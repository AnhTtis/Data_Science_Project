\section{Decomposability of t-JNs} \seclabel{decomposability}
%EXPLAIN DECOMPOSABLE: WHY DO WANT IT?
%
%MAIN REASON: THE PROJECTION DESCRIBES THE ``LIFE CYCLE'' OF INDIVIDUAL OBJECTS.
%
%SHOW THAT ONLY LOOKING AT INDIVIUDAL OBJECTS IS NOT SUFFICIENT, THEREFORE, WE NEED TO LOOK AT THE PROJECTION OF SETS OF IDENTIFIERS! (x, y separate not sufficient, also look at \sequence{x, y})
%
%PROJECTION ON SETS OF TYPES \dbtodo{FIX DEFINITION.}




%\andy{add some preliminaries}

\tpnids specify a class of nets with explicitly defined interactions between objects of different types within one system.
However, sometimes one may want to focus only on some behaviors exhibited by a given set of object types, by extracting a corresponding net from the original \tpnid model. 
We formalize this idea below.
\begin{definition}[Type projection] \deflabel{type_projection}
Let $N = (P_N, T_N, F_N, \alpha, \beta)$ be a \tpnid and $\Upsilon \subseteq \Lambda$ be a set of identifier types.
The \emph{type projection} of $\Upsilon$ on $N$ is a \tpnid $\project{\Upsilon}{N} = (P_\Upsilon, T_\Upsilon, F_\Upsilon, \alpha_\Upsilon, \beta_\Upsilon)$, where:
\begin{compactitem}
\item $P_\Upsilon = \set{p\in P\mid \Upsilon\subseteq\alpha(p)}$;
\item $T_\Upsilon = \set{t\in T\mid (\pre{t}\cup\post{t})\cap P\neq \emptyset}$;
\item $F_\Upsilon = F\cap ((P_\Upsilon\times T_\Upsilon)\cup (T_\Upsilon\times P_\Upsilon))$;
\item $\alpha_\Upsilon(p)=\Upsilon$, for each $p\in P_\Upsilon$;
\item $\beta_\Upsilon(f)=\restr{\beta(f)}{\type_{\varset}^{-1}(\Upsilon)}$, for each $f\in ((P_\Upsilon\times T_\Upsilon)\cup (T_\Upsilon\times P_\Upsilon))$.
\end{compactitem}
%	Given a t-JN $N=(P,T,F,\alpha,\beta)\in\mathcal{T}$ and a set of variables $X$ s.t. $X\subseteq vars(N)$ and $\exists f\in F$ s.t. $\beta(f)=X$ , then the projection of $N$ on $X$ is a t-JN $N'$ s.t.
%	\andy{add the definition; remember that the projection is a ``brand new'' net with renamed places and transitions (but transition labels should remain the same, i.e. $\ell(t)=\ell(t')$, where $\ell$ is the labeling function and $t'$ is just the primed copy of $t$)}
\end{definition}
%\andy{notice that it would make sense to treat types as sequences from the very beginning (we essentially use sequences in the above definition)}

With the next lemma we explore a property of typed Jackson nets that, in a nutshell, shows that \tjns are closed under the type projection. 
This also indirectly witnesses that \tjns provide a suitable formalism for specifying and manipulating systems with multiple communicating components.

\begin{lemma}
If $N = (P_N, T_N, F_N, \alpha, \beta)$ is a \tjn, then  $\project{\Upsilon}{N}$ is a \tjn as well, for any $\Upsilon \subseteq \type_\Lambda(N)$. 
\end{lemma}

\begin{proof} (sketch) Let us assume for simplicity that $N$ is atomic. Then, using rules from \defref{typed_jackson_net}, $N$ can be reduced to a single transition.  
Starting from this transition, one can construct a \tjn following the net graph construction from \defref{type_projection} using the same rules (but the identifier introduction one), proviso that arc inscriptions are always of type $\Upsilon$.
Then, it is easy to check that the constructed net is indeed the type projection of $\Upsilon$ on $N$.
\end{proof}

%\begin{property} \proplabel{projection_gives_tjn}
%	Let $\lambda \in \Lambda$ a type, and let $N = (P, T, F, \alpha, \beta)$ a \tjn. Then $\project{\lambda}{N}$ is a \tjn.
%\end{property}
%
%\begin{proof} \prflabel{projection_gives_tjn}
%	This proofs \propref{projection_gives_tjn}.
%\end{proof}

We define next how \tpnids can be composed and show that \tjns are not closed under the composition.
\begin{definition}[Composition] \deflabel{composition}
	Let $N = (P_N, T_N, F_N, \alpha_N, \beta_N)$ and \\$M = (P_M, T_M, F_M, \alpha_M, \beta_M)$ be two \tpnids. Their \emph{composition} is defined by: $$
	\compose{N}{M} = \left(P_N \cup P_M, T_N \cup T_M, F_N \cup F_M, \alpha_N \cup \alpha_M, \beta_N \cup \beta_M\right)
	$$
\end{definition}




\begin{figure}[t]
  \centering
	\begin{subfigure}{.45\textwidth}
		\centering
		 	\begin{tikzpicture}[->,>=stealth',auto,x=10mm,y=1cm,node distance=11mm and 3mm,thick,  every node/.style={scale=0.7}]
			\node[tr] (a) {$a$};
			\node[pl,right of = a, label=below:$p$] (p) {};
			\node[tr,right of = p] (b) {$c$};
			\node[pl,right of = b, label=below:$q$] (q) {};
			\node[tr,right of = q] (c) {$b$};
			\node[pl,right of = c, label=below:$r$] (r) {};
			\node[tr,right of = r] (d) {$d$};
			\path[->]
				(a) edge node[above,pos=.3]{$x$} (p)
				(p) edge node[above,pos=.3]{$x$} (b)
				(b) edge node[above,pos=.3]{$x$} (q)
				(q) edge node[above,pos=.3]{$x$} (c)
				(c) edge node[above,pos=.3]{$x$} (r)
				(r) edge node[above,pos=.3]{$x$} (d);  
			\end{tikzpicture}
        \caption{\tjn $N$\label{fig:tjn-n}}
	\end{subfigure}
	\begin{subfigure}{.45\textwidth}
		\centering
		\begin{tikzpicture}[->,>=stealth',auto,x=10mm,y=1cm,node distance=11mm and 3mm,thick,  every node/.style={scale=0.7}]
			\node[tr,] (a) {$a$};
			\node[pl,right of = a, label=below:$p$] (p) {};
			\node[tr,right of = p] (b) {$b$};
			\node[pl,right of = b, label=below:$s$] (s) {};
			\node[tr,right of = s] (c) {$c$};
			\node[pl,right of = c, label=below:$r$] (r) {};
			\node[tr,right of = r] (d) {$d$};
			\path[->]
				(a) edge node[above,pos=.3]{$x$} (p)
				(p) edge node[above,pos=.3]{$x$} (b)
				(b) edge node[above,pos=.3]{$x$} (s)
				(s) edge node[above,pos=.3]{$x$} (c)
				(c) edge node[above,pos=.3]{$x$} (r)
				(r) edge node[above,pos=.3]{$x$} (d);  
			\end{tikzpicture}
        \caption{\tjn $M$\label{fig:tjn-m}}
	\end{subfigure}
	\begin{subfigure}{.6\textwidth}
		\centering
			\begin{tikzpicture}[->,>=stealth',auto,x=10mm,y=1cm,node distance=11mm and 3mm,thick,  every node/.style={scale=0.7}]
			\node[tr] (a) {$a$};
			\node[pl,right of = a, label=below:$p$] (p) {};
			\node[tr,right of = p] (b) {$b$};
			\node[pl,right of = b, label=below:$s$,yshift=-1cm] (s) {};
			\node[pl,right of = b, label=below:$q$,yshift=1cm] (q) {};
			\node[tr,right of = s, yshift=1cm] (c) {$c$};
			\node[pl,right of = c, label=below:$r$] (r) {};
			\node[tr,right of = r] (d) {$d$};
			\path[->]
				(a) edge node[above,pos=.3]{$x$} (p)
				(p) edge node[above,pos=.3]{$x$} (b)
				(b) edge node[above,sloped]{$x$} (q)
				(q) edge node[above,sloped]{$x$} (c)
				(c) edge node[above,sloped]{$x$} (s)
				(s) edge node[above,sloped]{$x$} (b)
				(c) edge node[above,pos=.3]{$x$} (r)
				(r) edge node[above,pos=.3]{$x$} (d);  
			\end{tikzpicture}
        \caption{\tpnid $\compose{N}{M}$}
	\end{subfigure}
\caption{Although both $N$ and $M$ are \tjns, their composition is not}\label{fig:union-example}
\end{figure}

It is easy to see that the composition of two \tjns does not automatically result in a \tjn. Consider nets in \figref{union-example}. It is easy to see that both $N$ and $M$ can be obtained by applying \ref{itm:tjn_R2} from \defref{typed_jackson_net}. However, their composition %$\compose{N}{M}$ 
cannot be reduced to a single transition by consecutively applying rules from \defref{typed_jackson_net}.




\begin{figure}
	\centering
	\begin{tikzpicture}[->,>=stealth',auto,node distance=20mm and 12mm,thick,  every node/.style={scale=0.8}]

\node[tr] (a) {$a$};
\node[pl,right of = a, label=above:$p_x$,yshift=15mm,green1] (px) {};
\node[pl,right of = a, label=below:$p_{xy}$] (p) {};
\node[pl,right of = a, label=below:$p_y$,yshift=-15mm,cadmiumorange] (py) {};
\node[tr, right of = px] (b) {$b$};
\node[tr, right of = py] (c) {$c$};
\node[pl,right of = b, label=above:$q_x$,green1] (qx) {};
\node[pl,right of = c, label=below:$q_y$,cadmiumorange] (qy) {};
\node[pl,right of = b, label=below:$q_{xy}$,yshift=-15mm] (q) {};
\node[tr, right of = q] (d) {$d$};


\path[->,green1]
(a) edge node[above]{$x$} (px)
(px) edge node[above]{$x$} (b)
(b) edge node[above]{$x$} (qx)
(qx) edge node[above]{$x$} (d)
(px) edge node[right, pos=.7]{$x$} (c)
(c) edge node[left, pos=.2]{$x$} (qx)
;  

\path[->,cadmiumorange]
(a) edge  node[below]{$y$} (py)
(py) edge node[below]{$y$} (c)
(py) edge node[right,pos=.7]{$y$} (b)
(b) edge node[left,pos=.2]{$y$} (qy)
(c) edge node[below]{$y$} (qy)
(qy) edge node[below]{$y$} (d)
;  


\path[->,byzantium]
(a) edge node[above]{$xy$} (p)
(p) edge node[above,sloped, pos=.6]{$xy$} (b)
(p) edge node[below,sloped, pos=.6]{$xy$} (c)
(b) edge node[above,sloped,pos=.4]{$xy$} (q)
(c) edge node[below,sloped,pos=.4]{$xy$} (q) 
(q) edge node[above]{$xy$} (d)
;  
\end{tikzpicture}
	\caption{Composition of the projections on $\set{\lambda_1}$, $\set{\lambda_2}$ and $\set{\lambda_1,\lambda_2}$ on the \tjn $(a;[p,\tup{x,y}]; (b||c);[q,\tup{x,y}];d)$. Here, type assignments are as follows: 
	$\alpha(p_x)=\alpha(q_x)=\lambda_1$,
	$\alpha(p_y)=\alpha(q_y)=\lambda_2$ and 
	$\alpha(p)=\alpha(q)=\lambda_1\lambda_2$.}
	\label{fig:proj-union-example}
\end{figure}


A more surprising observation is that composing type projections of a \tjn may not result in a \tjn. Take for example the net from Figure~\ref{fig:proj-union-example}. 
Both its projections on $\set{\lambda_1}$ and $\set{\lambda_2}$ are \tjns. 
However, bringing them together using the composition operator results in a \tpnid that is not \tjn: indeed, since the ``copies'' of place $p$ appear in three places, and all such copies have same pre- and post-sets (and only differ by their respective types), it is impossible to apply identifier elimination rule \emph{R6} from \defref{typed_jackson_net}.

As one may observe from the above example, 
the only difference between $[p_{xy},\tup{\lambda_1,\lambda_2}]$ and its copies $p_x$ and $p_y$ is in their respective types, whereas the identifiers carried by $p_x$ and $p_y$ are always contained in $p_{xy}$, and thus both $p_x$ and $p_y$ can be seen as subsidiary with respect to $p_{xy}$. 
We formalize this observation using the notion of \emph{minor places}:  
a place $p$ is minor to some place $q$ if both $p$ and $q$ have identical pre- and post-sets, and the type of $q$ subsumes the one of $p$. 


\begin{definition}[Minor places] \deflabel{minor_places}
Let $N = (P_N, T_N, F_N, \alpha, \beta)$ be a \tpnid.
A place $p\in P$ is \emph{minor to} a place $q\in P$ iff the following holds:
\begin{compactitem}
\item  $\pre{p}=\pre{q}$, $\post{p}=\post{q}$ and $\alpha(p)\subset \alpha(q)$; 
\item  $\beta((t,p))=\restr{\beta((t,q))}{\type^{-1}(\alpha(p))}$, for each $t\in\pre{p}$;
\item  $\beta((p,t))=\restr{\beta((q,t))}{\type^{-1}(\alpha(p))}$, for each $t\in\post{p}$.
\end{compactitem}
\end{definition}

We show next that minor places can be added or removed without altering the overall behavior of the net.

\begin{lemma}
\label{lemma:minors}
Let $N = (P, T, F, \alpha, \beta)$ be a \tpnid with initial marking $m_0$ s.t. 
$m_0(p)=m_0(q)=\emptyset$, for $p,q\in P$, where $p$ is minor to $q$.
Let $N'=(P\setminus\set{p}, T, F\setminus(\set{(p,t)|t\in\post{p}}\cup\set{(t,p)|t\in\pre{p}}),\alpha,\beta)$ be a \tpnid obtained by eliminating from $N$ place $p$ .
Then $\transitionsystem{N,m_0}\sim^r \transitionsystem{N',m_0}$.
\end{lemma}
%\andy{we need to define bisimulation of two \tpnids}
\begin{proof}(sketch)
It is enough to define a relation $Q\subseteq \pnreachable{N}{m_0}\times \pnreachable{N'}{m_0}$ s.t. $(m,m')\in Q$ iff $m(r)=m'(r)$, for $r\in P\setminus\set{p}$, and $m(p)(\cname{id})=m'(q)(\cname{id})$, for all $\cname{id}\in\colset(\place)$, and $|m(p)|=|m'(q)|$.
Then the lemma statement directly follows from the firing rule of \tpnids and that pre- and post-sets of $p$ and $q$ coincide.
\end{proof}

\begin{figure}[t!]
  \centering
	\begin{subfigure}{.38\textwidth}
		\centering
		 	\begin{tikzpicture}[->,>=stealth',auto,x=10mm,y=1cm,node distance=15mm and 3mm,thick,  every node/.style={scale=0.7}]
			\node[tr] (a) {$a$};
			\node[pl,right of = a] (p1) {};
			\node[tr,right of = p1] (c) {$c$};			
			\node[pl,right of = c] (p5){};
			\node[tr,below of = p5] (e){$e$};
			 \node[pl,below of = e] (p6){};
			\node[pl,below of = a] (p2) {};
			\node[tr,below of = p2] (b){$b$};
			\node[pl,below of = c,label=left:$p$] (p3) {};
			\node[tr,below of = p3] (d) {$d$};
			\node[pl, left of = d] (p4){};
			\path[->]
				(a) edge node[above]{$x$} (p1)
				(a) edge node[right]{$x$} (p2)
				(p1) edge node[above]{$x$} (c)
				(p2) edge node[right]{$x$} (b)
				(p4) edge node[above]{$x$} (b)
				(d) edge node[above]{$x$}  (p4)
				(c) edge node[right]{$xy$} (p3)
				(p3) edge node[right]{$xy$}  (d)
				(c) edge node[above]{$y$}  (p5)
				(p5) edge node[right]{$y$} (e)
				(e) edge node[right]{$y$} (p6)
				(p6) edge node[above]{$y$} (d);  
			\end{tikzpicture}
        \caption{\tpnid $N$ with $\type(x)=\lambda_1$ and $\type(y)=\lambda_2$\label{fig:singleton-tjn-n}}
	\end{subfigure}
	\begin{subfigure}{.3\textwidth}
		\centering
		\begin{tikzpicture}[->,>=stealth',auto,x=10mm,y=1cm,node distance=15mm and 3mm,thick,  every node/.style={scale=0.7}]
			\node[tr] (a) {$a$};
			\node[pl,right of = a] (p1) {};
			\node[tr,right of = p1] (c) {$c$};			
			\node[pl,below of = a] (p2) {};
			\node[tr,below of = p2] (b){$b$};
			\node[pl,below of = c,label=left:$p_x$] (p3) {};
			\node[tr,below of = p3] (d) {$d$};
			\node[pl, left of = d] (p4){};
			\path[->]
				(a) edge node[above]{$x$} (p1)
				(a) edge node[right]{$x$} (p2)
				(p1) edge node[above]{$x$} (c)
				(p2) edge node[right]{$x$} (b)
				(p4) edge node[above]{$x$} (b)
				(d) edge node[above]{$x$}  (p4)
				(c) edge node[right]{$x$} (p3)
				(p3) edge node[right]{$x$}  (d);
			\end{tikzpicture}
        \caption{The projection of $\set{\lambda_1}$ on $N$ \label{fig:singleton-tjn-m}}
	\end{subfigure}
	\begin{subfigure}{.3\textwidth}
		\centering
		\begin{tikzpicture}[->,>=stealth',auto,x=10mm,y=1cm,node distance=15mm and 3mm,thick,  every node/.style={scale=0.7}]
			\node[tr] (c) {$c$};			
			\node[pl,right of = c] (p5){};
			\node[tr,below of = p5] (e){$e$};
			 \node[pl,below of = e] (p6){};
			\node[pl,below of = c,label=left:$p_y$] (p3) {};
			\node[tr,below of = p3] (d) {$d$};
			\path[->]
				(c) edge node[right]{$y$} (p3)
				(p3) edge node[right]{$y$}  (d)
				(c) edge node[above]{$y$}  (p5)
				(p5) edge node[right]{$y$} (e)
				(e) edge node[right]{$y$} (p6)
				(p6) edge node[above]{$y$} (d);  
			\end{tikzpicture}
        \caption{The projection of $\set{\lambda_2}$ on $N$ \label{fig:singleton-tjn-k}}
	\end{subfigure}
	\begin{subfigure}{.7\textwidth}
		\centering
		 	\begin{tikzpicture}[->,>=stealth',auto,x=10mm,y=1cm,node distance=15mm and 3mm,thick,  every node/.style={scale=0.7}]
			\node[tr] (a) {$a$};
			\node[pl,right of = a] (p1) {};
			\node[tr,right of = p1,xshift=5mm] (c) {$c$};			
			\node[pl,right of = c,xshift=5mm] (p5){};
			\node[tr,below of = p5] (e){$e$};
			 \node[pl,below of = e] (p6){};
			\node[pl,below of = a] (p2) {};
			\node[tr,below of = p2] (b){$b$};
			\node[pl,below of = c,label=left:$p_x$,xshift=-.5cm] (px) {};
			\node[pl,below of = c,label=right:$p_y$,xshift=.5cm] (py) {};

			\node[tr,below of = px,xshift=.5cm] (d) {$d$};
			\node[pl, left of = d] (p4){};
			\path[->]
				(a) edge node[above]{$x$} (p1)
				(a) edge node[right]{$x$} (p2)
				(p1) edge node[above]{$x$} (c)
				(p2) edge node[right]{$x$} (b)
				(p4) edge node[above]{$x$} (b)
				(d) edge node[above]{$x$}  (p4)
				(c) edge node[left]{$x$} (px)
				(px) edge node[left]{$y$}  (d)
				(c) edge node[right]{$y$} (py)
				(py) edge node[right]{$y$}  (d)
				(c) edge node[above]{$y$}  (p5)
				(p5) edge node[right]{$y$} (e)
				(e) edge node[right]{$y$} (p6)
				(p6) edge node[above]{$y$} (d);  
			\end{tikzpicture}
        \caption{The composition of  $\project{\set{\lambda_1}}{N}$ and $\project{\set{\lambda_2}}{N}$\label{fig:compose-singleton}}
	\end{subfigure}
\caption{\tpnid $N$ (\ref{fig:singleton-tjn-n}), its singleton projections  and their composition }\label{fig:singleton-example}
\end{figure}

Let us now address the reconstructability property.
In a nutshell, a net is reconstructable if composing all of its type projections 
returns the same net. 
This property is not that trivial to obtain.
For example, let us consider singleton projections (that is, projections $\project{\set{\lambda}}{N}$ obtained for each $\lambda\in\type_\Lambda(N)$)
of the net in \figref{singleton-example}.
It is easy to see that such projections ``ignore'' interactions between objects (or system components).
Thus, the composition of the singleton projections $\project{\set{\lambda_1}}{N}$ and $\project{\set{\lambda_2}}{N}$ from \figref{singleton-example} does not result in a model
that merges $p_x$ and $p_y$ in one place as the composition operator cannot recognize component interactions between such projections. 
This is reflected in \figref{compose-singleton}.





To be able to reconstruct the original model from its projections (or at least do it approximately well), 
one needs to consider a projection reflecting component interactions. In the case of the net from Figure~\ref{fig:singleton-tjn-n}, its non-singleton projection $\project{\set{\lambda_1,\lambda_2}}{N}$ is depicted in Figure~\ref{fig:interactions-n}.
Now, using this projection we can obtain a composition (see Figure~\ref{fig:compose-full}) that closely resembles $N$. 
Notice that, in this composition, copies of the interaction place $p$ appear three times as places $p_x$, $p_y$ and $p_{xy}$, respectively.
It is also easy to see that places $p_x$ and $p_y$ are minor to $p_{xy}$, and 
$\alpha(p)=\alpha(p_{xy})$ witnesses that $\project{\set{\lambda_1,\lambda_2}}{N}$ is the maximal projection defined over types of $N$ s.t. the correct type of $p$ is ``reconstructed''.
This leads us to the following result stipulating the reconstructability property of typed Jackson nets. 


\begin{figure}[t!]
  \centering
	\begin{subfigure}{.4\textwidth}
		\centering
		\begin{tikzpicture}[->,>=stealth',auto,x=10mm,y=1cm,node distance=15mm and 3mm,thick,  every node/.style={scale=0.7}]
			\node[tr] (c) {$c$};			
			 \node[pl,below of = c,label=right:$p_{xy}$] (p){};
			\node[tr,below of = p] (d) {$d$};
			\path[->]
				(c) edge node[right]{$xy$} (p)
				(p) edge node[right]{$xy$} (d);  
			\end{tikzpicture}
        \caption{The projection of $\set{\lambda_1,\lambda_2}$ on $N$ from Figure~\ref{fig:singleton-tjn-n} \label{fig:interactions-n}}
	\end{subfigure}\quad
	\begin{subfigure}{.5\textwidth}
		\centering
		 	\begin{tikzpicture}[->,>=stealth',auto,x=10mm,y=1cm,node distance=15mm and 3mm,thick,  every node/.style={scale=0.7}]
			\node[tr] (a) {$a$};
			\node[pl,right of = a] (p1) {};
			\node[tr,right of = p1,xshift=13mm] (c) {$c$};			
			\node[pl,right of = c,xshift=13mm] (p5){};
			\node[tr,below of = p5] (e){$e$};
			 \node[pl,below of = e] (p6){};
			\node[pl,below of = a] (p2) {};
			\node[tr,below of = p2] (b){$b$};
			\node[pl,below of = c,label=left:$p_x$,xshift=-13mm] (px) {};
			\node[pl,below of = c,label=right:$p_y$,xshift=13mm] (py) {};
			\node[pl,below of = c,label=right:$p_{xy}$] (p) {};

			\node[tr,below of = px,xshift=13mm] (d) {$d$};
			\node[pl, left of = d] (p4){};
			\path[->]
				(a) edge node[above]{$x$} (p1)
				(a) edge node[right]{$x$} (p2)
				(p1) edge node[above]{$x$} (c)
				(p2) edge node[right]{$x$} (b)
				(p4) edge node[above]{$x$} (b)
				(d) edge node[above]{$x$}  (p4)
				(c) edge node[left]{$x$} (px)
				(px) edge node[left]{$y$}  (d)
				(c) edge node[right]{$xy$} (p)
				(p) edge node[right]{$xy$}  (d)
				(c) edge node[right]{$y$} (py)
				(py) edge node[right]{$y$}  (d)
				(c) edge node[above]{$y$}  (p5)
				(p5) edge node[right]{$y$} (e)
				(e) edge node[right]{$y$} (p6)
				(p6) edge node[above]{$y$} (d);  
			\end{tikzpicture}
			  \caption{The composition  $\project{\set{\lambda_1}}{N}\composeOperator\project{\set{\lambda_2}}{N}\composeOperator\project{\set{\lambda_1,\lambda_2}}{N}$  for $N$ from Figure~\ref{fig:singleton-tjn-n} \label{fig:compose-full}}
%        \caption{The composition  $\biguplus\limits_{\Upsilon\subseteq\type_\Lambda(N)}\project{\Upsilon}{N}$ for $N$ from Figure~\ref{fig:singleton-tjn-n} \label{fig:compose-full}}
	\end{subfigure}
\caption{Adding the projection $\project{\set{\lambda_1,\lambda_2}}{N}$ reflecting interactions to the composition results in the original net $N$ modulo places minor to $p$ (such as $p_x$ and $p_y$). }\label{fig:compose-interactions}
\end{figure}

\begin{theorem}\thmlabel{reconstructability}
Let $N = (P, T, F, \alpha, \beta)$ be a \tjn. 
Then $\transitionsystem{N,\emptyset}\sim^r \transitionsystem{N',\emptyset}$, where\\ $N'=\biguplus\limits_{\emptyset\subset\Upsilon\subseteq\type_\Lambda(N)}\project{\Upsilon}{N}$. 
\end{theorem}
\begin{proof} (sketch)
The proof immediately follows from the next observation.
Among all possible projections, for each place $p\in P$ there exists a projection $\project{\Upsilon}{N}$ such that $\alpha(p)=\Upsilon$. This also means that $\project{\Upsilon}{N}$ contains $p$ and that all other projections $\project{\Upsilon'}{N}$ with $\Upsilon'\subset\Upsilon$ will at most include the minors of $p$. 
Following \defref{composition}, it is easy to see that the composition of all the projections yields a \tjn identical to $N$ modulo additional place minors introduced by some of the projections. Showing that the obtained net is bisimilar to $N$ can be done by analogy with Lemma~\ref{lemma:minors}.
\end{proof}

	
Notice that the above result can be made stronger if all the additional minors (i.e., minors that were not present originally in $N$) are removed using reduction rules from \defref{typed_jackson_net}. For simplicity, given a \tpnid $N$ with the set of places $P$, we denote by $\lfloor P \rfloor$ the set of its minor places. 

\begin{corollary}\corlabel{reconstructability}
Let $N$ be a \tjn and $N'$ is as in \thmref{reconstructability}. 
Then $(N,\emptyset)\leftrightsquigarrow(N',\emptyset)$, if $\lfloor P \rfloor = \lfloor P' \rfloor$, where $P$ and $P'$ are respectively the sets of places of $N$ and $N'$.
\end{corollary}
The above result can be obtained by complementing the proof of \thmref{reconstructability} with a step that applies finitely many \tjn reduction rules to all the minor places that are in $N'$ and not in $N$.

\endinput


\begin{property} \proplabel{composition_gives_tjn}
	Let $N = (P, T, F, \alpha, \beta)$ a \tjn, and suppose $N_\gamma = \project{\gamma}{N}$ and $N_\delta = \project{\delta}{N}$ two projections. Then their composition $\compose{N_\gamma}{N_\delta}$ is a \tjn.
\end{property}

\begin{proof} \prflabel{composition_gives_tjn}
	This proofs \propref{composition_gives_tjn}.
\end{proof}


\andy{elaborate more on this property: simply say that behavioral correlations between types are not enough and demonstrate that on a simple example with two subnets for types x and xyz}
\begin{property} \proplabel{projection_not_simulated}
	Let $N = (P, T, F, \alpha, \beta)$ a \tpnid, and $\lambda, \gamma \in \Lambda$ two types such that $\lambda \subseteq \gamma$.
	Then the type projection on $\lambda$ is not simulated by the type projection on $\gamma$, that is, $\delaysim{\project{\lambda}{N}}{}{\project{\gamma}{N}}$ does not hold.
\end{property}

\begin{proof} \prflabel{projection_not_simulated}
	This proofs \propref{projection_not_simulated} by counterexample, see whiteboard.
\end{proof}

\begin{definition}[Projection closure] \deflabel{projection_closure}
	Let $\lambda \in \Lambda$ be a type. Given a \tpnid $N = (P_N, T_N, F_N, \alpha, \beta)$, its \emph{closed $\lambda$-projection} $\projectClosure{\lambda}{N} = (P, T, F, \alpha, \beta)$ is a \tpnid with
	\begin{compactitem}
		\item $P = \set{p \in P_N \mid \alpha(p) \subseteq \lambda}$
		\item $T = \set{t \in T_N \mid ({}_N^\bullet t \cup t^\bullet_N) \cap P \neq \emptyset}$, \dbtodo{are these presets and postsets defined somewhere?}
		\item $F = F_N \cap ((P \times T) \cup (T \times P))$
	\end{compactitem}
\end{definition}

\begin{corollary} \corlabel{closure_projection_is_itself}
	The closed projection over all $\Lambda$ for \tpnid $N$ is  isomorphic equivalent to $N$:
	$$
	\projectClosure{\Lambda}{N} = N
	$$
\end{corollary}

\begin{proof} \deflabel{closure_projection_is_itself}
	This proofs \corref{closure_projection_is_itself}.
\end{proof}

\begin{theorem} \thmlabel{decomposability}
	Let $N = (P, T, F, \alpha, \beta)$ a \tpnid, and $\gamma \subseteq \Lambda$ a set of types. Then the following holds.
	$$
	\Composed{\lambda \in \gamma^*}  \strongbisim{\project{\lambda}{N}}{}{\projectClosure{\gamma}{N}}
	$$
\end{theorem}

\begin{proof} \prflabel{decomposability}
	This proofs \thmref{decomposability}.
\end{proof}


\begin{corollary} \corlabel{decomposability}
	Let $N = (P, T, F, \alpha, \beta)$ a \tjn, then the following holds.
	$$
	\Composed{\lambda \in \Lambda^*} \project{\lambda}{N} = N
	$$
\end{corollary}

\begin{proof} \prflabel{arg1}
	Follows directly from \correffull{closure_projection_is_itself} and \thmreffull{decomposability}.
\end{proof}
%
%
%\begin{property} \proplabel{composition_gives_tjn}
%	Let $N = (P, T, F, \alpha, \beta)$ a \tjn, and suppose $N_\gamma = \project{\gamma}{N}$ and $N_\delta = \project{\delta}{N}$ two projections. Then their composition $\compose{N_\gamma}{N_\delta}$ is a \tjn.
%\end{property}
%
%\begin{proof} \prflabel{composition_gives_tjn}
%	This proofs \propref{composition_gives_tjn}.
%\end{proof}
%
%
%\andy{elaborate more on this property: simply say that behavioral correlations between types are not enough and demonstrate that on a simple example with two subnets for types x and xyz}
%\begin{property} \proplabel{projection_not_simulated}
%	Let $N = (P, T, F, \alpha, \beta)$ a \tpnid, and $\lambda, \gamma \in \Lambda$ two types such that $\lambda \subseteq \gamma$.
%	Then the type projection on $\lambda$ is not simulated by the type projection on $\gamma$, that is, $\delaysim{\project{\lambda}{N}}{}{\project{\gamma}{N}}$ does not hold.
%\end{property}
%
%\begin{proof} \prflabel{projection_not_simulated}
%	This proofs \propref{projection_not_simulated} by counterexample, see whiteboard.
%\end{proof}
%
%\begin{definition}[Projection closure] \deflabel{projection_closure}
%	Let $\lambda \in \Lambda$ be a type. Given a \tpnid $N = (P_N, T_N, F_N, \alpha, \beta)$, its \emph{closed $\lambda$-projection} $\projectClosure{\lambda}{N} = (P, T, F, \alpha, \beta)$ is a \tpnid with
%	\begin{compactitem}
%		\item $P = \set{p \in P_N \mid \alpha(p) \subseteq \lambda}$
%		\item $T = \set{t \in T_N \mid ({}_N^\bullet t \cup t^\bullet_N) \cap P \neq \emptyset}$, \dbtodo{are these presets and postsets defined somewhere?}
%		\item $F = F_N \cap ((P \times T) \cup (T \times P))$
%	\end{compactitem}
%\end{definition}
%
%\begin{corollary} \corlabel{closure_projection_is_itself}
%	The closed projection over all $\Lambda$ for \tpnid $N$ is  isomorphic equivalent to $N$:
%	$$
%	\projectClosure{\Lambda}{N} = N
%	$$
%\end{corollary}
%
%\begin{proof} \deflabel{closure_projection_is_itself}
%	This proofs \corref{closure_projection_is_itself}.
%\end{proof}
%
%\begin{theorem} \thmlabel{decomposability}
%	Let $N = (P, T, F, \alpha, \beta)$ a \tpnid, and $\gamma \subseteq \Lambda$ a set of types. Then the following holds.
%	$$
%	\Composed{\lambda \in \gamma^*}  \strongbisim{\project{\lambda}{N}}{}{\projectClosure{\gamma}{N}}
%	$$
%\end{theorem}
%
%\begin{proof} \prflabel{decomposability}
%	This proofs \thmref{decomposability}.
%\end{proof}
%
%
%\begin{corollary} \corlabel{decomposability}
%	Let $N = (P, T, F, \alpha, \beta)$ a \tjn, then the following holds.
%	$$
%	\Composed{\lambda \in \Lambda^*} \project{\lambda}{N} = N
%	$$
%\end{corollary}
%
%\begin{proof} \prflabel{arg1}
%	Follows directly from \correffull{closure_projection_is_itself} and \thmreffull{decomposability}.
%\end{proof}
>>>>>>> 1cae204e8fc233fb1cd2f50e31b7b50be10dbbf6


%
%% everything below here is old
%
%
%For the sake of simplicity, we make the following assumption:\andy{elaborate more on this assumption}
%%Note tha\\t from \defref{typed_jackson_nets} it follows that transitions have unique actions, and can thus be identified precisely..
%\begin{assumption}
%	\tjns have unique transition labels; \db{No longer necessary I think, since we can just state that $\ell$ maps to unique actions, and there is a bijection between \A and $\Lambda$}.
%\end{assumption}
%
%
%\begin{figure}
%	\centering
%	\includegraphics[width=\linewidth]{figs/screenshot_extended_paper_definition_projection}
%	\caption{\dbtodo{Definition of type projection is needed.}}
%	\label{fig:screenshotextendedpaperdefinitionprojection}
%\end{figure}
%
%
%Now, we define a full t-JN decomposition. In the nutshell, the full decomposition is a set of t-JNs obtained from the original one by constructing named projections for every inscription in the original net.
%
%\begin{definition}[Full decomposition]
%	Idea: given a t-JN $N$, build a set $dec(N)$ consisting of named projections, where a each projection is obtained for a set take from one of the net inscriptions.
%	\andy{add the definition (has to be constructive, algorithm-like)}
%\end{definition}
%
%COMPOSITION OPERATOR
%
%Now, we introduce a special operator that, given two t-JNs, puts them into a synchronous composition, where the synchronization is realized only for same-labeled transitions.
%We start with an example explaining the idea behind this operator.
%
%\begin{example}
%	\label{ex:tjn-projections}
%	\andy{rewrite/complete the example + show which rules are used to construct the original net $N$}
%	Consider the following typed t-JN $N$.
%	\begin{center}
%		\begin{tikzpicture}[->,>=stealth',auto,x=15mm,y=1.0cm,node distance=15mm and 7mm,thick, scale=0.8, every node/.style={scale=0.8}]
%
%			\node[tr] (a) {$a$};
%			\node[pl,right of = a] (p1) {};
%			\node[tr, right of = p1] (b) {$b$};
%			\node[tr, below of = p1] (c) {$c$};
%			\node[pl,right of = c] (p2) {};
%			\node[tr,right of = p2] (d) {$d$};
%			\node[pl,right of = d] (p3) {};
%			\node[tr,right of = p3] (f) {$f$};
%			\node[pl,below of = p2,yshift=4mm] (p4){};
%			\node[tr,right of = p4] (e) {$e$};
%			\node[pl,right of = e] (p5) {};
%			\node[pl,below of = p4,yshift=4mm] (p6){};
%			\node[tr,right of = p6] (g) {$g$};
%			\node[pl,right of = g] (p7) {};
%
%
%
%			\path[->]
%			(a) edge node{$x$} (p1)
%			(p1) edge node{$x$} (b)
%			(p1) edge node{$x$} (c)
%			(c) edge node{$xyz$} (p2)
%			(p2)  edge node{$xyz$} (d)
%			(d) edge node{$xyz$} (p3)
%			(p3) edge node{$xyz$}  (f)
%			(f.north) edge[above,in=-30,out=150] node{$x$} (p1)
%			(c)  edge[above,bend right=10] node{$y$} (p4)
%			(p4) edge node{$y$} (e)
%			(e)  edge node{$y$} (p5)
%			(p5) edge[above,bend right=10] node{$y$} (f)
%			(c.south) edge[above,bend right=20] node{$z$}  (p6)
%			(p6) edge[above] node{$z$} (g)
%			(g) edge[above] node{$z$} (p7)
%			(p7) edge[above,bend right=20] node{$z$} (f.south)
%			;
%		\end{tikzpicture}
%	\end{center}
%
%	For the above net, we construct the following named projections.
%	\begin{enumerate}
%		\item For $X=\set{x}$, we get the following net:
%		      \begin{center}
%			      \begin{tikzpicture}[->,>=stealth',auto,x=15mm,y=1.0cm,node distance=15mm and 7mm,thick,scale=0.8, every node/.style={scale=0.8}]
%				      \node[tr] (a) {$a$};
%				      \node[pl,right of = a] (p1) {};
%				      \node[tr, right of = p1] (b) {$b$};
%				      \node[tr, below of = p1] (c) {$c$};
%				      \node[pl,right of = c] (p2) {};
%				      \node[tr,right of = p2] (d) {$d$};
%				      \node[pl,right of = d] (p3) {};
%				      \node[tr,right of = p3] (f) {$f$};
%				      \path[->]
%				      (a) edge node{$x$} (p1)
%				      (p1) edge node{$x$} (b)
%				      (p1) edge node{$x$} (c)
%				      (c) edge node{$x$} (p2)
%				      (p2)  edge node{$x$} (d)
%				      (d) edge node{$x$} (p3)
%				      (p3) edge node{$x$}  (f)
%				      (f.north) edge[above,in=-30,out=150] node{$x$} (p1);
%			      \end{tikzpicture}
%		      \end{center}
%		\item For $X=\set{y}$, we get the following net:
%		      \begin{center}
%			      \begin{tikzpicture}[->,>=stealth',auto,x=15mm,y=1.0cm,node distance=15mm and 7mm,thick,scale=0.8, every node/.style={scale=0.8}]
%				      \node[tr] (c) {$c$};
%				      \node[pl,right of = c] (p2) {};
%				      \node[tr,right of = p2] (d) {$d$};
%				      \node[pl,right of = d] (p3) {};
%				      \node[tr,right of = p3] (f) {$f$};
%				      \node[pl,below of = p2,yshift=4mm] (p4){};
%				      \node[tr,right of = p4] (e) {$e$};
%				      \node[pl,right of = e] (p5) {};
%				      \path[->]
%				      (c) edge node{$y$} (p2)
%				      (p2)  edge node{$y$} (d)
%				      (d) edge node{$y$} (p3)
%				      (p3) edge node{$y$}  (f)
%				      (c)  edge[above,bend right=10] node{$y$} (p4)
%				      (p4) edge node{$y$} (e)
%				      (e)  edge node{$y$} (p5)
%				      (p5) edge[above,bend right=10] node{$y$} (f);
%			      \end{tikzpicture}
%		      \end{center}
%		\item For $X=\set{z}$, we get the following net:
%		      \begin{center}
%			      \begin{tikzpicture}[->,>=stealth',auto,x=15mm,y=1.0cm,node distance=15mm and 7mm,thick,scale=0.8, every node/.style={scale=0.8}]
%				      \node[tr] (c) {$c$};
%				      \node[pl,right of = c] (p2) {};
%				      \node[tr,right of = p2] (d) {$d$};
%				      \node[pl,right of = d] (p3) {};
%				      \node[tr,right of = p3] (f) {$f$};
%				      \node[pl,below of = p2,yshift=4mm] (p4){};
%				      \node[tr,right of = p4] (e) {$g$};
%				      \node[pl,right of = e] (p5) {};
%				      \path[->]
%				      (c) edge node{$z$} (p2)
%				      (p2)  edge node{$z$} (d)
%				      (d) edge node{$z$} (p3)
%				      (p3) edge node{$z$}  (f)
%				      (c)  edge[above,bend right=10] node{$z$} (p4)
%				      (p4) edge node{$z$} (e)
%				      (e)  edge node{$z$} (p5)
%				      (p5) edge[above,bend right=10] node{$z$} (f);
%			      \end{tikzpicture}
%		      \end{center}
%		\item Finally, for $X=\set{x,y,z}$, we obtain the following t-JN:
%		      \begin{center}
%			      \begin{tikzpicture}[->,>=stealth',auto,x=15mm,y=1.0cm,node distance=15mm and 7mm,thick,scale=0.8, every node/.style={scale=0.8}]
%				      \node[tr] (c) {$c$};
%				      \node[pl,right of = c] (p2) {};
%				      \node[tr,right of = p2] (d) {$d$};
%				      \node[pl,right of = d] (p3) {};
%				      \node[tr,right of = p3] (f) {$f$};
%				      \path[->]
%				      (c) edge node{$xyz$} (p2)
%				      (p2)  edge node{$xyz$} (d)
%				      (d) edge node{$xyz$} (p3)
%				      (p3) edge node{$xyz$}  (f);
%			      \end{tikzpicture}
%		      \end{center}
%	\end{enumerate}
%
%	Now, we would like to see how to merge the obtained projections. By simply juxtaposing the above projections, we  obtain the following net:
%
%	\begin{center}
%		\begin{tikzpicture}[->,>=stealth',auto,x=15mm,y=1.0cm,node distance=15mm and 7mm,thick, scale=0.8, every node/.style={scale=0.8}]
%
%			\node[tr] (a) {$a$};
%			\node[pl,right of = a] (p1) {};
%			\node[tr, right of = p1] (b) {$b$};
%			\node[tr, below of = p1, yshift=-2cm] (c) {$c$};
%			\node[pl,right of = c,xshift=.5cm] (p2) {};
%			\node[pl,above of = p2,yshift=-6mm] (p2x) {};
%			\node[pl,below of = p2,yshift=6mm] (p2z) {};
%			\node[pl,above of = p2x,yshift=-6mm] (p2y) {};
%			\node[tr,right of = p2,xshift=.5cm] (d) {$d$};
%			\node[pl,right of = d,xshift=.5cm] (p3) {};
%			\node[tr,right of = p3,xshift=.5cm] (f) {$f$};
%			\node[pl,above of = p3,yshift=-6mm] (p3x) {};
%			\node[pl,above of = p3x,yshift=-6mm] (p3y) {};
%			\node[pl,below of = p3,yshift=6mm] (p3z) {};
%			\node[pl,below of = p2,yshift=-4mm] (p4){};
%			\node[tr,right of = p4,xshift=.5cm] (e) {$e$};
%			\node[pl,right of = e,xshift=.5cm] (p5) {};
%			\node[pl,below of = p4,yshift=4mm] (p6){};
%			\node[tr,right of = p6,xshift=.5cm] (g) {$g$};
%			\node[pl,right of = g,xshift=.5cm] (p7) {};
%
%			\path[->]
%			(a) edge node{$x$} (p1)
%			(p1) edge node{$x$} (b)
%			(p1) edge[left] node{$x$} (c)
%			(c) edge[above] node{$xyz$} (p2)
%			(p2)  edge[above] node{$xyz$} (d)
%			(d) edge[above] node{$xyz$} (p3)
%			(p3) edge[above] node{$xyz$}  (f)
%			(f.north) edge[above,in=-25,out=90] node{$x$} (p1)
%			(c)  edge[above,bend right=30,pos=.7] node{$y$} (p4)
%			(p4) edge node{$y$} (e)
%			(e)  edge node{$y$} (p5)
%			(p5) edge[above,bend right=30,pos=.2] node{$y$} (f)
%			(c.south) edge[above,bend right=20,pos=.8] node{$z$}  (p6)
%			(p6) edge[above] node{$z$} (g)
%			(g) edge[above] node{$z$} (p7)
%			(p7) edge[above,bend right=20,pos=.2] node{$z$} (f.south)
%			;
%
%			\path[->,red!60]
%			(c.north) edge[above,bend left=15] node{$x$} (p2x)
%			(p2x) edge[above,bend left=15] node{$x$} (d.north)
%			(d.north) edge[above,bend left=15] node{$x$} (p3x)
%			(p3x) edge[above,bend left=15] node{$x$} (f.north)
%			;
%
%			\path[->,blue!60]
%			(c.north) edge[above,bend left=25] node{$y$} (p2y)
%			(p2y) edge[above,bend left=25] node{$y$} (d.north)
%			(d.north) edge[above,bend left=25] node{$y$} (p3y)
%			(p3y) edge[above,bend left=25] node{$y$} (f.north)
%			;
%
%			\path[->,violet!60]
%			(c.south) edge[above,bend right=15] node{$z$} (p2z)
%			(p2z) edge[above,bend right=15] node{$z$} (d.south)
%			(d.south) edge[above,bend right=15] node{$z$} (p3z)
%			(p3z) edge[above,bend right=15] node{$z$} (f.south)
%			;
%		\end{tikzpicture}
%	\end{center}
%
%	\andy{\textbf{describe the merge here!}}
%\end{example}
%
%The above example gives an intuition of how the composition operator works for t-JNs that have a few same-labeled transitions in common. Next we formally define this operator.
%
%% %The above example gives an intuition of how the merge should not be performed in cases when typed nets with potentially partially shared identifier types are brought together. Instead, we propose the merge operator $\oplus$ that resolves the above issue. 
%% %In the nutshell, this operator creates a parallel composition of the discovered behaviors and, in case juxtaposed nets have overlapping subnets (that are not represented by single transitions), chooses the subnet that puts in relation more object types. 
%
%\begin{definition}[Synchronous product????]
%	The composition of two t-JNs $N_1 = (\places_1, \transitions_1, \flow_1, \alpha_1, \beta_1,\ell_1)$ and $N_2 = (\places_2, \transitions_2, \flow_2, \alpha_2, \beta_2,\ell_2)$ is a t-JN $N_1\oplus N_2= (\places, \transitions, \flow, \alpha, \beta,\ell)$, where:
%	\begin{itemize}
%		\item \andy{complete the definition, use the old definition as the basis}
%	\end{itemize}
%\end{definition}
%
%% %%%%%%%
%% %% OLD DEFINITION OF COMPOSITION 
%% %%%%%%%
%% %\begin{definition}
%% %The synchronous composition of two t-JNs $N_1 = (\places_1, \transitions_1, \flow_1, \alpha_1, \beta_1,\ell_1)$ and $N_2 = (\places_2, \transitions_2, \flow_2, \alpha_2, \beta_2,\ell_2)$ is a t-JN $N_1\oplus N_2= (\places, \transitions, \flow, \alpha, \beta,\ell)$, where:
%% %\begin{itemize}
%% %\item $\places= \set{p\in\places_1 \mid \not\exists p'\in\places_2,t\in \transitions_1,t'\in\transitions_2:\beta_1(p,t)\subset\beta_2(p',t') \land \ell_1(t)=\ell_2(t')}
%% %\cup \set{p\in\places_2 \mid \not\exists p'\in\places_1,t\in \transitions_2,t'\in\transitions_1:\beta_2(p,t)\subset\beta_1(p',t') \land \ell_2(t)=\ell_1(t')}$
%% %	\item $\transitions=\set{(t_1,t_2)\in \transitions_1\times \transitions_2\mid \ell_1(t_1)=\ell_2(t_2)}$
%% %	\item $\flow=\set{(p,(t_1,t_2))\in \places\times\transitions \mid (p,t_1)\in \flow_1 \text{ or }  (p,t_2)\in \flow_2} \cup \set{((t_1,t_2),p)\in \transitions\times\places \mid (t_1,p)\in \flow_1 \text{ or } (t_2,p)\in \flow_2}$ 
%% %	\item $\alpha(p)=\begin{cases} \alpha_1(p), & \text{ if }p\in\places_1 \\ \alpha_2(p), & \text{ if }p\in\places_2\end{cases}$
%% %	\item $\beta=\begin{cases} \beta_1(x,y), & \text{ if }x,y\in\places_1\cup\transitions_1 \\ \beta_2(x,y), & \text{ if }x,y\in\places_2\cup\transitions_2\end{cases}$
%% %	\item $\ell((t_1,t_2)):=\ell(t_1)=\ell(t_2)$
%% %\end{itemize}
%% %\end{definition}
%
%MAIN PROPERTY OF COMPOSITION: BISIMULATION RELATION
%
%
%The following proposition naturally follows from the above definitions.  %Given the full decomposition of $D$ of some t-JN, can reconstruct that very t-JN from all the t-JNs from $D$. 
%
%\begin{proposition}
%	Let $N$ be a t-JN and $dec(N)$ its full decomposition. Then $N\sim\bigoplus_{N'\in dec(N)} N'$.
%\end{proposition}
%\begin{proof}
%\end{proof}
%
%
%
%\db{I think anything after here is irrelevant for now; but I've kept it in just in case.}
%
%\begin{definition}[Dominant places]
%	Let $N=(\places, \transitions, \flow, \alpha, \beta,\ell)$ be a t-JN.
%	\andy{this definition introduces the concept of dominant places (i.e., those connected to transitions with the arcs that locally related the biggest number of objects)}
%\end{definition}
%
%How to ensure that in case of overlapping arc inscriptions, we will only consider places related to the dominating one? For that we employ Rule 3 from Section~\ref{sec:typedJN} and reformulate it as a non-dominant place reduction rule.
%
%\begin{definition}[Non-dominant place reduction]
%	Let $N=(\places, \transitions, \flow, \alpha, \beta,\ell)$ be a t-JN.
%	\andy{here we should pay attention to the fact that $\sigma$, $\mu$ and $\nu$ from Rule 3 are identifiable}
%\end{definition}
%
%\andy{here we could introduce a fixpoint algorithm (sound, complete and terminating) for the non-dominant place reduction}
%
%\begin{proposition}
%	A t-JN and its recomposition without non-dominant places are isomorphic.
%\end{proposition}
%\begin{proof}
%\end{proof}
%
%
%%% Miscellaneous comments still in the tex.
%
%% %\andy{Now discuss the synchronization operator by first giving intuition on it, and then provide a formal definition. Here we already see that the gluing the transitions without considering relations between objects leads already to a situation in which we have two subnets with $c\to d \to f$, one for $x$ and the other fro $xyz$. Now, intuitively, we would say that the $c\to d \to f$ sub-net for $x$ is subsumed by the same one for $xyz$, and thus should be simply dropped. What about other cases? What if we have for the same $x$ a subnet with $c\to d \to d' \to d'' \to f$, whereas for $xyz$ it remains as before? This would mean that we have a loop $d'\to d''$ for $x$ that ``intersects'' with the dominant behavior  $c\to d \to f$ for $xyz$. Can this even happen? }
%
%
%% %\andy{If we consider a simple composition working as a ``classical'' union, then the above result holds. If we consider dominant places (as it's already taken care of in the definition), the obtained composition of nets is going to be isomorphic to the original $N$.}