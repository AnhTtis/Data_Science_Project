\section{The $\nu$-miner} \seclabel{nu_miner}
Here we show that the Inductive Miner has nice property $\B$, which means that we can use it in the framework.




\subsection{TJN-miner under the $\mathtt{ind}$-regime}
Here we study the instantiation of the framework when $\D$ is Inductive Miner.
We call this instantiation the $\mathtt{ind}$-regime and refer to the respective TJN-miner as $\D^{\mathtt{ind}}_{id}$.


\andy{introduce process trees and Jackson types}

The following proposition\todo{or lemma?}  establishes a connection between process trees and untyped Jackson nets.

\begin{proposition}
	Process trees are untyped Jackson nets.
	\andy{reformulate}
\end{proposition}

\andy{do we need a proposition/lemma for the typed case?}

Now, we would like to recall the main properties of the inductive miner discovery algorithm.
\andy{add those properties}
We next show that the $\D_{id}$ property from Definition~\ref{def:discovery-property} holds for the original inductive miner algorithm.

\begin{proposition}
	The property from Definition~\ref{def:discovery-property} holds for Inductive Miner.
	\andy{reformulate}
\end{proposition}

Next we show a couple of rediscoverability-related results.


\begin{proposition}
	Let $N$ be a t-JN and $L$ is its $\E$-complete t-JN log.
	Then $\restr{N}{O} \sim \D_{id}^{\mathtt{ind}}(\restr{L}{O})$, for all $(n,O)\in\E$.
\end{proposition}


\begin{proposition}
	Let $N$ be a t-JN and $L$ is its $\E$-complete t-JN log.
	Then $N\sim \bigoplus_{O\in\O_\E}\D_{id}^{\mathtt{ind}}(\restr{L}{O})$.
\end{proposition}
