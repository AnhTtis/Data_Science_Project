%%%%%%%%%%%%%%%%%%%%%%%%%%%%%%%%%%%%%%%%%%%%%%%%%%%%%%%%%%%%%%%%%%%%%%%%%%%%%%%
\section{Conclusion} 
\seclabel{discussion_conclusion}
%%%%%%%%%%%%%%%%%%%%%%%%%%%%%%%%%%%%%%%%%%%%%%%%%%%%%%%%%%%%%%%%%%%%%%%%%%%%%%%
% 2023-03-17: Artem
In this paper, we studied typed Jackson Nets to model systems of interacting processes, a class of well-structured process models describing manipulations of object identifiers.
As we show, this class of nets has an important property of reconstructability.
In other words, the composition of the projections on all possible type combinations returns the model of the original system.
Ignoring the interactions between processes results in less accurate, or even wrong, models.
Similar problems occur in the discovery of systems of interacting processes, such as  object-centric process discovery, where event logs are flattened for each object.

% 2023-03-17: Artem
This paper provides a formal foundation for the composition of block-structured nets, and uses this to develop a framework for the discovery of systems of interacting processes.
We link the notion of event logs used for process discovery to system executions, and show that it is not possible to observe whether an event log is generated by a system of interacting processes, or by a projection of the system.
These properties form the key ingredients of the framework.
We show under what conditions a process discovery algorithm (that guarantees rediscoverability) can be used to discover the individual processes and their interactions, and how these can be combined to rediscover a model of interacting processes that is bisimilar to the original system that generated the event logs.

%Similar to Agent Miner~\cite{TourPKS2022agentdiscovery}, our framework applies a divide-and-conquer strategy.
%It proceeds by projecting the input event log into several event logs, one for each %combination of objects (the divide step), then discovering models from the projected event logs (the conquer step), and, finally, composing the discovered models into the overall system.
%Differently from Agent Miner, the division step is driven by objects manipulated by and not agents participating in the processes.
%Given the discovery algorithms used to conquer the projected event logs guarantee rediscoverability, the overall algorithm rediscovers the original system.
%This result is grounded in the reconstructability property of typed Jackson Nets.
%The discovered systems are live and identifier sound by construction, which are two desired correctness properties for business processes.
% 2023-03-17: Artem
Although typed Jackson Nets have less expressive power than formalisms like Object-centric Petri nets~\cite{aalstB20_discovering}, proclets~\cite{Fahland2019} or interacting artifacts~\cite{LuNWF15}, this paper shows the limitations and potential pitfalls of discovering interacting processes. 
This work aims to lay formal foundations for object-centric process discovery.
As a next step, we plan to implement the framework and tune our algorithms to discover useful models from industrial datasets.

\smallskip
\noindent
\textbf{Acknowledgements.}
Artem Polyvyanyy was in part supported by the Australian Research Council project DP220101516.

%%%%%%%%%%%%%%%%%%%%%%%%%%%%%%%%%%%%%%%%%%%%%%%%%%%%%%%%%%%%%%%%%%%%%%%%%%%%%%%




% MOVE TO DISCUSSION?
%To address this limitation, researchers have developed object-centric approaches, which incorporate information about data objects into the process model. Broadly speaking, there are two main streams of research when it comes to object-centric approaches using \pns. The first idea is to represent objects implicitly, most prominently done by proclets~\cite{proclet_paper_probably_by_dirk_in_stead_of_wil?}, where life-cycles of objects are modeled with (classical) \pns. The generation and synchronization of objects in the process is done using special ``ports'', annotated with multiplicity constraints. Correctness of proclets is still an open topic; \dbtodo{potentially due to the ports that arent defined as normal petri nets}. Objects can also be represented explicitly, as done in~\cite{deFrutosEscrig2005}, where systems comprising of various processes are modeled as a tuple of \pns, with added synchronization transitions. While objects are modeled explicitly, their life-cycle becomes implicit. The $\nu$-net~\cite{RosaVelardo2006,RosaVelardo2007} builds upon \cite{deFrutosEscrig2005} by adding name generation (generation of fresh tokens, with tokens representing objects) in a formal manner. Various models extend these $\nu$-nets, such as typed Petri nets with identifiers (\tpnids)~\cite{vanderWerf2022} and Catalog-nets~\cite{catalog_nets_citation}. \dbtodo{final sentence on correcntess and decidability}.

% MOVE TO DISCUSSION?
%None of these models are sound with respect to the objects they model. As such, in this paper, we focus on \emph{typed Jackson nets} (\tjns), a type of process model that builds upon the classical language of Petri nets but adds data-awareness. A \tjn is a \tpnid that is \emph{identifier sound} by construction~\cite{vanderWerf2022}, meaning that they are able to track the movement of data objects through the process and ensure that the data remains consistent and accurate.
%We propose a framework for discovering these \tjns. The framework uses the notion of \emph{decomposability} -- the composition of the projections of \tjns onto its identifiers give rise to a bisimilar model -- and focuses on \textit{rediscoverability}.