\section{Preliminaries} \seclabel{preliminaries}

%% SETS
Let $S$ and $T$ be two possibly infinite sets.
% Powerset + set size
%\db{I don't think we use powersets anywhere?}
The powerset of $S$ is denoted by $\powerset{S} = \set{ S' \mid S' \subseteq S}$ and $\size{S}$ denotes the cardinality of $S$.
%The cardinality of $S$ is denoted by $\size{S}$.
% Disjoint sets
Two sets $S$ and $T$ are \emph{disjoint} if $S \cap T = \emptyset$, with $\emptyset$ denoting the empty set.
% Cartesian product
The cartesian product of two sets $S$ and $T$, is defined by $S \times T = \set{(a,b) \mid a \in S, b \in T}$.
% Generalized cartesian product
The generalized cartesian product for some set $S$ and and sets $T_s$ for $s \in S$ is defined as $\Pi_{s \in S} T_s = \set{f: S \to \bigcup_{s \in S} T_s \mid \forall s \in S: f(s) \in T_s}$.
% Range
Given a relation $R \subseteq S \times T$, its range is defined by $\rng{R} = \set{y \in T \mid \exists x \in S : (x,y)\in R}$.
% Domain
Similarly, the domain of $R$ is defined by $\dom{R} = \set{x \in S \mid \exists y \in T: (x, y) \in R}$.
% Restriction
Restricting the domain of a relation to a set $U$ is defined by $\proj{R}{U} = \{(a,b) \in R \mid a \in U \}$.

%% Multiset / Bag
A \emph{multiset} $m$ over $S$ is a mapping of the form $m:S\rightarrow \naturals$, where $\naturals = \{0, 1, 2, \ldots\}$ denotes the set of natural numbers.
For $s \in S$, $m(s) \in \naturals$ denotes the number of times $s$ appears in multiset $m$. We write $s^n$ if $m(s)=n$.
For $x \not\in S$, $m(x) = 0$.
% Set of all finite multisets
We use $\setmult{S}$ to denote the set of all finite multisets over $S$ and overload $\emptyset$ to also denote the empty multiset. The size of a multiset is defined by $|m|=\sum_{s\in S}m(s)$.
% Support
The support of $m\in\setmult{S}$ is the set of elements that appear in $m$ at least once: $\setsupp{m} = \set{s\in S\mid m(s) > 0}$.
Given two multisets $m_1$ and $m_2$ over $S$:
\begin{inparaenum}[\it (i)]
    \item $m_1 \subseteq m_2$ (resp., $m_1 \subset m_2$) iff $m_1(s) \leq m_2(s)$ (resp., $m_1(s) < m_2(s)$) for each $s \in S$;
    \item $(m_1 + m_2)(s) = m_1(s) + m_2(s)$ for each $s \in S$; and
    \item if $m_1 \subseteq m_2$, $(m_2 - m_1)(s) = m_2(s) - m_1(s)$ for each $s \in S$.
\end{inparaenum}
%

%% Sequences
% Let $R : B \rightarrow S$ be an injective function. Renaming is a function $\rho_R : \setmult B \to \setmult S$ defined element-wise by $\rho_R(m)(b) = m(a)$ if $R(a) = b$ and $\rho_R(m)(b) = 0$ otherwise.
%\todo{simplify sequences when the main draft is ready}
A \emph{sequence} over $S$ of length $n \in \naturals$ is a function $\sigma : \{1,\ldots,n\} \to S$.
If $n > 0$ and $\sigma(i) = a_i$, for $1\leq i \leq n$, we write $\sigma = \sequence{a_1, \ldots, a_n}$.
The length of a sequence $\sigma$ is denoted by $\size \sigma$.
The sequence of length $0$ is called the \emph{empty sequence}, and is denoted by $\emptysequence$.
The set of all finite sequences over $S$ is denoted by $S^*$.
We write $a \in \sigma$ if there is $1 \leq i \leq \size \sigma$ such that $\sigma(i) = a$ and $\setsupp{\sigma} = \{a \in S \mid \exists 1 \leq i \leq \size \sigma : \sigma(i) = a \}$.
\emph{Concatenation} of two sequences $\nu,\gamma \in S^*$, denoted by $\sigma = \nu \concat \gamma$, is a sequence defined by $\sigma : \{ 1, \ldots, \size \nu+\size \gamma\}\rightarrow S$, such that $\sigma(i) = \nu(i)$ for $1 \leq i \leq \size \nu$, and $\sigma(i) = \gamma(i - \size \nu)$ for $\size \nu+1 \leq i \leq \size \nu+\size \gamma$.
Projection of sequences on a set $T$ is defined inductively by $\proj{\emptysequence}{T} = \emptysequence$, $\proj{(\sequence{a}\concat\sigma)}{T} = \sequence{a}\concat\proj{\sigma}{T}$ if $a \in T$ and $\proj{(\sequence{a}\concat\sigma)}{T} = \proj{\sigma}{T}$ otherwise.
%\todo{Andy: I prefer using a substitution for a sequence and then call renaming a ``special type'' of it}
Renaming a sequence with an injective function $r : S \rightarrow T$ is defined inductively by $\rho_r(\emptysequence) = \emptysequence$, and  $\rho_r(\sequence{a}\concat\sigma) = \sequence{r(a)}\concat\rho_r(\sigma)$.
Renaming is extended to multisets of sequences as follows: given a multiset $m \in \setmult{(S^*)}$, we define $\rho_r(m) = \sum_{\sigma\in\setsupp{m}} \sigma(m)\cdot\rho_r(\sigma)$.
For example, \smash{$\rho_{\{x\mapsto a, y \mapsto b\}}(\sequence{x,y}^3) = \sequence{a,b}^3$}.

% Graphs
A \emph{directed graph} is a pair $(V,A)$ where $V$ is the set of vertices, and $A \subseteq V \times V$ the set of arcs. 
Two graphs $G_1 = (V_1, A_1)$ and $G_2 = (V_2, A_2)$ are \emph{isomorphic}, denoted by $G_1 \leftrightsquigarrow G_2$, if a bijection $b : V_1 \rightarrow V_2$ exists, such that $(v_1, v_2) \in A_1$ iff $(b(v_1), b(v_2)) \in A_2$.

% LTS
Given a finite set $A$ of (action) labels, a \emph{(labeled) transition system} (LTS) over $A$ is a tuple $\transitionsystem{A} = (\states,A,\istate, \to)$, where $S$ is the (possibly infinite) set of \emph{states}, $\istate$ is the \emph{initial state} and $\to\ \subset (\states\times (A \cup \{\tau\}) \times \states)$ is the \emph{transition relation}, where $\tau\not\in A$ denotes the silent action~\cite{vanGlabbeek1993}.
In what follows, we write $\state \xrightarrow{a} \state'$ for $(\state,a,\state') \in \to$. 
%
%\jmw{Add language here?}
%
Let $r : A \to (A' \cup \{\tau\})$ be an injective, total function.
Renaming $\transitionsystem{}$ with $r$ is defined as $\rename{r}{\transitionsystem{}} = (\states, A \setminus A', \istate, \to')$ with $(\state,r(a),\state') \in \to'$ iff $(\state,a,\state') \in \to$.
Given a set $T$, hiding is defined as $\hide{T}{\transitionsystem{}} = \rename{h}{\transitionsystem{}}$ with $h : A \rightarrow A \cup \{\tau\}$ such that $h(t) = \tau$ if $t \in T$ and $h(t)=t$ otherwise.
Given $a\in A$, $p \xdasharrow{~a~} q$ denotes a \emph{weak transition relation} that is defined as follows:
\begin{inparaenum}[\it (i)]
    \item $p \xdasharrow[->]{~a~} q$ iff $p (\xrightarrow{\tau})^* q_1\xrightarrow{a}q_2 (\xrightarrow{\tau})^* q$;
    \item $p \xdasharrow[->]{\ensuremath{~\tau~}} q$ iff $p (\xrightarrow{\tau})^* q$.
\end{inparaenum}
Here, $(\xrightarrow{\tau})^*$ denotes the reflexive and transitive closure of $\xrightarrow{\tau}$.

%\db{We don't use weak bisimulation; remove!}
%\db{This would be a good place to add isomorphism, definde on LTS. We use $\equiv$ to denote it.}
%\begin{definition}[Strong and weak bisimulation] \deflabel{bisimulation}\deflabel{strong-bisimulation}\deflabel{weak-bisimulation}
    Let $\transitionsystem{1} = (\states_1, A, \state_{01}, \to_1)$ and $\transitionsystem{2} = (\states_2, A, \state_{02},\to_2)$ be two LTSs.
    A relation $R\subseteq(\states_1 \times \states_2)$ is called a \emph{strong simulation}, denoted as $\transitionsystem{1} \prec_R \transitionsystem{2}$, if for every pair $(p,q)\in R$ and $a\in A \cup \{\tau\}$, it holds that if $p\xrightarrow{a}_1 p'$, then there exists $q'\in\states_2$ such that $q \xrightarrow{a}_2 q'$ and $(p',q')\in R$. Relation $R$ is a \emph{weak simulation}, denoted by $\transitionsystem{1} \preccurlyeq_R \transitionsystem{2}$, iff  for every pair $(p,q)\in R$ and $a\in A \cup \{\tau\}$ it holds that if $p\xrightarrow{a}_1 p'$, then $a = \tau$ and $(p', q) \in R$, or there exists $q'\in\states_2$ such that $q \xdasharrow{~a~}_{\hspace{-.5ex} 2}~q'$ and $(p',q')\in R$.
%
    Relation $R$ is called a strong (weak) \emph{bisimulation}, denoted by $\transitionsystem{1} \sim_R \transitionsystem{2}$ ($\transitionsystem{1} \approx_R \transitionsystem{2}$) if both $\transitionsystem{1} \prec \transitionsystem{2}$ ($\transitionsystem{1} \preccurlyeq_R \transitionsystem{2}$) and $\transitionsystem{2} \prec_{R^{-1}} \transitionsystem{1}$ ($\transitionsystem{2} \preccurlyeq_{R^{-1}} \transitionsystem{1}$).
    Given a strong (weak) (bi)simulation $R$, we say that a state $p\in\states_1$ is strongly (weakly) rooted (bi)similar to $q\in\states_2$, written $p \sim^r_R  q$ (correspondingly, $p \approx^r_R q$), if  $(p,q)\in R$.
    The relation is called \emph{rooted} iff $(s_{01}, s_{02}) \in R$. A rooted relation is indicated with a superscript $^r$.
%\end{definition}

%Finally, $\transitionsystem{1}$ is said to be strongly (weakly) bisimilar to $\transitionsystem{2}$, written $\transitionsystem{1}  \sim \transitionsystem{2}$ (correspondingly, $\transitionsystem{1}  \approx \transitionsystem{2}$), if $s_{01}  \sim s_{02}$ ($s_{01}  \approx s_{02}$).

% Definition Petri net
A weighted \pn is a 4-tuple $(\places,\transitions,\flow, W)$ where
$\places$ and $\transitions$ are two disjoint sets of \emph{places} and \emph{transitions}, respectively, $\flow \subseteq ((\places \times \transitions) \cup (\transitions \times \places))$ is the \emph{flow relation}, and $W : \flow \rightarrow \naturals^+$ is a \emph{weight function}.
For $x\in\places\cup\transitions$, we write $\pre{x}=\set{y\mid (y,x)\in\flow}$ to denote the \emph{preset} of $x$ and $\post{x}=\set{y\mid (x,y)\in\flow}$ to denote the \emph{postset} of $x$.
We lift the notation of preset and postset to sets element-wise.
If for a \pn no weight function is defined, we assume $W(f) = 1$ for all $f \in F$.
% Definition marking
A \emph{marking} of $N$ is a multiset $m \in \setmult P$, where $m(p)$ denotes the number of \emph{tokens} in place $\place \in \places$. If $m(\place) > 0$, place $\place$ is called \emph{marked} in marking $m$.
A \emph{marked \pn} is a tuple \markednet{N}{m} with $N$ a weighted \pn with marking $m$.
% Enabledness
A transition $\transition\in \transitions$ is enabled in $\markednet{N}{m}$, denoted by $\pnenabled{\markednet{N}{m}}{\transition}$ iff $W((\place,\transition)) \leq m(\place)$ for all $\place \in \pre{\transition}$. An enabled transition can \emph{fire}, resulting in marking $m'$ iff $m'(\place) + W((\place, \transition)) = m(\place) + W((\transition, \place))$, for all $\place\in\places$, and is denoted by $\fire{\markednet{N}{m}}{\transition}{\markednet{N}{m'}}$. We lift the notation of firings to sequences. A sequence $\sigma \in T^*$ is a \emph{firing sequence} iff $\sigma = \emptysequence$, or markings $m_0, \ldots, m_n$ exist such that $\fire{(N,m_{i-1})}{\sigma(i)}{(N,m_{i})}$ for $1 \leq i \leq \size \sigma = n$, and is denoted by $\fire{(N,m_0)}{\sigma}{(N,m_n)}$. If the context is clear, we omit the weighted Petri net $N$.
The set of reachable markings of \markednet{N}{m} is defined by $\pnreachable{N}{m} = \{ m' \mid \exists \sigma \in \transitions^* : \fire{m}{\sigma}{m'} \}$.
The set of all possible finite firing sequences of  \markednet{N}{m} is denoted by $\mathcal{L}(N,m_0) = \{ \sigma \in \seq{T} \mid \fire{m}{\sigma}{m'} \}$.
The semantics of a marked Petri net $(N,m)$ with $N = (\places, \transitions, \flow, W)$ is defined by the LTS $\Gamma_{N,m} = (\setmult \places, T, m_0, \to)$ with $(m,t,m') \in \to$ iff $\fire{m}{t}{m'}$.
%Isomorphism on petri nets
A Petri net $N = (P, T, F, W)$ has underlying graph $(P \cup T, F)$.
Two Petri nets $N$ and $N'$ are isomorphic, denoted using $N \leftrightsquigarrow N'$, if their underlying graphs are.

% Workflow net
A \emph{workflow net} (WF-net for short) is a tuple $N=(\places, \transitions, \flow,W,\inp,\outp)$ such that:
\begin{inparaenum}[\it (i)]
	\item $(\places, \transitions, \flow, W)$ is a weighted Petri net;
	\item $\inp,\outp\in\places$ are the source and sink place, respectively, with $\pre{\inp} =\post{\outp}= \emptyset$;
	\item every node in $\places \cup \transitions$ is on a directed path from $\inp$ to $\outp$.
\end{inparaenum}
$N$ is called \emph{$k$-sound} for some $k \in \naturals$ iff
\begin{inparaenum}[\it (i)]
	\item it is proper completing, i.e., for all reachable markings $m \in \pnreachable{N}{[\inp^k]}$, if $[\outp^k]\subseteq m$, then $m = [\outp^k]$;
	\item it is weakly terminating, i.e., for any reachable marking $m \in \pnreachable{N}{[\inp^k]}$, the final marking is reachable, i.e., $[\outp^k] \in \pnreachable{N}{m}$; and
	\item it is quasi-live, i.e., for all transitions $t\in\transitions$, there is a marking $m\in\pnreachable{N}{[\inp]}$ such that $\pnenabled{m}{t}$.
\end{inparaenum}
The net is called \emph{sound} if it is $1$-sound.
If it is $k$-sound for all $k\in\naturals$, it is called \emph{generalized sound}~\cite{vanHee2003}.

%\db{@Daniel: For later, potentially we need to add something about object centric logs here.}
