%%% This macros file has been compiled by D. Barenholz
%%% Using sources from D. Barenholz, A. Rivkin, J.M.E.M van der Werf,
%%% and potentially more authors.

%%% Font Setup
\newcommand{\mathname}[1]{\ensuremath{\mathit{#1}}}
\newcommand{\setname}{\mathname}
\newcommand{\propername}[1]{\text{\textsf{\small #1}}\xspace}
\newcommand{\variablename}[1]{\ensuremath{\mathtt{#1}}}
\newcommand{\functionname}{\variablename}
\newcommand{\funsym}[1]{\ensuremath{\mathtt{#1}}}
\newcommand{\setsym}[1]{\ensuremath{\mathit{#1}}}
\newcommand{\cname}[1]{\ensuremath{\mathtt{#1}}\xspace} % constant name
\newcommand{\typename}[1]{\ensuremath{\mathtt{#1}}}
\newcommand{\concept}[1]{\textit{#1}}
\newcommand{\attribute}[1]{\texttt{#1}}


\newcommand{\A}{\ensuremath{\mathcal{A}}\xspace}
\newcommand{\B}{\ensuremath{\mathcal{B}}\xspace}
\newcommand{\C}{\ensuremath{\mathcal{C}}\xspace}
\newcommand{\D}{\ensuremath{\mathcal{D}}\xspace}
\newcommand{\E}{\ensuremath{\mathcal{E}}\xspace}
\newcommand{\F}{\ensuremath{\mathcal{F}}\xspace}
\newcommand{\G}{\ensuremath{\mathcal{G}}\xspace}
\renewcommand{\H}{\ensuremath{\mathcal{H}}\xspace}
\newcommand{\I}{\ensuremath{\mathcal{I}}\xspace}
\newcommand{\J}{\ensuremath{\mathcal{J}}\xspace}
\newcommand{\K}{\ensuremath{\mathcal{K}}\xspace}
\renewcommand{\L}{\ensuremath{\mathcal{L}}\xspace}
\newcommand{\M}{\ensuremath{\mathcal{M}}\xspace}
\newcommand{\N}{\ensuremath{\mathcal{N}}\xspace}
\renewcommand{\O}{\ensuremath{\mathcal{O}}\xspace}
\renewcommand{\P}{\ensuremath{\mathcal{P}}\xspace}
\newcommand{\Q}{\ensuremath{\mathcal{Q}}\xspace}
\newcommand{\R}{\ensuremath{\mathcal{R}}\xspace}
\renewcommand{\S}{\ensuremath{\mathcal{S}}\xspace}
\newcommand{\T}{\ensuremath{\mathcal{T}}\xspace}
\newcommand{\U}{\ensuremath{\mathcal{U}}\xspace}
\newcommand{\V}{\ensuremath{\mathcal{V}}\xspace}
\newcommand{\W}{\ensuremath{\mathcal{W}}\xspace}
\newcommand{\X}{\ensuremath{\mathcal{X}}\xspace}
\newcommand{\Y}{\ensuremath{\mathcal{Y}}\xspace}
\newcommand{\Z}{\ensuremath{\mathcal{Z}}\xspace}

%%% Overline and Underline
\newcommand{\ol}[1]{\ensuremath{\overline{#1}}}
\newcommand{\ul}[1]{\ensuremath{\underline{#1}}}

%%% Drawing arrows
\newcommand{\ra}{\ensuremath{\rightarrow}}
\newcommand{\Ra}{\ensuremath{\Rightarrow}}
\newcommand{\la}{\ensuremath{\leftarrow}}
\newcommand{\La}{\ensuremath{\Leftarrow}}
\newcommand{\lra}{\ensuremath{\leftrightarrow}}
\newcommand{\Lra}{\ensuremath{\Leftrightarrow}}
\renewcommand{\iff}{\Lra}
\newcommand{\lora}{\ensuremath{\longrightarrow}}
\newcommand{\Lora}{\ensuremath{\Longrightarrow}}
\newcommand{\lola}{\ensuremath{\longleftarrow}}
\newcommand{\Lola}{\ensuremath{\Longleftarrow}}
\newcommand{\lolra}{\ensuremath{\longleftrightarrow}}
\newcommand{\Lolra}{\ensuremath{\Longleftrightarrow}}
\newcommand{\ua}{\ensuremath{\uparrow}}
\newcommand{\Ua}{\ensuremath{\Uparrow}}
\newcommand{\da}{\ensuremath{\downarrow}}
\newcommand{\Da}{\ensuremath{\Downarrow}}
\newcommand{\uda}{\ensuremath{\updownarrow}}
\newcommand{\Uda}{\ensuremath{\Updownarrow}}

% \newcommand{\xdasharrow}[2][->,>=angle 90]{
%     % correct vertical setting by egreg:
%     % http://tex.stackexchange.com/a/59660/13304
%     \tikz[baseline=-\the\dimexpr\fontdimen22\textfont2\relax]{
%         \node[anchor=south,font=\scriptsize, inner ysep=1.5pt,outer xsep=2.5pt](x){\ensuremath{#2}};
%         \draw[shorten <=3.4pt,shorten >=3.4pt,dashed,#1](x.south west)--(x.south east);
%     }
% }
\makeatletter
\newcommand{\xdasharrow}[2][->,>=angle 90]{\tikz[baseline=-\the\dimexpr\fontdimen22\textfont2\relax]{\node[anchor=south,font=\scriptsize, inner ysep=1.5pt,outer xsep=2.5pt](x){\ensuremath{#2}};\draw[shorten <=3.4pt,shorten >=3.4pt,dashed,#1](x.south west)--(x.south east);}}
\makeatother

%%% Delimiters
\newcommand{\quotes}[1]{{\lq\lq #1\rq\rq}}
\newcommand{\Quotes}{\quotes}
\newcommand{\singlequote}[1]{{\lq\ #1\rq}}
\newcommand{\Singlequote}{\quotes}
\newcommand{\set}[1]{\ensuremath{\left\{#1\right\}}}
\newcommand{\Set}{\set}
\newcommand{\multiset}[1]{\ensuremath{\left[#1\right]}}
\newcommand{\Multiset}{\multiset}
\newcommand{\bag}{\multiset}
\newcommand{\Bag}{\bag}
\newcommand{\size}[1]{\ensuremath{\left|{#1}\right|}}
\newcommand{\Size}{\size}
\newcommand{\sequence}[1]{\ensuremath{\left\langle #1\right\rangle}}
\newcommand{\Sequence}{\sequence}
\newcommand{\tuple}[1]{\ensuremath{\left( #1\right)}}
\newcommand{\tup}[1]{\ensuremath{\left\langle #1\right\rangle}}
\newcommand{\Tuple}{\tuple}

%%% Common Sets
\newcommand{\naturals}{\ensuremath{\mathbb{N}}\xspace}
\newcommand{\Naturals}{\naturals}
\newcommand{\integers}{\ensuremath{\mathbb{Z}}\xspace}
\newcommand{\Integers}{\integers}
\newcommand{\fractions}{\ensuremath{\mathbb{Q}}\xspace}
\newcommand{\Fractions}{\fractions}
\newcommand{\reals}{\ensuremath{\mathbb{R}}\xspace}
\newcommand{\Reals}{\reals}
\newcommand{\complex}{\ensuremath{\mathbb{C}}\xspace}
\newcommand{\Complex}{\complex}
\newcommand{\booleans}{\ensuremath{\mathbb{B}}\xspace}
\newcommand{\Booleans}{\booleans}
\newcommand{\true}{\ensuremath{\mathsf{true}}\xspace}
\newcommand{\false}{\ensuremath{\mathsf{false}}\xspace}
\newcommand{\strings}{\ensuremath{\mathbb{S}}\xspace}
\newcommand{\Strings}{\strings}

%%% Sets
%% We already have these:
% \newcommand{\set}[1]{\ensuremath{\left\{#1\right\}}}
% \newcommand{\Set}{\set}
% \newcommand{\size}[1]{\ensuremath{\left|{#1}\right|}}
% \newcommand{\Size}{\size}
\newcommand{\powerset}[1]{\ensuremath{\mathcal{P}(#1)}\xspace}
\newcommand{\union}{\ensuremath{\cup}\xspace}
\newcommand{\intersect}{\ensuremath{\cap}\xspace}
\newcommand{\Union}{\ensuremath{\bigcup}\xspace}
\newcommand{\Intersect}{\ensuremath{\bigcap}\xspace}
\newcommand{\setmult}[1]{\ensuremath{#1^\oplus}\xspace}
\newcommand{\setsupp}[1]{\ensuremath{\mathit{supp}\left(#1\right)}}
\newcommand{\setsuppp}[1]{\ensuremath{\mathit{supp}(#1)}}
\newcommand{\genprod}[2]{\ensuremath{\prod_{#1}#2}\xspace}


%%% Multisets (Bags)
%% We already have these:
% \newcommand{\multiset}[1]{\ensuremath{\left[#1\right]}}
% \newcommand{\Multiset}{\multiset}
% \newcommand{\bag}{\multiset}
% \newcommand{\Bag}{\bag}
\newcommand{\bunion}{+}
\newcommand{\bminus}{-}
\newcommand{\emptybag}{\ensuremath{\emptyset}}
\newcommand{\bagset}[1]{\ensuremath{\naturals^{#1}}}

%%% Functions, Relations
%\newcommand{\func}[1]{\ensuremath{\mathit{#1}}}
\newcommand{\func}[1]{\ensuremath{ #1 }}
\newcommand{\partialfuncarrow}{\ensuremath{\rightharpoonup}}
\newcommand{\emptyfunction}{\ensuremath{\emptyset}}
\newcommand{\dom}[1]{\textsc{dom}(\func{#1})}
\newcommand{\Dom}{\dom}
\newcommand{\img}[1]{\textsc{img}(\func{#1})}
\newcommand{\Img}{\img}
\newcommand{\rng}[1]{\textsc{rng}(\func{#1})}
\newcommand{\Rng}{\rng}
\newcommand{\inverse}[1]{\ensuremath{{#1}^{-1}}}
\newcommand{\rel}[3]{\ensuremath{{#2}\,{#1}\,{#3}}}
\newcommand{\bottom}{\ensuremath{\bot}}
\newcommand{\charfunc}{\ensuremath{\chi}}

%%% Sequences
%% We already have these:
% \newcommand{\sequence}[1]{\ensuremath{\left\langle #1\right\rangle}}
% \newcommand{\Sequence}{\sequence}
\newcommand{\concat}{\cdot}
\newcommand{\seq}[1]{\ensuremath{{#1}^{*}}}
\newcommand{\emptysequence}{\ensuremath{\epsilon}}
\newcommand{\weave}[2]{\ensuremath{\left(#1 \parallel #2\right)}}
\newcommand{\parikh}[1]{\ensuremath{\overrightarrow{#1}}}
\newcommand{\length}[1]{\size{#1}}
\newcommand{\subsequence}[3]{\ensuremath{#1_{[#2..#3]}}}

%% Graphs
\newcommand{\graphunion}{\union}
\newcommand{\graphintersect}{\intersect}
\newcommand{\graphminus}{\ensuremath{\setminus}}

%% LTS
\newcommand{\transitionsystem}[1]{\ensuremath{\Gamma_{#1}}\xspace}
\newcommand{\states}{\ensuremath{S}\xspace}
\newcommand{\istate}{\ensuremath{s_0}\xspace}
\newcommand{\state}{\ensuremath{s}\xspace}
% \newcommand{\hide}[1]{\ensuremath{\funsym{\hat H}_{#1}}}

\newcommand{\lang}[1]{\ensuremath{\mathcal{L}(#1)}}
\newcommand{\traces}[1]{\ensuremath{\mathcal{T}(#1)}}
\newcommand{\lts}[1]{\ensuremath{\mathcal{N}(#1)}}
\newcommand{\reach}[1]{\ensuremath{\mathcal{R}(#1)}}

\newcommand{\renameop}[1]{\ensuremath{\rho_{#1}}}
\newcommand{\rename}[2]{\ensuremath{\renameop{#1}(#2)}}
\newcommand{\hideop}[1]{\ensuremath{\hat{\cname{H}}_{#1}}}
\newcommand{\hide}[2]{\ensuremath{\hideop{#1}(#2)}}

\newcommand{\myfire}[4]{\ensuremath{(#1 : #2 #3 #4) }}
\newcommand{\mydoublefire}[6]{\ensuremath{(#1 : #2 #3 #4 #5 #6) }}
\newcommand{\mydotfire}[9]{\ensuremath{(#1 : #2 #3 #4 #5 \ldots #6 #7 #8 #9) }}

\newcommand{\mayArrow}[1]{\ensuremath{\stackrel{#1}{\dashrightarrow}}}
\newcommand{\MayArrow}[1]{\ensuremath{\mathop{\stackrel{#1}{= \Rightarrow}}}}
\newcommand{\mayFire}[4]{\myfire{#1}{#2}{\mayArrow{#3}}{#4}}
\newcommand{\MayFire}[4]{\myfire{#1}{#2}{\MayArrow{#3}}{#4}}

\newcommand{\exitArrow}[1]{\ensuremath{\stackrel{#1}{\longrightarrow}}}
\newcommand{\ExitArrow}[1]{\ensuremath{\mathop{\stackrel{#1}{ \Longrightarrow}}}} % ignore warning - this is correct
\newcommand{\exitFire}[4]{\myfire{#1}{#2}{\exitArrow{#3}}{#4}}
\newcommand{\ExitFire}[4]{\myfire{#1}{#2}{\ExitArrow{#3}}{#4}}

\newcommand{\smallarrow}[1]{\ensuremath{\stackrel{{#1}}{\rightarrow}}}
\newcommand{\arrow}[1]{\ensuremath{\stackrel{{#1}}{\longrightarrow}}}
\newcommand{\Arrow}[1]{\ensuremath{\stackrel{{#1}}{\Longrightarrow}}}

\newcommand{\smallenabled}[3]{\ensuremath{(#1 : #2 \smallarrow{#3})}}
\newcommand{\enabled}[3]{\ensuremath{(#1 : #2 \arrow{#3})}}
\newcommand{\smallfires}[4]{\ensuremath{(#1~: {#2} \smallarrow{#3} {#4})}}
\newcommand{\smalldoublefires}[6]{\ensuremath{(#1 : #2 \smallarrow{#3} #4 \smallarrow{#5} #6 )}}
\newcommand{\doublesinglefire}[6]{\ensuremath{(#1 : #2 \Arrow{#3} #4 \arrow{#5} #6 )}}
\newcommand{\doublesingledoublefire}[8]{\ensuremath{(#1 : #2 \Arrow{#3} #4 \arrow{#5} #6 \Arrow{#7} #8 )}}
\newcommand{\fires}[4]{\ensuremath{(#1 : #2 \arrow{#3} #4)}}
\newcommand{\doublefires}[6]{\ensuremath{(#1 : #2 \arrow{#3} #4 \arrow{#5} #6 )}}
\newcommand{\triplefires}[8]{\ensuremath{(#1 : #2 \arrow{#3} #4 \arrow{#5} #6 \arrow{#7} #8)}}
\newcommand{\Fires}[4]{\ensuremath{(#1 : #2 \Arrow{#3} #4)}}
\newcommand{\Doublefires}[6]{\ensuremath{(#1 : #2 \Arrow{#3} #4 \Arrow{#5} #6 )}}
\newcommand{\Triplefires}[8]{\ensuremath{(#1 : #2 \Arrow{#3} #4 \Arrow{#5} #6 \Arrow{#7} #8)}}
\newcommand{\dotfires}[9]{\ensuremath{(#1 : #2 \arrow{#3} #4 \arrow{#5} \ldots \arrow{#6} #7 \arrow{#8} #9 )}}
\newcommand{\Dotfires}[9]{\ensuremath{(#1 : #2 \Arrow{#3} #4 \Arrow{#5} \ldots \Arrow{#6} #7 \Arrow{#8} #9 )}}

\newcommand{\strongsim}[3]{\ensuremath{#1\ {}_s\!\!\preceq_{#2} #3}}
\newcommand{\strongbisim}[3]{\ensuremath{#1\ {}_s\!\!\simeq_{#2} #3}}
\newcommand{\weaksim}[3]{\ensuremath{#1\ {}_w\!\!\preceq_{#2} #3}}
\newcommand{\weakbisim}[3]{\ensuremath{#1\ {}_w\!\!\simeq_{#2} #3}}
\newcommand{\delaysim}[3]{\ensuremath{#1\ \preceq_{#2} #3}}
\newcommand{\delaybisim}[3]{\ensuremath{#1\ \simeq_{#2} #3}}
\newcommand{\isomorphic}[3]{\ensuremath{#2 \cong_{#1} #3}}

\newcommand{\algequiv}{\ensuremath \equiv_{alg}}

%% Petri nets
\newcommand{\pn}{Petri net\xspace}
\newcommand{\pns}{Petri nets\xspace}
\newcommand{\wfnet}{WF-net\xspace}
\newcommand{\wfnets}{WF-nets\xspace}

\newcommand{\net}[1]{\ensuremath{#1 = (P, T, F)}}
\newcommand{\weightednet}[1]{\ensuremath{#1 = (P, T, F, W)}}
\newcommand{\markednet}[2]{\ensuremath{\left(#1, #2\right)}}
\newcommand{\marking}{\ensuremath{m}}
\newcommand{\initialmarking}{\ensuremath{m_0}}
\newcommand{\markings}[1]{\ensuremath{\mathbb{M}\left(#1\right)}}

\newcommand{\places}{\ensuremath{P}}
\newcommand{\place}{\ensuremath{\mathit{p}\xspace}}
\newcommand{\transitions}{\ensuremath{T}}
\newcommand{\transition}{\ensuremath{\mathit{t}\xspace}}
\newcommand{\silenttransition}{\ensuremath{\tau}\xspace}
\newcommand{\flow}{\ensuremath{F}}
\newcommand{\arc}{\ensuremath{a}}
\newcommand{\weights}{\ensuremath{W}}

\newcommand{\pre}[1]{\ensuremath{{}^\bullet}{#1}}
\newcommand{\post}[1]{\ensuremath{{#1}^{\bullet}}}
\newcommand{\postnet}[2]{\ensuremath{{#2}^{\bullet}_{#1}}}
\newcommand{\prenet}[2]{\ensuremath{{\,{}_{#1}^{\,\,\,\bullet}}{#2}}}

\newcommand{\netunion}{\ensuremath{\cup}}

\newcommand{\firesym}[1]{\ensuremath{[#1\rangle}}
\newcommand{\fire}[3]{\ensuremath{#1 \firesym{#2} #3}}
% \newcommand{\pnenabled}[3]{\ensuremath{(#1,#2)\firesym{#3}}}
\newcommand{\pnenabled}[2]{\ensuremath{#1\firesym{#2}}}
\newcommand{\pnfires}[4]{\ensuremath{(#1,#2)\firesym{#3}(#1,#4)}}
\newcommand{\pnreachable}[2]{\ensuremath{\mathcal{R}(#1,#2)}}


%% Proving
\newcommand{\qedboxfull}{\vrule height 5pt width 5pt depth 0pt}
\newcommand{\qedfull}{\hfill{\qedboxfull}}

%% Universes
\newcommand{\eventuniverse}{\ensuremath{\mathcal{E}}}
\newcommand{\caseuniverse}{\ensuremath{\mathcal{C}}}
\newcommand{\eventloguniverse}{\ensuremath{\mathcal{L}}}
\newcommand{\loguniverse}{\eventloguniverse}
\newcommand{\attributeuniverse}{\ensuremath{\mathcal{A}}}
\newcommand{\valueuniverse}{\ensuremath{\mathcal{V}}}
\newcommand{\contextmodeluniverse}{\ensuremath{\mathcal{I}}}

%% Event logs
\newcommand{\attrdef}{\ensuremath{{\#}}}
\newcommand{\attr}[2]{\ensuremath{\attrdef_{\textrm{#1}}(#2)}}
\newcommand{\notdefined}{\ensuremath{\bot}}

\newcommand{\traceattrdef}{\ensuremath{\textrm{trace}}}
\newcommand{\traceattr}[1]{\ensuremath{\attr{trace}{#1}}}
\newcommand{\tracesof}[1]{\ensuremath{\hat{#1}}}
\newcommand{\caseattrdef}{\ensuremath{\textrm{case}}}
\newcommand{\caseattr}[1]{\ensuremath{\attr{case}{#1}}}
\newcommand{\firstattrdef}{\ensuremath{\textrm{first}}}
\newcommand{\firstattr}[1]{\ensuremath{\attr{first}{#1}}}
\newcommand{\lastattrdef}{\ensuremath{\textrm{last}}}
\newcommand{\lastattr}[1]{\ensuremath{\attr{last}{#1}}}
\newcommand{\nextattrdef}{\ensuremath{\textrm{next}}}
\newcommand{\nextattr}[1]{\ensuremath{\attr{next}{#1}}}

%% Typed Petri nets with Identifiers
\newcommand{\tpnid}{t-PNID\xspace}
\newcommand{\tpnids}{t-PNIDs\xspace}
\newcommand{\jn}{JN\xspace}
\newcommand{\jns}{JNs\xspace}
\newcommand{\tjn}{t-JN\xspace}
\newcommand{\tjns}{t-JNs\xspace}
\newcommand{\id}{\func{id}\xspace}
\newcommand{\Id}{\func{Id}\xspace}
\newcommand{\type}{\funsym{type}\xspace}

\newcommand{\godelsymb}{\ensuremath \mathscr{G}\xspace}
\newcommand{\godel}[1]{\ensuremath \godelsymb\left( #1 \right)}

\newcommand{\invar}[1]{{\setsym{In}({#1})}}
\newcommand{\outvar}[1]{{\setsym{Out}({#1})}}
\newcommand{\var}[1]{{\setsym{Var}({#1})}}
\newcommand{\newvar}[1]{{\setsym{Emit}({#1})}}
\newcommand{\delvar}[1]{{\setsym{Collect}({#1})}}
\newcommand{\inp}{\ensuremath{\mathit{in}}}
\newcommand{\outp}{\ensuremath{\mathit{out}}}

\newcommand{\idset}{\I\xspace}
\newcommand{\varset}{\V\xspace}
\newcommand{\labelset}{\ensuremath{\Lambda}\xspace}
\newcommand{\colset}{\funsym{C}}

\newcommand{\restr}[2]{\left.#1\right|_{#2}}


% Sequences
\newcommand{\proj}[2]{\ensuremath{{#1}_{{\mid {#2}}}}}
\newcommand{\projection}{\proj}
\newcommand{\projsub}[3]{\ensuremath{{{#1}}_{#2\mid{#3}}}}

% Type projection
\newcommand{\project}[2]{\ensuremath{\pi_{#1}\left(#2\right)}}
\newcommand{\projectClosure}[2]{\ensuremath{\pi^+_{#1}\left(#2\right)}}

\newcommand{\composeOperator}{\ensuremath{\uplus}}
\newcommand{\compose}[2]{\ensuremath{#1 \composeOperator #2}}
\newcommand{\composed}[1]{\ensuremath{\composeOperator{#1}}}
\newcommand{\Composed}[1]{\ensuremath{\biguplus_{#1}}}

% Process trees
\newcommand{\ptsequenceop}{\ensuremath \rightarrow}
\newcommand{\ptselfloopop}{\ensuremath \circlearrowleft}
\newcommand{\ptparallelop}{\ensuremath \wedge}
\newcommand{\ptchoiceop}{\ensuremath \times}

\newcommand{\ptsequence}[1]{\ensuremath \ptsequenceop\left( #1 \right)}
\newcommand{\ptselfloop}[2]{\ensuremath \ptselfloopop \left( #1 , #2 \right)}
\newcommand{\ptparallel}[1]{\ensuremath \ptparallelop \left( #1 \right)}
\newcommand{\ptchoice}[1]{\ensuremath \ptchoiceop \left( #1  \right)}

% Jackson Types
\newcommand{\jnsequenceop}{\ensuremath ; }
\newcommand{\jnparallelop}{\ensuremath \parallel}
\newcommand{\jnchoiceop}{\ensuremath + }
\newcommand{\jnselfloopop}{\ensuremath \# }

\newcommand{\jnsequence}[2]{\ensuremath \left( #1 \jnsequenceop #2 \right) }
\newcommand{\jnparallel}[2]{\ensuremath \left( #1 \jnparallelop #2 \right)}
\newcommand{\jnchoice}[2]{\ensuremath \left( #1 \jnchoiceop #2 \right) }
\newcommand{\jnselfloop}[2]{\ensuremath \left( #1 \jnselfloopop #2 \right) }
\newcommand{\jnplace}[2]{\ensuremath \left[#1, #2 \right]}

% Typed Jackson Nets
\newcommand{\jnnewidentifierop}{\ensuremath \triangleleft}
\newcommand{\jnnewidentifier}[4]{\ensuremath \left( #1 \jnnewidentifierop \tuple{#2, #3, #4}\right)}

%%% Styling and spacing
\newcommand{\inlinetitle}[1]{\smallskip\noindent\textbf{#1.}\xspace}
\makeatletter
\g@addto@macro\normalsize{%ignore warning
    \setlength{\abovecaptionskip}{-2pt}
    \setlength{\belowcaptionskip}{12pt}
    \setlength\abovedisplayskip{3pt}
    \setlength\belowdisplayskip{3pt}
    \setlength\abovedisplayshortskip{3pt}
    \setlength\belowdisplayshortskip{3pt}
}

%%% Labels
%% Ignore warnings: just don't add spaces when defining labels
\newcommand{\figlabel}[1]{\label{fig:#1}}
\newcommand{\tbllabel}[1]{\label{tbl:#1}}
\newcommand{\deflabel}[1]{\label{def:#1}}
\newcommand{\thmlabel}[1]{\label{thm:#1}}
\newcommand{\lmlabel}[1]{\label{lm:#1}}
\newcommand{\corlabel}[1]{\label{cor:#1}}
\newcommand{\proplabel}[1]{\label{prop:#1}}
\newcommand{\seclabel}[1]{\label{sec:#1}}
\newcommand{\chaplabel}[1]{\label{chap:#1}}
\newcommand{\aplabel}[1]{\label{ap:#1}}
\newcommand{\algolabel}[1]{\label{algo:#1}}
\newcommand{\eqlabel}[1]{\label{eq:#1}}
\newcommand{\tablabel}[1]{\label{tab:#1}}
\newcommand{\partlabel}[1]{\label{part:#1}}
\newcommand{\exlabel}[1]{\label{ex:#1}}
\newcommand{\rklabel}[1]{\label{rk:#1}}
\newcommand{\asslabel}[1]{\label{ass:#1}}
\newcommand{\prflabel}[1]{\label{prf:#1}}

\newcommand{\figref}[1]{Fig.~\ref{fig:#1}}
\newcommand{\tblref}[1]{Tbl.~\ref{tbl:#1}}
\newcommand{\defref}[1]{Def.~\ref{def:#1}}
\newcommand{\thmref}[1]{Thm.~\ref{thm:#1}}
\newcommand{\lmref}[1]{Lm.~\ref{lm:#1}}
\newcommand{\corref}[1]{Cor.~\ref{cor:#1}}
\newcommand{\propref}[1]{Prop.~\ref{prop:#1}}
\newcommand{\secref}[1]{Section~\ref{sec:#1}}
\newcommand{\chapref}[1]{Chap.~\ref{chap:#1}}
\newcommand{\apref}[1]{Appendix~\ref{ap:#1}}
\newcommand{\algoref}[1]{Algo.~\ref{algo:#1}}
\newcommand{\tabref}[1]{Tbl.~\ref{tab:#1}}
\newcommand{\partref}[1]{Part~\ref{part:#1}}
\newcommand{\exref}[1]{Ex~\ref{ex:#1}}
\newcommand{\rkref}[1]{Rk~\ref{rk:#1}}
\newcommand{\assref}[1]{Ass~\ref{ass:#1}}
\newcommand{\prfref}[1]{Prf~\ref{prf:#1}}

\newcommand{\figreffull}[1]{Figure~\ref{fig:#1}}
\newcommand{\tblreffull}[1]{Table~\ref{tbl:#1}}
\newcommand{\defreffull}[1]{Definition~\ref{def:#1}}
\newcommand{\thmreffull}[1]{Theorem~\ref{thm:#1}}
\newcommand{\lmreffull}[1]{Lemma~\ref{lm:#1}}
\newcommand{\correffull}[1]{Corollary~\ref{cor:#1}}
\newcommand{\propreffull}[1]{Property~\ref{prop:#1}}
\newcommand{\secreffull}[1]{Section~\ref{sec:#1}}
\newcommand{\chapreffull}[1]{Chapter~\ref{chap:#1}}
\newcommand{\apreffull}[1]{Appendix~\ref{ap:#1}}
\newcommand{\algoreffull}[1]{Algorithm~\ref{algo:#1}}
\newcommand{\tabreffull}[1]{Table~\ref{tab:#1}}
\newcommand{\partreffull}[1]{Part~\ref{part:#1}}
\newcommand{\exreffull}[1]{Example~\ref{ex:#1}}
\newcommand{\rkreffull}[1]{Remark~\ref{rk:#1}}
\newcommand{\assreffull}[1]{Assumption~\ref{ass:#1}}
\newcommand{\prfreffull}[1]{Proof~\ref{prf:#1}}
