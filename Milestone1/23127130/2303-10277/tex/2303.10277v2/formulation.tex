

\subsection{Safety index synthesis}

Consider a control affine system~\footnote{Although our method assumes control affine dynamics, it is applicable to non-control affine systems, since we can always have a control affine form through dynamics extension~\cite{liu2016algorithmic}.}
\begin{align}
    (\Sigma_1) \quad & \dot x = f(x) + g(x) u\label{eq:x_dyn}
\end{align}
where $x \in X \subseteq \real^{n_x}$, $u \in U \subseteq \real^{n_u}$. We assume $U$ is a polytope, which is a common case in practice.
\begin{asp} [Polytope U]
    The control limits $U$ of the concrete system is a polytope: $U = \{u\mid A u \leq b\}$.
\end{asp}

A user-defined safety specification $\phi_0(x):\real^n \to \real$ is a continuous function and implicitly defines a connected and closed set $\XX_s := \{x\mid\phi_0(x)\leq0\}$ called the safe set. We are interested in keeping the state in a subset of the user-defined safe set: $S \subseteq \XX_s$. The problem can be expressed as a forward invariance problem: 
\begin{align}
   x(t_0) \in S \implies \forall t > t_0, x(t) \in S. 
\end{align}
In energy function-based methods, $S$ is defined by a designable safety index $\phi$: $S := \{x\mid\phi(x)\leq0\}$. The forward invariance can be guaranteed if the safe control constraint: $\dot \phi(x,u) < \gamma(x)$ is persistently feasible, where $\gamma(x)$ depends on the method. This paper considers the safe control constraint used in Safe Set Algorithm (SSA)~\cite{liu2014control}, corresponding to the following persistent feasibility condition:

% To realize forward invariance, control inputs have to follow certain safe control constraints, which we can denote by $\dot \phi(x,u) < 0$. The safe control constraints are different in different methods. This paper considers the control constraints used in SSA~\cite{liu2014control}, which defines $S := \{\phi\}_{<0} \cap \XX_s$ with a designable safety index $\phi$ and requires $\dot \phi(x,u) < 0$ when $\phi(x) = 0$. The forward invariance of $S$ can be guaranteed if $\phi$ is persistently feasible.

% An intuitive strategy to realize forward invariance is to find a control that makes $\dot \phi_0 < 0$ whenever the system reaches the safe set boundary $\phi_0=0$, in which case $S=\XX_s$. However, due to control limits and dynamics, such a control may not exist. Therefore, we need to design a safety index $\phi$ such that 1. its zero-sublevel-set $S:=\{\phi\}_{<0}\cap \XX_s \subseteq \XX_s$; 2. $\phi$ is persistently feasible, which means:

\begin{defn}[Persistent feasibility]\label{def:x_feasibility}
    A safety index $\phi$ is persistently feasible if $\forall x \in X$ such that $\phi(x)=0$, there always exists $u\in U$ such that $\dot \phi(x,u) < 0$.
\end{defn}

However, designing a persistently feasible safety index to ensure forward invariance is not easy, especially for high-dimensional applications. Suppose $\phi$ is parameterized by $\theta$, the problem can be formulated as 
\begin{align}
    \min_{\theta} \left|B_{\theta}^*\right| := \min_{\theta} \left|\{x \mid \phi_{\theta}(x) = 0, \inf_{u}\dot\phi_{\theta}(x,u) \geq 0 \}\right|,\label{eq:B*}
\end{align}
where $B^*_\theta$ denotes the set of states on the boundary of $\phi_\theta$ that have no feasible safe control, $|B^*_\theta|$ is the volume of $B^*_\theta$. The goal is to optimize $\theta$ such that $\left|B^*_\theta\right|=0$. The task is difficult because computing $\left|B^*_\theta\right|$ is generally intractable for high dimensional systems.


% A discretization-based numerical method was proposed to verify the persistent feasibility of general systems~\cite{wei2021safe}. 

% \begin{lem}\label{lemma: feasibility}
%     Suppose 1) we sample a finite state subset $B \subset X$ such that  $\forall x \in X$, $\min_{x' \in X } \|x - x'\| \leq \delta$, where $\delta$ is a constant representing the sampling density; 2) $\forall x' \in B$, there exists a feasible safe control $u$, \st $\dot \phi(x', u) \leq - \gamma(\phi(x')) -\epsilon,\  \forall f(x') , g(x')$,
%     % \begin{align}
%     % \dot \phi(x', u) \leq - \gamma(\phi(x') -\epsilon,\  \forall f(x') , g(x'). \label{eq: sample_feasibility}
%     % \end{align}
%     where $\epsilon$ is a scalar depending on the Lipschitz constants of the dynamics and $\phi$. Then $\phi$ is persistently feasible.
% \end{lem}

% However, this method does not scale to high-dimensional systems because the number of samples grows exponentially with the dimensions.

\subsection{System abstraction}

To design a persistent feasible safety index for high-dimensional applications. We observe that, often, not all states are needed to check the satisfaction or feasibility of a safety specification as in \cref{eq:B*}. For instance, when considering collision avoidance of a tool held by a robot arm (\cref{fig:idea}), only the relative distance and velocity from the tool to the obstacle (2 dimensions) are required, instead of all 14 dimensions of the robot arm.

System $\Sigma_1$ is called the \textit{concrete system}. Suppose we want to design the safety index on a space $Z$ which is defined by a smooth, surjective map $z = \Phi(x)$, where $\Phi: \real^{n_x} \to \real^{n_z}$, $n_z \leq n_x$. Then we can define a system
\begin{align}
    (\Sigma_2) \quad & \dot z = f_z(z) + g_z(z) v\label{eq:z_dyn},
\end{align}
where $z \in Z \subseteq \real^{n_z}$, $v \in V \subseteq \real^{n_v}$. $v$ is an affine transformation of $u$, which will be presented in \cref{lem:affine_control}.
\begin{asp}
    We assume $g_z$ is of full column rank. Otherwise, the dimension of abstract safe control can be reduced. Therefore, $g_z(z)^{-1}$ always exists.
\end{asp}
\begin{defn}[$\Phi$-related and Abstraction] A system $\Sigma_2$ is $\Phi$-related to a system $\Sigma_1$ if for every trajectory $x(t)$, $z(t) = \Phi(x(t))$ is a trajectory of $\Sigma_2$. $\Sigma_2$ is an \textit{abstraction} of $\Sigma_1$ if it is $\Phi$-related to $\Sigma_1$~\cite{pappas2002consistent}.
\end{defn}
\cite{pappas2000hierarchically} proves that given a control system $\Sigma$ and any smooth map $\Phi$, there always exists a control system which is $\Phi$-related to $\Sigma$. \cite{pappas2002consistent} provides a method to construct the smallest $\Sigma_2$ on $Z$ that is $\Phi$-related to $\Sigma_1$. %
If $f_z(z)$ and $g_z(z)$ are unknown, they can be constructed with this method.
% However, this method can not provide a consistent abstraction under control limits. 

% We are interested in designing a persistent feasible safety index in an abstracted system $\Sigma_2$ because it is usually of much lower dimensions. However, the abstracted $\Sigma_2$ overapproximates the abstracted trajectories of the concrete system $\Sigma_1$ which may result in trajectories that $\Sigma_2$ may generate but are infeasible for $\Sigma_1$. In order for the abstraction to be property-preserving, it has to be a consistent abstraction. Consistency requires high-level objectives to be uniformly implementable by the concrete system, which is a property-dependent condition.

% In our application, systems usually operate near the control limit. We need to make the abstraction consistent under control limits such that the abstract control is uniformly implementable by the concrete system.
However, $\Phi$-relatedness is insufficient for designing the safety index on the abstraction. Because an abstract control at $z$ may not be implementable at all states $x = \Phi^{-1}(z)$ by the concrete system. For example, as in \cref{fig:idea}, a horizontal acceleration is implementable at $x_1$ but not at $x_2$. To enable designing safety index on the abstraction, we define consistent abstraction of constraint feasibility as follows:
\begin{defn}[Feasibility consistent abstraction]\label{def:con-abs}
    Let $\Sigma_1$ and $\Sigma_2$ be two control systems and $\Phi:X \to Z$ be a smooth map. Given a safety index $\phi$ defined on $X$. $\Sigma_2$ is a feasibility consistent abstraction of $\Sigma_1$ iff there exists a safety index $\phi_z: Z\mapsto \real$ such that the following conditions are satisfied: 1) $\phi_z(z) = \phi_z(\Phi(x)) =\phi(x)$; 2) $\forall z~ \st \phi_z(z)=0$, if $\exists v \in V, \st \dot{\phi_z}(z, v) < 0$, then  $\exists u \in U, \st \dot \phi(x, u) < 0$, $\forall x, \st \Phi(x) = z$.
    % \begin{align}
    %     &\forall z, \st \phi_z(z) = 0,\ \exists v \in V, \st \dot{\phi_z}(z, v) < 0 \implies 
    %     \label{eq:z_feasibility}\\
    %     &\forall x, \st \Phi(x) = z, \phi(x) = 0,\ \exists u \in U, \st \dot \phi(x, u) < 0 \label{eq:x_feasibility}.
    % \end{align}
\end{defn}
% \begin{defn}[Uniformly implementable]
% An abstraction $\Sigma_2$ is uniformly implementable by $\Sigma_1$ if for every trajectory $z(t)$, $\forall x(0)\ \st \Phi(x(0)) = z(0)$, there exists a trajectory $x(t)$ such that $\Phi(x(t)) = z(t)$.    
% \end{defn}

Consistent abstraction has been studied without considering control limits~\cite{pappas2002consistent, pappas2000hierarchically}. That is, when $U = \real^{n_u}$ and $V = \real^{n_v}$. But in reality, $U$ is usually a bounded set. How to design the abstraction under control limits has not been studied. Besides, the safety constraint $\dot \phi(x,u) < 0$ is state-dependent which introduces another difficulty. 

% We are particularly interested in the direction \cref{eq:z_feasibility} $\implies$ \cref{eq:x_feasibility} because it enables us to design safety index on the abstracted system.

In the following, we first show how to construct feasibility-consistent abstractions under control limits, and characterize when persistent feasibility on the concrete system can be guaranteed by ensuring persistent feasibility on the abstracted system, that is, when condition 2) in \cref{def:con-abs} can be satisfied. Then we show how to ensure persistent feasibility on the abstracted system by designing a safety index.

% System $\Sigma_2$ can produce as trajectories all functions of the form $z(t) = \Phi(x(t))$, where $x(t)$ is a trajectory of system $\Sigma_1$. 
% However, the abstract control limits $V$ depends on the concrete state $x$.

% However, the challenge of system abstraction is that the abstract control limit (acceleration range of the end effector) changes with the underlying concrete state. As shown in Fig. \cref{fig:idea}, at a nominal pose, the end effector can accelerate in all directions. But when the arm is all straight, the end effector can only accelerate in vertical directions. And the acceleration range is also different. Existing work on system abstraction usually makes simplification by assuming a uniform control range~\cite{}. But such a simplification is often unrealistic, such as shown in Fig. \cref{fig:idea}.

% The following theorem gives a sufficient and necessary condition of $\Phi$-relatedness.
% $\Phi$-relatedness is defined through a differential geometry view of control systems. 

 % \wth{seems not very necessary to show this theorem, given a subjective smooth $\Phi$, $\Phi$ relatedness is natural for control systems}
% \begin{thm}[$\Phi$-Related Control Systems]
%     Let $\Sigma_1$ and $\Sigma_2$ be two control systems and $\Phi:X \to Z$ be a smooth map. Then $\Sigma_2$ is $\Phi$-related to $\Sigma_1$ if and only if for every trajectory $x(t)$, $z(t) = \Phi(x(t))$ is a trajectory of $\Sigma_2$.
% \end{thm}

% We need to define control systems from a differential geometry view.
% \begin{defn}
%     Let $X$ be a differentiable manifold, and denote the tangent space of M at $p\in X$ by $T_pX$. Then let $TX = \cup_{p\in X} T_pX$ be the tangent bundle of $X$.
% \end{defn}
% \begin{defn} (Control systems): A control system $S = (U,F)$ consists of a fiber bundle $\pi: U \to X$ and a smooth map $F: U \to TX$ which is fiber preserving. Given
% \end{defn}

% \begin{defn}[$\Phi$-Related Control Systems]: Let $S_X=(U_X, F_X)$ with $\pi_X:U_X\toX$ and $S_Z=(U_Z,F_Z)$ with $\pi_Z:U_Z\to Z$ be two control systems.
% \end{defn}

 % $\Sigma_2$ is also called a $\Phi$-related control system of $\Sigma_1$~\cite{pappas2002consistent}.
