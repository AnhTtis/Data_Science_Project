
Energy function-based methods have been extensively studied to ensure control-level safety for robotic systems in various applications, such as industrial robots in manufacturing or autonomous vehicles in transportation \cite{wei2019safe}. Typical approaches include the control barrier function method and the safe set algorithm (with safety index). These techniques aim to map dangerous states to high energy and safe states to low energy. Safety is guaranteed if a realizable control always exists that dissipates the energy whenever the state is in danger, which is known as \textit{persistent feasibility}.
Persistent feasibility can be guaranteed by offline energy function synthesis (known as the barrier function or the safety index)~\cite{wei2022safe}. Formal guarantees can be provided for general nonlinear systems with up to seven dimensions in states~\cite{bansal2017hamilton}.
% Energy function-based methods involve two steps: (1) offline synthesis of the energy function (e.g., barrier function, safety index), and (2) online enforcement of energy dissipation. Ensuring persistent feasibility by designing the safety index has been extensively studied for low-dimensional systems. Formal guarantees can be provided for general nonlinear systems with up to six to seven dimensions in states~\cite{bansal2017hamilton}.
% Neural Gaits are limited to periodic controllers \cite{rodriguez2022neural},


However, existing energy function-based methods face challenges in ensuring persistent feasibility for general high-dimensional applications. For example, rule-based methods \cite{liu2014control} only apply for specific types of system dynamics; evolutionary optimization-based synthesis does not scale well for high dimensions \cite{wei2022safe} due to the curse of dimensionality; adversarial optimization lacks formal guarantees \cite{liu2022safe}; and Sum-of-Square-based methods are restricted to certain polynomial specifications \cite{clark2021verification, zhao2022safety}. Thus, methods to formally ensure safety for general high-dimensional applications are needed.

Our observation, as illustrated in \cref{fig:idea}, is that high-dimensional system models are typically redundant with respect to the safety specification. For instance, if we only need to ensure collision avoidance between a drill and a human, there is no need to verify the feasibility of safe control for all robot states. Similarly, when addressing collision avoidance for a legged robot, the focus is primarily on its center of mass rather than each leg's states~\cite{zhao2021model}. Hence, a simplified model of the robot can be used to replace the full dynamic model for safe control to reduce dimensionality.



\begin{figure}[tb]
    \centering
    % \includegraphics[width=\linewidth]{img/idea.pdf}
    \includegraphics[width=\linewidth]{img/abstract-hor.pdf}
    % \caption{System abstraction for safety index design. Consider a robot arm holding a drill interacting with a human. Assume the robot's state is $x$, and the relative distance and velocity between the human and the drill are $d$ and $\dot d$. The safety specification is keeping the drill away from the human ($\phi_0 = 1-d < 0$). We wish to design a safety controller that ensures $\phi_0$ is always satisfied.
    % $\phi_0$ only considers distance, therefore may lead to failure (it can be too late to brake when the velocity is large). Therefore, typically we need to design a safety index $\phi$ that considers both $d$ and $\dot d$. 
    \caption{An illustration of abstract safe control to prevent collision between a drill and a human. The conventional safe control approach is to model the full kinematic chain of the robot system with state $x$. Our proposed approach considers abstract states $z=[d,\dot d]$ (relative distance and velocity) and a scalar M that considers constraints imposed by the full kinematic chain (whose values are different for $x_1$ and $x_2$).
    %How to handle this non-uniform constraint is the major challenge this paper aims to solve. 
    % For example, $x_1$ and $x_2$ correspond to the same z, but only at $x_1$, the drill can have relative acceleration away from the human. The varying abstract control limits prevent us from designing the safety controller on the abstracted system.
    \vspace{-5mm}
    }
    \label{fig:idea}
\end{figure}



A common approach to simplify dynamics is by employing a two-layer hierarchical architecture through system abstraction~\cite{pappas2000hierarchically}. The high-level system uses abstracted dynamics, while the low-level system uses the original concrete dynamics. 
However, existing system abstraction methods~\cite{smith2019continuous, yin2020optimization} can not ensure persistent feasibility. Because they usually assume the abstract model has uniform control constraints, which may lead to an unrealizable abstract control that breaks the feasibility. For instance, in \cref{fig:idea}, although both situations correspond to the same abstract state $z$, the feasible abstract control (Cartesian acceleration of the drill) in these two situations is different. In particular, at $x_2$, the horizontal acceleration of the drill is unrealizable due to singularity. 
% And in \cite{zhao2021model}, a doggo robot is abstracted as a mobile agent with a uniform acceleration limit. However, the actual range can be zero if the doggo is lied down.
% not all abstract control (acceleration of the drill) can be realized by the concrete system
To enable safe control through system abstraction, the notion of \textit{abstraction consistency} is essential to ensure that the high-level objective is always realizable by the concrete controller. Consistent abstraction of controllability~\cite{pappas2000hierarchically} and local accessibility \cite{pappas2002consistent} have been studied. However, it remains unclear on how to design a consistent abstraction for persistent feasibility, since the constraint is state-dependent, and control limits must be considered. 

To address these challenges, we propose a consistent abstraction for persistent feasibility, which allows us to design and verify the energy function (in the following discussion, we call it as the safety index) on a low-dimensional abstracted system. We specifically augment the abstract state with a scalar to account for different control constraints imposed by the concrete system for same abstract states.
%The non-uniform control constraint at different abstract state can be handled by extending the design of the abstraction with an additional state, which allows for different reactions under varying control limits given the same abstract state. 
For example, we extend $z$ in \cref{fig:idea} to $\hat z = [d, \dot d, M]$, where $M = \max |\ddot d|$. By mapping $x_1$ and $x_2$ to different $\hat z$, we can ensure that $\ddot d$ is always realizable by choosing it from the range $[-M,M]$. We will present a general method for designing this extended abstraction. Then we prove that a persistent feasible safety index synthesized on the extended abstraction guarantees persistent feasibility in the concrete system because all abstract controls are realizable. Lastly, we discuss how to design a persistent feasible safety index on the extended abstraction.

The abstraction not only reduces dimensionality but also enhances the transferability of the synthesized safety index. The safety index synthesized on such an extended abstraction can be directly applied to other concrete systems that have the same safety specifications (which implies same abstraction), as long as certain criteria is met. %For example, the safety index for a robot end effector can be directly applied to the center of mass a legged robot center of mass. 
%The persistent feasibility is maintained if a scalar metric of the new concrete system remains above a threshold, which characterizes the abstracted control limits and changes monotonically with the degree of freedom. 
This transferability is especially useful for systems with time varying structures, such as a robot arm with changing end effectors. Once the safety index is feasible for a robot arm on the extended abstraction, it can be directly applied to the robot arm with different end effectors, to be shown in \cref{sec: experiment}.



% \cite{cavarischia2007hierarchical} considers exact abstraction mapping, which guarantees the designed controls are tractable. However, these methods only consider linear systems and usually assume a fixed control limit. However, the actual control limit after abstraction is typically no longer fixed. Such an assumption may not hold or lead to conservative control design.
% \cite{smith2019continuous} synthesis a low level controller for nonlinear systems with Sum-of-Square polynomials. However, the
% \cite{yin2020optimization} 

% Our method:
% exact mapping, no model mismatch
% considers a state-varying control limit
% provide feasibility guarantee
% previous methods rely on specific dynamics: linear, polynomial
% fail to guarantee feasibility
% Our observation is that these methods do not consider the correlation between the abstract control limit and the concrete system state.

% We propose to synthesize the safety index by leveraging an exact abstraction mapping. 
% We address these challenges by proposing an extended abstraction method. 
% Furthermore, because the abstraction is inferred from the safety specification, the safety index can be applied to different concrete systems if the safety specification is the same. For example, a robot arm holding different tools with collision avoidance specifications can share the same abstracted system of relative distance and relative velocity to obstacles. And in our method, the feasibility of the synthesized safety index only depends on three scalar parameters that characterize the abstracted control limits. The abstraction is guaranteed consistent for all concrete systems whose these parameters are below a threshold. Three scalar parameters monotonically change with the degree of freedom, which makes our method particularly useful for systems having real-time varying structures, such as a robot arm with changing end effectors. Once the abstraction is verified consistent for a robot arm, it can be directly applied to the robot arm with different end effectors.

% \begin{align}
%     \phi = d_{min}^2 - d + k \dot d
% \end{align}
% \begin{align*}
%     \ddot d \in V \quad V = ?
% \end{align*}

% \begin{align}
%     z = [d, \dot d], v = [\ddot d].
% \end{align}

% where $x$ is the system state, we can consider $d,\ \dot d$ as the latent system state, and $\ddot d$ as the latent system control input. This lower dimensional space enables us to use analytical and numerical methods to design a feasible safety index. We wish to guarantee the feasibility of the safety index on the latent space, and consequently guarantees the feasibility on the original space.
% and compute safe control though this latent space.

% However, it is difficult to decide the control limit for the latent system. The range of $\ddot d$ depends on the original state $x$. If we want to do control directly on the latent system, we need to give a range of $\ddot d$ that only depends on the latent state $d$ and $\dot d$, which is generally difficult to compute because it involves computing the preimage of the latent state.
% And if we compute a uniform $\ddot d$ range, it is potentially very conservative. For example, consider a robot arm end effector case. When the robot arm is fully extended (all links in a row), and the obstacle is in line with the arm, we have $\ddot d \in [0,0]$. And because we need to take the intersection of $\ddot d$ range of all states, we can only have $\ddot d \in [0,0]$.

% In this work, we propose a method to achieve latent space safe control and avoid the difficult and conservative control limit computation.

