% Use only LaTeX2e, calling the article.cls class and 12-point type.

\documentclass[12pt]{article}

% Users of the {thebibliography} environment or BibTeX should use the
% scicite.sty package, downloadable from *Science* at
% http://www.sciencemag.org/authors/preparing-manuscripts-using-latex 
% This package should properly format in-text
% reference calls and reference-list numbers.

\usepackage{scicite}
\usepackage{graphicx}
\usepackage{times}
\makeatletter\def\author#1{\gdef\@author{\hskip-\dimexpr(\tabcolsep)\hskip1pt\parbox{\dimexpr\textwidth-1pt}{\centering#1}}}\makeatother

% The preamble here sets up a lot of new/revised commands and
% environments.  It's annoying, but please do *not* try to strip these
% out into a separate .sty file (which could lead to the loss of some
% information when we convert the file to other formats).  Instead, keep
% them in the preamble of your main LaTeX source file.


% The following parameters seem to provide a reasonable page setup.

\topmargin 0.0cm
\oddsidemargin 0.2cm
\textwidth 16cm 
\textheight 21cm
\footskip 1.0cm


%The next command sets up an environment for the abstract to your paper.

\newenvironment{sciabstract}{%
	\begin{quote} \bf}
	{\end{quote}}



% Include your paper's title here

\title{Vortices as Brownian Particles in Turbulent Flows} 


% Place the author information here.  Please hand-code the contact
% information and notecalls; do *not* use \footnote commands.  Let the
% author contact information appear immediately below the author names
% as shown.  We would also prefer that you don't change the type-size
% settings shown here.

\author
{Kai Leong Chong$^{1}$, Jun-Qiang Shi$^{2}$, Guang-Yu Ding$^{1}$, Shan-Shan Ding$^{2}$, Hao-Yuan Lu$^{2}$,
Jin-Qiang Zhong$^{2\ast}$ and Ke-Qing Xia$^{3,1\ast}$\\
~\\
\normalsize{$^{1}$Department of Physics, The Chinese University of Hong Kong,}\\
\normalsize{Shatin, Hong Kong, China}\\
\normalsize{$^{2}$School of Physics Science and Engineering, Tongji University,}\\
\normalsize{Shanghai, 200092, China}\\	
\normalsize{$^{3}$Center for Complex Flows and Soft Matter Research and Department of Mechanics and Aerospace Engineering, Southern University of Science and Technology}\\
\normalsize{Shenzhen, 518055, China}\\
~\\
\normalsize{$^\ast$To whom correspondence should be addressed; E-mail: jinqiang@tongji.edu.cn; xiakq@sustech.edu.cn}
}

% Include the date command, but leave its argument blank.

\date{}



%%%%%%%%%%%%%%%%% END OF PREAMBLE %%%%%%%%%%%%%%%%



\begin{document} 
	
	% Double-space the manuscript.
	
	\baselineskip24pt
	
	% Make the title.
	
	\maketitle 
	


\begin{sciabstract}
The traditional view of vortex motion is that the effect of inertia should be neglected since the vortex does not have distinct density or mass difference from their environment. Here, we demonstrate through both experiment and numerical simulation that the movement of vortices in a rotating turbulent flow resembles that of inertial Brownian particles,  i.e. they initially move ballistically, and then diffusively after certain critical time. Moreover, the transition from ballistic to diffusive behaviors is direct, as predicted by Langevin, without first going through the hydrodynamic memory regime. In the spatial domain, however, the vortices exhibit organized structures, as if they are performing tethered random motion.  Our results imply that vortices actually have inertia-induced memory such that their short term movement can be predicted and their motion can be well described in the framework of Brownian motions. 
\end{sciabstract}
    
\section*
  {Introduction}
Brownian motion is an example of stochastic processes that occur widely in nature \cite{pusey2011brownian}. Einstein was the first to provide a theoretical explanation for the movement of pollen particles in a thermal bath \cite{einstein1905molekularkinetischen}. Later, Langevin considered the inertia of the particles and predicted that the motion of particles would be ballistic in short time and then changes over to a diffusive one after certain time \cite{langevin1908theorie}. Because this transition occurs in a very short timescale, its direct observation had to wait for over one hundred years \cite{huang2011direct}. 

However, the ``pure" Brownian motion, as predicted by Langiven, is never observed in liquid systems, i.e., the mean squared displacement (MSD) of the particles changes directly from a $t^2$ dependence to a $t$ dependence. Rather, the transition spans a broad range of time scales, as is the case in Ref. \cite{huang2011direct}.  This slow and smooth transition is caused by the so-called hydrodynamic memory effect \cite{vladimirsky1945hydrodynamical}, which arises as the surrounding fluid displaced by moving particles reacting back through entrainment, thereby generating long-range correlations \cite{hinch1975application}. This also manifests in the spectrum of the stochastic force in the Langevin equation being ``colored" \cite{franosch2011resonances,jannasch2011inertial}.
The hydrodynamic memory effect has been observed in a number of systems, for instance, colloidal suspensions \cite{zhu1992scaling}, particles suspended in air \cite{kim1973long} and trapped particles in optical tweezers \cite{huang2011direct,franosch2011resonances,Kheifets2014Science}.

%Brownian motion is found to exist in many other systems besides the pollen particle. For instance, the particles in membranes in biological systems \cite{saffman1975brownian}, artificial microscale swimmers \cite{howse2007self}, nanocrystal suspensions \cite{brites2016instantaneous} and silica spheres in air \cite{li2010measurement}. It is also plays a role in the pre-planetary dust aggregation \cite{blum2000growth}. 

In the studies of Brownian motion, a common assumption is that the objects should have distinct density or mass difference from their environment such that inertia plays a role initially \cite{langevin1908theorie}. Here we demonstrate, by both experiment and numerical simulations, that vortices in highly turbulent flows behave like inertial particles performing pure Brownian motion, i.e. their MSD changes sharply from a $t^2$ dependence to a $t$ dependence without being influenced by the hydrodynamic effect. The system here is thermally-driven rotating turbulent flows in which the convective Taylor columns move two-dimensionally in a highly turbulent background flow that serves as a heat bath. Our results suggest that within a well-determined time, the inertia of vortex becomes effective such that it persists to drift along the previous direction. This may entail the capability of predicting the vortex motion within certain period of time in astro-and geo-physical systems.

In many situations in astrophysics, geophysics and meteorology, thermal convection occurs while being influenced by rotation. The existence of Coriolis force leads to the formation of vortices \cite{hopfinger1993vortices}, which appear ubiquitously in nature.  For instance, tropical cyclones in the atmosphere \cite{emanuel2003tropical}, oceanic vortices \cite{flament1996three}, long-lived giant red spot in Jupiter \cite{marcus2004prediction}. Another intriguing example is the convective Taylor columns in the Earth's outer core, which is believed to play a major role in the Earth's dynamo \cite{roberts2000geodynamo}, and is therefore closely related to the Earth's magnetic field variation and the corresponding seismic activities \cite{varotsos1984physical}. A challenge in the astro-and geophysical research communities is whether one can predict the movement of vortices within certain period of time. 

A model system used in the study of  vortices in convective flows is the so-called rotating Rayleigh-B\'enard (RB) convection \cite{chandrasekhar2013hydrodynamic,ahlers2009rmp,vorobieff2002turbulent,clercx2018mixing} which is a fluid layer of fixed height ($H$) heated from below and cooled in the above while being rotated about the vertical axis at an angular velocity $\Omega$. Here the temperature difference destabilizes the flow such that convection occurs when the thermal driving is sufficiently strong. Three dimensionless parameters are used to characterize the flow dynamics of this sytem, which are  the Rayleigh number $Ra=\alpha g \Delta TH^3/\kappa\nu$, the Prandtl number $Pr=\nu/\kappa$ and the Ekman number $Ek=\nu/2\Omega H^2$. Here $\alpha$, $\kappa$, and $\nu$ are the thermal expansion coefficient, thermal diffusivity, and kinematic viscosity of the fluid; $g$ is the gravitational acceleration and $\Delta T$ the temperature difference across the fluid layer.

\begin{figure*}[h!]
	\centering
	\setlength{\unitlength}{\textwidth}
	\includegraphics[width=1.0\unitlength]{./figure1.pdf}
	\caption{Snapshots of (a) the temperature $\theta$ and (b) streamlines originating from the lower thermal boundary layer. (c) Snapshots of $Q/Q_{std}$ (see main text for the definition of Q) taken horizontally at the edge of thermal boundary layer for $Ek=4\times10^{-5}$ and $Ra=10^8$, and a demonstration of the extracted vortex. The locations of vortex center are marked as yellow crosses.}
	\label{fig1}
\end{figure*}

In the absence of rotation, fragmented thermal plumes detach from the thermal boundary layer and transport to the opposite boundary layer. When rotation is present, especially when its effect becomes non-negligible, vortical structures emerge which can be seen as fluid parcels  spiraling up or down (Fig. \ref{fig1}). It is known that these vortical plumes arise from Ekman pumping and can enhance heat transport \cite{zhong2009prl}. When rotation becomes rapid yet not too strong so the flow is not completely laminarized, the Taylor-Proudman effect \cite{proudman1916motion,taylor1923stability} becomes dominant which suppresses the flow variation along the axis of rotation. The resultant flow field is the long-lived columnar structure extending throughout the entire cell height  as convective Talyor columns \cite{boubnov1986experimental,grooms2010prl}. %These vortices are not steady but move around (largely two-dimensionally) as they interact with the background turbulent flow. 

\begin{figure*}[h!]
	\centering
	\setlength{\unitlength}{\textwidth}
	\includegraphics[width=1.0\unitlength]{./figure2.pdf}
	\caption{(a) The mean square displacement (MSD) of the vortices as a function of time (b) Normalized MSD as a function of $t/t_c$. The solid line represents a fit of Eq. (4) to the data. In both (a) and (b), solid symbols denote numerical results at $Ra=1\times10^8$, and open symbols denote experimental results at $Ra=3\times10^7$. (c) Diffusion coefficient of vortices $D$ (open symbols) and the characteristic timescale for motion transition $t_c$ (solid symbols) as a function $Ra/Ra_c$. (d) Velocity autocorrelation function VACF versus $t/t_c$ for different $Ek$. The dashed line represents $C(t) = \frac{2D}{t_c}exp(-t/t_c)$. The solid line indicates a power law decay for the VACF.}
	\label{fig2}
\end{figure*}

The parameter range of the study is such that, for experiment, $Ra$ is fixed at $3\times10^7$ while $Ek$ is varied from $3.36\times10^{-5}$ to $2.68\times10^{-4}$. For direct numerical simulation (DNS), $Ra$ varies from $10^7$ to $10^9$ while $Ek$ changes from $1.5\times10^{-6}$ to $4\times10^{-4}$. In experiment, we use a cylindrical convection cell of the lateral dimension to height aspect ratio $\Gamma=3.8$, with rigorous thermal control at the wall boundaries \cite{zhong2015dynamics,ding2019temperature}. In DNS, periodic boundary condition is adopted with $\Gamma=2$. In addition, we consider only the influence of the Coriolis force but neglecting the effect of centrifugal force in the simulation. This condition is valid in the experiment for small enough Froude number (usually for $Fr = \Omega^2L/2g \ll 0.05$ \cite{zhong2009prl}). In order to compare experimental and numerical results, all the physical parameters are made dimensionless, using the buoyancy timescale (also known as the free-fall timescale), the temperature difference across the fluid layer and the system height. The vortices are identified and extracted using the so-called Q-criterion \cite{hunt1988eddies} (for details, see Materials and Methods). Figure \ref{fig1}c shows a typical field of Q quantity and the examples of extracted vortices. 

\section*
  {Results}
\subsection*{Horizontal motion of the vortices}
We first examine the motion of vortices by tracking their positional change from a sequence of snapshots with a time interval of smaller than $1$ buoyancy time unit, so that the movement of vortex is smooth in this time frame. With the obtained trajectories, the statistical behavior of the vortices can be characterized by their MSD, $\langle \vec{r}^2(t) \rangle = \frac{1}{N}\sum^N_{i=1}(\vec{r}_i(\tau+t)-\vec{r}_i(\tau))^2$, where $N$ is the total number of trajectories. Figure \ref{fig2}a shows the MSD versus time $t$ from both simulation ($Ra=1\times10^8$) and experiment ($Ra=3\times10^7$) for various values of $Ek$. For each $Ek$, the time scale spans three decades. The MSDs for different $Ek$ and $Ra$ are seen to exhibit the same behavior, i.e., at short time the vortex motion is ballistic, and the motion becomes diffusive after certain time. This is demonstrated more clearly by the power law dependence $\langle \vec{r}^2(t) \rangle \sim t^{\alpha}$, with the exponent $\alpha =2$ for small values of $t$ and that changes to $1$ for larger times.  It is interesting to note that this transition from ballistic to diffusive motion resembles that of Brownian particles in a thermal bath. Because there is relatively small density difference between fluids in the vortex and in the surrounding, the vortex motion should be over-damped, meaning a negligibly small inertia compared to the viscous damping. Therefore, the existence of ballistic behavior is particularly striking. If the vortices can be treated as Brownian particles, their motion can then be described by the solution of a Langevin equation \cite{langevin1908theorie}:

\begin{equation}
\ddot{\vec{r}}=-\dot{\vec{r}}/t_c+\vec{\xi }(t),
\end{equation}
\begin{equation}
\left \langle \vec{\xi }(t) \right \rangle=0,
\end{equation}
\begin{equation}
\left \langle \vec{\xi }(t') \cdot \vec{\xi }(t'') \right \rangle=\frac{2D}{t_c^2}\delta (t'-t'')
\end{equation}

where $\vec{\xi }(t)$ is a stochastic force with white-noise spectrum, which characterizes the turbulent background fluctuations; $t_c$ is a characteristic timescale separating the ballistic and diffusive regimes,  and $D$ is the diffusion coefficient of vortices in thermal turbulence. From the Langevin equation one obtains the MSD:

\begin{equation}
\label{msdeq}
\frac{\left \langle \vec{r}^2(t) \right \rangle}{2Dt_c}=\frac{t}{t_c}(1-\frac{t_c}{t}(1-exp(-\frac{t}{t_c})))
\end{equation}

The above expression can be used to fit the measured MSD to obtain $D$ and $t_c$ for each $Ek$ and $Ra$. By plotting $\left \langle \vec{r}^2(t) \right \rangle/2Dt_c$ versus $t/t_c$ one finds that all the measured MSDs collapse excellently onto a single curve, which implies that the dynamics of vortex motion is the same for the various values of $Ra$ and $Ek$. The solid line in Fig. \ref{fig2}b represents a fit of Eq. \ref{msdeq} to the data points. The excellent agreement, including both the ballistic and the diffusive behaviors and the sharp transition between the two regimes, suggests that the two-dimensional motion of the vortices exhibit a "pure Brownian" behavior. Note that the two fitting parameters $D$ and $t_c$ in the equation depend on both $Ra$ and $Ek$. It is thus remarkable that when plotted against $Ra/Ra_c$ (where  $Ra_c=8.7Ek^{-4/3}$ is the critical Rayleigh number for the onset of convection \cite{chandrasekhar2013hydrodynamic}), both $D$ and $t_c$ collapse nicely onto a single trend as shown in Fig. \ref{fig2}c. This suggests that the rescaled $Ra$  can serve as a suitable parameter to describe the dynamics of vortex motion. When $Ra/Ra_c$ increases the diffusivity of the vortex motion increases monotonically, indicating a greater level of turbulent fluctuation in the background flows. It is worthy of noting that when $Ra \geq 10Ra_c$, $t_c$ approaches to one, i.e. it becomes the buoyancy timescale. This suggests that the buoyancy time becomes the dominant scale in controlling the ballistic to diffusive transition of the vortex motion when $Ra$ becomes much larger than $Ra_c$.

We have shown that the hydrodynamic memory effect is absent in the Brownian motion of the vortices. This can be demonstrated rigorously by examining the velocity autocorrelation function (VACF) of the vortex motion, $C(t)=\langle V(\tau)V(\tau+t) \rangle$. Figure \ref{fig2}d shows that in the range of $0 \leq t/t_c \leq 5$, results of the VACF from both simulation and experiment are best described by an exponential function $C(t) = \frac{2D}{t_c}exp(-t/t_c)$. We note that the hydrodynamic memory effects, which are often observed in the motion of Brownian particles, leads to a long-time tail of algebraic decay in the VACF: $C(t) \sim t^{-3/2}$ \cite{hinch1975application,widom1971velocity}. Our observation of a pure exponential decay of the VACF suggests that the vortices are subjected to a single stochastic driving force from the surrounding flows. This driving force, which comes from the background turbulent fluctuations and comprise the complex hydrodynamic interactions between the vortical structures and the surrounding fluid, is the only stochastic source of the system represented by a Gaussian white noise in the Langevin equation. Consequently, the vortex motion exhibits strictly pure Brownian motion, and the transition from the ballistic to the diffusive regimes is sharp.

%For this trends of $D$ and $t_c$, it can be understood as follows: Away from the onset of convection, the convective flow becomes more turbulent, it will reduce the correlation time for the persistent motion and thus result in smaller $t_c$. Also for stronger the flow, the more energetic of the vortex that leads to larger $D$.

\subsection*{Vortex distribution}

Despite the Brownian-like motion, the spatial distribution of the vortices, however, is not random, rather, they exhibit patterned structures. We show in Fig. \ref{fig3}a horizontal slices of the instantaneous normalized $Q$ field taken at the edge of thermal boundary layer for several rotation rates. As $Ek$ varies from $4\times10^{-5}$ to $7\times10^{-6}$, several changes in vortex distribution can be identified. First, the number of vortices increases with the rotation rate such that the initially dilute and randomly-distributed vortices becomes highly concentrated and clustered. Second, when the rotation rate becomes sufficiently high, the vortices tend to form a vortex-grid structure. Zooming in to a local region for the case of highest rotation rate clearly reveals that there is a regular pattern for such vortex-grid structure: Vortices represented by reddish color form a square lattice, with bluish localized areas in-between denoting regions of high stress according to the definition of $Q$. This suggests that the regions of strong normal stress may generate a pinning effect that helps to form the lattice-like pattern.

\begin{figure*}[h!]
	\centering
	\setlength{\unitlength}{\textwidth}
	\includegraphics[width=1.0\unitlength]{./figure3.pdf}
	\caption{(a) Snapshots of $Q/Q_{std}$ taken horizontally at the edge of thermal boundary layer for, from left to right, $Ek=4\times10^{-5}$, $1\times10^{-5}$ and $7\times10^{-6}$ at $Ra=10^8$. (b) Radial distribution function $g(r)$ as a function of $r/a$, where $a$ is the average radius of vortices. (c) The maximum value $g_{max}$  of the radial distribution function versus $Ra/Ra_c$ (the case of $Ra=3\times10^7$ is from experiment, the others are from DNS).}
	\label{fig3}
\end{figure*}

The spatial structure of the vortices can be quantified by the radial distribution function $g(r)$, which is defined as the ratio of the actual number of vortices lying within an annulus region of $r$ and $r+\Delta r$ to the expected number for uniform distribution, such that $g(r)$ equals to one signifies randomly distributed vortices. Figure \ref{fig3}b shows $g(r)$ versus the distance $r$ between vortices  normalized by the average radius {\it a} of the vortices ($a$ is evaluated from the average area of vortex, assuming a shape of a perfect circle). It is seen that the value of $g(r)$ is close to zero when $r$ becomes smaller than the diameter of a vortex, as it should be the case. This feature becomes more robust with decreasing Ek, since the convective Taylor columns become more rigid as rotation rate increases. As $r$ increases and for not too rapid rotation ($Ek>10^{-5}$), $g(r)$ will gradually saturate at the value of one, implying a random spatial distribution of the vortices. In contrast, for $Ek\leq10^{-5}$, a sharp peak in $g(r)$ appears at $r \approx 4a$ before eventually decaying to the value of one. This suggests the emergence of short range order, which corresponds well to the lattice-like structure shown in the magnified picture in Fig. \ref{fig3}a. Here it is seen that the vortices are not closely packed but separated by a localized region of strong stress with the same size as that of a vortex. As a result, the distance between the center of vortices are approximately $4a$, corresponding to the peak position of $g(r)$.

For the smallest $Ek=7\times10^{-6}$, $g(r)$ even exhibits multiple peaks which is an evidence for the existence of a vortex-lattice with a size beyond the nearest neighbours. Figure \ref{fig3}c plots the maximum value of $g(r)$ against the rescaled $Ra$, it is interesting to observe that data points for different $Ra$ and $Ek$ collapse onto a single curve. For $Ra\geq10Ra_c$, $g_{max}$ is close to one, indicating the random distribution of vortices under the influence of strong turbulent fluctuation. In contrast, for $Ra<10Ra_c$, $g_{max}$ increases with decreasing $Ra/Ra_c$. We see that vortices exhibit certain spatial order, despite their motion in the temporal domain being random. How to reconcile this apparent contradiction?

\begin{figure*}[h!]
	\centering
	\setlength{\unitlength}{\textwidth}
	\includegraphics[width=1.0\unitlength]{./figure4.pdf}
	\caption{Trajectories of vortices: (a) $Ek=1\times10^{-4}$ and (b) $Ek=7\times10^{-6}$; in both cases $Ra=1\times10^8$. The blue dots indicate the end of the trajectories. (c) The average separation ($d_{v}$) between vortices (open symbols) and the 75th percentile of the distance ($d_{75}$) traveled by vortices (solid symbols), as a function of $Ra/Ra_c$ for $Ra=1\times10^8$ (simulation, red symbols) and $Ra=3\times10^7$ (experiment, blue symbols). (d) The maximum radial distribution function $g_{max}$ versus $\beta$, defined as the ratio between the Brownian timescale and the relaxation timescale.}
	\label{fig4}
\end{figure*}

Figures \ref{fig4}a,b show the trajectories of vortices for the cases of slow and fast rotation, respectively. It is seen that the vortex motion is actually very localized. This can be made more quantitative by comparing $d_{75}$, which is the 75th percentile of the distance traveled by vortices, to the mean vortex separation $d_v$, as shown in Fig. \ref{fig4}c. It is clear that the majority of vortices during their lifetime do not travel far enough to "see" or interact with other vortices, as if they are tethered. 

The spatial structure of vortices with short range order may be understood from the competition between two dynamical processes, as characterized by the vortex's relaxation timescale and its Brownian timescale, respectively. Here the relaxation timescale $t_s$ is defined as $1/\langle \parallel S \parallel \rangle_{x,y,t}$ where $\langle \parallel S \parallel \rangle_{x,y,t}$ is the magnitude of normal stress averaging over time and over horizontal plane at the edge of thermal BL. And the Brownian timescale is defined as $a^2/D$ where $a$ is the vortex radius. The ratio of the two timescales $\beta= \langle \parallel S \parallel \rangle_{x,y,t}a^2/D$ measures the tendency to form vortex aggregation, and this ratio is somewhat similar to the P\'eclet number used in the study of Stokesian dynamics of colloidal dispersions \cite{brady1988stokesian}. While both become larger for stronger rotation, their relative strength determines the spatial distribution of vortices. For $\beta \leq 1$, vortex motion is dominated by the diffusion, and thus any vortex structure induced by normal stress would be destroyed by the rapid diffusion (with large $D$) and therefore, the distribution of vortices appears to be random. On the other hand, for $\beta > 1$, diffusion loses out to stress and vortex aggregations form.. In Fig. \ref{fig4}d, we plot the peak value of radial distribution function $g(r)$ versus $\beta$. Indeed, $g_{max}$ starts increasing from one when $\beta$ becomes larger than unity.

\section*
  {Discussion}
We have shown that the motion of vortices in rotating thermal convection resembles that of inertial particles performing Brownian motion, with a sharp transition from ballistic to diffusive regimes without first experiencing the intermediate hydrodynamic memory regime. This ``pure" Brownian motion, as originally predicted by Langevin, has not been observed for particles in liquid systems. 
This is despite the fact that the vortices exhibit a certain level of spatial organization, so that their overall behavior is tethered random motion. The finding that the vortices have an apparent inertia, and therefore the existence of memory in their motion before transition to diffusive behavior, may have some astro- or geophysical implications. 

One example is the possibility to predict the motion of vortices within the ballistic regime.
Here we estimate the corresponding transition timescale for several cases in astro- and geophysical systems using the buoyancy timescale $\sqrt{H^4/(\nu\kappa Ra)}$. For the Earth's liquid outer core, one can estimate $t_c$ is in the order of hour to year. This estimation is based on the accepted value of $Ra$ ranges from $10^{22}$ to $10^{30}$ and $Ek\approx10^{-15}$. And the physical parameters $H\approx2\times10^6$ m, $\nu\approx10^{-6}$ m$^2$s$^{-1}$ and $\kappa\approx10^{-5}$ m$^2$s$^{-1}$ are used. In astrophysics, there is a conventional thought that the short term variation (timescale of years or less) of Earth's magnetic field should be primarily caused by external sources, such as the solar wind \cite{bloxham1985secular}. Based on our estimate, the inertia of a vortex should be another significant factor for the columnar vortex movement in the Earth's core and thus it is an example of internal sources affecting the short term variation in the Earth's magnetic field. It also hints at the potential of forecasting the activity of Earth's magnetic field and the related seismic activities.
    
\section*
  {Materials and Methods}
\subsection*{Experimental set-ups}
The experimental apparatus had been used for several previous investigations of turbulent rotating RB convection \cite{zhong2015dynamics,ding2019temperature}. For the present study we installed a new cylindrical cell that had a diameter $L$ = 240.0 mm and a height  $H$ = 63.0 mm, yielding an aspect ratio $\Gamma$ = 3.8. Its bottom plate, made of 35 mm thick oxygen-free copper, had a finely machined top surface that fit closely into a Plexiglas side wall, and was heated from below by a uniformly-distributed electric wire heater. The top plate of the cell was a 5 mm thick sapphire disc that was cooled from above through circulating temperature-controlled water. For flow visualization and velocity measurement, a particle-image-velocimetry (PIV) system was installed that consists of three main components: a solid-state laser with the light-sheet optics; neutrally-buoyant particles suspended in the flow; and a CCD camera. Both the convection apparatus and the PIV system are mounted on a rotating table that operates in a range of the rotating rate 0$ \leq \Omega \leq$2.5 rad/s. The measuring region of the velocity field presented in this work was a central square area of 164 mm$\times$136 mm of the horizontal plane at a fluid height $z=H/4$. In each velocity map 103$\times$86 velocity vectors were obtained with a spatial resolution of 1.6 mm. For a given Ek number, we took image sequences consisting of 18000 velocity maps at time intervals of 0.5 sec, corresponding to an acquisition time of 2.5 hrs.

\subsection*{Numerical method}
We consider the Navier-Stokes equation in Cartesian coordinate with Oberbeck-Boussinesq approximation.
\begin{equation}
\label{equ:governing_equation1}
\nabla\cdot u=0% \label{equ:continuity}
\end{equation}
\begin{equation}
\label{equ:governing_equation2}
\frac{\partial u}{\partial t}+u\cdot\nabla u=-\nabla p+\nu\nabla^2u-2\Omega\times u+\theta e_z %\label{equ:momentumn}
\end{equation}
\begin{equation}
\label{equ:governing_equation3}
\frac{\partial \theta}{\partial t}+u\cdot\nabla \theta=\kappa\nabla^2\theta %\label{equ:temperature}
\end{equation}
where $u$, $p$ and $\theta=[T-(T_{hot}+T_{cold})/2]/\Delta T$ are velocity, pressure and reduced temperature respectively, where $\Delta T=T_{hot}-T_{cold}$.  The governing equation is solved in non-dimensional form. Physical quantities in the governing equation are non-dimensionalized by $x_{ref}=H$, $u_{ref}=\left(\alpha gH\Delta T\right)^{1/2}$, $T_{ref}=\Delta T$ and $t_{ref}=x_{ref}/u_{ref}$. 
Equation \ref{equ:governing_equation3} is solved by the multiple-resolution version of the {\it CUPS} \cite{Chong2018}, which is a fully parallelized direct numerical simulation (DNS) code based on finite volume method with 4th order precision. Temperature and velocity are discretized in a staggered grid. In thermal convection with $Pr>1$, the smallest length scale is the Batchelor length $\eta_b/H=\eta_k/Pr$, where $\eta_k$ is the Kolmogorov length scale $\eta_k/H=\left(Ra\epsilon_u/Pr\right)^{-1/4}$, and $\epsilon_u=\nu\sum_i\sum_j\left(\frac{\partial u_i}{\partial x_j}\right)^2$ is the dimensionless viscous dissipation. Both Batchelor and Kolmogorov length scales are spatially intermittent, depending on the local viscous dissipation. Since boundaries are no-slip in our simulation, strong shearing occurs near boundaries and induces large $\epsilon_u$  and small length scale. To resolve the small-scale flow structures with small length scale, we use structured meshes which are refined near top and bottom boundaries, and equidistant in the two horizontal directions. As $\eta_b$ is smaller than $\eta_k$, the temperature requires higher spatial resolution than the momentum. Thus, the resolution requirement is determined by the temperature solver, which entails unneccessary compution for the momentum solver using the traditional method. To increase computational efficiency without any sacrifice in precision, we use a multiple-resolution strategy, which means the momentum equation is solved in a coarser grid than the temperature. The grid spacing used in our simulations resolve both the Batchelor and Kolmogorov length scales. The temporal integraton of the governing equations is carried out by an explicit Euler-leapfrog scheme, that the convective and diffusive terms are updated using the leapfrog and the Euler forward method respectively. We refer to Refs \cite{Chong2018,kaczorowski2013jfm} for the details of the code.


\subsection*{Extraction of vortices}
We extract vortices based on the Q-criterion \cite{hunt1988eddies} that considers the quantity Q defined by $Q=0.5(\parallel \omega \parallel^2-\parallel S \parallel^2)$ where $\omega$ is the vorticity tensor and $S$ is the rate-of-strain tensor and $\parallel A \parallel = \sqrt{Tr(AA^T)}$. Here a single vortex is defined by the connected region satisfying $Q>Q_{std}$, with $Q_{std}$ being the standard deviation of Q, which can discern the vortices from background fluctuations. The center of a vortex can be further identified by the location with maximum Q.

\bibliography{references}
\bibliographystyle{ScienceAdvances}

\section*{Acknowledgments}
\textbf{Funding:} This work was supported in part by the Hong Kong Research Grants Council under Grant Nos. 14301115, 14302317 and a NSFC$/$RGC Joint Research Grant N\_CUHK437$/$15, and through a Hong Kong PhD Fellowship (K.L.C). The experimental studies at Tongji Univ. were supported by the National Science
Foundation of China under Grant Nos. 11572230, 11772235, and a NSFC/RGC Joint Research Grant No.1561161004. \textbf{Author contributions:} K.-Q.X.  and J.-Q.Z.  conceived and designed research. K.L.C and G.-Y.D. conducted the numerical simulations. J.-Q.S., S-S.D. and H.-Y.L. conducted the experiments. K.L.C., J.-Q.Z. and K.-Q.X. wrote the manuscript. All authors contributed to the analysis of the data. \textbf{Competing interests:} The authors declare that they have no competing interests. \textbf{Data and materials availability:} All data needed to evaluate the conclusions in the paper are present in the paper. Additional data related to this paper may be requested from the authors.

\end{document}    
  