\documentclass[aps,pra,reprint,superscriptaddress]{revtex4-2}

\usepackage{graphicx}% Include figure files
\usepackage[caption=false,subrefformat=parens]{subfig}
\usepackage{dcolumn}% Align table columns on decimal point
\usepackage{bm}% bold math
\usepackage{amsmath,amssymb}
\usepackage{hyperref}% add hypertext capabilities
\usepackage{braket}
\usepackage[T1]{fontenc}

% Math operator
\DeclareMathOperator{\grad}{grad}% grad
\DeclareMathOperator{\divergence}{div}% div
\DeclareMathOperator{\rot}{rot}% rot
\DeclareMathOperator{\Tr}{Tr}% Tr
\DeclareMathOperator{\Real}{Re}% Re
\DeclareMathOperator{\Imag}{Im}% Im
\DeclareMathOperator{\sgn}{sgn}% sgn

% Exchange commands
% \exchange{#1}{#2}
\makeatletter
\newcommand\exchange[2]{\let\@tempa#1\let#1#2\let#2\@tempa}
\makeatother

\exchange{\Gamma}{\varGamma}
\exchange{\Delta}{\varDelta}
\exchange{\epsilon}{\varepsilon}
\exchange{\Theta}{\varTheta}
\exchange{\Lambda}{\varLambda}
\exchange{\Xi}{\varXi}
\exchange{\Pi}{\varPi}
\exchange{\Phi}{\varPhi}
\exchange{\Psi}{\varPsi}
\exchange{\Omega}{\varOmega}

\begin{document}

\title{Designing nontrivial one-dimensional Floquet topological phases using a spin-1/2 double-kicked rotor}

\author{Yusuke Koyama}
\affiliation{Department of Applied Physics, Nagoya University, Nagoya 464-8603, Japan}
\author{Kazuya Fujimoto}
\affiliation{Department of Physics, Tokyo Institute of Technology, 2-12-1 Ookayama, Meguro-ku, Tokyo 152-8551, Japan}
\author{Shuta Nakajima}
\affiliation{Center for Quantum Information and Quantum Biology, Osaka University, Toyonaka, Osaka 560-0043, Japan}
\author{Yuki Kawaguchi}
\affiliation{Department of Applied Physics, Nagoya University, Nagoya 464-8603, Japan}

\date{March 24, 2023}

\begin{abstract}
A quantum kicked rotor model is one of the promising systems to realize various Floquet topological phases.
We consider a double-kicked rotor model for a one-dimensional quasi-spin-1/2 Bose-Einstein condensate with spin-dependent and spin-independent kicks which are implementable for cold atomic experiments.
We theoretically show that the model can realize all the Altland-Zirnbauer classes with nontrivial topology in one dimension.
In the case of class CII, we show that a pair of winding numbers $(w_0,w_\pi)\in 2\mathbb{Z}\times 2\mathbb{Z}$ featuring the edge states at zero and $\pi$ quasienergy, respectively, takes various values depending on the strengths of the kicks.
We also find that the winding numbers change to $\mathbb{Z}$ when we break the time-reversal and particle-hole symmetries by changing the phase of a kicking lattice.
We numerically confirm that the winding numbers can be obtained by measuring the mean chiral displacement in the long time limit in the present case with four internal degrees of freedom.
We further propose two feasible methods to experimentally realize the spin-dependent and spin-independent kicks required for various topological phases.
\end{abstract}

\maketitle

\section{\label{sec:introduction}Introduction}
Periodic driving to design topological phases has become one of the fundamental techniques in recent studies of condensed matter physics.
A time-periodic field, such as laser fields radiating to electrons in solids and shaking external potentials for ultracold atoms, introduces periodicity in the frequency space, enabling us to manipulate the band topology~\cite{Oka_2009_Photovoltaic-Hall-effect-in-graphene,Lindner_2011_Floquet-topological-insulator-in-semiconductor-quantum-wells,Goldman_2014_Light-induced-gauge-fields-for-ultracold-atoms,Goldman_2016_Topological-quantum-matter-with-ultracold-gases-in-optical-lattices,Eckardt_2017_Colloquium:-Atomic-quantum-gases-in-periodically-driven-optical-lattices,Zhang_2018_Topological-quantum-matter-with-cold-atoms}.
It is also possible to construct discrete time evolution by a series of unitary operators, such as quantum walks~\cite{Kitagawa_2010_Exploring-topological-phases-with-quantum-walks,Kitagawa_2012_Observation-of-topologically-protected-bound-states-in-photonic-quantum-walks,Kitagawa_2012_Topological-phenomena-in-quantum-walks:-elementary-introduction-to-the-physics-of-topological-phases} and micro motions~\cite{Kitagawa_2010_Topological-characterization-of-periodically-driven-quantum-systems,Rudner_2013_Anomalous-Edge-States-and-the-Bulk-Edge-Correspondence-for-Periodically-Driven-Two-Dimensional-Systems,Carpentier_2015_Topological-Index-for-Periodically-Driven-Time-Reversal-Invariant-2D-Systems,Nathan_2015_Topological-singularities-and-the-general-classification-of-FloquetBloch-systems}, where one can design the effective Hamiltonian, defined as a logarithm of a single-period time-evolution operator, to be topologically nontrivial.
Moreover, since the topology of time-periodic systems, so-called Floquet systems, are distinct from the static ones~\cite{Eckardt_2017_Colloquium:-Atomic-quantum-gases-in-periodically-driven-optical-lattices,Cooper_2019_Topological-bands-for-ultracold-atoms,Roy_2017_Periodic-table-for-Floquet-topological-insulators,Yao_2017_Topological-invariants-of-Floquet-systems:-General-formulation-special-properties-and-Floquet-topological-defects,Ozawa_2019_Topological-photonics,Harper_2020_Topology-and-Broken-Symmetry-in-Floquet-Systems,Rudner_2020_Band-structure-engineering-and-non-equilibrium-dynamics-in-Floquet-topological-insulators}, one can access exotic phenomena that are absent in static systems, e.g., the $\pi$ modes observed in ultracold atoms~\cite{Wintersperger_2020_Realization-of-an-anomalous-Floquet-topological-system-with-ultracold-atoms,Xie_2020_Topological-Quantum-Walks-in-Momentum-Space-with-a-Bose-Einstein-Condensate} and photonic crystals~\cite{Kitagawa_2012_Observation-of-topologically-protected-bound-states-in-photonic-quantum-walks,Gao_2016_Probing-topological-protection-using-a-designer-surface-plasmon-structure,Mukherjee_2017_Experimental-observation-of-anomalous-topological-edge-modes-in-a-slowly-driven-photonic-lattice,Maczewsky_2017_Observation-of-photonic-anomalous-Floquet-topological-insulators,Cheng_2019_Observation-of-Anomalous-pi-Modes-in-Photonic-Floquet-Engineering}.

A kicked rotor is a typical Floquet system~\cite{Casati_1979_Stochastic-behavior-of-a-quantum-pendulum-under-a-periodic-perturbation,Izrailev_1990_Simple-models-of-quantum-chaos:-Spectrum-and-eigenfunctions,Raizen_1999_Quantum-Chaos-with-Cold-Atoms,Sadgrove_2011_A-Pseudoclassical-Method-for-the-Atom-Optics-Kicked-Rotor:-from-Theory-to-Experiment-and-Back} and can host various Floquet topological phases.
A quantum version of the kicked rotor was first considered in the context of quantum chaos and Anderson localization~\cite{Casati_1979_Stochastic-behavior-of-a-quantum-pendulum-under-a-periodic-perturbation,Izrailev_1990_Simple-models-of-quantum-chaos:-Spectrum-and-eigenfunctions}, which were experimentally demonstrated using a Bose-Einstein condensate (BEC) periodically subjected to optical lattice pulses~\cite{Chabe_2008_Experimental-Observation-of-the-Anderson-Metal-Insulator-Transition-with-Atomic-Matter-Waves,Lemarie_2010_Critical-State-of-the-Anderson-Transition:-Between-a-Metal-and-an-Insulator}.
A particular interest is in the case when the period of the kicking pulses is on resonance~\cite{Moore_1995_Atom-Optics-Realization-of-the-Quantum-delta-Kicked-Rotor,Ryu_2006_High-Order-Quantum-Resonances-Observed-in-a-Periodically-Kicked-Bose-Einstein-Condensate}, for which the effective Hamiltonian reduces to a tight-binding model in the momentum space.
Due to this momentum space lattice structure, topological phases can appear in the kicked rotor systems, and internal degrees of freedom play essential roles in emergence of topological phases.
For an on-resonance kicked rotor, spin internal degrees of freedom have been experimentally introduced by using a spin-dependent optical lattice, and a discrete-time quantum walk in the momentum space was demonstrated~\cite{Summy_2016_Quantum-random-walk-of-a-Bose-Einstein-condensate-in-momentum-space,Dadras_2018_Quantum-Walk-in-Momentum-Space-with-a-Bose-Einstein-Condensate,Dadras_2019_Experimental-realization-of-a-momentum-space-quantum-walk}.
Furthermore, such a kicked rotor model is theoretically shown to have the sublattice degrees of freedom when we apply two kicks in a single period setting the free evolution time between kicks to meet a high-order quantum resonance condition~\cite{Wang_2013_Exponential-quantum-spreading-in-a-class-of-kicked-rotor-systems-near-high-order-resonances}.
With these methods, exotic Floquet topological phases were theoretically studied~\cite{Zhou_2018_Floquet-topological-phases-in-a-spin-1/2-double-kicked-rotor,Zhou_2019_Non-Hermitian-Floquet-topological-phases-in-the-double-kicked-rotor,Zhou_2021_Floquet-Second-Order-Topological-Phases-in-Momentum-Space,Bolik_2022_Detecting-topological-phase-transitions-in-a-double-kicked-quantum-rotor}.

In this paper, we propose a kicked rotor model with both the spin and sublattice degrees of freedom that can realize all the Altland-Zirnbauer (AZ) classes with nontrivial topology in one dimension~\cite{Schnyder_2008_Classification-of-topological-insulators-and-superconductors-in-three-spatial-dimensions,Kitaev_2009_Periodic-table-for-topological-insulators-and-superconductors,Ryu_2010_Topological-insulators-and-superconductors:-tenfold-way-and-dimensional-hierarchy,Roy_2017_Periodic-table-for-Floquet-topological-insulators,Yao_2017_Topological-invariants-of-Floquet-systems:-General-formulation-special-properties-and-Floquet-topological-defects}.
The AZ classification characterizes a system according to the time-reversal symmetry, the particle-hole symmetry, and the chiral symmetry.
The chiral operator $\hat{\Gamma}$ is a unitary operator given by $\hat{\Gamma}=\hat{T}\hat{C}$, which always satisfies $\hat{\Gamma}^2=1$.
The time-reversal operator $\hat{T}$ and the particle-hole operator $\hat{C}$ are anti-unitary operators and their squares take $+1$ or $-1$.
Here, we would like to emphasize that an anti-unitary operator whose square is $-1$ is achieved only in a system more than two internal degrees of freedom.
Thus, to realize all the symmetry classes in the AZ classification, we need four internal degrees of freedom.
Among them, we have a particular interest in class CII, which has both the time-reversal and particle-hole symmetries with $\hat{T}^2=\hat{C}^2=-1$, and hence four internal degrees of freedom are required for emergence of class CII.
Although the class CII models have been theoretically proposed~\cite{Gentile_2015_Edge-States-and-Topological-Insulating-Phases-Generated-by-Curving-a-Nanowire-with-Rashba-Spin-Orbit-Coupling,Liu_2019_Fractional-charged-edge-states-in-ladder-topological-insulators,Zhou_2020_Floquet-topological-phases-with-fourfold-degenerate-edge-modes-in-a-driven-spin-1/2-Creutz-ladder,Zhou_2020_Non-Hermitian-Floquet-Phases-with-Even-Integer-Topological-Invariants-in-a-Periodically-Quenched-Two-Leg-Ladder,Malard_2020_Multicriticality-in-a-one-dimensional-topological-band-insulator}, the corresponding experiments have never been reported.
Besides class CII, classes AIII, BDI, D, and DIII host nontrivial topological phases in one dimension.
Classes AIII and BDI are experimentally realized in cold atomic systems~\cite{Meier_2016_Observation-of-the-topological-soliton-state-in-the-SuSchriefferHeeger-model,Meier_2018_Observation-of-the-topological-Anderson-insulator-in-disordered-atomic-wires,Xie_2019_Topological-characterizations-of-an-extended-SuSchriefferHeeger-model,Xie_2020_Topological-Quantum-Walks-in-Momentum-Space-with-a-Bose-Einstein-Condensate}.
In this paper, we show that the single kicked rotor model can access all the above classes by changing the parameters of the kicking lattices.

We consider a double-kicked rotor (DKR) model for a one-dimensional quasi-spin-1/2 BEC.
Here, the key ingredient is to simultaneously apply spin-dependent and spin-independent kicks.
We examine the symmetry of the Floquet operator and find parameter sets for realizing each AZ class.
We further investigate the phase diagram for the case of class CII in the parameter space of the kick strengths.
We confirm that a pair of winding numbers $(w_0, w_\pi)\in 2\mathbb{Z}\times 2\mathbb{Z}$ featuring the edge states at zero and $\pi$ quasienergy, respectively, take various values depending on the kick strengths.
We also find that the winding numbers change to $\mathbb{Z}$ when we break the time-reversal and particle-hole symmetries by changing the phase of a kicking lattice.
The winding numbers can be experimentally measured from the mean chiral displacement (MCD) for chiral symmetric systems~\cite{Cardano_2017_Detection-of-Zak-phases-and-topological-invariants-in-a-chiral-quantum-walk-of-twisted-photons,Meier_2018_Observation-of-the-topological-Anderson-insulator-in-disordered-atomic-wires,Maffei_2018_Topological-characterization-of-chiral-models-through-their-long-time-dynamics,Xie_2019_Topological-characterizations-of-an-extended-SuSchriefferHeeger-model,Xie_2020_Topological-Quantum-Walks-in-Momentum-Space-with-a-Bose-Einstein-Condensate,DErrico_2020_Bulk-detection-of-time-dependent-topological-transitions-in-quenched-chiral-models,St-Jean_2021_Measuring-Topological-Invariants-in-a-Polaritonic-Analog-of-Graphene,Xiao_2021_Observation-of-topological-phase-with-critical-localization-in-a-quasi-periodic-lattice}.
We numerically confirm that in the present case with the four internal degrees of freedom, the time average of the MCD converges to the winding number in the long time limit.

The rest of this paper is organized as follows.
In Sec.~\ref{sec:model}, we introduce the DKR model for a quasi-spin-1/2 BEC and calculate the time evolution operator over a single period.
In Sec.~\ref{sec:symmetry}, we discuss the symmetry properties of our system and derive the condition for realizing Floquet topological phases.
In Sec.~\ref{sec:topological_phase}, we study Floquet topological phases especially in class CII.
In Sec.~\ref{sec:experiment}, we discuss how to experimentally realize our model.
In Sec.~\ref{sec:conclusion}, we conclude this work.

\section{\label{sec:model}On-resonance spin-1/2 double-kicked rotor model}
We consider the DKR model for a one-dimensional quasi-spin-1/2 BEC, which has two spin degrees of freedom and two sublattice degrees of freedom.
We start with a general form of the spin-1/2 DKR (SDKR) model that includes the spin-dependent and spin-independent kicks.
The Hamiltonian is given by
\begin{align}
    \hat{H}(t) &= \frac{\hat{p}^2}{2M} \otimes \hat{\sigma}_0 + \hat{H}_1 \sum_{m=-\infty}^{\infty} \delta(t - mT) \notag \\
    &\quad + \hat{H}_2 \sum_{m=-\infty}^{\infty} \delta(t - T_2 - mT),\label{eq:SDKR_Hamiltonian} \\
    \hat{H}_j &= \lambda_j^0\cos\left(\frac{2\pi}{a}\nu_j^0\hat{x} + \alpha_j^0\right) \otimes \hat{\sigma}_0 \notag \\
    &\quad + \lambda_j\cos\left(\frac{2\pi}{a}\nu_j\hat{x} + \alpha_j\right) \otimes \bm{n}_j\cdot\hat{\bm{\sigma}} \quad (j=1, 2), \label{eq:SDKR_kick_Hamiltonian}
\end{align}
where $M$ is the atomic mass, $\hat{x}$ and $\hat{p}$ are the position and momentum of the atom, and $\hat{\bm{\sigma}}=(\hat{\sigma}_x,\hat{\sigma}_y,\hat{\sigma}_z)$ and $\hat{\sigma}_0$ are the vector of Pauli matrices and the identity matrix, respectively, in the spin space.
Here, we assume that the interatomic interactions are sufficiently weak and thus negligible.
The BEC is simultaneously kicked by the spin-independent and spin-dependent optical lattices twice in the single period $T$.
The duration between the first ($j=1$) and second ($j=2$) kicks is $T_2$.
The two optical lattices in each kick ($j=1, 2$) have the strengths $\lambda_j^0, \lambda_j$, the wave lengths $a/\nu_j^0, a/\nu_j$ ($\nu_j^0, \nu_j\in\mathbb{Z}$), and the phases $\alpha_j^0,\alpha_j$, respectively.
We require that $\nu_{1,2}^0$ and $\nu_{1,2}$ have no common divisor in order to ensure that the spatial period of Eq.~(\ref{eq:SDKR_Hamiltonian}) is $a$.
The unit vector $\bm{n}_j$ specifies the spin dependence of the optical lattice.
We choose $\bm{n}_1\nparallel\bm{n}_2$ so that the Hamiltonian $\hat{H}(t)$ is not block-diagonalized.
Without loss of generality, we can choose $\alpha_1^0=0$.
We will show below that this model belongs to the classes CII, AIII, BDI, D, or DIII depending on the choices of the parity of $\nu_{1,2}^0, \nu_{1,2}$ and the other phases $\alpha_2^0, \alpha_{1,2}$.

Because the Hamiltonian (\ref{eq:SDKR_Hamiltonian}) is time periodic, we investigate the symmetry property of the system by calculating the time evolution operator over a single period, i.e., the Floquet operator, $\hat{\mathcal{U}}(T)=\mathcal{T} \exp[-i\int_{-0}^{T-0}\hat{H}(t)\,dt/\hbar]$, where $\mathcal{T}$ is the time ordering operator.
By integrating Eq.~(\ref{eq:SDKR_Hamiltonian}) from $t=-0$ to $t=T-0$, we obtain
\begin{align}
    \hat{\mathcal{U}}(T) &= e^{-\frac{i}{\hbar}\frac{\hat{p}^2}{2M} \otimes \hat{\sigma}_0 (T-T_2)} e^{-\frac{i}{\hbar}\hat{H}_2} e^{-\frac{i}{\hbar}\frac{\hat{p}^2}{2M} \otimes \hat{\sigma}_0 T_2} e^{-\frac{i}{\hbar}\hat{H}_1}. \label{eq:SDKR_Floquet_op}
\end{align}
Because the Hamiltonian (\ref{eq:SDKR_Hamiltonian}) is also spatially periodic with the period $a$, the quasimomentum $\hbar(2\pi/a) \beta$ ($0\leq\beta<1$) becomes a good quantum number.
In other words, the atoms with a given $\beta$ have momentum restricted to
\begin{align}
    p&=\hbar\frac{2\pi}{a}(l+\beta) \quad (l\in\mathbb{Z}),
\end{align}
during the time evolution.
It follows that when we start from a BEC with atoms in the $\beta=0$ state, we can rewrite the momentum operator $\hat{p}$ as
\begin{align}
    \hat{p} &= \hbar\frac{2\pi}{a}\hat{l},
\end{align}
where $\hat{l}$ is the discretized momentum operator whose eigenvalues are all integers.
Here, we choose $T$ and $T_2$ as~\cite{Moore_1995_Atom-Optics-Realization-of-the-Quantum-delta-Kicked-Rotor,Wang_2013_Exponential-quantum-spreading-in-a-class-of-kicked-rotor-systems-near-high-order-resonances,Wang_2008_Proposal-of-a-cold-atom-realization-of-quantum-maps-with-Hofstadters-butterfly-spectrum}
\begin{align}
    T &= 4\pi \frac{M}{\hbar}\left(\frac{a}{2\pi}\right)^2 = \frac{h}{4E_\mathrm{R}}, \\
    T_2 &= \frac{T}{4} = \pi \frac{M}{\hbar}\left(\frac{a}{2\pi}\right)^2 = \frac{h}{16E_\mathrm{R}}, \label{eq:quantum_resonance_condition}
\end{align}
where $E_\mathrm{R}=\hbar^2 (\pi/a)^2/2M$ is the recoil energy of the optical lattice in Eq.~(\ref{eq:SDKR_kick_Hamiltonian}) with $\nu=1$.
Then, we obtain the Floquet operator for the on-resonance SDKR (ORSDKR) model:
\begin{align}
    \hat{\mathcal{U}}(T) &= e^{+i\frac{\pi}{2}\hat{l}^2 \otimes \hat{\sigma}_0} e^{-\frac{i}{\hbar}\hat{H}_2} e^{-i\frac{\pi}{2}\hat{l}^2 \otimes \hat{\sigma}_0} e^{-\frac{i}{\hbar}\hat{H}_1}. \label{eq:2_ORSDKR_Floquet_op}
\end{align}
The point here is that due to the choice of $T_2$ in Eq.~(\ref{eq:quantum_resonance_condition}), the first and the third factors in Eq.~(\ref{eq:SDKR_Floquet_op}) become the identity ($i$ and $-i$, respectively) for even (odd) eigenvalues of $\hat{l}$, introducing sublattice structure in the momentum space.

Due to the appearance of the sublattice structure, we decompose the momentum lattice basis $\{\ket{l}\}$ into even $l$ sites (A sites) and odd $l$ sites (B sites) and rewrite $\ket{l=2n}=\ket{n}\otimes\ket{\mathrm{A}}$ and $\ket{l=2n+1}=\ket{n}\otimes\ket{\mathrm{B}}$~\cite{Wang_2013_Exponential-quantum-spreading-in-a-class-of-kicked-rotor-systems-near-high-order-resonances,Zhou_2019_Non-Hermitian-Floquet-topological-phases-in-the-double-kicked-rotor}.
We further move to the quasiposition basis $\{\ket{\theta}\}$ defined by
\begin{align}
    \ket{\theta} &= \sum_{n=-\infty}^{\infty} \ket{n}\frac{1}{\sqrt{2\pi}}e^{i\theta n} \quad (-\pi<\theta\leq\pi). \label{eq:Fourier_quasiposition}
\end{align}
Here, $\theta$ is regarded as the ``Bloch wave number'' for the tight-binding model in the momentum space.
Indeed, the Floquet operator is block diagonal in terms of $\theta$ as $\hat{\mathcal{U}}(T) = \int_{-\pi}^{\pi} d\theta \ket{\theta}\bra{\theta} \otimes \hat{U}(\theta)$ where
\begin{gather}
    \hat{U}(\theta) = e^{-i \hat{h}_2(\theta)} e^{-i \hat{h}_1(\theta)}, \label{eq:ORSDKR_Floquet_op_quasiposition} \\
    \hat{h}_j(\theta) = \Lambda_j^0(\theta) \hat{h}_{\tau j}(\theta, \nu_j^0) \otimes \hat{\sigma}_0 + \Lambda_j(\theta) \hat{h}_{\tau j}(\theta, \nu_j) \otimes \bm{n}_j\cdot\hat{\bm{\sigma}}, \label{eq:ORSDKR_Hamiltonian} \\
    \hat{h}_{\tau j}(\theta, \nu) =
    \begin{cases}
        \hat{\tau}_0 & (\text{even }\nu) \\
        \bm{m}_j(\theta)\cdot\hat{\bm{\tau}} & (\text{odd }\nu)
    \end{cases} \quad (j=1, 2). \label{eq:factor_sublattice}
\end{gather}
Here, $\hat{\bm{\tau}}=(\hat{\tau}_x,\hat{\tau}_y,\hat{\tau}_z)$ and $\hat{\tau}_0$ are the vector of Pauli matrices and the identity matrix, respectively, acting on the sublattice space, and we define
\begin{align}
    \Lambda_j^0(\theta) &= \frac{\lambda_j^0}{\hbar} \cos\left(\frac{\nu_j^0}{2}\theta - \alpha_j^0\right), \\
    \Lambda_j(\theta) &= \frac{\lambda_j}{\hbar} \cos\left(\frac{\nu_j}{2}\theta - \alpha_j\right) \quad (j=1, 2), \\
    \bm{m}_1 &= \left(\cos\frac{\theta}{2}, \sin\frac{\theta}{2}, 0\right), \\
    \bm{m}_2 &= \left(-\sin\frac{\theta}{2}, \cos\frac{\theta}{2}, 0\right).
\end{align}
The detailed derivation of Eq.~(\ref{eq:ORSDKR_Floquet_op_quasiposition}) is given in Appendix~\ref{appx:basis}.

\section{\label{sec:symmetry}Symmetry properties}
\begin{table}
    \caption{\label{tab:AZ_class}Symmetry properties and topological invariants of the AZ classes that host nontrivial static and Floquet topological phases in one dimension~\cite{Schnyder_2008_Classification-of-topological-insulators-and-superconductors-in-three-spatial-dimensions,Kitaev_2009_Periodic-table-for-topological-insulators-and-superconductors,Ryu_2010_Topological-insulators-and-superconductors:-tenfold-way-and-dimensional-hierarchy,Roy_2017_Periodic-table-for-Floquet-topological-insulators,Yao_2017_Topological-invariants-of-Floquet-systems:-General-formulation-special-properties-and-Floquet-topological-defects}.}
    \begin{ruledtabular}
        \begin{tabular}{cccccc}
            Class & $\hat{T}^2$ & $\hat{C}^2$ & $\hat{\Gamma}^2$ & Static & Floquet \\
            \hline
            AIII & $0$ & $0$ & $1$ & $\mathbb{Z}$ & $\mathbb{Z}\times\mathbb{Z}$ \\
            BDI & $+1$ & $+1$ & $1$ & $\mathbb{Z}$ & $\mathbb{Z}\times\mathbb{Z}$ \\
            D & $0$ & $+1$ & $0$ & $\mathbb{Z}_2$ & $\mathbb{Z}_2\times\mathbb{Z}_2$ \\
            DIII & $-1$ & $+1$ & $1$ & $\mathbb{Z}_2$ & $\mathbb{Z}_2\times\mathbb{Z}_2$ \\
            CII & $-1$ & $-1$ & $1$ & $2\mathbb{Z}$ & $2\mathbb{Z}\times 2\mathbb{Z}$
        \end{tabular}
    \end{ruledtabular}
\end{table}

\begin{table*}
    \caption{\label{tab:class_ORSDKR}Optical lattice parameters and the corresponding symmetry operators for realizing symmetry classes in the ORSDKR.
    We list parameter sets with which the Floquet operator cannot be block-diagonalized.
    Without loss of generality, we choose $\alpha_1^0=0$.
    For the parameter sets with $\lambda_1^0=0$, we choose $\alpha_1=0$.
    We describe $\alpha_{1,2}^0,\alpha_{1,2} \in [0,\pi)$ because adding $\pi$ to them preserves the symmetry of the system.
    The symbol ``-'' in the columns $\nu_{1,2}^0,\nu_{1,2},\alpha_{1,2}^0,\alpha_{1,2}$ means that the corresponding kick must not be included, i.e., the corresponding $\lambda_{1,2}^0,\lambda_{1,2}$ must be zero.
    The symbol ``$*$'' in the columns $\nu_{1,2}^0,\nu_{1,2},\alpha_{1,2}^0,\alpha_{1,2}$ means no restriction except not belonging to the other classes.
    $\bm{n}_\perp$ is an unit vector perpendicular to both $\bm{n}_1$ and $\bm{n}_2$.}
    \begin{ruledtabular}
        \begin{tabular}{cccccccccccc}
            Class & $\nu_1^0$ & $\nu_1$ & $\nu_2^0$ & $\nu_2$ & $\alpha_1^0$ & $\alpha_1$ & $\alpha_2^0$ & $\alpha_2$ & $\hat{T}$ & $\hat{C}$ & $\hat{\Gamma}$ \\
            \hline
            AIII & odd & odd & odd & odd & $*$ & $*$ & $*$ & $*$ & - & - & $\hat{\tau}_z \otimes \hat{\sigma}_0$ \\
            AIII & - & odd & - & even & - & $*$ & - & $*$ & - & - & $\hat{\tau}_0 \otimes \bm{n}_\perp\cdot\hat{\bm{\sigma}}$ \\
            AIII & - & even & - & odd & - & $*$ & - & $*$ & - & - & $\hat{\tau}_0 \otimes \bm{n}_\perp\cdot\hat{\bm{\sigma}}$ \\
            AIII & odd & even & odd & even & $*$ & $*$ & $*$ & $*$ & - & - & $\hat{\tau}_z \otimes \bm{n}_\perp\cdot\hat{\bm{\sigma}}$ \\
            BDI & odd & odd & odd & odd & $0$ & $0$ & $\pi/2$ & $\pi/2$ & $\hat{\tau}_0 \otimes \bm{n}_\perp\cdot\hat{\bm{\sigma}} \hat{\sigma}_y \hat{K}$ & $\hat{\tau}_z \otimes \bm{n}_\perp\cdot\hat{\bm{\sigma}} \hat{\sigma}_y \hat{K}$ & $\hat{\tau}_z \otimes \hat{\sigma}_0$ \\
            D & odd & odd & $*$ & $*$ & $0$ & $0$ & $\pi/2$ & $\pi/2$ & - & $\hat{\tau}_z \otimes \bm{n}_\perp\cdot\hat{\bm{\sigma}} \hat{\sigma}_y \hat{K}$ & - \\
            D & odd & even & $*$ & $*$ & $0$ & $\pi/2$ & $\pi/2$ & $\pi/2$ & - & $\hat{\tau}_z \otimes \bm{n}_\perp\cdot\hat{\bm{\sigma}} \hat{\sigma}_y \hat{K}$ & - \\
            DIII & - & odd & - & even & - & $0$ & - & $\pi/2$ & $\hat{\tau}_z \otimes \hat{\sigma}_y \hat{K}$ & $\hat{\tau}_z \otimes \bm{n}_\perp\cdot\hat{\bm{\sigma}} \hat{\sigma}_y \hat{K}$ & $\hat{\tau}_0 \otimes \bm{n}_\perp\cdot\hat{\bm{\sigma}}$ \\
            DIII & odd & even & odd & even & $0$ & $\pi/2$ & $\pi/2$ & $\pi/2$ & $\hat{\tau}_0 \otimes \hat{\sigma}_y \hat{K}$ & $\hat{\tau}_z \otimes \bm{n}_\perp\cdot\hat{\bm{\sigma}} \hat{\sigma}_y \hat{K}$ & $\hat{\tau}_z \otimes \bm{n}_\perp\cdot\hat{\bm{\sigma}}$ \\
            CII & odd & odd & odd & odd & $0$ & $\pi/2$ & $\pi/2$ & $0$ & $\hat{\tau}_0\otimes\hat{\sigma}_y \hat{K}$ & $\hat{\tau}_z\otimes\hat{\sigma}_y \hat{K}$ & $\hat{\tau}_z\otimes\hat{\sigma}_0$
        \end{tabular}
    \end{ruledtabular}
\end{table*}

We derive the conditions for the system to belong to each AZ class.
For Floquet topological systems, it is convenient to describe the symmetry property in terms of the Floquet operator $\hat{U}(\theta)$.
Since $\theta$ corresponds to the ``Bloch wave number'', the time-reversal symmetry, the particle-hole symmetry, and the chiral symmetry for the Floquet operator are respectively given by~\cite{Roy_2017_Periodic-table-for-Floquet-topological-insulators,Yao_2017_Topological-invariants-of-Floquet-systems:-General-formulation-special-properties-and-Floquet-topological-defects}
\begin{subequations}\label{eq:symmetries_Floquet}
    \begin{align}
        \hat{T}\hat{U}(\theta)\hat{T}^{-1} &= \hat{U}^{-1}(-\theta), \label{eq:time_reversal_Floquet} \\
        \hat{C}\hat{U}(\theta)\hat{C}^{-1} &= \hat{U}(-\theta), \label{eq:particle_hole_Floquet} \\
        \hat{\Gamma}\hat{U}(\theta)\hat{\Gamma}^{-1} &= \hat{U}^{-1}(\theta), \label{eq:chiral_Floquet}
    \end{align}
\end{subequations}
where $\hat{T}$, $\hat{C}$, and $\hat{\Gamma}$ are the time-reversal, particle-hole, and chiral operators, respectively.
The former two are anti-unitary operators, whereas the last one, $\hat{\Gamma}$, is a unitary operator equal to $\hat{T}\hat{C}$ up to a phase factor.
In Table~\ref{tab:AZ_class}, we summarize the symmetry properties and topological invariants of the AZ classes that have nontrivial topological phases in one dimension.

Because we are considering the time-dependent Hamiltonian, the time-reversal and chiral symmetries exist only under proper choices of the origin of the time axis.
For the case of $\hat{U}(\theta)$ in Eq.~(\ref{eq:ORSDKR_Floquet_op_quasiposition}) composed of two successive unitary operators, we can define the Floquet operators in the two symmetric time frames as~\cite{Asboth_2013_Bulk-boundary-correspondence-for-chiral-symmetric-quantum-walks}
\begin{align}
    \hat{U}_1(\theta) &= e^{-i \hat{h}_1(\theta) / 2} e^{-i \hat{h}_2(\theta)} e^{-i \hat{h}_1(\theta) / 2}, \label{eq:symmetric_time_frame_1} \\
    \hat{U}_2(\theta) &= e^{-i \hat{h}_2(\theta) / 2} e^{-i \hat{h}_1(\theta)} e^{-i \hat{h}_2(\theta) / 2}. \label{eq:symmetric_time_frame_2}
\end{align}
It follows that both $\hat{U}_1(\theta)$ and $\hat{U}_2(\theta)$ satisfy Eq.~(\ref{eq:symmetries_Floquet}) when both $e^{-i \hat{h}_1(\theta)}$ and $e^{-i \hat{h}_2(\theta)}$ satisfy Eq.~(\ref{eq:symmetries_Floquet}), which in turn is rewritten as
\begin{subequations}
    \begin{align}
        \hat{T} \hat{h}_j(\theta) \hat{T}^{-1} &= \hat{h}_j(-\theta), \label{eq:time_reversal_Hamiltonian} \\
        \hat{C} \hat{h}_j(\theta) \hat{C}^{-1} &= -\hat{h}_j(-\theta), \label{eq:particle_hole_Hamiltonian} \\
        \hat{\Gamma} \hat{h}_j(\theta) \hat{\Gamma}^{-1} &= -\hat{h}_j(\theta). \label{eq:chiral_Hamiltonian}
    \end{align}
    \label{eq:symmetries_Hamiltonian}
\end{subequations}

We first consider the case of class CII.
What we have to do is to find the operators $\hat{T}$, $\hat{C}$, and $\hat{\Gamma}$ that satisfy Eq.~(\ref{eq:symmetries_Hamiltonian}) with $\hat{h}_j$ given in Eq.~(\ref{eq:ORSDKR_Hamiltonian}).
Here, $\hat{T}$ and $\hat{C}$ should satisfy $\hat{T}^2=\hat{C}^2=-1$.
First, for the system to have the chiral symmetry, $\hat{h}_j(\theta)$ should not include the identity matrix, $\hat{\tau}_0\otimes\hat{\sigma}_0$, which leads to that both $\nu_1^0$ and $\nu_2^0$ are odd integers.
The chiral operator that anti-commutes with the first term of Eq.~(\ref{eq:ORSDKR_Hamiltonian}) is $\hat{\Gamma}=\hat{\tau}_z\otimes\hat{\sigma}_0$, which also anti-commutes with the second term of Eq.~(\ref{eq:ORSDKR_Hamiltonian}) only when $\nu_1$ and $\nu_2$ are also odd integers.
For this $\hat{\Gamma}$, a possible pair of the time-reversal and particle-hole operators are $\hat{\tau}_0\otimes\hat{\sigma}_y \hat{K}$ and $\hat{\tau}_z\otimes\hat{\sigma}_y \hat{K}$, where $\hat{K}$ is the complex conjugate operator:
The changes in $\hat{h}_j(\theta)$ under the other choices of $\hat{T}$ and $\hat{C}$ are not written in simple sign changes as in Eqs.~(\ref{eq:time_reversal_Hamiltonian}) and (\ref{eq:particle_hole_Hamiltonian}).
Because we have chosen $\alpha_1^0=0$ for which $\Lambda_1^0(\theta)$ is an even function of $\theta$, we choose $\hat{T}=\hat{\tau}_0\otimes \hat{\sigma}_y \hat{K}$ and $\hat{C}=\hat{\tau}_z\otimes \hat{\sigma}_y \hat{K}$ such that they satisfy Eqs.~(\ref{eq:time_reversal_Floquet}) and (\ref{eq:particle_hole_Floquet}).
It follows that $\Lambda_2(\theta)$  [$\Lambda_1(\theta)$ and $\Lambda_2^0(\theta)$] should be an even function [odd functions] of $\theta$, resulting in $\alpha_2=0$ and $\alpha_1=\alpha_2^0=\pm\pi/2$.
This is the condition for the system to belong class CII.

We stress here that the existence of the spin-independent lattice is crucial for the realization of class CII system.
Otherwise, $\hat{h}_1(\theta)$ and $\hat{h}_2(\theta)$ commute with the same operator $\hat{\tau}_z\otimes\bm{n}_\perp\cdot\hat{\bm{\sigma}}$ with $\bm{n}_\perp$ being a unit vector perpendicular to both $\bm{n}_1$ and $\bm{n}_2$, which means that the system is merely a combination of the systems written by $2\times 2$ matrices.

Similarly to the class CII case, we obtain the conditions for the ORSDKR to belong to the symmetry classes that possess nontrivial topological phases in one dimension.
The detailed derivation is given in Appendix~\ref{appx:symmetry}, and we summarize the result in Table~\ref{tab:class_ORSDKR}.

\section{\label{sec:topological_phase}Topological phases in class CII}
\subsection{\label{subsec:winding_number}Winding number}
Since the ORSDKR has the chiral symmetry, we can characterize the topological phases with the winding numbers calculated as follows.
Defining the effective Hamiltonian $\hat{H}_{j,\mathrm{eff}}(\theta)$ from the relation $\hat{U}_j(\theta)=e^{-i\hat{H}_{j,\mathrm{eff}}(\theta) T/\hbar}$ ($j=1,2$), we introduce $Q$ matrix as~\cite{Asboth_2013_Bulk-boundary-correspondence-for-chiral-symmetric-quantum-walks}
\begin{align}
    \hat{Q}_j(\theta) &= \sin\frac{\hat{H}_{j,\mathrm{eff}}(\theta) T}{\hbar} = \frac{\hat{U}_j^\dagger(\theta) - \hat{U}_j(\theta)}{2i}.
\end{align}
In our model, the chiral operator $\hat{\Gamma}=\hat{\tau}_z\otimes\hat{\sigma}_0$ is diagonal, and hence $Q$ matrix is block off-diagonal,
\begin{align}
    \hat{Q}_j(\theta) &= \begin{pmatrix}
        0 & \hat{q}_j(\theta) \\
        \hat{q}_j^\dagger(\theta) & 0
    \end{pmatrix}.
\end{align}
By using the off-diagonal element $\hat{q}_j(\theta)$, we can calculate the winding number as
\begin{align}
    w_j &= \int_{-\pi}^{\pi} \frac{i}{2\pi} \Tr \left[\hat{q}_j^{-1}(\theta) \frac{\partial}{\partial \theta} \hat{q}_j(\theta)\right] \,d\theta. \label{eq:winding_number}
\end{align}
Here, $w_1$ and $w_2$ are the winding numbers calculated for the Floquet operators $\hat{U}_1$ and $\hat{U}_2$, respectively, in the symmetric time frames.
On the other hand, the winding numbers related to the number of edge states at quasienergies $\epsilon=0$ and $\pi\hbar/T$, if the system has edges, are given by~\cite{Asboth_2013_Bulk-boundary-correspondence-for-chiral-symmetric-quantum-walks}
\begin{align}
    w_0 = \frac{w_1+w_2}{2},\quad w_\pi = \frac{w_1-w_2}{2}. \label{eq:winding_number_Floquet}
\end{align}

\subsection{\label{subsec:phase_diagram}Phase diagram}
To satisfy the class CII symmetries, $\bm{n}_1\nparallel\bm{n}_2$ is required.
Otherwise, both $\hat{h}_1(\theta)$ and $\hat{h}_2(\theta)$ commute with the same operator $\hat{\tau}_0\otimes\bm{n}_1\cdot\hat{\bm{\sigma}}$.
We remember that all $\nu_{1,2}^0$ and $\nu_{1,2}$ must be odd numbers.
The cases $(\nu_{1,2}^0,\nu_{1,2})=(1,1)$ and $(1,3)$ would be experimentally implementable (see Sec.~\ref{sec:experiment}).
Here, we show the results for $(\nu_{1,2}^0,\nu_{1,2})=(1,1)$.
We have not seen significant difference for other choices of $(\nu_{1,2}^0,\nu_{1,2})$.

\begin{figure*}
    \includegraphics[width=\linewidth]{fig1.pdf}
    \caption{\label{fig:winding_number}Topological phase diagram of the ORSDKR.
    (a) Winding number $w_0$ at quasienergy $\epsilon=0$ and (b) winding number $w_\pi$ at quasienergy $\epsilon=\pi\hbar/T$.
    The parameters are fixed as $\lambda_1^0=0.3\pi\hbar$, $\lambda_2^0=0.4\pi\hbar$, $\bm{n}_1\cdot\bm{n}_2=1/\sqrt{2}(\neq\pm 1)$, $\nu_{1,2}^0=1$, $\nu_{1,2}=1$, $\alpha_1^0=\alpha_2=0$, and $\alpha_1=\alpha_2^0=-\pi/2$.
    The green dashed lines and the yellow filled circles indicate the values of $\lambda_{1,2}$ used for Figs.~\ref{fig:winding_number_CII_AIII} and \ref{fig:MCD}, respectively.}
\end{figure*}

Figure~\ref{fig:winding_number} shows the winding numbers $w_0$ and $w_\pi$ as functions of the kick strengths $\lambda_1$ and $\lambda_2$.
Here, we set the parameters as $\lambda_1^0=0.3\pi\hbar$, $\lambda_2^0=0.4\pi\hbar$, $\bm{n}_1\cdot\bm{n}_2=1/\sqrt{2}(\neq\pm 1)$, $\nu_{1,2}^0=1$, $\nu_{1,2}=1$, $\alpha_1^0=\alpha_2=0$, and $\alpha_1=\alpha_2^0=-\pi/2$.
We confirm that both $w_0$ and $w_\pi$ are even integers, being consistent with $2\mathbb{Z}\times 2\mathbb{Z}$ in Table~\ref{tab:AZ_class}.
Figure~\ref{fig:winding_number} shows that we can access various phases by changing the kick strengths.

It is interesting to see how the phase diagram changes as the symmetry changes.
In the ORSDKR model, it is possible to change the system from class CII to class AIII by slightly changing one of the phases $\alpha$ from the fixed values for class CII.
(See the first and last rows in Table~\ref{tab:class_ORSDKR}.)

\begin{figure*}
    \includegraphics[width=\linewidth]{fig2.pdf}
    \subfloat{\label{fig:winding_number_CII}}
    \subfloat{\label{fig:winding_number_AIII}}
    \caption{\label{fig:winding_number_CII_AIII}Winding numbers $w_0$ (blue solid line) and $w_\pi$ (orange dashed line) of the ORSDKR in (a) class CII and (b) class AIII as functions of the kick strength $\lambda_2$.
    (a) The parameters are the same as those in Fig.~\ref{fig:winding_number} and we choose $\lambda_1=2.5\pi\hbar$ (green dashed lines in Fig.~\ref{fig:winding_number}).
    (b) The parameters are the same as those of (a) except for $\alpha_2$:
    We choose $\alpha_2=0.1\pi$ so that the system does not have the time-reversal and particle-hole symmetries.
    At the point where one of the winding numbers in (a) jumps by 2, a new phase with an odd winding number appears in (b).}
\end{figure*}

We show the winding numbers $w_0$ and $w_\pi$ as functions of the kick strength $\lambda_2$ in class CII [Fig.~\subref*{fig:winding_number_CII}] and AIII [Fig.~\subref*{fig:winding_number_AIII}].
Here, we fix $\lambda_1=2.5\pi\hbar$ and choose $\alpha_2=0$ for class CII [Fig.~\subref*{fig:winding_number_CII}] and $0.1\pi$ for class AIII [Fig.~\subref*{fig:winding_number_AIII}].
The other parameters are the same as those in Fig.~\ref{fig:winding_number}.
At small $\lambda_2$, the winding numbers for class CII and AIII have similar structures.
However, there is a significant difference that at the point where one of the winding numbers in class CII changes by 2, a new phase with an odd winding number arises in class AIII (see, e.g., at $\lambda_2/\pi\hbar=0.3$).
The deviation of the winding numbers between the two classes enlarges as $\lambda_2$ becomes large.
We note that the structures of $w_0$ and $w_\pi$ in class AIII approaches the one in class CII as $\alpha_2$ becomes closer to zero.

It is also possible to change the system from class CII to class BDI using the same lattice potential but with changing $\alpha_1$ and $\alpha_2$ by $\pi/2$.
In this case, the winding numbers change drastically because the symmetry operators undergo discrete change (see Table~\ref{tab:class_ORSDKR}).

Here, we comment on the bulk-edge correspondence for the ORSDKR in class CII.
We can theoretically consider a finite system in the momentum space by artificially restricting the possible momentum value to a finite range~\cite{Zhou_2018_Floquet-topological-phases-in-a-spin-1/2-double-kicked-rotor,Zhou_2019_Non-Hermitian-Floquet-topological-phases-in-the-double-kicked-rotor}.
By calculating the eigenenergies $\epsilon$ and eigenstates $\ket{\psi}$ of $\hat{\mathcal{U}}(T)$ through $\hat{\mathcal{U}}(T) \ket{\psi} = e^{-i\epsilon T/\hbar} \ket{\psi}$, we can obtain the edge states in a finite system.
We have confirmed that the winding numbers $(w_0,w_\pi)$ and the number of edge states $(n_0,n_\pi)$ at zero and $\pi$ quasienergies are related as $(n_0,n_\pi)=(2|w_0|,2|w_\pi|)$.

\subsection{\label{sec:MCD}Mean chiral displacement}
The winding numbers can be experimentally measured from the MCD for chiral symmetric systems~\cite{Cardano_2017_Detection-of-Zak-phases-and-topological-invariants-in-a-chiral-quantum-walk-of-twisted-photons,Meier_2018_Observation-of-the-topological-Anderson-insulator-in-disordered-atomic-wires,Maffei_2018_Topological-characterization-of-chiral-models-through-their-long-time-dynamics,Xie_2019_Topological-characterizations-of-an-extended-SuSchriefferHeeger-model,Xie_2020_Topological-Quantum-Walks-in-Momentum-Space-with-a-Bose-Einstein-Condensate,DErrico_2020_Bulk-detection-of-time-dependent-topological-transitions-in-quenched-chiral-models,St-Jean_2021_Measuring-Topological-Invariants-in-a-Polaritonic-Analog-of-Graphene,Xiao_2021_Observation-of-topological-phase-with-critical-localization-in-a-quasi-periodic-lattice}.
The MCD is the signed momentum distribution of the chiral symmetric system and defined as
\begin{align}
    C_j(t) &= 4\Tr[\hat{\rho}_0 \hat{\mathcal{U}}_j^\dagger(t) (\hat{n}\otimes\hat{\Gamma}) \hat{\mathcal{U}}_j(t)], \label{eq:MCD}
\end{align}
where $\hat{n}$ is the momentum operator defined by $\hat{n}\ket{n}=n\ket{n}$ with $\ket{n}$ being the momentum lattice basis defined below Eq.~(\ref{eq:2_ORSDKR_Floquet_op}), and $\hat{\mathcal{U}}_j(t)$ ($j=1,2$) is the time evolution operator corresponding to the Floquet operator $\hat{U}_j(\theta)$ defined in the symmetric time frame.
For example, when $t=mT$ ($m=1,2,\cdots$), we have $\hat{\mathcal{U}}_j(t)=[\hat{\mathcal{U}}_j(T)]^m$ with $\hat{\mathcal{U}}_j(T)=\int_{-\pi}^{\pi} d\theta \ket{\theta}\bra{\theta} \otimes \hat{U}_j(\theta)$.
The initial density matrix $\hat{\rho}_0$ is here assumed to be
\begin{align}
    \hat{\rho}_0 &= \ket{n=0}\bra{n=0} \otimes \frac{\hat{\tau}_0 \otimes \hat{\sigma}_0}{4}. \label{eq:initial_density_matrix}
\end{align}
Equation~(\ref{eq:MCD}) is experimentally evaluated as follows:
Prepare the initial state in a pure state $\ket{\psi_{\tau \sigma}} = \ket{n=0} \otimes \ket{\tau}\otimes\ket{\sigma}$, ($\tau=\mathrm{A,B}$ and $\sigma=\uparrow,\downarrow$), and observe the time evolution of the expectation value of $\hat{n}\otimes\hat{\Gamma}$.
Repeating this procedure for the initial states with all the combinations of $\tau = \mathrm{A, B}$ and $\sigma = \uparrow, \downarrow$, we obtain the expectation value for each initial state and then take the summation for them, which becomes the MCD.
We note that $\ket{\psi_{\tau \sigma}}$ is the eigenstate of $\hat{\Gamma} = \hat{\tau}_z \otimes \hat{\sigma}_0$ and expressed as $\ket{\psi_{\mathrm{A} \sigma}} = \ket{l=0} \otimes \ket{\sigma}$ and $\ket{\psi_{\mathrm{B} \sigma}} = \ket{l=1} \otimes \ket{\sigma}$, where $\ket{l=0}$ and $\ket{l=1}$ are the eigenstates of $\hat{p}$ with eigenvalues $0$ and $h/a$, respectively.
Thus, from the measurement of the momentum distribution for each spin state after a certain period, we can obtain the MCD.

On the other hand, in the quasiposition basis the MCD at $t=mT$ is written as
\begin{align}
    C_j(mT) &= \int_{-\pi}^{\pi} \frac{d\theta}{2\pi} \Tr\left[[\hat{U}_j^\dagger(\theta)]^m \hat{\Gamma} i \frac{\partial}{\partial \theta} [\hat{U}_j(\theta)]^m\right], \label{eq:MCD_trace_position}
\end{align}
which converges to the winding number $w_j$ as $m\to\infty$~\cite{Maffei_2018_Topological-characterization-of-chiral-models-through-their-long-time-dynamics}.
Thus, from the MCDs $C_1(mT)$ and $C_2(mT)$ defined with the Floquet operators in the symmetric time frames $\hat{U}_1(\theta)$ and $\hat{U}_2(\theta)$, we obtain the winding numbers $w_0$ and $w_\pi$ through
\begin{align}
    C_0(mT) &= \frac{C_1(mT) + C_2(mT)}{2},\\
    C_\pi(mT) &= \frac{C_1(mT) - C_2(mT)}{2}, \label{eq:MCD_0_pi}
\end{align}
at a sufficiently long time.
The convergence becomes faster when we take the time average of the MCDs.
Here, we consider the time average of the MCD at $t=mT$ as
\begin{align}
    \overline{C}_j(t=mT) &= \frac{1}{m} \sum_{k=1}^{m} C_j(kT), \label{eq:MCD_average}
\end{align}
which converges to $w_j$ as $m\to\infty$~\cite{Zhou_2018_Floquet-topological-phases-in-a-spin-1/2-double-kicked-rotor,Zhou_2019_Non-Hermitian-Floquet-topological-phases-in-the-double-kicked-rotor}.

\begin{figure}
    \includegraphics[width=\linewidth]{fig3.pdf}
    \caption{\label{fig:MCD}MCDs $C_0=(C_1+C_2)/2$ (blue solid curve) and $C_\pi=(C_1-C_2)/2$ (orenge solid curve), and their time averages $\overline{C_0}$ (green dashed curve) and $\overline{C_\pi}$ (red dotted curve).
    The parameters are the same as those in Fig.~\ref{fig:winding_number} and we choose $\lambda_1=3.0\pi\hbar$ and $\lambda_2=0.7\pi\hbar$ (yellow filled circles in Fig.~\ref{fig:winding_number}), for which a pair of the winding numbers is $(w_0,w_\pi)=(4,-2)$.
    $C_0$ and $C_\pi$ oscillate and converge to $w_0$ and $w_\pi$, whereas $\overline{C}_0$ and $\overline{C}_\pi$ converge more rapidly.}
\end{figure}

Figure~\ref{fig:MCD} shows the time evolution of the MCDs $C_0$ and $C_\pi$ in class CII.
The parameters are the same as those in Fig.~\ref{fig:winding_number} and we choose $\lambda_1=3.0\pi\hbar$ and $\lambda_2=0.7\pi\hbar$.
The corresponding points are depicted with the yellow filled circles in Fig.~\ref{fig:winding_number}, for which a pair of the winding numbers is given by $(w_0,w_\pi)=(4,-2)$.
The MCDs $C_0$ and $C_\pi$ converge to the winding numbers $w_0$ and $w_\pi$, respectively, as time evolves.
One can clearly see that the time averaged MCDs $\overline{C}_0$ and $\overline{C}_\pi$ converge to $w_0$ and $w_\pi$ faster than $C_0$ and $C_\pi$.

\section{\label{sec:experiment}Proposal for experimental setup}
\begin{figure*}
    \includegraphics[width=0.9\linewidth]{fig4.pdf}
    \subfloat{\label{fig:setup_lin_circ}}
    \subfloat{\label{fig:setup_lin_enclosing}}
    \caption{\label{fig:setup}(a) Optical lattices created by linearly and circularly polarized lasers.
    A spin-dependent lattice for $\ket{\uparrow}\equiv\ket{F=2,m_F=2}$ and $\ket{\downarrow}\equiv\ket{F=1,m_F=1}$ of $^{87}$Rb can be created by the standing wave of the circularly polarized 789~nm laser~\cite{Wen_2021_Experimental-study-of-tune-out-wavelengths-for-spin-dependent-optical-lattice-in-87Rb-Bose--Einstein-condensation}.
    A spin-independent lattice for both spin states is created by the standing wave of the linearly polarized 789~nm laser.
    To avoid unwanted interference between the spin-dependent and spin-independent lattices, the laser frequency of the circularly polarized laser and the linearly polarized laser is slightly (typically a few hundred MHz) deviated from each other.
    Despite this small frequency difference, the spin-dependent and spin-independent lattices can be regarded as having the same lattice constant in the region of the atom cloud.
    The phase of each lattice can be switched by changing the laser frequency of each lattice beam.
    The spin-dependent global phase shift can be removed by a spin-dependent offset cancel beam created by a running laser beam with opposite circular polarization.
    This spin-dependent offset cancel beam also cancels unwanted spin-dependent confinement/deconfinement of radial direction of the spin-dependent lattice beam.
    (b) Optical lattices created by linearly polarized lasers with an enclosing angle.
    The phase difference $\alpha_{j +} - \alpha_{j -} = 2\alpha_{j +}$ between two optical lattice potentials for $\ket{\uparrow}\equiv\ket{F=1,m_F=1}$ and $\ket{\downarrow}\equiv\ket{F=1,m_F=-1}$ can be tuned by changing the linear polarization vectors enclosing an angle $\Theta$ which can be switched by using an electro-optical modulator~\cite{Mandel_2003_Coherent-Transport-of-Neutral-Atoms-in-Spin-Dependent-Optical-Lattice-Potentials,Mandel_2003_Controlled-collisions-for-multi-particle-entanglement-of-optically-trapped-atoms}.}
\end{figure*}

In this section, we discuss how to experimentally implement the Hamiltonian~(\ref{eq:SDKR_Hamiltonian}).
After explaining the method to choose the direction of $\bm n_{1,2}$ in Sec.~\ref{subsec:spin_dependence_z}, we propose two methods to simultaneously create spin-dependent and spin-independent lattice potentials in Secs.~\ref{subsec:linear_circular_polarization} and \ref{subsec:inclined_linear_polarization}.
The experimental setup is summarized in Fig.~\ref{fig:setup}.

\subsection{\label{subsec:spin_dependence_z}Spin dependence of the kick}
We first show that the direction specifying spin dependence of the kick can be arbitrarily chosen if we can implement $\hat{\sigma}_z$-dependent optical lattice
\begin{align}
    \hat{V}_z &= V(\hat{x}) \otimes \hat{\sigma}_z, \label{eq:kick_z}
\end{align}
and the spin rotation operator proposed in Refs.~\cite{Summy_2016_Quantum-random-walk-of-a-Bose-Einstein-condensate-in-momentum-space,Dadras_2018_Quantum-Walk-in-Momentum-Space-with-a-Bose-Einstein-Condensate,Dadras_2019_Experimental-realization-of-a-momentum-space-quantum-walk}
\begin{align}
    \hat{M}(\alpha,\chi) &= e^{-i(\alpha/2)(\sin\chi \hat{\sigma}_x - \cos\chi \hat{\sigma}_y)}. \label{eq:coin_op}
\end{align}
Using Eq.~(\ref{eq:coin_op}), the spin direction in Eq.~(\ref{eq:kick_z}) can be arbitrarily rotated as
\begin{align}
    \hat{M}(-\alpha,\chi) \hat{V}_z \hat{M}(\alpha,\chi) &= V(\hat{x}) \otimes \bm{n}\cdot\hat{\bm{\sigma}}, \label{eq:kick_potential_arbitrary}
\end{align}
where $\bm{n} = (\sin\alpha \cos\chi, \sin\alpha \sin\chi, \cos\alpha)$.
When the lattice potential in Eq.~(\ref{eq:kick_z}) is applied during an infinitesimal time $\Delta t$, the whole time evolution operator is given by
\begin{align}
    \hat{M}(-\alpha,\chi) e^{-i \hat{V}_z \Delta t / \hbar} \hat{M}(\alpha,\chi) &= e^{-i V(\hat{x}) \otimes \bm{n}\cdot\hat{\bm{\sigma}} \Delta t / \hbar}.
\end{align}
Thus, we can realize the kick potential with arbitrary spin dependence as given in Eq.~(\ref{eq:kick_potential_arbitrary}) by sandwiching the kick $\hat{V}_z$ between $\hat{M}(-\alpha,\chi)$ and $\hat{M}(\alpha,\chi)$.

\subsection{\label{subsec:linear_circular_polarization}Optical lattices created by linearly and circularly polarized lasers}
The spin-independent and spin-dependent kicks can be realized by applying lasers with linear and circular polarizations to a quasi-spin-1/2 BEC.
From the discussion in Sec.~\ref{subsec:spin_dependence_z}, the required spin-independent and spin-dependent potentials are $V^0(\hat{x}) \otimes \hat{\sigma}_0$ and $V(\hat{x}) \otimes \hat{\sigma}_z$, respectively, where
\begin{align}
    V^0(x) &= \lambda^0 \cos\left(\frac{2\pi}{a} \nu^0 x + \alpha^0\right), \label{eq:kick_potential_x_spin_independent} \\
    V(x) &= \lambda \cos\left(\frac{2\pi}{a} \nu x + \alpha\right). \label{eq:kick_potential_x_spin_dependent}
\end{align}
We consider a mixture of $^{87}$Rb BECs in the hyperfine levels $\ket{\uparrow}\equiv\ket{F=2,m_F=2}$ and $\ket{\downarrow}\equiv\ket{F=1,m_F=1}$.
The spin rotation between the two hyperfine states is realized by appliying microwave pulses~\cite{Summy_2016_Quantum-random-walk-of-a-Bose-Einstein-condensate-in-momentum-space,Dadras_2018_Quantum-Walk-in-Momentum-Space-with-a-Bose-Einstein-Condensate,Dadras_2019_Experimental-realization-of-a-momentum-space-quantum-walk,Mandel_2003_Coherent-Transport-of-Neutral-Atoms-in-Spin-Dependent-Optical-Lattice-Potentials,Mandel_2003_Controlled-collisions-for-multi-particle-entanglement-of-optically-trapped-atoms}.

These potentials can be realized by two counter-propagating laser beams with linear parallel polarization (lin~$\parallel$~lin) and circular parallel polarization (circ~$\parallel$~circ) [see Fig.~\subref*{fig:setup_lin_circ}].
According to Ref.~\cite{Wen_2021_Experimental-study-of-tune-out-wavelengths-for-spin-dependent-optical-lattice-in-87Rb-Bose--Einstein-condensation}, these laser beams induce an energy shift for each state $\ket{F,m_F}$ given by
\begin{align}
    \Delta U_{\ket{F,m_F}}^{\varepsilon\parallel\varepsilon}(x) &= U_{\ket{F,m_F}}^{\varepsilon\parallel\varepsilon}\cos^2 kx, \label{eq:energy_shift}
\end{align}
where $\varepsilon=\mathrm{lin}$ and circ is for linear polarization and for circular polarization, and $k$ is the wavenumber of the laser beam.
If the wavelength is sufficiently far from the corresponding optical transitions, the amplitude in the lin~$\parallel$~lin configuration satisfies $U_{\ket{F=2,m_F=2}}^\mathrm{lin\parallel lin}=U_{\ket{F=1,m_F=1}}^\mathrm{lin\parallel lin}$ (see Fig.~3 in Ref.~\cite{Wen_2021_Experimental-study-of-tune-out-wavelengths-for-spin-dependent-optical-lattice-in-87Rb-Bose--Einstein-condensation}), which corresponds to the spin-independent optical lattice.
On the other hand, at the wavelength of about 789~nm, the amplitude in the circ~$\parallel$~circ configuration satisfies $U_{\ket{F=2,m_F=2}}^{\mathrm{circ}\parallel\mathrm{circ}}=-U_{\ket{F=1,m_F=1}}^{\mathrm{circ}\parallel\mathrm{circ}}$ (see Fig.~5 in Ref.~\cite{Wen_2021_Experimental-study-of-tune-out-wavelengths-for-spin-dependent-optical-lattice-in-87Rb-Bose--Einstein-condensation}), which corresponds to the spin-dependent optical lattice.
Thus, at this wavelength, both spin-independent and spin-dependent optical lattice pulses can be realized.
Note that since Eq.~(\ref{eq:energy_shift}) has a constant offset term coming from $\cos^2 kx = (1 + \cos 2kx)/2$, the resulting potentials are not directly equal to $V^0(\hat{x}) \otimes \hat{\sigma}_0$ for $\varepsilon=\mathrm{lin}$ and $V(\hat{x}) \otimes \hat{\sigma}_z$ for $\varepsilon=\mathrm{circ}$; $U_{\ket{F,m_F}}^\mathrm{lin\parallel lin}/2$ induces a grobal phase shift, and hence, it is irrelevant; $U_{\ket{F,m_F}}^{\mathrm{circ}\parallel\mathrm{circ}}/2$ creates a spin-dependent phase shift, which should be removed by, e.g., applying a running laser beam with opposite circular polarization [see Fig.~\subref*{fig:setup_lin_circ}].
It is worth noticing that although the wavelength of the spin-dependent lattice is fixed due to the nature of the spin dependence, we can change the wavelength of the spin-independent lattice in a vast range.
For example, when we use the linearly polarized beams with the wavelength of 394.5~nm or 1578~nm for the spin-independent lattice, we can realize $(\nu_{1,2}^0,\nu_{1,2}) = (2,1),(1,2)$, respectively.
Also, we can even realize $(\nu_{1,2}^0,\nu_{1,2}) = (1, n)$ ($n>2$) by introducing an optical lattice generated by the interference of two laser beams propagating at a relative angle of $\theta_\mathrm{L}$.
For example, by intersecting two $\lambda_\mathrm{L} = 1064~\mathrm{nm}$ laser beams with relative angle $\theta_\mathrm{L}=53.4^\circ$, we can create an optical lattice with a lattice spacing of $d=\lambda_\mathrm{L}/[2\sin(\theta_\mathrm{L}/2)]=1184~\mathrm{nm}$, which corresponds $n=3$ case.

\subsection{\label{subsec:inclined_linear_polarization}Optical lattices created by linearly polarized lasers with an enclosing angle}
In this subsection, we propose a more feasible setup for the class CII case in Table~\ref{tab:class_ORSDKR} than the setup in Sec.~\ref{subsec:linear_circular_polarization}.
More concretely, Eq.~(\ref{eq:SDKR_kick_Hamiltonian}) with $\nu_j^0=\nu_j$ ($j=1,2$) can be realized by using only a pair of linearly polarized counter-propagating beams with an enclosing angle, whereas the setup in Sec.~\ref{subsec:linear_circular_polarization} requires two pairs of counter-propagating beams and an offset cancel beam.
The required kick Hamiltonian is Eq.~(\ref{eq:SDKR_kick_Hamiltonian}) with $\bm{n}_j=(0, 0, 1)$, which is written as
\begin{gather}
    \hat{H}_j = \begin{pmatrix}
        \hat{H}_{j +} & 0 \\
        0 & \hat{H}_{j -}
    \end{pmatrix}, \\
    \hat{H}_{j \pm} = \lambda_j^0\cos\left(\frac{2\pi}{a}\nu_j^0\hat{x} + \alpha_j^0\right) \pm \lambda_j\cos\left(\frac{2\pi}{a}\nu_j\hat{x} + \alpha_j\right). \label{eq:Hamiltonian_experiment}
\end{gather}
Assuming $\nu_j^0=\nu_j$, we can rewrite Eq.~(\ref{eq:Hamiltonian_experiment}) as
\begin{gather}
    \hat{H}_{j \pm} = \lambda_{j \pm} \cos\left(\frac{2\pi}{a}\nu_j\hat{x} + \alpha_{j \pm}\right), \label{eq:Hamiltonian_CII} \\
    \lambda_{j \pm} = \sqrt{(\lambda_j^0)^2 + (\lambda_j)^2 \pm 2 \lambda_j^0 \lambda_j \cos(\alpha_j^0 - \alpha_j)}, \\
    \tan\alpha_{j \pm} = \frac{\lambda_j^0 \sin\alpha_j^0 \pm \lambda_j \sin\alpha_j}{\lambda_j^0 \cos\alpha_j^0 \pm \lambda_j \cos\alpha_j}.
\end{gather}
Here, we especially focus on the class CII case in Table~\ref{tab:class_ORSDKR}.
Substituting $\alpha_1^0=0$, $\alpha_1=-\pi/2$, $\alpha_2^0=-\pi/2$, $\alpha_2=0$, and $\nu_{1,2}^0=\nu_{1,2}=1$, we obtain
\begin{align}
    \lambda_{1+} = \lambda_{1-} &= \sqrt{(\lambda_1^0)^2 + (\lambda_1)^2}, \label{eq:lambda1p} \\
    \alpha_{1+} = -\alpha_{1-} &= -\tan^{-1}\frac{\lambda_1}{\lambda_1^0}, \label{eq:alpha1p} \\
    \lambda_{2+} = \lambda_{2-} &= \sqrt{(\lambda_2^0)^2 + (\lambda_2)^2}, \label{eq:lambda2p} \\
    \alpha_{2+} = -\alpha_{2-} &= -\tan^{-1}\frac{\lambda_2^0}{\lambda_2}. \label{eq:alpha2p}
\end{align}
Thus, the optical potential that we need is the one that acts on spin-up and down atoms with the same amplitude and the phase difference $\alpha_{j+}-\alpha_{j-}=2\alpha_{j+}$.

To realize the quasi-spin-1/2 system belonging to class CII, we consider a mixture of $^{87}$Rb BECs in the hyperfine levels $\ket{\uparrow}\equiv\ket{F=1,m_F=1}$ and $\ket{\downarrow}\equiv\ket{F=1,m_F=-1}$.
The spin rotation between the two hyperfine states can be realized by using the Raman transition.
However, these two hyperfine states are unstable against the collision toward the $\ket{F=1,m_F=0}$ state, and the experiment using these two hyperfine states is limited to a short time.
Nevertheless, since this setup has a much simpler laser alignment than the scheme in Sec.~\ref{subsec:linear_circular_polarization}, it is worth considering.
We consider that the atoms are periodically kicked by two counter-propagating pulse lasers with linear polarization vectors enclosing an angle $\Theta$~\cite{Mandel_2003_Coherent-Transport-of-Neutral-Atoms-in-Spin-Dependent-Optical-Lattice-Potentials,Mandel_2003_Controlled-collisions-for-multi-particle-entanglement-of-optically-trapped-atoms} [see Fig.~\subref*{fig:setup_lin_enclosing}].
The optical lattice potentials for the two internal degrees of freedom are given by~\cite{Jaksch_2005_The-cold-atom-Hubbard-toolbox}
\begin{align}
    V_{\ket{F=1,m_F=1}}(x) &= \frac{3}{4} V_+(x) + \frac{1}{4} V_-(x), \\
    V_{\ket{F=1,m_F=-1}}(x) &= \frac{1}{4} V_+(x) + \frac{3}{4} V_-(x), \\
    V_\pm(x) &= A \cos^2\left(k x \pm \frac{\Theta}{2}\right),
\end{align}
where $A$ is the strength of the optical lattice pulse and $k$ is the wave number of the laser.
After some calculations, we obtain
\begin{align}
    V_{\ket{F=1,m_F=\pm 1}}(x) &= \frac{A}{2} + \lambda \cos(2kx \pm \alpha), \\
    \lambda &= \frac{A}{4} \sqrt{3\cos^2 \Theta + 1}, \\
    \tan\alpha &= \frac{1}{2} \tan\Theta.
\end{align}
Therefore, we should set $\lambda=\lambda_{j+}$ and $\alpha=\alpha_{j+}$ to realize Eq.~(\ref{eq:Hamiltonian_CII}) with Eqs.~(\ref{eq:lambda1p})--(\ref{eq:alpha2p}).
The angle $\Theta=\tan^{-1}(2\tan\alpha_{j +})$ determines the phase difference $2\alpha_{j +}$ between the potentials for up and down spins.
The relations between experimental parameters $A_j, \Theta_j$ and model parameters $\lambda_j^0, \lambda_j$ are given by
\begin{align}
    A_1 &= 2\sqrt{(\lambda_1^0)^2 + 4(\lambda_1)^2}, \label{eq:A1} \\
    \tan\Theta_1 &= -2\frac{\lambda_1}{\lambda_1^0}, \label{eq:theta1} \\
    A_2 &= 2\sqrt{4(\lambda_2^0)^2 + (\lambda_2)^2}, \label{eq:A2} \\
    \tan\Theta_2 &= -2\frac{\lambda_2^0}{\lambda_2}, \label{eq:theta2}
\end{align}
where the additional subscript $j=1,2$ of $A_j$ and $\Theta_j$ corresponds to the potentials for $\hat{H}_j$.
Solving Eqs.~(\ref{eq:A1})--(\ref{eq:theta2}) with respect to $\lambda_j^0$ and $\lambda_j$, we obtain
\begin{align}
    \lambda_1^0 &= \pm\frac{A_1}{2} \cos\Theta_1, \\
    \lambda_1 &= \mp\frac{A_1}{4} \sin\Theta_1, \\
    \lambda_2^0 &= \pm\frac{A_2}{4} \sin\Theta_2, \\
    \lambda_2 &= \mp\frac{A_2}{2} \cos\Theta_2,
\end{align}
where we take either the double signs in the same order.

In a similar way, one can consider the other cases of $\nu_j^0=\nu_j$ and $\alpha_j^0=\alpha_j=0$ or $\pm\pi/2$, e.g., classes BDI and D in Table~\ref{tab:class_ORSDKR}.
However, these classes require the lattice potentials with same phases and different amplitudes for spin-up and down atoms, and thus another quasi-spin mixture is needed.

\section{\label{sec:conclusion}Conclusion}
We have studied Floquet topological phases realized in the ORSDKR model for a one-dimensional quasi-spin-1/2 BEC.
Since class CII requires four internal degrees of freedom, we have combined spin and sublattice degrees of freedom, which are achieved by the on-resonance condition.
Using both spin-dependent and spin-independent kicks, we can realize all the AZ classes with nontrivial topology in one dimension as well as class CII.
We have calculated the winding numbers characterizing the topological phases in the ORSDKR model, which take various values depending on the strengths and phases of the kicking lattices.
These values can be experimentally measured via the MCDs, which, in our model, can be obtained from the momentum distributions of the BEC in the long time limit.

The ORSDKR model would be experimentally implementable using optical lattice pulses.
As discussed in Sec.~\ref{sec:experiment}, this model can be realized by simultaneously applying two kinds of optical lattice pulses with linear and circular parallel polarizations to the quasi-spin-1/2 BEC.
With the precise control of the phase difference and the compensation of the offset terms in the optical lattices, all the topologically nontrivial classes in one dimension can be accessed.
For the class CII case, it can also be realized by applying the two counter-propagating linearly polarized lasers with the polarization vectors inclined at a certain angle.

It is important to consider how to observe the edge states in momentum space and topological phase transitions.
Finite-size momentum lattice systems can be created by two-photon Bragg diffraction processes~\cite{Gadway_2015_Atom-optics-approach-to-studying-transport-phenomena,Meier_2016_Atom-optics-simulator-of-lattice-transport-phenomena} and topological edge states have been observed~\cite{Meier_2016_Observation-of-the-topological-soliton-state-in-the-SuSchriefferHeeger-model,An_2017_Direct-observation-of-chiral-currents-and-magnetic-reflection-in-atomic-flux-lattices}.
Using this method, edge states for the ORSDKR model might also be detected.
We note that our model undergoes topological phase transitions by changing the parameters of the kicking lattices.
Through the MCDs or edge states, the dynamics of the topological phase transitions might be observed.

The ORSDKR model can be extended to study many-body and non-Hermitian physics in multiple dimensions.
One can introduce interatomic interactions neglected in this paper.
These interactions might induce dynamical instability and localization~\cite{Zhang_2004_Transition-to-Instability-in-a-Kicked-Bose-Einstein-Condensate,Monteiro_2009_Nonlinear-Resonances-in-delta-Kicked-Bose-Einstein-Condensates,See-Toh_2022_Many-body-dynamical-delocalization-in-a-kicked-one-dimensional-ultracold-gas,Cao_2022_Interaction-driven-breakdown-of-dynamical-localization-in-a-kicked-quantum-gas}.
It is also possible to investigate the non-Hermitian regime for the ORSDKR model as Ref.~\cite{Zhou_2019_Non-Hermitian-Floquet-topological-phases-in-the-double-kicked-rotor}.
The non-Hermitian effects can be experimentally introduced by considering spontaneous emission~\cite{Zhou_2022_Engineering-non-Hermitian-skin-effect-with-band-topology-in-ultracold-gases,Liang_2022_Dynamic-Signatures-of-Non-Hermitian-Skin-Effect-and-Topology-in-Ultracold-Atoms}.
Furthermore, the ORSDKR model in higher dimentions could be achieved by applying additional lasers from multiple directions.

\begin{acknowledgements}
This work was supported by JSPS KAKENHI (Grants No. JP18K03538, No. JP19H01824, No. JP19K14628, No. 20H01843, No. 21H01009, No. 21H05185), JST ERATO-FS (Grant No. JPMJER2105), and the Program for Fostering Researchers for the Next Generation (IAR, Nagoya University) and Building of Consortia for the Development of Human Resources in Science and Technology (MEXT).
\end{acknowledgements}

\appendix
\section{\label{appx:basis}Derivation of Eq.~(\ref{eq:ORSDKR_Floquet_op_quasiposition})}
We start with a general kicking potential excluding spin degrees of freedom
\begin{align}
    \hat{V} &= \int_{0}^{a} dx\, V(x) \ket{x} \bra{x}
\end{align}
with $V(x + a) = V(x)$.
By using the Fourier transforms,
\begin{align}
    \braket{x|l} &= \frac{1}{\sqrt{a}}e^{i\frac{2\pi}{a}l x}, \label{eq:Fourier_momentum_lattice_1} \\
    \ket{x} &= \sum_{l=-\infty}^{\infty} \ket{l} \frac{1}{\sqrt{a}}e^{-i\frac{2\pi}{a}l x}, \label{eq:Fourier_momentum_lattice_2} \\
    \ket{l} &= \int_{0}^{a} dx \ket{x} \frac{1}{\sqrt{a}}e^{i\frac{2\pi}{a}l x}, \label{eq:Fourier_momentum_lattice_3}
\end{align}
we represent the potential in the momentum lattice basis $\ket{l}$,
\begin{align}
    \hat{V} &= \sum_{l=-\infty}^{\infty} \sum_{k=-\infty}^{\infty} V_{k} \ket{l + k} \bra{l}, \label{eq:potential_momentum_lattice}
\end{align}
where $V_{k}$ is the Fourier coefficient of $V(x)$ given by
\begin{align}
    V_{k} &= \frac{1}{a} \int_{0}^{a} V(x) e^{-i\frac{2\pi}{a}k x} \,dx.
\end{align}

Due to the translational invariance of Eq.~(\ref{eq:2_ORSDKR_Floquet_op}), $\hat{l} \to \hat{l} + 2$, the momentum lattice $\{\ket{l}\}$ is decomposed into the even and odd lattice sites corresponding to the sublattices A and B.
We define Pauli matrices on the sublattice as
\begin{align}
    \hat{\tau}_0 &= \ket{\mathrm{A}}\bra{\mathrm{A}} + \ket{\mathrm{B}}\bra{\mathrm{B}}, \\
    \hat{\tau}_x &= \ket{\mathrm{A}}\bra{\mathrm{B}} + \ket{\mathrm{B}}\bra{\mathrm{A}}, \\
    \hat{\tau}_y &= -i\ket{\mathrm{A}}\bra{\mathrm{B}} + i \ket{\mathrm{B}}\bra{\mathrm{A}}, \\
    \hat{\tau}_z &= \ket{\mathrm{A}}\bra{\mathrm{A}} - \ket{\mathrm{B}}\bra{\mathrm{B}}.
\end{align}
Here, we choose the new unit cell such that new $n$th unit cell contains old $(2n)$th and $(2n+1)$th momentum lattice.
By using the relations $\ket{l=2n}=\ket{n}\otimes\ket{\mathrm{A}},\ket{l=2n+1}=\ket{n}\otimes\ket{\mathrm{B}}$, we rewrite the potential as
\begin{align}
    \hat{V} &= \sum_{n=-\infty}^{\infty} \sum_{m=-\infty}^{\infty} \ket{n + m} \bra{n} \notag \\
    &\qquad\qquad \otimes (V_{2m} \hat{\tau}_0 + V_{2m + 1} \hat{\tau}_- + V_{2m - 1} \hat{\tau}_+).
    \label{eq:potential_momentum_sublattice}
\end{align}

We further Furier transform the momentum lattice basis $\{\ket{n}\}$ to the quasiposition basis $\{\ket{\theta}\}$:
\begin{align}
    \braket{n|\theta} &= \frac{1}{\sqrt{2\pi}}e^{i\theta n}, \label{eq:Fourier_quasiposition_1} \\
    \ket{\theta} &= \sum_{n=-\infty}^{\infty} \ket{n}\frac{1}{\sqrt{2\pi}}e^{i\theta n}, \label{eq:Fourier_quasiposition_2} \\
    \ket{n} &= \int_{-\pi}^{\pi} d\theta \ket{\theta}\frac{1}{\sqrt{2\pi}}e^{-i\theta n}, \label{eq:Fourier_quasiposition_3}
\end{align}
The resulting potential is given by
\begin{align}
    \hat{V} &= \int_{-\pi}^{\pi} d\theta \ket{\theta} \bra{\theta} \otimes \hat{V}(\theta), \label{eq:kick_potential_quasiposition}
\end{align}
with
\begin{align}
    \hat{V}(\theta) &= \sum_{m=-\infty}^{\infty} (V_{2m} \hat{\tau}_0 + V_{2m + 1} \hat{\tau}_- + V_{2m - 1} \hat{\tau}_+) e^{-i m\theta}.
\end{align}
By using $V_m^*=V_{-m}$ due to the reality of $V(x)$, we can rewrite $\hat{V}(\theta)$ as
\begin{align}
    \hat{V}(\theta) &= \sum_{m=-\infty}^{\infty} (V_{2m} e^{-i m\theta} \hat{\tau}_0 + V_{2m + 1} e^{-i m\theta} \hat{\tau}_- \notag \\
    &\qquad\qquad + V_{2m + 1}^* e^{i m\theta} \hat{\tau}_+) \notag \\
    &= \sum_{m=-\infty}^{\infty} [V_{2m} e^{-im\theta} \hat{\tau}_0 + 2\Real(V_{2m+1}e^{-im\theta}) \hat{\tau}_x \notag \\
    &\qquad\qquad + 2\Imag(V_{2m+1}e^{-im\theta}) \hat{\tau}_y].
    \label{eq:potential_quasiposition}
\end{align}

For the case of ORSDKR model, the $x$ dependence of the kicking potential is given by Eqs.~(\ref{eq:kick_potential_x_spin_independent}) and (\ref{eq:kick_potential_x_spin_dependent}), whose Fourier coefficient is given by
\begin{align}
    V_{k} &= \frac{\lambda}{2} (e^{i\alpha} \delta_{k, \nu} + e^{-i\alpha} \delta_{k, -\nu}).
\end{align}
It follows that $\hat{V}(\theta)$ includes only the $\hat{\tau}_0$ term (the $\hat{\tau}_x$ and $\hat{\tau}_y$ terms) for even (odd) $\nu$, resulting in
\begin{align}
    \hat{V}(\theta) &= \lambda \cos\left(\frac{\nu}{2}\theta - \alpha\right) \hat{h}_{\tau 1}(\theta, \nu),
\end{align}
where $\hat{h}_{\tau 1}(\theta,\nu)$ is defined in Eq.~(\ref{eq:factor_sublattice}).
Re-introducing the spin dependence, we can rewrite the kick terms $e^{-\frac{i}{\hbar}\hat{H}_{1,2}}$ in Eq.~(\ref{eq:2_ORSDKR_Floquet_op}) as
\begin{align}
    e^{-\frac{i}{\hbar}\hat{H}_1} &= \int_{-\pi}^{\pi} d\theta \ket{\theta}\bra{\theta} \otimes e^{-i\hat{h}_1(\theta)}, \\
    e^{-\frac{i}{\hbar}\hat{H}_2} &= \int_{-\pi}^{\pi} d\theta \ket{\theta}\bra{\theta} \otimes e^{-i\hat{\tilde{h}}_2(\theta)},
\end{align}
where $\hat{\tilde{h}}_2(\theta)$ is defined by
\begin{align}
    \hat{\tilde{h}}_2(\theta) &= \Lambda_2^0(\theta) \hat{h}_{\tau 1}(\theta, \nu_2^0) \otimes \hat{\sigma}_0 + \Lambda_2(\theta) \hat{h}_{\tau 1}(\theta, \nu_2) \otimes \bm{n}_2\cdot\hat{\bm{\sigma}}.
\end{align}

The remaining operators in Eq.~(\ref{eq:Fourier_quasiposition}), i.e., the free time evolution operators $e^{\pm i \frac{\pi}{2}\hat{l}^2\otimes\hat{\sigma}_0}$, are also diagonal in the quasiposition basis and given by
\begin{align}
    e^{\pm i \frac{\pi}{2}\hat{l}^2\otimes\hat{\sigma}_0} &= e^{\pm i\frac{\pi}{4}} \sum_{n} \ket{n}\bra{n} \otimes e^{\mp i\frac{\pi}{4}\hat{\tau}_z\otimes\hat{\sigma}_0} \notag \\
    &= e^{\pm i\frac{\pi}{4}} \int_{-\pi}^{\pi} d\theta \ket{\theta}\bra{\theta} \otimes e^{\mp i\frac{\pi}{4}\hat{\tau}_z\otimes\hat{\sigma}_0}.
\end{align}
Acting on $e^{-i\hat{\tilde{h}}_2(\theta)}$ from the both sides, they work as a $\pi/2$ rotation in the sublattice space:
\begin{align}
    e^{-i\frac{\pi}{4}\hat{\tau}_z\otimes\hat{\sigma}_0} e^{-i\hat{\tilde{h}}_2(\theta)} e^{i\frac{\pi}{4}\hat{\tau}_z\otimes\hat{\sigma}_0} &= e^{-i\hat{h}_2(\theta)}.
\end{align}
Getting all factors together, we finally obtain Eq.~(\ref{eq:ORSDKR_Floquet_op_quasiposition}).

\section{\label{appx:symmetry}Symmetry properties of Eqs.~(\ref{eq:symmetric_time_frame_1}) and (\ref{eq:symmetric_time_frame_2})}
We discuss the symmetry properties of the Floquet operators $\hat{U}_{1,2}(\theta)$ in the symmetric time frames [Eqs.~(\ref{eq:symmetric_time_frame_1}) and (\ref{eq:symmetric_time_frame_2})].
The result is summarized in Table~\ref{tab:class_ORSDKR}.
We note that $\hat{U}_{1,2}(\theta)$ satisfy Eq.~(\ref{eq:symmetries_Floquet}) when both $e^{-i \hat{h}_1(\theta)}$ and $e^{-i \hat{h}_2(\theta)}$ satisfy Eq.~(\ref{eq:symmetries_Floquet}), which is rewritten as Eq.~(\ref{eq:symmetries_Hamiltonian}).
Since each term appearing in $\hat{h}_j(\theta)$ can be factorized as $\Lambda_j(\theta) \hat{h}_{\tau j}(\theta,\nu) \otimes \hat{h}_{\sigma j}$, where $\Lambda_j(\theta)=\lambda_j \cos(\nu_j\theta/2 - \alpha_j)/\hbar$, $\hat{h}_{\tau j}(\theta,\nu)=\hat{\tau}_0$ or $\bm{m}_j(\theta)\cdot\hat{\bm{\tau}}$, and $\hat{h}_{\sigma j}=\hat{\sigma}_0$ or $\bm{n}_j\cdot\hat{\bm{\sigma}}$ ($j=1,2$), these factors should be invariant up to sign under the unitary or antiunitary operations corresponding to the required symmetry.
We list up the operations that make $\Lambda_j(\theta)$, $\hat{h}_{\tau j}(\theta,\nu)$, and $\hat{h}_{\sigma j}$ invariant in Tables~\ref{tab:kick}, \ref{tab:sublattice}, and \ref{tab:spin}, respectively.
Table~\ref{tab:factor} is the list for the $4\times 4$ matrix $\hat{h}_{\tau j}(\theta,\nu) \otimes \hat{h}_{\sigma j}$ that is obtained by combining the operations for the $2\times 2$ matrices $\hat{h}_{\tau j}(\theta,\nu)$ (Table~\ref{tab:sublattice}) and $\hat{h}_{\sigma j}$ (Table~\ref{tab:spin}).

In the following discussions, we assume $\bm{n}_1\nparallel\bm{n}_2$ and $\bm{n}_1\not\perp\bm{n}_2$.
The former is required for the Floquet operator not to be block-diagonalized.
Otherwise, $\hat{\tau}_0 \otimes \bm{n}_1\cdot\hat{\bm{\sigma}}$ commutes with both $\hat{h}_1(\theta)$ and $\hat{h}_2(\theta)$.
On the other hand, the latter is assumed for simplicity: If $\bm{n}_1\perp\bm{n}_2$, additional symmetry operations may exist for each class.

We can find a chiral operator as follows.
A possible chiral operator is one of the unitary operators, $\hat{\tau}_z \otimes \hat{\sigma}_0$, $\hat{\tau}_0 \otimes\bm{n}_\perp\cdot\hat{\bm{\sigma}}$, and $\hat{\tau}_z \otimes\bm{n}_\perp\cdot\hat{\bm{\sigma}}$, in Table~\ref{tab:factor}.
To satisfy Eq.~(\ref{eq:chiral_Hamiltonian}), we must choose either of $\hat{\tau}_0$ or $\bm{m}_j\cdot\hat{\bm{\tau}}$ as $\hat{h}_{\tau j}(\theta,\nu)$ so that $\hat{h}_{\tau j}(\theta,\nu) \otimes \hat{h}_{\sigma j}$ acquires a minus sign under that unitary operation.
Namely, only the terms $\hat{h}_{\tau j}(\theta,\nu) \otimes \hat{h}_{\sigma j}$ with the entry ``$-$'' in Table~\ref{tab:factor} are allowed.
This determines $\nu_j^0,\nu_j$ to be odd or even.

We can also find a time-reversal operator and a particle-hole operator in the almost same way.
The difference is whether or not to consider the sign change of $\Lambda_j(\theta)$ under $\theta \to -\theta$.
A possible time-reversal operator or particle-hole operator is one of the antiunitary operators, $\hat{\tau}_0 \otimes \hat{\sigma}_y \hat{K}$, $\hat{\tau}_z \otimes \hat{\sigma}_y \hat{K}$, $\hat{\tau}_0 \otimes\bm{n}_\perp\cdot\hat{\bm{\sigma}}\hat{\sigma}_y \hat{K}$, and $\hat{\tau}_z \otimes \bm{n}_\perp\cdot\hat{\bm{\sigma}}\hat{\sigma}_y \hat{K}$, whose squares are $-1$, $-1$, $+1$, and $+1$, respectively, in Table~\ref{tab:factor}.
For a given $\hat{h}_{\tau j}(\theta,\nu) \otimes \hat{h}_{\sigma j}$, we choose $\alpha_j^0,\alpha_j$ so that the whole sign change of $\Lambda_j(\theta) \hat{h}_{\tau j}(\theta,\nu) \otimes \hat{h}_{\sigma j}$ is plus or minus under that time-reversal or particle-hole operation, respectively.
Namely, if the sign change of $\hat{h}_{\tau j}(\theta,\nu) \otimes \hat{h}_{\sigma j}$ is plus (minus) for a time-reversal operator, we choose $\alpha_j^0,\alpha_j=0$ ($\pm\pi/2$), and for a particle-hole operator, $\alpha_j^0,\alpha_j=\pm\pi/2$ ($0$) (see Tables~\ref{tab:kick}).
For the case of class CII discussed in Sec.~\ref{sec:symmetry}, the time-reversal and particle-hole operators are uniquely determined as $\hat{T}=\hat{\tau}_0 \otimes \hat{\sigma}_y \hat{K}$ and $\hat{C}=\hat{\tau}_z \otimes \hat{\sigma}_y \hat{K}$, respectively, by choosing $\alpha_1^0=0$.

\begin{table}
    \caption{\label{tab:kick}Sign change of the coefficient $\Lambda_j(\theta)=\lambda_j \cos(\nu_j\theta/2 - \alpha_j) / \hbar$. Because $\Lambda_j(\theta)$ should be an odd or even function of $\theta$, $\alpha_j$ can take only $0, \pi$ or $\pm\pi/2$.}
    \begin{ruledtabular}
        \begin{tabular}{ccc}
            Operation & $\alpha_j=0, \pi$ & $\alpha_j=\pm\pi/2$ \\
            \hline
            $\theta \to -\theta$ & $+$ & $-$
        \end{tabular}
    \end{ruledtabular}
\end{table}

\begin{table}
    \caption{\label{tab:sublattice}Sign change of sublattice factor $\hat{h}_{\tau j}(\theta,\nu)=\hat{\tau}_0$, $\bm{m}_1(\theta)\cdot\hat{\bm{\tau}}$, or $\bm{m}_2(\theta)\cdot\hat{\bm{\tau}}$ under unitary and antiunitary operations. Here, operator $\hat{S}$ acts on $\hat{h}_{\tau j}(\theta,\nu)$ as $\hat{S} \hat{h}_{\tau j}(\theta,\nu) \hat{S}^{-1}$.}
    \begin{ruledtabular}
        \begin{tabular}{cccc}
            Operation & $\hat{\tau}_0$ & $\bm{m}_1(\theta)\cdot\hat{\bm{\tau}}$ & $\bm{m}_2(\theta)\cdot\hat{\bm{\tau}}$ \\
            \hline
            Unitary: \\
            $\hat{\tau}_0$ & $+$ & $+$ & $+$ \\
            $\hat{\tau}_z$ & $+$ & $-$ & $-$ \\
            Antiunitary: \\
            $\hat{K},\theta \to -\theta$ & $+$ & $+$ & $-$ \\
            $\hat{\tau}_z\hat{K},\theta \to -\theta$ & $+$ & $-$ & $+$
        \end{tabular}
    \end{ruledtabular}
\end{table}

\begin{table}
    \caption{\label{tab:spin}Sign change of spin factor $\hat{h}_{\sigma j}=\hat{\sigma}_0,\bm{n}_j\cdot\hat{\bm{\sigma}}$ under unitary and antiunitary operations. Here, operator $\hat{S}$ acts on $\hat{h}_{\sigma j}$ as $\hat{S} \hat{h}_{\sigma j} \hat{S}^{-1}$. $\bm{n}_{j\perp}$ represents an arbitrary unit vector perpendicular to $\bm{n}_j$.}
    \begin{ruledtabular}
        \begin{tabular}{ccc}
            Operation & $\hat{\sigma}_0$ & $\bm{n}_j\cdot\hat{\bm{\sigma}}$ \\
            \hline
            Unitary: \\
            $\hat{\sigma}_0$ & $+$ & $+$ \\
            $\bm{n}_j\cdot\hat{\bm{\sigma}}$ & $+$ & $+$ \\
            $\bm{n}_{j\perp}\cdot\hat{\bm{\sigma}}$ & $+$ & $-$ \\
            Antiunitary: \\
            $\hat{\sigma}_y \hat{K}, \theta \to -\theta$ & $+$ & $-$ \\
            $\bm{n}_j\cdot\hat{\bm{\sigma}} \hat{\sigma}_y\hat{K}, \theta \to -\theta$ & $+$ & $-$ \\
            $\bm{n}_{j\perp}\cdot\hat{\bm{\sigma}}\hat{\sigma}_y \hat{K}, \theta \to -\theta$ & $+$ & $+$
        \end{tabular}
    \end{ruledtabular}
\end{table}

\begin{table*}
    \caption{\label{tab:factor}Sign change of $\hat{h}_{\tau j}(\theta,\nu) \otimes \hat{h}_{\sigma j}$ under unitary and antiunitary operations. Here, operator $\hat{S}$ acts on $\hat{h}_{\tau j}(\theta,\nu) \otimes \hat{h}_{\sigma j}$ as $\hat{S} \hat{h}_{\tau j}(\theta,\nu) \otimes \hat{h}_{\sigma j} \hat{S}^{-1}$. $\bm{n}_\perp$ represents an unit vector perpendicular to both $\bm{n}_1$ and $\bm{n}_2$. Here, we assume $\bm{n}_1\nparallel\bm{n}_2$ and, for simplicity, $\bm{n}_1\not\perp\bm{n}_2$.}
    \begin{ruledtabular}
        \begin{tabular}{cccccccc}
            Operation & $\hat{\tau}_0 \otimes \hat{\sigma}_0$ & $\bm{m}_1\cdot\hat{\bm{\tau}} \otimes \hat{\sigma}_0$ & $\bm{m}_2\cdot\hat{\bm{\tau}} \otimes \hat{\sigma}_0$ & $\hat{\tau}_0 \otimes \bm{n}_1\cdot\hat{\bm{\sigma}}$ & $\bm{m}_1\cdot\hat{\bm{\tau}} \otimes \bm{n}_1\cdot\hat{\bm{\sigma}}$ & $\hat{\tau}_0 \otimes \bm{n}_2\cdot\hat{\bm{\sigma}}$ & $\bm{m}_2\cdot\hat{\bm{\tau}} \otimes \bm{n}_2\cdot\hat{\bm{\sigma}}$ \\
            \hline
            Unitary: & & & & & & & \\
            $\hat{\tau}_z \otimes \hat{\sigma}_0$ & $+$ & $-$ & $-$ & $+$ & $-$ & $+$ & $-$ \\
            $\hat{\tau}_0 \otimes \bm{n}_\perp\cdot\hat{\bm{\sigma}}$ & $+$ & $+$ & $+$ & $-$ & $-$ & $-$ & $-$ \\
            $\hat{\tau}_z \otimes \bm{n}_\perp\cdot\hat{\bm{\sigma}}$ & $+$ & $-$ & $-$ & $-$ & $+$ & $-$ & $+$ \\
            Antiunitary: & & & & & & & \\
            $\hat{\tau}_0 \otimes \hat{\sigma}_y \hat{K}, \theta \to -\theta$ & $+$ & $+$ & $-$ & $-$ & $-$ & $-$ & $+$ \\
            $\hat{\tau}_z \otimes \hat{\sigma}_y \hat{K}, \theta \to -\theta$ & $+$ & $-$ & $+$ & $-$ & $+$ & $-$ & $-$ \\
            $\hat{\tau}_0 \otimes \bm{n}_\perp\cdot\hat{\bm{\sigma}}\hat{\sigma}_y \hat{K}, \theta \to -\theta$ & $+$ & $+$ & $-$ & $+$ & $+$ & $+$ & $-$ \\
            $\hat{\tau}_z \otimes \bm{n}_\perp\cdot\hat{\bm{\sigma}}\hat{\sigma}_y \hat{K}, \theta \to -\theta$ & $+$ & $-$ & $+$ & $+$ & $-$ & $+$ & $+$
        \end{tabular}
    \end{ruledtabular}
\end{table*}

Now, we consider the other classes BDI, DIII, D, and AIII that can be achieved in the ORSDKR.
Without loss of generality, we choose $\alpha_1^0=0$.
In the case of $\lambda_1^0=0$, we choose $\alpha_1=0$.

\begin{description}
    \item[Class BDI]
    A Hamiltonian in class BDI has the symmetries satisfying $\hat{T}^2=+1$, $\hat{C}^2=+1$, and $\hat{\Gamma}^2=1$.
    The antiunitary operators that square to $+1$ in Table~\ref{tab:factor} are $\hat{\tau}_0 \otimes \bm{n}_\perp\cdot\hat{\bm{\sigma}} \hat{\sigma}_y \hat{K}$ and $\hat{\tau}_z \otimes \bm{n}_\perp\cdot\hat{\bm{\sigma}} \hat{\sigma}_y \hat{K}$.
    Therefore, the chiral operator is $\hat{\Gamma}=\hat{\tau}_z \otimes \hat{\sigma}_0$, which is possible when $\nu_j^0$ and $\nu_j$ are all odd from the first row of Table~\ref{tab:factor}.
    From the choice of $\alpha_1^0=0$, the time-reversal and particle-hole operators are determined as $\hat{T}=\hat{\tau}_0 \otimes \bm{n}_\perp\cdot\hat{\bm{\sigma}} \hat{\sigma}_y \hat{K}$ and $\hat{C}=\hat{\tau}_0 \otimes \bm{n}_\perp\cdot\hat{\bm{\sigma}} \hat{\sigma}_y \hat{K}$, respectively.
    Finally, according to the signs in Table~\ref{tab:factor}, the remaining phases are determined as $\alpha_1=0,\alpha_2^0=\alpha_2=\pi/2$ modulo $\pi$.

    \item[Class DIII]
    A Hamiltonian in class DIII has the symmetries satisfying $\hat{T}^2=-1$, $\hat{C}^2=+1$, and $\hat{\Gamma}^2=1$.
    In the case of $\lambda_1^0\neq 0$, choosing $\alpha_1^0=0$ determines $\nu_1^0$ to be odd, $\hat{T}=\hat{\tau}_0 \otimes \hat{\sigma}_y \hat{K}$, and $\hat{C}=\hat{\tau}_z \otimes \bm{n}_\perp\cdot\hat{\bm{\sigma}} \hat{\sigma}_y \hat{K}$.
    It follows that the chiral operator is given by $\hat{\Gamma}=\hat{\tau}_z \otimes \bm{n}_\perp\cdot\hat{\bm{\sigma}}$, with which the possible terms in $\hat{h}_j(\theta)$ are given by odd $\nu_2^0$ and even $\nu_{1,2}$ with $\alpha_2^0=\alpha_{1,2}=\pi/2$ modulo $\pi$.
    On the other hand, in the case of $\lambda_1^0=0$, $\lambda_1$ should be nonzero so that the Floquet operator is not block-diagonalized.
    In this case, choosing $\alpha_1=0$ determines $\nu_1$ to be odd, $\hat{T}=\hat{\tau}_z \otimes \hat{\sigma}_y \hat{K}$, and $\hat{C}=\hat{\tau}_z \otimes \bm{n}_\perp\cdot\hat{\bm{\sigma}} \hat{\sigma}_y \hat{K}$.
    The chiral operator $\hat{\Gamma}=\hat{\tau}_0 \otimes \bm{n}_\perp\cdot\hat{\bm{\sigma}}$ requires $\lambda_2^0=0$.
    From the symmetry properties of $\hat{T}$ and $\hat{C}$, we obtain $\alpha_2=\pi/2$ modulo $\pi$, with which $\nu_2$ can take both even and odd.
    We note, however, that for the case of $\lambda_1^0=\lambda_2^0=0$, $\nu_1$ and $\nu_2$ should have different parity so that the Floquet operator is not block-diagonalized.
    Thus, $\nu_2$ is determined to be even.

    \item[Class D]
    A Hamiltonian in class D has the particle-hole symmetry satisfying $\hat{C}^2=+1$.
    Choosing $\alpha_1^0=0$ determines $\nu_1^0$ to be odd and $\hat{C}=\hat{\tau}_z \otimes \bm{n}_\perp\cdot\hat{\bm{\sigma}} \hat{\sigma}_y \hat{K}$.
    In this case, we must choose $\nu_1$ to be odd (even) with $\alpha_1=0$ ($\pi/2$) modulo $\pi$.
    $\nu_2^0$ and $\nu_2$ can take both even and odd as long as $\alpha_2^0=\alpha_2=\pi/2$ modulo $\pi$.
    We should choose $\nu_2^0$ and $\nu_2$ not to belong to the other classes.
    We can also assume $\lambda_1^0=0, \lambda_1\neq 0$ and $\alpha_1=0$.
    However, this does not change the particle-hole operator, and the obtained result is included in the case of odd $\nu_1$ in the above argument.

    \item[Class AIII]
    A Hamiltonian in class AIII has the chiral symmetry satisfying $\hat{\Gamma}^2=1$.
    We can choose $\hat{\tau}_z \otimes \hat{\sigma}_0$, $\hat{\tau}_0 \otimes \bm{n}_\perp\cdot\hat{\bm{\sigma}}$, and $\hat{\tau}_z \otimes \bm{n}_\perp\cdot\hat{\bm{\sigma}}$ as chiral operators.
    For $\hat{\Gamma}=\hat{\tau}_z \otimes \hat{\sigma}_0$, we must choose $\nu_{1,2}^0,\nu_{1,2}$ to be odd.
    For $\hat{\Gamma}=\hat{\tau}_0 \otimes \bm{n}_\perp\cdot\hat{\bm{\sigma}}$, we must choose $\lambda_{1,2}^0=0$.
    For $\hat{\Gamma}=\hat{\tau}_z \otimes \bm{n}_\perp\cdot\hat{\bm{\sigma}}$, we must choose $\nu_{1,2}^0$ to be odd and $\nu_{1,2}$ to be even.
    In all cases, we should choose $\alpha_{1,2}^0,\alpha_{1,2}$ not to belong to the other classes.
\end{description}

\bibliography{reference}

\end{document}