In this section we discuss the computational hypotheses necessary for the algorithms in the next section. 
This section is based on the work \cite{east2019computing} in which East, Egry-Nagy, Mitchell and Péresse provide efficient algorithms for all basic computational questions on finite semigroups (which include monoids). Although we limit our scope to the case of transformation monoids, methods described \cite{east2019computing} allow the algorithms described below to be applied to other interesting classes of monoids. Moreover, they can theoretically be applied to any finite monoid using a Cayley embedding in a full transformation monoid. In general however, this is very inefficient and not feasible in practice.

Following the authors of \cite{east2019computing}, we make the following fundamental assumptions that we can compute:  
\begin{itemize}
	\item Assumption I : a product of two elements of the monoid.
	\item Assumption II : the image and kernel of a transformation (note that we do not explicitly use this assumption, but that it is necessary for the algorithms of \cite{east2019computing} that we do use).
	\item Assumption III : Green pairs.
	\item Assumption IV : Given $h \in {}_M\stab(H)$ compute the corresponding element in $\Gamma(H)$ (understood as a permutation group of the image common to all elements of $H$ as seen in Example \ref{ex:schu_as_symm}), and similarly on the right.
\end{itemize}

Not only do we directly need to be able to do these computations for our own algorithm, but they are also prerequisite for the algorithms from \cite{east2019computing}. As such, we refer to the top of Section 5.2 of \cite{east2019computing} on how to realize these computations in the case of transformation monoids.

We, again, refer to \cite{east2019computing} for the specific algorithms meeting our computational prerequisites.
\begin{itemize}
	\item Computing the \schu groups: \cite[Algorithm 4]{east2019computing}
	\item Checking membership of an element in a Green class: \cite[Algorithms 7 \& 8]{east2019computing}.
	\item Finding idempotents: \cite[Algorithm 10]{east2019computing}. This algorithm also allows for finding the regular $\jc$-classes.
	\item Decomposing the monoid in $\rc, \lc$ and $\jc$-classes : \cite[Algorithm 11]{east2019computing} and its discussion. Note that by storing this decomposition, we can, given an element of the monoid, find the classes that contain it.
	\item Obtaining a representative of a Green class: this is given by the data structure representing the Green classes described at the top of \cite[Section 5.4]{east2019computing}.
\end{itemize}

Finally, we require the following points that, although they are not described in \cite{east2019computing}, are easily obtained from it.

\begin{itemize}
	\item Computing a set $C_M$ of character equivalency representatives: given Proposition \ref{prop:char_equiv}, this can be done in four steps:
	\begin{enumerate}
		\item compute a set $\mathcal{E}$ of idempotent representatives of the regular $\jc$-classes,
		\item compute $\Gamma(\hc(e))$ for each $e \in \mathcal{E}$,
		\item compute a set $C_e$ of representatives of the conjugacy classes of $\Gamma(\hc(e))$ for each $e \in \mathcal{E}$, using for instance the procedure described in \cite{hulpke2000conjugacy},
		\item for each $e \in \mathcal{E}$ and $c \in C_e$ compute the corresponding element of $\hc(e)$ as in Example \ref{ex:iso_idem}. 
	\end{enumerate}
	\item Computing $\tau_a$ as in Proposition \ref{prop:bij_cano_conj} :  given $g \in \Gamma(H)$, $\tau_a(g)$ is simply, seen as an element of $\symm(\ker a)$ :
	\[a\inv\{i\} \mapsto (g\cdot a)\inv\{g\cdot a(i)\},\]
	which can be computed in $O(n)$. Note that this is a special case of the application described in \cite[Proposition 3.11 (a)]{east2019computing}
	\item Testing that two elements $g, g'$ in $\Gamma'(\hc(a))$ are conjugated : $\Gamma'(H)$ is represented as a subgroup of $\symm(\ker a)$ and known procedures, such as the one described in \cite{butler1994inductive}, can be used.
	\item Computing the cardinality of a conjugacy class of a \schu group: for instance, the computer algebra system GAP uses the method described in \cite{hulpke2000conjugacy}.
\end{itemize}