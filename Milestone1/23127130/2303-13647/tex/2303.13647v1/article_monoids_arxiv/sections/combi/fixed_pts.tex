	Consider the problem of counting the number of elements of the set $\fix_{G}(h, k)$ %$ = \{g \in G \sepp hgk = g\}$ 
	where $G$ is a finite group and $h, k \in G$. 
	If $\fix_{G}(h, k)$ is non-empty, it contains an element $\gamma$ such that $h\gamma k = \gamma$, or equivalently $h = \gamma k\inv \gamma\inv$. So for any $g \in \fix_G(h, k)$ we have:
	\[hgk = g \Leftrightarrow \gamma k\inv \gamma\inv g k = g \Leftrightarrow \gamma\inv g k = k\gamma\inv g.\]
	This means that $g \in \gamma C_G(k)$ where $C_G(k)$ is the centralizer of $k$ in $G$. Because the other inclusion is obvious, we get a description of $|\fix_{G}(h, k)|$: either $h$ and $k\inv$ are conjugated in which case there are $|C_G(k)|$ fixed points, or they are not, and there are no fixed points.
	In the case of a monoid, this reasoning mostly breaks: we crucially used the invertibility property, which monoids lack. The \schu groups seem to be ideal candidates to get back some of this invertibility. 
	In this section we clarify the role of the \schu groups for counting fixed points, how to give meaning to "$h\mul{H}$ and $\mul{H}k$ are in the same conjugacy class", and how to factorize our previous remark over all the $\hc$-classes of the same $\jc$-class.
	
	As the bijections between $\lc$ (and $\rc$) classes will play an major role in the remainder of this section, we introduce the following notations.
	\begin{notation}
		Given $R, R'$ two $\rc$-classes in the same $\jc$-class, we say that $(\lambda, \lambda')$ is a \emph{left Green pair} with respect to $(R, R')$ if:
		\begin{itemize}
			\item $\lambda R = R'$ and $\lambda'R' = R$.
			\item $(\lambda\lambda')\mul{R} = \Id_R$ and $(\lambda'\lambda)\mul{R'} = \Id_{R'}$
		\end{itemize}
		Similarly, given two $\lc$-classes $L, L'$ in the same $\jc$-classes, $(\rho, \rho')$ is a \emph{right Green pair} with respect to $(L, L')$ if:
		\begin{itemize}
			\item $L\rho = L'$ and $L'\rho' = L$.
			\item $\mul{L}(\rho\rho') = \Id_L$ and $\mul{L'}(\rho'\rho) = \Id_{L'}$
		\end{itemize}
	\end{notation}
	
	Using Green pairs, one can transport the problem of counting fixed points in an arbitrary $\hc$-class to a reference $\hc$-class.
	
	\begin{prop}\label{prop:transition}
		Let $H_1, H_2 \subset J$ be two $\hc$-classes contained in the same $\jc$-class. Let $\lambda, \lambda', \rho, \rho'$ such that:
		\begin{itemize}
			\item $(\lambda, \lambda')$ is a left Green pair with respect to $(\rc(H_1), \rc(H_2))$,
			\item $(\rho, \rho')$ is a right Green pair with respect to $(\lc(H_1), \lc(H_2))$.
		\end{itemize}
		Finally, let $(h, k) \in {}_M\stab(\rc(H_2)) \times \stab_M(\lc(H_2))$ and define $(h', k') = (\lambda' h \lambda, \rho k \rho')$. Then the maps $x \mapsto \lambda'x\rho'$ and $x \mapsto \lambda x\rho$ are reciprocal bijections between the sets $\fix_{H_2}(h, k) = \{a \in H_2 \sepp hak = a\}$ and $\fix_{H_1}(h', k')=\{a \in H_1 \sepp h'ak' = a\}$.
	\end{prop}
	
	\begin{proof}
		First notice that Green's Lemma give us the existence of $\lambda, \lambda', \rho, \rho'$ respecting the hypothesis we demand, and also gives that $x \mapsto \lambda'x\rho'$ and $x \mapsto \lambda x\rho$ are reciprocal bijections between $H_1$ and $H_2$.
		Let $a_1$ be an element of $H_1$ and denote by $a_2 = \lambda a_1 \rho$. Then:
		\[ha_2k = a_2 \Leftrightarrow \lambda'ha_2k\rho' = \lambda'a_2\rho' \Leftrightarrow (\lambda' h \lambda)a_1(\rho k \rho') = a_1 \Leftrightarrow h'a_1k' = a_1\]
		so these bijections restrict to $\fix_{H_1}(h', k')$ and $\fix_{H_2}(h, k)$.
	\end{proof}
	
	%\todo{Une image avec le nombre de pts fixes dans chaque H-class}
	
	Keeping in mind our computational goals, transporting the problem of counting fixed points from $H_2$ to $H_1$ is helpful, as for the price of 4 monoid multiplications, we can use a lot of precomputations specific to a particular $\hc$-class, avoiding the repetition of multiple similar computations for each $\hc$-class. 
	
	The question is now to determine the fixed points in a single $\hc$-class, using our previous remark on conjugacy. Let us first clarify the idea of elements of the left and right \schu groups being in the same conjugacy class.
	
	\begin{prop}\label{prop:bij_cano_conj}
		Given and $\hc$-class $H$, $a \in H$ and $g \in \Gamma(H)$, we define $\tau_a(g)$ as the unique element of $\Gamma'(H)$ such that $g \cdot a = a \cdot \tau_a(g)$. Then $\tau_a : \Gamma(H) \longrightarrow \Gamma'(H)$ is an anti-isomorphism\footnote{Note that some authors equip the right \schu group with reversed composition, and thus obtain an isomorphism instead of anti-isomorphism.}. Moreover $\tau_a$ gives rise to a bijection between the conjugacy classes of $\Gamma(H)$ and $\Gamma'(H)$ that is independent of the choice of $a$.
	\end{prop}
	
	\begin{proof}
		The first part is known since \cite{schu}. We want to check that for $a \in H, g \in \Gamma(H)$, the conjugacy class of $\tau_a(g)$ is defined independently of $a$. Take any $a, b \in H$. By definition of $\Gamma(H)$, there exist some $h \in \Gamma(H)$ such that $b = h \cdot a$. So :
		\[b \cdot \tau_a(g) = (h \cdot a) \cdot \tau_a(g) = h \cdot (g \cdot a) = hgh\inv\cdot(h \cdot a) = b \cdot \tau_b(hgh\inv). \]
		Since $\Gamma'(H)$ acts freely, this means that $\tau_a(g) = \tau_b(h)\tau_b(g)\tau_b(h)\inv$ and thus $\tau_a(g)$ is conjugated with $\tau_b(g)$, which proves that the conjugacy class of $\tau_a(g)$ is indeed defined independently of $a$. Finally, as $\tau_a$ is a group morphism, the image are of two conjugated elements are conjugated, meaning that $\tau_a$ does indeed induces bijection between the conjugacy classes of the left and right \schu groups, independently of the choice of $a$.
	\end{proof}

	In the next proposition, we formalize the idea of searching the fixed points as some centralizer, but in the context of a monoid.

	\begin{prop}\label{prop:centralisateur}
		Let $H$ be a $\hc$-class, $a\in H$ and $(h, k) \in {}_M\stab(\rc(H)) \times \stab_M(\lc(H))$. Then
		\[|\fix_H(h, k)| = \left\{\begin{aligned}
		&|C_{\Gamma'(H)}(\mul{H}k)| & & \textrm{ if } \tau_a(h\mul{H})\inv \in \overline{\mul{H}k}\\
		&0 & & \textrm{ otherwise}
		\end{aligned}\right.\]
		where $\overline{\mul{H}k}$ is the conjugacy class of $\mul{H}k$ in $\Gamma'(H)$ and $C_{\Gamma'(H)}(\mul{H}k)$ is the centralizer in $\Gamma'(H)$ of $\mul{H}k$.
	\end{prop}
	
	\begin{proof}
		For simplicity, we commit an abuse of notation by denoting $h\mul{H}$ as $h$ and $\mul{H}k$ as $k$. Let $a$ be any element of $H$. 
		\begin{align*}
		\fix_H(h,k) & = \{b \in H \sepp hbk=b\}\\
		& = \{a \cdot g\sepp g \in \Gamma'(H) \textrm{ and } ha\cdot gk = a\cdot g \}\\
		& = \{a\cdot g\sepp g \in \Gamma'(H) \textrm{ and } a \cdot \tau_a(h)gk = a\cdot g \}\\
		& = \{a\cdot g\sepp g \in \Gamma'(H) \textrm{ and } \tau_a(h)gk = g\}.
		\end{align*}
		The last equality comes from the fact that $\Gamma'(H)$ acts freely so we can simplify the $a$. Suppose that $\fix_H(h, k)$ is non-empty and let $\gamma \in \Gamma'(H)$ such that $\tau_a(h)\gamma k = \gamma$. Then, for any $g \in \Gamma'(H)$ :
		\[\tau_a(h)gk = g \Leftrightarrow g \inv \tau_a(h)gk = e \Leftrightarrow g\inv\gamma k\inv\gamma\inv g k\ = e \Leftrightarrow [\gamma\inv g, k] = e\]
		where $[\cdot,\cdot]$ is the commutation bracket. This means that 
		\[\fix_H(h,k) = \{a \cdot g \sepp g \in \gamma C_{\Gamma'(H)}(k)\}.\]
		Note that because, again, $\Gamma'(H)$ acts freely, $\fix_H(h,k)$ has the same cardinality as $C_{\Gamma'(H)}(k)$ and that, from Proposition \ref{prop:bij_cano_conj} this is independent from the choice of $a$ which proves the result. 
	\end{proof}
	
	\begin{lined}
		\begin{ex}
			Consider $a = [1\ 2\ 2\ 3] \in T_4$ and $H = \hc(a)$. We have $\im a = \{1, 2, 3\}$ and $\ker a = \{\{1\},\{2,3\},\{4\}\}$. Notice that $H$ is not a group since $a^2 = [1\ 2\ 2\ 2] \notin H$. Considering the \schu groups as symmetric groups on the image and kernel common to all elements of $H$ as in Example \ref{ex:schu_as_symm}, we have $\Gamma(H) = \symm(\im a)$ and $\Gamma'(H) = \symm(\ker a)$.
			
			Let us first check for fixed points under the action of $h = [1\ 2\ 3\ 4]$ on the left and $k = [2\ 1\ 1\ 4]$ on the right. Seen as an element of $\Gamma(H)$, $h$ corresponds to $\Id_{\im a}$ and $k$ corresponds to $(\{2, 3\}\ \{1\})$ in $\Gamma'(H)$. Since we have $\tau_a(h) = \Id_{\ker a}$, it follows that $|\fix_H(h,k)| = 0$.
			
			If we now take $h$ to be $[1\ 3\ 2\ 4]$, the corresponding element in $\Gamma(H)$ is $(2\ 3)$ and $\tau_a(h) = (\{2, 3\}\ \{4\})$. Since $(\{2, 3\}\ \{4\})$ and $(\{2, 3\}\ \{1\})$ are conjugated in $\symm(\ker a)$, the set of fixed points is non-empty. Their centralizers have cardinal 2 and one can indeed check that $[2\ 3\ 3\ 1]$ and $[3\ 2\ 2\ 1]$ are the only fixed points in $H$.
		\end{ex}
	\end{lined}

	Putting together the previous results, we get the following Corollary on the cardinality of $\fix_J(h, k)$.
	
	\begin{cor}\label{cor:pt_fix_counting}
		Let $J$ be a $\jc$-class, $a$ any element of $J$ and denote by $H_0 = \hc(a), R_0 = \rc(a), L_0 = \lc(a)$. Let $h, k$ be any elements of $M$. We denote by:
		\begin{itemize}
			\item $(\lambda_R, \lambda'_R)$ a left Green pair with respect to $(R_0, R)$ for each $\rc$-class $R \subset J$,
			\item $(\rho_L, \rho'_L)$ a right Green pair with respect to $(L_0, L)$ for each $\lc$-class $L \subset J$,
			\item $S_{\rc}(h) = \{R \subset J \sepp R \textrm{ is a } \rc\textrm{-class and } hR = R\}$,
			\item $S_{\lc}(k) = \{L \subset J \sepp R \textrm{ is a } \lc\textrm{-class and } Lk = L\}$
		\end{itemize}
		Denoting the set of conjugacy classes of $\Gamma'(H_0)$ as $C$, we further define two vectors:
		\begin{itemize}
			\item $r_J(h) = (|C_{\Gamma'(H)}(g)|\cdot|\{R \in S_{\rc}(h)\sepp \tau_a(\lambda_R' h \lambda_R) \in \bar{g}\}|)_{\bar{g}\in C}$,
			\item $l_J(k) = (|\{L \in S_{\lc}(k)\sepp \rho_L' k \rho_L \in \bar{g}\}|)_{\bar{g}\in C}$. 
		\end{itemize}
		Then $\fix_J(h, k)$ has cardinality the dot product of $r_J(h)$ with $l_J(k)$.
	\end{cor}
	