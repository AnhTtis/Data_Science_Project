One of the major features of the finite group representation theory is the fact that all the information on a representation can be summarized in its \emph{character}. This (partially) carries over to monoid representation theory, as we shall see in this section where we reformulate the results of the previous section in terms of characters.

\begin{dftn}
	If $V$ is a finite dimension $\kbf M$-module, its \emph{character} is the map from $M$ to $\kbf$ defined by $\chi_{\kbf M}^V : m \longmapsto \tr(v \mapsto m \cdot v)$.
\end{dftn}

We recall the following well know facts about characters.  Proofs for fact 2 and 3 are respectively (ii) and (iii) of \cite[Proposition 7.12]{steinberg2016representation}\footnote{Note that for Fact 3, our reference deals only with the case $M = M'$, but the proof is the same.}.

\begin{prop}\label{prop:facts_on_char}
	\begin{enumerate}
		\item Let $V$ be a $\kbf M$-module. We have $\chi_{\kbf M}^V = \chi_{\kbf M^{op}}^{V^*}$.
		\item Consider the short exact sequence of $\kbf M$-modules :
		\[0 \longrightarrow A \longrightarrow B \longrightarrow B/A \longrightarrow 0.\]
		Then $\chi_{\kbf M}^{B/A} = \chi_{\kbf M}^B - \chi_{\kbf M}^A$.
		\item Consider $M, M'$ two finite monoids, $V$ a $\kbf M$-module and $W$ a $\kbf M'^{op}$-module. Then $\chi_{\kbf M \otimes \kbf M'}^{V \otimes W} = \chi_{\kbf M}^{V} \chi_{\kbf M'}^{W}$.
	\end{enumerate}
\end{prop}

The previous properties are simply extensions of similar properties on groups, and their proof is similar. From groups, we also keep in the case of monoids the linear independence of irreducible characters (see \cite[Theorem 7.7]{steinberg2016representation} for reference):

\begin{prop}\label{prop:char_libres}
	The irreducible characters $\{\chi_{\kbf M}^S \sepp S \textrm{ is a simple } \kbf M-\module\}$ are linearly independent as $\kbf$ valued functions.
\end{prop}

This, together with the second point in the Proposition \ref{prop:facts_on_char}, has a nice consequence. As we are interested in finite dimensional module over finite monoids, those modules have a composition series. Say that a $\kbf M$-module $V$, has $S$ as a composition factor with multiplicity $[V:S]$ for any simple $\kbf M$-module $S$. Then:
\[\chi_{\kbf M}^V = \sum_S [V:S]\chi_{\kbf M}^S.\]
In that way, since characters of the simple modules are linearly independent, the character of a module can be seen as a record of its composition factors.

The question of where to compute characters is worth asking: in the case of groups, one needs only to compute the character for a transversal of conjugacy classes to get its value everywhere. The case of monoids was described for the first time by McAlister in \cite{mcalister1972characters}. 

\begin{dftn}\label{def:conj}
	We say that two elements $m, m'$ in $M$ are in the same \emph{generalized conjugacy class} or \emph{character equivalency class} if for every $\kbf M$-module $V$, $\chi_M^V(m) = \chi_M^V(m')$. We note $C_M$ the set of generalized conjugacy classes.
\end{dftn}

\begin{prop}(\cite[Proposition 2.5]{mcalister1972characters})\label{prop:char_equiv}
	Let $\mathcal{E} = \{e_1, \dots, e_n\}$ be idempotent representatives of the regular $\jc$-classes of $M$ and for each $e_i$ let $\mathcal{C}_i = \{c_{i,1}, \dots, c_{i,m_i}\}$ be representatives of the conjugacy classes of $G_{e_i}$. Then the set $\mathcal{C}_M = \bigcup_{e_i \in \mathcal{E}} \mathcal{C}_i$ is a set of representatives of character equivalency classes of $M$.
\end{prop}

We can now recall the definition of the character table of a monoid.

\begin{dftn}
	Let $\irr_M$ be the set of isomorphism classes of simple $\kbf M$-modules and $C_M$ as in definition \ref{def:conj}.
	The \emph{character table} of $M$ over $\kbf$ is the (square) matrix defined by :
	\[X(M) = (\chi_{\kbf M}^V(m))_{V \in \irr_M, m \in C_M}.\]
	Moreover, if $e \in M$ is an idempotent, we define $X_e(M)$ as the matrix obtained by extracting from $X(M)$ only the lines corresponding to simple modules with apex $e$.
\end{dftn}

Finally, we can apply the language of characters to Proposition \ref{prop:quotient_semisimple}, which yield a formula for computing the character table of $M$ over $\kbf$ given the character tables of the groups $G_e$ over $k$.

\begin{prop}\label{prop:formula_char}
	Let $e \in M$ be an idempotent, $G_e$ be the maximal subgroup at $e$. 
	We have the formula for $X_e(M)$ : 
	\[X_e(M) = {}^tX(G_e)\inv \cdot \left(\chi_{\kbf M \otimes \kbf G_e^{op}}^{\kbf\lc(e)}(m, g) - \chi_{\kbf M \otimes \kbf G_e^{op}}^{N_e(\kbf\lc(e))}(m, g)\right)_{g \in C_{G_e}, m \in C_M}\]
	where the dot is the matrix product.
\end{prop}

\begin{proof}
	First, we have, because of Proposition \ref{prop:facts_on_char}-2, we have:
	\[\chi_{\kbf M \otimes \kbf G_e^{op}}^{\kbf\lc(e)/N_e(\kbf\lc(e))} = 
	\chi_{\kbf M \otimes \kbf G_e^{op}}^{\kbf\lc(e)}(m, g) - \chi_{\kbf M \otimes \kbf G_e^{op}}^{N_e(\kbf\lc(e))}(m, g)\]
	Then, from Proposition \ref{prop:quotient_semisimple}, we know that:
	\[\begin{aligned}
		\chi_{\kbf M \otimes \kbf G_e^{op}}^{\kbf\lc(e)/N_e(\kbf\lc(e))}(m, g) 
		& = \chi_{\kbf M \otimes \kbf G_e^{op}}^{\bigoplus_{V \in \irr_e} V^{\#} \otimes V^*}\\
		& = \sum_{V\in\irr_e} \chi_{\kbf M \otimes \kbf G_e^{op}}^{V^{\#} \otimes V^*}\\
		& = \sum_{V\in\irr_e} \chi_{\kbf M}^{V^{\#}}(m)\chi_{\kbf G_e}^V(g)\\
	\end{aligned}
	\]
	This last sum is clearly the dot product between the column of $X(G_e)$ indexed by $g$ and the column of $X_e(M)$ indexed by $m$. That is, the coefficient in position $(g, m)$ of $\chi_{M-G_e}^{\lc(e)/\rad(\lc(e))}$ is equal to the coefficient in position $(g, m)$ of ${}^tX(G_e)\cdot X_e(M)$, which, together with Proposition \ref{prop:facts_on_char}(ii), proves the equality.
\end{proof}