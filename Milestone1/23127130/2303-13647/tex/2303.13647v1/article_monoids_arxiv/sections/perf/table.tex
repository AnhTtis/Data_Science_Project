%\todo{would it be possible to visualize the contribution of the various optimisations: Sage naive (btw: name it after the algorithm, not the platform); plain formula; exploit the Green-structure; exploit the combinatorics of conjugates}

As shown in Table \ref{tab:char_tab}, the computation of the character table takes much longer. This is due to the fact that, to compute the radical of $\kbf\lc(e)$ for an idempotent $e$, we must solve a linear system of size $|\rc(e)|\times|\lc(e)|$ which necessitates $O(|\rc(e)|^2|\lc(e)|)$ arithmetic operations. In the case of the full transformation semigroup $T_n$, if $e$ as $k$ elements in its image, $|\lc(e)| = k! \times \binom{n}{k}$, while $|\rc(e)| = k! \times S(n,k)$ where $S(n, k)$ is a Stirling number of the second kind, which gives $|\rc(e)| \sim k^n$. The size of that linear system becomes rapidly untractable. Moreover, once we have a basis of $N_e(\kbf L)$ of cardinality $d$, we still have to compute the $C_M^2$ character values in $O(d^2)$ operations each. Experiments show that the computation time of the character tables of the maximal subgroups is small in comparison of all radical related computations. As can be seen on Figure \ref{fig:plot_char_table}, time and memory usage are in lockstep (at least for big enough monoids) and the limiting factor is memory (the test on random monoids fail for the random monoids of the form $R(9,8)$ by exceeding the 16GB memory capacity of our testing machine).

\begin{table}[!h]
	\centering
	
	\makebox[\textwidth][c]{
		\begin{tabular}{c|c|c|c}
			Monoid & Cardinality & Coefficients & Ours\\% & Monoid & Cardinality & Coefficients & Ours \\
			\hline
			$T_3$ & 27 & $6^2$ & 27 ms \\
			$T_4$ & 256 & $11^2$ & 151 ms\\
			$T_5$ & 3125 & $18^2$ & 1.74 s\\
			$T_6$ & 46656 & $29^2$ & 29.8 s\\
			$T_7$ & 823543 & $44^2$ & 11.0 min\\  
			%$R(4,3)$ & 80 & $4^2$ & 29 ms & 
			%$R(6,5)$ & 5453 & $16^2$ & 4.78 s\\
			%$R(5,3)$ & 150 & $11^2$ & 78 ms & 
			%$R(7,6)$ & 129337 & $19^2$ & 28.8 min\\
			%$R(5,4)$ & 754 & $14^2$ & 378 ms &
			% 823543 & & 2.369 &\\
		\end{tabular}
	}
	
	\caption{Computation time of the character table.}
	\label{tab:char_tab}
\end{table}

\vspace{-10mm}
\begin{figure}[h!]
	\centering
	
	\makebox[\textwidth][c]{
		\includegraphics[width = 0.6\textwidth]{figures/char_time.eps}
		\hfill
		\includegraphics[width = 0.6\textwidth]{figures/char_mem.eps}
	}
	
	\caption{Computation time and memory usage for computing the character table using Propositions \ref{prop:formula_char} and \ref{prop:equations}.
		\\
		The blue points correspond to the random monoids, the yellow ones to $T_n$ for $n \in \intint{3, 7}$. As before, the yellow points are excluded of the linear regression although in this case, $T_n$ behave more or less like the randomly chosen monoids. The measured complexity on random monoids is slighly more than linear in time and memory.}
	\label{fig:plot_char_table}
\end{figure}

%	We present in Figure \ref{fig:char_tab} the result of such a computation: the character table of $T_6$.
%	
%	%% Figure : bichar T_6
%	\begin{figure}[h!]
	\begin{outdent}
	\setlength{\arraycolsep}{4pt}
	\medmuskip = 1mu % default: 4mu plus 2mu minus 4mu
	\[\left(\begin{smallmatrix}%{ccccccccccccccccccccccccccccc}
	1 & \bar1 & 1 & 1 & \bar1 & \bar1 & 1 & \bar1 & 1 & 1 & \bar1 & \cdot & \cdot & \cdot & \cdot & \cdot & \cdot & \cdot & \cdot & \cdot & \cdot & \cdot & \cdot & \cdot & \cdot & \cdot & \cdot & \cdot & \cdot \\
	5 & \bar3 & 2 & 1 & \bar1 & \cdot & \cdot & 1 & \bar1 & \bar1 & 1 & \cdot & \cdot & \cdot & \cdot & \cdot & \cdot & \cdot & \cdot & \cdot & \cdot & \cdot & \cdot & \cdot & \cdot & \cdot & \cdot & \cdot & \cdot \\
	9 & \bar3 & \cdot & 1 & 1 & \cdot & \bar1 & \bar3 & 1 & \cdot & \cdot & \cdot & \cdot & \cdot & \cdot & \cdot & \cdot & \cdot & \cdot & \cdot & \cdot & \cdot & \cdot & \cdot & \cdot & \cdot & \cdot & \cdot & \cdot \\
	5 & \bar1 & \bar1 & 1 & 1 & \bar1 & \cdot & 3 & \bar1 & 2 & \cdot & \cdot & \cdot & \cdot & \cdot & \cdot & \cdot & \cdot & \cdot & \cdot & \cdot & \cdot & \cdot & \cdot & \cdot & \cdot & \cdot & \cdot & \cdot \\
	10 & \bar2 & 1 & \bar2 & \cdot & 1 & \cdot & 2 & \cdot & 1 & \bar1 & \cdot & \cdot & \cdot & \cdot & \cdot & \cdot & \cdot & \cdot & \cdot & \cdot & \cdot & \cdot & \cdot & \cdot & \cdot & \cdot & \cdot & \cdot \\
	16 & \cdot & \bar2 & \cdot & \cdot & \cdot & 1 & \cdot & \cdot & \bar2 & \cdot & \cdot & \cdot & \cdot & \cdot & \cdot & \cdot & \cdot & \cdot & \cdot & \cdot & \cdot & \cdot & \cdot & \cdot & \cdot & \cdot & \cdot & \cdot \\
	5 & 1 & \bar1 & 1 & \bar1 & 1 & \cdot & \bar3 & \bar1 & 2 & \cdot & \cdot & \cdot & \cdot & \cdot & \cdot & \cdot & \cdot & \cdot & \cdot & \cdot & \cdot & \cdot & \cdot & \cdot & \cdot & \cdot & \cdot & \cdot \\
	10 & 2 & 1 & \bar2 & \cdot & \bar1 & \cdot & \bar2 & \cdot & 1 & 1 & \cdot & \cdot & \cdot & \cdot & \cdot & \cdot & \cdot & \cdot & \cdot & \cdot & \cdot & \cdot & \cdot & \cdot & \cdot & \cdot & \cdot & \cdot \\
	9 & 3 & \cdot & 1 & \bar1 & \cdot & \bar1 & 3 & 1 & \cdot & \cdot & \cdot & \cdot & \cdot & \cdot & \cdot & \cdot & \cdot & \cdot & \cdot & \cdot & \cdot & \cdot & \cdot & \cdot & \cdot & \cdot & \cdot & \cdot \\
	5 & 3 & 2 & 1 & 1 & \cdot & \cdot & \bar1 & \bar1 & \bar1 & \bar1 & \cdot & \cdot & \cdot & \cdot & \cdot & \cdot & \cdot & \cdot & \cdot & \cdot & \cdot & \cdot & \cdot & \cdot & \cdot & \cdot & \cdot & \cdot \\
	1 & 1 & 1 & 1 & 1 & 1 & 1 & 1 & 1 & 1 & 1 & \cdot & \cdot & \cdot & \cdot & \cdot & \cdot & \cdot & \cdot & \cdot & \cdot & \cdot & \cdot & \cdot & \cdot & \cdot & \cdot & \cdot & \cdot \\
	5 & \bar3 & 2 & 1 & \bar1 & \cdot & \cdot & 1 & \bar1 & \bar1 & 1 & 1 & \bar1 & 1 & 1 & \bar1 & \bar1 & 1 & \cdot & \cdot & \cdot & \cdot & \cdot & \cdot & \cdot & \cdot & \cdot & \cdot & \cdot \\
	24 & \bar8 & 3 & \cdot & \cdot & 1 & \bar1 & \cdot & \cdot & \cdot & \cdot & 4 & \bar2 & 1 & \cdot & \cdot & 1 & \bar1 & \cdot & \cdot & \cdot & \cdot & \cdot & \cdot & \cdot & \cdot & \cdot & \cdot & \cdot \\
	30 & \bar4 & \bar3 & 2 & 2 & \bar1 & \cdot & \cdot & \cdot & \cdot & \cdot & 5 & \bar1 & \bar1 & 1 & 1 & \bar1 & \cdot & \cdot & \cdot & \cdot & \cdot & \cdot & \cdot & \cdot & \cdot & \cdot & \cdot & \cdot \\
	36 & \cdot & \cdot & \bar4 & \cdot & \cdot & 1 & \cdot & \cdot & \cdot & \cdot & 6 & \cdot & \cdot & \bar2 & \cdot & \cdot & 1 & \cdot & \cdot & \cdot & \cdot & \cdot & \cdot & \cdot & \cdot & \cdot & \cdot & \cdot \\
	30 & 4 & \bar3 & 2 & \bar2 & 1 & \cdot & \cdot & \cdot & \cdot & \cdot & 5 & 1 & \bar1 & 1 & \bar1 & 1 & \cdot & \cdot & \cdot & \cdot & \cdot & \cdot & \cdot & \cdot & \cdot & \cdot & \cdot & \cdot \\
	24 & 8 & 3 & \cdot & \cdot & \bar1 & \bar1 & \cdot & \cdot & \cdot & \cdot & 4 & 2 & 1 & \cdot & \cdot & \bar1 & \bar1 & \cdot & \cdot & \cdot & \cdot & \cdot & \cdot & \cdot & \cdot & \cdot & \cdot & \cdot \\
	6 & 4 & 3 & 2 & 2 & 1 & 1 & \cdot & \cdot & \cdot & \cdot & 1 & 1 & 1 & 1 & 1 & 1 & 1 & \cdot & \cdot & \cdot & \cdot & \cdot & \cdot & \cdot & \cdot & \cdot & \cdot & \cdot \\
	10 & \bar2 & 1 & \bar2 & \cdot & 1 & \cdot & 2 & \cdot & 1 & \bar1 & 4 & \bar2 & 1 & \cdot & \cdot & 1 & \bar1 & 1 & \bar1 & 1 & 1 & \bar1 & \cdot & \cdot & \cdot & \cdot & \cdot & \cdot \\
	45 & \bar3 & \cdot & \bar3 & 1 & \cdot & \cdot & \bar3 & 1 & \cdot & \cdot & 15 & \bar3 & \cdot & \bar1 & 1 & \cdot & \cdot & 3 & \bar1 & \cdot & \bar1 & 1 & \cdot & \cdot & \cdot & \cdot & \cdot & \cdot \\
	30 & 2 & \bar3 & 2 & \cdot & \bar1 & \cdot & 6 & \cdot & \cdot & \cdot & 10 & \cdot & \bar2 & 2 & \cdot & \cdot & \cdot & 2 & \cdot & \bar1 & 2 & \cdot & \cdot & \cdot & \cdot & \cdot & \cdot & \cdot \\
	45 & 9 & \cdot & 1 & \bar1 & \cdot & \cdot & \bar3 & \bar1 & \cdot & \cdot & 15 & 3 & \cdot & \bar1 & \bar1 & \cdot & \cdot & 3 & 1 & \cdot & \bar1 & \bar1 & \cdot & \cdot & \cdot & \cdot & \cdot & \cdot \\
	15 & 7 & 3 & 3 & 1 & 1 & \cdot & 3 & 1 & \cdot & \cdot & 5 & 3 & 2 & 1 & 1 & \cdot & \cdot & 1 & 1 & 1 & 1 & 1 & \cdot & \cdot & \cdot & \cdot & \cdot & \cdot \\
	10 & 2 & 1 & \bar2 & \cdot & \bar1 & \cdot & \bar2 & \cdot & 1 & 1 & 6 & \cdot & \cdot & \bar2 & \cdot & \cdot & 1 & 3 & \bar1 & \cdot & \bar1 & 1 & 1 & \bar1 & 1 & \cdot & \cdot & \cdot \\
	40 & 8 & 1 & \cdot & \cdot & \bar1 & \cdot & \cdot & \cdot & \bar2 & \cdot & 20 & 2 & \bar1 & \cdot & \cdot & \bar1 & \cdot & 8 & \cdot & \bar1 & \cdot & \cdot & 2 & \cdot & \bar1 & \cdot & \cdot & \cdot \\
	20 & 8 & 2 & 4 & \cdot & 2 & \cdot & \cdot & \cdot & 2 & \cdot & 10 & 4 & 1 & 2 & \cdot & 1 & \cdot & 4 & 2 & 1 & \cdot & \cdot & 1 & 1 & 1 & \cdot & \cdot & \cdot \\
	5 & 3 & 2 & 1 & 1 & \cdot & \cdot & \bar1 & \bar1 & \bar1 & \bar1 & 4 & 2 & 1 & \cdot & \cdot & \bar1 & \bar1 & 3 & 1 & \cdot & \bar1 & \bar1 & 2 & \cdot & \bar1 & 1 & \bar1 & \cdot \\
	15 & 7 & 3 & 3 & 1 & 1 & \cdot & 3 & 1 & \cdot & \cdot & 10 & 4 & 1 & 2 & \cdot & 1 & \cdot & 6 & 2 & \cdot & 2 & \cdot & 3 & 1 & \cdot & 1 & 1 & \cdot \\
	1 & 1 & 1 & 1 & 1 & 1 & 1 & 1 & 1 & 1 & 1 & 1 & 1 & 1 & 1 & 1 & 1 & 1 & 1 & 1 & 1 & 1 & 1 & 1 & 1 & 1 & 1 & 1 & 1
	\end{smallmatrix}\right)\]
	
	\end{outdent}
	\caption{Character table of $T_6$ \\ {\small Minus signs are represented as bars and zeros as points for legibility. Note that the diagonal blocks are actually the character tables of the maximal subgroups of $T_6$, that is the symmetric groups on 1 up to 6 elements.}}
	\label{fig:char_tab}
\end{figure}
