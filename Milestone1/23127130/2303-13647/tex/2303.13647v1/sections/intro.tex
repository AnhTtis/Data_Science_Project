The last two decades have seen the development of a new dynamic around the study of monoid representation theory. This is due to applications to certain types of discrete Markov chains and specially Markov chains used to randomly generate combinatorial objects first uncovered in the seminal article of Brown \cite{Brown.2000}.
This has lead to an exploration of the combinatorial properties of monoid representations, for instance in \cite{AyyerSchillingSteinbergThiery.2014.RTrivivalMarkovChains}, \cite{AyyerSchillingThiery.2014.SpectralGap} or \cite{Thiery.CartanMatrixMonoid}.

In this last article \cite{Thiery.CartanMatrixMonoid}, Thiéry gives a formula for the Cartan matrix of a finite monoid of $M$ in terms of number of fixed points and the character table of $M$. More precisely, the formula involves computing the cardinality of the set $\{s \in M \sepp hsk = s\}$ for any $h, k \in M$. In this paper, we set out to use this formula to effectively compute the Cartan matrix of the algebra of $M$ over a perfect field $\kbf$ of null characteristic.

Two difficulties have to be overcome in the pursuit of this goal. Firstly, the cardinality of many interesting families of monoids tends the increase very quickly. For instance, the cardinality of the \emph{full transformation monoid} $T_n$ of all functions from $\intint{1, n}$ to $\intint{1, n}$ is $n^n$, making the naive computation of $|\{s \in M \sepp hsk = s\}|$ impractical even for small $n$. To remedy this we provide an algorithm to efficiently compute this statistic.
Secondly, to use the formula one has to compute the character table of the monoid. The Clifford-Munn-Ponizovskii Theorem (such as presented in \cite{ganyushkin2009irreducible}) gives an explicit description of the simple $\kbf M$-modules and technically makes the computation of the character table possible, provided that we know how to compute the simple modules associated to certain groups. However, this approach is rather convoluted and inefficient. We also note that although some results on character tables are known in the case of many interesting families of monoids, no algorithms are available to compute the character table of an arbitrary finite monoid. Thus, for the general case, we prove a formula for the character table that allows for computation exploiting the Green structure of the monoid for increased efficiency.

In the first section of this paper, we present the results necessary for our fixed-points counting algorithm, with a particular emphasis on the notions of Green classes and \schu groups.
In the second section, we prove a formula for the character table of a finite monoid, after recalling the necessary module and character theoretic results.
In the third section, we give and discuss the algorithms and equation systems used for computing the Cartan matrix with two focuses: the algorithms for fixed point counting and the equation system to compute the character table.
Finally, in the last section, we discuss the performance of these algorithms in terms of execution time and size of tractable problems.