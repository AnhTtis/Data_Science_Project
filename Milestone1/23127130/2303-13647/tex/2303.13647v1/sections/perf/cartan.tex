	Finally, for the computation of the Cartan Matrix, the previous timings show that the vast majority of the computation time is spent computing the character table of the monoid. As the computation of the combinatorial bicharacter is more than a hundred times faster than the computation of the character table, this is a clear invitation to improve in particular the computation of the character of the radical of the $\lc$-classes. In Table \ref{tab:cartan}, we show some timings for that computation, and a comparison with Sage generalist algorithm (based on the Peirce decomposition of the monoid algebra) for the computation of the Cartan Matrix: despite its limitations our specialized algorithm allows for the handling of larger objects. Indeed, our algorithm has near linear performance with respect to cardinality, while sage's has roughly cubic complexity.

\begin{table}[!h]
	\centering
	
	\makebox[\textwidth][c]{
		\begin{tabular}{c|c|c|c}
			Monoid & Coefficients & Sage's & Ours\\% & Monoid & Coefficients & Sage's & Ours \\
			\hline
			$T_3$ & $6^2$ & 575 ms & 56 ms \\ 
			$T_4$ & $11^2$ & 5.23 min & 173 ms \\
			$T_5$ & $18^2$ & {\color{red} >2h} & 1.82 s\\
			$T_6$ & $29^2$ & $\cdots$ & 34.6 s\\   
			$T_7$ & $44^2$ & $\cdots$ & 11.5 min\\
			%$R(4,3)$ & $4^2$ & 2.06 & 39 ms & 
			%$R(6,5)$ & $16^2$ & {\color{red} >1h} & 4.83 s\\
			%$R(5,3)$ & $11^2$ & 5.8 min & 120 ms &  &
			%$R(7,6)$ & $18^2$ & $\cdots$ & 28.1 min \\ 
			%$R(5,4)$ & $14^2$ & 5.8 min & 417 ms &
		\end{tabular}
	}
	
	\caption{Computation time of the Cartan matrix. \\ {\small In the case $T_5$, Sage's algorithm was interrupted before the end of the computation.}}
	\label{tab:cartan}
\end{table}

\begin{figure}[h!]
	\centering
	
	\makebox[\textwidth][c]{
		\includegraphics[width = 0.6\textwidth]{figures/cart_time.eps}
		\hfill
		\includegraphics[width = 0.6\textwidth]{figures/cart_mem.eps}
	}
	
	\caption{Computation time and memory usage for computing the Cartan Matrix using Proposition \ref{prop:formule_cartan}.
		\\
		The blue points correspond to the random monoids, the yellow ones to $T_n$ for $n \in \intint{3, 7}$. As before, the yellow points are excluded of the linear regression although in this case, $T_n$ behave more or less like the randomly chosen monoids. The measured complexity on random monoids is slighly more than linear in time and memory.}
	\label{fig:plot_cartan}
\end{figure}

Again, time and memory usage are in lockstep, and although memory fails before time for $T_8$ and onward and for the monoids of the form $R(9, 8)$, using the regression we can predict a computation time of around 6 hours on our testing machine if it was not memory limited.

An example of a Cartan matrix obtained using our Algorithms and Thiéry's formula is pictured in Figure \ref{fig:cartanT7}. 

\begin{figure}[h!]
	\centering
	\includegraphics{./sections/algo/cartan7.png}
	\caption{Cartan Matrix of $T_7$ \\ {\small For legibility, the entries are represented as grey values. The entries are integers from 0 (in white) to 4 (the single black pixel).}}
	\label{fig:cartanT7}
\end{figure}