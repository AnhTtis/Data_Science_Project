In the case of Algorithm \ref{algo:j_class}, we can give some analysis of the time complexity in terms of the Green structure of the particular $\jc$-class Algorithm \ref{algo:j_class} is applied to.

\begin{prop}
	Consider a $\jc$-class $J$ containing $n_L$ $\lc$-classes, $n_R$ $\rc$-classes, containing an $\hc$-class $H$ with $n_C$ conjugacy classes in $\Gamma'(H)$, and let $C_M$ be a set representatives of the character equivalence classes, as before. Then, the Algorithm \ref{algo:j_class} does:
	\begin{itemize}
		\item $n_C$ cardinality computations of conjugacy classes of $\Gamma'(H)$ (assuming memoization to be able to do a lookup in step 2-a of Algorithm \ref{algo:j_class}, instead of computing it on the fly),
		\item $O(|C_M|(n_L + n_R))$ monoid multiplications, Green class membership tests and conjugacy class of $\Gamma'(H)$ membership tests,
		\item $O(|C_M|n_R)$ computations of $\tau_a$,
		\item $O(|C_M|n_L)$ conjugacy class of $\Gamma'(H)$ cardinality lookups,
		\item $n_C|C_M|^2$ integer multiplications.
	\end{itemize}
\end{prop}

\begin{proof}
	This simply results from an inspection of Algorithm \ref{algo:j_class}.
\end{proof}

As explained before, we cannot meaningfully extend this analysis to Algorithm \ref{algo:j_class}. We can get a similar result by inspection of Algorithm \ref{algo:l_class}: it is essentially the same algorithm, except that it is applied on only one $\lc$-class and that we don't need the final integer multiplications ate the end.

\begin{figure}[h!]
	\centering
	
	\makebox[\textwidth][c]{
		\includegraphics[width = 0.6\textwidth]{figures/fp_time.eps}
		\hfill
		\includegraphics[width = 0.6\textwidth]{figures/fp_mem.eps}
	}
	
	\caption{Computation time and memory usage of Algorithm \ref{algo:M_char}.
		\\
		The blue points correspond to the random monoids, the yellow ones to $T_n$ for $n \in \intint{3, 9}$. The yellow points are excluded of the linear regression as the algorithm is "anormaly" efficient on them: we mesure a complexity on the full transformation monoids of approximately $O(\sqrt{|T_n|})$ while the measured complexity on random monoids is about $O(n^{0.76})$ in time and memory.}
	\label{fig:plot_bichar}
\end{figure}

Note that we do not provide a cumulative formula for the complexity of Algorithm \ref{algo:j_class} as for instance the complexity of a conjugacy class membership test heavily depends on the algorithm used by the computer algebra system, that can itself vary depending on the characteristics of the \schu groups. This makes the task of providing a meaningful evaluation of the global complexity of the algorithm quite difficult, mainly because expressing the complexity of those "elementary" operations of monoid multiplications, membership testing, etc... in terms of the same parameters is not straightforward and in some cases even unknown as noted in \cite{butler1994inductive}.
However, we can at least compare this to the naive algorithm of testing if every element of $J$ is a fixed point which demands $O(n_Ln_R|H|^2|C_M|^2)$ monoid multiplications: as long as the complexity of the more complex operations of Green class or conjugacy class membership testing remains limited in terms of monoid multiplications, our complexity is better. For instance, in the case of the monoid $T_n$, all the required operations can be done on $O(n)$, making Algorithm \ref{algo:j_class} (and, in turn, Algorithm \ref{algo:M_char}) more efficient than the naive algorithm, as can be seen in Table \ref{tab:bichar}, with a sub linear (with respect to cardinality) measured complexity (Figure \ref{fig:plot_bichar}).

\begin{table}[!h]
	\centering
	
	\begin{tabular}{c|c|c|c|c}
		Monoid & Cardinality & Coefficients & Naive & Ours \\
		\hline
		$T_3$ & 27 & $6^2$ & 29 ms & 18 ms\\
		%$R(4,3)$ & 80 & $4^2$ & 35 ms & 12 ms\\
		%$R(5,3)$ & 150 & $11^2$ & 53 ms & 34 ms\\
		$T_4$ & 256 & $11^2$ & 92 ms & 63 ms\\
		%$R(5,4)$ & 754 & $14^2$ & 228 ms & 59 ms\\
		$T_5$ & 3125 & $18^2$ & 1.44 s & 113 ms\\
		%$R(6,5)$ & 5453 & $16^2$ & 2.00 s & 120 ms\\
		
		$T_6$ & 46656 & $29^2$ & 53.0 s & 0.34 s\\
		%$R(7,6)$ & 129337 & $19^2$ & 70.0 s & 2.23 s\\
		$T_7$ & 823543 & $44^2$ & >30 min & 1.59 s\\ 
		%$R(8,9)$ & 1765334 & $29^2$ & $\cdots$ & 43,7 s\\
		$T_8$ & 16777216 & $66^2$ & $\cdots$ & 8.86 s\\ 
		$T_9$ & 387420489 & $96^2$ & $\cdots$ & 56.7 s\\
	\end{tabular}
	
	\caption{Computation time of the regular representation bicharacter.}
	\label{tab:bichar}
\end{table}