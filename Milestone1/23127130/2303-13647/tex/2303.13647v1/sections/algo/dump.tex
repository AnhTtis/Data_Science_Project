\todo{Order everything logically. Do the same in the previous section}
\paragraph{Basic computations}
\begin{itemize}
	\item \textbf{Product} We suppose that for any $a \in M, i \in \intint{1, n}$, $a(i)$ can be computed in $O(1)$. Then the product of $a, b \in M$ can be computed in $O(n)$ by successive applications.
	\item \textbf{Green Pairs} Let $a, a' \in M$. We suppose that $\im(a)$ and $\im(a')$ have the same cardinality. Consider the bijective map
	\[\lambda : \left\{\begin{aligned}
	\im(a) &\longrightarrow \im(a')\\
	a(i) &\mapsto a'(i)
	\end{aligned}\right..\]
	We have $\lambda \circ a = a'$. In particular, recall from Example \ref{ex:green_tn} that if two elements are in the same $\lc$-class in a transformation monoid, they have the same image. Thus, we can \emph{almost} say that $(\lambda, \lambda\inv)$ is a left Green pair for $\rc(a), \rc(a')$. However, note that $\lambda$ is not, in general, an element of $M$ (or even of $T_n$). Still, we know that such a Green pair exists in $M$ and the argument of Proposition \ref{prop:schu_acts_freely} still applies to left composition by $\lambda$ and $\lambda\inv$: since the action is entirely determined by the image of $a$, computing the product by a "real" Green pair sending $a$ to $a'$ yields the same result as left composition by our "fake" Green pair. This construction and reasoning can be applied to two elements $a, a'$ with the same kernel to get right Green pairs. In both cases, the Green pairs can be computed in $O(n)$.
	\item \textbf{Elements of the \schu groups} Consider the case in the previous point where $a$ and $a'$ are in the same $\hc$-class $H$. In that case the set of  $\lambda$ we obtain forms a group, and since every element of ${}_M\stab(H)$ corresponds to a "fake" Green pair, the set of all such $\lambda$ is a group isomorphic to the left \schu group $\Gamma(H)$. The same construction on the right gives a group isomorphic to $\Gamma'(H)$. Given elements $h \in {}_M\stab(H)$ and $k \in \stab_M(H)$, the corresponding elements $h\mul{H} \in \Gamma(h)$ and $\mul{H}k \in \Gamma'(H)$ can be computed in $O(n)$.
	\item \textbf{Morphism from $\Gamma(H)$ to $\Gamma'(H)$} Let $a \in H$, $g \in \Gamma(H)$ and recall that $\tau_a(g)$ is defined as the unique element of $\Gamma'(H)$ such that $g\cdot a = a \cdot \tau_a(g)$. Given $a$ and $g$, $\tau_a(g)$ can be computed in $O(n)$. \todo{justifier. Je pense que le O(nlogn) que donne James pourrait être un O(n))}
\end{itemize}

\paragraph{Computing the decomposition of a $\jc$-class in $\lc$ and $\rc$-classes.}
\& related tasks

\paragraph{Finding idempotents / Finding a regular $\hc$-class.}

\paragraph{Computing $C_M$}

\paragraph{Checking membership}
