In this section we first recall essential and elementary results on the Green structure of finite monoids and on \schu groups. The informed reader may skip this first paragraph, with the exception of the notations (\ref{not:times}) that are used throughout this paper. We then use these results to devise a fixed point counting method.

%\begin{dftn}
%	A finite monoid is a triple $(M, \cdot, 1_M)$ where $M$ is a finite set, $\cdot : M \times M \rightarrow M$ is an associative internal binary composition on $M$ and $1_M$ is the \emph{neutral element of $M$} satisfying $1_Mx=x1_M=x$ for any $x\in M$. As per usual convention, we will designate monoids by their underlying set and simply note $1$ for the neutral element.
%\end{dftn}

In the totality of this paper, we assume that all monoids are finite. We will often use the following special case of finite monoid to illustrate the various results presented hereafter.

%\begin{lined}
	\begin{dftn}[The full transformation monoid]
		Consider the set $T_n$ of all transformations of the set $\{1, \dots, n\}$, equipped with the multiplication given by map composition: $\forall f,g \in T_n, fg = f \circ g$.
		This is a monoid, aptly named the \emph{full transformation monoid}. A submonoid of $T_n$ is called a transformation monoid and $n$ is called its \emph{rank}.
	\end{dftn}
	
%	\begin{ex}
%		For a field $\kbf$ and $n \in \NN$, $(M_n(\kbf), \times)$ is a monoid.
%	\end{ex}
%	
%	\begin{ex}
%		$\symm_n$, and more generally any finite group, is a monoid.
%	\end{ex}
%\end{lined}

