Note that if we choose an element $a \in M$ and denote by $L$ its $\lc$-class and $H$ its $\hc$-class, we can equip $\kbf L$ with a $\kbf M-\module-\kbf\Gamma'(H)$ structure. $\kbf L$ is already a $\module-\kbf\Gamma'(H)$ by definition of $\Gamma'(H)$. We can also make it into a $\kbf M-\module$ by setting, for every $m \in M$ and $l \in L$:
\[m\cdot l = \left\{\begin{aligned}
& ml \textrm{ if } ml \in L\\
& 0 \textrm{ otherwise}
\end{aligned}\right..\]
This is well defined, as $ml \notin L$ implies that $l >_{\lc} ml$ and so for every $m' \in M$, $l >_{\lc} m'ml \notin L$ : once fallen out of $L$, we cannot climb back in.

We have previously stated that the representation theory of monoids is an extension of the representation theory of some subgroups. This mainly expressed using the two following functors.

\begin{dftn}
	Let $e \in M$ be an idempotent, $\lc(e)$ its $\lc$-class, $G_e$ the associated maximal subgroup. We define the two following maps :
	\begin{align*}
	\indu_{G_e}^M : & \left\{ \begin{aligned}
	\kbf G_e\mathrm{-mod} & \longrightarrow \kbf M\mathrm{-mod}\\
	V & \longmapsto \kbf\lc(e) \otimes_{\kbf G_e} V
	\end{aligned} \right.\\
	N_e : & \left\{ \begin{aligned}
	\kbf M\mathrm{-mod} & \longrightarrow \kbf M\mathrm{-mod}\\
	V & \longmapsto \{v \in V \sepp eMv = 0\}
	\end{aligned} \right..
	\end{align*}
\end{dftn}

The idempotents and their maximal subgroups play a central role in the theory. One can show (see for instance \cite[Proposition 1.14]{pin}) that if $e, f$ are two idempotents in the same $\jc$-class, there are some $x, x' \in M$ such that $xx' = e$ and $x'x = f$ and that $G_e \cong G_f$. A $\jc$-class containing an idempotent is called a \emph{regular} $\jc$-class.

We are almost ready to state the Clifford-Munn-Ponizovskii theorem, which is the central piece connecting group and monoid representation theory. We will need the notion of \emph{apex} of a $\kbf M$-module. A proof of the Clifford-Munn-Ponizovskii Theorem can be found in \cite[Section 5.2]{steinberg2016representation}.

\begin{dftn}
	Let $V$ be a $\kbf M-\module$, we denote its annihilator in $M$ by $\ann_M(V) = \{m \in M\sepp mV = \{0\}\}$. 
	This is clearly an two-sided ideal of $M$ and as such is an union of $\jc$-classes.
	A regular $\jc$-class $J$ is said to be the \emph{apex} of $V$ if $\ann_M(V) = I_J$ where $I_J = \{J \not \leql \jc(s) \sepp s \in M\}$. If $e \in J$ is an idempotent, we also say that $V$ has apex $e$.
\end{dftn}

\begin{thm}[Clifford-Munn-Ponizovskii]\label{thm:CMP}
	Let $M$ be a finite monoid, $e \in M$ an idempotent and $\kbf$ be a field.
	\begin{enumerate}
		\item There is a bijection between isomorphism classes of simple $M$-modules with apex $e$ and isomorphism classes of simple $G_e$-modules given by :
		\[V \longmapsto V^{\#} = \indu_{G_e}^M(V) / \rad(\indu_{G_e}^M(V)).\]
		The reciprocal bijection is given by $S \longmapsto eS$.
		\item $\rad(\indu_{G_e}^M(V)) = N_e(\rad(\indu_{G_e}^M(V)))$
		\item Every simple $M$-module has an apex.
		\item Every composition factor of $\indu_{G_e}^M(V)$ with the exception of $V^{\#}$ has an apex strictly $\jc$-greater than $e$. Moreover, $V^{\#}$ has apex $e$ and is a factor of multiplicity one.
	\end{enumerate}
\end{thm}

This allows us the following description of the $\lc$-class of an idempotent $e$.

\begin{prop}\label{prop:L_is_sum}
	Let $e \in M$ be an idempotent and $G_e$ be the maximal subgroup at $e$. Let $\irr_e$ be a set of representatives of the isomorphism classes of simple $\kbf G_e$-modules. Then: 
	$$\kbf \lc(e) = \bigoplus_{V\in \irr_e} \indu^M_{G_e}(V) \otimes_{\kbf} V^*$$
	where $V^*$ is the dual of $V$.
\end{prop}

\begin{proof}
	By definition, $\indu^M_{G_e}(V) = \lc(e) \otimes_{\kbf G_e} V$. Now, since direct sum and tensor product over a ring with identity commute :
	\[\bigoplus_{V\in \irr_e} \indu^M_{G_e}(V)\otimes V^* = \bigoplus_{V\in \irr_e} \kbf\lc(e) \otimes_{G_e} V\otimes V^* = \kbf\lc(e) \otimes_{G_e} \left(\bigoplus_{V\in \irr_e} V\otimes V^*\right)\]
	Because $\kbf$ is of null characteristic, $\kbf G_e$ is semi-simple. By the Wedderburn-Artin theorem, $\kbf G_e = \bigoplus_{V\in \irr_e} V\otimes V^*$ so :
	\[\bigoplus_{V\in \irr_e} \indu^M_{G_e}(V) \otimes V^* = \kbf\lc(e) \otimes_{\kbf G_e} \kbf G_e = \lc(e)\]
	since $\kbf G_e$ is a ring with identity.
\end{proof}

Note that this puts in relation three kinds of modules : the simple $\kbf G_e$-modules, which are well understood, $\kbf \lc(e)$ which is understood as well because it is a combinatorial module\footnote{That is, the multiplication of an element of the basis of the module by an element of $M$ is either an other element of the basis or 0.}, and finally the modules $\indu^M_{G_e}(V)$ which contain, in a sense, the simple $\kbf M$-modules that we are after. According to the Clifford-Munn-Ponizovskii Theorem, we still need to remove the radical of each $\indu^M_{G_e}(V)$ factor. Proposition \ref{prop:rad_is_ne} puts the radical in a form similar to Theorem \ref{thm:CMP} while Proposition \ref{prop:quotient_semisimple} does exactly this. Lemma \ref{lem:PG} and its Corollary are technical results on radicals used in the proof of Proposition \ref{prop:rad_is_ne}.

\begin{lemme}\label{lem:PG}
	Let $A, B$ be two finite dimensional algebras over a perfect field $\kbf$. Then: 
	\[\rad(A\otimes B) = \rad(A)\otimes B + A \otimes \rad(B).\]
\end{lemme}

While we haven't be able to find a source for that claim it seems to be folklore in the algebra representation community. For the sake of completeness, we reproduce a proof communicated to us by Pr. Pierre-Guy Plamondon\footnote{Pr. Pierre-Guy Plamondon, Laboratoire de Mathématiques de Versailles, Université Paris-Saclay, \href{https://www.imo.universite-paris-saclay.fr/~plamondon/}{website}.}.

\begin{proof}
	Because $\kbf$ is perfect, Wedderburn's Principal Theorem applies and we get the decompositions $A = A' \oplus \rad(A), B = B' \oplus \rad(B)$, with $A'$ and $B'$ semi-simple algebras. To prove the result, we show that $\rad(A)\otimes B + A \otimes \rad(B)$ is a nil radical and that \[\faktor{A \otimes B}{\rad(A)\otimes B + A \otimes \rad(B)}\] is semi-simple.
	Let us first show that the quotient is semi-simple.
	We have:
	\[
	\begin{aligned}
		A\otimes B & = (A' \oplus \rad(A)) \otimes (B' \oplus \rad(B))\\
		& = A'\otimes B' \oplus A'\otimes\rad(B) \oplus \rad(A)\otimes B' \oplus \rad(A)\otimes\rad(B).
	\end{aligned}
	\]
	On the other hand, the same decompositions give us 
	\[A \otimes \rad(B) \oplus \rad(A) \otimes B = A'\otimes\rad(B) \oplus \rad(A)\otimes B' \oplus \rad(A)\otimes\rad(B).\]
	Finally,
	\[\faktor{A\otimes B}{A \otimes \rad(B) \oplus \rad(A) \otimes B} = A' \otimes B'\]
	which, since $A', B'$ are semi-simple, is also semi-simple. $A \otimes \rad(B) \oplus \rad(A) \otimes B$ is also nil, because $\rad(B)$ and $\rad(A)$ are, so, indeed, $\rad(A \otimes B) = A \otimes \rad(B) \oplus \rad(A) \otimes B$.
\end{proof}

From Lemma \ref{lem:PG}, we get the following Corollary by recalling that if $V$ is a $A$-module, $\rad_A(V) = \rad(A)\cdot V$.

\begin{cor}
	Let $A, B$ be two finite dimensional unitary algebras over a perfect field. If $V_A \otimes V_B$ is a $A-\module-B$ (or equivalently a $A\otimes B^{op}-\module$), then 
	\[\rad_{A\otimes B^{op}}(V_A \otimes V_B) = \rad_A(V_A)\otimes B + A\otimes \rad_B(V_B).\]
\end{cor}

This allows us to identify the radical of $\kbf \lc(e)$.

\begin{prop}\label{prop:rad_is_ne}
	Let $M$ be a finite monoid, $e \in S$ an idempotent $G_e$ be the maximal subgroup at $e$ and $\kbf$ be a perfect field. 
	Then: 
	$$\rad_{\kbf M \otimes \kbf G_e^{op}}(\kbf \lc(e)) = N_e(\kbf \lc(e)).$$
\end{prop}

\begin{proof}
	Using Lemma \ref{lem:PG}, for $V$ a simple $G_e$-module, we have that :
	\[\begin{aligned}
	\rad_{\kbf M \otimes \kbf G_e^{op}}& (\indu^S_{G_e}(V) \otimes_{\kbf} V^*) \\
	= & \rad_{\kbf M} \indu_{G_e}^M(V) \otimes V^* + \indu_{G_e}^M(V) \otimes \rad_{\kbf G_e^{op}}(V^*)\\
	\overset{(1)}{=} & \rad_{\kbf M} \indu_{G_e}^M(V) \otimes V^*\\
	\overset{(2)}{=} & N_e(\indu_{G_e}^M(V)) \otimes V^*
	\end{aligned}\]
	where equality (1) comes from the simplicity of $V^*$ as a $\kbf G_e^{op}$-module and (2) is the second point of Theorem \ref{thm:CMP}.
	
	Since radical and direct sums commute, denoting by $\irr_e$ a set of representatives of the isomorphism classes of simple $\kbf G_e$-modules, we know that:
	\[\rad_{\kbf M \otimes \kbf G_e^{op}}(\lc(e)) = \bigoplus_{V \in \irr_e} N_e(\indu_{G_e}^M(V)) \otimes V^*.\]
	It remains to be seen why 
	\[\bigoplus_{V \in \irr_e} N_e(\indu_{G_e}^M(V)) \otimes V^* = N_e(\kbf\lc(e)).\]
	It is clear the direct sum on the left is a subset of the set on the right. 
	For the other inclusion, we see that if $V, V'$ are $\kbf M$-modules, $N_e(V \oplus V') = N_e(V) \oplus N_e(V')$. 
	Given the proposition \ref{prop:L_is_sum}, it is enough to show that $N_e(\indu_{G_e}^M(V)) \otimes V^* = N_e(\indu_{G_e}^M(V) \otimes V^*)$. 
	Let $x \in \indu_{G_e}^M(V) \otimes V^*$ be such that for every $m \in M, emx = 0$. $x$ can be writen as $\sum_i (\sum_j x_{i,j} b_j) \otimes b'_i$ where $\{b_j\}_j$ is a basis of $\indu_{G_e}^M(V)$ and $\{b'_i\}_i$ is a basis of $V^*$. For every $m \in M$, we have :
	\[em \cdot x = em \cdot \sum_i (\sum_j x_{i,j} b_j) \otimes b'_i = \sum_i (em \cdot \sum_j x_{i,j} b_j) \otimes b'_i = 0\]
	that is, for every $b'_i$ we get $em \cdot \sum_j x_{i,j}b_j = 0$ so $\sum_j x_{i,j}b_j \in N_e(\indu_{G_e}^M(V))$ which means $x \in N_e(\indu_{G_e}^M(V)) \otimes V^*$.
\end{proof}

\begin{prop}\label{prop:quotient_semisimple}
	Let $e \in M$ be an idempotent and $G_e$ be the maximal subgroup at $e$. Let $\irr_e$ be a set of representatives of the isomorphism classes of simple $\kbf G_e$-modules.
	Then:
	\[\kbf \lc(e) / \rad_{\kbf M \otimes \kbf G_e^{op}}(\kbf\lc(e)) \cong \bigoplus_{V \in \irr_e} V^{\#} \otimes V^*.\]
\end{prop}

\begin{proof}
	From Proposition \ref{prop:rad_is_ne}, we have a decomposition of $\rad_{\kbf M \otimes \kbf G_e^{op}}(\lc(e))$ as a direct sum adapted to the decomposition of $\kbf\lc(e)$ as $\bigoplus_{V\in \irr_e} \indu^S_{G_e}(V) \otimes_{\kbf} V^*$. So:
	\[\kbf \lc(e) / \rad_{\kbf M \otimes \kbf G_e^{op}}(\kbf\lc(e)) \cong \bigoplus_{V \in \irr_e} (\indu^S_{G_e}(V) \otimes_{\kbf} V^*) / (N_e(\indu^S_{G_e}(V))\otimes V^*).\]
	From Theorem \ref{thm:CMP}, we know that
	\[0 \longrightarrow N_e(\indu^S_{G_e}(V)) \longrightarrow \indu^S_{G_e}(V) \longrightarrow V^{\#} \longrightarrow 0\]
	is a short exact sequence. Since $N_e(\indu^S_{G_e}(V))\otimes V^*$ a submodule of $\indu^S_{G_e}(V)\otimes V^*$ and because tensor product is right exact we have a short exact sequence :
	\[0 \longrightarrow N_e(\indu^S_{G_e}(V))\otimes V^* \longrightarrow \indu^S_{G_e}(V)\otimes V^* \longrightarrow V^{\#}\otimes V^* \longrightarrow 0\]
	which proves the result.
\end{proof}