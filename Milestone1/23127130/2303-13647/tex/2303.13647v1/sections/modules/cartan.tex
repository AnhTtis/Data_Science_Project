We are, at last, in measure to state the formula from Thiéry for the Cartan Matrix. Without getting into the specific details, the Cartan matrix can be seen as measure of how "not semi-simple" the algebra of the monoid is. 
We use a non standard definition of the Cartan matrix, first given in \cite[Definition 2.6]{Thiery.CartanMatrixMonoid}. A formal proof that this is equivalent to the usual definition can be found in \cite[Corolary 7.28]{steinberg2016representation}.

\begin{dftn}
	Let $\{S_1, \dots, S_n\}$ be a set of representatives of the isomorphism classes of simple $\kbf M$-modules. The simple $\kbf M \otimes \kbf M^{op}$ modules are the $S_i \otimes S_j^*$ for all $i, j \in \intint{1, n}$. Denote by $[\kbf M : S_i \otimes S_j^*]$ the multiplicity of $S_i \otimes S_j^*$ as a composition factor of $\kbf M$.
	
	The Cartan matrix of $\kbf M$ is defined by:
	\[C(\kbf M) = ([\kbf M : S_i \otimes S_j^*])_{i, j}\] 
\end{dftn}

In other words, the Cartan matrix is a recording of the multiplicities of the composition factors of $\kbf M$ as a $\kbf M \otimes \kbf M^{op}$ module. But so is its character! The difference being that the character of $\kbf M$ as it is computed is expressed in the basis of the character equivalency classes of $M \times M^{op}$ while the Cartan matrix is expressed directly in the basis of the simple modules. Since the basis change between the two is precisely given by the character table and hence, we have the Thiéry's Formula for the Cartan matrix.

\begin{prop}\label{prop:formule_cartan}
	The Cartan matrix is given by the formula:
	\[C(\kbf M) = {}^tX_M\inv B X_M\inv\]
	where $B = (|\{s \in M \sepp msm'\}|)_{m,m' \in C_M}$
\end{prop}