In the remainder of this paper, $\kbf$ is a perfect field of null characteristic and $M$ is still a finite monoid. Furthermore, we suppose that $\kbf$ is "big enough", meaning that if not algebraically closed, at least a splitting field for the characteristic polynomials of the elements of $M$ seen as linear maps on $\kbf M$.

We will discuss the representation theory of a monoid $M$ over $\kbf$ using the language of modules. That is, a representation of $M$ over $\kbf$ will be a $\kbf$-vector space $V$ equipped with a linear action of the algebra $\kbf M$. As is usage, whenever the action is on the left we will say that $V$ is a $\kbf M-\module$ and a $\module - \kbf M$ if the action is on the right. If $M$, $M'$ are two finite, possibly different monoids, a $\kbf M - \module - \kbf M'$ is simply simultaneously a $\kbf M-\module$ and a $\module-\kbf M'$. For a monoid $M$, we denote by $M^{op}$ the \emph{opposite} monoid, with multiplication defined by $m\cdot_{M^{op}}m' = m'm$. We will use liberally the fact that a $\kbf M - \module - \kbf M'$ is naturally a $\kbf M \otimes \kbf M'^{op} - \module$ and a $\kbf(M \times M'^{op}) - \module$, and reciprocally. In the totality of this paper, we assume that the modules are finite dimensional as vector spaces over $\kbf$. Because of this, the Jordan-Hölder Theorem applies and the set of composition factors counted with multiplicities of a module is independent of the choice of a composition series. If $S$ is a $\kbf M$-module and $S$ is a simple $\kbf M$-module, we denote by $[V:S]$ the multiplicity of $S$ as a composition factor of $V$.


In this section, we deal with monoid representation theory, with the goal in mind to compute the character table of $M$. Using the Munn-Clifford-Ponizovskii, this can largely be reduced to group representation theory. Stated differently, the representation theory of a monoid $M$ is an extension of the representation theory of certain groups embedded in $M$. The groups in question are precisely the groups of Definition \ref{dftn:idem_and_maxgrp}.
In the first part of this section, we use this fact to find a description of an $\lc$-class containing an idempotent $e$ quotiented by its radical as a product of simple $\kbf G_e$-modules and simple $\kbf M$-modules. In the second part, we translate this decomposition in terms of characters, which gives us the formula we seek.
Finally, we recall and discuss the formula for the Cartan matrix from Thiéry \cite{Thiery.CartanMatrixMonoid}.