\subsection{Proof of main lemma for Phase 2 (Lemma \ref{lemma:phase2combined})}\label{sec:phase2proof}


\begin{proof}[Proof of Lemma \ref{lemma:phase2combined}]
At the beginning of the proof we would like to recall that as described in Remark \ref{choiceofconstants} the constants 
$\hat{c}_1,\hat{c}_2,\hat{c}_3,\hat{c}_4,\hat{c}_5,\hat{c}_6,\hat{c}_7 $
are chosen such that 
$\hat{c}_1, \hat{c}_2, \hat{c}_3^{4/5} \ll \hat{c}_4 \hat{c}_5 $, $\hat{c}_4 \ll \hat{c}_5 \ll \hat{c}_6 \ll 1$ holds.
Set
\begin{equation*}
    t_2:= \min \left\{ t \in \mathbb{N} : \ \sglmin \left( \LAT \Zt \right) \ge  \sqrt{\frac{\sglmin (X)}{8}} \text{ and } t \ge t_1  \right\}.
\end{equation*}
We show by induction that it holds that for $t_1 \le t \le t_2$ that
\begin{align}
    \sglmin \left( \LAT \Zt \right) &\ge  \left(1 + \frac{1}{8}\mu\sglmin(X)  \right)^{t-t_1} \sglmin \left( \LAT \Ztone \right),  \label{ineq:phase2hyp1} \\
    \norm{\Zt \Qtp} & \le \left( 1+ \frac{\mu}{1500} \sglmin(X)  \right)^{t-t_1} \norm{\Ztone Q_{t_1,\bot}},  \label{ineq:phase2hyp2} \\
    \norm{\LAPT P_{\Zt \Qt }} & \le \frac{\hat{c}_4}{\kappa^2},    \label{ineq:phase2hyp3} \\
    \norm{\Zt} &\le 2 \sqrt{\norm{X}}, \label{ineq:phase2hyp4}\\
    \norm{\tilZtT \Zt } &\le  \norm{ \widetilde{Z}_{t_1}^T Z_{t_1} } + 400\mu^2 \left( t-t_1 \right) \norm{X}^3    \le  \frac{\hat{c}_7 \norm{X}}{\kappa^{4}},  \label{ineq:phase2hyp_balanc1}\\
    \norm{ \tilZtT \Zt \Qtp } &\le \frac{\hat{c}_4 \hat{c}_5 \sqrt{ \norm{X}} }{\kappa^{3} } \left( 1+ \frac{\mu}{1500} \sglmin(X)  \right)^{t-t_1} \norm{\Ztone Q_{t_1,\bot}},  \label{ineq:phase2hyp_balanc2}\\
    \norm{ \tilZtT \PZQ } &\le \frac{\hat{c}_4 \hat{c}_6 \sqrt{\norm{X}}}{\kappa^3}.  \label{ineq:phase2hyp_balanc3}
\end{align}        
Before establishing these inequalities, we note that from \eqref{ineq:phase2hyp1} and the definition of $t_2$ the upper bound on $t_2-t_1$ given by inequality \eqref{ineq:t2bound} directly follows. 

We are going to prove these inequalities by induction. For that, we note first that the base case $t=t_1$ follows directly from the assumptions in this lemma.\\
To show the induction step $ t\rightarrow t+1$ for $t<t_2$ we first note that 
\begin{align}
\norm{\Deltat} =& \norm{(\Bcal^* \Bcal - \I)  \left( \symA - \Zt \ZtT +  \tilZt \tilZtT \right) } \nonumber \\
\le&\norm{(\Bcal^* \Bcal - \I)  \left( \symA  \right) }+\norm{(\Bcal^* \Bcal - \I)  \left( \Zt \Qt \QtT \ZtT \right) }+\norm{(\Bcal^* \Bcal - \I)  \left( \Zt \Qtp \QtpT \ZtT \right) } \nonumber \\
&+\norm{(\Bcal^* \Bcal - \I)  \left(  \tilZt \Qt \QtT \tilZtT \right) }+\norm{(\Bcal^* \Bcal - \I)  \left(  \tilZt \Qtp \QtpT \tilZtT \right) } \nonumber \\
\stackrel{(a)}{\le} & \delta \sqrt{r} \left(  \norm{\symA} + \norm{\Zt \Qt}^2 + \norm{\tilZt \Qt}^2  \right) +   \delta \left( \nucnorm{ \Zt\Qtp \QtpT \ZtT } + \nucnorm{ \tilZt\Qtp \QtpT \tilZtT }   \right) \nonumber \\
\stackrel{(b)}{=}& \delta \sqrt{r} \left(  \norm{\symA} + 2\norm{\Zt \Qt}^2  \right) + 2\delta \nucnorm{ \Zt\Qtp \QtpT \ZtT }  \nonumber\\
\stackrel{(c)}{\le} & \delta \sqrt{r}  \left(  9 \norm{X} + 2\nucnorm{ \Zt\Qtp \QtpT \ZtT }   \right), \label{ineq:phase2aux2}
\end{align}
where inequality $(a)$ follows from Lemma \ref{lemma:RIPlemma} and the restricted isometry property of the measurement operator $\mathcal{B}$.
In equality $(b)$ we used the symmetry between $\Zt$ and $\tilZt$ (see Lemma \ref{lemma:symmetry}) and in inequality $(c)$ we used the induction hypothesis \eqref{ineq:phase2hyp4}.
Next, we are going to show inequality \eqref{ineq:phase2hyp2} for $t+1$.
For that, we observe that 
\begin{equation}
\nucnorm{ \Zt\Qtp \QtpT \ZtT } \le \left( k -r \right) \norm{\Zt \Qtp}^2. \label{ineq:phase2aux3}
\end{equation}
To estimate this expression further, we observe that due to the induction hypothesis \eqref{ineq:phase2hyp2} it holds
\begin{align}
\norm{\Zt \Qtp} &\le  \left( 1+ \frac{\mu}{1500} \sglmin(X)  \right)^{t-t_1} \norm{\Ztone Q_{t_1,\bot}} \nonumber \\
& \stackrel{(a)}{\le} \exp \left(    \frac{2 \ln \left( \frac{ \sqrt{ \sglmin \left(X\right) } }{\sqrt{8} \sglmin \left( \LAT \Ztone \right) }   \right)}{\ln \left( 1 +\frac{\mu}{8} \sglmin \left(X\right)  \right)} \ln \left(  1+ \frac{\mu}{1500} \sglmin(X)   \right)   \right) \norm{\Ztone Q_{t_1,\bot}} \nonumber \\
& \stackrel{(b)}{\le} \exp \left(  \frac{ \ln \left( \frac{ \sqrt{ \sglmin \left(X\right) } }{\sqrt{8} \sglmin \left( \LAT \Ztone \right) }   \right)   }{5}  \right) \norm{\Ztone Q_{t_1,\bot}} \nonumber \\
& =  \left(  \frac{ \sglmin \left(X\right)  }{ 8 \sglmin^2 \left( \LAT \Ztone \right)  } \right)^{1/10}  \norm{\Ztone Q_{t_1,\bot}} \nonumber \\
& \stackrel{(c)}{\le}  \left(  \frac{ \sglmin \left(X\right)  }{ 8 \sglmin^2 \left( \LAT P_{Z_{t_1} Q_{t_1}} \right) \sglmin^2 \left(\Ztone Q_{t_1}  \right)  } \right)^{1/10}  \norm{\Ztone Q_{t_1,\bot}} \nonumber \\
& \stackrel{(d)}{\le}  \left(  \frac{ \sglmin \left(X\right)  }{ 128 \norm{ \Ztone Q_{t_1,\bot} }^2  } \right)^{1/10}  \norm{\Ztone Q_{t_1,\bot}} \nonumber \\
& = \left( \frac{1}{128} \right)^{1/10}  \left( \sglmin \left(X\right) \right)^{1/10}   \norm{ \Ztone Q_{t_1,\bot}}^{4/5}   \label{ineq:phase2aux5} \\
& \stackrel{(e)}{\le} \frac{ \hat{c}_3^{4/5} \sqrt{ \sglmin \left(X\right) } }{ \kappa^{7/2} k^{4/5}}, \label{ineq:phase2aux4}
\end{align}
where in inequality $(a)$ we have used the upper bound on $ t-t_1 \le t_2 -t_1$ in inequality \eqref{ineq:t2bound}.
In inequality $(b)$ we have used the elementary inequalities $ \frac{x}{1-x} \le \ln \left(1+x\right) \le x $ and the assumption on the step size $\mu \le \frac{ \hat{c}_2 }{\kappa^4 \norm{X}}$. 
Inequality $(c)$ follows from the fact that $ \sglmin \left( \LAT \Ztone \right) \ge \sglmin \left(   \LAT P_{Z_{t_1} Q_{t_1}}  \right) \sglmin \left( \Ztone Q_{t_1} \right)  $.
Note that it follows from assumption \eqref{ineq:phase2assump2} that $\sglmin \left(   \LAT P_{Z_{t_1} Q_{t_1}}  \right)  \ge 1/2 $.
Together with assumption \eqref{ineq:phase2assump1} this implies inequality $(d)$. 
Inequality $(e)$ follows again from assumption \eqref{ineq:phase2assump1}.
We remark that with an analogous computation as in the above inequality chain, we also can show that 
\begin{equation}\label{eqref:phase2aux12}
\begin{split}
\norm{ \tilZtT \Zt \Qtp }
&\le \frac{c_4 c_5 \sqrt{\norm{X}}}{\kappa^3} \left(  1 + \frac{\mu \sglmin \left(X\right)}{1500} \right)^{t-t_1} \norm{ Z_{\tone} Q_{\tone, \bot} }\\
&\le \left( \frac{1}{128} \right)^{1/10} \frac{ \hat{c}_4 \hat{c}_5 \sqrt{\norm{X}}}{\kappa^3} \left( \sglmin \left(X\right) \right)^{1/10}   \norm{ \Ztone Q_{t_1,\bot}}^{4/5}  
\le   \frac{ \hat{c}_3^{4/5} \hat{c}_4 \hat{c}_5 \sqrt{ \norm{X} \sglmin \left(X\right) } }{ \kappa^{7/2} k^{4/5}}.
\end{split}
\end{equation}
Combining \eqref{ineq:phase2aux3} and \eqref{ineq:phase2aux4} and inserting this into \eqref{ineq:phase2aux2} it follows that 
\begin{equation}\label{ineq:phase2aux11}
    \norm{\Deltat} \le 11 \delta \sqrt{r} \norm{X} \le \frac{11 \hat{c}_1}{\kappa^2} \sglmin \left(X\right),
\end{equation}
where in the last inequality we have used our assumption $\delta \le \frac{ \hat{c}_1}{\kappa^{3} \sqrt{r} }$.
Thus, we conclude that all the assumptions for Lemma \ref{lemma:sigmingrowth} are fulfilled.
It follows that 
\begin{equation}\label{ineq:phase2aux1}
    \sglmin(\LAT\Zplus) \ge \sglmin(\LAT\Zplus\Qt) \ge \sglmin(\LAT\Zt)\left(1 + \frac{1}{4}\mu\sglmin(X) - \mu\sglmin^2(\LAT\Zt)\right).
\end{equation}
Since we assumed $t < t_2 $, which implies by the definition of $t_2$ that $ \sglmin \left( \LAT \Zt \right) < \sqrt{\frac{ \sglmin(X) }{8}} $, we obtain that 
\begin{equation*}
  \sglmin \left(\LAT \Zplus\right) \ge \left(1 + \frac{1}{8}\mu\sglmin(X)  \right)  \sglmin(\LAT\Zt).
\end{equation*}
This implies \eqref{ineq:phase2hyp1} for $t+1$.
Note that \eqref{ineq:phase2aux1} also implies that $\LAT\Zplus\Qt$ has full rank.
Hence, we can apply Lemma \ref{lemma:noisetermgrowth} and by choosing the absolute constants $\hat{c}_1$, $\hat{c}_2$, and $\hat{c}_4 $ small enough we obtain that
\begin{equation}
\norm{\Zplus\Qplusp} \le \left (1 - \frac{\mu}{2}\norm{\Zt\Qtp}^2 + \frac{ \mu \sglmin(X)}{3000} \right) \norm{\Zt\Qtp}  + 2\mu\sqrt{\norm{X}}\norm{\tilZtT\Zt \Qtp}.
\end{equation} 
We obtain that
\begin{align*}
 \norm{\Zplus\Qplusp} 
 \stackrel{(a)}{\le} & \left (1 - \frac{\mu}{2}\norm{\Zt\Qtp}^2 + \frac{ \mu   \sglmin(X)}{3000} + \frac{ 2\mu \hat{c}_4 \hat{c}_5 \norm{X} }{\kappa^{3}} \right) \left( 1+ \frac{\mu}{1500} \sglmin(X)  \right)^{t-t_1} \norm{\Ztone Q_{t_1,\bot}}\\
 \stackrel{(b)}{\le} & \left( 1+ \frac{\mu}{1500} \sglmin(X)  \right)^{t+1-t_1} \norm{\Ztone Q_{t_1,\bot}}.
\end{align*}
Inequality $(a)$ is due to induction hypotheses \eqref{ineq:phase2hyp2} and \eqref{ineq:phase2hyp_balanc2}.
Inequality $(b)$ follows from choosing the absolute constants $\hat{c}_4$ and $ \hat{c}_5$ to be small enough.
This implies inequality \eqref{ineq:phase2hyp2} for $t+1$.


Next, we observe that the assumptions of Lemma \ref{lemma:anglecontrol} are satisfied and hence it follows that 
\begin{equation}\label{ineq:intern111}
\begin{split}
    &\norm{\LAPT\P{\Zplus Q_{t+1} }} \le \\
&\left(1 - \frac{1}{4}\mu\sglmin \left(X\right) \right)\norm{\LAPT\P{\Zt\Qt}} 
+  2 \mu  \sqrt{\norm{X}} \norm{ \tilZtT \P{\Zt\Qt} }
+C\mu\frac{\sqrt{\norm{X}}\norm{\tilZtT\Zt \Qtp}}{\sglmin\left(\Zt\Qt\right)} + C \mu \norm{\Deltat} +  C\mu^2\norm{X}^2.
\end{split}
\end{equation}
In order to proceed, we note that 
\begin{align}
    \frac{\norm{\tilZtT\Zt \Qtp}}{\sglmin\left(\Zt\Qt\right)} 
    \le & \frac{ \norm{\tilZtT\Zt \Qtp}}{\sglmin\left( \LAT \Zt\right)} \nonumber \\
    \stackrel{(a)}{\le} & \frac{ \hat{c}_4 \hat{c}_5 \sqrt{ \norm{X} } \left( 1+ \frac{\mu}{1500} \sglmin(X)  \right)^{t-t_1} \norm{\Ztone Q_{t_1,\bot}}}{ \kappa^{3} \left(1 + \frac{1}{8}\mu\sglmin(X)  \right)^{t-t_1} \sglmin \left( \LAT \Ztone \right)} \nonumber\\
    \le & \frac{ \hat{c}_4 \hat{c}_5 \sqrt{ \norm{X}  }   \norm{\Ztone Q_{t_1,\bot}}}{ \kappa^{3} \sglmin \left( \LAT \Ztone \right)} \nonumber \\
    \le & \frac{ \hat{c}_4 \hat{c}_5 \sqrt{ \norm{X}  }   \norm{\Ztone Q_{t_1,\bot}}}{ \kappa^{3} \sglmin \left( \LAT P_{\Ztone Q_{t_1}} \right) \sglmin \left( \Ztone Q_{t_1} \right) } \nonumber \\
    \stackrel{(b)}{\le} & \frac{4 \hat{c}_4 \hat{c}_5 \sqrt{ \norm{X}  } }{ \kappa^{3}}, \label{ineq:aux23}
\end{align}
where in inequality $(a)$ we have used the induction hypotheses \eqref{ineq:phase2hyp1} and \eqref{ineq:phase2hyp_balanc2}.
Inequality $(b)$ is due to the assumption \eqref{ineq:phase2assump1} and the induction hypothesis \eqref{ineq:phase2hyp3}.
Combining this inequality chain with inequality \eqref{ineq:intern111} we obtain that
\begin{align*}
    &\norm{\LAPT\P{\Zplus Q_{t+1}}}  \\
    \le &\left(1 - \frac{1}{4}\mu\sglmin \left(X\right) \right)\norm{\LAPT\P{\Zt\Qt}} 
    +  2 \mu  \sqrt{\norm{X}} \norm{ \tilZtT \P{\Zt\Qt} }
    +\frac{4C \hat{c}_4 \hat{c}_5 \mu  \norm{X} }{ \kappa^{3}}+ C \mu \norm{\Deltat} +  C\mu^2\norm{X}^2\\
    \stackrel{(a)}{\le} &\left(1 - \frac{1}{4}\mu\sglmin \left(X\right) \right) \frac{c_4}{\kappa^2} 
    +  \frac{ 2 \mu \hat{c}_4  \hat{c}_6 \norm{X} }{\kappa^3}
    +\frac{4 \mu C \hat{c}_4 \hat{c}_5    \norm{X} }{ \kappa^{3}}+ \frac{11 C \hat{c}_1 \mu \sglmin \left(X\right) }{\kappa^2}   + \frac{ \mu C \hat{c}_2  \sglmin (X)}{\kappa^3} \\
    = & \left( \hat{c}_4 - \mu \left( \frac{\hat{c}_4}{4} - 2 \hat{c}_4  \hat{c}_6 - 4C \hat{c}_4 \hat{c}_5 - 11 C \hat{c}_1 - \frac{ C \hat{c}_2}{\kappa}  \right) \sglmin \left(X\right)  \right) \frac{1}{ \kappa^2 } \\
    \stackrel{(b)}{\le} & \frac{ \hat{c}_4 }{\kappa^2}.
\end{align*}
In inequality $(a)$ we have used the induction hypotheses \eqref{ineq:phase2hyp3} and \eqref{ineq:phase2hyp_balanc3}, inequality \eqref{ineq:phase2aux11}, and the assumption that $\mu \le \frac{ \hat{c}_2 }{\kappa^4 \norm{X}}$.
In inequality $(b)$ we used that the constants $\hat{c}_1$ and $\hat{c}_2$ are chosen small enough compared to $\hat{c}_4$ and, moreover, the constants $\hat{c}_5$ and $\hat{c}_6$ are chosen small enough (compared to $1$).
This shows the \eqref{ineq:phase2hyp3} for $t+1$.

Next, recall from \eqref{ineq:phase2aux4} that $\norm{\Zt \Qtp} \le  \frac{ \hat{c}_3^{4/5}  \sqrt{ \sglmin \left(X\right) } }{ \kappa^{7/2} k^{4/5}} $, which allows us to apply Lemma \ref{lemma:normcontrolled}, which yields $\norm{\Zt}\le 2 \sqrt{\norm{X}} $.
This verifies \eqref{ineq:phase2hyp4} for $t+1$.

Moreover, we note that it follows from Lemma \ref{lemma:balancedbase} that 
\begin{equation*}
    \norm{\tilZplusT \Zplus} \le \norm{\tilZtT \Zt} + 400 \mu^2 \norm{X}^3.
\end{equation*}
Inserting this into the induction hypothesis \eqref{ineq:phase2hyp_balanc1} and using the assumption \eqref{ineq:phase2assump_balanc1}, we obtain that 
\begin{equation*}
    \norm{ \tilZplusT \Zplus} \le  \norm{ \widetilde{Z}_{t_1}^T Z_{t_1} } + 400\mu^2 \left( t+1-t_1 \right) \norm{X}^3    \le  \frac{ \hat{c}_7 \norm{X} }{\kappa^{4}},
\end{equation*}
which proves inequality \eqref{ineq:phase2hyp_balanc1} for $t+1$.
We obtain from Lemma \ref{lemma:balancednessperp} that
     \begin{align*}
        &\norm{\tilZplus^T \Zplus \Qplusp} \le  \norm{ \tilZtT \Zt\Qtp } \\
        &+ C\mu \left(   \left( \norm{\LAPT\P{\Zt\Qt}} + \mu\norm{X} \right)\beta  +  \mu \norm{X}^2 \right)  \sqrt{\norm{ X }} \norm{ \Zt\Qtp }+ 8 \mu \beta \norm{\Zt \Qtp}^2,
    \end{align*}
 where we have set $\beta := \norm{\LAPT \PZQ}  \norm{X}  + \norm{ \Zt \Qtp  }^2 + \norm{\Deltat}$. 
It follows from induction hypothesis \eqref{ineq:phase2hyp3}, inequality \eqref{ineq:phase2aux4}, and inequality \eqref{ineq:phase2aux11} that 
\begin{align*}
\beta
\le \frac{ \left( \hat{c}_4 +\hat{c}_3^{8/5} + 11 \hat{c}_1 \right) \sglmin (X)}{\kappa} .
\end{align*}
We obtain that
\begin{align*}
    &\norm{\tilZplus^T \Zplus \Qplusp} \le  \norm{ \tilZtT \Zt\Qtp } \\
    &+ C\mu \left( \left(  \hat{c}_4 +\hat{c}_3^{8/5} + 11 \hat{c}_1 \right)  \left( \norm{\LAPT\P{\Zt\Qt}} + \mu\norm{X} \right) \frac{\sglmin (X) }{\kappa}+  \mu \norm{X}^2 \right)  \sqrt{\norm{ X }} \norm{ \Zt\Qtp } \\ 
    &+ \frac{ 8 \left( \hat{c}_4 +\hat{c}_3^{8/5} + 11 \hat{c}_1 \right) \mu \sglmin (X)  \norm{\Zt \Qtp}^2}{\kappa} \\
    \stackrel{(a)}{\le} & \norm{ \tilZtT \Zt\Qtp } + C\mu \left( \left( \hat{c}_4 +\hat{c}_3^{8/5} +11 \hat{c}_1 \right)  \left( \frac{ \hat{c}_4}{\kappa^2} + \frac{ \hat{c}_2}{\kappa^4}\right) \frac{\sglmin (X) }{\kappa}+  \frac{ \hat{c}_2 \norm{X}}{\kappa^4} \right)  \sqrt{\norm{ X }} \norm{ \Zt\Qtp }\\ 
    & + 8 \hat{c}_3^{4/5} \left( \hat{c}_4 + \hat{c}_3^{8/5} + 11 \hat{c}_1 \right) \mu \frac{ \sglmin (X) \sqrt{\norm{X}}  \norm{\Zt \Qtp}}{\kappa^{9/2} k^{4/5} }\\
    \le & \norm{ \tilZtT \Zt\Qtp }  + \mu \left( C \left( \hat{c}_4+ \hat{c}_3^{8/5} + 11 \hat{c}_1 \right) \left( \hat{c}_4 +\hat{c}_2\right) + C \hat{c}_2 + 8 \hat{c}_3^{4/5} \left( \hat{c}_4 + \hat{c}_3^{8/5} +11 \hat{c}_1  \right)  \right) \\
    & \cdot \sglmin (X)  \frac{\sqrt{\norm{X}}  \norm{\Zt \Qtp}}{\kappa^3}  \\
    \stackrel{(b)}{\le} & \norm{ \tilZtT \Zt\Qtp }  +  \hat{c}_4 \hat{c}_5 \mu  \sglmin (X)  \frac{ \sqrt{\norm{X}}  \norm{ \Zt \Qtp}}{1500 \kappa^3}  \\
    \stackrel{(c)}{\le} &  \frac{ \hat{c}_4 \hat{c}_5 \sqrt{\norm{X}}}{\kappa^3} \left( 1+ \frac{\mu}{1500} \sglmin(X)  \right)^{t+1-t_1} \norm{\Ztone Q_{t_1,\bot}}.
\end{align*}
Inequality $(a)$ follows from inequalities \eqref{ineq:phase2hyp3}, \eqref{ineq:phase2aux4}, and the assumption $ \mu \le \frac{ \hat{c}_2}{\norm{X} \kappa^4}$.
Inequality $(b)$ follows from the fact that the constants $ \hat{c}_1$, $\hat{c}_2$, and  $ \hat{c}_3$ are chosen small enough (compared to $\hat{c}_4 \hat{c}_5$) and that $\hat{c}_4$ is chosen small enough compared to $\hat{c}_5$.
Inequality $(c)$ is due to inequalities \eqref{ineq:phase2hyp2} and \eqref{ineq:phase2hyp_balanc2}.
This shows \eqref{ineq:phase2hyp_balanc2} for $t+1$.

In order to prove \eqref{ineq:phase2hyp_balanc3} for $t+1$ we note that from Lemma \ref{ref:balancednessangle} it follows that 
    \begin{align*}
        &\norm{ \tilZplus^T P_{\Zplus \Qplus}  } \\
        \le & \left( 1-  \frac{\mu}{4} \sglmin \left( X \right) \right) \norm{\tilZt^T \PZQ } +4 \mu \norm{X} \norm{  \Zt \Qtp  }
        + 2\mu \norm{ \tilZt^T \Zt  } \sqrt{\norm{X}}+ \frac{\mu  \norm{\tilZt^T \Zt\Qtp}  \sglmin (X) }{ \sglmin \left( \Zt\Qt \right)  } \\
        &  +800 \mu^2 \norm{X}^{5/2}\\
        \stackrel{(a)}{\le} & \left( 1-  \frac{\mu}{4} \sglmin \left( X \right) \right) \frac{ \hat{c}_4 \hat{c}_6 \sqrt{\norm{X} } }{\kappa^{3}} +  \frac{ 4\mu \hat{c}_3^{4/5} \sqrt{ \sglmin \left(X\right) } \norm{X} }{ \kappa^{7/2} k^{4/5}}
        + 2\mu  \frac{ \hat{c}_7 \norm{X}^{3/2}  }{\kappa^{4}} + \frac{4 \mu  \hat{c}_4 \hat{c}_5 \sqrt{ \norm{X} } \sglmin (X) }{ \kappa^3 } \\
        &  + 800 \mu \hat{c}_2 \frac{\sqrt{\norm{X}} \sglmin (X) }{\kappa^{3}}\\
        = & \left( \hat{c}_4 \hat{c}_6   - \mu \left(  \frac{  \hat{c}_4 \hat{c}_6}{4} -  \frac{ 4  \hat{c}_3^{4/5} }{ k^{4/5}} -  2  \hat{c}_7-  4   \hat{c}_4 \hat{c}_5  - 800 \hat{c}_2 \right)  \sglmin (X)   \right) \frac{\sqrt{\norm{X}}}{\kappa^3}\\
        \stackrel{(b)}{\le} & \frac{ \hat{c}_4 \hat{c}_6 \sqrt{\norm{X}}}{ \kappa^{3}},
    \end{align*}
where in inequality $(a)$ we have used the induction hypotheses \eqref{ineq:phase2hyp_balanc1} and \eqref{ineq:phase2hyp_balanc3}, the assumption $\mu \le \frac{\hat{c}_2}{\kappa^4 \norm{X}}$, and inequalities \eqref{ineq:phase2aux4} and \eqref{ineq:aux23}.
Inequality $(b)$ follows from the fact that the constants $\hat{c}_2$, $\hat{c}_3$, and $\hat{c}_7$ are chosen small enough compared to $\hat{c}_4 \hat{c}_5 \ll \hat{c}_4 \hat{c}_6$ and $ \hat{c}_5$ is chosen small enough compared to $\hat{c}_6$.

Hence, we have verified the inequalities \eqref{ineq:phase2hyp1},  \eqref{ineq:phase2hyp2},  \eqref{ineq:phase2hyp3},  \eqref{ineq:phase2hyp4}, and the three conditions regarding the imbalance matrix (\eqref{ineq:phase2hyp_balanc1}, \eqref{ineq:phase2hyp_balanc2}, and \eqref{ineq:phase2hyp_balanc3})  for $t+1$. Thus, the induction step is completed.\\



In order to complete the proof it remains to show that inequalities \eqref{ineq:phase2final1}--\eqref{ineq:phase2final_balanc3} hold for $t=t_2$.
For that, we set 
\begin{equation*}
    \gamma := \left( 1 +\frac{\mu \sglmin (X)}{1500} \right)^{t-t_1} \norm{\Ztone Q_{t_1,\bot}}.
\end{equation*}
Note that upper bound in line \eqref{ineq:gammabound} follows directly from \eqref{ineq:phase2aux5} (whereas the lower bound is immediate).
Next, we note that for $t=t_2$ inequality \eqref{ineq:phase2final1} follows directly from the definition of $t_2$.
Inequality \eqref{ineq:phase2final2} is due to \eqref{ineq:phase2aux5} and the definition of $\gamma$.
Moreover, inequality \eqref{ineq:phase2final3}, respectively inequality \eqref{ineq:phase2final4}, follow directly from \eqref{ineq:phase2hyp3}, respectively \eqref{ineq:phase2hyp4}, applied to $t=t_2$.
Analogously, inequalities \eqref{ineq:phase2final_balanc1} and \eqref{ineq:phase2final_balanc3} regarding the imbalance matrix follow from \eqref{ineq:phase2hyp_balanc1} and \eqref{ineq:phase2hyp_balanc3} with $t=t_2$.
Inequality \eqref{ineq:phase2final_balanc2} follows from \eqref{eqref:phase2aux12} with $t=t_2$ and the definition of $\gamma$. 


\end{proof}