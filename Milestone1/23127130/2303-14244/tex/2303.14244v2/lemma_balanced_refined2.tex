\subsection{Proof of Lemma \ref{ref:balancednessangle}: Controlling $\norm{\tilZt^T \PZQ } $}\label{sec:balancednessangle}




\begin{proof}[Proof of Lemma \ref{ref:balancednessangle}]
    Recall that 
    \begin{align*}
        \Zplus &= \left(\Id - \mu M_t \right) \Zt, \\
        \tilZplus &=  \left(\Id + \mu M_t \right) \tilZt,
    \end{align*}
    where 
    \begin{equation*}
        M_t=  \Zt \ZtT - \tilZt \tilZtT -\symA + \Deltat.
    \end{equation*}
    Analogously, as in the proof of Lemma \ref{lemma:anglecontrol} we set 
    \begin{equation*}
        H:=\left( \I - \mu M_t \right) \left(\I + K\right) \P{\Zt\Qt},
    \end{equation*}
    where
    \begin{equation*}
        K:=\Zt\Qtp\QtpT\Qplus(\P{\Zt\Qt}^T\Zt\Qt\QtT\Qplus)^{-1}\P{\Zt\Qt}^T.
    \end{equation*}
    In the proof of Lemma \ref{lemma:anglecontrol} we have seen that $P_{\Zplus \Qplus}$ has the same column span as $H$. 
    It follows that
    \begin{equation*}
        \norm{\tilZplusT P_{\Zplus \Qplus}}= \norm{ \tilZplusT H \left(H^TH\right)^{-1/2}}.    
    \end{equation*} 
    We calculate that 
    \begin{align*}
        \tilZplusT P_{\Zplus \Qplus}
        =& \tilZt^T \left( \Id + \mu M_t \right) \left( \Id - \mu M_t \right)  \left(\I + K\right) \PZQ \left(H^TH\right)^{-1/2} \\
        =& \underbrace{ \tilZt^T \PZQ \left(H^TH\right)^{-1/2}}_{=:(I)} 
        + \underbrace{ \tilZt^T K \PZQ \left(H^TH\right)^{-1/2}}_{=:(II)}
         -\mu^2  \underbrace{ \tilZt M_t^2 \left( \Id +K \right) \PZQ  \left(H^TH\right)^{-1/2}}_{=:(III)}.
    \end{align*}
    We are going to estimate the spectral norm of the summands $(I)$, $(II)$, and $(III)$ individually.
    Before that, we will first derive bounds for $\norm{M_t} $, $\norm{ \QtpT\Qplus  }$, $\norm{K} $, and $ \sglmin \left(H\right) $. 
    First, we note that 
    \begin{align}
     \norm{M_t}
    \le & \norm{X}+ \norm{\Zt \ZtT} + \norm{\tilZt \tilZtT} + \norm{\Deltat} \nonumber \\
    = & \norm{X}+2 \norm{\Zt}^2 + \norm{\Deltat}\nonumber  \\
    \le & 10 \norm{X}, \label{ineq:balanceintern1}
    \end{align}
    where in the last line we used the assumptions $ \norm{\Zt} \le 2 \sqrt{\norm{X}} $ and $\norm{\Deltat} \le c \sglmin \left( X\right) $. Next, we derive an upper bound for $\norm{ \QtpT\Qplus  }$. It follows from Lemma \ref{lemma:aux1} that
    \begin{align}
        \norm{ \QtpT\Qplus  } \le&  \mu\left(\norm{\Zt\Qt}\norm{\Zt\Qtp}+ 40 \mu \norm{X} \norm{\Zt\Qt}^2 \right)\norm{\LAPT\P{\Zt\Qt}} + 4\mu\frac{\sqrt{\norm{X}}\norm{\tilZtT\Zt \Qtp}}{\sglmin\left(\Zt\Qt\right)} + 4 \mu \norm{\Deltat} \nonumber \\
        \stackrel{(a)}{\le}&  \mu\left( 2 \sqrt{\norm{X}} \norm{\Zt\Qtp}+ 160 \mu \norm{X}^2  \right) \norm{\LAPT \PZQ } + 4 \mu\frac{\sqrt{\norm{X}}\norm{\tilZtT\Zt \Qtp}}{\sglmin\left(\Zt\Qt\right)} + 4 \mu \norm{\Deltat} \nonumber\\
        \stackrel{(b)}{\le}& C_3 \mu c \sglmin \left(X\right), \label{ineq:balanceintern2additional}
    \end{align}
    where in inequality $(a)$ we used the assumption $\norm{\Zt} \le 2 \sqrt{\norm{X}}$
    and in inequality $(b)$ we used the assumptions $ \norm{ \Zt \Qtp} \le c \sqrt{\sglmin (X)} $, $\norm{\Deltat} \le c \sglmin \left( X\right)  $, $\norm{\tilZtT\Zt \Qtp} \le \frac{c}{\kappa} \sglmin \left(\Zt \Qt \right) \sqrt{\norm{X}} $, $ \mu \le \frac{c}{\norm{X} \kappa}  $, and $\norm{\LAPT \PZQ } \le \frac{c}{\kappa} $,
    where $C_3>0$ is an absolute constant chosen large enough.
    Then we derive an upper bound for $\norm{K} $. For that, we compute that
    \begin{align}
    \norm{K} 
    &\le \frac{ \norm{ \Zt\Qtp } \norm{ \QtpT \Qplus }  }{ \sglmin \left( \P{\Zt\Qt}^T\Zt\Qt \right) \sglmin \left( \QtT\Qplus  \right) } \nonumber\\
    &\stackrel{(a)}{\le} \frac{ 2 \norm{ \QtpT\Qplus }  }{ \sglmin \left( \QtT\Qplus  \right) } \nonumber\\
    %&= \frac{ 2 \norm{ \QtpT\Qplus }  }{ \sqrt{1- \norm{\QtpT\Qplus}} }.
    & \stackrel{(b)}{\le} 4 \norm{ \QtpT\Qplus  } \nonumber\\
    & \stackrel{(c)}{\le} C_1 \mu c \sglmin \left(X\right)\stackrel{(d)}{\le} \frac{1}{8}. \label{ineq:balanceintern2}
    \end{align}
    In inequality $(a)$ we used the assumption $ \norm{ \Zt \Qtp} \le 2  \sglmin \left( \Zt \Qt \right) $. In inequality $(b)$ we have used $  \sglmin \left( \QtT\Qplus  \right) \ge 1/2 $, which follows from Lemma \ref{lemma:aux1}.
    %(Since we can apply Lemma \ref{lemma:angle:auxiliarylemma1} the assumption $\sglmin \left( \LAT \Zplus \right) \ge \frac{1}{2} \sglmin \left( \Zt \Qt \right) $ is fulfilled.)
     In inequality $(c)$ we used inequality \eqref{ineq:balanceintern2additional}, where $C_1: = 4C_3$. Inequality $(d)$ follows from our assumption on the step size $\mu$ and by choosing the absolute constant $c>0$ small enough. In order to control $\sglmin \left( H \right) $ we observe that
    \begin{align}
       \sglmin \left( H \right) & \ge \left( 1 - \mu \norm{M_t} \right) \left( 1- \norm{K}  \right)  \norm{\PZQ} \nonumber  \\
        &= \left( 1 - \mu \norm{M_t} \right) \left( 1- \norm{K}  \right) \nonumber \\
        &\ge 1/2, \label{ineq:balanceintern3}
    \end{align}
    where in the last inequality we used inequalities \eqref{ineq:balanceintern1}, \eqref{ineq:balanceintern2}, and our assumption on the step size $\mu$.\\ 

    \noindent Now we are in a position to derive upper bounds for the spectral norms of the terms $(I)$, $(II)$, and $(III)$.\\

    \noindent\textbf{Estimation of (I):}
    Our goal is to derive an upper bound for the spectral norm of 
    \begin{equation*}
    (I)=\tilZt^T \PZQ \left(H^TH\right)^{-1/2}.
    \end{equation*}
    For that, we compute first
    \begin{align*}
        H^T H
        =& \P{\Zt\Qt}^T \left(\I + K^T\right) \left( \I - \mu M_t \right)^2 \left(\I + K\right) \P{\Zt\Qt},
    \end{align*}
    which can be rewritten as 
    \begin{align*}
        H^T H =& \P{\Zt\Qt}^T \left(\I - 2\mu M_t +K +K^T + F \right) \P{\Zt\Qt} \\
        =& \I - 2  \mu \P{\Zt\Qt}^T M_t  \PZQ +  \P{\Zt\Qt}^T \left( K + K^T +F \right) \PZQ, %+  \P{\Zt\Qt}^T K^T  \P{\Zt\Qt} +  \P{\Zt\Qt}^T F  \P{\Zt\Qt}  
    \end{align*}
    where
    \begin{align}
        F:=& \left(\I + K^T\right) \left( \I - \mu M_t \right)^2 \left(\I + K\right) -(\I - 2\mu M_t +K +K^T)\nonumber \\
        =& \mu^2 M_t^2 - 2 \mu K^T M_t - 2 \mu M_t K +K^T K+  \mu^2 K^T M_t^2 - 2\mu K^TM_tK + \mu^2 M_t^2K + \mu^2 K^TM_t^2 K. \label{equ:definitionF}
    \end{align}
    Since $\norm{K} \le 1/8 \le 1 $ due to inequality \eqref{ineq:balanceintern2} we obtain that 
    \begin{align}
        \norm{F} \le& \mu^2 \norm{M_t}^2 + 4 \mu \norm{M_t} \norm{K} + \norm{K}^2 + 2 \mu^2 \norm{K} \norm{M_t}^2 + 2\mu \norm{M_t} \norm{K}^2  + \mu^2 \norm{M_t}^2 \norm{K}^2 \nonumber \\
        \le & \mu^2 \norm{M_t}^2 + 4 \mu \norm{M_t} \norm{K} + \norm{K}^2 + 2 \mu^2 \norm{M_t}^2 + 2\mu \norm{M_t} \norm{K}  + \mu^2 \norm{M_t}^2\\
        \stackrel{(a)}{\le} & 16 \mu^2 \norm{M_t}^2 + 13 \norm{K}^2 \nonumber \\
        \stackrel{(b)}{\le} & 16 \mu^2 \norm{M_t}^2 + 13 \norm{K} \nonumber  \\
        \stackrel{\eqref{ineq:balanceintern1},\eqref{ineq:balanceintern2} }{\le} & 1600 \mu^2 \norm{X}^2 + 13  C_1 \mu \frac{ c \sglmin \left(X\right) }{\kappa} \nonumber \\
        \stackrel{(c)}{\le} & C_2 \mu c \sglmin \left(X\right) \label{ineq:balancingintern16} \\
        \stackrel{(d)}{\le } & 1, \label{ineq:balancingintern17}
    \end{align}
    where inequality $(a)$ follows from the elementary inequality $ab \le \frac{a^2}{2} + \frac{b^2}{2} $.
    In inequality $(b)$ we used again that $\norm{K} \le 1$, which is due to inequality \eqref{ineq:balanceintern2}. Inequality $(c)$ follows from the assumption $ \mu \le \frac{c}{ \norm{X} \kappa} $ and by choosing the constant $C_2>0$ large enough.
    Inequality $(d)$ follows from our assumption on the step size $\mu$ and by choosing the absolute constant $c>0$ small enough.
    Since $ \norm{K}\le \frac{1}{8}$, $\norm{M_t}\le 10 \norm{X}$, by our assumption $ \mu \le \frac{c}{\kappa \norm{X}}$ we can apply Lemma \ref{MatrixTaylor} and obtain that 
    \begin{align*}
        \left( H^T H \right)^{-1/2} =\Id + \mu \PZQ^T M_t  \PZQ - \frac{1}{2} \P{\Zt\Qt}^T \left( K + K^T +F \right) \PZQ + G, 
    \end{align*}
    where $G$ is a symmetric matrix, which satisfies 
    \begin{equation}\label{ineq:balancingGdefinition}
        \norm{G} \le 3 \norm{ -2  \mu \PZQ^T M_t  \P{\Zt\Qt} +  \P{\Zt\Qt}^T \left( K + K^T +F \right) \PZQ }^2.
    \end{equation}
    It follows that 
    \begin{align*}
        \tilZt^T \PZQ \left( H^T H\right)^{-1/2}
       =& \tilZt^T \PZQ  \left( \I + \mu   \P{\Zt\Qt}^T M_t  \PZQ \right) \\
       & - \frac{1}{2} \tilZt^T \PZQ  \PZQ^T \left( K + K^T +F \right) \PZQ + \tilZt^T \PZQ G.
    \end{align*}
    In particular, we obtain that 
    \begin{align}\label{ineq:internaandb}
        \norm{  \tilZt^T \PZQ \left( H^T H\right)^{-1/2} }
        \le \underbrace{ \norm{  \tilZt^T \PZQ  \left( \I + \mu   \P{\Zt\Qt}^T M_t  \PZQ \right) }}_{=:(\square)}
        + \frac{1}{2} \underbrace{ \norm{ \tilZt^T \PZQ } \left( 2 \norm{K} + \norm{F}  +\norm{G}   \right)}_{=:( \square \square )}  .
    \end{align}
    We are going to estimate the two summands individually.
    In order to estimate $( \square )$ we first compute that 
    \begin{align*}
        &\tilZt^T \PZQ  \left( \I + \mu   \P{\Zt\Qt}^T M_t  \PZQ \right)\\
        =&\tilZt^T \PZQ  \left( \I - \mu   \P{\Zt\Qt}^T \left( \symA- \Zt \ZtT + \tilZt \tilZtT - \Deltat \right)  \PZQ \right) \\
        =&\tilZt^T \PZQ  \left( \I - \mu   \P{\Zt\Qt}^T \left( \symA + \tilZt \tilZtT - \Deltat \right)  \PZQ \right)+ \mu \tilZt^T \PZQ \PZQT \Zt \ZtT  \PZQ\\
        =&\tilZt^T \PZQ  \left( \I - \mu   \P{\Zt\Qt}^T \left( \symA + \tilZt \tilZtT - \Deltat \right)  \PZQ \right)+ \mu \tilZt^T \Zt \ZtT  \PZQ-\mu \tilZt^T \PZQperp \PZQperpT \Zt \ZtT  \PZQ.
    \end{align*}
    It follows that 
    \begin{align}
        (\square) =& \norm{\tilZt^T \PZQ  \left( \I + \mu   \P{\Zt\Qt}^T M_t  \PZQ \right)} \nonumber \\
        \le& \norm{ \tilZt^T \PZQ   }  \norm{  \I - \mu   \P{\Zt\Qt}^T \left( \symA + \tilZt \tilZtT - \Deltat  \right) \PZQ } + \mu \norm{ \tilZt^T \Zt  } \norm{\Zt} + \mu \norm{\tilZt} \norm{ \PZQperpT \Zt  } \norm{\Zt} \nonumber \\
        \stackrel{(a)}{\le}& \norm{ \tilZt^T \PZQ   }  \norm{  \I - \mu   \P{\Zt\Qt}^T \left( \symA + \tilZt \tilZtT - \Deltat  \right) \PZQ } + 2\mu \norm{ \tilZt^T \Zt  }\sqrt{\norm{X}} + 4 \mu \norm{X} \norm{ \PZQperpT \Zt  } \nonumber \\
        \stackrel{(b)}{\le}& \norm{ \tilZt^T \PZQ   }  \norm{  \I - \mu   \P{\Zt\Qt}^T \left( \symA + \tilZt \tilZtT - \Deltat  \right) \PZQ } + 2\mu \norm{ \tilZt^T \Zt  }\sqrt{\norm{X}} + 4 \mu \norm{X} \norm{  \Zt \Qtp  }, \label{ineq:internbalancing14}
    \end{align}
    where in inequality (a) we used the assumption $\norm{\tilZt} = \norm{\Zt} \le 2\sqrt{\norm{X}} $ and in inequality (b) we used the fact that $ \norm{ \PZQperpT \Zt  } =  \norm{ \PZQperpT \Zt \Qtp  } \le \norm{ \Zt \Qtp  } $. Furthermore, we have that 
    \begin{align}
    &\norm{ \I - \mu   \P{\Zt\Qt}^T \left( X + \tilZt \tilZtT - \Deltat  \right) \PZQ } \nonumber \\
    =& \norm{ \I - \mu   \P{\Zt\Qt}^T \left( \LA \Sigma_X \LAT - \tilLA \Sigma_X \tilLAT + \tilZt \tilZtT - \Deltat  \right) \PZQ } \nonumber\\
    \le &  \norm{  \I - \mu   \P{\Zt\Qt}^T \LA \Sigma_X \LAT \PZQ - \mu \PZQT \tilZt \tilZtT \PZQ  }\nonumber\\
    &+ \mu  \norm{  \PZQT \tilLA \Sigma_X \tilLAT \PZQ  }
    + \mu  \norm{  \PZQT \Deltat \PZQ  }\nonumber\\
    \le & \norm{  \I - \mu   \P{\Zt\Qt}^T \LA \Sigma_X \LAT \PZQ  }
    + \mu  \norm{  \PZQT \tilLA \Sigma_X \tilLAT \PZQ  } + \mu  \norm{  \PZQT \Deltat \PZQ }\nonumber\\
    \le &  1- \mu \sglmin \left(    \P{\Zt\Qt}^T \LA \Sigma_X \LAT \PZQ \right)
    + \mu  \norm{  \PZQT \tilLA \Sigma_X \tilLAT \PZQ  } + \mu  \norm{   \Deltat }. \label{ineq:internbalancing13}
    \end{align}
    We observe that 
    \begin{align}\label{ineq:internbalancing11}
    \sglmin \left(    \P{\Zt\Qt}^T \LA \Sigma_X \LAT \PZQ \right)    
    \ge \sglmin \left(  \P{\Zt\Qt}^T \LA \right)^2 \sglmin \left( \Sigma_X \right)
    \ge \frac{3}{4} \sglmin \left( X \right),
    \end{align}
    where we have used the assumption $ \norm{\LAPT \PZQ} \le \frac{c}{\kappa} $.
    Since $ \norm{ \tilLAT \PZQ  } \le    \norm{\LAPT \PZQ}  $, the same assumption also implies 
    \begin{align}\label{ineq:internbalancing12}
        \norm{  \PZQT \tilLA \Sigma_X \tilLAT \PZQ  } \le \norm{  \PZQT \tilLA } \norm{ \Sigma_X } \le \frac{1}{8} \sglmin \left(X\right) .
    \end{align}
    Inserting inequalities \eqref{ineq:internbalancing11} and \eqref{ineq:internbalancing12} into \eqref{ineq:internbalancing13} and using the assumption $ \norm{ \Deltat} \le c \sglmin \left(X\right) $ it follows that 
    \begin{align}
    \norm{ \I - \mu   \P{\Zt\Qt}^T \left( \symA  + \tilZt \tilZtT - \Deltat  \right) \PZQ } 
    \le 1- \frac{\mu}{2} \sglmin \left( X \right). %\label{ineq:internbalancing14} 
    \end{align}
    Inserting the above inequality into \eqref{ineq:internbalancing14}  we obtain that 
    \begin{align}
        (\square)\le \left( 1-  \frac{\mu}{2} \sglmin \left( X \right) \right) \norm{\tilZt^T \PZQ } 
        + 2\mu \norm{ \tilZt^T \Zt  } \sqrt{\norm{X}}+ 4 \mu \norm{X} \norm{  \Zt \Qtp  }. \label{ineq:balancing:upperbounda}
    \end{align}
    Next, we are going to estimate term $(\square \square)$ in \eqref{ineq:internaandb}.
    First, we note that it follows from inequality \eqref{ineq:balancingGdefinition} that 
    \begin{align*}
        \norm{G} 
        \le &3 \norm{ -2  \mu \PZQ^T M_t  \P{\Zt\Qt} +  \P{\Zt\Qt}^T \left( K + K^T +F \right) \PZQ }^2\\
        \le & 3 \left( 2\mu \norm{M_t} + 2 \norm{K} +\norm{F} \right)^2 \\
        \le & 24 \mu^2 \norm{M_t}^2 + 6 \left(  2 \norm{K} +\norm{F}  \right)^2,
    \end{align*}
    where in the last line we used the elementary inequality $ (a+b)^2 \le 2a^2 + 2b^2 $.
    It follows that 
    \begin{align}
        (\square \square) = & \norm{ \tilZt^T \PZQ } \left( 2 \norm{K} + \norm{F}  +\norm{G}   \right) \nonumber \\
        \le &  \norm{ \tilZt^T \PZQ } \left( 2 \norm{K} + \norm{F}  + 24 \mu^2 \norm{M}^2 + 6 \left(  2 \norm{K} +\norm{F}  \right)^2   \right) \nonumber \\
        \stackrel{\eqref{ineq:balanceintern1}}{\le} &  \norm{ \tilZt^T \PZQ } \left( 2 \norm{K} + \norm{F}  + 2400 \mu^2 \norm{X}^2 + 6 \left(  2 \norm{K} +\norm{F}  \right)^2   \right) \nonumber \\
        \le &  \norm{ \tilZt^T \PZQ } \left( 2 \norm{K} + \norm{F}  + 2400 \mu c \sglmin \left( X \right) + 6 \left(  2 \norm{K} +\norm{F}  \right)^2   \right), \label{ineq:internbalancing15} 
    \end{align}
    where in the last line we used the assumption $ \mu \le \frac{c}{\norm{X} \kappa} $.
    Inserting our bounds for $\norm{K}$ and $\norm{F}$ into \eqref{ineq:internbalancing15} and obtain that
    \begin{align}
        (\square \square)
        \stackrel{(i)}{\le} & \norm{ \tilZt^T \PZQ } \left( 2 \norm{K} + \norm{F}  + 2400 \mu c \sglmin \left( X \right) + 18 \left(  2 \norm{K} +\norm{F}  \right)   \right) \nonumber \\
        \stackrel{(ii)}{\le} & C_3\mu c \sglmin (X)\norm{ \tilZt^T \PZQ }. \label{ineq:balancing:upperboundb}
    \end{align}
    In inequality $(i)$ we used that $\norm{K} \le 1 $ and $\norm{F}\le 1$ which we have shown above.
    In inequality $(ii)$ we used inequalities \eqref{ineq:balanceintern2} and \eqref{ineq:balancingintern16}. 
    $C_3>0$ is an absolute constant chosen large enough.
    Inserting the upper bounds for $(\square)$ and $(\square\square)$ (inequalities \eqref{ineq:balancing:upperbounda} and \eqref{ineq:balancing:upperboundb}) into \eqref{ineq:internaandb}  we obtain that  
    \begin{align*}
        &\norm{(I)} \le (\square) + \frac{1}{2}(\square\square) \\
        \le & \left( 1-  \frac{\mu}{2} \sglmin \left( X \right) \right) \norm{\tilZt^T \PZQ } 
        + 2\mu \norm{ \tilZt^T \Zt  } \sqrt{\norm{X}}  +  4 \mu \norm{X} \norm{  \Zt \Qtp  }+  \frac{1}{2}C_3\mu c \sglmin (X)\norm{ \tilZt^T \PZQ }  \\
        \le & \left( 1-  \frac{\mu}{4} \sglmin \left( X \right) \right) \norm{\tilZt^T \PZQ } 
        + 2\mu \norm{ \tilZt^T \Zt  } \sqrt{\norm{X}} +4 \mu \norm{X} \norm{  \Zt \Qtp  } ,
    \end{align*}
    where the last line follows since the absolute constant $c>0$ has been chosen small enough.\\

    \noindent\textbf{Estimation of (II):}
    By inserting the definition of $K$ we obtain that
    \begin{align*}
        \tilZt^T K \PZQ \left(H^TH\right)^{-1/2} 
        &= \tilZt^T \Zt\Qtp\QtpT\Qplus(\P{\Zt\Qt}^T\Zt\Qt\QtT\Qplus)^{-1}\P{\Zt\Qt}^T \PZQ \left(H^TH\right)^{-1/2}\\
        &= \tilZt^T \Zt\Qtp\QtpT\Qplus \left(  \QtT\Qplus \right)^{-1} \left(\P{\Zt\Qt}^T\Zt\Qt \right)^{-1} \left(H^TH\right)^{-1/2}. 
    \end{align*}
    It follows that 
    \begin{equation}\label{ineq:balancingintern18}
    (II) = \norm{\tilZt^T K \PZQ \left(H^TH\right)^{-1/2} }
    \le  \frac{\norm{\tilZt^T \Zt\Qtp} \norm{ \QtpT\Qplus }}{\sglmin \left(  \QtT\Qplus  \right) \sglmin \left( \Zt\Qt \right) \sglmin \left( H\right)}.    
    \end{equation}
    In particular, using inequality \eqref{ineq:balanceintern2additional}, \eqref{ineq:balanceintern3} and $  \sglmin \left( \QtT\Qplus  \right) \ge 1/2 $, which follows from Lemma \ref{lemma:aux1}, we obtain that 
    \begin{align*}
        (II) \le & 4C_3 \frac{\mu c  \norm{\tilZt^T \Zt\Qtp}  \sglmin (X) }{ \sglmin \left( \Zt\Qt \right)  } \\
        \le & \frac{\mu \norm{\tilZt^T \Zt\Qtp}  \sglmin (X) }{ \sglmin \left( \Zt\Qt \right)  },
    \end{align*}
    where the last line follows since the absolute constant $c>0$ has been chosen small enough.

    \noindent\textbf{Estimation of (III):}
    We note that
    \begin{align*}
        \norm{(III)}
        &=\norm{ \tilZt M_t^2 \left( \Id + K \right) \PZQ  \left(H^TH\right)^{-1/2}  }\\
        & \le \frac{ \norm{\tilZt} \norm{M_t}^2 \left( 1+ \norm{K} \right)}{\sglmin \left( H \right)} \\
        &\stackrel{(a)}{\le}  400 \norm{\tilZt} \norm{X}^2 
        \stackrel{(b)}{\le}  800 \norm{X}^{5/2},
    \end{align*}
    where in $(a)$ we used inequalities \eqref{ineq:balanceintern1},  \eqref{ineq:balanceintern2}, and \eqref{ineq:balanceintern3} and in inequality $(b)$ we used the assumption $  \norm{\Zt} \le 2 \sqrt{\norm{X}} $ and that by symmetry $ \norm{\Zt}= \norm{\tilZt} $, see Lemma \ref{lemma:symmetry}.\\

    By combining the upper bounds for the spectral norms of $(I)$, $(II)$, and $(III)$ we conclude that
    \begin{align*}
        \norm{ \tilZplusT P_{\Zplus \Qplus}} \le & \norm{(I)} + \norm{(II)} + \norm{(III)}\\
        \le & \left( 1-  \frac{\mu}{4} \sglmin \left( X \right) \right) \norm{\tilZt^T \PZQ } 
        + 2\mu \norm{ \tilZt^T \Zt  } \sqrt{\norm{X}}+ 4 \mu \norm{X} \norm{  \Zt \Qtp  }\\
        &+  \frac{\mu \norm{\tilZt^T \Zt\Qtp}  \sglmin (X) }{ \sglmin \left( \Zt\Qt \right)  }   + 800 \mu^2  \norm{X}^{5/2}. 
    \end{align*}
\end{proof}