\section{Proof of the main result}\label{sec:proofmainresult}


\subsection{Outline of the proof}\label{sec:proofoutline}

In the following, we will give an outline of our proof.
As we will, see our proof consists of the following three steps.
\begin{enumerate}
    \item \textbf{Symmetrization:}
    First, we will show how the asymmetric matrix sensing problem in equation  can be equivalently reformulated as a matrix sensing problem with symmetric matrices.
    With this reformulation, we will be able to use some of the tools developed in \cite{stoger2021small} in the next two steps.
    However, while on a first glance this reformulated problem might resemble the one in \cite{stoger2021small}, there is a key difference.
    Namely, in \cite{stoger2021small}, it was assumed that both the ground truth and the learned matrices are positive semidefinite.
    Instead, in the scenario in this paper the ground truth matrix will have both positive and negative eigenvalues and we train one positive definite and one negative definite matrix.
    The dynamics of this two matrices are coupled with each other, which will lead to important changes in the proof as we will point out below.  
    \item \textbf{Decomposition of the learned matrices into signal and nuisance part:}
    In the second step, we will discuss how to decompose both training matrices into a signal and nuisance part.
    As in \cite{stoger2021small} the idea is that in the third step we then can show the signal matrices, which have rank $r$, will converge (approximately) to the positive definite part, respectively the negative definite part, of the ground truth matrix,
    whereas the nuisance terms will stay small.
    \item \textbf{Three-Phase Analysis:}
    To analyse the dynamics of gradient descent we will utilize the three-phase-analysis introduced in \cite{stoger2021small}.
    In the first phase, the \textit{alignment/spectral phase},
    we will show how, similar to a spectral initialization, the subspaces spanned by the signal parts of the learned get gradually more aligned with the subspace spanned by the ground truth matrix.
    In the second phase, the \textit{saddle avoidance phase}, we will show how the singular values of the signal parts are growing until they reach a certain basin of attraction of the ground truth matrices.
    In the third phase, the \textit{local convergence phase}, we will prove that the signal parts of the learned matrices converge linearly to the ground truth matrix.

    As already pointed out in the description of the first step,
    the key difficulty will be to deal with the fact that the dynamics of the two learned matrices are coupled with each other. 
    In Section \ref{sec:threephase} we will describe in detail how we deal with this.
\end{enumerate}

In the following, we give a proof the main result, Theorem \ref{theorem:main}.
\subsection{Symmetrization}

The first step in our proof is to show that the asymmetric model can be equivalently formulated as a symmetric model.
For that, we first define the symmetric measurement matrices $B_i \in \text{Sym}_{n_1+n_2}$ by
\begin{equation*}
    B_i:= \frac{1}{\sqrt{2}} \begin{pmatrix}
        0 & A_i \\
        A_i^T & 0 
    \end{pmatrix}
    \quad \text{for all } i \in \left[m\right]
\end{equation*}
and the associated linear measurement operator 
 $ \Bcal : \text{Sym}_{n_1 + n_2} \longrightarrow \R^m $ 
via
\begin{equation}\label{eq:prelimintern1}
   \left(  \Bcal \left( S \right) \right)_i := \innerproduct{B_i,S}
\end{equation}
for $ i \in \left[m\right] $ and for all symmetric matrices $S \in \text{Sym}_{n_1+n_2}$.
We define
\begin{equation*}
\symA :=  \begin{bmatrix}
     0 &  X\\
        X^T &  0
\end{bmatrix}
\end{equation*}
and 
\begin{equation}\label{equ:Ztdefinition}
\Zt := \frac{1}{\sqrt{2}}\begin{bmatrix}
 V_t\\
 W_t
\end{bmatrix}\quad \text{and}
\quad
\tilZt := \frac{1}{\sqrt{2}}\begin{bmatrix}
 V_t\\
- W_t
\end{bmatrix}
 \quad 
 \text{ for all } t.
\end{equation}
We observe via a straightforward calculation that 
\begin{equation*}
\frac{1}{\sqrt{2}} \Bcal \left(\symA\right)=  \Acal \left(X\right) 
\quad \text{ and } \quad 
\frac{1}{\sqrt{2}} \Bcal \left( \Zt \ZtT - \tilZt \tilZtT \right) = \Acal \left( V_t W_t^T \right)
\end{equation*}
The last two equalities imply that
\begin{equation*}
    \Loss \left( V_t, W_t  \right) = \frac{1}{2} \twonorm{\Acal \left(X- V_t W_t^T\right) }^2  =\frac{1}{4} \twonorm{\Bcal \left( \symA- \Zt \ZtT + \tilZt \tilZtT \right) }^2.
\end{equation*}
This motivates the definition of the following loss function
\begin{equation*}
\Losssym \left( Z, \tilde{Z} \right):=  \frac{1}{4} \twonorm{\Bcal \left( \symA- Z Z^T + \tilde{Z} \tilde{Z}^T  \right) }^2,
\end{equation*}
where $Z \in \R^{(n_1 + n_2) \times k} $ and $\tilde{Z} \in \R^{(n_1 + n_2) \times k}$.
We observe that
\begin{align*}
\nabla_Z \Losssym \left( Z, \tilde{Z} \right) &= - \left[ \left(   \Bcal^* \Bcal \right) \left( \symA- Z Z^T + \tilde{Z} \tilde{Z}^T  \right) \right] Z,\\
\nabla_{\tilde{Z}} \Losssym \left( Z, \tilde{Z} \right) &=  \left[ \left(   \Bcal^* \Bcal \right) \left( \symA- Z Z^T + \tilde{Z} \tilde{Z}^T  \right) \right] \tilde{Z}.
\end{align*}
Here, $\Bcal^*$ denotes the adjoint of the measurement operator $\Bcal$. 
It follows from a straightforward calculation that 
\begin{equation*}
\nabla_{Z} \Losssym \left( \Zt, \tilZt \right) =  \begin{pmatrix}
    \nabla_V \Loss \left( V_t,W_t \right) \\
    \nabla_W \Loss \left( V_t,W_t \right) 
\end{pmatrix}
\quad 
\text{and}
\quad
\nabla_{\tilde{Z}} \Losssym \left( \Zt, \tilZt \right)=  \begin{pmatrix}
    \nabla_V \Loss \left( V_t,W_t \right) \\
    -\nabla_W \Loss \left( V_t,W_t \right) 
\end{pmatrix}.
\end{equation*}
These two equations imply that
\begin{align}
\Zplus &= \Zt  + \mu \left[ \left(   \Bcal^* \Bcal \right) \left( \symA- \Zt \ZtT+ \tilZt\tilZtT  \right) \right] \Zt, \label{def:Ztdefinition} \\
 \tilZplus &= \tilZt - \mu \left[ \left(   \Bcal^* \Bcal \right) \left( \symA- \Zt \ZtT+ \tilZt\tilZtT  \right) \right] \tilZt.\label{def:tilZtdefinition}
\end{align}
This shows that a gradient descent iteration in the symmetrized loss function $ \mathcal{L}_{ \text{sym} } $ is equivalent to a gradient descent step in the original loss function $\mathcal{L}$.
In the following, we analyze the trajectories of $\Zt$ and $ \tilZt$ in the symmetric formulation.

Using this reformulation, we can use some of the proof techniques developed in \cite{stoger2021small} to analyze the gradient descent trajectory.
However, let us stress that there is a key difference compared to the model in \cite{stoger2021small}.
Namely, in \cite{stoger2021small}, it was assumed that the ground truth matrix $\symA$ is positive semidefinite. Thus, one can set $\tilZt=0$ and one only needs to optimize over $\Zt$.
However, in the scenario analyzed in this paper, the ground truth matrix $\symA$ has positive as well as negative eigenvalues. 
As already mentioned in the introduction, what makes the analysis in this paper now much more challenging is the fact that the trajectories of the two matrices $\Zt$ and $\tilZt$ are coupled with each other.
To deal with this additional difficulty we need a refined analysis, see below.

\subsection{Decomposition into signal and nuisance term}
A crucial ingredient of our analysis will be the decomposition of our matrices $\Zt$ and $\tilZt$ into a signal part, which will converge to the ground truth signal, and a nuisance part, which will stay small during the training.

In order to define this decomposition, recall first that the singular value decomposition of the ground truth matrix $X$ is given by $X=P_X \Sigma_X Q_X^T $. 
Then we define the following two matrices
\begin{equation}\label{equ:LXdefinition}
L_X := \frac{1}{\sqrt{2}}\begin{bmatrix}
 P_{ X}\\
 Q_{ X}
\end{bmatrix}, \quad \text{and}
\quad
\widetilde{L_X} := \frac{1}{\sqrt{2}}\begin{bmatrix}
 P_{ X}\\
- Q_{ X}
\end{bmatrix},
\end{equation}
whose columns are orthonormal.
This allows us to write the eigendecomposition of $\symA$ as
\begin{equation*}
\symA  = L_{ X} \Sigma_XL_{ X}^T - \widetilde{L_X} \Sigma_X\widetilde{L_X}^T.
\end{equation*}
Here, $ L_{ X} \Sigma_XL_{ X}^T$ represents the positive semidefinite part of $\symA$ and $ \widetilde{L_X} \Sigma_X\widetilde{L_X}^T$ represents the negative semidefinite part of $ \symA $.
Now consider the matrix $\LAT \Zt  \in \mathbb{R}^{r \times k}$.
Under the assumption that this matrix has full rank $r$ (which, as we will prove, holds true during training) we can denote its singular value decomposition by $ \LAT \Zt = P_t\Sigma_t Q_t^T$ with $Q_t\in \mathbb{R}^{k\times r}$.
Moreover, we denote by $\Qtp \in \R^{k\times (k-r)} $ a matrix with orthonormal columns, whose column span is orthogonal to the column span of $\Qt$.

Using these definitions, similarly as in \cite{stoger2021small}, we can decompose $Z_t$ into
\begin{equation*}
Z_t = 
\underset{ \text{signal part} }{\underbrace{Z_tQ_tQ_t^T}}
+ \underset{ \text{nuisance part} }{ \underbrace{Z_tQ_{t,\bot}Q_{t,\bot}^T}}.
\end{equation*}
Note that it follows immediately from the definition of $\Qtp$ that $ \LAT \Zt \Qtp =0 $.


Next, we decompose $\tilZt$ into its signal and nuisance part:
\begin{equation*}
    \tilZt = \underset{ \text{signal part} }{ \underbrace{ \tilZt \Qt \QtT}} +  \underset{ \text{nuisance part} }{\underbrace{\tilZt \Qtp \QtpT}} .  
\end{equation*}
%Note that due to equations \eqref{equ:Ztdefinition} and \eqref{equ:LXdefinition} the singular value decomposition of $\LtilAT \tilZt$ can be written as $\LtilAT \tilZt= \tilde{P_t}  \Sigma_t Q_t$, i.e., the matrix $\LtilAT \tilZt$ has the same singular values and the same right singular vectors as the matrix $\LAT\Zt$.
Note that due to equations \eqref{equ:Ztdefinition} and \eqref{equ:LXdefinition} we have that $ \LAT \Zt = \LtilAT \tilZt=  P_t \Sigma_t Q_t$, i.e., the matrix $\LtilAT \tilZt$ and the matrix $\LAT\Zt$ are the same.
For this reason, we can also use the matrix $Q_t$ to define the decomposition of $\tilZt$ into its signal and nuisance part, i.e., 
\begin{equation*}
    \tilZt
    =
    \tilZt \Qt \QtT 
    +
    \tilZt \Qtp \QtpT.
\end{equation*}

The following lemma collects a few facts regarding the symmetry between $\Zt$ and $\tilZt$.
These are a direct consequence of equations \eqref{equ:Ztdefinition} and \eqref{equ:LXdefinition}, which is why we skip the short proof.
\begin{lemma}\label{lemma:symmetry}[Symmetry between $\Zt$ and $\tilZt$]
Assume that $L_X^T \Zt$ has rank $r$
and let $\Zt$ and $\tilZt$ as defined in this section.
Then the norms are equal:
\begin{align*}
    \norm{\Zt}&=\norm{\tilZt}, \\
    \norm{\Zt \Qt}&=\norm{\tilZt \Qt},\\
    \norm{\Zt \Qtp}&=\norm{\tilZt \Qtp}.
\end{align*}
Moreover, the following two identities hold:
\begin{align*}
    \LtilAPT \tilZt \Qt &=  \LAPT \Zt \Qt,\\
    \LtilAPT P_{\tilZt \Qt} &= \LAPT \PZQ,
\end{align*}
where we have set
\begin{equation*}
    \LtilAP
    :=
    \begin{bmatrix}
        \Id_{n_1} & 0 \\
        0 & -\Id_{n_2}
    \end{bmatrix}
    \cdot 
    \LAP.
\end{equation*}
Note that $\LtilAP \in \mathbb{R}^{(n_1+n_2) \times (n_1+n_2-r)} $ is a matrix with orthonormal columns, whose span is orthogonal to the span of $ \LtilA $.
%(Here, $\LtilAP \in \mathbb{R}^{(n_1+n_2) \times (n_1+n_2-r)} $ denotes a matrix with orthonormal columns, whose span is orthogonal to the span of $ \LtilA $\cx{The right definition should be the tilde function of $ \LAP $}). 
\end{lemma}
To see why the decomposition of $\Zt$ into the signal and noise part is useful we note that 
\begin{align}
\Zplus &= \Zt  - \mu \left[ \left(   \Bcal^* \Bcal \right) \left(  \Zt \ZtT- \tilZt\tilZtT -\symA  \right) \right] \Zt \nonumber \\
&=\Zt  - \mu  \left( \Zt \ZtT- \tilZt\tilZtT -\symA \right)\Zt 
+ \mu \underset{=:\Deltat}{ \underbrace{ \left[ \left( \Id -  \Bcal^* \Bcal \right) \left(  \Zt \ZtT- \tilZt\tilZtT -\symA \right) \right] }} \Zt. \label{equ:decompositiongradient}
\end{align}
The expression $\Deltat$ can be interpreted as a perturbation term.
To control the spectral norm of the perturbation term, $\norm{\Deltat}$, we rely on the RIP.
However, since the RIP only applies for low-rank matrices, 
we decompose $\Deltat$ into a term involving the low-rank signal part, which we expect to have good control over, and a high-rank term involving the nuisance term as follows:
\begin{align*}
    \Deltat
    =&
    \left[ \left( \Id -  \Bcal^* \Bcal \right) \left(  \Zt \Qt \QtT \ZtT- \tilZt \Qt \QtT \tilZtT -\symA \right) \right]\\
    &+
    \left[ \left( \Id -  \Bcal^* \Bcal \right) \left(  \Zt \Qtp \QtpT \ZtT- \tilZt \Qtp \QtpT \tilZtT  \right) \right].
\end{align*}
The fact that we have much sharper control over the first term than over the second term is also reflected by the following lemma,
whose straightforward proof has been deferred to Appendix \ref{sec:proofauxiliarylemma}.
This also indicates that we need to deal with the signal part of $\Zt$ differently than with the nuisance part in our proof.
\begin{lemma}\label{lemma:RIPlemma}
    Assume that the linear measurement operator $\mathcal{A}: \R^{n_1 \times n_2} \rightarrow \R^m $ has the restricted isometry property (RIP) of order $2r+1$ with constant $\delta >0$.
    Then, for all iterations $t$, it holds that 
    \begin{equation}\label{ineq:RIPclaim1}
        \begin{split}
        & \norm{ \left(\Id - \mathcal{B}^* \mathcal{B} \right) \left( \Zt \PZQ \PZQT \ZtT -\tilZt \PZQ \PZQT \tilZtT   -\symA \right)  }\\
         \le  &\delta \sqrt{r} \norm{ \Zt \PZQ \PZQT \ZtT -\tilZt \PZQ \PZQT \tilZtT   -\symA }.
        \end{split} 
    \end{equation}
    Moreover, for all $t$, we have that
    \begin{equation}\label{ineq:RIPclaim2}
        \begin{split}
        &\norm{ \left(\Id - \mathcal{B}^* \mathcal{B} \right) \left( \Zt \PZQperp \PZQperpT \ZtT -\tilZt\PZQperp \PZQperpT \tilZtT  \right)  } \\
         \le &\left( k - r \right) \delta \norm{  \Zt \PZQperp \PZQperpT \ZtT -\tilZt\PZQperp \PZQperpT \tilZtT  }.
        \end{split}
    \end{equation}
    and
    \begin{equation}\label{ineq:RIPclaim3}
        \begin{split}
        \norm{ \left(\Id - \mathcal{B}^* \mathcal{B} \right) \left( \Zt  \ZtT -\tilZt \tilZtT  \right)  }  \le  \delta \nucnorm{  \Zt  \ZtT -\tilZt \tilZtT  }.
        \end{split}
    \end{equation}
\end{lemma}

\subsection{Three-Phase Analysis}\label{sec:threephase}
In our convergence analysis, we need to control several quantities related to $\Zt$ and $\tilZt$.
The first three quantities, which we keep track of in our analysis, have also been studied in \cite{stoger2021small}:
\begin{align*}
    \sglmin \left(\Zt \Qt\right) \tag*{(magnitude of the signal part),}\\
    \norm{ \Zt \Qtp } \tag*{(magnitude of the nuisance part),}\\
    \norm{ \LAT \PZQ } \tag*{(angle between column spaces of signal part and ground truth).}
\end{align*}
Due to the symmetry between $\Zt$ and $\tilZt$ there is no need to control the corresponding quantities for $\tilZt$, see Lemma \ref{lemma:symmetry}.

In contrast to \cite{stoger2021small}, it does not suffice to only control these three norms to analyze the asymmetric scenario.
The reason is that the dynamics of $\Zt$ and $\tilZt$ are coupled with each other as can be seen from equations \eqref{def:Ztdefinition} and \eqref{def:tilZtdefinition}.
Inspecting equation \eqref{equ:decompositiongradient}, it is also natural to require that the spectral norm of the matrix $\tilZtT \Zt$ stays small, which is why in our analysis we control the norm
\begin{align*}
    \norm{ \tilZtT \Zt } 
    \tag*{(magnitude of the imbalance term)}.
\end{align*}
A direct calculation shows that
\begin{align*}
    \tilZplusT \Zplus 
    = 
    \tilZtT \Zt 
    + 
    \mu^2 \tilZtT \left( \Bcal^* \Bcal \left( \symA -\Zt \ZtT + \tilZtT \tilZt  \right) \right)^2 \Zt
    =
    \tilZtT \Zt 
    + 
    O \left( \mu^2 \right).
\end{align*}
Note that due to the choice of our initialization, we have that $\norm{\tilde{Z}_0^T Z_0} = O \left( \alpha^2 \right) $.
Thus, be choosing the step size small enough one could achieve that $\norm{ \tilZtT \Zt }$ stays at order $ O \left(\alpha^2 \right) $ during training.
Indeed, choosing the step size small enough or analyzing gradient flow is how several related works studying gradient flow deal with this issue, see, e.g., \cite{bah2019learning}. 

However, as our experiments show, see Section \ref{sec:experiments}, when using more realistic step sizes, the quantity $\norm{ \tilZtT \Zt }$ will increase during training significantly.
For this reason, it turns out that it does not suffice to only control $\norm{ \tilZtT \Zt }$ and we need a more fine-grained analysis of the coupling between $\Zt$ and $\tilZt$.
This will be achieved via controlling the following two norms
\begin{align*}
    \norm{\tilZtT \Zt \Qtp} \tag*{(imbalance term parallel to nuisance part),}\\
    \norm{\tilZtT \PZQ} \tag*{(imbalance term parallel to signal part).}
\end{align*}
The first term can be interpreted as measuring how strong the nuisance part of the signal $\Zt$ is coupled with $\tilZt$.
The second term can be seen as a projection of $\tilZt$ onto the span of the signal part of $\Zt$.
Note that this is different from $\norm{\tilZtT \Zt \Qt}$ as the singular values of $\Zt \Qt$  are not taken into account in the expression $\norm{\tilZtT \PZQ} $.

\subsubsection{Analysis of Phase 1}
Phase 1, the \textit{Spectral Phase}, is based on the observation that, due to our small, random initialization, in the first few iterations $\Zt \ZtT $ and $\tilZt \tilZtT$ are both very small compared to the (symmetrized) signal $\symA$.  
Hence, we can approximate the gradients of the loss function by
\begin{align*}
\nabla_Z \Losssym \left( \Zt, \tilde{\Zt} \right) 
= - \left[ \left(   \Bcal^* \Bcal \right) \left( \symA- \Zt \ZtT + \tilZt \tilZtT  \right) \right] \Zt  
\approx - \left[ \left(   \Bcal^* \Bcal \right) \left( \symA  \right) \right] \Zt 
\end{align*}
and by
\begin{align*}
\nabla_{\tilde{Z}} \Losssym \left( \Zt, \tilZt \right) 
=  \left[ \left(   \Bcal^* \Bcal \right) \left( \symA- \Zt \ZtT + \tilZt \tilZtT  \right) \right] \tilZt 
\approx  \left[ \left(   \Bcal^* \Bcal \right) \left( \symA  \right) \right] \tilZt.
\end{align*}
Due to equation \eqref{def:Ztdefinition} this implies that in the first few iterations we can approximate the iterates $\Zt$ by 
\begin{align*}
    \Zt 
    \approx 
    \left( \Id + \mu \left[ \left(   \Bcal^* \Bcal \right) \left( \symA  \right) \right] \right) Z_{t-1}
    \approx \left( \Id + \mu \left[ \left(   \Bcal^* \Bcal \right) \left( \symA  \right) \right] \right)^{t} Z_{0}.
\end{align*}
Analogously, due to equation \eqref{def:tilZtdefinition} we can approximate the iterates $\tilZt$ by 
\begin{align*}
    \tilZt
    \approx 
    \left( \Id - \mu \left[ \left(   \Bcal^* \Bcal \right) \left( \symA  \right) \right] \right)^t \tilde{Z}_0.
\end{align*}
This observation allows us to prove that after the first few iterations the signal parts of $\Zt$ and $\tilZt$ are well aligned with the eigenvectors of the ground truth signal $X$.
This is made precise in Lemma \ref{lemma:spectralmain} below.
The proof of this lemma is similar to the analysis of the spectral phase in \cite{stoger2021small}, which is why its complete proof has been deferred to Appendix \ref{sec:spectralphaseanalysis}.
We also refer to \cite{stoger2021small} for a more detailed discussion of the intuition behind the spectral phase.
\begin{lemma}\label{lemma:spectralmain}
Assume that $V_0 = \alpha V \in \R^{n_1 \times k} $ and $W_0= \alpha W \in \R^{n_2 \times k}$ for some fixed parameter $\alpha >0$, where the matrices $V$ and $W$ have i.i.d. entries with distribution $ \mathcal{N} \left(0,1\right) $.
Let the iterates $\left\{ \Zt; \tilZt \right\}_{t\in \mathbb{N}} $ be defined as in equations \eqref{def:Ztdefinition} and \eqref{def:tilZtdefinition}.
Moreover, assume that 
\begin{equation}\label{spectral:alphaassumption}
    \alpha \le 
    \sqrt{\frac{\norm{X}  }{ C_1 \max \left\{  n_1 + n_2;k \right\} }} \left( \frac{c \varepsilon \left( \sqrt{k} -\sqrt{r-1} \right) }{6\kappa^2  \sqrt{\max \left\{  n_1 + n_2; k \right\}} } \right)^{34\kappa }
\end{equation}
for some $0 < \varepsilon <1 $ and for some constant $0< c \le 1/32$.
Moreover, assume that
\begin{equation*}
    \norm{X- \left( \mathcal{A}^* \mathcal{A} \right) \left(X\right) } \le \frac{c}{\kappa^2} \sglmin \left( X \right)
\end{equation*}
and that $ \mu \le \frac{\tilde{c}}{\sglmin (X)} $ for a sufficiently small absolute constant $\tilde{c}>0$.
Then, with probability at least $1 -  C_2 \exp \left( - c_1 k \right) + \left( C_3\varepsilon \right)^{k-r+1}  $, after 
\begin{equation}\label{tstarbound}
    \tstar 
    := \left\lceil \frac{\ln\left(\frac{c\sglmin \left(L_{F_1}^T Z_0\right)}{2\kappa^{2} \norm{Z_0} }\right)}{\ln \left(1 - \frac{\mu\sglmin(X)}{8} \right)} \right\rceil
    \le \frac{17 \ln\left(\frac{6\kappa^{2} \sqrt{ \max \left\{ n_1+n_2; k \right\} } }{c\varepsilon \left( \sqrt{k} - \sqrt{r-1} \right) }\right)}{ \mu \sglmin \left(X\right) }
\end{equation}
iterations, the following inequalities hold:
\begin{align}
\norm{L_{X,\bot}^{T} P_{Z_{t_\star}Q_{t_\star}}}&\le \frac{28c}{\kappa^2}, \label{spectral:final1} \\
\sigma_{\min}(Z_{t_\star}Q_{t_\star}) 
&\ge \left(\frac{2\kappa^{2}  }{c}\right)^{8\kappa} \frac{\alpha \varepsilon \left( \sqrt{k} -\sqrt{r-1} \right)}{4} ,\label{spectral:final2}\\
\norm{Z_{t_\star}Q_{t_\star,\bot}} 
&\le \min \left\{ 2 \sglmin \left( \Ztstar Q_{\tstar} \right); \  \left( \frac{6\kappa^2  \sqrt{\max \left\{  n_1+n_2; k \right\} }}{ c \varepsilon \left( \sqrt{k} - \sqrt{r -1} \right)} \right)^{ 16 \kappa} \cdot \frac{12c \alpha \sqrt{\max \left\{ n_1 + n_2; k \right\} }}{ \kappa^2 } \right\} ,\label{spectral:final3}\\
\norm{Z_{\tstar}} &\le 2 \sqrt{\norm{X}},\label{spectral:final4}\\
\norm{\tilde{Z}_{\tstar}^T Z_{\tstar}  } 
&\le  \frac{ c \norm{X}}{\kappa^4}, \label{spectral:finalbalanc1} \\ 
\norm{\tilde{Z}_{\tstar}^T Z_{\tstar} Q_{\tstar, \bot}  }
&\le  \frac{c \sqrt{\norm{X}} }{\kappa^3} \norm{\Ztstar Q_{\tstar,\bot} },\label{spectral:finalbalanc2} \\ 
\norm{\tilde{Z}_{\tstar}^T P_{ Z_{\tstar} Q_{\tstar} }  }
&\le \frac{ c \sqrt{\norm{X}} }{\kappa^3}  . \label{spectral:finalbalanc3}
\end{align}
Here, $C_1, C_2, C_3, c_1 >0$ are two absolute constants.
\end{lemma}


\subsubsection{Analysis of Phase 2}
In Phase 2, the \textit{saddle avoidance phase}, we show that $\sglmin \left( \LAT \Zt \right) $ grows exponentially until we have that $ \sglmin \left( \LAT \Zt \right) \ge  \sqrt{\frac{\sglmin (X)}{8}} $.
Moreover, we show that in Phase 2, that the spectral norm of the nuisance term, $\norm{ \Zt \Qtp}$, is growing much slower than $ \sglmin \left(  \LAT \Zt \right)$.
This is captured by the following two lemmas, whose proofs have been deferred to Appendix \ref{sec:sigmingrowth} and Appendix \ref{sec:noisetermgrowth}.
\begin{lemma}\label{lemma:sigmingrowth}
Assume that $\mu \le \frac{c}{\norm{X}\kappa}$, $\norm{\Zt} \le 2\sqrt{\norm{X}}$, $\norm{\LAPT\P{\Zt\Qt}} \le \frac{c}{\kappa}$, $ \norm{\Deltat} \le c \sglmin(X) $, and that $ \LAT \Zt \Qt$ is invertible. Then it holds that
\begin{equation}\label{lemma1result}
\sglmin(\LAT\Zplus) \ge \sglmin(\LAT\Zplus\Qt) \ge \sglmin(\LAT\Zt)\left(1 + \frac{1}{4}\mu\sglmin(X) - \mu\sglmin^2(\LAT\Zt)\right).
\end{equation}
Here, $c>0$ is an absolute constant chosen small enough.
\end{lemma}


\begin{lemma}\label{lemma:noisetermgrowth}
Let $0 \le \varepsilon \le 1$.
Assume that $\mu \le \frac{c\varepsilon}{\norm{X}\kappa}$,  $\norm{\Zt} \le 2\sqrt{\norm{X}}$, $ \norm{\Deltat} \le c\varepsilon \sglmin (X) $, and $\norm{\LAPT\P{\Zt\Qt}} \le \frac{ c \varepsilon }{ \kappa }$.
Moreover, assume that $\LAT\Zplus\Qt$ and $\LAT\Zt\Qt$ have full rank.
Then it holds that 
\begin{equation*}
\norm{\Zplus\Qplusp} \le \left(1 - \frac{\mu}{2}\norm{\Zt\Qtp}^2 + \mu \varepsilon  \sglmin(X) \right) \norm{\Zt\Qtp} + 2\mu\sqrt{\norm{X}}\norm{\tilZtT\Zt \Qtp}.
\end{equation*}
Here, $c>0$ is an absolute constant chosen small enough.
\end{lemma}
We observe that in order to apply Lemma \ref{lemma:sigmingrowth} and Lemma \ref{lemma:noisetermgrowth}
we need to control several key quantities, which are described in the beginning of Section \ref{sec:threephase}.

The following lemma controls the angle between the positive eigenvectors ground truth signal $\symA$ and the signal part of $\Zt$.
The proof of Lemma \ref{lemma:anglecontrol} has been deferred to Appendix \ref{sec:anglecontrol}.
\begin{lemma}\label{lemma:anglecontrol}
Assume that $\mu \le \frac{c}{\norm{X}\kappa}$, $\norm{\LAPT\P{\Zt\Qt}} \le c\kappa^{-1}$, $ \norm{\Zt} \le 2\sqrt{\norm{X}}$, and $\norm{\Deltat} \le c \sglmin(X) $.
Moreover, assume that $\norm{\Zt\Qtp}\le \min \left\{ c \kappa^{-1/2} \sqrt{\sglmin(X)}; 2\sglmin(\Zt\Qt) \right\}$, $\norm{\tilZtT\Zt \Qtp} \le \sqrt{\norm{X}}\sglmin\left(\Zt\Qt\right)$, and that $\Zt \Qt$ has rank $r$.
Then it holds that
\begin{align*}
&\norm{\LAPT\P{\Zplus Q_{t+1} }} \le \\
&\left(1 - \frac{1}{4}\mu\sglmin \left(X\right) \right)\norm{\LAPT\P{\Zt\Qt}} 
+  2 \mu  \sqrt{\norm{X}} \norm{ \tilZtT \P{\Zt\Qt} }
+C\mu\frac{\sqrt{\norm{X}}\norm{\tilZtT\Zt \Qtp}}{\sglmin\left(\Zt\Qt\right)} + C \mu \norm{\Deltat} +  C\mu^2\norm{X}^2.
\end{align*}
Here, $c>0$ is an absolute constant chosen small enough and $C>0$ is an absolute constant chosen large enough.
\end{lemma}
The next lemma, whose proof has been deferred to Appendix \ref{sec:normcontrolled}, shows the spectral norm of $\Zt$ will never be too large compared to $ \sqrt{ \norm{X} }$.
%\cx{Compared to $\sqrt{X}$?}
\begin{lemma}\label{lemma:normcontrolled}
Suppose that $\norm{\Zt} \le 2 \sqrt{\norm{X}}$, $ \norm{\Deltat} \le \frac{1}{100} \norm{X} $, $\norm{\Zt \Qtp}\le \frac{ \sqrt{\norm{X}}}{100}$, $\norm{\LAPT P_{\Zt \Qt}} \le \frac{1}{100} $, and $ \mu \le \frac{1}{100 \norm{X}} $.
Then it holds that
\begin{equation*}
    \norm{\Zplus} \le 2 \sqrt{\norm{X}}.
\end{equation*}
\end{lemma}
The next three lemmas allow us to control the quantities related to the imbalance term, $ \norm{\tilZtT \Zt} $, $ \norm{ \tilZt \Zt \Qtp } $, and $ \norm{ \tilZt \PZQ } $.
In particular, using these lemmas we can prove that the trajectories of $\tilZt$ and $\Zt$ are sufficiently decoupled during training. The first lemma controls the growth of $\norm{\tilZtT \Zt} $.
\begin{lemma}\label{lemma:balancedbase}
Assume that $ \norm{\Zt} \le 2 \sqrt{\norm{X}} $
and $\norm{\Deltat} \le  \norm{X} $.
Then it holds that
\begin{equation*}
    \norm{\tilZplusT \Zplus} \le \norm{\tilZtT \Zt} + 400 \mu^2 \norm{X}^3.
\end{equation*}
\end{lemma}
The proof of Lemma \ref{lemma:balancedbase} has been deferred to Appendix \ref{sec:balancedbase}.
Note that in particular, Lemma \ref{lemma:balancedbase} shows that the growth is quadratic in the step size $\mu$.

The next lemma controls the growth of the balancedness term parallel to the nuisance part, $\tilZtT \Zt \Qtp $.
Note that it shows that its growth is upper bound by the size of the nuisance part $\norm{\Zt \Qtp}$.
Using this lemma we can show that  $\tilZtT \Zt \Qtp $ stays small relative to $\sqrt{\norm{X}} \norm{\Zt \Qtp} $.
\begin{lemma}\label{lemma:balancednessperp}
Assume that $\LAT \Zplus \Qt $ has full rank.
Moreover, assume that $\max \left\{ \norm{ \Zt }; \norm{\Zplus } \right\} \le 2 \sqrt{\norm{X}}$, $ \norm{\LAPT \PZQ} \le c $,  $\mu \le \frac{c}{\norm{X} \kappa}$, $\norm{\Zt} \le  2\sqrt{\norm{X}} $, and $ \norm{\Deltat} \le c \sglmin (X) $, where $c>0$ is an absolute constant chosen small enough.
Set 
\begin{equation*}
\beta := \norm{\LAPT \PZQ}  \norm{X}  + \norm{ \Zt \Qtp  }^2 + \norm{\Deltat}.
\end{equation*}
Then it holds that
 \begin{align*}
    &\norm{\tilZplus^T \Zplus \Qplusp} \le  \norm{ \tilZtT \Zt\Qtp } \\
    &+ C\mu \left(   \left( \norm{\LAPT\P{\Zt\Qt}} + \mu\norm{X} \right)\beta  +  \mu \norm{X}^2 \right)  \sqrt{\norm{ X }} \norm{ \Zt\Qtp }+ 8 \mu \beta \norm{\Zt \Qtp}^2,
\end{align*} 
where $C>0$ is an absolute constant chosen large enough.
\end{lemma}
The proof of Lemma \ref{lemma:balancednessperp} has been deferred to Appendix \ref{sec:balancednessperp}.
Next, Lemma \ref{ref:balancednessangle}, whose proof has been deferred to Appendix \ref{sec:balancednessangle}, allows us to prove that $\norm{ \tilZplusT \PZQ } $ stays bounded.
\begin{lemma}\label{ref:balancednessangle}
    Assume that $ \norm{\Zt} \le 2 \sqrt{\norm{X}} $, $ \norm{ \Zt \Qtp} \le \min \left\{  2 \sglmin \left( \Zt \Qt \right); c \sqrt{\sglmin (X)}  \right\} $, $\norm{\Deltat} \le c \sglmin \left( X\right)  $,  $ \mu \le \frac{c}{\norm{X} \kappa}  $, $\norm{\tilZtT\Zt \Qtp} \le \frac{c}{\kappa} \sglmin \left(\Zt \Qt \right) \sqrt{\norm{X}}$, and $\norm{\LAPT \PZQ } \le \frac{c}{\kappa} $, where $c>0$ is an absolute constant chosen sufficiently small.
    Then it holds that
    \begin{equation*}
        \begin{split}
        &\norm{ \tilZplus^T P_{\Zplus \Qplus}  } \\
        \le & \left( 1-  \frac{\mu}{4} \sglmin \left( X \right) \right) \norm{\tilZt^T \PZQ } 
        +4 \mu \norm{X} \norm{  \Zt \Qtp  }
        + 2\mu \norm{ \tilZt^T \Zt  } \sqrt{\norm{X}}+ \frac{\mu  \norm{\tilZt^T \Zt\Qtp}  \sglmin (X) }{ \sglmin \left( \Zt\Qt \right)  } \\
        &  + 800 \mu^2  \norm{X}^{5/2}.
        \end{split}
    \end{equation*}
\end{lemma}
Now we have all ingredients in place to prove Lemma \ref{lemma:phase2combined}, which is the main lemma for the second training phase.
The proof of Lemma \ref{lemma:phase2combined} has been deferred to Appendix \ref{sec:phase2proof}.
\begin{lemma}\label{lemma:phase2combined}
    Assume that the measurement operator $\mathcal{A}$ satisfies the rank-$(2r+1)$ restricted isometry property with constant $\delta \le  \frac{\hat{c}_1}{\kappa^{3} \sqrt{r} }$
    and assume that the step size satisfies $ \mu \le \frac{\hat{c}_2}{\kappa^{4} \norm{X}} $.
    Furthermore, assume that $ \sglmin \left( \LAT Z_{t_1} \right) \le \sqrt{ \frac{\sglmin (X)}{8} } $.
    Let $\left\{ \Zt \right\}_{t \in \mathbb{N}}$ and $\left\{ \tilZt \right\}_{t \in \mathbb{N}}$ be the iterates as defined in \eqref{def:Ztdefinition} and \eqref{def:tilZtdefinition}.
    Assume that after $t_1$ iterations we have that
    \begin{align}
        \norm{\Ztone Q_{t_1,\bot}} & \le \min \left\{  2 \sglmin (\Ztone Q_{t_1}) ;  \frac{ \hat{c}_3 \sqrt{ \sglmin \left(X\right) }}{ k \kappa^{35/8} } \right\}, \label{ineq:phase2assump1} \\
        \norm{\LAPT P_{Z_{t_1} Q_{t_1}} } & \le \frac{\hat{c}_4}{\kappa^2},  \label{ineq:phase2assump2} \\
        \norm{\Ztone} &\le 2 \sqrt{\norm{X}},  \label{ineq:phase2assump3} \\
        %\norm{ \widetilde{Z}_{t_1}^T Z_{t_1} } &\le ????   \label{ineq:phase2assump_balanc1}\\
        \norm{ \widetilde{Z}_{t_1}^T Z_{t_1} Q_{t_1,\bot} }  &\le \frac{\hat{c}_4 \hat{c}_5 \sqrt{ \norm{X} } }{\kappa^{3} }  \norm{\Ztone Q_{t_1,\bot}},    \label{ineq:phase2assump_balanc2}\\
        \norm{\widetilde{Z}_{t_1}^T P_{Z_{t_1} Q_{t_1 } }  }  &\le  \frac{ \hat{c}_4  \hat{c}_6 \sqrt{\norm{X}}}{\kappa^{3}}.   \label{ineq:phase2assump_balanc3}
    \end{align}
    Moreover, assume that 
    \begin{align} \label{ineq:phase2assump_balanc1}
        \norm{ \widetilde{Z}_{t_1}^T Z_{t_1} } + 400 \mu^2 \Bigg\lceil  \frac{\ln \left( \frac{ \sqrt{ \sglmin \left(X\right) } }{\sqrt{8} \sglmin \left( \LAT \Ztone \right) }   \right)}{\ln \left( 1 +\frac{\mu}{8}  \sglmin \left(X\right)   \right)} \Bigg\rceil \norm{X}^3
         \le \frac{\hat{c}_7}{\kappa^{4}} \norm{X}.
    \end{align}
    Then there is a natural number $t_2 \ge t_1$ and a constant $\gamma \in \mathbb{R}$ with
    \begin{align}
        t_2-t_1 &\le \Bigg\lceil \frac{\ln \left( \frac{ \sqrt{ \sglmin \left(X\right) } }{\sqrt{8} \sglmin \left( \LAT \Ztone \right) }   \right)}{\ln \left( 1 +\frac{\mu}{8}  \sglmin \left(X\right)   \right)} \Bigg\rceil, \label{ineq:t2bound} \\
        \norm{\Ztone Q_{\tone, \bot}} \le \gamma & \le \left( \frac{1}{128} \right)^{1/10}  \left( \sglmin \left(X\right) \right)^{1/10}   \norm{ \Ztone Q_{t_1,\bot}}^{4/5}  \label{ineq:gammabound} 
    \end{align}
    such that after $t_2$ iterations it holds that
    \begin{align}
        \sglmin \left( \LAT \Zttwo \right) &\ge  \sqrt{\frac{\sglmin (X)}{8}}, \label{ineq:phase2final1} \\
        \norm{\Zttwo Q_{t_2,\bot}} & %\le  \left( \frac{1}{128} \right)^{1/10}  \left( \sglmin \left(X\right) \right)^{1/10}   \norm{ \Ztone Q_{t_1,\bot}}^{4/5}
        \le \gamma ,   \label{ineq:phase2final2}  \\
        \norm{\LAPT P_{Z_{t_2} Q_{t_2} } } & \le \frac{\hat{c}_4}{\kappa^2}, \label{ineq:phase2final3}  \\
        \norm{\Zttwo} &\le 2 \sqrt{\norm{X}},  \label{ineq:phase2final4}  \\
        %\norm{B_{t_2}} &\le ????. \label{ineq:phase2final5} 
        \norm{ \widetilde{Z}_{t_2}^T Z_{t_2}  } &\le \norm{ \widetilde{Z}_{t_1}^T Z_{t_1} } + 400\mu^2 \left( t_2-t_1 \right) \norm{X}^3,  \label{ineq:phase2final_balanc1}\\ 
        \norm{ \widetilde{Z}_{t_2}^T Z_{t_2} Q_{t_2,\bot} }  & %\le \left( \frac{1}{128} \right)^{1/10} \frac{  \hat{c}_4 \hat{c}_5 \sqrt{\norm{X}} \left( \sglmin \left(X\right) \right)^{1/10}   \norm{ \Ztone Q_{t_1,\bot}}^{4/5}}{\kappa^{3}}
        \le \frac{ \hat{c}_4 \hat{c}_5 \sqrt{\norm{X}} \gamma }{\kappa^3},   \label{ineq:phase2final_balanc2}\\
        \norm{\widetilde{Z}_{t_2}^T P_{Z_{t_2} Q_{t_2 } }  }  &\le  \frac{\hat{c}_4 \hat{c}_6 \sqrt{\norm{X}}}{\kappa^3}.   \label{ineq:phase2final_balanc3}
    \end{align}
Here, $\hat{c}_1, \hat{c}_2, \hat{c}_3, \hat{c}_4, \hat{c}_5, \hat{c}_6, \hat{c}_7>0$ are absolute constants chosen small enough.
\end{lemma}
\begin{remark}\label{choiceofconstants}
In fact, in our proof we will show a bit more than the lemma suggests. 
Namely, we will prove that inequalities \eqref{ineq:phase2final2}--\eqref{ineq:phase2final_balanc3} hold for all $t$ such that $t_1 \le t \le t_2$.

As it turns out in our proof, one needs to be a bit careful on how to choose the constant.
In fact, we will need to choose the constants $\hat{c}_1, \hat{c}_2, \hat{c}_3, \hat{c}_4, \hat{c}_5, \hat{c}_6, \hat{c}_7>0$ such that the following relationships are satisfied:
$\hat{c}_1, \hat{c}_2, \hat{c}_3^{4/5} \ll \hat{c}_4 \hat{c}_5 $, $\hat{c}_4 \ll \hat{c}_5 \ll \hat{c}_6 \ll 1$.
Here, $a \ll b$ means that the constants $a$ and $b$ must be chosen such that they satisfy $a \le cb$ where $c$ is an absolute constant chosen sufficiently small.
\end{remark}
\subsubsection{Analysis of Phase 3}
Phase 3, the \textit{refinement phase}, begins after we have that $ \sglmin \left( \LAT \Zt \right) \ge \sqrt{\frac{\sglmin (X)}{8}} $.
In the following, we prove that in Phase 3 the learned signal $\Zt \ZtT - \tilZt \tilZtT $ converges to $\symA$ with respect to the spectral norm.
In other words, we show that 
\begin{equation}\label{term:intern2}
\norm{ \symA - \Zt \ZtT + \tilZt \tilZtT }
\end{equation}
becomes small.
A key difficulty in our analysis is now that the column spans of the matrices $\Zt$ and $\tilZt$ are not orthogonal to each other.
For this reason, individually estimating $ \norm{\LA \Sigma_X \LAT - \Zt\ZtT} $ and $ \norm{  \widetilde{L_X} \Sigma_X\widetilde{L_X}^T - \tilZt \tilZtT} $ leads to suboptimal bounds for \eqref{term:intern2}.
However, we can deal with this by using inequality \eqref{term:intern3} in Lemma \ref{SpecLossboundedlemma} below.  
\begin{lemma}\label{SpecLossboundedlemma}
Assume that $\mu \le \frac{ c}{ \kappa \norm{X} }$, $\norm{\LAPT\P{\Zt\Qt}} \le \frac{c }{\kappa }$, $\norm{\Zt}\le 2\sqrt{\norm{X}}$, and 
\begin{equation*}
\norm{\Deltat} \le \frac{c}{\kappa} \norm{\symA - \Zt\ZtT + \tilZt\tilZtT}.
\end{equation*}
Moreover, assume that $\sglmin(\Zt\Qt) \ge \sqrt{\frac{\sglmin(X)}{8}}$ and $\norm{\Zt\Qtp}\le c \sqrt{\sglmin(X)}$.
Then it holds that
\begin{equation*}
\norm{\LAPT \left( \symA - \Zt\ZtT + \tilZt \tilZtT \right) } \le 5\norm{\LAT \left( \symA - \Zt \ZtT + \tilZt \tilZtT \right) } + 4\norm{\Zt\Qtp}^2
\end{equation*}
and
\begin{equation}\label{term:intern3}
\norm{ \symA - \Zt\ZtT + \tilZt \tilZtT } 
\le 6\norm{\LAT \left( \symA - \Zt\ZtT + \tilZt \tilZtT \right) } 
+ 4\norm{\Zt\Qtp}^2.
\end{equation}
Here, $c>0$ is a sufficiently small absolute constant.
\end{lemma}
The proof of Lemma \ref{SpecLossboundedlemma} been deferred to Appendix \ref{sec:localconvergence}.
Due to inequality \eqref{term:intern3} to show that term \eqref{term:intern2} becomes small, it suffices now to show that 
\begin{equation*}
    \norm{\LAT \left( \symA - \Zt\ZtT + \tilZt \tilZtT \right) }
\end{equation*}
becomes sufficiently small.
For this task, we use the following lemma, whose proof has been deferred to Appendix \ref{sec:localconvergence}.
\begin{lemma}\label{lemma:localconvergence}
Under the assumptions of Lemma \ref{SpecLossboundedlemma} it holds that
\begin{align*}
    &\norm{\LAT(\symA - \Zplus\ZplusT + \tilZplus \tilZplusT)} \\
    \le &\left(1 - \frac{\mu}{128}\sglmin(X) \right)\norm{\LAT \left( \symA - \Zt\ZtT + \tilZt\tilZtT \right)} +  \frac{\mu}{20}\sglmin(X)\norm{\Zt\Qtp}^2.
\end{align*}
\end{lemma}
It is worth noting that both Lemma \ref{SpecLossboundedlemma} and Lemma \ref{lemma:localconvergence} do not not require any assumption on the imbalance matrix $\tilZtT \Zt $.

In \cite{stoger2021small} two similar lemmas are used to prove linear convergence in Phase 3.
However, the proof of Lemma \ref{SpecLossboundedlemma} in the paper at hand is significantly more involved due to the appearance of the additional $\tilZt$ term.

With this lemma in place and using the technical Lemmas \ref{lemma:anglecontrol}, \ref{lemma:normcontrolled}, \ref{lemma:balancedbase}, \ref{lemma:balancednessperp}, and \ref{ref:balancednessangle} from Phase 2, we are able to prove the following central lemma, which describes the training behaviour in the third phase.
Its proof has been deferred to Appendix \ref{sec:phase3proof}.
\begin{lemma}\label{lemma:phase3combined}
    Assume that $\mathcal{A}$ satisfies the rank-$(2r+1)$ restricted isometry property with constant $\delta < \frac{\hat{c}_1}{\kappa^3 \sqrt{r}} $.
    Furthermore, assume that the step size satisfies $ \mu \le \frac{\hat{c}_2}{\kappa^{4} \norm{X}} $.
    Let $\left\{ \Zt \right\}_{t \in \mathbb{N}}$ and $\left\{ \tilZt \right\}_{t \in \mathbb{N}}$ be the iterates defined in equations \eqref{def:Ztdefinition} and \eqref{def:tilZtdefinition}.
    Assume that there is a natural number $t_2$ and a positive real number $\gamma$ with 
    \begin{equation}\label{phase3:gammabound}
        \gamma 
        \le c_3 \min \left\{  \frac{ \sqrt{ \sglmin (X)}}{\kappa^{9/2}}; \frac{ \sqrt{\norm{X} } }{k^{4/3}} \right\} 
    \end{equation}
    such that
    \begin{align}
        \sglmin \left( \LAT Z_{t_2} \right)  &\ge  \sqrt{\frac{\sglmin (X)}{8}}, \label{ineq:phase3assump1}\\
        \norm{\Zttwo Q_{t_2,\bot}} 
        & \le \gamma, \label{ineq:phase3assump2}\\
        \norm{\LAPT P_{Z_{t_2}} Q_{t_2} } & \le \frac{\hat{c}_4}{\kappa^2},\label{ineq:phase3assump3}  \\
        \norm{\Zttwo} &\le 2 \sqrt{\norm{X}}, \label{ineq:phase3assump4} \\
        \norm{\tilde{Z}^T_{t_2} Z_{t_2} Q_{t_2, \bot} } 
        &\le \frac{\hat{c}_4 \hat{c}_5 \sqrt{\norm{X}}}{\kappa^3} \cdot \gamma \label{ineq:phase3assumpbalanc2}\\
        \norm{ \tilde{Z}^T_{t_2} P_{Z_{t_2} Q_{t_2}}} &\le \frac{\hat{c}_4 \hat{c}_6 \sqrt{\norm{X}}}{\kappa^3}. \label{ineq:phase3assumpbalanc3} 
    \end{align}
    Moreover, assume that 
    \begin{equation}\label{ineq:phase3assump_balanc1}
        \norm{ \widetilde{Z}_{t_2}^T Z_{t_2} } + 120000\mu \frac{ \ln \left( \frac{9  \sqrt{\norm{X}} }{200 k \gamma} \right) \norm{X}^3 }{ \sglmin (X)}  \le \frac{\hat{c}_7}{\kappa^{4}} \norm{X}.
    \end{equation}
    Then there is a natural number $t_3  \ge t_2$ with
    \begin{equation*}
        t_3-t_2 \le \frac{300 \ln \left( \frac{9  \sqrt{\norm{X}} }{200 k \gamma} \right) }{\mu \sglmin (X)}
    \end{equation*}
    such that after $t_3$ iterations it holds that
    \begin{equation}
        \norm{\symA - Z_{t_3} Z^T_{t_3}+  \tilde{Z}_{t_3} \tilde{Z}^T_{t_3} } \lesssim k \gamma  \sqrt{\norm{X} }. \label{phase3:finalerror}
    \end{equation}
The constants $\hat{c}_1,\hat{c}_2,\hat{c}_3,\hat{c}_4,\hat{c}_5,\hat{c}_6,\hat{c}_7>0$ are the same constants as those appearing in Lemma \ref{lemma:phase2combined}.
\end{lemma}


