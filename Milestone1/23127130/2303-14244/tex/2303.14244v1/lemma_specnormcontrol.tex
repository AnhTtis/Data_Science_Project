\subsection{Proof of Lemma \ref{lemma:normcontrolled}: Controlling $\norm{\Zt}$}\label{sec:normcontrolled}



\begin{proof}[Proof of Lemma \ref{lemma:normcontrolled}]
    We note that
     \begin{align}
    \Zplus &= \Zt -\mu \left( \Zt \ZtT - \tilZt \tilZtT - \symA \right) \Zt + \mu \Deltat \Zt \nonumber \\
    & =\left( \Id - \mu  \Zt \ZtT\right)\Zt + \mu \tilZt \tilZtT \Zt - \mu  \symA \Zt + \mu \Deltat \Zt. \label{ineq:aux52}
     \end{align}
    To deal with this expression, we compute that
     \begin{align*}
        &\tilZt \tilZtT \Zt \\
        =&   \tilZt \Qt \QtT \tilZtT \Zt \Qt \QtT +   \tilZt \Qt \QtT \tilZtT \Zt \Qtp \QtpT +  \tilZt \Qtp \QtpT \tilZtT \Zt \\
        =&   \tilZt \Qt \QtT \tilZtT P_{\tilZt Q_t} P_{\tilZt Q_t}^T P_{\Zt Q_t} P_{\Zt Q_t}^T  \Zt \Qt \QtT +   \tilZt \Qt \QtT \tilZtT \Zt \Qtp \QtpT +  \tilZt \Qtp \QtpT \tilZtT \Zt .
     \end{align*}
     We obtain that
     \begin{align*}
         \norm{\tilZt \tilZtT \Zt} \le& \norm{  P_{\tilZt Q_t}^T P_{\Zt Q_t} } \norm{\tilZt \Qt}^2  \norm{\Zt \Qt}  + \norm{\tilZt \Qt}^2  \norm{\Zt \Qtp}  + \norm{\tilZt \Qtp}^2 \norm{\Zt} \\
         \stackrel{(a)}{\le} & \left( \norm{  P_{\tilZt Q_t}^T P_{\Zt Q_t} } \norm{ \Zt}^2 + \norm{\Zt}  \norm{\Zt \Qtp}  +  \norm{\Zt \Qtp}^2   \right) \norm{\Zt}\\
         \stackrel{(b)}{\le} & \left( 4 \norm{  P_{\tilZt Q_t}^T P_{\Zt Q_t} } \norm{X} + \frac{\norm{X}}{30}   \right) \norm{\Zt},
     \end{align*}   
    where in inequality $(a)$ we have used that due to symmetry it holds that $\norm{\Zt \Qt} = \norm{\tilZt \Qt}  $ and $ \norm{\Zt \Qtp} = \norm{\tilZt \Qtp} $. 
    Inequality $(b)$ follows from the assumptions $\norm{\Zt} \le 2 \sqrt{\norm{X}} $ and $\norm{\Zt \Qtp}\le \frac{ \sqrt{\norm{X}}}{100} $. 
    Next, we note that
     \begin{align*}
        \norm{  P_{\tilZt Q_t}^T P_{\Zt Q_t} } &\le  \norm{  P_{\tilZt Q_t}^T \LAP \LAPT P_{\Zt Q_t} } +  \norm{  P_{\tilZt Q_t}^T \LA \LAT P_{\Zt Q_t} }  \\
        &\le \norm{ \LAPT P_{\Zt Q_t} } + \norm{  P_{\tilZt Q_t}^T \LA  }\\
        &= \norm{ \LAPT P_{\Zt Q_t} } + \norm{  P_{\Zt \Qt}^T \tilLA  }\\
        &\le  \norm{  \LAPT P_{\Zt Q_t}  } + \norm{ P_{\Zt \Qt}^T \LAP }\\
        &\le \frac{1}{50}  ,
    \end{align*}
    where we have used the assumption $\norm{\LAPT L_{\Zt \Qt}} \le \frac{1}{100}  $.
     Hence, we have shown that 
     \begin{equation}\label{ineq:aux51}
        \norm{\tilZt \tilZtT \Zt} 
        \le \left( \frac{4}{50}+\frac{1}{30} \right)  \norm{X} \norm{\Zt}
        \le \frac{\norm{X} \norm{\Zt}}{5}.
     \end{equation}
    From \eqref{ineq:aux52} and the triangle inequality it follows that
     \begin{align}
     \norm{ \Zplus } 
     &\le  \norm{ \left( \Id - \mu  \Zt \ZtT\right)\Zt } + \mu \norm{ \tilZt \tilZtT \Zt  } + \mu \norm{\symA} \norm{\Zt}  + \mu  \norm{ \Deltat} \norm{\Zt} \nonumber  \\
     &\stackrel{(a)}{=}  \norm{ \left( \Id - \mu  \Zt \ZtT\right)\Zt } +   \mu \norm{ \tilZt \tilZtT \Zt  }+   \mu \norm{X} \norm{\Zt} + \mu \norm{ \Deltat} \norm{\Zt} \nonumber \\ 
     &\stackrel{(b)}{\le}  \norm{ \left( \Id - \mu  \Zt \ZtT\right)\Zt } +   \mu  \frac{ \norm{X} \norm{\Zt}}{5} +   \mu \norm{X} \norm{\Zt} + \mu \norm{ \Deltat} \norm{\Zt} \nonumber \\ 
     &\stackrel{(c)}{\le}  \norm{ \left( \Id - \mu  \Zt \ZtT\right)\Zt } +  2 \mu \norm{X} \norm{\Zt} \nonumber \\
     & \stackrel{(d)}{=} \left( 1- \mu \norm{\Zt}^2\right) \norm{\Zt} +  2 \mu \norm{X} \norm{\Zt} \nonumber\\
    &= \left( 1- \mu \norm{\Zt}^2 + 2 \mu \norm{X} \right) \norm{\Zt}. \label{ineq:aux24}
     \end{align}
    In equality $(a)$ we used the fact that $\norm{\symA}=\norm{X}$, which follows from the definition of $\symA$.
    Inequality $(b)$ follows from \eqref{ineq:aux51}. 
    In inequality $(c)$ we used the assumption $ \norm{\Deltat} \le \frac{\norm{X}}{100} $.
    Equality $(d)$ follows from the singular value decomposition of $ \left( \Id - \mu  \Zt \ZtT\right)\Zt $ and $\Zt$, the fact that the function $x \mapsto (1-\mu x^2)x $ is increasing in the interval $ x\in \left( 0,\frac{1}{\sqrt{3 \mu} } \right) $, as well as the assumptions $ \mu \le \frac{c}{\norm{X}} $ and $\norm{\Zt} \le 2\sqrt{\norm{X}} $.
    
    In order to deduce the claim from inequality \eqref{ineq:aux24}, we will distinguish two cases.
    First, we assume that $ \norm{\Zt} < \frac{3}{2} \sqrt{\norm{X}} $. 
    Then the claim $\norm{\Zplus} \le 2\sqrt{\norm{X}} $ follows immediately from inequality \eqref{ineq:aux24} combined with the assumption $ \mu \le \frac{1}{100 \norm{X}}  $. 
    For the second case, we assume that $ \frac{3}{2} \sqrt{\norm{X}} \le \norm{\Zt} \le 2 \sqrt{ \norm{X} } $. 
    Then we can verify that \eqref{ineq:aux24} implies that $ \norm{\Zplus} \le \norm{\Zt} $.
    Since we assumed that $ \norm{\Zt} \le 2 \sqrt{ \norm{X} } $, this implies in particular that $ \norm{\Zplus} \le 2 \sqrt{ \norm{X} } $. 
    This finishes the proof.
\end{proof}