
\begin{proof}[Proof of Lemma \ref{lemma:RIPlemma}]
First we compute that 
\begin{equation}\label{ineq:RIPaux6} 
    \begin{split}
    \Zt \PZQ \PZQT \ZtT -\tilZt \PZQ \PZQT \tilZtT 
    &=   \frac{1}{2} \begin{bmatrix}
        V_t \PZQ \\
        W_t \PZQ
       \end{bmatrix} 
       \begin{bmatrix}
        V_t \PZQ \\
        W_t \PZQ
       \end{bmatrix}^T
       - \frac{1}{2} \begin{bmatrix}
        V_t \PZQ \\
        -W_t \PZQ
       \end{bmatrix} 
       \begin{bmatrix}
        V_t \PZQ \\
        -W_t \PZQ
       \end{bmatrix}^T \\
    &= 
    \begin{bmatrix}
    0 & V_t \PZQ \PZQT W_t^T \\
    W_t \PZQ \PZQT V_t^T & 0.
    \end{bmatrix}.
    \end{split}
\end{equation}
It follows that 
\begin{align*}
    \innerproduct{ B_i, \Zt \PZQ \PZQT \ZtT -\tilZt \PZQ \PZQT \tilZtT  }
    = \sqrt{2} \innerproduct{ A_i,V_t \PZQ \PZQT W_t^T  }.
\end{align*}
Analogously, we can compute that
\begin{align*}
\innerproduct{ B_i, \symA } = \sqrt{2} \innerproduct{ A_i,  X }.
\end{align*}
We obtain that
\begin{align*}
     &\left( \mathcal{B}^* \mathcal{B} \right) \left( \Zt \PZQ \PZQT \ZtT -\tilZt \PZQ \PZQT \tilZtT   -\symA \right)\\
    =& \sum_{i=1}^m B_i \innerproduct{ B_i, \Zt \PZQ \PZQT \ZtT -\tilZt \PZQ \PZQT \tilZtT   -\symA } \\
    =& \sqrt{2}  \sum_{i=1}^m B_i \innerproduct{ A_i,V_t \PZQ \PZQT W_t^T -X  }\\
    =&  \sum_{i=1}^m  \begin{bmatrix}
        0 & A_i \\
        A_i^T & 0 
    \end{bmatrix} \innerproduct{ A_i,V_t \PZQ \PZQT W_t^T  -X }  \\
    =&
     \begin{bmatrix}
        0 & \sum_{i=1}^m  A_i  \innerproduct{ A_i,V_t \PZQ \PZQT W_t^T  -X } \\
        \sum_{i=1}^m  A_i^T  \innerproduct{ A_i,V_t \PZQ \PZQT W_t^T  -X } & 0 
    \end{bmatrix}
    \\
    =& 
    \begin{bmatrix}
        0 & \left( \mathcal{A}^* \mathcal{A} \right) \left(  V_t \PZQ \PZQT W_t^T  - X \right) \\
        \left[ \left( \mathcal{A}^* \mathcal{A} \right) \left(  V_t \PZQ \PZQT W_t^T  - X \right) \right]^T & 0 
    \end{bmatrix}.
\end{align*}
Combining this last equality with equation \eqref{ineq:RIPaux6} and the definition of $\symA$ we obtain that
\begin{align*}
    &\left(\Id - \mathcal{B}^* \mathcal{B} \right) \left( \Zt \PZQperp \PZQperpT \ZtT -\tilZt\PZQperp \PZQperpT \tilZtT - \symA  \right)\\
    =     
    &\begin{bmatrix}
        0 & \left( \Id - \mathcal{A}^* \mathcal{A} \right) \left(   V_t \PZQ \PZQT W_t^T  -X \right)    \\
        \left[ \left( \Id - \mathcal{A}^* \mathcal{A} \right) \left(  V_t \PZQ \PZQT W_t^T  - X \right) \right]^T  &  0
    \end{bmatrix}.
\end{align*}
This implies that 
\begin{equation}\label{ineq:RIPaux1}
    \begin{split}
    &\norm{ \left(\Id - \mathcal{B}^* \mathcal{B} \right) \left( \Zt \PZQ \PZQT \ZtT -\tilZt \PZQ \PZQT \tilZtT   -\symA \right)  }\\
    =
    &\norm{  \left( \Id - \mathcal{A}^* \mathcal{A} \right) \left(   V_t \PZQ \PZQT W_t^T  -X\right)  }.
    \end{split}
\end{equation}
Note that $V_t \PZQ \PZQT W_t^T  - X $ has rank at most $2r$.
Hence, one can use the Restricted Isometry Property to show that (see, e.g., \cite[Lemma 7.3]{stoger2021small} or \cite{candesplan}) 
\begin{equation}\label{ineq:RIPaux2}
    \norm{  \left( \Id - \mathcal{A}^* \mathcal{A} \right) \left(   V_t \PZQ \PZQT W_t^T  -X \right)  }
    \le \delta \sqrt{r} \norm{ V_t \PZQ \PZQT W_t^T  - X  }.
\end{equation}
To estimate the right-hand side further note that it follows from equation \eqref{ineq:RIPaux6}  and the definition of $\symA$ that
\begin{equation*}
\begin{bmatrix}
    0 & V_t \PZQ \PZQT W_t^T  -X \\
    W_t \PZQ \PZQT V_t^T  -X^T & 0 
\end{bmatrix}
=\Zt \PZQ \PZQT \ZtT -\tilZt \PZQ \PZQT \tilZtT   -\symA,
\end{equation*}
which implies that
\begin{equation}\label{ineq:RIPaux5}
    \norm{ V_t \PZQ \PZQT W_t^T  - X  } = \norm{ \Zt \PZQ \PZQT \ZtT -\tilZt \PZQ \PZQT \tilZtT   -\symA }.
\end{equation}
By combining inequalities \eqref{ineq:RIPaux1}, \eqref{ineq:RIPaux2}, and \eqref{ineq:RIPaux5} we obtain inequality \eqref{ineq:RIPclaim1}.

It remains to prove inequality \eqref{ineq:RIPclaim2}. 
Using an analogous computation as in the proof of inequality \eqref{ineq:RIPaux1} we can show that 
\begin{equation}\label{ineq:RIPaux3}
    \begin{split}
    \norm{ \left(\Id - \mathcal{B}^* \mathcal{B} \right) \left( \Zt \PZQperp \PZQperpT \ZtT -\tilZt \PZQperp \PZQperpT \tilZtT   \right)  }
    =
    &\norm{  \left( \Id - \mathcal{A}^* \mathcal{A} \right) \left(   V_t \PZQperp \PZQperpT W_t^T  \right)  }.    
    \end{split}
\end{equation} 
Next, consider the singular value decomposition $V_t \PZQperp \PZQperpT W_t^T= \sum_{i=1}^{k-r} \sigma_i v_i w_i^T $.
It follows that
\begin{align}
    \norm{  \left( \Id - \mathcal{A}^* \mathcal{A} \right) \left(   V_t \PZQperp \PZQperpT W_t^T  \right)  }
    &\le \sum_{i=1}^{k-r} \sigma_i \norm{  \left( \Id - \mathcal{A}^* \mathcal{A} \right) \left(  v_i w_i^T \right)  } \nonumber\\
    &\stackrel{(a)}{\le}  \delta  \sum_{i=1}^{k-r} \sigma_i \nonumber \\
    &= \delta \nucnorm{ V_t \PZQperp \PZQperpT W_t^T } \nonumber \\
    &\le \delta \left( k-r \right) \norm{ V_t \PZQperp \PZQperpT W_t^T } \nonumber \\
    &\stackrel{(b)}{=} \delta \left( k-r \right) \norm{ \Zt \PZQperp \PZQperpT \ZtT -\tilZt \PZQperp \PZQperpT \tilZtT  } \label{ineq:RIPaux4},
\end{align}
where inequality $(a)$ follows from the Restricted Isometry Property, see, e.g., \cite[Proof of Lemma 7.3]{stoger2021small}. 
Equality $(b)$ follows from the equality 
\begin{equation*}
\begin{pmatrix}
    0 & V_t \PZQperp \PZQperpT W_t^T \\
    W_t \PZQperp \PZQperpT V_t^T   & 0
\end{pmatrix}
    = \Zt \PZQperp \PZQperpT \ZtT -\tilZt \PZQperp \PZQperpT \tilZtT.    
\end{equation*}
Combining inequalities \eqref{ineq:RIPaux3} and \eqref{ineq:RIPaux4} we obtain inequality \eqref{ineq:RIPclaim2}.

Inequality \eqref{ineq:RIPclaim3} can be proven similarly as inequality \eqref{ineq:RIPclaim2}, which is why we omit the details. 
This completes the proof of Lemma \ref{lemma:RIPlemma}.
\end{proof}
