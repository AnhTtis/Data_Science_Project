The goal of this section is to prove Lemma \ref{lemma:spectralmain}.
For that, we will closely trace the proof for the spectral phase in \cite{stoger2021small}.
First, we need to introduce several definitions.
We define
\begin{equation*}
    F:= \left( \Bcal^* \Bcal \right) (\symA).
\end{equation*}
Moreover, for all natural numbers $t$ we define
\begin{align}
\Zt' &:= \left(\I + \mu F \right)^tZ_0, \label{def:Ztprime}\\
G_t &:= \left(\I + \mu F \right)^t, \nonumber\\
E_t &:= \Zt - \Zt'.\nonumber
\end{align}
Denote by $Z_{t}:=\sum_{i=1}^{k} \sigma_{i} u_{i} v_{i}^{T}$ the singular value decomposition of $Z_{t}$.
We define $M_{t}:=\sum_{i=1}^{r} \sigma_{i} u_{i} v_{i}^{T}$ and $N_{t}:=\sum_{i=r+1}^{k} \sigma_{i} u_{i} v_{i}^{T}$.

The first lemma shows how close the iterates $\Zt$ stay to power method iterates $\Zt'$ with respect to the spectral norm.
\begin{lemma}\label{lemma:spectral1}
Assume that $\norm{Z_0}^2\le  \frac{\norm{F}}{16}$ and that $\mathcal{A}$ satisfies the restricted isometry property of order $2r+1 $ with constant $\delta <1 $. Then, for all integers $t$ such that 
\begin{equation*}
0 \le t \le \frac{\ln\left(\frac{\norm{F}}{16  \min \left\{ k;n_1+n_2\right\} \norm{Z_0}^2}\right)}{3\ln \left( 1+\mu  \norm{F} \right)}
\end{equation*}
it holds that
\begin{equation}\label{ineq:spectral_aux1}
\norm{E_t} \le \frac{16}{ \norm{F} } \min \left\{  k; n_1 + n_2  \right\}  (1 + \mu \norm{F})^{3t} \norm{Z_0}^3 \le \norm{Z_0}.
\end{equation}
\end{lemma}
\begin{proof}[Proof of Lemma \ref{lemma:spectral1}]
We define
\begin{equation*}
 \hat{E}_{i}:= \mu \left[  \Bcal^* \Bcal (Z_{i-1}Z_{i-1}^{T} - \tilde{Z}_{i-1}\tilde{Z}_{i-1}^T)\right] Z_{i-1}.
\end{equation*}
To prove the lemma, we will first establish the following auxiliary equation for any natural number $t \ge 1$. 
\begin{equation}\label{ineq:claim}
Z'_{t}-Z_{t} = \sum_{i=1}^{t} \left( \Id + \mu F \right)^{t-i}\hat{E}_{i}.
\end{equation}
\noindent \textbf{Proof of the equation \eqref{ineq:claim}:}
We will prove this equation via induction.
For $t=1$ we note that
\begin{align*}
    Z_1 = & Z_0 - \mu \left[  \left(  \Bcal^* \Bcal \right) (Z_0Z_0^T - \tilde{Z}_0\tilde{Z}_0^T - \symA) \right] Z_0 \\
    = & (\Id + \mu F)Z_0 - \mu  \left[  \left(\Bcal^* \Bcal \right) (Z_0Z_0^T - \tilde{Z}_0\tilde{Z}_0^T) \right] Z_0 \\
    = & Z_1' - \hat{E}_1,
\end{align*}
which proves the claim for $t = 1$.
Now assume that equation \eqref{ineq:claim} holds for a natural number $t\ge 2$.
We obtain that
\begin{align*}
    \Zplus = & \Zt - \mu  \left[  \left(\Bcal^* \Bcal \right) \left(\Zt\ZtT - \tilZt\tilZtT - \symA \right) \right] \Zt \\
    = & \left(\Id + \mu F \right) \Zt - \mu \left[ \left(\Bcal^* \Bcal \right)  \left(\Zt\ZtT - \tilZt\tilZtT \right) \right] \Zt \\
    \stackrel{(a)}{=} & \left( \left( \Id + \mu F \right) Z'_t - \sum_{i=1}^{t} \left( \Id + \mu F \right)^{t-i+1}\hat{E}_{i}\right) - \hat{E}_{t+1} \\
    \stackrel{(b)}{=} & Z'_{t+1} - \sum_{i=1}^{t} \left(\Id + \mu F \right)^{t-i+1}\hat{E}_{i} - \hat{E}_{t+1}\\ 
    = & Z'_{t+1} - \sum_{i=1}^{t+1} \left( \Id + \mu F \right)^{t+1-i}\hat{E}_{i},
\end{align*}
where in equality $(a)$ we have used the induction hypothesis and in equality $(b)$ we used the definition of $Z'_{t+1} $.
This shows the induction step for $t + 1$ and, hence, equation \eqref{ineq:claim} is shown for any natural number $t\ge 1$.\\

\noindent  In order to bound $\norm{E_t} = \|Z'_{t}-Z_{t}\|$ we will again proceed by induction. First, note that for the induction step $t=0$ inequality \eqref{ineq:spectral_aux1} holds true since $Z_0=Z'_0$.
Now let $t\ge 1$ be a natural number.
We observe that for any natural number $ t \ge 1  $ we have that
\begin{align*}
    \norm{ \hat{E}_{i} } \le & \mu  \norm{ \left( \Bcal^* \Bcal \right) \left(Z_{i-1}Z_{i-1}^{T} - \tilde{Z}_{i-1}\tilde{Z}_{i-1}^T \right) }  \norm{Z_{i-1}} \\
    \stackrel{(a)}{\le} & \left(1+\delta\right) \mu \nucnorm{  (Z_{i-1}Z_{i-1}^{T} - \tilde{Z}_{i-1}\tilde{Z}_{i-1}^T) }  \norm{Z_{i-1} } \\
    \le & 2 \mu \nucnorm{  (Z_{i-1}Z_{i-1}^{T} - \tilde{Z}_{i-1}\tilde{Z}_{i-1}^T) } \norm{Z_{i-1}} \\
    \le & 2 \mu \left(  \nucnorm{  \left(Z_{i-1}Z_{i-1}^{T}} + \nucnorm{ \tilde{Z}_{i-1}\tilde{Z}_{i-1}^T \right) }\right) \norm{Z_{i-1}} \\ 
    \stackrel{(b)}{=} & 4 \mu  \nucnorm{  Z_{i-1}Z_{i-1}^{T}}  \norm{Z_{i-1}} \\
    = & 4 \mu  \fbnorm{  Z_{i-1}}^2  \norm{Z_{i-1}} \\
    \le & 4 \mu \min \left\{ k;n_1+n_2\right\} \norm{Z_{i-1}}^3 \\
    \le & 4\mu  \min \left\{ k;n_1+n_2\right\}   \left( \norm{Z'_{i-1}} + \norm{ E_{i-1}} \right)^3 \\
    \le & 16\mu \min \left\{ k;n_1+n_2\right\}  \left( \norm{Z'_{i-1}}^3 + \norm {E_{i-1} }^3 \right) \\
    \stackrel{(c)}{\le} & 16\mu \min \left\{ k;n_1+n_2\right\}  \left( \norm{ \left( \Id + \mu F \right)^{3i-3} } \norm{Z_0}^3 + \norm{ E_{i-1} }^3 \right) \\
    \le & 16\mu \min \left\{ k;n_1+n_2\right\}  \left(  \left( 1 + \mu \norm{F} \right)^{3i-3}  \norm{Z_0}^3 + \norm{ E_{i-1} }^3 \right) \\
    \stackrel{(d)}{\le} & 32\mu \min \left\{ k;n_1+n_2\right\}  \left(1 + \mu \norm{F} \right)^{3i-3} \norm{Z_0}^3,
\end{align*}
where in inequality $(a)$ we used the Restricted Isometry Property (see inequality \eqref{ineq:RIPclaim3} in Lemma \ref{lemma:RIPlemma}).
Equality $(b)$ follows from equations \eqref{def:Ztdefinition} and \eqref{def:tilZtdefinition}.
Inequality $(c)$ follows from the definition of $Z'_{i-1}$ and inequality $(d)$ from the induction hypothesis that $ \norm{ E_{i-1} } \le \norm{Z_0}$ for $i \le t$.
Using equation \eqref{ineq:claim} we obtain that
\begin{align*}
    \norm{E_t} = &  \norm{Z_{t} - Z'_{t}} \\
    \le &\sum_{i=1}^{t} \norm{ \left( \Id + \mu F \right)^{t-i} } \norm{ \hat{E}_{i} } \\
    \le &32\mu \min \left\{ k;n_1+n_2\right\} \sum_{i=1}^{t}(1 + \mu \norm{F} )^{t+2i-3} \norm{Z_0}^3 \\
    = &32 \mu \min \left\{ k;n_1+n_2\right\} (1 + \mu \norm{F})^{t-1}\frac{(1 + \mu \norm{F} )^{2t} - 1}{(1 + \mu\norm{F})^{2}-1} \norm{Z_0}^3 \\
    \le &32 \mu \min \left\{ k;n_1+n_2\right\} (1 + \mu \norm{F})^{t-1}\frac{(1 + \mu \norm{F} )^{2t+1}}{2\mu \norm{F}} \norm{Z_0}^3 \\
    = &\frac{16}{\norm{F}}  \min \left\{ k;n_1+n_2\right\} \left(1 + \mu\norm{F} \right)^{3t}  \norm{Z_0}^3.
\end{align*}
By the assumption $t \le \frac{\ln\left(\frac{ \norm{F}}{16  \min \left\{ k;n_1+n_2\right\} \norm{Z_0}^2}\right)}{3\ln \left(1+\mu \norm{F}  \right)}$ we have that
\begin{equation*}
\frac{16}{\norm{F}}  \min \left\{ k;n_1+n_2\right\} \left(1 + \mu \norm{F}\right)^{3t}  \norm{Z_0}^3 \le \frac{16}{ \norm{F} }\cdot\frac{\norm{F}}{16  \norm{Z_0}^2 } \norm{Z_0}^3 = \norm{Z_0}.
\end{equation*}
Combining the above two inequalities we have shown the induction step for $t\ge 1$, which finishes the proof.
\end{proof}

Let $Z_{t}=\sum_{i=1}^{k} \sigma_{i} u_{i} v_{i}^{T}$ be the singular value decomposition of $Z_{t}$.
Define $M_{t}:=\sum_{i=1}^{r} \sigma_{i} u_{i} v_{i}^{T}$ and $N_{t}:=\sum_{i=r+1}^{k} \sigma_{i} u_{i} v_{i}^{T}$.
Moreover, recall that by construction 
\begin{align*}
F= \left(\Bcal^* \Bcal \right) \left( \symA \right) = \begin{pmatrix}
    0 &  \left(\mathcal{A}^* \mathcal{A} \right)  (X) \\
   \left[  \left( \mathcal{A}^* \mathcal{A}\right)  (X) \right]^T & 0
\end{pmatrix}
\end{align*}
has the same number of positive and negative eigenvalues. 
Denote by $F_1 $ the subspace spanned by the eigenvectors corresponding to the $r$ largest eigenvalues of $F$.
By $L_{F_1}$ we denote an orthonormal matrix whose column span is equal to $F_1$.

To prove Lemma \ref{lemma:spectralmain}, we also need the following two technical lemmas.
\begin{lemma}\label{lemma:spectral2}
    Assume that
    \begin{equation}\label{ineq:specbound4}
    \lambda_{r+1}(G_{t}) \norm{Z_0}+ \norm{E_{t}}<\lambda_{r}(G_{t}) \sglmin \left( L_{F_1}^T Z_0 \right).
    \end{equation}
    Then it holds that
    \begin{align}
    \sigma_{r}(Z_{t}) &\ge \lambda_{r}(G_{t})\sigma_{\min}(L_{F_1}^T Z_0) - \norm{E_t}, \label{ineq:specbound1}\\
    \sigma_{r+1}(Z_{t}) &\le \lambda_{r+1} \left(G_{t} \right ) \norm{Z_0} + \norm{E_t}, \label{ineq:specbound2}\\
    \norm{ L_{F_1,\bot}^T P_{M_t}}  &\le \frac{\lambda_{r+1}(G_t) \norm{Z_0}+ \norm{E_t}}{ \lambda_{r}\left( G_t \right)\sglmin \left( L_{F_1}^T Z_0 \right) - \lambda_{r+1}(G_t) \norm{Z_0}- \norm{E_t}}. \label{ineq:specbound3}
    \end{align}
    \end{lemma}
\begin{lemma}\label{lemma:spec3}
    Assume that $ \norm{ \LAPT P_{M_{t}}} \le \frac{1}{8}$. Then the following inequalities hold:
    \begin{align}
        \sigma_{r}(Z_tQ_t)&\ge \frac{1}{2}\sigma_{r} \left(Z_t \right),\label{SpecZtQt}\\
        \norm{ \LAPT P_{Z_tQ_t}} &\le 7 \norm{\LAPT P_{M_{t}}},\label{Specangle}\\
        \norm{Z_tQ_{t,\bot}}&\le 2\sigma_{r+1} \left(Z_t\right).\label{SpecZtQtp}
    \end{align}
\end{lemma}
These two lemmas are analogous to Lemma 8.3 and Lemma 8.4 in \cite{stoger2021small} and can be proven with exactly the same arguments, which is why we skip the details. 

The next lemma shows that after a certain amount of iterations, the signal part $ \Zt \Qt$ is sufficiently well-aligned with the ground truth signal and,
moreover, that the singular values of the signal part and the spectral norm of the nuisance part are sufficiently separated.
\begin{lemma}\label{lemma:spec4}
Assume that 
\begin{equation}\label{spectral:closeness}
    \norm{  \mathcal{A}^* \mathcal{A} \left( X \right) -X   } \le \frac{ c \sglmin (X)}{\kappa^2}
\end{equation}
and
\begin{equation}\label{definition:tstar}
t_\star := 
\left\lceil     \frac{\ln\left(\frac{c\sglmin \left(L_{F_1}^T Z_0\right)}{2\kappa^{2} \norm{Z_0} }\right)}{\ln \left(1 - \frac{\mu\sglmin(X)}{8} \right)} \right\rceil
\le \frac{\ln\left(\frac{\norm{F}}{16  \min \left\{ k;n_1+n_2\right\} \norm{Z_0}^2}\right)}{3\ln \left( 1+\mu  \norm{F} \right)}
\end{equation}
for a positive constant $ c \le \frac{1}{32}$.
Moreover, assume that the step size $\mu$ satisfies $ \mu \le \frac{\tilde{c}}{\sglmin (X)} $, where $\tilde{c}>0$ is a sufficiently small absolute constant.
Then it holds that
\begin{align}
\norm{L_{X,\bot}^{T} P_{Z_{t_\star}Q_{t_\star}}}&\le \frac{28c}{\kappa^2},\label{ineq:spectral2}\\
\sigma_{\min}(Z_{t_\star}Q_{t_\star}) 
&\ge \left(\frac{2\kappa^{2} \norm{Z_0} }{c\sglmin \left(L_{F_1}^T Z_0\right)}\right)^{2\kappa} \frac{\sglmin \left( L_{F_1}^T Z_0 \right)}{4} ,\label{ineq:spectral3}\\
\norm{Z_{t_\star}Q_{t_\star,\bot}} 
&\le \min \left\{ 2 \sglmin \left( \Ztstar Q_{\tstar} \right); \  \left( \frac{2\kappa^2 \norm{Z_0}}{c \sglmin \left( L_{F_1}^T Z_0 \right)} \right)^{ 16 \kappa} \cdot \frac{4c \sglmin \left( L_{F_1}^T Z_0 \right)}{ \kappa^2 } \right\} .\label{ineq:spectral4}
\end{align}
\end{lemma}

\begin{proof}[Proof of Lemma \ref{lemma:spec4}]
Before proving inequalities \eqref{ineq:spectral2}, \eqref{ineq:spectral3}, and \eqref{ineq:spectral4}, we first prove the following auxiliary inequality
\begin{equation}
\gamma:= \frac{\lambda_{r+1}(G_{t_\star}) \norm{Z_0}+\norm{E_{t_\star}}}{\lambda_{r}(G_{t_\star}) \sglmin\left( L_{F_1}^T Z_0  \right)} \le \frac{c}{\kappa^2}. \label{ineq:spectral1} 
\end{equation}
For that, we recall that
\begin{equation*}
 F
 = \left(\Bcal^* \Bcal \right) \left( \symA \right)
 = \begin{pmatrix}
    0 &   \left( \mathcal{A}^* \mathcal{A} \right)  (X)  \\
   \left[  \left(\mathcal{A}^* \mathcal{A}  (X) \right) \right]^T  & 0
\end{pmatrix}.   
\end{equation*}
It follows by Weyl's inequality and assumption \eqref{spectral:closeness} that 
\begin{equation}\label{ineq:Weyl1}
    \lambda_{r} \left( F \right)
    =\sigma_{r} \left(\left(\mathcal{A}^* \mathcal{A}  (X) \right)  \right)
    \ge \sigma_{\min} \left( X \right) - \norm{\left(\mathcal{A}^* \mathcal{A}  (X) \right)-X }
    \ge \frac{\sigma_{\min} \left( X \right)}{2}.
\end{equation}
Again using Weyl's inequality, assumption \eqref{spectral:closeness},  and, in addition, $ \sigma_{r+1 } (X) =0  $ we can derive that
\begin{equation}\label{ineq:Weyl2}
    \lambda_{r+1} \left( F \right)  
    =\sigma_{r+1} \left(\left(\mathcal{A}^* \mathcal{A}  (X) \right)  \right)
    \le \norm{\left(\mathcal{A}^* \mathcal{A}  (X) \right)-X }
    \le \frac{\sigma_{\min} \left( X \right)}{4}.
\end{equation}
Next, we note that for any $1\le i \le n_1+n_2$ it holds that
\begin{align*}
    \lambda_{i} \left( G_{t_\star} \right) 
    = \lambda_{i} \left( \left( 1+ \mu F  \right)^{t_\star}  \right)
    = \left(  \lambda_{i} \left( 1 + \mu F \right)  \right)^{t_\star}
    = \left(  1+ \mu \lambda_{i} \left( F \right)  \right)^{t_\star},
\end{align*}
since we have assumed that our step size satisfies $\mu \le \frac{\tilde{c}}{\sglmin (X)} \le \frac{1}{2 \norm{F}}$ (where the second inequality is also due to \eqref{spectral:closeness}).
Combining this observation with inequalities \eqref{ineq:Weyl1} and \eqref{ineq:Weyl2} it follows that
\begin{align}
    \lambda_{r} \left( G_ {t_\star}\right) & \ge  \left(  1+ \mu \frac{\sglmin (X)}{2}  \right)^{t_\star},\label{ineq:intern401}\\
    \lambda_{r+1} \left( G_{t_\star}\right) &\le \left(  1+ \mu \frac{\sglmin (X)}{4}  \right)^{t_\star}.  \label{ineq:intern402}
\end{align}
Thus, we obtain that
\begin{align*}
\gamma 
\stackrel{(a)}{\le} & \frac{\left(  1+ \mu \frac{\sglmin (X)}{4}  \right)^{t_\star} \norm{Z_0} + \norm{E_{t_\star} }}{\left(  1+ \mu \frac{\sglmin (X)}{2}  \right)^{t_{\star}} \sglmin \left( L_{F_1}^T Z_0  \right)}\\
\stackrel{(b)}{\le} & \frac{\left( \left(  1+ \mu \frac{\sglmin (X)}{4}  \right)^{t_{\star}} +1 \right) \norm{Z_0} }{\left(  1+ \mu \frac{\sglmin (X)}{2}  \right)^{t_\star} \sglmin \left( L_{F_1}^T Z_0  \right)}\\
\le & \left( \frac{1+ \frac{\mu \sglmin (X)}{4}}{1+ \frac{\mu \sglmin (X)}{2}}  \right)^{t_{\star}}  \frac{ 2\norm{Z_0} }{\sglmin \left( L_{F_1}^T Z_0  \right) }\\ 
= & \left( 1 - \mu \frac{ \frac{ \sglmin (X)}{4}}{1+ \frac{\mu \sglmin (X)}{2}}  \right)^{t_{\star}}  \frac{ 2\norm{Z_0} }{ \sglmin \left( L_{F_1}^T Z_0  \right) }\\ 
\stackrel{(c)}{\le} & \left( 1 - \frac{\mu \sglmin (X)}{8}  \right)^{t_{\star}}  \frac{ 2\norm{Z_0} }{\sglmin \left( L_{F_1}^T Z_0  \right) },\\ 
\end{align*}
where in inequality $(a)$ we used the definition of $\gamma$ and inequalities \eqref{ineq:intern401} and \eqref{ineq:intern402}.
Inequality $(b)$ follows from $\norm{E_{t_{\star}}} \le \norm{Z_0} $, which is a consequence of Lemma \ref{lemma:spectral1} and the definition of $t_{\star}$, see assumption \eqref{definition:tstar}.
Inequality $(c)$ is due to $\mu \le \frac{2}{\sglmin (X)}$, which follows from our assumption on the step size $\mu$.
It follows from the definition of $\tstar$, see \eqref{definition:tstar}, that
\begin{align}\label{ineq:gammabound1}
    \gamma
    \le \exp \left( t_{\star} \ln \left( 1 - \mu \frac{ \sglmin (X)  }{8}  \right) \right)  \frac{ 2\norm{Z_0} }{ \sglmin \left( L_{F_1}^T Z_0  \right) }
    \le \frac{c}{\kappa^2}.
\end{align}
This shows inequality \eqref{ineq:spectral1} and now we are in a position to prove inequalities \eqref{ineq:spectral2}, \eqref{ineq:spectral3}, and \eqref{ineq:spectral4}.\\

\noindent\textbf{Proof of inequality \eqref{ineq:spectral2}:}
First, we note that it follows from the Davis-Kahan $\sin \Theta$-Theorem (see \cite{kahanbound}) and assumption \eqref{spectral:closeness} that
\begin{equation}\label{spectral:DavisKahanapplication}
    \norm{ \LAPT L_{F_1} }
    =\norm{ L_{F_1,\bot}^T \LA } 
    \le \frac{  \norm{F-\symA}}{\lambda_{r} (\symA)-\norm{F-\symA}} 
    = \frac{\norm{   \left(\mathcal{A}^* \mathcal{A}  (X) \right) - X }}{\sigma_{\min} (X)-\norm{   \left(\mathcal{A}^* \mathcal{A}  (X) \right) -X}} 
    \le \frac{2c}{\kappa^2}.
\end{equation}
Next, we observe that
\begin{align}\label{ineq:intern369}
    \frac{ \lambda_{r+1} \left(G_{t_\star} \right) \norm{Z_0}+ \norm{E_{t_\star} } }{\lambda_{r}(G_{t_\star}) \sglmin\left( L_{F_1}^T Z_0 \right)- \lambda_{r+1} \left(G_{t_\star} \right) \norm{Z_0}- \norm{E_{t_\star}}}
    \le \frac{2 \left( \lambda_{r+1} \left(G_{t_\star} \right) \norm{Z_0}+ \norm{E_{t_\star}}\right)  }{\lambda_{r}(G_{t_\star}) \sglmin\left( L_{F_1}^T Z_0 \right) }
    \le \frac{2c}{\kappa^2},
\end{align}
where we have used in both inequalities that $ \gamma \le \frac{c}{\kappa^2} \le \frac{1}{2} $, see inequality \eqref{ineq:gammabound1}.
We also observe that
\begin{align}
   \norm{ \LAPT  P_{M_{t_{\star}}}} 
   &\le \norm{ \LAPT L_{F_1} L_{F_1}^T P_{M_{t_{\star}}}}+\norm{L_{X, \bot}^TL_{F_1,\bot} L_{F_1,\bot}^T P_{M_{t_{\star}}}} \nonumber \\
   &\le \norm{ \LAPT L_{F_1} }+\norm{ L_{F_1,\bot}^T P_{M_{t_{\star}}}}\nonumber\\
   &\stackrel{(a)}{\le} \frac{2c}{\kappa^2} +  \frac{\lambda_{r+1} \left(G_{\tstar} \right) \norm{Z_0}+ \norm{E_{\tstar}}}{ \lambda_{r}\left( G_{\tstar} \right)\sglmin \left( L_{F_1}^T Z_0 \right) - \lambda_{r+1}\left(G_{\tstar} \right) \norm{Z_0}- \norm{E_{\tstar}}}\nonumber\\
   &\stackrel{(b)}{\le} \frac{4c}{\kappa^2} 
   \stackrel{(c)}{\le}  \frac{1}{8}, \label{ineq:intern357}
\end{align}
where in inequality $(a)$ we used \eqref{spectral:DavisKahanapplication} and  \eqref{ineq:specbound3} in Lemma \ref{lemma:spectral2}.
(Lemma \ref{lemma:spectral2} is applicable since assumption \eqref{ineq:specbound4} is fulfilled due to \eqref{ineq:spectral1}.)
Inequality $(b)$ follows from inequality \eqref{ineq:spectral1} and the definition of $\gamma$.
Inequality $(c)$ is due to our assumption $c\le \frac{1}{32}$.
Now we can prove inequality \eqref{ineq:spectral2} by observing that
\begin{align}
\norm{\LAPT P_{Z_{t_\star}Q_{t_\star}}}
&\stackrel{(a)}{\le} 7 \norm{ \LAPT P_{M_{t_\star}} } \nonumber \\
&\le 7 \norm{ \LAPT \left( L_{F_1} L_{F_1}^T + L_{F_1, \bot} L_{F_1,\bot}^T \right) P_{M_{t_\star}} }\nonumber\\
&\le 7 \left(   \norm{ \LAPT L_{F_1} } + \norm{ L_{F_1,\bot}^T  P_{M_{t_\star}} } \right)\nonumber\\
& \stackrel{(b)}{\le} 7\norm{ \LAPT  L_{F_1} } + \frac{7 \left( \lambda_{r+1} \left(G_{t_\star} \right) \norm{Z_0}+ \norm{E_{t_\star} } \right)}{\lambda_{r}(G_{t_\star}) \sglmin\left( L_{F_1}^T Z_0 \right)- \lambda_{r+1} \left(G_{t_\star} \right) \norm{Z_0}- \norm{E_{t_\star}}}, \label{ineq:spectral765}
\end{align}
where for inequality $(a)$ we used inequality \eqref{Specangle} in Lemma \ref{lemma:spec3}, where this lemma is applicable since we have that $ \norm{\LAPT P_{Z_{t_\star}Q_{t_\star}} } \le \frac{1}{8}$ due to inequality \eqref{ineq:intern357}.
Inequality $(b)$ follows from Lemma \ref{lemma:spectral2}.
(The assumption in this lemma is fulfilled since we have that $\gamma \le \frac{c}{\kappa^2}<1$.)
Inserting inequality \eqref{spectral:DavisKahanapplication} in \eqref{ineq:spectral765}, we obtain that
\begin{equation*}
\norm{\LAPT P_{Z_{t_\star}Q_{t_\star}}}\le  7\norm{ \LAPT  L_{F_1} } + \frac{14 c }{\kappa^2} \stackrel{\eqref{spectral:DavisKahanapplication}}{\le} \frac{28c}{\kappa^2}. 
\end{equation*}
This proves inequality \eqref{ineq:spectral2}.\\

\noindent\textbf{Proof of inequality \eqref{ineq:spectral3}:}
Recall from the proof of inequality \eqref{ineq:spectral2} that both Lemma \ref{lemma:spectral2} and Lemma \ref{lemma:spec3} are applicable.
Then we note that
\begin{align}
\sglmin \left(Z_{t_\star}Q_{t_\star}\right) 
&\stackrel{(a)}{\ge} \frac{1}{2}\sigma_{r}(Z_{t_\star}) \nonumber\\
&\stackrel{(b)}{\ge} \frac{1}{2} \left(\lambda_{r} \left(G_{t_\star} \right)\sigma_{\min} \left(L_{F_1}^{T} Z_0 \right)-\norm{E_{t_\star}} \right)\nonumber \\
& \stackrel{(c)}{\ge}\frac{1}{4}\lambda_{r}(G_{t_\star})\sigma_{\min}(L_{F_1}^{T} Z_0), \label{spectral:intern62}
\end{align}
where in inequality $(a)$ we used inequality \eqref{SpecZtQt} in Lemma \ref{lemma:spec3}.
Inequality $(b)$ follows from inequality \eqref{ineq:specbound1} in Lemma \ref{lemma:spectral2}.
Inequality $(c)$ is a consequence of $\gamma \le 1/2$.
In order to proceed, we note that 
\begin{align}
    \lambda_{r} \left( G_\tstar \right)
    &\stackrel{(a)}{\ge} \left( 1 + \frac{\mu \sglmin (X)}{2} \right)^\tstar \nonumber \\
    & = \exp \left( \tstar \ln \left(  1 + \frac{\mu \sglmin (X)}{2}\right) \right)\nonumber\\
    &\stackrel{(b)}{\ge} \exp \left( \frac{\ln\left(\frac{c\sglmin \left(L_{F_1}^T Z_0\right)}{2\kappa^{2} \norm{Z_0} }\right)}{\ln \left(1 - \frac{\mu\sglmin(X)}{8} \right)} \ln \left( 1 + \frac{\mu \sglmin (X)}{2} \right) \right)\nonumber\\
    &\stackrel{(c)}{\ge} \exp \left( 2 \ln\left(\frac{2\kappa^{2} \norm{Z_0} }{c\sglmin \left(L_{F_1}^T Z_0\right)}\right)  \sglmin (X)  \right)\nonumber\\
    &= \left(\frac{2\kappa^{2} \norm{Z_0}}{c\sglmin \left(L_{F_1}^T Z_0\right) }\right)^{2\kappa}.\label{spectral:intern61}
\end{align}
For inequality $(a)$ we used inequality \eqref{ineq:intern401} and for inequality $(b)$ we used the definition of $\tstar$.
Inequality $(c)$ follows from the elementary inequality $ \frac{x}{1-x} \le \ln (1+x) $ and from the assumption $ \mu \le \frac{\tilde{c}}{\sglmin (X)} $.
By combining inequalities \eqref{spectral:intern62} and \eqref{spectral:intern61} we obtain inequality \eqref{ineq:spectral3}.\\

\noindent\textbf{Proof of inequality \eqref{ineq:spectral4}:}
Again, recall from the proof of inequality \eqref{ineq:spectral2} that both Lemma \ref{lemma:spectral2} and Lemma \ref{lemma:spec3} are applicable.
We observe that
\begin{align}
\norm{Z_{t_\star}Q_{t_\star,\bot}}
&\stackrel{(a)}{\le} 2\sigma_{r+1}(Z_{\tstar})\nonumber\\
&\stackrel{(b)}{\le} 2 \left(\lambda_{r+1}(G_{t_\star})    \norm{Z_0}+ \norm{E_{t_\star}} \right)\nonumber \\
&\stackrel{(c)}{\le} \frac{2c \lambda_{r}(G_{\tstar})\sigma_{\min}( L_{F_1}^T Z_0)}{\kappa^2}. \label{spectral:intern51}
\end{align}
Inequality $(a)$ follows from \eqref{SpecZtQtp}, see Lemma \ref{lemma:spec3}, and inequality $(b)$ follows from \eqref{ineq:specbound2}, see Lemma \ref{lemma:spectral2}.
Inequality $(c)$ follows from $\gamma \le \frac{c}{\kappa^2}$, see inequality \eqref{ineq:spectral1}.
By combining inequalities \eqref{spectral:intern62} and \eqref{spectral:intern51} and using that $ c \le 1/32 $ we observe that 
\begin{equation}\label{spectral:intern81}
 \norm{Z_{t_\star}Q_{t_\star,\bot}} \le 2 \sglmin \left( \Ztstar Q_{\tstar} \right). 
\end{equation}
Note that by Weyl's inequality
\begin{equation*}
    \lambda_{r} \left( F \right)
    = \sigma_{r} \left( \mathcal{A}^* \mathcal{A} \left( X \right) \right)
    = \sigma_{\min} \left( X  \right) + \norm{  \left( \mathcal{A}^* \mathcal{A} \right) \left(X\right) -X  }
    \le \frac{5}{4} \sigma_{\min} \left( X\right).
\end{equation*}
Thus, we can compute 
\begin{align*}
    \lambda_{r} \left(G_{\tstar}\right)
    &= \left( 1+ \mu \lambda_{r} \left( F \right) \right)^\tstar \\
    &\le \left( 1+ \frac{5 \mu \sigma_{\min} \left(X\right)  }{4}  \right)^\tstar\\
    &=  \exp \left( \tstar \ln \left( 1+ \frac{5 \mu \sigma_{\min} \left(X\right) }{4} \right) \right)\\
    &\stackrel{(a)}{\le} \exp \left( \left( \frac{  \ln \left( \frac{c \sglmin \left( L_{F_1}^T Z_0 \right)}{2\kappa^2 \norm{Z_0}} \right)}{\ln \left( 1- \frac{\mu \sglmin \left( X \right)}{8} \right)} +1 \right) \ln \left( 1+ \frac{5 \mu \sigma_{\min} \left(X\right) }{4} \right) \right)\\
    &\stackrel{(b)}{\le} 2 \left( \frac{2\kappa^2 \norm{Z_0}}{c \sglmin \left( L_{F_1}^T Z_0 \right)} \right)^{16 \kappa}.
\end{align*}
For inequality $(a)$ we used the definition of $\tstar$ and for inequality $(b)$ we used the elementary inequality $ \frac{x}{1-x} \le \ln \left(1+x\right) \le x $ and the assumption that $ \mu \le \frac{ \tilde{c} }{\sglmin (X)} $.
By combining this inequality chain with inequality \eqref{spectral:intern51} we obtain that 
\begin{equation}\label{spectral:intern82}
\norm{ \Ztstar Q_{t_\star,\bot}} \le \left( \frac{2\kappa^2 \norm{Z_0}}{c \sglmin \left( L_{F_1}^T Z_0 \right)} \right)^{16 \kappa} \frac{4c \sglmin \left( L_{F_1}^T Z_0 \right)}{\kappa^2}.
\end{equation}
By combining \eqref{spectral:intern81} and \eqref{spectral:intern82} we obtain inequality \eqref{ineq:spectral4}. 
This finishes the proof of Lemma \ref{lemma:spec4}.
\end{proof}
We also need to check that after the spectral phase also the conditions related to the imbalance term are fulfilled.
\begin{lemma}\label{lemma:spec5}
Let $0< c\le 1/32$. Assume that
\begin{equation*}
    \norm{X - \left(\mathcal{A}^* \mathcal{A}\right) \left(X\right) } \le \frac{c \sglmin \left(X\right)}{\kappa^2}
\end{equation*}
and that the measurement operator $\mathcal{A}$ satisfies the restricted isometry property of order $2r+1$ with constant $\delta <1$.
Moreover, assume that $ \mu \le \frac{\tilde{c}}{\sglmin (X)} $, where $\tilde{c}>0$ is a sufficiently small absolute constant.
If 
\begin{equation*}
    \tstar := \left\lceil\frac{\ln\left(\frac{c\sglmin \left(L_{F_1}^{T} Z_0\right)}{2\kappa^{2}\norm{Z_0}}\right)}{\ln\left(1 - \frac{\mu\sglmin(X)}{8}\right)}\right\rceil 
    \le \frac{\ln\left(\frac{\norm{F}}{16  \min \left\{ k;n_1+n_2\right\} \norm{Z_0}^2}\right)}{3\ln \left( 1+\mu  \norm{F} \right)},
\end{equation*}
then it holds that 
\begin{align}
    \norm{Z_{\tstar}} &\le 4 \left(\frac{2\kappa^{2} \norm{Z_0}}{c\sglmin \left(L_{F_1}^{T}Z_0 \right)}\right)^{32 \kappa} \norm{Z_0},\label{ineq:spectral_balanc4}\\
    \norm{\tilde{Z}_{\tstar}^T Z_{\tstar}  }&\le 6 \left(\frac{2\kappa^{2} \norm{Z_0}}{c\sglmin(L_{F_1}^{T}Z_0)}\right)^{32 \kappa}  \norm{Z_0}^2, \label{ineq:spectral_balanc1} \\ 
    \norm{\tilde{Z}_{\tstar}^T Z_{\tstar} Q_{\tstar, \bot}  }&\le  4 \left(\frac{2\kappa^{2} \norm{Z_0}}{c\sglmin \left(L_{F_1}^{T}Z_0 \right)}\right)^{32 \kappa}  \norm{Z_0}  \norm{\Ztstar Q_{\tstar, \bot}} ,\label{ineq:spectral_balanc2} \\ 
    \norm{\tilde{Z}_{\tstar}^T P_{ Z_{\tstar} Q_{\tstar} }  }&\le \frac{228 c}{\kappa^2}  \left( \frac{2\kappa^2 \norm{Z_0}}{ c \sglmin \left( L_{F_1}^T Z_0 \right) } \right)^{32 \kappa} \norm{Z_0} . \label{ineq:spectral_balanc3}
\end{align}
\end{lemma}

\begin{proof}
\noindent\textbf{Proof of inequality \eqref{ineq:spectral_balanc4}:}
For that, we first observe that
\begin{align}
\norm{Z_\tstar}
&\le \norm{Z_\tstar'} + \norm{E_\tstar}\nonumber\\
&\le \left( 1 + \mu \norm{F} \right)^\tstar \norm{Z_0} + \norm{E_\tstar}\nonumber\\
&\stackrel{(a)}{\le} \left( 1 + \mu \norm{F} \right)^\tstar \norm{Z_0} + \norm{Z_0}\nonumber\\
&\le 2 \left( 1 + \mu \norm{F} \right)^\tstar \norm{Z_0},\label{ineq:spectralintern1}
\end{align}
where inequality $(a)$ follows from $ \norm{E_\tstar}\le \norm{Z_0}$, see Lemma \ref{lemma:spectral1}.
Moreover, we note that
\begin{align}
    \left( 1 + \mu \norm{F} \right)^\tstar 
    &= \exp \left(  \tstar \ln \left( 1 + \mu \norm{F} \right)  \right) \nonumber \\
    &\stackrel{(a)}{\le} \exp \left(  \left( \frac{\ln\left(\frac{c\sglmin \left(L_{F_1}^{T}Z_0 \right)}{2\kappa^{2} \norm{Z_0}}\right)}{\ln \left(1 - \frac{\mu\sglmin(X)}{8}\right)} +1 \right)  \ln  \left( 1 + \mu \norm{F} \right) \right)\nonumber\\
    &\stackrel{(b)}{\le} \exp \left(   \ln\left(\frac{2\kappa^{2} \norm{Z_0}}{c\sglmin(L_{F_1}^{T}Z_0)}\right) \frac{16  \norm{F}}{ \sglmin (X) } + \mu \norm{F}  \right) \nonumber\\
    &\stackrel{(c)}{\le}   \left(\frac{2\kappa^{2} \norm{Z_0}}{c\sglmin(L_{F_1}^{T}Z_0)}\right)^{32 \kappa}  \exp \left( \mu \norm{F} \right) \nonumber \\ 
    &\stackrel{(d)}{\le}  2 \left(\frac{2\kappa^{2} \norm{Z_0}}{c\sglmin(L_{F_1}^{T}Z_0)}\right)^{32 \kappa}. \label{spectral:intern12}
\end{align}
Here, inequality $(a)$ follows from the definition of $\tstar$ and inequality $(b)$ follows from the elementary inequality $ \frac{x}{1-x} \le \ln (1+x) \le x  $ as well as the assumption $ \mu \le \frac{\tilde{c}}{\sglmin \left(X\right)} $.
In inequalities $(c)$ and $(d)$ we have used that $\norm{F} \le 2 \norm{X} $, which follows from
\begin{equation*}
    \norm{F}
    = \norm{\left( \mathcal{A}^* \mathcal{A} \right) \left(X\right) }
     \le \norm{X} +  \norm{ X - \left(  \mathcal{A}^* \mathcal{A} \right) \left(X\right) } \le 2 \norm{X}.
\end{equation*}
Moreover, in inequality $(d)$ we have used our assumption on the step size $\mu$.
Combining inequality \eqref{spectral:intern12} with inequality \eqref{ineq:spectralintern1}, we obtain inequality \eqref{ineq:spectral_balanc4}.\\


\noindent\textbf{Proof of inequality \eqref{ineq:spectral_balanc1}:}
In order to show inequality \eqref{ineq:spectral_balanc1}, we first introduce the following notation for all natural numbers $t\ge 1$
\begin{align*}
    \tilZt' &:= \left( \Id -\mu F \right)^t \tilde{Z}_0, \\
    \tilde{E}_t &:= \tilZt - \tilZt'.
\end{align*}
Before proving inequality \eqref{ineq:spectral_balanc1}, we will first show that 
\begin{equation}\label{ineq:spectralsymmetry}
\tilde{E}_t=
\underset{=:D}{\underbrace{
\begin{pmatrix}
    \Id & 0 \\
    0 & -\Id 
\end{pmatrix}}}
E_t.
\end{equation}
To show this, we note that $\tilde{Z}_0'= \tilde{Z}_0 = DZ_0 = Z_0'$, see \eqref{def:Ztprime}. 
Then, it follows by induction that
\begin{align*}
  \tilZt'
=\left( \Id - \mu F \right) \tilde{Z}_{t-1}'
=\left( \Id - \mu F \right)  \begin{pmatrix}
    \Id & 0 \\
    0 & -\Id 
\end{pmatrix} Z_{t-1}'
= 
\begin{pmatrix}
    \Id & 0 \\
    0 & -\Id 
\end{pmatrix}
\left(\Id + \mu F\right) Z_{t-1}' = 
\begin{pmatrix}
    \Id & 0 \\
    0 & -\Id 
\end{pmatrix}
\Zt'.
\end{align*}
Since we have that $\Zt = D \tilZt $ (see \eqref{equ:Ztdefinition}) equation \eqref{ineq:spectralsymmetry} follows from the definition of $E_t$ and $\tilde{E}_t'$. 
Next, we compute that 
\begin{equation*}
\tilZtstar'^T Z_{\tstar}' 
= \tilde{Z}_0^T \left(  \Id -\mu F  \right)^\tstar \left(  \Id + \mu F  \right)^\tstar Z_0 
=  \tilde{Z}_0^T \left(  \Id - \mu^2 F^2  \right)^\tstar Z_0.
\end{equation*}
This implies that 
\begin{align}
\norm{ \tilZtstar'^T Z_{\tstar}' } &\le  \norm{ \tilde{Z}_0 } \norm{\Id - \mu^2 F^2 }^\tstar \norm{ Z_0 } \nonumber \\
&=\norm{\Id - \mu^2 F^2 }^\tstar \norm{ Z_0 }^2\nonumber \\
&\le  \norm{ Z_0 }^2, \label{spectral:intern11}
\end{align}
where in the last line we used that $F^2$ is a positive semidefinite matrix, $\norm{F} \le 2 $, and our assumption on the step size $\mu$. 
Next, we note that 
\begin{equation*}
    \tilZtstar^T Z_{\tstar} 
    = \left( \tilZtstar' + \tilEtstar \right)^T \left( \Ztstar' + \Etstar \right)
    =  \tilZtstar'^T \Ztstar' + \tilEtstar^T \Ztstar'   + \tilZtstar'^T \Etstar   +\tilEtstar^T \Etstar.
\end{equation*}
This implies that 
\begin{align}
\norm{  \tilZtstar^T \Ztstar} 
&\stackrel{\eqref{spectral:intern11}}{\le}   \norm{ Z_0 }^2 + \norm{ \tilEtstar } \norm{Z_{\tstar}'} + \norm{\Etstar} \norm{\tilZtstar}  + \norm{\tilEtstar} \norm{\Etstar}    \nonumber \\ 
&\stackrel{(a)}{=}  \norm{ Z_0 }^2 +  2 \norm{Z_{\tstar}'} \norm{\Etstar}  +\norm{\Etstar}^2.\nonumber \\ 
&\stackrel{(b)}{\le}   \norm{ Z_0 }^2 +  2 \norm{Z_{\tstar}'} \norm{Z_0}  +\norm{Z_0}^2\nonumber\\  
&\le 2  \norm{ Z_0 }^2 +  2 \norm{Z_{\tstar}'} \norm{Z_0} \nonumber\\  
&\le 2  \norm{ Z_0 }^2 +  2  \left( 1 + \mu \norm{F} \right)^\tstar  \norm{Z_0} \nonumber\\  
&\stackrel{(c)}{\le} 2  \norm{ Z_0 }^2 + 4\left(\frac{2\kappa^{2} \norm{Z_0}}{c\sglmin(L_{F_1}^{T}Z_0)}\right)^{32 \kappa} \norm{Z_0}^2. \label{spectral:intern6}
\end{align}
Equation $(a)$ is due to $\norm{\tilEtstar}= \norm{\Etstar}$ and $ \norm{\tilZtstar} = \norm{\Ztstar}$, which is a consequence of the symmetry between $\tilZtstar$ and $\Ztstar$, see Lemma \ref{lemma:symmetry} and equation \eqref{ineq:spectralsymmetry}.
In inequality $(b)$ we used that $\norm{\Etstar}\le \norm{Z_0}$, which is a consequence of Lemma \ref{lemma:spectral1}.
Inequality $(c)$ can be obtained by using \eqref{spectral:intern12}.
It follows that
\begin{align*}
 \norm{  \tilZtstar^T \Ztstar} 
& \le 6 \left(\frac{2\kappa^{2} \norm{Z_0}}{c\sglmin(L_{F_1}^{T}Z_0)}\right)^{32 \kappa}  \norm{Z_0}^2.
\end{align*}
This proves inequality \eqref{ineq:spectral_balanc1}.\\


\noindent\textbf{Proof of inequality \eqref{ineq:spectral_balanc2}:}
We observe that 
\begin{equation*}
    \norm{\tilde{Z}_{\tstar}^T Z_{\tstar} Q_{\tstar, \bot}  }
    \le \norm{\tilZtstar} \norm{\Ztstar Q_{\tstar, \bot}}
    \stackrel{(a)}{\le} 4 \left(\frac{2\kappa^{2} \norm{Z_0}}{c\sglmin \left(L_{F_1}^{T}Z_0 \right)}\right)^{32 \kappa} \norm{Z_0}\norm{\Ztstar Q_{\tstar, \bot}},
\end{equation*}
where in $(a)$ we used inequality \eqref{ineq:spectral_balanc4} and the fact that $ \norm{\tilZtstar} = \norm{\Ztstar} $, which is a consequence of Lemma \ref{lemma:symmetry}.\\
 
\noindent\textbf{Proof of inequality \eqref{ineq:spectral_balanc3}:}
We compute that 
\begin{align}
\norm{\tilde{Z}_{\tstar}^T P_{ Z_{\tstar} Q_{\tstar} }  } 
&\le \norm{ Q_{\tstar}^T \tilde{Z}_{\tstar}^T P_{ Z_{\tstar} Q_{\tstar} }  } + \norm{ Q_{\tstar,\bot}^T \tilde{Z}_{\tstar}^T P_{ Z_{\tstar} Q_{\tstar} }  } \nonumber  \\ 
&\le \norm{ P^T_{ \tilde{Z}_{\tstar} Q_{\tstar} }  P_{ Z_{\tstar} Q_{\tstar} }  } \norm{ \tilde{Z}_{\tstar} } + \norm{ \tilde{Z}_{\tstar} Q_{\tstar,\bot}  } \nonumber \\   
&= \norm{ P^T_{ \tilde{Z}_{\tstar} Q_{\tstar} }  P_{ Z_{\tstar} Q_{\tstar} }  } \norm{ Z_{\tstar} } + \norm{ Z_{\tstar} Q_{\tstar,\bot}  }, \label{spectral:intern1}
\end{align}
where the last line follows from $\norm{\Ztstar}= \norm{\tilZtstar}$ and $ \norm{ Z_{\tstar} Q_{\tstar,\bot}  } = \norm{ \tilZtstar Q_{\tstar,\bot}  }$, see Lemma \ref{lemma:symmetry}.
Next, we compute that
\begin{align}
    \norm{ P^T_{ \tilde{Z}_{\tstar} Q_{\tstar} }  P_{ Z_{\tstar} Q_{\tstar} }  } 
    &\le \norm{ P^T_{ \tilde{Z}_{\tstar} Q_{\tstar} } \LA \LAT  P_{ Z_{\tstar} Q_{\tstar} }  } + \norm{ P^T_{ \tilde{Z}_{\tstar} Q_{\tstar} } \LAP \LAPT   P_{ Z_{\tstar} Q_{\tstar} }  } \nonumber \\   
    &\le \norm{ P^T_{ \tilde{Z}_{\tstar} Q_{\tstar} } \LA } \norm{ \LAT  P_{ Z_{\tstar} Q_{\tstar} }  } + \norm{ P^T_{ \tilde{Z}_{\tstar} Q_{\tstar} } \LAP } \norm{ \LAPT   P_{ Z_{\tstar} Q_{\tstar} }  } \nonumber \\   
    &\le \norm{ P^T_{ \tilde{Z}_{\tstar} Q_{\tstar} } \LA } +  \norm{ \LAPT   P_{ Z_{\tstar} Q_{\tstar} }  } \nonumber \\   
    &\le \norm{ P^T_{ \tilde{Z}_{\tstar} Q_{\tstar} } \widetilde{L_{X,\bot}} } +  \norm{ \LAPT   P_{ Z_{\tstar} Q_{\tstar} }  } \nonumber \\   
    &= 2 \norm{ \LAPT   P_{ Z_{\tstar} Q_{\tstar} }  }, \label{spectral:intern2}  
\end{align}
where in the last equality we have used the fact that $\LAPT   P_{ Z_{\tstar} Q_{\tstar} } = \widetilde{L_{X,\bot}}^T  P_{ \tilde{Z}_{\tstar} Q_{\tstar} }  $, see Lemma \ref{lemma:symmetry}.
By combining inequalities \eqref{spectral:intern1} and \eqref{spectral:intern2} we obtain that 
\begin{equation*}
    \norm{\tilde{Z}_{\tstar}^T P_{ Z_{\tstar} Q_{\tstar} }  }\le 2  \norm{Z_{\tstar}} \norm{\LAPT P_{ Z_{\tstar} Q_{\tstar}   } } + \norm{ Z_{\tstar} Q_{\tstar,\bot}  } . 
\end{equation*}
Thus, using inequalities \eqref{ineq:spectral2} and \eqref{ineq:spectral4} from Lemma \ref{lemma:spec4} and inequality \eqref{ineq:spectral_balanc4} we obtain that
\begin{align*}
    \norm{\tilde{Z}_{\tstar}^T P_{ Z_{\tstar} Q_{\tstar} }  }
&\le \frac{224 c}{\kappa^2}  \left( \frac{2\kappa^2 \norm{Z_0}}{ c \sglmin \left( L_{F_1}^T Z_0 \right) } \right)^{32 \kappa} \norm{Z_0}
+ \left( \frac{2\kappa^2 \norm{Z_0}}{c \sglmin \left( L_{F_1}^T Z_0 \right)} \right)^{16 \kappa} \cdot \frac{4c \sglmin \left( L_{F_1}^T Z_0 \right)}{ \kappa^2 }\\      
&\le \frac{228 c}{\kappa^2}  \left( \frac{2\kappa^2 \norm{Z_0}}{ c \sglmin \left( L_{F_1}^T Z_0 \right) } \right)^{32 \kappa} \norm{Z_0},
\end{align*}
where the second inequality we used to $  \sglmin \left( L_{F_1}^T Z_0 \right)\le \norm{Z_0}$.
This completes the proof.
\end{proof}

The next lemma tells us how small one needs to choose the initialization $Z_0$ such that inequality \eqref{ineq:spectralinternclaim} below holds. 
The right-hand side of inequality \eqref{ineq:spectralinternclaim} can be interpreted as the maximal number of iterations for which the gradient descent iterates $\Zt$ can be approximated by the power method iterates $\Zt'$,
whereas the number on the left-hand side gives an upper bound on the number of iterations which are needed to align the subspace of the learned signal $\Zt$ with the subspace of the true signal.
\begin{lemma}\label{lemma:tstarcheck}
 Assume that 
 \begin{equation}\label{ineq:Z0smallness}
     \norm{Z_0} \le \sqrt{\frac{\norm{X} }{ 24 \min \left\{  n_1 + n_2; k \right\} }} \left( \frac{c \sglmin \left( L_{F_1}^T Z_0 \right)}{2\kappa^2 \norm{Z_0}} \right)^{ 18 \kappa }.
 \end{equation}   
 Moreover, assume that $ \frac{2}{3} \norm{X} \le \norm{F} \le \frac{4}{3} \norm{X} $ and  $ \mu \le \frac{\tilde{c}}{\sglmin \left(X\right)} $ for a sufficiently small absolute constant $\tilde{c}>0$.
 Then it holds for any constant $0<c\le 1$ that 
 \begin{equation}\label{ineq:spectralinternclaim}
    t_{\star}:=
     \left\lceil \frac{\ln\left(\frac{c\sglmin \left(L_{F_1}^T Z_0\right)}{2\kappa^{2} \norm{Z_0} }\right)}{\ln \left(1 - \frac{\mu\sglmin(A)}{8} \right)} \right\rceil
     \le \frac{ \ln \left( \frac{\norm{F}}{16 \min \left\{ k; n_1 + n_2 \right\}\norm{Z_0}^2}  \right)  }{3 \ln \left(1 + \mu \norm{F}\right)}.
 \end{equation}
\end{lemma}

\begin{proof}[Proof of Lemma \ref{lemma:tstarcheck}]
    We observe that to prove the claim it suffices to show that
    \begin{align*}
        3 \ln \left( 1+ \mu \norm{F} \right) \left(  \frac{\ln\left(\frac{c\sglmin \left(L_{F_1}^T Z_0\right)}{2\kappa^{2} \norm{Z_0} }\right)}{\ln \left(1 - \frac{\mu\sglmin(X)}{8} \right)}   + 1\right) 
        \le  \ln \left( \frac{\norm{F}}{16 \min \left\{  n_1 + n_2; k \right\}\norm{Z_0}^2}  \right). 
    \end{align*}
Next, we note that 
\begin{align*}
  3 \ln \left( 1+ \mu \norm{F} \right) \left(  \frac{\ln\left(\frac{c\sglmin \left(L_{F_1}^T Z_0\right)}{2\kappa^{2} \norm{Z_0} }\right)}{\ln \left(1 - \frac{\mu\sglmin(X)}{8} \right)}   + 1\right) 
  &\stackrel{(a)}{\le} 3 \mu \norm{F}  \left( \frac{8 \ln \left( \frac{2 \kappa^2 \norm{Z_0}}{c \sglmin \left( L_{F_1}^T Z_0 \right)} \right)}{  \mu \sglmin (X)   } +1  \right)      \\
  &\stackrel{(b)}{\le} 4 \mu \norm{X}  \left( \frac{8 \ln \left( \frac{2 \kappa^2 \norm{Z_0}}{c \sglmin \left( L_{F_1}^T Z_0 \right)} \right)}{  \mu \sglmin (X)   }  +1 \right)      \\
  &= 32 \kappa \ln \left( \frac{2\kappa^2 \norm{Z_0}}{c \sglmin \left( L_{F_1}^T Z_0 \right)} \right) + 4 \mu \norm{X}\\
  &\stackrel{(c)}{\le} 36 \kappa \ln \left( \frac{2\kappa^2 \norm{Z_0}}{c \sglmin \left( L_{F_1}^T Z_0 \right)} \right),
\end{align*}
where in inequality $(a)$ we used the elementary inequality $\ln \left( 1+x\right) \le x$ and the assumption that $\mu\le \frac{\tilde{c}}{ \sglmin \left(X\right)}$.
Inequality $(b)$ follows from the assumption $ \norm{F} \le \frac{4}{3} \norm{X}$.
For inequality $(c)$ we used the assumption that $\mu \le \frac{\tilde{c}}{ \sglmin (A)}$ with a sufficiently small absolute constant $\tilde{c} >0$.
Thus, we observe that \eqref{ineq:spectralinternclaim} holds, if we have that
\begin{equation*}
    \left( \frac{2\kappa^2 \norm{Z_0}}{c \sglmin \left( L_{F_1}^T Z_0 \right)} \right)^{36 \kappa }  
    \le \frac{\norm{F}}{16 \min \left\{  n_1 + n_2; k \right\}\norm{Z_0}^2}.
\end{equation*}
By rearranging terms and using the assumption $ \frac{2}{3} \norm{X} \le \norm{F} $, we see that this is implied by assumption \eqref{ineq:Z0smallness}.
\end{proof}
So far, our results hold for any deterministic initialization $Z_0$.
The next lemma utilizes the fact that $Z_0$ is a random matrix.
\begin{lemma}\label{lemma:probability}
Assume that $V_0 = \alpha V \in \R^{n_1 \times k} $ and $W_0= \alpha W \in \R^{n_2 \times k}$ for some fixed parameter $\alpha >0$, where the matrices $V$ and $W$ have i.i.d. entries with distribution $ \mathcal{N} \left(0,1\right) $.
Then, with probability at least $1-  C_1 \exp \left( - c_1 \max \left\{ n_1 +n_2; k \right\} \right)  $, it holds that
\begin{equation}\label{ineq:prob1}
   \frac{ \alpha \sqrt{ \max \left\{  n_1 +n_2; k \right\} } }{2}  
   \le  \norm{Z_0} 
   \le 3 \alpha \sqrt{ \max \left\{ n_1+n_2; k \right\} } .
\end{equation}
Moreover, for any $\varepsilon>0$, with probability at least $1- \left( C_2 \varepsilon  \right)^{k -r+1} -  \exp \left( - c_2 k \right) $ we have that
\begin{equation}\label{ineq:prob2}
    \sglmin \left( L_{F_1}^T Z_0 \right) \ge \alpha \varepsilon \left( \sqrt{k} - \sqrt{r-1} \right).
\end{equation}
Here, $C_1, C_2, c_1, c_2>0$ are some fixed numerical constants.
\end{lemma}
\begin{proof}[Proof of Lemma \ref{lemma:probability}]
Recall that
\begin{equation*}
    Z_0=
    \begin{pmatrix}
        V_0 \\
        W_0
    \end{pmatrix}
    = \alpha \begin{pmatrix}
        V\\
        W
    \end{pmatrix}
    = \alpha Z 
     \in \R^{ \left( n_1 + n_2 \right) \times k }.
\end{equation*} 
Since $Z$ has i.i.d.~entries with distribution $\mathcal{N} \left(0,1\right)$, 
it is well-known (see, e.g., \cite[Corollary 7.3.3]{vershynin2018high}) that with probability at least $1- 2 \exp \left( -\frac{\max \left\{ n_1+n_2; k \right\} }{C} \right)  $ we must have that 
\begin{equation*}
    \norm{Z}
     \le 3\sqrt{ \max \left\{ n_1+n_2; k \right\} }  . 
\end{equation*}
This implies the second inequality in \eqref{ineq:prob1}.
To prove the first inequality in \eqref{ineq:prob1} it suffices to note that when $k \le n_1 +n_2$ then it holds that
\begin{equation*}
    \norm{Z}
    \ge \twonorm{Ze_1}
    \ge \frac{\sqrt{n_1+n_2}}{2},
\end{equation*}
where the second inequality holds with probability at least  $ 1 - O \left( \exp \left( -\frac{n_1 + n_2}{C}\right) \right) $.
Analogously, if $k \ge n_1 +n_2$, we have that
\begin{equation*}
    \norm{Z}
    \ge \twonorm{Z^Te_1}
    \ge \frac{\sqrt{k}}{2},
\end{equation*}
with probability at least $ 1 - O \left( \exp \left( -\frac{k}{C} \right) \right) $.
Since $Z_0= \alpha Z$, by choosing the fixed numerical constants $C_1, c_1 >0$ appropriately this shows the first inequality in \eqref{ineq:prob1}.

In order to prove inequality \eqref{ineq:prob2}, we note that $ \sglmin \left( L_{F_1}^T Z_0 \right) = \alpha \sglmin \left( L_{F_1}^T Z \right) $.
Note that $L_{F_1} \in \R^{ (n_1 + n_2) \times r }$ is a fixed matrix (conditional on the measurement matrices $ \left\{ A_i \right\}_{i=1}^m$), which describes a subspace of dimension $r$.
In particular, due to the rotation invariance of the Gaussian distribution, the matrix $ L_{F_1}^T Z $ is again a random matrix with i.i.d. entries with distribution $ \mathcal{N} \left(0,1\right)$ (conditional on $\left\{ A_i \right\}_{i=1}^m$).
Thus it follows from \cite[Theorem 1.1]{rudelson_vershynin} that for every $\varepsilon>0$  with probability at least $1 - \left( C_2 \varepsilon \right)^{k-r+1} - \exp \left( -c_2k  \right) $ it holds that 
\begin{equation*}
\sglmin \left( L_{F_1}^T Z_0 \right)  = \alpha \sglmin \left( L_{F_1}^T Z \right) \ge \alpha \varepsilon \left( \sqrt{k} - \sqrt{r} \right).
\end{equation*}
 Choosing the fixed numerical constants $C_2,c_2>0$ appropriately this implies the second claim in Lemma \ref{lemma:probability}.
\end{proof}
Now we have all ingredients in place to prove the main result for the spectral phase.
\begin{proof}[Proof of Lemma \ref{lemma:spectralmain}]
First, we note that due to Lemma \ref{lemma:probability} we have with probability at least $1 - C_2 \exp \left( - c_1 k \right)  + \left( C_3 \varepsilon \right)^{k-r+1}  $ that 
\begin{align}
   \frac{ \alpha \sqrt{ \max \left\{  n_1 +n_2; k \right\} } }{2}  
   \le  \norm{Z_0} 
   &\le 3\alpha  \sqrt{ \max \left\{ n_1 + n_2; k \right\} },  \label{ineq:spectralintern13}\\
    \sglmin \left( L_{F_1}^T Z_0 \right)
    & \ge \alpha \varepsilon \left( \sqrt{k} - \sqrt{r-1} \right),\label{ineq:spectralintern14}
\end{align}
where $C_2,C_3, c_1>0 $ are some fixed absolute constants.
We first check that the condition
 \begin{equation*}
    t_{\star}=
     \left\lceil \frac{\ln\left(\frac{c\sglmin \left(L_{F_1}^T Z_0\right)}{2\kappa^{2} \norm{Z_0} }\right)}{\ln \left(1 - \frac{\mu\sglmin(X)}{8} \right)} \right\rceil
     \le \frac{ \ln \left( \frac{\norm{F}}{16 \min \left\{  n_1 + n_2; k \right\}\norm{Z_0}^2}  \right)  }{3 \ln \left(1 + \mu \norm{F}\right)}
 \end{equation*}
 is satisfied. Due to Lemma \ref{lemma:tstarcheck} it suffices to note that 
 \begin{align*}
     \norm{Z_0} 
     &\stackrel{(a)}{\le} 3\alpha \sqrt{\max \left\{  n_1 +n_2; k \right\}}\\
     &\stackrel{(b)}{\le} 3 \sqrt{\frac{\norm{X} }{ C_1 \min \left\{  n_1 + n_2; k \right\} }} \left( \frac{c \varepsilon \left( \sqrt{k} -\sqrt{r-1} \right) }{6\kappa^2 \sqrt{ \max \left\{ k; n_1 + n_2 \right\} }} \right)^{ 18 \kappa }\\
     &\stackrel{(c)}{\le} \sqrt{\frac{\norm{X} }{ 24 \min \left\{  n_1 + n_2; k \right\} }} \left( \frac{c \sglmin \left( L_{F_1}^T Z_0 \right)}{2\kappa^2 \norm{Z_0}} \right)^{18 \kappa },
 \end{align*}   
 where inequality $(a)$ follows from \eqref{ineq:spectralintern13} and inequality $(b)$ follows from our assumption on $\alpha$.
 Inequality $(c)$ follows from \eqref{ineq:spectralintern13} and \eqref{ineq:spectralintern14} (and from choosing the constant $C_2 >0$ sufficiently large). 
Thus, we can apply Lemmas \ref{lemma:spec4} and \ref{lemma:spec5} and by combining these lemmas with inequalities \eqref{ineq:spectralintern13}, \eqref{ineq:spectralintern14} and assumption \eqref{spectral:alphaassumption} we obtain that
 \begin{align*}
 \norm{L_{A,\bot}^{T} P_{Z_{t_\star}Q_{t_\star}}}
 &\le \frac{28c}{\kappa^2},\\
 \sigma_{\min}(Z_{t_\star}Q_{t_\star}) 
 &\ge \left(\frac{2\kappa^{2} \norm{Z_0} }{c\sglmin \left(L_{F_1}^T Z_0\right)}\right)^{ 2 \kappa} \frac{\sglmin \left( L_{F_1}^T Z_0 \right)}{4} 
 \ge \left(\frac{2\kappa^{2}  }{c} \right)^{ 2 \kappa} \frac{ \alpha \varepsilon \left( \sqrt{k} -\sqrt{r-1} \right) }{4}, \\
 \norm{Z_{t_\star}Q_{t_\star,\bot}} 
 &\le \min \left\{ 2 \sglmin \left( \Ztstar Q_{\tstar} \right); \  \left( \frac{2\kappa^2 \norm{Z_0}}{c \sglmin \left( L_{F_1}^T Z_0 \right)} \right)^{ 16 \kappa} \cdot \frac{4c \sglmin \left( L_{F_1}^T Z_0 \right)}{ \kappa^2 } \right\} \\
 &\le \min \left\{ 2 \sglmin \left( \Ztstar Q_{\tstar} \right); \  \left( \frac{6\kappa^2  \sqrt{\max \left\{  n_1+n_2; k \right\} }}{ c \varepsilon \left( \sqrt{k} - \sqrt{r -1} \right)} \right)^{ 16 \kappa} \cdot \frac{12 c \alpha  \sqrt{\max \left\{  n_1 +n_2; k \right\}}}{ \kappa^2 } \right\},\\
 \norm{Z_{\tstar}} 
 &\le 4 \left(\frac{2\kappa^{2} \norm{Z_0}}{c\sglmin \left(L_{F_1}^{T}Z_0 \right)}\right)^{32 \kappa} \norm{Z_0}
 \le 12 \alpha \left(\frac{6\kappa^2  \sqrt{\max \left\{  n_1+n_2; k \right\} }}{ c\varepsilon \left( \sqrt{k} - \sqrt{r -1} \right)}\right)^{32 \kappa}  \sqrt{\max \left\{  n_1 +n_2;k \right\}}\\ 
 &\le 2 \sqrt{\norm{X}},\\
 \norm{\tilde{Z}_{\tstar}^T Z_{\tstar}  }
 &\le 6 \left(\frac{2\kappa^{2} \norm{Z_0}}{c\sglmin(L_{F_1}^{T}Z_0)}\right)^{32 \kappa}  \norm{Z_0}^2 
 \le 54 \alpha^2 \left(\frac{6\kappa^2  \sqrt{\max \left\{  n_1+n_2; k \right\} }}{ c\varepsilon \left( \sqrt{k} - \sqrt{r -1} \right)}\right)^{32 \kappa}  \max \left\{  n_1 +n_2;k \right\}   \\ 
 &\le \frac{ c \norm{X}}{  \kappa^4},\\
 \norm{\tilde{Z}_{\tstar}^T Z_{\tstar} Q_{\tstar, \bot}  }
 &\le  4 \left(\frac{2\kappa^{2} \norm{Z_0}}{c\sglmin \left(L_{F_1}^{T}Z_0 \right)}\right)^{32 \kappa}  \norm{Z_0}  \norm{\Ztstar Q_{\tstar, \bot}}  \\ 
 &\le  12 \alpha \left(\frac{6\kappa^2  \sqrt{\max \left\{  n_1+n_2; k \right\} }}{ c \varepsilon \left( \sqrt{k} - \sqrt{r -1} \right)}\right)^{32 \kappa}  \sqrt{\max \left\{  n_1 +n_2 ; k \right\}}  \norm{\Ztstar Q_{\tstar, \bot}}\\
 &\le \frac{ c \sqrt{\norm{X}}}{ \kappa^3 } \norm{\Ztstar Q_{\tstar, \bot} },\\
 \norm{\tilde{Z}_{\tstar}^T P_{ Z_{\tstar} Q_{\tstar} }  }
 &\le \frac{228 c}{\kappa^2}  \left( \frac{2\kappa^2 \norm{Z_0}}{ c \sglmin \left( L_{F_1}^T Z_0 \right) } \right)^{32 \kappa} \norm{Z_0} \\ 
 &\le \frac{674 \alpha c}{\kappa^2}  \left(  \frac{6\kappa^2  \sqrt{\max \left\{  n_1+n_2;k \right\} }}{ c \varepsilon \left( \sqrt{k} - \sqrt{r -1} \right)}\right)^{32 \kappa}  \sqrt{\max \left\{  n_1 +n_2; k \right\}} \\
 &\le \frac{ c \sqrt{\norm{X}} }{\kappa^3}.
 \end{align*}
This shows the inequalities \eqref{spectral:final1}-\eqref{spectral:finalbalanc3}.
To finish the proof we note that
 \begin{align*}
    t_{\star}=
     \left\lceil \frac{\ln\left(\frac{c\sglmin \left(L_{F_1}^T Z_0\right)}{2\kappa^{2} \norm{Z_0} }\right)}{\ln \left(1 - \frac{\mu\sglmin(X)}{8} \right)} \right\rceil
     &\le \frac{\ln\left(\frac{c\sglmin \left(L_{F_1}^T Z_0\right)}{2\kappa^{2} \norm{Z_0} }\right)}{\ln \left(1 - \frac{\mu\sglmin(X)}{8} \right)} +1 \\ 
     &\le \frac{16 \ln\left(\frac{2\kappa^{2} \norm{Z_0} }{c\sglmin \left(L_{F_1}^T Z_0\right)}\right)}{ \mu \sglmin \left(X\right) } +1 \\ 
     &\le \frac{17 \ln\left(\frac{6\kappa^{2} \sqrt{ \max \left\{ n_1+n_2; k \right\} } }{c\varepsilon \left( \sqrt{k} - \sqrt{r-1} \right) }\right)}{ \mu \sglmin \left(X\right) }.
 \end{align*}
 This shows inequality \eqref{tstarbound}.
 Thus, the proof is complete.
\end{proof}
