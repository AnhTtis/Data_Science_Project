\subsection{Proof of Lemma \ref{lemma:sigmingrowth}: Controlling $\sglmin(\LAT\Zt)$}\label{sec:sigmingrowth}

\begin{proof}[Proof of Lemma \ref{lemma:sigmingrowth}]
First, we note that it follows from $ \symA = \LA \Sigma_X \LAT - \tilLA \Sigma_X \tilLAT $ that $\LAT \symA = \Sigma_X \LAT $.
We note that
\begin{align}\label{lemma1eq}
    \LAT\Zplus\Qt = & \LAT \left( \I - \mu \left(\Zt\ZtT - \tilZt\tilZtT - \symA \right)  -\mu \Deltat \right) \Zt\Qt \nonumber\\
    = &  \left(\I + \mu\Sigma_X \right) \LAT\Zt\Qt - \mu\LAT \left(\Zt\ZtT - \tilZt\tilZtT \right) \Zt\Qt  -\LAT \Deltat \Zt \Qt \nonumber\\
    = & \left(\I + \mu\Sigma_X \right) \LAT\Zt\Qt - \mu\LAT \left(\Zt\ZtT - \tilZt\tilZtT \right)\left(\LA\LAT + \LAP\LAPT\right)\Zt\Qt -\LAT \Deltat \Zt \Qt \nonumber\\
    = & \left(\I + \mu\Sigma_X \right) \LAT\Zt\Qt - \mu\LAT\Zt\ZtT\LA\LAT \Zt\Qt + \mu\LAT\tilZt\tilZtT\LA\LAT\Zt\Qt  \nonumber\\
    & - \mu\LAT \left(\Zt\ZtT - \tilZt\tilZtT \right) \LAP\LAPT\Zt\Qt -\LAT \Deltat \Zt \Qt \nonumber\\
    = & \left(\I + \mu\Sigma_X + \mu\LAT\tilZt\tilZtT\LA \right)\LAT\Zt\Qt \left(\I - \mu\QtT\ZtT\LA\LAT\Zt\Qt\right) \nonumber\\
    & + \mu^2\underbrace{ \left(\Sigma_X + \LAT\tilZt\tilZtT\LA \right)\LAT\Zt\ZtT\LA\LAT\Zt\Qt}_{=:U_1} \nonumber\\
    & - \mu\underbrace{\LAT \left(\Zt\ZtT - \tilZt\tilZtT \right)\LAP\LAPT\Zt\Qt}_{=:U_2} \nonumber\\
    &- \mu \underbrace{\LAT \Deltat \Zt \Qt}_{=: U_3}.
\end{align}

To further simplify \eqref{lemma1eq}, we want to transform $U_i$ into the form $D_i\LAT\Zt\Qt(\I - \mu\QtT\ZtT\LA\LAT\Zt\Qt)$ for $i=1,2,3$, where $D_i$ are matrices to be determined.
Note that using the assumption that $ \LAT \Zt \Qt$ is invertible we can compute that
\begin{align}\label{lemma1LAPTZtQteq}
    \Zt\Qt = &\Zt\Qt(\LAT\Zt\Qt)^{-1}\LAT\Zt\Qt \nonumber\\
    = & \P{\Zt\Qt}\P{\Zt\Qt}^T\Zt\Qt \left(\LAT\P{\Zt\Qt}\P{\Zt\Qt}^T\Zt\Qt\right)^{-1}\LAT\Zt\Qt \nonumber\\
    = & \P{\Zt\Qt}\P{\Zt\Qt}^T\Zt\Qt\left(\P{\Zt\Qt}^T\Zt\Qt\right)^{-1}\left(\LAT\P{\Zt\Qt}\right)^{-1}\LAT\Zt\Qt \nonumber\\
    = & \P{\Zt\Qt}\left(\LAT\P{\Zt\Qt}\right)^{-1}\LAT\Zt\Qt.
\end{align}
Also note that for any matrix $K$ we have the identity
\begin{equation*}
    \left(\I - \mu KK^T \right)  K 
    %= X - \mu XX^TX 
    = K\left(\I - \mu K^T K\right).
\end{equation*}
In particular, when $\I - \mu KK^T$ is invertible, it follows that
\begin{equation}\label{lemma1Xeq}
    K = \left(\I - \mu KK^T \right)^{-1}K \left(\I - \mu K^TK \right).
\end{equation}
Moreover, using the assumptions $\mu\le\frac{c}{\kappa\norm{X}}$ and $\norm{\Zt}\le 2\sqrt{\norm{X}}$, we have that
\begin{equation}\label{lemma1muineq}
    \mu\le \frac{c}{\norm{X}\kappa} \le \frac{1}{8\norm{X}} \le \frac{1}{2\norm{\Zt}^2} \le \frac{1}{2\norm{\LAT\Zt}^2}.
\end{equation}
In particular, this implies that $\I - \mu  \LAT\Zt\Qt \QtT \ZtT \LA  $ is invertible. 
Hence, we can set $K := \LAT\Zt\Qt$ and equation \eqref{lemma1Xeq} yields 
\begin{equation}\label{lemma2Xeq}
    \LAT\Zt\Qt = \left(\I - \mu \LAT\Zt\Qt \QtT \ZtT \LA  \right)^{-1} \LAT\Zt\Qt \left(\I - \mu  \QtT \ZtT \LAT \LAT\Zt\Qt  \right).
\end{equation}
Using \eqref{lemma1LAPTZtQteq} and \eqref{lemma2Xeq} we can rewrite $U_1,U_2,U_3$ in the following way:
\begin{align}\label{U1eq}
    U_1 = & \left(\Sigma_X + \LAT\tilZt\tilZtT\LA \right)\LAT\Zt\ZtT\LA\LAT\Zt\Qt \nonumber \\
    = &\underbrace{\left(\Sigma_X + \LAT\tilZt\tilZtT\LA\right)\LAT\Zt\ZtT\LA\left(\I - \mu\LAT\Zt\Qt\QtT\ZtT\LA\right)^{-1}}_{=:D_1}\LAT\Zt\Qt \left(\I - \mu\QtT\ZtT\LA\LAT\Zt\Qt \right),
\end{align}
\begin{align}\label{U2eq}
    U_2 
   = & \LAT\left(\Zt\ZtT - \tilZt\tilZtT\right)\LAP\LAPT \Zt\Qt \nonumber\\
    = &\LAT\left(\Zt\ZtT - \tilZt\tilZtT\right)\LAP \LAPT\P{\Zt\Qt}\left(\LAT\P{\Zt\Qt}\right)^{-1}\LAT\Zt\Qt \nonumber\\
    = &\underbrace{\LAT\left(\Zt\ZtT - \tilZt\tilZtT\right)\LAP \LAPT\P{\Zt\Qt}\left(\LAT\P{\Zt\Qt}\right)^{-1}(\I - \mu\LAT  \Zt\Qt\QtT\ZtT\LA)^{-1}}_{=:D_2} \nonumber\\
    & \cdot\LAT\Zt\Qt\left(\I - \mu\QtT\ZtT\LA\LAT\Zt\Qt\right),
\end{align}
\begin{align}\label{U3eq}
    U_3 &= \LAT \Deltat \Zt \Qt \nonumber \\
    &= \LAT \Deltat \P{\Zt\Qt}(\LAT\P{\Zt\Qt})^{-1}\LAT\Zt\Qt \nonumber \\
    &=  \underbrace{ \LAT \Deltat \P{\Zt\Qt}(\LAT\P{\Zt\Qt})^{-1}  (\I - \mu \LAT\Zt\Qt \QtT \ZtT \LA  )^{-1} }_{=:D_3} \LAT\Zt\Qt(\I - \mu  \QtT \ZtT \LAT \LAT\Zt\Qt  ). 
\end{align}
Combining \eqref{lemma1eq}, \eqref{U1eq}, \eqref{U2eq}, and \eqref{U3eq} we conclude that
\[\LAT\Zplus\Qt = \left(\I + \mu\Sigma_X + \mu^2 D_1 - \mu D_2 - \mu D_3\right)\LAT\Zt\Qt \left(\I - \mu\QtT\ZtT\LA\LAT\Zt\Qt \right).\]
It follows that
\begin{align}\label{mahdieq}
    &\sglmin \left(\LAT\Zplus\Qt \right) \nonumber\\
    \ge &\sglmin \left(\I + \mu\Sigma_X + \mu^2 D_1 - \mu D_2 -\mu D_3 \right) \sglmin \left(\LAT\Zt\Qt \left(\I - \mu\QtT\ZtT\LA\LAT\Zt\Qt \right) \right) \nonumber\\
    \stackrel{(a)}{=} &\sglmin\left(\I + \mu\Sigma_X + \mu^2 D_1 - \mu D_2 - \mu D_3   \right) \sglmin \left(\LAT\Zt\Qt \right) \left(1 - \mu\sglmin^2 \left(\LAT\Zt\Qt \right) \right) \nonumber\\
    = &\sglmin\left(\I + \mu\Sigma_X + \mu^2 D_1 - \mu D_2 - \mu D_3   \right) \sglmin\left(\LAT\Zt\right) \left(1 - \mu\sglmin^2(\LAT\Zt) \right) \nonumber\\
    \stackrel{(b)}{\ge} &\left( \sglmin \left( \I + \mu\Sigma_X \right) - \mu^2\norm{D_1} - \mu\norm{D_2} - \mu \norm{D_3} \right) \sglmin\left(\LAT\Zt\right) \left(1 - \mu\sglmin^2\left(\LAT\Zt \right) \right) \nonumber\\
    \stackrel{(c)}{=} &\left(1 + \mu\sglmin(X) - \mu^2\norm{D_1} - \mu\norm{D_2} - \mu \norm{D_3}  \right)\sglmin\left(\LAT\Zt\right) \left(1 - \mu\sglmin^2 \left(\LAT\Zt\right) \right).
\end{align}
Equation (a) can be derived from the singular value decomposition of $\LAT\Zt\Qt$ and \eqref{lemma1muineq}. In inequality (b) we use Weyl's inequality. Equation (c) holds due to the fact that $\Sigma_X$ is a positive definite matrix.
In order to prove the desired inequality \eqref{lemma1result}, we need to bound $\norm{D_1}$, $\norm{D_2}$, and $ \norm{D_3} $.
To this aim, first note that using \eqref{lemma1muineq} and the assumption $\norm{\LAPT\P{\Zt\Qt}}\le \frac{c}{\kappa}\le 1/2$ we have that
\begin{align}\label{ineq:intern1}
    \norm{(\I - \mu\LAT\Zt\Qt\QtT\ZtT\LA)^{-1}} = &\frac{1}{\sglmin(\I - \mu\LAT\Zt\Qt\QtT\ZtT\LA)}=\frac{1}{1 - \mu\norm{\LAT\Zt\Qt\QtT\ZtT\LA}} \le 2, 
\end{align}
and
\begin{align}\label{ineq:intern2}
    \norm{(\LAT\P{\Zt\Qt})^{-1}}= &\frac{1}{\sglmin(\LAT\P{\Zt\Qt})} = \frac{1}{\sqrt{1 - \norm{\LAPT\P{\Zt\Qt}}^2}} \le 2.
\end{align}
Combining the latter two inequalities with the assumptions $\norm{\LAPT\P{\Zt\Qt}}\le c\kappa^{-1}$ and $\norm{\Zt} \le 2\sqrt{\norm{X}}$ and the fact that $\norm{\Zt} = \norm{\tilZt} $, we have
\begin{align*}
    \norm{D_1} 
    \le &\norm{\Sigma_X + \LAT\tilZt\tilZtT\LA} \norm{\LAT\Zt}\norm{\ZtT\LA}\norm{\I-\mu\LAT  \Zt\Qt\QtT\ZtT\LA)^{-1}} \\
    \stackrel{\eqref{ineq:intern1}}{\le} &2(\norm{X} + \norm{\LAT\tilZt}^2)\norm{\LAT\Zt}^2 \\
    \le &2(\norm{X} + \norm{\tilZt}^2)\norm{\Zt}^2 \\
    \le &40\norm{X}^2,
\end{align*}
\begin{align*}
    \norm{D_2} \le & \norm{\LAT \left(\Zt\ZtT - \tilZt\tilZtT \right) \LAP }\norm{\LAPT\P{\Zt\Qt}}\norm{(\LAT\P{\Zt\Qt})^{-1}}\norm{\left(\I - \mu\LAT\Zt\Qt\QtT\ZtT\LA \right)^{-1}} \\
    \stackrel{\eqref{ineq:intern1},\eqref{ineq:intern2}}{\le} & 4c\kappa^{-1}\norm{\LAT \left( \Zt\ZtT - \tilZt\tilZtT \right) \LAP} \\
    \le & 4c\kappa^{-1}\norm{\Zt\ZtT - \tilZt\tilZtT} \\
    \le & 32c\sglmin(X),
\end{align*}
and
\begin{align*}
    \norm{D_3} \le & \norm{ \LAT \Deltat \P{\Zt\Qt}(\LAT\P{\Zt\Qt})^{-1}  \left(\I - \mu \LAT\Zt\Qt \QtT \ZtT \LA  \right)^{-1}}\\
    \le & \norm{ \Deltat  } \norm{ \left(\LAT\P{\Zt\Qt} \right)^{-1} } \norm{ \left(\I - \mu \LAT\Zt\Qt \QtT \ZtT \LA  \right)^{-1} } \\
    \stackrel{\eqref{ineq:intern1},\eqref{ineq:intern2}}{\le}
    & 4 \norm{ \Deltat  } \\
    \le & 4 c \sglmin \left(X \right),
\end{align*}
where in the last line we used the assumption $ \norm{ \Deltat  } \le c \sglmin (X)  $.
Plugging these three bounds into \eqref{mahdieq} we have the following chain of inequalities which complete the proof of the lemma.
\begin{align*}
    &\sglmin \left(\LAT\Zplus\Qt \right) \nonumber\\
    \ge & \left(1 + \mu\sglmin \left(X \right) - \mu^2\norm{D_1} - \mu\norm{D_2} - \mu \norm{D_3} \right)\sglmin \left(\LAT\Zt\right) \left(1 - \mu\sglmin^2 \left(\LAT\Zt \right) \right) \nonumber\\
    \ge &\left(1 + \mu\sglmin(X) - 40\mu^2\norm{X}^2- 36c\mu\sglmin(X) \right) \sglmin(\LAT\Zt) \left(1 - \mu\sglmin^2 \left(\LAT\Zt \right) \right) \\
    \stackrel{(a)}{\ge} & \left(1 + \frac{1}{2}\mu\sglmin \left(X \right) \right) \sglmin(\LAT\Zt) \left(1 - \mu\sglmin^2(\LAT\Zt)\right) \\
    = &\left(1 + \frac{1}{2}\mu \left(1 - \mu\sglmin^2 \left( \LAT\Zt \right)\right) \sglmin(X) - \mu\sglmin^2 \left(\LAT\Zt \right) \right)\sglmin \left( \LAT\Zt \right) \\
   \stackrel{(b)}{\ge} & \left(1 + \frac{1}{4}\mu\sglmin(X) - \mu\sglmin^2 \left(\LAT\Zt \right) \right) \sglmin \left(\LAT\Zt \right).
\end{align*}
Inequality (a) follows from the assumption $\mu \le \frac{c}{\norm{X}\kappa}$ and by choosing the absolute constant $c>0$ small enough. 
In inequality (b) we used inequality \eqref{lemma1muineq}.
\end{proof}

