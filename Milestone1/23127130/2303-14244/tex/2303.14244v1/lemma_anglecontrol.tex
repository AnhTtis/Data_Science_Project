\subsection{Proof of Lemma \ref{lemma:anglecontrol}: Controlling $\norm{ \LAPT \P{\Zt\Qt} }$}\label{sec:anglecontrol}
We define the inverse of the square root of a symmetric positive definite matrix $D = \P{D}\Sigma_{D}\P{D}^T$ by $D^{-1/2} = \P{D}\Sigma_{D}^{-1/2}\P{D}^T$, where $\left(\Sigma_{D}^{-1/2}\right)_{ii} = \frac{1}{\sqrt{D_{ii}}}$. 
In the following, we will need the following technical lemma, which gives a bound on the first-order Taylor approximation of the matrix inverse square root.
\begin{lemma}\label{MatrixTaylor}
Let $D$ be a symmetric matrix such that $\norm{D} \leq 1/2$. Then it holds that
\[\norm{(\I + D)^{-1/2} - \I + \frac{1}{2}D} \leq 3\norm{D}^{2}.\]
\end{lemma}
For the straightforward proof of Lemma \ref{MatrixTaylor} we refer to  \cite[Lemma B.2]{stoger2021small}. 
To prove Lemma \ref{lemma:anglecontrol} we will need the following technical lemma.
\iffalse
\begin{lemma}\label{lemma:angle:auxiliarylemma1}
Assume that $\mu \le \frac{c}{\norm{X}}$ , $\norm{\Zt} \le 2\sqrt{\norm{X}}$, $\norm{\LAPT\P{\Zt\Qt}} \le c\kappa^{-1}$, and $\norm{\Deltat} \le c \sglmin(X)$.
Then it holds that
\begin{equation*}
\sglmin\left(\LAT\Zplus\right) \ge \frac{1}{2}\sglmin\left(\Zt\Qt\right),
\end{equation*}
where $c>0$ is an absolute constant chosen small enough.
\end{lemma}
\begin{proof}

\end{proof}
\fi

\begin{lemma}\label{lemma:aux1}
Assume that $\mu \le \frac{c}{\norm{X}\kappa}$, $\norm{\LAPT\P{\Zt\Qt}} \le c\kappa^{-1}$, $\norm{\Zt\Qtp} \le 2\sglmin(\Zt\Qt)$, $\norm{\Zt}\le 2\sqrt{\norm{X}}$, and $\norm{\Deltat} \le c \sglmin (X) $.
Then it holds that
\begin{equation}\label{ineq:aux1}
\norm{\QtpT\Qplus} \le \mu\left(\norm{\Zt\Qt}\norm{\Zt\Qtp}+ 40 \mu \norm{X} \norm{\Zt\Qt}^2 \right)\norm{\LAPT\P{\Zt\Qt}} + 4\mu\frac{\sqrt{\norm{X}}\norm{\tilZtT\Zt \Qtp}}{\sglmin\left(\Zt\Qt\right)} + 4\mu \norm{\Deltat}.
\end{equation}
Moreover, we have that
\begin{equation*}
\sglmin(\QtT\Qplus) \ge \frac{1}{2}.
\end{equation*}
Here, $c>0$ is an absolute constant chosen small enough.
\end{lemma}
\begin{proof}
We recall that
\begin{equation*}
    \Zplus = \left( \I -\mu M_t \right) \Zt,
\end{equation*}
where
\begin{equation*}
    M_t = \Zt\ZtT - \tilZt\tilZtT - \symA +\Deltat.
\end{equation*}
As shown in inequality \eqref{ineq:boundforM} we have
\begin{equation}\label{ineq:boundforM1}
    \norm{M_t} \le 10 \norm{X}.
\end{equation}
We observe that
\begin{align*}
    \QtpT\Qplus = &\QtpT\ZplusT\LA(\LAT\Zplus\ZplusT\LA)^{-1/2}.
\end{align*}
Due to $\LAT\Zt\Qtp = 0$ we have
\begin{align*}
    \LAT\Zplus\Qtp = &\LAT\left(\I - \mu\left(\Zt\ZtT - \tilZt\tilZtT - \symA + \Deltat\right)\right)\Zt\Qtp \nonumber\\
    = &-\mu\LAT \left(\Zt\ZtT - \tilZt\tilZtT - \symA + \Deltat \right)\Zt\Qtp \nonumber\\
    = &-\mu\LAT\left(\Zt\ZtT - \tilZt\tilZtT\right)\Zt\Qtp + \mu\LAT \left(\LA\Sigma_X\LAT - \tilLA \Sigma_X\tilLAT \right)\Zt\Qtp -\mu \LAT \Deltat \Zt \Qtp \nonumber\\
    = &-\mu\LAT \left(\Zt\ZtT - \tilZt\tilZtT \right) \Zt\Qtp  -  \mu \LAT \Deltat \Zt \Qtp \nonumber\\
    = &-\mu\LAT\Zt\Qt\QtT\ZtT\LAP\LAPT\Zt\Qtp + \mu\LAT\tilZt\tilZtT\Zt\Qtp -\mu \LAT\Deltat \Zt \Qtp.
\end{align*}
Combining the above two equalities we obtain that
\begin{align} \label{ineq:LemmaC2le}
    &\norm{\QtpT\Qplus} \nonumber\\
    = &\norm{(\LAT\Zplus\ZplusT\LA)^{-1/2}(-\mu\LAT\Zt\Qt\QtT\ZtT\LAP\LAPT\Zt\Qtp + \mu\LAT\tilZt\tilZtT\Zt\Qtp -\mu \LAT\Deltat \Zt \Qtp )} \nonumber\\
    \le &\mu\norm{(\LAT\Zplus\ZplusT\LA)^{-1/2}\LAT\Zt\Qt\QtT\ZtT\LAP\LAPT\Zt\Qtp} \nonumber\\ 
    & + \mu\norm{(\LAT\Zplus\ZplusT\LA)^{-1/2}\LAT\tilZt\tilZtT\Zt\Qtp} + \mu \norm{ (\LAT\Zplus\ZplusT\LA)^{-1/2} \LAT\Deltat \Zt \Qtp    } \nonumber\\
    \le &\mu\norm{(\LAT\Zplus\ZplusT\LA)^{-1/2}\LAT\Zt\Qt}\norm{\QtT\ZtT\LAP}\norm{\LAPT\Zt\Qtp} \nonumber\\ 
    & + \mu\norm{(\LAT\Zplus\ZplusT\LA)^{-1/2}}\norm{\tilZt} \norm{ \tilZtT \Zt \Qtp}  + \mu \norm{ (\LAT\Zplus\ZplusT\LA)^{-1/2}} \norm{\Deltat} \norm{ \Zt \Qtp } \nonumber\\
    \le &\mu\norm{(\LAT\Zplus\ZplusT\LA)^{-1/2}\LAT\Zt\Qt}\norm{\LAPT\P{\Zt\Qt}}\norm{\Zt\Qt}\norm{\Zt\Qtp} \nonumber\\ 
    & + \mu\frac{ \norm{\tilZt} \norm{ \tilZtT \Zt \Qtp}  }{\sglmin(\LAT\Zplus)} + \mu \frac{ \norm{\Deltat} \norm{ \Zt \Qtp }}{\sglmin(\LAT\Zplus)}.
\end{align}
To deal with the first summand in the penultimate line in the inequality chain above, we note that
\begin{align} \label{ineq:LemmaC2firstsummand}
    & \norm{\left(\LAT\Zplus\ZplusT\LA \right)^{-1/2}\LAT\Zt\Qt} \nonumber\\
    = &\norm{ \left(\LAT\Zplus\ZplusT\LA \right)^{-1/2}\LAT \left(\Zplus+\mu M_t \Zt \right)\Qt} \nonumber\\
    \le &\norm{ \left(\LAT\Zplus\ZplusT\LA \right)^{-1/2}\LAT\Zplus\Qt} + \mu\norm{ \left(\LAT\Zplus\ZplusT\LA \right)^{-1/2}\LAT M_t \Zt\Qt} \nonumber\\
    \le &\norm{\left( \LAT\Zplus\ZplusT\LA \right)^{-1/2}\LAT\Zplus} + \mu\norm{ \left(\LAT\Zplus\ZplusT\LA \right)^{-1/2}}\norm{ M_t }\norm{\Zt\Qt} \nonumber\\
    = & 1 + \mu\frac{\norm{ M_t }\norm{\Zt\Qt}}{\sglmin(\LAT\Zplus)}.
\end{align}
To deal with $\sglmin\left(\LAT\Zplus\right)$ appearing in the denominator, we compute that
\begin{align} \label{ineq:LemmaC2sglminLATZplus}
    \sglmin\left(\LAT\Zplus\right) \ge &\sglmin\left(\LAT\Zplus\Qt\right) \nonumber\\
    = &\sglmin\left(\LAT\left( \I -\mu M_t \right) \Zt\Qt\right) \nonumber\\
    = &\sglmin\left(\LAT\left( \I -\mu M_t \right) \P{\Zt\Qt}\P{\Zt\Qt}^T \Zt\Qt\right) \nonumber\\
    \ge &\sglmin\left(\LAT\left( \I -\mu M_t \right) \P{\Zt\Qt}\right)\sglmin\left(\P{\Zt\Qt}^T \Zt\Qt\right) \nonumber\\
    = &\sglmin\left(\LAT\P{\Zt\Qt} - \mu \LAT M_t \P{\Zt\Qt}\right)\sglmin\left(\Zt\Qt\right) \nonumber\\
    \ge &\left(\sglmin\left(\LAT\P{\Zt\Qt}\right) - \mu \norm{\LAT M_t\P{\Zt\Qt}}\right)\sglmin\left(\Zt\Qt\right) \nonumber\\
    \stackrel{(a)}{\ge} &\left(\sqrt{1 - \norm{\LAPT\P{\Zt\Qt}}^2} - 10\mu\norm{X}\right)\sglmin\left(\Zt\Qt\right) \nonumber\\
    \stackrel{(b)}{\ge} &\frac{1}{2}\sglmin\left(\Zt\Qt\right).
\end{align}
In inequality (a) we used the inequality \eqref{ineq:boundforM1}.
Inequality (b) follows from the assumptions $\mu \le \frac{c}{\norm{X}\kappa}$ and $\norm{\LAPT\P{\Zt\Qt}} \le c\kappa^{-1}$.

Inserting \eqref{ineq:LemmaC2firstsummand} and \eqref{ineq:LemmaC2sglminLATZplus} into \eqref{ineq:LemmaC2le} we obtain that
\begin{align}
    &\norm{\QtpT\Qplus} \nonumber \\
    \le &\mu\left(1 + \mu\frac{\norm{M_t}\norm{\Zt\Qt}}{\sglmin\left(\LAT\Zplus\right)}\right)\norm{\LAPT\P{\Zt\Qt}}\norm{\Zt\Qt}\norm{\Zt\Qtp} + \mu\frac{ \norm{\tilZt} \norm{ \tilZtT \Zt \Qtp}  + \norm{\Deltat} \norm{ \Zt \Qtp } }{\sglmin\left(\LAT\Zplus\right)} \nonumber \\
    \le &\mu\left(1 + 2\mu\frac{\norm{M_t}\norm{\Zt\Qt}}{\sglmin\left(\Zt\Qt\right)}\right)\norm{\LAPT\P{\Zt\Qt}}\norm{\Zt\Qt}\norm{\Zt\Qtp} + 2\mu\frac{ \norm{\tilZt} \norm{ \tilZtT \Zt \Qtp}   + \norm{\Deltat} \norm{ \Zt \Qtp } }{\sglmin\left(\Zt\Qt\right)}\nonumber \\
    \stackrel{(a)}{\le} &\mu\norm{\LAPT\P{\Zt\Qt}}(\norm{\Zt\Qt}\norm{\Zt\Qtp} + 4\mu\norm{M_t}\norm{\Zt\Qt}^2) + 4 \mu \frac{\norm{\tilZt}\norm{\tilZtT\Zt \Qtp}}{\sglmin\left(\Zt\Qt\right)}  +4\mu \norm{\Deltat} \nonumber\\
    \stackrel{(b)}{\le} &\mu \left( \norm{\Zt\Qt}\norm{\Zt\Qtp}+ 40 \mu \norm{X} \norm{\Zt\Qt}^2  \right) \norm{\LAPT\P{\Zt\Qt}} + 4\mu \frac{\sqrt{\norm{X}}\norm{\tilZtT\Zt \Qtp}}{\sglmin\left(\Zt\Qt\right)}+4\mu \norm{\Deltat} \label{ineq:lemmashown1} \\
    \stackrel{(c)}{\le} &\mu \left( \norm{\Zt\Qt}\norm{\Zt\Qtp}+ 40 \mu \norm{X} \norm{\Zt\Qt}^2  \right) + 4\mu \frac{\sqrt{\norm{X}}\norm{\tilZtT}\norm{\Zt \Qtp}}{\sglmin\left(\Zt\Qt\right)}+4\mu \norm{\Deltat} \nonumber \\
    \stackrel{(d)}{\le} &\mu \left( 4\norm{X}+ 160 \mu \norm{X}^2  \right) + 8\mu \frac{\norm{X}\norm{\Zt \Qtp}}{\sglmin\left(\Zt\Qt\right)}+4\mu \norm{\Deltat} \nonumber \\
    \stackrel{(e)}{\le} &\frac{1}{2}. \nonumber
\end{align}
In inequality $(a)$ we used the assumption $\norm{\Zt\Qtp} \le 2\sglmin(\Zt\Qt)$. 
Inequalities $(b)$ follows from the assumption that $\norm{\Zt} \le 2\sqrt{\norm{X}}$ and inequality \eqref{ineq:boundforM1}.
Inequality $(c)$ follows from $ \norm{\LAPT\P{\Zt\Qt}}\le 1$ and the submultiplicativity of the spectral norm. 
Inequalities $(d)$ follows from the assumption that $\norm{\Zt} \le 2\sqrt{\norm{X}}$ and the fact that $\norm{\tilZt} =\norm{\Zt} $.
Inequality $(e)$ follows then from the assumptions  $\norm{\Deltat} \le c \sglmin (A)  $, $\norm{\Zt \Qtp} \le 2 \sglmin \left(\Zt \Qt \right)$, and $\mu \le \frac{c}{\kappa \norm{X}}$.
Note that inequality \eqref{ineq:lemmashown1} implies inequality \eqref{ineq:aux1}.
From the last line in the above inequality chain it follows that
\[\sglmin(\QtT\Qplus) = \sqrt{1 - \norm{\QtpT\Qplus}^2} \ge \frac{1}{2}.\]
This finishes the proof.
\end{proof}
Now we have all ingredients in place to give a proof of Lemma \ref{lemma:anglecontrol}.
\begin{proof}[Proof of Lemma \ref{lemma:anglecontrol}]
First, we recall that
\begin{equation*}
    \Zplus = \left( \I -\mu M_t \right) \Zt,
\end{equation*}
where
\begin{equation*}
    M_t = \Zt\ZtT - \tilZt\tilZtT - \symA +\Deltat.
\end{equation*}
As shown in inequality \eqref{ineq:boundforM} we have
\begin{equation}\label{ineq:boundforM2}
    \norm{M_t} \le 10 \norm{X}.
\end{equation} It follows that
\begin{align*}
    \Zplus\Qplus = & (\I - \mu M_t )\Zt\Qplus \\
    = & (\I-\mu M_t )\Zt\Qt\QtT\Qplus + (\I - \mu M)\Zt\Qtp\QtpT\Qplus \\
    = & (\I-\mu M_t )\P{\Zt\Qt}\P{\Zt\Qt}^T\Zt\Qt\QtT\Qplus + \left(\I - \mu M_t \right) \Zt\Qtp\QtpT\Qplus.
\end{align*}
Note that $\P{\Zt\Qt}^T\Zt\Qt\QtT\Qplus$ is invertible since $\P{\Zt\Qt}^T\Zt\Qt$ is invertible by assumption and $\QtT\Qplus$ is invertible by Lemma \ref{lemma:aux1}.
It follows that
\begin{align*}
    &(\I - \mu M_t)\Zt\Qtp\QtpT\Qplus \\
    = &(\I - \mu M_t)\Zt\Qtp\QtpT\Qplus(\P{\Zt\Qt}^T\Zt\Qt\QtT\Qplus)^{-1}\P{\Zt\Qt}^T\Zt\Qt\QtT\Qplus \\
    = &(\I - \mu M_t)\underbrace{\Zt\Qtp\QtpT\Qplus(\P{\Zt\Qt}^T\Zt\Qt\QtT\Qplus)^{-1}\P{\Zt\Qt}^T}_{=:K}\P{\Zt\Qt}\P{\Zt\Qt}^T\Zt\Qt\QtT\Qplus \\
    = &(\I - \mu M_t)K\P{\Zt\Qt}\P{\Zt\Qt}^T\Zt\Qt\QtT\Qplus.
\end{align*}
Hence, we obtain that
\[\Zplus\Qplus = (\I - \mu M_t)(\I + K)\P{\Zt\Qt}\P{\Zt\Qt}^T\Zt\Qt\QtT\Qplus.\]
Since $\P{\Zt\Qt}^T\Zt\Qt\QtT\Qplus$ is invertible, the span of the left singular vectors of 
\begin{equation*}
H:=\left(\I - \mu M_t \right) \left(\I + K \right)\P{\Zt\Qt}
\end{equation*}
is the same as the span of the left singular vectors of $\Zplus\Qplus$, which yields that
\[\norm{\LAPT\P{\Zplus\Qplus}} = \norm{\LAPT\P{H}} = \norm{\LAPT\P{H}\Q{H}^T} = \norm{\LAPT H(H^TH)^{-1/2}}.\]
%\ds{Maybe we should check that $\Id+K$ is invertible. This is true but maybe we should add the argument.}
For simplicity of notations, we define 
\begin{equation*}
B:= \left(\I - \mu M_t \right) \left(\I + K \right) - \I= K-\mu M_t -\mu M_t K.
\end{equation*}
It follows that
\begin{align*}
    (H^TH)^{-1/2} 
    = & \left(\P{\Zt\Qt}^T\left(\I + B\right)^T(\I + B)\P{\Zt\Qt}\right)^{-1/2} \\
    = & \left(\P{\Zt\Qt}^T\left(\I + B^T + B + B^TB\right)\P{\Zt\Qt}\right)^{-1/2} \\
    = & \left(\I + \underbrace{ \P{\Zt\Qt}^T\left(B^T + B\right)\P{\Zt\Qt} + \P{\Zt\Qt}^TB^TB\P{\Zt\Qt}}_{=:D} \right)^{-1/2}.
\end{align*}
Next, we want to apply Lemma \ref{MatrixTaylor}.
For that we need to check that $ \norm{D} \le 1/2 $ holds.
In the following we are going to show $ \norm{B} \le 1/6$, which implies $ \norm{D} \le 1/2 $.
For that, we note first that
\begin{align}
    \norm{K} = &\norm{\Zt\Qtp\QtpT\Qplus(\P{\Zt\Qt}^T\Zt\Qt\QtT\Qplus)^{-1}\P{\Zt\Qt}^T}  \nonumber\\
    = &\norm{\Zt\Qtp\QtpT\Qplus(\QtT\Qplus)^{-1}(\P{\Zt\Qt}^T\Zt\Qt)^{-1}\P{\Zt\Qt}^T}\nonumber \\
    \le &\norm{\Zt\Qtp}\norm{\QtpT\Qplus}\norm{(\QtT\Qplus)^{-1}}\norm{(\P{\Zt\Qt}^T\Zt\Qt)^{-1}}\norm{\P{\Zt\Qt}^T} \nonumber \\
    = &\frac{\norm{\Zt\Qtp}\norm{\QtpT\Qplus}}{\sglmin(\QtT\Qplus)\sglmin(\P{\Zt\Qt}^T\Zt\Qt)}\nonumber \\
    = &\frac{\norm{\Zt\Qtp}\norm{\QtpT\Qplus}}{\sglmin(\QtT\Qplus)\sglmin(\Zt\Qt)} \nonumber \\
    \stackrel{(a)}{\le} &4\norm{\QtpT\Qplus}  \nonumber\\
    \stackrel{(b)}{\le} & 4 \mu\left(\norm{\Zt\Qt}\norm{\Zt\Qtp}+ 40 \mu \norm{X} \norm{\Zt\Qt}^2 \right)\norm{\LAPT\P{\Zt\Qt}} + 16 \mu\frac{\sqrt{\norm{X}}\norm{\tilZtT\Zt \Qtp}}{\sglmin\left(\Zt\Qt\right)} + 16 \mu \norm{\Deltat} \nonumber \\
    \stackrel{(c)}{\le} & 700c \mu \sglmin \left(X\right) \norm{\LAPT\P{\Zt\Qt}} + 16\mu\frac{\sqrt{\norm{X}}\norm{\tilZtT\Zt \Qtp}}{\sglmin\left(\Zt\Qt  \right)} +16 \mu \norm{\Deltat} \stackrel{(d)}{\le} \frac{1}{12}. \label{ineq:aux7}
\end{align}
Inequality $(a)$ follows from Lemma \ref{lemma:aux1} and from the assumption $\norm{\Zt \Qtp} \le 2 \sglmin \left(\Zt \Qt \right)  $.
Inequality $(b)$ follows from using Lemma \ref{lemma:aux1} again.
In inequality $(c)$ we use the assumptions $\norm{\Zt} \le 2\sqrt{\norm{X}} $, $\mu \le \frac{c}{\norm{X} \kappa} $, and $ \norm{ \Zt \Qtp} \le c \kappa^{-1/2} \sqrt{ \sglmin \left(X\right) }$.
Inequality $(d)$ follows from the assumptions $\norm{\Zt\Qtp} \le c\kappa^{-1/2}\sqrt{\sglmin(X)}$, $\norm{\tilZtT\Zt \Qtp} \le \sqrt{\norm{X}}\sglmin\left(\Zt\Qt\right)$, $ \norm{\Deltat} \le c \sglmin(X) $, and $\norm{\Zt} \le 2\sqrt{\norm{X}}$.
Next, we note that
\begin{align}
    \norm{B} \le &\mu\norm{M_t} + \norm{K} + \mu\norm{M_tK} \nonumber \\
    \le &\mu\norm{M_t} + \norm{K} + \mu\norm{M_t}\norm{K} \nonumber \\
    \stackrel{(a)}{\le} & \frac{3}{2} \mu\norm{ M_t } + \norm{K}  \label{ineq:aux5} \\
    \le  & \frac{3}{2} \mu \left( \norm{ \Zt\ZtT - \tilZt\tilZtT-\symA }  + \norm{\Deltat}  \right) + \norm{K}        \nonumber  \\
     \stackrel{(b)}{\le} & \frac{3}{2} \mu \left( 9 \norm{X} + \norm{\Deltat}  \right) + \norm{K}      \nonumber    \\
    \le & 1/6, \nonumber
\end{align}
where in inequality $(a)$ we used that $ \norm{K} \le 1/12 \le 1/2 $.
In inequality $(b)$ we used the assumption $ \norm{\Zt} \le 2 \sqrt{\norm{X}} $.
Hence, we have shown $\norm{B} \le 1/6 $, which implies $\norm{D} \le 1/2$.
Thus, we can apply Lemma \ref{MatrixTaylor} and it follows that
\begin{align*}
    & \left(\I + \P{\Zt\Qt}^T(B^T + B)\P{\Zt\Qt} + \P{\Zt\Qt}^TB^TB\P{\Zt\Qt} \right)^{-1/2} \\
    = & \I - \frac{1}{2} \left(\P{\Zt\Qt}^T(B^T + B)\P{\Zt\Qt} + \P{\Zt\Qt}^TB^TB\P{\Zt\Qt} \right) + C,
\end{align*}
where $C$ is a matrix such that
\begin{align}\label{ineq:intern3}
\norm{C} \le 3\norm{\P{\Zt\Qt}^T \left(B^T + B\right)\P{\Zt\Qt} + \P{\Zt\Qt}^TB^TB\P{\Zt\Qt}}^2.
\end{align}
By a direct computation we obtain that
\begin{align*}
    &\LAPT H \left(H^TH \right)^{-1} \\
    = &\LAPT \left(\I + B \right)\P{\Zt\Qt} \left(\I-\frac{1}{2} \left(\P{\Zt\Qt}^T \left(B^T + B \right) \P{\Zt\Qt} + \P{\Zt\Qt}^TB^TB\P{\Zt\Qt} \right) + C \right) \\
    = &\LAPT \left(\I + B -\frac{1}{2}\P{\Zt\Qt}\P{\Zt\Qt}^T \left(B^T + B \right) \right) \P{\Zt\Qt} - \frac{1}{2}\LAPT B\P{\Zt\Qt}\P{\Zt\Qt}^T \left(B^T + B \right)\P{\Zt\Qt} \\ 
    &-\LAPT \left(\I + B \right)\P{\Zt\Qt} \left( \frac{1}{2}\P{\Zt\Qt}^TB^TB\P{\Zt\Qt} - C \right) \\
    = &\LAPT \left(\I -\mu M_t + \mu\frac{1}{2}\P{\Zt\Qt}\P{\Zt\Qt}^T \left(M_t^T + M_t \right) \right) \P{\Zt\Qt} \\
    & + \LAPT \left(K - \frac{1}{2}\P{\Zt\Qt}\P{\Zt\Qt}^T \left( K^T + K \right) \right) \P{\Zt\Qt} \\
    & -\mu\LAPT \left(M_t K - \frac{1}{2}\P{\Zt\Qt}\P{\Zt\Qt}^T \left(K^TM_t^T + M_t K \right) \right)\P{\Zt\Qt} \\
    & -\frac{1}{2}\LAPT B\P{\Zt\Qt}\P{\Zt\Qt}^T \left( B^T + B \right) \P{\Zt\Qt} \\
    & -\LAPT \left(\I + B\right)\P{\Zt\Qt} \left(\frac{1}{2}\P{\Zt\Qt}^TB^TB\P{\Zt\Qt} - C\right).
\end{align*}
It follows that
\begin{align}
    &\norm{\LAPT H \left(H^TH \right)^{-1}} \nonumber \\
    \le & \norm{\LAPT \left(\I - \mu M_t + \mu\frac{1}{2}\P{\Zt\Qt}\P{\Zt\Qt}^T \left( M_t^T + M_t \right) \right)\P{\Zt\Qt}} \nonumber \\ 
    & + 2\norm{K} + 2\mu\norm{M_t K} + \frac{1}{2}\norm{B}\norm{B^T + B} + \left(1 + \norm{B} \right) \left(\norm{B}^2 + \norm{C} \right) \nonumber \\
    \le & \norm{\LAPT \left(\I - \mu M_t + \mu\frac{1}{2}\P{\Zt\Qt}\P{\Zt\Qt}^T \left(M_t^T + M_t \right) \right)\P{\Zt\Qt}} \nonumber \\ 
    & + 2\norm{K} + 2\mu\norm{ M_t }\norm{K} + 2\norm{B}^2 + \norm{B}^3 + \left(1 + \norm{B} \right)\norm{C}. \label{ineq:aux4}
\end{align}
In order to proceed, we first note that
\begin{align*}
    &\LAPT \left(\I - \mu M_t + \mu\frac{1}{2}\P{\Zt\Qt}\P{\Zt\Qt}^T \left( M_t^T + M_t \right) \right) \P{\Zt\Qt} \\
    =&\LAPT \left(\I - \mu M_t + \mu  \P{\Zt\Qt}\P{\Zt\Qt}^T M_t \right)\P{\Zt\Qt} \\
    = &\LAPT \left(\I - \mu \left(\I - \P{\Zt\Qt}\P{\Zt\Qt}^T \right) \left(\Zt\ZtT - \tilZt\tilZtT - \symA + \Deltat \right) \right)\P{\Zt\Qt} \\
    = &\LAPT\P{\Zt\Qt} - \mu\LAPT \left(\I - \P{\Zt\Qt}\P{\Zt\Qt}^T \right)\left(\Zt\ZtT - \tilZt\tilZtT\right)\P{\Zt\Qt} \\
    & + \mu\LAPT \left(\I - \P{\Zt\Qt}\P{\Zt\Qt}^T\right) \symA\P{\Zt\Qt} + \mu \LAPT \left(\I - \P{\Zt\Qt}\P{\Zt\Qt}^T \right) \Deltat \P{\Zt\Qt}     \\
    = &\LAPT\P{\Zt\Qt} - \mu\LAPT\P{\Zt\Qt,\bot}\P{\Zt\Qt,\bot}^T(\Zt\Qtp\QtpT\ZtT - \tilZt\tilZtT)\P{\Zt\Qt} \\
    & - \mu\LAPT\tilLA\Sigma_X\tilLAT\LAP\LAPT\P{\Zt\Qt} - \mu\LAPT\P{\Zt\Qt}\P{\Zt\Qt}^T\symA\P{\Zt\Qt} \\
    & +  \mu \LAPT  \P{\Zt\Qt, \perp}\P{\Zt\Qt, \perp}^T \Deltat \P{\Zt\Qt} \\
    =& \underbrace{ \left( \I - \mu\LAPT\tilLA\Sigma_X\tilLAT\LAP \right) \LAPT\P{\Zt\Qt} \left(\I - \mu\P{\Zt\Qt}^T\symA\P{\Zt\Qt} \right) }_{=:(i)} \\
    & - \mu^2 \underbrace{ \LAPT\tilLA\Sigma_X\tilLAT\LAP\LAPT\P{\Zt\Qt}\P{\Zt\Qt}^T\symA\P{\Zt\Qt}}_{=:(ii)} \\
    & - \mu \underbrace{\LAPT\P{\Zt\Qt,\bot}\P{\Zt\Qt,\bot}^T\Zt\Qtp\QtpT\ZtT\P{\Zt\Qt}}_{=:(iii)} \\
    & + \mu \underbrace{ \LAPT\P{\Zt\Qt,\bot}\P{\Zt\Qt,\bot}^T\tilZt\tilZtT\P{\Zt\Qt} }_{=:(iv)} + \mu \underbrace{ \LAPT  \P{\Zt\Qt, \perp}\P{\Zt\Qt, \perp}^T \Deltat \P{\Zt\Qt}}_{=:(v)}.
\end{align*}
In the next step, we estimate the spectral norm of the individual summands.
We first note that
\begin{align*}
   \norm{(i)}= &\norm{ \left(\I - \mu\LAPT\tilLA\Sigma_X\tilLAT\LAP \right)\LAPT\P{\Zt\Qt} \left(\I - \mu\P{\Zt\Qt}^T\symA\P{\Zt\Qt} \right)} \\
    \stackrel{(a)}{\le} &\norm{\LAPT\P{\Zt\Qt} \left(\I - \mu\P{\Zt\Qt}^T\symA\P{\Zt\Qt} \right)} \\
    %\le &\norm{\LAPT\P{\Zt\Qt} \left(\I - \mu\P{\Zt\Qt}^T\symA\P{\Zt\Qt} \right)} \\
    \le &\norm{\LAPT\P{\Zt\Qt}}\norm{\I - \mu\P{\Zt\Qt}^T\symA\P{\Zt\Qt}} \\
    = &\norm{\LAPT\P{\Zt\Qt}}\norm{\I - \mu\P{\Zt\Qt}^T \left(\LA\Sigma_X\LAT - \tilLA\Sigma_X\tilLAT \right) \P{\Zt\Qt}} \\
    \le &\norm{\LAPT\P{\Zt\Qt}} \left(1 - \mu\lambda_{\text{min}} \left(\P{\Zt\Qt}^T\LA\Sigma_X\LAT\P{\Zt\Qt} \right) + \mu\norm{\P{\Zt\Qt}^T\tilLA\Sigma_X\tilLAT\P{\Zt\Qt}} \right) \\
    \le &\norm{\LAPT\P{\Zt\Qt}} \left(1 - \mu\sglmin^2(\LAT \P{\Zt\Qt})\sglmin(X) + \mu\norm{\tilLAT\P{\Zt\Qt}}^2\norm{X} \right) \\
    = &\norm{\LAPT\P{\Zt\Qt}}\left(1 - \mu(1 - \norm{\LAPT\P{\Zt\Qt}}^2)\sglmin(X) + \mu\norm{\LAPT\P{\Zt\Qt}}^2\norm{X} \right) \\
    \stackrel{(b)}{\le} &\norm{\LAPT\P{\Zt\Qt}} \left(1 - \frac{1}{2}\mu\sglmin(X) \right).
\end{align*}
In inequality $(a)$ we used the assumption $\mu \le \frac{c}{\norm{X}\kappa}$ and in inequality $(b)$ we used the assumption that $\norm{\LAPT\P{\Zt\Qt}}\le \frac{c}{\kappa}$.
Next, we note that
\begin{align*}
    \norm{(ii)} =  \norm{\LAPT\tilLA\Sigma_X\tilLAT\LAP\LAPT\P{\Zt\Qt}\P{\Zt\Qt}^T\symA\P{\Zt\Qt}} \le  \norm{X}^2.
\end{align*}
Moreover, we have that
\begin{align*}
    \norm{(iii)}=& 
    \norm{\LAPT\P{\Zt\Qt,\bot}\P{\Zt\Qt,\bot}^T\Zt\Qtp\QtpT\ZtT\P{\Zt\Qt}}  \\
    =&\norm{\LAPT\P{\Zt\Qt,\bot}\P{\Zt\Qt,\bot}^T\Zt\Qtp\QtpT\ZtT\LAP\LAPT\P{\Zt\Qt}}  \\
    \le &\norm{\Zt\Qtp}^2\norm{\LAPT\P{\Zt\Qt}} \\
    \le &c^2\sglmin(X)\norm{\LAPT\P{\Zt\Qt}},
\end{align*}
where in the last line we use the assumption $\norm{\Zt\Qtp}\le c\sqrt{\sglmin(X)}$.
Next, we note that
\begin{align*}
    \norm{(iv)}= &\norm{\LAPT\P{\Zt\Qt,\bot}\P{\Zt\Qt,\bot}^T\tilZt\tilZtT\P{\Zt\Qt}}  \\
    \le & \norm{\tilZt} \norm{ \tilZtT \P{\Zt\Qt} }\\
    \le & 2 \sqrt{\norm{X}} \norm{ \tilZtT \P{\Zt\Qt} },
\end{align*}
where in the last line we used that $  \norm{\tilZt} = \norm{\Zt} $ due to symmetry and that by assumption $ \norm{\Zt} \le 2\sqrt{\norm{X}} $.
Moreover, we note that
\begin{align*}
    \norm{(v)}&= \norm{ \LAPT  \P{\Zt\Qt, \perp}\P{\Zt\Qt, \perp}^T \Deltat \P{\Zt\Qt} } \le \norm{ \Deltat }. %\le c\sglmin (A).
\end{align*}
By adding up all summands, we obtain that
\begin{equation}\label{ineq:aux6}
    \begin{split}
        &\norm{\LAPT(\I - \mu M_t + \mu\frac{1}{2}\P{\Zt\Qt}\P{\Zt\Qt}^T(M_t^T + M_t))\P{\Zt\Qt}} \\
        \le &\left(1 - \left(\frac{1}{2} - c^2\right) \mu\sglmin(X) \right)\norm{\LAPT\P{\Zt\Qt}} +  \mu^2\norm{X}^2 
        + 2 \mu  \sqrt{\norm{X}} \norm{ \tilZtT \P{\Zt\Qt} }
        + \mu \norm{\Deltat}. % c  \sglmin (A).   
    \end{split}
\end{equation}

Furthermore, we observe that
\begin{align*}
    \norm{C} \stackrel{(a)}{\le} &3\norm{\P{\Zt\Qt}^T \left(B^T + B \right) \P{\Zt\Qt} + \P{\Zt\Qt}^TB^TB\P{\Zt\Qt}}^2 \\
    \le &3 \left(2\norm{B} + \norm{B}^2 \right)^2 \\
    \stackrel{(b)}{\le} &27\norm{B}^2.
\end{align*}
where $(a)$ is inequality \eqref{ineq:intern3}. Inequality $(b)$ follows from $\norm{B}\le 1/6 \le 1$, which we have shown before. 
As a result, we obtain
\begin{align*}
&\norm{\LAPT H(H^TH)^{-1}} \\
    \stackrel{\eqref{ineq:aux4}}{\le} & \norm{\LAPT \left(\I - \mu M_t+ \mu\frac{1}{2}\P{\Zt\Qt}\P{\Zt\Qt}^T \left(M_t^T + M_t \right) \right)\P{\Zt\Qt}} \\ 
    & + 2\norm{K} + 2\mu\norm{M_t}\norm{K} + 2\norm{B}^2 + \norm{B}^3 + \left( 1 + \norm{B} \right) \norm{C} \\
    \stackrel{(a)}{\le} & \norm{\LAPT \left(\I - \mu M_t + \mu\frac{1}{2}\P{\Zt\Qt}\P{\Zt\Qt}^T \left(M_t^T + M_t \right) \right)\P{\Zt\Qt}} + 3\norm{K} + 57\norm{B}^2 \\
    \stackrel{\eqref{ineq:aux5}}{\le} & \norm{\LAPT \left(\I - \mu M_t + \mu\frac{1}{2}\P{\Zt\Qt}\P{\Zt\Qt}^T \left(M_t^T + M_t \right) \right)\P{\Zt\Qt}} + 3\norm{K} + 57 \left( \frac{3\mu}{2} \norm{M_t} + \norm{K}\right)^2 \\
    \stackrel{(b)}{\le} & \norm{\LAPT \left(\I - \mu M_t + \mu\frac{1}{2}\P{\Zt\Qt}\P{\Zt\Qt}^T \left(M_t^T + M_t \right) \right) \P{\Zt\Qt}} + 200\norm{K} + 500 \mu^2\norm{M_t}^2 \\
    \stackrel{\eqref{ineq:aux6}, \eqref{ineq:aux7}}{\le} & \left(1 - \left(\frac{1}{2} - c^2\right) \mu\sglmin(X) \right)\norm{\LAPT\P{\Zt\Qt}} +  \mu^2\norm{X}^2 +  2 \mu  \sqrt{\norm{X}} \norm{ \tilZtT \P{\Zt\Qt} }+   \mu \norm{\Deltat}  \\
    & + 140000c\mu\sglmin(X)\norm{\LAPT\P{\Zt\Qt}} + 3200\mu\frac{\sqrt{\norm{X}}\norm{\tilZtT\Zt \Qtp}}{\sglmin\left(\Zt\Qt\right)} +3200 \mu \norm{\Deltat}\\
    &+ \tilde{C} \mu^2\norm{X}^2 + \tilde{C} \mu^2 \norm{\Deltat} \\
    \le& \left(1 - \frac{1}{4}\mu\sglmin \left(X\right) \right)\norm{\LAPT\P{\Zt\Qt}} 
    +  2 \mu  \sqrt{\norm{X}} \norm{ \tilZtT \P{\Zt\Qt} }
    +C\mu\frac{\sqrt{\norm{X}}\norm{\tilZtT\Zt \Qtp}}{\sglmin\left(\Zt\Qt\right)}\\
    & + C \mu \norm{\Deltat} +  C\mu^2\norm{X}^2.
\end{align*}
In inequality $(a)$ we used that $\norm{B}\le 1/6\le 1$, $ \norm{C} \le 27 \norm{B}^2 $, and  $2\mu \norm{M_t} \le 1 $. 
The latter follows directly from the definition of $M_t$ combined with our assumptions.
In inequality $(b)$ we used the elementary inequality $ab \le \frac{a^2 + b^2}{2} $ and that $\norm{K} \le 1/12$.
The last two inequalities follow from the assumption on our step size $\mu$, by choosing the absolute constants $ \tilde{C}, C>0$ large enough, and by choosing the absolute constant $c>0$ small enough.
\end{proof}