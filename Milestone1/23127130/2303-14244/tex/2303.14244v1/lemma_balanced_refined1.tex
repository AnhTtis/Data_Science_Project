\subsection{Proof of Lemma \ref{lemma:balancednessperp}: Controlling $\norm{ \tilZtT \Zt\Qtp }$}\label{sec:balancednessperp}



\begin{proof}[Proof of Lemma \ref{lemma:balancednessperp}]

We first note that 
\begin{equation*}
    \tilZplus^T \Zplus \Qplusp =  \tilZplus^T \LAP \LAPT \Zplus \Qplusp. 
\end{equation*}
Analogously as in the proof of Lemma \ref{lemma:noisetermgrowth} by using the assumption that $\LAT \Zplus \Qt $ has full rank we can derive that 
\begin{equation*}
    \LAPT\Zplus\Qplusp = -\LAPT\P{\Zplus\Qt}(\LAT\P{\Zplus\Qt})^{-1}\LAT\Zplus\Qtp\QtpT\Qplusp + \LAPT\Zplus\Qtp\QtpT\Qplusp.    
\end{equation*}
Hence, we obtain by the triangle inequality that
\begin{align*}
    &\norm{\tilZplus^T \Zplus \Qplusp} \\
 \le & \norm{\tilZplus^T \LAP \LAPT\P{\Zplus\Qt}(\LAT\P{\Zplus\Qt})^{-1}\LAT\Zplus\Qtp\QtpT\Qplusp  }
 + \norm{\tilZplus^T \LAP \LAPT\Zplus\Qtp\QtpT\Qplusp  } \\
 \le & \underbrace{ \norm{\tilZplus^T \LAP \LAPT\P{\Zplus\Qt}(\LAT\P{\Zplus\Qt})^{-1}\LAT\Zplus\Qtp }}_{=:(I)} 
 + \underbrace{\norm{\tilZplus^T \LAP \LAPT\Zplus\Qtp }}_{=:(II)}.
\end{align*}
We estimate the two summands individually.\\

\noindent \textbf{Bounding (I):} 
First, we recall that
\begin{align*}
    \Zplus   &= \left( \I -\mu M_t \right) \Zt, \\
    \tilZplus&=\left( \I +\mu M_t \right) \tilZt,
\end{align*}
where
\begin{equation*}
    M_t= \Zt\ZtT - \tilZt\tilZtT - \symA +\Deltat.
\end{equation*}
As shown in inequality \eqref{ineq:boundforM} we have
\begin{equation}\label{ineq:boundforM3}
\norm{M_t} \le 10 \norm{X}.
\end{equation}
Then we compute that
\begin{align*}
    \LAT\Zplus\Qtp =& \LAT\left(\Id - \mu M_t\right) \Zt \Qtp \\
    =& \LAT\left(\Id - \mu M_t\right) \LAP \LAPT \Zt \Qtp \\
    =& - \mu\LAT M_t \LAP \LAPT \Zt \Qtp.
\end{align*}
By the submultiplicativity of the spectral norm it follows that 
\begin{align}
    (I) = & \norm{\tilZplus^T \LAP \LAPT\P{\Zplus\Qt}(\LAT\P{\Zplus\Qt})^{-1}\LAT\Zplus\Qtp } \nonumber \\
    = & \norm{- \mu\tilZplus^T \LAP \LAPT\P{\Zplus\Qt}(\LAT\P{\Zplus\Qt})^{-1}\LAT M_t \LAP \LAPT \Zt \Qtp } \nonumber \\
    \le & \mu   \norm{\tilZplus}  \frac{ \norm{ \LAPT\P{\Zplus\Qt} }}{ \sglmin(\LAT\P{\Zplus\Qt}) } \norm{ \LAT M_t\LAP } \norm{ \Zt \Qtp}. \label{ineq:secbalancing1}
\end{align}
Analogously as in Lemma \ref{lemma:anglecontrol} (see inequality \eqref{ineq:aux1212}) we can show that 
\begin{align}
    \norm{\LAPT\P{\Zplus\Qt}} 
    &\le  2\norm{\LAPT\P{\Zt\Qt}} + 20\mu\norm{X}. \label{ineq:balancingaux1} 
%    & \le 24 \beta. 
\end{align}
Note that due to our assumption on $ \mu $, $\norm{\Deltat}$, and $ \norm{\LAPT\P{\Zt\Qt}}  $ this also implies that $  \norm{\LAPT\P{\Zplus\Qt}}  \le \frac{1}{2}$, which implies that $\sglmin(\LAT\P{\Zplus\Qt})  \ge 1/2 $. 
Thus, from \eqref{ineq:secbalancing1} it follows that
\begin{align*}
    (I) 
    \le& 2 \mu \norm{\tilZplus}  \norm{ \LAPT\P{\Zplus\Qt} } \norm{ \LAT M_t\LAP } \norm{ \Zt \Qtp}\\
    \le& 4 \mu \sqrt{\norm{X}}  \norm{ \LAPT\P{\Zplus\Qt} } \norm{ \LAT M_t\LAP } \norm{ \Zt \Qtp}\\
    \le& 8 \mu  \left( \norm{\LAPT\P{\Zt\Qt}} + 10\mu\norm{X} \right) \norm{ \LAT M_t\LAP } \sqrt{\norm{X}} \norm{ \Zt \Qtp}.
\end{align*}
In the second inequality we used the assumption $ \norm{\Zplus} \le 2 \sqrt{\norm{X}} $ and that by symmetry it holds that $ \norm{\Zplus} = \norm{\tilZplus} $. In the third inequality we used inequality \eqref{ineq:balancingaux1}.\\
\noindent \textbf{Bounding (II):}
First, we observe that
\begin{align*}
    &\tilZplus^T \LAP \LAPT\Zplus\Qtp \\
    =& \tilZtT \left( \I +\mu M_t  \right) \LAP \LAPT  \left( \I -\mu M_t \right) \Zt\Qtp \\
    =&  \tilZtT \LAP \LAPT \Zt\Qtp  + \mu \tilZtT \left(  M_t  \LAP \LAPT -  \LAP \LAPT M_t \right) \Zt\Qtp - \mu^2 \tilZtT M_t \LAP \LAPT  M_t \Zt\Qtp \\
    =&  \tilZtT \Zt\Qtp  + \mu \tilZtT \left(  M_t  -  \LAP \LAPT M_t \right) \LAP \LAPT\Zt\Qtp - \mu^2 \tilZtT M_t \LAP \LAPT  M_t \Zt\Qtp \\
    =&  \tilZtT\Zt\Qtp  + \mu \tilZtT \LA \LAT M_t \LAP \LAPT  \Zt\Qtp - \mu^2 \tilZtT M_t \LAP \LAPT  M_t \Zt\Qtp.
\end{align*}
It follows from the triangle inequality that 
\begin{align*}
    (II) \le &  \norm{\tilZtT \Zt\Qtp} + \mu \norm{\tilZtT \LA \LAT M_t \LAP \LAPT  \Zt\Qtp} + \mu^2 \norm{\tilZtT M_t \LAP \LAPT  M_t \Zt\Qtp} \\
    \le &  \norm{\tilZtT \Zt\Qtp} + \mu \norm{ \LAT \tilZt} \norm{  \LAT M_t \LAP} \norm{ \Zt\Qtp } + \mu^2  \norm{\tilZt} \norm{ M_t}^2  \norm{ \Zt\Qtp }.
\end{align*}
Next, we note that
\begin{align*}
    \norm{ \LAT \tilZt} = & \norm{ \tilLAT \Zt} \\
    \le & \norm{ \LAPT \Zt} \\
    \le & \norm{ \LAPT \Zt \Qt} + \norm{ \LAPT \Zt \Qtp} \\
    \le & \norm{ \LAPT \PZQ } \norm{ \Zt } + \norm{\Zt \Qtp}.
\end{align*}
It follows that 
\begin{align*}
    (II)\le & \norm{\tilZtT \Zt\Qtp} + \mu \left(\norm{ \LAPT \PZQ } \norm{ \Zt } + \norm{\Zt \Qtp}\right)\norm{  \LAT M_t \LAP} \norm{ \Zt\Qtp } \\
    &+ \mu^2  \norm{\tilZt} \norm{M_t}^2  \norm{ \Zt\Qtp } \\
    \le & \norm{\tilZtT \Zt\Qtp} + \mu \left(2\norm{ \LAPT \PZQ } \sqrt{\norm{X}} + \norm{\Zt \Qtp}\right)\norm{  \LAT M_t \LAP} \norm{ \Zt\Qtp } \\
    &+ 200\mu^2 \norm{X}^{\frac{5}{2}}  \norm{ \Zt\Qtp },
\end{align*}
where in the second inequality we used the inequality \eqref{ineq:boundforM3}, the assumption $ \norm{\Zt} \le 2 \sqrt{\norm{X}}$ and that by symmetry it holds that $ \norm{\Zt} = \norm{\tilZt} $.\\

\noindent \textbf{Combining the estimates:}
By combining the previous two steps we obtain that 
\begin{align}
    & \norm{\tilZplus^T \Zplus \Qplusp} \nonumber \\
    \le & 8 \mu  \left( \norm{\LAPT\P{\Zt\Qt}} + 10\mu\norm{X} \right) \norm{ \LAT M_t\LAP } \sqrt{\norm{X}} \norm{ \Zt \Qtp} +  \norm{ \tilZtT \Zt\Qtp } \nonumber \\
    & + \mu \left(2\norm{ \LAPT \PZQ } \sqrt{\norm{X}} + \norm{\Zt \Qtp}\right)\norm{  \LAT M_t \LAP} \norm{ \Zt\Qtp } + 200\mu^2 \norm{X}^{\frac{5}{2}} \norm{ \LAPT  M_t}^2  \norm{ \Zt\Qtp } \nonumber\\
    =&  \norm{ \tilZtT \Zt\Qtp }  + 2\mu \left(  \left( 5\norm{\LAPT\P{\Zt\Qt}} + 40\mu\norm{X} \right) \norm{ \LAT M_t\LAP }  + 100\mu \norm{X}^2 \right)  \sqrt{\norm{ X }} \norm{ \Zt\Qtp }\nonumber\\
    &+\mu \norm{ \LAT M_t \LAP } \norm{\Zt \Qtp}^2. \label{ineq:secbalancing4}
\end{align}
Next, we estimate $ \norm{\LAT M_t\LAP} $. For that, we first calculate 
\begin{align*}
    \LAT M_t\LAP 
    &= \LAT \left(   \Zt\ZtT - \tilZt\tilZtT - \symA +\Deltat  \right)  \LAP \\
    &=  \LAT   \Zt\ZtT  \LAP -  \LAT  \tilZt\tilZtT  \LAP  +  \LAT \Deltat  \LAP \\
    &=  \LAT   \Zt \Qt \QtT \ZtT  \LAP  -  \LAT  \tilZt \Qt \QtT  \tilZtT  \LAP  -  \LAT  \tilZt \Qtp \QtpT  \tilZtT  \LAP  +  \LAT \Deltat  \LAP. 
\end{align*}
It follows that 
\begin{align*}
    \norm{ \LAT M_t\LAP } \le& \norm{ \Zt \Qt } \norm{ \LAPT   \Zt \Qt } + \norm{ \LAT  \tilZt \Qt } \norm{ \tilZt \Qt } + \norm{ \tilZt \Qtp  }^2 + \norm{\Deltat}\\
    = & \norm{ \Zt \Qt } \norm{ \LAPT   \Zt \Qt } + \norm{ \tilLAT  \Zt \Qt } \norm{ \tilZt \Qt } + \norm{ \tilZt \Qtp  }^2 + \norm{\Deltat}\\
    \le& \norm{     \Zt \Qt } \norm{ \LAPT   \Zt \Qt } + \norm{ \LAPT  \Zt \Qt } \norm{   \tilZt \Qt  } + \norm{ \tilZt \Qtp  }^2 + \norm{\Deltat}\\
    \stackrel{(a)}{=}& 2 \norm{     \Zt \Qt } \norm{ \LAPT   \Zt \Qt } + \norm{ \Zt \Qtp  }^2 + \norm{\Deltat} \\
    \le& 2 \norm{\LAPT \PZQ}  \norm{     \Zt \Qt }^2  + \norm{ \Zt \Qtp  }^2 + \norm{\Deltat},
\end{align*}
where in equality $(a)$ we used the symmetry between $\Zt$ and $\tilZt$, see Lemma \ref{lemma:symmetry}.
Using the assumption $ \norm{\Zt} \le 2 \sqrt{\norm{X}} $, we obtain that
\begin{align}
    \norm{ \LAT M_t\LAP }
    &\le  8 \norm{\LAPT \PZQ}  \norm{X}  + \norm{ \Zt \Qtp  }^2 + \norm{\Deltat}\le 8 \beta,\label{ineq:secbalancing3} 
\end{align}
where we have set $ \beta := \norm{\LAPT \PZQ}  \norm{X}  + \norm{ \Zt \Qtp  }^2 + \norm{\Deltat}$.
Inserting \eqref{ineq:secbalancing3} into \eqref{ineq:secbalancing4} we obtain that
\begin{align*}
    &\norm{\tilZplus^T \Zplus \Qplusp} \\   
    \le &\norm{ \tilZtT \Zt\Qtp }  + 2\mu \left(  8 \left( 5\norm{\LAPT\P{\Zt\Qt}} + 40\mu\norm{X} \right) \beta + 100 \mu \norm{X}^2 \right)  \sqrt{\norm{ X }} \norm{ \Zt\Qtp } + 8 \mu \beta \norm{\Zt \Qtp}^2\\
    \le &\norm{ \tilZtT \Zt\Qtp }  + C\mu \left(   \left( \norm{\LAPT\P{\Zt\Qt}} + \mu\norm{X} \right) \beta +  \mu \norm{X}^2 \right)  \sqrt{\norm{ X }} \norm{ \Zt\Qtp }+ C \mu \beta \norm{\Zt \Qtp}^2,
\end{align*}
where the last line follows by choosing the constant $C>0$ large enough.
This proves the claim.
\end{proof}








