%\section{Additional Numerical Experiments}
%\label{addexp}
\subsection{Impact of step size on balancedness}
% parameters: $n_1 = 100, n_2 = 50, r = 5, r = 10, m = 2000, \alpha = 0.00001$.
In this section, we focus on understanding how different choices of step size impact the spectral norm of the imbalance term $\norm{V_t^T V_t - W_t^T W_t}$.
To this aim, we fix $k=10$ and $\alpha= 10^{-5}$ and we choose different step sizes $\mu$.
In Figure \ref{fig:balancedness_stepsize1}, we show how the evolution of $\norm{V_t^T V_t - W_t^T W_t} $ changes with different step sizes.
We observe that the spectral norm of the imbalance term, $\norm{V_t^T V_t - W_t^T W_t}$, behaves qualitatively similarly regardless of the choice of the step size.
The norm stays first constant and then grows rapidly, after which it stays roughly constant again.
This indicates that most of the growth of this spectral norm happens during the second phase when the signal is growing.
However, what changes with different step sizes is the threshold which $\norm{V_t^T V_t - W_t^T W_t} $ converges to at the end of training. 
Indeed, Figure \ref{fig:balancedness_stepsize1} indicates that larger step sizes lead to a larger threshold after training.
\begin{figure}[t]
    \centering
    \includegraphics[scale=0.7]{Figures/Figures_Balance_vs_stepsize.pdf}
    \caption{Evolution of the spectral norm of the imbalance term $\norm{V_t^T V_t - W_t^T W_t}$ during training with different step sizes}
     \label{fig:balancedness_stepsize1}
\end{figure}

In the next experiment, we want to examine how the value of this threshold at the end of training depends on the step size $\mu$.
For that, we repeat the experiment for $\mu \norm{X} = \{0.01, 0.02, \ldots, 0.10\}$ and show the relations between threshold and the step size.
% parameters: $n_1 = 100, n_2 = 50, r = 5, r = 10, m = 2000, T = 3000$.
The results are depicted in Figure \ref{fig:balancedness_stepsize2}.
We observe that at the end of training the spectral norm of the imbalance term $\norm{V_t^T V_t - W_t^T W_t} $ depends linearly on the step size $\mu$.
We note that this observation is well-aligned with our theory. In particular, we infer from Lemma \ref{lemma:balancedbase} that for each iteration $\norm{V_t^T V_t - W_t^T W_t} $ grows by an additive term which scales quadratically with the step size $\mu$.
Furthermore, Theorem \ref{theorem:main} shows that the number of iterations needed for convergence is proportional to the inverse of the step size $\mu$. Combining these two results, our theory predicts that the scaling for the threshold after convergence should be linear in the step size $\mu$.
\begin{figure}[t]
    \centering
    \includegraphics[scale=0.7]{Figures/Figures_Balancevsstepsize_dynamic.pdf}
    \caption{Spectral norm of the imbalance term $\norm{V_t^T V_t - W_t^T W_t}$ at convergence with different choices of the step size $\mu$}
    \label{fig:balancedness_stepsize2}
\end{figure}

\subsection{Evolution of the two additional couplings of the trajectory}
In this section, we focus on the behavior of  the two additional coupling of the trajectories of the factors $V_t$ and $W_t$ studied in this paper (see Section \ref{sec:threephase}). For that, we set $k=10$, the initialization scale $\alpha = 10^{-6}$ and the step size $\mu = \frac{1}{100\|X\|}$. 


In the first experiment, depicted in Figure \ref{fig:BalanceVariant2}, we compare the evolution of the spectral norm of the imbalance term $\norm{V_t^T V_t - W_t^T W_t}$ and its nuisance part $\norm{(V_t^T V_t - W_t^T W_t)\Qtp} = 2\norm{\tilZt^T\Zt\Qtp}$ during training. We observe that the nuisance part is significantly smaller than the total imbalancedness. This phenomenon inspires us to do a tighter analysis of $\norm{\tilZt^T\Zt\Qtp}$ (see Lemma \ref{lemma:balancednessperp} for details). We note that this careful analysis is critical to our convergence analysis, allowing us to show good generalization and convergence with only a modest number of iterations.

In the next experiment, we show the evolution of the angle between the imbalance matrix and the signal direction $2\norm{\tilZtT\P{\Zt\Qt}}$ in Figure \ref{fig:BalanceAngle2}. We observe that this quantity remains small during training matching our analysis for this quantity in Lemma \ref{ref:balancednessangle}.



\begin{figure}[t]
    \centering\includegraphics[scale=0.7]{Figures/Figures_BalanceVariant2.pdf}
    \caption{Evolution of the spectral norm of the imbalance term $\norm{V_t^T V_t - W_t^T W_t}$ and its nuisance part $2\|\tilde{Z}_t^T Z_t Q_{t,\bot}\|$ during training.}
    \label{fig:BalanceVariant2}
\end{figure}

\begin{figure}[t]
    \centering  \includegraphics[scale=0.7]{Figures/Figures_BalanceAngle2.pdf}
    \caption{Evolution of the angle between the imbalance matrix and the signal direction $2\norm{\tilZtT\P{\Zt\Qt}}$ during training.}
    \label{fig:BalanceAngle2}
\end{figure}