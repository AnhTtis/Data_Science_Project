
\subsection{Proof of Lemma \ref{lemma:normcontrolled}}

\begin{lemma}\label{lemma:normcontrolled}
Suppose that $\norm{\Zt} \le 2$, $ \norm{\Deltat} \le c \norm{A} $, $\norm{B_t}\le \frac{1}{2} \norm{A}$, and $ \mu \le \frac{c}{\norm{A}} $.
Then it holds that
\begin{equation*}
    \norm{\Zplus} \le 2 \sqrt{\norm{A}}.
\end{equation*}
\end{lemma}

\begin{proof}
 First, we note that
 \begin{align*}
\Zplus &= \Zt -\mu \left( \Zt \ZtT - \tilZt \tilZtT + \symA \right) \Zt + \mu \Deltat \Zt\\
& =\left( \Id - \mu  \Zt \ZtT\right)\Zt + \mu \tilZt \tilZtT \Zt + \mu \cdot \symA \Zt + \mu \Deltat \Zt.
 \end{align*}   
 It follows that
 \begin{align}
 \norm{ \Zplus } 
 &\le  \norm{ \left( \Id - \mu  \Zt \ZtT\right)\Zt } + \mu \norm{ \tilZt} \norm{ \tilZtT \Zt  } + \mu \norm{\symA} \norm{\Zt}  + \mu  \norm{ \Deltat} \norm{Zt} \nonumber  \\
 &\stackrel{(a)}{=}  \norm{ \left( \Id - \mu  \Zt \ZtT\right)\Zt } +   \mu  \norm{B_t} \norm{\Zt} +   \mu \norm{A} \norm{\Zt} + \mu \norm{ \Deltat} \norm{Zt} \nonumber \\ 
 &\stackrel{(b)}{\le}  \norm{ \left( \Id - \mu  \Zt \ZtT\right)\Zt } +  2 \mu \norm{A} \norm{\Zt} \nonumber \\
 & \stackrel{(c)}{=} \left( 1- \mu \norm{\Zt}^2\right) \norm{\Zt} +  2 \mu \norm{A} \norm{\Zt} \nonumber\\
&= \left( 1- \mu \norm{\Zt}^2 + 2 \mu \norm{A} \right) \norm{\Zt}. \label{ineq:aux24}
 \end{align}
In equality $(a)$ we used the definition of $B_t$, the fact that $\norm{\tilZt} =\norm{\Zt} $, and that $\norm{\symA}=\norm{A}$, which follows from the definition of $\symA$.
In inequality $(b)$ we used the assumptions $\norm{B_t} \le c \norm{A}  $ and $ \norm{\Deltat} \le c \norm{A} $.
Equality $(c)$ follows from the singular value decomposition of $ \left( \Id - \mu  \Zt \ZtT\right)\Zt $ and $\Zt$, the fact that the function $x \mapsto (1-x^2)x $ is increasing in the interval $ x\in (0,\frac{1}{\sqrt{3}}) $, as well as the assumptions $ \mu \le \frac{c}{\norm{A}} $ and $\norm{\Zt} \le 2\sqrt{\norm{A}} $.

In order to show the claim, we will distinguish two cases.
First, we assume that $ \norm{\Zplus} < \frac{3}{2} \sqrt{\norm{A}} $. 
Then the claim $\norm{\Zplus} \le 2\sqrt{\norm{A}} $ follows immediately from inequality \eqref{ineq:aux24} combined with the assumption $ \mu \le \frac{c}{\norm{A}}  $. 
For the second case, we assume that $ \frac{3}{2} \norm{A} \le \norm{\Zplus} \le 2 \sqrt{ \norm{A} } $. 
Then we can verify that \eqref{ineq:aux24} implies that $ \norm{\Zplus} \le \norm{\Zt} $.
Since we assumed that $ \norm{\Zt} \le 2 \sqrt{A} $, this implies in particular that $ \norm{\Zplus} \le 2 \sqrt{ \norm{A} } $. 
This finishes the proof.
\end{proof}
