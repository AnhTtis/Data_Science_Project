%!TEX root = main.tex

In this section we investigate $f_2$ on \algGlobalSemoAlt.
The main result of this section is the following theorem.

\begin{theorem}
	\label{thm::OptSolf2}
	Let $G=(V,E)$ be an instance of the $W$-separator problem.
	Using the fitness function $f_2$, the expected number of iterations of \algGlobalSemoAlt until an optimal solution is sampled is upper bounded by $\O(n^3(\log n + \opt) + n^2 \cdot 4^{\opt \cdot W} )$.
\end{theorem}
 
First we give a brief overview of a reducible structure concerning the $W$-separator problem associated with the objectives in the fitness function $f_2$.   
The structure we will use is commonly known as crown decomposition.
Roughly speaking, it is a division of the set of vertices into three parts consisting of a crown, a head, and a body, with the head separating the crown from the body.
Under certain conditions concerning the crown and head vertices, which we will clarify in a moment, it is possible to show that there exists an optimal $W$-separator which contains the head vertices and reduces the given instance by removing the crown vertices.
Recall that the parameter $k$ in the decision variant of the $W$-separator asks for a $W$-separator of size at most $k$.
Kumar and Lokshtanov~\cite{DBLP:conf/iwpec/KumarL16} provide such a reducible structure and state that it is in a graph as long as the size of it is greater than $2kW$.
The structure is called a (strictly) reducible pair and consists of crown and head vertices.

For an instance $G=(V,E)$ of the $W$-separator problem we say that $Y=\{y_v \in \R_{\geq 0}\}_{v \in V}$ is a \emph{fractional $W$-separator} of $G$ if $Y$ is a feasible solution according to the LP formulation of the $W$-separator problem.
It is not difficult to see that the objective of any optimal fractional $W$-separator is smaller than $\opt$, i.e.~$\lpPrim(G) \leq \opt$.
In principle, the LP objective is useful for finding a strictly reducible pair, since the head vertices in an optimal fractional W separator must have value one.
Unfortunately, it is unknown whether each vertex that has value one in an optimal fractional $W$-separator is part of an optimal solution.
This leads to the challenge of filtering out the right vertices, where the uncovered-objective - and in particular the structural properties of a strictly reducible pair - come into play.


\paragraph{\textbf{Reducible Structure of the $\boldsymbol W$-Separator Problem}}
In the following, we briefly summarize the definitions and theorems of \cite{DBLP:conf/esa/Casel0INZ21,fomin2019kernelization,DBLP:conf/iwpec/KumarL16}. 
For a vertex set $B \subseteq V$, denote by $\B$ the partitioning of $B$ according to the connected components of $G[B]$.

\begin{definition}[(strictly) reducible pair]
\label{def::RedPair}
For a graph $G=(V,E)$, a pair $(A,B)$ of vertex disjoint subsets of $V$ is a \emph{reducible pair} if the following conditions are satisfied:
\begin{itemize}
	\item $N(B) \subseteq A$.
	\item
	The size of each $C \in \B$ is at most $W$.
	\item There is an assignment function $g \colon \B \times A \to \N_0$, such that
	\begin{itemize}
		\item for all $C \in \B$ and $a \in A$, if $g(C,a) \ne 0$, then $a \in N(C)$
		\item for all $a \in A$ we have $\sum_{C \in \B} g(C,a) \geq 2W-1$,
		\item for all $C \in \B$ we have  $\sum_{a \in A} g(C,a) \leq |C|$,
	\end{itemize} 
\end{itemize} 
In addition, if there exists an $a \in A$ such that $\sum_{Q \in \B} g(C,a) \geq 2 W$, then $(A,B)$ is a \emph{strictly reducible pair}.
\end{definition}

Next, we explain roughly the idea behind a reducible pair $(A,B)$.
The head and crown vertices correspond to $A$ and $B$, respectively.
That is, we want $A$ to be part of our $W$-separator, and if that is the case, then no additional vertex from $B$ is required to be in the solution since the components $C \in \B$ are isolated after removing $A$ from $G$ with $|C| \leq W$. 
Let $G=(V,E)$ be a graph. 
We say that $P_1, \dots, P_m \subseteq V$ is a \emph{($W+1$)-packing} if for all $i,j \in [m]$ with $i \neq j$ the induced subgraph $G[P_i]$ is connected, $|P_i| \geq W+1$, and $P_i \cap P_j = \varnothing$.
Note that for a $W$-separator $S \subseteq V$, it holds that $S \cap P_i \neq \varnothing$ for all $i \in [m]$.
Thus, the size of a ($W+1$)-packing is a lower bound on the number of vertices needed for a $W$-separator.

\begin{lemma}[\cite{DBLP:conf/iwpec/KumarL16}, Lemma 17]
\label{lemma::packingReduciblePair}
Let $(A,B)$ be a reducible pair in $G$.
There is a ($W+1$)-packing $P_1, \dots, P_{|A|}$ in $G[A \cup B]$, such that $|P_{i} \cap A| = 1$ for all $i \in [|A|]$.
\end{lemma}

Essentially, \cref{lemma::packingReduciblePair} provides a lower bound of $|A|$ vertices for a $W$-separator in $G[A \cup B]$.
On the other hand, $A$ is a $W$-separator of $G[A \cup B]$ while $A$ separates $B$ from the rest of the graph.
This properties basically admits the following theorem.

\begin{theorem}[\cite{DBLP:conf/iwpec/KumarL16}, Lemma 18]
\label{thm::saveReduction}
Let $(G,k)$ be an instance of the $W$-separator problem, and $(A,B)$ be a reducible pair in $G$.
$(G,k)$ is a yes-instance if and only if $(G - (A \cup B), k - |A|)$ is a yes-instance.
\end{theorem}

Finally, we clarify why a strictly reducible pair exists if the size of $G$ is larger than $2kW$.
To do so, we make use of a lemma derivable from  \cite{DBLP:conf/esa/Casel0INZ21,DBLP:conf/iwpec/KumarL16}.
A proof is given in the appendix~(\cref{appendix::f2}). 

\begin{lemma}
\label{lemma::balancedExpansionStrictlyReducibleExist}
Let $G=\left(A \cup B, E\right)$ be a graph and $W \in \N_0$.
Let $\B$ be the connected components given as vertex sets of $G[B]$, where for each $C \in \B$ we have $|C| \leq W$ and no $C \in \B$ is isolated, i.e.~$N(C) \neq \varnothing$.
If $|B| \geq (2W-1)|A|+1$, then there exists a non-empty strictly reducible pair $(A',B')$, where $A' \subseteq A$ and $B' \subseteq B$.
\end{lemma}

We conclude the preliminary section with a lemma that connects strictly reducible pairs with the size of the graph.

\begin{lemma}[\cite{fomin2019kernelization}, Lemma~6.14]
\label{lemma::existenceReducible}
Let $(G,k)$ be an instance of the $W$-separator problem, such that each component in $G$ has size at least $W+1$.
If $|V| > 2Wk$ and $(G,k)$ is a yes-instance, then there exists a strictly reducible pair $(A,B)$ in $G$. 
\end{lemma}

\paragraph*{\textbf{Running time analysis}} 
Let $(A,B)$ be a strictly reducible pair. 
We say $(A,B)$ is a minimal strictly reducible pair if there does not exist a strictly reducible pair $(A',B')$ with $A' \subset A$ and $B' \subseteq B$.
Clearly, it can happen that reducible pairs arises after a reduction is executed.
Therefore, we say $(A_1,B_1), \dots, (A_m,B_m)$ is a \emph{sequence of minimal strictly reducible pairs} if for all $i \in [m]$ the tuple $(A_i,B_i)$ is a minimal strictly reducible pair in $G - \bigcup_{j=1}^{i-1} A_j$.
Note that the definition of such a sequence implies that those tuples are pairwise disjoint, i.e., $(A_i \cup B_i) \cap (A_j \cup B_j) = \varnothing$ for all $i,j \in [m]$ with $i \neq j$.
The proof of \cref{thm::OptSolf2} can essentially be divided into three phases:
\begin{enumerate}
\item%[Part (i)]
\label{part1}
Let $(A_1,B_1), \dots, (A_m,B_m)$ be a sequence of minimal strictly reducible pairs in $G$, such that $G - \bigcup_{i\in [m]} A_i$ contains no minimal strictly reducible pair. 
The first phase is to show that after a polynomial number of iterations of \algGlobalSemoAlt with fitness $f_2$, a search point $X \in \{0, 1\}^n$ exists in the population $\P$, such that $\lpPrim(G) = |X_1| + \lpPrim(G[u(X)])$ and there is a fractional optimal $W$-separator $Y = \{y_v \in \R_{\geq 0}\}_{v \in u(X)}$ with $y_v < 1$ for each $v \in V$.
We will prove that in this case $G[u(X)]$ contains no strictly reducible pair, and that because of the equality relation $\lpPrim(G) = |X_1| + \lpPrim(G[u(X)])$ all the head vertices $A_i$ for $i \in [m]$ are in $X_1$.
That is, there is an optimal $W$-separator which contains a subset of $X_1$.
\item%[Part (ii)]
\label{part2}
The second phase is to filter $\bigcup_{i=1} A_i$ from $|X_1|$ so that we obtain a search point $X'$ that selects only those as 1-bits.
Once an $X$ as described in Phase~{\ref{part1}} is guaranteed to be in the population, the algorithm \algGlobalSemoAlt takes in expectation FPT-time to reach $X'$.
Finally, it is important that $X'$ remains in the population once we have found it.
We show this by taking advantage of the structural properties of a reducible pair in combination with the uncovered-objective.
\item
\label{part3}
For the last phase, we know by \cref{lemma::existenceReducible} already that $u(X')$ has size at most $2 \cdot \opt \cdot W$. 
Once we ensure that $X'$ is in $\P$ and stays there, we prove that \algGlobalSemoAlt finds in expectation an optimal solution in FPT-time.
\end{enumerate}

In phase~\ref{part1}, we essentially make use of the LP objective.
To prove that it works successfully, we will show the following two lemmas.

\begin{lemma}
\label{lemma::firstStepOptLP}
Using the fitness function $f_2$, the expected number of iterations of \algGlobalSemoAlt where the population $\P$ contains a search point $X \in \{0,1\}^n$ such that $\lpPrim(G)=\lpPrim(G[u(X)]) + |X_1|$ and there is an optimal fractional $W$-separator $\{y_v \in \R_{\geq 0}\}_{v \in u(X)}$ of $G[u(X)]$ with $y_v<1$ for every $v \in u(X)$ is upper bounded by $\O(n^3(\log n + \opt))$.
Moreover, once $\P$ contains such a search point at any iteration, the same holds for all future iterations.
\end{lemma} 

\begin{lemma}
\label{lemma::ingredientParateo1}
Let $(A_1,B_1), \dots, (A_m,B_m)$ be a sequence of minimal strictly reducible pairs in $G$, such that $G - \bigcup_{i=1} A_i$ contains no minimal strictly reducible pair.
Let $X \in \{0,1\}^n$ be a sample, such that there is a an optimal fractional $W$-separator $\{y_v \in \R_{\geq 0}\}_{v \in u(X)}$ of $G[u(X)]$ with $y_v<1$ for each $v \in u(X)$.
If $|X_1| + \lpPrim(G[u(X)]) = \lpPrim(G)$, then $A_i \subseteq X_1$ and $B_i \cap X_1 = \varnothing$ for all $i \in [m]$.
\end{lemma}

We guide the rest of this section by using $X \in \{0,1\}^n$ to denote a search point and $(A_1,B_1), \dots, (A_m,B_m)$ as a sequence of minimal strictly reducible pairs,
where $\aall := \bigcup_{i=1}^m A_i$ and $\ball := \bigcup_{i=1}^m B_i$.
For the Algorithm \algGlobalSemoAlt it is unlikely to jump from a uniformly random search point immediately to a search point satisfying \cref{lemma::ingredientParateo1}. 
To guarantee a stepwise progress,
we want that under the condition $\lpPrim(G)=\lpPrim(G[u(X)]) + |X_1|$ at each time $\aall \subsetneq X_1$ there exists a vertex of $v \in \aall \setminus X_1$ in an optimal fractional $W$-separator of $G[u(X)]$ which must have value one.
For this purpose, the characterization of minimal strictly reducible pairs by optimal fractional $W$-separators is useful.

\begin{lemma}[\cite{fomin2019kernelization}, Corollary~6.19 and Lemma~6.20]
\label{lemma::onesInLPReduciblePair}
Let $G=(V,E)$ be an instance of the $W$-separator problem and let $\{y_v \in \R_{\geq 0}\}_{v \in V}$ be an optimal fractional $W$-separator of $G$.
If $G$ contains a minimal strictly reducible pair $(A,B)$, then $y_v=1$ for all $v \in A$ and $y_u = 0$ for all $u \in B$. 	
\end{lemma}

From \cref{lemma::onesInLPReduciblePair} we can derive that if $(\aall \cup \ball) \cap X_1 = \varnothing$, such a vertex $v$ must exist, but the question is what happens if the intersection is not empty.
In particular, we want to avoid vertices of $\ball$ being in $X_1$, since reducible pairs in $G$ may then no longer exist in $G[u(X)]$.
We start with the proof of \cref{lemma::firstStepOptLP} and show later how it is related to a sequence of minimal strictly reducible pairs.
The first lemma is a simple but useful observation.


\begin{lemma}
\label{lemma::lowerBoundLP}
For every $X \in \{0,1\}^n$ it holds that $\lpPrim(G) \leq |X_1| + \lp(G[u(X)])$.
\end{lemma}

If \cref{lemma::lowerBoundLP} is true, it is not difficult to derive that if we have that it holds with equality for a search point $X$, then $f_2(X)$ is a pareto optimal vector of the fitness function $f_2$, as given below as a corollary.

\begin{corollary}
\label{corollary::paretoOptLP}
If a search point $X \in \{0,1\}^n$ satisfy $|X_1| + \lp(G[u(X)]) = \lpPrim(G)$, then the vector $(|X_1|,*,\lp(G[u(X)])$ is a pareto optimal vector of the fitness function $f_2$.
\end{corollary}

The next lemma ensures that removing vertices with value one in an optimal fractional $W$-separator does not affect the objective of a fractional $W$-separator of the remaining graph.

\begin{lemma}[\cite{fomin2019kernelization}, Corollary 6.17]
\label{lemma::onesInLPSameObjective}
Let $G=(V,E)$ be an instance of the $W$-separator problem and let $\{y_v \in \R_{\geq 0}\}_{v \in V}$ be an optimal fractional $W$-separator of $G$.
Let $V' \subseteq V(G)$, such that $y_v = 1$ for all $v \in V'$.
Then, $\{y_v \mid v \in V \setminus V'\}$ is an optimal fractional $W$-separator of $G-V'$, i.e., ~$\sum_{v \in V \setminus V'} y_v = \lpPrim(G-V')$.
\end{lemma}

\cref{corollary::paretoOptLP,lemma::onesInLPSameObjective} allow incremental progress in the set of 1-bits with respect to search points $X \in \P$ that satisfy $|X_1| + \lp(G[u(X)]) = \lpPrim(G)$ without backstepping.
With this ingredient we can prove \cref{lemma::firstStepOptLP} (see~Section~\ref{appendix::f2} for a proof).
Since $f_3$ has one less objective than $f_2$, one can derive the following lemma.
\begin{lemma}
\label{corollary::firstStepOptLP}
Using the fitness function $f_3$, the expected number of iterations of \algGlobalSemoAlt where the population $\P$ contains no search point $X \in \{0,1\}^n$ such that $\lpPrim(G)=\lpPrim(G[u(X)]) + |X_1|$ and there is an optimal fractional $W$-separator $\{y_v \in \R_{\geq 0}\}_{v \in u(X)}$ of $G[u(X)]$ with $y_v<1$ for every $v \in u(X)$ is upper bounded by $\O(n^2(\log n + \opt))$.
\end{lemma}

Our next goal is to prove \cref{lemma::ingredientParateo1}.
To identify the head vertices $\aall$ with respect to an optimal fractional $W$-separator, we want to ensure that the selection of the vertices of $\ball$ are distinguishable so that it cannot come to a conflict with \cref{lemma::firstStepOptLP}.
To do this, we will make use of the LP-objective and show that for a search point $X$ with $X_1 \cap \ball \neq \varnothing$ we have $\lpPrim(G) < \lpPrim(G[u(X)]) + |X_1|$.
Let $(A,B)$ be a minimal strictly reducible pair in $G$.
The essential idea is to use $(W+1)$-packings in $G[A \cup B]$, since they provide lower bounds for $W$-separators.
From \cref{lemma::packingReduciblePair} one can deduce that $G[A \cup B]$ contains a maximum ($W+1$)-packing $\Q$ of size $|A|$, since every vertex of $A$ is contained exactly in one element of $\Q$.
Inspired by ideas on how to find crown decompositions in weighted bipartite graphs from~\cite{DBLP:conf/esa/Casel0INZ21,DBLP:conf/iwpec/KumarL16}, we prove that removing vertices from $B$ only partially affects the size of the $(W+1)$-packing in $G[A \cup B]$, as stated in the following lemma.

\begin{lemma}
\label{lemma::PackingSizeLargeEnough}
Let $(A,B)$ be a minimal strictly reducible pair in $G=(V,E)$ and let $S \subset A \cup B$ with $|S| \leq |A|$.
If $S \cap B \neq \varnothing$, then $G[(A \cup B)] - S$ contains a packing of size $|A| - |S| + 1$.
\end{lemma}

In contrast, note that removing vertices $S \subseteq A$ from $G[A \cup B]$ would decrease the size of a $(W+1)$-packing by $|S|$,
i.e., a maximum $(W+1)$-packing in $G[A \cup B]-S$ has size $|A|-|S|$.
We moved the proof of \cref{lemma::PackingSizeLargeEnough} to the appendix (Section~\ref{appendix::f2}), since it is more technical and too long given the space constraints.
Essentially, we make use of the following two lemmas and properties of network flows.
In particular, these lemmas describe the new properties we have found for minimal strictly reducible pairs and may be of independent interest.

\begin{lemma}
	\label{lemma::decideWhichA}
	Let $(A,B)$ be a minimal strictly reducible pair in $G$ with parameter $W$.
	Then, for every $a^* \in A$ there is an assignment function $g \colon \B \times A \to \N_0$ like in \cref{def::RedPair} that satisfies $\sum_{C \in \B} g(C,a^*) \geq 2W$ and $\sum_{C \in \B} g(C,a) \geq 2W -1$ for every $a \in A \setminus \{a^*\}$.
\end{lemma}

Concerning \cref{lemma::decideWhichA}, we remark that the new feature to before is that the particular vertex (in the lemma $a^*$) can be chosen arbitrarily.

\begin{lemma}
	\label{lemma::SizeNeoghborhood}
	Let $(A,B)$ be a minimal strictly reducible pair in $G$ with parameter $W$.
	Then, for every $A' \subseteq A$ we have $|V(\B_{A'})| \geq |A'|(2W-1)+1$.
\end{lemma}

To conclude the Phase~\ref{part1} we need to prove \cref{lemma::ingredientParateo1}.
Equipped with \cref{lemma::PackingSizeLargeEnough} we may prove statements about the LP-objective if $X_1 \cap \ball \neq \varnothing$.
In doing so, we prove another relation with respect to such a sequence, which fits the proof and will be useful later.

\begin{lemma}
\label{lemma::RedPairLPNoSequence}
Let $(A_1,B_1), \dots, (A_m,B_m)$ be a sequence of minimal strictly reducible pairs and let $X_1 \in \{0,1\}^n$ be a sample.
\begin{enumerate}
	\item If $X_1 = \bigcup_{i=1}^m A_i$, then $\lpPrim(G) = \lpPrim(G[u(X)]) + |X_1|$.
	\label{enum::1}
	\item If $X_1 \cap B_\ell \neq \varnothing$ for an $\ell \in [m]$, then $\lpPrim(G(u[X])) + |X_1| > \lpPrim(G)$.
	\label{enum::2}
\end{enumerate}
\end{lemma}

Suitable for \cref{lemma::firstStepOptLP} we have characterized the case $X_1 \cap \ball \neq \varnothing$.
It remains to give a relation to this lemma when $\aall \cap X_1 \neq \varnothing$ and $\aall \subsetneq X_1$.
In particular, we want to ensure that in this case at least one vertex of $\aall \setminus X_1$ must be one in an optimal fractional $W$-separator of $G[u(X)]$.

\begin{lemma}
\label{lemma::CrownStaysCrown}
Let $(A,B)$ be a minimal strictly reducible pair in $G$ and let $\hat{A} \subset A$.
Then, there is a partition $A_1, \dots, A_m$ of $A \setminus \hat{A}$ with disjoint vertex sets $B_1, \dots, B_m \subseteq B$, such that for each $i \in [k]$ the tuple $(A_i,B_i)$ is a minimal strictly reducible pair in $G - \hat{A}$.  
\end{lemma}

By \cref{lemma::onesInLPReduciblePair} we already know that the head vertices of a minimal strictly reducible pair in an optimal fractional $W$-separator have value one.
\cref{lemma::CrownStaysCrown} ensures that if some of the head vertices are removed, the value of the remaining head vertices in the respective optimal fractional solution remain one.
The proof of \cref{lemma::ingredientParateo1} can be found in the appendix (Section~\ref{appendix::f2}) and concludes Phase~\ref{part1}.

Next, we prove that Phase~\ref{part2} works successfully.
After Phase~\ref{part1}, we have a search point $X$ in the population $\P$ with $\aall \subseteq X_1$ such that $\lpPrim(G) = \lpPrim(G[u(X)]) + |X_1|$.
Consequently, $|X_1| \leq \opt$ and therefore we can prove that \algGlobalSemoAlt reaches a search point $X'$ with $X'_1 = \aall$ from $X$ in FPT-time.

\begin{lemma}
\label{lemma::XReducedInPop}
Let $G=(V,E)$ be an instance of the $W$-separator problem, and let $(A_1,B_1)$, $\dots$, $(A_m,B_m)$ be a sequence of minimal strictly reducible pairs in $G$, such that $G - \bigcup_{i=1} A_i$ contains no strictly reducible pair.
Using the fitness function $f_2$, the expected number of iterations of \algGlobalSemoAlt until the population $\P$ contains a search point $X$ with $X_1 = \bigcup_{i=1}^m A_i$ is upper bounded by $\O\left(n^3(\log n + \opt) + 2^{\opt}\right)$.
\end{lemma}

The question that remains is whether we keep $X'$ in the population once we find it.
This is where the uncovered-objective and the structural properties of minimal strictly reducible pairs come into play.

\def\xdiff{X_{\texttt{dif}}}
\def\vdiff{V_{\texttt{dif}}}

\begin{lemma}
\label{lemma::paretoOptiSol}
Let $X \in \{0,1\}^n$ and let $(A_1,B_1), \dots, (A_m,B_m)$ be a sequence of minimal strictly reducible pairs in $G$, such that $G - \bigcup_{i=1} A_i$ contains no strictly reducible pair.
If $X_1 = \bigcup_{i=1}^m A_i$, then $X$ is a pareto optimal solution.
\end{lemma}
\begin{proof}
Let $A = \bigcup_{i=1}^m A_i$ and $B = \bigcup_{i=1}^m B_i$.
We will prove that if there is a search point $X'$ that dominates $X$, then $G - A$ contains a minimal strictly reducible contradicting the precondition of the lemma.
Note that a minimal strictly reducible pair in $G - A$ have to be in $G[u(X)]$ as the other components in $G-A$ have size at most $W$.

If $X_1 = \bigcup_{i=1}^m A_i$, then by \cref{lemma::RedPairLPNoSequence} $\lpPrim(G) = \lpPrim(G[u(X)]) + |X_1|$.
That is, we might restrict to solutions $X' \in \{0,1\}^n$ with $\lpPrim(G) = \lpPrim(G[u(X')]) + |X'_1|$ as well as $|X'_1| = |X_1|$ and can focus on the objective $u(X)$ and $u(X')$, respectively.
Note that it cannot happen that $|X'_1| < |X_1|$ and $\lpPrim(G[u(X')]) \leq \lpPrim(G[u(X)])$ as $\lpPrim(G) = \lpPrim(G[u(X)]) + |X_1| > \lpPrim(G[u(X')]) + |X'_1|$ contradicting \cref{lemma::lowerBoundLP}.

W.l.o.g.~we can assume that every connected component of $G-X_1=G-A$ of size at most $W$ is also in $G[B]$, i.e., $V \setminus u(X) = A \cup B$.
Let $g_1 \colon \B_1 \times A_1 \to \N_0,\dots,g_m \colon \B_m \times A_m \to \N_0$ be the according assignment functions of $(A_1,B_1), \dots, (A_m,B_m)$ and let $g \colon \B \times A \to \N_0$ be defined as $g(a,C) = g_i(a,C)$ if $a \in A_i$, $C \in \B_i$ and otherwise $g(a,C) = 0$.
Suppose there is such an $X' \neq X$ as described above with $|u(X')| \leq |u(X)|$, or equivalently $|V \setminus u(X')| \geq |V \setminus u(X)|$.
Note that $|V \setminus u(X)| - |A| \geq |A|(2W-1)+1$, 
since for each $a \in A$ and at least for one $a' \in A$ we have $\sum_{C \in \B} g(C,a) \geq 2W-1$ and $\sum_{C \in \B} g(C,a') \geq 2W$.

We define $V(\B_{\Tilde{A}}) \subseteq B$ for $\Tilde{A} \subseteq A$ as the vertices in the components $\{C \in \B \mid N(C) \cap \Tilde{A} \neq \varnothing\}$. 
Let $\xdiff = X'_1 \setminus  X_1 = X'_1 \setminus A$ and let $\vdiff = (V \setminus u(X')) \setminus (V \setminus u(X))$.
Note that $\vdiff \subseteq u(X)$ and $|\xdiff| = |A \setminus X'_1|$ by $|A| = |X'_1|$.
From $\lpPrim(G) = \lpPrim(G[u(X')]) + |X'_1|$, we obtain by \cref{lemma::ingredientParateo1} that $\xdiff \cap B = \varnothing$ and therefore $\xdiff \subseteq \vdiff$ as no vertex of $\xdiff$ is in $A \cup B = V \setminus u(X)$.
Thus, by the assignment function $g$ each vertex in $A' = A \setminus X'_1$ is in a connected component of size at least $W+1$ of $G[u(X')]$ and therefore $A'$ as well as $V(\B_{A'})$ are not in $V \setminus u(X')$.
Furthermore, for at least one $j \in [m]$ we have $A_j \subsetneq X'_1$ and by \cref{lemma::SizeNeoghborhood} we obtain that $|V(\B_{A_j \setminus X'_1})| \geq |A_j \setminus X'_1| (2W-1)+1$.
That is, to satisfy now $|V \setminus u(X')| \geq |V \setminus u(X)|$ we must have $|\vdiff| - |\xdiff| \geq |\xdiff|(2W-1)+1$ as at least $|A \setminus X'_1|(2W-1)+1+|A \setminus X'_1|$ vertices are in $V \setminus u(X)$ that are not in $V \setminus u(X')$.
Observe that each connected component $C$ of $G[\vdiff - \xdiff]$ satisfy $N(C) \cap \xdiff \neq \varnothing$ and $|C| \leq W$.
From this, we obtain that $G[\vdiff]$ contains a strictly reducible pair $(\hat{A},\hat{B})$ with $\hat{A} \subseteq \xdiff$ and $\hat{B} \subseteq \vdiff \setminus \xdiff$ by \cref{lemma::balancedExpansionStrictlyReducibleExist} and therefore also a minimal strictly reducible pair $(\hat{A}',\hat{B}')$ with $\hat{A}' \subseteq \hat{A}$ and $\hat{B}' \subseteq \hat{B}$.
In particular, $(\hat{A}',\hat{B}')$ is a minimal strictly reducible in $G-A$, since $\vdiff \subseteq u(X)$ and the connected components in $G[\vdiff \setminus \xdiff]$ exist identically in $G[u(X)]-\xdiff$.
\end{proof}
% The final proof of \cref{thm::OptSolf2} can be found in the appendix (Section \ref{appendix::f2}) and shows that 
We are ready for the final theorem of this section, which shows that Phase~\ref{part3} also works successfully.
\paragraph{Proof of \cref{thm::OptSolf2}:}
Let $(A_1,B_1),\dots,(A_m,B_m)$ be a sequence of minimal strictly reducible pairs, such that $G-\bigcup_{i=1}^m A_i$ contains no strictly reducible pair.
Furthermore, let $\P$ be a population with respect to $f_2$ in the algorithm \algGlobalSemoAlt. 
By \cref{lemma::XReducedInPop} we have a search point $X \in \P$ with $X_1 = \bigcup_{i=1}^m A_i$ after $\O\left(n^3(\log n + \opt) + 2^{\opt}\right)$ iterations  in expectation. %of \algGlobalSemoAlt using the fitness function $f_2$.
Moreover, $X$ is a pareto optimal solution by \cref{lemma::paretoOptiSol}.

Since $G[u(X)]$ contains no strictly reducible pair, we can derive from \cref{lemma::existenceReducible} that $|V(G[u(X)])| \leq 2 \cdot \opt \cdot W$. 
The algorithm \algGlobalSemoAlt calls with $1/3$ probability the mutation that flips every vertex $u(X)$ with $1/2$ probability in $X$.
That is, reaching a state $X'$ from $X$, such that $X'_1 = V^*$ has a probability of at least $\Omega\left(2^{-2 \cdot \opt \cdot W}\right)$, where  selecting $X'$ in $\P$ has probability $\Omega(1/n^2)$ (cf.~\cref{lemma::singleFlipAndBoundedPop}).
Thus, once $X$ is contained in $\P$ it takes in expectation $\O\left(n^2 \cdot 4^{\opt \cdot W} \right)$ iterations reaching $X'$.
As a result, the algorithm needs in total $\O\left(n^3(\log n + \opt) + n^2 \cdot 4^{\opt \cdot W} \right)$ iterations finding an optimal $W$-separator in expectation.