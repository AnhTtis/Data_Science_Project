%!TEX root = main.tex



\subsection*{Proof of \cref{lemma::singleFlipAndBoundedPop}}
\begin{proof}
	We start to bound the size of the population $\P$.
	Given $f_1$ or $f_2$ for the bit sequences $\P$, the first and second entries each have at most $(n+1)$ distinct values, and we keep at most one for each possibility.
	Therefore, we have $|\P| \leq (n+1)^2$ and thus selecting a certain search point $X \in \P$ has probability $\Omega(1/n^2)$.
	In \algGlobalSemo we mutate every bit in $X$ with probability $1/n$.
	That is, we obtain a probability of $1/n \cdot (1 - 1/n)^{n-1} \geq 1/ne \in \Omega(1/n)$ to flip a certain bit in $X$.
	This probability decreases only by a constant factor of $1/3$ in \algGlobalSemoAlt and keeps therefore the probability by $\Omega(1/n)$ for this event.
	As a result, selecting a certain search point $X \in \P$ and flipping only one single bit in it has probability $\Omega(1/n^3)$ for both algorithms.
\end{proof}

\subsection*{Proof of \cref{lemma::zeroSol}}

\begin{proof}
	Let $\P \neq \varnothing$ and let $X^{\min} \in \P$ be the bit string in $\P$ with the minimal number of ones, where $i := |X^{\min}_1|$.
	By \cref{lemma::singleFlipAndBoundedPop} the probability that $X^{\min}$ is chosen in one iteration of both algorithms is $\Omega(1/n^2)$.
	If we consider \algGlobalSemo, then the probability that the mutation of $X^{\min}_1$ results in a bit string $X$ with $|X_1| < i$ is at least 
	$i/n \cdot (1 - 1/n)^{n-1} \geq i/ne$, which is the probability that only one 1-bit is flipped and nothing else.
	Note that the probability changes by a factor of $1/3$ in the algorithm \algGlobalSemoAlt.
	Thus, the probability for both algorithms that the population gets after an iteration a bit string that has less number of ones compared to the previous population is $\Omega(i/n^3)$.
	This in turn means that the expected number of iterations this happens is $\O(n^3/i)$.  
	Using the method of fitness based partitions \cite{DBLP:conf/ppsn/Sudholt10} and summing up over the different values of $i$ we obtain an expected time of $\sum_{i=1}^{n} \O(n^3/i) = \O(n^3 \log n)$ that $0^n$ is in $\P$.
\end{proof}