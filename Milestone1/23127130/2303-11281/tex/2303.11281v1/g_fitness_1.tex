%!TEX root = main.tex

In this section we investigate the fitness $f_1$ on \algGlobalSemoAlt.
We will prove that the algorithm finds an optimal $W$-separator in expectation in FPT-runtime with the parameters $\opt$ and $W$.
Recall that the parameter $k$ in the decision variant of the $W$-separator asks for a $W$-separator of size at most $k$.
A more general variant, known as \textit{weighted component order connectivity problem}, was studied in~\cite{DBLP:journals/algorithmica/DrangeDH16} by Drange et al.
They achieve a $\O(k^2 W + W^2 k)$ vertex-kernel, which also holds for the $W$-separator problem.

\begin{theorem}[\cite{DBLP:journals/algorithmica/DrangeDH16}, Theorem~15]
	\label{thm::degreeKernel}
	The $W$-separator admits a kernel with at most $kW(k+W)+k$ vertices, where $k$ is the solution size.
\end{theorem}

Essentially, they use the following \emph{reduction rule}: as long as there is a vertex with degree greater than $k + W$, the vertex is included in the solution set and may be removed from the instance.

It is not difficult to see that this vertex must be included in the solution, since otherwise we would have to take more than $k$ vertices from its neighborhood to get a feasible solution.
After using this reduction rule exhaustively each vertex in the reduced instance has degree at most $k+W$.
Consequently, in the reduced instance, each vertex of a $W$-separator is connected to at most $k+W$ connected components after its removal, where each of those components has size at most $W$.  
A simple calculation provides finally the vertex-kernel stated in \cref{thm::degreeKernel}.

Now, we make use of the degree-objective from $f_1$ to find a search point that selects those vertices which can be safely added to an optimal solution according to the reduction rule.

\begin{lemma}
	\label{lemma::reducedInstance_f1}
	Using the fitness function $f_1$, the expected number of iterations of %\algGlobalSemo or
	\algGlobalSemoAlt where the population $\P$ contains a solution $X$ in which for all $u \in u(X)$ and for all $v \in X_1$ we have $d(u) \leq \opt+W$ and $d(v) > \opt+W$ is bounded by $\O(n^3(\opt + \log n))$. 
\end{lemma}
%\begin{proof}
%	Let $V' = \{v_1, \dots, v_\ell\} \subseteq V$ be the vertices with degree larger than $k+W$ (\emph{reducible vertices}), such that $d(v_i) \geq d(v_j)$ for $i>j$.
%	Observe that if $V' = \varnothing$, then $0^n$ is already the desired search point and we are done by \cref{lemma::zeroSol}.
%	For $i \in \{0,1,\dots,\ell\}$ let $X^i$ be a search point with $|X^i_1| = i$ and $\sum_{v \in X^i_1} d(v) = \sum_{j=1}^i d(v_j)$.
%	In particular, $X^\ell$ is the desired search point according to the lemma.
%	First observe that $X^i$ can only be dominated by a search point $X$ if $|X_1| \leq |X^i_1|$ and $-\sum_{v \in X_1} d(v) \leq - \sum_{v \in X^i_1} d(v)$,
%	which is only possible if $|X_1| = i = |X^i_1|$ and $-\sum_{v \in X} d(v) = - \sum_{v \in X'_1} d(v)$ as $X^i_1$ contains only the vertices of largest degree.
%	Consequently, $X^i \subseteq V'$ and once $X^i$ is in the population $\P$ the vector $(|X^i_1|, *, -\sum_{v \in X_1} d(v))$ is pareto optimal.
%	
%	Let $\P$ be a population with $0^n \in \P$ and let $s < \ell$ be the largest integer such that $X^s \in \P$.
%	Note that $X^0 = 0^n$.
%	Let $v \in u(X^s)$ be a vertex that satisfies $d(v) = \max_{u \in u(X^s)} d(u)$.
%	By \cref{lemma::singleFlipAndBoundedPop} \algGlobalSemoAlt flips only a certain bit from a certain search point of $\P$ with probability $\Omega(1/n^3)$ and
%	thus, mutating $X^s$ to a search point $X^{s+1}$ takes in expectation $\O(n^3)$ iterations.
%	Using the method of fitness based partitions \cite{DBLP:conf/ppsn/Sudholt10} and summing up over the different values of $s$ leads to $\sum_{i=1}^{\ell} \O(n^3) \leq \sum_{i=1}^{\opt} \O(n^3) = \O(\opt \cdot n^3)$ expected number of iterations having $X^\ell$ in the population $\P$, once $0^n \in \P$.
%	By \cref{lemma::zeroSol} the expected number of iterations such that $0^n$ is in the population $\P$ is $\O(n^3 \log n)$.
%	As a result, we have $X^\ell \in \P$ after $\O(n^3(\opt + \log n))$ iterations in expectation. 
%\end{proof}

With \cref{lemma::reducedInstance_f1} in hand we can upper bound the expected number of iterations that \algGlobalSemoAlt takes to find an optimal $W$-separator with respect to the fitness $f_1$.
Note that the uncovered-objective of $f_1$ ensures that the algorithms \algGlobalSemoAlt converge to a feasible solution and that a search point $X$ with $f_1(X) = (\opt, 0, *)$ corresponds to an optimal $W$-separator.   

\begin{theorem}
	\label{thm::fitness1Opt}
	Using the fitness function $f_1$, the expected number of iterations of \algGlobalSemoAlt until it finds a minimum $W$-separator in $G=(V,E)$ is upper bounded by $\O\left(n^3(\opt + \log n) + n^2 \cdot 2^q \right)$, where $q=\opt \cdot W(\opt+W)+\opt$.
\end{theorem}
%\paragraph{Proof of \cref{thm::fitness1Opt}:}
%%\begin{proof}
%First observe that once we have a search point $X$ according to \cref{lemma::reducedInstance_f1} in the population that this is pareto optimal, since there is no other search point with same or less number of selected vertices that yields to a smaller value of $-\sum_{v \in X_1} d(v)$.
%On the other hand, by \cref{thm::degreeKernel} we know that the uncovered-objective $u(X)$ has size at most $q=\opt \cdot W(\opt+W)+\opt$
%while $X_1$ contains only vertices that have to be in the optimal solution.
%In expectation, $X$ is in the population after $\O(n^3(\opt + \log n))$ iterations of \algGlobalSemoAlt (cf.~\cref{lemma::reducedInstance_f1}).
%Thus, given $X$ is selected in the Algorithm~\algGlobalSemoAlt we obtain a probability of $1/3 \cdot 2^q$ flipping $m \leq \opt$ vertices of $u(X)$ that lead to an optimal solution. % and therefore, given $X$ is selected this event happens in expectation after $2^{kW(k+W)+k}$ iterations of \algGlobalSemoAlt.
%That is, if $X \in \P$ we have a probability of $\Omega\left(1/n^2 \cdot 2^q\right)$ reaching the optimal solution in one iteration, where the additional factor $1/n^2$ comes from selecting $X$ (cf.~\cref{lemma::singleFlipAndBoundedPop}).
%Consequently, the expected number of iterations reaching an optimal solution is upper bounded by $\O\left(n^3(\opt + \log n) + n^2 \cdot 2^q \right)$.
%%\end{proof}