%!TEX root = main.tex

In this section we investigate $f_2$ on \algGlobalSemoAlt.
The main result of this section is the following theorem.
% Due to the space constraints we moved most of the proofs of this section to the appendix (Section~\ref{appendix::f2}).
%In particular, we prove the following theorem. \zienain{Sam: the preceding sentence doesn't say much. Maybe: "The main result of this section is the following theorem".}

\begin{theorem}
	\label{thm::OptSolf2}
	Let $G=(V,E)$ be an instance of the $W$-separator problem.
	Using the fitness function $f_2$, the expected number of iterations of \algGlobalSemoAlt until an optimal solution is sampled is upper bounded by $\O(n^3(\log n + \opt) + n^2 \cdot 4^{\opt \cdot W} )$.
\end{theorem}
 
First we give a brief overview of a reducible structure concerning the $W$-separator problem associated with the objectives in the fitness function $f_2$.   
The structure we will use is commonly known as crown decomposition.
Roughly speaking, it is a division of the set of vertices into three parts consisting of a crown, a head, and a body, with the head separating the crown from the body.
Under certain conditions concerning the crown and head vertices, which we will clarify in a moment, it is possible to show that there exists an optimal $W$-separator which contains the head vertices and reduces the given instance by removing the crown vertices.
Recall that the parameter $k$ in the decision variant of the $W$-separator asks for a $W$-separator of size at most $k$.
Kumar and Lokshtanov~\cite{DBLP:conf/iwpec/KumarL16} provide such a reducible structure and state that it is in a graph as long as the size of it is greater than $2kW$.
The structure is called a (strictly) reducible pair and consists of crown and head vertices.

For an instance $G=(V,E)$ of the $W$-separator problem we say that $Y=\{y_v \in \R_{\geq 0}\}_{v \in V}$ is a \emph{fractional $W$-separator} of $G$ if $Y$ is a feasible solution according to the LP formulation of the $W$-separator problem. % (see \cref{fig::LPWSep}).
It is not difficult to see that the objective of any optimal fractional $W$-separator is smaller than $\opt$, i.e.~$\lpPrim(G) \leq \opt$.
In principle, the LP objective is useful for finding a strictly reducible pair, since the head vertices in an optimal fractional W separator must have value one.
Unfortunately, it is unknown whether each vertex that has value one in an optimal fractional $W$-separator is part of an optimal solution.
This leads to the challenge of filtering out the right vertices, where the uncovered-objective - and in particular the structural properties of a strictly reducible pair - come into play.


\paragraph{\textbf{Reducible Structure of the $\boldsymbol W$-Separator Problem}}
In the following, we briefly summarize the definitions and theorems of \cite{DBLP:conf/esa/Casel0INZ21,fomin2019kernelization,DBLP:conf/iwpec/KumarL16}. 
For a vertex set $B \subseteq V$, denote by $\B$ the partitioning of $B$ according to the connected components of $G[B]$.

\begin{definition}[(strictly) reducible pair]
\label{def::RedPair}
For a graph $G=(V,E)$, a pair $(A,B)$ of vertex disjoint subsets of $V$ is a \emph{reducible pair} if the following conditions are satisfied:
\begin{itemize}
	\item $N(B) \subseteq A$.
	\item
	The size of each $C \in \B$ is at most $W$.
	\item There is an assignment function $g \colon \B \times A \to \N_0$, such that
	\begin{itemize}
		\item for all $C \in \B$ and $a \in A$, if $g(C,a) \ne 0$, then $a \in N(C)$
		\item for all $a \in A$ we have $\sum_{C \in \B} g(C,a) \geq 2W-1$,
		\item for all $C \in \B$ we have  $\sum_{a \in A} g(C,a) \leq |C|$,
	\end{itemize} 
\end{itemize} 
In addition, if there exists an $a \in A$ such that $\sum_{Q \in \B} g(C,a) \geq 2 W$, then $(A,B)$ is a \emph{strictly reducible pair}.
\end{definition}

Next, we explain roughly the idea behind a reducible pair $(A,B)$.
The head and crown vertices correspond to $A$ and $B$, respectively.
That is, we want $A$ to be part of our $W$-separator, and if that is the case, then no additional vertex from $B$ is required to be in the solution since the components $C \in \B$ are isolated after removing $A$ from $G$ with $|C| \leq W$. 
Let $G=(V,E)$ be a graph. 
We say that $P_1, \dots, P_m \subseteq V$ is a \emph{($W+1$)-packing} if for all $i,j \in [m]$ with $i \neq j$ the induced subgraph $G[P_i]$ is connected, $|P_i| \geq W+1$, and $P_i \cap P_j = \varnothing$.
Note that for a $W$-separator $S \subseteq V$, it holds that $S \cap P_i \neq \varnothing$ for all $i \in [m]$.
Thus, the size of a ($W+1$)-packing is a lower bound on the number of vertices needed for a $W$-separator.

\begin{lemma}[\cite{DBLP:conf/iwpec/KumarL16}, Lemma 17]
\label{lemma::packingReduciblePair}
Let $(A,B)$ be a reducible pair in $G$.
There is a ($W+1$)-packing $P_1, \dots, P_{|A|}$ in $G[A \cup B]$, such that $|P_{i} \cap A| = 1$ for all $i \in [|A|]$.
\end{lemma}

Essentially, \cref{lemma::packingReduciblePair} provides a lower bound of $|A|$ vertices for a $W$-separator in $G[A \cup B]$.
On the other hand, $A$ is a $W$-separator of $G[A \cup B]$ while $A$ separates $B$ from the rest of the graph.
This properties basically admits the following theorem.

\begin{theorem}[\cite{DBLP:conf/iwpec/KumarL16}, Lemma 18]
\label{thm::saveReduction}
Let $(G,k)$ be an instance of the $W$-separator problem, and $(A,B)$ be a reducible pair in $G$.
$(G,k)$ is a yes-instance if and only if $(G - (A \cup B), k - |A|)$ is a yes-instance.
\end{theorem}

Finally, we clarify why a strictly reducible pair exists if the size of $G$ is larger than $2kW$.
To do so, we make use of a lemma derivable from  \cite{DBLP:conf/esa/Casel0INZ21,DBLP:conf/iwpec/KumarL16}.
A proof is given in the appendix~(\cref{appendix::f2}). 

\begin{lemma}
\label{lemma::balancedExpansionStrictlyReducibleExist}
Let $G=\left(A \cup B, E\right)$ be a graph and $W \in \N_0$.
Let $\B$ be the connected components given as vertex sets of $G[B]$, where for each $C \in \B$ we have $|C| \leq W$ and no $C \in \B$ is isolated, i.e.~$N(C) \neq \varnothing$.
If $|B| \geq (2W-1)|A|+1$, then there exists a non-empty strictly reducible pair $(A',B')$, where $A' \subseteq A$ and $B' \subseteq B$.
\end{lemma}

%To prove \cref{lemma::balancedExpansionStrictlyReducibleExist} we use a structural lemma provided by Casel et al.~\cite{DBLP:conf/esa/Casel0INZ21} for potential vertices of a strictly reducible pair with additional properties for the unincluded vertices.
%For a better understanding, it is convenient to think by an assignment function $g \colon \B \times A \to \N_0$ on flows, which sends fractional vertex-parts of components $\B$ to $A$.
%For $g$ we define $\overrightarrow{\B}_a :=\left\{C \in \B \mid g\left(C,a\right) > 0 \right\}$ for $a \in A$, $\overrightarrow{\B}_{A'} := \bigcup_{a \in A'} \overrightarrow{\B}_a$ for $A' \subseteq A$ and $\overrightarrow{\B}_U :=\left\{C \in \B \mid \sum_{a \in A} g\left(C,a\right) < |C| \right\}$.
%
%\begin{definition}[fractional balanced expansion]
%\label{definition::fbe}
%Let $G=\left(A \cup B, E\right)$ be a graph, where for each component $C \in \B$ we have $|C| \leq W$. 
%%Let $\B$ be the connected components given as vertex sets of $G[B]$, where $|C| \leq W$ for each $C \in \B$.
%For $q \in \N_0$, a partition $A_1\cup A_2$ of $A$ and $g \colon \B \times A \to \N_0$, the tuple $\left(A_1,A_2,g,q\right)$ is called \emph{fractional balanced expansion} if:
%\begin{enumerate}
%	\item for all $C \in \B$ and $a \in A$, if $g(C,a) \ne 0$, then $a \in N(C)$,
%	\item $\sum_{C \in \B} g\left(C,a\right)\, \begin{cases}\geq \ q, & a\in A_1\\ 
%		\leq \ q, & a\in A_2\end{cases}$  
%	\item $\forall C \in \B \colon \sum_{a \in A} g\left(C,a\right) \leq |C|$ %(capacity)
%	\item  $N\left(\overrightarrow{\B_U} \cup \overrightarrow{\B}_{A_1}\right) \subseteq A_1$ %, (separator)\\
%	%where $\overrightarrow{\B}_a :=\left\{C \in \B \mid g\left(C,a\right) > 0 \right\}$ for $a \in A$, $\overrightarrow{\B}_{A'} := \bigcup_{a \in A'} \overrightarrow{\B}_a$ for $A' \subseteq A$ and $\overrightarrow{\B}_U :=\left\{C \in \B \mid \sum_{a \in A} g\left(C,a\right) < |C| \right\}$  
%\end{enumerate}
%\end{definition}
%
%The vertices $A_1$ in a fractional balanced extension correspond to the head vertices, and the question we are interested in is when they are nonempty. 
%
%\begin{lemma}[\cite{DBLP:conf/esa/Casel0INZ21}, Lemma~4]
%\label{lemma::balancedExpansionReducibleExist}
%Let $G=\left(A \cup B, E\right)$ be a graph, where for each $C \in \B$ we have $|C| \leq W$ and no $C \in \B$ is isolated, i.e.~$N(C) \neq \varnothing$.
%%Let $\B$ be the connected components given as vertex sets of $G[B]$, 
%For every $q \in \N_0$, if $w(B) \geq q|A|$, then there exists a fractional balanced expansion $\left(A_1,A_2,g,q\right)$ with $A_1 \neq \varnothing$. 
%\end{lemma}
%
%Equipped with \cref{lemma::balancedExpansionReducibleExist} we can prove \cref{lemma::balancedExpansionStrictlyReducibleExist}, i.e.~the existence of a strictly reducible pair with respect to the vertex-size. 
%
%\paragraph{Proof of \cref{lemma::balancedExpansionStrictlyReducibleExist}:}
%By the precondition of the lemma we obtain by \cref{lemma::balancedExpansionReducibleExist} the existence of a fractional balanced expansion $\left(A_1,A_2,g,2W-1\right)$ with $A_1 \neq \varnothing$ in $G$.
%By the properties of the fractional balanced expansion we have $N\left(\overrightarrow{\B}_U \cup \overrightarrow{\B}_{A_1}\right) \subseteq A_1$
%%(see \cref{definition::fbe} for the definitions of $\overrightarrow{\B_U}$ and $\overrightarrow{\B}_{A_1}$)
%and $\sum_{b \in B} g\left(C,a\right) \geq 2W-1$ for all $a \in A_1$.
%That is, $(A'=A_1, B' = \bigcup_{C \in \B_U \cup \overrightarrow{\B}_{A_1} } C)$ is a reducible pair.
%Since $\sum_{C \in \B} g\left(C,a\right) \leq 2W-1$ for all $a \in A_2$ and no $C \in \B$ is isolated, we have $|B| - \sum_{a \in A_2} \sum_{C \in \B} g(C,a) \geq |A|(2W-1) + 1 - |A_2| (2W-1) = |A_1|(2W-1)+1$.
%Thus, either there is already an $a \in A_1 = A'$ with $\sum_{C \in  \B} g\left(C,a\right) \geq 2W$, or $\overrightarrow{\B_U} \neq \varnothing$ and we can assign one more unit of $\overrightarrow{\B}_U$ to a vertex of $A_1 = A'$ which has a component $C \in \overrightarrow{\B_U}$ in its neighborhood.
%This in turn shows that $(A',B')$ is a strictly reducible pair.
%%That is, $A' = A_1$ and $B' = V(\overrightarrow{\B}_U \cup \overrightarrow{\B}_{A_1})$ is a strictly reducible pair.
%\vspace{0.2cm}

We conclude the preliminary section with a lemma that connects strictly reducible pairs with the size of the graph.
% by repeating the short proof from \cite{DBLP:conf/iwpec/KumarL16} to understand the connection of a strictly reducible pair with the size of the input graph.

\begin{lemma}[\cite{fomin2019kernelization}, Lemma~6.14]
\label{lemma::existenceReducible}
Let $(G,k)$ be an instance of the $W$-separator problem, such that each component in $G$ has size at least $W+1$.
If $|V| > 2Wk$ and $(G,k)$ is a yes-instance, then there exists a strictly reducible pair $(A,B)$ in $G$. 
\end{lemma}
%\begin{proof}
%Since $(G,k)$ is a yes-instance, there is a solution $S \subset V$ of the $W$-separator problem in $G$ such that $|S| \leq k$.
%After removing $S$ from $G$ each connected component of $G_S$ has size at most $W$ and is not isolated as each component in $G$ has size at least $W+1$.
%Moreover, $|V \setminus S| > 2kW - k = k(2W-1) \geq k(2W-1) + 1$.
%Thus, $G(A \cup B,E)$ with $A=S$ and $B=V \setminus S$ satisfies the precondition of \cref{lemma::balancedExpansionStrictlyReducibleExist}, which in turn ensures the existence of strictly reducible pair.
%\end{proof}

%Note that \cref{lemma::existenceReducible} implies that if the input graph $G$ contains no strictly reducible pair, then $|V(G)| \leq 2kW$.

\paragraph*{\textbf{Running time analysis}} 
%The proof of \cref{thm::OptSolf2} can essentially be divided into three phases, which we will explain after introducing few notations.
Let $(A,B)$ be a strictly reducible pair. 
We say $(A,B)$ is a minimal strictly reducible pair if there does not exist a strictly reducible pair $(A',B')$ with $A' \subset A$ and $B' \subseteq B$.
Clearly, it can happen that reducible pairs arises after a reduction is executed.
Therefore, we say $(A_1,B_1), \dots, (A_m,B_m)$ is a \emph{sequence of minimal strictly reducible pairs} if for all $i \in [m]$ the tuple $(A_i,B_i)$ is a minimal strictly reducible pair in $G - \bigcup_{j=1}^{i-1} A_j$.
Note that the definition of such a sequence implies that those tuples are pairwise disjoint, i.e., $(A_i \cup B_i) \cap (A_j \cup B_j) = \varnothing$ for all $i,j \in [m]$ with $i \neq j$.
The proof of \cref{thm::OptSolf2} can essentially be divided into three phases:
\begin{enumerate}
\item%[Part (i)]
\label{part1}
Let $(A_1,B_1), \dots, (A_m,B_m)$ be a sequence of minimal strictly reducible pairs in $G$, such that $G - \bigcup_{i\in [m]} A_i$ contains no minimal strictly reducible pair. 
The first phase is to show that after a polynomial number of iterations of \algGlobalSemoAlt with fitness $f_2$, a search point $X \in \{0, 1\}^n$ exists in the population $\P$, such that $\lpPrim(G) = |X_1| + \lpPrim(G[u(X)])$ and there is a fractional optimal $W$-separator $Y = \{y_v \in \R_{\geq 0}\}_{v \in u(X)}$ with $y_v < 1$ for each $v \in V$.
%Note that the equality implies that there is a optimal fractional $W$-separator where the vertices $X_1$ has value one.
We will prove that in this case $G[u(X)]$ contains no strictly reducible pair, and that because of the equality relation $\lpPrim(G) = |X_1| + \lpPrim(G[u(X)])$ all the head vertices $A_i$ for $i \in [m]$ are in $X_1$.
That is, there is an optimal $W$-separator which contains a subset of $X_1$.
\item%[Part (ii)]
\label{part2}
The second phase is to filter $\bigcup_{i=1} A_i$ from $|X_1|$ so that we obtain a search point $X'$ that selects only those as 1-bits.
Once an $X$ as described in Phase~{\ref{part1}} is guaranteed to be in the population, the algorithm \algGlobalSemoAlt takes in expectation FPT-time to reach $X'$. 
%$h(\opt,W) \cdot n^{\O(1)}$ for a computable function $h$ to reach $X'$. 
Finally, it is important that $X'$ remains in the population once we have found it.
We show this by taking advantage of the structural properties of a reducible pair in combination with the uncovered-objective.
\item%[Part (iii)]
\label{part3}
For the last phase, we know by \cref{lemma::existenceReducible} already that $u(X')$ has size at most $2 \cdot \opt \cdot W$. 
Once we ensure that $X'$ is in $\P$ and stays there, we prove that \algGlobalSemoAlt finds in expectation an optimal solution in FPT-time.
\end{enumerate}

In phase~\ref{part1}, we essentially make use of the LP objective.
To prove that it works successfully, we will show the following two lemmas.

\begin{lemma}
\label{lemma::firstStepOptLP}
Using the fitness function $f_2$, the expected number of iterations of \algGlobalSemoAlt where the population $\P$ contains a search point $X \in \{0,1\}^n$ such that $\lpPrim(G)=\lpPrim(G[u(X)]) + |X_1|$ and there is an optimal fractional $W$-separator $\{y_v \in \R_{\geq 0}\}_{v \in u(X)}$ of $G[u(X)]$ with $y_v<1$ for every $v \in u(X)$ is upper bounded by $\O(n^3(\log n + \opt))$.
Moreover, once $\P$ contains such a search point at any iteration, the same holds for all future iterations.
\end{lemma} 
%\zienain{Marcus: Why do we have this negation ('no search point') in the statement? Ziena: The negation says in principle that we do not need to worry that such a solution becomes dominated since we proof there have to be one after blah blah}

\begin{lemma}
\label{lemma::ingredientParateo1}
Let $(A_1,B_1), \dots, (A_m,B_m)$ be a sequence of minimal strictly reducible pairs in $G$, such that $G - \bigcup_{i=1} A_i$ contains no minimal strictly reducible pair.
Let $X \in \{0,1\}^n$ be a sample, such that there is a an optimal fractional $W$-separator $\{y_v \in \R_{\geq 0}\}_{v \in u(X)}$ of $G[u(X)]$ with $y_v<1$ for each $v \in u(X)$.
If $|X_1| + \lpPrim(G[u(X)]) = \lpPrim(G)$, then $A_i \subseteq X_1$ and $B_i \cap X_1 = \varnothing$ for all $i \in [m]$.
\end{lemma}

We guide the rest of this section by using $X \in \{0,1\}^n$ to denote a search point and $(A_1,B_1), \dots, (A_m,B_m)$ as a sequence of minimal strictly reducible pairs,
where $\aall := \bigcup_{i=1}^m A_i$ and $\ball := \bigcup_{i=1}^m B_i$.
For the Algorithm \algGlobalSemoAlt it is unlikely to jump from a uniformly random search point immediately to a search point satisfying \cref{lemma::ingredientParateo1}. 
To guarantee a stepwise progress,
we want that under the condition $\lpPrim(G)=\lpPrim(G[u(X)]) + |X_1|$ at each time $\aall \subsetneq X_1$ there exists a vertex of $v \in \aall \setminus X_1$ in an optimal fractional $W$-separator of $G[u(X)]$ which must have value one.
For this purpose, the characterization of minimal strictly reducible pairs by optimal fractional $W$-separators is useful.

\begin{lemma}[\cite{fomin2019kernelization}, Corollary~6.19 and Lemma~6.20]
\label{lemma::onesInLPReduciblePair}
Let $G=(V,E)$ be an instance of the $W$-separator problem and let $\{y_v \in \R_{\geq 0}\}_{v \in V}$ be an optimal fractional $W$-separator of $G$.
If $G$ contains a minimal strictly reducible pair $(A,B)$, then $y_v=1$ for all $v \in A$ and $y_u = 0$ for all $u \in B$. 	
\end{lemma}

From \cref{lemma::onesInLPReduciblePair} we can derive that if $(\aall \cup \ball) \cap X_1 = \varnothing$, such a vertex $v$ must exist, but the question is what happens if the intersection is not empty.
In particular, we want to avoid vertices of $\ball$ being in $X_1$, since reducible pairs in $G$ may then no longer exist in $G[u(X)]$.
We start with the proof of \cref{lemma::firstStepOptLP} and show later how it is related to a sequence of minimal strictly reducible pairs.
The first lemma is a simple but useful observation.


\begin{lemma}
\label{lemma::lowerBoundLP}
For every $X \in \{0,1\}^n$ it holds that $\lpPrim(G) \leq |X_1| + \lp(G[u(X)])$.
\end{lemma}
%\begin{proof}
%The fractional $W$-separator of $G[u(X)]$ with an objective value of $LP(G[u(X)])$ can be extended to a fractional $W$-separator of $G$ by the vertices $X_1$.
%Since this solution would be feasible for the linear program of the $W$-separator problem, we obtain the desired inequality.
%\end{proof}

If \cref{lemma::lowerBoundLP} is true, it is not difficult to derive that if we have that it holds with equality for a search point $X$, then $f_2(X)$ is a pareto optimal vector of the fitness function $f_2$, as given below as a corollary.

\begin{corollary}
\label{corollary::paretoOptLP}
If a search point $X \in \{0,1\}^n$ satisfy $|X_1| + \lp(G[u(X)]) = \lpPrim(G)$, then the vector $(|X_1|,*,\lp(G[u(X)])$ is a pareto optimal vector of the fitness function $f_2$.
\end{corollary}

The next lemma ensures that removing vertices with value one in an optimal fractional $W$-separator does not affect the objective of a fractional $W$-separator of the remaining graph.

\begin{lemma}[\cite{fomin2019kernelization}, Corollary 6.17]
\label{lemma::onesInLPSameObjective}
Let $G=(V,E)$ be an instance of the $W$-separator problem and let $\{y_v \in \R_{\geq 0}\}_{v \in V}$ be an optimal fractional $W$-separator of $G$.
Let $V' \subseteq V(G)$, such that $y_v = 1$ for all $v \in V'$.
Then, $\{y_v \mid v \in V \setminus V'\}$ is an optimal fractional $W$-separator of $G-V'$, i.e., ~$\sum_{v \in V \setminus V'} y_v = \lpPrim(G-V')$.
\end{lemma}

\cref{corollary::paretoOptLP,lemma::onesInLPSameObjective} allow incremental progress in the set of 1-bits with respect to search points $X \in \P$ that satisfy $|X_1| + \lp(G[u(X)]) = \lpPrim(G)$ without backstepping.
With this ingredient we can prove \cref{lemma::firstStepOptLP} (see~Section~\ref{appendix::f2} for a proof).
%\paragraph{Proof of \cref{lemma::firstStepOptLP}:}
%%\begin{proof}
%Let $\P$ be a population with $0^n \in \P$.
%Let $s \geq 0$ be the largest integer that denotes a sample $X^s$ in $\P$ with $|X^s_1| = s$ and $\lpPrim(G) = \lpPrim(G[u(X^s)]) + |X^s_1|$.
%Note that $X^0=0^n$.
%By \cref{corollary::paretoOptLP} the tuple $(|X^s_1|,*,\lpPrim(u(X^s)))$ is pareto optima, i.e.,~$s$ can never decrease.
%%Let $v \in u(X^s)$ be vertex with $\lpPrim(G) = \lpPrim(G[u(X^s)] \setminus \{v\}) + |X^s_1 \cup \{v\}|$.
%%Note that if $v$ not exists, then we are done.
%If there is an optimal fractional $W$-separator $\{y_v \in \R_{\geq 0}\}_{v \in u(X)}$ of $G[u(x)]$, such that $y_v<1$ for every $v \in u(X)$ then we are done.
%Otherwise, there is a $v \in u(X)$ with $y_v=1$.
%By \cref{lemma::onesInLPSameObjective} we obtain that $\lpPrim(G) = \lpPrim(G[u(X^s) \setminus \{v\}]) + |X^s_1 \cup \{v\}|$.
%Thus, mutating $X^s$ by only flipping the according entry $v$ in $X^s$ results in a search point $X^{s+1}$.
%By \cref{lemma::singleFlipAndBoundedPop} this event has probability $\Omega(1/n^3)$ in the algorithm \algGlobalSemoAlt or \algGlobalSemoAltTwo, and happens at most $\opt$ times as $\lpPrim(G) \leq \opt$.
%That is, $s \leq \opt$ and mutating $X^s$ to a search point $X^{s+1}$ happens after $\O(n^3)$ iterations in expectation.
%Moreover, by \cref{lemma::zeroSol} the search point $0^n$ is in the population $\P$ after $\O(n^3 \log n)$ iterations in expectation. 
%By using the method of fitness based partitions \cite{DBLP:conf/ppsn/Sudholt10}, the fact that a search point described in the lemma is not in $\P$ is upper bounded by $\O(n^3(\log n + \opt))$ iterations in expectation.
%%\end{proof}
%\vspace{0.2cm}
Since $f_3$ has one less objective than $f_2$, one can derive the following lemma.
%, essentially replacing the \cref{lemma::singleFlipAndBoundedPop,lemma::zeroSol} with the \cref{corollary::singleFlipAndBoundedPop,corollary::zeroSol} in the proof of \cref{lemma::firstStepOptLP}.
\begin{lemma}
\label{corollary::firstStepOptLP}
Using the fitness function $f_3$, the expected number of iterations of \algGlobalSemoAlt where the population $\P$ contains no search point $X \in \{0,1\}^n$ such that $\lpPrim(G)=\lpPrim(G[u(X)]) + |X_1|$ and there is an optimal fractional $W$-separator $\{y_v \in \R_{\geq 0}\}_{v \in u(X)}$ of $G[u(X)]$ with $y_v<1$ for every $v \in u(X)$ is upper bounded by $\O(n^2(\log n + \opt))$.
\end{lemma}

Our next goal is to prove \cref{lemma::ingredientParateo1}.
To identify the head vertices $\aall$ with respect to an optimal fractional $W$-separator, we want to ensure that the selection of the vertices of $\ball$ are distinguishable so that it cannot come to a conflict with \cref{lemma::firstStepOptLP}.
%characterized so that .
To do this, we will make use of the LP-objective and show that for a search point $X$ with $X_1 \cap \ball \neq \varnothing$ we have $\lpPrim(G) < \lpPrim(G[u(X)]) + |X_1|$.
Let $(A,B)$ be a minimal strictly reducible pair in $G$.
The essential idea is to use $(W+1)$-packings in $G[A \cup B]$, since they provide lower bounds for $W$-separators. %in $G[A \cup B]$.
From \cref{lemma::packingReduciblePair} one can deduce that $G[A \cup B]$ contains a maximum ($W+1$)-packing $\Q$ of size $|A|$, since every vertex of $A$ is contained exactly in one element of $\Q$.
Inspired by ideas on how to find crown decompositions in weighted bipartite graphs from~\cite{DBLP:conf/esa/Casel0INZ21,DBLP:conf/iwpec/KumarL16}, we prove that removing vertices from $B$ only partially affects the size of the $(W+1)$-packing in $G[A \cup B]$, as stated in the following lemma.

\begin{lemma}
\label{lemma::PackingSizeLargeEnough}
Let $(A,B)$ be a minimal strictly reducible pair in $G=(V,E)$ and let $S \subset A \cup B$ with $|S| \leq |A|$.
If $S \cap B \neq \varnothing$, then $G[(A \cup B)] - S$ contains a packing of size $|A| - |S| + 1$.
\end{lemma}

In contrast, note that removing vertices $S \subseteq A$ from $G[A \cup B]$ would decrease the size of a $(W+1)$-packing by $|S|$,
i.e., a maximum $(W+1)$-packing in $G[A \cup B]-S$ has size $|A|-|S|$.
We moved the proof of \cref{lemma::PackingSizeLargeEnough} to the appendix (Section~\ref{appendix::f2}), since it is more technical and too long given the space constraints.
Essentially, we make use of the following two lemmas and properties of network flows.
In particular, these lemmas describe the new properties we have found for minimal strictly reducible pairs and may be of independent interest.

\begin{lemma}
	\label{lemma::decideWhichA}
	Let $(A,B)$ be a minimal strictly reducible pair in $G$ with parameter $W$.
	Then, for every $a^* \in A$ there is an assignment function $g \colon \B \times A \to \N_0$ like in \cref{def::RedPair} that satisfies $\sum_{C \in \B} g(C,a^*) \geq 2W$ and $\sum_{C \in \B} g(C,a) \geq 2W -1$ for every $a \in A \setminus \{a^*\}$.
\end{lemma}

Concerning \cref{lemma::decideWhichA}, we remark that the new feature to before is that the particular vertex (in the lemma $a^*$) can be chosen arbitrarily.

\begin{lemma}
	\label{lemma::SizeNeoghborhood}
	Let $(A,B)$ be a minimal strictly reducible pair in $G$ with parameter $W$.
	Then, for every $A' \subseteq A$ we have $|V(\B_{A'})| \geq |A'|(2W-1)+1$.
\end{lemma}

%To prove \cref{lemma::PackingSizeLargeEnough}, we recall some definitions and introduce new ones.
%Let $(A,B)$ be a minimal strictly reducible pair with an assignment function $g$ (cf.~\cref{def::RedPair}).
%We use $\B$ to denote the connected components of $G[B]$ as vertex sets.
%Moreover, recall the definitions $\overrightarrow{\B}_a :=\left\{C \in \B \mid g\left(C,a\right) > 0 \right\}$ for $a \in A$, $\overrightarrow{\B}_{A'} := \bigcup_{a \in A'} \overrightarrow{\B}_a$ for $A' \subseteq A$ and $\overrightarrow{\B}_U :=\left\{C \in \B \mid \sum_{a \in A} g\left(C,a\right) < |C| \right\}$.
%For a subset $\B' \subseteq \B$ we define $V(\B') := \bigcup_{C \in \B'} C$.
%We abuse notation and use $N(\B')$ to denote $N(V(\B'))$.
%For $A' \subseteq A$ we define $\B_{A'} := \{C \in \B \mid N(C) \subseteq A'\}$. 
%
%Mainly, we found new properties concerning minimal strictly reducible pairs which are useful for the proof of \cref{lemma::PackingSizeLargeEnough}.
%One of them is that the particular vertex in $A$, which is guaranteed to have a larger mapping of fractional vertices from components in $\B$ over $g$, can be chosen arbitrarily.
%To prove this, we construct two similar flow networks $H$ and $H_a$ for $a \in A$ with respect to $A, B$ and $g$, which we will call \emph{$g$-embedded flow networks}.
%Basically, it describes the assignments from $\B$ to $A$ via $g$ with the advantage that we can make use of the properties provided by a flow network. 
%For the proofs it is more convenient to reduce the assignments of each $a \in A$ that are given via $g$ arbitrary from $\oB_{a}$ so that we have $\sum_{C \in \B} g(C,a) = 2W-1$.
%Now we describe the procedure getting the $g$-embedded flow network $H = \left(A \cup \B \cup \left\{s,t\right\}, \overrightarrow{E}, c\right)$.
%We add the vertices $s,t$, $s$ as source and $t$ as sink, and arcs $\overrightarrow{E}$ with a capacity function $c\colon\overrightarrow{E} \to \mathbb{N}$ defined as follows:
%connect every $C \in \B$ through an arc $\overrightarrow{sC}$ with capacity $|C|$ and connect every $a \in A$ through an arc $\overrightarrow{at}$ with capacity $2W-1$. 
%Moreover, for every $C \in \B$ and $a \in A$ add an arc $\ora{Ca}$ to $\ora{E}$ if $a \in N(C)$ with capacity $|C|$.
%Let $Z = \left\{z_e \in \N_0 \mid \ora{e} \in \ora{E} \right\}$ denote the flow values in $H$.
%We embed now $g$ naturally in $H$.
%For this, we set $z_{\ora{Ca}} = g(C,a)$ for every $a \in A$ and $C \in \B$.
%To ensure now that the flow conversation is ensured for the vertices $V(H) \setminus \{s,t\}$ we set $z_{\ora{sC}} = \sum_{a \in A} g(C,a)$ for each $C \in \B$ and $z_{\ora{at}} = \sum_{C \in \B} g(C,a)$ for each $a \in A$.
%It is not difficult to see that no arc capacity is violated and hence we have feasible flow from $s$ to $t$ in $H$.
%We denote by $F = |A|(2W-1)$ the total flow value $\sum_{a \in A} z_{\ora{at}}$.   
%Note that we have $z_{\ora{at}} = 2W-1$ for each $a \in A$, which means that these arcs are \emph{saturated} as $c(\ora{at}) = 2W-1$ and therefore $Z$ is a maximum flow in $H$.
%
%For $a \in A$ the flow network $H_a$ is basically the network $H$ with the difference that $c(\ora{at}) = 2W$.
%We will accordingly denote the flow in $H_a$ by $Z^a = \left\{z^a_e \in \N_0 \mid \ora{e} \in \ora{E} \right\}$ and its flow value by $F_a$.
%Note that this time we do not know, whether we have a maximum flow in hand as $\ora{at}$ is not saturated in $H_a$.
%
%\begin{lemma}
%\label{lemma::flowEverywhere}
%Let $(A,B)$ be a minimal strictly reducible pair in $G$.
%Then, for every $a \in A$ the maximum flow from $s$ to $t$ in $H_a$ is $|A|(2W-1) + 1$.
%\end{lemma}
%
%\begin{proof}
%Recall by the definition of a strictly reducible pair that there is a $g$ that satisfy already $\sum_{C \in \B} g(C,a) \geq 2W -1$ for each $a \in A$ and for an $a' \in A$ even $\sum_{C \in \B} g(C,a') \geq 2W$.
%Basically, we want to show that this particular element in $A$ can be chosen arbitrarily if $(A,B)$ is a minimal strictly reducible pair.
%If this is not possible, then we will prove that there is a smaller strictly reducible pair $(A',B')$ with $A' \subset A$ and $B' \subseteq B$ contradicting the minimality of $(A,B)$.
%
%Consider the $g$-embedded flow networks $H$ and $H_a$ for each $a \in A$.
%For each $a \in A$ we search for an augmenting path in the residual graph of $H_a$ with respect to $Z^a$ (in the beginning $Z^a = Z$) from $s$ to $t$, or in other words, we will check, whether a larger maximum $s$-$t$-flow $F=|A|(2W-1)$ as in $H$ is realizable in $H_a$. 
%Observe that an augmenting path in $H_a$ with respect to $Z^a$ has to run through $\ora{at}$. 
%Let $A_1$ and $A_2$ be a partition of $A$, where for $a \in A_1$ the networks $H_a$ allow a greater total flow value from $s$ to $t$ than $H$, i.e.~$F_a = F+1 = |A|(2W-1) + 1$.
%Note that if we would have $F_a = F+1$ for all $a \in A$, i.e.,~$A_1=A$, then the lemma holds and we are done.
%Moreover, note that $A_1 \neq \varnothing$, since %$(A,B)$ is a strictly reducible pair and 
%before reducing the assignments of $g$ for the embedding in $H$ there is a particular vertex $a \in A$ with $\sum_{C \in \B} g(C,a) \geq 2W$.
%
%Assume $A_1 \neq A$.
%We prove in this case that there is a strictly reducible pair $(A_1,B_1)$ with $B_1 \subset B$, which would contradict the minimality of $(A,B)$.
%Let $C' \in \B'$ be the components in $\B$, where for an $a_1 \in A_1$ we have $z^{a_1}_{\ora{Ca}} > 0$ in $Z^{a_1}$.
%If $N(C') \cap A_2 = \varnothing$ for all $C' \in \B'$, then $(A_1, \bigcup_{C' \in \B'} C')$ is a strictly reducible pair and we are done, since $N(\B') \subseteq A_1$ and we can generate a corresponding assignment function $\B' \times A_1 \to \N_0$ through the flows $Z^{a_1}$ for an arbitrary $a_1 \in A_1$.
%Note that $N(B) \subseteq A$ by the definition of a strictly reducible pair, which in turn means that $\bigcup_{C' \in \B'} C'$ has no neighbor outside $A$.
%Thus, we can assume that there is a $C' \in \B'$ with $N(C') \cap A_2 \neq \varnothing$, let us say $a_2 \in A_2$ is connected to $C'$.
%If $\sum_{a \in A} z_{\ora{C'a}} = \sum_{a \in A} g(C',a) = 0$, then $a_2$ cannot be in $A_2$ as $F_{a_2}=F+1$ is easily realizable as there is a augmenting flow path from $s$ to $t$ via $s,C',a_2,t$ in $H_{a_2}$, and $a_2$ would be in $A_1$.
%Let $a_1 \in A_1$ be a vertex with $z_{\ora{C'a_1}}>0$.
%Observe that the initial flow values $z_{\ora{Ca_1}}>0$ for $C \in \B$ in the network $H_{a_1}$, i.e., before augmenting flow in $H_{a_1}$, can only be decreased if we use the corresponding residual arc $\ora{a_1C}$ in a residual $s$-$t$-path $P$.
%But this cannot happen as $\ora{a_1t}$ is an arc in $P$, which means a residual arc $\ora{a_1C}$ cannot be part of the path $P$.
%From this, we obtain that $z^{a_1}_{\ora{C'a_1}} \geq z_{\ora{C'a_1}}>0$.
%For the following step observe that $z^{a_1}_{\ora{a_1t}} = 2W$ and $z^{a_1}_{\ora{at}} = 2W-1$ for all $a \in A \setminus \{a_1\}$.
%With these facts in hand we can easily transform $Z^{a_1}$ to a flow in $H_{a_2}$ with a total flow value of $F+1$ by subtracting one unit flow from $z^{a_1}_{\ora{C'a_1}}$ and $z^{a_1}_{\ora{a_1t}}$ and adding one unit of flow to $z^{a_2}_{\ora{C'a_2}}$ and $z^{a_2}_{\ora{a_2t}}$.
%However, this contradicts that $a_2 \in A_2$.
%\end{proof}
%
%The next two corollaries are easily derived from \cref{lemma::flowEverywhere} and describes the new properties we found for minimal strictly reducible pairs.
%
%\begin{corollary}
%\label{corollary::decideWhichA}
%Let $(A,B)$ be a minimal strictly reducible pair in $G$ with parameter $W$.
%Then, for every $a^* \in A$ there is an assignment function $g \colon \B \times A \to \N_0$ like in \cref{def::RedPair} that satisfies $\sum_{C \in \B} g(C,a^*) \geq 2W$ and $\sum_{C \in \B} g(C,a) \geq 2W -1$ for every $a \in A \setminus \{a^*\}$.
%\end{corollary}
%
%\begin{corollary}
%\label{corollary::SizeNeoghborhood}
%Let $(A,B)$ be a minimal strictly reducible pair in $G$ with parameter $W$.
%Then, for every $A' \subseteq A$ we have $|V(\B_{A'})| \geq |A'|(2W-1)+1$.
%\end{corollary}
%
%To prove \cref{lemma::PackingSizeLargeEnough}, we use the procedure from \cite{DBLP:conf/esa/Casel0INZ21,DBLP:conf/iwpec/KumarL16} on how to find a ($W+1$)-packing of size $|A|$ in $G[A \cup B]$.
%We refine this procedure with respect to the missing vertices of $A \cup B$ using \cref{lemma::flowEverywhere} and the augmenting flow properties of network flows.
%
%\paragraph{Proof of \cref{lemma::PackingSizeLargeEnough}:}
%%\begin{proof} 
%To prove the lemma we will careful pack the vertices in $G[A \cup B] - S$ into connected subgraphs of size $W+1$, such that we obtain a $(W+1)$-packing of size $|A|-|S|+1$.
%Let $(A,B)$ be a minimal strictly reducible pair in $G$. %with an according assignment function $g$.
%We define for an assignment function $g \colon \B \times A \to \N_0$ an edge-weighted graph $G_g := (A \cup \B, E', w: E' \to \N_0)$, where $E' := \{aC \in A \times \B \mid g(C,a) > 0\}$ and $w_{Ca} = g(C,a)$ for every $Ca \in E'$.
%First we demonstrate how a \emph{cycle canceling process} in $G_g$ works without changing the assignments from $\B$ to $A$, i.e., maintaining $\sum_{C \in N\left(a\right)} w_{Ca} = \sum_{C \in \B} g(C,a)$ for all $a \in A$, while ensuring $\sum_{a \in N\left(C\right)} w_{Ca} = \sum_{a \in A} g(C,a) \leq |C|$ for all $C \in \B$.
%After that we demonstrate how the \emph{packing process} works to find a $(W+1)$-packing of size $|A|$ in $G[A \cup B]$.
%Both processes was demonstrated identically already in~\cite{DBLP:conf/esa/Casel0INZ21,DBLP:conf/iwpec/KumarL16}.
%We repeat the explanation of them because they are elementary to understand how we find our desired packing in $G[A \cup B]-S$ of size at least $|A|-|S|+1$.
%
%Let $g$ be the assignment function of the minimal strictly reducible pair $(A,B)$.
%Suppose there exists a cycle $Q \subseteq G_g$.
%We pick an edge $Ca \in E'(Q)$ with minimum weight $\min_{Ca \in E\left(Q\right)} w_{Ca} = x$.
%We remove $Q$ from $G_g$ as follows.
%First, we reduce the weight of $w_{Ca}$ by $x$.
%Then, traversing the cycle $Q$ from $C$ in direction of its neighbor that is not $a$, we alternate between increasing and decreasing the weight of the visited edge by $x$ until we reach $a$.
%In the end we obtain at least one zero weight edge, namely $Ca$, and we remove all edges of weight zero from $G_g$. 
%Since $G_g$ is bipartite, the number of edges in the cycle is even.
%Thus, we will maintain the weight $\sum_{C \in N\left(a\right)} w_{Ca} = \sum_{C \in \B} g(C,a)$ for all $a \in A$ during this process.
%Furthermore, we transfer weight on an edge incident to a $C \in \B$ to another edge, which in turn is also incident to this vertex $C$.
%As a result, the condition $\sum_{a \in N\left(C\right)} w_{Ca} = \sum_{a \in A} g(C,a) \leq |C|$ for all $C \in \B$ is still satisfied.
%We repeat this process until $G_g$ becomes a forest.
%Moreover, if $w_{Ca} > 0$, then $a \in N(C)$ for every $C \in \B$ and $a \in A$, since $g(C,a) > 0$ leads to the same and we add no new edge to $G_g$. 
%
%Next, we give the packing process, i.e., we show how to find the $(W+1)$-packing in $G[A \cup B]$ of size $|A|$ 
%We transform accordingly the resulting weights back to a new assignment function $g' \colon \B \times A \to \N_0$ by setting $g'(C,a) = w_{Ca}$.
%Note that $g'$ satisfies the condition of \cref{def::RedPair}, and that $G_{g'}$ is a forest.
%Let $T$ be a tree in $G_{g'}$ rooted at an $a_r \in A$.
%We have $\sum_{C \in \B} g'(C,a) \geq 2W-1$ for every $a \in A_1$ in the tree.
%Since $G_{g'}$ is a bipartite graph, every child of $a \in A_1$ is in $C \in \B$ and vice versa.
%Let $\B_T$ and $A_T$ be the vertices in $T$ of $\B$ and $A$, respectively.
%We construct now the desired packing by mapping the elements of $\B_T$ to $A_T$.
%Let $f \colon \B \to A$ be a function that describes for each $C \in \B_T$ to which $a \in A$ it belongs to. 
%We set $f(C) = a$ for every $a \in A_T$ if $C \in \B_T$ is a child vertex of $a$.
%Note that $a \in N(C)$ if $f(C) = a$.
%Thus, for $a_r$ we obtain $\left|\bigcup_{C \in f^{-1}(a_r)} C \right| = \sum_{C \in \B} g'(C,a_r) \geq 2W-1$.
%For all other vertices $a' \in A_T$ that are in the tree, we only lose its father vertex $C'$, which in turn has also a father $a'' \in A_T \setminus \{a\}$.
%That is, $g'(C',a'') \geq 1$.
%It follows that $g'(C',a') \leq |C| - 1 \leq W - 1$ for all those vertices and we obtain $\left|\bigcup_{C \in f^{-1}(a)} C \right| \geq \sum_{C \in \B} g'(C,a) - (W-1) \geq (2W-1)-(W-1) = W$ for all $a \in A_T \setminus \{a_r\}$.
%We execute this process with all trees in $G_g$ and get our desired packing $\Q$ by setting each $Q_a = \{a\} \cup f^{-1}(a)$ for $a \in A$ as an element of it.
%
%We know how to find a $(W+1)$-packing $\Q$ of size $|A|$ in $G[A \cup B]$ yet.
%As a reminder, the properties for each $Q \in \Q$ are that $|Q| \geq W+1$ and $G[Q]$ is connected (\emph{pack properties}), and that the elements in $\Q$ are pairwise disjoint.
%Our goal is now to find a $(W+1)$-packing in $G[A \cup B]-S$ of size at least $|A|-|S|+1$.
%Observe by the disjointedness of $\Q$ that after removing $S$ from $\Q$ at most $|S|$ elements in $\Q$ can lose its pack properties and if this is exactly $|S|$, then every $s \in S$ has an exclusive $Q \in \Q$ where it is contained.
%Otherwise, either we have ~$|S \cap Q| \geq 2$ for some $Q \in \Q$, or there is an $s \in S$ that has no intersection with an element in $\Q$. Consequently, the elements in $\Q$ containing no vertex from $S$ are at least $|\Q| - |S| + 1 = |A|-|S|+1$ many and form the desired packing satisfying the lemma.
%
%We can assume that every $s \in S$ is contained in exactly one $Q \in \Q$.
%Let $S_A = A \cap S$ and $S_B = B \cap S$.
%Note by the precondition of the lemma that $S_B \neq \varnothing$ and $S_A \neq A$.
%For $s \in S_B$ let $\B(s)$ denote the component in $\B$ where $s$ is contained.
%Since every $s \in S$ is contained in a $Q \in \Q$, we can assume for $s \in S_B$ that $\B(s)$ occurs in one of the trees of $G_{g'}$ and therefore $\sum_{a \in A} g'(\B(s),a) > 0$ holds.
%To get now the desired ($W+1$)-packing we use the properties of the root node $a_r$ while constructing $\Q$.
%Recall that we have $\left|\bigcup_{C \in f^{-1}(a_r)} C \right| = \sum_{C \in \B} g'(C,a_r) \geq 2W-1$ and that we can fix a root node arbitrary from $A$.
%%Suppose there is an $s' \in S$, such that for every $a \in A$ we have $g'(\B(s'),a) < W$.
%Let $a' \in A$ be a vertex with $g'(\B(s'),a) > 0$, which exists as $\sum_{a \in A} g'(\B(s'),a) > 0$.
%Let $\Q'$ be the $(W+1)$-packing after the packing process with $f' \colon \B \to A$ as the corresponding assignment where $a'$ is the root in one of the trees of $G_{g'}$.
%We can assume that $a' \notin S$ and only $\B(s')$ contains a vertex of $S$ among the child vertices, as  otherwise, we have after the packing process an element $Q \in \Q'$ with $|S \cap Q| \geq 2$ and we are able to find the desired ($W+1$)-packing in $G[A \cup B]-S$. 
%If $g'(\B(s'),a) < W$, we obtain $\left|\bigcup_{C \in f'^{-1}(a') \setminus \{\B(s')\} } C  \right|  = \sum_{C \in \B \setminus \{\B(s')\}} g'(C,a_r) \geq 2W-1-(W-1)=W$.
%This in turn means that $\left( \{a'\} \cup \bigcup_{C \in f'^{-1}(a')} C \right) \setminus S$ satisfies the pack properties and we are able to construct the desired $(W+1)$-packing, because at most $|S|-1$ elements in $\Q'$ after removing the vertices of $S$ in $\Q'$ can violate the pack properties.
%
%Thus, we can assume that $g'(\B(s'),a') = W$.
%If $\sum_{C \in \B} g'(C,a') \geq 2W$, then we are done by the explanations above as $\left|\bigcup_{C \in f'^{-1}(a') \setminus \{\B(s')\} } C  \right|  = \sum_{C \in \B \setminus \{\B(s')\}} g'(C,a_r) \geq 2W-W=W$.
%If not, we create a valid assignment function $g''$ like in \cref{def::RedPair} with $\sum_{C \in \B} g''(C,a') \geq 2W$, where $G_{g''}$ is as forest and $\B(s')$ is adjacent to $a'$ in $G_{g''}$. 
%For this, we construct a $g'$-embedded flow network $H_{a'}$.
%By \cref{lemma::flowEverywhere} we know that there is an augmenting path from $s$ to $t$ and by the construction of $H_{a'}$ that has to cross the residual arc $\ora{a't}$.
%This augmenting path $P$ cannot decrease the initial flow values $z^{a'}_{\ora{Ca'}}$ as an augmenting path need to use a residual arc $\ora{a'C}$ to do this, which would contradict that $P$ is a path as it has to use the arc $\ora{a't}$.
%In particular, after extracting the resulting assignment function $\hat{g}$ from $Z^{a'}$, i.e., setting $\hat{g}(C,a) = z^{a'}_{\ora{Ca}}$ for all $a \in A$ and $C \in \B$, we have $\hat{g}(\B(s'),a') = W$.
%Observe that due to $\sum_{C \in \B} \hat{g}(C,a') \leq |W|$ the vertex $\B(s')$ has degree one in $G_{\hat{g}}$.
%That is, if we execute the cycle canceling process in $G_{\hat{g}}$ the vertex $\B(s')$ cannot be in any cycle of $G_{\hat{g}}$ during this process.
%As a result, after extracting the resulting assignment function after the cycle canceling process this corresponds to the desired $g''$ where $\B(s')$ is adjacent to $a'$ in $G_{g''}$.
%
%\vspace{0.2cm}

To conclude the Phase~\ref{part1} we need to prove \cref{lemma::ingredientParateo1}.
Equipped with \cref{lemma::PackingSizeLargeEnough} we may prove statements about the LP-objective if $X_1 \cap \ball \neq \varnothing$.
In doing so, we prove another relation with respect to such a sequence, which fits the proof and will be useful later.


%Before doing this, we use the following property concerning LP's for the $W$-separator problem.
%
%\begin{lemma}
%\label{lemma::partitionLP}
%Let $G=(V,E)$ be a graph and let $V_1,V_2$ be a partition of $V$.
%For every $W \in \N$ of the $W$-separator problem we have $\lpPrim(G) \geq \lpPrim(G[V_1]) + \lpPrim(G[V_2])$.
%\end{lemma}    
%\begin{proof}
%Let $\{y_v\}_{v \in V}$ be a fractional $W$-separator of $G$.
%For every connected subgraph $G'$ of size $W+1$ in $G[V_1]$ or $G[V_2]$ we have $\sum_{v \in V(G')} y_v \geq 1$. 
%Consequently, $\{y_v\}_{v \in V_1}$ and $\{y_v\}_{v \in V_2}$ are fractional $W$-separators of $G[V_1]$ and $G[V_2]$, which shows that the desired inequality holds.
%\end{proof}
%
%First, we give a relation of the LP-objective to a minimal strictly reducible pair $(A,B)$ in case of $B \cap X_1 \neq \varnothing$, which we will use to prove a relation of the LP-objective to a sequence of minimal strictly reducible pair.
%
%\begin{lemma}
%\label{lemma::RedPairLPNoB}
%Let $(A,B)$ be an minimal strictly reducible pair in $G$ and let $X_1 \in \{0,1\}^n$ be a search point.
%If $X_1 \cap B \neq \varnothing$, then $\lpPrim(G(u[X])) + |X_1| > \lpPrim(G)$.
%\end{lemma}
%\begin{proof}
%Let $\oAB = V \setminus (\AB)$, $X_1^{\AB} = X_1 \cap (\AB)$ and $X_1^{\oAB} = X_1 \cap \oAB$.
%Note $\AB$ and $\oAB$ are a partition of $V$ as well as $X_1^{\AB}$ and $X_1^{\oAB}$ a partition of $X_1$.
%And recall that for every vertex bipartition $V_1,V_2$ of a graph $G'$ that $\lpPrim(G') \geq \lpPrim(G'[V_1]) + \lpPrim(G'[V_2])$ (cf.~\cref{lemma::RedPairLPNoB}).
%Moreover, observe that $\lpPrim(G[u(X)]) = \lpPrim(G - X_1)$.
%Thus, we have 
%\begin{align*}
%	&\lpPrim(G[u(X)]) + |X_1|\\
%	&=\lpPrim(G-X_1) + |X_1|\\
%	&\geq \lpPrim(G[\oAB] - X_1^{\oAB}) + \lpPrim(G[\AB] - X_1^{\AB}) + |X_1|\\
%	&= \lpPrim(G[\oAB] - X_1^{\oAB}) + |X_1^{\oAB}| + \lpPrim(G[\AB] - X_1^{\AB}) + |X_1^{\AB}|\\
%	&\geq  \lpPrim(G[\oAB]) + \lpPrim(G[\AB] - X_1^{\AB}) + |X_1^{\AB}|,
%\end{align*}
%where the last inequality follows by \cref{lemma::lowerBoundLP}.
%By \cref{lemma::onesInLPReduciblePair} and \cref{lemma::onesInLPSameObjective} we obtain that $\lpPrim(G) = \lpPrim(G-A) + |A|$.
%Note that $\lpPrim(G-A) =  \lpPrim(G[\oAB])$ by the separation properties of $A$ and hence, if we can show that $\lpPrim(G[\AB] - X_1^{\AB}) + |X_1^{\AB}| > |A|$ we are done.
%Clearly, if $|X_1^{\AB}| > |A|$ there is nothing to prove.
%Assume $|X_1^{\AB}| \leq |A|$.
%Let $\{y_v \in \R_{\geq 0}\}_{v \in V}$ be an optimal fractional $W$-separator of $G[\AB] - X_1^{\AB}$.
%Because $B \cap X_1^{\AB} \neq \varnothing$ and $|X_1^{\AB}| \leq |A|$ we know by \cref{lemma::PackingSizeLargeEnough} that $G[\AB] - X_1^{\AB}$ contains a ($W+1$)-packing $\Q$ of size $|A| - |X_1^{\AB}| + 1$.
%This in turn means that $\lpPrim(G[\AB] - X_1^{\AB}) \geq |A| - |X_1^{\AB}| + 1$, since the elements in $\Q$ are pairwise disjoint and for each $Q \in \Q$ we have $\sum_{v \in Q} y_v \geq 1$ for any feasible fractional $W$-separator.
%In summary, we obtain
%\begin{align*}
%	&\lpPrim (G[u(X)]) + |X_1|\\  
%	&\geq  \lpPrim(G[\oAB]) + \lpPrim(G[\AB] - X_1^{\AB}) + |X_1^{\AB}|\\ 
%	&= \lpPrim(G - A) + \lpPrim(G[\AB] - X_1^{\AB}) + |X_1^{\AB}|\\
%	&\geq \lpPrim(G - A) + |A| - |X_1^{\AB}| + 1 + |X_1^{\AB}|\\
%	&= \lpPrim(G - A) + |A| + 1\\
%	&= \lpPrim(G) + 1\\ 
%	&> \lpPrim(G).
%\end{align*}
%\end{proof}
%
%We extend the relation of the LP-objective from $X$ to a sequence of minimal strictly reducible pairs.
%In doing so, we prove another relation with respect to such a sequence, which fits the proof and will be useful later.

\begin{lemma}
\label{lemma::RedPairLPNoSequence}
Let $(A_1,B_1), \dots, (A_m,B_m)$ be a sequence of minimal strictly reducible pairs and let $X_1 \in \{0,1\}^n$ be a sample.
\begin{enumerate}
	\item If $X_1 = \bigcup_{i=1}^m A_i$, then $\lpPrim(G) = \lpPrim(G[u(X)]) + |X_1|$.
	\label{enum::1}
	\item If $X_1 \cap B_\ell \neq \varnothing$ for an $\ell \in [m]$, then $\lpPrim(G(u[X])) + |X_1| > \lpPrim(G)$.
	\label{enum::2}
\end{enumerate}
\end{lemma}
%\begin{proof}
%Let $A^i := \bigcup_{j=1}^i A_j$.
%By \cref{lemma::onesInLPReduciblePair} and \ref{lemma::onesInLPSameObjective} we have $\lpPrim(G - A^i) = \lpPrim(G - A^{i-1}) + |A_i|$ for every $i \in [m]$.
%Starting with 
%$\lpPrim(G) = \lpPrim(G - A^1) + |A_1|$ and stepwise substituting $\lpPrim(G - A^{i})$ for $i \in [m]$ yields 
%$$\lpPrim(G) = \lpPrim(G - A^m) + |A_1| + \dots + |A_m|.$$
%Thus, if $X_1 = \bigcup_{i=1}^m A_i$, then $\lpPrim(G) = \lpPrim(G[u(X)]) + |X_1|$ proving \ref{enum::1}.
%
%Suppose $X_1 \cap B_\ell \neq \varnothing$.
%Observe that $|A_i| = \lpPrim(G[A_i \cup B_i])$ by \cref{lemma::onesInLPReduciblePair} and $\lpPrim(G[A_i \cup B_i] - A_i) = 0$.
%Hence, we obtain
%$$\lpPrim(G) = \lpPrim(G - A^m) + \lpPrim(G[A_1 \cup B_1]) + \dots + \lpPrim(G[A_m \cup B_m]).$$
%That is, to guarantee $\lpPrim(G(u[X])) + |X_1| = \lpPrim(G)$ we must have for every $i \in [m]$ that $\lpPrim(G[A_i \cup B_i] - X_1) + |X_1 \cap (A_i \cup B_i)| = \lpPrim(G[A_i \cup B_i])$.
%Note that to compensate $\lpPrim(G[A_i \cup B_i] - X_1) + |X_1 \cap (A_i \cup B_i)| > \lpPrim(G[A_i \cup B_i])$ we need for a $j \in [m] \setminus \{i\}$ that $\lpPrim(G[A_j \cup B_j] - X_1) + |X_1 \cap (A_j \cup B_j)| < \lpPrim(G[A_j \cup B_j])$, however, this is not possible as $\lpPrim(G[A_p \cup B_p] - X_1) + |X_1 \cap (A_p \cup B_p)| \geq \lpPrim(G[A_p \cup B_p])$ for all $p \in [m]$ by \cref{lemma::lowerBoundLP}.
%Finally, we have $\lpPrim(G[A_\ell \cup B_\ell] - X_1) + |X_1 \cap (A_\ell \cup B_\ell)| > \lpPrim(G[A_\ell \cup B_\ell])$ by \cref{lemma::RedPairLPNoB}, which proves \ref{enum::2}.
%\end{proof}

Suitable for \cref{lemma::firstStepOptLP} we have characterized the case $X_1 \cap \ball \neq \varnothing$.
It remains to give a relation to this lemma when $\aall \cap X_1 \neq \varnothing$ and $\aall \subsetneq X_1$.
In particular, we want to ensure that in this case at least one vertex of $\aall \setminus X_1$ must be one in an optimal fractional $W$-separator of $G[u(X)]$.

\begin{lemma}
\label{lemma::CrownStaysCrown}
Let $(A,B)$ be a minimal strictly reducible pair in $G$ and let $\hat{A} \subset A$.
Then, there is a partition $A_1, \dots, A_m$ of $A \setminus \hat{A}$ with disjoint vertex sets $B_1, \dots, B_m \subseteq B$, such that for each $i \in [k]$ the tuple $(A_i,B_i)$ is a minimal strictly reducible pair in $G - \hat{A}$.  
\end{lemma}
%\begin{proof}
%After removing $\hat{A}$ from $A$ and $V(\B_{\hat{A}})$ from $B$ we obtain by \cref{corollary::decideWhichA} that $\left(A \setminus \hat{A},B \setminus V(\B_{\hat{A}})\right)$ is a strictly reducible pair.
%This in turn means that there exists a nonempty minimal strictly reducible pair with $A_1 \subseteq (A \setminus \hat{A})$ and $B_1 \subseteq B \setminus V(\B_{\hat{A}})$.
%Since $N(B_1) \subseteq A_1$, we can use similar arguments for $\hat{A}' = \hat{A} \cup A_1$ and $V(\B_{\hat{A}'})$ as for $\hat{A}$ and $V(\B_{\hat{A}})$ to obtain that there exists a minimal strictly reducible pair $(A_2,B_2)$ with $A_2 \subseteq A \setminus \hat{A}'$ and $B_2 \subseteq B \setminus V(\B_{\hat{A}'})$.
%Thus, we can continue this process until $\hat{A} \cup A_1 \cup A_2 \dots $ is equal to $A$.
%\end{proof}

By \cref{lemma::onesInLPReduciblePair} we already know that the head vertices of a minimal strictly reducible pair in an optimal fractional $W$-separator have value one.
\cref{lemma::CrownStaysCrown} ensures that if some of the head vertices are removed, the value of the remaining head vertices in the respective optimal fractional solution remain one.
The proof of \cref{lemma::ingredientParateo1} can be found in the appendix (Section~\ref{appendix::f2}) and concludes Phase~\ref{part1}.

%\paragraph{Proof of \cref{lemma::ingredientParateo1}:} 
%%\begin{proof}
%As $|X_1| + \lpPrim(G[u(X)]) = \lpPrim(G)$ we obtain by \cref{lemma::RedPairLPNoSequence} that $X_1 \cap B_i = \varnothing$ for every $i \in [m]$.
%Remains to show that $A_i \subseteq X_1$ for every $i \in [m]$.
%Assume $A_j$ is a set with $A_j \subsetneq X_1$ for $j \in [m]$, where $A_i \subseteq X_1$ for $i \in [j-1]$.
%Let $g$ be an assignment function like in \cref{def::RedPair} for $(A_j,B_j)$.
%Since no vertex of $B_j$ is in $X_1$, the vertices $A' = A_j \setminus X_1$ with $B' = \bigcup_{C \in \B_{A'}} C$ have to be in $u(X)$ as every $a \in A'$ satisfies $\sum_{C \in \B_{A'}} g(C,a) \geq 2W-1$.
%Note that if $X_1 \cap V(\B_{A'}) = \varnothing$, then $\sum_{C \in \B_{A'}} g(C,a) \geq 2W-1$ for an $a \in A'$ implies that the vertex $a$ is in a connected subgraph of size at least $W+1$ in $G-X_1$.
%Combining the fact $\bigcup_{i=1}^{j-1} A_i \subseteq X_1$ with \cref{lemma::CrownStaysCrown}  we obtain that there is a partition $\hat{A}_1, \dots, \hat{A}_m$ of $A'$ with disjoint vertex sets $\hat{B}_1, \dots, \hat{B}_p \subseteq \bigcup_{C \in \B_{A'}} C$, such that for each $i \in [k]$ the tuple $(A_i,B_i)$ is a minimal strictly reducible pair in $G[u(X)]$.
%However, by \cref{lemma::onesInLPSameObjective} we have then $y_a=1$ for every $a \in A'$ contradicting the precondition of the lemma.
%\vspace{0.2cm}
%\end{proof}

Next, we prove that Phase~\ref{part2} works successfully.
After Phase~\ref{part1}, we have a search point $X$ in the population $\P$ with $\aall \subseteq X_1$ such that $\lpPrim(G) = \lpPrim(G[u(X)]) + |X_1|$.
Consequently, $|X_1| \leq \opt$ and therefore we can prove that \algGlobalSemoAlt reaches a search point $X'$ with $X'_1 = \aall$ from $X$ in FPT-time.

\begin{lemma}
\label{lemma::XReducedInPop}
Let $G=(V,E)$ be an instance of the $W$-separator problem, and let $(A_1,B_1), \dots, (A_m,B_m)$ be a sequence of minimal strictly reducible pairs in $G$, such that $G - \bigcup_{i=1} A_i$ contains no strictly reducible pair.
Using the fitness function $f_2$, the expected number of iterations of \algGlobalSemoAlt until the population $\P$ contains a search point $X$ with $X_1 = \bigcup_{i=1}^m A_i$ is upper bounded by $\O\left(n^3(\log n + \opt) + 2^{\opt}\right)$.
\end{lemma}
%\begin{proof}
%Let $\P$ be a population with respect to $f_2$ in the algorithm \algGlobalSemoAltTwo.
%By \cref{lemma::firstStepOptLP} we have in expectation a solution $X$ in $\P$ such that $\lpPrim(G[u(X)]) + |X_1| = LP(G)$ and there is an optimal fractional $W$-separator $\{y_v \in \R_{\geq 0}\}_{v \in u(X)}$ of $G[u(X)]$ with $y_v<1$ for every $v \in u(X)$ after $\O(n^3(\log n + \opt))$ iterations.
%
%By \cref{lemma::ingredientParateo1} we have $A_i \subseteq X_1$.
%Observe that $|X_1| \leq \opt$ as $\lpPrim(G[u(X)]) + |X_1| = LP(G)$.
%The algorithm \algGlobalSemoAlt calls with $1/3$ probability the mutation that flips every vertex $X_1$ with $1/2$ probability in $X$.
%That is, reaching a state $X'$ with $X'_1 = \bigcup_{i=1}^m A_i$ has probability $\Omega(2^{-\opt})$ and selecting $X$ in $\P$ has probability $\Omega(1/n^2)$.
%From this, in expectation it takes $\O\left(2^{\opt} n^2\right)$ iterations reaching $X'$ from $X$.
%\end{proof}


The question that remains is whether we keep $X'$ in the population once we find it.
This is where the uncovered-objective and the structural properties of minimal strictly reducible pairs come into play.

\def\xdiff{X_{\texttt{dif}}}
\def\vdiff{V_{\texttt{dif}}}

\begin{lemma}
\label{lemma::paretoOptiSol}
Let $X \in \{0,1\}^n$ and let $(A_1,B_1), \dots, (A_m,B_m)$ be a sequence of minimal strictly reducible pairs in $G$, such that $G - \bigcup_{i=1} A_i$ contains no strictly reducible pair.
If $X_1 = \bigcup_{i=1}^m A_i$, then $X$ is a pareto optimal solution.
\end{lemma}
\begin{proof}
Let $A = \bigcup_{i=1}^m A_i$ and $B = \bigcup_{i=1}^m B_i$.
We will prove that if there is a search point $X'$ that dominates $X$, then $G - A$ contains a minimal strictly reducible contradicting the precondition of the lemma.
Note that a minimal strictly reducible pair in $G - A$ have to be in $G[u(X)]$ as the other components in $G-A$ have size at most $W$.

If $X_1 = \bigcup_{i=1}^m A_i$, then by \cref{lemma::RedPairLPNoSequence} $\lpPrim(G) = \lpPrim(G[u(X)]) + |X_1|$.
That is, we might restrict to solutions $X' \in \{0,1\}^n$ with $\lpPrim(G) = \lpPrim(G[u(X')]) + |X'_1|$ as well as $|X'_1| = |X_1|$ and can focus on the objective $u(X)$ and $u(X')$, respectively.
Note that it cannot happen that $|X'_1| < |X_1|$ and $\lpPrim(G[u(X')]) \leq \lpPrim(G[u(X)])$ as $\lpPrim(G) = \lpPrim(G[u(X)]) + |X_1| > \lpPrim(G[u(X')]) + |X'_1|$ contradicting \cref{lemma::lowerBoundLP}.

W.l.o.g.~we can assume that every connected component of $G-X_1=G-A$ of size at most $W$ is also in $G[B]$, i.e., $V \setminus u(X) = A \cup B$.
Let $g_1 \colon \B_1 \times A_1 \to \N_0,\dots,g_m \colon \B_m \times A_m \to \N_0$ be the according assignment functions of $(A_1,B_1), \dots, (A_m,B_m)$ and let $g \colon \B \times A \to \N_0$ be defined as $g(a,C) = g_i(a,C)$ if $a \in A_i$, $C \in \B_i$ and otherwise $g(a,C) = 0$.
Suppose there is such an $X' \neq X$ as described above with $|u(X')| \leq |u(X)|$, or equivalently $|V \setminus u(X')| \geq |V \setminus u(X)|$.
Note that $|V \setminus u(X)| - |A| \geq |A|(2W-1)+1$, 
since for each $a \in A$ and at least for one $a' \in A$ we have $\sum_{C \in \B} g(C,a) \geq 2W-1$ and $\sum_{C \in \B} g(C,a') \geq 2W$.

%Let $(A,B)$ be a strictly reducible pair.
We define $V(\B_{\Tilde{A}}) \subseteq B$ for $\Tilde{A} \subseteq A$ as the vertices in the components $\{C \in \B \mid N(C) \cap \Tilde{A} \neq \varnothing\}$. 
Let $\xdiff = X'_1 \setminus  X_1 = X'_1 \setminus A$ and let $\vdiff = (V \setminus u(X')) \setminus (V \setminus u(X))$.
Note that $\vdiff \subseteq u(X)$ and $|\xdiff| = |A \setminus X'_1|$ by $|A| = |X'_1|$.
From $\lpPrim(G) = \lpPrim(G[u(X')]) + |X'_1|$, we obtain by \cref{lemma::ingredientParateo1} that $\xdiff \cap B = \varnothing$ and therefore $\xdiff \subseteq \vdiff$ as no vertex of $\xdiff$ is in $A \cup B = V \setminus u(X)$.
Thus, by the assignment function $g$ each vertex in $A' = A \setminus X'_1$ is in a connected component of size at least $W+1$ of $G[u(X')]$ and therefore $A'$ as well as $V(\B_{A'})$ are not in $V \setminus u(X')$.
Furthermore, for at least one $j \in [m]$ we have $A_j \subsetneq X'_1$ and by \cref{lemma::SizeNeoghborhood} we obtain that $|V(\B_{A_j \setminus X'_1})| \geq |A_j \setminus X'_1| (2W-1)+1$.
That is, to satisfy now $|V \setminus u(X')| \geq |V \setminus u(X)|$ we must have $|\vdiff| - |\xdiff| \geq |\xdiff|(2W-1)+1$ as at least $|A \setminus X'_1|(2W-1)+1+|A \setminus X'_1|$ vertices are in $V \setminus u(X)$ that are not in $V \setminus u(X')$.
Observe that each connected component $C$ of $G[\vdiff - \xdiff]$ satisfy $N(C) \cap \xdiff \neq \varnothing$ and $|C| \leq W$.
From this, we obtain that $G[\vdiff]$ contains a strictly reducible pair $(\hat{A},\hat{B})$ with $\hat{A} \subseteq \xdiff$ and $\hat{B} \subseteq \vdiff \setminus \xdiff$ by \cref{lemma::balancedExpansionStrictlyReducibleExist} and therefore also a minimal strictly reducible pair $(\hat{A}',\hat{B}')$ with $\hat{A}' \subseteq \hat{A}$ and $\hat{B}' \subseteq \hat{B}$.
In particular, $(\hat{A}',\hat{B}')$ is a minimal strictly reducible in $G-A$, since $\vdiff \subseteq u(X)$ and the connected components in $G[\vdiff \setminus \xdiff]$ exist identically in $G[u(X)]-\xdiff$.
\end{proof}
% The final proof of \cref{thm::OptSolf2} can be found in the appendix (Section \ref{appendix::f2}) and shows that 
We are ready for the final theorem of this section, which shows that Phase~\ref{part3} also works successfully.
\paragraph{Proof of \cref{thm::OptSolf2}:}
Let $(A_1,B_1),\dots,(A_m,B_m)$ be a sequence of minimal strictly reducible pairs, such that $G-\bigcup_{i=1}^m A_i$ contains no strictly reducible pair.
Furthermore, let $\P$ be a population with respect to $f_2$ in the algorithm \algGlobalSemoAlt. 
By \cref{lemma::XReducedInPop} we have a search point $X \in \P$ with $X_1 = \bigcup_{i=1}^m A_i$ after $\O\left(n^3(\log n + \opt) + 2^{\opt}\right)$ iterations  in expectation. %of \algGlobalSemoAlt using the fitness function $f_2$.
Moreover, $X$ is a pareto optimal solution by \cref{lemma::paretoOptiSol}.

Since $G[u(X)]$ contains no strictly reducible pair, we can derive from \cref{lemma::existenceReducible} that $|V(G[u(X)])| \leq 2 \cdot \opt \cdot W$. 
The algorithm \algGlobalSemoAlt calls with $1/3$ probability the mutation that flips every vertex $u(X)$ with $1/2$ probability in $X$.
That is, reaching a state $X'$ from $X$, such that $X'_1 = V^*$ has a probability of at least $\Omega\left(2^{-2 \cdot \opt \cdot W}\right)$, where  selecting $X'$ in $\P$ has probability $\Omega(1/n^2)$ (cf.~\cref{lemma::singleFlipAndBoundedPop}).
Thus, once $X$ is contained in $\P$ it takes in expectation $\O\left(n^2 \cdot 4^{\opt \cdot W} \right)$ iterations reaching $X'$.
As a result, the algorithm needs in total $\O\left(n^3(\log n + \opt) + n^2 \cdot 4^{\opt \cdot W} \right)$ iterations finding an optimal $W$-separator in expectation.