\documentclass{amsart}
\usepackage[utf8]{inputenc}

%%%%%%%%%%%%%%%%% PACKAGES %%%%%%%%%%%%%%%%%

\usepackage{amssymb}
\usepackage{tikz}
\usepackage{tikz-cd}
\usetikzlibrary{matrix,arrows,calc,decorations.pathmorphing,decorations.pathreplacing,positioning}
\usepackage{graphicx}
\usepackage{caption}
\usepackage{url}
\usepackage{color}
\usepackage[all]{xy}
\SelectTips{cm}{}

%%%%%%%%%%%%%%%%%% COMMANDS %%%%%%%%%%%%%%%%

\DeclareMathOperator{\Br}{\mathsf{Br}}
\DeclareMathOperator{\Up}{\mathsf{U}}
\DeclareMathOperator{\id}{\mathsf{id}}
\DeclareMathOperator{\G}{\mathsf{G}}
\def\Ofrak{\mathfrak{O}}

\def\op{\mathrm{op}}
\def\surj{\mathrm{surj}}
\def\inj{\mathrm{inj}}
\def\Fin{\mathrm{Fin}}
\def\Set{\mathrm{Set}}
\def\I{\mathbb{I}}

\def\NQCS{\mathbf{NQCS}}
\def\ActionOperads{\mathbb{AOp}}

\def\CS{\mathbf{CS}}
\def\preCS{\mathbf{preCS}}

 \newtheorem{thm}{Theorem}[section]
  \newtheorem{taggedtheoremx}{Theorem}
 \newenvironment{taggedtheorem}[1]
 {\renewcommand\thetaggedtheoremx{#1}\taggedtheoremx}
 {\endtaggedtheoremx}
 \newtheorem{cor}[thm]{Corollary}
 \newtheorem{lem}[thm]{Lemma}
 \newtheorem{prop}[thm]{Proposition}
 
 \theoremstyle{definition}
 \newtheorem{defn}[thm]{Definition}
   \newtheorem{rem}[thm]{Remark}
   \newtheorem{examp}[thm]{Example}

\newcommand{\ja}[1]{\color{blue} \{JA: #1\} \color{black}}
\newcommand{\fc}[1]{\color{red} \{FC: #1\} \color{black}}
\newcommand{\vc}[1]{\color{blue} \{VC: #1\} \color{black}}
\newcommand{\jg}[1]{\color{purple} \{JG: #1\} \color{black}}



\newcommand{\brbinom}[2]{\genfrac{[}{]}{0pt}{}{#1}{#2}}

\numberwithin{equation}{section}


\begin{document}


%%%%%%%%%%%%%%%%%%%%%%%%%%%%%%%%%%%%%%%%%%%%%

\title{Cloning systems and action operads}

\author[J. Aramayona]{Javier Aramayona}
\address{Javier Aramayona: Instituto de Ciencias Matem\'aticas (ICMAT). Nicol\'as Cabrera, 13--15. 28049, Madrid, Spain}
\thanks{J.A. was supported by grant CEX2019-000904-S funded by MCIN/AEI/ 10.13039/501100011033 and by grant PGC2018-101179-B-I00.  F.C. was supported by grant SI3/PJI/2021-00505 from Comunidad de Madrid. V.C. was partially supported by grants PID2020- 117971GB-C21 funded by MCIN/AEI/10.13039/501100011033, US-1263032 (US/JUNTA/FEDER, UE), P20 01109 (JUNTA/FEDER, UE)  and FPU17/01871. J.J.G. was supported by 
grants PID2020-117971GB-C22 and CEX2020-001084-M funded by MCIN/AEI/10.13039/501100011033 and grant 2021-SGR-00697 funded by the Catalan Government.}
\author[F. Cantero]{Federico Cantero Morán}
\address{Federico Cantero Morán: Departamento de Matemáticas, Universidad Autónoma de Madrid \& ICMAT. Calle Francisco Tomás y Valiente, 7. 28049, Madrid, Spain}
\author[V. Carmona]{V\'ictor Carmona} 
\address{Víctor Carmona: Departamento de  Álgebra \& IMUS, Universidad de Sevilla. Avda Reina Mercedes, s/n, 41012, Sevilla, Spain}
\author[J.J. Gutiérrez]{Javier J. Guti\'errez}
\address{Javier J. Guti\'errez: Departament de Matem\`atiques i Inform\`atica, Universitat de Barcelona. Gran Via de les Corts Catalanes 585. 08007, Barcelona, Spain}
\date{\today}


\begin{abstract}
Action operads and cloning systems are, respectively, the main ingredients in Thumann's and Witzel--Zaremsky's approaches for axiomatically constructing Thompson-like groups. 
In this paper, we prove that action operads are equivalent to cloning systems that admit a certain extra structure, and which we call {\em bilateral} cloning systems. In addition, we describe their relation with crossed interval groups and product categories. 
\end{abstract}



\maketitle


\section{Introduction}
The umbrella term {\em Thompson-like groups} makes reference to a vast family of groups that are in some way reminiscent of one of the classical groups $F$, $T$ and $V$ of R. Thompson \cite{CFP}. Apart from these three groups, prominent examples of Thompson-like groups include Stein's groups of PL homeomorphisms \cite{Ste}; Guba--Sapir's  diagram groups \cite{GuSa}; Brin's higher-dimensional Thompson groups \cite{Bri04}; Belk--Forrest's rearrangement groups of fractals \cite{BeFo}; the braided Thompson group of Brin \cite{Bri07} and Dehornoy \cite{Deh}; Wahl's  ribbon Thompson group \cite{Wahl}; the  {\em asymptotic mapping class groups} of surfaces and higher-dimensional manifolds \cite{FK04, FK08,FK09,AF20,ABF+21,GLU20}, etc. 




As may be appreciated from the above list of examples, Thompson-like groups arise in a variety of different ways.
With this motivation, the independent results of Witzel--Zaremsky \cite{WZ} and Thumann \cite{Th} offer unified frameworks for constructing Thompson-like groups. Witzel-Zaremsky achieve this in terms of {\em cloning systems}, which provide a recipe for ``twisting'' a direct limit of groups into a Thompson-like group. In turn, Thumann \cite{Th} uses the theory of {\em operads} in order to construct Thompson-like groups, which in this setting arise as the fundamental group of a certain category associated to an operad. 

The purpose of this paper is to establish a dictionary between cloning systems and a certain type of operad called {\em action operad} \cite{Zhang,Corner-Gurski,Yoshida,Yau}, which we consider without constants (see Section \ref{section:action_operads} for definitions). Our first result is: 

\begin{thm}
Every action operad gives rise to a cloning system.
\end{thm}



In fact, we precisely determine the extent to which a converse to the above theorem holds; more concretely, we will prove that every action operad comes from cloning systems that admit a certain extra structure, and  that we call {\em bilateral cloning systems}, see Section \ref{section:cloning_systems}. We stress that many of the known cloning systems are in fact bilateral; see Section \ref{section:cloning_systems} for examples and non-examples. In this language, our main results may be summarized as follows: 

\begin{thm} \label{thm:main2}
There is an explicit bijective correspondence between action operads and bilateral cloning systems.
\end{thm}


As will become apparent, our methods actually yield the equivalence of the categories of bilateral cloning systems
and  action operads, respectively. 


\medskip

\noindent {\bf Further results.} Action operads have been related to crossed interval groups in \cite{Yoshida}. In Section \ref{section:crossed}, we explicitly explain how crossed interval groups yield cloning systems, in terms of what we will call {\em inert crossed demi-interval groups} (see Section \ref{section:crossed}), and compare Yoshida's result to ours. Finally, in Section \ref{section:props} we introduce a third construction using PROs ({\em product categories}) which also yields bilateral cloning systems. The following diagram summarizes the relation between all of these results: 
$$
\begin{tikzcd}[ampersand replacement=\&]
   \begin{matrix}
       \text{action operads}
   \end{matrix}\ar[r,leftrightarrow]\ar[d] \& 
   \begin{matrix}
       \text{bilateral}\\
       \text{cloning systems}
   \end{matrix}\ar[d] \ar[r,leftrightarrow] \& 
   \text{cloning PROs}   \\
     \begin{matrix}
       \text{inert crossed}\\
       \text{demi-interval groups}
   \end{matrix}\ar[r]
   \& \text{cloning systems}
\end{tikzcd}
$$


\medskip

\noindent{\bf Future work.} This is the first of two papers devoted to the relation between operads and Thompson groups. In a forthcoming paper we will show that, if $A$ is an action operad and $C$ is its associated cloning system, then the fundamental group of a certain $A$\nobreakdash-operad is isomorphic to the Thompson group of the cloning system $C$.


\medskip

\noindent{\bf Plan of the paper.} In Section \ref{section:cloning_systems} we give a brief introduction to (bilateral) cloning systems, and give some examples. Section \ref{section:action_operads} offers an abridged overview of action operads. Section \ref{sec:braids} is devoted to the proof of Theorem \ref{thm:main2} in the special case of braid groups. These ideas are then generalized in Sections \ref{sec:operadtocloning} and \ref{sect:FromCStoActionOperads}. Finally, in Sections \ref{section:crossed} and \ref{section:props}, we will give further interpretations of our results in terms of crossed interval groups and PROs, respectively. 




\medskip

\noindent{\bf Acknowledgements.} This project started with some informal conversations at the {\em IX Encuentro de J\'ovenes Top\'ologos}, held in Seville in 2021. We are grateful to the organization for their hospitality and support. The second author thanks An\'ibal Medina for a very enlightning conversation about Joyal duality.  


\section{Cloning systems}\label{section:cloning_systems}
In this section we offer a brief introduction to Witzel--Zaremsky's  cloning systems \cite{WZ}, and define their {\em bilateral} counterparts; we refer the interested reader to \cite{WZ,Zaremsky} for a detailed account on cloning systems. 

We start with a specific example, which appears as Example 2.9 in \cite{WZ}, and that will serve to establish some notation for the sequel. In what follows, $\Sigma_n$ stands for the symmetric group on $n$ elements. 

\begin{examp}[Cloning system for symmetric groups] Let $\Sigma_{\bullet}=\{\Sigma_n\}_{n\ge 1}$ be the family of symmetric groups. For every $n\ge 1$, let $\lambda_{n}:\Sigma_n \to \Sigma_{n+1}$ be the injective homomorphism obtained by fixing the last element, that is,
$$
\lambda_n(\sigma)(i)=\sigma(i) \mbox{ for $1\le i\le n$ and } \lambda_n(\sigma)(n+1)=n+1,
$$
for every $\sigma\in\Sigma_n$. For every $n\ge 1$ and $1\le j\le n$, let $c^n_j: \Sigma_n\to \Sigma_{n+1}$ be the injective map given by, thinking about permutations pictorially as strand diagrams, ``repeating'' the $j$-th strand; that is, 
$$
c_j^n(\sigma)(i)=
\left\{
\begin{array}{ll}
\sigma(i) & \mbox{ if $i\le j$ and $\sigma(i)\le\sigma(j)$,} \\
\sigma(i)+1 & \mbox{if $i<j$ and $\sigma(i)>\sigma(j)$,}\\
\sigma(i-1) & \mbox{if $i>j+1$ and $\sigma(i-1)<\sigma(j)$,}\\
\sigma(i-1)+1 & \mbox{if $i\ge j+1$ and $\sigma(i-1)\ge \sigma(j)$,}
\end{array}
\right.
$$
for every $\sigma\in\Sigma_n$.
As shown in \cite[Example 2.9]{WZ} he families of morphisms $\lambda$ and $c$ interact with each other, and satisfy certain obvious compatibility properties, detailed in \cite[Proposition 2.6]{WZ}. The \emph{cloning system} for the family of symmetric groups is the triple $(\Sigma_\bullet, \lambda, c)$, subject to these compatibility conditions. 
\end{examp}

The maps $c_j$ above are called {\em cloning maps}, for obvious reasons; note that they are not group homomorphisms. The notion of a cloning system is a generalization of the above example to arbitrary families of groups. 


\begin{defn}\label{defn:CloningSystem}A \emph{cloning system} is a quadruple $(\G_{\bullet},\iota,\kappa,\pi)$, where
\begin{itemize}
    \item $\G_{\bullet}=\{\G_n\}_{n\geq 1}$ is a family of groups,
    \item $\iota=\{\iota_n\colon\G_n\to\G_{n+1}\}_{n\ge 1}$ is a family of injective homomorphisms,
    \item $\kappa=\{\kappa^n_{j}\colon\G_{n}\to\G_{n+1}\}_{n\ge 1,\,1\leq j\leq n}$ is a family of maps, called \emph{cloning maps}, and
    \item $\pi= \{\pi_n \colon \G_n\to \Sigma_n\}_{n\ge 1}$ is a  homomorphism,
\end{itemize}
 subject to the following compatibility conditions:
\begin{itemize}
    \item[(i)]\label{it:i} $\pi_{n+1} \circ \iota_n=\lambda_n \circ \pi_n$, for all $n\ge 1$;  
    \item[(ii)]\label{it:ii} $\pi_{n+1} \circ \kappa^n_j = c^n_j \circ \pi_n$, for all $n\ge 1$ and all $1\le j\le n$; 
    \item[(iii)]\label{it:iii} $\iota_{n+1}\circ\kappa^n_j= \kappa_j^{n+1}\circ \iota_n$, for all $n\ge 1$ and all $1\le j\le n$;
    \item[(iv)]\label{it:iv} $\kappa^{n+1}_{j+1} \circ \kappa_l^n = \kappa_l^{n+1} \circ \kappa_{j}^{n}$, for all $n$ and all $l< j\le n$;
    \item[(v)]\label{it:v} $\kappa^n_j(g\cdot h)=\kappa^n_{\pi_n(h)(j)}(g)\cdot\kappa^n_j(h)$, for all $n$, all $g,h \in \G_n$ and all $j\le n$.
\end{itemize}
\end{defn}
%\jg{He quitado la condición de que las $\kappa$'s sean inyectivas después de hablarlo con Federico. Aunque en el artículo del user's guide lo pide, en el Witzel-Zelemsky no lo hace y para la comparación con action operads y demás, es mejor que no lo sean.}
\begin{rem} It is important to note some differences between the Definition \ref{defn:CloningSystem} and the definition of cloning system of~\cite{WZ,Zaremsky}. First, we use a functional convention for composition of maps, that is, the composition of two functions $f\colon X\to Y$ and $g\colon Y\to Z$ is denoted by $g\circ f$, defined as $(g\circ f)(x)=g(f(x))$. Second, 
the original definition of cloning system in \cite{Zaremsky} requires injective homomorphisms $\iota_{n,m}\colon \G_{n}\to \G_{m}$ for every $m> n\geq 1$; however, our Definition \ref{defn:CloningSystem} implies that it suffices to consider injective maps $\iota_{n,n+1}=\iota_n$ for all $n\geq 1$.  
\end{rem}


We now introduce the notion of a \emph{bilateral} cloning system. In a nutshell, in the same way that the maps $\iota$ of Definition \ref{defn:CloningSystem} informally correspond to ``adding elements on the right'', a bilateral cloning system comes equipped with  a family of ``dual'' maps $\zeta$ that correspond to ``adding elements on the left''. 

We now proceed to formalize this idea. First, some notation: for the family of symmetric groups, we denote by $\rho_n: \Sigma_n \to \Sigma_{n+1}$ the injective homomorphism that fixes the first element, that is, 
$$
\rho_n(\sigma)(i)=\sigma(i-1)+1 \mbox{ for $2\le i\le n+1$ and } \rho_n(\sigma)(1)=1,
$$
for every $\sigma\in\Sigma_n$.
\begin{defn}\label{defn:NCCloningSystem} A \emph{bilateral cloning system} is a quintuple $(\G_{\bullet},\iota, \zeta, \kappa,\pi)$, where $(\G_{\bullet},\iota,\kappa, \pi)$ is a cloning system and $\zeta=\{\zeta_n\colon \G_n\to\G_{1+n}\}_{n\ge 1}$ is an additional family of injective homomomorphisms satisfying the following conditions:
\begin{itemize}
    \item[(i')] $\pi_{n+1} \circ \zeta_n=\rho_{n}\circ \pi_n $, for all $n\ge 1$;  
    \item[(iii')] \label{it:iiib} $\zeta_n\circ \kappa_j^n = \kappa_{j+1}^{n+1}\circ\zeta_{n}$, for all $n\ge 1$ and all $1\le j\le n$;
	\item[(vi)] $\zeta_{n+1}\circ \iota_n = \iota_{n+1}\circ\zeta_{n}$, for all $n\ge 1$;
	\item[(vii)]\label{it:iii'} $\iota_{n+1}\circ \iota_n = \kappa_{n+1}^{n+1}\circ \iota_n$, for all $n\ge 1$;
 \item[(vii')]\label{it:iiib'} $\zeta_{n+1}\circ \zeta_n = \kappa_{1}^{n+1}\circ \zeta_n$, for all $n\ge 1$;
   	\item[(iv+)] $\kappa^{n+1}_{j+1}\circ \kappa^n_{j} = \kappa_{j}^{n+1}\circ \kappa_j^n$, for all $1\le j\le n$;
	\item[(viii)]$\kappa_i^n(m)(g)\cdot \nu_i^m(n)(h) = \nu^m_{\pi(g)(i)}(n)(h)\cdot \kappa^n_i(m)(g)$, for all $m,n\geq 0$ and all $g\in \G_n$ and $h\in\G_m$;
	\item[(ix)] $\iota_n(m)(g)\cdot \zeta_m(n)(h)=\zeta_m(n)(h)\cdot \iota_n(m)(g)$, for all $m,n\geq 0$ and all $g\in\G_n$ and $h\in\G_m$.
\end{itemize}
\end{defn}

The morphisms $\iota_n(m)$, $\zeta_m(n)$, $\kappa_j^n(m)$ and $\nu_j^m(n)$ appearing in conditions (viii) and (ix) are the maps
$$
\iota_n(r)\colon\G_{n}\longrightarrow\G_{n+r},\quad \kappa^n_j(m)\colon \G_{n}\longrightarrow\G_{n+m}, \quad
\zeta_n(l)\colon\G_{n}\longrightarrow\G_{l+n}, 
$$
 and 
$$
\nu^n_j(m)\colon\G_{n}\to\G_{n+m-1}
$$
defined by  $\iota_n(r)=\iota_{n+r-1}\circ\cdots\circ\iota_{n}$ for all $n,r\ge 1$; $\kappa_j^n(m)=\kappa_j^{n+m-2}\circ\cdots\circ \kappa_j^{n}$ for all $n,m\ge 1$ and all $1\le j\le n$; $\zeta_n(l)=\zeta_{n+l-1}\circ\cdots\circ\zeta_{n}$ for all $n,l\ge 1$; and  $\nu_j^n(m)=\zeta_{m+n-j}(j-1)\circ\iota_n(m-j)=\iota_{n+j-1}(m-j)\circ\zeta_n(j-1)$ for all $j,m,n\ge 1$. Note that the morphisms $\zeta_n(l)$ and $\nu_j^n(m)$ only make sense for bilateral cloning systems.


\begin{rem}
Although in Definition~\ref{defn:CloningSystem} and Definition~\ref{defn:NCCloningSystem} we require  the maps  $\iota$ and $\zeta$ to be injective, this  is not essential for the comparison of bilateral cloning system and action operads, as we will see in the next section. 
\end{rem}

In order to get a grip on the intuition behind the definitions above, we next describe perhaps the primordial example of a (bilateral) cloning system, namely that of braid groups; we refer the reader to \cite{WZ} for details.   

\begin{examp}[Braid groups]\label{examp:cloning}
    
 Let $\Br_n$ denote the braid group on $n$ strands. For every $n\ge 1$ there is a canonical surjective group homomorphism $\pi_n\colon\Br_n\to\Sigma_n$ by sending each braid to its underlying permutation. We also have inclusion maps $\iota_n\colon \Br_n\to\Br_{n+1}$ corresponding to ``adding one strand on the right'', and cloning maps
$$
\kappa_{j}^n\colon \Br_n\longrightarrow \Br_{n+1}
$$
given by duplicating the $j$-th strand into two parallel strands; see Figure \ref{fig:Kappa Braid} for an example, and \cite[Example~3.3]{Zaremsky} for details.
\begin{figure}[htp]
    \centering
    \includegraphics[width=6cm]{kappa_Br.pdf}
    \caption{Cloning map $\kappa_2^4$ on $\Br_4$.}
    \label{fig:Kappa Braid}
    \end{figure}

Together, the three families of maps defined above endow the collection of all braid groups $\Br_{\bullet}=\{\Br_n\}_{n\ge 1}$ with the cloning system structure $\Br=(\Br_{\bullet},\iota,\kappa,\pi)$, see \cite{WZ} for a proof. 

Moreover, in analogy with the maps $\iota_n$, we can also define inclusion maps $\zeta_n\colon \Br_n\to\Br_{n+1}$ that informally correspond to ``adding one strand on the left''. Equipped with these maps, one readily checks that $\Br=(\Br_{\bullet},\iota,\zeta,\kappa,\pi)$ is a bilateral cloning system. For illustrative purposes, Figures \ref{fig:AxiomVI} to \ref{fig:AxiomXI} depict particular instances of some of the conditions of the bilateral cloning system structure of $\Br_{\bullet}$.
\end{examp}  
 \begin{figure}
 \centering
 \includegraphics[width=7cm]{NonChiral_additional_axioms_vi.pdf}
    \caption{Condition (vii), $\kappa_3^3\circ \iota=\iota\circ\iota$.}
    \label{fig:AxiomVI}
    \end{figure}
 \begin{figure}
    \centering
    \includegraphics[width=7cm]{NonChiral_additional_axioms_vii.pdf}
    \caption{Condition (vi), $\iota\circ \zeta=\zeta\circ \iota$.}
    \label{fig:AxiomVIII}
    \end{figure}
\begin{figure}
    \centering
    \includegraphics[width=10cm]{NonChiral_additional_axioms_x.pdf}
    \caption{Condition (viii).}
    \label{fig:AxiomX}
    \end{figure}
 \begin{figure}
    \centering
    \includegraphics[width=10cm]{NonChiral_additional_axioms_xi.pdf}
    \caption{Condition (ix).}
    \label{fig:AxiomXI}
    \end{figure}

Other examples of bilateral cloning systems are the {\em mock symmetric groups} and the {\em loop braid groups} (also known as {\em symmetric automorphisms of free groups}); see \cite{WZ}; as well as the {\em signed symmetric groups} and the {\em twisted braid groups} \cite{Zaremsky}. 

\medskip

Next, we discuss a simple example of a cloning system from \cite{WZ} which, as we will see, is not bilateral.

\begin{examp}[Direct powers]\label{examp:DirectPowers}
Let $G$ be a group and denote by $G^n$ the $n$-fold direct product of $G$ with itself. Write  $\iota_n: G^n \to G^{n+1}$ for the map that adds the identity (in $G$) as last entry, let $\pi_n: G^n \to \Sigma_n$ the trivial homomorphism, and consider the map $\kappa^n_j:G^n \to G^{n+1}$ that duplicates the $j$-th entry. Then, the quadruple $(\{G^n\}_{n\ge 1}, \iota, \pi, \kappa)$ is  a cloning system on the set of direct powers of $G$. 

Despite the fact that there is an obvious map $\zeta_n: G^n \to G^{n+1}$ that adds the identity (in $G$) as first entry, one may check that the maps $\kappa$ and $\zeta$ do not satisfy condition (viii) of the definition of a bilateral cloning system.
\end{examp}

Finally, we discuss another example of a cloning system introduced in \cite{Zaremsky}, which again  will not be bilateral.


\begin{examp}[Upper triangular matrices]\label{examp:UpperTriangularMatrices}
Let $U_n$ denote the group of invertible $n\times n$ upper triangular matrices with real coefficients. Consider the obvious inclusion map $\iota_n: U_n \to U_{n+1}$ given by adding a 1 as the lowermost element on the diagonal. 

There are cloning maps $\kappa_j^n: U_n \to U_{n+1}$ that informally correspond to a certain duplication of the $j$-th column that preserves the upper triangular structure of the matrix, and which becomes apparent just by giving the following particular example; see \cite{Zaremsky} for details: 

\[
\kappa^3_2\begin{pmatrix}
1 & 2 & 3 \\ 0 & 4 & 5 \\ 0 & 0 & 6
\end{pmatrix}  = 
\begin{pmatrix}
1 & 2 & 2 & 3 \\ 0 & 4 & 0 & 0 \\ 0 & 0 & 4 & 5 \\ 0 & 0 & 0 & 6
\end{pmatrix}
\]
Setting $\pi_n: U_n \to \Sigma_n$ to be the trivial homomorphism, the set $U_\bullet = \{U_n\}_{n \ge 1}$ acquires the cloning system structure $U = (U_\bullet, \iota, \kappa, \pi)$. 

Observe that one could define another inclusion map $\zeta_n: U_n \to U_{n+1}$ by adding a 1 as the uppermost element of the diagonal; however, as in the previous example, the interaction of the maps $\zeta$ and $\kappa$ does not satisfy condition (viii) of the definition of a bilateral cloning system.
\end{examp}





 

\section{Action operads}\label{section:action_operads}
We start this section recalling the classical notion of an {\em operad}: 




\begin{defn}\label{defn:Operad} 
A \emph{symmetric operad} $\mathsf{O}$ on sets is a triple $(\mathsf{O}, \{\circ_i\}_i, \id)$, where
\begin{itemize}
\item $\mathsf{O}=\{O(n)\}_{n\ge 1}$ is a family of sets, and each $O(n)$ is equipped with a right $\Sigma_n$-action, for every $n\ge 1$,
\item ${\id}\in O(1)$ is called the \emph{unit} of the operad, and \item $\{\circ_i\}_i$ is a family of maps
$$
\circ_i\colon O(n)\times O(m)\to O(n+m-1) \quad\mbox{for all $n,m\ge 1$ and $1\le i\le n$},
$$ called \emph{$\circ_i$-operations} or \emph{partial composition products},
such that ${\id} \circ_i y=y$ and $x\circ_i {\id}=x$ for every $x,y\in O(n)$. Moreover, these $\circ_i$\nobreakdash-operations satisfy certain associativity and equivariance axioms, which are not relevant to the discussion here; see, for example, \cite[Definition 11]{Markl} for details.
\end{itemize}


A \emph{morphism of operads} $f\colon\mathsf{O}\to \mathsf{P}$ consists of maps $f_n\colon O(n)\to P(n)$ for $n\ge 1$, that are compatible with the unit and $\circ_i$-operations of $\mathsf{O}$ and $\mathsf{P}$. If we forget about all the symmetric group actions on the sets $O(n)$, we have the notion of a \emph{non-symmetric operad}.
\end{defn}

\begin{rem} Note that Definition \ref{defn:Operad} avoids nullary operations in an operad, i.e.\ there is no $O(0)$ in $\mathsf{O}$. In this work, we will only consider operads without nullary operations, or in other words, without constants. This choice is not essential, but it is made to have a clearer connection between action operads and cloning systems.
\end{rem}


\begin{examp}[Symmetric groups]
The family of symmetric groups $\Sigma_{\bullet}=\{\Sigma_n\}_{n\ge 1}$ is an operad in sets, with $\id\in\Sigma_1$ the trivial permutation of $\Sigma_1$. The partial composition products $\circ_i\colon\Sigma_n\times \Sigma_m\to \Sigma_{m+n-1}$ are defined as follows: if $\sigma\in \Sigma_n$ and $\tau\in\Sigma_m$, then $\sigma\circ_i\tau$ is the permutation of $\Sigma_{m+n-1}$ obtained by ``inserting'' $\tau$ in $\sigma$ at the $i$th place as a block and rearranging the indices accordingly. Figure~\ref{fig:CompositionProduct Sigma} shows an example of the composition product $\circ_2\colon\Sigma_3\times \Sigma_2\to\Sigma_4$.
\begin{figure}[h]
    \centering
    \includegraphics[width=5cm]{CompositionProduct_Sigma.pdf}
    \caption{\scriptsize
    $\left(\begin{matrix}
     1 & 2 & 3\\
     2 & 3 & 1
    \end{matrix}\right)\circ_2
    \left(\begin{matrix}
     1 & 2 \\
     2 & 1
    \end{matrix}\right)
    = 
    \left(\begin{matrix}
     1 & 2 & 3 & 4\\
     2 & 4 & 3 & 1
    \end{matrix}\right)$.}
    \label{fig:CompositionProduct Sigma}
    \end{figure}
\end{examp}
We now introduce the concept of an {\em action operads}, which essentially is a family of groups satisfying certain properties that allow to define operads with equivariance relative to this family; the reader is referred to \cite{Corner-Gurski} for a detailed treatment of action operads and their properties. They have been also studied under the name \emph{group operads} \cite{Zhang, Yoshida, Yau}

\begin{defn}\label{defn:ActionOperad} An \emph{action operad without constants}, or simply an \emph{action operad}, is a quadruple $(\G_{\bullet},\pi,\{\circ_i\}_i,\id)$, where
\begin{itemize}
    \item $\G_\bullet=\{\G_n\}_{n\ge 1}$ is a family of groups, and $(\G_{\bullet},\{\circ_i\}_i,\id)$ is a non-symmetric operad on sets, where $\{\circ_i\}_i$ are the partial composition products of the operad and $\id\in \G_1$ is the unit. The associativity of the partial composition products yields that if $f\in \G_n$, $g\in \G_m$ and $h\in \G_l$, then we have that
\begin{align*}
(f\circ_i g)\circ_j h &= \begin{cases}
(f\circ_j h) \circ_{i+l-1}g & \text{if $j<i$},\\
 (f \circ_{j-m+1} h)\circ_i g& \text{if $j\geq i+m$},\\
f\circ_i (g\circ_{j-i+1}h) & \text{if $j=i,\ldots, i+m-1$},
\end{cases} \\
f\circ_i \id &= f, \\
\id\circ_1 g &= g;
\end{align*}
    \item $\pi: \G_\bullet \to \Sigma_{\bullet}$ is a map of operads which is also a levelwise group homomorphism, that is, $\pi_n\colon\G_n \to \Sigma_n$ is a homomorphism for all $n\ge 1$;
\item For every $f, f' \in \G_n$, $g,g'\in \G_{m}$, we have that
\begin{align*}
(f\cdot f')\circ_i(g\cdot g') &= (f\circ_{\pi_n(f')(i)}g)\cdot (f'\circ_i g'),
\end{align*}
with the multiplication taking place in the group $\G_{n+m-1}$.
\end{itemize}
\label{def:actionoperad}
\end{defn}


 Note that the partial composition products $\circ_i$ are not group homomorphisms in general, that action operads are not assumed to be symmetric operads, and that they have no nullary operations. It follows from the axioms that the unit element $\id\in \G_1$ of the operad $\G_{\bullet}$ is precisely the unit $e_1$ of the group $\G_1$. 

 Observe that, both in a cloning system and in an action operad the group, $\G_n$ acts on the set $\{1,\ldots,n\}$ via the homomorphism $\pi_n$ for every $n$. If these maps $\pi$ are understood from the context, for $g\in \G_n$ we will write $g(i)$ instead of $\pi_n(g)(i)$.


Finally, note that an action operad with trivial $\pi$ is the same thing as a non-symmetric operad on  groups.    


\section{Proof of Theorem \ref{thm:main2} for braid groups}
\label{sec:braids}
Before we embark on the proof of our main theorem in its full generality, we explain the situation with the special case of braid groups. As will become apparent, the general argument is an abstraction of the ideas introduced here, and will be treated in the next two sections.

Recall from Example~\ref{examp:cloning} the (bilateral) cloning system $\Br=(\Br_{\bullet},\iota,\zeta,\kappa,\pi)$ for braid groups, where $\pi$ is the canonical projection onto the set of symmetric groups, the maps $\iota$ and $\zeta$ add one strand on the right or the left, respectively, and $\kappa$ are the cloning maps given by duplicating a strand.

$\Br=\big(\Br_{\bullet},\pi\big)$ be the collection of all braid groups equipped with the canonical projection homomorphisms $\pi_n\colon\Br_{n}\to \Sigma_n$. 

Note that the collection $\Br_{\bullet}$ can be endowed with an action operad structure with respect to the ``substitution maps'' $\circ_{j}\colon \Br_{n+1}\times\Br_m\to \Br_{n+m}$, which correspond to replacing the $j$-th strand of the first braid by the second braid. Figure  \ref{fig:CloningToOperad} contains a depiction of this operation; for a more detailed discussion, see \cite[Definition 1.6 and Example 1.12(2)]{Corner-Gurski} for details.


We now explain how to obtain one structure from the other, illustrating the main ideas of the procedure with several pictures.

\subsection{From the action operad to the (bilateral) cloning system.}
The morphisms $\iota$ and the cloning maps $\kappa$ for the cloning system are obtained by using the identity $e_2\in \Br_2$ and the composition products $\circ_i$, as explained in Figures~\ref{fig:OperadToIota} and~\ref{fig:OperadToKappa}, respectively. Thus one obtains the cloning system $\Br$ described in Example~\ref{examp:cloning}. 
\begin{figure}
    \centering
    \includegraphics[width=10cm]{Operad_to_iota.pdf}
    \caption{Constructing the map $\iota$. Insert the given braid in the first strand of $e_2$.}
    \label{fig:OperadToIota}
    \end{figure} 
        
    \begin{figure}
    \centering
    \includegraphics[width=10cm]{Operad_to_kappa.pdf}
    \caption{Constructing the map $\kappa_{j}$. Replace the $j$th strand of the given braid by $e_2$.}
    \label{fig:OperadToKappa}
    \end{figure}
In order to get a bilateral cloning system, the maps $\zeta$ are defined similarly to the maps $\iota$, but replacing the second strand of $e_2$ instead of the first one, that is, using $\circ_2$ instead of $\circ_1$; see Figure~\ref{fig:OperadToZeta}.
\begin{figure}
    \centering
    \includegraphics[width=10cm]{Operad_to_zeta.pdf}
    \caption{Constructing the map $\zeta$. Insert the given braid in the second strand of $e_2$.}
    \label{fig:OperadToZeta}
    \end{figure}    

\subsection{From the bilateral cloning system to the action operad} We now explain how to use the bilateral cloning system structure on $\Br$ in order to build an action operad structure on $\Br_{\bullet}$. To see how the composition product maps $\circ_j$ are obtained, we proceed as follows. First, we replicate $m$ times the $j$th strand of the first braid using the cloning maps, where $m$ is the number of strands of the second braid. Then we add, by using $\iota$ and $\zeta$ strands to the right and left, respectively, of the second braid, so that it has the same number of strands as the cloned braid. Finally, we just multiply the two braids obtained this way. Figure~\ref{fig:CloningToOperad} shows an example of this construction. 
 \begin{figure}
    \centering
    \includegraphics[width=10cm]{Cloning_to_Operad.pdf}
    \caption{Construction the maps $\circ_j$ from the cloning system data.}
    \label{fig:CloningToOperad}
    \end{figure}

One can check that the maps $\circ_j$ built above satisfy all the properties required for  endowing $\Br_{\bullet}$ with an operad structure in sets. For instance, Figure~\ref{fig:AssociativityI} and Figure~\ref{fig:AssociativityII} show the  verification of the associativity axiom. 
 \begin{figure}
    \centering
    \includegraphics[width=7cm]{Associativity_of_operad_I.pdf}
    \caption{First half of the associativity axiom. The colors represent the application of $\iota$, $\zeta$ and $\kappa$ in each step.}
    \label{fig:AssociativityI}
    \end{figure}
    \begin{figure}
    \centering
    \includegraphics[width=7cm]{Associativity_of_operad_II.pdf}
    \caption{Second half of the associativity axiom.
    }
    \label{fig:AssociativityII}
    \end{figure}

Finally, a simple computation, depicted in Figure~\ref{fig:Compatibility}, shows that the maps $\circ_j$ are compatible with the underlying group structure of the braid groups, thus proving that $(\Br_{\bullet},\pi, \{\circ_i\}_i, e_1)$ is indeed an action operad. 
 \begin{figure}
    \centering
    \includegraphics[width=7cm]{Compatibility_of_partial_products.pdf}
    \caption{Compatibility axiom of the action operad.}
    \label{fig:Compatibility}
    \end{figure}

\section{From action operads to cloning systems}
\label{sec:operadtocloning}
In this section, we prove one of the implications of Theorem \ref{thm:main2} in  full generality; more concretely, we explain how to obtain a (bilateral) cloning system from an barbitrary action operad.

Let $\G=(\G_{\bullet},\pi, \{\circ_i\}_i, \id)$ be an action operad, and
%which, recall from Definition~\ref{def:actionoperad}, is a collection of groups $\{\G_n\}_{n\ge 1}$ equipped with homomorphisms $\pi_n\colon\G_n\to\Sigma_n$ and composition products  $(\circ_j\colon \G_{n+1}\times\G_{m}\to \G_{n+m})_{j,n,m}$ satisfying certain compatibility relations.
 denote by $e_n\in \G_n$ the identity element of  $G_n$. If $g\in G_n$ and $i\in \{1,\ldots,n\}$, we denote by $g(i) = \pi(g)(i)$. From the action operad structure, one can define the following maps
$$
\kappa^n_j,\iota_n,\zeta_n\colon \G_n\longrightarrow \G_{n+1}
$$
by setting
$$
\kappa^n_j(g) = g\circ_j e_2,\quad
\zeta_n(g) = e_2\circ_2 g\quad\mbox{ and }\quad
\iota_n(g) = e_2\circ_1 g.
$$
The following observation will be useful in what follows:
\begin{rem}\label{rem:properties} Note that in an action operad we always have that $e_n\circ_i e_m = e_{n+m-1}$ because $e_n\circ_i e_m = (e_n\cdot e_n) \circ_i (e_m\cdot e_m) = (e_n\circ_i e_m)\cdot (e_n\circ_i e_m)$. Moreover, the following hold
$$
    \nu_j^n(m)(g) = e_m\circ_j g\quad\mbox{ and }\quad
    \kappa_j^n(m)(g) = g\circ_j e_m,
$$
where $\kappa_j^n(m)$ and $\nu_j^n(m)$ are defined as in Definition~\ref{defn:NCCloningSystem}.
\end{rem}

\begin{prop}\label{prop:OperadtoCS} 
Let $\G=(\G_{\bullet},\pi, \{\circ_i\}_i, \id)$ be an action operad. Then, the quintuple $(\G_{\bullet}, \iota, \zeta, \kappa,\pi)$, with $\iota$, $\zeta$ and $\kappa$ as defined above, is a bilateral cloning system. 
\end{prop}
\begin{proof}
We will prove it in several steps. We will make essential use of  the fact that $\G$ is an action operad, the definition of the morphisms $\iota$, $\zeta$ and $\kappa$, and Remark~\ref{rem:properties}. First, we prove that the maps $\iota$ and $\zeta$ are homomorphisms; indeed,
\[\iota(f\cdot g) = e_2\circ_1 (f\cdot g) =
(e_2\cdot e_2)\circ_1(f\cdot g)=
(e_2\circ_{1} f)\cdot (e_2\circ_1 g) = \iota(f)\cdot \iota(g),\]
\[\zeta(f\cdot g) = e_2\circ_2 (f\cdot g) =
(e_2\cdot e_2)\circ_2(f\cdot g)=
(e_2\circ_{2} f)\cdot (e_2\circ_2 g) = \zeta(f)\cdot \zeta(g).\]
We now show that the quadruple $(\G_\bullet,\iota,\kappa,\pi)$ is a cloning system by checking that it satisfies properties (i)--(v) of Definition~\ref{defn:CloningSystem}.
\begin{itemize}
\item[(i)] $\pi_{n+1}\circ \iota_n = \lambda_n\circ\pi_n$. For every $g\in \G_n$ we have that
\begin{align*}
\pi_{n+1}(\iota_n(g)) = \pi_{n+1}(e_2\circ_1 g) = \pi_{2}(e_2)\circ_1 \pi_{n}(g) = e_2\circ_1 \pi_{n}(g) = \lambda_n(\pi_n(g)).
\end{align*}
\item[(ii)] $\pi_{n+1}\circ \kappa^n_j = c_j^n\circ\pi_n$. For every $g\in\G_n$ and all $1\le j\le n$ we have that
\begin{align*}
\pi_{n+1}(\kappa^n_j (g)) = \pi_{n+1}(g\circ_j e_2) = \pi_{n}(g)\circ_j \pi_{2}(e_2) = \pi_{n}(g)\circ_j e_2 = c^n_j(\pi_n(g)).
\end{align*}
\item[(iii)] $\iota_{n+1}\circ \kappa_j^n = \kappa^{n+1}_j\circ\iota_n$. For every $g\in\G_n$ and all $1\le j\le n$ we have that
\begin{align*}
\iota_{n+1}\circ \kappa_j^n(g) &= \iota_{n+1}(g\circ_j e_2) = e_2\circ_1(g\circ_j e_2)  \\&=  ( e_2\circ_1 g)\circ_j e_2 =  \iota_n(g)\circ_j e_2 = \kappa^{n+1}_j(\iota_n(g)).\end{align*}
\item[(iv)] $\kappa^{n+1}_l\circ\kappa_j^n = \kappa_{j+1}^{n+1}\circ \kappa_l^n$. For every $g\in\G_n$ and all $l< j\le n$ we have that
\begin{align*}
\kappa^{n+1}_l(\kappa^n_j(g)) &= \kappa^{n+1}_l(g\circ_j e_2) = (g\circ_j e_2) \circ_l e_2 \\
&=(g\circ_l e_2) \circ_{j+1} e_2 = \kappa^n_l(g) \circ_{j+1} e_2 = \kappa^{n+1}_{j+1}(\kappa^n_l(g)).
\end{align*}
\item[(v)] $\kappa_j^n(g\cdot h) = \kappa^n_{h(j)}(g)\cdot \kappa_j^n(h)$ for every $g,h \in \G_n$. For all $j\le n$ we have that
\begin{align*}
\kappa_j^n(g\cdot h) &= (g\cdot h)\circ_j e_2 = (g\cdot h)\circ_j (e_2\cdot e_2) \\
&= (g\circ_{h(j)} e_2)\cdot (h\circ_j e_2)
= \kappa^n_{h(j)}(g)\cdot \kappa_j^n(h)
\end{align*}
\end{itemize}
Finally, we prove that $(\G_{\bullet}, \iota, \zeta, \kappa,\pi)$ is a bilateral cloning system by checking properties (i'), (iii'), (vi), (vii), (vii'), (iv+), (ix), (x) of Definition~\ref{defn:NCCloningSystem}. 
First, properties (i'), (iii') and (vii') are proved as properties (i), (iii) and (vii) simply by replacing $\circ_1$ by $\circ_2$ and $\iota$ by $\zeta$.
\begin{itemize}
\item[(vi)] $\zeta_{n+1}\circ \iota_n = \iota_{n+1}\circ\zeta_{n}$. For every $g\in \G_n$ and all $n\ge 1$ we have that
\begin{align*}
    \zeta_{n+1}(\iota_n(g)) &= e_2\circ_2(e_2\circ_1 g) = (e_2\circ_2 e_2)\circ_2 g = e_3\circ_2 g \\&= (e_2\circ_1 e_2)\circ_2 g = e_2\circ_1 (e_2\circ_2 g) = \iota_{n+1}(\zeta_{n}(g)).
\end{align*}
\item[(vii)] $\iota_{n+1}\circ\iota_n = \kappa^{n+1}_{n+1}\circ \iota_n$. For every $g\in \G_n$ and all $n\ge 1$ we have that
\begin{align*}
\iota_{n+1}(\iota_{n}(g)) &= e_2\circ_1(e_2\circ_1 g) = (e_2\circ_1 e_2)\circ_1 g = e_3\circ_1 g, \\
\kappa^{n+1}_{n+1}(\iota_n(g)) &= (e_2\circ_1 g)\circ_{n+1} e_2 = (e_2\circ_2 e_2)\circ_1 g = e_3\circ_1 g.
\end{align*}
\item[(iv+)] $\kappa^{n+1}_{j+1}\circ \kappa^n_{j} = \kappa_{j}^{n+1}\circ \kappa_j^n$. For all $g\in \G_n$ and all $1\le j\le n$ we have that 
\begin{align*}
\kappa^{n+1}_{j+1}(\kappa^n_{j}(g)) &=
(g\circ_j e_2)\circ_{j+1} e_2 = g\circ_j(e_2\circ_2 e_2) = g\circ_2 e_3  \\ &= g\circ_j (e_2\circ_1 e_2) = (g\circ_j e_2)\circ_j e_2 = \kappa_{j}^{n+1}(\kappa_j^n(g)).
\end{align*}
\item[(viii)] $\kappa_j^n(m)(g)\cdot \nu^m_j(n)(h) = \nu^m_{g(j)}(n)(h) \cdot \kappa_j^n(m)(g)$ for every $g\in\G_n$ and $h\in\G_m$. For all $m,n\geq 0$ we have that
\begin{align*}
\kappa_j^n(m)(g)\cdot \nu^m_j(n)(h) &=
(g\circ_j e_m)\cdot (e_n\circ_j h) \\
&= (g\cdot e_n)\circ_{j} (e_m\cdot h) \\
&= (e_n\cdot g)\circ_{j} (h\cdot e_m) \\
&= (e_n\circ_{g(j)} h)\cdot (g\circ_{j} e_m) \\
&= \nu^m_{g(j)}(n)(h)\cdot \kappa_{j}^n(m)(g).
\end{align*}

\item[(ix)] $\iota_n(m)(g)\cdot \zeta_m(n)(h) = \zeta_m(n)(h)\cdot \iota_n(m)(g)$ for every $g\in\G_n$ and $h\in\G_m$. For all $m,n\ge 0$ we have that
\begin{align*}
    \iota_n(m)(g)\cdot \zeta_m(n)(h) &= (e_{m+1}\circ_{1} g)\cdot (e_{n+1}\circ_{n+1} h) \\
    &= (e_{m+1}\circ_1 g)\cdot((e_2\circ_1 e_n)\circ_{n+1} h) \\
    &= (e_{m+1}\circ_1 g)\cdot((e_2\circ_2 h)\circ_1 e_n) \\
    &= (e_{m+1}\cdot (e_2\circ_2 h))\circ_1 (g\cdot e_n) \\
    &= ((e_2\circ_2 h)\cdot e_{m+1})\circ_1 (e_n\cdot g) \\
    &= ((e_2\circ_2 h)\cdot (e_{2}\circ_2 e_m))\circ_1 (e_n\cdot g) \\
    &= ((e_2\circ_2 h)\circ_1 e_n)\cdot ((e_2\circ_2e_m)\circ_1 g) \\
    &= ((e_2\circ_1 e_n)\circ_{n+1} h)\cdot ((e_{m+1})\circ_1 g) \\
    &= \zeta_m(n)(h)\cdot \iota_n(m)(g).
\end{align*}
\end{itemize}
Therefore, $(\G_{\bullet}, \iota, \zeta, \kappa,\pi)$ is a bilateral cloning system as we wanted to show.
\end{proof}


\section{From cloning systems to action operads}\label{sect:FromCStoActionOperads}
We now explain how to construct an action operad from a bilateral cloning system. Let $\G=(\G_{\bullet},\iota,\zeta,\kappa,\pi)$ be a bilateral cloning system. The following identities can be derived from the identites in the definition of a bilateral cloning system (recall from Definition~\ref{defn:NCCloningSystem} the construction of the maps $\kappa_i^n(m)$ and $\nu_i^n(m)$):

\[
\kappa_i^{n+m-1}(l)\circ \kappa_j^n(m)=
\begin{cases}
    \kappa_{j}^n(m+l-1) & \text{if $j\leq i < j+m$}, \\
    \kappa_{j+l-1}^{n+l-1}(m)\circ\kappa_i^n(l) & \text{if $i<j$}.
\end{cases}
\]
\[
    \kappa_j^{n+m-1}(l)\circ \nu^n_i(m) = \begin{cases}
		\nu^{n+l-1}_i(m)\circ \kappa_{j-i+1}^n(l) & \text{if $i\leq j< i+n$},\\
		\nu^{n}_{i+l-1}(m+l-1) & \text{if $j< i$}, \\
		\nu^{n}_i(m+l-1) & \text{if $j\geq i+n$},
		\end{cases}
  \]
\begin{align*}
    \nu_j^{n+m-1}(l)\circ \nu_i^n(m)& =\nu_{j+i-1}^n(m+l-1),
\end{align*}
\begin{align*}
    \kappa_j^n(m)(f\cdot g) &= \kappa_{g(j)}^n(m)(f)\cdot \kappa_j^n(m)(g) &\mbox{if } 1\leq j\leq n,
\end{align*}
\begin{multline*}
        \nu_j^l(n+m-1)(g)\cdot \nu_{i+l-1}^m(n+l-1)(h) \\ =  \nu_{i+l-1}^m(n+l-1)(h)\cdot \nu_j^l(n+m-1)(g)\qquad \mbox{if $j<i$}
\end{multline*}
Note that $\nu^n_i(m)(g)$ acts trivially on all $j<i$ and all $j\geq i+m$ and that $\kappa^n_i(m)(e_n) = e_{n+m-1}$, because
\[\kappa^n_i(e_n) = \kappa^n_i(e_n\cdot e_n) = \kappa^n_i(e_n)\cdot \kappa^n_i(e_n)\]
and therefore $\kappa^n_i(e_n) = e_{n+1}$. 

In order to define an action operad structure on $\G_{\bullet}$ and since we already have a map $\pi\colon \G_{\bullet}\to \Sigma_{\bullet}$, it is enough to define the partial composition products $\{\circ_i\}_i$, which we do it as follows. The map
\[
\circ_i\colon \G_n\times \G_m\longrightarrow \G_{n+m-1}
\]
is the following composition
\[\xymatrix{
\G_{n}\times \G_{m}\ar[rr]^-{\kappa_{i}^n(m)\times \nu_i^m(n)} &&
\G_{n+m-1}\times\G_{n+m-1} \ar[r]^-{\cdot} &
\G_{n+m-1},
}\]
that is, $f\circ_i g = \kappa_i^n(m)(f)\cdot \nu^m_i(n)(g)$, for every $f\in\G_n$ and $g\in \G_m$. We also set $\id = e_1\in \G_1$.

\begin{prop}\label{prop:CloningtoOper}
    Let $\G=(\G_{\bullet}, \iota, \zeta, \kappa,\pi)$ be a bilateral cloning system. Then the quadruple $(\G_{\bullet},\pi, \{\circ_i\}_i, \id)$, with $\{\circ_i\}_i$ as defined above is an action operad.
\end{prop}
\begin{proof} First, we check that $e_1$ is indeed a unit for the partial composition product. For every $f\in\G_n$ and every $g\in \G_m$ we have that
\begin{align*}
    f\circ_i e_1 &= \kappa^n_i(1)(f)\cdot \nu_i^1(n)(e_1) = f\cdot e_n = f, \\
    e_1\circ_1 g &= \kappa^1_1(m)(e_1) \cdot \nu_1^m(1)(g) = e_m\cdot g = g.
\end{align*}
Second, we show the associativity for the partial composition product. Let $f\in \G_n$, $g\in \G_m$ and $h\in \G_l$. Then
    \begin{align}\notag
        &(f\circ_i g)\circ_j h \\ \notag &=(\kappa_i^n(m)(f)\cdot \nu^m_i(n)(g))\circ_j h \\ \notag
&= \kappa^{n+m-1}_j(l)\Big(\kappa_i^n(m)(f)\cdot \nu_i^m(n)(g)\Big)\cdot \nu_j^l(n+m-1)(h) \\ \label{eq:309}
&=
\Big(\kappa_{g^*(j)}^{n+m-1}(l)(\kappa_i^n(m)(f))\Big)\cdot \Big(\kappa_{j}^{n+m-1}(l)(\nu_i^m(n)(g))\Big)\cdot \Big(\nu_j^l(n+m-1)(h)\Big), 
\end{align}
where we denote $g^*=\nu_i^m(n)(g)$. Depending on the indexes $i$ and $j$, we have to consider the following three cases.

\bigskip
\noindent {\it Case 1.} Suppose first that $j<i$. In this case $g^*(j) = j$ and we have that 
$$
\kappa_j^{n+m-1}(l)\circ \kappa_i^n(m) = \kappa_{i+l-1}^{n+l-1}(m)\circ \kappa_j^n(l)\quad\mbox{and}
$$
$$
\kappa_j^{n+m-1}(l)\circ \nu_i^m(n) = \nu_{i+l-1}^m(n+l-1).
$$
Then \eqref{eq:309} becomes
\[\Big(\kappa_{i+l-1}^{n+l-1}(m)(\kappa_j^n(l)(f))\Big)\cdot \Big(\nu_{i+l-1}^m(n+l-1)(g)\Big)\cdot \Big(\nu_j^l(n+m-1)(h)\Big). \]
On the other hand, we have that
    \begin{align}\notag 
        &(f\circ_j h)\circ_{i+l-1} g  \\ \notag
&=(\kappa_j^n(l)(f)\cdot \nu_j^l(n)(h))\circ_{i+l-1} g \\ \notag      
&= \kappa_{i+l-1}^{n+l-1}(m)\Big(\kappa_j^n(l)(f)\cdot \nu_j^l(n)(h)\Big)\cdot \nu_{i+l-1}^m(n+l-1)(g) \\ \label{eq:310}
&=
\Big(\kappa_{h^*(i+l-1)}^{n+l-1}(m)(\kappa_j^n(l)(f))\Big)\cdot \Big(\kappa_{i+l-1}^{n+l-1}(m)(\nu_j^l(n)(h))\Big)\cdot \Big(\nu_{i+l-1}^m(n+l-1)(g)\Big),
\end{align}
where $h^*=\nu_j^l(n)(h)$. Now, \eqref{eq:309} and \eqref{eq:310} are equal since 
$$
h^*(i+l-1) = \nu_j^l(n)(h)(i+l-1) = i+l-1\quad\mbox{and}
$$
$$
\kappa_{i+l-1}^{n+l-1}(m)(\nu_j^l(n)(h)) = \nu_j^l(n+m-1)(h).
$$
The last equality holds because $j<i$ and therefore $i+l-1\ge j+l$.

\bigskip
\noindent{\it Case 2.} Suppose now that $j\geq i+m$. Again $g^*(j)=j$ and we have that
$$
\kappa_j^{n+m-1}(l)\circ \kappa_i^n(m) = \kappa_{i}^{n+l-1}(m)\circ \kappa_{j-m+1}^n(l)\quad\mbox{and}
$$
$$
\kappa_j^{n+m-1}(l)\circ \nu_i^m(n) = \nu_i^m(n+l-1).
$$
The fist equality holds because $j\ge i+m$ and therefore $i<j-m+1$.
Then \eqref{eq:309} becomes
\[\Big(\kappa_{i}^{n+l-1}(m)(\kappa_{j-m+1}^n(l)(f))\Big)\cdot \Big(\nu_i^m(n+l-1)(g)\Big)\cdot \Big(\nu_j^l(n+m-1)(h)\Big). \]
On the other hand,
    \begin{align}\notag 
        &(f\circ_{j-m+1} h)\circ_{i} g \\ \notag
&= (\kappa_{j-m+1}^n(l)(f)\cdot \nu_{j-m+1}^l(n)(h))\circ_i g \\ \notag  &= \kappa_{i}^{n+l-1}(m)\Big(\kappa_{j-m+1}^n(l)(f)\cdot \nu_{j-m+1}^l(n)(h)\Big)\cdot \nu_i^m(n+l-1)(g) \\ \label{eq:311}
&=
\Big(\kappa_{h^*(i)}^{n+l-1}(m)(\kappa_{j-m+1}^n(l)(f))\Big)\!\cdot \!\Big(\kappa_{i}^{n+l-1}(m)(\nu_{j-m+1}^l(n)(h))\Big)\!\cdot \!\Big(\nu_i^m(n+l-1)(g)\Big),
\end{align}
where $h^*=\nu_{j-m+1}^l(n)(h)$. But \eqref{eq:309} and \eqref{eq:311} are equal since $h^*(i) = i$ and
$$
\kappa_i^{n+l-1}(m)(\nu_{j-m+1}^l(n)(h)) = \nu_j^l(n+m-1)(h).
$$
The last equality holds because $j\ge i+m$ and therefore $i<j-m+1$.

\bigskip
\noindent {\it Case 3.} Suppose that $i\leq j<i+m$. In this case, $i\leq g^*(j)<i+m$ and we have that
$$
\kappa_{g^*(j)}^{n+m-1}(l)\circ \kappa_i^n(m) = \kappa_i^n(m+l-1)\quad\mbox{and}
$$
$$
\kappa_j^{n+m-1}(l)\circ \nu_i^m(n) = \nu_i^{m+l-1}(n)\circ \kappa_{j-i+1}^m(l).
$$
Then \eqref{eq:309} becomes
\[\Big(\kappa_i^n(m+l-1)(f)\Big)\cdot \Big(\nu_i^{m+l-1}(n)( \kappa_{j-i+1}^m(l)(g))\Big)\cdot \Big(\nu_j^l(n+m-1)(h)\Big).
\]
On the other hand, 
\begin{align} \notag
&f\circ_i(g\circ_{j-i+1} h) \\ \notag
&=f\circ_i (\kappa_{j-i+1}^m(l)(g)\cdot \nu_{j-i+1}^l(m)(h)) \\ \notag
&= \kappa_i^n(m+l-1)(f)\cdot \nu_i^{m+l-1}(n)\Big(\kappa_{j-i+1}^m(l)(g)\cdot \nu_{j-i+1}^l(m)(h)\Big)\\
&=\Big(\kappa_i^n(m+l-1)(f)\Big)\cdot \Big(\nu_i^{m+l-1}(n)(\kappa_{j-i+1}^m(l)(g))\Big)\cdot \Big(\nu_i^{m+l-1}(n)(\nu_{j-i+1}^l(m)(h))\Big).\label{eq:312}
\end{align}
Again \eqref{eq:309} and \eqref{eq:312} are equal since 
$$
\nu_i^{m+l-1}(n)(\nu_{j-i+1}^l(m)(h))=\nu_j^l(n+m-1)(h).
$$

Finally, in order to prove the product rule for the action operad, we will use condition (viii) in the definition of a bilateral cloning system. Let $f,f'\in G_n$ and $g,g'\in G_m$. Then we have that
\begin{align*}
(f\circ_{f'(i)} g)\cdot (f'\circ_i g')
&= \Big(\kappa_{f'(i)}^n(m)(f)\cdot \nu_{f'(i)}^m(n)(g)\Big)\cdot \Big(\kappa_i^n(m)(f')\cdot \nu_i^m(n)(g')\Big) \\
&= \Big(\kappa_{f'(i)}^n(m)(f)\cdot  \kappa_i^n(m)(f')\Big)\cdot \Big(\nu_{f'(i)}^m(n)(g)\cdot \nu_i^m(n)(g')\Big) \\
&= \Big(\kappa_{i}^n(m)(f\cdot f')\Big)\cdot \Big(\nu_{i}^m(n)(g\cdot g')\Big) \\
&= (f\cdot f')\circ_i (g\cdot g').\qedhere
\end{align*}
\end{proof}

\begin{thm}
There is an explicit bijective correspondence between action operads and bilateral cloning systems.     
\end{thm}
\begin{proof} The constructions given in Proposition~\ref{prop:OperadtoCS} and Proposition~\ref{prop:CloningtoOper} are inverses of each other. Let $(\G_\bullet,\iota,\zeta,\kappa,\pi)$ be a bilateral cloning system, and let $(\hat{\G}_\bullet,\hat{\iota},\hat{\zeta},\hat{\kappa},\hat{\pi})$ be the bilateral cloning system that arises from the action operad associated to it. It is clear that $\hat{\G}_\bullet = \G_\bullet$ and that $\hat{\pi} = \pi$. For the maps $\iota$ we have that
\begin{align*}
\hat{\iota}_n(g) &= e_2\circ_1 g = \kappa_1^2(n)(e_2)\cdot \nu^n_1(2)(g) = e_{n+1}\cdot \iota_n(1)(g) = \iota_n(g),
\end{align*}
and in the same way, we obtain that $\hat{\zeta}_n(g)=\zeta_n(g)$. For the cloning maps, we have that
\begin{align*}
\hat{\kappa}_j^n(g) &= g\circ_j e_2 = \kappa_j^n(2)(g)\cdot \nu_j^2(n)(e_2) = \kappa_j^n(2)(g)\cdot e_{n+1} = \kappa_j^n(g).
\end{align*}

Conversely, let $(\hat{\G}_\bullet,\hat{\pi},\{\hat{\circ}_i\}_i,\hat{\id})$ be the action operad associated to the bilateral cloning system obtained from an action operad $(\G_\bullet,\pi,\{\circ_i\}_i,\id)$. It is clear that $\hat{\G}_\bullet = \G_\bullet$, $\hat{\pi}=\pi$ and $\hat{\id}=\id$. Regarding the partial composition products, let $f\in \G_n$ and $g\in\G_m$. Then we have that
\begin{align*}
f\operatorname*{\hat{\circ}}\nolimits_i g &= \kappa_i^n(m)(f)\cdot \nu^m_i(n)(g) \\ 
&= (f\circ_i e_m)\cdot (e_n\circ_i g) \\
&= (f\cdot e_n)\circ_i (e_m \cdot g) \\
&= f\circ_i g,
\end{align*}
where we have used Remark~\ref{rem:properties} for the second equality and the product rule of the action operad for the third. 
\end{proof}


\begin{rem} As mentioned in Section \ref{section:cloning_systems}, most of the known examples of cloning systems are bilateral. Consequently, for all those examples, we have identified an action operad structure on them. However, a natural question arises: is it possible to interpret the remaining Example~\ref{examp:DirectPowers} and Example~\ref{examp:UpperTriangularMatrices}, in operadic terms? The answer is yes. Both examples have in common that their structural map $\pi$ is trivial and that the only condition that fails for them to be bilateral cloning systems is axiom (viii). 

Repeating the construction of Proposition \ref{prop:CloningtoOper} for these examples, and noticing that axiom (viii) is only applied to show the compatibility of the group multiplication with the $\circ_i$ products, one gets the structure of a non-symmetric operad in sets (see Definition \ref{def:actionoperad}). The reason why they are not bilateral is that a bilateral cloning system with $\pi$ trivial corresponds under the correspondence with a  non-symmetric operad in groups. 
\end{rem}









%%%%%%%%%%%%%%%%%%%%%%%%%%%%%%%%%%%%%%%%%%%%





\section{Cloning systems as crossed groups}\label{section:crossed}


Let $(\G_{\bullet},\pi,\iota,\kappa)$ be a cloning system. If we forget all the structure except the cloning maps, we can interpret the pair $(\G_\bullet,\kappa)$ as a diagram
$$
\begin{tikzcd}[ampersand replacement=\&]
\G_1
\ar[r, "\kappa_1" description] \& \G_2
\ar[r, "\kappa_1" description, shift left=2]\ar[r, "\kappa_2" description, shift right = 2] \& \G_3 
\ar[r, "\kappa_1" description, shift left=4]\ar[r, "\kappa_2" description]\ar[r, "\kappa_3" description, shift right=4] \& \;\cdots 
\end{tikzcd}
$$
which resembles the diagram representing a simplicial set $X$, but considering only the degeneracy operations
$$
\begin{tikzcd}[ampersand replacement=\&]
X[0] \ar[r, "s_0" description] \& X[1] \ar[r, "s_0" description, shift left=2]\ar[r, "s_1"' description, shift right=2] \& X[2] \ar[r, "s_0" description, shift left=3]\ar[r, "s_1" description] \ar[r, "s_2"' description, shift right=3] \& X[3] \ar[r, "s_0"description, shift left=5]\ar[r, "s_1" description, shift left=2]\ar[r, "s_2" description, shift right=2]\ar[r, "s_3"' description,shift right=5] \& X[4] \ar[r, "s_0" description,shift left=6]\ar[r, "s_1" description, shift left=3]\ar[r, "s_2" description]\ar[r, "s_3" description, shift right=3]\ar[r, "s_4"' description, shift right=6] \& \;\cdots 
\end{tikzcd}
$$
Let us formalise this viewpoint. Let $[n]=\{0,1,\ldots,n\}$ be the finite ordinal of cardinality $n+1$.
\begin{defn}
    The \emph{simplicial category} $\Delta$ has objects the non-empty finite ordinals and morphisms the order-preserving maps between them. There are two distinguished families of morphisms
\begin{align*}
    \delta^n_j\colon &[n-1]\longrightarrow [n], & \sigma^n_j\colon &[n+1]\longrightarrow [n], & 0\leq j\leq n,
\end{align*}
called \emph{cofaces} and \emph{codegeneracies}, respectively, and defined as
\begin{align*}
    \delta_j^n(i)&=\begin{cases}
        i & \text{if $i<j$} \\
        i+1 & \text{if $i\geq j$}
    \end{cases}
&
    \sigma_j^n(i)&=\begin{cases} i &\text{if $i\leq j$} \\ i-1 & \text{if $i>j$}
\end{cases}
\end{align*}
These morphisms satisfy the following relations, called \emph{cosimplicial identities}
\begin{align*}\label{eq:007}
\delta^n_{j}\circ \delta^{n-1}_{i+1} &= \delta^n_{i}\circ \delta^{n-1}_j, \qquad j\leq i, \\
\sigma^n_{j+1}\circ \sigma^{n+1}_i &= \sigma^n_i\circ \sigma^{n-1}_{j},\qquad i\leq j, \\
\delta^{n+1}_i\circ \sigma^n_j &= \begin{cases}
    \sigma^n_{j-1}\circ \delta^{n+1}_i & i<j, \\
    \id & i=j,j+1, \\
    \sigma^n_j\circ \delta^{n+1}_{i-1} & i>j+1,
\end{cases}
\end{align*}
and generate all the morphisms in the category in the sense that every morphism can be expressed as a composite of cofaces and codegeneracies. 
\end{defn}
\begin{defn}
Let $\Delta_{\surj}\subset \Delta$ be the category of non-empty finite ordinals with order-preserving surjections between them. It is the subcategory of $\Delta$ generated by the maps $s_j^n$.
\end{defn}
\begin{defn}
    Let $\Set$ be the category of sets and functions. A~\emph{simplicial set} $X$ is a functor $X\colon\Delta^\op\to \Set$. The maps $X(\delta^n_i)$ are called \emph{faces} and denoted $d^n_i$ and the maps $X(\sigma^n_i)$ are called \emph{degeneracies} and denoted $s^n_i$. A \emph{demi-simplicial set}\footnote{It seems these objects have not been considered previously in the literature. Since simplicial sets without degeneracies are called \emph{semi-simplicial sets}, we have chosen to replace the prefix \emph{semi-} by its french version \emph{demi-}. Additionally, the first letter of each prefix specifies whether we are removing degeneracies or face maps.} $X$ is a functor $X\colon\Delta^\op_\surj\to \Set$.
\end{defn}

Faces and degeneracies satisfy the so-called \emph{simplicial identites} which are the dual of the cosimplicial identites mentioned above. In the particular case of the relation involving only degeneracies, we get the identity
\begin{equation}
s^{n+1}_i\circ s^n_{j+1}=s^{n-1}_{j}\circ s^n_i,\qquad i\leq j.\label{eq:deg_rel}
\end{equation}

In what follows, when we refer to conditions in roman numbers, we mean the conditions satisfied by cloning systems and bilateral cloning systems from Definition~\ref{defn:CloningSystem} and Definition~\ref{defn:NCCloningSystem}.

Observe that \eqref{eq:deg_rel} is exactly the same relation satisfied by the maps $\kappa^n_j$ with the extra condition (iv+) except that the subindexes are shifted by one (the first map is $\kappa_1$ not $\kappa_0$). Therefore we have the following consequence. 
\begin{lem} Let $\G_{\bullet}=\{\G_n\}_{n\geq 1}$ be a family of groups and let $\kappa=\{\kappa^n_{j}\colon\G_{n}\to\G_{n+1}\}_{n\ge 1,\,1\leq j\leq n}$ be a family of maps. A pair $(\G_\bullet,\kappa)$ satisfying conditions \emph{(iv)} and \emph{(iv+)} is the same as a demi-simplicial set with values on groups. $\hfill\qed$
\end{lem}
\begin{examp}\label{ex:demi_symm} Let us build the demi-simplicial set associated to the cloning system of symmetric groups. Note that an order-preserving surjection $f\colon [n]\to [m]$ is completely determined by the cardinality of the preimages $f^{-1}(0),\ldots,f^{-1}(m)$. Every permutation $h\in \Sigma_{m+1}$ of $[m]$ induces a permutation of the preimages $f^{-1}(0),\ldots,f^{-1}(m)$, and therefore a block permutation on $[n]$, that we denote by $\Phi(f)(h)\in \Sigma_{n+1}$. This defines a functor $\Phi\colon \Delta_\surj^\op\to \Set$ with $\Phi([n]) = \Sigma_{n+1}$.

There is an action of $\Sigma_{m+1}$ on the set of order-preserving surjections from $[n]$ to $[m]$, that sends a permutation $g$ and a surjection $f$ to the unique surjection $h$ such that the cardinality of $h^{-1}(g(i))$ is equal to the cardinality of $f^{-1}(i)$. We denote this surjection $h$ by $f_g$. Observe now that the map $\Phi(f)\colon \Sigma_{m+1}\to \Sigma_{n+1}$ is not a group homomorphism but satisfies that 
\begin{equation*}\label{eq:610}
    \Phi(f)(g\cdot g') = \Phi(f_{g'})(g)\cdot\Phi(f)(g'),
\end{equation*}
which applied to degeneracies is the same as condition (v) in a cloning system
\begin{equation}\label{eq:611}\Phi(s_i)(g\cdot g') = \Phi(s_{g'(i)})(g)\cdot \Phi(s_i)(g').
\end{equation}
The functor $\Phi$, defined on order-presering surjections between ordinals, can be extended to any order-preserving map between ordinals. Indeed, since every map factors as a surjection followed by an injection, it is enough to define it on injections, which we can do as follows. Note that to give an order-preserving injective function $f\colon [n]\to [m]$ is equivalent to specify the complement of the image $A = [m]\smallsetminus f([n])$. There is an action of $\Sigma_{m+1}$ on the set of order-preserving injections from $[n]$ to $[m]$, that sends a permutation $g$ and an injection $f$ to the unique injection $h$ such that $[m]\smallsetminus h([n]) = [m]\smallsetminus g(f([m]))$. We denote this injection $h$ by $f_g$. If $g\in \Sigma_{m+1}$ is a permutation of $[m]$, define $\Phi(f)(g) = f_g^{-1}\circ g\circ f$. This defines a functor $\Phi\colon \Delta^\op\to \Set$, and hence a simplicial set.

 Observe now that the map $\Phi(f)\colon \Sigma_{m+1}\to \Sigma_{n+1}$ is not a group homomorphism but satisfies that 
\begin{equation*}\label{eq:613}
    \Phi(f)(g\cdot g') = \Phi(f_{g'})(g)\cdot\Phi(f)(g').
\end{equation*}
\end{examp}
The following definition of crossed simplicial groups is a characterization taken from~\cite[Proposition 1.7]{FL}, where $\Phi$ denotes the simplicial set constructed above.
\begin{defn}
    A \emph{crossed simplicial group} is a simplicial set $\Psi\colon \Delta^\op\to \Set$ with values on groups together with a natural transformation $\pi\colon \Psi\to \Phi$ such that 
\begin{align*}
    \Psi(s_i)(g\cdot g') &= \Psi(s_{\pi(g)(i)})(g')\cdot\Psi(s_i)(g'), \\ 
    \Psi(d_i)(g\cdot g') &= \Psi(d_{\pi(g)(i)})(g')\cdot\Psi(d_i)(g'),
    \end{align*} and $\pi([n])\colon \Psi([n])\to \Sigma_n$ is a group homomorphism. A \emph{crossed demi-simplicial group} is defined in the same way, replacing the category $\Delta$ by the category $\Delta_\surj$. 
\end{defn}

The following result follows immediately from the previous discussion and the definition of crossed demi-simplicial group.
\begin{lem} Let $\G_{\bullet}=\{\G_n\}_{n\geq 1}$ be a family of groups, $\pi=\{\pi_n\colon \G_n\to\Sigma_n\}_{n\ge 1}$ a family of group homomorphisms and $\kappa=\{\kappa^n_{j}\colon\G_{n}\to\G_{n+1}\}_{n\ge 1,\,1\leq j\leq n}$ a family of maps. A triple $(\G_\bullet,\pi,\kappa)$ satisfying conditions {\rm (ii)}, {\rm (iv)}, {\rm (iv+)} and {\rm (v)} is the same as a crossed demi-simplicial group. $\hfill\qed$
\end{lem}

Now we would like to incorporate the homomorphisms $\iota$ to the picture. We will incorporate the morphisms $\zeta$ at the same time.

\begin{defn} For each $n\geq 1$, the $n^{\text{th}}$ \emph{interval} is the set $\langle n\rangle = \{-\infty,1,\ldots,n,\infty\}$. The \emph{interval category} $\I$ is the category whose objects are all intervals and whose morphisms are order-preserving maps that preserve $-\infty$ and $\infty$. The subcategory $\I_\surj$ has the same objects as $\I$ and its morphisms are the order-preserving surjective maps that preserve $-\infty$ and $\infty$.\end{defn}
The interval category can be introduced as the Joyal dual of the simplicial category \cite{Joyal} or as the image of the faithful embedding $\alpha\colon \I\to \Delta$ that sends $\langle n\rangle = \{-\infty,1,\ldots,n,\infty\}$ to $[n+1] = \{0,1,\ldots,n+1\}$, and an interval map yields a simplicial map by interpreting $-\infty$ as $0$ and $\infty$ as $n+1$. Here we are interested in the second description, and we will blur de difference between maps in $\I$ and their images under the embedding $\alpha$.
\begin{defn}
    An \emph{inert crossed interval group} is a presheaf $\Psi\colon \I^\op\to \Set$ with values on groups together with a natural transformation $\pi\colon \Psi\to \Phi\circ \alpha$ such that 
    \begin{align*}\label{eq:product_cig}
    \Psi(s_i)(g\cdot g') &= \Psi(s_{\pi(g)(i)})(g')\cdot\Psi(s_i)(g'), \\ 
    \Psi(d_i)(g\cdot g') &= \Psi(d_{\pi(g)(i)})(g')\cdot\Psi(d_i)(g'),
    \end{align*} and $\pi([n])\colon \Psi([n])\to \Sigma_n$ is a group homomorphism, where $\Phi$ is the simplicial set constructed in Example~\ref{ex:demi_symm}. An \emph{inert crossed demi-interval group} is defined in the same way, by replacing $\I$ by $\I_\surj$.
\end{defn}
Observe now that if $\Phi$ is an inert crossed interval group, the maps $\Phi(s_0)$ and $\Phi(s_{n+1})$ are group homomorphisms. In fact, we can interpret bilateral cloning systems as an inert crossed demi-interval groups by setting $\kappa^n_i = s^{n+1}_i$, $\zeta_n = s^{n+1}_0$ and $\iota_n = s^{n+1}_{n+1}$. Thus, we have the following result.
\begin{lem} A quintuple $(\G_\bullet,\pi,\iota,\zeta,\kappa)$ satisfying all the conditions of a bilateral cloning system except possibly conditions {\rm (viii)} and {\rm (ix)} is the same as an inert crossed demi-interval group.
\end{lem}

\begin{rem} Crossed interval groups have been studied in \cite{Batanin-Markl} and \cite{Yoshida}. The adjective \emph{inert} corresponds to any of the two equivalent properties defined in \cite[Lemma 4.3]{Yoshida}.

In that paper, Yoshida studied the relation between crossed interval groups and action operads. He established that every action operad determines a crossed interval group, and found two properties that characterise the crossed interval groups that come from an operad: operadicness and tameness. Under the above lemma, tameness corresponds to property (ix), while operadicness corresponds to property (viii) plus being inert. Yoshida studies action operads with constants (operations of arity 0), while we study action operads without constants. The existence of constants in the action operad corresponds to the injective morphisms in the simplicial or the interval category.

We note that Examples 3.1 and 3.4 in \cite{Zaremsky}, which do not come from an action operad, do come from an inert crossed demi-interval group (with trivial natural transformation $\pi$). 
\end{rem}





\section{Bilateral cloning systems as PROs}\label{section:props}

In this section, we present another viewpoint for cloning systems in terms of PROs (product categories) that will be used in a forthcoming piece of work about the construction of Thompson groups using the different viewpoints in this paper, and the equivalence between them. We begin by recalling the definition of PRO.
\begin{defn} A \emph{PRO} $\Ofrak$ is a quadruple $(\Ofrak,\odot,\otimes,\id)$ where:
\begin{itemize}
    \item $\Ofrak$ is a collection of sets $\Big(\Ofrak\brbinom{n}{m}\Big)_{n,m\geq 1}$,
    \item $\odot$ is an associative product (\emph{vertical product}) with two-sided unit $\id$
    $$
    \odot\colon \Ofrak\brbinom{n}{t}\times \Ofrak\brbinom{s}{n}\longrightarrow \Ofrak\brbinom{s}{t},\quad  \id_{n}\in\Ofrak\brbinom{n}{n},
    $$
    \item $\otimes$ is an associative product (\emph{horizontal product})
    $$
    \otimes\colon \Ofrak\brbinom{n_1}{m_1}\times \Ofrak\brbinom{n_2}{m_2}\longrightarrow \Ofrak\brbinom{n_1+n_2}{m_1+m_2}.
    $$
\end{itemize}
Moreover, the vertical and horizontal product are required to satisfy the \emph{interchange law}
$$
(f\otimes g)\odot (p\otimes q)=(f\odot p)\otimes (g\odot q),
$$
whenever it makes sense, and  $\id_{n}\otimes\id_{m}=\id_{n+m}$ for all $n,m\geq 1$.
\end{defn}

More abstractly, a \emph{PRO} $\mathfrak{O}$ is a strict (non unital) monoidal category equipped with a strict monoidal functor $(\mathbb{N}_{\geq 1},+)\to (\mathfrak{O},\otimes)$ which is an isomorphism on objects. We do not consider monoidal units, that is, units for $\otimes$, because they will create nullary operations that we are avoiding.



\begin{examp}[Symmetric groups]
The family of symmetric groups $\{\Sigma_n\}_{n\ge 1}$ yields a very simple PRO by setting
$$
\Sigma\brbinom{n}{m}= \begin{cases}
    \Sigma_n & \mbox{if $n=m$}, \\
    \emptyset & \mbox{if $n\neq m$}.
\end{cases}
$$
The vertical product $\odot$ is just the group structure of the symmetric groups and the horizontal product $\otimes$ is the block product of permutations (see Figure ~\ref{fig:HorizontalProductSigma}).
\begin{figure}[h]
    \centering
    \includegraphics[width=6cm]{HorizontalProductSigma.pdf}
    \caption{Block product of $(1\;4)(2\;3)$ and $(1\;2\;3)$.}
    \label{fig:HorizontalProductSigma}
    \end{figure}
We denote by $\Sigma$ the PRO of symmetric groups. Additionally, we consider that $\Sigma$ is equipped with \emph{cloning maps} for any $n$-tuple of positive integers $(m_1,\dots,m_n)$, 
$
c(m_1,\dots,m_n)\colon \Sigma\brbinom{n}{n}\to \Sigma\brbinom{m_1+\dots+m_n}{m_1+\dots+m_n},
$ 
obtained by replacing the $i$th strand by $m_i$ strands for all $1\leq i\leq n$.
\end{examp}

We will also need the concept of a \emph{morphism of PROs} $f\colon \Ofrak\to\Ofrak'$, which consists of maps $f_{n,m}\colon\Ofrak\brbinom{n}{m}\to \Ofrak'\brbinom{n}{m}$ for every $n,m\ge 1$ that preserve the PRO structure.

Next, we define a special type of PROs, by adding some extra structure, which will turn out to be equivalent to bilateral cloning systems. 

\begin{defn}
A \textit{cloning PRO} consists of the following data:
\begin{itemize}
    \item a PRO $(\G,\odot,\otimes,\id)$ such that the vertical product on $\G\brbinom{n}{n}$ admits inverses, that is, $\G\brbinom{n}{n}$ is a group for very $n$, and $\G\brbinom{n}{m}=\emptyset$ if $n\neq m$, 
    \item a morphism of PROs $\pi\colon\G\to\Sigma$,
    \item cloning maps 
    $\kappa(\underline{m})\colon \G\brbinom{n}{n}\rightarrow\G\brbinom{m_1+\dots+m_n}{m_1+\dots+m_n}$ for every $n\geq 1$ and every $n$-tuple $\underline{m}=(m_1,\dots,m_n)$ with $m_i\geq 1$, satisfying $\kappa(1,\stackrel{(n)}{\dots}, 1)=\id_{\G_n}$, and  the associativity condition
    $$
    \kappa(\underline{r}^1\sqcup\cdots\sqcup\underline{r}^{n})\circ\kappa(\underline{m})=\kappa\Big(\sum_{j_1}r^1_{j_1},\dots,\sum_{j_n}r^n_{j_n}\Big) 
    ,
    $$
    where $\underline{r}^i=(r^i_1,\dots,r^i_{m_i})$ and  
    $
    \underline{r}^1\sqcup\cdots\sqcup\underline{r}^n=(r^1_1,\dots,r^1_{m_1},\dots,r^n_1,\dots,r^n_{m_n}).
    $
\end{itemize}
Moreover, the PRO structure is compatible with $\pi$ and $\kappa$ as stated by the following conditions:
\begin{enumerate}
    \item $\pi$ commutes with  $\kappa$: $\pi\circ\kappa(\underline{m})=c(\underline{m})\circ\pi$.
     \item Identities and $\kappa$: $\kappa(\underline{m})(\id_{n})=\id_{m_1+\cdots+m_n}$.
    \item Horizontal composition and $\kappa$:
    $$\kappa(\underline{m})(f)\otimes \kappa(\underline{n})(g)= \kappa(\underline{m}\sqcup\underline{n})(f\otimes g).$$
    \item Vertical composition and $\kappa$:
    $$\kappa(\underline{m})(h\odot g)=\kappa(\underline{m}_{\pi(g)})(h)\odot \kappa(\underline{m})(g),$$ where  $\underline{m}_{\pi(g)}=(m_{\pi(g)(1)},\dots,m_{\pi(g)(n)})$. % provided that $\underline{m}=(m_1,\dots,m_n)$ \ja{quitamos lo del provided?}.
    \item[($\star$)] Twisted interchange law: $$\kappa(\underline{m})(f)\odot (g_{1}\otimes \cdots\otimes g_{n})=(g_{\pi(f)(1)}\otimes \cdots \otimes g_{\pi(f)(n)})\odot \kappa(\underline{m})(f).$$
\end{enumerate}
\end{defn}

After these definitions, we prove that every bilateral cloning system gives rise to a cloning PRO. More concretely: 

\begin{prop}\label{prop:NCCStoPRO} A bilateral cloning system $(\G_{\bullet},\iota,\zeta,\kappa,\pi)$ defines a cloning PRO  $(\G,\odot,\otimes,\id,\pi,\kappa)$ in a functorial way.
\end{prop}
\begin{proof}
Set $\G\brbinom{n}{n}=\G_n$, the vertical product $\odot$ to be the group multiplication and $\id_n=e_n$. Define the cloning maps
$$
\kappa(\underline{m})=\kappa(m_1,\dots,m_n)=\kappa_1(m_1)\circ\cdots \circ \kappa_{n}(m_n),
$$
and the horizontal product $\otimes$
$$
\begin{tikzcd}
\otimes\colon \G\brbinom{n}{n}\times \G\brbinom{m}{m}\ar[rr,"\iota(m)\times \zeta(n)"] && \G\brbinom{n+m}{n+m}\times\G\brbinom{n+m}{n+m}\ar[r, "\cdot"] & \G\brbinom{n+m}{n+m} 
\end{tikzcd}.
$$

We know check that $(\G,\odot,\otimes,\id,\pi,\kappa)$ is a cloning PRO by verifying all the axioms. Let $f\in\G_n$, $g\in\G_m$ and $h\in \G_r$.
\begin{itemize}
    \item Associativity of $\otimes$: 
\begin{align*}
(f\otimes g)\otimes h &=  \iota(r)\big(\iota(m)(f)\cdot \zeta(n)(g)\big)\cdot \zeta(n+m)(h)\\
&= \iota(r+m)(f)\cdot \iota(r)(\zeta(n)(g))\cdot \zeta(n+m)(h)\\
&= f\otimes (g\otimes h),
\end{align*}
by condition (vi) of bilateral cloning systems and since $\iota$ and $\zeta$ are homomorphisms.
\item Interchange law for $\odot$ and $\otimes$: 
\begin{align*}
(f\otimes g)\odot (p\otimes q) &=  \Big(\iota(m)(f)\cdot \zeta(n)(g)\Big)\cdot\Big(\iota(m)(p)\cdot\zeta(n)(q)\Big)\\
&= \iota(m)(f)\cdot \iota(m)(p)\cdot \zeta(n)(g)\cdot\zeta(n)(q)\\
&= (f\odot p)\otimes (g\odot q),
\end{align*}
by condition (ix) and since $\iota$ and $\zeta$ are homomorphisms.

\item $\pi\colon \G\to \Sigma$ is a morphism of PROs, since $\pi$ commutes with $\iota, \kappa,\zeta$ and it consists of group homomorphisms.

\item Associativity conditions for $\kappa$ hold by conditions (iv) and (iv+) (axioms for compositions of $\kappa$'s for bilateral cloning systems).

\item Commutation of $\kappa$ and $\pi$ holds by the analogous axiom for bilateral cloning systems.

\item $\kappa$ acting on identities. Since
$$
\kappa_j(r)(e_m)=\kappa_j(r)(e_m)\cdot\kappa_j(r)(e_m)
$$
it follows that $\kappa_j(r)(e_m)=e_{m+r-1}$.

\item Relation between $\kappa$ and $\odot$. By iterating condition (v) we get 
\begin{multline*}
\kappa(\underline{m})(f\odot p) =  \kappa_1(m_1)\circ\cdots\circ \kappa_n(m_n)(f\cdot p)\\
=  \Big(\kappa_{p(1)}(m_1)\circ\cdots\circ\kappa_{p(n)}(m_n)(f)\Big)\cdot \Big(\kappa_1(m_1)\circ\cdots\circ\kappa_n(m_n)(p)\Big)\\
= \kappa(\underline{m}_{p})(f)\odot \kappa(\underline{m})(p).
\end{multline*}

\item Relation between $\kappa$ and $\otimes$. Let us analyze what happens for $\kappa_j(r)$ instead of a general $\kappa(\underline{m}^1\sqcup\underline{m}^2)$, because the general case will follow by combining these simple cases by conditions (iv) and (iv+). 

\begin{itemize}
    \item[\textbf{Case 1:}] ($1\leq j\leq n$). 
    \begin{align*}
 \kappa_j(r)(f\otimes g) &=  \kappa_j(r)\big(\iota(m)(f)\cdot\zeta(n)(g)\big)\\
 &=  \kappa_{\zeta(n)(g)(j)}(r)(\iota(m)(f))\cdot \kappa_j(r)(\zeta(n)(g))\\
 &=\kappa_j(r)(\iota(m)(f))\cdot \kappa_j(r)(\zeta(n)(g))\\
 &=\iota(m)\big(\kappa_j(r)(f)\big)\cdot \zeta(r+n)(g)\\
 &= \kappa_j(r)(f)\otimes g.
 \end{align*}
We have applied conditions (v), (iii), (vii') and that $\zeta(n)(g)$ is constant over $\{1,\dots, n\}$.
    \item[\textbf{Case 2:}] ($n<j\leq m+n$).
    \begin{align*}
 \kappa_j(r)(f\otimes g) 
 &=  \kappa_{n+g(j)}(r)(\iota(m)(f))\cdot \kappa_j(r)(\zeta(n)(g))\\
 &=\iota(m+r)(f)\cdot \zeta(n)(\kappa_j(r)(g))\\
 &= f\otimes \kappa_j(r)(g).
 \end{align*}
We have applied conditions (v), (iii'), (vii) and that $\zeta(n)(g)(j)=n+g(j)>n$.
\end{itemize}

\item Twisted interchange law. Let us just check the case $f\in \G_2$ ($n=2$) and $g_i\in \G_{m_i}$, since the general case follows from the same ideas.
 \begin{align*}
\kappa(m_1,m_2)(f)\odot (g_1\otimes g_2) 
 &= (\kappa_1(m_1)\circ\kappa_2(m_2))(f)\cdot \Big(\iota(m_2)(g_1)\cdot\zeta(m_1)(g_2)\Big) \\
 &\overset{(a)}{=} \Big(\kappa_1(m_1)\big(\kappa_2(m_2)(f)\big)\cdot \nu_{1}(m_2)(g_1)\Big)\cdot \zeta(m_1)(g_2)\\
 &\overset{(b)}{=} \Big(\nu_{f(1)}(m_2)(g_1)\cdot \kappa_{1}(m_1)\big(\kappa_{2}(m_2)(f)\big)\Big)\cdot \zeta(m_1)(g_2)\\
  &\overset{(c)}{=} \nu_{f(1)}(m_2)(g_1)\cdot \kappa_{1}(m_1)\Big(\kappa_{2}(m_2)(f)\cdot \nu_2(1)(g_2)\Big)\\
  &\overset{(d)}{=} \nu_{f(1)}(m_2)(g_1)\cdot \kappa_{1}(m_1)\Big(\nu_{f(2)}(1)(g_2)\cdot\kappa_{2}(m_2)(f)\Big)\\
   &\overset{(e)}{=} \nu_{f(1)}(m_2)(g_1)\cdot \nu_{f(2)}(m_1)(g_2)\cdot\kappa(m_1,m_2)(f)\\
 &\overset{\phantom{(e)}}{=} (g_{f(1)}\otimes g_{f(2)})\odot \kappa(m_1,m_2)(f).
 \end{align*}
In the above equalities, we have applied: $(a)$ definition of $\nu$; $(b)$ condition (viii); $(c)$ condition (vii') and definition of $\nu$; $(d)$ condition (viii); and $(e)$ conditions (vii) and (iv+). \hfill\qedhere
\end{itemize}
\end{proof}

Next, we prove that bilateral cloning systems come from cloning PROs: 

\begin{prop}\label{prop:PROsToNCCS} A cloning PRO $(\G,\odot,\otimes,\id,\pi,\kappa)$ determines a bilateral cloning system $(\G_{\bullet},\iota,\zeta,\kappa,\pi)$ in a functorial way.
\end{prop}
\begin{proof}
Set $\G_n=\G\brbinom{n}{n}$, the product $\cdot=\odot$ and $e_n=\id_n$. Define
$$
\kappa_j(m)=\kappa(1,\dots,1,m^{(j)},1,\dots,1), \quad 
\iota(f)=f\otimes \id \quad \text{and} \quad \zeta(f)=\id\otimes f.
$$

The only non straightforward axioms to check are the following:
\begin{itemize}
    \item $\iota$ is a homomorphism of groups. We have that,
    $$
    \iota(m)(f\cdot g)= (f\odot g)\otimes \id_m= (f\otimes \id_m)\odot(g\otimes \id_m)= \iota(m)(f)\cdot\iota(m)(g)
    $$
    by the interchange law and since $\id_m$ is idempotent. A similar argument proves that $\zeta$ is also a homomorphism of groups.
    \item Axioms (iii) and (vii). We have that
    $$
    \kappa_j(r)\iota(m)(f)=\left\lbrace 
    \begin{array}{lcl}
    \kappa_j(r)(f)\otimes \id_m=\iota(m)\kappa_j(r)(f) && \text{if }1\leq j\leq n,\\[4mm]
   f\otimes \id_{m+r-1}=\iota(m+r-1)(f) && \text{otherwise},
    \end{array}
    \right.
    $$
    where $f\in\G\brbinom{n}{n}$. The corresponding conditions (iii') and (vii') for $\zeta$ can be deduced in the same way.
    \item Axiom (vi). We have that 
    $$
    \zeta(m)\iota(n)(f)=\id_m\otimes f\otimes \id_n= \iota(n)\zeta(m)(f)
    $$
    by the associativity of the horizontal product $\otimes$.
    \item Axiom (viii). We have that
    \begin{align*}
 \kappa_j(m)(f)\cdot \nu_j(n)(g)&= \kappa(1,\dots,m^{(j)},\dots, 1)(f)\odot(\id_{j-1}\otimes\, g\otimes \id_{n-j+1})\\
 &= (\id_{f(j)-1}\otimes\, g\otimes \id_{n-f(j)+1})\odot \kappa(1,\dots,m^{(j)},\dots, 1)(f)\\ 
 &=\nu_{f(j)}(n)(g)\cdot \kappa_j(m)(f),
 \end{align*}
 by the twisted interchange law and the identity $\id_{r}\otimes\id_{s}=\id_{r+s}$.
    \item Axiom (ix). We have that
     \begin{align*}
 \iota(m)(f)\cdot \zeta(n)(g)&= (f\otimes \id_m)\odot(\id_n\otimes\, g)\\
 &= (f\odot \id_n)\otimes (\id_m\odot\, g)\\ 
 &= (\id_n\odot\, f)\otimes (g\odot \id_m)\\
  &= (\id_n\otimes\, g)\odot (f\otimes \id_m)\\
 &=\zeta(n)(g)\cdot \iota(m)(f),
 \end{align*}
    by the interchange law and since $\id_r$ is an identity for $(\G\brbinom{r}{r},\odot)$. \hfill\qedhere
\end{itemize}
\end{proof}

Finally, we have the following theorem, which ties together the results in this section: 


\begin{thm} The functorial constructions of Propositions \ref{prop:NCCStoPRO} and \ref{prop:PROsToNCCS} induce an isomorphism of categories
$$
\begin{Bmatrix}
\text{bilateral}\\
\text{cloning systems}
\end{Bmatrix} \cong 
\begin{Bmatrix}
\text{cloning PROs}
\end{Bmatrix}.
$$
\end{thm}
\begin{proof}
Let us first check that
$$
\begin{Bmatrix}
\text{bilateral}\\
\text{cloning systems}
\end{Bmatrix} \to 
\begin{Bmatrix}
\text{cloning PROs}
\end{Bmatrix}\to 
\begin{Bmatrix}
\text{bilateral}\\
\text{cloning systems}
\end{Bmatrix}.
$$
is the identity. Let $(\G_{\bullet},\iota,\zeta,\kappa,\pi)$ be a bilateral cloning system, and let $(\hat{\G}_{\bullet},\hat{\iota},\hat{\zeta},\hat{\kappa},\hat{\pi})$ be the image of the previous bilateral cloning system under the composition of functors. By definition $\hat{\G}=\G$ as groups and $\hat{\pi}=\pi$. For $\hat{\iota}$, we have that
\begin{align*}
\hat{\iota}(g) &= g\otimes \id = \iota(1)(g)\cdot \zeta(n)(e_1)=\iota(g)\cdot e_{n+1}=\iota(g).
\end{align*}
The maps $\zeta$ behave similarly, and
\begin{align*}
\hat{\kappa}_j(g) &= \kappa(1,\dots,2^{(j)},\dots,1)(g)=\big(\kappa_1(1)\circ\cdots\circ\kappa_j(2)\circ\cdots\circ\kappa_n(1)\big)(g)=\kappa_j(g).
\end{align*}

Conversely, let $(\G,\odot,\otimes,\id,\pi,\kappa)$ be a cloning PRO and let $(\hat{\G},\hat{\odot},\hat{\otimes},\hat{\id},\hat{\pi},\hat{\kappa})$ be the image of it under the other composition of functors. It is clear that $\hat{\G} = \G$, $\hat{\odot}=\odot$, $\hat{\id}=\id$, $\hat{\pi}=\pi$ and $\hat{\kappa}=\kappa$. Finally, regarding the horizontal product we have that
\begin{align*}
f\,\hat{\otimes}\, g &= \iota(m)(f)\cdot \zeta(n)(g)= (f\otimes \id_m)\odot (\id_n\otimes g)\\ 
&= (f\odot \id_n)\otimes(\id_m \odot g) = f\otimes g. \qedhere
\end{align*}
\end{proof}


\bibliographystyle{alpha}
\bibliography{references.bib}

\end{document}
