\documentclass{amsart}
\usepackage[utf8]{inputenc}

%%%%%%%%%%%%%%%%% PACKAGES %%%%%%%%%%%%%%%%%

\usepackage{amssymb}
\usepackage{tikz}
\usepackage{tikz-cd}
\usetikzlibrary{matrix,arrows,calc,decorations.pathmorphing,decorations.pathreplacing,positioning}
\usepackage{graphicx}
\usepackage{caption}
\usepackage{url}
\usepackage{color}
\usepackage[all]{xy}
\SelectTips{cm}{}

%%%%%%%%%%%%%%%%%% COMMANDS %%%%%%%%%%%%%%%%

\DeclareMathOperator{\Br}{\mathsf{Br}}
\DeclareMathOperator{\RBr}{\mathsf{RBr}}
\DeclareMathOperator{\URBr}{\mathsf{uRBr}}
\DeclareMathOperator{\Up}{\mathsf{U}}
\DeclareMathOperator{\id}{\mathsf{id}}
\DeclareMathOperator{\G}{\mathsf{G}}
\DeclareMathOperator{\circulo}{\circ}
\def\Ofrak{\mathfrak{O}}

\DeclareMathOperator{\Emb}{\mathsf{Emb}}
\DeclareMathOperator{\Conf}{\mathsf{Conf}}


\def\bC{\mathbb{C}} 
\def\C{\mathsf{D}}
\def\fr{\mathrm{fr}}
\def\PP{\mathsf{P}}

\def\twist{\mathrm{twist}}
\def\OO{\mathsf{O}}
\def\EE{\mathsf{E}}
\def\lra{\longrightarrow}

\def\op{\mathrm{op}}
\def\surj{\mathrm{surj}}
\def\inj{\mathrm{inj}}
\def\Fin{\mathrm{Fin}}
\def\Set{\mathrm{Set}}
\def\I{\mathbb{I}}

\def\NQCS{\mathbf{NQCS}}
\def\ActionOperads{\mathbb{AOp}}

\def\CS{\mathbf{CS}}
\def\preCS{\mathbf{preCS}}

 \newtheorem{thm}{Theorem}[section]
  \newtheorem{taggedtheoremx}{Theorem}
 \newenvironment{taggedtheorem}[1]
 {\renewcommand\thetaggedtheoremx{#1}\taggedtheoremx}
 {\endtaggedtheoremx}
 \newtheorem{cor}[thm]{Corollary}
 \newtheorem{lem}[thm]{Lemma}
 \newtheorem{prop}[thm]{Proposition}
 
 \theoremstyle{definition}
 \newtheorem{defn}[thm]{Definition}
   \newtheorem{rem}[thm]{Remark}
   \newtheorem{examp}[thm]{Example}

\newcommand{\ja}[1]{\color{blue} \{JA: #1\} \color{black}}
\newcommand{\fc}[1]{\color{red} \{FC: #1\} \color{black}}
\newcommand{\vc}[1]{\color{blue} \{VC: #1\} \color{black}}
\newcommand{\jg}[1]{\color{purple} \{JG: #1\} \color{black}}
\newcommand{\old}[1]{\color{green} \{OLD: #1\} \color{black}}



\newcommand{\brbinom}[2]{\genfrac{[}{]}{0pt}{}{#1}{#2}}

\numberwithin{equation}{section}


\begin{document}


%%%%%%%%%%%%%%%%%%%%%%%%%%%%%%%%%%%%%%%%%%%%%

\title{Cloning systems and action operads}

\author[J. Aramayona]{Javier Aramayona}
\address{Javier Aramayona: Instituto de Ciencias Matem\'aticas (ICMAT). Nicol\'as Cabrera, 13--15. 28049, Madrid, Spain}
\email{javier.aramayona@icmat.es}
\thanks{J.A. was supported by grant CEX2019-000904-S funded by MCIN/AEI/ 10.13039/501100011033 and by grant PGC2018-101179-B-I00.  F.C. was supported by grant SI3/PJI/2021-00505 from Comunidad de Madrid. V.C. was partially supported by grants PID2020- 117971GB-C21 funded by MCIN/AEI/10.13039/501100011033, US-1263032 (US/JUNTA/FEDER, UE), P20 01109 (JUNTA/FEDER, UE)  and FPU17/01871. J.J.G. was supported by 
grants PID2020-117971GB-C22 and CEX2020-001084-M funded by MCIN/AEI/10.13039/501100011033 and grant 2021-SGR-00697 funded by the Catalan Government.}
\author[F. Cantero]{Federico Cantero Morán}
\address{Federico Cantero Morán: Departamento de Matemáticas, Universidad Autónoma de Madrid \& ICMAT. Calle Francisco Tomás y Valiente, 7. 28049, Madrid, Spain}
\email{federico.cantero@uam.es}
\author[V. Carmona]{V\'ictor Carmona} 
\address{Víctor Carmona: Departamento de  Álgebra \& IMUS, Universidad de Sevilla. Avda Reina Mercedes, s/n, 41012, Sevilla, Spain}
\email{vcarmona1@us.es}
\author[J.J. Gutiérrez]{Javier J. Guti\'errez}
\address{Javier J. Guti\'errez: Departament de Matem\`atiques i Inform\`atica, Universitat de Barcelona. Gran Via de les Corts Catalanes 585. 08007, Barcelona, Spain}
\email{javier.gutierrez@ub.edu}
\date{\today}


\begin{abstract}
Action operads and cloning systems are, respectively, the main ingredients in Thumann's and Witzel--Zaremsky's approaches for axiomatically constructing Thompson-like groups. 
In this paper, we prove that action operads are equivalent to cloning systems that admit a certain extra structure, and which we call {\em restricted operadic} cloning systems. In addition, we describe their relation with crossed interval groups and product categories. 
\end{abstract}

\maketitle

\section{Introduction}
Graph neural networks (GNNs) have undergone rapid development and become increasingly popular for learning graph data \cite{welling2016semi, velivckovic2017graph, xu2018powerful}.
GNNs are usually trained in an end-to-end manner while getting enough labeled data is arduously expensive and sometimes even impractical to access. This motivates some recent advances in pre-training GNNs ~\cite{hu2019strategies,Hu2020GPTGNNGP,Qiu2020GCCGC,Lu2021LearningTP}. 
The key insight of pre-training GNNs is to learn transferable knowledge from a collection of unlabeled graph data, hoping that the learned knowledge can be easily adapted to downstream tasks.
In view of the great success of pre-training in other fields like computer vision and natural language processing~\cite{devlin2018bert,he2020momentum},  graph pre-training is {highly expected} to be an effective means to improve downstream performance.


\begin{figure}[t]
    \centering
    {\includegraphics[width=1\columnwidth]{figure/motivation.pdf}}
    \caption{Comparison of {existing methods} and {proposed W2PGNN} to answer \emph{when to pre-train} GNNs.}    
    \label{fig:example}
\end{figure}

However, the intuition that graph pre-trained model would ideally benefit the downstream is far from the truth in the area of graph pre-training.
Instead, graph pre-trained models can lead to \emph{negative transfer} on many downstream tasks, especially when the graphs used for pre-training are not necessarily from the same domain as the {downstream} data~\cite{hu2019strategies, Qiu2020GCCGC}.
For example, the closed triangles ($\vcenter{\hbox{\includegraphics[width=2.4ex,height=2.4ex]{figure/s2.pdf}}}$) and open triangles  ($\vcenter{\hbox{\includegraphics[width=2.4ex,height=2.4ex]{figure/s1.pdf}}}$) might yield different interpretations in molecular networks (unstable vs. stable in terms of chemical property) from those in social networks (stable vs. unstable in terms of social relationship); such distinct or reversed semantics does not contribute to transferability, and even exacerbates the problem of negative transfer.


To avoid the negative transfer, recent efforts focus on  \emph{what to pre-train} and \emph{how to pre-train},  \emph{i.e.}, design/adopt graph pre-training models with a variety of self-supervised tasks to capture different patterns~\cite{Qiu2020GCCGC,you2020graph,Lu2021LearningTP} and fine-tuning strategies to enhance downstream performance~\cite{Hu2019PreTrainingGN,Han2021AdaptiveTL,Zhang2022FineTuningGN,Xia2022TowardsEA}.
However, there do exist some cases that no matter how advanced the pre-training/fine-tuning method is, the transferability from pre-training data to downstream data still cannot be guaranteed. This is because the underlying assumption of deep learning models is that the test data should share a similar distribution as the training data.
Therefore, it is a necessity to understand \emph{when to pre-train}, \emph{i.e.}, under what situations the ``graph pre-train and fine-tune'' paradigm should be adopted.

Towards the answer of when to pre-train GNNs, one straight-forward way illustrated in Figure~\ref{fig:example}(a) is to train and evaluate on all candidates of pre-training models and fine-tuning strategies, and then the resulting best downstream performance would tell us whether pre-training
% ``pre-train and fine-tune'' 
is a sensible choice. If there exist $l_1$ pre-training models and $l_2$ fine-tuning strategies,  such a process would be very costly as you should make $l_1 \times l_2$ ``pre-train and fine-tune'' attempts.
Another approach is to utilize graph metrics to measure the similarity between pre-training and downstream data, \emph{e.g.}, density, clustering coefficient and etc. However, it is a daunting task to enumerate all hand-engineered graph features or find the dominant features that influenced similarity.
Moreover, the graph metrics only measure the pair-wise similarity between two graphs, which cannot be directly and accurately applied to the practical scenario where pre-training data contains multiple graphs.


In this paper, we propose a W2PGNN framework to answer
\emph{\underline{w}hen \underline{to} \underline{p}re-train \underline{GNN}s from a graph data generation perspective}.
% aim to address the problem of when to pre-train GNNs 
The high-level idea is that instead of performing effortful graph pre-training/fine-tuning or making comparisons between the pre-training and downstream data, we study the complex generative mechanism from the pre-training data to the downstream data (Figure~\ref{fig:example}(b)).
We say that downstream data can benefit from pre-training data (\emph{i.e.}, has high feasibility of performing pre-training), 
if it can be generated with high probability by a graph generator that summarizes the topological characteristic of pre-training data.



The major challenge is how to obtain an appropriate graph generator, hoping that it not only inherits the transferable topological patterns of the pre-training data, but also is endowed with the ability to generate feasible downstream graphs.
To tackle the challenge, we propose to design a graph generator based on graphons.
We first fit the pre-training graphs into different graphons to construct a \emph{graphon basis}, where each graphon (\emph{i.e.}, element of the graphon basis) identifies a collection of graphs that share common transferable patterns. We then define a \emph{graph generator} as {a convex combination of elements in a graphon basis}, which serves as a comprehensive and representative summary of pre-training data.  All of these possible generators constitute the \emph{generator space}, from which graphs generated form the solution space for the downstream data that can benefit from pre-training.

Accordingly, the feasibility of performing pre-training can be measured as the highest probability of downstream data being generated from any graph generator in the generator space, which can be formulated as an optimization problem.
However, this problem is still difficult to solve due to the large search space of graphon basis. We propose to reduce the search space to three candidates of graphon basis, \emph{i.e.,} topological graphon basis, domain graphon basis, and integrated graphon basis, to mimic different {generation mechanisms} from pre-training to downstream data. Built upon the reduced search space, the feasibility can be approximated efficiently.





Our major contributions are concluded as follows:
\begin{itemize}[leftmargin=*,topsep=0pt]
\item \textbf{Problem and method.} To the best of our knowledge, we are the first work to study the problem of when to pre-train GNNs. We propose a W2PGNN framework
to answer the question from a data generation perspective, which tells us the feasibility of performing graph pre-training before conducting effortful pre-training and fine-tuning.




\item \textbf{Broad applications.}
W2PGNN provides several practical applications: (1) provide the application scope of a graph pre-trained model, (2) measure the feasibility of performing pre-training for a downstream data
and (3) choose the pre-training data so as to maximize downstream performance with limited resources.


\item \textbf{Theory and Experiment.} 
We theoretically and empirically justify the effectiveness of W2PGNN.
Extensive experiments {on real-world graph datasets from multiple domains} show that the proposed method can provide an accurate estimation of pre-training feasibility and the selected pre-training data can benefit the downstream performance.


\end{itemize}




\section{Cloning systems}\label{section:cloning_systems}

In this section we offer a brief introduction to Witzel--Zaremsky's  cloning systems \cite{WZ}, and define their {\em bilateral} counterparts; we refer the interested reader to \cite{WZ,Zaremsky} for a detailed account on cloning systems. 

We start with a specific example, which appears as Example 2.9 in \cite{WZ}, and that will serve to establish some notation for the sequel. In what follows, $\Sigma_n$ stands for the symmetric group on $n$ elements. 

\begin{examp}[Cloning system for symmetric groups] Let $\Sigma_{\bullet}=\{\Sigma_n\}_{n\ge 1}$ be the family of symmetric groups. For every $n\ge 1$, let $\lambda_{n}:\Sigma_n \to \Sigma_{n+1}$ be the injective homomorphism obtained by fixing the last element, that is,
$$
\lambda_n(\sigma)(i)=\sigma(i) \mbox{ for $1\le i\le n$ and } \lambda_n(\sigma)(n+1)=n+1,
$$
for every $\sigma\in\Sigma_n$. For every $n\ge 1$ and $1\le j\le n$, let $c^n_j: \Sigma_n\to \Sigma_{n+1}$ be the injective map given by, thinking about permutations pictorially as strand diagrams, ``repeating'' the $j$-th strand; that is, 
$$
c_j^n(\sigma)(i)=
\left\{
\begin{array}{ll}
\sigma(i) & \mbox{ if $i\le j$ and $\sigma(i)\le\sigma(j)$,} \\
\sigma(i)+1 & \mbox{if $i<j$ and $\sigma(i)>\sigma(j)$,}\\
\sigma(i-1) & \mbox{if $i>j+1$ and $\sigma(i-1)<\sigma(j)$,}\\
\sigma(i-1)+1 & \mbox{if $i\ge j+1$ and $\sigma(i-1)\ge \sigma(j)$,}
\end{array}
\right.
$$
for every $\sigma\in\Sigma_n$.
As shown in \cite[Example 2.9]{WZ} the families of morphisms $\lambda$ and $c$ interact with each other, and satisfy certain obvious compatibility properties, detailed in \cite[Proposition 2.6]{WZ}. The \emph{cloning system} for the family of symmetric groups is the triple $(\Sigma_\bullet, \lambda, c)$, subject to these compatibility conditions. 
\end{examp}

The maps $c_j$ above are called {\em cloning maps}, for obvious reasons; note that they are not group homomorphisms. The notion of a cloning system is a generalization of the above example to arbitrary families of groups. 


\begin{defn}\label{defn:CloningSystem}A \emph{cloning system} is a quadruple $(\G_{\bullet},\iota,\kappa,\pi)$, where
\begin{itemize}
    \item $\G_{\bullet}=\{\G_n\}_{n\geq 1}$ is a family of groups,
    \item $\iota=\{\iota_n\colon\G_n\to\G_{n+1}\}_{n\ge 1}$ is a family of injective homomorphisms,
    \item $\kappa=\{\kappa^n_{j}\colon\G_{n}\to\G_{n+1}\}_{n\ge 1,\,1\leq j\leq n}$ is a family of maps, called \emph{cloning maps}, and
    \item $\pi= \{\pi_n \colon \G_n\to \Sigma_n\}_{n\ge 1}$ is a  homomorphism,
\end{itemize}
 subject to the following compatibility conditions:
\begin{itemize}
    \item[(i)]\label{it:i} $\pi_{n+1} \circ \iota_n=\lambda_n \circ \pi_n$, for all $n\ge 1$;  
    \item[(ii)]\label{it:ii} $(\pi_{n+1}(\kappa^n_j(g)))(i) = (c^n_j(\pi_n(g))(i)$, for all $n\ge 1$ and all $1\le j\le n$, all $i\neq j,j+1$, all $g\in G_n$;
    \item[(iii)]\label{it:iii} $\iota_{n+1}\circ\kappa^n_j= \kappa_j^{n+1}\circ \iota_n$, for all $n\ge 1$ and all $1\le j\le n$;
    \item[(iv)]\label{it:iv} $\kappa^{n+1}_{j+1} \circ \kappa_l^n = \kappa_l^{n+1} \circ \kappa_{j}^{n}$, for all $n$ and all $l< j\le n$;
    \item[(v)]\label{it:v} $\kappa^n_j(g\cdot h)=\kappa^n_{\pi_n(h)(j)}(g)\cdot\kappa^n_j(h)$, for all $n$, all $g,h \in \G_n$ and all $j\le n$.
\end{itemize}
\end{defn}
%\jg{He quitado la condición de que las $\kappa$'s sean inyectivas después de hablarlo con Federico. Aunque en el artículo del user's guide lo pide, en el Witzel-Zelemsky no lo hace y para la comparación con action operads y demás, es mejor que no lo sean.}
\begin{rem} It is important to note some differences between Definition \ref{defn:CloningSystem} and the definition of cloning system of~\cite{WZ,Zaremsky}. First, we use a functional convention for composition of maps, that is, the composition of two functions $f\colon X\to Y$ and $g\colon Y\to Z$ is denoted by $g\circ f$, defined as $(g\circ f)(x)=g(f(x))$. Second, 
the original definition of cloning system in \cite{Zaremsky} requires injective homomorphisms $\iota_{n,m}\colon \G_{n}\to \G_{m}$ for every $m> n\geq 1$; however, our Definition \ref{defn:CloningSystem} implies that it suffices to consider injective maps $\iota_{n,n+1}=\iota_n$ for all $n\geq 1$.  
\end{rem}

Witzel and Zaremsky observe that cloning systems often satisfy the following strengthed version of these axioms. 
\begin{itemize}
   	\item[(iv+)] $\kappa^{n+1}_{j+1}\circ \kappa^n_{j} = \kappa_{j}^{n+1}\circ \kappa_j^n$, for all $1\le j\le n$;
    \item[(vii)]\label{it:iii'} $\iota_{n+1}\circ \iota_n = \kappa_{n+1}^{n+1}\circ \iota_n$, for all $n\ge 1$;    
\end{itemize}
\begin{rem}
    Condition (vii) was part of the original definition of cloning system: In~\cite[Definition 2.18]{WZ} the cloning maps $\kappa$ are required to be a \emph{family of cloning maps}, which must satisfy two conditions presented at the beginning of page 18 in that paper. These two conditions correspond to Conditions (iii) and (vii) in this article.

In \cite[page 17]{WZ} (see also \cite{Bri07}) the Hedge monoid ${\mathcal H}$ is introduced, together with a surjective map ${\mathcal F}\to {\mathcal H}$ from the monoid of forests to the monoid of hegdes. The colimit of the groups in a cloning system comes with an action of ${\mathcal F}$. Condition~(iv+) in this article is equivalent to require that that action factors through the hegde monoid (which holds for most examples; see Observation 2.11 and paragraph before Observation 2.19 in \cite{WZ}).

\end{rem}


We now introduce the notion of a \emph{bilateral} cloning system. In a nutshell, in the same way that the maps $\iota$ of Definition \ref{defn:CloningSystem} informally correspond to ``adding elements on the right'', a bilateral cloning system comes equipped with  a family of ``dual'' maps $\zeta$ that correspond to ``adding elements on the left''. 

We now proceed to formalize this idea. For the family of symmetric groups, we denote by $\rho_n: \Sigma_n \to \Sigma_{n+1}$ the injective homomorphism that fixes the first element, that is, 
$$
\rho_n(\sigma)(i)=\sigma(i-1)+1 \mbox{ for $2\le i\le n+1$ and } \rho_n(\sigma)(1)=1,
$$
for every $\sigma\in\Sigma_n$.


\begin{defn}\label{defn:NCCloningSystem} A \emph{bilateral cloning system} is a quintuple $(\G_{\bullet},\iota, \zeta, \kappa,\pi)$, where $(\G_{\bullet},\iota,\kappa, \pi)$ is a cloning system satisfying conditions (iv+) and (vii), and $\zeta=\{\zeta_n\colon \G_n\to\G_{1+n}\}_{n\ge 1}$ is an additional family of injective homomomorphisms satisfying the following conditions:
\begin{itemize}
    \item[(i')] $\pi_{n+1} \circ \zeta_n=\rho_{n}\circ \pi_n $, for all $n\ge 1$;  
    \item[(iii')] \label{it:iiib} $\zeta_n\circ \kappa_j^n = \kappa_{j+1}^{n+1}\circ\zeta_{n}$, for all $n\ge 1$ and all $1\le j\le n$;
    \item[(vi)] $\zeta_{n+1}\circ \iota_n = \iota_{n+1}\circ\zeta_{n}$, for all $n\ge 1$;
    \item[(vii')]\label{it:iiib'} $\zeta_{n+1}\circ \zeta_n = \kappa_{1}^{n+1}\circ \zeta_n$, for all $n\ge 1$;
 \end{itemize}
 A bilateral cloning system is called \emph{restricted} if it additionally satisfies the following condition:
 \begin{itemize}
        \item[(ii+)] $\pi_{n+1} \circ \kappa^n_j = c^n_j \circ \pi_n$, for all $n\ge 1$ and all $1\le j\le n$;
 \end{itemize}
\end{defn}

\begin{defn} A bilateral cloning system is called \emph{operadic} if the following additional conditions are satisfied:
\begin{itemize}
	\item[(viii)]$\kappa_i^n(m)(g)\cdot \nu_i^m(n)(h) = \nu^m_{\pi(g)(i)}(n)(h)\cdot \kappa^n_i(m)(g)$, for all $m,n\geq 0$ and all $g\in \G_n$ and $h\in\G_m$;
	\item[(ix)] $\iota_n(m)(g)\cdot \zeta_m(n)(h)=\zeta_m(n)(h)\cdot \iota_n(m)(g)$, for all $m,n\geq 0$ and all $g\in\G_n$ and $h\in\G_m$.
\end{itemize}
\end{defn}


The morphisms $\iota_n(m)$, $\zeta_m(n)$, $\kappa_j^n(m)$ and $\nu_j^m(n)$ appearing in conditions (viii) and (ix) are the maps
$$
\iota_n(r)\colon\G_{n}\longrightarrow\G_{n+r},\quad \kappa^n_j(m)\colon \G_{n}\longrightarrow\G_{n+m}, \quad
\zeta_n(l)\colon\G_{n}\longrightarrow\G_{l+n}, 
$$
 and 
$$
\nu^n_j(m)\colon\G_{n}\to\G_{n+m-1}
$$
defined by  $\iota_n(r)=\iota_{n+r-1}\circ\cdots\circ\iota_{n}$ for all $n,r\ge 1$; $\kappa_j^n(m)=\kappa_j^{n+m-2}\circ\cdots\circ \kappa_j^{n}$ for all $n,m\ge 1$ and all $1\le j\le n$; $\zeta_n(l)=\zeta_{n+l-1}\circ\cdots\circ\zeta_{n}$ for all $n,l\ge 1$; and  $\nu_j^n(m)=\zeta_{m+n-j}(j-1)\circ\iota_n(m-j)=\iota_{n+j-1}(m-j)\circ\zeta_n(j-1)$ for all $j,m,n\ge 1$. Note that the morphisms $\zeta_n(l)$ and $\nu_j^n(m)$ only make sense for bilateral cloning systems.


\begin{rem}
Although in Definition~\ref{defn:CloningSystem} and Definition~\ref{defn:NCCloningSystem} we require  the maps  $\iota$ and $\zeta$ to be injective, this  is not essential for the comparison of bilateral cloning system and action operads, as we will see in the next section. 
\end{rem}


In order to get a grip on the intuition behind the definitions above, we next describe perhaps the primordial example of a (bilateral) cloning system, namely that of braid groups; we refer the reader to \cite{WZ} for details.   

\begin{examp}[Braid groups]\label{examp:cloning}
    
 Let $\Br_n$ denote the braid group on $n$ strands. For every $n\ge 1$ there is a canonical surjective group homomorphism $\pi_n\colon\Br_n\to\Sigma_n$ by sending each braid to its underlying permutation. We also have inclusion maps $\iota_n\colon \Br_n\to\Br_{n+1}$ corresponding to ``adding one strand on the right'', and cloning maps
$$
\kappa_{j}^n\colon \Br_n\longrightarrow \Br_{n+1}
$$
given by duplicating the $j$-th strand into two parallel strands; see Figure \ref{fig:Kappa Braid} for an example, and \cite[Example~3.3]{Zaremsky} for details.
\begin{figure}[htp]
    \centering
    \includegraphics[width=6cm]{figures/kappa_Br.pdf}
    \caption{Cloning map $\kappa_2^4$ on $\Br_4$.}
    \label{fig:Kappa Braid}
    \end{figure}

Together, the three families of maps defined above endow the collection of all braid groups $\Br_{\bullet}=\{\Br_n\}_{n\ge 1}$ with the cloning system structure $\Br=(\Br_{\bullet},\iota,\kappa,\pi)$, see \cite{WZ} for a proof. 

Moreover, in analogy with the maps $\iota_n$, we can also define inclusion maps $\zeta_n\colon \Br_n\to\Br_{n+1}$ that informally correspond to ``adding one strand on the left''. Equipped with these maps, one readily checks that $\Br=(\Br_{\bullet},\iota,\zeta,\kappa,\pi)$ is a bilateral cloning system. For illustrative purposes, Figures \ref{fig:AxiomVI} to \ref{fig:AxiomXI} depict particular instances of some of the conditions of the bilateral cloning system structure of $\Br_{\bullet}$.
\end{examp}  
 \begin{figure}
 \centering
 \includegraphics[width=7cm]{figures/NonChiral_additional_axioms_vi.pdf}
    \caption{Condition (vii), $\kappa_3^3\circ \iota=\iota\circ\iota$.}
    \label{fig:AxiomVI}
    \end{figure}
 \begin{figure}
    \centering
    \includegraphics[width=7cm]{figures/NonChiral_additional_axioms_vii.pdf}
    \caption{Condition (vi), $\iota\circ \zeta=\zeta\circ \iota$.}
    \label{fig:AxiomVIII}
    \end{figure}
\begin{figure}
    \centering
    \includegraphics[width=10cm]{figures/NonChiral_additional_axioms_x.pdf}
    \caption{Condition (viii).}
    \label{fig:AxiomX}
    \end{figure}
 \begin{figure}
    \centering
    \includegraphics[width=10cm]{figures/NonChiral_additional_axioms_xi.pdf}
    \caption{Condition (ix).}
    \label{fig:AxiomXI}
    \end{figure}

Other examples of bilateral cloning systems are the {\em mock symmetric groups} and the {\em loop braid groups} (also known as {\em symmetric automorphisms of free groups}); see \cite{WZ}; as well as the {\em signed symmetric groups} and the {\em twisted braid groups} \cite{Zaremsky}. 
The latter two bilateral cloning systems are not restricted.
\medskip

Next, we discuss two examples of (bilateral) cloning systems from \cite{WZ} and \cite{Zaremsky} which, as we will see, are not operadic.

\begin{examp}[Direct powers]\label{examp:DirectPowers}
Let $G$ be a group and denote by $G^n$ the $n$-fold direct product of $G$ with itself. Write  $\iota_n: G^n \to G^{n+1}$ for the map that adds the identity (in $G$) as last entry, let $\pi_n: G^n \to \Sigma_n$ the trivial homomorphism, and consider the map $\kappa^n_j:G^n \to G^{n+1}$ that duplicates the $j$-th entry. Then, the quadruple $(\{G^n\}_{n\ge 1}, \iota, \pi, \kappa)$ is  a cloning system on the set of direct powers of $G$. 

Despite the fact that there is an obvious map $\zeta_n: G^n \to G^{n+1}$ that adds the identity (in $G$) as first entry, one may check that the maps $\kappa$ and $\zeta$ do not satisfy condition (viii) of the definition of an operadic bilateral cloning system.
\end{examp}


\begin{examp}[Upper triangular matrices]\label{examp:UpperTriangularMatrices}
Let $U_n$ denote the group of invertible $n\times n$ upper triangular matrices with real coefficients. Consider the obvious inclusion map $\iota_n: U_n \to U_{n+1}$ given by adding a 1 as the lowermost element on the diagonal. 

There are cloning maps $\kappa_j^n: U_n \to U_{n+1}$ that informally correspond to a certain duplication of the $j$-th column that preserves the upper triangular structure of the matrix, and which becomes apparent just by giving the following particular example; see \cite{Zaremsky} for details: 

\[
\kappa^3_2\begin{pmatrix}
1 & 2 & 3 \\ 0 & 4 & 5 \\ 0 & 0 & 6
\end{pmatrix}  = 
\begin{pmatrix}
1 & 2 & 2 & 3 \\ 0 & 4 & 0 & 0 \\ 0 & 0 & 4 & 5 \\ 0 & 0 & 0 & 6
\end{pmatrix}
\]
Setting $\pi_n: U_n \to \Sigma_n$ to be the trivial homomorphism, the set $U_\bullet = \{U_n\}_{n \ge 1}$ acquires the cloning system structure $U = (U_\bullet, \iota, \kappa, \pi)$. 

Observe that one could define another inclusion map $\zeta_n: U_n \to U_{n+1}$ by adding a 1 as the uppermost element of the diagonal; however, as in the previous example, the interaction of the maps $\zeta$ and $\kappa$ does not satisfy condition (viii) of the definition of a bilateral cloning system.
\end{examp}



\section{Action operads}\label{section:action_operads}
We start this section recalling the classical notion of an {\em operad}: 




\begin{defn}\label{defn:Operad} 
A \emph{symmetric operad} $\EE$ on sets is a triple $(\EE, \{\circ_i\}_i, \id)$, where
\begin{itemize}
\item $\EE=\{\EE(n)\}_{n\ge 1}$ is a family of sets, and each $\EE(n)$ is equipped with a right $\Sigma_n$-action, for every $n\ge 1$,
\item ${\id}\in \EE(1)$ is called the \emph{unit} of the operad, and \item $\{\circ_i\}_i$ is a family of maps
$$
\circ_i\colon \EE(n)\times \EE(m)\to \EE(n+m-1) \quad\mbox{for all $n,m\ge 1$ and $1\le i\le n$},
$$ called \emph{$\circ_i$-operations} or \emph{partial composition products},
such that ${\id} \circ_i y=y$ and $x\circ_i {\id}=x$ for every $x,y\in \EE(n)$. Moreover, these $\circ_i$\nobreakdash-operations satisfy certain associativity and equivariance axioms, which are spelled out in, for example, \cite[Definition 11]{Markl}.
\end{itemize}


A \emph{morphism of operads} $f\colon\EE\to \mathsf{P}$ consists of maps $f_n\colon \EE(n)\to \mathsf{P}(n)$ for $n\ge 1$, that are compatible with the unit and $\circ_i$-operations of $\EE$ and $\mathsf{P}$. If we forget about all the symmetric group actions on the sets $\EE(n)$, we have the notion of a \emph{non-symmetric operad}.
\end{defn}

\begin{rem} Note that Definition \ref{defn:Operad} avoids nullary operations in an operad, i.e.\ there is no $\EE(0)$ in $\EE$. In this work, we will only consider operads without nullary operations, or in other words, without constants. This choice is not essential, but it is made to have a clearer connection between action operads and cloning systems.
\end{rem}

\begin{examp}[Symmetric groups]
The family of symmetric groups $\Sigma_{\bullet}=\{\Sigma_n\}_{n\ge 1}$ has the structure of a non-symmetric operad in sets, with $\id\in\Sigma_1$ the trivial permutation of $\Sigma_1$. The partial composition products $\circ_i\colon\Sigma_n\times \Sigma_m\to \Sigma_{m+n-1}$ are defined as follows: if $\sigma\in \Sigma_n$ and $\tau\in\Sigma_m$, then $\sigma\circ_i\tau$ is the permutation of $\Sigma_{m+n-1}$ obtained by ``inserting'' $\tau$ in $\sigma$ at the $i$-th place as a block and rearranging the indices accordingly. Figure~\ref{fig:CompositionProduct Sigma} shows an example of the composition product $\circ_2\colon\Sigma_3\times \Sigma_2\to\Sigma_4$.
\begin{figure}[h]
    \centering
    \includegraphics[width=5cm]{figures/CompositionProduct_Sigma.pdf}
    \caption{\scriptsize
    $\left(\begin{matrix}
     1 & 2 & 3\\
     2 & 3 & 1
    \end{matrix}\right)\circ_2
    \left(\begin{matrix}
     1 & 2 \\
     2 & 1
    \end{matrix}\right)
    = 
    \left(\begin{matrix}
     1 & 2 & 3 & 4\\
     2 & 4 & 3 & 1
    \end{matrix}\right)$.}
    \label{fig:CompositionProduct Sigma}
    \end{figure}
\end{examp}

\subsection{Action operads}
We now introduce the concept of an {\em action operad}, which essentially is a family of groups satisfying certain properties that allow to define operads with equivariance relative to this family; the reader is referred to \cite{Corner-Gurski} for a detailed treatment of action operads and their properties. They have been also studied under the name \emph{group operads} \cite{Zhang, Yoshida, Yau}

\begin{defn}\label{defn:ActionOperad} An \emph{action operad without constants} $\G$, or simply an \emph{action operad}, is a quadruple $(\G_{\bullet},\pi,\{\circ_i\}_i,\id)$, where
\begin{enumerate}
    \item\label{cond:ac:1} $\G_\bullet=\{\G_n\}_{n\ge 1}$ is a family of groups, and $(\G_{\bullet},\{\circ_i\}_i,\id)$ is a non-symmetric operad on sets, where $\{\circ_i\}_i$ are the partial composition products of the operad and $\id\in \G_1$ is the unit. The associativity of the partial composition products yields that if $f\in \G_n$, $g\in \G_m$ and $h\in \G_l$, then we have that
\begin{align*}
(f\circ_i g)\circ_j h &= \begin{cases}
(f\circ_j h) \circ_{i+l-1}g & \text{if $j<i$},\\
 (f \circ_{j-m+1} h)\circ_i g& \text{if $j\geq i+m$},\\
f\circ_i (g\circ_{j-i+1}h) & \text{if $j=i,\ldots, i+m-1$},
\end{cases} \\
f\circ_i \id &= f, \\
\id\circ_1 g &= g;
\end{align*}
    \item\label{cond:ac:2} $\pi: \G_\bullet \to \Sigma_{\bullet}$ is a map of operads which is also a levelwise group homomorphism, that is, $\pi_n\colon\G_n \to \Sigma_n$ is a homomorphism for all $n\ge 1$;
\item\label{cond:ac:3} For every $f, f' \in \G_n$, $g,g'\in \G_{m}$, we have that
\begin{align*}
(f\cdot f')\circ_i(g\cdot g') &= (f\circ_{\pi(f')(i)}g)\cdot (f'\circ_i g'),
\end{align*}
with the multiplication taking place in the group $\G_{n+m-1}$.
\end{enumerate}
\label{def:actionoperad}
\end{defn}

 Note that the partial composition products $\circ_i$ are not group homomorphisms in general, that action operads are not assumed to be symmetric operads, and that they have no nullary operations. It follows from the axioms that the unit element $\id\in \G_1$ of the operad $\G_{\bullet}$ is precisely the unit $e_1$ of the group $\G_1$.

 Observe that, both in a cloning system and in an action operad, the group $\G_n$ acts on the set $\{1,\ldots,n\}$ via the homomorphism $\pi_n$ for every $n$. If these maps $\pi$ are understood from the context, for $g\in \G_n$ we will write $g(i)$ instead of $\pi_n(g)(i)$.


Finally, note that an action operad with trivial $\pi$ is the same thing as a non-symmetric operad on  groups.    

\subsection{General action operads} Inspired by the definition of cloning system, we relax as follows the definition of action operad. Note that this relaxation does not diminish the hability of such an operad to hold the equivariance of other operads. In fact, we believe that this should be the adequate definition of action operad. 
\begin{defn}
    A \emph{general action operad} $\G$ is a tuple $(\G_\bullet,\pi,\{\circ_i\}_i,\id)$ satisfying Conditions \eqref{cond:ac:1} and \eqref{cond:ac:3} together with the following:
    \begin{itemize}
        \item[(2')] $\pi_n\colon \G_n\to \Sigma_n$ is a levelwise group homomorphism for each $n\geq 1$ such that if $g\in \G_n$ and $h\in \G_m$, then the action of the symmetric group elements $\pi(g)\circ_i \pi(h)$ and $\pi(g\circ_i h)$ coincide on $$\{1,2,\ldots,i-1,i+m,i+m+1,\ldots,n+m-1\}.$$
    \end{itemize}
\end{defn}
Here is an example of general action operad that is not an action operad. It corresponds to the cloning system of signed symmetric groups or to the hyperoctahedral inert demi-interval group (see Section \ref{section:crossed}).

The group $\Sigma^{\pm}_n$ is the signed symmetric group, that consist on permutations $g$ of the set$\{1,-1,2,-2,\ldots,n,-n\}$ such that $g(i) = - g(-i)$. The composition $g\circ_i h$ is defined as follows: if $g(+i)$ is positive, then $g\circ_i h$ is obtained by cabling the signed symmetry $h$ in the $i$-th strand of the permutation $g$. If $g(+i)$ is negative, then $g\circ_i h$ is obtained by cabling the signed symmetry $h'$ in the $i$-th strand of $g$, where $h'(j) = -h(n-j+1)$ for a positive $j$. A signed permutation $g$ is determined by the pair $(\sigma,A)$, where $\sigma$ is the underlying permutation and $A$ is the set of $i$'s such that $g(+i)$ is negative. With this notation, the composition of $(g,A)$ and $(g',A')$ is $(g\circ g',g'(A)\triangle A')$, where the symbol $\triangle$ denotes symmetric difference.

From the classification of inert crossed interval groups and the relation between action operads and crossed interval groups, it is reasonable to think that the signed symmetric general action operad is final in the category of general action operads. This allows to simplify as follows the definition of general action operad:
\begin{lem}
    If we require that the target of the map $\pi$ is the signed symmetric group instead of the symmetric group, then
    Condition {\rm (2')} in the definition of general action operad can be replaced by the following: 
    \begin{itemize}
        \item[(2'')] $\pi: \G_\bullet \to \Sigma^{\pm}_{\bullet}$ is a map of operads and a levelwise group homomorphism. 
    \end{itemize}
\end{lem}
\begin{proof} Suppose that $\pi\colon \G_n\to \Sigma_n$ is a family of maps satisfying Condition (2'). We will define a family of maps $\pi'\colon \G_n\to \Sigma^{\pm}_n$ satisfying Condition (2'').

We first define the map $\pi'_n\colon \G_n\to \Sigma^{\pm}_n$. If $g\in \G_n$, define $\pi'(g) = (\pi(g),A)$ with $A = \{j\mid \pi(g\circ_2 e_2)\neq \pi(g)\circ_2 e_2\}$. To check that $\pi'$ is a group homomorphism, we have that, if $g,h\in \G_n$ and $\pi'(g) = (\sigma,A)$ and $\pi'(h) = (\sigma',A')$, and $\pi'(g\cdot h) = (\sigma'',A'')$, then $\sigma'' = \sigma\cdot \sigma'$ and
    \[
        (g\cdot h)\circ_j e_2 = (g\cdot h)\circ_j (e_2\cdot e_2) = (g\circ_{\sigma'(j)} e_2)\cdot (g\circ_2 e_2)
    \]
    The product $\pi((g\cdot h)\circ_j e_2)$ is either equal to $\pi(g\cdot h)\circ_j e_2$ or differs from it by a twist of the entries $\{j,j+1\}$. Inspection shows that $A'' = g'(A)\triangle A'$.

    One still has to show that $\pi'$ is an operad map. By the multiplication rule it is enough to check it for the compositions $g\circ_j e_m$. Since the latter are iterated compositions of the form $g\circ_2 e_2$, it is enough to check it in this last case, in which is true by definition.

    Finally, if $\pi'\colon \G_n\to \Sigma_n^\pm$ is a family of maps satisfying Condition (2''), then composing them with the homomorphism $\Sigma_n^{\pm}\to \Sigma_n$ that forgets the signs yields a family of maps satisfying $(2')$.
\end{proof}

\begin{rem}
    This lemma has its counterpart in the world of cloning systems: Condition (ii) in the definition of cloning system can be replaced by the following:
    \begin{itemize}
        \item[(ii')] $\pi_n\colon \G_n\to \Sigma_{n}^\pm$ is a levelwise group homomorphism such that the identity $\pi_{n+1}\circ \kappa_j^n = c_j^n\circ \pi_n$ holds for all $n\geq 1$ and all $1\leq j\leq n$,
    \end{itemize}
where $c_j^n$ are the cloning maps of the signed symmetric cloning system. In order the make sense of axioms (v) and (viii), we are implicitly using the fact that $\Sigma^\pm_n$ acts on $\{1,2,\ldots,n\}$ through the homomorphism $\Sigma_n^{\pm}\to \Sigma_n$ that forgets the signs.
\end{rem}


\subsection{Fundamental groups of $\G$-operads and  unoriented ribbon braids}

Fix an (general) action operad $\G$. Then, one may define a $\G$-operad as in \cite[Def.~1.14]{Corner-Gurski}, \cite[Def.~4.2.6, Prop. 4.3.1]{Yau} or \cite[Def.~2.30]{Zhang}, i.e., as a non-symmetric operad $\EE$ with an action $\EE(n)\times \G_n\to \EE(n)$ such that the composition
\[
    \EE(n)\times \big(\EE(n_1)\times \ldots\times \EE(n_k)\big)\lra \EE(n_1+\ldots+n_k)
\]
is equivariant with respect to the map
\[
    \G_n\times (\G_{n_1}\times \ldots\times \G_{n_k})\lra \G_{n_1+\ldots+n_k}.
\]
Alternatively, if one considers operads with identities as we do, one requires the composition $\circ_i\colon\EE(n)\times \EE(m)\to \EE(n+m-1)$ to be equivariant with respect to the map $\circ_i\colon\G_n\times \G_m\to \G_{n+m-1}$. 
%\footnote{\fc{Esto ya no hace falta, y quedaba un poco offtopic} If $M$ is a manifold and $f\colon L\to M$ is a fibre bundle, the configuration space $C_n(M;L)$ is the quotient of the space $\{(x_1,\ldots,x_n)\in L^n\mid f(x_i)\neq f(x_j)\}$ by the action of the symmetric group $\Sigma_n$. Consider the following fibre bundles:
%\begin{itemize}
%    \item The trivial bundle $M\to M$.
%    \item $S(T\bC) = \{(p,v)\mid p\in \bC, v\in T_p \bC, \|v\| = 1\}$, the unit sphere bundle of the tangent bundle of the complex plane $\bC$.
%    \item $P(T\bC) = \{(p,L)\mid p\in \bC, L\in P(T_p \bC)\}$ is the fibrewise projective bundle of the tangent bundle of $\bC$.
%\end{itemize}
%The fundamental group of $C_n(\bC;\bC)$ is the braid group $\Br_n$, the fundamental group of $C_n(\bC;S(T\bC))$ is the \emph{ribbon braid group} $\RBr_n$ and the fundamental group of $C_n(\bC;P(T\bC))$ is the \emph{unoriented ribbon braid group} $\URBr_n$. 

%The ordered configurations in $\bC$ and their framed counterpart are homotopy equivalent to the spaces of operations in the little $2$-discs operad $\C(2)$ and the framed little $2$-discs operad $\C^\fr(2)$. These symmetric operads can be endowed with a good basepoint, and therefore the fundamental groups of their quotients by the symetric group are action operads \cite[Theorem~3.4]{Zhang} 

%The ordered configurations with labels in $P(T\bC)$ do not form an operad, because the information of a line through the origin is not enough to yield an identification of a neighbourhood of each point with $\bC$.

%\vc{DEBERIA DAR UN OPERAD IGUALMENTE, NO CREO QUE HAYA NINGUN PROBLEMA CON $P(T\bC)$.}
%\fc{Yo creo que no es un operad}
%\vc{$\{\Conf_n(\bC)\}_{n}$ corresponde con little discs con trivializacion del fibrado tangente, es decir el fibrado principal sobre el grupo trivial (piénsalo como $\{\Emb^{\mathsf{fr}}(\bC^{\sqcup n},\bC)\}_n$ y está claro porqué es un operad). $\{\Conf^{\mathsf{fr}}_n(\bC)\}_{n}$, que lo estamos llamando framed pero yo prefiero llamarlo oriented, corresponde con el fibrado principal con grupo $\mathsf{SO}(2)=\mathbb{S}^1$ (piénsalo como $\{\Emb^{\mathsf{or}}(\bC^{\sqcup n},\bC)\}$). El que falta es el que correspondería con el grupo proyectivo $\mathsf{PSO}(2)$ (cociente de $\{\pm \id\}\subset \mathsf{SO}(2)$ si no me equivoco) y por tanto con los embeddings que preservan esta estructura proyectiva salvo homotopía. Esto tiene sentido?? 
%
%Esto me lleva a pensar que el unoriented ribbon braid operad se podría definir como $\{\pi_1\big(\Emb^{\mathsf{pso}}(\bC^{\sqcup n},\bC)/\Sigma_n\big)\}_n$.
%}

% Nonetheless, we will see that the fundamental groups of $C_n(\bC;P(T\bC))$ do form an operad because the loop has the missing information to build a canonical identification of a neighbourhood of each point with $\bC$.
%}

Our goal in this subsection is to provide more examples of (general) action operads. For that purpose, we briefly discuss a construction based on fundamental groups of topological $\G$-operads that produces (general) action operads over $\G$.

%Due to applications of the fundamental group functor, we need choices of basepoints compatible with the $\G$-action and the composition products. That is the reason for the following detour.

First, observe that the forgetful functor from $\G$-operads to non-symmetric operads admits a left adjoint which simply adds a free  right $\G$-action,
 $$
(-)_{\G}\colon \begin{Bmatrix}
\text{non-symmetric}\\
\text{operads}
\end{Bmatrix}  \longrightarrow \begin{Bmatrix}
\G\text{-operads}
\end{Bmatrix}, \quad \mathsf{P}\mapsto \mathsf{P}_{\G},
 $$
 where $\mathsf{P}_{\G}(n)=\mathsf{P}(n)\times \G_{n}$ and the operation $\circ_i$ is defined as the composition
 $$
 \begin{tikzcd}[ampersand replacement=\&]
     \mathsf{P}_{\G}(n)\times \mathsf{P}_{\G}(m)\ar[d,"\cong"',"\text{switch}"] \\ 
     \mathsf{P}(n)\times \mathsf{P}(m)\times \G_{n}\times\G_{m}
     \ar[d,"\circ_i\times \circ_i"] \\ \mathsf{P}(n+m-1)\times \G_{n+m-1}\ar[d, equal]\\
     \mathsf{P}_{\G}(n+m-1)
 \end{tikzcd}\quad .
 $$
Of course, the same discussion applies to topological operads.
 
 Applying this left adjoint functor to the (non-symmetric) associative operad $\mathsf{As}$, characterized by $\mathsf{As}(n)=*$ for any $n\geq 1$ (and $\mathsf{As}(0)=\emptyset$), we obtain $\mathsf{As}_{\G}$. A map from $\mathsf{As}_{\G}$ selects basepoints in a coherent way in a $\G$-operad. In fact, 
 \begin{defn} Let $\EE$ be a topological $\G$-operad. Then, a \emph{good $\G$-basepoint} for $\EE$ is a map of topological $\G$-operads $\eta\colon\mathsf{As}^h_{\G}\to \EE$, where $\mathsf{As}^h_{\G}$ is a topological $\G$-operad homotopy equivalent to $\mathsf{As}_{\G}$.
 \end{defn}

We assume without loss of generality that for any $k\geq 1$ there is a choice of basepoint $\mu_k\in \mathsf{As}_{\G}^{h}(k)$ representing the plain $k$-ary multiplication, that is, lying in the connected component corresponding to $(*,e_k)\in \mathsf{As}_{\G}(k)$. By definition, there are paths $\mu_n\circ_i\mu_m\simeq \mu_{n+m-1}$ in $ \mathsf{As}_{\G}^h(n+m-1)$ and homotopies relating natural concatenations of those paths. For that reason, we fix basepoints $\epsilon_k:=\eta(\mu_k)\in \EE(k)$ for paths and loops as \cite[Section 3]{Zhang} in the sequel.

%Through $\eta\colon \mathsf{As}_{\G}^{h}\to \EE$, we transfer this structure to $\EE$ obtaining the data required to deal with the basepoints $\epsilon_k:=\eta(\mu_k)\in \EE(k)$ when considering paths and loops in $\EE(k)$ as done in \cite[Section 3]{Zhang}. \fc{la frase "we transfer this structure" es un poco ambigua. ¿Qué estructura exactamente se está preservando? ¿Es necesario señalarlo?}

Now, assume that the action of $\G_n$ on $\EE(n)$ is a covering action (see \cite[Section 1.3]{Hatcher}) and that $\EE(n)$ is path-connected. Then, we have a homotopy fiber sequence $\G_n\hookrightarrow \EE(n)\twoheadrightarrow \EE(n)/\G_n$, which induces a short exact sequence of groups
\begin{equation}\label{eqt:ses pi1 of Goperads}
    1\longrightarrow \pi_1\big(\EE(n),\epsilon_n\big)\longrightarrow \pi_1\big(\EE(n)/\G_n,[\epsilon_n]\big)\overset{\delta_n}{\longrightarrow} \G_n \longrightarrow 1,
\end{equation}
and an isomorphism of relative homotopy groups
\begin{equation}\label{eqt:pi1 of Goperads}
    \pi_1\big(\EE(n)/\G_n,[\epsilon_n]\big)\cong \pi_1\big(\EE(n);\G_n\!\cdot\, \epsilon_n,\epsilon_n\big).
\end{equation}
The left-hand side in (\ref{eqt:pi1 of Goperads}) has a group structure, while the right-hand side is easily seen to form an operad. Altogether, we have the following generalization of \cite[Theorem 3.4]{Zhang}:% (see also Theorem 3.8):
 \begin{prop} Let $(\G,\pi,\circ_i,\id)$ be an (general) action operad, $\EE$ be a topological $\G$-operad without constants so that: {\rm (i)} $\EE$ is equipped with a good $\G$-basepoint, {\rm (ii)} $\EE(n)$ is path-connected for any $n\geq 1$, and {\rm (iii)} $\G$ acts on $\EE$  via covering actions. Then, the sequence of groups $\pi_1\big(\EE(n)/\G_n,[\epsilon_n]\big)$ together with the maps $$\pi_1(\EE(n)/\G_n,[\epsilon_n])\overset{\delta_n}{\lra} \G_n\overset{\pi}{\lra} \Sigma_n$$ conforms an (general) action operad, denoted $\pi_1(\EE,\G)$. Moreover, $\pi_1(\EE,\G)$ lies over $\G$, i.e.\ the connecting homomorphisms $\{\delta_n\}_{n\geq 1}$ in (\ref{eqt:ses pi1 of Goperads}) yield a morphism $\delta\colon\pi_1(\EE,\G)\to \G$ of (general) action operads. 
 \end{prop}

 
 The next two examples were considered in \cite{Wahl}, \cite{Zhang}, \cite{Corner-Gurski} and \cite{Yau}. The third one is a general action operad, but not an action operad.

\begin{examp}
    Consider the action operad $\G = \Sigma$, and take $\EE$ to be the little $2$-discs operad $\C_2$ with its natural right $\G$-action: 
    $$
    \begin{tikzcd}[ampersand replacement=\&]
    \C_2(n)\times \Sigma_n \ar[rr] \&\& \C_2(n) \\[-6mm]
    \big((x_1,\dots,x_n),g\big) \ar[rr, mapsto] \&\& \big(x_{g(1)},\dots,x_{g(n)}\big)
    \end{tikzcd}.
    $$
    %if $g\in \Sigma_n$ and $(x_1,\ldots,x_n)\in \C_2(n)$, then define $(x_1,\ldots,x_n)\cdot g$ as $(x_{g(1)},\ldots,x_{g(n)})$.
    Now, observe that the little $1$-discs operad $\C_1$ is homotopy equivalent to the symmetric operad $\mathsf{Ass}=\mathsf{As}_{\Sigma}$, and the inclusion $\C_1\to \C_2$ is a good $\G$-basepoint. The operad $\Br = \pi_1(\C_2,\Sigma)$ is the action operad of \emph{braid groups}.
\end{examp}

\begin{examp}
    The previous example remains valid if we replace the little $2$-discs operad $\C_2$ by its framed version $\C^{\fr}_2\simeq \C_2\rtimes \mathsf{SO}(2)$. The resulting operad $\RBr = \pi_1(\C^{\fr}_2,\Sigma)$ is the action operad of \emph{ribbon braid groups}. Recall that 
    $$\RBr_k\cong \Br_k\rtimes\, \mathbb{Z}^{\times k},$$
    where $\mathbb{Z}^{\times k}$ accounts for the number of full twists on each ribbon. 
\end{examp}

\begin{examp}
    Consider the general action operad $\G = \Sigma^{\pm}$, and take $\EE$ to be the framed little $2$-discs operad $\C^\fr_2$ with the following $\G$-action:
    $$
    \begin{tikzcd}[ampersand replacement=\&]
    \C_2^{\fr}(n)\times \Sigma_n^{\pm} \ar[rr] \&\& \C_2^{\fr}(n) \\[-6mm]
    \big((x_1,\dots,x_n),(g,A)\big) \ar[rr, mapsto] \&\& \big(x^A_{g(1)},\dots,x^A_{g(n)}\big)
    \end{tikzcd},
    $$
    %if $(g,A)\in \Sigma^{\pm}_n$ and $(x_1,\ldots,x_n)\in \C^{\fr}_2(n)$, then define $(x_1,\ldots,x_n)\cdot(g,A)$ as $(x_{g(1)}^A,\ldots,x_{g(n)}^A)$,
    where $x_i^A = x_i$ for any $i\notin A$ and $x_i^A$ is the precomposition of the embedding $x_i\colon D^2\to D^2$ with a $\pi$-rotation otherwise. Now, observe that the ``unoriented little $1$-discs operad'' $\C^{\mathsf{un}}_1\simeq \C_1\rtimes \OO(1)$ is homotopy equivalent to the operad $\mathsf{As}_{\Sigma^{\pm}}$ since 
    $$
    \C_1^{\mathsf{un}}(k)\simeq \C_1(k)\times \OO(1)^{\times k} \quad \text{ and } \quad  
    \mathsf{As}_{\Sigma^{\pm}}(k)\cong \Sigma_{k}^{\pm}\cong \Sigma_k\times \{\pm \id\}^{\times k}. 
    $$ 
    Hence, the inclusion $\C^{\mathsf{un}}_1\to \C^{\fr}_2$ determined by the inclusion $\C_1\to \C_2$ and the homomorphism $\OO(1)\to \mathsf{SO}(2)$, $-\id\mapsto e^{i\pi}$, is a good $\G$-basepoint. The operad $\URBr = \pi_1(\C^{\fr}_2,\Sigma^{\pm})$ is the general action operad of \emph{unoriented ribbon braid groups}. One can identify unoriented ribbon braid groups in a similar manner to the case of plain ribbon braids, i.e.\  $\URBr_{k}\cong \Br_k\rtimes\, \mathbb{Z}^{\times k}$, but now $\mathbb{Z}^{\times k}$ accounts  for the number of half-twists on each ribbon. With such a description, the canonical morphism $\URBr\to \Sigma^{\pm}$ sends $(\beta_k;n_1,\dots,n_k)$ to $(\pi(\beta_k),A_{\underline{n}})$, where $\pi(\beta_k)$ is the underlying permutation of the braid $\beta_k$ and $A_{\underline{n}}$ is given by the set of $i$'s so that $n_i$ is odd. See Figure \ref{fig:FramedUnordConfAndRibbon} for an illustration of the difference between $\RBr$ and $\URBr$.

    \begin{figure}[h]
    \centering
    \includegraphics%[width=5cm]
    {figures/FramedUnordConfAndRibbon.pdf}
    \caption{From left to right, the operation $\sigma\circ_1\widehat{\tau}=(\sigma;1,0)$ in $\RBr_2$ and the operation $\sigma\circ_1\tau=(\sigma;1,0)$ in $\URBr_2$.}
    \label{fig:FramedUnordConfAndRibbon}
    \end{figure}
\end{examp}

By construction, there is a commutative diagram of general action operads
\begin{equation}\label{eq:braid_maps}
\begin{tikzcd}[ampersand replacement=\&]
   \Br \ar[rd, bend right=40, dashed]\ar[r, dashed] \& \RBr \ar[r] \ar[d] \& \URBr \ar[d] \&[-4mm] (\beta_k;n_1,\dots,n_k) \ar[r, mapsto]\ar[d,mapsto] \& (\beta_k;2n_1,\dots,2n_k)\ar[d,mapsto]\\
   \& \Sigma \ar[r] \& \Sigma^{\pm} \& \pi(\beta_k) \ar[r] \& (\pi(\beta_k),\emptyset)
\end{tikzcd}.
\end{equation}
Note that $\RBr$ is an action operad, while $\URBr$ equipped with the obvious projection $\URBr\to \Sigma$, $(\beta_{k};n_1,\dots,n_k)\mapsto \pi(\beta_{k})$ is just a general action operad since, for example, the square
$$
\begin{tikzcd}[ampersand replacement=\&]
\URBr_1\times \URBr_{2}\ar[r,"\circ_1"] \ar[d] \& \URBr_2\ar[d]\\
\Sigma_1\times \Sigma_2 \ar[r,"\circ_1"'] \& \Sigma_2
\end{tikzcd}
$$
does not commute. For instance, take $(\tau,e_2)=\big((e_1;1),(e_2;0,0)\big)\in \URBr_1\times \URBr_2$ and follow the two directions to obtain $e_2\neq (1,2)$ in $\Sigma_2$ (see Figure \ref{fig:Composition in twisted braid}).

\begin{rem}
Under the correspondence between general action operads and operadic cloning systems, the unordered ribbon braided operad $\URBr$ corresponds to the cloning system of twisted braid groups described in \cite[Example~4.2]{WZ}. 

In \cite[3.5.3]{Th}, Thumann introduces the following braided operad $\EE$: Consider first the free braided operad generated in arity $1$ by an operation $\tau$ and in arity $2$ by an operation $\sigma$. The operad $\EE$ is defined as the result of quotienting this operad by the relation $\tau \circ_1 e_2 = (\sigma\circ_1 \tau)\circ_2 \tau$. This operad is isomorphic to the underlying operad of $\URBr$, and the braid action comes from the maps \eqref{eq:braid_maps}: the operation $\tau$ is the half-twist of a single ribbon, while the operation $\sigma$ is the generator of the braid group. The relation is depicted in Figure \ref{fig:Composition in twisted braid}.

 \begin{figure}[h]
    \centering
    \includegraphics[width=6cm]{figures/Composition_in_twisted_braid.pdf}
    \caption{The relation $\tau\circ_1 e_2= (\sigma\circ_1 \tau)\circ_2\tau$ seen in the general action operad $\URBr$.}
    \label{fig:Composition in twisted braid}
    \end{figure}
\end{rem}




%The operad fundamental group of the operad $\OO$ considered as a $\Br$-operad will be different from the fundamental group of $\OO$ considered as a $\URBr$-operad, and therefore they will yield different Thompson groups. In fact, only the latter will be isomorphic to the Thompson group associated to the cloning system of twisted braid groups. \fc{tengo serias dudas sobre esto. Creo que las operaciones de aridad 1 borran la diferencia}




\section{Proof of Theorem \ref{thm:main2} for braid groups}
\label{sec:braids}
Before we embark on the proof of our main theorem in its full generality, we explain the situation with the special case of braid groups. As will become apparent, the general argument is an abstraction of the ideas introduced here, and will be treated in the next two sections.

Recall from Example~\ref{examp:cloning} the (bilateral) cloning system $\Br=(\Br_{\bullet},\iota,\zeta,\kappa,\pi)$ for braid groups, where $\pi$ is the canonical projection onto the set of symmetric groups, the maps $\iota$ and $\zeta$ add one strand on the right or the left, respectively, and $\kappa$ are the cloning maps given by duplicating a strand.

$\Br=\big(\Br_{\bullet},\pi\big)$ be the collection of all braid groups equipped with the canonical projection homomorphisms $\pi_n\colon\Br_{n}\to \Sigma_n$. 

Note that the collection $\Br_{\bullet}$ can be endowed with an action operad structure with respect to the ``substitution maps'' $\circ_{j}\colon \Br_{n+1}\times\Br_m\to \Br_{n+m}$, which correspond to replacing the $j$-th strand of the first braid by the second braid. Figure  \ref{fig:CloningToOperad} contains a depiction of this operation; for a more detailed discussion, see \cite[Definition 1.6 and Example 1.12(2)]{Corner-Gurski} for details.


We now explain how to obtain one structure from the other, illustrating the main ideas of the procedure with several pictures.

\subsection{From the action operad to the (bilateral) cloning system.}
The morphisms $\iota$ and the cloning maps $\kappa$ for the cloning system are obtained by using the identity $e_2\in \Br_2$ and the composition products $\circ_i$, as explained in Figures~\ref{fig:OperadToIota} and~\ref{fig:OperadToKappa}, respectively. Thus one obtains the cloning system $\Br$ described in Example~\ref{examp:cloning}. 
\begin{figure}
    \centering
    \includegraphics[width=10cm]{figures/Operad_to_iota.pdf}
    \caption{Constructing the map $\iota$. Insert the given braid in the first strand of $e_2$.}
    \label{fig:OperadToIota}
    \end{figure} 
        
    \begin{figure}
    \centering
    \includegraphics[width=10cm]{figures/Operad_to_kappa.pdf}
    \caption{Constructing the map $\kappa_{j}$. Replace the $j$th strand of the given braid by $e_2$.}
    \label{fig:OperadToKappa}
    \end{figure}
In order to get a bilateral cloning system, the maps $\zeta$ are defined similarly to the maps $\iota$, but replacing the second strand of $e_2$ instead of the first one, that is, using $\circ_2$ instead of $\circ_1$; see Figure~\ref{fig:OperadToZeta}.
\begin{figure}
    \centering
    \includegraphics[width=10cm]{figures/Operad_to_zeta.pdf}
    \caption{Constructing the map $\zeta$. Insert the given braid in the second strand of $e_2$.}
    \label{fig:OperadToZeta}
    \end{figure}    

\subsection{From the (operadic) bilateral cloning system to the action operad} We now explain how to use the bilateral cloning system structure on $\Br$ in order to build an action operad structure on $\Br_{\bullet}$. To see how the composition product maps $\circ_j$ are obtained, we proceed as follows. First, we replicate $m$ times the $j$th strand of the first braid using the cloning maps, where $m$ is the number of strands of the second braid. Then we add, by using $\iota$ and $\zeta$ strands to the right and left, respectively, of the second braid, so that it has the same number of strands as the cloned braid. Finally, we just multiply the two braids obtained this way. Figure~\ref{fig:CloningToOperad} shows an example of this construction. 
 \begin{figure}
    \centering
    \includegraphics[width=10cm]{figures/Cloning_to_Operad.pdf}
    \caption{Constructing the substitution maps $\circ_j$ from the cloning system data.}
    \label{fig:CloningToOperad}
    \end{figure}

One can check that the maps $\circ_j$ built above satisfy all the properties required for  endowing $\Br_{\bullet}$ with an operad structure in sets. For instance, Figure~\ref{fig:AssociativityI} and Figure~\ref{fig:AssociativityII} show the verification of the associativity axiom. 
 \begin{figure}
    \centering
    \includegraphics[width=7cm]{figures/Associativity_of_operad_I.pdf}
    \caption{First half of the associativity axiom. The colors represent the application of $\iota$, $\zeta$ and $\kappa$ in each step.}
    \label{fig:AssociativityI}
    \end{figure}
    \begin{figure}
    \centering
    \includegraphics[width=7cm]{figures/Associativity_of_operad_II.pdf}
    \caption{Second half of the associativity axiom.
    }
    \label{fig:AssociativityII}
    \end{figure}

Finally, a simple computation, depicted in Figure~\ref{fig:Compatibility}, shows that the maps $\circ_j$ are compatible with the underlying group structure of the braid groups, thus proving that $(\Br_{\bullet},\pi, \{\circ_i\}_i, e_1)$ is indeed an action operad. 
 \begin{figure}
    \centering
    \includegraphics[width=7cm]{figures/Compatibility_of_products_partial_products.pdf}
    \caption{Compatibility axiom of the action operad.}
    \label{fig:Compatibility}
    \end{figure}



\section{From action operads to cloning systems}
\label{sec:operadtocloning}
In this section, we prove one of the implications of Theorem \ref{thm:main2} in  full generality; more concretely, we explain how to obtain a (bilateral) cloning system from an arbitrary action operad.

Let $\G=(\G_{\bullet},\pi, \{\circ_i\}_i, \id)$ be an action operad, and
%which, recall from Definition~\ref{def:actionoperad}, is a collection of groups $\{\G_n\}_{n\ge 1}$ equipped with homomorphisms $\pi_n\colon\G_n\to\Sigma_n$ and composition products  $(\circ_j\colon \G_{n+1}\times\G_{m}\to \G_{n+m})_{j,n,m}$ satisfying certain compatibility relations.
 denote by $e_n\in \G_n$ the identity element of  $G_n$. If $g\in G_n$ and $i\in \{1,\ldots,n\}$, we denote by $g(i) = \pi(g)(i)$. From the action operad structure, one can define the following maps
$$
\kappa^n_j,\iota_n,\zeta_n\colon \G_n\longrightarrow \G_{n+1}
$$
by setting
$$
\kappa^n_j(g) = g\circ_j e_2,\quad
\zeta_n(g) = e_2\circ_2 g\quad\mbox{ and }\quad
\iota_n(g) = e_2\circ_1 g.
$$
The following observation will be useful in what follows:
\begin{rem}\label{rem:properties} Note that in an action operad we always have that $e_n\circ_i e_m = e_{n+m-1}$ because $e_n\circ_i e_m = (e_n\cdot e_n) \circ_i (e_m\cdot e_m) = (e_n\circ_i e_m)\cdot (e_n\circ_i e_m)$. Moreover, the following hold
$$
    \nu_j^n(m)(g) = e_m\circ_j g\quad\mbox{ and }\quad
    \kappa_j^n(m)(g) = g\circ_j e_m,
$$
where $\kappa_j^n(m)$ and $\nu_j^n(m)$ are defined as in Definition~\ref{defn:NCCloningSystem}.
\end{rem}

\begin{prop}\label{prop:OperadtoCS} 
Let $\G=(\G_{\bullet},\pi, \{\circ_i\}_i, \id)$ be an action operad. Then, the quintuple $(\G_{\bullet}, \iota, \zeta, \kappa,\pi)$, with $\iota$, $\zeta$ and $\kappa$ as defined above, is a restricted operadic  cloning system. If $\G=(\G_{\bullet},\pi, \{\circ_i\}_i, \id)$ is a general action operad, then $(\G_{\bullet}, \iota, \zeta, \kappa,\pi)$ is an operadic cloning system.
\end{prop}
\begin{proof}
We will prove it in several steps. We will make essential use of  the fact that $\G$ is an action operad, the definition of the morphisms $\iota$, $\zeta$ and $\kappa$, and Remark~\ref{rem:properties}. First, we prove that the maps $\iota$ and $\zeta$ are homomorphisms; indeed,
\[\iota(f\cdot g) = e_2\circ_1 (f\cdot g) =
(e_2\cdot e_2)\circ_1(f\cdot g)=
(e_2\circ_{1} f)\cdot (e_2\circ_1 g) = \iota(f)\cdot \iota(g),\]
\[\zeta(f\cdot g) = e_2\circ_2 (f\cdot g) =
(e_2\cdot e_2)\circ_2(f\cdot g)=
(e_2\circ_{2} f)\cdot (e_2\circ_2 g) = \zeta(f)\cdot \zeta(g).\]
We now show that the quadruple $(\G_\bullet,\iota,\kappa,\pi)$ is a cloning system by checking that it satisfies properties (i)--(v) of Definition~\ref{defn:CloningSystem}.
\begin{itemize}
\item[(i)] $\pi_{n+1}\circ \iota_n = \lambda_n\circ\pi_n$. For every $g\in \G_n$ we have that
\begin{align*}
\pi_{n+1}(\iota_n(g)) = \pi_{n+1}(e_2\circ_1 g) = \pi_{2}(e_2)\circ_1 \pi_{n}(g) = e_2\circ_1 \pi_{n}(g) = \lambda_n(\pi_n(g)).
\end{align*}
\item[(ii+)] $\pi_{n+1}\circ \kappa^n_j = c_j^n\circ\pi_n$. For every $g\in\G_n$ and all $1\le j\le n$ we have that
\begin{align*}
\pi_{n+1}(\kappa^n_j (g)) = \pi_{n+1}(g\circ_j e_2) = \pi_{n}(g)\circ_j \pi_{2}(e_2) = \pi_{n}(g)\circ_j e_2 = c^n_j(\pi_n(g)).
\end{align*}
\item[(iii)] $\iota_{n+1}\circ \kappa_j^n = \kappa^{n+1}_j\circ\iota_n$. For every $g\in\G_n$ and all $1\le j\le n$ we have that
\begin{align*}
\iota_{n+1}\circ \kappa_j^n(g) &= \iota_{n+1}(g\circ_j e_2) = e_2\circ_1(g\circ_j e_2)  \\&=  ( e_2\circ_1 g)\circ_j e_2 =  \iota_n(g)\circ_j e_2 = \kappa^{n+1}_j(\iota_n(g)).\end{align*}
\item[(iv)] $\kappa^{n+1}_l\circ\kappa_j^n = \kappa_{j+1}^{n+1}\circ \kappa_l^n$. For every $g\in\G_n$ and all $l< j\le n$ we have that
\begin{align*}
\kappa^{n+1}_l(\kappa^n_j(g)) &= \kappa^{n+1}_l(g\circ_j e_2) = (g\circ_j e_2) \circ_l e_2 \\
&=(g\circ_l e_2) \circ_{j+1} e_2 = \kappa^n_l(g) \circ_{j+1} e_2 = \kappa^{n+1}_{j+1}(\kappa^n_l(g)).
\end{align*}
\item[(v)] $\kappa_j^n(g\cdot h) = \kappa^n_{h(j)}(g)\cdot \kappa_j^n(h)$ for every $g,h \in \G_n$. For all $j\le n$ we have that
\begin{align*}
\kappa_j^n(g\cdot h) &= (g\cdot h)\circ_j e_2 = (g\cdot h)\circ_j (e_2\cdot e_2) \\
&= (g\circ_{h(j)} e_2)\cdot (h\circ_j e_2)
= \kappa^n_{h(j)}(g)\cdot \kappa_j^n(h)
\end{align*}
\end{itemize}
Finally, we prove that $(\G_{\bullet}, \iota, \zeta, \kappa,\pi)$ is a bilateral cloning system by checking properties (i'), (iii'), (vi), (vii), (vii'), (iv+), (ix), (x) of Definition~\ref{defn:NCCloningSystem}. 
First, properties (i'), (iii') and (vii') are proved as properties (i), (iii) and (vii) simply by replacing $\circ_1$ by $\circ_2$ and $\iota$ by $\zeta$.
\begin{itemize}
\item[(vi)] $\zeta_{n+1}\circ \iota_n = \iota_{n+1}\circ\zeta_{n}$. For every $g\in \G_n$ and all $n\ge 1$ we have that
\begin{align*}
    \zeta_{n+1}(\iota_n(g)) &= e_2\circ_2(e_2\circ_1 g) = (e_2\circ_2 e_2)\circ_2 g = e_3\circ_2 g \\&= (e_2\circ_1 e_2)\circ_2 g = e_2\circ_1 (e_2\circ_2 g) = \iota_{n+1}(\zeta_{n}(g)).
\end{align*}
\item[(vii)] $\iota_{n+1}\circ\iota_n = \kappa^{n+1}_{n+1}\circ \iota_n$. For every $g\in \G_n$ and all $n\ge 1$ we have that
\begin{align*}
\iota_{n+1}(\iota_{n}(g)) &= e_2\circ_1(e_2\circ_1 g) = (e_2\circ_1 e_2)\circ_1 g = e_3\circ_1 g, \\
\kappa^{n+1}_{n+1}(\iota_n(g)) &= (e_2\circ_1 g)\circ_{n+1} e_2 = (e_2\circ_2 e_2)\circ_1 g = e_3\circ_1 g.
\end{align*}
\item[(iv+)] $\kappa^{n+1}_{j+1}\circ \kappa^n_{j} = \kappa_{j}^{n+1}\circ \kappa_j^n$. For all $g\in \G_n$ and all $1\le j\le n$ we have that 
\begin{align*}
\kappa^{n+1}_{j+1}(\kappa^n_{j}(g)) &=
(g\circ_j e_2)\circ_{j+1} e_2 = g\circ_j(e_2\circ_2 e_2) = g\circ_2 e_3  \\ &= g\circ_j (e_2\circ_1 e_2) = (g\circ_j e_2)\circ_j e_2 = \kappa_{j}^{n+1}(\kappa_j^n(g)).
\end{align*}
\item[(viii)] $\kappa_j^n(m)(g)\cdot \nu^m_j(n)(h) = \nu^m_{g(j)}(n)(h) \cdot \kappa_j^n(m)(g)$ for every $g\in\G_n$ and $h\in\G_m$. For all $m,n\geq 0$ we have that
\begin{align*}
\kappa_j^n(m)(g)\cdot \nu^m_j(n)(h) &=
(g\circ_j e_m)\cdot (e_n\circ_j h) \\
&= (g\cdot e_n)\circ_{j} (e_m\cdot h) \\
&= (e_n\cdot g)\circ_{j} (h\cdot e_m) \\
&= (e_n\circ_{g(j)} h)\cdot (g\circ_{j} e_m) \\
&= \nu^m_{g(j)}(n)(h)\cdot \kappa_{j}^n(m)(g).
\end{align*}

\item[(ix)] $\iota_n(m)(g)\cdot \zeta_m(n)(h) = \zeta_m(n)(h)\cdot \iota_n(m)(g)$ for every $g\in\G_n$ and $h\in\G_m$. For all $m,n\ge 0$ we have that
\begin{align*}
    \iota_n(m)(g)\cdot \zeta_m(n)(h) &= (e_{m+1}\circ_{1} g)\cdot (e_{n+1}\circ_{n+1} h) \\
    &= (e_{m+1}\circ_1 g)\cdot((e_2\circ_1 e_n)\circ_{n+1} h) \\
    &= (e_{m+1}\circ_1 g)\cdot((e_2\circ_2 h)\circ_1 e_n) \\
    &= (e_{m+1}\cdot (e_2\circ_2 h))\circ_1 (g\cdot e_n) \\
    &= ((e_2\circ_2 h)\cdot e_{m+1})\circ_1 (e_n\cdot g) \\
    &= ((e_2\circ_2 h)\cdot (e_{2}\circ_2 e_m))\circ_1 (e_n\cdot g) \\
    &= ((e_2\circ_2 h)\circ_1 e_n)\cdot ((e_2\circ_2e_m)\circ_1 g) \\
    &= ((e_2\circ_1 e_n)\circ_{n+1} h)\cdot ((e_{m+1})\circ_1 g) \\
    &= \zeta_m(n)(h)\cdot \iota_n(m)(g).
\end{align*}
\end{itemize}
Therefore, $(\G_{\bullet}, \iota, \zeta, \kappa,\pi)$ is a restricted operadic cloning system as we wanted to show.

If $(\G_{\bullet}, \iota, \zeta, \kappa,\pi,\id)$ were a general action operad, then the proof of Condition (ii+) would restrict instead to a proof of Condition $(ii)$.
\end{proof}





\section{From cloning systems to action operads}\label{sect:FromCStoActionOperads}
We now explain how to construct (general) action operads from a subclass of bilateral cloning systems. Let $\G=(\G_{\bullet},\iota,\zeta,\kappa,\pi)$ be a bilateral cloning system. The following identities can be derived from the identites in the definition of a bilateral cloning system (recall from Definition~\ref{defn:NCCloningSystem} the construction of the maps $\kappa_i^n(m)$ and $\nu_i^n(m)$):

\[
\kappa_i^{n+m-1}(l)\circ \kappa_j^n(m)=
\begin{cases}
    \kappa_{j}^n(m+l-1) & \text{if $j\leq i < j+m$}, \\
    \kappa_{j+l-1}^{n+l-1}(m)\circ\kappa_i^n(l) & \text{if $i<j$}.
\end{cases}
\]
\[
    \kappa_j^{n+m-1}(l)\circ \nu^n_i(m) = \begin{cases}
		\nu^{n+l-1}_i(m)\circ \kappa_{j-i+1}^n(l) & \text{if $i\leq j< i+n$},\\
		\nu^{n}_{i+l-1}(m+l-1) & \text{if $j< i$}, \\
		\nu^{n}_i(m+l-1) & \text{if $j\geq i+n$},
		\end{cases}
  \]
\begin{align*}
    \nu_j^{n+m-1}(l)\circ \nu_i^n(m)& =\nu_{j+i-1}^n(m+l-1),
\end{align*}
\begin{align*}
    \kappa_j^n(m)(f\cdot g) &= \kappa_{g(j)}^n(m)(f)\cdot \kappa_j^n(m)(g) &\mbox{if } 1\leq j\leq n,
\end{align*}
\begin{multline*}
        \nu_j^l(n+m-1)(g)\cdot \nu_{i+l-1}^m(n+l-1)(h) \\ =  \nu_{i+l-1}^m(n+l-1)(h)\cdot \nu_j^l(n+m-1)(g)\qquad \mbox{if $j<i$}
\end{multline*}
Note that $\nu^n_i(m)(g)$ acts trivially on all $j<i$ and all $j\geq i+m$ and that $\kappa^n_i(m)(e_n) = e_{n+m-1}$, because
\[\kappa^n_i(e_n) = \kappa^n_i(e_n\cdot e_n) = \kappa^n_i(e_n)\cdot \kappa^n_i(e_n)\]
and therefore $\kappa^n_i(e_n) = e_{n+1}$. 

In order to define an action operad structure on $\G_{\bullet}$ and since we already have a map $\pi\colon \G_{\bullet}\to \Sigma_{\bullet}$, it is enough to define the partial composition products $\{\circ_i\}_i$, which we do it as follows. The map
\[
\circ_i\colon \G_n\times \G_m\longrightarrow \G_{n+m-1}
\]
is the following composition
\[\xymatrix{
\G_{n}\times \G_{m}\ar[rr]^-{\kappa_{i}^n(m)\times \nu_i^m(n)} &&
\G_{n+m-1}\times\G_{n+m-1} \ar[r]^-{\cdot} &
\G_{n+m-1},
}\]
that is, $f\circ_i g = \kappa_i^n(m)(f)\cdot \nu^m_i(n)(g)$, for every $f\in\G_n$ and $g\in \G_m$. We also set $\id = e_1\in \G_1$.

\begin{prop}\label{prop:CloningtoOper}
    Let $\G=(\G_{\bullet}, \iota, \zeta, \kappa,\pi)$ be an restricted operadic cloning system. Then the quadruple $(\G_{\bullet},\pi, \{\circ_i\}_i, \id)$, with $\{\circ_i\}_i$ as defined above is an action operad. If the restricted condition is dropped, then we obtain a general action operad.
\end{prop}
\begin{proof} First, we check that $e_1$ is indeed a unit for the partial composition product. For every $f\in\G_n$ and every $g\in \G_m$ we have that
\begin{align*}
    f\circ_i e_1 &= \kappa^n_i(1)(f)\cdot \nu_i^1(n)(e_1) = f\cdot e_n = f, \\
    e_1\circ_1 g &= \kappa^1_1(m)(e_1) \cdot \nu_1^m(1)(g) = e_m\cdot g = g.
\end{align*}
Second, we show the associativity for the partial composition product. Let $f\in \G_n$, $g\in \G_m$ and $h\in \G_l$. Then
    \begin{align}\notag
        &(f\circ_i g)\circ_j h \\ \notag &=(\kappa_i^n(m)(f)\cdot \nu^m_i(n)(g))\circ_j h \\ \notag
&= \kappa^{n+m-1}_j(l)\Big(\kappa_i^n(m)(f)\cdot \nu_i^m(n)(g)\Big)\cdot \nu_j^l(n+m-1)(h) \\ \label{eq:309}
&=
\Big(\kappa_{g^*(j)}^{n+m-1}(l)(\kappa_i^n(m)(f))\Big)\cdot \Big(\kappa_{j}^{n+m-1}(l)(\nu_i^m(n)(g))\Big)\cdot \Big(\nu_j^l(n+m-1)(h)\Big), 
\end{align}
where we denote $g^*=\nu_i^m(n)(g)$. Depending on the indexes $i$ and $j$, we have to consider the following three cases.

\bigskip
\noindent {\it Case 1.} Suppose first that $j<i$. In this case $g^*(j) = j$ and we have that 
$$
\kappa_j^{n+m-1}(l)\circ \kappa_i^n(m) = \kappa_{i+l-1}^{n+l-1}(m)\circ \kappa_j^n(l)\quad\mbox{and}
$$
$$
\kappa_j^{n+m-1}(l)\circ \nu_i^m(n) = \nu_{i+l-1}^m(n+l-1).
$$
Then \eqref{eq:309} becomes
\[\Big(\kappa_{i+l-1}^{n+l-1}(m)(\kappa_j^n(l)(f))\Big)\cdot \Big(\nu_{i+l-1}^m(n+l-1)(g)\Big)\cdot \Big(\nu_j^l(n+m-1)(h)\Big). \]
On the other hand, we have that
    \begin{align}\notag 
        &(f\circ_j h)\circ_{i+l-1} g  \\ \notag
&=(\kappa_j^n(l)(f)\cdot \nu_j^l(n)(h))\circ_{i+l-1} g \\ \notag      
&= \kappa_{i+l-1}^{n+l-1}(m)\Big(\kappa_j^n(l)(f)\cdot \nu_j^l(n)(h)\Big)\cdot \nu_{i+l-1}^m(n+l-1)(g) \\ \label{eq:310}
&=
\Big(\kappa_{h^*(i+l-1)}^{n+l-1}(m)(\kappa_j^n(l)(f))\Big)\cdot \Big(\kappa_{i+l-1}^{n+l-1}(m)(\nu_j^l(n)(h))\Big)\cdot \Big(\nu_{i+l-1}^m(n+l-1)(g)\Big),
\end{align}
where $h^*=\nu_j^l(n)(h)$. Now, \eqref{eq:309} and \eqref{eq:310} are equal since 
$$
h^*(i+l-1) = \nu_j^l(n)(h)(i+l-1) = i+l-1\quad\mbox{and}
$$
$$
\kappa_{i+l-1}^{n+l-1}(m)(\nu_j^l(n)(h)) = \nu_j^l(n+m-1)(h).
$$
The last equality holds because $j<i$ and therefore $i+l-1\ge j+l$.

\bigskip
\noindent{\it Case 2.} Suppose now that $j\geq i+m$. Again $g^*(j)=j$ and we have that
$$
\kappa_j^{n+m-1}(l)\circ \kappa_i^n(m) = \kappa_{i}^{n+l-1}(m)\circ \kappa_{j-m+1}^n(l)\quad\mbox{and}
$$
$$
\kappa_j^{n+m-1}(l)\circ \nu_i^m(n) = \nu_i^m(n+l-1).
$$
The fist equality holds because $j\ge i+m$ and therefore $i<j-m+1$.
Then \eqref{eq:309} becomes
\[\Big(\kappa_{i}^{n+l-1}(m)(\kappa_{j-m+1}^n(l)(f))\Big)\cdot \Big(\nu_i^m(n+l-1)(g)\Big)\cdot \Big(\nu_j^l(n+m-1)(h)\Big). \]
On the other hand,
    \begin{align}\notag 
        &(f\circ_{j-m+1} h)\circ_{i} g \\ \notag
&= (\kappa_{j-m+1}^n(l)(f)\cdot \nu_{j-m+1}^l(n)(h))\circ_i g \\ \notag  &= \kappa_{i}^{n+l-1}(m)\Big(\kappa_{j-m+1}^n(l)(f)\cdot \nu_{j-m+1}^l(n)(h)\Big)\cdot \nu_i^m(n+l-1)(g) \\ \label{eq:311}
&=
\Big(\kappa_{h^*(i)}^{n+l-1}(m)(\kappa_{j-m+1}^n(l)(f))\Big)\!\cdot \!\Big(\kappa_{i}^{n+l-1}(m)(\nu_{j-m+1}^l(n)(h))\Big)\!\cdot \!\Big(\nu_i^m(n+l-1)(g)\Big),
\end{align}
where $h^*=\nu_{j-m+1}^l(n)(h)$. But \eqref{eq:309} and \eqref{eq:311} are equal since $h^*(i) = i$ and
$$
\kappa_i^{n+l-1}(m)(\nu_{j-m+1}^l(n)(h)) = \nu_j^l(n+m-1)(h).
$$
The last equality holds because $j\ge i+m$ and therefore $i<j-m+1$.

\bigskip
\noindent {\it Case 3.} Suppose that $i\leq j<i+m$. In this case, $i\leq g^*(j)<i+m$ and we have that
$$
\kappa_{g^*(j)}^{n+m-1}(l)\circ \kappa_i^n(m) = \kappa_i^n(m+l-1)\quad\mbox{and}
$$
$$
\kappa_j^{n+m-1}(l)\circ \nu_i^m(n) = \nu_i^{m+l-1}(n)\circ \kappa_{j-i+1}^m(l).
$$
Then \eqref{eq:309} becomes
\[\Big(\kappa_i^n(m+l-1)(f)\Big)\cdot \Big(\nu_i^{m+l-1}(n)( \kappa_{j-i+1}^m(l)(g))\Big)\cdot \Big(\nu_j^l(n+m-1)(h)\Big).
\]
On the other hand, 
\begin{align} \notag
&f\circ_i(g\circ_{j-i+1} h) \\ \notag
&=f\circ_i (\kappa_{j-i+1}^m(l)(g)\cdot \nu_{j-i+1}^l(m)(h)) \\ \notag
&= \kappa_i^n(m+l-1)(f)\cdot \nu_i^{m+l-1}(n)\Big(\kappa_{j-i+1}^m(l)(g)\cdot \nu_{j-i+1}^l(m)(h)\Big)\\
&=\Big(\kappa_i^n(m+l-1)(f)\Big)\cdot \Big(\nu_i^{m+l-1}(n)(\kappa_{j-i+1}^m(l)(g))\Big)\cdot \Big(\nu_i^{m+l-1}(n)(\nu_{j-i+1}^l(m)(h))\Big).\label{eq:312}
\end{align}
Again \eqref{eq:309} and \eqref{eq:312} are equal since 
$$
\nu_i^{m+l-1}(n)(\nu_{j-i+1}^l(m)(h))=\nu_j^l(n+m-1)(h).
$$
Regarding the second condition, it is clear that from axiom $(ii)$ we obtain a levelwise homomorphism $\pi$ to the family of symmetric groups as in the definition of general action operad. From axiom $(ii+)$ we obtain an operad map $\pi$ as in the definition of action operad.

Finally, in order to prove the product rule for the action operad, we will use condition (viii) in the definition of a bilateral cloning system. Let $f,f'\in G_n$ and $g,g'\in G_m$. Then we have that
\begin{align*}
(f\circ_{f'(i)} g)\cdot (f'\circ_i g')
&= \Big(\kappa_{f'(i)}^n(m)(f)\cdot \nu_{f'(i)}^m(n)(g)\Big)\cdot \Big(\kappa_i^n(m)(f')\cdot \nu_i^m(n)(g')\Big) \\
&= \Big(\kappa_{f'(i)}^n(m)(f)\cdot  \kappa_i^n(m)(f')\Big)\cdot \Big(\nu_{f'(i)}^m(n)(g)\cdot \nu_i^m(n)(g')\Big) \\
&= \Big(\kappa_{i}^n(m)(f\cdot f')\Big)\cdot \Big(\nu_{i}^m(n)(g\cdot g')\Big) \\
&= (f\cdot f')\circ_i (g\cdot g').\qedhere
\end{align*}
\end{proof}

\begin{thm}\label{thm:BijectionActOpdsAndOperadicCS}
There is an explicit bijective correspondence between action operads and restricted operadic cloning systems. There is an explicit bijective correspondence between general action operads and operadic cloning systems.      
\end{thm}
\begin{proof} The constructions given in Proposition~\ref{prop:OperadtoCS} and Proposition~\ref{prop:CloningtoOper} are inverses of each other. Let $(\G_\bullet,\iota,\zeta,\kappa,\pi)$ be a restricted operadic cloning system, and let $(\hat{\G}_\bullet,\hat{\iota},\hat{\zeta},\hat{\kappa},\hat{\pi})$ be the restricted operadic cloning system that arises from the action operad associated to it. It is clear that $\hat{\G}_\bullet = \G_\bullet$ and that $\hat{\pi} = \pi$. For the maps $\iota$ we have that
\begin{align*}
\hat{\iota}_n(g) &= e_2\circ_1 g = \kappa_1^2(n)(e_2)\cdot \nu^n_1(2)(g) = e_{n+1}\cdot \iota_n(1)(g) = \iota_n(g),
\end{align*}
and in the same way, we obtain that $\hat{\zeta}_n(g)=\zeta_n(g)$. For the cloning maps, we have that
\begin{align*}
\hat{\kappa}_j^n(g) &= g\circ_j e_2 = \kappa_j^n(2)(g)\cdot \nu_j^2(n)(e_2) = \kappa_j^n(2)(g)\cdot e_{n+1} = \kappa_j^n(g).
\end{align*}

Conversely, let $(\hat{\G}_\bullet,\hat{\pi},\{\hat{\circ}_i\}_i,\hat{\id})$ be the action operad associated to the restricted operadic cloning system obtained from an action operad $(\G_\bullet,\pi,\{\circ_i\}_i,\id)$. It is clear that $\hat{\G}_\bullet = \G_\bullet$, $\hat{\pi}=\pi$ and $\hat{\id}=\id$. Regarding the partial composition products, let $f\in \G_n$ and $g\in\G_m$. Then we have that
\begin{align*}
f\operatorname*{\hat{\circ}}\nolimits_i g &= \kappa_i^n(m)(f)\cdot \nu^m_i(n)(g) \\ 
&= (f\circ_i e_m)\cdot (e_n\circ_i g) \\
&= (f\cdot e_n)\circ_i (e_m \cdot g) \\
&= f\circ_i g,
\end{align*}
where we have used Remark~\ref{rem:properties} for the second equality and the product rule of the action operad for the third. 

The second statement is proven analogously.
\end{proof}


\begin{rem} As mentioned in Section \ref{section:cloning_systems}, most of the known examples of cloning systems are bilateral and even operadic. Consequently, for all those examples, we have identified a general action operad structure on them. However, a natural question arises: is it possible to interpret the remaining Example~\ref{examp:DirectPowers} and Example~\ref{examp:UpperTriangularMatrices}, in operadic terms? The answer is yes. Both examples have in common that their structural map $\pi$ is trivial and that the only condition that fails for them to be operadic bilateral cloning systems is axiom (viii). 

Repeating the construction of Proposition \ref{prop:CloningtoOper} for these examples, and noticing that axiom (viii) is only applied to show the compatibility of the group multiplication with the $\circ_i$ products, one gets the structure of a non-symmetric operad \emph{in sets} (see Definition \ref{def:actionoperad}). The reason why they are not operadic is that an operadic cloning system with $\pi$ trivial has an associated non-symmetric operad \emph{in groups} via Theorem \ref{thm:BijectionActOpdsAndOperadicCS}.
\end{rem}



\section{Cloning systems as crossed groups}\label{section:crossed}

Action operads have a close relationship with crossed simplicial groups \cite{Zhang, Yoshida}. In this section we review this relationship, and explain the relationship with cloning systems. As we will see, cloning systems satisfying (iv+) can be interpreted as ``crossed interval groups'', while cloning systems satisfying (iv+) without maps $\iota$ are the same as crossed simplicial groups.


Let $(\G_{\bullet},\pi,\iota,\kappa)$ be a cloning system. If we forget all the structure except the cloning maps, we can interpret the pair $(\G_\bullet,\kappa)$ as a diagram
$$
\begin{tikzcd}[ampersand replacement=\&]
\G_1
\ar[r, "\kappa_1" description] \& \G_2
\ar[r, "\kappa_1" description, shift left=2]\ar[r, "\kappa_2" description, shift right = 2] \& \G_3 
\ar[r, "\kappa_1" description, shift left=4]\ar[r, "\kappa_2" description]\ar[r, "\kappa_3" description, shift right=4] \& \;\cdots 
\end{tikzcd}
$$
which resembles the diagram representing a simplicial set $X$, but considering only the degeneracy operations
$$
\begin{tikzcd}[ampersand replacement=\&]
X[0] \ar[r, "s_0" description] \& X[1] \ar[r, "s_0" description, shift left=2]\ar[r, "s_1"' description, shift right=2] \& X[2] \ar[r, "s_0" description, shift left=3]\ar[r, "s_1" description] \ar[r, "s_2"' description, shift right=3] \& X[3] \ar[r, "s_0"description, shift left=5]\ar[r, "s_1" description, shift left=2]\ar[r, "s_2" description, shift right=2]\ar[r, "s_3"' description,shift right=5] \& X[4] \ar[r, "s_0" description,shift left=6]\ar[r, "s_1" description, shift left=3]\ar[r, "s_2" description]\ar[r, "s_3" description, shift right=3]\ar[r, "s_4"' description, shift right=6] \& \;\cdots 
\end{tikzcd}
$$
Let us formalise this viewpoint. Let $[n]=\{0,1,\ldots,n\}$ be the finite ordinal of cardinality $n+1$.
\begin{defn}
    The \emph{simplicial category} $\Delta$ has objects the non-empty finite ordinals and morphisms the order-preserving maps between them. There are two distinguished families of morphisms
\begin{align*}
    \delta^n_j\colon &[n-1]\longrightarrow [n], & \sigma^n_j\colon &[n+1]\longrightarrow [n], & 0\leq j\leq n,
\end{align*}
called \emph{cofaces} and \emph{codegeneracies}, respectively, and defined as
\begin{align*}
    \delta_j^n(i)&=\begin{cases}
        i & \text{if $i<j$} \\
        i+1 & \text{if $i\geq j$}
    \end{cases}
&
    \sigma_j^n(i)&=\begin{cases} i &\text{if $i\leq j$} \\ i-1 & \text{if $i>j$}
\end{cases}
\end{align*}
These morphisms satisfy the following relations, called \emph{cosimplicial identities}
\begin{align*}\label{eq:007}
\delta^n_{j}\circ \delta^{n-1}_{i+1} &= \delta^n_{i}\circ \delta^{n-1}_j, \qquad j\leq i, \\
\sigma^n_{j+1}\circ \sigma^{n+1}_i &= \sigma^n_i\circ \sigma^{n-1}_{j},\qquad i\leq j, \\
\delta^{n+1}_i\circ \sigma^n_j &= \begin{cases}
    \sigma^n_{j-1}\circ \delta^{n+1}_i & i<j, \\
    \id & i=j,j+1, \\
    \sigma^n_j\circ \delta^{n+1}_{i-1} & i>j+1,
\end{cases}
\end{align*}
and generate all the morphisms in the category in the sense that every morphism can be expressed as a composite of cofaces and codegeneracies. 
\end{defn}
\begin{defn}
Let $\Delta_{\surj}\subset \Delta$ be the category of non-empty finite ordinals with order-preserving surjections between them. It is the subcategory of $\Delta$ generated by the maps $s_j^n$.
\end{defn}
\begin{defn}
    Let $\Set$ be the category of sets and functions. A~\emph{simplicial set} $X$ is a functor $X\colon\Delta^\op\to \Set$. The maps $X(\delta^n_i)$ are called \emph{faces} and denoted $d^n_i$ and the maps $X(\sigma^n_i)$ are called \emph{degeneracies} and denoted $s^n_i$. A \emph{demi-simplicial set}\footnote{It seems these objects have not been considered previously in the literature. Since simplicial sets without degeneracies are called \emph{semi-simplicial sets}, we have chosen to replace the prefix \emph{semi-} by its french version \emph{demi-}. Additionally, the first letter of each prefix specifies whether we are removing degeneracies or face maps.} $X$ is a functor $X\colon\Delta^\op_\surj\to \Set$.
\end{defn}

Faces and degeneracies satisfy the so-called \emph{simplicial identities} which are the dual of the cosimplicial identities mentioned above. In the particular case of the relation involving only degeneracies, we get the identity
\begin{equation}
s^{n+1}_i\circ s^n_{j+1}=s^{n-1}_{j}\circ s^n_i,\qquad i\leq j.\label{eq:deg_rel}
\end{equation}

In what follows, when we refer to conditions in roman numbers, we mean the conditions satisfied by cloning systems and bilateral cloning systems from Definition~\ref{defn:CloningSystem} and Definition~\ref{defn:NCCloningSystem}.

Observe that \eqref{eq:deg_rel} is exactly the same relation satisfied by the maps $\kappa^n_j$ with the extra condition (iv+) except that the subindexes are shifted by one (the first map is $\kappa_1$ not $\kappa_0$). Therefore we have the following consequence. 
\begin{lem} Let $\G_{\bullet}=\{\G_n\}_{n\geq 1}$ be a family of groups and let $\kappa=\{\kappa^n_{j}\colon\G_{n}\to\G_{n+1}\}_{n\ge 1,\,1\leq j\leq n}$ be a family of maps. A pair $(\G_\bullet,\kappa)$ satisfying conditions \emph{(iv)} and \emph{(iv+)} is the same as a demi-simplicial set with values on groups. $\hfill\qed$
\end{lem}
\begin{examp}\label{ex:demi_symm} Let us build the demi-simplicial set associated to the cloning system of symmetric groups. Note that an order-preserving surjection $f\colon [n]\to [m]$ is completely determined by the cardinality of the preimages $f^{-1}(0),\ldots,f^{-1}(m)$. Every permutation $h\in \Sigma_{m+1}$ of $[m]$ induces a permutation of the preimages $f^{-1}(0),\ldots,f^{-1}(m)$, and therefore a block permutation on $[n]$, that we denote by $\Phi(f)(h)\in \Sigma_{n+1}$. This defines a functor $\Phi\colon \Delta_\surj^\op\to \Set$ with $\Phi([n]) = \Sigma_{n+1}$.

There is an action of $\Sigma_{m+1}$ on the set of order-preserving surjections from $[n]$ to $[m]$, that sends a permutation $g$ and a surjection $f$ to the unique surjection $h$ such that the cardinality of $h^{-1}(g(i))$ is equal to the cardinality of $f^{-1}(i)$. We denote this surjection $h$ by $f_g$. Observe now that the map $\Phi(f)\colon \Sigma_{m+1}\to \Sigma_{n+1}$ is not a group homomorphism but satisfies that 
\begin{equation*}\label{eq:610}
    \Phi(f)(g\cdot g') = \Phi(f_{g'})(g)\cdot\Phi(f)(g'),
\end{equation*}
which applied to degeneracies is the same as condition (v) in a cloning system
\begin{equation}\label{eq:611}\Phi(s_i)(g\cdot g') = \Phi(s_{g'(i)})(g)\cdot \Phi(s_i)(g').
\end{equation}
The functor $\Phi$, defined on order-presering surjections between ordinals, can be extended to any order-preserving map between ordinals. Indeed, since every map factors as a surjection followed by an injection, it is enough to define it on injections, which we can do as follows. Note that to give an order-preserving injective function $f\colon [n]\to [m]$ is equivalent to specify the complement of the image $A = [m]\smallsetminus f([n])$. There is an action of $\Sigma_{m+1}$ on the set of order-preserving injections from $[n]$ to $[m]$, that sends a permutation $g$ and an injection $f$ to the unique injection $h$ such that $[m]\smallsetminus h([n]) = [m]\smallsetminus g(f([m]))$. We denote this injection $h$ by $f_g$. If $g\in \Sigma_{m+1}$ is a permutation of $[m]$, define $\Phi(f)(g) = f_g^{-1}\circ g\circ f$. This defines a functor $\Phi\colon \Delta^\op\to \Set$, and hence a simplicial set.

 Observe now that the map $\Phi(f)\colon \Sigma_{m+1}\to \Sigma_{n+1}$ is not a group homomorphism but satisfies that 
\begin{equation*}\label{eq:613}
    \Phi(f)(g\cdot g') = \Phi(f_{g'})(g)\cdot\Phi(f)(g').
\end{equation*}
\end{examp}
The following definition of crossed simplicial groups is a characterization taken from~\cite[Proposition 1.7]{FL}, where $\Phi$ denotes the simplicial set constructed above.
\begin{defn}
    A \emph{crossed simplicial group} is a simplicial set $\Psi\colon \Delta^\op\to \Set$ with values on groups together with a levelwise group homomorphism $\pi\colon \Psi\to \Phi$ such that $\pi(s_i^{\Psi}(g))[j] = s_i^{\Phi}(\pi(g))[j]$ for all $j\neq i,i+1$ and
\begin{align*}
    \Psi(s_i)(g\cdot g') &= \Psi(s_{\pi(g)(i)})(g')\cdot\Psi(s_i)(g'), \\ 
    \Psi(d_i)(g\cdot g') &= \Psi(d_{\pi(g)(i)})(g')\cdot\Psi(d_i)(g'),
    \end{align*} and $\pi([n])\colon \Psi([n])\to \Sigma_n$ is a group homomorphism. A \emph{crossed demi-simplicial group} is defined in the same way, replacing the category $\Delta$ by the category $\Delta_\surj$. 
\end{defn}

The following result follows immediately from the previous discussion and the definition of crossed demi-simplicial group.
\begin{lem} Let $\G_{\bullet}=\{\G_n\}_{n\geq 1}$ be a family of groups, $\pi=\{\pi_n\colon \G_n\to\Sigma_n\}_{n\ge 1}$ a family of group homomorphisms and $\kappa=\{\kappa^n_{j}\colon\G_{n}\to\G_{n+1}\}_{n\ge 1,\,1\leq j\leq n}$ a family of maps. A triple $(\G_\bullet,\pi,\kappa)$ satisfying conditions {\rm (ii)}, {\rm (iv)}, {\rm (iv+)} and {\rm (v)} is the same as a crossed demi-simplicial group. $\hfill\qed$
\end{lem}

Now we would like to incorporate the homomorphisms $\iota$ to the picture. We will incorporate the morphisms $\zeta$ at the same time.

\begin{defn} For each $n\geq 1$, the $n^{\text{th}}$ \emph{interval} is the set $\langle n\rangle = \{-\infty,1,\ldots,n,\infty\}$. The \emph{interval category} $\I$ is the category whose objects are all intervals and whose morphisms are order-preserving maps that preserve $-\infty$ and $\infty$. The subcategory $\I_\surj$ has the same objects as $\I$ and its morphisms are the order-preserving surjective maps that preserve $-\infty$ and $\infty$.\end{defn}
The interval category can be introduced as the Joyal dual of the simplicial category \cite{Joyal} or as the image of the faithful embedding $\alpha\colon \I\to \Delta$ that sends $\langle n\rangle = \{-\infty,1,\ldots,n,\infty\}$ to $[n+1] = \{0,1,\ldots,n+1\}$, and an interval map yields a simplicial map by interpreting $-\infty$ as $0$ and $\infty$ as $n+1$. Here we are interested in the second description, and we will blur de difference between maps in $\I$ and their images under the embedding $\alpha$. Therefore, from now on we will write simply $\Phi$ for the crossed interval group $\Phi\circ \alpha$.
\begin{defn}
    An \emph{inert crossed interval group} is a presheaf $\Psi\colon \I^\op\to \Set$ with values on groups together with a levelwise group homomorphism $\pi_n\colon \Psi([n])\to \Phi([n])$ such that $\pi(s_i^{\Psi}(g))[j] = s_i^{\Phi}(\pi(g))[j]$ for all $j\neq i,i+1$ and
    \begin{align*}\label{eq:product_cig}
    \Psi(s_i)(g\cdot g') &= \Psi(s_{\pi(g)(i)})(g')\cdot\Psi(s_i)(g'), \\ 
    \Psi(d_i)(g\cdot g') &= \Psi(d_{\pi(g)(i)})(g')\cdot\Psi(d_i)(g'),
    \end{align*} 
     An \emph{inert crossed demi-interval group} is defined in the same way, by replacing $\I$ by $\I_\surj$.
\end{defn}
Observe now that if $\Phi$ is an inert crossed interval group, the maps $\Phi(s_0)$ and $\Phi(s_{n+1})$ are group homomorphisms. In fact, we can interpret bilateral cloning systems as an inert crossed demi-interval groups by setting $\kappa^n_i = s^{n+1}_i$, $\zeta_n = s^{n+1}_0$ and $\iota_n = s^{n+1}_{n+1}$. Thus, we have the following result.
\begin{lem} A quintuple $(\G_\bullet,\pi,\iota,\zeta,\kappa)$ satisfying all the conditions of a bilateral cloning system is the same as an inert crossed demi-interval group. 
\end{lem}

\begin{rem}
    Since crossed interval groups embedd into crossed simplicial groups, every action operad gives rise to a crossed simplicial group. This is carefully developed in \cite[2.4]{Zhang}.
\end{rem}

\begin{rem} Crossed interval groups have been studied in \cite{Batanin-Markl} and \cite{Yoshida}. The adjective \emph{inert} corresponds to any of the two equivalent properties defined in \cite[Lemma 4.3]{Yoshida}.

In that paper, Yoshida studied the relation between crossed interval groups and action operads. He established that every action operad determines a crossed interval group, and found three properties that characterise the crossed interval groups that come from an operad: operadicness, tameness and ``factoring through the symmetric group''. Under the above lemma, tameness corresponds to property (ix), while operadicness corresponds to property (viii) plus being inert and ``factoring through the symmetric group'' corresponds to (ii+). Yoshida studies action operads with constants (operations of arity 0), while we study action operads without constants. The existence of constants in the action operad corresponds to the injective morphisms in the simplicial or the interval category.

We note that Examples 3.1 and 3.4 in \cite{Zaremsky}, which do not come from an action operad, do come from an inert crossed demi-interval group (with trivial homomorphisms $\pi$). 
\end{rem}

%\fc{
%\begin{rem}
%    By the classification result of Loday and Fiedorowicz, we can replace the requirement of having a 
%    \begin{itemize}
        %\item ``levelwise group homomorphism $\pi\colon \Psi\to \Phi$ such that $\pi(s_i^{\Psi}(g))[j] = s_i^{\Phi}(\pi(g))[j]$ for all $j\neq i,i+1$''.
%    \end{itemize} by the following: Let $\Phi'$ be the crossed interval group corresponding to the hyperoctahedral crossed simplicial group. The $\pi'\colon \Psi\to \Phi'$ is a natural transformation of set-valued functors that is a levelwise group homomorphism. The levelwise group homomorphism $\pi\colon \Psi\to \Phi$ is then recovered as the composition $\Psi\to \Phi'\to \Phi$.
%\end{rem}
%}



\section{Cloning systems and PROs}\label{section:props}

In this section, we present another viewpoint for cloning systems in terms of PROs (product categories) that will be used in a forthcoming piece of work about the construction of Thompson groups. We begin by recalling the definition of PRO.
\begin{defn} A \emph{PRO} $\Ofrak$ is a quadruple $(\Ofrak,\odot,\otimes,\id)$ where:
\begin{itemize} 
    \item $\Ofrak$ is a collection of sets $\Big(\Ofrak\brbinom{n}{m}\Big)_{n,m\geq 1}$,
    \item $\odot$ is an associative product (\emph{vertical product}) with two-sided unit $\id$
    $$
    \odot\colon \Ofrak\brbinom{n}{t}\times \Ofrak\brbinom{s}{n}\longrightarrow \Ofrak\brbinom{s}{t},\quad  \id_{n}\in\Ofrak\brbinom{n}{n},
    $$
    \item $\otimes$ is an associative product (\emph{horizontal product})
    $$
    \otimes\colon \Ofrak\brbinom{n_1}{m_1}\times \Ofrak\brbinom{n_2}{m_2}\longrightarrow \Ofrak\brbinom{n_1+n_2}{m_1+m_2}.
    $$
\end{itemize}
Moreover, the vertical and horizontal product are required to satisfy the \emph{interchange law}
$$
(f\otimes g)\odot (p\otimes q)=(f\odot p)\otimes (g\odot q),
$$
whenever it makes sense, and  $\id_{n}\otimes\id_{m}=\id_{n+m}$ for all $n,m\geq 1$.
\end{defn}

More abstractly, a \emph{PRO} $\mathfrak{O}$ is a strict (non unital) monoidal category equipped with a strict monoidal functor $(\mathbb{N}_{\geq 1},+)\to (\mathfrak{O},\otimes)$ which is an isomorphism on objects. We do not consider monoidal units, that is, units for $\otimes$, because they will create nullary operations that we are avoiding.



\begin{examp}[Symmetric groups]
The family of symmetric groups $\{\Sigma_n\}_{n\ge 1}$ yields a very simple PRO by setting
$$
\Sigma\brbinom{n}{m}= \begin{cases}
    \Sigma_n & \mbox{if $n=m$}, \\
    \emptyset & \mbox{if $n\neq m$}.
\end{cases}
$$
The vertical product $\odot$ is just the group structure of the symmetric groups and the horizontal product $\otimes$ is the block product of permutations (see Figure ~\ref{fig:HorizontalProductSigma}).
\begin{figure}[h]
    \centering
    \includegraphics[width=6cm]{figures/HorizontalProductSigma.pdf}
    \caption{Block product of $(1\;4)(2\;3)$ and $(1\;2\;3)$.}
    \label{fig:HorizontalProductSigma}
    \end{figure}
We denote by $\Sigma$ the PRO of symmetric groups. Additionally, we consider that $\Sigma$ is equipped with \emph{cloning maps} for any $n$-tuple of positive integers $(m_1,\dots,m_n)$, 
$
c(m_1,\dots,m_n)\colon \Sigma\brbinom{n}{n}\to \Sigma\brbinom{m_1+\dots+m_n}{m_1+\dots+m_n},
$ 
obtained by replacing the $i$-th strand by $m_i$ strands for all $1\leq i\leq n$.
\end{examp}

\begin{examp}[Signed symmetric groups]
The family of signed symmetric groups $\{\Sigma_{n}^{\pm}\}_{n\ge 1}$ provides a slightly more complicated PRO. First, define 
$$
\Sigma^{\pm}\brbinom{n}{m}= \begin{cases}
    \Sigma_{n}^{\pm} & \mbox{if $n=m$}, \\
    \emptyset & \mbox{if $n\neq m$}.
\end{cases}
$$
The vertical product $\odot$ is just the group structure of the signed symmetric groups and the horizontal product $\otimes$ is the block product of signed permutations, which is defined analogously to the block product of ordinary permutations.
We denote by $\Sigma^{\pm}$ the PRO of signed symmetric groups. Additionally, we consider that $\Sigma^{\pm}$ is equipped with \emph{cloning maps} for any $n$-tuple of positive integers $(m_1,\dots,m_n)$, 
$
c(m_1,\dots,m_n)\colon \Sigma^{\pm}\brbinom{n}{n}\to \Sigma^{\pm}\brbinom{m_1+\dots+m_n}{m_1+\dots+m_n},
$ 
obtained by ``plumbing'' $m_i$ strands in the $i$-th component for $1\leq i\leq n$ subject to the rules:
\begin{itemize}
\item if $g(+i)$ is positive, the new $m_i$ strands are plumbed with the identity permutation;
\item if $g(+i)$ is negative, the new $m_i$ strands are plumbed with the permutation 
$$
\left(\begin{matrix}
1 & 2 & \cdots & m_i-1 & m_i\\
m_i & m_i-1 & \cdots & 2 & 1
\end{matrix}\right).
$$
\end{itemize}
\begin{figure}[h]
    \centering
    \includegraphics{figures/Kappa_Signed.pdf}
    \caption{Cloning map on $\Sigma^{\pm}$.}
    \label{fig:KappaSigned}
    \end{figure}
\end{examp}


We will also need the concept of a \emph{morphism of PROs} $f\colon \Ofrak\to\Ofrak'$, which consists of maps $f_{n,m}\colon\Ofrak\brbinom{n}{m}\to \Ofrak'\brbinom{n}{m}$ for every $n,m\ge 1$ that preserve the PRO structure.

Next, we define two special types of PROs, by adding some extra structure, which will turn out to be equivalent to operadic cloning systems and their restricted counterparts. 

\begin{defn}
A \textit{cloning PRO} consists of the following data:
\begin{itemize}
    \item a PRO $(\G,\odot,\otimes,\id)$ such that the vertical product on $\G\brbinom{n}{n}$ admits inverses, that is, $\G\brbinom{n}{n}$ is a group for very $n$, and $\G\brbinom{n}{m}=\emptyset$ if $n\neq m$, 
    \item a morphism of PROs $\pi\colon\G\to\Sigma^{\pm}$,
    \item cloning maps 
    $\kappa(\underline{m})\colon \G\brbinom{n}{n}\rightarrow\G\brbinom{m_1+\dots+m_n}{m_1+\dots+m_n}$ for every $n\geq 1$ and every $n$-tuple $\underline{m}=(m_1,\dots,m_n)$ with $m_i\geq 1$, satisfying $\kappa(1,\stackrel{(n)}{\dots}, 1)=\id_{\G_n}$, and  the associativity condition
    $$
    \kappa(\underline{r}^1\sqcup\cdots\sqcup\underline{r}^{n})\circ\kappa(\underline{m})=\kappa\Big(\sum_{j_1}r^1_{j_1},\dots,\sum_{j_n}r^n_{j_n}\Big) 
    ,
    $$
    where $\underline{r}^i=(r^i_1,\dots,r^i_{m_i})$ and  
    $
    \underline{r}^1\sqcup\cdots\sqcup\underline{r}^n=(r^1_1,\dots,r^1_{m_1},\dots,r^n_1,\dots,r^n_{m_n}).
    $
\end{itemize}
Moreover, the PRO structure is compatible with $\pi$ and $\kappa$ as stated by the following conditions:
\begin{enumerate}
    \item $\pi$ commutes with  $\kappa$: $\pi\circ\kappa(\underline{m})=c(\underline{m})\circ\pi$.
     \item Identities and $\kappa$: $\kappa(\underline{m})(\id_{n})=\id_{m_1+\cdots+m_n}$.
    \item Horizontal composition and $\kappa$:
    $$\kappa(\underline{m})(f)\otimes \kappa(\underline{n})(g)= \kappa(\underline{m}\sqcup\underline{n})(f\otimes g).$$
    \item Vertical composition and $\kappa$:
    $$\kappa(\underline{m})(h\odot g)=\kappa(\underline{m}_{\pi(g)})(h)\odot \kappa(\underline{m})(g),$$ where  $\underline{m}_{\pi(g)}=(m_{\pi(g)(1)},\dots,m_{\pi(g)(n)})$. % provided that $\underline{m}=(m_1,\dots,m_n)$ \ja{quitamos lo del provided?}.
    \item[($\star$)] Twisted interchange law: $$\kappa(\underline{m})(f)\odot (g_{1}\otimes \cdots\otimes g_{n})=(g_{\pi(f)(1)}\otimes \cdots \otimes g_{\pi(f)(n)})\odot \kappa(\underline{m})(f).$$
\end{enumerate}
\end{defn}
\begin{defn}
    A cloning PRO $(\G,\odot,\otimes, \id,\pi,\kappa)$ is said to be \emph{restricted} if the morphism of PROs $\pi\colon \G\to \Sigma^{\pm}$ factors through the PRO of symmetric groups $\Sigma$.
\end{defn}


After these definitions, we prove that every (restricted) operadic cloning system gives rise to a (restricted) cloning PRO. More concretely: 

\begin{prop}\label{prop:NCCStoPRO} A (restricted) operadic cloning system $(\G_{\bullet},\iota,\zeta,\kappa,\pi)$ defines a (restricted) cloning PRO  $(\G,\odot,\otimes,\id,\pi,\kappa)$ in a functorial way.
\end{prop}
\begin{proof}
Set $\G\brbinom{n}{n}=\G_n$, the vertical product $\odot$ to be the group multiplication and $\id_n=e_n$. Define the cloning maps
$$
\kappa(\underline{m})=\kappa(m_1,\dots,m_n)=\kappa_1(m_1)\circ\cdots \circ \kappa_{n}(m_n),
$$
and the horizontal product $\otimes$
$$
\begin{tikzcd}
\otimes\colon \G\brbinom{n}{n}\times \G\brbinom{m}{m}\ar[rr,"\iota(m)\times \zeta(n)"] && \G\brbinom{n+m}{n+m}\times\G\brbinom{n+m}{n+m}\ar[r, "\cdot"] & \G\brbinom{n+m}{n+m} 
\end{tikzcd}.
$$

We now check that $(\G,\odot,\otimes,\id,\pi,\kappa)$ is a (restricted) cloning PRO by verifying all the axioms. Let $f\in\G_n$, $g\in\G_m$ and $h\in \G_r$.
\begin{itemize}
    \item Associativity of $\otimes$: 
\begin{align*}
(f\otimes g)\otimes h &=  \iota(r)\big(\iota(m)(f)\cdot \zeta(n)(g)\big)\cdot \zeta(n+m)(h)\\
&= \iota(r+m)(f)\cdot \iota(r)(\zeta(n)(g))\cdot \zeta(n+m)(h)\\
&= f\otimes (g\otimes h),
\end{align*}
by condition (vi) of bilateral cloning systems and since $\iota$ and $\zeta$ are homomorphisms.
\item Interchange law for $\odot$ and $\otimes$: 
\begin{align*}
(f\otimes g)\odot (p\otimes q) &=  \Big(\iota(m)(f)\cdot \zeta(n)(g)\Big)\cdot\Big(\iota(m)(p)\cdot\zeta(n)(q)\Big)\\
&= \iota(m)(f)\cdot \iota(m)(p)\cdot \zeta(n)(g)\cdot\zeta(n)(q)\\
&= (f\odot p)\otimes (g\odot q),
\end{align*}
by condition (ix) and since $\iota$ and $\zeta$ are homomorphisms.

\item $\pi\colon \G\to \Sigma^{\pm}$ (resp.\ $\pi\colon \G\to \Sigma$) is a morphism of PROs, since $\pi$ commutes with $\iota, \kappa,\zeta$ and it consists of group homomorphisms.

\item Associativity conditions for $\kappa$ hold by conditions (iv) and (iv+) (axioms for compositions of $\kappa$'s for operadic/bilateral cloning systems).

\item Commutation of $\kappa$ and $\pi$ holds by the analogous axiom for (restricted) operadic cloning systems, i.e.\ (ii') (resp.\ (ii+)).

\item $\kappa$ acting on identities. Since
$$
\kappa_j(r)(e_m)=\kappa_j(r)(e_m)\cdot\kappa_j(r)(e_m),
$$
it follows that $\kappa_j(r)(e_m)=e_{m+r-1}$.

\item Relation between $\kappa$ and $\odot$. By iterating condition (v) we get 
\begin{multline*}
\kappa(\underline{m})(f\odot p) =  \kappa_1(m_1)\circ\cdots\circ \kappa_n(m_n)(f\cdot p)\\
=  \Big(\kappa_{p(1)}(m_1)\circ\cdots\circ\kappa_{p(n)}(m_n)(f)\Big)\cdot \Big(\kappa_1(m_1)\circ\cdots\circ\kappa_n(m_n)(p)\Big)\\
= \kappa(\underline{m}_{p})(f)\odot \kappa(\underline{m})(p).
\end{multline*}

\item Relation between $\kappa$ and $\otimes$. Let us analyze what happens for $\kappa_j(r)$ instead of a general $\kappa(\underline{m}^1\sqcup\underline{m}^2)$, because the general case will follow by combining these simple cases by conditions (iv) and (iv+). 

\begin{itemize}
    \item[\textbf{Case 1:}] ($1\leq j\leq n$). 
    \begin{align*}
 \kappa_j(r)(f\otimes g) &=  \kappa_j(r)\big(\iota(m)(f)\cdot\zeta(n)(g)\big)\\
 &=  \kappa_{\zeta(n)(g)(j)}(r)(\iota(m)(f))\cdot \kappa_j(r)(\zeta(n)(g))\\
 &=\kappa_j(r)(\iota(m)(f))\cdot \kappa_j(r)(\zeta(n)(g))\\
 &=\iota(m)\big(\kappa_j(r)(f)\big)\cdot \zeta(r+n)(g)\\
 &= \kappa_j(r)(f)\otimes g.
 \end{align*}
We have applied conditions (v), (iii), (vii') and that $\zeta(n)(g)$ is constant over $\{1,\dots, n\}$.
    \item[\textbf{Case 2:}] ($n<j\leq m+n$).
    \begin{align*}
 \kappa_j(r)(f\otimes g) 
 &=  \kappa_{n+g(j)}(r)(\iota(m)(f))\cdot \kappa_j(r)(\zeta(n)(g))\\
 &=\iota(m+r)(f)\cdot \zeta(n)(\kappa_j(r)(g))\\
 &= f\otimes \kappa_j(r)(g).
 \end{align*}
We have applied conditions (v), (iii'), (vii) and that $\zeta(n)(g)(j)$ in this case is $n+g(j)$ which is strictly bigger than $n$ (recall that we are always assuming that elements in $\G_m$ or $\Sigma^{\pm}_m$ act on $\{1,\dots, m\}$ through their canonical map into $\Sigma_m$).
\end{itemize}

\item Twisted interchange law. Let us just check the case $f\in \G_2$ ($n=2$) and $g_i\in \G_{m_i}$, since the general case follows from the same ideas.
 \begin{align*}
\kappa(m_1,m_2)(f)\odot (g_1\otimes g_2) 
 &= (\kappa_1(m_1)\circ\kappa_2(m_2))(f)\cdot \Big(\iota(m_2)(g_1)\cdot\zeta(m_1)(g_2)\Big) \\
 &\overset{(a)}{=} \Big(\kappa_1(m_1)\big(\kappa_2(m_2)(f)\big)\cdot \nu_{1}(m_2)(g_1)\Big)\cdot \zeta(m_1)(g_2)\\
 &\overset{(b)}{=} \Big(\nu_{f(1)}(m_2)(g_1)\cdot \kappa_{1}(m_1)\big(\kappa_{2}(m_2)(f)\big)\Big)\cdot \zeta(m_1)(g_2)\\
  &\overset{(c)}{=} \nu_{f(1)}(m_2)(g_1)\cdot \kappa_{1}(m_1)\Big(\kappa_{2}(m_2)(f)\cdot \nu_2(1)(g_2)\Big)\\
  &\overset{(d)}{=} \nu_{f(1)}(m_2)(g_1)\cdot \kappa_{1}(m_1)\Big(\nu_{f(2)}(1)(g_2)\cdot\kappa_{2}(m_2)(f)\Big)\\
   &\overset{(e)}{=} \nu_{f(1)}(m_2)(g_1)\cdot \nu_{f(2)}(m_1)(g_2)\cdot\kappa(m_1,m_2)(f)\\
 &\overset{\phantom{(e)}}{=} (g_{f(1)}\otimes g_{f(2)})\odot \kappa(m_1,m_2)(f).
 \end{align*}
In the above equalities, we have applied: $(a)$ definition of $\nu$; $(b)$ condition (viii); $(c)$ condition (vii') and definition of $\nu$; $(d)$ condition (viii); and $(e)$ conditions (vii) and (iv+). \hfill\qedhere
\end{itemize}
\end{proof}

Next, we prove that (restricted) operadic cloning systems come from (restricted) cloning PROs: 

\begin{prop}\label{prop:PROsToNCCS} A (restricted) cloning PRO $(\G,\odot,\otimes,\id,\pi,\kappa)$ determines a (restricted) operadic cloning system $(\G_{\bullet},\iota,\zeta,\kappa,\pi)$ in a functorial way.
\end{prop}
\begin{proof}
Set $\G_n=\G\brbinom{n}{n}$, the product $\cdot=\odot$ and $e_n=\id_n$. Define
$$
\kappa_j(m)=\kappa(1,\dots,1,m^{(j)},1,\dots,1), \quad 
\iota(f)=f\otimes \id \quad \text{and} \quad \zeta(f)=\id\otimes f.
$$

The only non straightforward axioms to check are the following:
\begin{itemize}
    \item $\iota$ is a homomorphism of groups. We have that,
    $$
    \iota(m)(f\cdot g)= (f\odot g)\otimes \id_m= (f\otimes \id_m)\odot(g\otimes \id_m)= \iota(m)(f)\cdot\iota(m)(g)
    $$
    by the interchange law and since $\id_m$ is idempotent. A similar argument proves that $\zeta$ is also a homomorphism of groups.
    \item Axioms (iii) and (vii). We have that
    $$
    \kappa_j(r)\iota(m)(f)=\left\lbrace 
    \begin{array}{lcl}
    \kappa_j(r)(f)\otimes \id_m=\iota(m)\kappa_j(r)(f) && \text{if }1\leq j\leq n,\\[4mm]
   f\otimes \id_{m+r-1}=\iota(m+r-1)(f) && \text{otherwise},
    \end{array}
    \right.
    $$
    where $f\in\G\brbinom{n}{n}$. The corresponding conditions (iii') and (vii') for $\zeta$ can be deduced in the same way.
    \item Axiom (vi). We have that 
    $$
    \zeta(m)\iota(n)(f)=\id_m\otimes f\otimes \id_n= \iota(n)\zeta(m)(f)
    $$
    by the associativity of the horizontal product $\otimes$.
    \item Axiom (viii). We have that
    \begin{align*}
 \kappa_j(m)(f)\cdot \nu_j(n)(g)&= \kappa(1,\dots,m^{(j)},\dots, 1)(f)\odot(\id_{j-1}\otimes\, g\otimes \id_{n-j+1})\\
 &= (\id_{f(j)-1}\otimes\, g\otimes \id_{n-f(j)+1})\odot \kappa(1,\dots,m^{(j)},\dots, 1)(f)\\ 
 &=\nu_{f(j)}(n)(g)\cdot \kappa_j(m)(f),
 \end{align*}
 by the twisted interchange law and the identity $\id_{r}\otimes\id_{s}=\id_{r+s}$.
    \item Axiom (ix). We have that
     \begin{align*}
 \iota(m)(f)\cdot \zeta(n)(g)&= (f\otimes \id_m)\odot(\id_n\otimes\, g)\\
 &= (f\odot \id_n)\otimes (\id_m\odot\, g)\\ 
 &= (\id_n\odot\, f)\otimes (g\odot \id_m)\\
  &= (\id_n\otimes\, g)\odot (f\otimes \id_m)\\
 &=\zeta(n)(g)\cdot \iota(m)(f),
 \end{align*}
    by the interchange law and since $\id_r$ is an identity for $(\G\brbinom{r}{r},\odot)$. \hfill\qedhere
\end{itemize}
\end{proof}

Finally, we have the following theorem, which ties together the results in this section: 


\begin{thm} The functorial constructions of Propositions \ref{prop:NCCStoPRO} and \ref{prop:PROsToNCCS} induce isomorphisms of categories
$$
\begin{Bmatrix}
\text{(restricted) operadic}\\
\text{cloning systems}
\end{Bmatrix} \cong 
\begin{Bmatrix}
\text{(restricted)}\\
\text{cloning PROs}
\end{Bmatrix}.
$$
\end{thm}
\begin{proof}
Let us first check that
$$
\begin{Bmatrix}
\text{(restricted) operadic}\\
\text{cloning systems}
\end{Bmatrix} \to 
\begin{Bmatrix}
\text{(restricted)}\\
\text{cloning PROs}
\end{Bmatrix}\to 
\begin{Bmatrix}
\text{(restricted) operadic}\\
\text{cloning systems}
\end{Bmatrix}.
$$
is the identity. Let $(\G_{\bullet},\iota,\zeta,\kappa,\pi)$ be a (restricted) operadic cloning system, and let $(\hat{\G}_{\bullet},\hat{\iota},\hat{\zeta},\hat{\kappa},\hat{\pi})$ be the image of the previous (restricted) operadic cloning system under the composition of functors. By definition $\hat{\G}=\G$ as groups and $\hat{\pi}=\pi$. For $\hat{\iota}$, we have that
\begin{align*}
\hat{\iota}(g) &= g\otimes \id = \iota(1)(g)\cdot \zeta(n)(e_1)=\iota(g)\cdot e_{n+1}=\iota(g).
\end{align*}
The maps $\zeta$ behave similarly, and
\begin{align*}
\hat{\kappa}_j(g) &= \kappa(1,\dots,2^{(j)},\dots,1)(g)=\big(\kappa_1(1)\circ\cdots\circ\kappa_j(2)\circ\cdots\circ\kappa_n(1)\big)(g)=\kappa_j(g).
\end{align*}

Conversely, let $(\G,\odot,\otimes,\id,\pi,\kappa)$ be a (restricted) cloning PRO and let the tuple  $(\hat{\G},\hat{\odot},\hat{\otimes},\hat{\id},\hat{\pi},\hat{\kappa})$ denote the image of it under the other composition of functors. It is clear that $\hat{\G} = \G$, $\hat{\odot}=\odot$, $\hat{\id}=\id$, $\hat{\pi}=\pi$ and $\hat{\kappa}=\kappa$. Finally, regarding the horizontal product we have that
\begin{align*}
f\,\hat{\otimes}\, g &= \iota(m)(f)\cdot \zeta(n)(g)= (f\otimes \id_m)\odot (\id_n\otimes g)\\ 
&= (f\odot \id_n)\otimes(\id_m \odot g) = f\otimes g. \qedhere
\end{align*}
\end{proof}


%\bibliographystyle{aomalpha}
%\bibliography{references.bib}
% This must be in the first 5 lines to tell arXiv to use pdfLaTeX, which is strongly recommended.
\pdfoutput=1
% In particular, the hyperref package requires pdfLaTeX in order to break URLs across lines.

\documentclass[11pt]{article}

% Remove the "review" option to generate the final version.
%\usepackage[review]{ACL2023}
\usepackage{ACL2023}

% Standard package includes
\usepackage{times}
\usepackage{latexsym}

% For proper rendering and hyphenation of words containing Latin characters (including in bib files)
\usepackage[T1]{fontenc}
% For Vietnamese characters
% \usepackage[T5]{fontenc}
% See https://www.latex-project.org/help/documentation/encguide.pdf for other character sets

% This assumes your files are encoded as UTF8
\usepackage[utf8]{inputenc}

% This is not strictly necessary, and may be commented out.
% However, it will improve the layout of the manuscript,
% and will typically save some space.
\usepackage{microtype}

% This is also not strictly necessary, and may be commented out.
% However, it will improve the aesthetics of text in
% the typewriter font.
\usepackage{inconsolata}


% If the title and author information does not fit in the area allocated, uncomment the following
%
%\setlength\titlebox{10cm}
%
% and set <dim> to something 5cm or larger.

%%%%%%%%%%%%%%%%%%%%%%%%%%%%%%%%%%
\usepackage{graphicx}
\usepackage{amsfonts}
\usepackage{amsmath}
\usepackage{bigdelim}
\usepackage{diagbox}
\usepackage{amsthm}
\usepackage{makecell}
\usepackage{mathtools}
\usepackage{booktabs}
\usepackage[shortlabels]{enumitem}
\graphicspath{ {figs/} }

\theoremstyle{remark}
\newtheorem*{question}{Question}

\newcommand{\tk}[1]{\textcolor{blue}{{#1}}}
\newcommand{\sy}[1]{\textcolor{red}{{#1}}}
\newcommand{\mg}[1]{\textcolor{purple}{{#1}}}
\newcommand{\lh}[1]{\textcolor{green}{{#1}}}
\newcommand{\lc}[1]{\textcolor{green}{{#1}}}

% Rounded color box
\definecolor{light_blue}{HTML}{cfdfff}
\usepackage[most]{tcolorbox}
\tcbset{on line, 
        boxsep=1pt, left=0pt,right=0pt,top=0pt,bottom=0pt,
        colframe=white,colback=light_blue,  
        highlight math style={enhanced}
        }

\newcommand{\quash}[1]{}  %Anything in \quash is ignored
\newcommand{\gpt}{\textsc{GPT-2}}
\newcommand{\bert}{\textsc{BERT}}
\newcommand{\bertlarge}{\textsc{BERT-large}}
\newcommand{\mask}{\texttt{[MASK]}}
\newcommand{\cls}{\texttt{[CLS]}}
\newcommand{\sep}{\texttt{[SEP]}}
\newcommand{\mat}{\texttt{mat}}
\newcommand{\id}{\texttt{id}}
\newcommand{\matl}{\texttt{mat}_{\ell \rightarrow \ell'}}
\newcommand{\matattnl}{\texttt{mat\_attn}_{\ell \rightarrow \ell'}}
\newcommand{\matffl}{\texttt{mat\_ffn}_{\ell \rightarrow \ell'}}
\newcommand{\matlnl}{\texttt{mat\_ln1\_ln2}_{\ell \rightarrow \ell'}}
\newcommand{\idl}{\texttt{id}_{\ell \rightarrow \ell'}}
\newcommand{\matlL}{\texttt{mat}_{\ell \rightarrow L}}
\newcommand{\matattnlL}{\texttt{mat\_attn}_{\ell \rightarrow L}}
\newcommand{\matfflL}{\texttt{mat\_ffn}_{\ell \rightarrow L}}
\newcommand{\matlnlL}{\texttt{mat\_ln1\_ln2}_{\ell \rightarrow L}}
\newcommand{\idlL}{\texttt{id}_{\ell \rightarrow L}}

\definecolor{blue(munsell)}{rgb}{0.0, 0.5, 0.69}
%%%%%%%%%%%%%%%%%%%%%%%%%%%%%%%%%%

\title{Jump to Conclusions: Short-Cutting Transformers\\With Linear Transformations}

% Author information can be set in various styles:
% For several authors from the same institution:
% \author{Author 1 \and ... \and Author n \\
%         Address line \\ ... \\ Address line}
% if the names do not fit well on one line use
%         Author 1 \\ {\bf Author 2} \\ ... \\ {\bf Author n} \\
% For authors from different institutions:
% \author{Author 1 \\ Address line \\  ... \\ Address line
%         \And  ... \And
%         Author n \\ Address line \\ ... \\ Address line}
% To start a seperate ``row'' of authors use \AND, as in
% \author{Author 1 \\ Address line \\  ... \\ Address line
%         \AND
%         Author 2 \\ Address line \\ ... \\ Address line \And
%         Author 3 \\ Address line \\ ... \\ Address line}

\author{Alexander Yom Din$^{1}$ ~~~~~ Taelin Karidi$^{1}$ ~~~~~ Leshem Choshen$^{1}$ ~~~~~
Mor Geva$^{2}$ 
\vspace{0.2cm} \\
$^1$Hebrew University of Jerusalem ~~~ $^2$Google Research \\
\small{\texttt{\{alexander.yomdin, taelin.karidi, leshem.choshen\}@mail.huji.ac.il}}, \small{\texttt{pipek@google.com}}}

\quash{
\author{Alexander Yom Din \\
  Hebrew University of Jerusalem \\ \texttt{alexander.yomdin@mail.huji.ac.il} \\\And
  Taelin Karidi \\
  Hebrew University of Jerusalem \\
  \texttt{taelin.karidi@mail.huji.ac.il} \\\And
  Leshem Choshen \\
  Hebrew University of Jerusalem \\ \texttt{leshem.choshen@mail.huji.ac.il} \\\And
  Mor Geva \\
  Google Research \\
  \texttt{pipek@google.com} \\}
}

\begin{document}
\maketitle



\begin{abstract}
% \vspace{-1em}
The diffusion-based generative models have achieved remarkable success in text-based image generation. However, since it contains enormous randomness in generation progress, it is still challenging to apply such models for real-world visual content editing, especially in videos. 
In this paper, we propose \texttt{FateZero}, a zero-shot text-based editing method on real-world videos without per-prompt training or use-specific mask. 
\RM{Specifically, different from a pipeline of two independent inversion and then generation stages, we find the intermediate attention maps during inversions store better structure and motion information. We thus reform them to temporally casual attention and replace them in the generation progress. To further reduce the unnecessary semantic leakage of source video and enhance the editing quality, we then remix the temporally casual attentions via the cross-attention features of the source prompt as the mask.}
To edit videos consistently, we propose several techniques based on the pre-trained models. Firstly, in contrast to the straightforward DDIM inversion technique, our approach captures intermediate attention maps during inversion, which effectively retain both structural and motion information. These maps are directly fused in the editing process rather than generated during denoising. To further minimize semantic leakage of the source video, we then fuse self-attentions with a blending mask obtained by cross-attention features from the source prompt. Furthermore, we have implemented a reform of the self-attention mechanism in denoising UNet by introducing spatial-temporal attention to ensure frame consistency.
Yet succinct, our method is the first one to show the ability of zero-shot text-driven video style and local attribute editing from the trained text-to-image model. We also have a better zero-shot shape-aware editing ability based on the text-to-video model~\cite{tuneavideo}. \RM{Besides video, our unified method also achieves state-of-the-art performance in zero-shot image editing.\chenyang{Need exp or remove the zero-shot image}} Extensive experiments demonstrate our superior temporal consistency and editing capability than previous works.
% The code will be released.
% \chenyang{emphasize: our observation at inversion time} \xiaodong{replacing the bold part to the actual pipeline: \textbf{Specifically, we work on replacing and mixing the attention maps between the inversion and generation since the self-attention map keeps the structure of the original natural image and the cross-attention is semantic-related, after remixing, we replace them in the corresponding generation steps for denoising.}}
% \footnote{Since there is no general video diffusion model is publicly available, we use one-shot video generation method~(Tune-A-Video~\cite{tuneavideo}) as the pretrained video diffusion model for zero-shot video editing\xiaodong{can be removed if we actually zero-shot on video}.}.
\end{abstract}
\section{Introduction}

The ability to reason about plans is critical for performing long-horizon tasks \citep{erol1996hierarchical, sohn2018hierarchical, sharma-etal-2022-skill}, compositional generalization \citep{corona-etal-2021-modular} and generalization to unseen tasks and environments \citep{shridhar2020alfred}.
Consider a simple long-horizon planning scenario where a robot is tasked with preparing a meal and serving it on the table. 
This presents a non-trivial planning problem since the agent needs to understand the sequence of operations required to perform the task and search for the relevant objects in the unfamiliar environment by interacting with various objects. %



Large language models have been recently shown to possess commonsense knowledge about the world such as object affordances and physical dynamics \citep{ouyang2022training,chowdhery2022palm}.
Early approaches considered text based environments and fine-tuned PLMs to predict actions given the history of past observations and actions \citep{jansen-2020-visually,micheli-fleuret-2021-language,yao-etal-2020-keep}.
Recent work has used this ability to reason about plans from text instructions in simulated household environments with simplifying assumptions such as text-only environment observations or feedback \citep{huang2022language,ahn2022can,li2022pre,logeswaran-etal-2022-shot}.


We focus on \emph{visually grounded planning} with PLMs --- the ability to adapt plans based on interaction and visual feedback from the environment.
While PLMs have strong planning commonsense priors, predictions from a PLM may not be directly realizable in the environment since the observation and action spaces are unknown.
This requires \emph{grounding} the PLM in the environment and adapting it to observe visual feedback, which is highly non-trivial.
Some prior works assume the availability of a pre-trained affordance function \citep{ahn2022can} or a success detector \citep{mirchandani2021ella}.
Notably, SayCan \citep{ahn2022can} completely decouples the PLM from observation information by selecting actions that have both high affordability (through a pre-trained affordance model) and high PLM likelihood.
Although this partially addresses the grounding problem, the use of visual feedback for action affordance alone is limited.
Often an agent must choose one of many affordable actions using information from observations.
For example, a driving agent should re-navigate and possibly turn around when encountering a ``road closed'' sign, but both turning around and driving forward are indistinguishable to SayCan because they are both affordable and the PLM is blind to observations.

Another workaround explored in prior work is translating the information in the visual observations to text using a pre-trained captioning system \citep{shridhar2021alfworld,huang2022language}.
However, it can be difficult to faithfully describe an image in words and information is lost in this inherently noisy process, which limits the information available to the planner.



Recent work shows that PLMs can be adapted for various natural language tasks by inserting tunable embeddings or soft prompts at the input of the PLM (also called prompt tuning or prefix tuning)~\citep{li-liang-2021-prefix,lester-etal-2021-power}.
This approach also extends to multi-modal understanding tasks such as image captioning \citep{mokady2021clipcap} and VQA \citep{tsimpoukelli2021multimodal} where images are encoded as soft prompts and finetuned for the target task.
Transformer based architectures have also been successfully applied to offline Reinforcement Learning in recent work \citep{chen2021decision,janner2021offline,li2022pre,reid2022can}.

Taking inspiration from these works, we propose the simple approach of embedding visual observations (`visual prompts') and \textit{directly inserting them as PLM input embeddings}.
The visual encoder and PLM are jointly trained for the target task, an approach we call \textbf{\oursfull}~(\ours).
By teaching the PLM to use observations for planning in an end to end manner, we remove the dependency on external data such as captions and affordability information that was used in prior work.
We show that this simple approach performs better than prior PLM-based planning approaches on two embodied planning benchmarks based on ALFWorld~\citep{shridhar2021alfworld} and Virtualhome~\cite{puig2018virtualhome}.



\section{Related Work}

%Here we summarize prior work on transfer learning and property inference.

%\shortsection{Transfer Learning}
%%Transfer learning reuses features learned by pre-trained models for new tasks, with the pretext that inherent similarities in the generic features will be useful for the downstream tasks and hence reducing their cost of downstream training. Specifically, the downstream model trainer will use a pre-trained upstream model as the starting point for the downstream training, with inclusion of (or replacement with) the task-specific classification layer/module. The downstream model is then trained by either updating all layers of the model (including ones reused from upstream model) or freezing some earlier layers of the reused parts as the ``feature extractor'' and only updating the rest. The latter approach is more popular as the reused feature extractors can already learn useful feature representations and the training cost is also much lower and affordable for individuals with limited computational resources. We study the vulnerability of the latter transfer learning approach in this paper. 


%\shortsection{Transfer Learning} 
Several works have demonstrated risks associated with transfer learning across a variety of attack goals. Wang et al.~\cite{wang2018great} and Yao et al.~\cite{yao2019latent} consider manipulating the upstream model such that the fine-tuned downstream models contain backdoors, misclassifying test inputs that contain predefined backdoor triggers. These transfer manipulations are tailored to their particular attack goals and cannot be applied for the property inference goal considered in this paper. Zou et al.~\cite{zou2020privacy} study the threat of membership inference attacks on transfer learning, but with normally trained upstream models.  
%\dnote{its clear that the goals are different for these attacks, but how similar are the methods?} \ynote{similarity of the methods? more details about the methods? do not know what is expected here}
%In contrast, we investigate the possibility of boosting the effectiveness of property inference by manipulating the upstream model training. % Schuster et al.~\cite{schuster2020humpty} show that the attacker can modify the corpus on which the word embedding is trained such that the downstream NLP models which use that embedding will behave abnormally.

%\shortsection{Property Inference}
The risk of property inference was introduced by Ateniese et al.~\cite{ateniese2015hacking}, % introduces the threat of inferring properties of the training data from pre-trained models, 
and several subsequent works have developed property inference (also known as distribution inference) attacks~\cite{Wang2022GroupPI, suri2022formalizing, Jurez2022BlackBoxAF, Hartmann2022DistributionIR}.
% Ganju et al.~\cite{ganju2018property} and Suri and Evans~\cite{suri2022formalizing} 
These works study property inference against normally trained models, and they launch attacks using a variety of black-box and white-box attacks. All the white-box attacks use meta-classifiers, which take the permutation-invariant representation~\cite{ganju2018property} of the model parameters as the features. We use the state-of-the-art white-box attack~\cite{suri2022formalizing} in our experiments.
%We will use the state-of-the-art white-box method proposed by Ganju et al.~\cite{ganju2018property} and later extended by suri et al.~\cite{suri2022formalizing} in this paper.
%\dnote{do we use these attacks?} 
Melis et al.~\cite{melis2019exploiting} and Zhang et al.~\cite{zhang2021leakage} focus on property inference in distributed training scenarios. In their settings, the attacker is a participant in the global model training and conducts property inference using meta-classifiers that are trained on model outputs or gradients. Similarly, Suri et al.~\cite{suri2022subject} focus on federated learning settings where the attacker is a participant (or the central server) that utilizes black-box attacks for inferring membership of data from particular subjects. %\dnote{if we use black-box attacks, explain which ones, or how ours are related to previous ones} 
For our experiments, We improve the black-box meta-classifier proposed by Zhang et al.~\cite{zhang2021leakage} using the ``query tuning'' technique in Xu et al.~\cite{xu2019detecting}. 

The closest works to ours are Chase et al.~\cite{saeed} and Chaudhari et al.~\cite{Chaudhari2022SNAPEE}, which both consider a scenario where the attacker can manipulate some of the training data of the model to induce a model that significantly increases property inference risk.
% \dnote{it enables precise property inference attacks?}.
These works assume an adversary with the ability to poison the victim's training data, while the adversary in our scenario has no access to the victim's training data, and therefore, their methods are not applicable.
% \dnote{example how different from ours, and why the methods are not applicable}
%Thus, their methods are not applicable to our transfer learning scenario.
%Their methods rely on inducing certain behavior correlated with the properties to be inferred, and thus are not applicable to our transfer learning scenario. \anote{Still a bit unclear why that is the case.}
%
There are also works similar to ours that leverage ``adversarial initializations'' for attack purposes.
% \cite{grosse2019adversarial, boenisch2021curious, wen2022fishing, fowl2021robbing}.
Grosse et al.~\cite{grosse2019adversarial} focus on scenarios where the attacker can control the parameter initialization of a model, and demonstrate that the attacker can use special initializations to damage the performance of the trained model. %This attack is orthogonal to ours.
Other works \cite{boenisch2021curious, wen2022fishing, fowl2021robbing} show that the malicious central server in a federated learning protocol can reconstruct some training samples via falsifying the global model in some training rounds and then analyzing the submitted gradients. These kinds of attacks do not apply to our transfer-learning scenario since the attacker cannot access the downstream gradients, and can only manipulate the upstream training.

\iffalse %%%%%%%%%%%%%%%%%%%%%%%%%%%%%%%%

In this section, we provide the background and also the summary of prior attacks on transfer learning (Section~\ref{sec:transfer_learning}) and property inference (Section~\ref{sec:property_inference}). Then, we introduce the closely related manipulation attacks against machine learning models to boost different privacy risks in Section~\ref{sec:active_inference_attacks}.

%\anote{Do we really need a dedicated section for this? It's barely 2 paragraphs right now.}

%\dnote{the most closely related work to ours are works that attempt to amplify inference attacks by poisoning models, the two most relevant I know of are \url{https://www.computer.org/csdl/proceedings-article/sp/2022/131600b569/1CIO8nmuota} and \url{https://arxiv.org/abs/2204.00032}, but need to look thoroughly for others. We should definitely be describing this and relating it to our work, probably in the introduction. Most of what is here is Background, but should be clear what this section is for (not muddling background and related work)}

\subsection{Transfer Learning} \label{sec:transfer_learning}
Transfer learning reuses features learned by pre-trained models for new tasks, with the pretext that inherent similarities in generic features can be useful for downstream tasks, thus reducing the cost of downstream training. Specifically, the downstream model trainer uses a pre-trained upstream model as the starting point for downstream training, with the inclusion (or replacement) of task-specific classification layers/modules. The downstream model is then trained by either updating all layers of the model (including ones reused from the upstream model) or freezing some earlier layers of the reused parts as the ``feature extractor'' and only updating the rest. The latter approach is more popular as the reused feature extractors can already learn useful feature representations and the training cost is also much lower and affordable for individuals with limited computational resources. We study the vulnerability of the latter transfer learning approach in this paper. 
%mainly in two ways:  1) all the layers (including ones reused from ) and tune the full model; the other one is to freeze some earlier layers of the model as the feature extractor and only tune the rest later layers. The second update strategy could achieve better efficiency since the frozen layers can already produce meaningful feature representations~\cite{wang2018great,yao2019latent}, and we will study the transfer learning using this strategy. 

Recently, various attacks have been proposed for the transfer learning setting, but with different attack goals from ours. Wang et al.~\cite{wang2018great} generate adversarial examples against black-box student models that transfer knowledge from publicly available teacher models without repeated queries. Yao et al.~\cite{yao2019latent} propose to manipulate the upstream model such that the downstream models derived from the upstream model contain backdoors, which would misclassify test inputs that contain some predefined backdoor triggers. Zou et al.~\cite{zou2020privacy} study the threat of membership inference attacks on transfer learning and the upstream models are trained normally. In contrast, we investigate the possibility of boosting the effectiveness of property inference by manipulating the upstream model training. Schuster et al.~\cite{schuster2020humpty} show that the attacker can modify the corpus on which the word embedding is trained such that the downstream NLP models which use that embedding will behave abnormally.

%This additionally allows model trainers to achieve satisfactory performance with limited training samples, leading to reduced computational costs. The most common approach reuses parameters in the earlier layers of the pre-trained model, either by fixing them as the feature extractor or just using them for initialization, to conduct downstream training.

\subsection{Property Inference} \label{sec:property_inference}

\shortsection{Property Inference Attacks} In property inference attacks, the adversary aims to infer some sensitive properties of some data, given a model trained on it. For example, the adversary may be interested in sensitive properties like the presence of people of a specific race in the dataset~\cite{ateniese2015hacking, melis2019exploiting}), or even be curious about the 
the statistics of the training set (e.g, the ratio of people with a specific gender~\cite{saeed, ganju2018property, suri2022formalizing, zhang2021leakage}).


Ateniese et al.~\cite{ateniese2015hacking} were the first to identify the threat of inferring properties of the training data from pre-trained models. Ganju et al.~\cite{ganju2018property} and Suri and Evans~\cite{suri2022formalizing} 
study property inference against normally trained models, and they launch attacks using white-box meta-classifiers, which utilize the permutation-invariance representation~\cite{ganju2018property} of the model parameters, while other works focus on distributed training~\cite{zhang2021leakage} where the attacker is a participant in the global model training and conducts property inference using meta-classifiers trained on model outputs. Similarly, Suri et al.~\cite{suri2022subject} focus on federated learning, where the attacker is a participant (or the central server) that utilizes black-box attacks for inferring membership of data from particular subjects. Chase et al.~\cite{saeed} propose an active property inference attack for data poisoning scenarios, which we will cover and compare to in Section~\ref{sec:active_inference_attacks}.

%The closest work to ours are by Chase et al.~\cite{saeed} and Tramer et al.~\cite{tramer2022truth}. In their work, the attacker can manipulate some of the training data of the model such that a model trained (from scratch) on the poisoned data has an increased inference risk. However, their methods are not applicable to the transfer learning scenario. 
%In this work, we will focus on the property inference in transfer learning scenarios in which the attacker releases the upstream model and infer sensitive properties of the downstream models tuned from that upstream model.
% 

\shortsection{Defenses}
Defending against property inference attacks is an open problem. There are no studies in the current literature on active adversaries, and only a couple on passive ones. Ma et. al.~\cite{ma2021nosnoop} propose a defense against property inference attacks on data batches in the  collaborative learning setting. However, adversaries in the transfer-learning setting do not have access to batch-wise gradients of the downstream trainer. Chen and Ohrimenko~\cite{chen2022protecting} utilize mechanisms that add carefully-crafted noise to features to provide theoretical guarantees against inference adversaries, but focus on query-based access to the underlying dataset, not a machine learning model trained on it. These existing defenses thus do not apply to our threat model.

%propose a framework that reduces property inference to Boolean functions of individual members, posing the ratio of members satisfying the given function in a dataset as the property. These property inference attacks have since then been proposed as distribution inference attacks~\cite{suri2022formalizing}, presenting such attacks as inferring properties of the distributions used to sample datasets, differentiating them from exact inference attacks like dataset inference~\cite{maini2021dataset}. Nearly all property inference attacks use meta-classifiers to perform inference: training models on versions of datasets with and without the target property, followed by training a meta-classifier on top of these classifiers's model representations. These representations can take several forms: using model weights themselves with permutation-invariance~\cite{ganju2018property}, or model activations or logits for a generated set of query points~\cite{xu2019detecting}. However, the capability of such approaches is limited: the most that these attacks have been shown to work is medium-sized convolutional networks on the CelebA dataset~\cite{suri2022formalizing}.


\subsection{Active Privacy Attacks} \label{sec:active_inference_attacks}
% Perhaps the closely related works to ours as ones that proactively enhance the effectiveness of privacy attacks by manipulating the model training process in certain ways~\cite{saeed, melis2019exploiting, nasr2019comprehensive, tramer2022truth}. 
%shown that the adversary can, by using proactive ways, achieve stronger attacks that infer private information from deep learning systems~\cite{nasr2019comprehensive, melis2019exploiting, tramer2022truth, saeed}. In this section, we introduce the ones that are close to ours.

In the decentralized federated learning training, by submitting specially crafted gradients to the central server, malicious agents can increase membership inference risk~\cite{nasr2019comprehensive} and property inference risks~\cite{melis2019exploiting} of other benign agents' training data. However, these attacks do not apply to transfer learning scenario, as the attacker cannot control model gradients of downstream training. In the centralized setting, researchers propose attacks to poison the victim's training data such that the impacts of attribute inference and membership inference~\cite{tramer2022truth} and property inference~\cite{saeed} attacks are amplified on the poisoned model.
The ability to poison the victim's data is a threat model orthogonal to ours, since we have no access to the victim's downstream data. While there is scope to combine such approaches for stronger attacks (albeit with stronger access assumptions), we choose to focus on the scenario with no read/write access to the victim's data.

\fi %%%%%%%%%%%%%%%%%%%%%%%%%%%%%%%%

\section{Linear Shortcut Across Blocks}
\label{sec:layer_jump}

To use a hidden representation from layer $\ell<L$ as a final representation, we propose to cast it using linear regression, while skipping the computation in-between these layers. More generally, this approach can be applied to cast any $\ell$-th hidden representation to any subsequent layer $\ell'>\ell$.


\subsection{Method}
\label{subsec:methodology_linear_shortcut}

Given a source layer $\ell$ and a target layer $\ell'$ such that $0 \leq \ell < \ell' \leq L$, our goal is to learn a mapping
%$A_{\ell', \ell} \in \mathbb{R}^{d_h \times d_h}$
from hidden representations at layer $\ell$ to those at layer $\ell'$. To this end, we first collect a set of corresponding hidden representation pairs $(h^\ell, h^{\ell'})$. Concretely, we run a set $\mathcal{T}$ of input sequences through the model, and for each input $s$, we extract the hidden representations $h_{i_s}^{\ell}, h_{i_s}^{\ell'}$, where $i_s$ is a random position in $s$.
Next, we learn a matrix $A_{\ell', \ell} \in \mathbb{R}^{d_h \times d_h}$ by fitting linear regression over $\mathcal{T}$, i.e., $A_{\ell', \ell}$ is a numerical minimizer for:
$$ A \mapsto \sum_{s \in \mathcal{T}} || A \cdot h_{i_s}^\ell - h_{i_s}^{\ell'} ||^2,$$ 
and define the mapping of a representation $h$ from layer $\ell$ to layer $\ell'$ as:
\begin{equation}
\label{eq:linear_jump}
    \matl{} (h) \coloneqq A_{\ell', \ell} \cdot h.
\end{equation}


\subsection{Baseline}
\label{subsec:baseline}

We evaluate 
% our method against 
the prevalent approach of ``reading'' hidden representations directly, without any transformation. 
Namely, the propagation of a hidden representation from layer $\ell$ to layer $\ell'$ is given by the identity function, dubbed \id{}:

$$ \idl{} (h) \coloneqq h.$$

% Notably, 
This baseline 
assumes that representations at different layers operate in the same linear space.

\subsection{Quality of Fit}
\label{subsec:experiments_r2}

We first evaluate our method by measuring how well the learned linear mappings approximate the representations at the target layer. To this end, we calculate the (coordinate-averaged) $r^2$-score of our mapping's outputs with respect to the representations obtained from a full inference pass, and compare to the same for the \id{} baseline.


\paragraph{Models.}

We use \gpt{} \cite{radford2019language}, a decoder-only auto-regressive LM, with $L = 48$, $d_h = 1600$, and \bert{} \cite{devlin-etal-2019-bert}, an encoder-only model trained with masked language modeling, with $L=24$, $d_h=1024$.
% \footnote{\label{footnote:hf}We use models and data from Huggingface \cite{wolf-etal-2020-transformers,lhoest-etal-2021-datasets}.}
%For masked token prediction, we use a masked LM head pre-trained for our \bert{} model.

% \footnote{Specifically, we use the Huggingface Transformers \cite{wolf-etal-2020-transformers} implementations of all these models.}

%\sy{We use \gpt{} \cite{radford2019language}, a decoder-only auto-regressive LM, coming in four scales; $\texttt{gpt2}$ ($L = 12$, $d_h = 768$), $\texttt{gpt2-medium}$ ($L = 24$, $d_h = 1024$), $\texttt{gpt2-large}$ ($L = 36$, $d_h = 1280$) and $\texttt{gpt2-xl}$ ($L = 48$, $d_h = 1600$). Also, we use \bert{} \cite{devlin-etal-2019-bert}, an encoder-only model trained with masked language modeling, coming in two scales;  \texttt{bert-base-uncased} ($L=12$, $d_h=768$) and \texttt{bert-large-uncased} ($L=24$, $d_h=1024$). For masked token prediction, we use masked LM heads pre-trained for our models. Specifically, we use the Huggingface Transformers \cite{wolf-etal-2020-transformers} implementations of all these models. The plots presented in this section are for $48$-layered \gpt{} and $24$-layered \bert{}.}

%\sy{We use \gpt{} \cite{radford2019language}, a decoder-only auto-regressive LM, in the Huggingface \cite{wolf-etal-2020-transformers} implementation\footnote{\url{https://huggingface.co/gpt2}}, coming in four scales; $\texttt{gpt2}$ ($L = 12$, $d_h = 768$), $\texttt{gpt2-medium}$ ($L = 24$, $d_h = 1024$), $\texttt{gpt2-large}$ ($L = 36$, $d_h = 1280$) and $\texttt{gpt2-xl}$ ($L = 48$, $d_h = 1600$). Also, we use \bert{} \cite{devlin-etal-2019-bert}, an encoder-only model trained with masked language modeling, in the Hugginface implementation, coming in two scales;  \texttt{bert-base-uncased}\footnote{\url{https://huggingface.co/bert-base-uncased}} ($L=12$, $d_h=768$) and \texttt{bert-large-uncased}\footnote{\url{https://huggingface.co/bert-large-uncased}} ($L=24$, $d_h=1024$). For masked token prediction, we use the \texttt{BertForMaskedLM} heads from Huggingface, pretrained for these models. The plots presented in this section are for $48$-layered \gpt{} and $24$-layered \bert{}.}

\paragraph{Data.}
We sample random sentences from Wikipedia,
% \footref{footnote:hf} 
collecting 9,000 (resp. 3,000) sentences for the training set $\mathcal{T}$ (resp. validation set $\mathcal{V}$).\footnote{We use sentences rather than full documents to simplify the analysis.}
%\sy{We use two data sources to evaluate our method. One is Wikiepdia \cite{lhoest-etal-2021-datasets}\footnote{\url{https://huggingface.co/datasets/wikipedia}}; we use \texttt{spaCy}\footnote{\url{https://spacy.io/}} to divide documents into sentences\footnote{We use sentences rather than full documents to simplify the analysis.}\footnote{We pick randomly a Wikipedia document and then pick randomly a sentence ending in a newline character in it. \sy{[maybe this footnote is not needed?]}}, collecting 9,000 (resp. 3,000) random sentences for the training set $\mathcal{T}$ (resp. validation set $\mathcal{V}$). The second is a news article sentences dataset, the 10K English 2020 news sentences corpus
% \footnote{\url{https://downloads.wortschatz-leipzig.de/corpora/eng_news_2020_10K.tar.gz}} from the Leipzig Corpora Collection \cite{goldhahn-etal-2012-building}, which we randomly divide into a training set $\mathcal{T}$ consisting of 9,000 examples and a validation set $\mathcal{V}$ consisting of 1,000 examples.
% We truncate sentences to the maximal token length allowed by the model \mg{do we ever need to truncate? a sentence has about 10 words and the max. input len is thousands} \sy{[I surely did not need to in Leipzig, but discovered (via a transformers runtime warning) that I do need to for some (probably a minority) of the Wikipedia sentences. This probably has to do with that it is not really ``sentences" necessarily, for example, I noticed that it has some listings or something like that (bulleted items)... So some minority might get very long I guess...]}.
For each example $s$, we select a random position $i_s$ and extract the hidden representations $h_{i_s}^{\ell}$ at that position from all the layers.
For \bert{}, we first replace the input token at position $i_s$ with a \mask{} token, as our motivation is interpreting predictions, which are obtained via masked tokens in \bert{} (see \S\ref{subsec:BERT}).
Thus, in this case, the hidden representations we consider
%in the case of \bert{}
are of \mask{} tokens only.
%As we observed highly similar results for the two data sources across all our experiments, throughout the paper we will mainly report results for Wikipedia (except for \S\ref{sec:robustness}, where we cross-validate).


\begin{figure}[t]
\includegraphics[scale=0.2]{figs/r2_scores_48.pdf}
% \includegraphics[width=\columnwidth]{figs/r2_scores_48.pdf}
\caption{The coordinate-averaged $r^2$-score of $\matl{}$ (left) and $\idl{}$ (right) (\gpt{}).}
\label{fig:r2_scores}
\end{figure}


\begin{figure}[t]
\setlength{\belowcaptionskip}{-10pt}
\includegraphics[scale=0.2]{figs/bertmask_r2_scores_24.pdf}
% \includegraphics[width=\columnwidth]{figs/bertmask_r2_scores_24.pdf}
\caption{The coordinate-averaged $r^2$-score of $\matl{}$ (left) and $\idl{}$ (right) (\bert{}).}
\label{fig:bertmask_r2_scores}
\end{figure}



\paragraph{Evaluation.}
For every pair of layers $\ell, \ell'$, such that $0 \leq \ell < \ell' \leq L$, we use the training set $\mathcal{T}$ to fit linear regression as described in \S\ref{subsec:methodology_linear_shortcut}, and obtain a mapping $\matl{}$. 
Next, we evaluate the quality of $\matl{}$ as well as of $\idl{}$ using the $r^2$-coefficient, uniformly averaged over all coordinates. Concretely, we compute the $r^2$-coefficient of each of the predicted representations $\matl{} (h_{i_s}^{\ell})$ and $\idl{} (h_{i_s}^{\ell})$ versus the true representations $h_{i_s}^{\ell'}$
over all $s \in \mathcal{V}$.
%as we vary $s \in \mathcal{V}$.
%for every $s \in \mathcal{V}$.



\paragraph{Results.}
Results for \gpt{} and \bert{} are presented in Figs.~\ref{fig:r2_scores} and~\ref{fig:bertmask_r2_scores}, respectively.
In both models, \mat{} consistently yields better approximations than \id{}, as it obtains higher $r^2$-scores (in blue) across the network. 
This gap between \mat{} and \id{} is especially evident in \bert{}, where \id{} completely fails to map the representations between most layers, suggesting that hidden representations are modified  substantially by every transformer block.
Overall, this highlights the shortcoming of existing practices to inspect representations in the same linear space, and the gains from using our method to approximate future layers.
% in the network.
\section{Linear Shortcut for Language Modeling}
\label{sec:prediction}

We saw that our method approximates future hidden representations substantially better than a naive propagation. 
In this section, we will show that this improvement also translates to better predictive abilities from earlier layers. Specifically, we will use our method to estimate how often intermediate representations encode the final prediction, in the context of two fundamental LM tasks; next token prediction and masked token prediction.

\paragraph{Evaluation Metrics.}
Let $h, h' \in \mathbb{R}^{d_h}$ be a final representation and a substitute final representation obtained by some mapping, and denote by $\delta (h), \delta (h') \in \mathbb{R}^{d_v}$ their corresponding output probability distributions (obtained through projection to the output vocabulary -- see details below). 
We measure the prediction quality of $h'$ with respect to $h$ using two metrics:
\begin{itemize}
[leftmargin=*,topsep=1pt,parsep=1pt]
    \item \textbf{Precision@$k$} ($\uparrow$ is better): This checks whether the token with the highest probability according to $\delta(h')$ appears in the top-$k$ tokens according to $\delta(h)$. Namely, we sort $\delta(h)$ and assign a score of $1$ if $\arg\max(\delta(h'))$ appears in the top-$k$ tokens by $\delta(h)$, and $0$ otherwise.
    
    \item \textbf{Surprisal} ($\downarrow$ is better): We measure the minus log-probability according to $\delta(h)$, of the highest-probability token according to $\delta(h')$. Intuitively, low values mean that the model sees the substitute result as probable and hence not surprising.
\end{itemize}

\noindent We report the average Precision@$k$ and Surprisal over the validation set $\mathcal{V}$.



\subsection{Next Token Prediction}
\label{subsec:next_token_prediction_task}

Auto-regressive LMs output for every position a probability distribution over the vocabulary for the next token. Specifically, the output distribution for every position $i$ is given by $\delta (h_i^L)$, where:
\begin{equation}\label{eq:output_distribution}
    \delta (h) = \texttt{softmax} ( E^\top \cdot h) \in \mathbb{R}^{d_v}
\end{equation}
For some LMs, including \gpt{}, a layer normalization $\texttt{ln\_f}$ is applied to the final layer representation before this conversion (i.e., computing $\delta (\texttt{ln\_f}(h))$ rather than $\delta (h)$).

Recall that our goal is to measure how well this distribution can be estimated from intermediate representations, i.e. estimating $\delta (h_i^L)$ from $\delta (h_i^\ell)$ where $\ell<L$. To this end, we first run examples from the validation set through the model, while extracting for each example $s$ the hidden representation of a random position $i_s$ at every layer. Next, we apply our mappings $\matlL{}$ and the $\idlL{}$ baseline to cast the hidden representations of every layer $\ell$ to final layer substitutes (see \S\ref{sec:layer_jump}). Last, for each layer, we convert its corresponding final-layer substitute to an output distribution (Eq.~\ref{eq:output_distribution}) and compute the average Precision@$k$ (for $k=1,5,10$) and Surprisal scores with respect to the final output distribution, over the validation set.

\paragraph{Results.}
Figs.~\ref{fig:pre} and~\ref{fig:surp} show the average Precision@$k$ and Surprisal scores per layer in $48$-layered \gpt{}, respectively (the plots for the other \gpt{} models are presented in \S\ref{sec:app_scale}). Across all layers, \mat{} outperforms \id{} in terms of both scores, often by a large margin (e.g. till layer $44$ the Precision@$1$ achieved by \mat{} is bigger than that of $\id{}$ by more than $0.2$). 
This shows that linear mappings enable not just better estimation of final layer representations, but also of the predictions they induce. Moreover, the relatively high Precision@$k$ scores of \mat{} in early layers ($0.62$-$0.82$ for $k=10$, $0.52$-$0.74$ for $k=5$, and $0.28$-$0.45$ for $k=1$) suggest that early representations already encode a good estimation of the final prediction. Also, the substantially lower Surprisal scores of \mat{} compared to \id{} imply that our method allows for a more representative reading into the layer-wise prediction-formation of the model than allowed through direct projection to the vocabulary.

\begin{figure}[t]
\centering
\includegraphics[scale=0.4]{figs/pre_48.pdf}
\caption{Precision@$k$ ($k = 1,5, 10$) of $\matlL{}$ and $\idlL{}$ for next token prediction in $48$-layered \gpt{}.}
\label{fig:pre}
\end{figure}

\begin{figure}[t]
\centering
\includegraphics[scale=0.35]{figs/surp_48.pdf}
\caption{Surprisal for $\matlL$ and the baseline $\idlL{}$ ($48$-layered \gpt{} next token prediction task). A 95\% confidence interval surrounds the lines.}
\label{fig:surp}
\end{figure}

\subsection{Masked Token Prediction}
\label{subsec:BERT}

We now conduct the same experiment for the task of masked language modeling, where the model predicts a probability distribution of a masked token in the input rather than the token that follows the input. Unlike next token prediction, where the output distribution is computed from representations of varying input tokens, in masked token prediction the output is always obtained from representations of the same input token (i.e. \texttt{[MASK]}).

For this experiment, we use \bert{}, on top of which we use a pretrained masked language model head $\delta$; given a token sequence $s$, a \mask{} token inside it and its final representation $h$, $\delta (h) \in \mathbb{R}^{d_v}$
 is a probability distribution over tokens giving the model's assessment
 of the likelihood of tokens to be fitting in place of the \mask{} token in $s$.


\begin{figure}[t]
\centering
\includegraphics[scale=0.4]{figs/bertmask_pre_24.pdf}
\caption{Precision@$k$ ($k = 1,5, 10$) for  $\matlL{}$ and the baseline $\idlL{}$ ($24$-layered \bert{} masked token prediction task).}
\label{fig:bertmask_pre}
\end{figure}

\begin{figure}[t]
\centering
\includegraphics[scale=0.35]{figs/bertmask_surp_24.pdf}
\caption{Surprisal for $\matlL{}$ and the baseline $\idlL{}$ ($24$-layered \bert{} masked token prediction task). A 95\% confidence interval surrounds the lines.}
\label{fig:bertmask_surp}
\end{figure}

\paragraph{Results.}
Figs.~\ref{fig:bertmask_pre} and~\ref{fig:bertmask_surp} present the average Precision@$k$ and Surprisal scores per layer in $24$-layered \bert{} (the plots for the $12$-layered \bert{} model are presented in \S\ref{sec:app_scale}), overall showing trends similar to those observed for next token prediction in \gpt{} (\S\ref{subsec:next_token_prediction_task}). This is despite the differences between the two tasks and the considerable architectural differences between \bert{} and \gpt{}.
Notably, the superiority of \mat{} over \id{} in this setting is even more prominent; 
while \mat{}'s precision is between $0.2-0.6$ in the first ten layers (Fig.~\ref{fig:bertmask_pre}), \id{}'s precision for all values of $k$ is close to zero, again strongly indicating that our method allows for better reading into early layer hidden representations. 
More generally, \mat{} improves the Precision@$1$ of \id{} by more than $17\%$ at most layers, and unveils that a substantial amount of predictions ($>25\%$ starting from layer $3$) appear already in the very first layers.
Interestingly, the (rough) divide between the first half of layers and last half of layers for $\id{}$ in Figs.~\ref{fig:bertmask_pre},~\ref{fig:bertmask_surp} seems to align with the two-hump shape of the blue region for $\mat{}$ in Fig.~\ref{fig:bertmask_r2_scores}.

\paragraph{Analysis.}
We manually compare the predictions of our mapping $\matlL{}$ with $\idlL{}$, for a $24$-layered \bert{} model.  Concretely, we select 50 random sentences from the Leipzig dataset. Next, for each layer $\ell$, we manually analyze how many of the top-$5$ tokens according to $\matlL{}$ and $\idlL{}$ fit into context. We consider a token to fit into context if it is grammatically plausible within the sentence (see Tab.~\ref{tab:manual} for concrete examples).
In the resulting $1250$ instances (i.e. $50$ sentences $\times$ $25$ representations), we observe a substantially higher plausibility rate of $85.36\%$ for \mat{} compared to $52.8\%$ for \id{}. In fact, only in less than $4.3\%$ of the instances there are more plausible tokens among the top-$5$ tokens according to \id{} than among the top-$5$ tokens according to \mat{}, further supporting the Surprisal results above.

\begin{table*}
\footnotesize
\setlength{\belowcaptionskip}{-15pt}
\begin{tabular}{p{0.3\linewidth}ccccc}
& $\texttt{id}_{4 \rightarrow 24}$ & $\texttt{mat}_{4 \rightarrow 24}$ & $\texttt{id}_{12 \rightarrow 24}$ & $\texttt{mat}_{12 \rightarrow 24}$ & $\texttt{id}_{24 \rightarrow 24}$ \\ \midrule
\multirow{5}{=}{aldridge had shoulder surgery in \mask{}.} & fellowship & \tcbox{time} & cyclist & \tcbox{2009} & \tcbox{september} \\
& employment & \tcbox{it} & emergencies & \tcbox{2008} & \tcbox{november} \\
& agreement & her & seniors & \tcbox{2010} & \tcbox{december} \\
& \#\#ostal & them & cycling & \tcbox{2006} & \tcbox{august} \\
& \#\#com & work & \tcbox{pennsylvania} & \tcbox{2007} & \tcbox{july} \\ \midrule
\multirow{5}{=}{on your next view you will be asked to \mask{} continue reading.} & \#\#com & be & be & be & \tcbox{please} \\
& accreditation & get & undergo & \tcbox{please} & \tcbox{simply} \\ 
& $	\copyright$ & go & spartans & help & \tcbox{also} \\ 
& fellowship & \tcbox{help} & seniors & \tcbox{simply} & \tcbox{again} \\ 
& summer & have & * & say & \tcbox{immediately} \\ \bottomrule
\end{tabular}
\caption{Examples of top-$5$ predictions at layers $4$, $12$ and $24$, under the mappings $\matlL{}$ and $\idlL{}$, for a $24$-layered \bert{} model. Grammatically plausible predictions (according to a human annotator) are marked in \tcbox{blue}. Note that at layer $24$ the predictions of $\matlL{}$ and $\idlL{}$ are the same (by definition).} 
\label{tab:manual}
\end{table*}

\section{Implication to Early Exiting}
\label{sec:applications}

%The fact that it is often possible to approximate
The possibility of approximating
the final prediction already in the early layers has important implications for efficiency; applying our linear mapping instead of executing transformer blocks of quadratic time complexity, could save a substantial portion of the computation. In this section, we demonstrate this in the context of early exiting.

When 
% performing transformer model inference under 
using an early exit strategy \cite{schwartz-etal-2020-right, xin-etal-2020-deebert, schuster2022confident}, one aims at deciding dynamically at which layer to stop the computation and ``read'' the prediction from the hidden representation of that layer.
More precisely, under a confidence measure paradigm, one decides to stop the computation for a position $i$ at layer $\ell$ based on a confidence criterion, that is derived from casting the hidden representation $h_i^\ell$ as a final-layer representation and converting it to an output probability distribution. Specifically, following \citet{schuster2022confident}, a decision to exit is made if the difference between the highest and the second highest probabilities is bigger than $$ 0.9 \cdot \lambda + 0.1 \cdot {\rm exp} (-4 i / N),$$
where $N$ is the average length of the input until position $i_s$ for $s \in \mathcal{V}$, and $\lambda$ is a hyper-parameter.

\begin{figure}[t]
\setlength{\belowcaptionskip}{-10pt}
\centering
\includegraphics[width=\columnwidth]{figs/ee_gpt2bert.pdf}
\caption{Precision@$1$ with early exit and ``fixed exit'', applied to the $24$-layer \gpt{} for next token prediction (left) and the $24$-layer \bert{} for masked token prediction (right). Varying the confidence parameter $\lambda$, the $x$-coordinate is the average number of layers processed before an early exit decision is reached.}
\label{fig:ee_gpt2bert}
\end{figure}

\quash{
\begin{figure}[t]
\setlength{\belowcaptionskip}{-10pt}
\centering
\includegraphics[scale=0.35]{figs/ee_pre1_24.pdf}
\caption{Precision@$1$ for the various early exit methods, and previous ``fixed exit'' methods for comparison ($24$-layer \gpt{} next token prediction task). Varying the confidence parameter $\lambda$, the $x$-coordinate is the average number of layers processed before an early exit decision is reached.}
\label{fig:ee_pre1}
\end{figure}
}

\paragraph{Experiment.}
We assess the utility of our mapping $\matlL{}$ for early exit as a plug-and-play replacement for $\idlL{}$, through which intermediate representations are cast into final-layer representations.
We use \gpt{} for the next token prediction and \bert{} for masked token prediction (both with 24 layers).
We run each of the models over the validation set examples, while varying the confidence parameter $\lambda$ and using either $\idlL{}$ or $\matlL{}$ for casting intermediate representations.
Furthermore, we compare these early exit variants to the ``fixed exit'' strategy from \S\ref{sec:prediction}, where the computation is stopped after a pre-defined number of layers rather than relying on a dynamic decision.
We evaluate each variant in terms of both prediction's accuracy, using the Precision@$1$ metric (see \S\ref{sec:prediction}), and efficiency, measured as the average number of transformer layers processed during inference.


\paragraph{Results.}
%Figs.~\ref{fig:ee_pre1} and~\ref{fig:bertmask_ee_pre1}
Fig.~\ref{fig:ee_gpt2bert}
plots the average Precision@$1$ score against the average number of layers processed, for $24$-layer \gpt{} and $24$-layer \bert{}. For both models, under an early exit strategy our mapping \mat{} again provides a substantial improvement over \id{}.
For example, aiming at $95\%$ average precision, \mat{} saves $\sim3.3$ ($13.8$\%) layers in \gpt{} compared to only $\sim1.4$ ($5.9$\%) layers by \id{}, and $\sim4.8$ ($20$\%) layers in \bert{} versus $\sim3.5$ ($14.6$\%) layers by \id{}.
These results highlight the potential gains prominent early exit methods can obtain by using our method.
Notably, in both models and for each of the mapping methods, early exit obtains better results than fixed layer exit, as expected. 

\quash{
\begin{figure}[t]
\setlength{\belowcaptionskip}{-10pt}
\centering
\includegraphics[scale=0.35]{figs/bertmask_ee_pre1_24.pdf}
\caption{Precision@$1$ for the various early exit methods, and previous ``fixed exit'' methods for comparison ($24$-layer \bert{} masked token prediction task). Varying the confidence parameter $\lambda$, the $x$-coordinate is the average number of layers processed before an early exit decision is reached.}
\label{fig:bertmask_ee_pre1}
\end{figure}
}
\section{Linear Shortcut Across Sub-Modules}
\label{sec:submodules}

% Our experiments show that
% , despite the commonly-applied simplification by interpretability works, transformer layers do not operate in the same linear space and 
% there is a major gap in approximating future representations using an identity mapping (\S\ref{sec:layer_jump}, \S\ref{sec:prediction}).
% Here, 
In this section, we investigate whether discrepancies across layers result from specific sub-modules or are a general behaviour of all sub-modules in the network.  
This is done by extending our approach to test how well particular components in transformer blocks can be linearly approximated. 


\paragraph{Method.}

Consider \gpt{} for definiteness, then:
% we have 
$$ \texttt{b}_{\ell} = \texttt{b}_{\ell}^{\texttt{ffn}} \circ \texttt{b}_{\ell}^{\texttt{attn}}$$ 
% with
\begin{equation}\label{eq:attn} \texttt{b}^{\texttt{attn}}_{\ell} (H) = \texttt{attn}_{\ell} (\texttt{ln1}_{\ell} (H)) + H,\end{equation} 
where $\texttt{attn}_{\ell}$ is
%a multi-head self-attention
a MHSA
layer and \texttt{ln1} is a layer normalization (LN), and 
$$ \texttt{b}^{\texttt{ffn}}_{\ell} (H) = \texttt{ffn}_{\ell} (\texttt{ln2}_{\ell} (H)) + H,$$  
where $\texttt{ffn}_{\ell}$ is
%a feed-forward network
an FFN
layer and $\texttt{ln2}$ is a LN.
\quash{
Given a block $\texttt{b}_\ell$ and one of its sub-modules $\texttt{ln1}_\ell, \ \texttt{attn}_\ell, \ \texttt{ln2}_\ell$, or $\texttt{ffn}_\ell$, we fit linear regression approximating the output of the sub-module given its input and then use it in order to define mappings, as we now describe.
}
Given a block $\texttt{b}_\ell$ and one of its sub-modules $\texttt{ln1}_\ell, \ \texttt{attn}_\ell, \ \texttt{ln2}_\ell$, or $\texttt{ffn}_\ell$, we fit linear regression approximating the output of the sub-module given its input, and then use it to define mappings $\matattnl{}$, $\matlnl{}$ and $\matffl{}$.
%We provide the definition of $\matattnl{}$ below, and that of the other two in App. \ref{sec:app_submodule_skip_description}.
We provide the formal definitions of these mappings in App. \ref{sec:app_submodule_skip_description}.
\iffalse
\paragraph{$\matattnl{}$.}
%Illustrating this on $\texttt{attn}_\ell$ for definiteness,
For an input $s$, let $v^\ell_{i_s}$ be the vector at position $i_s$ in the output of $\texttt{attn}_\ell (\texttt{ln1}_\ell (H^{\ell - 1}))$. We denote by $A_\ell^{\texttt{attn}} \in \mathbb{R}^{d_h \times d_h}$ the matrix numerically minimizing 
$$ A \mapsto \sum_{s \in \mathcal{T}} || A \cdot \texttt{ln1}_\ell (h^{\ell-1}_{i_s}) - v^\ell_{i_s}||^2,$$
and define an attention sub-module replacement (Eq.~\ref{eq:attn}) by $$
\texttt{b}^{\overline{\texttt{attn}}}_\ell (h) \coloneqq A_{\ell}^{\texttt{attn}} \cdot \texttt{ln1}_\ell (h) + h. $$
We then define a mapping between two layers ${\ell \rightarrow \ell'}$ by:
$$ \matattnl{} (h) \coloneqq $$
$$ \texttt{b}^{\texttt{ffn}}_{\ell'} ( \texttt{b}^{\overline{\texttt{attn}}}_{\ell'} ( \ldots (\texttt{b}^{\texttt{ffn}}_{\ell+1} ( \texttt{b}^{\overline{\texttt{attn}}}_{\ell+1} (h)))\ldots)).$$ 
Namely, when applying each $\ell''$-th block, $\ell < \ell'' \leq \ell'$, we replace its attention sub-module $\texttt{attn}_{\ell''}$ by its linear approximation.
%In an analogous way, we consider the mappings $\matffl{}$ and $\matlnl{}$, where in the latter we perform the linear shortcut both for \texttt{ln1} and for \texttt{ln2} (see~\S\ref{sec:app_submodule_skip_description} for precise descriptions).
Importantly, unlike the original attention module, the approximation $\texttt{b}^{\overline{\texttt{attn}}}_\ell$ operates on each position independently, and therefore applying $\matattnl{}$ disables any contextualization between the layers $\ell$ and $\ell'$. Note that this is not the case for $\matffl{}$ and $\matlnl{}$, which retain the self-attention sub-modules and operate contextually.
\fi

\paragraph{Evaluation.}


We analyze the $24$-layered \gpt{}, and proceed completely analogously to \S\ref{subsec:next_token_prediction_task}, evaluating the Precision@$1$ and Surprisal metrics for the mappings $\matattnlL{}$, $\matfflL{}$ and $\matlnlL{}$.

\begin{figure}[t]
\setlength{\belowcaptionskip}{-0pt}
\centering
%\includegraphics[scale=0.2]
\includegraphics[width=\columnwidth]{figs/parts_presurp_24.pdf}
\caption{Precision@$1$ and Surprisal for the various sub-module linear mappings, and $\matlL{}$ for comparison ($24$-layer \gpt{} next token prediction task). A 95\% confidence interval surrounds the Surprisal lines.}
\label{fig:parts_presurp}
\end{figure}

\quash{
\begin{figure}[t]
\centering
\includegraphics[scale=0.4]{figs/parts_pre1_24.pdf}
\caption{Precision@$1$ for the various sub-module linear shortcut mappings, and the mapping $\matlL{}$ for comparison (\gpt{} next token prediction task).}
\label{fig:parts_pre1}
\end{figure}

\begin{figure}[t]
\centering
\includegraphics[scale=0.35]{figs/parts_surp_24.pdf}
\caption{Surprisal for the various sub-module linear shortcut mappings, and the mapping $\matlL{}$ for comparison (\gpt{} next token prediction task). A 95\% confidence interval surrounds the lines.}
\label{fig:parts_surp}
\end{figure}
}

\paragraph{Results.}
Fig.~\ref{fig:parts_presurp} shows the average Precision@$1$ and Surprisal scores per layer.
From a certain layer (\textasciitilde$7$), all sub-module mappings achieve better results than the full-block mapping $\matlL{}$. Thus, it is not just the cumulative effect of all the sub-modules in the transformer block that is amenable to linear approximation, but also individual sub-modules can be linearly approximated. 
Furthermore, the linear approximation of attention sub-modules is less harmful than that of the FFN or LN sub-modules. 
% Hypothetically, 
A possible reason is that the linear replacement of FFN or LN ``erodes'' the self-attention computation after a few layers. 
Moreover, the good performance of $\matattnlL{}$ suggests that contextualization often exhausts itself in early layers; speculatively, it is only in more delicate cases that the self-attention of late layers adds important information. Last, remark the sharp ascent of the scores for layer normalization in layers $5$-$8$, for which we do not currently see a particular reason. To conclude, we see that the possibility of linear approximation permeates
%the various
transformer components.


\section{Related Work}

Recently, there was a lot of interest in utilizing intermediate representations in transformer-based LMs, both for interpretability and for efficiency.

In the direction of interpretability, one seeks to understand the prediction construction process of the model \cite{tenney-etal-2019-bert, voita-etal-2019-bottom}.

More recent works use mechanistic interpretability and view the inference pass as a residual stream of information \cite{dar2022analyzing,geva-etal-2022-transformer}. Additionally, there are works on probing, attempting to understand what features are stored in the hidden representations \cite{adi2017finegrained, conneau-etal-2018-cram,liu-etal-2019-linguistic}. Our work is different in that it attempts to convert intermediate representations into a final-layer form, which is interpretable by design.

In the direction of efficiency, there is the thread of work on early exit, where computation is cut at a dynamically-decided earlier stage \cite{schwartz-etal-2020-right,xin-etal-2020-deebert,schuster2022confident}. Other works utilize a fixed early stage network to parallelize inference \citep{leviathan2022fast, chen2023accelerating}. However, intermediate representations are directly propagated in these works, which we show is substantially worse than our approach. Moreover, our method requires training considerably less parameters than methods such as \citet{schuster-etal-2021-consistent}, that learn a different output softmax for each intermediate layer.  

More broadly, skipping transformer layers and analyzing the linearity properties of transformer components have been discussed in prior works \cite{Zhao2021of,mickus-etal-2022-dissect,wang-etal-2022-skipbert,lamparth2023analyzing}.


\section{Conclusion and Future Work}

We present a simple and effective method for enhancing utilization of hidden representations in transformer-based LMs, that uses 
pre-fitted context-free and token-uniform linear mappings.
Through a series of experiments on different data sources, model architectures and scales, we show that our method consistently outperforms the prevalent practice of interpreting representations in the final-layer space of the model, yielding better approximations of succeeding representations and the predictions they induce, thus allowing a more faithful interpretation of the model's prediction-formation.
We demonstrate the practicality of our method for improving computation efficiency, saving a substantial amount of compute on top of prominent early exiting approaches. 
Also, by extending our method to sub-modules, 
% more specifically the attention sub-modules, 
we observe that replacing a part of the transformer inference by a non-contextual linear computation often results in a small deterioration of the prediction.
This opens new research directions for improving model efficiency,
% and parallelizability.
% including breaking the computation into several parallelizable tasks.
including breaking the computation into parallel tasks.

\section*{Limitations}

Although we see in this work that there is more linear structure to transformer inference than could be explained solely by the residual connection, we do not elucidate a reason for that. We also do not try to formulate formal criteria according to which to judge, in principle, the quality of ways of short-cutting transformer inference in-between layers. In addition, our experiments cover only English data.


%\section*{Ethics Statement}
%Scientific work published at ACL 2023 must comply with the ACL Ethics Policy.\footnote{\url{https://www.aclweb.org/portal/content/acl-code-ethics}} We encourage all authors to include an explicit ethics statement on the broader impact of the work, or other ethical considerations after the conclusion but before the references. The ethics statement will not count toward the page limit (8 pages for long, 4 pages for short papers).

\section*{Acknowledgements}

We thank Tal Schuster for constructive comments.

% Entries for the entire Anthology, followed by custom entries
\bibliography{anthology,custom}
\bibliographystyle{acl_natbib}

\appendix

\section{Descriptions of $\matattn{}$, $\matff{}$ and $\matln{}$}
\label{sec:app_submodule_skip_description}

Here we detail the definitions of the mappings $\matattnl{}$, $\matffl{}$ and $\matlnl{}$ utilized in \S\ref{sec:submodules}.

\paragraph{Description of $\matattnl{}$.}
%Illustrating this on $\texttt{attn}_\ell$ for definiteness,
For an input $s$, let $v^\ell_{i_s}$ be the vector at position $i_s$ in the output of $\texttt{attn}_\ell (\texttt{ln1}_\ell (H^{\ell - 1}))$. We denote by $A_\ell^{\texttt{attn}} \in \mathbb{R}^{d_h \times d_h}$ the matrix numerically minimizing 
$$ A \mapsto \sum_{s \in \mathcal{T}} || A \cdot \texttt{ln1}_\ell (h^{\ell-1}_{i_s}) - v^\ell_{i_s}||^2,$$
and define an attention sub-module replacement (Eq.~\ref{eq:attn}) by $$
\texttt{b}^{\overline{\texttt{attn}}}_\ell (h) \coloneqq A_{\ell}^{\texttt{attn}} \cdot \texttt{ln1}_\ell (h) + h. $$
We then define a mapping between two layers ${\ell \rightarrow \ell'}$ by:
$$ \matattnl{} (h) \coloneqq $$
$$ \texttt{b}^{\texttt{ffn}}_{\ell'} ( \texttt{b}^{\overline{\texttt{attn}}}_{\ell'} ( \ldots (\texttt{b}^{\texttt{ffn}}_{\ell+1} ( \texttt{b}^{\overline{\texttt{attn}}}_{\ell+1} (h)))\ldots)).$$ 
Namely, when applying each $\ell''$-th block, $\ell < \ell'' \leq \ell'$, we replace its attention sub-module $\texttt{attn}_{\ell''}$ by its linear approximation.
%In an analogous way, we consider the mappings $\matffl{}$ and $\matlnl{}$, where in the latter we perform the linear shortcut both for \texttt{ln1} and for \texttt{ln2} (see~\S\ref{sec:app_submodule_skip_description} for precise descriptions).
Importantly, unlike the original attention module, the approximation $\texttt{b}^{\overline{\texttt{attn}}}_\ell$ operates on each position independently, and therefore applying $\matattnl{}$ disables any contextualization between the layers $\ell$ and $\ell'$. Note that this is not the case for $\matffl{}$ and $\matlnl{}$, which retain the self-attention sub-modules and operate contextually.

\paragraph{Description of $\matffl{}$.}
Let $v^\ell_{i_s}$ be the vector at position $i_s$ in the output of $\texttt{ln2}_{\ell} (\texttt{b}_\ell^{\texttt{attn}} (H^{\ell - 1}))$, for a given input $s$. We denote by $A_\ell^{\texttt{ffn}} \in \mathbb{R}^{d_h \times d_h}$ the matrix numerically minimizing 
$$ A \mapsto \sum_{s \in \mathcal{T}} || A \cdot v^{\ell}_{i_s} - \texttt{ffn}_{\ell} (v^\ell_{i_s})||^2,$$
and define a replacement of the feed-forward sub-module $\texttt{b}_{\ell}^{\texttt{ffn}}$ by $$ \texttt{b}^{\overline{\texttt{ffn}}}_\ell (H) \coloneqq A_{\ell}^{\texttt{ffn}} \cdot \texttt{ln2}_\ell (H) + H.$$
We then define a mapping between two layers ${\ell \rightarrow \ell'}$ by:
$$ \matffl{} (H) \coloneqq $$
$$ \texttt{b}^{\overline{\texttt{ffn}}}_{\ell'} ( \texttt{b}^{\texttt{attn}}_{\ell'} ( \ldots (\texttt{b}^{\overline{\texttt{ffn}}}_{\ell+1} ( \texttt{b}^{\texttt{attn}}_{\ell+1} (H))\ldots)).$$

\paragraph{Description of $\matlnl{}$.}
Let $v^\ell_{i_s}$ be the vector at position $i_s$ in the output of $\texttt{b}^{\texttt{attn}}_{\ell} (H^{\ell - 1})$, for a given input $s$. We denote by $A_\ell^{\texttt{ln1}} \in \mathbb{R}^{d_h \times d_h}$ the matrix numerically minimizing 
$$ A \mapsto \sum_{s \in \mathcal{T}} || A \cdot h^{\ell}_{i_s} - \texttt{ln1}_{\ell} (h^\ell_{i_s})||^2$$ and we denote by $A_\ell^{\texttt{ln2}} \in \mathbb{R}^{d_h \times d_h}$ the matrix numerically minimizing $$ A \mapsto \sum_{s \in \mathcal{T}} || A \cdot v^{\ell}_{i_s} - \texttt{ln2}_{\ell} (v^\ell_{i_s})||^2.$$ We define a replacement of the block $\texttt{b}^{\texttt{attn}}_{\ell}$ by \begin{equation} \texttt{b}^{\overline{\texttt{ln1}}}_\ell (H) \coloneqq \texttt{attn}_{\ell} (A_{\ell}^{\texttt{ln1}} \cdot H) + H\end{equation} and we define a replacement of the block $\texttt{b}^{\texttt{ffn}}_{\ell}$ by \begin{equation} \texttt{b}^{\overline{\texttt{ln2}}}_\ell (H) \coloneqq \texttt{ffn}_{\ell} (A_{\ell}^{\texttt{ln2}} \cdot H) + H.\end{equation}
We then define a mapping between two layers ${\ell \rightarrow \ell'}$ by:
$$ \matlnl{} (H) \coloneqq $$
$$ \texttt{b}^{\overline{\texttt{ln2}}}_{\ell'} ( \texttt{b}^{\overline{\texttt{ln1}}}_{\ell'} ( \ldots (\texttt{b}^{\overline{\texttt{ln2}}}_{\ell+1} ( \texttt{b}^{\overline{\texttt{ln1}}}_{\ell+1} (H))\ldots)).$$


\end{document}


\end{document}
