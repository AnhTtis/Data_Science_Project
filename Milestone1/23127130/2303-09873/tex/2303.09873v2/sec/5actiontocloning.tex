

\section{From action operads to cloning systems}
\label{sec:operadtocloning}
In this section, we prove one of the implications of Theorem \ref{thm:main2} in  full generality; more concretely, we explain how to obtain a (bilateral) cloning system from an arbitrary action operad.

Let $\G=(\G_{\bullet},\pi, \{\circ_i\}_i, \id)$ be an action operad, and
%which, recall from Definition~\ref{def:actionoperad}, is a collection of groups $\{\G_n\}_{n\ge 1}$ equipped with homomorphisms $\pi_n\colon\G_n\to\Sigma_n$ and composition products  $(\circ_j\colon \G_{n+1}\times\G_{m}\to \G_{n+m})_{j,n,m}$ satisfying certain compatibility relations.
 denote by $e_n\in \G_n$ the identity element of  $G_n$. If $g\in G_n$ and $i\in \{1,\ldots,n\}$, we denote by $g(i) = \pi(g)(i)$. From the action operad structure, one can define the following maps
$$
\kappa^n_j,\iota_n,\zeta_n\colon \G_n\longrightarrow \G_{n+1}
$$
by setting
$$
\kappa^n_j(g) = g\circ_j e_2,\quad
\zeta_n(g) = e_2\circ_2 g\quad\mbox{ and }\quad
\iota_n(g) = e_2\circ_1 g.
$$
The following observation will be useful in what follows:
\begin{rem}\label{rem:properties} Note that in an action operad we always have that $e_n\circ_i e_m = e_{n+m-1}$ because $e_n\circ_i e_m = (e_n\cdot e_n) \circ_i (e_m\cdot e_m) = (e_n\circ_i e_m)\cdot (e_n\circ_i e_m)$. Moreover, the following hold
$$
    \nu_j^n(m)(g) = e_m\circ_j g\quad\mbox{ and }\quad
    \kappa_j^n(m)(g) = g\circ_j e_m,
$$
where $\kappa_j^n(m)$ and $\nu_j^n(m)$ are defined as in Definition~\ref{defn:NCCloningSystem}.
\end{rem}

\begin{prop}\label{prop:OperadtoCS} 
Let $\G=(\G_{\bullet},\pi, \{\circ_i\}_i, \id)$ be an action operad. Then, the quintuple $(\G_{\bullet}, \iota, \zeta, \kappa,\pi)$, with $\iota$, $\zeta$ and $\kappa$ as defined above, is a restricted operadic  cloning system. If $\G=(\G_{\bullet},\pi, \{\circ_i\}_i, \id)$ is a general action operad, then $(\G_{\bullet}, \iota, \zeta, \kappa,\pi)$ is an operadic cloning system.
\end{prop}
\begin{proof}
We will prove it in several steps. We will make essential use of  the fact that $\G$ is an action operad, the definition of the morphisms $\iota$, $\zeta$ and $\kappa$, and Remark~\ref{rem:properties}. First, we prove that the maps $\iota$ and $\zeta$ are homomorphisms; indeed,
\[\iota(f\cdot g) = e_2\circ_1 (f\cdot g) =
(e_2\cdot e_2)\circ_1(f\cdot g)=
(e_2\circ_{1} f)\cdot (e_2\circ_1 g) = \iota(f)\cdot \iota(g),\]
\[\zeta(f\cdot g) = e_2\circ_2 (f\cdot g) =
(e_2\cdot e_2)\circ_2(f\cdot g)=
(e_2\circ_{2} f)\cdot (e_2\circ_2 g) = \zeta(f)\cdot \zeta(g).\]
We now show that the quadruple $(\G_\bullet,\iota,\kappa,\pi)$ is a cloning system by checking that it satisfies properties (i)--(v) of Definition~\ref{defn:CloningSystem}.
\begin{itemize}
\item[(i)] $\pi_{n+1}\circ \iota_n = \lambda_n\circ\pi_n$. For every $g\in \G_n$ we have that
\begin{align*}
\pi_{n+1}(\iota_n(g)) = \pi_{n+1}(e_2\circ_1 g) = \pi_{2}(e_2)\circ_1 \pi_{n}(g) = e_2\circ_1 \pi_{n}(g) = \lambda_n(\pi_n(g)).
\end{align*}
\item[(ii+)] $\pi_{n+1}\circ \kappa^n_j = c_j^n\circ\pi_n$. For every $g\in\G_n$ and all $1\le j\le n$ we have that
\begin{align*}
\pi_{n+1}(\kappa^n_j (g)) = \pi_{n+1}(g\circ_j e_2) = \pi_{n}(g)\circ_j \pi_{2}(e_2) = \pi_{n}(g)\circ_j e_2 = c^n_j(\pi_n(g)).
\end{align*}
\item[(iii)] $\iota_{n+1}\circ \kappa_j^n = \kappa^{n+1}_j\circ\iota_n$. For every $g\in\G_n$ and all $1\le j\le n$ we have that
\begin{align*}
\iota_{n+1}\circ \kappa_j^n(g) &= \iota_{n+1}(g\circ_j e_2) = e_2\circ_1(g\circ_j e_2)  \\&=  ( e_2\circ_1 g)\circ_j e_2 =  \iota_n(g)\circ_j e_2 = \kappa^{n+1}_j(\iota_n(g)).\end{align*}
\item[(iv)] $\kappa^{n+1}_l\circ\kappa_j^n = \kappa_{j+1}^{n+1}\circ \kappa_l^n$. For every $g\in\G_n$ and all $l< j\le n$ we have that
\begin{align*}
\kappa^{n+1}_l(\kappa^n_j(g)) &= \kappa^{n+1}_l(g\circ_j e_2) = (g\circ_j e_2) \circ_l e_2 \\
&=(g\circ_l e_2) \circ_{j+1} e_2 = \kappa^n_l(g) \circ_{j+1} e_2 = \kappa^{n+1}_{j+1}(\kappa^n_l(g)).
\end{align*}
\item[(v)] $\kappa_j^n(g\cdot h) = \kappa^n_{h(j)}(g)\cdot \kappa_j^n(h)$ for every $g,h \in \G_n$. For all $j\le n$ we have that
\begin{align*}
\kappa_j^n(g\cdot h) &= (g\cdot h)\circ_j e_2 = (g\cdot h)\circ_j (e_2\cdot e_2) \\
&= (g\circ_{h(j)} e_2)\cdot (h\circ_j e_2)
= \kappa^n_{h(j)}(g)\cdot \kappa_j^n(h)
\end{align*}
\end{itemize}
Finally, we prove that $(\G_{\bullet}, \iota, \zeta, \kappa,\pi)$ is a bilateral cloning system by checking properties (i'), (iii'), (vi), (vii), (vii'), (iv+), (ix), (x) of Definition~\ref{defn:NCCloningSystem}. 
First, properties (i'), (iii') and (vii') are proved as properties (i), (iii) and (vii) simply by replacing $\circ_1$ by $\circ_2$ and $\iota$ by $\zeta$.
\begin{itemize}
\item[(vi)] $\zeta_{n+1}\circ \iota_n = \iota_{n+1}\circ\zeta_{n}$. For every $g\in \G_n$ and all $n\ge 1$ we have that
\begin{align*}
    \zeta_{n+1}(\iota_n(g)) &= e_2\circ_2(e_2\circ_1 g) = (e_2\circ_2 e_2)\circ_2 g = e_3\circ_2 g \\&= (e_2\circ_1 e_2)\circ_2 g = e_2\circ_1 (e_2\circ_2 g) = \iota_{n+1}(\zeta_{n}(g)).
\end{align*}
\item[(vii)] $\iota_{n+1}\circ\iota_n = \kappa^{n+1}_{n+1}\circ \iota_n$. For every $g\in \G_n$ and all $n\ge 1$ we have that
\begin{align*}
\iota_{n+1}(\iota_{n}(g)) &= e_2\circ_1(e_2\circ_1 g) = (e_2\circ_1 e_2)\circ_1 g = e_3\circ_1 g, \\
\kappa^{n+1}_{n+1}(\iota_n(g)) &= (e_2\circ_1 g)\circ_{n+1} e_2 = (e_2\circ_2 e_2)\circ_1 g = e_3\circ_1 g.
\end{align*}
\item[(iv+)] $\kappa^{n+1}_{j+1}\circ \kappa^n_{j} = \kappa_{j}^{n+1}\circ \kappa_j^n$. For all $g\in \G_n$ and all $1\le j\le n$ we have that 
\begin{align*}
\kappa^{n+1}_{j+1}(\kappa^n_{j}(g)) &=
(g\circ_j e_2)\circ_{j+1} e_2 = g\circ_j(e_2\circ_2 e_2) = g\circ_2 e_3  \\ &= g\circ_j (e_2\circ_1 e_2) = (g\circ_j e_2)\circ_j e_2 = \kappa_{j}^{n+1}(\kappa_j^n(g)).
\end{align*}
\item[(viii)] $\kappa_j^n(m)(g)\cdot \nu^m_j(n)(h) = \nu^m_{g(j)}(n)(h) \cdot \kappa_j^n(m)(g)$ for every $g\in\G_n$ and $h\in\G_m$. For all $m,n\geq 0$ we have that
\begin{align*}
\kappa_j^n(m)(g)\cdot \nu^m_j(n)(h) &=
(g\circ_j e_m)\cdot (e_n\circ_j h) \\
&= (g\cdot e_n)\circ_{j} (e_m\cdot h) \\
&= (e_n\cdot g)\circ_{j} (h\cdot e_m) \\
&= (e_n\circ_{g(j)} h)\cdot (g\circ_{j} e_m) \\
&= \nu^m_{g(j)}(n)(h)\cdot \kappa_{j}^n(m)(g).
\end{align*}

\item[(ix)] $\iota_n(m)(g)\cdot \zeta_m(n)(h) = \zeta_m(n)(h)\cdot \iota_n(m)(g)$ for every $g\in\G_n$ and $h\in\G_m$. For all $m,n\ge 0$ we have that
\begin{align*}
    \iota_n(m)(g)\cdot \zeta_m(n)(h) &= (e_{m+1}\circ_{1} g)\cdot (e_{n+1}\circ_{n+1} h) \\
    &= (e_{m+1}\circ_1 g)\cdot((e_2\circ_1 e_n)\circ_{n+1} h) \\
    &= (e_{m+1}\circ_1 g)\cdot((e_2\circ_2 h)\circ_1 e_n) \\
    &= (e_{m+1}\cdot (e_2\circ_2 h))\circ_1 (g\cdot e_n) \\
    &= ((e_2\circ_2 h)\cdot e_{m+1})\circ_1 (e_n\cdot g) \\
    &= ((e_2\circ_2 h)\cdot (e_{2}\circ_2 e_m))\circ_1 (e_n\cdot g) \\
    &= ((e_2\circ_2 h)\circ_1 e_n)\cdot ((e_2\circ_2e_m)\circ_1 g) \\
    &= ((e_2\circ_1 e_n)\circ_{n+1} h)\cdot ((e_{m+1})\circ_1 g) \\
    &= \zeta_m(n)(h)\cdot \iota_n(m)(g).
\end{align*}
\end{itemize}
Therefore, $(\G_{\bullet}, \iota, \zeta, \kappa,\pi)$ is a restricted operadic cloning system as we wanted to show.

If $(\G_{\bullet}, \iota, \zeta, \kappa,\pi,\id)$ were a general action operad, then the proof of Condition (ii+) would restrict instead to a proof of Condition $(ii)$.
\end{proof}


