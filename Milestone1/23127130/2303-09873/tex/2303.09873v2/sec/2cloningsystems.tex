

\section{Cloning systems}\label{section:cloning_systems}

In this section we offer a brief introduction to Witzel--Zaremsky's  cloning systems \cite{WZ}, and define their {\em bilateral} counterparts; we refer the interested reader to \cite{WZ,Zaremsky} for a detailed account on cloning systems. 

We start with a specific example, which appears as Example 2.9 in \cite{WZ}, and that will serve to establish some notation for the sequel. In what follows, $\Sigma_n$ stands for the symmetric group on $n$ elements. 

\begin{examp}[Cloning system for symmetric groups] Let $\Sigma_{\bullet}=\{\Sigma_n\}_{n\ge 1}$ be the family of symmetric groups. For every $n\ge 1$, let $\lambda_{n}:\Sigma_n \to \Sigma_{n+1}$ be the injective homomorphism obtained by fixing the last element, that is,
$$
\lambda_n(\sigma)(i)=\sigma(i) \mbox{ for $1\le i\le n$ and } \lambda_n(\sigma)(n+1)=n+1,
$$
for every $\sigma\in\Sigma_n$. For every $n\ge 1$ and $1\le j\le n$, let $c^n_j: \Sigma_n\to \Sigma_{n+1}$ be the injective map given by, thinking about permutations pictorially as strand diagrams, ``repeating'' the $j$-th strand; that is, 
$$
c_j^n(\sigma)(i)=
\left\{
\begin{array}{ll}
\sigma(i) & \mbox{ if $i\le j$ and $\sigma(i)\le\sigma(j)$,} \\
\sigma(i)+1 & \mbox{if $i<j$ and $\sigma(i)>\sigma(j)$,}\\
\sigma(i-1) & \mbox{if $i>j+1$ and $\sigma(i-1)<\sigma(j)$,}\\
\sigma(i-1)+1 & \mbox{if $i\ge j+1$ and $\sigma(i-1)\ge \sigma(j)$,}
\end{array}
\right.
$$
for every $\sigma\in\Sigma_n$.
As shown in \cite[Example 2.9]{WZ} the families of morphisms $\lambda$ and $c$ interact with each other, and satisfy certain obvious compatibility properties, detailed in \cite[Proposition 2.6]{WZ}. The \emph{cloning system} for the family of symmetric groups is the triple $(\Sigma_\bullet, \lambda, c)$, subject to these compatibility conditions. 
\end{examp}

The maps $c_j$ above are called {\em cloning maps}, for obvious reasons; note that they are not group homomorphisms. The notion of a cloning system is a generalization of the above example to arbitrary families of groups. 


\begin{defn}\label{defn:CloningSystem}A \emph{cloning system} is a quadruple $(\G_{\bullet},\iota,\kappa,\pi)$, where
\begin{itemize}
    \item $\G_{\bullet}=\{\G_n\}_{n\geq 1}$ is a family of groups,
    \item $\iota=\{\iota_n\colon\G_n\to\G_{n+1}\}_{n\ge 1}$ is a family of injective homomorphisms,
    \item $\kappa=\{\kappa^n_{j}\colon\G_{n}\to\G_{n+1}\}_{n\ge 1,\,1\leq j\leq n}$ is a family of maps, called \emph{cloning maps}, and
    \item $\pi= \{\pi_n \colon \G_n\to \Sigma_n\}_{n\ge 1}$ is a  homomorphism,
\end{itemize}
 subject to the following compatibility conditions:
\begin{itemize}
    \item[(i)]\label{it:i} $\pi_{n+1} \circ \iota_n=\lambda_n \circ \pi_n$, for all $n\ge 1$;  
    \item[(ii)]\label{it:ii} $(\pi_{n+1}(\kappa^n_j(g)))(i) = (c^n_j(\pi_n(g))(i)$, for all $n\ge 1$ and all $1\le j\le n$, all $i\neq j,j+1$, all $g\in G_n$;
    \item[(iii)]\label{it:iii} $\iota_{n+1}\circ\kappa^n_j= \kappa_j^{n+1}\circ \iota_n$, for all $n\ge 1$ and all $1\le j\le n$;
    \item[(iv)]\label{it:iv} $\kappa^{n+1}_{j+1} \circ \kappa_l^n = \kappa_l^{n+1} \circ \kappa_{j}^{n}$, for all $n$ and all $l< j\le n$;
    \item[(v)]\label{it:v} $\kappa^n_j(g\cdot h)=\kappa^n_{\pi_n(h)(j)}(g)\cdot\kappa^n_j(h)$, for all $n$, all $g,h \in \G_n$ and all $j\le n$.
\end{itemize}
\end{defn}
%\jg{He quitado la condición de que las $\kappa$'s sean inyectivas después de hablarlo con Federico. Aunque en el artículo del user's guide lo pide, en el Witzel-Zelemsky no lo hace y para la comparación con action operads y demás, es mejor que no lo sean.}
\begin{rem} It is important to note some differences between Definition \ref{defn:CloningSystem} and the definition of cloning system of~\cite{WZ,Zaremsky}. First, we use a functional convention for composition of maps, that is, the composition of two functions $f\colon X\to Y$ and $g\colon Y\to Z$ is denoted by $g\circ f$, defined as $(g\circ f)(x)=g(f(x))$. Second, 
the original definition of cloning system in \cite{Zaremsky} requires injective homomorphisms $\iota_{n,m}\colon \G_{n}\to \G_{m}$ for every $m> n\geq 1$; however, our Definition \ref{defn:CloningSystem} implies that it suffices to consider injective maps $\iota_{n,n+1}=\iota_n$ for all $n\geq 1$.  
\end{rem}

Witzel and Zaremsky observe that cloning systems often satisfy the following strengthed version of these axioms. 
\begin{itemize}
   	\item[(iv+)] $\kappa^{n+1}_{j+1}\circ \kappa^n_{j} = \kappa_{j}^{n+1}\circ \kappa_j^n$, for all $1\le j\le n$;
    \item[(vii)]\label{it:iii'} $\iota_{n+1}\circ \iota_n = \kappa_{n+1}^{n+1}\circ \iota_n$, for all $n\ge 1$;    
\end{itemize}
\begin{rem}
    Condition (vii) was part of the original definition of cloning system: In~\cite[Definition 2.18]{WZ} the cloning maps $\kappa$ are required to be a \emph{family of cloning maps}, which must satisfy two conditions presented at the beginning of page 18 in that paper. These two conditions correspond to Conditions (iii) and (vii) in this article.

In \cite[page 17]{WZ} (see also \cite{Bri07}) the Hedge monoid ${\mathcal H}$ is introduced, together with a surjective map ${\mathcal F}\to {\mathcal H}$ from the monoid of forests to the monoid of hegdes. The colimit of the groups in a cloning system comes with an action of ${\mathcal F}$. Condition~(iv+) in this article is equivalent to require that that action factors through the hegde monoid (which holds for most examples; see Observation 2.11 and paragraph before Observation 2.19 in \cite{WZ}).

\end{rem}


We now introduce the notion of a \emph{bilateral} cloning system. In a nutshell, in the same way that the maps $\iota$ of Definition \ref{defn:CloningSystem} informally correspond to ``adding elements on the right'', a bilateral cloning system comes equipped with  a family of ``dual'' maps $\zeta$ that correspond to ``adding elements on the left''. 

We now proceed to formalize this idea. For the family of symmetric groups, we denote by $\rho_n: \Sigma_n \to \Sigma_{n+1}$ the injective homomorphism that fixes the first element, that is, 
$$
\rho_n(\sigma)(i)=\sigma(i-1)+1 \mbox{ for $2\le i\le n+1$ and } \rho_n(\sigma)(1)=1,
$$
for every $\sigma\in\Sigma_n$.


\begin{defn}\label{defn:NCCloningSystem} A \emph{bilateral cloning system} is a quintuple $(\G_{\bullet},\iota, \zeta, \kappa,\pi)$, where $(\G_{\bullet},\iota,\kappa, \pi)$ is a cloning system satisfying conditions (iv+) and (vii), and $\zeta=\{\zeta_n\colon \G_n\to\G_{1+n}\}_{n\ge 1}$ is an additional family of injective homomomorphisms satisfying the following conditions:
\begin{itemize}
    \item[(i')] $\pi_{n+1} \circ \zeta_n=\rho_{n}\circ \pi_n $, for all $n\ge 1$;  
    \item[(iii')] \label{it:iiib} $\zeta_n\circ \kappa_j^n = \kappa_{j+1}^{n+1}\circ\zeta_{n}$, for all $n\ge 1$ and all $1\le j\le n$;
    \item[(vi)] $\zeta_{n+1}\circ \iota_n = \iota_{n+1}\circ\zeta_{n}$, for all $n\ge 1$;
    \item[(vii')]\label{it:iiib'} $\zeta_{n+1}\circ \zeta_n = \kappa_{1}^{n+1}\circ \zeta_n$, for all $n\ge 1$;
 \end{itemize}
 A bilateral cloning system is called \emph{restricted} if it additionally satisfies the following condition:
 \begin{itemize}
        \item[(ii+)] $\pi_{n+1} \circ \kappa^n_j = c^n_j \circ \pi_n$, for all $n\ge 1$ and all $1\le j\le n$;
 \end{itemize}
\end{defn}

\begin{defn} A bilateral cloning system is called \emph{operadic} if the following additional conditions are satisfied:
\begin{itemize}
	\item[(viii)]$\kappa_i^n(m)(g)\cdot \nu_i^m(n)(h) = \nu^m_{\pi(g)(i)}(n)(h)\cdot \kappa^n_i(m)(g)$, for all $m,n\geq 0$ and all $g\in \G_n$ and $h\in\G_m$;
	\item[(ix)] $\iota_n(m)(g)\cdot \zeta_m(n)(h)=\zeta_m(n)(h)\cdot \iota_n(m)(g)$, for all $m,n\geq 0$ and all $g\in\G_n$ and $h\in\G_m$.
\end{itemize}
\end{defn}


The morphisms $\iota_n(m)$, $\zeta_m(n)$, $\kappa_j^n(m)$ and $\nu_j^m(n)$ appearing in conditions (viii) and (ix) are the maps
$$
\iota_n(r)\colon\G_{n}\longrightarrow\G_{n+r},\quad \kappa^n_j(m)\colon \G_{n}\longrightarrow\G_{n+m}, \quad
\zeta_n(l)\colon\G_{n}\longrightarrow\G_{l+n}, 
$$
 and 
$$
\nu^n_j(m)\colon\G_{n}\to\G_{n+m-1}
$$
defined by  $\iota_n(r)=\iota_{n+r-1}\circ\cdots\circ\iota_{n}$ for all $n,r\ge 1$; $\kappa_j^n(m)=\kappa_j^{n+m-2}\circ\cdots\circ \kappa_j^{n}$ for all $n,m\ge 1$ and all $1\le j\le n$; $\zeta_n(l)=\zeta_{n+l-1}\circ\cdots\circ\zeta_{n}$ for all $n,l\ge 1$; and  $\nu_j^n(m)=\zeta_{m+n-j}(j-1)\circ\iota_n(m-j)=\iota_{n+j-1}(m-j)\circ\zeta_n(j-1)$ for all $j,m,n\ge 1$. Note that the morphisms $\zeta_n(l)$ and $\nu_j^n(m)$ only make sense for bilateral cloning systems.


\begin{rem}
Although in Definition~\ref{defn:CloningSystem} and Definition~\ref{defn:NCCloningSystem} we require  the maps  $\iota$ and $\zeta$ to be injective, this  is not essential for the comparison of bilateral cloning system and action operads, as we will see in the next section. 
\end{rem}


In order to get a grip on the intuition behind the definitions above, we next describe perhaps the primordial example of a (bilateral) cloning system, namely that of braid groups; we refer the reader to \cite{WZ} for details.   

\begin{examp}[Braid groups]\label{examp:cloning}
    
 Let $\Br_n$ denote the braid group on $n$ strands. For every $n\ge 1$ there is a canonical surjective group homomorphism $\pi_n\colon\Br_n\to\Sigma_n$ by sending each braid to its underlying permutation. We also have inclusion maps $\iota_n\colon \Br_n\to\Br_{n+1}$ corresponding to ``adding one strand on the right'', and cloning maps
$$
\kappa_{j}^n\colon \Br_n\longrightarrow \Br_{n+1}
$$
given by duplicating the $j$-th strand into two parallel strands; see Figure \ref{fig:Kappa Braid} for an example, and \cite[Example~3.3]{Zaremsky} for details.
\begin{figure}[htp]
    \centering
    \includegraphics[width=6cm]{figures/kappa_Br.pdf}
    \caption{Cloning map $\kappa_2^4$ on $\Br_4$.}
    \label{fig:Kappa Braid}
    \end{figure}

Together, the three families of maps defined above endow the collection of all braid groups $\Br_{\bullet}=\{\Br_n\}_{n\ge 1}$ with the cloning system structure $\Br=(\Br_{\bullet},\iota,\kappa,\pi)$, see \cite{WZ} for a proof. 

Moreover, in analogy with the maps $\iota_n$, we can also define inclusion maps $\zeta_n\colon \Br_n\to\Br_{n+1}$ that informally correspond to ``adding one strand on the left''. Equipped with these maps, one readily checks that $\Br=(\Br_{\bullet},\iota,\zeta,\kappa,\pi)$ is a bilateral cloning system. For illustrative purposes, Figures \ref{fig:AxiomVI} to \ref{fig:AxiomXI} depict particular instances of some of the conditions of the bilateral cloning system structure of $\Br_{\bullet}$.
\end{examp}  
 \begin{figure}
 \centering
 \includegraphics[width=7cm]{figures/NonChiral_additional_axioms_vi.pdf}
    \caption{Condition (vii), $\kappa_3^3\circ \iota=\iota\circ\iota$.}
    \label{fig:AxiomVI}
    \end{figure}
 \begin{figure}
    \centering
    \includegraphics[width=7cm]{figures/NonChiral_additional_axioms_vii.pdf}
    \caption{Condition (vi), $\iota\circ \zeta=\zeta\circ \iota$.}
    \label{fig:AxiomVIII}
    \end{figure}
\begin{figure}
    \centering
    \includegraphics[width=10cm]{figures/NonChiral_additional_axioms_x.pdf}
    \caption{Condition (viii).}
    \label{fig:AxiomX}
    \end{figure}
 \begin{figure}
    \centering
    \includegraphics[width=10cm]{figures/NonChiral_additional_axioms_xi.pdf}
    \caption{Condition (ix).}
    \label{fig:AxiomXI}
    \end{figure}

Other examples of bilateral cloning systems are the {\em mock symmetric groups} and the {\em loop braid groups} (also known as {\em symmetric automorphisms of free groups}); see \cite{WZ}; as well as the {\em signed symmetric groups} and the {\em twisted braid groups} \cite{Zaremsky}. 
The latter two bilateral cloning systems are not restricted.
\medskip

Next, we discuss two examples of (bilateral) cloning systems from \cite{WZ} and \cite{Zaremsky} which, as we will see, are not operadic.

\begin{examp}[Direct powers]\label{examp:DirectPowers}
Let $G$ be a group and denote by $G^n$ the $n$-fold direct product of $G$ with itself. Write  $\iota_n: G^n \to G^{n+1}$ for the map that adds the identity (in $G$) as last entry, let $\pi_n: G^n \to \Sigma_n$ the trivial homomorphism, and consider the map $\kappa^n_j:G^n \to G^{n+1}$ that duplicates the $j$-th entry. Then, the quadruple $(\{G^n\}_{n\ge 1}, \iota, \pi, \kappa)$ is  a cloning system on the set of direct powers of $G$. 

Despite the fact that there is an obvious map $\zeta_n: G^n \to G^{n+1}$ that adds the identity (in $G$) as first entry, one may check that the maps $\kappa$ and $\zeta$ do not satisfy condition (viii) of the definition of an operadic bilateral cloning system.
\end{examp}


\begin{examp}[Upper triangular matrices]\label{examp:UpperTriangularMatrices}
Let $U_n$ denote the group of invertible $n\times n$ upper triangular matrices with real coefficients. Consider the obvious inclusion map $\iota_n: U_n \to U_{n+1}$ given by adding a 1 as the lowermost element on the diagonal. 

There are cloning maps $\kappa_j^n: U_n \to U_{n+1}$ that informally correspond to a certain duplication of the $j$-th column that preserves the upper triangular structure of the matrix, and which becomes apparent just by giving the following particular example; see \cite{Zaremsky} for details: 

\[
\kappa^3_2\begin{pmatrix}
1 & 2 & 3 \\ 0 & 4 & 5 \\ 0 & 0 & 6
\end{pmatrix}  = 
\begin{pmatrix}
1 & 2 & 2 & 3 \\ 0 & 4 & 0 & 0 \\ 0 & 0 & 4 & 5 \\ 0 & 0 & 0 & 6
\end{pmatrix}
\]
Setting $\pi_n: U_n \to \Sigma_n$ to be the trivial homomorphism, the set $U_\bullet = \{U_n\}_{n \ge 1}$ acquires the cloning system structure $U = (U_\bullet, \iota, \kappa, \pi)$. 

Observe that one could define another inclusion map $\zeta_n: U_n \to U_{n+1}$ by adding a 1 as the uppermost element of the diagonal; however, as in the previous example, the interaction of the maps $\zeta$ and $\kappa$ does not satisfy condition (viii) of the definition of a bilateral cloning system.
\end{examp}
