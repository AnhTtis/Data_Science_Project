\section{Introduction}
The umbrella term {\em Thompson-like groups} makes reference to a vast family of groups that are in some way reminiscent of one of the classical groups $F$, $T$ and $V$ of R. Thompson \cite{CFP}. Apart from these three groups, prominent examples of Thompson-like groups include Stein's groups of PL homeomorphisms \cite{Ste}; Guba--Sapir's  diagram groups \cite{GuSa}; Brin's higher-dimensional Thompson groups \cite{Bri04}; Belk--Forrest's rearrangement groups of fractals \cite{BeFo}; the braided Thompson group of Brin \cite{Bri07} and Dehornoy \cite{Deh}; Wahl's  ribbon Thompson group \cite{Wahl}; the  {\em asymptotic mapping class groups} of surfaces and higher-dimensional manifolds \cite{FK04, FK08,FK09,AF20,ABF+21,GLU20}, etc. 




As may be appreciated from the above list of examples, Thompson-like groups arise in a variety of different ways.
With this motivation, the independent results of Witzel--Zaremsky \cite{WZ} and Thumann \cite{Th} offer unified frameworks for constructing Thompson-like groups. Witzel-Zaremsky achieve this in terms of {\em cloning systems}, which provide a recipe for ``twisting'' a direct limit of groups into a Thompson-like group. In turn, Thumann \cite{Th} uses the theory of {\em operads} in order to construct Thompson-like groups, which in this setting arise as the fundamental group of a certain category associated to an operad (cf.\ \cite{FioreLeinster}). 

The purpose of this paper is to establish a dictionary between cloning systems and a certain type of operads called {\em action operads} \cite{Zhang,Corner-Gurski,Yoshida,Yau}, which we consider without constants (see Section \ref{section:action_operads} for definitions). Our first result is the following: 

\begin{thm}
Every action operad gives rise to a cloning system.
\end{thm}



In fact, we precisely determine to which extent a converse to the above theorem holds; more concretely, we will prove that every action operad comes from a cloning system that admits a certain extra structure, and  that we call {\em restricted operadic cloning system}, see Section \ref{section:cloning_systems}. We stress that many of the known cloning systems are, in fact, operadic and bilateral; see Section \ref{section:cloning_systems} for examples and non-examples. In this language, our main results may be summarized as follows: 

\begin{thm} \label{thm:main2}
There is an explicit bijective correspondence between action operads and restricted operadic cloning systems.
\end{thm}


As will become apparent, our methods actually yield the equivalence of the categories of restricted operadic cloning systems
and action operads, respectively.

There is a wider class of cloning systems that yield operads: the \emph{operadic} cloning systems. In fact, operads that arise from these cloning systems comply with all the roles of action operads (i.e., operads that support the equivariance of other operads). These will be defined in Section \ref{section:action_operads} with the name of \emph{general action operads}.


\medskip

\noindent {\bf Further results.} Action operads have been related to crossed simplicial groups in \cite{Zhang} and to crossed interval groups in \cite{Yoshida}. In Section \ref{section:crossed}, we review this relationship and we extend it to cloning systems. Finally, in Section \ref{section:props} we introduce a third construction using PROs ({\em product categories}) which also yields operadic cloning systems. The following diagram summarizes the relation between all of these results: 
$$
\begin{tikzcd}[ampersand replacement=\&]
   \begin{matrix}
       \text{action operads}
   \end{matrix}\ar[r,leftrightarrow]\ar[d] \& 
   \begin{matrix}
       \text{restricted operadic}\\
       \text{cloning systems}
   \end{matrix}\ar[d] \ar[r,leftrightarrow] \& 
    \begin{matrix}
        \text{restricted} \\
       \text{cloning PROs}
   \end{matrix}
   \\
   \begin{matrix}
       \text{general}\\
       \text{action operads}
   \end{matrix}\ar[r,leftrightarrow] \ar[d]
   \& 
   \begin{matrix}
       \text{operadic} \\ \text{cloning systems}
   \end{matrix} \ar[d] \& \text{cloning PROs} \ar[l, leftrightarrow] \ar[u, leftarrow] \ar[d]
   \\
     \begin{matrix}
       \text{inert crossed}\\
       \text{demi-interval groups}
   \end{matrix}\ar[r,leftrightarrow]
   \&
   \begin{matrix}
    \text{bilateral}\\ \text{cloning systems}
   \end{matrix}\ar[r]
   \& \text{cloning systems}
\end{tikzcd}
$$

\medskip

\noindent{\bf Future work.} This is the first of two papers devoted to the relation between operads and Thompson groups. In a forthcoming paper we will show that, if $A$ is an action operad and $C$ is its associated cloning system, then the fundamental group of a certain $A$\nobreakdash-operad is isomorphic to the Thompson group of the cloning system $C$.


\medskip

\noindent{\bf Plan of the paper.} In Section \ref{section:cloning_systems} we give a brief introduction to cloning systems, and give some examples. Section \ref{section:action_operads} offers an abridged overview of action operads. Section \ref{sec:braids} is devoted to the proof of Theorem \ref{thm:main2} in the special case of braid groups. These ideas are then generalized in Sections \ref{sec:operadtocloning} and \ref{sect:FromCStoActionOperads}. Finally, in Sections \ref{section:crossed} and \ref{section:props}, we will give further interpretations of our results in terms of crossed interval groups and PROs, respectively. 




\medskip

\noindent{\bf Acknowledgements.} This project started with some informal conversations at the {\em IX Encuentro de J\'ovenes Top\'ologos}, held in Seville in 2021. We are grateful to the organization for their hospitality and support. The second author thanks An\'ibal Medina for a very enlightning conversation about Joyal duality.  