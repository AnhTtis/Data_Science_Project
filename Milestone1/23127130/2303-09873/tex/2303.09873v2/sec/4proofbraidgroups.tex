

\section{Proof of Theorem \ref{thm:main2} for braid groups}
\label{sec:braids}
Before we embark on the proof of our main theorem in its full generality, we explain the situation with the special case of braid groups. As will become apparent, the general argument is an abstraction of the ideas introduced here, and will be treated in the next two sections.

Recall from Example~\ref{examp:cloning} the (bilateral) cloning system $\Br=(\Br_{\bullet},\iota,\zeta,\kappa,\pi)$ for braid groups, where $\pi$ is the canonical projection onto the set of symmetric groups, the maps $\iota$ and $\zeta$ add one strand on the right or the left, respectively, and $\kappa$ are the cloning maps given by duplicating a strand.

$\Br=\big(\Br_{\bullet},\pi\big)$ be the collection of all braid groups equipped with the canonical projection homomorphisms $\pi_n\colon\Br_{n}\to \Sigma_n$. 

Note that the collection $\Br_{\bullet}$ can be endowed with an action operad structure with respect to the ``substitution maps'' $\circ_{j}\colon \Br_{n+1}\times\Br_m\to \Br_{n+m}$, which correspond to replacing the $j$-th strand of the first braid by the second braid. Figure  \ref{fig:CloningToOperad} contains a depiction of this operation; for a more detailed discussion, see \cite[Definition 1.6 and Example 1.12(2)]{Corner-Gurski} for details.


We now explain how to obtain one structure from the other, illustrating the main ideas of the procedure with several pictures.

\subsection{From the action operad to the (bilateral) cloning system.}
The morphisms $\iota$ and the cloning maps $\kappa$ for the cloning system are obtained by using the identity $e_2\in \Br_2$ and the composition products $\circ_i$, as explained in Figures~\ref{fig:OperadToIota} and~\ref{fig:OperadToKappa}, respectively. Thus one obtains the cloning system $\Br$ described in Example~\ref{examp:cloning}. 
\begin{figure}
    \centering
    \includegraphics[width=10cm]{figures/Operad_to_iota.pdf}
    \caption{Constructing the map $\iota$. Insert the given braid in the first strand of $e_2$.}
    \label{fig:OperadToIota}
    \end{figure} 
        
    \begin{figure}
    \centering
    \includegraphics[width=10cm]{figures/Operad_to_kappa.pdf}
    \caption{Constructing the map $\kappa_{j}$. Replace the $j$th strand of the given braid by $e_2$.}
    \label{fig:OperadToKappa}
    \end{figure}
In order to get a bilateral cloning system, the maps $\zeta$ are defined similarly to the maps $\iota$, but replacing the second strand of $e_2$ instead of the first one, that is, using $\circ_2$ instead of $\circ_1$; see Figure~\ref{fig:OperadToZeta}.
\begin{figure}
    \centering
    \includegraphics[width=10cm]{figures/Operad_to_zeta.pdf}
    \caption{Constructing the map $\zeta$. Insert the given braid in the second strand of $e_2$.}
    \label{fig:OperadToZeta}
    \end{figure}    

\subsection{From the (operadic) bilateral cloning system to the action operad} We now explain how to use the bilateral cloning system structure on $\Br$ in order to build an action operad structure on $\Br_{\bullet}$. To see how the composition product maps $\circ_j$ are obtained, we proceed as follows. First, we replicate $m$ times the $j$th strand of the first braid using the cloning maps, where $m$ is the number of strands of the second braid. Then we add, by using $\iota$ and $\zeta$ strands to the right and left, respectively, of the second braid, so that it has the same number of strands as the cloned braid. Finally, we just multiply the two braids obtained this way. Figure~\ref{fig:CloningToOperad} shows an example of this construction. 
 \begin{figure}
    \centering
    \includegraphics[width=10cm]{figures/Cloning_to_Operad.pdf}
    \caption{Constructing the substitution maps $\circ_j$ from the cloning system data.}
    \label{fig:CloningToOperad}
    \end{figure}

One can check that the maps $\circ_j$ built above satisfy all the properties required for  endowing $\Br_{\bullet}$ with an operad structure in sets. For instance, Figure~\ref{fig:AssociativityI} and Figure~\ref{fig:AssociativityII} show the verification of the associativity axiom. 
 \begin{figure}
    \centering
    \includegraphics[width=7cm]{figures/Associativity_of_operad_I.pdf}
    \caption{First half of the associativity axiom. The colors represent the application of $\iota$, $\zeta$ and $\kappa$ in each step.}
    \label{fig:AssociativityI}
    \end{figure}
    \begin{figure}
    \centering
    \includegraphics[width=7cm]{figures/Associativity_of_operad_II.pdf}
    \caption{Second half of the associativity axiom.
    }
    \label{fig:AssociativityII}
    \end{figure}

Finally, a simple computation, depicted in Figure~\ref{fig:Compatibility}, shows that the maps $\circ_j$ are compatible with the underlying group structure of the braid groups, thus proving that $(\Br_{\bullet},\pi, \{\circ_i\}_i, e_1)$ is indeed an action operad. 
 \begin{figure}
    \centering
    \includegraphics[width=7cm]{figures/Compatibility_of_products_partial_products.pdf}
    \caption{Compatibility axiom of the action operad.}
    \label{fig:Compatibility}
    \end{figure}
