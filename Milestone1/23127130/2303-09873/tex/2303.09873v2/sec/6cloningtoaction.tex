

\section{From cloning systems to action operads}\label{sect:FromCStoActionOperads}
We now explain how to construct (general) action operads from a subclass of bilateral cloning systems. Let $\G=(\G_{\bullet},\iota,\zeta,\kappa,\pi)$ be a bilateral cloning system. The following identities can be derived from the identites in the definition of a bilateral cloning system (recall from Definition~\ref{defn:NCCloningSystem} the construction of the maps $\kappa_i^n(m)$ and $\nu_i^n(m)$):

\[
\kappa_i^{n+m-1}(l)\circ \kappa_j^n(m)=
\begin{cases}
    \kappa_{j}^n(m+l-1) & \text{if $j\leq i < j+m$}, \\
    \kappa_{j+l-1}^{n+l-1}(m)\circ\kappa_i^n(l) & \text{if $i<j$}.
\end{cases}
\]
\[
    \kappa_j^{n+m-1}(l)\circ \nu^n_i(m) = \begin{cases}
		\nu^{n+l-1}_i(m)\circ \kappa_{j-i+1}^n(l) & \text{if $i\leq j< i+n$},\\
		\nu^{n}_{i+l-1}(m+l-1) & \text{if $j< i$}, \\
		\nu^{n}_i(m+l-1) & \text{if $j\geq i+n$},
		\end{cases}
  \]
\begin{align*}
    \nu_j^{n+m-1}(l)\circ \nu_i^n(m)& =\nu_{j+i-1}^n(m+l-1),
\end{align*}
\begin{align*}
    \kappa_j^n(m)(f\cdot g) &= \kappa_{g(j)}^n(m)(f)\cdot \kappa_j^n(m)(g) &\mbox{if } 1\leq j\leq n,
\end{align*}
\begin{multline*}
        \nu_j^l(n+m-1)(g)\cdot \nu_{i+l-1}^m(n+l-1)(h) \\ =  \nu_{i+l-1}^m(n+l-1)(h)\cdot \nu_j^l(n+m-1)(g)\qquad \mbox{if $j<i$}
\end{multline*}
Note that $\nu^n_i(m)(g)$ acts trivially on all $j<i$ and all $j\geq i+m$ and that $\kappa^n_i(m)(e_n) = e_{n+m-1}$, because
\[\kappa^n_i(e_n) = \kappa^n_i(e_n\cdot e_n) = \kappa^n_i(e_n)\cdot \kappa^n_i(e_n)\]
and therefore $\kappa^n_i(e_n) = e_{n+1}$. 

In order to define an action operad structure on $\G_{\bullet}$ and since we already have a map $\pi\colon \G_{\bullet}\to \Sigma_{\bullet}$, it is enough to define the partial composition products $\{\circ_i\}_i$, which we do it as follows. The map
\[
\circ_i\colon \G_n\times \G_m\longrightarrow \G_{n+m-1}
\]
is the following composition
\[\xymatrix{
\G_{n}\times \G_{m}\ar[rr]^-{\kappa_{i}^n(m)\times \nu_i^m(n)} &&
\G_{n+m-1}\times\G_{n+m-1} \ar[r]^-{\cdot} &
\G_{n+m-1},
}\]
that is, $f\circ_i g = \kappa_i^n(m)(f)\cdot \nu^m_i(n)(g)$, for every $f\in\G_n$ and $g\in \G_m$. We also set $\id = e_1\in \G_1$.

\begin{prop}\label{prop:CloningtoOper}
    Let $\G=(\G_{\bullet}, \iota, \zeta, \kappa,\pi)$ be an restricted operadic cloning system. Then the quadruple $(\G_{\bullet},\pi, \{\circ_i\}_i, \id)$, with $\{\circ_i\}_i$ as defined above is an action operad. If the restricted condition is dropped, then we obtain a general action operad.
\end{prop}
\begin{proof} First, we check that $e_1$ is indeed a unit for the partial composition product. For every $f\in\G_n$ and every $g\in \G_m$ we have that
\begin{align*}
    f\circ_i e_1 &= \kappa^n_i(1)(f)\cdot \nu_i^1(n)(e_1) = f\cdot e_n = f, \\
    e_1\circ_1 g &= \kappa^1_1(m)(e_1) \cdot \nu_1^m(1)(g) = e_m\cdot g = g.
\end{align*}
Second, we show the associativity for the partial composition product. Let $f\in \G_n$, $g\in \G_m$ and $h\in \G_l$. Then
    \begin{align}\notag
        &(f\circ_i g)\circ_j h \\ \notag &=(\kappa_i^n(m)(f)\cdot \nu^m_i(n)(g))\circ_j h \\ \notag
&= \kappa^{n+m-1}_j(l)\Big(\kappa_i^n(m)(f)\cdot \nu_i^m(n)(g)\Big)\cdot \nu_j^l(n+m-1)(h) \\ \label{eq:309}
&=
\Big(\kappa_{g^*(j)}^{n+m-1}(l)(\kappa_i^n(m)(f))\Big)\cdot \Big(\kappa_{j}^{n+m-1}(l)(\nu_i^m(n)(g))\Big)\cdot \Big(\nu_j^l(n+m-1)(h)\Big), 
\end{align}
where we denote $g^*=\nu_i^m(n)(g)$. Depending on the indexes $i$ and $j$, we have to consider the following three cases.

\bigskip
\noindent {\it Case 1.} Suppose first that $j<i$. In this case $g^*(j) = j$ and we have that 
$$
\kappa_j^{n+m-1}(l)\circ \kappa_i^n(m) = \kappa_{i+l-1}^{n+l-1}(m)\circ \kappa_j^n(l)\quad\mbox{and}
$$
$$
\kappa_j^{n+m-1}(l)\circ \nu_i^m(n) = \nu_{i+l-1}^m(n+l-1).
$$
Then \eqref{eq:309} becomes
\[\Big(\kappa_{i+l-1}^{n+l-1}(m)(\kappa_j^n(l)(f))\Big)\cdot \Big(\nu_{i+l-1}^m(n+l-1)(g)\Big)\cdot \Big(\nu_j^l(n+m-1)(h)\Big). \]
On the other hand, we have that
    \begin{align}\notag 
        &(f\circ_j h)\circ_{i+l-1} g  \\ \notag
&=(\kappa_j^n(l)(f)\cdot \nu_j^l(n)(h))\circ_{i+l-1} g \\ \notag      
&= \kappa_{i+l-1}^{n+l-1}(m)\Big(\kappa_j^n(l)(f)\cdot \nu_j^l(n)(h)\Big)\cdot \nu_{i+l-1}^m(n+l-1)(g) \\ \label{eq:310}
&=
\Big(\kappa_{h^*(i+l-1)}^{n+l-1}(m)(\kappa_j^n(l)(f))\Big)\cdot \Big(\kappa_{i+l-1}^{n+l-1}(m)(\nu_j^l(n)(h))\Big)\cdot \Big(\nu_{i+l-1}^m(n+l-1)(g)\Big),
\end{align}
where $h^*=\nu_j^l(n)(h)$. Now, \eqref{eq:309} and \eqref{eq:310} are equal since 
$$
h^*(i+l-1) = \nu_j^l(n)(h)(i+l-1) = i+l-1\quad\mbox{and}
$$
$$
\kappa_{i+l-1}^{n+l-1}(m)(\nu_j^l(n)(h)) = \nu_j^l(n+m-1)(h).
$$
The last equality holds because $j<i$ and therefore $i+l-1\ge j+l$.

\bigskip
\noindent{\it Case 2.} Suppose now that $j\geq i+m$. Again $g^*(j)=j$ and we have that
$$
\kappa_j^{n+m-1}(l)\circ \kappa_i^n(m) = \kappa_{i}^{n+l-1}(m)\circ \kappa_{j-m+1}^n(l)\quad\mbox{and}
$$
$$
\kappa_j^{n+m-1}(l)\circ \nu_i^m(n) = \nu_i^m(n+l-1).
$$
The fist equality holds because $j\ge i+m$ and therefore $i<j-m+1$.
Then \eqref{eq:309} becomes
\[\Big(\kappa_{i}^{n+l-1}(m)(\kappa_{j-m+1}^n(l)(f))\Big)\cdot \Big(\nu_i^m(n+l-1)(g)\Big)\cdot \Big(\nu_j^l(n+m-1)(h)\Big). \]
On the other hand,
    \begin{align}\notag 
        &(f\circ_{j-m+1} h)\circ_{i} g \\ \notag
&= (\kappa_{j-m+1}^n(l)(f)\cdot \nu_{j-m+1}^l(n)(h))\circ_i g \\ \notag  &= \kappa_{i}^{n+l-1}(m)\Big(\kappa_{j-m+1}^n(l)(f)\cdot \nu_{j-m+1}^l(n)(h)\Big)\cdot \nu_i^m(n+l-1)(g) \\ \label{eq:311}
&=
\Big(\kappa_{h^*(i)}^{n+l-1}(m)(\kappa_{j-m+1}^n(l)(f))\Big)\!\cdot \!\Big(\kappa_{i}^{n+l-1}(m)(\nu_{j-m+1}^l(n)(h))\Big)\!\cdot \!\Big(\nu_i^m(n+l-1)(g)\Big),
\end{align}
where $h^*=\nu_{j-m+1}^l(n)(h)$. But \eqref{eq:309} and \eqref{eq:311} are equal since $h^*(i) = i$ and
$$
\kappa_i^{n+l-1}(m)(\nu_{j-m+1}^l(n)(h)) = \nu_j^l(n+m-1)(h).
$$
The last equality holds because $j\ge i+m$ and therefore $i<j-m+1$.

\bigskip
\noindent {\it Case 3.} Suppose that $i\leq j<i+m$. In this case, $i\leq g^*(j)<i+m$ and we have that
$$
\kappa_{g^*(j)}^{n+m-1}(l)\circ \kappa_i^n(m) = \kappa_i^n(m+l-1)\quad\mbox{and}
$$
$$
\kappa_j^{n+m-1}(l)\circ \nu_i^m(n) = \nu_i^{m+l-1}(n)\circ \kappa_{j-i+1}^m(l).
$$
Then \eqref{eq:309} becomes
\[\Big(\kappa_i^n(m+l-1)(f)\Big)\cdot \Big(\nu_i^{m+l-1}(n)( \kappa_{j-i+1}^m(l)(g))\Big)\cdot \Big(\nu_j^l(n+m-1)(h)\Big).
\]
On the other hand, 
\begin{align} \notag
&f\circ_i(g\circ_{j-i+1} h) \\ \notag
&=f\circ_i (\kappa_{j-i+1}^m(l)(g)\cdot \nu_{j-i+1}^l(m)(h)) \\ \notag
&= \kappa_i^n(m+l-1)(f)\cdot \nu_i^{m+l-1}(n)\Big(\kappa_{j-i+1}^m(l)(g)\cdot \nu_{j-i+1}^l(m)(h)\Big)\\
&=\Big(\kappa_i^n(m+l-1)(f)\Big)\cdot \Big(\nu_i^{m+l-1}(n)(\kappa_{j-i+1}^m(l)(g))\Big)\cdot \Big(\nu_i^{m+l-1}(n)(\nu_{j-i+1}^l(m)(h))\Big).\label{eq:312}
\end{align}
Again \eqref{eq:309} and \eqref{eq:312} are equal since 
$$
\nu_i^{m+l-1}(n)(\nu_{j-i+1}^l(m)(h))=\nu_j^l(n+m-1)(h).
$$
Regarding the second condition, it is clear that from axiom $(ii)$ we obtain a levelwise homomorphism $\pi$ to the family of symmetric groups as in the definition of general action operad. From axiom $(ii+)$ we obtain an operad map $\pi$ as in the definition of action operad.

Finally, in order to prove the product rule for the action operad, we will use condition (viii) in the definition of a bilateral cloning system. Let $f,f'\in G_n$ and $g,g'\in G_m$. Then we have that
\begin{align*}
(f\circ_{f'(i)} g)\cdot (f'\circ_i g')
&= \Big(\kappa_{f'(i)}^n(m)(f)\cdot \nu_{f'(i)}^m(n)(g)\Big)\cdot \Big(\kappa_i^n(m)(f')\cdot \nu_i^m(n)(g')\Big) \\
&= \Big(\kappa_{f'(i)}^n(m)(f)\cdot  \kappa_i^n(m)(f')\Big)\cdot \Big(\nu_{f'(i)}^m(n)(g)\cdot \nu_i^m(n)(g')\Big) \\
&= \Big(\kappa_{i}^n(m)(f\cdot f')\Big)\cdot \Big(\nu_{i}^m(n)(g\cdot g')\Big) \\
&= (f\cdot f')\circ_i (g\cdot g').\qedhere
\end{align*}
\end{proof}

\begin{thm}\label{thm:BijectionActOpdsAndOperadicCS}
There is an explicit bijective correspondence between action operads and restricted operadic cloning systems. There is an explicit bijective correspondence between general action operads and operadic cloning systems.      
\end{thm}
\begin{proof} The constructions given in Proposition~\ref{prop:OperadtoCS} and Proposition~\ref{prop:CloningtoOper} are inverses of each other. Let $(\G_\bullet,\iota,\zeta,\kappa,\pi)$ be a restricted operadic cloning system, and let $(\hat{\G}_\bullet,\hat{\iota},\hat{\zeta},\hat{\kappa},\hat{\pi})$ be the restricted operadic cloning system that arises from the action operad associated to it. It is clear that $\hat{\G}_\bullet = \G_\bullet$ and that $\hat{\pi} = \pi$. For the maps $\iota$ we have that
\begin{align*}
\hat{\iota}_n(g) &= e_2\circ_1 g = \kappa_1^2(n)(e_2)\cdot \nu^n_1(2)(g) = e_{n+1}\cdot \iota_n(1)(g) = \iota_n(g),
\end{align*}
and in the same way, we obtain that $\hat{\zeta}_n(g)=\zeta_n(g)$. For the cloning maps, we have that
\begin{align*}
\hat{\kappa}_j^n(g) &= g\circ_j e_2 = \kappa_j^n(2)(g)\cdot \nu_j^2(n)(e_2) = \kappa_j^n(2)(g)\cdot e_{n+1} = \kappa_j^n(g).
\end{align*}

Conversely, let $(\hat{\G}_\bullet,\hat{\pi},\{\hat{\circ}_i\}_i,\hat{\id})$ be the action operad associated to the restricted operadic cloning system obtained from an action operad $(\G_\bullet,\pi,\{\circ_i\}_i,\id)$. It is clear that $\hat{\G}_\bullet = \G_\bullet$, $\hat{\pi}=\pi$ and $\hat{\id}=\id$. Regarding the partial composition products, let $f\in \G_n$ and $g\in\G_m$. Then we have that
\begin{align*}
f\operatorname*{\hat{\circ}}\nolimits_i g &= \kappa_i^n(m)(f)\cdot \nu^m_i(n)(g) \\ 
&= (f\circ_i e_m)\cdot (e_n\circ_i g) \\
&= (f\cdot e_n)\circ_i (e_m \cdot g) \\
&= f\circ_i g,
\end{align*}
where we have used Remark~\ref{rem:properties} for the second equality and the product rule of the action operad for the third. 

The second statement is proven analogously.
\end{proof}


\begin{rem} As mentioned in Section \ref{section:cloning_systems}, most of the known examples of cloning systems are bilateral and even operadic. Consequently, for all those examples, we have identified a general action operad structure on them. However, a natural question arises: is it possible to interpret the remaining Example~\ref{examp:DirectPowers} and Example~\ref{examp:UpperTriangularMatrices}, in operadic terms? The answer is yes. Both examples have in common that their structural map $\pi$ is trivial and that the only condition that fails for them to be operadic bilateral cloning systems is axiom (viii). 

Repeating the construction of Proposition \ref{prop:CloningtoOper} for these examples, and noticing that axiom (viii) is only applied to show the compatibility of the group multiplication with the $\circ_i$ products, one gets the structure of a non-symmetric operad \emph{in sets} (see Definition \ref{def:actionoperad}). The reason why they are not operadic is that an operadic cloning system with $\pi$ trivial has an associated non-symmetric operad \emph{in groups} via Theorem \ref{thm:BijectionActOpdsAndOperadicCS}.
\end{rem}
