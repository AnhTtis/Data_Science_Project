

\section{Cloning systems and PROs}\label{section:props}

In this section, we present another viewpoint for cloning systems in terms of PROs (product categories) that will be used in a forthcoming piece of work about the construction of Thompson groups. We begin by recalling the definition of PRO.
\begin{defn} A \emph{PRO} $\Ofrak$ is a quadruple $(\Ofrak,\odot,\otimes,\id)$ where:
\begin{itemize} 
    \item $\Ofrak$ is a collection of sets $\Big(\Ofrak\brbinom{n}{m}\Big)_{n,m\geq 1}$,
    \item $\odot$ is an associative product (\emph{vertical product}) with two-sided unit $\id$
    $$
    \odot\colon \Ofrak\brbinom{n}{t}\times \Ofrak\brbinom{s}{n}\longrightarrow \Ofrak\brbinom{s}{t},\quad  \id_{n}\in\Ofrak\brbinom{n}{n},
    $$
    \item $\otimes$ is an associative product (\emph{horizontal product})
    $$
    \otimes\colon \Ofrak\brbinom{n_1}{m_1}\times \Ofrak\brbinom{n_2}{m_2}\longrightarrow \Ofrak\brbinom{n_1+n_2}{m_1+m_2}.
    $$
\end{itemize}
Moreover, the vertical and horizontal product are required to satisfy the \emph{interchange law}
$$
(f\otimes g)\odot (p\otimes q)=(f\odot p)\otimes (g\odot q),
$$
whenever it makes sense, and  $\id_{n}\otimes\id_{m}=\id_{n+m}$ for all $n,m\geq 1$.
\end{defn}

More abstractly, a \emph{PRO} $\mathfrak{O}$ is a strict (non unital) monoidal category equipped with a strict monoidal functor $(\mathbb{N}_{\geq 1},+)\to (\mathfrak{O},\otimes)$ which is an isomorphism on objects. We do not consider monoidal units, that is, units for $\otimes$, because they will create nullary operations that we are avoiding.



\begin{examp}[Symmetric groups]
The family of symmetric groups $\{\Sigma_n\}_{n\ge 1}$ yields a very simple PRO by setting
$$
\Sigma\brbinom{n}{m}= \begin{cases}
    \Sigma_n & \mbox{if $n=m$}, \\
    \emptyset & \mbox{if $n\neq m$}.
\end{cases}
$$
The vertical product $\odot$ is just the group structure of the symmetric groups and the horizontal product $\otimes$ is the block product of permutations (see Figure ~\ref{fig:HorizontalProductSigma}).
\begin{figure}[h]
    \centering
    \includegraphics[width=6cm]{figures/HorizontalProductSigma.pdf}
    \caption{Block product of $(1\;4)(2\;3)$ and $(1\;2\;3)$.}
    \label{fig:HorizontalProductSigma}
    \end{figure}
We denote by $\Sigma$ the PRO of symmetric groups. Additionally, we consider that $\Sigma$ is equipped with \emph{cloning maps} for any $n$-tuple of positive integers $(m_1,\dots,m_n)$, 
$
c(m_1,\dots,m_n)\colon \Sigma\brbinom{n}{n}\to \Sigma\brbinom{m_1+\dots+m_n}{m_1+\dots+m_n},
$ 
obtained by replacing the $i$-th strand by $m_i$ strands for all $1\leq i\leq n$.
\end{examp}

\begin{examp}[Signed symmetric groups]
The family of signed symmetric groups $\{\Sigma_{n}^{\pm}\}_{n\ge 1}$ provides a slightly more complicated PRO. First, define 
$$
\Sigma^{\pm}\brbinom{n}{m}= \begin{cases}
    \Sigma_{n}^{\pm} & \mbox{if $n=m$}, \\
    \emptyset & \mbox{if $n\neq m$}.
\end{cases}
$$
The vertical product $\odot$ is just the group structure of the signed symmetric groups and the horizontal product $\otimes$ is the block product of signed permutations, which is defined analogously to the block product of ordinary permutations.
We denote by $\Sigma^{\pm}$ the PRO of signed symmetric groups. Additionally, we consider that $\Sigma^{\pm}$ is equipped with \emph{cloning maps} for any $n$-tuple of positive integers $(m_1,\dots,m_n)$, 
$
c(m_1,\dots,m_n)\colon \Sigma^{\pm}\brbinom{n}{n}\to \Sigma^{\pm}\brbinom{m_1+\dots+m_n}{m_1+\dots+m_n},
$ 
obtained by ``plumbing'' $m_i$ strands in the $i$-th component for $1\leq i\leq n$ subject to the rules:
\begin{itemize}
\item if $g(+i)$ is positive, the new $m_i$ strands are plumbed with the identity permutation;
\item if $g(+i)$ is negative, the new $m_i$ strands are plumbed with the permutation 
$$
\left(\begin{matrix}
1 & 2 & \cdots & m_i-1 & m_i\\
m_i & m_i-1 & \cdots & 2 & 1
\end{matrix}\right).
$$
\end{itemize}
\begin{figure}[h]
    \centering
    \includegraphics{figures/Kappa_Signed.pdf}
    \caption{Cloning map on $\Sigma^{\pm}$.}
    \label{fig:KappaSigned}
    \end{figure}
\end{examp}


We will also need the concept of a \emph{morphism of PROs} $f\colon \Ofrak\to\Ofrak'$, which consists of maps $f_{n,m}\colon\Ofrak\brbinom{n}{m}\to \Ofrak'\brbinom{n}{m}$ for every $n,m\ge 1$ that preserve the PRO structure.

Next, we define two special types of PROs, by adding some extra structure, which will turn out to be equivalent to operadic cloning systems and their restricted counterparts. 

\begin{defn}
A \textit{cloning PRO} consists of the following data:
\begin{itemize}
    \item a PRO $(\G,\odot,\otimes,\id)$ such that the vertical product on $\G\brbinom{n}{n}$ admits inverses, that is, $\G\brbinom{n}{n}$ is a group for very $n$, and $\G\brbinom{n}{m}=\emptyset$ if $n\neq m$, 
    \item a morphism of PROs $\pi\colon\G\to\Sigma^{\pm}$,
    \item cloning maps 
    $\kappa(\underline{m})\colon \G\brbinom{n}{n}\rightarrow\G\brbinom{m_1+\dots+m_n}{m_1+\dots+m_n}$ for every $n\geq 1$ and every $n$-tuple $\underline{m}=(m_1,\dots,m_n)$ with $m_i\geq 1$, satisfying $\kappa(1,\stackrel{(n)}{\dots}, 1)=\id_{\G_n}$, and  the associativity condition
    $$
    \kappa(\underline{r}^1\sqcup\cdots\sqcup\underline{r}^{n})\circ\kappa(\underline{m})=\kappa\Big(\sum_{j_1}r^1_{j_1},\dots,\sum_{j_n}r^n_{j_n}\Big) 
    ,
    $$
    where $\underline{r}^i=(r^i_1,\dots,r^i_{m_i})$ and  
    $
    \underline{r}^1\sqcup\cdots\sqcup\underline{r}^n=(r^1_1,\dots,r^1_{m_1},\dots,r^n_1,\dots,r^n_{m_n}).
    $
\end{itemize}
Moreover, the PRO structure is compatible with $\pi$ and $\kappa$ as stated by the following conditions:
\begin{enumerate}
    \item $\pi$ commutes with  $\kappa$: $\pi\circ\kappa(\underline{m})=c(\underline{m})\circ\pi$.
     \item Identities and $\kappa$: $\kappa(\underline{m})(\id_{n})=\id_{m_1+\cdots+m_n}$.
    \item Horizontal composition and $\kappa$:
    $$\kappa(\underline{m})(f)\otimes \kappa(\underline{n})(g)= \kappa(\underline{m}\sqcup\underline{n})(f\otimes g).$$
    \item Vertical composition and $\kappa$:
    $$\kappa(\underline{m})(h\odot g)=\kappa(\underline{m}_{\pi(g)})(h)\odot \kappa(\underline{m})(g),$$ where  $\underline{m}_{\pi(g)}=(m_{\pi(g)(1)},\dots,m_{\pi(g)(n)})$. % provided that $\underline{m}=(m_1,\dots,m_n)$ \ja{quitamos lo del provided?}.
    \item[($\star$)] Twisted interchange law: $$\kappa(\underline{m})(f)\odot (g_{1}\otimes \cdots\otimes g_{n})=(g_{\pi(f)(1)}\otimes \cdots \otimes g_{\pi(f)(n)})\odot \kappa(\underline{m})(f).$$
\end{enumerate}
\end{defn}
\begin{defn}
    A cloning PRO $(\G,\odot,\otimes, \id,\pi,\kappa)$ is said to be \emph{restricted} if the morphism of PROs $\pi\colon \G\to \Sigma^{\pm}$ factors through the PRO of symmetric groups $\Sigma$.
\end{defn}


After these definitions, we prove that every (restricted) operadic cloning system gives rise to a (restricted) cloning PRO. More concretely: 

\begin{prop}\label{prop:NCCStoPRO} A (restricted) operadic cloning system $(\G_{\bullet},\iota,\zeta,\kappa,\pi)$ defines a (restricted) cloning PRO  $(\G,\odot,\otimes,\id,\pi,\kappa)$ in a functorial way.
\end{prop}
\begin{proof}
Set $\G\brbinom{n}{n}=\G_n$, the vertical product $\odot$ to be the group multiplication and $\id_n=e_n$. Define the cloning maps
$$
\kappa(\underline{m})=\kappa(m_1,\dots,m_n)=\kappa_1(m_1)\circ\cdots \circ \kappa_{n}(m_n),
$$
and the horizontal product $\otimes$
$$
\begin{tikzcd}
\otimes\colon \G\brbinom{n}{n}\times \G\brbinom{m}{m}\ar[rr,"\iota(m)\times \zeta(n)"] && \G\brbinom{n+m}{n+m}\times\G\brbinom{n+m}{n+m}\ar[r, "\cdot"] & \G\brbinom{n+m}{n+m} 
\end{tikzcd}.
$$

We now check that $(\G,\odot,\otimes,\id,\pi,\kappa)$ is a (restricted) cloning PRO by verifying all the axioms. Let $f\in\G_n$, $g\in\G_m$ and $h\in \G_r$.
\begin{itemize}
    \item Associativity of $\otimes$: 
\begin{align*}
(f\otimes g)\otimes h &=  \iota(r)\big(\iota(m)(f)\cdot \zeta(n)(g)\big)\cdot \zeta(n+m)(h)\\
&= \iota(r+m)(f)\cdot \iota(r)(\zeta(n)(g))\cdot \zeta(n+m)(h)\\
&= f\otimes (g\otimes h),
\end{align*}
by condition (vi) of bilateral cloning systems and since $\iota$ and $\zeta$ are homomorphisms.
\item Interchange law for $\odot$ and $\otimes$: 
\begin{align*}
(f\otimes g)\odot (p\otimes q) &=  \Big(\iota(m)(f)\cdot \zeta(n)(g)\Big)\cdot\Big(\iota(m)(p)\cdot\zeta(n)(q)\Big)\\
&= \iota(m)(f)\cdot \iota(m)(p)\cdot \zeta(n)(g)\cdot\zeta(n)(q)\\
&= (f\odot p)\otimes (g\odot q),
\end{align*}
by condition (ix) and since $\iota$ and $\zeta$ are homomorphisms.

\item $\pi\colon \G\to \Sigma^{\pm}$ (resp.\ $\pi\colon \G\to \Sigma$) is a morphism of PROs, since $\pi$ commutes with $\iota, \kappa,\zeta$ and it consists of group homomorphisms.

\item Associativity conditions for $\kappa$ hold by conditions (iv) and (iv+) (axioms for compositions of $\kappa$'s for operadic/bilateral cloning systems).

\item Commutation of $\kappa$ and $\pi$ holds by the analogous axiom for (restricted) operadic cloning systems, i.e.\ (ii') (resp.\ (ii+)).

\item $\kappa$ acting on identities. Since
$$
\kappa_j(r)(e_m)=\kappa_j(r)(e_m)\cdot\kappa_j(r)(e_m),
$$
it follows that $\kappa_j(r)(e_m)=e_{m+r-1}$.

\item Relation between $\kappa$ and $\odot$. By iterating condition (v) we get 
\begin{multline*}
\kappa(\underline{m})(f\odot p) =  \kappa_1(m_1)\circ\cdots\circ \kappa_n(m_n)(f\cdot p)\\
=  \Big(\kappa_{p(1)}(m_1)\circ\cdots\circ\kappa_{p(n)}(m_n)(f)\Big)\cdot \Big(\kappa_1(m_1)\circ\cdots\circ\kappa_n(m_n)(p)\Big)\\
= \kappa(\underline{m}_{p})(f)\odot \kappa(\underline{m})(p).
\end{multline*}

\item Relation between $\kappa$ and $\otimes$. Let us analyze what happens for $\kappa_j(r)$ instead of a general $\kappa(\underline{m}^1\sqcup\underline{m}^2)$, because the general case will follow by combining these simple cases by conditions (iv) and (iv+). 

\begin{itemize}
    \item[\textbf{Case 1:}] ($1\leq j\leq n$). 
    \begin{align*}
 \kappa_j(r)(f\otimes g) &=  \kappa_j(r)\big(\iota(m)(f)\cdot\zeta(n)(g)\big)\\
 &=  \kappa_{\zeta(n)(g)(j)}(r)(\iota(m)(f))\cdot \kappa_j(r)(\zeta(n)(g))\\
 &=\kappa_j(r)(\iota(m)(f))\cdot \kappa_j(r)(\zeta(n)(g))\\
 &=\iota(m)\big(\kappa_j(r)(f)\big)\cdot \zeta(r+n)(g)\\
 &= \kappa_j(r)(f)\otimes g.
 \end{align*}
We have applied conditions (v), (iii), (vii') and that $\zeta(n)(g)$ is constant over $\{1,\dots, n\}$.
    \item[\textbf{Case 2:}] ($n<j\leq m+n$).
    \begin{align*}
 \kappa_j(r)(f\otimes g) 
 &=  \kappa_{n+g(j)}(r)(\iota(m)(f))\cdot \kappa_j(r)(\zeta(n)(g))\\
 &=\iota(m+r)(f)\cdot \zeta(n)(\kappa_j(r)(g))\\
 &= f\otimes \kappa_j(r)(g).
 \end{align*}
We have applied conditions (v), (iii'), (vii) and that $\zeta(n)(g)(j)$ in this case is $n+g(j)$ which is strictly bigger than $n$ (recall that we are always assuming that elements in $\G_m$ or $\Sigma^{\pm}_m$ act on $\{1,\dots, m\}$ through their canonical map into $\Sigma_m$).
\end{itemize}

\item Twisted interchange law. Let us just check the case $f\in \G_2$ ($n=2$) and $g_i\in \G_{m_i}$, since the general case follows from the same ideas.
 \begin{align*}
\kappa(m_1,m_2)(f)\odot (g_1\otimes g_2) 
 &= (\kappa_1(m_1)\circ\kappa_2(m_2))(f)\cdot \Big(\iota(m_2)(g_1)\cdot\zeta(m_1)(g_2)\Big) \\
 &\overset{(a)}{=} \Big(\kappa_1(m_1)\big(\kappa_2(m_2)(f)\big)\cdot \nu_{1}(m_2)(g_1)\Big)\cdot \zeta(m_1)(g_2)\\
 &\overset{(b)}{=} \Big(\nu_{f(1)}(m_2)(g_1)\cdot \kappa_{1}(m_1)\big(\kappa_{2}(m_2)(f)\big)\Big)\cdot \zeta(m_1)(g_2)\\
  &\overset{(c)}{=} \nu_{f(1)}(m_2)(g_1)\cdot \kappa_{1}(m_1)\Big(\kappa_{2}(m_2)(f)\cdot \nu_2(1)(g_2)\Big)\\
  &\overset{(d)}{=} \nu_{f(1)}(m_2)(g_1)\cdot \kappa_{1}(m_1)\Big(\nu_{f(2)}(1)(g_2)\cdot\kappa_{2}(m_2)(f)\Big)\\
   &\overset{(e)}{=} \nu_{f(1)}(m_2)(g_1)\cdot \nu_{f(2)}(m_1)(g_2)\cdot\kappa(m_1,m_2)(f)\\
 &\overset{\phantom{(e)}}{=} (g_{f(1)}\otimes g_{f(2)})\odot \kappa(m_1,m_2)(f).
 \end{align*}
In the above equalities, we have applied: $(a)$ definition of $\nu$; $(b)$ condition (viii); $(c)$ condition (vii') and definition of $\nu$; $(d)$ condition (viii); and $(e)$ conditions (vii) and (iv+). \hfill\qedhere
\end{itemize}
\end{proof}

Next, we prove that (restricted) operadic cloning systems come from (restricted) cloning PROs: 

\begin{prop}\label{prop:PROsToNCCS} A (restricted) cloning PRO $(\G,\odot,\otimes,\id,\pi,\kappa)$ determines a (restricted) operadic cloning system $(\G_{\bullet},\iota,\zeta,\kappa,\pi)$ in a functorial way.
\end{prop}
\begin{proof}
Set $\G_n=\G\brbinom{n}{n}$, the product $\cdot=\odot$ and $e_n=\id_n$. Define
$$
\kappa_j(m)=\kappa(1,\dots,1,m^{(j)},1,\dots,1), \quad 
\iota(f)=f\otimes \id \quad \text{and} \quad \zeta(f)=\id\otimes f.
$$

The only non straightforward axioms to check are the following:
\begin{itemize}
    \item $\iota$ is a homomorphism of groups. We have that,
    $$
    \iota(m)(f\cdot g)= (f\odot g)\otimes \id_m= (f\otimes \id_m)\odot(g\otimes \id_m)= \iota(m)(f)\cdot\iota(m)(g)
    $$
    by the interchange law and since $\id_m$ is idempotent. A similar argument proves that $\zeta$ is also a homomorphism of groups.
    \item Axioms (iii) and (vii). We have that
    $$
    \kappa_j(r)\iota(m)(f)=\left\lbrace 
    \begin{array}{lcl}
    \kappa_j(r)(f)\otimes \id_m=\iota(m)\kappa_j(r)(f) && \text{if }1\leq j\leq n,\\[4mm]
   f\otimes \id_{m+r-1}=\iota(m+r-1)(f) && \text{otherwise},
    \end{array}
    \right.
    $$
    where $f\in\G\brbinom{n}{n}$. The corresponding conditions (iii') and (vii') for $\zeta$ can be deduced in the same way.
    \item Axiom (vi). We have that 
    $$
    \zeta(m)\iota(n)(f)=\id_m\otimes f\otimes \id_n= \iota(n)\zeta(m)(f)
    $$
    by the associativity of the horizontal product $\otimes$.
    \item Axiom (viii). We have that
    \begin{align*}
 \kappa_j(m)(f)\cdot \nu_j(n)(g)&= \kappa(1,\dots,m^{(j)},\dots, 1)(f)\odot(\id_{j-1}\otimes\, g\otimes \id_{n-j+1})\\
 &= (\id_{f(j)-1}\otimes\, g\otimes \id_{n-f(j)+1})\odot \kappa(1,\dots,m^{(j)},\dots, 1)(f)\\ 
 &=\nu_{f(j)}(n)(g)\cdot \kappa_j(m)(f),
 \end{align*}
 by the twisted interchange law and the identity $\id_{r}\otimes\id_{s}=\id_{r+s}$.
    \item Axiom (ix). We have that
     \begin{align*}
 \iota(m)(f)\cdot \zeta(n)(g)&= (f\otimes \id_m)\odot(\id_n\otimes\, g)\\
 &= (f\odot \id_n)\otimes (\id_m\odot\, g)\\ 
 &= (\id_n\odot\, f)\otimes (g\odot \id_m)\\
  &= (\id_n\otimes\, g)\odot (f\otimes \id_m)\\
 &=\zeta(n)(g)\cdot \iota(m)(f),
 \end{align*}
    by the interchange law and since $\id_r$ is an identity for $(\G\brbinom{r}{r},\odot)$. \hfill\qedhere
\end{itemize}
\end{proof}

Finally, we have the following theorem, which ties together the results in this section: 


\begin{thm} The functorial constructions of Propositions \ref{prop:NCCStoPRO} and \ref{prop:PROsToNCCS} induce isomorphisms of categories
$$
\begin{Bmatrix}
\text{(restricted) operadic}\\
\text{cloning systems}
\end{Bmatrix} \cong 
\begin{Bmatrix}
\text{(restricted)}\\
\text{cloning PROs}
\end{Bmatrix}.
$$
\end{thm}
\begin{proof}
Let us first check that
$$
\begin{Bmatrix}
\text{(restricted) operadic}\\
\text{cloning systems}
\end{Bmatrix} \to 
\begin{Bmatrix}
\text{(restricted)}\\
\text{cloning PROs}
\end{Bmatrix}\to 
\begin{Bmatrix}
\text{(restricted) operadic}\\
\text{cloning systems}
\end{Bmatrix}.
$$
is the identity. Let $(\G_{\bullet},\iota,\zeta,\kappa,\pi)$ be a (restricted) operadic cloning system, and let $(\hat{\G}_{\bullet},\hat{\iota},\hat{\zeta},\hat{\kappa},\hat{\pi})$ be the image of the previous (restricted) operadic cloning system under the composition of functors. By definition $\hat{\G}=\G$ as groups and $\hat{\pi}=\pi$. For $\hat{\iota}$, we have that
\begin{align*}
\hat{\iota}(g) &= g\otimes \id = \iota(1)(g)\cdot \zeta(n)(e_1)=\iota(g)\cdot e_{n+1}=\iota(g).
\end{align*}
The maps $\zeta$ behave similarly, and
\begin{align*}
\hat{\kappa}_j(g) &= \kappa(1,\dots,2^{(j)},\dots,1)(g)=\big(\kappa_1(1)\circ\cdots\circ\kappa_j(2)\circ\cdots\circ\kappa_n(1)\big)(g)=\kappa_j(g).
\end{align*}

Conversely, let $(\G,\odot,\otimes,\id,\pi,\kappa)$ be a (restricted) cloning PRO and let the tuple  $(\hat{\G},\hat{\odot},\hat{\otimes},\hat{\id},\hat{\pi},\hat{\kappa})$ denote the image of it under the other composition of functors. It is clear that $\hat{\G} = \G$, $\hat{\odot}=\odot$, $\hat{\id}=\id$, $\hat{\pi}=\pi$ and $\hat{\kappa}=\kappa$. Finally, regarding the horizontal product we have that
\begin{align*}
f\,\hat{\otimes}\, g &= \iota(m)(f)\cdot \zeta(n)(g)= (f\otimes \id_m)\odot (\id_n\otimes g)\\ 
&= (f\odot \id_n)\otimes(\id_m \odot g) = f\otimes g. \qedhere
\end{align*}
\end{proof}
