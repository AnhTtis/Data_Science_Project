

\section{Cloning systems as crossed groups}\label{section:crossed}

Action operads have a close relationship with crossed simplicial groups \cite{Zhang, Yoshida}. In this section we review this relationship, and explain the relationship with cloning systems. As we will see, cloning systems satisfying (iv+) can be interpreted as ``crossed interval groups'', while cloning systems satisfying (iv+) without maps $\iota$ are the same as crossed simplicial groups.


Let $(\G_{\bullet},\pi,\iota,\kappa)$ be a cloning system. If we forget all the structure except the cloning maps, we can interpret the pair $(\G_\bullet,\kappa)$ as a diagram
$$
\begin{tikzcd}[ampersand replacement=\&]
\G_1
\ar[r, "\kappa_1" description] \& \G_2
\ar[r, "\kappa_1" description, shift left=2]\ar[r, "\kappa_2" description, shift right = 2] \& \G_3 
\ar[r, "\kappa_1" description, shift left=4]\ar[r, "\kappa_2" description]\ar[r, "\kappa_3" description, shift right=4] \& \;\cdots 
\end{tikzcd}
$$
which resembles the diagram representing a simplicial set $X$, but considering only the degeneracy operations
$$
\begin{tikzcd}[ampersand replacement=\&]
X[0] \ar[r, "s_0" description] \& X[1] \ar[r, "s_0" description, shift left=2]\ar[r, "s_1"' description, shift right=2] \& X[2] \ar[r, "s_0" description, shift left=3]\ar[r, "s_1" description] \ar[r, "s_2"' description, shift right=3] \& X[3] \ar[r, "s_0"description, shift left=5]\ar[r, "s_1" description, shift left=2]\ar[r, "s_2" description, shift right=2]\ar[r, "s_3"' description,shift right=5] \& X[4] \ar[r, "s_0" description,shift left=6]\ar[r, "s_1" description, shift left=3]\ar[r, "s_2" description]\ar[r, "s_3" description, shift right=3]\ar[r, "s_4"' description, shift right=6] \& \;\cdots 
\end{tikzcd}
$$
Let us formalise this viewpoint. Let $[n]=\{0,1,\ldots,n\}$ be the finite ordinal of cardinality $n+1$.
\begin{defn}
    The \emph{simplicial category} $\Delta$ has objects the non-empty finite ordinals and morphisms the order-preserving maps between them. There are two distinguished families of morphisms
\begin{align*}
    \delta^n_j\colon &[n-1]\longrightarrow [n], & \sigma^n_j\colon &[n+1]\longrightarrow [n], & 0\leq j\leq n,
\end{align*}
called \emph{cofaces} and \emph{codegeneracies}, respectively, and defined as
\begin{align*}
    \delta_j^n(i)&=\begin{cases}
        i & \text{if $i<j$} \\
        i+1 & \text{if $i\geq j$}
    \end{cases}
&
    \sigma_j^n(i)&=\begin{cases} i &\text{if $i\leq j$} \\ i-1 & \text{if $i>j$}
\end{cases}
\end{align*}
These morphisms satisfy the following relations, called \emph{cosimplicial identities}
\begin{align*}\label{eq:007}
\delta^n_{j}\circ \delta^{n-1}_{i+1} &= \delta^n_{i}\circ \delta^{n-1}_j, \qquad j\leq i, \\
\sigma^n_{j+1}\circ \sigma^{n+1}_i &= \sigma^n_i\circ \sigma^{n-1}_{j},\qquad i\leq j, \\
\delta^{n+1}_i\circ \sigma^n_j &= \begin{cases}
    \sigma^n_{j-1}\circ \delta^{n+1}_i & i<j, \\
    \id & i=j,j+1, \\
    \sigma^n_j\circ \delta^{n+1}_{i-1} & i>j+1,
\end{cases}
\end{align*}
and generate all the morphisms in the category in the sense that every morphism can be expressed as a composite of cofaces and codegeneracies. 
\end{defn}
\begin{defn}
Let $\Delta_{\surj}\subset \Delta$ be the category of non-empty finite ordinals with order-preserving surjections between them. It is the subcategory of $\Delta$ generated by the maps $s_j^n$.
\end{defn}
\begin{defn}
    Let $\Set$ be the category of sets and functions. A~\emph{simplicial set} $X$ is a functor $X\colon\Delta^\op\to \Set$. The maps $X(\delta^n_i)$ are called \emph{faces} and denoted $d^n_i$ and the maps $X(\sigma^n_i)$ are called \emph{degeneracies} and denoted $s^n_i$. A \emph{demi-simplicial set}\footnote{It seems these objects have not been considered previously in the literature. Since simplicial sets without degeneracies are called \emph{semi-simplicial sets}, we have chosen to replace the prefix \emph{semi-} by its french version \emph{demi-}. Additionally, the first letter of each prefix specifies whether we are removing degeneracies or face maps.} $X$ is a functor $X\colon\Delta^\op_\surj\to \Set$.
\end{defn}

Faces and degeneracies satisfy the so-called \emph{simplicial identities} which are the dual of the cosimplicial identities mentioned above. In the particular case of the relation involving only degeneracies, we get the identity
\begin{equation}
s^{n+1}_i\circ s^n_{j+1}=s^{n-1}_{j}\circ s^n_i,\qquad i\leq j.\label{eq:deg_rel}
\end{equation}

In what follows, when we refer to conditions in roman numbers, we mean the conditions satisfied by cloning systems and bilateral cloning systems from Definition~\ref{defn:CloningSystem} and Definition~\ref{defn:NCCloningSystem}.

Observe that \eqref{eq:deg_rel} is exactly the same relation satisfied by the maps $\kappa^n_j$ with the extra condition (iv+) except that the subindexes are shifted by one (the first map is $\kappa_1$ not $\kappa_0$). Therefore we have the following consequence. 
\begin{lem} Let $\G_{\bullet}=\{\G_n\}_{n\geq 1}$ be a family of groups and let $\kappa=\{\kappa^n_{j}\colon\G_{n}\to\G_{n+1}\}_{n\ge 1,\,1\leq j\leq n}$ be a family of maps. A pair $(\G_\bullet,\kappa)$ satisfying conditions \emph{(iv)} and \emph{(iv+)} is the same as a demi-simplicial set with values on groups. $\hfill\qed$
\end{lem}
\begin{examp}\label{ex:demi_symm} Let us build the demi-simplicial set associated to the cloning system of symmetric groups. Note that an order-preserving surjection $f\colon [n]\to [m]$ is completely determined by the cardinality of the preimages $f^{-1}(0),\ldots,f^{-1}(m)$. Every permutation $h\in \Sigma_{m+1}$ of $[m]$ induces a permutation of the preimages $f^{-1}(0),\ldots,f^{-1}(m)$, and therefore a block permutation on $[n]$, that we denote by $\Phi(f)(h)\in \Sigma_{n+1}$. This defines a functor $\Phi\colon \Delta_\surj^\op\to \Set$ with $\Phi([n]) = \Sigma_{n+1}$.

There is an action of $\Sigma_{m+1}$ on the set of order-preserving surjections from $[n]$ to $[m]$, that sends a permutation $g$ and a surjection $f$ to the unique surjection $h$ such that the cardinality of $h^{-1}(g(i))$ is equal to the cardinality of $f^{-1}(i)$. We denote this surjection $h$ by $f_g$. Observe now that the map $\Phi(f)\colon \Sigma_{m+1}\to \Sigma_{n+1}$ is not a group homomorphism but satisfies that 
\begin{equation*}\label{eq:610}
    \Phi(f)(g\cdot g') = \Phi(f_{g'})(g)\cdot\Phi(f)(g'),
\end{equation*}
which applied to degeneracies is the same as condition (v) in a cloning system
\begin{equation}\label{eq:611}\Phi(s_i)(g\cdot g') = \Phi(s_{g'(i)})(g)\cdot \Phi(s_i)(g').
\end{equation}
The functor $\Phi$, defined on order-presering surjections between ordinals, can be extended to any order-preserving map between ordinals. Indeed, since every map factors as a surjection followed by an injection, it is enough to define it on injections, which we can do as follows. Note that to give an order-preserving injective function $f\colon [n]\to [m]$ is equivalent to specify the complement of the image $A = [m]\smallsetminus f([n])$. There is an action of $\Sigma_{m+1}$ on the set of order-preserving injections from $[n]$ to $[m]$, that sends a permutation $g$ and an injection $f$ to the unique injection $h$ such that $[m]\smallsetminus h([n]) = [m]\smallsetminus g(f([m]))$. We denote this injection $h$ by $f_g$. If $g\in \Sigma_{m+1}$ is a permutation of $[m]$, define $\Phi(f)(g) = f_g^{-1}\circ g\circ f$. This defines a functor $\Phi\colon \Delta^\op\to \Set$, and hence a simplicial set.

 Observe now that the map $\Phi(f)\colon \Sigma_{m+1}\to \Sigma_{n+1}$ is not a group homomorphism but satisfies that 
\begin{equation*}\label{eq:613}
    \Phi(f)(g\cdot g') = \Phi(f_{g'})(g)\cdot\Phi(f)(g').
\end{equation*}
\end{examp}
The following definition of crossed simplicial groups is a characterization taken from~\cite[Proposition 1.7]{FL}, where $\Phi$ denotes the simplicial set constructed above.
\begin{defn}
    A \emph{crossed simplicial group} is a simplicial set $\Psi\colon \Delta^\op\to \Set$ with values on groups together with a levelwise group homomorphism $\pi\colon \Psi\to \Phi$ such that $\pi(s_i^{\Psi}(g))[j] = s_i^{\Phi}(\pi(g))[j]$ for all $j\neq i,i+1$ and
\begin{align*}
    \Psi(s_i)(g\cdot g') &= \Psi(s_{\pi(g)(i)})(g')\cdot\Psi(s_i)(g'), \\ 
    \Psi(d_i)(g\cdot g') &= \Psi(d_{\pi(g)(i)})(g')\cdot\Psi(d_i)(g'),
    \end{align*} and $\pi([n])\colon \Psi([n])\to \Sigma_n$ is a group homomorphism. A \emph{crossed demi-simplicial group} is defined in the same way, replacing the category $\Delta$ by the category $\Delta_\surj$. 
\end{defn}

The following result follows immediately from the previous discussion and the definition of crossed demi-simplicial group.
\begin{lem} Let $\G_{\bullet}=\{\G_n\}_{n\geq 1}$ be a family of groups, $\pi=\{\pi_n\colon \G_n\to\Sigma_n\}_{n\ge 1}$ a family of group homomorphisms and $\kappa=\{\kappa^n_{j}\colon\G_{n}\to\G_{n+1}\}_{n\ge 1,\,1\leq j\leq n}$ a family of maps. A triple $(\G_\bullet,\pi,\kappa)$ satisfying conditions {\rm (ii)}, {\rm (iv)}, {\rm (iv+)} and {\rm (v)} is the same as a crossed demi-simplicial group. $\hfill\qed$
\end{lem}

Now we would like to incorporate the homomorphisms $\iota$ to the picture. We will incorporate the morphisms $\zeta$ at the same time.

\begin{defn} For each $n\geq 1$, the $n^{\text{th}}$ \emph{interval} is the set $\langle n\rangle = \{-\infty,1,\ldots,n,\infty\}$. The \emph{interval category} $\I$ is the category whose objects are all intervals and whose morphisms are order-preserving maps that preserve $-\infty$ and $\infty$. The subcategory $\I_\surj$ has the same objects as $\I$ and its morphisms are the order-preserving surjective maps that preserve $-\infty$ and $\infty$.\end{defn}
The interval category can be introduced as the Joyal dual of the simplicial category \cite{Joyal} or as the image of the faithful embedding $\alpha\colon \I\to \Delta$ that sends $\langle n\rangle = \{-\infty,1,\ldots,n,\infty\}$ to $[n+1] = \{0,1,\ldots,n+1\}$, and an interval map yields a simplicial map by interpreting $-\infty$ as $0$ and $\infty$ as $n+1$. Here we are interested in the second description, and we will blur de difference between maps in $\I$ and their images under the embedding $\alpha$. Therefore, from now on we will write simply $\Phi$ for the crossed interval group $\Phi\circ \alpha$.
\begin{defn}
    An \emph{inert crossed interval group} is a presheaf $\Psi\colon \I^\op\to \Set$ with values on groups together with a levelwise group homomorphism $\pi_n\colon \Psi([n])\to \Phi([n])$ such that $\pi(s_i^{\Psi}(g))[j] = s_i^{\Phi}(\pi(g))[j]$ for all $j\neq i,i+1$ and
    \begin{align*}\label{eq:product_cig}
    \Psi(s_i)(g\cdot g') &= \Psi(s_{\pi(g)(i)})(g')\cdot\Psi(s_i)(g'), \\ 
    \Psi(d_i)(g\cdot g') &= \Psi(d_{\pi(g)(i)})(g')\cdot\Psi(d_i)(g'),
    \end{align*} 
     An \emph{inert crossed demi-interval group} is defined in the same way, by replacing $\I$ by $\I_\surj$.
\end{defn}
Observe now that if $\Phi$ is an inert crossed interval group, the maps $\Phi(s_0)$ and $\Phi(s_{n+1})$ are group homomorphisms. In fact, we can interpret bilateral cloning systems as an inert crossed demi-interval groups by setting $\kappa^n_i = s^{n+1}_i$, $\zeta_n = s^{n+1}_0$ and $\iota_n = s^{n+1}_{n+1}$. Thus, we have the following result.
\begin{lem} A quintuple $(\G_\bullet,\pi,\iota,\zeta,\kappa)$ satisfying all the conditions of a bilateral cloning system is the same as an inert crossed demi-interval group. 
\end{lem}

\begin{rem}
    Since crossed interval groups embedd into crossed simplicial groups, every action operad gives rise to a crossed simplicial group. This is carefully developed in \cite[2.4]{Zhang}.
\end{rem}

\begin{rem} Crossed interval groups have been studied in \cite{Batanin-Markl} and \cite{Yoshida}. The adjective \emph{inert} corresponds to any of the two equivalent properties defined in \cite[Lemma 4.3]{Yoshida}.

In that paper, Yoshida studied the relation between crossed interval groups and action operads. He established that every action operad determines a crossed interval group, and found three properties that characterise the crossed interval groups that come from an operad: operadicness, tameness and ``factoring through the symmetric group''. Under the above lemma, tameness corresponds to property (ix), while operadicness corresponds to property (viii) plus being inert and ``factoring through the symmetric group'' corresponds to (ii+). Yoshida studies action operads with constants (operations of arity 0), while we study action operads without constants. The existence of constants in the action operad corresponds to the injective morphisms in the simplicial or the interval category.

We note that Examples 3.1 and 3.4 in \cite{Zaremsky}, which do not come from an action operad, do come from an inert crossed demi-interval group (with trivial homomorphisms $\pi$). 
\end{rem}

%\fc{
%\begin{rem}
%    By the classification result of Loday and Fiedorowicz, we can replace the requirement of having a 
%    \begin{itemize}
        %\item ``levelwise group homomorphism $\pi\colon \Psi\to \Phi$ such that $\pi(s_i^{\Psi}(g))[j] = s_i^{\Phi}(\pi(g))[j]$ for all $j\neq i,i+1$''.
%    \end{itemize} by the following: Let $\Phi'$ be the crossed interval group corresponding to the hyperoctahedral crossed simplicial group. The $\pi'\colon \Psi\to \Phi'$ is a natural transformation of set-valued functors that is a levelwise group homomorphism. The levelwise group homomorphism $\pi\colon \Psi\to \Phi$ is then recovered as the composition $\Psi\to \Phi'\to \Phi$.
%\end{rem}
%}
