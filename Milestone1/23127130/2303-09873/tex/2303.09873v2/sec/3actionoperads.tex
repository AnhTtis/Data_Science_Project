

\section{Action operads}\label{section:action_operads}
We start this section recalling the classical notion of an {\em operad}: 




\begin{defn}\label{defn:Operad} 
A \emph{symmetric operad} $\EE$ on sets is a triple $(\EE, \{\circ_i\}_i, \id)$, where
\begin{itemize}
\item $\EE=\{\EE(n)\}_{n\ge 1}$ is a family of sets, and each $\EE(n)$ is equipped with a right $\Sigma_n$-action, for every $n\ge 1$,
\item ${\id}\in \EE(1)$ is called the \emph{unit} of the operad, and \item $\{\circ_i\}_i$ is a family of maps
$$
\circ_i\colon \EE(n)\times \EE(m)\to \EE(n+m-1) \quad\mbox{for all $n,m\ge 1$ and $1\le i\le n$},
$$ called \emph{$\circ_i$-operations} or \emph{partial composition products},
such that ${\id} \circ_i y=y$ and $x\circ_i {\id}=x$ for every $x,y\in \EE(n)$. Moreover, these $\circ_i$\nobreakdash-operations satisfy certain associativity and equivariance axioms, which are spelled out in, for example, \cite[Definition 11]{Markl}.
\end{itemize}


A \emph{morphism of operads} $f\colon\EE\to \mathsf{P}$ consists of maps $f_n\colon \EE(n)\to \mathsf{P}(n)$ for $n\ge 1$, that are compatible with the unit and $\circ_i$-operations of $\EE$ and $\mathsf{P}$. If we forget about all the symmetric group actions on the sets $\EE(n)$, we have the notion of a \emph{non-symmetric operad}.
\end{defn}

\begin{rem} Note that Definition \ref{defn:Operad} avoids nullary operations in an operad, i.e.\ there is no $\EE(0)$ in $\EE$. In this work, we will only consider operads without nullary operations, or in other words, without constants. This choice is not essential, but it is made to have a clearer connection between action operads and cloning systems.
\end{rem}

\begin{examp}[Symmetric groups]
The family of symmetric groups $\Sigma_{\bullet}=\{\Sigma_n\}_{n\ge 1}$ has the structure of a non-symmetric operad in sets, with $\id\in\Sigma_1$ the trivial permutation of $\Sigma_1$. The partial composition products $\circ_i\colon\Sigma_n\times \Sigma_m\to \Sigma_{m+n-1}$ are defined as follows: if $\sigma\in \Sigma_n$ and $\tau\in\Sigma_m$, then $\sigma\circ_i\tau$ is the permutation of $\Sigma_{m+n-1}$ obtained by ``inserting'' $\tau$ in $\sigma$ at the $i$-th place as a block and rearranging the indices accordingly. Figure~\ref{fig:CompositionProduct Sigma} shows an example of the composition product $\circ_2\colon\Sigma_3\times \Sigma_2\to\Sigma_4$.
\begin{figure}[h]
    \centering
    \includegraphics[width=5cm]{figures/CompositionProduct_Sigma.pdf}
    \caption{\scriptsize
    $\left(\begin{matrix}
     1 & 2 & 3\\
     2 & 3 & 1
    \end{matrix}\right)\circ_2
    \left(\begin{matrix}
     1 & 2 \\
     2 & 1
    \end{matrix}\right)
    = 
    \left(\begin{matrix}
     1 & 2 & 3 & 4\\
     2 & 4 & 3 & 1
    \end{matrix}\right)$.}
    \label{fig:CompositionProduct Sigma}
    \end{figure}
\end{examp}

\subsection{Action operads}
We now introduce the concept of an {\em action operad}, which essentially is a family of groups satisfying certain properties that allow to define operads with equivariance relative to this family; the reader is referred to \cite{Corner-Gurski} for a detailed treatment of action operads and their properties. They have been also studied under the name \emph{group operads} \cite{Zhang, Yoshida, Yau}

\begin{defn}\label{defn:ActionOperad} An \emph{action operad without constants} $\G$, or simply an \emph{action operad}, is a quadruple $(\G_{\bullet},\pi,\{\circ_i\}_i,\id)$, where
\begin{enumerate}
    \item\label{cond:ac:1} $\G_\bullet=\{\G_n\}_{n\ge 1}$ is a family of groups, and $(\G_{\bullet},\{\circ_i\}_i,\id)$ is a non-symmetric operad on sets, where $\{\circ_i\}_i$ are the partial composition products of the operad and $\id\in \G_1$ is the unit. The associativity of the partial composition products yields that if $f\in \G_n$, $g\in \G_m$ and $h\in \G_l$, then we have that
\begin{align*}
(f\circ_i g)\circ_j h &= \begin{cases}
(f\circ_j h) \circ_{i+l-1}g & \text{if $j<i$},\\
 (f \circ_{j-m+1} h)\circ_i g& \text{if $j\geq i+m$},\\
f\circ_i (g\circ_{j-i+1}h) & \text{if $j=i,\ldots, i+m-1$},
\end{cases} \\
f\circ_i \id &= f, \\
\id\circ_1 g &= g;
\end{align*}
    \item\label{cond:ac:2} $\pi: \G_\bullet \to \Sigma_{\bullet}$ is a map of operads which is also a levelwise group homomorphism, that is, $\pi_n\colon\G_n \to \Sigma_n$ is a homomorphism for all $n\ge 1$;
\item\label{cond:ac:3} For every $f, f' \in \G_n$, $g,g'\in \G_{m}$, we have that
\begin{align*}
(f\cdot f')\circ_i(g\cdot g') &= (f\circ_{\pi(f')(i)}g)\cdot (f'\circ_i g'),
\end{align*}
with the multiplication taking place in the group $\G_{n+m-1}$.
\end{enumerate}
\label{def:actionoperad}
\end{defn}

 Note that the partial composition products $\circ_i$ are not group homomorphisms in general, that action operads are not assumed to be symmetric operads, and that they have no nullary operations. It follows from the axioms that the unit element $\id\in \G_1$ of the operad $\G_{\bullet}$ is precisely the unit $e_1$ of the group $\G_1$.

 Observe that, both in a cloning system and in an action operad, the group $\G_n$ acts on the set $\{1,\ldots,n\}$ via the homomorphism $\pi_n$ for every $n$. If these maps $\pi$ are understood from the context, for $g\in \G_n$ we will write $g(i)$ instead of $\pi_n(g)(i)$.


Finally, note that an action operad with trivial $\pi$ is the same thing as a non-symmetric operad on  groups.    

\subsection{General action operads} Inspired by the definition of cloning system, we relax as follows the definition of action operad. Note that this relaxation does not diminish the hability of such an operad to hold the equivariance of other operads. In fact, we believe that this should be the adequate definition of action operad. 
\begin{defn}
    A \emph{general action operad} $\G$ is a tuple $(\G_\bullet,\pi,\{\circ_i\}_i,\id)$ satisfying Conditions \eqref{cond:ac:1} and \eqref{cond:ac:3} together with the following:
    \begin{itemize}
        \item[(2')] $\pi_n\colon \G_n\to \Sigma_n$ is a levelwise group homomorphism for each $n\geq 1$ such that if $g\in \G_n$ and $h\in \G_m$, then the action of the symmetric group elements $\pi(g)\circ_i \pi(h)$ and $\pi(g\circ_i h)$ coincide on $$\{1,2,\ldots,i-1,i+m,i+m+1,\ldots,n+m-1\}.$$
    \end{itemize}
\end{defn}
Here is an example of general action operad that is not an action operad. It corresponds to the cloning system of signed symmetric groups or to the hyperoctahedral inert demi-interval group (see Section \ref{section:crossed}).

The group $\Sigma^{\pm}_n$ is the signed symmetric group, that consist on permutations $g$ of the set$\{1,-1,2,-2,\ldots,n,-n\}$ such that $g(i) = - g(-i)$. The composition $g\circ_i h$ is defined as follows: if $g(+i)$ is positive, then $g\circ_i h$ is obtained by cabling the signed symmetry $h$ in the $i$-th strand of the permutation $g$. If $g(+i)$ is negative, then $g\circ_i h$ is obtained by cabling the signed symmetry $h'$ in the $i$-th strand of $g$, where $h'(j) = -h(n-j+1)$ for a positive $j$. A signed permutation $g$ is determined by the pair $(\sigma,A)$, where $\sigma$ is the underlying permutation and $A$ is the set of $i$'s such that $g(+i)$ is negative. With this notation, the composition of $(g,A)$ and $(g',A')$ is $(g\circ g',g'(A)\triangle A')$, where the symbol $\triangle$ denotes symmetric difference.

From the classification of inert crossed interval groups and the relation between action operads and crossed interval groups, it is reasonable to think that the signed symmetric general action operad is final in the category of general action operads. This allows to simplify as follows the definition of general action operad:
\begin{lem}
    If we require that the target of the map $\pi$ is the signed symmetric group instead of the symmetric group, then
    Condition {\rm (2')} in the definition of general action operad can be replaced by the following: 
    \begin{itemize}
        \item[(2'')] $\pi: \G_\bullet \to \Sigma^{\pm}_{\bullet}$ is a map of operads and a levelwise group homomorphism. 
    \end{itemize}
\end{lem}
\begin{proof} Suppose that $\pi\colon \G_n\to \Sigma_n$ is a family of maps satisfying Condition (2'). We will define a family of maps $\pi'\colon \G_n\to \Sigma^{\pm}_n$ satisfying Condition (2'').

We first define the map $\pi'_n\colon \G_n\to \Sigma^{\pm}_n$. If $g\in \G_n$, define $\pi'(g) = (\pi(g),A)$ with $A = \{j\mid \pi(g\circ_2 e_2)\neq \pi(g)\circ_2 e_2\}$. To check that $\pi'$ is a group homomorphism, we have that, if $g,h\in \G_n$ and $\pi'(g) = (\sigma,A)$ and $\pi'(h) = (\sigma',A')$, and $\pi'(g\cdot h) = (\sigma'',A'')$, then $\sigma'' = \sigma\cdot \sigma'$ and
    \[
        (g\cdot h)\circ_j e_2 = (g\cdot h)\circ_j (e_2\cdot e_2) = (g\circ_{\sigma'(j)} e_2)\cdot (g\circ_2 e_2)
    \]
    The product $\pi((g\cdot h)\circ_j e_2)$ is either equal to $\pi(g\cdot h)\circ_j e_2$ or differs from it by a twist of the entries $\{j,j+1\}$. Inspection shows that $A'' = g'(A)\triangle A'$.

    One still has to show that $\pi'$ is an operad map. By the multiplication rule it is enough to check it for the compositions $g\circ_j e_m$. Since the latter are iterated compositions of the form $g\circ_2 e_2$, it is enough to check it in this last case, in which is true by definition.

    Finally, if $\pi'\colon \G_n\to \Sigma_n^\pm$ is a family of maps satisfying Condition (2''), then composing them with the homomorphism $\Sigma_n^{\pm}\to \Sigma_n$ that forgets the signs yields a family of maps satisfying $(2')$.
\end{proof}

\begin{rem}
    This lemma has its counterpart in the world of cloning systems: Condition (ii) in the definition of cloning system can be replaced by the following:
    \begin{itemize}
        \item[(ii')] $\pi_n\colon \G_n\to \Sigma_{n}^\pm$ is a levelwise group homomorphism such that the identity $\pi_{n+1}\circ \kappa_j^n = c_j^n\circ \pi_n$ holds for all $n\geq 1$ and all $1\leq j\leq n$,
    \end{itemize}
where $c_j^n$ are the cloning maps of the signed symmetric cloning system. In order the make sense of axioms (v) and (viii), we are implicitly using the fact that $\Sigma^\pm_n$ acts on $\{1,2,\ldots,n\}$ through the homomorphism $\Sigma_n^{\pm}\to \Sigma_n$ that forgets the signs.
\end{rem}


\subsection{Fundamental groups of $\G$-operads and  unoriented ribbon braids}

Fix an (general) action operad $\G$. Then, one may define a $\G$-operad as in \cite[Def.~1.14]{Corner-Gurski}, \cite[Def.~4.2.6, Prop. 4.3.1]{Yau} or \cite[Def.~2.30]{Zhang}, i.e., as a non-symmetric operad $\EE$ with an action $\EE(n)\times \G_n\to \EE(n)$ such that the composition
\[
    \EE(n)\times \big(\EE(n_1)\times \ldots\times \EE(n_k)\big)\lra \EE(n_1+\ldots+n_k)
\]
is equivariant with respect to the map
\[
    \G_n\times (\G_{n_1}\times \ldots\times \G_{n_k})\lra \G_{n_1+\ldots+n_k}.
\]
Alternatively, if one considers operads with identities as we do, one requires the composition $\circ_i\colon\EE(n)\times \EE(m)\to \EE(n+m-1)$ to be equivariant with respect to the map $\circ_i\colon\G_n\times \G_m\to \G_{n+m-1}$. 
%\footnote{\fc{Esto ya no hace falta, y quedaba un poco offtopic} If $M$ is a manifold and $f\colon L\to M$ is a fibre bundle, the configuration space $C_n(M;L)$ is the quotient of the space $\{(x_1,\ldots,x_n)\in L^n\mid f(x_i)\neq f(x_j)\}$ by the action of the symmetric group $\Sigma_n$. Consider the following fibre bundles:
%\begin{itemize}
%    \item The trivial bundle $M\to M$.
%    \item $S(T\bC) = \{(p,v)\mid p\in \bC, v\in T_p \bC, \|v\| = 1\}$, the unit sphere bundle of the tangent bundle of the complex plane $\bC$.
%    \item $P(T\bC) = \{(p,L)\mid p\in \bC, L\in P(T_p \bC)\}$ is the fibrewise projective bundle of the tangent bundle of $\bC$.
%\end{itemize}
%The fundamental group of $C_n(\bC;\bC)$ is the braid group $\Br_n$, the fundamental group of $C_n(\bC;S(T\bC))$ is the \emph{ribbon braid group} $\RBr_n$ and the fundamental group of $C_n(\bC;P(T\bC))$ is the \emph{unoriented ribbon braid group} $\URBr_n$. 

%The ordered configurations in $\bC$ and their framed counterpart are homotopy equivalent to the spaces of operations in the little $2$-discs operad $\C(2)$ and the framed little $2$-discs operad $\C^\fr(2)$. These symmetric operads can be endowed with a good basepoint, and therefore the fundamental groups of their quotients by the symetric group are action operads \cite[Theorem~3.4]{Zhang} 

%The ordered configurations with labels in $P(T\bC)$ do not form an operad, because the information of a line through the origin is not enough to yield an identification of a neighbourhood of each point with $\bC$.

%\vc{DEBERIA DAR UN OPERAD IGUALMENTE, NO CREO QUE HAYA NINGUN PROBLEMA CON $P(T\bC)$.}
%\fc{Yo creo que no es un operad}
%\vc{$\{\Conf_n(\bC)\}_{n}$ corresponde con little discs con trivializacion del fibrado tangente, es decir el fibrado principal sobre el grupo trivial (piénsalo como $\{\Emb^{\mathsf{fr}}(\bC^{\sqcup n},\bC)\}_n$ y está claro porqué es un operad). $\{\Conf^{\mathsf{fr}}_n(\bC)\}_{n}$, que lo estamos llamando framed pero yo prefiero llamarlo oriented, corresponde con el fibrado principal con grupo $\mathsf{SO}(2)=\mathbb{S}^1$ (piénsalo como $\{\Emb^{\mathsf{or}}(\bC^{\sqcup n},\bC)\}$). El que falta es el que correspondería con el grupo proyectivo $\mathsf{PSO}(2)$ (cociente de $\{\pm \id\}\subset \mathsf{SO}(2)$ si no me equivoco) y por tanto con los embeddings que preservan esta estructura proyectiva salvo homotopía. Esto tiene sentido?? 
%
%Esto me lleva a pensar que el unoriented ribbon braid operad se podría definir como $\{\pi_1\big(\Emb^{\mathsf{pso}}(\bC^{\sqcup n},\bC)/\Sigma_n\big)\}_n$.
%}

% Nonetheless, we will see that the fundamental groups of $C_n(\bC;P(T\bC))$ do form an operad because the loop has the missing information to build a canonical identification of a neighbourhood of each point with $\bC$.
%}

Our goal in this subsection is to provide more examples of (general) action operads. For that purpose, we briefly discuss a construction based on fundamental groups of topological $\G$-operads that produces (general) action operads over $\G$.

%Due to applications of the fundamental group functor, we need choices of basepoints compatible with the $\G$-action and the composition products. That is the reason for the following detour.

First, observe that the forgetful functor from $\G$-operads to non-symmetric operads admits a left adjoint which simply adds a free  right $\G$-action,
 $$
(-)_{\G}\colon \begin{Bmatrix}
\text{non-symmetric}\\
\text{operads}
\end{Bmatrix}  \longrightarrow \begin{Bmatrix}
\G\text{-operads}
\end{Bmatrix}, \quad \mathsf{P}\mapsto \mathsf{P}_{\G},
 $$
 where $\mathsf{P}_{\G}(n)=\mathsf{P}(n)\times \G_{n}$ and the operation $\circ_i$ is defined as the composition
 $$
 \begin{tikzcd}[ampersand replacement=\&]
     \mathsf{P}_{\G}(n)\times \mathsf{P}_{\G}(m)\ar[d,"\cong"',"\text{switch}"] \\ 
     \mathsf{P}(n)\times \mathsf{P}(m)\times \G_{n}\times\G_{m}
     \ar[d,"\circ_i\times \circ_i"] \\ \mathsf{P}(n+m-1)\times \G_{n+m-1}\ar[d, equal]\\
     \mathsf{P}_{\G}(n+m-1)
 \end{tikzcd}\quad .
 $$
Of course, the same discussion applies to topological operads.
 
 Applying this left adjoint functor to the (non-symmetric) associative operad $\mathsf{As}$, characterized by $\mathsf{As}(n)=*$ for any $n\geq 1$ (and $\mathsf{As}(0)=\emptyset$), we obtain $\mathsf{As}_{\G}$. A map from $\mathsf{As}_{\G}$ selects basepoints in a coherent way in a $\G$-operad. In fact, 
 \begin{defn} Let $\EE$ be a topological $\G$-operad. Then, a \emph{good $\G$-basepoint} for $\EE$ is a map of topological $\G$-operads $\eta\colon\mathsf{As}^h_{\G}\to \EE$, where $\mathsf{As}^h_{\G}$ is a topological $\G$-operad homotopy equivalent to $\mathsf{As}_{\G}$.
 \end{defn}

We assume without loss of generality that for any $k\geq 1$ there is a choice of basepoint $\mu_k\in \mathsf{As}_{\G}^{h}(k)$ representing the plain $k$-ary multiplication, that is, lying in the connected component corresponding to $(*,e_k)\in \mathsf{As}_{\G}(k)$. By definition, there are paths $\mu_n\circ_i\mu_m\simeq \mu_{n+m-1}$ in $ \mathsf{As}_{\G}^h(n+m-1)$ and homotopies relating natural concatenations of those paths. For that reason, we fix basepoints $\epsilon_k:=\eta(\mu_k)\in \EE(k)$ for paths and loops as \cite[Section 3]{Zhang} in the sequel.

%Through $\eta\colon \mathsf{As}_{\G}^{h}\to \EE$, we transfer this structure to $\EE$ obtaining the data required to deal with the basepoints $\epsilon_k:=\eta(\mu_k)\in \EE(k)$ when considering paths and loops in $\EE(k)$ as done in \cite[Section 3]{Zhang}. \fc{la frase "we transfer this structure" es un poco ambigua. ¿Qué estructura exactamente se está preservando? ¿Es necesario señalarlo?}

Now, assume that the action of $\G_n$ on $\EE(n)$ is a covering action (see \cite[Section 1.3]{Hatcher}) and that $\EE(n)$ is path-connected. Then, we have a homotopy fiber sequence $\G_n\hookrightarrow \EE(n)\twoheadrightarrow \EE(n)/\G_n$, which induces a short exact sequence of groups
\begin{equation}\label{eqt:ses pi1 of Goperads}
    1\longrightarrow \pi_1\big(\EE(n),\epsilon_n\big)\longrightarrow \pi_1\big(\EE(n)/\G_n,[\epsilon_n]\big)\overset{\delta_n}{\longrightarrow} \G_n \longrightarrow 1,
\end{equation}
and an isomorphism of relative homotopy groups
\begin{equation}\label{eqt:pi1 of Goperads}
    \pi_1\big(\EE(n)/\G_n,[\epsilon_n]\big)\cong \pi_1\big(\EE(n);\G_n\!\cdot\, \epsilon_n,\epsilon_n\big).
\end{equation}
The left-hand side in (\ref{eqt:pi1 of Goperads}) has a group structure, while the right-hand side is easily seen to form an operad. Altogether, we have the following generalization of \cite[Theorem 3.4]{Zhang}:% (see also Theorem 3.8):
 \begin{prop} Let $(\G,\pi,\circ_i,\id)$ be an (general) action operad, $\EE$ be a topological $\G$-operad without constants so that: {\rm (i)} $\EE$ is equipped with a good $\G$-basepoint, {\rm (ii)} $\EE(n)$ is path-connected for any $n\geq 1$, and {\rm (iii)} $\G$ acts on $\EE$  via covering actions. Then, the sequence of groups $\pi_1\big(\EE(n)/\G_n,[\epsilon_n]\big)$ together with the maps $$\pi_1(\EE(n)/\G_n,[\epsilon_n])\overset{\delta_n}{\lra} \G_n\overset{\pi}{\lra} \Sigma_n$$ conforms an (general) action operad, denoted $\pi_1(\EE,\G)$. Moreover, $\pi_1(\EE,\G)$ lies over $\G$, i.e.\ the connecting homomorphisms $\{\delta_n\}_{n\geq 1}$ in (\ref{eqt:ses pi1 of Goperads}) yield a morphism $\delta\colon\pi_1(\EE,\G)\to \G$ of (general) action operads. 
 \end{prop}

 
 The next two examples were considered in \cite{Wahl}, \cite{Zhang}, \cite{Corner-Gurski} and \cite{Yau}. The third one is a general action operad, but not an action operad.

\begin{examp}
    Consider the action operad $\G = \Sigma$, and take $\EE$ to be the little $2$-discs operad $\C_2$ with its natural right $\G$-action: 
    $$
    \begin{tikzcd}[ampersand replacement=\&]
    \C_2(n)\times \Sigma_n \ar[rr] \&\& \C_2(n) \\[-6mm]
    \big((x_1,\dots,x_n),g\big) \ar[rr, mapsto] \&\& \big(x_{g(1)},\dots,x_{g(n)}\big)
    \end{tikzcd}.
    $$
    %if $g\in \Sigma_n$ and $(x_1,\ldots,x_n)\in \C_2(n)$, then define $(x_1,\ldots,x_n)\cdot g$ as $(x_{g(1)},\ldots,x_{g(n)})$.
    Now, observe that the little $1$-discs operad $\C_1$ is homotopy equivalent to the symmetric operad $\mathsf{Ass}=\mathsf{As}_{\Sigma}$, and the inclusion $\C_1\to \C_2$ is a good $\G$-basepoint. The operad $\Br = \pi_1(\C_2,\Sigma)$ is the action operad of \emph{braid groups}.
\end{examp}

\begin{examp}
    The previous example remains valid if we replace the little $2$-discs operad $\C_2$ by its framed version $\C^{\fr}_2\simeq \C_2\rtimes \mathsf{SO}(2)$. The resulting operad $\RBr = \pi_1(\C^{\fr}_2,\Sigma)$ is the action operad of \emph{ribbon braid groups}. Recall that 
    $$\RBr_k\cong \Br_k\rtimes\, \mathbb{Z}^{\times k},$$
    where $\mathbb{Z}^{\times k}$ accounts for the number of full twists on each ribbon. 
\end{examp}

\begin{examp}
    Consider the general action operad $\G = \Sigma^{\pm}$, and take $\EE$ to be the framed little $2$-discs operad $\C^\fr_2$ with the following $\G$-action:
    $$
    \begin{tikzcd}[ampersand replacement=\&]
    \C_2^{\fr}(n)\times \Sigma_n^{\pm} \ar[rr] \&\& \C_2^{\fr}(n) \\[-6mm]
    \big((x_1,\dots,x_n),(g,A)\big) \ar[rr, mapsto] \&\& \big(x^A_{g(1)},\dots,x^A_{g(n)}\big)
    \end{tikzcd},
    $$
    %if $(g,A)\in \Sigma^{\pm}_n$ and $(x_1,\ldots,x_n)\in \C^{\fr}_2(n)$, then define $(x_1,\ldots,x_n)\cdot(g,A)$ as $(x_{g(1)}^A,\ldots,x_{g(n)}^A)$,
    where $x_i^A = x_i$ for any $i\notin A$ and $x_i^A$ is the precomposition of the embedding $x_i\colon D^2\to D^2$ with a $\pi$-rotation otherwise. Now, observe that the ``unoriented little $1$-discs operad'' $\C^{\mathsf{un}}_1\simeq \C_1\rtimes \OO(1)$ is homotopy equivalent to the operad $\mathsf{As}_{\Sigma^{\pm}}$ since 
    $$
    \C_1^{\mathsf{un}}(k)\simeq \C_1(k)\times \OO(1)^{\times k} \quad \text{ and } \quad  
    \mathsf{As}_{\Sigma^{\pm}}(k)\cong \Sigma_{k}^{\pm}\cong \Sigma_k\times \{\pm \id\}^{\times k}. 
    $$ 
    Hence, the inclusion $\C^{\mathsf{un}}_1\to \C^{\fr}_2$ determined by the inclusion $\C_1\to \C_2$ and the homomorphism $\OO(1)\to \mathsf{SO}(2)$, $-\id\mapsto e^{i\pi}$, is a good $\G$-basepoint. The operad $\URBr = \pi_1(\C^{\fr}_2,\Sigma^{\pm})$ is the general action operad of \emph{unoriented ribbon braid groups}. One can identify unoriented ribbon braid groups in a similar manner to the case of plain ribbon braids, i.e.\  $\URBr_{k}\cong \Br_k\rtimes\, \mathbb{Z}^{\times k}$, but now $\mathbb{Z}^{\times k}$ accounts  for the number of half-twists on each ribbon. With such a description, the canonical morphism $\URBr\to \Sigma^{\pm}$ sends $(\beta_k;n_1,\dots,n_k)$ to $(\pi(\beta_k),A_{\underline{n}})$, where $\pi(\beta_k)$ is the underlying permutation of the braid $\beta_k$ and $A_{\underline{n}}$ is given by the set of $i$'s so that $n_i$ is odd. See Figure \ref{fig:FramedUnordConfAndRibbon} for an illustration of the difference between $\RBr$ and $\URBr$.

    \begin{figure}[h]
    \centering
    \includegraphics%[width=5cm]
    {figures/FramedUnordConfAndRibbon.pdf}
    \caption{From left to right, the operation $\sigma\circ_1\widehat{\tau}=(\sigma;1,0)$ in $\RBr_2$ and the operation $\sigma\circ_1\tau=(\sigma;1,0)$ in $\URBr_2$.}
    \label{fig:FramedUnordConfAndRibbon}
    \end{figure}
\end{examp}

By construction, there is a commutative diagram of general action operads
\begin{equation}\label{eq:braid_maps}
\begin{tikzcd}[ampersand replacement=\&]
   \Br \ar[rd, bend right=40, dashed]\ar[r, dashed] \& \RBr \ar[r] \ar[d] \& \URBr \ar[d] \&[-4mm] (\beta_k;n_1,\dots,n_k) \ar[r, mapsto]\ar[d,mapsto] \& (\beta_k;2n_1,\dots,2n_k)\ar[d,mapsto]\\
   \& \Sigma \ar[r] \& \Sigma^{\pm} \& \pi(\beta_k) \ar[r] \& (\pi(\beta_k),\emptyset)
\end{tikzcd}.
\end{equation}
Note that $\RBr$ is an action operad, while $\URBr$ equipped with the obvious projection $\URBr\to \Sigma$, $(\beta_{k};n_1,\dots,n_k)\mapsto \pi(\beta_{k})$ is just a general action operad since, for example, the square
$$
\begin{tikzcd}[ampersand replacement=\&]
\URBr_1\times \URBr_{2}\ar[r,"\circ_1"] \ar[d] \& \URBr_2\ar[d]\\
\Sigma_1\times \Sigma_2 \ar[r,"\circ_1"'] \& \Sigma_2
\end{tikzcd}
$$
does not commute. For instance, take $(\tau,e_2)=\big((e_1;1),(e_2;0,0)\big)\in \URBr_1\times \URBr_2$ and follow the two directions to obtain $e_2\neq (1,2)$ in $\Sigma_2$ (see Figure \ref{fig:Composition in twisted braid}).

\begin{rem}
Under the correspondence between general action operads and operadic cloning systems, the unordered ribbon braided operad $\URBr$ corresponds to the cloning system of twisted braid groups described in \cite[Example~4.2]{WZ}. 

In \cite[3.5.3]{Th}, Thumann introduces the following braided operad $\EE$: Consider first the free braided operad generated in arity $1$ by an operation $\tau$ and in arity $2$ by an operation $\sigma$. The operad $\EE$ is defined as the result of quotienting this operad by the relation $\tau \circ_1 e_2 = (\sigma\circ_1 \tau)\circ_2 \tau$. This operad is isomorphic to the underlying operad of $\URBr$, and the braid action comes from the maps \eqref{eq:braid_maps}: the operation $\tau$ is the half-twist of a single ribbon, while the operation $\sigma$ is the generator of the braid group. The relation is depicted in Figure \ref{fig:Composition in twisted braid}.

 \begin{figure}[h]
    \centering
    \includegraphics[width=6cm]{figures/Composition_in_twisted_braid.pdf}
    \caption{The relation $\tau\circ_1 e_2= (\sigma\circ_1 \tau)\circ_2\tau$ seen in the general action operad $\URBr$.}
    \label{fig:Composition in twisted braid}
    \end{figure}
\end{rem}




%The operad fundamental group of the operad $\OO$ considered as a $\Br$-operad will be different from the fundamental group of $\OO$ considered as a $\URBr$-operad, and therefore they will yield different Thompson groups. In fact, only the latter will be isomorphic to the Thompson group associated to the cloning system of twisted braid groups. \fc{tengo serias dudas sobre esto. Creo que las operaciones de aridad 1 borran la diferencia}

