\section{sparse-arrays}

\subsection{attributes}

Sparse-Arrays are meant to represent dense-arrays perfectly, with the assertion that elements which are zero are neglected\footnote{sometimes approximately zero, in which case the representation is approximate.}.
%, the map is given \S\ref{densesparse} and \S\ref{sparsedense}. 
The sparse-arrays may be broken into 2 primary parts: An 2-dimensional dense index-array and 1-dimensional dense data-array . %Dense referring to 
The sparse index-array is represented by:
\begin{align*}
    \text{Sparse Index-Array} \equiv A\left[ \quad | \quad \right]\quad\quad,
\end{align*}
with the vertical-line, $|$, separating this 2d array's two indices. 
With the left-side representing the dense-index columns, and the right-side indexing the enumeration of the sparse-tuple entry. In order to draw parallel with the dense-representation, the dense-indices are often shown explicitly included.
For example for a tensor $A$, with dense-indices $ijkl$, and sparse-index $I$ (enumerating all the non-zero values):
\begin{align*}
    \text{Dense  Representation} &\equiv A\left[ i, j, k, l \right] \\
    \text{Sparse Representation} &\equiv A\left[ i, j, k, l \,|\, I\, \right]\quad\quad.
\end{align*}
Notice, the dense representation does not have the $I$ (sparse) index.
Additionally, a small bit of information is the shape of the dense-representation of the array.
Therefore the complete attributes of the sparse-array include: shape-array ($s$), array-of-tuples/index-array ($A_\text{index}$), and the data-array ($A_\text{data}$). The full anatomy of a sparse-array is thus given in fig. \ref{sparsedecomp}.

%In addition to the data and index arrays, definitions of the sparse-tensor in both \cite{PyTorch} and \cite{tensorflow2015} define the \textit{dense-shape-array}. 



%\caution although sparse-tensors perfect mimetic dense-tensors, they have their own structure....
%An additional attribute included here is the sparse-tensor \textit{label}, this will give names to the axis/dense-index/columns of the sparse-tensor useful for partial-tracing.
 %and \cite{tensorflow2015}

\begin{figure}[h!]
\begin{center}
\begin{tikzpicture}
    \draw[<-] (-1, 0) -- (-1, 2);
    \draw[->] (-1, 3) -- (-1, 5.0);
    %\node(asd) at (-1.0 2.5)   (aa) {$I$};
    \node(draw) at (-1, 2.5)  (a) {$I$};
    \draw[<-] (0, -0.5) -- (1.5, -0.5);
    \draw[<-] (4, -0.5) -- (2.5, -0.5);
    \node(draw) at (2.0, -0.5)  (a) {$i$};
    
    \draw (0,0) -- (0,5) -- (4,5) -- ( 4,0) -- (0,0);
    \draw (4.5,5) -- (5.0,5) -- ( 5.0,0) -- ( 4.5,0) -- (4.5,5);
    \draw (0,5.25) -- (0,5.75) -- (4,5.75) -- ( 4,5.25) -- (0,5.25);

    \node(draw) at (2.00, 5.50)   (a) {shape};
    \node(draw) at (4.75, 2.50)   (a) {\rotatebox{270}{data}};
    \node(draw) at (2.00, 2.50)   (a) {index-array};
    
\end{tikzpicture}
\end{center}
\caption{\label{sparsedecomp} The decomposition of the sparse-array into a shape, array-of-tuples, and data arrays.}
\end{figure}

\subsubsection{shape}

The shape-array is a 1-dimensional array which gives the shape of the array in it's dense-representation. This is useful for the mapping between sparse-arrays and dense-arrays.

\subsubsection{indices}

The sparse-array index-array or \textit{indices} or \textit{array-of-tuples} is by convention lexicographically-ordered (ssort, by convention in ascending-order). This helps in identifying duplicates, and other sparse-array operations. 
The index-array captures the structure of the array.

\subsubsection{data}

For us the \textit{data-arrays} are 1-dimensional with a length matching the index-arrays, and being in 1-1 correspondence with the index-array. That is the list-index (of the index-array) matches the data-array index. These arrays are typically of \texttt{float} datatypes, and have no restriction on duplicates, with a slot-representation of $A_\text{data}\left[\quad\right]$.



\subsection{sparsity}

The sparsity is defined by (assuming $A_\text{index}$ is in a proper format, only filled with unique entries):
\begin{align*}
    \text{sparsity} &= \frac{\text{\text{total elements in proper sparse-tensor}}}{\text{total elements in dense-tensor}}\quad\quad \\
    &= \frac{\text{len}(A_\text{data})}{\prod_i \text{shape}[i]} \quad\quad.
\end{align*}




\subsection{well-ordering \label{clean}}

Suppose we have a ssorted index-array, $A_\text{index}\left[\!\left[ \quad | \quad \right]\!\right]$, and it's uniques array, $u$ (indicating the first appearance of a unique element in the sorted array). Then the duplicate elements are all the elements in the ssorted sparse-tensor between two successive unique elements, $u[i] : u[i+1]$ (not including $u[i+1]$). Therefore, we sum all these elements of $A\left[\!\left[ \quad | \quad \right]\!\right]$ for all successive-elements in $u$. This creates an array with unique data values for each unique index-tuple:
\begin{align*}
    A_\text{index}\left[\!\left[\!\left[ \quad | \quad \right]\!\right]\!\right] &= A_\text{index}\left[\!\left[ \quad \,\,|\,\,u \,\,\right]\!\right]\quad, \\
    A_\text{data}\left[ \quad \right] &= \sum_{i\in|u|} A_\text{data}\left[\,\, u[i] : u[i+1] \,\,\right]\quad\quad.
\end{align*}
Note $u[i+1] = |u|$ for the last unique-array's index, $i = |u|$.

%\subsubsection{Sparse-to-Dense array \label{sparsedense}}

%In order to convert the sparse-representation into the dense-representation. An array of shape is given
%In order to initialize the dense-array


%\subsubsection{Dense-to-Sparse array \label{densesparse}}
%Reshape Dense Tensor with default tuple-array, and remove entries with data entry of ``0'' (or within some tolerance).

%This algorithm necessitates a complete element-wise search, to find nonzero entries. And thus has a time-complexity of $\mathcal{O}\sim N^\eta$, with $\eta$ being the dimension of the dense-array.