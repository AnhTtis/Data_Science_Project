\documentclass[prb,preprint]{revtex4-2} 
\usepackage[utf8]{inputenc}  
\usepackage[T1]{fontenc}  
\usepackage{helvet}
\usepackage{amsmath}
\usepackage{graphicx}
\usepackage{amssymb} 
\usepackage{url}
\usepackage{braket}

\newcommand{\SO}{\mathrm{SO}}
\newcommand{\SU}{\mathrm{SU}}
\newcommand{\hspin}{spin-$1/2$ }

\begin{document}
%\title{The spinorial ball: a device to visualize half-spin rotations}
\title{The spinorial ball: a macroscopic object of spin-1/2}
\author{Samuel Bernard-Bernardet}
\affiliation{DotWave Lab, Chamb\'{e}ry, France}
\author{David Dumas}
\affiliation{University of Illinois at Chicago, USA}
\author{Benjamin Apffel}
\email{benjamin.apffel@epfl.ch}
\affiliation{Laboratory of Wave Engineering, EPFL, Lausanne, Switzerland}
\date{2022}
%Rouge = texte ajouté
%Bleu = phrase à enlever
\begin{abstract}
Historically, the observation of half-spin particles was one of the most surprising features of quantum mechanics.  They are often described as `objects that do not come back to their initial state after one turn but do after two turns'.
There are macroscopic implementations using constraints such as clamping a belt or ribbon that purport to show similar behavior (the ``Dirac belt trick'' ). However, a demonstration of an unconstrained macroscopic object with half-spin behavior remains elusive. In this article, we propose to fill this gap and introduce the \emph{spinorial ball}. It consists of a translucent plastic ball with internal LED illumination that behaves as a freely movable macroscopic half-spin object. It provides a new tool to introduce and visualize half-integer spins as well as the covering group homomorphism from $\SU(2)$ to $\SO(3)$, and offers in particular a clear visualization of the different homotopy classes of $\SO(3)$. We discuss its development and function, and how one can mimic quantum measurement and wave function collapse using this the spinorial ball. The entire system is open source hardware, with build details, models, 3d printing files, etc., provided under an open source license.
\end{abstract}
\maketitle
\section{Introduction}
\label{sec:intro}
\subsection{Scope and context}
The goal of this article is to introduce a new object (the spinorial ball) with the goal of providing insight on how half-integer spins behave. In particular, the spinorial ball can help to understand why half-spins are often considered as ``half-rotations'' or as ``an object that does not come back to its original state after one turn''. Although the authors have tried to make this presentation self-contained, it may be easier to read for those with basic knowledge of quantum mechanics, quantization of orbital angular momentum and \hspin. For instance \citep{basdevant_quantum_2005} provides a compact introduction and \citep{cohen-tannoudji_quantum_1986} offers comprehensive details. Some familiarity with group theory and the matrix exponential could also help for some technical aspects (see for instance \cite{appel_mathematics_2007}). Whenever possible we focus on the underlying concepts and physical intuitions, at times omit technical details for the sake of such focus.  We provide references for proofs that are omitted for this reason.

\subsection{The Stern-Gerlach experiment}
We first discuss the reason that quantized spin and half-integer spin were introduced in quantum mechanics during the 1920s. These concepts were used to explain experimental results that were inconsistent with classical angular momentum theory such as anomalous Zeeman effect and the Stern-Gerlach experiment \cite{gerlach_experimentelle_1922}. The latter consists of sending a beam of particles through a magnetic field gradient and measuring their deflection as depicted in figure \ref{fig:1}a. The historical experiment used neutral silver atoms on which no Lorentz forces $q \vec v \wedge \vec B$ apply. The atoms were prepared in their ground state, and experience a magnetic deflection primarily due to  interaction between the outermost (47$^{\mathrm{th}}$) electron and the field. For simplicity, we therefore consider a roughly equivalent thought experiment that uses a beam of electrons and discard the Lorentz force that would acts equally on all of them. Doing the experiment, one observes that the beam strikes a planar target in two separate spots, symmetrically arranged about the expected location, as in figure \ref{fig:1}d. The rest of this introduction aims to explain how quantum mechanics and half-integer spin  explain such results while classical physics fails to do so.

Classically, one can model the electron as a sphere of mass $m$ carrying a total charge $q$, which that can rotate around an axis $\vec n$ with angular velocity $\Omega$. As each part of the sphere is charged, its rotation will induce many small current loops that give rise to a net magnetic moment $\vec \mu$. The latter can be expressed as a function of the angular momentum  $\vec L \propto \Omega \vec n$ through the formula
\begin{equation}
\vec \mu = -\frac{q}{2m} \vec L 
\label{eq:muL}
\end{equation}

If one assumes that the magnetic field $\vec B$ depends only on the $z$ coordinate (for instance $\vec B = B(z) \vec y$), the force experienced by the sphere deviates from the Lorentz force $q\vec v \wedge \vec B$ on a non-spinning sphere of equal charge by
\begin{equation}
\vec F = -\vec \nabla (\vec \mu . \vec B)= - (\vec \mu . \vec \nabla) \vec B = -\mu_z \frac{\partial B}{\partial z} \vec y
\label{eq:forceB}
\end{equation}
with $\mu_z = \vec \mu . \vec z$ the projection of the magnetic moment on the $z$ axis. As no particular care is taken in the preparation of the electron beam, one would assume that the rotation axis $\vec n$ and rotation speed $\Omega$ of each electron would be random. From equation (\ref{eq:muL}), we deduce that $\vec \mu$ can point in any direction and have arbitrary norm. The deflection force (\ref{eq:forceB}) being proportional to the projection of the latter along the $\vec z$ axis, one would thus expect some continuous distribution of arrival locations of the beam. This is in strong contrast to the two spots observed experimentally. A solution to this contradiction was found through a quantum-mechanical formalism.

\subsection{Orbital and spin angular momentum in quantum mechanics}
Observing only two spots in the deflected beam suggests that some quantities in the problem can only take quantized values instead of continuous ones. Indeed, it turns out that the quantum-mechanical formalism predicts that the projection of the orbital momentum $\vec L$ along any axis is itself quantized. One can show that the wave functions associated to well-defined angular momentum projection along, say, the $\vec z$ axis are of the form 
\begin{equation}
 \Psi(r, \theta, \phi) = f(r, \theta) \times e^{-i \frac{L_z}{\hbar} \phi}
 \label{eq:waveFuncPhi}
\end{equation} with $\phi$ the azimuthal angle in spherical coordinates. This wave function needs to be single-valued under $\phi \rightarrow \phi + 2 \pi$, which constrains $L_z$ to be of the form $L_z = n \hbar$ with $n$ an integer. This is a significant improvement, as quantizing $\mu_z$ in (\ref{eq:forceB}) explains the discrete nature of the experimental results. However, it still fails to explain the precise results, as one would expect a spot in the center of the screen associated to $L_z = 0$ rather than two spots arranged on either side of the center.  
And as we will see below, taking $L_z$ to be a multiple of $\hbar$ ultimately leads to a prediction of an odd number of spots in total.

\begin{figure}
\centering
\includegraphics[width=15cm]{fig0}
\caption{(a) Sketch of the (simplified) Stern-Gerlach experiments : a beam of electrons is deviated by a magnetic field gradient resulting in two distinct spots (Lorentz force has been discarded here). The different theoretical predictions by respectively (b) classical, (c) orbital and (d) spin angular momentum are shown bellow. }
\label{fig:1}
\end{figure}

A solution to these further problems was found by looking at the problem from a more algebraic point of view. In quantum mechanics, angular momenta along ($\vec x, \vec y, \vec z)$ are associated to linear operators ($L_x, L_y, L_z$) that satisfy the following commutation relations
\begin{equation}
[L_x, L_y] = i \hbar L_z, \quad [L_y, L_z] = i \hbar L_x, \quad [L_z, L_x] = i \hbar L_y 
\end{equation}
These relations imply that the eigenvalues of $L_z$, which are the possible measurement results, will span either a set of the form $I_n = \{ -n, ..., -1, 0, 1, ..., n \} \times \hbar$ or of the form $S_n = \{-n + \frac{1}{2}, ..., -\frac{1}{2}, \frac{1}{2}, ..., n-\frac{1}{2} \} \times \hbar$. Case $I_n$ was discussed above, and can be interpreted as the quantum version of the rotating sphere with associated wave functions for the electron. Such angular momentum are referred to as \textit{orbital angular momentum}. The $2n + 1$ eigenvalues in this case correspond directly to a prediction of $2n+1$ spots on the screen, including one in the middle.

 The other types of sets $S_n$ appear more interesting for our problem. For example if one chooses $n=1$, there are two possible values $\{ - \frac{\hbar}{2}, \frac{\hbar}{2} \}$. Therefore, if one assumes that the relation (\ref{eq:muL}) still holds, the theory predicts exactly two spots, both offset from the center, as observed in the Stern-Gerlach experiments. However, those half-integer eigenvalues can not be associated to \textit{orbital momentum} nor can they be interpreted as physical rotational motion. Indeed the corresponding wave functions (\ref{eq:waveFuncPhi}) would be $\Psi \propto e^{\pm i \phi/2}$ which are ill-defined: $\phi \rightarrow \phi + 2 \pi$ gives $\Psi \rightarrow - \Psi$. Instead, this case must thus be interpreted using another type of angular momentum called \emph{spin angular momentum} (or simply \emph{spin}) and denoted ${\vec S}$. The spin is not associated to the physical rotation of an object, yet it possesses the algebraic properties of an orbital angular momentum. A magnetic momentum $\vec \mu$ is also associated to it through the formula $\vec \mu = \frac{g q}{2m} \vec S$ that is very similar to (\ref{eq:muL}) except that an extra dimensionless constant $g$ (the Land\'{e} factor) has been added.  Moreover, one can show that while conservation of angular momentum in quantum mechanics does not apply to $\vec L$ or $\vec S$ individually, it does apply to the sum $\vec L + \vec S$ \cite{thomson_modern_2013}. This provides evidence that spin and orbital momentum appear to have equivalent roles in quantum mechanics.
 
\subsection{Macroscopic \hspin}
Nevertheless, the existence of half integer angular momentum is puzzling for the intuition. In particular, if one thinks of it as the orbital momentum of a particular object, a rotation of $2\pi$ acts on the corresponding wave function as $\Psi_m \rightarrow - \Psi_m$. This would correspond to an object that does not come back on its original state after a rotation of angle $2 \pi$, but that does after a rotation of angle $4\pi$. There exist some macroscopic objects that exhibit such behavior due to external constraints. The classical \textit{Dirac belt trick}, for example, involves rotating an object that is attached to a wall by a belt, pair of strings, or ribbon as in Figure \ref{fig:1b}a (cf.~\cite[pp.~43-44]{penrose_rindler_1984}). After one complete rotation of the object about an axis, the belt acquires a twist that cannot be removed except through further rotation of the object.  But if one applies a second full turn to the object, the resulting double twist in the belt can be removed by applying a series of translations to the object without rotating it at all (see for instance videos   
\cite{HiseBeltTrickVideo}
\cite{alex_mason_httpswwwyoutubecomwatchvnat-esrextq_2008}). In a certain sense, this example exhibits behavior in which $2\pi$ rotation is not the identity while  $4\pi$ is, which is similar to \hspin. Although this system allows one to gain some intuition, and the belt can be seen as providing a mathematical model for spin angular momentum, the analogy between a free electron and the belt is not straightforward. In particular, in this model it is required that one side of the belt is fixed, breaking translational freedom, and the series of translations needed to restore the original state after a rotation by $4\pi$ has no direct analog in the quantum theory of \hspin.

Another system based on coupled mechanical pendulums has also been described \cite{leroy_simulating_2006, leroy_simulating_2010}, but the effective \hspin that emerges takes place in an abstract space of parameters rather than in the physical space of the demonstration. Therefore, a direct physical implementation of a half-integer spin is still missing. 

\begin{figure}
\centering
\includegraphics[width=16cm]{fig0bs}
\caption{(a) The \textit{Dirac belt} trick : when a belt fixed to a wall is rotated by one turn, one can not remove the twists. If a second turn is performed, all the twists can be removed without any additional rotation. (b) Picture (with removed background) of the spinorial ball. When the ball rotates, the colors displayed on the faces change continuously and come back to their original values after two turns (but not after one).}
\label{fig:1b}
\end{figure}

We propose to fill this gap by introducing the \emph{spinorial ball}.  While we primarily focus on its implementation as a manipulable electronic device, a mobile version \cite{spinPhone} (which works by rotating the phone) and a browser-based simulation \cite{noauthor_httpproxy-informatiquefrquball_nodate} are available as well. We briefly describe here its key characteristics in a qualitative way before exploring the details of its design, construction, and function in later sections. The spinorial ball is a polyhedron of roughly spherical shape, with LEDs illuminating each face (figure \ref{fig:1b}b) in a certain color. The face colors change smoothly as the ball is rotated, and while one full turn about an axis will not bring the LED colors back to their initial state, two full turns will do so (see figure \ref{fig:3}c for an example). This behavior is the same regardless of the initial orientation of the spinorial ball and works for any axis of rotation. In this way, the colors of the object behave like \hspin.

The goal of the article is to explain how the spinorial ball works. We first introduce the rotation group $\SO(3)$ to describe the possible rotation on the physical ball, and the group $\SU(2)$ that acts on spin angular momentum of \hspin particles. We then discuss the group homomorphism $T$ between $\SU(2)$ and $\SO(3)$ and its consequences. The homomorphism is at the heart of the spinorial ball, as it links the physical orientation of the ball (an element of $\SO(3)$) with its color evolution (an element of $\SU(2)$). Having introduced all the needed concepts, we come back in the last section to the spinorial ball and show how all essential features of \hspin naturally transcribe to this macroscopic object.

\section{Rotation group}
\subsection{Definition}
The possible rigid motions of an object, apart from translations, are elements of the \textit{rotation group} in $\mathbb{R}^3$ that will be denoted $\SO(3)$. Formally, it is defined as all the $3 \times 3$ real matrices (or linear transformations of $\mathbb{R}^3$) $R$ such that
\begin{itemize}
\item[(P1)] preserve the length of all vectors : $R^\dag R = \mathbb{I}_3$
\item[(P2)] preserve the orientation of space : $\det R = 1$
\end{itemize}

One can verify that $\SO(3)$ is a non-commutative group under the operation of matrix product (or composition of motions). In particular, the product and the inverse of a rotation is also a rotation. Moreover, one can show that any element $R \neq \mathbb{I}_3$ of $\SO(3)$ admits a unique (up to sign) unit vector $\vec n$, called its \textit{rotation axis}, such that $R \vec n = \vec n$. In the plane orthogonal to $\vec n$, one can moreover show that $R$ acts as two-dimensional rotation with an angle $\theta$. This means that all the elements of $\SO(3)$ correspond to rotations of space, justifying the name ``rotation group'' for $\SO(3)$. Conversely, any physical rotation of space with an angle $\Psi$ around the axis $\vec n$ can be encoded in an element of $\SO(3)$ that we write $R_{\vec n} (\Psi)$. For instance, the rotation of angle $\Psi$ around the $x$-axis (that is, taking $\vec n = \vec x$) is
\begin{equation}
R_{\vec x} (\Psi) = \begin{bmatrix}
1 && 0 && 0 \\
0 && \cos{\Psi} && -\sin{\Psi} \\
0 && \sin{\Psi} && \cos{\Psi} \\
\end{bmatrix}
\label{eq:rotz}
\end{equation}
which satisfies (P1) and (P2) and therefore belongs to $\SO(3)$.

\subsection{Generators of rotations}
There exists a convenient way to specify any rotation, i.e.~any element of $\SO(3)$. Let us consider first an infinitesimal rotation $d\Psi$ around one of the coordinate unit vectors $\vec x$, $\vec y$, $\vec z$. 
These rotation matrices can be written in the form $R_{\vec n}(d \Psi) = \mathbb{I}_3 - i\, d\Psi J_n$ with 
\begin{equation}
J_x = \begin{bmatrix}
0 && 0 && 0 \\
0 && 0 && -i \\
0 && i && 0 \\
\end{bmatrix}
\quad
J_y = \begin{bmatrix}
0 && 0 && i \\
0 && 0 && 0 \\
-i && 0 && 0 \\
\end{bmatrix}
\quad
J_z = \begin{bmatrix}
0 && -i && 0 \\
i && 0 && 0 \\
0 && 0 && 0  \quad
\end{bmatrix}
\end{equation}
More precisely, these expressions give the derivatives of $R_{\vec n}(\Psi)$ as a function of $\Psi$, evaluated at $\Psi=0$.  (The expression for $J_x$ follows directly from \eqref{eq:rotz} above.)

To generate a macroscopic rotation of $\Psi$ around $\vec x$, one can compose $N \gg 1$ rotations of $\Psi/N$, which leads to the expression $R_{\vec{x}}(\Psi) = \lim_{N \to\infty} \left( \mathbb{I}_3 - i\frac{\Psi}{N} J_x \right)^N = e^{-i \Psi J_x}$, the exponential being taken in the sense of matrices. In the same manner, two matrices $J_y$ and $J_z$ that can be used to generates the rotations around $\vec y$ and $\vec z$ respectively. More generally, any rotation of angle $\Psi$ around a unitary axis $\vec n$ can be written as 
\begin{equation}
R_{\vec n} (\Psi) = e^{-i\Psi \vec J \cdot \vec n}
\label{eq:expRot}
\end{equation}
where $\vec J = (J_x, J_y, J_z)$.
The three matrices $ -i \vec J = (-iJ_x, -iJ_y, -iJ_z)$ are thus enough to generate all the rotations through the exponential, and are called \textit{generators} of $\SO(3)$.
(In more mathematical terminology, the same matrices are called a basis of the Lie algebra $\mathfrak{so}(3)$.)

The generators satisfy the commutation relations :
\begin{equation}
[J_x, J_y] = i J_z, \quad [J_y, J_z] = i J_x, \quad [J_z, J_x] = i J_y 
\label{eq:comRot}
\end{equation}
which mean that up to a $\hbar$ factor, they can be identified with angular momentum operators in quantum mechanics. This is the angular counterpart of the fact that that linear momentum arises from the generators of the group of translations.

The rotation group provides a clear mathematical description of the ball's physical motion. We now aim to present the equivalent of rotations for \hspin.

\section{Rotation and spin}
\subsection{Spin-1/2 and $\SU(2)$}
\label{subsec:spinSU2}
In quantum mechanics, a \hspin is described as a unit vector $s$ in $\mathbb{C}^2$ that we denote by
\begin{equation}
s=a \ket{\uparrow} + b \ket{\downarrow}, \quad (a, b) \in \mathbb{C}^2, \quad |a|^2 + |b|^2 = 1
\end{equation}
where $\ket{\uparrow}$ and $\ket{\downarrow}$ are basis vectors corresponding to up and down spin states. In the following, any unit vector in $\mathbb{C}^2$ will be called a spinor. The spinorial ball maintains an internal state that is a spinor, which is displayed on the LED panels of the ball in a way that is described in the next section.

As in the case of the ball in $\mathbb{R}^3$, one can consider all the $2 \times 2$ complex matrices $S$ acting on $\mathbb{C}^2$ that satisfy the two-dimensional analogues of (P1) and (P2). This set of matrices is once again a group that is usually denoted $\SU(2)$, and is the analogue of rotations for spinors. After a bit of algebra, one can show that each element $S$ of $\SU(2)$ has the form
\begin{equation}
S = \begin{bmatrix}
a && -b^* \\
b && a^* \\
\end{bmatrix}, \quad (a, b) \in \mathbb{C}^2, \quad |a|^2 + |b|^2 = 1
\label{eq:SU2}
\end{equation} 
and conversely, any matrix of this form belongs to $\SU(2)$. Note that the first column of this matrix is a spinor, so we have a bijective correspondence between spinors and elements of $\SU(2)$.  This correspondence will be used latter, and justifies that we now focus on $\SU(2)$ to study \hspin. 

\subsection{Generators of $\SU(2)$}
As for rotations in $\SO(3)$, any element of $\SU(2)$ can be written in exponential form as
\begin{equation}
S_{\vec n} (\Psi) = e^{-i \Psi \vec S \cdot \vec n}
\label{eq:expSpin}
\end{equation}
where $\Psi$ is a real number, $\vec n$ is a unit vector of $\mathbb{R}^3$, and where we have defined three generators of $\SU(2)$, denoted $\vec S = (S_x, S_y, S_z)$, by
\begin{equation}
 S_x = \frac{1}{2} \begin{bmatrix}
0 && 1 \\
1 && 0 \\
\end{bmatrix}
\quad S_y = \frac{1}{2} \begin{bmatrix}
0 && -i \\
i && 0 \\
\end{bmatrix}
\quad S_z = \frac{1}{2} \begin{bmatrix}
1 && 0 \\
0 && -1 \\
\end{bmatrix}
\label{eq:genSU2}
\end{equation}
Up to a factor $1/2$, these are also the \emph{Pauli matrices} of quantum mechanics. They obey the commutation relations :
\begin{equation}
[S_x, S_y] = i S_z, \quad [S_y, S_z] = i S_x, \quad [S_z, S_x] = i S_y 
\label{eq:comSpin}
\end{equation}


The commutation relations (\ref{eq:comRot}) and (\ref{eq:comSpin}), combined with the formulas (\ref{eq:expRot}) and (\ref{eq:expSpin}) suggest a strong connection between $\SO(3)$ and $\SU(2)$. The next section describes the link between the two groups that is at the heart of the spinorial ball, that aims to relate physical rotation of the ball and the rotation of a spinor.

\subsection{Group homomorphism between $\SU(2)$ and $\SO(3)$}

The previous considerations strongly suggest a relation between $\SU(2)$ and $\SO(3)$ might involve identifying generators $S_x, S_y, S_z$ with $J_x, J_y, J_z$ respectively. The formulas (\ref{eq:expRot}) and (\ref{eq:expSpin}) suggest more generally that one can associate an element of $\SO(3)$ to an element of $\SU(2)$ through the map
\begin{equation}
\begin{array}{l|rcl}
T : & SU(2) & \longrightarrow & \SO(3) \\
    &S_{\vec n} (\Psi) =e^{-i \Psi \vec S \cdot \vec n} & \longmapsto & R_{\vec n} (\Psi) =e^{-i \Psi \vec J \cdot \vec n} \end{array}
\end{equation}

Indeed, this map is well-defined and has the important property that it is compatible with the group structure: for any $S_1, S_2$ in $\SU(2)$, one has $T(S_1 S_2) = T(S_1) T(S_2)$ in $\SO(3)$ \cite{appel_mathematics_2007}. In other words, taking the product of two elements in $\SU(2)$ and applying $T$ is the same as first applying $T$ to both of them and then taking the product in $\SO(3)$ (see figure \ref{fig:2}a). Due to this property, the map $T$ is said to be a \textit{group homomorphism} from $\SU(2)$ to $\SO(3)$.

However, $T$ can not be inverted as it is not injective: In $\SO(3)$, $R_{\vec n} (\Psi)$ is a rotation of $\Psi$ around $\vec n$. Therefore, if one chooses $\Psi  + 2 \pi$ instead of $\Psi$, we get the same rotation, meaning that (see for instance formula (\ref{eq:rotz}))
\begin{equation}
R_{\vec n}(\Psi+2\pi) = R_{\vec n} (\Psi) 
\end{equation}
However, this is not true of the parameterization of $\SU(2)$ introduced above. In fact, one can show that the Rodrigue-Euler formula holds
\begin{equation}
S_{\vec n} (\Psi) = \cos{\left( \frac{\Psi}{2} \right)} \mathbb{I}_2 - 2 i \sin{\left( \frac{\Psi}{2} \right) } \vec S \cdot \vec n 
\label{eq:RodrigueEuler}
\end{equation}
so that one has in particular
\begin{equation}
S_{\vec n} (\Psi + 2\pi) = -S_{\vec n} (\Psi).
\label{eq:minusSign}
\end{equation}
Thus $\pm S_{\vec n}(\Psi)$ are associated through $T$ to the same rotation
\begin{equation}
T(S_{\vec n} (\Psi + 2\pi)) = R_{\vec n}(\Psi+2\pi) = R_{\vec n} (\Psi) = T(S_{\vec n} (\Psi)).
\end{equation}
Therefore we see here that taking $\Psi \rightarrow \Psi + 2\pi$ has no effect on the rotation in $\SO(3)$ but changes the sign in $\SU(2)$. This is of course directly linked to the \hspin properties discussed in section \ref{sec:intro}. One can moreover show that each element of $\SO(3)$ arises from exactly two elements of $\SU(2)$, and $T$ is therefore called a \textit{double cover}.

Obtaining a non-invertible map is more than just an artifact of our construction; it reflects a genuine difference between the groups $\SU(2)$ and $\SO(3)$.  In fact, these two spaces are not homeomorphic, meaning that there is no continuous bijection between them. This is discussed further in \ref{subsec:pathSO3}.
% Continuity of inverse follows from compactness of SO(3)

\begin{figure}
\centering
\includegraphics[width=16cm]{fig1}
\caption{(a) The map $T$ is a group homomorphism: The image of a product is the product of images, i.e. $T(S_1)T(S_2) = T(S_1 S_2)$.  (b) The group homomorphism $T$ maps $\SU(2)$ onto $\SO(3)$, but each elements of $\SO(3)$ is the image of two elements in $\SU(2)$.  In fact, any continuous path of $\SO(3)$ is the image of two continuous paths in $\SU(2)$, its \emph{lifts}. (c) A loop $\Gamma_1$ in $\SO(3)$ which is continuously contractible to a point has a lift with the same property, but a non-contractible path (e.g. $\Gamma_2$) may list to an arc that joints two opposite points of $\SU(2)$.}
\label{fig:2}
\end{figure}

\subsection{Path lift}
\label{subsec:pathLift}
As each element of $\SO(3)$ admits two preimages, it is impossible to invert the map $T : \SU(2) \rightarrow \SO(3)$. Therefore, acting on the spinor by rotating the ball seems to be ill-defined. This is correct from a \textit{global} point of view but one can still perform \textit{local} inversion.

Furthermore, there is a distinguished way to perform such inversion for a path of rotations, as opposed to considering individual elements.  We define a path $\Gamma$ in $\SO(3)$ as a continuous function $\Gamma : t \in [0, 1] \rightarrow R(t) \in \SO(3)$, i.e.~a continuous one-parameter family of rotation matrices.  The elements $\Gamma(0)$ and $\Gamma(1)$ are the endpoints of that path. One can for instance consider $\Gamma(t) = R_{\vec n} (\Psi t)$ that goes continuously from $\Gamma(0) = \mathbb{I}_3$ to $\Gamma(1)=R_{\vec n} (\Psi)$. As discussed before, each element of such a path has two preimages $\pm S_{\vec n}(\Psi t)$ in $\SU(2)$. Moreover, there exist exactly two continuous paths in $\SU(2)$, namely $\{S_{\vec n} (\Psi t) \}_{0\leq t \leq 1}$ and $\{ -S_{\vec n} (\Psi t) \}_{0\leq t \leq 1}$, that map to $\Gamma$ under $T$ as shown in figure \ref{fig:2}b. Those paths go from $\mathbb{I}_2$ to $S_{\vec n}(\Psi)$ and from $-\mathbb{I}_2$ to $-S_{\vec n}(\Psi)$ respectively. If one specifies $\mathbb{I}_2$ as the starting point in $\SU(2)$, then there exists only one way to continuously lift a path from $\SO(3)$ to $\SU(2)$. In other words, there exists a unique continuous way to associate a preimage of \textit{each} element of the path, once the preimage of \textit{one} element of the path is chosen. 

One can generalize this result and show that any continuous path in $\SO(3)$ is associated to two continuous path in $\SU(2)$. As before, once the preimage of the starting point is chosen, there is unique way to continuously lift the rest of the path. This will be the key ingredient to realize a spinorial ball, as it provides a recipe to go from the physical rotation of the ball to an action on the spinor.

\subsection{Homotopy classes in $\SO(3)$}
\label{subsec:pathSO3}
Before coming to the spinorial ball, we briefly discuss how the lifting of paths from $\SO(3)$ to $\SU(2)$ gives some insight into why these spaces are topologically different. For this we will consider \textit{loops}, i.e.~paths with the same starting and ending point: $\Gamma(0) = \Gamma(1)$. We then define the \textit{homotopy classes} as collections of loops that can be continuously deformed to one to another. For instance, in the plane $\mathbb{R}^2$ any loop can be continuously deformed into any another, and there is therefore only one homotopy class.  But as we will see, $\SO(3)$ and $\SU(2)$ have different numbers of homotopy classes.

Consider as in figure \ref{fig:2}c two loops $\Gamma_1$ and $\Gamma_2$ in $\SO(3)$ going from $\mathbb{I}_3$ to $\mathbb{I}_3$. When those loops are lifted in $\SU(2)$ as $\gamma_1$ and $\gamma_2$ starting from $\mathbb{I}_2$, all we can conclude from the construction is that the endpoints $\gamma_1(1)$ and $\gamma_2(1)$ are both $T$-preimages of $\mathbb{I}_3$.
It may happen that one of these lifted paths is a loop, i.e.~$\gamma_1(1) = \mathbb{I}_2$, while the other is an arc joining $\pm \mathbb{I}_2$, i.e.~$\gamma_2(1) = -\mathbb{I}_2$.
Both of these possibilities actually occur for loops in $\SO(3)$, and  making a continuous deformation of a loop in $\SO(3)$ does not change which behavior is seen.
Thus there are at least two types of loops in $\SO(3)$: the ones that remain loops when lifted to $\SU(2)$, and the ones that open up to arcs from $\mathbb{I}_2$ to $-\mathbb{I}_2$. 

Therefore, $\SO(3)$ admits at least two homotopy classes, and one can show that there are in fact exactly two. On the other hand, $\SU(2)$ is the unit sphere of $\mathbb{C}^2$ (see (\ref{eq:SU2})), which is equivalently the unit sphere of $\mathbb{R}^4$.
Arguments similar to the one sketched for $\mathbb{R}^2$ above can be used to show that the unit sphere in $\mathbb{R}^{4}$ has a single homotopy class.

From here, it can be seen that there is no bijective continuous map from $\SU(2)$ to $\SO(3)$, as this would ultimately imply that they have the same number of homotopy classes. More details and comprehensive proofs can be found in e.g.~\cite{appel_mathematics_2007}.


\section{The spinorial ball}

\subsection{General principle}
\label{subsec:generalPrinc}
Our physical realization of an object exhibiting \hspin behavior consists of a truncated icosahedron, which is a polyhedron with 20 hexagonal faces and 12 pentagonal faces.  Each face is a translucent LED panel.
At any time, the hexagonal faces are all illuminated in the same solid color, and the same is true of the pentagonal faces.  Thus, the object displays two colors at any time. This may seem an unusual choice for an object that is meant to be rotated, as it is difficult to visually discern the orientation of a roughly spherical object with such a symmetric pattern.  However this is by design: \emph{The spinorial ball displays its orientation through the two colors of its faces.} 

More than that, the ball displays a spinor:  A \hspin state $(a, b) \in \mathbb{C}^2$ is represented by displaying the complex number $a$ on the pentagons and $b$ on the hexagons. Complex numbers are encoded as colors using saturation and hue (see color map and examples on figure \ref{fig:3}a). The ball is equipped with an electronic gyroscope to continuously track changes to its orientation.  It therefore knows the path of its rotations in $\SO(3)$, and lifts that path continuously to $\SU(2)$ in real time so it can display the effect on the spinor as changes to the LED 
colors.

To make this more precise, let us assume that we initialize the ball at $t = 0$ so its LED panels display the spinor $(1,0)$ (figure \ref{fig:3}c). After each time step $dt$, the gyroscope returns the current rotation matrix $R_c \in \SO(3)$ that describes how the object has been rotated since its initialization. Thus for example, if the ball has been stationary since startup, the sensor will return $\mathbb{I}_3$ after each time step. If we then start to rotate the ball and we measure the new rotation matrix after each step $dt$, we can represent each of the rotation matrices thus obtained as a small change to the previous one: From time $(k-1)\,dt$ to $k \,dt$, $R_c$ changes by an additional rotation by $\delta_k$ around $\vec{n}_k$. Thus the matrices returned by the gyroscopes are given by the time series
\begin{equation}
\begin{array}{c|ccccc}
   \textrm{Time} \quad  & 0 & dt & 2 dt & ... &  n dt \\ 
   \hline
  R_c & \mathbb{I}_3 & \quad R_{\vec{n}_1} (\delta_1)\quad &  \quad R_{\vec{n}_2} (\delta_2 )R_{\vec{n}_1} (\delta_1) \quad& ... & \quad \prod_{k=1}^n R_{\vec{n}_k} (\delta_k) \quad \\
\end{array}
\end{equation}
As the time step $dt$ is quite small, this series is a discrete analogue of a continuous path in $\SO(3)$.
Just as we did with continuous paths, this path can be lifted to two paths in $\SU(2)$:
\begin{equation}
\begin{array}{c|ccccc}
   \textrm{Time} \quad  & 0 & dt & 2 dt & ... &  n dt \\ 
   \hline
  S_c & \mathbb{I}_2 & \quad S_{\vec{n}_1} (\delta_1)\quad &  \quad S_{\vec{n}_2} (\delta_2 )S_{\vec{n}_1} (\delta_1) \quad& ... & \quad \prod_{k=1}^n S_{\vec{n}_k} (\delta_k) \quad \\
 & -\mathbb{I}_2 & \quad -S_{\vec{n}_1} (\delta_1)\quad &  \quad -S_{\vec{n}_2} (\delta_2 )S_{\vec{n}_1} (\delta_1) \quad& ... & \quad - \prod_{k=1}^n S_{\vec{n}_k} (\delta_k) \quad \\
\end{array}
\label{eq:pathSU2}
\end{equation}

That is, once the ball is initialized (corresponding to choosing $\pm \mathbb{I}_2$ as a starting point), there exists a unique continuous way to transport the physical rotation of the ball to a path in $\SU(2)$. Once found, the element $S$ of $\SU(2)$ is mapped on its associated spinor $s=(a,b)$ (section \ref{subsec:spinSU2}) and displayed on the LED panels as illustrated in figure \ref{fig:3}b.

\begin{figure}
\centering
\includegraphics[width=15cm]{fig2}
\caption{(a) We encode a complex number using the saturation (for the modulus) and the hue (for the argument) of colors. (b) Three examples of $\SU(2)$ elements represented using two color panels and corresponding pictures of the ball. (c) Evolution of the spinorial ball starting from the state $s_0 = \ket \uparrow$ after a rotation of respectively one turn ($2 \pi)$ around $\vec z$, two turn ($4 \pi)$ around $\vec z$, half-turn ($\pi)$ around $\vec y$ and quarter-turn ($\pi/2$) around $\vec x$.}
\label{fig:3}
\end{figure}

While motivated by \hspin physics, the spinorial ball can equivalently be seen  as a physical model of path lifting from $\SO(3)$ to $\SU(2)$. The LED panel shows the evolution of the spinor along the path in $\SU(2)$ when the physical ball is transformed according to a path in $\SO(3)$.

As indicated above, the actual orientation of the ball (relative to its starting orientation) is also ``visible'' at all times: The colors of the LED panels indicate an element of $\SU(2)$, and the rotation the object has undergone is the image of that element by $T$.

\subsection{Practical implementation}

The panels of the ball are 3D printed in translucent plastic and glued together in order to form a truncated icosahedron. An RGB LED is placed on the inner side of each panel.  The inner cavity of the ball contains an a battery, gyroscope (Bosch BNO055), and an Arduino-compatible microcontroller board to which the gyroscope and all of the LEDs are connected.
As is common in designs with many RGB LEDs, we use LED modules with integrated logic (WS8211 ICs) allowing them to receive color data over a serial bus, so all of the LEDs to be controlled with just three pins of the microcontroller.
It turns out that for practical reasons (computation speed or interpolation purposes, for instance), many commercial gyroscopes encode rotations directly with elements of $\SU(2)$ or equivalently through quaternions. Our discussion in section \ref{subsec:generalPrinc} is therefore slightly incorrect, as our gyroscope does not return after each time step the rotation matrix $R$ but one of the two associated elements $\pm S$ of $\SU(2)$. 
In a sense this means the sensor data is more direct---as it provides a spinor---but the sign of the spinor that is returned may change discontinuously.  The firmware of the spinball therefore chooses to use the spinor directly or multiply it by $-1$ according to which is closer to its previous state.  It then sends new color data to the LEDs to display the updated spinor.  All details, including the schematics, code, and 3D printing models needed to build a spinorial ball are available at \cite{gitSam}, and are free to use for any research or teaching activity.

\subsection{Classical examples}
We now give several examples of the spinorial ball evolution, the starting point always being $S_0 = \mathbb{I}_2$ corresponding to spinor $s_0 = \ket \uparrow$. The corresponding pictures of the spinorial ball after each transformation are shown in figure \ref{fig:3}c. 

\begin{itemize}
\item A full turn around the $\vec z$ axis (or any other axis) has the effect of moving in $\SU(2)$ from $\mathbb{I}_2$ to $-\mathbb{I}_2$, corresponding to the spinor transformation $\ket \uparrow \rightarrow - \ket \uparrow$.  The colors of the pentagonal and hexagonal faces are each inverted, i.e.~replaced with opposite hues of equal saturation.
\item Two full turns around $\vec z$ (or any other axis) restores the colors to their original state.

\item When one rotates by an angle $\phi$ around the $\vec z$ axis, one obtains spinor $\ket \uparrow \rightarrow e^{-i \phi/2} \ket \uparrow$. Although it is usually discarded in quantum mechanics, the phase factor $e^{-i \phi/2}$ can be observed on the spinorial ball.
\item A half-turn around the $\vec y$ axis corresponds to a path from $\mathbb{I}_2$ to $-2iS_{y}$, corresponding to $\ket \uparrow \rightarrow \ket \downarrow$. In other words, when the ball is turned upside down, the spinor goes from up-state to down-state.
\item A quarter-turn around the $\vec x$  axis corresponds to a path in $\SU(2)$ from $\mathbb{I}_2$ to $\frac{1}{\sqrt{2}} \left ( \mathbb{I}_2 -2i S_x \right )$ corresponding to $\ket \uparrow \rightarrow  \frac{1}{\sqrt{2}} \left ( \ket \uparrow - i \ket \downarrow \right )$
\end{itemize}

The interested reader can manipulate a virtual version of the ball on our browser-based simulation using WebGL-based 3D graphics \citep{noauthor_httpproxy-informatiquefrquball_nodate}. A mobile version that changes along the phone's rotation and displays the corresponding spinor is also available \cite{spinPhone}.

\subsection{Visualizing homotopy classes with the ball}
The homotopy classes of $\SO(3)$ can also be seen with the spinorial ball. A loop in $\SO(3)$ is a rotation process that results in the same orientation of the physical ball at the end as at the beginning.
If the colors of the hexagons and pentagons displayed at the end and the beginning are the same, then the path in $\SO(3)$ lifts as a loop $\mathbb{I}_2 \rightarrow \mathbb{I}_2$ in $\SU(2)$.  But if the colors change (and are opposite according to the color map in figure \ref{fig:3}), the path lifts as an arc $\mathbb{I}_2 \rightarrow -\mathbb{I}_2$.  Equivalently, in terms of spins, if the spinorial ball is initialized in $\ket \uparrow$ state, then a loop of rotations corresponds to either $\ket \uparrow \rightarrow \ket \uparrow$ or $\ket \uparrow \rightarrow - \ket \uparrow$ depending on the homotopy class of the physical rotation that has been applied. The spinorial ball can therefore be used as a homotopy class detector, similar to the Dirac belt but without any tether or constraint on its motion.

Note that the previous remark also explains why a spin-1/2 must come back to its initial state after two turns. After one turn, the path in $SU(2)$ is $\mathbb{I}_2 \rightarrow -\mathbb{I}_2$. But if one performs a second turn, the corresponding path will be another arc $-\mathbb{I}_2 \rightarrow \mathbb{I}_2$. When chained, those two arcs therefore form a closed loop $\mathbb{I}_2 \rightarrow \mathbb{I}_2$ and the spin comes back to its initial state. 

\subsection{Multiply elements of $\SU(2)$ with the ball}
\label{subsec:multiplySU2}
The spinorial ball can also be used to multiply the elements of $\SU(2)$. Let us assume that at a given time, the current rotation matrix is $R_c$ in $\SO(3)$, associated to $S_c$ in $\SU(2)$. We will now apply a small rotation $R$ associated to $\pm S$ in $\SU(2)$. The new matrix is $R \times R_c$ and as in equation (\ref{eq:pathSU2}), the new matrix in $\SU(2)$ is then $\pm S \times S_c$.  Here we have used that $T$ is a group homomorphism:  if it were not, lifting $R \times R_c$ might differ from lifting $R$ as $S$ and $R_c$ as $S_c$ separately and then multiplying them in $\SU(2)$.

We still need to pick a ``good'' sign for the new element of $\SU(2)$. As $R$ is close to $\mathbb{I}_3$, one of its lifts $S$ is close to $\mathbb{I}_2$ and the other ($-S$) is close from $-\mathbb{I}_2$. Since we look for continuous evolution in $\SU(2)$, the natural choice is thus $S \times S_c$, i.e.~to make $S_c$ change by as little as possible while remaining a lift of $R_c$.  If the ball were turned by a very large angle in a very short time, it might not be possible to decide which lift of $R$ to use; however, the frequent readings taken by the gyroscope prevents this from happening in practice.

Thus we see that rotating the ball corresponds to multiply two elements of $\SU(2)$. For instance, rotating the ball of 180 degree around the $\vec x$ axis corresponds to multiplying the current element by $S = -2i S_x$ while a complete turn around any axis corresponds to a multiplication by $-\mathbb{I}_2$, what is consistent with previous examples. 

Finally, we mention that $\SU(2)$ can also be identified with the set of unit elements of the field of quaternions. Therefore, the ball can also be used to multiply unitary quaternions. In particular, the multiplication table of the quaternion group (a group of 8 unit quaternions) can be visualized by applying sequences of 180 degree rotations around the axis $\vec x, \vec y, \vec z$.

\subsection{Simulating measurement and wave-function collapse}
As a last application, we describe how the spinorial ball can simulate a quantum measurement. In quantum mechanics, if one performs a measurement on the state $s = a \ket \uparrow +  b \ket \downarrow$, the results is $\hbar/2$ (resp. $-\hbar/2$) with probability $|\braket{+|s}|^2=|a|^2$ (resp. $|\braket{-|s}|^2=|b|^2$). After the measurement, the spin state is either $\ket \uparrow$ or $\ket \downarrow$ depending on the result. That is, the state ``collapses'' from the spinor $s$ to one of the two eigenstates (an instance of so-called \emph{wave-function collapse}).

This can be implemented on the spinorial ball using an external button or other triggering mechanism that corresponds to a quantum measurement being taken. When that happens, if the ball is in a state given by spinor $s$, we compute the probability $p=|\braket{+|s}|^2$  and pick a uniform random number $t$ between $0$ and $1$. If $t < p$, the display on the LED panel is reinitialized in the state $\ket \uparrow$, while if $t \geq p$ it is reinitialized in the state $\ket \downarrow$. In doing so, the spinorial ball mimics the corresponding wave function collapse.

\section{Conclusion and perspectives}
In this article, we have described a macroscopic object that is able to move freely while exhibiting the essential characteristics of \hspin. The object exhibits the connection between $\SU(2)$ and $\SO(3)$ in a visual way and shows how paths can be lifted from the latter to the former once an initial point is fixed. 

This type of object suggests several avenues for further work.  First, it could be extended to any integer or half-integer spin by increasing the number of colored tiles on the ball and modifying the way they transform under rotation according to the spin matrices in the corresponding representation spaces. Also, we discussed in the last section how one could emulate quantum measurement for a single spin. It would be interesting to extend this to the case of \emph{two} macroscopic \hspin objects.  Communication between their driving electronics might be used to simulate entanglement.  Perhaps one could simulate the failure in Bell's inequalities in this way.

We also believe that this device could be used to popularize quantum mechanics for undergraduate students, as a practical exercise for students in electronics or engineering, or simply as an interesting toy that offers the possibility of opening a gateway to interesting mathematics and physics for any careful observer.
\section{Acknowledgment}
The authors would like to acknowledge Emmanuel Fort, Tony Jin, David Martin and Marc Abboud for insightful discussions and feedback.  Parts of this project were developed during a semester program at the Institute for Computational and Experimental Research in Mathematics (ICERM) at Brown University, and at ``Les Gustins'' Summer School with support of Jean Baud and Ingénieurs et Scientifiques de France - Sillon Alpin (IESF-SA). The authors thank these organizations and acknowledge attendees of these programs for stimulating discussions. The authors have no conflicts to disclose.
\bibliographystyle{unsrt}
\bibliography{biblio}
\end{document}
