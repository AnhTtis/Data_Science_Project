\documentclass[conference]{IEEEtran}
\IEEEoverridecommandlockouts
% The preceding line is only needed to identify funding in the first footnote. If that is unneeded, please comment it out.
\usepackage{cite}
\usepackage{amsmath,amssymb,amsfonts}
\usepackage{algorithmic}
\usepackage{graphicx}
\usepackage{textcomp}
\usepackage{xcolor}
\usepackage{hyperref}
%\usepackage[square,numbers]{natbib}


\newcommand{\norm}[1]{\left\lVert#1\right\rVert}

\newtheorem{theorem}{Theorem}[section]
\newtheorem{proposition}{Proposition}[section]
\newtheorem{corollary}{Corollary}[theorem]
\newtheorem{lemma}[theorem]{Lemma}
\newtheorem{definition}{Definition}[section]
\newcommand{\opL}{\mathcal L}
\newcommand{\opR}{\mathcal R}
\renewcommand{\Re}{\operatorname{Re}}
\renewcommand{\Im}{\operatorname{Im}}
\newcommand{\mean}[1]{\mathbb E#1}
\newcommand{\cov}{\text{Cov}}
\newcommand{\trace}[1]{\text{Tr}\left[#1\right]}
\newcommand{\dd}[1]{\mathrm{d}#1}
\newcommand{\traceLim}[2][17.5]{ \text{Tr}_{#1}\left[#2\right] }

\def\BibTeX{{\rm B\kern-.05em{\sc i\kern-.025em b}\kern-.08em
    T\kern-.1667em\lower.7ex\hbox{E}\kern-.125emX}}
\begin{document}



\title{Gradient flow on extensive-rank positive semi-definite matrix denoising}

%\author{\IEEEauthorblockN{1\textsuperscript{st} Antoine Bodin}
\author{\IEEEauthorblockN{Antoine Bodin and Nicolas Macris}
\IEEEauthorblockA{
School of Computer and Communication Sciences\\
SMILS - IC - Ecole Polytechnique F\'ed\'erale de Lausanne\\
%\textit{Ecole Polytechnique F\'ed\'erale de Lausanne}\\
antoine.bodin@epfl.ch and nicolas.macris@epfl.ch}
%\and
%\IEEEauthorblockN{1\textsuperscript{st} Nicolas Macris}
%\IEEEauthorblockA{Communication Theory Laboratory \\
%School of Computer and Communication Sciences,\\
%\textit{Ecole Polytechnique F\'ed\'erale de Lausanne}\\
%nicolas.macris@epfl.ch}
}

\maketitle

\begin{abstract}
In this work, we present a new approach to analyze the gradient flow for a positive semi-definite matrix denoising problem in an extensive-rank and high-dimensional regime. We use recent linear pencil techniques of random matrix theory to derive fixed point equations which track the complete time evolution of the matrix-mean-square-error of the problem. The predictions of the resulting fixed point equations are validated by numerical experiments. In this short note we briefly illustrate a few predictions of our formalism by way of examples, and in particular we uncover continuous phase transitions in the extensive-rank and high-dimensional regime, which connect to the classical phase transitions of the low-rank problem in the appropriate limit. The formalism has much wider applicability than shown in this communication.
\end{abstract}

\begin{IEEEkeywords}
Random Matrix Theory, Linear Pencils, Gradient Flow, Matrix Denoising, Extensive-Rank, Phase transitions
\end{IEEEkeywords}

\section{Introduction}

Matrix denoising and factorization play a crucial role in a variety of data science tasks such as matrix sensing, phase retrieval or synchronisation, or matrix completion. The problem consists in reducing the amount of noise or irrelevant information present in a dataset, allowing for more accurate analysis and interpretation of the data, as well as better computational efficiency and modeling by way of dimensionality reduction. The literature on the subject is immense and we refer to \cite{ChenChi, chi2019nonconvex} for recent overviews of applications and theory in various settings and formulations.

In this contribution we focus on the study of gradient-flow for the following statistical formulation for positive definite matrix denoising. We consider a "ground truth" signal $X^* \in \mathbb R^{n \times d}$ with randomly sampled independent entries $X^*_{ij} \sim \mathcal N(0,\frac{1}{n})$ where the dimensions $n,d$ are  such that $\phi = \frac{d}{n}$ is fixed. Then we define the corrupted data matrix $Y \in \mathbb R^{n \times n}$
\begin{equation}\label{model1}
    Y = X^* X^{*T} + \frac{1}{\sqrt{\lambda}} \xi
\end{equation}
where $\xi$ is an additive symmetric random noise with $\xi_{ij} = \xi_{ji} \sim \mathcal N(0,\frac{1}{n})$ and $\lambda$ is (proportional to) the signal-to-noise ratio. The objective is to estimate the ground truth positive semi-definite matrix $X^* X^{*T}$ from the corrupted data matrix $Y$ with a matrix $XX^T$ such that $X \in \mathbb R^{n \times m}$ where $m$ is set from the fixed ratio $\psi = \frac{m}{n}$. Note that we allow $d$ and $m$ to be different. 
The estimator studied in this contribution is given by the gradient flow $X(t)$ ($t$ is  time) for an objective function with regularization parameter $\mu$, defined as 
\begin{equation}
    \mathcal H(X) = \frac{1}{4 d} \norm{Y - X X^{T}}_{F}^{2} + \frac{\mu}{2d} \norm{X}_{F}^{2} 
\end{equation}
where $\norm{ \cdot }_F$ is the Frobenius norm. The initialization of gradient flow is $X(0)= X_0 \in \mathbb R^{n \times m}$ random with i.i.d matrix elements from  
$\mathcal N(0,\frac{1}{n})$.
As a measure of performance we adopt the expected matrix-mean-square-error 
\begin{equation}
\mathbb{E}\mathcal E = \frac{1}{d} \mathbb{E}\norm{X^* X^{*T} - X X^{T}}_{F}^{2} 
\end{equation}
where the expectation is over $\xi$, $X^*$, $X_0$.
Note that the objective function and performance measure are not the same and can be thought of as "training" and "generalization" errors in the language of machine learning.

{\bf  Summary of main contributions}: 
\begin{itemize}
\item
We derive a set of analytical fixed point equations whose solutions allow to compute the full performance curve $t \to \mathbb{E}\mathcal{E}_t$ 
for the extensive-rank and high-dimensional regime where $m, d, n$ all tend to infinity while $\phi, \psi$ are kept fixed (results 1 and 2 in Sec. \ref{sec:results}). Continuous time average behaviour of gradient flow is a proxy for the usual discrete gradient descent algorithm, and has the advantage that it is more amenable to analytical study. The numerical experiments confirm that (a) $\mathcal{E}_t$ concentrates over its expectation; (b) theoretical predictions of gradient flow agree with gradient descent. See Fig. \ref{fig:graph1}.
\item
We further push the analysis of these equations in the time limit $t=+\infty$ and display specific examples where a critical value $\lambda_c$ can be calculated such that: (a) for $\lambda\leq\lambda_c$ the performance error of gradient flow is no better than the one of the null-estimator $X=0$; (b) for $\lambda>\lambda_c$ better estimation is possible; (c) the phase transition between the two regimes is a continuous type phase transition. These results are displayed on Fig. \ref{fig:graph3}.
\item
We analyze the limit $\phi = \psi \to 0$ (after $n, m, d$ have been sent to infinity) and derive a connection with the low-rank setting. It turns out that the matrix-mean-square-error curve (when $t\to +\infty$) tends to the one of the rank-one problem and the phase transition reduces to the well known BBP transition at $\lambda_c=1$. 
\end{itemize}
 
We use tools based on modern results in random matrix theory. Central to our derivations, is the formalism of \emph{linear-pencils}, that initially appeared in \cite{spectra, mingo2017free} and has been further improved recently in the context of neural networks \cite{adlam2020neural, bodin2021model, bodin.2212.06757}. In particular we make use of extensions provided in \cite{bodin.2212.06757} to derive closed-form expressions of non-trivial averages over $\xi$, $X^*$, $X_0$ appearing in traces of complicated "rational" expressions of these random matrices. Although these techniques have not yet always been rigorously proven they have been used successfully in different applications, and the predictions are confirmed by numerical experiments. In addition, we use holomorphic functional calculus for matrices \cite{schwartz1958linear}.

{\bf  Brief review of literature}:
The full time-evolution of gradient flow for the rank-one problem (the so-called spiked Wigner model with $d=m=1$) has been solved and rigorously analyzed in much the same spirit as the present work in \cite{pmlr-v134-bodin21a} with the difference that the spike is constrained to lie on a sphere all along the evolution. 
%The {\it limiting} performance for $t\to +\infty$ is the same as the one where the spherical constraint is relaxed to a ridge term (like in \eqref{model1}) with $\mu= \frac{1}{\lambda}$. 
For the present extensive-rank setting rigorous or even analytical results on the whole time-evolution are scarce. Closely connected to our work is the recent paper \cite{tarmoun2021understanding}. An essential difference however is that in \cite{tarmoun2021understanding} the initialization $X(0)= X_0$ is taken to have eigenvectors aligned with those of $Y$ (this pre-processing can be implemented empirically in practice). Moreover the authors do not carry out the random matrix averages fully analytically. Gradient flow has been studied in a variety of settings more or less related to the present one, see  \cite{gunasekar2017implicit, chou2020gradient, saxe2013exact, mannelli2019passed, BenArous-et-al-2022, Liang-Sen-Sur-2022}.

%In contrast with former work , the matrix $X(t=0)=X_0$ is initialized randomly with $[X_0]_{ij} \sim \mathcal N(0,\frac{1}{n})$ so that the eigenvectors of $X_0 X_0^T$ have not been preprocessed to be aligned with those of $Y$.

Bayesian approaches are quite well understood for the low-rank problem (mainly rank-one). This context is quite different from the present one. To begin with it is not dynamical. One studies the Minimum-Mean-Square-Estimator (MMSE) computed as the conditional expectation of the signal with respect to the Bayesian posterior probability distribution \cite{Montanari-Richard-2014, lelarge2019fundamental, luneau2020high, barbier2019adaptive, miolane2017fundamental, pourkamali2021mismatched, pourkamali2022mismatched, CamilliContucciMingione, barbier2022price}. Bayesian-optimal as well as mismatched estimation settings have been well studied with rigorous results on the mutual information, the MMSE, the cross-entropy, and the problem displays a rich phenomenology of first and higher order phase transitions depending on the nature of the priors. Related dynamics of the Approximate Message Passing (AMP) algorithms is also well understood for these problems \cite{DBLP:conf/isit/LesieurMLKZ17, Lesieur_2017, Montanari2017EstimationOL} . The realm of extensive-rank within such Bayesian and AMP approaches is quite open and very timely \cite{Kabashima_2016, barbier2021statistical, maillard2021perturbative, troiani2022optimal, camilli2022matrix}.

Finally other types of non-dynamical approach belong to the class of spectral methods like Principal Component Analysis (PCA).  The low rank case is covered by \cite{baik2005phase, Pch2004TheLE, benaych2011eigenvalues}. 
For the extensive-rank setting the results are scarce and little is known except for ensembles of rotation invariant signals for which an interesting class of Rotation Invariant Estimators (RIE) has been proposed \cite{bun2017cleaning}.

  
%In this work, we consider a planted signal $X^* \in \mathbb R^{n \times d}$ randomly sampled from an independent entries such that $X^*_{ij} \sim \mathcal N(0,\frac{1}{n})$ where the dimensions $n,d$ are  such that $\phi = \frac{d}{n}$ is fixed. Then we define the a symmetric random noise $\xi$ with $\xi_{ij} = \xi_{ji} \sim \mathcal N(0,\frac{1}{n})$ and the corrupted data matrix $Y \in \mathbb R^{n \times n}$ with signal-to-noise ratio $\lambda$. as follows:
%\begin{equation}
%    Y = X^* X^{*T} + \frac{1}{\sqrt{\lambda}} \xi
%\end{equation}
%The objective of this work is to estimate the planted signal $X^* X^{*T}$ from the corrupted data matrix $Y$ with a matrix $XX^T$ such that $X \in \mathbb R^{n \times m}$ where $m$ is set from the fixed ratio $\psi = \frac{m}{n}$. This estimation is realized by the minimization with the gradient-flow algorithm of an objective function (or the training error) with a regularization parameter $\mu$ defined as follows:
%\begin{equation}
%    \mathcal H(X) = \frac{1}{4 d} \norm{Y - X X^{T}}_{F}^{2} + \frac{\mu}{2d} \norm{X}_{F}^{2}
%\end{equation}
%Where $\norm{ \cdot }_F$ is the Frobenius norm. Further, to measure the performance of the estimation, the matrix mean-square error (or generalization error) $\mathcal E(X) = \frac{1}{d} \norm{X^* X^{*T} - X X^{T}}_{F}^{2}$ is used. In contrast with former work \cite{tarmoun2021understanding}, the matrix $X(t=0)=X_0$ is initialized randomly with $[X_0]_{ij} \sim \mathcal N(0,\frac{1}{n})$ so that the eigenvectors of $X_0 X_0^T$ have not been preprocessed to be aligned with those of $Y$. 






%\input{main-preliminaries}
%\documentclass{amsart}
%%%%%% GENERAL MATH COMMANDS
% Reals
\newcommand{\R}{{\mathbb R}}
% Integers
\newcommand{\Z}{{\mathbb Z}}
% Naturals
\newcommand{\N}{{\mathbb N}}
% Expectation
\DeclareMathOperator*{\E}{\mathbb{E}}
% ^th notation
\newcommand{\tth}{^{\text{th}}}
% Small dots for integer range [a .. b]
\newcommand{\sdots}{\,..\,}
% Vectorized version of matrix
\newcommand{\matvec}{\mbox{vec}}

% := sign
\newcommand{\defeq}{\vcentcolon=}
% Zero function
\newcommand{\zf}{\mathbf{0}}
% Vector of ones
\newcommand{\ones}{\mathbf{1}}

% Argmin and argmax definitions
\DeclareMathOperator*{\argmax}{arg\,max}
\DeclareMathOperator*{\argmin}{arg\,min}


%%%%% PROBLEM STATEMENT NOTATION 
% \newcommandtwoopt{\St}[2][t][]{{S_{#1}^{#2}}} % State
\newcommand{\task}[1][i]{{\mathcal{T}_{#1}}} % Task, optionally takes index
\newcommand{\tasks}{\{ \task \}_{i=1}^N}
\newcommand{\losst}[1][i]{{l_{#1}}}
\newcommand{\lossv}[1][i]{{l_{#1}^{\textrm{val}}}}
\newcommand{\tasktarget}{{\mathcal{T}_{\textrm{target}}}}
\newcommand{\lossttarget}{l_{\textrm{target}}}
\newcommand{\lossvtarget}{l_{\textrm{target}}^{\textrm{val}}}
\newcommand{\lossttargetit}{l_{\textrm{target}}^{(k)}}
\newcommand{\losstotal}{l^{\textrm{total}}}
\newcommand{\lossopt}{l^*}

\newcommand{\thetait}[2]{\theta_{#1}^{(#2)}}
\newcommand{\phit}[1]{\phi^{(#1)}}
\newcommand{\hist}[2]{S_{#1}^{(#2)}}
\newcommand{\grad}[2]{G_{#1}^{(#2)}}

\newcommand{\Alg}{\textup{\textbf{Opt}}}
\newcommand{\MetaAlg}{\textup{\textbf{MetaOpt}}}

%%%%% Theorems
\newtheoremstyle{mytheoremstyle} % name
    {\topsep}                    % Space above
    {\topsep}                    % Space below
    {\itshape}                   % Body font
    {}                           % Indent amount
    {\scshape}                   % Theorem head font
    {.}                          % Punctuation after theorem head
    {.5em}                       % Space after theorem head
    {}  % Theorem head spec (can be left empty, meaning ‘normal’)
\theoremstyle{mytheoremstyle}
\theoremstyle{plain}
\newtheorem{theorem}{Theorem}
\newtheorem{proposition}{Proposition}
\newtheorem{assumption}{Assumption}
\newtheorem{definition}{Definition}
\newtheorem{lemma}{Lemma}
\theoremstyle{remark}
\newtheorem{remark}{Remark}

%
%\begin{document}
%\section{Main results}

In this section we state our main result, which is the existence condition for $d$-GS structures. We then describe the bijections between the different incarnations of $d$-GS structures.

\begin{thm}\label{thm:main}
Let $d\geq 3$ and let $G$ be a $d$-map. There exists a $d$-GS wood (resp. labeling, marked orientation, angular orientation) for $G$ if and only if $G$ is $d$-adapted (that is, if its simple non-facial cycles have length at least $d$). 

Moreover for any fixed $d$, there is an algorithm which takes as input a $d$-adapted map, and computes a $d$-GS wood (resp. labeling, marked orientation, angular orientation) in linear time in the number of vertices.

Lastly, the set $\bW_G$ of $d$-GS woods of $G$, the set $\bL_G$ of $d$-GS labelings of $G$, the set $\bM_G$ of $d$-GS marked orientations of $G$, and the set $\bA_G$ of $d$-GS angular orientations of $G$ are all in bijection.
\end{thm}

%\fig{width=\linewidth}{def-incarnations}{Conditions defining the $d$-grand-Schnyder structures.}


We will prove the existence result stated in Theorem~\ref{thm:main} in Section~\ref{sec:proof-existence}, and we will also describe there the algorithm for computing $d$-GS structures. %In \cite{OB-EF:Schnyder} an existence proof was given for the special case of Schnyder decompositions (which correspond to $d$-GS structures on $d$-angulations) by using a ``min-cut max-flow'' type of argument. It may be possible to adapt this argument in the general case, however the proof in the present article is different: it is constructive and can be used to define a linear time algorithm for computing $d$-GS structures.}
\\


In Section~\ref{sec:lattice} we will show that the set of $d$-GS structure of a given $d$-adapted map can be given a \emph{lattice structure} (in the sense of partially ordered sets), and characterize the covering operations. 
%The order relation in the lattice is easiest to describe in terms of the angular orientation incarnation: a $d$-GS angular orientation $\cA\in \bA_G$ is less than $\cA'\in \bA_G$ if $\cA'$ can be obtained from $\cA$ be repeatedly flipping some counterclockwise simple cycle into a clockwise cycle (more precisely, decreasing the weight of such a cycle, and increasing the weight of the opposite cycle; see Section~\ref{sec:lattice} for details).\\


For the rest of this section, we focus on defining the bijections between the different incarnations of $d$-GS structures.
From now on, we fix an integer $d\geq 3$ and a $d$-map $G$. The outer vertices of $G$ are denoted by $v_1,\ldots,v_d$, and they appear in clockwise order around the outer face of $G$. The bijections between the sets $\bW_G$, $\bL_G$, $\bM_G$ and $\bA_G$ are represented in Figure~\ref{fig:bij-labeling-marked},~\ref{fig:bij-marked-angular} and~\ref{fig:bij-labeling-wood}. We start with the bijections between $\bL_G$, $\bM_G$ and $\bA_G$, which are easier to prove.


%We delay the proof of Proposition~\ref{prop:bij-alpha}. The difficulty is to show that the image of a $d$-GS labeling by $\th$ is a $d$-GS wood, and the image of a $d$-GS labeling by $\th$ is a $d$-GS wood.
%The difficulty lies in establishing that for any $d$-GS labeling $\cL\in \bL_G$, the image $\th(\cL)$ is a $d$-GS wood.


First, we define the bijection $\Phi$ between $d$-GS labelings and $d$-GS marked orientations. Roughly speaking, the bijection $\Phi$ is as follows: the marks encode the label jumps around a face, while the arc weights encode the label jumps around vertices. The bijection $\Phi$ is represented in Figure~\ref{fig:bij-labeling-marked}. We call \emph{marked orientation} of $G$ a weighted orientation of $G$ together with assigning a number of marks to each of its inner corners. 

\begin{definition}
Given a $d$-GS labeling $\cL$ of $G$, we define a marked orientation $\Phi(\cL)$ of $G$ as follows. 
For an inner corner $c$, we denote by $c^-$ the corner preceding $c$ in clockwise order around the face containing $c$, and by $c^+$ the corner following $c$ in clockwise order around the vertex incident to $c$.
\begin{compactitem}
\item The number of marks of an inner corner $c$ in $\Phi(\cL)$ is $\delta-1$, where $\delta$ is the label jump from $c^-$ to $c$.
\item The weight of outer arcs in $\Phi(\cL)$ is 0. 
Let $a$ be an inner arc, and let $c$ be the corner preceding $a$ in clockwise order around the initial vertex of $a$. Then the weight of the inner arc $a$ in $\Phi(\cL)$ is $\delta +\eps-1$, where $\delta$ is label jump from $c^-$ to $c$ and $\eps$ is label jump from $c$ to $c^+$.
\end{compactitem}
%Let $a$ be an inner arc. Let $f$ be the face on the left of $a$, let $v$ be the origin of $a$ and let $c$ be the corner of $f$ incident to $v$. The weight of the inner arc $a$ in $\Phi(\cL)$ is $\delta+\eps$
% $c$ and $c'$ be the corners preceding and following $a$ clockwise order around $v$. The weight of the inner arc $a$ in $\Phi(\cL)$ is the sum of the number of marks in $c$ and the label jump from $c$ to $c'$.
\end{definition}



\fig{width=\linewidth}{bij-labeling-marked}{Bijection $\Phi$ from $d$-GS labelings to $d$-GS marked orientations.}


\begin{prop}\label{prop:bij-beta}
The map $\Phi$ is a bijection between the set $\bL_G$ of $d$-GS labelings of $G$ and the set $\bM_G$ of $d$-GS marked orientations of $G$. 
\end{prop}

\begin{proof}
We first show that the image of a $d$-GS labeling by $\Phi$ is a $d$-GS marked orientation. 
Let $\cL\in \bL_G$ and let $\cM=\Phi(\cL)$. 
First observe that Condition~(L3) for $\cL$ translates into Condition~(M3) for $\cM$. 
Also, Condition (L1) for $\cL$ implies Condition (M1) for $\cM$ as well as the part of Condition~(M2) about weights around vertices. The part of Condition (M2) about edges (weight $d-2$ for every inner edge) is a consequence of the fact that the sum of counterclockwise label jumps around inner edges is $d$, which holds for any $d$-GS labeling by Lemma~\ref{lem:ccw-jumps-edges}. Lastly, Condition (L0) for $\cL$ implies that for any inner arc $a$ whose initial vertex is an outer vertex $v_i$, the weight $\om(a)$ of $a$ is equal to the number of marks $m(a)$ in the corner of $v_i$ on the left of $a$. Moreover, Condition (M1) gives $m(a)\leq d-\deg(f)$ and condition (M3) gives $\om(a)\geq d-\deg(f)$. This gives (M0). Hence, $\cM$ is a $d$-GS marked orientation.


By the above, $\Phi$ is a map from $\bL_G$ to $\bM_G$. It is also clear that $\Phi$ is injective because knowing $\Phi(\cL)$ allows one to determine all the label jumps between ``adjacent corners'' (corners that are consecutive around faces or vertices) which together with Condition (L0) determines the corner labeling $\cL$. The label jumps determined from a marked orientation is indicated in the bottom part of Figure~\ref{fig:bij-labeling-marked}, and provides a tentative inverse mapping for $\Phi$.
In order to prove that $\Phi$ is surjective, we need to take a closer look at this inverse mapping. Namely, we need to check that, starting from a marked orientation $\cM\in \bM_G$, the `jump assignments'' determined as indicated in Figure~\ref{fig:bij-labeling-marked} can be satisfied by a $d$-labeling of $G$ (recall that a \emph{$d$-labeling} is simply an assignment of a value in $[d]$ to each inner corner of $G$).


The \emph{corner-graph} of $G$ is the directed graph $C_G$ defined as follows: the vertices of $C_G$ are the inner corners of $G$, and there is an oriented edge from a corner $c$ to a corner $c'$ in $C_G$ if $c'$ is the corner following $c$ in clockwise order around a face or vertex. The corner graph is represented in Figure~\ref{fig:angular-map}(c).
Note that the corner graph comes with an embedding in the plane determined by $G$; it has three types of inner faces corresponding to the inner vertices, inner faces, and inner edges of $G$ respectively. 
A \emph{$d$-jump function} for $G$ is an assignment of a number in $\{0,1,\ldots,d-1\}$ to each oriented edge of the corner graph $C_G$. A $d$-jump function is called \emph{exact} if its values are equal to the label jumps of a $d$-labeling of $G$. It is easy to see that a $d$-jump function $f$ is exact if and only if every simple cycle $C$ of $C_G$ satisfies:
\begin{equation}\label{eq:jumps-exact}
\sum_{a\in C^+}f(a)-\sum_{a\in C^-}f(a)\in d\ZZ,
\end{equation}
where $C^+$ (resp. $C^-$) is the set of edges of $C_G$ appearing clockwise (resp. counterclockwise) on~$C$. Indeed~\eqref{eq:jumps-exact} is the condition that ensures that labels can be ``propagated'' according to the jump assignments without encountering any conflict. Furthermore, it is easy to check that~\eqref{eq:jumps-exact} holds if and only if it holds for every cycle which is the contour of an inner face of $C_G$. 

In conclusion, if one fixes an assignment of jumps from every inner corner to the next corner around the faces and vertices of $M$, this assignment of jumps corresponds to a $d$-labeling of $G$ if and only if the sum of assigned jumps around each vertex, face and edge of $G$ is a multiple of~$d$. 
Using this criteria, we see that for any marked orientation $\cM\in\bM_G$, Conditions (M1) and (M2) imply that the `jump assignments'' determined as indicated in Figure~\ref{fig:bij-labeling-marked} can be realized by a $d$-labeling of $G$. There is a unique such $d$-labeling such that the corners incident to $v_1$ have label 1, and we denote this $d$-labeling by $\bPhi(\cM)$. Using Condition (M0) and (M1) for $\cM$, we see that the jump assignments for $\bPhi(\cM)$ along the outer edges of $G$ are equal to 1. This implies that $\bPhi(\cM)$ satisfies Condition (L0). Lastly, it is easy to see that Conditions (M1) and (M2) for $\cM$ imply Condition (L1) for $\bPhi(\cM)$, and that Condition (M3) implies Condition (L3), while Condition (L2) holds for $\bPhi(\cM)$ by definition of the jump assignments. Thus, $\bPhi$ is a map from $\bM_G$ to $\cL_G$. 

Finally, it is clear that $\bPhi\circ \Phi=\Id_{\cL_G}$ and $\Phi\circ \bPhi=\Id_{\cM_G}$, hence these are inverse bijections between the $d$-GS labelings and the $d$-GS marked orientations of $G$. 
\end{proof}

Next, we define the bijection $\Psi$ between $d$-GS marked orientations and $d$-GS angular orientations of $G$. This bijection is represented in Figure~\ref{fig:bij-marked-angular}.
In the angular map $G^+$, the \emph{original arcs} are those on original edges (edges of $G$) while the \emph{star arcs} are those on star edges. 
%that have been added to construct $G^+$. 

\begin{definition}
Given a $d$-GS marked orientation $\cM$ of $G$ with arc weights denoted by $\om$, we define a weighted orientation $\Psi(\cM)$ of $G^+$ as follows.
% an angular orientation $\Psi(\cM)$ of $G$ as follows.
\begin{compactitem}
\item The weight in $\Psi(\cM)$ of an original inner arc $a$ is $\om(a)+\deg(f)-d$, where $f$ is the face of $M$ at the left of $a$. The weight of outer arcs in $\Psi(\cM)$ is 0.
\item Let $c$ be an inner corner of $G$, let $m$ be the number of marks of $c$ and let $f$ be the face of $G$ containing $c$. The weight in $\Psi(\cM)$ of the star arc from the star vertex $v_f$ to $c$ is~$m$, and the weight of the star arc from $c$ to $v_f$ is $d-\deg(f)-m$.
\end{compactitem}
\end{definition}


\fig{width=\linewidth}{bij-marked-angular}{Bijection $\Psi$ between $d$-GS marked orientations and $d$-GS angular orientations.}

\begin{prop}\label{prop:bij-gamma}
The map $\Psi$ is a bijection between the set $\bM_G$ of $d$-GS marked orientations of $G$ and the set $\bA_G$ of $d$-GS angular orientations of $G$. 
\end{prop}

\begin{proof}
We first show that the image of a $d$-GS marked orientation by $\Psi$ is a $d$-GS angular orientation. 
Let $\cM\in \bM_G$ and let $\cA=\Psi(\cM)$. It is clear that Conditions (M0) and (M1) for $\cM$ imply Conditions (A0) and (A1) respectively for $\cA$. Also Condition (M2) for an inner edge $e$ of $\cM$ (weight $d-2$ in $\cM$) implies condition (A2) for $e$ (weight $d-2-(d-\deg(f))-(d-\deg(f'))=\deg(f)+\deg(f')-d-2$ in $\cA$). Lastly, Condition (M2) for a vertex $v$ of $M$ (outgoing weight $d+\#marks$ in $\cM$) implies Condition (A2) for $v$. Indeed, for any original arc $a$ out of $v$, with $a'$ the preceding star arc in clockwise
order around $v$, the sum of the weights (in $\cA$) of $a$ and $a'$ is equal to $\om(a)-m$, where $m$ is the number of marks in the corner preceding $a$. 

It is easy to invert the mapping $\Psi$. Given a $d$-GS angular orientation $\cA$ of $G$ with arc weights denoted by $\om^+$, we define a marked orientation $\bPsi(\cM)$ of $G$ as follows.
\begin{compactitem}
\item The weight in $\bPsi(\bM)$ of an arc $a$ of $M$ is $\om^+(a)-\deg(f)+d$, where $f$ is the face of $M$ at the left of $a$. The weight of outer arcs in $\bPsi(\cA)$ is 0.
\item Let $c$ be an inner corner of $G$ in a face $f$. The number of marks of $c$ in $\bPsi(\cA)$ is the weight of the star arc from the star vertex $v_f$ to $c$.
\end{compactitem}

As before, it is easy to see that Conditions (A0), (A1) and (A2) for $\cA$ imply Conditions (M0), (M1) and (M2) respectively for $\bPsi(\cM)$. Moreover, Condition (M3) for $\bPsi(\cM)$ is immediate from the definition, hence $\bPsi(\cM)$ is a $d$-GS marked orientation.
Lastly, it is clear that $\bPsi\circ\Psi=\Id_{\bM_G}$ and $\Psi\circ\bPsi=\Id_{\bA_G}$, thus $\Psi,\bPsi$ are bijections.
\end{proof}




Lastly, we define a bijection $\Th$ between $d$-GS labelings and $d$-GS woods of $G$. This bijection is represented in Figure~\ref{fig:bij-labeling-wood}. Recall from Section~\ref{subsec:GS-woods} that we can interpret a $d$-GS wood $(W_1,\ldots,W_d)$ of $G$ in terms of ``arc colorings'': we say that an arc $a$ has color $i$ if it belongs to the spanning tree $W_i$, where $W_i$ is oriented toward its root $v_{i}$ as usual. By Condition (W1), the colors of a given arc are cyclically consecutive, and it will be convenient to use some notation for such intervals. For elements $i,j$ of $[d]$, we denote by $[i:j[$ the set of integers $\{i,i+1,i+2,\ldots,j-1\}$ modulo $d$. More precisely, if $i\leq j$ then $[i:j[:=\{k\in [d] \mid i\leq k<j\}$ and if $j<i$, then $[i:j[:=\{k\in [d]\mid k\geq i\textrm{ or }k<j\}$ (note the special case $[i:i[=\emptyset$). Roughly speaking, the bijection $\Th$ from $d$-GS labelings to $d$-GS woods is obtained by assigning to each arc $a$ the set of colors $[i:j[$, where $i,j$ are the labels preceding and following $a$. The precise definition is as follows.



\begin{definition}[Bijection between labelings and woods]
Given a $d$-GS labeling $\cL$ of $G$, we define a tuple $\Th(\cL)=(W_1,\ldots,W_d)$ of subsets of arcs of $G$ (interpreted as an arc coloring) as follows.
\begin{compactitem}
\item For all $i\in [d]$, the outer arc from $v_i$ to $v_{i+1}$ has all the colors except $i$, while the outer arc from $v_{i+1}$ to $v_{i}$ has no color.
%For all $k\in[d]$, the set $W_k$ contains all the outer arcs $(v_i,v_{i+1})$ for $k\in [d]$ (with the usual convention $v_{d+1}=v_1$) except those incident to $v_k$ and $v_{k+1}$.
\item An inner arc $a$ of $G$ has color set $[i:j[$, where $i$ and $j$ are the labels of the corners at the left and right of the arc $a$ respectively, at the initial vertex of~$a$. 
%For all $k\in[d]$, an inner arc $a$ of $G$ is in $W_k$ if and only if $k$ is in the set $[i:j[$, where $i$ and $j$ are the labels of the corners at the left and right of the arc $a$ respectively at the initial vertex of $a$. 
\end{compactitem}
Given a $d$-GS wood $\cW$ of $G$ (interpreted in terms of arc coloring), we define a corner labeling $\bTh(\cW)$ as follows.
\begin{compactitem}
\item The inner corners incident to the outer vertex $v_{i}$ receive label $i$.
\item A corner $c$ incident to an inner vertex $v$ has label $i$ if it is between the outgoing arcs of color $i-1$ and $i$ in clockwise order around $v$.
\end{compactitem}
\end{definition}

\fig{width=\linewidth}{bij-labeling-wood}{Bijection $\Th$ between $d$-GS labelings and $d$-GS woods.}

\begin{remark}
In the definition of $\bTh$, the rule for setting the corner labels for inner vertices and outer vertices can be unified, up to using the appropriate convention. Namely, given a $d$-GS wood $\cW$ of $G$, one can add an outgoing edge of color $k$ from $v_k$ to $v_{k-1}$ for all $k\in[d]$. With this convention, every vertex $v$ of $G$ has one outgoing arc of each color, and each inner corner incident to $v$ has label $i$ if it is between the outgoing arcs of color $i-1$ and $i$ in clockwise order around $v$.
\end{remark}


\begin{prop}\label{prop:bij-theta}
The map $\Th$ is a bijection between the set $\bL_G$ of $d$-GS labelings of $G$ and the set $\bW_G$ of $d$-GS woods of $G$. The map $\bTh$ is the inverse bijection. 
\end{prop}

The proof of Proposition~\ref{prop:bij-theta} is a bit technical and we postpone it to Section~\ref{sec:remaining-proofs}. 

%%%%%%%%%%%%%%%%%%%%%%%%%%%%%%%%%%%%%%%%%%%%%%%%%%%%%%%%%%%%%%%%%%



%% The rest of this section is devoted to the proof of Proposition~\ref{prop:bij-theta}.
%% \begin{lemma}\label{lem:image-theta}
%% Let $G$ be a $d$-map. For any $d$-GS labeling $\cL$ of $G$, the image $\Th(\cL)$ is a $d$-GS wood.
%% \end{lemma}
%% \begin{proof}
%% Let $\cL\in \bL_G$, and let $\cW=(W_1,\ldots,W_d)=\Th(\cL)$ be the corresponding arc coloring (where $W_i$ is the set of arcs of color $i$). 
%% %Before proving that the subsets of arcs $W_1,\ldots, W_d$ are trees, 
%% We observe that $\cW$ satisfy Condition (W1) because in $\cL$ the sum of label-jumps in clockwise order around any inner vertex is $d$. 
%% Next, we show that $\cW$ satisfy Condition (W2). 
%% Let $a$ be an inner arc of $G$ oriented from $u$ to $v$. We assume that $a$ has color $i$ in $\cW$, and that $v$ is an inner vertex. 
%% We want to show that $a$ appears strictly between the outgoing arc of color $i+1$ and the outgoing arc of color $i+1$ around $v$. 
%% Let $i_1,i_2,i_3,i_4$ be the labels in counterclockwise order around $a$ as indicated in Figure~\ref{fig:labels-around-edge}.
%% Since the arc $a$ has color $i$, we have $i\in [i_1:i_2[$. Moreover, since the label jumps are all positive around faces by Condition (L2), we have $i-1,i, i+1\in [i_4:i_1[\cup [i_1:i_2[\cup [i_1:i_3[$. By Lemma~\ref{lem:ccw-jumps-edges}, the sum of label jumps counterclockwise around the arc $a$ is equal to $d$, hence the sets $[i_4,i_1[$, $[i_1:i_2[$, $[i_2:i_3[$, and $[i_3:i_4[$ are disjoint (and give a partition of $[d]$). This implies that the arcs of color $i-1$, $i$ and $i+1$ will appear in this order in clockwise order around $v$ starting at the corner labeled $i_4$ and ending at the corner labeled $i_3$, which proves that the arc $a$ satisfies (W2).

%% \fig{width=.3\linewidth}{labels-around-edge}{Corner labels around an edge.}

%% Next, we show that the set of arcs $W_i$ is a spanning tree oriented toward $v_{i}$ for all $i\in [d]$. 
%% By (W1), every inner vertex is incident to exactly 1 outgoing arc of color $i$ for all $i\in[d]$. 
%% Let $v$ be an inner vertex. We consider the path $P_i(v)$ starting at $v$ which is obtained by following the arcs of color $i$ until reaching an outer vertex or an inner vertex already visited. Suppose for contradiction that there exists an inner vertex $v$ and a color $i\in [d]$, such that $P_i(v)$ ends at an inner vertex, so that $P_i(v)$ contains a cycle $C_i(v)$. We pick $v$ and $i$ such that the number of faces contained in the cycle $C_i(v)$ is minimal. Suppose first that $C_i(v)$ is directed clockwise. In this case, because of Condition (W2), for every vertex $u$ on $C_i(v)$, the outgoing edge of color $i+1$ at $u$ is either the same as the outgoing arc of color $i$ or goes strictly inside $C_i(v)$. This implies that the path $P_{i+1}(u)$ cannot reach vertices laying outside of the region enclosed by $C_i(v)$. By the minimality condition on $C_i(v)$, we conclude that $C_{i+1}(u)$ is equal to $C_i(v)$ for any vertex $u$ on $C_i(v)$. Repeating the argument, we get $C_j(u)=C_i(v)$ for all $j\in [d]$. We reach a contradiction because, by definition of $\be$, no arc can have all the colors in $[d]$. Similarly, if one supposes that $C_i(v)$ is directed counterclockwise, then one can prove that $C_{i-1}(u)$ is equal to $C_i(v)$ for any vertex $u$ on $C_i(v)$, and this leads to a contradiction as before. This concludes the proof that for all $v$, the path $P_i(v)$ reaches an outer vertex. Lastly, since the labeling $\cL$ satisfy (L0) it is easy to see that the outer vertices $v_i$ and $v_{i+1}$ are not incident to ingoing inner arcs of color $i$. Thus, all the paths $P_i(v)$ end at an outer vertex distinct from $v_{i}, v_{i+1}$, which implies that the set of arcs $W_i$ forms a spanning tree oriented toward $v_{i}$. Moreover $v_i,v_{i+1}$ are no incident to ingoing arcs of color $i$, hence $\cW$ satisfies (W0).

%% It remains to show that $\cW$ satisfies (W3). Consider an inner arc $a$ oriented from $u$ to $v$. Let $f$ be the face at the right of $a$. Let $i_1,i_2,i_3,i_4$ be the labels in counterclockwise order around $a$ as indicated in Figure~\ref{fig:labels-around-edge}. 
%% Suppose that $a$ has color $i$, or that $a$ is strictly between the outgoing arcs of color $i$ and $i+1$ in clockwise order around $u$. Suppose also that the number $\eps:=|[i_3:i_4[|$ of colors of the arc $-a$ satisfies $d-\deg(f)-\eps>0$. 
%% In order to prove that the arc $a$ satisfy condition (W3), it suffices to prove that $[i:i+2+d-\deg(f)-\eps[\subseteq [i_4:i_3[$. Since, by Lemma~\ref{lem:ccw-jumps-edges}, the sum of label jumps counterclockwise around $a$ is $d$, we have $[i_4:i_3[=[i_4:i_2[\cup [i_1:i_3[$. Moreover, under our hypotheses, $i\in[i_4:i_2[$, and $[i+1:i+1+\delta[\subseteq [i_1:i_3[$, where $\delta=|[i_2:i_3[|$.
%% Lastly, by Condition (L3) we have $\delta\geq d+1-\deg(f)-\eps$, hence $[i+1:i+2+d-\deg(f)-\eps[\subseteq [i_4:i_3[$. This concludes the proof of that Condition (W3) holds. Thus $\cW$ is a $d$-GS wood.
%% \end{proof}

%% \begin{lemma}\label{lem:image-theta-inverse}
%% Let $G$ be a $d$-map. For any $d$-GS wood $\cW$ of $G$, the image $\bTh(\cW)$ is a $d$-GS labeling.
%% \end{lemma}

%% \begin{proof}
%% Let $\cW$ be a $d$-GS wood of $G$, and let $\cL=\bTh(\cW)$. Condition (L0) holds for $\cL$ by definition of $\bTh$. Moreover it is clear that Condition (W1) for $\cW$ implies that the sum of label jumps clockwise around inner vertices is always $d$. In order to establish that $\cL$ satisfies (L1) and (L2) we need to establish the two technical results.

%% \noindent \textbf{Claim 1:} For every inner edge $e$ of $G$, the 4 corners incident to $e$ cannot all have the same label.

%% Suppose for the sake of contradiction, that an inner edge $e$ has its 4 incident corners labeled $i$. Let $u,v$ be the endpoints of $e$. Without loss of generallity, we can suppose that $v$ is an inner vertex. Consider the paths $P_1(v),\ldots,P_d(v)$ of color $1,2,\ldots,d$ starting at $v$ as defined in Section~\ref{subsec:GS-woods}. Let $R_i(v)$ be the region delimited by the paths $P_{i-1}(v)$ and $P_i(v)$. Since the corners incident to $e$ have label $i$, the edge $e$ is in $R_i(v)$, hence $u$ is in $R_i(v)$. By Lemma~\ref{lem:vi-not-in-Ri} this implies that $u\neq v_i$. Since $u$ is incident to some corners labeled $i$, we conclude that $u$ is not an outer vertex. Hence both $u$ and $v$ are inner vertex. Moreover $u$ is in $R_i(v)$, and symmetrically $v$ is in $R_i(u)$.
%% By Corollary~\ref{cor:containment-regions} the region $R_i(u)$ is strictly contained in $R_i(v)$, and $R_i(v)$ is strictly contained in $R_i(u)$, which gives a contradiction. Hence Claim 1 holds




%% Next, we look at the label situation around inner edges. Let $a$ be an inner arc of $G$ and let $i_1,i_2,i_3,i_4$ be the labels of the incident corners as indicated in Figure~\ref{fig:labels-around-edge}. 

%% \noindent \textbf{Claim 2:} The sum of label jumps in counterclockwise order around $a$ is $d$, and moreover $i_2\neq i_3$ and $i_1\neq i_4$.


%% Let us first prove Claim 2 in the case $i_1\neq i_2$. If $i_1\neq i_2$ then by property (W2) of $\cW$ we have $[i_1:i_2[\cap [i_3,i_4[=\emptyset$, hence the sum of label jumps in counterclockwise order around $a$ is $d$. Moreover, still by property (W2), $i_2\neq i_3$ and $i_1\neq i_4$ so Claim 2 holds. 
%% By symmetry, if $i_3\neq i_4$, then Claim 2 holds.
%% Lastly, if $i_1=i_2$ and $i_3=i_4$, then $i_1=i_2\neq i_3=i_4$ by Claim 1, which implies again that the sum of label jumps in counterclockwise order around $e$ is $d$, and Claim 2 holds again.

%% Claim 2 trivially implies that $\cL$ satisfies (L2). Moreover, Claim 2 implies that the sum of label jumps clockwise around every inner face is at least $d$ (since it is a multiple of $d$ and cannot be 0). Next, we use Equation~\eqref{eq:sum-jumps-relation} between the sum of label jumps around vertices, edges and faces.
%% Using Claim 2, we get
%% \begin{equation}\label{eq:sum-jumps-relation2}
%% d(|F|+|V|)\leq \sum_{f\in F}\cwjump(f)+ \sum_{v\in V}\cwjump(v)=d+\sum_{e\in E}\ccwjump(e)=d(1+|E|),
%% \end{equation}
%% where $V,F,E$ are the set of inner vertices, faces, and edges of $G$ respectively. By the Euler relation we have $|F|+|V|=1+|E|$, hence the inequality in~\eqref{eq:sum-jumps-relation2} is an equality. Thus the sum of label jumps clockwise around every inner fave is $d$. This complete the proof that $\cL$ satisfies (L1).

%% It remains to prove that $\cL$ satisfies (L3). Consider an inner arc $a$ oriented from $u$ to $v$ with incident corners labeled $i_1,i_2,i_3,i_4$ as indicated in Figure~\ref{fig:labels-around-edge}. Let $f$ be the face at the right of $a$, let $\delta=|[i_2,i_3[|$ and let $\eps=|[i_3,i_4[|$. We want to show $\delta+\eps\geq d-\deg(f)+1$.
%% If $d-\deg(f)-\eps<0$, then this inequality clearly holds (since $\delta>0$). Suppose now that $d-\deg(f)-\eps\geq 0$, and consider Consider (W3) of $\cW$. If $i_1\neq i_2$, then Condition (W3) applied to color $i=i_2-1$ of $a$ gives $\delta\geq d-\deg(f)-\eps+1$ as wanted. If $i_1=i_2$, then $a$ is between the outgoing arc of $W_i$ and $W_{i+1}$ around $u$ for $i=i_2-1$ and we also get $\delta\geq d-\deg(f)-\eps+1$ as wanted. This shows that $\cL$ satisfies (L3), which completes the proof that $\cL$ is a $d$-GS labeling.
%% \end{proof}

%% \begin{proof}[Proof of Proposition~\ref{prop:bij-theta}] Let $G$ be $d$-map. 
%% By Lemmas~\ref{lem:image-theta} and~\ref{lem:image-theta-inverse}, $\Th$ is a map from $\bL_G$ to $\bW_G$, and $\bTh$ is a map from $\bW_G$ to $\bL_G$. It is easy to see that $\bTh\circ \Th=\Id_{\bL_G}$ and $\Th\circ \bTh=\Id_{\bW_G}$. Thus these are inverse bijections. 
%% \end{proof}

%% Before closing this section, we prove Lemma~\ref{lem:W2'}. 
%% \begin{proof}
%% Let $G$ be a $d$-adapted map, and let $\bW_G'$ be the set of maps satisfying Conditions (W0), (W1) and (W2'). We clearly have the inclusion $\bW_G'\subseteq \bW_G$ and want to show $\bW_G'= \bW_G$. 

%% Let $\bTh'$ be the extension of the map $\bTh$ to $\bW_G'$ (with the same definition as $\bTh$). Let $\cW\in \bW_G'$ and let $\cL=\bTh'(\cW)$. We claim that $\cL$ is in $\bL_G$. We reason as in the proof of Lemma~\ref{lem:image-theta-inverse}, and since $\cW$ satisfies (W0), (W1) and (W2), we get that $\cL$ satisfy (L0), (L1), (L2). It remains to prove Condition (L3) for every arc $a$. We adopt the notation $i_1,i_2,i_3,i_4$, $f$, $\delta=|[i_2,i_3|$ and $\eps=|[i_3,i_4|$ from the proof of Lemma~\ref{lem:image-theta-inverse}. We need to show $\delta+\eps\geq d-\deg(f)+1$. If $i_1\neq i_2$ then Condition (W2') applied to color $i=i_2-1$ of $a$ gives $\delta\geq d-\deg(f)-\eps+1$ as wanted. We now consider the case $i_1=i_2$. Let $f'$ be the face at the left of $a$. Since $G$ is $d$-adapted, we must have $\deg(f)+\deg(f')-2\geq d$ (because $\deg(f)+\deg(f')-2$ is the length of a non-facial cycle of $G$: the contour of the face one would obtain by deleting $e$ and merging $f$ and $f'$). Moreover, $|[i_4,i_1[|\leq d-\deg(f')+1$ because the sum of label jumps around $f'$ is $d$ and each jump is at least 1.
%% Since the sum of label jumps in counterclockwise order around $a$ is $d$, one obtains
%% $$\delta+\eps=d-|[i_4,i_1[|\geq \deg(f')-1\geq d-\deg(f)+1.$$
%% Thus Condition (L3) holds and $\cL$ is in $\bL_G$.

%% Since the image of $\bTh'$ is in $\bL_G$, we can compose this map with $\Th$. For all $\cW\in \bW_G'$ one clearly has $\Th \circ \bTh'(\cW)=\cW$, hence $\cW$ is in the image of $\Th$, which is in $\bW_G$ by Lemma~\ref{lem:image-theta}. This concludes the proof that $\bW_G'= \bW_G$. 
%% \end{proof}

%\end{document}


\section{Sketch of Proof}\label{sec:sketchproof}
Our method relies on considering the interaction of the random matrices $X_0, X^*, \xi$. We treat each term $q_t$ and $p_t$ separately with the linear-pencil technique. In both cases, we first factor out the $X_0$ matrix, then decouple the time dependency from the remaining random matrix expressions, and finally factor-out $X^*, \xi$.

Our results are derived in the limit $n,m,d\to +\infty$. For a sequence of matrices $A_N \in \mathbb R^{N\times N}$ we use the notation $\traceLim[N]{A_N} = \lim_{N \to \infty} \frac{1}{N}\trace{A_N}$. As stated in Sec. \ref{sec:results} we assume that the limiting traces involved in the linear pencil method concentrate.

\subsection{Tracking the angle $q_t$}
The term $q_t = \traceLim[d]{Z^* Z_t}$ can be completely recovered from a sub-block of the following linear-pencil $M_q$:
\begin{align}
    M_q = \left(
        \begin{array}{c|||c||c|ccc|c}
            0 & I_d & 0 & 0 & 0 & 0 & 0\\ \hline \hline  \hline
            I_d & 0 & 0 & 0 & 0 & 0 & W_t\\ \hline \hline
            0 & 0 & 0 & X_0 & 0 & 0 & I_n\\ \hline
            0 & 0 & X_0^T & I_m & 0 & X_0^T & 0\\
            0 & 0 & 0 & 0 & L_t & I_n & 0\\
            0 & 0 & 0 & X_0 & I_n & 0  & 0\\ \hline
            0 & W_t^T & I_n& 0 & 0 & 0 & 0
        \end{array}
    \right)
\end{align}
Where $W_t = X^{*T} e^{tH}$ and $L_t = 2 \int_0^t e^{2sH} \dd s$. A recursive application of the Schur-complement to compute $M_q^{-1}$ shows that the block $(M_q^{-1})^{(1,1)}$ is the random matrix $X^{*T} Z_t X^*$. So in fact: $q_t = \traceLim[d]{(M_q^{-1})^{(1,1)}}$.

The random matrices $X_0, X^*, \xi$ are all independent and $X_0$ is not part of the terms $W_t, L_t$. Therefore, we can apply the linear-pencil theory on $M_q$ over the random-matrix $X_0$ while considering the other random matrices as fixed. To this end, we note the constant part $C_q = \mean_{X_0}[M_q]$, and consider matrix of sub-traces $g \in \mathbb R^{7 \times 7}$ such that for squared-blocks $ij$, $g_{ij} = \traceLim[N_i]{(M_q^{-1})^{(i,j)}}$ where $N_i$ is the size of the block $ij$ in $M_q^{-1}$.
Then we apply the fixed-point equation described in Appendix D of \cite{bodin.2212.06757} with $g_{ ij } = \frac{1}{N_i} \trace{ ((C_q - \eta(g) \otimes I)^{-1})^{(ij)} }$ where $\eta(g)$ is the matrix defined by:
\begin{align}
    \eta(g) = \left(
        \begin{array}{ccccccc}
            0 & 0 & 0 & 0 & 0 & 0 & 0 \\
            0 & 0 & 0 & 0 & 0 & 0 & 0 \\
            0 & 0 & \psi g_{44} & 0 & 0 & \psi g_{44} & 0 \\
            0 & 0 & 0 & g_{33} + g_{36} +  & 0 & 0 & 0 \\
             &  &  & g_{63} + g_{66} &  &  &  \\
            0 & 0 & 0 & 0 & 0 & 0 & 0 \\
            0 & 0 & \psi g_{44} & 0 & 0 & \psi g_{44} & 0 \\
            0 & 0 & 0 & 0 & 0 & 0 & 0 
        \end{array}
    \right)
\end{align}
Further inversion of $C_q - \eta(g) \otimes I$ leads to:
\begin{align}
    g_{11} &= \traceLim[d]{ g_{44} \psi W_t (g_{44} \psi L_t + I_n)^{-1} W_t^T } \\
    g_{44} & = \frac{1}{1-g_{66}} \\
    g_{66} &= -\traceLim[n]{ L_t (g_{44} \psi L_t + I_n)^{-1}  }
\end{align}
Let $\Gamma \subset \mathbb C$ be a contour enclosing the eigenvalues of $H$, we use the fact that for any functional $f$ which applies on the eigenvalues of a matrix we have $f(H) = \frac{-1}{2\pi i} \oint_{\Gamma} f(z) (H-zI_n)^{-1} \dd z$ to obtain:
\begin{equation*}
    g_{11}  = \frac{-1}{2\pi i} \oint_{\Gamma} 
    \frac{g_{44} \psi e^{2zt}}{
        1 + g_{44} \psi \int_0^t 2 e^{2sz} \dd s
    } 
    \traceLim[d]{ (H - z I_n)^{-1} Z^* } \dd z 
\end{equation*}
which leads with $Q(z) = \traceLim[d]{ X^{*T} (H - z I_n)^{-1} X^* } $ to:
\begin{align}
    g_{11}  = \frac{-1}{2\pi i} \oint_{\Gamma} 
    \frac{g_{44} \psi z}{
        \psi g_{44} (1 -  e^{-2tz}) + z e^{-2tz}
    }  Q(z) \dd z 
\end{align}
Similarly with $P(z) = \traceLim[n]{ (H - z I_n)^{-1} } $
\begin{equation}
    g_{66} \psi = \frac{-1}{2\pi i} \oint_{\Gamma}
    \frac{
        1 - e^{-2tz}
    }{
        \psi g_{44} (1 -  e^{-2tz}) + z e^{-2tz}
    } P(z) \dd z
\end{equation}
We find the equations from the main results with $\tilde q_t = \frac{1}{\psi g_{44}}$.


\subsection{Tracking the norm $p_t$}
The term $p_t = \traceLim[d]{Z_t^2}$ can also be recovered from a similar calculation but would lead to design a much larger linear-pencil. Another method is to track directly the eigenvalues of $Z_t$ with the trace of the resolvent: $h_{11} = \traceLim[n]{ (Z_t-zI_n)^{-1} }$ with $h$ the solution of the fixed point equation (Appendix D in \cite{bodin.2212.06757}) stemming from the following linear-pencil:
\begin{equation}
    M_p = \left(
        \begin{array}{c||c|ccc|c}
            -zI_n & 0 & 0 & 0 & 0 & e^{tH}\\ \hline \hline
            0 & 0 & X_0 & 0 & 0 & I_n\\ \hline
            0 & X_0^T & I_m & 0 & X_0^T & 0\\
            0 & 0 & 0 & L_t & I_n & 0\\
            0 & 0 & X_0 & I_n & 0  & 0\\ \hline
            e^{tH} & I_n& 0 & 0 & 0 & 0
        \end{array}
    \right)
\end{equation}
Which yields the set of equations:
\begin{align*}
    h_{11} & = -\traceLim[n]{
        \left(L_t + \frac{1}{h_{33}} I_n\right)
        \left(e^{2tH} + z L_t + \frac{z}{h_{33}} I_n\right)^{-1}
    } \\
    h_{33} & =
    1 - 
    \frac{1}{\psi} \traceLim[n]{
        \left(z L_t + e^{2tH}\right)
        \left(e^{2tH} + z L_t + \frac{z}{h_{33}} I_n\right)^{-1}
    }
\end{align*}
Using the contour integration technique, we obtain:
\begin{equation}
    h_{11} = \frac{-1}{2\pi i} \oint_{\Gamma}
    -\frac{
        \frac{1}{h_{33}} + \int_0^t 2e^{2sx} \dd s
    }{
        \frac{z}{h_{33}} + e^{2tx} + z \int_0^t 2e^{2sx} \dd s
    } P(x) \dd x
\end{equation}
which is reduced to:
\begin{equation}
    h_{11}(z) = \frac{-1}{2\pi i} \oint_{\Gamma}
    -\frac{
        1 + e^{-2tx} (\frac{x}{h_{33}} - 1)
    }{
        x + z + z e^{-2tx} (\frac{x}{h_{33}} -1)
    } P(x) \dd x
\end{equation}
Similarly for $h_{33}$:
\begin{equation}
    h_{33}(z) = 1 + \frac{1}{\psi} \frac{-1}{2\pi i} \oint_{\Gamma}
    -\frac{
        (x + z - z e^{2tx}) P(x) \dd x
    }{
        x + z + z e^{-2tx} (\frac{x}{h_{33}} -1)
    } 
\end{equation}
Two possible ways to retrieve $p_t$ from $h_{11}$ and $h_{33}$: either with $\phi p_t = \frac{-1}{2\pi i} \oint_\Gamma z^2 h_{11}(z) \dd z$, or $\phi p_t =  - \frac12 \frac{\partial^{(2)}}{\partial z^2} \left( \frac{1}{z} h_{11}(\frac{1}{z}) \right) \rvert_{z=0}$.  In both cases, there is an additional level of complexity in terms of calculation as it either requires a double-contour integration, or computing derivative and second derivative of the given functions yielding further new equations.

\subsection{Quantities $Q(z),P(z)$}
There remains to calculate the terms $Q(z),P(z)$ which depends only on the random matrices $X^*, \xi$ and can be done altogether with the linear-pencil:
\begin{equation}
    M_{z} = \left(
        \begin{matrix}
            I_n & X^* & 0 & 0\\
            0 & I_d & X^{*T} & 0\\
            0 & 0 & (z + \mu) I_n- \frac{1}{\sqrt \lambda} \xi & X^*\\
            0 & 0 & X^{*T} & I_d
        \end{matrix}
    \right)
\end{equation}
Using the kernel $K = (H-zI_n)^{-1}$, we can calculate the inverse:
\begin{equation}
    M_z^{-1} = \left(
        \begin{matrix}
            I_n & -X^* & -Z^*K & Z^* K X^*\\
            0 & I_d & X^{*T} K & -X^{*T} K X^*\\
            0 & 0 & -K & K X^*\\
            0 & 0 & -X^{*T} K & I_d - X^{*T} K X^*
        \end{matrix}
    \right)    
\end{equation}
So that $Q(z) = -f_{13}$ and $P(z) = f_{33}$ where we $f$ is the analog of $g$ and $h$ with the former linear-pencils. In particular we expect the following structure:
\begin{equation}
    f = \left(
        \begin{matrix}
            1 & 0 & -\phi Q(z) & 0 \\
            0 & 1 & 0 & - Q(z) \\
            0 & 0 & -P(z) & 0 \\
            0 & 0 & 0 & 1-Q(z)
        \end{matrix}
    \right)
\end{equation}
We can further compute the fixed point equation with:
\begin{equation}
    \eta(f) = \left(
        \begin{matrix}
            0 & 0 & f_{22} \phi + f_{24} \phi & 0\\
            0 & f_{31} & 0 & f_{33}\\
            0 & 0 & \frac{f_{33}}{\lambda} + f_{42} \phi + f_{44} \phi & 0\\
            0 & f_{31} & 0 & f_{33}
        \end{matrix}
    \right)
\end{equation}
After some algebraic reductions, we obtain the degree 3 polynomials given in equation \eqref{eq:Q_and_P}. In general, these equations have multiple solutions but only one corresponds to the analytic solution associated to the appropriate trace of resolvent.

\section{Conclusion}
% a conclusion saying that our results can be made more general with other structures of $Y$
Our work primarily shows how we can take advantage of random matrix techniques to derive fixed-point equations solving the time evolution of the matrix-mean-square-error in the high-dimensional limit. Although we choose a specific data model, as future considerations, the matrix $H$ can be generalized to other structures for which the same methods would apply. In particular, if only the noise structure changes, then only $\rho_Q$ and $\rho_P$ are changed. We will come back to these issues in a more extensive and detailed contribution.

\vskip 0.25cm
{\bf Acknowledgments} The work of A. B is supported by Swiss National Fondation Grant no 200020 182517. We also acknowledge instructive discussions with Farzad Pourkamali and Jean Barbier.

\bibliographystyle{IEEEtran}
\bibliography{IEEEabrv,bibliography}

%\bibliographystyle{abbrvnat}
%\bibliography{bibliography}



\end{document}
