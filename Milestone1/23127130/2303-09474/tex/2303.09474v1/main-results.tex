\section{Results}\label{sec:results}

\subsection{Preliminaries}
We simplify the notations by introducing the variables $Z = X X^T$ and $Z^* = X^* X^{*T}$ and the \emph{order parameters} $p$ and $q$ such that $\mathcal E = r - 2 q + p$ with:
\begin{equation}
    q = \frac{1}{d} \trace{Z^*  Z} \quad \quad p = \frac{1}{d} \trace{Z^2} \quad  \quad r = \frac{1}{d} \trace{(Z^*)^2}
\end{equation}
In the rank-one setting, $p$ can be seen as a norm of the estimator while $q$ represents the angle with the ground-truth.
We consider the gradient flow
\begin{equation}
\frac{\dd X_t}{\dd t} = -\phi \nabla \mathcal H(X_t)
\end{equation}
and track the evolution of the matrix mean-square error $\mathcal E_t$ through the quantities $q_t$ and $p_t$. The factor $\phi$ amounts to a rescaling of time which leads to more convenient expressions.
With the additional notation $H = Y - \mu I_n$, expanding the gradient provides: $\frac{\dd X_t}{\dd t} = (H-Z_t) X_t$, which in turns provides the matrix Riccati differential equation:
\begin{equation}
    \frac{\dd Z_t}{\dd t} = H Z_t + Z_t H - 2Z_t^2
\end{equation}
A general solution of this matrix differential equation is (see e.g., \cite{tarmoun2021understanding}):
\begin{equation}\label{eq:Zt}
    Z_t = e^{tH} X_0 \left(I_m + 2 X_0^T \int_0^t e^{2sH} \dd s X_0 \right)^{-1} X_0^T e^{tH}
\end{equation}
This formula is valid regardless of the dimensions $n,m,d$. In particular, when $m=d=1$ this is the solution of the rank-1 gradient flow.
In the high-rank case, it is not straightforward a priori how to track the evolution of the matrix $Z_t$ as firstly the rank of $X_0 X_0^T$ and $X^* X^{*T}$ (or $Y$ or $H$) are not necessarily equal when $d \neq m$, and secondly because the eigenvectors of the two matrices are not aligned at the initialization.

In the following, we will consider the high-dimensional limit $n,m,d \to \infty$ with $d/n$ and $m/n$ fixed and make the following assumptions:
\begin{itemize}
    \item The limits of traces $p_t = \frac{1}{d}{\rm Tr}[Z_t^2]$, $q_t= \frac{1}{d}{\rm Tr}[Z^*Z_t]$ (and $\mathcal{E}_t$) concentrate on their expectation, as well as related traces used in the \emph{linear-pencils}  method in Sec. \ref{sec:sketchproof}.
    \item We assume that $H$ has a limiting spectral distribution whose support can be enlaced in a finite contour $\Gamma \subset \mathbb C$.
\end{itemize}
To keep notations lighter we shall abusively denote by $p_t$, $q_t$, $\mathcal{E}_t$ their limiting deterministic values.

\subsection{Main results}

The MSE $\mathcal E_t$ of the problem is completely given by $q_t$, $p_t$ and the constant $r$ which in the high-dimensional limit is found to be $r = 1 + \phi$ from the second moment of the Marchenko-Pastur law \cite{marchenko1967distribution}. The main contribution of this paper is the self-consistent set of equations that fully track $q_t$ and $p_t$:

\textbf{(Result 1)} In the high dimensional limit, the overlap $q_t$ evolves according the integral:
\begin{equation}\label{eq:qt}
    q_t = \int_{\mathbb R} 
    \frac{z \rho_Q(z) \dd z}{
        1 - e^{-2tz} + z \tilde q_t e^{-2tz}
    }
\end{equation}
with the auxiliary function $\tilde q_t$ solution of the fixed-point equation:
\begin{equation}\label{eq:tilde_qt}
    \psi \tilde q_t = 1 + \int_{\mathbb R}  \frac{
        (1- e^{-2tz}) \rho_P(z) \dd z
    }{
        \frac{1}{\tilde q_t} (1 - e^{-2tz}) + z e^{-2tz}
    }
\end{equation}
and $\rho_P, \rho_Q$ are given by their inverse Stieltjes transforms $P(z), Q(z)$. These are the analytic solutions of the degree 3 polynomials such that $-zP(z) \to 1$ when $|z| \to \infty$ and $-zQ(z) \to 1$ when $|z| \to \infty$ where:
\begin{equation*}
    P^{3} + P^{2} \left( \lambda (\mu + z) + 1\right) + P \lambda \left(  \mu + z -  \phi + 1\right) + \lambda = 0
\end{equation*}
\begin{equation}\label{eq:Q_and_P}
    Q^{3} \phi 
    +
    Q^{2} \left(\mu + z - 2 \phi - 1 - \frac{1}{\lambda}\right)
    - Q \left( \mu + z - \phi - 2 \right) = 1 
\end{equation}




\textbf{(Result 2)} In the high-dimensional limit, the eigenvalue distribution of $Z_t$ is found by the inverse Stieltjes-Transform of $h_t(z)$ where:
\begin{align}\label{eq:pt}
    h_t(z) & = \frac{-1}{2\pi i} \oint_{\Gamma}
    -\frac{
        (1 + e^{-2tx} (\frac{x}{\tilde h_t(z)} - 1)) P(x) \dd x
    }{
        x + z + z e^{-2tx} (\frac{x}{\tilde h_t(z)} -1)
    }  \\
    \tilde h_t(z) & = 1 + \frac{1}{\psi} \frac{-1}{2\pi i} \oint_{\Gamma}
    -\frac{
        (x + z - z e^{2tx}) P(x) \dd x
    }{
        x + z + z e^{-2tx} (\frac{x}{\tilde h_t(z)} -1)
    }
\end{align}
in particular, we find:
\begin{equation}\label{eq:derivative_pt}
    p_t = -\frac{1}{2\phi} \left.  \frac{\partial^{(2)}}{\partial z^2} \left( \frac{1}{z} h_t\left(\frac{1}{z}\right) \right) \right\rvert_{z=0}
\end{equation}
Note that a similar system of equations as \eqref{eq:qt} can be derived by calculating the first and second derivatives in $z$ as given by  \eqref{eq:derivative_pt} and using the integrands in \eqref{eq:pt}. However the resulting formulas are too cumbersome to be presented here.

\subsection{Discussions and experiments}

Figure \ref{fig:graph1} provides an example of the calculation of $q_t$ through time compared with experimental runs: we see a good agreement between the curves and the prediction.
% code to include graph1.png
\begin{figure}[h]
    \centering
    \includegraphics[width=0.6\linewidth]{images/graph1.png}
    \vskip -0.2cm
    \caption{Comparison of $q_t$ evolution with $10$ runs of a gradient descent with $n=100$, $m=25$, $d=75$ and $\lambda=10^4$ and $\mu=0$. }
    \label{fig:graph1}
\end{figure}

\textbf{Asymptotic Limit $t\to \infty$:}
An interesting question is to study the asymptotics of $q_t$ when $t \to \infty$. We take the ansatz that $q_t \sim \gamma e^{2 \alpha t}$ in this limit with $\alpha>0$ and $\gamma > 0$ another constant and plug this in equation \eqref{eq:tilde_qt}:
\begin{align}
    \psi & = \frac{1}{\tilde q_t} + \int_{\mathbb R}  \frac{
        (1- e^{-2tz}) \rho_P(z) \dd z
    }{
       1 - e^{-2tz} + z \tilde q_t e^{-2tz}
    } \\
    & \simeq \frac{e^{-2\alpha t}}{\gamma}
    + \int_{\mathbb R}  \frac{
        (1- e^{-2tz}) \rho_P(z) \dd z
    }{
       1 - e^{-2tz} + z \gamma e^{-2 (z-\alpha) t} 
    } \\
    & \simeq  \int_\alpha^\infty \rho_P(z) \dd z = 1-F_P(\alpha)
\end{align}
With $F_P$ the CDF of $P$. Such a solution exists when we can find $\alpha$ such that $F_P(\alpha) = 1 - \psi$, effectively selecting the proportion $\psi$ of the eigenvalues of $H$ in the interval $(\alpha, +\infty)$. Due to the assumption $\alpha >0$, a further condition for the existence of such an $\alpha$ is $F_P(0) < 1 - \psi \leq 1$ or: $0 \leq \psi < 1 - F_P(0) \leq 1$. This implies that the ansatz is valid in the \emph{under-parameterized} regime $(m<n)$. The asymptotic limit is thus given by
$q_\infty = \lim_{t \to \infty} q_t = \int_{\alpha}^\infty z \rho_Q(z) \dd z$.
%\begin{equation}\label{eq:q_t_asymptotic}
%    q_\infty = \lim_{t \to \infty} q_t = \int_{\alpha}^\infty z \rho_Q(z) \dd z
%\end{equation}
Note that the alternative ansatz that $\tilde q_t$ converges towards a finite limit leads to a similar solution as but with $\alpha = 0$.

A similar line of reasoning lead us to consider the term $p_\infty = \frac{1}{\phi} \int_{\alpha}^\infty z^2 \rho_P(z) \dd z$ and thus a asymptotic mean square error:
\begin{equation}\label{eq:mse_asymptotic}
    \mathcal E_\infty = r - \int_\alpha^\infty \left(2 z \rho_Q(z) - \frac{1}{\phi} z^2 \rho_P(z) \right) \dd z
\end{equation}
As an example, for $\phi = \psi = 1$ and $\mu = \frac{1}{\lambda}$, and $\alpha=0$ we expect from formula \eqref{eq:mse_asymptotic} that $\mathcal E_\infty = r$ when the support of $\rho_P$ and $\rho_Q$ is located below $0$. This can be found by studying the discriminant $\Delta_P(\lambda, z)$ of the polynomial solved by $P$: because it is a order $3$ polynomial with coefficients in $\mathbb R$ when $z \in \mathbb R$, either the solutions are all real $(\Delta_P > 0)$ implying $\rho_P(z) = 0$, or one is real and two are complex conjugate $(\Delta_P = 0)$ implying $\rho_P(z)>0$. At a specific $\lambda$, the support of $\rho_P$ is located below $0$ and touches $z=0$. This $\lambda_c$ is solution of $\Delta_P(\lambda_c, 0) = 0$ which provides the solution $\lambda_c = \frac{4}{27}$. The whole error curve at $t = +\infty$ is shown in Figure \ref{fig:graph3}.
% code to include graph3.png
\begin{figure}[h]
    \centering
    \includegraphics[width=0.6\linewidth]{images/graph3.png}
    \vskip -0.3cm
    \caption{Experimental and theoretical $\mathcal E_\infty$ with $\phi=\psi=1, \mu = \frac{1}{\lambda}$ }
    \label{fig:graph3}
\end{figure}
The choice $\mu = \frac{1}{\lambda}$ is natural from a Bayesian point-of-view because it would correspond to the situation where the statistician matches its prior to the ground-truth when $\psi=\phi$.


\textbf{Low rank limit when $\phi = \psi \to 0$:} We bring to the reader's attention that the objective function $\mathcal H$ when $d=1$ and $n\to \infty$ with $\mu = \frac{1}{\lambda}$ corresponds precisely to the spiked-Wigner problem. This suggest to look at the limit $\phi = \psi \to 0$. In this situation, we expect $\alpha$ should be close to the maximum eigenvalue of the bulk of $\rho_Q$. We make the following observation in Figure \ref{fig:graph2}: as $\phi$ decreases, $\rho_P$ in blue has two bulks of eigenvalues, one of which disappears as $\phi$ grows. On the other hand, $\rho_Q$ in orange displays also two bulks at the same locations but the second bulk develops a mass as $\phi \to 0$. Therefore, we expect that $\alpha$ adjusts itself to the maximum eigenvalue of the first bulk of $\rho_P$. Furthermore, interestingly we see that these two bulks are getting closer when $\lambda$ is closer to $1$ as seen in Figure \ref{fig:graph4}.
%code to include graph2.png
\begin{figure}[h]
    \centering
    \includegraphics[width=0.9\linewidth]{images/graph2.png}
    \vskip -0.3cm
    \caption{Bulk of eigenvalues for $\lambda = 5$ and different values of $\phi$}
    \label{fig:graph2}
\end{figure}
%\vskip -0.5cm
\begin{figure}[h]
    \centering
    \includegraphics[width=0.9\linewidth]{images/graph4.png}
    \vskip -0.3cm
    \caption{Bulk of eigenvalues for $\lambda = 2$ and different values of $\phi$}
    \label{fig:graph4}
\end{figure}

With these observations, we expect that $Q(z)$ has a pole $z = 1$ in the limit $\phi \to 0$. Let's consider a polynomial equation of $\hat Q$ solving the reduced polynomial equation of $Q$ with $\phi = 0$:
\begin{equation}
 \hat Q^{2} (z - 1) - \hat Q (z - 2 + \frac{1}{\lambda}) - 1 = 0 .
 \end{equation}
In order to find a potential pole, we consider $\mathcal Q(z) = (1-z) \hat Q(z)$ and check for potential limits of $\mathcal Q$ when $z\to 1$. First of all, injecting $\mathcal Q$ in the former polynomial equation, we find:
\begin{equation}
\left(\mathcal Q^{2} \lambda + \mathcal Q (\lambda z - 2 \lambda + 1) - \lambda z + \lambda\right)/(z-1) = 0 .
\end{equation}
Therefore, on the upper-complex plane we find the numerator equals $0$, and by analytic continuation, the limit $z \to 1$ follows: $\mathcal Q(1) (\mathcal Q(1) \lambda - \lambda +1) = 0$ so $\mathcal Q(1) \in \{ 0, 1 - \frac{1}{\lambda} \}$. It is interesting to notice the connection with the usual Bayesian overlap of the spiked Wigner model - since 
$\mathcal Q(1)$ represents the squared overlap $q_\infty$ in the limit $\phi\to 0$. Pushing further this analysis for $P(z)$ allows to eventually get $p_\infty$ and $\mathcal{E}_\infty$ in the limit $\phi\to 0$ and check the connection with the Bayesian MMSE of the spiked Wigner model. 

%Pushing this analysis further for $P(z)$ one could also compute $p_\infty$ in the limit $\phi\to 0$ and eventually show 
%that $\mathcal{E}_\infty$ tends to the matrix-MMSE of spiked Wigner model. 

%In a similar vein, the limiting value of $p_\infty = \frac{1}{\phi} \int_{\alpha}^\infty z^2 \rho_P(z) \dd z$ can be obtained: as it can be seen on Figure \ref{fig:graph2}, the second bulk of $\rho_P$ dissipates when $\phi \to 0$ while the first bulk remains. However, there is a scaling factor $\frac{1}{\phi}$ which needs to be considered in the expression of $p_\infty$, so the vanishing second bulks needs in fact to be considered carefully. 