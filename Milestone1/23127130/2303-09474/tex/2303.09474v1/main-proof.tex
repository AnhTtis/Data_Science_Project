\section{Sketch of Proof}\label{sec:sketchproof}
Our method relies on considering the interaction of the random matrices $X_0, X^*, \xi$. We treat each term $q_t$ and $p_t$ separately with the linear-pencil technique. In both cases, we first factor out the $X_0$ matrix, then decouple the time dependency from the remaining random matrix expressions, and finally factor-out $X^*, \xi$.

Our results are derived in the limit $n,m,d\to +\infty$. For a sequence of matrices $A_N \in \mathbb R^{N\times N}$ we use the notation $\traceLim[N]{A_N} = \lim_{N \to \infty} \frac{1}{N}\trace{A_N}$. As stated in Sec. \ref{sec:results} we assume that the limiting traces involved in the linear pencil method concentrate.

\subsection{Tracking the angle $q_t$}
The term $q_t = \traceLim[d]{Z^* Z_t}$ can be completely recovered from a sub-block of the following linear-pencil $M_q$:
\begin{align}
    M_q = \left(
        \begin{array}{c|||c||c|ccc|c}
            0 & I_d & 0 & 0 & 0 & 0 & 0\\ \hline \hline  \hline
            I_d & 0 & 0 & 0 & 0 & 0 & W_t\\ \hline \hline
            0 & 0 & 0 & X_0 & 0 & 0 & I_n\\ \hline
            0 & 0 & X_0^T & I_m & 0 & X_0^T & 0\\
            0 & 0 & 0 & 0 & L_t & I_n & 0\\
            0 & 0 & 0 & X_0 & I_n & 0  & 0\\ \hline
            0 & W_t^T & I_n& 0 & 0 & 0 & 0
        \end{array}
    \right)
\end{align}
Where $W_t = X^{*T} e^{tH}$ and $L_t = 2 \int_0^t e^{2sH} \dd s$. A recursive application of the Schur-complement to compute $M_q^{-1}$ shows that the block $(M_q^{-1})^{(1,1)}$ is the random matrix $X^{*T} Z_t X^*$. So in fact: $q_t = \traceLim[d]{(M_q^{-1})^{(1,1)}}$.

The random matrices $X_0, X^*, \xi$ are all independent and $X_0$ is not part of the terms $W_t, L_t$. Therefore, we can apply the linear-pencil theory on $M_q$ over the random-matrix $X_0$ while considering the other random matrices as fixed. To this end, we note the constant part $C_q = \mean_{X_0}[M_q]$, and consider matrix of sub-traces $g \in \mathbb R^{7 \times 7}$ such that for squared-blocks $ij$, $g_{ij} = \traceLim[N_i]{(M_q^{-1})^{(i,j)}}$ where $N_i$ is the size of the block $ij$ in $M_q^{-1}$.
Then we apply the fixed-point equation described in Appendix D of \cite{bodin.2212.06757} with $g_{ ij } = \frac{1}{N_i} \trace{ ((C_q - \eta(g) \otimes I)^{-1})^{(ij)} }$ where $\eta(g)$ is the matrix defined by:
\begin{align}
    \eta(g) = \left(
        \begin{array}{ccccccc}
            0 & 0 & 0 & 0 & 0 & 0 & 0 \\
            0 & 0 & 0 & 0 & 0 & 0 & 0 \\
            0 & 0 & \psi g_{44} & 0 & 0 & \psi g_{44} & 0 \\
            0 & 0 & 0 & g_{33} + g_{36} +  & 0 & 0 & 0 \\
             &  &  & g_{63} + g_{66} &  &  &  \\
            0 & 0 & 0 & 0 & 0 & 0 & 0 \\
            0 & 0 & \psi g_{44} & 0 & 0 & \psi g_{44} & 0 \\
            0 & 0 & 0 & 0 & 0 & 0 & 0 
        \end{array}
    \right)
\end{align}
Further inversion of $C_q - \eta(g) \otimes I$ leads to:
\begin{align}
    g_{11} &= \traceLim[d]{ g_{44} \psi W_t (g_{44} \psi L_t + I_n)^{-1} W_t^T } \\
    g_{44} & = \frac{1}{1-g_{66}} \\
    g_{66} &= -\traceLim[n]{ L_t (g_{44} \psi L_t + I_n)^{-1}  }
\end{align}
Let $\Gamma \subset \mathbb C$ be a contour enclosing the eigenvalues of $H$, we use the fact that for any functional $f$ which applies on the eigenvalues of a matrix we have $f(H) = \frac{-1}{2\pi i} \oint_{\Gamma} f(z) (H-zI_n)^{-1} \dd z$ to obtain:
\begin{equation*}
    g_{11}  = \frac{-1}{2\pi i} \oint_{\Gamma} 
    \frac{g_{44} \psi e^{2zt}}{
        1 + g_{44} \psi \int_0^t 2 e^{2sz} \dd s
    } 
    \traceLim[d]{ (H - z I_n)^{-1} Z^* } \dd z 
\end{equation*}
which leads with $Q(z) = \traceLim[d]{ X^{*T} (H - z I_n)^{-1} X^* } $ to:
\begin{align}
    g_{11}  = \frac{-1}{2\pi i} \oint_{\Gamma} 
    \frac{g_{44} \psi z}{
        \psi g_{44} (1 -  e^{-2tz}) + z e^{-2tz}
    }  Q(z) \dd z 
\end{align}
Similarly with $P(z) = \traceLim[n]{ (H - z I_n)^{-1} } $
\begin{equation}
    g_{66} \psi = \frac{-1}{2\pi i} \oint_{\Gamma}
    \frac{
        1 - e^{-2tz}
    }{
        \psi g_{44} (1 -  e^{-2tz}) + z e^{-2tz}
    } P(z) \dd z
\end{equation}
We find the equations from the main results with $\tilde q_t = \frac{1}{\psi g_{44}}$.


\subsection{Tracking the norm $p_t$}
The term $p_t = \traceLim[d]{Z_t^2}$ can also be recovered from a similar calculation but would lead to design a much larger linear-pencil. Another method is to track directly the eigenvalues of $Z_t$ with the trace of the resolvent: $h_{11} = \traceLim[n]{ (Z_t-zI_n)^{-1} }$ with $h$ the solution of the fixed point equation (Appendix D in \cite{bodin.2212.06757}) stemming from the following linear-pencil:
\begin{equation}
    M_p = \left(
        \begin{array}{c||c|ccc|c}
            -zI_n & 0 & 0 & 0 & 0 & e^{tH}\\ \hline \hline
            0 & 0 & X_0 & 0 & 0 & I_n\\ \hline
            0 & X_0^T & I_m & 0 & X_0^T & 0\\
            0 & 0 & 0 & L_t & I_n & 0\\
            0 & 0 & X_0 & I_n & 0  & 0\\ \hline
            e^{tH} & I_n& 0 & 0 & 0 & 0
        \end{array}
    \right)
\end{equation}
Which yields the set of equations:
\begin{align*}
    h_{11} & = -\traceLim[n]{
        \left(L_t + \frac{1}{h_{33}} I_n\right)
        \left(e^{2tH} + z L_t + \frac{z}{h_{33}} I_n\right)^{-1}
    } \\
    h_{33} & =
    1 - 
    \frac{1}{\psi} \traceLim[n]{
        \left(z L_t + e^{2tH}\right)
        \left(e^{2tH} + z L_t + \frac{z}{h_{33}} I_n\right)^{-1}
    }
\end{align*}
Using the contour integration technique, we obtain:
\begin{equation}
    h_{11} = \frac{-1}{2\pi i} \oint_{\Gamma}
    -\frac{
        \frac{1}{h_{33}} + \int_0^t 2e^{2sx} \dd s
    }{
        \frac{z}{h_{33}} + e^{2tx} + z \int_0^t 2e^{2sx} \dd s
    } P(x) \dd x
\end{equation}
which is reduced to:
\begin{equation}
    h_{11}(z) = \frac{-1}{2\pi i} \oint_{\Gamma}
    -\frac{
        1 + e^{-2tx} (\frac{x}{h_{33}} - 1)
    }{
        x + z + z e^{-2tx} (\frac{x}{h_{33}} -1)
    } P(x) \dd x
\end{equation}
Similarly for $h_{33}$:
\begin{equation}
    h_{33}(z) = 1 + \frac{1}{\psi} \frac{-1}{2\pi i} \oint_{\Gamma}
    -\frac{
        (x + z - z e^{2tx}) P(x) \dd x
    }{
        x + z + z e^{-2tx} (\frac{x}{h_{33}} -1)
    } 
\end{equation}
Two possible ways to retrieve $p_t$ from $h_{11}$ and $h_{33}$: either with $\phi p_t = \frac{-1}{2\pi i} \oint_\Gamma z^2 h_{11}(z) \dd z$, or $\phi p_t =  - \frac12 \frac{\partial^{(2)}}{\partial z^2} \left( \frac{1}{z} h_{11}(\frac{1}{z}) \right) \rvert_{z=0}$.  In both cases, there is an additional level of complexity in terms of calculation as it either requires a double-contour integration, or computing derivative and second derivative of the given functions yielding further new equations.

\subsection{Quantities $Q(z),P(z)$}
There remains to calculate the terms $Q(z),P(z)$ which depends only on the random matrices $X^*, \xi$ and can be done altogether with the linear-pencil:
\begin{equation}
    M_{z} = \left(
        \begin{matrix}
            I_n & X^* & 0 & 0\\
            0 & I_d & X^{*T} & 0\\
            0 & 0 & (z + \mu) I_n- \frac{1}{\sqrt \lambda} \xi & X^*\\
            0 & 0 & X^{*T} & I_d
        \end{matrix}
    \right)
\end{equation}
Using the kernel $K = (H-zI_n)^{-1}$, we can calculate the inverse:
\begin{equation}
    M_z^{-1} = \left(
        \begin{matrix}
            I_n & -X^* & -Z^*K & Z^* K X^*\\
            0 & I_d & X^{*T} K & -X^{*T} K X^*\\
            0 & 0 & -K & K X^*\\
            0 & 0 & -X^{*T} K & I_d - X^{*T} K X^*
        \end{matrix}
    \right)    
\end{equation}
So that $Q(z) = -f_{13}$ and $P(z) = f_{33}$ where we $f$ is the analog of $g$ and $h$ with the former linear-pencils. In particular we expect the following structure:
\begin{equation}
    f = \left(
        \begin{matrix}
            1 & 0 & -\phi Q(z) & 0 \\
            0 & 1 & 0 & - Q(z) \\
            0 & 0 & -P(z) & 0 \\
            0 & 0 & 0 & 1-Q(z)
        \end{matrix}
    \right)
\end{equation}
We can further compute the fixed point equation with:
\begin{equation}
    \eta(f) = \left(
        \begin{matrix}
            0 & 0 & f_{22} \phi + f_{24} \phi & 0\\
            0 & f_{31} & 0 & f_{33}\\
            0 & 0 & \frac{f_{33}}{\lambda} + f_{42} \phi + f_{44} \phi & 0\\
            0 & f_{31} & 0 & f_{33}
        \end{matrix}
    \right)
\end{equation}
After some algebraic reductions, we obtain the degree 3 polynomials given in equation \eqref{eq:Q_and_P}. In general, these equations have multiple solutions but only one corresponds to the analytic solution associated to the appropriate trace of resolvent.