% Formerly part of Lemma 3
Consider $(Z)_{i,:}$, the $i^\th$ row of $Z$. For all rows, there are exactly two nonzero entries $(Z)_{i,j_1}$, $(Z)_{i,j_2}$, both equal to $0.5$. We can determine $j_1$ and $j_2$ for $i$ as follows; first, choose $n$ such that $nN_x+1 \leq i \leq (n+1)N_x$. 

If $n$ is even, then
$\{j_1, j_2\} = \{(\frac{n}{2}+1)N_x+i, (\frac{n}{2}+2)N_x+i\}$
    
If $n$ is odd, then:
$\{j_1, j_2\} = \{\frac{n-1}{2}N_x+i, \frac{n+1}{2}N_x+i\}$

% Former example

\subsection{Analysis of trajectory sets}

Theorem \ref{thm:equal_trajsets} can easily be checked in code; \Red{it forms the basis for an algorithm for local region size selection in the next section.} In the remainder of this section, we aim to gain a deeper understanding of what conditions lead to equality and inequality of the trajectory sets. For simplicity, we begin with the case of $T=1$.

\begin{equation}
\begin{aligned}
    \Z &= \begin{bmatrix} I & 0 & 0 \\ -A & I & -B
    \end{bmatrix} \\
    \Zi &= \begin{bmatrix} I & 0 \\ CA & C \\ -B^\top CA & -B^\top C
    \end{bmatrix}
\end{aligned}
\end{equation}
where $C = (I+BB^\top)^{-1}$. Then,

\begin{equation}
    Z^h = \begin{bmatrix} 0 & I-C & CB \\ 0 & B^\top C & I-B^\top CB
    \end{bmatrix}
\end{equation}

For pedagogical reasons, we will first analyze the case of $B=I$, then extend the analysis to more general $B$. For $B=I$, we have
\begin{equation} \label{eq:z2}
    Z^h = \frac{1}{2} \begin{bmatrix} 0 & I & I \\ 0 & I & I
    \end{bmatrix}
\end{equation}

\begin{proposition} \label{prop:z_properties}
\textit{(Matrix structure)} For $T=1, B=I$, the following are true:
\begin{enumerate}
    \item The $i^\th$ row of $\Zb$ has exactly two nonzero values, located at columns $j_1$ and $j_2$. $(Z^h)_{i,j_1} = (Z^h)_{i,j_2} = \frac{1}{2}$
    \item Consider the $i^\th$ row of $\Zb$. There exists exactly one value of $k \neq i$ such that the the two rows are equal, i.e. $(Z^h)_{i,:} = (Z^h)_{k,:}$
    \item The $j^\th$ column of $\Zb$ has exactly two nonzero values.
    \item Consider the $i^\th$ row of $\Zb$ with nonzero values located at columns $j_1$ and $j_2$. It is true that $(Z^hX)_{:,j_1} = (Z^hX)_{:,j_2}$
\end{enumerate}
\end{proposition}
\begin{proof}
These follow directly from the structure of $Z^h$, $X$, and $\Zb$.
\end{proof}

\begin{proposition} \label{prop:rank_preserving_cols}
\textit{(Rank-preserving columns)} Let $\Cfrak$ be a set of column indices for $Z^hX$.
Define the set
\begin{equation*}
\begin{aligned}
\Crank(k) = \{ nN_x + k : \mod(n,3) \neq 0 \\ n=0 \ldots \nPhi \}
\end{aligned}
\end{equation*}

For $T=1$, $B=I$, $\rank(Z^hX) = \rank(Z^hX)_{:,\Cfrak}$ if and only if $\forall k \in \{ 1 \ldots N_uT \}, \exists j \in \Cfrak \cap \Crank(k)$.
\end{proposition}
\begin{proof}
This follows from the structure of $Z^h$ per \eqref{eq:z2} and of $X$ (Definition \ref{defn:augmented_state}). All columns corresponding to indices in $\Crank(k)$ are scalar multiples of one another; thus, to preserve rank, at least one column corresponding to each $k$ needs to be preserved. The modulo condition arises from the fact that the first block column of $Z^h$ (and all its column replicates, which result from postmultiplying by $X$) are zero.
\end{proof}

Proposition \ref{prop:rank_preserving_cols} will be useful in assessing the dimension of $\Ycal^h_\Lcal(X_0)$, since post-multiplying $Z^hX$ by $(I-\Fi F)$ has the effect of replacing certain columns of $Z^hX$ with zeros, as we show next.

\begin{lemma} \label{lemm:post_select_column} \textit{(Post-multiplication selects columns)} Let $\Cnonzero$ be the set of columns in $F$ with at least one nonzero value, i.e. 
\begin{equation}
\begin{aligned}
    \Cnonzero & = \{ j: \exists i, (F)_{i,j} \neq 0 \} \\
    & = \{ j: \exists i \in \Lfrak, (Z)_{i,j} \neq 0 \}
\end{aligned}
\end{equation}
For $T=1, B=I$, we have that ~\\
$Z^hX(I-\Fi F)_{:,j} = 
\begin{cases}
0 & j \in \Cnonzero \\
(Z^hX)_{:,j} & \text{otherwise}
\end{cases}$
\end{lemma}
\begin{proof}
Define the indices of unique rows of $F$:
\begin{equation}
    \Runique = \{ i: \forall k \neq i, (F)_{i,:} \neq (F)_{k,:} \}
\end{equation}
Note that if a row is not unique in $F$, then by  Proposition \ref{prop:z_properties}-2, there must be exactly one other row that is identical to it. We can construct $\Fi$, the pseudoinverse of $F$, as follows, and confirm that $F\Fi = I$
\begin{equation}
    (\Fi)_{:,i} = 
    \begin{cases}
    2(F)_{i,:}^\top & i \in \Runique \\
    (F)_{i,:}^\top & \text{otherwise}
    \end{cases}
\end{equation}

Define pair of column indices are nonzero at the same rows, i.e.
\begin{equation}
    \Cpaired = \{ (j_1, j_2): \exists i, (F)_{i,j_1} = (F)_{i,j_2} = \frac{1}{2}\}
\end{equation}

By points 1-3 listed in Proposition \ref{prop:z_properties}, we have
\begin{equation}
    (\Fi F)_{i,j} =
    \begin{cases}
    \frac{1}{2} & (i,j) \in \Cpaired \quad \text{or} \quad i=j \in \Cnonzero \\
    0 & \text{otherwise}
    \end{cases}
\end{equation}

Then, we see that
\begin{equation}
    (I-\Fi F)_{:,j} = 
    \begin{cases}
        \frac{1}{2}e_j - \frac{1}{2}e_i & j \in \Cnonzero \\
        e_j & \text{otherwise}
    \end{cases}
\end{equation}
where $i$ is such that $(i,j) \in \Cpaired$ (such an $i$ is guaranteed to exist for all $j$; see Proposition \ref{prop:z_properties}-1), and $e_i$ and $e_j$ are the $i^\th$ and $j^\th$ basis column vectors. Clearly, if $j \notin \Cnonzero$, $Z^hX(I-\Fi F)_{:,j} = Z^hX_{:,j}$, as desired. If $j \in \Cnonzero$, then $Z^hX(I-\Fi F)_{:,j} = \frac{1}{2}Z^hX_{:,j} - \frac{1}{2}Z^hX_{:,i} = 0$ by Proposition \ref{prop:z_properties}-4.
\end{proof}

\begin{theorem} \label{thm:equality_t1bI} \textit{(Equality of trajectory sets)} For $T=1, B=I$, $\Ycal_\Lcal(x_0) = \Ycal(x_0)$ if and only if the following holds: 
\begin{equation} \label{eq:equality_t1bI}
\begin{aligned}
& \forall k \in \{1 \ldots N_uT \}, \\ 
& \exists j \in \Crank(k) \quad \text{s.t.} \quad \forall i \in \Lfrak, (Z)_{i,j} = 0
\end{aligned}
\end{equation}
\end{theorem}
\begin{proof}
First, note that \eqref{eq:equality_t1bI} is equivalent to the following:
\begin{equation*}
\forall k \in \{1 \ldots N_uT \}, \exists j \in \Crank(k) \quad \text{s.t.} \quad j \notin \Cnonzero
\end{equation*} 
Applying Lemma \ref{lemm:post_select_column}, we see that this is equivalent to $Z^hX(I-\Fi F) = (Z^hX)_{:,\Cfrak}$, where $\forall k \in \{1 \ldots N_uT \}, \exists j \in \Crank(k) \cap \Cfrak$. Applying Proposition \ref{prop:rank_preserving_cols} gives the desired result.
\end{proof}

We now give a simple example of a case in which the local and unconstrained trajectory sets are equal. Let $T=1, B=I$. Consider a spanning graph embedded in a square mesh network. We constrain each node to only communicate to its immediate neighbors, i.e. nodes $i$ and $j$ may only communicate if $e_{ij} \in \Ecal$\footnote{This corresponds to neighborhood size $d=1$ in the SLS literature}. For dense $x_0$, the local and unconstrained trajectory sets are equal.

We now sum up our findings so far from this pedagogical exercise of $T=1$. The trajectory set is an image space of some matrix (Lemma \ref{lemm:yhset_image}). The localized trajectory set is an image space of the same matrix, but with some columns replaced by zeros (Lemmas \ref{lemm:yhlset_image}, \ref{lemm:post_select_column}). Due to the structure of this matrix (Propositions \ref{prop:z_properties}, \ref{prop:rank_preserving_cols}), a number of columns can be replaced by zeros without reducing the rank of the matrix. If rank is not reduced, then the localized trajectory set is equal to the trajectory set without locality constraints, and the inclusion of locality constraints does not introduce performance degradation.

The above analysis can be extended to the case of $T=1$, $B$ quasi-diagonal.

\begin{definition}
Let $m < n$. $B \in \Rbb^{n \times m}$ is \textit{quasi-diagonal} if $\exists$ diagonal matrix $D \in \Rbb^{n \times n}$ such that $B = (D)_{:,\Cfrak}$
\end{definition}

For any system where each state has no more than one actuator, and each actuator acts on no more than one state, we can index the actuators appropriately such that actuation matrix $B$ is quasi-diagonal. In this case, we can extend Proposition \ref{prop:z_properties}

\begin{proposition} \label{prop:z_properties_quasidiag}
\textit{(Matrix structure)} For $T=1, B$ quasi-diagonal, the following are true:
\begin{enumerate}
    \item The $i^\th$ row of $\Zb$ has exactly two nonzero values.
    \item Consider the $i^\th$ row of $\Zb$. There exists exactly one value of $k \neq i$ such that $(Z^h)_{i,:}$ and $(Z^h)_{k,:}$ have the same sparsity pattern, i.e. nonzeros are located at the same indices. Furthermore, $(Z^h)_{i,:} = \alpha (Z^h)_{k,:}$ for some scalar constant $\alpha$.
    \item The $j^\th$ column of $\Zb$ has exactly two nonzero values.
    \item Consider the $i^\th$ row of $\Zb$ with nonzero values located at columns $j_1$ and $j_2$. It is true that $(Z^hX)_{:,j_1} = \alpha (Z^hX)_{:,j_2}$ for some scalar constant $\alpha$
\end{enumerate}
\end{proposition}

Compared to the case of $B=I$, we see that sparsity properties are preserved, though more numerical bookkeeping is required. Nonetheless, we can analogously extend Proposition \ref{prop:rank_preserving_cols}, Lemma \ref{lemm:post_select_column}, and Theorem \ref{thm:equality_t1bI} to quasi-diagonal $B$. The punchline is the same; there are groups of rank-preserving columns in $Z^hX$, and post-multiplying $Z^hX$ by $(I-\Fi F)$ \Red{(i.e. adding local communication constraints)} has the effect of replacing certain columns of $Z^hX$ with zeros. If, after post-multiplication, sufficient rank-preserving columns have not been replaced with zeros, then the locality constraints do not introduce performance degradation in the overall MPC problem. 

For longer horizon sizes, we are no longer able to analytically write out all relevant expressions as we do above. However, similar ideas apply; the structure of $Z^hX$ necessarily contains columns that can be replaced with zeros or otherwise altered, without affecting rank. \Red{Something like, we will use this idea for the algorithm in the next section.}

Note that in the $T=1$ case, $A$ does not affect the analysis of trajectory spaces; this is not true for longer horizons. In general, $A$ and $B$ represent the interconnection topology of the system, and play a part in determining the communication topology. 