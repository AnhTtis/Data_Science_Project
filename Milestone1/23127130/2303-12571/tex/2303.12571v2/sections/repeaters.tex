%twin-ax repeaters
To overcome attenuation due to the up to 6.25\,m twin-ax cable length in the sTGC trigger path (see Table\,\ref{tab:twinaxAttenuation}), repeaters are placed roughly midway along the twin-ax signal paths to guarantee error-free operation.
A 4.8\,Gb/s signalling rate requires good transmission of several odd harmonics of the signalling rate.
The 640\,MHz LVDS transmission is less sensitive to the high frequency attenuation, but does suffer from the attenuation of the fine wires (30\,AWG) in the cable.
The 4.8\,Gb/s serial repeaters restore the signals from the pad-TDS to the Pad Trigger and from the strip-TDS to the Router;
640\,Mb/s parallel LVDS repeaters restore signals from the Pad Trigger to the strip-TDS.
See Figure\,\ref{fig:LL_NSW_ElxOvr} for their locations.
For more details see\,\cite{repeaters}.
%? \red{Can we afford space for pictures?}

% Table generated by Excel2LaTeX from sheet 'Sheet1'
\begin{table}[h]
  \centering
  \caption{Attenuation versus frequency for the  ``MiniSAS'' twin-ax ribbon cable (3M~SL8800 Series MiniSAS cables) from\,\cite{twin-ax}. Silver-plated twin-ax cables were finally preferred because they have reduced attenuation at higher frequencies and lower prices. Since a large number of tin-plated cables were already in stock  they were used in the connection from Pad Trigger to sFEB (640\,Mb/s).}
    \begin{tabular}{lrrrrrrr}
    \toprule
    {\textbf{Frequency (GHz)}} & \multicolumn{1}{c}{\textbf{0.5}} & \textbf{1.0} & \multicolumn{1}{l}{\textbf{2.0}} & \multicolumn{1}{l}{\textbf{5.0}} & \multicolumn{1}{l}{\textbf{10.0}} & \multicolumn{1}{l}{\textbf{15.0}} & \multicolumn{1}{l}{\textbf{20.0}} \\
    \midrule
    {Tin plating (dB/m)} & -0.90 & -1.4  & -2.2  & -4.0  & -7.5  & -10.9 & -14.6 \\
    {Silver plating (dB/m)} & -0.85 & -1.2  & -1.7  & -3.2  & -4.9  & -6.8  & -8.8 \\
    \bottomrule
    \end{tabular}%
  \label{tab:twinaxAttenuation}%
\end{table}%


%\subsubsection{4.8 serial repeaters}
\para{4.8\,Gb/s serial repeaters}
\label{sec:serialRepeaters}
Tests showed that for error-free operation with a safe margin, serial repeaters are necessary for twin-ax cable lengths beyond 4\,m.
A serial repeater for a single 4-pair cable is housed in a small shielded copper box compatible with the width of the cable for easy mounting inline with the cables.
They are powered from a nearby L1DDC.
The quad-channel serial repeater chips\,\cite{ds100br410} are the same as those used as signal conditioners in the Pad Trigger and Router.
Receive equalization, transmit de-emphasis and transmit voltage can all be set by means of pin jumpers.
PRBS test data transmitted from the TDS ASICs showed a bit error rate less than $10^{-14}$.
Different sets of parameters and all cable lengths combinations were used to select the parameters with the lowest transmission error rates.
It was possible to select a common setting for all cable combinations.
A thermal simulation, and subsequent tests using temperature probes, confirmed that the repeater chip was adequately cooled by the copper box.
Power consumption at 2.5\,V is 213\,mW per repeater board. A total of 880 serial Repeater boards were built.
%\red{Do we want an eye-diagram?}

%\subsubsection{LVDS repeaters}
\para{LVDS repeaters}
\label{sec:LVDSRepeaters}
A 6.25\,m twin-ax cable carries the seven serial 640\,Mb/s LVDS lines from the Pad Trigger to the strip-TDS.
The eye-diagram of a test at 640\,Mb/s with a 5\,m cable showed attenuation and an 8\,m cable was unstable with barely an eye.
Consequently, to be safe with a 6.25\,m cable, it was decided to install repeaters.
Single channel Micrel SY58605U 3.2\,Gbps Precision LVDS buffers\,\cite{SY58605U} were used to regenerate the signals.
The high power consumption (a bit less than 1\,W per 7-bit connection) required active cooling.
The LVDS repeaters were located on the spokes of the Wheel, behind the Large Sectors, where cold water for cooling was available.
Boards with six repeaters each, are powered with 2.5\,V from a FEASTMP pluggable module DC-DC converter\,\cite{FEASTMP}.
Two cooling bars are soldered on either side of a copper pipe, carrying cold water.
Two boards are mounted on either side, in thermal contact with the copper bars.
All four boards are enclosed in a common metallic shielding.
A total of 144 LVDS Repeater boards were built.

%\red{All the repeaters were tested in a loop back configuration in a setup based on the Xilinx VC707 platform and a custom mezzanine card.
%The yield was higher than 99\%.-- This is QAQC, remove.}