%\section{Boards}
\label{sec:boards}
The electronics boards developed for the NSW upgrade have several commonalities:
%\begin{itemize}\itemsep-4pt
\begin{itemize}[topsep=2pt, itemsep=0pt, parsep=1pt]
\item DC (non-isolated) Point-of-Load power from FEAST ASICs\,\cite{FEAST2.1} or FEAST modules\,\cite{FEASTMP}
\item SCA ASIC for configuration and monitoring
\item LVDS (Low Voltage Differential Signalling) or Scalable Low Voltage Signaling (SLVS)\cite{slvs} used for most on-board chip-to-chip communication
\item \gls{MiniSAS} twin-ax ribbon cable\,\cite{twin-ax,8F36} used for board-to-board communication, carrying SLVS, LVDS and 4.8\,Gb/s serial signals.
           In all, there are about 40\,km of MiniSAS twin-ax cables.
\item Active water cooling of all boards (except the Serial Repeaters). To prevent leaks, the system runs with less than atmospheric pressure.
\end{itemize}

\subsection{\MM Front-end board -- MMFE8}
\label{sec:mmfe8}

The \gls{MMFE8} board is the front-end electronics for \MM detectors. It is the interface between the detector, the trigger (ADDC), and data acquisition (L1DDC) electronics. Due to the high number of readout channels on the \MM detector ($\sim$2.1\,M), 4096 MMFE8 boards are needed, each handling 512 channels. The board must meet demanding space and electrical requirements and constraints. The board dimensions are 215\mm in length by 60\mm wide. It is comprised of 14 electrical layers and is 2.54\mm thick.  A block diagram of the MMFE8 is shown in Figure\,\ref{fig:mmfe8_schematics}.

The interface to the detector is achieved via two \gls{ZEBRA}\textsuperscript{\textregistered} elastomeric connectors\,\cite{zebra}.  The type and layout of the connectors was carefully studied. The pitch on the board was chosen to be 400\um with a contact width of 200\um which is compatible with the detector pitch.
The connector has six lines of through wires per strip which ensure contact between the detector strips and the pads on the MMFE8 PCB.
A precision machined slot between the two connectors aligns the board to the detector.
The compression of $\sim$150\um is also regulated with mechanical cams on top of the board where a ground strip connects the board analogue ground to the detector ground through six contacts. On either side of the connector, four lines provide geographical information through an on-detector encoding\footnote{It was unfortunately found during integration that due to wrong connectivity of the pins in the MMFE8 board, the geographical decoding could not be achieved.}.

The VMM inputs on the board are protected through a dedicated diode network which was extensively tested and was found to protect the ASIC from any discharges on the detector\,\cite{Iakovidis:2675779}.  On each channel, a Transient Voltage Suppressor (\gls{TVS}), a Semtech \textmu Clamp\,\cite{semtech}, is able to mitigate fast transients.
Following the TVS diodes, a 10\,$\Omega$ resistor connects to one of the four channels of a TVS Diode Array, SP3004\,\cite{sp3004}.  This diode array mitigates slower input transients.
Another 10\,$\Omega$ resistor is placed after the SP3004\footnote{Early in the project, the NUP4114-D was used. It was discovered that the functional details of this protection diode were modified between 2012 and 2014, but without changing the part number. Its new functionality was not adequate to the NSW needs, so the SP3004 was used instead. }.

\begin{figure}[t]
\centering
\includegraphics[width=1.0\textwidth]{figures/GI_mmfe8_schematic_v2}
\caption{The MMFE8 block diagram showing the \gls{ESD} protection,
         the VMM Front-end ASICs, the Readout Controller ASIC, the Slow Control Adapter ASIC, the FEAST DC-to-DC converter ASICs, and the MiniSAS twin-ax connectors for the different interfaces.}
\label{fig:mmfe8_schematics}
\end{figure}

																																																																																																																																																																																																																														The board contains eight VMM ASICs, one ROC ASIC, one SCA ASIC and three FEAST ASICs. The on-board FEAST ASICs provide power to all the ASICs. Two FEAST ASICs set for 1.3\,V\footnote{Studies showed that the VMM must be provided with at least 1.2\V and therefore the 1.3\V ensures that voltage by compensating for small voltage drops along the lines.} provide power to the eight VMM's analog section (one FEAST per four VMMs).  One FEAST provides 1.2\V to the digital supplies of all the VMMs and also to the ROC and SCA ASICs\footnote{The SCA ASIC is designed to function at 1.5\,V. Discussions and validation tests showed that the SCA can function properly at 1.2\V with negligible impact on the integrated ADC performance. This was done to avoid integrating another FEAST on the board due to space constraints.}.
The SCA provides a unique 32-bit ID for the board.
The input voltage of 11\,V is provided through the Low Voltage Distributor Board (LVDB) described in Section\,\ref{sec:powerDistribution}.
The power consumption of the board was measured to be $\sim$16\,W.


The interface to both ADDC and L1DDC boards is realised through MiniSAS cables and connectors\,\cite{MiniSASconnector,twin-ax,8F36}. The connector to the ADDC carries the eight ART signals. The connection to the L1DDC carries the TTC signals and bunch crossing clock to the MMFE8 as well as the four data lines to the GBTx (one per SROC). The SCA E-link is carried as well in the same connector, removing ambiguity as to which ROC and VMMs are configured and reset by the SCA. The on-board ROC provides dedicated trigger, test pulse, reset and various clock signals to each VMM.

The design of the MMFE8 was iterated such that it almost reached the theoretical noise levels with respect to the ones defined by the VMM for the \MM detector capacitance, which is estimated to be $\sim$150\,pF/m\,\cite{Iakovidis:2675779}. To achieve that, the FEASTMP2.1\cite{FEASTMP} design was adopted.

\subsection{sTGC Front-end boards -- sFEB and pFEB}
\label{sec:sfeb_pfeb}

The strip and pad Front-end boards\,\cite{Miao_2020} interface to detector strips, pads and wire groups, providing their data to the readout path on receiving a Level-1 trigger signal and to the trigger path on every bunch crossing.
There are separate radiation-tolerant Front-end boards for strips and pads+wires.
Both are the result of several demonstrator and prototype boards, including those with earlier versions of the VMM and those with FPGA readout instead of the Readout Controller (ROC) ASIC.
VMM ASICs (see Section\,\ref{sec:VMM}) provide data to both trigger and readout paths.
An SCA ASIC provides a unique 32-bit ID for the board.

For each gas gap, detector signals connect to adapter boards on each of the two radial edges of the detector.
A strip Front-end board (sFEB) reads out strips from one edge, and a pad+wire Front-end board (pFEB) reads out pads and wires from the other.
In total, there are about 30 wire groups, up to 112 pads, and up to 400 strips reading out an sTGC gas gap.
The adapter boards route the signals to high-density (10\,$\times$\,30 contacts) low-profile matrix interposers (Samtec GFZ)\,\cite{GFZ} (one for pads, two for strips) on the adapter board to which the Front-end boards connect.

Careful placement, layout and shielding of the on-board FEAST DC-DC power converters was essential and is described in\,\cite{Miao_2020}.

\para{Input transient voltage suppression and signal conditioning:}
Gas detectors have high voltage discharges and the front-end VMM ASIC inputs must be protected.
Each of the VMM inputs on the board is protected through a dedicated diode network which was extensively tested and was found to protect the ASIC from any discharges on the detector\,\cite{Iakovidis:2675779}.
Figure\,\ref{fig:TVS_Pi_blockDiagram} shows the input circuitry that protects against transients and compensates for the high rate and high charge signals for strips, pads and wires.
\begin{figure}[h]
\centering
\includegraphics[width=0.75\textwidth]{figures/GI_pfeb_sfeb_schematic.pdf}
%\includegraphics[width=0.93\textwidth]{figures/PM_TVS_Pi_blockDiagram.pdf}
\caption{The sTGC FEB analog input circuit which protects against transients and compensates for high rate and high charge signals for pads\,(blue, left), wires\,(red, middle) and strips\,(yellow, right).}
\label{fig:TVS_Pi_blockDiagram}
\end{figure}
The first stage is a Transient Voltage Suppressor (TVS), a Semtech \textmu Clamp\,\cite{semtech}, which can mitigate fast transients.
The last stage is a 10\,$\Omega$ resistor that connects to one of the four channels of a diode array, SP3004\,\cite{sp3004}.
This diode array mitigates slower input transients.
In between, are schemes that compensate for the sTGC high rate and high charge signals, which include a level of uncertainty. Different schemes are needed for wires, pads and strips\,\cite{pi-networks}.

The amplitude and time structure of the sTGC signal require the use of an attenuator circuit in the case of the pads and wires, and a pull-up resistor circuit in the case of the strips, in order for the VMM to operate optimally under the conditions of the HL-LHC.

\vspace{3pt}\noindent\textit{Pad and wire input $\uppi$-networks:}
At the operational voltage of the sTGC, a single cavern background hit can induce a 50\,pC charge into a single sTGC pad, which exceeds the design requirements of the VMM front-end\footnote{At the time of the VMM design the input signal of the sTGC pads was underestimated and the input current would exceed the specifications.}.
This charge is sufficient to saturate the VMM feedback currents (configurable), causing an undesired and overly long recovery time.
To minimize this effect, a $\uppi$-network was implemented\,\cite{pi-networks} and is shown in Figure\,\ref{fig:TVS_Pi_blockDiagram} (opposite polarity for wires and pads).
The network acts as a charge divider, reducing the charge input to the ASIC front end.
The $\uppi$-network capacitor value was optimized to preserve efficiency and minimize recovery time.
Due to the difference in the pad capacitance along the detector radius, a different capacitor is used for the inner, middle, and outer quadruplet.
This circuit is directly implemented on the Front-end boards.
The optimization was based on radiation data (test-beam plus backgrounds), simulation, and charge injection studies.
In these studies, the optimal attenuation factor was found to be approximately 5:1, corresponding to 200\,pF, 330\,pF, and 470\,pF capacitors in the $\uppi$-network for inner, middle, and outer
%Q1, Q2 and Q3
pads FEBs respectively and 200\,pF for wire FEBs.

\vspace{3pt}\noindent\textit{Input pull-up resistor for strips:}
The signal formation of the sTGC detector is characterized by three components with different timescales:
The first component has a characteristic timescale of $\sim$20\,ns and is due to the electron avalanche drifting towards the wires.
The second component has a characteristic timescale of tens of microseconds and is due to the ion drift towards the cathode planes.
The third component has a characteristic timescale of milliseconds and is due to the charge induction within the resistive layer following the ion arrival to the layer,
and hence has the opposite polarity to the previous two\footnote{Sufficiently high resistance would neglect this component, but in the sTGC detector, in order to dissipate charge from the high background, it is not sufficiently high and a bipolar shaped signal is formed.}.
At the high rates of the HL-LHC, the second component from different signals overlap, causing a constant current into the VMM, which can be compensated by the Front-end, but strong bipolar shaped signals cannot.
Hence, although not necessary, a pull-up resistor was implemented for the strips, as shown in Figure\,\ref{fig:TVS_Pi_blockDiagram}.
In this case, the VMM baseline restoration circuit is protected by providing an additional constant feedback current.
A resistor of 400\,k$\Omega$ connected to supply voltage (1.2\,V nominal) was chosen.

\para{Strip front-end boards:}
The sTGC strip Front-end board is a dense 14-layer circuit board (27.5\,cm\,$\times$\,7.6\,cm) comprised of six or eight VMMs, one ROC, one SCA, six FEAST, and three or four strip-TDS ASICs.
The outer two of the three quadruplets have fewer strips and so require only six VMM and three strip-TDS ASICs to be populated on the board.
%The same board, rotated 180\textdegree\ around its z-axis, is used for both rising and falling edges. Therefore the radial coordinates of one are reversed from the other.
Boards mounted on one edge of the detector are rotated 180\textdegree\ around its $z$-axis with respect to the other edge.
Therefore the radial coordinates of one are reversed from the other.
Board power consumption is 21\,W.
Figure\,\ref{fig:sFEB_block_V01} shows a block diagram of the board.

\begin{figure}[ht]
\centering
\includegraphics[width=1.0\textwidth]{figures/sFEB_block_V01.pdf}
\caption{Block diagram of the sTGC strip Front-end board that handles the strips in one sTGC gas gap, showing
         its ASICS, connectors and power blocks.
        % When configured for 640\,Mb/s, the readout E-links use a second twin-ax line.
        The second MiniSAS connector is used only for the inner quadruplet.}
        % the GFZ input connector, the ESD block, the VMM front-end ASIC, the strip TDS ASIC, the GBT-SCA ASIC,
        % the Readout Controller ASIC, the FEAST DC-DC converters,
        % the MiniSAS cable connectors for the connections to the Pad Trigger, Router and L1DDC,
        % and their interconnections.
        % The 2.5\,V power is supplied by a precision voltage reference.}
\label{fig:sFEB_block_V01}
\end{figure}

The seven serial LVDS lines from the Pad Trigger include four separate Band-id lines for the four strip-TDS ASIC positions on the board, a frame, BCID and a clock.
The BCID, clock and frame are distributed to the four strip-TDS ASICs by three 1-to-4 multiplexers.
Rather than an additional FEAST, three 2.5\,V precision voltage references, TL431AIDBZR\,\cite{TL431}, power them from the 10\,V supply.

\para{Pad+wire front-end boards:}
The pad+wire Front-end board is a dense 12-layer circuit board (16.3\,cm\,$\times$\,7.6\,cm) comprised of three VMMs, one ROC, one pad-TDS, one SCA and three FEAST ASICs.
The sTGC pads connect to two of the VMMs, wire groups to the third VMM.
%The same board, rotated 180\textdegree\ around its z-axis, is used for both rising and falling edges. Therefore the wire azimuthal coordinates of one are reversed from the other.
Boards mounted on one edge of the detector are rotated 180\textdegree\ around its $z$-axis with respect to the other edge.
Therefore the wire azimuthal coordinates for half the layers are reversed from the other and the pad numbering differs for different layers.
Board power consumption is 9\,W.
Figure\,\ref{fig:pFEB_block_V01} shows a block diagram of the board.


\begin{figure}[ht]
\centering
\includegraphics[width=0.92\textwidth]{figures/pFEB_block_V01.pdf}
\caption{Block diagram of the sTGC pad Front-end board that handles the pads and wires in one sTGC gas gap, showing
         its ASICS, connectors and power blocks.}
\label{fig:pFEB_block_V01}
\end{figure}

%\FloatBarrier

%%%%%%%%%%%%%%%%%%%%%%%%%%%%%%%%%%%%%%%%%%%%%%%%%

%\begin{comment}
%
%
%pad-FEB: The pad-FEB will be responsible for the readout of both pads and wires.
%Trigger path: The output of the pad-TDS is routed to a Serial Attachment (SATA) connector which is connected to the Pad Trigger Board via a 4:1 SATA-to-MiniSAS cable.
%Since the wires do not participate in the trigger, the wire VMM direct outputs are not used.
%Refs to VMM and TDS section.
%
%Readout path:
%ref to VMM section
%
%TVS
%Pi network
%Noise, Shielding and EMI
%power: FEASTs +
%cooling
%
%pFEBs an sFEBs power consumption is 9\,W and 21\,W, respectively.
%
%FEAST circuits. Care was taken to move the FEAST circuits away from the detector signal traces.
%
%Pi networks.
%
%The noise study on the bench shows negligible levels of noise coupling from the FEAST DC-DC converters to the sensitive VMM analog circuit.
%Integration studies with the full-size sTGC quadruplets indicate no noticeable noise coupling from the digital circuit.
%The pick-up noise from the detector due to the FEAST switching current irradiation has been minimized to a negligible level
%thanks to the attentive layout and the effective shielding of the FEAST modules.
%
%The high demand to accommodate up to 512 analog channels with a dense input network in a very constrained space,
%minimize the conducted and irradiated noise from the onboard switching regulators and digital ASICs,
%and handle a large number of high-speed inter-connects among highly integrated ASICs,
%all bring orders of magnitude complexity as well as great challenges to the design
%comparing with the previous generation FEB's used in similar experiments.
%Moreover, special care is needed for the FEB's to deal with the harsh radiation and the magnet field environments at the same time.
%Detailed analysis and strategic discussions on how these challenges have been addressed are presented
%
%
%I. Ravinovich, The New Small Wheel background rates predictions based on CSC and 2 TGC Run-2
%measurements, ATL-COM-MUON-2016-006 (2016).
%
%\end{comment}


\subsection{Aggregation of several electrical links to optical links -- L1DDC}
\label{sec:L1DDC}
The Level-1 Data Driver Cards (L1DDC)\,\cite{Gkountoumis:2019ye, Gkountoumis:2779645, Gkountoumis:2016mrs, Gkountoumis:2016dfm} are high-bandwidth radiation-tolerant data aggregator boards based on the GBTx ASIC (Section\,\ref{sec:GBTx}).
On one side they connect through E-links with multiple front-end boards via twin-ax cables; on the other side, they connect through two or more fibres with FELIX.
To accommodate Phase\,2, a Front-end board can provide up to four 320\,Mb/s E-links for L1A readout data; only one is needed for the LHC Run-3.
The L1DDC is completely transparent to the data being transmitted or received.
MiniSAS cables make the connections to the front-end boards.
The optical interfaces are the VTRx and VTTx (see Section\,\ref{sec:VTRxVTTx}).
The boards are powered by FEAST DC-DC converters for 1.5\,V and 2.5\,V.
There are three variants of L1DDC's with the same basic functionality, but with different mechanics and channel counts.
One of them is part of the \MM electronics, and two of them of the sTGC electronics.

\para{The Rim-L1DDC:} Supports the Pad Trigger and eight Router cards in the sTGC trigger path electronics. Those electronics are in a dedicated crate called ``the Rim crate'' since it is located in the rim of the NSW structure.
The Rim-L1DDC consists of two independent boards, one primary and one auxiliary, sharing the same PCB.
Each includes one GBTx, one VTRx and nine MiniSAS connectors. This provides redundancy to the system since a complete sTGC trigger sector depends on it.
The Pad and Router boards are connected to both primary and secondary sections. The on-board SCA's are able to switch on/off the Pad Trigger and Router boards by controlling the enable/disable signal of their FEASTs.
Tests of the Rim-L1DDC showed that its GBTx ASIC's 160\,MHz E-link clock jitter was marginally good as a reference clock for the Xilinx\,7-family FPGA transceivers of the Router and Pad Trigger.
For that reason, a dedicated clock is provided through an additional fibre to the Rim-L1DDC. This direct low jitter 160\,MHz clock (Section\,\ref{sec:DirectClock}) is received through an additional VTRx and is distributed to the Router and Pad Trigger boards via low jitter fan-outs.
The block diagram of the Rim-L1DDC is shown in Figure\,\ref{fig:LL_Rim_E-links}.  The power consumption of the board was measured to be 8\,W.
%The block diagram of the Rim-L1DDC is shown in Figure\,\ref{fig:rimBlock}.


\begin{figure}[h]
\centering
\includegraphics[width=0.95\textwidth]{figures/LL_Rim_E-links_V02}
\caption{The two redundant sections of the Rim-L1DDC and their connections to the Router and Pad Trigger boards}
\label{fig:LL_Rim_E-links}
\end{figure}

%\begin{figure}[ht]
%\centering
%\includegraphics[width=1.0\textwidth]{figures/GI_RIML1BlockDiagram}
%\caption{Rim-L1DDC block diagram.}
%\label{fig:rimBlock}
%\end{figure}

\vspace{-10pt}
\para{sTGC-L1DDC:} interfaces with three Front-end boards, either strip or pad+wire. Two GBTx ASICs for two independent bi-directional (VTRx) links; only one is used for pad+wire FEBs. The second strip GBTx provides the extra bandwidth needed by Phase\,2.
In all, 21~E-links connect to a sTGC-L1DDC.
The board environment is monitored by an SCA ASIC.
In addition, by means of the SCAs second redundant E-link, both GBTx ASICs can communicate with the SCA.
The second GBTx though is configured through the \ItwoC of the SCA.
%?\red{LL: Incorrect? See Panos diagram: figures/LL_sTGC-L1DDC_GBTxConfig_v01.pdf}
The L1DDC's are placed on the upper edge of the sTGC quadruplet.
The power consumption of the sTGC-L1DDC is $\sim$4\,W\footnote{Higher consumption up to 6\,W is measured on the sTGC L1DDC but is due to the consumption of the repeaters which are powered through the L1DDC.}.

\para{\MM -L1DDC:}  interfaces with eight Front-end boards and one ADDC board. It features three GBTx ASICs: one is a full-duplex transceiver (VTRx) and two are simplex transmitters (VTTx) for the extra readout bandwidth needed by Phase\,2.
The GBTx connected to the VTRx is configured directly from its IC link.
The two GBTx ASICs connected to the VTTx  are configured via the on-board SCA ASIC.
The SCA also monitors the temperature and voltage levels of the board and the VTRx's Rx signal strength.
One E-link connects to the SCA on the ADDC board; another distributes the TTC signals (BCR and BC clock) to the ADDC board.
In all, 41~E-links connect to a MM-L1DDC.
One layer of \MM is readout by two L1DDC boards. The L1DDC's are installed along the edges of the outer layers of the \MM quadruplet.
The board power consumption is $\sim$5.5\,W.

%\FloatBarrier


\subsection{\MM trigger data serializer -- ADDC}
\label{sec:addc}
The Address in Real Time Data Driver Card (ADDC)\,\cite{ADDC, ADDCprod,8376567,9058723} features two ART ASICs, two GBTx ASICs, one GBT-SCA ASIC, one VTTx optical transmitter and two FEAST ASICs.
The ADDC block diagram is shown in Figure\,\ref{fig:block_ADDC}.
Each ART ASIC receives the 320\,Mb/s ART data from 32 VMMs on four MMFE8 front-end boards.
In total 64 VMM ART signals are driven through eight MiniSAS cables to the ADDC. Each ART ASIC then transmits the data to the onboard GBTx.
\begin{figure}[b]
\centering
\includegraphics[width=0.9\textwidth]{figures/GI_addc_architecture2}
\caption{The ADDC block diagram. Each ADDC receives 64 inputs of ART data using two ART ASICs. Output to the Trigger Processor is via two GBTx ASICs.}
\label{fig:block_ADDC}
\end{figure}

The hit selection within the two ART ASICs has been explained in Section\,\ref{sec:ART}. Each GBTx ASIC transmits the data then to the Trigger Processor through one transmission channel of the VTTx.  The configuration and control of the ART and GBTx ASIC is achieved by the on-board GBT-SCA ASIC. The ASICs are powered through two on-board FEAST ASICs configured at 1.5\,V and 2.5\,V respectively.

Since the ADDC features only two transmission optical lines to minimize the fibre connections, each ADDC is connected to the GBTx on one L1DDC through a  MiniSAS cable.  Through this connection the following lines are provided to the on-board ASICs:
\vspace{-5pt}
\begin{itemize}\itemsep-4pt
\item{SCA E-link: (three pairs,  BC clock, Tx/Rx data at 80\,Mb/s)}
\item{Two reference BC clocks, one for each of the GBTx ASICs}
\item{The BCR (Bunch Crossing Reset) signal}
\end{itemize}

\vspace{-6pt}\noindent Each of the GBTx ASICs is transmitting by itself the recovered and generated 40.079\MHz and 160.316\MHz clocks to each ART ASIC respectively.
In total 512 ADDC boards communicate with the 4096 front-end boards.

The ADDCs are placed on the edges of \MM detector along with the front-end boards (MMFE8) and the L1DDC boards.			 % "edges" is more descriptive than "sides".
The latency of the ADDC was measured to be $\sim$187\,ns, of which around 143\,ns is the latency of the GBTx plus VTTx dual optical transmitter module.

%\FloatBarrier

\subsection{sTGC Pad Trigger}
\label{sec:pad_trigger}

The sTGC Pad Trigger is a pre-trigger board that finds tracks passing through a pointing tower of logical sTGC pads.
See Figure\,\ref{fig:LL_PadSelect}.
Its purpose is to limit the on-detector trigger electronics to send out only the charge data from those bands of strips that pass through the pad tower coincidences.
Up to four towers can be found.
It receives the binary pad hits from 24 Front-end boards, three per layer, in a sector.
The band of 17 strips in each layer that pass through each tower is identified by a Band-id.
The Band-ids of triggered towers are sent to the relevant strip-TDS ASICs which then send the charge information of each strip in the band to the Trigger Processor via the Router.
One Pad Trigger per sector resides in the Rim Crate for that sector.
Its context and block diagram are shown in Figure\,\ref{fig:RV_Pad_block_schema}.
The internal blocks of the FPGA are shown in Figure\,\ref{fig:RV_Pad_Trigger_FPGA}.

\begin{figure}[ht]
\centering
\includegraphics[width=0.72\textwidth]{figures/RV_Pad_block_schema.pdf}
\caption{Pad Trigger context and block diagram.
         Not shown: The output of the Trigger block is also sent to the Readout block.}
\label{fig:RV_Pad_block_schema}
\end{figure}
\begin{figure}[ht]
\centering
\includegraphics[width=0.99\textwidth]{figures/RV_Pad_Trigger_FPGA.pdf}
\caption{Pad Trigger FPGA block diagram.
         Not shown: The output to the Trigger Processor is also read out to FELIX.}
\label{fig:RV_Pad_Trigger_FPGA}
\end{figure}

Serial repeater chips\,\cite{ds100br410} condition the 24 4.8\,Gb/s input links from the pad-TDS ASICs.
They are configured via the \ItwoC channels of the on-board SCA ASIC.
After deserialization, each vector of pad hits can be delayed in programmable steps of 4.17\,ns to align them in time.

\para{Band-finding algorithm:} Valid 3-out-of-4 pointing towers for each 4-layer wedge were defined using the ATLAS off-line simulation to track single muons through the sTGC layers of a sector.
The simulation's pad id were mapped to a Pad Trigger input-id.
For each tower, the list of pad input-ids (one per layer), the band-id and $\phi$-id were put in a list of trigger patterns.
There is one comparator for each of about 4700 trigger patterns that compares the input pads to the list of pads in each trigger pattern.
The match is done taking into account the 3-out-of-4 majority logic per each quadruplet.
The comparators are grouped by their corresponding Band-id.
A one or two bunch crossing coincidence window is determined by the choice of a bit file specifically generated for that choice.
The current choice is a two BC window.
% from Band-id\,6 to\,87. \red{What happened to Band-ids 1-5 and 88-92?}
% Some Band-ids cannot be be generated because not enough layers in the band due chambers have constant R boundairies and not constant theta,
% or not all pads are connected to pTDS channels (104 of 112).
% Also holes in the VMMs that prevent contiguous strips from being contiguous TDS channels. Each comparator group produces a ``1'' if there is a match, ``0'' otherwise.
The lists of trigger patterns for large and small sectors are defined in separate FPGA bit files.
An output vector of comparator results, one bit per band, is passed to a priority encoder which selects a maximum of four Band-ids to be sent to the strip-TDS ASICs on the sFEBs.
%An output vector of comparator results, one bit per band, is passed to a priority encoder which selects a maximum of four Band-ids and their respective  $\phi$-ids to be sent to the strip-TDS ASICs on the strip-FEB's.
The priority encoder also takes into account that a Band-id may need to be sent to two adjacent strip-TDS ASICs.
In this case the maximum number of Band-ids is reduced by one.
The selected Band-ids are finally directed to their corresponding strip-TDS via cable connections to each sFEB.
Note that each strip-TDS has a configurable lookup table that defines which strips belong to a given Band-id in that layer and that at most three of the four strip-TDS positions on a sFEB are populated.
The Pad Trigger output to each sFEB consists of up to four Band-ids (a maximum of one per TDS ASIC), the BCID, a frame and a clock, and is sent via seven serial LVDS lines (See Section\,\ref{sec:sTDS}.).
%The Pad Trigger output to a strip-FEB, consisting of up to three Band-ids, the BCID, the $\phi$-id, a frame and a clock, is sent via seven serial LVDS lines (See Section\,\ref{sec:sTDS}.).
The maximum of four Band-ids may be distributed over any of the up to ten strip-TDS ASICs in each layer.
The Band-ids, BCID, and $\phi$-id along with some flags are also sent to the Trigger Processor via two redundant serial links at 4.8\,Gb/s, 8b/10b encoded, via a dual VTTx optical transmitter
(See Section\,\ref{sec:VTRxVTTx}.).

Configuration parameters, partially listed in Table\,\ref{tab:PadTrigConfig}, are set through the SCA server via the Rim-L1DDC and the on-board SCA ASIC. The FPGA configuration of the Pad Trigger board can be realised either by the SCA server via the on-board SCA's JTAG channel or by a Xilinx programmer via a connector. This is configurable through on-board jumpers.


%\noindent\red{
%How is the $\phi$-id handled? \\
%Can we make the choice of one or two BC window optional? Perhaps with separate bit files?} this is configurable through a 5-bit register
%?how is phi-ID handled? Each pattern line comes with its phi-ID (from simulation) so the phi ID is easy picked up once the matching logic gives a yes
%?readout window how centered? There are two I2C programmable registers. One is the L1 latency, which is the time distance in BCIDs corresponding to the first event to be read. The other register is the readout window width, the number of older events to be read after the first one.

% Table generated by Excel2LaTeX from sheet 'Sheet1'
\begin{table}[htbp]
  \centering
  \caption{Partial list of Pad Trigger configuration parameters.}
           % More parameters will be added as the Pad Trigger developments progress.
    \begin{tabular}{l}
    \toprule
%    \textbf{Parameter} \\
%    \midrule
    BC counter offset  \\
    Enable readout on L1A  \\
    Enable readout on self-trigger   \\
    Readout window BC offset of the first BC to be read on a L1A   \\
    Readout window size,  in BC units (0\,=\,1\,BC, 1\,=\,2\,BC, ...) \\
    pFEB input delay in steps of 4.17\,ns (0\,=\,4.17\,ns, 1\,=\,8.33\,ns, ...) \,4 bits per input \\
    24-bit mask to force all pad hits of the corresponding pFEB to 0 \\
    24-bit mask to force all pad hits of the corresponding pFEB to 1 \\
    Enable Orbit Count Reset (OCR) mode (where idle state is forced \\ ~~~~~until OCR is received). See Section\,\ref{sec:BCsync}. \\
    \bottomrule
    \end{tabular}%
  \label{tab:PadTrigConfig}%
\end{table}%

In addition to the trigger path output, all inputs and outputs are saved in a latency buffer until a Level-1 Accept is received via the TTC system.
In response to the Level-1 Accept, data for a configurable BC window is sent out on an E-link to the swROD via FELIX.
The input vectors of pad hits are compressed using a sparse binary compression algorithm. See Chapter\,11.5 of\,\cite{salomon2010handbook}.
%\red{Is a hit map sent if the compression result would be bigger than the hit map?}

For redundancy, the Pad Trigger is connected to both of the primary and secondary Rim-L1DDC (See Section\,\ref{sec:L1DDC}.).
The SCA on the Pad Trigger is connected to both as well.
This allows the SCA to choose which Rim-L1DDC provides the 160\,MHz reference clock for the Pad Trigger's gigabit transceivers.
See Section\,\ref{sec:L1DDC}.
The choice of which Rim-L1DDC supplies the E-links is defined by the FPGA design, i.e.\ an FPGA bit file is specific to one Rim-L1DDC or the other.
% Perhaps a later firmware version will have a set multiplexers in the FPGA logic that select either the primary or secondary Rim-L1DDC E-links, programmable via the SCA \ItwoC.}

The Pad Trigger's components are cooled by conduction through heat-conductive gap pads to a copper plate that is water cooled.
Special attention was paid to cool the VTTx optical transmitters which are heat sensitive.
The Pad Trigger board consumes 25\,W from a 9.2\,V supply.
%\FloatBarrier




\subsection{sTGC Router}
\label{sec:router}

The sTGC Router\,\cite{HU2022167504} serves as a packet switch for routing sTGC strip charge information from the strip-TDS to the sTGC Trigger Processor.
It is implemented in a Xilinx FPGA\footnote{XC7A200TFFG1156} from the Artix-7 family.
There is one Router per layer for each sTGC sector.
There are four TDS ASIC positions on a strip Front-end board;
depending on the board location three or four are populated.
%Although all four TDS positions are connected to the Router, only three links will be alive.
A scheme was developed to maintain low and fixed-latency packet multiplexing through the Router\,\cite{RouterFixedLat2, RouterFixedLat1}.
The Router's latency from first bit in to first bit out is 97\,ns.
The Router is unaware of the ATLAS run state and does not send any data in response to a Level-1 trigger.
A context diagram is shown in Figure\,\ref{fig:RouterContext}.

\begin{figure}[h!]
   \centering
   \includegraphics[width=0.75\textwidth]{figures/RouterContext}
   \caption{Context diagram of the Router for a layer showing the four electrical inputs from each of the three strip Front-end boards in one layer of a sector and the four fibre outputs.
             }
   \label{fig:RouterContext}
   \end{figure}

On every bunch crossing the sTGC Pad Trigger chooses up to four bands of strips and sends their band-ids to the strip TDS ASICs that contain those bands.
The selected TDS ASICs transmit the strip data for the band of strips they contain to the Router.
The other strip TDS ASICs transmit a null packet.
The Router inputs are twin-ax electrical serial streams at 4.8\,Gb/s.
The input serial streams are deserialized, unscrambled and aligned to a common clock.
The Router then routes the up to four packets containing the strip charges for a chosen band to the Trigger Processor via the Router's four 4.8\,Gb/s output fibres.
Null packets (containing the sector-id, layer and fibre number) are sent out when there are less than four strip-charge packets.
These are also used in confirming the connectivity of the inputs.

The on-board serial repeater chips\,\cite{ds100br410} condition the 4.8\,Gb/s links from each strip-TDS.
Their parameters were optimized and set with soldered jumpers.
Miniature Transmitter (\gls{MTx}) optical transmitters\,\cite{Zhao:2016czy, Xiao:2016dvu} are used to drive the output fibres.
They are configured via an SCA \ItwoC master.

The on-board SCA  ASIC is used to control the Router.
The sector-id is set in the FPGA via the SCA \gls{GPIO}.
The twin-ax cables from the inner and middle radii front-end boards were made the same length.
The cable from the outer Front-end board is shorter by one or two clock equivalent periods.
In order to align all the inputs, the Router inserts a delay, either one (or two) 160\,MHz clocks, for signals from the outer front-end boards of the large (or small) sectors.

The Router FPGA can be configured from on-board Flash memory or by FELIX via the \gls{JTAG} port of the SCA.
The Router's components are cooled by conduction through heat-conductive gap pads to a copper plate that is water cooled.

\para{Radiation tolerance}

The Router's tolerance to total ionization dose was shown acceptable and reported in\,\cite{RouterRadTol, HU2022167504}.
SEU's in the FPGA fabric or its configuration memory can cause malfunction or halting of the Router operation.
The critical logic sections that handle control and initialization are protected by Triple Modular Redundancy (TMR)\,\cite{TMR}.
The goal is to retain a steady data flow, while allowing bit upsets in the data stream to be mitigated by using the redundancy of multiple detector layers.

Configuration memory bits are protected with Error-Correction Code (\gls{ECC}) and Cyclic Redundancy Check (\gls{CRC}).
However, SEU's may accumulate in essential bits and result in functional failures.
Upsets in configuration memory are handled by the Soft Error Mitigation (SEM) tool from Xilinx\,\cite{XilinxSEM} which monitors the integrity of the configuration
memory and can fix up to two bits upset simultaneously.

% \red{Add reboot from Flash and SCA config via JTAG.}
If the TMR and SEM protection is corrupted by multiple bit upsets, the Router FPGA can be recovered through reconfiguration, by:
1)~locally reading from the on-board flash memory, which can be updated to the latest version, or
2)~by running JTAG remotely via the SCA JTAG port.
The JTAG programming has the highest priority by default, in case multiple programming paths are available.

Finally, in case of any non-recoverable functional interrupt, e.g.\ single-event functional interrupt (SEFI),
the Router power can be cycled by disabling its FEAST DC-DC converters via a signal from the SCA on the Rim-L1DDC.
Further details of the Router's SEU mitigation strategy may be found in\,\cite{RouterSEU}.
% \,\cite{RouterRadTol, RouterSEU, RouterFixedLat2, RouterFixedLat1}

%\FloatBarrier

\subsection{Twin-ax serial and LVDS repeaters}
%twin-ax repeaters
To overcome attenuation due to the up to 6.25\,m twin-ax cable length in the sTGC trigger path (see Table\,\ref{tab:twinaxAttenuation}), repeaters are placed roughly midway along the twin-ax signal paths to guarantee error-free operation.
A 4.8\,Gb/s signalling rate requires good transmission of several odd harmonics of the signalling rate.
The 640\,MHz LVDS transmission is less sensitive to the high frequency attenuation, but does suffer from the attenuation of the fine wires (30\,AWG) in the cable.
The 4.8\,Gb/s serial repeaters restore the signals from the pad-TDS to the Pad Trigger and from the strip-TDS to the Router;
640\,Mb/s parallel LVDS repeaters restore signals from the Pad Trigger to the strip-TDS.
See Figure\,\ref{fig:LL_NSW_ElxOvr} for their locations.
For more details see\,\cite{repeaters}.
%? \red{Can we afford space for pictures?}

% Table generated by Excel2LaTeX from sheet 'Sheet1'
\begin{table}[h]
  \centering
  \caption{Attenuation versus frequency for the  ``MiniSAS'' twin-ax ribbon cable (3M~SL8800 Series MiniSAS cables) from\,\cite{twin-ax}. Silver-plated twin-ax cables were finally preferred because they have reduced attenuation at higher frequencies and lower prices. Since a large number of tin-plated cables were already in stock  they were used in the connection from Pad Trigger to sFEB (640\,Mb/s).}
    \begin{tabular}{lrrrrrrr}
    \toprule
    {\textbf{Frequency (GHz)}} & \multicolumn{1}{c}{\textbf{0.5}} & \textbf{1.0} & \multicolumn{1}{l}{\textbf{2.0}} & \multicolumn{1}{l}{\textbf{5.0}} & \multicolumn{1}{l}{\textbf{10.0}} & \multicolumn{1}{l}{\textbf{15.0}} & \multicolumn{1}{l}{\textbf{20.0}} \\
    \midrule
    {Tin plating (dB/m)} & -0.90 & -1.4  & -2.2  & -4.0  & -7.5  & -10.9 & -14.6 \\
    {Silver plating (dB/m)} & -0.85 & -1.2  & -1.7  & -3.2  & -4.9  & -6.8  & -8.8 \\
    \bottomrule
    \end{tabular}%
  \label{tab:twinaxAttenuation}%
\end{table}%


%\subsubsection{4.8 serial repeaters}
\para{4.8\,Gb/s serial repeaters}
\label{sec:serialRepeaters}
Tests showed that for error-free operation with a safe margin, serial repeaters are necessary for twin-ax cable lengths beyond 4\,m.
A serial repeater for a single 4-pair cable is housed in a small shielded copper box compatible with the width of the cable for easy mounting inline with the cables.
They are powered from a nearby L1DDC.
The quad-channel serial repeater chips\,\cite{ds100br410} are the same as those used as signal conditioners in the Pad Trigger and Router.
Receive equalization, transmit de-emphasis and transmit voltage can all be set by means of pin jumpers.
PRBS test data transmitted from the TDS ASICs showed a bit error rate less than $10^{-14}$.
Different sets of parameters and all cable lengths combinations were used to select the parameters with the lowest transmission error rates.
It was possible to select a common setting for all cable combinations.
A thermal simulation, and subsequent tests using temperature probes, confirmed that the repeater chip was adequately cooled by the copper box.
Power consumption at 2.5\,V is 213\,mW per repeater board. A total of 880 serial Repeater boards were built.
%\red{Do we want an eye-diagram?}

%\subsubsection{LVDS repeaters}
\para{LVDS repeaters}
\label{sec:LVDSRepeaters}
A 6.25\,m twin-ax cable carries the seven serial 640\,Mb/s LVDS lines from the Pad Trigger to the strip-TDS.
The eye-diagram of a test at 640\,Mb/s with a 5\,m cable showed attenuation and an 8\,m cable was unstable with barely an eye.
Consequently, to be safe with a 6.25\,m cable, it was decided to install repeaters.
Single channel Micrel SY58605U 3.2\,Gbps Precision LVDS buffers\,\cite{SY58605U} were used to regenerate the signals.
The high power consumption (a bit less than 1\,W per 7-bit connection) required active cooling.
The LVDS repeaters were located on the spokes of the Wheel, behind the Large Sectors, where cold water for cooling was available.
Boards with six repeaters each, are powered with 2.5\,V from a FEASTMP pluggable module DC-DC converter\,\cite{FEASTMP}.
Two cooling bars are soldered on either side of a copper pipe, carrying cold water.
Two boards are mounted on either side, in thermal contact with the copper bars.
All four boards are enclosed in a common metallic shielding.
A total of 144 LVDS Repeater boards were built.

%\red{All the repeaters were tested in a loop back configuration in a setup based on the Xilinx VC707 platform and a custom mezzanine card.
%The yield was higher than 99\%.-- This is QAQC, remove.}

\subsection{Direct low jitter FPGA transceiver reference clock}
\input{sections/directClock}

\subsection{Board and ASIC counts}
%\red{Not sure where this section should go.}   \\
% counts of boards and ASICs
\label{sec:CountsOfBoardAndASICs}
Ten types of custom electronics boards were designed for the NSW electronics system; over 8000 of them were installed.
Four custom ASICs were designed; over 49,000 of them were installed.
Table\,\ref{tab:boardsASICs} itemizes the counts of NSW boards and ASICs.

% Table generated by Excel2LaTeX from sheet 'Sheet1'
\begin{table}[htbp]
  \centering
  \caption{The numbers of the various boards and ASICs in the NSW system.
  \local{This table summarizes the table in\,\cite{BoardASICcounts}.} }
    \begin{tabular}{lrllrl}
    \toprule
    \textbf{Board} & \textbf{Quantity} & \textbf{Comment} & \textbf{ASIC} & \textbf{Quantity} & \textbf{Comment} \\
    \cmidrule (r){1-3}   \cmidrule (l){4-6}
    MMFE8         &  4096 &                     & VMM   & 40,192 &            \\
    pad-FEB       &   768 &                     & pTDS  &  768   &            \\
    strip-FEB     &   768 &                     & sTDS  & 2304   &            \\
    ADDC          &   512 &                     & ART   & 1024   & 2\,/\,ADDC \\
    MM L1DDC      &   512 &                     & ROC   & 5632   & one\,/\,FEB \\
    sTGC L1DDC    &   512 &		        & GBTx  & 3712   &            \\
    Rim-L1DDC     &    32 &                     & SCA   & 7520   & one\,/\,board  \\
    Pad Trigger   &    32 &                     & VTRx  & 1920   &         \\
    Router        &   256 &                     & VTTx  & 1056   &         \\
    Serial repeater & 768 &                     &       &        &         \\
    LVDS repeater   & 128 &                     &       &        &         \\
    Direct clock  &     2 &  		        & MTx   &  512   &         \\
    FELIX FPGAs   &    60 &                     & FEAST IC & 5760 &        \\
    Trigger       &    16 &  2 sectors each	&       &       & \\
    ~~~~Processor &       &			&       &       & \\
    \bottomrule
    \end{tabular}%
  \label{tab:boardsASICs}%
\end{table}%

