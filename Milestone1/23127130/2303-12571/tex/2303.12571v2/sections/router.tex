\label{sec:router}

The sTGC Router\,\cite{HU2022167504} serves as a packet switch for routing sTGC strip charge information from the strip-TDS to the sTGC Trigger Processor.
It is implemented in a Xilinx FPGA\footnote{XC7A200TFFG1156} from the Artix-7 family.
There is one Router per layer for each sTGC sector.
There are four TDS ASIC positions on a strip Front-end board;
depending on the board location three or four are populated.
%Although all four TDS positions are connected to the Router, only three links will be alive.
A scheme was developed to maintain low and fixed-latency packet multiplexing through the Router\,\cite{RouterFixedLat2, RouterFixedLat1}.
The Router's latency from first bit in to first bit out is 97\,ns.
The Router is unaware of the ATLAS run state and does not send any data in response to a Level-1 trigger.
A context diagram is shown in Figure\,\ref{fig:RouterContext}.

\begin{figure}[h!]
   \centering
   \includegraphics[width=0.75\textwidth]{figures/RouterContext}
   \caption{Context diagram of the Router for a layer showing the four electrical inputs from each of the three strip Front-end boards in one layer of a sector and the four fibre outputs.
             }
   \label{fig:RouterContext}
   \end{figure}

On every bunch crossing the sTGC Pad Trigger chooses up to four bands of strips and sends their band-ids to the strip TDS ASICs that contain those bands.
The selected TDS ASICs transmit the strip data for the band of strips they contain to the Router.
The other strip TDS ASICs transmit a null packet.
The Router inputs are twin-ax electrical serial streams at 4.8\,Gb/s.
The input serial streams are deserialized, unscrambled and aligned to a common clock.
The Router then routes the up to four packets containing the strip charges for a chosen band to the Trigger Processor via the Router's four 4.8\,Gb/s output fibres.
Null packets (containing the sector-id, layer and fibre number) are sent out when there are less than four strip-charge packets.
These are also used in confirming the connectivity of the inputs.

The on-board serial repeater chips\,\cite{ds100br410} condition the 4.8\,Gb/s links from each strip-TDS.
Their parameters were optimized and set with soldered jumpers.
Miniature Transmitter (\gls{MTx}) optical transmitters\,\cite{Zhao:2016czy, Xiao:2016dvu} are used to drive the output fibres.
They are configured via an SCA \ItwoC master.

The on-board SCA  ASIC is used to control the Router.
The sector-id is set in the FPGA via the SCA \gls{GPIO}.
The twin-ax cables from the inner and middle radii front-end boards were made the same length.
The cable from the outer Front-end board is shorter by one or two clock equivalent periods.
In order to align all the inputs, the Router inserts a delay, either one (or two) 160\,MHz clocks, for signals from the outer front-end boards of the large (or small) sectors.

The Router FPGA can be configured from on-board Flash memory or by FELIX via the \gls{JTAG} port of the SCA.
The Router's components are cooled by conduction through heat-conductive gap pads to a copper plate that is water cooled.

\para{Radiation tolerance}

The Router's tolerance to total ionization dose was shown acceptable and reported in\,\cite{RouterRadTol, HU2022167504}.
SEU's in the FPGA fabric or its configuration memory can cause malfunction or halting of the Router operation.
The critical logic sections that handle control and initialization are protected by Triple Modular Redundancy (TMR)\,\cite{TMR}.
The goal is to retain a steady data flow, while allowing bit upsets in the data stream to be mitigated by using the redundancy of multiple detector layers.

Configuration memory bits are protected with Error-Correction Code (\gls{ECC}) and Cyclic Redundancy Check (\gls{CRC}).
However, SEU's may accumulate in essential bits and result in functional failures.
Upsets in configuration memory are handled by the Soft Error Mitigation (SEM) tool from Xilinx\,\cite{XilinxSEM} which monitors the integrity of the configuration
memory and can fix up to two bits upset simultaneously.

% \red{Add reboot from Flash and SCA config via JTAG.}
If the TMR and SEM protection is corrupted by multiple bit upsets, the Router FPGA can be recovered through reconfiguration, by:
1)~locally reading from the on-board flash memory, which can be updated to the latest version, or
2)~by running JTAG remotely via the SCA JTAG port.
The JTAG programming has the highest priority by default, in case multiple programming paths are available.

Finally, in case of any non-recoverable functional interrupt, e.g.\ single-event functional interrupt (SEFI),
the Router power can be cycled by disabling its FEAST DC-DC converters via a signal from the SCA on the Rim-L1DDC.
Further details of the Router's SEU mitigation strategy may be found in\,\cite{RouterSEU}.
% \,\cite{RouterRadTol, RouterSEU, RouterFixedLat2, RouterFixedLat1}

%\FloatBarrier