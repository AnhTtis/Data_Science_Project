\label{sec:intro}

The New Small Wheel (NSW)\,\cite{nswTDR} is an upgrade of the innermost forward station of the Muon Spectrometer at the ATLAS experiment\,\cite{ATLAS_2008} at CERN.
The High Luminosity LHC (HL-LHC) will provide substantially increased luminosity; therefore, higher background rates are expected.
The upgrades to ATLAS for HL-LHC are done in two phases: Phase\,1 for Run\,3, which began in mid-2022 and Phase\,2 for Run\,4, which is expected to begin in mid-2029.
The initial configuration of the Small Wheels of the Muon Spectrometer does not allow the rejection of the fake triggers from the increased background,	and moreover, efficiency loss is expected at high particle rate.
The New Small Wheel will reduce the significant background of fake triggers from track segments that do not originate from the interaction point by providing a track segment to the ATLAS Level 1 trigger logic to match with hit coincidences in the Big Wheel\,\cite{l1_muon_twiki}.
%\red{This needs work and maybe even the usual diagram}
Also, it will cope with a ten-fold increase in the ATLAS Level-1 trigger rate.
It was installed in the ATLAS cavern in 2021 and is undergoing  commissioning.
Operating in a high background radiation region (up to about 20\,kHz/cm$^{2}$ at the expected HL-LHC luminosity of \lumihllhchigh),
%%%% HL-LHC's peak luminosity is expected to be 5 � 10^34 cm^-2 /s and would most likely be pushed to 7.5 � 10^34 cm^-2 /s
it reconstructs muon tracks with high precision and furnishes pointing track segments to the ATLAS \Lone trigger.
In this manuscript, the overall design of the New Small Wheel electronics is presented.


The NSW employs two gaseous detector technologies: sTGC\,\cite{ABUSLEME201685} and \MM\,\cite{ALEXOPOULOS2010161}.
The sTGC provides bunch crossing assignment with high radial resolution from strips and rough $\phi$ resolution from pads;
the \MM strips provide even better radial resolution, and a good $\phi$ coordinate due to its stereo strips layout.
Both technologies are used for triggering and for track reconstruction.
For triggering, the sTGC is expected to provide better timing resolution than Micromegas and a higher angular resolution due to its greater separation between its first four and last four layers.
This performance is still to be demonstrated during Run\,3 operation of the NSW.
%due to its calculation of strip centroids and its longer level arm.

The NSW envelope is a disk, $\sim$10\,m in diameter and $\sim$1\,m thick.
It lies between the Endcap Liquid Argon Calorimeter and the Endcap Toroid, centred at $z=7.3$\,m.
Each NSW comprises 16 sectors, 8 small and 8 large alternating each other.
Each sector consists of eight layers of each detector technology, arranged along the beam axis as follows: a 4-layer sTGC wedge, two 4-layer \MM wedges on either side of a support structure and another 4-layer sTGC wedge.
There are $\sim$2.1\,million \MM strips (4 of the 8 layers have stereo strip layout) $\sim$282\,k sTGC strip, $\sim$47\,k sTGC pad and $\sim$24\,k sTGC wire analog readouts
for a total of $\sim$2.45\,million channels.
The NSW electronics follows the NSW detector organization in a hierarchy from endcaps, sectors, layers, and finally, radial position.
A large and a small sector cover one octant.
The sectors overlap slightly, but are independent, with no communication between them.
%Signals flow only vertically up and down the hierarchy.
%, although low voltage power modules may supply multiple sectors.

The NSW performance criteria are demanding.
In particular, the precision reconstruction of track segments for the offline analysis requires a spatial resolution of $\sim$100\um per layer to provide good momentum resolution along the radial direction,
over a 4\,m active radius surface. The track segments for the \Lone trigger must be reconstructed online with a polar angular resolution of approximately 1\,mrad,
in order to match the 1\,mrad angular resolution of the middle and outer muon stations planned for the Phase\,2 upgrade for the HL- LHC.
The trigger requires that for a valid NSW segment, the angle between it and the infinite momentum track from the interaction point be less than $\pm$15\,mrad. (The angle is configurable up to\,$\pm$15\,mrad in steps of 1\,mrad.)
In Phase\,1, NSW segments found at every bunch crossing are extrapolated to match the \emph{hit} coincidences in the muon stations downstream from the endcap toroid magnet.
In Phase\,2, they are used with \emph{segments} in the downstream muon stations to measure a track's transverse momentum, $p_\mathrm{T}$, for the trigger algorithm.

\vspace{12pt}
\noindent Constraints strongly affecting the design of the electronics include:
\begin{itemize}[topsep=2pt, itemsep=0pt, parsep=1pt]
   \item Sophisticated Front-end readout is required for 2.45\,million analog channels with differing signal characteristics.
   \item Tightly limited latency for the Phase\,1 trigger path of $\sim$1.1\us including $\sim$500\ns of time-of-flight, fibre and cable propagation
   \item Radiation at the inner radius (ten-year total dose of 350\,kRad, simulated, with safety factors, requires radiation and Single Event Upset tolerant ASICs and testing of any commodity electronics used).
            FPGAs can be used only on the rim of the wheel and only if they use SEU mitigation techniques.
   \item Magnetic fields up to 5\,kG strongly limit the use of non-air-core inductors in filters and DC-DC converters.
   \item Limited space implies dense circuit boards, active cooling.
   \item Limited access after installation requires redundancy.
   \item Since the on-detector electronics cannot be replaced for Phase\,2, the Phase\,2 requirements must be met with Phase\,1 technology.
   \item The decision between a single level and a two-level hardware trigger could not be made before freezing the NSW ASIC requirements,
            thus requiring the readout ASIC to support both.
   \item Need to support two detector technologies with as many shared elements as possible.
   \item Need for state-of-the-art gigabit serial transceivers and interconnections
   \item Need of on-board voltage conversion requires very careful design of the power distribution to maintain a high signal-to-noise ratio.
\end{itemize}

\para{Actual clock frequencies}
All the NSW clocks are inherited from the LHC bunch crossing clock, which is 40.079\,MHz.
However, for convenience in this document, the bunch crossing (BC) clock and its integer multiples are referred to as 40\,MHz, 80\,MHz, 160\,MHz, etc.
For example, the actual frequency of the 160\,MHz clock is 160.316\,MHz.

\subsection{Radiation and magnetic field tolerance}
The radiation level diminishes as the radius increases; see Table\,\ref{tab:radEnv}.
It is high enough to force using  ASICs on the detectors, but FPGAs are manageable, with Single Event Upset (SEU) mitigation, on the rim of the wheel.
All on-detector components were tested to confirm that they withstand the expected Total Ionisation Dose (TID).
All on-detector non-ASIC components are stateless, so their SEUs are not fatal.
The modern jitter cleaners required by FPGA gigabit transceivers have both analog and digital logic and, unfortunately, were found to have fatal Single Event Effects even in the radiation environment on the Rim.
See\,\cite{Ameel, amideiDCDC, ATL-MUON-PUB-2022-001, NSWenv}.

The magnetic field, combined from the solenoid and endcap toroid, is neither uniform in radius, $R$, nor azimuth, $\phi$.
Its highest value lies exactly on the Rim and the edges of the detectors where the electronic boards are located.
Either air-core inductors, or inductors that could be shielded and oriented to minimize the effect of the magnetic field, were used\,\cite{NSWenv}.

\begin{table}[ht!]
\centering
   \caption{Simulated radiation loads and magnetic fields, from\,\cite{NSWenv}  for the NSW after 10\,years at high luminosity LHC, $\mathcal{L}=\rm 5{\times}10^{34} cm^{-2} s^{-1}$,
              for Total Ionization Dose (TID),
              Non-Ionizing Energy Loss (NIEL),
              Single Event Effect (SEE).
              Safety factors are not included.}
   \label{tab:radEnv}
   \begin{tabular}{l l l}
     \toprule
     \textbf{ } & \textbf{Inner radius (R=1\,m)} & \textbf{Outer Rim (R=5\,m)}\\
     \midrule
       TID ($\gamma$) & 780\,Gy  & 26\,Gy\\
       NIEL (fast neutrons) & $\rm 3.9{\times}10^{13} \        \emph{n}/cm^2$ &  $\rm 1.2{\times}10^{12} \   \emph{n}/cm^2$\\
       SEE (protons)        & $\rm 6.7{\times}10^{12} \        \emph{p}/cm^2$ &  $\rm 2.2{\times}10^{11} \   \emph{p}/cm^2$\\
       B\,field & $\le$\,1\,kG & max 5\,kG\\
    \bottomrule
    \end{tabular}
\end{table}

%%%%%%%%%%%%%%%%%%%%%%%%%%%%%%%%%%%%%%%%%%%%%%%%%%%%%%%%%%%%%%%%%%
% Text originally here has been moved to overall_architecture.tex
