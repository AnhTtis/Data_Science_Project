\label{sec:TDS}

The sTGC trigger data serializer (\gls{TDS}) ASIC\,\cite{Wang:2017ols, Wang:2015msa, Wang:2019tay} prepares the trigger data from the VMM for either pads or strips and serializes it for transmission, in the case of pads, to the Pad Trigger, in the case of strips to the Router on the rim of the NSW detector. The data are transferred ultimately to the Trigger Processor.
It is mounted on the Front-end boards and can be configured in either strip or pad mode by connecting a pin to  high or low.

Both the 160\,MHz TDS logic clock and serializer reference clock are generated by an on-chip PLL from the BC clock supplied by the Read Out Controller.
The TDS is configured via the \ItwoC master of the on-board SCA ASIC.
An on-chip pseudo-random binary sequence generator (PRBS-31) is provided for serializer and link testing.

The TDS is fabricated in IBM\,130\,nm \gls{CMOS} technology and is packaged as a 400-pin Ball Grid Array (BGA).
It uses a 1.5\,V supply for both the logic part and the serializer and consumes about 0.9\,W.
A block diagram of the TDS ASIC, including both strip and pad parts, is shown in Figure\,\ref{fig:JW_TDS_blockdiagram}.

\begin{figure}[t]
\centering
\includegraphics[width=0.999\textwidth]{figures/JW_TDS_blockdiagram.pdf}
\caption{Block diagram of the TDS ASIC. Strip mode, top; Pad mode, bottom.}
\label{fig:JW_TDS_blockdiagram}
\end{figure}


\subsubsection{Pad TDS mode}  %%%%%%%%%%   P A D   T D S  %%%%%%%%%%%%%%%%%%%%%%%%%%%%%%

The pad TDS receives up to 104 pad Time-over-Threshold (ToT) differential signals from two VMM ASICs and transmits the data at 4.8\,Gb/s to the Pad Trigger every 25\,ns.
The rising edge of the ToT signal is used to capture a pad hit and assign its BCID.
Since the 104 pads cover an area of about 3\,m$^2$, the on-detector routing lengths from a pad to the VMM and pad-TDS can differ by up to 2.6\,m, with a propagation time of about 7\,ns/m\,\footnote{Calculated by the PCB layout program and verified approximately by measurements of some pads with a Time Domain Reflectometer.}.
In addition to the differences in time-of-flight from the IP for the pads within a detector,
this difference in propagation time could result in different BCID assignments for pad signals even though they belong to tracks from the same collision.
To compensate for this, each channel has a configurable delay, of up to eight 3.125\,ns steps\,\cite{Wang:2017iuz}.
Assuming that the earliest and latest pad signal arrivals (considering the earliest arrival signal from each pad)
are within 25\,ns of each other,
shifting the phase of the incoming BC clock enables them to be assigned the same BCID.

For serial transmission, 116 bits (104 pad bits plus a 12-bit BCID) are split into four consecutive frames: a 26-bit frame followed by three 30-bit frames.
The four frames are scrambled following a scheme used by the 10\,Gb/s Ethernet physical layer implementation\footnote{IEEE Standard 802.3-2012 with the polynomial function $1+x^{39}+x^{58}$}.
An unscrambled 4-bit header, 0b1010, is prefixed to the 26-bit scrambled frame, marking it as the first of the four 30-bit frames.
Scrambling enables 116 data bits to be transferred in one bunch crossing at 4.8\,Gb/s.
The latency of the pad-TDS from the end of a BC to the first bit of a pad frame exiting the serializer core is 31\,ns.

%\begin{figure}[h]
%\centering
%\includegraphics[width=0.93\textwidth]{figures/Pad_TDS_BlkDiagram_V02}
%\caption{The processing pipeline in the pad-TDS ASIC}
%\label{fig:pTDS_flow}
%\end{figure}


\subsubsection{Strip TDS mode}  %%%%%%%%%%  S T R I P   T D S  %%%%%%%%%%%%%%%%%%%%%%%%%%%
\label{sec:sTDS}

Each of the two VMM ASICs connected to one strip-TDS, sends 6-bit charge data through 64 independent serial lines, each representing the induced charge on a strip.
The strip-TDS holds the vector of charges, along with its BCID, in a circular buffer.
The Pad Trigger, after finding a coincidence in a tower of pads in eight sTGC layers, sends the ID of the band of strips in each layer that passes through that tower to the strip-TDS that holds the charges of that band.
A configurable look-up table in each TDS contains the channel number of the first strip in each band that it holds.
Strip-TDS ASICs that do not hold charges of any band are sent band-id\,0xff, which is not in its look-up table.
The band-id is the same for all layers, but the 17 strips comprising that band differ from layer-to-layer since the tower points to the interaction point.
When the request arrives from the Pad Trigger, if the charges for strips corresponding to the band-id are in the circular buffer and their BCID matches the requested BCID,
they are transmitted at 4.8\,Gb/s to the Trigger Processor via the Router.
Only the charges of the outer 14 or inner 14 strips are transmitted; a flag indicates which.
The data is sent in four 30-bit packets in one BC.
The packet format is shown in\,\cite{HU2022167504}.
This repeats for every bunch crossing.
The latency of the strip-TDS from arrival of the request from the Pad Trigger until the first bit of data frame exits the serializer core is 75\,ns.

\subsubsection{Test functions}
The TDS has several embedded functional self-tests in case of any system malfunctions, system commissioning and to enable testing without inputs.
The 4.8\,Gb/s serializer can be driven by an on-chip pseudo-random binary sequence generator\,\cite{PRBS}, PRBS-31, for testing and commissioning the link.
Specifically, the test modes are: \\
\textbf{Bypass Trigger mode:} In strip mode, probes inputs from an individual or a programmable number of strip channels bypassing the channel ring buffer and without trigger input from the Pad Trigger.
In pad mode, a similar diagnosis function provides access to an individual or a programmable group of input channels. \\
\textbf{TDS-Router Training Frame:} For strip mode, sends fake test frames to the Router without relying on the input from the Pad Trigger.\\
%
\textbf{Global-Test mode:} performs a full strip-mode function and output without the input from the Pad Trigger or the Front-end electronics.

%A detailed description can be found in\,\cite{Wang:2017ols}.

%\begin{figure}[h]
%\centering
%\includegraphics[width=0.97\textwidth]{figures/sTDS_flow.png}
%\caption{The processing pipeline in the strip-TDS ASIC}
%\label{fig:sTDS_flow}
%\end{figure}
%
%\FloatBarrier

\subsubsection{The 4.8\,Gb/s serializer}   %%%%%%%%%%  S E R I A L I Z E R  %%%%%%%%%%%%%%%%%%%%%%%%%%%

The 4.8\,Gb/s serializer for the TDS was developed and prototyped prior to the remainder of the TDS\,\cite{Wang:2015msa}.
Rather than developing such a challenging circuit from scratch, it was adapted from the CERN GBTx serializer, changed from loading 120 bits at 40\,MHz to loading 4\,$\times$\,30 bits at 160\,MHz.
CERN's ePLL design\,\cite{Poltorak_2012} was also used.
The metalization layers were also changed to match the variant of the IBM\,130\,nm CMOS process used by the other NSW ASICs so that all could be fabricated on the same wafer.
The serializer's output into a 100\,$\Omega$ load is about $\pm$\,500\,mV.

A jitter analysis of the transmission showed a total jitter of 49.7\,ps at a bit-error-ratio (\gls{BER}) of $10^{-12}$.
A BER test with an embedded PRBS checker inside a Xilinx\,7 FPGA was also performed.
Error-free running for three days was achieved, which corresponds to a BER less than $10^{-15}$.

\subsubsection{Radiation tolerance}
To mitigate Single Event Upsets, the TDS employs Triple Modular Redundancy\,(TMR)\,\cite{TMR} for the following:
Serializer, Serial protocol logic,
BCID, BC clock generator, BC clock phase shift, size of the matching window of the strip channel,
all configuration registers, Pad Trigger \gls{LUT}.