\section{Configuration}
\label{sec:configuration}
All NSW electronics boards have a Slow Control Adapter ASIC (SCA)\,\cite{GBT-SCA}, see Section\,\ref{sec:SCA} for configuration of the ASICs on the board and readout of board temperatures and voltages.
The FPGA-based Trigger Processor includes firmware, SCAx\,\cite{SCAxIEEE}, that emulates the  \ItwoC sub-device of the SCA ASIC, providing access to registers and memories in the FPGA.
On the software side, a dedicated OPC\,UA server\,\cite{opcuaserver} for the SCA facilitates the SCA communications for the configuration and monitoring of the electronics
%(SCAx supports only the SCA's \ItwoC sub-device).
The use of OPC\,UA enables the interface to the SCA to be shared by both the Detector Control System and the Data Acquisition System.
It is an industry-standard protocol with an advanced ecosystem of tools.
The details of the configuration path are shown in Figure\,\ref{fig:LL_SCAOPC}.
See\,\cite{Tzanis:2021ttb,Tzanis:2022xje} for additional details.
Many configuration parameters are tuned through the calibration procedures described in Section\,\ref{sec:calibration}.

\begin{figure}[h]
\centering
\includegraphics[width=0.8\textwidth]{figures/LL_SCAOPC_v02.pdf}
\caption{The Configuration path with OPC\,UA clients and servers.}
\label{fig:LL_SCAOPC}
\end{figure}
