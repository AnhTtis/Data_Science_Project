\label{sec:pad_trigger}

The sTGC Pad Trigger is a pre-trigger board that finds tracks passing through a pointing tower of logical sTGC pads.
See Figure\,\ref{fig:LL_PadSelect}.
Its purpose is to limit the on-detector trigger electronics to send out only the charge data from those bands of strips that pass through the pad tower coincidences.
Up to four towers can be found.
It receives the binary pad hits from 24 Front-end boards, three per layer, in a sector.
The band of 17 strips in each layer that pass through each tower is identified by a Band-id.
The Band-ids of triggered towers are sent to the relevant strip-TDS ASICs which then send the charge information of each strip in the band to the Trigger Processor via the Router.
One Pad Trigger per sector resides in the Rim Crate for that sector.
Its context and block diagram are shown in Figure\,\ref{fig:RV_Pad_block_schema}.
The internal blocks of the FPGA are shown in Figure\,\ref{fig:RV_Pad_Trigger_FPGA}.

\begin{figure}[ht]
\centering
\includegraphics[width=0.72\textwidth]{figures/RV_Pad_block_schema.pdf}
\caption{Pad Trigger context and block diagram.
         Not shown: The output of the Trigger block is also sent to the Readout block.}
\label{fig:RV_Pad_block_schema}
\end{figure}
\begin{figure}[ht]
\centering
\includegraphics[width=0.99\textwidth]{figures/RV_Pad_Trigger_FPGA.pdf}
\caption{Pad Trigger FPGA block diagram.
         Not shown: The output to the Trigger Processor is also read out to FELIX.}
\label{fig:RV_Pad_Trigger_FPGA}
\end{figure}

Serial repeater chips\,\cite{ds100br410} condition the 24 4.8\,Gb/s input links from the pad-TDS ASICs.
They are configured via the \ItwoC channels of the on-board SCA ASIC.
After deserialization, each vector of pad hits can be delayed in programmable steps of 4.17\,ns to align them in time.

\para{Band-finding algorithm:} Valid 3-out-of-4 pointing towers for each 4-layer wedge were defined using the ATLAS off-line simulation to track single muons through the sTGC layers of a sector.
The simulation's pad id were mapped to a Pad Trigger input-id.
For each tower, the list of pad input-ids (one per layer), the band-id and $\phi$-id were put in a list of trigger patterns.
There is one comparator for each of about 4700 trigger patterns that compares the input pads to the list of pads in each trigger pattern.
The match is done taking into account the 3-out-of-4 majority logic per each quadruplet.
The comparators are grouped by their corresponding Band-id.
A one or two bunch crossing coincidence window is determined by the choice of a bit file specifically generated for that choice.
The current choice is a two BC window.
% from Band-id\,6 to\,87. \red{What happened to Band-ids 1-5 and 88-92?}
% Some Band-ids cannot be be generated because not enough layers in the band due chambers have constant R boundairies and not constant theta,
% or not all pads are connected to pTDS channels (104 of 112).
% Also holes in the VMMs that prevent contiguous strips from being contiguous TDS channels. Each comparator group produces a ``1'' if there is a match, ``0'' otherwise.
The lists of trigger patterns for large and small sectors are defined in separate FPGA bit files.
An output vector of comparator results, one bit per band, is passed to a priority encoder which selects a maximum of four Band-ids to be sent to the strip-TDS ASICs on the sFEBs.
%An output vector of comparator results, one bit per band, is passed to a priority encoder which selects a maximum of four Band-ids and their respective  $\phi$-ids to be sent to the strip-TDS ASICs on the strip-FEB's.
The priority encoder also takes into account that a Band-id may need to be sent to two adjacent strip-TDS ASICs.
In this case the maximum number of Band-ids is reduced by one.
The selected Band-ids are finally directed to their corresponding strip-TDS via cable connections to each sFEB.
Note that each strip-TDS has a configurable lookup table that defines which strips belong to a given Band-id in that layer and that at most three of the four strip-TDS positions on a sFEB are populated.
The Pad Trigger output to each sFEB consists of up to four Band-ids (a maximum of one per TDS ASIC), the BCID, a frame and a clock, and is sent via seven serial LVDS lines (See Section\,\ref{sec:sTDS}.).
%The Pad Trigger output to a strip-FEB, consisting of up to three Band-ids, the BCID, the $\phi$-id, a frame and a clock, is sent via seven serial LVDS lines (See Section\,\ref{sec:sTDS}.).
The maximum of four Band-ids may be distributed over any of the up to ten strip-TDS ASICs in each layer.
The Band-ids, BCID, and $\phi$-id along with some flags are also sent to the Trigger Processor via two redundant serial links at 4.8\,Gb/s, 8b/10b encoded, via a dual VTTx optical transmitter
(See Section\,\ref{sec:VTRxVTTx}.).

Configuration parameters, partially listed in Table\,\ref{tab:PadTrigConfig}, are set through the SCA server via the Rim-L1DDC and the on-board SCA ASIC. The FPGA configuration of the Pad Trigger board can be realised either by the SCA server via the on-board SCA's JTAG channel or by a Xilinx programmer via a connector. This is configurable through on-board jumpers.


%\noindent\red{
%How is the $\phi$-id handled? \\
%Can we make the choice of one or two BC window optional? Perhaps with separate bit files?} this is configurable through a 5-bit register
%?how is phi-ID handled? Each pattern line comes with its phi-ID (from simulation) so the phi ID is easy picked up once the matching logic gives a yes
%?readout window how centered? There are two I2C programmable registers. One is the L1 latency, which is the time distance in BCIDs corresponding to the first event to be read. The other register is the readout window width, the number of older events to be read after the first one.

% Table generated by Excel2LaTeX from sheet 'Sheet1'
\begin{table}[htbp]
  \centering
  \caption{Partial list of Pad Trigger configuration parameters.}
           % More parameters will be added as the Pad Trigger developments progress.
    \begin{tabular}{l}
    \toprule
%    \textbf{Parameter} \\
%    \midrule
    BC counter offset  \\
    Enable readout on L1A  \\
    Enable readout on self-trigger   \\
    Readout window BC offset of the first BC to be read on a L1A   \\
    Readout window size,  in BC units (0\,=\,1\,BC, 1\,=\,2\,BC, ...) \\
    pFEB input delay in steps of 4.17\,ns (0\,=\,4.17\,ns, 1\,=\,8.33\,ns, ...) \,4 bits per input \\
    24-bit mask to force all pad hits of the corresponding pFEB to 0 \\
    24-bit mask to force all pad hits of the corresponding pFEB to 1 \\
    Enable Orbit Count Reset (OCR) mode (where idle state is forced \\ ~~~~~until OCR is received). See Section\,\ref{sec:BCsync}. \\
    \bottomrule
    \end{tabular}%
  \label{tab:PadTrigConfig}%
\end{table}%

In addition to the trigger path output, all inputs and outputs are saved in a latency buffer until a Level-1 Accept is received via the TTC system.
In response to the Level-1 Accept, data for a configurable BC window is sent out on an E-link to the swROD via FELIX.
The input vectors of pad hits are compressed using a sparse binary compression algorithm. See Chapter\,11.5 of\,\cite{salomon2010handbook}.
%\red{Is a hit map sent if the compression result would be bigger than the hit map?}

For redundancy, the Pad Trigger is connected to both of the primary and secondary Rim-L1DDC (See Section\,\ref{sec:L1DDC}.).
The SCA on the Pad Trigger is connected to both as well.
This allows the SCA to choose which Rim-L1DDC provides the 160\,MHz reference clock for the Pad Trigger's gigabit transceivers.
See Section\,\ref{sec:L1DDC}.
The choice of which Rim-L1DDC supplies the E-links is defined by the FPGA design, i.e.\ an FPGA bit file is specific to one Rim-L1DDC or the other.
% Perhaps a later firmware version will have a set multiplexers in the FPGA logic that select either the primary or secondary Rim-L1DDC E-links, programmable via the SCA \ItwoC.}

The Pad Trigger's components are cooled by conduction through heat-conductive gap pads to a copper plate that is water cooled.
Special attention was paid to cool the VTTx optical transmitters which are heat sensitive.
The Pad Trigger board consumes 25\,W from a 9.2\,V supply.
%\FloatBarrier


