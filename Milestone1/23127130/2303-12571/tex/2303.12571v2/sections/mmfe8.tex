\label{sec:mmfe8}

The \gls{MMFE8} board is the front-end electronics for \MM detectors. It is the interface between the detector, the trigger (ADDC), and data acquisition (L1DDC) electronics. Due to the high number of readout channels on the \MM detector ($\sim$2.1\,M), 4096 MMFE8 boards are needed, each handling 512 channels. The board must meet demanding space and electrical requirements and constraints. The board dimensions are 215\mm in length by 60\mm wide. It is comprised of 14 electrical layers and is 2.54\mm thick.  A block diagram of the MMFE8 is shown in Figure\,\ref{fig:mmfe8_schematics}.

The interface to the detector is achieved via two \gls{ZEBRA}\textsuperscript{\textregistered} elastomeric connectors\,\cite{zebra}.  The type and layout of the connectors was carefully studied. The pitch on the board was chosen to be 400\um with a contact width of 200\um which is compatible with the detector pitch.
The connector has six lines of through wires per strip which ensure contact between the detector strips and the pads on the MMFE8 PCB.
A precision machined slot between the two connectors aligns the board to the detector.
The compression of $\sim$150\um is also regulated with mechanical cams on top of the board where a ground strip connects the board analogue ground to the detector ground through six contacts. On either side of the connector, four lines provide geographical information through an on-detector encoding\footnote{It was unfortunately found during integration that due to wrong connectivity of the pins in the MMFE8 board, the geographical decoding could not be achieved.}.

The VMM inputs on the board are protected through a dedicated diode network which was extensively tested and was found to protect the ASIC from any discharges on the detector\,\cite{Iakovidis:2675779}.  On each channel, a Transient Voltage Suppressor (\gls{TVS}), a Semtech \textmu Clamp\,\cite{semtech}, is able to mitigate fast transients.
Following the TVS diodes, a 10\,$\Omega$ resistor connects to one of the four channels of a TVS Diode Array, SP3004\,\cite{sp3004}.  This diode array mitigates slower input transients.
Another 10\,$\Omega$ resistor is placed after the SP3004\footnote{Early in the project, the NUP4114-D was used. It was discovered that the functional details of this protection diode were modified between 2012 and 2014, but without changing the part number. Its new functionality was not adequate to the NSW needs, so the SP3004 was used instead. }.

\begin{figure}[t]
\centering
\includegraphics[width=1.0\textwidth]{figures/GI_mmfe8_schematic_v2}
\caption{The MMFE8 block diagram showing the \gls{ESD} protection,
         the VMM Front-end ASICs, the Readout Controller ASIC, the Slow Control Adapter ASIC, the FEAST DC-to-DC converter ASICs, and the MiniSAS twin-ax connectors for the different interfaces.}
\label{fig:mmfe8_schematics}
\end{figure}

																																																																																																																																																																																																																														The board contains eight VMM ASICs, one ROC ASIC, one SCA ASIC and three FEAST ASICs. The on-board FEAST ASICs provide power to all the ASICs. Two FEAST ASICs set for 1.3\,V\footnote{Studies showed that the VMM must be provided with at least 1.2\V and therefore the 1.3\V ensures that voltage by compensating for small voltage drops along the lines.} provide power to the eight VMM's analog section (one FEAST per four VMMs).  One FEAST provides 1.2\V to the digital supplies of all the VMMs and also to the ROC and SCA ASICs\footnote{The SCA ASIC is designed to function at 1.5\,V. Discussions and validation tests showed that the SCA can function properly at 1.2\V with negligible impact on the integrated ADC performance. This was done to avoid integrating another FEAST on the board due to space constraints.}.
The SCA provides a unique 32-bit ID for the board.
The input voltage of 11\,V is provided through the Low Voltage Distributor Board (LVDB) described in Section\,\ref{sec:powerDistribution}.
The power consumption of the board was measured to be $\sim$16\,W.


The interface to both ADDC and L1DDC boards is realised through MiniSAS cables and connectors\,\cite{MiniSASconnector,twin-ax,8F36}. The connector to the ADDC carries the eight ART signals. The connection to the L1DDC carries the TTC signals and bunch crossing clock to the MMFE8 as well as the four data lines to the GBTx (one per SROC). The SCA E-link is carried as well in the same connector, removing ambiguity as to which ROC and VMMs are configured and reset by the SCA. The on-board ROC provides dedicated trigger, test pulse, reset and various clock signals to each VMM.

The design of the MMFE8 was iterated such that it almost reached the theoretical noise levels with respect to the ones defined by the VMM for the \MM detector capacitance, which is estimated to be $\sim$150\,pF/m\,\cite{Iakovidis:2675779}. To achieve that, the FEASTMP2.1\cite{FEASTMP} design was adopted.