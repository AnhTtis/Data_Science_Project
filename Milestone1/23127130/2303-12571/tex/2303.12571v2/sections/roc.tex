\label{sec:ROC}
The ReadOut Controller (\gls{ROC})\,\cite{Coliban:2016uys, Popa:2019trf, Popa:2020sbm, Popa:2020poz,Popa2022} is a highly configurable data aggregation ASIC designed specifically for the NSW.  The block diagram of the ROC ASIC is shown in Figure\,\ref{fig:block_ROC}.
The ROC must provide readout of the hits buffered from the VMMs in response to a Level-1 trigger at 100\,kHz for Phase\,1 and at 1\,MHz for Phase\,2\,\cite{1MHzReadout}.
Initial specifications for the two-level hardware trigger required handling a latency of 60\,$\upmu$s with a consequent very deep buffer for hit data.
After ATLAS decided on a single-level hardware trigger the latency was reduced to 10\,$\upmu$s, but the ROC ASIC had already been produced.

\begin{figure}[b]
\centering
\includegraphics[width=0.99\textwidth]{figures/GI_ROC_Architecture}
\caption{Block diagram of the ROC ASIC\,\cite{Popa2022}}
\label{fig:block_ROC}
\end{figure}

The ROC receives 8b/10b encoded data from up to eight VMM3a ASICs. There is a dedicated ``Capture'' module for each VMM. The data are first de-serialized and then a dedicated mechanism determines the correct 8b/10b stream alignment and then decodes it. The data parity is also checked and if an error occurs, the relevant counters are incremented. If no data appear then a configurable timeout is asserted.
The decoded L0 packets are enqueued separately for each VMM.
A configurable crossbar allows routing of the data from one to eight VMMs to each of up to four SROC modules.
Each SROC is able to transmit 8b/10b encoded data through a configurable up to 320\,Mb/s E-link to the L1DDC.  Two 320\,Mb/s E-links can be combined to produce a 640\,Mb/s output.
For Phase 1, generally one 320 Mb/s E-link per ROC is sufficient.
For Phase 2, more than one E-link is needed, especially at the inner radius.
VMM occupancy varies strongly with radius.
Front-end boards at inner and outer radii require different numbers of E-links.
The crossbar allows routing VMM outputs to E-links in order to optimize the number of E-links needed.
For load balancing, an SROC can combine high and low occupancy VMMs on the same E-link.
The VMM data integrity is also checked at the SROC level.
If, for example, the ROC receives the so-called ``Magic'' BCID, which is a value outside the range of the expected LHC values, the ROC understands that the VMM FIFOs overflowed and data will not be received from this VMM for the configurable value (on the VMM) of skipped triggers.

The ASIC implements several output formats including dummy hits signalling overflow from the VMM, as well as hits that discard the TDC (TDO) value of the VMM to decrease the bandwidth. Moreover, Busy-On/Busy-Off symbols are injected in the output data stream, if enabled, to signal almost-full buffers in the ROC. The configuration and status of the ASIC is accessed by a dedicated I$^2$C interface through the SCA.

The ROC ASIC receives the TTC stream and BC clock from the GBTx. The alignment of the input stream is determined by detecting the positive edge of the BC clock signal in the readout clock domain. All its internal and externally supplied clock signals are generated by four ePLL blocks\,\cite{Poltorak_2012} driven by the BC clock. It supplies the clock and the TTC commands to all the ART, TDS and VMM ASICs. The three ePLLs supplying the signals and clocks to the other ASICs include phase-shifting circuits. This gives the ability to forward the relevant TTC commands and clocks with a configurable phase. The TTC FIFO (also called BC FIFO) buffers the Level-1 triggers.
In response to the Level-1 trigger, the ASIC checks, aggregates, re-formats and filters the L0 packets from the associated VMM Capture modules, building output packets that are pushed into the SROC FIFO ready to be read out.
The ePLLs are configured and monitored through a separate register bank through a dedicated I$^2$C interface of the SCA.



During the integration of the ROC ASIC and the SCA, it was found that the I$^2$C read-back implementation of the ROC was incompatible with the I$^2$C standard chosen by the SCA. The I$^2$C read-back was therefore emulated using dedicated GPIO lines and a software ``Bit Banger'' application\,\cite{opcuaserver}.

The ROC implements Triple Modular Redundancy\,(TMR)\,\cite{TMR} for all the configuration registers, state machines, control of the bunch crossing FIFO and readout logic to mitigate the effects of SEUs. An SEU counter is accessible by the SCA chip (through the configuration and register bank) to monitor the chip operation in the ATLAS environment.

%The protection areas are summarized in Table\,\ref{radtableroc}.

%\begin{table}[h]
%\caption{Single Event Upset protection schema in the ROC}
%\vspace{5pt}
%\label{radtableroc}
%\centering
%\setlength{\tabcolsep}{3pt}
%\begin{tabular}{lc}  %{ p{7cm}p{7cm}  }
%\toprule
% Block & Type of protection \\
%\midrule
%Registers & TMR\\
% State Machines & TMR\\
% Bunch crossing clock domain & TMR\\
% Readout clock domain & TMR\\
%Flip-Flops & TMR\\
%
%\bottomrule
%\end{tabular}
%\end{table}

%\FloatBarrier