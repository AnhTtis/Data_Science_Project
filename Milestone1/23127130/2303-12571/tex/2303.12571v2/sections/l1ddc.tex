\label{sec:L1DDC}
The Level-1 Data Driver Cards (L1DDC)\,\cite{Gkountoumis:2019ye, Gkountoumis:2779645, Gkountoumis:2016mrs, Gkountoumis:2016dfm} are high-bandwidth radiation-tolerant data aggregator boards based on the GBTx ASIC (Section\,\ref{sec:GBTx}).
On one side they connect through E-links with multiple front-end boards via twin-ax cables; on the other side, they connect through two or more fibres with FELIX.
To accommodate Phase\,2, a Front-end board can provide up to four 320\,Mb/s E-links for L1A readout data; only one is needed for the LHC Run-3.
The L1DDC is completely transparent to the data being transmitted or received.
MiniSAS cables make the connections to the front-end boards.
The optical interfaces are the VTRx and VTTx (see Section\,\ref{sec:VTRxVTTx}).
The boards are powered by FEAST DC-DC converters for 1.5\,V and 2.5\,V.
There are three variants of L1DDC's with the same basic functionality, but with different mechanics and channel counts.
One of them is part of the \MM electronics, and two of them of the sTGC electronics.

\para{The Rim-L1DDC:} Supports the Pad Trigger and eight Router cards in the sTGC trigger path electronics. Those electronics are in a dedicated crate called ``the Rim crate'' since it is located in the rim of the NSW structure.
The Rim-L1DDC consists of two independent boards, one primary and one auxiliary, sharing the same PCB.
Each includes one GBTx, one VTRx and nine MiniSAS connectors. This provides redundancy to the system since a complete sTGC trigger sector depends on it.
The Pad and Router boards are connected to both primary and secondary sections. The on-board SCA's are able to switch on/off the Pad Trigger and Router boards by controlling the enable/disable signal of their FEASTs.
Tests of the Rim-L1DDC showed that its GBTx ASIC's 160\,MHz E-link clock jitter was marginally good as a reference clock for the Xilinx\,7-family FPGA transceivers of the Router and Pad Trigger.
For that reason, a dedicated clock is provided through an additional fibre to the Rim-L1DDC. This direct low jitter 160\,MHz clock (Section\,\ref{sec:DirectClock}) is received through an additional VTRx and is distributed to the Router and Pad Trigger boards via low jitter fan-outs.
The block diagram of the Rim-L1DDC is shown in Figure\,\ref{fig:LL_Rim_E-links}.  The power consumption of the board was measured to be 8\,W.
%The block diagram of the Rim-L1DDC is shown in Figure\,\ref{fig:rimBlock}.


\begin{figure}[h]
\centering
\includegraphics[width=0.95\textwidth]{figures/LL_Rim_E-links_V02}
\caption{The two redundant sections of the Rim-L1DDC and their connections to the Router and Pad Trigger boards}
\label{fig:LL_Rim_E-links}
\end{figure}

%\begin{figure}[ht]
%\centering
%\includegraphics[width=1.0\textwidth]{figures/GI_RIML1BlockDiagram}
%\caption{Rim-L1DDC block diagram.}
%\label{fig:rimBlock}
%\end{figure}

\vspace{-10pt}
\para{sTGC-L1DDC:} interfaces with three Front-end boards, either strip or pad+wire. Two GBTx ASICs for two independent bi-directional (VTRx) links; only one is used for pad+wire FEBs. The second strip GBTx provides the extra bandwidth needed by Phase\,2.
In all, 21~E-links connect to a sTGC-L1DDC.
The board environment is monitored by an SCA ASIC.
In addition, by means of the SCAs second redundant E-link, both GBTx ASICs can communicate with the SCA.
The second GBTx though is configured through the \ItwoC of the SCA.
%?\red{LL: Incorrect? See Panos diagram: figures/LL_sTGC-L1DDC_GBTxConfig_v01.pdf}
The L1DDC's are placed on the upper edge of the sTGC quadruplet.
The power consumption of the sTGC-L1DDC is $\sim$4\,W\footnote{Higher consumption up to 6\,W is measured on the sTGC L1DDC but is due to the consumption of the repeaters which are powered through the L1DDC.}.

\para{\MM -L1DDC:}  interfaces with eight Front-end boards and one ADDC board. It features three GBTx ASICs: one is a full-duplex transceiver (VTRx) and two are simplex transmitters (VTTx) for the extra readout bandwidth needed by Phase\,2.
The GBTx connected to the VTRx is configured directly from its IC link.
The two GBTx ASICs connected to the VTTx  are configured via the on-board SCA ASIC.
The SCA also monitors the temperature and voltage levels of the board and the VTRx's Rx signal strength.
One E-link connects to the SCA on the ADDC board; another distributes the TTC signals (BCR and BC clock) to the ADDC board.
In all, 41~E-links connect to a MM-L1DDC.
One layer of \MM is readout by two L1DDC boards. The L1DDC's are installed along the edges of the outer layers of the \MM quadruplet.
The board power consumption is $\sim$5.5\,W.

%\FloatBarrier
