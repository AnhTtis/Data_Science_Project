The ASICs described in this section were developed by the CERN Electronic Systems for Experiments Group for use in several experiments.

\subsection{GBTx -- GigaBit Transceiver}
\label{sec:GBTx}

The GBTx ASIC\,\cite{Moreira:2009pem, Wyllie:2012cua}, aggregates many slow (2,\,4, or 8~bits per bunch crossing, i.e.\ 80, 160 or 320\,Mb/s)
serial data links called \emph{E-links} into a single serial link running at 4.8\,Gb/s.
It provides one such link in each direction; the two directions are completely independent.
The GBTx ASIC is radiation hard and uses forward error correction to assure uncorrupted data transmission. %mitigate Single Even Upsets.
Its net throughput is 3.2\,Gb/s with error correction, or 4.48\,Gb/s without error correction in ``Widebus mode'' used by the ADDC (See Section\,\ref{sec:addc}.).
The 120-bit payload frames are transported with fixed latency, synchronous with the LHC bunch crossing clock.
One GBTx can transport event, configuration, control and monitoring streams, and, by virtue of its fixed latency, the bunch crossing clock and TTC (Timing, Trigger and Control\,\cite{TTC, Gallno:1999ws}) signals.
The GBT-FPGA firmware\,\cite{GBT-FPGA, GBT_FPGA2} implements the GBTx protocol into an FPGA.

\noindent\textbf{E-links}
consist of a serial input, a serial output and an output clock, all of whose rates are separately configurable.
\gls{E-links} use the SLVS\,\cite{slvs} standard.
The  properties of different E-links are somewhat constrained (see\,\cite{FelixUserGuide}).
%See the FELIX User Guide for E-link rules\,\cite{FelixUserGuide}.
The output data and the output clock are transmitted with a fixed phase with respect to the recovered BC clock. %40\,MHz link clock.
To accurately capture the input data, each E-link's internal receive clock phase with respect to data arrival must be calibrated.
Data is transmitted transparently.
In the NSW, all E-links carry 8b/10b\,\cite{8b10b} encoded data, except the SCA E-links which carry High-Level Data Link Control protocol (HDLC)\,\cite{HDLC} encoded data and the TTC links which are not encoded.
The 8b/10b encoding further reduces the throughput to 2.56\,Gb/s.
Control symbols in the 8b/10b standard are used to delineate start and end of packets.
The jitter of the E-link clocks is of the order of 5\,ps which is within the working range of the data acquisition but is marginal for \gls{FPGA} gigabit transceiver reference clocks\,\cite{Elinkjitter}.
To accurately capture the input data, each E-link's internal receive clock phase with respect to data arrival must be calibrated. See Section\,\ref{sec:PhaseAlign}.

\subsection{SCA -- Slow Control Adapter}
\label{sec:SCA}
The \gls{SCA} \gls{ASIC} (Slow Control Adapter)\,\cite{GBT-SCA}, developed by CERN, is used to configure all the ASICs in the NSW.
It is part of the GBT chip-set and communicates to FELIX through a 80\,Mb/s GBTx E-link.
The E-link is encoded in \gls{HDLC}.
It provides communication to other chips using \glslink{I2C}{I$^2$C}, \gls{SPI}, JTAG protocols as well as General Purpose I/O.
The industry-standard OPC\,UA\,\cite{OPC-UA} was chosen as the software interface to the SCA to be easily compatible with the Detector Control System's SCADA program and to take advantage of industry standard software. Although initially the ASIC was featuring a faster ADC with more accurate sampling, in the end,  this could not be realised and a slower implementation based on a Wilkinson architecture was adopted.
The ADC implemented is a 12-bit ADC\,\cite{SCA-ADC} based on a single-slope Wilkinson architecture with a range up to 1.0\,V with a maximum conversion rate of 3.5\,kHz.
Conversions are triggered by software commands.
This implementation underestimates the VMM's measured ENC and scaling should be implemented (See Section\,\ref{sec:SCAconfig}).


\subsection{FEAST -- DC-to-DC converter}
\label{sec:FEAST}
The CERN \gls{FEAST ASIC}\,\cite{FEAST2.1} provides up to 10\,W of Point-of-Load non-isolated DC power from $\sim$10\,V input.
\gls{FEASTMP} pluggable modules\,\cite{FEASTMP} are also used.
The output voltage is set by a resistor.
Several voltages between 1.2\,V and 3.3\,V are required by the various NSW boards.
The device is radiation tolerant for Total Ionization Dose (TID) above 200\,Mrad(Si) and has an adjustable switching frequency between 1\,and 3\,MHz. For the FEAST ASICs in the NSW, the range is between 1.3 and 1.7\,MHz.

\subsection{VTRx, VTTx -- optical interfaces}
\label{sec:VTRxVTTx}
The \gls{VTRx} and \gls{VTTx} are radiation tolerant bi-directional and dual transmitter optical link interfaces respectively.
They target data transmission between the on-detector and off-detector electronics at rates up to 5\,Gb/s with an emphasis on High luminosity LHC level radiation resistance, low power dissipation and low mass components.
They are protocol-agnostic and use a standard \gls{LC}-LC optical connector mounted on a pluggable module.
Their development was a joint ATLAS-CMS project, the Versatile Link project\,\cite{versatile, Xiang:2012vua,Vasey:2012xjw}.
Care must be taken to adequately cool them which was foreseen in the NSW project.