\label{sec:VMM}
The VMM\,\cite{9724214, Iakovidis_2020, vmmuserguide} is a custom Application-Specific Integrated Circuit (ASIC). It is designed to be the front-end ASIC of both the Micromegas and sTGC detectors of the New Small Wheels.  For the NSW, it is packaged in a 400-ball $21\times 21\, \textrm{mm}^2$ Ball Grid Array (\gls{BGA}) with 1\,mm ball pitch.

\subsubsection{Requirements}
The 64 channels with highly configurable parameters meet the processing needs of signals from all sources of both detector types:

\paragraph{The Micromegas}\hspace{-0.3cm}signals from the anode strips (negative polarity signals), depending on the chosen gas
gain and shaper integration time, can be up to a maximum 250\,fC, but typically half or even smaller charge is expected. The fast electron current is followed by the positive ion current which typically lasts for $\sim$150\,ns\,\cite{georgePhd}. In addition to the current signal duration and maximum input charge, the other relevant parameter is the electrode (anode strip) capacitance which varies from about 50 to 300\pF depending on the length of the strips. The noise is a critical parameter for the Micromegas determined by the requirement of single primary electron detection with a threshold  five times the RMS noise, a gas gain of  10,000, and the maximum possible electrode capacitance of 300\,pF. These conditions determine the
required noise level to be at 0.5\,fC or about 3,000 electrons RMS.

\paragraph{The sTGC}\hspace{-0.3cm}feature three different types of active elements on a detector: strips, wires, and pads. All three are read out via the VMM. Strips provide the precision radial coordinate measurement for track reconstruction, wires the azimuthal coordinate; pads are used for a ``pre-trigger'' that requires a configurable coincidence performed by the Pad Trigger (See Sections\,\ref{sec:trigpath} and \ref{sec:pad_trigger}) firmware among the signals of pads in consecutive layers. The wire signals have negative polarity, while both the strip and pad signals are positive. Hence the need for the VMM to handle both polarities. The total charge and the long ion tail  impose specific requirements on the processing of the sTGC signals.  The VMM should recover from wire and pad signals of 6\pC and 3\,pC, respectively, within 250\ns while maintaining linearity up to 2\,pC.
For pads, it should provide the Time-Over-Threshold (ToT) and recover within 1\us from high charges up to 50\,pC. The pads impose challenging requirements since their capacitance can be up to 3\,nF. For the strips, an average charge of 1\pC is expected while the input capacitance is $\sim$200\,pF. As mentioned above, the sTGC signals span a very large range from 1\pC on a given
strip to about 50\pC on a pad. The dynamic range for the precision strip measurement is 2\,pC. The need to measure 2.5\% of this charge with a 2\% resolution and a 200\pF electrode capacitance, requires a noise level for a 25\ns integration time to be about 1\,fC RMS.
The noise for the signals from the pads with much larger capacitance (up to 3\nF) is
substantially higher.

Both detectors have similar readout requirements as the architecture is the same. A maximum trigger latency of 10\,$\muup$s must be supported, so the VMM needs deep enough FIFOs to buffer the data for this length of time. Moreover, a hit rate of up to 1\MHz per channel is expected.
On the other hand, the two detectors have different trigger requirements. The Micromegas need to provide the address of the VMM channel that fired first in a bunch crossing.
For the sTGC pad pre-trigger, which requires binary pad hits to select relevant strips, the VMM provides a Time-over-Threshold signal.
For the sTGC trigger, the VMM provides a 6-bit charge measurement of the strips within $\sim$50\,ns.
The sTGC wires do not participate in the trigger formation.

The VMM will  operate in a harsh radiation environment, see Table\,\ref{tab:radEnv} and\,\cite{Ameel, amideiDCDC, ATL-MUON-PUB-2022-001, nswTDR}.  Design techniques are applied to mitigate issues that may affect the operation of the ASIC under the above-targeted conditions.  Although \gls{TID} (Total Ionisation Dose) may degrade the performance of the ASIC,  the VMM3a was tested for TID tolerance in the  $^{60}$Co source irradiation facility at BNL for the expected radiation and no performance degradation was noticed.  Single event upsets (\gls{SEU}), though, become increasingly more serious. To overcome SEU's   in the vulnerable logic blocks (see Table\,\ref{radtablevmm}), Dual Interlocked storage Cells (\gls{DICE})\,\cite{DICE,DICE2} and Triple Module Redundancy (\gls{TMR})\,\cite{TMR} protection techniques are used.
For the large storage elements such as the latency \gls{FIFO}, an upset is just flagged once detected and the FIFO is reset.

\begin{table}[h]
\caption{ Single Event Upset protection schema in the VMM}
\vspace{5pt}
\label{radtablevmm}
\centering
\setlength{\tabcolsep}{3pt}
\begin{tabular}{ p{7cm}p{7cm}  }
\toprule
 Block & Type of protection \\
\midrule
  Global configuration and channel registers & DICE\\
  VMM State Machine & TMR\\
 Bunch Crossing Counter   & TMR\\
 L0 FIFO Control  & TMR\\
 L0 Event Builder & TMR\\
 L0 Accept register, NSkip Circuit & TMR\\
  Latency FIFO & Parity on pointer, FIFO reset on parity error\\

\bottomrule
\end{tabular}
\end{table}


\subsubsection{Architecture}
The analog front-end section of each channel integrates a three-stage low-noise charge amplifier\,(\gls{CA}) followed by a third-order shaper.  The charge amplifier implements a programmable input polarity, a test capacitor connected to the integrated pulse generator, a power-down option, a fast recovery option for very high-charge events, and several programmable bias adjustments to accommodate a broad range of signals. The input \gls{MOSFET} is a \mbox{p-channel}. It is followed by a dual cascode stage and a mirrored rail-to-rail output stage. The shaper features programmable peaking time of 25,\,50,\,100, and 200\,ns.  The gain is adjustable in eight values (0.5,\,1,\,3,\,4.5,\,6,\,9,\,12,\,16\,mV/fC). A low-frequency non-linear feedback baseline holder\,(BLH) stabilizes the output baseline, referenced to an on-chip band-gap reference circuit set at 160\,mV. The BLH has a programmable bandwidth that allows the user to enable either a mild or a strong (effective bipolar shape) compensation introduced to handle the \gls{sTGC} long current due to the long drift of ions.

Following the analog front-end is the mixed-signal section that includes discrimination, peak and timing detection measurements and the corresponding analog-to-digital conversions. The threshold is adjusted by a 10-bit Digital to Analog Converter (DAC) common to all channels plus a local 5-bit trimming \gls{DAC} independently adjustable in each channel in 31 steps of approximately 1\,mV each. The peak detector
measures the peak amplitude and stores its output (PDO) in an analog memory.  Additionally, it provides the timing signal at the time of the peak of the analog pulse. The time detector measures the timing using a time-to-amplitude converter (\gls{TAC}) which arms at the rising edge of the bunch crossing clock (CKBC) and latches at its falling edge. The time detector output (TDO) value is stored in an analog memory. The ramp duration can be configured to 60, 100, 350 or 650\,ns.
For the NSW, the value of 60\ns is used; this is enough to cover the duration of the CKBC while the TAC is in its linear range. The  block diagram of one of the 64 identical channels is shown in Figure\,\ref{fig:GI_VMM_architecture},  delimited with a dashed box, along with the relevant parts shared by all the 64 channel circuits and signals.

The VMM neighbor option triggers the two neighbors of a triggered channel, irrespective of whether they cross threshold.
This allows raising the threshold while still digitizing the edges of a spatial charge distribution. The functionality is applicable across different VMMs through dedicated electrical lines.
%This lowers rate, which lowers dead time,
%but the two neighboring channels will also be dead when the channel above threshold triggers, thus increasing their dead time.

\begin{figure}[ht]
\centering
\includegraphics[width=0.99\textwidth]{figures/GI_VMM_architecture}
\caption{Overview of the VMM architecture}
\label{fig:GI_VMM_architecture}
\end{figure}

The mixed-signal part of the ASIC is followed by three current mode ADCs per channel. A pedestal of 150\mV is subtracted before digitization. This way the range of the ADC's is increased. The 10-bit and 8-bit ADC's are two-stage conversion digitizers providing charge and time measurements, respectively. The per-channel dead-time is driven by the 10-bit \gls{ADC} conversion that is configurable down to $\sim$250\ns, giving an effective rate of $\sim$4\,MHz per channel. The 8-bit ADC and a coarse 12-bit BC counter provide a 20-bit time stamp.  Each channel has a direct dedicated output (\gls{DDO}) where the 6-bit ADC provides the same charge measurement from the \gls{PDO} but in a much faster dedicated path with $\sim$50\ns dead-time.  The channel remains inactive until the 10-bit ADC completes the conversion. The ASIC provides the ability to interrupt the 10-bit conversion once the 6-bit conversion finishes, such that the DDO dead-time is small. In that case, the 10-bit information is unusable. The same DDO can be configured to provide pulses indicating Time-Over-Threshold\,(\gls{ToT}), Time-To-Peak\,(\gls{TtP}), Peak-To-Threshold\,(\gls{PtT}) or a Pulse-at-Peak\,(\gls{PtP}) of 10\ns duration. The address of the channel that registered the first hit per CKBC cycle, is output on a dedicated per-chip serial line. This is called the Address-in-Real-Time\,(\gls{ART}).


\subsubsection{Readout schema}
Although the VMM features more than one readout schema, the so-called ``L0'' mode is designed for operation within the ATLAS experiment. The output of the 10-bit and 8-bit ADCs enter into a 64-deep FIFO per channel called the ``Latency FIFO''.  Given the size of this FIFO and the 250\ns dead-time per channel, a maximum guaranteed latency of 16.0\,$\upmu$s where no data is lost can be achieved. This is larger than the minimum 10\,$\upmu$s required by ATLAS Trigger-DAQ{\,\cite{CERN-LHCC-2017-020}}.
Note that each channel is autonomous and this FIFO is filled asynchronously.

Each channel has a Level-0 Selector circuit that is connected to the output of the channel's latency FIFO.
The selector finds events within the BCID window (configurable at a maximum size of 8 BC's and the BC is offset by the latency) of a Level-0 Accept and copies them to the ``L0 channel'' FIFO.  If a channel's L0 selection circuit does not find a hit within the BC window,  a ``no data'' item is passed to the ``L0 channel'' FIFO. In this way the ``L0 channel'' of all 64 channels overflow synchronously. The ``L0 BCID'' FIFO is made deeper, 32-deep, than the ``L0 channel'' FIFO. This way, once the ``L0 channel'' overflows, the VMM data can still indicate on which BC this happened and can skip a configurable number of triggers to recover from the overflow while still maintaining its synchronous data-taking. The buffering scheme of the VMM L0 is shown in Figure\,\ref{fig:VMM3intBuf_V04}.

\begin{figure}[ht]
\centering
\includegraphics[width=0.65\textwidth]{figures/VMM3intBuf_V04}
\caption{Overview of the L0 buffer}
\label{fig:VMM3intBuf_V04}
\end{figure}

The data transfer from the VMM is done via two serial lines running at 160\,MHz with Double Data Rate (DDR), giving a total bandwidth of 640\,Mb/s. Two lines are used to reduce the clock rate. The Readout Controller supplies the clock for this transfer. The data is encoded in 8b/10b with one or more comma characters transmitted continuously between Level-0 events. The 8b/10b encoding reduces the effective bandwidth to 512\,Mb/s.

\subsubsection{Trigger outputs}
The VMM must provide different trigger primitives for the four different detector elements connected.
This is achieved through different data outputs and specific configuration. It is possible to turn off the \gls{SLVS} drivers of trigger output lines not in use in order to reduce power consumption.% and the possibility of digital interference with the VMM's front-end operation.

\paragraph{The \gls{Micromegas} trigger}\hspace{-0.3cm}utilises the fine strip pitch and the ionisation spread across the path of a particle crossing the detector at an angle as used in the $\muup$TPC method\,\cite{ALEXOPOULOS2019125}.
The VMM sends out this first address through a dedicated serial line called Address in Real Time (ART).
This signal is asynchronous to the bunch crossing clock so the VMM can be configured to align it with the BC clock.
The ART clock is provided to the VMM externally from the Readout Controller (ROC); see Section\,\ref{sec:ROC}.
The ART signals can be provided at threshold crossing or at peak found, depending on the configuration.
It can be optionally clocked at both edges of the 160\,MHz clock.


\paragraph{The sTGC trigger}\hspace{-0.3cm}is done in two steps as described in Section\,\ref{sec:trigpath}:
1) A pre-trigger: Overlapping induced cathode pads define a candidate track segment by a configurable coincidence in a projective tower in both quadruplets in a sector made by the Pad Trigger module.
Each VMM connected to pads operates its direct outputs in Time-over-Threshold (ToT) mode.
Each channel self-resets at the end of the timing pulse, thus providing continuous and independent operation of all 64 channels.
2) The projective tower defines the bands of strips (if any) to be read out from each TDS in each layer. The VMM channels connected to the strips digitize the charge through the fast 6-bit ADC and provide them through the direct output.  The channel reset occurs after the last bit has been shifted out. The data are clocked at both edges of the 160\,MHz clock.
