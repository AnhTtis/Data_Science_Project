\section{Calibration}
\label{sec:calibration}

%List of VMM, ART, GBTx, TDS, Pad Trigger calibs, configurable phases, configurable delays, BCID offsets and BC clock phases

\subsection{Phase alignments}
\label{sec:PhaseAlign}
The primary clock source in the NSW electronics is the Bunch Crossing (BC) clock which is distributed by the ATLAS TTC system\,\cite{TTC}. The NSW electronics generates all the other needed clocks based on the BC clock. Since the data generated in one board may not be aligned with the clock needed to decode these data on another board, a clock-data alignment needs to be performed. This operation is performed by shifting the clock phase with respect to the data such that the decoding is correct.  The following parts need to be aligned in NSW electronics:
%\begin{itemize}\itemsep-2pt

\para{ROC TTC reception:}The ROC ASIC receives and distributes the eight TTC bits per BC forwarded by FELIX that are representing the TTC signals specified in Section\,\ref{sec:overall}. Since the eight bits have no begin or end marker, the BC clock needs to be shifted such that the TTC bits are correctly extracted.

\para{VMM-ROC data line:}The VMM transmits data to ROC using a 160\,MHz clock. When not transmitting data, the chip transmits K28.5 idle characters. Although the ROC provides the data clock to the VMM, the signal received is shifted with respect to the transmitted clock. To compensate for that, ROC generates an internal copy of this clock which can be shifted accordingly and hence decode the data correctly.
Errors detected by the ROC while decoding are flagged in a status register that can be read out.
%Moreover, the ROC implements registers that can be read out flagging and error while decoding.


\para{VMM-ART serial stream:}The VMM transmits serially flag and address information to the ART
using a 160\,MHz clock provided by the ROC. The ART shifts the phase of the input serial stream and decodes the received data with its local 160\,MHz clock.

\para{TDS calibration:} Two VMMs transmit serially 6-bit charge information (after a flag) to TDS using a 160\,MHz clock provided by the TDS to which it is connected.
%TDS using a 160\,MHz clock provided by the TDS.
For each of the two VMMs, the phase of the TDS must be calibrated so that the TDS sampling point allows a correct decoding of the data.
The TDS BC clock is provided by the ROC; its phase has to be calibrated to correctly receive the Pad Trigger data.
The relative time offset between the reception of the pad data and the reception of the strip data needs to be calibrated so that the strip data matches the BCID requested by the Pad Trigger.

\para{Trigger Processor - Router, ADDC, Pad Trigger:}The Trigger Processor receives serial streams from the ADDC, the Router and the Pad Trigger boards. In all cases, the streams are deskewed to account for differences in cable lengths and
deserialization. Deskewing requires reading the BCID of each incoming stream independently,
and aligning them with a per-stream delay in units of the local decoding clock. This clock is
240\,MHz for the Pad Trigger and 320\,MHz for the Trigger Processor.

\para {GBTx:}To deserialize the input data stream from, e.g.\ the ROC ASIC, the phase of the GBTx E-link receive clock must be adjusted to sample the incoming data at the correct time.
The GBTx output E-link clock is used by the front-end ASICs to transmit data back to the GBTx. The correct phase of the receiving clock depends on the ASICs internal delays and the cable delay between GBTx and Front-end.
The GBTx features phase aligner circuits, one per E-group each with eight adjustable channels, as all the E-links in an E-group may have different phases.
The phase aligner can operate in three modes: static, automatic phase tracking and initial training with learned static phase selection. The latter two are proposed for the NSW operation but performance is still to be demonstrated.
%\end{itemize}


\subsection{SCA-ADC based calibration}
\label{sec:SCAconfig}
This type of calibration does not involve the L1A data. It uses the ADC implemented in the SCA to sample the Monitor Output (MO) of the VMM. Since the MO is routed on a connector physically on the front-end boards, the VMM is configured to copy the MO to the PDO output which is routed through a voltage divider, in the case of MMFE8, to the ADC multiplexer of the SCA. The voltage divider is necessary since the MO output has a range up to 1.2\,V, while the ADC of the SCA has a 1\,V range. The sTGC Front-end board does not implement a divider. All the measured parameters are needed for the optimal configuration of the NSW electronics to acquire data. The procedure is to configure the VMMs to output different parameters, sample the outputs and store the data.
Figure\,\ref{fig:GI_calib}\,(left) shows the schematic of the parts involved. The configuration software transmits commands through the OPC client to the SCA OPC-UA server which, via FELIX, propagates the commands to the Front-end ASICs. Then, through the same path, the SCA is instructed to sample the ADC output and the data are made available through the SCA server to a custom data handler.


\begin{figure}[ht]
\centering
\includegraphics[width=0.48\textwidth]{figures/GI_SCA_Calib}
\includegraphics[width=0.48\textwidth]{figures/GI_dt_calib}
\caption{Left: Schematic representation of the NSW electronics setup indicating the paths involved in the SCA-ADC based calibration. Commands are transmitted through the OPC client and server via FELIX to the on-detector electronics. The SCA ADC is instructed to sample the analog output of the VMM and data are transmitted back to the custom data handler software. \\
Right: Schematic representation of the data-taking path in the calibration procedure. The dedicated configuration of the DAQ system (partition) configures the system with the OPC client-server through FELIX; the TTC is configured to produce a sequence of test pulses and L1A and L0A signals. Data are captured by the swROD.}
\label{fig:GI_calib}
\end{figure}


%\begin{figure}[ht]
%\centering
%\includegraphics[width=0.5\textwidth]{figures/GI_SCA_Calib}
%\caption{Schematic representation of the NSW electronics setup indicating the paths involved in the SCA-based calibration. Commands are transmitted through the OPC client-server and through FELIX, to the on-detector electronics. The SCA ADC is instructed to sample the analog output of the VMM and data are transmitted back to the custom data handler software.}
%\label{fig:GI_SCA_Calib}
%\end{figure}

\noindent The following types of calibration are foreseen and possible through this procedure:
%\begin{itemize}\itemsep-2pt

\para{Baseline and ENC:}The \gls{MO} output of the VMM is configured to be sent out for each channel at a time. In this configuration the analog output of the amplifier after shaping is made available. The channel baseline is sampled in a configurable number of samples from which the mean voltage level can be derived (baseline) and the spread (\gls{ENC} estimation). It is worth mentioning that the SCA ADC has a single slope Wilkinson architecture; hence, the slow ramp is ideal for slowly varying parameters but underestimates the fast ones. Consequently, the noise estimation on the detector is underestimated. To establish the real noise level on the detector, a correction factor needs to be applied.

\para{Pulser and threshold DAC:}Through this measurement, the correct threshold can be applied to the electronics, and the input charge can be converted from DAC counts to mV.

\para{Channel trimmers:}During this calibration, the individual 5-bit trimmer of each VMM channel can be set such that the amount of charge from baseline to the discrimination level is equalised across the channels.
%\end{itemize}

Illustration of the results of these calibrations can be found here\,\cite{9724214}. Through the same procedure, the temperature and the band-gap voltage of the VMM can be measured for monitoring.


\subsection{Data-driven calibration}
Through this type of calibration, several parameters can be extracted that can be used as calibration parameters to reprocess the acquired data. The sequence of this calibration type involves the full data-taking with L1A data. The data are generated through the internal VMM pulser that is driven through a dedicated TTC bit such that they are synchronous along the system.  For this sequence, a dedicated
configuration of the data-acquisition system which involves the configuration procedure through the OPC client-server is used. The TTC system is also part of the calibration since it needs to produce a specific time sequence of signals to drive the VMM pulser and produce a L1A and L0A to read back the data. The data are captured by an instance of the swROD not connected to the ATLAS High Level Trigger. The schematic of this sequence is shown in Figure\,\ref{fig:GI_calib}\,(right).
The following types of calibration are required through this procedure:
%\begin{itemize}\itemsep-4pt

\para{VMM channel gain calibration:}By injecting different amounts of charge in the VMM and storing the PDO, the gain of the electronics can be established which allows correction of the charge measurement results. During this procedure, the time-walk can also be measured using the \gls{TDO} data.

\para{Time-to-Amplitude Converter (TAC) calibration:}The VMM records as TDO a voltage level which is the amplitude of the TAC starting either at peak or threshold and stopping at the falling edge of the bunch crossing clock. The command to inject charge arrives via the TTC system and hence is synchronous to the system and the bunch crossing clock.
The ROC has a configurable delay between the bunch crossing clock and the clock used to inject the charge. By stepping this delay, the slope of the TAC ramp can be measured and used to translate ADC counts into meaningful time units.
%\end{itemize}

\subsection{Configurable delays}
As mentioned above, the clock distribution in NSW has to follow paths of different length to reach various front-end boards. In addition, the time-of-flight from particles originating from the ATLAS interaction point (IP) differs along the radius of the detector.  The sTGC pad-generated signals are transmitted through PCB traces with significantly different lengths; hence they must be aligned.
To compensate for the above-mentioned effects, configurable delays have been implemented in different ASICs. Those delays must be tuned to assign the correct time stamp on the recorded event.