\label{sec:sfeb_pfeb}

The strip and pad Front-end boards\,\cite{Miao_2020} interface to detector strips, pads and wire groups, providing their data to the readout path on receiving a Level-1 trigger signal and to the trigger path on every bunch crossing.
There are separate radiation-tolerant Front-end boards for strips and pads+wires.
Both are the result of several demonstrator and prototype boards, including those with earlier versions of the VMM and those with FPGA readout instead of the Readout Controller (ROC) ASIC.
VMM ASICs (see Section\,\ref{sec:VMM}) provide data to both trigger and readout paths.
An SCA ASIC provides a unique 32-bit ID for the board.

For each gas gap, detector signals connect to adapter boards on each of the two radial edges of the detector.
A strip Front-end board (sFEB) reads out strips from one edge, and a pad+wire Front-end board (pFEB) reads out pads and wires from the other.
In total, there are about 30 wire groups, up to 112 pads, and up to 400 strips reading out an sTGC gas gap.
The adapter boards route the signals to high-density (10\,$\times$\,30 contacts) low-profile matrix interposers (Samtec GFZ)\,\cite{GFZ} (one for pads, two for strips) on the adapter board to which the Front-end boards connect.

Careful placement, layout and shielding of the on-board FEAST DC-DC power converters was essential and is described in\,\cite{Miao_2020}.

\para{Input transient voltage suppression and signal conditioning:}
Gas detectors have high voltage discharges and the front-end VMM ASIC inputs must be protected.
Each of the VMM inputs on the board is protected through a dedicated diode network which was extensively tested and was found to protect the ASIC from any discharges on the detector\,\cite{Iakovidis:2675779}.
Figure\,\ref{fig:TVS_Pi_blockDiagram} shows the input circuitry that protects against transients and compensates for the high rate and high charge signals for strips, pads and wires.
\begin{figure}[h]
\centering
\includegraphics[width=0.75\textwidth]{figures/GI_pfeb_sfeb_schematic.pdf}
%\includegraphics[width=0.93\textwidth]{figures/PM_TVS_Pi_blockDiagram.pdf}
\caption{The sTGC FEB analog input circuit which protects against transients and compensates for high rate and high charge signals for pads\,(blue, left), wires\,(red, middle) and strips\,(yellow, right).}
\label{fig:TVS_Pi_blockDiagram}
\end{figure}
The first stage is a Transient Voltage Suppressor (TVS), a Semtech \textmu Clamp\,\cite{semtech}, which can mitigate fast transients.
The last stage is a 10\,$\Omega$ resistor that connects to one of the four channels of a diode array, SP3004\,\cite{sp3004}.
This diode array mitigates slower input transients.
In between, are schemes that compensate for the sTGC high rate and high charge signals, which include a level of uncertainty. Different schemes are needed for wires, pads and strips\,\cite{pi-networks}.

The amplitude and time structure of the sTGC signal require the use of an attenuator circuit in the case of the pads and wires, and a pull-up resistor circuit in the case of the strips, in order for the VMM to operate optimally under the conditions of the HL-LHC.

\vspace{3pt}\noindent\textit{Pad and wire input $\uppi$-networks:}
At the operational voltage of the sTGC, a single cavern background hit can induce a 50\,pC charge into a single sTGC pad, which exceeds the design requirements of the VMM front-end\footnote{At the time of the VMM design the input signal of the sTGC pads was underestimated and the input current would exceed the specifications.}.
This charge is sufficient to saturate the VMM feedback currents (configurable), causing an undesired and overly long recovery time.
To minimize this effect, a $\uppi$-network was implemented\,\cite{pi-networks} and is shown in Figure\,\ref{fig:TVS_Pi_blockDiagram} (opposite polarity for wires and pads).
The network acts as a charge divider, reducing the charge input to the ASIC front end.
The $\uppi$-network capacitor value was optimized to preserve efficiency and minimize recovery time.
Due to the difference in the pad capacitance along the detector radius, a different capacitor is used for the inner, middle, and outer quadruplet.
This circuit is directly implemented on the Front-end boards.
The optimization was based on radiation data (test-beam plus backgrounds), simulation, and charge injection studies.
In these studies, the optimal attenuation factor was found to be approximately 5:1, corresponding to 200\,pF, 330\,pF, and 470\,pF capacitors in the $\uppi$-network for inner, middle, and outer
%Q1, Q2 and Q3
pads FEBs respectively and 200\,pF for wire FEBs.

\vspace{3pt}\noindent\textit{Input pull-up resistor for strips:}
The signal formation of the sTGC detector is characterized by three components with different timescales:
The first component has a characteristic timescale of $\sim$20\,ns and is due to the electron avalanche drifting towards the wires.
The second component has a characteristic timescale of tens of microseconds and is due to the ion drift towards the cathode planes.
The third component has a characteristic timescale of milliseconds and is due to the charge induction within the resistive layer following the ion arrival to the layer,
and hence has the opposite polarity to the previous two\footnote{Sufficiently high resistance would neglect this component, but in the sTGC detector, in order to dissipate charge from the high background, it is not sufficiently high and a bipolar shaped signal is formed.}.
At the high rates of the HL-LHC, the second component from different signals overlap, causing a constant current into the VMM, which can be compensated by the Front-end, but strong bipolar shaped signals cannot.
Hence, although not necessary, a pull-up resistor was implemented for the strips, as shown in Figure\,\ref{fig:TVS_Pi_blockDiagram}.
In this case, the VMM baseline restoration circuit is protected by providing an additional constant feedback current.
A resistor of 400\,k$\Omega$ connected to supply voltage (1.2\,V nominal) was chosen.

\para{Strip front-end boards:}
The sTGC strip Front-end board is a dense 14-layer circuit board (27.5\,cm\,$\times$\,7.6\,cm) comprised of six or eight VMMs, one ROC, one SCA, six FEAST, and three or four strip-TDS ASICs.
The outer two of the three quadruplets have fewer strips and so require only six VMM and three strip-TDS ASICs to be populated on the board.
%The same board, rotated 180\textdegree\ around its z-axis, is used for both rising and falling edges. Therefore the radial coordinates of one are reversed from the other.
Boards mounted on one edge of the detector are rotated 180\textdegree\ around its $z$-axis with respect to the other edge.
Therefore the radial coordinates of one are reversed from the other.
Board power consumption is 21\,W.
Figure\,\ref{fig:sFEB_block_V01} shows a block diagram of the board.

\begin{figure}[ht]
\centering
\includegraphics[width=1.0\textwidth]{figures/sFEB_block_V01.pdf}
\caption{Block diagram of the sTGC strip Front-end board that handles the strips in one sTGC gas gap, showing
         its ASICS, connectors and power blocks.
        % When configured for 640\,Mb/s, the readout E-links use a second twin-ax line.
        The second MiniSAS connector is used only for the inner quadruplet.}
        % the GFZ input connector, the ESD block, the VMM front-end ASIC, the strip TDS ASIC, the GBT-SCA ASIC,
        % the Readout Controller ASIC, the FEAST DC-DC converters,
        % the MiniSAS cable connectors for the connections to the Pad Trigger, Router and L1DDC,
        % and their interconnections.
        % The 2.5\,V power is supplied by a precision voltage reference.}
\label{fig:sFEB_block_V01}
\end{figure}

The seven serial LVDS lines from the Pad Trigger include four separate Band-id lines for the four strip-TDS ASIC positions on the board, a frame, BCID and a clock.
The BCID, clock and frame are distributed to the four strip-TDS ASICs by three 1-to-4 multiplexers.
Rather than an additional FEAST, three 2.5\,V precision voltage references, TL431AIDBZR\,\cite{TL431}, power them from the 10\,V supply.

\para{Pad+wire front-end boards:}
The pad+wire Front-end board is a dense 12-layer circuit board (16.3\,cm\,$\times$\,7.6\,cm) comprised of three VMMs, one ROC, one pad-TDS, one SCA and three FEAST ASICs.
The sTGC pads connect to two of the VMMs, wire groups to the third VMM.
%The same board, rotated 180\textdegree\ around its z-axis, is used for both rising and falling edges. Therefore the wire azimuthal coordinates of one are reversed from the other.
Boards mounted on one edge of the detector are rotated 180\textdegree\ around its $z$-axis with respect to the other edge.
Therefore the wire azimuthal coordinates for half the layers are reversed from the other and the pad numbering differs for different layers.
Board power consumption is 9\,W.
Figure\,\ref{fig:pFEB_block_V01} shows a block diagram of the board.


\begin{figure}[ht]
\centering
\includegraphics[width=0.92\textwidth]{figures/pFEB_block_V01.pdf}
\caption{Block diagram of the sTGC pad Front-end board that handles the pads and wires in one sTGC gas gap, showing
         its ASICS, connectors and power blocks.}
\label{fig:pFEB_block_V01}
\end{figure}

%\FloatBarrier

%%%%%%%%%%%%%%%%%%%%%%%%%%%%%%%%%%%%%%%%%%%%%%%%%

%\begin{comment}
%
%
%pad-FEB: The pad-FEB will be responsible for the readout of both pads and wires.
%Trigger path: The output of the pad-TDS is routed to a Serial Attachment (SATA) connector which is connected to the Pad Trigger Board via a 4:1 SATA-to-MiniSAS cable.
%Since the wires do not participate in the trigger, the wire VMM direct outputs are not used.
%Refs to VMM and TDS section.
%
%Readout path:
%ref to VMM section
%
%TVS
%Pi network
%Noise, Shielding and EMI
%power: FEASTs +
%cooling
%
%pFEBs an sFEBs power consumption is 9\,W and 21\,W, respectively.
%
%FEAST circuits. Care was taken to move the FEAST circuits away from the detector signal traces.
%
%Pi networks.
%
%The noise study on the bench shows negligible levels of noise coupling from the FEAST DC-DC converters to the sensitive VMM analog circuit.
%Integration studies with the full-size sTGC quadruplets indicate no noticeable noise coupling from the digital circuit.
%The pick-up noise from the detector due to the FEAST switching current irradiation has been minimized to a negligible level
%thanks to the attentive layout and the effective shielding of the FEAST modules.
%
%The high demand to accommodate up to 512 analog channels with a dense input network in a very constrained space,
%minimize the conducted and irradiated noise from the onboard switching regulators and digital ASICs,
%and handle a large number of high-speed inter-connects among highly integrated ASICs,
%all bring orders of magnitude complexity as well as great challenges to the design
%comparing with the previous generation FEB's used in similar experiments.
%Moreover, special care is needed for the FEB's to deal with the harsh radiation and the magnet field environments at the same time.
%Detailed analysis and strategic discussions on how these challenges have been addressed are presented
%
%
%I. Ravinovich, The New Small Wheel background rates predictions based on CSC and 2 TGC Run-2
%measurements, ATL-COM-MUON-2016-006 (2016).
%
%\end{comment}
