\section{Trigger latency}
\label{sec:latency}

The ATLAS Run-3 trigger latency is required to not exceed the limits imposed by those detectors that are unchanged from the previous running period.
Considerable design and implementation effort was made to minimize the NSW latency so as not to be on the critical path.
For example, a low-latency path dedicated to trigger fibres is used between the detector and the radiation-protected room (USA15);
Pad Trigger Band-ids are sent directly to the sTGC Trigger Processor to allow it to prepare the needed look-up table values and
the internal routing of strip data before the strip data arrives.
Table\,\ref{tab:latency} shows the large contributions to latency due to the size of the NSW and the length of the trigger path.
Details of the measured latencies and the signal path may be found in\,\cite{NSWlatency}.

% Table generated by Excel2LaTeX from sheet 'Sheet1'
\begin{table}[htbp]
  \centering
  \caption{Contributions to the trigger latency}
    \begin{tabular}{lr}
    \toprule
    \textbf{item} & \textbf{ns} \\
    \midrule
    longest time-of-flight from the IP &  31 \\
    electronic modules       & 405 \\
    serializer-deserializers & 189 \\
    Pad traces \& twin-ax cables &  136 \\
    All fibres               & 329 \\
%?  All fibres    & 471  \\
    ~~~~including from the Large Box to   &      \\
    ~~~~the Trigger Processor (42.5\,m)   &  212 \\
    \bottomrule
    \end{tabular}%
  \label{tab:latency}%
\end{table}%

%\FloatBarrier