\label{sec:mmtrigger}
\para{The Micromegas trigger,} although being a 2.1\,M channel system, utilises the Address in Real Time (ART) output of the VMM to scale down the system to $\sim$262\,k channels for the trigger. The concept utilises the fine pitch of the \MM detectors and the spread of ionisation charge for particles crossing the detector at an angle\,\cite{georgePhd}. The VMM sends out the address of the channel that presents the earliest signal in every bunch crossing.
For the fine $\sim$0.45\,mm strip pitch, this is a good approximation of the coordinate perpendicular to the strips direction; see Figure\,\ref{fig:ART_concept}.
\begin{figure}[ht]
\centering
%\includegraphics[width=0.55\textwidth]{figures/ART_concept.pdf}
\includegraphics[width=0.7\textwidth]{figures/GI_art_concept.pdf}
\caption{A simulated event showing the ionisation from a particle crossing the \MM detector at an angle.
The address of the strip to which the charge arrives first is output as an ART signal\,\cite{georgePhd}.}
\label{fig:ART_concept}
\end{figure}
The address is sent to the ART ASIC on the ADDC board which receives the addresses of 32~VMMs and selects eight of them to be sent out to the Trigger Processor.
The 32~ART ASICs in a sector collect the addresses from all the eight layers and transmit them serially on 32~fibres to the Trigger Processor (one per sector) which forms track segments.
Since the drift time of the \MM detectors can extend up to 150\,ns, the ART ASIC has the option to mask the input of a VMM which has already provided the strip with the earliest time within the drift time of the \MM.
Therefore, signals originating from the same particle track, but from a different strip, can be discarded if desired.
