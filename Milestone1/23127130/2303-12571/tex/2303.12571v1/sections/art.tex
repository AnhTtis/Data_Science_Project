\label{sec:ART}
The ART ASIC is part of the trigger path of the \MM chambers.
It receives the prompt ``Address in Real-Time'' (ART) data from 32 VMM front-end ASICs which consist of the address of the first arriving hit in each 64-channel VMM for a given bunch crossing.  It then selects up to eight  addresses for each bunch-crossing and sends the data to the \MM Trigger Processor via one GBTx chip configured to operate in ``Wide'' mode, i.e.\ without Forward Error Correction (FEC).

\noindent In particular, the ART ASIC performs the following functions:\vspace{-6pt}\begin{itemize}\itemsep-4pt
\item Deserialize each ART stream and phase-align the hits to the BC clock.
\item Selects the strip addresses of up to a fixed number of hits by means of cascaded priority encoders.
\item Append the 5-bit geographical physical VMM address to the strip address of each hit (defined by the cable connections).
\item Send the ART addresses and the 12-bit BCID to the \MM Trigger Processor via a GBTx.

\end{itemize}


The block diagram of the ART ASIC is shown in Figure\,\ref{fig:block_ART}.  The ASIC receives its TTC stream, configuration and clock signals from the GBTx (which receives them through the Level-1 Data Driver Card (\gls{L1DDC}) downlink, see Section\,\ref{sec:addc}). The ASIC has the option to output the Bunch Crossing Reset (\gls{BCR}) \gls{TTC} signal received to another ART ASIC\footnote{The second ART on the ADDC.}.
The ASIC is configured through the SCA using an I$^2$C bus.  The ASIC is packaged in a 128-pin LQFP package.
The latency of the ART ASIC was measured in several test setups to be $\sim$44\,ns.
Further information can be found in\,\cite{ARTASIC}.

\begin{figure}[ht]
\centering
\includegraphics[width=0.95\textwidth]{figures/GI_art_architecture_nolabel}
\caption{Block diagram of the ART ASIC}
\label{fig:block_ART}
\end{figure}



\paragraph{Programmable delays:}\hspace{-8pt}The purpose of the Programmable Delay block is to be able to skew the input signals to the local clock phase and adjust the ART stream to it.  To perform that, the ASIC uses four copies of the 8-channel Phase Aligner core developed at CERN\,\cite{Tavernier_2012}.% which can automatically align the ART signals with the 160\MHz clock provided.

\paragraph{Deserializer:}\hspace{-8pt}The first signal that precedes the ART 6-bit address is a flag.
This is generated by the VMM and in DDR mode, the flag pulse raises asynchronously to the ART clock (CKART) and is kept high through the next two falling edges of the 160\MHz clock, being lowered by the second falling edge.
Optionally, the rising edge of the flag can be registered to the CKART clock. A 10\ns reset period is applied after each ART sequence.  The eight flip-flops form an 8-bit DDR shift register which deserializes the incoming data stream.

\paragraph{Programmable dead time:}\hspace{-8pt}This block creates an artificial dead time for each VMM input which is controllable via configuration. Subsequent data on a particular input is ignored for between 0 and 7 BC's. This prevents responding to any subsequent ART signals from the same particle crossing the \MM.
See Figure\,\ref{fig:ART_concept}.

\paragraph{Priority selection:}\hspace{-8pt}
The Hit Selection circuit is based on eight layers of cascaded priority encoders.
The first priority encoder selects the first ART flag (most significant bit which is not zero from the 32-bit ART flag word).
The 32-bit word is readout as the first ART and then removed from the array. The consequent word (without the first ART) is presented to the following stage. The second priority encoder selects the second non-zero bit from the ART flag word in the same manner as the first stage.
The operation is cascaded eight times to select a maximum of eight non-zero flags.
The operation is illustrated in Figure\,\ref{fig:art_priorityenc}. If there are more than eight ART signals in the same BC window,  then the ART ASIC selects only eight of them, based on the priority scheme explained. The ART can be configured to give priority to different radii of the detector\,\cite{nswTDR}.


\begin{figure}[ht]
\centering
\includegraphics[width=0.9\textwidth]{figures/GI_art_priority_encoders}
\caption{Hit map generation circuitry based on priority encoders. The duration of the procedure is $\sim$3\,ns.}%timing was provided by Sorin
\label{fig:art_priorityenc}
\end{figure}


\paragraph{Output:}\hspace{-8pt}For each bunch crossing, the ART ASIC transmits to the Trigger Processor through the GBTx, the following data:
\vspace{-6pt}\begin{itemize}\itemsep-4pt
\item{Up to eight  VMM ART signals which had a hit in that bunch crossing}
\item{The 12-bit BCID in which the ART signals occurred}
\item{Other information (error flags, parity bits)}
\end{itemize}

\noindent The Wide Bus mode of the GBTx chip (see Section\,\ref{sec:GBTx}) allows for 112-bits to be transmitted in one bunch crossing on two 80\,MHz clock edges. The ART data are transmitted to the GBTx chip in two batches:\vspace{-6pt}\begin{itemize}\itemsep-4pt
\item{The first 56-bits contain the selected VMM hit list based on the flag bits issued. This can be configured in two different modes: either transmit a ``Hit Map'' 32-bit word where each bit corresponds to one of the 32 VMMs connected to the ART ASIC or, the ``Hit Address'' option containing a 5-bit VMMID for each of the VMMs selected. In both cases, the 12-bit BCID is transmitted along with the Hit Information.}
\item{The second batch of 56-bits contains the eight 6-bit VMM ART addresses of the VMMs selected along with an 8-bit parity bit for each word.}
\end{itemize}



\paragraph{Debugging mode:}\hspace{-8pt}Besides the normal operation modes, the ASIC implements a debugging mode where different parts of the system are bypassed or fixed, or, repeating calibration patterns are transmitted. The following debugging modes are implemented:
\vspace{-6pt}
\begin{itemize}\itemsep-4pt
\item{Full bypass mode where the output of the programmable delays are directly made available to the outputs of the ASIC. This is used to verify and measure the propagation delay during ASIC initial verification and it is not accessible during normal operation in ATLAS.}
\item{Priority encoders bypass mode where the input channels are connected directly to the output logic. This mode is controlled by a configuration register and allows the verification of the transmission between VMM chips and ART ASIC bypassing any selection, while permitting the definition of the propagation delays set in the Programmable Delays. }
\item{Fixed output calibration pattern where a fixed pattern is sent continuously. The pattern is stored in local configuration registers and is accessible via the configuration path which allows the verification of the data transmission.}
\end{itemize}


\paragraph{Radiation protection:}\hspace{-8pt}The ART ASIC is designed with several protection mechanisms in order to ensure protection against SEU events or to flag uncertain conditions.
The SEUs in the data will not seriously affect the trigger. However, simple parity bits are calculated by the input deserializer circuits and can be used downstream to tag possible data corruption.  The circuits protected with triple modular redundancy (TMR)\,\cite{TMR} or parity bits are shown in Table\,\ref{radtable_ART}.


\begin{table}[h]
\centering
\caption{TMR protection scheme in the ART ASIC}
%\vspace{5pt}
\label{radtable_ART}
%\setlength{\tabcolsep}{3pt}
\begin{tabular}{lll}
\toprule
       &   Data     & State machine \\
 Block & protection & protection \\
\midrule
  Programmable Delays & No & -- \\
 \gls{DDR} Deserializers & Parity bit & Yes\\
 Programmable dead time   & -- & Yes\\
 Priority Selection & No & --\\
 BCID Counter & Yes & --\\
 Output Logic & No & Yes\\
  Register Matrix & Yes & Yes\\
  \ItwoC  Slave & Yes & Yes\\
\bottomrule
\end{tabular}
\end{table}
