\label{sec:asics}

The four NSW ASICs (VMM, ROC, TDS, ART) are fabricated in the Global Foundries 130\,nm 8RF-DM CMOS process on a common 8-inch wafer.
The reticle plan is shown in Figure\,\ref{fig:reticle}.


\begin{figure}[ht]
\centering
\includegraphics[width=0.45\textwidth]{figures/reticle}
\caption{The NSW wafer reticle, showing the locations and relative sizes of two copies of the VMM3a, one ROC1 and one ROC2 along with the TDSvII and the ART2 ASIC. There are other ASICs for the ATLAS Calorimeter in the same reticle. As seen, the reticle contains two versions of the ROC ASIC. This was done in case the ROC2 version had an issue and the fall-back solution for scheduling reasons was the ROC1.  There are 86 reticles in the 8-inch wafer. Those on the perimeter, however, could not be used.  Each wafer contains 113 VMM3a, 66 ART2, 62 ROC1, 65 ROC2 and 65 TDSvII chips. The number that can be extracted depends on how the wafer is cut. The reticle size is $20798.32 \times 20795\,$\textmu m.}
\label{fig:reticle}
\end{figure}

% Functional details of the various ASICs.

\subsection{VMM -- Mixed-signal front-end ASIC}
\label{sec:VMM}
The VMM\,\cite{9724214, Iakovidis_2020, vmmuserguide} is a custom Application-Specific Integrated Circuit (ASIC). It is designed to be the front-end ASIC of both the Micromegas and sTGC detectors of the New Small Wheels.  For the NSW, it is packaged in a 400-ball $21\times 21\, \textrm{mm}^2$ Ball Grid Array (\gls{BGA}) with 1\,mm ball pitch.

\subsubsection{Requirements}
The 64 channels with highly configurable parameters meet the processing needs of signals from all sources of both detector types:

\paragraph{The Micromegas}\hspace{-0.3cm}signals from the anode strips (negative polarity signals), depending on the chosen gas
gain and shaper integration time, can be up to a maximum 250\,fC, but typically half or even smaller charge is expected. The fast electron current is followed by the positive ion current which typically lasts for $\sim$150\,ns\,\cite{georgePhd}. In addition to the current signal duration and maximum input charge, the other relevant parameter is the electrode (anode strip) capacitance which varies from about 50 to 300\pF depending on the length of the strips. The noise is a critical parameter for the Micromegas determined by the requirement of single primary electron detection with a threshold  five times the RMS noise, a gas gain of  10,000, and the maximum possible electrode capacitance of 300\,pF. These conditions determine the
required noise level to be at 0.5\,fC or about 3,000 electrons RMS.

\paragraph{The sTGC}\hspace{-0.3cm}feature three different types of active elements on a detector: strips, wires, and pads. All three are read out via the VMM. Strips provide the precision radial coordinate measurement for track reconstruction, wires the azimuthal coordinate; pads are used for a ``pre-trigger'' that requires a configurable coincidence performed by the Pad Trigger (See Sections\,\ref{sec:trigpath} and \ref{sec:pad_trigger}) firmware among the signals of pads in consecutive layers. The wire signals have negative polarity, while both the strip and pad signals are positive. Hence the need for the VMM to handle both polarities. The total charge and the long ion tail  impose specific requirements on the processing of the sTGC signals.  The VMM should recover from wire and pad signals of 6\pC and 3\,pC, respectively, within 250\ns while maintaining linearity up to 2\,pC.
For pads, it should provide the Time-Over-Threshold (ToT) and recover within 1\us from high charges up to 50\,pC. The pads impose challenging requirements since their capacitance can be up to 3\,nF. For the strips, an average charge of 1\pC is expected while the input capacitance is $\sim$200\,pF. As mentioned above, the sTGC signals span a very large range from 1\pC on a given
strip to about 50\pC on a pad. The dynamic range for the precision strip measurement is 2\,pC. The need to measure 2.5\% of this charge with a 2\% resolution and a 200\pF electrode capacitance, requires a noise level for a 25\ns integration time to be about 1\,fC RMS.
The noise for the signals from the pads with much larger capacitance (up to 3\nF) is
substantially higher.

Both detectors have similar readout requirements as the architecture is the same. A maximum trigger latency of 10\,$\muup$s must be supported, so the VMM needs deep enough FIFOs to buffer the data for this length of time. Moreover, a hit rate of up to 1\MHz per channel is expected.
On the other hand, the two detectors have different trigger requirements. The Micromegas need to provide the address of the VMM channel that fired first in a bunch crossing.
For the sTGC pad pre-trigger, which requires binary pad hits to select relevant strips, the VMM provides a Time-over-Threshold signal.
For the sTGC trigger, the VMM provides a 6-bit charge measurement of the strips within $\sim$50\,ns.
The sTGC wires do not participate in the trigger formation.

The VMM will  operate in a harsh radiation environment, see Table\,\ref{tab:radEnv} and\,\cite{Ameel, amideiDCDC, ATL-MUON-PUB-2022-001, nswTDR}.  Design techniques are applied to mitigate issues that may affect the operation of the ASIC under the above-targeted conditions.  Although \gls{TID} (Total Ionisation Dose) may degrade the performance of the ASIC,  the VMM3a was tested for TID tolerance in the  $^{60}$Co source irradiation facility at BNL for the expected radiation and no performance degradation was noticed.  Single event upsets (\gls{SEU}), though, become increasingly more serious. To overcome SEU's   in the vulnerable logic blocks (see Table\,\ref{radtablevmm}), Dual Interlocked storage Cells (\gls{DICE})\,\cite{DICE,DICE2} and Triple Module Redundancy (\gls{TMR})\,\cite{TMR} protection techniques are used.
For the large storage elements such as the latency \gls{FIFO}, an upset is just flagged once detected and the FIFO is reset.

\begin{table}[h]
\caption{ Single Event Upset protection schema in the VMM}
\vspace{5pt}
\label{radtablevmm}
\centering
\setlength{\tabcolsep}{3pt}
\begin{tabular}{ p{7cm}p{7cm}  }
\toprule
 Block & Type of protection \\
\midrule
  Global configuration and channel registers & DICE\\
  VMM State Machine & TMR\\
 Bunch Crossing Counter   & TMR\\
 L0 FIFO Control  & TMR\\
 L0 Event Builder & TMR\\
 L0 Accept register, NSkip Circuit & TMR\\
  Latency FIFO & Parity on pointer, FIFO reset on parity error\\

\bottomrule
\end{tabular}
\end{table}


\subsubsection{Architecture}
The analog front-end section of each channel integrates a three-stage low-noise charge amplifier\,(\gls{CA}) followed by a third-order shaper.  The charge amplifier implements a programmable input polarity, a test capacitor connected to the integrated pulse generator, a power-down option, a fast recovery option for very high-charge events, and several programmable bias adjustments to accommodate a broad range of signals. The input \gls{MOSFET} is a \mbox{p-channel}. It is followed by a dual cascode stage and a mirrored rail-to-rail output stage. The shaper features programmable peaking time of 25,\,50,\,100, and 200\,ns.  The gain is adjustable in eight values (0.5,\,1,\,3,\,4.5,\,6,\,9,\,12,\,16\,mV/fC). A low-frequency non-linear feedback baseline holder\,(BLH) stabilizes the output baseline, referenced to an on-chip band-gap reference circuit set at 160\,mV. The BLH has a programmable bandwidth that allows the user to enable either a mild or a strong (effective bipolar shape) compensation introduced to handle the \gls{sTGC} long current due to the long drift of ions.

Following the analog front-end is the mixed-signal section that includes discrimination, peak and timing detection measurements and the corresponding analog-to-digital conversions. The threshold is adjusted by a 10-bit Digital to Analog Converter (DAC) common to all channels plus a local 5-bit trimming \gls{DAC} independently adjustable in each channel in 31 steps of approximately 1\,mV each. The peak detector
measures the peak amplitude and stores its output (PDO) in an analog memory.  Additionally, it provides the timing signal at the time of the peak of the analog pulse. The time detector measures the timing using a time-to-amplitude converter (\gls{TAC}) which arms at the rising edge of the bunch crossing clock (CKBC) and latches at its falling edge. The time detector output (TDO) value is stored in an analog memory. The ramp duration can be configured to 60, 100, 350 or 650\,ns.
For the NSW, the value of 60\ns is used; this is enough to cover the duration of the CKBC while the TAC is in its linear range. The  block diagram of one of the 64 identical channels is shown in Figure\,\ref{fig:GI_VMM_architecture},  delimited with a dashed box, along with the relevant parts shared by all the 64 channel circuits and signals.

The VMM neighbor option triggers the two neighbors of a triggered channel, irrespective of whether they cross threshold.
This allows raising the threshold while still digitizing the edges of a spatial charge distribution. The functionality is applicable across different VMMs through dedicated electrical lines.
%This lowers rate, which lowers dead time,
%but the two neighboring channels will also be dead when the channel above threshold triggers, thus increasing their dead time.

\begin{figure}[ht]
\centering
\includegraphics[width=0.99\textwidth]{figures/GI_VMM_architecture}
\caption{Overview of the VMM architecture}
\label{fig:GI_VMM_architecture}
\end{figure}

The mixed-signal part of the ASIC is followed by three current mode ADCs per channel. A pedestal of 150\mV is subtracted before digitization. This way the range of the ADC's is increased. The 10-bit and 8-bit ADC's are two-stage conversion digitizers providing charge and time measurements, respectively. The per-channel dead-time is driven by the 10-bit \gls{ADC} conversion that is configurable down to $\sim$250\ns, giving an effective rate of $\sim$4\,MHz per channel. The 8-bit ADC and a coarse 12-bit BC counter provide a 20-bit time stamp.  Each channel has a direct dedicated output (\gls{DDO}) where the 6-bit ADC provides the same charge measurement from the \gls{PDO} but in a much faster dedicated path with $\sim$50\ns dead-time.  The channel remains inactive until the 10-bit ADC completes the conversion. The ASIC provides the ability to interrupt the 10-bit conversion once the 6-bit conversion finishes, such that the DDO dead-time is small. In that case, the 10-bit information is unusable. The same DDO can be configured to provide pulses indicating Time-Over-Threshold\,(\gls{ToT}), Time-To-Peak\,(\gls{TtP}), Peak-To-Threshold\,(\gls{PtT}) or a Pulse-at-Peak\,(\gls{PtP}) of 10\ns duration. The address of the channel that registered the first hit per CKBC cycle, is output on a dedicated per-chip serial line. This is called the Address-in-Real-Time\,(\gls{ART}).


\subsubsection{Readout schema}
Although the VMM features more than one readout schema, the so-called ``L0'' mode is designed for operation within the ATLAS experiment. The output of the 10-bit and 8-bit ADCs enter into a 64-deep FIFO per channel called the ``Latency FIFO''.  Given the size of this FIFO and the 250\ns dead-time per channel, a maximum guaranteed latency of 16.0\,$\upmu$s where no data is lost can be achieved. This is larger than the minimum 10\,$\upmu$s required by ATLAS Trigger-DAQ{\,\cite{CERN-LHCC-2017-020}}.
Note that each channel is autonomous and this FIFO is filled asynchronously.

Each channel has a Level-0 Selector circuit that is connected to the output of the channel's latency FIFO.
The selector finds events within the BCID window (configurable at a maximum size of 8 BC's and the BC is offset by the latency) of a Level-0 Accept and copies them to the ``L0 channel'' FIFO.  If a channel's L0 selection circuit does not find a hit within the BC window,  a ``no data'' item is passed to the ``L0 channel'' FIFO. In this way the ``L0 channel'' of all 64 channels overflow synchronously. The ``L0 BCID'' FIFO is made deeper, 32-deep, than the ``L0 channel'' FIFO. This way, once the ``L0 channel'' overflows, the VMM data can still indicate on which BC this happened and can skip a configurable number of triggers to recover from the overflow while still maintaining its synchronous data-taking. The buffering scheme of the VMM L0 is shown in Figure\,\ref{fig:VMM3intBuf_V04}.

\begin{figure}[ht]
\centering
\includegraphics[width=0.65\textwidth]{figures/VMM3intBuf_V04}
\caption{Overview of the L0 buffer}
\label{fig:VMM3intBuf_V04}
\end{figure}

The data transfer from the VMM is done via two serial lines running at 160\,MHz with Double Data Rate (DDR), giving a total bandwidth of 640\,Mb/s. Two lines are used to reduce the clock rate. The Readout Controller supplies the clock for this transfer. The data is encoded in 8b/10b with one or more comma characters transmitted continuously between Level-0 events. The 8b/10b encoding reduces the effective bandwidth to 512\,Mb/s.

\subsubsection{Trigger outputs}
The VMM must provide different trigger primitives for the four different detector elements connected.
This is achieved through different data outputs and specific configuration. It is possible to turn off the \gls{SLVS} drivers of trigger output lines not in use in order to reduce power consumption.% and the possibility of digital interference with the VMM's front-end operation.

\paragraph{The \gls{Micromegas} trigger}\hspace{-0.3cm}utilises the fine strip pitch and the ionisation spread across the path of a particle crossing the detector at an angle as used in the $\muup$TPC method\,\cite{ALEXOPOULOS2019125}.
The VMM sends out this first address through a dedicated serial line called Address in Real Time (ART).
This signal is asynchronous to the bunch crossing clock so the VMM can be configured to align it with the BC clock.
The ART clock is provided to the VMM externally from the Readout Controller (ROC); see Section\,\ref{sec:ROC}.
The ART signals can be provided at threshold crossing or at peak found, depending on the configuration.
It can be optionally clocked at both edges of the 160\,MHz clock.


\paragraph{The sTGC trigger}\hspace{-0.3cm}is done in two steps as described in Section\,\ref{sec:trigpath}:
1) A pre-trigger: Overlapping induced cathode pads define a candidate track segment by a configurable coincidence in a projective tower in both quadruplets in a sector made by the Pad Trigger module.
Each VMM connected to pads operates its direct outputs in Time-over-Threshold (ToT) mode.
Each channel self-resets at the end of the timing pulse, thus providing continuous and independent operation of all 64 channels.
2) The projective tower defines the bands of strips (if any) to be read out from each TDS in each layer. The VMM channels connected to the strips digitize the charge through the fast 6-bit ADC and provide them through the direct output.  The channel reset occurs after the last bit has been shifted out. The data are clocked at both edges of the 160\,MHz clock.

%\FloatBarrier

\subsection{ROC -- Read out Controller ASIC}
\label{sec:ROC}
The ReadOut Controller (\gls{ROC})\,\cite{Coliban:2016uys, Popa:2019trf, Popa:2020sbm, Popa:2020poz,Popa2022} is a highly configurable data aggregation ASIC designed specifically for the NSW.  The block diagram of the ROC ASIC is shown in Figure\,\ref{fig:block_ROC}.
The ROC must provide readout of the hits buffered from the VMMs in response to a Level-1 trigger at 100\,kHz for Phase\,1 and at 1\,MHz for Phase\,2\,\cite{1MHzReadout}.
Initial specifications for the two-level hardware trigger required handling a latency of 60\,$\upmu$s with a consequent very deep buffer for hit data.
After ATLAS decided on a single-level hardware trigger the latency was reduced to 10\,$\upmu$s, but the ROC ASIC had already been produced.

\begin{figure}[b]
\centering
\includegraphics[width=0.99\textwidth]{figures/GI_ROC_Architecture}
\caption{Block diagram of the ROC ASIC\,\cite{Popa2022}}
\label{fig:block_ROC}
\end{figure}

The ROC receives 8b/10b encoded data from up to eight VMM3a ASICs. There is a dedicated ``Capture'' module for each VMM. The data are first de-serialized and then a dedicated mechanism determines the correct 8b/10b stream alignment and then decodes it. The data parity is also checked and if an error occurs, the relevant counters are incremented. If no data appear then a configurable timeout is asserted.
The decoded L0 packets are enqueued separately for each VMM.
A configurable crossbar allows routing of the data from one to eight VMMs to each of up to four SROC modules.
Each SROC is able to transmit 8b/10b encoded data through a configurable up to 320\,Mb/s E-link to the L1DDC.  Two 320\,Mb/s E-links can be combined to produce a 640\,Mb/s output.
For Phase 1, generally one 320 Mb/s E-link per ROC is sufficient.
For Phase 2, more than one E-link is needed, especially at the inner radius.
VMM occupancy varies strongly with radius.
Front-end boards at inner and outer radii require different numbers of E-links.
The crossbar allows routing VMM outputs to E-links in order to optimize the number of E-links needed.
For load balancing, an SROC can combine high and low occupancy VMMs on the same E-link.
The VMM data integrity is also checked at the SROC level.
If, for example, the ROC receives the so-called ``Magic'' BCID, which is a value outside the range of the expected LHC values, the ROC understands that the VMM FIFOs overflowed and data will not be received from this VMM for the configurable value (on the VMM) of skipped triggers.

The ASIC implements several output formats including dummy hits signalling overflow from the VMM, as well as hits that discard the TDC (TDO) value of the VMM to decrease the bandwidth. Moreover, Busy-On/Busy-Off symbols are injected in the output data stream, if enabled, to signal almost-full buffers in the ROC. The configuration and status of the ASIC is accessed by a dedicated I$^2$C interface through the SCA.

The ROC ASIC receives the TTC stream and BC clock from the GBTx. The alignment of the input stream is determined by detecting the positive edge of the BC clock signal in the readout clock domain. All its internal and externally supplied clock signals are generated by four ePLL blocks\,\cite{Poltorak_2012} driven by the BC clock. It supplies the clock and the TTC commands to all the ART, TDS and VMM ASICs. The three ePLLs supplying the signals and clocks to the other ASICs include phase-shifting circuits. This gives the ability to forward the relevant TTC commands and clocks with a configurable phase. The TTC FIFO (also called BC FIFO) buffers the Level-1 triggers.
In response to the Level-1 trigger, the ASIC checks, aggregates, re-formats and filters the L0 packets from the associated VMM Capture modules, building output packets that are pushed into the SROC FIFO ready to be read out.
The ePLLs are configured and monitored through a separate register bank through a dedicated I$^2$C interface of the SCA.



During the integration of the ROC ASIC and the SCA, it was found that the I$^2$C read-back implementation of the ROC was incompatible with the I$^2$C standard chosen by the SCA. The I$^2$C read-back was therefore emulated using dedicated GPIO lines and a software ``Bit Banger'' application\,\cite{opcuaserver}.

The ROC implements Triple Modular Redundancy\,(TMR)\,\cite{TMR} for all the configuration registers, state machines, control of the bunch crossing FIFO and readout logic to mitigate the effects of SEUs. An SEU counter is accessible by the SCA chip (through the configuration and register bank) to monitor the chip operation in the ATLAS environment.

%The protection areas are summarized in Table\,\ref{radtableroc}.

%\begin{table}[h]
%\caption{Single Event Upset protection schema in the ROC}
%\vspace{5pt}
%\label{radtableroc}
%\centering
%\setlength{\tabcolsep}{3pt}
%\begin{tabular}{lc}  %{ p{7cm}p{7cm}  }
%\toprule
% Block & Type of protection \\
%\midrule
%Registers & TMR\\
% State Machines & TMR\\
% Bunch crossing clock domain & TMR\\
% Readout clock domain & TMR\\
%Flip-Flops & TMR\\
%
%\bottomrule
%\end{tabular}
%\end{table}

%\FloatBarrier
%\FloatBarrier

\subsection{TDS -- sTGC Trigger Data Serializer ASIC}
\label{sec:TDS}

The sTGC trigger data serializer (\gls{TDS}) ASIC\,\cite{Wang:2017ols, Wang:2015msa, Wang:2019tay} prepares the trigger data from the VMM for either pads or strips and serializes it for transmission, in the case of pads, to the Pad Trigger, in the case of strips to the Router on the rim of the NSW detector. The data are transferred ultimately to the Trigger Processor.
It is mounted on the Front-end boards and can be configured in either strip or pad mode by connecting a pin to  high or low.

Both the 160\,MHz TDS logic clock and serializer reference clock are generated by an on-chip PLL from the BC clock supplied by the Read Out Controller.
The TDS is configured via the \ItwoC master of the on-board SCA ASIC.
An on-chip pseudo-random binary sequence generator (PRBS-31) is provided for serializer and link testing.

The TDS is fabricated in IBM\,130\,nm \gls{CMOS} technology and is packaged as a 400-pin Ball Grid Array (BGA).
It uses a 1.5\,V supply for both the logic part and the serializer and consumes about 0.9\,W.
A block diagram of the TDS ASIC, including both strip and pad parts, is shown in Figure\,\ref{fig:JW_TDS_blockdiagram}.

\begin{figure}[t]
\centering
\includegraphics[width=0.999\textwidth]{figures/JW_TDS_blockdiagram.pdf}
\caption{Block diagram of the TDS ASIC. Strip mode, top; Pad mode, bottom.}
\label{fig:JW_TDS_blockdiagram}
\end{figure}


\subsubsection{Pad TDS mode}  %%%%%%%%%%   P A D   T D S  %%%%%%%%%%%%%%%%%%%%%%%%%%%%%%

The pad TDS receives up to 104 pad Time-over-Threshold (ToT) differential signals from two VMM ASICs and transmits the data at 4.8\,Gb/s to the Pad Trigger every 25\,ns.
The rising edge of the ToT signal is used to capture a pad hit and assign its BCID.
Since the 104 pads cover an area of about 3\,m$^2$, the on-detector routing lengths from a pad to the VMM and pad-TDS can differ by up to 2.6\,m, with a propagation time of about 7\,ns/m\,\footnote{Calculated by the PCB layout program and verified approximately by measurements of some pads with a Time Domain Reflectometer.}.
In addition to the differences in time-of-flight from the IP for the pads within a detector,
this difference in propagation time could result in different BCID assignments for pad signals even though they belong to tracks from the same collision.
To compensate for this, each channel has a configurable delay, of up to eight 3.125\,ns steps\,\cite{Wang:2017iuz}.
Assuming that the earliest and latest pad signal arrivals (considering the earliest arrival signal from each pad)
are within 25\,ns of each other,
shifting the phase of the incoming BC clock enables them to be assigned the same BCID.

For serial transmission, 116 bits (104 pad bits plus a 12-bit BCID) are split into four consecutive frames: a 26-bit frame followed by three 30-bit frames.
The four frames are scrambled following a scheme used by the 10\,Gb/s Ethernet physical layer implementation\footnote{IEEE Standard 802.3-2012 with the polynomial function $1+x^{39}+x^{58}$}.
An unscrambled 4-bit header, 0b1010, is prefixed to the 26-bit scrambled frame, marking it as the first of the four 30-bit frames.
Scrambling enables 116 data bits to be transferred in one bunch crossing at 4.8\,Gb/s.
The latency of the pad-TDS from the end of a BC to the first bit of a pad frame exiting the serializer core is 31\,ns.

%\begin{figure}[h]
%\centering
%\includegraphics[width=0.93\textwidth]{figures/Pad_TDS_BlkDiagram_V02}
%\caption{The processing pipeline in the pad-TDS ASIC}
%\label{fig:pTDS_flow}
%\end{figure}


\subsubsection{Strip TDS mode}  %%%%%%%%%%  S T R I P   T D S  %%%%%%%%%%%%%%%%%%%%%%%%%%%
\label{sec:sTDS}

Each of the two VMM ASICs connected to one strip-TDS, sends 6-bit charge data through 64 independent serial lines, each representing the induced charge on a strip.
The strip-TDS holds the vector of charges, along with its BCID, in a circular buffer.
The Pad Trigger, after finding a coincidence in a tower of pads in eight sTGC layers, sends the ID of the band of strips in each layer that passes through that tower to the strip-TDS that holds the charges of that band.
A configurable look-up table in each TDS contains the channel number of the first strip in each band that it holds.
Strip-TDS ASICs that do not hold charges of any band are sent band-id\,0xff, which is not in its look-up table.
The band-id is the same for all layers, but the 17 strips comprising that band differ from layer-to-layer since the tower points to the interaction point.
When the request arrives from the Pad Trigger, if the charges for strips corresponding to the band-id are in the circular buffer and their BCID matches the requested BCID,
they are transmitted at 4.8\,Gb/s to the Trigger Processor via the Router.
Only the charges of the outer 14 or inner 14 strips are transmitted; a flag indicates which.
The data is sent in four 30-bit packets in one BC.
The packet format is shown in\,\cite{HU2022167504}.
This repeats for every bunch crossing.
The latency of the strip-TDS from arrival of the request from the Pad Trigger until the first bit of data frame exits the serializer core is 75\,ns.

\subsubsection{Test functions}
The TDS has several embedded functional self-tests in case of any system malfunctions, system commissioning and to enable testing without inputs.
The 4.8\,Gb/s serializer can be driven by an on-chip pseudo-random binary sequence generator\,\cite{PRBS}, PRBS-31, for testing and commissioning the link.
Specifically, the test modes are: \\
\textbf{Bypass Trigger mode:} In strip mode, probes inputs from an individual or a programmable number of strip channels bypassing the channel ring buffer and without trigger input from the Pad Trigger.
In pad mode, a similar diagnosis function provides access to an individual or a programmable group of input channels. \\
\textbf{TDS-Router Training Frame:} For strip mode, sends fake test frames to the Router without relying on the input from the Pad Trigger.\\
%
\textbf{Global-Test mode:} performs a full strip-mode function and output without the input from the Pad Trigger or the Front-end electronics.

%A detailed description can be found in\,\cite{Wang:2017ols}.

%\begin{figure}[h]
%\centering
%\includegraphics[width=0.97\textwidth]{figures/sTDS_flow.png}
%\caption{The processing pipeline in the strip-TDS ASIC}
%\label{fig:sTDS_flow}
%\end{figure}
%
%\FloatBarrier

\subsubsection{The 4.8\,Gb/s serializer}   %%%%%%%%%%  S E R I A L I Z E R  %%%%%%%%%%%%%%%%%%%%%%%%%%%

The 4.8\,Gb/s serializer for the TDS was developed and prototyped prior to the remainder of the TDS\,\cite{Wang:2015msa}.
Rather than developing such a challenging circuit from scratch, it was adapted from the CERN GBTx serializer, changed from loading 120 bits at 40\,MHz to loading 4\,$\times$\,30 bits at 160\,MHz.
CERN's ePLL design\,\cite{Poltorak_2012} was also used.
The metalization layers were also changed to match the variant of the IBM\,130\,nm CMOS process used by the other NSW ASICs so that all could be fabricated on the same wafer.
The serializer's output into a 100\,$\Omega$ load is about $\pm$\,500\,mV.

A jitter analysis of the transmission showed a total jitter of 49.7\,ps at a bit-error-ratio (\gls{BER}) of $10^{-12}$.
A BER test with an embedded PRBS checker inside a Xilinx\,7 FPGA was also performed.
Error-free running for three days was achieved, which corresponds to a BER less than $10^{-15}$.

\subsubsection{Radiation tolerance}
To mitigate Single Event Upsets, the TDS employs Triple Modular Redundancy\,(TMR)\,\cite{TMR} for the following:
Serializer, Serial protocol logic,
BCID, BC clock generator, BC clock phase shift, size of the matching window of the strip channel,
all configuration registers, Pad Trigger \gls{LUT}.
%\FloatBarrier

\subsection{ART -- \MM trigger data aggregator and serializer ASIC}
\label{sec:ART}
The ART ASIC is part of the trigger path of the \MM chambers.
It receives the prompt ``Address in Real-Time'' (ART) data from 32 VMM front-end ASICs which consist of the address of the first arriving hit in each 64-channel VMM for a given bunch crossing.  It then selects up to eight  addresses for each bunch-crossing and sends the data to the \MM Trigger Processor via one GBTx chip configured to operate in ``Wide'' mode, i.e.\ without Forward Error Correction (FEC).

\noindent In particular, the ART ASIC performs the following functions:\vspace{-6pt}\begin{itemize}\itemsep-4pt
\item Deserialize each ART stream and phase-align the hits to the BC clock.
\item Selects the strip addresses of up to a fixed number of hits by means of cascaded priority encoders.
\item Append the 5-bit geographical physical VMM address to the strip address of each hit (defined by the cable connections).
\item Send the ART addresses and the 12-bit BCID to the \MM Trigger Processor via a GBTx.

\end{itemize}


The block diagram of the ART ASIC is shown in Figure\,\ref{fig:block_ART}.  The ASIC receives its TTC stream, configuration and clock signals from the GBTx (which receives them through the Level-1 Data Driver Card (\gls{L1DDC}) downlink, see Section\,\ref{sec:addc}). The ASIC has the option to output the Bunch Crossing Reset (\gls{BCR}) \gls{TTC} signal received to another ART ASIC\footnote{The second ART on the ADDC.}.
The ASIC is configured through the SCA using an I$^2$C bus.  The ASIC is packaged in a 128-pin LQFP package.
The latency of the ART ASIC was measured in several test setups to be $\sim$44\,ns.
Further information can be found in\,\cite{ARTASIC}.

\begin{figure}[ht]
\centering
\includegraphics[width=0.95\textwidth]{figures/GI_art_architecture_nolabel}
\caption{Block diagram of the ART ASIC}
\label{fig:block_ART}
\end{figure}



\paragraph{Programmable delays:}\hspace{-8pt}The purpose of the Programmable Delay block is to be able to skew the input signals to the local clock phase and adjust the ART stream to it.  To perform that, the ASIC uses four copies of the 8-channel Phase Aligner core developed at CERN\,\cite{Tavernier_2012}.% which can automatically align the ART signals with the 160\MHz clock provided.

\paragraph{Deserializer:}\hspace{-8pt}The first signal that precedes the ART 6-bit address is a flag.
This is generated by the VMM and in DDR mode, the flag pulse raises asynchronously to the ART clock (CKART) and is kept high through the next two falling edges of the 160\MHz clock, being lowered by the second falling edge.
Optionally, the rising edge of the flag can be registered to the CKART clock. A 10\ns reset period is applied after each ART sequence.  The eight flip-flops form an 8-bit DDR shift register which deserializes the incoming data stream.

\paragraph{Programmable dead time:}\hspace{-8pt}This block creates an artificial dead time for each VMM input which is controllable via configuration. Subsequent data on a particular input is ignored for between 0 and 7 BC's. This prevents responding to any subsequent ART signals from the same particle crossing the \MM.
See Figure\,\ref{fig:ART_concept}.

\paragraph{Priority selection:}\hspace{-8pt}
The Hit Selection circuit is based on eight layers of cascaded priority encoders.
The first priority encoder selects the first ART flag (most significant bit which is not zero from the 32-bit ART flag word).
The 32-bit word is readout as the first ART and then removed from the array. The consequent word (without the first ART) is presented to the following stage. The second priority encoder selects the second non-zero bit from the ART flag word in the same manner as the first stage.
The operation is cascaded eight times to select a maximum of eight non-zero flags.
The operation is illustrated in Figure\,\ref{fig:art_priorityenc}. If there are more than eight ART signals in the same BC window,  then the ART ASIC selects only eight of them, based on the priority scheme explained. The ART can be configured to give priority to different radii of the detector\,\cite{nswTDR}.


\begin{figure}[ht]
\centering
\includegraphics[width=0.9\textwidth]{figures/GI_art_priority_encoders}
\caption{Hit map generation circuitry based on priority encoders. The duration of the procedure is $\sim$3\,ns.}%timing was provided by Sorin
\label{fig:art_priorityenc}
\end{figure}


\paragraph{Output:}\hspace{-8pt}For each bunch crossing, the ART ASIC transmits to the Trigger Processor through the GBTx, the following data:
\vspace{-6pt}\begin{itemize}\itemsep-4pt
\item{Up to eight  VMM ART signals which had a hit in that bunch crossing}
\item{The 12-bit BCID in which the ART signals occurred}
\item{Other information (error flags, parity bits)}
\end{itemize}

\noindent The Wide Bus mode of the GBTx chip (see Section\,\ref{sec:GBTx}) allows for 112-bits to be transmitted in one bunch crossing on two 80\,MHz clock edges. The ART data are transmitted to the GBTx chip in two batches:\vspace{-6pt}\begin{itemize}\itemsep-4pt
\item{The first 56-bits contain the selected VMM hit list based on the flag bits issued. This can be configured in two different modes: either transmit a ``Hit Map'' 32-bit word where each bit corresponds to one of the 32 VMMs connected to the ART ASIC or, the ``Hit Address'' option containing a 5-bit VMMID for each of the VMMs selected. In both cases, the 12-bit BCID is transmitted along with the Hit Information.}
\item{The second batch of 56-bits contains the eight 6-bit VMM ART addresses of the VMMs selected along with an 8-bit parity bit for each word.}
\end{itemize}



\paragraph{Debugging mode:}\hspace{-8pt}Besides the normal operation modes, the ASIC implements a debugging mode where different parts of the system are bypassed or fixed, or, repeating calibration patterns are transmitted. The following debugging modes are implemented:
\vspace{-6pt}
\begin{itemize}\itemsep-4pt
\item{Full bypass mode where the output of the programmable delays are directly made available to the outputs of the ASIC. This is used to verify and measure the propagation delay during ASIC initial verification and it is not accessible during normal operation in ATLAS.}
\item{Priority encoders bypass mode where the input channels are connected directly to the output logic. This mode is controlled by a configuration register and allows the verification of the transmission between VMM chips and ART ASIC bypassing any selection, while permitting the definition of the propagation delays set in the Programmable Delays. }
\item{Fixed output calibration pattern where a fixed pattern is sent continuously. The pattern is stored in local configuration registers and is accessible via the configuration path which allows the verification of the data transmission.}
\end{itemize}


\paragraph{Radiation protection:}\hspace{-8pt}The ART ASIC is designed with several protection mechanisms in order to ensure protection against SEU events or to flag uncertain conditions.
The SEUs in the data will not seriously affect the trigger. However, simple parity bits are calculated by the input deserializer circuits and can be used downstream to tag possible data corruption.  The circuits protected with triple modular redundancy (TMR)\,\cite{TMR} or parity bits are shown in Table\,\ref{radtable_ART}.


\begin{table}[h]
\centering
\caption{TMR protection scheme in the ART ASIC}
%\vspace{5pt}
\label{radtable_ART}
%\setlength{\tabcolsep}{3pt}
\begin{tabular}{lll}
\toprule
       &   Data     & State machine \\
 Block & protection & protection \\
\midrule
  Programmable Delays & No & -- \\
 \gls{DDR} Deserializers & Parity bit & Yes\\
 Programmable dead time   & -- & Yes\\
 Priority Selection & No & --\\
 BCID Counter & Yes & --\\
 Output Logic & No & Yes\\
  Register Matrix & Yes & Yes\\
  \ItwoC  Slave & Yes & Yes\\
\bottomrule
\end{tabular}
\end{table}

%\FloatBarrier





