\section{FELIX}
\label{sec:felix}

The Front End Link eXchange (FELIX), developed by the ATLAS Trigger and DAQ project\,\cite{Panduro-Vasquez:2022oR, Levinson:2799865, Paramonov:2021jpz, PanduroVazquez:2020mnk, felixHW, Trovato:2019pui, FelixUserGuide}, %% (\red{these refs already done before})
interfaces 4.8\,Gb/s optical links from GBTx ASICs that aggregate several slow serial copper ``E-links'', to an industry standard Ethernet network.
These slow links carry Front-end readout, configuration, calibration and detector monitoring data.
Acting similarly to a network switch, FELIX routes these slow links individually between Front-end electronics and software processes on the network, such as those shown in Figure\,\ref{fig:LL_NSW_ElxOvr}.
Furthermore, it distributes the TTC (Timing, Trigger and Control) signals, including the LHC Bunch Crossing clock, to all the NSW electronics.
FELIX is built from custom PCIe FPGA cards\,\cite{felixHW} hosted in commercial Linux servers, each equipped with a high-performance Ethernet interface card; see Figure\,\ref{fig:FELIXblkdiag}.
FELIX provides a common platform for some ATLAS Phase\,1 subsystems and will do so for all ATLAS subsystems in Phase\,2.
The use of commodity components and the sharing of a common platform reduces hardware, firmware and software effort.

The New Small Wheel uses the version of the FELIX FPGA card with 24 4.8\,Gb/s links in each direction in so-called GBT mode.
Each From-detector optical link carries data from between 9 and 21 slow, copper twin-ax serial links, ``E-links''; To-detector links carry between 7~and 18 serial E-links.
The slow serial E-links are aggregated to the 4.8\,Gb/s fibre by the GBTx ASICs on L1DDC boards or directly by the Trigger Processor Carrier FPGA.

%To accommodate Phase\,2, the Front-end boards provide up to four 320\,Mb/s E-links for L1A readout data; only one is needed for Phase\,1.
%The L1DDC's however, must support the Phase\,2 E-links aas well.
%The number of readout E-links for \MM L1DDC's requires two extra to-FELIX fibres beyond the normal pair of to- and from-FELIX fibres.
%The sTGC strip-L1DDC requires one extra to-FELIX fibre. (But both directions were implemented.)
%The sTGC pad-L1DDC does not require an additional to-FELIX fibre.

\begin{figure}[th]
\centering
\includegraphics[width=0.85\textwidth]{figures/FELIXblkdiag_v02.pdf}
\caption{Block diagram of the FELIX server. A server may contain up to two FPGA cards.
         The  Network Interface Card (NIC) for NSW has two 25\,Gb/s ports and supports RDMA\,\cite{RDMA}.} % Mellanox ConnectX-5
\label{fig:FELIXblkdiag}
\end{figure}

FELIX receives TTC information from an ALTI\,\cite{ALTItwiki} Timing, Trigger \& Control (TTC) module via a fibre connection.
It also asserts BUSY on a dedicated electrical line should its FELIX FPGA or server buffers be near to overflowing.

For Phase\,1, there are 60 FELIX FPGA boards in 30 servers.
Phase\,2 requires 40~additional boards (based on 24~input links per FPGA).
Although the Phase\,2 fibres reach the radiation-protected underground room (USA15), the Phase\,2 FELIX boards and servers will not be installed until Phase\,2.

%\noindent Table\,\ref{tab:FLXcount} shows the numbers of FELIX FPGA cards in the NSW system.
%Although the Phase\,2 fibres reach USA15, the Phase\,2 FELIX will not be installed until Phase\,2.
%
%% Table generated by Excel2LaTeX from sheet 'Sheet1'
%\begin{table}[htbp]
%  \centering
%  \caption{Numbers of FELIX FPGA boards (two per FELIX server) The number for Phase\,2 assumes 24 input links per FPGA.}
%    \begin{tabular}{rccc}
%    \toprule
%          & \textbf{Phase 1} & \textbf{Phase 2} & \textbf{Total} \\
%    \midrule
%    \multicolumn{1}{l}{\textbf{sTGC FE}} & 32    & 16    & 48 \\
%    \multicolumn{1}{l}{\textbf{MM FE}} & 24    & 24    & 48 \\
%    \multicolumn{1}{l}{\textbf{sTGC Trigger}} & 2     & --    & 2 \\
%    \multicolumn{1}{l}{\textbf{MM Trigger}} & 2     & --    & 2 \\
%    \midrule
%    \textbf{Total} & \textbf{60} & \textbf{40} & \textbf{100} \\
%    \bottomrule
%    \end{tabular}%
%  \label{tab:FLXcount}%
%\end{table}%


\subsection{Geographic names}
\label{sec:geonames}

There are over 22,000 E-links in the NSW.
For the software to have the ability to address them according to their data type and the exact region of the NSW to which they are connected,
each one is given a geographical or logical name, a so-called Detector Resource Name\,\cite{geonames}.
Referencing by geographic name is much clearer, less error prone and easier to maintain than by its physical FELIX connection.
Also switching to spare fibres or spare FELIX servers is transparent to the software.
The names are independent of specific connections of the multi-fibre bundles from the detector to FELIX boards and servers.
The mapping is done via ``FELIX-ID''s\,\cite{FELIXID}, with NSW experts maintaining the translation from fibres and E-links to FELIX-IDs and
FELIX experts maintaining the translation from FELIX-IDs to  FELIX servers.
%A look-up-table, defined and maintained by NSW experts, maps each name to a ``FELIX Resource Identifier'' (FELIX-ID or FID)\,\cite{FELIXID}, which is a specific E-link in a specific fibre in a multi-fibre bundle arriving in USA15 from the detector.
%A second look-up-table, defined and maintained by FELIX experts, maps FELIX-ID's to a specific FELIX server, board and connector.
%Geographical names are a string of location and attribute values, separated by ``/''s. For example:
An example geographical name is:
``{\small\textsf{MM-A/V0/L1A/strip/S4/L3/R11/E3}}''.
It refers to L1A data from Micromegas, Version\,0, Endcap\,A, strips, Sector\,4, Layer\,3, Radial position\,11, Readout controller E-link\,3.
Corresponding to the string is a 32-bit binary representation that can be used within data records to identify their data.
