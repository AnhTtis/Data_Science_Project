\section{Power distribution and grounding}

\subsection{Power}
\label{sec:powerDistribution}
All the NSW on-detector electronics utilise the FEAST DC-DC converter\,\cite{FEAST2.1} for the on-board Point-of-Load DC regulators. The input power to the FEASTs is provided by power supplies developed by CAEN\,\cite{caen}.
The Low Voltage power distribution\,\cite{nswlv} adopts a two step voltage conversion. The New Generation Power Supply (NGPS)\,\cite{ngps} located in the US15 service area of ATLAS provide 280\,V to the on-wheel  Intermediate Conversion Stage (\gls{ICS}) modules which are based on the EASY BRIC system\,\cite{ics}.
The ICS modules convert the 280\,VDC to 11\,V (configurable) and supply the voltage to the on-detector electronics through a Low-Voltage-Distributor-Board (LVDB). One LVDB can supply up to eight Front-end boards (analog section) and up to four digital boards (e.g.\ L1DDC, digital section).
%sLow voltage power modules may supply multiple sectors.

\subsection{Grounding}
\label{sec:grounding}
The grounding for the New Small Wheel tries to follow the general ATLAS guidelines as much as possible\,\cite{Blanchot:1073170} in a ``Star'' topology. Each detector technology sector of the NSW is considered an isolated system attached to its mechanical frame via isolating attachments. The low inductance,  low resistance connection to ground must not carry current. The ``Analog'' ground of the Front-end boards should connect directly to the detector ground with the shortest path possible.  The Low Voltage lines are connected to a floating power supply.  All ground, drain, and shield connections in LVDS cables should be AC coupled to the ground plane of the Front-end Board only. The other end will be DC coupled on the digital electronics. The grounding scheme with principles and rules that were implemented can be found here\,\cite{Levinson:2845656}. The integration of the electronics in the wheel is not described in this manuscript.