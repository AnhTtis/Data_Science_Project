%direct clock

\label{sec:DirectClock}

Initially the E-Link and programmable clocks of the GBTx ASIC were used as 160\,MHz reference clocks for the 4.8\,Gb/s serial receivers in the 288 sTGC Pad Triggers and Routers.
They finally proved to be marginal during the data integrity tests due to their high jitter of about $4.3\,\mathrm{ps}$ and $10\,\mathrm{ps}$ respectively as shown in the phase noise plot in Figure\,\ref{fig:phase noise}.
\begin{figure}[b]
\centering
\includegraphics[width=0.9\textwidth]{figures/phase_noise.PNG}
\caption{Phase noise (random--rms) jitter of the direct and GBTx clocks compared to Xilinx's phase noise mask for the CPLL and QPLL.
         The integrated bandwidth for GBTx E-link and programmable clocks is 1\,kHz to 20\,MHz. The integrated bandwidth for the clock distributor is 1 kHz to 30 MHz\,\cite{directClock}.}
\label{fig:phase noise}
\end{figure}
The jitter cleaner with the required low jitter was tested and found to suffer from fatal single event upsets in the foreseen radiation environment\,\cite{ATL-MUON-PUB-2022-001}.
Consequently, an additional clock distribution scheme\,\cite{directClock} was designed to deliver a low jitter clock directly
from the radiation-protected room outside the ATLAS collision cavern to each Rim L1DDC via $63\,\mathrm{m}$ OM3 fibres.
A clock distributor board per endcap receives the timing, trigger and control (TTC) stream from the ALTI module\,\cite{ALTItwiki} and
recovers a 160\,MHz clock based on the LHC bunch crossing clock, using the Clock and Data Recovery (\gls{CDR}) ADN2814 chip from Analog Devices.
The recovered clock is fanned out to four Si5345 jitter cleaners from Silicon Labs.
Each of their eight outputs is further fanned out by 1:4 fan-outs, Si53306, from Silicon Labs with an ultra-low additive jitter of $50\,\mathrm{fs}$.
The cleaned clocks then drive three 12-channel Avago/Foxconn miniPOD optical transmitters\,({\small AFBR-812FH1Z})
which drive 32 fibres -- one to each of the two redundant Rim L1DDC's per sector.
Each section of the Rim-L1DDC can then be configured by its SCA to forward either the direct clock or an E-link clock to the trigger boards in its Rim crate.
The received clock is chosen and fanned out by means of ultra-low jitter (less than $300\,\mathrm{fs}$ additive jitter) fan-outs.

The clock distributor board is a \gls{VME}\,6U board, designed with several measures\,\cite{directClock} taken to minimize jitter.
The jitter cleaners are configured via \ItwoC using an on-board commercial Ethernet to \ItwoC adapter card.
The VME backplane provides only power to the board.
