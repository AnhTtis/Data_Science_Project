\label{sec:sTGCtrigger}
\para{The sTGC trigger}
Ideally, one would read out all the strips in a sector directly to that sector's Trigger Processor.
However, reading out a 6-bit charge for each of 282,000 strips on every bunch crossing would require the huge bandwidth of almost 70\,Tb/s.
The power, cooling, and cost of current electronics are prohibitive.
In order to reduce the number of strips to be transferred to the Trigger Processor, the NSW uses eight-layer towers of sTGC pads pointing to the interaction point to provide a pre-trigger.
See Figure\,\ref{fig:LL_PadSelect}.
\begin{figure}[ht]
\centering
\includegraphics[width=0.9\textwidth]{figures/LL_PadSelect_V02}
\caption{A band of strips in each layer is selected by a particle making a 3-out-of-4 hit coincidence in a pointing tower of sTGC pads in each quadruplet.
The pads in half of the layers are shifted by half a pad in both directions to increase the resolution.
Eight-layer towers pointing to the interaction point are defined by the overlapping physical pads (shown in grey) which identify a logical pad (in red) in each layer.}
\label{fig:LL_PadSelect}
\end{figure}

The pre-trigger per sector is formed by the Pad Trigger board (Section\,\ref{sec:pad_trigger}) on the rim of the NSW.
Coincidences between layers of the towers identify up to four bands of strips in each of the eight layers.
The Pad Trigger signals the TDS ASICs that contain those bands to transmit the strip charges to the Trigger Processor via the Router (Section\,\ref{sec:router}).
Note the zigzag path in Figure\,\ref{fig:LL_NSW_ElxOvr} and its consequent significant increase in trigger latency.
The Trigger Processor receives the strip charges and calculates a centroid for each layer.
These are used to calculate $r$ and $\Delta\theta$ of a track segment and to apply the $\Delta\theta$ cut mentioned above.