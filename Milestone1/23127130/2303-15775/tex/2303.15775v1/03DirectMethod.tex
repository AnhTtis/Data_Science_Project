\section{Direct Method of the Calculus of Variations}\label{diectmethod}
In this section, we employ the direct method of the calculus of variations to obtain an axially symmetric minimizer of the Willmore energy with prescribed isoperimetric ratio $\sigma$. The analysis follows with only minor modifications, the arguments in \cite{choskiveneroni}.

\subsection{Functional Space}\label{functionalspacesection}
We first define a suitable space of profile curves.
\begin{definition}[Admissible Profile Curves]\label{classPdef}
A curve $\gamma=(\gamma_1,\gamma_2):[0,1]\rightarrow \mathcal H^2$ is called an \emph{admissible profile curve} if:
\begin{enumerate}[(1)]
    \item $\gamma\in C^1([0,1])\cap W^ {2,2}_{\operatorname{loc}}((0,1))$.
    \item $\gamma_1(0)=\gamma_1(1)=0$ and $\gamma_1(t)>0$ for all $t\in(0,1)$.
    \item $\gamma$ is parameterized proportional to arc length.
    \item $\Sigma_\gamma$ has curvature bounded in $L^2(\mu_{\Sigma_\gamma})$. That is 
    $$\int_{\Sp^2}k_1^2+k_2^2 d\mu_{\Sigma_\gamma}=\int_0^1 (k_1^2+k_2^2)2\pi|\dot\gamma|\gamma_1 dt<\infty.$$
\end{enumerate}
We denote the set of all admissible profile curves by $\mathcal P$.
\end{definition}

We wish to deduce the existence of a curve $\gamma\in\mathcal P$ that minimizes the Willmore energy under all curves in $\mathcal P$ with prescribed isoperimetric ratio. Given $\sigma\in(0,1)$ we put
$$
\mathcal F_\sigma:=
\left\{
\gamma\in\mathcal P\ \bigg|\ A[\gamma]=1\hspace{.2cm}\textrm{and}\hspace{.2cm}V[\gamma]=\frac{\sigma}{6\sqrt\pi}
\right\}
\hspace{.5cm}\textrm{and}\hspace{.5cm}
\beta(\sigma):=\inf\set{\mathcal W[\gamma]\ |\  \gamma\in\mathcal F_\sigma}.$$
Note that in the definition of $\mathcal F_\sigma$, we prescribe the area and the volume of $\gamma$. However, as both the Willmore energy $\mathcal W$ and the isoperimetric ratio $\mathcal I$ are invariant under scaling, we have 
\begin{equation}\label{betaDefinition}
\beta(\sigma)
=\inf\set{\mathcal W[\gamma]\ |\  \gamma\in\mathcal F_\sigma}
=\inf\set{\mathcal W[\gamma]\ |\  \gamma\in\mathcal P\textrm{ with }\mathcal I[\gamma]=\sigma}.
\end{equation}


Unfortunately, the space $\mathcal P$ does not have sufficient compactness properties and we require a weakened notation of admissible profile curves. 
\begin{definition}[Weak Profile Curves]\label{PWeakDefinition}
We say that $\gamma:[0,1]\rightarrow\mathcal H^2$ is a \emph{weak profile curve} if:
\begin{enumerate}[(1)]
    \item $\gamma\in W^{1,2}((0,1))$ (so in particular $\gamma\in C^0([0,1])$). Additionally  $\gamma_1(0)=\gamma_1(1)=0$ and $\gamma_1(t)>0$ for almost all $t\in[0,1]$.
    \item $|\dot\gamma|=L[\gamma]$ almost everywhere.
    \item $\gamma\in W^{2,2}_{\operatorname{loc}}(U)$ for all $U\subset[0,1]$ satisfying $\gamma_1>0$ on $U$.
    \end{enumerate}
    By (1) and (3), $\ddot\gamma$, and thus also $k_1$, is defined almost everywhere even though $\dot\gamma$ is not weakly differentiable on $(0,1)$. This allows the formulation of:
    \begin{enumerate}[(1)]\setcounter{enumi}{3}
    \item $\Sigma_\gamma$ has curvature bounded in $L^2(\mu_{\Sigma_\gamma})$. That is 
    $$\int_0^1 (k_1^2+k_2^2)2\pi|\dot\gamma|\gamma_1 dt<\infty.$$
\end{enumerate}
The set of weak profile curves is denoted by $\mathcal P^w$. 
\end{definition}


Clearly $\mathcal P\subsetneq\mathcal P^w$. However, the next theorem establishes that weak profile curves correspond exactly to chaining several admissible profile curves from $\mathcal P$ together. For a proof, we refer to Lemma 6 in \cite{choskiveneroni}.
\begin{theorem}[Regularity of Weak Profile Curves]\label{weakandstrongPconnection}
Let $\gamma\in \mathcal P^w$, put $M:=\int_0^1 (k_1^2+k_2^2)2\pi|\dot\gamma|\gamma_1 dt$ and let $n_0:=\lfloor\frac{M}{8\pi}\rfloor$. Then there exists $m\leq n_0$ and points $0=\tau_0<\tau_1...<\tau_m=1$ such that (up to linear reparamterization) $\gamma|_{[\tau_i,\tau_{i+1}]}$ belongs to $\mathcal P$. Additionally, when $t\rightarrow\tau_i^ \pm$ we have
$$\lim_{t\rightarrow\tau_i^ \pm}\dot\gamma_2(t)=0
\hspace{.5cm}\textrm{and}\hspace{.5cm}
\lim_{t\rightarrow\tau_i^ \pm}\dot\gamma_1(t)=\pm L[\gamma].$$
\end{theorem}

As an immediate consequence, we obtain the following corollary:
\begin{korollar}
Let $\gamma\in\mathcal P$. Then 
\begin{equation}\label{velocityatends}
    \dot\gamma_1(0)=L[\gamma],\ \dot\gamma_1(1)=-L[\gamma]\hspace{.5cm}\textrm{and}\hspace{.5cm}\dot\gamma_2(0)=\dot\gamma_2(1)=0.
\end{equation}
Consequently $\Sigma_\gamma$ is a $C^1$ surface.
\end{korollar}

\paragraph{Properties of Profile Curves}\ \\
We collect several properties of (weak) profile curves. 
The following theorem is Lemma 2 from \cite{choskiveneroni}.
\begin{theorem}[Choski and Veneroni]\label{choskiveneroniresult}
Let $\gamma\in\mathcal P$. Then 
\begin{equation}\label{lengthbound}
\frac{A[\gamma]}{2\pi\operatorname{diam}(\Sigma_\gamma)}\leq L[\gamma]\leq \frac{\sqrt{A[\gamma]}}{2\pi}\left(
\left(\int_{\Sp^2} k_1^2 d\mu_{\Sigma_\gamma}\right)^{\frac12}+
\left(\int_{\Sp^2}  k_2^2 d\mu_{\Sigma_\gamma}\right)^{\frac12}
\right)
.
\end{equation}
\end{theorem}
As an immediate consequence of this theorem, we can also bound the inverse arc length. Indeed, using $\gamma_1(0)=0$, we can bound the diameter of $\Sigma_\gamma$ by estimating
$$|f_\gamma(t,\theta)-f_\gamma(t',\theta')|
\leq |\gamma(t)-\gamma(t')|+|\gamma_1(t')|\ \left|\begin{pmatrix}
\cos\theta-\cos\theta'\\
\sin\theta-\sin\theta'
\end{pmatrix}
\right|
\leq 3L[\gamma].$$
Consequently $\operatorname{diam}(\Sigma_\gamma)\leq 3L[\gamma]$ and thus
\begin{equation}\label{inverseLbound}
\frac1{L[\gamma]^2}\leq\frac{2\pi\operatorname{diam}(\Sigma_\gamma)}{A[\gamma]L[\gamma]}\leq \frac{6\pi}{A[\gamma]}.
\end{equation}

Additionally, many facts about smooth surfaces remain true for surfaces $\Sigma_\gamma$ with $\gamma\in\mathcal P$. We collect these below and refer to Appendix \ref{proofpropertiesofP} for their proofs. 

\begin{lemma}[Li-Yau Inequality]\label{WillmoreLowerBound}\ \\
Let $\gamma\in\mathcal P$. Then $\mathcal W[\Sigma_\gamma]\geq 4\pi$ and equality is only true if $\Sigma_\gamma$ is a sphere and $f_\gamma:\Sp^2\rightarrow \Sigma_\gamma$ is a diffeomorphism. 
\end{lemma}

A direct consequence from Lemma \ref{WillmoreLowerBound} is $\beta(\sigma)\geq 4\pi$ for all $\sigma\in(0,1]$. Additionally, since $\mathcal W[\Sp^2]=4\pi$ we deduce $\beta(1)=4\pi$ and the minimizers are precisely rounds spheres.

\begin{lemma}[Gauß-Bonnet Theorem]\label{gausbonnetlemma}\ \\
Let $\gamma\in\mathcal P$. Then $\Sigma_\gamma$ satisfies the Gauß-Bonnet theorem. That is
\begin{equation}\label{gaussbonnettheorem}
\int_{\Sp^2}k_1k_2d\mu_{\Sigma_\gamma}=4\pi.
\end{equation}
\end{lemma}

\begin{lemma}[Hopf Theorem]\label{hopftypetheorem}\ \\
Let $\gamma\in \mathcal P$ and assume that $\Sigma_\gamma$ has constant mean curvature $H_0$. Then $\Sigma_\gamma$ is a sphere.
\end{lemma}


\begin{lemma}\label{gamma1ddotgammainL1}
    Let $\gamma\in\mathcal P^w$, $L:=L[\gamma]$ and $U\subset[0,1]$ be measurable such that $\inf_U \gamma_1>0$. Then $\gamma_1|\ddot\gamma|^2\in L^1((0,1))$, $|\ddot\gamma|^2\in L^1(U)$ and
    $$\int_0^1 \gamma_1(t)|\ddot\gamma(t)|^2dt\leq \frac{L^3}{2\pi}\int_0^1 k_1^22\pi L\gamma_1(t) dt
    \hspace{.5cm}\textrm{and}\hspace{.5cm}
    \int_U|\ddot\gamma(t)|^2dt\leq \frac{2L^3}{\pi\inf_U\gamma_1}\mathcal W[\gamma].$$
\end{lemma}
\begin{proof}
    In view of Theorem \ref{weakandstrongPconnection}, we can assume that $\gamma\in\mathcal P$ to estblish the first esimtate. By Definition \ref{classPdef} we have $\gamma\in W^{2,2}(\epsilon,1-\epsilon)$ for all small $\epsilon>0$. We use the definition of $\mathcal P$ and Equation \eqref{ArcLength_K_Identity} to deduce that for some $C$ independent of $\epsilon$
    $$C\geq \int_\epsilon^{1-\epsilon}k_1^2 2\pi|\dot\gamma|\gamma_1 dt=\frac{2\pi}{L^3}\int_\epsilon^{1-\epsilon}|\ddot\gamma|^2\gamma_1 dt.$$
    The first part of the lemma follows by letting $\epsilon\rightarrow 0^+$ and using the theorem of Beppo-Levi. Let $\gamma\in\mathcal P^w$. To establish the second estimate, it suffices to bound $\int_0^1 k_1^2 2\pi L\gamma_1dt$. By Theorem \ref{weakandstrongPconnection}, there exists $m\in\N$ such that $\gamma$ is built from chaining together $m$ many $\mathcal P$-curves. We use the Gauß-Bonnet theorem (Lemma \ref{gaussbonnettheorem}) to obtain $\int_0^1 k_1k_22\pi|\dot\gamma|\gamma_1 dt=4n_0\pi\geq 0$ and estiamte 
    $$\int_0^1 k_1^2 2\pi |\dot\gamma|\gamma_1 dt\leq 4\mathcal W[\gamma]-2\int_0^1 k_1k_22\pi |\dot\gamma|\gamma_1 dt\leq 4\mathcal W[\gamma]. $$
\end{proof}

\subsection{Topology}
We now introduce a notion of convergence in $\mathcal P^w$. The following is essentially the same definition as Definition 6 in \cite{choskiveneroni}. Their definition, however, allows for systems of curves, which we exclude here. Additionally, we have eliminated the concept of measure-function couples. 
\begin{definition}\label{convergencedefinition}
We say that a sequence $(\gamma^ {(n)})\subset \mathcal P^w$ \emph{converges to $\gamma^*\in \mathcal P^w$ in $\mathcal P^w$} if:
\begin{enumerate}[(1)]
    \item For $i=1,2$ we have $\gamma^ {(n)}_i\rightarrow\gamma^*_i$ in $C^ 0([0,1])$.
    \item For $i=1,2$ we have $\dot\gamma^ {(n)}_i\rightarrow\dot\gamma^*_i$ in $L^ 2((0,1))$.
    \item For $i=1,2$ we have $\ddot\gamma^ {(n)}_i\gamma_1^{(n)}\rightarrow\ddot\gamma^*_i\gamma_1^*$ weakly in $L^ 1((0,1))$.
\end{enumerate}
\end{definition}
We note that the third condition is well-defined by Lemma \ref{gamma1ddotgammainL1}. 

\paragraph{Lower Semicontinuity}\ \\
We show that the notation of convergence in Definition \ref{convergencedefinition} is strong enough to imply lower semicontinuity of our problem. Let $\gamma^ {(n)}\rightarrow \gamma^*$ in $\mathcal P^w$. Then $\gamma^ {(n)}\rightarrow\gamma^*$ in $C^ 0([0,1])$ and $\dot\gamma^ {(n)}\rightarrow\dot\gamma^*$ in $L^2((0,1))$. Hence 
\begin{align}
    A[\gamma^{(n)}]&=2\pi\int_0^1|\dot\gamma^ {(n)}(t)|\gamma_1^ {(n)}(t) dt\rightarrow A[\gamma^*],\label{areaconv}\\
    V[\gamma^{(n)}]&=\pi\left|\int_0^1\dot\gamma^ {(n)}_2(t)\left(\gamma_1^ {(n)}(t)\right)^2 dt\right|\rightarrow V[\gamma^*].\label{volconv}
\end{align}
In particular $\mathcal I[\gamma^{(n)}]\rightarrow\mathcal I[\gamma^*]$. So, the area, volume and isoperimetric ratio are actually continuous with respect to the convergence from Definition \ref{convergencedefinition}. Moreover, in Lemmas \ref{lowersemiconLemma} and \ref{lemmaprinciplacuvaturesconv} in the appendix we prove
\begin{align}
&\mathcal W[\gamma^*]\leq\liminf_{n\rightarrow\infty}\mathcal W[\gamma^{(n)}],\label{Willmorelowersemicont}\\
%---------
&\int_0^1\left((k_1^* )^2+(k_2^* )^2\right)2\pi|\dot\gamma^*|\gamma_1^*dt
\leq\liminf_{n\rightarrow\infty}\int_0^1 \left((k_1^ {(n)})^2+(k_2^ {(n)})^2\right)2\pi|\dot\gamma^{(n)}|\gamma_1 ^{(n)}dt.\label{curvsqlowersemicont}
\end{align}


\subsection{Compactness}
The following compactness theorem is essentially due to Choski's and Veneroni's paper \cite{choskiveneroni} (see Subsection 3.4). However, as we have slightly changed the definition of convergence in $\mathcal P^w$, their proof must also be slightly modified. For the reader's convenience, we provide the proof in Appendix \ref{compactnessappendix}.
\begin{theorem}[Compactness Theorem]\label{compactnesstheorem}
Let $(\gamma^{(n)})\subset\mathcal P^w$ such that:
\begin{enumerate}[(1)]
    \item There exists $R>0$ such that $\gamma^ {(n)}(t)\in B_R(0)$ for all $t\in[0,1]$ and $n\in\N$.
    \item There exists $C<\infty$ such that 
    $$\sup_n \int_0^1\left[ \left(k_1^{(n)}\right)^2+\left(k_2^{(n)}\right)^2\right]2\pi\gamma_1^ {(n)}|\dot\gamma^ {(n)}|dt\leq C.$$
    \item There exist $\theta>0$ and $C<\infty$ such that  $0<\theta\leq A[\gamma^ {(n)}]\leq C$ for all $n\in\N$.
\end{enumerate} 
Then there exists $\gamma^*\in \mathcal P^w$  and a subsequence $\gamma^ {(n_k)}$ such that $\gamma^ {(n_k)}\rightarrow\gamma^*$ in $\mathcal P^w$.
\end{theorem}

\subsection{Existence of a Weak Minimizer}
To prove the existence of a minimizer $\gamma\in\mathcal F_\sigma$ satisfying $\mathcal W[\gamma]=\beta(\sigma)$, we require the following theorem, that is essentially due to Schygulla \cite{schygulla}.
\begin{theorem}[Schygulla]\label{schygulla}\ \\
 Let $\beta$ be as in Equation \eqref{betaDefinition}. Then $\beta(\sigma)<8\pi$ for all $\sigma\in(0,1]$.
\end{theorem}
Schygulla proves this for surfaces without requiring axial symmetry. Therefore Theorem \ref{schygulla} cannot be cited directly from his paper. However, inspecting his proof, it is straightforward to see that it can easily be adapted for our purposes. We comment on the necessary changes in Appendix \ref{schygullapp}.\\

Using Theorem \ref{schygulla}, we can now prove the existence of a weak minimizer.
\begin{theorem}[Existence of Weak Minimizer]\label{weakminimizerexistence}\ \\
For all $\sigma\in(0,1]$ there exists $\gamma\in\mathcal F_\sigma$ such that $\beta(\sigma)=\mathcal W[\gamma]$.
\end{theorem}
\begin{proof}
For $\sigma=1$, any round sphere is a minimizer. For $\sigma\in(0,1)$ let $\gamma^{(n)}\subset \mathcal P$ such that $\mathcal W[\gamma^{(n)}]\rightarrow\beta(\sigma)$. As $\beta(\sigma)<8\pi$, we can assume without loss of generality that $\mathcal W[\gamma^{(n)}]\leq 8\pi-\rho_0$ for all $n\in\N$ and some $\rho_0>0$. The Willmore energy $\mathcal W$ and the isoperimetric ratio $\mathcal I$ are invariant under translations. So, by defining
$$\tilde\gamma^{(n)}(t):=\gamma^{(n)}(t)-(0,\gamma^{(n)}_2(0)),$$
we can further assume that $\gamma^{(n)}(0)=0$ for all $n\in\N$. Let $L_n:=L[\gamma^{(n)}]$. For every individual $\gamma^{(n)}$, the Gauß-Bonnet theorem (see Lemma \ref{gausbonnetlemma}) holds and hence
\begin{align}
&\int_0^1\left(\left(k_1^{(n)}\right)^2+\left(k_2^{(n)}\right)^2\right)2\pi L_n \gamma_1^{(n)}(t)dt\nonumber\\
=&4\mathcal W[\gamma^{(n)}]-2\int_{0}^1k_1^{(n)}k_2^{(n)}2\pi L_n \gamma_1^{(n)}(t)dt\nonumber\\
\leq & 4\cdot(8\pi-\rho_0)-8\pi.\label{compactnesssecondassumpion}
\end{align}
In view of Theorem \ref{choskiveneroniresult}, we get $L_n\leq C$. For any $t\in[0,1]$ and $n\in\N$ we can now estimate 
$$|\gamma^{(n)}(t)|=|\gamma^{(n)}(t)-\gamma^{(n)}(0)|\leq \int_0^1 |\dot\gamma^{(n)}(t)|dt\leq C.$$
Finally $A[\gamma^ {(n)}]\equiv 1$. Thus, all assumptions from the compactness theorem (Theorem \ref{compactnesstheorem}) are satisfied and we can extract a subsequence, again denoted by $\gamma^{(n)}$, that converges to a limit curve $\gamma^*\in\mathcal P^w$. Let $L_*:=L[\gamma^*]$. We will now prove that $\gamma^*\in\mathcal P$. Using the lower semicontinuity results from Equations \eqref{Willmorelowersemicont} and \eqref{curvsqlowersemicont} and Estimate \eqref{compactnesssecondassumpion}, we get 
\begin{align} 
\mathcal W[\gamma^*]\leq\beta(\sigma)
\hspace{.5cm}\textrm{and}\hspace{.5cm}
\int_0^1\left(\left(k_1^{*}\right)^2+\left(k_2^{*}\right)^2\right)2\pi L_* \gamma_1^{*}(t)dt&<3\cdot 8\pi.\label{leq38pi}
\end{align}
So, by Theorem \ref{weakandstrongPconnection}, there is at most one $\tau\in(0,1)$ such that $\gamma^*_1(\tau)=0$. Suppose that such a $\tau$ exists. Then, by Theorem \ref{weakandstrongPconnection}, $\gamma^*$ restricted to $[0,\tau]$ and $[\tau,1]$, up to reparameterization, belong to $\mathcal P$. Applying the Li-Yau-inequality (Lemma \ref{WillmoreLowerBound}) produces the contradiction
\begin{equation}\label{reallyinP}
8\pi-\rho_0\geq \mathcal W[\gamma^*]
=
\mathcal W[\gamma^*|_{[0,\tau]}]+\mathcal W[\gamma^*|_{[\tau,1]}]
\geq 2\cdot 4\pi.
\end{equation}
This shows $\gamma_{1}^*(t)>0$ for all $t\in(0,1)$ and hence $\gamma^*\in\mathcal P$. Using the continuity of the area and volume functional from Equations \eqref{areaconv} and \eqref{volconv}, we deduce $\gamma^*\in\mathcal F_\sigma$ and therefore $\mathcal W[\gamma^*]\geq \beta(\sigma)$. In view of Estimate \eqref{leq38pi} we get $\mathcal W[\gamma^*]=\beta(\sigma)$ and hence $\gamma^*$ is a minimizer.
\end{proof}