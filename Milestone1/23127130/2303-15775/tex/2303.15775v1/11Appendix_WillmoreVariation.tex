\subsection{First Variation of the Willmore Energy}
Let $I=[s_0,s_1]$ be an interval in $\R$ and $\gamma=(\gamma_1,\gamma_2)\in C^\infty(I,\mathcal H^2)$. We assume that the axisymmetric surface defined by 
$$f:I\times[0,2\pi)\rightarrow\R^3,\ f(s, \theta):=\begin{bmatrix}
\gamma_1(s)\cos\theta\\
\gamma_1(s)\sin\theta\\
\gamma_2(s)
\end{bmatrix}$$
is smooth. Given a variation $\phi:(-\epsilon_0,\epsilon_0)\rightarrow\mathcal H^2$ of $\gamma$, we can define a variation of $f$ by putting 
$$\Phi:(-\epsilon_0,\epsilon_0)\times I\times[0,2\pi)\rightarrow\R^3,\ \Phi(\epsilon,s,\theta):=\begin{bmatrix}
\phi_1(\epsilon,s)\cos\theta\\
\phi_1(\epsilon,s)\sin\theta\\
\phi_2(\epsilon,s)
\end{bmatrix}.$$
Considering Equations \eqref{metricprofilecurve} for the metric, \eqref{principalcurvatures} for the principal curvatures of $f$ and the formula for $W[f]$ from Equation \eqref{WillmoreOperator}, we observe that $W$ only depends on $s$ and not on $\theta$. This justifies the notation $W[\gamma]$ for the very same function. Using the formulas for the normals of $\Sigma_\gamma$ and $\gamma$ from Equation \eqref{normalsprofilecurve} we further have
$\langle\Phi'(0),n\rangle=\langle\phi'(0), \nu\rangle$. 
Using the surface element from Equation \eqref{metricprofilecurve}, we can rewrite the surface integral in Equation \eqref{classicalformula} as
$$\int_{I\times\Sp^1}W[f]\langle\Phi'(0),n\rangle d\mu_f=2\pi\int_{s_0}^{s_1} W[\gamma]\langle\phi'(0), \nu\rangle \gamma_1|\dot\gamma|ds.$$
Let us now consider the boundary term. We assume that $\gamma$ is not closed -- that is $\gamma(s_0)\neq\gamma(s_1)$. Additionally, we assume that $\gamma_1(s_0)\neq 0\neq \gamma_1(s_1)$. In this case $\partial\Sigma_\gamma\neq\emptyset$ and
$$\partial\Sigma_\gamma=\set{f(s,\theta)\ |\ s\in\set{s_0,s_1},\ \theta\in[0,2\pi)}.$$
We focus on the boundary component at $s=s_0$. The interior conormal is given by
\begin{equation}\label{boundarycomputation1}
\eta=\frac1{|\dot\gamma(s)|}\frac\partial{\partial s}\bigg|_{s=s_0}
\hspace{.5cm}\textrm{and}\hspace{.5cm}
Df\eta=\frac1{|\dot\gamma|}\frac{\partial f}{\partial s}\bigg|_{s=s_0}.
\end{equation}
The tangential part of $\Phi'(0)$ is $\Phi'(0)-\langle\Phi'(0), n\rangle n$ and 
\begin{equation}\label{boundarycomputation2}
    \langle \Phi'(0)-\langle\Phi'(0), n\rangle n, Df\eta\rangle=\frac1{|\dot\gamma(s_0)|}\langle \Phi'(0), \frac{\partial f}{\partial s}\bigg|_{s=s_0}\rangle=\frac1{|\dot\gamma(s_0)|}\langle \phi'(0), \dot\gamma(s_0)\rangle.
\end{equation}
Combining Equations \eqref{boundarycomputation1} and \eqref{boundarycomputation2} and using the formula fom Equation \eqref{WillmoreBoundaryterm}, we get
\begin{align*}
\int_{s_0\times\Sp^1}\omega(\eta)dS_f=&\frac12\left[
\frac{\varphi(s_0) H'(s_0)-\varphi'(s_0) H(s_0)}{|\dot\gamma(s_0)|}
-
\frac12\frac{H(s_0)^2}{|\dot\gamma(s_0)|}\langle\dot\gamma(s_0), \phi'(0)\rangle 
\right]\int_0^{2\pi}\gamma_1(s_0)d\theta\\
%--------------------
=&\pi\left[
\frac{\varphi(s_0) H'(s_0)-\varphi'(s_0) H(s_0)
-\frac12H(s_0)^2\langle \dot\gamma(s_0),\phi'(0)\rangle}
{|\dot\gamma(s_0)|}
\right]\gamma_1(s_0).
\end{align*}

Computing the boundary term at $s=s_1$ is achieved by a similar computation. In total, we get the formula
\begin{equation}\label{variationformulastandard}
\begin{aligned}
    \delta\mathcal W[\gamma]\phi'(0)=&2\pi\int_{s_0}^{s_1} W[\gamma]\langle\phi,\nu\rangle \gamma_1|\dot\gamma|ds\\
    &-\pi\left[
\frac{\varphi(s) H'(s)-\varphi'(s) H(s)
-\frac12H^2(s)\langle \dot\gamma(s),\phi'(0)\rangle}
{|\dot\gamma(s)|}
\right]\gamma_1(s)\bigg|_{s=s_0}^{s=s_1}.
\end{aligned}
\end{equation}


\subsection{Existence of Variations}\label{variationexistenceappendix}
\begin{lemma}\label{variationexistence}
Let $\sigma\in(0,1)$, $\gamma\in\mathcal F_\sigma$ and $\eta\in C^\infty_0((0,1))$ such that $\delta\mathcal I[\gamma]\eta=0$. Then there exists $\Phi:(-\epsilon_0,\epsilon_0)\rightarrow\mathcal F'_\sigma$ such that $\Phi'(0)=\eta$.
\end{lemma}
\begin{proof}
Since $\sigma\in(0,1)$, Lemma \ref{psi0existencelemma} provides $\psi_0\in C^\infty_0((0,1))$ such that $\delta\mathcal I[\gamma]\psi_0=1$. For small $\epsilon,\rho\in\R$ we have $\gamma+\epsilon\eta+\rho\psi_0\in\mathcal P'$ as $\operatorname{supp}(\eta),\ \operatorname{supp}(\psi_0)\subset(0,1)$. Now, we define the operator 
$$Q(\epsilon,\rho):=\mathcal I[\gamma+\epsilon\eta+\rho\psi_0].$$
Clearly $Q(0,0)=\sigma$ and 
$$\frac d{d\rho}\bigg|_{\rho=0}Q(0,\rho)=\delta\mathcal I[\gamma]\psi_0=1\neq 0.$$
So, by the implicit function theorem, there exist $\epsilon_0, \rho_0>0$ and a smooth map $\kappa:(-\epsilon_0,\epsilon_0)\rightarrow (-\rho_0,\rho_0)$ such that for $\epsilon\in(-\epsilon_0,\epsilon_0)$ and $\rho\in(-\rho_0,\rho_0)$ we have
$$Q(\epsilon,\rho)=\sigma
\hspace{.5cm}\Leftrightarrow\hspace{.5cm}
\rho=\kappa(\epsilon).
$$
Moreover $\kappa'(0)=0$ as
$$0=\frac d{d\epsilon}\bigg|_{\epsilon=0}Q(\epsilon,\kappa(\epsilon))=\delta\mathcal I[\gamma]\eta+\kappa'(0)\delta\mathcal I[\gamma]\psi_0=\kappa'(0).$$
Therefore $\Phi(\epsilon):=\gamma+\epsilon\eta+\kappa(\epsilon)\psi_0$ provides a variation as claimed. 
\end{proof}

\noindent
\textit{Proof of Lemma \ref{eulerlagangeataxis}.}\ \\
Using $\varphi'(0)=0$ and $\operatorname{supp}(\psi_0)\subset(0,1)$, we have $\gamma+\epsilon\chi\cdot(\varphi\circ\gamma_1)e_2+\rho\psi_0\in\mathcal P'$. This allows the definition of the operator 
$$Q(\epsilon,\rho):=\mathcal I[\gamma+\epsilon\chi\cdot(\varphi\circ\gamma_1)e_2+\rho\psi_0].$$
Note $Q(0,0)=\sigma$ and 
$$\frac d{d\rho}\bigg|_{\rho=0} Q(0,\rho)=\delta\mathcal I[\gamma]\psi_0=1\neq 0.$$
So, by the implicit function theorem, there exist $\epsilon_0, \rho_0>0$ and a diffeomorphism $\kappa:(-\epsilon_0,\epsilon_0)\rightarrow (-\rho_0,\rho_0)$ such that for $\epsilon\in(-\epsilon_0,\epsilon_0)$ and $\rho\in(-\rho_0,\rho_0)$ we have
$$Q(\epsilon,\rho)=\sigma
\hspace{.5cm}\Leftrightarrow\hspace{.5cm}
\rho=\kappa(\epsilon).
$$
Moreover we have $\kappa'(0)=-\delta\mathcal I[\gamma]\left(\chi\cdot(\varphi\circ\gamma_1) e_2\right)$ as 
$$0=\frac d{d\epsilon}\bigg|_{\epsilon=0}Q(\epsilon,\kappa(\epsilon))=\delta\mathcal I[\gamma]\left(\chi\cdot(\varphi\circ\gamma_1) e_2\right)+\kappa'(0)\delta\mathcal I[\gamma]\psi_0.$$
Therefore $\Phi(\epsilon):=\gamma+\epsilon\chi\cdot(\varphi\circ\gamma_1) e_2+\kappa(\epsilon)\psi_0$ provides a variation as claimed. 
\qed



\subsection{Proof of Theorem \ref{schygulla}}\label{schygullapp}

Schygulla's proof in \cite{schygulla} is based on studying the scaled catenoid. For $a>0$ he considers 
$$g_a:\R\times[0,2\pi)\rightarrow\R^3,\ g_a(s,\theta):=\left(
a\cosh\left(\frac sa\right)\cos(\theta),a\cosh\left(\frac sa\right)\sin(\theta),s
\right).$$
Next, he introduces the inversion $I(x):=e_3+\frac{x-e_3}{|x-e_3|^2}$, defines $f_a:=I\circ g_a$ and puts 
$$\Sigma_a:=f_a(\R\times[0,2\pi))\cup\set{e_3}.$$


$\Sigma_a$ is smooth away from $e_3$. Additionally, there exist $R>0$, functions $u^\pm:B_R(0)\rightarrow\R$ and a neighbourhood of $e_3$ in which $\Sigma_a=\operatorname{graph}(u^+)\cup\operatorname{graph}(u^-)$. Direct computation shows $u^\pm\in C^{1,\alpha}(B_R(0))\cap W^{2,p}(B_R(0))$ for all $\alpha\in(0,1)$ and $p\in[1,\infty)$. Another direct computation shows $\mathcal W[\Sigma_a]=8\pi$.\\

We now sketch the proof of Theorem \ref{schygulla} by following Schygulla's arguments and commenting on the necessary alterations. For details, we refer to Lemma 1 in Schygulla's article \cite{schygulla}.
\begin{proof}
Let $\varphi_\pm\in C^\infty_0(B_R)$ be axially symmetric and smooth functions --  that is $\varphi(x)=\varphi(Tx)$ for all $T\in \operatorname{SO}(2)$ -- that satisfy $\varphi_{\pm}(0)=\pm 1$. For small $\epsilon>0$, we define variations
$$\Phi_\pm^\epsilon:B_R(0)\rightarrow\R^3,\ \Phi^\epsilon_\pm(x):=(x, u_\pm(x)+\epsilon\varphi_\pm(x)).$$
Following Schygulla's computations, it follows that there exists $c_0>0$ such that
$$\frac d{d\epsilon}\bigg|_{\epsilon=0}\mathcal W[\Phi_\pm^\epsilon]=\mp c_0\varphi_\pm(0).$$
We modify $\Sigma_a$ and construct a surface $\Sigma_a^\epsilon$ by replacing the graphs of $u^\pm$ with the graphs of $u_\pm+\epsilon\varphi_\pm$ with small $\epsilon>0$. Consequently $\mathcal W[\Sigma_a^\epsilon]<8\pi-\epsilon c_0$ for small $\epsilon>0$. Additionally, $\Sigma_a^\epsilon$ can be parameterized over $\Sp^2$ and is axially symmetric. Finally, we note $\mathcal I[\Sigma_a]\rightarrow 0$, as $a\rightarrow0^+$. Given $\sigma\in(0,1)$ we can therefore choose $a$ and $\epsilon$ so small such that simultaneously
\begin{align*}
    \mathcal W[\Sigma_a^\epsilon]&\leq \mathcal W[\Sigma_a]-\epsilon c_0=8\pi-\epsilon c_0,\\
    \mathcal I[\Sigma_a^\epsilon]&<\sigma.
\end{align*}
Approximating $\Sigma_a^\epsilon$ with smooth surfaces, we deduce that there exists a smooth, axially symmetric embedding $F_0:\Sp^2\rightarrow\R^3$ with $\mathcal W[F_0]<8\pi$ and $\mathcal I[F_0]<\sigma$. Theorem 5.2 in \cite{removability} allows us to deduce that the Willmore flow $F(t)$ with initial datum $F(0)=F_0$ exists for all times and converges to a sphere. As $F_0$ is axially symmetric and the solution to the Willmore flow is unique  (Proposition 1.1 in \cite{kuwert2002gradient}), it follows that $F(t)$ is axially symmetric for all $t\geq 0$. Since $\mathcal I[F(0)]<\sigma$ and $\mathcal I[F(t)]\rightarrow 1$ as $t\rightarrow\infty$, there exists a time $t_0>0$ such that $\mathcal I[F(t_0)]=\sigma$. Finally, by definitions of $\beta(\sigma)$ and the dissipation of Willmore energy along the Willmore flow, we get
$$\beta(\sigma)\leq \mathcal W[F(t_0)]\leq \mathcal W[F(0)]<8\pi.$$
\end{proof}


