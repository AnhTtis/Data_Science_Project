\section{Technicalities regarding the Direct Method}

\subsection{Lower Semi Continuity}
The proof of Equations \eqref{Willmorelowersemicont} and \eqref{curvsqlowersemicont} is the consequence of two lemmas. First, we establish the following:
\begin{lemma}\label{lowersemiconLemma}
Let $\rho_n,\rho\in C^0([0,1],\R^+_0)$ and $f_n,f:[0,1]\rightarrow\R$ be measurable such that $\rho_n\rightarrow\rho$ in $C^0([0,1])$ and $f_n\rho_n\rightarrow f\rho$ weakly in $L^1((0,1))$. Further assume $f^2\rho\in L^1((0,1))$ and $\|f_n^2\rho_n\|_{L^1((0,1))}\leq C<\infty$ for all $n\in\N$. Then
$$\int_0^1 f^2(t)\rho(t)dt\leq\liminf_{n\rightarrow\infty}\int_0^1 f_n^2(t)\rho_n(t)dt.$$
\end{lemma}
All integrals are well-defined since $f^2\rho\geq 0$ and $f_n^2\rho_n\geq 0$. 
\begin{proof}
For $R>0$ we put $\phi_R:=f1_{|f|\leq R}\in L^\infty((0,1))$. Using $f_n\rho_n\rightarrow f\rho$ weakly in $L^1((0,1))$ and $\rho_n\rightarrow\rho$ in $C^0([0,1])$, we estimate
\begin{align*}
    \int_0^1 f(t)\phi_R(t)\rho(t)dt=&\lim_{n\rightarrow\infty}\int_0^1 f_n(t)\phi_R(t)\rho_n(t)dt\\
                                \leq &\liminf_{n\rightarrow\infty}\left[\left(\int_0^1 f_n(t)^2 \rho_n(t)dt\right)^{\frac12}\left(\int_0^1 \phi_R(t)^2\rho_n(t)dt\right)^{\frac12}\right]\\
                                \leq & \liminf_{n\rightarrow\infty}\left(\int_0^1 f_n(t)^2 \rho_n(t)dt\right)^{\frac12}\left(\int_0^1 \phi_R(t)^2\rho(t)dt\right)^{\frac12}.
\end{align*}    
Note that $\phi_R\rightarrow f$ pointwise and $|\phi_R|\leq |f|$. As $|f|^2\rho\in L^1((0,1))$ we can let $R\rightarrow\infty$ and get the claimed estimate. 
\end{proof}

 \begin{lemma}\label{lemmaprinciplacuvaturesconv}
Let $\gamma^{(n)},\gamma^*\in\mathcal P^w$ such that $\gamma^{(n)}\rightarrow \gamma^*$ in $\mathcal P^w$ and 
$$\sup_n\int_0^1 \left((k_1^ {(n)})^2+(k_2^ {(n)})^2\right) 2\pi \gamma_1^ {(n)}|\dot\gamma^ {(n)}|dt<\infty.$$
Then $k_i^{(n)}\gamma_1^{(n)}\rightarrow k_i^*\gamma_1^*$ weakly in $L^1((0,1))$ for $i=1,2$. 
\end{lemma}

\begin{proof}
Let $\phi\in L^ \infty((0,1))$, $L_n:=L[\gamma^{(n)}]$ and $L_*:=L[\gamma^*]$. Then $L_n\rightarrow L_*$ and, as $\gamma^*\in\mathcal P^ w$, we have $L_*>0$. We put $R=\begin{bmatrix}
0 & -1\\
1 & 0
\end{bmatrix}$ and estimate 
$$\left|\int_0^1 \langle\ddot\gamma^{(n)},R\dot\gamma^{(n)}-R\dot\gamma^{*}\rangle\gamma_1^{(n)}\phi dt\right|
\leq 
\|\phi\|_{L^\infty}\left[\int_0^1|\ddot\gamma^{(n)}|^2\gamma_1^{(n)}dt\right]^{\frac12}\left(\sup_n \gamma_1^{(n)}\right)^{\frac12}\|\dot\gamma^{(n)}-\dot\gamma^*\|_{L^2}.$$
$\||\ddot\gamma^ {(n)}|^2\gamma_1^ {(n)}\|_{L^1((0,1))}$ is bounded by combining the assumptions of the lemma with Lemma \ref{gamma1ddotgammainL1}. Using $\gamma^{(n)}\rightarrow\gamma^*$ in $C^0([0,1])$ and $W^ {1,2}((0,1))$, we deduce that 
\begin{equation}\label{erroerestiamtekconvergenceeq1}
\lim_{n\rightarrow\infty}\left|\int_0^1 \langle\ddot\gamma^{(n)},R\dot\gamma^{(n)}-R\dot\gamma^{*}\rangle\gamma_1^{(n)}\phi dt\right|=0.
\end{equation}
Note that $R\dot\gamma^*\phi\in L^\infty$. So, by weak convergence in $L^1$, we get 
$$\lim_{n\rightarrow\infty}\int_0^1\langle \ddot\gamma^{(n)} ,R\dot\gamma^{*}\rangle\gamma_1^{(n)}\phi dt
=
\lim_{n\rightarrow\infty}\int_0^1 \langle\ddot\gamma^{*}, R\dot\gamma^{*}\rangle\gamma_1^{*}\phi dt
.$$
In view of Equation \eqref{erroerestiamtekconvergenceeq1} as well as Equation \eqref{principalcurvatures}, we get $k_1^{(n)}\gamma_1^{(n)}\rightarrow k_1^*\gamma_1^*$ weakly in $L^1((0,1))$.\ \\


Next, we take care of $k_2^{(n)}$. As $\dot\gamma^{(n)}\rightarrow\dot\gamma^*$ in $L^2((0,1))$ we get 
 $$\int_0^1 k_2^{(n)}\gamma_1^{(n)}(t)\phi(t)dt=\frac{2\pi}{ L_n}\int_0^1 \dot\gamma_2^{(n)}(t) \phi(t)dt\rightarrow \int_0^1 k_2^*\gamma_1^*(t)\phi(t)dt.$$
\end{proof}

\noindent
\textit{Proof of Estimates \eqref{Willmorelowersemicont} and \eqref{curvsqlowersemicont}.}\ \\
   The proof for both estimates is essentially the same. To prove Estimate \eqref{Willmorelowersemicont}, we pass to a subsequence that realizes the $\liminf$ and assume without loss of generality that the $\liminf$ is finite. Consequently 
   \begin{equation}\label{liminfsuprealization}
   \sup_{n\in\N}\int_0^1\left(k_1^ {(n)}+k_2^ {(n)}\right)^22\pi|\dot\gamma^ {(n)}|\gamma_1^ {(n)}dt<\infty.\\
  \end{equation}
   We wish to apply Lemma \ref{lowersemiconLemma} with the functions $f_n=k_1^ {(n)}+k_2^ {(n)}$, $f=k_1^*+k_2^*$, $\rho_n=\gamma_1^ {(n)}$ and $\rho=\gamma_1^*$. To justify this choice we first use Lemma  \ref{lemmaprinciplacuvaturesconv} to get $(k_1^{(n)}+k_2^{(n)})\gamma_1^{(n)}\rightarrow (k_1^*+k_2^*)\gamma_1^*$ weakly in $L^1((0,1))$. Additionally we have Estimate \eqref{liminfsuprealization} and $(k_1^*+k_2^*)^2\gamma_1^*\in L^1((0,1))$ by definition of the class $\mathcal P^ w$.  Estimate \eqref{Willmorelowersemicont} now follows from Lemma \ref{lowersemiconLemma}. To get  Estimate \eqref{curvsqlowersemicont}, the only change is to take $f_n=k_i^ {(n)}$ and $f=k_i^*$ for $i=1,2$. \qed

\subsection{Compactness}\label{compactnessappendix}
The following lemma is due to Choski and Veneroni \cite{choskiveneroni} (see Lemma 5). We present a modified proof. 

\begin{lemma}\label{oscbound}
Let $\gamma\in\mathcal P^w$, $L:=L[\gamma]$ and $(a,b)\subset [0,1]$ such that $\gamma_1(t)>0$ for all $t\in(a,b)$. Then 
\begin{align*} 
|\dot\gamma_1(b^-)-\dot\gamma_1(a^+)|&\leq \frac L{2\pi}\int_a^b |k_1k_2|2\pi L\gamma_1(t)dt,\\
|\dot\gamma_2(b^-)-\dot\gamma_2(a^+)|&\leq\sqrt{\frac{L}{2\pi}}  \left(\int_a^b k_1^2 2\pi L\gamma_1 dt\right)^{\frac12}
    \left(\int_a^b\frac{\dot\gamma_1^2}{\gamma_1}dt\right)^{\frac12}.
\end{align*}
In  particular, if $\dot\gamma$ is continuous at $a$ and $b$, the one sided limits can be replaced by $\dot\gamma(a)$ and $\dot\gamma(b)$ respectively. 
\end{lemma}
Before presenting the proof, we have the following remark: In the proof, we use Theorem \ref{weakandstrongPconnection} to deduce that $\dot\gamma$ has one-sided limits at $a$ and $b$. The analog of Theorem \ref{weakandstrongPconnection} in \cite{choskiveneroni} is Lemma 6 and the proof uses Lemma 5 (the analog of our Lemma \ref{oscbound}). However, the continuity statement we use here is Corollary 2 in \cite{choskiveneroni} and proven without using the analog of our Lemma \ref{oscbound}. 
\begin{proof}
Let $\epsilon>0$ be small. Then $\gamma_1\geq c_0(\epsilon)>0$ on $[a+\epsilon, b-\epsilon]$ so that $\gamma\in W^{2,2}((a+\epsilon, b-\epsilon))$ by Definition \ref{PWeakDefinition}. 
Using Equation \eqref{ArcLength_K_Identity}, we compute 
$$
    \frac1{2\pi}\int_{a+\epsilon}^{b-\epsilon}|k_1k_2|2\pi L\gamma_1 dt
    \geq 
    \frac1L\left|\int_{a+\epsilon}^{b-\epsilon}\ddot\gamma_1(t)dt\right|
    =
    \frac{|\dot\gamma_1(b-\epsilon)-\dot\gamma_1(a+\epsilon)|}{L}.
$$
Now we let $\epsilon\rightarrow 0^+$. To do so, we use $|k_1k_2|2\pi L\gamma_1\in L^1((0,1))$ by definition of the class $\mathcal P^w$ and Theorem \ref{weakandstrongPconnection} to get the first estimate. For the second estimate, we argue similarly and write 
\begin{align*}
 \frac{|\dot\gamma_2(b-\epsilon)-\dot\gamma_2(a+\epsilon)|}{L}&=
 \frac1L\left|\int_{a+\epsilon}^{b-\epsilon}\ddot\gamma_2(t)dt\right|\\
 &=  \left|\int_{a+\epsilon}^{b-\epsilon}k_1 \dot\gamma_1 dt\right|\\
 &=  \frac1{\sqrt{2\pi L}}\left|\int_{a+\epsilon}^{b-\epsilon}k_1 \frac{\dot\gamma_1}{\sqrt{\gamma_1}} \sqrt{2\pi L\gamma_1} dt\right|\\
 &\leq  \frac1{\sqrt{2\pi L}}\left[
    \left(\int_a^b k_1^2 2\pi L\gamma_1 dt\right)^{\frac12}
    \left(\int_a^b\frac{\dot\gamma_1^2}{\gamma_1}dt\right)^{\frac12}
 \right].
\end{align*}
Note that while valid, we may have estimated by $+\infty$ in the last step. Letting $\epsilon\rightarrow 0^+$, the second estimate claimed in the lemma follows.
\end{proof}

For the proof of Theorem \ref{compactnesstheorem}, we require the space of bounded variations.
\begin{definition}
Let $\alpha,\beta\in\R$. For a measurbale function $f:[\alpha,\beta]\rightarrow\R$ we put
$$V_\alpha^\beta(f):=\sup\left\{\sum_{i=0}^{N-1}|f(t_{i+1})-f(t_i)|\ \bigg|\ N\in\N\textrm{ and }\alpha=t_0<t_1<...<t_{N-1}<t_N=\beta\right\}.$$
$f$ is said to have \emph{bounded variation} if $V_\alpha^\beta(f)<\infty$. If additionally $f\in L^1((\alpha,\beta))$, we define
$\|f\|_{\operatorname{BV}((\alpha,\beta))}:=\|f\|_{L^1((\alpha,\beta))}+V_\alpha^\beta(f)$. 
The space of all $f$ with $\|f\|_{\operatorname{BV}((\alpha,\beta))}<\infty$ is denoted by $\operatorname{BV}((\alpha,\beta))$. 
\end{definition}
We stress that to define $V_\alpha^\beta(f)$, we consider a function $f:[\alpha,\beta]\rightarrow\R$ and not the equivalence class of all functions $\tilde f$ such that $\tilde f=f$ almost everywhere. We require the following theorem. For a proof, we refer to Theorem 2.35 and Proposition 2.38 in \cite{leoni2017first}.
\begin{theorem}[Helly's Selection Theorem]\label{helly}
Let $(f_n)$ be a sequence of measurable functions $f_n:[\alpha,\beta]\rightarrow\R$ and $c\in[\alpha,\beta]$ such that $|f_n(c)|+V_\alpha^\beta(f_n)\leq C<\infty$ for all $n\in\N$. Then, there exists a subsequence $f_{n_a}$ and a measurable function $f:[\alpha,\beta]\rightarrow\R$ such that $f_{n_a}\rightarrow f$ pointwise and
$$V_\alpha^\beta(f)\leq\liminf_{a\rightarrow\infty}V_\alpha^\beta(f_{n_a}).$$
\end{theorem}
We establish the following corollary:
\begin{korollar}\label{BVCompactInL1}
Let $(f_n)\subset \operatorname{BV}((\alpha,\beta))$ be a sequence such that $\|f_n\|_{\operatorname{BV}((\alpha,\beta))}\leq C<\infty$. Then, there exist $f\in \operatorname{BV}((\alpha,\beta))$ and a subsequence  $(f_{n_a})$ such that $f_{n_a}\rightarrow f$ pointwise and in $L^1((\alpha,\beta))$. 
\end{korollar}
\begin{proof}
We first claim that $f_n(\alpha)$ is bounded. Once this is shown, we can use Helly's selection theorem to get a subsequence $f_{n_a}$ such that $f_{n_a}\rightarrow f$ pointwise. For $x\in(\alpha,\beta)$ we estimate 
\begin{align*}
|f_{n_a}(x)|&\leq |f_{n_a}(\alpha)|+|f_{n_a}(\alpha)-f_{n_a}(x)|\\
&\leq  |f_{n_a}(\alpha)|+|f_{n_a}(\alpha)-f_{n_a}(x)|+|f_{n_a}(x)-f_{n_a}(\beta)|\\
&\leq \sup_{a\geq 0}\left(|f_{n_a}(\alpha)|+V_\alpha^\beta(f_{n_a})\right).
\end{align*}
Therefore $f_{n_a}$ is bounded in $L^\infty((\alpha,\beta))$ and hence $f\in L^\infty((\alpha,\beta))$. Using Lebesgue's theorem, we obtain 
$$0=\lim_{a\rightarrow\infty}\int_\alpha^\beta |f(t)-f_{n_a}(t)|dt.$$
To prove that $f_n(\alpha)$ is bounded, let $x_n\in[\alpha,\beta]$ such that $|f_n(x_n)|<\inf_{[\alpha,\beta]}|f_n|+1$. We have the estimate 
$$C\geq \int_\alpha^\beta |f_n(x)|dx\geq \int_\alpha^\beta |f_n(x_n)|-1 dx= |f_n(x_n)|-(\beta-\alpha).$$
So, $f_n(x_n)$ is bounded and we can estimate
$$|f_n(\alpha)|\leq |f_n(x_n)|+|f_n(x_n)-f_n(\alpha)|+|f_n(\beta)-f_n(x_n)|\leq |f_n(x_n)|+V_\alpha^\beta(f_n)\leq C.$$
\end{proof}

Let $U\subset\R$ be open. Then there exist at most countably many and uniquely determined disjoint intervals $I_j=(\alpha_j,\beta_j)$ such that $U=\bigcup_j I_j$. This is e.g. shown in \cite{pugh2003real}, Chapter 2, Section 1, Theorem 9. Using this fact, we extend the notion of the space of bounded variations to general open sets of $\R$.

\begin{definition}
Let $U\subset\R$ be open, $m\in\N\cup\set\infty$ and $U=\bigcup_{j=1}^m(\alpha_j,\beta_j)$ be the unique decomposition of $U$ into disjoint, open intervals. For a measurable function $f:U\rightarrow\R$, we put 
$$V_U(f):=\sum_{j=1}^m V_{\alpha_j}^{\beta_j}(f).$$
$f$ is said to have \emph{bounded variation} if $V_U(f)<\infty$. If additionally $f\in L^1(U)$, we define $\|f\|_{\operatorname{BV}(U)}:=\|f\|_{L^1(U)}+V_U(f)$. The space of all $f$ with $\|f\|_{\operatorname{BV}(U)}<\infty$ is denoted by $\operatorname{BV}(U).$
\end{definition}

By a usual diagonal argument, we get the following corollary from Corollary \ref{BVCompactInL1}
\begin{korollar}\label{BVCompactInL1Mk2}
    Let $U\subset\R$ be open and $(f_n)\subset \operatorname{BV}(U)$ such that $\|f_n\|_{\operatorname{BV}(U)}\leq C<\infty$. Then there exists $f\in \operatorname{BV}(U)$ and a subsequence $(f_{n_a})$ such that $f_{n_a}\rightarrow f$ in $L^1(U)$. 
\end{korollar}


Finally, we establish Theorem \ref{compactnesstheorem}. The following proof is due to Choski and Veneroni \cite{choskiveneroni} (see Proposition 2) with only minor modifications.
\renewcommand{\proofname}{\textit{Proof of Theorem \ref{compactnesstheorem}.}}
\begin{proof}
We put $L_n:=L[\gamma^{(n)}]$. By the assumptions of Theorem \ref{compactnesstheorem}, $\gamma^{(n)}$ is bounded in $C^0([0,1])$ and combining the assumptions of Theorem \ref{compactnesstheorem} with Theorem \ref{choskiveneroniresult} shows that $\dot\gamma^ {(n)}$ is bounded in $L^\infty((0,1))$. In particular, we can assume without loss of generality that $L_n\rightarrow L_*\in[0,\infty)$.
\ \\
\ \\
\noindent
\textbf{Step 1 - Weak Limit}\ \\
Since $\gamma^{(n)}$ is bounded in $C^0$ and $|\dot\gamma^{(n)}|$ is bounded in $L^\infty((0,1))$ and therefore also in $L^2((0,1))$, we can use the Arzel\`a-Ascoli theorem to deduce that, after passing to a subsequence, $\gamma^{(n)}\rightarrow\gamma^*$ in $C^0([0,1])$ and $\dot\gamma^{(n)}\rightarrow\dot\gamma^* $ weakly in $L^2((0,1))$.
Next, we prove by contradiction that $L_*>0$. Suppose $L_*=0$, then, as $\gamma^ {(n)}\rightarrow\gamma^*$ in $C^ 0([0,1])$, we get
$$\theta\leq\liminf_{n\rightarrow\infty}A[\gamma^{(n)}]=2\pi\liminf_{n\rightarrow\infty}\int_0^1\gamma_1^{(n)}(t)L_n dt=0.$$
\ \\
\noindent
\textbf{Step 2 - Convergence in $W^ {1,2}$}\ \\
By Lemma \ref{oscbound}, the sequence $\dot\gamma^{(n)}_1$ is bounded in $\operatorname{BV}((0,1))$ since 
$$
V_0^1(\dot\gamma_1^{(n)})\leq \frac{L_n}{2\pi}\int_0^1 |k_1^{(n)}k_2^{(n)}|2\pi L_n\gamma_1^{(n)}(t) dt\leq C.
$$
Using Corollary \ref{BVCompactInL1}, we deduce that there exists $\sigma\in L^1((0,1))$ such that  $\dot\gamma^{(n)}_1\rightarrow\sigma$ strongly in $L^1((0,1))$ along a subsequence and since $\dot\gamma^{(n)}\rightarrow\dot\gamma^*$ weakly in $L^2((0,1))$, we get $\sigma=\gamma_1^*$. As $\dot\gamma^{(n)}_1$ is bounded in $L^\infty((0,1))$, we get, $\dot\gamma_1^ {(n)}\rightarrow\dot\gamma_1^*$ in $L^ p((0,1))$ for all $p\in[1,\infty)$. Indeed
$$\int_0^1 |\dot\gamma_1^ {(n)}-\dot\gamma_1^ {(m)}|^ p dt\leq \|\dot\gamma_1^ {(n)}-\dot\gamma_1^ {(m)}\|_{L^ \infty}^ {p-1}\|\dot\gamma_1^ {(n)}-\dot\gamma_1^ {(m)}\|_{L^1((0,1))}\leq C\|\dot\gamma_1^ {(n)}-\dot\gamma_1^ {(m)}\|_{L^1((0,1))}$$
shows that $\dot\gamma_1^ {(n)}$ is Cauchy in $L^ p((0,1))$ and hence converges to some limit. By uniqueness of the limit in $L^1((0,1))$ we get $\dot\gamma_1^ {(n)}\rightarrow\dot\gamma_1^*$ in $L^ p((0,1))$ for all $p\in[1,\infty)$. \\

Next, we claim $\gamma_1^*(t)>0$ almost everywhere. To see this, let $E:=\set{t\in[0,1]\ |\ \gamma_1^*(t)=0}$. Clearly $E$ is compact and, by uniform convergence, $\gamma_1^{(n)}\leq\epsilon$ on $E$ for all $n\geq n_0(\epsilon)$. Using the second assumption of Theorem \ref{compactnesstheorem}, we get
$$C\geq \int_0^1 \left(k_2^{(n)}\right)^22\pi L_n\gamma_1^{(n)}dt=\frac{2\pi}{L_n}\int_0^1 \frac{\left(\dot\gamma_2^{(n)}\right)^2}{\gamma_1^{(n)}}dt\geq \frac{2\pi}{L_n \epsilon}\int_E\left(\dot\gamma_2^{(n)}\right)^2dt.$$
This shows $\dot\gamma^{(n)}_2\rightarrow 0$ in $L^2(E)$. Combining this with $\dot\gamma_1^ {(n)}\rightarrow\dot\gamma_1^*$ in $L^2((0,1))$, we get
$$|E|L_*^2=\lim_{n\rightarrow\infty}|E|L_n^2=\lim_{n\rightarrow\infty}\int_E|\dot\gamma^ {(n)}|^2dt=\int_E(\dot\gamma_1^*)^2dt.$$ 
Now assume $|E|>0$. Then $|\dot\gamma_1^*(t)|=L_*$ for almost all $t\in E$. Using Rademacher's theorem, we deduce that $\gamma$ is differentiable at almost all $t\in E$. Since the classical and the weak derivative agree almost everywhere, we deduce that there exists $t_0\in E$ such that $\gamma$ is differentiable in $t_0$ and $|\dot\gamma^*_1(t_0)|=L_*>0$. But by definition of $E$ we also have $\gamma_1^*(t_0)=0$, which implies that there exists $t\in[0,1]$ close to $t_0$ such that $\gamma_1^*(t)<0$, which is a contradiction.\\

Next, we prove that along a subsequence $\dot\gamma_2^{(n)}\rightarrow\dot\gamma_2$ in $L^1((0,1))$. For $k\in\N$ we put 
$A_k:=\set{t\in[0,1]\ |\ \gamma_1(t)>2k^{-1}}$. For $n\geq n_0(k)$ we have $\gamma_1^{(n)}\geq\frac1k$ on $A_k$. Using  Lemma \ref{oscbound}, we deduce that $\dot\gamma_2^{(n)}$ is bounded in $\operatorname{BV}(A_k)$ for all $k$ with 
$$V_{A_k}(\dot\gamma_2^{(n)})\leq \sqrt{\frac{L_n}{2\pi}}\left(\int_0^1 (k_1^{(n)})^2 2\pi L_n\gamma_1^{(n)}dt\right)^{\frac12}k^{\frac12}\left(\int_0^1 (\dot\gamma_1^{(n)})^2dt\right)^{\frac12}\leq C\sqrt k.$$
After passing to a subsequence we deduce $\dot\gamma_2^{(n)}\rightarrow\dot\gamma_2^*$ in $L^1(A_k)$ for all $k\in\N$. We estimate
$$\int_0^1|\dot\gamma_2^{(n)}-\dot\gamma_2^*|dt\leq 
\int_{A_k}|\dot\gamma_2^{(n)}-\dot\gamma_2^*|dt+(L_n+L_*)|[0,1]\backslash A_k|.$$
$|[0,1]\backslash A_k|\rightarrow 0$ as $k\rightarrow\infty$ since $|E|=0$. Thus $\dot\gamma_2^{(n)}\rightarrow\dot\gamma_2^*$ in $L^1((0,1))$ and, using the $L^\infty$-bound, also in $L^p((0,1))$ for all $1\leq p<\infty$.\\

In particular we have $\dot\gamma^{(n)}\rightarrow\dot\gamma^*$ in $L^1((0,1))$, which implies that for all $0\leq a<b\leq 1$
$$L_*(b-a)=\lim_{n\rightarrow\infty}L_n(b-a)=\lim_{n\rightarrow\infty}\int_a^b |\dot\gamma^{(n)}|dt=\int_a^b|\dot\gamma^*|dt.$$
Taking $a=0$ and $b=1$ shows $L[\gamma^*]=L_*$ and taking $b=a+\delta$ and letting $\delta\rightarrow 0$ shows $|\dot\gamma^*(a)|=L_*$ for almost every $a\in[0,1]$. 
\ \\
\ \\
\noindent
\textbf{Step 3 - Limit in $\mathcal P^ w$}\ \\
Let $U\subset(0,1)$ be open such that $\gamma_1^*>0$ on $\bar U$. We prove that $\gamma^*\in W^ {2,2}(U)$. First, as $\gamma^ {(n)}\rightarrow\gamma^*$ in $C^ 0([0,1])$ we know that there exists $n_0(U)$ such that for all $n\geq n_0(U)$ we have $\inf_U \gamma_1^{(n)}\geq c_0>0$. By Lemma \ref{gamma1ddotgammainL1}, $(\ddot\gamma^ {(n)})\subset L^2(U)$ is a bounded sequence and hence there exists $\xi\in L^2(U)$ such that $\ddot\gamma^ {(n)}\rightarrow\xi$ weakly in $L^2(U)$ along a subsequence. For $\varphi\in C^ \infty_0(U)$ we get 
$$\int_U\xi\varphi dt=\lim_{n\rightarrow\infty}\int_U\ddot\gamma^ {(n)}\varphi dt=-\lim_{n\rightarrow\infty}\int_U\dot\gamma^ {(n)}\dot\varphi dt=-\int_U\dot\gamma^*\dot\varphi dt.$$
We deduce $\gamma^*\in W^ {2,2}_{\operatorname{loc}}([\gamma_1^*>0])$. It remains to prove that the curvature of $\gamma^*$ is bounded in $L^2$. As $\gamma^ {(n)}\rightarrow\gamma^*$ in $W^ {1,2}((0,1))$ we have $\dot\gamma^ {(n_a)}\rightarrow\dot\gamma^*$ pointwise almost everywhere along a suitable subsequence and clearly $\gamma^ {(n)}\rightarrow\gamma^*$ pointwise. Hence, by Fatou's lemma and the assumptions of Theorem \ref{compactnesstheorem}
\begin{align*}
    \int_0^1 (k_2^*)^22\pi L\gamma_1^* dt&=
    \int_0^1 \frac{(\dot\gamma^*_2)^2}{\gamma_1^* L_*}2\pi dt\\
    & \leq \liminf_{a\rightarrow\infty}\int_0^1\frac{(\dot\gamma^{(n_a)}_2)^2}{\gamma_1^{(n_a)} L_{n_a}}2\pi dt\\
    &= \liminf_{a\rightarrow\infty}\int_0^1 \left(k_2^ {(n_a)}\right)^2 2\pi L_{n_a}\gamma_1^ {(n_a)}dt\\
    &\leq C.
\end{align*}
Next, we consider $k_1^*$. Let $U\subset (0,1)$ such that $\bar U\cap E=\emptyset$. Since $(\ddot\gamma^{(n)})\subset L^2(U)$ is bounded and $\gamma_1^{(n)}\rightarrow\gamma_1^*$ in $C^0([0,1])$, we may estimate  
\begin{equation}\label{substiutionestimate}
\left|\int_U |\ddot\gamma^ {(n)}|^2(\gamma_1^*-\gamma_1^ {(n)})dt\right|\leq \|\gamma^ {(n)}_1-\gamma_1^*\|_{C^ 0}\sup_n \|\ddot\gamma^ {(n)}\|_{L^2(U)}^2\rightarrow 0.
\end{equation}
Along a subsequence $n_a$ we have $\ddot\gamma^ {(n_a)}\rightarrow\ddot\gamma^*$ weakly in $L^2(U)$. Let $\varphi\in C^ \infty_0(U)$. Using Estimate \eqref{substiutionestimate}, we can compute 
\begin{align*} 
\int_U|\ddot\gamma^*|^2|\gamma_1^*|dt
&=\lim_{a\rightarrow\infty}\int_U|\ddot\gamma^ {(n_a)}|^2\gamma_1^*dt
=\lim_{a\rightarrow\infty}\int_U|\ddot\gamma^ {(n_a)}|^2\gamma_1^ {(n_a)}dt
\leq \sup_n\int_0^1 |\ddot\gamma^ {(n)}|^2\gamma_1^ {(n)}dt.
\end{align*}
Consequently $|\ddot\gamma^*|^2\gamma_1^*\in L^1(U)$ with norm bounded independent of $U$. Using $|E|=0$ and Equation \eqref{ArcLength_K_Identity}, we get 
$$\int_0^1(k_1^*)^22\pi\gamma_1^*|\dot\gamma^*|dt
=\sup\left\{
\frac{2\pi}{L^3_*} \int_U |\ddot\gamma^*|^2\gamma_1^* dt\ \bigg|\ U\subset[0,1],\ \inf_U\gamma_1^*>0\right\}<\infty.$$
In total we have shown $\gamma^*\in\mathcal P^ w$.\ \\
\ \\
\noindent
\textbf{Step 4 - Convergence in $\mathcal P^ w$}\ \\
Considering Definition \ref{convergencedefinition} and that $\gamma^{(n)}\rightarrow\gamma^*$ in $C^0([0,1])$ and $W^{1,2}((0,1))$ has already been established, we are left with proving that $\ddot\gamma^{(n)}\gamma_1^{(n)}\rightarrow\ddot\gamma^*\gamma_1^*$ weakly in $L^1((0,1))$. Let $\xi\in \mathcal P^ w$. By Theorem \ref{weakandstrongPconnection}, there exist $0=t_0<t_1<...<t_N=1$ such that $\xi_1(t)=0$ precisely when $t=t_i$ for some $i$. For small $\epsilon>0$, we have $\xi\in W^{2,2}(t_i+\epsilon, t_{i+1}-\epsilon)$. So, for $\phi\in C^ \infty([0,1])$, we have
$$\int_{t_i+\epsilon}^ {t_{i+1}-\epsilon}\ddot\xi\xi_1\phi dt
=
\dot\xi\xi_1\phi\bigg|_{t_i+\epsilon}^ {t_{i+1}-\epsilon}
-
\int_{t_i+\epsilon}^ {t_{i+1}-\epsilon}\dot\xi(\dot\xi_1\phi+\xi_1\dot\phi)dt.$$
We sum over $i$ and wish to take $\epsilon\rightarrow 0^+$. As $\xi\in\mathcal P^ w$ we have $\ddot\xi\xi_1\in L^1((0,1))$ by Lemma \ref{gamma1ddotgammainL1}. Using $\phi,\dot\xi\in L^ \infty((0,1))$ and $\xi_1(t_i)=0$, the boundary term vanishes for $\epsilon\rightarrow 0^ +$. So, we deduce 
$$\int_{0}^ {1}\ddot\xi\xi_1\phi
=
-
\int_{0}^ 1\dot\xi_1(\dot\xi_1\phi+\xi_1\dot\phi)dt.$$
As $\gamma^ {(n)}\rightarrow\gamma^*$ in $C^ 0([0,1])$ and in $W^ {1,2}((0,1))$ we can compute 
\begin{align}
    &\lim_{n\rightarrow\infty}\int_{0}^ {1}\ddot\gamma^ {(n)}\gamma_1^ {(n)}\phi dt
=-\lim_{n\rightarrow\infty }
\int_{0}^ 1\dot\gamma^ {(n)}(\dot\gamma_1^ {(n)}\phi+\gamma_1^ {(n)}\dot\phi)dt\nonumber\\
=&-
\int_{0}^ 1\dot\gamma^ {*}(\dot\gamma_1^ {*}\phi+\gamma_1^ {*}\dot\phi)dt=\int_{0}^ {1}\ddot\gamma^ {*}\gamma_1^ {*}\phi dt.\label{identityforsmoothtests}
\end{align}
Finally, we prove that Equation \eqref{identityforsmoothtests} is also valid for $\phi\in L^2((0,1))$.  Let $\phi\in L^2((0,1))$ and $\epsilon>0$. We choose $\phi_\epsilon\in C^ \infty([0,1])$ such that $\|\phi_\epsilon-\phi\|_{L^2((0,1))}\leq \epsilon$ and estimate 
\begin{equation}\label{phiapprox1}
\left|\int_0^1 \ddot\gamma^*\gamma_1^*(\phi-\phi_\epsilon) dt\right|
\leq \left(\int_0^1|\ddot\gamma^*|^2\gamma_1^*dt\right)^ {\frac12}\left(\sup\gamma_1^*\right)^ {\frac12}\|\phi-\phi_\epsilon\|_{L^2((0,1))}\leq C\epsilon.
\end{equation}
The last step is justified by Lemma \ref{gamma1ddotgammainL1}. Similarly, using Lemma \ref{gamma1ddotgammainL1} and the assumptions of Theorem \ref{compactnesstheorem} we estimate 
\begin{align}
&\limsup_{n\rightarrow\infty}\left|\int_0^1 \ddot\gamma^{(n)}\gamma_1^{(n)}(\phi -\phi_\epsilon)dt\right|\nonumber\\
\leq &\sup_{n\in\N}\left[\left(\int_0^1|\ddot\gamma^{(n)}|^2\gamma_1^{(n)}dt\right)^ {\frac12}\left(\sup\gamma_1^{(n)}\right)^ {\frac12}\right]\|\phi-\phi_\epsilon\|_{L^2((0,1))}\leq C\epsilon.\label{phiapprox2}
\end{align}
Combining Estimates \eqref{phiapprox1} and \eqref{phiapprox2} and letting $\epsilon\rightarrow0^+$ shows that Equation \eqref{identityforsmoothtests} is valid for $\phi\in L^2((0,1))$ also. 
\end{proof}
\renewcommand{\proofname}{\textit{Proof.}}

