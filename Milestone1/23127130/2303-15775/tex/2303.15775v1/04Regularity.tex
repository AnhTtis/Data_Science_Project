\section{Regularity of Minimizers}\label{regularityawayfromaxis}
In this section, we prove that the weak minimizers from Theorem \ref{weakminimizerexistence} are smooth.
\begin{theorem}\label{regularitytheorem1}
Let $\sigma\in(0,1]$ and $\gamma\in\mathcal F_\sigma$ such that $\beta(\sigma)=\mathcal W[\gamma]$. Then $\Sigma_\gamma$ is a smooth surface. 
\end{theorem}
For $\sigma=1$, we have already shown that the minimizers in $\mathcal P$ are round spheres. So it suffices to consider the case $\sigma\in(0,1)$. The proof is split into two parts. In the first step, we prove that the profile curve $\gamma$ is smooth on $(0,1)$. This shows that $\Sigma_\gamma$ is smooth away from the axis of rotation. In the second step, we consider the two intersection points $\gamma_2(0)e_3$ and $\gamma_2(1)e_3$ of $\Sigma_\gamma$ with the axis of rotation and show that here locally $\Sigma_\gamma$ is a smooth graph. 

\subsection{Euler-Lagrange Equation}
Our main tool for improving the regularity of $\Sigma_\gamma$ is a suitable Euler-Lagrange equation. However, deriving this is a little involved when we restrict ourselves to the class $\mathcal P$ due to the many restrictions imposed on curves $\gamma\in\mathcal P$. 
\begin{definition}[Generalized Profile Curves]
We say that $\gamma:[0,1]\rightarrow \mathcal H^2$ is a \emph{generalised profile curve} if
\begin{enumerate}[(1)]
    \item $\gamma\in C^1([0,1])\cap W^ {2,2}_{\operatorname{loc}}((0,1))$,
    \item $\gamma_1(0)=\gamma_1(1)=0$ and $\gamma_1(t)>0$ for all $t\in(0,1)$,
    \item $\Sigma_\gamma$ has curvature bounded in $L^2(\Sigma_\gamma)$: $\int_{\Sp^2} k_1^2+k_2^2d\mu_{\Sigma_\gamma}<\infty$. 
\end{enumerate}
We denote the set of all generalized profile curves by $\mathcal P'$. 
\end{definition}
Clearly $\mathcal P\subset\mathcal P'$. 
We consider the isoperimetric ratio functional
$$\mathcal I:\mathcal P'\rightarrow\R_0^+,\ \mathcal I[\gamma]:=6\sqrt\pi\frac{V[\gamma]}{A[\gamma]^{\frac32}}
\hspace{.5cm}\textrm{ and put }\hspace{.5cm}\
\mathcal F'_\sigma:=\set{\gamma\in\mathcal P'\ |\ \mathcal I[\gamma]=\sigma}.$$
As both the Willmore energy $\mathcal W$ and the isoperimetric ratio $\mathcal I$ are invariant under scaling and reparameterization, we have 
$$
\beta(\sigma)
=
\inf\left\{\mathcal W[\gamma]\ |\ \gamma\in\mathcal F_\sigma\right\} 
=
\inf\left\{\mathcal W[\gamma]\ |\ \gamma\in\mathcal F'_\sigma\right\} .
$$
Consequently, given a minimizer $\gamma\in\mathcal F_\sigma$ and a variation $\Phi:(-\epsilon_0,\epsilon_0)\rightarrow\mathcal F'_\sigma$ satisfying $\Phi(0)=\gamma$ we have 
\begin{equation}\label{eqeq0willmorevar}
0=\frac d{d\epsilon}\bigg|_{\epsilon=0}\mathcal W[\Phi(\epsilon)]=\delta\mathcal W[\gamma]\Phi'(0).
\end{equation}
\paragraph{Variation Formulas}\ \\
In the following, we require the variation of the isoperimetric ratio functional. Let $\gamma\in\mathcal P'$ with $V[\gamma]>0$ and $\phi\in C^\infty([0,1])$, not necessarily vanishing at $t=0,1$. Using the definition of $\mathcal I$ as well as Equations \eqref{areadef} and \eqref{volumedef} it is easy to see that
\begin{align}
    \delta\mathcal I[\gamma]\phi&=\frac{6\sqrt\pi}{A[\gamma]^{\frac32}}
    \left[
    \delta V[\gamma]\phi-\frac32\frac{V[\gamma]}{A[\gamma]}\delta A[\gamma]\phi
    \right],\label{var1}\\
    \delta V[\gamma]\phi&=\pm_\gamma\pi\int_0^1 \gamma_1^2(t)\dot\phi_2+2\gamma_1(t)\dot\gamma_2(t)\phi_1(t)dt,\label{var2}\\
\delta A[\gamma]\phi&=2\pi\int_0^1|\dot\gamma(t)|\phi_1(t)+\frac{\gamma_1(t)\langle\dot\phi(t),\dot\gamma(t)\rangle}{|\dot\gamma(t)|}dt.\label{var3}
\end{align}
The $\pm_\gamma$ in Equation \eqref{var2} is the sign of the integral in Equation \eqref{volumedef}. Now suppose that $\phi\in C^\infty_0((0,1))$. Then, as $\gamma\in W^{2,2}_{\operatorname{loc}}((0,1))$, we can integrate the integrals in Equations \eqref{var2} and \eqref{var3} by parts. Using $\nu$ from Equation \eqref{normalsprofilecurve} we get
\begin{align}
    \delta V[\gamma]\phi&=\mp_\gamma 2\pi\int_0^1 \langle \nu,\phi\rangle|\dot\gamma|\gamma_1 dt,\label{var4}\\
    \delta A[\gamma]\phi&=-2\pi\int_0^1 H[\gamma]\langle \nu,\phi\rangle |\dot\gamma|\gamma_1 dt.\label{var5}
\end{align}
The following lemma proves that the constraint of prescribed isoperimetric ration $\sigma\neq 1$ is non-degenerate.
\begin{lemma}\label{psi0existencelemma}
Let $\gamma\in\mathcal P$. Either $\Sigma_\gamma$ is a sphere or there exists $\psi_0\in C_0^\infty((0,1))$ such that 
$$\frac d{d\epsilon}\bigg|_{\epsilon=0}\mathcal I[\gamma+\epsilon\psi_0]=1.$$
\end{lemma}
\begin{proof}
If no $\phi\in C_0 ^ \infty((0,1))$ satisfying $\delta \mathcal I[\gamma]\phi\neq 0$ exists, then we can use Equations \eqref{var1}, \eqref{var4} and \eqref{var5} to get
$$\pm_\gamma2\pi\int_0^1 \langle \nu,\phi\rangle |\dot\gamma|\gamma_1 dt
=
2\pi\frac{3 V[\gamma]}{2A[\gamma]}\int_0^1 H[\gamma]\langle \nu,\phi\rangle |\dot\gamma|\gamma_1 dt \hspace{.5cm}\textrm{for all }\phi\in C^ \infty_0((0,1)).$$
By the fundamental lemma of the calculus of variations analysis $H\equiv \pm_\gamma\frac{2A[\gamma]}{2V[\gamma]}$ and by Lemma \ref{hopftypetheorem} -- the Hopf theorem -- $\Sigma_\gamma$ is a sphere.
\end{proof}

\subsection{Regularity Away from the Axis of Rotation}
Let $\sigma\in(0,1)$ and $\gamma\in\mathcal F_\sigma$ such that $\mathcal W[\gamma]=\beta(\sigma)$. We derive a suitable Euler-Lagrange equation for $\gamma$ on $(0,1)$. Using Lemma \ref{psi0existencelemma}, we choose $\psi_0\in C_0^\infty((0,1))$ such that $\delta\mathcal I[\gamma]\psi_0=1$. For any smooth $\eta\in C_0^\infty((0,1))$ we consider the vector field 
$$\tilde\eta:=\eta-\alpha(\eta)\psi_0\hspace{.5cm}\textrm{with}\hspace{.5cm}\alpha(\eta):=\delta\mathcal I[\gamma]\eta.$$
In Lemma \ref{variationexistence} we prove that for any such $\eta$ there exists a variation $\Phi_\eta:(-\epsilon_0,\epsilon_0)\rightarrow\mathcal F_\sigma'$ such that $\Phi_\eta'(0)=\tilde\eta$. So, by Equation \eqref{eqeq0willmorevar} we get 
\begin{equation}\label{weakeulerlagrangefirstappe}
0=\frac d{d\epsilon}\bigg|_{\epsilon=0}\mathcal W[\Phi_\eta(\epsilon)]=\delta\mathcal W[\gamma]\tilde\eta
\hspace{.5cm}\textrm{and hence}\hspace{.5cm}\delta\mathcal W[\gamma]\eta=\left(\delta\mathcal W[\gamma]\psi_0\right)\delta\mathcal I[\gamma]\eta.
\end{equation}

Using this Euler-Lagrange equation, we can now derive regularity of $\gamma$ as long as we \emph{`stay away from the axis of rotation'}. To do so, we require a general regularity result, which we have moved to Appendix \ref{GeneralRegularityAppendix}.

\begin{korollar}\label{gammasmooththeorem1}
    Let $\sigma\in(0,1)$, $\gamma\in\mathcal F_\sigma$ such that $\mathcal W[\gamma]=\beta(\sigma)$, $0<\tau_1<\tau_2<1$ and $U:=(\tau_1,\tau_2)$. Assume that $\kappa_1(\gamma;U)=\inf_U\gamma_1>0$
    and that there exists $\psi_0\in C^2_0(U)$ such that $\delta\mathcal I[\gamma]\psi_0=1$.
    Then $\gamma\in C^\infty(U)$ and for all $m\in\N$
    $$\|\gamma-\gamma(0)\|_{C^m((\tau_1,\tau_2))}\leq C(m,\kappa(\gamma;U),\|\psi_0\|_{C^2(U)}).$$
    The constant is decreasing in the second slot. 
\end{korollar}
\begin{proof}
    Without loss of generality, we assume $\gamma(0)=0$. We wish to apply Theorem \ref{GeneralRegularityTheorem}. To do so, we first establish an appropriate differential equation. Using Equation \eqref{weakeulerlagrangefirstappe}, we get
    $$\delta\mathcal W[\gamma]=\left(\delta\mathcal W[\gamma]\psi_0\right)\delta\mathcal I[\gamma](\phi)=I(\phi)\hspace{.5cm}\textrm{for all }\phi\in C^\infty_0(U,\R^2).$$
    Here $I(\phi)$ is an in Equation \eqref{IFUnctionalPDE} with parameters $\alpha$ and $\beta$ that, in view of Equations \eqref{var1}, \eqref{var2} and \eqref{var3}, are given as
    $$\alpha=\pm_\gamma\frac{6\sqrt\pi^3}{A[\gamma]^{\frac32}}\left(\delta\mathcal W[\gamma]\psi_0\right)
    \hspace{.5cm}\textrm{and}\hspace{.5cm}
    \beta=-\frac{3\pi}{A[\gamma]}\frac{6\sqrt\pi V[\gamma]}{A[\gamma]^{\frac32}}\left(\delta\mathcal W[\gamma]\psi_0\right).$$
    Next, we collect the appropriate bounds to satisfy assumption \eqref{uniformestimate}.  We have $\mathcal W[\gamma]=\beta(\sigma)\leq 8\pi$. Using $A[\gamma]=1$, Theorem \ref{choskiveneroniresult} and Estimate \eqref{inverseLbound}, we get $L[\gamma]+L[\gamma]^{-1}\leq C$. Combining the bound for $L[\gamma]$ with Lemma \ref{gamma1ddotgammainL1}, we have $\|\ddot\gamma\|_{L^2(U)}\leq C(\kappa(\gamma;U),U)$.
     Since $|\dot\gamma|=L[\gamma]\leq C$ and we assume $\gamma(0)=0$, we obtain $\|\gamma\|_{C^0}\leq C$.\\
     
     
    To estimate $\alpha$ and $\beta$ we recall $A[\gamma]=1$ and $V[\gamma]=\frac\sigma{6\sqrt\pi}$. It remains to estimate  $|\delta\mathcal W[\gamma]\psi_0|$. Considering Equations \eqref{principalcurvatures} and \eqref{willmoredef} it is easy to see that 
    $$|\delta\mathcal W[\gamma]\psi_0|\leq C(\|\gamma\|_{W^{1,\infty}(U)},\|\ddot\gamma\|_{L^2(U)},\kappa(\gamma;U))\|\psi_0\|_{C^2(U)}.$$
    In fact, we essentially establish this estimate in Step 2 in the proof of Lemma \ref{regularitystep1}.\\

    Summarising the above, we see that we can satisfy Estimate \eqref{uniformestimate} with a parameter $M$ that depends only on $U$, $\kappa(\gamma;U)$ and $\|\psi_0\|_{C^2(U)}$. So, Theorem \ref{GeneralRegularityTheorem} implies the corollary. 
\end{proof}

Combining Lemma \ref{psi0existencelemma}, $\gamma_1(t)>0$ for all $t\in(0,1)$ and Corollary \ref{gammasmooththeorem1}, we immediately get:

\begin{korollar}\label{gammasmooththeorem}
    Let $\sigma\in(0,1)$ and $\gamma\in\mathcal F_\sigma$ such that $\mathcal W[\gamma]=\beta(\sigma)$. Then $\gamma\in C^\infty((0,1))$. 
\end{korollar}






\subsection{Regularity At the Axis of Rotation}\label{RegularityAtAxisSubsection}
We continue to fix $\sigma\in(0,1)$ and $\gamma\in\mathcal F_\sigma$ such that $\mathcal W[\gamma]=\beta(\sigma)$. By Definition \ref{classPdef} and Corollary \ref{gammasmooththeorem} we have $\gamma\in C^\infty((0,1))\cap C^1([0,1])$. The main ingredient in the discussion of the regularity of $\Sigma_\gamma$ near the axis of rotation is the study of an appropriate Euler-Lagrange equation:

\begin{lemma}[Euler-Lagrange Equation]\label{eulerlagrangeequation}
There exists $\Lambda\in\R$ such that $\gamma$ satisfies the following Euler-Lagrange equation on $(0,1)$:
$$\Delta_g H+ \frac12H(H^2-4K)=\Lambda(\pm_\gamma4\sqrt\pi-\sigma H)$$
\end{lemma}
\begin{proof}
Let $\phi\in C^\infty_0((0,1))$ and $\psi_0$ be as in Lemma \ref{psi0existencelemma}. Then $\delta\mathcal W[\gamma](\phi-\alpha(\phi)\psi_0)=0$ by Equation \eqref{weakeulerlagrangefirstappe} . As $\phi$ is supported in $(0,1)$ where $\gamma$ is smooth, we can use Equation \eqref{variationformulastandard} for the variation of Willmore energy from the appendix to get 
$$\delta \mathcal W[\gamma]\phi=2\pi\int_0^1 W[\gamma]\langle\phi,\nu\rangle\gamma_1 |\dot\gamma|dt.$$
We put $\Lambda:=-3(\delta\mathcal W[\gamma]\psi_0)$ and recall from Equations \eqref{var1}, \eqref{var4} and \eqref{var5} that 
$$\alpha(\phi)=\delta\mathcal I[\gamma]\phi=-2\pi\sigma \int_0^1 \left(\pm_\gamma\frac1{V[\gamma]}-\frac32\frac{H}{A[\gamma]}\right)\langle\phi,\nu\rangle \gamma_1|\dot\gamma|ds.$$
Using $A[f]=1$, $V[f]=\frac\sigma{6\sqrt\pi}$, the explicit formula for the Willmore operator as well as the fundamental lemma of the calculus of variations, we get 
$$\frac12\left(\Delta_g H+\frac12H(H^2-4K)\right)=W[\gamma]=\frac13\Lambda \sigma \left(\pm_\gamma\frac{6\sqrt\pi}\sigma-\frac32 H\right).$$

\end{proof}
 To continue the investigation of the regularity of $\Sigma_\gamma$, we will now introduce a more suitable description of the surface $\Sigma_\gamma$ close to $(0,0,\gamma_2(0))$. 

\paragraph{Graph Representation near $(0,0,\gamma_2(0))$}\ \\
Near $(0,0,\gamma_2(0))$ we can write the surface $\Sigma_\gamma$ as a graph 
$$(r\cos\theta,r\sin\theta, u(r))\hspace{.5cm}\textrm{where}\hspace{.5cm} r\in[0, r_0),\ \theta\in[0,2\pi).$$
Indeed, putting $r(t):=\gamma_1(t)$ we have $r\in C^1([0,1])$ and $r'(0)=L[\gamma]\neq 0$. By the inverse function theorem, there exists $t_0>0$ such that $\gamma_1:[0, t_0]\rightarrow\gamma_1([0,t_0])$ is a $C^1$-diffeomorphism and we may define
$$u:[0, r_0)\rightarrow\R,\ u(r):=\gamma_2(\gamma_1^ {-1}(r))\hspace{.5cm}\textrm{where }r_0=\gamma_1(t_0).$$
Since $\gamma\in C^\infty((0, t_0])$, we get $u\in C^1([0, r_0])\cap C^ \infty((0, r_0])$. Additionally, $\dot\gamma_2(0)=0$ implies $u'(0)=0$.\\

Throughout the rest of this subsection, we fix $u$ as above.\\

We now wish to write the Euler-Lagrange equation from Lemma \ref{eulerlagrangeequation} in the graph parameterization. Using a result from \cite{dziuk2006error}, the authors of \cite{chenODE} derive an elegant formula for the Willmore operator. Letting $w:=u'(r)$ and $v:=\sqrt{1+w^2}$ we deduce that $w$ solves the following ODE on $(0, r_0)$:
\begin{equation}\label{ODE}
\frac1r\frac d{dr}\left[ r\frac1{v^ 5}\left(w''+\left(\frac{w}{r}\right)'-\varphi(w)\right)
\right]=W[\gamma]=
a- b H
\end{equation}
In view of Lemma \ref{eulerlagrangeequation} $a$ and $b$ are explicitly given by $a=\pm_\gamma4\sqrt\pi\Lambda$ and $b=\sigma\Lambda$. Furthermore, $\varphi(w)$ is computed in Lemma 2.2 in \cite{chenODE} and given by
\begin{equation}\label{varphiofwdef}
\varphi(w):=\frac{5w}{2(1+w^2)}(w')^2+\frac{w^ 3}{2r^2}(3+w^2).
\end{equation}
Note from Equation \eqref{graphH}, that $H$ is also of the form $r^{-1}\partial_r(...)$. This allows us to reduce the order of the ODE in Equation \eqref{ODE}. Indeed, multiplying Equation \eqref{ODE} by $r$, inserting Equation \eqref{graphH} for $H$, integrating and finally multiplying by $\frac{v^ 5}r$ shows that there exists $\lambda\in\R$ such that
\begin{equation}\label{ODEamRand}
\left\{
\begin{aligned}
&w''+\left(\frac{w}{r}\right)'-\varphi(w)
=
\frac12 v^ 5 a r+bv^ 4w+\frac\lambda rv^ 5,\\
&w(0)=0.
\end{aligned}
\right.
\end{equation}

\paragraph{Regularity of $u$}\ \\
$u$ satisfies Equation \eqref{ODEamRand} on $(0,r_0)$. We note that Equation \eqref{ODEamRand} allows for nonsmooth solutions. For example, the first remark in Section 3 in \cite{removability} discusses that the inverted catenoid solves \eqref{ODEamRand} on $(0, r_0)$ with $a=b=0$. It is also shown that the graph function $u_{ic}$ describing the inverted catenoid satisfies $u_{ic}\in C^ {1,\alpha}([0,r_0])$ for all $\alpha\in(0,1)$ and $u_{ic}''\in L^p((0, r_0))$ for all $p\in[1,\infty)$ but $u_{ic}\not\in C^{1,1}((0,r_0))$. In the next lemma, we prove that our solution $u$ possesses at least the same regularity.\\

The following lemma is essentially the same as Lemma 3.3 in \cite{chenODE}. Here the same analysis is carried out for the unconstrained Willmore equation. We show that the same argument can be applied when including the Lagrange multiplier in Equation \eqref{ODEamRand}.\\


\begin{lemma}\label{regularityatend01}
There exists $\xi\in C^1([0, r_0])$ such that $u'(r)=\frac12\lambda r\ln(r)+\xi(r)$. In particular, $u'\in C^{0,\alpha}([0,r_0])$ and $u''\in L^ p((0, r_0))$ for all $\alpha\in (0,1)$ and $p\in[1,\infty)$.
\end{lemma}
\begin{proof}
The proof is separated into six steps.\\

\noindent
\textbf{Step 1}\ \\
Note  $w\varphi(w)\geq 0$ by the definition of $\varphi(w)$ in Equation \eqref{varphiofwdef}.  We multiply Equation \eqref{ODEamRand} by $w$, choose any $\rho\in (0, r_0)$  and integrate:
$$\int_r ^ \rho w''w+\left(\frac wt\right)' w dt
\geq \frac a2 \int_ r^ \rho v^ 5 t w dt+
b\int_r^ \rho v^ 4 w^2 dt+\lambda \int_r ^ \rho \frac{v^ 5}{t} w dt$$
We already know that $u$ is $C^1([0, r_0))$ and satisfies $u'(0)=0$. Hence $v$ and $w$ are bounded and we get the estimate  
$$\int_r ^ \rho w''w+\left(\frac wt\right)' w dt
\geq 
-C\left(1+\int_r^ \rho\frac{|w|}tdt\right).$$
$C$ depends on $\sigma$, $\lambda$ and on the Lagrange multiplier. We will not make this dependence explicit in the notation. Integrating by parts gives 
$$-w'(r)w(r)-\frac{w(r)^2}r-\int_r^ \rho (w'(t))^2+\frac{ ww'}{t}dt
\geq 
-C(\rho)\left(1+\int_r^ \rho\frac{|w|}tdt\right).$$
We have included the boundary terms at $t=\rho$ in the constant $C(\rho)$. Note that $C(\rho)<\infty$ for all $\rho\in(0, r_0)$ as $w\in C^\infty((0, r_0))$. Next, we integrate by parts again and get
$$-w'(r)w(r)-\frac{w(r)^2}r+\frac{w(r)^2}{2r}-\int_r^ \rho (w'(t))^2+\frac{ w(t)^2}{2t^2}dt
\geq 
-C(\rho)\left(1+\int_r^ \rho\frac{|w|}tdt\right).$$
Rearranging this estimate gives 
\begin{align}
\int_r^ \rho (w'(t))^2+\frac{ w(t)^2}{2t^2}dt
&\leq C(\rho)\left(1+\int_r^ \rho\frac{|w|}tdt\right)+\frac{w(r)^2}{2r}-w(r)\frac{w(r) +rw'(r)}r\nonumber\\
&= C(\rho)\left(1+\int_r^ \rho\frac{|w|}tdt\right)+\frac{w(r)^2}{2r}-w(r)\frac{(r w(r))'}r.\label{step1result}
\end{align}

%-----------------------
\noindent
\textbf{Step 2}\ \\
Let again $\rho\in(0,r_0)$. This time we integrate Equation \eqref{ODEamRand} directly and use $w'+\frac wr=\frac1r(wr)'$ to get 
\begin{equation}\label{step2begin}
\frac {(w t)'}t\bigg|_{t=\rho}-\frac {(w r)'}r=\int_ r^ \rho \varphi(w)+\lambda\frac{ v^ 5-1}t+\frac12 a v^ 5 t + b v^ 4wdt-\lambda\ln\frac r\rho.
\end{equation}
Let $\epsilon>0$. Since $w=u'\in C^0([0, r_0))$ and $w(0)=0$, we can choose $\rho(\epsilon)$ small enough such that $|w(r)|\leq \epsilon$ for all $r\in[0,\rho(\epsilon)]$. For $\rho\leq\rho(\epsilon)$ we get 
\begin{align*}
    \left|\frac{(r w(r))'}{r}\right| 
    &\leq C(\epsilon)+\left|\lambda\ln\frac r\rho\right|+\int_r^ \rho
    \frac{5 |w(t)| w'(t)^2}{2(1+w(t)^2)}+w(t)^2\frac{3|w(t)|+|w(t)|^3}{2 t^2}+|\lambda|\left|\frac{v^ 5-1}{t}\right|dt\\
    &\hspace{.5cm}+C\int_r^ \rho t v^ 5+ v^ 4|w| dt.
\end{align*}
The constant $C(\epsilon)$ is due to the boundary terms at $\rho(\epsilon)$. The second constant $C$ does not depend on $\epsilon$. Using $u'(0)=w(0)=0$, we can shrink $\rho(\epsilon)$ to achieve
\begin{align}
    \left|\frac{(r w(r))'}{r}\right| 
    &\leq C(\epsilon)\ln\frac1r+\epsilon \int_r^ \rho w'(t)^2 +\frac{w(t)^2}{2t^2} dt\hspace{.5cm}\textrm{for all }r\in(0, \rho(\epsilon)).\label{step2result}
\end{align}

%-------------------------
\noindent
\textbf{Step 3}\ \\
Let $\epsilon>0$ be arbitrary. By taking $\rho=\rho(\epsilon)$ small enough in Estimate \eqref{step1result} we may insert Estimate \eqref{step2result} to get 
\begin{align*} 
\int_ r^ {\rho}w'(t)^2+\frac{w(t)^2}{2t^2} dt&\leq C(\epsilon)\left(1+\int_r^ \rho\frac{|w|}{t}dt\right)+\frac{w(r)^2}{2r}\\
&\hspace{1cm}+|w(r)|\left(C(\epsilon)\ln\frac1r+\epsilon \int_r^ \rho w'(t)^2 +\frac{w(t)^2}{2t^2} dt\right).
\end{align*}
By potentially shrinking $\rho$ we can assume without loss of generality that $|w|\leq \epsilon$ on $[0,\rho]$ and so, in particular $|w(r)|\leq \epsilon$. Choosing $\epsilon$ small enough, we can absorb the second integral on the right to the left and obtain that for some small enough $\rho$ and $r\in(0,\rho)$
\begin{equation}\label{step3result}
\int_r^ {\rho}w'(t)^2+\frac{w(t)^2}{2t^2} dt
\leq
C(\rho)\left(\frac{w(r)^2}{2r}+|w(r)|\ln\frac1r+\int_r^ \rho\frac{|w(t)|}tdt+1\right).
\end{equation}

%--------------------------
\noindent
\textbf{Step 4}\ \\
Let $\rho$ be small enough so that Estimate \eqref{step3result} holds. Inserting Estimate \eqref{step3result} into Estimate \eqref{step2result} with $\epsilon=1$ and using that $|w|$ is bounded, we get
\begin{align*} 
\left|\frac{(r w(r))'}{r}\right| 
    \leq &
    C\ln\frac1r+C\left(\frac{w(r)^2}{2r}+|w(r)|\ln\frac1r+\int_r^ \rho\frac{|w(t)|}tdt+1\right) \\
    \leq &
    C\ln\frac1r+C\left(\frac{w(r)^2}{2r}+\int_r^ \rho\frac{1}tdt\right).
\end{align*}
We multiply by $r$. Clearly $|\ln(\rho)-\ln(r)|\leq C\ln(\frac1r)$ as $r\leq \rho$. Let $\mu\in(0,1)$. By potentially shrinking $\rho$ we can assume that $|w|\leq C^ {-1}\mu$ on $[0,\rho]$. Hence 
$$|(r w)'|\leq C(w(r)^2+r\ln\frac1r)\leq \mu |w|+Cr\ln\frac1r.$$ 
In general, for any weakly differentiable $g$ we have $|g|'\leq |g'|$. As $|rw|=r|w|$ we get 
$$r|w|'+|w|=(r|w|)'=|rw|'\leq |(rw)'|\leq \mu|w|+Cr\ln\frac1r \hspace{.5cm}\textrm{for all $r\in(0,\rho]$}.$$
Multiplying with $r^ {1-\mu}$ we get 
$$(r^ {1-\mu} |w|)'=r^ {1-\mu}|w|'+(1-\mu) r^ {-\mu}|w|\leq C r^ {1-\mu} \ln\frac1r.$$
We integrate this inequality from $r_1<r$ to $r$ and subsequently send $r_1\rightarrow 0^+$. Since $w\in C^0([0,r_0))$, we get
$$r^ {1-\mu}|w(r)|\leq C\int_0^ r t^ {1-\mu}\ln\frac1 t dt\leq C(\mu) r^{2-\mu}\ln\frac1r.$$
Taking e.g. $\mu=\frac12$ shows that there exists $\rho_0>0$ such that
\begin{equation}\label{step4result}
|w(r)|\leq Cr\ln\frac1r\hspace{.5cm}\textrm{for all }0<r<\rho_0.
\end{equation}

%------------------------
\noindent
\textbf{Step 5}\ \\
 Using Estimate \eqref{step4result} we deduce the following three estimates for $r\in(0,\rho_0)$:
\begin{align}
    \frac{w(r)^2}r&\leq C\frac{r^ 2\ln^2\frac1r}{r}\leq Cr\ln^2\frac1r\label{step5estimate1}\\
    |w(r)|\ln\frac1r&\leq Cr\ln^2\frac1r\label{step5estimate2}\\
    \int_0^ r\frac{|w(t)|}tdt&\leq C\int_0^  r \ln\frac1t dt\leq C\label{step5estimate3}
\end{align}
After potentially shrinking $\rho_0$, we may use Estimate \eqref{step3result}. Inserting Estimates \eqref{step5estimate1}-\eqref{step5estimate3} into Estimate \eqref{step3result}, we get
\begin{equation}\label{step5ausgangspunkt}
\int_r^ {\rho_0}w'(t)^2+\frac{w(t)^2}{2t^2} dt
\leq C.
\end{equation}
Potentially shrinking $\rho_0$ even further, we may assume $|w|\leq\frac12$ and, recalling that $v=\sqrt{1+w^2}$, we get
\begin{equation}\label{step5helpestimates}
\left|\frac{5w}{2(1+w^2)}(w')^2\right|
\leq C|w'|^2,
\hspace{.5cm}
\left|\frac{3w^3+w^ 5}{2t^2}\right|
\leq 
C\frac{w^2}{t^2}
\hspace{.5cm}\textrm{and}\hspace{.5cm}
\left|\frac{v^5-1}t\right|\leq C\frac{w^2}t.
\end{equation}
%----
Combining Estimates \eqref{step5ausgangspunkt} and \eqref{step5helpestimates}, we obtain
\begin{equation}\label{step5result}
\int_0^ {\rho_0}\left|
\frac{5w}{2(1+w^2)}(w')^2
+
\frac{3w^3+w^ 5}{2t^2}
+\lambda\frac{v^ 5-1}t
\right|dt<\infty.
\end{equation}

%----------------------
\noindent
\textbf{Step 6}\ \\
We choose $\rho$ small enough, such that the results from all previous steps are valid. Considering the integral appearing in Equation \eqref{step2begin} from Step 2 we introduce
$$\psi(r):=-\int_ r^ \rho \varphi(w)+\lambda\frac{ v^ 5-1}t+\frac12 a v^ 5 t + b v^ 4wdt.$$
Recalling the definition of $\varphi(w)$ from Equation \eqref{varphiofwdef} and using Estimate \eqref{step5result} as well as $v\in C^ 0([0,\rho])$ we deduce $\psi\in C^0([0,\rho])$. Rewriting Equation \eqref{step2begin} we have shown that there exists a constant $\tilde c_0\in\R$ such that
$$(w(r) r)'=r\psi(r)+\tilde c_0+\lambda r\ln(r).$$
This implies that there exist constants $c_0, c_1\in\R$ such that
$$w(r)r=\int_0 ^r t\psi(t)dt +c_0 r+c_1 +\frac12 \lambda r^2 \ln(r)-\frac14\lambda r^2.$$
Letting $r\rightarrow 0^+$ we get that $c_1=0$ so that
$$w(r)-\frac12 \lambda r\ln(r)-c_0+\frac14\lambda r=\frac1r\int_0 ^r t\psi(t)dt=:h(r).$$
It remains to prove that $h\in C^1([0,\rho])$. First note, that $\psi\in C^0([0,\rho])$ is bounded. Hence $h$ is continuous at $r=0$ with $h(0)=0$. Indeed
$$|h(r)|\leq \frac Cr\int_0 ^r tdt\leq Cr\rightarrow 0,\hspace{.5cm}\textrm{as $r\rightarrow 0^+$}.$$
On $(0,\rho]$ we can differentiate the definition of $h$ to derive the equation $\frac{(r h(r))'}{r}=\psi(r)$. 
As $\psi\in C^0([0,\rho])$ we may use Lemma \ref{oderegularitylemma} from the appendix and get $h\in C^1([0,\rho])$. Finally, as $u$ is smooth on $[\rho, r_0]$, the lemma follows. 
\end{proof}

The next step is to derive an expansion of the mean curvature $H$.

\begin{lemma}\label{HExpansionLemma}
There exists a function $H_0(r)\in C^ {1,\alpha}([0, r_0])$ for all $\alpha\in(0,1)$ such that $H(r)=-\lambda\ln(r)+H_0(r)$.
\end{lemma}
\begin{proof}
Writing out the Willmore operator in Equation \eqref{ODE} in terms of $H$, we get 
$$\Delta_g H+ \frac12H(H^2-4K)=a-bH.$$
We put $f(r):=a-bH-\frac12 H(H^2-4K)$. Using Equations \eqref{graphmetric} and \eqref{graphH} we may write
$$\frac1{r \sqrt{1+u'(r)^2}}\frac{d}{dr}\left[\frac{r}{\sqrt{1+u'(r)^2}}H'(r)\right]
=\Delta_g H(r)
=f(r).$$
Lemma \ref{regularityatend01} gives $u'(r)=\frac12\lambda r\ln(r)+\xi$ with $\xi\in C^1([0, r_0])$. As $u'(0)=0$ we have $\xi(0)=0$. Using Equations \eqref{graphmetric} and \eqref{graphicalPrincipalCurvatures} allows us to estimate both $|H(r)|\leq C\ln\frac1r$ and $|K(r)|\leq C\ln^2\frac1r$. Therefore $|f(r)|\leq C\ln^3\frac1r$ and for all $p\in[1,\infty)$
\begin{equation}\label{docheinname}
\frac{d}{dr}\left[\frac r{\sqrt{1+u'(r)^2}}H'(r)\right]=r \sqrt{1+u'(r)^2} f(r)=:z(r)\in L^p((0, r_0)).
\end{equation}
Since $u\in C^ \infty((0,r_0])\cap C^1([0,r_0])$, $|H|\leq C\ln\frac1r$ and $|K|\leq C\ln^2\frac1r$ we get $z\in C^ \infty((0,r_0])\cap C^ 0([0,r_0])$. Integrating Equation \eqref{docheinname} implies that there exists $c_0\in\R$ such that  
$$H'(r)=\frac{c_0}{r}\sqrt{1+u'(r)^2}+\frac{\sqrt{1+u'(r)^2}}r\int_0^r z(s) ds.$$
We put $H_0(r):=H(r)-c_0\ln(r)$. Clearly $H_0\in C^ \infty((0,r_0])$ and we can compute 
\begin{align*}
    H_0''(r)=&\frac d{dr}\left[c_0\frac{\sqrt{1+u'(r)^2}-1}r+\frac{\sqrt{1+u'(r)^2}}r\int_0 ^r z(s)ds\right]\\
    =&-c_0\frac{\sqrt{1+u'(r)^2}-1}{r^2}+c_0\frac{u'(r) u''(r)}{r\sqrt{1+u'(r)^2}}
    -\frac{\sqrt{1+u'(r)^2}}{r^2}\int_0 ^r
    z(s)ds
    \\
    &+\frac{u'(r)u''(r)}{r\sqrt{1+u'(r)^2}}\int_0 ^r z(s)ds
    +\frac{\sqrt{1+u'(r)^2}}rz(r).
\end{align*}
We have $|u'(r)|\leq Cr\ln\frac1r$ and $|u''(r)|\leq C\ln\frac1r$. Also $|z(r)|\leq Cr\ln^3\frac1r$. This gives 
$$
|H_0''(r)|\leq C\left[\ln^2\frac1r+\frac 1{r^2}\int_0^r s\ln^3\frac1sds+\ln^2\frac1r\int_0^r s\ln^3\frac1sds+\ln^3\frac1r\right]\leq C\ln^3\frac1r.
$$
Hence $|H_0''(r)|\leq C\ln^3\frac1r\in L^p((0, r_0))$ for all $p\in[1,\infty)$. Using Lemma \ref{derivativeinL1} we get $H_0\in C^1([0,r_0])$ and thus $H_0\in W^ {2,p}((0,r_0))$ for all $p<\infty$. By the Sobolev embedding theorem we deduce $H_0\in C^{1,\alpha}([0,r_0])$ for all $\alpha\in(0,1)$. Recalling the definition of $H_0$, this gives
\begin{equation}\label{Hexpansionversion0}
H(r)=c_0\ln(r)+H_0(r)\hspace{.5cm}\textrm{with $H_0\in C^ {1,\alpha}([0, r_0])$}.
\end{equation}
It remains to check that $c_0=-\lambda$. We recall $u'(r)=\frac\lambda2 r\ln(r)+\xi(r)$ where $\xi\in C^1([0,r_0])$. In particular $u'\in C^0([0,r_0])$. Using $u'(0)=0$ and Equation \eqref{graphicalPrincipalCurvatures} we compute 
$$ \frac{H(r)}{\ln(r)}=-\frac{u''(r)}{\ln(r)}\frac1{\sqrt{1+u'(r)^2}^3}-\frac{u'(r)}{r\ln(r)}\frac1{\sqrt{1+u'(r)^2}}
\rightarrow-\lambda\hspace{.5cm}\textrm{as }r\rightarrow0^+.$$
In view of Equation \eqref{Hexpansionversion0}, we obtain $c_0=-\lambda$.
\end{proof}

\paragraph{Higher Regularity}\ \\
In the next lemma, we prove that the parameter $\lambda$ from Equation \eqref{ODEamRand} vanishes. In view of Lemma \ref{regularityatend01} we then get $u\in C^2([0,r_0])$. The key ingredient that allows this further improvement of regularity is the fact that the surface $\Sigma_\gamma$ minimizes the Willmore energy inside the class $\mathcal F_\sigma$.\\ 

Recall that $\gamma_1$ is a $C^1$-diffeomorphism from $[0,t_0]$ onto $[0,r_0]$ and that $u:[0,r_0]\rightarrow\R,\ u(r):=\gamma_2(\gamma_1^{-1}(r))$. We put $r_1:=\frac{r_0}2$, $t_1:=\gamma_1^{-1}(r_1)\in(0, t_0)$ and choose $\chi\in C^\infty([0,1])$ such that $\chi\equiv 1$ on $[0, t_1]$ and $\chi\equiv 0$ on $[\frac{t_0+t_1}2,1]$. For $\varphi\in C^\infty_0([0,r_1))$ satisfying $\varphi'(0)=0$ we consider the variation
$$
\Phi:(-\epsilon_0,\epsilon_0)\times[0,1]\rightarrow\mathcal H^2,\ \Phi(\epsilon, t):=\gamma(t)+\epsilon\chi(t)\varphi(\gamma_1(t))e_2\hspace{.5cm}\textrm{and put}\hspace{.5cm}
\phi(t):=\frac{\partial\Phi}{\partial\epsilon}\bigg|_{\epsilon=0}.$$
We note that $\Phi(\epsilon, t)\equiv \gamma(t)$ for $t\geq t_1$ and $\Phi(\epsilon,\cdot)\in\mathcal P'$ for all small $\epsilon$. However, the variation $\Phi$ does not preserve the isoperimetric ratio. Employing the function $\psi_0\in C^\infty_0((0,1))$ from Lemma \ref{psi0existencelemma} we may define an admissible variational vector field by putting 
\begin{equation}\label{Vdef}
\eta:=\phi-\alpha(\phi)\psi_0
\hspace{.5cm}\textrm{with}\hspace{.5cm}
\alpha(\phi)=\delta\mathcal I[\gamma]\phi.
\end{equation}

We require the following lemma, 
\begin{lemma}\label{eulerlagangeataxis}
There exists a variation $\tilde\Phi:(-\epsilon_0,\epsilon_0)\rightarrow\mathcal F_\sigma'$ such that $\tilde\Phi'(0)=\eta$. In particular $\delta \mathcal W[\gamma]\eta=0$ by Equation \eqref{weakeulerlagrangefirstappe}. Additionally 
$$0=\delta \mathcal I[\gamma] \eta=-3\pi\lim_{\tau\rightarrow0^+}\int_\tau^1\left( \pm_\gamma 4\sqrt\pi-\sigma H[\gamma]\right)\langle\nu,\eta\rangle L\gamma_1 dt.$$
\end{lemma}
\begin{proof}
    The existence of $\tilde\Phi$ is established in Appendix \ref{variationexistenceappendix}. Here we only establish the formula. Note that $\eta=\chi\cdot(\varphi\circ\gamma_1)e_2-\alpha(\phi)\psi_0\in C^1([0,1])$ since $\gamma\in C^1([0,1])$. Using Equation \eqref{var3}, $\gamma_1(0)=0$ and the fact that $\eta$ vanishes at $1$, we compute 
    \begin{align}
    \delta A[\gamma]\eta=&2\pi\lim_{\tau\rightarrow0^+}\int_\tau^1L\eta_1+\frac{\gamma_1(t)\langle\dot \eta(t),\dot\gamma(t)\rangle}{L}dt\nonumber\\
                        =&-2\pi\lim_{\tau\rightarrow0^+}\left[\frac{\gamma_1(\tau)\langle \eta(\tau), \dot\gamma(\tau)\rangle}L
                            +\int_\tau^1 H[\gamma]\langle n,\eta\rangle L\gamma_1(t)dt\right]\nonumber\\
                        =&-2\pi\lim_{\tau\rightarrow0^+}\int_\tau^1 H[\gamma]\langle \nu,\eta\rangle L\gamma_1(t)dt.\label{uberlegung1}
\end{align}
Using Equation \eqref{var2}, $\gamma_1(0)=0$, $\eta\equiv 0$ near $1$ and the regularity of $\gamma$ we write 
\begin{align}
    \delta V[\gamma]\eta=&\mp_\gamma2\pi\int_0^1 \langle\nu,\eta\rangle L\gamma_1(t) dt\mp_\gamma\pi\gamma_1^2(0)\eta_2(0)\nonumber\\
                        =&\mp_\gamma2\pi\lim_{\tau\rightarrow 0^+}\int_\tau^1 \langle\nu,\eta\rangle L\gamma_1(t) dt.\label{uberlegung2}
\end{align}
Inserting Equations \eqref{uberlegung1} and \eqref{uberlegung2} into Equation \eqref{var1} and using $A[\gamma]=1$ and $V[\gamma]=\frac{\sigma}{6\sqrt\pi}$, we get
\begin{align}
    \delta\mathcal I[\gamma]\eta=&\frac{6\sqrt\pi}{A[\gamma]^{\frac32}}\left[\delta V[\gamma]\eta-\frac{3 V[\gamma]}{2 A[\gamma]}\delta A[\gamma]\eta\right]\nonumber\\
     %----------           
        =&-6\sqrt\pi\lim_{\tau\rightarrow0^+}2\pi\int_\tau^1\left( \pm_\gamma 1-\frac{\sigma}{4\sqrt\pi}H[\gamma]\right)\langle\nu,\eta\rangle L\gamma_1 dt.\label{isoperimetricepsiloin}
\end{align}
\end{proof}

\newpage
\begin{lemma}\label{uisc2}
$\lambda=0$ and consequently by Lemmas \ref{regularityatend01} and \ref{HExpansionLemma}, $u\in C^2([0,r_0])$ and $H\in C^1([0,r_0])$.
\end{lemma}

\begin{proof}
Throughout the proof, we use both the profile curve and the graph function parameterization. To shorten the notation, we denote the mean curvature by $\tilde H$ when working in the graph function parameterization and by $ H$ when using the profile curve parameterization. That is: $\tilde H(r)=\tilde H(\gamma_1(s))=H(s)$.\\

Let $\varphi\in C^\infty_0([0, r_1))$ such that $\varphi'(0)=0$ and define $\eta$ as in Equation \eqref{Vdef}. By Lemma \ref{eulerlagangeataxis} there exists a variation $\tilde\Phi:(-\epsilon_0,\epsilon_0)\rightarrow\mathcal F_\sigma'$ such that $\frac{\partial\tilde\Phi(\epsilon)}{\partial\epsilon}\big|_{\epsilon=0}=\eta$ and we get the Euler-Lagrange equation 
\begin{equation}\label{weakeulerlagrangelambda}
0=\delta\mathcal W[\gamma]\eta=\delta \mathcal W[\gamma]\phi-\alpha(\phi)\delta\mathcal W[\gamma]\psi_0.
\end{equation}


\paragraph{The first term}\ \\
$\phi$ is supported on $[0, t_1]$. Here $\chi\equiv 1$ and hence
\begin{align*}
    \delta\mathcal W[\gamma]\phi&=\frac d{d\epsilon}\bigg|_{\epsilon=0}\mathcal W[\gamma+\epsilon\phi]=\frac{2\pi}4 \frac d{d\epsilon}\bigg|_{\epsilon=0}\int_0^{r_1} \tilde H[\gamma+\epsilon\phi]^2 \sqrt{1+(u'+\epsilon\varphi')^2}rdr\\
   =& \frac{2\pi}4 \int_0^{r_1}
   \left(
   2 \tilde H(r)\delta \tilde  H[\gamma]\phi\sqrt{1+u'(r)^2}+ \tilde  H(r)^2 \frac{u'(r)\varphi'(r)}{\sqrt{1+u'(r)^2}} 
   \right)rdr.
\end{align*}
Let us consider $\delta  \tilde H[\gamma]\phi$. By Lemma \ref{regularityatend01} there exists $\xi\in C^1([0,r_1])$ such that $u'(r)=\frac\lambda 2r\ln(r)+\xi(r)$. Since $u'(0)=0$ we have $\xi(0)=0$. Also $u''(r)=\frac\lambda2\ln(r)+\tilde\xi(r)$ where $\tilde\xi=\xi'+\frac\lambda2$ is continuous on $[0,r_0]$. Using Equation \eqref{graphicalPrincipalCurvatures} shows that for all $p\in[1,\infty)$
$$\delta \tilde H[\gamma]\phi=
-\frac{\varphi''(r)}{\sqrt{1+u'(r)^2}^3}
+3\frac{u''(r)u'(r)\varphi'(r)}{\sqrt{1+u'(r)^2}^5}
-\frac{\varphi'(r)}{r\sqrt{1+u'(r)^2}}
+\frac{u'(r)^2\varphi'(r)}{r\sqrt{1+u'(r)^2}^3}
\in L^p((0,r_1)).$$
Additionally, $ \tilde H\in L^p((0, r_0))$ for all $p\in[1,\infty)$ by Lemma \ref{HExpansionLemma}. Let $\rho>0$ be a small parameter. We put $\gamma_\rho:=\gamma|_{[\gamma_1^{-1}(\rho), \gamma_1^{-1}(r_1)]}$. Using Lebesgue's theorem, we get
\begin{align}
    \delta\mathcal W[\gamma]\phi
   &= \lim_{\rho\rightarrow0^+}\frac{2\pi}4 \int_\rho ^{r_1}  \left(2 \tilde H(r)\delta \tilde H[\gamma]\phi
   \sqrt{1+u'(r)^2}+  \tilde H(r)^2 \frac{u'(r)\varphi'(r)}{\sqrt{1+u'(r)^2}} \right) rdr\nonumber\\
   &= \lim_{\rho\rightarrow0^+}\delta\mathcal W[\gamma_\rho]\phi.\label{limitfirstterm}
\end{align}
Since $\gamma\in C^\infty((0,1))$ we have  $\gamma_\rho\in C^\infty([\gamma_1^{-1}(\rho),\gamma_1^{-1}(r_1)])$ for all small $\rho>0$. Using Equation \eqref{variationformulastandard} from the appendix and using that $\phi$ is compactly supported in $[0,\gamma_1^{-1}(r_1))$, we get
\begin{align}
\delta\mathcal W[\gamma_\rho]\phi=2\pi\int_{\gamma_1^{-1}(\rho)}^{\gamma_1^{-1}(r_1)} W[\gamma]\langle \phi, \nu\rangle \gamma_1 |\dot\gamma|dt
+B(\rho)\nonumber\\
=2\pi\int_{\gamma_1^{-1}(\rho)}^{1} W[\gamma]\langle \phi, \nu\rangle \gamma_1 |\dot\gamma|dt
+B(\rho).
\label{firsttermresult}
\end{align}
Here we have introduced the boundary term 
\begin{equation}\label{Bdefinition}
B(\rho)=
\pi\left[
\frac{\varphi(\gamma_1(s))   H'(s)-(\varphi\circ\gamma_1)'(s) H(s)
-\frac12H^2(s)\langle \dot\gamma(s),\phi(s)\rangle}
{|\dot\gamma(s)|}
\right]\gamma_1(s)\bigg|_{s=\gamma_1^{-1}(\rho)}.
\end{equation}



\paragraph{The second term}\ \\
As $\psi_0\in C^\infty_0((0,1))$, its support lies in an interval $(a,b)$ with $0<a<b<1$ where $\gamma\in C^\infty([a,b])$. We use Equation \eqref{variationformulastandard} from the appendix. As $\psi_0$ has compact support in $(a,b)$ we get
\begin{equation}\label{secondtermresult}
\delta\mathcal W[\gamma]\psi_0=2\pi\int_a^bW[\gamma]\langle \psi_0, \nu\rangle \gamma_1|\dot\gamma|ds
=2\pi\lim_{\rho\rightarrow 0^+}\int_{\gamma_1^{-1}(\rho)}^1W[\gamma]\langle \psi_0, \nu\rangle \gamma_1|\dot\gamma|ds.
\end{equation}


\paragraph{Combining the results}\ \\
By Lemma \ref{eulerlagrangeequation}, $\gamma$ satisfies the Euler-Lagrange equation $W[\gamma]=\Lambda (\pm_\gamma4\sqrt\pi-\sigma H[\gamma])$ on $(0,1)$. Using Lemma \ref{eulerlagangeataxis}, we get
 \begin{align}
     &\lim_{\rho\rightarrow0^+}\left[2\pi\int_{\gamma_1^{-1}(\rho)}^1W[\gamma]\langle \eta, \nu\rangle \gamma_1|\dot\gamma|ds\right]\nonumber\\
    =&
    \lim_{\rho\rightarrow0^+}\left[2\pi\Lambda \int_{\gamma_1^{-1}(\rho)}^1(\pm_\gamma4\sqrt\pi-\sigma H[\gamma])\langle \eta, \nu\rangle \gamma_1|\dot\gamma|ds\right]\nonumber\\
   =&
    -\frac{2\Lambda}3\delta\mathcal I[\gamma]\eta
    =0.\label{surfaceintegralscombined}
 \end{align}
Combining Equations \eqref{weakeulerlagrangelambda}, \eqref{limitfirstterm}, \eqref{firsttermresult},  \eqref{secondtermresult} and  \eqref{surfaceintegralscombined} we deduce 
$
\lim_{\rho\rightarrow0^+}B(\rho)=~0.
$
We now rewrite the boundary term in terms of the graph function $u$. Note 
$$
r=\gamma_1(s)
\hspace{.5cm}\textrm{and consequently}\hspace{.5cm}
\frac\partial{\partial s}=\frac{\partial r}{\partial s}\frac\partial{\partial r}=\dot\gamma_1(s)\frac{\partial}{\partial r}.$$
Hence $\dot\gamma_2(s)=\dot\gamma_1(s)(\gamma_2\circ\gamma_1^{-1})'(r)=\dot\gamma_1(s) u'(r)$ and therefore $|\dot\gamma(s)|=\dot\gamma_1(s)\sqrt{1+u'(r)^2}$. Inserting into Equation \eqref{Bdefinition}, we get
$$B(\rho)=\pi r\frac{\varphi(r) \tilde H'(r)-\varphi'(r) \tilde H(r)-\frac12 \tilde H^2(r)u'(r)\varphi(r)}{\sqrt{1+u'(r)^2}}\bigg|_{r=\rho}.$$
Using the formulas $u'(r)=\frac\lambda2 r\ln(r)+\xi(r)$ where $\xi\in C^1([0, r_0])$ satisfies $\xi(0)=0$ from Lemma \ref{regularityatend01}, $ \tilde H(r)=-\lambda\ln(r)+ \tilde H_0(r)$ with $ \tilde H_0\in C^{1,\alpha}([0,r_0))$ from Lemma \ref{HExpansionLemma}, the smoothness of $\varphi$ and $\varphi'(0)=0$, we get 
$$0=\lim_{\rho\rightarrow 0^+} B(\rho)=\lim_{\rho\rightarrow0^+}\left[\pi\rho\frac{\varphi(\rho)}{\sqrt{1+u'(\rho)^2}}\left(-\frac\lambda\rho +\tilde H_0'(\rho)\right)\right]=-\pi\lambda\varphi(0).$$
As $\varphi(0)$ is arbitrary we deduce $\lambda=0$. 
\end{proof}







Having improved the regularity of $u$ to $C^2$, we can now easily deduce that $u$ is, in fact, smooth.
\begin{lemma}\label{uissmooth}
$u\in C^\infty([0,r_0))$. 
\end{lemma}
\begin{proof}
The proof is inductive. Let $k\geq 0$ and suppose that we know $H\in C^k([0,r_0))$. By Equation \eqref{graphH} 
$$-\frac1r\frac{d}{dr}\left[r\frac{u'(r)}{\sqrt{1+u'(r)^2}}\right]=H\in C^k([0,r_0)).$$
Lemma \ref{oderegularitylemma} implies $u'\in C^{k+1}([0,r_0))$ and thus $u\in C^{k+2}([0,r_0))$. In view of Equation \eqref{graphicalPrincipalCurvatures} we get $k_1\in C^k([0, r_0))$ and since $u'(0)=0$ 
$$k_2(r)=-\frac{u'(r)}{r\sqrt{1+u'(r)^2}}=-\frac1{\sqrt{1+u'(r)^2}}\int_0^1 u''(sr) ds\hspace{.5cm}\textrm{is also of class }C^k([0, r_0)).$$
Thus $K=k_1k_2\in C^ k([0,r_0))$. Writing out the Euler-Lagrange equation from Lemma \ref{eulerlagrangeequation}, we get
$$\frac1{r\sqrt{1+u'(r)^2}}\frac d{dr}\left[\frac r{\sqrt{1+u'(r)^2}}\frac {dH}{dr}\right]
=
\Delta_g H
=-\frac12 H(H^2-4K)+a-bH\in C^k([0, r_0)).
$$
Using $u'\in C^{k+1}([0, r_0))$ and putting $\alpha(r):=\frac 1{\sqrt{1+u'(r)^2}}\frac {dH}{dr}$, we get $\frac{(r\alpha)'}r\in C^ k([0,r_0))$. Lemma \ref{oderegularitylemma} implies $\alpha\in C^{k+1}([0, r_0))$. 
By definition of $\alpha$ and using $u'\in C^{k+1}([0, r_0))$ we deduce $H'(r)\in C^{k+1}([0, r_0))$ and consequently $H\in C^{k+2}([0, r_0))$.\\

We have shown the following two implications: $H\in C^k([0,r_0) \Rightarrow H\in C^{k+2}([0,r_0))$ and $H\in C^k([0,r_0)) \Rightarrow u\in C^{k+2}([0,r_0))$. Since we have shown that $H\in C^1([0, r_0))$ in Lemma \ref{uisc2}, we deduce $u\in C^\infty([0,r_0))$.
\end{proof}

In Lemma \ref{uissmooth}, we have shown that the graph function $u$ describing $\gamma$ near $t=0$ is smooth. By the same arguments, the graph function $\tilde u$ describing $\gamma$ near $t=1$ is also smooth.
As an easy corollary, we obtain that $\gamma$ is smooth up to $t=0$ and $t=1$.

\begin{korollar}\label{gammasmoothuptoBoundaryCorollary}
    $\gamma\in C^\infty([0,1])$
\end{korollar}
\begin{proof}
    In view of Corollary \ref{gammasmooththeorem} we already have $\gamma\in C^\infty((0,1))$. Recall that we chose $t_0>0$ and $r_0:=\gamma_1(t_0)$ such that $\gamma_1:[0, t_0)\rightarrow [0, r_0)$ is a $C^1$-diffeomorphism. In particular, as $\dot\gamma_1(0)=L:=L[\gamma]>0$, we have $\dot\gamma_1(t)>0$ for all $t\in[0, t_0)$. From Lemma \ref{uissmooth}, we know that  $u:[0,r_0)\rightarrow\R,\ u(r):=\gamma_2(\gamma_1^{-1}(r))$
    is in $C^\infty([0, r_0))$. Clearly $\gamma_2(t)=u(\gamma_1(t))$. Differentiating and using $|\dot\gamma(t)|^2=L^2$ yields 
    $$L^2-\dot\gamma_1(t)^2=\dot\gamma_2(t)^2=u'(\gamma_1(t))^2\dot\gamma_1(t)^2\hspace{.5cm}\textrm{and so}\hspace{.5cm}\dot\gamma_1(t)^2=\frac{L^2}{1+u'(\gamma_1(t))^2}.$$
    Since $\dot\gamma_1(t)\geq 0$ for all $t\in[0, t_0)$ we can take the square root and obtain 
    $$\dot\gamma_1(t)=\frac{L}{\sqrt{1+u'(\gamma_1(t))^2}}.$$
    As $\gamma_1\in C^0([0, t_0))$ and $u'\in C^\infty([0, r_0))$, we deduce that $\gamma_1\in C^\infty([0, t_0))$ and consequently $\gamma_2=u\circ\gamma_1\in C^\infty([0, t_0))$. This establishes the smoothness of $\gamma$ near $t=0$. The smoothness near $t=1$ can be established by the same method. 
\end{proof}


We now prove Theorem \ref{regularitytheorem1} by demonstrating that $u^{(2k+1)}(0)=0$ for all integers $k\geq 0$. Indeed the same proof shows the analog result for the graph function $\tilde u$ describing $\gamma$ near $t=1$. 

\begin{lemma}[Proof of Theorem \ref{theorem1} (apart from $\beta(\sigma)\rightarrow 8\pi$ for $\sigma\rightarrow 0^+$)]\label{Sigmaissmooth}\ \\
$u^{(2k+1)}(0)=0$ for all $k\in\N_0$ and consequently  $\Sigma_\gamma$ is smooth.
\end{lemma}
\begin{proof}
The proof is inductive. By Equation \eqref{ODEamRand} we have $w(0)=u'(0)=0$. We show that if there exists $n\geq 0$ such that $u^{(2k+1)}(0)=0$  for all $0\leq k\leq n$, then $u^{(2n+3)}(0)=0$. By Lemma \ref{uisc2} and Equation \eqref{ODEamRand} we have 
\begin{equation}\label{SmoothnessFinalStepODE}
u'''+\left(\frac{u'}{r}\right)'
=
\frac12 \sqrt{1+u'(r)^2}^ 5 a r+b(1+u'(r)^2)^2u'(r)+\varphi(u').
\end{equation}
Since $u\in C^\infty([0, r_0)$ and $u^{(2k+1)}(0)=0$ for $0\leq k\leq n$, the function
$$U:(-r_0,r_0)\rightarrow\R,\ U(r):=\left\{
\begin{aligned}
u(r) &\hspace{.5cm} \textrm{ if }r\geq 0,\\
u(-r) &\hspace{.5cm}\textrm{ if } r<0
\end{aligned}
\right.$$
is smooth on $[0,r_0)$, $(-r_0, 0]$ and $(-r_0,r_0)\backslash\set 0$. However, the derivatives at $0$ of order$\geq 2n+3$ might not match up so that only $U\in C^{2n+2}((-r_0,r_0))$ is guaranteed. Consequently, the function 
$$F(r):=\frac12 \sqrt{1+U'(r)^2}^ 5 a r+b(1+U'(r)^2)^2U'(r)+\varphi(U')$$
is of class $C^{2n}((-r_0, r_0))$. Indeed, considering the definition of $\varphi(U')$ from Equation \eqref{varphiofwdef}, we see that only the term $\frac{(U'(r))^3}{2r^2}(3+(U'(r))^2)$ is of concern. However, using that $u'(0)=0$, we get 
$$\frac{U'(r)^2}{r^2}=\left(\int_0^1 U''(sr)ds\right)^2,\hspace{.5cm}\textrm{which is of class }C^{2n}(-r_0, r_0).$$
Additionally, it is readily checked that $F(-r)=-F(r)$. Thus $F^{(2n)}(0)=0$. Differentiating Equation \eqref{SmoothnessFinalStepODE} $2n$ times, we get 
\begin{equation}\label{differentiatedODE}
u^{(2n+3)}+\left(\frac{u'}r\right)^{(2n+1)}=F^{(2n)}(r)\hspace{.5cm}\textrm{on }[0, r_0).
\end{equation}
Since $u\in C^\infty([0, r_0))$ we may write
$$\left(\frac{u'}r\right)^{(2n+1)}
=
\left(\int_0^1 u''(rs)ds\right)^{(2n+1)}
=
\int_0^1 u^{(2n+3)}(rs)s^{2n+1} ds.$$
Now we consider Equation \eqref{differentiatedODE} the limit $r\rightarrow 0^+$. Since $F^{(2n)}(0)=0$ we get
\begin{equation}\label{nextderivative0}
0=\lim_{r\rightarrow 0^+}\left(u^{(2n+3)}(r)+\int_0^ 1  u^{(2n+3)}(rs)s^{2n+1} ds\right)=
\left(1+\frac1{2n+2}\right)u^{(2n+3)}(0)
\end{equation}
and hence $u^{(2n+3)}(0)=0$.
\end{proof}

