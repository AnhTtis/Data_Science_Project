\section{Formation of the Catenoidal Neck}\label{catenoidNeckSection}
Throughout this section let $\sigma_k\rightarrow 0^+$ and $\gamma^{(k)}\in\mathcal F_{\sigma_k}^{0+}$ such that $\mathcal W[\gamma^{(k)}]=\beta(\sigma_k)$. Additionally, let $L_k:=L[\gamma^{(k)}]$ and
$$\epsilon_k:=\inf_{t\in[\frac18,\frac78]}\gamma_1^{(k)}(t).$$
\begin{lemma}\label{catenoidLemma01}
    \begin{enumerate}[(1)]
    \item If $(t_k)\subset [\frac18,\frac78]$ is a sequence of times such that $\gamma_1^{(k)}$ has a local minimum at $t_k$, then $t_k\rightarrow\frac12$ and $\gamma_1^{(k)}(t_k)\rightarrow 0$. Moreover, there exists $k_0\in\N$ such that the minima are strict for all $k\geq k_0$.
    \item  $\epsilon_k\rightarrow 0^+$ as $k\rightarrow{\infty}$. Additionally, $\tau_k\rightarrow\frac12$ for every sequence $(\tau_k)\subset [\frac18,\frac78]$ satisfying $\gamma_1^{(k)}(\tau_k)=\epsilon_k$.
    \end{enumerate} 
\end{lemma}
    \begin{proof}
   Clearly, the second part is a special case of the first part. \\
   
        Now let $(t_k)\subset[\frac18,\frac78]$ be a sequence of local minima for $\gamma_1^{(k)}$.  After passing to a subsequence, we have $t_k\rightarrow t^*\in[\frac18,\frac78]$. We prove that $t^*=\frac12$ by contradiction. Clearly $\dot\gamma_1^{(k)}(t_k)=0$ and $\ddot\gamma_1^{(k)}(t_k)\geq 0$. If $t^*\neq\frac12$, we can use $\gamma^{(k)}\rightarrow\kappa$ in $C^\infty([0,1]\backslash\{\frac12\})$ to obtain $\dot\kappa_1(t_*)=0$ and $\ddot\kappa_1(t^*)\geq 0$. However, since $\kappa_1(t)=|\sin(2\pi t)|$, this is impossible. So $t^*=\frac12$ and 
        $$h_k:=\gamma_1^{(k)}(t_k)\rightarrow\kappa_1(\frac12)=0.$$
        Finally, we prove that for large $k$, the minima at $t_k$ are strict. For each $k$ we have $\dot \gamma_1^{(k)}(t_k)=0$ and $\ddot\gamma_1^{(k)}(t_k)\geq 0$. If $\ddot\gamma_1^{(k)}(t_k)>0$, the local minimum at $t_k$ is strict. If $\ddot\gamma_1^{(k)}(t_k)=0$ we also have $\dddot \gamma_1^{(k)}(t_k)=0$ and $\ddddot\gamma_1^{(k)}(t_k)\geq 0$ as otherwise $t_k$ could not be a local minimum of $\gamma_1^{(k)}$. If $\dot\gamma_1^{(k)}(t_k)=\ddot\gamma_1^{(k)}(t_k)=\dddot\gamma_1^{(k)}(t_k)=0$ and $\ddddot\gamma_1^{(k)}(t_k)>0$ the local minimum at $t_k$ is strict.  We prove by contradiction that 
        \begin{equation}\label{deriavtives0contrdiction}
        \dot\gamma_1^{(k)}(t_k)=\ddot\gamma_1^{(k)}(t_k)=\dddot\gamma_1^{(k)}(t_k)=\ddddot\gamma_1^{(k)}(t_k)=0\end{equation}
         along a subsequence is impossible. So let us assume that we had \eqref{deriavtives0contrdiction} along a subsequence again denoted by $\gamma^{(k)}$. Since $\dot\gamma_1^{(k)}(t_k)=0$ we have $|\dot\gamma_2^{(k)}(t_k)|=L_k\neq 0$ so that differentiating $|\dot\gamma^{(k)}(t)|^2=L_k^2$ implies
         $\ddot\gamma_2^{(k)}(t_k)=\dddot\gamma_2^{(k)}(t_k)=\ddddot\gamma_2^{(k)}(t_k)=0$.     
         Using Equation \eqref{principalcurvatures} we get 
         $$k_1^{(k)}(t_k)=\dot k_1^{(k)}(t_k)=\ddot k_1^{(k)}(t_k)=0=\dot k_2^{(k)}(t_k)=\ddot k_2^{(k)}(t_k)
         \hspace{.5cm}\textrm{and}\hspace{.5cm}|k_2^{(k)}(t_k)|=\frac1{h_k}.$$
         This implies that $|H[\gamma^{(k)}](t_k)|=\frac1{h_k}$, $K[\gamma^{(k)}](t_k)=0$ and $(\Delta_g H)[\gamma^{(k)}](t_k)=0$. Inserting into the Euler-Lagrange equation from Lemma \ref{eulerlagrangeequation}, we get
         \begin{equation}\label{MoonZappa}
         \frac1{2h_k^3}=\frac12 |H[\gamma^{(k)}](t_k)|^3=2|W[\gamma^{(k)}](t_k)|=\left|\Lambda_k\left(\pm_k4\sqrt\pi\pm_k\frac{\sigma_k}{h_k}\right)\right|. 
        \end{equation}
         By Corollary \ref{Lagrangemultipliergoto0Lemma} we have $\Lambda_k\rightarrow 0$. So, Equation \eqref{MoonZappa} implies $\frac12=|\Lambda_k(\pm_k4\sqrt\pi h_k^3\pm_k\sigma_kh_k^2)|\rightarrow 0$ as $k\rightarrow\infty$, which is a contradiction.        
    \end{proof}

By Theorem \ref{asymptoticsthm}, the curves $\gamma^{(k)}$ asymptotically look like two circles. The fact that these are joined by a catenoidal shape is derived by showing that near $t=\frac12$, they solve approximately $H[\gamma^{(k)}]=0$. The precise formulation of this observation is given in the following lemma.

\begin{lemma}\label{MinimalSurfaceLemma}
    Let $a_k\uparrow\frac12$ and $b_k\downarrow\frac12$. Then 
    $$\lim_{k\rightarrow\infty}\mathcal W\left[\gamma^{(k)}\big|_{[a_k, b_k]}\right]
    =
    \frac\pi2\lim_{k\rightarrow\infty}\int_{a_k}^{b_k}H[\gamma_k]^2\gamma_1^{(k)}|\dot\gamma^{(k)}|dt=0.$$
\end{lemma}
\begin{proof}
    The proof is by contradiction. Assume that along a subsequence again denoted by $\gamma^{(k)}$ we had 
    \begin{equation}
        \frac\pi2\int_{a_k}^{b_k}H[\gamma_k]^2\gamma_1^{(k)}|\dot\gamma^{(k)}|dt\geq \epsilon^*>0.
    \end{equation}
    For $\delta>0$ we put $I_\delta:=[0,\frac12-\delta)\cup(\frac12+\delta, 1]$. For each $\delta>0$ there exists $k_0(\delta)$ such that $I_\delta\cap(a_k, b_k)=\emptyset$ for all $k\geq k_0(\delta)$. Note that $\mathcal W[\gamma^{(k)}]=\beta(\sigma_k)\leq8\pi$. Taking any $\delta>0$ and $k\geq k_0(\delta)$, we can therefore estimate
    \begin{align*} 
    8\pi\geq &\mathcal W[\gamma^{(k)}]\\
    =&\frac\pi2\int_0^1 H[\gamma^{(k)}]^2 |\dot\gamma^{(k)}|\gamma_1^{(k)}dt\\
    \geq & \epsilon^*+\frac\pi2\int_{I_\delta} H[\gamma^{(k)}]^2 |\dot\gamma^{(k)}|\gamma_1^{(k)}dt.
    \end{align*}
    By Theorem \ref{asymptoticsthm}, we have $\gamma^{(k)}\rightarrow\kappa$ in $C^\infty(I_\delta)$ as $k\rightarrow\infty$. So, letting $k\rightarrow\infty$, we obtain 
    $$8\pi\geq \epsilon^*+\frac\pi2\int_{I_\delta} H[\kappa]^2 |\dot\kappa|\kappa_1dt=\epsilon^*+\mathcal W\left[\kappa\big|_{I_\delta}\right].$$
    We have $\mathcal W[\kappa]=8\pi$, so, letting $\delta\rightarrow0^+$, we arrive at a contradiction. 
\end{proof}

The remaining arguments are essentially based on the following blow-up lemma.

\begin{lemma}[Blow-Up Lemma]\label{blowuplemma}
    Let $t_k\rightarrow\frac12$ and put $h_k:=\gamma_1^ {(k)}(t_k)$. Assume that there is $\rho_0>0$ independent of $k$ such that $\gamma_1^ {(k)}(t_k+h_kt)\geq \frac12 h_k$ for all $|t|\leq \rho_0$. Then 
    $$\Gamma^{(k)}(t):=\frac1{h_k}\left(\gamma^ {(k)}(t_k+h_k t)-(0,\gamma_2^ {(k)}(t_k))\right)$$
   is bounded in $C^m([-\rho,\rho])$ for all $\rho\in(0,\rho_0)$ and $m\in\N$. Additionally, if along some subsequence $\Gamma^{(k_l)}\rightarrow \Gamma^*$ in $C^2([-\rho,\rho])$, the limit $\Gamma^*$ satisfies $H[\Gamma^*]=0$. 
\end{lemma}
\begin{proof}
By Lemma \ref{eulerlagrangeequation}, the curves $\gamma^{(k)}$ satisfy an Euler Lagrange equation. Using the well-known scaling of the Willmore operator and the mean curvature, we get
\begin{align*}
W[\Gamma^{(k)}](t)=&h_k^3W[\gamma^{(k)}](t_k+h_k t)\\
    =&2\Lambda_k(\pm_{\gamma^{(k)}}4\sqrt\pi h_k^3-\sigma_k h_k^3H[\gamma^{(k)}](t_k+h_k t))\\
    =&2\Lambda_k(\pm_{\gamma^{(k)}} 4\sqrt\pi h_k^3-\sigma_k h_k^2H[\Gamma^{(k)}](t)).
\end{align*}
Using Equation \eqref{variationformulastandard} and  Equations \eqref{var2}, \eqref{var3}, \eqref{var4}, \eqref{var5} we deduce that for all $\phi\in C^\infty_0((-\rho,\rho))$
$$\delta\mathcal W[\Gamma^{(k)}]\phi=2\Lambda_k\sigma_k h_k^2 \delta A[\Gamma^{(k)}]\phi-8\sqrt\pi\Lambda_k h_k^3 \delta V[\Gamma^{(k)}]\phi.$$
 $h_k\rightarrow 0^+$ since $t_k\rightarrow\frac12$ and by Corollary \ref{Lagrangemultipliergoto0Lemma} we also know $\Lambda_k\rightarrow 0$. We have $|\dot\Gamma^{(k)}|=L_k\rightarrow L_*>0$ and hence $L_k+L_k^{-1}$ is bounded. By scaling invariance of the Willmore energy we have $\mathcal W[\Gamma^{(k)}|_{(-\rho_0,\rho_0)}]\leq \mathcal W[\gamma^{(k)}]\leq 8\pi$. Next we exploit the assumption $\gamma_1^{(k)}(t_k+h_k t)\geq\frac12 h_k$. First, by definition of $\Gamma^{(k)}$, it implies $\Gamma_1^{(k)}(t)\geq \frac12$ for all $|t|< \rho_0$. Second, using Lemma \ref{gamma1ddotgammainL1}, we get 
$$\int_{-\rho_0}^{\rho_0}|\ddot\Gamma^{(k)}(t)|^2 dt=h_k \int_{t_k-h_k\rho_0}^{t_k+h_k\rho_0}|\ddot\gamma(s)|^2ds\leq C.$$
Finally, since by construction $\Gamma^{(k)}(0)=(1,0)$ and $|\dot\Gamma^{(k)}|=L_k\leq C$, we obtain $|\Gamma^{(k)}|\leq C$. We have proven that the assumptions of Theorem \ref{GeneralRegularityTheorem} are satisfied with $M$ and $\kappa(\Gamma^{(k)};(-\rho,\rho))\geq \kappa_0>0$ independent of $k$. Therefore, the uniform $C^m$ bounds follow from Theorem \ref{GeneralRegularityTheorem}. \\

Now assume that after passing to a subsequence $\Gamma^{(k)}\rightarrow\Gamma^*$ in $C^2$. We use the invariance of the Willmore energy under scaling and apply Lemma \ref{MinimalSurfaceLemma} to compute 
  \begin{align*}
  \frac\pi2\int_{-\rho}^ {\rho}H[\Gamma^*]^2 |\dot\Gamma^*|\Gamma^*_1dt=&\frac\pi2\lim_{k\rightarrow\infty}
  \int_{-\rho}^ {\rho}H[\Gamma^{(k)}]^2 |\dot\Gamma^{(k)}|\Gamma^{(k)}_1dt\\
  =&\lim_{k\rightarrow\infty}\mathcal W[\Gamma^ {(k)}|_{[-\rho,\rho]}]\\
   =&\lim_{k\rightarrow\infty}\mathcal W[\gamma^ {(k)}|_{[t_k-h_k\rho,t_k+h_k\rho]}]\\
   =&0.
  \end{align*}
Since $\Gamma_1^*\geq \frac12$ on $(-\rho,\rho)$ and $|\dot\Gamma^*|=L_*> 0$ we deduce  $H[\Gamma^*]\equiv0$. 
\end{proof}

Next, we establish that for $k$ large enough, the minimal value $\epsilon_k$ is only attained once.

\begin{lemma}[Proof of Theorem \ref{CatenoidNeckTheorem}, parts (1) and (2)] \label{catenoidLemma02}\ \\
    There exists $k_0\in\N$ such that for all $k\geq k_0$ there exists a unique $\tau_k\in[\frac18,\frac78]$ satisfying $\gamma_1^{(k)}(\tau_k)=\epsilon_k$. Moreover, there exists $\rho_0>0$ such that $\dot\gamma_1(t)<0$ for $t\in(\tau_k-\rho_0,\tau_k)$ and $\dot\gamma_1(t)>0$ for $t\in(\tau_k, \tau_k+\rho_0)$.
\end{lemma}
Note that part (3) of Theorem \ref{CatenoidNeckTheorem} follows by combining Lemmas \ref{catenoidLemma01} and \ref{catenoidLemma02}.
\begin{proof}
    We choose a sequence $(\tau_k)\subset[\frac18,\frac78]$ such that $\gamma_1^{(k)}(\tau_k)=\epsilon_k$ and prove by contradiction that there exists $\rho_0>0$ such that $\dot\gamma_1^{(k)}(t)\neq 0$ for all $t\in(\tau_k-\rho_0,\tau_k+\rho_0)\backslash\set{\tau_k}$. Once this is shown, uniqueness follows since any other sequence $\tau_k'$ would satisfy $|\tau_k-\tau_k'|\rightarrow 0$ by Lemma \ref{catenoidLemma01}.\\
    
    Assuming the above claim was false and potentially passing to a subsequence again denoted by $\gamma^{(k)}$, we obtain a sequence $\tilde\tau_k\neq\tau_k$ satisfying $|\tilde\tau_k-\tau_k|\rightarrow 0$ and $\dot\gamma_1^{(k)}(\tilde\tau_k)=0$. For fixed $k$ either $\ddot\gamma_1^{(k)}(\tilde\tau_k)\leq 0$ or $\ddot\gamma_1^{(k)}(\tilde\tau_k)> 0$. Suppose that the second case is true. Then $\gamma_1^{(k)}$ has a local minimum at $\tilde\tau_k$.  By Lemma \ref{catenoidLemma01}, the minima are strict and hence we deduce that $\gamma_1^{(k)}$ must attain a local maximum at some $\hat\tau_k$ in between $\tau_k$ and $\tilde\tau_k$. Then $\dot\gamma^{(k)}(\hat\tau_k)=(0,\pm L_k)$ and $\ddot\gamma_1^{(k)}(\hat\tau_k)\leq 0$. In the first case,  $\tilde\tau_k$ has the same properties: $\dot\gamma^{(k)}(\tilde\tau_k)=(0,\pm L_k)$ and $\ddot\gamma_1^{(k)}(\tilde \tau_k)\leq 0$.\\
    
    So, in any case we have produced a sequence $\tau_k\neq \hat\tau_k\rightarrow\frac12$ such that after potentially passing to another subsequence we have $\ddot\gamma_1^{(k)}(\hat\tau_k)\leq 0$ and $\dot\gamma_1^{(k)}(\hat\tau_k)=(0,\pm L_k)$. We will assume that $+$ is the correct sign. If `$-$' is correct, the argument needs no modification.\\
    
     Since $\gamma^{(k)}\rightarrow\kappa$ in $C^0([0,1])$ we deduce $h_k:=\gamma_1^{(k)}(\hat\tau_k)\rightarrow \kappa_1(\frac12)=0$.
    For a parameter $\rho_0>0$, which we will choose shortly, we define the curve 
    $$\Gamma^{(k)}:[-\rho_0,\rho_0]\rightarrow\mathcal H^2,\ \Gamma^{(k)}(t):=\frac1{h_k}\left(\gamma^{(k)}(\hat\tau_k+h_k t)-\left(0, \gamma^{(k)}_2(\hat\tau_k)\right)\right).$$
    Using $L_k|=\dot\gamma^{(k)}|\rightarrow L[\kappa]=L_*$ we deduce that $L_k$ is bounded and estimate 
    \begin{equation}\label{gamma1klowerbound}
    \gamma_1^{(k)}(\hat\tau_k+h_k t)=h_k+h_k\int_0^t\dot\gamma_1^{(k)}(\hat\tau_k+h_k s)ds\geq h_k(1-L_k |t|)\geq h_k(1-C\rho_0).
    \end{equation}
 Choosing $\rho_0$ small enough, we obtain $\gamma^ {(k)}_1(\hat\tau_k+h_k t)\geq \frac12h_k$. We fix any $\rho\in (0,\rho_0)$. By Lemma \ref{blowuplemma} we deduce that $\Gamma^ {(k)}$ is bounded in $C^ m([-\rho,\rho])$ for all $m\in\N$. Consequently there exists a subsequence, again denoted by $\Gamma^ {(k)}$, that converges in $C^2([-\rho,\rho])$ to some limit $\Gamma^*$.  Lemma \ref{blowuplemma} gives $H[\Gamma^*]=0$. Since $\Gamma^ {(k)}(0)=(1,0)$, $\dot\Gamma^ {(k)}(0)=(0,L_k)$ and $\ddot\Gamma^ {(k)}_1(0)\leq 0$ we obtain $\Gamma^*(0)=(1,0)$, $\dot\Gamma^*(0)=(0,L_*)$ and $\ddot\Gamma^*_1(0)\leq 0$. Inserting into $H[\Gamma^*]=0$, we produce the contradiction 
 $$0=k_1[\Gamma^*](0)+k_2[\Gamma^*](0)
 =-\frac{\ddot\Gamma_1^*(0)\dot\Gamma^*_2(0)}{L_*^3}+\frac{\dot\Gamma_2^*(0)}{\Gamma_1^*(0)L_*}=-\frac{\ddot\Gamma_1^*(0)}{L_*^2}+1\geq 1.$$
Consequently there exists $\rho_0>0$ such that $\dot\gamma_1^{(k)}(t)\neq 0$ for all $t\in(\tau_k-\rho_0,\tau_k+\rho_0)$. We argue for example that $\dot\gamma_1^{(k)}> 0$ when $t\in(\tau_k,\tau_k+\rho)$. Indeed, if this were not true, then $\dot\gamma_1(t)<0$ would have to be true. But that would imply that for small $t>0$
$$\epsilon_k=\inf_{[\frac18,\frac78]}\gamma_1^{(k)}\leq \gamma_1^{(k)}(\tau_k+t)=\gamma_1^{(k)}(\tau_k)+\int_0^t \dot\gamma_1^{(k)}(\tau_k+s)ds<\epsilon_k.$$
\end{proof}

\begin{lemma}\label{gammakModificationLemma}
    There exists $k_1\in\N$ such that for $k\geq k_1$, the curves $\hat\gamma^{(k)}(s):=(\gamma_1^{(k)}(1-s),\gamma_2^{(k)}(1-s)-\gamma_2^{(k)}(1))$ satisfy $\hat\gamma^{(k)}\in \mathcal F_\sigma^{0+}$ and $\mathcal W[\hat\gamma^{(k)}]=\beta(\sigma_k)$.
\end{lemma}
\begin{proof}
    It is clear that $\hat\gamma^{(k)}\in\mathcal P$, $\mathcal W[\gamma^{(k)}]=\mathcal W[\hat\gamma^{(k)}]$ and that $\mathcal I[\gamma^{(k)}]=\mathcal I[\hat\gamma^{(k)}]$. Additionally $\hat\gamma^{(k)}(0)=0$.
    Let $\epsilon>0$ satisfy 
    $$0<\epsilon<\int_0^1 \kappa_2(t)dt.$$
    Since $\gamma^{(k)}\rightarrow \kappa$ in $C^0([0,1])$ we can choose $k_1(\epsilon)$ such that for all $k\geq k_1$ 
    $$
\int_0^1\hat\gamma_2^{(k)}(t) dt
=
\int_0^1\gamma_2^{(k)}(t) dt-\gamma_2^{(k)}(1)
\geq 
\int_0^1\kappa_2(s) ds-\kappa_2(1)-\epsilon
= 
\int_0^1\kappa_2(s) ds-\epsilon
>0.
$$
\end{proof}
By Lemma \ref{catenoidLemma02} we know that for $k$ large enough, $\gamma_1^{(k)}$ attains its minimum on $[\frac18,\frac78]$ only once, namely at $t=\tau_k$. At $t=\tau_k$ we then have $\dot\gamma^{(k)}(\tau_k)=(0,\pm_k L_k)$. Using Lemma \ref{gammakModificationLemma}, we can now modify the sequence $\gamma^{(k)}$ and ensure that for all $k\geq k_1$
\begin{equation}\label{assumeptionLk}
    \dot\gamma^{(k)}(\tau_k)=(0,- L_k).
\end{equation}

\begin{lemma}[Proof of Theorem \ref{CatenoidNeckTheorem}, part (4)]\ \\
    Let $\gamma^{(k)}$ now additionally satisfy Equation \eqref{assumeptionLk}. For all $R>0$ there exists $k_0(R)\in\N$ such that for all $k\geq k_0(R)$ the function 
    $$\Gamma^ {(k)}:[-R,R]\rightarrow\mathcal H^2,\ \Gamma^ {(k)}(t):=\frac1{\epsilon_k}\left(\gamma^ {(k)}(\tau_k+\epsilon_k t)-(0,\gamma_2^ {(k)}(\tau_k))\right)$$
    is well-defined. Additionally, putting 
    $$\Gamma^*(t):=\left(\sqrt{1+L_*^2t^2},\ -\operatorname{asinh}(L_*t)\right)\hspace{.5cm}\textrm{where }L_*=\sqrt{\frac\pi2},$$
    we have $\Gamma^{(k)}\rightarrow \Gamma^*$ in $C^m([-R, R])$ for all $m\in\N$. 
\end{lemma}
\begin{proof}
    For $R>0$ we choose $k_0(R)$ so large that $|\tau_k+\epsilon_k t|\in(\frac18,\frac78)$ for all $|t|\leq R+1$ and $k\geq k_0(R)$. By definition of $\epsilon_k$ we get $\Gamma^ {(k)}_1(t)\geq 1$ for all $t\in[-R,R]$. Therefore we can use Lemma \ref{blowuplemma} and deduce that $\Gamma^ {(k)}$ is bounded in $C^ m([-R,R])$ for all $m\in\N$. We show that any subsequence contains another subsequence that converges to the claimed limit. Let $\Gamma^ {(k)}$ be some subsequence, again denoted by $\Gamma^ {(k)}$. By the uniform $C^ m$ bound we can assume that $\Gamma^ {(k)}\rightarrow\Gamma^*$ in $C^2([-R,R])$ after potentially passing to yet another subsequence and by Lemma \ref{blowuplemma}, the limit satisfies $H[\Gamma^*]=0$. By assumption we have $\dot\gamma_2 ^{(k)}(\tau_k)=-L_k$ and since $\tau_k$ is a minimum of $\gamma_1^{(k)}$ we have $\dot\gamma_1^{(k)}(\tau_k)=0$ and $\ddot\gamma_1^{(k)}(\tau_k)\geq 0$. By Theorem \ref{asymptoticsthm}, we have $L_k=L[\gamma^{(k)}]\rightarrow L[\kappa]=\sqrt{\frac\pi2}$. Exploiting the $C^2$-convergence, we get
    $$\Gamma^ *(0)=(1,0),\hspace{.5cm} \dot\Gamma^*(0)=(0, -L_*)\hspace{.5cm}\textrm{and}\hspace{.5cm}\ddot\Gamma^*_1(0)\geq 0.$$
    Since $|\dot\Gamma^*(t)|=\lim_{k\rightarrow\infty}|\dot\Gamma^{(k)}(t)|=\lim_{k\rightarrow\infty}L_k=L_*$, the limit $\Gamma^*$ is parameterized proportional to arc length. Using Equation \eqref{ArcLength_H_Identity}, we get
    \begin{equation}\label{LimitH0Equations}
    \ddot\Gamma_1^*=\frac{(\dot\Gamma_2^*)^2}{\Gamma_1^*}
    \hspace{.5cm}\textrm{and}\hspace{.5cm}
    \ddot\Gamma_2^*=-\frac{\dot\Gamma_1^*\dot\Gamma_2^*}{\Gamma_1^*}.
    \end{equation}
    Using the first identity and $|\dot\Gamma^*|^2=L_*^2$, we compute 
   $$
        \frac12\frac{d^2}{dt^2}\Gamma_1^*(t)^2=\ddot\Gamma_1^*\Gamma_1^*+(\dot\Gamma_1^*)^2=(\dot\Gamma_2^*)^2+(\dot\Gamma_1^*)^2=L_*^2.
   $$
So, using  $\Gamma_1^*(0)=1$, $\dot\Gamma_1^*(0)=0$  and $\Gamma_1^*(t)\geq 0$, we deduce  $\Gamma_1^*(t)=\sqrt{1+L_*^2 t^2}$. Now, using the second identity in Equation \eqref{LimitH0Equations}, we compute
    $$\frac d{dt}(\dot\Gamma_2^*\Gamma_1^*)=\ddot\Gamma_2^*\Gamma_1^*+\dot\Gamma_1^*\dot\Gamma_2^*=0
     \hspace{.5cm}\textrm{and hence}\hspace{.5cm}
     \dot\Gamma_2^*(t)\Gamma_1^*(t)=\dot\Gamma_2^*(0)\Gamma_1^*(0).$$
     We have $\dot\Gamma_2^*(0)=-L_*$ and $\Gamma^*(0)=(1,0)$, which implies 
     $$\dot\Gamma_2^*(t)=\frac{-L_*}{\Gamma_1^*(t)}=\frac{-L_*}{\sqrt{1+L_*^2 t^2}}\hspace{.5cm}\textrm{and hence}\hspace{.5cm}
     \Gamma_2^*(t)=-\operatorname{asinh}(L_*t).$$
\end{proof}