\section{Properties of Profile Curves}\label{Appendix1}
\subsection{Proofs for Subsection \ref{functionalspacesection}}\label{proofpropertiesofP}
We begin by proving Lemma \ref{WillmoreLowerBound}
\begin{proof}
We put $L:=L[\gamma]$ and $d\mu_\gamma:=2\pi|\dot\gamma|\gamma_1 dt$ so that $k_1^2+k_2^2\in L^1(\mu_\gamma)$ by definition of the class $\mathcal P$. Hence $|k_1k_2|\in L^1(\mu_\gamma)$ and we may estimate 
\begin{align}
\mathcal W[\gamma]=&\frac{2\pi}4\int_0^1 (k_1+k_2)^2\gamma_1|\dot\gamma(t)|dt\nonumber\\ &\geq 2\pi\int_0 ^1 k_1 k_2 |\dot\gamma|\gamma_1 dt
=\lim_{\epsilon\rightarrow0^+} 2\pi\int_\epsilon^{1-\epsilon} k_1 k_2 |\dot\gamma|\gamma_1 dt.\label{WillLB01}
\end{align}
By definition of the class $\mathcal P$ we have $\gamma\in W^{2,2}((\epsilon,1-\epsilon))$. So we may use Equation \eqref{ArcLength_K_Identity} and deduce
$$2\pi\int_\epsilon^{1-\epsilon} k_1k_2|\dot\gamma|\gamma_1dt=-\frac{2\pi}L\int_\epsilon^ {1-\epsilon}\ddot\gamma_1(t)dt=\frac{2\pi}L(\dot\gamma_1(\epsilon)-\dot\gamma_1(1-\epsilon)).$$
We insert into Equation \eqref{WillLB01} and wish to let $\epsilon\rightarrow 0^ +$. To do so, note that $\dot\gamma$ is continuous by the definition of the class $\mathcal P$. So, using Equation \eqref{velocityatends}, we find
\begin{equation}\label{gaussbonnet}
\mathcal W[\Sigma_\gamma]\geq 2\pi\int_0^1 k_1k_2|\dot\gamma|\gamma_1 dt=\lim_{\epsilon\rightarrow0^ +}2\pi\int_\epsilon^{1-\epsilon} k_1k_2|\dot\gamma|\gamma_1dt
= \frac{2\pi}L(L-(-L))=4\pi.
\end{equation}
Equality only occurs when $(k_1+k_2)^2=4k_1k_2$ almost everywhere, which is true only if $k_1=k_2$ almost everywhere. This gives 
$$\ddot\gamma_2\dot\gamma_1-\ddot\gamma_1\dot\gamma_2=L^3k_1=L^3 k_2=\frac{\dot\gamma_2}{\gamma_1}L^2.$$
Multiplying this equation by $\dot\gamma_2$ and using $\langle\dot\gamma,\ddot\gamma\rangle=0$ yields 
\begin{equation}\label{globalgamma1ode}
\ddot\gamma_1\gamma_1=\dot\gamma_1^2-L^2.
\end{equation}
As $\gamma_1>0$ on $(0,1)$ and $\gamma\in C^1([0,1])$, we deduce $\gamma_1\in C^2((0,1))\cap C^1([0,1])$. Since $\dot\gamma_1\in C^0([0,1])$, $\dot\gamma_1(0)=L$ and $\dot\gamma_1(1)=-L$ there exists a non-empty, open interval $I\subset(0,1)$ such that $\dot\gamma_1(t)\not\in\set{0,\pm L}$ for all $t\in I$. On $I$, the following computation is valid:
$$-\frac12\frac d{dt}\ln(L^2-\dot\gamma_1^2)=\frac{\dot\gamma_1\ddot\gamma_1}{L^2-\dot\gamma_1^2}\overset{\eqref{globalgamma1ode}}=-\frac{\dot\gamma_1}{\gamma_1}=-\frac d{dt}\ln(\gamma_1(t))$$
From this, we deduce that there exists $K>0$ such that $L^2-\dot\gamma_1^2=K\gamma_1^2(t)$ for all $t\in I$. Differentiating gives $\dot\gamma_1(\ddot\gamma_1+K\gamma_1)=0$. As $\dot\gamma_1(t)\neq 0$ we deduce that for some $a,b\in\R$ we get
\begin{equation}\label{gamma1sol}
\gamma_1(t)=a\cos(\sqrt K t)+b\sin(\sqrt K t)\hspace{.5cm}\textrm{for all $t\in I$}.
\end{equation}
Using $\gamma_1>0$ for all $t\in(0,1)$ and Equation \eqref{globalgamma1ode}, the Picard-Lindeöff theorem implies that Equation \eqref{gamma1sol} is valid for all $t\in(0,1)$. Using $\dot\gamma_2^2=L^2-\dot\gamma_1^2$, it is then readily seen that $\gamma$ traces out a semicircle and hence $\Sigma_\gamma$ is a sphere. 
\end{proof}

Note that in Equation \eqref{gaussbonnet}, we have verified Lemma \ref{gausbonnetlemma} as we have shown that
$$
\int_{\Sp^2}k_1k_2d\mu_{\Sigma_\gamma}
=2\pi\int_0^1 k_1k_2 |\dot\gamma|\gamma_1 dt
=4\pi.
$$

Next, we prove Lemma \ref{hopftypetheorem}
\begin{proof}
First we note $H_0\neq 0$ as $4\pi\leq \mathcal W[\gamma]=\frac14 A[\gamma]H_0^2$ by Lemma \ref{WillmoreLowerBound}. Let $L:=L[\gamma]$. Using the assumption $H=H_0$ and Equation \eqref{ArcLength_H_Identity}, we get
\begin{equation}\label{hopf0}
    H_0\dot\gamma_2=-\frac{\ddot\gamma_1}L+\frac{\dot\gamma_2^2}{L\gamma_1}
    \hspace{.5cm}\textrm{and}\hspace{.5cm}
     H_0\dot\gamma_1=\frac{\ddot\gamma_2}L+\frac{\dot\gamma_1\dot\gamma_2}{L\gamma_1}.
\end{equation}
Using $\gamma\in C^1([0,1])$, $\gamma_1>0$ on $(0,1)$ and Equation \eqref{hopf0}, we get $\gamma\in C^\infty((0,1))$. Using the second identity in Equation \eqref{hopf0}, we compute 
\begin{equation}\label{sphereidentity1}
\frac d{dt}(\dot\gamma_2\gamma_1)=\ddot\gamma_2\gamma_1+\dot\gamma_2\dot\gamma_1=LH_0\gamma_1\dot\gamma_1
\hspace{.5cm}\textrm{and hence}\hspace{.5cm}
\dot\gamma_2\gamma_1=\frac12LH_0\gamma_1^2.
\end{equation}
There is no integration constant since $\gamma_1(0)=0$. Using $\gamma_1(t)>0$ for $t\in (0,1)$, we may deduce $\dot\gamma_2=\frac12 LH_0\gamma_1$. Inserting into the first identity in Equation \eqref{hopf0} and putting $\omega:=\frac{LH_0}2$, we get 
$$\ddot\gamma_1+\omega^2\gamma_1=0\hspace{.5cm}\textrm{and hence}\hspace{.5cm}\gamma_1(t)=a\cos(\omega t)+b\sin(\omega t).$$
Combining $\gamma_1(0)=\gamma_1(1)=0$ and $\gamma_1(t)>0$ for $t\in(0,1)$ we deduce $a=0$, $b>0$ and $\omega=\pi$.  Using $\dot\gamma_2=\omega \gamma_1$ we get $\gamma_2(t)=\gamma_2(0)+b(1-\cos(\pi t))$. Finally, since $|\dot\gamma|=L$, we conclude $b=\frac L\pi$.
\end{proof}