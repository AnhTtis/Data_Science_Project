\section{Introduction}
The isoperimetric ratio $\mathcal I(f)$ of a smooth embedding $f:\Sp^2\rightarrow\R^3$ is defined as
$$\mathcal I[f]:=6\sqrt\pi\frac{V[f]}{A[f]^{\frac32}},$$
where $A[f]$ and $V[f]$ denote the surface area of $f$ and the volume enclosed by $f$, respectively. By the classical isoperimetric inequality 
$$V[f]\leq\frac1{6\sqrt\pi} A[f]^{\frac32}$$
we have $\mathcal I(f)\in(0,1]$ with $\mathcal I(f)=1$ occurring only for round spheres. For $\sigma\in(0,1]$, Schygulla \cite{schygulla} introduces the problem of minimizing the Willmore energy 
$$\mathcal W[f]:=\frac14\int_{\Sp^2} H[f]^2 d\mu_f$$
under all embeddings $f\in C^\infty(\Sp^2,\R^3)$ with $\mathcal I[f]=\sigma$. This problem is motivated by the models for the shape of blood cells by  Canham \cite{canham} and Helfrich \cite{helfrich}, who independently proposed to model the bending energy of a blood cell by a quadratic expression in its curvature. Helfrich's article \cite{helfrich} also motivates the isoperimetric constraint: He prescribes the area and the volume. However, as the Willmore energy is invariant under scaling, these two constraints reduce to prescribing the isoperimetric ratio. By Hopf's theorem, the constraint of prescribed isoperimetric ratio $\mathcal I[f]=\sigma$ is nondegenerate for $\sigma<1$ so that by Lagrange's theorem, the minimizers satisfy the Euler-Lagrange equation 
\begin{equation}\label{IntroPDE}
\Delta_g H+\frac12 H(H^2-4K)=\Lambda\sigma\left(\frac1{V[f]}-\frac{3 H}{2A[f]}\right),
\end{equation}
where $H$ and $K$ denote the mean and the Gauß curvature of $f$ respectively.  In this article, we investigate the axially symmetric case. For $\sigma\in(0,1]$, we introduce 
$$\mathcal S_\sigma:=\set{f\in C^\infty(\Sp^2,\R^3)\textrm{ embedding }|\ \mathcal I[f]=\sigma\textrm{ and }f\circ T=f\textrm{ for all }T\in\operatorname{SO}(2)}.$$
Additionally, we let $\beta(\sigma):=\inf_{f\in\mathcal S_\sigma}\mathcal W[f]$. A convenient characterization of the surfaces in $\mathcal S_\sigma$ is achieved using their \emph{profile curves}. Indeed, after translation, every $f\in\mathcal S_\sigma$ is of the form 
\begin{equation}\label{introrevolutioneq}
f_\gamma:[0,1]\times \Sp^1\rightarrow\R^3,\ f_\gamma(s,\omega):=\begin{bmatrix}
\gamma_1(s)\omega\\
\gamma_2(s)
\end{bmatrix}
\end{equation}
where $\gamma\in\mathcal F_\sigma:=\set{\gamma\in\mathcal P\ |\ \mathcal I[f_\gamma]=\sigma}$ and $\mathcal P$ denotes the set of admissible profile curves
$$\mathcal P:=\left\{
\gamma\in C^\infty([0,1],\R^2)\ \big|\ 
\begin{array}{l}
     \gamma_1(0)=\gamma_1(1)=0\textrm{ and } \gamma_1(t)>0 \textrm{ for all }t\in(0,1)
\end{array}
\right\}.$$
Conversely, using Equation \eqref{introrevolutioneq}, any $\gamma\in\mathcal P$ can be assigned a surface of spherical type in $\R^3$ so that we can identify $f_\gamma$ and $\gamma$. 

\begin{theorem}\label{theorem1}
For all $\sigma\in(0,1]$, there exists $f\in\mathcal S_\sigma$ such that $\mathcal W[f]= \beta(\sigma)$. Moreover, $\beta(\sigma)< 8\pi$ for all $\sigma\in(0,1]$ and $\beta(\sigma)\rightarrow 8\pi$ as $\sigma\rightarrow 0^+$. 
\end{theorem}

The proof of \ref{theorem1} is split into three steps. First, the existence of a weak minimizing profile $\gamma\in W^{1,2}((0,1))\cap W^{2,2}_{\operatorname{loc}}((0,1))$ is established using the direct method of the calculus of variations. The arguments used here follow those executed by Choski and Venernoni in \cite{choskiveneroni}. Next, the smoothness of $\gamma$ on $(0,1)$ -- away from the axis of rotation --  is established using elliptic regularity arguments. 
To study regularity at the axis of rotation, a careful analysis of the local graph representation 
$$\set{(r\omega, u(r))\ |\ r\in[0, r_0),\ \omega\in \mathbb S^1}$$
of $f_\gamma$ near the axis of rotation is required. For the unconstrained Willmore equation, such analysis has been carried out in \cite{chenODE} by Chen and Li. We adapt their analysis to prove
\begin{equation}\label{introsingularsolution}
u(r)=\lambda r^2\ln(r)+C^2([0, r_0)).
\end{equation}
The parameter $\lambda$ is a residue, as has been studied by Kuwert and Schätzle in \cite{removability} and cannot be eliminated by ODE arguments alone, as the Euler-Lagrange equation \eqref{IntroPDE} allows for singular solutions of the form in Equation \eqref{introsingularsolution}.\\

After proving the existence of smooth minimizers, we investigate their behavior in the limit $\sigma\rightarrow0^+$. This has already been done by Schygulla \cite{schygulla}. He proved that as $\sigma\rightarrow0^+$, his minimizers converge to a double sphere in the sense of measures. In \cite{kuwertli}, this result was later refined by Kuwert and Li, who prove that asymptotically, Schygulla's minimizers look like two spheres joined by a catenoidal neck. We prove that our axially symmetric minimizers display the same behavior. Restricting to the class 
$$\mathcal F^{0+}_\sigma:=\left\{\gamma\in\mathcal F_\sigma,\ \bigg|\ \gamma(0)=(0,0),\ A[f_\gamma]=1\textrm{ and }\int_0^1 \gamma_2(t)dt\geq 0\right\},$$
we eliminate the invariance of the problem under vertical translation, scaling and reflection -- thereby out ruling the trivial obstructions to convergence. Characterizing the solutions to the Euler-Lagrange equation in the limit $\sigma\rightarrow0^+$ we then establish:

\begin{theorem}[Convergence to a Double Sphere]\label{doublesphereconvergence}
    Let $(\sigma_k)\subset(0,1]$ be a sequence of isoperimetric ratios $\sigma_k\rightarrow 0^+$ and $(\gamma^{(k)})_k$ be a sequence of profile curves $\gamma^{(k)}\in\mathcal F^{0+}_{\sigma_k}$ satisfying $\beta(\sigma_k)=\mathcal W[f_{\gamma^{(k)}}]$. Then as $k\rightarrow\infty$,
    $$\gamma^{(k)}(t)\rightarrow\kappa(t):=(0,R)+R(|\sin(2\pi t)|, -\cos(2\pi t))\hspace{.5cm}\textrm{with }R:=\frac{1}{\sqrt{8\pi}}$$
   in $C^\infty([0,1]\backslash \{\frac12\})$ and in $W^{1,2}((0,1))$.
\end{theorem}

Since $\kappa$ is not smooth at $t=\frac12$, it is clear that the curves $\gamma^{(k)}$ have to develop some form of singularity near $t=\frac12$. We prove that near $t=\frac12$, the profile curves solve approximately the equation $H=0$. Using a suitable blow-up argument, we then establish the following theorem:   

\begin{theorem}[Formation of a Catenoidal Neck]\label{CatenoidNeckTheorem} Let $\sigma_k\rightarrow0^+$ and $\gamma^{(k)}\in\mathcal F^{0+}_{\sigma_k}$ satisfy $\mathcal{W}[f_{\gamma^{(k)}}]=\beta(\sigma_k)$. There exist $k_0\in\N$ and $\rho>0$ such that the following statements hold:
\begin{enumerate}[(1)]
    \item For $k\geq k_0$ there exists a unique $\tau_k\in[\frac18,\frac78]$ such that $\gamma_1^{(k)}(\tau_k)=\epsilon_k:=\inf_{[\frac18,\frac78]}\gamma_1^{(k)}$.
    \item $\dot\gamma_1^{(k)}<0$ on $[\tau_k-\rho,\tau_k)$ and $\dot\gamma_1^{(k)}>0$ on $(\tau_k,\tau_k+\rho]$. 
    \item The parameters from $(1)$ satisfy $\epsilon_k\rightarrow 0^+$ and $\tau_k\rightarrow\frac12$.
    \item After potentially putting
$\hat\gamma^{(k)}(t):=\left(\gamma_1^{(k)}(1-t),\ \gamma_2^{(k)}(1-t)-\gamma_2^{(k)}(1)\right)$,
it can be assumed that $\dot\gamma_2^{(k)}(\tau_k)\leq 0$ for all $k\geq k_0$ and the following convergence is in $C^\infty_{\operatorname{loc}}(\R)$:
    $$\frac1{\epsilon_k}\left(\gamma^{(k)}(\tau_k+\epsilon_k t)-(0,\gamma_2^{(k)}(\tau_k))\right)\rightarrow \left(\sqrt{1+L_*^2 t^2}, -\operatorname{asinh}(L_*t)\right)\hspace{.5cm}\textrm{with }L_*:=\sqrt{\frac\pi2}$$
\end{enumerate} 
\end{theorem}


The results from Theorems \ref{doublesphereconvergence} and \ref{CatenoidNeckTheorem} are in agreement with the numerically computed minimizers in \cite{seifert} (see Figure 9).  Using numerical methods, several authors have computed candidates for the minimizers. For an overview, we refer to \cite{seifert}. However, we mention that Helfrich has already investigated axially symmetric minimizers in the aforementioned \cite{helfrich} and in a joint work with Deuling in \cite{helfrichDeuling}. \\

The first rigorous treatment of Willmore minimizers with isoperimetric constraint was perhaps carried out in \cite{Nagasawa}, where a one-parameter family of critical points bifurcating from the sphere is constructed. After the aforemetnioned article \cite{schygulla} from Schygulla, the isoperimetric problem for surfaces of higher genus was studied in \cite{KellerIso,kusnerISO,MondinoScharrerISO}.\\

Regarding the axisymmetric case, Choski and Veneroni \cite{choskiveneroni} study systems of axisymmetric surfaces $S=(\Sigma_1,...,\Sigma_N)$ where each $\Sigma_i$ is either an embedded sphere or an embedded torus that minimize the Helfrich energy under the generalized isoperimetric constraint 
$$6\sqrt\pi\frac{V[S]}{A[S]^{\frac32}}=\sigma\in(0,1]\hspace{.5cm}\textrm{where }A[S]=\sum_{i=1}^N A[\Sigma_i]\textrm{ and }V[S]=\sum_{i=1}^N V[\Sigma_i].$$
After the initial submission of this mansucript, the author was informed that Rupp \cite{Ruppiso} recently estblished the existence of critical Willmore spheres with prescribed isoperimetric ratio by analyzing the \emph{`Willmore flow with prescribed isoperimetric ratio'}. Restricting to axially symmetric initial datums, his results establish the existence of axially symmetric solutions to the Euler-Lagrange Equation \eqref{IntroPDE}.\\

\cite{scharrer2022embedded} studies axisymmetric tori minimizing the Willmore energy with an isoperimetric constraint. \\




This article is structured as follows. Section \ref{PreliminariesSection} collects the notation and formulas we require throughout this article. In Section \ref{diectmethod}, the existence of a weak minimizer is proven, and in Section \ref{regularityawayfromaxis}, its smoothness is established. Sections \ref{doublespheresection} and \ref{catenoidNeckSection} contain the proofs of Theorems \ref{doublesphereconvergence} and \ref{CatenoidNeckTheorem}. Technical arguments are moved to the \hyperref[Appendix1]{appendix}.
