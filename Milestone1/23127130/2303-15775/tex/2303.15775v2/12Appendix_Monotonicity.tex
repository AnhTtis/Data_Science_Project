\subsection{The Monotonicity Formula}
Let $\Sigma$ be a smooth $2$-dimensional manifold without boundary, $f\in C^1(\Sigma,\R^3)$ be an immersion and $\mu_\Sigma$ denote the surface measure on $\Sigma$ that is induced by $f$. For $p\in\R^3$, the \emph{muliplicity} of $p$ is defined as 
\begin{equation}\label{multiplicityDef}
\theta^2_{f}(p):=\lim_{r\rightarrow0^+}\frac{\mu_{\Sigma}(B_r(p))}{\pi r^2}.
\end{equation}
If $\Sigma$ is compact, this expression is well-defined and
\begin{equation}\label{multiplicitiyisNumberPreImages}
\theta^2_f(p)=|\set{x\in\Sigma\ |\ f(x)=p}|.
\end{equation}

The following result is due Simon \cite{SimonMonotonicity}. Its applicability in the present setting has essentially been established by Kuwert and Schätzle \cite{removability}, Appendix A.
\begin{lemma}[Monotonicity Formula]\label{monotonicityformula}
Let $\gamma\in \mathcal P \cap  C^\infty((0,1),\mathcal H^2)$. The surface $f_\gamma:[0,1]\times[0,2\pi)\rightarrow\R^3$ as defined in Equation \eqref{fgammadef} is $C^1$-immersed. Putting $\theta^2(p):=\theta^2_{f_\gamma}(p)$, we have
$$\frac1\pi\int_{\Sp^2}\left|
\frac14\vec H+\frac{(f_\gamma(x)-p)^\perp}{|f_\gamma(x)-p|^2}
\right| d\mu_{\Sigma_\gamma}(x)
\leq 
\frac1{4\pi}\mathcal W[f_\gamma]-\theta^2(p).$$
Here $\vec H$ is the almost everywhere defined mean curvature vector of $f_\gamma$ and $(f_\gamma(x)-p)^\perp$ is the component of $f_\gamma(x)-p$ that is orthogonal to $Df_\gamma(x)( T_x \Sp^2)$. 
\end{lemma}

\begin{proof}
We put $f:=f_\gamma$ and $\mu:=\mu_{\Sigma_\gamma}$. We use the monotonicity formula proved in Appendix A of \cite{removability}. We may apply their results as the mean curvature $H$ is in $L^2(\mu)$. Indeed, by the definition of the class $\mathcal P$ (see Definition \ref{classPdef})
$$
\int_{\Sp^2} H^2 d\mu
\leq 2\int_{\Sp^2} k_1^2+k_2^2 d\mu
<\infty.
$$
Using Equation (A3) from \cite{removability}, we get 
\begin{align}
    &\sigma^{-2}\mu(B_\sigma (x))
    +\int_{f^{-1}(B_\rho(p)\backslash B_\sigma(p))}\left|\frac14 \vec H+\frac{(f(x)-p)^\perp}{|f(x)-p|^2}\right|^2 d\mu\nonumber\\
    =&
    \rho^{-2}\mu(B_\rho (x))
    +\frac1{16}\int_{f^{-1}(B_\rho(p)\backslash B_\sigma(p))} \vec H^2 d\mu
    +
    R_{p,\rho}-R_{p,\sigma}\label{monotonicityformulaeq1}
\end{align}
where for $\alpha=\sigma,\rho$
$$R_{p,\alpha}:=\frac1{2\alpha^2}\int_{f^{-1}(B_\alpha(p))}\langle(x-p),\vec H\rangle d\mu.$$
The Equation after (A6) in \cite{removability} implies $R_{p,\sigma}\rightarrow 0$ as $\sigma\rightarrow 0^+$. We note that $\mu(B_\rho(p))\leq A[f]<\infty$ by definition of the class $\mathcal P$. Hence $\rho^{-2}\mu_{\Sigma}(B_\rho(p))\rightarrow 0$ for $\rho\rightarrow\infty$. So Equation (A13) in \cite{removability} implies $\lim_{\rho\rightarrow \infty}R_{p,\rho}=0$. 
Letting $\rho\rightarrow\infty$ and $\sigma\rightarrow 0^+$ in Equation \eqref{monotonicityformulaeq1} then gives
$$\liminf_{\sigma\rightarrow 0^+}\left[
\sigma^{-2}\mu_{\Sigma}(B_\sigma (x))
    +\int_{\Sp^2\backslash f^{-1}(B_\sigma(p))}\left|\frac14 \vec H+\frac{(f(x)-p)^\perp}{|f(x)-p|^2}\right|^2 d\mu
\right]
\leq 
\frac14\mathcal W[f].
$$
The lemma follows by using Fatou's lemma and Equation \eqref{multiplicityDef}.
\end{proof}

\begin{lemma}\label{SphereOrInvertedCat}
Let $\gamma\in\mathcal P\cap C^\infty((0,1))$ and denote by $\vec H$ the almost everywhere defined mean curvature vector of $f_\gamma$. Suppose that there exists $p\in\R^3$ such that
\begin{equation}\label{ssumptioncritical}
\frac14\vec H(s)=-\frac{(f_\gamma(s,\theta)-p)^\perp}{|f_\gamma(s,\theta)- p|^2}.
\end{equation}
Then $\Sigma_\gamma$ is either a sphere or the inversion of a scaled and vertically translated catenoid. In particular, only the second case is possible if $\mathcal W[\Sigma_\gamma]=8\pi$.
\end{lemma}

\begin{proof}
By vertical translation of $\gamma$ and rotational invariance, we can assume $p=Re_1$ for some $R\geq 0$. Let $L:=L[\gamma]$. Using Equation \eqref{normalsprofilecurve}, we compute 
\begin{align*} 
(f_\gamma(s,\theta)-Re_1)\cdot n(s,\theta)&=\begin{bmatrix}
\gamma_1\cos\theta-R\\
\gamma_1\sin\theta\\
\gamma_2
\end{bmatrix}
\cdot
\frac1L\begin{bmatrix}
-\dot\gamma_2\cos\theta\\
-\dot\gamma_2\sin\theta\\
\dot\gamma_1
\end{bmatrix}
=\frac{-\gamma_1\dot\gamma_2+\dot\gamma_1\gamma_2+R\dot\gamma_2\cos\theta}L.
\end{align*}
A quick computation shows $|f_\gamma(s,\theta)-Re_1|^2=\gamma_1^2+\gamma_2^2+R^2-2R\gamma_1\cos\theta$. Inserting into Equation \eqref{ssumptioncritical}, we get
\begin{equation}\label{newequationLemmaA13}
\frac14 H(s)\left(\gamma_1^2+\gamma_2^2+R^2-2R\gamma_1\cos\theta\right)
=
-\frac{-\gamma_1\dot\gamma_2+\dot\gamma_1\gamma_2+R\dot\gamma_2\cos\theta}L.
\end{equation}
\paragraph{The case $R>0$}\ \\
Comparing the coefficients of $\cos\theta$ in Equation \eqref{newequationLemmaA13}and using $R\neq 0$ we derive\\ $\frac{\dot\gamma_2}L=\frac12H(s)\gamma_1(s)$. By using Equation \eqref{ArcLength_H_Identity}, we get
$$
\frac{2\dot\gamma_2^2}L
=H\gamma_1\dot\gamma_2
=-\frac{\gamma_1\ddot\gamma_1}L+\frac{\dot\gamma_2^2}{L}.
$$
We use $\dot\gamma_2^2=L^2-\dot\gamma_1^2$ and get $-\ddot\gamma_1\gamma_1=\dot\gamma_2^2=L^2-\dot\gamma_1^2$. Next, we put $x(t):=\gamma_1(\frac tL)$. $x$ satisfies the following problem
\begin{equation}\label{xproblem}
    \left\{\begin{aligned}
    \ddot x x&=\dot x^2-1,\\
    x(0)&=x(L)=0,\\
    \dot x(0)&=1,\\
    x(t)&>0\hspace{.5cm}\textrm{for }t\in(0,L).
\end{aligned}\right.
\end{equation}
Since $|\dot\gamma(t)|\leq L$ we have $|\dot x|\leq 1$. Suppose that there exists $t_0\in(0,L)$ such that $\dot x(t_0)^2=1$. Then by the Picard-Lindelöff theorem $x(t)=x(t_0)\pm (t-t_0)$. This, however, contradicts  $x(0)=x(L)=0$. So $x^2(t)<1$ for all $t\in(0,L)$ and hence the following computation is admissible for $t\in(0,L)$:
$$-\frac12\frac d{dt}\ln(1-\dot x^2)=\frac{\dot x\ddot x}{1-\dot x^2}=-\frac{\dot x}x=-\frac d{dt}\ln(x)$$
Integrating and subsequent exponentiation proves that there exists a positive constant $K$ such that 
$1-\dot x^2=K x^2$. Differentiating this identity gives $\dot x(\ddot x+Kx)=0$ Since $\dot x(0)=1$ we have $\dot x\neq 0$ for small times $t\in[0, t_0]$ and get 
$$
x(t)=a\sin(\sqrt K t)+b\cos(\sqrt K t)\hspace{.5cm}\textrm{for }t\in[0,t_0].$$
As $x(t)>0$, Equation \eqref{xproblem} and the Picard-Lindelöff theorem imply that this formula is valid not only on $[0, t_0]$ but for all $t\in[0,L]$. Since $x(0)=x(L)=0$ we get $b=0$ and $\sqrt K=L^{-1}n\pi$ for some $n\in\N$. Moreover, as $x>0$ on $(0,L)$ and $\dot x(0)=1$, we get 
$$x(t)=\frac L\pi\sin\left(\frac \pi L t\right)\hspace{.5cm}\textrm{for all $t\in[0, L]$}.$$
Let $y(t):=\gamma_2(\frac tL)$. As $\dot y^2=1-\dot x^2=\sin^2(\frac\pi L t)$, we have $\dot y(t)=\pm\sin(\frac\pi L t)$ and hence
$$y(t)=y(0)\mp\frac L\pi\cos\left(\frac \pi L t\right).$$
The formulas for $x(t)$ and $y(t)$ imply that $\gamma$ parameterises a circle of radius $\frac L\pi$ with center at $(0, y(0))$. In particular $\mathcal W[\gamma]=4\pi$.\\

\paragraph{The Case $R=0$}\ \\
Let $I:\R^3\backslash\set0\rightarrow\R^3\backslash\set 0$, $I(x):=\frac{x}{|x|^2}$. Then $I^*\delta=\frac1{|x|^4}\delta$ and  
$$H[I\circ f_\gamma]=H[f_\gamma, I^*\delta]=|f_\gamma|^2H[f_\gamma]+4\langle f_\gamma, n\rangle=0.$$
This shows that the surface defined by the profile curve $\Gamma(s):=\frac{\gamma(s)}{|\gamma(s)|^2}$ is a minimal surface. Let $\alpha$ denote the maximal solution to the problem
$$
\left\{
\begin{aligned}
    \dot\alpha(t)&=\frac{|\gamma(\alpha(t))|^2}{L},\\
    \alpha(0)&=\frac12.
\end{aligned}
\right.
$$
We define $\Gamma^*(t):=\Gamma(\alpha(t))$ and compute 
$$|\dot\Gamma^*(s)|=\left|\left(\frac{\dot\gamma}{|\gamma|^2}-2\frac{\langle\gamma,\dot\gamma\rangle}{|\gamma|^4}\gamma\right)\dot\alpha\right|=1.$$
Consequently, $\Gamma^*$ is parameterized by arc length and satisfies $H[\Gamma^*]=0$. Following the arguments after Equation \eqref{LimitH0Equations} we get $\Gamma_1^*(t)=\sqrt{a+bt+t^2}$ and $\dot\Gamma_2^*(t)\Gamma_1^*(t)=c$ for constants $a>0$ and $b,c\in\R$. By potentially redefining $\Gamma^*(t)\rightarrow\Gamma^*(-t)$ we may further assume $b\geq 0$. Since $|\dot\Gamma^*|=1$ we have 
$$1=(\dot\Gamma_1^*)^2+(\dot\Gamma_2^*)^2=\frac{c^2+(t+\frac b2)^2}{(\Gamma_1^*)^2}
\hspace{.5cm}\textrm{and hence}\hspace{.5cm}
t^2+bt+a=(t+\frac b2)^2+c^2.$$
If $c=0$ we deduce $\Gamma^*(t)=(t+\frac b2,\Gamma_2^*(0))$ for $t\geq -\frac b2$. Hence $\Sigma_{\Gamma^*}$ is part of a a plane $P$ orthogonal to the $x_3$-axis and consequently $\Sigma_\gamma$ is part of a sphere or, if $0\in P$, part of $P$. The second case is impossible. Indeed, if $\gamma(s)=(\gamma_1(s), c_0)$ for some $c_0\in\R$, we get $L=|\dot\gamma|=|\dot\gamma_1|$. Since $\gamma_1(0)=0$ and $\gamma_1(t)\geq 0$ for all $t\in[0,1]$ we get $\gamma_1(t)=Lt$, which contradicts $\gamma_1(1)=0$. So, if $c=0$, $\Sigma_\gamma$ is part of a sphere and by the definition of the class $\mathcal P$, it then follows that $\Sigma_\gamma$ is a sphere and hence $\mathcal W[\Sigma_\gamma]=4\pi$. If $c\neq 0$, we have 
\begin{align*}
\Gamma_1^*(t)&=\sqrt{\left(t+\frac b2\right)^2+c^2},\\
    \Gamma_2^*(t)&=\Gamma_2^*(0)+\int_0^t \frac {cdt}{\sqrt{(t+\frac b2)^2+c^2}}=\Gamma_2^*(0)-|c|\operatorname{asinh}(\frac b{2c})+|c|\operatorname{asinh}(\frac tc+\frac b{2c}).
\end{align*}
This is the profile curve of a scaled and vertically translated catenoid. Hence, $\Sigma_\gamma$ is the inversion of a scaled and vertically translated catenoid and $\mathcal W[\Sigma_\gamma]=8\pi$.
\end{proof}



