\section{Preliminaries}\label{PreliminariesSection}

\paragraph{General Notation}\ \\
For $k\in\N_0\cup\set\infty$ and $U\subset\R^n$ open, we denote the set of $C^k$-functions with compact support in $U$ by $C^k_0(U)$.
The right half plane in $\R^2$ is denoted by
$$\mathcal H^2:=\set{(x_1,x_2)\in\R^2\ |\ x_1\geq 0}.$$
If $Q$ is some functional, we put 
$$\delta Q[x]v:=\frac d{dt}\bigg|_{t=0} Q[x+tv].$$


\paragraph{Immersions}\ \\
Let $f:\Sp^2\rightarrow\R^3$ be a sufficiently regular immersion and $n$ denote a regular unit normal vector field along $f$. The first and second fundamental forms are respectively denoted by
$$g_{ij}:=\langle\partial_i f,\partial_j f\rangle 
\hspace{.5cm}\textrm{and}\hspace{.5cm}
h_{ij}:=\langle\partial_{ij} f, n\rangle.$$
The surfaces measure on $\Sp^2$ induced by $f$ is denoted by $\mu_f$. In local coordinates $d\mu_f=\sqrt{\det g}dx$. Let $k_1$ and $k_2$ denote the principal curvatures of $f$ -- that is, the eigenvalues of the Weingarten operator $g^{-1}h$. The mean curvature $H$ and the Gauß curvature $K$ are given by
$$H=k_1+k_2
\hspace{.5cm}\textrm{and}\hspace{.5cm}
K=k_1k_2.$$
For example, considering the inclusion $\imath:\Sp^2\hookrightarrow\R^3$ with the interior normal field $n=-\imath$, we get $H[f]=2$ and $K[f]=1$.\\
 

\paragraph{Functionals}\ \\
Let $f:\Sp^2\rightarrow\R^3$ be a sufficiently regular immersion. The Willmore energy of $f$ is defined as 
\begin{equation}\label{willmoreenergydef}
\mathcal W[f]:=\frac14\int_{\Sp^2} H[f]^2 d\mu_f.
\end{equation}
For any of the two choices of continuous normal vector field $n$ along $f$, the surface area, the volume and the isoperimetric ratio of $f$ are respectively defined as
\begin{equation}\label{areavoldef}
A[f]:=\int_{\Sp^2} d\mu_f,
\hspace{.5cm}
V[f]:=\left|\int_{\Sp^2}f_2 \langle n, e_2\rangle d\mu_f\right|
\hspace{.5cm}\textrm{and}\hspace{.5cm}
\mathcal I[f]:=6\sqrt\pi\frac{V[f]}{A[f]^{\frac32}}.
\end{equation}
To justify the definition of volume, we have the following lemma:
\begin{lemma}\label{generalvolumelemma}
Let $f\in C^1(\Sp^2,\R^3)$ be an embedding. Then there exists a bounded $C^1$-domain $\Omega\subset\R^3$ with $\partial\Omega=f(\Sp^2)$ and which satisfies $|\Omega|=V[f]$.
\end{lemma}
\begin{proof}
The existence of $\Omega$ is due to the Jordan-Brouwer separation theorem (see e.g. \cite{MayerBook}, Corollary 5.24). Let $n_{\operatorname{int}}$ denote the interior unit normal along $\partial\Omega$. Since $\Omega\in C^1$, we can use Gauß's theorem to compute
$$|\Omega|
=-\int_{\partial\Omega}x_2\langle n_{\operatorname{int}}, e_2\rangle dS
=-\int_{\Sp^2}f_2\langle n_{\operatorname{int}}, e_2\rangle d\mu_f.$$
This establishes Equation \eqref{areavoldef} when $n$ in the interior normal and thereby also for the exterior normal $n_{\operatorname{ext}}=-n_{\operatorname{int}}$.
\end{proof}


\paragraph{First Variation of the Willmore Energy}\ \\
Let $f:\Sigma\rightarrow\R^3$ be a smoothly immersed surface and $\Phi:(-\epsilon_0,\epsilon_0)\rightarrow \R^3$ be a smooth variation of $f$. Denote by $n$ the unit normal field used to define the scalar mean curvature $H=\langle\vec H, n\rangle$. The following formula is, for example, derived in \cite{AK}, Theorem 1. 
\begin{equation}\label{classicalformula}
\frac{d}{d\epsilon}\bigg|_{\epsilon=0}\mathcal W[\Phi(\epsilon)]=\int_{\Sigma}W[f]\langle \Phi'(0), n\rangle d\mu_f+\int_{\partial\Sigma}\omega(\eta) dS_f.
\end{equation}
In this formula, $\eta$ is the interior unit normal along $\partial\Sigma$ and splitting the variational vector field $\Phi'(0)$ into a normal and a tangential part by writing $\Phi'(0)=\varphi n + Df X$, the Willmore operator $W[f]$ and the one-form appearing in Equation \eqref{classicalformula} are given by
\begin{align}
    W[f]&=\frac12\left(\Delta_g H+\frac12H(H^2-4K)\right),\label{WillmoreOperator}\\
%--------------
\omega(\eta)&=\frac12\left(\varphi\frac{\partial H}{\partial\eta}-H\frac{\partial\varphi}{\partial\eta}-\frac12 H^2 \langle Df\eta, Df X\rangle\right).\label{WillmoreBoundaryterm}
\end{align}

\subsection{Graphical Surfaces}
Let $r_0>0$ and $u:[0,r_0)\rightarrow\R$ be sufficiently regular. We consider the axisymmetric surface $\Sigma_u$ generated by rotating the graph of $u$ around the $x_3$-axis:
$$f_u:[0,r_0)\times[0,2\pi)\rightarrow\R^3,\ f_u(r,\theta):=(r\cos\theta,r\sin\theta, u(r))$$
We collect the geometric data of $\Sigma_u$. The first fundamental form is given by
\begin{equation}\label{graphmetric}
g(r,\theta)=\begin{bmatrix}
1+u'(r)^2 & \\
 & r^2
\end{bmatrix}
\hspace{.5cm}\textrm{and}\hspace{.5cm}\sqrt{\det g}=r\sqrt{1+u'(r)^2}.
\end{equation}
Next, we introduce a normal field $n$, which we choose to coincide with the interior unit normal when $u(r)=\sqrt{1-r^2}$ -- that is when $\Sigma_u$ is a sphere.
$$n(r,\theta):=
\frac{1}{\sqrt{1+u'(r)^2}}
\begin{bmatrix}
u'(r)\cos\theta\\
u'(r)\sin\theta\\
-1
\end{bmatrix}$$
Using this normal, the scalar second fundamental form is given by 
\begin{equation}\label{graphsecondff}
h=-\frac1{\sqrt{1+u'(r)^2}}\begin{bmatrix}
u''(r) & \\
 & ru'(r)
\end{bmatrix}.
\end{equation}
Combining Equations \eqref{graphmetric} and \eqref{graphsecondff}, we obtain the following formulas for the two principal curvatures:
\begin{equation}\label{graphicalPrincipalCurvatures}
k_1=-\frac{u''(r)}{\sqrt{1+u'(r)^2}^3}
\hspace{.5cm}\textrm{and}\hspace{.5cm}
k_2=-\frac{u'(r)}{r\sqrt{1+u'(r)^2}}
\end{equation}
Using these, it is readily checked that the mean curvature $H$ of $\Sigma_u$ is given by
\begin{equation}\label{graphH}
H=k_1+k_2=-\frac1r\frac{d}{dr}\left[r\frac{u'(r)}{\sqrt{1+u'(r)^2}}\right].
\end{equation}

\subsection{Profile Curves}
Let $\gamma:[0,1]\rightarrow\mathcal H^2$ be a sufficiently regular curve. We begin by recalling that the arc length of $\gamma$ is defined as 
\begin{equation}
    L[\gamma]:=\int_0^1|\dot\gamma(t)|dt.
\end{equation}
We consider the surface $\Sigma_\gamma$ generated by rotating the graph of $\gamma$ around the $x_2$-axis. Concretely, it is given by the parameterization
\begin{equation}\label{fgammadef}
f_\gamma:[0,1]\times[0,2\pi)\rightarrow\R^3,\ f_\gamma(t,\theta):=\begin{bmatrix}
\gamma_1(t)\cos\theta\\
\gamma_1(t)\sin\theta\\
\gamma_2(t)
\end{bmatrix}.
\end{equation}
In the following, we collect the geometric data of $\Sigma_\gamma$. The first fundamental form is given by
\begin{equation}\label{metricprofilecurve}
g(t,\theta)=\begin{bmatrix}
|\dot\gamma(t)|^2 & \\
 & \gamma_1(t)^2
\end{bmatrix}
\hspace{.5cm}\textrm{and}\hspace{.5cm}\sqrt{\det g}=\gamma_1(t)|\dot\gamma(t)|.
\end{equation}
We denote the surface element on $\Sigma_\gamma$ by $d\mu_{\Sigma_\gamma}:=\gamma_1|\dot\gamma_1|dtd\theta$. Next, we define normal fields $n$ along $\Sigma_\gamma$ and $\nu$ along $\gamma$. 
\begin{equation}\label{normalsprofilecurve}
n(t,\theta)=\frac1{|\dot\gamma(t)|}\begin{bmatrix}
-\dot\gamma_2(t)\cos\theta\\
-\dot\gamma_2(t)\sin\theta\\
\dot\gamma_1(t)
\end{bmatrix}
\hspace{.5cm}\textrm{and}\hspace{.5cm}
\nu(t)=\frac1{|\dot\gamma(t)|}\begin{bmatrix}
-\dot\gamma_2(t)\\
\dot\gamma_1(t)
\end{bmatrix}
\end{equation}
This normal is chosen, such that it agrees with the interior normal along $\Sigma_\gamma$ when $\gamma(t)=(\cos(\pi t), \sin(\pi t))$ -- that is when $\gamma$ parameterizes a circle run through counterclockwise. Using this normal, we get the following formula for the scalar second fundamental form of $\Sigma_\gamma$:
\begin{equation}\label{curvesecondff}
h(t,\theta):=\frac1{|\dot\gamma(t)|}\begin{bmatrix}
    \ddot\gamma_2\dot\gamma_1-\ddot\gamma_1\dot\gamma_2 & \\
     & \gamma_1\dot\gamma_2
\end{bmatrix}
\end{equation}
Combining Equations \eqref{metricprofilecurve} and \eqref{curvesecondff} gives the following formulas for the two principal curvatures:
\begin{equation}\label{principalcurvatures} 
k_1=\frac{\ddot\gamma_2\dot\gamma_1-\ddot\gamma_1\dot\gamma_2}{|\dot\gamma|^3}
\hspace{.5cm}\textrm{and}\hspace{.5cm}
k_2=\frac{\dot\gamma_2}{\gamma_1|\dot\gamma|}
\end{equation}

\paragraph{Functionals}\ \\
Using Equations \eqref{willmoreenergydef}, \eqref{areavoldef}, \eqref{metricprofilecurve} and \eqref{normalsprofilecurve}, we get the following formula for the Willmore energy, the area and the volume of $\Sigma_\gamma$:
\begin{align}
    \mathcal W[\gamma]&:=\mathcal W[\Sigma_\gamma]=\frac\pi2\int_0^1 (k_1(t)+k_2(t))^2 \gamma_1(t)|\dot\gamma(t)| dt\label{willmoredef}\\
    A[\gamma]&:=A[\Sigma_\gamma]=2\pi\int_0^1 \gamma_1(t)|\dot\gamma(t)|dt\label{areadef}\\
    V[\gamma]&:=V[\Sigma_\gamma]=\pi\left|\int_0^1 \gamma_1(t)^2\dot\gamma_2(t)dt\right|\label{volumedef}
\end{align}
We also put $\mathcal I[\gamma]:=\mathcal I[\Sigma_\gamma]$. In view of Lemma \ref{generalvolumelemma}, we have the following result:
\begin{lemma}\label{volumegauss}
    Let $\gamma\in C^1([0,1],\mathcal H^2)$ be injective, $\dot\gamma(t)\neq 0$ for all $t\in[0,1]$ and $\gamma_1(0)=\gamma_1(1)=0$ as well as $\dot\gamma_2(0)=\dot\gamma_2(1)=0$. Then $\Sigma_\gamma$ is the boundary of a bounded $C^1$ domain $\Omega$ and $|\Omega|=V[\gamma]$. 
\end{lemma}

\paragraph{Arc Length Parameterization}\ \\
A regular curve $\gamma:[0,1]\rightarrow\mathcal H^2$ is \emph{parameterized proportional to arc length} if $|\dot\gamma(t)|=L[\gamma]=:L$ for all $t\in[0,1]$. These curves enjoy several beneficial properties. First 
\begin{equation}\label{velocity_acc_Perp_Eq}
    0=\frac12\frac{d}{dt}|\dot\gamma(t)|^2=\langle\dot\gamma(t),\ddot\gamma(t)\rangle=\dot\gamma_1(t)\ddot\gamma_1(t)+\dot\gamma_2(t)\ddot\gamma_2(t).
\end{equation}
Considering Equation \eqref{principalcurvatures}, it is then straightforward to check that 
\begin{equation}\label{ArcLength_H_Identity}
\dot\gamma_1 H= \frac{\ddot\gamma_2}L+\frac{\dot\gamma_1\dot\gamma_2}{\gamma_1 L}
\hspace{.5cm}\textrm{and}\hspace{.5cm}
\dot\gamma_2 H= -\frac{\ddot\gamma_1}L+\frac{\dot\gamma_2^2}{\gamma_1 L}.
\end{equation}
Additionally, it is readily checked that
\begin{equation}\label{ArcLength_K_Identity}
    k_1^2=\frac{|\ddot\gamma|^2}{L^4}
    \hspace{.5cm}\textrm{and}\hspace{.5cm}
    K=k_1k_2=-\frac{\ddot\gamma_1}{\gamma_1 L^2}.
\end{equation}



