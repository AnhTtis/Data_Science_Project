\section{General Regularity Theory}\label{GeneralRegularityAppendix}
Let $U\subset\R$ be an open interval and $\gamma\in W^{1,\infty}(U,\mathcal H^2)\cap W^ {2,2}(U,\mathcal H^2)$. For $\alpha,\beta\in\R$ we define the functional 
\begin{equation}\label{IFUnctionalPDE}
I(\phi)
:=
\alpha\int_0^1 \gamma_1^2(t)\dot\phi_2(t)+2\gamma_1(t)\dot\gamma_2(t)\phi_1(t)dt
+
\beta\int_0^1|\dot\gamma(t)|\phi_1(t)+\frac{\gamma_1(t)\langle\dot\phi(t),\dot\gamma(t)\rangle}{|\dot\gamma(t)|}dt.
\end{equation}
We assume that $\gamma$ satisfies the equation
\begin{equation}\label{regappendixweakeq}
\delta\mathcal W[\gamma]\phi=I(\phi)\hspace{.5cm}\textrm{for all }\phi\in C^\infty_0(U).
\end{equation}
The goal of this section is to prove the following regularity theorem:

\begin{theorem}[Regularity Theorem]\label{GeneralRegularityTheorem}
    Let $\gamma\in W^{1,\infty}(U,\mathcal H^2)\cap W^{2,2}(U,\mathcal H^2)$ satisfy Equation \eqref{regappendixweakeq}. Further, we assume that $|\dot\gamma(t)|=L>0$ and
    \begin{align}
    &|\alpha|+|\beta|+L+L^{-1}+\|\gamma\|_{C^0(U)}+\|\ddot\gamma\|_{L^2(U)}+\mathcal W[\gamma]\leq M,\label{uniformestimate}\\
    &\kappa(\gamma;U):=\inf_{t\in U}\gamma_1(t)>0.\label{kappadef}
    \end{align}
    Then $\gamma\in C^\infty(U)$ and $\|\gamma\|_{C^m(U)}\leq C(M, \kappa(\gamma;U),U,m)$. The constant is increasing in the first and decreasing in the second slot. 
\end{theorem}

The proof is split into several lemmas. We remark that Eichmann and Grunau have carried out similar arguments in \cite{eichmann} for the Willmore equation without the isoperimetric Lagrange multiplier. 




\begin{lemma}\label{regularitystep1}
    Under the assumptions of Theorem \ref{GeneralRegularityTheorem} $\gamma\in W^{2,\infty}(U)$, $H\in L^\infty(U)$ and 
    $$\|\gamma\|_{W^{2,\infty}(U)}+\|H\|_{L^\infty(U)}\leq C(M,\kappa(\gamma;U),U).$$
\end{lemma}
\begin{proof}
Let $U=(\tau_1,\tau_2)$. Throughout the proof $C$ will denote a constant that is allowed to depend on $M$, $\kappa(\gamma;U)$ and $U$ and that may change from line to line.\\
\ \\
\noindent
\textbf{Step 1: Constructing an Appropriate Test Function}\ \\
Given $\varphi=(\varphi_1,\varphi_2)\in C_0^\infty(U,\R^2)$ define 
$$\zeta(t):=\int_{\tau_1}^{t} \int_{\tau_1} ^s \varphi(r)dr ds.$$
Let $\tau_1<r_1<r_2<\tau_2$ such that $\operatorname{supp}(\varphi)\subset[r_1,r_2]$. Then $\zeta(t)=0$ when $t\in[\tau_1,r_1]$ and $\zeta(t)=a+bt$ when $t\in[r_2,\tau_2]$ where 
$$a=\int_{\tau_1}^{\tau_2}\int_{\tau_1}^s\varphi(r)dr ds-b\tau_2\hspace{.5cm}\textrm{and}\hspace{.5cm} b=\int_{\tau_1}^{\tau_2} \varphi(r)dr.$$
We choose a smooth function $\chi\equiv 0$ on $[\tau_1,\tau_1+\frac13(\tau_2-\tau_1)]$ and $\chi\equiv 1$ on $[\tau_1+\frac23(\tau_2-\tau_1),\tau_2]$ and define 
\begin{equation}\label{phiconstruction}
\phi(t):=\zeta(t)-\chi(t)(a+bt).
\end{equation}
It is easy to see that $\phi\in C^ \infty_0(U)$ and that $\|\phi\|_{C^1(U)}\leq C\|\varphi\|_{L^1(U)}$. \\

\noindent
\textbf{Step 2: Weak Euler-Lagrange Equation}\ \\
We compute 
\begin{equation}\label{weakeulerLagrange01}
    I(\phi)=\delta\mathcal W[\gamma]\phi=\pi\int_{\tau_1}^ {\tau_2} H\left(
    \delta H\phi
    \right)\gamma_1|\dot\gamma|dt
    +
    \frac\pi2\int_{\tau_1}^ {\tau_2} H^2 \left(
    |\dot\gamma|\phi_1+\gamma_1\frac{\langle\dot\gamma,\dot{\phi}\rangle}{|\dot\gamma|}
    \right)dt.
\end{equation}
Considering the definition of $I(\phi)$ it is easy to see that $|I(\phi)|\leq\|\phi\|_{C^1}$.
Using Estimates \eqref{uniformestimate} and \eqref{kappadef} we estimate 
\begin{align}
\left|\int_{\tau_1}^ {\tau_2} H^2
\left(|\dot\gamma|\phi_1+\gamma_1 \frac{\langle\dot\gamma,\dot{\phi}\rangle}{|\dot\gamma|} \right)dt\right|
    \leq
    &\left(L+\|\gamma_1\|_{C^0}\right) \| \phi\|_{C^1}\int_{\tau_1}^ {\tau_2} H^2  dt\nonumber\\
    %----------
\leq & \frac{C\|\phi\|_{C^1}}{2\pi L\kappa(\gamma;U)}\int_{\tau_1}^ {\tau_2} H^22\pi\gamma_1 |\dot\gamma|dt\nonumber\\
\leq &C\|\phi\|_{C^1}.\label{int2estimate0}
\end{align}
We recall that $H=k_1+k_2$. Using the formulas for $k_1$ and $k_2$ from Equation \eqref{principalcurvatures} it is easy to derive the following equations:
\begin{align}
    \delta k_1\phi&=\frac{\ddot{\phi}_2\dot\gamma_1+\ddot\gamma_2\dot{\phi}_1-\ddot{\phi}_1\dot\gamma_2-\ddot\gamma_1\dot{\phi}_2}{L^3}
    -
    3\frac{\ddot\gamma_2\dot\gamma_1-\ddot\gamma_1\dot\gamma_2}{L^4}\frac{\langle\dot\gamma,\dot{\phi}\rangle}{L}\label{vark1}\\
    %---------------------
    \delta k_2\phi&=\frac{\dot{\phi}_2}{\gamma_1 L}-\frac{\dot\gamma_2}{\gamma_1^2L}\phi_1-\frac{\dot\gamma_2}{\gamma_1L^2}\frac{\langle\dot\gamma,\dot{\phi}\rangle}{L}\label{vark2}
\end{align}
Next, we estimate the terms in $\delta H\phi$ that do not contain $\ddot{\phi}$. Using $|\dot\gamma|=L$ and Estimate \eqref{uniformestimate} we get
\begin{equation}\label{kestimates}
|\delta k_2\phi|
\leq
C\|\phi\|_{C^1}
\hspace{.5cm}\textrm{and}\hspace{.5cm}
\left|\frac{\ddot\gamma_2\dot{\phi}_1-\ddot\gamma_1\dot{\phi}_2}{L^3}
    -
    3\frac{\ddot\gamma_2\dot\gamma_1-\ddot\gamma_1\dot\gamma_2}{L^4}\frac{\langle\dot\gamma,\dot\phi\rangle}{L}\right|
\leq 
C\|\phi\|_{C^1}|\ddot\gamma|.
\end{equation}
Using Equation \eqref{weakeulerLagrange01} and inserting Estimates \eqref{int2estimate0} and \eqref{kestimates} yields
\begin{align} 
\left|\int_{\tau_1}^ {\tau_2} H(\ddot{\phi}_2\dot\gamma_1-\ddot{\phi}_1\dot\gamma_2)\gamma_1|\dot\gamma|dt\right|
\leq 
&C\|\phi\|_{C^1}\left[1+\int_{\tau_1}^ {\tau_2} |H|\gamma_1|\dot\gamma_1|dt+\int_{\tau_1}^{\tau_2} |H|\ |\ddot\gamma|\gamma_1|\dot\gamma_1|dt\right]\nonumber \\
\leq
&C\|\phi\|_{C^1}(1+\|\ddot\gamma\|_{L^2{((\tau_1,\tau_2))}})\nonumber \\
\leq 
&C\|\phi\|_{C^1}.\label{step2finalestimate0}
\end{align}
Here we have first used the Hölder inequality and the bound on the Willmore energy from Estimate \eqref{uniformestimate}.\\

\noindent
\textbf{Estimate Auxiliary Terms}\ \\
For the moment, we abbreviate $\xi:=\chi(a+bt)$. We use Estimate \eqref{uniformestimate} to estimate 
\begin{align}
&\left|\int_{\tau_1}^ {\tau_2} H(\ddot\xi_2\dot\gamma_1-\ddot\xi_1\dot\gamma_2)\gamma_1|\dot\gamma|dt\right|\nonumber
\leq 
2\|\ddot\xi\|_{C^0}\int_{\tau_1}^ {\tau_2}  |H|\ |\dot\gamma|^2\gamma_1dt\\
\leq &C\|\ddot\xi\|_{C^0}\sqrt{\mathcal W[\gamma]}
\leq C\|\xi\|_{C^2}.\label{smoothestimate}
\end{align}
We wish to use this estimate to replace $\phi$ in Estimate \eqref{step2finalestimate0} with $\zeta$. To do so, we recall $\phi=\zeta-\chi(a+bt)$. Using $\ddot\zeta=\varphi$, Estimate \eqref{smoothestimate} as well as the fact that $|a|+|b|\leq C\|\varphi\|_{L^1}$ we get
\begin{equation}\label{endresultstep2}
\left|\int_{\tau_1}^ {\tau_2} H(\varphi_2\dot\gamma_1-\varphi_1\dot\gamma_2)\gamma_1|\dot\gamma|dt\right|
\leq 
C\|\varphi\|_{L^1((0,1))}.
\end{equation}
In particular, choosing $\varphi=(\varphi_1,0)$ we find that for all $\varphi_1\in C^\infty_0((\tau_1,\tau_2))$ we have 
\begin{equation}\label{firstregest}
\left|\int_{\tau_1}^ {\tau_2} H\varphi_1\dot\gamma_2\gamma_1|\dot\gamma|dt\right|
\leq 
C\|\varphi_1\|_{L^1((\tau_1,\tau_2))}.
\end{equation}

\noindent
\textbf{Regularity Improvement}\ \\
Estimate \eqref{firstregest} allows us to extend the integral expression  $\int_{\tau_1}^ {\tau_2} H\dot\gamma_2\gamma_1|\dot\gamma|\varphi_1dt$ to a linear and continuous functional $\Lambda: L^1((\tau_1,\tau_2))\rightarrow\R$. As $(L^1((\tau_1,\tau_2)))'=L^\infty((\tau_1,\tau_2))$ we may deduce  
 $H\dot\gamma_2\gamma_1|\dot\gamma|\in L^\infty((\tau_1,\tau_2))$ with 
 $$\|H\dot\gamma_2 \gamma_1|\dot\gamma|\ \|_{L^\infty((\tau_1,\tau_2))}\leq C.$$
 As $\gamma_1\geq\kappa(\gamma;U)$ on $(\tau_1,\tau_2)$ and $|\dot\gamma|=L$ we obtain $\|H\dot\gamma_2\|_{L^\infty((\tau_1,\tau_2))}\leq C$. Using $\varphi=(0,\varphi_2)$ one can derive a similar estimate for $\|H\dot\gamma_1\|_{L^\infty((\tau_1,\tau_2))}$. Combining both estimates, we get
$$\|H\|_{L^\infty((\tau_1,\tau_2))}=\frac{\|H\dot\gamma\|_{L^\infty((\tau_1,\tau_2))}}{L}\leq\frac1L(\|H\dot\gamma_1\|_{L^\infty((\tau_1,\tau_2))}+\|H\dot\gamma_2\|_{L^\infty((\tau_1,\tau_2))})\leq C.$$
Using the identities from Equation \eqref{ArcLength_H_Identity}, we deduce $\gamma\in W^{2,\infty}((\tau_1,\tau_2))$ and the estimate $\|\ddot\gamma\|_{L^\infty((\tau_1,\tau_2))}\leq C$. 
\end{proof}


\begin{lemma}\label{regularitystep2}
        Under the assumptions of Theorem \ref{GeneralRegularityTheorem} $\gamma\in W^{3,\infty}(U)$, $H\in W^{1,\infty}(U)$ and 
    $$\|\gamma\|_{W^{3,\infty}(U)}+\|H\|_{W^{1,\infty}(U)}\leq C(M,\kappa(\gamma;U),U).$$
\end{lemma}
\begin{proof}
Let again $U=(\tau_1,\tau_2)$. As in the proof of Lemma \ref{regularitystep1} $C$ always refers to a constant depending on $M$, $\kappa(\gamma;U)$ and $U$. Also, the rest of the proof is very similar to the proof of Lemma \ref{regularitystep1} and we only sketch the parts that need to be altered. First, the test function has to get modified. Given $\varphi=(\varphi_1,\varphi_2)\in C_0^ \infty((\tau_1,\tau_2),\R^2)$ we put
$$\zeta(t):=\int_0^ t\varphi(r)dr.$$
Let $[r_1,r_2]=\operatorname{supp}\varphi$. Then $\zeta(t)=0$ for $t\in[\tau_1, r_1]$ and $\zeta(t)=a$ when $t\in[r_2,\tau_2]$ where
$$a=\int_{\tau_1}^ {\tau_2}\varphi(r) dr.$$
We choose the same smooth function $\chi$ as in the proof of Lemma \ref{regularitystep1} and put $\phi(t):=\zeta(t)-a\chi(t)$. It is easy to see that $\phi\in C^ \infty_0(U)$ and that $\|\phi\|_{C^ 0}\leq C\|\varphi\|_{L^1}$ and $\|\dot\phi\|_{L^1}\leq C\|\varphi\|_{L^1}$. 
Testing the weak Euler-Lagrange equation with $\phi$ as we did in Equation \eqref{weakeulerLagrange01} and using the estimates derived in Lemma \ref{regularitystep1} we get
\begin{equation}\label{step2estimatefehlthiername}
\left|\int_{\tau_1}^ {\tau_2} \hspace{-.3cm}H\left(\delta H\phi\right)\gamma_1|\dot\gamma|dt\right|\leq 
C\left[\int_{\tau_1}^ {\tau_2}\hspace{-.3cm} H^2\left(|\dot\gamma|\ |\phi_1|+\gamma_1 |\dot{\phi}|\right) dt+|I(\phi)|\right]
\leq
C\|\phi\|_{W^ {1,1}(U)}.
\end{equation}
Next, we take care of the terms in $\delta H$ that do not contain $\ddot{\phi}$. Recalling Equations \eqref{vark1} and \eqref{vark2} and using the result of Lemma \ref{regularitystep1}, we estimate 
\begin{align}
\left|\delta H\phi-\frac{\ddot{\phi}_2\dot\gamma_1-\ddot{\phi}_1\dot\gamma_2}{L^3}\right|
&=
|\delta k_2\phi|
+
\left|\frac{\ddot\gamma_2\dot{\phi}_1-\ddot\gamma_1\dot{\phi}_2}{L^3}
    -
    3\frac{\ddot\gamma_2\dot\gamma_1-\ddot\gamma_1\dot\gamma_2}{L^4}\frac{\langle\dot\gamma,\dot\phi\rangle}{L}\right|\nonumber\\
&\leq 
C\left(|\phi|+|\dot{\phi}|\right).\label{kestimates2}
\end{align}
Inserting into Estimate \eqref{step2estimatefehlthiername} and using the estimate for $\|H\|_{L^\infty((\tau_1,\tau_2))}$ from Lemma \ref{regularitystep1}, we get
$$\left|\int_{\tau_1}^ {\tau_2} H(\ddot{\phi}_2\dot\gamma_1-\ddot{\phi}_1\dot\gamma_2)\gamma_1|\dot\gamma|dt\right|
\leq 
C\|\phi\|_{W^ {1,1}(U)}.
$$
As in the proof of Lemma \ref{regularitystep1}, we now insert the formula for $\phi$ and estimate the auxiliary terms to get 
\begin{equation}\label{lemma2est}
\left|\int_{\tau_1}^ {\tau_2} H(\dot\varphi_2\dot\gamma_1-\dot\varphi_1\dot\gamma_2)\gamma_1|\dot\gamma|dt\right|
\leq 
C\|\phi\|_{W^ {1,1,}}\leq C\|\varphi\|_{L^1}.
\end{equation}
Now we argue similarly to as we did in the final step of the proof of Lemma \ref{regularitystep1}. First we take $\varphi=(\varphi_1,0)$ and consider the densely defined linear operator
$$
\Lambda:L^1((\tau_1,\tau_2))\rightarrow\R,
\hspace{.5cm}
\Lambda(\varphi_1):=\int_{\tau_1}^{\tau_2} H\dot\varphi_1\dot\gamma_2\gamma_1|\dot\gamma|dt.
$$
Estimate \eqref{lemma2est} allows us to extend $\Lambda$ to a continuous linear map on all of $L^1((\tau_1,\tau_2))$. Now, by the Riesz representation theorem (see e.g. \cite{hirzebruch}, Theorem 19.2), there exists a function $h\in L^ \infty((\tau_1,\tau_2))$ such that $\Lambda(\varphi)=\int_{\tau_1}^ {\tau_2} h \varphi dt$. Going back to $\varphi\in C_0^ \infty((\tau_1,\tau_2))$ we get 
$$
\int_{\tau_1}^ {\tau_2} h\varphi_1dt
=
\int_{\tau_1}^ {\tau_2} H\gamma_1\dot\gamma_2|\dot\gamma|\dot\varphi_1 dt\hspace{.5cm}\textrm{for all }\varphi_1\in C^ \infty_0((\tau_1,\tau_2)).$$
Thus $H\gamma_1\dot\gamma_2|\dot\gamma|$ is weakly differentiable and Estimate \eqref{lemma2est} implies 
$$\|(H\gamma_1\dot\gamma_2|\dot\gamma|)'\|_{L^ \infty((\tau_1,\tau_2))}\leq C.$$
Using the results from Lemma \ref{regularitystep1} as well as $\gamma_1\geq\kappa(\gamma;U)$ on $U=(\tau_1,\tau_2)$ we obtain $H\dot\gamma_2\in W^ {1,\infty}((\tau_1,\tau_2))$ with $\|H\dot\gamma_2\|_{W^ {1,\infty}((\tau_1,\tau_2))}\leq C$. Taking $\varphi=(0,\varphi_2)$ instead, we can derive a similar estimate for $H\dot\gamma_1$. The lemma follows as in the proof of Lemma \ref{regularitystep1} by using the identities in Equation \eqref{ArcLength_H_Identity}.
\end{proof}

\begin{lemma}
    Under the assumptions of Theorem \ref{GeneralRegularityTheorem} $\gamma\in W^{k+2,\infty}(U)$, $H\in W^{k,\infty(U)}$ for all $k\in\N_0$ and 
    $$\|\gamma\|_{W^{k+2,\infty}(U)}+\|H\|_{W^{k,\infty}(U)}\leq C(M,\kappa(\gamma;U),U,k).$$
\end{lemma}
\begin{proof}
As before, $C$ will always refer to a constant depending on $M$, $\kappa(\gamma; U)$ and $U$ and this time also $k$. The proof is inductive. The case $k=1$ follows from Lemma \ref{regularitystep2}. Therefore it suffices to prove the following implication for $k\geq 1$:
\begin{equation}\label{implication}
\left\{\begin{aligned} 
&H\in W^{k,\infty}(U)\\
&\|H\|_{W^ {k,\infty}(U)}\leq C\\
&\gamma\in W^{k+2,\infty}(U)\\
&\|\gamma\|_{W^ {k+2,\infty}(U)}\leq C
\end{aligned}\right\}
\hspace{.5cm}\Rightarrow \hspace{.5cm}
\left\{\begin{aligned} 
&H\in W^{k+1,\infty}(U)\\
&\|H\|_{W^ {k+1,\infty}(U)}\leq C\\
&\gamma\in W^{k+3,\infty}(U)\\
&\|\gamma\|_{W^ {k+3,\infty}(U)}\leq C
\end{aligned}\right\}
\end{equation}
To establish \eqref{implication} we take $\varphi=(\varphi_1,\varphi_2)\in C_0^ \infty(U,\R^2)$ and choose the test function $\phi:=\varphi^ {(k-1)}$
Testing the weak version of the Euler-Lagrange equation with $ \phi$ as we did in Equation \eqref{weakeulerLagrange01} gives 
\begin{align}
 &\left|\int_{\tau_1}^ {\tau_2} H\left(\delta H\phi\right)\gamma_1|\dot\gamma|dt\right|
\leq 
C\left|
\int_{\tau_1}^ {\tau_2}H^2\left(
|\dot\gamma|\phi_1+\gamma_1\frac{\langle\dot\gamma,\dot{\phi}\rangle}{|\dot\gamma|}
\right)dt
\right|+C|I(\phi)|\nonumber\\
%---------------
    \leq & C\left|\int_{\tau_1}^ {\tau_2}H^2\left(|\dot\gamma|\varphi^{(k-1)}_1(t)+\gamma_1\frac{\langle\dot\gamma,\varphi^{(k)}(t)\rangle}{|\dot\gamma|}\right)dt\right|+C|I(\varphi^{(k-1)})|.\label{longestimate} 
\end{align}
 First, we establish a suitable estimate for $I(\varphi^ {(k-1)})$. By assumption $\gamma\in W^ {k+2,\infty}(U)$. Therefore we can integrate the integrals in Equation \eqref{IFUnctionalPDE} by parts to obtain 
$|I(\varphi^ {(k-1)})| \leq C\|\varphi\|_{L^1}$. 
Also using the assumption $H\in W^ {k,\infty}(U)$ and $\gamma\in W^ {k+2,\infty}(U)$, we can rewrite the integral on the right in Estimate \eqref{longestimate} by integrating by parts several times and estimate it by $C\|\varphi\|_{L^1}$. In total, we deduce
\begin{equation}\label{step3zwischenfehltname}
\left|
\int_{\tau_1}^ {\tau_2} H\left(\delta H\phi\right)\gamma_1|\dot\gamma|dt
\right|
\leq 
C\|\varphi\|_{L^1}.
\end{equation}
Next, we again estimate all terms in $\delta H\phi$ that do not contain $\ddot{\phi}$. Using Equations \eqref{vark1} and \eqref{vark2}, $\gamma\in W^ {k+2,\infty}$ and repeated integration by parts we get
\begin{align*}
   & \left|\int_{\tau_1}^ {\tau_2}H\left(
   \delta H\phi-\frac{\dot\gamma_1\ddot{\phi}_2-\dot \gamma_2\ddot{ \phi}_1}{L^3}
   \right)\gamma_1|\dot\gamma|dt\right|\\
   %-------------
   \leq & \left|\int_{\tau_1}^ {\tau_2}H
   \left(
        \delta k_2\phi
    +
    \frac{\ddot\gamma_2\dot{\phi}_1-\ddot\gamma_1\dot{\phi}_2}{L^3}
    -
    3\frac{\ddot\gamma_2\dot\gamma_1-\ddot\gamma_1\dot\gamma_2}{L^4}\frac{\langle\dot\gamma,\dot{\phi}\rangle}{L}\right)
   \gamma_1|\dot\gamma|dt\right|\\
   \leq & C\|\varphi\|_{L^1}.
\end{align*}
Combining with Estimate \eqref{step3zwischenfehltname}, we find 
$$\left|\int_{\tau_1}^ {\tau_2} H
\frac{\dot\gamma_1\varphi^ {(k+1)}_2-\dot\gamma_2\varphi^ {(k+1)}_1}{L^3}\gamma_1 |\dot\gamma| dt\right|
\leq
C\|\varphi\|_{L^1((\tau_1,\tau_2))}.
$$
As in the proof of Lemma \ref{regularitystep1} this now implies $H\in W^ {k+1,\infty}((\tau_1,\tau_2))$ and hence $\gamma\in W^ {k+3,\infty}((\tau_1,\tau_2))$. This establishes Implication \eqref{implication} and thereby the lemma.
\end{proof}












