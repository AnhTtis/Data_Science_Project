\section{Convergence to a Double Sphere}\label{doublespheresection}
Let $\sigma_k\rightarrow0^+$ and  $\gamma^{(k)}\in\mathcal F_{\sigma_k}$ satisfy $\mathcal W[\gamma^{(k)}]=\beta(\sigma_k)$. Since vertical translations $(\gamma_1,\gamma_2)\mapsto (\gamma_1,\gamma_2+c)$ and the reflection $(\gamma_1,\gamma_2)\mapsto (\gamma_1,-\gamma_2)$ leave the Willmore energy and the isoperimetric ratio invariant, we cannot expect the profile curves $\gamma^{(k)}$ to converge. To rule out both obstructions, we introduce 
$$\mathcal F^ {0+}_\sigma:=\set{\gamma\in\mathcal F_\sigma\ |\ \gamma(0)=0\textrm{ and }\int_0^1 \gamma_2(t) dt\geq 0},$$
and stress that $\beta(\sigma)=\inf\set{\mathcal W[\gamma]\ |\ \gamma\in\mathcal F_\sigma}
=\inf\set{\mathcal W[\gamma]\ |\ \gamma\in\mathcal F_\sigma^{0+}}$. Now, considering a sequence $\gamma^{(k)}\in\mathcal F_{\sigma_k}^{0+}$ convergence to a limit is possible. Indeed, letting $R:=\frac1{\sqrt{8\pi}}$ and 
$$\kappa:[0,1]\rightarrow \R^2,\ \kappa(t):=(0, R)+R(|\sin(2\pi t)|, -\cos(2\pi t)),$$
we prove the following:

\begin{theorem}[Convergence to a Double Sphere]\label{asymptoticsthm}
Let $(\sigma_k)\subset (0,1)$ such that $\sigma_k\rightarrow 0^+$ and let $\gamma^ {(k)}\in\mathcal F_{\sigma_k}^ {0+}$ such that $\mathcal W[\gamma^{(k)}]=\beta(\sigma_k)$.
Then $\beta(\sigma_k)\rightarrow 8\pi$ and $\gamma^{(k)}\rightarrow\kappa$ in $W^ {1,2}((0,1))$ and in $C^\infty_{\operatorname{loc}}([0,1]\backslash\{\frac12\})$.
\end{theorem}

In particular, we note that Theorem \ref{asymptoticsthm} implies  Theorem \ref{doublesphereconvergence} and concludes the proof of Theorem \ref{theorem1}.\\
\ \\
For the entirety of this section we fix a sequence $\sigma_k\rightarrow 0^+$ as well as $\gamma^{(k)}\in \mathcal F^{0+}_{\sigma_k}$ such that $\beta(\sigma_k)=\mathcal W[\gamma^{(k)}]$. We will prove that there exists a subsequence $\gamma^{(k_l)}$ that converges to $\kappa$ in $W^{1,2}((0,1))$ and in $C^\infty_{\operatorname{loc}}([0,1]\backslash\{\frac12\})$. Once we have shown this, Theorem \ref{asymptoticsthm} follows by the usual subsequence argument. 

\begin{lemma}\label{convergentsubsequence}
There exists a subsequence $\gamma^{(k_l)}$ and a curve $\gamma^*\in\mathcal P^w$ such that $\gamma^{(k_l)}\rightarrow\gamma^*$ in the sense of Definition \ref{convergencedefinition}. Moreover, either $\gamma^*\in \mathcal P$ or there exists $\tau\in(0,1)$ such that, after reparameterization, $\gamma^*|_{[0,\tau]}$ and  $\gamma^*|_{[\tau,1]}$ both belong to $\mathcal P$.
\end{lemma}
\begin{proof}
The proof for the existence of $\gamma^*$ and the convergence is essentially the same as the one from Theorem \ref{weakminimizerexistence}. However, instead of Estimate \eqref{leq38pi} we only get 
$$\mathcal W[\gamma^*]\leq 8\pi\hspace{.5cm}\textrm{and}\hspace{.5cm}\int_0^1 \left((k_1^*)^2+(k_2^*)^2 \right)2\pi L_*\gamma_1^*(t)\leq 3\cdot 8\pi.$$
A priori Theorem \ref{weakandstrongPconnection} implies that there are at most 2 points $\tau\neq\tau'\in (0,1)$ such that $\gamma_1^*(\tau)=\gamma_1^*(\tau')=0$. However, following the arguments after Equation \eqref{leq38pi} implies that there can be at most one such point.
\end{proof}

Let $\delta>0$ be small. Depending on whether the limit $\gamma^*$ from Lemma \ref{convergentsubsequence} satisfies $\gamma^*_1(\tau)=0$ for some $\tau\in(0,1)$ or not, we put 
$$
I_\delta:=
\left\{
\begin{aligned}
(\delta,1-\delta) &\ \textrm{if $\gamma^*\in\mathcal P$},\\
(\delta,\tau-\delta)\cup (\tau+\delta,1-\delta)&\ \textrm{if $\gamma^*\not\in\mathcal P$}.
\end{aligned}
\right.
$$



We can now prove that along a subsequence $\gamma^{(k)}$ converges to $\gamma^*$ not only in $\mathcal P^w$ but in $C^m(I_\delta)$ for all $\delta>0$ and $m\in\N$. 

\begin{lemma}\label{smoothconvergence}
Let  $\gamma^ {(k_l)}$ and $\gamma^*$ be as in Lemma \ref{convergentsubsequence}. Then  $\gamma^*\in C^\infty(I_\delta)$ for all $\delta>0$ and $\gamma^ {(k_l)}\rightarrow\gamma^*$ in $C^m(I_\delta)$ for all  $m\in\N$ and small $\delta>0$.
\end{lemma}

\begin{proof}
We prove that for arbitrary $m\in\N$ the sequence $\gamma^ {(k_l)}$ is bounded in $W^ {m+1,\infty}(I_\delta)$. The Sobolev embedding $W^ {m+1,\infty}(I_\delta)\hookrightarrow C^ {m,\alpha}(\bar I_\delta)$ then shows that $\gamma^ {(k_l)}$ is bounded in $C^ {m,\alpha}(I_\delta)$. Once this is shown, we can use $\gamma^ {(k_l)}\rightarrow \gamma^*$ in $\mathcal P^w$ to deduce the lemma. From now on we write $\gamma^{(k)}$ instead of $\gamma^{(k_l)}$\\


 
\noindent
\textbf{Uniform Bounds in $W^ {m,\infty}$}\ \\
By Definition \ref{convergencedefinition} and Lemma \ref{convergentsubsequence} we get $\gamma^ {(k)}\rightarrow \gamma^*$ in $C^0([0,1])$. So we can assume that
\begin{equation}\label{kappauniform}
\inf_{t\in I_\delta}\gamma_1^{(k)}(t)
\geq 
\frac12\inf_{t\in I_\delta}\gamma_1^*(t)
=\frac12\kappa(\gamma^*;I_\delta)>0.
\end{equation}
Either $I_\delta=(\delta,1-\delta)$ or $I_\delta=(\delta,\tau-\delta)\cup(\tau+\delta,1-\delta)$. Let $(\tau_1,\tau_2)$ denote any of these intervals. We claim that there is $\psi^*\in C^ \infty_0((\tau_1,\tau_2))$ such that 
\begin{equation}\label{psistardef}
1\overset!=6\sqrt\pi \delta V[\gamma^*]\psi^*
=\pm_{\gamma^*}6\sqrt\pi^\frac32 \int_{\tau_1}^ {\tau_2} (\gamma_1^*)^2\dot\psi^*_2+2\gamma_1^*\dot\gamma_2^*\psi^*_1 dt.
\end{equation}
Indeed, otherwise, we could use partial integration to get for all $\psi^*\in C^ \infty_0((\tau_1,\tau_2))$
$$
0=\int_{\tau_1}^ {\tau_2} (\gamma_1^*)^2\dot\psi^*_2+2\gamma_1^*\dot\gamma_2^*\psi^*_1 dt
=2\int_{\tau_1}^ {\tau_2} \gamma_1^*\langle\dot\gamma^*, (-\psi_2^*,\psi_1^*)\rangle dt.
$$
Using $\gamma^*\in W^{1,2}((0,1))$ and $\gamma_1^*>0$ almost everywhere we conclude $\dot\gamma^*= 0$ almost everywhere in $(\tau_1,\tau_2)$. This is a contradiction as $\gamma^*\in\mathcal P^w$ and hence $|\dot\gamma^*|=L[\gamma^*]$ almost everywhere. \\

Now let $\psi^*$ be as in Equation \eqref{psistardef}. Using $\gamma^{(k)}\rightarrow\gamma^*$ in $C^0([0,1])$ and $W^{1,2}((0,1))$ and $\sigma_k\rightarrow 0$, we can use Equations \eqref{var1}-\eqref{var3} to get $1=\lim_{k\rightarrow\infty}\delta\mathcal I[\gamma^{(k)}]\psi^*$. This proves that, after potentially ignoring the first couple of terms in the sequence, there exists a sequence $\zeta_k\subset(0,\infty)$ converging to $1$ such that 
$$1=\zeta_k\left(\delta\mathcal I[\gamma^{(k)}]\psi^*\right)=\delta\mathcal I[\gamma^{(k)}](\zeta_k\psi^*).$$
This allows us to use Corollary \ref{gammasmooththeorem1} with $U=(\tau_1,\tau_2)$ and $\psi_0^ {(k)}=\zeta_k\psi^*$ to prove the lemma. Indeed, Estimate \eqref{kappauniform} provides us with a uniform lower bound for $\kappa(\gamma^{(k)};U)$ and, as $\zeta_k\rightarrow 1$, the $C^2$-norms of $\psi^ {(k)}_0$ are bounded uniformly.
\end{proof}

We now prove a first part of Theorem \ref{asymptoticsthm}.
\begin{lemma}\label{Lemma5point4}
    Let $(\sigma_k)\subset (0,1)$ such that $\sigma_k\rightarrow 0^+$ and let $\gamma^ {(k)}\in\mathcal F_{\sigma_k}^ {0+}$ such that $\mathcal W[\gamma^{(k)}]=\beta(\sigma_k)$.
Then $\beta(\sigma_k)\rightarrow 8\pi$ and $\gamma^{(k)}\rightarrow\kappa$ in $W^ {1,2}((0,1))$ and in $C^\infty_{\operatorname{loc}}((0,1)\backslash\{\frac12\})$.
\end{lemma}
\begin{proof}
Combining Lemmas \ref{convergentsubsequence} and \ref{smoothconvergence} we have $\gamma^{(k)}\rightarrow\gamma^*$ in $\mathcal P^w$ and in $C^m(I_\delta)$ for all $m\in\N$ and $\delta>0$ along a subsequence again denoted by $\gamma^{(k)}$. Additionally, either $\gamma^*\in\mathcal P$ or there exists $\tau\in(0,1)$ such that $\gamma_1^*(\tau)=0$ and $\gamma^*|_{[0,\tau]}$,  $\gamma^*|_{[\tau,1]}$ both belong to $\mathcal P$. In either case, we can use the weak lower semicontinuity from Equation \eqref{Willmorelowersemicont} and deduce
\begin{equation}\label{limitWillmoreupperbound}
\mathcal W[\gamma^*]\leq \liminf_{k\rightarrow\infty}\mathcal W[\gamma^{(k)}]\leq\limsup_{k\rightarrow\infty}\mathcal W[\gamma^{(k)}]\leq 8\pi.
\end{equation}
Using $\mathcal I[\gamma^ {(k)}]=\sigma_k\rightarrow 0^ +$, $A[\gamma^ {(k)}]\equiv 1$ and the continuity of the area and volume with respect to the $\mathcal P^w$ convergence (see Equations \eqref{areaconv} and \eqref{volconv}), we get 
\begin{equation}\label{areaANDvolume}
A[\gamma^*]=\lim_{k\rightarrow\infty}A[\gamma^ {(k)}]\equiv 1
\hspace{.5cm}\textrm{and}\hspace{.5cm}
V[\gamma^*]=\lim_{k\rightarrow\infty}V[\gamma^ {(k)}]=0.
\end{equation}

\paragraph{The first case: $\gamma^*\in\mathcal P$}\ \\
We claim that there exist $\tau<\tau'\in[0,1]$ such that $\gamma(\tau)=\gamma(\tau')=:(p_1,p_2)$. Indeed, if $\gamma^*$ was injective, then $V[\gamma^*]>0$ by Lemma \ref{volumegauss} which contradicts Equation \eqref{areaANDvolume}. By Lemma \ref{monotonicityformula} we conclude $\mathcal W[\gamma^*]\geq 8\pi$. Combining this with Estimate \eqref{limitWillmoreupperbound}, we deduce $\mathcal W[\gamma^*]=8\pi$. Putting $p:=(p_1,0,p_2)$, we are in the equality case of the monotonicity formula from Lemma \ref{monotonicityformula} and get 
$$\frac14\vec H[f_{\gamma^*}]=\frac{(f_{\gamma^*}-p)^\perp}{|f_{\gamma^*}-p|^2}.$$
Using Lemma \ref{SphereOrInvertedCat} and $\mathcal W[\Sigma_{\gamma^*}]=8\pi$, we deduce that $\Sigma_{\gamma^*}$ is the inversion of a scaled and vertically translated catenoid and thus encloses a positive volume, which contradicts Equation \eqref{areaANDvolume}.

\paragraph{The Second Case: $\gamma^*\not\in\mathcal P$}\ \\
Let $\alpha^{(1)}:=\gamma^*|_{[0,\tau]}$ and $\alpha^{(2)}:=\gamma^*|_{[\tau,1]}$. Lemma \ref{WillmoreLowerBound} gives $\mathcal W[\alpha^{(i)}]\geq 4\pi$ for $i=1,2$. This shows
\begin{equation}\label{twospheresequation}
8\pi\leq \mathcal W[\alpha^{(1)}]+ \mathcal W[\alpha^{(2)}]= \mathcal W[\gamma^*]\leq 8\pi.
\end{equation}
Hence $\mathcal W[\gamma^*]=8\pi$ and in view of Estimate \eqref{limitWillmoreupperbound} we get $\beta(\sigma_k)=\mathcal W[\gamma^{(k)}]\rightarrow8\pi$. Additionally, Estimate \eqref{twospheresequation} gives $\mathcal W[\alpha^{(i)}]= 4\pi$ for $i=1,2$ so that by the Li-Yau inequality (Lemma \ref{WillmoreLowerBound}), both curves are the profile curve of a sphere. Since $\alpha^{(i)}\in\mathcal P$, are parameterized proportional to arc length, there exist $z_i\in\R$, $\epsilon_i\in\set{\pm1}$ and $\omega_i, R_i\in\R_0^+$, such that
\begin{align*}
    &\alpha^{(1)}:[0,\tau]\rightarrow \mathcal H^2,\ \alpha^{(1)}(t)=(0, z_1)+ R_1(\sin(\omega_1 t), \epsilon_1\cos(\omega_1 t)),\\
    %----------------
   &\alpha^{(2)}:[\tau,1]\rightarrow \mathcal H^2,\ \alpha^{(2)}(t)=(0, z_2)+ R_2(\sin(\omega_2(t-\tau)), \epsilon_1\cos(\omega_2(t-\tau))).
\end{align*}
Using Equation \eqref{areaANDvolume} we get
\begin{align*}
    1&=A[\gamma^*]=A[\gamma^ {(1)}]+A[\gamma^ {(2)}]=4\pi(R_1^2+R_2^2),\\
    0&=V[\gamma^*]=V[\gamma^ {(1)}]+V[\gamma^ {(2)}]=\frac43\pi|\epsilon_1R_1^3+\epsilon_2 R_2^3|.
\end{align*}
These imply $\epsilon_2=-\epsilon_1=:-\epsilon$ and $R_1=R_2=(8\pi)^{-\frac12}=:R$. Since $\gamma^*$ is parameterized proportional to arc length, we have $\omega_1 R=|\dot\alpha^{(1)}|=|\dot\gamma^*|=|\dot\alpha^{(2)}|=\omega_2 R$
and hence $\omega_1=\omega_2=:\omega$. Note $\alpha^{(1)}_1(t)>0$ for $t\in(0,\tau)$ and $\alpha^{(1)}_1(\tau)=0$. This proves $\omega\tau=\pi$. Similarly  $\alpha^{(2)}_1(t)>0$ for $t\in(\tau,1)$ and $\alpha^{(2)}_1(1)=0$ give $\omega(1-\tau)=\pi$. Therefore $\omega=2\pi$ and $\tau=\frac12$. By continuity of $\gamma^*$ at $\tau$ 
$$z_1+\epsilon R\cos(\omega\tau)=z_2-\epsilon R\cos(\omega(\tau-\tau)).$$
Using $\omega=2\pi$ and $\tau=\frac12$ we get $z_1=z_2=:z$. As $\gamma^*(0)=(0,0)$, we get $\alpha_2^{(1)}(0)=0$ and hence $z=-\epsilon R$.  Finally, since $\gamma^{(k)}\in\mathcal F^{0+}_{\sigma_k}$ we get
$$0\leq \lim_{k\rightarrow\infty}\int_0^1\gamma_2^{(k)}(t)dt
=\int_0^1\gamma_2^*(t)dt
=\frac{z_1+z_2}2
=z
=-\epsilon R.$$
So $\epsilon=-1$ and $\gamma^*=\kappa$.
\end{proof} 

It remains to study the asymptotic behaviour of $\gamma^ {(k)}$ near $t=0$ and $t=1$ when $k\rightarrow\infty$. 
\begin{korollar}\label{Lagrangemultipliergoto0Lemma}
We denote by $\Lambda_k$ the Lagrange multiplier from Lemma \ref{eulerlagrangeequation}. Then $\Lambda_k\rightarrow 0$ as $k\rightarrow\infty$
\end{korollar}
\begin{proof}
By Lemma \ref{eulerlagrangeequation}
\begin{equation}\label{LambdakGoestoZeroEquation}
|\Lambda_k|=\left|\frac{2W[\gamma^{(k)}]}{\pm_{\gamma^{(k)}} 4\sqrt\pi-\sigma_k H[\gamma^{(k)}]}\right|.
\end{equation}
Let $I:=(\frac18,\frac38)$. By Lemma \ref{Lemma5point4} we have $\gamma^{(k)}\rightarrow\kappa$ in $C^r(I)$ for all $r\in\N$ and hence  
$$W[\gamma^{(k)}]\bigg|_{I}\rightarrow W[\kappa]\bigg|_{I}\equiv 0
\hspace{.5cm}\textrm{and}\hspace{.5cm}
H[\gamma^{(k)}]\bigg|_{I}\rightarrow H[\kappa]\bigg|_{I}\equiv 2.$$
In particular, $\sigma_k H[\gamma^{(k)}]\rightarrow0$ on $I$ and consequently $\Lambda_k\rightarrow 0$ by Equation \eqref{LambdakGoestoZeroEquation}.
\end{proof}

For all $k\in\N$ we define 
$$\tau_k:=\sup\{t\geq 0\ |\ \dot\gamma_1^ {(k)}(s)>0\hspace{.2cm}\textrm{for all $0\leq s\leq t$}\}.$$
We prove that $\tau_k$ is bounded from below. In fact, we even prove the following fact:

\begin{lemma}\label{lowerboundeddotgamma1Lemma}
   Let $\epsilon>0$. There exist $k_0(\epsilon)\in\N$ and $t_0(\epsilon)>0$ such that 
   $$\dot\gamma_1^ {(k)}(t)\geq L[\gamma^{(k)}](1-\epsilon)\hspace{.5cm}\textrm{for all $k\geq k_0(\epsilon)$ and }t\in[0, t_0(\epsilon)].$$
\end{lemma}
\begin{proof}
    The proof is by contradiction. Assuming that the lemma is false, there exists $\epsilon>0$ such that after passing to a subsequence, we have $\dot\gamma_1^{(k)}(t_k)<L_k(1-\epsilon)$ for a sequence $t_k\rightarrow 0^+$ and where $L_k:=L[\gamma^{(k)}]$. By Lemma \ref{Lemma5point4} we have $\gamma^ {(k)}\rightarrow\kappa$ in $W^ {1,2}((0,1))$ and in $C^\infty_{\operatorname{loc}}((0,1)\backslash\{\frac12\})$ and therefore $L_k\rightarrow L_*:=L[\kappa]$.  For small $\delta>0$, we recall $I_\delta:=(\delta,\frac12-\delta)\cup(\frac12+\delta,1-\delta)$.
    Denoting the principal curvatures of $\kappa$ by $k_i^*$ we have
    \begin{equation}\label{Limitdeltarightarrow0eq}
   \lim_{\delta\rightarrow0^+}\frac{2\pi}4\int_{I_\delta}(k_1^*+k_2^*)^2 \kappa_1|\dot\kappa| dt
   = \mathcal W[\kappa]
   = 8\pi. 
    \end{equation}    
    For each $\delta>0$ there exists $k_0(\delta)>0$ such that $t_k<\delta$ for all $k\geq k_0(\delta)$. Using that $\mathcal W[\gamma^{(k)}]=\beta(\sigma_k)\leq8\pi$  we estimate 
    \begin{align}
    8\pi\geq \mathcal W[\gamma^{(k)}]=&\frac{2\pi}4\int_0^1 (k_1^{(k)}+k_2^{(k)})^2 \gamma_1^{(k)}|\dot\gamma^{(k)}|dt\nonumber\\
    \geq &\frac{2\pi}4\int_{I_\delta} (k_1^{(k)}+k_2^{(k)})^2 \gamma_1^{(k)}|\dot\gamma^{(k)}|dt+\frac{2\pi}4\int_{0}^{t_k} (k_1^{(k)}+k_2^{(k)})^2 \gamma_1^{(k)}|\dot\gamma^{(k)}|dt.\label{8piContradictionsetupspellmistake}
    \end{align}
     We derive a suitable estimate for the second integral. By Corollary \ref{gammasmoothuptoBoundaryCorollary} we know $\gamma^ {(k)}\in C^ \infty([0, t_k])$ and using Equation \eqref{ArcLength_K_Identity}, we get
     $$k_1^{(k)}k_2^{(k)}\gamma_1^ {(k)}|\dot\gamma^ {(k)}|=-\frac1{L_k}\ddot\gamma_1^ {(k)}\hspace{.5cm}\textrm{for $t\in[0,t_k]$}.$$
     Using this identity, $\dot\gamma_1^ {(k)}(0)=L_k$ and $\dot\gamma_1^ {(k)}(t_k)\leq L_k(1-\epsilon)$ we may compute
    \begin{align*}
    &\int_0^{t_k}(k_1^{(k)}+k_2^{(k)})^2 \gamma_1^{(k)}|\dot\gamma^{(k)}|dt
    \geq 
    2\int_0 ^{t_k}k_1^{(k)}k_2^{(k)} |\dot\gamma^{(k)}|\gamma_1^{(k)}dt
    =
-\frac2{L_k}\int_0^{t_k}\ddot\gamma_1^{(k)}(t) dt\\
=&-\frac2{L_k}\left(\dot\gamma_1^ {(k)}(t_k)-\dot\gamma_1^ {(k)}(0)\right)\geq  2L_k-2(L_k-\epsilon)
    =2\epsilon.
    \end{align*}
    Inserting into Estimate \eqref{8piContradictionsetupspellmistake}, we deduce that for all $\delta>0$ and $k\geq k_0(\delta)$ we have  
    $$
    8\pi\geq \frac{2\pi}4\int_{I_\delta} (k_1^{(k)}+k_2^{(k)})^2 \gamma_1^{(k)}|\dot\gamma^{(k)}|dt+\pi\epsilon.
   $$
   Since $\gamma^{(k)}\rightarrow \kappa$ in $C^r(I_\delta)$ for all $r\in\N$, we can send $k\rightarrow\infty$ and deduce that for all $\delta>0$
   $$
    8\pi\geq \frac{2\pi}4\int_{I_\delta} (k_1^*+k_2^*)^2 \kappa_1|\dot\kappa|dt+\pi\epsilon.
   $$
   In view of Equation \eqref{Limitdeltarightarrow0eq}, we arrive at a contradiction by letting $\delta\rightarrow0^+$. 
\end{proof}

Putting $L_*:=L[\kappa]$ we have $L_k\rightarrow L_*$. So, by Lemma \ref{lowerboundeddotgamma1Lemma}, there exists $k_0>0$ and $t_0>0$ such that for all $k\geq k_0$ and $t\in[0, t_0]$ we have $\dot\gamma_1^ {(k)}(t)\geq\frac14L_*>0$.
Consequently, each $\gamma_1^ {(k)}|_{[0, t_0]}$ is a smooth diffeomorphism onto its image. We also note that 
$$\gamma_1^ {(k)}(t_0)=\int_0^{t_0}\dot\gamma_1^ {(k)}(t)dt\geq \frac{t_0 L_*}4=:r_0.$$
This implies that for all $k\geq k_0$
\begin{equation}\label{definitionofuk}
u_k(r):[0, r_0]\rightarrow\R,\ u_k(r):=\gamma_2^{(k)}\circ \left(\gamma_1^{(k)}\right)^{-1}(r)
\end{equation}
is well-defined and smooth. Since $\gamma^{(k)}\rightarrow\kappa$ in $C^\infty_{\operatorname{loc}}((0,1)\backslash\{0\})$, we deduce that 
\begin{equation}\label{LimitofGraphfunction01}
u_k(r)\rightarrow u^*(r):= R-\sqrt{R^2-r^2}\hspace{.5cm}\textrm{ in $C^\infty_{\operatorname{loc}}((0, r_0]))$}\hspace{.2cm}\textrm{ where $R=\frac1{\sqrt{8\pi}}$.}
\end{equation}
We wish to prove that the convergence is even in $C^\infty([0, r_0]))$. 

\begin{korollar}\label{Asymptotics_Derivativesmall}
   For all $\epsilon>0$ there exist $k_0(\epsilon)\in\N$ and $r_1(\epsilon)\in(0, r_0]$ such that $|u_k'(r)|\leq\epsilon$ for all $k\geq k_0(\epsilon)$ and $r\in[0, r_1(\epsilon)]$.
\end{korollar}
\begin{proof}
For small $\epsilon>0$ we take $k_0(\epsilon)$ and $t_0(\epsilon)$ as in Lemma \ref{lowerboundeddotgamma1Lemma} and restrict ourselves to $k\geq k_0(\epsilon)$. We put $\rho_k(\epsilon):=\gamma_1^ {(k)}(t_0(\epsilon))$. Note that for $r\in[0,\rho_k(\epsilon)]$ we have $(\gamma^ {(k)}_1)^ {-1}(r)\in[0, t_0(\epsilon)]$. Hence $\dot\gamma_1^ {(k)}((\gamma^ {(k)}_1)^ {-1}(r))\geq L_k(1-\epsilon)$ and therefore 
$$\left|\dot\gamma_2^{(k)}\left(\left(\gamma^ {(k)}_1\right)^ {-1}(r)\right)\right|=\left(L_k^2-\left(\dot\gamma_1^{(k)}\left(\left(\gamma^ {(k)}_1\right)^ {-1}(r)\right)\right)^2\right)^{\frac12}\leq L_k\sqrt{1-(1-\epsilon)^2}.$$
So, for all $k\geq k_0(\epsilon)$ and  $r\leq \rho_k(\epsilon)$ we get 
    $$|u_k'(r)|=\left|\dot\gamma_2^ {(k)}\left(\left(\gamma_1^{(k)}\right)^{-1}(r)\right)\cdot\frac1{\dot\gamma_1^ {(k)}\left(\left(\gamma_1^{(k)}\right)^{-1}(r)\right)}\right|
    \leq \frac{L_k\sqrt{2\epsilon-\epsilon^2}}{L_k(1-\epsilon)}=\frac{\sqrt{2-\epsilon}}{1-\epsilon}\sqrt\epsilon.$$
 This implies that the lemma is proven as soon as a uniform lower bound for $\rho_k(\epsilon)$ is established. To do so, we use the defining property of $t_0(\epsilon)$ from Lemma \ref{lowerboundeddotgamma1Lemma} and estimate  
$$\rho_k(\epsilon)=\gamma_1^{(k)}(t_0(\epsilon))=\int_0^{t_0(\epsilon)}\dot\gamma_1^{(k)}(t) dt\geq t_0(\epsilon) L_k(1-\epsilon).$$
Since $L_k\rightarrow L_*>0$ we have $\inf_k L_k>0$ and so $\rho_k(\epsilon)\geq r_1(\epsilon)>0$ independent of $k$. 
\end{proof}


Corollary \ref{Asymptotics_Derivativesmall} allows us to essentially repeat the proof of Lemma \ref{regularityatend01} to establish a uniform bound for $u_k''$. 


\begin{lemma}\label{AsymptoticsSecondDerivativeBound}
There exists $C>0$ and $r_0>0$ such that $|u_k''(r)|\leq C$ for all $k\in\N$ and $r\in[0, r_0]$.
\end{lemma}
\begin{proof}
As each individual $u_k\in C^\infty([0, r_0))$, we only need to prove a uniform bound $|u_k''|\leq C$ for all $k\geq k_1$ with some arbitrary $k_1$. To do so, we follow the proof of Lemma \ref{regularityatend01} and put $w_k:=u_k'$ and $v_k:=\sqrt{1+w_k^2}$. In view of Equation \eqref{ODEamRand}, we have 
\begin{equation}\label{ODEamRand_Neu}
w_k''+\left(\frac{w_k}{r}\right)'-\varphi(w_k)
=
\frac12 v_k^ 5 a_k r+b_kv_k^ 4w_k.
\end{equation}
Here we used that $\lambda_k=0$ by Lemma \ref{uisc2}. Also we note that $a_k=\pm_{\gamma^{(k)}}4\sqrt\pi \Lambda_k$ and $b_k=\sigma_k\Lambda_k$. Corollary \ref{Lagrangemultipliergoto0Lemma} gives $\Lambda_k\rightarrow0$ as $k\rightarrow \infty$ so that $a_k$ and $b_k$ are bounded. Additionally, Corollary \ref{Asymptotics_Derivativesmall} and $u_k\rightarrow u^*$ in $C^\infty((0, r_0])$ together imply that $w_k=u_k'$ is bounded.\\

\noindent
\textbf{Step 1}\ \\
Note  $w_k\varphi(w_k)\geq 0$ by the definition of $\varphi(w)$ in Equation \eqref{varphiofwdef}. We multiply Equation \eqref{ODEamRand_Neu} by $w_k$, choose any $\rho\in (0, r_0)$  and integrate to get
$$\int_r ^ \rho w_k''w_k+\left(\frac {w_k}t\right)' w_k dt
\geq \frac {a_k}2 \int_ r^ \rho v_k^ 5 t w_k dt+
b_k\int_r^ \rho v_k^ 4 w_k^2 dt\geq -C.$$
$C$ is uniform over $k$ as $a_k$, $b_k$ and $w_k$ are bounded. Performing the same calculations that lead to Estimate \eqref{step1result}, we derive
\begin{align}
\int_r^ \rho (w_k'(t))^2+\frac{ w_k(t)^2}{2t^2}dt
&\leq C(\rho)+\frac{w_k(r)^2}{2r}-w_k(r)\frac{w_k(r) +rw_k'(r)}r\nonumber\\
&= C(\rho)+\frac{w_k(r)^2}{2r}-w_k(r)\frac{(r w_k(r))'}r.\label{step1result_Neu}
\end{align}
The constant $C(\rho)$ includes the values $w_k(\rho)$ and $w_k'(\rho)$ but is uniform over $k$. Indeed, as $u_k\rightarrow u^*$ in $C^\infty_{\operatorname{loc}}((0, r_0])$, we have $w_k(\rho)\rightarrow (u^*)'(\rho)$ and $w_k'(\rho)\rightarrow  (u^*)''(\rho)$.\\

%-----------------------
\noindent
\textbf{Step 2}\ \\
Let again $\rho\in(0,r_0)$. This time we integrate Equation \eqref{ODEamRand_Neu} directly and use $w_k'+\frac {w_k}r=\frac1r(w_kr)'$ to get 
\begin{equation}\label{step2begin_Neu}
\frac {(w_k t)'}t\bigg|_{t=\rho}-\frac {(w_k r)'}r=\int_ r^ \rho \varphi(w_k)(t)+\frac12 a_k v_k^ 5 t + b_k v_k^ 4w_kdt.
\end{equation}
Given $\epsilon>0$, we use Corollary \ref{Asymptotics_Derivativesmall} and choose $k_0(\epsilon)\in\N$ and $\rho(\epsilon):=r_1(\epsilon)\in(0, r_0]$ such that $|w_k(r)|\leq \epsilon$ for all $k\geq k_0$ and $r\in(0, \rho(\epsilon))$. This allows us to perform the same manipulations that lead to Estimate \eqref{step2result} and get
\begin{align}
    \left|\frac{(r w_k(r))'}{r}\right| 
    &\leq C(\epsilon)+\epsilon \int_r^ \rho w_k'(t)^2 +\frac{w_k(t)^2}{2t^2} dt\hspace{.5cm}\textrm{for all }r\in(0, \rho(\epsilon)).\label{step2result_Neu}
\end{align}
Again $C(\epsilon)$ contains the boundary values $w_k(\rho(\epsilon))$ and $w_k'(\rho(\epsilon))$ which are, however, controlled uniformly over $k$.\\ 

%-------------------------
\noindent
\textbf{Step 3}\ \\
Let $\epsilon>0$ be arbitrary. By taking $\rho=\rho(\epsilon)$ small enough in Estimate \eqref{step1result_Neu} we may insert Estimate \eqref{step2result_Neu} to get 
\begin{align*} 
\int_ r^ {\rho}w_k'(t)^2+\frac{w_k(t)^2}{2t^2} dt&\leq C(\epsilon)+\frac{w_k(r)^2}{2r}
+|w_k(r)|\left(C(\epsilon)+\epsilon \int_r^ \rho w_k'(t)^2 +\frac{w_k(t)^2}{2t^2} dt\right).
\end{align*}
By potentially shrinking $\rho$ we can assume without loss of generality that $|w_k|\leq \epsilon$ on $[0,\rho]$ and so, in particular $|w_k(r)|\leq \epsilon$. Choosing $\epsilon$ small enough, we can absorb the integral on the right to the left and obtain that for some small enough $\rho$ and $r\in(0,\rho)$
\begin{equation}\label{step3result_Neu}
\int_r^ {\rho}w_k'(t)^2+\frac{w_k(t)^2}{2t^2} dt
\leq
C\left(\frac{w_k(r)^2}{2r}+|w_k(r)|+1\right).
\end{equation}

%--------------------------
\noindent
\textbf{Step 4}\ \\
Let $\rho$ be small enough so that Estimate \eqref{step3result_Neu} holds. Inserting Estimate \eqref{step3result_Neu} into Estimate \eqref{step2result_Neu} with $\epsilon=1$ and using that $|w_k|$ is bounded uniformly, we get
\begin{align*} 
\left|\frac{(r w_k(r))'}{r}\right| 
    \leq &
    C+C\left(\frac{w_k(r)^2}{2r}+|w_k(r)|+1\right) 
    \leq 
    C\left(\frac{w_k(r)^2}{2r}+1\right).
\end{align*}
We multiply by $r$ and choose any $\mu\in(0,1)$. By using Corollary \ref{Asymptotics_Derivativesmall} and potentially shrinking $\rho$ we can assume that $|w_k|\leq C^ {-1}\mu$ on $[0,\rho]$. Hence 
$$|(r w_k)'|\leq C(w_k(r)^2+r)\leq \mu |w_k|+Cr.$$ 
Using the same justification as in the proof of Lemma \ref{regularityatend01}, we compute
$$r|w_k|'+|w_k|=(r|w_k|)'=|rw_k|'\leq |(rw_k)'|\leq \mu|w_k|+Cr\hspace{.5cm}\textrm{for all $r\in(0,\rho]$},$$
multiply with $r^ {-\mu}$ and get 
$$(r^ {1-\mu} |w_k|)'=r^ {1-\mu}|w_k|'+(1-\mu) r^ {-\mu}|w_k|\leq C r^ {1-\mu} .$$
We integrate this inequality from $0$ to $r$ and get
\begin{equation}\label{step4result_Neu}
r^ {1-\mu}|w_k(r)|\leq C\int_0^ r t^ {1-\mu} dt\leq C(\mu) r^{2-\mu}
\hspace{.5cm}\textrm{and hence}\hspace{.5cm}
|w_k(r)|\leq Cr.
\end{equation}

We now deviate from the proof of Lemma \ref{regularityatend01}.\\


\noindent
\textbf{Step 5}\ \\
Inserting \eqref{step4result_Neu} into \eqref{step3result_Neu}, we get
$$\int_r^ {\rho}w_k'(t)^2+\frac{w_k(t)^2}{2t^2} dt
\leq
C.$$
Inserting this into Estimate \eqref{step2result_Neu} with $\epsilon=1$, we obtain $|(w_k(r) r)'|\leq Cr$. Combining this with Estimate \eqref{step4result_Neu}, we get 
$$r|w_k'(r)|\leq |rw_k'(r)+w_k(r)|+|w_k(r)|\leq \left|(w_k(r) r)'\right|+|w_k(r)|\leq Cr.$$
Hence $w_k'(r)$ is bounded on $[0, \rho]$. The lemma follows by recalling that $u_k\rightarrow u^*$ in $C^\infty([\rho, r_0])$.
\end{proof}

Corollary \ref{Asymptotics_Derivativesmall} and Lemma \ref{AsymptoticsSecondDerivativeBound} prove that $u_k'$ and $u_k''$ are bounded. To deduce $u_k\rightarrow u^*$ in $C^\infty([0, r_0])$, we prove that all higher derivatives are bounded as well.  

    
\begin{lemma}\label{ukfinalconvergencecorollary}
$u_k\rightarrow u^*$ in $C^\infty([0, r_0])$.
\end{lemma}
\begin{proof}
    In view of Equation \eqref{LimitofGraphfunction01} it suffices to prove that $(u_k)\subset C^m((0, r_0))$ is bounded for all $m\geq 0$. We have $\gamma^{(k)}\rightarrow\kappa$ in $C^0([0,1])$ and hence $\gamma_2^{(k)}$ is a bounded sequence of  functions. By the definition of $u_k$ in Equation \eqref{definitionofuk} we deduce that $(u_k)\subset C^0([0, r_0])$ is bounded. Combining this with
        Corollary \ref{Asymptotics_Derivativesmall} and Lemma \ref{AsymptoticsSecondDerivativeBound}, we obtain that $(u_k)\subset C^2([0, r_0])$ is bounded.\\
        
    We put $w_k:=u_k'$ and $v_k:=\sqrt{1+w_k^2}$. By Equation \eqref{ODEamRand_Neu} 
    $$w_k''+\left(\frac{w_k'}r\right)'=\frac12 v_k^5 a_k r+b_k v_k^4 w_k+\frac{5w_k}{2(1+w_k^2)}(w_k')^2+\frac{w_k^3}{2r^2}(3+w_k^2)=:f_k(r).$$


    To prove that $(w_k)\subset C^m([0, r_0])$ is bounded for all $m\geq 1$ we argue by induction. $m=1$ has already been established. Assuming that $(w_k)\subset C^m([0, r_0])$ is bounded for some $m\geq 1$, we first use $w_k(0)=u_k'(0)=0$ to rewrite
    \begin{equation}\label{Newformulaforf}
    f_k(r)=
    \frac12 v_k^5 a_k r+b_k v_k^4w_k +\frac{5w_k}{2(1+w_k^2)}(w_k')^2+\frac{w_k}{2}(3+w_k^2)\left(\int_0^1 w_k'(s r) ds\right)^2.
    \end{equation}
    As $(w_k)\subset  C^{m}([0, r_0])$ is bounded, Equation \eqref{Newformulaforf} implies that $(f_k)\subset C^{m-1}([0, r_0])$ is bounded. Also $f\in C^\infty((0, r_0])$. So, using Lemma \ref{uniformboundedLemma} we deduce that $(w_k^{(m+2)})\subset C^0([0, r_0])$ is bounded. Noting that $w_k^{(m+1)}(r_0)$ is bounded since $u_k\rightarrow u^*$ in $C^\infty_{\operatorname{loc}}((0, r_0])$, we may use Lemma \ref{uniformboundedLemma} to get that $w_k^{(m+1)}\subset C^0([0,r_0])$ is bounded and the inductive step follows.\\

    Since $u_k'=w_k$ and $(u_k)\subset C^0([0, r_0])$ is bounded, we have shown that $(u_k)\subset C^m([0, r_0])$ is bounded for all $m\in\N_0$. 
\end{proof}

Finally, we prove Theorem \ref{asymptoticsthm} In view of Lemma \ref{Lemma5point4}, we only need to prove that $\gamma^{(k)}$ is bounded in $C^m([0,t_0]\cup[1-t_0,1])$ for some small $t_0>0$ and all $m\in\N_0$.\\

\noindent
\textit{Proof of Theorem \ref{asymptoticsthm}}\ \\
Let $L_*:=L[\kappa]$ and $t_0>0$ such that $\gamma^{(k)}_1|_{[0,t_0]}$ are smooth diffeomorphisms for all $k\in\N$. By Lemma \ref{lowerboundeddotgamma1Lemma}, we may assume $\dot\gamma_1^{(k)}(t)\geq\frac{L_*}2>0$ for all $k\geq k_0$ and $t\in[0, t_0]$ after potentially shrinking $t_0$. We establish that $(\gamma^{(k)})\subset C^m([0, t_0])$ is bounded for all $m\in\N$ by an inductive argument. For $m=0$ this follows from the fact that $\gamma^{(k)}\rightarrow\kappa$ in $C^0([0,1])$. Now for the inductive step. Suppose that for some $m\in\N_0$ we already know that $\gamma^{(k)}\subset C^m([0, t_0])$ is bounded.
Recalling $\gamma_2^{(k)}(t)=u_k(\gamma_1^{(k)}(t))$ and $L_k^2=|\dot\gamma^{(k)}(t)|^2$, we deduce 
$$L_k^2-\left(\dot\gamma_1^{(k)}(t)\right)^2=(u_k'(\gamma_1^{(k)}(t))^2\left(\dot\gamma_1^{(k)}(t)\right)^2
\hspace{.5cm}\textrm{and so}\hspace{.5cm}
\left(\dot\gamma_1^{(k)}(t)\right)^2=\frac{L_k^2}{ 1+ u_k'(\gamma_1^{(k)}(t))^2}.$$
Using that $\dot\gamma_1^{(k)}(t)\geq 0$ for all $t\in[0, t_0]$ we can take the square root and obtain
\begin{equation}\label{gamma1odewithu}
    \dot\gamma_1^{(k)}(t)=\frac{L_k}{\sqrt{1+u_k'(\gamma_1^{(k)}(t))^2}}\hspace{.5cm}\textrm{when $t\in[0, t_0]$}.
\end{equation}
By Lemma \ref{ukfinalconvergencecorollary}
$(u_k)\subset C^l([0, r_0])$ is bounded for all $l\in\N$. Using Equation \eqref{gamma1odewithu} it then follows that $\gamma_1^{(k)}\subset C^{m+1}([0, t_0])$ is bounded. Since $\gamma_2^{(k)}=u_k\circ \gamma_1^{(k)}$, we deduce that $\gamma^{(k)}$ is bounded in $C^{m+1}([0, t_0])$, which completes the inductive step.\\


Finally, we note that a similar discussion proves that $(\gamma^{(k)})$ is bounded $C^m([1-t_1,1])$ for some (potentially small) $t_1>0$ and all $m\in\N$. Theorem \ref{asymptoticsthm} follows.  
\qed


