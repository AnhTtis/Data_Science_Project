\section{Calculus Lemmas  }
\begin{lemma}\label{oderegularitylemma}
Let $n\in\N_0$, $u\in C^1((0,1))\cap C^0([0,1))$, $h\in C^n([0,1)$ and 
$$\frac{(x u(x))'}x=h(x).$$
Then $u\in C^{n+1}([0,1))$.
\end{lemma}
\begin{proof}
First, we consider the case $n=0$. We have 
\begin{equation}\label{odeRegLemmaEq1}
xu(x)=0\cdot u(0)+\int_0^x th(t)dt.
\end{equation}
So, for $x\in (0,1)$ we get 
$$u'(x)=\frac d{dx}\left[\frac1x\int_0^x th(t)dt\right]=h(x)-\frac1{x^2}\int_0^x th(t)dt.$$
We claim that $u'(x)\rightarrow \frac12 h(0)$ as $x\rightarrow 0$. Indeed
$$\frac1{x^2}\int_0^x h(t)tdt=\int_0^1 s h(sx) ds\rightarrow h(0)\int_0^1 s ds=\frac12 h(0).$$
Now we consider the case $n>0$. Rewriting Equation \eqref{odeRegLemmaEq1} we get
$$u(x)=x\int_0^1h(tx ) t dt$$
and deduce $u\in C^n([0,1))$. Next, compute 
\begin{align*}
    u'(x)=&\int_0^1 th(tx) dt+x\int_0^1 t^2 h'(tx) dt\\
        &\int_0^1 th(tx) dt
            +x\left[\frac {t^2h(tx)}x\bigg|_{t=0}^{t=1}-\int_0^1 2t\frac{h(tx)}xdt\right]\\
        =&h(x)-\int_0^1 t h(tx)dt.
\end{align*}
Clearly, this last formula is $C^n([0,1])$, which proves the lemma. 
\end{proof}

\begin{lemma}\label{uniformboundedLemma}
Let $r_0>0$, $f\in C^0([0, r_0])\cap C^1((0, r_0])$ and $w\in C^2([0, r_0])$ such that 
$$w''(r)+\left(\frac wr\right)'=f(r).$$
Then $|w''(r)|\leq \frac 43 \sup_{0\leq\rho\leq r}|f(\rho)|$ and consequently $|w'(r)|\leq |w'(r_0)|+\frac43 r_0\sup_{0\leq\rho\leq r_0}|f(\rho)|$.
\end{lemma}
\begin{proof}
    We compute 
    $$
    \left(\frac{(wr)'}r\right)'
    =
    \left(w'+\frac wr\right)'
    =
    f.
    $$
Integrating this equation from $r>0$ to $r_0$ shows 
$$-\frac{(w(r)r)'}r+w'(r_0)+\frac{w(r_0)}{r_0}=\int_r^{r_0} f(s)ds=\int_0^{r_0} f(s)ds-\int_0^rf(s)ds.$$
We put $c_0:=w'(r_0)+\frac{w(r_0)}{r_0}-\int_0^{r_0} f(s)ds$. Then 
$$\frac{(rw(r))'}r=c_0+\int_0^r f(s)ds.$$
We multiply by $r$. Another integration from $r$ to $r_0$ shows 
\begin{align*}
r_0 w(r_0)-rw(r)&=\frac12 c_0(r_0^2-r^2)+\int_r^{r_0}x\int_0^x f(s)dsdx\\
&=-\frac12 c_0 r^2 +\frac12 c_0 r_0^2+\int_0^{r_0}x\int_0^x f(s)dsdx-\int_0^{r}x\int_0^x f(s)dsdx.
\end{align*}
This shows that there exist constants $c_1,c_2\in\R$ such that
$$rw(r)=c_1 r^2+c_2+\int_0^r \int_0^x xf(s)dsdx.$$
Taking $r=0$ shows $c_2=0$. Next, we substitute $x=ry$ and obtain 
$$w(r)=c_1r+r\int_0^1\int_0^{ry}y f(s)dsdy.$$
We compute two $r$-derivatives. 
\begin{align*} 
w''(r)=&2\int_0^1 \frac\partial{\partial r}\int_0^{yr}yf(s)dsdy+r\int_0^1 \frac{\partial^2}{\partial r^2}\int_0^{ry} yf(s)dsdy\\
=&2\int_0^1 y^2f(yr)dy+r\int_0^1 y^3 f'(yr)dy\\
=&2\int_0^1 y^2f(yr)dy+\int_0^1 y^3\frac{\partial}{\partial y} f(yr)dy\\
=&2\int_0^1 y^2f(yr)dy+y^3 f(yr)\bigg|_{y=0}^{y=1}-\int_0^1 3y^2f(yr)dy\\
=& -\int_0^1 y^2f(yr)dy+ f(r)
\end{align*}
Consequently 
$$|w''(r)|
\leq \left(1+\int_0^1 y^2 dy\right)\sup_{0\leq\rho\leq r}|f(\rho)|
=\frac 43 \sup_{0\leq\rho\leq r}|f(\rho)|.$$
\end{proof}

\begin{lemma}\label{derivativeinL1}
Let $f\in C^ 1((0,\rho))$ such that $f'\in L^ 1((0,\rho))$. Then $f\in C^ 0([0,\rho])$.
\end{lemma}
\begin{proof}
We prove continuity at $x=0$. The case $x=\rho$ works similarly. Let $x<y\in(0,\rho]$. Then 
$$|f(x)-f(y)|\leq \int_{x}^ y|f'(t)|dt\leq \int_0^  y|f'(t)|dt.$$
This implies that $f(\epsilon_n)$ is Cauchy for any $\epsilon_n\rightarrow 0^ +$ and hence $f(0):=\lim_{x\rightarrow0^+}f(x)$ is well-defined. To see that $f$ is continuous at $x=0$, we use $f'\in L^1((0,1))$ to estimate
$$|f(0)-f(y)|\leq \int_0^y |f'(t)|dt\rightarrow 0,\hspace{.5cm}\textrm{as $y\rightarrow 0^ +$.}$$
\end{proof}