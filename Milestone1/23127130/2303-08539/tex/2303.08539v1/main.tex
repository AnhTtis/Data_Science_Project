\documentclass[a4paper]{amsart}
\usepackage{amsmath,amsthm,amssymb,amsfonts,amsbsy}
\usepackage{enumitem,color,graphicx}
\usepackage[all,pdf]{xy}


\usepackage[backref=page,hyperindex=true,CJKbookmarks=true,
colorlinks,linkcolor=blue,anchorcolor=red,citecolor=cyan]{hyperref}


\newtheorem{thm}{Theorem}[section]
\newtheorem{rmk}[thm]{Remark}
\newtheorem{defn}[thm]{Definition}
\newtheorem{prop}[thm]{Proposition}
\newtheorem{lem}[thm]{Lemma}

\newtheorem{claim}[thm]{Claim}
\newtheorem{cor}[thm]{Corollary}
%\newtheorem{qst}[thm]{Question}

\theoremstyle{plain}
\newtheorem{maintheorem}{Theorem}
\renewcommand{\themaintheorem}{\Alph{maintheorem}}
\newtheorem{main-cor}[maintheorem]{Corollary}

\numberwithin{equation}{section}
\newcommand{\quand}{\quad\text{and}\quad}

\renewcommand*{\backref}[1]{}
\renewcommand*{\backrefalt}[4]{%
    \ifcase #1 (Not cited.)%
    \or        (Cited on page~#2.)%
    \else      (Cited on pages~#2.)%
    \fi}

%%%%%%%%%%%%%%%%%%%%%%%%%%%%%%%%%%%%%%%%%%%%%%%%%%%%%%%%%%%%%%%

\address{Department of Mathematics, 
            Southern University of Science and Technology, 
            Shenzhen, Guangdong, China 518055}
\email{xiamy@sustech.edu.cn}

\title[Kan-type transitivity]{Topological transitivity of Kan-type partially hyperbolic diffeomorphisms}
\author{Mingyang Xia}

\date{\today}
\subjclass[2020]{37D05; 37D30; 37E05}
\keywords{Kan example; partially hyperbolic diffeomorphism; transitivity.}

%%%%%%%%%%%%%%%%%%%%%%%%%%%%%%%%%%%%%%%%%%%%%%%%%%%%%%%%%%%%%%%

\begin{document}

\begin{abstract}
    \begin{sloppypar}
	It is presented the topological transitivity of 
    a class of diffeomorphisms on the thickened torus, 
    including the partially hyperbolic example introduced by Ittai Kan in 1994,
    which is well known for the first systems with intermingled basins phenomenon.
    \end{sloppypar}
\end{abstract}

\maketitle


\section{Introduction}
In 1994, Kan \cite{Kan94} constructed two important partially hyperbolic systems, 
which inspired a lot of studies from the statistical and topological viewpoints.

Precisely speaking, for the non-invertible case, 
Kan constructed a $C^\infty$ skew-product map $f:M\to M$
on the 2-dimensional cylinder $M=\mathbb{S}^1\times[0,1]$ defined by
\begin{align}\label{Ex-endo}
	f(x,t)=\left(3x, t+\frac{t(1-t)}{32}\cos(2\pi x)\right), 
	~~ \forall (x,t)\in\mathbb{S}^1\times[0,1].
\end{align}
For the invertible case, 
Kan constructed a $C^\infty$ skew-product diffeomorphism 
on the thickened torus $M=\mathbb{T}^2\times[0,1]$ defined by
\begin{align}\label{Ex-diffeo}
	f(x,y,t)=\left(3x+y, 2x+y, t+\frac{t(1-t)}{32}\cos(2\pi x)\right),
	~~\forall (x, y,t)\in\mathbb{T}^2\times[0,1].
\end{align}

From the statistical viewpoint, it is well known that 
\emph{Kan's constructions admit two physical measures whose basins are intermingled.} 
Recall that for a $C^2$ map $f$ on a compact manifold $M$, 
the basin $\mathcal{B}(\mu)$ of an $f$-invariant Borel probability measure $\mu$ is defined by
    $$
    \mathcal{B}(\mu)= \left\{ x\in M : 
    \lim\limits_{n\to\infty} 
    \frac{1}{n}  \sum_{k=0}^{n-1}\phi(f^k(x)) =\int_{M}\phi d\mu,
    ~~\forall \phi \in C^0(M,\mathbb{R})   \right\}.
    $$
Then $\mu$ is called a \emph{physical measure} (or $Sinai-Ruelle-Bowen$ measure) 
if the basin $\mathcal{B}(\mu)$ has positive Lebesgue measure.
The existence and finiteness of physical measures 
are crucial in the study of statistical behavior of dynamical systems, 
as is shown by Sinai, Ruelle and Bowen \cite{Si72,Ru76,BR75} that 
uniformly hyperbolic systems only have finitely many physical measures
while the union of their basins has the full Lebesgue measure in the manifold.
See for instance \cite{ABV00,CYZ20,HYY20} for recent advances with weak hyperbolicity.
Moreover, two physical measures $\mu_1$ and $\mu_2$ have 
\emph{intermingled basins} 
if for any non-empty open set $U\subseteq M$,
    $$ 
    Leb(\mathcal{B}(\mu_1)\cap U)>0 \quand Leb(\mathcal{B}(\mu_2)\cap U)>0 .
    $$ 
Kan's examples continue to be a source of interesting research in dynamical systems,
which have been extensively studied, especially in the aspect of intermingled basins, 
see for instance \cite{MW05,DVY16,BP18,UV18} for recent developments. 


From the perspective of topology, 
it has been shown that a typical class of Kan's non-invertible systems is indecomposed 
by appending two conditions (see \cite[Proposition 11.2]{BDV05}).
In fact, it is proved this kind of cylinder endomorphisms are \emph{topologically transitive},
that is, for any two non-empty open sets $U,V\subseteq M$, 
there is a positive integer $m$ such that $f^m(U)\cap V\neq\varnothing.$
Recently, Gan and Shi \cite{GS19} show that Kan-like cylinder endomorphisms,
including the Example (\ref{Ex-endo}),
are robustly topologically mixing within the $C^2$ boundary preserving maps.

These facts indeed indicate an interesting difference 
between the measure and the topology: 
these constructions admit two intermingled physical measures 
but are topologically indecomposed.
This makes Kan's examples more important to some extent.
The contrast of measurable and topological properties 
is an interesting subject for the study of dynamics.
Very recently, there are some beautiful results 
focusing on topological transitivity of skew-products (see \cite{Ok17,CO21}), 
and measures of maximal entropy for some general systems 
related to the Kan's endomorphism (see \cite{NBRV21,RT22}).

As for the situation of Kan's invertible systems, 
there are few studies related to Kan's constructions.
In contrast to the robust manner of the intermingled basins \cite{IKS08}, 
recently, Ures and V\'{a}squez \cite{UV18} establish the non-robust intermingled basins phenomenon 
on $\mathbb{T}^3$, providing the constructions are not accessible.
Soon after, appending extra constructions, 
a family of topologically transitive diffeomorphisms on $\mathbb{T}^2\times \mathbb{S}^1$ 
is constructed by inserting a blender in the Kan's example (\ref{Ex-diffeo}) 
and embedding into boundaryless manifold $\mathbb{T}^2\times \mathbb{S}^1$ (see \cite[Theorem 1.4]{CGS18}).
%In the case of skew-products with $\mathbb{S}^1$, 
%there is also an interesting work related to topological transitivity in \cite{Ok17}. 
 
However, it is still unknown whether the Kan's original 
example on $\mathbb{T}^2\times[0,1]$ is topologically transitive or not.
This paper is devoted to giving an affirmative answer 
for Kan-type partially hyperbolic diffeomorphism (see Definition \ref{def:Kan-type}). 
By this result, we complete the final piece of topological indecomposability of Kan's examples.


\begin{maintheorem}\label{main-thm}
	Let $F:\mathbb{T}^2\times[0,1]\circlearrowleft$ be a 
    Kan-type partially hyperbolic diffeomorphism 
    with $C^2$ regularity, defined by $$F(x,t)=(Ax,\phi_x(t)),$$ 
	where $A:\mathbb{T}^2\circlearrowleft$ is an Anosov toral automorphism 
	fixing two points $p,q\in \mathbb{T}^2$ and  
        $$\dfrac{\ln\phi_p'(0)}{\ln\phi_q'(0)}\notin\mathbb{Q},$$  
	then $F$ is topologically transitive.
\end{maintheorem}

This result also holds for the case 
$A$ is an Anosov diffeomorphism on $\mathbb{T}^n$ based on the exactly same argument,
we just show the bare bone proof to make it more clear.
Comparing to the argument in \cite[Proposition 11.2]{BDV05}, 
we keep the original assumptions on the Kan's examples without additional ones.
Moreover, there is a property behind the Kan's examples, 
called mostly contracting (see \cite{Kan94,ABV00,BV00}), 
which requires
    $$
    \int_{\mathbb{T}^2}\phi_x'(0)dx<0
    \quand
    \int_{\mathbb{T}^2}\phi_x'(1)dx<0.
    $$
As stated in Definition \ref{def:Kan-type}, we do not need this condition either.

Naturally, the Example (\ref{Ex-diffeo}) belongs to 
the category of Kan-type partially hyperbolic diffeomorphisms, 
so there is the following result as a corollary.  

\begin{main-cor}\label{main-cor}
	The Kan's example on $\mathbb{T}^2\times[0,1]$ 
	\begin{align*}
		f(x,y,t)=\left(3x+y, 2x+y, t+\frac{t(1-t)}{32}\cos(2\pi x)\right)
	\end{align*}
	is topologically transitive.
\end{main-cor}
\begin{proof}[Proof of Corollary \ref{main-cor}]
	It satisfies to prove the irrational condition of Theorem \ref{main-thm}.
    Note that there are two fixing points $p=({1}/{2},0)$ and $q=(0,0)$ 
    with $\phi_p'(0)={31}/{32}$ and $\phi_q'(0)={33}/{32}$,
	which exactly have the irrational relation. 
	In fact, if not, we have
        $$ 
        \dfrac{\ln\phi_p'(0)}{\ln\phi_q'(0)}
        =\dfrac{\ln31-\ln32}{\ln33-\ln32}=-\dfrac{m}{n},
        $$
	where $m,n\in\mathbb{N}$ and $\gcd(m,n)=1$. 
	%Note that $m>n$ by the concavity of the logarithmic function.
	Rewriting this equality, we just get
    	$$
    	3^m\cdot 11^m\cdot 31^n=2^{5m+5n},
        $$ 
	which is an obvious contradiction by the Fundamental Theorem of Arithmetic.
 
\end{proof}


\section{Preliminaries}\label{sec-pre}


\subsection{Kan-type partially hyperbolic diffeomorphism}

There are some characterizations in the Kan's original example (\ref{Ex-diffeo}),
so we introduce the following general class of partially hyperbolic diffeomorphisms, 
with attempting to capture the main features of Kan's construction on $\mathbb{T}^2\times[0,1]$. 

Recall the Pole maps on the unit interval at first.

\begin{defn}
    Let $\phi:[0,1]\to[0,1]$ be a diffeomorphism fixing two endpoints.
    \begin{itemize}
    	\item $\phi$ is called an North–South Pole map (abbr. NS-map) 
    	if $0<\phi'(0)<1$, $\phi'(1)>1$ and 
    	$$ \phi(t)<t, \quad \forall t\in(0,1). $$ 
    	
    	\item $\phi$ is called an South–North Pole map (abbr. SN-map)
    	if $0<\phi'(1)<1$, $\phi'(0)>1$ and 
    	$$ \phi(t)>t, \quad \forall t\in(0,1). $$ 
    \end{itemize}
\end{defn}

\begin{defn}\label{def:Kan-type}
	Let $F:\mathbb{T}^2\times[0,1]\circlearrowleft$ be a $C^2$ skew-product defined by            
        $$F(x,t)=(Ax,\phi_x(t)),$$ 
	where $A$ is an Anosov toral automorphism on $\mathbb{T}^2$ fixing two points $p$ and $q$.
	$F$ is called a Kan-type partially hyperbolic diffeomorphism 
	if the following conditions are satisfied:
	\begin{enumerate}
		\item[($K_1$)] $\forall$ $x\in \mathbb{T}^2$, $\phi_x(0)=0$ and $\phi_x(1)=1$, 
		    i.e., $F$ preserves boundary components.
		\item[($K_2$)] $\forall$ $t\in[0,1]$, $\phi_p$ is NS-map and $\phi_q$ is SN-map.
		\item[($K_3$)] $\forall$ $(x,t)\in \mathbb{T}^2\times[0,1]$, 
		    $\|A^{-1}\|^{-1}< \phi_x'(t) <\|A\|$, 
		    i.e., $F$ is partially hyperbolic.
	\end{enumerate}
\end{defn}

\iffalse
\begin{figure}[htbp]
	\centering
	\includegraphics[width=8cm]{Transitive-mechanism.pdf}
	\caption{Transitive mechanism}
	\label{TM}
\end{figure}
\fi


\subsection{Regularity of holonomy map}

For the sake of delicate analysis, 
it is required some regularity for the holonomy dynamics at least. 
It has been proved in \cite{PSW97} and \cite{PSW00} that 
the strong stable foliation is $C^1$ when restricted to a center-stable leaf 
if the partially hyperbolic diffeomorphism $f$ is a dynamically coherent 
and center-bunched $C^2$ diffeomorphism.

\begin{thm}[\cite{PSW97}, Theorem B]\label{Thm:C1-holonomy}
	Let $f$ be a $C^2$ partially hyperbolic diffeomorphism 
	satisfying the dynamically coherent and center bunching conditions, 
	then the holonomy map defined by the strong stable or strong unstable foliation 
	is $C^1$ local diffeomorphism. 
\end{thm}

\begin{rmk}
	For $C^2$ Kan-type partially hyperbolic diffeomorphism, 
	the holonomy map defined by the strong stable or strong unstable foliation
	is $C^1$ diffeomorphism locally,
	since dynamically coherent condition is satisfied 
	by Item $(K_3)$ in Definition \ref{def:Kan-type} with the skew-product form, and 
	center bunching condition is also satisfied 
	by Item $(K_3)$ with the setting of one-dimensional center.
\end{rmk}


\subsection{Analysis in the one-dimensional dynamics}

Before we start to deal with the topological transitivity of Kan-type systems, 
we need to get some preparation on the one-dimensional center dynamics.
Here we are going to present an intersection result to warm up.

Firstly we introduce the following classical linearization theorem by Sternberg, 
for the convenience of providing $C^1$-charts in one-dimensional dynamics.

\begin{thm}[\cite{Na11}, Theorem 3.6.2]\label{SLT}
	Let $f$ be a $C^2$ diffeomorphism 
	from a neighborhood containing $0$ in $\mathbb{R}^+$ onto its image.
	If $f'(0)=\alpha\neq1$, then there is a $C^1$ local diffeomorphism 
	$h$ onto its image with $h'(0)=1$, 
	and $ h\circ f=\alpha\cdot h$ near $0$.
\end{thm}

Now we present the intersection result in a quantitative way 
with the map $h$ working as the holonomy map later.
Note that the inverse of NS-map is SN-map and vice versa.

\begin{prop}\label{prop-Duminy-id}
	Let $h:[0,1]\to[0,1]$ be an orientation-preserving $C^1$ diffeomorphism. 
	Assume that $f$ and $g$ are $C^2$ NS-maps
	satisfying $\ln \alpha$ and $\ln \beta$ are rationally independent, 
	where $\alpha=f'(0)$ and $\beta=g'(0)$. 
	
	Then for any $x\in(0,1)$, the set 
    	$$
    	\left\lbrace f^{-k}\circ h^{-1}\circ g^l(x):k,l\in\mathbb{N}^*\right\rbrace
    	$$
	is dense in $[0,1]$.
	
	In particular, for any intervals $I,J \subseteq [0,1]$, 
	there exist infinitely many pairs of integers $k_n,l_n>0$ such that 
    	$$
    	\big( h\circ f^{k_n}(I) \big) \bigcap g^{l_n}(J) \neq \varnothing
    	$$
	and 		
    	$$
    	\dfrac{k_n}{l_n} \to \dfrac{\ln\beta}{\ln\alpha} 
    	\quad \text{as} \quad n\to \infty.
    	$$
	
	Moreover, denote by $I=[a,b]$ and $J=[c,d]$, there exists $\rho>0$ such that 
    	$$
    	\dfrac{| f^{k_n}(I) \bigcap 
    	\big( h^{-1}\circ g^{l_n}(J) \big) |}{f^{k_n}(b)-f^{k_n+1}(b)}
        \geqslant\rho
    	\quand	
    	\dfrac{| \big( h\circ f^{k_n}(I) \big) \bigcap g^{l_n}(J)|}
        {g^{l_n}(d)-g^{l_n+1}(d)}
    	\geqslant\rho. 
    	$$
\end{prop}

\begin{proof}[Proof of Proposition \ref{prop-Duminy-id}]
	By Sternberg linearization theorem, 
	there exist two $C^1$ local diffeomorphisms $h_1,h_2$ such that
    	$
    	h_1'(0)=h_2'(0)=1,
    	$
	with the following conjugate equations holding:
	\begin{align*}
	   h_1\circ f(t)&=\alpha \cdot h_1(t),\\
	   h_2\circ g(t)&=\beta  \cdot h_2(t),
	\end{align*}
	for any $t$ near the hyperbolic sink $0$.
	
	Pick a small $\delta >0$ such that the conjugate equations both hold on $[0,\delta]$,
	then there is a commutative diagram as follows:	
	\begin{equation*}
	   \begin{gathered} \xymatrix{
    		f:[0,\delta]\circlearrowleft \ar[r]^h \ar[d]_{h_1} &g:[0,\delta]\circlearrowleft  \ar[d]_{h_2}\\
    		\alpha:[0,h_1(\delta)]\circlearrowleft \ar[r]^{\tilde{h}}  &\beta:[0,h_2(\delta)]\circlearrowleft \\}	
        \end{gathered}
	\end{equation*}

	
	We are going to show there exist infinitely many intersection iterates at first. 
	\begin{lem}\label{lem-InterInf}
	    For any $x\in (0,\delta]$ and $I=[a,b]\subseteq [0,\delta]$, 
	    there exist infinitely many integers $k_n,l_n>0$ such that 
        	$$
        	g^{l_n}(x) \in h\circ f^{k_n}([a,b]),
        	$$ 
        where these iterate numbers $k_n,l_n$ satisfy 
        	$$
        	\dfrac{k_n}{l_n} \to \dfrac{\ln\beta}{\ln\alpha} 
        	\quad\quad \text{as} \quad n\to \infty.
        	$$ 
	\end{lem}
	\begin{proof}[Proof of Lemma \ref{lem-InterInf}]
        From the conjugate viewpoint,
        we are going to show (see Figure \ref{IntMech})
        	\begin{align}\label{prop-aim-inter}
        		\beta^l(\tilde{x}) \in \tilde{h}\circ \alpha^k([\tilde{a},\tilde{b}]),
        	\end{align}
        where $\tilde{x}=h_2(x)$, $\tilde{I}=h_1(I)=[\tilde{a},\tilde{b}]$ 
        and $\tilde{h}=h_2 \circ h \circ h^{-1}_1$.
        %(tilde means taking the transformation by the $C^1$-charts).


        \begin{figure}[htbp]
        	\centering
        	\includegraphics[width=8.6cm]{InfIntMech.pdf}
        	\caption{Intersections dynamics}
        	\label{IntMech}
        \end{figure}

         
	
        Note that $\tilde{h}'(0)=h'(0)=\theta>0,$
    	we can write
        	$$
        	\tilde{h}^{-1}(t)=\theta^{-1} \cdot t + R(t), 
        	$$
    	with the remainder term
        	$$
        	|R(t)| \leqslant M(t)\cdot t \quad {\rm for} \quad  t\in [0,h_2(\delta)],
        	$$
    	where $M(t)\to 0$ as $t\to 0$.
	
    	Then for any $\tilde{x}\in(0,h_2(\delta)]$ and interval $\tilde{I}\subseteq[0,h_1(\delta)]$, 
    	there are two positive numbers 
    	$\eta=\eta(I)$ and $\epsilon=\epsilon(I)<\eta$ such that
        \begin{align}\label{ToGene}
        	[\eta^* \tilde{x}-\epsilon,\eta^* \tilde{x}+\epsilon] 
        	\subseteq \theta\cdot\tilde{I}, 
        \end{align} 
    	for any positive number $\eta^*\in [\eta-\epsilon,\eta+\epsilon]$.
	
    	Since $\ln \alpha$ and $\ln \beta$ are rationally independent, 
    	for the given number $\ln\eta$, we have infinitely many pairs integers $k_n,l_n\to\infty$ such that
    	\begin{align}\label{ToGene=1}
    	    -k_n\ln\alpha+l_n\ln\beta \to \ln\eta, \quad {\rm i.e.,} 
    	    \quad \alpha^{-k_n}\cdot\beta^{l_n} \to \eta,
    	\end{align}
    	so we can take $N>0$ such that, for all $n>N$, 
        the corresponding infinitely many pairs integers $k_n,l_n$ 
    	satisfy the inequality:
    	\begin{align}\label{ToGene1}
        	\alpha^{-k_n}\cdot\beta^{l_n}\triangleq\eta^*_n
        	\in [\eta-\epsilon,\eta+\epsilon].
    	\end{align}
    	Meanwhile, note that $\beta<1$, for the integers $k_n,l_n\to\infty$, we have
        	$$
        	|\alpha^{-k_n}\cdot R(\beta^{l_n}(\tilde{x}))|
        	\leqslant \alpha^{-k_n}\cdot M(\beta^{l_n}\tilde{x}) 
        	\cdot \beta^{l_n}\tilde{x} \to 0,
        	$$
    	so we can enlarge this $N$ such that, 
    	for all $n>N$, these corresponding infinitely many pairs integers $k_n,l_n$ 
    	also satisfy the inequality: 
    	\begin{align}\label{ToGene2}
    	    |\alpha^{-k_n}\cdot R(\beta^{l_n}(\tilde{x}))| 
    	    < \theta^{-1}\cdot\epsilon.
    	\end{align}
	
    	Hence, combining the Inequalities 
        (\ref{ToGene}), (\ref{ToGene1}) and (\ref{ToGene2}), 
    	we just get
    	\begin{align*}
    	    \alpha^{-k_n}\circ \tilde{h}^{-1}\circ \beta^{l_n} (\tilde{x})
    	    &= \alpha^{-k_n}(\theta^{-1}\cdot \beta^{l_n}(\tilde{x}) + 
    	    R( \beta^{l_n}(\tilde{x}) ) ),\\
    	    &= \theta^{-1}\cdot\alpha^{-k_n}\cdot \beta^{l_n}(\tilde{x}) + \alpha^{-k_n}\cdot 
    	    R( \beta^{l_n}(\tilde{x})),\\
    	    &\in\theta^{-1}\cdot[\eta^*_n \tilde{x}-\epsilon,\eta^*_n \tilde{x}+\epsilon] 
    	    \subseteq \tilde{I}.
    	\end{align*}
	
	
    	This implies that there exist infinitely many pairs integers $k_n,l_n>0$ such that 
    	the desired Inclusion (\ref{prop-aim-inter}) holds,
    	with these $k_n,l_n$ satisfying that
        	$$
        	\dfrac{k_n}{l_n} \to \dfrac{\ln\beta}{\ln\alpha} 
        	\quad\quad {\rm as} \quad n\to \infty.
        	$$ 
	\end{proof}
	
	
	According to this Lemma, for any given intervals $I,J \subseteq [0,1]$, 
	we naturally have 
        	$$
        	\big(h\circ f^{k_n}(I)\big) \bigcap g^{l_n}(J) \neq \varnothing.
        	$$
	
	Moreover, we can make slight modifications for the choices of $k_n$ and $l_n$,
	to get a uniform lower bound for the proportions 
    between the length of these intersections 
	and the length of the corresponding iterates of the fundamental domain.
	
	Precisely, for the given intervals $I,J \subseteq [0,1]$, 
	we take their middle thirds and denote them by $I_0,J_0$ respectively. 
	By the argument of the proof above, 
    for the corresponding number $\eta_0=\eta_0(I_0)>0$,
	there are also infinitely many pairs integers $k_n^0,l_n^0 >0$ with
    	$$
        \big(h\circ f^{k_n^0}(I_0)\big) \bigcap g^{l_n^0}(J_0) \neq \varnothing,
        $$	
    and
        $$
    	\alpha^{-k_n^0}\cdot\beta^{l_n^0} \to \eta_0
    	\quad {\rm as} \quad n\to \infty.
    	$$
	By taking these iterates $k_n^0,l_n^0$ 
    on the previous intervals $I=[a,b],J=[c,d]$,
	we have the following refined result based on distortion control estimations.

 
	\begin{lem}\label{lem-InterSpe}
	    There is a constant $\rho>0$ such that
    	    $$
        	\dfrac{|f^{k^0_n}(I) \bigcap 
        	\big(h^{-1}\circ g^{l^0_n}(J)\big) |}{f^{k^0_n}(b)-f^{k^0_n+1}(b)}
        	\geqslant\rho
        	\quand	
        	\dfrac{| \big( h\circ f^{k^0_n}(I) \big)
        	\bigcap g^{l^0_n}(J)|}{g^{l^0_n}(d)-g^{l^0_n+1}(d)}
        	\geqslant\rho. 
        	$$
	\end{lem}
	\begin{proof}[Proof of Lemma \ref{lem-InterSpe}]
        It satisfies to prove the existence of a constant $\rho$ with
            $$\dfrac{| \big( h\circ f^{k^0_n}(I) \big)
        	\bigcap g^{l^0_n}(J)|}{g^{l^0_n}(d)-g^{l^0_n+1}(d)}
        	\geqslant\rho, 
            $$
        since $h$ is an orientation-preserving diffeomorphism.
    For the intersection iterates of the middle intervals $I_0=[a_0,b_0]$ and $J_0=[c_0,d_0]$,
	    there are only two possibilities 
        for the intersections of the previous intervals $I$ and $J$ now.
	
	
    \emph{Case 1. $h\circ f^{k_n^0}(I)$ is not totally contained in $g^{l_n^0}(J)$.}     
    	Without loss of generality, we assume $g(d)<c$, then we have
        	$$
        	\left( g^{l_n^0}(c),g^{l_n^0}(c_0) \right) \subseteq h\circ f^{k_n^0}(I) 
        	\quad{\rm or}\quad
        	\left( g^{l_n^0}(d_0),g^{l_n^0}(d) \right) \subseteq h\circ f^{k_n^0}(I),
        	$$
    	since we have $k_n^0,l_n^0$ 
    	with the intersection for middle thirds of the intervals $I,J$. 
    	
        Thus, by the distortion control argument, 
        we can get the following estimations: 
        \begin{align*}
        	\dfrac{|\big(h\circ f^{k_n^0}(I) \big)
            \bigcap g^{l_n^0}(J)|}{g^{l_n^0}(d)-g^{l_n^0+1}(d)}
        	&\geqslant \dfrac{g^{l_n^0}(c_0)-g^{l_n^0}(c)}
            {g^{l_n^0}(d)-g^{l_n^0+1}(d)} \\
        	&\geqslant \dfrac{c_0-c}{d-g(d)} 
            e^{\ln{(g^{l_n^0})'(\xi_1)}-\ln{(g^{l_n^0})'(\xi_2)}}\\
        	&\geqslant 
        	\dfrac{c_0-c}{d-g(d)} e^{-G_{12} \cdot 
            \sum\limits_{j=0}^{l_n^0-1}{|g^j(\xi_1)-g^j(\xi_2)|}}\\
        	&\geqslant 
        	\dfrac{c_0-c}{d-g(d)} e^{-G_{12}\cdot d}\triangleq \rho_c,
        \end{align*}
    	or
        \begin{align*}
        	\dfrac{|\big(h\circ f^{k_n^0}(I) \big)
            \bigcap g^{l_n^0}(J)|}{g^{l_n^0}(d)-g^{l_n^0+1}(d)}
        	&\geqslant \dfrac{g^{l_n^0}(d)-g^{l_n^0}(d_0)}
            {g^{l_n^0}(d)-g^{l_n^0+1}(d)} \\
        	&\geqslant \dfrac{d-d_0}{d-g(d)} 
            e^{\ln{(g^{l_n^0})'(\xi_3)}-\ln{(g^{l_n^0})'(\xi_4)}}\\
        	&\geqslant 
        	\dfrac{d-d_0}{d-g(d)} e^{-G_{12} \cdot 
            \sum\limits_{j=0}^{l_n^0-1}{|g^j(\xi_3)-g^j(\xi_4)|}}\\
        	&\geqslant 
        	\dfrac{d-d_0}{d-g(d)} e^{-G_{12}\cdot d}\triangleq \rho_d,
    	\end{align*}
	    where
        	$$
        	G_{12}=\dfrac{\max\limits_{t\in[0,\tilde{\delta}]}
        	{|g''(t)|}}{\min\limits_{t\in[0,\tilde{\delta}]}{|g'(t)|}}
        	%\quad {\rm and} \quad \xi_i \in (g(d),d) \quad {\rm for} \quad i=1,2,3,4.
        	$$
	    and	
        	$$
        	\xi_i \in (g(d),d) \quad {\rm for} \quad i=1,2,3,4.
        	$$

	
	\emph{Case 2. $h\circ f^{k_n^0}(I)$ is exactly totally contained in $g^{l_n^0}(J)$.} 
        So we directly have
        	$$
        	\dfrac{|\big(h\circ f^{k_n^0}(I)\big) \bigcap 
        	g^{l_n^0}(J)|}{g^{l_n^0}(d)-g^{l_n^0+1}(d)}	=
        	\dfrac{|h\circ f^{k_n^0}(I)|}{g^{l_n^0}(d)-g^{l_n^0+1}(d)}.
        	$$ 
         
	    Similarly, we deal with the situation 
	    from the conjugate viewpoint at first. 
        Denote by $\tilde{I}=h_1(I)=[\tilde{a},\tilde{b}]$ 
        and $\tilde{J}=h_2(J)=[\tilde{c},\tilde{d}]$. 
    	By $\tilde{h}'(0)>0,$
    	there exists $\theta'>0$ with 
        	$$
        	\tilde{h}'(t)\geqslant \theta' \quad {\rm for} \quad t\in [0,h_1(\delta)].
        	$$
    	Note that $\alpha<1$, we can always take $k_n^0$ large enough 
    	to guarantee $\alpha^{k_n^0}(\tilde{b})<h_1(\delta),$
    	which implies 
        	$$
        	\tilde{h}(\alpha^{k_n^0}(\tilde{b}))-\tilde{h}(\alpha^{k_n^0}(\tilde{a}))
        	\geqslant \theta'\cdot \alpha^{k_n^0}\cdot(\tilde{b}-\tilde{a}).
        	$$
    	At the same time, by taking $k_n^0,l_n^0$ large enough, 
    	we can get some number $\eta_0'\in(0,\eta_0^{-1})$ satisfying   
        	$$
        	\alpha^{k_n^0}\cdot\beta^{-l_n^0}\geqslant \eta_0'.
        	$$
    	In fact, this also comes from the Limit (\ref{ToGene=1}), that is,  
        	$$
        	\alpha^{k_n^0}\cdot\beta^{-l_n^0} \to 
            \eta^{-1} \quad {\rm  as} \quad k_n^0,l_n^0\to\infty.
        	$$
    	Thus we can get the following estimation:
        \begin{align*}
        	\dfrac{|\tilde{h}\circ \alpha^{k_n^0}(\tilde{I})|}{\beta^{l_n^0}(\tilde{d})-\beta^{l_n^0+1}(\tilde{d})}
        	&= \dfrac{\tilde{h}\circ \alpha^{k_n^0}(\tilde{b})
        	-\tilde{h}\circ \alpha^{k_n^0}(\tilde{a}) }
        	{\beta^{l_n^0}(\tilde{d})-\beta^{l_n^0+1}(\tilde{d})}\\
        	&\geqslant \dfrac{\theta'\cdot \alpha^{k_n^0}\cdot(\tilde{b}-\tilde{a}) }
        	{\beta^{l_n^0}\cdot(\tilde{d}-\beta\tilde{d})}\\	
        	&\geqslant \theta'\eta_0'\dfrac{\tilde{b}-\tilde{a}}{\tilde{d}-\beta\tilde{d}}
        	\triangleq \tilde{\rho_0}.
        \end{align*}
	
	Because $h_2'(0)=1$ and $0$ is the sink, 
        taking $\delta$ small enough, we have $\theta_2'>0$ with
        	$$
        	\dfrac{\min\limits_{t\in[0,h_2(\delta)]}{(h_2^{-1})'(t)}}
        	{\max\limits_{t\in[0,h_2(\delta)]}{(h_2^{-1})'(t)}} \geqslant \theta_2'.
        	$$
        Moreover, we get
        \begin{align*}
        	\dfrac{|h\circ f^{k_n^0}(I) 
            \bigcap g^{l_n^0}(J)|}{g^{l_n^0}(d)-g^{l_n^0+1}(d)}
        	&=	\dfrac{|h\circ f^{k_n^0}(I)|}{g^{l_n^0}(d)-g^{l_n^0+1}(d)}\\
        	&=	\dfrac{|h_2^{-1}\circ\tilde{h}\circ \alpha^{k_n^0}(\tilde{I})|}
        	{h_2^{-1}\circ(\beta^{l_n^0}(\tilde{d}))-h_2^{-1}
            \circ(\beta^{l_n^0+1}(\tilde{d}))}\\
        	&\geqslant
        	\theta_2'\dfrac{|\tilde{h}\circ \alpha^{k_n^0}(\tilde{I})|}
        	{\beta^{l_n^0}(\tilde{d})-\beta^{l_n^0+1}(\tilde{d})}\\
        	&\geqslant
        	\theta_2'\cdot\tilde{\rho_0}\triangleq \rho_0.
        \end{align*}
	
	Finally, combining two cases, we can take
        	$$
        	\rho=\min\left\{\rho_0,\rho_c,\rho_d\right\}>0,
        	$$
        then we obtain the desired uniform low bound $\rho$ satisfying
        	$$	
        	\dfrac{|h\circ f^{k_n^0}(I) \bigcap g^{l_n^0}(J)|}{g^{l_n^0}(d)-g^{l_n^0+1}(d)}\geqslant\rho. 
        	$$
         
	\iffalse
	Finally, as $h$ is an orientation-preserving diffeomorphism, 
	   by the similar argument, we will also have $\rho_2>0$ such that 
        	$$
        	\dfrac{|f^{k_n^0}(I) \bigcap h^{-1}\circ g^{l_n^0}(J)|}{f^{k_n^0}(b)-f^{k_n^0+1}(b)}\geqslant\rho_2.
        	$$
    	Hence pick 
    	   $$\rho=\min\{\rho_1,\rho_2\},$$	
    	we just complete the whole proof of Lemma \ref{lem-InterSpe}.
	\fi
 
	\end{proof}
	
	Combining these two Lemmas, 
	we just complete the proof of Proposition \ref{prop-Duminy-id}.	
	
\end{proof}


\section{Kan-type transitivity}

Now we present the proof of Theorem \ref{main-thm} in detail 
to show Kan-type transitivity.


    \begin{figure}[htbp]
    	\centering
    	\includegraphics[width=7.7cm]{KanTranDyn.pdf}
    	\caption{Kan-type dynamics}
    	\label{KanDyn}
    \end{figure}


\begin{proof}[Proof of Theorem \ref{main-thm}]
	
	For any given non-empty open sets $U,V\subseteq\mathbb{T}^2\times[0,1]$, 
	we are going to show the existence of some positive integer $m$ 
	with $F^m(U)\cap V\neq\varnothing$. 
    To be clear, we divide the proof into three steps.  
	

\textbf{\textit{Step 1. Holonomy maps with specific sizes.}} 


    At first, we define the holonomy map on the torus.    
	By the hyperbolicity of $A$, there are transversal $A$-invariant foliations 
	$\mathcal{L}^u$ and $\mathcal{L}^s$ on $\mathbb{T}^2$. 
	Denote by $h^s_p$ the holonomy map near $p$ along the stable manifolds, 
    	$$
    	h^s_p:\mathcal{L}^u_{loc}(p)\to\mathcal{L}^u_{loc}(\tilde{p}), ~~ h^s_p(x)=\mathcal{L}_{loc}^s(x)\cap\mathcal{L}^u_{loc}(\tilde{p}),
    	$$
	for any $\tilde{p}\in\mathcal{L}^s(p)$. 
	
	Since the partially hyperbolic diffeomorphism $F$ has interval center fibers, 
	for $F$-invariant strong unstable manifolds $\mathcal{W}^u$ and strong stable manifolds $\mathcal{W}^s$ 
    on $\mathbb{T}^2\times[0,1]$, 
	there is a local map $H^s_p$ projecting to $h^s_p$. 
    We give this construction carefully.
	
	For simplicity, we take the abusing symbols between $(p,0)$ and $p$ 
	without making confusions on understanding, 
	by regarding $\Gamma_0=\mathbb{T}^2\times\{0\}$ as $\mathbb{T}^2$.
	We denote by $I_p$ the center interval leaf containing the point $p$, 
	and $\pi:\mathbb{T}^2\times[0,1]\to\Gamma_0$ the canonical projection 
    along the center direction.
	
	Note that 
    	$$
    	\overline{\cup_{k>0} F^{-k}(\mathcal{W}^s_{loc}(I_p))} = \mathbb{T}^2\times[0,1],
    	$$
	for the given open set $U\subseteq\mathbb{T}^2\times[0,1]$, 
    we take an integer $k_0^{s}>0$ such that 
    	$$
    	F^{k_0^{s}}(U)\cap\mathcal{W}^s_{loc}(I_p)\neq\varnothing.
    	$$
	
	Moreover, taking a point 
    $$\tilde{p}\in \left(\pi\circ F^{k^s_0}(U)\right)\cap\mathcal{L}^s(p) \subseteq \Gamma_0,$$
	we redefine the holonomy map $h_p^{s}$ on $\Gamma_0$ 
	with the specific sizes $\epsilon_p$ and $\tilde{\epsilon_{p}}$,
    	$$
    	h^s_p:\mathcal{L}^u_{\epsilon_p}(p)\to\mathcal{L}^u_{\tilde{\epsilon_p}}(\tilde{p}),~~
    	h^s_p(x)=\mathcal{L}_{loc}^s(x)\cap\mathcal{L}^u_{\tilde{\epsilon_p}}(\tilde{p}).
    	$$
	Here we have an interval $\tilde{J}_{\tilde{p}}\subseteq I_{\tilde{p}} \cap F^{k^s_0}(U)$. 
	
	Then choosing $\delta_u < \epsilon_p $, 
	we have $$\mathcal{W}^u_{\delta_u}(I_p)\subseteq\mathcal{L}^u_{\epsilon_p}(p) \times [0,1],$$
	and we define the holonomy map $H_p^s$ on $\mathbb{T}^2\times[0,1]$ 
    along the strong stable manifolds by 
    	$$
    	H^s_p:\mathcal{L}^u_{\epsilon_p}(p) \times [0,1]
    	\to\mathcal{L}^u_{\tilde{\epsilon_p}}(\tilde{p}) \times [0,1], ~~
    	H^s_p(z)=\mathcal{W}_{loc}^s(z)\cap
        \mathcal{L}^u_{\tilde{\epsilon_p}}(\tilde{p}) \times [0,1].
    	$$
	Here we have an interval $J_p\subseteq I_p$ with $H_p^s(J_p)=\tilde{J}_{\tilde{p}}$.
	
	Thus, denote by $U_0 = F^{k^s_0}(U)$ and take 
    	$$
    	\delta_p=\min\limits_{t\in I_p} \max
        \limits_{z\in\mathcal{W}^u_{\delta_u}(t)}\{d(\pi(z),p)\},	
    	$$ 
	we get, by decreasing $\delta_u$ (hence $\delta_p$), 
    every center interval $J_{p'}$ in $\Gamma^{cu}_p\triangleq\mathcal{W}^u_{\delta_p}(J_p)$ satisfies 
    	$$
    	\tilde{J}_{\tilde{p'}}\triangleq H_p^s(J_{p'})\subseteq U_0 
    	\quad {\rm and} \quad 
    	\Gamma^{cu}_{\tilde{p'}}\triangleq
        \mathcal{W}^u_{\delta_p}(\tilde{J}_{\tilde{p'}}) \subseteq U_0,
    	$$
	where $p'\in \mathcal{L}^u_{\delta_p}(p)$ and $\tilde{p'}=\pi\circ H_p^s(p')$.
	
	In the same manner as above, we deal with the another part: 
	For the given open set $V\subseteq\mathbb{T}^2\times[0,1]$, we choose $l_0^u>0$ with 
    	$$
    	F^{-l_0^{u}}(V)\cap\mathcal{W}^u_{loc}(I_q)\neq\varnothing.
    	$$
	Denote by $V_0 = F^{-l_0^{u}}(V)$, 
	we take 
    $$\tilde{q}\in \left(\pi\circ V_0\right) \cap \mathcal{L}^u(q)$$
    with corresponding sizes $\epsilon_q, \tilde{\epsilon_q}$ 
	such that the following holonomy map $H^u_q$ is well-defined
    	$$
    	H^u_q:\mathcal{L}^s_{\epsilon_q}(q) \times [0,1]
    	\to\mathcal{L}^s_{\tilde{\epsilon_q}}(\tilde{q}) \times [0,1], ~~
    	H^u_q(z)=\mathcal{W}_{loc}^u(z)\cap
        \mathcal{L}^s_{\tilde{\epsilon_q}}(\tilde{q}) \times [0,1].
    	$$
	Moreover, for the interval $\tilde{J}_{\tilde{q}}\subseteq I_{\tilde{q}} \cap V_0$, 
	we have $H_q^u(J_q)=\tilde{J}_{\tilde{q}}$ with some interval $J_q\subseteq I_q$.
	%and we will fix the iterate number $l_0^u$ with some consideration by the way.
	Thus, take $\delta_s<\epsilon_q$ and denote by 
    	$$
    	\delta_q=\min\limits_{t\in I_q} \max\limits_{z\in\mathcal{W}^u_{\delta_s}(t)}\{d(\pi(z),q)\},	
    	$$ 
	we get, by decreasing $\delta_s$ (hence $\delta_q$), 
	every center interval $J_{q'}$ 
    in $\Gamma^{cs}_q\triangleq\mathcal{W}^s_{\delta_q}(J_q)$ satisfies 
    	$$
    	\tilde{J}_{\tilde{q'}}\triangleq H_q^u(J_{q'})\subseteq V_0 
    	\quad {\rm and} \quad 
        \Gamma^{cs}_{\tilde{q'}}\triangleq
    	\mathcal{W}^s_{\delta_q}(\tilde{J}_{\tilde{q'}}) \subseteq V_0,
    	$$
	where $q'\in \mathcal{L}^s_{\delta_q}(q)$ and $\tilde{q'}=\pi\circ H_q^u(q')$.
	
	Here we mention that, 
	before picking the sizes $\delta_u,\delta_s$, 
	we take the positive integers $k_0^s,l_0^u$ large enough such that 
	\begin{align*}%\label{kl-bd-choice}
    	d_s(p,\tilde{p})&<d_c(b,\phi_p(b)),\\
    	d_u(q,\tilde{q})&<d_c(d,\phi_q^{-1}(d)),
	\end{align*}
	where we denote by $J_p=[a,b]$ and $J_q=[c,d]$. 
	In fact, this comes from the partial hyperbolicity, 
	which means there exist $\lambda,\mu\in(0,1)$ such that, 
    for any $(x,t)\in\mathbb{T}^2\times [0,1]$,
    	$$
    	\|A^{-1}\|^{-1}=\lambda<\mu< \phi_x'(t)<\mu^{-1} <\lambda^{-1}=\|A\|.
    	$$
	At the same time, note that 
    the center fiber is straight and $\phi_p,\phi_q^{-1}$ are NS-maps, 
	we also take the positive integers $k_0^s,l_0^u$ large enough such that 
	there exists $Q>0$ such that 
	\begin{align*}%\label{kl-Q-choice}
    	d_s(x,H_p^s(x)) &\leqslant Q\cdot d_s(\pi(x),\pi(H_p^s(x))),\\
    	d_u(y,H_q^u(y)) &\leqslant Q\cdot d_u(\pi(y),\pi(H_q^u(y))),
	\end{align*}
	where $x,y$ are near the boundary $\Gamma_0$ with 
    	$$
    	d_c(x,\pi(x))<b 
    	\quad {\rm and} \quad  
    	d_c(y,\pi(y))<d .
    	$$   


\textbf{\textit{Step 2. Intersections in center dynamics and the projection.}}
	
	
	Now we apply Proposition \ref{prop-Duminy-id} to the following setting. 
	Take $r\in\mathcal{L}^u(p)\pitchfork\mathcal{L}^s(q)$, 
	and denote the holonomy maps from $p$ to $q$ by 
	\begin{align*}
    	H^u_p:\mathcal{L}^s_{loc}(p) \times [0,1]
    	\to\mathcal{L}^s_{loc}(r) \times [0,1],\\
    	H^s_q:\mathcal{L}^u_{loc}(r) \times [0,1]
    	\to\mathcal{L}^u_{loc}(q) \times [0,1],
	\end{align*}
	then for $C^2$ NS-maps $\phi_p=f$, $\phi_q^{-1}=g$, 
	and $C^1$ local diffeomorphism $H\triangleq H_q^s\circ H_p^u=h$,
	we have infinitely many pairs of integers $k_n,l_n>0$ such that 	
	\begin{align}\label{Intersection-bd-1}
    	(H\circ \phi_p^{k_n}(J_p)) \cap \phi_q^{-l_n}(J_q) \neq \varnothing,
	\end{align}
	and we also have $\rho >0 $ satisfying 
	\begin{align}\label{Intersection-bd-1-left}
    	\dfrac{| \phi_p^{k_n}(J_p) \cap (H^{-1}\circ\phi_q^{-l_n}(J_q)) |}
    	{\phi_p^{k_n}(b)-\phi_p^{k_n+1}(b)}\geqslant \rho,
	\end{align}
	and
	\begin{align}\label{Intersection-bd-1-right}
    	\dfrac{| H\circ\phi_p^{k_n}(J_p) \cap (\phi_q^{-l_n}(J_q)) |}
    	{\phi_q^{-l_n}(d)-\phi_q^{-l_n-1}(d)}\geqslant \rho.
	\end{align}
	
	In other words, for the $k_n,l_n$ large enough with
    	$$
    	\lambda^{-k_n}\cdot\delta_p>2d_u(p,r) 
    	\quad {\rm and} \quad 
    	\lambda^{-l_n}\cdot\delta_q>2d_s(q,r),
    	$$ 
	we can apply Proposition \ref{prop-Duminy-id} to get the intersection 
    from the Item (\ref{Intersection-bd-1}):
	\begin{align*}
	   F^{k_n}(\Gamma_p^{cu}) \cap F^{-l_n}(\Gamma_q^{cs}) \neq \varnothing.
	\end{align*}
	Thus, there exists a center interval $J_r$ in $I_r$ such that 
	\begin{align}\label{Intersection-Ir}
	   J_r= F^{k_n}(J_{p'}) \cap F^{-l_n}(J_{q'}),
	\end{align}
	for center intervals $J_{p'} $ in $\mathcal{W}^u_{\delta_p}(J_p)=\Gamma^{cu}_p$ 
	and $J_{q'}$ in $\mathcal{W}^s_{\delta_q}(J_q)=\Gamma^{cs}_q$.


	
	\begin{figure}[htbp]
		\centering
		\includegraphics[width=7.7cm]{HolCenProj.pdf}
		\caption{Dynamics under the projection}
		\label{HolProj}
	\end{figure}
	

 
	Moreover, since the choices of $\Gamma^{cu}_p$ and $\Gamma^{cs}_q$ satisfy
    	$$
    	\mathcal{W}^u_{\delta_p}(H_p^s(J_{p'})) \subseteq U_0
    	\quad {\rm and} \quad
    	\mathcal{W}^s_{\delta_q}(H_q^u(J_{q'})) \subseteq V_0,
    	$$
	so under the iterates of $F$, we actually get 
    from the Equality (\ref{Intersection-Ir}):
	\begin{align*}
    	F^{k_n}(\Gamma_p^{cu}) \cap F^{-l_n}(V_0) \neq \varnothing,\\
    	F^{k_n}(U_0) \cap F^{-l_n}(\Gamma_q^{cs}) \neq \varnothing.
	\end{align*}
	Here we also obtain $F^{k_n}(U_0)$ and $F^{-l_n}(V_0)$ intersect 
    under the natural projection of $\pi$,
	that is, there exists $\tilde{r}\in\Gamma_0$ 
    near $r$ (see Figure \ref{HolProj}) satisfying 
    	$$
    	\tilde{r}
    	=   \mathcal{L}^u_{loc}(r_s) \pitchfork \mathcal{L}^s_{loc}(r_u)
    	~\in~ \big(\pi\circ F^{k_n}(U_0)\big) \cap \big(\pi\circ F^{-l_n}(V_0)\big),
    	$$
	where 
	\begin{align*}
    	r_s &\triangleq A^{k_n}(\tilde{p'}) = A^{k_n}(h_p^s(p')),\\
    	r_u &\triangleq A^{-l_n}(\tilde{q'})= A^{-l_n}(h_q^u(q')).
	\end{align*}
	

\textbf{\textit{Step 3. The contradiction argument by partial hyperbolicity.}}

 
	Finally, we show the contradiction 
    if it has large distortion in the center direction.
	Denote by 
        $$J_{p'}=[a',b'] \quand J_{q'}=[c',d'],$$
	from the Equality (\ref{Intersection-Ir}), without loss of generality, 
	we suppose in center leaf $I_r$: 
    	$$	
    	c_r<a_r < d_r<b_r,
    	$$
	where
    	$$
    	[a_r,b_r]=F^{k_n}([a',b']) 
        \quad{\rm and}\quad [c_r,d_r]=F^{-l_n}([c',d']).
    	$$
	In what follows, we are going to prove 
        $$ F^{k_n}(U_0) \cap F^{-l_n}(V_0) \neq\varnothing. $$


    	
	\begin{figure}[htbp]
		\centering
		\includegraphics[width=8.6cm]{HolCenDist.pdf}
		\caption{Holonomy center distortion}
		\label{HolDist}
	\end{figure}

 
	
	Otherwise, we can assume the order in $I_{\tilde{r}}$:
	\begin{align}\label{Contradic-0}
    	{\tilde{a_r}}'=\mathcal{W}^u_{loc}(\tilde{a_r})\cap I_{\tilde{r}} 
        > {\tilde{d_r}}'=\mathcal{W}^s_{loc}(\tilde{d_r})\cap I_{\tilde{r}},
	\end{align}
	where 
    	$$
    	\tilde{a_r} \triangleq H_p^s(a_r)\in I_{r_s}
    	\quad{\rm and}\quad
    	\tilde{d_r} \triangleq H_q^u(d_r)\in I_{r_u}.
    	$$
	Then we have $e_r>d_r$ in $I_r$ (see Figure \ref{HolDist}) such that
	\begin{align*}
    	H_q^u(e_r) =\tilde{e_r}>\tilde{d_r} 
    	\quad {\rm and} \quad
    	{\tilde{a_r}}'=\mathcal{W}^s_{loc}(\tilde{e_r})\cap I_{\tilde{r}}.
	\end{align*}
	Note that $e_r>d_r>a_r,$ i.e., $d_c(a_r,e_r)>d_c(a_r,d_r),$ 
	so we get 
	\begin{align}\label{Contradic+0}
    	&d_s(a_r,\tilde{a_r}) + d_u(\tilde{a_r},{\tilde{a_r}}') 
    	+ d_s({\tilde{a_r}}',\tilde{e_r}) + d_u(\tilde{e_r},e_r) \nonumber\\
    	&> d_c(a_r,e_r)  >d_c(a_r,d_r).
	\end{align}	
	
	
	On the one hand, we have  
    	$$
    	d_c(a_r,d_r) = |H_p^u\circ \phi_p^{k_n}(J_p)) 
        \cap H_q^s\circ\phi_q^{-l_n}(J_q)|.
    	$$
	By the Inequalities (\ref{Intersection-bd-1-left}) and (\ref{Intersection-bd-1-right}),
	there exist constants $K_1$ and $K_2$ 
    that are only dependent on $H_p^u$ and $H_q^s$ respectively, such that
	\begin{align}\label{Intersection-bd+1-left}
    	d_c(a_r,d_r)
    	&\geqslant	K_1\cdot | \phi_p^{k_n}(J_p) 
            \cap (H^{-1}\circ\phi_q^{-l_n}(J_q)) | \nonumber\\
    	&\geqslant  K_1\cdot \rho 
            \cdot (\phi_p^{k_n}(b)-\phi_p^{k_n+1}(b)) \nonumber\\
    	&>			K_1\rho\mu^{k_n}d_c(b,\phi_p(b)) \triangleq D_1^n,
	\end{align}
	and		
	\begin{align}\label{Intersection-bd+1-right}
    	d_c(a_r,d_r)
    	&\geqslant	K_2\cdot | H\circ\phi_p^{k_n}(J_p) 
            \cap (\phi_q^{-l_n}(J_q)) | \nonumber\\
    	&\geqslant	K_2\cdot \rho \cdot	
            (\phi_q^{-l_n}(d)-\phi_q^{-l_n-1}(d))	\nonumber\\
    	&> 			K_2\rho\mu^{l_n}d_c(d,\phi_q^{-1}(d)) \triangleq D_2^n.
	\end{align}
	Here, recall that the constant $\mu\in(0,1)$ satisfies 
	   $\phi_x'(t)\in (\mu,\mu^{-1})$.
    %for $(x,t)\in \mathbb{T}^2\times[0,1].$
	
	On the other hand, we can get 
	\begin{align*}
    	&d_s(a_r,\tilde{a_r}) + d_u(\tilde{a_r},{\tilde{a_r}}') 
        	+ d_s({\tilde{a_r}}',\tilde{e_r}) +  d_u(\tilde{e_r},e_r) \\ 
    	&\leqslant	Q\cdot  \{ d_s(r,r_s) + d_u(r_s,\tilde{r}) 
        	+ d_s(\tilde{r},r_u)              +  d_u(r_u, r) \}       \\
    	&=			2Q\cdot \{ d_s(r,r_s) + d_u(r_u, r) \}\\
    	&\leqslant	2Q\cdot \{ d_s(A^{k_n}(p'),A^{k_n}(\tilde{p'})) 
            + d_u(A^{-l_n}(q'),A^{-l_n}(\tilde{q'})) \} \\
    	&\leqslant  2Q\cdot \{ \lambda^{k_n} d_s(p',\tilde{p'}) 	
            + \lambda^{l_n} d_u(q',\tilde{q'}) 	     \} \\
    	&=			2Q\cdot \{ \lambda^{k_n} d_s(p,\tilde{p})		
            + \lambda^{l_n}d_u(q,\tilde{q})		     \} ,
	\end{align*}
	that is,  we have
	\begin{align}\label{Contradic-2}
    	& d_s(a_r,\tilde{a_r}) + d_u(\tilde{a_r},{\tilde{a_r}}') 
    	+ d_s({\tilde{a_r}}',\tilde{e_r}) 
        + d_u(\tilde{e_r},e_r)\nonumber\\
    	& \leqslant  2Q\cdot\lambda^{k_n} d_s(p,\tilde{p}) 
        + 2Q\cdot\lambda^{l_n}d_u(q,\tilde{q})
    	\triangleq R_1^n+R_2^n.
	\end{align}
	Here, note that the $A$-invariant foliations 
    $\mathcal{L}^u$ and $\mathcal{L}^s$ on $\Gamma_0$ are parallel lines, 
	and recall that the uniform constant $Q$   
    comes from the choices of $k_0^s$ and $l_0^u$ above. 
	
	By the partial hyperbolicity of $F$, we have $\lambda<\mu$, 
	so we will get a contradiction when taking $k_n,l_n$ large enough. 
	In fact, by the choices of $k_0^s$ and $l_0^u$, we have
    	$$
    	d_s(p,\tilde{p})<d_c(b,\phi_p(b))
    	\quad {\rm and} \quad
    	d_u(q,\tilde{q})<d_c(d,\phi^{-1}_q(d)),
    	$$
	and note that $\lambda^n/\mu^n\to 0$ and 
    these constants $\rho,K_1,K_2,Q$ are all independent of $k_n,l_n$, 
	we can take $k_n,l_n$ large enough such that 
    	$$
    	D_i^n\geqslant 2\cdot R_i^n
    	\quad {\rm for} \quad i=1,2.
    	$$
	Thus, from the Inequalities (\ref{Intersection-bd+1-left}), 
    (\ref{Intersection-bd+1-right}) and (\ref{Contradic-2}), 
	we will get
	\begin{align*}
    	&d_s(a_r,\tilde{a_r}) + d_u(\tilde{a_r},{\tilde{a_r}}') 
    	+ d_s({\tilde{a_r}}',\tilde{e_r}) + d_u(\tilde{e_r},e_r)\\
    	&\leqslant R_1^n + R_2^n
    	\leqslant 2\cdot\max\limits_{i=1,2}\{R_i^n\}
    	\leqslant \max\limits_{i=1,2}\{D_i^n\}
    	< d_c(a_r,d_r),
	\end{align*}	
	which exactly contradicts the Inequality (\ref{Contradic+0}). 
	So the Assumption (\ref{Contradic-0}) does not hold, we just obtain
    	$$
    	F^{k_n}(U_0) \cap F^{-l_n}(V_0) \neq\varnothing.
    	$$
	
	Hence for the given $U,V$, by taking some $k_n,l_n$ large enough,
	we just obtain the desired positive integer 
    	$$
    	m=k_n+k_0^s+l_n+l_0^u,
    	$$ 
	such that 
    	$$
    	F^{m}(U) \cap V \neq\varnothing.
    	$$ 
     
	This ends the proof of Theorem \ref{main-thm}.
	
\end{proof}

%\section*{Acknowledgement}

\bibliographystyle{amsalpha}
\bibliography{Bib-Kan2023}


\end{document}