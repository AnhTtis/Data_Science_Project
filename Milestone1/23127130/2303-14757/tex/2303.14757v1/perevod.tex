\documentclass{article}
\usepackage[T2A]{fontenc}
\usepackage[cp1251]{inputenc}
\usepackage[english,russian]{babel}
\usepackage[tbtags]{amsmath}
\usepackage{amsfonts,amssymb,mathrsfs,amscd}
%для подключения графики используются стандартные команды, но кроме файла *.eps
%необходимо наличие в текущей директории соответствующего файла *.pdf
%-------------------------------------------------
\usepackage[hyper]{msb-a}



\usepackage[all]{xy}
\usepackage{tikz}

\makeatletter
\gdef\No{{\select@language{russian}\textnumero}}
\makeatother

%\JournalName{МАТЕМАТ�?ЧЕСК�?Й СБОРН�?К}
\JournalName{}
%Пустой аргумент приводит к исчезновению всех атрибутов журнала "Математический
%сборник", файл можно представить в любой другой журнал
%-------------------------------------------------
\numberwithin{equation}{section}
%-------------------------------------------------
\theoremstyle{plain}
\newtheorem{theorem}{Theorem}
\newtheorem{lemma}{|Lemma}[section]
\newtheorem{propos}{Proposition}
\newtheorem{corollary}{Corollary}
%-------------------------------------------------
\theoremstyle{definition}
\newtheorem{definition}{Definition}
\newtheorem{proof}{Proof}\def\theproof{}
\newtheorem{remark}{Remark}

\newtheorem{prav}{Ruler}

%-------------------------------------------------
\def\Re{\operatorname{Re}}
\def\Im{\operatorname{Im}}
\def\const{\mathrm{const}}
\def\RR{\mathbb R}
\def\CC{\mathbb C}
\def\NN{\mathbb N}
\def\RS{\mathfrak R}
\def\bz{\mathbf z}
\def\sH{\mathscr H}
\def\HH{\mathscr H}
\def\mro{\widehat\rho}
%-------------------------------------------------
\begin{document}

\title{ Models of representations for classical series of  Lie algebras }
\author[D.\,V.~Artamonov]{D.\,V.~Artamonov}
\address{Lomonosov Moscow State University}
\email{artamonov@econ.msu.ru}
%второй автор
%\author[I.\,I.~Ivanov]{�?.\,�?.~�?ванов}
%\address{}
%\email{}

%\date{19.03.2008}
\udk{517.588}

\maketitle

\begin{fulltext}

\begin{abstract}
A model of representations of a Lie algebra is a representation which a direct sum of all irreducible finite dimensional representations taken with multiplicity  $1$. 
In the paper an explicit construction of a model of representation for all  series of classical Lie algebras is given. The construction does not differ much for different series. The space of the model is constructed as a space of polynomial solutions of a system of partial differential equations. The equations in this system are constructed form relations between minors of matrices from the corresponding Lie group.  This system has a simplification which is very close to the GKZ system, that is satisfied by
$A$-hypergeometric functions.
 % Построение основано на исследований  некоторой системы уравнений в частных производных, называемой антисимметризованной системой Гельфанда-Капранова-Зелевинского, введённой в  \cite{a5}. Эта система тесно связана с  системой  ГКЗ, которой удовлетворяют  $A$-гипергеометрические функции.
% С помощью этой модели даётся новый поход к определению базиса Гельфанда-Цетлина для классических серий алгебр Ли.

Bibliography: 25 items.
\end{abstract}

\begin{keywords}
Lie algebras, hypergeometric functions, the Gelfand-Tsetlin base.
\end{keywords}

\markright{A-GKZ models of representations}

%\footnotetext[0]{Работа выполнена при поддержке РФФ�? (грант \No~08-01-00317) и
%Программы поддержки ведущих научных школ РФ (грант \No~НШ-3906.2008.1).}

\section{Introduction}

%Пусть дана алгебра Ли  $\mathfrak{g}$.
A model of representations of a Lie algebra is a representation which is a direct sum of all it's finite dimensional irreducible representations taken with multiplicity    $1$.  % Классическим примером такой конструкции является теорема, утверждающая, что пространство функций на  $SO_3(\mathbb{R})$  является моделью для предсталвений  $\mathfrak{o}_3(\mathbb{R})$.



Thus one can think about  the classical Weyl construction as a model of representations of the algebra  $\mathfrak{gl}_n$  \cite{w}.   There exist analogs of the Weyl construction for other classical Lie algebras  \cite{fh}, and even for some exceptional Lie algebras \cite{hu}.
In physical literature the models based on the language of creation and annihilation operators  are use. Thus such an approach in the case of the series   $A$ is used in \cite{bb}  and further papers of these authors. But when one tries to generalize this language to the the case of the series $C$ one faces difficulties. Thus usually small dimensions are considered   \cite{h}, \cite{h1}, \cite{h2}, \cite{h3}.
Also the Zhelobenko's construction is a model  \cite{zh}.


Actually all these three constructions are similar. The construction of the present paper is also similar to them.

Let us continue  the discussion of known models. There exist numerous models of combinatorial  nature, it seems to be impossible to give short review.  Let us return to the mentioned above papers by  Biedenharn and coauthors  \cite{bb}-\cite{h3}.
In paper  \cite{bf} (see also  \cite{bf1}) there was constructed a model of representations for the algebra    $\mathfrak{sl}_3$.   These two papers were the closing paper in a  big series of papers where  where the authors tried to explicitly the Clebsh-Gordan, Rachah coefficients that describe a splitting  of tensor products of representations into irreducible. Flath joint to this activity on it's final stage and although he dealt with  the classical objects he named the classical representation theory the  "mathematical golden mine"  \cite{bf1}.
Inspired by  \cite{bf},  Gelfand and Kapranov wrote a  paper   \cite{gz}, where the notion of a model of representations was introduced.
% - прямой суммы конечномерных неприводимых представлений, взятых с кратностью  $1$.
In this paper some models of geometric nature for all classical  Lie algebras over  $\mathbb{C}$ were constructed. Such written with an inspiration papers devoted to the classical representation theory were the starting point of the present paper.


The present paper can be considered as a continuation of the paper   \cite{a4} ans as a generalization of it's result to other classical Lie algebras.
In  \cite{a4}  for the algebra  $\mathfrak{gl}_n$  we consider the function of independent variables  $A_X$, $X\subset\{1,...,n\}$, anti-symmetric in  $X$.  It turns out that the space of polynomials in these variables satisfying some system  of partial differential equations called the antisymmetrized system of Gelfand-Kapranov-Zelevinsky  (A-GKZ for short, see \cite{a5}).  This model is called the A-GKZ model.  Moreover in this model one naturally constructs a base in each representation. An advantage of this model is the fact that one has both an explicit base and an explicit formulas for the scalar product. This makes possible to do some non-trivial calculations. Thus in \cite{sm},  using the A-GKZ model in the case  $n=3$ explicit simple formulas for an arbitrary Clebsh-Gordan coefficient are obtained. 

Also in  \cite{a4} using   $A$-hypergeometric function a base in Zhelobenko's realization was obtained (in it's construction the A-GKZ system plays a crucial role). In this base one manages to write explicitly the action of generators of the algebra  $\mathfrak{gl}_n$. This construction is called the GKZ base for the Zhelobenko's model.  In   \cite{a4} this base is an important tool in establishing a relation  between the constructed base of the  A-GKZ model and the Gelfand-Tsetlin base. 

There exist analogs of the GKZ base for other Lie algebras of small dimensions \cite{a3}, \cite{a6}.


In the present paper in Section  \ref{systemy}  we construct an analogue of the GKZ and A-GKZ systems for the Lie algebras of the series   $B$, $C$, $D$.  This leads to an A-GKZ model for these algebras (see Section  \ref{modela}).  It is remarkable that the constructions for different series are essentially the same.
In the A-GKZ model one naturally constructs a base.
%  В разделе  \ref{gttl} устанавливается его связь с базисом Гельфанда-Цетлина. Отметим, что в настоящей работе найдена процедура, работающая сразу для всех серий и значительно более изящная, чем в работе  \cite{a4}.


Also for the series $B$, $C$, $D$  we construct a GKZ base in the Zhelobenko's model and establish formulas for the action of generators in this base  (see  Section \ref{zhlb}). Using this result in Section   \ref{monomy} we construct other basis in the Zhelobenko's realization. %получается ответ на такой естественный вопрос - какие мономы от миноров образуют базис в реализации Желобенко?


Also we investigate a relation between the constructed base in the A-GKZ model and the Gelfand-Tsetlin base. 

Firstly we suggest a new point of view to the notion of the Gelfand-Tsetlin diagram  (see Section  \ref{dgc},  Definition  \ref{d1}).
There exists a bijective correspondence between the objects introduced in Definition  \ref{d1} and the traditional Gelfand-Tsetlin diagrams.  But in our approach the formulas for the action of generators and so on are written more natural (see Section \ref{gkzt}).
The Gelfand-Tselin diagrams in our sense  are indexing base vectors in irreducible representations. Of course there are numerous constructions of sets that solve the same problem.  Usually they are constructed as sets of integer points in some polytops (the string polytps of Berenstein-Zelevinsky-Littelmann \cite{abz}, the polytops of Vinberg-Littelmann-Feigin-Fourier  \cite{ffl}).
The objects introduced in   \ref{d1}  can also be obtained as integer points in some polytop. But we use the term the Gelfand-Tsetlin diagram since one can easily restore the highest weights for a chain of subalgebras  that appears  in the standard procedure of construction of the Gelfand-Tsetlin base (diagrams for classical series in the traditional  sense can be found in\cite{m}).


Secondly we prove that the transition matrix relation the base of the A-GKZ model and the Gelfand-Tsetlin base is triangular. We show that the Gelfand-Tsetlin base is nothing but an orthogonalization of the A-GKZ base.

%Во-вторых даётся  новое определение базис, не вовлекающее процедуру ограничения на подалгебры, см. Теорему \ref{tgc}, излагающую конструкцию которая может быть положена в основу определения базиса Гельфанда-Цетлина. Замечательно, что  эта конструкция в сущности одна и та же для всех серий алгебр Ли.


The structure of the paper is the following. Section  \ref{oob}, \ref{km}, \ref{seriaa}  are introductory. In  \ref{oob}  the  basic notions are introduce. In \ref{km} the Zhelobenko's model is discussed, here we obtain an explicit description of the Zhelobenko's model for the series   $B$, $C$, $D$,  supplementing the results of the book  \cite{zh}. In  \ref{seriaa} we present the results for the series  $A$, which are modifications of the results from  \cite{a4}.


The main result of the paper can be found in Sections  \ref{systemy}, \ref{dgc}, \ref{mdl}. In   \ref{systemy}  the A-GKZ system for the series $B$, $C$, $D$ is introduced, in Section  \ref{dgc}  a new definition of a Gelfand-Tsetlin diagram for this series is given,  in Section \ref{mdl}  the A-GKZ model and the GKZ base for the Zhelobenko's model are constructed.

Finely in Section \ref{gttl}  a relation of the constructed base of the A-GKZ model and the Gelfand-Tsetlin base  is discussed.

\section{The basic objects}
\label{oob}

In this Section we give an definition of an important class of functions that plays a crucial role in the present paper. We present a system of equations satisfied by these functions. Also we present the Lie algebras used in the present paper. 

\subsection{ A $\Gamma$-series}
\label{r2}

 A detailed information about a $\Gamma$-series can be found  in \cite{GG}.

Let $\mathcal{B}\subset \mathbb{Z}^N$ be a lattice, let $\gamma\in \mathbb{Z}^N$ be a fixed vector. Define a {\it  hypergeometric 
	$\Gamma$-series }  in variables  $z_1,...,z_N$ by the formula

\begin{equation}
\label{gmr}
\mathcal{F}_{\gamma}(z,\mathcal{B})=\sum_{b\in
	\mathcal{B}}\frac{z^{b+\gamma}}{\Gamma(b+\gamma+1)},
\end{equation}
where  $z=(z_1,...,z_N)$.  The following  multi-index notation  is used

$$
z^{b+\gamma}:=\prod_{i=1}^N
z_i^{b_i+\gamma_i},\,\,\,\Gamma(b+\gamma+1):=\prod_{i=1}^N\Gamma(b_i+\gamma_i+1).
$$

Note that  if at least one of the components of the vector  $b+\gamma$ is negative integer, then the corresponding summand in     \eqref{gmr} vanishes. Due to this fact in the considered in the paper    $\Gamma$-series there are finitely many terms.  For simplicity we shall write factorials instead of $\Gamma$-functions.


An $A$-hypergeometric function satisfies a system of partial differential equations called the Gelfand-Karpanov-Zelevinsky (shortly GKZ) system, which consists of equations of two types.

{\bf 1.} Let  $a=(a_1,...,a_N)$ be a vector orthogonal to the lattice  $\mathcal{B}$, then

\begin{equation}
\label{e1}
a_1z_1\frac{\partial}{\partial z_1}\mathcal{F}_{\gamma}+...+a_Nz_N\frac{\partial}{\partial z_N}\mathcal{F}_{\gamma}=(a_1\gamma_1+...+a_N\gamma_N)\mathcal{F}_{\gamma},
\end{equation}
it is enough to consider only the base vectors of the lattice orthogonal to  $\mathcal{B}$.


{\bf 2.} Let  $b\in \mathcal{B}$ и $b=b_+-b_-$, where all coordinates of the vectors  $b_+$, $b_-$ are non-negative.  Chose in these vectors non-zero elements  
$b_+=(...b_{i_1},....,b_{i_k}...)$,  $b_-=(...b_{j_1},....,b_{j_l}...)$. Then

\begin{equation}
\label{e2} (\frac{\partial }{\partial
	z_{i_1}})^{b_{i_1}}...(\frac{\partial}{\partial z_{i_k}})^{b_{i_k}}
\mathcal{F}_{\gamma}=(\frac{\partial }{\partial
	z_{j_1}})^{b_{j_1}}...(\frac{\partial }{\partial z_{j_l}})^{b_{j_l}} \mathcal{F}_{\gamma}.
\end{equation}

It is enough to consider only base vectors  $b\in \mathcal{B}$.




\subsection{ Algebras $\mathfrak{o}_{2n+1}$,  $\mathfrak{o}_{2n}$, $\mathfrak{sp}_{2n}$}
\label{algl}

The Lie algebras  $\mathfrak{o}_{2n}$, $\mathfrak{sp}_{2n}$ are considered as subalgebras in the Lie algebra of all  $2n\times 2n$ matrices,   whose rows and columns are indexed by   $i,j=-n,...,-1,1,...,n$, and the algebra   $\mathfrak{o}_{2n+1}$   is a subalgebra  in the Lie algebra of all $(2n+1)\times (2n+1)$ matrices, whose rows and columns are indexed by    $i,j=-n,...,-1,0,1,...,n$.


The algebras $\mathfrak{o}_{2n+1}$ and   $\mathfrak{o}_{2n}$
are generated by matrices

\begin{equation}\label{fbd}F_{i,j}=E_{i,j}-E_{-j,-i},\end{equation}

where $i,j=-n,...,-1,0,1,...,n$ in the case $\mathfrak{o}_{2n+1}$ and  $i,j=-n,...,-1,1,...,n$   in the case $\mathfrak{o}_{2n}$.

The algebra $\mathfrak{sp}_{2n} $  is generated by the matrices
\begin{equation}\label{fd}F_{i,j}=E_{i,j}-sign(i)sign(j)E_{-j,-i},\end{equation}
where $i,j=-n,...,-1,1,...,n$.


Denote the Lie algebras  $\mathfrak{o}_{2n+1}$,  $\mathfrak{o}_{2n}$ or  $\mathfrak{sp}_{2n}$ as $g_n$.

To be able to talk about a Gelfand-Tsetlin base one needs to fix a chain of subalgebras. %Подалгебру  $\mathfrak{o}_{2n-1}$,   $\mathfrak{o}_{2n-2}$, $\mathfrak{sp}_{2n-2} $ мы выбираем как
%порождённую  $F_{i,j}$ для $i,j=-n,...,-2,2,...,n$ в случае серий $C$, $D$ и   $i,j=-n,...,-2,0,2,...,n$ в случае серии $B$. При дальнейшем выборе
A subalgebra $g_{n-k}\subset g_{n}$ is defined as a span  $<F_{i,j}>_{i,j\neq \pm 1,...,\pm (k-1)}$.

%Также мы будем использовать  алгебру
%  $\mathfrak{gl}_{n+1}$ всех матриц $(n+1)\times (n+1)$,  чьи строки и столбцы занумерованы индексами  $i,j=-n,...,-1,1$. Эта алгебра порождена матрицами  $E_{i,j}$,   $i,j=-n,...,-1,1$.  Выделим подалгебру   $\mathfrak{gl}_{n-1}$ как линейную оболочку матриц  $E_{i,j}$,    $i,j=-n,...,-2$.





\section{ The Zhelobenko's model}
\label{km}

In this Section we present a model of representations realized in the space of functions  on the corresponding Lie group. Zhelobenko (see  \cite{zh})  has proved the Theorem  \ref{tzh0}  which describes the space of this model in the case of the series  $A$. We formulate and prove the Theorem   \ref{tzh},  which is a generalization of the Zhelobenko's result to the cases of other series. 

In these Theorems the space of a representation is described as a space of solution of  a system of partial differential equations. A more explicit description is given in the Theorem \ref{ost1}.

% В разделе ??? обсуждается построение базиса в этом просранстве.

\subsection{Functions on a group}

%Так как модель Желобенко весьма изящна





Consider a space of function of a group
$G$. Onto a function
$f(g)$,  $g\in G$ an element $X\in G_{}$ acts by right shifts  
\begin{equation}
\label{xf}
(Xf)(g)=f(gX).
\end{equation}

Thus the space  $Fun$ of all function of a group    $G$ is a representation of   $G$ and thus of the Lie algebra    $Lie G$.

Let  $G=Sp_{2n}$, $SO_{2n+1}$, $SO_{2n+1}$  and let
$a_{i}^{j}$, $i,j=-n,...,n$\footnote{In the cases $G=Sp_{2n}$, $O_{2n}$ one takes $i,j=-n,...\hat{0}...,n$, and in the case  $G=SO_{2n+1}$  one takes $i,j=-n,...,0,...,n$} -
be a function of a matrix element which is a function on a group $G$. Here $j$ is a row index, and  $i$ is a column index.




%, но между  дифференциальными операторами мы соотношений вводить не будем.






Also put

\begin{equation}
\label{dete}
a_{i_1,...,i_k}:=det(a_i^j)_{i=i_1,...,i_k}^{j \in\text{the first  $k$ rows}}.
\end{equation}


That is one takes a determinant of a submatrix in a matrix $(a_i^j)$,
оformed by the first  $k$ rows and columns  
$i_1,...,i_k$


Also in the case of the series   $D$ one puts
\begin{equation}
\bar{a}_{i_1,...,i_n}:=det(a_i^j)_{i=i_1,...,i_n}^{j=-n,...,-2,1}.
\end{equation}


One has

\begin{equation}
(a_{-n,...,-2,-1})^{-1}= \bar{a}_{-n,...,-2,1}.
\end{equation}

Using  formulas \eqref{xf}, \eqref{dete} one can obtain formulas for the action of generators of the algebra onto these determinants.  To do it let us formally \footnote{The expression  \eqref{edet1} should be understood symbolically since it does not define an action of    $E_{i,j}$, because it does not respect the relations between determinant in the cases of groups   $Sp$,  $SO$.} introduces an action of an operator
$E_{i,j}$ onto a determinant:
% действует путём действия на
%столбцовые индексы

\begin{equation}
\label{edet1}
E_{i,j}a_{i_1,...,i_k}=a_{\{i_1,...,i_k\}\mid_{j\mapsto i}},
\end{equation}


where  $.\mid_{j\mapsto i}$ is a substitution of   $j$ instead of
$i$, and in the case when  $j\notin \{i_1,...,i_k\}$, then one obtains zero.
Now for the Lie algebra of the series  $B, C, D$ one defines an action of  $F_{i,j}$  through the action of  $E_{i,j}$.



Now let us give an explicit formulas for the highest vector of a given highest weight  (see  \cite{bb}, 
\cite{zh}). The vector

\begin{align}\begin{split}
\label{stv0} &v_0=\prod_{k=-n}^{-2} (a_{-n,...,-k})^{m_{-k}-m_{-k+1}}
a_{-n,...,-2,-1}^{m_{-1}} \text{  for the series $B$, $C$, $D$ in the case
	$m_{-1}\geq 0$},\\
&v_0=\prod_{k=-n}^{-2} (a_{-n,...,-k})^{m_{-k}-m_{-k+1}}
\bar{a}_{-n,...,-2,1}^{-m_{-1}} \text{ for the series  $D$ in the case $m_{-1}<
	0$}
\end{split}\end{align}
%где  $1$ есть тождественно равная единице функция,
is a highest vector for  $g_{n}$ with the highest weight
$[m_{-n},...,m_{-1}]$ for all the series  $B$, $C$, $D$.
%  и  $[m_{-n},...,m_{-1},0]$ для серии  $A$.
Note that in the case of the integer highest weight this is a polynomial function. In the case  $B$, $D$
when the highest weight is half-integer the last factor has a fractional exponent.  Thus one obtains a multi-valued function. It becomes single-valued when one passes toe the group   $Spin$.


\begin{definition}
A direct sum of subrepresentations in  $Fun$,  generated by highest vectors \eqref{stv0}  is called the Zhelobenko's model. The space of this model is denoted by     $Zh$.
\end{definition}


\subsection{Relations between determinant}

\subsubsection{ The Plucker relations}

For all series between the minors $a_{i_1,...,i_k}$ of an  $m\times m$  matrix one has  the Plucker relations 

\begin{equation}
\sum_{t=1}^{k+1} (-1)^t a_{i_1,...,i_{k-1},j_t} a_{j_1,...,j_{t-1}, j_{t+1},...,j_{k+1}}=0
\end{equation}

These relations are sufficient conditions that guarantee that the collection of numbers  $a_{i_1,...,i_k}$ is a collection of minors of type \eqref{dete} of some matrix.
That is one has.

\begin{lemma}[see  \cite{rel}]
In the case of the series 	  $A$   the ideal  $I_{\mathfrak{gl}_m}$ of relations between the determinants   $a_X:=a_{i_1,...,i_k}$ is generated by the Plucker relations.
\end{lemma}



\subsubsection{The Jacobi relations}

In the case of the series $B$, $C$, $D$  there exist other relations.
To derive them let us use the fact that between minors of a matrix   $(a_i^j)$ and it's inverse $((a^{-1})_i^j)$ there exist the Jacobi relations \cite{gm}:

$$
a_{i_1,...,i_k}^{j_1,...,j_k}=det(a)(-1)^{\sum_{p=1}^k i_p+\sum_{q=1}^k j_q}(a^{-1})^{\widehat{i_1},...,\widehat{i_k}}_{\widehat{j_1},...,\widehat{j_k}}
$$

Let us write the corollaries of these relations in the cases   $B$, $C$, $D$. Let  $X$ be an element of the corresponding Lie group.  Then 


\begin{align}
\begin{split}
\label{omg}
& X^t\Omega X=\Omega \Leftrightarrow X^{-1}=\Omega^{-1}X^t\Omega,\,\,\,\,\,  \Omega=(\omega_{i,j}),
\text{ where }\\
&\omega_{i,j}=\begin{cases} +1,\,\, i=-j, i<0\\ -1,\,\, i=-j, i>0\\ 0 \text{  otherwise }  \end{cases} \text{ for the series  }C,\,\,\,\,\,\,\,\,\,
\omega_{i,j}=\begin{cases} +1,\,\, i=-j,\\  0 \text{ otherwise }  \end{cases} \text{ for the series  }B,D
\end{split}
\end{align}

Note that  $\Omega^{-1}=-\Omega$  in the case of the series  $C$  and  $\Omega^{-1}=\Omega$  for the series   $B$, $D$.
% Поэтому

%\begin{align*}
% &(\Omega^{-1} x\Omega)_{i}^j=\sum_{k,l}
%\end{align*}

Thus for the matrix elements one has 
\begin{align*}
&a_{i}^j=a_{-j}^{-i}\text{  for the series  $B,D$},\,\,\,\,\,\,\,a_{i}^j=sing(i)sign(j)a_{-j}^{-i}\text{ for the series  $C$}
\end{align*}



Since for all groups one has $det(a)=1$, we obtain:

\begin{align}
\begin{split}
\label{rwo}
& a_{i_1,...,i_k}^{-n,...,-n+k-1}=(-1)^{i_1+...+i_k}(-1)^{-n+...+(-n+k-1)}(a^{-1})^{\widehat{i_1},...,\widehat{i_k}}_{\widehat{-n},...,\widehat{-n+k-1}}=\\
&=(-1)^{i_1+...+i_k}(-1)^{\frac{(-2n+k-1)k}{2}}\cdot\begin{cases} (a_{-j}^{-i})^{\widehat{i_1},...,\widehat{i_k}}_{\widehat{-n},...,\widehat{-n+k-1}}    \text{ for the series  $B$, $D$},  \\ (sing(i)sing(j)a_{-j}^{-i})^{\widehat{i_1},...,\widehat{i_k}}_{\widehat{-n},...,\widehat{-n+k-1}}    \text{ for the series  $C$}  \\   \end{cases}=\\&
=(-1)^{i_1+...+i_k}(-1)^{\frac{(-2n+k-1)k}{2}}\cdot\begin{cases} (a_{i}^{j})_{\widehat{-i_1},...,\widehat{-i_k}}^{\widehat{n},...,\widehat{n-k+1}}    \text{ for the series  $B$, $D$},  \\ (sing(i)sing(j)a_{i}^{j})_{\widehat{-i_1},...,\widehat{-i_k}}^{\widehat{n},...,\widehat{n-k+1}}    \text{ for the series  $C$} \end{cases}
\end{split}
\end{align}

\begin{remark}
	\label{rmmk}
Note that when we were deriving 	the equality \eqref{rwo}  we were considering for the matrices on the right and on the left in the equality  $X^{-1}=\Omega^{-1}X^t\Omega$ the minors formed by the first  {\it columns }.
\end{remark}
Thus we have proved the following statement. 

\begin{lemma}%[Соотношения Якоби для определителей]
	\label{lj}
	For the series  $B$, $C$,  $D$ one has
	
	\begin{equation}
	\label{sb}
	a_{i_1,...,i_k}=\pm  a_{\widehat{-i_1},...,\widehat{-i_k}}
	\end{equation}
	For the series $D$  one also has
	
	\begin{equation}
	\label{sd}
	%: & a_{i_1,...,i_k}=(-1)^{i_1+...+i_k}(-1)^{\frac{(-2n+k-1)k}{2}}a_{\widehat{-i_1},...,\widehat{-i_k}} ,\\
	\bar{a}_{i_1,...,i_k}=\pm \bar{a}_{\widehat{-i_1},...,\widehat{-i_k}}.
	\end{equation}
	
	Here
	
	\begin{equation}
	\label{znk}\pm =s\cdot(-1)^{i_1+...+i_k}(-1)^{\frac{(-2n+k-1)k}{2}},
	\end{equation}  where  $s=1$ in the cases  $B$, $D$  and   $s$ equals $-1$ to the power which equals to the number of rows and columns with negative indices in the case of the series  $C$.
	
\end{lemma}

 The statement \eqref{sd} for the series $D$  can be obtained using reordering or rows.
 


%  выглядят так

%\begin{align*}
% & a_{i_1,...,i_k}=(-1)^{I}a_{\widehat{-i_1},...,\widehat{-i_k}} \text{для серии} B,D,\\
% & a_{i_1,...,i_k}=(-1)^{I}(-1)^ka_{\widehat{-i_1},...,\widehat{-i_k}} \text{для серии} C.
%\end{align*}

Introduce a shorter notation for determinants. If  $X\subset \{-n,...,n\}$, then $a_X:=a_{i_1,...,i_k}$. Analogously introduce a notation   $\bar{a}_X$ in the case of the algebra  $\mathfrak{o}_{2n}$ and  $|X|=n$. In these notations  \eqref{sb} one writes

$$
a_X=\pm a_{\widehat{-X}}.
$$


One has the following statement.

\begin{lemma}
	\label{lms}  Let  $G$ be a group $Sp_{2n}$, $SO_{2n+1}$, $SO_{2n+1}$.  Than is a group of matrices that preserve the bilinear form with the matrix $\Omega=(\omega_{i,j})$,  where $\omega_{i,j}$ are defined in  \eqref{omg}.
	Then the ideal of relations between the determinants $a_X$ of type \eqref{dete} is generated by the Plucker and the Jacobi relations.  The same is true in the case of the series  $D$,  where instead of determinants  $a_X$ of the order  $n$  one takes the determinans $\bar{a}_X$.
	
\end{lemma}
\begin{proof}

Firstly consider the determinants  $a_X$.
%Случай, когда  для серии  $D$ мы   вместо определителей порядка  $n$  берём  $\bar{a}_X$ рассматривается аналогично.
One has.

\begin{propos}
	\label{prp}
	Let us be given matrices  $O_1$ and $O_2\in G$. If all their minors constructed on columns that belong to an arbitrary subset   $X$ and first consecutive rows  (i.e. minors of type \eqref{dete}) are equal, then $O_1=TO_2$, where $T$ is a lower-unitriangular matrix. Analogously  if all their minors constructed on  rows that belong to an arbitrary subset  $X$ and first consecutive columns then $O_1=O_2T$, where  $T$ is an upper-unitriangular matrix.
\end{propos}

%�?так, знание определителей $a_X$ определяет элемент $g\in G$ с точностью до умножения на нижне-унитреугольную матриц.

Let us return to the proof of the Lemma \ref{lms}.
Using the equality  \eqref{rwo} and the Remark  \ref{rmmk}, one obtains the Corollary.
\begin{corollary} 
Let for minors of $O$ one has 
$a_{X}=\pm a_{\widehat{-X}}$.
Then one has  $\Omega^{-1} O^t\Omega=O^{-1}T$,  
where $T$ is an upper-unitriangular matrix.  \end{corollary}


%Рассмотрим матрицу  $O_1=\Omega O^t\Omega$  и  $O_2=O^{-1}$. Для них соотношение $a^1_X=a^2_X$  для них есть в точности соотношение   $a_{X}=a_{\widehat{-X}}$.

Consider now the ideal  $I_{g_n}$ of relations between the determinants. The following   {\bf fact } takes place:  the ideal  $I_{g_n}$ is a set of polynomials  $f(a_X)$, such that their vanishing is a necessary and a sufficient condition providing that   $a_X$ of a matrix $O$, such  that  $\Omega^{-1} O^t\Omega=O^{-1}$.

The fact that that the condition   $a_{X}=\pm a_{\widehat{-X}}$ is necessary was proved above, one needs to prove that the presented in the formulation of the Lemma conditions are sufficient. 


Consider an ideal in the ring of polynomials in variables  $a_X$, generated by  $I_{\mathfrak{gl}_m}$  (where $m$ is a size of matrices that form  $G$)  and the Jacobi relations  $a_{X}=\pm a_{\widehat{-X}}$. The fact that the relations from the ideal $I_{\mathfrak{gl}_m}$  hold provide that these variables can be considered as minors of a matrix $O$.

The considerations \eqref{rwo} and the Remark  \ref{rmmk} show that the Jacobi relations provide that   $\pm \Omega O^t\Omega=O^{-1}T$ for an upper-unitriangular matrix $T$, where $\pm=-$ for the series  $C$ and  $+$ for the series  $B$, $D$.
This equality is equivalent to the following one:  $\pm O\Omega O^t\Omega=T $, from here one gets   that $T\Omega=O\Omega O^t$ is a skew-symmetric matrix for the series   $C$ and a symmetric matrix for the series $B$, $D$. Thus there exists a low-unitrangular matrix  $X$, such that $X^{}T \Omega X^{t}=\Omega$.

For the matrix $XO$ one has  $\pm \Omega(XO)^t\Omega=(XO)^{-1}$, that is  $XO$  belongs to the considered group    $G$.  Since  $X$  is a low-unitrangular matrix then the determinants of type   \eqref{dete} for matrices  $XO$ and  $O$ coincide.

Thus the relations from the ideal  $I_{\mathfrak{gl}_m}$ and the Jacobi relations provide that   $a_X$  are minors of a matrix   $XO\in G$.  As we have noted this implies that  the ideal generated by  $I_{gl_n}$ and the Jacobi relations is the ideal of all relations between the determinants   $a_X$ for the group  $G$.


The case of the series  $D$  and determinants  $\bar{a}_X$  is reduced to the considered case by reordering of rows.   

\end{proof}

\subsubsection{ Some useful relations}


Let us derive some relations that are corollaries of the Plucker and the Jacobi  relations that we shall use below. 

\begin{propos}
	\label{dopson}
	In the case of the series $B$ one has
	
	\begin{equation}
	\label{s22}
	a_{\pm n,...,\widehat{\pm i}...,\pm 1,0}^{2}=-2\cdot a_{\pm n,...,\widehat{\pm i}...,\pm 1,-i}a_{\pm n,...,\widehat{\pm i}...,\pm 1,i}
	\end{equation}
	The choice of the sign on both sides is  the same.
 
\end{propos}
\begin{proof}

Using a group automorphism that acts as a permutation of coordinates  $1,...,n$, one concludes that one can consider just the case   $i=1$

One has the Plucker relations

\begin{equation}
\label{s211}
a_{\pm n,...,\pm 2,-1}a_{\pm n,...,\pm 2,1,0}+a_{\pm n,...,\pm 2,1}a_{\pm n,...,\pm 2,-1,0}+a_{\pm n,...,\pm 2,,0}a_{\pm n,...,\pm 2,,-1,1}=0
\end{equation}

Take the Jacobi relations 

\begin{align*}
&a_{\pm n,...,\pm 2,1,0}= a_{\pm n,...,\pm 2,-1},&  a_{\pm n,...,\pm 2,-1,0}= a_{\pm n,...,\pm 2,-1},&  a_{\pm n,...,\pm 2,-1,1}=- a_{\pm n,...,\pm 2,0}.
\end{align*}

Applying them in  \eqref{s211}  one obtains \eqref{s22} for  $i=1$.

\end{proof}

%Константу  в  \eqref{s22} несложно вычислить. Например, если  $i=1$, то она равна  $-2$.





\subsection{ The conditions defining an irreducible representation in the Zhelobenko's model }
\label{rzd3}

\subsubsection{The general theorem}

The indicators system is the system of differential equations of the following form:
\begin{align}
\begin{split}
\label{indsys}
&L_{-n,-n+1}^{r_{-n}+1}f=0,...,L_{-2,-1}^{r_{-2}+1}f=0,\,\,\,,L_{-1,1}^{r_{-1}+1}f=0\text{in the case of the series $A$, $C$},\\
&L_{-n,-n+1}^{r_{-n}+1}f=0,...,L_{-2,-1}^{r_{-2}+1}f=0,\,\,\,,L_{-1,0}^{r_{-1}+1}f=0\text{ in the case of the series  $B$,}\\
&L_{-n,-n+1}^{r_{-n}+1}f=0,...,L_{-2,-1}^{r_{-2}+1}f=0,\,\,\,L_{-2,1}^{r_{-1}+1}f=0\text{
	in the case of the series  $D$}.
\end{split}
\end{align}

Here  $L_{-i,-j}$ is an operator acting onto a function  $f(a)$
 as a left infinitesimal shift by 
$F_{-i,-j}$ for the series $B$, $C$, $D$ and by $E_{-i,-j}$ for the series $A$.
%Показатель   $r_{-i}$
% определяются равенством $<\alpha_i,[m]>$, где $\alpha_i$ корень, соответствующий элементу $F_{-i,-j}$ или $E_{-i,-j}$,

The exponent  $r_{-i}$  is defined as follows:


\begin{align}
\begin{split}
\label{rb}
&r_{-n}=m_{-n}-m_{-n+1},...,r_{-2}=m_{-2}-m_{-1},\,\, r_{-1}=m_{-1}\text{in the case of the series  $A$, $C$,}\\
&r_{-n}=m_{-n}-m_{-n+1},...,r_{-2}=m_{-2}-m_{-1},\,\, r_{-1}=2m_{-1}\text{ in the case of the series  $B$,}\\
&r_{-n}=m_{-n}-m_{-n+1},...,r_{-2}=m_{-2}-|m_{-1}|,\,\,
r_{-1}=m_{-2}+|m_{-1}|\text{ in the case of the series $D$}.
\end{split}
\end{align}



In \cite{zh}  the following was proved.
\begin{theorem}\label{tzh0}
	For the series  $A$ one has the following statement. Let the group   $GL_{n+1}$ act in the space with coordinates indexed by   $-n,...,-1,1$\footnote{such an indexation is used for a formulation of analogous results for other series}. Then the space of functions on the group   $GL_{n+1}$ is an irreducible representation with the highest weight  $[m_{-n},...,m_{-1},0]$ and the highest vector  \eqref{stv0}
	is defined by the following conditions.
	
	
	\begin{enumerate}
		%\item $f$ есть многочлен от матричных элементов.
		\item $L_{-}f=0$, where  $L_{-}$ is a left infinitesimal shift by an arbitrary element of $GL_{n+1}$ corresponding to a negative root.
		\item $L_{-i,-i}f=m_{-i}f$, where $L_{-i,-i}$, $i=-1,1,...,n$ is a left infinitesimal shift by an element of the group 
		$GL_{n+1}$, corresponding to a Cartan element    $E_{-i,-i}$.
		\item $f$  satisfies the indicator system.% $L_{\alpha_{-i}}^{r_{-i}+1}f=0$,  $i=1,...,n$.
	\end{enumerate}
	
\end{theorem}

This  Theorem describes the space of the model $Zh$ in the case of the series $A$.
Let us prove an analogous statement for the series  $B$, $C$, $D$.


\begin{theorem}
	\label{tzh}
	In the Zhelobenko's realization for the series  $B$, $C$, $D$ an irreducible representation with the highest vector  \eqref{stv0}
	is defined by the conditions $1,3$  and the condition $2$ where $E_{-i,-i}$ is changed to	$F_{-i,-i}$ in the definition of   $L_{-i,-i}$.
	%, а также модифицированным условием $1$:
	
	%\begin{itemize}
	%  \item Если старший вес целочисленные и  неотрицательный, то  $f$ есть многочлен от определителей.
	%  \item Если в случае серии  $D$  старший вес целочисленные, но $m_{-1}<0$, то  $f$ есть линейная комбинация произведений определителей. В каждое произведение определитель входит в целой неотрицательной %степени, кроме  $a_{-n,...,-2,-1}$, который входит в степенях  $0,-1,-2,...,m_{-1}$.
	%  \item Если старший вес полуцелый и  неотрицательный, то  $f$ есть линейная комбинация произведений определителей. В каждое произведение определитель входит в целой неотрицательной степени, кроме  %$a_{-n,...,-2,-1}$, который входит в степенях  $m_{-1},m_{-1}-1,...,\frac{1}{2},-\frac{1}{2}$.
	%  \item Если  в случае серии  $D$ старший вес полуцелый и  $m_{-1}<0$, то  $f$ есть линейная комбинация произведений определителей. В каждое произведение определитель входит в целой неотрицательной %степени, кроме  $a_{-n,...,-2,-1}$, который входит в степенях  $\frac{1}{2},-\frac{1}{2},....,m_{-1}$.
	%\end{itemize}
	
	
	
	%\begin{enumerate}
	%\item $f$ есть многочлен от определителей порядков $1,...,n$; а также
	%$a_{-n,...,-2,-1}^{1/2}$ в случае полуцелого старшего веса для серий $B$, $D$; и ещё  $(a_{-n,...,-2,-1})^{-1}$
	%в случае серии $D$ и $m_{-1}<0$. При этом степень  $(a_{-n,...,-2,-1})^{-1}$ не выше  $m_{-1}$.
	%\end{enumerate}
\end{theorem}

The  proceeding part of  Section   \ref{rzd3} is devoted to the proof of this Theorem.

\begin{proof}
	
	The scheme of the proof is the following. First of all one gives formulas for the action of an operator of left infinitesimal shift onto determinant. Using them one proves that the highest vector satisfies the conditions 1-3. The main difficult is to prove that the highest vector satisfies the indicator system. Then one easily proves that an arbitrary vector satisfies the conditions  1-3. Finally one proves that among  the functions that satisfy the conditions   1-3 there is nothing but functions that vectors of the representation with the highest vector 	\eqref{stv0}.
	
	
Let us proceed to realization of this plan.	
Consider the determinant
	%$$
	$a_{i_1,...,i_k}$
%	=det(a_i^j)_{i=i_1,...,i_k}^{j=-n,...,-n+k-1}$, $k\leq n$
	%$$
and let us find an action of the left infinitesimal shift onto it. Let us introduce a temporary notation for determinants
	
	$$
	a^{-n,...,-n+k-1}_{i_1,...,i_k},
	$$
	where we write also the indices of rows (on the up). Then the operator   $L_{-i,-j}$ of the left infinitesimal shift acts onto row indices $-n,...,-n+k-1$ by the following ruler. For the series  $A$  the operator of the left infinitesimal shift $E_{-i,-j}$   acts
	% на верхние индексы $\{-n,...,-n+k-1\}$
	by the ruler
	
	\begin{equation}
	\label{lij}
	L_{-i,-j}a^{-n,...,-n+k-1}_{i_1,...,i_k}=a^{\{-n,...,-n+k-1\}\mid_{-i\mapsto -j}}_{i_1,...,i_k}.
	\end{equation}
	
	 The action  of the operator of the left infinitesimal shift by   $F_{-i,-j}$ for the series $B$,
	$C$, $D$  can be expressed through these operators.
	
	Analogously one proves that 
	
	\begin{equation}
	\label{fii}
	L_{-i,-i}a^{-n,...,-n+k-1}_{i_1,...,i_k}=\begin{cases}
	a^{-n,...,-n+k-1}_{i_1,...,i_k}\text{ if  }
	-i\in\{-n,...,-n+k-1\},\,\,\, \\ \text{ $0$ otherwise }\end{cases}
	\end{equation}
	
	Onto a product of determinants $L_{-i,-i}, L_{-i,-j}$ act by the Leibniz ruler.
	
	Let us prove that the conditions 1-3 for a representation with the highest weight  \eqref{stv0}  are satisfied.
	
	Let us prove that the conditions  1-3  are satisfied for the highest vector  \eqref{stv0}.  The fact that the conditions  1 and 2 are satisfied is proved directly. One has to prove that the condition 3 is satisfied.
	
	From the formula  \eqref{lij} it follows that the operators  $L_{-i,-i+1}$ for $i=-n,...,-3$  being acting onto  \eqref{stv0}  actually are acting onto  $a_{-n,...,-i}^{m_{-i}-m_{-i+1}}$ only.  The operator $L_{-2,-1}$ for the series $A$, $B$, $C$ and  $D$  in the case  $m_{-1}\geq 0$,  the operator $L_{-2,1}$
	for the series  $D$ in the case  $m_{-1}<0$
	being acting onto  \eqref{stv} is acting onto $a_{-n,...,-2}^{m_{-2}-m_{-1}}$ only.
	The operator  $L_{-1,1}$ for the series  $A$, $C$,  the operator $L_{-1,0}$  for the series  $B$ are acting onto determinants of order  $n$ only.
	
	In the case of the series   $D$  one also has the following operators. In the case $m_{-1}\geq 0$ one has the operator $L_{-2,1}$,
	and in the case  $m_{-1}< 0$ -  one has an operator  $L_{-2,-1}$. Both the act onto determinants of orders   $n-1$ and $n$.
	
	Since the exponents  $r_{-i}=m_{-i}-m_{-i+1}$ of the determinant  $a_{-n,...,-n+i-1}$ is integer then by the formula  \eqref{lij}  the equation    $L_{-i,-i+1}^{r_{-i}+1}v_0=0$, $i=-n,...,-2$  is satisfied.  By the same reason the equation  $L_{-1,1}^{r_{-1}+1}v_0=0$  is satisfied for the series  $A$, $C$.
	
	
Let us check that equality  $L_{-1,0}^{m_{-1}+1}v_0=0$ holds for the series  $B$.
	To prove it we need the following formulas. In the case of the series   $B$ one has
	
	
	\begin{align*}
	&L_{-1,0}a_{-n,...,-2,-1}^{-n,...,-2,-1}=a_{-n,...,-2,-1}^{-n,...,-2,0},\,\,\,L_{-1,0}^2a_{-n,...,-2,-1}^{-n,...,-2,-1}=-a_{-n,...,-2,-1}^{-n,...,-2,1},\\&L_{-1,0}^3a_{-n,...,-2,-1}^{-n,...,-2,-1}=0.
	\end{align*}
	
	Using these formula one derives that the equation $L_{-1,0}^{2m_{-1}+1}v_0=0$ holds in the case of the series  $B$ when the highest weight in integer.
	
 Also one has
	
	
	\begin{align*}
	&L_{-1,0}(a_{-n,...,-2,-1}^{-n,...,-2,-1})^{1/2}=\frac{1}{2}a_{-n,...,-2,-1}^{-n,...,-2,0}(a_{-n,...,-2,-1}^{-n,...,-2,-1})^{-1/2},\\
	&L_{-1,0}^2a_{-n,...,-2,-1}^{-n,...,-2,-1}=-\frac{1}{2}a_{-n,...,-2,-1}^{-n,...,-2,1}(a_{-n,...,-2,-1}^{-n,...,-2,-1})^{-1/2}-\\&-
	\frac{1}{4}(a_{-n,...,-2,-1}^{-n,...,-2,0})^2(a_{-n,...,-2,-1}^{-n,...,-2,-1})^{-3/2}=0
	\end{align*}
	
	To obtain the last equality one uses a relation  $$a_{-n,...,-2,-1}^{-n,...,-2,1}a_{-n,...,-2,-1}^{-n,...,-2,-1}=-\frac{1}{2}(a_{-n,...,-2,-1}^{-n,...,-2,0})^2.$$
	%оно может быть доказано так же, как это сделано при доказательстве Леммы \ref{ssp4} ниже.
From these equalities  it follows that $L_{-1,0}^{2m_{-1}+1}v_0=0$ in the case of the series  $B$  when the highest weight is half-integer.
	
	
Now consider the case of the series  $D$ when $m_{-1}\geq 0$. Let us check the the equality $L_{-2,1}^{m_{-2}+m_{-1}+1}v_0=0$ holds. One has 
	\begin{align*}
	&L_{-2,1}a_{-n,...,-2}^{-n,...,-2}=a_{-n,...,-2}^{-n,...,1},\,\,\, L_{-2,1}a_{-n,...,-2}^{-n,...,1}=0,\\
	&L_{-2,1}a_{-n,...,-2,-1}^{-n,...,-2,-1}=2a_{-n,...,-2,-1}^{-n,...,1,-1},\,\,\, L_{-2,1}a_{-n,...,-2,-1}^{-n,...,1,-1}=a_{-n,...,-2,-1}^{-n,...,1,2},\,\,\, L_{-2,1}a_{-n,...,-2,-1}^{-n,...,1,2}=0.
	\end{align*}
	
	Thus  $v_0$ is mapped to zero under the action of   $L_{-2,1}$ to the power which equals to a sum of   $1$  and the size of the determinant 	$a_{-n,...,-2}$ and twice of the size of the determinant    $a_{-n,...,-2,-1}$. That is under the action of   $L_{-2,1}$  to the power 
	$1+(m_{-2}-m_{-1})+2m_{-1}=1+m_{-2}+m_{-1}$.
	
In the case of the series  $D$ when $m_{-1}< 0$.  Let us check the the equality $L_{-2,-1}^{m_{-2}+m_{-1}+1}v_0=0$ holds.   One has
	
	\begin{align*}
	&L_{-2,-1}a_{-n,...,-2}^{-n,...,-2}=a_{-n,...,-2}^{-n,...,-1},\,\,\, L_{-2,-1}a_{-n,...,-2}^{-n,...,-1}=0,\\
	&L_{-2,-1}a_{-n,...,-2,1}^{-n,...,-2,1}=2a_{-n,...,-2,-1}^{-n,...,-1,1},\,\,\, L_{-2,-1}a_{-n,...,-2,-1}^{-n,...,-1,1}=-a_{-n,...,-2,-1}^{-n,...,-1,2},\,\,\, L_{-2,1}a_{-n,...,-2,-1}^{-n,...,-1,2}=0.
	\end{align*}
	
		Thus $v_0$ is mapped to zero under the action of    $L_{-2,-1}$ to the power
	$1+(m_{-2}-m_{-1})+2m_{-1}=1+m_{-2}+m_{-1}$.
	
	
So  the highest vector satisfies the indicator system.
	
	The fact that the conditions  1-3 do hold for an arbitrary vector follows from the fact that left and right shifts commute and an arbitrary vector can be written as a linear combination of right shifts of the highest vector.
	
	% Проверим выполнение условия 1, для этого необходимо проверить, что если старший вектор имеет вид \eqref{stv}, то все векторы представления удовлетворяют условию  $1$.
	
	%В случае положительного  целочисленного старшего веса \eqref{stv} есть полином от определителей, так что все вектора представления также есть полиномы от определителей.
	
	%В случае  серии  $D$  и  целочисленного старшего веса с  $m_{-1}$ старшего веса \eqref{stv} есть полином от определителей порядка  $1,...,n-1$ и определителя  %$(a_{-n,....,-2,-1})^{-1}=\bar{a}_{-n,...,-2,1}$, причём степень по последнему определителю равна  $-m_{-1}$. Так  что  вектора представления также есть полиномы от определителей, причём
	
	%В случае положительного  полуцелого  старшего веса \eqref{stv}  можно представить как  $f\cdot a_{-n,...,-2,-1}^{1\2}$, где  $f$ - полином от определителей. �?з элементов алгебры Ли  на %$a_{-n,...,-2,-1}^{1\2}$ нетривиально действуют только  элементы $F_{0,-i}$. �?меем
	
	%$$
	%a_{-n,...,-2,-1}^{1\2}\mapsto_{F_{0,-i}}
	%$$
	
	
	
	
Now one has to prove that among functions satisfying the conditions 1-3,  there is nothing else bu functions that form a representation with the highest vector  \eqref{stv0}.
	
	%	Для этого воспользуемся следующим утверждением из \cite{zh}.
	
	%	\begin{prop}\label{claizh}При ограничении функции, определённой на всей группе  $G$, на подгруппу верхнетреугольных унипотентых матриц  $Z$ функции, образующие неприводимое представление со старшим вектором \eqref{stv} переходят биективно в функции на  $Z$, которые  образуют в реализации на пространстве функций на   $Z$ неприводимое представление с тем же старшим весом.
	%		В реализации в пространстве функций на  $Z$  функции, образующие это неприводимое представление,   выделяются условием  .
	%	\end{prop}
	
	
	%	�?так,
Let us be given a function on   $G$, that satisfies the conditions 1-3.  Then it's restriction to the subgroup $Z\subset G$ of upper-unitriangular matrices satisfies the indicator system. According to  \cite{zh} this restriction belongs to a realization of the representation in the space of functions on the group   $Z$ (in \cite{zh} this realization in the space of functions on   $Z $ is given for all series).  Using the results of   \cite{zh}, one gets that the initial function on   $G$ belongs to the representation with the highest weight  \eqref{stv0}.
	
	
\end{proof}


\subsubsection{Solution of the indicator system and the equations $L_{-i,-i}f=m_{-i}f$}
\label{islm}
%Возьмём функцию от определителей

In the proof of the Theorem   \ref{tzh}  the formulas for the action of $L_{-i,-i}$ were obtained. One gets using them the following  Lemma.

\begin{lemma}
	\label{lfmf}
Solution of the system $L_{-i,-i}f=m_{-i}f$ that are functions of determinants are described as follows.  If one represents this function as  a sum of monomials in determinants  then in each monomial the sum of exponents of determinant of size  $n-i+1$ equals
	$r_{-i}$ for $i=n,...,2$.  Also the sum of exponents of determinants of size $n$ equals $|m_{-1}|$.
	% при $m_{-1}\geq 0$ и
	%$-m_{-1}$ при  $m_{-1}<0$.
	
\end{lemma}


Now has to find solutions of the indicator system among these functions.

\begin{lemma}
	\label{l1} If the highest weight is integer non-negative then the solutions of the indicator system are polynomials in determinants that satisfy the conditions of the Lemma   \ref{lfmf}.
\end{lemma}

\begin{proof}
Let highest vector
  \eqref{stv0} be a polynomial in determinants. Since under the action of the algebra  of polynomials in determinant is invariant then an arbitrary vector of the representation is also presented as a polynomial in determinants. Using this argument one immediately derives the statement of the Lemma from   \eqref{lij}.
\end{proof}


\begin{lemma}
	\label{l2} If the highest vector in half-integer then among the functions that satisfy the conditions of the Lemma 
	\ref{lfmf}, the solutions of the indicator system are the functions of type    р
	
	\begin{align}
	\begin{split}
	\label{f12}
	&f=\sum_{\alpha}\sqrt{a_{\pm n,...,\pm 2,\pm 1}}\cdot f_{\alpha}\text{ in the case of the series $B$  and of $D$ when $m_{-1}\geq 0$},\\
	&f=\sum_{\alpha}\sqrt{\bar{a}_{\pm n,...,\pm 2,\pm 1}}\cdot f_{\alpha}\text{
		in the case of the series  $D$ and $m_{-1}< 0$}.
	\end{split}
	\end{align}
	here $\alpha$ denotes a choice of  $+$ or $-$  for all the indices  $\pm n,...,\pm 1 $.   In the case of the series  $B$ no conditions on the choice of the signs are imposed. I the case of the series   $D$  when $m_{-1}\geq 0$  the sign  $+$ is chosen in such  way that the parity of the quantity of  places with $-$ is the same as the parity of $n$,  and in the case  $m_{-1}< 0$ the parity  of the quantity of  places with $-$ must be the same as the parity of   $n-1$.
	
All $f_{\alpha}$  are polynomials in determinants and the sum of exponents of determinants of size  $n-i+1$ equals
	$r_{-i}$ for $i=n,...,2$. In the case of the series  $D$ and $m_{-1}<0$  one takes as determinants of order   $n$ the determinants of type   $\bar{a}_X$.	
	 In all cases the sum of exponents of  determinants order $n$ equals  $|m_{-1}|-\frac{1}{2}$.
\end{lemma}

\begin{proof}
	Consider the case $B$.
	
First of all let us show that functions representing vectors of the representation with the highest weight \eqref{stv0} have the form	\eqref{f12}.
	
%	Сперва покажем, что решение индикаторной системы имеет вид
%	\eqref{f12}.   Функция   от определителей  	\eqref{f12} автоматически удовлетворяет
%	условию 1 из Теоремы \ref{tzh}. Условия 2 и 3 мы считаем
%	выполненными,  так что  эта функция есть вектор представления со
%	старшим вектором \eqref{stv0}.
	
Consider first a representation with the highest vector  $\sqrt{a_{-n,...,-2,-1}}$,  that is the spinor representation 
	
	
	\begin{propos}
		\label{spb} For the algebra $\mathfrak{o}_{2n+1}$
		a representation with the highest vector  $\sqrt{a_{-n,...,-2,-1}}$ is a span of functions of type  
		
		\begin{equation}
		\label{spn}
		\sqrt{a_{\pm n,...,\pm 2,\pm 1}}
		\end{equation}
		
		
	\end{propos}
	
	\begin{proof}
 Let us show that when one applies   $F_{p,q}$ to  $
	\sqrt{a_{\pm n,...,\pm 2,\pm 1}}
	$ one gets a span of functions of the same type. Then the considered span  is a representation  that contains the spinor representation  since  $\sqrt{a_{-n,...,-2,-1}}$ is the highest vector of the spinor representation. Then fact that the considered span and the spinor representations do coincide follows form the fact that  they both have the dimension $2^n$.
	
	
 The considerations are the same for all choices of signs. To avoid cumbersome notations let us consider just the case when the sign  $-$ is chosen everywhere.  Then one obtains a non-zero result for the action onto  $\sqrt{a_{-n,...,-2,-1}}$  only for the operators $F_{0,-i}$, $F_{j,-i}$, $i,j>0$.  In these cases one gets
	
	\begin{align}
	\begin{split}
	\label{ffa}
	&F_{0,-i}:\,\,\sqrt{a_{-n,...,-2,-1}} \mapsto\frac{(-1)^i a_{-n,...,\widehat{-i},...,-1,0}}{2\sqrt{a_{-n,...,-1}}},\\
	&F_{j,-i}:\,\,\sqrt{a_{-n,...,-2,-1}} \mapsto\frac{(-1)^i a_{-n,...,\widehat{-i},...,-1,j}-(-1)^j   a_{-n,...,\widehat{-j},...,-1,i} }{2\sqrt{a_{-n,...,-1}}}.
	\end{split}
	\end{align}
	
	
Consider the first equality in  \eqref{ffa}.
One has the relations \eqref{s22}.  One a square root of this relation one gets 
	
	\begin{equation}
	\label{fl1}
	F_{0,-i}\sqrt{a_{-n,...,-2,-1}}=\frac{\sqrt{-2}}{2} \cdot \sqrt{a_{-n,...,\widehat{-i},...,-1,i}}
	\end{equation}
	
%	Константу несложно вычислить. Например, для $i=1$ она равна  $\frac{\sqrt{-2}}{2}$.
	
 Now consider the second equality in  \eqref{ffa}. Let us show that
	$$
	\frac{(-1)^i a_{-n,...,\widehat{-i},...,-1,j}-(-1)^j   a_{-n,...,\widehat{-j},...,-1,i}}{2\sqrt{a_{-n,...,-1}}}
	$$
is a linear combination of functions of type  \eqref{spn}.  Without loss of generality one can put  $i=1$, $j=2$.  That is we are considering the fraction 
	
	$$
	\frac{  a_{-n,...,-3,-2,2}- a_{-n,...,-3,1,-1}}{2\sqrt{a_{-n,...,-1}}}
	$$
	
	
	
	
To avoid cumbersome notations let us omit the first indices. One has the Plucker relation
	
	$$
	a_{...,-2,2}a_{...,-2,-1,1}+  a_{...,-2,1}a_{...,-2,2,-1}+ a_{...,-2,-1}a_{...,-2,1,2} =0,
	$$
	
thus
	
	\begin{align*}
	&  a_{...,-2,2}=-\frac{   a_{...,-2,1}a_{...,-2,2,-1}+ a_{...,-2,-1}a_{...,-2,1,2}  }{a_{...,-2,-1,1}}=\frac{   a_{...,-2,1}a_{...,0,-1}}{a_{...,-2,0}}+\frac{ a_{...,-2,-1}a_{...,0,1}  }{a_{...,-2,0}}
	\end{align*}
	
	
	
Now let us use \eqref{s22}, one gets
	
	\begin{align*}
	&  a_{...,-2,2}=\frac{   a_{...,-2,1}\sqrt{a_{...,2,-1}a_{...,-2,-1}}}{\sqrt{a_{...,-2,1}a_{...,-2,-1}}}+\frac{ a_{...,-2,-1}\sqrt{a_{...,-2,1}a_{...,2,1}}  }{\sqrt{a_{...,-2,1}a_{...,-2,-1}}}=\\&=\sqrt{a_{...,-2,1}a_{...,2,-1}}+\sqrt{a_{...,-2,-1}a_{...,2,1}}
	\end{align*}
	
	
Analogously one has the Plucker relation 
	
	$$
	a_{...,-1,1}a_{...,-1,-2,2}+a_{...,-1,2}a_{...,-1,1,-2}+a_{...,-1,-2}a_{...,-1,2,1}=0
	$$
	
	
	
	
From here one gets
	
	\begin{align*}
	& a_{...,-1,1}=-\frac{a_{...,-1,2}a_{...,-1,1,-2}+a_{...,-1,-2}a_{...,-1,2,1}}{a_{...,-1,-2,2}}=
	\frac{a_{...,-1,2}a_{...,-2,0}}{a_{...,0,-1}}-\frac{a_{...,-1,-2}a_{...,0,2}}{a_{...,-1,0}}
	\end{align*}
	
	%Отметим, что мы в первом выражении берём тот же
Use the relation  \eqref{s22},  one gets	
	
	\begin{align*}
	& a_{...,-1,1}=
	\frac{a_{...,-1,2}\sqrt{a_{...,-2,-1}a_{...,-2,1}}}{\sqrt{a_{...,-2,-1}  a_{...,2,-1}} }+\frac{a_{...,-1,-2}\sqrt{a_{...,-1,2}  a_{...,1,2} } }{\sqrt{ a_{...,-1,2} a_{...,-1,-2}    }}=\\&=
	-\sqrt{a_{...,2,-1}a_{...-2,1}}+\sqrt{a_{...,-1,-2}a_{...,1,2}}
	\end{align*}
	
Thus one has
	
	$$
	a_{...,-2,2}+a_{...,-1,1}=2\sqrt{a_{...,-1,-2}a_{...,1,2}}
	$$
	
The Jacobi relations imply that $a_{...,-1,1}=a_{....,-2,2}$, thus one has 
	
	\begin{equation}
	\label{sd}
	a_{...,-1,1}=\sqrt{a_{...,-1,-2}a_{...,1,2}}
	\end{equation}
	
	 %подставляя выражения для  $a_{...,-2,2}$ и  $a_{...,-1,1}$ в  
	 Let us substitute our results into  \eqref{ffa}, for $i=1,j=2$. One gets 
	
	\begin{equation}
	\label{chtn}
	F_{2,-1}\sqrt{a_{-n,...,-1}}= \sqrt{a_{-n,...,1,2}},
	\end{equation}
	
	\end{proof}
	
	
Now consider the case of an arbitrary highest weight. The  highest vector
can be represented as
	
	$$
	v_0=v'_0(a_{-n,...,-2,-1})^{\frac{1}{2}},
	$$
 where  $v'_0$  is a polynomial in determinants. An arbitrary vector $f$
can be written as a linear combination of vectors that are obtained by the action onto the highest vector of the operators $\prod_{i<j} F_{i,j}^{p_{i,j}}$.  As a result one obtains  a vector of type 
	\eqref{f12}.
	
	
Thus the Zhelobenko's model is contained in the space of functions of type  \eqref{f12}.   Now one has to prove that all function of this type belong to the  Zhelobenko's model. For this it is enough to check that each function of the form \eqref{f12},
that satisfies the conditions of Lemma \ref{lfmf}, is a solution of the indicator system. 
	
The operators $L_{-i,-i+1}$, $i=n,...,2$  are acting onto the determinant of size  $n-i+1$ only. Such determinants occur in   $f_{\alpha}$ only, that there exponents are non-negative integers. The sum of the exponents of determinants of size  $i$ equals  $r_{-n+i-1}$. Thus the conditions of Lemma \ref{lfmf} are satisfied.  Thus the conditions 
	$L_{-i,-i+1}^{r_{-i}+1}f=0 $ for $i=n,...,2$ hold.
	
Now consider the equation
	$L_{-1,0}^{2m_{-1}+1}f=L_{-1,0}^{2[m_{-1}]+1+1}f=0$,  where
	$[m_{-1}]$ is the integer part.  The operator $L_{-1,0}^{2[m_{-1}]+1+1}$ acts by the Leibniz ruler onto each summand in 
	\eqref{f12} as follows.
	
	Either  $L_{-1,0}^{2[m_{-1}]+1+1}$ acts onto the second factor  $f_{\alpha}$. In this case one obtains    $0$, since in the polynomials  $f_{\alpha}$ the sum of exponents of determinants of size $n$ equals  $[m_{-1}]$,  but such polynomials are annihilated by$L_{-1,0}^{2[m_{-1}]+1}$.
	
	Either   $L_{-1,0}^{2[m_{-1}]+1}$ acts onto the second factor   $f_{\alpha}$,  and 
	$L_{-1,0}$ acts onto the first factor.  In this case one obtains  $0$  by the same reason.
	
	Either $L_{-1,0}^{2[m_{-1}]+2-k}$ acts onto the second factor   and  $L_{-1,0}^{k}$  acts onto the first factor, here $k\geq 2$.  Since the first factor is a vector of the representation with the highest weight  $[\frac{1}{2},...,\frac{1}{2}]$, then under the action of  $L_{-1,0}^{2}$
 it vanishes.
	
 Thus the vectors of type \eqref{f12}  are vanishing under the action of 
	$L_{-1,0}^{2m_{-1}+1}$.  In the case of the series $B$ the Lemma is proved.
	
Now consider the case of the series  $D$.   The scheme of the proof is the same as in the case of the series    $B$.  First of all one considers the spinor representations with the highest weights  $[\frac{1}{2},...,\frac{1}{2},\frac{1}{2}]$,   $[\frac{1}{2},...,\frac{1}{2},-\frac{1}{2}]$.
	
	
	\begin{propos}
		For the algebra  $\mathfrak{o}_{2n}$ the representations with the highest vectors    $\sqrt{a_{-n,...,-2,-1}}$, $\sqrt{\bar{a}_{-n,...,-2,1}}$  coincide with the spans of functions 
		
		\begin{align*}
		&  < \sqrt{a_{\pm n,...,\pm 2,\pm1}}> ,\text{ the parity of the number of $-$ equals to the parity of $n$},\\
		&  < \sqrt{\bar{a}_{\pm n,...,\pm 2,\pm1}}>, \text{ the parity of the number of $-$ equals to the parity of $n-1$}
		\end{align*}
		
		
	\end{propos}
	
	\begin{proof}
	
The scheme of the proof is the same as for the Propostion \ref{spb}.
	One has an embedding of the Lie algebras $\mathfrak{o}_{2n}\subset\mathfrak{o}_{2n+1}$, induced by an embedding of root systems.  Thus the determinants  $a_{\pm n,...,\pm 1}$ and also  $\bar{a}_{\pm n,...,\pm 1}$ can be considered as functions on the group $SO_{2n+1}$. Thus we can use the formula  \eqref{chtn}.
	
Using it one obtains that under the action of $\mathfrak{o}_{2n}$  onto a function $\sqrt{a_{\pm n,...,\pm 2,\pm1}}, \sqrt{\bar{a}_{\pm n,...,\pm 2,\pm1}} $ one obtains a linear combination of functions of the same type. At the same time, the parity of the number of minuses in the indexes is preserved. Thus, both linear spans under consideration are representations. The dimensions of each of these linear spans are equal  $2^{n-1}$.
	
	
	
The irreducible representations with the highest vectors $\sqrt{a_{-n,...,-2,-1}}$, $\sqrt{\bar{a}_{-n,...,-2,1}}$ also have dimension $2^{n-1}$.
	
From these two facts one obtains the statement of the Proposition.
	
	
	\end{proof}
	
Further considerations in the case of the series  $D$  are analogous to considerations in the case of the series   $B$. The only change is a replacement of $L_{-1,0}$ onto  $L_{-2,1}$.
	
\end{proof}


Thus we have proved the following Theorem.

\begin{theorem}
	\label{ost1}
	The space of an irreducible representation with the highest vector \eqref{stv0} in the Zhelobenko's model is described as follows.
In the cases of the series  $A$, $B$, $C$   and the series $D$ with $m_{-1}\geq 0$    consider the space of functions of determinants   $a_{X}$, $|X|\leq n$.  And  in the case of the series $D$ with  $|m_{-1}|<0$  consider the space of functions of determinants  $a_{X}$, $|X|< n$  and $\bar{a}_X$, $|X|=n$.
	
The functions must meet the following requirement. In the case of an integer of highest weight, they are polynomials in the determinants satisfying the conditions of the Lemma \ref{lfmf}.
In the case of an integer of highest weight they are function of type \eqref{f12},  satisfying the conditions of the Lemma   \ref{l2}.
\end{theorem}


\section{ The GKZ system for the series  $A$. The A-GKZ system}
\label{seriaa}
In this s section  we introduce two important systems of partial differential equations. Solutions of one of these systems (GKZ systems) are used to build a basis in the Zhelobenko model, and solutions of the other (A-GKZ system) are used to build a new representation model.


The results of this section concern only the $A$ series, essentially all of them were obtained in \cite{a4}, but we modify them insignificantly to adapt them for consideration of other series. It is shown how proofs of modified statements can be obtained from proofs of similar statements in \cite{a4}.




\subsection{The Gelfand-Tsetlin lattice. The vectors $v_{\alpha}$. The number $\mathcal{K}$}
\label{rgc}


Consider the Lie algebra $\mathfrak{gl}_m$, which we identify with the algebra of matrices whose rows and columns are indexed by the numbers $-n,...,\hat{0},...,n$ (in the case $m=2n$) or $-n,...,0,....n$ (in the case $m=2n+1$). Le us associate with it a shifted lattice in the space $\mathbb{Z}^{N}$, whose coordinates are numbered by proper subsets $X\subset \{-n,...,n\}$. Here $N$ is the number of possible proper subsets of $X\subset\{-n,...,n\}$. Such a strange at first glance indexing is taken in order to obtain a lattice that will be used in further reasoning for the series $B$, $C$, $D$.






The definition of the Gelfand-Tsetlin lattice for the $A$ series, which is given below, differs somewhat from that given in \cite{a4}. The essence of the difference is as follows. The idea of a homogeneous system defining the Gelfand-Tsetlin lattice is that the inhomogeneous versions of this system represent naive conditions on the vector of monomial exponents from determinants, which can appear in the decomposition of a function corresponding to the vector of the Gelfand-Tsetlin basis.
Accordingly, this system should depend on the chain of subalgebras underlying the construction of the Gelfand-Tsetlin basis. In \cite{a4}, a chain is used in which the subalgebra $\mathfrak{gl}_{m-k}$ is formed by matrices having nonzero elements of the first $m-k$ rows and columns relative to the standard order on the set $\{-n,...,n\}$. In the present paper, we use a chain in which the subalgebra $\mathfrak{gl}_{m-k}$ is formed by matrices having nonzero elements in $m-k$ rows and columns with the smallest indices relative to the ordering 





\begin{equation}\label{porya}1\succ -1\succ 2\succ -2\succ...\succ 0 ,\end{equation} 

This is done in order to use the results for the $A$ series in further consideration of other series.



We also introduce the value





	
\begin{equation}
\label{snk}
s(p,q):=\#\{t:t\leq p\,\, \& \,\,t\succ q\}     %\begin{cases}
%0, p<0,\\ \#\{t:t<0\,\, \& \,\,t\succ q\},\\
%1,p>0
%\end{cases}, \text{если  $0\notin \{-n,...,n\}$},  \,\,\,\,  s(q)=0 \text{ иначе}
\end{equation}

%Теперь определим важную решётку в пространстве  $\mathbb{Z}^N$.


\begin{definition}
Define the Gelfand-Tselin lattice $\mathcal{B}_{GC}^{\mathfrak{gl}_m}\subset \mathbb{Z}^N$ for the series   $A$ by the following homogeneous system of equations 




	\begin{align}
	\begin{split}
	\label{sar}
	&\delta\in \mathcal{B}_{GC}^{\mathfrak{gl}_m} \Leftrightarrow  \forall p,q \in \{-n,...,n\},p\preceq q  \text{ one has  }:\\&\sum_{X\text{ contains }\geq (p+n+1-s(p,q))\text{ elements }\preceq  q} \delta_{X}=0.
	\end{split}
	\end{align}
	

The resulting lattice is called the Gelfand-Tsetlin lattice for the series $A$.
\end{definition}

Let us construct the generators of this lattice. Take subsets  $Y_{i},i=-n,...,n$  of the following type

\begin{align}
\begin{split}
\label{yi}
&Y_{-n}=\{-n\},\,\,\, Y_{n}=\{-n,n\},\\
&...\\
& Y_{-i}=\{-n,...,-i\},\,\,\,\, Y_{i}=\{-n,...,-i,i\},\,\,\, i>0
\end{split}
\end{align}

Note that  $Y_{-i}=Y_{-i-1}\cap\{-i\}$, $Y_{i}=Y_{-i}\cap\{i\}$.  Also note that 

$$
\{j: j\succ Y_{-i}\}=\{j:j\succeq i\},\,\,\, \{j: j\succ Y_{i}\}=\{j:j\succeq -i+1\},
$$

where the index is greater than the subset if  it is greater than each element of the subset.


Define the vectors (in the case  $i\geq 0$)

\begin{align}
\begin{split}
\label{va0}
&v_{\alpha}=e_{Y_{-i-1},-i,X}-e_{Y_{-i-1},j,X}-e_{Y_{-i-1},-i,y,X}
+e_{Y_{-i-1},j,y,X}\text{ for }-i\prec j\prec y \prec X\\
&v_{\alpha}=e_{Y_{-i},i,X}-e_{Y_{-i},j,X}-e_{Y_{-i},i,y,X}
+e_{Y_{-i},j,y,X}\text{ for }i\prec j\prec y \prec X
\end{split}
\end{align}

Here  $e_{Z}$  is a unit base vector corresponding to a coordinate  $Z$.  Let us write  (for  $i\in \{-n,...,n\}$)
 
 \begin{equation}
 \label{va}
 v_{\alpha}=e_{i,Z}-e_{j,Z}-e_{i,y,Z}
 +e_{j,y,Z},\end{equation}

where  $Z=Y_{i-1}\cup X$
for $i\leq 0$ and  $Z=Y_{-i}\cup X$
for $i >0$.
\begin{lemma}\label{lm}
The vectors  $v_{\alpha}$ are generting the lattice  $\mathcal{B}_{GC}^{\mathfrak{gl}_m}$.
\end{lemma}

\begin{proof}

Let's first chose some of the vectors \eqref{va} (more precisely, it's better to look at the formula \eqref{va0}) and let us prove that they form a basis in the lattice. This will be enough to prove the lemma. To select vectors, we fix $\pm i$, $j$, $y$ and consider vectors \eqref{va0} for all possible $X$. Let's construct a graph in which these subsets  $X$ will be the vertices, and edges will be pairs of subsets of the form $X_1=X, X_2=\{x\}\cup X$. Let's choose such a collection of subsets that the corresponding subgraph is a tree, and it is  the maximum  collection with respect to the extension while preserving this property.



For fixed $\pm i$, $j$, $y$ and chosen $X$, we construct vectors \eqref{va}. Let us show that they form a basis in the solution space of the \eqref{sar} system. The proof of this fact repeats the basic steps of the proof of Proposition 1 from \cite{a4}.



The step {\bf 1} from this proof (linear independence of the selected set of vectors \eqref{va}) literally repeats the reasoning from step {\bf 1} in \cite{a4}.
	
The step {\bf 2} consists in verifying that the vectors \eqref{va} are solutions of the system \eqref{sar}. This is done by direct substitution.


The step {\bf 3} consists in calculating the dimension of the solution space and in comparing it with the number of selected vectors \eqref{va}. This step is somewhat different from the corresponding reasoning in \cite{a4}. We first notice that the number of equations in \eqref{sar} is equal to the number of elements of the Gelfand-Tsetlin diagram. In particular, it coincides with the number of equations for a similar system from \cite{a4} for $\mathfrak{gl}_{m}$. Next, we notice that the number of vectors of the form \eqref{va} selected at the step {\bf 1} is equal to the number of vectors selected at step {\bf 1} in \cite{a4}. Indeed, in \cite{a4}, the selection is carried out like this. The complete set of indices $\{1,...,m\}$ is considered. Next, for each fixed index $i$, we consider the choice of $j<y<X$ belonging to $\{i+1,...,m\}$ and satisfying the conditions of the step ${\bf 1}$. The cardinality of the set $\{i+1,...,m\}$ from which $j<y<X$ is selected takes the values $m-1,...,2$.

	
In a situation of the present paper,  for 
 of a fixed $i\in \{-n,...,n\}$ the selection of $j\prec y\prec X$ is made from a subset of $\{j:j\succeq i\}$. In this case, $j\prec y\prec X$ must satisfy the conditions of the step ${\bf 1}$. 
 When iterating over $i=1,-1,...$ the cardinality of the set $\{j:j\succeq i\}$, from which the choice is $j\prec y\prec X$ is done takes the values $m-1,...,2$. Since the selection rules for $X$ at  the step {\bf 1} are the same in the present  paper and in \cite{a4}, the numbers selected by the vector in these two constructions are the same.
	
In \cite{a4} it is verified that the number of selected vectors is equal to the dimension of the solution space. The dimensions of the solution space both in the present paper and in  \cite{a4} are equal to the number of variables minus the number of equations, which in turn is equal to the number of elements of the Gelfand-Tsetlin diagram. Hence, it turns out that the dimension of the solution space and the number of selected vectors coincide in the situation of this work. Step {\bf 3} is completed.
	

	
	
\end{proof}



We have presented the  vectors $v_{\alpha}$ generating the lattice $\mathcal{B}_{GC}^{\mathfrak{gl}_m}$. Let's select a basis in this set and denote by $\mathcal{K}$ the number of vectors in this basis.




\subsection{The GKZ system for the series  $A$}

Let's write  the GKZ system associated with the lattice $\mathcal{B}_{GC}^{\mathfrak{gl}_m}$. In the Section \ref{r2}, a GKZ system associated with an arbitrary lattice was defined. In this section we will write only equations of the second type, which are determined by the lattice $\mathcal{B}_{GC}^{\mathfrak{gl}_m}$. This will be the GKZ system for the $A$ series.

Namely, let $A_X$ be variables that are skew-symmetric over the set of $X$, but they do not obey any relations.
Then we  call the GKZ system for the $A$ series a system consisting of differential equations constructed by vectors \eqref{va}

\begin{equation}
\label{gkza1}
(\frac{\partial^2}{\partial A_{i,Z}\partial A_{j,y,Z} }-\frac{\partial^2}{\partial A_{j,Z}\partial A_{i,y,Z} })\mathcal{F}
\end{equation}

\begin{lemma}[\cite{GG}]
	\label{lgkz}
In the space of polynomial solution of the system   \eqref{gkza1} there exists a base $\mathcal{F}_{\gamma}(A;\mathcal{B}_{GC}^{\mathfrak{gl}_m})$, consisting of   $\Gamma$-series.  We take that shift vectors $\gamma$ that represent different elements of the factor-space  $\mathbb{C}^N /\mathcal{B}_{GC}^{\mathfrak{gl}_m}$ and such that one of the representatives of this class has only non-negative coordinates  (see Section   \ref{dgca}).
\end{lemma}

In the formulation, the word <<polynomial>> can be replaced by <<decomposing into power series>>, but it should be understood that the system has other solutions decomposing into power-logarithmic series.

\subsection{The A-GKZ system for the series  $A$}
\label{agkzs}

Let us call the following system the A-GKZ system for the series   $A$:

\begin{equation}
%\begin{align}
%\begin{split}
\label{gkza}
(\frac{\partial^2}{\partial A_{i,Z }\partial A_{j,y,Z} }-\frac{\partial^2}{\partial A_{j,Z}\partial A_{i,y,Z} }+\frac{\partial^2}{\partial A_{y,Z}\partial A_{i,j,Z} })F
%\end{split}
%\end{align}
\end{equation}
\begin{remark}

An important observation is that with an equation of the A-GKZ system (i.e. with the vector \eqref{va}))  a Plucker relation for determinants is associated 
	\begin{equation}
	\label{spgl}
	a_{i,Z} a_{y,Z}	- a_{j,Z} a_{i,y,Z}	+a_{y,Z} a_{i,j,Z}	=0
	\end{equation}
\end{remark}


With the vector   $v_{\alpha}$ of type \eqref{va}  one relates a vector 


\begin{equation}
\label{ra}
r_{\alpha}=e_{y,Z}-e_{j,Z}-e_{i,y,Z}+e_{i,j,Z}
\end{equation}

Now Let us give formulas for some solutions of the A-GKZ system. Let $t,s\in\mathbb{Z}^\mathcal{K}_{\geq 0}$, where $\mathcal{K}$ be the number of independent vectors $v_{\alpha}$. Introduce the functions




\begin{equation}
\label{fs}
\mathcal{F}_{\gamma}^s(A;\mathcal{B}_{GC}^{\mathfrak{gl}_m})=\sum_{t\in\mathbb{Z}^{\mathcal{K}}}\frac{(t+1)...(t+s)A^{\gamma+tv}}{s!(\gamma+tv)!},
\end{equation}
in the formula a multi-index notation is udes  $tv:=t_1v_1+...+t_{\mathcal{K}}v_{\mathcal{K}}$, $sr:=s_1r_1+...+s_{\mathcal{K}}r_{\mathcal{K}}$  and also a multi-index notation for factorials. Put


\begin{equation}
\label{ffpr}
F_{\gamma}(A;\mathcal{B}_{GC}^{\mathfrak{gl}_m})=\sum_{s\in \mathbb{Z}^{\mathcal{K}}}(-1)^s\mathcal{F}_{\gamma-sr}^s(A;\mathcal{B}_{GC}^{\mathfrak{gl}_m}).
\end{equation}

Note that this series does depend on the choice of the vector  $\gamma$, not only on the class $\gamma\,\,mod \mathcal{B}_{GC}^{\mathfrak{gl}_m}$.
As in   \cite{a4}  (see also  \cite{a5}) one proves

\begin{propos}
	\label{sagkz}
The functions  $F_{\gamma}(A;\mathcal{B}_{GC}^{\mathfrak{gl}_m})$, for vectors  $\gamma$ chosen in the Lemma \ref{lgkz}, form a base in the space of polynomial solutions of the A-GKZ system.
\end{propos}




\begin{remark}\label{rem1}


In \cite{a4} the following statement is proved. It's  proof  is again without any change is valid in the considered in the present paper situation.  Let $a_{i_1,...,i_k}$ be a function of the form \eqref{dete} for the group $GL_m$. Let $I_{\mathfrak{gl}_m}$ be the ideal of relations between these minors, considered as an ideal in the polynomial ring $\mathbb{C}[A]$.

	
To any relations their corresponds a differential operator by the ruler   \begin{equation}\label{aa}A_X\mapsto \frac{\partial}{\partial A_X}.\end{equation} Thus one gets an ideal   $\bar{I}_{\mathfrak{gl}_m}$
in the ring of differential operators with constant coefficients.	
	%	Множество его решений (функций, обращающихся в ноль под действием этого идеала) очевидно  $\mathfrak{gl}_m$-инвариантно.
t can be noted that the equations of the A-GKZ system are obtained by the \eqref{aa} rule from {\it of some} relations of the Plucker type.
	
In  \cite{a4} it is verified that, nevertheless, the solution space of the ideal $\bar{I}_{\mathfrak{gl}_m}$ coincides with the solution space of the A-GKZ system.
	
	%	Тот факт, что рассматриваемое представление есть прямая сумма конечномерных неприводимых представлений, взятых с кратностью  $1$, следует из Предложения \ref{sagkz}.
	
\end{remark}

\section{ Models of representations for  $\mathfrak{gl}_m$}

The results of this section also concern only the $A$ series, they were obtained in \cite{a4}.


As before, let's take the Lie algebra $\mathfrak{gl}_m$, where $m=2n$ or $2n+1$, acting in a space with coordinates $-n,...,n$, where in the first case $0$ is thrown out, and in the second - not. Next, $\{-n,...,n\}$ denotes such an index set.


\subsection{Gelfand-Tsetlin diagrams and shift vectors}
\label{dgca}


With a Gelfand-Tsetlin diagram  $(m_{p,q})$
%,
 %$p,q=-n,...,n$,  $p \leq q$ 
for the algebra  $\mathfrak{gl}_m$ one associates a shifted lattice $\Pi=\gamma+\mathcal{B}_{GC}^{\mathfrak{gl}_m}$, defined by an inhomogeneous system of equations 


\begin{align}
\begin{split}
\label{s1}
&\delta\in \Pi \Leftrightarrow  \forall p,q\in\{-n,...,n\}, p\preceq q \text{ one has }:\\&\sum_{X\text{  contains }\geq (p+n+1-s(p,q))\text{ elements } \preceq q} \delta_{X}=m_{p,q},
\end{split}
\end{align}

where  $s(p,q)$  was defined in   \eqref{snk}.

%где мы используем порядок.

\begin{lemma} There exists a one-to-one correspondence between the Gelfand-Tsetlin diagrams   $(m_{p,q})$  and shifted lattices $\Pi=\gamma+\mathcal{B}_{GC}^{\mathfrak{gl}_m}$ (using the system  \eqref{s1}) that contain at least one vector with only non-negative coordinates.
\end{lemma}	

The elements of the Gelfand-Tsetlin diagram, understood in the usual sense, are restored using the equalities \eqref{s1}.
In this approach, diagrams related to the representation of the highest weight $[m_{-n},...,m_{n}]$ are defined as a class $mod\mathcal{B}_{GC}^{\mathfrak{gl}_m}$ of the vectors $\gamma$, such that  $\sum_{X: |X|=p}\gamma_X=m_{p+n-1-s(p)}$, $p\in\{-n,...,n\}$.

Thus, {\it is possible to define } the Gelfand-Tsetlin diagram as a shifted lattice $\Pi=\gamma+\mathcal{B}_{GC}^{\mathfrak{gl}_m}$ of integer vectors, such that in this shifted lattice there is a representative having only non-negative coordinates.


Then the Gelfand-Tsetlin diagram can be called $\mathfrak{gl}_{m-q}$-maximal if in its representative $\gamma$ (and, therefore, in all its vectors) the coordinates corresponding to subsets of $X$ turn to be zero, in the case when the monomial $A_X$ is not $\mathfrak{gl}_{m-q}$ is the highest vector under the action of  $\mathfrak{gl}_{m}$ on these variables defined in \eqref{dax}.





\subsection{The GKZ base in the Zhelobenko's realization}

In \cite{a4} the following was proved.


\begin{theorem}
	\label{osa0}
Consider $\Gamma$-series $\mathcal{F}_{\gamma}(A;\mathcal{B}_{GC}^{\mathfrak{gl}_m})$, such that $\Pi=\gamma+\mathcal{B}_{GC}^{\mathfrak{gl}_m}$ are Gelfand-Tsetlin diagrams in the sense of the previous section. Then $\mathcal{F}_{\gamma}(A;\mathcal{B}_{GC}^{\mathfrak{gl}_m})$ is the basis of the Zhelobenko's model $Zh$, and this basis is consistent in the decomposition of $Zh$ into the sum of irreducible representations.	
	
	

\end{theorem}




\subsection{An A-GKZ model of representations  $\mathfrak{gl}_m$}

Consider the action of the Lie algebra $\mathfrak{gl}_m$ on the independent variables $A_{X}$, defined by the formula

\begin{equation}
\label{dax}
E_{i,j}=\sum_{X}A_{X,i}\frac{\partial}{\partial A_{X,j}}
\end{equation}


One has.

\begin{theorem}[\cite{a4}]
	\label{osa}
The space of polynomial solutions of the A-GKZ system is invariant under the action of $\mathfrak{gl}_m$. This space is a direct sum of finite-dimensional irreducible representations each taken with multiplicity $1$.
\end{theorem}

Thus, the solution space A-GKZ in the case of $\mathfrak{gl}_m$ is a model of finite-dimensional irreducible representations. It is not difficult to remark that this model is naturally identified with a subspace in the tensor product of standard representations.





\subsection{A relation with the Gelfand-Tselin base }

The following statement is proved in \cite{a4}. Introduce on the set of vectors considered by $mod\mathcal{B}_{GC}^{\mathfrak{gl}_m}$, the ordering

\begin{equation}
\label{poryadok}
\gamma \preceq \delta \Leftrightarrow \gamma=\delta-s r\,\,\,\,\, mod  \mathcal{B}_{GC}^{\mathfrak{gl}_m},\,\,\,\, s\in \mathbb{Z}^{\mathcal{K}}_{\geq 0},
\end{equation}

where ${\mathcal{K}}$  is  the number of independent vectors  $v_{\alpha}$  of type  \eqref{va}, the vectors  $r_{\alpha}$  were defined in   \eqref{ra}. We use a notation  $sr:=s_1r_1+...+s_{\mathcal{K}}r_{\mathcal{K}}$.


According to the section \ref{dgca}, the Gelfand-Tsetlin diagrams are in bijective correspondence with the shift vectors considered by $mod\mathcal{B}_{GC}^{\mathfrak{gl}_m}$, such that there is a vector with non-negative coordinates in this class. Then you can introduce a notation $G_{\delta}$ for the Gelfand-Tsetlin base vector corresponding to the diagram $\delta$.




\begin{theorem}
	\begin{enumerate}
		%	\item Базисы  $\mathfrak{F}_{\delta}$ и  $G_{\delta}$ связаны верхнетреугольным преобразованием относительно порядка  $\prec$.
		\item The basis  $F_{\delta}$ and  $G_{\delta}$  are related by a lower-triangular transformation relatively the ordering   $\prec$.
		\item   $G_{\delta}$ is a orthogonalization of  $F_{\delta}$.
	\end{enumerate}
\end{theorem}

In  \cite{a4} a transformation that relates $F_{\delta}$ and  $G_{\delta}$ was found explicitly.


\section{The GKZ and the A-GKZ systems for the series  $B$, $C$, $D$}

\label{systemy}
To obtain systems for which the analogs of the Theorems \ref{osa0}, \ref{osa} hold, one needs to add new equations to the GKZ and A-GKZ systems for the $A$ series. These equations are related to new (non-Plucker) relations arising for the series $B$, $C$, $D$.






\subsection{ An auxiliary lattice  $\mathcal{B}^{g_n}$. Variables  $B_X$}



Introduce a lattice

\begin{equation}\bar{\mathcal{B}}^{g_n}=\mathcal{B}_{GC}^{\mathfrak{gl}_n}\subset \bar{L}= \mathbb{Z}^N,\,\,\,\,\, h_{\kappa}=e_X-e_{\widehat{-X}} ,\end{equation}

The lattice $\bar{L}=\mathbb{Z}^N$ is interpreted as the set of exponents of Laurent monomials in variables $A_X$ introduced earlier.


In the case of the series $C$ we define  an auxiliary lattice of $\mathcal{B}^{g_n}$ just as the lattice    $\bar{\mathcal{B}}^{g_n}$ defined above. For the series $B$, $D$, further constructions are needed.


Namely, consider the variables $B_X$ associated with the previously introduced variables $A_X$ according to the following rule. If $X$ or $\widehat{-X}$ coincides with $\{\pm n,...,\pm 1\}$, then $B_{X}=\sqrt{A_{X}}$, in other cases $B_X=A_X$. When one passes to the Zhelobenko model in the first case $\sqrt{a_{X}}$ is substituted instead of $B _X$ (or $\sqrt{\bar{a}_{X}}$ for the $D$ series, when $|X|=n$, $m_{-1}<0$), and in the second case one substitutes just  $a_X$ (or $\bar{a}_{X}$ for the series $D$, for $|X|=n$, $m_{-1}<0$).
Note that a construction of this type (a transition to variables, some of which coincide with the determinants, and some are the roots of the determinants in order for the spinor representation to be realized in the space of {\it polynomials} in
these variables) for the case of $\mathfrak{o}_5$ is implemented in all details in \cite{a6}.



Take the lattice $L=\mathbb{Z}^N$ of exponents of monomials in variables $B_X$.
One can naturally embed the lattice  $\bar{L}$ in $L$.
This gives the embedding $\bar{\mathcal{B}}^{g_n}\subset\bar{L}\subset L=\mathbb{Z}^N$.
Define $\mathcal{B}^{g_n}$ as a sublattice in $L$, which is the image of $\bar{\mathcal{B}}^{g_n}$ for this embedding.





%\begin{definition}
Introduce the degrees of variables  $B_X$:
\begin{equation}
\label{podst}
degB_X=1, \,\,\, X\neq \begin{cases}\{\pm n,...,\pm 1\},\\   \{\pm n,...,\pm 1,0\}  \end{cases},	degB_X=\frac{1}{2}, \,\,\,  \text{ otherwise }.
\end{equation}

\subsection{The GKZ system for  $g_n$ in the case of the series  $B$, $D$}

Using the lattice $\mathcal{B}^{g_n}$, it is possible to build a system of GKZ  and then the A-GKZ system. But then one needs to add some new equations so that for the series $B$, $D$ the analogue of the statement formulated in the Remark \ref{rem1} holds. These new equations correspond to the Jacobi relations \eqref{sb}, as well as the root of the relations \eqref{s22}. Despite the addition of these equations, we  call the resulting systems {\it the GKZ and A-GKZ systems } for the series $B$, $D$.




Consider functions in {\it independent} variables $B_X$, indexed by proper subsets in $\{-n,...,n\}$. In the case of the $B$ series, the index $0$  is included.



\begin{definition}

  {\it The GKZ system for the series  $B$, $D$} is a system consisting of equations constructed form the vectors  \eqref{va}:

\begin{equation}\label{ggkz}
\begin{cases}
& \mathcal{O}_{\alpha}\mathcal{F}:=(\frac{\partial^{\tau_1+\tau_2}}{\partial^{\tau_1} B_{i,Z}\partial^{\tau_2} B_{j,y,Z}}-\frac{\partial^{\tau_3+t_4}}{\partial^{\tau_3}B_{j,Z}\partial^{\tau_4} B_{i,y,Z}}\mathcal{F}=0,\\
%& \text{ индексные множества содержат  $\leq n$ элементов},\\
&  \tau_{i}=2,\text{ if the corresponding variable is of type $B_{\pm n,...,\pm 1}$ or  $B_{\pm n,...,\pm 1,0}$},\\& \text{ and $\tau_{i}=1$ otherwise},\\
&(\frac{\partial}{\partial B_X}-\pm\frac{\partial}{\partial B_{\widehat{-X}}})\mathcal{F}=0,\text{  the sign is defined in   \eqref{znk}},\\&
(\frac{\partial }{\partial B_{\pm n,...,\hat{\pm i},...,\hat{\pm j},...,\pm 1,-i,i}}-\frac{\partial^2 }{\partial   B_{\pm n,...,\hat{\pm i},...,\hat{\pm j},...,\pm 1,-i,-j} 
\partial  B_{\pm n,...,\hat{\pm i},...,\hat{\pm j},...,\pm 1,i,j} })F=0,\\
& (\frac{\partial^2}{\partial B_{\pm n,...,\hat{i},...,\pm 1 ,-i}\partial B_{\pm n,...,\hat{i},...,\pm 1 ,i}}-\frac{\sqrt{-1}}{\sqrt{2}}\frac{\partial}{\partial  B_{\pm n,...,\hat{i},...,\pm 1 ,0}})\mathcal{F}=0\text{ only for the series }B
\end{cases}
\end{equation}
\end{definition}

The equation of the second type is called  {\it the Jacobi equation}.





\subsection{The A-GKZ system for the series  $B$, $D$}

\label{ser1}




	

\begin{definition}   {\it The A-GKZ system for the series  $B$, $D$} looks as follows
\begin{equation}
\label{agkzgn}
\begin{cases}
&\bar{ \mathcal{O}}_{\alpha}\mathcal{F}:=(\frac{\partial^{\tau_1+\tau_2}}{\partial^{\tau_1} B_{i,Z}\partial^{\tau_2} B_{j,y,Z}}-\frac{\partial^{\tau_3+t_4}}{\partial^{\tau_3}B_{j,Z}\partial^{\tau_4} B_{i,y,Z}}+ \frac{\partial^{\tau_5+\tau_6}}{\partial^{\tau_5}B_{y,Z}\partial^{\tau_6} B_{i,j,Z}})F=0,\\
%& \text{ индексные множества содержат  $\leq n$ элементов},\\
&  \tau_i=2,\text{ if the corresponding variable is of type $B_{\pm n,...,\pm 1}$ или  $B_{\pm n,...,\pm 1,0}$},\\& \text{ and $\tau_i=1$ otherwise},\\
&(\frac{\partial}{\partial B_X}-\pm\frac{\partial}{\partial B_{\widehat{-X}}})F=0,\text{ the sign is defined in  \eqref{znk}},\\
\\&
(\frac{\partial }{\partial B_{\pm n,...,\hat{\pm i},...,\hat{\pm j},...,\pm 1,-i,i}}-\frac{\partial^2 }{\partial   B_{\pm n,...,\hat{\pm i},...,\hat{\pm j},...,\pm 1,-i,-j} 
	\partial  B_{\pm n,...,\hat{\pm i},...,\hat{\pm j},...,\pm 1,i,j} })F=0,\\
& (\frac{\partial^2}{\partial B_{\pm n,...,\hat{i},...,\pm 1 ,-i}\partial B_{\pm n,...,\hat{i},...,\pm 1 ,i}}-\frac{\sqrt{-1}}{\sqrt{2}}\frac{\partial^2}{\partial^2  B_{\pm n,...,\hat{i},...,\pm 1 ,0}})F=0\text{ only for the series }B.
\end{cases}
\end{equation}
\end{definition}


Let us construct basis in the spaces of solutions of the GKZ and A-GKZ systems.

For an arbitrary vector $\delta\in\mathbb{Z}^N$ one introduces functions analogous to the functions \eqref{fs}.




\begin{equation}
\label{fss}
\mathfrak{f}_{\delta}^s(B)=\sum_{t\in\mathbb{Z}^{\mathcal{K}}}\frac{(t+1)...(t+s)B^{\delta+tv}}{s!(\delta+tv)!},
\end{equation}

we use a chosen basis  $v_{\alpha}$ of the auxiliary lattice   $\mathcal{B}^{g_n}$.
Put

$$
\mathfrak{f}_{\delta}(B):= \mathfrak{f}_{\delta}^0(B)
$$

Then

\begin{equation}
\label{reshenije}
f_{\delta}(B):=\sum_{s\in \mathbb{Z}^{\mathcal{K}}_{\geq 0}}(-1)^s \mathfrak{f}_{\delta}^s(B)
\end{equation}


Both functions  $\mathfrak{f}_{\delta}(B)$  and $f_{\delta}(B)$ for  $\delta\in\mathbb{Z}^N$ are polynomials.

The following ruler of differentiation holds

\begin{equation}
\label{prdf}
\frac{\partial}{\partial B_X} \mathfrak{f}^s_{\delta}=\mathfrak{f}^s_{\delta-e_X},\,\,\,\frac{\partial}{\partial B_X} f_{\delta}=f_{\delta-e_X},
\end{equation}
%\begin{remark}

%	Мы знаем базис в пространстве решений системы  А-ГКЗ для серии  $A$, состоящий из функций  .  Эти функции дифференцируются по правилу


%	$$\frac{\partial}{\partial A_X}F_{\gamma}=F_{\gamma-e_X},$$

where $e_X$ is the unit basis vector corresponding to the coordinate $B_X$.

Also, as in the case of the $A$ series, it is proved that $\mathfrak{f}_{\delta}(B)$ and $f_{\delta}(B)$ are solutions of systems consisting of equations of the first type for the systems \eqref{ggkz} and \eqref{agkzgn} respectively.


Introduce vectors



%\begin{equation}
\begin{align}
\begin{split}
\label{hkappa} 
&h_{\kappa}=e_X-e_{\widehat{-X}},\,\,\,\,\,  \kappa=1,...,T\\
&x_{\chi}=e_{\pm n,...,\hat{\pm i},...,\hat{\pm j},...,\pm 1,-i,i}-
e_{\pm n,...,\hat{\pm i},...,\hat{\pm j},...,\pm 1,-i,-j}-\\&-e_{\pm n,...,\hat{\pm i},...,\hat{\pm j},...,\pm 1,i,j}, \,\,\,\,\,  \chi=1,...,Z
\\&w_{\epsilon}=e_{\pm n,...,\hat{i},...,\pm 1 ,-i}+e_{\pm n,...,\hat{i},...,\pm 1 ,i}-e_{\pm n,...,\hat{i},...,\pm 1 ,0},\,\,\,\,\,\epsilon=1,...,E.
\end{split}
\end{align}
%\end{equation}
%знак $\pm$ определяется согласно правилу  \eqref{znk}.


%:Пусть  $\kappa=1,...,$

Note that when one fixes $\kappa$  then one fixes a subset of $X=\{i_1,...,i_t\}$. Introduce the functions



	\begin{align}
	\begin{split}
\label{sol1}
&\mathcal{F}^s_{\delta}(B):=\sum_{t_1\in \mathbb{Z}^T,t_2\in\mathbb{Z}^{E},t_{3}\in \mathbb{Z}^{Z}} \Big (  \prod_{\kappa}(\pm_{\kappa} 1)^{t_1^{\kappa}}\cdot (\frac{\sqrt{-1}}{2})^{\sum_{\epsilon} t_2^{\epsilon}} \Big) \cdot \mathfrak{f}^s_{\delta+t_1 h+t_2 w+t_3x}(B ) \text{ for the series }B,\\
&\mathcal{F}^s_{\delta}(B):=\sum_{t_1\in \mathbb{Z}^K,t_{3}\in \mathbb{Z}^{Z}} \Big (  \prod_{\kappa}(\pm_{\kappa} 1)^{t_1^{\kappa}} \Big) \cdot \mathfrak{f}^s_{\delta+t_1 h+t_3x}(B ) \text{  for the series }D
\end{split}
\end{align}
		\begin{equation}
\label{sol}
F_{\delta}(B):=\sum_{s\in\mathbb{Z}^{\mathcal{K}}_{\geq 0}}(-1)^s \mathcal{F}^s_{\delta}(B)
\end{equation}


One uses a notation $t_1 h:=t_1^{1} h_1+...+t_1^{T} h_T$,  $t_2 w:=t_2^{1} w_1+...+t_2^{E} w_E$, $t_3 \chi:=t_3^{1} x_1+...+t_3^{Z} x_{Z}$.
Introduce a definition

\begin{definition}
The Gelfand-Tselin lattice is defined as follows
\begin{align*}
&\mathcal{B}^{g_n}_{GC}:=<\mathcal{B}^{g_n},h_{\kappa}, w_{\epsilon},x_{\chi}>\,\,\,\text{ for the series }B,\\
&\mathcal{B}^{g_n}_{GC}:=<\mathcal{B}^{g_n},h_{\kappa},x_{\chi}>\,\,\,\text{ for the series }D
\end{align*}
\end{definition}


Using  \eqref{prdf} one gets


\begin{theorem}	
	\label{tagkz}
The space of polynomial solutions of the GKZ system for the series $B$, $D$ has a basis consisting of  function	$\mathcal{F}_{\delta}(B)=\mathcal{F}^0_{\delta}(B)$


for various $mod\mathcal{B}^{g_n}_{GC}$ vectors $\delta\in\mathbb{Z}^N$, such that this function is nonzero. The $ sign(\pm_{\kappa} 1)$ is defined according to the rule \eqref{znk}.

The space of polynomial solutions of the A-GKZ system for $g_n$ has the basis (with the same choice $\delta$) of the function $F_{\delta}(B)$


\end{theorem}	


%	Смысл обозначения   $B^{g_n}_{GC}$  будет объяснён ниже.

%	Функции  \eqref{sol}  при этом зависимы.

\begin{corollary}
	\label{gkzagkz}
There is a one-to-one correspondence between the spaces of polynomial solutions of GKZ system \eqref{ggkz} and the A-GKZ system \eqref{agkzgn}.
\end{corollary}	
%\end{remark}
%\end{proof}

This theorem is a manifestation of the following general idea in the analytical theory of differential equations: given a  system of equations, a simplified system is constructed, which is solved explicitly (GKZ is a "simplification" of A-GKZ). Then, for each solution of the simplified system, the solution of the initial system is constructed. There are several formalizations of this idea  \cite{br}, \cite{st}.




\subsection{The systems for the tensor representations in the case of the series $B$, $C$, $D$}
\label{s2}

It is known that in the case of the $C$ series, any finite-dimensional representation
\footnote{Thus, the GKZ and A-GKZ systems for the $C$ series are the systems constructed in this Section.} is realized as a sub-representation in the tensor power of the standard representation. In the case of series
$B$, $D$ it is possible to set the problem of construction of models only for such representations. To build a model of tensor representations, we will use simpler GKZ and A-GKZ systems than in the previous section.




%Также именно так построенные системы мы и будем называть системами  ГКЗ и  А-ГКЗ для серии  $C$.


Consider the independent variables $A_X$, antisymmetric by $X$. When one passes to the Zhelobenko model for all $X$, $a_X$ is substituted instead of $A_X$.


% (или  $\bar{a}_X$    для серии  $D$, при  $|X|=n$, если хотим получить представление с $m_{-1}<0$).


Take the auxiliary lattice $\bar{\mathcal{B}}^{g_n}$ constructed above. We denote it now also by  $\mathcal{B}^{g_n}$. Take a GKZ system based on it and add to it   the Jacobi equation.
 
 
\begin{definition}   {\it The GKZ system for the tensor representations} is the following system of equations

\begin{equation}\label{ggkzt}
\begin{cases}
& \mathcal{O}_{\alpha}\mathcal{F}:=(\frac{\partial^2}{\partial A_{i,Z}\partial A_{j,y,Z}}-\frac{\partial^2}{\partial A_{j,Z}\partial A_{i,y,Z}})\mathcal{F}=0,\\
&(\frac{\partial}{\partial A_X}-\pm\frac{\partial}{\partial A_{\widehat{-X}}})\mathcal{F}=0,\text{  the sign is defined in   \eqref{znk}}
\end{cases}
\end{equation}

\end{definition}

\begin{definition}  {\it The A-GKZ system for the tensor representations} is the following system of equations

\begin{equation}
\label{agkzgnt}
\begin{cases}
&\bar{\mathcal{O}}_{\alpha}F=(\frac{\partial^2}{\partial A_{i,Z}\partial A_{j,y,Z}}-\frac{\partial^2}{\partial A_{j,Z}\partial A_{i,y,Z}}+\frac{\partial^2}{\partial A_{y,Z}\partial A_{i,j,Z}})F=0,\\
&(\frac{\partial}{\partial A_X}-\pm\frac{\partial}{\partial A_{\widehat{-X}}})F=0.
\end{cases}
\end{equation}
\end{definition}


\begin{definition}
The Gelfand-Tsetlin lattice for the tensor representations  is define as follows
	$$
	\mathcal{B}^{g_n}_{GC}:=<\mathcal{B}^{g_n},h_{\kappa}>
	$$
\end{definition}

The solutions of the A-GKZ system are constructed as follows 
	\begin{align}
\begin{split}
\label{sol10}
&\mathcal{F}^s_{\delta}(A):=\sum_{t_1\in \mathbb{Z}^K} \Big (  \prod_{\kappa}(\pm_{\kappa} 1)^{t_1^{\kappa}} \Big) \cdot \mathfrak{f}^s_{\delta+t_1 h}(A ),
\end{split}
\end{align}

where $\mathfrak{f}^s_{\delta}(A)$ is defined by the same formula  \eqref{fss},  where however one uses the variables  $A_X$ and another auxiliary lattice   $\mathcal{B}^{g_n}$.
\begin{equation}
\label{sol0}
F_{\delta}(A):=\sum_{s\in\mathbb{Z}^{\mathcal{K}}_{\geq 0}}(-1)^s \mathcal{F}^s_{\delta}(A)
\end{equation}

An analogue of the Theorem \ref{tagkz} and the Corollary  \ref{gkzagkz} holds.



\section{The Gelfand-Tsetlin diagrams for the algebra  $g_n$}
\label{dgc}


By analogy with the Section \ref{dgca} let use a definition of a Gelfand-Tsetlin diagram for the algebra $g_n$.

One defines  Gelfand-Tsetlin diagrams as follows.



\begin{definition}
	\label{d1}
	
	
	A Gelfand-Tsetlin diagram for the algebra $g_n$ is a shifted lattice $\Pi=\gamma+\mathcal{B}_{GC}^{g_n}$ consisting of integer vectors $\gamma\in\mathbb{Z}^N$, such that in $\Pi$ there is a vector that has only non-negative coordinates.
	

\end{definition}

In this case,  to a diagram $\delta$ there corresponds to a set of higher weights \begin{equation}\label{wtk}wt_{n-k+1}(\delta)=[m_{-n,k},...,m_{-1+(k-1),k}]\end{equation} for algebras $g_{n-k+1}$, $k=1,...,n$ constructed by the rule




\begin{align}
\begin{split}
\label{strok}
&m_{p,q}=\sum_{X: X\text{ contains } \geq (n+p+1)\text{ indices whose absolute value} \geq q}\gamma_X-\\&- \sum_{X: X\text{  contains } \geq (-p-1)\text{ indices whose absolute value  }  \geq q}\gamma_X,
\\&p=-n,...,-1,\,\,\,\,\,\, q=1,...,n, p\leq -q+1
\end{split}
\end{align}

This definition is correct, since it does not depend on the choice of the representative of $\gamma$ in the equivalence class $mod \mathcal{B}_{GC}^{\mathfrak{g}_n}$.


\begin{definition}
The row   $[m_{-n,1},..., m_{-1,1}]$ is called the highest weight of a diagram.
\end{definition}

Another definition of Gelfand-Tsetlin diagrams is more familiar, in which some tables are constructed that encode the basic vectors of a finite-dimensional irreducible representation of $g_n$ (see \cite{m}). At the same time, these tables contain the rows $m_{p,q}$, defined in \eqref{strok}.

It is shown below that diagrams in the sense of the definition of \ref{d1} are also in one-to-one correspondence with the basic vectors of  finite-dimensional irreducible representations of $g_n$.
From this we can conclude that what is defined in \ref{d1} is a manifestation of the same essence that is encoded by the Gelfand-Tsetlin diagram in the classical, combinatorial sense.

The advantage of the definition of \ref{d1}  becomes clear when one calculates the formulas for the action of generators. If you think in terms of vectors $\gamma$, these formulas look very simple and natural (see  \eqref{deistviel}).



\section{Models of representations in ther cases  $B$, $C$, $D$}
\label{mdl}

\subsection{The action }

\label{razddeist}

There is an action of the algebra $g_n$ on the variables $A_X$, given by the formulas \eqref{deistvie}, where an expression $F_{i,j}$ through $E_{i,j}$ is used.

The action of $g_n$ onto the variables $B_X$ in the case of the series $B$, $D$ for $X$ and $\widehat{-X}\neq \{\pm n,...,\pm 1\}$ is defined in the same way as onto $A_X$. In the case of $X$ or $\widehat{-X}=\{\pm n,...,\pm 1\}$, the action is defined so that $B_X$ for these $X$ form a spinor representation. To obtain such formulas, one needs to take the formulas of the action $g_n$ on the roots of the determinants $\sqrt{a_X}$ (or $\sqrt{\bar{a}_X}$. If one wants to obtain a representation with $m_{-1}<0$ for the series $D$) and replace $\sqrt{a_X}$ (or $\sqrt{\bar{a}_X}$) by $B_X$. Thus, in the case of the $D$ series, there are two ways to construct an action.



The explicit form of the resulting formulas was discussed when proving the Sentence \ref{spb}, see formulas \eqref{fl1}, \eqref{chtn}.

Thus, the spaces of polynomials $\mathbb{C}[A]$ and $\mathbb{C}[B]$ become representation spaces of the algebra $g_n$.


\subsection{The plan}
\label{plan}
Next, we want to get the following results..

\begin{enumerate}
	\item  To prove that Gelfand-Tsetlin diagrams in the sense of the definition \ref{d1} encode the basic vectors of irreducible finite-dimensional representations.
	\item To prove that the solution space of the A-GKZ system is a  model of representations.
	\item  To construct a base of the Zhelobenko model consisting of $\Gamma$-series or monomials.
\end{enumerate}


To achieve these goals, we first prove that the solution space of the A-GKZ system coincides with the solution space of the ideal of relations between determinants. It follows from this that the solution space A-GKZ {\it contains } a representation model, and the number of Gelfand-Tsetlin diagrams in the sense of \ref{d1} for a fixed highest weight {\it is not greater than } the dimension of the representation of this highest weight.

Next, we check that the span of $\Gamma$-series in determinants constructed from Gelfand-Tsetlin diagrams in the sense of \ref{d1} coincides with the Zhelobenko model. It follows from this that the number of Gelfand-Tsetlin diagrams in the sense of \ref{d1} for a fixed highest weight {\it is not less than } the dimension of the representation of this highest weight.

From this we can already conclude that the number of Gelfand-Tsetlin diagrams in the sense of \ref{d1} for a fixed highest weight {\it is equal to } the dimension of the representation.


Hence, by dimension reasons, one  concludes that {\bf 1) } the solution space of the A-GKZ system is a  model of representations (and the basic solutions $F_{\gamma}$ form a base of the A-GKZ model); {\bf 2)} $\Gamma$ are series in determinants constructed using the Gelfand-Tsetlin  diagrams in the sense of definition  \ref{d1},  form the basis of the A-GKZ model.


These   goals are achieved in the sections  \ref{cel1}; \ref{cel2}; \ref{monomy}.

\subsection{Invariance of the solution space of the A-GKZ system}

\label{modela}

This section discusses the A-GKZ systems defined in the Sections  \ref{ser1}, \ref{s2}.



\begin{propos}
	\label{prdl}
The solution space of the A-GKZ system is $g_n$-invariant.
\end{propos}

\begin{proof}



We consider separately the A-GKZ systems form Sections  \ref{ser1}, \ref{s2}.
Let's prove the statement first for the A-GKZ system from the Section\ref{ser1}.


Let  $I_{g_n}\subset \mathbb{C}[A]$  be an ideal of relations between determinants  $a_X$ (see the Remark  \ref{rem1}).
Define the ideal $\bar{I}_{g_n}$ in the ring of differential operators with constant coefficients $\mathbb{C}[\frac{\partial}{\partial A_X}]$, which is obtained from   $I_{g_n}$ by the substitution

$$
A_X\mapsto  \frac{\partial }{ \partial  A_X}.
$$


The action of $g_n$ onto variables $A_X$ gives an action  of $g_n$ on $\mathbb{C}[\frac{\partial}{\partial_X}]$.
Moreover, since the ideal $I_{g_n}$ is invariant, the ideal $\bar{I}_{g_n}$ is also invariant, and hence its space of polynomial solutions, which we denote as $Sol_{\bar{I}_{g_n}}$, is also invariant.

Let us show that $Sol_{\bar{I}_{g_n}}$ coincides with the space of polynomial solutions of the A-GKZ system from the Section \ref{s2}.
In the Lemma \ref{lms} it was proved that the ideal $I_{g_n}$ is generated by the ideal $I_{\mathfrak{gl}_m}$ ($m=2n$ for series $C$, $D$, and $m=2n+1$ for series $B$) and Jacobi relations.  Thus

$$
Sol_{\bar{I}_{g_n}}=Sol_{\bar{I}_{\mathfrak{gl}_m}} \cap Sol_{  Jacobi},
$$

where $Jacobi$ is a set of equations of the second type for the A-GKZ system from the section \ref{s2}. For the case of the $A$ series,
literally repeating the reasoning
from \cite{a4}, one 
gets that $Sol_{\bar{I}_{\mathfrak{gl}_m}}$
coincides with the solution space of the A-GKZ system for the $A$ series. But the A-GKZ system from the  Section 
\ref{s2} is a union of the A-GKZ system for the $A$ series and the Jacobi equations. So, $Sol_{\bar{I}_{g_n}}$  coincides with the space of polynomial solutions of the A-GKZ system from the section \ref{ser1}. Hence, the space of polynomial solutions of the A-GKZ system of the \ref{ser1} section is invariant.


To  proceed to the consideration of the A-GKZ system from the section \ref{s2},
let us do  the following conclusion from the $\mathfrak{gl}_m$-invariance of the space of polynomial solutions of the A-GKZ system for the $A$ series. If $x\in g_n$, then



\begin{equation}
\label{como}
[x,\mathcal{O}_{\alpha}]\in \bar{I}_{\mathcal{O}_{}}=\bar{I}_{\mathfrak{gl}_{m}},
\end{equation}

where $\bar{I}_{\mathcal{O}_{}}$ is an ideal in the space of differential operators with constant coefficients generated by A-GKZ operators $\mathcal{O}_{\beta}$ for the series $A$. Since for the series $A$one has $Sol_{\bar{I}_{\mathfrak{gl}_m}}=Sol_{AGKZ}$, then the differential operator constructed according to an arbitrary (not necessarily of the form \eqref{spgl}) Plucker relation belongs to $\bar{I}_{\mathcal{O}_{}}$. Here $Sol_{AGKZ}$ is the space of polynomial solutions of the A-GKZ system.

Now consider the A-GKZ system for $g_n$ from the Section \ref{ser1}.
Consider  a substitution of  variables $A_X$ instead of the variables $B_X$ by the rule:



\begin{equation}
\label{podst}
A_X\mapsto B_X, \,\,\, X\neq \begin{cases}\{\pm n,...,\pm 1\},\\   \{\pm n,...,\pm 1,0\}  \end{cases},	A_X\mapsto B^2_X, \,\,\,  \text{ othewise}.
\end{equation}


Directly (using transformations leading \eqref{ffa} to \eqref{fl1} and \eqref{chtn}), it is verified that this substitution is consistent with the action modulo the ideal $I'_{g_n}\subset \mathbb{C}[B]$ generated by the Plucker relations \eqref{spgl}, Jacobi relations and relations



\begin{align}
\begin{split}
\label{sob}
&B_{\pm n,...,\hat{i},...,\pm 1,i}B_{\pm n,...,\hat{i},...,\pm 1,-i}=\frac{\sqrt{-1}}{\sqrt{2}} B_{\pm n,...,\hat{i},...,\pm 1,i,0},\,\,\, \text{ only for the series  }B,\\
& B_{\pm n,...,\hat{\pm i},...,\hat{\pm j},...,\pm 1,-i,i}=B_{\pm n,...,\hat{\pm i},...,\hat{\pm j},...,\pm 1,-i,-j}B_{\pm n,...,\hat{\pm i},...,\hat{\pm j},...,\pm 1,i,j}
\end{split}
\end{align}


the square of which is a consequence of the Plucker and Jacobi relations.


Now one can check that the ideal $\bar{I}'_{g_n}$ in the space $\mathbb{C}[\frac{\partial}{\partial B}]$ is invariant. For generators $\bar{I}'_{g_n}$ corresponding to the Plucker relations, this is checked using the equality \eqref{como} and the consistency of the substitution \eqref{podst} and the action $g_n$. For generators $\bar{I}'_{g_n}$ constructed by the Jacobi and \eqref{sob} relations, this is checked by direct calculation. Hence, the space of polynomial solutions $\bar{I'_{g_n}}$, which is nothing but the space of polynomial solutions of the A-GKZ system from the section \ref{s2}, is also invariant.



\end{proof}


\begin{lemma}\label{ostnach} The solution space of the A-GKZ system contains a representation model.
	
\end{lemma}

\begin{proof}

According to the Proposition \ref{prdl}, the solution space of the A-GKZ system for $g_n$ is a representation of $g_n$.
Consider in the space $Sol_{AGKZ}$ solutions of A-GKZ for $g_n$ a finite-dimensional subspace of functions $Sol_{AGKZ}^{l_{-n},...,l_{-1}}$, of a fixed homogeneous degree $l_{p}$ by $A_X$ (or $B_X$) with $|X|=(p+n+1)$ and $(-p-1)$, $p=1,...,n$. One has



$$
Sol_{AGKZ}=\bigoplus_{l_{-n},...,l_{-1}}Sol_{AGKZ}^{l_{-n},...,l_{-1}}
$$


Under the action of $g_n$, homogeneous powers over determinants of the same size are preserved. Note also that the basis \eqref{sol} is consistent with this decomposition of the solution space A-GKZ for $g_n$



Put $m_{p}=l_p+l_{p+1}+...$.
Then in   $Sol_{AGKZ}^{l_{-n},...,l_{-1}}$  the following $g_n$-highest vector of the weight  $[m_{-n},m_{-n+1},...,m_{-1}]$ is contained

\begin{equation}
\label{stv}
(A_{-n}+\pm A_{\widehat{n}})^{m_{-n}-m{-n+1}}( A_{-n,-n+1}+\pm A_{\widehat{n-1,n}})^{m_{-n+1}-m_{-n+2}}....,
\end{equation}

Or, in the case of series $B$, $D$ is a similar expression with variables $B$. Moreover, in the case of the $D$ series
 the last $n$-th factor is taken to the power  $|m_{-1}|$. The signs in brackets are defined in \eqref{znk}.

Thus  $Sol_{AGKZ}^{l_{-n},...,l_{-1}}$ contains an irreducible representation of the highest weight $[m_{-n},m_{-n+1},...,m_{-1}]$.
% Для доказательства Теоремы  \ref{ost} остаётся проверить, что $dimSol_{AGKZ}^{l_{-n},...,l_{-1}}$ равна размерности неприводимое представление веса $[m_{-n},m_{-n+1},...,m_{-1}]$.

%Для этого достаточно построить {\it базис } в пространстве $Sol_{AGKZ}^{l_{-n},...,l_{-1}}$.  �?меется базис \eqref{sol} в данном пространстве решений, индексированный диаграммами Гельфанда-Цетлина для  $g_n$.

\end{proof}


Since in
$Sol_{AZ}^{l_{-n},...,l_{-1}}$ there is a base \eqref{sol} indexed by Gelfand-Tsetlin diagrams for $g_n$ in the sense of \ref{d1}, we come to a conclusion

\begin{corollary}
\label{grsn}
Fix a highest weight $[m_{-n},...,m_{-1}]$. Then the number of Gelfand-Tsetlin diagrams in the sense of \ref{d1} with such a high weight is no more than the dimension of this irreducible representation.
\end{corollary}

%Аналогично вводится скалярное произведение в случае переменных  $B$.
\subsection{ $\Gamma$-series in the Zhelobenko's realization}
\label{zhlb}

We use a short notation 

$$
\mathcal{F}_{\gamma}:=\mathcal{F}_{\gamma}(A) \text{ or }\mathcal{F}_{\gamma}(B),\,\,\,
F_{\gamma}:=F_{\gamma}(A) \text{ or }F_{\gamma}(B).
$$

\subsubsection{A formulation of the result: $\Gamma$-series generate the Zhelobenko's realization}

%Выяснить формулы действия генераторов алгебры  $F_{i,j}$ на базисные функции $F_{\gamma}(A)$ затруднительно.  Поэтому мы построим другую модель, в которой формулы для действия будут выписаны явно. Кроме того эта новая модель будет важным инструментом для установления связи  базиса  $F_{\gamma}(A)$ с базисом Гельфанда-Цетлина.

%�?так,

Consider $\Gamma$-series that are solutions of the GKZ system for the corresponding algebra $g_n$. Substitute the determinants $a_X$ in them instead of $B_X$ and $A_X$ according to the substitution rule described above. Let $W\subset Fun$ be their linear span.

\begin{theorem}
\label{gkzt}
The space $W$ is a representation model of the algebra $g_n$.
\end{theorem}
From this Theorem, the coincidence of $W$ and the Zhelobenko model immediately follows.

\begin{proof}
Consider the functions $\mathcal{F}_{\gamma}$ with a  fixed homogeneous degree $l_i$ by determinants of the same size $i$. Let's check that when the generator $F_{i,j}$ acts on a function $\mathcal{F}_{\gamma}$, a linear combination of the functions $\mathcal{F}_{\delta}$ of the same homogeneous degree is obtained.
To do this, we will use the Principal Lemma.





\subsubsection{The formulation of the Principal Lemma}


\begin{lemma} 
	\label{osnovl}	
	Let  us be given a $\Gamma$ series  $\mathfrak{f}_{\gamma}(c)$ in variables $c_X=a_{X}$ or $\sqrt{a_X}$. The lattice on which this series is constructed has a basis $v_{\alpha}$, $\alpha=1,...,K$, such that for $\alpha=1,...,{\mathcal{K}}$, ${\mathcal{K}}<K$ these generators have the following properties. They have the form
	

	$$
	v_{\alpha}=\tau_1\cdot e_{X_1}+\tau_2\cdot e_{X_2}-\tau_3\cdot e_{X_3}-\tau_4\cdot e_{X_4},
	$$
where  $e_X$  is a unit vector corresponding to the coordinate $a_X$, $\tau_i\in\mathbb{Z}_{\geq 0}$.
There exists a relation between variables of the form is associated with each such generative
	
	\begin{equation}
	\label{sbv}
	c^{\tau_1}_{X_1}c^{\tau_2}_{X_2}-c^{\tau_3}_{X_3}c^{\tau_4}_{X_4}+c^{\tau_5}_{X_5}c^{\tau_6}_{X_6}=0,
	\end{equation}		
	
	
	
for some  $X_5,X_6$, $\tau_5$, $\tau_6$. With each generator  $v_{\alpha}$  one assocates a vector
	
	$$
	r_{\alpha}=\tau_1\cdot e_{X_1}+\tau_2\cdot e_{X_2}-\tau_5\cdot e_{X_5}-\tau_6\cdot e_{X_6}.
	$$
	
Then modulo the relations   \eqref{sbv} one has
	
	
	\begin{equation}
	\label{umn}
c_X\mathfrak{f}_{\gamma}(c)=\sum_{s\in\mathbb{Z}^{\mathcal{K}}_{\geq 0}} C^{\gamma}_s\mathfrak{f}_{\gamma+e_X+sr}(c).
	\end{equation}
	
%	где  $Plk$ - идеал соотношений, порожденный  \eqref{sbv}.
	
\end{lemma}





\subsubsection{The proof of the Principle Lemma }


Take the independent variables $C_X$, antisymmetric by $X$.
For a polynomial $g(C)$  denote as $$g(\frac{\partial}{\partial C})$$  a differential operator obtained by replacing each variable $C_{X}$ by differentiating $\frac{\partial}{\partial C_X}$.



Denote as $Plk$ the ideal generated by the relations


$$
C^{\tau_1}_{X_1}C^{\tau_2}_{X_2}-C^{\tau_3}_{X_3}C^{\tau_4}_{X_4}+C^{\tau_5}_{X_5}C^{\tau_6}_{X_6}=0,
$$


With this relation, one relates a system A-GKZ, composed of equations.
\begin{align*}
\Big( (\frac{\partial }{\partial  C_{X_1}})^{\tau_1} (\frac{\partial }{\partial  C_{X_2}})^{\tau_2}-(\frac{\partial }{\partial  C_{X_3}})^{\tau_3} (\frac{\partial }{\partial  C_{X_4}})^{\tau_4}+(\frac{\partial }{\partial  C_{X_5}})^{\tau_5} (\frac{\partial }{\partial  C_{X_6}})^{\tau_6} \Big)f=0,
\end{align*}

and  the equation of the system of GKZ corresponding to the rest of the generating  $v_{\alpha}$, $\alpha=k+1,...,K$.
% Такая система имеет базис в пространстве решений  $F_{\omega}(C)$, конструируемый также, как



There is a basis in the space of solution of  this system

  $$f_{\delta}(C)=\sum_{s\in \mathbb{Z}_{\geq 0}^{\mathcal{K}}}(-1)^s\mathfrak{f}_{\delta-sr}^s(C)$$ в where $\mathfrak{f}_{\delta-sr}^s(C)$ is defined by analogy with \eqref{fss}.


Let us do the following observation (similar to how it is done in \cite{a4})


\begin{lemma}
	$\lambda_1 g_1(c)+...+\lambda_l g_l(c)=0\,\,\, mod Plk $, if
	$$
	(\lambda_1 g_1(\frac{\partial }{ \partial C})+...+\lambda_l g_l(\frac{\partial }{\partial C}))f_{\delta}(C)=0
	$$
	for all  $C$.
\end{lemma}


Put  $\delta+v:=\delta+v_1+...+v_{\mathcal{K}}$.
Literally, as well as in \cite{a4}, the following statement is proved

\begin{lemma}
	\label{lma}
	\begin{equation}\label{rwo}
\mathfrak{f}_{\gamma}(\frac{\partial }{\partial C})f_{\delta}(C)=\sum_{s\in\mathbb{Z}^{\mathcal{K}}_{\geq 0}}\mathfrak{f}_{\gamma+v+sr}^s(1)f_{\delta-\gamma-sr}(C),
	\end{equation}
	
	where  $\mathfrak{f}_{\gamma+v+sr}^s(1)$ is a   result of substitution of units instead of all arguments  $C$.
\end{lemma}


Let us show that a similar equality exists for the functions $\mathcal{F}_{\gamma}(c)$ defined in \eqref{sol1}. Indeed, this function is the sum of $\Gamma$-series of the form $\mathfrak{f}_{\delta}$, for each of which has equality of the form
\eqref{rwo}. Summing them up, one gets the following equality

	\begin{equation}\label{rwo1}
\mathcal{F}_{\gamma}(\frac{\partial }{\partial C})f_{\delta}(C)=\sum_{s\in\mathbb{Z}^{\mathcal{K}}_{\geq 0}}\mathcal{F}_{\gamma+v+sr}^s(1)f_{\delta-\gamma-sr}(C),
\end{equation}

From here, literally as well as in \cite{a4}, we can conclude

\begin{equation}
\label{bxf}
c_X\mathcal{F}_{\gamma}(c)=\sum_{s\in\mathbb{Z}^k_{\geq 0}} const^{\gamma}_s\cdot \mathcal{F}_{\gamma+e_X+sr}(c)% \,\,\, mod Plk
\end{equation}

The Principal Lemma is proved.


\subsubsection{Coefficient in the Principal Lemma}

One can give a formula for the coefficients in the expansion \eqref{umn}. Literally, as in \cite{a4}, one gets:



\begin{align}
\begin{split}
\label{cs}
&const^{\gamma}_s=\frac{\mathcal{F}_{\gamma+v}^{s}(1)}{\mathcal{F}^s_{\gamma+v+e_X+sr}(1)}-\sum_{p\in \mathbb{Z}^{\mathcal{K}}_{\geq 0}: s-p\in \mathbb{Z}^{\mathcal{K}}_{\geq 0},p\neq s}\frac{\mathcal{F}_{\gamma+v}^{p}(1)
	\mathcal{F}_{\gamma+v+pr+e_X+(s-p)r}^{s-p}(1)}{\mathcal{F}_{\gamma+v+pr+e_X+(s-p)r}(1)\mathcal{F}_{\gamma+v+pr+e_X}(1)}=\\
&=
\frac{\mathcal{F}_{\gamma+v}^{s}(1)}{\mathcal{F}^s_{\gamma+v+e_X+sr} (1) }-\sum_{p\in \mathbb{Z}^{\mathcal{K}}_{\geq 0}: s-p\in \mathbb{Z}^{\mathcal{K}}_{\geq 0},p\neq s}\frac{\mathcal{F}_{\gamma+v}^{p}(1)\mathcal{F}_{\gamma+v+e_X+sr}^{s-p}(1)}{\mathcal{F}_{\gamma+v+e_X+sr}(1)
	\mathcal{F}_{\gamma+v+pr+e_X}(1)}
\end{split}
\end{align}



\subsubsection{The action of generators. The end of the proof of the Theorem \ref{gkzt}}

Recall that $c_X=a_X$ or $\sqrt{a_X}$. The action of $F_{i,j}$ on $a_X$ is given by the formula \eqref{edet1}, the action on $\sqrt{a_X}$ is described in the proof of the Proposition \ref{spb}, see formulas \eqref{fl1}.

In both cases one can write


\begin{equation}
F_{i,j}=\sum_{Y_1,Y_2} c_{Y_1}\frac{\partial}{\partial c_{Y_2}}.
\end{equation}


Using the rule of differentiation  of  $\Gamma$-series and the Principle Lemma \ref{osnovl}, one   gets
\begin{lemma}
	
	\begin{equation}
	\label{deistviel}
	F_{i,j}\mathcal{F}_{\gamma}=\sum_{Y_1}\sum_{s\in \mathbb{Z}^{\mathcal{K}}_{\geq\ 0}}
	const_{s}^{\gamma-e_{Y_2}} \cdot \mathcal{F}_{\gamma-e_{Y_2}+e_{Y_1}+sr}(c)
	\end{equation}
	
	
\end{lemma}

\begin{corollary}
	The linear span of functions $\mathcal{F}_{\gamma}$ is a representation.
\end{corollary}

Let's return to the proof of the \ref{gkzt} Theorem. In the Theorem \ref{ost1}, the conditions for the function necessary and sufficient for it to be included in the Zhelobenko model are written out.


Applying it, one sees that $W\subset Yar$ is a sub-representation containing all finite-dimensional irreducible representations. Then $W=Zh$.
The \ref{gkzt} theorem is proved.



\end{proof}


As a corollary one gets.

\begin{corollary}
	\label{grsv}

Fix the highest weight of $[m_{-n},...,m_{-1}]$. Then the number of Gelfand-Tsetlin diagrams in the sense of \ref{d1} with this highest weight is not less than the dimension of this irreducible representation.

\end{corollary}


\subsubsection{The Gelfand-Tsetlin diagrams} 

\label{cel1}

Using the Corollaries  \ref{grsn}, \ref{grsv}, one gets.

\begin{theorem}\label{chdrzm}

Fix the highest weight  $[m_{-n},...,m_{-1}]$. Then the number of Gelfand-Tsetlin diagrams in the sense of \ref{d1} with this weight is equal to the dimension of this irreducible representation.

\end{theorem}

Thus, the first goal from the \ref{plan} section has been achieved.

\subsection{The A-GKZ model}
\label{cel2}


Using the Lemma \ref{ostnach}, the Theorem \ref{chdrzm} and using dimension arguments, one gets the following result.


\begin{theorem}\label{osnt0} 

There exists an A-GKZ model formed by the space of solutions of the A-GKZ system. In the case of series $B$, $D$, it has a basis consisting  of functions $F_{\gamma}(B)$, and in the case of series $C$ it has a basis consisting of functions $F_{\gamma}(A)$.One takes as $\gamma$ all possible Gelfand-Tsetlin diagrams in the sense of the definition of \ref{d1}. Those functions for which
the diagram has the highest weight $[m_{-n},...,m_{-1}]$,
% \% \ begin{equation}\label{s}\sum_{X: |X|=n+p+1 \text{or } -p-1}\gamma_X=|m_p|-|m_{p+1}|,\,\,\, p=-n,...,-1\end{equation}
form a basis in the representation space of the highest weight $[m_{-n},...,m_{-1}]$.

 
\end{theorem}

Due to the independence of the variables $A_X$, one can introduce an invariant
scalar product on the polynomials $f(A)$, $g(A)$ by the formula


\begin{equation}
\label{skp}
<f(A),g(A)>:=f(\frac{d}{dA}) g(A)\mid_{A=0},
\end{equation}


where $f(\frac{d}{dA}) $ is the result of substituting in $f$ the differential operators $\frac{d}{dA_X}$ instead of the variables $A_X$.


Similarly, a scalar product is introduced in the case of variables $B$.


Thus, the second goal from the \ref{plan} section has been achieved.


\subsection{Basis in the Zhelobenko's realization}
\label{monomy}

In the previous Sections, the functions $\mathcal{F}_{\gamma}(A)$ and $\mathcal{F}_{\gamma}(B)$ were constructed. In Sections \ref{ser1}, \ref{s2} when defining variables $A_X$, $B_X$ it was also indicated which substitution of determinants in these variables should be carried out when  one passes to the Zhelobenko's realization. Let us do such a substitution. The resulting function in both cases is denoted as $\mathcal{F}_{\gamma}(a) $.





\begin{theorem}

The functions $\mathcal{F}_{\gamma}(a)$, constructed for various Gelfand-Tsetlin diagrams in the sense of the definition of \ref{d1}, form the basis of the Zhelobenko's realization.
	

\end{theorem}

\begin{proof}

Since all these functions satisfy the conditions of the \ref{gkst} Theorem, they all belong to the Zhelobenko model and it remains to prove the linear independence of these functions.

The linear span of the function $\mathcal{F}_{\gamma}(a)$, for Gelfand-Tsetlin diagrams in the sense of the definition of \ref{d1}, having the highest weight $[m_{-n},...,m_{-1}]$, form a sub-representation in the Zhelobenko's model of the highest weight $[m_{-n},...,m_{-1}]$. Their number is equal to the dimension of this representation. So they are independent, and therefore form a basis.

	
	
\end{proof}



Note that we not only proved that the linear span of $\mathcal{F}_{\gamma}(a)$ is a  model of representations of $g_n$, but also we explicitly wrote out formulas for the action of algebra generators.


Let's prove another  statement. When defining the Gelfand-Tsetlin lattice $\mathcal{B}_{GC}^{g_n}$, its embedding into the lattice $\mathbb{Z}^N$ of exponents (integers or half-integers) of monomials in determinants was given (see the section \ref{systemy}). The image of the vector $\gamma\in \mathcal{B}_{GC}^{g_n} $ with this embedding is denoted as $\bar{\gamma}$. A similar construction can be implemented with vectors of shifted lattices. So for a Gelfand-Tsetlin diagram $\Pi$, we choose a representative of $\gamma$, then we construct a monomial in the determinants $a^{\bar{\gamma}}$.





\begin{theorem}

For every Gelfand-Tsetlin diagram in the sense of the definition of \ref{d1}, we construct a monomial $a^{\bar{\gamma}}$ according to the above procedure. Then these monomials form the basis of the Zhelobenko's realization.

	
\end{theorem}
\begin{proof}
	
	Indeed, by Lemma \ref{lma} one has
	
	$$
	\mathcal{F}_{\gamma}(a)=a^{\bar{\gamma}}+h.o.t.,
	$$
	where  $h.o.t.$   is a sum of monomials  $a^{\bar{\delta}}$, and  $\gamma\prec \delta$, where the ordering $\prec$ defined in the same way as in \eqref{poryadok}, with the replacement of the lattice by $\mathcal{B}_{GC}^{g_n}$.
	
	In particular one sees that  if $\gamma=\gamma'\,\,\, mod \mathcal{B}_{GC}^{g_n}$, then
	
	$$
	a^{\bar{\gamma}}=a^{\bar{\gamma'}}+h.o.t.
	$$

Consider a set of vectors $\gamma\in\mathbb{Z}_{\geq 0}^N$, which are representatives of Gelfand-Tsetlin diagrams with the highest weight $[m_{-n},...,m_{-1}]$. Then the functions $\mathcal{F}_{\gamma}(a)$ form the basis
of this representation in the Zhelobenko model, and the set of functions $a^{\bar{\gamma}}$ is related with $\mathcal{F}_{\gamma}(a)$ by an upper-triangular transformation. Hence, the set $a^{\bar{\gamma}}$ also forms a basis.
 
	
	
\end{proof}
Thus, the third goal from the \ref{plan} section has been achieved.


\section{ A relation to the Gelfand-Tsetlin base}
\label{gttl}
Let's establish a relation between the bases $F_{\gamma}(A)$ (or $F_{\gamma}(B)$) in the A-GKZ realization and the basis $\mathcal{F}_{\gamma}(a)$ in the Zhelobenko's realization with the Gelfand-Tsetlin basis $G_{\gamma}$.


A Gelfand-Tsetlin type base can be defined as an eigenbasiss for the following maximal commutative subalgebra in $GT\subset U(g_n)$, called the Gelfand-Tsetlin algebra. For the subalgebra chain $g_1\subset g_2\subset ...\subset g_n$ take the centers of universal wrappers $Z(U(g_1)),...,Z(U(g_n))$ and generate a subalgebra with them in $GT\subset U(g_n)$ This is the Gelfand-Tsetlin algebra. You can explicitly write out the generators




\begin{equation}
\label{sso}
C_{p}^q=\sum_{i_1,...,i_p\text{ modulo }\geq q}F_{i_1,i_2}F_{i_2,i_3}...F_{i_{2p},i_1}
\end{equation}


It is directly verified that these generators are self-adjoint with respect to the scalar product \eqref{skp}.

It is known that the proper basis for the algebra $GT$, that is, the Gelfand-Tsetlin basis, is not unique in the case of the series $B$, $C$, $D$. Let us  give a construction of some Gelfand-Tsetlin basis. To do this, we first prove the following statement



\begin{lemma}
\label{lmff}
The scalar product $<\mathcal{F}_{\gamma_1}(a),\mathcal{F}_{\gamma_2}(a)>$ can be nonzero only if there exists $\omega$ such that $\gamma_1\preceq \omega$, $\gamma_2\preceq\omega$.

\end{lemma}

\begin{proof}


For certainty, we  conduct arguments using variables $A_X$. When using variables $B_X$, the reasoning is literally the same.

To calculate the scalar product, let's go to the A-GKZ realization. Then the vector $\mathcal{F}_{\gamma_1}(a)$ in the Zhelobenko's realization  is written in A-GKZ  realization as a function of the form $\mathcal{FF}_{\gamma_1}(A):=\mathcal{F}_{\gamma_1}(A)+h(A)$, where $h(A)\in I_{g_n}$. Using the formula \eqref{skp} and taking into account that $F_{\delta}(A)$ is vanished by the ideal  $\bar{I}_{g_n}$, one gets

	
	$$
	<\mathcal{F}_{\gamma_1}(A)+h(A),F_{\delta}(A)>=	<\mathcal{F}_{\gamma_1}(A),F_{\delta}(A)>.
	$$
	
	By definition  $<\mathcal{F}_{\gamma_1}(A),F_{\delta}(A)>$ can be non-zero only if   $\gamma_1\preceq \delta$. Then  $\mathcal{FF}_{\gamma_1}=\sum_{s\in \mathbb{Z}^{\mathcal{K}}_{\geq 0}} c_{\gamma_1}^s F_{\gamma_1+sr}(A)$. Thus
	 	$$<\mathcal{F}_{\gamma_1}(a),\mathcal{F}_{\gamma_2}(a)>=\sum_{s_1,s_1\in\mathbb{Z}^{\mathcal{K}}_{\geq 0}   }c_{\gamma_1}^{s_1} c_{\gamma_2}^{s_2} <F_{\gamma_1+s_1r}(A),  F_{\gamma_2+s_2r}(A)  > .$$


Considering  the support, we conclude that the scalar product of $F_{\gamma_1+s_1r}(A)$ and $F_{\gamma_2+s_2r}(A) $ can be nonzero only if there exists $\omega$ such that $\gamma_1+s_1r, \gamma_2+s_2r\preceq\omega$
This condition is equivalent to the condition from the formulation of the Lemma.

   
\end{proof}


\begin{corollary}

There exists an orthogonal basis $\mathcal{G}_{\gamma}$ in the Zhelobenko's model, expressed in terms of the basis $\mathcal{F}_{\gamma}$ in an upper-triangular way with respect to the ordering $\prec$, that is, $$\mathcal{G}_{\gamma}(a)=\sum_{s\in\mathbb{Z}^{\mathcal{K}}_{\geq 0}} d_{\gamma}^s\cdot\mathcal{F}_{\gamma+sr}(a),\,\,\,\, d_{\gamma}^0=1.$$

\end{corollary}



\begin{theorem}
	\label{tgc}
	The base $\mathcal{G}_{\gamma}$ is the Gelfand-Tselin base.
\end{theorem}
\begin{proof}


The Gelfand-Tsetlin base is an eigenbase for the Gelfand-
Tsetlin algebra. Its generators \eqref{sso} are self-adjoint with respect to the scalar product \eqref{skp}, so that the space of a finite-dimensional irreducible representation $V\subset Zh$ is represented as an orthogonal direct sum of proper subspaces for $GT$. The set of eigenvalues defining one of the direct summands is given by the set of $g_{n-k}$-higher weights arising from the decomposition of $V$ into the sum of irreducibles under the standard procedure for limiting the algebra $g_n\downarrow g_{n-k}$. We introduce the notation $mu=\{\mu_n,...,\mu_1\}$ for the set of $g_n,...,g_1$-higher weights. The corresponding term in $V$ is denoted as $V_{\mu}$.


Also, for the Gelfand-Tsetlin diagram $\gamma$, we introduce the notation (see \eqref{wtk})


 
 $$
 V^{\gamma}:=V_{\mu},\,\,\,\text{ where } \mu=\{wt_{n}(\delta),....,wt_{1}(\delta)\}.
 $$

Let us prove the following statement.

\begin{propos}
	\label{pfv}
	$\mathcal{F}_{\gamma}(a)\in\oplus_{s\in\mathbb{Z}^{\mathcal{K}}_{\geq 0}} V^{\gamma+sr}$
\end{propos}

\begin{proof}
	

Consider first the monomial $a^{\delta}$. Applying the Jacobi relation if necessary, we can assume that $a^{\delta}$ depends only on the determinants $a_X$ with $|X|\leq n$.


Let's introduce the {\it raising} operation.
Let us first consider a realization in the space of functions on the subgroup $Z$ of upper-unitriangular matrices on the corresponding group $G$ (see \cite{zh}). Denote as $z_{i,j}$  functions of matrix elements on $Z$. In this case, a finite-dimensional irreducible representation of $V$ is realized in the space of polynomials in variables $z_{i,j}$ (including in the case of a half-integer of the highest weight) satisfying the indicator system \eqref{indsys}, the indicators in which are determined by the highest weight according to the rule \eqref{rb}.
 
	
 The monomial $a^{\delta}$ being restricted to $Z$ is written as a function $f(z_{i,j})$. In this case, one can assume that
$i<-j$ in the case of series $B$, $D$ and $i\leq -j$ in the case of series $C$. The remaining variables $z_{i,j}$ are expressed in terms of these variables as polynomials.

A function that is a $g_{n-k}$-the highest vector depends only on the variables $z_{i,j}$, $j\in\{-k,...,\hat{0},...,k\}$.


Let's The function that is $g_{n-k}$-the highest vector depends only on the variables $z_{i,j}$, $j\in\{-k,...,\hat{0},...,k\}$.


Let's define the {\it raise} procedure like this. It consists in the fact that we apply the increasing operators $F_{i,j}$, $i<j$ until we get $g_{n-k}$-the highest vector. Let's agree that this is done in the following order. First we apply $F_{-n,-(n-1)}$ to the maximum  possible power  (until the result is a non-zero result), then $F_{-n,-(n-2)}$ to the maximum  possible power, then $F_{-(n-1),-(n-2)}$, etc. That is, the operators are applied according to the order
 
 \begin{equation}
 \label{proprem}
 F_{-n,-(n-1)};  F_{-n,-(n-2)} ; F_{-(n-1),-(n-2)} ; F_{-(n,-(n-3))};...
 \end{equation}
 

Generally speaking, the operator $F_{i,j}$, $i<j$ in the realization under consideration is written as follows
 

$$
F_{i,j}=\frac{\partial}{\partial z_{i,j}}+\sum_{t<i}z_{t,i}\frac{\partial }{\partial z_{t,j}}-\pm\frac{\partial}{\partial z_{-j,-i}}+\sum_{t<-j}z_{t,-i}\frac{\partial }{\partial z_{t,j}},
$$	

where $\pm=+$ for series $B$, $D$ and $sign(i)sign(j)$ for series $C$. But in the considered order of application, the operators $F_{i,j}$ are written simply as
 


$$
F_{i,j}=\frac{\partial}{\partial z_{i,j}}-\pm\frac{\partial}{\partial z_{-j,-i}}.
$$


And taking into account on which $z_{i,j}$ our function depends on, one has $F_{i,j}=\frac{\partial}{\partial z_{i,j}}$.

The action of {\it raising } on a monomial $a^{\delta}$ can be described without going to the realization in the space of functions on $Z$. This operation acts as a substitution for $a_X\mapsto a_{X'}$, where $X'$ is obtained by the maximum possible left shift of all indexes whose absolute value $\geq k$.




Each function $f(z_{i,j})$ can be written as follows

\begin{equation}
\label{fpovysh}
f=\sum_{\alpha}c_{\alpha}\cdot \prod_{|s|>k}\frac{z_{r,s}^{\alpha_{r,s}}}{\alpha_{r,s}!}\cdot f_{\alpha}(z_{i,j}),
\end{equation}


where $\alpha$ is some index listing the terms in $f$ of the specified type.
In this case, the function $f_{\alpha}(z_{i,j})$ depends only on the variables $z_{i,j}$, $j\in \{-k,...,\hat{0},...,k\}$.
When applying the {\it raising } operation in the expression \eqref{fpovysh} one gets $\sum_ is obtained{\alpha}c_{\alpha}f_{\alpha}(z_{i,j})$, where the sum of $\alpha$ is taken, such that the vector


$$
[\alpha]:=(\alpha_{-n,-(n-1)}+\alpha_{(n-1),n},\alpha_{-n,-(n-2)}+\alpha_{(n-2),n}, \alpha_{-(n-1),-(n-2)}+\alpha_{(n-2),(n-1)},...)
$$


is the maximum relative to the lexicographic order. Such terms we call the  {\it maximal}. According to our assumption, on which $z_{i,j}$ the function $f$ depends, made at the beginning of the proof, we have $\alpha_{(n-1),n}=\alpha_{(n-2),n}...=0$. So {\it is the maximum } term is the only one and it corresponds to the maximum in the lexicographic sense of the vector of indicators $\alpha$.


The fact that $f\in V_{\mu_0}\oplus V_{\mu}\oplus...$ means the following. When using the increasing operators $F_{p,q}$ in some other order, a function is obtained at one of the steps, which is the sum of s highest vector $V_{\mu}$ and some other summand. This situation occurs exactly when there are terms in the expression \eqref{fpovysh} that are not {\it maximal}.

The weight $weight(f_{\alpha})$ of the resulting $g_{n-k}$-highest vector is calculated as follows. Let
% $weight(\delta)$ denotes the weight of the vector $a^{\delta}$, and
$h.weight$ denotes $g_n$-the highest weight of the representation in question. Then


 
 
 \begin{equation}
 weight(f_{\alpha})=h.weight-\sum_{r<s< -k}\alpha_{r,s}(e_{r}-e_{s}),
 \end{equation}

where $e_{r},e_{s}$ are unit vectors for the corresponding weight components. At the same time, the resulting vector has $weight(f_{\alpha})$ the first $(n-k)$ coordinates are taken.


When one passes from a maximum term to a non-maximal one to the weight $weight(f_{\alpha})$ a vector of the form $[0,...,1,...,-1,...,0]$  is added.



 The same change in $weight(f_{\alpha})$ occurs when calculating the weight of $weight(f_{\alpha})$ corresponding to the {\it maximum } term, but for the vector of exponents $\delta+r_{\alpha}$, where $r_{\alpha}=e_{-n,...,-r-1,-k}-e_{-n,...,-r-1,-r}-e_{-n,...,-r-1,-s,-k}+e_{-n,...,-r-1,-r,-s}$.

It follows that $a^{\delta}\in \oplus_{s\in\mathbb{Z}^{{\mathcal{K}}}_{\geq 0}}V^{\delta+sr}$.

\end{proof}



Let us return to the proof of the \ref{tgc} Theorem. Using the Proposition \ref{fpfv}, one gets that $\mathcal{G}_{\gamma}(a)$ also belongs to $\oplus_{s\in\mathbb{Z}^{\mathcal{K}}_{\geq 0}}V^{\gamma+sr}$. Since $\mathcal{G}_{\gamma}(a)$ are obtained by the orthogonalization procedure, then $\mathcal{G}_{\gamma}(a)$ is orthogonal to all vectors of the form $\mathcal{F}_{\gamma+sr}(a)$, $s\in\mathbb{Z}^{\mathcal{K}}_{\geq 0}$, $s\neq 0$. Due to the Lemma \ref{lmff}, $\mathcal{G}_{\gamma}(a)$ is also orthogonal to all other $\mathcal{F}_{\delta}(a)$, $\delta\neq\gamma+sr$ $mod\mathcal{B}_{GC}^{g_n}$. So $\mathcal{G}_{\gamma}(a)$ is orthogonal to $\oplus_{s\in\mathbb{Z}^{\mathcal{K}}_{\geq 0},s\neq 0}V^{\gamma+sr}$. It follows that $\mathcal{G}_{\gamma}(a)\inf^{\gamma}$.



This means that the basis $\mathcal{G}_{\gamma}(a)$ is consistent with the orthogonal decomposition of $V$ into the direct sum of the eigenspaces of $GT$. So $\mathcal{G}_{\gamma}(a)$ is the Gelfand-Tsetlin basis.



\end{proof}
 

Applying now  the same reasoning as in \cite{a4} for the $A$ series, one gets the following statement.


\begin{theorem}


The lower-triangular orthogonalization of the $F_{\gamma}$ basis with respect to the order \eqref{poryadok} is a Gelfand-Tsetlin type basis $\mathcal{G}_{\gamma}$.

\end{theorem} 


To prove this statement, we consider the function $G_{\gamma}$ in the variables $A_X$ or $B_X$, representing the vector $\mathcal{G}_{\gamma}$ in A-GKZ realization. Then using the same arguments as for the $A$ series, it is shown that there exists an expression of the form
 

 
 $$
 G_{\gamma}=\sum_{s\in\mathbb{Z}_{\geq 0}^{\mathcal{K}} }d_{\gamma}^s\cdot F_{\gamma-sr}
 $$

This expression is an orthogonalization of Gram-Schmidt.
The Gram matrix for the $F_{\gamma}$ basis can be explicitly written out using the same reasoning as in \cite{a4} for the $A$ series. With its help, you can explicitly write out the transition matrix from $F_{\gamma}$ to $G_{\gamma}$. 
 


\end{fulltext}
\begin{thebibliography}{99}
	



\RBibitem{w}
\by  W.~Fulton
\book Young tableaux with applications to Representation Theory and Geometry.
\publaddr Cambridge
\publ  Cambridge University Press
\yr 1977


\RBibitem{fh}
\by  W.~Fulton,  J.~Harris
\book Representation THeory. A first course
\publaddr New York
\publ  Springer-Verlag
\yr 1991


\Bibitem{hu}
\by J.\,-S.~Huang,  C.\,-B.~Zhu
\paper Weyl's construction and tensor power  decompositions for  $G_2$
\jour Proceedings of AMS
\vol 123
\issue 3
\pages  925--934
\yr 1999


\Bibitem{bb}
\by  G.\,E.~Biedenharn,  L.\,C.~Baid
\paper On the representations of semisimple Lie Groups II
\jour J. Math. Phys.
\vol 4
\issue 12
\pages 1449--1466
\yr 1963



\Bibitem{h}
\by  W.\,J.~Holman
\paper Representation Theory of $SP(4)$ and $SO(5)$
\jour J. Math. Phys.
\vol 10
%\issue 12
\pages  1710--1716
\yr 1969

\Bibitem{h1}
\by  O.~Castanos,
E.~Chacon, M.~Moshinsky
\paper  Boson realization of $SP(4)$. $I$. The matrix formulation
\jour J. Math. Phys.
\vol 26
%\issue 12
\pages  2107--2123
\yr 1985


\Bibitem{h2}
\by  O.~Castanos,
P.~Kramer, M.~Moshinsky
\paper  Boson realization of $SP(4,R)$. $II$. The generating kernel formulation
\jour J. Math. Phys.
\vol 27
%\issue 12
\pages  924--935
\yr 1986


\Bibitem{h3}
\by   J.\,P.~Draayer, A.\,I.~Georgieva, M.\,I.~Ivanov
\paper  Deformations of the boson $sp(4,R)$ representation and its subalgebras
\jour  J. Phys. A: Math. Gen.
\vol 34
%\issue 12
\pages  2999--3014
\yr 2001


\RBibitem{zh}
\by D.\,P.~Zhelobenko
\book Compact Lie Groups and Their Representations
\publaddr Providence, R. I.
\publ  AMS
\yr 1973


\Bibitem{bf}
\by L.\,C.~Biedenharn,
D.\,E.~Flath,
\paper  On the structure of tensor operators in $SU_3$
\jour  Comm. Math. Phys.
\vol 93
\issue 2
\pages  143--170
\yr 1984

\Bibitem{bf1}
\by
D.\,E.~Flath
\paper On  $\mathfrak{so}_8$  and tensor operators of $\mathfrak{sl}_3$
\jour   Bulletin of AMS (New  Series)
\vol 1
\issue 1
\pages   97--100
\yr 1987

\RBibitem{gz}
\by
 I. M. Gel'fand, A. V. Zelevinskii
\paper Models of representations of classical groups and their hidden symmetries
\jour Funct. Anal. Appl.
\vol 18
\issue 3
\pages  183-198
\yr 1984

\RBibitem{a4}
\by
D.\,V.~Artamonov
\paper A functional realization of the Gelfand-Tselin base
\jour Izvestia RAN (submitted to publication)
%\vol 18
%\issue 3
%\pages  183--198
%\yr 1984

\Bibitem{a5}
\by D.\,V.~Artamonov
\paper Antisymmetrization of the Gelfand-Kapranov-Zelevinskij Systems
\jour J. Math. Sci.
\vol 155
\issue 5
\pages 535--542
\yr 2021	

\RBibitem{sm}
\by
D.\,V.~Artamonov
\paper Formulas for calculating the $3j$-symbols of the representations of the Lie algebra $gl_3$ for the Gelfand–Tsetlin bases
\jour Siberian Mathematical Journal
\vol 63
\issue 4
\pages 717--735
\yr 2022


\RBibitem{a3}
\by
D.\,V.~Artamonov
\paper A Gelfand–Tsetlin-type basis for the algebra $sp_4$ and hypergeometric functions
\jour Theoret. and Math. Phys.
\vol 206
\issue 3
\pages 243-257
\yr 2021


\RBibitem{a6}
\by
D.\,V.~Artamonov
\paper Functional approach to a Gelfand–Tsetlin-type basis for   $\mathfrak{o}_5$
\jour Theoret. and Math. Phys.
\vol 211
\issue 1
\pages 443-459
\yr 2022


\Bibitem{abz}
\by
 A.~Berenstein, A.~Zelevinsky
\paper Tensor product multiplicities, canonical bases and totally positive varieties
\jour  Invent.
Math.
\vol 143
\issue 1
\pages 77--128
\yr 2001


\Bibitem{ffl}
\by
E.~Feigin, G.~Fourier, P.~Littelmann
\paper  PBW filtration and bases for irreducible modules in
type $A_n$
\jour Transformation Groups
\vol 16
\issue 1
\pages 71-89
\yr 2011


\RBibitem{m}
\by  A.\,I.~Molev
\book  Yangians and classical Lie algebras 
\publaddr Providence, R. I.
\publ  AMS
\yr 2017

\RBibitem{GG}
\by
 I. M. Gel'fand, M. I. Graev, V. S. Retakh
\paper General hypergeometric systems of equations and series of hypergeometric type
\jour Russian Math. Surveys
\vol 47
\issue 4
\pages  1--88
\yr 1992

\RBibitem{rel}
\by  E.~Miller, B.~Sturmfels
\book  Combinatorial commutative algebra, in: Graduate Texts in Mathematics
\inbook  Graduate Texts in Mathematics
\vol 227
\publaddr New-York
\publ  Springer-Verlag
\yr 2005


\RBibitem{gm}
\by F.\, R. ~Gantmacher
\book   Matrix Theory
\publaddr Providence, R. I.
\publ  AMS
\yr 2000

\RBibitem{br}
\by  A.\,D.~Bruno
\book  
Power Geometry in Algebraic and Differential Equations
\publaddr North-Holland.
\publ  Elsevir
\yr 2000

\RBibitem{st}
\by   M.~Saito, B.~Sturmfels, N.~Takayama
\book   Grobner Deformations of Hypergeometric Differential Equations
\publaddr Berlin Heidelberg
\publ Springer-Verlag
\yr 2000


\end{thebibliography}
\end{document}







