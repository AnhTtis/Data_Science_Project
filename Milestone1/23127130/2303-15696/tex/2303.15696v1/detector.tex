
The \texttt{detector} module is designed to organize the data products related to a \gls{gw} detector and provides functions to create and process its internal data. %
It is formally defined as a subclass of the \texttt{Bilby} \texttt{Interferometer} class~\cite{Ashton:2018jfp}. %
In what follows, we describe the additional features we have developed, and refer the reader to the \texttt{Bilby} documentation for the built-in properties inherited from the parent class. %

By default, the \texttt{Interferometer} object is initialized by taking geometrical information of a \gls{gw} detector such as \texttt{latitude}, \texttt{longitude} and \texttt{elevation} as arguments. %
\begin{Verbatim}[commandchars=\\\{\},frame=leftline,framesep=1.5ex,framerule=0.8pt,fontsize=\small]
\PY{k+kn}{from} \PY{n+nn}{pygwb} \PY{k+kn}{import} \PY{n}{detector}
\PY{n}{my\PYZus{}detector} \PY{o}{=} \PY{n}{detector}\PY{o}{.}\PY{n}{Interferometer}\PY{p}{(}\PY{o}{*}\PY{n}{args}\PY{p}{,} \PY{o}{*}\PY{o}{*}\PY{n}{kwargs}\PY{p}{)}
\end{Verbatim}

While the above method allows the user to customize the detector's specification, one can initialize the object based on existing \gls{gw} detectors as done in \texttt{bilby}'s \texttt{Interferometer} class by calling the {\tt get\_empty\_interferometer} method. %

Once initialized, this object provides several ways to read in and process timeseries data, all of which internally call the \texttt{preprocessing} module, using information such as a channel name to query the data, or pointing to a {\tt numpy} array or a {\tt gwpy} object directly. %
Additionally, the {\tt Interferometer} object includes processing methods relying on the {\tt spectral} module described in Sec.~\ref{sec:spectral} to calculate naive and averaged spectrograms from the stored timeseries data.

A pair of {\tt Interferometer} objects can be used to initialize a {\tt Baseline} object, as described below. These are then imported as attributes of the \texttt{Baseline} object and store data products specific to each detector.
