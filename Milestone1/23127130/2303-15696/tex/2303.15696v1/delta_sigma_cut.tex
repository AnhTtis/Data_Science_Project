In general, the noise level in ground-based detectors changes slowly on time-scales of tens of minutes to hours. %\kt{In spectral, it is said that the noise is stationary on the order of a few minutes.}. 
The variance $\sigma^2_{\rm GW}$~(see Eq.~\eqref{eq:Variance}) associated to each segment is an indicator of that level of noise, which typically changes at roughly the percent level from one data segment to the next. %
However, there are occasional very loud disturbances to the detectors, such as glitches, which violate the Gaussianity of the noise. %
Auto-gating procedures are in place, as explained in Sec.~\ref{sec:preproc}, to remove loud glitches from the data; however the procedure does not remove all non-stationarities. %  
To avoid biases due to these noise events, an automated technique to exclude them from the analysis has been developed~\cite{LIGO_S4}. %
To this end, the {\tt pygwb} package includes the {\tt delta-sigma cut} module, which flags specific segments to be cut from the analyzed set. %
Note that inverse-noise-weighting, as explained in Sec. \ref{sec:postproc}, also reduces the effect of non-Gaussian noise artifacts. %

The ``\textit{$\Delta\sigma$} cut'' calculation consists in comparing the $\sigma_{\rm GW}$ of a segment $t$, $\sigma_t$, to that of its nearest neighbors and flagging it for removal in case their values differ by more than a chosen threshold. %
%, segment $I$ is removed from the analysis to avoid a bias in the point estimate, Eq.~(\ref{eq:Omega}), and $\sigma$, Eq.~(\ref{eq:Variance}). 
%The non-stationarity condition can be expressed as 
Conceptually, the calculation is based on the simple inequality,
%
\begin{equation}
\label{eq:dsc_condition}
    \frac{|\sigma_i - \sigma_{i+1}| + |\sigma_i - \sigma_{i-1}|}{2\sigma_i}>{\rm threshold}\,,
\end{equation}
%
where $i$ is a segment index. %
However, in practice we perform an analogous, more sophisticated calculation, which compares the naive and average segment variances ${\sigma}_{t, \alpha}$ and $\bar{\sigma}_{t, \alpha}$. %
These are derived from the unweighted naive and average segment variances computed with Eq.~\eqref{eq:Variance} using naive and average \glspl{psd} per segment (see Sec.~\ref{sec:spectral} for details), respectively, which are then reweighted by the index $\alpha$, as shown in Eq.~\eqref{eq:sigma_alpha}. %
The final expression employed in the calculation is 
\begin{equation}
    \frac{|\bar{\sigma}_{t, \alpha} b_{\rm avg} - \sigma_{t, \alpha} b_{\rm nav} |} {\bar{\sigma}_{t, \alpha} b_{\rm avg}}>{\rm threshold}\,,
\end{equation}
which also takes into account the bias factors that arise due to the different impacts of windowing on naive and average quantities (see App.~\ref{sec:app_window} for details). %
Past analyses have used a threshold of 0.2, as this has been shown to yield a Gaussian distribution for the remaining (un-cut) segment variances~\cite{LIGO_S5}. %
For more details on this choice see~\cite{Meyers_thesis}.
\begin{figure}
    \centering
    \includegraphics[width=0.55\linewidth]{deltasigmacut_1.pdf}
    \caption{In this plot, power-law spectra with different spectral indices are compared to the O3 sensitivity curve of \gls{ligo}-Livingston. Each power law is sensitive to a different frequency band. This makes it necessary to repeat the \textit{$\Delta\sigma$ cut} assuming different $\alpha$, since this allows to check for noise fluctuations in the whole range of frequencies analyzed. The O3 sensitivity curve for \gls{ligo}-Livingston was retrieved from \cite{O3_sens_curve}.}
    \label{fig:delta_sigma_cut_alphas}
\end{figure}

The $\Delta\sigma$ cut calculation is performed %over cross-correlated data obtained 
assuming different spectral indices $\alpha$ as each power law is sensitive to a different frequency band (see Fig.~\ref{fig:delta_sigma_cut_alphas}). %
The union of all the segments flagged for each $\alpha$ is taken, leading to a full list of segments to discard from the analysis. %
The default choice of $\alpha$ values in the {\tt delta-sigma cut} module is $ \alpha=\{-5, 0, 3\}$, as this adequately covers most of the frequency band of \gls{lvk} searches, from 20-1726 Hz~\cite{LIGO_O3}, at current sensitivity. %
These may be easily modified by the user. %
This would be especially recommended if the search were carried out over a different set of frequencies, or for data from detectors with a spectral sensitivity different than that for Advanced LIGO, Advanced Virgo, or KAGRA. %
Often, the value of $\alpha=5$ is also considered, and was employed in the most recent LVK isotropic search~\cite{O3-iso-KAGRA:2021kbb}. %
The analysis performed at a spectral index $\alpha=-5$ is mostly sensitive to non-stationary effects in the $\sim 15-50$ Hz range, while in the case of $\alpha = 0$ the analysis is sensitive to effects between $\sim 40-80$ Hz, for $\alpha = 3$ from $\sim 90-500$ Hz, and finally $\alpha = 5$ is most sensitive to fluctuations at the higher frequencies, above $\sim 500$ Hz. %
These higher frequencies are not always included in this sort of analysis due to reduced sensitivity in this range, hence $\alpha = 5$ is not a default value used for the cut.

% \footnote{Work described in the following internal reference: https://stochastic-alog.ligo.org/aLOG/index.php?callRep=339959}

As the \textit{$\Delta\sigma$} cut only compares neighboring segments, long stretches of loud noise--contaminated data can pass the test and be included in the analysis. %
We are currently working to improve this by monitoring and flagging longer stretches of non-stationary noise and prolonged loud noise conditions.
