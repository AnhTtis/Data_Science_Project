Given measurements $\{\hat \Omega_i\}$ with uncertainties $\sigma^2_i$, as shown in Sec.\ref{sec:pe} the following likelihood function can be used to perform parameter estimation on the \gls{gwb}:
	\begin{equation}
    \label{eq:likelihood-again}
    p(\{\hat \Omega_f\} | {\bm \Theta})
    	= \mathcal{N} \exp\left[
        	-\frac{1}{2}\sum_f\frac{\left(\hat \Omega_f -  \Omega_{\rm M}(f|{\bm \Theta})\right)^2}{\sigma_f^2}\right].
    \end{equation}
Here, the $\{\hat \Omega_f\}$ are a set of estimators for the \gls{gw} energy density at discrete frequencies $f$, $\Omega_{\rm M}(f|{\bm \Theta})$ is a model for the energy density with parameters ${\bm \Theta}$, and $\mathcal{N}$ is a normalization constant.
We will consider only a single baseline and neglect the sum over detector pairs $IJ$ appearing in Eq.~\eqref{eq:likelihood}; if multiple detector pairs exist, the derivation below can be replicated for each pair.

Eq.~\eqref{eq:likelihood-again} assumes that our estimators $\{\hat \Omega_f\}$ are direct, unbiased measurements of the underlying energy-density spectrum.
In general, however, the imperfect amplitude and phase calibration of \gls{gw} detectors will break this assumption.
We can account for calibration uncertainty by amending our likelihood to introduce a new parameter $\lambda$:
	\begin{equation}
    \label{eq:likelihood-calibration-uncertainty}
    p(\{\hat \Omega_f\} | {\bm \Theta},\lambda)
    	= \mathcal{N} \exp\left[
        	-\frac{1}{2}\sum_f\frac{\left(\hat \Omega_f -  \lambda\Omega_{\rm M}(f|{\bm \Theta})\right)^2}{\sigma_f^2}\right].
    \end{equation}
The parameter $\lambda$ is an unknown multiplicative factor that encapsulates potential calibration inaccuracy.
In the case of perfect amplitude calibration ($\lambda=1$), then $\{\hat \Omega_f\}$ are direct measurements of the underlying (unknown) energy spectrum.
But if our calibration is imperfect ($\lambda\ne1$), then $\{\hat \Omega_f\}$ are instead measurements of some multiple $\lambda \Omega(f)$ of the \gls{gwb} spectrum.
Although we do not know $\lambda$, it is possible to estimate the \textit{uncertainty} on instrumental calibration.
We will therefore model $\lambda$ itself as an unknown variable drawn from a normal distribution centered at 1 (corresponding to perfect calibration) but with a variance $\epsilon^2$:
	\begin{equation}
    p(\lambda) \propto 
    	\exp\left[-\frac{1}{2\epsilon^2}\left(\lambda-1\right)^2\right],
    \end{equation}
where $\epsilon$ is a known amplitude calibration uncertainty. 
Additionally, we impose the constraint that $\lambda$ be positive: we expect errors in the amplitude of strain measurements but not their \textit{sign}.
In this case, the probability distribution for $\lambda$ becomes
	\begin{equation}
    \label{eq:plambda}
    p(\lambda) = \sqrt{\frac{2}{\pi}}
  		\frac{1}{\epsilon\left[1
        	+\mathrm{Erf}(\frac{1}{\sqrt{2\epsilon^2}})\right]}
        \exp\left[-\frac{1}{2\epsilon^2}\left(\lambda-1\right)^2\right],
    \end{equation}
normalized to unity on the interval $\lambda\in(0,\infty)$.
Eq. \eqref{eq:plambda} is our prior on $\lambda$.

We can now use Eq. \eqref{eq:plambda} to marginalize our likelihood (Eq.~\eqref{eq:likelihood-calibration-uncertainty}) over the unknown calibration factor $\lambda$.
The marginalized likelihood is given by
	\begin{equation}
    \begin{aligned}
    p(\{\hat \Omega_f \} | {\bm \Theta} )
        &= \int p(\{\hat \Omega_f \} | {\bm\Theta},\lambda) \,p(\lambda) d\lambda \\
        &= \mathcal{N} \sqrt{\frac{2}{\pi}}
  			\frac{1}{\epsilon\left[1
        		+\mathrm{Erf}(\frac{1}{\sqrt{2\epsilon^2}})\right]}
            \int_0^\infty \exp\left[
            	-\frac{1}{2}\sum_f \frac{\left(\hat \Omega_f - \lambda\Omega_{\rm M}(f|{\bm \Theta})\right)^2}{\sigma^2_f}
                -\frac{1}{2}\frac{\left(\lambda-1\right)^2}{\epsilon^2}
                \right] d\lambda.
    \end{aligned}
    \end{equation}
If we define 
	\begin{equation}
    A({\bm \Theta}) = \frac{1}{\epsilon^2}+\sum_f\frac{\Omega_{\rm M}(f|{\bm \Theta})^2}{\sigma^2_f},
    \end{equation}
    \begin{equation}
    B({\bm \Theta}) = \frac{1}{\epsilon^2}+\sum_f\frac{\hat \Omega_f \Omega_{\rm M}(f|{\bm\Theta})}{\sigma^2_f},
    \end{equation}
and
	\begin{equation}
    C({\bm \Theta}) = \frac{1}{\epsilon^2}+\sum_f\frac{\hat \Omega^2_f}{\sigma^2_f},
    \end{equation}
the marginal likelihood can be more concisely expressed as
	\begin{equation}
    \label{eq:likelihood-calib-2}
    p(\{\hat \Omega_f \} | {\bm \Theta} )
    	= \mathcal{N} \sqrt{\frac{2}{\pi}}
  			\frac{1}{\epsilon\left[1 +\mathrm{Erf}(\frac{1}{\sqrt{2\epsilon^2}})\right]}
            \int_0^\infty \exp\left[-\frac{1}{2}\left(
            	A({\bm \Theta})\lambda^2 - 2B({\bm \Theta})\lambda + C({\bm \Theta})
                \right)\right] d\lambda;
    \end{equation}
this expression can be analytically integrated to obtain
	\begin{equation}
    p(\{\hat \Omega_f \} | {\bm \Theta} ) =
    	\mathcal{N} \frac{1}{\epsilon\sqrt{A({\bm \Theta})}}
        \left[\frac{1+\mathrm{Erf}(\frac{B({\bm \Theta})}{\sqrt{2A({\bm \Theta})}})}
        	{1+\mathrm{Erf}(\frac{1}{\sqrt{2\epsilon^2}})}\right]
        \exp\left[-\frac{1}{2}\left(C({\bm \Theta})-\frac{B({\bm \Theta})^2}{A({\bm \Theta})}\right)\right].
    \end{equation}

Marginalization of calibration uncertainty is built into the {\tt pygwb\_pe} module, and this calculation is automatically triggered when passing a calibration error $\epsilon\neq 0$. Additional information on the treatment of calibration uncertainties can be found in \cite{Whelan:2012ur}.