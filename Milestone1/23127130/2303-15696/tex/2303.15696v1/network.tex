

The {\tt network} module is designed to handle two different tasks. %
Its primary purpose is to combine results from different {\tt Baseline} objects. %
Similarly to the {\tt Baseline} object, the {\tt Network} object imports {\tt Baseline} objects as attributes which may be invoked through the {\tt Network}. %
In addition to this functionality, it can also be used to simulate cross-correlated data across a network of detectors. %
Both signal-only and signal and noise data can be simulated using a {\tt Network} object. %
The {\tt network} module handles all data generation by querying the {\tt simulator} module. %

The {\tt Network} object can be initialized in two ways. %
By default it is initialized through a list of {\tt Interferometer} objects. 
\begin{Verbatim}[commandchars=\\\{\},frame=leftline,framesep=1.5ex,framerule=0.8pt,fontsize=\small]
\PY{k+kn}{from} \PY{n+nn}{pygwb} \PY{k+kn}{import} \PY{n}{network}
\PY{n}{HLV\PYZus{}network} \PY{o}{=} \PY{n}{network}\PY{o}{.}\PY{n}{Network}\PY{p}{(}\PY{l+s+s1}{\PYZsq{}}\PY{l+s+s1}{HLV}\PY{l+s+s1}{\PYZsq{}}\PY{p}{,} \PY{p}{[}\PY{n}{H1}\PY{p}{,} \PY{n}{L1}\PY{p}{,} \PY{n}{V1}\PY{p}{]}\PY{p}{)}
\end{Verbatim}

It is also possible to initialize a {\tt Network} using a list of {\tt Baseline} objects, to streamline the combination of results from different baselines which already contain final data products. 
\begin{Verbatim}[commandchars=\\\{\},frame=leftline,framesep=1.5ex,framerule=0.8pt,fontsize=\small]
\PY{n}{HLV\PYZus{}network} \PY{o}{=} \PY{n}{network}\PY{o}{.}\PY{n}{Network}\PY{o}{.}\PY{n}{from\PYZus{}baselines}\PY{p}{(}\PY{l+s+s1}{\PYZsq{}}\PY{l+s+s1}{HLV}\PY{l+s+s1}{\PYZsq{}}\PY{p}{,} \PY{p}{[}\PY{n}{HL\PYZus{}baseline}\PY{p}{,} \PY{n}{HV\PYZus{}baseline}\PY{p}{,} \PY{n}{LV\PYZus{}baseline}\PY{p}{]}\PY{p}{)}
\end{Verbatim}



The combined point estimate and sigma spectra are stored as attributes of the {\tt Network}. %
These are combined by performing an inverse-noise--weighted average over the individual {Baseline} final spectra, % 
assuming the data are uncorrelated between baselines, i.e., assuming each baseline provides independent information. %
This is a valid approximation when working in the large noise limit. %
Further details can be found in \cite{Allen:1999}.

%To conclude, we highlight one more functionality of the {\tt network}. %
The {\tt Network} is also designed to produce appropriately correlated simulated data for a network of interferometers. %
The {\tt Network} can either simulate data from scratch, or add simulated data to pre-existing data, if the interferometers used to initialize the object contain strain data. %
The latter is simply done as strain adds coherently in the time domain. %
This functionality relies on the {\tt simulator} module which performs the data simulation, as discussed in Sec. \ref{Sec:Simulator}. 

\iffalse
{\footnotesize{
\begin{minted}[mathescape, linenos]{python}
HLV_network.set_interferometer_data_from_simulator(Intensity_GW, N_segments=7, inject_into_data_flag=True)
\end{minted}
}}
\fi

