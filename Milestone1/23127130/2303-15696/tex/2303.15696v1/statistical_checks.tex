With the {\tt statistical\_checks} module, we provide a tool to perform initial statistical analyses of a {\tt pygwb} run result set, and visualize them in pre-formatted plots. %
We identify five broad categories of checks. %

The first set calculates the running point estimate for $\hat\Omega_{\rm ref}^\alpha$ and $\sigma_{\rm ref}^\alpha$ quantities as a function of time, as more data segments are added to the analysis. %
The values of $\alpha$ and $f_{\rm ref}$ are those used in the analysis and may not be changed at this point. %
The running averages are cumulative weighted averages of time--ordered segments, and do not take segment-by-segment correlation into account. % 
In case of detection, these converge to a biased point estimate and $\sigma$, as proper postprocessing is not applied (see Sec.~\ref{sec:postproc}). %
However, the visualization of running quantities is extremely useful to identify trends in the data, and ultimately will flag a possible detection. %
The module also provides a linear trend analysis, fitting the evolution of the parameters described above as a function of time. %

The second set focuses on the \gls{snr} spectrum as a function of frequency, defined as
\begin{equation}
    {\rm SNR}_f = \frac{\hat\Omega^\alpha_{{\rm ref}, f}}{\sigma^\alpha_{{\rm ref}, f}}\,.
\end{equation}
The absolute value, real, and imaginary part of the \gls{snr} are calculated, as well as the cumulative \gls{snr}. %
An example of these plots using the first sub-set of O3 data further described in Sec. \ref{Sec:O3Data} is given in Fig.~\ref{fig:FreqPlots}. %
These plots are a faithful representation of the ``noisiness'' of each frequency bin and how much each bin contributes to the analysis. %

\begin{figure}[t]
    \centering
    \includegraphics[width=0.47\textwidth]{statchecks_1.png}
    \includegraphics[width=0.47\textwidth]{statchecks_2.png}
    \caption{\textbf{Left}: Absolute value of the \gls{snr} spectrum as a function of frequencies. \textbf{Right}: Sigma spectrum as a function of frequency.}
    \label{fig:FreqPlots}
\end{figure}
\begin{figure}[h!]
    \centering
    \includegraphics[width=.43\textwidth]{statchecks_3.png}
    \raisebox{0.575\height}{\includegraphics[width=.43\textwidth]{statchecks_4.png}}
    \caption{\textbf{Left}: Point estimate, sigma and deviates $\Delta {\rm SNR}_i$ as a function of time before the delta-sigma cut (red) and after the cut (blue). \textbf{Right}: Distribution of the deviates $\Delta {\rm SNR}_i$ as a function of time before the delta-sigma cut (red) and after the cut (blue).}
    \label{fig:DscExamplePlots}
\end{figure}
The third set of checks produces the \gls{ift} of the point estimate spectrum, which should peak around zero seconds in case of a detection. 
Time-shifting the data in two detectors by more than the coherence time between the two detectors breaks the coherence between the two data streams, removing any evidence of a GWB signal. Note that the coherence time is determined by the bandwidth of our signal, which is of order 100 Hz, resulting in a coherence time of 10 ms. %
Hence, a \gls{gwb} signal will only peak around zero time lag between the output of the detectors. %

The fourth set studies the effect of the $\Delta\sigma$ data quality cut described in Sec. \ref{Sec:DeltaSigma} on the analysis run. %
To this end, we display several quantities before and after the cut is applied to the data, including the segment values of $\hat\Omega^\alpha_{{\rm ref}, i}$, $\sigma^\alpha_{{\rm ref}, i}$, and $\Delta\sigma^\alpha_{{\rm ref}, i}$, and the deviations in SNR,
\begin{equation}
    \Delta {\rm SNR}_i = \frac{\hat\Omega^\alpha_{{\rm ref}, i}-\langle\hat\Omega^\alpha_{{\rm ref}}\rangle}{\sigma^\alpha_{{\rm ref}, i}} \,,  
\end{equation}
as a function of time. %
Here angle brackets indicate an arithmetic mean over all segments $i$. %
We also plot a histogram of the values of $\Delta {\rm SNR}$ before and after the cut. %
This distribution should be centred around $0$, with a smooth narrower distribution after the application of the $\Delta\sigma$ cut. %
We additionally plot the $\Delta {\rm SNR}$ as a function of individual $\sigma^\alpha_{{\rm ref}, i}$. %
Finally, we plot the distribution of the ratios $\sigma^2_{{\rm ref}, i}/\langle\sigma^2_{{\rm ref}, i}\rangle$, which should peak around $1$. %
Some representative plots are shown as an example in Fig.~\ref{fig:DscExamplePlots}.



The last set of checks concerns a \gls{ks} test that is used to verify that the $\Delta {\rm SNR}_i$ are consistent with a Gaussian distribution. %
The \gls{ks} test implementation of this module returns the \gls{ks} test statistic, which is the maximal deviation from the Gaussian cumulative distribution function, as well as the p-value. %
These values can be used to make statements about the Gaussianity of the data \cite{KStestRef}. In addition, the cumulative distribution function is plotted for the data as well as for a Gaussian distribution. %