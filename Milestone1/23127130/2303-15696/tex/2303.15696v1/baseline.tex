The {\tt Baseline} module is by design the core of the {\tt pygwb} stochastic analysis. %
Its main role is to manage the cross-correlation between {\tt Interferometer} data products, combine these into a single cross-spectrum, which represents the point estimate of the analysis, and calculate the associated error, as introduced in Sec.~\ref{sec: GWB analysis}.

The standard initialization of a {\tt Baseline} object simply requires a pair of {\tt Interferometer} objects. %
\begin{Verbatim}[commandchars=\\\{\},frame=leftline,framesep=1.5ex,framerule=0.8pt,fontsize=\small]
\PY{k+kn}{from} \PY{n+nn}{pygwb} \PY{k+kn}{import} \PY{n}{baseline}
\PY{n}{H2H2\PYZus{}baseline} \PY{o}{=} \PY{n}{baseline}\PY{o}{.}\PY{n}{Baseline}\PY{p}{(}\PY{l+s+s2}{\PYZdq{}}\PY{l+s+s2}{H1\PYZhy{}H2}\PY{l+s+s2}{\PYZdq{}}\PY{p}{,} \PY{n}{H1}\PY{p}{,} \PY{n}{H2}\PY{p}{)}
\end{Verbatim}

Here {\tt H1} and {\tt H2} are {\tt Interferometer} objects. % 
It is also possible to load a previously stored {\tt Baseline} object in {\tt pickle} format by calling the relevant class method. %

The data loaded into the {\tt Interferometer} objects are automatically imported into the {\tt Baseline} object upon initialization. % 
The {\tt Baseline} object relies on the {\tt spectral} module to calculate cross-correlations between the data streams, following the methodology shown in Sec.~\ref{sec:spectral}. %
Similarly, it relies on the {\tt postprocessing} module to obtain the point estimate $\hat{\Omega}^\alpha_{\rm ref}$ and its variance $\sigma^\alpha_{\rm ref}$, as described in Eqs.~(\ref{eq:ptest_postpoc}--\ref{eq:sigma_postpoc}). %
The user may choose to calculate point estimate and sigma spectra or point values; in the latter case, the spectra are automatically stored to facilitate subsequent analyses. %

Calculating $\hat{\Omega}^\alpha_{\rm ref}$, as well as performing parameter estimation on the GWB spectrum, requires the two-detector \gls{orf}, $\gamma_{IJ}$, shown in Eq.~\eqref{eq:orf}. %
The \gls{orf} is calculated at {\tt Baseline} object initialization, then stored as an attribute. %
By default, we assume \gls{gr}, which presents two independent degrees of freedom for the strain field, typically $A=\{+, \times\}$ in the transverse-traceless gauge. For a precise derivation of this function and detector response definitions, see for example~\cite{Romano_2017}. %

The {\tt Baseline} object is also equipped to probe circularly polarized backgrounds~\cite{Seto:2007tn}, and non-\gls{gr} polarizations in the \gls{gwb}, such as scalar and vector backgrounds~\cite{TeVeS}. %
This requires selecting a different choice of $A$, according to the chosen polarization type, which can be declared when calculating $\hat{\Omega}^\alpha_{\rm ref}$ or the \gls{orf} directly. %
Details on the expressions for non-\gls{gr} $\gamma_{IJ}$ functions may be found in the appendix of~\cite{TeVeS}.


