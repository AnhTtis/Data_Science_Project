The role of the {\tt spectral} module is to compute, for each time segment of duration $T$, the discrete frequency domain quantities $C_{IJ, f}$, $P_{I, f}$ and $P_{J, f}$ used in Eqs.~\eqref{eq:Omega} and \eqref{eq:Variance}. The one-sided cross- and auto-power spectral densities $C_{IJ}$ and $P_I$, respectively, of a single segment are defined as
\begin{equation} \label{eq:csd_psd}
    C_{IJ, f} = \frac{2}{T} \tilde s_{I, f}^* \tilde s_{J, f}\,, \ \ \ \ P_{I,f} = \frac{2}{T} |\tilde s_{I, f}|^2 \ ,
\end{equation}
where $\tilde s_f$ are \glspl{dft} of $s(t_k)$ %$t_k = 0,\,\delta t, \, 2 \times \delta t, \,\cdots, \, T-\delta t$, 
 defined by
\begin{equation}
   \tilde s_{f} \equiv \sum_{t_k=0}^{T-\delta t} s(t_k) \, e^{-i 2 \pi m t_k / T} \,,
\end{equation}
where %$f = 0,\, \delta f,\, 2 \times \delta f,\, \cdots,\, \frac{1}{2 \times \delta t}$. $f = m\,\delta f$, where
$f=m \delta f$, with $m$ a natural number between 0 and $1/(2\,\delta t \,\delta f)$, and $\delta f$ the desired frequency resolution, chosen such that $1/(2\delta t \delta f)$ is an integer. %

The segmented data are windowed before calculating Fourier transforms to avoid spectral leakage due to discontinuities at the ends of the segments. %
The user may define their own choice of window, which defaults to the \textit{Hann} window if none is selected. %
The {\tt spectral} module uses methods from {\tt scipy.signal} to calculate spectrograms $\tilde s^t_{f}$ of the given data, which are then used to calculate the list of $C^t_{IJ, f}$, $P^t_{I, f}$, and $P^t_{J, f}$ quantities, corresponding to different time segments labelled by $t$ in the dataset. %
By default, these are calculated with a 50\% time overlap to account for the impact of the windowing. However, the user may redefine the overlap between consecutive segments to be used throughout the analysis to better suit any choice of window. %~\arr{style comment: some sections don't have separation between paragraphs. Maybe we should decide upon one style or the other.}%

Different averaging procedures are employed to reduce the fluctuations in the spectra estimates and compress the data. %
The procedures we employ are selected to minimize sensitivity loss. %
In the estimates of $P^t_{I, f}$ and $P^t_{J, f}$ %, instead of performing a DFT on the entire segment of data of duration $T$, 
we employ Welch's estimation method of \glspl{psd}~\cite{pwelch}, which is known to produce minimum variance estimates of the \gls{psd}, implemented as follows. %
Each segment is divided into sub-segments of duration $1/\delta f$ which are \gls{dft}ed individually. %
The auto-correlated power $|\tilde s_{I,f}|^2$ is then averaged over the sub-segments %to get $|s_I(f)|^2$ for the whole segment which is then used in Eq.~\ref{eq:csd_psd}
to obtain estimates of $P^t_{I, f}$ and $P^t_{J, f}$ for time $t$. %
This procedure returns spectra at the desired frequency resolution $\delta f$, which is typically much larger than the original resolution $1/T$ Hz. %
%\air{it would be good to mention the overlap factor here.}

As the power varies slowly with frequency\footnote{The power varies slowly with frequency except in very few bins, where narrow-band spectral artifacts or {\it lines} are present, as discussed in Sec.~\ref{sec:notch}.}, we can average over neighboring frequencies using a process known as coarse-graining~\cite{https://doi.org/10.48550/arxiv.2106.13785}. %
This is the default procedure employed in the \gls{csd} estimation. %
The resulting spectra are returned at the desired frequency resolution $\delta f$. %, we average $s_I^*(f) s_J(f)$ over neighboring frequency bins to get the desired frequency resolution of $\delta f$. %
Note that the data are zero-padded before calculating Fourier transforms for $C^t_{IJ, f}$ to avoid wrap-around problems arising from finite data \cite{LIGO_S1, whelan_CC_dcc, numerical_recipes2007}, and hence coarse-graining is required to achieve the desired frequency resolution. %
% coarse-graining mostly arises due to the zero-padding which is needed to avoid the wrap-around problems and needs to be done before taking the Fourier transform.
Zero-padding simply entails appending a vector of zeros equal to the length of the segment before taking the Fourier transform. %


\begin{figure}
    \centering
    \includegraphics[width=0.5\textwidth]{spectral_1.pdf}
    \caption{An example of cross- and auto-power spectral densities of the Hanford and Livingston detector data during O3.}
    \label{fig:psd_csd_spectra}
\end{figure}

\begin{figure}
    \centering
    \includegraphics[width=0.7\textwidth]{spectral_2.png}
    \caption{An example spectrogram showing two hours of LIGO   Hanford data during O3. The visible vertical columns correspond to noisy segments, which are usually removed from the analysis (see Sec.~\ref{Sec:DeltaSigma}).}
    \label{fig:psd_spectrogram}
\end{figure}
%Depending on the user input, it is possible to further average $P_I(f)$ and $P_J(f)$ over estimates from neighboring segments.
To further reduce fluctuations in the \gls{psd} estimates, the $P^t_{I, f}$ quantities are averaged over neighboring segments to obtain the final estimate $\bar{P}^t_{I, f}$ of the \gls{psd} at a given time $t$. %
This is appropriate as the noise in \gls{gw} detectors is (most often) approximately stationary over periods of a few minutes. %
We often refer to the initial (un-barred) quantities as ``naive'' and the final (barred) quantities as ``average'' estimates in the rest of this paper, to avoid confusion. %
By default, only nearest neighbors are used for the calculation, such that the \gls{psd} at time $t$ is an average of the naive \glspl{psd} calculated for times $t-T$ and $t+T$. %
The user may define any even number $D$ of segments to be used to perform this average, which are taken before and after the reference time $t$ such that the \gls{psd} is averaged over naive \glspl{psd} at times $t-D T/2$ and $t+DT/2$. % %

Fig.~\ref{fig:psd_csd_spectra} shows the cross- and auto-power \glspl{psd} of 192 s of data from the Hanford and Livingston detectors during O3, while Fig.~\ref{fig:psd_spectrogram} shows a two-hour spectrogram of Hanford data during O3, produced with the {\tt spectral} module. % 
