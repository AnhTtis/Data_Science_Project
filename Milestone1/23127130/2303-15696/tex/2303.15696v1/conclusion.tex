We present a new Python--based package tailored to \gls{gwb} searches with current ground-based interferometers. %
We opt for a modular code, where each module performs specific tasks of the \gls{gwb} data analysis. %
The modularity of the code results in large flexibility and offers the possibility to customize the pipeline according to one's own needs. %
With the use of Python language, the user-friendliness and flexibility of the code, we aim to bring \gls{gwb} searches to the wider \gls{gw} community, as the detection of a \gls{gwb} with ground-based interferometers draws potentially closer. %
With increasing amounts of \gls{gw} data, {\tt pygwb} also answers the need for an open-source \gls{gwb} data analysis tool. %

In this paper, we show the application of the {\tt pygwb} package to mock datasets, illustrating how the various modules can be assembled to form a search pipeline, and showing what a \gls{gwb} detection could look like with our analysis approach. %
To conclude, we run the {\tt pygwb} pipeline on real \gls{gw} data from the third observing run (O3) of the \gls{lvk} collaboration, and recover results in agreement with published results. %
Both analyses serve as a validation of the software.

The {\tt pygwb} package is designed to evolve along the way and address new analysis needs as they arise. %
This is facilitated by the structure and the format of {\tt pygwb}, and the management of the online Git repository. %
The {\tt pygwb} team invites input from the broader community, under the form of Git issues and pull requests. %
New contributors to the code are always welcome. %
Official updates and releases of the code will be handled and reviewed internally by the software and review teams, which are due to evolve. %  

We are aware other analysis methodologies exist which accommodate different features of specific \glspl{gwb}, such as potential anisotropy~\cite{Ain:2018zvo}, and the intermittency of the \gls{bbh} background~\cite{Smith:2017vfk, Lawrence:2023buo}. %
We look forward to interfacing with these methods and, where useful and appropriate, improving the current codebase to support and encompass more analysis schemes.

Finally, we are particularly excited at the prospect of broadening the scope of the package to include support for next generation detectors such as \gls{et}~\cite{ET_science} and \gls{ce}~\cite{CE_science}. %
While the science cases and design properties of these detectors are still under development, there is evident interest in targeting \glspl{gwb} with these detectors within the community~\cite{ET-first-MDC-PhysRevD.86.122001, ET-second-MDC-PhysRevD.89.084046, Sathyaprakash:2011bh}, and a notable increase in sensitivity compared to present-day interferometers is expected.  

\section*{Acknowledgements}

The author list of this paper includes, in order: all {\tt pygwb} code authors, in order of successful GitLab merge requests, at time of writing; %
code reviewers and testers, in alphabetical order. %

We would like to thank the LVK stochastic group for its continued support. %
Special thanks to G. Cella and J. Suresh for valuable comments on the manuscript. %useful discussions and validations. %
%PnP reviewers
%Andrew Miller

%{\color{red} add funding info here (author list order):}
AIR is supported by the NSF award 1912594.
ARR is supported in part by the Strategic Research Program “High-Energy Physics” of the Research Council of the Vrije Universiteit Brussel and by the iBOF “Unlocking the Dark Universe with Gravitational Wave Observations: from Quantum Optics to Quantum Gravity” of the Vlaamse Interuniversitaire Raad and by the FWO IRI grant I002123N “Essential Technologies for the Einstein Telescope”. %
KT is supported by FWO-Vlaanderen through grant number 1179522N. %
PMM is supported by the NANOGrav Physics
Frontiers Center, National Science Foundation (NSF), award number 2020265.
LT is supported by the National Science Foundation through OAC-2103662 and PHY-2011865.
KJ is supported by FWO-Vlaanderen via grant number 11C5720N. %
F.D.L. is supported by a FRIA Grant of the Belgian Fund for Research, F.R.S.-FNRS. %
JL was supported by NSF Award PHY-2207270. %
DD is supported by the NSF as a part of the LIGO Laboratory
AM is supported by the European Union’s Horizon 2020 research and
innovation programme under the Marie Skłodowska-Curie grant agreement No. 754510. %
JDR was supported in part by NSF Award PHY-2207270 and start-up funds provided by Texas Tech University. VM was supported in part by the NSF award PHY-2110238.%

This material is based upon work supported by NSF’s LIGO Laboratory 
which is a major facility fully funded by the 
National Science Foundation.
LIGO was constructed by the California Institute of Technology 
and Massachusetts Institute of Technology with funding from 
the National Science Foundation, 
and operates under cooperative agreement PHY-1764464. 
Advanced LIGO was built under award PHY-0823459.
The authors also gratefully acknowledge the support of the Science
and Technology Facilities Council (STFC) of the United Kingdom, the Max-Planck-Society (MPS), and the
State of Niedersachsen/Germany for support of the construction of Advanced LIGO and construction and
operation of the GEO 600 detector. Additional support for Advanced LIGO was provided by the Australian
Research Council. 
The authors gratefully acknowledge the Italian Istituto Nazionale di Fisica Nucleare
(INFN), the French Centre National de la Recherche Scientifique (CNRS) and the Netherlands Organization for Scientific Research (NWO), for the construction and operation of the Virgo detector and the creation
and support of the EGO consortium. The authors also gratefully acknowledge research support from these
agencies as well as by the Council of Scientific and Industrial Research of India, the Department of Science
and Technology, India, the Science \& Engineering Research Board (SERB), India, the Ministry of Human
Resource Development, India, the Spanish Agencia Estatal de Investigaci\'on (AEI) the Spanish Ministerio de Ciencia e Innovaci\'on and Ministerio de Universidades, the Conselleria de Fons Europeus, Universitat i Cultura and the Direcci\'o General de Pol\'itica Universitaria i Recerca del Govern de les Illes Balears, the Conselleria d’Innovaci\'on, Universitats, Ci\'encia i Societat Digital de la Generalitat Valenciana and the CERCA Programme Generalitat de Catalunya, Spain, the National Science Centre of Poland and the European Union – European Regional Development Fund; Foundation for Polish Science (FNP), the Swiss
National Science Foundation (SNSF), the Russian Foundation for Basic Research, the Russian Science
Foundation, the European Commission, the European Social Funds (ESF), the European Regional Development Funds (ERDF), the Royal Society, the Scottish Funding Council, the Scottish Universities Physics
Alliance, the Hungarian Scientific Research Fund (OTKA), the French Lyon Institute of Origins (LIO), the
Belgian Fonds de la Recherche Scientifique (FRS-FNRS), Actions de Recherche Concertées (ARC) and
Fonds Wetenschappelijk Onderzoek – Vlaanderen (FWO), Belgium, the Paris ˆIle-de-France Region, the
National Research, Development and Innovation Office Hungary (NKFIH), the National Research Foundation of Korea, the Natural Science and Engineering Research Council Canada, Canadian Foundation for
Innovation (CFI), the Brazilian Ministry of Science, Technology, and Innovations, the International Center
for Theoretical Physics South American Institute for Fundamental Research (ICTP-SAIFR), the Research
Grants Council of Hong Kong, the National Natural Science Foundation of China (NSFC), the Leverhulme
Trust, the Research Corporation, the Ministry of Science and Technology (MOST), Taiwan, the United States
Department of Energy, and the Kavli Foundation. The authors gratefully acknowledge the support of the
NSF, STFC, INFN and CNRS for provision of computational resources
The authors are grateful for computational resources provided by the LIGO Laboratory and supported by NSF Grants PHY-0757058 and PHY-0823459. 
This work carries LIGO document number P2300048. %