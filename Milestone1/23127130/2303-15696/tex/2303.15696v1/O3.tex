
In this section we present results from the application of the {\tt pygwb} analysis suite to the full \gls{ligo} Hanford and \gls{ligo} Livingston O3 dataset~\cite{gwosc}. %
We set upper limits on the signal from a \gls{sgwb} and confirm these are consistent with previously published collaboration results~\cite{LIGO_O3}.

The O3 data run collected between April 1, 2019 and March 27, 2020, divided into two sub-sets with an interruption between October 1 and November 1, 2019, with a total coincident livetime of 205.4 days between \gls{ligo} Hanford and \gls{ligo} Livingston. %
These are reduced to 196 days after {\it category 1 vetoes}\footnote{``Category 1'' vetoes flag data which are unsuitable for analysis, such as incorrectly calibrated data, data collected during atypical operation of the instruments, and data with severe data quality issues.}~\cite{Virgo:2022kwz, Abbott_2018} and external non-stationarity cuts are applied (for details, see~\cite{LIGO_O3}). %
The {\tt pygwb} analysis is implemented with the workflow described above. %
The O3 data, natively sampled at 16384 Hz, are downsampled to 4096 Hz and high-pass filtered at 11 Hz. %

The time-averaged O3 \gls{ligo} Hanford -- Livingston cross-correlation spectrum is presented in Fig.~\ref{fig:O3_crosscorr_spectrum}. %
Our $\Delta\sigma$ threshold excludes 7.8\% of the analyzed time (see Sec.~\ref{Sec:DeltaSigma} for implementation details). %
This result matches the previous stochastic non-stationarity cut published in~\cite{LIGO_O3} within 1\%, with the previous cut excluding an extra 0.06\% of the time. %
We believe this small variation to be due to a different window bias factor used in the two analyses (the bias factor calculation used here is outlined in App.~\ref{sec:app_window}). %

We calculate broad-band integrated estimates between $20 - 1726$ Hz of $\Omega_{\rm GW}(f_{\rm ref})$ for different power-law spectral models, applying the released O3 notchlist~\cite{O3IsotropicDataset} to exclude known problematic frequencies~\cite{Covas_2018}. %
A summary of the values for the point estimate and uncertainty for these is presented in Table~\ref{tab:O3_results}. %
The uncertainties $\sigma^\alpha_{\rm ref}$ agree within $1\%$ with previously published LVK results, presented in~\cite{LIGO_O3}. %
The point estimates for $\Omega^\alpha_{\rm ref}$ fluctuate notably more than the uncertainties. %
We believe this to be due to small differences in the analyses, to which the point estimates are more sensitive, such as individual start and end time of each pipeline job, and the differences in the non-stationarity cuts described above. %

Finally, we perform parameter estimation to constrain $\Omega_{\rm GW} (f_{\rm ref}= 25 \text{Hz})\equiv \Omega_{25}$ and the spectral index $\alpha$ with O3 data. %
We employ a log-uniform prior on $\Omega_{25}$ spanning $[10^{-13},10^{-6}]$, and present results for two different priors on $\alpha$: a uniform prior between $[-4, 4]$ and a Gaussian prior centered around 0 with norm 3.5 (the latter matches the choice in ~\cite{LIGO_O3}). %
To account for calibration uncertainty, we marginalize over the uncertainty parameter $\lambda$ as described in App.~\ref{sec:app_calibration}, with combined calibration error for Hanford and Livingston of $1.48\%$, as in~\cite{LIGO_O3}. %
Parameter estimation confirms $\Omega_{25}$ is consistent with 0 and $\alpha$ remains unconstrained, as may be seen in Fig.~\ref{fig:O3_pe}. % 
These results agree with the previous parameter estimation carried out in~\cite{LIGO_O3}.

\begin{figure}[t]
    \centering
    \includegraphics[width=0.75\textwidth]{O3_1.png}
    \caption{Estimated cross-correlation spectrum $\hat{\Omega}^0_{25}\pm \hat{\sigma}^0_{25}$ from O3 data. By eye, it is possible to spot several narrowband artifacts (lines) which are subsequently excluded from our analysis.}
    \label{fig:O3_crosscorr_spectrum}
\end{figure}

Our results are quoted at the value of the Hubble parameter $H_0 = 67.9$ km/s/Mpc, in line with published results. %
This is not the built-in value of $H_0$, defined in Sec.~\ref{sec:postproc}; however, rescaling is straightforward as it is an overall multiplication factor, which may be changed when post-processing the run with the {\tt pygwb$\_$combine} script or manually using the built-in functions of the {\tt OmegaSpectrum} object, as explained in Sec.s~\ref{sec:postproc} and~\ref{sec: pipeline}.

We would like to note that this entire analysis was carried out on a large computing cluster and completed in less than five hours of human time. %
This is an example of the computational efficiency of our package.

\begin{table}[h]
    \centering
    \begin{tabular}{c|c|c||c|c|c}
       ~$\alpha$~  &  ~$\hat{\Omega}^\alpha_{25}\times 10^9$~ & ~$\hat{\Omega}_{\rm LVK}\times 10^9$ & ~prior~ & ~$\Omega_{\rm pe}$ ($95\%$ UL) ~& ~$\alpha_{\rm pe}$~\\
       \hline 
       \hline 
       
         0 & $-3.4\pm 8.1$& $-2.1\pm 8.2$ & {\sc uniform} & $5.44\times 10^9$ & $-0.8^{+2.8}_{-2.2}$\\
         2/3 & $-4.5\pm 6.1$& $-3.4\pm 6.1$ & ~{\sc gaussian}~ & $4.06\times 10^9$ & $-0.5^{+2.8}_{-2.8}$ \\
         3 & $-1.5\pm 0.9$& $-1.3\pm 0.9$ & & 
    \end{tabular}
    \caption{Summary of {\tt pygwb} search results on O3 dataset. On the left, three columns summarising point estimates from the weighted optimal statistic, at different spectral indices $\alpha$. On the right, three columns summarising Bayesian upper limits (UL) with log-uniform prior on $\Omega^0_{25}$ and either uniform or Gaussian prior on $\alpha$. These results are consistent with no detection of the amplitude of the background ($\Omega^\alpha_{\rm ref}$ is consistent with 0), nor its spectral shape ($\alpha$ remains unconstrained).}
    \label{tab:O3_results}
\end{table}

\begin{figure}
    \centering
    \includegraphics[width=0.4\linewidth]{O3_2.pdf}
    \includegraphics[width=0.4\linewidth]{O3_3.pdf}
    \caption{Parameter estimation results with {\tt pygwb$\_$pe} on LVK O3 data, using a log-uniform prior on $\Omega_{25}$, and a uniform prior (left) or a Gaussian prior on $\alpha$ (right), as described in the text. The priors are denoted by the gray dashed lines. }
    \label{fig:O3_pe}
\end{figure}