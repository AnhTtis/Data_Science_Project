
Since the first direct \gls{gw} detection~\cite{PhysRevLett.116.061102}, the field of \gls{gw} astrophysics has exploded, now encompassing a wide range of instrumental and observational campaigns across the globe. %
These detection efforts monitor a vast range of frequencies, from the nanohertz to the kilohertz, and are sensitive to a multitude of \gls{gw} sources emitting therein. %
While the \gls{gw} sources in each band may present extremely different characteristics, a potential candidate for {\it all} \gls{gw} measurements is a \gls{gwb}, given by the collection of all \glspl{gw} too faint to be individually resolved, or by the incoherent overlap of a large number of signals in the same band~\cite{Regimbau:2011rp, Christensen_review, SGWB_review2022_AIR}. %
This sort of signal has been targeted in several different datasets~\cite{LIGO_S1, LIGO_S4, LIGO_S5, LIGO_O1, LIGO_O2, LIGO_O3} using search methods which estimate the \gls{gw} strain signal power modelling the signal as stochastic, frequently resorting to cross-correlation of multiple independent observations~\cite{Allen:1999}. %
These searches are often referred to as stochastic searches by the \gls{gw} detection community, and these backgrounds are often referred to as \glspl{sgwb}, even though, in practice, not all target background signals are fully described by stochastic variables\footnote{To avoid confusion, in this paper we will use the term \gls{sgwb} to refer to signals that are indeed defined as stochastic fields.}, and this definition may imply an approximation. %
So far, no confident detection of a \gls{gwb} has been claimed.

With this paper we present {\tt pygwb}~\cite{pygwb_pypi}, a new Python--based package tailored to searches for isotropic \glspl{gwb} with current ground-based interferometers, namely the \gls{ligo}~\cite{aLIGO_sensitivity}, the Virgo observatory~\cite{Acernese_2014}, and the KAGRA detector~\cite{KAGRA_design_doc}, and with the potential to be expanded and adapted to several other detection efforts. %
The core analysis tools, described in detail in what follows, are heavily inspired by the \gls{lvk} stochastic analysis code, {\tt stochastic.m}. %
The latter consists of a set of {\tt MATLAB} scripts easily parallelizable on a high-throughput computing cluster, and has been used in \gls{lvk} data analysis for the past data acquisition runs~\cite{LIGO_S4, LIGO_S5, LIGO_O1, LIGO_O2, LIGO_O3}. %
These include the 3 observing runs: O1 (September 2015 to January 2016), O2 (November 2016 to August 2017), and O3 (April 2019 to March 2020), performed with Advanced \gls{ligo} Hanford and Livingston, and Advanced Virgo~\cite{Acernese_2014} for part of O2 and O3. % 
Data from Virgo has been included in stochastic analyses as of the latest observing run. %
The analysis consists in the calculation of an optimal statistic~\cite{Allen:1999} from the data of multiple interferometers, which is directly related to the amplitude of the \gls{gwb} signal.

A notable change throughout the years of stochastic \gls{gw} analyses has been the constant shift towards Bayesian parameter estimation~\cite{Mandic_2012, LIGO_O3}. %
To date, there is no preferred stochastic parameter estimation software, and different groups have employed private scripts. %
To extend the scope of the stochastic search beyond the optimal statistic, we include a parameter estimation module in {\tt pygwb} based on the {\tt Bilby} package~\cite{Ashton:2018jfp} which allows the user to test both predefined and user-defined models and obtain posterior distributions on the parameters of interest. %

The steady inflow of ever-improving \gls{gw} data open for analysis~\cite{gwosc} has been a catalyst for open-source \gls{gw} data analysis codebase development. %
By adopting the Python language and focusing on user-friendliness, flexibility, and portability, we intend to introduce stochastic searches to the wider \gls{gw} community. %
Detecting a \gls{gwb} with ground-based interferometers will be a community effort, and we expect search pipelines to evolve along the way. %
The format and structure of {\tt pygwb} facilitates this evolution, and conversely, the package is suitable for beginners approaching \gls{gwb} data analysis for the first time. %


%``We want an easy to use, customizable and flexible code which the community will be able to contribute to as we approach detection.''

This paper is structured as follows. %
In Sec. \ref{sec: GWB analysis}, concepts related to the characterization and detection methods of a \gls{gwb} are reviewed. %
A detailed overview of the individual modules that make up the {\tt pygwb} package follows in Sec. \ref{sec: modules}, outlining the steps of \gls{lvk} stochastic analyses. %
Several manager objects which store relevant data and handle the analysis internally are described in Sec.~\ref{sec: manager objects}. % 
The built-in {\tt pygwb} pipeline which combines individual modules and performs the search for an isotropic \gls{gwb} is presented in Sec.~\ref{sec: pipeline}. %
To test the capabilities of the pipeline, mock datasets with a variety of simulated signals are analyzed in Sec.~\ref{sec: MDC}. %
To conclude, results from the analysis of the third \gls{lvk} collaboration observing run, O3, are presented and compared with collaboration results in Sec.~\ref{Sec:O3Data}.