
Ground-based laser interferometers present many narrow-frequency noise artifacts which are typically persistent in time, and are generally referred to as noise lines. %
Some examples are calibration lines and mechanical resonances \cite{LIGO:2021ppb,Virgo:2022ysc,vanRemortel_2022}. %
The {\tt notch} module provides the framework to properly deal with these noise lines in the case of the search for an isotropic \gls{gwb}. %
The solution is to ``notch out'' these noise lines, i.e., set the values of the spectra at the affected frequency bins to zero. %
Note that the {\tt notch} module is not built to identify these lines, as this is typically done by detector characterization experts working closely with instrumentalists running the detectors. %
Rather, the final product of the {\tt notch} module is a frequency mask which may be applied to the relevant spectra in the analysis. %

The key object of the {\tt notch} module is the {\tt StochNotchList}, which is a list of {\tt StochNotch} objects. %
A {\tt StochNotch} object represents a physical noise line which has been identified and needs to be removed from the data analysis. %, e.g. a calibration line. %
The object has a minimum and maximum frequency indicating the contaminated frequency region. %
Furthermore, it also comes with a descriptive string which allows the user to keep track of the reason why the line was notched. %
All the different {\tt StochNotch} objects for a certain analysis are then stored in the {\tt StochNotchList} which contains the entire list of lines to be notched from the analysis. %

The notch mask used to apply a set of notches within the analysis is constructed conservatively, such that any frequency that has overlapping frequency content with the noise lines defined in the {\tt StochNotchList} will be removed when applying the notch mask. %
%Concretely, the \gls{gwb} estimator $\hat\Omega^\alpha_{{\rm ref},f}$ is calculated at the discrete frequency $f$. %
To maintain generality, we discuss here a generic estimated spectrum $\hat\Omega_{f}$, where its value at frequency $f$ estimates $\Omega_{\textrm{GW}}(f)$ in the frequency range [$f - \delta f/2,f + \delta f/2$], where $\delta f$ is the chosen frequency resolution, as defined in Sec.~\ref{sec:spectral}. %
If a noise line has any overlap with the interval [$f - \delta f/2,f + \delta f/2$], the $f$ frequency bin is excluded. %
This implies that a hypothetical delta-peak noise line at $f +\delta f/2$, leads to notching both $f$ as well as $f + \delta f$. %
\begin{figure}
    \centering
    \includegraphics[width = 0.55\textwidth]{notch_1.pdf}
    \caption{Example of how the notching of noise lines (orange curve) applied to the discrete measurements of the spectrum $\hat\Omega_{\textrm{GW}, f}$ (blue stars) leads to a final set of measurements (red dots). % after excluding possible contaminate frequency bins. 
    The vertical shaded regions indicate the bins, where even bins are white and odd bins are light blue.    
    The orange line traces out the noise lines such that a noise line is present where the orange curve is zero. 
    The analyzed data spans $[5.0, \,6.875]$ Hz, in the un-shaded region. %
    In this example there are five noise lines, from left to right: a noise line ending at the lowest frequency bin, a noise line entirely contained in one frequency bin, a noise line spread across two frequency bins, a noise line spread across multiple frequency bins, and a noise line from bin-edge to bin-edge. %
    %The data analysed is represented by the blue stars, which is a vector starting at 5Hz, up to 6.875Hz, and a 0.125Hz resolution. 
    %The green stars are the remaining values of $C_{IJ}$ which are not zero after applying the notch mask based on the noise lines represented by the orange curve.
    After our notching procedure, the data is reduced to the bins marked by the red dots. %
    For visual convenience we have changed the amplitude in these remaining frequency bins by a factor 0.9. }
    \label{fig:notch_example}
\end{figure}
We present the creation of a notch mask with an example in Fig. \ref{fig:notch_example}, which illustrates how our conservative notching strategy excludes frequency bins based on different scenarios of noise lines. 

The current code is set up to apply the same notches to an entire stretch of data, which can be considered ``time-independent'' notching. %
To allow for time-dependent notching we could either use the current {\tt notch} module and split the analysis in different segments, each having their own notch list. Alternatively, one could extend the current module with an additional parameter which keeps track of which times have to be notched. Since typically the majority of the notched lines in the search for an isotropic \gls{gwb} with data from the \gls{ligo} and Virgo detectors are present during the entire dataset, the possible gain of implementing time-dependent notching is expected to be limited.