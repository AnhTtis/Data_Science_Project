\subsubsection{Stationary and continuous stochastic gravitational-wave background}
\label{MDCSimulator}
We employ the {\tt Network} (Sec.~\ref{sec:network}) to generate a stationary and continuous \gls{sgwb} signal with a fixed \gls{psd}, $S_h(f)$. %
This allows us to simultaneously test the module and the whole analysis pipeline. %
%We now perform a more rigorous test of the {\tt simulator} module and use the {\tt pygwb} pipeline to recover the injection. 
The injected \gls{sgwb} is scale-invariant, i.e., $\Omega_{\rm GW}(f)$ is constant over frequencies, 
\begin{equation}
    \Omega_{\rm inj}(f)= 1.06 \times 10^{-7}\,.   
\end{equation}
This is converted to $S_h(f)$ using the relation in Eq.~\eqref{eq:omegatoI}. %
The noise $P_n(f)$ is taken to be Gaussian, colored using the the Advanced \gls{ligo} noise \gls{psd} \cite{aLIGO_sensitivity}. %
One hundred days of consecutive data are simulated at a sampling rate of $1024$ Hz. %

Each of the one hundred days is analyzed separately, and we recover a distribution of $\hat\Omega^0_{\rm 25}$ point estimates, shown in Fig.~\ref{fig:hundreddaydataset} (left), using $\alpha=0$ and $f_{\rm ref} = 25$ Hz in the pipeline. %
Analyzing one hundred days separately allows us to construct a distribution of recovered point estimates, which is useful to assess the ability of the {\tt simulator} module to inject a stochastic stationary signal. %
%This test assesses both the ability to inject a signal correctly, as well as the ability to recover the injected signal using the {\tt pygwb} pipeline. 
We then perform parameter estimation on the combined one hundred days, presented in Fig. \ref{fig:hundreddaydataset} (right). We assume a log-uniform prior from $10^{-11} - 10^{-6}$ for $\Omega_{\rm ref}$ and Gaussian prior with mean 0 and standard deviation 1.5 for $\alpha$. %
This shows a recovery within $1\sigma$ for $\Omega_{\rm ref} = 1.06 \times 10^{-7}$ and within $2\sigma$ for the spectral index $\alpha_{\rm inj} = 0$. 

The tests above illustrate that the {\tt simulator} module is able to successfully inject a stochastic stationary signal and that the {\tt pygwb} pipeline is able to recover this injection.

\begin{figure}[t]
    \centering
    \includegraphics[width=0.45\textwidth]{simtest_1.pdf}
    \includegraphics[width=0.3\textwidth]{simtest_2.pdf}
    \caption{\textbf{Left}: Distribution of the recovered point estimate for each day in the dataset. The injected value is denoted by the red line. $\overline{\Omega}_{\rm ref}$ and $\sigma_{\rm ref}$ denote the mean and the standard deviation of the one hundred point estimates. \textbf{Right}: Parameter estimation performed on the one hundred days, obtained assuming a log-uniform prior from $10^{-11} - 10^{-6}$ for $\Omega_{\rm ref}$ and Gaussian prior with mean 0 and standard deviation 1.5 for $\alpha$. The injected values are denoted by the black lines, while the contours represent the 1$\sigma$, 2$\sigma$, and $3\sigma$ contours. The vertical dashed lines represent the $1 \sigma$ confidence interval.}
    \label{fig:hundreddaydataset}
\end{figure}
\subsubsection{Gravitational-wave background from a coalescing compact binary population}


%A compact binary coalescence follows the inspiralling phase, with GWs emission, of a binary system, where two massive stars have collapsed into a neutron star or a black hole. %
The inspiral and merger of two compact objects emit a characteristic \gls{gw} signal. %
We generate datasets containing a \gls{gwb} signal resulting from the superposition of \gls{gw} signals from a set of \gls{cbc} populations including \glspl{bbh} and \glspl{bns}. %
 To simulate the signals, we employ the code used in the past for the \gls{et} \glspl{mdsc} \citep{ET-first-MDC-PhysRevD.86.122001, ET-second-MDC-PhysRevD.89.084046, ET-second-MDC-low-f-PhysRevD.93.024018} and for the Advanced \gls{ligo} and Advanced Virgo \gls{mdsc} \citep{HLV-MDC-PhysRevD.92.063002}. %
The Monte Carlo algorithm that we use for the generation of a compact binary population up to redshift $z=10$ is extensively described in \cite{ET-first-MDC-PhysRevD.86.122001} and \cite{Regimbau:2014nxa}. We summarize below the main steps of the simulations.

To generate a \gls{cbc} population we assume a merger rate per unit redshift~\citep{Belczynski:2006br, Berger:2006ik, Belczynski:2000wr, Bulik:2003kr},
\begin{equation}
    \frac{\text{d}R(z)}{\text{d}z} = \frac{\text{d}V_{\rm c}}{\text{d}z} {r_c}(z),
\end{equation}
where $\text{d}V_{\rm c}/\text{d}z$ is the co-moving volume element and ${r_c}$ the coalescence rate as a function of redshift~\cite{Regimbau:2011rp}. %
The element of co-moving volume assumes a $\Lambda$CDM cosmology from Planck 2018~\cite{Planck2018} (Hubble parameter $H_0 = 67.7\, \mathrm{km/s/Mpc}$, $\Omega_{m}=0.31$ and $\Omega_{\Lambda}=1-\Omega_{m}$). %
We assume a coalescence rate normalized to a local rate $r_c(0)=1\,\mathrm{Mpc^{-3}\, Myr^{-1}}$ for \gls{bns} coalescences and  $r_c(0)=3\,\mathrm{Mpc^{-3}\, Myr^{-1}}$ for \gls{bbh} coalescences, assuming the star formation rate from \cite{Hopkins:2006bw} and a minimum delay time between binary formation and merger of $20\, \mathrm{Myr}$ for \glspl{bns} and $50\, \mathrm{Myr}$ for \glspl{bbh}; see \citep{Dominik:2012kk, Neijssel_2019} for more details. %
These choices give rise to a dataset composed by $87 \%$ of \glspl{bns} and $13 \%$ of \glspl{bbh}. % 

The time intervals $\tau$ between consecutive \gls{cbc} events in our population are obtained by sampling an exponential distribution $P(\tau) = \exp(-\tau/\bar{\tau})$, where $\bar{\tau}$ is the average time between consecutive events. This is consistent with the assumption that the coalescence times $t_c$ of the events behave as a Poisson process~\cite{Regimbau:2014nxa}. %
The coalescence redshift is drawn from the normalized coalescence rate $p(z)=\bar{\tau}\text{d}R/\text{d}z(z)$ within $z \in [0,10]$. %
The sky position $\hat{n}$ of each source is generated isotropically on the sky. %
The \gls{gw} polarization angle $\psi$, the phase angle $\phi_0$ at the coalescence time, and the cosine of the inclination angle of the orbital plane to the line of sight $\iota$ are all drawn from uniform distributions. %
The mass function of the components in the \glspl{bbh} is chosen to be a \gls{plpp} from the preferred case presented in the \gls{lvk} collaboration \gls{cbc} population inference paper~\citep{O3_pop_paper_LIGOScientific:2021psn} or a simple \gls{pl}~\citep{O2_pop_paper_LIGOScientific:2020kqk}, while the \gls{bns} masses are drawn from uniform distribution between 1 and 3 $M_{\odot}$. %\kt{Since the mass range for BNS is mentioned, should the one for BBH also be mentioned?}
The \gls{bbh} mass functions are used to label the two datasets presented below. %
Spins are neglected in both cases. %
\begin{figure}[t]
    \centering
\includegraphics[width=0.5\textwidth]{MDC_1.pdf}
    \caption{$\Omega_{\mathrm{GW}}(f)$ for the dataset corresponding to the \gls{plpp} (blue) and the \gls{pl} (red) models for the mass function. The black line is the power-law integrated sensitivity (PI) curve for an observation time of six months and an expected \gls{snr} = 5, assuming the HL baseline with Advanced \gls{ligo} plus design sensitivity. The simulated signals intersect the PI curve, hence they are expected to be detected with an \gls{snr} of at least 5.}
    \label{fig:Omega_CBC_injection}
\end{figure}

For each source, the signal waveform is generated in the time domain. %
For \glspl{bns}, we use the {\tt TaylorT4} time-domain waveform \citep{TaylorT4_Buonanno:2002fy}. %
For \gls{bbh} signals, we use the {\tt EOBNRv2} \citep{EOBNRv2_Buonanno:2009zt} % 
time-domain waveform from numerical relativity. %
These are then injected into the \gls{ligo} Hanford and Livingston detectors, with the addition of colored Gaussian noise generated from the \gls{ligo} A+ Design~\cite{MDC_sensitivity_curves, aplus_design_curve} expected sensitivity curve, to produce the final datasets. %


Following the above prescriptions, we generate two six-months datasets with sampling frequency 1024 Hz, labelled \gls{plpp} and \gls{pl}, formed by two different \gls{cbc} populations. %
The two populations differ by the mass distributions of the \glspl{bbh} and the average time between consecutive events, as seen in Table \ref{tab:MDC_CBC_results}. %
The latter is chosen such that the \gls{gwb} amplitudes of the two datasets match, for ease of comparison. % 
Furthermore, to obtain a \gls{snr} large enough to confidently detect the injected \gls{gwb}, a small amplification of the signal is required. %
To this end, the amplitude of the \gls{cbc} waveforms is multiplied by 1.5 and 1.7 for the \gls{plpp} and the \gls{pl} datasets, respectively, resulting in an injected value of $\Omega_{\mathrm{ref}} = 2.05 \times 10^{-9}$. %

The $\Omega_{\mathrm{GW}}(f)$ spectrum relative to the each dataset is obtained by summing the contributions from individual coalescences \citep{HLV-MDC-PhysRevD.92.063002}, and is illustrated in Fig.~\ref{fig:Omega_CBC_injection}. %
As may be observed, in the case of \gls{cbc} signals $\Omega_{\mathrm{GW}}(f)$ increases as $f^{2/3}$ from the inspiral phase (and then as $f^{5/3}$ from the \gls{bbh} merger phase) before reaching a peak and steeply decreasing~\cite{Marassi:2011si}. %
This motivates fixing the spectral index parameter to $\alpha=2/3$ in our searches. %
Fig.~\ref{fig:Omega_CBC_injection} also shows the \gls{pi} curve \citep{PI_curves:PhysRevD.88.124032} for the Hanford-Livingston baseline, assuming the design A+ sensitivity for the two detectors \citep{aplus_design_curve}, an observation time $T_{\rm obs} = 6$ months, and a desired sensitivity of \gls{snr}=5. %
Given that the \gls{pi} curve is almost tangent to $\Omega_{\mathrm{ref}}$ of the two datasets, we expect to observe the \gls{gwb} signals with \gls{snr} $\sim 5$. %

%\noindent \textbf{Analysis and results}

\begin{figure}[t]
    \centering
    \includegraphics[width=0.4\linewidth]{MDC_2.pdf}
    \includegraphics[width=0.4\linewidth]{MDC_3.pdf}
    \caption{PE results. Left: Corner plot obtained from running the parameter estimation over the \gls{plpp} dataset.
    Right: Corner plot obtained from running the parameter estimation over the \gls{pl} dataset.
    Each plot shows the posteriors on $\Omega_{\mathrm{ref}}$ and $\alpha$ obtained assuming a log-uniform prior on $\Omega^0_{\mathrm{ref}}$ from $10^{-11}$--$10^{-8}$ and a Gaussian prior on $\alpha$ with mean 2/3 and standard deviation of 1.5, respectively, denoted by the gray dashed lines. The injected values are represented by the black lines, indicating a recovery of both the amplitude of the signal and $\alpha$ within $1\, \sigma$. The vertical blue dashed lines represent the $2 \sigma$ confidence interval.}
    \label{fig:MDC_pe}
\end{figure}
We analyze the two datasets in the frequency band $20 - 500$ Hz, using a frequency resolution of $1/32$ Hz and a segment duration of 192 s. %
We choose $\alpha = 2/3$, $f_{\rm ref} = 25$ Hz, and $H0=67.7$ km/s/Mpc for this analysis. %
The results of the analysis are summarized in Table \ref{tab:MDC_CBC_results}. %
We recover the \gls{plpp} injection within $1\, \sigma$, and observe it with \gls{snr} = 5.4, while recovering the \gls{pl} injection within $1\, \sigma$, with \gls{snr} = 5.0. %
We attribute the differences in the recoveries to the specific data and noise realizations within the datasets. %
% A larger average time between two successive binary mergers, as in the case of the \gls{plpp} dataset, results in a more discontinuous \gls{gwb}, which requires a longer amount of time to be observed with the techniques used in this work. %

We then proceed with estimating the parameters $\alpha$ and $\Omega^\alpha_{\mathrm{ref}}$ modelling $\Omega_{\mathrm{GW}}(f)$ as a simple power-law in frequency as given by Eq.~\eqref{eq:omega-plaw}. %
We assume a log-uniform prior over $\Omega^0_{\mathrm{ref}}$ in the range $[\Omega^0_{\mathrm{min}},\,\Omega^0_{\mathrm{max}}]=[10^{-11}, \,10^{-8}]$, and a Gaussian prior on $\alpha$ with mean $2/3$ and standard deviation $(\log_{10}{\Omega^0_{\mathrm{min}}}-\log_{10}{\Omega^0_{\mathrm{min}}})/2 = 1.5$. %
Note that the priors in $\Omega^\alpha_{\mathrm{ref}}$ are defined for $\alpha = 0$. %
The choice of the prior over $\alpha$ can be understood as follows. %
The log-uniform prior over $\Omega^0_{\mathrm{ref}}$ induces some implicit prior over $\alpha$ that can be shown to be a triangular prior centred on $\alpha = 0$ and non-zero for $|\alpha| \leq (\log_{10}{\Omega^0_{\mathrm{max}}}-\log_{10}{\Omega^0_{\mathrm{min}}})$. %
To avoid a vanishing prior outside of this range, we choose a Gaussian prior for $\alpha$ with standard deviation comparable with the triangular prior, centered on $\alpha =2/3$ to better match the injected \gls{gwb}. %

Parameter estimation corner plots are shown in Fig.~\ref{fig:MDC_pe}. %
For both datasets, $\Omega_{\mathrm{ref}}$ and $\alpha$ are recovered within $1\, \sigma$. %
The log-Bayes factors $\mathcal{B}^{\rm GW}_{\rm noise}$ are $11.1$ and $9.2$ for the \gls{plpp} and \gls{pl} datasets, respectively, indicating strong evidence~\cite{bayes_factor_doi:10.1080/01621459.1995.10476572} for the presence of signal over noise only. 

\begin{table}[h]
    \centering
    \begin{tabular}{c|c|c|c|c|c}
       {\sc dataset}  & $\tau$ (s) & $a$ &  $  ~\qty(\hat{\Omega}^{2/3}_{25} \pm \hat{\sigma}^{2/3}_{25}) \times 10^{9}$~ & ~SNR~ & ~$\mathcal{B}^{\mathrm{GW}}_{\mathrm{noise}}$~\\
    \hline
    \hline
        PLPP & ~60~  & ~1.5~  & $2.09 \pm 0.39$ & ~5.4~ & ~11.1~\\
        PL & ~54.7~ & ~1.7~ & $1.94 \pm 0.39$ & ~5.0~ & ~9.2~
    \end{tabular}
    \caption{Parameters and results of each dataset. The first row refers to the \gls{plpp} dataset, while the second row to the \gls{pl} one. The second and third columns display the average time between two successive binary mergers, $\tau$, and the waveform amplification factor, $a$. The last three columns illustrate the recovered point estimate with $1\, \sigma$ uncertainty on the quantity $\hat{\Omega}_{\mathrm{ref}}^\alpha$ ($f_{\mathrm{ref}} = 25\, \mathrm{Hz}$, $\alpha=2/3$), the corresponding \gls{snr}, and the log-Bayes factor $\mathcal{B}^{\mathrm{GW}}_{\mathrm{noise}}$.
    }
    \label{tab:MDC_CBC_results}
\end{table}