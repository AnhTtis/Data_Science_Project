\documentclass[%
%reprint,
%superscriptaddress,
%groupedaddress,
%unsortedaddress,
%runinaddress,
%frontmatterverbose, 
%preprint,
%preprintnumbers,
nofootinbib,
%nobibnotes,
%bibnotes,
%sor,
%jor,
 %aps
amsmath,amssymb,
%onecolumn,
%pra,
%prb,
%rmp,
%prstab,
%prstper,
%floatfix,
]{aastex631}

\usepackage{graphicx}
\usepackage{dcolumn}
\usepackage{bm}
\usepackage{braket}
\usepackage{hyperref}
% %\usepackage[symbol]{footmisc}
\usepackage{xcolor, soul}
\usepackage{pygtex}
\usepackage{physics, amsmath}
\usepackage{inputenc}
% \usepackage[normalem]{ulem} 

%%%%%%%%%% acronym definitions %%%%%%%%%%%%%%
\usepackage[acronym]{glossaries}
\makeglossaries
%% Acronyms
\newacronym[plural=GWs,firstplural=gravitational waves (GWs)]{gw}{GW}{gravitational wave}
\newacronym[plural=BHs,firstplural=black holes (BHs)]{bh}{BH}{black hole}
\newacronym[plural=CBCs,firstplural=compact binary coalescences (CBCs)]{cbc}{CBC}{compact binary coalescence}
\newacronym[plural=SGWBs,firstplural=stochastic gravitational-wave backgrounds (SGWBs)]{sgwb}{SGWB}{stochastic gravitational-wave background}
\newacronym[plural=GWBs, firstplural=gravitational-wave backgrounds (GWBs)]{gwb}{GWB}{gravitational-wave background}
\newacronym[plural=DFTs, firstplural=discrete Fourier transforms (DFTs)]{dft}{DFT}{discrete Fourier transform}
\newacronym[plural=MVUEs, firstplural=minimum-variance unbiased estimators(MVUEss)]{mvue}{MVUE}{minimum-variance unbiased estimator}
\newacronym{ligo}{LIGO}{Laser Interferometer Gravitational-wave Observatory}
\newacronym{lvk}{LVK}{LIGO, Virgo, and KAGRA Collaboration}
\newacronym{lisa}{LISA}{Laser Interferometer Space Antenna}
\newacronym{et}{ET}{Einstein Telescope}
\newacronym{ce}{CE}{Cosmic Explorer}
\newacronym[plural=BBHs,firstplural=binary black holes (BBHs)]{bbh}{BBH}{binary black hole}
\newacronym[plural=BNSs,firstplural=binary neutron stars (BNSs)]{bns}{BNS}{binary neutron star}
\newacronym[plural=MDSCs,firstplural=mock data and science challenges (MDSCs)]{mdsc}{MDSC}{mock data and science challenges}
\newacronym[plural=IFTed,firstplural=inverse--discrete Fourier transformed (IFTed)]{ift}{IFT}{inverse--discrete Fourier transform}
\newacronym{plpp}{PLPP}{power-law-plus-peak}
\newacronym{pl}{PL}{power-law}
\newacronym{snr}{SNR}{signal-to-noise ratio}
\newacronym{psd}{PSD}{power spectral density}
\newacronym[plural=ORFs, firstplural=overlap reduction functions (ORFs)]{orf}{ORF}{overlap reduction function}
\newacronym{asd}{ASD}{amplitude spectral density}
\newacronym{csd}{CSD}{cross-spectral density}
\newacronym{gwosc}{GWOSC}{Gravitational-wave Open Science Center}
\newacronym{ks}{KS}{Kolmogorov-Smirnov}
\newacronym{gr}{GR}{General Relativity}
\newacronym{pi}{PI}{power-law integrated sensitivity}

% \newcommand{\air}[1]{\textcolor{orange}{[{\bf AIR}: #1]}}
% \newcommand{\lt}[1]{\textcolor{teal}{[{\bf LT}: #1]}}
% \newcommand{\fdl}[1]{\textcolor{violet}{[{\bf FDL}: #1]}}
% \newcommand{\arr}[1]{\textcolor{pink}{[{\bf ARR}: #1]}}
% \newcommand{\kt}[1]{\textcolor{magenta}{[{KT}: #1]}}
% \newcommand{\mla}[1]{\textcolor{brown}{[{\bf MLA}: #1]}}
% \newcommand{\kj}[1]{\textcolor{blue}{[{KJ}: #1]}}
% \newcommand{\sk}[1]{\textcolor{brown}{[{SK}: #1]}}
% \newcommand{\pmm}[1]{\textcolor{purple}{[{PMM}: #1]}}

\begin{document}
\nolinenumbers
\title{{\tt pygwb}: Python-based library for gravitational-wave background searches}

\author{Arianna I. Renzini}
\email{arenzini@caltech.edu}
\affiliation{LIGO Laboratory,  California  Institute  of  Technology,  Pasadena,  California  91125,  USA}
\affiliation{Department of Physics, California Institute of Technology, Pasadena, California 91125, USA}
\author{Alba Romero-Rodríguez}
 \affiliation{Theoretische Natuurkunde, Vrije Universiteit Brussel, Pleinlaan 2, B-1050 Brussels, Belgium} 
\author{Colm Talbot}
\affiliation{Kavli Institute for Astrophysics and Space Research, Massachusetts Institute of Technology, 77 Massachusetts Ave, Cambridge, MA 02139, USA}
\author{Max Lalleman}
 \affiliation{Universiteit Antwerpen, Prinsstraat 13, 2000 Antwerpen, Belgium} 
\author{Shivaraj Kandhasamy}
\affiliation{Inter-University Centre for Astronomy and Astrophysics, Pune 411007, India}
\author{Kevin Turbang}
\affiliation{Universiteit Antwerpen, Prinsstraat 13, 2000 Antwerpen, Belgium}
\affiliation{Theoretische Natuurkunde, Vrije Universiteit Brussel, Pleinlaan 2, B-1050 Brussels, Belgium} 
\author{Sylvia Biscoveanu}
\affiliation{Kavli Institute for Astrophysics and Space Research, Massachusetts Institute of Technology, 77 Massachusetts Ave, Cambridge, MA 02139, USA}
\affiliation{LIGO Laboratory, Massachusetts Institute of Technology, 185 Albany St, Cambridge, MA 02139, USA}
\author{Katarina Martinovic}
\affiliation{Theoretical Particle Physics and Cosmology Group,  Physics Department, \\ King's College London, University of London, Strand, London WC2R 2LS, United Kingdom}
\author{Patrick Meyers}
\affiliation{Theoretical Astrophysics Group, California Institute of Technology, Pasadena, CA 91125, USA}
\author{Leo Tsukada}
\affiliation{Department of Physics, The Pennsylvania State University, University Park, Pennsylvania 16802, USA}
\affiliation{Institute for Gravitation and the Cosmos, The Pennsylvania State University, University Park, Pennsylvania 16802, USA}
\author{Kamiel Janssens}
 \affiliation{Universiteit Antwerpen, Prinsstraat 13, 2000 Antwerpen, Belgium}
 \affiliation{Universit\'e C$\hat{o}$te d’Azur, Observatoire C$\hat{o}$te d’Azur, ARTEMIS, Nice, France}
\author{Derek Davis}
\affiliation{LIGO Laboratory,  California  Institute  of  Technology,  Pasadena,  California  91125,  USA}
\affiliation{Department of Physics, California Institute of Technology, Pasadena, California 91125, USA}
\author{Andrew Matas}
\affiliation{Max Planck Institute for Gravitational Physics (Albert Einstein Institute), D-14476 Potsdam, Germany}
\author{Philip Charlton}
\affiliation{OzGrav, Charles Sturt University, Wagga Wagga, New South Wales 2678, Australia}
\author{Guo-Chin Liu}
\affiliation{Department of Physics, Tamkang University, Danshui Dist., New Taipei City 25137, Taiwan}
\author{Irina Dvorkin}
\affiliation{Institut d’Astrophysique de Paris, Sorbonne Universit\'e \& CNRS, UMR 7095, 98 bis bd Arago, F-75014 Paris, France}
\affiliation{Université Paris Cité, CNRS, Astroparticule et Cosmologie, F-75013 Paris, France}
\author{Sharan Banagiri}
\affiliation{Center for Interdisciplinary Exploration and Research in Astrophysics (CIERA), Northwestern University, 1800 Sherman Ave, Evanston, IL 60201, USA}
\author{Sukanta Bose}
\affiliation{Inter-University Centre for Astronomy and Astrophysics, Pune 411007, India}
\author{Thomas Callister}
\affiliation{Kavli Institute for Cosmological Physics, The University of Chicago, 5640 S. Ellis Ave., Chicago, IL 60615, USA}
\author{Federico De Lillo}
\affiliation{Centre for Cosmology, Particle Physics and Phenomenology (CP3),\\
Universit\'e catholique de Louvain, Louvain-la-Neuve, B-1348, Belgium}
\author{Luca D'Onofrio}
\affiliation{Universit\`a di Napoli "Federico II", Dipartimento di Fisica "Ettore Pancini", Compl. Univ. di Monte S. Angelo, Via Cinthia 21, I-80126, Napoli, Italy}
\affiliation{INFN, Sezione di Napoli, Compl. Univ. di Monte S. Angelo, Edificio G, Via Cinthia, I-80126, Napoli, Italy}
\author{Fabio Garufi}
\affiliation{Universit\`a di Napoli "Federico II", Dipartimento di Fisica "Ettore Pancini", Compl. Univ. di Monte S. Angelo, Via Cinthia 21, I-80126, Napoli, Italy}
\affiliation{INFN, Sezione di Napoli, Compl. Univ. di Monte S. Angelo, Edificio G, Via Cinthia, I-80126, Napoli, Italy}
\author{Gregg Harry}
\affiliation{Physics Department, American University, Washington, DC 20016, USA}
\author{Jessica Lawrence}
\affiliation{Department of Physics, Texas Tech University, Lubbock, TX 79409, USA}
\author{Vuk Mandic}
\affiliation{School of Physics and Astronomy, University of Minnesota, Minneapolis, MN 55455, USA}
\author{Adrian Macquet}
\affiliation{Institut de F\'{\i}sica d’Altes Energies (IFAE), The Barcelona Institute of Science and Technology, Campus UAB, 08193 Bellaterra (Barcelona) Spain}
\author{Ioannis Michaloliakos}
\affiliation{Department of Physics, University of Florida, Gainesville, Florida 32611, USA}
\author{Sanjit Mitra}
\affiliation{Inter-University Centre for Astronomy and Astrophysics, Pune 411007, India}
\author{Kiet Pham}
\affiliation{School of Physics and Astronomy, University of Minnesota, Minneapolis, MN 55455, USA}
\author{Rosa Poggiani}
\affiliation{Universit\`a di Pisa, I-56127 Pisa, Italy}
\affiliation{INFN, Sezione di Pisa, I-56127 Pisa, Italy}
\author{Tania Regimbau}
\affiliation{LAPP, CNRS, 9 Chemin de Bellevue, 74941 Annecy-le-Vieux, France}
\author{Joseph D. Romano}
\affiliation{Department of Physics, Texas Tech University, Lubbock, TX 79409, USA}
\author{Nick van Remortel}
 \affiliation{Universiteit Antwerpen, Prinsstraat 13, 2000 Antwerpen, Belgium}
\author{Haowen Zhong}
\affiliation{School of Physics and Astronomy, University of Minnesota, Minneapolis, MN 55455, USA}


\begin{abstract}
The collection of gravitational waves (GWs) that are either too weak or too numerous to be individually resolved is commonly referred to as the gravitational-wave background (GWB). % would arise from the superposition of gravitational waves that are too weak to be individually resolved. %
A confident detection and model-driven characterization of such a signal will provide invaluable information about the evolution of the Universe and the population of GW sources within it. %
We present a new, user-friendly Python--based package for gravitational-wave data analysis to search for an isotropic GWB in ground--based interferometer data. %
We employ cross-correlation spectra of GW detector pairs to construct an optimal estimator of the Gaussian and isotropic GWB, and Bayesian parameter estimation to constrain GWB models. %
The modularity and clarity of the code allow for both a shallow learning curve and flexibility in adjusting the analysis to one's own needs. %
We describe the individual modules which make up {\tt pygwb}, following the traditional steps of stochastic analyses carried out within the LIGO, Virgo, and KAGRA Collaboration. %
We then describe the built-in pipeline which combines the different modules and validate it with both mock data and real GW data from the O3 Advanced LIGO and Virgo observing run. %
We successfully recover all mock data injections and reproduce published results. %

\end{abstract}

%\maketitle

%\tableofcontents

\section{Introduction}
\section{Introduction}

The increasing complexity of source code poses a key challenge to the reliability of large-scale software systems. Software bugs in these systems can lead to safety issues~\cite{bug_safety} for users around the world as well as cause non-negligible financial losses~\cite{bug_loss}. As such, developers have to spend a large amount of time and effort on bug fixing. Consequently, \aprfull (\apr), designed to automatically generate patches to fix software bugs, has attracted wide attention from both academia and industry~\cite{long2016prophet, legoues2012genprog, long2015spr, lou2020can, tufano2018empstudy}. 


To achieve \apr, one popular approach is known as Generate-and-Validate (G\&V)~\cite{qi2015gv, ghanbari2019prapr, lou2020can, le2016hdrepair, legoues2012genprog, wen2018capgen, hua2018sketchfix, martinez2016astor, koyuncu2020fixminder, liu2019tbar, liu2019avatar}, which is typically based on the following pipeline: First, fault localization techniques~\cite{wong2016fl, abreu2007ochiai, zhang2013injecting, papadakis2015metallaxis, li2019deepfl, li2017transforming} are applied to determine the suspicious locations in programs where bugs are likely to exist. Then, the buggy locations are used by the \apr tools to generate a list of patches that replace buggy lines with correct lines. Afterward, each patch is validated against the original test suite to identify any \emph{plausible patches} (i.e., passing all tests in the test suite). Finally, to determine the \emph{correct patches}, developers examine the list of plausible patches to see if any of them can correctly fix the bug. 

Traditional \apr tools can mainly be categorized into heuristic-based~\cite{legoues2012genprog, le2016hdrepair, wen2018capgen}, constraint-based~\cite{mechtaev2016angelix, le2017s3, demacro2014nopol, long2015spr} and \template~\cite{ghanbari2019prapr, hua2018sketchfix, martinez2016astor, liu2019tbar, liu2019avatar}. Among these traditional tools, \template \apr tools~\cite{ghanbari2019prapr, liu2019tbar, benton2020effectiveness} have been able to achieve state-of-the-art results. \Template \apr tools typically leverage pre-defined templates (e.g., adding a nullness check) for bug fixing. However, since these fix templates are typically handcrafted, the number and types of bugs they are able to fix can be limited. 



To address the limitations of traditional \apr, researchers have proposed various \learning \apr tools~\cite{li2020dlfix, chen2018sequencer, jiang2021cure, lutellier2020coconut, zhu2021recoder, ye2022rewardrepair} based on the \nmtfull (\nmt) architecture~\cite{sutskever2014mt} where the input is the buggy code snippets and the goal is to translate the buggy code snippets into a fixed version. To accomplish this, \learning \apr tools require supervised training datasets with pairs of both buggy and fixed code snippets in order to learn how to perform this translation step. These training data are usually obtained by mining historical bug fixes using heuristics/keywords~\cite{dallmeier2007benchmark}, which can be imprecise for identifying bug-fixing commits; even the actual bug-fixing commits can include irrelevant code changes, leading to further pollution in the dataset~\cite{xia2022alpharepair}.
% 
Moreover, it can be hard for such \apr tools to generalize and fix bug types unseen during training. 



To better leverage recent advances in \plmfull{s} (\plm{s}), researchers~\cite{xia2022alpharepair, xia2023repairstudy, kolak2022patch, prenner2021codexws} have directly applied \plm{s} to generate patches without bug-fixing datasets. These \llm-based \apr tools work by either directly generating a complete code function~\cite{prenner2021codexws, xia2023repairstudy} or predict/infill the correct code snippet given its surrounding context~\cite{xia2022alpharepair, xia2023repairstudy}. By directly using \llm{s} that are pre-trained on billions of open-source code snippets, \llm-based \apr tools can achieve state-of-the-art performance on many repair datasets~\cite{xia2022alpharepair}. 


% 
%
%

Traditional \apr tools have long used the insight of the \emph{plastic surgery hypothesis}~\cite{barr2014plastic} where it states that the code ingredients to fix a bug already exist within the same project. Traditional \apr tools have manually designed pattern-~\cite{ghanbari2019prapr, saha2017elixir} or heuristic-based~\cite{jiang2018simfix, legoues2012genprog} approaches to finding and using such relevant code ingredients to generate fixes for bugs. However, the plastic surgery hypothesis has been largely ignored in \llm-based \apr. In fact, \llm provides a unique opportunity to fully automate the plastic surgery hypothesis idea via fine-tuning (learning project-specific information via model updates from the buggy project) and prompting (directly providing relevant code ingredients to the model), and make it directly applicable to different languages (since the \llm{s} are typically multi-lingual).%
Moreover, despite the intensive manual efforts involved, traditional \apr tools still cannot fully leverage project-specific information due to large search space for leveraging/composing existing code ingredients. In contrast, the project-specific information can effectively leveraged by \llm{s} due to their power in code understanding/vectorization, e.g., even partial/imprecise information may still guide \llm{s} in correct patch generation!
 To this end, we ask the question: \emph{How useful is the plastic surgery hypothesis in the era of \plm{s}}?








\mypara{Our Work.} To answer the question, we present \ourtech{\xspace} -- a \llm-based approach that automatically utilizes the plastic surgery hypothesis by systematically combining multiple fine-tuning and prompting strategies for \apr. \ourtech fine-tunes \plm{s} using two novel domain-specific training strategies: \textbf{\epfinetune} -- we fine-tune using the original buggy project by aggressively masking out a high percentage of tokens, which allows \plm to learn project-specific code tokens and programming styles; and \textbf{\rofinetune} -- which only masks out a single continuous code sequence per training sample, allowing the model to get used to the final \csapr task of predicting a single continuous code sequence. Furthermore, we directly leverage the ability for \plm{s} to understand natural language instructions and introduce a novel prompting strategy, \textbf{\idprompting}, which uses information retrieval and static analysis to obtain a list of relevant identifiers for the buggy lines. While such relevant identifiers are critical for fixing some difficult bugs, they may not be seen by the \llm during inference due to limited context window size. Through the use of prompting, we directly tell the model to use these extracted identifiers (relevant code ingredients) to generate the correct code. Finally, to perform repair, we combine all four model variants (including the base model, both fine-tuned models and the base model with prompting) for the final repair.





While our insight of leveraging the plastic surgery hypothesis for \llm-based \apr is generalizable across different types of \plm{s}, to implement \ourtech, we choose a recent \plm{\xspace}, \ctfive~\cite{wang2021codet5}, which is pre-trained on millions of open-source code snippets. \ctfive is an encoder-decoder model trained using \mspfull (\msp) objective where a percentage of tokens are masked out and each continuous masked token sequence is referred to as a masked span. Also, although we only extract relevant identifiers from the current buggy project (since this paper focuses on the plastic surgery hypothesis), our work can be easily extended to obtain other code information (such as relevant statements or functions) from other sources, such as  the massive pre-training corpora~\cite{husain2020codesearchnet} or historical bug-fixing datasets~\cite{jiang2019infer}, which can provide more coding knowledge for \llm{s}. Besides, although we mainly focus on using traditional string comparison algorithms for information retrieval in this paper, these techniques can be easily replaced by other frequency-based retrieval~\cite{robertson2009probabilistic} and neural search (or embedding-based search)~\cite{reimers2019sentence}.
  In summary, this paper makes the following contributions:


%


\begin{itemize}[noitemsep, leftmargin=*, topsep=0pt]
    \item \textbf{Dimension.} This paper is the first to revisit the important plastic surgery hypothesis in the era of \llm{s}. It opens up a new dimension for \llm-based \apr to incorporate previously neglected information from the buggy project itself to boost \apr performance. Furthermore, it demonstrates the promising future of retrieval-based prompting for modern \llm-based \apr.
    \item \textbf{Implementation.} We implement \ourtech based on the recent \ctfive model. We augment the model using two novel fine-tuning strategies: \epfinetune and \rofinetune, along with a novel prompting strategy based on information retrieval and static analysis: \idprompting. We combine the patches generated by all four models together and perform patch ranking to speed up \apr.% 
    \item \textbf{Evaluation Study.} We conduct an extensive evaluation against state-of-the-art \apr tools. On the widely studied \dfj 1.2 and 2.0 datasets~\cite{just2014dfj}, \ourtech is able to achieve the new state-of-the-art results of 89 and 44 correct bug fixes (15 and 8 more than best baseline) respectively.  Furthermore, we perform a broad ablation study to justify our design. \ourtech demonstrates for the first time that the plastic surgery hypothesis can substantially boost \llm-based \apr and advance state-of-the-art \apr, while being fully automated and general. Moreover, even partial/imprecise code ingredients may still effectively guide \llm{s} for \apr!
\end{itemize}


\section{The isotropic stochastic analysis}
\section{PoseRAC Model}
\label{sec4}

\begin{figure*}[t]
\centering
\includegraphics[width=1.0\textwidth]{figure5.pdf}
\caption{Overview of our proposed PoseRAC. For a input video, the repetitive count can be obtained through Pose Estimation, Transformer Encoder, Pose Mapping and Action-trigger, where only the Encoder and the Pose Mapping need to be trained. We use Triplet Margin Loss to train the Encoder while Binary Cross Entropy Loss to train both the Encoder and the Pose Mapping. In addition to achieving the state-of-the-art performance so far, the biggest highlight of our PoseRAC is that it is lightweight enough to be easily trained on a CPU.}
\label{fig5}
\end{figure*}

Given a video $V={\{x_i\}}^{T}_{1}\in \mathbb{R}^{C\times H\times W\times T}$ with $T$ RGB frames, repetitive action counting model aims to predict a certain value $Y$, which is the number of repetitive actions. In this section, we will introduce our PoseRAC in detail.

\subsection{Model Overview}

As shown in Figure \ref{fig5}, PoseRAC consists of four parts. 

\begin{itemize}

\item The first is a state-of-the-art and lightweight Pose Estimation Network~($\S\ref{first}$), which is used to estimate the poses represented by lots of human pose key points from each frame of the original video sequence. 

\item The second is a simple Transformer Encoder~($\S\ref{second}$) to embed the key points of poses into high-level feature space, where the same class have similar distances, while the distances of different classes are far apart.

\item The third is a Pose Mapping Module~($\S\ref{third}$), where the unique mapping relationship between the salient poses and the action classes can be learned. Each pose can be mapped to the action class with the highest probability after the previous encoding.

\item The fourth part is a lightweight Action-trigger Module~($\S\ref{fourth}$). When we get the salient action classification results of all frames of the entire video sequence, we can use this module to calculate the repetition count in a short time.

\end{itemize}

\subsection{Pose Estimation Network}
\label{first}
Our model first converts the video sequence into a sequence of human pose key points, which can be defined as: 
\begin{equation}
\begin{split}
&V={\{x_i\}}^{T}_{1}\in \mathbb{R}^{C\times H\times W\times T}\\
&V\xrightarrow{\mathrm{Pose Estimation}} P={\{p_i\}}^{T}_{1}\in \mathbb{R}^{D\times K\times T}
\end{split}
\label{eq1}
\end{equation}
where each $x_i$ represents a single RGB frame, and each $p_i$ represents the key points of each frame. To express the key points of each frame, we use $D\times K$ sequence, which includes two parts, one ($K$) is the number of key points to fully represent the current pose, the other ($D$) is the dimension of each key point, generally three, which are the two coordinates of the planes and the depth estimation.

Here we use state-of-the-art pose estimation models such as Vitpose\cite{xu2022vitpose} and BlazePose\cite{bazarevsky2020blazepose}. The pose estimation algorithms themselves are not designed by us, but we introduce pose information into the action counting task, which is a novel design not explored by previous work.

Moreover, our pose-level poses estimation processes the primitive information of video, which is similar to the feature extraction network in all video-level algorithms such as I3D\cite{carreira2017quo}, VideoSwinTransformer\cite{liu2022video}, and TSN\cite{wang2016temporal}. But the difference is that the result of video-level incorporates all information, while pose-level only produces core information, which greatly improves the performance. Additionally, using pose information can contribute to the lightweight of model. For instance, for a 1024-frame video, video-level feature extraction with an output dimension of 512 would produce a data volume of $1024\times 512=524288$, while using pose information with 33 key points produces a data volume of only $1024\times 33 \times 3=101376$.

\subsection{Encoding Poses with Transformer}
\label{second}
Here we specify our data representation for the Transformer Encoder, which requires input batch size, sequence length, and embedding dimensions. In our pose-level approach, each frame is a batch, the number of key points in each frame is the sequence length, and the feature dimension of each key point is the embedding dimension.

First we get the pose of each frame ${p_i}\in \mathbb{R}^{D\times K}$ through the Pose Estimation Network, where $i\in {1, 2, \dots, T}$ is the frame index, $K$ is the number of key points, and $D$ is the dimension of each key point. We further define $p_i = {\{k_j\}}^{K}_{1}$ to represent each key point, where $k_j\in \mathbb{R}^D$, and we embed it to obtain richer information. Our embedding projection $\mathrm{\bf{E}}$ is a simple MLP network with ReLU as the activation function. These calculations can be defined as:
\begin{equation}
\begin{split}
\mathrm{\bf{Z}}^0 = [\mathrm{\bf{E}}(k_1), \mathrm{\bf{E}}(k_2), \dots, \mathrm{\bf{E}}(k_K)]^T
\end{split}
\end{equation}
where $\mathrm{\bf{E}}(k_j)\in \mathbb{R}^{D^{\prime}}$ is the embedding feature. Then the next Transformer takes $\mathrm{\bf{Z}}^0$ as input and encodes it with self-attention. Given $\mathrm{\bf{Z}}^0\in \mathbb{R}^{K\times D^{\prime}}$ with $K$ key point features, each of which is $D^{\prime}$-dimensional, $\mathrm{\bf{Z}}^0$ is projected using $\mathrm{\bf{W}}_Q\in \mathbb{R}^{D^{\prime}\times D_q}$, $\mathrm{\bf{W}}_K\in \mathbb{R}^{D^{\prime}\times D_k}$, $\mathrm{\bf{W}}_V\in \mathbb{R}^{D^{\prime}\times D_v}$, where $D_k=D_q$, to extract feature representations query($\mathrm{\bf{Q}}$), key($\mathrm{\bf{K}}$) and value($\mathrm{\bf{V}}$), which can be defined as:
\begin{equation}
\begin{split}
&\mathrm{\bf{Q}}=\mathrm{\bf{Z}}^0\times \mathrm{\bf{W}}_Q\\
&\mathrm{\bf{K}}=\mathrm{\bf{Z}}^0\times \mathrm{\bf{W}}_K\\
&\mathrm{\bf{V}}=\mathrm{\bf{Z}}^0\times \mathrm{\bf{W}}_V
\end{split}
\end{equation}
and the output of self-attention can be computed as:
\begin{equation}
\begin{split}
\mathrm{\bf{Attn}}=\mathrm{Softmax}(\frac{\mathrm{\bf{Q}}\mathrm{\bf{K}}^T}{\sqrt{D_q}})\mathrm{\bf{V}}
\end{split}
\end{equation}
where $\mathrm{\bf{Attn}}\in \mathbb{R}^{K\times D^{\prime}}$. Also, we use common multi-head self-attention (MHSA) to make several self-attention operations calculate in parallel.

Now we introduce the overall architecture of Transformer Encoder, which has $L$ layers with each layer consisting of MHSA and MLP blocks. Also, LayerNorm and Residual Connection are applied before and after every MHSA or MLP block, respectively. Because the number of key points of each frame is  a bit less, so our encoder does not include the downsampling module that other models may have. The overall process can be defined as:
\begin{equation}
\begin{split}
&\mathrm{\bf{\hat{Z}}}^l = \mathrm{MHSA}(\mathrm{LN}(\mathrm{\bf{Z}}^{l-1})) + \mathrm{\bf{Z}}^{l-1}\\
&\mathrm{\bf{Z}}^l = \mathrm{MLP}(\mathrm{LN}(\mathrm{\bf{\hat{Z}}}^l)) + \mathrm{\bf{\hat{Z}}}^l
\end{split}
\end{equation}
where $\mathrm{\bf{Z}}^{l-1}$, $\mathrm{\bf{\hat{Z}}}^l$, $\mathrm{\bf{Z}}^l\in \mathbb{R}^{K\times D^{\prime}}$.


\subsection{Pose Mapping}
\label{third}
Taking the Encoder output $\mathrm{\bf{Z}}^L\in \mathbb{R}^{K\times D^{\prime}}$ as input, Pose Mapping module outputs probability scores $\mathrm{\bf{S}}\in \mathbb{R}^{C}$ of the current frame over all action classes. We perform binary classification after Sigmoid activation for each class, with the two salient poses of each class represented by the same bit data. To realize such a module, we use a very lightweight MLP network, which avoids the complexity. First, the two dimensions $K$ and $D^{\prime}$ of $\mathrm{\bf{Z}}^L$ are flattened into $\mathbb{R}^{KD^{\prime}}$, and then it passes through an MLP module, where the output channels is set to $C$, which can be defined as:
\begin{equation}
\begin{split}
\mathrm{\bf{S}} = \sigma(\mathrm{MLP}(\mathrm{Flatten}(\mathrm{\bf{Z}}^L)))
\end{split}
\end{equation}
where $\sigma$ represents the Sigmoid activation function.

With such Pose Mapping, we can obtain the scores of single frame. It should be noted that we extract the poses of all frames, and use the convenience of matrix operations to obtain scores in parallel, which is actually consistent with the idea of mini batch. So at last, we combine the scores of all frames to get the video score matrix $\mathrm{\bf{\hat{S}}}\in \mathbb{R}^{C\times T}$, where $T$ represents the number of frames in the current video. 


\subsection{Action-trigger Module}
\label{fourth}
We use the lightweight Action-trigger Module to obtain the final output $Y$, the repetitive action count, which has a time complexity of $\mathcal{O}(n)$. First, we get the scores $S_c\in \mathbb{R}^T$ of a given action class from $\mathrm{\bf{\hat{S}}}$. Then, we scan all frames and use the action-trigger mechanism to count when the two salient poses of the action class occur sequentially. We set upper and lower bounds to distinguish the scores of the two salient poses, which cluster non-salient poses in the middle and easily classify the salient poses to the two ends.

\subsection{Losses and Metric Learning}

The modules need to be trained are Embedding, Transformer Encoder and Pose Mapping, and because we perform binary classification for each class, so we use the Binary Cross Entropy Loss, which can be defined as follows:
\begin{gather}
\mathcal{L}_{bce} = -\frac{1}{N}\sum\limits_{i=1}^{N}(\frac{1}{C}\sum\limits_{j=1}^{C}loss(i,j))  \\
 loss(i,j)=y_{ij}\log p_{ij} + (1-y_{ij})\log(1-p_{ij})
\end{gather}
where $N$ represents the batch size (in our method, each frame is a batch), $C$ represents the number of classes, $y$ and $p$ are the labels and our predictions, respectively.

Moreover, we use Metric Learning to improve our Encoder and introduce the Pose Triplet Loss. Given a pose, Encoder produces higher-level features $\mathrm{\bf{Z}}^L$, which should be more representative. As shown in Figure \ref{fig5}, we achieve this with Triplet Margin Loss function, which selects anchors, same class positive samples, and different classes negative samples in a batch. It can be expressed as:
\begin{equation}
\begin{split}
\mathcal{L}_{tri} = \mathrm{max}(\mathrm{CS}(a,p)-\mathrm{CS}(a,n)+\mathrm{margin},0)
\end{split}
\end{equation}
where $a$, $p$, $d$ are anchors, positive and negative samples, and $\mathrm{CS}$ represents the Cosine Similarity to measure the distance between features. We pay more attention to hard samples, where the distances between anchors and negative samples are even smaller than those of positive samples. After Metric Learning, the poses of each action can be distinguishable, which cluster in the high-level space.

At last, our overall training combines these two losses:
\begin{equation}
\begin{split}
\mathcal{L} = \mathcal{L}_{bce} + \alpha\mathcal{L}_{tri}
\end{split}
\end{equation}
where $\alpha$ is the weight factor to control the two losses in the same numeric scale.
\subsection{Implementation Details}

\noindent{\bf Training.} We use the \emph{RepCount-pose} and \emph{UCFRep-pose} dataset we created to train our model. Only the frames with salient poses are inputted into the network instead of the entire video to speed up the fitting.

\noindent{\bf Inference.} During inference, the entire video sequence is inputted into the model. The poses of all frames pass through the Encoder and Pose Mapping, and then enter the Action-trigger Module to output the repetitive count.

\section{Individual modules}
\label{sec: modules}
What follows is a detailed step-by-step presentation of the stochastic analysis pipeline. % 
We follow the natural structure of the code for clarity as we introduce each module individually. %
To start, we present the {\tt preprocessing} module which pre-conditions the time-domain strain from \gls{gw} detectors for spectral analysis. %
In {\tt spectral}, we explain the power spectrum and cross-spectrum calculations, which produce the $P_{I, f}$ and $C_{IJ, f}$ spectra in Eqs.~\eqref{eq:Omega} and~\eqref{eq:Variance}. %
We then describe {\tt postprocessing}, which includes the averaging procedures employed over large datasets to obtain an optimal estimate of the signal amplitude starting from the quantities in Eqs.~\eqref{eq:Omega} and~\eqref{eq:Variance}, and knowledge of the expected spectral shape. %
In {\tt delta-sigma cut} and {\tt notch}, we present modules which focus on data quality checks, and the implementation of relevant time-domain and frequency-domain data cuts. % 
We then describe the built-in parameter estimation module {\tt pe}, based on {\tt Bilby}~\cite{Ashton:2018jfp}, a Python--based Bayesian inference library widely used in GW data analysis. %
Finally, we present the {\tt simulator} module, which includes different mock-data injection techniques for \gls{gwb} study and detection validation. %

A schematic of the {\tt pygwb} package is presented in Fig.~\ref{fig:flowchart}. %
This includes the manager objects {\tt Interferometer}, {\tt Baseline}, and {\tt Network}, presented in Sec.~\ref{sec: manager objects}.

\begin{figure}
    \centering
    \includegraphics[width = 0.65\textwidth]{pygwb_modules.pdf}
    \caption{Schematic overview of {\tt pygwb} analysis flow. In blue squares, we show the manager objects of the code that handle the analysis internally. These manager objects query (red arrows) different modules for specific objects, calculations, or quantities (rounded bubbles), imported (grey arrows) by either internal (i.e., within {\tt pygwb}) or external modules (i.e., outside of {\tt pygwb}). Internal modules are indicated in red, while external modules are indicated in green. }
    \label{fig:flowchart}
\end{figure}

%To be organised according to the pipeline worflow - yet good as standalone sections.

\subsection{\tt preprocessing}\label{sec:preproc}
\begin{table*}[htb]
\normalsize
\begin{center}
\renewcommand\arraystretch{1.2}
\setlength{\tabcolsep}{2.4mm}{
\begin{tabular}{lll}
\toprule
\textbf{Training transformation} & \textbf{Test transformation} & \textbf{CAM transformation} \\
\midrule
\texttt{RandomResizedCrop(224),}       & \texttt{Resize(256),}        & \texttt{Resize((224,224)),} \\
\texttt{RandomHorizontalFlip(0.5),}    & \texttt{CenterCrop(224),}    & \texttt{ToTensor(),} \\
\texttt{ColorJitter(63/255),}          & \texttt{ToTensor(),}         & \texttt{Normalize().} \\
\texttt{ToTensor(),}                   & \texttt{Normalize().}        & \\
\texttt{Normalize().}                  &                            & \\
\bottomrule
\end{tabular}}
\end{center}
\caption{Uniform image preprocessing methods for the three datasets. ``CAM transformation'' denotes the preprocessing method for generating CAM~\cite{zhou2016cam}. The mean and standard deviation parameters of \texttt{Normalize()} are omitted. Note that FOSTER~\cite{wang2022foster} additionally applies AutoAugment~\cite{cubuk2019autoaugment} in the training transformation, and we followed it for fair comparison.}
\label{tab_preprocessing}
\end{table*}

\subsection{\tt spectral}\label{sec:spectral}
The role of the {\tt spectral} module is to compute, for each time segment of duration $T$, the discrete frequency domain quantities $C_{IJ, f}$, $P_{I, f}$ and $P_{J, f}$ used in Eqs.~\eqref{eq:Omega} and \eqref{eq:Variance}. The one-sided cross- and auto-power spectral densities $C_{IJ}$ and $P_I$, respectively, of a single segment are defined as
\begin{equation} \label{eq:csd_psd}
    C_{IJ, f} = \frac{2}{T} \tilde s_{I, f}^* \tilde s_{J, f}\,, \ \ \ \ P_{I,f} = \frac{2}{T} |\tilde s_{I, f}|^2 \ ,
\end{equation}
where $\tilde s_f$ are \glspl{dft} of $s(t_k)$ %$t_k = 0,\,\delta t, \, 2 \times \delta t, \,\cdots, \, T-\delta t$, 
 defined by
\begin{equation}
   \tilde s_{f} \equiv \sum_{t_k=0}^{T-\delta t} s(t_k) \, e^{-i 2 \pi m t_k / T} \,,
\end{equation}
where %$f = 0,\, \delta f,\, 2 \times \delta f,\, \cdots,\, \frac{1}{2 \times \delta t}$. $f = m\,\delta f$, where
$f=m \delta f$, with $m$ a natural number between 0 and $1/(2\,\delta t \,\delta f)$, and $\delta f$ the desired frequency resolution, chosen such that $1/(2\delta t \delta f)$ is an integer. %

The segmented data are windowed before calculating Fourier transforms to avoid spectral leakage due to discontinuities at the ends of the segments. %
The user may define their own choice of window, which defaults to the \textit{Hann} window if none is selected. %
The {\tt spectral} module uses methods from {\tt scipy.signal} to calculate spectrograms $\tilde s^t_{f}$ of the given data, which are then used to calculate the list of $C^t_{IJ, f}$, $P^t_{I, f}$, and $P^t_{J, f}$ quantities, corresponding to different time segments labelled by $t$ in the dataset. %
By default, these are calculated with a 50\% time overlap to account for the impact of the windowing. However, the user may redefine the overlap between consecutive segments to be used throughout the analysis to better suit any choice of window. %~\arr{style comment: some sections don't have separation between paragraphs. Maybe we should decide upon one style or the other.}%

Different averaging procedures are employed to reduce the fluctuations in the spectra estimates and compress the data. %
The procedures we employ are selected to minimize sensitivity loss. %
In the estimates of $P^t_{I, f}$ and $P^t_{J, f}$ %, instead of performing a DFT on the entire segment of data of duration $T$, 
we employ Welch's estimation method of \glspl{psd}~\cite{pwelch}, which is known to produce minimum variance estimates of the \gls{psd}, implemented as follows. %
Each segment is divided into sub-segments of duration $1/\delta f$ which are \gls{dft}ed individually. %
The auto-correlated power $|\tilde s_{I,f}|^2$ is then averaged over the sub-segments %to get $|s_I(f)|^2$ for the whole segment which is then used in Eq.~\ref{eq:csd_psd}
to obtain estimates of $P^t_{I, f}$ and $P^t_{J, f}$ for time $t$. %
This procedure returns spectra at the desired frequency resolution $\delta f$, which is typically much larger than the original resolution $1/T$ Hz. %
%\air{it would be good to mention the overlap factor here.}

As the power varies slowly with frequency\footnote{The power varies slowly with frequency except in very few bins, where narrow-band spectral artifacts or {\it lines} are present, as discussed in Sec.~\ref{sec:notch}.}, we can average over neighboring frequencies using a process known as coarse-graining~\cite{https://doi.org/10.48550/arxiv.2106.13785}. %
This is the default procedure employed in the \gls{csd} estimation. %
The resulting spectra are returned at the desired frequency resolution $\delta f$. %, we average $s_I^*(f) s_J(f)$ over neighboring frequency bins to get the desired frequency resolution of $\delta f$. %
Note that the data are zero-padded before calculating Fourier transforms for $C^t_{IJ, f}$ to avoid wrap-around problems arising from finite data \cite{LIGO_S1, whelan_CC_dcc, numerical_recipes2007}, and hence coarse-graining is required to achieve the desired frequency resolution. %
% coarse-graining mostly arises due to the zero-padding which is needed to avoid the wrap-around problems and needs to be done before taking the Fourier transform.
Zero-padding simply entails appending a vector of zeros equal to the length of the segment before taking the Fourier transform. %


\begin{figure}
    \centering
    \includegraphics[width=0.5\textwidth]{spectral_1.pdf}
    \caption{An example of cross- and auto-power spectral densities of the Hanford and Livingston detector data during O3.}
    \label{fig:psd_csd_spectra}
\end{figure}

\begin{figure}
    \centering
    \includegraphics[width=0.7\textwidth]{spectral_2.png}
    \caption{An example spectrogram showing two hours of LIGO   Hanford data during O3. The visible vertical columns correspond to noisy segments, which are usually removed from the analysis (see Sec.~\ref{Sec:DeltaSigma}).}
    \label{fig:psd_spectrogram}
\end{figure}
%Depending on the user input, it is possible to further average $P_I(f)$ and $P_J(f)$ over estimates from neighboring segments.
To further reduce fluctuations in the \gls{psd} estimates, the $P^t_{I, f}$ quantities are averaged over neighboring segments to obtain the final estimate $\bar{P}^t_{I, f}$ of the \gls{psd} at a given time $t$. %
This is appropriate as the noise in \gls{gw} detectors is (most often) approximately stationary over periods of a few minutes. %
We often refer to the initial (un-barred) quantities as ``naive'' and the final (barred) quantities as ``average'' estimates in the rest of this paper, to avoid confusion. %
By default, only nearest neighbors are used for the calculation, such that the \gls{psd} at time $t$ is an average of the naive \glspl{psd} calculated for times $t-T$ and $t+T$. %
The user may define any even number $D$ of segments to be used to perform this average, which are taken before and after the reference time $t$ such that the \gls{psd} is averaged over naive \glspl{psd} at times $t-D T/2$ and $t+DT/2$. % %

Fig.~\ref{fig:psd_csd_spectra} shows the cross- and auto-power \glspl{psd} of 192 s of data from the Hanford and Livingston detectors during O3, while Fig.~\ref{fig:psd_spectrogram} shows a two-hour spectrogram of Hanford data during O3, produced with the {\tt spectral} module. % 


\subsection{\tt postprocessing}\label{sec:postproc}
Once a set of data, comprised of an uninterrupted stretch of timeseries data, has been pre-processed and average \gls{psd} and \gls{csd} estimates have been calculated for each segment of data within the set, one can combine those separate time segments to construct a final, time-averaged estimate of the \gls{gwb} amplitude. %

Due to the aggressive windowing choice we typically make, and the subsequent overlapping of time segments, we must be careful in combining time segments together. %
The overlapping and windowing cause overlapping time segments to be correlated with one another. %
Within each processed set, individual time segments must be combined while accounting for this covariance. %
A detailed calculation and discussion of this covariance can be found in~\cite{lazz_romano_windowing_note}, while effective approximations to that full calculation can also be used (see, e.g. Sec. IIIB of~\cite{AinDalvi2015}). %

To start, we construct the estimate of the GWB in a single segment $t$. %
As detailed in Sec.~\ref{sec: GWB analysis}, the \gls{gwb} search is often framed in terms of constructing a point estimate for $\Omega_{\rm GW}(f_{\rm ref})$, the energy density of the \gls{gwb} at the specific frequency $f_{\textrm{ref}}$, assuming a power-law for the \gls{gwb} with spectral index $\alpha$. %
We refer to the estimator of this quantity as $\hat \Omega^\alpha_{\mathrm{ref}},$ in general, and for a single time segment of data, it can be constructed using a weighted average over the individual frequency bin estimators $\hat{\Omega}_f$ and ${\sigma}_f$ described in Eqs.~\eqref{eq:Omega} and~\eqref{eq:Variance} calculated per segment $t$, as
%
\begin{align}
\hat\Omega^\alpha_{\textrm{ref}, t} = \frac{\sum_{f} \hat\Omega_{t,f} H_{{\rm ref}, \alpha}(f)\bar\sigma^{-2}_{t, f}}{\sum_f H_{{\rm ref}, \alpha}^2(f) \bar\sigma^{-2}_{t,f}},\label{eq:omega_alpha}\\
\sigma^\alpha_{\textrm{ref},t} = \left[\sum_f H^2_{{\rm ref}, \alpha}(f)\bar\sigma_{t,f}^{-2}\right]^{-\frac{1}{2}},\label{eq:sigma_alpha}
\end{align}
% where $w(f)$ is the re-weighting function
% \begin{equation}
% w(f) = \left(\frac{f}{f_{\textrm{ref}}} \right)^{\alpha}.
% \end{equation}
where the rescaling $H_{{\rm ref}, \alpha}(f)$ is defined in Eq.~\eqref{eq:Salpha}. %
The average variance spectrum per segment, $\bar\sigma^2_{t, f}$, is calculated using average \glspl{psd} described in Sec.~\ref{sec:spectral}. %
These broadband quantities can be calculated for each time segment $t$, and then this set of estimators at each time can be combined to account for the overlap between time segments discussed above. %
We first lay out how to perform this combination assuming we have calculated the quantities above for each individual time segment. %
Then, we discuss how to alternatively average the estimators in each frequency bin over time independently, before combining them into an integrated quantity at the end. %
The latter calculation is normalized such that it gives the same result as the former. %
To avoid heavy notation we drop the bars that indicate average quantities in the rest of this section -- all variances used for the following calculations are average variances as defined above. %

To construct an estimator for the \gls{gwb} using a set of measurements in short, overlapping time segments, we first combine the segments that are non-overlapping. %
If the overlap between segments is $50\%$ or less, then this amounts to separately performing inverse-noise-weighted averaging over the even- and odd-indexed segments:
\begin{align}
\label{eq:odd_even_sigma}
    \sigma_{\mathrm{odd}}^2 &= \frac{1}{\sum_{t\in \mathrm{odd}}\sigma_t^{-2}}\\
    \label{eq:odd_even_omega}
    \Omega_{\textrm{odd}} &= \frac{\sum_{t\in \textrm{odd}}\Omega_t\sigma_{t}^{-2}}{\sum_{t\in\textrm{odd}}\sigma_{t}^{-2}},
\end{align}
where %we have assumed we have averaged over frequency first. %
the quantities $\Omega_t\equiv\hat\Omega^\alpha_{\textrm{ref}, t}$ and $\sigma_t\equiv\sigma^\alpha_{\textrm{ref}, t}$ for each time segment $t$. %
Analogous expressions are calculated for $\Omega_{\textrm{even}}$ and $\sigma_{\textrm{even}}$. %
Subscripts refer to even/odd time segments, and we drop here the subscripts $_{\rm GW}$, $_{\rm ref}$, and $_\alpha$ used to construct the integrated quantities to lighten the notation. %
We refer to the final, frequency- and time-averaged estimate as $\hat\Omega_\mathrm{ref}$ for now.

Next, we calculate the cross-covariance between point estimates in the odd and even segment combinations~\cite{lazz_romano_windowing_note},
\begin{align}
\sigma_{oe}^2 =\sigma_{eo}^2 &\equiv \langle \Omega_\mathrm{odd}\Omega_\mathrm{even}\rangle - \langle \Omega_\mathrm{odd}\rangle\langle \Omega_\mathrm{even}\rangle\\
&= \frac{1}{2}\frac{\bar{w}^4_\mathrm{ovl}}{\bar{w}^4}\left[\sigma_{\mathrm{odd}}^2 + \sigma_{\mathrm{even}}^2-\frac{1}{2}\sigma_{\mathrm{odd}}^2\sigma_{\mathrm{even}}^2\left(\sigma_{1}^{-2} + \sigma^{-2}_{2M-1}\right)\right],
\end{align}
where $M$ is the number of \textit{independent} segments and so $2M-1$ is the total number of overlapping segments, with the {\it window factors} $\bar{w}^4_\mathrm{ovl}$ and $\bar{w}^4$ as defined in App.~\ref{sec:app_window}. %
For the sake of compactness, we rewrite this as
\begin{align}
    \sigma_{oe}^2 &= \frac{k}{2}\sigma_{\mathrm{odd}}^2\sigma_{\mathrm{even}}^2 \sigma_{IJ}^{-2}\,,\\
    \sigma_{IJ}^2 &= \left[\sigma_{\mathrm{odd}}^{-2} 
 + \sigma_{\mathrm{even}}^{-2} - \frac{1}{2}\left(\sigma_{1}^{-2} + \sigma^{-2}_{2M-1}\right)\right]^{-1}\,,
\end{align}
where $k = \bar{w}^4_\mathrm{ovl} / \bar{w}^4$.

The covariance matrix between even/odd segment sets is then defined as
\begin{align}
    \bm{C} = \begin{pmatrix}\sigma_{\mathrm{odd}}^2 & \sigma_{\mathrm{oe}}^2 \\
            \sigma_{\mathrm{oe}}^2 & \sigma_{\mathrm{even}}^2
    \end{pmatrix},
\end{align}
which we use to construct the optimal combination of segments to obtain the point estimate $\hat{\Omega}_{\mathrm{ref}}$ and its variance $\sigma_{\mathrm{ref}}^2$. These are given by:
\begin{align}
    \hat\Omega_{\mathrm{ref}} &= \frac{\sum_{i=1}^2 \lambda_i \Omega_i}{\sum_{j=1}^2 \lambda_j},\label{eq:ptest_postpoc}\\
    \sigma_{\mathrm{ref}}^2 &= b^2_{\rm avg}\qty({\sum_{k=1}^2 \lambda_k})^{-2}\sum_{i=1}^2\sum_{j=1}^2 \lambda_i C_{ij} \lambda_j, \label{eq:sigma_postpoc}
\end{align}
with
\begin{align}
    \lambda_i = \sum_{j=1}^2 \left(\bm C^{-1}\right)_{ij}\,,
    \label{eq:lambda_i_broadband}
\end{align}
where $i,\,j$ indices label odd/even quantities. %
The bias factor $b_{\rm avg}$ which arises due to harsh windowing of the data has been included in Eq.~\eqref{eq:sigma_postpoc}. %
The derivation of the bias factor is described in App.~\ref{sec:app_window}. %
If combining over non-overlapping segments, then $\sigma_{\mathrm{oe}}^2 = 0,$ and this method reduces to the typical inverse-noise-weighted average that one would expect. %

The above expressions are for a broadband estimator, but in practice the {\tt postprocessing} module combines over time segments before combining over frequency bins. %
We refer to the estimated narrowband quantities as $\hat\Omega_{\mathrm{ref}, f}$ and $\sigma_{\mathrm{ref}, f}$. This notation indicates that, once a power-law spectral model is applied, the estimate in a frequency bin represents an estimate of the GWB at the reference frequency of the power law, assuming the chosen spectral shape.

We normalize $\hat\Omega_{\mathrm{ref}, f}$ and $\sigma_{\textrm{ref}, f}$ such that, when performing a weighted average over frequency bins \textit{after} combining  overlapping time segments we get the same result as Eqs.~(\ref{eq:ptest_postpoc}) and (\ref{eq:sigma_postpoc})  (which assume construction of a broadband estimator \textit{before} combining overlapping time segments). This results in the following expression for $\sigma_{\textrm{ref}, f}^{-2}$,
\begin{align}
\sigma_{\textrm{ref}, f}^{-2} &= b^{-2}_{\rm avg}\frac{\left[\sigma_{\mathrm{odd}, f}^{-2} + \sigma_{\mathrm{even}, f}^{-2} - k\sigma_{IJ, f}^{-2}\right]}{1 - \frac{k^2}{4}\sigma_{\mathrm{odd}}^2\sigma_{\mathrm{even}}^2\sigma_{IJ}^{-4}}\,,
\end{align}
and a corresponding expression for $\hat\Omega_{\mathrm{ref}, f}$,
\begin{align}
\hat\Omega_{\textrm{ref}, f} &= \frac{\Omega^{}_{\textrm{odd}, f}\sigma_{\mathrm{odd}, f}^{-2}\left(1 - \frac{k}{2}\sigma_{\mathrm{odd}}^2\sigma_{IJ}^{-2}\right) + \Omega^{}_{\textrm{even}, f}\sigma_{\mathrm{even}, f}^{-2}\left(1 - \frac{k}{2}\sigma_{\mathrm{even}}^2\sigma_{IJ}^{-2}\right)}{\sigma_{\mathrm{odd}, f}^{-2} + \sigma_{\mathrm{even}, f}^{-2} - k\sigma_{IJ, f}^{-2}}\,.
\end{align}
The even and odd estimators for each frequency bin are defined as in Eqs. (\ref{eq:odd_even_sigma}) and (\ref{eq:odd_even_omega}), except applied to individual bin-by-bin estimators calculated at each time segment. As discussed above, these expressions have been normalized such that 
\begin{align}
   \hat\Omega_{\textrm{ref}} = \frac{\sum_{f}\hat\Omega_{\textrm{ref}, f} \sigma_{\textrm{ref}, f}^{-2}}{\sum_f \sigma_{\textrm{ref}, f}^{-2}}\,, \\ 
   \sigma_{\textrm{ref}}^{2} = \left[\sum_f \sigma_{\textrm{ref}, f}^{-2}\right]^{-1}\,.
\end{align}

The \texttt{postprocessing} module implements the above expressions to estimate $\hat\Omega_{\mathrm{ref}, f}$ and $\sigma_{\mathrm{ref}, f}$ at a fixed $\alpha$, and returns them in form of an \texttt{OmegaSpectrum} object, which sub-classes the classic {\tt gwpy.FrequencySeries} and adds two key attributes: the spectral index $\alpha$ and the reference frequency $f_{\rm ref}$ at which the spectrum is calculated. %
By default, \texttt{pygwb} assumes a power-law spectral index $\alpha=0$ and a reference frequency $f_{\rm ref} = 25$ Hz when constructing the above estimators. % 
To explicitly include the $\alpha$ dependence in our results, we refer to the final postprocessed spectra as $\hat\Omega^\alpha_{\mathrm{ref}, f}$ and $\sigma^\alpha_{\mathrm{ref}, f}$.

One of the advantages of averaging over time before averaging over frequency is that one can reweight $\hat\Omega^\alpha_{\mathrm{ref}, f}$ and $\sigma^\alpha_{\mathrm{ref}, f}$ to be estimators for different choices of $\alpha$ without needing to average over all time segments again for a new choice of $\alpha$. %
The \texttt{OmegaSpectrum} object has a built-in method to perform a reweighting to change either $f_\mathrm{ref}$ or $\alpha$ used to calculate $\Omega_\mathrm{ref}$, employing the relation %
\begin{equation}
    \Omega^{{\rm ref}_1, \alpha_1}_{\rm GW}(f) = \Omega^{{\rm ref}_2, \alpha_2}_{\rm GW}(f) \frac{H_{{\rm ref}_1, \alpha_1}(f)}{H_{{\rm ref}_1, \alpha_2}(f)}\,,    
\end{equation}
derived using Eq.~\eqref{eq:omega-plaw}, which implies the following relation between amplitudes at different reference frequencies,
\begin{equation}
    \Omega_{\rm ref_1} = \Omega_{\rm ref_2} \frac{H_{{\rm ref}_2, \alpha}(f)}{H_{{\rm ref}_1, \alpha}(f)}\,.
\end{equation}
This allows to quickly calculate time- and frequency-averaged estimates of the \gls{gwb} amplitude associated with a specific power-law model. %

The default Hubble constant $H_0$, required in the scaling $S_0(f)$ in Eq.~\eqref{eq:S0}, is chosen to be $H_0 = 67.7$~km/(Mpc$\cdot$s), drawn from the Planck 2018 observations~\cite{Planck2018} and imported directly from the {\tt astropy} package. %
This is an attribute of the \texttt{OmegaSpectrum} and may be re-set by the user.

\subsection{\tt delta-sigma cut}\label{Sec:DeltaSigma}
In general, the noise level in ground-based detectors changes slowly on time-scales of tens of minutes to hours. %\kt{In spectral, it is said that the noise is stationary on the order of a few minutes.}. 
The variance $\sigma^2_{\rm GW}$~(see Eq.~\eqref{eq:Variance}) associated to each segment is an indicator of that level of noise, which typically changes at roughly the percent level from one data segment to the next. %
However, there are occasional very loud disturbances to the detectors, such as glitches, which violate the Gaussianity of the noise. %
Auto-gating procedures are in place, as explained in Sec.~\ref{sec:preproc}, to remove loud glitches from the data; however the procedure does not remove all non-stationarities. %  
To avoid biases due to these noise events, an automated technique to exclude them from the analysis has been developed~\cite{LIGO_S4}. %
To this end, the {\tt pygwb} package includes the {\tt delta-sigma cut} module, which flags specific segments to be cut from the analyzed set. %
Note that inverse-noise-weighting, as explained in Sec. \ref{sec:postproc}, also reduces the effect of non-Gaussian noise artifacts. %

The ``\textit{$\Delta\sigma$} cut'' calculation consists in comparing the $\sigma_{\rm GW}$ of a segment $t$, $\sigma_t$, to that of its nearest neighbors and flagging it for removal in case their values differ by more than a chosen threshold. %
%, segment $I$ is removed from the analysis to avoid a bias in the point estimate, Eq.~(\ref{eq:Omega}), and $\sigma$, Eq.~(\ref{eq:Variance}). 
%The non-stationarity condition can be expressed as 
Conceptually, the calculation is based on the simple inequality,
%
\begin{equation}
\label{eq:dsc_condition}
    \frac{|\sigma_i - \sigma_{i+1}| + |\sigma_i - \sigma_{i-1}|}{2\sigma_i}>{\rm threshold}\,,
\end{equation}
%
where $i$ is a segment index. %
However, in practice we perform an analogous, more sophisticated calculation, which compares the naive and average segment variances ${\sigma}_{t, \alpha}$ and $\bar{\sigma}_{t, \alpha}$. %
These are derived from the unweighted naive and average segment variances computed with Eq.~\eqref{eq:Variance} using naive and average \glspl{psd} per segment (see Sec.~\ref{sec:spectral} for details), respectively, which are then reweighted by the index $\alpha$, as shown in Eq.~\eqref{eq:sigma_alpha}. %
The final expression employed in the calculation is 
\begin{equation}
    \frac{|\bar{\sigma}_{t, \alpha} b_{\rm avg} - \sigma_{t, \alpha} b_{\rm nav} |} {\bar{\sigma}_{t, \alpha} b_{\rm avg}}>{\rm threshold}\,,
\end{equation}
which also takes into account the bias factors that arise due to the different impacts of windowing on naive and average quantities (see App.~\ref{sec:app_window} for details). %
Past analyses have used a threshold of 0.2, as this has been shown to yield a Gaussian distribution for the remaining (un-cut) segment variances~\cite{LIGO_S5}. %
For more details on this choice see~\cite{Meyers_thesis}.
\begin{figure}
    \centering
    \includegraphics[width=0.55\linewidth]{deltasigmacut_1.pdf}
    \caption{In this plot, power-law spectra with different spectral indices are compared to the O3 sensitivity curve of \gls{ligo}-Livingston. Each power law is sensitive to a different frequency band. This makes it necessary to repeat the \textit{$\Delta\sigma$ cut} assuming different $\alpha$, since this allows to check for noise fluctuations in the whole range of frequencies analyzed. The O3 sensitivity curve for \gls{ligo}-Livingston was retrieved from \cite{O3_sens_curve}.}
    \label{fig:delta_sigma_cut_alphas}
\end{figure}

The $\Delta\sigma$ cut calculation is performed %over cross-correlated data obtained 
assuming different spectral indices $\alpha$ as each power law is sensitive to a different frequency band (see Fig.~\ref{fig:delta_sigma_cut_alphas}). %
The union of all the segments flagged for each $\alpha$ is taken, leading to a full list of segments to discard from the analysis. %
The default choice of $\alpha$ values in the {\tt delta-sigma cut} module is $ \alpha=\{-5, 0, 3\}$, as this adequately covers most of the frequency band of \gls{lvk} searches, from 20-1726 Hz~\cite{LIGO_O3}, at current sensitivity. %
These may be easily modified by the user. %
This would be especially recommended if the search were carried out over a different set of frequencies, or for data from detectors with a spectral sensitivity different than that for Advanced LIGO, Advanced Virgo, or KAGRA. %
Often, the value of $\alpha=5$ is also considered, and was employed in the most recent LVK isotropic search~\cite{O3-iso-KAGRA:2021kbb}. %
The analysis performed at a spectral index $\alpha=-5$ is mostly sensitive to non-stationary effects in the $\sim 15-50$ Hz range, while in the case of $\alpha = 0$ the analysis is sensitive to effects between $\sim 40-80$ Hz, for $\alpha = 3$ from $\sim 90-500$ Hz, and finally $\alpha = 5$ is most sensitive to fluctuations at the higher frequencies, above $\sim 500$ Hz. %
These higher frequencies are not always included in this sort of analysis due to reduced sensitivity in this range, hence $\alpha = 5$ is not a default value used for the cut.

% \footnote{Work described in the following internal reference: https://stochastic-alog.ligo.org/aLOG/index.php?callRep=339959}

As the \textit{$\Delta\sigma$} cut only compares neighboring segments, long stretches of loud noise--contaminated data can pass the test and be included in the analysis. %
We are currently working to improve this by monitoring and flagging longer stretches of non-stationary noise and prolonged loud noise conditions.


\subsection{\tt notch}\label{sec:notch}

Ground-based laser interferometers present many narrow-frequency noise artifacts which are typically persistent in time, and are generally referred to as noise lines. %
Some examples are calibration lines and mechanical resonances \cite{LIGO:2021ppb,Virgo:2022ysc,vanRemortel_2022}. %
The {\tt notch} module provides the framework to properly deal with these noise lines in the case of the search for an isotropic \gls{gwb}. %
The solution is to ``notch out'' these noise lines, i.e., set the values of the spectra at the affected frequency bins to zero. %
Note that the {\tt notch} module is not built to identify these lines, as this is typically done by detector characterization experts working closely with instrumentalists running the detectors. %
Rather, the final product of the {\tt notch} module is a frequency mask which may be applied to the relevant spectra in the analysis. %

The key object of the {\tt notch} module is the {\tt StochNotchList}, which is a list of {\tt StochNotch} objects. %
A {\tt StochNotch} object represents a physical noise line which has been identified and needs to be removed from the data analysis. %, e.g. a calibration line. %
The object has a minimum and maximum frequency indicating the contaminated frequency region. %
Furthermore, it also comes with a descriptive string which allows the user to keep track of the reason why the line was notched. %
All the different {\tt StochNotch} objects for a certain analysis are then stored in the {\tt StochNotchList} which contains the entire list of lines to be notched from the analysis. %

The notch mask used to apply a set of notches within the analysis is constructed conservatively, such that any frequency that has overlapping frequency content with the noise lines defined in the {\tt StochNotchList} will be removed when applying the notch mask. %
%Concretely, the \gls{gwb} estimator $\hat\Omega^\alpha_{{\rm ref},f}$ is calculated at the discrete frequency $f$. %
To maintain generality, we discuss here a generic estimated spectrum $\hat\Omega_{f}$, where its value at frequency $f$ estimates $\Omega_{\textrm{GW}}(f)$ in the frequency range [$f - \delta f/2,f + \delta f/2$], where $\delta f$ is the chosen frequency resolution, as defined in Sec.~\ref{sec:spectral}. %
If a noise line has any overlap with the interval [$f - \delta f/2,f + \delta f/2$], the $f$ frequency bin is excluded. %
This implies that a hypothetical delta-peak noise line at $f +\delta f/2$, leads to notching both $f$ as well as $f + \delta f$. %
\begin{figure}
    \centering
    \includegraphics[width = 0.55\textwidth]{notch_1.pdf}
    \caption{Example of how the notching of noise lines (orange curve) applied to the discrete measurements of the spectrum $\hat\Omega_{\textrm{GW}, f}$ (blue stars) leads to a final set of measurements (red dots). % after excluding possible contaminate frequency bins. 
    The vertical shaded regions indicate the bins, where even bins are white and odd bins are light blue.    
    The orange line traces out the noise lines such that a noise line is present where the orange curve is zero. 
    The analyzed data spans $[5.0, \,6.875]$ Hz, in the un-shaded region. %
    In this example there are five noise lines, from left to right: a noise line ending at the lowest frequency bin, a noise line entirely contained in one frequency bin, a noise line spread across two frequency bins, a noise line spread across multiple frequency bins, and a noise line from bin-edge to bin-edge. %
    %The data analysed is represented by the blue stars, which is a vector starting at 5Hz, up to 6.875Hz, and a 0.125Hz resolution. 
    %The green stars are the remaining values of $C_{IJ}$ which are not zero after applying the notch mask based on the noise lines represented by the orange curve.
    After our notching procedure, the data is reduced to the bins marked by the red dots. %
    For visual convenience we have changed the amplitude in these remaining frequency bins by a factor 0.9. }
    \label{fig:notch_example}
\end{figure}
We present the creation of a notch mask with an example in Fig. \ref{fig:notch_example}, which illustrates how our conservative notching strategy excludes frequency bins based on different scenarios of noise lines. 

The current code is set up to apply the same notches to an entire stretch of data, which can be considered ``time-independent'' notching. %
To allow for time-dependent notching we could either use the current {\tt notch} module and split the analysis in different segments, each having their own notch list. Alternatively, one could extend the current module with an additional parameter which keeps track of which times have to be notched. Since typically the majority of the notched lines in the search for an isotropic \gls{gwb} with data from the \gls{ligo} and Virgo detectors are present during the entire dataset, the possible gain of implementing time-dependent notching is expected to be limited.

\subsection{\tt pe}\label{sec:pe}
Starting from an estimate of the \gls{gwb} spectrum $\hat{\Omega}_{{\rm GW}, f}$, with variance $\sigma^{2}_{{\rm GW}, f}$, it is possible to place stringent constraints on the \gls{gwb} amplitude using a hybrid frequentist-Bayesian approach. %
We consider the general case where we have a set of \gls{gwb} measurements $\hat{\Omega}^{IJ}_{{\rm GW}, f}$ from different detector pairs, or {\it baselines}, $IJ$. %
We define a Gaussian likelihood for $B$ pairs of detectors, 
\begin{equation}
\label{eq:likelihood}
p\qty(\hat{\Omega}^{IJ}_{{\rm GW}, f} | \mathbf{\Theta}) \propto\exp\left[  -\frac{1}{2} \sum_{IJ}^B \sum_f \left(\frac{\hat{\Omega}^{IJ}_{{\rm GW}, f}  - \Omega_{\rm M}(f|\mathbf{\Theta})}{\sigma^{IJ}_{{\rm GW}, f}}\right)^2  \right],
\end{equation}
where $\Omega_{\rm M}(f|\mathbf{\Theta})$ is the \gls{gwb} model and $\mathbf{\Theta}$ are its parameters. %
Bayes' theorem is used to obtain posterior distributions on the model parameters, %given the likelihood defined in Eq.~(\ref{eq:likelihood}) and the priors on those parameters:
\begin{equation}\label{eq:likelihood_params}
    p\qty(\mathbf{\Theta}|\hat{\Omega}^{IJ}_{{\rm GW}, f}) \propto p\qty(\hat{\Omega}^{IJ}_{{\rm GW}, f}| \mathbf{\Theta})\,p(\mathbf{\Theta})\,,
\end{equation}
where the priors $p(\mathbf{\Theta})$ are employed. %
In practice, when performing parameter estimation on a large dataset, we take the post-processed, {\it unweighted} (i.e., $\alpha=0$) estimate $\hat{\Omega}^{0, IJ}_{{\rm ref}, f}$ to be the measured \gls{gwb} spectrum in each frequency bin, and plug it into Eq.~\eqref{eq:likelihood}. %
Note that it is necessary for the input spectra used in parameter estimation to be unweighted as any other value would constitute a model choice and bias results. %

Within the {\tt pygwb} package, we include the {\tt pe} module to perform parameter estimation as an integral part of the analysis, which naturally follows the computation of the optimal estimate of the \gls{gwb}. %
This is a notable improvement compared to previous LVK analyses, where data products and parameter estimation were handled independently by packages in different programming languages. % %$\hat{C}^{IJ}(f_k)$ and $\sigma_{IJ}(f_k)$, after which a separate parameter estimation analysis followed.
Furthermore, the {\tt pe} module is a simple and user-friendly toolkit for any model builder to constrain their physical models with \gls{gw} data. %

The {\tt pe} module is built on class inheritance, with {\tt GWBModel} as the parent class. %
The methods of the parent class are functions shared between different \gls{gwb} models, e.g., the likelihood formulation in Eq.~(\ref{eq:likelihood}), as well as the noise likelihood, given by Eq.~(\ref{eq:likelihood}) with $\Omega_{\rm M}(f|\mathbf{\Theta})\equiv0$. %
It is possible to include calibration uncertainty by modifying the {\tt calibration\_epsilon} parameter, which defaults to 0. %
For details on the marginalization over calibration uncertainty, see App.~\ref{sec:app_calibration} and \cite{Whelan:2012ur}. %
The \gls{gw} polarization used for analysis is user-defined, and defaults to standard \gls{gr} polarization (i.e., tensor). %, still giving the user the flexibility to change the GW polarisation and explore models that predict a scalar or vector polarisation of the \gls{gwb}. %
More details on possible polarization choices can be found in Sec.~\ref{sec:baseline}. %
In our implementation of {\tt pe}, we rely on the {\tt Bilby} package~\cite{Ashton:2018jfp} to perform parameter space exploration, and employ the sampler {\tt dynesty} by default \cite{dynesty}. %
The user has flexibility in choosing the sampler as well as the sampler settings. %

Child classes in the {\tt pe} module inherit attributes and methods from the {\tt GWBModel} class. %
Each child class represents a single \gls{gwb} model, and combined they form a catalog of available \gls{gwb} models that may be probed with \gls{gw} data. %
The inheritance structure of the module makes it straightforward to expand the catalog, allowing users of the {\tt pygwb} package to add their own $\Omega_{\rm M}(f|\mathbf{\Theta})$ models. %
The flexibility of the {\tt pe} module allows the user to combine several \gls{gwb} models defined within the module. %
A particularly useful application of this is the modelling of a \gls{gwb} in the presence of correlated magnetic noise, as discussed in \cite{Meyers_2020}, or the simultaneous estimation of astrophysical and cosmological \gls{gwb}s \cite{PhysRevD.103.043023}. %
The {\tt pygwb} documentation~\cite{docs} contains information on the existing models in the catalog, with a description of the GWB models and their parameters. 

\subsection{\tt simulator}
\label{Sec:Simulator}
\vspace{0.2cm}
\section{Planning Framework}
\label{sec:fram}
\noindent %-block diagram with do path prediction, then trajectory generation with core and execute for ego
We tackle motion planning in structured environments by searching the velocity space $v$ over future times $s$. As depicted in Figure \ref{fig:simulator}, ROPT initially receives latest positions 
$\textbf{x}_i$, velocities $v_i$ and given map paths for the green ego car and $N_o$ other red cars (subscripted by $j$).\footnote{A traffic situation consists in this way of $N_o+1$ participants indexed with $i$.} Without prior knowledge, 
other trajectories are predicted on their respective paths with constant velocity up to a prediction horizon $s_h$. The goal of ROPT is now to optimize parameters $\boldsymbol{\uptheta}$ from
multiple velocity profiles $v^m$ for the ego agent. For this purpose, we alternate between adjusting $\boldsymbol{\uptheta}$ and evaluating risks $R(t)$, utility $U(t)$ and comfort $O(t)$ 
of the arising dynamic scene for the current time $t$. Once a defined cost threshold is satisfied for each sample, $v^m$ with the lowest cost is chosen and executed within a simulation step $\Delta t$ to obtain accelerations $a_i$ 
and jerks $r_i$. %actually when difference between function value or input value is under threshold %\footnote{We choose a plain double integrator, however vehicle dynamic models, such as the two-track model, can be added here if necessary.} %\footnote{Vehicle dynamic models, such as the two-track model, can be added if necessary. For our purposes, we choose however a plain double integrator.} 
In doing so, the simulator either updates other vehicles from measured fixed trajectories or controls them with their own planners. %Vehicle dynamic models, such as the two-track model, can be hereby added if necessary. For our purposes, we choose however a plain double integrator. %For the simulation, we use a simple double integrator. 


\subsection{Trajectory Optimization} 
\label{sec:trajgen}

%-Snake Trajectory \\
In complex scenarios with more than one risk source (i.e., driving in curve while crossing crowded intersection), the cost functional is non-convex. To overcome local minima, velocity shapes with high degrees of freedom are necessary. We choose for ROPT $n=4$ segments having fixed length $s_l=\unit[2.5]{\text{sec}}$ but variable end velocities $v_{p,n}$ (see left-hand side of Figure \ref{fig:snakes}, whereby $p$ stands for one parameter in the parameter set $\boldsymbol{\uptheta}$). %The first ramp starting the current velocity $v_0$ 
This allows to proactively plan tactical maneuvers, such as consecutively braking, keeping velocity and accelerating back. After each step $\Delta t$, the resulting ``snake'' profile is then time-shifted by an offset $o$ to match the new start velocity $v_0$ with same slopes $v_{p,n}$ for faster convergence. %After finding an optimal trajectory, $v_{p,n}$ can thus stay similar with less computational effort.        
Because $v(s)$ is discontinuous, we furthermore introduce an adjustable first lag $\lambda_{p,0}$ in the acting acceleration $a_0$. The right-hand side of Figure \ref{fig:snakes} shows that the following ramp transitions are supplementary smoothed with a Gaussian filter $h_g$.  %for reduced peaks

\begin{figure}[t!]
      \centering
      \resizebox{\linewidth}{!}{
      \import{img/}{smoothed_snake.pdf_tex}}
      \caption{Left: Parameters and shift of chosen velocity snake. Right: Lag implementation and corner smoothing.} %exemplary
      \label{fig:snakes}
\end{figure}

ROPT uses the non-gradient Powell's optimization method \cite{powell1964} which iteratively fits for $\boldsymbol{\uptheta}$ a quadratic function to three evaluation points and finds its vertex. %for each entry in theta %\footnote{On average, it takes less than 20 iterations to reach a suitable ego trajectory.} %A suitable ego trajectory is found usually in maximum 20 iterations. 
Soft constraints are set with penalizations for exceeding the minimal/maximal values $v_{\text{max}}$, $\lambda_{\text{min}}$, $a_{\text{min}}$ and $a_{\text{max}}$. Altogether, the optimization problem can thus be formulated as   
\begin{equation}
\text{min} \hspace{0.1cm} f \hspace{-0.05cm} \underbrace{(v_{p,1}, v_{p,2}, v_{p,3}, v_{p,4}, \lambda_{p,0})}_{\mbox{\footnotesize decision variables $\boldsymbol{\uptheta}$}} = \underbrace{R(t) - U(t) - O(t)}_{\mbox{\footnotesize fitness function $f$}},
%\text{min} f(\theta) = \underbrace{(\tau_0, v_1, v_2, v_3, v_4)}_{\mbox{\footnotesize decision variables}} = \underbrace{R(t) + O(t) - U(t)}_{\mbox{\footnotesize fitness function}} = C(t)
\label{eq: optimization}
\end{equation}
%\vspace{-0.1cm}
\begin{equation}
%\vspace{-0.4cm}
\text{subject to } v_{p,n} \leq  v_{\text{max}}, \hspace{0.15cm} \lambda_{p,0}  \geq \lambda_{\text{min}}, \hspace{0.15cm} a_{\text{min}} \leq  a_{p,n} \leq a_{\text{max}} \nonumber
%\text{subject to } \hspace{0.4cm} &\unit[0]{m/\text{sec}} \leq  v_i \leq  v_{\text{max}} \hspace{0.2cm} \nonumber \\ 
%&a_{\text{min}} \leq  a_i \leq a_{\text{max}} \\ %, \hspace{0.2cm} 
%&\unit[0]{\text{sec}} \leq  \tau_0 \leq \tau_{\text{max}} \nonumber
\label{eq: constraints}
\end{equation}
\noindent with segment accelerations $a_{p,n}$. A suitable ego maneuver is usually attained in less than 20 cycles. If not, we force the termination after a firm cycle number. %If not, we terminate the optimization after a fixed cycle number.  %For real-time capabilities, the optimization ends by default after 30 cycles. 
Besides the optimized snakes, we also sample fixed trajectories in our implementation: one constant velocity trajectory as well as one emergency stop and one acceleration trajectory. All trajectories are always evaluated in terms of their fitness, and one is in the end selected for behavior execution. Here, we introduce an hysteresis so that a switch to a different trajectory $v^m$ is done when the risk $R(t)$ of the new trajectory is relatively and absolutely smaller for a set period of time. 

\subsubsection{Smoothing Discrete Snake Function} 
%-picture with smoothing and tangent points and then next the smoothed verlauf of velocity \\
\noindent If we assume instantaneous actuation with fixed direct velocity points, ROPT may create trajectories which are unfeasible in real vehicles. The effective jerk $r(s)$ from $v(s)$ 
requires a continuous velocity curve. For this reason, we extrapolate the initial acceleration $a_0$ for the time $\lambda_{p,0}$ and blend its velocity line with the old ramp $(v_0,v_{p,1})$  according to   %and a weight factor $w(s)$
\begin{equation}
v(s) = v_0 +  (\lambda_{p,0}-s)a_0 + \frac{s}{s_l} (v_{p,1} - v_0)
\end{equation}
whereby $s=[0,\lambda_{p,0}]$. Afterwards, $v(s)$ is convoluted for the complete prediction interval $s_h$ with a Gaussian function %having constants $\sigma_s^2$ and $\mu_s$ given by
\begin{equation}
h_g(s)=N(\sigma_s^2,\mu_s=0). 
\end{equation}
We set the variance $\sigma_s^2$ and use $\mu_s=0$ to achieve further smoothness of the overall velocity curve without overshooting. As the derivatives $a(s)$ and $r(s)$ are numerically recalculated after the smoothing steps for $v(s)$, errors from asynchronicity are prevented. 

Consequently by optimizing $\lambda_{p,0}$, ROPT is able to influence the course of $r(s)$ (i.e., gradual actuation). Limits for $\lambda_{p,0}$ have to be enforced even when high-risk situations occur. The average brake lag to decelerate at once from $0$ to $a_{\text{min}}$ amounts to $\lambda_b=\unit[0.4]{\text{sec}}$ and engine acceleration to $a_{\text{max}}$ takes $\lambda_e = \unit[0.8]{\text{sec}}$.\footnote{In contrast, the action of taking the foot off the brake or gas pedal has immediate effect on the car.} With this in mind, we qualify the minimal lag threshold $\lambda_{\text{min}}$ depending on the acceleration $a_0$ as
\begin{equation}
\text{if } a_0 \geq 0 \text{: } \lambda_{\text{min}} = \frac{a_0}{a_{\text{max}}} \lambda_e , \hspace{0.2cm} \text{else: } \lambda_{\text{min}} = |\frac{a_0}{a_{\text{min}}}| \lambda_b. 
\end{equation}
%-$\tau_{\text{min}}=\unit[0.4]{\text{sec}}$ if $a<0$, $\tau_{\text{min}}=\unit[0.8]{\text{sec}}$ if $a>0$
Compared to employing continuous polynoms, our modified snake behaves smoothly and does not require the solution of a linear equation system to map $\boldsymbol{\uptheta}$ to the function shape.
In non-risky scenarios, ROPT is hence able to concentrate on comfortable behaviors. %has the advantage that there is no need of solving a linear equation system.

\subsection{Risk, Utility and Comfort Prediction}
\label{sec:trajeval}
\noindent
In the following, we look at one future plan $v^m(s)$ for the ego vehicle combined with constant velocities $v_j(s)$ of the other vehicles. %The evaluation step treats the contained dynamic 
This subsection describes the computation of the entire accumulated future costs $R(t)$, $U(t)$ and $O(t)$ contained in the resulting scene state sequence $\textbf{z}_{t:t+s}$, starting from time $t$ and evolving over $s$.  %\ref{sec:trajeval}

For the risk analysis, we can only postulate that $\textbf{z}_{t:t+s}$ will happen with a certain probability (e.g. because of sensor inaccuracies or unkown drivers' intention). % from collision and curve
%ROPT combines two probabilistic methods a Gaussian method for an instantaneous collision probability with the survival analysis
%[6] to retrieve an accumulated critical event probability. 
On this account, ROPT models accident occurences within an inhomogeneous Poisson process. The total event rate $\tau^{-1}(\textbf{z}_{t+s})$ describes the \makebox[\linewidth][s]{mean time between events. When subdivided into an escape}  \par
\noindent rate $\tau_0$ (behavioral options mitigating dangers) and critical rates of collisions $\tau^{-1}_{\text{crit},j}$ as well as of losing control in curves $\tau^{-1}_{\text{curv}}$, we gain %\footnote{To compute $\tau_{\text{coll}}$, we take the overlap of 2D normal distributions around the expected positions between car pairs and $\tau_{\text{coll}}$ compares }
\begin{equation}
\tau^{-1}(\textbf{z}_{t+s}) = \tau_0 + \sum_j  \tau^{-1}_{\text{coll,j}}+\tau^{-1}_{\text{curv}}.
\vspace{-0.04cm}
\label{eq:taucrit}
\end{equation} 
In Equation (\ref{eq:taucrit}), normal distributions are additionally defined for the positions and velocities growing after each prediction step size $\Delta s$. While $\tau^{-1}_{\text{coll}}$ is dependant on the distances $d_j(s)$ of car pairs, $\tau^{-1}_{\text{curv}}$ takes the lateral ego acceleration $a_{\text{y}}(s)$ into account.\footnote{For further details about the Gaussian method, please refer to \cite{puphal2018}.} 

Since the conveyed kinetic energy in a casualty is proportional to the operating masses $m_i$ and velocity vectors $\textbf{v}_i$, we use for collision and curve damage
\begin{equation}
D_{\text{coll},j}(s;t,\Delta s) = D_0+ \frac{m_1m_j}{2(m_1+m_j)}\| \textbf{v}_j - \textbf{v}_1 \|^2,
\end{equation}
\begin{equation}
D_{\text{curv}}(s;t, \Delta s) = D_0 + \frac{1}{2}m_1 \| \textbf{v}_1 \|^2
\end{equation}
with an offset $D_0$. Anytime a crash is not possible conditional to kinodynamics of the cars, $D_{\text{coll},j}$ is set to $0$.
%The kinetic energy of the accident (proportional to velocity vectors $v_i$) is 
By introducing a survival probability that the ego entity will not be engaged in an event during $[t, t+s]$ via
\begin{equation}
S(s;t,\textbf{z}_{t:t+s})=\exp\{-\int_o^s \tau^{-1}(\textbf{z}_{t+s'}) \, ds'\},
\end{equation}
we eventually obtain $R(t)$ as the temporal integration of rates, damages and probabilities%\footnote{A straightforward numerical calculation of the integral is sufficient with small $\Delta s$.}
\begin{equation}
R(t) = \int_0^{\infty} (\sum_j \tau_{\text{coll},j}^{-1}D_{\text{coll},j}+\tau_{\text{curv}}^{-1}D_{\text{curv}})S \,ds.
\label{eq: risk}
\end{equation} 
A straightforward numerical calculation of the integral is sufficient with small $\Delta s$, e.g. we utilize $\unit[0.05]{\text{sec}}$.

Besides minimizing risk, ROPT maximizes benefit (i.e., utility and comfort) as well. The considered utility consists of the overall needed time to arrive at the goal affected from the ego velocity $v_1$ and deviations to the desired velocity $v_d$. The components are weighted with driver-specific constants $b^t$ and $b^d$ to retrieve
\begin{equation}
U(t) = \int_0^{\infty} (b^t |v_1| + b^d |v_1 - v_d|) S \,ds.
\label{eq: utility}
\end{equation} 
Comfort returns are granted if the behavior does not change (ego acceleration $a_1\approx 0$) and the approach to planned $a_1$ is slow (ego jerk $j_1\approx 0$) so that 
\begin{equation}
O(t) = \int_0^{\infty} -(b^c |a_1| + b^j |j_1|) S \,ds.
\label{eq: comfort}
\end{equation} 
Calibrating the occuring parameters $b^c$ and $b^j$ in combination with $b^t$ plus $b^d$ allows to reproduce different driver characteristics, such as conservative versus sporty. The costs are therefore expressed in the same unit \euro \hspace{0.04cm}. For higher $s$, we also consider the survival function $S$ in Equation (\ref{eq: utility}) and (\ref{eq: comfort}) so that predicted benefits cannot surpass risk factors. % High-risk situations should not be surpassed for benefit reasons.


\section{Manager objects}
\label{sec: manager objects}
{\tt pygwb} counts three manager objects the user can interface with: {\tt Interferometer}, {\tt Baseline}, and {\tt Network}, which are defined in the {\tt detector}, {\tt baseline}, and {\tt network} modules, respectively. Each object is in charge of storing and saving relevant data, and handles data analysis internally. %
The manager objects are designed such that the user need never call a method from a module directly, but rather will invoke the manager which queries the relevant module to perform the calculation. %
For details on how to use these objects, see the complete set of tutorials in the {\tt pygwb} online documentation~\cite{docs}. %

%\air{guidline: it's `a' detector object, and `the' detector module.}

\subsection{\tt detector}\label{sec:detector}

\section{Experimental apparatus and data sample}
\label{sec:detector}

A description of the ALICE detector and its performances can be found in Refs.~\cite{ALICE:2008ngc,ALICE:2014sbx}.
At forward rapidity ($2.5<y<4.0$) the production of quarkonium states is measured in the muon spectrometer down to $\pt = 0$ via their dimuon decay channel. 
The muon spectrometer of ALICE consists of a ten interaction length thick  front absorber to filter muons, five tracking stations of two planes of cathode pad chambers each (MCH), a dipole magnet with a field integral of 3 Tm surrounding the third tracking station, a 1.2 m thick iron wall to absorb secondary hadrons escaping from the front absorber and low momentum muons coming mainly from $\pi$ and K decays, and two trigger stations made of two planes of resistive plate chambers each (MTR)~\cite{ALICE:2011zqe}. 
The silicon pixel detector (\SPD) and scintillator arrays (\VZERO) are also used in this analysis. The \VZERO counters, two arrays of 32 scintillator tiles each, cover $2.8 \leq \eta \leq 5.1$ (\VZEROA) and $-3.7 \leq \eta  \leq -1.7 $  (\VZEROC) and provide trigger information. The minimum bias (MB) trigger requirement consists of a logical AND of  a signal in \VZEROA and in \VZEROC. The \SPD,  two cylindrical layers covering $|\eta| \leq  2.0$ and $|\eta| \leq 1.4$ for the inner and outer layers, respectively, is dedicated to the  vertex reconstruction and allows estimating pile-up. 
The maximum interaction rate for the analysed data sample is 260 kHz, and the maximum pile-up probability is about $5\times10^{-3}$, negligible for this measurement.


The $\Jpsi$ pair analysis is performed using data from pp collisions at \thirteen collected from 2016 to 2018. 
The event sample was selected using the dimuon trigger condition, which 
is defined as  the  coincidence between the MB requirement and two opposite-charge sign track segments in the muon spectrometer trigger stations. Each track segment in the trigger stations is required to have a transverse momentum, evaluated online, larger than about 0.5~\GeVc. 
%
Only events passing a selection criterion to remove beam--background collisions contamination, based on the timing information from the V0 arrays, are considered in the analysis. 

When multiple primary vertices are reconstructed by the \SPD, the event is tagged as pile-up and removed from this analysis. 
In order to avoid acceptance biases on the reconstructed \SPD tracklets, events with a displaced vertex with respect to centre of the \SPD detector along the beam direction are discarded according to the requirement $|v_{z}| \leq 10$~cm. 
These selections allowed us to keep the pile-up below $0.3\%$ for the analysed events, also for events with two muon pairs with an invariant mass above 2 \GeVmass. 
Considering the above selections, the total number of dimuon triggered events in the sample sums up to $587.4 \times 10^{6}$ events and corresponds to an integrated luminosity of $24.11 \pm  0.01 (\rm{stat.}) \pm 0.80 (\rm{syst.}) \, \rm{pb}^{-1}$.



\subsection{\tt baseline}\label{sec:baseline}
The {\tt Baseline} module is by design the core of the {\tt pygwb} stochastic analysis. %
Its main role is to manage the cross-correlation between {\tt Interferometer} data products, combine these into a single cross-spectrum, which represents the point estimate of the analysis, and calculate the associated error, as introduced in Sec.~\ref{sec: GWB analysis}.

The standard initialization of a {\tt Baseline} object simply requires a pair of {\tt Interferometer} objects. %
\begin{Verbatim}[commandchars=\\\{\},frame=leftline,framesep=1.5ex,framerule=0.8pt,fontsize=\small]
\PY{k+kn}{from} \PY{n+nn}{pygwb} \PY{k+kn}{import} \PY{n}{baseline}
\PY{n}{H2H2\PYZus{}baseline} \PY{o}{=} \PY{n}{baseline}\PY{o}{.}\PY{n}{Baseline}\PY{p}{(}\PY{l+s+s2}{\PYZdq{}}\PY{l+s+s2}{H1\PYZhy{}H2}\PY{l+s+s2}{\PYZdq{}}\PY{p}{,} \PY{n}{H1}\PY{p}{,} \PY{n}{H2}\PY{p}{)}
\end{Verbatim}

Here {\tt H1} and {\tt H2} are {\tt Interferometer} objects. % 
It is also possible to load a previously stored {\tt Baseline} object in {\tt pickle} format by calling the relevant class method. %

The data loaded into the {\tt Interferometer} objects are automatically imported into the {\tt Baseline} object upon initialization. % 
The {\tt Baseline} object relies on the {\tt spectral} module to calculate cross-correlations between the data streams, following the methodology shown in Sec.~\ref{sec:spectral}. %
Similarly, it relies on the {\tt postprocessing} module to obtain the point estimate $\hat{\Omega}^\alpha_{\rm ref}$ and its variance $\sigma^\alpha_{\rm ref}$, as described in Eqs.~(\ref{eq:ptest_postpoc}--\ref{eq:sigma_postpoc}). %
The user may choose to calculate point estimate and sigma spectra or point values; in the latter case, the spectra are automatically stored to facilitate subsequent analyses. %

Calculating $\hat{\Omega}^\alpha_{\rm ref}$, as well as performing parameter estimation on the GWB spectrum, requires the two-detector \gls{orf}, $\gamma_{IJ}$, shown in Eq.~\eqref{eq:orf}. %
The \gls{orf} is calculated at {\tt Baseline} object initialization, then stored as an attribute. %
By default, we assume \gls{gr}, which presents two independent degrees of freedom for the strain field, typically $A=\{+, \times\}$ in the transverse-traceless gauge. For a precise derivation of this function and detector response definitions, see for example~\cite{Romano_2017}. %

The {\tt Baseline} object is also equipped to probe circularly polarized backgrounds~\cite{Seto:2007tn}, and non-\gls{gr} polarizations in the \gls{gwb}, such as scalar and vector backgrounds~\cite{TeVeS}. %
This requires selecting a different choice of $A$, according to the chosen polarization type, which can be declared when calculating $\hat{\Omega}^\alpha_{\rm ref}$ or the \gls{orf} directly. %
Details on the expressions for non-\gls{gr} $\gamma_{IJ}$ functions may be found in the appendix of~\cite{TeVeS}.




\subsection{\tt network}\label{sec:network}
\begin{figure}
   \vspace{-0.5cm}
   \center
   \includegraphics[width=0.48\textwidth]{misc/figures/Figure_revised_activation_layer.pdf}
   \vspace{-0.3cm}
   \caption{
   Our CIM installs an ``extension'' on the backbone (e.g., ResNet-18~\cite{he2016deep,rebuffi2017icarl} with four blocks) by adding a learnable activation function (e.g., PAU~\cite{molina2019pade}) at the position of the original activation function (i.e., ReLU~\cite{nair2010rectified}). Such ``extension'' results in a new network branch (in \textcolor[rgb]{0.576, 0.769, 0.490}{green}), whose weight layer parameters are directly copied from the original branch (in \textcolor[rgb]{0.427, 0.620, 0.922}{blue}).
   }
   \label{figure: revised_activation_layer}
   \vspace{-0.4cm}
\end{figure} 

\section{Analysis pipeline}
\label{sec: pipeline}

The previous sections contain a detailed description of each of the modules of the {\tt pygwb} package. %
We now present an overview of the package analysis scripts,  which combine the various modules into a \gls{gwb} analysis pipeline. %
The pipeline has several default values which may be changed according to the user's requirements. %
However, we note that thanks to the flexibility of the {\tt pygwb} package, one can also easily construct an ad-hoc pipeline. %

\subsection{\tt pygwb\_pipe}
\begin{table}[h!]
    \centering
    \footnotesize
    \begin{tabular}{c|c|c}
        Parameter & Default value & Description \\
        \hline
        \hline
        \multicolumn{3}{c}{Script arguments}\\
        \hline
        {\tt output\_path} & {\tt ""} &
                        Output data path \\
        {\tt calc\_pt\_est} & {\tt True} & 
                        If {\tt True}, calculate point estimates \\
        {\tt apply\_dsc} & {\tt True} &
                         If {\tt True}, apply $\Delta\sigma$ cut \\
        {\tt pickle\_out} & {\tt True} &
                        If {\tt True}, pickle post-processed baseline \\
        {\tt wipe\_ifo} & {\tt True} & If {\tt True}, set interferometer strain data to 0\\
        \hline
        \multicolumn{3}{c}{Data specifics}\\
        \hline
        {\tt interferometer\_list} & {\tt ["H1", "L1"]} & List of (2) interferometers \\

        {\tt t0} & 0 & Analysis start time \\
        {\tt tf} & 100 & Analysis end time \\
        {\tt data\_type} & {\tt public} & Data accessibility \\
        {\tt channel} & {\tt GWOSC-16KHZ\_R1\_STRAIN} & Data channel name \\
        \hline
        \multicolumn{3}{c}{Pre-processing}\\
        \hline
        {\tt tag} & C00 & Descriptive data tag \\
        {\tt new\_sample\_rate} & 4096 Hz & Downsampled sample rate \\
        {\tt input\_sample\_rate} & 16384 Hz & Input sample rate \\
        {\tt cutoff\_frequency} & 11 Hz & Lower frequency cutoff \\
        {\tt segment\_duration} & 192 s & Individual segment duration \\
        {\tt number\_cropped\_seconds} & 2 s & Preprocessing cropped seconds \\
        {\tt window\_downsampling} & hamming & Downsampling window \\
        {\tt ftype} & fir & Downsampling filter \\
        {\tt time\_shift} & 0 s & Time shift duration \\
        \hline
        \multicolumn{3}{c}{Gating}\\
        \hline
        {\tt gate\_data} & {\tt False} & If {\tt True}, self-gate data \\
        {\tt gate\_tzero} & 1 s & 0 time half-width duration \\
        {\tt gate\_tpad} & 0.5 s & Gating window tapering \\
        {\tt gate\_threshold} & 50 & Gating threshold \\
        {\tt cluster\_window} & 0.5 & Gating cluster window \\
        {\tt gate\_whiten} & {\tt True} & If {\tt True}, whiten data before gating \\
        \hline
        \multicolumn{3}{c}{Spectral density estimation}\\
        \hline
        {\tt frequency\_resolution} & 1/32 Hz & Output frequency resolution \\
        %{\tt coarse\_grain} & {\tt True} &If {\tt True}, {\color{red} **} \\
        {\tt overlap\_factor} & 0.5 & Consecutive segment fractional overlap \\
        {\tt N\_average\_segments\_welch\_psd} & 2 & Average \gls{psd} segment number \\
        {\tt zeropad\_csd} & {\tt True} & If true zeropad the \gls{csd}\\
        \hline
        \multicolumn{3}{c}{FFT window specifics}\\
        \hline
        {\tt window\_fft\_dict} & {\tt hann} & FFT window parameter dictionary\\
        \hline
        \multicolumn{3}{c}{Postprocessing}\\
        \hline
        {\tt polarization} & {\tt tensor} & ORF polarization basis \\
        {\tt alpha} & 0 & Spectral index $\alpha$ \\
        {\tt fref} & 25 Hz & Reference frequency $f_{\rm ref}$ \\
        {\tt flow} & 20 Hz & Lowest frequency included \\
        {\tt fhigh} & 1726 Hz & Highest frequency included \\
        \hline
        \multicolumn{3}{c}{Data quality specifics}\\
        \hline
        {\tt notch\_list\_path} & {\tt ""} & Notch list file path \\
        {\tt calibration\_epsilon} & 0 & Calibration coefficient \\
        {\tt alphas\_delta\_sigma\_cut} & [-5, 0, 3] & List of $\Delta\sigma$ cut spectral indices \\
        {\tt delta\_sigma\_cut} & 0.2 & $\Delta\sigma$ cut cutoff value \\
        {\tt return\_naive\_and\_averaged\_sigmas} & {\tt False} & If {\tt True}, return both $\sigma$ and $\bar{\sigma}$  \\
        & & used in $\Delta\sigma$ calculation \\
        \hline
        \multicolumn{3}{c}{Output specifics}\\
        \hline
        {\tt save\_data\_type} & {\tt npz} & Output datatype \\
        \hline
        \multicolumn{3}{c}{Local data locations}\\
        \hline
        {\tt local\_data\_path\_dict} & {\tt\{\}} & Dictionary of local data paths         \end{tabular}
    \caption{Default parameters for the {\tt pygwb$\_$pipe} script as well as the {\tt Parameters} dataclass. Most of these choices reflect defaults chosen in the past when analysing LIGO and Virgo data. Notably, the default start and end times for the analysis are not meaningful and represent placeholders for the user-defined times. A default initialization file is included in the package with meaningful start and end times present in the O3 open dataset.}
    \label{tab:parameters}
\end{table}

The core script of our analysis suite, {\tt pygwb$\_$pipe}, is designed to carry out the bulk of the stochastic analysis. %
It combines the {\tt pygwb} modules in order to go from the unprocessed data to the optimally averaged $\hat{\Omega}^\alpha_{{\rm ref}, f}$ and $\sigma^\alpha_{{\rm ref}, f}$ spectra for a single baseline. % 
 To read in the analysis parameters, {\tt pygwb\_pipe} interfaces with the {\tt parameters} module, specifically designed to handle the analysis parameters, either passed through an initialization file ({\tt param\_file}) or declared in the command line. %
 The module includes the {\tt Parameters} dataclass which stores the chosen parameters. % 
 The pipeline may be run from the command line as follows.
\begin{Verbatim}[commandchars=\\\{\},frame=leftline,framesep=1.5ex,framerule=0.8pt,fontsize=\small]
\PY{n}{pygwb\PYZus{}pipe} \PY{o}{\PYZhy{}}\PY{o}{\PYZhy{}}\PY{n}{param\PYZus{}file} \PY{p}{\PYZob{}}\PY{n}{path\PYZus{}to\PYZus{}param\PYZus{}file}\PY{p}{\PYZcb{}}
\end{Verbatim}


All {\tt param\_file} parameters may be alternatively passed from the command line directly. % 
If a mixture of parameter file and command line parameters are passed, the latter will override their corresponding values stored in the parameter file. %
Additionally, a set of pipeline--specific parameters may be passed from the command line for ease of use, such as whether to apply data quality cuts. %
A full list of parameters and their description may be found in Table \ref{tab:parameters}. %


After reading in the parameters, two {\tt Interferometer} objects are created accordingly, %, using the {\tt Interferometer} method {\tt from$\_$parameters}. %
and data are loaded in and pre-processed using the {\tt preprocessing} module. %
Depending on the value of the {\tt gate\_data} parameter in the initialization file, the gating outlined in Sec.~\ref{sec:preproc} also takes place at this stage. %
Subsequently, a baseline object is created using the pair of interferometer objects. %
Recall that the {\tt baseline} module plays a central role in the pipeline and handles the computation of the various quantities of interest, including the (average) \gls{psd}s and \gls{csd}s of the baseline, relying on the {\tt spectral} module. %
This is described in more detail in Sec.s~\ref{sec:spectral} and~\ref{sec:baseline}.

The delta-sigma cut is then performed,
and optimally averaged spectra and overall point estimate are calculated with the relevant {\tt Baseline} methods. %
The delta-sigma cut is applied by default, but may also be calculated and applied at a later stage. %
Finally, the spectra, the overall point estimate, and the pickled baselines (if requested), are saved as output. %
By default, the output is in {\tt numpy} binary file format, {\tt npz}. %

In realistic scenarios, we analyze year-long datasets and running {\tt pygwb\_pipe} in series is sub-optimal. %
However, a long dataset can be split into smaller jobs and parallelized on a cluster. %
The output of each job is then combined into a single set of result spectra $\hat\Omega^\alpha_{{\rm ref}, f}$ and $\sigma_{{\rm ref}, f}$ using the {\tt pygwb\_combine} script. %
The latter simply takes a weighted average over all jobs, assuming each job is an independent measurement of the signal. %
At this stage it is possible to implement final post-processing choices, such as re-weighting the spectra to a desired $\alpha$ and $f_{\rm ref}$, as well as change the default Hubble constant $H_0$ at which results are reported. %

Details on running the pipeline and combination scripts may be found in our online documentation~\cite{docs}. %



\subsection{\tt statistical\_checks}\label{Sec:StatChecks}
With the {\tt statistical\_checks} module, we provide a tool to perform initial statistical analyses of a {\tt pygwb} run result set, and visualize them in pre-formatted plots. %
We identify five broad categories of checks. %

The first set calculates the running point estimate for $\hat\Omega_{\rm ref}^\alpha$ and $\sigma_{\rm ref}^\alpha$ quantities as a function of time, as more data segments are added to the analysis. %
The values of $\alpha$ and $f_{\rm ref}$ are those used in the analysis and may not be changed at this point. %
The running averages are cumulative weighted averages of time--ordered segments, and do not take segment-by-segment correlation into account. % 
In case of detection, these converge to a biased point estimate and $\sigma$, as proper postprocessing is not applied (see Sec.~\ref{sec:postproc}). %
However, the visualization of running quantities is extremely useful to identify trends in the data, and ultimately will flag a possible detection. %
The module also provides a linear trend analysis, fitting the evolution of the parameters described above as a function of time. %

The second set focuses on the \gls{snr} spectrum as a function of frequency, defined as
\begin{equation}
    {\rm SNR}_f = \frac{\hat\Omega^\alpha_{{\rm ref}, f}}{\sigma^\alpha_{{\rm ref}, f}}\,.
\end{equation}
The absolute value, real, and imaginary part of the \gls{snr} are calculated, as well as the cumulative \gls{snr}. %
An example of these plots using the first sub-set of O3 data further described in Sec. \ref{Sec:O3Data} is given in Fig.~\ref{fig:FreqPlots}. %
These plots are a faithful representation of the ``noisiness'' of each frequency bin and how much each bin contributes to the analysis. %

\begin{figure}[t]
    \centering
    \includegraphics[width=0.47\textwidth]{statchecks_1.png}
    \includegraphics[width=0.47\textwidth]{statchecks_2.png}
    \caption{\textbf{Left}: Absolute value of the \gls{snr} spectrum as a function of frequencies. \textbf{Right}: Sigma spectrum as a function of frequency.}
    \label{fig:FreqPlots}
\end{figure}
\begin{figure}[h!]
    \centering
    \includegraphics[width=.43\textwidth]{statchecks_3.png}
    \raisebox{0.575\height}{\includegraphics[width=.43\textwidth]{statchecks_4.png}}
    \caption{\textbf{Left}: Point estimate, sigma and deviates $\Delta {\rm SNR}_i$ as a function of time before the delta-sigma cut (red) and after the cut (blue). \textbf{Right}: Distribution of the deviates $\Delta {\rm SNR}_i$ as a function of time before the delta-sigma cut (red) and after the cut (blue).}
    \label{fig:DscExamplePlots}
\end{figure}
The third set of checks produces the \gls{ift} of the point estimate spectrum, which should peak around zero seconds in case of a detection. 
Time-shifting the data in two detectors by more than the coherence time between the two detectors breaks the coherence between the two data streams, removing any evidence of a GWB signal. Note that the coherence time is determined by the bandwidth of our signal, which is of order 100 Hz, resulting in a coherence time of 10 ms. %
Hence, a \gls{gwb} signal will only peak around zero time lag between the output of the detectors. %

The fourth set studies the effect of the $\Delta\sigma$ data quality cut described in Sec. \ref{Sec:DeltaSigma} on the analysis run. %
To this end, we display several quantities before and after the cut is applied to the data, including the segment values of $\hat\Omega^\alpha_{{\rm ref}, i}$, $\sigma^\alpha_{{\rm ref}, i}$, and $\Delta\sigma^\alpha_{{\rm ref}, i}$, and the deviations in SNR,
\begin{equation}
    \Delta {\rm SNR}_i = \frac{\hat\Omega^\alpha_{{\rm ref}, i}-\langle\hat\Omega^\alpha_{{\rm ref}}\rangle}{\sigma^\alpha_{{\rm ref}, i}} \,,  
\end{equation}
as a function of time. %
Here angle brackets indicate an arithmetic mean over all segments $i$. %
We also plot a histogram of the values of $\Delta {\rm SNR}$ before and after the cut. %
This distribution should be centred around $0$, with a smooth narrower distribution after the application of the $\Delta\sigma$ cut. %
We additionally plot the $\Delta {\rm SNR}$ as a function of individual $\sigma^\alpha_{{\rm ref}, i}$. %
Finally, we plot the distribution of the ratios $\sigma^2_{{\rm ref}, i}/\langle\sigma^2_{{\rm ref}, i}\rangle$, which should peak around $1$. %
Some representative plots are shown as an example in Fig.~\ref{fig:DscExamplePlots}.



The last set of checks concerns a \gls{ks} test that is used to verify that the $\Delta {\rm SNR}_i$ are consistent with a Gaussian distribution. %
The \gls{ks} test implementation of this module returns the \gls{ks} test statistic, which is the maximal deviation from the Gaussian cumulative distribution function, as well as the p-value. %
These values can be used to make statements about the Gaussianity of the data \cite{KStestRef}. In addition, the cumulative distribution function is plotted for the data as well as for a Gaussian distribution. %

\section{Testing}

To comprehensively test the {\tt pygwb} analysis suite, we employ an efficient workflow to analyze datasets of increasing complexity. %
The datasets considered in this paper are:
\begin{enumerate}
    \item {\bf Continuous \gls{sgwb}:} A loud stationary and continuous stochastic signal generated with the {\tt simulator} module, injected in Advanced \gls{ligo} Hanford and \gls{ligo} Livingston assuming design A+ sensitivity~\cite{aplus_design_curve}. 
    \item {\bf Realistic \gls{cbc} \gls{gwb}:} A realistic background of merging \glspl{bbh} and \glspl{bns}, injected in Advanced \gls{ligo} Hanford and \gls{ligo} Livingston assuming design A+ sensitivity.
    \item {\bf O3 dataset:} The full Advanced \gls{ligo} Hanford and \gls{ligo} Livingston dataset from the third \gls{lvk} observing run~\cite{LIGO_O3}.
\end{enumerate}
The continuous \gls{sgwb} (dataset 1) is an idealized observing scenario, as our stochastic model matches the target signal perfectly by design, and as the signal is stationary and continuous our approach is optimal~\cite{Drasco-Flanagan:2003, Lawrence:2023buo}. %
The \gls{cbc} background (dataset 2) is a realistic scenario where the target signal is generated according to astrophysical models, informed by GW detections. %
In this case the signal is non-Gaussian, and we expect our approach to be un-biased~\cite{HLV-MDC-PhysRevD.92.063002, ET-first-MDC-PhysRevD.86.122001, ET-second-MDC-PhysRevD.89.084046} but sub-optimal~\cite{Drasco-Flanagan:2003, Lawrence:2023buo}, due to the intermittent nature of the signal which is not taken into account in the search method. %
For more details on the time-domain characteristics of these two types of signals and the detection challenges these present, see for example~\cite{Regimbau2022}. %
Finally, the O3 Advanced LIGO dataset (dataset 3) presents all the complexity of analyzing real GW detector data, which includes non-stationary noise, a large data volume, and expensive computational requirements. %

We handle large datasets by splitting the data into smaller {\tt pygwb$\_$pipe} jobs, assuming each job is independent; %
these are then combined using the {\tt pygwb$\_$combine} script (see Sec.~\ref{sec: pipeline} for details). %
We then employ a parameter estimation script, {\tt pygwb$\_$pe}, based on the {\tt pe} module described in Sec.~\ref{sec:pe}, to perform parameter estimation on specific models. %
For more details on how to run this sort of analysis, we refer users to the online documentation for the most up-to-date workflow instructions~\cite{docs}. %
In the following, we present the different datasets and summarize our analysis results.

\subsection{Mock data}
\label{sec: MDC}

\subsubsection{Stationary and continuous stochastic gravitational-wave background}
\label{MDCSimulator}
We employ the {\tt Network} (Sec.~\ref{sec:network}) to generate a stationary and continuous \gls{sgwb} signal with a fixed \gls{psd}, $S_h(f)$. %
This allows us to simultaneously test the module and the whole analysis pipeline. %
%We now perform a more rigorous test of the {\tt simulator} module and use the {\tt pygwb} pipeline to recover the injection. 
The injected \gls{sgwb} is scale-invariant, i.e., $\Omega_{\rm GW}(f)$ is constant over frequencies, 
\begin{equation}
    \Omega_{\rm inj}(f)= 1.06 \times 10^{-7}\,.   
\end{equation}
This is converted to $S_h(f)$ using the relation in Eq.~\eqref{eq:omegatoI}. %
The noise $P_n(f)$ is taken to be Gaussian, colored using the the Advanced \gls{ligo} noise \gls{psd} \cite{aLIGO_sensitivity}. %
One hundred days of consecutive data are simulated at a sampling rate of $1024$ Hz. %

Each of the one hundred days is analyzed separately, and we recover a distribution of $\hat\Omega^0_{\rm 25}$ point estimates, shown in Fig.~\ref{fig:hundreddaydataset} (left), using $\alpha=0$ and $f_{\rm ref} = 25$ Hz in the pipeline. %
Analyzing one hundred days separately allows us to construct a distribution of recovered point estimates, which is useful to assess the ability of the {\tt simulator} module to inject a stochastic stationary signal. %
%This test assesses both the ability to inject a signal correctly, as well as the ability to recover the injected signal using the {\tt pygwb} pipeline. 
We then perform parameter estimation on the combined one hundred days, presented in Fig. \ref{fig:hundreddaydataset} (right). We assume a log-uniform prior from $10^{-11} - 10^{-6}$ for $\Omega_{\rm ref}$ and Gaussian prior with mean 0 and standard deviation 1.5 for $\alpha$. %
This shows a recovery within $1\sigma$ for $\Omega_{\rm ref} = 1.06 \times 10^{-7}$ and within $2\sigma$ for the spectral index $\alpha_{\rm inj} = 0$. 

The tests above illustrate that the {\tt simulator} module is able to successfully inject a stochastic stationary signal and that the {\tt pygwb} pipeline is able to recover this injection.

\begin{figure}[t]
    \centering
    \includegraphics[width=0.45\textwidth]{simtest_1.pdf}
    \includegraphics[width=0.3\textwidth]{simtest_2.pdf}
    \caption{\textbf{Left}: Distribution of the recovered point estimate for each day in the dataset. The injected value is denoted by the red line. $\overline{\Omega}_{\rm ref}$ and $\sigma_{\rm ref}$ denote the mean and the standard deviation of the one hundred point estimates. \textbf{Right}: Parameter estimation performed on the one hundred days, obtained assuming a log-uniform prior from $10^{-11} - 10^{-6}$ for $\Omega_{\rm ref}$ and Gaussian prior with mean 0 and standard deviation 1.5 for $\alpha$. The injected values are denoted by the black lines, while the contours represent the 1$\sigma$, 2$\sigma$, and $3\sigma$ contours. The vertical dashed lines represent the $1 \sigma$ confidence interval.}
    \label{fig:hundreddaydataset}
\end{figure}
\subsubsection{Gravitational-wave background from a coalescing compact binary population}


%A compact binary coalescence follows the inspiralling phase, with GWs emission, of a binary system, where two massive stars have collapsed into a neutron star or a black hole. %
The inspiral and merger of two compact objects emit a characteristic \gls{gw} signal. %
We generate datasets containing a \gls{gwb} signal resulting from the superposition of \gls{gw} signals from a set of \gls{cbc} populations including \glspl{bbh} and \glspl{bns}. %
 To simulate the signals, we employ the code used in the past for the \gls{et} \glspl{mdsc} \citep{ET-first-MDC-PhysRevD.86.122001, ET-second-MDC-PhysRevD.89.084046, ET-second-MDC-low-f-PhysRevD.93.024018} and for the Advanced \gls{ligo} and Advanced Virgo \gls{mdsc} \citep{HLV-MDC-PhysRevD.92.063002}. %
The Monte Carlo algorithm that we use for the generation of a compact binary population up to redshift $z=10$ is extensively described in \cite{ET-first-MDC-PhysRevD.86.122001} and \cite{Regimbau:2014nxa}. We summarize below the main steps of the simulations.

To generate a \gls{cbc} population we assume a merger rate per unit redshift~\citep{Belczynski:2006br, Berger:2006ik, Belczynski:2000wr, Bulik:2003kr},
\begin{equation}
    \frac{\text{d}R(z)}{\text{d}z} = \frac{\text{d}V_{\rm c}}{\text{d}z} {r_c}(z),
\end{equation}
where $\text{d}V_{\rm c}/\text{d}z$ is the co-moving volume element and ${r_c}$ the coalescence rate as a function of redshift~\cite{Regimbau:2011rp}. %
The element of co-moving volume assumes a $\Lambda$CDM cosmology from Planck 2018~\cite{Planck2018} (Hubble parameter $H_0 = 67.7\, \mathrm{km/s/Mpc}$, $\Omega_{m}=0.31$ and $\Omega_{\Lambda}=1-\Omega_{m}$). %
We assume a coalescence rate normalized to a local rate $r_c(0)=1\,\mathrm{Mpc^{-3}\, Myr^{-1}}$ for \gls{bns} coalescences and  $r_c(0)=3\,\mathrm{Mpc^{-3}\, Myr^{-1}}$ for \gls{bbh} coalescences, assuming the star formation rate from \cite{Hopkins:2006bw} and a minimum delay time between binary formation and merger of $20\, \mathrm{Myr}$ for \glspl{bns} and $50\, \mathrm{Myr}$ for \glspl{bbh}; see \citep{Dominik:2012kk, Neijssel_2019} for more details. %
These choices give rise to a dataset composed by $87 \%$ of \glspl{bns} and $13 \%$ of \glspl{bbh}. % 

The time intervals $\tau$ between consecutive \gls{cbc} events in our population are obtained by sampling an exponential distribution $P(\tau) = \exp(-\tau/\bar{\tau})$, where $\bar{\tau}$ is the average time between consecutive events. This is consistent with the assumption that the coalescence times $t_c$ of the events behave as a Poisson process~\cite{Regimbau:2014nxa}. %
The coalescence redshift is drawn from the normalized coalescence rate $p(z)=\bar{\tau}\text{d}R/\text{d}z(z)$ within $z \in [0,10]$. %
The sky position $\hat{n}$ of each source is generated isotropically on the sky. %
The \gls{gw} polarization angle $\psi$, the phase angle $\phi_0$ at the coalescence time, and the cosine of the inclination angle of the orbital plane to the line of sight $\iota$ are all drawn from uniform distributions. %
The mass function of the components in the \glspl{bbh} is chosen to be a \gls{plpp} from the preferred case presented in the \gls{lvk} collaboration \gls{cbc} population inference paper~\citep{O3_pop_paper_LIGOScientific:2021psn} or a simple \gls{pl}~\citep{O2_pop_paper_LIGOScientific:2020kqk}, while the \gls{bns} masses are drawn from uniform distribution between 1 and 3 $M_{\odot}$. %\kt{Since the mass range for BNS is mentioned, should the one for BBH also be mentioned?}
The \gls{bbh} mass functions are used to label the two datasets presented below. %
Spins are neglected in both cases. %
\begin{figure}[t]
    \centering
\includegraphics[width=0.5\textwidth]{MDC_1.pdf}
    \caption{$\Omega_{\mathrm{GW}}(f)$ for the dataset corresponding to the \gls{plpp} (blue) and the \gls{pl} (red) models for the mass function. The black line is the power-law integrated sensitivity (PI) curve for an observation time of six months and an expected \gls{snr} = 5, assuming the HL baseline with Advanced \gls{ligo} plus design sensitivity. The simulated signals intersect the PI curve, hence they are expected to be detected with an \gls{snr} of at least 5.}
    \label{fig:Omega_CBC_injection}
\end{figure}

For each source, the signal waveform is generated in the time domain. %
For \glspl{bns}, we use the {\tt TaylorT4} time-domain waveform \citep{TaylorT4_Buonanno:2002fy}. %
For \gls{bbh} signals, we use the {\tt EOBNRv2} \citep{EOBNRv2_Buonanno:2009zt} % 
time-domain waveform from numerical relativity. %
These are then injected into the \gls{ligo} Hanford and Livingston detectors, with the addition of colored Gaussian noise generated from the \gls{ligo} A+ Design~\cite{MDC_sensitivity_curves, aplus_design_curve} expected sensitivity curve, to produce the final datasets. %


Following the above prescriptions, we generate two six-months datasets with sampling frequency 1024 Hz, labelled \gls{plpp} and \gls{pl}, formed by two different \gls{cbc} populations. %
The two populations differ by the mass distributions of the \glspl{bbh} and the average time between consecutive events, as seen in Table \ref{tab:MDC_CBC_results}. %
The latter is chosen such that the \gls{gwb} amplitudes of the two datasets match, for ease of comparison. % 
Furthermore, to obtain a \gls{snr} large enough to confidently detect the injected \gls{gwb}, a small amplification of the signal is required. %
To this end, the amplitude of the \gls{cbc} waveforms is multiplied by 1.5 and 1.7 for the \gls{plpp} and the \gls{pl} datasets, respectively, resulting in an injected value of $\Omega_{\mathrm{ref}} = 2.05 \times 10^{-9}$. %

The $\Omega_{\mathrm{GW}}(f)$ spectrum relative to the each dataset is obtained by summing the contributions from individual coalescences \citep{HLV-MDC-PhysRevD.92.063002}, and is illustrated in Fig.~\ref{fig:Omega_CBC_injection}. %
As may be observed, in the case of \gls{cbc} signals $\Omega_{\mathrm{GW}}(f)$ increases as $f^{2/3}$ from the inspiral phase (and then as $f^{5/3}$ from the \gls{bbh} merger phase) before reaching a peak and steeply decreasing~\cite{Marassi:2011si}. %
This motivates fixing the spectral index parameter to $\alpha=2/3$ in our searches. %
Fig.~\ref{fig:Omega_CBC_injection} also shows the \gls{pi} curve \citep{PI_curves:PhysRevD.88.124032} for the Hanford-Livingston baseline, assuming the design A+ sensitivity for the two detectors \citep{aplus_design_curve}, an observation time $T_{\rm obs} = 6$ months, and a desired sensitivity of \gls{snr}=5. %
Given that the \gls{pi} curve is almost tangent to $\Omega_{\mathrm{ref}}$ of the two datasets, we expect to observe the \gls{gwb} signals with \gls{snr} $\sim 5$. %

%\noindent \textbf{Analysis and results}

\begin{figure}[t]
    \centering
    \includegraphics[width=0.4\linewidth]{MDC_2.pdf}
    \includegraphics[width=0.4\linewidth]{MDC_3.pdf}
    \caption{PE results. Left: Corner plot obtained from running the parameter estimation over the \gls{plpp} dataset.
    Right: Corner plot obtained from running the parameter estimation over the \gls{pl} dataset.
    Each plot shows the posteriors on $\Omega_{\mathrm{ref}}$ and $\alpha$ obtained assuming a log-uniform prior on $\Omega^0_{\mathrm{ref}}$ from $10^{-11}$--$10^{-8}$ and a Gaussian prior on $\alpha$ with mean 2/3 and standard deviation of 1.5, respectively, denoted by the gray dashed lines. The injected values are represented by the black lines, indicating a recovery of both the amplitude of the signal and $\alpha$ within $1\, \sigma$. The vertical blue dashed lines represent the $2 \sigma$ confidence interval.}
    \label{fig:MDC_pe}
\end{figure}
We analyze the two datasets in the frequency band $20 - 500$ Hz, using a frequency resolution of $1/32$ Hz and a segment duration of 192 s. %
We choose $\alpha = 2/3$, $f_{\rm ref} = 25$ Hz, and $H0=67.7$ km/s/Mpc for this analysis. %
The results of the analysis are summarized in Table \ref{tab:MDC_CBC_results}. %
We recover the \gls{plpp} injection within $1\, \sigma$, and observe it with \gls{snr} = 5.4, while recovering the \gls{pl} injection within $1\, \sigma$, with \gls{snr} = 5.0. %
We attribute the differences in the recoveries to the specific data and noise realizations within the datasets. %
% A larger average time between two successive binary mergers, as in the case of the \gls{plpp} dataset, results in a more discontinuous \gls{gwb}, which requires a longer amount of time to be observed with the techniques used in this work. %

We then proceed with estimating the parameters $\alpha$ and $\Omega^\alpha_{\mathrm{ref}}$ modelling $\Omega_{\mathrm{GW}}(f)$ as a simple power-law in frequency as given by Eq.~\eqref{eq:omega-plaw}. %
We assume a log-uniform prior over $\Omega^0_{\mathrm{ref}}$ in the range $[\Omega^0_{\mathrm{min}},\,\Omega^0_{\mathrm{max}}]=[10^{-11}, \,10^{-8}]$, and a Gaussian prior on $\alpha$ with mean $2/3$ and standard deviation $(\log_{10}{\Omega^0_{\mathrm{min}}}-\log_{10}{\Omega^0_{\mathrm{min}}})/2 = 1.5$. %
Note that the priors in $\Omega^\alpha_{\mathrm{ref}}$ are defined for $\alpha = 0$. %
The choice of the prior over $\alpha$ can be understood as follows. %
The log-uniform prior over $\Omega^0_{\mathrm{ref}}$ induces some implicit prior over $\alpha$ that can be shown to be a triangular prior centred on $\alpha = 0$ and non-zero for $|\alpha| \leq (\log_{10}{\Omega^0_{\mathrm{max}}}-\log_{10}{\Omega^0_{\mathrm{min}}})$. %
To avoid a vanishing prior outside of this range, we choose a Gaussian prior for $\alpha$ with standard deviation comparable with the triangular prior, centered on $\alpha =2/3$ to better match the injected \gls{gwb}. %

Parameter estimation corner plots are shown in Fig.~\ref{fig:MDC_pe}. %
For both datasets, $\Omega_{\mathrm{ref}}$ and $\alpha$ are recovered within $1\, \sigma$. %
The log-Bayes factors $\mathcal{B}^{\rm GW}_{\rm noise}$ are $11.1$ and $9.2$ for the \gls{plpp} and \gls{pl} datasets, respectively, indicating strong evidence~\cite{bayes_factor_doi:10.1080/01621459.1995.10476572} for the presence of signal over noise only. 

\begin{table}[h]
    \centering
    \begin{tabular}{c|c|c|c|c|c}
       {\sc dataset}  & $\tau$ (s) & $a$ &  $  ~\qty(\hat{\Omega}^{2/3}_{25} \pm \hat{\sigma}^{2/3}_{25}) \times 10^{9}$~ & ~SNR~ & ~$\mathcal{B}^{\mathrm{GW}}_{\mathrm{noise}}$~\\
    \hline
    \hline
        PLPP & ~60~  & ~1.5~  & $2.09 \pm 0.39$ & ~5.4~ & ~11.1~\\
        PL & ~54.7~ & ~1.7~ & $1.94 \pm 0.39$ & ~5.0~ & ~9.2~
    \end{tabular}
    \caption{Parameters and results of each dataset. The first row refers to the \gls{plpp} dataset, while the second row to the \gls{pl} one. The second and third columns display the average time between two successive binary mergers, $\tau$, and the waveform amplification factor, $a$. The last three columns illustrate the recovered point estimate with $1\, \sigma$ uncertainty on the quantity $\hat{\Omega}_{\mathrm{ref}}^\alpha$ ($f_{\mathrm{ref}} = 25\, \mathrm{Hz}$, $\alpha=2/3$), the corresponding \gls{snr}, and the log-Bayes factor $\mathcal{B}^{\mathrm{GW}}_{\mathrm{noise}}$.
    }
    \label{tab:MDC_CBC_results}
\end{table}
\subsection{O3}
\label{Sec:O3Data}

In this section we present results from the application of the {\tt pygwb} analysis suite to the full \gls{ligo} Hanford and \gls{ligo} Livingston O3 dataset~\cite{gwosc}. %
We set upper limits on the signal from a \gls{sgwb} and confirm these are consistent with previously published collaboration results~\cite{LIGO_O3}.

The O3 data run collected between April 1, 2019 and March 27, 2020, divided into two sub-sets with an interruption between October 1 and November 1, 2019, with a total coincident livetime of 205.4 days between \gls{ligo} Hanford and \gls{ligo} Livingston. %
These are reduced to 196 days after {\it category 1 vetoes}\footnote{``Category 1'' vetoes flag data which are unsuitable for analysis, such as incorrectly calibrated data, data collected during atypical operation of the instruments, and data with severe data quality issues.}~\cite{Virgo:2022kwz, Abbott_2018} and external non-stationarity cuts are applied (for details, see~\cite{LIGO_O3}). %
The {\tt pygwb} analysis is implemented with the workflow described above. %
The O3 data, natively sampled at 16384 Hz, are downsampled to 4096 Hz and high-pass filtered at 11 Hz. %

The time-averaged O3 \gls{ligo} Hanford -- Livingston cross-correlation spectrum is presented in Fig.~\ref{fig:O3_crosscorr_spectrum}. %
Our $\Delta\sigma$ threshold excludes 7.8\% of the analyzed time (see Sec.~\ref{Sec:DeltaSigma} for implementation details). %
This result matches the previous stochastic non-stationarity cut published in~\cite{LIGO_O3} within 1\%, with the previous cut excluding an extra 0.06\% of the time. %
We believe this small variation to be due to a different window bias factor used in the two analyses (the bias factor calculation used here is outlined in App.~\ref{sec:app_window}). %

We calculate broad-band integrated estimates between $20 - 1726$ Hz of $\Omega_{\rm GW}(f_{\rm ref})$ for different power-law spectral models, applying the released O3 notchlist~\cite{O3IsotropicDataset} to exclude known problematic frequencies~\cite{Covas_2018}. %
A summary of the values for the point estimate and uncertainty for these is presented in Table~\ref{tab:O3_results}. %
The uncertainties $\sigma^\alpha_{\rm ref}$ agree within $1\%$ with previously published LVK results, presented in~\cite{LIGO_O3}. %
The point estimates for $\Omega^\alpha_{\rm ref}$ fluctuate notably more than the uncertainties. %
We believe this to be due to small differences in the analyses, to which the point estimates are more sensitive, such as individual start and end time of each pipeline job, and the differences in the non-stationarity cuts described above. %

Finally, we perform parameter estimation to constrain $\Omega_{\rm GW} (f_{\rm ref}= 25 \text{Hz})\equiv \Omega_{25}$ and the spectral index $\alpha$ with O3 data. %
We employ a log-uniform prior on $\Omega_{25}$ spanning $[10^{-13},10^{-6}]$, and present results for two different priors on $\alpha$: a uniform prior between $[-4, 4]$ and a Gaussian prior centered around 0 with norm 3.5 (the latter matches the choice in ~\cite{LIGO_O3}). %
To account for calibration uncertainty, we marginalize over the uncertainty parameter $\lambda$ as described in App.~\ref{sec:app_calibration}, with combined calibration error for Hanford and Livingston of $1.48\%$, as in~\cite{LIGO_O3}. %
Parameter estimation confirms $\Omega_{25}$ is consistent with 0 and $\alpha$ remains unconstrained, as may be seen in Fig.~\ref{fig:O3_pe}. % 
These results agree with the previous parameter estimation carried out in~\cite{LIGO_O3}.

\begin{figure}[t]
    \centering
    \includegraphics[width=0.75\textwidth]{O3_1.png}
    \caption{Estimated cross-correlation spectrum $\hat{\Omega}^0_{25}\pm \hat{\sigma}^0_{25}$ from O3 data. By eye, it is possible to spot several narrowband artifacts (lines) which are subsequently excluded from our analysis.}
    \label{fig:O3_crosscorr_spectrum}
\end{figure}

Our results are quoted at the value of the Hubble parameter $H_0 = 67.9$ km/s/Mpc, in line with published results. %
This is not the built-in value of $H_0$, defined in Sec.~\ref{sec:postproc}; however, rescaling is straightforward as it is an overall multiplication factor, which may be changed when post-processing the run with the {\tt pygwb$\_$combine} script or manually using the built-in functions of the {\tt OmegaSpectrum} object, as explained in Sec.s~\ref{sec:postproc} and~\ref{sec: pipeline}.

We would like to note that this entire analysis was carried out on a large computing cluster and completed in less than five hours of human time. %
This is an example of the computational efficiency of our package.

\begin{table}[h]
    \centering
    \begin{tabular}{c|c|c||c|c|c}
       ~$\alpha$~  &  ~$\hat{\Omega}^\alpha_{25}\times 10^9$~ & ~$\hat{\Omega}_{\rm LVK}\times 10^9$ & ~prior~ & ~$\Omega_{\rm pe}$ ($95\%$ UL) ~& ~$\alpha_{\rm pe}$~\\
       \hline 
       \hline 
       
         0 & $-3.4\pm 8.1$& $-2.1\pm 8.2$ & {\sc uniform} & $5.44\times 10^9$ & $-0.8^{+2.8}_{-2.2}$\\
         2/3 & $-4.5\pm 6.1$& $-3.4\pm 6.1$ & ~{\sc gaussian}~ & $4.06\times 10^9$ & $-0.5^{+2.8}_{-2.8}$ \\
         3 & $-1.5\pm 0.9$& $-1.3\pm 0.9$ & & 
    \end{tabular}
    \caption{Summary of {\tt pygwb} search results on O3 dataset. On the left, three columns summarising point estimates from the weighted optimal statistic, at different spectral indices $\alpha$. On the right, three columns summarising Bayesian upper limits (UL) with log-uniform prior on $\Omega^0_{25}$ and either uniform or Gaussian prior on $\alpha$. These results are consistent with no detection of the amplitude of the background ($\Omega^\alpha_{\rm ref}$ is consistent with 0), nor its spectral shape ($\alpha$ remains unconstrained).}
    \label{tab:O3_results}
\end{table}

\begin{figure}
    \centering
    \includegraphics[width=0.4\linewidth]{O3_2.pdf}
    \includegraphics[width=0.4\linewidth]{O3_3.pdf}
    \caption{Parameter estimation results with {\tt pygwb$\_$pe} on LVK O3 data, using a log-uniform prior on $\Omega_{25}$, and a uniform prior (left) or a Gaussian prior on $\alpha$ (right), as described in the text. The priors are denoted by the gray dashed lines. }
    \label{fig:O3_pe}
\end{figure}

\section{Conclusions}
\section{Conclusion}\label{sec:conclusion}
In this work, we focus on addressing the fundamental challenge of OOD detection tasks, which is how to fully understand the semantic discrepancy between the ID/OOD samples. We reveal that the key to success in the realistic SCOOD task is to allocate as many ID samples in the unlabeled set correctly as possible. To this end, we propose a novel uncertainty-aware optimal transport scheme that introduces class-specific energy scores as guidance for effective label assignment. Experimental results show that our method achieves better performance than previous state-of-the-art methods on SCOOD benchmarks.

\textbf{Limitations.} In addition to temperature scaling, other techniques such as feature clipping applied in ReAct~\cite{sun2021react} also enhance the performance of energy score, so how to obtain an OOD score that best fits the SCOOD task can be further explored. Moreover, a setting highly related to SCOOD has been proposed in \cite{katz2022training} and formulated as a constrained optimization problem. We will also theoretically analyze these practical OOD settings in our feature work.

% \section*{Acknowledgments}
\textbf{Acknowledgments.} 
This work is supported by National Key R\&D Program of China under Grant 2020AAA0105701, National Natural Science Foundation of China (NSFC) under Grants 61872327, Major Special Science and Technology Project of Anhui, National Natural Science Foundation of China (62033012) and Ant Group through Ant Research Intern Program.


\appendix
\section{Window functions and bias factors}\label{sec:app_window}
The window factors, $\bar{w}^4_\mathrm{ovl}$ and $\bar{w}^4$, used in Sec.~\ref{sec:postproc} are defined as in Eqs.~(34) and (24) in~\cite{lazz_romano_windowing_note}. They are used to correct for the effect windowing has on our estimate of the variances. Actually, these corrections should include contributions from the autocorrelation function (PSD) of the individual detectors or their cross-correlation (see, e.g. Eqs.~(22) and (32) of the same note). However, if the frequency response of the window is sufficiently strongly peaked around zero, then we can treat the transformed windows as delta functions~\cite{whelan_CC_dcc} and our expressions for these quantities reduce to
\begin{align}
    \bar{w}^4 = \frac{1}{N}\sum_{i=1}^Nw_i^4,
\end{align}
where $w_i$ represents the $i^{\textrm{th}}$ sample of the Hann window we use. Likewise, we need to account for the covariance between point estimates calculated in adjacent time segments. The point estimates are each quadratic in the data, windowed, and use 50\% overlapping segments of data, and so we must account for the overlapping of the windows applied to the two segments
\begin{align}
   \bar{w}_{\textrm{ovl}}^4 = \frac{1}{N/2}\sum_{i=N/2+1}^N w_i^2 w_{i-N/2}^2,
\end{align}
where we see this now as the cross-correlation of the pieces of the two windows that overlap for the two segments.

When calculating the variance of our point estimate, we must estimate the quantity 
$\left(P_{1, f} P_{2, f}\right)^{-1}$, which is the expression that appears in the Gaussian likelihood used to construct our optimal estimators~\cite{Matas:2020roi}, and is therefore the relevant quantity when considering the variance of the point estimates. We briefly summarize how to properly estimate this quantity based on the discussion in Appendix B of ~\cite{Matas:2020roi}, noting that they do not consider the effect of windowing, which we also discuss below. 

For a segment of length $T$ we calculate estimators for the PSDs, $\hat P_{I,f}$, where $I=1,2$ labels the detector, using Welch's method~\cite{welch_method_and_window_factor}. We break our time segment $T$ into 50\% overlapping chunks, calculate the PSD in each chunk, and average those estimates together. If we want a PSD with frequency resolution $\Delta f$ then we have $K$ overlapping segments where $K=2T\Delta f -1$. We can assume our (noisy) estimators for the individual PSDs are unbiased and can be written as the true PSD plus some small deviation, $\hat P_{1,f} = P_{1,f} + \delta P_{1,f}$. We now look at the quantity of interest in calculating our variance
\begin{align}
   \frac{1}{\hat P_{1,f}\hat P_{2,f}} =&\frac{1}{\left[P_{1,f} + \delta P_{1,f}\right]\left[P_{2,f} + \delta P_{2,f}\right]}.
   \end{align}
   We can expand the denominator, take the expectation value of both sides, and use the fact that $\langle\delta P_{I,f}\rangle=0$ and $\langle\delta P_{I,f}^2\rangle = \textrm{var} P_{I,f}$, where $I=1,2$ labels the detector. This gives us
   \begin{align}
    \Braket{\frac{1}{\hat P_{1, f}\hat P_{2, f}}}\approx&\frac{1}{P_{1, f}  P_{2, f}} \left(1 + \frac{\textrm{var}P_{1, f}}{ P_{1, f}^2} + \frac{\textrm{var}P_{2, f}}{ P_{2, f}^2} + \cdots\right)\\
    =&\frac{1}{ P_{1, f}  P_{2, f}}\left(
    1 + \frac{2\kappa}{K} \right).
\end{align}
This expression can be compared to Eq. (B8) in~\cite{Matas:2020roi}, noting that we have an extra term in the variance of our PSDs, $\kappa$. This term reduces the ``effective'' number of averages we perform due to our windowing, where we apply a Hann window with amplitude $\{w_i\}$ at each sample $i$, as well as the overlapping of our chunks of data. The correction factor is given by~\cite{welch_method_and_window_factor}
\begin{align}
   \kappa = \left[1 + 2\left(\frac{\sum_{i=N/2+1}^N w_i w_{i-N/2}}{\sum_{i=1}^Nw_i^2}\right)^2\frac{K-1}{K}\right].
\end{align}
In practice, we ignore the term $(K-1)/K$, as it leads to extra corrections that are $\mathcal {O}(K^{-2})$ that are quite small.

We can now define a bias correction factor based on the windowing we choose and the number of averages used in constructing $\hat P_{I,f}$. 
Defining $N_{\textrm{eff}} = \kappa^{-1}K$, we have
\begin{align}
\hat\sigma^{-2}(f) = \left(1 + \frac{2}{N_\textrm{eff}}\right) \sigma^{-2}(f),
\end{align}
where we have used simplified notation again where the hat indicates our estimator for Eq.~(\ref{eq:Variance}) and the unhatted indicates the true value.

Taking the square root of both sides and inverting it gives us
\begin{align}
    \sigma = b(N_\textrm{eff})\hat\sigma,
\end{align}
where the bias factor, $b(N_{\textrm{eff}})$, is given by
\begin{align}
    b(N_{\textrm{eff}}) = \frac{N_{\textrm{eff}}}{N_{\textrm{eff}}-1},
\end{align}
assuming $N_{\textrm{eff}}$ is large. %
In Sec.~\ref{Sec:DeltaSigma}, two different bias factors are discussed. In one case, the ``naive'' $\sigma$ is estimated using one segment of length $T$, which results in fewer effective averages, and a larger bias correction than our typical estimate of $\sigma$ which uses two adjacent segments of length $T$ and there twice as many averages.

\section{Marginalizing over calibration uncertainty}\label{sec:app_calibration}
Given measurements $\{\hat \Omega_i\}$ with uncertainties $\sigma^2_i$, as shown in Sec.\ref{sec:pe} the following likelihood function can be used to perform parameter estimation on the \gls{gwb}:
	\begin{equation}
    \label{eq:likelihood-again}
    p(\{\hat \Omega_f\} | {\bm \Theta})
    	= \mathcal{N} \exp\left[
        	-\frac{1}{2}\sum_f\frac{\left(\hat \Omega_f -  \Omega_{\rm M}(f|{\bm \Theta})\right)^2}{\sigma_f^2}\right].
    \end{equation}
Here, the $\{\hat \Omega_f\}$ are a set of estimators for the \gls{gw} energy density at discrete frequencies $f$, $\Omega_{\rm M}(f|{\bm \Theta})$ is a model for the energy density with parameters ${\bm \Theta}$, and $\mathcal{N}$ is a normalization constant.
We will consider only a single baseline and neglect the sum over detector pairs $IJ$ appearing in Eq.~\eqref{eq:likelihood}; if multiple detector pairs exist, the derivation below can be replicated for each pair.

Eq.~\eqref{eq:likelihood-again} assumes that our estimators $\{\hat \Omega_f\}$ are direct, unbiased measurements of the underlying energy-density spectrum.
In general, however, the imperfect amplitude and phase calibration of \gls{gw} detectors will break this assumption.
We can account for calibration uncertainty by amending our likelihood to introduce a new parameter $\lambda$:
	\begin{equation}
    \label{eq:likelihood-calibration-uncertainty}
    p(\{\hat \Omega_f\} | {\bm \Theta},\lambda)
    	= \mathcal{N} \exp\left[
        	-\frac{1}{2}\sum_f\frac{\left(\hat \Omega_f -  \lambda\Omega_{\rm M}(f|{\bm \Theta})\right)^2}{\sigma_f^2}\right].
    \end{equation}
The parameter $\lambda$ is an unknown multiplicative factor that encapsulates potential calibration inaccuracy.
In the case of perfect amplitude calibration ($\lambda=1$), then $\{\hat \Omega_f\}$ are direct measurements of the underlying (unknown) energy spectrum.
But if our calibration is imperfect ($\lambda\ne1$), then $\{\hat \Omega_f\}$ are instead measurements of some multiple $\lambda \Omega(f)$ of the \gls{gwb} spectrum.
Although we do not know $\lambda$, it is possible to estimate the \textit{uncertainty} on instrumental calibration.
We will therefore model $\lambda$ itself as an unknown variable drawn from a normal distribution centered at 1 (corresponding to perfect calibration) but with a variance $\epsilon^2$:
	\begin{equation}
    p(\lambda) \propto 
    	\exp\left[-\frac{1}{2\epsilon^2}\left(\lambda-1\right)^2\right],
    \end{equation}
where $\epsilon$ is a known amplitude calibration uncertainty. 
Additionally, we impose the constraint that $\lambda$ be positive: we expect errors in the amplitude of strain measurements but not their \textit{sign}.
In this case, the probability distribution for $\lambda$ becomes
	\begin{equation}
    \label{eq:plambda}
    p(\lambda) = \sqrt{\frac{2}{\pi}}
  		\frac{1}{\epsilon\left[1
        	+\mathrm{Erf}(\frac{1}{\sqrt{2\epsilon^2}})\right]}
        \exp\left[-\frac{1}{2\epsilon^2}\left(\lambda-1\right)^2\right],
    \end{equation}
normalized to unity on the interval $\lambda\in(0,\infty)$.
Eq. \eqref{eq:plambda} is our prior on $\lambda$.

We can now use Eq. \eqref{eq:plambda} to marginalize our likelihood (Eq.~\eqref{eq:likelihood-calibration-uncertainty}) over the unknown calibration factor $\lambda$.
The marginalized likelihood is given by
	\begin{equation}
    \begin{aligned}
    p(\{\hat \Omega_f \} | {\bm \Theta} )
        &= \int p(\{\hat \Omega_f \} | {\bm\Theta},\lambda) \,p(\lambda) d\lambda \\
        &= \mathcal{N} \sqrt{\frac{2}{\pi}}
  			\frac{1}{\epsilon\left[1
        		+\mathrm{Erf}(\frac{1}{\sqrt{2\epsilon^2}})\right]}
            \int_0^\infty \exp\left[
            	-\frac{1}{2}\sum_f \frac{\left(\hat \Omega_f - \lambda\Omega_{\rm M}(f|{\bm \Theta})\right)^2}{\sigma^2_f}
                -\frac{1}{2}\frac{\left(\lambda-1\right)^2}{\epsilon^2}
                \right] d\lambda.
    \end{aligned}
    \end{equation}
If we define 
	\begin{equation}
    A({\bm \Theta}) = \frac{1}{\epsilon^2}+\sum_f\frac{\Omega_{\rm M}(f|{\bm \Theta})^2}{\sigma^2_f},
    \end{equation}
    \begin{equation}
    B({\bm \Theta}) = \frac{1}{\epsilon^2}+\sum_f\frac{\hat \Omega_f \Omega_{\rm M}(f|{\bm\Theta})}{\sigma^2_f},
    \end{equation}
and
	\begin{equation}
    C({\bm \Theta}) = \frac{1}{\epsilon^2}+\sum_f\frac{\hat \Omega^2_f}{\sigma^2_f},
    \end{equation}
the marginal likelihood can be more concisely expressed as
	\begin{equation}
    \label{eq:likelihood-calib-2}
    p(\{\hat \Omega_f \} | {\bm \Theta} )
    	= \mathcal{N} \sqrt{\frac{2}{\pi}}
  			\frac{1}{\epsilon\left[1 +\mathrm{Erf}(\frac{1}{\sqrt{2\epsilon^2}})\right]}
            \int_0^\infty \exp\left[-\frac{1}{2}\left(
            	A({\bm \Theta})\lambda^2 - 2B({\bm \Theta})\lambda + C({\bm \Theta})
                \right)\right] d\lambda;
    \end{equation}
this expression can be analytically integrated to obtain
	\begin{equation}
    p(\{\hat \Omega_f \} | {\bm \Theta} ) =
    	\mathcal{N} \frac{1}{\epsilon\sqrt{A({\bm \Theta})}}
        \left[\frac{1+\mathrm{Erf}(\frac{B({\bm \Theta})}{\sqrt{2A({\bm \Theta})}})}
        	{1+\mathrm{Erf}(\frac{1}{\sqrt{2\epsilon^2}})}\right]
        \exp\left[-\frac{1}{2}\left(C({\bm \Theta})-\frac{B({\bm \Theta})^2}{A({\bm \Theta})}\right)\right].
    \end{equation}

Marginalization of calibration uncertainty is built into the {\tt pygwb\_pe} module, and this calculation is automatically triggered when passing a calibration error $\epsilon\neq 0$. Additional information on the treatment of calibration uncertainties can be found in \cite{Whelan:2012ur}.
\bibliographystyle{aasjournal}
\bibliography{pygwb_bib}

\end{document}
