Starting from an estimate of the \gls{gwb} spectrum $\hat{\Omega}_{{\rm GW}, f}$, with variance $\sigma^{2}_{{\rm GW}, f}$, it is possible to place stringent constraints on the \gls{gwb} amplitude using a hybrid frequentist-Bayesian approach. %
We consider the general case where we have a set of \gls{gwb} measurements $\hat{\Omega}^{IJ}_{{\rm GW}, f}$ from different detector pairs, or {\it baselines}, $IJ$. %
We define a Gaussian likelihood for $B$ pairs of detectors, 
\begin{equation}
\label{eq:likelihood}
p\qty(\hat{\Omega}^{IJ}_{{\rm GW}, f} | \mathbf{\Theta}) \propto\exp\left[  -\frac{1}{2} \sum_{IJ}^B \sum_f \left(\frac{\hat{\Omega}^{IJ}_{{\rm GW}, f}  - \Omega_{\rm M}(f|\mathbf{\Theta})}{\sigma^{IJ}_{{\rm GW}, f}}\right)^2  \right],
\end{equation}
where $\Omega_{\rm M}(f|\mathbf{\Theta})$ is the \gls{gwb} model and $\mathbf{\Theta}$ are its parameters. %
Bayes' theorem is used to obtain posterior distributions on the model parameters, %given the likelihood defined in Eq.~(\ref{eq:likelihood}) and the priors on those parameters:
\begin{equation}\label{eq:likelihood_params}
    p\qty(\mathbf{\Theta}|\hat{\Omega}^{IJ}_{{\rm GW}, f}) \propto p\qty(\hat{\Omega}^{IJ}_{{\rm GW}, f}| \mathbf{\Theta})\,p(\mathbf{\Theta})\,,
\end{equation}
where the priors $p(\mathbf{\Theta})$ are employed. %
In practice, when performing parameter estimation on a large dataset, we take the post-processed, {\it unweighted} (i.e., $\alpha=0$) estimate $\hat{\Omega}^{0, IJ}_{{\rm ref}, f}$ to be the measured \gls{gwb} spectrum in each frequency bin, and plug it into Eq.~\eqref{eq:likelihood}. %
Note that it is necessary for the input spectra used in parameter estimation to be unweighted as any other value would constitute a model choice and bias results. %

Within the {\tt pygwb} package, we include the {\tt pe} module to perform parameter estimation as an integral part of the analysis, which naturally follows the computation of the optimal estimate of the \gls{gwb}. %
This is a notable improvement compared to previous LVK analyses, where data products and parameter estimation were handled independently by packages in different programming languages. % %$\hat{C}^{IJ}(f_k)$ and $\sigma_{IJ}(f_k)$, after which a separate parameter estimation analysis followed.
Furthermore, the {\tt pe} module is a simple and user-friendly toolkit for any model builder to constrain their physical models with \gls{gw} data. %

The {\tt pe} module is built on class inheritance, with {\tt GWBModel} as the parent class. %
The methods of the parent class are functions shared between different \gls{gwb} models, e.g., the likelihood formulation in Eq.~(\ref{eq:likelihood}), as well as the noise likelihood, given by Eq.~(\ref{eq:likelihood}) with $\Omega_{\rm M}(f|\mathbf{\Theta})\equiv0$. %
It is possible to include calibration uncertainty by modifying the {\tt calibration\_epsilon} parameter, which defaults to 0. %
For details on the marginalization over calibration uncertainty, see App.~\ref{sec:app_calibration} and \cite{Whelan:2012ur}. %
The \gls{gw} polarization used for analysis is user-defined, and defaults to standard \gls{gr} polarization (i.e., tensor). %, still giving the user the flexibility to change the GW polarisation and explore models that predict a scalar or vector polarisation of the \gls{gwb}. %
More details on possible polarization choices can be found in Sec.~\ref{sec:baseline}. %
In our implementation of {\tt pe}, we rely on the {\tt Bilby} package~\cite{Ashton:2018jfp} to perform parameter space exploration, and employ the sampler {\tt dynesty} by default \cite{dynesty}. %
The user has flexibility in choosing the sampler as well as the sampler settings. %

Child classes in the {\tt pe} module inherit attributes and methods from the {\tt GWBModel} class. %
Each child class represents a single \gls{gwb} model, and combined they form a catalog of available \gls{gwb} models that may be probed with \gls{gw} data. %
The inheritance structure of the module makes it straightforward to expand the catalog, allowing users of the {\tt pygwb} package to add their own $\Omega_{\rm M}(f|\mathbf{\Theta})$ models. %
The flexibility of the {\tt pe} module allows the user to combine several \gls{gwb} models defined within the module. %
A particularly useful application of this is the modelling of a \gls{gwb} in the presence of correlated magnetic noise, as discussed in \cite{Meyers_2020}, or the simultaneous estimation of astrophysical and cosmological \gls{gwb}s \cite{PhysRevD.103.043023}. %
The {\tt pygwb} documentation~\cite{docs} contains information on the existing models in the catalog, with a description of the GWB models and their parameters. 