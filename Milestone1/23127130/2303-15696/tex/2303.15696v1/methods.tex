\label{sec: GWB analysis}

A \gls{sgwb} is characterized by its spectral emission, which is the target of stochastic \gls{gw} searches. %
    The spectrum is typically parametrized by the \gls{gw} fractional energy density spectrum $\Omega_{\rm GW}(f)$, such that
\begin{equation}
	\Omega_{\rm GW}(f) = \frac{1}{\rho_\mathrm{c}} \frac{\text{d}\rho_{\rm GW} (f)}{\text{d}\ln f}\,,
\label{eq:Omegarho}
\end{equation}
where $\text{d}\rho_{\rm GW}$ is the energy density of \glspl{gw} in the frequency band $f$ to $f+\text{d}f$, and $\rho_{\rm c}$ is the critical energy density in the Universe. When integrated over $\text{d}\log f$, $\Omega_{\rm GW}(f)$ gives the total dimensionless \gls{gw} energy density. %
The $\Omega_{\rm GW}(f)$ spectrum is thus directly related to the intensity of \glspl{gw}. %
Specifically, from Eq.~\eqref{eq:Omegarho} it follows that~\cite{Allen:1999}
\begin{equation}
	\Omega_{\rm GW}(f) = \frac{4\pi^2f^3}{\rho_\mathrm{c} G}S_h(f)\,,
\label{eq:omegatoI}
\end{equation}
where the strain spectral density $S_h(f)$ is defined as the polarization--averaged second moment of the stochastic \gls{gw} strain field, decomposed into its polarization components $h_{+}$ and $h_{\times}$,
\begin{equation}
\langle h^{}_+(f,\,\hat{\bm n})\,h^\ast_+(f',\,\hat{\bm n}')\rangle + \langle h^{}_\times(f,\,\hat{\bm n})\,h^\ast_\times(f',\,\hat{\bm n}')\rangle  = \,\delta^{(2)}(\bm{n},\bm{n'})\,\delta(f-f')\,
 S_h(f,\,\hat{\bm n})\,,
\label{eq:intensity}
\end{equation}
assuming statistical homogeneity. %
The unit vectors $\hat{\bm n}$, $\hat{\bm n}'$ span the 2-sphere, while $f\in \mathbb{R}$. %
In the plane wave formalism, $h_{+}$ and $h_{\times}$ in Eq.~\eqref{eq:intensity} are the Fourier coefficients of the time-domain strain fields. %
If these are stochastically distributed, these give rise to a \gls{sgwb} which we describe solely through the statistical moments of the distribution. % 
In particular, a Gaussian \gls{sgwb} is fully described by its second moments, hence the spectral density in Eq.~\eqref{eq:intensity} is the primary target of a search which assumes the signal to be both stochastic and Gaussian. %
More details on these quantities can be found for example in~\cite{Romano_2017}. %

Laser interferometers such as \gls{ligo} and Virgo are sensitive to the strain field in the time domain coming from all directions, $h(t)$. %
These detectors measure the \gls{gw} strain filtered through a linear response function $F$ (see definition in~\cite{Romano_2017}) plus a detector noise component $n$, which we may write in shorthand as
\begin{equation}
    d(t) = F(t) \star h(t) + n(t),
\end{equation}
where ``$\star$'' indicates a convolution operation. % 
Given that the \gls{sgwb} signal is weak and hard to distinguish from instrumental noise, cross-correlating two independent, time-coincident datastreams with uncorrelated noise is an effective way to construct an estimator for $\Omega_{\rm GW}(f)$. % 
We assume our target stochastic \gls{gw} signal is stationary, Gaussian, and isotropic. %
We further assume the detector noise is Gaussian and uncorrelated between detectors, which is a fair assumption in the case of ground-based interferometers at current detector sensitivity\footnote{In future detectors, correlated noise will become a significant problem, and quite a few methods for mitigating it have been proposed, including Wiener filtering and Bayesian parameter estimation~\cite{ThraneChristensen2013,ThraneChristensen2014,CoughlinChristensen2016,HimemotoTaruya2017,CoughlinCirone2018,HimemotoTaruya2019,Meyers_2020,JanssensMartinovic2021,JanssensBall2023,HimemotoNishizawa2023}.} (after specific mitigation)~\cite{LIGO_O3,JanssensBall2023}, and that the noise amplitude is much larger than the signal amplitude. %
Under these assumptions\footnote{Failure of stationarity or Gaussianity implies the estimator is sub-optimal, yet still valid~\cite{Drasco-Flanagan:2003, Lawrence:2023buo}; failure of isotropy would also induce a bias, and the target signal would be ill-defined~\cite{PhysRevD.107.023024}.}, it has been shown~\cite{SGWBH1H2, Romano_2017} that the cross-correlation--based \gls{mvue} of $\Omega_{\rm GW}$ at a frequency bin $f$ and the corresponding variance is given by,
%
\begin{equation}\label{eq:Omega}
    \hat{\Omega}_{{\rm GW}, f} = \frac{\Re[C_{IJ, f}]}{\gamma_{IJ}(f) S_0(f)} \ ,
\end{equation}
%
and
%
\begin{equation}\label{eq:Variance}
    \sigma^2_{{\rm GW,} f} = \frac{1}{2 T \Delta f} \frac{P_{I, f} P_{J, f}}{\gamma^2_{IJ}(f) S^2_0(f)},
\end{equation}
%
where $C_{IJ, f}$ is the one-sided \gls{csd} and $P_{I, f}$ is the one-sided (auto-)\gls{psd} of strain data $d_t$ from two detectors $(I,J)$, as defined below in Sec.~\ref{sec:spectral}\footnote{Note that, in previous works, the notation $C_{IJ}$ was used to define the cross-correlation statistic itself \cite{LIGO_O3}. This is not the case in this paper.}. %
Note that throughout this work we will denote continuous functions of the frequency with the notation $(f)$, whereas discrete functions of the frequency will be denoted with a subscript $_f$. %
Typically, in this paper, discrete functions of frequency are estimators for continuous functions, and in Equations such as Eqs.~\eqref{eq:Omega} and ~\eqref{eq:Variance} which mix discrete and continuous functions our notation implies that continuous functions are evaluated at the discrete set of frequencies for which we know the value of the discrete functions. % 
In the above, $T$ is the duration of data used to produce the above spectral densities, and $\gamma_{IJ}(f)$ is the cross-correlated \gls{gw} response, or \gls{orf}, which is the polarization-- and sky-- averaged cross-correlation of the individual detector responses, $F_I$. The \gls{orf} normalized for a pair of perpendicular-arm interferometers is given by~\cite{Allen:1999}
%
\begin{equation}
    \label{eq:orf}
    \gamma_{IJ}(f)  = \frac{5}{8\pi} \sum_{A}\int_{S^2} d \hat{\bm n} F^A_I(f, \hat{\bm n})F^A_J(f, \hat{\bm n}) e^{-i 2\pi f \hat{\bm n}\cdot({\bm x}_I-{\bm x}_J)}\,,
\end{equation}
%
where $\hat{\bm n}$ is the unit vector on the sky, in an arbitrary basis\footnote{The \gls{orf} in {\tt pygwb} is calculated in geocentric coordinates.}, ${\bm x}_I-{\bm x}_J$ is the difference between the position vectors of the two detectors $I$ and $J$ respectively, and $A$ spans the polarization basis. %
The \gls{orf} quantifies the reduction in sensitivity of the cross-correlation stochastic search due to the detectors not being co-aligned and co-located, and having different non-trivial responses. %
The function $S_0$ is defined as~\cite{Romano_2017, SGWB_review2022_AIR}
\begin{equation}
    S_0(f) = \frac{3H_0^2}{10\pi^2}\frac{1}{f^3},
    \label{eq:S0}
\end{equation}
and converts a \gls{gw} strain power spectrum into a fractional energy density. % 
The derivation of $S_0$ is shown in~\cite{Allen:1999}, and note that it includes the normalization factor of the \gls{orf}, $5/8\pi$, which ensures $\gamma_{IJ}(f)\equiv 1$ for co-aligned, co-located detectors.

There are two important considerations to make regarding the estimator in Eqs.~\eqref{eq:Omega} and~\eqref{eq:Variance}. %
Firstly, the implementation of a \gls{dft} over a finite time $T$ in the estimator of the continuous non-periodic quantity $\Omega_{\rm GW}(f)$ may create spectral artifacts, as seen in~\cite{numerical_recipes2007, whelan_CC_dcc}. %
We outline how this is handled in Sec.~\ref{sec:spectral}. %
Secondly, as the estimator is initially derived as a minimal variance estimator in the time domain~\cite{Allen:1999}, the narrow-band frequency estimator in Eq.~\eqref{eq:Omega} is actually obtained from a broad-band one, as will be clarified in Sec.~\ref{sec:postproc}. %
In the rest of this paper, we refer to $\hat{\Omega}_{{\rm GW}, f}$ as the optimal estimator of the signal spectrum ${\Omega}_{\rm GW} (f)$. % 
The optimality of the estimator can either be justified by the proof that this is an \gls{mvue}, or equivalently by showing that it maximizes a reasonable likelihood for the data. When performing parameter estimation as outlined in Sec.~\ref{sec:pe}, we in fact employ a Gaussian likelihood which is maximized by $\hat{\Omega}_{{\rm GW}, f}$.

In stochastic analyses with current interferometers, we take advantage of long observing times to improve detection statistics. %
In practice, the data are segmented into smaller chunks and analyzed individually before they are optimally combined to produce an estimate. This is convenient due to potential non-stationarities in the detector noise over both short time-scales, such as the length of an individual data segment, and long time-scales, such as the total observation time, as well as reducing computational costs. % 
Assuming each time segment is independent, we perform a weighted average over all segments to calculate $\hat{\Omega}_{{\rm GW}, f}$ for long observations. %
This average can be thought of as an approximation to the ensemble averages in Eq.~\eqref{eq:intensity}. Hence the more independent observations are averaged over, the better the measurement. The averaging procedure is described in full in Sec.~\ref{sec:postproc}.


The narrow-band statistic of Eqs.~\eqref{eq:Omega} and~\eqref{eq:Variance} assumes each frequency bin is independent. The information from each bin can be combined under the assumption of a known \gls{gw} spectral density distribution. In \gls{gwb} analyses, it is most common to assume a power-law spectral shape for $\Omega_{\rm GW}$,
\begin{equation}
    \Omega_{\rm GW}(f) = \Omega_{\rm ref}\left(\frac{f}{f_{\rm ref}}\right)^{\alpha}\,,
    \label{eq:omega-plaw}
\end{equation}
where $\alpha$ is the spectral index of the signal, and $f_{\rm ref}$ is a reference frequency, and $\Omega_{\rm ref}$ is defined as $\Omega_{\rm ref}\equiv\Omega_{\rm GW}(f_{\rm ref})$. %
Under this assumption, the rescaling
\begin{equation}
    \label{eq:Salpha}
    H_{\rm ref, \alpha}(f) = \qty(\frac{f}{f_{\rm ref}})^\alpha
\end{equation}
can be used to re-weight the estimate of the spectrum $\hat\Omega_{{\rm GW}, f}$, obtained for $\alpha=0$, to optimize the statistic for a specific spectral index $\alpha$ at a chosen reference frequency $f_{\rm ref}$, reducing the search to the estimation of a single number, $\Omega_{\rm ref}$. This procedure is referred to as {\it re-weighting} and is clarified in Sec.~\ref{sec:postproc}. %
Alternatively, it is also possible to keep $\alpha$ as a free parameter in the analysis, and estimate both $\Omega_{\rm ref}$ and $\alpha$ from the data. This is described in Sec.~\ref{sec:pe}.

