Once a set of data, comprised of an uninterrupted stretch of timeseries data, has been pre-processed and average \gls{psd} and \gls{csd} estimates have been calculated for each segment of data within the set, one can combine those separate time segments to construct a final, time-averaged estimate of the \gls{gwb} amplitude. %

Due to the aggressive windowing choice we typically make, and the subsequent overlapping of time segments, we must be careful in combining time segments together. %
The overlapping and windowing cause overlapping time segments to be correlated with one another. %
Within each processed set, individual time segments must be combined while accounting for this covariance. %
A detailed calculation and discussion of this covariance can be found in~\cite{lazz_romano_windowing_note}, while effective approximations to that full calculation can also be used (see, e.g. Sec. IIIB of~\cite{AinDalvi2015}). %

To start, we construct the estimate of the GWB in a single segment $t$. %
As detailed in Sec.~\ref{sec: GWB analysis}, the \gls{gwb} search is often framed in terms of constructing a point estimate for $\Omega_{\rm GW}(f_{\rm ref})$, the energy density of the \gls{gwb} at the specific frequency $f_{\textrm{ref}}$, assuming a power-law for the \gls{gwb} with spectral index $\alpha$. %
We refer to the estimator of this quantity as $\hat \Omega^\alpha_{\mathrm{ref}},$ in general, and for a single time segment of data, it can be constructed using a weighted average over the individual frequency bin estimators $\hat{\Omega}_f$ and ${\sigma}_f$ described in Eqs.~\eqref{eq:Omega} and~\eqref{eq:Variance} calculated per segment $t$, as
%
\begin{align}
\hat\Omega^\alpha_{\textrm{ref}, t} = \frac{\sum_{f} \hat\Omega_{t,f} H_{{\rm ref}, \alpha}(f)\bar\sigma^{-2}_{t, f}}{\sum_f H_{{\rm ref}, \alpha}^2(f) \bar\sigma^{-2}_{t,f}},\label{eq:omega_alpha}\\
\sigma^\alpha_{\textrm{ref},t} = \left[\sum_f H^2_{{\rm ref}, \alpha}(f)\bar\sigma_{t,f}^{-2}\right]^{-\frac{1}{2}},\label{eq:sigma_alpha}
\end{align}
% where $w(f)$ is the re-weighting function
% \begin{equation}
% w(f) = \left(\frac{f}{f_{\textrm{ref}}} \right)^{\alpha}.
% \end{equation}
where the rescaling $H_{{\rm ref}, \alpha}(f)$ is defined in Eq.~\eqref{eq:Salpha}. %
The average variance spectrum per segment, $\bar\sigma^2_{t, f}$, is calculated using average \glspl{psd} described in Sec.~\ref{sec:spectral}. %
These broadband quantities can be calculated for each time segment $t$, and then this set of estimators at each time can be combined to account for the overlap between time segments discussed above. %
We first lay out how to perform this combination assuming we have calculated the quantities above for each individual time segment. %
Then, we discuss how to alternatively average the estimators in each frequency bin over time independently, before combining them into an integrated quantity at the end. %
The latter calculation is normalized such that it gives the same result as the former. %
To avoid heavy notation we drop the bars that indicate average quantities in the rest of this section -- all variances used for the following calculations are average variances as defined above. %

To construct an estimator for the \gls{gwb} using a set of measurements in short, overlapping time segments, we first combine the segments that are non-overlapping. %
If the overlap between segments is $50\%$ or less, then this amounts to separately performing inverse-noise-weighted averaging over the even- and odd-indexed segments:
\begin{align}
\label{eq:odd_even_sigma}
    \sigma_{\mathrm{odd}}^2 &= \frac{1}{\sum_{t\in \mathrm{odd}}\sigma_t^{-2}}\\
    \label{eq:odd_even_omega}
    \Omega_{\textrm{odd}} &= \frac{\sum_{t\in \textrm{odd}}\Omega_t\sigma_{t}^{-2}}{\sum_{t\in\textrm{odd}}\sigma_{t}^{-2}},
\end{align}
where %we have assumed we have averaged over frequency first. %
the quantities $\Omega_t\equiv\hat\Omega^\alpha_{\textrm{ref}, t}$ and $\sigma_t\equiv\sigma^\alpha_{\textrm{ref}, t}$ for each time segment $t$. %
Analogous expressions are calculated for $\Omega_{\textrm{even}}$ and $\sigma_{\textrm{even}}$. %
Subscripts refer to even/odd time segments, and we drop here the subscripts $_{\rm GW}$, $_{\rm ref}$, and $_\alpha$ used to construct the integrated quantities to lighten the notation. %
We refer to the final, frequency- and time-averaged estimate as $\hat\Omega_\mathrm{ref}$ for now.

Next, we calculate the cross-covariance between point estimates in the odd and even segment combinations~\cite{lazz_romano_windowing_note},
\begin{align}
\sigma_{oe}^2 =\sigma_{eo}^2 &\equiv \langle \Omega_\mathrm{odd}\Omega_\mathrm{even}\rangle - \langle \Omega_\mathrm{odd}\rangle\langle \Omega_\mathrm{even}\rangle\\
&= \frac{1}{2}\frac{\bar{w}^4_\mathrm{ovl}}{\bar{w}^4}\left[\sigma_{\mathrm{odd}}^2 + \sigma_{\mathrm{even}}^2-\frac{1}{2}\sigma_{\mathrm{odd}}^2\sigma_{\mathrm{even}}^2\left(\sigma_{1}^{-2} + \sigma^{-2}_{2M-1}\right)\right],
\end{align}
where $M$ is the number of \textit{independent} segments and so $2M-1$ is the total number of overlapping segments, with the {\it window factors} $\bar{w}^4_\mathrm{ovl}$ and $\bar{w}^4$ as defined in App.~\ref{sec:app_window}. %
For the sake of compactness, we rewrite this as
\begin{align}
    \sigma_{oe}^2 &= \frac{k}{2}\sigma_{\mathrm{odd}}^2\sigma_{\mathrm{even}}^2 \sigma_{IJ}^{-2}\,,\\
    \sigma_{IJ}^2 &= \left[\sigma_{\mathrm{odd}}^{-2} 
 + \sigma_{\mathrm{even}}^{-2} - \frac{1}{2}\left(\sigma_{1}^{-2} + \sigma^{-2}_{2M-1}\right)\right]^{-1}\,,
\end{align}
where $k = \bar{w}^4_\mathrm{ovl} / \bar{w}^4$.

The covariance matrix between even/odd segment sets is then defined as
\begin{align}
    \bm{C} = \begin{pmatrix}\sigma_{\mathrm{odd}}^2 & \sigma_{\mathrm{oe}}^2 \\
            \sigma_{\mathrm{oe}}^2 & \sigma_{\mathrm{even}}^2
    \end{pmatrix},
\end{align}
which we use to construct the optimal combination of segments to obtain the point estimate $\hat{\Omega}_{\mathrm{ref}}$ and its variance $\sigma_{\mathrm{ref}}^2$. These are given by:
\begin{align}
    \hat\Omega_{\mathrm{ref}} &= \frac{\sum_{i=1}^2 \lambda_i \Omega_i}{\sum_{j=1}^2 \lambda_j},\label{eq:ptest_postpoc}\\
    \sigma_{\mathrm{ref}}^2 &= b^2_{\rm avg}\qty({\sum_{k=1}^2 \lambda_k})^{-2}\sum_{i=1}^2\sum_{j=1}^2 \lambda_i C_{ij} \lambda_j, \label{eq:sigma_postpoc}
\end{align}
with
\begin{align}
    \lambda_i = \sum_{j=1}^2 \left(\bm C^{-1}\right)_{ij}\,,
    \label{eq:lambda_i_broadband}
\end{align}
where $i,\,j$ indices label odd/even quantities. %
The bias factor $b_{\rm avg}$ which arises due to harsh windowing of the data has been included in Eq.~\eqref{eq:sigma_postpoc}. %
The derivation of the bias factor is described in App.~\ref{sec:app_window}. %
If combining over non-overlapping segments, then $\sigma_{\mathrm{oe}}^2 = 0,$ and this method reduces to the typical inverse-noise-weighted average that one would expect. %

The above expressions are for a broadband estimator, but in practice the {\tt postprocessing} module combines over time segments before combining over frequency bins. %
We refer to the estimated narrowband quantities as $\hat\Omega_{\mathrm{ref}, f}$ and $\sigma_{\mathrm{ref}, f}$. This notation indicates that, once a power-law spectral model is applied, the estimate in a frequency bin represents an estimate of the GWB at the reference frequency of the power law, assuming the chosen spectral shape.

We normalize $\hat\Omega_{\mathrm{ref}, f}$ and $\sigma_{\textrm{ref}, f}$ such that, when performing a weighted average over frequency bins \textit{after} combining  overlapping time segments we get the same result as Eqs.~(\ref{eq:ptest_postpoc}) and (\ref{eq:sigma_postpoc})  (which assume construction of a broadband estimator \textit{before} combining overlapping time segments). This results in the following expression for $\sigma_{\textrm{ref}, f}^{-2}$,
\begin{align}
\sigma_{\textrm{ref}, f}^{-2} &= b^{-2}_{\rm avg}\frac{\left[\sigma_{\mathrm{odd}, f}^{-2} + \sigma_{\mathrm{even}, f}^{-2} - k\sigma_{IJ, f}^{-2}\right]}{1 - \frac{k^2}{4}\sigma_{\mathrm{odd}}^2\sigma_{\mathrm{even}}^2\sigma_{IJ}^{-4}}\,,
\end{align}
and a corresponding expression for $\hat\Omega_{\mathrm{ref}, f}$,
\begin{align}
\hat\Omega_{\textrm{ref}, f} &= \frac{\Omega^{}_{\textrm{odd}, f}\sigma_{\mathrm{odd}, f}^{-2}\left(1 - \frac{k}{2}\sigma_{\mathrm{odd}}^2\sigma_{IJ}^{-2}\right) + \Omega^{}_{\textrm{even}, f}\sigma_{\mathrm{even}, f}^{-2}\left(1 - \frac{k}{2}\sigma_{\mathrm{even}}^2\sigma_{IJ}^{-2}\right)}{\sigma_{\mathrm{odd}, f}^{-2} + \sigma_{\mathrm{even}, f}^{-2} - k\sigma_{IJ, f}^{-2}}\,.
\end{align}
The even and odd estimators for each frequency bin are defined as in Eqs. (\ref{eq:odd_even_sigma}) and (\ref{eq:odd_even_omega}), except applied to individual bin-by-bin estimators calculated at each time segment. As discussed above, these expressions have been normalized such that 
\begin{align}
   \hat\Omega_{\textrm{ref}} = \frac{\sum_{f}\hat\Omega_{\textrm{ref}, f} \sigma_{\textrm{ref}, f}^{-2}}{\sum_f \sigma_{\textrm{ref}, f}^{-2}}\,, \\ 
   \sigma_{\textrm{ref}}^{2} = \left[\sum_f \sigma_{\textrm{ref}, f}^{-2}\right]^{-1}\,.
\end{align}

The \texttt{postprocessing} module implements the above expressions to estimate $\hat\Omega_{\mathrm{ref}, f}$ and $\sigma_{\mathrm{ref}, f}$ at a fixed $\alpha$, and returns them in form of an \texttt{OmegaSpectrum} object, which sub-classes the classic {\tt gwpy.FrequencySeries} and adds two key attributes: the spectral index $\alpha$ and the reference frequency $f_{\rm ref}$ at which the spectrum is calculated. %
By default, \texttt{pygwb} assumes a power-law spectral index $\alpha=0$ and a reference frequency $f_{\rm ref} = 25$ Hz when constructing the above estimators. % 
To explicitly include the $\alpha$ dependence in our results, we refer to the final postprocessed spectra as $\hat\Omega^\alpha_{\mathrm{ref}, f}$ and $\sigma^\alpha_{\mathrm{ref}, f}$.

One of the advantages of averaging over time before averaging over frequency is that one can reweight $\hat\Omega^\alpha_{\mathrm{ref}, f}$ and $\sigma^\alpha_{\mathrm{ref}, f}$ to be estimators for different choices of $\alpha$ without needing to average over all time segments again for a new choice of $\alpha$. %
The \texttt{OmegaSpectrum} object has a built-in method to perform a reweighting to change either $f_\mathrm{ref}$ or $\alpha$ used to calculate $\Omega_\mathrm{ref}$, employing the relation %
\begin{equation}
    \Omega^{{\rm ref}_1, \alpha_1}_{\rm GW}(f) = \Omega^{{\rm ref}_2, \alpha_2}_{\rm GW}(f) \frac{H_{{\rm ref}_1, \alpha_1}(f)}{H_{{\rm ref}_1, \alpha_2}(f)}\,,    
\end{equation}
derived using Eq.~\eqref{eq:omega-plaw}, which implies the following relation between amplitudes at different reference frequencies,
\begin{equation}
    \Omega_{\rm ref_1} = \Omega_{\rm ref_2} \frac{H_{{\rm ref}_2, \alpha}(f)}{H_{{\rm ref}_1, \alpha}(f)}\,.
\end{equation}
This allows to quickly calculate time- and frequency-averaged estimates of the \gls{gwb} amplitude associated with a specific power-law model. %

The default Hubble constant $H_0$, required in the scaling $S_0(f)$ in Eq.~\eqref{eq:S0}, is chosen to be $H_0 = 67.7$~km/(Mpc$\cdot$s), drawn from the Planck 2018 observations~\cite{Planck2018} and imported directly from the {\tt astropy} package. %
This is an attribute of the \texttt{OmegaSpectrum} and may be re-set by the user.