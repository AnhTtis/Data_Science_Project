\clearpage
\begin{table}[tbp]
\caption{Technical Majors }
\label{tab:majors}\vspace{2 mm}
\par
\centering
\begin{tabular}{ll}
\toprule
Transportation                        & Finance, Physics                          \\
Systems engineering                   & Finance, IT                               \\
Quantitative Econ, Financial Modeling & Finance and Information Syatems           \\
quantitative analysis                 & Environmental Engineering                 \\
Quantatitive finance                  & Engineering                               \\
Quant econ course                     & Eletrical, Ocean, Civil Engineering       \\
Physics and Economics                 & Eletrical Engineering, CS                 \\
Physics and applied math              & Electronics and communication enginerring \\
Physics                               & Electrical engineering                    \\
PhD Pharmaceutical sciences           & Electrical Electronic Engineering         \\
PhD in Pharmacology                   & Electrical Computer Engineering           \\
PhD in neuroscience                   & Electrical and electronics engineering    \\
PhD Biomedical/Medical Engineering    & Data analyst course                       \\
Computer                              & Computer Science                          \\
Mehcanical Engineering                & Computer engineering                      \\
Medical Chemistry                     & Computer engineering                      \\
Mechanical Engineering                & Computational biology, bioinformatics     \\
Mathematics, Finance                  & Computatioanl finance, information system \\
Mathematics, economics                & Civil Engineering                         \\
Mathematics, acturial studies         & Civil engineering                         \\
Mathematics minor                     & Chemistry, environmental engineering      \\
Mathematics and statistics            & Chemistry engineering                     \\
Mathematics and physics               & Chemistry and Applied mathematics         \\
Mathematics                           & Chemistry                                 \\
Mathematical science                  & Chemical Engineering                      \\
Mathematical economics analysis       & Biophysics and biochemistry               \\
Mathematical economics                & biophysics                                \\
Mathematic, Physics                   & Bio-Organic Chemistry                     \\
Mathematic and economics              & biomedical engineering                    \\
Math and Economics                    & BioMathematics                            \\
Materials Science                     & Biology and chemistry                     \\
Material metallurgical engineering    & Biology and Biochemistry                  \\
Masters in Data Science               & Biochemistry                              \\
Master of engineering                 & behavioral science                        \\
M.A. in Neuroscience                  & Information systems                       \\
Industrial Engineering                &                                           \\
Industrial and system engineering     &                                           \\
Industrial and management engineering &                                           \\
general engineering                   &                                           \\
Financial Modeling, Engineering       &                             \\   
\bottomrule
\end{tabular}
\end{table}
\clearpage


\clearpage
\begin{table}[tbp]
\caption{WRDS Financial Ratio }
\label{tab:WRDS_Financial_Ratio}\vspace{2 mm}
\par
\centering
\begin{tabular}{lll}
\toprule
Variable Name    & Financial Ratio                                & Category            \\
\midrule
capital\_ratio   & Capitalization Ratio                           & Capitalization      \\
debt\_invcap     & Long-term Debt/Invested Capital                & Capitalization      \\
equity\_invcap   & Common Equity/Invested Capital                 & Capitalization      \\
totdebt\_invcap  & Total Debt/Invested Capital                    & Capitalization      \\
at\_turn         & Asset Turnover                                 & Efficiency          \\
inv\_turn        & Inventory Turnover                             & Efficiency          \\
pay\_turn        & Payables Turnover                              & Efficiency          \\
rect\_turn       & Receivables Turnover                           & Efficiency          \\
sale\_equity     & Sales/Stockholders Equity                      & Efficiency          \\
sale\_invcap     & Sales/Invested Capital                         & Efficiency          \\
sale\_nwc        & Sales/Working Capital                          & Efficiency          \\
cash\_debt       & Cash Flow/Total Debt                           & Financial Soundness \\
cash\_lt         & Cash Balance/Total Liabilities                 & Financial Soundness \\
cfm              & Cash Flow Margin                               & Financial Soundness \\
curr\_debt       & Current Liabilities/Total Liabilities          & Financial Soundness \\
debt\_ebitda     & Total Debt/EBITDA                              & Financial Soundness \\
dltt\_be         & Long-term Debt/Book Equity                     & Financial Soundness \\
fcf\_ocf         & Free Cash Flow/Operating Cash Flow             & Financial Soundness \\
int\_debt        & Interest/Average Long-term Debt                & Financial Soundness \\
int\_totdebt     & Interest/Average Total Debt                    & Financial Soundness \\
invt\_act        & Inventory/Current Assets                       & Financial Soundness \\
lt\_debt         & Long-term Debt/Total Liabilities               & Financial Soundness \\
lt\_ppent        & Total Liabilities/Total Tangible Assets        & Financial Soundness \\
ocf\_lct         & Operating CF/Current Liabilities               & Financial Soundness \\
profit\_lct      & Profit Before Depreciation/Current Liabilities & Financial Soundness \\
rect\_act        & Receivables/Current Assets                     & Financial Soundness \\
short\_debt      & Short-Term Debt/Total Debt                     & Financial Soundness \\
cash\_conversion & Cash Conversion Cycle (Days)                   & Liquidity           \\
cash\_ratio      & Cash Ratio                                     & Liquidity           \\
curr\_ratio      & Current Ratio                                  & Liquidity           \\
quick\_ratio     & Quick Ratio (Acid Test)                        & Liquidity           \\
accrual          & Accruals/Average Assets                        & Other               \\
adv\_sale        & Avertising Expenses/Sales                      & Other               \\
rd\_sale         & Research and Development/Sales                 & Other               \\
staff\_sale      & Labor Expenses/Sales                           & Other               \\
aftret\_eq       & After-tax Return on Average Common Equity      & Profitability       \\
aftret\_equity   & After-tax Return on Total Stockholders Equity  & Profitability       \\
aftret\_invcapx  & After-tax Return on Invested Capital           & Profitability       \\
efftax           & Effective Tax Rate                             & Profitability       \\
\bottomrule
\end{tabular}
\end{table}
\clearpage

\clearpage
\begin{table}[tbp]
\ContinuedFloat
\caption{continued }
\vspace{2 mm}
\par
\centering
\begin{tabular}{lll}
\toprule
Variable Name    & Financial Ratio                                & Category            \\
\midrule
gpm              & Gross Profit Margin                            & Profitability       \\
GProf            & Gross Profit/Total Assets                      & Profitability       \\
npm              & Net Profit Margin                              & Profitability       \\
opmad            & Operating Profit Margin After Depreciation     & Profitability       \\
opmbd            & Operating Profit Margin Before Depreciation    & Profitability       \\
pretret\_earnat  & Pre-tax Return on Total Earning Assets         & Profitability       \\
pretret\_noa     & Pre-tax return on Net Operating Assets         & Profitability       \\
ptpm             & Pre-tax Profit Margin                          & Profitability       \\
roa              & Return on Assets                               & Profitability       \\
roce             & Return on Capital Employed                     & Profitability       \\
roe              & Return on Equity                               & Profitability       \\
de\_ratio        & Total Debt/Equity                              & Solvency            \\
debt\_assets     & Total Debt/Total Assets                        & Solvency            \\
debt\_at         & Total Debt/Total Assets                        & Solvency            \\
debt\_capital    & Total Debt/Capital                             & Solvency            \\
intcov           & After-tax Interest Coverage                    & Solvency            \\
intcov\_ratio    & Interest Coverage Ratio                        & Solvency            \\
bm               & Book/Market                                    & Valuation           \\
capei            & Shillers Cyclically Adjusted P/E Ratio         & Valuation           \\
dpr              & Dividend Payout Ratio                          & Valuation           \\
evm              & Enterprise Value Multiple                      & Valuation           \\
pcf              & Price/Cash flow                                & Valuation           \\
pe\_exi          & P/E excluding Extraordinary Items (Diluted)             & Valuation           \\
pe\_inc          & P/E including Extraordinary Items (Diluted)             & Valuation           \\
ps               & Price/Sales                                    & Valuation           \\
ptb              & Price/Book                                     & Valuation          \\
\bottomrule
\end{tabular}
\end{table}
\clearpage




\begin{table}[tbp]
	\caption{Predictable Forecast Errors}
 \label{tab:over_gmm}
	\begin{footnotesize}      
        This table report GMM regression results of regressing forecast error at $t+1$ on the information at time $t$. Forecast error at $t+1$ is $\text{Forecast Error}_{t+1} \coloneqq p_{t+1} - E_t p_{t+1} $, where $E_t$ is forecast based on date $t$ data. In columns (1), forecast $E_t p_{t+1}$ is the median earnings forecast of all analysts forecasts conducted during fiscal year t. In columns (3), forecast $E_t p_{t+1}$ is the median earnings forecast of all machine learning forecasts conducted during fiscal year t. Earnings are converted from EPS forecast to total earnings. Investment$_{t}$ capital expenditure at time $t$. Debt issuance$_{t}$ is net debt issuance at time $t$. All variables are scaled by total assets.
	\end{footnotesize}	
	\begin{center}
            \begin{tabular}{lcccc}
            \toprule
             & (1)              & (2)                   \\
             & Forecast   Error & Forecast   Error  \\
             & Analysts   & Machine  \\
             \midrule
            Investment&     -0.472***   & -0.345** \\
                          & (0.003)   & (0.025)  \\
            \midrule
            Firm FE & Yes        & Yes           \\
            Year FE & Yes       & Yes            \\
            Period  & 1994-2018 & 1994-2018  \\
            N       & 54536     & 53324      \\
            \bottomrule
            \end{tabular}
	\end{center}
\end{table}
\clearpage

%\begin{table}[tbp]
%	\caption{Predictable Forecast Errors Using Large Learning Rates}
  %      \label{tab:largelr}
%	\begin{footnotesize}    
 %        This table presents the results of an OLS regression analysis of the forecast error at time $t + 1$ and Wald test results from seemingly unrelated regressions. The forecast error at $t + 1$ is defined as $\text{Forecast Error}_{t+1} = p_{t+1} - F_t p_{t+1}$, where $F_t p_{t+1}$ is the average of median monthly earnings forecast based on data from time $t$, and $p_{t+1}$ represents the actual earnings at time $t + 1$. Forecasts are calculated from various machine learning models that use different hyper-parameters. The default hyper-parameters are set as the following values: depth of the tree is $2$ and number of trees in the forest is $50$. Table reports the results using different learning rates. Investment$_{t}$ is capital expenditure at time $t$. The $\chi^2$ of the Wald test is used to test the difference of regression coefficients from default hyper parameters, with the results of the test presented in the table. The p-value of the test is also included in brackets.
   %     \end{footnotesize}	
	%\begin{center}
        %\begin{tabular}{lcccc}
 %           \toprule
  %           & (1)              & (2)              & (3)              & (4)              \\
   %          & Forecast   Error & Forecast   Error & Forecast   Error & Forecast   Error \\
    %         & Analysts  &  Analysts   & Machine  & Machine  \\
     %        \cmidrule(lr){2-5}
      %       Learning rate & 0.15 & 0.25  & 0.3 & 0.5 \\
       %      \midrule
        %    Investment& -0.106***       & -0.107***       & -0.105***       & -0.099***       \\
%                & (-7.766)        & (-7.285)        & (-7.262)        & (-6.132)        \\
 %               \midrule
  %          $\chi^2$      & 0.019	&0.012	 & 0.072&1.261  \\
   %         & [0.890]	&[0.911]	 &[0.788]	& [0.261]  \\
    %        Firm FE & Yes             & Yes             & Yes             & Yes             \\
   %         Year FE & Yes             & Yes             & Yes             & Yes             \\
    %        Period  & 1994-2018       & 1994-2018       & 1994-2018       & 1994-2018       \\
     %       N       & 53321           & 53321           & 53321           & 53321           \\
      %      AdjR2   & 0.21            & 0.20            & 0.20            & 0.18           \\
       %     \bottomrule
        %    \end{tabular}
%	\end{center}
%\end{table}
%\clearpage


\begin{figure}[hbt!]
	\caption{Earning Predictions and Realizations}
	\label{fig:earnings_pre_realized}
 This figure shows the relation between predicted earnings and realized earnings. Earnings are scaled by total assets, which is equal to $\frac{\text{Predicted EPS*Total Shares Outstanding}}{\text{Total Assets}}$. The x-axis and y-axis are truncated at -0.5 to 0.5 to exclude outliers in the figure being presented.
	\begin{center}
        \subcaption*{Panel A}
	\begin{minipage}{0.55\textwidth}
		\centering
		\includegraphics[width=1.0\linewidth]{figures/Results_Analysts_Realized.png}
	\end{minipage}
        \subcaption*{Panel B}
        \begin{minipage}{0.55\textwidth}
        \centering
        \includegraphics[width=1.0\linewidth]{figures/Results_ML_Realized.png}
        \end{minipage}
	\end{center}
\end{figure}
\clearpage

\subsection{Predictable Forecast Errors}\label{sec:appframework}
In this section, we present a simple model of predictable forecast errors to clarify the overreaction tests and the sources of forecasting errors. We analyze how machine learning algorithm may improve the predictive power, and how future forecast errors could be correlated with historical information.

Assume analysts or Machine predicts the the value of EPS $y_{i,t}$ for firm $i$ in period $t$. $y_{i,t}$ follows the following process
\begin{equation}
    y_{i,t+1} = \delta y_{i,t} + f(z_{i,t})  + \varepsilon_{i,t+1}\label{process}
\end{equation}
where $f(z_{i,t})$ can be interpreted as firm/type specific component, but determined by information at time $t$. And the forecast is
\begin{equation}
    F_t y_{i,t+1} = \hat{\delta} y_{i,t} + \hat{f}(z_{i,t}) + \theta\hat\delta\varepsilon_{i,t},
\end{equation}
where $\theta>0$ captures the behavioral bias as in \cite{bordalo2021real}.

%The MSE of the forecast is then
%\begin{align}
 %   E(y_{i,t+1} - F_t y_{i,t+1})^2
 %   & =E((\delta - \hat{\delta})y_{i,t} + ( f(z_{i,t}) - \hat{f}(z_{i,t})) - \theta \hat\delta\varepsilon_{i,t} + \varepsilon_{i,t+1})^2 \label{s12}\\
%    & = (\delta - \hat{\delta})^2 E(y_{i,t})^2 + E( f(z_{i,t}) - \hat{f}(z_{i,t}))^2 + \theta^2\hat\delta^2 E(\varepsilon_{i,t})^2 + E(\varepsilon_{i,t+1})^2 \label{s22}.
%\end{align}
%Notice that to go from Equation \ref{s12} to Equation \ref{s22}, it is assumed that all four components in Equation \ref{s12} are uncorrelated with each other.

Without loss of generality assume that $\delta = \hat{\delta}$. Forecast Error is then
\begin{align}
    fe_{i,t+1} & \equiv y_{i,t+1} - E_t y_{i,t+1}  =  \delta y_{i,t} + f(z_{i,t})  + \varepsilon_{i,t+1} - (\hat{\delta} y_{i,t} + \hat{f}(z_{i,t})) - \theta\hat\delta \varepsilon_{i,t}  \\
    & = (\delta - \hat{\delta})y_{i,t} - ( f(z_{i,t}) - \hat{f}(z_{i,t})) - \theta\hat\delta \varepsilon_{i,t}\\
    & =  - \theta \hat\delta\varepsilon_{i,t} - ( f(z_{i,t}) - \hat{f}(z_{i,t}))\label{ferror}
\end{align}

Let $x_{i,t}$ be the observed firm fundamental (e.g., investment) that is linearly related to the forecasted variable $x_{i,t} = \alpha y_{i,t} + \epsilon^x_{i,t}$. We calculate the correlation between investment and forecast errors instead of using lagged earnings and forecast errors in order to address the potential measurement errors in EPS that could potentially affect our analysis (\citet{bordalo2021real}). Calculating the correlation between future forecast errors and current period information, we have
\begin{align}
    cov & (fe_{i,t+1},x_{i,t})  =
     cov( - \theta \hat\delta\varepsilon_{i,t} - ( f(z_{i,t}) - \hat{f}(z_{i,t})) , \alpha \varepsilon_{i,t} ) \label{s2} \\
    & = - \alpha\theta\hat\delta var(\varepsilon_{i,t})\label{s3}.
\end{align}
To get the Equation \ref{s2}, we assume that $f(z_{i,t})-\hat f(z_{i,t})$ is uncorrelated with information prior to time $t$. Equation \ref{s3} says when predictors exhibit behavioral bias ($\theta>0$), forecast error is negatively related to current information $x_{it}$.

With the existence of firm-level heterogeneity, machine learning could improve predictive power by allowing more factors and more complex functional forms. In this setting, we can still utilize the correlation between forecast errors and firm fundamentals to test forecast overreaction. Furthermore, this calculation demonstrates that disregarding firm-level heterogeneity $f(z_{it})$ does not impact the extent of overreaction. The analysis implies that even though machine learning models may produce improved forecasts (i.e., lower mean squared error), it does not necessarily imply that machine learning forecasts exhibit less overreaction.

\subsection{Model solution and proof}\label{sec:soln} 
\subsubsection{Equilibrium solution}
In this section, we provide detailed proof for Proposition \ref{prop:x}. To solve the model, we first solve for investors' equity demand as a function of the firm's equity issuance. We begin by solving for the investors' demand as a function of share price. Then we express their demand as a function of equity supply using the asset market clearing condition. Lastly, we derive the equilibrium solution from the firm's maximization problem. 

\textit{Step 1: Solve for investors' optimal portfolios, given information sets and equity issuance.}

From the investors' optimization problem, we can derive investors' demand for equity from the first order condition of Equation \ref{demand}:
\begin{align}
	& q_{jt} %= \frac{A_tE_j(y_t|\mathcal{I}_{t}) -p_tr_t}{\alpha A_t^2 V_j(y_t|\mathcal{I}_{t})}
	= \frac{A_t E[y_t|\mathcal{I}_{t-1},A_t,\eta_{jt},p_t]
		- p_tr_t}{\alpha A_t^2\rho_t^{-1}}\label{opt_demand},
\end{align}
where $E[y_t|\mathcal{I}_{t-1},A_t,\eta_{jt},p_t]
= \rho_t^{-1}(\rho_y\mu_{yt} + \rho_\eta \eta_{jt} + \rho_{pt} \eta_{pt})$.

The numerator is investors' expected net payoff from holding firm equity. The first term $E[y_t|\mathcal{I}_{t-1},A_t,\eta_{jt},p_t]$ is the expected payoff from one share of firm stock. However, holding firm stocks means investors cannot receive the return from riskless asset -- this is where the second term $p_tr_t$ comes from. The final demand is the net expected return adjusted by investor risk aversion and stock volatility. Note that the diagnostic expectation component $\theta\delta\epsilon_{t-1}$ is embedded in $\eta_{jt}$.%Recall that all investors believe the private signals they receive are unbiased, so $\eta_{jt} = y_t+\xi_{jt}$ with $\xi_{jt}\sim N(0, \rho_{\eta }^{-1})$.

\textit{Step 2: Solve for investor demand as a function of equity issuance using the market clearing condition.}

From the market clearing condition, we have
%\begin{align*}
%5	& \frac{A_t\rho_t^{-1}(\rho_y\mu_{yt}
%		+\rho_\eta (y_{t}+(1-\lambda)(\theta\delta\epsilon_{t-1}+f_t))
%		+\rho_{pt}\eta_{pt})
%		- p_tr_t}{\alpha A_t^2\rho_t^{-1}} =  x_t + x_t,\\
%	& p_t r_t %= A_t\rho_t^{-1}\rho_y\mu_{yt} + A_t\rho_t^{-1}\rho_{\eta }(y_t + (1-\lambda)(\theta\delta\epsilon_{t-1}+f_t))+A_t\rho_t^{-1}\rho_{pt}(y_t+e_t)-\alpha A_t^2\rho_t^{-1}( x_t + x_t),\\
%	 = A_t\rho_t^{-1}\rho_y\mu_{yt} + A_t\rho_t^{-1}\rho_{\eta }(1-\lambda)(\theta\delta\epsilon_{t-1}+f_t)
%	-\alpha A_t^2\rho_t^{-1} x_t
%	+ A_t\rho_t^{-1}\rho_{\eta }(y_t - A_t\frac{\alpha}{\rho_{\eta %}}x_t)+A_t\rho_t^{-1}\rho_{pt}(y_t+e_t).
%\end{align*}
\begin{align*}
	& p_t r_t 
	 = \frac{A_t}{\rho_t} [\rho_y\mu_{yt} + \rho_{\eta }(1-\lambda)(\theta\delta\epsilon_{t-1}+f_t) 
	+ \rho_{\eta }(y_t - A_t\frac{\alpha}{\rho_{\eta }}x_t)+\rho_{pt}(y_t+e_t) 
 -\alpha A_t x_t].
\end{align*}

Previously, we guess that the price $p_t$ is equivalent to a signal $\eta_{pt}=y_t+e_t$ to investors. Both sides of the equation above should be $y_t+e_t$ measurable in order for this guess to be true. It turns out that when $e_t$ equals the value below, the above equation holds, thus our guess is verified:
\[e_t = -A_t\frac{\alpha}{\rho_{\eta }}\tilde x_t.\]
% When supply high, price is low. Investor j thinks this could be because other investors' demand is low as the signals they receive \eta is low. So a high x translates to a signal suggesting a low z.
Therefore,
\begin{align}
	& p_t %= \frac{A_t\rho_t^{-1}}{r_t}[\rho_y\mu_{yt} + \rho_{\eta }(1-\lambda)(\theta\delta\epsilon_{t-1}+f_t)
	%-\alpha A_t x_t
	%+(\rho_{\eta }+\rho_{pt})(y_t-A_t\frac{\alpha}{\rho_{\eta }}x_t)],\\
	= \frac{A_t}{r_t\rho_t }[\Gamma_t - \alpha A_t x_t +
	(\rho_{\eta }+\rho_{pt})(y_t-A_t\frac{\alpha}{\rho_{\eta }}\tilde x_t)]\label{Tdemand},
\end{align}
where $\Gamma_t\equiv \rho_y\mu_{yt} + \rho_{\eta }(1-\lambda)(\theta\delta\epsilon_{t-1}+f_t)$, $\rho_{pt} = \frac{\rho_\eta^2}{\alpha^2A_t^2}\rho_x$\footnote{Note that $A_t$ is public information known to all at the very beginning of each period.}.

Equation \ref{Tdemand} gives the demand function of the risky asset. Holding everything else constant, a rise in $\theta$ leads to a rise in the stock price $p_t$ if the recent firm-level shock $\epsilon_{t-1}$ is positive. Intuitively, if investors overestimate the value of $y_t$ following a positive shock, they are willing to pay more to get one share of stock.

\textit{Step 3: Solve for firm equity issuance.}

From the first order condition of the firm's maximization problem, we can obtain the supply function. The firm owner does not use any signals to update her belief on $y_t$, thus the first order condition gives
\begin{align*}
	& E[p_t +  x_t\frac{\partial p_t}{\partial  x_t}|\mathcal{I}_{t-1}, A_t] - \phi( x_t-x_{t-1}) = 0,\\
	& E[p_t|\mathcal{I}_{t-1}, A_t] -  x_t \frac{\alpha A_t^2\rho_t^{-1}}{r_t} - \phi( x_t -  x_{t-1}) = 0.
\end{align*}

Plug Equation \ref{Tdemand} into the equation above. Given that the noisy supply has a mean of 0, we have
\begin{align}
	& \frac{A_t\rho_t^{-1}}{r_t}[\Gamma_t - \alpha A_t x_t +
	(\rho_{\eta }+\rho_{pt})\mu_{yt}] -  x_t \frac{\alpha A_t^2\rho^{-1}}{r_t} - \phi( x_t - x_{t-1}) = 0.\label{eq}
\end{align}
Rearranging Equation \ref{eq} gives us the equilibrium value of $ x_t$ :
	\begin{align}
		 x_t %=\frac{\frac{A_t\rho^{-1}}{r_t}[ \rho_y\mu_{yt} + \rho_{\eta }((1-\lambda)(\mu_{yt}+\theta\delta\epsilon_{t-1}+f_t)+\lambda \mu_{yt}) + \rho_p\mu_{yt}]-\tilde\phi_1 + \phi_2\bar x_{t-1}}{2\frac{\alpha A_t^2\rho^{-1}}{r_t} + \phi_2}
  =\frac{\frac{A_t}{\rho_t r_t}[ (\rho_y+\rho_{\eta }+\rho_{pt})\mu_{yt} + \rho_{\eta }(1-\lambda)(\theta\delta\epsilon_{t-1}+f_t) ] + \phi x_{t-1}}
		{2\frac{\alpha A_t^2}{\rho_t r_t} + \phi}\label{eissuance},
	\end{align}
	where $\rho_t =(\rho_y+\rho_{\eta }+\rho_{pt})$, $\rho_{pt} = \frac{\rho_\eta^2}{\alpha^2A_t^2}\rho_x$, $\mu_{yt} = (1-\delta)\mu_y + \delta y_{t-1}$.

\subsubsection{Proof of corollaries}
Bellow we give the proof for Corollary \ref{col:lambda}.
\begin{proof}
The overreaction of equity issuance to the recent shock is
\begin{align*}
   & \frac{\partial x_t}{\partial \epsilon_{t-1}}
   = \frac{\frac{ A_t}{\rho_t r_t}\rho_\eta(1-\lambda)\theta\delta}{2\frac{\alpha A_t^2}{\rho_t r_t} + \phi}.
\end{align*}
The response of this overreaction to changes in $\lambda$ is
\begin{align*}
   & \frac{\partial x_t}{\partial \epsilon_{t-1}\partial \lambda}
   = -\frac{\frac{A_t}{\rho_t r_t}\rho_\eta\theta\delta}{2\frac{\alpha A_t^2}{\rho_t r_t} + \phi} < 0.
\end{align*}
Therefore, the overreaction decreases as $\lambda$ rises.
\end{proof}

\subsection{Prompt used in ChatGPT}\label{sec:ai} 
For each EPS announcement month of a given firm, we use the information 12 months before that date to make predictions. For example, for the EPS to be announced at month $t$, the information we use is that available at month $t-12$. We do this instead of using all dates prior to the announcement in order to save computation time. Then for each firm-month, we submit the above prompt to ChatGPT-4o, and save the EPS it gives back to us. We use this EPS as the AI prediction. Then we calculate the difference between this and the actual EPS to get the AI prediction error.


Below is an example of the prompt we used to generate GenAI prediction (we insert the values and simple descriptions of all of our predictors used in our baseline prediction as in Table \ref{tab:WRDS_Financial_Ratio}).

``
You are a helpful assistant. I would like you to make predictions about firm's earnings per share using the information I give you. I know the following information about a public firm in the US. Please predict the firm's earning per share. Please just give a single number of EPS, and I do not want a range. When you report the EPS, please keep 3 decimals and use the following format: `the eps I predict is 4.000'\\
\indent Unemployment Rate 2\%, Real GDP Growth 4\%, Short-Term Debt/Total Debt 0.5..."\footnote{The values of unemployment, real GDP growth and the percentage of short term debt are made up and are only used for illustration purpose.}

\