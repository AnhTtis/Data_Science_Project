\documentclass[11pt,english]{article}

\usepackage{amssymb,amsfonts,amsmath}
\usepackage{mathtools}
\usepackage{setspace}
\usepackage{sourceserifpro}
\usepackage[T1]{fontenc}
\usepackage{graphicx}
\usepackage[left=1in,right=1in,top=1in,bottom=1in]{geometry}
\usepackage[authoryear]{natbib}
\usepackage{enumerate}
\usepackage{hyperref}
\usepackage{booktabs}
\usepackage{pdflscape}
\usepackage{tabularx}
\usepackage{longtable}
\usepackage{epstopdf}
\usepackage[utf8]{inputenc}
\usepackage{multirow}
\usepackage{subcaption}
\usepackage[flushleft]{threeparttable}
\usepackage{tikz}
\usepackage{bbm}
\usepackage{forest}
\usepackage{float}
\usepackage{rotating}

\usepackage{algorithm}
\usepackage{algpseudocode}
%\newtheorem{assumption}[theorem]{Assumption}

\hypersetup{
colorlinks=true,
linkcolor=blue,
filecolor=magenta,
urlcolor=blue,
citecolor=blue,
}
\urlstyle{same}
\def\sym#1{\ifmmode^{#1}\else\(^{#1}\)\fi}
\newcommand*\file[1]{\href{run:#1.pdf}{#1}}
\urlstyle{same}
\newtheorem{theorem}{Theorem}[section]
\newtheorem{lemma}[theorem]{Lemma}
\newtheorem{proposition}[theorem]{Proposition}

\newtheorem{corollary}[theorem]{Corollary}
\newenvironment{proof}[1][Proof]{\begin{trivlist}
\item[\hskip \labelsep {\bfseries #1}]}{\end{trivlist}}

\begin{document}
\setstretch{1.5}
\begin{titlepage}
\title{Behavioral Machine Learning? \\ Computer Predictions of Corporate Earnings also Overreact
\thanks{Respective affiliations are Murray
Frank, murra280@umn.edu,
University of Minnesota, Minneapolis, MN. Jing Gao, gao00268@umn.edu,
University of Minnesota, Minneapolis, MN. Keer Yang,
kkeyang@ucdavis.edu, University of California at Davis, Davis, CA  }
}

\author{Murray Z. Frank \and Jing Gao \and Keer Yang}

\date{March 22,  2023}
\maketitle
\thispagestyle{empty}

\begin{abstract}
\begin{singlespace}
There is considerable evidence that machine learning algorithms have better predictive abilities than humans in various financial settings. But, the literature has not tested whether these algorithmic predictions are more rational than human predictions. We study the predictions of corporate earnings from several algorithms, notably linear regressions and a popular algorithm called Gradient Boosted Regression Trees (GBRT). On average, GBRT outperformed both linear regressions and human stock analysts, but it still overreacted to news and did not satisfy  rational expectation as normally defined. By reducing the learning rate, the magnitude of overreaction can be minimized, but it comes with the cost of poorer out-of-sample prediction accuracy. Human stock analysts who have been trained in machine learning methods overreact less than traditionally trained analysts. Additionally, stock analyst predictions reflect information not otherwise available to machine algorithms.

%\noindent Key Words: Expected profit, gradient boosting, firm
%financing decisions, behavioral finance  \\
%\noindent JEL Classification: G17, G32, G40
\end{singlespace}

\end{abstract}
\clearpage


\end{titlepage}

\tableofcontents
%\listoftables

\DeclareGraphicsExtensions{.pdf,.png,.gif,.jpg}
\newpage

%%%%%%%%%%%%%%%%%%%%%%%%%%%%%%%%%%%%%%%%%%%%%%%%%%%%
\section{Introduction}

According to \citet*{bordalo2022overreaction} there is strong evidence that the expectations of professional forecasters, corporate managers, consumers, and investors are often biased towards overreaction. This evidence seriously challenges the rational expectations and efficient markets hypotheses. This bias is said to be rooted in human psychology, notably selective recall and a tendency to focus on salient information. This perspective can also help to explain certain financial and macroeconomic phenomena that are otherwise difficult to understand, \citep{bordalo2020overreaction,bordalo2021real,bordalo2021diagnostic}.

While there is evidence that machine learning has better predictive power than humans in some finance finance settings, it is unclear whether machine algorithms produce more rational predictions than humans. To address this question, we study the predictions of corporate earnings from several algorithms, notably linear regressions\footnote{In untabulated results we also find that predictions generated using the classical Fama-Macbeth method to estimate parameters \citep{fama1973risk,fama2006profitability} also overreact.} and a popular algorithm called Gradient Boosted Regression Trees (GBRT), see \citet{friedman2001greedy, friedman2002stochastic, chen2016xgboost}. We test for overreaction following the approach used in \citep{bordalo2021real}. On average, GBRT outperformed both linear regressions and human stock analysts. However, like human forecasts, the machine learning algorithm predictions also overreact to news and do not satisfy rational expectation as normally defined.

It is of interest to see if we can control the degree of overreaction of the algorithm. Our first attempt was to alter the variables in the data set fed to the algorithm. This had little impact on the overreaction. Next we consider the fact that the main results use hyper-parameters selected by cross-validation. We therefore examines alternative, less optimized hyper-parameters to see if we can get rid of overreaction. The overreaction can be reduced by lowering the learning rate, but this also reduces the algorithm's responsiveness to new information. If the algorithm ignores news it does not overreact to the news. 

We paid considerable attention to comparing the computer algorithmic predictions to those of human stock analysts. When compared to predictions from linear regressions and from professional stock analysts, the machine learning predictions overreaction is not as strong. However due to the overreaction the machine learning generated predictions reject the hypothesis of rational expectations as normally defined. 

Using hand collected data on stock analysts' training we distinguish those with training in adopting machine learning methods from those with a more conventional training. Analysts with a machine learning background overreact less to news, even after we standardize on the firms being covered. It is worth stressing that while use of machine learning methods is helpful in some respects it does not strictly dominate human analysts. We show that the human stock analyst predictions contain information beyond that contained in standard databases of macroeconomic variables and firm accounting attributes. 

To understand the implications of the growing use of machine learning for equity market equilibrium, we provide a simple model that shows the effect of increasing the number of tech-oriented analysts. Our model is based on \citet{begenau2018big}. In the model equilibrium, the presence of a growing number of technically trained stock analysts reduces the magnitude of overreactions. Furthermore, the change to the training of stock analysts affects the amount of equity issued in the equilibrium. The greater rationality by in effect reducing the amount of overoptimism on the part of investors in response to a good shock, means that there is reduced demand for risky assets. Since equity is a risky claim, there is less demand for equity. In the equilibrium this means that less equity is issued due to the increasing importance of technically trained stock analysts. While we do not have a direct test of this implication, empirically we distinguish earlier years (1994-2012) to more recent years (2013-2018) since it has been claimed that in recent years machine learning methods are much more widely used. Overreaction does seem to be weaker in the more recent years.


\paragraph{Related Literature}
This paper is related to a growing body of literature that investigates the predictive power of machine learning in finance. Our empirical setting is similar to that of \citet{van2022man} and \cite{cao2021man}, who also compare the predictive power between analysts and machine learning models. Specifically, we find that using a gradient boosting regression tree model leads to a similar magnitude of decrease in mean squared errors as documented in \citet{van2022man}. However, in our setting to predict annual earnings, the efficacy of the machine learning model is not as strong as in \citet{cao2021man}. Though \citet{cao2021man} finds similar evidence that machine learning models perform worse after the 2008 financial crisis. Our paper is the first to document the overreacting behavioral of machine learning corporate earnings forecasts.

\citet{bianchi2022belief} uses machine learning as a benchmark to study the belief distortion of individual analysts. Relative to that paper our results are novel in that, rather than using machine learning as a benchmark to study belief distortion, we study the quality and the rationality of the machine learning predictions themselves. \citet{bianchi2022belief} find that machine learning does not overreact in their empirical setting studying the time series of macro variables. Our results differ in that our machine learning forecasts of firm earnings overreact. We also provide novel results that show the difference between the effects of aggregate shocks from firm-specific shocks. Predictions for the aggregate shocks exhibit some overreaction, but notably less than is found for firm-specific shocks.

The rest of the paper is organized as follows. Section \ref{sec:Data} describes the sources of information we feed into the prediction algorithms. Empirical strategies are explained in Section \ref{sec:strategies} and the test for overreaction is described. Section \ref{sec:MainResults} provides key test results. Information about predictions by stock analysts is provided in Section \ref{sec:analysts}. Section \ref{sec:model} provides a formal model that we use to help interpret the evidence. The conclusion is Section \ref{sec:conclusion}.
%%%%%%%%%%%%%%%%%%%%%%%%%%%%%%%%%%%%%%%%%%%%%%%%%%%%
\section{Data}
\label{sec:Data}
Our main empirical analysis uses four data sources: (1) Firm earnings forecasts (EPS) by equity analysts, sourced from IBES; (2) Firm financial data from Compustat, and stock return data from the Center for Research in Security Prices (CRSP); (3) Macro time series downloaded from Federal Reserve Bank of Philadelphia real time data; and (4) Manually collected background information on equity analysts obtained from LinkedIn and cross-references with FINRA brokercheck.

The IBES database provides micro-level data on firm-level forecasts made by equity analysts. These analysts make monthly forecasts before the end of the fiscal year. We use equity analysts' forecasts of firms' annual earnings per share (EPS) with a forecasting horizon ranging from 12 to 23 months. The forecasting horizon is the difference in months, between the fiscal year end month of the realized annual EPS to be predicted, and the month when the analyst made the forecast. For each firm and each month, we take the median value of the forecasts as the median consensus forecast. We use realized annual EPS data from IBES Actual files.

%%%%%%%%%%%%%%%%%%%%%%%%%%%%%%%%%%%%%%%%%%%%%%%%%%%%

\section{Empirical Strategies}
\label{sec:strategies}
\subsection{Using ML to Make Predictions}
The machine learning predictions mimics the forecasts by equity analysts. Equity analysts made forecasts monthly on the annual EPS of the firm. For EPS of firm i in each fiscal year $y$, we ask the machine learning model to make predictions each month from 12 to 23 months prior to the end month of the fiscal year, using the most updated information available during the month in fiscal year $y$. For prediction made in year $y$ at year-month $t$, we use observations in the past 5 years, from $y-5$ to $y-1$, to train the model. Suppose our forecast horizon is $h$ month, $h = 12, 13.., 23$. We made forecast using information available in year $y$ at year-month $t$, denote as $\Phi_{i,t}$. Because of forecast horizon is $h$, we want to predict EPS of firm $i$ at year-month $t+h$, $t+h$ should be firm i' fiscal year end month. If the year-month $t+h$ is not firm's fiscal year end month, we do not make predictions.

The information set $\Phi_{i,t}$ of firm $i$ at year-month $t$ includes: (1) last annual EPS, the last three quarters' quarterly EPS; (2) 67 financial ratios from the Financial Ratios Suite by WRDS, which provide firm-month level information that is available at year-month $t$; (3) macro variables calculated using Federal Reserve Bank of Philadelphia real-time data; and (4) the median consensus forecast of analysts at each firm-month. We also use the monthly median consensus forecast of analysts to incorporate the idea of the importance of analysts' private information. For each investment horizon $h$ at each month-year $t$ (denote $y$ as the calendar year of the month-year $t$), we use a training sample consisting of five-year observations prior to the forecast date year $y$. The training sample includes the ${X_{i,\tau}, Y_{i,\tau+h}}$ where $\tau$ are within year $y-5$ and $y-1$, and $Y_{i,\tau+h}$ is publicly available by the end of $y-1$.\footnote{This restriction is stricter than $\tau+h \leq t$ because earnings are usually reported with a lag.} We retrain the model each year to make forecasts for the following year.

If we are given the information set $\Phi_{i,t}$ in year $y$ at year-month $t$ and we want to forecast the annual EPS realization in the fiscal-year end month $t+h$, we can make predictions based on the following model:
\begin{equation}
    F_{t} X_{i,t+h} = f_y(\Phi_{i,t})
\end{equation}
where $f_y$ is trained using the past 5-year observations. $F_{t} X_{i,t+h}$ is the machine learning model's year-month $t$ forecast for firm $i$'s annual EPS realization at the fiscal year end month $t+h$. Similar to analysts, the machine learning model could make forecast at any month before fiscal year end month.

\paragraph{Gradient boosting regression tree model}
The machine learning algorithm we used is gradient boosting regression tree model. Gradient boosting regression trees are a particularly prominent and
empirically successful prediction method used in many applications. It starts
with regression trees and then refines them iteratively by focusing on the
errors in the previous iteration. This idea of refinement by focusing on error
correction is known as boosting. Gradually an entire forest of trees is
constructed. Then an average across the set of trees is used as the model's
prediction.
\begin{figure}[H]
    \caption{Tree Example}
    \label{fig:TreeExample}
    \centering
    \begin{forest}
	for tree={l sep+=.8cm,s sep+=.5cm,shape=rectangle, rounded corners,
		draw, align=center,
		top color=white, bottom color=white}
	[{ $\pi_t < 0.5$}
	[ {ln $B_t/M_t < 1$} , edge label={node[midway,left]{True} }
	[{$\pi_{t+1} = 0.1$ }, edge label={node[midway,left]{True}}  ]
	[{$\pi_{t+1} = 0.5$ }, edge label={node[midway,right]{False} }  ]
	]
	[{$\pi_{t+1} = 0.8$ }, edge label={node[midway,right]{False} }]
	]
    \end{forest}
\end{figure}

Gradient boosting is the most widely adopted version of boosted forests. It
starts by estimating decision trees with fixed shallow depth. Then it computes
the residuals for the trees. At the next iteration more weight is devoted to
the cases in which the model fit poorly. In the end an ensemble of trees are
used to `vote' on the appropriate results. This generally reduces the bias in a
simple tree model while maintaining the low variance. The main drawback
relative to a simple tree is that forests
do not have such simple depictions that show how each variable affects the
final result.

There are three important hyper-parameters in gradient boosting: the depth of the
tree $d$ (max$\_$depth), the number of trees in the forest $M$ (n$\_$%
estimators), and learning rate $\gamma$ (learning$\_$rate). The default
parameters we use have the following values: depth of the tree is $2$, number of
trees in the forest is $50$, and learning rate is $0.1$. This set of hyper-paramters are obtained through cross-validation method using sample before 1994. We have
systematically carried out the analysis with the hyper-parameters optimized using cross-validation. The hyper-parameters were chosen to optimize the out-of-sample mean squared errors before 1994.  Except where noted, the Tables use the default hyper-parameters.

A more formal representation can be found in Section \ref{sec:ml_parameters} when we examine the variation in model hyper-parameters. Gradient boosting estimation was done using the software
GradientBoostingRegressor from
\url{https://scikit-learn.org/stable/modules/ensemble.html}, and also using XGBoost from
\citet{chen2016xgboost}. The results for the two implimentations of gradient boosting are very similar. The reported results in the Tables use the algorithm GradientBoostingRegressor from \url{https://scikit-learn.org/stable/}.


\paragraph{Linear Regressions}
Similarly, we could use linear regression models to forecast EPS. The forecasting procedures strictly follow the procedure used in gradient boosting forecasts. The only difference here is that the function $f_{y}$ used to fit the training sample is a linear regression model. We use the same forecasting variable $\Phi_{i,t}$ as in the gradient boosting model.

\paragraph{Mean Squared Errors}
We use average mean squared errors $MSE$ to measure the predictive performance of each model.
\begin{equation}
	MSE  = \frac{\sum_i(X_{i,t+1} - F_t X_{i,t+1} )^2}{N}
\end{equation}


\subsection{Testing for Overreaction}\label{sec:overreaction}
We follow the framework proposed by \citet{bordalo2021real} to test for overreaction in analysts' and machine forecasts. We start with the monthly consensus forecasts made by analysts and the monthly forecasts made by machine learning model. For each EPS of firm $i$ at fiscal year $t+1$, we have machine learning forecasts and consensus analysts forecasts made 12 to 23 months prior to the fiscal year end month. We take the average of the 12 median consensus forecasts. This is the average forecasts made during fiscal year $t$ on the EPS of firm at fiscal year $t+1$, denoted as $F_t X_{t+1}$.

We merge the firm-year level forecasts to standard firm financial data from COMPUSTAT. We then convert the per share forecasting error to per total assets forecasting error, by multiplying firm's total number of shares and divided by total assets. The results are similar when we instead scaled by total capital stock. All firm level variables are scaled by total assets. We further restrict our sample to forecast made from 1994 to 2018, with at least one analyst forecast, resulting in $54534$ firm-year observations.

To test overreaction, we run the following regression.
\begin{equation}
    x_{i,t+1} - F_{i,t} x_{i,t+1} = \beta y_{i,t} + \lambda_t + \mu_i + \varepsilon_{i,t}
\end{equation}
The dependent variable is the forecast error for EPS at fiscal year $t+1$. We regress the forecast error at $t+1$ on firm outcome at $t$ that captures the positive information at time $t$. In practice, following \citet{bordalo2021real}, we use investment, instead of EPS at time $t$, to eliminate the cross-contamination of measurement errors. The variable of interest is $\beta$. The econometric framework in Section \ref{sec:framework} shows that $\beta = \frac{-\theta\delta var(\varepsilon_{i,t})}{var(y_{i,t})}$, where $\varepsilon_{it}$ is the forecast error for firm $i$ at time $t+1$ and $y_{it}$ is the actual EPS for firm $i$ at time $t+1$. A negative $\beta$ indicates overreacting forecasts, while a positive $\beta$ indicates under-reacting forecasts. To address concerns about downward dynamic panel bias due to the inclusion of firm fixed effects, we provide GMM estimations in the appendix.

To examine the magnitude of overreaction, we also compare the difference in overreaction coefficients ($\beta$) using different forecast models. We use seemingly unrelated regressions to compare the regression coefficients using forecasting errors from different forecast models. We report $\chi^2$ statistics and p-values of Wald tests calculated using covariance estimated from the seemingly unrelated regressions. Because any contamination from measurement errors and fixed effects is the same when we use different forecast errors, we can directly compare the coefficients $\beta$. A lower $\beta$ indicates a lower overreaction coefficient ($\theta$).

\subsection{Predictable Forecast Errors}\label{sec:framework}
In this section, we provide a simple model of predictable forecast errors to clarify the sources of errors. We analyze how machine learning algorithm may improve the predictive power, and how future forecast errors could be correlated with historical information.

Assume analysts or Machine predicts the the value of EPS $x_{it}$ for firm $i$ in period $t$. $x_{it}$ follows the following process
\begin{equation}
    x_{i,t+1} = \delta x_{i,t} + f(z_{i,t})  + \varepsilon_{i,t+1}\label{process}
\end{equation}
where $f(z_{i,t})$ can be interpreted as firm/type specific component, but determined by information at time $t$. And the forecast is
\begin{equation}
    F_t x_{i,t+1} = \hat{\delta} x_{i,t} + \hat{f}(z_{i,t}) + \theta\hat\delta\varepsilon_{i,t}
\end{equation}

The MSE of the forecast is then
\begin{align}
    E(x_{i,t+1} - F_t x_{i,t+1})^2
    & =E((\delta - \hat{\delta})x_{i,t} - ( f(z_{i,t}) - \hat{f}(z_{i,t})) - \theta \hat\delta\varepsilon_{i,t} + \varepsilon_{i,t+1})^2 \label{s12}\\
    & = (\delta - \hat{\delta})^2 E(x_{i,t})^2 + E( f(z_{i,t}) - \hat{f}(z_{i,t}))^2 + \theta^2\hat\delta^2 E(\varepsilon_{i,t})^2 + E(\varepsilon_{i,t+1})^2 \label{s22}.
\end{align}
Notice that to go from Equation \ref{s12} to Equation \ref{s22}, it is assumed that all four components in Equation \ref{s12} are uncorrelated with each other.

Without loss of generality assume that $\delta = \hat{\delta}$. Forecast Error is then
\begin{align}
    fe_{i,t+1} & \equiv x_{i,t+1} - E_t x_{i,t+1}  =  \delta x_{i,t} + f(z_{i,t})  + \varepsilon_{i,t+1} - (\hat{\delta} x_{i,t} + \hat{f}(z_{i,t})) - \theta\hat\delta \varepsilon_{i,t}  \\
    & = (\delta - \hat{\delta})x_{i,t} - ( f(z_{i,t}) - \hat{f}(z_{i,t})) - \theta\hat\delta \varepsilon_{i,t}\\
    & =  - \theta \hat\delta\varepsilon_{i,t} - ( f(z_{i,t}) - \hat{f}(z_{i,t}))\label{ferror}
\end{align}

Calculating the correlation between future forecast errors and current period information, we have
\begin{align}
    cov & (fe_{i,t+1},x_{i,t})  = \\
    & cov( - \theta \hat\delta\varepsilon_{i,t} - ( f(z_{i,t}) - \hat{f}(z_{i,t})) , \rho x_{i,t-1} + f(z_{i,t-1})  + \varepsilon_{i,t} ) \label{s2} \\
    & = - \theta\hat\delta var(\varepsilon_{i,t})\label{s3}.
\end{align}
To go from the Equation \ref{s2} to Equation \ref{s3}, we assume that $f(z_{i,t}-\hat f(z_{i,t})$ is uncorrelated with information prior to time $t$.

With the existence of firm-level heterogeneity, machine learning could improve predictive power by allowing more factors and more complex functional forms. However, omitting $f(z_{it})$ does not cause more overreaction.


%%%%%%%%%%%%%%%%%%%%%%%%%%%%%%%%%%%%%%%%%%%%%%%%%%%%
\section{Empirical Results}
\label{sec:MainResults}
In this section, we first analyze the predictability of machine learning models versus analysts' forecasts by comparing the mean squared errors of each forecasting method. Then, we examine whether the machine learning algorithm exhibits overreacting behavior. We find that machine learning forecasts experience similar overreacting behavior, but with a smaller magnitude. We explore potential explanations by comparing machine learning models with different forecasting variables and regularization parameters. Finally, we examine whether similar overreacting behavior also exists when we use linear regression models.



\begin{table}[tbp]
	\caption{Summary Statistics}
	\label{tab:sumstat}
	\begin{footnotesize}
        This table reports the summary statistics of key variables. The MSE for analysts forecasts is 1.312.
	\end{footnotesize}	
	\begin{center}
        \subcaption{\% ML Win}
        \begin{tabular}{llccc}
        \toprule
        & Forecasting Variables         & Learning rate &  MSE & Ratio   of MSE: Analysts vs ML \\
        \midrule
        (1) & Analysts Forecast &      -      & 1.312 &       -                 \\
        \midrule
        & Gradient Boosting & & &  \\
        \cmidrule(lr){2-2}
        (2) & All                           & 0.1           & 1.255 & 1.045                          \\
        (3) & Lag EPS                       & 0.1           & 1.892 & 0.693                          \\
        (4) & All minus Analysts Forecast   & 0.1           & 1.5   & 0.875                          \\
        (5) & All plus Financial Statements & 0.1           & 1.253 & 1.047                          \\
        (6) & All                           & 0.01          & 2.579 & 0.509                          \\
        (7) & All                           & 0.04          & 1.427 & 0.919                          \\
        (8) & All                           & 0.2           & 1.259 & 1.042                         \\
        \midrule
        & Linear Regressions & & &  \\
        \cmidrule(lr){2-2}
        (9) & All                           & -           & 1.856 & 0.707                          \\
        (10) & Lag EPS                       & -           & 1.881 & 0.698                          \\
        (11) & All minus Analysts Forecast   & -           & 2.611 & 0.502                          \\
        (12) & All plus Financial Statements & -           & 1.850 & 0.709                          \\
          \bottomrule
        \end{tabular}
        \end{center}

\end{table}

\begin{table}[h!]
	\ContinuedFloat
 \caption{Summary Statistics (continued)}
        \subcaption{All Analysts}
        \begin{center}
        \begin{tabular}{lcccccc}
        \toprule
                         & Mean  & Median & Std   Dev & 25     & 75   & N     \\
        \midrule
        EPS               & 1.130 & 0.890 & 1.916 & 0.050 & 1.95 & 54534 \\
        Analyst Forecast  & 1.424 & 1.139 & 1.722 & 0.413 & 2.12 & 54534 \\
        ML Forecast       & 1.155 & 0.887 & 1.509 & 0.203 & 1.82 & 54534 \\
        Analyst Forecast Error & -0.031 & -0.007 & 0.109 & -0.039 & 0.01 \\
        ML Forecast Error & 0.000 & 0.020 & 0.158 & -0.019 & 0.05 \\
        Investment        & 0.058 & 0.039 & 0.061 & 0.019  & 0.07 & 54534 \\
        Debt Net Issuance & 0.017 & 0.000 & 0.090 & -0.016 & 0.03 & 54534 \\
        \bottomrule
        \end{tabular}
        \end{center}
        \subcaption{Tech and Non-Tech Analysts}
        \begin{center}
        \begin{tabular}{lcccccc}
        \toprule
                         & Mean  & Median & Std   Dev & 25     & 75   & N     \\
        \midrule
        \multicolumn{7}{l}{Sample period: 1994-2018} \\
        EPS                     & 1.351 & 1.06  & 2.169 & 0.11  & 2.28 & 39419 \\
        Non-Tech   Analyst      & 1.678 & 1.341 & 1.996 & 0.483 & 2.5  & 36155 \\
        Tech   Analyst          & 1.67  & 1.198 & 2.442 & 0.253 & 2.65 & 14901 \\
        Non-Tech   Analyst  MSE & 0.837 & 0.027 & 2.98  & 0.003 & 0.2  & 34935 \\
        Tech   Analyst MSE      & 1.617 & 0.039 & 5.79  & 0.005 & 0.34 & 14585 \\
        Investment              & 0.055 & 0.036 & 0.061 & 0.017 & 0.07 & 39773 \\
        \midrule
        \multicolumn{7}{l}{Sample period: 1994-2012} \\
        EPS                     & 1.248 & 1.05  & 1.812 & 0.23  & 2.08 & 25320 \\
        Non-Tech   Analyst      & 1.561 & 1.311 & 1.613 & 0.583 & 2.26 & 22929 \\
        Tech   Analyst          & 1.479 & 1.1   & 1.888 & 0.385 & 2.25 & 7922  \\
        Non-Tech   Analyst  MSE & 0.748 & 0.027 & 2.776 & 0.003 & 0.19 & 22093 \\
        Tech   Analyst MSE      & 1.408 & 0.043 & 5.28  & 0.005 & 0.32 & 7786  \\
        Investment              & 0.06  & 0.039 & 0.064 & 0.02  & 0.07 & 25479 \\
        \midrule
        \multicolumn{7}{l}{Sample period: 2013-2018} \\
        EPS                     & 1.535 & 1.1   & 2.685 & -0.12 & 2.82 & 14099 \\
        Non-Tech   Analyst      & 1.881 & 1.433 & 2.513 & 0.248 & 3.07 & 13226 \\
        Tech   Analyst          & 1.886 & 1.395 & 2.933 & 0.02  & 3.2  & 6979  \\
        Non-Tech   Analyst  MSE & 0.989 & 0.027 & 3.297 & 0.003 & 0.21 & 12842 \\
        Tech   Analyst MSE      & 1.857 & 0.035 & 6.315 & 0.004 & 0.35 & 6799  \\
        Investment              & 0.048 & 0.03  & 0.056 & 0.013 & 0.06 & 14294 \\
        \bottomrule
        \end{tabular}
        \end{center}
\end{table}
\clearpage

\subsection{Prediction MSE}
This section provides an overall evidence on the predictability of firm earnings. Table \ref{tab:sumstat} presents evidence on the efficacy of gradient boosting methods, analyst forecasts, and linear regression models. The first column indicates the variables used as forecasting variables. The set of factors labeled `All' is our benchmark set of factors used for prediction, described in Section \ref{sec:Data}. We also use different sets of factors in the model. The second column represents the learning rate hyper-parameter we use to control for over-fitting in gradient boosting tree models; the default number is $0.1$. The third column represents the mean squared error of forecasts of firm annual EPS. The last column indicates the ratio of the MSE of the analyst's forecast and the forecast model used in that row. A ratio larger than one indicates that the model in the row performs better than the median consensus analyst forecast.


Row (1) of Table \ref{tab:sumstat} presents the prediction results for the median consensus analyst forecast, which serves as our benchmark for mean squared error. The mean squared error of the analyst's forecast is $1.312$.

Row (2) of Table \ref{tab:sumstat} presents the prediction results for gradient boosting using all factors. The model performs relatively well, with an $MSE$ of 1.255. The analyst's forecast has a 4.5\% higher squared forecast error. The results suggest that our machine learning model improves predictability, but not as strongly as in other settings (\citet{cao2021man}). The improvement is similar to the estimation results from \citet{van2022man}.

Row (3) again uses the gradient boosting regression tree model but reduces the set of variables to only lagged EPS. As might be expected, the model estimation performance is significantly reduced. The mean squared error becomes 1.892, and the performance is worse than that of the analyst's forecasts.

\citet{cao2021man} argue that human forecasts can complement machine learning forecasts. In the main set of forecasts, we already include the median consensus analyst forecast as a predictor. To examine the importance of the analyst's forecast, we remove it from the factors. Surprisingly, when we remove the analyst's forecast as a predictor in row (4), the mean squared errors drop to 1.5, which is smaller than the mean squared error of the analyst's forecast itself. The results suggest that when forecasting firm earnings, the analyst's private assessment is extremely valuable. The information generated by analysts cannot be replaced by incorporating a large set of public financial ratios.

In row (5), we extend the set of variables to include all firm balance sheet variables. The mean squared errors are very similar to row (1), suggesting that the financial variables we used in the main forecasting set have already captured the firm heterogeneity in explaining earnings.

Rows (6) to (8) capture the predictability when we use different learning rate parameters. When the learning rate is very small, the model does not fit the training sample enough, which leads to relatively poor out-of-sample performance. As we increase the learning rate, the model's performance improves. However, row (8) shows that if the learning rate is too large, the model runs the risk of over-fitting the historical information, leading to a drop in prediction performance.

\begin{figure}[hbt!]
	\caption{Ratio of MSE}
	\label{fig:MSE}
 This figure shows the ratio of mean squared errors of human forecasts and machine forecasts at different years. A ratio smaller than one suggests that humans beat the machines in that year.
	\begin{center}
	\begin{minipage}{0.8\textwidth}
		\centering
		\includegraphics[width=1.0\linewidth]{figures/Results_ML3Wins_All.png}
	\end{minipage}
	\end{center}
\end{figure}

We also examine the time-series variations in the relative strength between human analysts and machine learning models. Figure \ref{fig:MSE} shows the ratio of mean squared errors of human forecasts and machine forecasts at different years. A ratio smaller than one suggests that humans beat the machines in that year. From 1994 to 2018, we observed 10 years where humans beat the machines, including 1994, 2003, 2004, 2005, 2009, 2010, 2011, 2012, 2013, and 2017. One possible explanation for humans beating the machines is that the training samples experienced negative shocks and were, therefore, less useful for future earnings predictions. The annual SP500 return in 2000, 2001, and 2022 were -10.1\%, -13.0\%, and 23.4\%, respectively. The annual percentage change of the SP500 index in 2008 was -38.5\%, while in 2015 was -0.73\%. When we include those observations in our 5-year training sample, we find that humans beat the machines.

The results are similar if we use the last three years' observations to train the model, but on average we get a lower decrease in MSE if we use only three years' observations. We also tried to include more years during recessions, as \citet{cao2021man} suggested, and found that the results are similar, with still lower increases in average mean squared errors.


\subsection{Machines Do Overreact}
In this section, we follow the empirical strategy defined in Section \ref{sec:overreaction}. The key question is whether the ML generated forecasts exhibit similar overreaction behavior to that found for human reported forecasts.

\citet{bordalo2021real} run linear regressions in which the dependent variable
is the error at date $t+1$ of stock analyst predictions from IBES. The
independent variables are date $t$ values of the forecast, profit, investment,
or debt issuance. In each of these regressions firm and year fixed effects are
included. According to \citet{bordalo2021real} under rational expectations the
slope coefficients should be zero. Under diagnostic expectations the slope
coefficients should be negative. That is because of beliefs that overreact both
to good and to bad news. \citet{bordalo2021real} use linear regressions with
fixed effects to carry out this test. We follow \citet{bordalo2021real} in Table \ref{tab:over_main}. We also add a columns (1) (2) that uses the prediction error from analysts.
\begin{table}[tbp]
	\caption{Predictable Forecast Errors}
        \label{tab:over_main}
	\begin{footnotesize}
        This table reports OLS regression results of regressing forecast error at $t+1$ on the information at time $t$. Forecast error at $t+1$ is $\text{Forecast Error}_{t+1} \coloneqq p_{t+1} - F_t p_{t+1} $, where $F_t$ is forecast based on date $t$ data. In columns (1) and (2), forecast $F_t p_{t+1}$ is the average of median monthly earnings forecast of all analysts' forecasts conducted during fiscal year t. In columns (3) and (4), forecast $F_t p_{t+1}$ is the average of median monthly earnings forecast of all machine learning forecasts conducted during fiscal year t. Earnings are converted from EPS forecast to total earnings. Investment$_{t}$ capital expenditure at time $t$. Debt issuance$_{t}$ is net debt issuance at time $t$. All variables are scaled by total assets.
	\end{footnotesize}	
	\begin{center}
            \begin{tabular}{lcccc}
            \toprule
             & (1)              & (2)              & (3)              & (4)              \\
             & Forecast   Error & Forecast   Error & Forecast   Error & Forecast   Error \\
             & Analysts  &  Analysts   & Machine  & Machine  \\
             \midrule
            Investment& -0.018*   & -0.143*** & -0.016**  & -0.107*** \\
                        & (-1.845)  & (-10.065) & (-1.978)  & (-8.175)  \\
            \midrule
            Firm FE & No        & Yes       & No        & Yes       \\
            Year FE & Yes       & Yes       & Yes       & Yes       \\
            Period  & 1994-2018 & 1994-2018 & 1994-2018 & 1994-2018 \\
            N       & 54536     & 53324     & 54536     & 53324     \\
            AdjR2   & 0.02      & 0.32      & 0.03      & 0.23     \\
            \bottomrule
            \end{tabular}
	\end{center}
\end{table}


Columns (1) and (2) confirm the widely documented empirical findings that human forecasts exhibit overreaction behavior. In Column (2), the regression coefficient is $0.143$, indicating that a one standard deviation increase in investment $(0.061)$ leads to an 8\% higher disappointment in earnings next year $(0.061 * 0.143 / 0.109)$. Columns (3) and (4) perform the same regressions as Columns (1) and (2), but using machine learning predictions instead of human forecasts. The magnitude of the regressions is reduced compared to Columns (1) and (2), but the results in all columns (1) to (4) are negative and statistically significant.

To address concerns about downward dynamic panel bias due to the inclusion of firm fixed effects, we provide GMM estimations in Appendix Table \ref{tab:over_gmm}. The GMM results give us a larger regression coefficient of $0.472$. Both gradient boosting and analyst forecasts exhibit overreacting behavior even when we correct for downward dynamic panel bias. The magnitude is comparable to the coefficient of $0.887$ in \citet{bordalo2021real}. The difference in magnitude could result from different sample firms.

\subsection{Why Do Machine Overreact?}\label{sec:ml_parameters}
To examine why machine learning forecasts exhibit such "irrational" behavior, we will analyze the number of forecasting variables used and the model's regularization parameter.

To examine the magnitude of overreaction, we compare the differences in overreaction coefficients ($\beta$) using different forecast models. We use seemingly unrelated regressions to compare the regression coefficients estimated using forecasting errors from different forecast models. We report $\chi^2$ statistics and p-values of Wald tests calculated using the covariance estimated from the seemingly unrelated regressions. Since any contamination from measurement errors and fixed effects is the same when we use different forecast errors, we can directly compare the coefficients $\beta$. A lower $\beta$ indicates a lower overreaction coefficient ($\theta$).

\begin{table}[tbp]
	\caption{Predictable Forecast Errors Using Difference Forecasting Variables}
         \label{tab:over_variables}
        \begin{footnotesize}
         This table presents the results of an OLS regression analysis of the forecast error at time $t + 1$ and Wald test results from seemingly unrelated regressions. The forecast error at $t + 1$ is defined as $\text{Forecast Error}_{t+1} = p_{t+1} - F_t p_{t+1}$, where $F_t p_{t+1}$ is the average of median monthly earnings forecast based on data from time $t$, and $p_{t+1}$ represents the actual earnings at time $t + 1$. Forecasts are calculated from various machine learning models that use different forecasting variables. In column (1), lagged annual earnings of the firm are used as the forecasting variables. In column (2), the forecasting variables include firm characteristics and lagged annual and quarterly earnings. In column (3), the forecasting variables include firm characteristics, analysts forecasts, and lagged annual and quarterly earnings. In column (4), the variables include firm characteristics, analysts forecasts, lagged annual and quarterly earnings, and financial statement items. All earnings are converted to total earnings from EPS forecasts and scaled by total assets. Investment$_{t}$ is capital expenditure at time $t$. Debt issuance$_{t}$ is net debt issuance at time $t$. The $\chi^2$ of the Wald test is used to test the difference of regression coefficients from column (3), with the results of the test presented in the table. The p-value of the test is also included in brackets.
        \end{footnotesize}	
	\begin{center}
            \begin{tabular}{lcccc}
            \toprule
             & (1)              & (2)              & (3)              & (4)           \\
             & Forecast   Error & Forecast   Error & Forecast   Error & Forecast   Error \\
             & Analysts  &  Analysts   & Machine  & Machine  \\
             \midrule
             Investment & -0.109*** & -0.114*** & -0.107*** & -0.109*** \\
             & (-5.431)  & (-7.692)   & (-8.174)  & (-8.388)  \\
             \midrule
             $\chi^2$ &0.0495 	 &1.705	& -	&2.221 \\
                    &[0.824] 	&[0.192]  &-	&[0.136]\\
            Firm FE & Yes             & Yes       & Yes       & Yes       \\
            Year FE & Yes           & Yes       & Yes       & Yes       \\
            Period  & 1994-2018  & 1994-2018 & 1994-2018 & 1994-2018 \\
            N       & 53321         & 53321     & 53321     & 53321     \\
            AdjR2   & 0.43          & 0.23      & 0.23      & 0.22     \\
            \bottomrule
            \end{tabular}
	\end{center}
\end{table}


Table \ref{tab:over_variables} shows the results obtained using different information sets to make predictions. Consistent with our simple illustrative model in Section \ref{sec:framework}, ignoring firm heterogeneity could lead to higher squared forecast errors but does not affect overreaction behavior.

In order to further investigate the phenomenon of overreaction in machine learning, we next analyze the impact of machine learning hyper-parameters on the behavior of machine learning forecasts. To fix ideas consider the following representation of the gradient boosting regression tree model,
\begin{equation}
	\hat{y} = F_M(x) = \sum_{m = 1}^{M} h_m(x)
\end{equation}
where $\hat{y}$ is the fitted value, $h_m$ is decision tree regressor with depth of $d$, and $M$ is the
number of trees in the forest. $F_M(x)$ is solved by using a greedy algorithm
framework,
\begin{equation}
	F_k(x) = F_{k-1}(x) + \gamma h_k
\end{equation}
where $\gamma$ is the learning rate. The learning rate shrinks the contribution
of each additional tree. $h_k$ is the newly added tree solved by minimize a loss
function $L$ given $F_{k-1}(x)$
\begin{equation}
	h_k = \arg \min_h \sum_{i }(L(y_i,F_{k-1} (x_i) + \gamma h(x_i) )
\end{equation}
There are three main hyper-parameters in gradient boosting: the depth of the
tree $d$ (max$\_$depth), the number of trees in the forest $M$ (n$\_$%
estimators), and learning rate $\gamma$ (learning$\_$rate).

The default parameters we start with have the following values: depth of the tree is $2$, number of
trees in the forest is $50$, and learning rate is $0.1$. We have
systematically analyzed the effect of using hyper-parameters optimized using cross-validation. The hyper-parameters were chosen to optimize the out-of-sample mean squared errors before 1994.  Except where noted, the Tables use the default hyper-parameters.

Over-reaction is fundamentally about the response to a shock. The responsiveness of the algorithm is primarily controlled by the learning rate. So we are interested in examining the impact of altering the learning rate away from a default or an optimized value. It should be noted that we find that other hyper-parameters do not have much of an affect on the estimated overreaction by the algorithm.

What does the learning rate do? It is a regularization strategy that shrinks the contribution of each additional tree. Larger learning rate leads to a higher probability of over-fitting given the training sample. When we set a high learning rate, the algorithm can capture the deep structure in the training sample which may be just transitory. Our interest is in the relationship between overreacting behavior of the algorithm predictions, and over-fitting parameters in the prediction model.

Table \ref{tab:over_lr} shows the results obtained using different hyper-parameters to make predictions. We find that a lower learning rate generates predictions that overreact less. This should make intuitive sense. Consider an extreme case. Suppose the algorithm entirely ignored all shocks. Then there would be no response at all. If a prediction does not change, it certainly is not going to change excessively.

\begin{table}[tbp]
	\caption{Predictable Forecast Errors Using Hyper-parameters}
        \label{tab:over_lr}
	\begin{footnotesize}
         This table presents the results of an OLS regression analysis of the forecast error at time $t + 1$ and Wald test results from seemingly unrelated regressions. The forecast error at $t + 1$ is defined as $\text{Forecast Error}_{t+1} = p_{t+1} - F_t p_{t+1}$, where $F_t p_{t+1}$ is the average of median monthly earnings forecast based on data from time $t$, and $p_{t+1}$ represents the actual earnings at time $t + 1$. Forecasts are calculated from various machine learning models that use different hyper-parameters. The default hyper-parameters are set as the following values: depth of the tree is $2$ and number of
        trees in the forest is $50$. Table reports the results using different learning rates. Investment$_{t}$ is capital expenditure at time $t$. Debt issuance$_{t}$ is net debt issuance at time $t$. The $\chi^2$ of the Wald test is used to test the difference of regression coefficients from column (3), with the results of the test presented in the table. The p-value of the test is also included in brackets.
        \end{footnotesize}	
	\begin{center}
        \begin{tabular}{lcccc}
            \toprule
             & (1)              & (2)              & (3)              & (4)              \\
             & Forecast   Error & Forecast   Error & Forecast   Error & Forecast   Error \\
             & Analysts  &  Analysts   & Machine  & Machine  \\
             \cmidrule(lr){2-5}
             Learning rate & 0.01 & 0.03  & 0.1 & 0.2 \\
             \midrule
            Investment& 0.027     & -0.075*** & -0.107*** & -0.110*** \\
                    & (0.781)   & (-4.099)    & (-8.174)  & (-7.718)  \\
            \midrule
            $\chi^2$      & 39.730	&10.464	 & -	&0.576  \\
            & [0.000]	&[0.001]	 &-	& [0.448]  \\
            Firm FE & Yes       & Yes            & Yes       & Yes       \\
            Year FE & Yes       & Yes             & Yes       & Yes       \\
            Period  & 1994-2018 & 1994-2018  & 1994-2018 & 1994-2018 \\
            N       & 53321     & 53321        & 53321     & 53321     \\
            AdjR2   & 0.76      & 0.66          & 0.23      & 0.20     \\
            \bottomrule
            \end{tabular}
	\end{center}
\end{table}



\subsection{Overreaction to Aggregate or Firm-Specific Shocks?}
The results from \citet{bianchi2022belief} suggest that macro-economic shocks might affect things differently from firm-specific shocks. So we follow \citet{chen2007price} and decompose machine forecast errors into a market-related variation and a firm-specific variation. We do not examine industry-specific variation because our machine learning models are not trained using industry-specific variables. Consider the following regression:

\begin{equation}
    e_{i,t} = \beta_i + \beta_{i,m} r_{m,t} + \varepsilon_{i,t}
\end{equation}
where $e_{i,t}$ is the forecast error of firm $i$ at year $t$, $r_{m,t}$ is the market forecast error at year $t$, which is the equal-weighted average forecast error at year $t$. We define market-related forecast error as $\hat{\beta}_{i,m} r_{m,t}$ and the firm-specific forecast error as $e_{i,t} - \hat{\beta}{i,m} r{m,t}$. In Table \ref{tab:over_decompose}, we regress market-related error and firm-specific error on investment separately.

\begin{table}[tbp]
	\caption{Decomposing Forecast Errors }
        \label{tab:over_decompose}
	\begin{footnotesize}
         This table presents the results of an OLS regression analysis of the forecast error at time $t + 1$.
         We decompose forecast errors into a market-related variation and a firm-specific variation, using the following regression $e_{i,t} = \beta_i + \beta_{i,m} r_{m,t} + \varepsilon_{i,t}$. $e_{i,t}$ is the forecast error of firm $i$ at year $t$, $r_{m,t}$ is the market forecast error at year $t$, which is the equal-weighted average forecast error at year $t$. We define market-related forecast error as $\hat{\beta}_{i,m} r_{m,t}$ and the firm-specific forecast error as $e_{i,t} - \hat{\beta}_{i,m} r_{m,t}$.
         The dependent variable in column (1) is the market-related forecasting errors of human analysts, while column (2) is the firm-specific forecasting errors of human analysts. The dependent variable in column (3) is the market-related forecasting errors of machine analysts, while column (4) is the firm-specific forecasting errors of machine analysts. Investment$_{t}$ is capital expenditure at time $t$.
        \end{footnotesize}	
	\begin{center}
        \begin{tabular}{lcccc}
            \toprule
             & (1)              & (2)              & (3)              & (4)              \\
             & Forecast   Error & Forecast   Error & Forecast   Error & Forecast   Error \\
             & Analysts  &  Analysts   & Machine  & Machine  \\
             & Market-Related & Firm-Specific & Market-Related & Firm-Specific \\
             \midrule
            Investment& -0.040***     & -0.103***      & -0.013**            & -0.093***            \\
             & (-4.919)      & (-9.349)       & (-2.023)            & (-8.178)             \\
            Firm FE & Yes           & Yes            & Yes                 & Yes                  \\
            Year FE & Yes           & Yes            & Yes                 & Yes                  \\
            Period  & 1994-2018     & 1994-2018      & 1994-2018           & 1994-2018            \\
            N       & 53321         & 53321          & 53321               & 53321                \\
            AdjR2   & 1.00          & 0.99           & 0.96                & 0.90                \\
            \bottomrule
            \end{tabular}
	\end{center}
\end{table}


Columns (1) and (2) show the results for analysts' forecasts. The dependent variable in column (1) is the market-related forecasting errors of human analysts, while column (2) is the firm-specific forecasting errors of human analysts. An increase in investment rate leads to disappointment in earnings forecasts by human analysts next year. In columns (3) and (4), we decompose the forecast errors using the gradient boosting model. The dependent variable in column (3) is the market-related forecasting errors of machine analysts, while column (4) is the firm-specific forecasting errors of machine analysts. The overreaction coefficient in column (3) is $-0.013$, while the overreaction coefficient in column (1) is $-0.040$, suggesting that applying machine learning largely reduces the market-related overreaction. The regression coefficients have similar magnitudes in columns (2) and (4).

\subsection{Linear Regression Models Also Overreact}
Our analysis is focused on gradient boosting predictions. That is motivated by the wide use and documented success of the algorithm. ``Gradient Tree Boosting or Gradient Boosted Decision Trees (GBDT) is a generalization of boosting to arbitrary differentiable loss functions, see the seminal work of \citet{friedman2001greedy}. GBDT is an accurate and effective off-the-shelf procedure that can be used for both regression and classification problems in a variety of areas...'' \url{https://scikit-learn.org/stable/modules/ensemble.html#gradient-boosting}

It is natural to wonder if our overreaction results are specific to that particular algorithm. They are not. To show this we consider predictions from perhaps the most familiar methods used in finance. In this section, we use linear regression models to forecast EPS. The forecasting procedures strictly follow the procedure used in gradient boosting forecasts. The only difference here is that the function $f_{y}$ used to fit the training sample is a linear regression model. We use the same forecasting variables $\Phi_{i,t}$ as in the gradient boosting model.

\begin{table}[tbp]
	\caption{Predictable Forecast Errors in Linear Models }
        \label{tab:over_ols}
	\begin{footnotesize}
         This table presents the results of an OLS regression analysis of the forecast error at time $t + 1$ and Wald test results from seemingly unrelated regressions. The forecast error at $t + 1$ is defined as $\text{Forecast Error}_{t+1} = p_{t+1} - F_t p_{t+1}$, where $F_t p_{t+1}$ is the average of monthly earnings forecast based on data from time $t$, and $p_{t+1}$ represents the actual earnings at time $t + 1$. Forecasts are calculated from linear regression models that use different forecasting variables. In column (1), lagged annual earnings of the firm are used as the forecasting variables. In column (2), the forecasting variables include firm characteristics and lagged annual and quarterly earnings. In column (3), the forecasting variables include firm characteristics, analysts forecasts, and lagged annual and quarterly earnings. In column (4), the variables include firm characteristics, analysts forecasts, lagged annual and quarterly earnings, and financial statement items. All earnings are converted to total earnings from EPS forecasts and scaled by total assets. Investment$_{t}$ is capital expenditure at time $t$. Debt issuance$_{t}$ is net debt issuance at time $t$. The $\chi^2$ of the Wald test is used to test the difference of regression coefficients from column (3), with the results of the test presented in the table. The p-value of the test is also included in brackets.
        \end{footnotesize}	
	\begin{center}
            \begin{tabular}{lcccc}
            \toprule
             & (1)              & (2)              & (3)              & (4)           \\
             & Forecast   Error & Forecast   Error & Forecast   Error & Forecast   Error \\
             & Analysts  &  Analysts   & Machine  & Machine  \\
             \midrule
             Investment & -0.138*** & -0.113*** & -0.128*** & -0.133*** \\
             & (-8.196)  & (-5.336)   & (-6.971)  & (-7.871)  \\
             \midrule
             $\chi^2$ &10.696 	 &0.368	& 3.537	&6.649 \\
                    &[0.001] 	&[0.544]  & [0.060]	&[0.010]\\
            Firm FE & Yes             & Yes       & Yes       & Yes       \\
            Year FE & Yes           & Yes       & Yes       & Yes       \\
            Period  & 1994-2018  & 1994-2018 & 1994-2018 & 1994-2018 \\
            N       & 53321         & 53321     & 53321     & 53321     \\
            AdjR2   & 0.22          & 0.44      & 0.22      & 0.23     \\
            \bottomrule
            \end{tabular}
	\end{center}
\end{table}



Table \ref{tab:over_ols} presents the forecasts calculated from linear regression models that use different forecasting variables. In column (1), lagged annual earnings of the firm were used as the forecasting variables. In column (2), the forecasting variables included firm characteristics and lagged annual and quarterly earnings. In column (3), the forecasting variables included firm characteristics, analysts' forecasts, and lagged annual and quarterly earnings. In column (4), the variables included firm characteristics, analysts' forecasts, lagged annual and quarterly earnings, and financial statement items.

Unlike the gradient boosting regression tree model, there is no hyper-parameter governing the level of regularization in OLS. The number of forecasting variables represents the information set used in the model, as well as the potential severity of over-fitting. Table \ref{tab:over_ols} shows that when too many forecasting variables are included in a linear model, the forecast variable has an overreacting coefficient quite similar to that of human analysts. We

In Table \ref{tab:largelr}, we also tested whether the machine learning model could generate overreaction similar to human analysts by increasing the learning rate parameter. Unlike linear models, we found that the machine learning forecast's overreaction is capped from above. The difference between linear models and the gradient boosting regression tree model is that the gradient boosting model embeds the idea of regularization into the optimization algorithm, making it less likely to create overfitting forecasts that exhibit overreaction similar to linear models.

%%%%%%%%%%%%%%%%%%%%%%%%%%%%%%%%%%%%%%%%%%%%%%%%%%%%
\section{Stock Analysts}
\label{sec:analysts}

So far, we have found that in a simulation setting, machine learning predictions overreact regardless of the information set used to make the prediction. However, the magnitude of overreaction is smaller for machine learning predictions compared to analysts' forecasts.

In this section, we use real analysts' forecasts to further analyze the impact of adopting machine learning models on the overreacting behavior of forecasts. To do so, we need to measure the probability that analysts adopt machine learning models when making predictions. Given our goal of measuring overreaction, an analyst-level measure is more granular and provides more meaningful inferences compared to a brokerage-level machine learning adoption measure.

We use analysts' technical backgrounds as a proxy for the probability of analysts adopting machine learning models. The underlying assumption is that analysts with advanced statistical training are more likely to adopt machine learning algorithms when making EPS forecasts, once they become available. We compare technical analysts' forecasts with those of non-technical analysts before and after the widespread availability of machine learning. This empirical strategy is similar to a difference-in-differences analysis. However, since technical and non-technical analysts have different stock coverage, we cannot directly implement a difference-in-differences framework. Instead, we compare the magnitude of the overreaction coefficients, represented by $\beta$, before and after the widespread availability of machine learning algorithm.

Although the gradient boosting regression tree model becomes popular after the publication of \citet{friedman2002stochastic}, it did not become widely accessible until the implementation of several open-source machine learning packages, including scikit-learn and tensorflow. Therefore, we use the year 2012, when scikit-learn became ``well-maintained and popular'' (\url{https://en.wikipedia.org/wiki/Scikit-learn}), as the threshold for pre- and post- periods.

\subsection{Technical Analysts}
We manually collect the technical backgrounds of analysts using their LinkedIn information. Our goal is to measure analysts' statistical tools, and therefore their ability and probability of incorporating machine learning in their EPS forecasts. Here the focus is on the analysts and their backgrounds.
\begin{figure}[hbt!]
	\caption{LinkedIn page}
	\label{fig:LinkedIn}
 %%https://www.linkedin.com/in/mohan-naidu/
	This figure gives an example of a technical analysts' LinkedIn page.
	\begin{center}
	\begin{minipage}{0.8\textwidth}
		\centering
		\includegraphics[width=1.0\linewidth]{figures/Fig_IN.png}
	\end{minipage}
	\end{center}
\end{figure}

To identify analysts' identity and educational background information, we use analysts who gave at least one earnings recommendation in 2018. We cross-check their affiliations on FINRA's brokercheck website and LinkedIn to maximize the likelihood of an accurate manual matching. In total, we successfully linked 858 analysts to their LinkedIn profiles.

To identify analysts' technical backgrounds start with self-reported technical skills, including machine learning, artificial intelligence, and advanced statistics. It should be noted that we do not use `quantitative modelling' as a keyword since it might refer to a much wider range to prediction methods. Figure \ref{fig:LinkedIn} gives an example of a technical analysts' LinkedIn page.

We use their higher education background to identify whether analysts have technical skills or not. We manually check the curriculum for each major and identify the majors in Table \ref{tab:majors} as technical majors. These majors are mostly STEM majors, but there are some discrepancies. For example, we do not find advanced statistical courses in the biology major's curriculum. So analysts with a general biology education are not generally included. However, if an analyst earns a degree in Computational Biology, we find that advanced statistical tools are required. Therefore, we label these analysts as technical analysts.\footnote{Our focus is different from \citet{babina2021artificial}, which focuses on AI-related skills instead of statistical skills. We aim to measure analysts' ability to understand and use advanced statistics and the probability of using them. Therefore, we do not only focus on AI-related skills.}

In total, we identified 173 analysts with a technical education background. We then distinguish our IBES sample to analysts whom we categorized as technical vs. non-technical. For each firm in each month, we calculated the median consensus forecast by technical analysts $F_t^{T} x_{it+1}$ and non-technical analysts $F_t^{NT} x_{it+1}$ separately.

To conduct the overreaction test, in each fiscal year $t+1$, we took the average of the 12 median consensus forecasts made during fiscal year $t+1$. Starting in 1994 and going to 2018 we have 14,901 firm-year observations for forecasts made by technical analysts and 36155 firm-year observations for forecasts made by non-technical analysts. It should be kept in mind that technical analysts and non-technical analysts may cover different stocks, and this may affect certain comparisons.


\subsection{Technical VS Non-Tech Analysts}
We first examine whether forecast made by tech analysts and non-tech analysts exhibit different overreacting behavioral after machine learning adoption in 2013. We do so by separately calculating forecasts made by tech analysts and non-tech analysts. We then conduct similar overreacting tests in the previous section.

\begin{table}[tbp]
\caption{Tech Analysts vs Non-Tech Analysts Background before and after 2013}
\label{tab:techinv}
\begin{footnotesize}
    This table report OLS regression results of regressing forecast error at $t+1$ on the information at time $t$. Forecast error at $t+1$ is $\text{Forecast Error}_{t+1} \coloneqq p_{t+1} - F_t p_{t+1} $, where $F_t$ is forecast based on date $t$ data. In columns (1) and (3), forecast $F_t p_{t+1}$ is the average of median monthly earnings forecast of tech analysts conducted during fiscal year t. In columns (2) and (4), forecast $F_t p_{t+1}$ is the average of median monthly earnings forecast of non-tech analysts conducted during fiscal year t. Earnings are converted from EPS forecast to total earnings. Investment$_{t}$ capital expenditure at time $t$.  All variables are scaled by total assets. Sample periods are reported.
\end{footnotesize}
    \begin{center}
    \subcaption{Full Sample}\label{tab:techinv_full}
    \begin{tabular}{lcccc}
    \toprule
     & (1)              & (2)              & (3)              & (4)              \\
     & Forecast   Error & Forecast   Error & Forecast   Error & Forecast   Error \\
     & Tech  &  Non-Tech   & Tech  & Non-Tech  \\
     \midrule
    Investment & -0.181*** & -0.127*** & -0.109    & -0.091**  \\
       & (-2.731)  & (-6.913)  & (-1.424)  & (-2.865)  \\
    \midrule
    Firm FE    & Yes       & Yes       & Yes       & Yes       \\
    Year FE    & Yes       & Yes       & Yes       & Yes       \\
    Period     & 1994-2012 & 1994-2012 & 2013-2018 & 2013-2018 \\
    N          & 7367      & 21309     & 6359      & 12316     \\
    AdjR2      & 0.25      & 0.28      & 0.25      & 0.31     \\
    \bottomrule
    \end{tabular}
    \end{center}
    \begin{center}
    \subcaption{Balanced Sample}\label{tab:techinv_bal}
    \begin{tabular}{lcccc}
    \toprule
     & (1)              & (2)              & (3)              & (4)              \\
     & Forecast   Error & Forecast   Error & Forecast   Error & Forecast   Error \\
     & Tech  &  Non-Tech   & Tech  & Non-Tech  \\
     \midrule
    Investment & -0.160**  & -0.119**  & -0.147**  & -0.113**  \\
            & (-2.395)  & (-1.986)  & (-2.254)  & (-2.069)  \\
    \midrule
    Firm FE & Yes       & Yes       & Yes       & Yes       \\
    Year FE & Yes       & Yes       & Yes       & Yes       \\
    Period  & 1994-2012 & 1994-2012 & 2013-2018 & 2013-2018 \\
    N       & 4902      & 4902      & 5100      & 5100      \\
    AdjR2   & 0.25      & 0.26      & 0.29      & 0.32     \\
    \bottomrule
    \end{tabular}
    \end{center}
\end{table}
%\clearpage


Table \ref{tab:techinv_full} reports OLS regression results of regressing forecast error at $t+1$ on the information at time $t$ using full sample. In columns (1) and (3), the forecast $E_t p_{t+1}$ is the average earnings forecast of tech analysts' forecasts conducted during fiscal year t. In columns (2) and (4), the forecast $E_t p_{t+1}$ is the average earnings forecast of non-tech analysts' forecasts conducted during fiscal year t. Column (1) shows that before 2013, a one percentage point increase in investment led to a 0.181 percentage point stronger disappointment in earnings next year for tech analysts' forecasts. The negative relationship is statistically significant. Column (2) shows that after 2013, a one percentage point increase in investment led to a 0.09 percentage point stronger disappointment in earnings next year for tech analysts' forecasts. For tech analysts, the decrease in overreacting coefficients after 2013 is $0.398 = 1 - 0.109/0.181$. Interestingly, the overreaction coefficient becomes insignificant after 2013. The coefficient in column (3) is still statistically significant if we do not cluster at the firm level.

When we examine non-tech analysts's the change in overreacting coefficient, the trend is similar. The overreacting pattern becomes weaker after 2013, perhaps due to the relatively recent trends of relying on machine learning models to make predictions. However, the decrease in overreaction is smaller. For non-tech analysts, the decrease in overreacting coefficients after 2013 is $0.252 = 1 - 0.095/0.127$.

Technical analysts and non-technical analysts may have different sample coverage, and the sample coverage may vary before and after 2013. To address this issue, we first restricted our sample to firms that were covered by both technical and non-technical analysts in the same year. However, as few firms were covered by the same analyst both before and after 2013, we calculated the Mahalanobis distance for firms before and after 2013 based on firm age, total assets, high tech industry, and investment rate. We dropped firms before 2013 that were significantly different from firms in 2013. We then ran similar regressions using our balanced sample, and the results are provided in Table \ref{tab:techinv_bal}. We found similar results: technical analysts, on average, had a much larger decrease in overreaction after 2013, while overreaction for non-technical analysts remained at similar magnitudes.

Overall, the results suggest that tech analysts become less over-reactive after adopting machine learning models after 2013. The results confirm our prediction that adopting machine learning models could reduce overreacting bias.

To examine whether the difference in overreacting can have a real impact, we test whether less overreaction by tech analysts could lead to lower investment reversal. The empirical framework we use to test investment reversal from overreacting is from \citet{bordalo2021real}. If EPS forecasts used by managers overreact to good news, the investment decision based on the overreacting EPS would be larger. Once earnings are realized, the manager would correct the over-investing by under-investing in the next period. Therefore, the higher the overreacting, the higher the next period investment reversal will be. Since realized forecasting errors capture error terms that are not directly related to the overreacting behavior, we use an instrumental variable approach, as used in \citet{bordalo2021real}. The difference between our setting and \citet{bordalo2021real} is that we use analysts' forecasts, whereas \citet{bordalo2021real} use managerial forecasts. We show that analysts' forecasts can indeed affect firms' investment decisions in Table \ref{tab:analystsinvestment}, even after controlling for realized profits.

We regress forecast error on investment rate, as shown in Table \ref{tab:techinv}, and use the model to calculate predicted forecast errors. We then regress the change in future investment rates to examine whether the predicted forecast errors lead to investment reversals. Because one stock may have both tech and non-tech analysts' forecasts, to address the problem that managers may receive both forecasts from tech and non-tech analysts, we restrict our sample to stocks covered only by one type of analysts. All regressions are done separately for non-tech analysts' forecasts and tech analysts' forecasts. The results are shown in Table \ref{tab:invrev}.
\begin{table}[tbp]
\caption{Investment Reversals}
\label{tab:invrev}
\begin{footnotesize}
    This table reports OLS regression results of regressing changes in investments at $t+1$ and predicted forecast errors. predicted forecast errors. Predicted Forecast Errors (Tech) are predicted value using regression coefficients in Table \ref{tab:techinv} for tech analysts. Predicted Forecast Errors (Non-Tech) are predicted values using regression coefficients in Table \ref{tab:techinv} for non-tech analysts.
\end{footnotesize}	
    \begin{center}
    \begin{tabular}{lcccc}
    \toprule
     & (1)              & (2)              & (3)              & (4)              \\
     & $\Delta$ Inv$_{t+1}$ & $\Delta$ Inv$_{t+1}$ & $\Delta$ Inv$_{t+1}$ & $\Delta$ Inv$_{t+1}$ \\
     \midrule
    Predicted Forecast  & 6.541***  &           & 4.533*    &           \\
    Errors (Tech)     & (5.114)  &           & (2.840)   &           \\
    Predicted Forecast  &           & 6.134***  &           & 5.333***  \\
    Errors (Non-Tech)  &           & (4.869)   &           & (5.422)   \\
    \midrule
    Firm FE & Yes       & Yes       & Yes       & Yes       \\
    Year FE & Yes       & Yes       & Yes       & Yes       \\
    Period  & 1994-2012 & 1994-2012 & 2013-2018 & 2013-2018 \\
    N       & 54        & 163       & 186       & 349       \\
    AdjR2   & 0.26      & 0.40      & 0.07     & 0.11     \\
        \bottomrule
    \end{tabular}
    \end{center}
\end{table}

The results are consistent with our findings of overreaction in Table \ref{tab:techinv}. In column (1), a one-percentage-point increase in predicted forecast errors by tech analysts leads to a 6.543-percentage-point increase in investment rates from 1994 to 2012. After 2013, column (3) shows that a one-percentage-point increase in predicted forecast errors by tech analysts leads to a 5.333-percentage-point increase in investment rates, which is lower than the investment reversal before 2013. These results are consistent with the findings in Table \ref{tab:techinv} that analysts' forecasts by tech analysts decrease after 2013. Similarly, we find that investment reversals caused by non-tech analysts' overreacting forecasts are smaller after 2013.

For tech analysts, the decrease in investment reversals after 2013 is $0.307 = 1- 4.533/6.541$. For non-tech analysts, the decrease in investment reversals after 2013 is $0.131 = 1- 5.333/6.134$. Firms covered by non-tech analysts experience a larger decrease in investment reversals after 2013, the result is consistent with our findings that tech analysts' forecasts have a larger decrease in overreacting after 2013 compared to non-tech analysts.

Overall, the results suggest that the decrease in overreacting by adopting machine learning algorithm may have real implications, firms covered by those analysts experience lower magnitude of investment reversal behavioral, potentially lead to altered investments. However, such overreaction cannot be totally eliminated given the prevalence of overreacting behavioral in machine learning algorithms.



\begin{table}[tbp]
    \caption{Tech Analyst Coverage}
    \label{tab:techcoverage}
    \begin{footnotesize}
        This table estimates average characteristics of the stocks that tech analysts at analyst-year level. Tech is a indicator variables equalling to 1 if the analyst is tech analyst, 0 otherwise. After 2013 is an indicator variable equaling to 1 if the forecast is made after 2013. Number of Stocks is the total number of stocks covered by the analyst. Age is the average age of the stocks covered by the analyst. Total assets is the average total assets of stocks covered by the analyst. High Tech is the percentage of high-tech stocks covered by the analyst. High-Tech stocks are firms with three digit SIC code of 283, 357, 366, 367, 382, 384, or 737.
    \end{footnotesize}	
    \subcaption{Coverage before and after 2013}
    \begin{center}
    \begin{tabular}{lcccc}
    \toprule
     & (1)              & (2)              & (3)              & (4)              \\
     & Num of Stocks & Age       & Total   Assets & High Tech \\
    \midrule
    Tech*After 2013 & 0.859***          & -0.215    & -0.018         & -0.002      \\
                    & (3.021)           & (-0.860)  & (-0.548)       & (-0.382)    \\
    \midrule
    Analyst FE      & Yes               & Yes       & Yes            & Yes         \\
    Year FE         & Yes               & Yes       & Yes            & Yes         \\
    Period          & 1994-2018         & 1994-2018 & 1994-2018      & 1994-2018   \\
    N               & 11703             & 11703     & 11703          & 11703       \\
    AdjR2           & 0.58              & 0.77      & 0.79           & 0.92       \\
    \bottomrule
    \end{tabular}
    \end{center}
    \subcaption{Tech Analysts Coverage}
    \begin{center}
    \begin{tabular}{lcccc}
    \toprule
     & (1)     & (2)     & (3)    & (4)     \\
     & Num of Stocks & Age    & Total Assets & High Tech \\
    \midrule
    Tech       & -0.008    & -1.687*** & -0.194*** & 0.220***  \\
           & (-0.043)  & (-7.362)  & (-6.301)  & (22.969)  \\
    \midrule
    Analyst FE & No        & No        & No        & No        \\
    Year FE    & Yes       & Yes       & Yes       & Yes       \\
    Period     & 1994-2018 & 1994-2018 & 1994-2018 & 1994-2018 \\
    N          & 11716     & 11716     & 11716     & 11716     \\
    AdjR2      & 0.06      & 0.06      & 0.05      & 0.04     \\
    \bottomrule
    \end{tabular}
    \end{center}
\end{table}

To rule out the possibility that our results are driven by a change in tech and non-tech analysts' coverage after 2013, we compare the average firm characteristics covered by tech and non-tech analysts before and after 2013, as shown in Table \ref{tab:techcoverage}. We find that tech analysts increased their number of stock coverage  after 2013, while all other firm characteristics, including age, total assets, and tech industry, remained unchanged. The results suggest that our findings are unlikely to be driven by a change in tech analysts' stock coverage.

%%%%%%%%%%%%%%%%%%%%%%%%%%%%%%%%%%%%%%%%%%%%%%%%%%%%
\section{Model}
\label{sec:model}
Both our simulation exercise and tech analysts results suggest that adopting machine learning could lead to smaller, but still non-negligible, improvements in forecasting accuracy. We use a model to study the impact of machine learning adoption on making forecasts. This is especially important because we want to understand what happens during the transition to a world of complex machine learning models.
\subsection{Setup}
Based on \cite{begenau2018big}, we build up a repeated static model. At the beginning of each period, a firm chooses its size, i.e., the number of shares to issue in order to maximize its net revenue. The firm uses the raised capital to finance its 1-period investment opportunity that has a stochastic payoff affected by both idiosyncratic and aggregate shocks. Then investors make portfolio choice decisions to maximize their expected utilities. In particular, the investors do not know the idiosyncratic profitability of the firm's investment. They will randomly receive private signals from two types of analysts, i.e., tech and non-tech analysts. With these signals, the investors will update their prior beliefs about idiosyncratic profitability and form posterior beliefs. Investors' expected utilities are based on their posterior beliefs. At the end of each period, the payoff and utilities are realized. In the next period, new investors arrive and the same sequence repeats.

\textbf{Firm} There is one firm that chooses how many shares $\bar x_t$ to issue in order to raise funds for investment. The capital raised from each share is invested in the firm's investment project whose payoff is $A_tz_t$. $A_t$ is the aggregate profitability, while $z_t$ is the idiosyncratic profitability. We introduce $A_t$ just to model the relationship between the aggregate economic state and analysts' forecast error. Otherwise, $A_t$ functions exactly the same as $z_t$ since in the current setting, we only have one firm in the economy. Because we are interested in the analysts' forecasts on the \textit{firm-level} profitability, we assume that $A_t$ is public information and is realized at the very beginning of each period. By contrast, the firm does not know the realized value of $z_t$ when it makes decisions. However, the firm owner has a prior belief about the distribution of $z_t$. Specifically, the prior is that $z_t\sim N(\mu_{zt},\rho_z^{-1})$ with $\mu_{zt} = E_{t-1}[z_t] = \delta z_{t-1} + f_{t-1}$  as in Equation \ref{process}. The $\delta z_{t-1}$ part is the AR(1) component of the $z_t$ process, and $f_{t-1}$ term models the relationship between $z_t$ and all other information known at time $t-1$. So the information set of the firm owner when she makes equity issuance decisions is $\mathcal{F}_{t}^{-} = \{\mathcal{I}_{t-1}, A_t\}$, where $\mathcal{I}_{t-1}$ contains all the information realized prior to time $t$. Conditional on this information set, the firm owner maximizes the current period net revenue from the sale of the firm. Therefore, she solves the following problem
\begin{align}
	& \max_{\bar x_t} E[\bar x_t p_t - \phi(\bar x_t,\bar x_{t-1})|\mathcal{F}_{t}^{-}], \label{firm}
\end{align}
where $\phi$ is equity issuance cost. The equity issuance cost includes both a fixed cost and a variable cost in quadratic form
\[\phi(x_t,x_{t-1}) = \phi_0 1_{(|\Delta\bar x_t|>0)} + \phi_1|\Delta\bar x_t| + \frac{\phi_2}{2}(\Delta\bar x_t)^2,\]
where $\Delta\bar x_t = \bar x_t-\bar x_{t-1}$.
%Different from a dynamic model, the equity issuance cost does not have the real-option effect, and its effect is to limit firm's growth. If we do not want to examine firm size and its link with $\lambda$, we do not necessarily need this. I choose to include this adjustment cost 1) follow what is done; 2) see what is the implication with adjustment cost; 3) in case we want it in the end.

Given the firm is the only supplier of its stock, its equity issuance affects the price of the stock $p_t$. Thus, the firm makes decisions knowing that its issuance affects the price of each share.

\textbf{Analysts}
There are two types of analysts with total measure of 1. $\lambda\in[0,1]$ of them are tech analysts who apply algorithms to make forecasts, $1-\lambda$ are non-tech analysts.

\textbf{Investors} There is a continuum of investors endowed with wealth $W_t$ and decide how to allocate their wealth between riskless and risky assets, taking prices as given. When making decisions, investors do not know the firm's idiosyncratic profitability $z_t$, but they have a common prior that $z_t\sim N(\mu_{zt},\rho_z^{-1})$ with $\mu_{zt} = E_{t-1}[z_t] = \delta z_{t-1} + f_{t-1}$  as in Equation \ref{process}. So investors have the same prior as the firm owner. The aggregate state $A_t$ is public information and follows an AR(1) process $log A_t = \delta_AlogA_{t-1} + \epsilon_{At}$, with $\epsilon_{At}\sim N(0,\sigma_A^2)$. The mean of $A_t$ is thus assumed to be 1. \footnote{Because our model is a repeated, static model, the dynamic process of $A_t$ does not affect agents' choices in each period. However, the introduction of aggregate profitability and its dynamics enable us to capture the relationship between the performance of machine learning algorithms and the aggregate state that is going to be clearer in Assumption 1.}

Investors update their beliefs on $z_t$ based on private signals they receive from analysts and the stock price. At the beginning of each period, each investor randomly draws an analyst forecast from the pool of analysts. By the law of large numbers, $\lambda$ investors end up receiving the forecast of tech analysts, and $1-\lambda$ from non-tech analysts. We call the former investors tech investors, and the latter non-tech investors. Investors do not know the exact type of signal they receive.

Investor $j$ only receives one private signal from one analyst: 1) if the signal is provided by a tech analyst, the signal is $\eta_{T,jt} = z_t +\xi_{T,jt}$ with $\xi_{T,jt} \sim N(0, \rho_{\eta }^{-1})$; 2) if the signal is provided by non-tech analysts, it is $\eta_{N,jt} = z_t +\xi_{N,jt}$ with $\xi_{N,jt}\sim N(\theta\delta\epsilon_{t-1}+\tilde f_t, \rho_{\eta }^{-1})$. $\xi_{T, jt}$ ($\xi_{N, jt}$) is the difference between the true profitability and the signal value, that is, the total forecast error, and is uncorrelated across investors. The mean of the error term of tech analysts' forecasts is normalized to 0, and that of the non-tech analysts is $\theta\delta\epsilon_{t-1} + \tilde f_t$. The first term $\theta\delta\epsilon_{t-1}$ captures the behavioral bias as in the first term of Equation \ref{ferror}, and $\tilde f_t$ is a scalar that captures the misspecification error in a reduced form as in the second term of Equation \ref{ferror}. Our empirical finding suggests that the MSE of forecasts varies with the aggregate state, thus we assume $\tilde f_t$ to be correlated with the aggregate state $A_t$. Specifically, the empirical finding suggests that, when the economy is in a recession (in our model, this refers to states where the realized aggregate profitability $A_t$ is lower than its mean), the MSE of machine learning forecasts is less unbiased compared to the good state. Right now, we do not impose any specific functional relationship between $\tilde f_t$ and $A_t$. The only assumption we have is:

% \textit{Assumption 1.} $\tilde f_t$ is higher when $A_t-\mu_A>0$ compared to when $A_t-\mu_A<0$.
\textit{Assumption 1.} The absolute value of $\tilde f_t$ is non-decreasing in $A_t-\mu_A$, and is smaller when $A_t \leq \mu_A$ compared to when $A_t>\mu_A$.
% The precision is assumed to be the same at $\rho_{\eta }$ which depends on the aggregate state $A_t$, that is, $\rho_{\eta }|(A_t = A_h) = \rho^h_{\eta t}$ and $\rho_{\eta }|(A_t = A_l) = \rho^l_{\eta t}$. $A$ larger, $f$ smaller.

Investors do not know about the potential biasedness of signals\footnote{Alternatively, we can model an economy where investors know the biasedness of non-tech forecasts, but do not know the specific type of signals they receive. In this case, investors rationally expect they receive a tech signal with probability $\lambda$, and a non-tech signal with probability $1-\lambda$. So from the investors' point of view, the signal follows a normal distribution with mean $(1-\lambda)(\theta\delta\epsilon_{t-1}+\tilde f_t)$. Thus, when an investor receives a signal $\eta_{jt}$, she knows that $\eta_{jt} - (1-\lambda)(\theta\delta\epsilon_{t-1}+\tilde f_t) \sim N(0,\rho_\eta^{-1})$.  As will become clear in Equation \ref{opt_demand}, the investor demand for equity is linear in investors' posterior expectation $z_t$. This implies that the stock price derived from the market clearing condition will correct for the forecast bias. We choose our current setting because if agents in the economy know tech forecasts are biased, they can fix it directly instead of reporting a biased forecast. The empirical evidence indeed shows that the reported forecasts are biased.} and therefore believe the private signal they receive is $\eta_{jt} = z_t + \xi_{jt}$, where $\xi_{jt}$ follows a normal distribution with mean 0
\[\xi_{jt}\sim N(0, \rho_{\eta }^{-1}).\]

Apart from the private signals, investors also learn about $z_t$ from the market price $p_t$, we assume and later on verify that observing this price is equivalent to observing a normally distributed signal $\eta_{pt} = z_t + e_t$ with $e_t\sim N(0,\rho_{pt}^{-1})$.

Then by Bayes' law, the perceived posterior distribution of $z_t$ is
\begin{align}
	& z_t|\eta_{jt},\eta_{pt} \sim N(\mu_{jt}, \rho_t^{-1}),
\end{align}
where $\rho_t^{-1}\equiv(\rho_z+\rho_{\eta }+\rho_{pt})^{-1}$ and $\mu_{jt}\equiv E[z_t|\eta_{jt},\eta_{pt}]
= \rho_t^{-1}(\rho_z\mu_{zt} + \rho_\eta \eta_{jt} + \rho_{pt} \eta_{pt})$.
\textit{However, if the investor knows the non-tech signal is biased, the correct $\mu_{jt}$ should be $\rho_t^{-1}(\rho_z\mu_{zt} + \rho_\eta (\eta_{jt}- \theta\beta\epsilon_{t-1} - \tilde f_t) + \rho_{pt} \eta_{pt})$.} Due to this bias, after a positive shock $\epsilon_{t-1}$, $1-\lambda$ investors who have received non-tech signals overestimate the firm's profitability.


\textbf{Investor portfolio choice}
After forming expectations about the firm's profitability, investors decide how to allocate their wealth between riskless and risky assets, taking prices as given.

\begin{align}
	& \max_{q_{jt}} U_{jt} = \alpha E_{jt}[W_{jt}|\mathcal{F}_{t}] - \frac{\alpha^2}{2}V_{jt}[W_{jt}|\mathcal{F}_{t}],\label{investor}\\
	& s.t.\ W_{jt} = r_t(W_t-q_{jt}p_t) + q_{jt}A_tz_t,
\end{align}
where $ \mathcal{F}_{t} = \{\eta_{jt},p_t, \mathcal{F}_{t}^-\}$ is investors' information set.

Plug the budget constraint into investor's utility, the maximization problem transforms into
\begin{align}
	\max_{q_{jt}} U_{jt} % & = \alpha E_j\left[r_tW_t + q_{jt}(A_tz_t - p_tr_t) |\mathcal{F}_{t}\right]
	%- \frac{\alpha^2}{2}V_j\left[r_tW_t + q_{jt}(A_tz_t - p_tr_t)|\mathcal{F}_{t}\right],\\
	& =  \alpha E_j\left[r_tW_t + q_{jt}(A_tz_t - p_tr_t)|\mathcal{F}_{t} \right]
	- \frac{\alpha^2A_t^2}{2}q_{jt}^2V_j(z_t|\mathcal{F}_{t}).
	\label{demand}
\end{align}

\textbf{Asset market clearing condition}
The equity market clears:
\begin{align}
	& \int^1_0 q_{jt}dj = \bar x_t + x_t,
\end{align}
where $x_t$ is noisy supply and is distributed normally $N(0,\rho_x^{-1})$. The introduction of noisy supply prevents fully revealing $z_t$ with a continuum of traders.

\subsection{Equilibrium}
\textbf{Equilibrium definition}
An equilibrium is a stock price $\{p_t\}$, firm equity issuance $\bar x_t$, investor demand for equity $\{q_{jt}\}_j$ such that:
\begin{enumerate}
    \item Given a common prior about the distribution of $z_t$, a perceived private signal $\eta_{jt}$ about idiosyncratic profitability $z_t$ and stock price $p_t$, each investor $j$ updates her perceived distribution of $z_t$, and $q_{jt}$ solves her optimization problem in Equation \ref{investor} based on her posterior belief.
    \item Asset market clears: $\int^1_0 q_{jt}dj = \bar x_t + x_t$.
    \item The equity issuance $\bar x_t$ solves the firm owner's net revenue maximization problem in Equation \ref{firm}.
\end{enumerate}

From the investors' optimization problem, we can derive investors' demand for equity from the first order condition of Equation \ref{demand}:
\begin{align}
	& q_{jt} %= \frac{A_tE_j(z_t|\mathcal{F}_{t}) -p_tr_t}{\alpha A_t^2 V_j(z_t|\mathcal{F}_{t})}
	= \frac{A_t\rho_t^{-1}(\rho_z\mu_{zt}+\rho_\eta\eta_{jt}+\rho_{pt}\eta_{pt})
		- p_tr_t}{\alpha A_t^2\rho_t^{-1}}\label{opt_demand}.
\end{align}
The diagnostic expectation component $\theta\delta\epsilon_{t-1}$ is embedded in $\eta_{jt}$. Recall that all investors believe the private signals they receive are unbiased, so $\eta_{jt} = z_t+\xi_{jt}$ with $\xi_{jt}\sim N(0, \rho_{\eta }^{-1})$.

From the market clearing condition, we have
\begin{align*}
%5	& \frac{A_t\rho_t^{-1}(\rho_z\mu_{zt}
%		+\rho_\eta (z_{t}+(1-\lambda)(\theta\delta\epsilon_{t-1}+\tilde f_t))
%		+\rho_{pt}\eta_{pt})
%		- p_tr_t}{\alpha A_t^2\rho_t^{-1}} = \bar x_t + x_t,\\
	& p_t r_t %= A_t\rho_t^{-1}\rho_z\mu_{zt} + A_t\rho_t^{-1}\rho_{\eta }(z_t + (1-\lambda)(\theta\delta\epsilon_{t-1}+\tilde f_t))+A_t\rho_t^{-1}\rho_{pt}(z_t+e_t)-\alpha A_t^2\rho_t^{-1}(\bar x_t + x_t),\\
	 = A_t\rho_t^{-1}\rho_z\mu_{zt} + A_t\rho_t^{-1}\rho_{\eta }(1-\lambda)(\theta\delta\epsilon_{t-1}+\tilde f_t)
	-\alpha A_t^2\rho_t^{-1}\bar x_t
	+ A_t\rho_t^{-1}\rho_{\eta }(z_t - A_t\frac{\alpha}{\rho_{\eta }}x_t)+A_t\rho_t^{-1}\rho_{pt}(z_t+e_t).
\end{align*}
Both sides should be $z_t+e_t$ measurable, therefore
\[e_t = -A_t\frac{\alpha}{\rho_{\eta }}x_t.\]
% When supply high, price is low. Investor j thinks this could be because other investors' demand is low as the signals they receive \eta is low. So a high x translates to a signal suggesting a low z.
Thus,
\begin{align}
	& p_t %= \frac{A_t\rho_t^{-1}}{r_t}[\rho_z\mu_{zt} + \rho_{\eta }(1-\lambda)(\theta\delta\epsilon_{t-1}+\tilde f_t)
	%-\alpha A_t\bar x_t
	%+(\rho_{\eta }+\rho_{pt})(z_t-A_t\frac{\alpha}{\rho_{\eta }}x_t)],\\
	= \frac{A_t\rho_t^{-1}}{r_t}[\Gamma_t - \alpha A_t\bar x_t +
	(\rho_{\eta }+\rho_{pt})(z_t-A_t\frac{\alpha}{\rho_{\eta }}x_t)]\label{Tdemand},
\end{align}
where $\Gamma_t\equiv \rho_z\mu_{zt} + \rho_{\eta }(1-\lambda)(\theta\delta\epsilon_{t-1}+\tilde f_t)$, $\rho_{pt} = \frac{\rho_\eta^2}{\alpha^2A_t^2}\rho_x$.

Equation \ref{Tdemand} gives the demand function of the risky asset. Holding everything else constant, a rise in $\theta$ leads to a rise in the stock price $p_t$ if the recent firm-level shock $\epsilon_{t-1}$ is positive. Intuitively, if investors overestimate the value of $z_t$ following a positive shock, they are willing to pay more in order to get one share of stock.

From the first order condition of the firm's maximization problem, we can also obtain the supply function (we focus on the interior solution). The firm owner does not use any signals to update her belief on $z_t$, thus the first order condition gives
\begin{align*}
	& E[p_t + \bar x_t\frac{\partial p_t}{\partial \bar x_t}|\mathcal{F}_{t}^-] - \phi_1(\bar x_t,\bar x_{t-1}) = 0,\\
	& E[p_t|\mathcal{F}_{t}^-] - \bar x_t \frac{\alpha A_t^2\rho_t^{-1}}{r_t} - \tilde\phi_1 - \phi_2(\bar x_t - \bar x_{t-1}) = 0,
\end{align*}
where $\tilde\phi_1 = \phi_1$ if $\Delta x_t>0$, $\tilde\phi_1 = -\phi_1$ otherwise.

Plug Equation \ref{Tdemand} into the equation above. Given that the noisy supply has a mean of 0, we have
\begin{align}
	& \frac{A_t\rho_t^{-1}}{r_t}[\Gamma_t - \alpha A_t\bar x_t +
	(\rho_{\eta }+\rho_{pt})\mu_{zt}] - \bar x_t \frac{\alpha A_t^2\rho_t^{-1}}{r_t} - \tilde\phi_1 - \phi_2(\bar x_t - \bar x_{t-1}) = 0.\label{eq}
\end{align}
Equation \ref{eq} gives us the equilibrium value of $\bar x_t$ :
\begin{proposition}
	In equilibrium, the number of shares issued is
	\begin{align}
		\bar x_t %=\frac{\frac{A_t\rho^{-1}}{r_t}[ \rho_z\mu_{zt} + \rho_{\eta }((1-\lambda)(\mu_{zt}+\theta\delta\epsilon_{t-1}+\tilde f_t)+\lambda \mu_{zt}) + \rho_p\mu_{zt}]-\tilde\phi_1 + \phi_2\bar x_{t-1}}{2\frac{\alpha A_t^2\rho^{-1}}{r_t} + \phi_2}
  =\frac{\frac{A_t\rho^{-1}}{r_t}[ (\rho_z+\rho_{\eta }+\rho_p)\mu_{zt} + \rho_{\eta }(1-\lambda)(\theta\delta\epsilon_{t-1}+\tilde f_t) ]-\tilde\phi_1 + \phi_2\bar x_{t-1}}
		{2\frac{\alpha A_t^2\rho^{-1}}{r_t} + \phi_2}\label{eissuance},
	\end{align}
	where $\rho^{-1}=(\rho_z+\rho_{\eta }+\rho_p)^{-1}$, $\rho_p = \frac{\rho_\eta^2}{\alpha^2A_t^2}\rho_x$.
\end{proposition}

Equation \ref{eissuance} gives the total shares of equity the firm issues in equilibrium. The first term in the numerator captures the effect of expected $z_t$ which combines the information from the prior belief, private signals and stock price. The remaining terms in the numerator represent the marginal equity issuance costs (interior solution). The denominator captures the sensitivity of price to demand plus the convex equity issuance cost.
% If the demand is more sensitive to price change, denominator is larger, \bar x smaller

\begin{corollary}
	The equilibrium level of equity issuance exhibits less overreaction to recent shocks when the economy has more tech analysts, i.e. when $\lambda$ increases.
\end{corollary}
\noindent\textbf{Proof of Corollary 6.2}

The overreaction of equity issuance to the recent shock is
\begin{align*}
   & \frac{\partial\bar x_t}{\partial \epsilon_{t-1}}
   = \frac{\frac{ A_t\rho^{-1}}{r_t}\rho_\eta(1-\lambda)\theta\delta}{2\frac{\alpha A_t^2\rho^{-1}}{r_t} + \phi_2}.
\end{align*}
The response of this overreaction to changes in $\lambda$ is
\begin{align*}
   & \frac{\partial\bar x_t}{\partial \epsilon_{t-1}\partial \lambda}
   = -\frac{\frac{A_t\rho^{-1}}{r_t}\rho_\eta\theta\delta}{2\frac{\alpha A_t^2\rho^{-1}}{r_t} + \phi_2} < 0.
\end{align*}
Therefore, the overreaction decreases as $\lambda$ rises. \hfill$\square$

This result is expected given that tech analysts exhibit lower overreaction to recent idiosyncratic shocks. So if we transit from an economy with few tech analysts to an economy with more tech analysts, the equity issuance we observe would be more rational in the sense that it overreacts less to firm-level shocks.
\begin{corollary}
In the special case where the equity issuance cost equals 0: i) Holding $f_t$ fixed, $\bar x_t$ is decreasing in $A_t$; ii) Holding $\tilde A_t$ fixed, $\bar x_t$ is non-decreasing in $\tilde f_t$. The sign of $\frac{\partial\bar x_t}{\partial A_t}$ is ambiguous once we account for the correlation between $A_t$ and $f_t$.
\end{corollary}
\noindent\textbf{Proof of Corollary 6.3}

If $\phi_1 = \phi_2 =0$, then
\begin{align*}
   & \bar x_t = \frac{(\rho_z+\rho_\eta +\frac{\rho_\eta^2}{\alpha^2A_t^2}\rho_x)\mu_{zt}+\rho_\eta(1-\lambda)(\theta\delta\epsilon_{t-1}+\tilde f_t)}{2\alpha A_t}.
\end{align*}
Holding $\tilde f_t$ constant, a rise in $A_t$ leads to an increase in the denominator and a decline in the numerator, both of which contributes to a lower $\bar x_t$ given that $\frac{\rho_\eta^2}{\alpha^2A_t^2}\rho_x$, $\bar x_t$ and $\alpha A_t$ are all non-negative\footnote{$\bar x_t$ is the total equity issued, not the change of equity issuance $\Delta x_t$. Our parameters will be set such that $\bar x_t\geq0$.}.

\noindent Holding $A_t$ constant, $\frac{\partial\bar x_t}{\partial\tilde f_t} = \frac{\rho_\eta(1-\lambda)}{2\alpha A_t} \geq 0$.

In the simplified problem above where the equity issuance cost is 0, $A_t$ affects the firm's equity issuance from two channels: the direct effects on the conditional mean and variance of the investment payoff and the indirect effect on the model misspecification error. The former tends to have a net negative impact on firm equity issuance, while the latter tends to have a positive effect because higher $\tilde f_t$ translates into a higher expected payoff. The overall impact is ambiguous. In the extreme case where the response of $\tilde f_t$ to $A_t$ is negligible, $\bar x_t$ is decreasing in $A_t$, while in the extreme case where a small increase in $A_t$ leads to a huge rise in $\tilde f_t$ which approaches $\infty$, $\bar x_t$ is increasing in $A_t$. Similar patterns hold for the value of $\bar x_t$ in Equation \ref{eissuance} (this is also verified by our numerical example below).\hfill$\square$

\noindent\textbf{Numerical example when $\tilde f_t$ is linear in $A_t$.} Given the sign of $\frac{\partial\bar x_t}{\partial A_t}$ is ambiguous, we use a numerical exercise to examine the response of $\bar x_t$ to $A_t$ under reasonable parameter values.


\begin{table}[h]
    \caption{Parameters in the Numerical Example}
    \label{tab:param}
   \begin{footnotesize}
    This table displays the parameter values used in the numerical example. The values of $\rho_x,\alpha,\mu_z,\phi_1,\phi_2,r$ are from \cite{begenau2018big}, and the values of $\rho_z,\delta,\theta$ are from \cite{bordalo2021real}.
    \end{footnotesize}	
    \begin{center}
    \begin{tabular}{ccccccccccc}
    \toprule
    & $\rho_z$ & $\rho_x$ & $\rho_\eta$ & $\alpha$ & $\mu_z$ & $\phi_1$ & $\phi_2$ & $r$ & $\delta$ & $\theta$ \\
     \midrule
    & 81.16 & 4 & 2.78 & 0.1 & 15 & 0.091 & 0.0004 & 1.01 & 0.785 & 0.991\\
     \bottomrule
    \end{tabular}
    \end{center}
\end{table}

It is not feasible for us to estimate the exact relationship between $\tilde f_t$ and $A_t$ from the data because the true functional form of $f_t$ is unknown. Here we consider a simple case by assuming $\tilde f_t = 500$ when $A_t>\mu_A=1$ and  $\tilde f_t = 100$ when $A_t\leq\mu_A$, consistent with Assumption 1. Table \ref{tab:param} displays the values of parameters used in this numerical example. The values of $\rho_x,\alpha,\mu_z,\phi_1,\phi_2,r$ are from \cite{begenau2018big}, and the values of $\rho_z,\delta,\theta$ are from \cite{bordalo2021real}. Admittedly, our setting is different from \cite{bordalo2021real}, so the parameter values in their setting may not exactly fit our model. Thus, the numerical example here should not be regarded as a formal estimation.  $\rho_\eta$ comes from the assumption that $\sigma_\eta=0.6$ ($\rho_\eta= 1/\sigma_\eta^2$) which is set to be slightly larger than $\sigma_x=0.5$ ($\rho_x=1/\sigma_x^2$). We consider a scenario where the firm just experienced a positive idiosyncratic shock such that $\epsilon_{t-1}=10$.
% \footnote{If we set $\epsilon_{t-1}=-10$, the results are similar because under the parameter values we choose, the magnitude of the procyclical model misspecification error outweighs that of the diagnostic component. That is, $|\theta\delta\epsilon_{t-1}|<\nu A_t$}.

\begin{figure}[hbt!]
        \caption{Numerical Example}
	\label{fig:numerical}
 This figure shows the change of $\bar x$ when the aggregate state $A$ changes for different $\lambda$'s. The solid line shows the result when there is no tech analyst in the economy, while the dashed line illustrates the result when 90\% analysts use algorithms to make forecasts. The parameter values are given in Table \ref{tab:param}. $\bar x_{t-1}$ is set to be 1000 (small enough so that the optimal $\bar x_t$ is an interior solution), $\epsilon_{t-1} = 10$.
	\begin{center}
	\begin{minipage}{0.8\textwidth}
		\centering
		\includegraphics[width=1.0\linewidth]{figures/num_example.png}
	\end{minipage}
	\end{center}
\end{figure}

Figure \ref{fig:numerical} illustrates the results for high and low $\lambda$'s. With parameter values in Table \ref{tab:param}, higher aggregate profitability ($A$ ranges from 0.9 to 1.1) leads to less equity issuance except for a jump at $A=\mu_A$. Moreover, when the economy has more tech analysts, the equity issuance tends to be smaller over the entire range of $A$. This is because we assume that $\tilde f_t$ is always positive, which implies that the non-tech analysts are always overoptimistic. When the economy has more tech analysts (a larger $\lambda$), the investors are overall more rational in the sense that they are less overoptimistic, and thus tend to demand fewer risky assets which in equilibrium drags down the equity issuance.
% However, as we only borrow from them the parameters related to the process of idiosyncratic productivity and diagnostic expectation which are relatively separate from


\section{Conclusion}
\label{sec:conclusion}

A key behavioral challenge to the rational model is the substantial evidence that a great many people  overreact to news \citep{bordalo2022overreaction}. We show that it is not just people that overreact. High quality machine algorithms also overreact - albeit less strongly than do people. This raises the issue of the underlying source of overreaction. We find that it appears to be connects to the learning rate. By reducing the learning rate below the rate chosen by cross-validation, overreaction is reduced, as the algorithm produces predictions that less fully reflect new information altogether.

Analysts who depend on machine learning algorithms overreact much less than do more traditional analysts. Despite this, replacing traditional analysts with machine learning algorithms is not costless. Traditional stock analysts, despite their behavioral limitations do have information that is not otherwise readily available for use in the algorithms.

%%%%%%%%%%%%%%%%%%%%%%%%%%%%%%%%%%%%%%%%%%%%%%%%%%%%
%\clearpage
\begin{singlespace}
%\bibliographystyle{jfe}
\bibliographystyle{aer}
\bibliography{Profit_Refs2022May}
\end{singlespace}
%%%%%%%%%%%%%%%%%%%%%%%%%%%%%%%%%%%%%%%%%%%%%%%%%%%%







\clearpage
\appendix
\section{Appendix}
\setcounter{table}{0}
\renewcommand{\thetable}{A\arabic{table}}
\clearpage
\begin{table}[tbp]
\caption{Technical Majors }
\label{tab:majors}\vspace{2 mm}
\par
\centering
\begin{tabular}{ll}
\toprule
Transportation                        & Finance, Physics                          \\
Systems engineering                   & Finance, IT                               \\
Quantitative Econ, Financial Modeling & Finance and Information Syatems           \\
quantitative analysis                 & Environmental Engineering                 \\
Quantatitive finance                  & Engineering                               \\
Quant econ course                     & Eletrical, Ocean, Civil Engineering       \\
Physics and Economics                 & Eletrical Engineering, CS                 \\
Physics and applied math              & Electronics and communication enginerring \\
Physics                               & Electrical engineering                    \\
PhD Pharmaceutical sciences           & Electrical Electronic Engineering         \\
PhD in Pharmacology                   & Electrical Computer Engineering           \\
PhD in neuroscience                   & Electrical and electronics engineering    \\
PhD Biomedical/Medical Engineering    & Data analyst course                       \\
Computer                              & Computer Science                          \\
Mehcanical Engineering                & Computer engineering                      \\
Medical Chemistry                     & Computer engineering                      \\
Mechanical Engineering                & Computational biology, bioinformatics     \\
Mathematics, Finance                  & Computatioanl finance, information system \\
Mathematics, economics                & Civil Engineering                         \\
Mathematics, acturial studies         & Civil engineering                         \\
Mathematics minor                     & Chemistry, environmental engineering      \\
Mathematics and statistics            & Chemistry engineering                     \\
Mathematics and physics               & Chemistry and Applied mathematics         \\
Mathematics                           & Chemistry                                 \\
Mathematical science                  & Chemical Engineering                      \\
Mathematical economics analysis       & Biophysics and biochemistry               \\
Mathematical economics                & biophysics                                \\
Mathematic, Physics                   & Bio-Organic Chemistry                     \\
Mathematic and economics              & biomedical engineering                    \\
Math and Economics                    & BioMathematics                            \\
Materials Science                     & Biology and chemistry                     \\
Material metallurgical engineering    & Biology and Biochemistry                  \\
Masters in Data Science               & Biochemistry                              \\
Master of engineering                 & behavioral science                        \\
M.A. in Neuroscience                  & Information systems                       \\
Industrial Engineering                &                                           \\
Industrial and system engineering     &                                           \\
Industrial and management engineering &                                           \\
general engineering                   &                                           \\
Financial Modeling, Engineering       &                             \\   
\bottomrule
\end{tabular}
\end{table}
\clearpage


\begin{table}[tbp]
	\caption{Predictable Forecast Errors}
 \label{tab:over_gmm}
	\begin{footnotesize}      
        This table report GMM regression results of regressing forecast error at $t+1$ on the information at time $t$. Forecast error at $t+1$ is $\text{Forecast Error}_{t+1} \coloneqq p_{t+1} - E_t p_{t+1} $, where $E_t$ is forecast based on date $t$ data. In columns (1), forecast $E_t p_{t+1}$ is the median earnings forecast of all analysts forecasts conducted during fiscal year t. In columns (3), forecast $E_t p_{t+1}$ is the median earnings forecast of all machine learning forecasts conducted during fiscal year t. Earnings are converted from EPS forecast to total earnings. Investment$_{t}$ capital expenditure at time $t$. Debt issuance$_{t}$ is net debt issuance at time $t$. All variables are scaled by total assets.
	\end{footnotesize}	
	\begin{center}
            \begin{tabular}{lcccc}
            \toprule
             & (1)              & (2)                   \\
             & Forecast   Error & Forecast   Error  \\
             & Analysts   & Machine  \\
             \midrule
            Investment&     -0.472***   & -0.345** \\
                          & (0.003)   & (0.025)  \\
            \midrule
            Firm FE & Yes        & Yes           \\
            Year FE & Yes       & Yes            \\
            Period  & 1994-2018 & 1994-2018  \\
            N       & 54536     & 53324      \\
            \bottomrule
            \end{tabular}
	\end{center}
\end{table}
\clearpage

\begin{table}[tbp]
	\caption{Predictable Forecast Errors Using Large Learning Rates}
        \label{tab:largelr}
	\begin{footnotesize}    
         This table presents the results of an OLS regression analysis of the forecast error at time $t + 1$ and Wald test results from seemingly unrelated regressions. The forecast error at $t + 1$ is defined as $\text{Forecast Error}_{t+1} = p_{t+1} - F_t p_{t+1}$, where $F_t p_{t+1}$ is the average of median monthly earnings forecast based on data from time $t$, and $p_{t+1}$ represents the actual earnings at time $t + 1$. Forecasts are calculated from various machine learning models that use different hyper-parameters. The default hyper-parameters are set as the following values: depth of the tree is $2$ and number of
        trees in the forest is $50$. Table reports the results using different learning rates. Investment$_{t}$ is capital expenditure at time $t$. The $\chi^2$ of the Wald test is used to test the difference of regression coefficients from default hyper parameters, with the results of the test presented in the table. The p-value of the test is also included in brackets.
        \end{footnotesize}	
	\begin{center}
        \begin{tabular}{lcccc}
            \toprule
             & (1)              & (2)              & (3)              & (4)              \\
             & Forecast   Error & Forecast   Error & Forecast   Error & Forecast   Error \\
             & Analysts  &  Analysts   & Machine  & Machine  \\
             \cmidrule(lr){2-5}
             Learning rate & 0.15 & 0.25  & 0.3 & 0.5 \\
             \midrule
            Investment& -0.106***       & -0.107***       & -0.105***       & -0.099***       \\
                & (-7.766)        & (-7.285)        & (-7.262)        & (-6.132)        \\
                \midrule
            $\chi^2$      & 0.019	&0.012	 & 0.072&1.261  \\
            & [0.890]	&[0.911]	 &[0.788]	& [0.261]  \\
            Firm FE & Yes             & Yes             & Yes             & Yes             \\
            Year FE & Yes             & Yes             & Yes             & Yes             \\
            Period  & 1994-2018       & 1994-2018       & 1994-2018       & 1994-2018       \\
            N       & 53321           & 53321           & 53321           & 53321           \\
            AdjR2   & 0.21            & 0.20            & 0.20            & 0.18           \\
            \bottomrule
            \end{tabular}
	\end{center}
\end{table}
\clearpage


\begin{table}[tbp]
	\caption{Analysts Forecast Errors}
 \label{tab:analystsinvestment}
	\begin{footnotesize}      
        This table report OLS regression regressing firm investment and debt on analysts forecast errors.
	\end{footnotesize}	
	\begin{center}
            \begin{tabular}{lcccc}
            \toprule
             & (1)       & (2)       & (3)             & (4)             \\
             & Inv$_{t+1}$   & Inv$_{t+1}$   & Debt Net Iss.$_{t+1}$ & Debt Iss.$_{t+1}$ \\
             \midrule
            Analysts Forecast Errors & -0.007*** & -0.007*** & -0.028***       & -0.032***       \\
                         & (-2.668)  & (-3.446)  & (-6.466)        & (-6.017)        \\
                         &           &           &                 &                 \\
            Profit & 0.035***  & 0.020***  & -0.001          & 0.017***        \\
                         & (10.683)  & (6.617)   & (-0.323)        & (4.174)         \\
            Firm FE      & No        & Yes       & No              & Yes             \\
            Year FE      & Yes       & Yes       & Yes             & Yes             \\
            Period       & 1994-2018 & 1994-2018 & 1994-2018       & 1994-2018       \\
            N            & 64218     & 62788     & 64218           & 62788           \\
            AdjR2        & 0.05      & 0.65      & 0.02            & 0.07   \\
            \bottomrule
            \end{tabular}
	\end{center}
\end{table}
\clearpage

%
\subsection{Gradient boosted regression trees}

\label{sec:gbrt}
Gradient boosting regression trees are a particularly prominent and
empirically successful prediction method used in many applications. It starts
with regression trees and then refines them iteratively by focusing on the
errors in the previous iteration. This idea of refinement by focusing on error
correction is known as boosting. Gradually an entire forest of trees is
constructed. Then an average across the set of trees is used as the model's
prediction.

Start with a regression tree. It is the basic building block for the method. A
regression tree uses numerical cutoffs to assign an observation to a branch. In
the simplest case there is a threshold number $\overline{x}$. If the
observation has $x>\overline{x}$ the data is assigned to the upper branch.  If
the observation has $x\leq \overline{x}$ the data is assigned to the lower
branch. Then within each branch there are subsequent partitions constructed.
Eventually an entire set of branches are constructed. The final set is called a
leaf, and it defines the set that the particular observation belongs to. A
range of rules can be used to define the number of branches, their order of
consideration and so on...

\[\text{[INSERT \autoref{fig:TreeExample} AROUND HERE]}\]

Figure \ref{fig:TreeExample} shows an example of a simple tree. It predicts
expected profit using current profit, and book to market. If current
profit is greater or equal to $0.5$, the decision tree predicts the
future profitability to be $0.8$. If current profit is less than $0.5$
and log value of book to market is less than $1$, predicted future
profit is $0.1$. Otherwise, the predicted profit is $0.5$.
The tree in Figure \ref{fig:TreeExample} can also be represented as,
\begin{equation}
\hat{y} = h(x) = \sum_{m = 1}^{3} c_m I \{ (x_1,x_2) \in R_m \}
\end{equation}
where $R_m$ is the partition of the input variables, and $c_m$ is the
predicted value assigned to the terminal leaf.

In the example it takes at most two ``branches'' to reach the
final terminal leaves. This is called the number of layers of the decision
tree. The more layers the tree has, the more complex the model is.
When a regression tree model is deep (more layers), the model tends to have low
bias but larger variance. While when the model is shallow (fewer layers), the
model becomes too simple, with low variance but large bias.

Regression trees are easy to interpret, but commonly do not predict well.
Ensembles of trees have been found to predict better, see \citet{hastie2009elements}. Random forest constructs many trees using independent bootstrap samples from the data to generate a forest. Boosting creates trees to improve on past prediction mistake instead of bootstrap samples from the original data. Boosting is more complex than bootstrapping, but it tends to improve final performance, see \citet{efron2016computer}.

Gradient boosting is the most widely adopted version of boosted forests. It
starts by estimating decision trees with fixed shallow depth. Then it computes
the residuals for the trees. At the next iteration more weight is devoted to
the cases in which the model fit poorly. In the end an ensemble of trees are
used to `vote' on the appropriate results. This generally reduces the bias in a
simple tree model while maintaining the low variance. The main drawback
relative to a simple tree as in Figure \ref{fig:TreeExample}, is that forests
do not have such simple depictions that show how each variable affects the
final result.

A more formal representation is,
\begin{equation}
\hat{y} = F_M(x) = \sum_{m = 1}^{M} h_m(x)
\end{equation}
where $h_m$ is decision tree regressor with depth of $d$, and $M$ is the
number of trees in the forest. $F_M(x)$ is solved by using a greedy algorithm
framework,
\begin{equation}
F_k(x) = F_{k-1}(x) + \gamma h_k
\end{equation}
where $\gamma$ is the learning rate. The learning rate shrinks the contribution
of each additional tree. $h_k$ is the newly added tree solved by minimize a loss
function $L$ given $F_{k-1}(x)$
\begin{equation}
h_k = \arg \min_h \sum_{i }(L(y_i,F_{k-1} (x_i) + \gamma h(x_i) )
\end{equation}


There are three important hyperparameters in gradient boosting: the depth of the
tree $d$ (max$\_$depth), the number of trees in the forest $M$ (n$\_$%
estimators), and learning rate $\gamma$ (learning$\_$rate). The default
parameters have the following values: depth of the tree is $3$, number of
trees in the forest is $100$, and learning rate is $0.1$. We have
systematically carried out the analysis with the default hyperparameters as
well as with hyperparameters optimized using cross-validation. The results are
very similar. Except where noted, the Tables use the default hyperparameters.

As in the Fama-MacBeth estimation, to predict profit at time $\tau+1$,
$\hat{\pi}_{\tau+1}$, we train the model using information before $\tau+1$,
$\{X_{t-1},\pi_{t} \}_{t\leq \tau}$. We then apply the model using factors
available at time $\tau$, $X_\tau$.

Gradient boosting estimation was done using the software
GradientBoostingRegressor from
https://scikit-learn.org/stable/modules/ensemble.html, and using XGBoost from
\citet{chen2016xgboost}. The results are very similar. The reported results in
the Tables use the algorithm GradientBoostingRegressor from scikit-learn.org.
For greater detail see the algorithm \ref{alg:GBRTPredict}.


\begin{algorithm}
\caption{GBRT Predictions}
\label{alg:GBRTPredict}
\begin{algorithmic}
\Procedure{GBRT}{$X_{i,t},\pi_{i,t+1}$}\Comment{Where $t$ - time year,
$i$ - firm, $X_{i,t}$ - predictors at time $t$, $\pi_{i,t+1}$ -
profitability at $t+1$ to predict}
\For{$ 1975 \leq T \leq 2014$}
\State Training sample are observations for all firms and $ 1964 \leq t 
\leq T-1$
\State Fit the following GBRT model using training sample data
\State $\pi_{i,t+1}  = f(X_{i,t}) $
\State Get estimated model  $\hat{f}_T$
\State Use information at time $T$ $X_{i,T}$ and fitted model 
$\hat{f}_T$ to predict profitability at $T+1$
\State $\hat{\pi}_{i,T+1}  = \hat{f}_T(X_{i,T})$
\EndFor
\EndProcedure
\end{algorithmic}
\end{algorithm}




\section{Methods to evaluate predictions}

There are many ways to evaluate predictions that have distinct justifications.
We report the results using in-sample $R^2$, out of sample $R^2$,
\citet{diebold2002comparing} t-tests, and estimation with cross-validation.

\subsection{In-sample $R^2$}

An in-sample $R^2$ is perhaps the best known method to assess the ability of a
model to account for the variation in the data. If a model is correctly
specified, then maximum likelihood estimated parameters will be optimal. As
argued by \citet{inoue2005sample} it is then appropriate to evaluate those
parameters within the sample.

Commonly we have less than full confidence that the structure of the model
being estimated is the actual true data generating process. Any model may be
reasonable, but it is almost certainly misspecified relative the true data
generating process. Accordingly, as is standard in the machine learning
literature \citep{efron2016computer}, we place much greater weight on the
out-of-sample performance as discussed in the next section.

Even if we are not confident that the model is correctly specified, an
in-sample $R^2$ is widely reported as a familiar diagnostic tool. In-sample
explanatory power is given by
\begin{equation}
R^2  =1 - \frac{\sum (\hat{\pi}_{i,t+1}  - \pi_{i,t+1} )^2}{\sum (\pi_{i,t+1} -\bar{\pi}_{i,t+1} )^2}.
\end{equation}
where $\bar{\pi}_{i,t+1}$ is the sample average profitability. Because this is in-sample, it is bounded below by 0. The
calculation steps are given in greater detail as Algorithm \ref{alg:InSampleR2}.

\begin{algorithm}
\caption{In-Sample $R^2$}
\label{alg:InSampleR2}
\begin{algorithmic}
\Procedure{IS}{$X_{i,t},\pi_{i,t+1}$}\Comment{Where $t$ - time year, 
$i$ - firm, $X_{i,t}$ - predictors at time $t$, $\pi_{i,t+1}$ - 
profitability at $t+1$ to predict}
\State Use observations such that $1975 \leq t \leq 2014$ as the whole 
sample, ($1976 \leq t+1 \leq 2015$)
\If {evaluate FM}
\ForAll{firm-year observations in the whole sample}
\State for all time period t, run cross-sectional regression for all 
firms at time t
\State $\pi_{i,t+1}  = \lambda_{0,t} + \lambda_{1,t} X_{i,t} +  
\varepsilon_{i,t}$
\State get estimated coefficients for each time period t $ 
\{\hat{\lambda}_{0,t} ,\hat{\lambda}_{1,t} \}$
\State compute the average coefficients $ \{\hat{\lambda}_{0} 
,\hat{\lambda}_{1} \}$
\EndFor
\ForAll{firm-year observations in the whole sample}
\State make prediction of profitability in the whole sample
\State $\hat{\pi}_{i,t+1}  = \hat{\lambda}_{0} + \hat{\lambda}_{1} 
X_{i,t}$
\State calculate the in-sample $R^2$ in the whole sam
\State $R^2_{IS} = 1 - \frac{\sum (\hat{\pi}_{i,t+1}  - \pi_{i,t+1} 
)^2}{\sum (\pi_{i,t+1} -\bar{\pi}_{i,t+1} )^2}$ where 
$\bar{\pi}_{i,t+1}$ is the average of profitability $\pi_{i,t+1}$ for 
the whole sample
\EndFor
\EndIf
\If {evaluate GBRT}
\ForAll{firm-year observations in the whole sample }
\State Fit the following GBRT model using the whole sample
\State $\pi_{i,t+1}  = f(X_{i,t}) $
\State Get estimated model  $\hat{f}$
\EndFor
\ForAll{firm-year observations in the whole sample}
\State Use fitted model $\hat{f}$ to predict profitability in the whole 
sample
\State $\hat{\pi}_{i,t+1}  = \hat{f}(X_{i,t})$
\State calculate the in-sample $R^2$ in the whole sam
\State $R^2_{IS} = 1 - \frac{\sum (\hat{\pi}_{i,t+1}  - \pi_{i,t+1} 
)^2}{\sum (\pi_{i,t+1} -\bar{\pi}_{i,t+1} )^2}$ where 
$\bar{\pi}_{i,t+1}$ is the average of profitability $\pi_{i,t+1}$ for 
the whole sample
\EndFor
\EndIf
\EndProcedure
\end{algorithmic}
\end{algorithm}


\subsection{Out-of-sample $R^2$}
Out-of-sample explanatory power is given by
\begin{equation}
R^2  =1 - \frac{\sum (\hat{\pi}_{i,t+1}  - \pi_{i,t+1} )^2}{\sum (\pi_{i,t+1} -\pi_{i,t} )^2}.
\end{equation}
If $\hat{\pi}_{i,t+1}$ is very far from $\pi_{i,t+1}$ then the numerator may be
much larger than the denominator and so the term being subtracted may be larger
than one. So the out-of-sample $R^2$ is not bounded below by zero. When that
happens the model is clearly not doing a good job of explaining the data.

\subsection{\citet{diebold2002comparing}}

A popular and useful method of evaluating model performance is the
\citet{diebold2002comparing} t-test, see also \citet{diebold2015comparing}.
Following \citet{gu2020empirical}, the test statistics DM is calculated as
\begin{eqnarray}
DM &=& \bar{d}^{1,2}_{t}/\hat{\sigma}(\bar{d}^{1,2}_{t}), \text{where} \\
d^{1,2}_{t+1} &=& \frac{1}{n} \sum_{i=1}^{n} ((e^2_{i,t+1})^2 - (e^1_{i,t+1})^2), \notag
\end{eqnarray}
$e^1_{i,t+1}, e^2_{i,t+1}$ is the prediction error for firm $i$ profitability at time $t+1$ using each method, and $\bar{d}^{1,2}_{t}$ and $\hat{\sigma}(\bar{d}^{1,2}_{t})$ are the mean and Newey-West standard error of the time series $d^{1,2}_{t}$, respectively.

\subsection{Cross-validation}
Cross-validation is a standard machine learning method intended to reduce
overfitting, see \citep{hastie2009elements, efron2016computer, bates2021cross}.
With a panel of data there is again an issue of how to deal with the time
dimension. We also want to ensure data comparability to the out-of-sample $R^2$
calculation.

Accordingly, firm-year observations from 1975 to 2014 are used as the data. The
data is randomly partitioned into 10-folds using subsamples that are stratified
by year. This ensures proper balance of years in each fold. No other factors
were used for further stratification.

The model is estimated 10 times and then an average is reported. In the first
round, the first fold is held out for validation and the remaining 9 folds are
used for parameter estimation.  In the second round, the second fold is held
out for validation and the remaining 9 folds are used for parameter estimation.
The process continues in this way until 10 estimates have been computed.

The reported out-of-sample $R^2$ for cross-validation is the average
out-of-sample $R^2$ across all 10 estimations. The calculation steps are given
in detail as Algorithm \ref{alg:KFoldCV}.

\begin{algorithm}
\caption{K-Fold Stratified Cross Validation}
\label{alg:KFoldCV}
\begin{algorithmic}
\Procedure{CV}{$k,X_{i,t},\pi_{i,t+1}$}\Comment{Where $t$ - time year, 
$i$ - firm, $X_{i,t}$ - predictors at time $t$, $\pi_{i,t+1}$ - 
profitability at $t+1$ to predict, K-fold, stratified by time $t$}
\State $K = 10$
\State Use observations such that $1975 \leq t \leq 2014$ as the full 
sample, ($1976 \leq t+1 \leq 2015$)
\State The sample is randomly split into K groups, $\{S_j\}_{j}^K$, 
stratified by year, meaning that each subsets contains roughly the same 
proportions of observation in each year. The K-Fold Stratified 
splitting is implemented using 
\href{https://scikit-learn.org/stable/modules/generated/sklearn.model_selection.StratifiedKFold.html}{sklearn.model\_selection.stratifiedkfold}
(scikit-learn 0.24.2).
\For{$ 1 \leq k \leq K$}
\State the training set $T_k$ is $\bigcup\limits_{j = 1, j \neq k }^{K} 
S_{j} $
\State the validation set  $V_k$ is $S_{j = k} $
\If {evaluate FM}
\ForAll{firm-year observations in the training set $T_k$}
\State for all time period t, run cross-sectional regression for all 
firms at time t
\State $\pi_{i,t+1}  = \lambda_{0,t} + \lambda_{1,t} X_{i,t} + 
\varepsilon_{i,t}$
\State get estimated coefficients for each time period t $ 
\{\hat{\lambda}_{0,t,k} ,\hat{\lambda}_{1,t,k} \}$
\State compute the average coefficients $ \{\hat{\lambda}_{0,k} 
,\hat{\lambda}_{1,k} \}$
\EndFor
\ForAll{firm-year observations in the validation set $V_k$}
\State make prediction of profitability in the validation set $V_k$
\State $\hat{\pi}_{i,t+1}  = \hat{\lambda}_{0,k} + \hat{\lambda}_{1,k} 
X_{i,t}$
\State calculate the out-of-sample $R^2$ at the validation set $V_k$
\State $R^2_{OOS,k} = 1 - \frac{\sum (\hat{\pi}_{i,t+1}  - \pi_{i,t+1} 
)^2}{\sum (\pi_{i,t+1} -\pi_{i,t} )^2}$ where ${i,t}\in V_k$
\EndFor
\EndIf
\If {evaluate GBRT}
\ForAll{firm-year observations in the training set $T_k$}
\State Fit the following GBRT model using training set $T_k$
\State $\pi_{i,t+1}  = f(X_{i,t}) $
\State Get estimated model  $\hat{f}_k$
\EndFor
\ForAll{firm-year observations in the validation set $V_k$}
\State Use fitted model $\hat{f}_k$ to predict profitability in the 
validation set $V_k$
\State $\hat{\pi}_{i,t+1}  = \hat{f}_k(X_{i,T})$
\State calculate the out-of-sample $R^2$ at the validation set $V_k$
\State $R^2_{OOS,k} = 1 - \frac{\sum (\hat{\pi}_{i,t+1}  - \pi_{i,t+1} 
)^2}{\sum (\pi_{i,t+1} -\pi_{i,t} )^2}$ where ${i,t}\in V_k$
\EndFor
\EndIf
\State Compute the average out-of-sample $R^2$ for the model being 
evaluated
\State $R^2_{\text{K-Fold}} = \frac{1}{K}\sum_{k = 1}^{K} R^2_{OOS,k}  $
\EndFor
\EndProcedure
\end{algorithmic}
\end{algorithm}


%\clearpage
%\section{Internet Appendix: Additional Tables}
%\label{sec:appendix_tables}
%





\end{document} 