



\section*{Introduction}
Let $\mathfrak{g}$ be the symplectic Lie algebra $\mathfrak{sp}_4(\bbC)$, i.e. the simple Lie algebra of type $C_2$. 
The irreducible $\mathfrak{g}$-modules are the highest weight modules $V(\lam)$, with $\lambda$ a  dominant weight. Given an arbitrary weight $\mu$, we denote by $d_{\lam,\mu}$ the \emph{weight multiplicity}, i.e. the dimension of the weight space $ V(\lam)_\mu$.  
The weight multiplicity $d_{\lam,\mu}$ admits a $q$-analogue, known as the Kostka--Foulkes polynomial $K_{\lam,\mu}(q)$, so that $K_{\lam,\mu}(1)=d_{\lam,\mu}$. The Kostka--Foulkes polynomials have a natural representation-theoretic interpretation since  their coefficients record the dimension of the graded pieces of the Brilinski--Konstant filtration on weight spaces. Additionally, these polynomials are also (up to renormalization) special cases of affine Kazhdan--Lusztig polynomials and have positive coefficients.

The goal of this paper is to give a combinatorial interpretation for the Kostka--Foulkes polynomials $K_{\lam,\mu}(q)$. 
Finding  such a combinatorial formula amounts to finding:
\begin{enumerate}
\item a set $\calB(\lambda)_\mu$ of cardinality $d_{\lam,\mu}$ parametrizing a basis of the $\mu$-weight space $V(\lambda)_{\mu}$.
\item a combinatorial statistic $c: \calB(\lambda)_\mu \rightarrow \mathbb{Z}_{>0}$, called the \emph{charge}, such that the Kostka--Foulkes polynomial $K_{\lambda,\mu}$ is a generating function of $\op{ch}$ on $\calB(\lambda)_\mu$,
\[K_{\lambda,\mu}(q) = \underset{T \in \calB(\lambda)_\mu}{\sum} q^{c(T)}. \]
\end{enumerate}
\noindent The set $\calB(\lam)_\mu$ has many known realizations, some of which are geometric, such as Littelmann paths, others algebro-geometric, such as Mirković--Vilonen cycles, and some purely combinatorial, such as semi-standard Young tableaux in type A or Kashiwara--Nakashima tableaux for classical types. An important feature that all of these models have in common is that they are endowed with a \textit{crystal structure}, that is, for each of these models the set $\calB(\lambda)=\bigcup \calB(\lam)_\mu$ has cardinality $\op{dim}(V(\lambda))$ and can be endowed with  the structure of a normal crystal (cf. \cite{BG01,BSch17} ).




In type $A$,  the charge statistic was first described by Lascoux and \break Schützenberger in 1978  using a combinatorial procedure on tableaux called cyclage   \cite{lscharge}. In 1995, Lascoux, Leclerc and Thibon \cite{llt95} provided another formulation of the charge statistic in terms of the crystal structure on tableaux.


In a recent work by the first named author \cite{Pat}, an alternative description of the charge statistic was obtained through a more geometric approach, which involves translating the problem of finding the charge onto the affine Grassmannian, where it becomes a variation problem for the hyperbolic localization functor.
 This geometric approach makes the problem of finding a charge statistic more accessible even beyond type $A$ (to this day this remains a mostly  open problem, except that in some special situations, such as row tableaux in type $C$ \cite{DGT} or in weight $0$ \cite{LL20}).  In fact, in the present paper we develop a similar strategy to construct a charge statistic in type $C_2$. We believe that this strategy can be further extended to cover groups of higher ranks. 
 
\subsection*{Charges via the affine Grassmannian}

 We now briefly recall the results in \cite{Pat}, at the heart of which lies the geometric Satake correspondence. Recall that the affine Grassmannian associated to $G^{\vee}$ is endowed with an action of the extended torus $ \hat{T} = T^{\vee} \times \mathbb{C}^{*}$. For $\lam\in X_+$ let $\bar{\Gr^\lam}$ denote the corresponding Schubert variety in the affine Grassmannian of $G^\vee$ (cf. \cite[\S 2.1.2.]{Pat}). For any regular $\eta \in \affX$ and any $\mu \leq \lam$ the hyperbolic localization induces a functor 
\[ \HL^\eta_\mu: \calD^b_{T^\vee \times \bbC^*}(\bar{\Gr^\lam})\raw \calD^b(pt)\cong \mathrm{Vect}^\bbZ,\]
where $\calD^b_{T^\vee \times \bbC^*}(\bar{\mathcal{G}r}^\lam)$ is the derived category of $T^\vee \times \bbC^*$-equivariant constructible sheaves on the Schubert variety $\bar{\Gr^\lam}$ with $\bbQ$-coefficients, and $\calD^b(pt)$ is the derived category of sheaves on a point, which is equivalent to the category of graded $\bbQ$-vector spaces (see \cite[\S 2.4]{Pat}).
In general, for any regular $\eta\in \affX_\bbQ$ we can define $\HL^\eta_\mu$ as $\HL^{N\eta}_\mu$, where $N$ is any positive integer such that $N\eta\in \affX$. By abuse of terminology, we are then allowed to refer to all the elements in $\affX_\bbQ$ as cocharacters.
 
 
 If $\eta$ is a dominant $T^{\vee}$ cocharacter, then the hyperbolic localization functors are the \textit{weight functors}, which send the intersection cohomology sheaf $IC_{\lambda}$ to the weight space $V(\lambda)_{\mu}$ of the simple module $V(\lambda)$. In this case, as in \cite{Pat},  we say that $\eta$ is in the \textit{MV region}, where $MV$ is short for Mirkovi\'c--Vilonen. 
 If $\eta$ is $\hat{T}$-dominant, that is, dominant for the affine root system, then the hyperbolic localization functors return graded vector spaces whose graded dimensions are renormalized Kostka--Foulkes polynomials. In this case, we say that $\eta$ is in the \textit{KL region}, where $KL$ is short for Kazhdan--Lusztig. 

Let $\htil^\eta_{\mu,\lam}(v) := \grdim(\HL^\eta_\mu(IC_\lam))$. The polynomials $\htil^\eta_{\mu,\lam}(v)$ are called \emph{renormalized $\eta$-Kazhdan--Lusztig polynomials}. We say that a function $r_{\eta}:\calB(\lam)\raw \bbZ$ is a $\eta$-\emph{recharge} for $\eta$ if we have
\[\htil^\eta_{\mu,\lam}(q^{\frac12})=\sum_{T\in \calB(\lam)_\mu} q^{r_{\eta}(T)}\in \bbZ[q^{\frac12},q^{-\frac12}].\]
If $\eta_{KL}$ is in the KL chamber and $\mu \in X_+$, then
\[K_{\mu,\lam}(q)=\htil^{\eta_{KL}}_{\mu,\lam}(q^{\frac12})q^{\frac12 \ell(\mu)}\] is a Koskta--Foulkes polynomial by \cite[Proposition 2.14]{Pat}. So if $r_{KL}$ is a recharge for $\eta_{KL}$  in the KL region, we obtain a charge statistic $c:\calB(\lam)\raw \bbZ$ by setting $c(T):=r_{KL}(T)+\frac 12 \ell(\wt(T))$. 
Notice that if $\wt(T)\in X_+$ this is equal to $c(T)=r_{KL}(T)+\langle \wt(T),\rho^\vee\rangle$.\\
 
 
 It turns out that the only situation where the hyperbolic localization functors change their value is when they ``cross'' a hyperplane of the form 
 
 \begin{align*}
 H_{\alpha^{\vee}} = \left\{\eta \in X_{*}( \hat{T}) | \langle \eta,\alpha^{\vee}\rangle = 0 \right\}
 \end{align*}
 
 \noindent
 where $\alpha^{\vee}$ is a positive real root in the root system corresponding to the Langlands dual group $G^{\vee}$. In \cite{Pat}, the first named author has observed a simple rule to compute the hyperbolic localization functor after crossing such a wall. Let $\eta_{1}$ and $\eta_{2}$ be two cocharacters on opposite sides of such a wall $H_{\alpha^{\vee}}$. Then by \cite[Proposition 2.33]{Pat} we have, for $\nu = s_{\alpha^{\vee}}(\mu)$ such that $\mu < \nu \leq \lambda$:
 
 
 \begin{align*}
\htil^{\eta_2}_{\nu,\lam}(v) &= v^{-2}\htil^{\eta_1}_{\nu,\lam}(v) \hbox{ and }\\
\htil^{\eta_2}_{\mu,\lam}(v) &= \htil^{\eta_1}_{\mu,\lam}(v) + (1 - v^{-2})\htil^{\eta_1}_{\nu,\lam}(v).
 \end{align*}



 
 
 
 In order to efficiently track these changes, \textit{swapping functions} $\phi : \calB(\lambda)_{\mu} \rightarrow \calB(\lambda)_{s_{\alpha^{\vee}}(\mu)}$ are constructed with the property that $r_{\eta_1}(T) - 1 = r_{\eta_1}(\phi (T))$. This allows a definition of a recharge statistic $r_{\eta_2}$ for $\eta_2$ given a recharge statistic $r_{\eta_1}$ for $\eta_1$. To do this, \textit{modified root operators} $e_{\alpha}, f_{\alpha}$ are constructed for any positive root $\alpha \in \Phi$. Combining this with the \textit{atomic decomposition} of the crystals $B(\lambda)$ in type $A_{n-1}$ given by Lecouvey--Lenart \cite{LL21}, which are obtained independently in \cite{Pat}, it is shown that the charge statistic giving the Kostka--Foulkes polynomials in type $A_{n-1}$ is given by the sum $\sum_{\alpha \in \Phi^{+}} \epsilon_{\alpha}(b)$.

  %, where $\varepsilon_{\alpha}$ is given by formula (\ref{normal}). 

 \subsection*{Results}

 Our main results consist of the atomic decomposition of the type $C_2$ crystals $\mathcal{B}(\lambda)$, as well as the construction of swapping functions. As a result we obtain the following formula for the  charge statistic in type $C_2$
 \begin{align*}
     c: \calB(\lam)_+ &\raw \bbN
     T\mapsto \eps_1(T)+\eps_2(T)+\eps_{12}(T)+\affeps_{21}(T)
 \end{align*}
 where $\affeps_{21}$ is not attached to a modified crystal operator, but rather depends on the atom in which $T$ sits.
This yields a positive combinatorial formula for the Kostka-- Foulkes polynomials. We outline our methodology below. 

 \subsubsection*{Atomic decompositions and charge statistics}
In \cite{Pat}, the first named author has shown that the LL atoms \cite{LL21} coincide with the connected components of the graph with same vertices as $\calB(\lambda)$, given by the $W$-closure of the $f_n$-orbits. This is one of the first constraints which appears when considering type $C_2$ crystals: the $W, f_2$ connected components are not atoms (cf. \Cref{def:atom}). This calls for an alternative approach. As in \cite{Pat}, the language of adapted strings will be an important tool for us. We first define an embedding of crystals (cf. \Cref{emb})

\[\Phi: \calB(\lambda) \rightarrow \calB(\lambda + 2\varpi_1).\]

We call the complement of $\Phi$ in $\calB(\lambda + 2\varpi_1)$ the \textit{principal preatom} $\calP(\lambda+2\varpi_1)$. If $\lambda = \lambda_1 \varpi_1 + \lambda_2 \varpi_2$ is such that $\lambda_1 \leq 1$, we define $\calP(\lambda): = \calB(\lambda)$.
The map $\Phi$ has an easy definition using the combinatorics of Kashiwara--Nakashima tableaux which allows to prove its properties directly, however, its reformulation in terms of adapted strings allows us to give equations describing the principal preatoms $\calP(\lambda)$, which we use throughout this work. A \textit{preatomic decomposition} of our crystal can be defined recursively. We show that the preatoms are stable under the $W$ and $f_2$ action, hence naturally generalize the LL atoms. Once the preatomic decomposition of our crystal has been defined, we are ready to define its atomic decomposition. In \Cref{atomemb} we show that there exists a weight-preserving injection

\[ \bPsi: \calP(\lambda) \rightarrow \calP(\lambda + \varpi_2) \]

\noindent
such that the sets $\calA(\lambda)$ given by the complement of $\bPsi$ for $\lambda_1 \neq 0$, respectively by the complement of $\bPsi^2$ for $\lambda_1 = 0$, are atoms. The map $\bPsi$ is defined explicitly on the string parameters for the reduced expression of the longest Weyl group element given by $s_2s_1s_2s_1$. An explicit description in terms of Kashiwara--Nakashima tableaux is provided in the appendix, although we do not need tableaux combinatorics in this paper. To show that the sets $\calA(\lambda)$ are  atoms, we resort to algebraic computations directly in the Hecke algebra. In particular, we make use of pre-canonical bases, introduced by Libedinsky--Patimo--Plaza in \cite{LPP}. In analogy to the Satake isomorphism, which in particular identifies the ungraded character of the character of $\calB(\lambda)$ with the specialization at $v =1$ of the corresponding element of the Kazhdan--Lusztig basis of the spherical Hecke algebra, in \Cref{PreatomPrecan} it is shown that the ungraded character corresponds to the specialization at $v =1$ of a modification $\tilN^{3}$ of the precanonical basis $\bfN^3$ introduced in (\ref{modifiedprecan3}).\\

In fact, the atomic and preatomic decompositions alone are already enough to define our charge statistic in type $C_2$. Let $T \in \calB(\lambda)$. We define in \Cref{preatomicnumber,defatomicnumber} the \emph{atomic number} $\at(T)$ and the \emph{preatomic number} $\pat(T)$ to be the positive integers such that 

\[T\in \calA(\lam-\at(T)\varpi_2-2\pat(T) \varpi_1)\subset \calP(\lam-2\varpi_1(T))\subset \calB(\lam).\]

A consequence of our main result reads as follows (cf. \Cref{maincharge}).

\begin{theorem*}
The function
$c:\calB(\lam)_+\raw \bbZ$ defined as
\[c(T)=\langle \lambda -\wt(T),\rho^\vee\rangle -\at(T)-\pat(T)\]
is a charge statistic.
\end{theorem*}

Indeed, our main result \Cref{maintheorem} consists in the construction of a recharge statistic $r_{\eta_i}$ for each $\eta_i$ in a family of cocharacters defined in \Cref{family} which goes between the KL and MV regions. In order to construct such recharge statistics, we need first to carefully study the geometry of atoms in type $C_2$.

\subsubsection*{Twisted Bruhat graphs and non-swappable staircases}
We consider \textit{twisted Bruhat graphs} associated to a fixed infinite reduced decomposition $y_{\infty}$ in the affine Weyl group, as in \cite{Pat}. For any $m \in \mathbb{Z}_{>0}$, let $y_m$ be the product of the first $m$ elements of $y_{\infty}$ and let $N(y_m)$ be its set of inversions. The idea is to start off by considering the Bruhat graph $\Gamma_\lambda$ of a given dominant integral weight $\lambda$, that is, the moment graph of the Schubert variety $\sch{\lambda}$. The vertices of the graph $\Gamma_\lambda$ are all the weights lesser than or equal to  $\lam$ in the dominance order. We have an edge $\mu_1\raw \mu_2$ in $\Gamma_\lambda$ if and only if $\mu_2-\mu_1$ is a multiple of a root and $\mu_1\leq \mu_2$. From $\Gamma_{\lambda}$ we obtain our twisted Bruhat graph $\Gamma^{m}_{\lambda}$ by inverting the orientation of all the arrows in $\Gamma_\lambda$ with label in $N(y_m)$. For $\mu \leq \lambda$, let $\Arr_m(\mu,\lam)$ be the set of arrows pointing to $\mu$ in $\Gamma_m^\lambda$ and by $\ell_m(\mu,\lambda):=|\Arr_m(\mu,\lam)|$ the number of those arrows (cf. \Cref{twistedarrowslambdamu}). Let $t_{m+1}:=y_{m+1}y_m^{-1}$. If $\mu < t_{m+1}\mu$ then surprisingly, for  the twisted Bruhat graphs in type $A$ (\cite[Prop. 4.14]{Pat}) the following holds: $\ell_m(\mu,\lambda)=\ell_m(t_{m+1}\mu,\lambda)-1$ if $\mu <t_{m+1}\mu \leq \lambda$. This implies that
$\ell_{m+1}(\mu,\lambda)=\ell_m(t_{m+1}\mu,\lambda)$. However, as we show in \Cref{exampleswap}, this property does not hold in type $C_2$. In \Cref{def:swappableedge} we define an edge $\mu \raw t_{m+1}\mu$ in $\Gamma_{\lambda}$ to be \emph{swappable} if and only if 
\[
 \ell_m(\mu,\lambda)= \ell_m(t_{m+1}\mu,\lambda)-1.
\]

The whole of Section 4 is dedicated to their classification. We  pay particular attention to non-swappable edges. In \Cref{nonswappablenumber} we define the number of non-swappable edges in the following sense:
\[\calN_m(\mu,\lambda):=|\{k \leq m \mid  \mu<t_k\mu \leq \lambda \text{ and }\mu\raw t_k\mu\text{ is not swappable}\}|.\]


An important property of non-swappable edges is that they will always ``be swappable'' in an atom isomorphic to $\mathcal{A}(\lambda - k\varpi_2)$ for large enough $k$. This leads to the notion of non-swappable staircases (cf. \Cref{def:nonswappable}). Essentially, a non-swappable staircase  over $(\mu,\lambda)$ consists of a sequence of edges of the form $e_i:=(\mu \raw \mu - (n+i)\alpha)$ such that $e_i$ is non-swappable in $\calA(\lambda + i \varpi_2)$. We define  $\affD_m(\mu,\lam)$ to be the length of the longest NS-staircase over $(\mu,\lam)$ where the label of every edge in $e_i$ is a root in $N(y_m)$. Moreover, in \Cref{truncatedns} we define the following statistic, which considers only NS-staircases lying in a single preatom:
\[ \calD_m(\mu,\lam,k):=\min(k,\affD_m(\mu,\lam-k\varpi_2)).\]

We are now ready to define the recharge statistics $r_{\eta_m}$, which we define in \Cref{N=0}. For $T\in  \calP(\lam)\subset \calB(\lam')$ with $\mu:=\wt(T)$.  We define
\[ r_m(T):= -\ell_m(\mu,\lama)+\calN_m(\mu,\lama)-\calD_m(\mu,\lambda,a)-\at(T)-2\pat(T)+\langle \lam',\rho^\vee\rangle.\]
Our main result, from which descends our explicit formula for the charge statistic in type $C_2$, is the following (cf. \Cref{maintheorem}).

\begin{theorem*}
The function $r_m:\calB(\lambda)\raw \bbZ$ is a recharge statistic for $\eta_m$ for any $m\in \bbN\cup \{\infty\}$.
\end{theorem*}

To prove our main theorem, we need to construct swapping functions. 

\subsubsection*{Swapping functions}

The existence of non-swappable edges in type $C_2$ means that we cannot define swapping functions within a single atom as in type $A_n$. In 
 \Cref{sec:swapping} the swapping functions we construct involve two elements from two different atoms within the same preatom. In order to determine which are the two atoms involved we need to introduce a new quantity, which we call the elevation $\Omega(e)$ of an edge $e$ that measures the height of the maximal staircases of non-swappable edges lying underneath it. For any $\mu\in X$ such that $\mu<t\mu\leq \lambda$ we define the swapping functions 
\[ \psi_{t\mu}:\calB(\lambda)_{t\mu}\ra 
\calB(\lambda)_{\mu}\]
as follows. Let $T\in \calB(\lam)_{t\mu}$ and assume that $T\in \calA(\lam-a\varpi_2)\subset \calP(\lam)$. Let $e:=(\mu \raw t\mu)\in E(\lambda-a\varpi_2)$. Then $\psi_{t\mu}(T)=T'$, where $T'$ is the only element of weight $\mu$ in $\calA(\lam-(a+\Omega(e))\varpi_2)\subset \calP(\lam)$. To prove \Cref{maintheorem} we show in \Cref{swapcheck} that 
\[r_{m+1}(T)=r_{m+1}(\psi_{t\mu}(T))+1.\]

In the proof we use many results on non-swappable staircases and non-swappable edges obtained in Section 4. 

\subsubsection*{Alternative formula}
In Section 6, we obtain an alternative formula for the charge statistic by focusing on a single element and counting how many times its recharge gets changed by a swapping function. The formula we obtain is in terms of the modified crystal operators, which we define in \Cref{def:modified}. 

Let $T\in \calA(\zeta)$ be such that  $\wt(T)=\mu$.
Let $\affeps_{21}(T)$ be the maximum integer such that 
$\mu+k\alpha_i\leq \zeta$. In Section 6 we show that 

\[ c(T)=\eps_1(T)+\eps_2(T)+\eps_{12}(T)+\affeps_{21}(T)\]
is a charge statistic on $\calB_+(\lam)$. Finally, we conjecture a formula for a charge statistic in type $C_3$, which is a natural generalization of our formula.  

\section*{Acknowledgements}
J.T. was supported by the grant UMO-2021/43/D/ST1/02290
 and partially supported by the grant UMO-2019/34/A/ST1/00263. 