\section{The charge and recharge statistics}\label{sec:swapping}

\subsection{A family of cocharacters}\label{sec:family}

We recall some definitions from \cite{Pat}.
Let $\affX=X\oplus \bbZ d$ be the cocharacter lattice of $T^\vee \times \bbC^*$, where $T^\vee$ is the maximal torus of $G^\vee$.
Let $\affX_\bbQ  := \affX \otimes_\bbZ \bbQ$ and $\affX_\bbR := \affX \otimes_\bbZ \bbR$.

 The \emph{KL region} is the subset of $\affX_\bbQ$ of the elements $\eta$ such that $\langle \alpha^\vee,\eta\rangle >0$ for all $\alpha^\vee \in \affPhi_+^\vee$. Concretely, an element in the KL region can be written as $\eta=\lambda+Cd$ where $\lambda\in X_{++}$ and $C>\langle \lambda,\beta^\vee\rangle$ for all $\beta^\vee \in \Phi^\vee_+$.
 The \emph{MV region} is the subset of $\affX_\bbQ$ consisting of elements of the form $\eta=\lambda+Cd$, with $\lambda\in X_{++}$ and $C=0$.

We call \emph{wall} a hyperplane in $\affX_\bbR$ of the form 
\[H_{\alpha^\vee}:=\{ \eta \in \affX_\bbR \mid \langle \eta,\alpha^\vee \rangle =0\}\subset \affX_\bbR\]
for $\alpha^\vee\in \affPhi^\vee$.
	For $\lambda\in X_+$, we denote by $\affPhi^\vee(\lambda)$ the set of all the labels present in the graph $\Gamma_\lambda$. We say that a wall $H_{\alpha^\vee}$ is a $\lambda$-wall if $\alpha^\vee\in \affPhi^\vee(\lambda)$.
We call \emph{$\lambda$-chamber} (or simply chamber, if $\lambda$ is clear from the context)  the intersection of $\affX_\bbQ$ with a connected component of 
\[\affX_\bbR \setminus \bigcup_{\alpha^\vee\in \affPhi^\vee(\lambda)}H_{\alpha^\vee}.\]
Two chambers are \emph{adjacent} if they are separated by a single $\lambda$-wall. 
The \emph{KL chamber} is the unique chamber containing the KL region and the  \emph{MV chamber} is the unique chamber containing the MV region.
We say that $\lambda\in X_\bbQ$ is \emph{regular} if it does not lie on any wall, and it is \emph{singular} otherwise.

For $\lam\in X_+$ let $\bar{\Gr^\lam}$ denote the corresponding Schubert variety in the affine Grassmannian of $G^\vee$ (cf. \cite[\S 2.1.2.]{Pat}). For any regular $\eta \in \affX$ and any $\mu \leq \lam$ the hyperbolic localization induces a functor 
\[ \HL^\eta_\mu: \calD^b_{T^\vee \times \bbC^*}(\bar{\Gr^\lam})\raw \calD^b(pt)\cong \mathrm{Vect}^\bbZ,\]
where $\calD^b_{T^\vee \times \bbC^*}(\bar{\mathcal{G}r}^\lam)$ is the derived category of $T^\vee \times \bbC^*$-equivariant constructible sheaves on the Schubert variety $\bar{\Gr^\lam}$ with $\bbQ$-coefficients, and $\calD^b(pt)$ is the derived category of sheaves on a point, which is equivalent to the category of graded $\bbQ$-vector spaces (see \cite[\S 2.4]{Pat}).
In general, for any regular $\eta\in \affX_\bbQ$ we can define $\HL^\eta_\mu$ as $\HL^{N\eta}_\mu$, where $N$ is any positive integer such that $N\eta\in \affX$. By abuse of terminology, we are then allowed to refer to all the elements in $\affX_\bbQ$ as cocharacters.

Let $\htil^\eta_{\mu,\lam}(v) := \grdim(\HL^\eta_\mu(IC_\lam))$, where $IC_\lam$ denotes the intersection cohomology sheaf on $\bar{\Gr^\lam}$. The polynomials $\htil^\eta_{\mu,\lam}(v)$ are called \emph{renormalized $\eta$-Kazhdan--Lusztig polynomials}. We say that a function $r:\calB(\lam)\raw \bbZ$ is a $\eta$-\emph{recharge} for $\eta$ if we have
\[\htil^\eta_{\mu,\lam}(q^{\frac12})=\sum_{T\in \calB(\lam)_\mu} q^{r(T)}\in \bbZ[q^{\frac12},q^{-\frac12}].\]
If $\eta_{KL}$ is in the KL chamber and $\mu \in X_+$, then
\[K_{\mu,\lam}(q)=\htil^{\eta_{KL}}_{\mu,\lam}(q^{\frac12})q^{\frac12 \ell(\mu)}\] is a Koskta--Foulkes polynomial by \cite[Proposition 2.14]{Pat}. So if $r_{KL}$ is a recharge for $\eta_{KL}$  in the KL region, we obtain a charge statistic $c:\calB(\lam)\raw \bbZ$ by setting $c(T):=r_{KL}(T)+\frac 12 \ell(\wt(T))$. 
Notice that if $\wt(T)\in X_+$ this is equal to $c(T)=r_{KL}(T)+\langle \wt(T),\rho^\vee\rangle$.


We specialize \cite[Definition 3.29]{Pat} to our setting. 
%\Jaz{There's no Section 6 in Pat new arxiv version. Are you referring to the published version?}\Leo{Corrected}

\begin{definition}
Let $\lambda\in X_+$. We call \emph{$\lambda$-parabolic region} the subset of $\affX_\bbQ$ consisting of regular cocharacters $\eta$ such that	\begin{itemize}
		\item $\langle \eta,\beta^\vee\rangle>0$ for every $\beta^\vee$ of the form 
		$M\delta -\alpha_1^\vee$ with $M>0$, or of the form $M\delta+\alpha^\vee$, with $\alpha\in \Phi_+$ and $M\geq 0$.
		\item $\langle \eta,\beta^\vee\rangle<0$ for every $\beta^\vee\in \affPhi^\vee_+(\lambda)$ of the form $M\delta-\alpha_i^\vee$ such that $M> 0$ and $i\in \{2,12,21\}$.
	\end{itemize} 
\end{definition} 


The walls that separate the  parabolic region from the KL region are precisely
\[ H_{M\delta-\alpha_i^\vee} \qquad \text{with }M>0\text{ and }i\in \{2,12,21\}.\]
Every cocharacter $\eta_P$ of the form
\[ \eta_P= A_1\varpi_1 + A_2\varpi_2 + Cd\]
with $0\ll A_1\ll C\ll A_2$ lies in the parabolic region.\footnote{More precisely, sufficient conditions are $0<A_1<C<\frac{A_2}{\gamma}$ where $\gamma=\max\{ M \mid M\delta-\beta^\vee \in \Phi^\vee(\lam)\}$.}



We consider the following family of cocharacters:
\begin{equation}\label{family}
	\eta:\bbQ_{\geq 0}\ra \affX_\bbQ,\qquad \eta(t)=\eta_P+t d.
\end{equation} 
Observe that $\eta(t)$ is in the KL chamber for $t\gg 0$. We can choose $t_0$ such that $\eta(t_0)$ is in the KL chamber and for any $i$ we choose $t_{i+1}<t_{i}$ so that $\eta(t_i)$ and $\eta(t_{i+1})$ lie in adjacent $\lambda$-chambers until we arrive at $t_M$ in the parabolic region. We can furthermore choose  $t_M=0$ and set $t_{M+1}=\ldots =t_{\infty}=0$ and $\eta_i:=\eta(t_i)$ for any $i\in \bbN \cup \{\infty\}$.

\subsection{Recharge statistics from the parabolic to the KL region}



Our goal is to attach a recharge statistic to each of the cocharacters $\eta_i$. 

Let $T\in \calB(\lambda)$. Recall that by \Cref{preatomicnumber,defatomicnumber} we have 
\[T\in \calA(\lam-\at(T)\varpi_2-2\pat(T) \varpi_1)\subset \calP(\lam-2\varpi_1(T))\subset \calB(\lam).\]


\begin{definition}\label{N=0}
Assume that $T\in  \calP(\lam)\subset \calB(\lam')$ with $\mu:=\wt(T)$. Let $a:=\at(T)$ and $p:=\pat(T)$ so that $\lam'=\lam+2p\varpi_1$. We define
\[ \sigma_m(T):= \ell_m(\mu,\lama)-\calN_m(\mu,\lama)+\calD_m(\mu,\lambda,a)+a+2p.\]
Let $r_m(T):=-\sigma_m(T)+\langle \lam',\rho^\vee\rangle=-\sigma_m(T)+\langle \lam,\rho^\vee\rangle+3p$.

\end{definition}

Our main result is the following.

\begin{theorem}
\label{maintheorem}
The function $r_m:\calB(\lambda)\raw \bbZ$ is a recharge statistic for $\eta_m$ for any $m\in \bbN\cup \{\infty\}$.
\end{theorem}

The proof that $r_i$ is a recharge for $\eta_i$ is divided in two parts. We first show directly in \Cref{sec:parabolic} that $r_\infty$ is a recharge statistic for $\eta_\infty=\eta(0)$, i.e. a recharge in the parabolic region, and then we construct for any $i$ swapping functions between $\eta_i$ and $\eta_{i+1}$. 
After putting everything together, this proves that $r_{KL}:=r_0$ is a recharge in the KL region, and we can easily obtain from that the following formula for a charge statistic in type $C_2$.

\begin{corollary}
\label{maincharge}
The function
$c:\calB(\lam)_+\raw \bbZ$ defined as
\[c(T)=\langle \lambda -\wt(T),\rho^\vee\rangle -\at(T)-\pat(T)\]
is a charge statistic.
\end{corollary}
\begin{proof}
By definition, we have $\calN_0=\calD_0=0$ and $\ell_0=\ell$. Hence
\[ c(T)=r_0(T)+\frac12\ell(\wt(T))=\langle \lambda,\rho^\vee\rangle -\frac12\ell(\wt(T))-\at(T)-2\pat(T)\]
is a charge statistic. We conclude since, for $T=\calB(\lam)_+$, we have $\ell(\wt(T))=2\langle \wt(T),\rho^\vee\rangle$.
\end{proof}




\subsection{Recharge in the parabolic region}\label{sec:parabolic}

Let $\eta_{MV}$ be a cocharacter in the MV region and $\eta_P$ be in the parabolic region. The only walls separating $\eta_{MV}$ from $\eta_P$ are of the form $H_{M\delta-\alpha_1^\vee}$, with $M>0$. We now from  \cite[Eq. (17)]{Pat} that
\[r_{MV}(T) = -\langle \rho^\vee, \wt(T)\rangle.\] 
is a recharge in the MV region. To construct a recharge in the parabolic region, after Levi branching, we can assume we are in rank $1$ and thus compute the recharge as illustrated in \cite[\S 3.4]{Pat}. In particular, it follows from \cite[Lemma 3.26]{Pat} that 
\[ r_P(T)=-\langle \rho^\vee,\wt(T)\rangle+\phi_1(T)-\ell^1(\wt(T))\]
is a recharge in the parabolic region. It remains to show the equality between $r_P$ and $r_{\infty}$.


% \begin{definition}
%     Assume $T\in \calA(\lam)$. For $i \in \{2,12,21\}$ let $\affphi_i(T):=\affphi_i(\wt(T),\lam)$.
% \end{definition}

%Let $T\in \calA(\lam-k\varpi_2)\subset \calP(\lam)\subset \calB(\lam')$. 
Let $T\in \calP(\lam)\subset \calB(\lam+2p\varpi)$ with $p=\pat(T)$ and let $a=\at(T)$ and $\mu=\wt(T)$.
At $m=\infty$, we have 
\begin{align}\label{sigmainf}
 \sigma_\infty(T) = \ell^1(\mu)+  \sum_{i\in \{2,12,21\}}\affphi_i(\mu,\lama) \nonumber \\
 -\calN_\infty(\mu,\lama)+\calD_\infty(\mu,\lam,a)+a+2p.
 %&=\ell^1(\mu)+\sum_{i\in \{2,12,21\}}\affphi_i(T)-\calN_\infty(\mu,\lama)+\calD_\infty(\mu,\lam,a)+a+2p
\end{align}

Our next goal is to simplify the expression \eqref{sigmainf}.

\begin{lemma}\label{phi21k}
We have $\affphi_{21}(\mu,\lama)+a=\affphi_{21}(\mu,\lambda)$.
\end{lemma}
\begin{proof}
    This follows directly from \Cref{affphicompute}.
\end{proof}


\begin{proposition}\label{phi12hard}
Let $\mu = \wt(T)$ and assume that $\mu_1\leq  0$. We have $\phi_2(T)=\affphi_2(\mu,\lama)$ and
\[\phi_{12}(T)=\affphi_{12}(\mu,\lama)-\calN_\infty(\mu,\lama)+\calD_\infty(\mu,\lambda,a).\]
\end{proposition}
The proof of \Cref{phi12hard} is rather long and technical and we postpone it to \Cref{sec:phi2}.

\begin{lemma}\label{sigmainfnicelemma}
Let $\mu=\wt(T)$. We have 
\begin{equation}\label{sigmainfnice}\sigma_\infty(T)=\ell^1(\mu)+\phi_2(T)+\phi_{12}(T)+\affphi_{21}(\mu,\lambda)+2p.
\end{equation}
\end{lemma}
\begin{proof}
If $\mu_1\leq 0$, this follows immediately from \Cref{phi21k} and \Cref{phi12hard}.

If $\mu_1>0$, then let $T'=s_1(T)$. Recall that atoms are stable under $s_1$ by \Cref{lemmaonPsi}. So the element $T'$ can also be characterized as the element in the same atom of $T$ with weight $s_1(\mu)$. %\Leo{Argh, there is maybe a  small problem in the definition of $\at$. We want them to be stable under $s_1$ so the definition we gave only work for $\mu_1\geq 0$. (Or $\mu_1\leq 0$?)}
Notice that $\affphi_{21}$, $\calN_\infty$ and $\calD_\infty$ are preserved by $s_1$, while  $\affphi_{2}(\mu,\lama)=\affphi_{12}(s_1(\mu),\lama)$ and $\ell^1(\mu)=\ell^1(s_1(\mu))- 1$. It follows that $\sigma_\infty(T)=\sigma_\infty(T')-1$. On the other hand, we also have $\phi_2(T)=\phi_{12}(T')$ and $\phi_{12}(T')=\phi_2(T)$, so we obtain the desired identity \eqref{sigmainfnice} for $T$ as well.
\end{proof}

\begin{proposition}\label{rP}
We have $r_P(T)=r_\infty(T)$ for any $T \in \calB(\lambda')$.
\end{proposition}
\begin{proof}
Let $\mu=\wt(T)$ and assume $T\in \calP(\lam)\subset \calB(\lam+2p\varpi_1)$. By \Cref{sigmainfnicelemma} we have
\[r_\infty(T)=-\ell^1(\mu)-\phi_2(T)-\phi_{12}(T)-\affphi_{21}(\mu,\lam)+\langle \lam,\rho^\vee\rangle+p.\]
So our claim is equivalent to
\[ \langle \lambda+\mu,\rho^\vee \rangle -\affphi_{21}(\mu,\lam)=\phi_1(T)+\phi_2(T)+\phi_{12}(T)-p=Z(T)-p.\]
By  \Cref{atomicnumber} and \Cref{affphicompute} we have
\begin{align*}\langle \mu+\lam,\rho\rangle -\affphi_{21}(\mu,\lam)%&=\frac32(\lam_1+\mu_1)+\lam_2+\mu_2-\min(\lam_1,\frac{\lam_1+\mu_1}{2},\lam_1+\mu_1)\\
		&= \lam_2+\mu_2+\frac32\lam_1+\frac32\mu_1-\min(\lam_1,\frac{\lam_1+\mu_1}{2},\lam_1+\mu_1)\\
		&= \lam_2+\mu_2+\lam_1+\mu_1-\min(\frac{\lam_1-\mu_1}{2},0,\frac{\lam_1+\mu_1}{2})\\
  & =Z(T) - p.\qedhere
\end{align*}
\end{proof}


\subsection{Computing \texorpdfstring{$\phi_2$}{psi}}
\label{sec:phi2}
It remains to prove the identity \Cref{phi12hard}.

We begin by considering the case $\at(T)=0$. The general case will follow by induction on the atomic number.





\begin{proposition}\label{atomicphi2}
    For any $T \in \calP(\lam)$ with $\wt(T)_1\leq 0$ we have $\phi_2(T)=\affphi_2(\wt(T),\lam-\at(T)\varpi_2)$.
\end{proposition}

\begin{proof}
Let $\mu \leq \lam$. Consider the multiset
\[ M_2(\mu,\lam):=\{ \phi_2(X) \mid X \in \calP(\lambda)\text{ with }\wt(X)=\mu\}\]
Since $\calP(\lambda)$ is a union of $f_2$-strings, we have an equality of multisets
\begin{equation}\label{M2alt}M_2(\mu,\lam)=\{\affphi(\mu,\lamk) \mid 0\leq k\leq \lam_2 \text{ with }\mu \leq \lamk\}.
\end{equation}
In fact, the $f_2$-strings contained in $\calP(\lambda)$ which pass through an element of weight $\mu$ are in bijection with the atoms in $\calP(\lambda)$ containing an element of weight $\mu$.

The claim now follows by induction on $\lam_2$. If $\lam_2=0$, then $\calP(\lambda)=\calA(\lambda)$ and $M_2(\mu,\lambda)=\{\phi_2(T)\} =\{\affphi_{2}(\mu,\lam)\}$.

If $\lam_2>0$, consider the embedding $\Psi:\calP(\lam-\varpi_2)\hookrightarrow \calP(\lam)$ from \Cref{defPsi}. The map $\Psi$ is weight preserving and we have  $\phi_2(\Psi(X))=\phi_2(X)$ and $\at(\Psi(X))=\at(X)+1$ for any $X\in \calP(\lam-\varpi_2)$ with $\wt(X)_1\leq 0$. 
If $T= \psi(X)$ for some $X\in \calP(\lam-\varpi_2)$, then $\phi_2(T)=\phi_2(X)=\affphi_2(\mu,\lam-\varpi_2-\at(X)\varpi_2)$ and the claim follows. Otherwise, we have $T\in \calA(\lam)=\calP(\lam)\setminus \Psi(\calP(\lam-\varpi_2))$
and by \eqref{M2alt} we see that
\[ \{\phi_2(T)\}=M_2(\mu,\lam)\setminus M_2(\mu,\lam-\varpi_2)=\{ \affphi_2(\mu,\lam)\}.\qedhere\]
\end{proof}


\begin{lemma}\label{phi1at0}
Let $T\in \calP(\lam)\subset \calB(\lam)$ and let $\mu=\wt(T)$. Assume $\mu_1< 0$ and $\at(T)=0$. Then we have $
    \phi_1(T) = \max(0,\mu_1+\mu_2-\lam_2,-\mu_2-\lam_2).$
\end{lemma}
% \begin{align}
% \phi_1(T) &= \max(0,\mu_1+\mu_2-\lam_2,-\mu_2-\lam_2),\label{claim0}\\ \phi_{12}(T) &= \affphi_{12}(\mu,\lam)-\calN_{\infty}(\mu,\lam).\label{claim12}
% \end{align}

\begin{proof}

%It is enough to prove the claims for $\pat(T)=0$. Otherwise $T=\Phi(T')$ and by induction we have $\phi_1(T)=\phi_1(T')+1$ and $\phi_{12}(T)=\phi_{12}(T')$.

Let $\stn(T)=(a,b,c,d)$.
By \Cref{atom0} we have 
\[\at(T)=0\iff (c=d=0) \text{ or }(b=\lam_1+2c-2d\text{ and }d\leq 1\text{ or }c= \lam_2+d)\]

As computed in \eqref{phi1stn}, we have
\[\phi_1(T)=\lam_1+2a-2b+2c-2d+\max(d,2c-b,b-2a).\]

%so the claim is equivalent to
%\[\lam_1+2a+4c-b-d-\min(2a+2c+d,b+2c,2b+d) = \max(0,\lam_1-b-d,-2\lam_2-b-d+2a+2c).\]
We now divide into several cases. 
Assume first $c=d=0$. Then the statement is equivalent
\begin{equation}\label{eqcd0}\lam_1+2a-b-\min(2a,b) = \max(0,\lam_1-b,-2\lam_2-b+2a).\end{equation}
Since $\mu_1=\lam_1-2b+2a <0$ and $b\leq \lam_1$, we have $2a\leq b$, so the LHS in \eqref{eqcd0} is $\lam_1-b$.
Moreover, $\lam_1-b\geq 0$ and $\lam_1-b\geq 2a-b-2\lam_2$ otherwise we get.
$\lam_1+2\lam_2< 2a<b.$ So the RHS in \eqref{eqcd0} is also equal to $\lam_1-b$.

We can now assume $b=\lam_1+2c-2d$, so we have
$\phi_1(T)=\max(-\lam_1+2a-2c+3d,0)$,
while the RHS  can be rewritten as 
$\max(0,d-2c,-2\lam_2+d+2a-\lam_1)$.
Moreover, we have $d-2c\leq 0$ and $\mu_1 = -\lam_1+2a-2c+2d\leq 0$.
%and $2a-2\lam_1-2c+4d\leq 2a-2\lam_1-2c+2d+2b=-\mu_1-\lam_1\leq 0$.

So it is enough to show that 
\begin{equation}\label{eq2}\max(0,\mu_1+d)=\max(0,\mu_1-2(c-\lam_2-d)+d)\end{equation}
The equality is clear if $c=\lam_2+d$ and it also follows if $d\leq 1$ since that both term vanish for $\mu_1<0$.%\Leo{Is this really false if $\mu_1=0$ land $d=1$? Weirdly enough yes, e.g. for (5,10,2,1) and $\lam=(8,4)$. It should not be a problem.}
\end{proof}
%
% In the case $c= \lam_2+d$ the equation \eqref{eq2} becomes clearly an identity. 
% %\[\max(0,2a-\lam_1-2\lam_2+d)=\max(0,-2\lam_2+d+2a-\lam_1)\]
% We finally assume $d\leq 1$. In this case  implies $2a+2d\leq \lam_1+2c\leq  \lam_1+2\lam_2+2d$, hence $2a+3d\leq \lam_1+2c$. On the other hand
% $d+2a< \lam_1 +2\lam_2+1$, so $\lam_1+2\lam_2\geq d+ 2a$.\Leo{Need to check again these computations!}

\begin{proposition}
\label{phi12at0}
Let $T\in \calP(\lam)$ and let $\mu=\wt(T)$. Assume $\mu_1\leq  0$ and $\at(T)=0$. Then we have $\phi_{12}(T)=\affphi_{12}(\mu,\lam)-\calN_{\infty}(\mu,\lam).$
\end{proposition}
\begin{proof}
If $\mu_1=0$, then $\phi_{12}(T)=\phi_2(T)$, $\affphi_{12}(T)=\affphi_{2}(T)$ and $\calN_\infty(\mu,\lam)=0$, so the claim follows from \Cref{atomicphi2}. We assume in the rest of the proof $\mu_1<0$. We can also assume that $T$ lies in the biggest preatom, i.e. that $\calP(\lam)\subset \calB(\lam)$. In fact, since $\Phi$ commutes with $s_1$ and $f_2$, the claim for the other preatoms easily follows by induction.

Recall now by \Cref{atomicphi2} that $\phi_2(T)=\affphi_2(\mu,\lam)$. We divide into three cases.

We assume first $\mu_1+\mu_2\leq \lam_2$ and $\mu_2\geq -\lam_2$. Notice that this precisely means that $\phi_1(T)=0$. By \Cref{Ninf}, we have in this case
\[ \affphi_{12}(T)-\calN_\infty(\mu,\lam)=\max(0,\frf{\lam_2+\mu_1+\mu_2})-\min(0,\frac{-\lam_1-\mu_1}{2}).\]
Let $\chi:=\lam_2+\mu_1+\mu_2$. Then by \Cref{atomicnumber,atomicphi2,phi1at0} we have
\begin{align*} \phi_{12}(T) &=Z(T)-\phi_1(T)-\phi_2(T)\\
&=\frac{\lam_1+\mu_1}{2}+\chi +\max(0,\frac{-\mu_1-\lam_1}{2})-\min(\chi,\frf{\chi},\lam_2).
\end{align*}

Notice that $\min(0,\frac{-\lam_1-\mu_1}{2})+\max(0,\frac{-\lam_1-\mu_1}{2})=\frac{-\lam_1-\mu_1}{2}$ and that $\lam_2\geq \frf{\chi}$. So our claim results equivalent to the easy-to-check identity
\[\chi -\min(\chi,\frf{\chi})=\max(0,\frc{\chi}).\]

We now assume that $\mu_1+\mu_2> \lam_2$ or that $\mu_2<-\lam_2$. In both cases we have from \Cref{octineq} that $\mu_1>-\lam_1$, so $Z(T)=\lam_1+\lam_2+\mu_1+\mu_2$. Moreover, we have from \Cref{Ncount} that $\calN_{\infty}(\mu,\lam)=0$, so the claim is equivalent to $Z(T)-\phi_1(T)-\phi_2(T)=\affphi_{12}(T)$.

If $\mu_1+\mu_2>\lam_2$, then 
$\affphi_2(T)=\frac{\lam_1-\mu_1}{2}+\lam_2$ and $\affphi_{12}(T)=\frac{\lam_1+\mu_1}{2}+\lam_2$, so the desired equality reduces to the identity
\[ \lam_1+\lam_2+\mu_1+\mu_2-\mu_1-\mu_2+\lam_2-\frac{\lam_1}{2}+\frac{\mu_1}{2}-\lam_2=\frac{\lam_1}{2}+\frac{\mu_1}{2}+\lam_2.\]
Finally, if $\mu_2<-\lam_2$, the desired equality reduces to the identity
 \[ \lam_1+\lam_2+\mu_1+\mu_2+\mu_2+\lam_2-\frac{\lam_1}{2}+\frac{\mu_1}{2}-\mu_1-\mu_2-\lam_2=\frac{\lam_1}{2}+\frac{\mu_1}{2}+\lam_2+\mu_2.\qedhere\]
\end{proof}



\begin{proposition}\label{phi2claim}
Let $T\in \calP(\lam)$ and let $\mu=\wt(T)$. Let $A:=\at(T)$.
	If $\mu_1\leq 0$, we have
\[\phi_{12}(T)=\affphi_{12}(\mu,\lam- A\varpi_2)-\calN_\infty(\mu,\lambda-A\varpi_2)+\calD_\infty(\mu,\lambda,A).\]
\end{proposition}

\begin{proof}
As in \Cref{phi12at0} we can assume that $\calP(\lam)\subset \calB(\lam)$.
We show the claim it by induction on $A$.
	If $A=0$, the claim immediately follows from \Cref{phi1at0} since $\calD_\infty(\mu,\lam,0)=0$. 

 If $A>0$, then $T=\Psi(U)$ for some $U\in \calP(\lambda-\omega_2)\subset \calB(\lam-\omega_2)$ with $\at(U)=A-1$.
By induction, we have 
\[ \phi_{12}(U)=\affphi_{12}(\mu,\lamA)-\calN_\infty(\mu,\lamA)+\calD_{\infty}(\mu,\lam-\varpi_2,A-1).\]
 So it suffices to show that, for any $U$ in $\calP(\lam-\varpi_2)\subset \calB(\lam-\varpi_2)$ with $\wt(U)=\mu$, we have
\begin{align}\label{Psiphi12} \phi_{12}(\Psi(U))-\phi_{12}(U) &=\calD_\infty(\mu,\lambda,A)-\calD_{\infty}(\mu,\lam-\varpi_2,A-1)\\
& =\min(A,\affD_\infty(\mu,\lamA))-\min(A-1,\affD_\infty(\mu,\lamA)).\nonumber\end{align}

Let $\stn(U)=(a,b,c,d)$. We know from \Cref{cor:phi12psi} that 
\[ \phi_{12}(\Psi(U))-\phi_{12}(U)=\begin{cases}
1& \text{if }d=0\text{ and }2a>b>2c\text{ or }d\neq 0,\lam_1\text{ and }b\geq 2a+d\\
0&\text{otherwise}.
\end{cases}\]
However, notice that we cannot have $d=0$ and $2a>b>2c$ since otherwise
$\mu_1=\lam_1+2a-2b+2c> \lam_1+2c-b\geq 0$.
It follows that \eqref{Psiphi12} is equivalent to showing that
\[ \affD_\infty(\mu,\lamA)\geq A \iff d\neq 0,\lam_1 \text{ and }b\geq 2a + d.\]
We show this in the following lemma.
\end{proof}

\begin{lemma}
Let $X\in \calP(\lam)\subset \calB(\lam)$ with $\mu=\wt(X)$ such that $\mu_1< 0$. Let $A:=\at(X)$ and $\stn(X)=(a,b,c,d)$. We have 
\[\affD_\infty(\mu,\lamA)> A \iff d\neq 0,\lam_1 \text{ and }b\geq 2a + d.
\]
\end{lemma}
\begin{proof}

Recall from \Cref{Dinf<0} that we have 
\[\affD_\infty(\mu,
\lamA)=\begin{cases}
	\max(0,\min(\lam_1,-\mu_1)-1) &\begin{array}{l}\text{if }\mu_1+\mu_2+\lam_2\geq A,\\
 \mu_1+\mu_2+A\leq \lam_2 \text{ and }\\
 \mu_1+\mu_2\not\equiv\lam_2-A \pmod{2};\end{array}\\
\begin{aligned}\max(0,\min(-\mu_1,\lam_1)+\lam_2\\-A+\mu_1+\mu_2) \end{aligned}&\begin{array}{c}\text{if } \mu_1+\mu_2+\lam_2<A;\end{array}\\
0 &\begin{array}{c}\text{otherwise.}\end{array}\end{cases}\]
Moreover,  from \Cref{atomformula} we have
\[A= \at(X)=\begin{cases} \min(c,\lam_1+2c-b)& \text{if }d=0\\
\lam_1+2c-2d-b+\min(\lam_2+d-c,d-1) &\text{if }d>0.\end{cases}\]

We divide the proof into three cases.

\noindent \textbf{First case: $d=0$.\;}
 We claim that in this case we  actually have $\affD_\infty(\mu,\lamA)=0$.
Notice that $\mu_1+\mu_2+\lam_2=\lam_1+2\lam_2-b\geq \lam_1+2c-b\geq A$. So we can also assume that $A\leq \lam_2-\mu_1-\mu_2=b-\lam_1$. Notice that this is equivalent to $c+\lam_1\geq b$ and $A=\lam_1+2c-b$. However, if $A=\lam_1+2c-b$ then $\mu_1+\mu_2+\lam_2+A\equiv 0 \pmod{2}$, and therefore $\affD_\infty(\mu,\lam-A\varpi_2)=0$.

\noindent \textbf{Second case: $d=\lam_1$.\;} In this case we have $A=\min(\lam_2+c-b,2c-b-1)$ and $\mu_1+\mu_2+\lam_2=2\lam_2-b$. It follows that $\mu_1+\mu_2+\lam_2\geq A$ if and only if $\lam_2\geq c$. Recall also that $b\leq 2c-\lam_1$.

Assume first $\lam_2\geq c$, so that $A=2c-b-1$ and $\mu_1+\mu_2+\lam_2\geq A$. The claim now follows since $\affD_\infty(\mu,\lam-A\varpi_2)\leq \lam_1-1\leq 2c-b-1$.

Assume now $\lam_2<c$ so that $A=\lam_2+c-b$ and $\mu_1+\mu_2+\lam_2\leq A$. The claim follows because, if $\affD_\infty(\mu,
\lamA)\geq 0$, then 
$\affD_\infty(\mu,
\lamA)\leq \lam_1+\lam_2+\mu_1+\mu_2-A=\lam_1+\lam_2-c\leq \lam_2+c-b=A$.

\noindent \textbf{Third case: $d\neq 0,\lam_1$.\;}
In this case we have $b=\lam_1-2d+2c$. Notice that $b\geq 2a+d$ is equivalent to $\lam_1-2a+2c> 3d$. We also have $A=\min(\lam_2+d-c,d-1)$ and $\mu_1+\mu_2+\lam_2=2\lam_2-2c+d$, so $\mu_1+\mu_2+\lam_2\geq A$ if and only if $\lam_2\geq c$.

Assume first $\lam_2\geq c$, so that $A=d-1$ and $\mu_1+\mu_2+\lam_2\geq A$. Notice that $\lam_1-1>A$ and also \[-\mu_1-1>A\iff \lam_1+2c-2a-2d-1>d-1\iff \lam_1+2c-2a>3d\]
Hence, $\affD_\infty(\mu,\lamA)>A$ if and only if $\lam_1+2c-2a>3d$.

Finally assume $\lam_2<c$ so that $A=\lam_2+d-c$ and $\mu_1+\mu_2+\lam_2<A$. In this case we have $\affD_\infty(\mu,\lamA)=\max(0,\min(0,\mu_1+\lam_1)+\lam_2+\mu_2-A)$.
We have $\mu_1+\lam_1+\lam_2+\mu_2-A=\lam_1+\lam_2-c>\lam_2+d-c=A$ and
\[\lam_2+\mu_2-A>A\iff \lam_1+2c-2a> 3d.\]
It follows that $\affD_\infty(\mu,\lamA)>A$ if and only if $\lam_1+2c-2a>3d$.
\end{proof}



\section{Swapping functions}

Recall the family of cocharacters $\{\eta_m\}_{m\in \bbN}$ introduced in \Cref{sec:family}.
The unique wall separating $\eta_m$ and $\eta_{m+1}$ is 
$H_{\alpha_{m+1}^\vee}$, where $\alpha_{m+1}^\vee \in \affPhi^\vee_+$ is the $(m+1)$-th root occurring in the sequence \eqref{reflectionorder}.  
As in \eqref{tm}, let $t:=t_{m+1}$  denote the corresponding reflection.
 For any $\mu\in X$ such that $\mu<t\mu\leq \lambda$ we define
\[ \psi_{t\mu}:\calB(\lambda)_{t\mu}\ra 
\calB(\lambda)_{\mu}\]
as follows. 
Let $T\in \calB(\lam)_{t\mu}$ and assume that $T\in \calA(\lam-a\varpi_2)\subset \calP(\lam)$ and let $e:=(\mu \raw t\mu)\in E(\lambda-a\varpi_2)$. Then $\psi_{t\mu}(T)=T'$, where $T'$ is the only element of weight $\mu$ in $\calA(\lam-(a+\Omega(e))\varpi_2)\subset \calP(\lam)$.

\begin{proposition}\label{swappingfunctionprop}
The collection of maps $\psi=\{\psi_\nu\}$ is a swapping function between $\eta_{m+1}$ and $\eta_m$. In particular, if $r_{m+1}$ is a recharge for $\eta_{m+1}$ then $r_m$ is a recharge for $\eta_m$. 
\end{proposition}

To prove \Cref{swappingfunctionprop} we need to check that for any $m$ and $T$ we have $r_{m+1}(T)=r_{m+1}(\psi_{t\mu}(T))+1$, or equivalently that $\sigma_{m+1}(T)=\sigma_{m+1}(\psi_{t\mu}(T))-1$. %This is in turn equivalent to $\sigma_{m}(\psi_{t\mu}(T))=\sigma_{m}(T)-1$\Leo{Is this clear?}














% Now, we need to divide into two cases.
% First we consider the case of a swappable edge, for which the swapping function only involves element within the same atom.

% \begin{proposition}
% Assume $T\in \calA(\lam-k\varpi_2)\subset \calP(\lam)$ with $t_{m+1}\mu =\wt(T)$ and that $(\mu\raw t_{m+1}\mu)\in E^S(\lam-k\varpi_2)$. Let $T$ be the element in the same atom of $T$ with weight $\mu$. Then we have
% \[\sigma_m(T)=\sigma_m(T')+1\]
% \end{proposition}
% \begin{proof}
% Since $\mu\raw t_m\mu$ is swappable we have $\ell_m(\mu,\lambda-k\varpi_2)=\ell_m(t_{m+1}\mu,\lamk)-1$. By \Cref{corcon} we have that  $\calN_m(\mu,\lambda-k\varpi_2)=0$ and by \Cref{Nintmuis0} we get $\calN_m(t_m\mu,\lamk)=0$. Finally, by \Cref{ifN=0thenD=0}, we also get $\calD_m(\mu,\lam,k)=\calD_m(t_{m+1}\mu,\lam,k)=0$ and the claim follows.
% \end{proof}

% \begin{proposition}
% Assume $T\in \calA(\lam-k\varpi_2)\subset \calP(\lam)$ with $t_{m+1}\mu =\wt(T)$ and that $(\mu\raw t_{m+1}\mu)\in E^S(\lam-k\varpi_2)$. Let $T$ be the element in the same atom of $T$ with weight $\mu$. Then we have
% \[\sigma_{m+1}(T)=\sigma_{m+1}(T')-1\]
% \end{proposition}
% \begin{proof}
% Since $\mu\raw t_m\mu$ is swappable we have $\ell_{m+1}(\mu,\lambda-k\varpi_2)=\ell_{m+1}(t_{m+1}\mu,\lamk)+1$. By \Cref{corcon} we have that  $\calN_{m+1}(\mu,\lambda-k\varpi_2)=0$ and by \Cref{Nintmuis0} we get $\calN_{m+1}(t_m\mu,\lamk)=0$. Finally, by \Cref{ifN=0thenD=0}, we also get $\calD_{m+1}(\mu,\lam,k)=\calD_{m+1}(t_{m+1}\mu,\lam,k)=0$ and the claim follows.
% \end{proof}

% We now consider the general case. \Leo{Is the first case necessary?}

% \begin{proposition}
% Assume $T\in \calA(\lam-k\varpi_2)\subset \calP(\lam)$ with $t_{m+1}\mu =\wt(T)$ and that $(\mu\raw t_{m+1}\mu)\in E^N(\lamk)$. Let $\Omega:=\Omega(\mu\raw t_{m+1}\mu)$ be its elevation.
% Let $T'$ be the element of weight $\mu$ in the atom of $\calA(\lam-(k+\Omega)\varpi_2)$.
% Then, we have 
% \begin{equation}\label{sigmans}\sigma_m(T)=\sigma_m(T')+1.\end{equation}
% \end{proposition}
% \begin{proof}
% %First of all, notice that such element $T'$ exists. In fact, we have $\mu\leq \lambda_i-k\varpi_2$. \Leo{better to prove it explicitly...} 

% By definition we have $\calD_m(\mu,\lam,k+\Omega)=\Omega-1$.
% Combining with \Cref{N=0}, our claim  \eqref{sigmans} becomes equivalent to
% \[\ell_m(t_{m+1}\mu,\lamk)=\ell_m(\mu,\lam-(k+\Omega)\varpi_2)+2\Omega-\calN_m(\mu,\lam-(k+\Omega)\varpi_2)\]

% Assume now $\mu_1>0$. By \Cref{corcon}, if $t_{m+1}\mu\leq \lam-(k+\Omega)\varpi_2$ we have $\mu \raw t_{m+1}\mu\in E^N(\lam-(k+\Omega)\varpi_2)$, so \Leo{k should be $a$}
% \[\calN_{m+1}(\mu,\lam-(k+\Omega)\varpi_2)-\calN_m(\mu,\lam-(k+\Omega)\varpi_2)=1-I_{\not\leq (\lambda-(k+\Omega)\varpi_2)}(t_{m+1}\mu)\]

% Using \Cref{Ncountgen}, this is equivalent to
% \[\ell_m(t_{m+1}\mu,\lamk)-\affell_m(t_{m+1}\mu,\lam-(k+\Omega)\varpi_2)=2\Omega-I_{\not\leq (\lambda-(k+\Omega)\varpi_2)}(t_{m+1}\mu)\]
% We can now conclude because 
% \[\affphi_{21}(t_{m+1}\mu,\lamk)-\affphi_{21}(t_{m+1}\mu,\lam-(k+\Omega)\varpi_2)=\Omega\]
% \[\affphi_{12}(t_{m+1}\mu,\lamk)-\affphi_{12}(t_{m+1}\mu,\lam-(k+\Omega)\varpi_2)=\Omega\]
% because $(t_{m+1}\mu)_2\leq -\lam_2$ (as proven in \Cref{claimtmu}). 
% \end{proof}

\begin{proposition}\label{swapcheck}
Assume $T\in \calA(\lam-k\varpi_2)\subset \calP(\lam)$ with $t\mu =\wt(T)$ and that $e:=(\mu\raw t\mu)\in E(\lama)$. %Let $\Omega:=\Omega(e)$ be its elevation.
 % be the element of weight $\mu$ in the atom of $\calA(\lam-(a+\Omega)\varpi_2)$.
Then, we have 
\begin{equation}\label{sigmans}\sigma_{m+1}(T)=\sigma_{m+1}(\psi_{t\mu}(T))-1.\end{equation}
\end{proposition}
\begin{proof} 
By \Cref{Nintmuis0} we have $\calN_{m+1}(t\mu,\lama)=0$ and by \Cref{ifN=0thenD=0} we also get $\calD_{m+1}(t\mu,\lam,a)=0$. Let $\Omega:=\Omega(e)$ and recall that $\psi_{t\mu}(T)$ is the element of weight $\mu$ in the atom $\calA(\lam-(a+\Omega)\varpi_2)\subset \calP(\lam)$.

First assume $\Omega=0$, or equivalently that $e$ is swappable. Since $\mu\raw t\mu$ is swappable, by definition we have $\ell_{m+1}(\mu,\lama)=\ell_{m+1}(t\mu,\lama)+1$. By \Cref{corcon} we have that  $\calN_{m+1}(\mu,\lama)=0$ and by  \Cref{ifN=0thenD=0}, we also get $\calD_{m+1}(\mu,\lam,a)$. The claim now easily follows.

Assume now $\Omega>0$, so $e$ is not swappable. Notice that $\calD_{m+1}(\mu,\lam,a+\Omega)=\Omega$.
Combining with \Cref{Nintmuis0}, our claim \eqref{sigmans} becomes equivalent to
\[\ell_{m+1}(t\mu,\lama)=\ell_{m+1}(\mu,\lam-(a+\Omega)\varpi_2)-\calN_{m+1}(\mu,\lam-(a+\Omega)\varpi_2)+2\Omega-1.\]

We can assume $\mu_1\geq 0$ as the case $\mu_1<0$ is symmetric. Because $e$ is not swappable, we have $m+1=4M$, $q_M\mu\not\leq \lam$ and $r_M\mu \not \leq \lam$. In particular, by \eqref{eqqu} and \eqref{eqru} we have
 $\ell^{12}_{m+1}=\affphi_{12}$ and $\ell^{21}_{m+1}=\affphi_{21}$. 
Using \Cref{Ncountgen}, our claim is then equivalent to
\begin{equation}\label{2Omegaeq}\ell_{m+1}(t\mu,\lama)-\affell_{m+1}(t\mu,\lam-(a+\Omega)\varpi_2)=2\Omega.
\end{equation}
%I_{\not\leq (\lambda-(a+\Omega)\varpi_2)}(t\mu)\]
By \Cref{affphicompute} we have  
\[\affphi_{21}(t\mu,\lama)-\affphi_{21}(t\mu,\lam-(a+\Omega)\varpi_2)=\Omega\]
\[\affphi_{12}(t\mu,\lama)-\affphi_{12}(t\mu,\lam-(a+\Omega)\varpi_2)=\Omega\]
because $(t_{m+1}\mu)_2\leq -\lam_2$ (as proven in \Cref{claimtmu}) and the identity \eqref{2Omegaeq} now follows directly from the definition of $\affell_{m+1}$. 
\end{proof}


% \begin{corollary}
% The number of swappable edges on the left of $\mu$ is 
% \[ \hat{\phi}_2(\mu,\lambda)-BG(\mu,\lambda)=
% \min\left(\max\left(0,\lce \frac{\mu_2+\lambda_2}{2}\rce\right)+\max\left(0,\frac{\lambda_1-\mu_1}{2}\right),\hat{\phi}_2(\mu,\lambda)\right).
% \]
% \end{corollary}





\begin{proof}[Proof of \Cref{swappingfunctionprop}.]
\Cref{swapcheck} precisely shows that $\psi$ is a swapping function for $r_{m+1}$. This means that we can obtain a new recharge $r'_m$ for $\eta_m$ by swapping the values of $r_{m+1}$ as indicated by $\psi$. It remains to show that $r_m=r_m'$. In other words, for $t=t_{m+1}$ and for any $\mu\leq t\mu$  we need to show that 
\begin{enumerate}
    \item if $\wt(T)=t\mu$ then $r_m(T)=r_{m+1}(\psi(T))=r_{m+1}(T)-1$;
    \item if $\wt(U)=\mu$ and $U\in \Ima(\psi_{t\mu})$ then $r_m(U)=r_{m+1}(\psi_{t\mu}^{-1}(U))=r_{m+1}(U)+1$;
    \item   if $\wt(U)=\mu$ and $U\not\in \Ima(\psi_{t\mu})$ then $r_m(U)=r_{m+1}(U)$.
\end{enumerate}

The first statement is clear since $r_m(T)-r_{m+1}(T)=\ell_{m+1}(t\mu,\tau)-\ell_{m}(t\mu,\tau)=-1$ by \Cref{Nintmuis0,ifN=0thenD=0}. Let now $U\in \calA(\zeta)$ with $\wt(U)=\mu$ and let $a:=\at(U)$. We need to compute
\begin{align}
     r_m(U)-r_{m+1}(U)=&\ell_{m+1}(\mu,\zeta)-\ell_m(\mu,\zeta)-\calN_{m+1}(\mu,\zeta)+\calN_m(\mu,\zeta)+ \nonumber\\&+\calD_{m+1}(\mu,\zeta+a\varpi_2,a)-\calD_m(\mu,\zeta+a\varpi_2,a)\label{rmT'}.
\end{align}
If $U\in \Ima(\psi_{t\mu})$, there exists $\Omega\in \bbN$ such that $e:=(\mu\raw t\mu)\in E(\zeta+\Omega\varpi_2)$ and $\Omega=\Omega(e)$. 
If $\Omega=0$, then $e$ is swappable and $t\mu\leq \zeta$. So \eqref{rmT'} simplifies to $r_m(U)-r_{m+1}(U)=\ell_{m+1}(\mu,\zeta)-\ell_m(\mu,\zeta)=1$. If $\Omega>0$, then we have by \Cref{corcon}.1 that $(\mu\raw t\mu)\in E^N(\zeta)$ or $t\mu\not \leq \zeta$. It follows that
\begin{equation}\label{Ndiff}
    \ell_{m+1}(\mu,\zeta)-\ell_m(\mu,\zeta)=\calN_{m+1}(\mu,\zeta)-\calN_m(\mu,\zeta)=\begin{cases} 1&\text{if }t\mu \leq \zeta\\
0&\text{if }t\mu \not \leq \zeta,\end{cases}
\end{equation}
so the first line in the RHS of \eqref{rmT'}  vanishes.
Since $\Omega\leq a$, the edge $e$ belongs to a truncated NS staircase over $(\mu,\zeta+a\varpi_2)$, hence $\calD_{m+1}(\mu,\zeta+a\varpi_2,a)-\calD_m(\mu,\zeta+a\varpi_2,a)=1$.

Finally, assume that $U\not\in \Ima(\psi_{t\mu})$. This means that $(\mu\raw t\mu)\not\in E^S(\zeta)$, so \eqref{Ndiff} holds again in this case. Moreover, there does not exists $k\leq a$ such that $f:=(\mu\raw t\mu)\in E^N(\zeta+k\varpi_2)$ with $\Omega(f)=k$, from which it follows that $\calD_{m+1}(\mu,\zeta+a\varpi_2,a)=\calD_m(\mu,\zeta+a\varpi_2,a)$ and \eqref{rmT'} can be simplified to $r_m(U)-r_{m+1}(U)=0$.
\end{proof}

\subsection{Alternative formula}
We can obtain an alternative formula for the charge statistic by focusing on a single element and counting how many times its recharge gets changed by a swapping function. In type $A$, this is discussed in \cite[Remark 5.6]{Pat}.

\begin{definition}
	We define $\Delta^\alpha:
	\calB(\lambda)\raw \bbZ$, for $\alpha\in \Phi_+$ as the \emph{total contribution} of the swapping functions along the direction $\alpha$. It is defined as 
	\[ \Delta^{\alpha}=\sum r_m(T)-r_{m-1}(T)\]
	where the sum runs over all $m$ such that the (unique) wall between the $\lambda$-chambers of $\eta_m$ and $\eta_{m-1}$ is of the form $H_{M\delta-\alpha^\vee}$.
	
	We write $\Delta^i:=\Delta^{\alpha_i}$  for $i\in \{1,2,21,12\}$.
\end{definition}

We have $r_{KL}-r_{MV}=\sum_{\alpha\in \Phi_+} \Delta^\alpha$.
Recall in type $A$ for any $\alpha\in \Phi_+$ we have $\Delta^{\alpha}(T)=\phi_\alpha(T)-\ell^\alpha(\wt(T))$. When we apply the swapping functions along the $\alpha_1$-direction, to go from the MV region to the parabolic region, we  regard $\calB(\lambda)$  as a crystal of type $A_1$.  It follows that $\Delta^1(T)=\phi_1(T)-\ell^1(\wt(T))$ as in \cite[Lemma 3.26]{Pat}. Moreover, if $T\in \calB_+(\lam)$, we have
\begin{equation}\label{Deltaepsilon}
    \Delta^1(T)=\phi_1(T)-\langle \wt(T),\alpha_1^\vee\rangle=\eps_1(T).
\end{equation}

%Notice that if $T\in \calB_+(\lambda)$ then all the contributions $r_m(T)-r_{m-1}(T)$ are either $0$ or $1$. Therefore $\Delta^\alpha(T)$ is exactly the number of times the $T$ is in the image of a swapping functions corresponding to a wall of the form $H_{M\delta-\alpha^\vee}$.

\begin{proposition}
	For $T\in \calA(\zeta)\subset \calB(\lambda)$ with $\wt(T)=\mu$ and $\at(T)=a$ we have 
	\begin{enumerate}
		\item $\Delta^{21}(T)=\affphi_{21}(\mu,\zeta)-\ell^{21}(\mu)(T)$.
		\item $\Delta^2(T)=\phi_2(T)-\ell^{2}(\wt(T))$
		\item $\Delta^{12}(T)=\phi_{12}(T)-\ell^{12}(\wt(T))$
	\end{enumerate}
\end{proposition}
\begin{proof}
By \Cref{21swappable}, the swaps in the $\alpha_{21}$-direction always occur  within the atom of $T$, so to compute $\Delta^{21}(T)$ we just need to consider the string of elements in the atom of $T$ of weights $\mu+k\alpha_{21}$. This means that we can compute $\Delta^{21}$ as in the rank one case and have $\Delta^{21}(T)=\affphi_{21}(T)-\ell^{21}(\mu)$.
	
	
	Assume first $\mu_1\leq 0$. Then the swapping occurring on $T$ in the $\alpha_2$ direction only occur within the atom of $T$, so as for $\Delta^{21}$, we have 
 \[\Delta^2(T)=\affphi_{2}(\mu,\zeta)-\ell^2(\mu)=\phi_2(T)-\ell^2(\mu),\]
where the second equality comes from \Cref{atomicphi2}.
 
	Assume now $\mu_1\geq 0$. Then by construction the number of swappable edges containing $\mu$ in the atom of $T$ is $\affphi_2(\mu,\zeta)-\calN_\infty(\mu,\zeta)$. Of these, there are $\ell^2(\mu)$ attached to roots $M\delta+\alpha_2^\vee$, which do not correspond to any crossed wall. Moreover, $T$ is also in the image of $\calD_\infty(\mu,\zeta+a\varpi_2,a)$ swapping functions, corresponding to non-swappable edges in atoms bigger than $\calA(\zeta)$. It follows that \[\Delta^2(T)=\affphi_2(\mu,\zeta)-\ell^2(\mu)-\calN_\infty(\mu,\zeta)+\calD_{\infty}(\mu,\zeta+a\varpi_2,a)=\phi_2(T)-\ell^2(\mu)\] by \Cref{phi2claim}.
	
	The proof of the formula for $\Delta^{12}$ is symmetric.
\end{proof}

Assume now that $T\in \calB_+(\lam)$. Then, as in \eqref{Deltaepsilon}, we have $\Delta^2(T)=\eps_2(T)$ and $\Delta^{12}(T)=\eps_{12}(T)$. 

\begin{definition}
Let $T\in \calA(\zeta)$ be such that  $\wt(T)=\mu$. We define $\affeps_{21}(T):=\affphi_{21}(\mu,\zeta)-\ell^{21}(\mu)$.
\end{definition}

Notice that $\affeps_{21}(T)$ can equivalently be defined as the largest integer $k$ such that $\wt(T)+k\leq \zeta$, for $T\in \calA(\zeta)$.

For $T\in \calB_+(\lam)$, we have $r_{KL}(T)-r_{MV}(T)=\eps_1(T)+\eps_2(T)+\eps_{12}(T)+\affeps_{21}(T)$. Since $r_{MV}(T)+\frac 12 \ell(\wt(T))=0$ for $T\in \calB_+(\lam)$, it follows that
\[ c(T)=\eps_1(T)+\eps_2(T)+\eps_{12}(T)+\affeps_{21}(T)\]
is a charge statistic on $\calB_+(\lam)$.

We conclude by giving a more explicit way to compute $\affeps_{21}(T)$.

\begin{definition}\label{defe21}
    Let $T$ be in the biggest atom, that is we assume $T\in \calA(\lam)\subset \calP(\lam)\subset \calB(\lam)$ and let $\str_2(T)=(a,b,c,d)$.    We define
    \[e_{21}^{\str}(T):=\begin{cases}
        (a-1,b-1,c,d) & \text{if }c=d=0\\
        (a-1,b-2,c,d+1) & \text{if }c>0\text{ and }d=0\\
        (a,b,c-1,d-1) & \text{if }d>0
    \end{cases}\]
    and $\bar{e}_{21}(T)$ as the element in $\calB(\lam)$ such that $\str_2(\bar{e}_{21}(T))=e_{21}^{\str}(T))$ if it exists, and $0$ otherwise.
    Finally, define $\affe_{12}(T)$ as $\bar{e}_{12}(T)$ if $\wt(T)_1\leq 0$ and $s_1(\bar{e}_{12}(s_1(T)))$ if $\wt(T)_1\geq 0$.
\end{definition}


\begin{proposition}
    Let $T\in \calA(\lam)\subset \calB(\lam)$
    \begin{itemize}
        \item If $\affe_{21}(T)\neq 0$, then $\affe_{21}(T)\in \calA(\lam)$ and $\affeps_{21}(T)>0$.
        \item If $\affe_{21}(T)=0$ and $\langle\wt(T),\alpha_{21}^\vee\rangle \geq 0$, then $\affeps_{21}(T)=0$.
    \end{itemize}  
\end{proposition}
\begin{proof}
    It can be easily verified by \Cref{atom0} that if $T\in \calA(\lam)$ and $\affe_{21}(T)\neq 0$, then also $\affe_{21}(T)\in \calA(\lam)$.

To prove the second statement, we introduce operators $f_{21}^{\str},\bar{f}_{21},\afff_{21}$, similarly to \Cref{defe21}, where $f_{21}^{\str}(T)$ is defined, for $T\in \calA(\lam)$ with $\str_2(T)=(a,b,c,d)$ as 
    \[f_{21}^{\str}(T)=\begin{cases}(a+1,b+1,c,d)& \text{if }b<\lam_1-2d+2c\\
    (a,b,c+1,d+1)& \text{if }d=0\text{ or }c=\lam_2+d\\
    (a-1,b+2,c,d-1)&\text{if }d=1\text{ and }c<\lam_2+d\end{cases}\]
    Again, one can verify via \Cref{atom0}, that if $T\in \calA(\lam)$ also $\afff_{21}(T)\in \calA(\lam)$. If $\affeps_{21}(T)\neq 0$, there exists $U\in \calA(\lam)$ with $\wt(U)=\wt(T)+\alpha_{21}$. Then, we have  $f_{21}(U)=T$, from which it follows that $e_{21}(T)=U\neq 0$, or $\afff_{21}(U)=0$. But we cannot have $\afff_{21}(U)=0$ and $\langle\wt(U),\alpha_{21}^\vee\rangle \geq 2$. For example, if $c=\lam_2+d$ or $d=0$, then $\bar{f}_{21}(T)=0$ only if $a=\lam_2+b-2c+2d$, which implies $\langle\wt(U),\alpha_{21}^\vee\rangle =\wt(U)_1+2\wt(U)_2=-b\leq 0$.
\end{proof}
The proposition implies that $\affeps_{21}$ is associated to the operator $\affe_{21}$. That is, we have $\affeps_{21}(T)=\max\{ k \mid \affe_{21}^k(T)\neq 0\}$.
Similar expressions for $\affe_{21}$ on the other atoms can be obtained recursively using the embeddings $\Phi$ and $\bPsi$.


We believe one can construct similar charge statistics in higher ranks.
\begin{conjecture}
Assume $\calB$ is a crystal of type $C_3$. Then there exists a function $\affeps_{32}:\calB\raw \bbZ_{\geq 0}$ such that
\[
\begin{aligned} c(T)=\eps_1(T)+\eps_2(T)+\eps_2(s_1(T))+\eps_3(T)+\eps_3(s_2(T))+\eps_3(s_1s_2(T))\\+\affeps_{32}(T)+\affeps_{32}(s_1(T))+\affeps_{32}(s_2s_1(T))\end{aligned}\]
    is a charge statistic on $\calB_+(\lam)$.
\end{conjecture}
Notice that if $\wt(T)=0$ the conjecture predicts that $c(T)=\eps_1(T)+2\eps_2(T)+3\eps_3(T)+3\affeps_{321}(T)$. We have checked in many examples that such a function exists on elements of weight $0$.

% We conclude by giving a more intrinsic description of $\affeps_{21}(T)$ in terms of the crystal operators.
% Let $T\in \calB_+(\lam)$. We can compute easily $\pat(T)$ by \Cref{atomicnumber}. So $T\in \calA(\zeta)\subset \calP(\lam-2\varpi_1\pat(T)))$ where $\zeta_1=\lam_1-2\pat(T)$. 
% Recall that by \Cref{lemmaonPsi}, since $\Psi(Z)=\bPsi(Z)$ for $\wt(Z)_1\leq 0$ and $\Psi(Z)=s_1\bPsi s_1(Z)$ for $\wt(Z)_1\geq 0$ for $U\in \calA(\zeta)$ we  have
% \begin{itemize}
%     \item if $\wt(U)_1\geq 0$ and $e_{12}(U)\neq 0$, then $e_{12}(U)\in \calA(\zeta)$;
%     \item if $\wt(U)_1\leq 0$ and $e_2(U)\neq 0$, then $e_2(U)\in \calA(\zeta)$;
%     \item  if $\wt(U)_1\geq 2$ and $f_{12}(U)\neq 0$, then $f_{12}(U)\in \calA(\zeta)$
%     \item if $\wt(U)_1\leq -2$ and $f_2(U)\neq 0$ then $f_2(U)\in \calA(\zeta)$.
% \end{itemize}
% Assume now $\wt(U)_1=0$. Then $e_{12}(



% \section{A formula for the charge statistic in type \texorpdfstring{$C_2$}{C2}}

% We describe here a conjecture for the charge in type $C_2$.

% Let $A(\lambda)$ be the largest preatom inside the crystal $B(\lambda)$, i.e. $A(\lambda)$ is the complement of $emb(B(\lambda-2\omega_1)$. By induction, we can define the charge on $emb(B(\lambda-2\omega_1)$ increasing by $1$ the charge on $B(\lambda-\omega_1)$. So it remains to define it on $A(\lambda)$.

% We have the crystal operator $e_2$ which preserves $A(\lambda)$. We define another operator $e_{12}$ by $s_1e_2s_1$, where $s_1$ is the simple reflection (which also preserve $A(\lambda)$. Notice that $e_{12}$ increases the weight by $\alpha_1+\alpha_2$. (recall that the roots are $\alpha_1,\alpha_2,\alpha_1+\alpha_2$ and $\alpha_1+2\alpha_2)$ 



% Let $G(b)$ be the maximum integer such that $e_{12}^{G(b)}e_2^{G(b)}(b)\neq \emptyset$. Then, we define
% \[R(b)=\begin{cases}2G(b) & \text{if }e_{12}^{G(b)+1}e_2^{G(b)}(b)=\emptyset\\
% 2G(b)+1 & \text{if }e_{12}^{G(b)+1}e_2^{G(b)}(b)\neq\emptyset
% \end{cases}\]

% \begin{definition}
% We call \emph{rectangle} of $A(\lambda)$, with $\lambda=\lambda_1\omega_1+\lambda_2\omega_2$, the subset of elements of weight $\mu=\mu_1\omega_1+\mu_2\omega_2$, with $|\mu_1|\leq \lambda_1$ and $|\mu_1+2\mu_2|\leq \lambda_1+2\lambda_2$
% \end{definition}

% To define the charge on $A(\lambda)$ we need to distinguish two cases. Let $wt(b)=\mu_1\omega_1+\mu_2\omega_2$. Then we can assume that $\mu_1\geq 0$. Otherwise, we define the charge on $b$ to be equal than the charge on $s_1(b)$.

% If $b$ is inside the rectangle, we simply define $C(b)=R(b)$. Let $b$ be outside the rectangle. Let $m$ be the maximum integer such that $wt(b)+m\alpha_1\leq \lambda$. Then we define
% \[ C(b)=\begin{cases} R(b)& \text{if }\epsilon_1(b)+R(b)<m\\
% R(b)-1&\text{otherwise}
% \end{cases}\]

% \begin{conjecture}
% The function $C(b)$ defines a charge statistic in type $C_2$
% \end{conjecture}

% We have checked the conjecture in many cases to be true. The existence of such a formula for the charge seems to hint at the existence of swapping functions. However, we have failed to find such functions for $\lambda=2\omega_1+3\omega_2$. 

