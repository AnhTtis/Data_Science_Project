
\section{The root system and Hecke algebra of type \texorpdfstring{$C_2$}{C2}}
\subsection{The root system and the affine Weyl group}

Let $(X,\Phi,X^\vee,\Phi^\vee)$ be the root datum of the reductive group $Sp(4,\bbC)$. The lattices $X$ and $X^\vee$ are isomorphic to $\bbZ^2$, with bases $\{\varpi_1,\varpi_2\}$ and $\{\varpi_1^\vee,\varpi_2^\vee\}$. Let $X_+$ and $X_+^\vee$ be the subsets of dominant weights and dominant coweights. 
Sometimes we use the notation $(\lambda_1,\lambda_2)$ to the denote the weight $\lambda=\lambda_1\varpi_1 +\lambda_2\varpi_2$.

The root system $\Phi\subset X$ is a root system of type $C_2$, with positive roots \[\{\alpha_1,\alpha_2,\alpha_{12}:=2\alpha_1+\alpha_2,\alpha_{21}:=\alpha_1+\alpha_2\}\] with $\alpha_2$ and $\alpha_{12}$ being the long roots. We have $\alpha_1=2\varpi_1-\varpi_2$ and $\alpha_2=-2\varpi_1+2\varpi_2$.


The coroot system $\Phi^\vee\subset X^\vee$ has positive coroots \[\{\alpha_1^\vee, \alpha_2^\vee,\alpha_{12}^\vee:=\alpha^\vee_1+\alpha^\vee_2,\alpha_{21}^\vee:=\alpha^\vee_1+2\alpha^\vee_2\}.\]
For any $i\in \{1,2,12,21\}$, $\alpha_i^\vee$ is the coroot corresponding to $\alpha_i$.

\begin{figure}[hbt!]
\begin{center}
\begin{tikzpicture}
\draw[->] (0,0) to (1,1) node[above] {$\alpha_{21}$};
\draw[->] (0,0) to (2,0) node[above] {$\alpha_{2}$};
\draw[->] (0,0) to (0,2) node[above] {$\alpha_{12}$};
\draw[->] (0,0) to (0,-2);
\draw[->] (0,0) to (-2,0);
\draw[->] (0,0) to (-1,-1);
\draw[->] (0,0) to (1,-1);
\draw[->] (0,0) to (-1,1) node[above] {$\alpha_{1}$};
\begin{scope}[scale=1.3, xshift=4cm]
\draw[->] (0,0) to (1,1) node[above] {$\alpha^\vee_{21}$};
\draw[->] (0,0) to (1,0) node[above] {$\alpha^\vee_{2}$};
\draw[->] (0,0) to (0,1) node[above] {$\alpha^\vee_{12}$};
\draw[->] (0,0) to (0,-1);
\draw[->] (0,0) to (-1,0);
\draw[->] (0,0) to (-1,-1);
\draw[->] (0,0) to (1,-1);
\draw[->] (0,0) to (-1,1) node[above] {$\alpha^\vee_{1}$};
\end{scope}
\end{tikzpicture}
\end{center}
\caption{The root system $\Phi$ and the coroot system $\Phi^\vee$.}\label{figrootsystems}
\end{figure}

Let $\rho$ be the half-sum of the positive roots and $\rho^\vee$ be the half-sum of negative roots. We have $\rho=2\alpha_1 +\frac32 \alpha_2$ and $\rho^\vee= \frac32 \alpha_1^\vee + 2\alpha_2^\vee$. 


We have $X/\bbZ \Phi\cong \bbZ/2\bbZ$ and the two classes are generated by $0$ and $\varpi_1$.

Let $\affW$ be the affine Weyl group of type $\tilde{C}_2$. The group $\affW$ has three simple reflections $s_0, s_1, s_2$ and has the following description as a Coxeter group:
\[ \affW \cong \langle s_0,s_1,s_2\mid s_0^2=s_1^2=s_2^2=(s_0s_2)^4=(s_1s_2)^4=(s_0s_1)^2=e\rangle .\]

Let $ \affX^\vee:=X^\vee\oplus \bbZ$ and let $\affPhi^\vee= \{ \alpha^\vee + m\delta \mid \alpha^\vee \in \Phi^\vee, m \in \bbZ\}$ be the corresponding affine root system. The positive roots in $\affPhi^\vee$ are \[\affPhi^\vee_{+} =\{ \alpha^\vee + m\delta \mid \alpha^\vee \in \Phi^\vee, m >0\} \cup \Phi^\vee_+\] and the simple roots are
 \[\affDelta^\vee = \{\alpha_1^\vee,\alpha_2^\vee,\alpha_0^\vee:=\delta -\alpha_{21}^\vee\}\]
There is a bijection between reflections in $\affW$ and positive roots $\affPhi^\vee_+$, with simple reflections corresponding to simple roots. For a reflection $t\in \affW$ we denote by $\alpha_t^\vee$ the corresponding root.



% Consider $\undc = s_0s_2s_1s_2$. Then $y_\infty:=\undc \undc \undc\ldots$ is an infinite reduced expression.
% Let $y_m$ be the element given by the first $m$ simple reflections in $y_{\infty}$.
% We order the roots in $N(y_\infty)$ as follows:
% \begin{multline}\label{reflectionorder}
% \delta-\alpha_{21}^\vee<\delta-\alpha_{12}^\vee<2\delta-\alpha_{21}^\vee< \delta -\alpha_{2}^\vee< 3\delta-\alpha_{21}^\vee<2\delta-\alpha_{12}^\vee< \\
% \ldots<M\delta-\alpha_{12}^\vee<2M\delta-\alpha_{21}^\vee<M\delta-\alpha_{2}^\vee<(2M+1)\delta-\alpha_{21}^\vee<\ldots \\ 
% \end{multline}
% The first $m$ roots in \eqref{reflectionorder} are precisely the elements of $N(y_m)$.
% We just write $\ell_m$ and $\leq_m$ for $\ell_{N(y_m)}$ and $\leq_{N(y_m)}$, the $y_m$-twisted length and the $y_m$-twisted Bruhat order on $\affW$.

% Recall that $X\cong \extW/W$. Hence, the twisted Bruhat order on $\extW$ also induces a Bruhat order on $X$. (Concretely, this means that we regard $\lambda\in X$ as a right coset in $\extW$ and denote by $\lambda_m\in \extW$ the element of minimal $y_m$-twisted length in the coset $\lambda$. We set $\ell_m(\lambda):=\ell_m(\lambda_m)$ and$\mu\leq_m \lambda$ if $\lambda_m\leq_m \mu_m$.



% Let $\Gamma_X$ denote the Bruhat graph of $X$.
% The vertices of $\Gamma_X$ are the weights $X$, and we have an arrow $\mu\raw \lambda$ if and only if there exists $\alpha^\vee\in \affPhi^\vee$ such that $s_{\alpha^\vee}(\mu)=\lambda$ and $\mu\leq \lambda$.

% 	For every $m\in \bbZ_{\geq 0} \cup \{ \infty\}$ let $\Gamma_X^m$ be the $y_m$\emph{-twisted Bruhat graph of}$X$. This is the the directed graph whose vertices are the weights $X$ and where there is an edge $\mu\raw \lambda$ if there exists $\alpha^\vee\in \affPhi^\vee$ such that $s_{\alpha^\vee}(\mu)=\lambda$ and $\mu<_m \lambda$. 


\subsection{The Hecke algebra and its pre-canonical bases}\label{sec:precan}

Recall from \cite{Knop, Lus83} the definition of the spherical Hecke algebra (see also \cite[\S 2.2]{LPP}).
We denote by $\calH$ the spherical Hecke algebra associated to the root system $\Phi$.
The Hecke algebra is the free module over $\bbZ[v,v^{-1}]$ with standard basis $\{\bfH_\lam\}_{\lam\in X_+}$ and a canonical basis, the Kazhdan-Lusztig basis, which we denote by $\{\undH_\lambda\}_{\lambda\in X_+}$.

The spherical Hecke algebra can be thought of as a deformation of the monoid algebra $\bbZ[X_+]$, which as an abelian group is free with basis $\{e^\lam\}_{\lam\in X_+}$. In fact, specializing at $v=1$, we obtain a ring homomorphism 
\begin{align*}(-)_{v=1}:\calH &\raw \bbZ[X_+]\\
\bfH_\lam& \mapsto e^\lam.
\end{align*}
If $\lam=\lam_1\varpi_1+\lam_2\varpi_2$ we write $\undH_{(\lam_1,\lam_2)}$ for $\undH_{\lam}$ and similarly for $\bfH$.

For $w\in W$ and $\lam\in X$ we denote by $w\cdot \lam = w(\lam+\rho)-\rho$ the dot action of $w$ on $\lam$.
We say that a weight $\lambda$ is singular if there exists $w\in W$ with $w(\lambda)=\lambda$. Clearly, a weight $\lambda$ is singular if and only if $\lambda+\rho$ is singular with respect to the dot action.


We extend to definition of $\undH_\lam$ to the whole $X$ by setting $\undH_\lam=0$ if $(\lam+\rho)$ is singular and $\undH_\lam=(-1)^{\ell(w)}\undH_{w\cdot \lam)}$ if $w\in W$ is such that $w\cdot \lam\in X_+$. Notice that in our setting $\lam=(\lam_1,\lam_2)$ is singular if and only if $\lam_1=-1$, $\lam_2=-1$, $\lam_1+\lam_2=-2$ or $\lam_1+2\lam_2=-3$.


Recall the definition of the pre-canonical bases. We have
\[ \bfN_\lam^i=\sum_{I \subset \Phi^{\geq i}} (-v^2)^{|I|}\undH_{\lam-\sum_{\alpha\in I}\alpha}\]
where $\Phi^{\geq i}$ is the subset of roots of height at least $i$. Notice that we have $\Phi^{\geq 3}=\alpha_{12}=2\varpi_1$ and $\Phi^{\geq 2}=\{\alpha_{12},\alpha_{21}\}=\{2\varpi_1,\varpi_2\}$. 
Recall by \cite[Theorem 1.2]{LPP} that $\bfN^1$ is the standard basis, while $\bfN^2$ is the atomic basis $\bfN$, that is we have
\[ \bfN^2_\lam=\bfN_\lam=\sum_{\mu \leq \lam} v^{2\langle\rho^\vee,\lam-\mu\rangle}\bfH_{\mu}.\]



It follows immediately from the definition that $\undH_\lam=\bfN^4_\lam$. 

\begin{example}
 Unfortunately, and contrary to the type $A$ situation, the coefficients of the $\undH$-basis in the $\bfN^3$-basis are in general not postive. For example, we have $\bfN^3_{(0,\lam_2)}=\undH_{(0,\lam_2)}+v^2\undH_{(0,\lam_2-1)}$. In particular, we get $\undH_{(0,1)} =\bfN^3_{(0,1)}-v^2\bfN^3_{(0,0)}$.
\end{example}

To recover positivity, we need to introduce a modification of the $\bfN^3$ basis. We define 
\begin{align}
\label{modifiedprecan3}
 \tilN^{3}_\lam=\begin{cases}\bfN^3_\lam & \text{if }\lam_1\neq 0\\
\undH_\lam & \text{if }\lam_1 =0
\end{cases}
\end{align}

\begin{lemma}
We have \[\undH_{(\lam_1,\lam_2)}=\sum_{i \leq \frf{\lam_1}} v^{2i}\tilN^3_{(\lam_1-2i,\lam_2)}\]
\end{lemma}
\begin{proof}
We prove it by induction on $\lam_1$.
The claim is clear if $\lam_1=0$.

If $\lam_1>0$, we have $\tilN^3_\lam=\undH_{\lam}-v^2\undH_{\lam-2\varpi_1}$.
If $\lam_1=1$ we have $\tilN^3_\lam=\undH_\lam$ since $\lam-2\varpi_1+\rho$ is singular. If $\lam_1\geq 2$, we get $\undH_{\lam}=\bfN^3_{\lam}+v^2\undH_{\lam-2\varpi_1}$ and the claim easily follows by induction.
\end{proof}

\begin{lemma}\label{precanN3}
We have \[\tilN^3_{(\lam_1,\lam_2)}=\begin{cases} \sum_{i \leq \lam_2} v^{2i}\bfN^2_{(\lam_1,\lam_2-i)} & \text{if }\lam_1>0\\
\sum_{i \leq \frf{\lam_2}} v^{4i}\bfN^2_{(\lam_1,\lam_2-2i)}&\text{if }\lam_1=0.\end{cases}\]
\end{lemma}
\begin{proof}
We have $\bfN^2_\lam=\bfN^3_\lam-v^2\bfN^3_{\lam-\varpi_2}$. If $\lam_1>0$ we get $\tilN^3_\lam=\bfN^3_\lam=\bfN^2_\lam+v^2\tilN^3_{\lam-\varpi_2}$ and the claim easily follows by induction on $\lam_2$.

If $\lam_1=0$ we have $\tilN^3_\lam=\tilN^3_{(0,\lam_2)}=\undH^3_{(0,\lam_2)}$ and 
\begin{align*}
 \bfN^2_{(0,\lam_2)}&=\undH_{(0,\lam_2)}-v^2\undH_{(-2,\lam_2)}-v^2\undH_{(0,\lam_2-1)}+v^4\undH_{(-2,\lam_2-1)} \\
 &=\undH_{(0,\lam_2)}+v^2\undH_{(0,\lam_2-1)}-v^2\undH_{(0,\lam_2-1)}-v^4\undH_{(0,\lam_2-2)}\\
 &=\undH_{(0,\lam_2)}-v^4\undH_{(0,\lam_2-2)}=\tilN_{(0,\lam_2)}-v^4\tilN_{(0,\lam_2-2)}
\end{align*}

If $\lam_2 \leq 1$ we get $\bfN^2_{(0,\lam_2)}=\tilN_{(0,\lam_2)}$. For $\lam_2\geq 2$ we have $\tilN^3_{(0,\lam_2)}=\bfN^2_{(0,\lam_2)}+v^3\tilN^3_{(0,\lam_2-2)}$ and the claim follows by induction.
\end{proof}

\begin{remark}
The decomposition of the $\undH$-basis in terms of the $\bfN$ basis has been already computed in \cite[Theorem 1.1]{BBP}  with different methods. 
Here we prefer to reprove it using the precanonical bases since, as it turns out, also the $\tilN^3$ basis has a natural combinatorial interpretation in terms of the crystal (cf. \Cref{PreatomPrecan}).
\end{remark}

\section{Crystals and Weyl group actions}



A (seminormal) crystal for a complex finite dimensional Lie algebra $\mathfrak{g}$ consists of a non-empty set $B$ together with maps 
\begin{align*}
\op{wt}:& B \longrightarrow X \\
e_{i},f_{i}:& B \longrightarrow B \sqcup \left\{ 0\right\}, i \in [1, \operatorname{rank}(\mathfrak{g})]
\end{align*}

\noindent such that for all $b,b' \in B$:


\begin{itemize}	
\item $b' = e_i(b)$ if and only if $b = f_i(b')$,
\item if $f_i(b) \neq 0 $ then $\textsf{wt}(f_i(b)) = \textsf{wt}(b)-\alpha_i$;
\item
if $e_i(b) \neq 0$, then
$\textsf{wt}(e_i(b)) = \textsf{wt}(b)+\alpha_i$, and
\item  $\phi_i(b)-\eps_i(b)=  \langle \textsf{wt}(b),\alpha_i^\vee  \rangle$,
\end{itemize}

\noindent where 
\
\begin{align*}
\eps_i(b)&=\max\{a \in \mathbb{Z}_{\geq 0} :e_i^a(b)\neq 0\} \hbox{ and } \\
       \phi_i(b)&=\max\{a \in \mathbb{Z}_{\geq 0 }:f_i^a(b)\neq 0\}.
       \end{align*}


\noindent 
To each such crystal $B$ is associated a \textit{crystal graph}, a coloured directed graph with vertex set $B$ and edges coloured by elements $i \in [1,\operatorname{rank}(\mathfrak{g})]$, where if $f_{i}(b) = b'$ there is an arrow $b \overset{i}{\rightarrow} b'$. A crystal is irreducible if its corresponding crystal graph is connected and finite. A seminormal crystal is called normal if it is isomorphic to the crystal of a representation of $\mathfrak{g}$. 
Irreducible normal crystals are thus indexed by dominant integral weights of $\mathfrak{g}$. % and correspond to irreducible representations of $U_{q}(\mathfrak{g})$.
We refer the reader to \cite{BSch17} for more background on crystals.

For  a dominant weight  $\lambda$ we denote by $\mathcal{B}(\lambda)$ the corresponding normal crystal associated to the irreducible representation of $\mathfrak{g}$ %$U_{q}(\mathfrak{sp}_{4}(\mathbb{C}))$ 
of highest weight $\lambda$. 

\subsection{Crystals of Kashiwara--Nakashima tableaux}

In type $C$ we can realize crystals using Kashiwara--Nakashima tableaux.


\begin{definition}
Let $n$ be a positive integer. A Kashiwara--Nakashima tableau (KN tableau for short) is a semi-standard Young tableau of shape a partition of at most $n$ parts, in the alphabet 
\[\mathcal{P}_{n} := \left\{1 < \cdots < n < \bar n < \cdots < \bar 1 \right\}\]

\noindent which satisfy the following conditions:
\begin{itemize}
 \item Each one of their columns is \textbf{admissible} (cf. \Cref{defAdm}).
 \item Their \textbf{splitting} is a semi-standard Young tableau (cf. \Cref{defSpl}). 
\end{itemize}
\end{definition}



\begin{definition}\label{defAdm}
Let $C$ be a semi-standard column in the alphabet $\mathcal{P}_{n}$ of length at most $n$. Let $Z = \left\{z_{1} > ... > z_{m} \right\}$ be the set of non-barred letters $z$ in $\mathcal{P}_{n}$ such that both $z$ and $\bar z$ both appear in $C$. We say that the column $C$ is \emph{admissible} if there exists a set $T = \left\{t_{1} > ... > t_{m} \right\}$ of non-barred letters that satisfy:
\begin{itemize}
 \item $t_{1} < z_{1}$ and is maximal with the property $t_1, \bar t_1 \notin C$; 
 \item $t_{i} < \min(t_{i-1}, z_{i})$, $t_i, \bar t_i \notin C$ and is maximal with these properties.
\end{itemize}
\end{definition}

\begin{definition}
\label{defSpl}
The split of a column is the two-column tableau $lC rC$ where $lC$ is the column obtained from $C$ by replacing $ z_{i}$ by $ t_{i}$ and possibly re-ordering, and $rC$ is obtained from $C$ by replacing $ \bar z_{i}$ by $\bar t_{i}$ and possibly re-ordering.  

The \textit{splitting} of a semi-standard Young tableau consisting of admissible columns is the concatenation of the splits of its columns.
\end{definition}

\begin{example}
Let $n = 2$. The column $\Skew(0:\hbox{\tiny{$2$}}|0: \hbox{\tiny{$\bar 2$}})$ is admissible (we have $Z=\{2\}$ and $T=\{1\}$), however, $\Skew(0:\hbox{\tiny{$1$}}|0: \hbox{\tiny{$\bar 1$}})$ is not. Notice that although each one of its columns is admissible, the tableau $\Skew(0:\hbox{\tiny{$2$}},\hbox{\tiny{$2$}}|0: \hbox{\tiny{$\bar 2$}}, \hbox{\tiny{$\bar 2$}} )$ is not KN, because its split, $\Skew(0:\hbox{\tiny{$1$}},\hbox{\tiny{$2$}}, \hbox{\tiny{$1$}},\hbox{\tiny{$2$}} |0: \hbox{\tiny{$\bar 2$}}, \hbox{\tiny{$\bar 1$}}, \hbox{\tiny{$\bar 2$}}, \hbox{\tiny{$\bar 1$}})$ is not semi-standard. 
\end{example}

\begin{definition}
\label{def:weightofatableau}
Let $T$ be a KN tableau. For $i \in \left\{1,2\right\}$ let $n_i(T)$ denote the number of $i$'s appearing in $T$ and let $n_{\bar i}(T)$ denote the number of $\bar i$'s. Let $t_{i}(T) = n_{i}(T) - n_{\bar i}(T)$. Let $\lambda_{1}(T) = t_{1}(T) - t_{2}(T)$ and $\lambda_{2}(T) = t_{2}(T)$. The weight of $T$ is defined to be $\wt(T) = (\lambda_{1}(T),\lambda_2(T)) = \lambda_1 (T)\varpi_1 + \lambda_2(T) \varpi_2$.
\end{definition}

\subsection{Words and signatures. Crystal operators and Weyl group action.}
The \textit{word} of a KN tableau $T$ is the reading of its entries, column by column, starting from the right most column and reading each column from top to bottom. We will denote the word of $T$ by $word(T)$. For example, if 

\begin{align}
\label{T}
T = \Skew(0:\hbox{\tiny{$1$}},\hbox{\tiny{$2$}}|0: \hbox{\tiny{$\bar 2$}}, \hbox{\tiny{$\bar 1$}} )
\end{align}

\noindent we have $word(T) = 2 \bar 1 1 \bar 2$. For each $1\leq i \leq n$, to a word $w\in \mathcal{P}_{n}$ we assign a labelling of the letters of $w$ by $+,-$ or no label. For $i \leq n-1$, label the letters $i, \overline{i+1}$ by $+$ and the letters $i+1, \overline{i}$ by $-$. If $i = n$, label $n$ by $+$ and $\overline{n}$ by $-$. The remaining letters remain without label. Finally, cancel out pairs of labels of the form $+ -$, that is, cancel out every label $+$ with the first $-$ to its right, starting from the left-most one. For example, if the sequence of labels is 
$-+\hbox{ }--\hbox{ }++ \hbox{ }$ (blank spaces mean no label), after the cancelling out process we obtain $-\hbox{ }\hbox{ }\hbox{ }-\hbox{ }++ \hbox{ }$. Like this, we obtain a sequence of labels which looks like this (after ignoring blank spaces):
\[(-)^r(+)^{s}\]

\noindent for some $r,s \in \mathbb{Z}_{\geq 0}$. This is the \textbf{i-signature} of $w$ (but we also keep a record of the position of the remaining labels). We will denote it by $\sigma_{i}(w)$. For example, the 1-signature of $word(T)$ as in (\ref{T}) is $- - + +$. Its 2-signature is empty. 
To apply the root operator $f_{i}$ to $T$, we replace in $T$ the letter $a$ which is tagged by the left-most $+$ in the i-signature of $word(T)$, by the letter $\bar a$, where $\overline{\bar a } = a$. If $s = 0$, then $f_{i}$ is not defined. To apply $e_{i}$, we replace in $T$ the letter $a$ which is tagged by the right-most $-$ in the i-signature of $word(T)$, by the letter $\bar a$, where $\overline{\bar a } = a$. If $r = 0$, then $e_{i}$ is not defined. 

\subsection{Plactic relations for words. }\label{sec:lecrelations}

Note that the definition of the crystal operators and therefore of the simple reflections make sense on arbitrary words in the alphabet $\mathcal{P}_{n}$. In \cite{lec02} the following plactic relations (R1-3) on words are introduced.


\begin{enumerate}
\item[\mylabel{R1}{$R1$}]
$yzx \sim yxz \hbox{ for } x \leq y < z\hbox{ with } z \neq \bar x \hbox{ and }
xzy \sim zxy \hbox{ for } x < y \leq z \hbox{ with } z \neq \bar x
$;
\medskip
\item[\mylabel{R2}{$R2$}]
$
y \overline{x-1}(x-1) \sim y x \overline{x}  \hbox{ and }
x \overline{x} y \cong \overline{x-1} (x-1)y \hbox{ for }
1 < x \leq n \hbox{ and } x \leq y \leq \bar x
$;
\medskip
\item[\mylabel{R3}{$R3$}] 
$w\sim w\setminus\{z,\overline{z}\}$, where $w\in \mathcal{P}^{*}_{n}$ and $z\in [n]$ are such that $w$ is a non-admissible column, $z$ is the
lowest non-barred letter in $w$ such that $N(z) = z+1$ and any proper factor of $w$ is an admissible column.
\end{enumerate}

\noindent
 These relations  define an equivalence relation $\cong$ on the word monoid $\mathcal{P}^{*}_{n}$. 
Each word $w \in \mathcal{P}^{*}_{n}$ is equivalent via plactic relations to the word of a unique KN tableau $P(w)$. Moreover, there is the following characterization. Let $u,v \in \mathcal{P}^{*}_{n}$ and let $U, V$ the connected components (both normal $U_{q}(\mathfrak{sp}(2n,\mathbb{C}))$-crystals) in which they lie. Then $u \cong v$ if and only if there exists a crystal isomorphism $\eta: U \rightarrow V$ such that $\eta (u) = v$.


\subsection{Weyl group actions and modified crystal operators}

Let $\sigma_{i}(word(T))=(-)^r(+)^{s}$ be the $i$-signature of $word(T)$ as defined in the previous paragraph.
To apply the simple reflection $s_{i}$ to $T$ do the following: 

\begin{itemize}
 \item If $r = s$, then $s_{i}(T) = T$.
 \item If $r>s, s_{i}(T) = e^{r-s}_{i}(T)$.
 \item If $s>r, s_{i}(T) = f^{s-r}_{i}(T)$.
\end{itemize}

Let $x = s_{i_1}\cdots s_{i_r} 
 \in W$. The action of $x$ on a KN tableau $T$ is defined by 
 
 \[ s_{i_1}(\cdots (s_{i_r}(T))).\]

 More generally, given a crystal $B$ there is an action of the Weyl group $W$ on $B$ where $s_i$ acts by reversing the $f_i$-string, i.e. for $T\in B$ with $r=\eps_i(T)$ and $s=\phi_i(T)$, we define $s_i(T)$ as $e_i^{r-s}(T)$ if $r\geq s$ and $f_i^{s-r}(T)$ if $s\geq r$.


For a proof that this defines an action of $W$ also on the weights, see \cite[Proposition 2.36]{BSch17}.  For any $x \in W$ we have $x(\wt(T))=\wt(x(T))$.

\begin{definition}
\label{def:modified}
    In analogy with \cite{Pat}, we introduce the modified crystal operator $e_{12}:=s_1 e_2 s_1$ and $f_{12}:=s_1 f_2 s_1$. 
\end{definition}

\begin{remark}
Unfortunately, we cannot just define $e_{21}$ as $s_2 e_1 s_2$ to be the modified crystal operator attached to the root $\alpha_{21}$. In fact, in our inductive procedure we need the crystal operator  to be constructed by conjugating the root of higher index, but it is not possible here since $\alpha_{21}$ and $\alpha_2$ lie in different orbits under the Weyl group ($\alpha_2$ is long while $\alpha_{21}$ is short).  One of the main hurdles of generalizing the charge statistic from type $A$ to type $C$ is in fact to find an appropriate replacement for this crystal operator in the charge formula.
\end{remark}
\subsection{Adapted strings}

There are two reduced expressions for the longest element $w_0$ of type $C_2$: $s_1s_2s_1s_2$ and $s_2s_1s_2s_1$. 
After fixing a reduced expression $\sigma=s_{i_1}s_{i_2}s_{i_3} s_{i_{4}}$ of $w_0$, an element $T\in\calB(\lam)$ is uniquely determined by a quadruple of non-negative integers $\str_\sigma(T)=(a,b,c,d)$, called the adapted string, such that $T = f_{i_1}^{a}f^{b}_{i_2}f_{i_3}^{c}f_{i_4}^{d}(v_{\lambda})$, where $v_{\lambda} \in \calB(\lambda)$ is the highest weight vertex. 
We abbreviate $\str_{s_1s_2s_1s_2}$ as $\str_1$ and $\str_{s_2s_1s_2s_1}$ as $\str_2$.
The adapted strings for each of the different reduced expressions form a cone, denoted by $C_1$ and $C_2$. The precise relation between these two cones has been given by Littelmann.

\begin{theorem}[{\cite[Prop. 2.4]{conescrystalspatterns}}]
\label{adaptedstring}
There exists piecewise linear mutually inverse bijections 
  $\theta_{12}: C_{1} \rightarrow C_{2}$ and $\theta_{21}: C_{2} \rightarrow C_{1}$, such that $\theta_{12}\circ \str_1=\str_2$ and $\theta_{21}\circ \str_2=\str_1$, given by $\theta_{12}(a,b,c,d) = (a',b',c',d')$, where
\begin{align*}
    a' &= \operatorname{max}\left\{d, c-b,b-a \right\} \\
    b' &= \operatorname{max}\left\{c,a-2b+2c,a+2d \right\} \\
    c' &= \operatorname{min}\left\{b, 2b-c+d, a+d\right\} \\
    d' &= \operatorname{min}\left\{a, 2b-c, c-2d \right\},
\end{align*}
\noindent 
and $\theta_{21}(a,b,c,d) = (a',b',c',d')$, where
\begin{align*}
    a' &= \operatorname{max}\left\{d, 2c-b, b-2a\right\} \\
    b' &= \operatorname{max}\left\{c, a+d, a+2c-b \right\} \\
    c' &= \operatorname{min}\left\{ b, 2b-2c+d,d+2a \right\} \\
    d' &= \operatorname{min}\left\{a, c-d, b-c \right\}.
\end{align*}
\end{theorem}


Moreover, Littelmann precisely characterizes the adapted strings which occur in a given crystal $\calB(\lam)$.


\begin{theorem}[{\cite[Corollary 2, Prop. 1.5]{conescrystalspatterns}}]
\label{Littelmannineq}
Let $\lambda=\lambda_1\varpi_1+\lambda_2\varpi_2$. Given $(a,b,c,d)\in \bbZ_{\geq 0}^4$, there exists $x\in \calB(\lambda)$ with $\stn(x)=(a,b,c,d)$ if and only if the following inequalities hold:
\begin{itemize}
    \item $b\geq c \geq d$
    \item $d\leq \lambda_1$
    \item $c\leq \lam_2+d$
    \item $b\leq \lam_1-2d+2c$
    \item $a\leq \lam_2+d-2c+b$
\end{itemize}
\end{theorem}