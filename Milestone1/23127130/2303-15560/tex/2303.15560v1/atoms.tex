\section{The atomic and preatomic decompositions}

In this section we introduce some important decompositions of the crystal $\calB(\lam)$.



\subsection{Preatoms}

We start by defining the preatomic decomposition. As we note in \Cref{remarkPreatomic}, the preatoms turn out to be a direct generalization of the LL atoms in type $A$, although they can contain several elements with the same weight.

% In analogy with \cite{Pat} we first introduce a decomposition by taking the $f_2$-closure of $W$-orbits. This is an intermediate step, and in fact an intermediate decomposition, to define an atomic decomposition of the crystal (cf. \cite{LL21} and \cite[\S 3.2]{Pat})
% % We consider a \textit{preatomic} decomposition as follows: in  $\calB(\lambda)$, take the sub-graph defined by the  $f_2$-closure of the $s_1$-orbit.  Each connected component of this sub-graph will be called a \textit{preatomic connected component}. \\

% \begin{definition}
%     Consider the equivalence relation $\sim$ of $\calB(\lam)$ generated by:
%     \begin{itemize}
%         \item $v \sim f_2(v)$ for any $v\in \calB(\lam)$ with $f_2(v)\neq 0$.
%         \item $v \sim s_1(v)$.
%     \end{itemize}
%     We call an equivalences class of $\sim$ a $Wf_2$-\textit{connected component}. 
% \end{definition}
%\Leo{We can probably simplify all and do not talk about connected components at all...}

%Let $J$ be the finite indexing set such that 
%\begin{align}
%\label{preatomicdec}
%\calB(\lambda) = \underset{i \in J}{\bigcup} %\mathcal{A}_{i} 
%\end{align}

%\noindent
 %is our decomposition into $Wf_2$-connected components.



 

%\begin{proposition}
\label{preatoms}
%Let $\lambda = \lambda_{1} \varpi_{1} + \lambda_{2} \varpi_{2}$. Then, if (\ref{preatomicdec}) is our pre-atomic decomposition, the pre-atomic decomposition for $\calB(\lambda +2\varpi_{1})$ is given by

%\begin{align}
%\label{preatomicdecconj}
%\calB(\lambda + 2\varpi_{1}) = \underset{i \in J \cup I}{\bigcup} \mathcal{A}_{i} 
%\end{align}

%\noindent
%where $I$ has cardinality $|I| = 1$ if $\lambda_{1}$ is even, and $I$ has cardinality $|I| = 2$ if $\lambda_{1}$ is odd. 
%\end{proposition}

%\noindent
%We will call the set $\mathcal{P}(\lambda) = \underset{i \in I}{\bigcup}\mathcal{A}_{i}$ the \textit{principal pre-atom} or simply the \textit{pre-atom} whenever there is no room for confusion.


\begin{proposition}
\label{emb}
There is an embedding of crystals $\Phi: \calB(\lambda) \rightarrow \calB(\lambda + 2 \varpi_{1})$.  
\end{proposition}

\begin{proof} 
We define the map $\Phi$ on Kashiwara-Nakashima tableaux as follows. Note that  since $n = 2$, all tableaux will have at most two rows. Let $T$ be a Kashiwara-Nakashima tableaux of shape a partition $[a,b]$. Then we replace the first row of $T$, say $r^{1} = \Skew(0:\hbox{\tiny{$r^{1}_1$}} , ... , \hbox{\tiny{$r^{1}_k$}})$, by $\Skew(0:\hbox{\tiny{$1$}},\hbox{\tiny{$r^1_1$}} , ... , \hbox{\tiny{$r^{1}_k$}}, \hbox{\tiny{$\bar 1$}})$. The resulting tableau will be denoted by $\Phi'(T)$. 

If $\Phi'(T)$ contains the column $\Skew(0:\hbox{\tiny{$ 1$}}|0: \hbox{\tiny{$\bar 1$}})$, we replace it with the column $\Skew(0:\hbox{\tiny{$2$}}|0: \hbox{\tiny{$\bar 2$}})$. The new tableau will be denoted by $\Phi(T)$. Note that by semi-standardness, since $T$ does not contain a column $\Skew(0:\hbox{\tiny{$ 1$}}|0: \hbox{\tiny{$\bar 1$}})$, $\Phi'(T)$ can contain at most one such column.

The map $\Phi$ is well defined:
the tableau $\Phi'(T)$ is clearly semi-standard; this implies that, in case $\Phi'(T) \neq \Phi(T)$, then the latter must be semi-standard as well. Assume then that $\Phi'(T) \neq \Phi(T)$. The $1$ in the column $\Skew(0:\hbox{\tiny{$ 1$}}|0: \hbox{\tiny{$\bar 1$}})$ of $\Phi'(T)$ is necessarily the right-most one, so all entries to its right in $\Phi'(T)$ must be strictly larger than $1$. In the second row of $\Phi'(T)$, the $\bar 1$ which  is replaced by $\bar 2$ to obtain $\Phi(T)$ has to be the left-most one, since otherwise $\Phi'(T)$ would contain the column $\Skew(0:\hbox{\tiny{$ 1$}}|0: \hbox{\tiny{$\bar 1$}})$, which is impossible, since it is not an admissible column. The last thing missing to check in order to establish that $\Phi(T)$ is indeed a KN tableau is that it does not contain as a sub tableau $\Skew(0:\hbox{\tiny{$ 2$}},\hbox{\tiny{$2$}}|0:\hbox{\tiny{$\bar 2$}}, \hbox{\tiny{$\bar 2$}})$. But this is impossible, because then $\Phi'(T)$ would necessarily have to contain $\Skew(0:\hbox{\tiny{$ 1$}},\hbox{\tiny{$2$}}|0:\hbox{\tiny{$\bar 1$}}, \hbox{\tiny{$\bar 2$}})$ as a sub-tableau, which is not semi-standard. Note  that, by construction, $\Phi$ is weight-preserving. The case $\Phi'(T) = \Phi(T)$ is left to the reader, as the arguments are very similar to the ones above. \\

It remains to show that $\Phi$ is injective and that it commutes with the crystal operators. We start with a lemma.

% Next we show that the map $\Phi$ preserves $Wf_2$-connected components, that is, it commutes with the crystal operator $f_{2}$ and the simple reflection $s_{1}$. In fact we will prove something stronger: $\Phi$ commutes with the crystal operators, whenever they are defined. We start with a lemma. 

\def \Phio {\Phi^{-\bar 1}}
\begin{lemma}
\label{importantembtool}
Let $T$ be a Kashiwara-Nakashima tableau, and let $w = \word(T)$ be its word. Then the word $\bar 1 w 1$ is plactic equivalent to $\word(\Phi(T))$. Moreover, the word $\bar 1 w 1$ is plactic equivalent to $\bar 1 u$, where $u$ is a word plactic equivalent to $w1$.
\end{lemma}

\begin{proof}

Let $r,s$ be positive integers such that the second row of $T$ has length $s$ and the first row, $s+r$. Let $a_{1} \leq \cdots \leq a_{r+s}$ be the entries in the first row, respectively $b_{1} \leq \cdots \leq b_{s}$ the entries in the second row of $T$ (if any).

Adding a $\bar 1$ at the end of the first row of a tableau just adds a $\bar 1$ at the beginning of its word. Let $\Phio(T)$ be the tableau obtained by removing the rightmost $\bar 1$ from $\Phi(T)$. It is then enough to show that $\word(\Phio(T))\cong w 1$. 

We actually prove a slightly stronger statement by induction on $s$: we have $\word(\Phio(T))\cong w 1$ and $\word(\Phio(T))=a_{r+s}v$, for some $v\in \calP_2^*$.

  
%  If $s=1$, we have 
% \begin{align*}
% w 1 =& a_{r+1}a _r\ldots a_{1} b_{1} 1 \cong a_{r+1}a _r\ldots a_{1} 1 b_{1} 
% \end{align*}
% \noindent
% by \ref{R1} 
% unless $b_1 = \bar 1$, in which case by \ref{R2} we have
% \begin{align*}
% w 1 =&a_{r+1}a _r\ldots a_{1} 2 \bar 2.
% \end{align*}
% Notice that the case $b_1=\bar 1$ precisely occurs when $\Phi'(T)$ contains the column $\Skew(0:\hbox{\tiny{1}}|0:\hbox{\tiny{$\bar 1$}})$. We see that in both cases we have $w1\cong \word(\Phio(T))$.

  The claim is clear if $s=0$. For $s>0$, consider  the tableau $U$ consisting of the first $s-1$ columns of $T$ and let $u=\word(U)$. We have $w=a_{r+s}\ldots a_{s+1}a_sb_s u$ and by induction we have
\[w1\cong a_{r+s}\ldots a_{s+1}a_sb_s\word(\Phio(U))=a_{r+s}\ldots a_{s+1}a_sb_s a_{s-1}v.\]

Since $a_{s-1}\leq a_s<b_s$ by Relation \ref{R1} in \S \ref{sec:lecrelations} we have 
\[a_{s}b_{s}a_{s-1} \cong a_{s} a_{s-1} b_{s}\]
\noindent except for when $b_{s} = \overline{a_{s-1}}$. Assume then we are in this case. Note that $b_{s} = \bar 2$ is impossible since semi-standardness alone then implies that $a_{s-1} = a_{s} = 2$ and $b_{s-1} = b_{s} = \bar 2$ but the tableau $\Skew(0:\hbox{\tiny{$2$}}, \hbox{\tiny{$2$}} |0: \hbox{\tiny{$\bar 2$}},\hbox{\tiny{$\bar 2$}})$ is not KN. Therefore the only option is $b_{s} = \bar 1$ and $a_{s-1} = 1$. In this case we have $a_s\in \{2,\bar 2\}$ and Relation \ref{R2} tells us that 
\[a_{s}\bar 1 1 \cong a_{s} 2 \bar 2.\]
Notice that the case $b_s=\bar 1$ precisely occurs when the $s$-th column of $\Phi'(T)$ is $\Skew(0:\hbox{\tiny{1}}|0:\hbox{\tiny{$\bar 1$}})$ and is replaced by $\Skew(0:\hbox{\tiny{2}}|0:\hbox{\tiny{$\bar 2$}})$ in $\Phi(T)$. From this we observe that in both cases we have $w1\cong \word(\Phio(T))$.
\end{proof}

% \noindent  Our assertion now follows by induction on $s$.  The base $s = 0$ is trivial. For $s>0$ we can assume without loss of generality that $r = 0$. If $s=1$, we have 
% \begin{align*}
% \bar 1 w 1 =&\bar 1  a_{1} b_{1} 1 \cong \bar 1  a_{1} 1 b_{1} 
% \end{align*}
% \noindent
% unless, by the argument above, $b_1 = \bar 1$, in which case
% \begin{align*}
% \bar 1 w 1 =&\bar 1 a_{1} b_{1} 1 \cong \bar 1 a_{1} 2 \bar 2.
% \end{align*}
% \noindent We therefore conclude that for $s=0,1$ the statement of \Cref{importantembtool} holds, since $b_{1} = \bar 1$ precisely when $\Phi'(T)$ contains the column $\Skew(0:\hbox{\tiny{1}}|0:\hbox{\tiny{$\bar 1$}})$.  
% For the inductive step, assume $s>1$. By the induction hypothesis we know that 

% \[\bar 1 a_{s-1}b_{s-1} \cdots a_{1}b_{1} 1 \cong \bar 1 u\]

% \noindent 
% where $u$ is a word which is plactic equivalent to
% $a_{s-1}b_{s-1} \cdots a_{1}b_{1}1$. Moreover by the discussion above, we know that $u = a_{s-1}v$. Therefore

% \begin{align*}
% \bar 1 w 1 &= \bar 1 a_s b_s \cdots a_1 b_1 1\\
%      &\cong \bar 1 a_s b_s a_{s-1} v.
% \end{align*}

% \noindent The statement of \Cref{importantembtool} follows from the argument given at the beginning of the proof.

We now go back to the proof of \Cref{emb}. From \Cref{importantembtool} we see immediately that $\Phi$ is injective.
Let $T$ be a KN tableau and $w = \word(T)$. We have $\sigma_{1}(\bar 1 w 1) = - \sigma_{1}(w) +$. This implies that, if $f_{1}$ is defined on $ w$ then it is also defined on $\bar 1 w 1$ and 
\begin{align}
\label{emb1}
f_{1}(\bar 1 w 1) = \bar 1 f_{1}(w) 1
\end{align}
Similarly, if $e_1(w)$ is defined, then $e_1(\bar 1 w 1) = \bar 1 e_1(w) 1$. 
We know by Lemma \ref{importantembtool}
that $\bar 1 w 1 \cong \word(\Phi(T))$ therefore
\begin{align}
\label{emb2}
     f_{1}(\word(\Phi(T))) \cong f_{1}(\bar 1 w 1) = \bar 1 f_{1}(w) 1\cong \word(\Phi(f_{1}(T))).
\end{align}

This implies that, since $ f_{1}(\Phi(T)), e_{1}(\Phi(T)), \Phi(e_{1}(T))$ and $\Phi(f_{1}(T))$ are $KN$ tableaux, we have
\begin{align}
\label{emb3}
f_{1}(\Phi(T)) = \Phi(f_{1}(T))\qquad
e_{1}(\Phi(T)) = \Phi(e_{1}(T))
\end{align}

\noindent as desired. Now, $\sigma_{2}(\bar 1w 1) = \sigma_{2}(w)$ by definition, so $e_2$ and $f_2$ are defined on $\Phi(T)$ if and only if are defined on $T$. Hence $f_2(\word(\Phi(T))=\word(\Phi(f_2(T)))$ and \eqref{emb3} hold after replacing $f_{1}$ by $f_{2}$ and $e_{1}$ by $e_{2}$. 
\end{proof}



\begin{corollary}
\label{corplactic}
Given a KN tableau $T$, the new tableau $\Phi(T)$ is defined by first column inserting the letter $1$ into $T$ using symplectic insertion and subsequently adding a $\bar 1$ at the end of the first row. 
\end{corollary}

\begin{proof}
The proof follows immediately from Lemma \ref{importantembtool}.
\end{proof}


\begin{corollary}\label{preatomsclosed}
The complement of $\Ima(\Phi)$ is closed under the action of $W$, under $e_2$ and  under outwards  $e_1$, i.e.
if $T\not \in \Ima(\Phi)$ and $\langle wt(T),\alpha_1^\vee\rangle\geq 0$ and $e_1(T)\neq 0$, then $e_1(T)\not \in \Ima(\Phi)$.
\end{corollary}
\begin{proof}
Since $\Phi$ commutes with $W$, its complement is union of $W$-orbits.
Let $T\not \in \Ima(\Phi)$.
We know that $\Phi(e_i(T))=e_i(\Phi(T))$ if $e_i(T)\neq 0$. %If $e_{i}(T) = 0$ then $f_{i}(T) \neq 0$ and $s_{i}(T) = f^{\Phi_{1}(T)}$. If $T$ is in the middle of its $i-$string then it is fixed by $s_{i}$. Note that in this case $\Phi(T)$ is also in the middle of its $i-$string in $\calB(\lambda)$. This follows from the action of $s_{i}$ on the weights and because $i-$strings are multiplicity-free. Therefore we have $\Phi(s_i(T))=s_i(\Phi(T))$.

 Assume $e_2(T)\neq 0$. If $e_2(T)=\Phi(T')$, then it follows from $\sigma_2(\Phi(w))=\sigma_2(w)$, that $f_{2}(T') \neq 0$ and therefore $T = f_{2}(\Phi(T')) = \Phi(f_{2}(T'))$, which is impossible.

Assume $e_1(T)\neq 0$ and $\langle \wt(T),\alpha_1^\vee\rangle\geq 0$. Assume $e_{1}(T) = \Phi(T')$. Since $\langle \wt(T'),\alpha_1^\vee\rangle =\langle \wt(T),\alpha_1^\vee\rangle +2 >0$, we have $f_1(T')\neq 0$, hence $T=f_1(\Phi(T'))=\Phi(f_1(T'))$, which is impossible.
% We have in this case $\sigma_{1}(\Phi(w)) = - \sigma_{1} +$. Therefore if we assume that $e_{1}(T) = \Phi(T')$ we have two options. Either $f_{1}(T') \neq 0$ and we argue as before that then $T = f_{1}(\Phi(T')) = \Phi(f_{1}(T'))$ or $f_{1}(T') = 0$ which contradicts our assumption that $\langle wt(T),\alpha_1^\vee\rangle\geq 0$. Note that in this case necessarily ($w \cong \word(T)$) we have that $\sigma_{1}(w) = u -$ and $\sigma_{1}(e_{1}(w)) = u +$. 
\end{proof}

%\Jaz{It seems to me that from this last argument follows that the complement in the i-string of $\calB(\lambda)$ of the image under $\phi$ of an 1-string in $\calB(\lambda - 2\varpi_{1})$ has cardinality at most two.  This would mean that preatoms are made up of: a) whole strings untouched by the image of $\phi$, and b) symmetric end-bits of 1-strings, at most two per string, one in each "outwards" direction. Does this make any sense?}

\begin{remark}
\label{remarkPreatomic}
    In analogy with \cite[Definition 3.18]{Pat} we can consider the connected components obtained as  $f_2$-closure of the $W$-orbits in the crystal graph.
    From \Cref{preatomsclosed} we see that preatoms are unions of the $f_2$-closure, and moreover, it turns out that for most $\lam$ (i.e. for $\lam_1>0$) each preatom consists of exactly one or two connected components, depending on the parity of $\lam_1$. In this sense, we can think of the preatoms in type $C_2$ as a direct generalization of the LL atoms in type $A$.
\end{remark}

%\Leo{I have commented a corollary which we don't need anyomore}
% \begin{corollary}\label{corImPhi}
% Every element in the complement of $\Ima(\Phi)$ is connected via $s_1,f_2$ and outwards $e_1$ to the highest weight element $b_\lambda \in \calB(\lambda)$.
% \end{corollary}
% \begin{proof}
% Let $T \not \in \Ima(\Phi)$. Let $T'\in  \calB(\lambda)$ be an element whose weight is maximal among those reachable from $T$. Since $s_1(wt(T'))=wt(s_1(T'))$, we have $\langle wt(T'),\alpha_1^\vee\rangle >0$.
% If $e_1(T') \neq 0$, then $wt(e_1(T')) = wt(T') + \alpha_1 > wt(T')$, which contradicts the maximality of $wt(T')$. The same happens if $e_2(T') \neq 0$. Finally, if $e_1(T')=e_2(T')=0$, then $T=b_\lambda$.
% \end{proof}


%{\color{red!40} Previously:  }
%Let $T \not \in \Ima(\phi)$. Let $T'\in  \calB(\lambda)$ be an element whose weight is dominant and maximal among those reachable from $T$.
%If $e_1(T')\neq 0$, then $wt(e_1(T'))$ is not dominant, so $e_2(e_1(T'))\neq 0$, contradicting the maximality of $T'$. If $e_2(T')\neq 0$, then $wt(e_2(T'))$ is not dominant, so $s_1(e_2(T'))=e_1^k e_2(T') \neq 0$ (for $k\in \{1,2\}$, contradicting the maximality of $T'$. Finally, if $e_1(T')=e_2(T')=0$, then $T=b_\lambda$.

\begin{definition}
For $\lambda$ such that $\lambda_{1} \geq 2$, we define the \textit{principal preatom} $\calP(\lambda)$ to be the complement of $\operatorname{Im}(\Phi)$ in $\calB(\lambda)$. If $\lambda_{1} \leq 1$, we define $\calP(\lambda):= \mathcal{B}(\lambda)$.

We define the \emph{preatomic decomposition} by induction on $\lam_1$. If $\lam_1\geq 2$, let
$\calB(\lam-2\varpi_1)=\bigsqcup \calP(\mu_i)$ be the preatomic decomposition. Then, the preatomic decomposition of $\calB(\lambda)$ is
\[ \calB(\lambda) = \calP(\lambda) \sqcup \bigsqcup \Phi(\calP(\mu_i)).\]
\end{definition}

Notice that all the preatoms in $\calB(\lam)$ are image of a principal preatom $\calP(\lam-2k\varpi_1)$ under the map $\Phi^k$. By \Cref{emb1str}
 we see that each preatom has a unique element of maximal weight. In particular, for any $\lam\in X$ every preatom of highest weight $\lam$ is isomorphic via some power of $\Phi$ to the principal preatom $\calP(\lam)\subset \calB(\lam)$. 
%We abuse notation and simply write $\calP=\calP(\lam)$ in this case.


%\Leo{Do we know that preatom contain one or two connected components? Do we actually need to show it?}


% Now, let $C_{1}, C_{2}$ be the cones of adapted strings for the reduced expressions 
% \[w_{0} = s_{1}s_{2}s_{1}s_{2}\hbox{ and }w_{0} = s_{2}s_{1}s_{2}s_{2}\]

% \noindent respectively. We will use Proposition  \cite[Prop.2.4]{conescrystalspatterns}, which states that 
%  the piecewise linear mutually inverse bijections 
%   $\theta_{12}: C_{1} \rightarrow C_{2}$ and $\theta_{21}: C_{2} \rightarrow C_{1}$ are given by $\theta_{12}(a,b,c,d) = (a',b',c',d')$ where: 
% \begin{align*}
%     a' &= \operatorname{max}\left\{d, c-b,b-a \right\} \\
%     b' &= \operatorname{max}\left\{c,a-2b+2c,a+2d \right\} \\
%     c' &= \operatorname{min}\left\{b, 2b-c+d, a+d\right\} \\
%     d' &= \operatorname{min}\left\{a, 2b-c, c-2d \right\}
% \end{align*}
% \noindent 
% and $\theta_{21}(a,b,c,d) = (a',b',c',d')$, where:
% \begin{align*}
%     a' &= \operatorname{max}\left\{d, 2c-b, b-2a\right\} \\
%     b' &= \operatorname{max}\left\{c, a+d, a+2c-b \right\} \\
%     c' &= \operatorname{min}\left\{ b, 2b-2c+d,d+2a \right\} \\
%     d' &= \operatorname{min}\left\{a, c-d, b-c \right\}
% \end{align*}


% For any $T \in \calB(\lambda)$, we will denote by $str(T) = (a,b,c,d)$ the adapted string for $T$ with respect to the reduced expression of the longest element $w_{0} = s_{1}s_{2}s_{1}s_{2}$, in particular $T = f_{1}^{a}f^{b}_{2}f_{1}^{c}f_{2}^{d}(v_{\lambda})$, where $v_{\lambda} \in \calB(\lambda)$ is the highest weight vertex.

\begin{proposition}\label{emb1str}
Let $T \in \calB(\lambda)$ and consider $\Phi : \calB(\lambda) \rightarrow \calB(\lambda + 2\varpi_{1})$. Then 
\begin{enumerate}
    \item If $str_1(T) = (a,b,c,d)$, we have $str_1(\Phi(T)) = (a+1,b+1,c+1,d)$.
    \item If $\str_2(T) = (a,b,c,d)$ we have $\str_2(\Phi(T))=(a,b+1,c+1,d+1)$.
\end{enumerate}
\end{proposition}

\begin{proof}
If $T = v_{\lambda}$ is the highest weight vector, then it follows from Lemma \ref{importantembtool} that 
\begin{align}
\label{embhwstring2}
\Phi(T) = f_{1}f_{2}f_{1}(v_{\lambda + 2\varpi_{1}}).
\end{align}
\noindent
In this case $str_1(T) = str_1(v_{\lambda}) = (0,0,0,0)$ so the claim follows since $(1,1,1,0)$ is an adapted string for $\Phi(T)$. For arbitrary $T \in \calB(\lambda)$ it follows from \Cref{emb} that 
\begin{align}
    \label{embstringbeta}
\Phi(T) = f_{1}^{a}f^{b}_{2}f_{1}^{c}f_{2}^{d}f_{1}f_{2}f_{1}(v_{\lambda + 2\varpi_{1}}). 
\end{align}
 We introduce the following notation: 
  \begin{align*}
 (a',b',c',d') &:=\str_1(f^{d}_{2}f_{1}f_{2}f_{1}(v_{\lambda + 2\varpi_{1}}))=\theta_{21}(d,1,1,1)\\
(a'',b'',c'',d'') &:=\str_1(f^{a'+c}_{1}f^{b'}_{2}f^{c'}_{1}f^{d'}_{2}(v_{\lambda + 2\varpi_{1}}))\\
(a''',b''',c''',d''') &:=\str_2(f^{a''+b}_{2}f^{b''}_{1}f^{c''}_{2}f^{d''}_{1}(v_{\lambda + 2\varpi_{1}})).
\end{align*}
 
By \Cref{adaptedstring}, we have $(a',b',c',d')=\theta_{12}(d,1,1,1)=(1,d+1,1,0)$. Moreover, it follows from \cite[Cor. 2, ii.]{conescrystalspatterns} that 
$(a'',b'',c'',d'')=(0,c+1,d+1,1)$ and $(a''',b''',c''',d''')=(1,b+1,c+1,d)$.
Putting all of this together we get that 
\begin{align*}
\Phi(T) &= f_{1}^{a}f^{b}_{2}f_{1}^{c}f_{2}^{d}f_{1}f_{2}f_{1}(v_{\lambda + 2\varpi_{1}}) \\
& = f_{1}^{a}f^{b}_{2}f_{1}^{a'+c}f_{2}^{b'}f_{1}^{c'}f_{2}^{d'}(v_{\lambda + 2\varpi_{1}}) \\
& = f_{1}^{a}f^{b+a''}_{2}f_{1}^{b''}f_{2}^{c''}f_{1}^{d''}(v_{\lambda + 2\varpi_{1}}) \\
& = f_{1}^{a+a'''}f^{b'''}_{2}f_{1}^{c'''}f_{2}^{d'''}(v_{\lambda + 2\varpi_{1}}). 
\end{align*}
Therefore
\[(a+a''',b'',c''',d''') = (a+1,b+1,c+1, d) = \str_1(\Phi(T)),\]
showing the first statement. The proof of the second statement is similar. It follows from Lemma \ref{importantembtool} that 
\begin{equation}
 \label{embhwstring}
\Phi(T) = f_{1}f_{2}f_{1}(v_{\lambda + 2\varpi_{1}}),    
\end{equation}
so that $\str_2(\Phi(v_{\lambda + 2\varpi_{1}})) = (0,1,1,1)$. Using \cite[Prop. 2.4]{conescrystalspatterns}
we get that 
\begin{align*}
    \Phi(T) &= f^{a}_{2}f^{b}_{1}f^{c}_{2}f^{d}_{1}(f_{1}f_{2}f_{1}(v_{\lambda + 2\varpi_{1}})) \\
    & = f^{a}_{2}f^{b}_{1}f^{c}_{2}f^{d+1}_{1}f_{2}f_{1}(v_{\lambda + 2\varpi_{1}})\\
    & = f^{a}_{2}f^{b}_{1}f_{1}f^{c+1}_{2}f^{d+1}_{1}(v_{\lambda + 2\varpi_{1}}) \\
    & = f^{a}_{2}f^{b+1}_{1}f^{c+1}_{2}f^{d+1}_{1} (v_{\lambda + 2\varpi_{1}}).
\end{align*}
This concludes the proof. 
\end{proof}

\begin{remark}
    Notice that one can avoid the recourse to tableaux combinatorics and use the equation in \Cref{emb1str} as the definition of $\Phi$. Then one can use the explicit description of the adapted strings in \Cref{adaptedstring} to the check that $\Phi$ is well defined and that has the desired properties.
\end{remark}
% We have a similar statement when we use adapted strings with respect to the reduced word $\stn:=str_{(2,1,2,1)}$
% \begin{corollary}
% Let $T \in \calB(\lambda)$ and consider $\Phi : \calB(\lambda) \rightarrow \calB(\lambda + 2\varpi_{1})$. Then if
% $\stn(T) = (a,b,c,d)$, we have $\stn(\Phi(T)) = (a,b+1,c+1,d+1)$. 
% \end{corollary}
% \begin{proof}


The description of the embedding $\Phi$ in terms of adapted strings allows us to give a convenient description of the elements in principal preatom $\calP(\lambda)\subset \calB(\lam)$.

\begin{corollary}
\label{preatominequalities}
There exists $T\in \calP(\lambda)$ with $\stn(T)=(a,b,c,d)$ if \textbf{and only if} all the inequalities in \Cref{Littelmannineq} hold and at least one of the following equations hold.%\Leo{Does this also work for $\lam_1\leq 1$?}\Jaz{Yes, because then $0 \leq d \leq \lambda_1$}
\begin{itemize}
    \item $d=0$
    \item $d=\lam_1$
    \item $b= \lam_1-2d +2c$
\end{itemize}
\end{corollary}
\begin{proof}
Let $T\in\calB(\lam)$ with $\str_2(a,b,c,d)$, so all the inequalities in \Cref{Littelmannineq} hold.
There exists $U\in \calB(\lambda-2\varpi_1)$ with $\str_2(U)=(a,b-1,c-1,d-1)$ so that $\Phi(U)=T$ if and only if all the inequalities in \Cref{Littelmannineq} hold for $(a,b-1,c-1,d-1)$ and $\lambda-2\varpi_1$, which written explicitly means that $d\geq 1$, $d\leq \lambda_1-1$ and $b\leq \lam_1-2d+2c-1$ (the others remain unchanged). The claim now easily follows for $\lambda_{1} \geq 2$. 
\end{proof}

\begin{definition}\label{preatomicnumber}
Let $T\in \calB(\lambda)$.
Let $\pat(T)\in \bbZ_{\geq 0}$ be such that $T\in \calP(\lambda-2\pat(T)\varpi_1)\subset \calB(\lambda)$. We call $\pat(T)$ the \emph{preatomic number} of $T$.

In other words, $\pat(T)$ is the maximum integer with $T\in \Ima(\Phi^{\pat(T)})$.
\end{definition}

We now compute the size of the preatoms using the precanonical bases from \Cref{sec:precan}.

\begin{definition}
Let $\calB^+(\lam)$ be the subset of $\calB(\lam)$ consisting of elements whose wight is dominant. For a subset of $C\subset \calB^+(\lam)$ we define the \emph{ungraded character of $C$} as 
 \[ [C]_{v=1} := \sum_{c\in C} e^{\wt(c)} \in \bbZ[X_+]\]
More generally, for a subset $C\subset \calB(\lam)$ stable under the $W$-action we define 
  \[ [C]_{v=1} := [C\cap \calB^+(\lam)]_{v=1}\]

\end{definition}

\begin{proposition}\label{PreatomPrecan}
We have $[\calB(\lam)]_{v=1}=(\undH_{\lam})_{v=1}$ and $[\calP(\lam)]_{v=1}=(\tilN^3_\lam)_{v=1}$.
\end{proposition}
\begin{proof}
The statement about $\calB(\lam)$ follows by the Satake isomorphism (see for example \cite{Knop}). The second statement follows  easily from the definition of $\tilN^3_\lam$. In fact, if $\lam_1\leq 1$ we have $\calB(\lam)=\calP(\lam)$. If $\lam_1\geq 2$ we have $\calP(\lam)=\calB(\lam)\setminus \Phi(\calB(\lam-2\varpi_1))$. Since $\Phi$ is weight preserving and injective, we have 
\[[\calP(\lam)]_{v=1}=[\calB(\lam)]_{v=1}-[\calB(\lam-2\varpi_1)]_{v=1}=(\undH_{\lam}-\undH_{\lam-2\varpi_1})_{v=1}=(\tilN^3_{\lam})_{v=1}.\qedhere\]
\end{proof}

\subsubsection{The preatomic \texorpdfstring{$Z$}{Z} function}

In analogy with \cite[Definition 3.23]{Pat} we define a $Z$ function in type $C$. 
\begin{definition}
For $T\in \calB(\lam)$, let $Z(T):=\phi_1(T)+\phi_2(T)+\phi_{21}(T)$. 
\end{definition}

The $Z$-function is not constant along preatoms but nevertheless can be used to give an explicit formula for the preatomic number.
\begin{proposition}\label{atomicnumber}
% On each preatom, the function $\phi_1(T)+\phi_2(T)+\phi_{21}(T)$ is constant on elements of the same weight.

% If $T$ belongs to a largest preatom of highest weight $\lambda=\lambda_1\varpi_1+\lambda_2\varpi_2$ and if $wt(T)=\mu_1\varpi_1+\mu_2\varpi_2$, then we have
% \[Z(T):=\phi_1(T)+\phi_2(T)+\phi_{12}(T) = \lambda_1+\lambda_2+\mu_1+\mu_2+\max\left(0,\frac{|\mu_1|-\lambda_1}{2}\right).\]

Assume $T\in \calB(\lam)$ and let $\mu:=\wt(T)$. Then we have 
\begin{equation}\label{Zform}Z(T)= \lambda_1+\lambda_2+\mu_1+\mu_2+\max\left(0,\frac{|\mu_1|-\lambda_1}{2}\right)+\pat(T).\qedhere\end{equation}
\end{proposition}

\begin{proof}
We show the claim by induction on $\pat(T)$.
    We first assume $\pat(T)=0$, or equivalently that $T\in \calP(\lam)\subset \calB(\lam)$. Let $(a,b,c,d)=\stn(T)$.  

Let $\bbT=(\bbQ\cup \{+\infty\}, \oplus, \odot)$ be the tropical semiring (see \cite{TropicalBook}), where 
 $x\oplus y=\min(x,y)$ denotes the tropical addition and $x\odot y= x+y$ is the tropical multiplication. We also write fractions in $\bbT$ for the tropical division, i.e. $\frac{x}{y}=x-y$. A tropical polynomial is the function expressing the minimum of several linear functions. A tropical rational function is the difference of two tropical polynomials.

    Our first goal is to reinterpret both sides of \eqref{Zform} as tropical rational functions in $a,b,c,d,\lam_1$ and $\lam_2$.
    For example, $\mu_1$ can be expressed as a tropical rational function: since we have $\mu_1=\lam_1+2a+2c-2b-2d$, we can write $\mu_1=\frac{\lam_1\odot a^{\odot 2} \odot c^{\odot 2}}{b^{\odot 2}\odot d^{\odot 2}}$.
    In the rest of this proof we make the notation lighter by simply writing $xy$ for $x\odot y$ and  $x^n$ for $x^{\odot n}$.
    Since $\pat(T)=0$ we can rewrite the RHS in \eqref{Zform} as 
    \[RHS(T):=\frac{\lam_1^2 \lam_2^2}{b d (1 \oplus \frac{\lam_1 ac}{bd}\oplus \frac{bd}{ac})}=\frac{ac\lam_1^2\lam_2^2}{a^2c^2\lam_1\oplus abcd\oplus b^2d^2}.\]
    Expressing the LHS of \eqref{Zform} is unfortunately a much longer computation. %, for which we need to resort to the help of the computer algebra software SageMath \cite{SageMath}.
    We have $Z(T)=\phi_2(T)\odot \phi_1(T) \odot \phi_{12}(T)$ and
    \begin{itemize}
        \item $\phi_2(T)=\frac{bd\lam_2}{ac^2}$
        \item$\phi_1(T)=\phi_1'\circ \theta_{21}(a,b,c,d)$, where $\phi_1'(a,b,c,d)=\frac{b^2d^2\lam_1}{ac^2}$ and $\theta_{21}$ is as in \Cref{adaptedstring}. 
        \item $\phi_{12}(T)=\phi_2\circ \theta_{12}\circ \sigma_1\circ \theta_{21}(a,b,c,d)$ where $\sigma_1(a,b,c,d)=(\frac{\lam_1b^2d^2}{ac^2},b,c,d)$ is the transformation expressing the   action of the simple reflection $s_1$ on $\str_1$.
    \end{itemize} 
    From this, we can obtain an explicit expression of $Z(T)$ as a tropical rational function. However, this is a rather unfeasible task to do by hand, so we resort to the help of the computer algebra software  \cite{SageMath}. In Sage we can simply compute $Z(t)$ by formally treating its three factors as ordinary rational functions in $\bbQ(a,b,c,d,\lam_1,\lam_2)$.

    Then, to check the claim, we need to show that $Z(T)=RHS(T)$ when $d=0$, $d=\lam_1$ or $b=\lam_1+2d-2c$. In other words, we need to show that, as tropical rational functions on the set of elements of the crystal, we get $Z(T)/RHS(T)=1$ if we specialize $d=1$,\footnote{Recall that $0\in \bbQ$ is the multiplicative unity in $\bbT$} $d=\lam_1$ or $b=\lam_1d^2/c^2$. Again, this can be  checked with the help of SageMath. In \Cref{appendix} we attach the code that proves our claim.

    Assume now $\pat(T)>0$, so $T=\Phi(T')$ for some $T'\in \calB(\lam-2\varpi_1)$. Since $\pat(T)=\pat(T')+1$, by induction it suffices to show that $Z(T)=Z(T')+1$.  From \Cref{emb1str} it follows that $\phi_1(T)=\phi_1(T')+1$ and $\phi_2(T)=\phi_2(T')$. Moreover, we have \[\phi_{12}(T)=\phi_2(s_1(T))=\phi_2(s_1(\Phi(T')))=\phi_2(\Phi(s_1(T')))=\phi_2(s_1(T'))=\phi_{12}(T')\]
    since $\Phi$ commutes with $s_1$, and the claim follows.
\end{proof}


\subsection{Atoms}

The goal of this section is to describe a finer decomposition of $\calB(\lam)$ into atoms.
\begin{definition}
\label{def:atom}
    We call a subset $A\subset \calB(\lam)$ an \emph{atom} if $[A]_{v=1}=(\bfN_\mu)_{v=1}$ for some $\mu \in X_+$. This means that there exists $\mu\in X_+$ such that every weight smaller or equal than $\mu$ in $X$ occurs exactly once as the weight of an element in $A$.

    An \emph{atomic decomposition} is a decomposition of $\calB(\lam)$ into atoms.
\end{definition}



\begin{proposition}
\label{atomemb}
 There is an injective weight-preserving map $\bPsi:\mathcal{P}(\lambda) \hookrightarrow \mathcal{P}(\lambda + \varpi_{2})$. If $\lam_1\neq 0$ then the set $\calA(\lambda+\varpi_2):=\calP(\lambda+\varpi_2) \setminus \bar{\Psi}(\calP(\lambda))$ is an atom. If $\lam_1=0$ then the set $\calA(\lambda+2\varpi_2):=\calP(\lambda+2\varpi_2) \setminus \bar{\Psi}^2(\calP(\lambda))$ is an atom.
\end{proposition}

We divide the proof into several steps. 
We begin by defining a map $\Psi$ directly in terms of the adapted strings (we give in \Cref{sec:psitableaux} an alternative construction in terms of KN tableaux.) The map $\bar{\Psi}$ is then obtaining by making $\Psi$ symmetric  along $s_1$.
Then we prove injectivity in \Cref{omega2} and that the complement is an atom in \Cref{atoms}.


\begin{lemma}
\label{welldefomega2}
Let $T\in \calP(\lambda)$ with $\stn(T)=(a,b,c,d)$. Then we have the following:
\begin{enumerate}
    \item If $d\in \{0,\lambda_1\}$, there exists $U\in \calP(\lambda+\varpi_2)$ with $\stn(U)=(a,b+1,c+1,d)$;
    \item If $d\not\in \{0,\lambda_1\}$, there exists $U\in \calP(\lambda+\varpi_2)$ with $\stn(U)=(a,b,c+1,d+1)$.
\end{enumerate}
\end{lemma}
\begin{proof}
%Assume first that $b\neq \lam_1-2d+2c$.
%Since $T\in\calP(\lambda)$, we have $d=0$ or $d=\lam_1$. 
Assume first $d=0$ and $d=\lam_1$. The Littelmann inequalities for $(a,b+1,c+1,d)$ and $\lambda+\varpi_2$ are implied by the original ones for $(a,b,c,d)$ and $\lambda$, so there exists such $U\in \calB(\lambda+\varpi_2)$. Since $d=0$ or $d=\lam_1$ we also see that $U\in \calP(\lambda+\varpi_2)$.

Assume now $d\neq 0$ and $d\neq \lambda_1$. Since $T\in \calP(\lambda)$ we have $b=\lam_1-2d+2c$. The Littelmann inequalities for $(a,b,c+1,d+1)$ and $\lambda+\varpi_2$ are:
\begin{itemize}
    \item $b\geq c+1 \geq d+1$,
    \item $d+1\leq \lambda_1$,
    \item $c+1\leq \lam_2+1+d+1$,
    \item $b\leq \lam_1-2d+2c$, and
    \item $a\leq \lam_2+d-2c+b$.
\end{itemize}
All these inequalites are implied by the original ones (and by $d\neq \lambda_1$) except $b\geq c+1$. However, if $b<c+1$ then $b=c$ and $c=\lam_1-2d+2c$ or, equivalently, $d=\frac12 (c+\lam_1)$. Since $d\leq c$ and $d< \lam_1$ this is impossible. It follows that there exists $U\in \calB(\lambda+\varpi_2)$ with $\stn(U)=(a,b,c+1,d+1)$. Moreover, $b=\lam_1-2(d+1)+2(c+1)$, so $U\in \calP(\lambda+\varpi_2)$
\end{proof}

\Cref{welldefomega2} ensures that the following function is well defined.
\begin{definition}\label{defPsi}
    We define $\Psi:\calP(\lambda)\raw \calP(\lambda+\varpi_2)$ as follows. Let $T\in \calP(\lam)$ with $\stn(T)=(a,b,c,d)$. Then $\Psi(T)=U$ with
    \[ \stn(U)=\begin{cases}
    (a,b+1,c+1,d) & \text{if }d=0\text{ or }d=\lam_1\\
    (a,b,c+1,d+1) &\text{otherwise}.\end{cases}\]

    We also define $\bPsi:\calP(\lambda)\raw \calP(\lambda+\varpi_2)$ as follows. 
    \[\bPsi(T)=\begin{cases}\Psi(T)& \text{if }\wt(T)_1\leq 0\\
    s_1(\Psi(s_1(T)))&\text{if }\wt(T)_1\geq 0\end{cases}\]
\end{definition}

\begin{lemma}\label{lemmaonPsi}
For $T\in \calP(\lam)$ we have:
\begin{enumerate}
\item $\wt(\Psi(T))=\wt(\bPsi(T))=\wt(T)$
\item $\phi_2(\Psi(T))=\phi_2(T)$.
\item If $f_2(T)\neq 0$ also $f_2(\Psi(T))=\Psi(f_2(T))$.
\item If $e_2(T)\neq 0$ also $e_2(\Psi(T))=\Psi(e_2(T))$.
\item $s_1(\bPsi(T))=\bPsi(s_1(T))$.
\end{enumerate}
\end{lemma}
\begin{proof}
This is clear by the definition of $\stn$.
\end{proof}

\begin{lemma}
\label{omega2}
The maps $\Psi,\bPsi:\calP(\lambda)\raw \calP(\lambda+\varpi_2)$ are injective.
\end{lemma}
\begin{proof}
It is enough to prove the statement for $\Psi$.
Assume $\Psi(T)=\Psi(U)$ with $T\neq U$. Let $\stn(T)=(a,b,c,d)$ and $\stn(T)=(a',b',c',d')$. We can assume that 
$d\not\in\{ 0,\lambda_1\}$, $d'\in \{0,\lam_1\}$ and that \[\stn(\Psi(T))=(a,b,c+1,d+1)=(a',b'+1,c'+1,d').\] 

It follows that $d'=d+1$, $c'=c$ %so $d=\lam_1-1$,
and $b'=b-1$. 
Since \[b-1=b'\leq \lam_1-2d'+2c'=\lam_1-2(d+1)+2c\] it follows that $b\leq \lam_1-2d+2c-1$. But this contradicts the fact that $b=\lam_1-2d+2c$.
\end{proof}

Recall the atomic basis $\bfN=\bfN^2$ of the spherical Hecke algebra from \Cref{sec:precan}.

\begin{proposition}
\label{atoms}
We have $[\calA(\lam)]_{v=1}=(\bfN_{\lam})_{v=1}$. In particular the set $\calA(\lam)$ is an atom.  
\end{proposition}
\begin{proof}
If $\lam_2=0$ we have $\calA(\lam)=\calP(\lam)$, so $[\calA(\lam)]_{v=1}=(\tilN_{\lam})_{v=1}=(\bfN_{\lam})_{v=1}$.

If $\lam_2=1$ and $\lam_1=0$ then we can easily check that $\calB(\lam)$ consists of a single atom. If $\lam_2>1$ and $\lam_1=0$ then we have
$\calA(\lam)=\calP(\lam)\setminus \bPsi^2(\calP(\lam-\varpi_2))$.
Since $\bPsi$ is injective and weight-preserving, we have by \Cref{precanN3} that
\[ [\calA(\lam)]_{v=1}=[\calP(\lam)]_{v=1}-[\calP(\lam-2\varpi_2)]_{v=1}=(\tilN^3_{\lam}-\tilN^3_{\lam-2\varpi_2})_{v=1}=(\bfN_\lam)_{v=1}.\]
Finally, assume $\lam_2 >0$ and $\lam_1>0$. Then, we have $\calA(\lam)=\calP(\lam)\setminus \bPsi(\calP(\lam-\varpi_2))$. Since $\bPsi$ is injective and weight-preserving, we have
\[ [\calA(\lam)]_{v=1}=[\calP(\lam)]_{v=1}-[\calP(\lam-\varpi_2)]_{v=1}=(\tilN^3_{\lam}-\tilN^3_{\lam-\varpi_2})_{v=1}=(\bfN_\lam)_{v=1}.\qedhere\]
\end{proof}



From this we can  obtain an atomic decomposition of $\calB(\lam)$. Because we already know how to decompose $\calB(\lam)$ into preatoms, it is enough to decompose each preatom $\calP(\lam)$ into atoms. If $\lam_2=0$ or if $\lam=(0,1)$ we have $\calP(\lam)=\calA(\lam)$. If $\lam_2>0$ and $\lam\neq (0,1)$ then we have 
\[ \calP(\lam) = \begin{cases}\calA(\lam) \sqcup \bPsi(\calP(\lam -\varpi_2))&\text{if }\lam_1>0\\
\calA(\lam) \sqcup \bPsi^2(\calP(\lam -2\varpi_2))&\text{if }\lam_1=0\\
\end{cases}\]
so, applying $\bPsi$, we obtain an atomic decomposition by induction.

\begin{remark}
It is worth noting that an atomic decomposition can also be obtained by taking the complement of $\Psi$ rather than $\bPsi$. The advantage of using $\bPsi$ is to ensure that atoms are stable under $s_1$. This stability is crucial, as our approach inherently relies on $s_1$-symmetry, as discussed for example in \Cref{21swappable}. It is therefore 
 essential to ensure that the structures we define are compatible with this symmetry.
\end{remark}

\begin{lemma}\label{phi1Psi}
Let $T\in \calP(\lam)$ with $\stn(T)=(a,b,c,d)$. Then 
\[ \phi_1(\Psi(T))=\begin{cases}
\phi_1(T)& \text{if }d=0\text{ and }2a>b>2c\text{ or }d\neq 0,\lam_1\text{ and }b> 2a+d\\
\phi_1(T)+1 \hspace{-2pt}&\text{otherwise}.
\end{cases}\]

Moreover, if $\phi_1(\Psi(T))=\phi_1(T)$ and $\mu_1\leq 0$, then $\phi_1(T)=0$
\end{lemma}


\begin{proof}
Let $\pi_1:\bbZ^4\raw \bbZ$ be the projection onto the first component. Then, we have 
\begin{equation}\label{phi1stn}\phi_1(T)=\pi_1(\theta_{21}(\stn(T)))+(\wt(T))_1=\lam_1+2a-2b+2c-2d+\max(d,2c-b,b-2a).\end{equation}
From here we see that, if $d=0$ or $d=\lam_1$, we have
\[\phi_1(\Psi(T))-\phi_1(T)=\max(d,2c-b+1,b-2a+1)-\max(d,2c-b,b-2a).\]
If $d=0$, then $\phi_1(\Psi(T))=\phi_1(T))$ if and only if $2a>b>2c$. If $d=\lam_1$, we have $2c-b\geq 2d-\lam_1=\lam_1$, so $\phi_1(\Psi(T))-\phi_1(T)=1$.

If $0<d<\lam_1$ and $b=\lam_1-2d+2c$, then \[\phi_1(\Psi(T))-\phi_1(T)=\max(d+1,2c-b+2,b-2a)-\max(d,2c-b,b-2a),\]
but $2c-b=2d-\lam_1<d$, so $\phi_1(\Psi(T))-\phi_1(T)=\max(d+1,b-2a)-\max(d,b-2a)$ and the claim easily follows.
%This follows from \[\phi_1(T)=\lam_1+2a+4c-b-d-\min(2a+2c+d,b+2c,2b+d).\]
\end{proof}


\begin{corollary}\label{cor:phi12psi}
Let $T\in \calP(\lam)$ with $\stn(T)=(a,b,c,d)$. Then 
\[ \phi_{12}(\Psi(T))=\begin{cases}
\phi_{12}(T)+1& \text{if }d=0\text{ and }2a>b>2c\text{ or }d\neq 0,\lam_1\text{ and }b> 2a+d\\
\phi_{12}(T)&\text{otherwise}.
\end{cases}\]
\end{corollary}
\begin{proof}
It follows from \Cref{atomicnumber} that
\[ \phi_{12}(\Psi(T))-\phi_{12}(T)=1-(\phi_1(\Psi(T))-\phi_1(T)),\]
so we conclude by \Cref{phi1Psi}.
\end{proof}




\begin{definition}\label{defatomicnumber}
Let $T\in \calB(\lambda)$.
%\Leo{Should we change $x$ and $y$ to capital letters?} \Jaz{I think it's better to leave them small because $X$ looks like a weight lattice.}\Leo{Ok, I use $T$ instead. (sorry, it was bugging me that we use two different conventions on the same page}
Let $\at(T)\in \bbZ_{\geq 0}$ be the maximum integer such that $T$ is in the image of $\bPsi^{\at(T)}:\calP(\lam-\at(T)\varpi_2)\raw \calP(\lam)$. We call $\at(T)$ the \emph{atomic number} of $T$.
\end{definition}

\begin{proposition}\label{atomformula}
Let $T\in \calP(\lambda)\subset \calB(\lam)$ with $\stn(T)=(a,b,c,d)$ and $\wt(T)_1\leq 0$.  We have
\[ \at(T)=\begin{cases} \min(c,\lam_1+2c-b)& \text{if }d=0\\
\lam_1+2c-2d-b+\min(\lam_2+d-c,d-1) &\text{if }d>0.\end{cases}\]
\end{proposition}

\begin{proof}
Notice that since $\wt(T)_1\leq 0$ we have $\Psi=\bPsi$.

First recall that by \Cref{Littelmannineq}, we have $0 \leq d \leq \lambda_{1}$. If $d = 0$, $\at(T)$ is the maximal amount we can substract simultaneously from $b$ and $c$, decreasing at the same time the value of $\lambda_{2}$ by the same amount, so that the inequalities and equalities mentioned in Corollary \ref{preatominequalities} still hold. Since $b \geq c$, we can focus only on $c$ and the inequality $\lambda_{1}+2c-b \geq 0$, which is the only other inequality describing $\mathcal{P}(\lambda)$ which is affected after reducing $b,c$ and $\lambda_{2}$ in equal amounts. Now if we decrease $b$ and $c$ simultaneously by the same amount, the quantity $\lambda_{1}+2c-b$ decreases by the same amount. Therefore, in this case $\at(T)= \min\left(c,\lambda_{1}+2c-b \right)$ as desired. 

Assume now  $d =\lam_1$. Recall that we need to find the maximal  $\at(T)$ such that the map $\Psi^{\at(T)}(U) = T$ for an element $U \in \mathcal{P}(\lambda-\at(T)\varpi_2)$. Recall that the definition of the map $\Psi$ depends on the value of $d$. Let $\psi_1$ and  $\psi_2$ be the two possible actions on adapted strings defined by $\Psi$, corresponding to the cases $0< d< \lambda_1$ and $d = 0,\lambda_1$ respectively (i.e. we have $\psi_1,\psi_2:\bbZ^4\raw \bbZ^4$ with $\psi_1(a,b,c,d)=(a,b,c+1,d+1)$ and $\psi_2(a,b,c,d)=(a,b+1,c+1,d)$). The definitions imply that we must have 
\[ \str_2(T) = \psi_{2}^{\at_2(T)}\psi_{1}^{\at_1(T)}(\str_2(U))\]
for some $\at_1(T),\at_2(T)\in \bbN$ with $\at_1(T)+\at_2(T)=\at(T)$.

Now, to calculate $\at_1(T)$ we first need to subtract the largest possible amount from $b$, $c$ and $\lambda_{2}$ such that our inequalities and equalities stated in Corollary \ref{preatominequalities} will still hold. Analogously to the case  $d=0$ we can conclude that this number is $\at_{1}(T) = \min(c,\lambda_{1}+2c-b -2d )$. %However, if $d <\lambda_{1}$, it follows from the third equality in Corollary \ref{preatominequalities} that $\at_{1}(T) = 0$. Therefore $\at_{1}(T)$ is only non-zero when $d = \lambda_{1}$. 
In this case the inequality $0 \leq \lambda_{1} + 2c-2d-b$ becomes $0 \leq 2c -\lambda_{1} -b \leq c $ since $c \leq b$. Therefore $\at_{1}(T) = \lambda_{1}+2c-b -2d$. To compute $\at_2(T)$ in this case, after already reducing $b,c$ and $\lambda_{2}$ by $\at_{1}(T)$ we need to further reduce $c' = c - \at_{1}(T)$ as well as $d$ and $\lambda_{2}' = \lambda_{2} - \at_{1}(T)$ by the maximal possible amount strictly smaller than $d$ such that the preatom inequalities/equalities will still hold. This amount is 
\[\at_{2}(T) =  \min \left(\lambda_{2}' +d - c', d-1 \right) = \min \left(\lambda_{2}+d-c, d-1 \right)\]

\noindent
since the inequality $\lambda_{2}+d - c'\geq 0$ is the only preatom inequality affected by decreasing $c,d$ and $\lambda_{2}$ simultaneously by the same amount. Moreover, it decreases precisely by this amount. 

Finally,  assume $0<d < \lambda_1$. As in the discussion above we have 
\[\str_2(T) = \psi_{2}^{\at_2(T)}(\str_2(U)),\]
and thus $\at(T)=\at_2(T)$. Moreover, if $0<d<\lam_1$ we have $b=\lam_1-2c+2d$ so we can also write $\at(T)=\lam_1+2c-2d-b+\at_2(T)=\lam_1+2c-2d-b+\at_2(T)+\min(\lam_2+d-c,d-1)$.
\end{proof}


\begin{corollary}\label{atom0}
Let $U\in \calP(\lambda)\subset \calB(\lam)$ with $\stn(U)=(a,b,c,d)$. Then $U \not \in \Psi(\calP(\lambda-\varpi_2))$ if and only if one of the following two conditions holds:
\begin{itemize}
    \item $b=\lam_1-2d+2c$ and ($d\leq 1$ or $c=\lam_2+d$)
    %\Jaz{Maybe delete if not used later on!}\Leo{We need it in 5.10, so we can leave it}
    ;
\item  $b<\lam_1-2d+2c$ and $c=d=0$.
\end{itemize}

\end{corollary}

\begin{proof}
We know that $U \notin \Psi(\mathcal{P}(\lambda-\varpi_2)) \iff \operatorname{at}(U) = 0$. First assume $\at(U) = 0$. If $b = \lambda_1 - 2d +2c$ then from \Cref{atomformula} we see that either $d \leq 1$ or if $d>1$, we must have $\min(\lambda_2 +d-c, d-1) = 0$. Since $d >1$ this implies that $\lambda_2 +d - c = 0$. If $b < \lambda_1 - 2d +2c$ then by \Cref{atomformula} $d>0$ is impossible, so $d = 0$ necessarily. Moreover, since $\at(U) = 0$ we must have $\operatorname{min}(c, \lambda_1 + 2c-b)$, but since the second term is strictly larger than zero by assumption, we conclude $c = 0$. Conversely, if $b = \lambda_1 - 2d +2c$ and $d \leq 1$, it follows directly from \Cref{atomformula} that $\at(U) = 0$. If $c = \lambda_2 + d$ and $d>1$ then $\at(U) = 0$ also by \Cref{atomformula}. Now, if $b < \lambda_1 - 2d +2c$ and $c = d = 0$ then $\at(U) = 0$ applying the first formula in \Cref{atomformula}. 
\end{proof}


\subsection{Example: The atomic decomposition \texorpdfstring{of $\calB(k\varpi_2)$}{of B(kw2)}}
\label{sec:lam1=0}
%\Leo{What should we do of this section?} \Jaz{Great that you made a general proof! What do you think if we leave it as an example? I changed the wording and shortened it a bit.}\Leo{Ok!}
 Let $B_{k} := \calB(k \varpi_{2})$. By definition $B_k$ consists of a single preatom. We describe now the atomic decomposition of $B_k$.
 Since $\lam_1=0$ we have $\str_2(T)=(a,b,c,0)$ for any $T\in B_k$. By \Cref{phi1Psi}, we see that $\phi_1(\Psi(T))=\phi_1(T)$ for any $T\in B_k$, hence $\Psi$ commutes with $s_1$ and we have $\Psi=\bPsi$. Then by \Cref{lemmaonPsi}.3, we see that $\Psi$ also commutes with $f_2$.
 
 Here we refer to the connected components under $W, f_2$ simply as \textit{connected components} (cf. \Cref{remarkPreatomic}). Notice that $\Psi$ preserves these connected components.
 We claim that the crystal $B_{k}$ has precisely $k+1$ connected components
\
\[
    B_{k} = \bigsqcup_{i = 0}^{k} B_{k}[i].
\]
and that $\Psi(B_{k-1}[i])=B_k[i]$.
In particular, it follows that $\calA(k\varpi_2)=B_k\setminus \Psi^2(B_{k-1})=B_k[k]\sqcup B_k[k-1]$.
\noindent

 The crystal $B_0$ consists of a single element, the empty tableau, so the claim is trivial. In $B_1$ there are two connected components. In fact, it is easy to see that 
\[B_{1}[0] = \left\{\Skew(0:\hbox{\tiny{$2$}}|0:\hbox{\tiny{$\bar 2$}}) \right\}\]

\noindent is fixed under the action of $f_{2}$ and $s_{1}$, and that its complement in $B_{1}$ is a connected component of cardinality $4$.

The weights of the elements in $\calA(k\varpi_2)$ form two square grid of side $k$ and $k+1$ as shown in \Cref{figurewts}, so $|\calA(k\varpi_2)|=(k+1)^2+k^2$. From this, it follows that $|B_k|-|B_{k-1}|=(k+1)^2$.


By induction, to show our claim it is enough to show that the complement of $\Psi(B_{k-1})$ in $B_{k}$ is a single connected component of cardinality $(k+1)^2$.


% To see this notice that in this case $d = 0$, so $\Psi (T)$ i given on the level of $str_{2}$ adapted strings by $(a,b,c,0) \mapsto (a,b+1,c+1,0)$.

The complement of $\Psi$ always contains the highest weight vector $T_k\in B_k$. Then, for $0\leq r\leq k$, the tableaux 
\[ f^{r}_{2}(T_{k})=
\Skew(0:\hbox{\tiny{1}},\hbox{\tiny{$\cdots $}}, \hbox{\tiny{1}}, \hbox{\tiny{1}}, \hbox{\tiny{$\cdots$}}, \hbox{\tiny{1}}|0:\hbox{\tiny{2}},\hbox{\tiny{$\cdots$}},\hbox{\tiny{2}},\hbox{\tiny{$\bar 2$}},\hbox{\tiny{$\cdots$}},\hbox{\tiny{$\bar 2$}} ), \]
in which there are $r$ column of the  form $\Skew(0:\hbox{\tiny{1}}|0:\hbox{\tiny{$\bar2$}})$,  are also in the same connected component as $T_k$. 
 We obtain $s_1(f_2^r(T))$ from $f_2^r(T)$ by replacing the columns of the form $\Skew(0:\hbox{\tiny{1}}|0:\hbox{\tiny{$\bar2$}})$ by columns of the form $\Skew(0:\hbox{\tiny{2}}|0:\hbox{\tiny{$\bar1$}})$.
The  tableaux $s_1(f_2^r(T))$ are the highest element in their $f_2$-string, and there are $k+1$ elements in their $f_2$-orbit,
 given by barring some of the $2$'s. So we have seen that
 there are at least $(k+1)^2$ elements in the connected components of $T_k$. Since $\Psi$ is an embedding and $|B_k|-|B_{k-1}|=(k+1)^2$, these are precisely all the elements in the complement of $\Psi$.
 
 
 % Each one of these tableaux, as long as it contains at  least one column of the form  $\Skew(0:\hbox{\tiny{1}}|0:\hbox{\tiny{$\bar2$}})$, is the highest weight of a $1$-string. In total, there are $k^{2}$ tableaux of this form in $B_{k}$. Note that  their reflections are again tableaux of this form, so that there are no more tableaux to count in the connected component of $B_{k}$ containing $T_k$, which we conclude to be of cardinality $(k+1)^2 = k^2 + 2k +1$. Our initial claims now follow by induction and the fact that $ \bPsi$ is injective.


%That the complement of $\Psi^2$ is an atom in $\calP((k+2)\varpi_2)$ can be seen from the discussion above together with the fact that in $[B_{k}]_{v =1}$ the weights that appear are precisely those of the form $(k-2l,k-2s)$ and $(k-1-2r, k-1-2u)$, for $0\leq l,s \leq k$ and $0 \leq r, u \leq k-1$. One observes then immediately that for any $T$ in this complement it must be true that $at(T) = 0$. 



 \begin{figure}
\begin{center}
 \begin{tikzpicture}[scale=0.5]
\draw[style=black] (3,3) to (-3,3) to (-3,-3) to (3,-3) to cycle;
\foreach \x in {0,1,2,3}{
    \foreach \y in {0,1,2,3}{
        \node[style=sred] at (3-2*\x,3-2*\y) {\small 1}; 
}}
\foreach \x in {0,1,2}{
    \foreach \y in {0,1,2}{
        \node[style=sred] at (2-2*\x,2-2*\y) {\small 2}; 
}}	
\foreach \x in {0,1}{
    \foreach \y in {0,1}{
        \node[style=sred] at (1-2*\x,1-2*\y) {\small 3}; 
}}	
\foreach \x in {0}{
    \foreach \y in {0}{
        \node[style=sred] at (0-2*\x,0-2*\y) {\small 4}; 
}}
 \end{tikzpicture}
 \end{center}
  \caption{The weight multiplicities of the crystal $B_3$.}
  \label{figurewts}
\end{figure}