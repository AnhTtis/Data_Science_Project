\appendix
\section{Proof of \texorpdfstring{\Cref{atomicnumber}}{the preatomic formula} with SageMath}
\label{appendix}
\begin{python}
R.<a,b,c,d,L1,L2>=PolynomialRing(QQ) 
#L1 and L2 represent lam_1 and lam_2
K=R.fraction_field()

def theta12(): 
    X = [a,b,c,d]
    X[0] = 1/K(1/d + b/c + a/b)
    X[1] = 1/K(1/c + b^2/(a*c^2) + 1/(a*d^2))
    X[2] = K(b+b^2*d/c+a*d)
    X[3] = K(a+b^2/c+c/d^2)
    F(a,b,c,d) = tuple(X)
    return F

def theta21():
    X = [a,b,c,d]
    X[0] = 1/K(1/d+b/c^2+a^2/b)
    X[1] = 1/K(1/c+1/(a*d)+b/(a*c^2))
    X[2] = K(b+b^2*d/c^2+a^2*d)
    X[3] = K(a+c/d+b/c)
    F(a,b,c,d) = tuple(X)
    return F

def RRTAux(P):  
#From a tropical polynomials we can remove coefficients bigger than 1.
#Moreover, we are only interested in the function on positive values of a,b,c,d,L1 and L2
#we can remove monomials which are divisible by another monomial, as the minimum is never 
#expressed exclusively by them.
    M = P.monomials()
    R = []
    for i in range(len(M)):
        for j in range(len(M)):
            if M[i].divides(M[j]) and i != j: 
                R.append(j)
    return sum([M[j] for j in range(len(M)) if not j in R])

def RemoveRedundantTerms(X):
    return RRTAux(X.numerator())/RRTAux(X.denominator())

t12 = theta12()
t21 = theta21()
s1(a,b,c,d) = (L1*b^2*d^2/(a*c^2),b,c,d)
phi2(a,b,c,d) = L2*b*d/(a*c^2)
Phi2= K(L2*b*d/(a*c^2))
phi1aux(a,b,c,d) = L1*b^2*d^2/(a*c^2)
Phi1 = K(phi1aux(*t21))
phi12aux1 = s1(*t21)
phi12aux2 = t12(*phi12aux1)
Phi12 = K(phi2(*phi12aux2))
Z = Phi2*Phi1*Phi12
RHS = K(L1^2*L2^2/(b*d*(1+L1*a*c/(b*d)+b*d/(a*c))))
Q = Z/RHS
\end{python}
We first compute the quotient $Q$ on the subset of elements in $\calP(\lam)$ such that $d=0$. 
\begin{python}
f1(a,b,c,d,L1,L2) = (a,b,c,1,L1,L2)
Q1 = RemoveRedundantTerms(K(Q(*f1)(a,b,c,d,L1,L2))) 
# Q(*f1) denotes composition of functions in Sage
print(Q1)
print(Q1.numerator()-Q1.denominator())

\end{python}
\begin{verbatim}
(a^3*c^3*L1 + a^2*c^4*L1 + a^2*b*c^2 + a*b*c^3 + a*b^2*c +
b^2*c^2 + b^3)/(a^3*c^3*L1 + a^2*c^4*L1 + a*c^5*L1 + 
a^2*b*c^2 + a*b*c^3 + a*b^2*c*L1 + b^2*c^2 + b^3)
-a*c^5*L1 - a*b^2*c*L1 + a*b^2*c
\end{verbatim}
There is one extra monomial ($ab^2c$) in the numerator  which does not occur  in the denominator
and two extra monomials ($ac^5\lam_1$ and $ab^2c\lam_1)$ in the denominator.
However, we have 
\begin{itemize}
\item $a + 2b + c + \lam_1 \geq  a + 2b + c \geq  \min(2a+b+2c,3b)$ 
\item
$a + 5c + \lam_1 \geq  a + b + 3c$ (because $b \leq \lam_1 + 2d - 2c)$).
\end{itemize}
Hence the minimum is never expressed by these monomials. So the quotient function $Q$ is constantly zero on the elements of the preatom with $d=0$.

Now we compute the quotient $Q$ on the subset of elements in $\calP(\lam)$ such that $d=\lam_1$.
\begin{python}
f2(a,b,c,d,L1,L2) = (a,b,c,L1,L1,L2)
Q2 = RemoveRedundantTerms(K(Q(*f2)(a,b,c,d,L1,L2)))
print(Q2)
print(Q2.numerator()-Q2.denominator())
\end{python}
\begin{verbatim}
(a^3*c^3*L1^2 + a^2*c^4*L1 + a^2*b*c^2*L1^2 + a*b*c^3*L1 + 
a*b^2*c*L1^2 + b^2*c^2*L1^2 + b^3*L1^3)/(a^3*c^3*L1^2 +
a^2*c^4*L1 + a^2*b*c^2*L1^2+ a*c^5 + a*b*c^3*L1 +
a*b^2*c*L1^2 +b^2*c^2*L1^2 + b^3*L1^3)
-a*c^5
\end{verbatim}
There is one extra monomial in the denominator: $ac^5$.
However, we have 
$a + 5c  \geq a + 2b+ c + 2\lam_1$  (because $b \leq 2c-\lam_1)$).
Hence the minimum is never expressed by this monomial and the tropical function $Q$ is constantly zero when $d=
lam_1$. Finally, we compute $Q$ for $b=\lam_1+2c-2d$.
\begin{python}
f3(a,b,c,d,L1,L2) = (a,L1*c^2/d^2,c,d,L1,L2)
Q3 =RemoveRedundantTerms(K(Q(*f3)(a,b,c,d,L1,L2)))
print(Q3)
print(Q3.numerator()-X.denominator())
\end{python}
\begin{verbatim}
(a^3*d^4 + a^2*c*d^3 + a^2*c*d^2*L1 + a*c^2*d^2+ a*c^2*d*L1 + 
c^3*d*L1 + c^3*L1^2)/(a^3*d^4 +a^2*c*d^3 + a^2*c*d^2*L1 + 
a*c^2*d^2 + a*c^2*d*L1+ c^3*d*L1 + a*c^2*L1^2 + c^3*L1^2)
-a*c^2*L1^2
\end{verbatim}
There is one extra monomial in the denominator: $ac^2\lam_1^2$.
However, we have 
$a + 2c +2\lam_1\geq  a + 2c + d+ \lam_1$  (because $d \leq  \lam_1$). This shows again that $Q$ is zero when $b=\lam_1+2d-2c$. Hence it is always zero on the preatom, concluding the proof of \Cref{atomicnumber} in the case $\pat(T)=0$.



\section{Combinatorial interpretation of the map \texorpdfstring{ \\ $\Psi: \mathcal{P}(\lambda) \rightarrow \mathcal{P}(\lambda + \varpi_{2})$}{the atomic embedding map}}\label{sec:psitableaux}
%\Jaz{What do you think if we include this section as an appendix? Or maybe just leave it out, it's maybe just too confusing and not used in the example I wrote anyway.}\Leo{Ok. Or at least we should start the section by saying that this is not needed elsewhere in the paper}

In this subsection we give a combinatorial description of the map $\Psi$ on Kashiwara--Nakashima tableaux. The reader interested in the thought process behind the present work might want to know that this was the first map $\Psi$ we obtained. Although it is not used in the rest of the paper, we include it for completeness. It may also be used to compute examples by hand directly. 

\begin{proposition}
Let $T \in \mathcal{P}(\lambda) \subset \calB(\lam)$. Then the Kashiwara--Nakashima tableau of $\Psi(T)$ can be obtained from the KN tableau corresponding to $T$ by the following algorithm.
Let $d'$ be the number of $\bar{1}$'s in the first row of $T$.

\begin{itemize}
\item Add the column $\Skew(0:\hbox{\tiny{$1$}} |0: \hbox{\tiny{$2$}} )$ on the left tableau.
\item If $d' = 0$ or $d'=\lam_1$ then
\begin{itemize}
    \item If $T$ has a column of the form $\Skew(0:\hbox{\tiny{$2$}} |0: \hbox{\tiny{$\bar 2$}} )$ (note that $T$ can have at most one such column), then replace this column by $\Skew(0:\hbox{\tiny{$ 2$}} |0: \hbox{\tiny{$\bar 1$}} )$
    \item If $T$ does not have column of the form $\Skew(0:\hbox{\tiny{$2$}} |0: \hbox{\tiny{$\bar 2$}} )$,  then replace the rightmost $1$ in the first row by a $2$.
    \item Replace the rightmost $2$ in the second row by $\bar 2$.
\end{itemize}
\item If $0<d'<\lam_1$ then replace the rightmost $2$ in the first row by a $\bar{1}$. If there is no $2$ in the first row, then replace the rightmost $\bar{2}$ in the first row by $\bar 1$ and the rightmost $2$ in the second row by $\bar 2$.
\end{itemize}



\end{proposition}

\begin{proof}
Let $v_{\lambda}$ and $v_{\lambda + \varpi_{2}}$ be the highest weight tableaux of shapes $\lambda$ and $\lambda + \varpi_{2}$, respectively. 
Let $\stn(T)=(a,b,c,d)$.
Note that, since $b\geq c \geq d$, it follows from the definition of the crystal operators that the letter $\bar 1$ appears in the first row of $T$ precisely $d$ times, so $d' = d$. 

Assume first that $d \in \left\{ 0,\lambda_{1}\right\}$ and recall that $\stn(\Psi(T))=(a,b+1,c+1,d)$. One obtains the above described combinatorial interpretation by making the following observations.
 
 \begin{itemize}
     \item The element $v_{\lambda + \varpi_{2}}$ is obtained from $v_{\lambda}$ by adding to its left the column  $\Skew(0:\hbox{\tiny{$1$}} |0: \hbox{\tiny{$2$}} )$.
     \item When the operator $f_{1}$ is first applied $d$ times to $v_{\lambda}$, it transforms the $d$ leftmost $1$'s into $2's$. Similarly, $f_1^d(v_{\lam+\varpi_2})$ is obtained by replacing the $d$ leftmost $1$ in $v_{\lam+\varpi_2}$ by $2$.
     \item To obtain $f_2^cf_1^d(v_\lam)$, the $2$ which have just been transformed, are replaced again by $\bar 2$. Since $c\geq d$, we must also transform $c-d$ $2$ in the second row into $\bar 2$. Similarly,
     $f_2^{c+1}f_1^d(v_{\lam+\varpi_2})$ is obtained from $f_2^cf_1^d(v_\lam)$ by adding $\Skew(0:\hbox{\tiny{$1$}} |0: \hbox{\tiny{$2$}} )$ on the left and replacing one extra $2$ in the second row by $\bar{2}$.

     \item To obtain $f_1^bf_2^cf_1^d(v_\lam)$ from $f_2^cf_1^d(v_\lam)$ we  transform all the $\bar 2$'s in the first row (if any) into $\bar 1$'s. Then, the rightmost $b-d$ boxes in the columns of size two are transformed according to the crystal rule: $1 \mapsto 2 $ as well as $\bar 2 \mapsto \bar 1$. Therefore if in $T$ there is a column of the form $\Skew(0:\hbox{\tiny{$2$}} |0: \hbox{\tiny{$\bar 2$}} )$, it means that $f_{1}$ modified the top box of this column last, because $f_{2}$ would never modify this column further as its $2$-signature is $+-$. Therefore, if we would apply it one more time, it would modify also the bottom box into $\bar 1$. If there is no such column in $T$, it must therefore mean that the operator $f^{b}_{1}$ either finished at the bottom of one of these columns or was not applied at all to the columns of size two. Applying this operator once more to $f^{b}_{1}f^{c+1}_{2}f^{d}_{1}v_{\lambda + \varpi_{2}}$ would then transform the rightmost $1$ in the first column  into $\bar 2$.
     \item Finally, note that applying the operator $f_{2}^a$ to   $f^{b+1}_{1}f^{c+1}_{2}f^{d}_{1}v_{\lambda + \varpi_{2}}$ or to $f^{b}_{1}f^{c}_{2}f^{d}_{1}v_{\lambda + \varpi_{2}}$ only transforms the letters $2$ into a $\bar 2$ (with the one exception of $2, \bar 2$ being in the same column already described above), hence the changes made by $f^{b}_{1}, f^{b+1}_{1}$ are not modified by $f_{2}^{a}$. 
 \end{itemize}
 
 This finishes the proof in case $d \in \left\{ 0,\lambda_{1} \right\}$. Now assume that $0< d < \lambda_{1}$. From the description above we conclude that $f^{2}_{a}f^{b}_{1}f^{c+1}_{2}f^{d+1}_{1}v_{\lambda + \varpi_{2}}$ is obtained from $f^{c}_{2}f^{d}_{1}v_{\lambda }$ by adding a highest weight column of shape $\varpi_{2}$ at the beginning of $f^{d}v_{\lambda }$  and by replacing the rightmost letter $2$ in the first row, if it exists (note that this is the case if and only if $a<b$), by $\bar 1$, and re-ordering the letters of that row.  \\
 
 If there is no $2$ in the first row, then we claim that the first row must contain at least one letter $\bar 2$. This follows from \Cref{preatominequalities} since in this case we have $b = \lambda_{1}-2d+2c \geq \lambda_{1} > d$. Then from the description above we deduce that $f^{2}_{2}f^{b}_{1}f^{c+1}_{2}f^{d+1}_{1}v_{\lambda + \varpi_{2}}$ is obtained from $f^{c}_{2}f^{d}_{1}v_{\lambda }$ by adding a highest weight column of shape $\varpi_{2}$ at the beginning of $f^{d}v_{\lambda }$  and by replacing the rightmost letter $\bar 2$ in the first row, by $\bar 1$, and, since $a \geq b$, by replacing the rightmost $2$ in tho second row by $\bar 2$. This last action comes from the fact that since a new $\bar 1$ was created in the first column, $f_{1}^{b}$ created one $1 \mapsto 2$ move less (which would then become a $2 \mapsto \bar 2$ by $f_{2}^{a}$). Hence necessarily $f^{a}_{2}$ will create one more move of the form $2 \mapsto \bar 2$ in the second row. 
\end{proof}
