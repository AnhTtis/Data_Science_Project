\section{Conclusion}

%We propose here \cqg{} to generate queries for entities that controls for the underlying intent. 
We propose here \cqg{}, a new approach to generate synthetic queries for entities that allows to control for the query intent and that can work in the absence of annotated data through the use of a weak-labeling function that leverages content metadata. We study the impact that the generated queries have on decreasing the retrievability bias and effectiveness, i.e. on helping the search engine surface more entities while avoiding negative effects on the relevance of results. 
Our experimental results in three different domains show that training dense retrieval models on synthetic queries from \cqg{} leads to significant decreases in the retrievability bias of the system with comparable effectiveness. We also demonstrate how to reduce the retrievability bias by suggesting queries generated by \cqg{}.

As future work, we believe important directions to be: (I) taking into account the interplay between recommendation and search in the measure of the accessibility of an entity, (II) improving the representation of entities for which most metadata information is not available and (III) study methods to reduce the retrievability of a system for re-ranking scenarios (IV) study the impact of increased content retrievability on content discovery.

