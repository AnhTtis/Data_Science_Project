\section{Introduction}
% \maryam{again here, like in the abstract I suggest starting with the broad context in plain English about what's the big practical problem you are solving, almost from a search engine user's/science journalist's point of view. The details of the intro is pretty good}

\definecolor{Mycolor}{HTML}{a3a3a3}
\definecolor{Mycolor2}{HTML}{fc0adc}

\begin{figure}[]
    \centering
    \includegraphics[width=0.42\textwidth]{img/tsne_plot.pdf}
    \caption{TSNE reduction of \textcolor{Mycolor}{queries} and \textcolor{Mycolor2}{entities} when embedded with a Bi-Encoder trained with query logs and clicked entities (left), and synthetic queries from our proposed method \cqg{}  (right). The left model surfaces the same four entities for most queries while six entities are never retrieved as the most similar entity. The right model distributes the queries better, i.e. has less retrievability bias.}
    \label{fig:intro_image} 
    % \vspace{-1.5cm} 
\end{figure}
 
% Topic and Background: entity search, narrow vs broad, other goals such as discovery
% An important goal of online platforms for music, podcast and books is that users discover entities, i.e. find an entity that user was not familiar with. 

In online content platforms, users can search for catalog entities\footnote{Catalog entities are items from a platform that can be retrieved and/or recommended to users. For example, the book \textit{``The Fellowship of the Ring by J.R.R Tolkien''} is an entity from an online book platform. We refer to such items as \emph{entities} throughout the paper.} that they are already familiar with, for example, they issue queries with the title of a track to listen to next or of a book they would like to read. This type of search based on bibliographic data (e.g. title, artist, author, etc) to find entities~\cite{bainbridge2003people} has been referred to as \emph{narrow} intent queries~\cite{hosey2019just}. However, user information needs are diverse and can be more complex depending on their current mindset~\cite{li2019search}. 

When users have an exploratory mindset, they have a higher tolerance and are prone to explore different alternatives through \emph{broad} queries. Non-focused information needs are generally complex and require multiple interactions. Many users solve such information needs outside the search engine of the platforms, by asking broad queries to other users in forums such as subreddits\footnote{See for example \url{https://www.reddit.com/r/musicsuggestions/} or /r/booksuggestions/.} as existing search systems are ineffective for broader intents. 

% Problem: retrievability bias
Broad intents are an opportunity to surface under-served entities that would not be discovered otherwise without affecting user satisfaction~\cite{tomasi2020query}. While approaches to promote the discovery of entities have been studied from the perspective of recommender systems~\cite{aziz2021leveraging,mehrotra2021algorithmic}, they do not generalize to search engines where there is an input query. A prerequisite to improving the discoverability of entities through search is that the entity is retrievable. ~\citet{azzopardi2008retrievability} defined the \emph{retrievability of a document as how many queries lead to the entity being surfaced in the top-k results.} 

% \hugues{Let's be cautious about interchanging the words discoverability and retrievability. Either we stick to the latter or explain how the former is tied to the latter.}

For example, if we assume that the users will only interact with the top-1 ranked entity of the list, the dense retrieval model used to embed queries and entities in the left Voronoi plot of Figure~\ref{fig:intro_image} would make one of the five leftmost entities appear for every query (they are closest neighbors in the embedded space). These entities would have a high concentration of retrievability, i.e. retrievability bias, compared to the remaining entities which have no query close in the embedding space. Retrievability bias limits exploration, as it becomes harder to discover new entities through search when they have low retrievability scores.

In this paper, we study the effect of \emph{generating queries} on the retrievability of the system. Although the implications of query generation techniques for training document/passage dense models have been studied in detail~\cite{liang2020embedding,ma2020zero,wang2021gpl,dai2022promptagator}, little attention has been given to generating queries for \emph{entities} and their impact on the effectiveness and \emph{retrievability bias}. Dense models have shown promising results for different retrieval tasks~\cite{lin2021pretrained}, requiring a significant amount of in-domain supervision data for training~\cite{thakur2021beir}. Query generation approaches have shown to be effective in generating training data for domains with a scarcity of labeled data~\cite{liang2020embedding,ma2020zero,bonifacio2022inpars}.

% \maryam{<< If you can somehow relate your query generation approach to "a significant amount of in-domain supervision data for training" argument here, this paragraph will nicely lead to the next one. Otherwise, it needs some sort of transition.}

% \begin{table}[]
% \caption{Example of generated queries by \cqg{} for the song ``\textit{The Art Of A Mourning Kind}'', the podcast ``\textit{2019 Alumni Game Pre-show! presented by Da'Sean Butler}'' and the book ``\textit{The Brothers Karamazov by Fyodor Dostoyevsky}''.}
% \label{table:intro_table}
% \begin{tabular}{@{}lcc@{}} 
% \toprule 
% Dataset & Intent = \narrow{} & Intent = \broad{} \\ \midrule
% \tracks{}    & \textit{the art of a}
%  & \textit{melodyc death metal} \\
% \podcasts{}   & \textit{da'sean m} &  \textit{west virginia university athletics} \\ 
% \books{}   & \textit{fyodor dostoyevsky} & \textit{mystery crime historical fiction} \\ 
% \bottomrule
% \end{tabular}
% \end{table}

Unlike previous approaches to query generation which are agnostic to search intents, we propose \cqg{} which controls for the underlying intent. By generating both narrow and broad queries for an entity we are able to (I) train the dense retrieval model for both types of intents and (II) suggest broader and more exploratory queries to users. With the use of weak supervision through the proposed \emph{weak labeling functions}, \cqg{} does not strictly require any training data to generate synthetic queries for a given entity.
% Table~\ref{table:intro_table} shows examples of \cqg{} queries for a \narrow{} or a \broad{} intent given the same entity. 
 
%  RQ and hypothesis
With our empirical evaluation using three datasets in the domains of music, podcasts, and books we set out to answer the following research question: \emph{\textbf{To what extent can we reduce the retrievability bias of entity search with automatically generated queries without significant impact in the effectiveness?}} 

% \maryam{When I start reading the paragraph below, it feels like "H1" is only related to/attempts to resolve "I", and "H2" is only related to/attempts to resolve "II." But I don't think this is the case? I suggest the below changes or something similar}

Considering that the retrievability of an entity depends on (I) the retrieval model which decides which entities are surfaced for each query and (II) the set of queries used for the estimation, we generate two retrievability debiasing hypotheses that focus on modifications to the retrieval model and the set of queries respectively. Our first hypothesis, \textbf{H1}, is that training dense retrieval models with \cqg{} queries will lead to less retrievability bias compared to training with real queries and their respective clicked entities. The click data is prone to different biases, for example, many queries will be issued for the most popular entities, i.e. popularity bias, and after training the model on such data and this bias will be reinforced in later interactions with the system. Conversely, with \cqg{} we can obtain pairs of query-entity to train the model for any given entity, which can be randomly sampled from the collection. 

Our second hypothesis, \textbf{H2}, is that suggesting broad queries using \cqg{} will lead to less retrievability bias. Narrow queries have by definition less relevant entities than broad queries. By assisting users in formulating their queries with the suggestion of broad queries we can potentially influence users' query behaviors and then have an impact on the query type distribution.


Our main findings and contributions are:
\vspace{-0.1cm}
\begin{itemize}
    \item We introduce \cqg{}, a novel method to generate queries for a given entity conditioned on a desired underlying intent (\narrow{} or \broad{}). We demonstrate two ways of using the generated queries: as training data for dense retrieval models and as query suggestions.

    \item We find positive evidence for \textbf{H1}: dense models fine-tuned on synthetic queries have significantly less retrievability bias than models fine-tuned on click data. When using the queries from the proposed \cqg{} we reduce the retrievability bias by 10\% in terms of Gini scores on average when compared to a model that uses the click data and make 9\% of the \tracks{} collection of entities retrievable---go from zero to non-zero retrievability score. 
    % By controlling for the intent of the synthetic queries, we show that with \cqg{} we can train models that are more effective for broad queries than using click data.
    
    \item Regarding \textbf{H2}, we show that applying \cqg{} for generating query suggestions can reduce the retrievability bias of the system up to 9\% percent and increase the number of entities that have non-zero retrievability 11\% for the \tracks{} collection when using a Bi-Encoder model that was trained with an unbiased set of queries.
\end{itemize}

Next, we describe the related work, followed by the proposed method in Section 3. Section 4 describes the experimental setup used to answer the research question, followed by our experiments in Section 5. We conclude the paper in Section 6.

% \maryam{briefly describe the road-map for the rest of the paper here for orienting the reader. "Next we describe the related work, in section x we describe the methodology, in section y the model, etc" }