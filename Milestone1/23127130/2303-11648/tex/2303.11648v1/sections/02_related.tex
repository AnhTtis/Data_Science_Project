

\section{Related Work}
We first discuss here related work on search for the domains considered here. Then we look into retrievability followed by a discussion on query generation techniques and their applications.

\subsection{Entity Search}

A number of studies have explored user behavior when searching for specific entities such as music tracks, products, and books~\cite{hosey2019just,garcia2018understanding,bainbridge2003people,laplante2008everyday}. While focused searches have the goal of finding a specific entity, non-focused searches involve broader intents, where the user is in an exploratory mindset~\cite{li2019search,su2018user,sguerra2022navigational}. 

The Social Book Search Lab CLEF~\cite{koolen2016overview} that ran from 2011 to 2016\footnote{The book corpus of the SBS tasks is no longer available.} enabled a number of studies in complex search for the book domain ~\cite{ullah2020social,bogers2017supporting,bogers2018m,ullah2021improving,chaa2018combining}. A richer document representation for books which contains for example reviews, tags, and controlled vocabulary was shown to have better retrieval effectiveness.  It has also been shown in the music domain that multiple sources of data such as metadata, audio features, tags, and lyrics lead to better effectiveness for downstream tasks~\cite{kim2020one}. For podcast search, the TREC2020 podcasts track~\cite{jones2021trec} revealed that adding the additional information of transcripts also leads to higher effectiveness when compared to using only the episode title and description. 

Another external source of information for entities that was shown to be useful for downstream tasks~\cite{stamatelatos2021point} is the concept of lists, where users group together a number of entities that are similar in a way. In the music domain, this is often referred to as playlists. The creation of lists with curated entities is also common in the domains of books~\cite{liu2014recommending} and movies~\cite{greene2013discovering}.


% A number of studies have explored user behavior when searching for entities such as music, products and books~\cite{hosey2019just,garcia2018understanding,bainbridge2003people,laplante2008everyday}. While on one side of the intent spectrum you have narrow searches which are focused and have the goal of finding a target, on the other side you have searches which involve broad intents, where the user is in a exploratory mindset~\cite{li2019search,su2018user,sguerra2022navigational}. More complex information needs that go beyond finding an entity by its bibliographic information, such as the name, require semantic matching between the query and a rich representation of the entity.~\cite{tomasi2020query} explored the identification of such broad queries in order to surface under-served entities, providing exposure to suppliers that might otherwise not be found.

\begin{figure*}[ht!]
    \centering
    \includegraphics[width=1\textwidth]{img/model_diagram.pdf}
    \caption{\textcolor{Brown}{Left}: Components of the \cqg{} method. Each entity is serialized by concatenating the values of each metadata, e.g. \textit{title: The Fellowship of the Ring [SEP] author names: J.R.R. Tolkien [...]}. (2) Labeled data (\textit{entity ; query ; intent}) is not strictly required, due to the use of weak labeling functions which output a query and intent for a given entity, e.g. (\textit{The Fellowship of the Ring ; fantasy book ; \broad{}}). (3) Control over the underlying intent (\narrow{} or \broad{}) when generating the query via prompting, e.g. ``\textit{Generate a query with \narrow{}/\broad{} intent from: <serialized\_entity>}''. \textcolor{Brown}{Right}: Different ways of using the proposed \cqg{} to improve the retrievability of the search system: modifying the ranker by fine-tuning on synthetic queries and modifying the set of queries by suggesting broad queries for the narrow intent queries issued.}
    \label{fig:model_diagram} 
\end{figure*}

% \subsubsection*{\textbf{{Retrievability}}}
\subsection{Retrievability}
% Intuitively, a system should have a retrievability bias towards documents that are relevant to more queries than to documents that are not relevant to most queries.
% ~\citet{wilkie2017algorithmic} showed that good systems make relevant documents more retrievable. However, e
%  ~\citet{traub2018impact} showed that less errors in optical character recognition decreased the bias in retrievability
% The intuition is that the complexity of patents often lead to term mismatches, vague terms, and difficulty in selecting relevant terms, which leads to a set of queries with high retrievability bias. Experiments show that the proposed query expansion techniques decreased the overall bias of the retrieval system.
% Finding which queries lead to a document, and thus increases its retrievability, can be also used to improve search transparency~\cite{li2022exposing}.

% The notion of the accessibility of a document was first introduced in IR by~\cite{azzopardi2008accessibility}. Intuitively, a more accessible document has a higher likelihood of it being retrieved than a less accessible document. 

To estimate the retrievability of a document~\cite{azzopardi2008accessibility,azzopardi2008retrievability} proposed to sum the popularity of the queries that retrieve the given document above a position that the user would actually look at (e.g. in the top-5 documents). Retrievability scores can be used to determine if a retrieval system has a concentration of retrievability, for example, to verify if certain types of documents are being surfaced more than others. For example,~\cite{roy2022studying} showed that for a collection with datasets and articles, the retrievability bias was stronger for datasets when compared with articles. 

Even though a system with less retrievability bias does not necessarily mean that the system is more effective, studies have found a correlation between the two~\cite{bashir2017retrieval,wilkie2013relating,wilkie2014retrievability}, suggesting that a measure of retrievability bias can potentially be used to select better retrieval systems. In order to reduce the retrievability bias of a system ~\cite{bashir2010improving} proposed a query expansion technique with a novel document selection process for pseudo-relevance feedback in the domain of patent search.~\citet{chakraborty2020retrievability} proposed to use retrievability of a document over a set of query variations to decide which documents to use for relevance feedback. Finding which queries lead to a document can be also used to improve search transparency~\cite{li2022exposing}.

% \subsubsection*{\textbf{Query Generation}}
\subsection{Query Generation}
% Extending doc2query, ~\cite{nogueira2019doc2query} replaced the query generator by a transformer model (T5), resulting in a strong baseline when combined with a sparse retrieval model such as BM25
Query generation techniques can be broadly categorized based on their input: documents or queries. For generating known-item \textbf{queries for a given document}, i.e. queries where the task is to find a previously seen document, techniques have been proposed that select a number of document terms based on different sampling methods~\cite{kumar2011algorithmic,azzopardi2007building,azzopardi2006automatic}. ~\citet{liu2021strong} tackled a similar problem with the additional constraint that the generated queries are also informative. Generating queries for a given document using a seq2seq model was first proposed by~\cite{nogueira2019document}. Unlike sampling methods proposed for generating known-item queries, a seq2seq approach such as \textit{docT5query}~\cite{nogueira2019doc2query} is able to generate queries where its terms do not occur in the input document, being able to mitigate the vocabulary mismatch problem. Similarly, ~\cite{liang2020embedding,ma2020zero,wang2021gpl} generate queries based on documents using a transformer encoder-decoder model, but instead of using the queries for document augmentation, they employ the queries as additional training data for training bi-encoders, leading to significant gains in retrieval effectiveness---specially in cross-domain evaluation settings. It has also been proposed to replace the fine-tuned encoder-decoder model to generate queries with little supervision by doing in-context learning with models such as GPT-3~\cite{dai2022promptagator,bonifacio2022inpars}.~\citet{zhuang2022bridging} used generated queries with the goal of improving the effectiveness of the emerging differentiable search indexes. Another recent direction for query generation is to incorporate explicit knowledge when generating queries, e.g. with the use of knowledge graphs~\cite{shen2022diversified,cho2022query,han2019inferring}.

% The idea is to mitigate the mismatch problem where at retrieval time the transformer model has to go from a short text to document ids while at train time it goes from long texts (the documents) to their respective ids.

% , where common approaches are to mine from query logs and use interaction data
Generating \textbf{queries for a given query} has also been shown to be useful in IR. For example by generating query suggestions or reformulations that help users explore and express their information needs~\cite{cao2008context,mei2008query}. Another objective is to generate query variants that can be used to obtain more effective ranking models by combining such variants for the given query~\cite{belkin1995combining,benham2019boosting}, and also to better evaluate ranking models~\cite{penha2022evaluating,zuccon2016query,bailey2017retrieval}.
% with the premise that the same information need can be instantiated in a number of different ways, while current evaluation schemes only use one query to test for effectiveness.

The closest to our problem is the generation of queries for product search.~\citet{lien2022leveraging} used textual data from the reviews associated with the documents (products) to generate queries automatically for the following products: headphones, tents, and conditioners. In the domain of movies,~\citet{10.1145/3404835.3463071} generated queries automatically for a document (a movie) based on a number of predefined semantic components such as genre and year. We propose here a method to generate queries that can take advantage of manually created functions as a weak supervision signal, and also employ pre-trained language models. Unlike previous methods to generate queries, \cqg{} does intent-aware generation, where it is possible to control for the underlying intent of the output query.