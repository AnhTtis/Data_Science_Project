%%
%% This is file `sample-authordraft.tex',
%% generated with the docstrip utility.
%%
%% The original source files were:
%%
%% samples.dtx  (with options: `authordraft')
%% 
%% IMPORTANT NOTICE:
%% 
%% For the copyright see the source file.
%% 
%% Any modified versions of this file must be renamed
%% with new filenames distinct from sample-authordraft.tex.
%% 
%% For distribution of the original source see the terms
%% for copying and modification in the file samples.dtx.
%% 
%% This generated file may be distributed as long as the
%% original source files, as listed above, are part of the
%% same distribution. (The sources need not necessarily be
%% in the same archive or directory.)
%%
%% Commands for TeXCount
%TC:macro \cite [option:text,text]
%TC:macro \citep [option:text,text]
%TC:macro \citet [option:text,text]
%TC:envir table 0 1
%TC:envir table* 0 1
%TC:envir tabular [ignore] word
%TC:envir displaymath 0 word
%TC:envir math 0 word
%TC:envir comment 0 0
%%
%%
%% The first command in your LaTeX source must be the \documentclass command.
\PassOptionsToPackage{table,xcdraw,dvipsnames}{xcolor}
% \documentclass[sigconf,authordraft,anonymous]{acmart}
\documentclass[sigconf]{acmart}
\usepackage{booktabs}
\usepackage{caption}
\usepackage{subcaption}
\usepackage{booktabs}
\usepackage{arydshln}
\usepackage{multirow}
\usepackage{amsmath}
\usepackage{float}
\usepackage{makecell}
\usepackage{pythonhighlight}

\settopmatter{printacmref=false}

\newcommand{\set}[1]{\mathcal{#1}}

\newcommand{\tracks}{\texttt{Tracks}}
\newcommand{\podcasts}{\texttt{Podcasts}}
\newcommand{\books}{\texttt{Books}}

\newcommand{\reptitle}{\textit{title}}
\newcommand{\reptitleaug}{\textit{title+aug}}

\newcommand{\clicks}{\texttt{Click}}
\newcommand{\qgen}{\texttt{QGen}}
\newcommand{\inpars}{\texttt{InPars}}
\newcommand{\cqg}{\texttt{CtrlQGen}}

\newcommand{\broad}{\texttt{broad}}
\newcommand{\narrow}{\texttt{narrow}}

\newcommand{\podcastbroad}{\texttt{Podcasts$_{\broad{}}$}}
\newcommand{\tracksbroad}{\texttt{Tracks$_{\broad{}}$}}

\newcommand{\narrowcols}{\textit{narrow-fields}}
\newcommand{\broadcols}{\textit{broad-fields}}
\newcommand{\broadcolsft}{\textit{broad-fields-ft}}

\newcommand{\bm}[1]{\texttt{BM25}}
\newcommand{\biencoder}[1]{\texttt{Bi-Encoder}}

\newcommand{\weaklabelsun}[1]{\texttt{WeakLabeling-Un}}
\newcommand{\weaklabelsip}[1]{\texttt{WeakLabeling-IP}}

\newcommand{\maryam}[1]{\textcolor{red}{#1}}
\newcommand{\hugues}[1]{\textcolor{green}{#1}}

%% NOTE that a single column version may required for 
%% submission and peer review. This can be done by changing
%% the \doucmentclass[...]{acmart} in this template to 
%% \documentclass[manuscript,screen]{acmart}
%% 
%% To ensure 100% compatibility, please check the white list of
%% approved LaTeX packages to be used with the Master Article Template at
%% https://www.acm.org/publications/taps/whitelist-of-latex-packages 
%% before creating your document. The white list page provides 
%% information on how to submit additional LaTeX packages for 
%% review and adoption.
%% Fonts used in the template cannot be substituted; margin 
%% adjustments are not allowed.

%%
%% \BibTeX command to typeset BibTeX logo in the docs
\AtBeginDocument{%
  \providecommand\BibTeX{{%
    \normalfont B\kern-0.5em{\scshape i\kern-0.25em b}\kern-0.8em\TeX}}}

%% Rights management information.  This information is sent to you
%% when you complete the rights form.  These commands have SAMPLE
%% values in them; it is your responsibility as an author to replace
%% the commands and values with those provided to you when you
%% complete the rights form.
% % \setcopyright{none}

% \setcopyright{acmcopyright}
% \copyrightyear{2018}
% \acmYear{2018}
% \acmDOI{XXXXXXX.XXXXXXX}

\copyrightyear{2023}
\acmYear{2023}
\setcopyright{acmlicensed}\acmConference[WWW '23]{Proceedings of the ACM Web Conference 2023}{May 1--5, 2023}{Austin, TX, USA}
\acmBooktitle{Proceedings of the ACM Web Conference 2023 (WWW '23), May 1--5, 2023, Austin, TX, USA}
\acmPrice{15.00}
\acmDOI{10.1145/3543507.3583261}
\acmISBN{978-1-4503-9416-1/23/04}

% These commands are for a PROCEEDINGS abstract or paper.
% \acmConference[Conference acronym 'XX]{Make sure to enter the correct
%   conference title from your rights confirmation emai}{June 03--05,
%   2018}{Woodstock, NY}
%
%  Uncomment \acmBooktitle if th title of the proceedings is different
%  from ``Proceedings of ...''!
%
%\acmBooktitle{Woodstock '18: ACM Symposium on Neural Gaze Detection,
%  June 03--05, 2018, Woodstock, NY} 
\acmPrice{15.00}
\acmISBN{978-1-4503-XXXX-X/18/06}


%%
%% Submission ID.
%% Use this when submitting an article to a sponsored event. You'll
%% receive a unique submission ID from the organizers
%% of the event, and this ID should be used as the parameter to this command.
%%\acmSubmissionID{123-A56-BU3}

%%
%% For managing citations, it is recommended to use bibliography
%% files in BibTeX format.
%%
%% You can then either use BibTeX with the ACM-Reference-Format style,
%% or BibLaTeX with the acmnumeric or acmauthoryear sytles, that include
%% support for advanced citation of software artefact from the
%% biblatex-software package, also separately available on CTAN.
%%
%% Look at the sample-*-biblatex.tex files for templates showcasing
%% the biblatex styles.
%%

%%
%% For managing citations, it is recommended to use bibliography
%% files in BibTeX format.
%%
%% You can then either use BibTeX with the ACM-Reference-Format style,
%% or BibLaTeX with the acmnumeric or acmauthoryear sytles, that include
%% support for advanced citation of software artefact from the
%% biblatex-software package, also separately available on CTAN.
%%
%% Look at the sample-*-biblatex.tex files for templates showcasing
%% the biblatex styles.
%%

%%
%% The majority of ACM publications use numbered citations and
%% references.  The command \citestyle{authoryear} switches to the
%% "author year" style.
%%
%% If you are preparing content for an event
%% sponsored by ACM SIGGRAPH, you must use the "author year" style of
%% citations and references.
%% Uncommenting
%% the next command will enable that style.
%%\citestyle{acmauthoryear}

%%
%% end of the preamble, start of the body of the document source.
\begin{document}

%%
%% The "title" command has an optional parameter,
%% allowing the author to define a "short title" to be used in page headers.
\title{Improving Content Retrievability in Search with Controllable Query Generation}

%%
%% The "author" command and its associated commands are used to define
%% the authors and their affiliations.
%% Of note is the shared affiliation of the first two authors, and the
%% "authornote" and "authornotemark" commands
%% used to denote shared contribution to the research.
\author{Gustavo Penha$^1$, Enrico Palumbo$^2$, Maryam Aziz$^3$, Alice Wang$^3$, Hugues Bouchard$^4$}
\affiliation{
\institution{Spotify}
\country{$^1$Netherlands, $^2$Italy, $^3$USA, $^4$Spain}
}
\email{{gustavop,enricop,maryama,alicew,hb}@spotify.com}  % emails


% \author{Gustavo Penha}
% \affiliation{%
% \institution{Spotify}
% % \city{Delft}
% \country{Netherlands}}
% \email{gustavop@spotify.com}

% \author{Enrico Palumbo}
% \affiliation{%
% \institution{Spotify}
% % \city{}
% \country{Italy}
% }
% \email{enricop@spotify.com}

% \author{Maryam Aziz}
% \affiliation{%
% \institution{Spotify}
% % \city{}
% \country{USA}
% }
% % \email{@spotify.com}

% \author{Alice Wang}
% \affiliation{%
% \institution{Spotify}
% % \city{}
% \country{USA}
% }
% % \email{@spotify.com}

% \author{Hugues Bouchard}
% \affiliation{%
% \institution{Spotify}
% % \city{}
% \country{Spain}
% }
% % \email{@spotify.com}



%%
%% By default, the full list of authors will be used in the page
%% headers. Often, this list is too long, and will overlap
%% other information printed in the page headers. This command allows
%% the author to define a more concise list
%% of authors' names for this purpose.
\renewcommand{\shortauthors}{}

%%
%% The abstract is a short summary of the work to be presented in the
%% article.
\begin{abstract}
% \maryam{I think it might be. better to open the abstract by giving a broader but crisp context and introduce the problem and your solution right away, something line "Users' search intents are often varied from "narrow" to "broad" but online content platforms are mainly tailored for narrow intent queries overlooking much of users' complex information needs and the content matching those needs. In this work, ...''}

% \hugues{ I would suggest to open the abstract by stressing the importance of retrievability, and the challenges in online platforms dedicated to music, podcast and books. Then, we can make the distinction between narrow and broad intents. And finally, propose the solution CtrlQGen.}

An important goal of online platforms is to enable content discovery, i.e. allow users to find a catalog entity they were not familiar with. A pre-requisite to discover an entity, e.g. a book, with a search engine is that the entity is \emph{retrievable}, i.e. there are queries for which the system will surface such entity in the top results. However, machine-learned search engines have a high retrievability bias, where the majority of the queries return the same entities. This happens partly due to the predominance of narrow intent queries, where users create queries using the title of an already known entity, e.g. in book search ``\textit{harry potter}''. The amount of broad queries where users want to discover new entities, e.g. in music search ``\textit{chill lyrical electronica with an atmospheric feeling to it}'', and have a higher tolerance to what they might find, is small in comparison.  We focus here on two factors that have a negative impact on the retrievability of the entities (I) the training data used for dense retrieval models and (II) the distribution of narrow and broad intent queries issued in the system. We propose \cqg{}, a method that generates queries for a chosen underlying intent---narrow or broad. We can use \cqg{} to improve factor (I) by generating training data for dense retrieval models comprised of diverse synthetic queries. \cqg{} can also be used to deal with factor (II) by suggesting queries with broader intents to users. Our results on datasets from the domains of music, podcasts, and books reveal that we can significantly decrease the retrievability bias of a dense retrieval model when using \cqg{}. First, by using the generated queries as training data for dense models we make 9\% of the entities retrievable---go from zero to non-zero retrievability. Second, by suggesting broader queries to users, we can make 12\% of the entities retrievable in the best case.
\end{abstract}

% In addition, search engines of online content platforms are tailored for narrow intent queries and are often not able to handle complex information needs.

%%
%% The code below is generated by the tool at http://dl.acm.org/ccs.cfm.
%% Please copy and paste the code instead of the example below.
%%
\begin{CCSXML}
% <ccs2012>
%  <concept>
%   <concept_id>10010520.10010553.10010562</concept_id>
%   <concept_desc>Computer systems organization~Embedded systems</concept_desc>
%   <concept_significance>500</concept_significance>
%  </concept>
%  <concept>
%   <concept_id>10010520.10010575.10010755</concept_id>
%   <concept_desc>Computer systems organization~Redundancy</concept_desc>
%   <concept_significance>300</concept_significance>
%  </concept>
%  <concept>
%   <concept_id>10010520.10010553.10010554</concept_id>
%   <concept_desc>Computer systems organization~Robotics</concept_desc>
%   <concept_significance>100</concept_significance>
%  </concept>
%  <concept>
%   <concept_id>10003033.10003083.10003095</concept_id>
%   <concept_desc>Networks~Network reliability</concept_desc>
%   <concept_significance>100</concept_significance>
%  </concept>
% </ccs2012>
\end{CCSXML}

% \ccsdesc[500]{Computer systems organization~Embedded systems}
% \ccsdesc[300]{Computer systems organization~Redundancy}
% \ccsdesc{Computer systems organization~Robotics}
% \ccsdesc[100]{Networks~Network reliability}

%%
%% Keywords. The author(s) should pick words that accurately describe
%% the work being presented. Separate the keywords with commas.
% \keywords{retrievability, dense retrieval, query generation, controllable generation, query suggestion}


%%
%% This command processes the author and affiliation and title
%% information and builds the first part of the formatted document.
\maketitle

\section{Introduction}

\label{sec:intro}

% \textit{"Drawing and colour are not separate at all; in so far as you paint, you draw. The more the colour harmonizes, the more exact the drawing becomes."} - Paul Cezanne.

Art is a reflection of the figments of human imagination. 
While many are limited in their practical creative capabilities, by pushing the boundaries of digital media, new ways can be created for casual artists and experts alike to express their ideas. At the same time, current neural generative art takes away much of the control from humans. In this work, we attempt to take a step towards restoring some of that control, enabling neural networks to complement users and naturally extend their skills rather than taking hold over the generative process.

% \orr{TBD - make the opening colorful : 1. Add quote:  2. Elaborate: art is a rendering of figments of imaginations of humans. Most people are limited in their drawing capabilities, and by pushing the boundaries we allow new ways for casual artists and experts alike in expressing ideas. At the same time, neural generative art takes a lot of the control away. Here, we want to give back some of this control to humans, such that neural networks complement them and compensate their lack of skills, rather than replacing them.}

% The field of image synthesis has been significantly propelled by neural generative models, particularly by the latest text-to-image models that predominantly rely on large language-image models ~\cite{balaji2022eDiff-I, ramesh2022dalle, rombach2021highresolution, imagen2022saharia}. These models have revolutionized the field of computer vision as they can produce astonishing visual outcomes from text prompts only.

The field of image synthesis has been significantly propelled by neural generative models, particularly by the latest text-to-image models that predominantly rely on large language-image models ~\cite{balaji2022eDiff-I, ramesh2022dalle, rombach2021highresolution, imagen2022saharia}. These models have revolutionized the field of computer vision, as they can produce astonishing visual outcomes from text prompts alone.

The ability of text-to-image models has sparked a wave of editing methods that utilize these models. Many of these techniques rely on prompt editing ~\cite{ fu2022shapecrafter, hertz2022prompt, kawar2022imagic,lin2022magic3d,mokady2022null, poole2022dreamfusion}. Nevertheless, simplifying the interface to text alone means users lack the necessary level of granularity to produce their exact desired outcomes.
% which is} insufficient for effectively editing local content. 
% editing and manipulating visual content, as users lack the necessary level of control to achieve their desired outcomes
Sketch-guided editing, on the other hand, provides intuitive control that aligns with user's conventional drawing and painting skills. By incorporating user-guided sketches into text-to-image models, powerful editing systems can be created, offering a high degree of flexibility and fine-grained control for manipulating visual content~\cite{zhang2023controlnet, voynov2022sketch}.

Although sketch-guided and text-driven methods have proven successful in generating and manipulating 2D images \cite{meng2022sdedit, voynov2022sketch, cheng2023wacv}, it immediately raises the intriguing question of whether a similar approach could be developed to edit 3D shapes. 
Since direct text-to-3D models require an abundance of data to scale, state-of-the-art 3D generative models, such as DreamFusion~\cite{poole2022dreamfusion} and Magic3D~\cite{lin2022magic3d}, which build on the capabilities of text-to-image models, may be considered as an alternative.
% Due to the difficulty of scaling general direct text-to-3D models, incorporating conditions into a text-to-3D model is not straightforward. Thus, state of the art 3D generative models, such as DreamFusion~\cite{poole2022dreamfusion} \orrc{and Magic3D~\cite{lin2022magic3d}}, which build on the capabilities of text-to-image models, may be considered as an alternative.
However, maintaining control via conditioning with such models remains a challenging task, as these generative pipelines optimize a Neural Radiance Field (NeRF) \cite{mildenhall2020nerf} by amortizing gradients from a multitude of 2D views. In particular, providing consistent sketches across all possible views presents a hurdle for users. Instead, a plausible user interface should act with guidance from as few views as possible, e.g. up to two or three.


In this paper, we present \textbf{SKED}, a \textbf{SK}etch-guided 3D \textbf{ED}iting technique. Our method acts on reconstructed or generated NeRF models. We assume a text prompt and a minimum of two sketches as input and provide output edits over the neural field faithful to the input conditions.
Meeting all input requirements can be challenging as the text prompt may not match the sketch's semantics, and sketch views may lack coherence.
To undertake this complex task, we conceptually break it down into two subtasks that are easier to handle: one that depends on pure geometric reasoning and the other that exploits the rich semantic knowledge of the generative model. These two subtasks work together, with the former providing a coarse estimate of location and boundary, and the latter adding and refining geometric and texture details through fine-grained operations.


Our experiments highlight the effectiveness of our approach for editing various pretrained NeRF instances. We introduce assorted accessories, objects, and artifacts, which are generated and blended into the original neural field seamlessly.
Finally, we validate our method through quantitative evaluations and ablation studies to assert the contribution of individual components in our method. 
% By presenting examples in the paper, we illustrate that our method can generate realistic 3D artifacts with accurate texture and geometry using only a few basic sketches.



% Due to the absence of a direct text-to-3D model, incorporating conditions into a text-to-3D model is not straightforward. Thus, 3D generative models, such as DreamFusion~\cite{poole2022dreamfusion}, which build on the capabilities of text-to-image models, may be considered as an alternative.
% However, this is a challenging task since DreamFusion generates a NeRF by integrating many different 2D views. It is very hard to provide consistent sketches across all possible views. The challenge is to use sketches as a guide on only a few views (e.g., two or three) and generate 3D edit of the existing NeRF that is subject to being edited. 

% In this paper, we present \textbf{SKED}, a \textbf{SK}etch-guided 3D \textbf{ED}iting technique, that takes as input a text prompt and a few (two or more) sketches and edits a 3D given object represented as a NeRF in a geometrically plausible and controlled way. 
% We acknowledge the difficulty of this task, as there are no existing text-to-3D generative models available for manipulating the geometry of the existing object based on a text prompt. 
% To undertake this complex task, we conceptually break it down into two simpler subtasks that are easier to handle: one that depends on pure geometric reasoning and the other that exploits the rich semantic knowledge of generative model. These two subtasks work together, with the former providing a coarse estimate of location and the latter adding and refining geometric and texture details through fine-grained operations.

% Our experiments showcase the effectiveness of our approach in performing sketch-guided text-based edits on different base nerfs by introducing various accessories, objects, and artifacts. We also conduct ablation studies and experiments to evaluate the performance of individual components in our method. By presenting examples in the paper, we illustrate that our method can generate realistic 3D artifacts with accurate texture and geometry using only a few basic sketches.

%\dcc{Add here the traditional paragraph that tell about what we achieved and evaluated}
  While  submodular optimization problems are generally NP-hard, the celebrated greedy algorithm \cite{nemhauser1978analysis} attains a $(1-1/e)$ approximation ratio for  submodular maximization subject to uniform matroids and a $1/2$ approximation ratio for general matroid constraints. As discussed in the introduction, the  continuous greedy algorithm \cite{calinescu2011maximizing} restores the $(1-1/e)$ approximation ratio by lifting the discrete problem to the continuous domain via the multilinear relaxation. %It is worth to mention here that the multilinear relaxation is a DR-submodular function, a.k.a. a continuous function with the diminishing returns property.

Stochastic submodular maximization, in which the objective is expressed as an expectation, has gained a lot of interest in the recent years \cite{asadpour2008stochastic, zhang2022stochastic, chen2018online}. Karimi et al. \cite{karimi2017stochastic} use a concave relaxation method that achieves the $(1-1/e)$ approximation guarantee, but only  for the class of submodular coverage functions. Hassani et al.~\cite{hassani2017gradient} provide projected gradients methods for the general case of stochastic submodular problems that achieve $1/2$ approximation guarantee.  Mokhtari et al. \cite{mokhtari2020stochastic} propose stochastic  conditional gradient methods for solving both minimization and maximization  stochastic submodular optimization problems. Their method for maximization, Stochastic Continous Greedy (SCG) can be interpreted as a stochastic variant of the continuous greedy algorithm \cite{vondrak2008optimal, calinescu2011maximizing} and achieves a tight $(1-1/e)$ approximation guarantee for monotone and submodular functions. %However, all these methods suffer from two sources of randomness (one comes from sampling the objective function and the other comes from estimating the multilinear relaxation via sampling its inputs).

Our work builds upon and relies on the approach by  \"{O}zcan et al.~\cite{ozcan2021submodular}, who studied ways of accelerating the computation of gradients via a polynomial estimator. Extending on the work of Mahdian et al.~\cite{mahdian2020kelly},  \"{O}zcan et al. show that submodular functions that can be written as compositions of (a) an analytic function and (b) a multilinear function can be arbitrarily well approximated via Taylor polynomials; in turn, this gives rise to a method for approximating their multilinear relaxation in a closed form, without sampling. We leverage this method in the context of stochastic submodular optimization, showing that it can also be applied in combination with SCG of Mokhtari et al.~\cite{mokhtari2020stochastic}: this eliminates one of the two sources of randomness, thereby reducing variance at the expense of added bias. From a technical standpoint, this requires controlling the error introduced by the bias of the polynomial estimator, while simultaneously accounting for the variance inherent in SCG, due to sampling instances.   %: this eliminates the latter source of randomness by utilizing the properties of deep submodular models that result from composition over multiple layers. In order to do so, we combine the stochastic continuous greedy algorithm proposed by Mokthari et al. \cite{mokhtari2020stochastic} with the deterministic estimator proposed by
\begin{figure*}[t!]
\includegraphics[width=1.0\linewidth, trim={0 0.3cm 0 0.1cm}, clip]{figures/architecture/architecture.pdf}
\vspace{-15pt}
\caption{
\textbf{Point2Vec pre-training.}
Our model divides the input point cloud into %
point patches using farthest point sampling (FPS) and $k$-NN aggregation.
We obtain patch embeddings by applying a mini-PointNet\,\colorsquare{m_pointnet} to each point patch (\emph{right}).
The teacher Transformer encoder\,\colorsquare{m_green} infers a contextualized %
representation for all patch embeddings which, after normalization and averaging over the last $K$ Transformer layers, serve as training targets.
The student's input is a masked view on the input data, \ie we randomly mask out a ratio of patch embeddings and only pass the remaining embeddings into the student Transformer encoder\,\colorsquare{m_blue}.
After applying a shallow decoder\,\colorsquare{m_red} on the outputs of the student, padded with learned mask embeddings\,\protect\maskembedding{}, we train the student and decoder to predict the latent teacher representation of the patch embeddings.
\vspace{-10pt}
}
\label{fig:model}
\end{figure*}
\section{Method}

The aim of this work is to unlock the full potential of data2vec-like\,\cite{baevski2022data2vec} pre-training on point clouds by addressing point cloud specific challenges.
To achieve this, we first summarize the technical concepts of data2vec (\refsec{method_d2v}) and show how to learn rich representations on point clouds using data2vec pre-training (\refsec{method_d2v_pcl}).
Finally, we propose \name{}, which accounts for the point cloud specific limitations of data2vec (\refsec{method_p2v}).

\subsection{Data2vec}\label{sec:method_d2v}
Data2vec\,\cite{baevski2022data2vec} is designed to pre-train Transformer-based models, which involve a feature encoder that maps the input data to a sequence of embeddings.
These embeddings are subsequently passed to a standard Transformer encoder to generate the final latent representations.
During pre-training, two versions of the Transformer encoder are kept: a \emph{student} and a \emph{teacher}.
The teacher is a momentum encoder, \ie its parameters $\Delta$ track the student's parameters $\theta$ by being updated after each training step according to an exponential moving average (EMA) rule\,\cite{caron2021dino, baevski2022data2vec, grill2020BYOL, he2020moco}: $\Delta \leftarrow \tau \Delta + (1-\tau)\theta$,
where $\tau \in [0,1]$ is the EMA decay rate.
The teacher provides the training targets, which the student predicts given a corrupted version of the same input.

In a first step, the teacher encodes the uncorrupted input sequence.
The training targets are then constructed by averaging the outputs of the last $K$ blocks of the teacher, which are normalized beforehand to prevent a single block from dominating the sum.
Due to the self-attention layers, these targets are \emph{contextualized}, \ie they incorporate global information from the whole input sequence.
This is an important difference to other masked-prediction methods such as BERT\,\cite{devlin2018bert} and MAE\,\cite{he2022mae}, where the targets only comprise local information, \eg a word or an image patch. %

The student is given a masked version of the same input, where some of the embeddings in the input sequence are substituted by a special learned \emph{mask embedding}. %
The student's task is to predict the targets corresponding to the masked parts of the input.
The model is trained by optimizing a Smooth L1 loss on the regressed targets. %







\subsection{Data2vec for Point Clouds}\label{sec:method_d2v_pcl}

To apply data2vec to point clouds, we utilize the same underlying model as Point-BERT\,\cite{yu2021pointbert} and Point-MAE\,\cite{pang2022pointmae}.
This model is well suited for data2vec pre-training: it extracts a sequence of patch embeddings from the input point cloud and feeds it to a standard Transformer encoder.
For downstream tasks, we append a task-specific head to the Transformer encoder (\refsec{experiments}).
Next, we describe the point cloud embedding and the Transformer in detail and conclude with a summary of data2vec for point clouds.


\parag{Point Cloud Embedding.}
First, we sample $n$ center points from the input point cloud using farthest point sampling (FPS)\,\cite{qi2017pointnetplusplus}.
Grouping the center points' $k$-nearest neighbors ($k$-NN) in the point cloud yields $n$ contiguous \emph{point patches}, \ie sub-clouds of $k$ elements.
Next, we normalize the point patches by subtracting the corresponding center point from the patch's points.
This untangles the positional and the structural information.
To account for the permutation-invariant property of point clouds, we employ a mini-PointNet\,\cite{qi2016pointnet} (\reffig{model}, \emph{right}) that maps each normalized point patch to a \emph{patch embedding}.

The mini-PointNet involves the following steps:
First, we map each point of a patch to a feature vector using a shared MLP.
Then, we concatenate max-pooled features to each feature vector.
The resulting feature vectors are then passed through a second shared MLP and a final max-pooling layer to obtain the patch embedding.

\paragraph{Transformer Encoder.}
The central component of the model is a standard Transformer encoder.
The patch embeddings form the input sequence to the Transformer encoder.
Since the point patches are normalized, the patch embeddings carry no positional information;
therefore, a two-layer MLP maps each center point to a position embedding, which is then added to the corresponding patch embedding.
Due to the special importance of positional information in point clouds, the position embeddings are added again before each subsequent Transformer block to ensure that the positional information is incorporated at every step of the encoding process.

\paragraph{\emakefirstuc{\datavec{}}.}

To establish a baseline, we apply the unmodified data2vec approach to the previously described underlying model of Point-BERT and Point-MAE.
Going forward, we will refer to this approach as \datavec{}.


\subsection{\emakefirstuc{\name{}}}\label{sec:method_p2v}
In \reffig{model}, we present the complete pipeline of our \name{} model.
Directly applying data2vec to point cloud data without modifications is not optimal, as the position embeddings are also added to the mask embeddings, revealing the overall shape of the point cloud to the student.
As positions are the only features for point clouds, this makes the masking far less effective, as noted by Pang \etal \cite{pang2022pointmae} in the context of masked autoencoders.

To solve this issue, we adopt an approach inspired by MAE\,\cite{he2022mae}, where we only feed the non-masked embeddings to the student\,\colorsquare{m_blue}.
A separate decoder\,\colorsquare{m_red}, implemented as a shallow Transformer encoder, takes the output of the student and the previously held-back masked embeddings\,\maskembedding{} as input and predicts the training targets.
In contrast to \datavec{}, this approach does not suffer from leaking positional information from the masked-out point patches to the student.
Moreover, utilizing an MAE-inspired setup provides additional benefits:
First, the student is more computationally efficient, as it only needs to process the non-masked embeddings.
Second, the model's inputs during fine-tuning are more similar to those during pre-training because the inputs during pre-training are no longer dominated by masked embeddings which are absent during fine-tuning.
This likely makes the learned representations more transferable to downstream tasks.


\begin{table}[ht!]
\footnotesize
\caption{Datasets metadata and statistics. Metadata columns 1--3 are considered to be \narrowcols{}, whereas 4--9 are \broadcols{}. In the experiments the broad columns which are free-text (\broadcolsft{}) are: Episode \& show description and Transcript for \podcasts{}, and User reviews and Description for \books{}.}
\label{table:dataset_statistics}
\begin{tabular}{@{}lllll@{}}
\toprule
 &  & \tracks & \podcasts & \books \\ \midrule
Metadata & \begin{tabular}[c]{@{}l@{}}(1)\\ (2)\\ (3)\\ (4)\\ (5)\\ (6)\\ (7)\\ (8)\\ (9)\end{tabular} & \begin{tabular}[c]{@{}l@{}}Title\\ Album name\\ Artist names\\ Release year\\ Language\\ Genres\\ Descriptors\\ Lyric\\ User Playlists\end{tabular} & \begin{tabular}[c]{@{}l@{}}Title\\ Show name\\ Host names\\ Ingested date\\ Language\\ Categories\\ Episode \& show description\\ Transcript\\  Topics\end{tabular} & \begin{tabular}[c]{@{}l@{}}Title\\ Series name\\ Author names\\ Publication year\\ Language\\ Genres\\ Description\\ User reviews\\ User lists\end{tabular} \\ \midrule
% Example & \begin{tabular}[c]{@{}l@{}}(1)\\ (2)\\ (3)\\ (4)\\ (5)\\ (6)\\ (7)\\ (8)\\ (9)\end{tabular} & \begin{tabular}[c]{@{}l@{}}Shake for Me - Live at {[}...{]}\\ The Essential Stevie Ray {[}...{]}\\ Stevie Ray Vaughan\\ 2002\\ EN\\ rock, instrumental rock {[}...{]}\\ modern, blues, barbecu {[}...{]}\\ You better wait, baby yo {[}...{]}\\ the essential stevie, dou {[}...{]}\end{tabular} & \begin{tabular}[c]{@{}l@{}}You asked for real raises, {[}...{]}\\ Planet Money\\ Robert Smith, Stacey Va {[}...{]}\\ 2021-11-25\\ EN\\ Business \& Technology, {[}...{]}\\ It's listener question time {[}...{]}\\ The economy explained. {[}...{]}\\ Taylor Swift\end{tabular} & \begin{tabular}[c]{@{}l@{}}Red Sapphire\\ The Sita Chronicles\\ Ashley Mayers\\ 2015\\ EN\\ fantasy, paranormal, you {[}...{]}\\ The young, timid dreame {[}...{]}\\ A captivating fantasy sag {[}...{]}\\ science fiction, teens an {[}...{]}\end{tabular} \\ \midrule
\multicolumn{2}{l}{\# docs} & 682k & 600k & 617k \\ \midrule
\multicolumn{2}{l}{\# queries} & 100k & 100k & 100k \\ \midrule
\multicolumn{2}{l}{\begin{tabular}[c]{@{}l@{}}\clicks{} \# qrels  \\ train/val/test\end{tabular}} & 75.9k/9.5k/9.5k & 14.4k/1.8k/1.8k & 117.5k/14.7k/14.7k \\ \midrule
% \multicolumn{2}{l}{\begin{tabular}[c]{@{}l@{}}Avg doc len\\ \reptitle{} (1)\end{tabular}} & 3.80 & 6.96 & 5.54 \\ \midrule
\multicolumn{2}{l}{Avg doc len} & 55.87 & 80.76 & 161.58 \\ \midrule
\multicolumn{2}{l}{Avg query len} & 1.96 & 3.06 & 4.47 \\ \bottomrule
\end{tabular}
\end{table}
\section{Experimental Setup}

In this section, we first describe the data used to test our hypotheses, followed by the implementation details of the methods and baselines as well as how we evaluate different approaches.

\subsection{Datasets}
In order to test our hypothesis and compare different methods to generate queries we rely on three datasets: \tracks{}, \podcasts{}, and \books{}. For each dataset, we have a set of entities (>600k entities), a set of 100k queries, and a set of relevance judgements. Table~\ref{table:dataset_statistics} describes the statistics of the datasets and examples of entities.

While the queries and entities from \textbf{\tracks{}} and \textbf{\podcasts{}} were extracted from a large-scale online platform the \textbf{\books{}} dataset is a subset of the Goodreads public dataset from~\cite{DBLP:conf/recsys/WanM18}~\footnote{\href{https://github.com/MengtingWan/goodreads}{https://github.com/MengtingWan/goodreads}}. The query sets from \tracks{} and \podcasts{} are a unique subset of randomly sampled entities and queries from the logs of a large scale online audio platform, where clicks for a given entity after issuing the query are considered to be the relevance signal in our experiments. 

We also use the number of distinct users for which the query was issued by, and use them as $o_{q}$ to calculate retrievability scores (see Section 4.3). Regarding the columns, as seen in Table~\ref{table:dataset_statistics}, the ones with numbers 1--3 are considered to be \narrowcols{}, whereas 4--9 are \broadcols{}. The broad columns which are free-text (\broadcolsft{}) are: \textit{Episode \& show description} and \textit{Transcript} for \podcasts{}, and \textit{User reviews} and \textit{Description} for \books{}.

Since the Goodreads dataset does not have any set of queries available, we generate a set of queries automatically: 75\% of the queries are narrow, generated by sampling words from the \narrowcols{}, and 25\% of them are from \broadcols{} and consider that as the relevance labels. This specific split of narrow and broad queries was chosen to simulate actual user behavior observed in the other two datasets (\tracks{} and \podcasts{}) where narrow queries are the majority but in a less extreme fashion. We use the number of ratings the entities from \books{} have as a proxy for the number of users that would issue such queries ($o_{q}$). 
% In the next subsection we explore ways of getting a dataset of broad queries for the music and podcast domains.

\subsubsection{Broad queries datasets}

Since the majority of the queries from \tracks{}, \podcasts{} and \books{} are \narrow{}, we also employ two smaller additional sets of queries and relevance labels that have an underlying \broad{} intent. They are $\tracks{}_{\broad{}}$ and $\podcasts{}_{\broad{}}$, containing a total of 1309 and 500 queries. The $\tracks{}_{\broad{}}$ is a sample of queries from the logs that have a high predicted probability of being broad based on the interaction signals the user had after issuing the query. If the user interacts with entities such as playlists and hubs more than tracks and albums they are more likely to be issuing a broad query. Based on this set, we get the clicked entities where the query does not match the title, artist, or album of the entity, avoiding cases where the query seems broad but is in fact a narrow interaction, e.g. query ``\textit{pop}'' and clicking a track with the title ``\textit{POP!}''. For $\podcasts{}_{\broad{}}$, there is no parallel for the \tracks{} playlists so we employ a set of manually curated pairs of broad queries and entities. Annotators were instructed to write a query relevant to the podcast episode while avoiding exact matches and matching diverse metadata fields.

\subsection{Implementation Details}

\subsubsection{Query generation models}
As baselines for generating synthetic queries, we first use \textbf{\qgen{}}, a common approach to generate queries from documents used in this manner in different previous work~\cite{nogueira2019doc2query,liang2020embedding,ma2020zero,wang2021gpl}. We rely on fine-tuning T5~\cite{raffel2019exploring} (\textit{t5-base}) on a subset of \clicks{} train set with 10k pairs query-entities. The second baseline for generating queries requires very little supervision signal: \textbf{\inpars{}}~\cite{bonifacio2022inpars}. The model uses in-context learning, i.e. few examples in the prompt of the document and expected query, and large language models. For a fair comparison, we randomly sample examples to use in the prompt every time we are generating the output queries, this way \inpars{} has access to the same amount of training pairs of query and entities as \qgen{}\footnote{This was shown to be effective for the validation sets of \podcasts{} and \books{}. For \tracks{} we did not observe the same, so we used a fixed prompt with the same two examples randomly selected from the dataset.}. We rely on the open \textit{bigscience/bloom-760m}\footnote{\href{https://bigscience.huggingface.co/blog/bloom}{https://bigscience.huggingface.co/blog/bloom}} release to do so\footnote{We explore larger GPT-3 models on the appendix and see that the larger 175B parameter one does not significantly improve over smaller models.}. For the \textbf{\cqg{}} implementation we also rely on the T5 (\textit{t5-base}) model. When generating the queries with T5, for both \qgen{} and \cqg{} we employ $do\_sample$=True and $top\_k$=10.

\subsubsection{Retrieval models}
For \textbf{\bm{}}~\cite{robertson1994some} we resort to the default hyperparameters and implementation provided by the PyTerrier toolkit~\cite{pyterrier2020ictir}. For the zero-shot \textbf{\biencoder{}} models, we rely on the SentenceTransformers~\cite{reimers-2019-sentence-bert} model releases\footnote{\url{https://www.sbert.net/docs/pretrained_models.html}}. The library uses Hugginface transformers for the pre-trained models such as BERT~\cite{devlin2018bert} and MPNet~\cite{song2020mpnet}. Specifically, we employ the pre-trained model \textit{all-mpnet-base-v2}. When fine-tuning the \biencoder{} models on the \clicks{} or synthetic datasets, we rely on the \textit{MultipleNegativesRankingLoss}, which uses in-batch random negatives to train the model. We fine-tune the dense models for a total of 10k steps. \textbf{Thus, all dense models were trained on the same amount of (synthetic or not) queries}. We use a batch size of 8, with 10\% of the training steps as warmup steps. The learning rate is 2e-5 and the weight decay is 0.01. We refer to the \biencoder{} model trained on \clicks{} data as \textbf{$\biencoder{}_{\clicks{}}$} and a \biencoder{} model trained on the queries from \cqg{} as \textbf{$\biencoder{}_{\cqg{}}$}.

\subsection{Evaluation Procedure}

To evaluate the effectiveness of the retrieval systems we use the recall at 100, $R@100$. The choice for R@100 is due to the objective of increasing the retrievability of items considering the first 100 options\footnote{A second stage re-ranker in this pipeline could be precision-focused if the retriever is able to find enough relevant and diverse options.}. We perform Students t-tests at the confidence level of 0.95 with Bonferroni correction to compare the difference between models with statistical significance. 

To evaluate how biased the retrieval system is in terms of retrievability, we first estimate the retrievability of an entity $e$ as defined by~\cite{azzopardi2008retrievability}: $r(\mathbf{e})=\sum_{\mathbf{q} \in \mathbf{Q}} o_{q} \cdot f\left(k_{e q}, c\right)$, where $\mathbf{Q}$ is the set of queries\footnote{The size of $\mathbf{Q}$ is 100k for all computations.}, $o_{q}$ is the weight of each query---here we use the number of users that issued the query---and $f\left(k_{e q}, c\right)$ is 1 if the entity $e$ is ranked above $c$ by the search system (in our experiments we set c=100) and 0 otherwise. In order to get a number that summarizes how concentrated or biased the retrievability scores are we calculate the Gini score~\cite{gastwirth1972estimation}: $G=\frac{\sum_{i=1}^{N}(2 * i-N-1) * r\left(\mathbf{e}_{\mathbf{i}}\right)}{N \sum_{j=1}^{N} r\left(\mathbf{e}_{\mathbf{j}}\right)}$, where G=1 means only one entity concentrates all the retrievability, and G=0 means every entity in the collection has the same retrievability score. In order to perform statistical testing for the Gini scores we follow~\cite{gamboa2010statistical}.
We begin by briefly comparing the performance of the three predictors for $\hat{\bm{x}}_i$ (FE, RK4, NN) before testing \cref{algo:simulator}.

\subsection{Predictor comparison}\label{subsec:results_predictors}
The approximator $\hat{\bm{x}}_i$ should have two properties: being fast and being accurate for large time-steps $\Delta t$. \Cref{fig:predictor_characteristics} shows the two properties. The left panel displays the error in the differential variable $\Delta \omega$ across 200 points of random initial conditions $\bm{x}_{0}$ and voltage parameterizations $\yparams{}_i$. In terms of accuracy, the \gls{NN} performs well and only for smaller time-steps ($\Delta t < 0.05 \si{\second}$), the RK4 approximation becomes more accurate. The RK-schemes exhibit the expected dependency of the time-step - the local truncation error should follow $\mathcal{O}(\Delta t)$ and $\mathcal{O}(\Delta t^4)$ for FE and RK4. The error of the \gls{NN} in contrast is near independent of $\Delta t$. At the same time, the \gls{NN} is the fastest approximator. While the FE approximator is similarly fast, its poor accuracy makes it undesirable and while more accurate, the RK4 scheme has the drawback of comparably long run-time.     

\begin{figure}[!th]
    \centering
    \includegraphics[width=0.95\linewidth]{figures_pdf/predictor_error.pdf}
    \caption{Predictor characteristics: (left) prediction error of $\Delta \omega$ for 200 predictions with random $\bm{x}_0$ and \yparams{}, (right) run-time per point. These results show that \gls{NN} constitute an accurate and fast predictor.}
    \label{fig:predictor_characteristics}
\end{figure}

\subsection{Simulator results}\label{subsec:results_simulators}

We now focus on the performance of the simulator in \cref{algo:simulator}. \Cref{fig:ieee9_prediction} displays the prediction for $\Delta t = \SI{0.05}{\second}$ using \gls{NN}-based approximators $\hat{\bm{x}}_i^{NN}$. 
\begin{figure}[!ht]
    \centering
    \includegraphics[width=0.95\linewidth]{figures_pdf/simulation_results.pdf}
    \caption{Prediction of a state trajectory ($\Delta \omega_i$) and an algebraic variable $V_i$ for time-step size $\Delta t = \SI{0.05}{\second}$ with a \gls{NN}-based simulator. The predictions (dashed lines) coincide with the ground truth (gray lines).}
    \label{fig:ieee9_prediction}
\end{figure}
The results correspond to the first row in \cref{tbl:simulator_results} where we report the maximum absolute errors of $V$ and $\Delta \omega$ along the trajectory and the run-time. \Cref{tbl:simulator_results} shows further results for the \gls{RK4}-based simulator, for time-steps of $\Delta t = \SI{0.1}{\second}$ and $\Delta t = \SI{0.15}{\second}$ and for collocation points at $\bm{T}=[0.3, 0.7]\Delta t$ and $\bm{T}=[0.1, 0.3, 0.5, 0.7, 0.9]\Delta t$, i.e., $s=2$ and $s=5$. The \gls{NN}-based simulator is consistently faster and more accurate than the \gls{RK}-based simulator, except for the case of $\Delta t = \SI{0.05}{\second}$. These results confirm the predictor characteristics observed in \cref{subsec:results_predictors}. The simulation run-time scales approximately inversely proportional with $\Delta t$. Deviations can arise due to varying numbers of iteration in \cref{algo:parameter_update}, however, in the reported cases, we observe usually 5-7 iterations. Increasing the number of collocation points $s$ results in a small increase in run-time in all cases. In terms of accuracy, we observe that more collocation points can lead to better accuracy, when the overall solution quality is good. However, for too large time-steps, here $\Delta t = \SI{0.15}{\second}$, the effect might reverse. The choice of the location of the collocation points, i.e., $\bm{T}$, also matters.%\bz{Can we bold the best numbers in a box? This would help to highlight what people should look at. There are a lot of numbers.}
\begin{table}[!ht]
\renewcommand{\arraystretch}{1.2}
\caption{Comparison of simulators with different predictor schemes}
\label{tbl:simulator_results}
\centering
\begin{tabular}{ccc|ccc}
\toprule
$\Delta t$ & Predictor & s & run-time & $\max |V_i - \hat{V}_i|$ & $\max | \omega_i - \hat{\omega}_i|$ \\
$[\si{\second}]$ & & & [\si{\second}] & $\times 10^{-2} [\si{\pu}]$ & $\times 10^{-3} [\si{\pu}]$ \\ \midrule
\multirow{2}{*}{$0.05$} & NN & $5$ & $\bm{1.85}$ & $0.82$ & $0.35$\\
 & RK4 & $5$ & $3.88$ & $\bm{0.31}$ & $\bm{0.29}$\\ \midrule
\multirow{4}{*}{$0.10$} & \multirow{2}{*}{NN} & $2$ & $\bm{0.98}$ & $2.71$ & $1.29$ \\
& & $5$ & $1.05$ & $\bm{1.32}$ & $\bm{0.62}$ \\
 & \multirow{2}{*}{RK4} & $2$ & $2.21$ & $5.07$ & $2.40$ \\
& & $5$ & $2.27$ & $3.88$ & $2.01$ \\ \midrule
\multirow{4}{*}{$0.15$} & \multirow{2}{*}{NN} & $2$ & $\bm{0.60}$ & $\bm{6.28}$ & $2.93$ \\
& & $5$ & $0.77$ & $6.36$ & $\bm{2.90}$ \\
 & \multirow{2}{*}{RK4} & $2$ & $1.19$ & $11.8$ & $5.06$ \\
& & $5$ & $1.46$ & $19.0$ & $7.75$ \\
% $\approx$ 0.020 & BDF & - & 0.83 & 0.07 & 0.26\\
% $\approx$ 0.045 & BDF & - & 0.67 & 7.73 & 3.36\\
\bottomrule
\end{tabular}%
\end{table}



%The last two rows stem from the \gls{DAE}-solver in Assimulo based on a variable order backward differentiation formula (BDF) with different tolerance levels. We observe that the required time-step size is significantly smaller than for the proposed simulator. The resulting run-times are comparable, however, neither method was optimized for run-tme in this study. \bz{I would delete the the last two rows and skip this discussion. Not sure how this helps the paper.}
This paper presented a comprehensive analysis of the use of \acrfull{PINN} for power system dynamic simulations. We show that \glspl{PINN} (i) are 10 to 1'000 times faster than conventional solvers, (ii) do not face issues of numerical instability unlike conventional solvers, and, (iii) achieve a decoupling between the power system size and the required solution time. However, \glspl{PINN} are less flexible (i.e. they do not easily handle parameter changes), and require an up-front training cost. Overall, this makes \gls{PINN}-based solutions well-suited for repetitive tasks as well as task where run-time speed is crucial, such as for screening.

Besides the comparison between conventional and \gls{NN}-based methods, this paper conducts a deeper analysis on the parameters that affect the performance of the \gls{NN} solutions. In that respect, we introduce a new \gls{NN} regularisation, called dtNN, as a intermediate step between \glspl{NN} and \glspl{PINN}. We show that \glspl{PINN} achieve overall higher levels of accuracy, and more balanced error distributions thanks to the evaluation of the collocation points.


% \section{Acknowledgments}
% \begin{acks}
% To Robert, for the bagels and explaining CMYK and color spaces.
% \end{acks}

%%
%% The next two lines define the bibliography style to be used, and
%% the bibliography file.
\bibliographystyle{ACM-Reference-Format}
\bibliography{sample-base}

\appendix
\section{Unsupervised Weak Labeling Functions}
In this appendix, we define the functions used in \weaklabelsun{}.
\subsection{Random Terms Selection}
Samples words from the entity $e$, given possible $P$ length percentages for the query with probability $Pr$.
\begin{python}
def sample_words(e, P, Pr):
    words = tokenize(e)
    p_words_to_sample = np.random.choice(P, 1, p=Pr)
    n = int(len(words) * p_words_to_sample)
    words = np.random.choice(words, n, replace=False)
    return words
\end{python}

\subsection{Query Variation Ordering}
Generates a query variation by shuffling two random words from the query.
\begin{python}
def qv_ordering(q):
    words = tokenize(q)
    idxs = [i for i in range(0, len(words))]
    p1, p2 = np.random.choice(idxs, 2, replace=False)
    words[p1], words[p2] = words[p2], words[p1]
    return " ".join(words)
\end{python}

\subsection{Query Variation Misspelling}
Generates a query variation by adding a misspelling error with $P$ probabilities of removing and addition.
\begin{python}
def qv_misspelling(q, P):
    t = np.random.choice(["rem", "mdf"], 1, p=P)
    idxs=[i for i in range(len(query))]
    l=string.ascii_letters
    if t == "rem":
        idx_rem = np.random.choice(idxs, 1)[0]
        qv = q[0:idx_rem] + q[idx_rem+1:]
    elif t == "mdf":
        idx_mdf = np.random.choice(idxs, 1)[0]
        char_add = np.random.choice(len(l), 1)[0]
        qv = q[0:idx_mdf] + l[char_add] + q[idx_mdf+1:]
    return qv
\end{python}

\subsection{Query Variation Prefix}
Generates a query variation by removing $P$ percentages of the suffix of the query with probabilities $Pr$.
\begin{python}
def qv_prefix_query(q, P, Pr):
    rem = np.random.choice(P, 1, p=Pr)    
    return q[:int((1-rem)*len(q))]
\end{python}

\subsection{Query from Free-Text Column by Summarization}
Generates a query by summarizing the value of a free-text column (\broadcolsft{}). For our experiments we rely on the pre-trianed summarizer model \textit{snrspeaks/t5-one-line-summary}\footnote{\url{https://huggingface.co/snrspeaks/t5-one-line-summary}}.
\begin{python}
from transformers import pipeline
def q_summarizer(ft, m):
    pipe = pipeline("text2text-generation", 
                    model = m)
    return pipe("summarize: {}".format(ft))[0]
\end{python}

% \subsection{Query from Free-Text Column by IDF selection}
% Generates a query by selecting terms using sampling based on top $T$ \% of terms sorted by IDF.

% \begin{python}
% def q_idf(ft, idf_dict, T):
%     idfs = {w: idf[w] for w in tokenize(ft)}
%     s = sorted(idfs.entities(), 
%                 key=operator.entitygetter(1))
%     w = [v[0] for v in s]
%     return " ".join(w[int(len(w)*T):])
% \end{python}

\section{Bias mitigation for the \clicks{} dataset}
In this appendix, we investigate if it is possible to mitigate the biases of the \clicks{} data with a simpler approach.

When fine-tuning the Bi-Encoder with \clicks{} data in our experiments we do not employ the same combination of queries and entities twice, even if that pair is highly popular in the logs. This is already a way of reducing the bias in the \clicks{} dataset. However, there are still many query variations that lead to the same entities, i.e. queries with different forms but with the same underlying information need, which are not removed when we get distinct queries for training. In order to mitigate this bias from the \clicks{} data by removing multiple queries that lead to the same entity, we randomly select only one of the queries for each entity to train the \biencoder{} model on. 

The result of this experiment is that such a bias mitigation strategy indeed improves the retrievability of the system: the Gini scores go from 0.856 to 0.803 for \tracks{} and from 0.763 to 0.713 for \podcasts{}. However the mitigated \clicks{} data approach still leads to 30\% and 5\% more retrievability bias than \cqg{}, for \tracks{} and \podcasts{} respectively.

\section{Scaling \inpars{} with GPT-3}
In this appendix, we test if increasing the model size of the InPars model using GPT-3 as the language model has a significant effect on the Bi-Encoder trained with such synthetic queries. 

We see from \ref{fig:gpt3} that this is not the case for both datasets, and the highest R@100 is reached when using the 1.2B GPT-3 model (\textit{babbage-001}). Similar results were found in the InPars paper~\cite{bonifacio2022inpars}.
\begin{figure}[H]
    \centering
    \includegraphics[width=.45\textwidth]{img/gpt3.png}
    \caption{Scaling \inpars{} baseline using GPT-3.}
    \label{fig:gpt3}
\end{figure}

\section{Overlap of generated queries}
In this appendix, we check if the queries generated by \cqg{} have a significant overlap with either the log queries (is \cqg{} just copying existing queries?) or with the input entity (is \cqg{} just copying words from the input entity?).

\subsection{With log queries from \clicks{}}
Out of the 10k narrow queries generated by 
\cqg{} to test the first hypothesis (results from Tables~\ref{table:results_narrow}), there are only 6\% and 12\% exact matches with set of queries from the logs (\clicks{}) for the \tracks{} and \podcasts{} datasets respectively. For the second hypothesis, out of the 376k broad queries generated, there are only 2\% and 1\% are exact matches with the set of queries from the logs. This shows the diversity of the generated queries from \cqg{}, and that it is not just copying input queries from the log.

\subsection{With the input entity}
Out of the 10k narrow queries generated by 
\cqg{} to test the first hypothesis, 25\% and 48\% are not a subset of the serialized entity for \tracks{} and \podcasts{} datasets respectively. For the second hypothesis, out of the 376k broad queries generated, a total of ~70\% of queries are not subsets of the serialized entity for both datasets. This shows that while for narrow queries substrings of the entity are the majority of the cases when generating broad queries this is not the case. Also, this indicates that for both cases the model is not always selecting parts of the input as the query.

\section{Dataset details}

For the \books{} dataset, we take into account the top two most-voted reviews and use the first 50 tokens. For the \tracks{} dataset, we use the first lyric line and the most frequent lyric line and employ a maximum of 25 descriptors. For the \podcasts{} dataset, we use the first 50 tokens of the description and of the transcript. For the \books{} and \tracks{} datasets we use a maximum of 25 playlists.

\end{document}
\endinput
%%
%% End of file `sample-authordraft.tex'.
