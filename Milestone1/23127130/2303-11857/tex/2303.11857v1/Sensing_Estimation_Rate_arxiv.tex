\documentclass[journal,10pt]{IEEEtran}

\IEEEoverridecommandlockouts
\usepackage{amsbsy}
\usepackage{amsthm}
\usepackage{amssymb}
%\makeatletter
%\thm@headfont{\sc}
%\makeatother
\usepackage{amsmath}
\usepackage{multirow}
\usepackage{epsfig}
\usepackage{subfigure}
\usepackage{graphics}
\usepackage{multicol}

%
\usepackage{epstopdf}
\usepackage{balance}
\usepackage{xcolor}
\usepackage{cite}
%\usepackage{citesort}
\usepackage{color}
\usepackage{algorithm}
\usepackage{algorithmic}
\usepackage{textcomp}
\usepackage{subfigure}
\usepackage{subfloat}
\usepackage{booktabs}
\usepackage{bm}
\usepackage{filecontents}
\usepackage{upgreek}
\usepackage{setspace}
\newtheorem{Def}{Definition}
\newtheorem{Them}{Theorem}
%\usepackage[marginal]{footmisc}
%\renewcommand{\thefootnote}{}
%\def\BibTeX{{\rm B\kern-.05em{\sc i\kern-.025em b}\kern-.08em
%		T\kern-.1667em\lower.7ex\hbox{E}\kern-.125emX}}
\def\RED{\color{red}}
\def\END{\color{black}}

\graphicspath{ {figures/}}


% correct bad hyphenation here
\hyphenation{op-tical net-works semi-conduc-tor}

\begin{document}

\title{\huge{Rethinking Estimation Rate for Wireless Sensing: \\ A Rate-Distortion Perspective}}

\author{Fuwang Dong,~\IEEEmembership{Member,~IEEE}, Fan Liu,~\IEEEmembership{Member,~IEEE}, Shihang Lu, Yifeng Xiong,~\IEEEmembership{Member,~IEEE}  \vspace{-2em}}
{}
% make the title area
\maketitle

% As a general rule, do not put math, special symbols or citations
% in the abstract or keywords.
\begin{abstract} Wireless sensing has been recognized as a key enabling technology for numerous emerging applications. For decades, the sensing performance was mostly evaluated from a reliability perspective, with the efficiency aspect widely unexplored. Motivated from both backgrounds of rate-distortion theory and optimal sensing waveform design, a novel efficiency metric, namely, the sensing estimation rate (SER), is defined to unify the information- and estimation- theoretic perspectives of wireless sensing. Specifically, the active sensing process is characterized as a virtual lossy data transmission through non-cooperative joint source-channel coding. The bounds of SER are analyzed based on the data processing inequality, followed by a detailed derivation of achievable bounds under the special cases of the Gaussian linear model (GLM) and semi-controllable GLM. As for the intractable non-linear model, a computable upper bound is also given in terms of the Bayesian Cram\'er-Rao bound (BCRB). Finally, we show the rationality and effectiveness of the SER defined by comparing to the related works.          
\end{abstract}

\begin{IEEEkeywords}
Sensing estimation rate, rate-distortion theory, waveform design, ISAC.  
\end{IEEEkeywords}



\IEEEpeerreviewmaketitle



\section{Introduction}\label{Introduction}

\subsection{Motivation and Related Works}
Sensing tasks can be roughly classified into three categories, \textit{detection}, \textit{estimation}, and \textit{recognition}, which are essentially deducing the desired parameters/features from the collected signals/data with respect to the targets. Most of the sensing performance metrics are reliability metrics, such as detection probability and mean square error (MSE), etc, with the effciency aspect of sensing widely unexplored \cite{liu2022integrated}. In light of the recent progress of integrated sensing and communications (ISAC) techniques \cite{liu2022integrated}, evaluating the sensing performance by a well-defined information quantity consistent with the communication metrics, is envisioned as an essential methodology in revealing the fundamental limits and performance trade-off of ISAC, which may provide profound insights into system design and resource allocation \cite{liu2022survey}.     

The rate-distortion theory, which initially characterizes the minimal information rate required to achieve a preset distortion in lossy data transmission \cite{thomas2006elements}, bridges the quantities between information and estimation. Based on this principle, the mutual information (MI) based radar waveform design has been presented \cite{bell1993information,tang2018spectrally}, where the MI between the observations and the target impulse response is maximized under the Gaussian linear model (GLM). In \cite{yang2007mimo}, the authors show that maximizing MI and minimizing minimum MSE (MMSE) through the waveform design lead to the same water-filling solution. Nevertheless, the operational meaning of the sensing MI, as well as its connection with various commonly employed sensing metrics still remain unclear, which limits the applications of the MI-based methods for sensing or ISAC systems.           

Recently, the capacity-distortion-cost trade-off of state-dependent memoryless channel was studied in \cite{ahmadipour2022information}, where the channel state parameters are simultaneously estimated through the backscattered signals while communicating with the users. Although this considered scenario is common in wireless communications, the yielded capacity-distortion trade-off is more applicable to the case where the sensing and communication channels are $\textit{strongly correlated}$, e.g., the communication user is also a target to be sensed. As a consequence, it is unable to address more general issue where there is only partial correlation between two channels. More related to this work, a radar estimation information rate has been defined in \cite{chiriyath2015inner} for target range estimation. Nevertheless, its operational meaning was not fully investigated in a rigorous manner. 

\subsection{Our Contributions}
To address the above issues, in this paper, we define the sensing estimation rate (SER) for the general sensing model from the perspective of rate-distortion theory and waveform design, where the sensing process is analogized as a virtual non-cooperative communication process. The SER is expected to be a stepping stone towards unifying the information-theoretic and estimation-theoretic metrics, thereby providing an analytical framework for depicting the performance gain of ISAC systems. The contributions are summarized as follows.       

\begin{itemize}
	\item Based on the rate-distortion theory, sensing process is characterized by an equivalent virtual lossy data transmission process. Namely, a virtual user ``transmits'' information of parameters through joint source-channel coding under the constraint of ``channel capacity''.
	
	\item The SER is defined as the minimal information rate required to achieve the lower bound of estimation error through the optimal waveform design, followed by the bounds analysis in terms of data processing inequality.  
	
	\item The achievable bounds are derived for the special cases of GLM and semi-controllable GLM. Furthermore, a computable upper bound of SER for the nonlinear Gaussian model is also given. 
	
	\item Under the proposed framework, we conclude that: (1) The results in \cite{yang2007mimo} is a weak version of SER; (2) The estimation information rate defined in \cite{chiriyath2015inner} is a special scalar case for SER; (3) The MI-based sensing system designs may be endowed with an operational estimation-theoretic meaning.  
\end{itemize} 



%\subsection{Integrated Sensing and Communication}\label{IntroISAC}
%Motivated by the need of extra frequency spectrum resources for the unprecedented proliferation of new wireless services, ISAC was initially proposed to release the frequency bands previously restricted to radar systems for shared use of communications \cite{ZL2019,CC2020}. The relevant research works span from radar and communication spectral coexistence, radar-communication cooperation, to dual-functional radar-communication (DFRC) design \cite{DW2017,LHzo2020,ComprehensiveS2021}. Recently, the concept and scope of ISAC have been formally given in \cite{liu2021integrated,CuiNet}, which attracts extensive research attention from both industry \cite{Huawei2021} and academia \cite{ZJ2021}. In contrast to the dedicated sensing or communication functionality, the ISAC design methodology exhibits two types of gains. First, the shared use of limited resources, namely, spectrum, energy and hardware platforms, results in improved efficiency for both sensing and communication (S\&C), and hence provides \textit{integration gain}. Second, mutual assistance between S\&C may further boost the dual performance, offering \textit{coordination gain} \cite{liu2021integrated}. Due to the numerous advantages offered by ISAC, it is envisioned to be a key enabler for many future applications including intelligent connected vehicles, Internet of Things, and smart homes and cities \cite{Automous2020,IoT2021}.
% 
%To evaluate the performance of a perceptive network, QoS metrics are needed for both S\&C services. While the communication QoS has been well studied from both efficiency (spectral and energy efficiency) and reliability (bit and symbol error rates), to the best of our knowledge, the sensing QoS still remains widely unexplored. In what follows, we first define sensing QoS for diverse applications.
%
%\subsection{The Definition of Sensing QoS}\label{DefinitionQoS}
%In general, sensing tasks can be roughly classified into four categories, i.e., detection, localization and tracking, imaging, and recognition \cite{liu2021integrated}. In this paper, we focus on the fundamental sensing QoS including the capability to detect, localize and track objects, and designate the QoS definition of the rest of sensing tasks as our future research.
%
%\textit{(1) Detection QoS:} Target detection refers to making binary or multiple decisions to identify the status of a target, e.g., present or absent. The common metrics include the probability of detection $P_\text{D}$, i.e., the probability that a target is declared when a target is in fact present, and the probability of false alarm $P_\text{FA}$, i.e., the probability that a target is declared but in fact absent. In radar application, it usually requires that the $P_\text{FA}$ has to be maintained below a pre-assigned threshold while maximizing the $P_\text{D}$, namely, the Neyman-Pearson criterion \cite{BookDetectionTheory}. A detailed description can be found in Section \ref{TarDetection}.
%
%\textit{(2) Localization QoS:} The localization of the static objects can be interpreted as parameter estimation problems for the time delay and angle of arrival (AoA). A straightforward metric to measure the localization performance is the mean squared error (MSE) between the true parameters and the estimated ones. However, the MSE is normally difficult to characterize. Alternatively, the Cr\'amer-Rao bound (CRB) for target estimation, which is known as a lower bound on the variance of an unbiased estimator, can be employed to measure the localization QoS \cite{BookEstimationTheory}. This will be detailed in Section \ref{TarLocalization}.    
%
%\textit{(3) Tracking QoS:} Target tracking refers to tracking the state variation (range, angle, velocity, etc.) of a moving target, e.g., a vehicle or a drone. Tracking tasks typically emerge in high-mobility ISAC scenarios, such as vehicle-to-everything (V2X) networks. In contrast to the conventional CRB that relies on the measured data only, a posterior Cr\'amer-Rao bound (PCRB) is introduced to measure the tracking QoS by considering the Fisher information provided by both the measured data as well as prior state models \cite{PCRB1998}. A detailed description can be found in Section \ref{TarTracking}.
% 
%Similar to the conventional communication-only scenarios, users in a perceptive network may also require different sensing QoS. For instance, important/sensitive targets (such as pedestrians, fast-moving vehicles, etc.) generally require high sensing QoS to prevent potential traffic accidents and to safeguard human lives. On the contrary, low sensing QoS could be tolerable to the static and inanimate objects. To this end, the base station (BS) can allocate the available resources to the users who have different S\&C needs, to improve the flexibility and capacity of the perceptive networks.
%     
%\subsection{Resource Allocation}\label{IntroRA}
%
%Intuitively, the more transmit resource is utilized to serve multiple users, the better S\&C QoS will be obtained. Unfortunately, resources (e.g., transmit power, bandwidth, etc.) are always limited for practical applications, resulting in a performance trade-off among the users. To that end, efficient resource allocation (RA) schemes are necessary to achieve optimal QoS performance. 
%
%On the one hand, the communication achievable rate depends on the power and bandwidth resources based on the Shannon's theorem. Hence, the bandwidth allocation and power control have been widely investigated to maximize the network's capacity \cite{KDcom2003,Propor2005,LTERA2013,GRA2017,NOMARA2018}. In particular, for the orthogonal frequency division multiplexing (OFDM) systems, the water-filling algorithm over inverse of the channel spectrum can be used to optimally allocate power and rate among subcarriers \cite{KDcom2003}. In \cite{Propor2005}, the authors further proposed a joint subchannel assignment and power allocation algorithm under the proportional fairness constraints. Moreover, the RA problems have also been extensively studied in various scenarios, including in the device-to-device communications in LTE-Advanced networks \cite{LTERA2013}, the slicing network based on 5G \cite{GRA2017}, and the non-orthogonal multiple access (NOMA) scheme \cite{NOMARA2018}.
%
%On the other hand, there is relatively less literature on the RA schemes for sensing compared to communications. In \cite{Yan2015}, the simultaneous multibeam and power allocation is developed to improve the worst-case tracking performance, where the number of beams is assumed less than the number of radiating elements. Additionally, the dwell time which determines the velocity estimation accuracy, is regarded as a limited resource to be allocated in phased array radar network \cite{Yan2017cooperative}. In \cite{ZHW2020}, the authors propose a power and bandwidth allocation method to minimize the PCRB of multiple targets in the tracking strategy.  
%
%There have been a number of works investigating the RA problems in ISAC systems. In \cite{SPL2017}, a DFRC transmitter that supports both communication and radar receivers was considered, where the power budget is allocated to radar waveform and information signal such that the probability of detection is maximized under the constraint of information rate requirement. In \cite{LY2017}, a power allocation scheme was proposed in the OFDM ISAC system, where the conditional mutual information (MI) between the random target impulse response and the received signal is adopted as the sensing performance metric. Although the conditional MI is a ``communication-friendly'' metric that has a similar expression to the communication information rate, and is applicable to all kinds of sensing tasks, the vague physical definition and weak correlation with traditional radar metrics limits its practical application. Moreover, the above-mentioned works mainly focused on power allocation for specific ISAC scenarios, without addressing the more general RA problem at a network level. The joint RA scheme for different kinds of system resources (e.g., power and bandwidth) for ISAC systems and networks still remains widely unexplored.

\begin{figure*}[!t]
	\centering
	\includegraphics[width=5in]{Rate_distortion_Revise.jpg}
	\caption{Active sensing as non-cooperative joint source-channel coding.}
	\label{rate_distortion_system}
\end{figure*}

\section{The Definition of Estimation Rate}\label{SystemModel}

\subsection{The Sensing System Model}
Let us consider a general sensing model under Gaussian channel, where the received signal can be expressed as
\begin{equation}\label{M1}
	\textbf{Y}=\textbf{H}(\bm{\upeta})\tilde{\textbf{X}}+\textbf{Z},
\end{equation}
where $\tilde{\textbf{X}} \in \mathbb{C}^{M_t \times T} $ represents the matrix of transmit waveform, $T$ is the number of discrete samples. $\textbf{Z} \in \mathbb{C}^{M_r \times T}$ denotes the additive noise which follows an independently and identically distributed (i.i.d.) circularly symmetric complex Gaussian distribution. $M_t$ and $M_r$ are the numbers of transmitting and receiving antennas. $\textbf{H}(\bm{\upeta}) \in \mathbb{C}^{M_r \times M_t}$ is the sensing channel matrix with respect to the latent parameters $\bm{\upeta} \in \mathbb{R}^{K}$. The sensing task aims at recovering $\bm{\upeta}$ from the noisy observation $\textbf{Y}$ via a well-designed estimator $\hat{\bm{\upeta}}$. Here, $\bm{\upeta}$  can be classified into several categories, which correspond to different sensing tasks.
\begin{itemize}
	\item \textit{Target Parameter Estimation} ($\bm{\upeta} \in \mathbb{R}^{K}$): In this case, $\bm{\upeta}$ refers to the physical parameters of targets, e.g., amplitude, delay, angle, Doppler, etc.
	
	\item \textit{Channel Estimation} ($\bm{\upeta}=\text{vec}({\textbf{H}})$): This case also corresponds to \textit{linear spectrum estimation}, where the receiver only requires the knowledge of channel state information (CSI) or channel impulse response (CIR).
	
	\item \textit{Target Detection} ($\bm{\upeta}\in \{0,1\}$): Here, $0$ and $1$ represent the absence and presence of the target in hypothesis testing, respectively. The estimator $\hat{\bm{\upeta}}$ is exactly the decision rule.    
\end{itemize} 

Furthermore, we highlight that (\ref{M1}) may describe a variety of sensing scenarios with different forms of $\textbf{H}(\bm{\upeta})$. Specifically, we take the following two commonly applied sensing models as examples.

$\bullet$ For angle estimation of targets using a MIMO radar equipped with co-located antennas, the CSI matrix $\textbf{H}(\bm{\upeta})$ has the following form \cite{li2007range}
\begin{equation}
\textbf{H}=\sum_{i=1}^L \alpha_i \textbf{a}(\theta_i)\textbf{b}^H(\theta_i),
\end{equation}
where $L$ is the number of the targets, $\alpha_i$ and $\theta_i$ are the complex reflection coefficient and the angle of the $i$-th target. $\textbf{a}(\theta)$ and $\textbf{b}(\theta)$ are the receiving and transmitting steering vectors, respectively. In this case, $\bm{\upeta}=[{\bm{\theta},\text{Re}\{\bm{\alpha}\}, \text{Im}\{\bm{\alpha}\}}]$.  

$\bullet$ For discrete-time orthogonal frequency division multiplexing (OFDM) system with $N$ subcarriers, the CIR matrix $\textbf{H}(\bm{\upeta})$ with sufficient cyclic prefix can be expressed by \cite{4600229} 
\begin{equation}
\textbf{H}=
{\scriptsize  \left[ {\begin{array}{*{20}{c}}
			{h(0,\tau_0)}&{0}&{\cdots}&{h(0,\tau_{L-1})}&{\cdots}&{h(0,\tau_1)}\\
			{h(1,\tau_1)}&{h(1,\tau_0)}&{0}&{\cdots}&{\cdots}&{h(0,\tau_2)}\\
			{\vdots}&{\vdots}&{\ddots}&{\vdots}&{\vdots}&{\vdots}\\
			{0}&{0}&{\cdots}&{h(N-1,\tau_{L-1})}&{\cdots}&{h(N-1,\tau_0)}
		\end{array}} 
\right]}
\end{equation}
where $h(n,m)$ represent the multi-path time- and frequency- selective fading channel at time $n$, with the expression of 
\begin{equation}
h(n,m)=\sum_{i=0}^L \alpha_i(n) \delta(m-\tau_i)e^{j2\pi f_{\text{D}_i} n},
\end{equation}
where $L$ is the number of path. The parameter vector $\bm{\upeta}=[\alpha_i, \tau_i, f_{\text{D}_i}]$ represent the time-varying path gain, time delay and Doppler frequency of the $i$-th path, respectively.

In this paper, we consider $\textbf{H}$ as an unstructured matrix that contains useful information about the targets for generic sensing model, rather than restricting to a specific scenario. The vector form of (\ref{M1}) can be rewritten by
\begin{equation}\label{VecModel}
	\textbf{y}=\textbf{X}\textbf{h}+\textbf{z},
\end{equation}  
where $\textbf{X}=\tilde{\textbf{X}}^T \otimes \textbf{I}_{M_r \times M_r}$ represents the equivalent waveform matrix. $\textbf{y}=\text{vec}(\textbf{Y})$, $\textbf{h}=\text{vec}(\textbf{H})$, $\textbf{z} = \text{vec}(\textbf{Z})\sim \mathcal{CN}(0, \sigma_z^2\textbf{I}_{M_r \times T})$, $\sigma_z^2$ is the variance of noise. 

We assume that the parameters take values on a set $\mathcal{A}$ with a prior distribution $p_{\bm{\upeta}}(\bm{\eta})$. The sensing performance can be measured by the expected distortion between the estimate and the ground truth, i.e.,   
\begin{equation} {\label{Dist}} 
	D=\mathbb{E}_{\bm{\upeta}}\left[d(\bm{\upeta},\hat{\bm{\upeta}})\right],
\end{equation}  
where $d:\mathcal{A} \times \hat{\mathcal{A}} \to \mathbb{R}$ is a distance metric of estimation error. The choice of $d$ depends on the specific sensing task. For instance, $D$ is the MSE with the Euclidean distance $d=\|\bm{\upeta}-\hat{\bm{\upeta}}\|_2^2$ for parameter estimation, and is equivalent to negative detection probability with the Hamming distance $d(\bm{\upeta},\hat{\bm{\upeta}})=\bm{\upeta} \oplus \hat{\bm{\upeta}}$ for target detection.

\subsection{The Definition of SER}
In an information-theoretic sense, wireless sensing and communication are intertwined with each other as an ``odd couple''. This motivates us to characterize wireless sensing as a procedure of non-cooperative joint source-channel coding. As shown in Fig. \ref{rate_distortion_system}, the target can be regarded as a virtual user which encodes the information of $\bm{\upeta}$, and communicates it to the sensing receiver through lossy data transmission. To be specific, the random parameter $\bm{\upeta}$ is regarded as a memoryless source, $\textbf{h}(\bm{\upeta})$ and $\textbf{y}$ are the input and output of the channel $p_{\textbf{y}|\textbf{h}}$, respectively. Finally, a distorted version of $\bm{\upeta}$ may be reconstructed from the noisy data $\textbf{y}$ at the receiver.   

It is evident that $\bm{\upeta} \to \textbf{h} \to \textbf{y} \to \hat{\bm{\upeta}}$ forms a Markov chain. In such a case, the rate-distortion function characterizes the minimum number of bits required to communicate $\bm{\upeta}$ to the receiver with a preset distortion $D$ between the source (ground truth of sensing parameters) and the reconstructed information (estimate). Furthermore, we highlight that the achievable distortion is determined by both the ``capacity'' of channel $p_{\textbf{y}|\textbf{h}}$ and the transmit waveform $\textbf{X}$. Accordingly, we provide the following definition of SER to bridge the quantities between the information and estimation.   

\begin{Def}
	The Estimation Rate $R_{\rm{E}}$ for sensing systems modeled by (\ref{VecModel}), is defined as
	\begin{equation}\label{ER}
		R_\text{E}(D_L)=\mathop { \rm{min} } \limits_{P_{\hat{\bm{\upeta}}|\bm{\upeta}}:\mathbb{E}\left[d(\bm{\upeta},\hat{\bm{\upeta}})\right] = D_L} I(\bm{\upeta};\hat{\bm{\upeta}}),
	\end{equation}
	where $I(\bm{\upeta};\hat{\bm{\upeta}})$ represents the MI between $\bm{\upeta}$ and $\hat{\bm{\upeta}}$, $D_L$ is the lower bound of the achievable distortion over all controllable operations, i.e., 
	\begin{equation}\label{DL}
		D_L = \mathop { \rm{min} }\limits_{\hat{\bm{\upeta}} \in \hat{\mathcal{A}}, \kern 2pt \mathbf{X} \in \mathcal{X}} \kern 5pt \mathbb{E}\left[d(\bm{\upeta},\hat{\bm{\upeta}})\right].
	\end{equation}
	Here, $\mathcal{X}$ represents the feasible set of transmit waveforms. Each element in $\mathcal{X}$ satisfies $ \| \mathbf{X}\|_F^2 \le TP_T$, where $P_T$ is the power budget at transmitter. 
\end{Def}

\textbf{Remark 1:} It should be highlighted that not only the choice of estimator $\hat{\bm{\upeta}}$, but also the waveform design $\textbf{X}$ is included in (\ref{DL}), which is different from the conventional rate-distortion function in lossy source coding. In other words, the SER defined can be interpreted as the minimal information rate to communicate $\bm{\upeta}$ from the target (virtual communication transmitter) to the sensing receiver through the channel $p_{\textbf{y}|\textbf{h}}$, such that the minimum estimation error (distortion) can be achieved at the receiver. 


\subsection{The Bounds for SER}
Note that the SER is a monotonically decreasing function of the distortion, namely, reaching a lower $D_L$ would require higher $R_\text{E}$, and vice versa. By recalling the Markov chain $\bm{\upeta} \to \textbf{h} \to \textbf{y} \to \hat{\bm{\upeta}}$, we have    
\begin{equation} \label{SourceCoding}
	R_{\text{E}} =\mathop { \rm{minimize} } \limits_{P_{\hat{\bm{\upeta}}|\bm{\upeta}}:\mathbb{E}\left[d(\bm{\upeta},\hat{\bm{\upeta}})\right] = D_L} I(\bm{\upeta};\hat{\bm{\upeta}}) \le I(\bm{\upeta};\hat{\bm{\upeta}}) \mathop \le \limits^{(a)} I(\textbf{y};\textbf{h}), 
\end{equation}
where $(a)$ is due to data processing inequality. $I(\textbf{y};\textbf{h})$ is the abbreviated form of $I(\textbf{y};\textbf{h}|\textbf{X})$ without causing confusion. Furthermore, $I(\textbf{y};\textbf{h})$ can be upper bounded by the following inequality
\begin{equation} \label{ChannelCoding}
	I(\textbf{y};\textbf{h}) \mathop \le \limits^{(b)} \mathop \text{max} \limits_{\textbf{X} \in \mathcal{X}}  I(\textbf{y};\textbf{h}) \triangleq I(\textbf{X}^*) \mathop \le \limits^{(c)} \sup_{\textbf{X} \in \mathcal{X}, p_\textbf{h}} I(\textbf{y};\textbf{h}) \triangleq C(\textbf{X}^*),
\end{equation}
where $I(\textbf{X}^*)$ is the MI with respect to optimal waveform $\textbf{X}^*$ for a given channel prior $p_\textbf{h}$. $C(\textbf{X}^*)$ is the maximum MI by maximizing over both channel prior $p_\textbf{h}$ and transmit waveform set $\mathcal{X}$. The conditions for the equality are given as follows.   
\begin{itemize}
	\item $(a)$ holds if and only if $I(\bm{\upeta};\textbf{y}|\hat{\bm{\upeta}})=0$;\footnote{In this case, we have $I(\bm{\upeta};\hat{\bm{\upeta}})=I(\textbf{y};\bm{\upeta})$ and apply the fact that $I(\textbf{y};\bm{\upeta}) = I(\textbf{y};\textbf{h})$, as $\textbf{h}$ is a deterministic function of $\bm{\upeta}$.}   
	
	\item $(b)$ holds by the optimal waveform design;
	
	\item $(c)$ holds when the prior distribution $p_{\textbf{h}}$ is Gaussian.  
\end{itemize}
Thus, according to the monotonicity of rate-distortion function, the distortion can be lower bounded by
\begin{equation} 
	D_L \ge D(I(\textbf{X}^*)) \ge D(C(\textbf{X}^*)).
\end{equation}

\textbf{Remark 2:} In the above discussions, the specific encoding and decoding methods are omitted since there is no actual coding happened. The bound derived is based on the extremum principle and the data processing inequality. This process is similar to the converse proof for the source-channel separation theorem, in the sense that (\ref{SourceCoding}) depends only on the prior (source) and the loss function, and (\ref{ChannelCoding}) depends only on the channel statistical model \cite{yp2023information}. It shows that the SER quantitatively connects the reliability metrics with the efficiency metrics by providing a lower bound $D(C(\textbf{X}^*))$ for any achievable distortion. 
  

%The ER defined bridges the quantities of information and estimation, which is expected to provide a useful tool for the fundamental limits analysis in ISAC systems. In the next section, we will provide a new rate-distortion perspective for the existing MI-based methods in ISAC system design under the framework of ER defined.

\section{The Achievable Bounds for Special Cases} \label{Example}
Since it is difficult to obtain explicit expressions of SER in general, in this section, we reveal further insights into the SER metric by focusing on the \textit{quadratic Gaussian problem}. To be more specific, we consider the MSE distortion metric and Gaussian source $\bm{\upeta}$. According to the results of rate-distortion function in \cite[Th.10.3.2 \& 10.3.3]{thomas2006elements}, we have\footnote{The coefficient $\frac{1}{2}$ is vanished since the complex variable is considered.}   
\begin{equation} \label{RI}
	R_{\text{E}}(D_L) = \log^+ \frac{\sigma_{\upeta}^2}{D_L}, 
\end{equation}
with $\sigma_{\upeta}^2$ being the variance of $\upeta$ for scalar case, and   
\begin{equation}\label{VRD}
	R_{\text{E}}(D_L) = \sum_{i=1}^M \log \frac{\sigma^2_{\upeta_i}}{D_i}, \kern 5pt 	D_i = \left\{
	\begin{aligned}
		\xi, \kern 2pt \text{if} \kern 2pt \xi < \sigma^2_{\upeta_i}, \\
		\sigma^2_{\upeta_i}, \kern 2pt \text{if} \kern 2pt \xi \ge \sigma^2_{\upeta_i},
	\end{aligned} \right.
\end{equation}
for $M$ independent Gaussian random variables with variance $\sigma_{\upeta_i}^2$, where $\xi$ is the inverse water-filling factor such that $\sum_{i=1}^M D_i=D_L$. In what follows, we will derive the achievable bounds for the cases of GLM, semi-controllable GLM and non-Gaussian model based on the explicit expressions of rate-distortion in (\ref{RI}) and (\ref{VRD}), and reveal the connections between the SER defined in this paper and the existing works.

\subsection{Case 1: GLM}\label{IRE}
Let us first consider the GLM model, which is typical in communication channel or radar impulse response estimation \cite{yang2007mimo,tang2018spectrally,bell1993information}. In such a case, we have $\bm{\upeta} = \textbf{h}$ with $\bm{\upeta} \sim \mathcal{CN}(\textbf{0},\bm{\Sigma}_{\bm{\upeta}})$, where $\bm{\Sigma}_{\bm{\upeta}}$ represents the covariance matrix. It is well known that the lower bound of the estimation error can be achieved by the MMSE estimator for this classical GLM, which leads to the following theorem. 
\begin{Them}
	For GLM, let $D_L = \rm{MMSE}(\mathbf{X}^*)$, where $\mathbf{X}^*$ is the optimal waveform that minimizes the MMSE. Under such a setting, the SER achieves its upper bound, i.e.,     
	\begin{equation}\label{T1}
		R_\text{E}(D_L)=I(\mathbf{X}^*)=C(\mathbf{X}^*).
	\end{equation}
\end{Them}
\textit{Proof}: The proof is relegated to Appendix \ref{AppendixA}. $\hfill\blacksquare$

\textbf{Remark 3}: We highlight that the optimal waveform is necessary since the first equality in (\ref{T1}) generally does not hold for arbitrary waveform in the vector case. The pioneering work \cite{yang2007mimo} has shown that minimizing MMSE and maximizing MI $I(\textbf{y};\bm{\upeta})$ lead to a same solution for radar waveform design. This result can be seen as a weak version of \textit{Theorem 1}, since (\ref{T1}) shows a completely equivalent relations. In addition, \textit{Theorem 1} also endows the MI-based sensing system design (e.g., \cite{tang2018spectrally}) with the estimation-theoretic meaning. That is, for impulse response estimation under GLM, maximizing MI is equivalent to achieving the SER, thereby reaching the MMSE.          

\subsection{Case 2: Semi-Controllable GLM}
In this subsection, we consider the parameter estimation problem with a linear (or approximate linear) function $\textbf{h}(\bm{\upeta})=\textbf{F}\bm{\upeta}$. An example can be found in \cite{liu2020radar}, where $\textbf{F}$ is the Jacobian matrix in the Kalman filtering. Although the system model is still a GLM by treating $\textbf{X}\textbf{F}$ as a new matrix, it is quite different from the model in \ref{IRE} in terms of waveform design. Since $\textbf{F}$ is a fixed (uncontrollable) matrix related to the channel state, we refer to the model $\textbf{y}=\textbf{X}\textbf{F}\bm{\upeta}+\textbf{z}$ as the semi-controllable GLM, leading to a different waveform design strategy.
\begin{Them}
	For the semi-controllable GLM, let $D_L = \rm{MMSE}(\mathbf{X}^*)$, we have
	\begin{equation}\label{T2}
		R_\text{E}(D_L) \le I(\mathbf{X}^*) = C(\mathbf{X}^*),
	\end{equation}
	where the equality holds if the right singular space of $\mathbf{F}$ is equal to the eigenspace of $\bm{\Sigma_{\bm{\upeta}}}$.
\end{Them}    
\textit{Proof}: The proof is relegated to Appendix \ref{AppendixB}. $\hfill\blacksquare$ 
  
{\it Theorem 2} shows that although maximizing MI and minimizing MMSE still leads to the same waveform solution $\mathbf{X}^*$, the SER and MI is not necessary equivalent from rate-distortion perspective. Namely, it may require less information rate than MI for achieving distortion $D_L$.
\vspace{-0.1em}
\subsection{Discussion on Non-Gaussian Model}
For non-linear $\textbf{h}(\bm{\upeta})$ with Gaussian distributed $\bm{\upeta}$, there are even no explicit expressions for MI and MMSE in general, since $\textbf{h}(\bm{\upeta})$ is not Gaussian. To this end, we consider the Bayesian Cram\'er-Rao bound (BCRB) of $\bm{\upeta}$, which provides a lower bound for the MSE of weakly unbiased estimators. Let $D_L = \text{BCRB}(\textbf{X}^*)$, we have     
\begin{equation}\label{NonGLM}
	R_\text{E}(D_L) \le \sum_{i=1}^M\log \left[\left(\sigma_z^{-2}\bm{\Lambda}_{\text{G}_{ii}}\lambda-1\right)^+ +1\right],
\end{equation}
where the equality holds if the right singular space of $\textbf{G}$ is equal to the eigenspace of $\textbf{J}_\text{P}^{-1}$. The definitions of the matrices $\textbf{G}$, $\textbf{J}_\text{P}$, $\bm{\Lambda}_{\text{G}}$ and the detailed proof are given in Appendix \ref{AppendixC}. 

\textbf{Remark 4}: The BCRB is known as a loose bound and is not asymptotically achieved by the MMSE, especially for the case where the Jacobian matrix $\textbf{F}_{\bm{\upeta}} = \partial \textbf{h}(\bm{\upeta})/\partial \bm{\upeta}$ depends on the parameter $\bm{\upeta}$. However, if the Jacobian matrix $\textbf{F}_{\bm{\upeta}}=\textbf{F}$ is independent to $\bm{\upeta}$, the BCRB is an asymptotically tight bound. 
	
By resorting to the definition in (\ref{NonGLM}), let us reconsider the non-linear scalar time-delay estimation in \cite{chiriyath2015inner} with the following received signal model  
\begin{equation}
	y(t)=\alpha x(t-\upeta)+z(t),
\end{equation} 
where the parameter $\upeta = \tau$ denote the time-delay of the reflecting signal. The BCRB of $\bm{\upeta}$ can be calculated by
\begin{equation}\label{CRB}
	\text{BCRB}_{\upeta} = (J_{\text{D}} + J_\text{P})^{-1}= (\frac{1}{\mathbb{E}_{\upeta}[\sigma^2_{\text{CRB}}]}+\frac{1}{\sigma^2_{\upeta}})^{-1},
\end{equation} 
where $\sigma^2_{\text{CRB}}=(8 \pi^2 B^2_\text{rms} \text{SNR})^{-1}$ represents the Cram\'er-Rao bound for a deterministic parameter $\eta$ and SNR is the signal to noise ratio. $B^2_\text{rms}=\int f^2 \left| {X(f)} \right|^2 df/\int \left| {X(f)} \right|^2 df$ is the effective bandwidth with $X(f)$ being the Fourier transform of the waveform $x(t)$. It is evident that we have $\mathbb{E}_{\upeta}[\sigma^2_{\text{CRB}}]=\sigma^2_{\text{CRB}}$, since $\sigma^2_{\text{CRB}}$ is independent to the true value of $\upeta$. In such a case, the BCRB is asymptotically tight and the equality in (\ref{NonGLM}) holds with scalar $\upeta$. Therefore, by combining Definition 1 and formula (\ref{RI}), the SER can be obtained by
\begin{equation}\label{BLS}
	R_\text{E}=\log \left(\frac{\sigma^2_{\upeta}}{\text{BCRB}_\upeta} \right)^+=\log \left(1+\frac{\sigma^2_{\upeta}}{\sigma^2_{\text{CRB}}}\right).
\end{equation} 

\textbf{Remark 5}: The radar estimation information rate has been defined in \cite{chiriyath2015inner} without giving a clear explanation. It should be noted that $R_{\text{E}}$ is consistent with radar estimation information rate given in \cite[formula (16)]{chiriyath2015inner} by omitting the coefficient of pulse repetition interval. However, the SER defined in this paper has a clear information-estimation relation from the perspective of rate-distortion theory and can be extended to more general scenarios.   
\vspace{-1.1em}

\section{Conclusion} 
This paper has proposed a novel efficiency metric for wireless sensing, namely, the sensing estimation rate (SER). Under the framework of the proposed SER, the sensing process can be analogized as an uncooperative lossy data transmission through joint source-channel coding. The bounds of SER and its achievability have been derived for the Gaussian linear model (GLM) and semi-controllable GLM. Finally, by comparing to the existing works, we show that the proposed framework may provide new information-theoretic insights into wireless sensing as well as optimal waveform design.
\vspace{-1.1em}
\begin{appendices}  
	\section{}\label{AppendixA} 
	\textit{Scalar Case}: The MMSE can be calculated by 
	\begin{equation}\label{MMSE_s} 
		D_L = \text{MMSE} = \frac{\sigma_{\upeta}^2}{1+|x|^2\sigma_{\upeta}^2/\sigma_z^2}=\frac{\sigma_{\upeta}^2}{1+\text{snr}},
	\end{equation} 
	where snr is defined by $|x|^2\sigma_{\upeta}^2/\sigma_z^2$. We have 
%	By substituting (\ref{MMSE_s}) into (\ref{RI}), we have 
	\begin{equation} 
		R_{\text{E}}(D_L) = \log^+ \sigma_{\upeta}^2/D_L  = \log(1+\text{snr}) = C.
	\end{equation} 
	
	\textit{Vector Case}: Here, we assume that the covariance matrix $\bm{\Sigma_{\bm{\upeta}}} \in \mathbb{C}^{M \times M}$ has full rank for simplicity and $M=M_rM_t$. The associated eigenvalue decomposition can be expressed as $\bm{\Sigma_{\bm{\upeta}}} = \textbf{U}\bm{\Lambda}\textbf{U}^H$, where $\bm{\Lambda}=\text{diag}\{\sigma^2_{\upeta_1},\cdots,\sigma^2_{\upeta_M}\}$ is a diagonal matrix with each diagonal entry given by a real and non-negative eigenvalue, the columns of the unitary matrix $\textbf{U}$ are the eigenvectors. The maximum MI for $I(\textbf{y};\bm{\upeta})$ is
	\begin{equation}\label{MaxMI}
		\begin{aligned}
			I(\textbf{X}^*) &= \mathop {\max} \limits_{\textbf{X} \in \mathcal{X}} \kern 2pt \log \left[\det (\sigma_z^2\bm{\Sigma_{\bm{\upeta}}}\textbf{X}^H\textbf{X}+\textbf{I})\right] \\
			%		&=\sum_{i=1}^M\log \left[\sigma_z^{-2}\sigma^2_{\eta_i}\left(\lambda-\frac{\sigma_z^2}{\sigma^2_{\eta_i}}\right)^++1\right] \\
			&=\sum_{i=1}^M\log \left[\left(\sigma_z^{-2}\sigma^2_{\upeta_i}\lambda-1\right)^+ +1\right],
		\end{aligned} 
	\end{equation}
	and the minimum MMSE is
	\begin{equation}\label{MinMMSE}
		\begin{aligned}
			\text{MMSE}(\textbf{X}^*) &= \mathop {\min} \limits_{\textbf{X} \in \mathcal{X}} \kern 2pt \text{tr} \left\{ \left( \sigma_z^{-2} \textbf{X}^H\textbf{X} + \bm{\Sigma_{\bm{\upeta}}}^{-1} \right)^{-1} \right\} \\
			&=\sum_{i=1}^M \frac{\sigma^2_{\upeta_i}}{\left(\sigma_z^{-2}\sigma^2_{\upeta_i}\lambda-1\right)^+ +1},  \\
		\end{aligned} 
	\end{equation}
	where $\lambda$ is the water-filling factor such that $\sum_{i=1}^M (\lambda-\sigma_z^2/\sigma^2_{\upeta_i})^+=P_T$. Moreover, (\ref{MaxMI}) and (\ref{MinMMSE}) have the same solution with the form of 
	\begin{equation}\label{solution}
		\textbf{X}^* = \bm{\Phi} \left( \text{diag}\left[(\lambda-\frac{\sigma_z^2}{\sigma^2_{\upeta_1}})^+, \cdots, (\lambda-\frac{\sigma_z^2}{\sigma^2_{\upeta_M}})^+\right]  \right)^{\frac{1}{2}} \textbf{U}^H  
	\end{equation}
	where $\bm{\Phi}$ is an $M_rT \times M$ matrix with orthogonal columns. The reader is referred to \cite{yang2007mimo} for details. Subsequently, it can be verified that
	\begin{equation}\label{DDD}
		D_L=\text{MMSE}(\textbf{X}^*), \kern 5pt D_i=\frac{\sigma^2_{\upeta_i}}{\left(\sigma_z^{-2}\sigma^2_{\upeta_i}\lambda-1\right)^+ +1},
	\end{equation}
	meet the inverse water-filling constraints in (\ref{VRD}) when we choose $\xi=\sigma_z^2/\lambda$. Thus, by substituting (\ref{DDD}) into (\ref{VRD}) and comparing to (\ref{MaxMI}), we have 
	\begin{equation}
		R_\text{E}(D_L) = I(\textbf{X}^*)=C(\textbf{X}^*),
	\end{equation}   
	which completes the proof.
	
	\section{}\label{AppendixB}
	The proof for the scalar case is straightforward as similar to that in Appendix \ref{AppendixA}, and hence is omitted. For the vector case, The maximum MI for $I(\textbf{y};\bm{\upeta})$ can be obtained by
	\begin{equation}\label{FMaxMI}
		\begin{aligned}
			I(\textbf{X}^*) &= \mathop {\max} \limits_{\textbf{X} \in \mathcal{X}} \kern 2pt \log \left[\det(\sigma_z^{-2}\textbf{F}\bm{\Sigma_{\bm{\upeta}}}\textbf{F}^H\textbf{X}^H\textbf{X}+\textbf{I})\right]\\
			&=\sum_{i=1}^M\log \left[\left(\sigma_z^{-2}\bm{\Lambda}_{\text{F}_{ii}}\lambda-1\right)^+ +1\right],
		\end{aligned}
	\end{equation} 
	where the diagonal matrix $\bm{\Lambda}_{\text{F}}$ satisfies the eigenvalue decomposition $\textbf{F}\bm{\Sigma_{\bm{\eta}}}\textbf{F}^H = \textbf{U}_\text{F}\bm{\Lambda}_{\text{F}}\textbf{U}_\text{F}^H$. The optimal solution $\textbf{X}^*$ of (\ref{FMaxMI}) has the similar form in (\ref{solution}) through replacing $\textbf{U}$ and $\sigma^2_{\upeta_i}$ by $\textbf{U}_\text{F}$ and diagonal entries of $\bm{\Lambda}_{\text{F}_{ii}}$, respectively. 
	
Now we will show that minimizing MMSE still has the same solution with maximizing MI. Note that the minimum MMSE can be expressed by
	\begin{equation}\label{MMSE_semi}
		\text{MMSE}(\textbf{X}^*) = \mathop {\min} \limits_{\textbf{X} \in \mathcal{X}} \kern 2pt \text{tr} \left\{ \left( \sigma_z^{-2} \textbf{F}^H\textbf{X}^H\textbf{X}\textbf{F} + \bm{\Sigma_{\bm{\upeta}}}^{-1} \right)^{-1} \right\}.
	\end{equation}
	Meanwhile, $I(\textbf{X}^*)$ in (\ref{FMaxMI}) can be recast by
	\begin{equation}
		\begin{aligned}
			\mathop {\max} \limits_{\textbf{X} \in \mathcal{X}} \kern 2pt \log \left[\det(\sigma_z^{-2}\textbf{F}^H\textbf{X}^H\textbf{X}\textbf{F}+\bm{\Sigma_{\bm{\upeta}}}^{-1})\right]+\log\left[\det(\bm{\Sigma_{\bm{\upeta}}})\right],\\
		\end{aligned}
	\end{equation}
	which has the same solution with
	\begin{equation} 
		-\mathop {\min} \limits_{\textbf{X} \in \mathcal{X}} \kern 2pt \log \left[\det \left[(\sigma_z^{-2}\textbf{F}^H\textbf{X}^H\textbf{X}\textbf{F}+\bm{\Sigma_{\bm{\upeta}}}^{-1})^{-1}\right]\right].
	\end{equation} 
	Since the function $\log\det(\cdot)$ and $\text{tr}(\cdot)$ are both monotonically increasing on matrix $\textbf{A}$ with respect to the positive semi-definite cone, then each optimal point that minimize $\text{tr}(\textbf{A})$ will also minimize $\log\det(\textbf{A})$ and vice versa \cite{naghibi2010optimal}. Thus, by recalling the eigenvalue decomposition of $\bm{\Sigma_{\bm{\upeta}}}$,  we have 
	\begin{equation}
		\begin{aligned}
			\text{MMSE}(\textbf{X}^*) &= \text{tr} \left\{ \left( \sigma_z^{-2} \bm{\Lambda}^{\frac{1}{2}}\textbf{U}^H\textbf{F}^H\textbf{X}^{*H}\textbf{X}^*\textbf{F}\textbf{U}\bm{\Lambda}^{\frac{1}{2}} + \textbf{I} \right)^{-1} \bm{\Lambda} \right\} \\
			&\mathop = \limits^{(d)} \text{tr} \left\{ \left( \sigma_z^{-2}\textbf{V} \tilde{\bm{\Lambda}}^H_\text{F}\textbf{U}^H_\text{F}\textbf{X}^H\textbf{X}\textbf{U}_\text{F}\tilde{\bm{\Lambda}}_\text{F}\textbf{V}^H + \textbf{I} \right)^{-1} \bm{\Lambda} \right\} \\
			&\mathop = \limits^{(e)} \text{tr} \left\{ \left( \sigma_z^{-2}\textbf{V} \bm{\Lambda}_\text{X}\bm{\Lambda}_\text{F}\textbf{V}^H + \textbf{I} \right)^{-1} \bm{\Lambda} \right\} \\
			&\mathop \ge \limits^{(f)} \sum_{i=1}^M \frac{\sigma^2_{\upeta_i}}{\left(\sigma_z^{-2}\bm{\Lambda}_{\text{F}_{ii}}\lambda-1\right)^+ +1}. 
		\end{aligned} 
	\end{equation}
	Here, $(d)$ holds due to the singular value decomposition of $\textbf{F}\textbf{U}\bm{\Lambda}^{\frac{1}{2}}=\textbf{U}_\text{F}\tilde{\bm{\Lambda}}_\text{F}\textbf{V}^H$, where $\tilde{\bm{\Lambda}}_\text{F} \in \mathbb{R}^{M \times M_rT} $ is the singular value matrix satisfying $\tilde{\bm{\Lambda}}_\text{F}\tilde{\bm{\Lambda}}^H_\text{F}=\bm{\Lambda}_{\text{F}}$. $(e)$ holds by substituting the expression of $\textbf{X}^*$, where diagonal matrix $\bm{\Lambda}_\text{X}$ represents the eigenvalue matrix obtained by the water-filling method. $(f)$ is due to the following lemma.
	
	\textit{Lemma 1}: For a positive definite Hermitian matrix $\textbf{A} \in \mathbb{C}^{M \times M}$ and a diagonal matrix $\bm{\Lambda}$ with its entries are positive, then the following inequality holds:
	\begin{equation} 
		\text{tr}({\textbf{A}^{-1}\bm{\Lambda}}) \ge \sum_{i=1}^{M} \frac{\bm{\Lambda}_{ii}}{\textbf{A}_{ii}},
	\end{equation}     
	the equality holds if and only if $\textbf{A}$ is diagonal. 
	
	The proof of \textit{Lemma 1} can be readily obtained through the similar procedure in \cite[Appendix I]{ohno2004capacity}. Again, according to the rate-distortion function (\ref{VRD}), we have
	\begin{equation}
		R_\text{E}(D_L) \le R_\text{E}\left(\sum_{i=1}^M \frac{\sigma^2_{\upeta_i}}{\left(\sigma_z^{-2}\bm{\Lambda}_{\text{F}_{ii}}\lambda-1\right)^+ +1}\right)=C(\textbf{X}^*),
	\end{equation}  
	where the equality holds if $\textbf{V}$ is identity matrix. It implies that the right singular space of $\textbf{F}$ is equal to the eigenspace of $\bm{\Sigma_{\bm{\upeta}}}$. 
	
	\section{}\label{AppendixC}
	The Bayesian Fisher information matrix (BFIM) of $\bm{\upeta}$ conditioned on $\textbf{X}$ can be expressed by
	\begin{equation}
		\begin{aligned}
			\textbf{J}_{\bm{\upeta}|\textbf{X}}&= \textbf{J}_{\text{D}} + \textbf{J}_\text{P}=\mathbb{E}_{\bm{\upeta}}\left[ \sigma^{-2}_z \textbf{F}^H_{\bm{\upeta}} \textbf{X}^H \textbf{X} \textbf{F}_{\bm{\upeta}}  \right]+ \textbf{J}_\text{P} \\
			&\mathop = \limits^{(g)} \sigma^{-2}_z \textbf{G}^H \textbf{X}^H \textbf{X} \textbf{G} + \textbf{J}_\text{P},
		\end{aligned}
	\end{equation} 
	where $\textbf{F}_{\bm{\upeta}} = \partial \textbf{h}(\bm{\upeta})/\partial \bm{\upeta}$ is the Jacobin matrix. $\textbf{J}_\text{P}$ is the prior Fisher information provided by the prior distribution $p_{\bm{\upeta}}(\bm{\eta})$, which can be obtained by
	\begin{equation}
		\textbf{J}_\text{P} = \mathbb{E} \left[ \frac{\partial \ln p_{\bm{\upeta}}(\bm{\eta})}{\partial \bm{\upeta}} \frac{\partial \ln p_{\bm{\upeta}}(\bm{\eta})}{\partial \bm{\upeta}^T}   \right].
	\end{equation} 
	$(g)$ follows the results of \cite[Proposition 2]{xiong2022flowing}, where $\textbf{G}$ can be expressed by 
	\begin{equation}
		\textbf{G}=\sqrt{\lambda_\Psi}\text{mat}_{M \times K}(\textbf{u}),
	\end{equation}   
	where $\text{mat}_{M \times K} (\textbf{u})$ denotes the $M \times K$ matrix satisfying $\text{vec}(\text{mat}_{M \times K} (\textbf{u}))=\textbf{u}$, and
	\begin{equation}
		\bm{\Psi}=\mathbb{E}\left[{\text{vec}(\textbf{F}_{\bm{\upeta}})\text{vec}(\textbf{F}_{\bm{\upeta}})^H}\right]=\lambda_\Psi\textbf{u}\textbf{u}^H.
	\end{equation}
	The rightmost equality is the eigenvalue decomposition of matrix $\bm{\Psi}$. The above process follows the principle of Choi representation theory, whose details can be found in \cite{xiong2022flowing}. Therefore, the minimum BCRB is   
	\begin{equation}
		\text{BCRB}_{\bm{\upeta}}=\mathop {\min} \limits_{\textbf{X} \in \mathcal{X}} \kern 2pt \text{tr} \left\{ \left( \sigma_z^{-2} \textbf{G}^H \textbf{X}^H \textbf{X} \textbf{G} + \textbf{J}_\text{P} \right)^{-1} \right\},
	\end{equation}
	which has the similar form with (\ref{MMSE_semi}). Following the same procedure in Appendix \ref{AppendixB}, we have 
	\begin{equation}
		R_\text{E}(\text{BCRB}_{\bm{\upeta}}) \le \sum_{i=1}^M\log \left[\left(\sigma_z^{-2}\bm{\Lambda}_{\text{G}_{ii}}\lambda-1\right)^+ +1\right],
	\end{equation}
	where the diagonal matrix $\bm{\Lambda}_{\text{G}}$ satisfies the eigenvalue decomposition $\textbf{G}\textbf{J}_{\text{P}}\textbf{G}^H = \textbf{U}_\text{G}\bm{\Lambda}_{\text{G}}\textbf{U}_\text{G}^H$. This completes the proof.
	
\end{appendices} 
      
\bibliographystyle{IEEEtran}
% argument is your BibTeX string definitions and bibliography database(s)
\bibliography{IEEEabrv,Sensing_Estimation_Rate_arxiv}


% that's all folks
\end{document}


