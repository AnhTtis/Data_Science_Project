% CVPR 2024 Paper Template; see https://github.com/cvpr-org/author-kit

\documentclass[10pt,twocolumn,letterpaper]{article}

%%%%%%%%% PAPER TYPE  - PLEASE UPDATE FOR FINAL VERSION
\usepackage{cvpr}              % To produce the CAMERA-READY version
% \usepackage[review]{cvpr}      % To produce the REVIEW version
% \usepackage[pagenumbers]{cvpr} % To force page numbers, e.g. for an arXiv version

% Import additional packages in the preamble file, before hyperref
% \usepackage[latin1]{inputenc}
\usepackage[british]{babel}
\usepackage[all]{xy}
\usepackage{amscd}
\usepackage{amssymb}
\usepackage{amsthm}
\usepackage{enumitem}
\usepackage{mathrsfs,bbm}
\usepackage{xcolor,graphicx}
\usepackage{graphics}
\usepackage{soul}
\usepackage{comment}
\usepackage[all]{xy}
\usepackage{amscd}
\usepackage{amssymb,amsmath,latexsym}
\usepackage{amsthm}
\usepackage{enumitem}
\usepackage{mathrsfs,bbm}
\usepackage{dsfont}
\usepackage{tikz-cd}
\usepackage[T1]{fontenc}
\usepackage[utf8]{inputenc}  
 %
%%%%%%%%%%%%%%%%%%%%%%%%%%%%%%%%%%
%pagestyle
%%%%%%%%%%%%%%%%%%%%%%%%%%%%%%%%%%
%\pagestyle{plain}
\textwidth=430pt
\headsep=.7cm
\evensidemargin=15pt
\oddsidemargin=15pt
\leftmargin=0cm
\rightmargin=0cm
%%
%%%%%%%%%%%%%%%%%%%%%%%
\newcommand*\fixitem {\item[]%
  \refstepcounter{enumi}\hskip-\leftmargin\labelenumi\hskip\labelsep}
\newtheorem*{mainthm}{Main Theorem}
\newtheorem*{mainthm1}{Theorem}
\newtheorem*{maincor}{Corollary}
\usepackage[colorlinks=true]{hyperref}
\DeclareMathOperator{\Forall}{\forall}
\DeclareMathOperator{\Exists}{\exists}
\DeclareMathOperator{\ord}{ord}
\newcommand{\phiD}{\varphi_D}
\newcommand{\phiDI}{\varphi_{\mathbf{D}_I}}
\newcommand{\phiDIj}{\varphi_{\mathbf{D}_I (j)}}
\newcommand{\phiH}{\varphi_H}
\newcommand{\phiTimes}{\phiD \otimes \phiH}
\newcommand{\phiTimesDI}{\varphi_{\mathbf{D}_I} \otimes \phiH}
\newcommand{\R}{\mathscr{A}}
\newcommand{\X}{\mathscr{X}}
\newcommand{\Xf}{\mathscr{X}_{(k_0 ,i)}[r_0]}
\newcommand{\Xfr}{\mathscr{X}_{(k_0,i)}[r]}
\newcommand{\hotimes}{\widehat{\otimes}}
\newcommand{\C}{\mathbb{C}_p}
\newcommand{\V}{\mathscr{V}}
\newcommand{\B}{\mathscr{B}}
\newcommand{\dualD}{\mathfrak{D}}
\newcommand{\Dg}{\mathbf{D}}
\newcommand{\DD}{\mathcal{D}^0}
\newcommand{\DDg}{\mathcal{D}}
\newcommand{\DV}{\mathcal{D}}
\newcommand{\W}{\mathscr{W}_N}
\newcommand{\Ao}{\mathbf{A}^\circ}
\newcommand{\AoK}{\mathbf{A}^\circ_{\K}}
\newcommand{\AK}{\mathbf{A}_{/\K}}
\newcommand{\OOO}{\mathscr{A}^\circ}
\newcommand{\K}{\mathcal{K}} 
\newcommand{\OK}{\mathcal{O}_{\K}}
\newcommand{\varprojlog}[1]{\underleftarrow{\log\!^{#1}}}
\newcommand{\T}{\mathscr{T}}
\newcommand{\TT}{\mathbf{T}}
\newcommand{\VV}{\mathbf{V}}
\newcommand{\HH}{\mathcal{H}}
\newcommand{\hh}{\mathcal{H}^+}
\newcommand{\HG}[2]{\mathcal{H}_{#1}(#2)}
\newcommand{\hhl}{\mathcal{H}^{+,[l]}}
\newcommand{\hhj}{\mathcal{H}^{+,[j]}}
\newcommand{\hhjj}{\mathcal{H}^{+,[l,l']}}
\newcommand{\GS}{G_{\mathbb{Q},S}}
\newcommand{\Rf}{R_{(k_0 ,i)}[r_0]}
\newcommand{\Rfr}{R_{(k_0 ,i)}[r]}
\newcommand{\parT}{\langle T\rangle}
\newcommand{\Zf}{Z_{(k_0 ,i)}[r_0]}
\newcommand{\Zfr}{\mathscr{Z}_{(k_0 ,i)}[r]}
\newcommand{\ZFf}{\mathscr{Z}_{(k_0 ,i)}[r_0]}
\newcommand{\ZFfr}{\mathscr{Z}_{(k_0 ,i)}[r]}
\newcommand{\ZF}{\mathscr{Z}}

% It is strongly recommended to use hyperref, especially for the review version.
% hyperref with option pagebackref eases the reviewers' job.
% Please disable hyperref *only* if you encounter grave issues, 
% e.g. with the file validation for the camera-ready version.
%
% If you comment hyperref and then uncomment it, you should delete *.aux before re-running LaTeX.
% (Or just hit 'q' on the first LaTeX run, let it finish, and you should be clear).
\definecolor{cvprblue}{rgb}{0.21,0.49,0.74}
\usepackage[pagebackref,breaklinks,colorlinks,citecolor=cvprblue]{hyperref}
\usepackage{times}
\usepackage{epsfig}
\usepackage{graphicx}
\usepackage{amsmath}
\usepackage[linesnumbered,ruled,vlined]{algorithm2e}
\usepackage{amssymb}
\usepackage{array}
\usepackage{booktabs}
\usepackage{amssymb}
\usepackage{pifont}
\usepackage{comment}
\usepackage{caption}
\usepackage[accsupp]{axessibility}
\usepackage[numbers]{natbib}


%% Define coment styles
\newcommand\commentfont[1]{\footnotesize\ttfamily\textcolor{blue}{#1}}
\SetCommentSty{commentfont}
\newcommand{\NEW}[1]{{\textcolor{orange}{NEW TEXT: #1}}}
\newcommand{\ANGIE}[1]{{\textcolor{teal}{Angie: #1}}}
\newcommand{\orl}[1]{{\textcolor{blue}{Or: #1}}}
\newcommand{\DAVID}[1]{{\textcolor{cyan}{David: #1}}}

% Additional styles
\newcolumntype{?}{!{\vrule width 1.2pt}}

% Extra symbols
\newcommand{\OURS}{UnScene3D}
\newcommand{\cmark}{\ding{51}}%
\newcommand{\xmark}{\ding{55}}%

%%%%%%%%% PAPER ID  - PLEASE UPDATE
\def\paperID{1852} % *** Enter the Paper ID here
\def\confName{CVPR}
\def\confYear{2024}

%%%%%%%%% TITLE - PLEASE UPDATE
\title{\OURS{}: Unsupervised 3D Instance Segmentation for Indoor Scenes}

%%%%%%%%% AUTHORS - PLEASE UPDATE
% \author{First Author\\
%Institution1\\
%Institution1 address\\
%{\tt\small firstauthor@i1.org}
% For a paper whose authors are all at the same institution,
% omit the following lines up until the closing ``}''.
% Additional authors and addresses can be added with ``\and'',
% just like the second author.
% To save space, use either the email address or home page, not both
%\and
%Second Author\\
%Institution2\\
%First line of institution2 address\\
%{\tt\small secondauthor@i2.org}
%}

\author{
David Rozenberszki$^{1}$~~~~~~
Or Litany$^{2,3}$~~~~~~
Angela Dai$^1$
\vspace{0.2cm} \\ 
$^1$Technical University of Munich~~~
$^2$Technion~~~
$^3$NVIDIA
\vspace{0.2cm} \\ 
\href{https://rozdavid.github.io/unscene3d}{https://rozdavid.github.io/unscene3d}
}

\begin{document}

\twocolumn[{%
\renewcommand\twocolumn[1][]{#1}%
\maketitle
\vspace{-1cm}
\begin{center}
    \centering
    \captionsetup{type=figure}
    \includegraphics[width=\textwidth,keepaspectratio]{figures/teaser.jpeg}
    \vspace{-0.7cm}
    \captionof{figure}{We propose \OURS{}, a fully-unsupervised 3D instance segmentation method, effectively separating semantic instances without requiring any manual annotations. 
    %Our method is based on geometric primitives ensuring crisps masks, and due to our self-training loop we can also obtain a dense set of predictions. 
    We utilize geometric primitives to ensure crisp masks, and due to our self-training loop, we can also obtain a dense set of predictions, even in cluttered indoor scenarios.
    }
\end{center}%
}]


\maketitle
\begin{abstract}
The current study investigated possible human-robot kinaesthetic interaction using a variational recurrent neural network model, called PV-RNN, which is based on the free energy principle.
Our prior robotic studies using PV-RNN showed that the nature of interactions between top-down expectation and bottom-up inference is strongly affected by a parameter, called the meta-prior, which regulates the complexity term in free energy.
% The current study examines how the behaviours of robots alter by changing the meta-prior $w$ in human-robot kinaesthetic interaction.
The current study examines how changing the meta-prior $w$ in the interaction phase affects the counter force generated when an experimenter attempts to induce movement pattern transitions familiar to the robot through its prior training.
The study also compares the counter force generated when trained transitions are induced by a human experimenter and when untrained transitions are induced.
Our experimental results indicated that (1) the human experimenter needs more/less force to induce trained transitions when $w$ is set with larger/smaller values, (2) the human experimenter needs more force to act on the robot when he attempts to induce untrained as opposed to trained movement pattern transitions.
Our analysis of time development of essential variables and values in PV-RNN during bodily interaction clarified the mechanism by which gaps in actional intentions between the human experimenter and the robot can be manifested as reaction forces between them.


%% Hiroki writing 2022-11-4
%Current study investigates the dynamics of the latent states during human-robot kinaesthetic interaction using PV-RNN.
%We have achieved to observe and analyse the internal state of an RNN model based on the free energy principle, during real-time human-robot interaction.
%Essential characteristics observed in the previous study of this variational recurrent neural network model, PV-RNN, is that by changing a meta prior $w$, the balance between the top-down intention and the bottom-up perceptual reality changes.
%In the current study, we examined how changing the weighting parameter $w$ between accuracy and complexity in free energy principle affects the humanoid robot's behaviour through human-robot interaction. We have conducted some human-robot kinaesthetic interaction experiments with various $w$ and quantitatively analysed the latent variable and the force applied to the humanoid robot. We have observed that the force required to change the robot's intention has increased, both when the top-down intention was strengthened by changing the $w$ and when corresponding switch of its primitive was against the experience of the RNN during its training. The study confirms through quantitative analysis that by increasing or decreasing the $w$ in PV-RNN, humanoid robot leads or follows the human counterpart during the human-robot kinaesthetic interaction.

\begin{comment}
Comment from Jun #2
・最後にQualitativeな結果(インパクト)が欲しい
・Current study investigates the problem on~と書き出すのが一般的
・最初の一文と最後の一文を対応させる
・最後の一文はもう少しAbstractかつ包括的に
\end{comment}

\begin{comment}
Comment from Jun #1
We investigated how the kinaesthetic human-robot interaction can affect the internal state of a model based on the free energy principle. 
=> how the internal state is affected is not the most important point in this study. This part should be rewritten.

The key function of this variational recurrent neural network model, PV-RNN, is that by changing a meta prior $w$, it takes a balance between the "complexity” term and the ”accuracy” term which corresponds to a top-down intention and a bottom-up perceptual reality in the free energy principle, respectively. 
=> This is not key function of PV-RNN. It is an essential characteristics observed in the previous study. The grammar after $w$ is something strange. Rewrite these.

This research has conducted a human-robot interaction experiment with a robotic agent in a kinaesthetic sense.
=> The sentence is not good. "in a kinaesthetic sense" is grammatically wrong.
MODIFIED => "In the current study human-robot interaction experiments using the kinaesthetic sense were conducted."

We investigated that when human forces the agent to switch primitives from one to another, larger force was required both when the human intention is conflictive against the top-down the intention of the agent and when the agent has a stronger top-down intention by modifying the $w$.
=> You should write the essential results of the experiments rather than what we investigated and also how these results could contribute to the studies on human-robot interaction.
\end{comment}

\end{abstract}
\section{Introduction}
\label{sec:intro}
\begin{figure}[t]
\begin{center}
    \includegraphics[width=1\linewidth]{figures/teaser.pdf}
\end{center}
\vspace{-0.1in}
\caption{\textbf{{\em Foggy} vs {\em Clear} NeRF.} Our \ournerf gets rid of reconstruction errors manifested as foggy ``floaters" in the density volume without additional input or significant computational overhead. 
%
Below are density profiles along a given ray before and after our geometry correction procedure, where we discard density peaks corresponding to floaters.
}
\label{fig:teaser}
\vspace{-0.2in}
\end{figure}



%The emergence of 
Neural Radiance Fields (NeRFs)~\cite{mildenhall2020nerf}  %and its variants 
have made revolutionary contributions in %photo-realistic 
novel view synthesis~\cite{barron2021mip,barron2022mip}, 
autonomous driving~\cite{rematas2022urban,tancik2022block}, digital human~\cite{hong2022headnerf,zhao2022humannerf}, and 3D content generation~\cite{eg3d,poole2022dreamfusion,lin2022magic3d}.
%by leveraging a multi-layer perceptron (MLP) to implicitly model the mapping from input 5D coordinates (i.e., 3D coordinates $\mathbf{x} = (x,y,z)$ and 2D viewing directions $\mathbf{d}=(\theta,\phi)$) to volume density $\sigma$ and view-dependent emitted radiance color $\mathbf{c} = (r,g,b)$. 
%
%They then use traditional volume rendering mechanisms on the obtained continuous 5D function (i.e., MLP) to generate novel views. 
To date, unfortunately, most NeRF-based methods encounter challenges when tackling large-scale cluttered scenes (e.g., Fig.~\ref{fig:teaser}):
\begin{enumerate}[leftmargin=0.16in, topsep=2pt,itemsep=-1ex,partopsep=1ex,parsep=1ex]
\item Input observations used for NeRF are often too sparse  compared to forward-facing or synthetic looking-inward scenes;
%\item Recovering fine-grained objects within a large volume is challenging for NeRF; %in capturing details accurately.
\item View-dependent visual effects give rise to ambiguity, resulting in a ``foggy" density field as shown in Fig.~\ref{fig:teaser}. 
%
Such artifacts are particularly pronounced in indoor scenes strewn with view-dependent appearances, such as specular highlights, glossy surface reflections from man-made objects. 
\end{enumerate}

Despite attempts to enhance NeRF's rendering quality given suboptimal input, such as using 3D conical frustums~\cite{barron2021mip,barron2022mip}, physically-grounded augmentations~\cite{chen2022aug}, and misalignment correction~\cite{jiang2022alignerf},  these challenges have yet to be fully resolved.
%
Depth supervision~\cite{deng2022depth, wei2021nerfingmvs} or proxy geometry~\cite{xu2021scalable,wu2022scalable} images can help alleviate the challenges in handling large-scale with sparse input, at the expense of %but they come at the cost of requiring 
expensive pre-processing or additional input.
%
Another line of work~\cite{wang2021neus, oechsle2021unisurf, wang2022neuris} achieves better reconstruction of surface geometry by using signed distances instead of volume density as scene representation. However, they sacrifice the ability to synthesize photo-realistic novel views.

%We observe that NeRF has been suffering from foggy ``floater" artifacts in large-scale cluttered scenes.
%
%Such artifacts are particularly pronounced in indoor scenes strewn with view-dependent appearances from man-made objects. 
%
To address the above issues, we propose an extension to NeRF, dubbed as {\bf \ournerf}, which enforces effective {\em appearance} and {\em geometry} constraints conducive to accurate colors and 3D densities estimation. We believe \ournerf can contribute beyond novel view synthesis, such as NeRF object detection~\cite{hu2022nerf}, NeRF object segmentation~\cite{zhi2021place, liu2022unsupervised, fan2022nerf,ren2022neural}, and NeRF registration~\cite{goli2022nerf2nerf}, where the rooms for improvement are substantial if more accurate color and density estimation are available.

Correspondingly, there are two steps in \ournerf. First, for appearance correction, the view-independent and view-dependent color components are predicted from the underlying 3D scene, which is combined to produce the final color estimation (Fig.~\ref{fig:toaster}).
%
The view-independent component (diffuse color and shading) captures the overall scene color, while the view-dependent component (highlights or reflections) captures color variations due to changes in viewing angle.
%
\ournerf then discards these view-dependent appearances in the training views to prevent them from interfering with the density estimation.
%
Second, a simple and effective geometry correction procedure will be performed to further eliminate the foggy ``floaters" or density errors. This geometry correction procedure is based on an assumption in line with traditional ray tracing in computer graphics.
\begin{comment}
% xh: basically copying method
On the other hand, ClearNeRF performs a geometric correction procedure performed on each traced ray during inference to refine the density estimation and better tackle the floater artifacts. 
%
The geometry correction procedure assumes that there should only be one salient peak along each traced ray during NeRF inference. 
Only the salient peak closest to the ray origin (the camera center) corresponds to  true geometry while the others will be manifested as foggy floaters hovering in the density volume. 
%
This assumption is in line with traditional ray tracing in computer graphics where in the absence of noise, only one intersection per ray should be returned to indicate the closest ray-object intersection.
%
\end{comment}
%%%%%%%%%%%
%As shown in Fig.~\ref{fig:teaser}, when reconstructing an indoor scene with sparse input and highly view-dependent objects, NeRF produces severe floating artifacts due to its attempt to explain view-dependent appearances.
%
Experiments verify that our proposed \ournerf can effectively get rid of floater artifacts without additional input.% or significant computational overhead. 


In summary, our contributions include the following:
\begin{itemize}[leftmargin=0.16in, topsep=2pt,itemsep=-1ex,partopsep=1ex,parsep=1ex]
    \item We propose a concise method for decomposing view-independent and view-dependent appearance during NeRF training and eliminate the interference of view-dependent appearance.
    \item We propose a geometric correction procedure performed on each traced ray during inference to refine the density estimation and better tackle the floater artifacts.
    \item Extensive experiments and ablations verify the effectiveness of our core designs and results in improvements over the vanilla NeRF and other state-of-the-art alternatives.
    %without additional computational resources or other inputs.
\end{itemize}




\section{Related work}

In recent years, large language models have improved significantly in various NLP areas, especially in generative tasks.
A lot of new concepts were introduced, starting from attention mechanism~\cite{bahdanau2014neural}, transformers~\cite{vaswani2017attention} to multitask, learning from instructions~\cite{wang2022super} and human feedback~\cite{wang2021putting}.
The last becomes extremely popular in the generative context including machine translation. 
% new architectures were proposed~\cite{radford2019language,brown2020language}, and, 
Consequently, the usage of machine translation tools has become a necessary compound for understanding a foreign language. 
Unfortunately, like other neural network-based algorithms, these tools are vulnerable to adversarial examples~\cite{DBLP:journals/corr/GoodfellowSS14}. 
Starting from text classification \cite{li-etal-2020-bert-attack,DBLP:conf/acl/EbrahimiRLD18,Li2018TextBuggerGA}, vulnerability and robustness received a lot of attention in the NLP community. 
For MT systems one of the pioneering works was~\cite{ebrahimi2018adversarial}, where authors proposed a character-level approach to generate adversarial examples.
% that neural MT systems are vulnerable to character-level perturbations, where only a few symbols in an input query are subject to change. 
Inheriting HotFlip~\cite{ebrahimi-etal-2018-hotflip} there were considered white-box and black-box settings, where only a few symbols in an input query are subject to change imitating typos.

While white-box optimization may yield stronger adversarial perturbations it implies access to the model's architecture and weights which is impractical in the case of online MT tools. 
In~\cite{wallace} there was considered a white-box universal approach to a targeted attack on conditional text generation. 
The authors modeled perturbation as an insertion of a trigger, a token sequence of small length, that results in a generated sequence similar to the target set of sentences. 
While during experiments certain triggers cause a model to produce sensitive racist output, they are generally meaningless and similarly to character-level attacks are easy to detect. 
Authors of~\cite{guo-etal-2021-gradient,9747475} reported high attack transferability making this approach promising for black-box setup, however,  the research is limited only to the GPT-2 model for generation task. 
The above papers use greedy techniques to walk through the searching space during the optimization, on the other hand, attacks on NLP models could be found via projection onto embeddings~\cite{wallace}, and for MT task this was discovered in~\cite{Seq2Sick,Sadrizadeh2023TargetedAA,sadrizadeh2023transfool}. 
In~\cite{zhang2021crafting}, it was shown that black-box optimization may yield transferable word-level attack that fools online translation tools, for example Baidu and Bing translators. 
This work proposed to use the word saliency as the measure of uncertainty. 
Masking candidates the saliency was estimated via additional BERT model~\cite{devlin2018bert}  which lead to strong readable and imperceptible adversaries, however, neither human evaluation was performed nor quantities results for online tools were given. In~\cite{wan2022paeg}, a gradient-based approach to generate phrase-level adversarial examples for neural MT systems was proposed. Similarly to~\cite{zhang2021crafting}, it is proposed to estimate the vulnerable word positions are estimated in an input phrase with the use of gradient information and replace corresponding words by the candidates computed with an auxiliary model.

% \mynote{actually we may underline that we do not generate adversarial examples per se (we arent aimed at misclassification), but rather generate inputs that are been translated though they should not}

% \mynote{TODO: Maybe add more criticism of zhang2021crafting and point out the differences in our approach.}

% \todopa{}{}{
% https://www.semanticscholar.org/paper/AdvAug\%3A-Robust-Adversarial-Augmentation-for-Neural-Cheng-Jiang/1e7d3a9846da556bc7b84ae1410d257b89448c30
% }

%\todopa{}{}{
%https://www.semanticscholar.org/paper/A-Targeted-Attack-on-Black-Box-Neural-Machine-with-Xu-Wang/2a46eb47e8742be29b16a5b83dc1a38616b24ce6
%}

%\todopa{}{}{https://www.semanticscholar.org/paper/PAEG\%3A-Phrase-level-Adversarial-Example-Generation-Wan-Yang/a6dd2a8debb5d5324c4f2be7fb7bb52ce109cbaf}

% \todopa{}{}{
% https://download.huan-zhang.com/events/srml2022/accepted/bhandari22lost.pdf
% }

%\todopa{}{}{http://fan-yao.com/paper/2021_SEED_nmtstroke.pdf}

% \todopa{}{}{https://arxiv.org/pdf/2303.01068v1.pdf}

%\todopa{kosinski2023theory}
%    {Theory of mind may have spontaneously emerged in large language models}
%    {https://arxiv.org/pdf/2302.02083.pdf}
%    {We can say that large language models are very clever now, etc...}

% \todopa{ebrahimi2018adversarial}
%     {On adversarial examples for character-level neural machine translation}
%     {https://arxiv.org/pdf/1806.09030.pdf}
%     {Very related work (see beamsearch in the text also)...}

% \todopa{zhang2021crafting}
%     {Crafting adversarial examples for neural machine translation}
%     {https://github.com/JHL-HUST/AdvNMT-WSLS}
%     {Very related work. See: ``Besides, WSLS exhibits strong transferability on attacking Baidu and Bing online translators.''}

% \todopa{sadrizadeh2023transfool}
%     {TransFool: An Adversarial Attack against Neural Machine Translation Models}
%     {https://arxiv.org/pdf/2302.00944.pdf}
%     {Very related work!}
%\section{Preliminaries}\label{sec:preliminaries}

%We leverage the principle of the normalized cut algorithm~\cite{shi2000normalized_cut} (NCut) to identify potential instances for 3D pseudo masks, lifted to the high-dimensional 3D scenario by using a geometric primitive basis, as discussed in Section~\ref{sec:oversegmentation}.
\NEW{We employ the principle of the NCut approach for pseudo-generation similarly to \cite{wang2023cut}, but lift it to support high-resolution 3D segmentation by operating on segment-level geometric primitives from self-supervised 2D and 3D features.}

NCut maximizes similarities within partitions and dissimilarities across partitions by minimizing the cost of a graph cut. This is formalized as: 
%
\begin{equation}
    NCut(A,B) = \frac{cut(A,B)}{assoc(A,V)} + \frac{cut(A,B)}{assoc(B,V)},
\end{equation}
where $A$ and $B$ are disjoint bipartitions from a full graph $V$, $cut$ measures the degree of dissimilarity computed as the total weight of edges that have been removed, and  $assoc$ represents the total connection within the partition.
Normalizing the cut cost function with the size of the partitions can solve the problem of single outlier node removal, which is one of the biggest difficulties of other graph cut algorithms~\cite{244673}.

While the minimum solution of this problem is intractable for practical applications, it can be rewritten as a generalized eigenvalue problem with adjacency matrix $W$ and degree matrix $D$, where $D(i,i) = \Sigma_jW(i,j)$:
%
\begin{equation} \label{eq:general_eigenval}
    (D-W)v = \lambda D v,
\end{equation}
%
Finding the second smallest eigenvalue $\lambda$  and its corresponding eigenvector $v$ is a close approximation for the minimized cost. 
From $v$, we obtain foreground separation by taking all node activations where the eigenvector components were larger than their mean. 
This method has been shown to be effective on intensity images, but combining it with deep features has demonstrated even stronger potential in the image domain \cite{wang2022tokencut,lis2022attentropy,wang2023cut}. 
While this multiple foreground objects could be directly predicted with a single pass by taking the eigenvectors in order, it was shown in \cite{wang2022tokencut} that a greedy iterative approach produces better results.




\section{Method}\label{sec:method}
\begin{figure*}
    \centering
    \includegraphics[width=\linewidth,keepaspectratio]{figures/pipelines/pipeline_figure_full.pdf}
    %\vspace{-0.7cm}
    \caption[]{
    \OURS{} first generates a set of pseudo masks (top) to initiate self-training (bottom) for unsupervised 3D instance segmentation.
    We leverage features from 3D self-supervised pre-training in combination with 2D self-supervised features on an input mesh.
    These multi-modal features are then aggregated on geometric segment primitives, integrating low- and high-level signals for pseudo mask segmentation.
    These initial pseudo masks are then used as supervision for a 3D transformer-based model to produce updated instance masks that are integrated into the supervision of multiple self-training cycles.
    Finally, we obtain clean and dense instance segmentation without using any manual annotations.
    } 
    \label{fig:full_pipeline}
\end{figure*}

\paragraph{Problem definition}
We propose an unsupervised learning-based method for 3D instance segmentation. Formally, we assume a set of training 3D scenes $\{X_i\}_{i=1}^{n_t}$, represented as meshes, where each scene $X_i$ contains an unknown set of $n_i$ objects. We aim to train a model that can predict for a previously unseen input scene $X$, a set of 3D masks representing the different object instances in that scene. 

\paragraph{Method overview}
We employ a two-stage scheme for unsupervised 3D instance segmentation.
First, we group scene points into contiguous regions based on geometric primitives, and aggregate self-supervised features in these regions to generate an initial set of pseudo masks using Normalized Cut.  This grouping technique enables efficient handling of high-dimensional 3D data.
%
We then follow a series of self-training cycles to refine the pseudo annotations.
An overview of our approach is shown in Figure~\ref{fig:full_pipeline}.

\subsection{Geometric oversegmentation}\label{sec:oversegmentation}
Normalized Cut~\cite{shi2000normalized_cut} (NCut) with deep features has been successfully applied in the 2D domain on dense graphs generated from image patches \cite{wang2022tokencut,lis2022attentropy,wang2023cut}. However, adopting this directly to 3D would be computationally infeasible due to the cubic growth with dimensionality.

We thus propose a solution in the form of geometric oversegmentation through graph coarsening. We begin by creating a graph where each node represents a mesh vertex. Then, we aggregate nodes with similar normal direction and color values and cluster them into contiguous mesh segments using the efficient method proposed by \cite{felzenszwalb2004efficient}. This process reduces the graph size by multiple orders of magnitude. 
%
Our geometry aware segments, as opposed to voxels or points, additionally provide regularization to the subsequent stage of feature aggregation for generating pseudo masks, which we will describe in the following section.

\subsection{Initial pseudo mask generation}\label{sec:dataset_gen}

We first predict an initial set of pseudo masks.  Additional masks will be added during the self-training phase described in Section~\ref{sec:self_train}. 
Thus, we favor generating a reliable set of masks at the cost of restricting to a sparse initial set (i.e., missing potential instances rather than generating noisy masks for them).

\paragraph{Feature aggregation}

We aim to employ a strong set of features for pseudo mask generation and thus consider complementary geometric and color signals from RGB-D scan data.
We leverage geometric 3D self-supervised features from a state-of-the-art 3D pre-training approach, Contrastive Scene Contexts (CSC)~\cite{hou2021exploring}.
We additionally consider 2D self-supervised features from DINO~\cite{caron2021emerging_dino}, extracted from the RGB images and projected to 3D using the corresponding camera poses.
Both the 3D and 2D features are aggregated within each of our geometry-aware segments. 

\paragraph{Masked foreground separation}
We apply NCut to our aggregated features to extract foreground regions $M$ as our initial pseudo masks. 
%
Starting with an empty set $M^0=\{\}$, we iteratively compute the adjacency matrix and retrieve the masks. 
%
That is, we start from $N$ geometric segments with their corresponding $D$-dimensional features $\mathcal{F} \in \mathcal{R}^{N\times D}$, and construct the similarity matrix $A = sim(\mathcal{F})$, where $sim$ denotes cosine similarity. 
Additionally, for the multi-modal setup we calculate similarity matrices $A_{2D}$ and $A_{3D}$ independently and take their weighted average to obtain the final scores. 
Empirically, we found this to be more robust than direct feature fusion of the different modalities, due to their different statistical characteristics.
%
\indent We obtain $W_j$ for Equation~\ref{eq:general_eigenval} by thresholding $A$ at $\tau_{cut}$, where $j$ denotes the $j^{th}$ NCut iteration. 
Using $W_j$, we solve for the second eigenvector $v_j$ and threshold it to retrieve the partition $m_j$. 
We keep all separated foregrounds in $M^0$, where for each upcoming iteration, we mask out the row and column vectors from $W_i$, where $m_i \in M^0$ was already accepted as a foreground instance and $i$ being the segment ids. 
This allows greedy separation of instances in order of confidence  in every cut iteration.
Examples of our generated pseudo masks are visualized in Figures \ref{fig:freemask_vs_ncut} and \ref{fig:self_training_refinement}. \\
%
\indent As the adjacency graph is unaware of the mesh connectivity, NCut often results in masks that span  spatially separated scene regions. 
In 3D, we can leverage knowledge of physical distance to constrain masks to be contiguous in the coarsened scene connectivity graph. We thus filter masks that have separated components, keeping only the ones that contain the item with the maximum absolute value in  $v_j$. Separation is performed before saving $m_j$ into $M^0$, thus allowing for repeated separation of every component. 
We iterate until the maximum number of instances $M^0 = \{m_i\}_{i=1}^{N_m}$ are obtained, or there are no segments left in the scene. 

\subsection{Self-Training}\label{sec:self_train}

Our initial pseudo masks can provide a set of proposed instances $M^0$; however, these pseudo masks are quite sparse in the scenes and sometimes over- or under-split nearby instances.
We thus refine the pseudo mask data through an iterative self-training strategy, producing final instance segmentation predictions $M'$ with more dense and complete instance proposals.

We leverage a state-of-the-art 3D transformer-based backbone~\cite{Schult23mask3d} for our self-training from pseudo mask data as supervision. 
Through multiple training cycles we save the proposals of the $t^{th}$ iteration into $M^{t}$, from the self-trained model, and save these masks as an extension to the original pseudo dataset obtaining $M^t \supseteq M^0$. 
From the second training iteration, we can extract the most confident $K$ predictions and sample these new instance proposals as an addition to the pseudo annotations. 
Further, we only accept new instances if the added information value is larger than a minimum threshold, which we measure by simple segment IoU scores. This way, we can effectively densify the originally sparse annotations, but without limiting the quality of the originally clean pseudo masks. 
%
\paragraph{Loss} \label{par:losses}
We adapt DropLoss \cite{wang2023cut} for our self-training cycles, which is robust to sparse data and missing annotations. 
In particular, we use a weighted combination of cross-entropy and Dice \cite{sudre2017generalised_diceloss} losses for bipartite-matching with pseudo annotations.
We then drop losses for backpropagation which do not have at least $\tau_{drop}$ overlap with the annotations from the previous cycle.

\subsection{Implementation Details}\label{sec:implementation}
%
\paragraph{Backbones.} 
We use a Res16UNet34C sparse-voxel UNet implemented in the MinkowskiEngine~\cite{choy20194d} for 3D pre-trained feature extraction as well as for the 3D transformer during self-training. 

\paragraph{Self-training.} We employ the 3D transformer architecture of \cite{Schult23mask3d}, initialized from scratch. 
The first self-training cycle is trained for 600 epochs with a batch size of 8 until convergence, which takes $\approx 3$ days on a single NVIDIA RTX A6000 GPU. 
Further self-training cycles are all initialized from the previous state and finetuned for an additional 50 epochs in $\approx 4$ hours and for a total of 4 training cycles to produce the final set of instance predictions $S$. 
For the Hungarian assignment, we take the original weighted combination of dice and binary cross-entropy losses and only apply the DropLoss condition in the backpropagation phase.

% \begin{table}[t]
%     \tablestyle{2pt}{1.05}
    
%     \centering
%     %\resizebox{1\columnwidth}{!}{
%     \begin{tabular}{@{}l|ccccccc}
%     	\toprule
%             \multicolumn{8}{c}{Untrimmed Spatial-Temporal Grounding}
%     	\toprule
%     	\multicolumn{1}{c}{} & \multicolumn{7}{c}{GroundingYouTube}  \\ 
%     	\cmidrule(lr){2-8} 
%     	\multirow{2}{*}{\textbf{Method}}    & \multirow{2}{*}{IoU+Point} &\multicolumn{6}{c}{mAP}  \\ 
%     	                                    &  & 0.1 & 0.2 & 0.3 & 0.4 & 0.5  & 0.1:0.5 \\ 
%     	\midrule
%     	MIL-NCE \citep{miech2020end} & 4.67 & 33.94 & 25.16 & 12.65 & 3.42 & 0.41  & 15.11 \\
%          CoMMA* \citep{tan2021look}   & 1.02 & 2.18  & 1.72 & 1.11 & 0.93 & 0.37 & 1.26\\
%              %Ours S3D                         & 7.78 & 39.43 & 31.47 & 19.38 & 9.14 & 3.79  & 20.64  \\
%              Ours S3D                      & 9.12 & 42.70  & 35.49 & 25.16 & 16.22 & 10.05  & 25.92 \\
%              \midrule
%             CLIP \citep{radford2021learning}  & 3.59 & 29.54  & 22.15 & 9.16 & 2.48 & 0.39 & 12.74 \\
%             CoMMA$\dagger$              & 1.68 & 3.51 & 2.32 & 1.88 & 0.99 & 0.40 & 1.82 \\
%     	   Ours                        & 10.09 & 42.81  & 36.05 & 25.84 & 17.10 & 11.35  & 26.63 \\
%             \midrule
%             GLIP \citep{li2022grounded}      &  1.24 & 2.83 & 2.10 & 1.52 & 0.96 & 0.37 & 1.56 \\
%     	\bottomrule
%     \end{tabular}
%     %\vspace{+0.3cm}
%     \caption{\textbf{Spatio-temporal localization on full videos}. Since our model learned global representations encoding temporal information and spatial correspondences across modalities, it achieves the best performance in spatio-temporal evaluation.
%     % \caption{\textbf{Spatial-temporal localization on full videos}. Our model learned both global representation which encodes temporal information. It also learned spatial correspondence across modalities, which ends up with the best performance in spatial temporal evaluation.
%     \label{tab:st_long}
%     %\vspace{-0.7cm}
%     }
%     %}
% \end{table}
\begin{table*}[t]
    \tablestyle{4pt}{1.05}
    \tiny
    \centering
    \resizebox{2\columnwidth}{!}{
    \begin{tabular}{@{}l|ccccccccccc}
    	\toprule
    	\multicolumn{5}{c}{} &\multicolumn{7}{c}{GroundingYoutube}  \\ 
    	\cmidrule(lr){6-12} 
    	\multirow{2}{*}{\textbf{Method}}  & \multirow{2}{*}{\textbf{Backbone}} & \multirow{2}{*}{\textbf{DataSet}} & \multirow{2}{*}{\textbf{Supervision}} & \multirow{2}{*}{\textbf{Modality}}  & \multirow{2}{*}{IoU+Point} &\multicolumn{6}{c}{mAP}  \\ 
    	  & & & & & & 0.1 & 0.2 & 0.3 & 0.4 & 0.5  & 0.1:0.5 \\ 
    	\midrule
    	
         CoMMA$\dagger$ \citep{tan2021look}  & S3D &HT250K & Self &VT& 1.02 & 2.18  & 1.72 & 1.11 & 0.93 & 0.37 & 1.26\\
         MIL-NCE \citep{miech2020end} & S3D* &HT100M & Self &VT& 4.67 & 33.94 & 25.16 & 12.65 & 3.42 & 0.41  & 15.11 \\
             %Ours S3D                         & 7.78 & 39.43 & 31.47 & 19.38 & 9.14 & 3.79  & 20.64  \\
             %\midrule
             Ours                   & S3D &HT100M & Self &VT  & \textbf{9.12} & \textbf{42.70}  & \textbf{35.49} & \textbf{25.16} & \textbf{16.22} & \textbf{10.05}  & \textbf{25.92} \\
             \midrule
            GLIP \citep{li2022grounded}   & Swin-L*  & Cap24M & Weak & IT &  1.24 & 2.83 & 2.10 & 1.52 & 0.96 & 0.37 & 1.56 \\
            CoMMA$\ddagger$   \citep{tan2021look} & CLIP &HT100M& Self & VT & 1.68 & 3.51 & 2.32 & 1.88 & 0.99 & 0.40 & 1.82 \\
            CLIP \citep{radford2021learning}& CLIP &HT100M & Self & IT& 3.59 & 29.54  & 22.15 & 9.16 & 2.48 & 0.39 & 12.74 \\
            RegionCLIP \citep{zhong2022regionclip}   & ResNet-101*  & CC3M & Weak & IT &  5.65 & 35.65 & 27.43 & 15.69 & 4.31 & 0.86 &  16.78 \\
            %\midrule
    	   Ours       & CLIP &HT100M & Self &VT  &10.09 & 42.81  & 36.05 & 25.84 & 17.10 & 11.35  & 26.63 \\
               Ours                    & CLIP* &HT100M & Self &VT  & \textbf{11.53} & \textbf{43.64}  & \textbf{36.94} & \textbf{26.78} & \textbf{19.45} & \textbf{14.61}  & \textbf{28.26} \\
               \midrule
               MIL-NCE(temp.)+RegionCLIP(spa.)   &  -  & - & - & VT  & 9.21  & 40.54  & 34.97  & 22.38  &  13.79 & 9.18  &  22.33  \\
    	\bottomrule
    \end{tabular}}
    %\vspace{+0.3cm}
    \caption{\textbf{Spatio-temporal grounding on GroundingYouTube full videos}.   
The proposed model learns global representations encoding global information and spatial correspondences across modalities, achieving a better performance in spatio-temporal evaluation compared to models trained on only spatial or temporal grounding. 
(V: video, I: image, T: text.) $^*$ indicates finetuned backbone.
    % \caption{\textbf{Spatial-temporal localization on full videos}. Our model learned both global representation which encodes temporal information. It also learned spatial correspondence across modalities, which ends up with the best performance in spatial temporal evaluation.
    \label{tab:st_long}
    \vspace{-0.3cm}
    }
    %}
\end{table*}

\section{Experiments}

\subsection{Datasets} \label{dataset}
\noindent \textbf{Training Data:} \textbf{HowTo100M dataset} contains 1.2M instructional videos along with their corresponding automatically generated speech (ASR).
The narrations may be inaccurate and do not always accurately depict the video scene.
%We randomly selected 200K  video clips from the \textit{Food and Entertaining} category for training. %, and thus, we mainly focus on instructional videos in the area of cooking and kitchen tasks. 
%
%\begin{table*}%[htpb]
    \centering
    \small
    \setlength{\tabcolsep}{4pt}
    \resizebox{\textwidth}{!}{
    \begin{tabu}{lr|ccccccccc|ccccccccc}
        \toprule
        \multirow{2}{*}{\bf Method} & \multirow{2}{*}{\bf \#Pairs} & \multicolumn{9}{c|}{\bf FT Retrieval \ \ R@1 / R@5 / R@10} & \multicolumn{9}{c}{\bf ZS Retrieval \ \ R@1 / R@5 / R@10} \\
        & & \multicolumn{3}{c}{MSRVTT} & \multicolumn{3}{c}{DiDeMo} & \multicolumn{3}{c|}{ActivityNet} & \multicolumn{3}{c}{MSRVTT} & \multicolumn{3}{c}{DiDeMo} & \multicolumn{3}{c}{ActivityNet}  \\
        \midrule
        ClipBERT~\cite{lei2021less}  & 5.4M & 22.0 & 46.8 & 59.9 & 20.4 & 48.0 & 60.8 & 21.3 & 49.0 & 63.5 & -& -& -& -& -& -& -& -& -\\
        VideoCLIP~\cite{xu2021videoclip}  & 136M & 30.9 & 55.4 & 66.8 & -& -& -& -& -& -& 10.4 & 22.2 & 30.0 & 16.6 & 46.9 & -& -& -& -\\
        Frozen~\cite{bain2021frozen}  & 5M & 31.0 & 59.5 & 70.5 & 34.6 & 65.0 & 74.7  & -& -& -& 18.7 & 39.5 & 51.6 & 20.2 & 46.4 & 58.5 & -& -& -\\
        ALPRO~\cite{li2022align}  & 5M & 33.9 & 60.7 & 73.2 & 35.9 & 67.5 & 78.8 & -& -& -& 24.1 & 44.7 & 55.4 & 23.8 & 47.3 & 57.9 & -& -& -\\
        VIOLET~\cite{fu2021violet}  & 138M & 34.5 & 63.0 & 73.4 & 32.6 & 62.8 & 74.7 & -& -& -& 25.9 & 49.5 & 59.7 & 23.5 & 49.8 & 59.8 & -& -& - \\
        All-in-one~\cite{wang2022all} & 138M & 37.9 & 68.1 & 77.1 & 32.7 & 61.4 & 73.5 & 22.4 & 53.7 & 67.7 & -& -& -& -& -& -& -& -& -\\
        LAVENDER~\cite{li2022lavender} & 30M & 40.7 & 66.9 & 77.6 & 53.4 & 78.6 & 85.3 &  - & -& -& -& -& -& -& -& -& -& -& -\\
        Singularity~\cite{lei2022revealing} & 17M & 42.7 & 69.5 & 78.1 & 53.1 & 79.9 & 88.1 & 48.9 & 77.0 & 86.3 & 34.0 & 56.7 & 66.7 & 37.1 & 61.7 & 69.9 & 30.6 & 55.6 & 66.9 \\
        OmniVL~\cite{wang2022omnivl} & 17M & 47.8 & 74.2 & 83.8 & 52.4 & 79.5 & 85.4 & -& -& -& 34.6 & 58.4 & 66.6 & 33.3 & 58.7 & 68.5 & -& -& -\\ 
        VINDLU~\cite{Cheng2022VindLUAR} & 25M & 46.5 & 71.5 & 80.4 & 61.2 & 85.8 & 91.0 & 55.0 & 81.4 & 89.7 & 32.0 & 54.6 & 62.0 & 36.9 & 61.7 & 70.5 & 30.9 & 57.0 & 68.2 \\
        \rowfont{\color{Gray}}
        CLIP4Clip~\cite{luo2022clip4clip} & 400M & 44.5 & 71.4 & 81.6 & 42.8 & 68.5 & 79.2 & 40.5 & 72.4 & 83.4 & 31.2 & 53.7 & 64.2 & -& -& -& -& -& -\\
        % \rowfont{\color{Gray}}
        % CLIP-Hhiker~\cite{bain2022clip} & 400M & 47.7 & 74.1 & 82.9 & -& -& -& 44.0 & 74.9 & 86.1 & -& -& -& -& -& -& -& -& -\\
        \rowfont{\color{Gray}}
        CLIP-ViP~\cite{xue2022clip} & 500M & 54.2 & 77.2 & 84.8 & 50.5 & 78.4 & 87.1 & 53.4 & 81.4 & 90.0 & -& -& -& -& -& -& -& -& -\\
        \rowfont{\color{Gray}}
        InternVideo~\cite{Wang2022InternVideoGV} & 646M & 55.2 & 79.6 & 87.5 & 57.9 & 82.4 & 88.9 & 62.2  & 85.9 & 93.2 & 40.7 & 65.3 & 74.1 & 31.5 & 57.6 & 68.2 & 30.7 & 57.4 & 70.2 \\
        \midrule
        \multirow{3}{*}{\Modelname-Base} & 5M & 46.3 & 72.7 & 82.0 & 54.8 & 83.0 & 89.0 & 52.1 & 80.5 & 89.6 & 29.6 & 52.8 & 61.9 & 33.4 & 58.3 & 67.0 & 28.3 & 53.0 & 64.2 \\
        & 17M & 50.6 & 75.4 & 83.5 & 60.8 & 85.1 & 91.0 & 56.1 & 82.5 & 91.2 & 35.5 & 59.3 & 68.6 & 41.9 & 66.7 & 75.0 & 33.8 & 59.1 & 70.4 \\
        & 25M & 51.0 & 76.5 & 84.2 & 61.6 & 86.8 & 91.5 & 58.3 & 83.9 & 91.5 & 35.2 & 57.8 & 66.0 & 41.2 & 65.4 & 74.9 & 35.5 & 60.6 & 71.8 \\
        \hline
        \multirow{3}{*}{\Modelname-Large} & 5M & 53.3 & 76.6 & 83.9 & 59.7 & 84.9 & 90.8 & 58.1 & 85.5 & 92.9 & 33.3 & 58.1 & 66.7 & 34.0 & 60.4 & 68.7 & 31.9 & 60.2 & 72.0 \\
        & 17M & \underline{56.5} & \underline{80.1} & \underline{87.4} & \underline{66.6} & \underline{89.9} & \textbf{93.7} & \underline{66.6} & \underline{88.6} & \underline{94.7} & \textbf{42.6} & \textbf{64.4} & \textbf{73.1} & \underline{46.4} & \underline{70.0} & \underline{78.8} & \textbf{42.8} & \textbf{69.6} & \textbf{79.8} \\
        & 25M & \textbf{58.8} & \textbf{81.0} & \textbf{87.1} & \textbf{70.4} & \textbf{90.1} & \underline{93.5} & \textbf{66.8} & \textbf{89.1} & \textbf{94.9} & \underline{40.7} & \underline{63.4} & \underline{71.8} & \textbf{48.6} & \textbf{72.9} & \textbf{80.0} & \underline{41.9} & \underline{68.9} & \underline{80.3} \\
        \bottomrule
    \end{tabu}
    }
    \vspace{-0.3cm}
    \caption{Comparison to the state-of-the-art text-to-video retrieval methods on MSRVTT, DiDeMo and AcitivityNet.
    \#Pairs denotes the number of pre-training pairs.
    ``FT'' and ``ZS'' refer to the fine-tuning and zero-shot results.
    }
    \label{tab:retrieval}
\end{table*}
%\input{tables/spatio_vhico}

%\vspace{-0.3cm}
\noindent\textbf{Downstream Datasets:} %\textbf{YouCook2}: For the text-to-video retrieval downstream task, we use the common YouCook2 dataset , which provides a human-generated caption for 3.5K video clips for cooking instruction. 
%blah blah ... \hkc{add some details here?}
\textbf{GroundingYoutube (GYT)} is used to evaluate the task of multi-action spatio-temporal grounding as described in Section \ref{sec:dataset:annotation}.
% , we annotated the dense spatio-temporal location information as described in Section \ref{sec:dataset:annotation}.
%for 512 verb-noun phrases. All occurrences of the specific phrase in the test video are hence annotated, allowing us to evaluate spatio-temporal grounding in full untrimmed videos.
\noindent\textbf{MiningYoutube (MYT)} \citep{kuehne2019mining} %: To evaluate the temporal grounding abilities, we leverage the MiningYoutube \citep{kuehne2019mining} dataset, as it 
provides temporal annotation and is limited to the domain of cooking instruction videos. %The dataset features 250 full instructional videos, which are annotated with 512 action classes and temporal boundary information. 
%We use it to evaluate the temporal grounding abilities.
%Here, temporal alignment, the task of finding the right temporal boundaries given the sequences of actions, is used during evaluation to relax the task of temporal detection. 
\noindent\textbf{YouCook-Interaction (YC-Inter)} \citep{tan2021look} is an extension of the YouCook2 dataset \citep{zhou2018towards} for cooking instruction providing bounding boxes for 6K selected frames. The bounding boxes usually comprise the hand and the tool mentioned in the respective sentence-wise annotation. %We evaluate the spatial grounding abilities of models on this dataset.
% \noindent\textbf{YouCook2-Interaction (YC-Inter)}  To evaluate the spatial grounding abilities of our system, we use the YouCook2-Interaction dataset \citep{tan2021look}, an extension of a subset of the YouCook2 dataset \citep{zhou2018towards} for cooking instruction, which provides bounding boxes for 6K selected frames. The bounding boxes usually comprise the hand and the tool mentioned in the respective sentence-wise annotation.    
To further benchmark on general video domains on the \textbf{V-HICO} dataset~\citep{li2021weakly} with 6.5k videos with human-object interaction bounding boxes annotations, 
% that have been semi-automatically curated from sentence captions, 
and \textbf{Daly} action dataset~\citep{weinzaepfel2016human}, featuring videos consisting of daily actions such as ``brushing teeth''.% and ``cleaning windows''.



\subsection{Baseline methods}

%The proposed system is compared to various multimodal methods based on self- and weak supervision: 
\textbf{Temporal}: MIL-NCE~\citep{miech2020end} utilizes S3D~\citep{xie2018rethinking} and word2vec~\citep{mikolov2013efficient}. CLIP~\citep{radford2021learning}, an image-text model with transformer. 
\textbf{Spatial}:
CoMMA~\citep{tan2021look}, SSL model ($\dagger$ for weights shared by the author\footnote{We thank the authors for providing code and weights.} $\ddagger$ trained with CLIP);  
GLIP~\citep{li2022grounded}, RegionCLIP~\citep{zhong2022regionclip}, SOTA weakly supervised grounding model. % trained with image-text pairs.
\textbf{Spatio-temporal}: We construct MIL-NCE+RegionCLIP following the inference pipeline in Figure \ref{fig:inference}. 
TubeDETR~\citep{yang2022tubedetr} and STCAT \citep{jin2022embracing} are supervised. 
More descriptions of the baselines are given in the Appendix \ref{sup:baseline}.
Details of the implementation and experimental settings can be found in the appendix \ref{backbone_and_training}. Inference setups for each baseline are described in Section \ref{inference_sup}.

%% \begin{table}[t]
%     \tablestyle{2pt}{1.05}
    
%     \centering
%     %\resizebox{1\columnwidth}{!}{
%     \begin{tabular}{@{}l|ccccccc}
%     	\toprule
%             \multicolumn{8}{c}{Untrimmed Spatial-Temporal Grounding}
%     	\toprule
%     	\multicolumn{1}{c}{} & \multicolumn{7}{c}{GroundingYouTube}  \\ 
%     	\cmidrule(lr){2-8} 
%     	\multirow{2}{*}{\textbf{Method}}    & \multirow{2}{*}{IoU+Point} &\multicolumn{6}{c}{mAP}  \\ 
%     	                                    &  & 0.1 & 0.2 & 0.3 & 0.4 & 0.5  & 0.1:0.5 \\ 
%     	\midrule
%     	MIL-NCE \citep{miech2020end} & 4.67 & 33.94 & 25.16 & 12.65 & 3.42 & 0.41  & 15.11 \\
%          CoMMA* \citep{tan2021look}   & 1.02 & 2.18  & 1.72 & 1.11 & 0.93 & 0.37 & 1.26\\
%              %Ours S3D                         & 7.78 & 39.43 & 31.47 & 19.38 & 9.14 & 3.79  & 20.64  \\
%              Ours S3D                      & 9.12 & 42.70  & 35.49 & 25.16 & 16.22 & 10.05  & 25.92 \\
%              \midrule
%             CLIP \citep{radford2021learning}  & 3.59 & 29.54  & 22.15 & 9.16 & 2.48 & 0.39 & 12.74 \\
%             CoMMA$\dagger$              & 1.68 & 3.51 & 2.32 & 1.88 & 0.99 & 0.40 & 1.82 \\
%     	   Ours                        & 10.09 & 42.81  & 36.05 & 25.84 & 17.10 & 11.35  & 26.63 \\
%             \midrule
%             GLIP \citep{li2022grounded}      &  1.24 & 2.83 & 2.10 & 1.52 & 0.96 & 0.37 & 1.56 \\
%     	\bottomrule
%     \end{tabular}
%     %\vspace{+0.3cm}
%     \caption{\textbf{Spatio-temporal localization on full videos}. Since our model learned global representations encoding temporal information and spatial correspondences across modalities, it achieves the best performance in spatio-temporal evaluation.
%     % \caption{\textbf{Spatial-temporal localization on full videos}. Our model learned both global representation which encodes temporal information. It also learned spatial correspondence across modalities, which ends up with the best performance in spatial temporal evaluation.
%     \label{tab:st_long}
%     %\vspace{-0.7cm}
%     }
%     %}
% \end{table}
\begin{table*}[t]
    \tablestyle{4pt}{1.05}
    \tiny
    \centering
    \resizebox{2\columnwidth}{!}{
    \begin{tabular}{@{}l|ccccccccccc}
    	\toprule
    	\multicolumn{5}{c}{} &\multicolumn{7}{c}{GroundingYoutube}  \\ 
    	\cmidrule(lr){6-12} 
    	\multirow{2}{*}{\textbf{Method}}  & \multirow{2}{*}{\textbf{Backbone}} & \multirow{2}{*}{\textbf{DataSet}} & \multirow{2}{*}{\textbf{Supervision}} & \multirow{2}{*}{\textbf{Modality}}  & \multirow{2}{*}{IoU+Point} &\multicolumn{6}{c}{mAP}  \\ 
    	  & & & & & & 0.1 & 0.2 & 0.3 & 0.4 & 0.5  & 0.1:0.5 \\ 
    	\midrule
    	
         CoMMA$\dagger$ \citep{tan2021look}  & S3D &HT250K & Self &VT& 1.02 & 2.18  & 1.72 & 1.11 & 0.93 & 0.37 & 1.26\\
         MIL-NCE \citep{miech2020end} & S3D* &HT100M & Self &VT& 4.67 & 33.94 & 25.16 & 12.65 & 3.42 & 0.41  & 15.11 \\
             %Ours S3D                         & 7.78 & 39.43 & 31.47 & 19.38 & 9.14 & 3.79  & 20.64  \\
             %\midrule
             Ours                   & S3D &HT100M & Self &VT  & \textbf{9.12} & \textbf{42.70}  & \textbf{35.49} & \textbf{25.16} & \textbf{16.22} & \textbf{10.05}  & \textbf{25.92} \\
             \midrule
            GLIP \citep{li2022grounded}   & Swin-L*  & Cap24M & Weak & IT &  1.24 & 2.83 & 2.10 & 1.52 & 0.96 & 0.37 & 1.56 \\
            CoMMA$\ddagger$   \citep{tan2021look} & CLIP &HT100M& Self & VT & 1.68 & 3.51 & 2.32 & 1.88 & 0.99 & 0.40 & 1.82 \\
            CLIP \citep{radford2021learning}& CLIP &HT100M & Self & IT& 3.59 & 29.54  & 22.15 & 9.16 & 2.48 & 0.39 & 12.74 \\
            RegionCLIP \citep{zhong2022regionclip}   & ResNet-101*  & CC3M & Weak & IT &  5.65 & 35.65 & 27.43 & 15.69 & 4.31 & 0.86 &  16.78 \\
            %\midrule
    	   Ours       & CLIP &HT100M & Self &VT  &10.09 & 42.81  & 36.05 & 25.84 & 17.10 & 11.35  & 26.63 \\
               Ours                    & CLIP* &HT100M & Self &VT  & \textbf{11.53} & \textbf{43.64}  & \textbf{36.94} & \textbf{26.78} & \textbf{19.45} & \textbf{14.61}  & \textbf{28.26} \\
               \midrule
               MIL-NCE(temp.)+RegionCLIP(spa.)   &  -  & - & - & VT  & 9.21  & 40.54  & 34.97  & 22.38  &  13.79 & 9.18  &  22.33  \\
    	\bottomrule
    \end{tabular}}
    %\vspace{+0.3cm}
    \caption{\textbf{Spatio-temporal grounding on GroundingYouTube full videos}.   
The proposed model learns global representations encoding global information and spatial correspondences across modalities, achieving a better performance in spatio-temporal evaluation compared to models trained on only spatial or temporal grounding. 
(V: video, I: image, T: text.) $^*$ indicates finetuned backbone.
    % \caption{\textbf{Spatial-temporal localization on full videos}. Our model learned both global representation which encodes temporal information. It also learned spatial correspondence across modalities, which ends up with the best performance in spatial temporal evaluation.
    \label{tab:st_long}
    \vspace{-0.3cm}
    }
    %}
\end{table*}

\begin{table*}[h]
    \tablestyle{7pt}{1.05}
    \tiny
    \centering
    \resizebox{2\columnwidth}{!}{
    \begin{tabular}{@{}l| cccc | c |c c| c c | c c }
    	\toprule
    	\multicolumn{4}{c}{} & \multicolumn{1}{c}{ } & \multicolumn{1}{c}{YC-Inter} & \multicolumn{2}{c}{GroundingYT}  & \multicolumn{2}{c}{V-HICO}   & \multicolumn{2}{c}{Daly}\\ 
    	\cmidrule(lr){6-6} \cmidrule(lr){7-8}  \cmidrule(lr){9-10} \cmidrule(lr){11-12}  
    	Method  & Backbone &Data&Super.&Mod.& Acc &  Acc & mAP &  Acc & mAP  &  Acc & mAP \\ 
    	\midrule
        MIL-NCE \citep{miech2020end} & S3D* &HT100M & Self &VT& 23.67  & 27.45  & 8.21 & 12.65 & 11.23 & 13.84 & 24.23 \\
    	CoMMA$\dagger$ \citep{tan2021look} & S3D &HT250K & Self &VT& 48.63   & 47.68 & 23.38 & 40.97 & 21.45 & 54.48 & 33.39 \\
        %\midrule
        Ours                       & S3D &HT100M & Self &VT & \textbf{53.98}   & \textbf{60.62} & \textbf{44.93} & \textbf{44.32} & \textbf{24.31} & \textbf{66.35} & \textbf{45.93} \\
         \midrule
         CLIP   \citep{radford2021learning}            & CLIP&HT100M & Self &IT &    14.10    & 12.50  & 3.49 &  29.23 & 12.51  & 18.02 & 27.28  \\
         CoMMA$\ddagger$  \citep{tan2021look}            & CLIP  &HT100M & Self &VT&   52.65     & 47.56 & 36.42 & 55.20 &  34.54& 61.06 & 44.37  \\
             RegionCLIP   \citep{zhong2022regionclip}            & RN50x4* & CC3M & Weak &IT &   51.56     &   52.84 &  23.42 & 57.92 & 37.82 & 67.12 & 48.62 \\
            GLIP   \citep{li2022grounded}            & Swin-L*&Cap24M & Weak &IT &   52.84      &   53.62 & 24.73 & \textbf{66.05} & 41.17 & - & - \\
            %\midrule
            Ours         & CLIP &HT100M & Self &VT& 57.10    &   55.49 & 43.12 & 60.71& 39.28 & 70.08 & 50.56 \\
            Ours                       & CLIP* &HT100M & Self &VT& \textbf{58.35}    &   \textbf{56.98} & \textbf{45.32} & 62.34& \textbf{41.56} & \textbf{71.35} & \textbf{52.78} \\
            %V-HICO   \citep{}            &  Faster R-CNN &  -      &  - & - & & 67.21 & - & - \\
            \midrule
            {\color{gray}TubeDETR \citep{yang2022tubedetr}}    &  {\color{gray}MDETR} & {\color{gray}Vid-STG} & {\color{gray} Full} & {\color{gray}VT} & {\color{gray}51.63}    &   {\color{gray}53.24} & {\color{gray} 41.76} & {\color{gray}63.23} & {\color{gray}40.87 } & {\color{gray}84.21} & {\color{gray} 62.98} \\
            {\color{gray}STCAT \citep{jin2022embracing}}    &  {\color{gray}ResNet-101} & {\color{gray}Vid-STG} & {\color{gray} Full} & {\color{gray}VT} & {\color{gray}54.47}    &   {\color{gray} 55.90} & {\color{gray}44.21 } & {\color{gray}65.34} & {\color{gray} 41.10 } & {\color{gray}85.42} & {\color{gray} 63.94} \\
    	\bottomrule
    \end{tabular}
    }
    \vspace{-0.2cm}
    \caption{\textbf{Video spatial grounding}. We evaluate the accuracy of the pointing game and the mean average precision. 
    We listed CNN-based methods on top and transformer-based methods in the middle. 
    Models learning global representations (MIL-NCE, CLIP) don't perform well on localization tasks, while our model outperforms other grounding methods. $^*$ indicates finetuned backbone.
    %Models learning global representations (MIL-NCE, CLIP) don't perform well on localization tasks, while our model outperforms other grounding methods. %We listed CNN-based methods on top and transfomer-based methods at the bottom. 
    %(Mod. indicates the modality used, where V: video, I: image, T: text. Super. indicates supervision.)
    %Our method generalized well on both video and image architectures. 
    % Daly GLIP is not workable since every class is action. OOV. V-HICO dataset the CLIP  generalized better to OOV, while word2vec getting worse performance. \bc{maybe we can add supervision: weakly, SSL} \bc{add pretraining data}
    \label{tab:spatial}
    \vspace{-0.5cm}
    }
   
    
\end{table*}

% \begin{table}[t]
%     % \tablestyle{2pt}{1.05}
    
%     \centering
%     %\resizebox{1\columnwidth}{!}{
%     \begin{tabular}{@{}l|cc|cc}
%     	\toprule
%     	\multicolumn{1}{c}{} & \multicolumn{2}{c}{YouCook-Interaction} & \multicolumn{2}{c}{MiningYoutube Grounding}  \\ 
%     	\cmidrule(lr){2-3} \cmidrule(lr){4-5} 
%     	Method  & Acc & IoU   & Acc & IoU \\ 
%     	\midrule
%     	CoMMA* \citep{tan2021look}   & 48.63 & -  & 47.68 & -  \\
%     	MIL-NCE \citep{miech2020end} & 23.67 & -  & 27.45 & -  \\
%     	Ours                        & 48.03 & -  & 47.35 & -  \\
%     	\bottomrule
%     \end{tabular}
%     \vspace{+0.3cm}
%     \caption{Evaluation on spatial-only evaluation using pointing game accuracy and attention heatmap IoU with GT bounding box. Models learning global representation doesn't perform well on localization tasks, while our model maintain comparable performance.
%     \label{tab:spatial}
%     %\vspace{-0.2cm}
%     }
%     %}
    
% \end{table}

\subsection{Downstream Tasks}


%We compare to the SOTA self-supervised method evaluated on spatial \citep{tan2021look} and temporal \citep{kuehne2019mining} grounding.

We considered the following downstream tasks to evaluate spatio-temporal grounding abilities of various models (detailed description is included in the appendix \ref{eval_metric}):

\noindent (i) \textbf{Spatio-temporal grounding in untrimmed video} is evaluated on the proposed Grounding Youtube dataset. The entire video and the respective pool of action instructions were provided. The model needs to localize each action step in time (start-time/end-time) and space (location in the video) as described in Figure \ref{fig:inference}. 
% We evaluate in two metrics: \textbf{IoU+Pointing game} combines spatial grounding~\citep{akbari2019multi} and temporal grounding~\citep{kuehne2019mining} metrics. %For each video frame, the prediction is correct when the model predicts the correct action for the frame. Also, given the predicted action as a query, the maximum point of the heatmap aims to lie within the desired bounding box. We then compute the Intersection over Union (IoU) over all the predictions with the GT to acquire the final score. 
% We also compute \textbf{video mAP} following previous evaluation~\citep{gu2018ava}, where we set IoU threshold between GT and predicted spatio-temporal tubes. A prediction is correct when it surpasses the IoU threshold. We compute the mAP over all classes. %We form a 3D prediction mask following Figure \ref{fig:inference} and compute IoU between our 3D heatmap and 3D tube.
We evaluate in two metrics: \textbf{IoU+Pointing game} combines the evaluation setting from the spatial grounding \citep{akbari2019multi} and temporal grounding \citep{kuehne2019mining} metrics. For each video frame, the prediction is correct when the model predicts the correct action for the frame. Also, given the predicted action as a query, the maximum point of the heatmap aims to lie within the desired bounding box. We then compute the Intersection over Union (IoU) over all the predictions with the GT to acquire the final score. 
We also compute \textbf{video mAP} following previous evaluation \citep{gu2018ava}, where we set IoU threshold between GT and predicted spatio-temporal tubes. A prediction is correct when it surpasses the IoU threshold. We then compute the mAP over all classes. We form a 3D prediction mask following Figure \ref{fig:inference} and compute IoU between our 3D heatmap and 3D tube.

\noindent (ii) \textbf{Spatial grounding} is given a text description to localize the region in the trimmed video. %We use GroundingYoutube, Youcook-Interaction, V-HICO, and Daly for evaluation. %Note that the evaluation is spatial only. It evaluates the results for each frame separately without considering the temporal information. 
It is evaluated using the \textbf{pointing game accuracy}. %Given the query text and video, we compute the attention heatmap on the video as described in Figure \ref{fig:inference}(b). 
If the predicted point lies in the ground truth bounding box, the result counts as a ``hit" and counts as ``miss" otherwise. The final accuracy is calculated as a ratio between hits to the total number of predictions $\frac{\text{\# hits}}{\text{\# hits} + \text{\# misses}}$. 
We also report the mean average precision \textbf{(mAP)} following the settings from V-HICO~\citep{li2021weakly}. %Given a human-object category as the text query, we aim to localize the spatial location in the video frame.
%The predicted location is correct if their Intersection over-Union (IoU) with ground truth bounding boxes is larger than 0.3. 
%Since we do not use any bounding box proposal tools or supervision, we create an attention heatmap as described in Figure \ref{fig:inference}(b) to create a mask for IoU computation. 
%We follow \citep{li2021weakly} and compute the mAP over all verb-object classes.


\noindent (iii) \textbf{Temporal grounding} \label{temporal_grounding}
provides videos with the respective actions and their ordering, including the background. The goal is to find the correct frame-wise segmentation of the video. We follow the inference procedure in \citep{kuehne2019mining} to compute the alignment given the similarity input matrix. The task is evaluated by intersection over detection (IoD), defined as $\frac{G \cap D}{D}$ the ratio between the intersection of ground-truth action $G$ and prediction $D$ to prediction $D$, and the Jaccard index, which is an (IoU) given as $\frac{G \cap D}{G \cup D}$.



\subsection{Comparison with state-of-the-art methods}\label{sota}
\noindent (i) \textbf{Spatio-temporal grounding in untrimmed video:}
We first compare the proposed method with other approaches designed for spatial or temporal grounding in Table \ref{tab:st_long}.
It shows that models without specific loss designs for spatial grounding (MIL-NCE~\citep{miech2020end}, CLIP~\citep{radford2021learning}) show good mAP scores but lower pointing game accuracy. Out of the two weakly supervised methods, GLIP~\citep{li2022grounded} and RegionCLIP~\citep{zhong2022regionclip}), trained with aligned image-text, RegionCLIP show significantly better performance in this setting, while both perform in a similar range in the spatial grounding scenario (see Table~\ref{tab:spatial}). We attribute this behavior to the fact that RegionCLIP distinguishes frames with relevant queries better from background than GLIP, leading to better temporal localization. 
We finally compare the strong baseline MIL-NCE+RegionCLIP, which combines two approaches specialized in temporal and spatial aspects, to our task. 
It shows that the proposed method improves over all other baselines underlining the need to incorporate global (temporal) and local (spatial) representations. 
%Experiments showed that combining a joint objective that learns spatial and temporal information jointly results in better performance than simply applying the best temporal and spatial model. 
% Also, such a combined objective also benefits more when the visual backbone is finetued as well. 
% We construct a split with single action shown in appendix \ref{single_action_stg}.
%Models designed for trimmed videos (CoMMA\citep{tan2021look}) or trained with aligned image-text (GLIP\citep{li2022grounded}, RegionCLIP\citep{zhong2022regionclip}) failed to capture the temporal dynamics, while models without specific loss designs for spatial grounding (MIL-NCE\citep{miech2020end}, CLIP\citep{radford2021learning}) were not able to ground the action in the correct region.
%Note that supervised spatio-temporal grounding approaches~\citep{yang2022tubedetr,jin2022embracing} are not directly applicable in this evaluation since such methods assume the given text query to be ground-truth. %The model must distinguish the correct text query from a pool of action lists. 
%We include an evaluation setting in the supplement where the GT-text queries were provided. \hkc{Do we? If not, we can probably comment the last 2 sentences}
%More experiment setting is in the supplement.

\begin{table}[h]
     \tablestyle{2pt}{1.05}
    
    \centering
    %\resizebox{1\columnwidth}{!}{
    \begin{tabular}{@{}l|ccccc}
    	\toprule
    	%\multicolumn{4}{c}{} &\multicolumn{2}{c}{MiningYoutube}  \\ 
    	%\cmidrule(lr){5-6} 
    	Method   & Backbone &Data & Super. & IoU & IoD \\ 
    	\midrule
    	Mining: MLP \cite{miech2020end} & TSM & MiningYT & Weak & 9.80 & 19.20    \\
             CoMMA* \cite{tan2021look} & S3D-word2vec & HT250K & Self & 2.05 & 5.63    \\
    	MIL-NCE \cite{miech2020end} & S3D-word2vec & HT100M & Self & 18.69 & 26.74    \\
    	Ours                       & S3D-word2vec & HT200K & Self  & 19.18 & 27.65   \\
    	%Ours                       & VAT& S3D-g  & 19.40 & 28.48   \\
            Ours                       & CLIP & HT200K & Self &  \textbf{19.88} & \textbf{28.50}   \\
             %\midrule
            % MCN \cite{chen2021multimodal}      &VAT& R152+RX101   & 23.10 & 32.04    \\
    	\bottomrule
    \end{tabular}
    \vspace{-0.3cm}
    \caption{\textbf{Temporal Grounding on MiningYoutube.} %Spatial-focused model CoMMA is not trained for temporal detection, which results in lower performance, while the proposed model combines global and local representation resulting in better temporal localization than one alone. %\bc{we should include setting without knowing the order}
    %\vspace{-0.5cm}
    \label{tab:temporal}
%    \vspace{-0.4cm}
    }
    %}
\end{table}

\noindent (ii)~\textbf{Spatial grounding: } 
 %We do not report mAP on Youcook interaction since the input is sentence descriptions instead of class.
Second, we compare the performance of the proposed framework to other methods on the task of spatial grounding, including models with weak supervision, as well as models trained in a fully supervised setting in Table \ref{tab:spatial}.
%As shown in Table \ref{tab:spatial}, models trained with global representations such as MIL-NCE and CLIP were not able to localize the text description compared to models learning local representations such as CoMMA, GLIP, RegionCLIP and our approach. 
In the instruction video domain (GYT and YC-Inter), the proposed approach achieves the best result among all weakly and self-supervised trained methods. In the general domain (V-HICO and Daly), the method also achieves competitive results, showing the generalizability of the model to other domains. 
%We attribute this to the transformer architecture in the text branch inheriting knowledge from the open domain during large-scale training, while in contrast the model's performance using word2vec dropped in these datasets. 
Note that in the Daly dataset, the classes are verbs, which are not detectable by the object-focused model GLIP. 
Compared to their weakly trained counterparts, fully-supervised model (TubeDETER~\citep{yang2022tubedetr}, STCAT~\citep{jin2022embracing}) achieve competitive performance in the general domain (V-HICO, Daly) and slightly lower performance in instruction domain (GYT, YC-Inter) due to the domain gap with respect to the training data.
\begin{figure}
       \centering
        \setlength{\tabcolsep}{1pt}
        {\scriptsize
        \begin{tabular}{c c c c c c c }
            { Original } &
            \multicolumn{2}{c}{  } &
            \multicolumn{4}{c}{$\longleftarrow$ Object level variations $\longrightarrow$} \\
            \includegraphics[width=0.185\linewidth]{images/ablation/chair.jpg} &
            \multicolumn{2}{c}{  } &
            \includegraphics[width=0.185\linewidth]{images/ablation/1_only_prompt_mixing/bench.jpg} &
            \includegraphics[width=0.185\linewidth]{images/ablation/1_only_prompt_mixing/stool.jpg} &
            \includegraphics[width=0.185\linewidth]{images/ablation/1_only_prompt_mixing/armchair.jpg} &
            \includegraphics[width=0.185\linewidth]{images/ablation/1_only_prompt_mixing/saddle.jpg} \\
            \multicolumn{3}{c}{  } &
            \multicolumn{4}{c}{ Only Prompt Mixing } \\
            \multicolumn{3}{c}{ } &
            \includegraphics[width=0.185\linewidth]{images/ablation/2_with_self_attn_injection/bench.jpg} &
            \includegraphics[width=0.185\linewidth]{images/ablation/2_with_self_attn_injection/stool.jpg} &
            \includegraphics[width=0.185\linewidth]{images/ablation/2_with_self_attn_injection/armchair.jpg} &
            \includegraphics[width=0.185\linewidth]{images/ablation/2_with_self_attn_injection/saddle.jpg} \\
            \multicolumn{3}{c}{  } &
            \multicolumn{4}{c}{ + Attention-Based Shape Localization } \\
            \multicolumn{3}{c}{ } &
            \includegraphics[width=0.185\linewidth]{images/ablation/3_background_blending/bench.jpg} &
            \includegraphics[width=0.185\linewidth]{images/ablation/3_background_blending/stool.jpg} &
            \includegraphics[width=0.185\linewidth]{images/ablation/3_background_blending/armchair.jpg} &
            \includegraphics[width=0.185\linewidth]{images/ablation/3_background_blending/saddle.jpg} \\
            \multicolumn{3}{c}{  } &
            \multicolumn{4}{c}{ + Controllable Background Preservation } \\
        \end{tabular}
        }
    \vspace{1mm}
    \captionof{figure}{
    Ablating our full object variations pipeline. Original image was crated using the prompt ``A \emph{chair} with a dog on it''. 
    }
    \vspace{-10pt}
    \label{fig:ablation}
\end{figure}

\section{Visualization On Demand} %Visualization Elements
\label{sec:visrisk}
Based on environment data and trajectory evaluation, we now present ways of communicating the situation and risks on a visual display to achieve an ADAS.
In this context, we employ a renderer that visualizes all the information in a joint Cartesian coordinate system (see section \ref{subsec:sim}). 
Once driving risks are detected, design elements are overlayed on the display with section \ref{subsec:active} and section \ref{subsec:warning}. 

\subsection{Simulator Environment}
\label{subsec:sim}
Nodes of the R-LDM have a range of potential attributes, such as the 3D position or geometrical shape of objects. 
% For instance, the road centerline is a polyline with bounderies to the left and right. Crosswalks have a defined width and buildings a polygonal outline description. 
In the renderer, we always visualize static and quasi-static data that lie in the field of view from the ego vehicle. 
For this, a local 3D model is generated by converting geographic points with (lat, lon, alt) into Cartesian coordinates of (x, y, z). 
% and project the positonal relations from a view perspective with a transformation matrix. 
Fig. \ref{fig:3Dsimulator} depicts an exemplary map section having several intersections in bird's-eye view.
% with several intersections, stop lines and crosswalks. 
On the top right, the first person view of a vehicle approaching a crosswalk is shown. 

The dynamic data is then added to this static view. A zoomed-in excerpt from the map is given at the bottom of Fig. \ref{fig:3Dsimulator} that includes a recorded GNSS trace (red).
We project the trace onto the connected lane center, which is pictured in green. 
% Because we project the ego position on the closest lane segment, on the bottom right the measured trace is changed in red and the aligned trace is marked in green.
Consequently, the virtual horizon and its possible paths are retrieved as described in section \ref{subsec:ldm}. 
We can lastly update and move the excerpt with the current position from the GNSS to obtain a live simulation.

\subsection{Proactive Support}
\label{subsec:active}
Communication of spatial as well as spatio-temporal relations is crucial for risk-averse driver support. 
% This has the reason that humans can estimate the time better than positions (especially for risks). 
% Velocity contains implicitly the time as well. 
Further sources of information are cause, likelihood and severity of a potential risks.  
% if a collision happens. 
The next step for RNS is the choice of suitable design elements. 
In this process, we suppose that we know where the ego vehicle is driving (i.e., the ego path) from its navigation route. 
Yet, for surrounding vehicles, all paths are considered.

\subsubsection{Hazard Route Element}
The so-called hazard route in Fig. \ref{fig:charts} is a concept that consists of a scale portraying distances to an upcoming risk element.
Furthermore, the geometrical area or length of risks is considered.
Risk is thus measured with respect to the ego path, ranging from the current position  $\Delta l \hspace{-0.03cm}=\hspace{-0.03cm} \unit[0]{m}$ to the end of the path $\Delta l_{h}$.
Here, the length $\Delta l_{h}$ can be chosen according to own preferences. 

At an upcoming intersection, risk is defined by the section of the path that lies within the junction.
Since risk corresponds to exposition time, we encode the path part from the intersection $I_z$ with a color, ranging from green for short intersections to red for long ones. 
%allgemein risiko entlang des pfades zu intersection zone
%share of junction segment to navigation route + 
%one case with large intersection far and one case with small intersection close
Fig. \ref{fig:charts}~a) gives two examples of the hazard route.
The left bar shows a large intersection (e.g. multi-lane four-way stop) in vicinity and the right bar has a small and consecutive medium junction. 
% In the case of collision risk, the intersection zone $I_z$ can be used.
% Depending on the value of $I_z$ (low, medium and large), the area is marked from green, to yellow until red for conveying the criticality. 
This emphasizes that we may include more than one intersection in our warnings.

\begin{figure}[t]
  \centering
  \includegraphics[width=0.95\linewidth]{./img/simulator.png}
  \caption{Rendered road network from two perspectives with the ego position being projected on the navigation route. \vspace{0.45cm}}
  \label{fig:3Dsimulator}
\end{figure}

\begin{figure}[t]
  \centering
  \resizebox{\linewidth}{!}{
  \import{img/}{velocity_scale_new.pdf_tex}}  
  \caption{Chart elements for proactive support. Hazard route (left) and velocity scale (right).} %\vspace{-0.3cm}}
  \label{fig:charts} 
\end{figure} 

\subsubsection{Velocity Scale Element}
The velocity scale, Fig. \ref{fig:charts}~b), is a second chart element which qualifies the difference between the current velocity of the vehicle $v_0$ and the target velocity $v_{\text{tar}}$ from the trajectory evaluation of section \ref{subsec:trajeval}. 
The scale shows possible velocity values, from standstill $v\hspace{-0.05cm}=\hspace{-0.05cm}\unit[0]{m/s}$ to a maximal velocity $v_{\text{max}}$. Depending on the difference $|v_0 \hspace{0.05cm} - \hspace{0.05cm} v_{\text{tar}}|$, the situation is rated as safe with $v_0 \hspace{-0.042cm} \approx \hspace{-0.042cm} v_{\text{tar}}$ (green, left), as dangerous with e.g. $v_0 \hspace{-0.05cm} < \hspace{-0.05cm} v_{\text{tar}}$ (yellow, middle) to critical with $v_0 \hspace{-0.07cm} \ll \hspace{-0.07cm} v_{\text{tar}}$ (red, right). The same cases hold true for the opposite circumstances, i.e., $v_0 \hspace{-0.032cm} > \hspace{-0.032cm} v_{\text{tar}}$. 
This velocity scale can be employed for curve or regulatory risks. 
Moreover, we may set an enforced speed limit as the target velocity $v_{\text{tar}}$ for proactive behavior, once there is no risk ahead. 
%\noindent -Warning vs behavior support \\
%-Ghost vehicle as in game \\

\subsection{Short-Term Warning Elements}
\label{subsec:warning}
In order to emphasize the criticality of the situation, we propose to add further intuitive warning elements as e.g. pop-up signs and lane colorings. 
The following elements augment the proactive elements.

\subsubsection{Pop-up Signs}
Explicit symbols indicate the risk cause accompanied with the event time for collisions ($s_E$), distances to the risk spot for turns (i.e., right curve with $d_r$ and left curve with $d_l$) or stopping distance for crosswalks ($d_c$). In Fig. \ref{fig:popups}~a), the pop-up signs are pictured. 
% Besides the velocity difference, the risk type is an indication for the severity of the situation.
%Examples for collision risk are car-to-car crash., curve risk can be  as a single-car accident and regulatory risks will be a car-to-object collision. 
We want to stress that this is just a selection and more risk causes can be added. 
The purpose is also to clarify the reason for the warning and give more human-understandable information.

\subsubsection{Colored Events}
Finally, we highlight lane parts or positions according to the corresponding risks.  
% the determined color rating from the hazard route and velocity scale and relate the risks to the simulator environment. 
In the instance of curve and regulatory risk, the lane is colored from the ego position up to the point of maximal risk. 
For collision risk, we mark the point of the closest encounter as a red cube.
An illustration for regulatory risk induced from a stop line is depicted in Fig. \ref{fig:popups}~b). Again, the color is defined by the deviation $|v_0-v_{\text{tar}}|$. It also shows the therein considered navigation route with length $\Delta l_h$ and another unlikely path. 

It should be noted that the visualization of warnings only occurs if the risks are actually present. 
%\textcolor{red}{improve language, repeat intersection zone and navigation route}
%eingrauen unlikely paths and navigation path and describe in text, maybe delete Iz -> put line from unlikely path to green arrow
Altogether, the RNS provides a variety of tools to analyze and circumvent critical situations in intersection scenarios, while not overloading the driver's awareness.

\begin{figure}[t]
  \centering
  \resizebox{\linewidth}{!}{
  \import{img/}{colored_lane_new.pdf_tex}}  
  \vspace{-0.53cm}
  \caption{Short-term warning elements. Selected pop-up warnings (left) and colored lane (right).}
  \label{fig:popups} 
\end{figure} 



\noindent (iii)~\textbf{Temporal grounding:}
We evaluate temporal grounding in Table \ref{tab:temporal}. Here, it shows that global representations also profit from local representation learning.%, achieving state-of-the-art results in temporally localizing actions in untrimmed videos. 
This hypothesis is further validated in the ablation studies in Table~\ref{tab:train_ablations}, where we ablate both losses for all three settings and show a consistent improvement in the joint loss formulation. 

%Our model achieved comparable results with 
%Called action step localization. Evaluated on Mining Youtube. 




%\input{tables/spatial_temporal_clip}






% \noindent (iv) \textbf{Spatio-temporal Clip :}
% \label{ST_clip}
% Following the current spatio-temporal datasets \citep{jiang2014thumos,gu2018ava} which aim to discriminate the action class from the background class in a short clip, we construct a clip level evaluation where the clip varies from 9 sec to 60 Section  Given an action step, we append the video segments before and after the steps with the same time length of the action step to form the final video clip. This results in 2,895 clips for the spatio-temporal clip grounding evaluation.
% For each clip, the  temporal action intervals occupy 33\% of corresponding videos, which demonstrates the difficulty of the setting. As shown in Table \ref{tab:st_clip}, we observe a similar trend as the full video evaluation where our model outperforms all the baselines. 




\subsection{Ablation study} 
%\vspace{-1mm}
We perform ablation studies with respect to all three settings, spatio-temporal grounding, as well as spatial and temporal grounding alone, reporting performance for spatio-temporal grounding on GroundingYT using mAP with IoU@0.4, on temporal grounding using MiningYT IoU, and on spatial grounding using YC-Inter. pointing game. Additional ablation are in appendix \ref{ablation_sup}. %For each setting, we use the same feature extractor for three modalities as described in Sec 4.1 for a fair comparison. 

% add summary here?
% as they are the less computational evaluation tasks.
%This subset of downstream tasks has been chosen for their simplicity of evaluation and because they cover a wide range of tasks.

\noindent\textbf{Frame selection strategy.} 
We perform an ablation on the possible frame selection strategies for our method (Figure \ref{fig:pipeline}(b) and Section \ref{sinkhorn_main}). In Table \ref{tab:frame_ablations}, \textit{None} uses all frames within the ASR boundary ($U=T$) as our video training data. 
\textit{Global} represents the [CLS] token in text and video. \textit{Local} uses the words and spatio-temporal tokens. In the setting Sinkhorn was not applied, the top $T$ frames with the highest similarity score were selected. When we set spatio-temporal tokens as the selection target, we sum over the scores with respect to each frame to acquire the frame similarity score.
%\textit{Global} utilizes the global sentence resp. frame [CLS] token as the query to rank the top $T$ similar frames as the selected frames for training. \textit{Local} uses the words resp spatial-temporal tokens instead of the CLS token as a query and selects the frames with the closest feature distance. 
It shows that selecting frames based on Sinkhorn selection leads to consistently better results as it enforces more variety of visual concepts but also captures frames with possible groundable objects. It further shows that word tokens are more suitable than the global text CLS token for frame selection. Finally, we see that depending on the task (spatial vs. temporal), a local resp. global representation is better, and a combination of both works best for spatio-temporal grounding. 
%, which improves overall performance.%, leading to better supervision.
We provide runtime analysis of such frame selection strategy in the appendix \ref{runtime}.
% \noindent\textbf{Number of frames for training.} We tested different video lengths $T$ used for training. As shown in Table \ref{subtab:ablations2}, selecting less frames for training significantly causes the performance to drop. We hypothesize that not only does the model fail to capture the temporal dynamics with less frames, but loses some frames with groundable objects in the sentence while training. We also found that when the number of frames increases, more irrelevant frames might be selected during training, which decreases the performance.
\begin{table}[!t]
  \centering\small
  \caption{%
    Ablation study on dual-form approximate rank loss.
  }
  \vspace{-3pt}
  % \renewcommand{\arraystretch}{0.8}
  \setlength{\tabcolsep}{2.4mm}{
    \begin{tabular}{l|cccccc}
    \toprule
    \multirow{2}{*}{Loss} & \multicolumn{2}{c}{\textbf{IoU = 0.1}} & \multicolumn{2}{c}{\textbf{IoU = 0.3}} & \multicolumn{2}{c}{\textbf{IoU = 0.5}} \\
    & R@1 & R@5 & R@1 & R@5 & R@1 & R@5  \\
    \midrule
    $\mathcal{L}_{bce}$  & 0.05  & 0.51 & 0.01 & 0.10 & 0.00 & 0.01 \\
    $\mathcal{L}_{nce}$  & 5.26  & 13.65 & 4.09 & 10.90 & 2.32 & 6.73 \\
    $\mathcal{L}_{ar}$   & 10.08  & 22.02 & 8.15 & 18.47 & 4.80 & 12.04 \\
    \midrule
    $\mathcal{L}_{dar}$  & \textbf{11.03}  & \textbf{22.99} & \textbf{8.83} & \textbf{19.48} & \textbf{5.23} & \textbf{13.18} \\
    \bottomrule
    \end{tabular}
  }
  \vspace{-8pt}
  \label{tab:ablation_loss}
\end{table}

%\vspace{-0.1cm}
\noindent\textbf{Global and local loss.} As mentioned in the spatio-temporal evaluation, both features contribute to the final grounding result. We test the model by ablating out each loss. 
Table \ref{tab:train_ablations} shows that each loss not only contributes to the spatio-temporal grounding on the GYT, but also that the whole is more than the sum of its parts (losses) since this task requires both spatial and temporal detection. The reduced impact of the global loss in the case of YC-Inter is that this is a pure spatial grounding dataset (no background frames) without temporal detection, and the local loss plays a more critical role. We observe the same patterns in the temporal grounding result for MYT, where spatial localization is not directly contributing to the final performance. We tried out the same ablation using in the S3D backbone in supplement.
%We provide runtime analysis of different losses in the appendix \ref{runtime}.
%By comparing the results for spatio-temporal grounding in untrimmed videos (Table 1) vs. spatial grounding in trimmed videos (Table 3),  we can further see the impact of the proposed joint representation.

%\bc{to appendix}




% adding the global loss improves the ground performance. This results also shows that spatial grounding benefits from global representation learning. In the spatio-temporal setting, the performance without a global or local loss outperforms other baselines.

% \noindent\textbf{Dataset for training.} As mentioned in Section \ref{dataset}, we trained models with data with food categories. In Table \ref{subtab:ablations4}, we also tested our model trained with a larger set of food and entertaining called HowTo370K used in \citep{han2022temporal}. The full set of HowTo100M contains a total of 1M long videos, which is five times the size of our dataset. We found training with our 200K videos reaches similar performance with much less training hours.

% \noindent\textbf{Affect of audio in training and testing.} Unlike text which describes a discrete concept as a target to ground, audio serves as a continuous representation that is highly relevant to the temporal information. For example, we can determine an action started when we hear a ``cracking'' sound. In Table \ref{subtab:ablations5}, we tested our model using the additional audio modality by expanding our architecture and loss from VT to VAT. We found when training and testing with audio, the spatio-temporal result increases while the spatial-only result remains the same. This validates our assumption that audio contributes more to temporal understanding. When we trained on audio and tested without audio, the performance increases over the VT model, showing that the audio serves as useful supervision for better video/text representations. More details are presented in the supplement. 

\subsection{Qualitative results}
\vspace{-1mm}
We visualize our spatio-temporal result in Figure \ref{fig:visualization}. For the GLIP model, we output the bounding box with the highest confidence score and visualize its center point. We found GLIP model focuses on the salient object while our model focuses more on human-object interaction.


 \section{Conclusion}
 In this paper, we have presented a tactile manipulation system that is able to rotate different objects without vision. We showed an end-to-end reinforcement learning framework to learn tactile dexterity over the proposed system. We carried out experiments both in simulation and real to demonstrate its effectiveness. Our work demonstrated that we are able to achieve tactile dexterity as humans in real for the first time. In the future, there are many promising future directions to investigate, such as exploring the use of a more dense contact sensor array and scaling up the system to solve more diverse tasks. We hope that our work can pave the way for more intelligent robot hands.


{
    \small
    \bibliographystyle{ieeenat_fullname}
    \bibliography{main}
}

\clearpage
\section{Appendix}

%% Qualitative figures leading
%
\begin{figure*}[!ht]
    \centering
    \includegraphics[width=0.75\linewidth, keepaspectratio]{figures/supplement/arkit_supplemental_scenes.jpg}
    \caption[]{Additional results on the ARKitScenes dataset \cite{dehghan2021arkitscenes}, compared to geometric clustering and oversegmentation-based baselines.} 
    \label{fig:arkit_supplement}
\end{figure*}
%
\begin{figure*}[!ht]
    \centering
    \includegraphics[width=0.85\linewidth,keepaspectratio]{figures/supplement/scannet_supplemental_scenes.jpg}
    \caption[]{Additional results on the ScanNet dataset \cite{dai2017scannet}, compared to geometric clustering and oversegmentation-based baselines.} 
    \label{fig:scannet_supplemental}
\end{figure*}

\subsection{\OURS{} as Data Efficient Pretraining}

We report additional qualitative details on the data efficient pretraining performance of \OURS{} in Table~\ref{tab:data_efficient}.

We also note that the 3D contrastive pre-training of CSC, similar to other 3D pre-training methods developed for non-transformer backbones \cite{xie2020pointcontrast, hou2021exploring, zhang_depth_contrast,nunes2022segcontrast}, was not beneficial for a transformer-based model. A similar observation was also reported in a recent pretraining method \cite{hou2023mask3d}. 
We thus also compare with CSC pretraining on their original 3D backbone (which demonstrated improvement over training from scratch on the same backbone).
Our approach can improves notably over both alternatives.
%
\begin{table*}[!ht]
\centering
\resizebox{\linewidth}{!}{
\begin{tabular}{ccccccccccccccccc}\toprule
& & \multicolumn{3}{c}{1\%} &  \multicolumn{3}{c}{5\%} &  \multicolumn{3}{c}{10\%} &  \multicolumn{3}{c}{20\%} & \multicolumn{3}{c}{50\%}
\\\cmidrule(lr){3-5}\cmidrule(lr){6-8}\cmidrule(lr){9-11}\cmidrule(lr){12-14}\cmidrule(lr){15-17}
  Model & Backbone  & AP@25  & AP@50 & AP   & AP@25  & AP@50 & AP & AP@25  & AP@50 & AP & AP@25  & AP@50 & AP & AP@25  & AP@50 & AP \\\midrule
Scratch & Bottom-up & 22.6 &14.1 &6.8 &45.5 &33.3 &18.1 &54.8 &39.2 &21.9 & 61.0 &43.4 &25.5 &67.0 &51.4 &30.3 \\
CSC \cite{hou2021exploring} & Bottom-up &35.6 &22.1 &12.5 &52.7 &39.9 &23.3 &59.8 & 43.8& 25.0& 63.8& 48.9& 29.6& 70.5& 56.0 & 33.6 \\
Scratch & Transformer & 24.7 & 9.3 & 4.6 & 48.1 & 27.6 & 16.3 & 59.2 & 39.1& 23.4& 66.4 & 49.6 & 33.1& \textbf{78.9}& 67.5& \textbf{49.8} \\
CSC  & Transformer & 17.0& 6.8& 3.8& 44.2& 22.7& 13.1& 55.2& 32.3& 19.1& 62.0& 41.2& 26.0& 73.7& 58.2&40.0 \\
Ours & Transformer & \textbf{43.5} & \textbf{28.4} & \textbf{15.8} & \textbf{63.2} & \textbf{46.8} & \textbf{28.3} & \textbf{70.3} & \textbf{55.7} & \textbf{36.7} & \textbf{72.4} & \textbf{60.7}& \textbf{41.5} & \textbf{78.9}& \textbf{68.0} & 48.2 \\ \bottomrule
\end{tabular}
}
\caption{Unsupervised class-agnostic pretraining with our method can also act as a powerful pretraining strategy, advancing over state of the art. 
We report pretraining with CSC \cite{hou2021exploring} and \OURS{}, and evaluate the downstream weakly-supervised instance segmentation performance on ScanNet with percentage of limited annoated scenes used denoted in the top row. 
As we found that CSC degraded performance when using a transformer-based backbone, we also report the performance of training from scratch and CSC on their originally proposed backbone of a sparse UNet with bottom-up voting.}
\label{tab:data_efficient}
\end{table*}

%%% NOISE Robust LOSSES
\subsection{The effect of noise robust losses.} 
We adopt DropLoss \cite{wang2023cut} for our self-training cycles, which is robust to sparse data and missing annotations. 
In particular, we use a weighted combination of cross-entropy and Dice \cite{sudre2017generalised_diceloss} losses for bipartite-matching with pseudo annotations.
We then drop losses for backpropagation which do not have at least $\tau_{drop}$ overlap with the annotations from the previous cycle.
We evaluate the effect of different noise robust losses for self-training in Table~\ref{tab:noise_robust_ablation}. 
We compare our baseline losses with a 3D extension of the projection loss of \cite{wang2022freesolo}, and our adaptation of  DropLoss from \cite{wang2023cut}.
Our approach does not penalize for missing pseudo masks, which enables more effective self-training to discover previously missed instances.

\begin{table}[!ht]
\centering
\small
\begin{tabular}{lcccc}\toprule
     & AP@25  & AP@50 & AP & AP Final\\\midrule
Initial Pseudo Masks &  19.9  &  10.0  & 5.9  & -\\
Baseline losses \cite{Schult23mask3d} &  42.3  &  16.9 &  7.2 &  14.2 \\
Projection loss \cite{wang2022freesolo} &  35.7  &  12.1  &  4.7 & 7.2 \\
%DropLoss@0.01 &  45.4  &  18.2  &  7.5  \\
%DropLoss@0.05 &  48.7  &  21.8  &  9.5  \\
%DropLoss@0.1 &  \textbf{52.8}  &  \textbf{23.4 } &  \textbf{10.3}  \\ \bottomrule
DropLoss \cite{wang2023cut} &  \textbf{52.9}  &  \textbf{23.2} &  \textbf{10.4}  & \textbf{15.9} \\ \bottomrule
\end{tabular}
\caption{
A 3D projection loss struggles with under-determined associations, while DropLoss helps \OURS{} to discover parts of the scene that were missed by the source supervision. We report all metrics after a single iteration and the AP scores after 4 iterations of self-training.}
\label{tab:noise_robust_ablation}
\end{table}

\subsection{Additional Qualitative Results}

We show more qualitative results from our method trained on ARKitScenes \cite{dehghan2021arkitscenes} in Figure \ref{fig:arkit_supplement} and on ScanNet \cite{dai2017scannet} in Figure \ref{fig:scannet_supplemental}. 

\subsection{Pseudo Mask Generation Ablations}

We also ablate the saliency threshold, oversegmentation parameters, and separation strategy  in our pseudo mask generation. If not explicitly stated otherwise in Table \ref{tab:freemask_vs_ncut}, we use both 2D and 3D modality features for the pseudo mask generation. 

\paragraph{What is the effect of the saliency threshold in pseudo mask generation?} 

 We threshold the saliency matrix $A$ with $\tau_{cut}=0.55$ for geometric-only features and $\tau_{cut}=0.65$ for combined modalities. Table~\ref{tab:cutler3d_tau} shows that our approach maintains robust performance across a large range of $\tau_{cut}$ thresholds used to estimate salient areas for pseudo masks. In this table we report results using features from combined modalities, but similar behaviour can be observed for the other scenarios as well.

\begin{comment}
    \begin{table}
\centering
\small
\begin{tabular}{lccc}\toprule
$\tau_{cut}$ & AP@25  & AP@50 & AP \\\midrule
0.40 &  10.9  &  3.9  &  1.7  \\
0.50 &  12.5  &  4.7  &  1.9  \\
\textbf{0.55} &  \textbf{13.8}  &  \textbf{4.7}  &  \textbf{2.0}  \\
0.60 &  13.4  &  4.5  &  1.8  \\
0.70 &  13.3  &  4.5  &  1.8  \\
0.80 &  8.7  &  3  &  1.2  \\ \bottomrule
\end{tabular}
\caption{
Our pseudo mask generation quality, as measured by AP metrics, maintains robustness to a large range of $\tau$ thresholds that extract saliency.
Note that this measures the quality of only the pseudo masks; our full approach with self-training produces significantly improved results. In this table we show results from 3D only features. 
}
\label{tab:cutler3d_tau}
\end{table}
\end{comment}

\begin{table}
\centering
\small
\begin{tabular}{lccc}\toprule
$\tau_{cut}$ & AP@25  & AP@50 & AP \\\midrule
0.40 & 16.7 & 9.0 & 5.2 \\
0.50 & 20.8 & 10.7 & 5.7 \\
0.55 & 21.0 & 10.8 & 5.7 \\
0.60 & 21.3 & 11.3 & 5.8 \\
\textbf{0.65} & \textbf{19.9} & \textbf{10.0} & \textbf{5.9} \\
0.70 & 18.2 & 9.9 & 5.6 \\
0.80 & 11.8 & 5.0 & 2.6 \\ \bottomrule
\end{tabular}
\caption{
Our pseudo mask generation quality, as measured by AP metrics, maintains robustness to a large range of $\tau$ thresholds that extract saliency.
Note that this measures the quality of only the pseudo masks; our full approach with self-training produces significantly improved results. In this table we show results and parameters used by our method in bold and report pseudo mask performance generated from both modalities.}
\label{tab:cutler3d_tau}
\end{table}

\paragraph{The effect of iterative mask densification.}

We designed a strategy to leverage a sparse set of relatively clean initial pseudo masks, which are progressively extended with confident self-predictions during later iterations. This leads to a 3x improvement over state of the art in the Average Precision Metric. We could also consider different mask refinement strategies using a mixture of segments, initial masks or self-trained instances.
Tab.~\ref{tab:mask_refinement} ablates a mask refinement strategy of discarding previous masks and retaining current predictions. We also consider using Felzenswalb segments directly instead of feature-based pseudo labels. Both these strategies lead to lower performance due to the increased presence of noisy labels, which dominate the training signal.

\vspace{-0.3cm}
\begin{table}[!ht]
\centering
\footnotesize
\begin{tabular}{lccc}
                               & AP@25 & AP@50 & AP    \\ \hline \hline
\multicolumn{1}{l|}{Felzenswalb Masks}   & 35.5   & 20.6  &  10.3  \\ \hline
\multicolumn{1}{l|}{Mask Refinement}   &   43.7 & 24.4  &  12.4  \\ \hline
\multicolumn{1}{l|}{Mask Addition (Ours)}   &  58.6  & 32.0  &  16.0  \\ \hline
\end{tabular}
%\vspace{-0.2cm}
\caption{Instead of using masks from previous iteration directly it is the best to keep the initial masks fixed, and iteratively sample plausible predictions to enrich the pseudo dataset during self-training. This method strikes a balance between relatively clean, but sparse labels and increasing number of confident samples. Finally, even though Felzenswalb oversegmentation yields to higher precision, then our initial mask prediction algorithm, it also includes more background into the training, and this way plateauing at a lower self-training performance.}
\vspace{-0.3cm}
\label{tab:mask_refinement}
\end{table}


\paragraph{Robustness to oversegmentation parameters.}
Table~\ref{tab:pseudomaskablation} shows that our approach maintains strong robustness to a wide range of oversegmentation parameters for our geometric segments (our used parameters denoted in bold).

\paragraph{Additional pseudo mask generation hyperparameters.} 
Additionally, we also test the effect of other hyperparameters in out \textit{NCut}-based pseudo mask generation module, including  used distance metrics in the similarity matrix and different methods to separate unconnected patches in the predicted foregrounds. 
During the foreground separation in the Normalized Cut algorithm, we had an additional condition for the minimum number of foreground segments for the bipartitions. This conditions was able effectively filter out suboptimal partitioning of the full graph leading to separated parts from the full instances. Reducing the size of this parameter can directly lead to a more dense set of initial pseudo masks, with the cost of higher false positive rate. In Table \ref{tab:pseudomaskablation} we report a sparser and denser version of the datasets with a minimum number of foregorund segments of 8 and 2 accordingly, and show the initial higher scores of the pseudo annotation doesn't necessarily propagate to better downstream self-trained performance. 

Finally, we also ablate the effect of our physical connectivity-based foreground separation introduced in Section 3.1. In our main method we separate all set of connected components in the foreground, but only keep the component with the highest eigenvector activation (\textit{Max}). As an alternative we also test a method where we calculate the highest average activation in the connected component (\textit{Avg.}), a method where we keep the component with the largest surface value (\textit{Largest}) and finally, to test the effect of this module, without any kind of connectivity-based separation (\textit{No Sep.}). 

\begin{table*}
\centering
\resizebox{\linewidth}{!}{
\begin{tabular}{ccccccccccccccc}\toprule
\multicolumn{4}{c}{Generation Params.} & \multicolumn{4}{c}{Initial Pseudo Mask} & \multicolumn{3}{c}{1 Iteration of Self-Training}  & \multicolumn{3}{c}{4 Iterations of Self-Training}
\\\cmidrule(lr){1-4}\cmidrule(lr){5-8}\cmidrule(lr){9-11}\cmidrule(lr){12-14}
  Segment Size & Metric  & Separation & Min. \# of Foreground & \# of Instances & AP@25  & AP@50 & AP  & AP@25  & AP@50 & AP & AP@25  & AP@50 & AP  \\\midrule
30 & Cos & Max & 8 & 2169 &  21.9 & 11.5 & 6.3  & 53.7 & 26.2 & 12.4 & 55.4 & 30.3 & 15.3\\
 \textbf{50} & \textbf{Cos} & \textbf{Max} & \textbf{8} & 1414 & \textbf{19.9} & \textbf{10.0} & \textbf{5.9}  & \textbf{52.9} & \textbf{23.2} & \textbf{10.4} & \textbf{58.5} & \textbf{32.2} & \textbf{15.9} \\
 100 & Cos & Max & 8 & 1090 &  17.4 & 8.0 & 4.2 & 33.1 & 10.2 & 3.9 & 39.6 & 13.7 & 5.3 \\
 200 & Cos & Max & 8 & 584 &  11.0 & 3.7 & 1.8  & 24.3 & 8.7 & 2.1 & 26.1 & 9.7 & 2.4\\
 400 & Cos & Max & 8 & 319  & 6.4 & 2.5 & 1.1  & 19.1 & 3.9 & 1.2 & 19.9 & 3.2 & 1.0 \\ \midrule
 50 & L2 & Max & 8 & 1539 &  20.1 & 10.6 & 5.4 &  49.0 & 21.7  & 9.8 & 55.3 & 38.4 & 14.3 \\
 100 & L2 & Max & 8 & 805 & 13.3 & 5.3 & 2.6 & 30.8 & 8.3 & 2.8  & 39.0 & 12.7 & 5.0\\ \midrule
 50 & Cos & No Sep. & 8 & 125 &  4.3 & 0.3 &  0.1  & 4.3 & 0.5 & 0.2 & 4.9 & 0.6 & 0.2 \\ 
 50 & Cos & Largest & 8 & 620 &  11.5 & 4.9 & 2.5 &  11.5 & 1.5 & 0.4 & 12.9 & 2.2 & 12.9\\
 50 & Cos & Avg.  & 8 & 1078 & 16.8 & 9.1 & 5.1  & 36.4 & 12.5 & 4.9 & 43.8 & 17.8 & 7.5 \\ \midrule \midrule
 30 & Cos & Max & 2 & 2909 & 29.0 & 15.6 & 8.7 & 53.6 & 28.6 & 14.2 & 54.2 & 29.8& 15.4\\
 50 & Cos & Max & 2 & 2512 &24.9 & 12.4 & 7.2 & 56.5 & 29.8 & 15.0  & 51.3 & 26.2& 12.6\\
 100 & Cos & Max & 2 & 2317 & 23.1 & 12.3 & 6.8 & 51.8 & 24.4 & 11.6  & 57.1 & 31.3& 15.6\\
 200 & Cos & Max & 2 & 2181 & 28.4 & 15.5 & 8.9 & 54.6 & 28.7 &  13.7 & 56.6&31.4 & 15.6\\
 400 & Cos & Max & 2 & 1373 & 20.6 & 11.1 & 6.3 & 51.0 & 24.8 &  11.8 & 55.8 & 30.3& 15.2\\ \midrule
 50 & L2 & Max & 2 & 2496 & 28.6 & 15.8 & 9.0 & 55.8  & 29.6 & 14.6 & 54.8& 30.3& 15.3\\
 100 & L2 & Max & 2 & 1668 & 23.4 & 12.7 & 7.3 & 53.1 & 25.0 & 11.3 & 56.3&27.7 &12.9 \\ \midrule
 50 & Cos & No Sep. & 2 & 159 & 0.2 & 0.5 & 3.6 &  5.4 & 0.6 & 0.3 & 3.9& 0.4& 0.2\\ 
 50 & Cos & Largest & 2 &1026 & 14.1 & 7.2 & 3.9 & 11.5 & 1.8 & 0.5  &14.5 & 2.5& 0.7\\
 50 & Cos & Avg.  & 2 & 2053 & 23.3 & 12.0 & 6.8 & 52.5 & 27.4 & 12.7 & 54.9& 29.9& 14.9\\ \bottomrule
\end{tabular}
}
%\vspace{-0.4cm}
\caption{
We denote the parameters used by our method in bold. 
We show that our method is robust to a wide range of numbers regarding segments sizes and different similarity metrics, and only degrades somewhat in performance when segments are constrained to be too large. 
We also show that the separation of physically distant foreground patches is important and it is beneficial to use the activation of the eigenvector for the best results. 
Finally, we show that denser initial mask predictions lead to quantitatively better initial pseudo annotations, and even better self-training performance after a single iteration, but underperforming in their final scores. This behaviour can be explained by the larger false positive ratio in the denser initial predictions, which is propagating through all iterations, but thanks to the noise robust losses and iterative refinement of predictions the sparse set of labels can be effectively used. In this table we report results using both modalities for the initial pseudo mask generation, and number predicted pseudo instances in the official validation split of the ScanNet dataset.}
\label{tab:pseudomaskablation}
\end{table*}

\begin{comment}
\begin{table}
\centering
\resizebox{\linewidth}{!}{
\begin{tabular}{cccccccccc}\toprule
\multicolumn{3}{c}{Generation Params.} & \multicolumn{3}{c}{Pseudo Mask Scores} \\\cmidrule(lr){1-3}\cmidrule(lr){4-6}
  Segment Size & Metric  & Separation & AP@25  & AP@50 & AP  \\\midrule
 30 & Cos & Max & 21.9 & 11.5& 6.3 \\
 \textbf{50} & \textbf{Cos} & \textbf{Max} & \textbf{19.9}& \textbf{10.0}& \textbf{5.9} \\
 100 & Cos & Max  & 17.4& 8.0& 4.2 \\
 200 & Cos & Max  & 11.0& 3.7& 1.8 \\
 400 & Cos & Max  & 6.4& 2.5& 1.1 \\ \midrule
 50 & L2 & Max  & 20.1& 10.6& 5.4 \\
 100 & L2 & Max  & 13.3& 5.3& 2.6\\ \midrule
 50 & Cos & No Sep.  & 4.3& 0.3&  0.1\\ 
 50 & Cos & Largest  & 11.5& 4.9& 2.5 \\
 50 & Cos & Avg.  & 16.8& 9.1& 5.1 \\ \bottomrule
\end{tabular}
}
%\vspace{-0.4cm}
\caption{
We denote the parameters used by our method in bold. 
We show that the method is robust to distance metrics, the smaller segment sizes work the best. Additionally we show that the separation of physically distant foreground patches is important and it is beneficial to use the activation of the eigenvector for the best results. }
\label{tab:pseudomaskablation}
\end{table}  
\end{comment}




\subsection{Comparison with methods from the 2D domain}

To ensure a fair evaluation of methods operating on different input domains in Table 1. we followed the established procedure of well-known baselines \cite{Dai20183DMVJ3,hou20193d,jaritz2019multi}. This involves using depth information to project 2D predictions into 3D such that all methods are evaluated in the same 3D domain and aggregate multiple predictions through consensus by majority voting or accepting the maximum confidence scores for every voxel location. 
We also show results evaluated against 2D ScanNet images by projecting our method's predictions into 2D in Tab.~\ref{tab:evaluation_2d}, and comparing it to the current state of the art 2D unsupervised segmentation method \cite{wang2023cut} which demonstrates the usefulness of 3D reasoning.

\vspace{-0.3cm}s
\begin{table}[!ht]
\centering
\resizebox{\columnwidth}{!}{
\begin{tabular}{lccc}
                               & AP@25 (2D) & AP@50 (2D) & AP (2D)    \\ \hline \hline
\multicolumn{1}{l|}{CutLER (2D)}   &  7.8  & 2.8  &  0.7  \\ \hline
\multicolumn{1}{l|}{Ours (projected)}   &  60.0  & 38.1  &  21.1  \\ \hline
\end{tabular}}
\caption{2D evaluation on ScanNet images.}
\vspace{-0.3cm}
\label{tab:evaluation_2d}
\end{table}

We also compare to weakly-supervised instance segmentation method SAM3D \cite{yang2023sam3d}, where powerful class-agnostic 2D masks are extracted by the powerful SAM model \cite{kirillov2023segment}. Here the projected 2D masks are merged into 3D masks iteratively with a bottom-up bidirectional merging approach to achieved cleaner and more view-independent 3D instances. A qualitative comparison on ScanNet can be seen in Table \ref{tab:sam3d_results}, with qualitative comparisons in Figure \ref{fig:sam3d_preds}.

\vspace{-0.3cm}
\begin{table}[!h]
\centering
\footnotesize
\begin{tabular}{lccc}
                               & AP@25 & AP@50 & AP    \\ \hline \hline
\multicolumn{1}{l|}{SAM3D}   & 37.2   &  11.8 &   3.7 \\ \hline
\multicolumn{1}{l|}{SAM3D with GT Segments}   & 47.6   & 24.1  & 10.8   \\ \hline
\multicolumn{1}{l|}{Ours}   & \textbf{58.5}   & \textbf{32.2}  &  \textbf{15.9}  \\ \hline
\end{tabular}
\vspace{-0.3cm}
\caption{\OURS{} achieves significantly better performance on ScanNet than SAM3D through our strong multi-modal reasoning. 
}
\vspace{-0.7cm}
\label{tab:sam3d_results}
\end{table}

\begin{figure}[!h]
    \centering
    \centering
    \includegraphics[width=\linewidth]{figures/rebuttal/sam3d_enebled_vs_ours_scene0645_00_cropped.png}
    \vspace{-0.8cm}
    \caption{ While SAM has powerful capabilities in crisp 2D mask generation, when aggregated on 3D, SAM3D tends to over-segment object instances. }
    \label{fig:sam3d_preds}
    \vspace{-0.5cm}
\end{figure}

SAM3D must resolve view inconsistencies and SAM’s tendency to over-segment objects, which results in SAM3D splitting instances, while \OURS{} is able to achieve complete masks through multi-modal reasoning.
We believe integrating SAM or other (weakly-) supervised 2D models into our pipeline to enable multi-modal reasoning is an interesting avenue for future work.

\subsection{Additional Implementation Details}

Here, we further explain the implementation details of our pseudo mask generation. 

\paragraph{Pseudo code for masked NCut}

We show the pseudo code-style implementation for the masked normalized cut algorithm generating multiple instances as pseudo masks. The full algorithm can be seen in \ref{alg:masked_ncut}. 

\begin{algorithm}[ht!]
\DontPrintSemicolon
\caption{Masked NCut on 3D segments} \label{alg:masked_ncut}
\KwData{$\mathcal{S} = \{s_i,  \dots, s_N\}$, $\mathcal{F} \in \mathcal{R}^{NxD}$, 
$\mathcal{C} = \{(s_1, s_k), (s_1, s_l), \dots \}$}
\KwResult{$\mathcal{M} = \{m_j,  \dots, m_M\}$}
$\mathcal{M} \gets \{\}$ \\
\While{$j \le max\_inst\_num$}{
  $\mathcal{F}' \gets \mathcal{F}$ \\
  $\mathcal{F}'[\mathcal{M}] \gets 0.$ \tcp*{Mask out previous insts.}
  $\mathcal{W} \gets \mathcal{F} \times \mathcal{F}^T$ \tcp*{Feature similarity}
  \tcp*[l]{Saliency with connected graph}
  $\mathcal{W}_{i,k} = \left\{\begin{array}{cl}
           1. & \text{if $\mathcal{W}_{i,k} \geq \tau_{cut}$} \\
           \epsilon               & \text{if $\mathcal{W}_{i,k} < \tau_{cut}$} \\
           \end{array}\right.$ \\  \label{algoline:saliency}
  $\mathcal{D}_{i,i} = \sum_{k} W_{i,k}$ \\
  \tcp*[l]{Get $2^{nd}$ smallest eigenvector}
  $\lambda, \mathbf{v} \gets eigh(\mathcal{D} - \mathcal{W}, \mathcal{D}, -2)$ \\
  $m_i = \left\{\begin{array}{cl}
           1 & \text{if $ v_i \geq mean( \mathbf{v} )$} \\
           0 & \text{if $ v_i < mean( \mathbf{v} )$} \\
           \end{array}\right.$ \\   \label{algoline:segment_foreground}
  \tcp*[l]{Invert bipartition if too large}
  \If{$sum(\mathbf{m}) > D / 2$}{
    $ \mathbf{m} = 1 - \mathbf{m} $ \\
    $ \mathbf{v} = -1. * \mathbf{v} $ \\ 
    }
  \tcp*[l]{Separate unconnected components}
  $ v_{max} = max(\mathbf{v}) $ \\
  $ \tilde{\mathbf{m}} = sep(\mathbf{v}, v_{max}, \mathcal{C})$ \\
  $M \gets M \cup \{\tilde{\mathbf{m}}\}$
}
\end{algorithm}

\paragraph{3D Adaptation of FreeMask}

We also evaluate an alternative pseudo mask segmentation algorithm besides the masked \textit{NCut} method. 
In the 2D domain FreeSOLO \cite{wang2022freesolo} also followed a two stage pipeline first generating the pseudo annotations, and then refine those predictions through a series of self-training cycles. We followed their intuition to take a self-supervised pretrained backbone and extract it's deep features at multiple levels of the decoder. 
While in standard pretrained UNet-style models early features represent global context, final features and local semantic meaning, intermediate features can act as an useful proxy to extract self-similar regions in the input samples. 
In our implementation we used the same backbone features of \cite{hou2021exploring,caron2021emerging_dino} for the same 2D-3D setup and extracted the penultimate layer features for the self-similarity calculation. 
Then sampled the feature space with the Furthest Point Sampling \cite{qi2017pointnet++} strategy to get a more limited set of anchor points, later used to extract self-similar regions. 
For every seed point we took similarity scores with the other features of the full scene and thresholded it to extract salient regions. 
Finally, we used the efficient Non Maximum Suppression implementation from \cite{wang2022freesolo} to sort the predicted salient areas and filter out overlapping regions. 
We also used average similarity score combined with the salient region area to get \textit{maskness scores} for every salient region, directly following the original implementation. 
We report comparative results of the masked \textit{NCut} algorithm and our FreeMask 3D adaptation after self-training in Table 3. of the main paper and in Table \ref{tab:freemask_vs_ncut} of the initial pseudo mask scores. 

\begin{table}[!ht]
\centering
\small
\begin{tabular}{lccccc}\toprule
           & Modality  & AP@25  & AP@50 & AP \\\midrule
FreeMask   &  3D  &  13.7  & 7,2  &  3.7   \\
Ours    &  3D & 13.8 & 4.7 & 2.0  \\  \midrule
FreeMask   &  2D  & 15.3 & 6.6 & 2.9  \\
Ours    &  2D  & 15.6 & 7.2 & 3.6   \\ \midrule
FreeMask   &  both  & 17.9 & 7.5 & 3.7  \\
Ours    &  both  & 19.9 & 10.0 &  5.9 \\ \bottomrule
\end{tabular}
\caption{We compare pseudo mask generation from 3D-only features (3D), color-only features (2D), and both color and geometry (both) signal, as well as with pseudo annotation generation algorithm FreeMask.
We compare the quality of the initial pseudo mask dataset using our masked \textit{NCut} algorithm and the adaptation of FreeMask~\cite{wang2022freesolo} to 3D. We see that the normalized cut-based method is superior for both modalities.}
\label{tab:freemask_vs_ncut}
\end{table}

We also note here that while there is a difference in the initial pseudo mask qualities for the different methods, the downstream performance is way more significant. This can explained by the nature of the pseudo masks. \textit{NCut} provides a clean and sparse set of annotation, which is easy to densify for following iterations. On the other hand, the more dense, but noisy FreeMask predictions remain in the training for the duration of the whole training, hindering the performance of the self-trained model with noisy supervision. 


% WARNING: do not forget to delete the supplementary pages from your submission 
% \clearpage
%\setcounter{page}{1}
\maketitlesupplementary
The supplementary material for our work  \textit{SC-VAE: Sparse Coding-based Variational Autoencoder with Learned ISTA} is structured as follows:
%Sec. \ref{section1_s} provides the detailed information of the encoder and decode architecture of the SC-VAE model. 
%Sec. \ref{section2_s} shows the visualization of the dictionary atoms.
%Sec. \ref{section3_s} shows the training loss on the ImageNet dataset with different number of downsampling (upsampling) blocks ($d$) in the encoder (decoder) of the SC-VAE model.
%Sec. \ref{section4_s} shows the visualization results of an unofficial implementation of VIT-VQGAN \cite{yu2021vector}. 
%Sec. \ref{section5_s} shows additional manipulation and interpolation results on FFHQ dataset. 
%Sec. \ref{section6_s} shows additional image patches clustering results on FFHQ and ImageNet datasets. 
%Sec. \ref{section7_s} shows additional unsupervised image segmentation results.
Section \ref{section1_s} details the encoder and decoder architecture of the SC-VAE model. In Section \ref{section2_s}, the dictionary atoms are visualized. In Section \ref{section3_s}, we provide the training losses on the ImageNet dataset when varying the number of downsampling (upsampling) blocks ($d$) in the encoder (decoder) of the SC-VAE model. In Section \ref{section4_s}, the visualized reconstruction results of an unofficial implementation of VIT-VQGAN \cite{yu2021vector} are provided. 
We provide  additional manipulation and interpolation results on the FFHQ dataset in Section \ref{section5_s}, while  additional clustering results of image patches on both FFHQ and ImageNet  are provided in Section \ref{section6_s}. Supplementary unsupervised image segmentation results are given in Section \ref{section7_s}.

%Additional results on image patches clustering and unsupervised image segmentation on FFHQ and ImageNet datasets are then presented in Sec. 2 and Sec. 3, respectively.
\setcounter{section}{0}

\section{The Encoder and Decoder Architecture of SC-VAE} \label{section1_s}
The SC-VAE model's encoder and decoder architecture mirrors that of VQGAN \cite{esser2021taming}. Details about the architecture are provided in Table \ref{figure:encoder_decoder}.
%The encoder and decoder architecture in the SC-VAE model are the same as the architecture used in VQGAN \cite{esser2021taming}, which is described in Table \ref{figure:encoder_decoder}. 
$H$, $W$ and $C$ denote the height, width
and the number of channels of an input image, respectively.
$C'$ and $C''$ represent the number of channels of the feature maps that are produced as outputs by the intermediate layers of the encoder and decode network.
In our experiment, $C'$ and $C''$ were set to $128$ and $512$, respectively. $n$ denotes the number of dimensions of each latent representation, which was set to $256$.
The variable $d$ represents the number of blocks used for downsampling and upsampling. Therefore, we can calculate the height ($h$) and width ($w$) of the encoder's output feature maps by dividing the height ($H$) and width ($W$) of input images by $2$ raised to the power of $d$.

\begin{table}[thbp!]
\centering
\caption{High-level architecture of the encoder and decoder of the SC-VAE model. $H$, $W$, and $C$ refer to the height, width, and the number of channels of an input image. 
$C'$ and $C''$ represent the number of channels of the feature maps from intermediate layers in the encoder and decoder networks. $n$ denotes the number of dimensions of each latent representation, while $d$ represents the number of downsampling (upsampling) blocks. Note that $h=\frac{H}{2^{d}}$, $w=\frac{W}{2^d}$.} 
\resizebox{1\linewidth}{!}{%
\begin{tabular}{c|c}
  \toprule
   &  $x\in \mathbb{R}^{H\times W\times C} $\\
   &  2D Convolution $\rightarrow \mathbb{R}^{H\times W\times C'}$\\
   &  $d \times$\{Residual Block, Downsample Block\} $\rightarrow \mathbb{R}^{h\times w\times C''}$\\
   &  Residual Block $\rightarrow \mathbb{R}^{h\times w\times C''}$\\
  Encoder &  Non-Local Block $\rightarrow \mathbb{R}^{h\times w\times C''}$\\
   &  Residual Block $\rightarrow \mathbb{R}^{h\times w\times C''}$\\
   &  Group Normalization \cite{wu2018group} $\rightarrow \mathbb{R}^{h\times w\times C''}$ \\
   &  Swish Activation Function \cite{ramachandran2017searching} $\rightarrow \mathbb{R}^{h\times w\times C''}$\\
   &  2D Convolution $\rightarrow E(x) \in \mathbb{R}^{h\times w\times n}$\\
  \midrule
   & $\tilde{E}(x)\in \mathbb{R}^{h\times w\times n} $  \\
   &  2D Convolution $\rightarrow \mathbb{R}^{h\times w\times C''}$  \\
   &  Residual Block $\rightarrow \mathbb{R}^{h\times w\times C''}$\\
    & Non-Local Block $\rightarrow \mathbb{R}^{h\times w\times C''}$\\
  Decoder & Residual Block $\rightarrow \mathbb{R}^{h\times w\times C''}$\\
    & $d\times$\{Residual Block, Upsample Block\} $\rightarrow \mathbb{R}^{H\times W\times C'}$\\
   & Group Normalization \cite{wu2018group} $\rightarrow \mathbb{R}^{H\times W\times C'}$\\
    & Swish  Activation Function \cite{ramachandran2017searching}
    $\rightarrow \mathbb{R}^{H\times W\times C'}$\\
    & 2D Convolution $\rightarrow G(\tilde{E}(x)) \in \mathbb{R}^{H\times W\times C}$\\
  \bottomrule
\end{tabular}}
\label{figure:encoder_decoder}
\end{table}

\section{Visualization of Dictionary Atoms}
\label{section2_s}
Figure \ref{figure:dictionary_visualization} demonstrates the $512$ columns (atoms) of the pre-determined Discrete Cosine Transform (DCT) dictionary. Each atom is of dimension $256$, which corresponds to the size of $16 \times 16$ images when shaped.
%We reshape all atoms into an image with a $16\times 16$ resolution.

\begin{figure}[tbp]
\centering
\includegraphics[width=8cm]{./Figures/visualization_of_dictionary.png}
\caption{$512$ atoms of the Discrete Cosine Transform (DCT) dictionary. All atoms were reshaped into a $16 \times 16$ image.}
\label{figure:dictionary_visualization}
\end{figure}

\section{Training Losses}  \label{section3_s}
%Training losses of inherent noises around the 140th epoch under different auxiliary dataset sizes (K)
Figures \ref{figure:TLImagenet32x32}, \ref{figure:TLImagenet16x16}, \ref{figure:TLImagenet4x4} and \ref{figure:TLImagenet1x1} show  the training losses over $120,000$ training steps on the
ImageNet dataset.
The number of downsampling (upsampling) blocks ($d$) in the encoder (decoder) of the SC-VAE model are $3, 4, 6$ and $8$, respectively.
%with the number of downsampling (upsampling) blocks ($d=3,4,6$ and $8$, respectively) in the encoder (decoder) of the SC-VAE model. 
%As is shown in these figures, the LISTA networks of the SC-VAE models converge to a fixed point no matter which downsampling (upsampling) block $d$ is used. However, SC-VAE suffer from image reconstruction when increasing $d$.
As depicted in these figures, the LISTA networks within the SC-VAE models consistently converge to a stable point regardless of the chosen downsampling (upsampling) block $d$. However, increasing $d$ leads to worse image reconstructions ($\mathcal{L}_{rec}$) in SC-VAE.

\begin{figure}[tbp]
\centering
\includegraphics[width=7.5cm]{./Figures/Imagenet32x32.png}
\caption{The training losses over $120,000$ training steps on the ImageNet dataset. The number of  downsampling (upsampling) blocks ($d$) in the encoder (decoder) of the SC-VAE model was set to $3$ and the height ($h$) and width ($w$) of latent representations were $32$. (a) Total loss $\mathcal{L}_{SC-VAE}$. (b) Image reconstruction loss $\mathcal{L}_{rec}$. (c)The mean of latent representations reconstruction loss $\frac{1}{hw}\mathcal{L}_{latent}$.}
\label{figure:TLImagenet32x32}
\end{figure}

\begin{figure}[tbp]
\centering
\includegraphics[width=7.5cm]{./Figures/Imagenet16x16.png}
\caption{The training losses over $120,000$ training steps on the ImageNet dataset. The number of  downsampling (upsampling) blocks ($d$) in the encoder (decoder) of the SC-VAE model was set to $4$ and the height ($h$) and width ($w$) of latent representations were $16$. (a) Total loss $\mathcal{L}_{SC-VAE}$. (b) Image reconstruction loss $\mathcal{L}_{rec}$. (c) The mean of latent representations reconstruction loss $\frac{1}{hw}\mathcal{L}_{latent}$.}
\label{figure:TLImagenet16x16}
\end{figure}

\begin{figure}[tbp]
\centering
\includegraphics[width=7.5cm]{./Figures/Imagenet4x4.png}
\caption{The training losses over $120,000$ training steps on the ImageNet dataset. The number of  downsampling (upsampling) blocks ($d$) in the encoder (decoder) of the SC-VAE model was set to $6$ and the height ($h$) and width ($w$) of latent representations were $4$. (a) Total loss $\mathcal{L}_{SC-VAE}$. (b) Image reconstruction loss $\mathcal{L}_{rec}$. (c) The mean of latent representations reconstruction loss $\frac{1}{hw}\mathcal{L}_{latent}$.}
\label{figure:TLImagenet4x4}
\end{figure}

\begin{figure}[tbp]
\centering
\includegraphics[width=7.5cm]{./Figures/Imagenet1x1.png}
\caption{The training losses over $120,000$ training steps on the ImageNet dataset. The number of  downsampling (upsampling) blocks ($d$) in the encoder (decoder) of the SC-VAE model was set to $8$ and the height ($h$) and width ($w$) of latent representations were $1$. (a) Total loss $\mathcal{L}_{SC-VAE}$. (b) Image reconstruction loss $\mathcal{L}_{rec}$. (c) The mean of latent representations reconstruction loss $\frac{1}{hw}\mathcal{L}_{latent}$.}
\label{figure:TLImagenet1x1}
\end{figure}

%\noindent
%\noindent\textbf{Learnbale ISTA.} The architecture of our Learnable ISTA network is shown in Table 2.
%\noindent\textbf{Attention Network for $\alpha$ Estimation.}  Our neural network architecture follows the backbone of PixelCNN++ [52], which is a U-Net [48] based on a Wide ResNet [72]. We replaced weight normalization [49] with group normalization [66] to make the implementation simpler. Our 32 × 32 models use four feature map resolutions (32 × 32 to 4 × 4), and our 256 × 256 models use six. All models have two convolutional residual blocks per resolution level and self-attention blocks at the 16 × 16 resolution between the convolutional blocks [6].






% \begin{table*}[!htbp]
% \centering
% \caption{High-level architecture of the Learnable ISTA of our SC-VAE. Note that $k$ is the number of the unfolded ISTA block.} 
% \begin{tabular}{c}
%   \toprule
%   Learnable ISTA \\
%   \midrule
%   $E(x)\in \mathbb{R}^{h\times w \times n} $ \\
%   Filter Matrix $\rightarrow \mathbb{R}^{h\times w\times K}$ \\
%   $k\times$\{Shrinkage Function, Mutual Inhibition Matrix, Addition Operator\} $\rightarrow \mathbb{R}^{h\times w\times K}$\\
%   Shrinkage function$\rightarrow Z\in \mathbb{R}^{h\times w\times K}$\\
%   \bottomrule
% \end{tabular}
% \end{table*}

\section{Image Reconstruction}  \label{section4_s}
Reconstruction results from unofficial implementation\footnote{https://github.com/thuanz123/enhancing-transformers} of VIT-VQGAN \cite{yu2021vector} are presented in Figure \ref{figure:ViT-VQGAN_Visualization}.
%Figures \ref{figure:ViT-VQGAN_Visualization} shows visualizations from unofficial implementation\footnote{https://github.com/thuanz123/enhancing-transformers} of VIT-VQGAN \cite{yu2021vector}. 
VIT-VQGAN \cite{yu2021vector} achieved visually appealing results. However, similar to VQ-GAN \cite{esser2021taming} and RQ-VAE \cite{lee2022autoregressive}, it faced challenges in accurately reconstructing intricate details and complex patterns.
%as VQ-GAN\cite{esser2021taming} and RQ-VAE\cite{lee2022autoregressive}. 
Additionally, its generalization performance was inferior to that of our model.

\begin{figure}[tbp]
\centering
\includegraphics[width=7.0cm]{./Figures/ViT-VQGAN_Visualization.png}
\caption{Image reconstructions from an unofficial implementation of VIT-VQGAN \cite{yu2021vector} and the SC-VAE models trained
on ImageNet dataset. Original images in the top two rows are
from the validation set of ImageNet dataset. Two external images are shown in the last two rows to demonstrate the generalizability of different methods. The numbers denote the shape of
latent codes and the learned codebook (dictionary) size, respectively.
SC-VAE achieved improved image reconstruction compared to VIT-VQGAN \cite{yu2021vector}. Zoom in to see the details in the red square area.}
\label{figure:ViT-VQGAN_Visualization}
\end{figure}

\section{Image Generation}  \label{section5_s}
Additional interpolation and manipulation results can be found in Figures \ref{figure:image_interpolation_supple} and \ref{figure:image_manipulation_supple}, respectively.

\begin{figure}[tbp]
\centering
\includegraphics[width=7.0cm]{./Figures/image_interpolation_supple.png}
\caption{Interpolation between the sparse code vectors of two samples from the SC-VAE$^{\dag}$ model trained on FFHQ.}
\label{figure:image_interpolation_supple}
\end{figure}

\begin{figure*}[tbp]
\centering
\includegraphics[width=14.5cm]{./Figures/image_manipulation_supple4.png}
\caption{Manipulating sparse code vectors on FFHQ. 
Each block contains five seed images used to infer the latent sparse code vector in the SC-VAE$^{\dag}$ model.
The disentangled attributes associated with the $i$-th component of a sparse code vector $z$ and a traversal range are shown on the top of each block.}
\label{figure:image_manipulation_supple}
\end{figure*}



% \begin{figure}[tbp]
% \centering
% \includegraphics[width=8cm]{./Figures/IG_Age.png}
% \caption{IG-Age.}
% \label{figure:IG_Age}
% \end{figure}

% \begin{figure}[tbp]
% \centering
% \includegraphics[width=8cm]{./Figures/IG_sunglasses.png}
% \caption{IG-sunglasses.}
% \label{figure:IG_sunglasses}
% \end{figure}

% \begin{figure}[tbp]
% \centering
% \includegraphics[width=8cm]{./Figures/IG_Azimuth.png}
% \caption{IG-azimuth.}
% \label{figure:IG_azimuth}
% \end{figure}

% \begin{figure}[tbp]
% \centering
% \includegraphics[width=8cm]{./Figures/IG_Fringe.png}
% \caption{IG-Fringe.}
% \label{figure:IG_Fringe}
% \end{figure}


% \begin{figure}[tbp]
% \centering
% \includegraphics[width=8cm]{./Figures/IG_skin color.png}
% \caption{IG-skin color.}
% \label{figure:IG_skin color}
% \end{figure}


% \begin{figure}[tbp]
% \centering
% \includegraphics[width=8cm]{./Figures/image_interpolation_supple.png}
% \caption{Interpolation in the latent space between two samples from a model trained on FFHQ.}
% \label{figure:interpolation}
% \end{figure}

\section{Image Patches Clustering}  \label{section6_s}
%Figures \ref{figure:s1} and \ref{figure:s2} exhibit more image patches clustering outcomes for the FFHQ and ImageNet datasets, respectively. 
Figures \ref{figure:s1} and \ref{figure:s2} showcase additional qualitative results of image patches clustering on FFHQ and ImageNet datasets, respectively.
These results were obtained utilizing the pre-trained SC-VAE$^\curlyvee$ model specific to each dataset with a downsampling block $d=4$.
\begin{figure*}[h!]
\centering
\includegraphics[width=16cm]{./Figures/patches_cluster_ffhq_supple_50.png}
\caption{50 randomly selected image patch clusters from the validation set of the FFHQ dataset generated by clustering the learned sparse code vectors of the pre-trained SC-VAE$^\curlyvee$ model
using the K-means algorithm. Each row represents one cluster. Image patches with similar patterns were grouped together.}
\label{figure:s1}
\end{figure*}

\begin{figure*}[h!]
\centering
\includegraphics[width=16cm]{./Figures/imagenet_cluster_patches_V3.png}
\caption{50 randomly selected image patch clusters from the validation set of the ImageNet dataset generated by clustering the learned sparse code vectors of the pre-trained SC-VAE$^\curlyvee$ model
using the K-means algorithm. Each row represents one cluster. Image patches with similar patterns were grouped together.}
\label{figure:s2}
\end{figure*}

% \begin{figure*}[h!]
% \centering
% \includegraphics[width=16cm]{./Figures/segmentation_ffhq_supple3.png}
% \caption{FFHQ.}
% \label{figure:5}
% \end{figure*}

\section{Unsupervised Image Segmentation} \label{section7_s}
\subsection{Qualitative Analysis on FFHQ and ImageNet}
%Figures \ref{figure:s3} and \ref{figure:s4} contain additional qualitative unsupervised image segmentation results on FFHQ and ImageNet datasets, respectively. 
%We utilized two SCVAE models that were pre-trained on the training set of the FFHQ and ImageNet dataset, respectively. These models had a downsampling block of $d = 3$ and a sparsity penalty of $\lambda = 2$. 
%We employed two SC-VAE$^\curlywedge$ models that had been pre-trained on the training sets of the FFHQ and ImageNet datasets, respectively. These models had a downsampling block $d=3$.
Additional qualitative unsupervised image segmentation results on the FFHQ and ImageNet datasets can be found in Figures \ref{figure:s3} and \ref{figure:s4}, respectively. We utilized two SC-VAE$^\curlywedge$ models pre-trained on the training sets of FFHQ and ImageNet, each employing a downsampling block $d=3$.
\subsection{Quantitative comparisons to prior work}
%Figure \ref{figure:Flower_CUB} shows more qualitative results on  Flowers \cite{nilsback2008automated} and Caltech-UCSD Birds-200-2011 (CUB) \cite{WahCUB_200_2011}. Flowers \cite{nilsback2008automated} consists of $8,189$ images of $102$ classes of flowers, with segmentation masks obtained by an automated algorithm developed specifically for segmenting flowers in color photographs \cite{nilsback2007delving}. CUB \cite{WahCUB_200_2011} consists of $11,788$ images of $200$ classes of birds and segmentation masks. Flowers and CUB contain $1,020$ and $1,000$ test images, respectively.
%Figure \ref{figure:Flower_CUB} shows more qualitative results on  Flowers \cite{nilsback2008automated} and Caltech-UCSD Birds-200-2011 (CUB) \cite{WahCUB_200_2011} datasets.
Figure \ref{figure:Flower_CUB} displays additional qualitative results from the Flowers \cite{nilsback2008automated} and Caltech-UCSD Birds-200-2011 (CUB) \cite{WahCUB_200_2011} datasets.\\
\subsubsection{Evaluation Metrics}
\textbf{Intersection of Union (IoU).} %The IoU score measures the overlap of two regions A and B by calculating the ratio of intersection over union, according to
The IoU score quantifies the overlap between two regions. This is achieved by evaluating the ratio of their intersection to their union.
\begin{align}
    \textup{IoU}(A, B) = \frac{|A\cap B|}{|A\cup B|}. \nonumber
\end{align}
%where we use the inferred mask and ground-truth mask as $A$ and $B$ respectively for evaluation.\\
$A$ denotes the ground-truth mask, while $B$ denotes the inferred mask.\\
%as $B$ for assessment purposes.\\
\textbf{DICE score.} Similarly, the DICE score is defined as:
\begin{align}
    \textup{Dice}(A, B) = \frac{2|A\cap B|}{|A|+ |B|}.\nonumber
\end{align}
\noindent
Higher is better for both scores.\\
\subsubsection{Dataset Details}
\textbf{Flowers.} The Flowers \cite{nilsback2008automated} dataset consists of $8,189$ images across $102$ different flower classes. Additionally, it includes segmentation masks generated by an automated algorithm designed explicitly for color photograph flower segmentation \cite{nilsback2007delving}. 
%The images in this dataset have large scale, pose and light variations.\\
The dataset contains images that exhibit substantial variations in scale, pose, and lighting.
Flowers \cite{nilsback2008automated} contains $1,020$ test images.\\
\textbf{CUB.} The CUB \cite{WahCUB_200_2011} dataset contains $11,788$ images covering $200$ bird classes, along with their segmentation masks. 
%Each image is further annotated with $15$ part locations and $1$ bounding box. We use theprovided bounding box to extract a center square from the image, and scale it to $128\times 128$ pixels.
Every image comes with annotations for $15$ part locations, $312$ binary attributes, and $1$ bounding box. We utilized the given bounding box to crop a central square from the image. The CUB dataset includes $1,000$ test images.\\
\textbf{ISIC-2016.} The ISIC-2016 \cite{gutman2016skin} dataset is a public challenge dataset dedicated to Skin Lesion Analysis for Melanoma Detection. Derived from the extensive International Skin Imaging Collaboration (ISIC) archive, it represents a significant collection of meticulously curated dermoscopic images of skin lesions. Within this challenge, a subset of $900$ images is designated as training data, while $379$ images serve as testing data, aiming to provide representative samples for analysis.
%The ISIC-2016 \cite{gutman2016skin} dataset is a public challenge dataset of Skin Lesion Analysis Towards Melanoma Detection released with ISBI 2016. This dataset is based on the International Skin Imaging Collaboration (ISIC) Archive, which is the largest publicly available collection of quality controlled dermoscopic images of skin lesions. The challenge employs a subset of representative images with $900$ images as training data and $379$ images as testing data.

%For all experiments, we resized the input images into a resolution of $256\times 256$ and  generated a $32\times 32$ binary mask for each image utilizing the pre-trained SC-VAE$^\curlywedge$ on ImageNet dataset, a spectral clustering algorithm and boundary connectivity information. The inferred binary mask and ground truth mask were resized to $128\times 128$ to calculate the IoU and DICE scores.
For our experiments, we resized the input images into a resolution of $256\times 256$.
Subsequently, we generated a binary mask of size $32\times 32$ per image by employing the pre-trained SC-VAE$^\curlywedge$ on the ImageNet dataset, along with a spectral clustering algorithm and boundary connectivity information \cite{zhu2014saliency}. To compute the IoU and DICE scores, both the inferred binary mask and the ground truth mask were resized to $128\times 128$.
%\subsubsection{Baseline Methods}
\label{section3}
\begin{figure*}[h!]
\centering
\includegraphics[width=16cm]{./Figures/segmen_ffhq_supple3.png}
%\caption{Additional unsupervised image segmentation results. Images are from the validation set of the FFHQ dataset.}
\caption{Additional unsupervised image segmentation results. These results were generated by grouping sparse code vectors into $5$ categories per image, utilizing the pre-trained SC-VAE$^{\curlywedge}$ model and the K-means algorithm. Images are from the validation set of the FFHQ dataset.}
\label{figure:s3}
\end{figure*}

\begin{figure*}[h!]
\centering
\includegraphics[width=16cm]{./Figures/segmentation_imagenet_supple.png}
%\caption{Additional unsupervised image segmentation results by applying K-means algorithm to cluster sparse code vectors per image into $5$ categories using the SC-VAE$^{\curlywedge}$ model. Images are from the validation set of the ImageNet dataset.}
\caption{Additional unsupervised image segmentation results. These results were generated by grouping sparse code vectors into $5$ categories per image, utilizing the pre-trained SC-VAE$^{\curlywedge}$ model and the K-means algorithm. Images are from the validation set of the ImageNet dataset.}
\label{figure:s4}
\end{figure*}

\begin{figure*}[tbp]
\centering
\includegraphics[width=18cm]{./Figures/flower_cub_isic2016_supple2.png}
\caption{Additional unsupervised image segmentation results on Flowers \cite{nilsback2008automated} (\textit{Left Panel}), CUB \cite{WahCUB_200_2011} (\textit{Middle Panel}) and ISIC-2016 \cite{gutman2016skin} (\textit{Right Panel}). (a) input image. (b) ground truth mask. (c) and (e) segmentation results by clustering sparse code vectors per image into $2$ or $3$ classes using a spectral clustering algorithm. (d) and (f) boundary connectivity information \cite{zhu2014saliency}
was used to decide the foreground and background.}
\label{figure:Flower_CUB}
\end{figure*}

\clearpage
\clearpage
{
   \small
   \bibliographystyle{ieee_fullname}
   \bibliography{egpaper_arxiv_V2}
}

\end{document}
