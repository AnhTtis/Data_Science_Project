\begin{table*}[!ht]
\centering
\begin{tabular}{lccccc}\toprule
& \multicolumn{2}{c}{Cluster Quality} & \multicolumn{3}{c}{ScanNet Precisions}
\\\cmidrule(lr){2-3}\cmidrule(lr){4-6}
           & Self-Train  & Use Images   & AP@25  & AP@50 & mAP \\\midrule
HDBSCAN \cite{mcinnes2017accelerated_hdbscan}    & \xmark  & \xmark  & 32.1 & 5.5 & 1.6 \\
3DUIS \cite{nunes2022unsupervised}    & \xmark  & \xmark & 30.5 & 7.3 & 2.3 \\
Felzenswalb \cite{felzenszwalb2004efficient}    & \xmark  & \xmark  & 38.9 & 12.7 & 5.0 \\
Masked NCut    & \xmark & \xmark  & 19.9 & 10.0 & 5.9 \\
Ours geom. only    & \cmark & \xmark & 52.5 & 26.9 & 13.8 \\
Ours    & \cmark  & \cmark & \textbf{58.6} & \textbf{32.0} & \textbf{16.0} \\ \bottomrule
\end{tabular}
\caption{Unsupervised, class agnostic instance segmentation methods on the indoor ScanNet \cite{dai2017scannet} dataset validation split. We can see that while previous point- and density based clustering methods perform poorly, and for cluttered indoor scenarios the assumption for free instances from ground separation will not hold we have to rely on a more high-level solution for iterative self training. While Masked NCut provides sparse annotations, their quality still enables downstream model training for higher quality instance predictions.}
\label{tab:scannet_results}
\end{table*}