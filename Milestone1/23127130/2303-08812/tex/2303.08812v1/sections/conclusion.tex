\section{Conclusions}
\label{sec:conclusion}

In this article, we have demonstrated the application of an SSCNN for event reconstruction on neutrino telescopes.
We have shown that these networks are capable of maintaining competitive performance on the tasks of energy and angular reconstruction while running on the $\mu$s time scale.
The speedup enables the SSCNN to process events at a rate well above that of the current neutrino telescopes trigger rate, which is expected to be representative of other neutrino telescopes currently operating or under construction, such as IceCube, KM3NeT, P-ONE, and Baikal-GVD.
Reaching this threshold makes the SSCNN a feasible option for online reconstruction at the detector site where resources are limited and where first guesses of the energy and direction of the neutrino are made.
As discussed in the introduction, this can have a substantial impact on current real-time analyses, where our first estimations can also be utilized in an alert-sending system, which will notify collaborators if the detector sees an interesting event. 
Additionally, these reconstructions can serve as seeds for more time-consuming reconstructions, and thus improving these first estimations will be beneficial to all subsequent analyses.