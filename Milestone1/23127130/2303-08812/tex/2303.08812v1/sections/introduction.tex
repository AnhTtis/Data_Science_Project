\section{Introduction}
\label{sec:intro}

In recent decades, the deployment of innovative experiments and the construction of advanced detectors have driven substantial advances in particle physics.
These detectors can probe previously unreachable parts of phase space, but in order to do so, they must generate an overwhelming amount of data that requires substantial computational resources to analyze~\cite{Acosta:2022sax}.
Machine learning (ML) has proven to be a valuable tool in enhancing the capabilities of current data sets, both improving the accuracy of reconstructions and reducing computational run-time~\cite{Feickert:2021ajf}.
In addition to improving offline data analysis, machine learning may enable improved data readout with accurate reconstructions that can operate at the detector trigger-level~\cite{Wang:2020fjr, Duarte:2018ite, Summers:2020xiy, Iiyama:2020wap, Coelho:2020zfu, Loncar:2020hqp, Duarte:2019fta, markus_atkinson_2020_4001022}.
% ML algorithms possess a fast decision-making ability and will play a key role in shaping the future of high-energy experiments by transforming the data they record.
This article presents a method to significantly shorten the execution time of ML reconstructions on high-energy neutrinos.

\begin{figure}[bht!]
\centering
\includegraphics[width=0.47\textwidth]{figures/event_rates.png}
\caption{\textbf{\textit{Event rates of triggers in different neutrino telescopes~\cite{IceCube_triggers, baikaltdr, km3net_trigger} compared to the run-times of various reconstruction methods.}} Sparse submanifold CNNs and their performance are detailed in this article. The CNN and maximum likelihood method run-times are taken from~\cite{Mirco:2017}. Notably, sparse submanifold CNNs can process events well above standard trigger rates in both ice- and water-based experiments.}
\label{fig:event_rates}
\end{figure}

% The IceCube Neutrino Observatory~\cite{IceCube:2008qbc, IceCube:2016zyt}---an array of 5,160 individual detection units called DOMs deployed in the glacial ice at the geographic south pole---observes high-energy neutrinos at a rate of approximately one every five minutes; however, for each neutrino that triggers the detector readout, one million cosmic-ray-muon-induced events are detected, corresponding to a rate of approximately 3~kHz. 

Typically, modern neutrino telescopes are comprised of thousands of optical modules (OMs) deployed in a transparent medium, such as water or ice.
While high-energy neutrinos are observed in these detectors at intervals of a few minutes, they are largely dominated by cosmic-ray-muon induced events, which results in a substantial amount of events which triggers the detector readout thousands of times every second.

In this article, we introduce a reconstruction method using a sparse submanifold CNN (SSCNN), which is fast enough to reconstruct events at this kHz-scale trigger-level rate.
At this level, experiments use simple reconstruction algorithms to be able to handle the event rate.
For example, the IceCube Neutrino Observatory currently uses \texttt{Linefit}~\cite{linefit} for trigger-level angular reconstruction.
This online reconstruction which runs on all recorded events is then used for cuts and as a seed for more advanced likelihood-based reconstruction~\cite{Bradascio:2019eub}. Although the likelihood-based reconstruction is able to achieve a significantly better median angular resolution in muon-induced tracks than ML-based reconstructions, they are generally slower and prone to failure due to mismodeling or improper seeding.

% Currently, IceCube uses \texttt{Linefit}~\cite{linefit} for trigger-level angular reconstruction.
% After cutting on the \texttt{LineFit} result to reduce the data rate to around 1~Hz, a more sophisticated, likelihood-based reconstruction is used~\cite{Bradascio:2019eub}.
% Although the likelihood-based reconstruction is able to achieve a significantly better median angular resolution in muon-induced tracks than ML-based reconstructions, it is prone to failure due to mismodeling or improper seeding.
Our SSCNN achieves better angular resolutions than methods such as \texttt{Linefit} while requiring a comparable run-time, enabling improved trigger-level cuts and serving as a better seed for the likelihood-based reconstruction. 
Fig.~\ref{fig:event_rates} summarizes typical event rates found in neutrino telescopes and compares these to the execution rate of various reconstructions.
% Fig.~\ref{fig:event_rates} summarizes the IceCube data rate at different processing levels and compares these to the execution rate of various reconstructions.

The ability to quickly reconstruct the direction of muons is crucial for two reasons.
First, the removal of background in neutrino telescopes are reliant on cuts which depend on the angle in which the interaction occurs in the detector. With a reliable, fast directional reconstruction, cosmic-ray-muon induced events and other backgrounds can be separated from neutrino signal. 
Moreover, a rapid reconstruction method could serve as part of an alert system that notifies researchers of events that are highly likely to be astrophysical neutrinos.
% The ability to quickly reconstruct the direction of muons travelling through IceCube is crucial for two reasons.
% First, most of the 3~kHz background events are downward-going muons produced in the cosmic-ray air showers.
% With a reliable, fast directional reconstruction, these can be separated from the neutrinos, which are able to cross the Earth and produce upward-going muons. 
% Additionally, IceCube publishes events with a high likelihood of being astrophysical neutrinos as Astronomer Telegrams~\cite{Rutledge:1998hc}.
In IceCube, the real-time follow up of such an event led to the observation of the first astrophysical neutrino source candidate, TXS 0506+056, by detecting a neutrino in coincidence with a gamma-ray flare~\cite{IceCube:2018cha}.
Our methodology will improve the capacity of neutrino telescopes to make prompt decisions, potentially enabling further discoveries.

While this article will concentrate on the implementation of SSCNN in IceCube, it should be noted that this general machine learning method is also applicable to water-based neutrino telescopes. The rest of this article is organized as follows.
In Sec.~\ref{sec:architecture} we motivate and introduce sparse submanifold convolutions; in Sec.~\ref{sec:events} we describe the data sets used for training and testing; in Sec.~\ref{sec:performance} we evaluate the performance of the network.
Finally, in Sec.~\ref{sec:conclusion} we conclude with some parting words. The code detailing our implementation of SSCNN has been made available at Ref.~\cite{GithubCode}.

