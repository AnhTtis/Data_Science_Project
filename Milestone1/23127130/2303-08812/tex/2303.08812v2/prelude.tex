\preprint{APS/123-QED}

\title{Trigger-Level Event Reconstruction for Neutrino Telescopes Using Sparse Submanifold Convolutional Neural Networks}

\author{Felix J. Yu}
%  \altaffiliation[Also at ]{Physics Department, XYZ University.}%Lines break automatically or can be forced with \\
\email{felixyu@g.harvard.edu}
\affiliation{Department of Physics and Laboratory for Particle Physics and Cosmology, Harvard University, Cambridge, MA 02138, US}

\author{Jeffrey Lazar}
\email{jlazar@icecube.wisc.edu}
\affiliation{Department of Physics and Laboratory for Particle Physics and Cosmology, Harvard University, Cambridge, MA 02138, US}
\affiliation{Department of Physics and Wisconsin IceCube Particle Astrophysics Center, \\
University of Wisconsin–Madison, Madison, WI 53706, USA}

\author{Carlos A. Arg\"{u}elles}
\email{carguelles@g.harvard.edu}
%  \homepage{http://www.Second.institution.edu/~Charlie.Author}
\affiliation{Department of Physics and Laboratory for Particle Physics and Cosmology, Harvard University, Cambridge, MA 02138, US}

\date{\today}% It is always \today, today,
             %  but any date may be explicitly specified

\begin{abstract}
Convolutional neural networks (CNNs) have seen extensive applications in scientific data analysis, including in neutrino telescopes. However, the data from these experiments present numerous challenges to CNNs, such as non-regular geometry, sparsity, and high dimensionality. Consequently, CNNs are highly inefficient on neutrino telescope data, and require significant pre-processing that results in information loss. We propose sparse submanifold convolutions (SSCNNs) as a solution to these issues and show that the SSCNN event reconstruction performance is comparable to or better than traditional and machine learning algorithms. Additionally, our SSCNN runs approximately 16 times faster than a traditional CNN on a GPU. As a result of this speedup, it is expected to be capable of handling the trigger-level event rate of IceCube-scale neutrino telescopes.
These networks could be used to improve the first estimation of the neutrino energy and direction to seed more advanced reconstructions, or to provide this information to an alert-sending system to quickly follow-up interesting events.

% \begin{description}
% \item[Usage]
% Secondary publications and information retrieval purposes.
% \item[Structure]
% You may use the \texttt{description} environment to structure your abstract;
% use the optional argument of the \verb+\item+ command to give the category of each item. 
% \end{description}
\end{abstract}

%\keywords{Suggested keywords}%Use showkeys class option if keyword
                              %display desired
\maketitle