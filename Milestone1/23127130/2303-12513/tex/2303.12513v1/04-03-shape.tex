\subsection{Shape Association Prediction}
\label{sec:shape}

\medskip \noindent \textbf{Task description.} Another salient visual feature of language is the association between concrete nouns or noun phrases and particular shapes. For example, the nouns \emph{wheel} and \emph{compass} have a circular association, while \emph{pyramid} and \emph{corn chip} have a triangular association. Shape associations have been studied in the psychological literature in contexts such as child language acquisition~\cite{yee2011function} and semantic memory representation~\cite{verdine2016shape}. Building on this line of research, we propose the task of \emph{shape association prediction} -- given a noun (or noun phrase) \W, identify the basic shape that is most associated with \W. Because the space of possible shapes is complex and difficult to categorize unambiguously, we restrict \W under consideration to nouns associated with a few basic shapes, as described below.

\medskip \noindent \textbf{Experimental details.} We construct the \textbf{ShapeIt} benchmark for shape associations\footnote{available in our code repository}. This contains 109 items total, each consisting of a noun or noun phrase along with the basic shape most associated with it from the set $\{rectangle, circle, triangle\}$. The benchmark was constructed by performing a user study requiring users to choose a shape associated with a given word, and selecting for only those words which were consistently classified by the users. Data collection methods used in constructing this benchmark are detailed in the supplementary material, along with further analysis of its contents. Probing methods used for this task are equivalent to the color association prediction task. Prompts used for probing include \emph{``A \SMASK shaped \W''} where \W is the shape associated word; the full list of prompts used is detailed in the supplementary material. We use both shape nouns (e.g. \emph{circle}) and associated adjectives (e.g. \emph{circular}) and report the highest accuracy achieved between these two settings.

\medskip \noindent \textbf{Evaluation metric.} We report categorical accuracy of predictions ($acc$) relative to the ground truth shape labels.


