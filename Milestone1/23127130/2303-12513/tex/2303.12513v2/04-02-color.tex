\subsection{Color Association Prediction}
\label{sec:color}

\medskip \noindent \textbf{Task description.} Some concepts are highly associated with particular colors---for example, the word \emph{banana} is highly associated with the color yellow, while a word like \emph{child} does not have a strong color association. These color associations have been widely studied in experimental psychology and neuroscience~\cite{bramao2011role,bannert2013decoding}. We propose a task of \emph{color association prediction} -- given a noun (or noun phrase) \W, identify the color with which \W is normally associated.

\medskip \noindent \textbf{Experimental details.} To probe our models for color associations, we use the MLM and SP methods described above. In particular, we conceive of this task as categorical prediction over a set of basic color words $C$. For example, using the prompt \emph{``A picture of a \SMASK \W''} where \W is the item being tested, our probing methods search for the most suitable color to place in the \SMASK slot. All prompts tested are listed in the supplementary material. For MLM probing, we predict the color $c \in C$ with the highest predicted probability in the \SMASK{} slot of the prompt. For SP, we predict $c^* = \arg \max s_c$ using similarity scores as defined above.

To test this method on our chosen text encoders, we use two datasets. The Color Terms Dataset (CTD)~\cite{bruni2012distributional} provides a list of 52 concrete words and their color. The Natural-Color Dataset (NCD) of fruit images~\cite{anwar2020image} is a colorization task containing images of 20 types of fruits and vegetables paired with colors. We use the provided list of fruits and colors as a fixed set of words with strong color associations, discarding the image data. For the latter, we filter objects with the color label \emph{purple} as this label contains multiple WordPiece tokens and thus is not directly comparable with MLM probing for models such as BERT. This results in 15 unique fruits and vegetables. For each model, we calculate color predictions using the probing methods described above out of the set: $\{$\emph{red, orange, yellow, green, blue, black, white, grey, brown}$\}$.

\medskip \noindent \textbf{Evaluation metrics.} We report categorical accuracy of predictions on the CTD and NCD datasets ($acc_{CTD}$ and $acc_{NCD}$) relative to the ground truth color labels.