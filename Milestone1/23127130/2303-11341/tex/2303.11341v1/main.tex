\documentclass{article}


\usepackage{PRIMEarxiv}

\usepackage[style=acmnumeric]{biblatex} % Replace 'my_custom_style' with the name of your .bbx file (without the .bbx extension)
\addbibresource{refs.bib}
% \usepackage{natbib}

\usepackage[utf8]{inputenc} % allow utf-8 input
\usepackage[T1]{fontenc}    % use 8-bit T1 fonts
\usepackage{hyperref}       % hyperlinks
\usepackage{url}            % simple URL typesetting
\usepackage{booktabs}       % professional-quality tables
\usepackage{amsfonts}       % blackboard math symbols
\usepackage{nicefrac}       % compact symbols for 1/2, etc.
\usepackage{microtype}      % microtypography
\usepackage{lipsum}
\usepackage{fancyhdr}       % header
\usepackage{graphicx}       % graphics




%Header
\pagestyle{fancy}
\thispagestyle{empty}
\rhead{ \textit{ }} 

% Update your Headers here
\fancyhead[LO]{Verifying Rules on Large-Scale NN Training via Compute Monitoring}
% \fancyhead[RE]{Firstauthor and Secondauthor} % Firstauthor et al. if more than 2 - must use \documentclass[twoside]{article}

  
%% Title
\title{A template for Arxiv Style
%%%% Cite as
%%%% Update your official citation here when published 
\thanks{\textit{\underline{Citation}}: 
\textbf{Authors. Title. Pages.... DOI:000000/11111.}} 
}

\author{
  Yonadav Shavit \\
  Harvard University \\
  \texttt{yonadav@g.harvard.edu}\\
  %% examples of more authors
  %% \AND
  %% Coauthor \\
  %% Affiliation \\
  %% Address \\
  %% \texttt{email} \\
  %% \And
  %% Coauthor \\
  %% Affiliation \\
  %% Address \\
  %% \texttt{email} \\
  %% \And
  %% Coauthor \\
  %% Affiliation \\
  %% Address \\
  %% \texttt{email} \\
}



\usepackage{xcolor}
\usepackage{siunitx}
\usepackage{amsmath}
\usepackage{amsthm}

\usepackage{array}




\newtheorem{theorem}{Theorem} 
\newtheorem{lemma}{Lemma}
\newtheorem{definition}{Definition}


\definecolor{DarkGreen}{rgb}{0.1,0.5,0.1}
\newcommand{\todo}[1]{\textcolor{DarkGreen}{[TODO: #1]}}
\newcommand{\rc}[1]{\textcolor{red}{[Rachel: #1]}}
\newcommand{\ys}[1]{\textcolor{blue}{[Yonadav: #1]}}
% \newcommand{\copied}[1]{\textcolor{orange}{[Copied: #1]}}
\newcommand{\future}[1]{}

% \newcommand{\ot}[1]{\textcolor{purple}{[Outline: #1]}}
\newcommand{\ot}[1]{\textcolor{purple}{[Outline: ]}}

\newcommand{\assumption}[1]{\textcolor{red}{ASSUMPTION: #1 \\}}
\newcommand{\oq}[1]{OPEN QUESTION: #1 \\}
\newcommand{\hparams}{q}
\newcommand{\monitoringperiod}{T_m}
\newcommand{\trainingperiod}{T}
\newcommand{\totalchips}{C}
\newcommand{\datasequence}{S(D)}
\newcommand{\rulecomputeflops}{H}
\newcommand{\modelweights}{W}
\newcommand{\totalmonitoringperiods}{k}
\newcommand{\computehours}{\computeflops}

\newcommand{\hashfunction}{\mathcal{H}}

%%
%% Submission ID.
%% Use this when submitting an article to a sponsored event. You'll
%% receive a unique submission ID from the organizers
%% of the event, and this ID should be used as the parameter to this command.
%%\acmSubmissionID{123-A56-BU3}

%%
%% For managing citations, it is recommended to use bibliography
%% files in BibTeX format.
%%
%% You can then either use BibTeX with the ACM-Reference-Format style,
%% or BibLaTeX with the acmnumeric or acmauthoryear sytles, that include
%% support for advanced citation of software artefact from the
%% biblatex-software package, also separately available on CTAN.
%%
%% Look at the sample-*-biblatex.tex files for templates showcasing
%% the biblatex styles.
%%

%%
%% The majority of ACM publications use numbered citations and
%% references.  The command \citestyle{authoryear} switches to the
%% "author year" style.
%%
%% If you are preparing content for an event
%% sponsored by ACM SIGGRAPH, you must use the "author year" style of
%% citations and references.
%% Uncommenting
%% the next command will enable that style.
%%\citestyle{acmauthoryear}

%%
%% end of the preamble, start of the body of the document source.

\title{What does it take to catch a Chinchilla?\\Verifying Rules on Large-Scale Neural Network Training via Compute Monitoring}

\begin{document}
%%
%% By default, the full list of authors will be used in the page
%% headers. Often, this list is too long, and will overlap
%% other information printed in the page headers. This command allows
%% the author to define a more concise list
%% of authors' names for this purpose.
%\renewcommand{\shortauthors}{Shavit}
\maketitle
\begin{abstract}
As advanced machine learning systems' capabilities begin to play a significant role in geopolitics and societal order, it may become imperative that (1) governments be able to enforce rules on the development of advanced ML systems within their borders, and (2) countries be able to verify each other's compliance with potential future international agreements on advanced ML development.
This work analyzes one mechanism to achieve this, by monitoring the computing hardware used for large-scale NN training.
The framework's primary goal is to provide governments high confidence that no actor uses large quantities of specialized ML chips to execute a training run in violation of agreed rules.
At the same time, the system does not curtail the use of consumer computing devices, and maintains the privacy and confidentiality of ML practitioners' models, data, and hyperparameters.
The system consists of interventions at three stages:
(1) using on-chip firmware to occasionally save snapshots of the the neural network weights stored in device memory, in a form that an inspector could later retrieve; (2) saving sufficient information about each training run to prove to inspectors the details of the training run that had resulted in the snapshotted weights; and (3) monitoring the chip supply chain to ensure that no actor can avoid discovery by amassing a large quantity of un-tracked chips.
The proposed design decomposes the ML training rule verification problem into a series of narrow technical challenges, including a new variant of the Proof-of-Learning problem [Jia et al. '21].
\end{abstract}



\section{Introduction}
Graph neural networks (GNNs) have undergone rapid development and become increasingly popular for learning graph data \cite{welling2016semi, velivckovic2017graph, xu2018powerful}.
GNNs are usually trained in an end-to-end manner while getting enough labeled data is arduously expensive and sometimes even impractical to access. This motivates some recent advances in pre-training GNNs ~\cite{hu2019strategies,Hu2020GPTGNNGP,Qiu2020GCCGC,Lu2021LearningTP}. 
The key insight of pre-training GNNs is to learn transferable knowledge from a collection of unlabeled graph data, hoping that the learned knowledge can be easily adapted to downstream tasks.
In view of the great success of pre-training in other fields like computer vision and natural language processing~\cite{devlin2018bert,he2020momentum},  graph pre-training is {highly expected} to be an effective means to improve downstream performance.


\begin{figure}[t]
    \centering
    {\includegraphics[width=1\columnwidth]{figure/motivation.pdf}}
    \caption{Comparison of {existing methods} and {proposed W2PGNN} to answer \emph{when to pre-train} GNNs.}    
    \label{fig:example}
\end{figure}

However, the intuition that graph pre-trained model would ideally benefit the downstream is far from the truth in the area of graph pre-training.
Instead, graph pre-trained models can lead to \emph{negative transfer} on many downstream tasks, especially when the graphs used for pre-training are not necessarily from the same domain as the {downstream} data~\cite{hu2019strategies, Qiu2020GCCGC}.
For example, the closed triangles ($\vcenter{\hbox{\includegraphics[width=2.4ex,height=2.4ex]{figure/s2.pdf}}}$) and open triangles  ($\vcenter{\hbox{\includegraphics[width=2.4ex,height=2.4ex]{figure/s1.pdf}}}$) might yield different interpretations in molecular networks (unstable vs. stable in terms of chemical property) from those in social networks (stable vs. unstable in terms of social relationship); such distinct or reversed semantics does not contribute to transferability, and even exacerbates the problem of negative transfer.


To avoid the negative transfer, recent efforts focus on  \emph{what to pre-train} and \emph{how to pre-train},  \emph{i.e.}, design/adopt graph pre-training models with a variety of self-supervised tasks to capture different patterns~\cite{Qiu2020GCCGC,you2020graph,Lu2021LearningTP} and fine-tuning strategies to enhance downstream performance~\cite{Hu2019PreTrainingGN,Han2021AdaptiveTL,Zhang2022FineTuningGN,Xia2022TowardsEA}.
However, there do exist some cases that no matter how advanced the pre-training/fine-tuning method is, the transferability from pre-training data to downstream data still cannot be guaranteed. This is because the underlying assumption of deep learning models is that the test data should share a similar distribution as the training data.
Therefore, it is a necessity to understand \emph{when to pre-train}, \emph{i.e.}, under what situations the ``graph pre-train and fine-tune'' paradigm should be adopted.

Towards the answer of when to pre-train GNNs, one straight-forward way illustrated in Figure~\ref{fig:example}(a) is to train and evaluate on all candidates of pre-training models and fine-tuning strategies, and then the resulting best downstream performance would tell us whether pre-training
% ``pre-train and fine-tune'' 
is a sensible choice. If there exist $l_1$ pre-training models and $l_2$ fine-tuning strategies,  such a process would be very costly as you should make $l_1 \times l_2$ ``pre-train and fine-tune'' attempts.
Another approach is to utilize graph metrics to measure the similarity between pre-training and downstream data, \emph{e.g.}, density, clustering coefficient and etc. However, it is a daunting task to enumerate all hand-engineered graph features or find the dominant features that influenced similarity.
Moreover, the graph metrics only measure the pair-wise similarity between two graphs, which cannot be directly and accurately applied to the practical scenario where pre-training data contains multiple graphs.


In this paper, we propose a W2PGNN framework to answer
\emph{\underline{w}hen \underline{to} \underline{p}re-train \underline{GNN}s from a graph data generation perspective}.
% aim to address the problem of when to pre-train GNNs 
The high-level idea is that instead of performing effortful graph pre-training/fine-tuning or making comparisons between the pre-training and downstream data, we study the complex generative mechanism from the pre-training data to the downstream data (Figure~\ref{fig:example}(b)).
We say that downstream data can benefit from pre-training data (\emph{i.e.}, has high feasibility of performing pre-training), 
if it can be generated with high probability by a graph generator that summarizes the topological characteristic of pre-training data.



The major challenge is how to obtain an appropriate graph generator, hoping that it not only inherits the transferable topological patterns of the pre-training data, but also is endowed with the ability to generate feasible downstream graphs.
To tackle the challenge, we propose to design a graph generator based on graphons.
We first fit the pre-training graphs into different graphons to construct a \emph{graphon basis}, where each graphon (\emph{i.e.}, element of the graphon basis) identifies a collection of graphs that share common transferable patterns. We then define a \emph{graph generator} as {a convex combination of elements in a graphon basis}, which serves as a comprehensive and representative summary of pre-training data.  All of these possible generators constitute the \emph{generator space}, from which graphs generated form the solution space for the downstream data that can benefit from pre-training.

Accordingly, the feasibility of performing pre-training can be measured as the highest probability of downstream data being generated from any graph generator in the generator space, which can be formulated as an optimization problem.
However, this problem is still difficult to solve due to the large search space of graphon basis. We propose to reduce the search space to three candidates of graphon basis, \emph{i.e.,} topological graphon basis, domain graphon basis, and integrated graphon basis, to mimic different {generation mechanisms} from pre-training to downstream data. Built upon the reduced search space, the feasibility can be approximated efficiently.





Our major contributions are concluded as follows:
\begin{itemize}[leftmargin=*,topsep=0pt]
\item \textbf{Problem and method.} To the best of our knowledge, we are the first work to study the problem of when to pre-train GNNs. We propose a W2PGNN framework
to answer the question from a data generation perspective, which tells us the feasibility of performing graph pre-training before conducting effortful pre-training and fine-tuning.




\item \textbf{Broad applications.}
W2PGNN provides several practical applications: (1) provide the application scope of a graph pre-trained model, (2) measure the feasibility of performing pre-training for a downstream data
and (3) choose the pre-training data so as to maximize downstream performance with limited resources.


\item \textbf{Theory and Experiment.} 
We theoretically and empirically justify the effectiveness of W2PGNN.
Extensive experiments {on real-world graph datasets from multiple domains} show that the proposed method can provide an accurate estimation of pre-training feasibility and the selected pre-training data can benefit the downstream performance.


\end{itemize}


\paragraph{Hypergeometric Sequences}

A hypergeometric sequence $\langle u_n\rangle_{n=0}^\infty$ is a sequence of rational numbers that satisfies 
a recurrence of the form \eqref{eq:rel}
where $f,g \in \Z[x]$ are polynomials, and  $f(x)$
has no non-negative integer zeros. 
By the latter requirement on~$f(x)$, the  recurrence~\eqref{eq:rel} 
uniquely defines an infinite sequence of rational numbers once the initial element $u_0$ 
is specified.

An instance of the Membership Problem for hypergeometric sequences consists of a recurrence~\eqref{eq:rel}, an initial value 
$u_0 \in \Q$, and a target $t \in \Q$.  
The problem asks to decide whether there exists \(n\in\N\) such that \(u_n = t\).
We say that such an instance is in \emph{standard form} if~(S1) the initial condition is $u_0=1$; (S2)~the polynomial $g(x)$ has no positive integer root; (S3)~the target $t$ is non-zero;
(S4)~the polynomials $f$ and $g$ have the same degree and leading coefficient.

For the purposes of deciding the Membership Problem, we can assume without loss of generality that all instances are in standard form.  An arbitrary instance can be transformed into one satisfying Condition~(S1) by 
multiplying the sequence and target by a suitable constant.  Instances of the  Membership Problem that fail to satisfy 
Conditions~(S2) and (S3) are trivially solvable.  The positive integer roots of~$g$ can be computed and for any such root $n_0$, we have $u_n=0$ for all $n\geq n_0$.
%
Finally,
for recurrences that fail Condition~(S4) we have that \[ \frac{u_n}{u_{n-1}}=\frac{g(n)}{f(n)} \] either converges to $0$ or diverges in absolute value.  Under the assumption that~$t\neq 0$, in each case we can compute an effective threshold $n_0$ such that $u_n\neq t$ for all $n\geq n_0$.

\paragraph{The $p$-adic valuation}
Let $p\in \N$ be a  prime.
 Denote by~$v_p:\Q \to \Z \cup\{\infty\}$
the $p$-adic valuation on~$\Q$. 
Recall that for a  non-zero  number~$x\in \Q$, 
$v_p(x)$ is the unique integer such that~$x$ can be written in the form
\[x=p^{v_p(x)}\; \frac{a}{b}\]
where $a,b\in \Z$ and $p$ divides neither $a$ nor $b$.
The value $v_p(0)$ is defined to be $\infty$. 
The valuation possesses two important properties:
\begin{enumerate}
	\item[-]$v_p(x+y)\geq \min\{v_p(x),v_p(y)\}$ \, (\emph{strong triangle inequality}),
	\item[-]$v_p(xy)=v_p(x)+v_p(y)$ \, (\emph{multiplicative property}).
\end{enumerate}



\paragraph{Asymptotic estimates for series over primes}
Given  ${\sim} \in {\{<,=,> \}}$ and $x\in \Q$,
we denote sums over primes \(p\in\N\) such that 
\(p \sim x\) by \(\sum_{p \sim x}\).
Let \(\pi(x) := \sum_{p\le x} 1\) count the number of primes of size at most~\(x\).
The following result is a consequence of the Prime Number Theorem.
    \begin{theorem}\label{thm:pnt} Let \(\pi(x)\) count the number of primes of size at most \(x\), then  
            \begin{equation*}
               \pi(x) = \frac{x}{\log x} + O\Bigl(\frac{x}{\log^2 x}\Bigr).
            \end{equation*}
    \end{theorem}

As an aside, an element \(a\in\Z\) is a \emph{square} modulo a  prime \(p\in \N\) if there exists an \(x\in\Z\) such that  \(x^2 \equiv a \pmod{p}\). 
An element \(a\in\Z\) is a \emph{quadratic residue} modulo  \(p\) if \(a\) is both a square modulo \(p\), and furthermore \(a\) and \(p\) are co-prime.
We denote by \(\mathcal{L}_p\) the set of quadratic residues modulo \(p\).



Recall the first of Mertens' three theorems \cite{mertens1874zahlentheorie}  (see also \cite[Theorem~4.10]{apostol1998introduction}),
\[
 \sum_{p \leq x } \frac{\log p}{p} =  \log x + O(1) \,.
 \] 
In the sequel we shall make use of the following refinement of Mertens' theorem.
\begin{proposition} \label{prop:apostol_primes_beta}
Suppose that \(a \in \Z\) is not a  perfect square. 
Then
\begin{equation*}
\sum_{p\leq x, \, a \in \mathcal{L}_p}  \frac{\log p}{p} =  \frac{1}{2}\log(x)+O(1).
\end{equation*}
\end{proposition}
\autoref{prop:apostol_primes_beta} appears in work by Selberg  \cite[Equation (3.3)]{selberg1950pnt-ap}
on an elementary proof of Dirichlet's theorem in arithmetic progressions.



\section{Solution Overview}\label{s.overview}

% Assume that in the section before, we highlighted the desire to monitor training runs, that they are large runs, that we wanted it to not be privacy-invading, and that they would be verifiably secure. Also that 
In this section, we outline a high-level technical plan, illustrated in Figure \ref{fig:overview}, for Verifiers to monitor Provers' ML chips for evidence that a large rule-violating training occurred.
\begin{figure}
    \centering
    \includegraphics[width=\linewidth]{figures/verification_process.pdf}
    \caption{Overview of the proposed monitoring framework.}
    \label{fig:overview}
\end{figure}
The framework revolves around chip inspections: the Verifier will inspect a sufficient random sample of the Prover's chips (Section \ref{s.sampling}), and confirm that none of these chips contributed to a rule-violating training run.
For the Verifier to ascertain compliance from simply inspecting a chip, we will need interventions at three stages: on the chip, at the Prover's data-center, and in the supply chain.
\begin{itemize}
    \item \emph{On the chip} (Section \ref{s.onchip}): When the Verifier gets access to a Prover's chip, they need to be able to confirm whether or not that chip was involved in a rule-violating training run.
    Given that rule violation depends only on the code that was run, our solution will necessitate that ML chips logging infrequent traces of their activity, with logging done via hardware-backed firmware.
    We suggest that ML chips' firmware occasionally log a copy of the current state of the chip's high-bandwidth memory to long-term storage, and in particular, that it logs the shard of the NN's weights stored in memory. 
    These \emph{weight-snapshots} can serve as a fingerprint of the NN training that took place on each chip.
    \item \emph{At the data-center} (Section \ref{s.datacenter}): The Verifier needs a way to interpret the chips' logs, and determine whether or not they are evidence for a rule-violating training run.
    To that end, the Prover, who is training the model, will be required to store a transcript of the training process --- including training data, hyperparameters, and intermediate weight checkpoints --- for each model they train.
    Using protocols similar to ``Proof-of-Learning'' \cite{jia2021proof}, these training transcripts may serve as provenance for the logged weight-snapshots, which are themselves the result of the same training process.
    In practice, for each (hash of a) weight-snapshot logged by a chip, the Prover provides the Verifier (the hashed version of) the matching training transcript.
    Then the Prover and Verifier jointly and securely verify that, with high probability, retraining using the training transcript would have indeed resulted in the logged weight-snapshot (and that no other valid training transcript could have resulted in that snapshot).
    Finally, now that the Verifier knows an approximate training transcript of the training run that had been executed on that chip at that time, they can examine properties of the training transcript to confirm that the Prover has complied with the agreed upon rules.
    \item \emph{At the supply chain} (Section \ref{s.supplychain}): The Verifier needs to know which ML chips the Prover owns, so that the Verifier can randomly inspect a representative sample of those chips, to confirm their ownership and that their logging mechanisms are reporting correctly.
    Without this chip-ownership verification step, a Prover might covertly acquire a large quantity of chips and use them for training without ever notifying the Verifier.
    Keeping track of chip-ownership is viable because the cutting-edge data-center chip supply chain is highly concentrated, meaning that chips originate from a few initial chokepoints and can be monitored thereafter.
\end{itemize}

These steps, put together, enable a chain of guarantees.
\begin{itemize}
\item
    When any organization wants to train a large rule-violating ML model, they must do so using chips that the Verifier is aware of.
\item
    These chips will occasionally log weight-snapshots.
    Each time a chip logs a weight-snapshot, the Prover must report the log to the Verifier, along with (hashes of) training transcripts that establish the provenance of that weight-snapshot.
\item
    If the Prover tries to hide the logs from every chip involved in the training run, the Verifier will eventually find out, because it will sample and physically inspect at least one of those chips with high probability.
\item
    Also, the hashed training transcripts that the Prover provides along with the logged weight-snapshot need to be authentic, reflecting the training run that was run on the chip.
    If they are not authentic, they will be caught by the Verifier's transcript verification procedure.
\item
    If the training transcript \emph{is} authentic, and the run violated the rules, then the Verifier can tell, and will catch the Prover.
\end{itemize}
    
Thus, so long as the Prover complies with the Verifier's steps, the Verifier will detect the Prover's rule-violation with high probability.
Just as in financial audits, a Prover's refusal to comply with the verification steps would itself represent an indication of guilt.
%In the next sections, we will explore the technical requirements for each of the three intervention stages, and highlight remaining open problems.

\subsection{Chip inspections}\label{s.inspection}
% \future{In an ideal world, chips' behavior could simply be guaranteed in hardware, and they could remotely communicate with the Verifier via a unique private key embedded on-chip.
% Unfortunately, in reality, it is generally understood within the hardware security community that there is as-of-yet no viable technology for ``tamper-proofing'' chips, and even secure hardware can be undermined if an attacker is provided physical access. \todo{cite}
% The Prover, being the owner of the chip, will certainly have such access.
% However, tamper-evidence technology is much more mature, and indeed many hardware attacks require physically damaging the chip in ways that would be nearly impossible to hide from a physical inspection by the Verifier.
%  \tdo{Add a note about what's the deal with non-invasive attacks.}
% }
Ideally, chips could remotely report their logs, with on-chip firmware and remote attestation being sufficient to guarantee that those logs were truthfully reported (see Section \ref{s.onchip}).
Such guarantees may be sufficient to deter most Provers from misbehavior.
Unfortunately, these measures may not restrain well-resourced adversarial Provers:
it is generally understood within the hardware security world that, given unlimited physical access to a chip (which the Prover certainly would have), a chip's attestation and signed-firmware-enforcement mechanisms can be undermined \cite{skorobogatov2005semi}.
However, by physically inspecting chips after the fact, it may be possible to reliably detect such tampering.
This is because most hardware attacks require physically damaging the chip in ways that would be nearly-impossible to hide from a Verifier's subsequent inspection, especially if ML chips are supplemented with tamper-evident packaging measures \cite{waksman2010tamper}.
Luckily, rather than needing to inspect every chip, we only need to inspect enough to catch at least one chip from each large training run (Section \ref{s.sampling}).

A chip inspection can be done in one of two ways.
The Verifier can send an inspector to the facility in which a sampled chip is stored, which may be cost-effective if the facility contains multiple chips selected for sampling.
Alternatively, the ML chips can be \emph{transported a neutral inspection site}, as these chips are generally housed in easily-removable servers. (Sensitive components like storage can be removed prior to transport.)
The transporting option may be more attractive to security-conscious Provers who would prefer not to allow Verifier inspectors into their data-centers.

A chip inspection involves several steps.
First, the Verifier confirms the chip's serial number (often burned into chips and accessible both digitally and physically) matches the serial number they requested from the Prover.
This confirms that this is the correct randomly-sampled chip, and also that the Prover still owns the chip and has not secretly resold it to an untrusted party.
Second, the Verifier checks the on-chip logs, from which they extract the weight snapshot hashes which they will check against the Prover's reported training transcripts (Section \ref{s.datacenter}).
Finally, the Verifier checks that the chip's logging mechanisms, such as its secure firmware authentication mechanism (Section \ref{s.onchip}), have not been physically tampered with.


\subsection{How many ML chips does the Verifier need to inspect?}\label{s.sampling}
Sections \ref{s.onchip}, \ref{s.datacenter}, and \ref{s.supplychain} will together guarantee that if the Verifier observes even a single ML chip that logged a weight snapshot hash during a rules-violating training run, then the Verifier can use that weight snapshot to prove that the Prover violated the rules.
For the Verifier to make sure they'd observe at least one such snapshot from a large training run with high probability (or alternatively, catch a chip that'd been tampered with), they would need to inspect at least $s$ of the Prover's $\totalchips$ chips every $T$ days.
We now seek to calculate $s$.

We define a large-scale training run as any training run using more than $\rulecomputeflops$ FLOPs.
We only seek to verify rules on training runs using $> \rulecomputeflops$ FLOPs, meaning that any weight-snapshots from the first $\rulecomputeflops$ FLOPs are not incriminating (as the Prover could have stopped the training run after that snapshot).
Catching a chip that happened to save a weight-snapshot at time $\rulecomputeflops + \epsilon$ is vanishingly unlikely for small $\epsilon$, so we instead provide guarantees on detecting the case where the Prover's training run is executed for $\geq 2 \rulecomputeflops$, substantially exceeding $\rulecomputeflops$.\footnote{We can always keep the detection threshold the same by cutting $\rulecomputeflops$ in half.
Also, the literature on neural scaling laws \cite{kaplan2020scaling, rosenfeld2021scaling, epoch2023scalinglawsliteraturereview} suggests that model loss decreases logarithmically at best given additional compute, so failing to catch a less-than-2x violation may have limited consequences.}
This means that the training run was executed for an additional second period of at least $\rulecomputeflops$ FLOPs, during which any weight snapshot would capture evidence of an ML model that had already been trained for at least $>\rulecomputeflops$ FLOPs.
From hereon, when we describe detecting a snapshot of a ``large-scale training run'', we refer to this second stage of the training run.
If a Verifier wanted to catch a Prover \emph{before} they had completed a $\rulecomputeflops$ rule-violating training run, they can simply shrink their target threshold to a fraction of $\rulecomputeflops$, and sample chips on a rolling basis.

The Verifier, whether due to specific suspicions or routine due diligence, wants to audit a particular Prover's $\totalchips$ chips.
Let $a$ be the FLOPs per day for those chips.
We will conservatively assume that NN training can be perfectly parallelized and utilize all the FLOPs on each chip.
Let $f$ be the expected number of weight-snapshots saved by the chip per day (Section \ref{s.onchip}) snapshots are Poisson distributed.
The Verifier wants to catch a training run within at most $T$ days of its concluding, though the training run itself may have lasted for a longer period
\footnote{For simplicity, we assume $\trainingperiod$ is divisible by $\monitoringperiod$}
$\trainingperiod > \monitoringperiod$.
This implies that the Prover must have used a minimum
of $\frac{\rulecomputeflops}{a\trainingperiod}$ chips.
One might expect that by increasing $\trainingperiod$, the Prover could use fewer chips and therefore evade detection, but this turns out to not to meaningfully improve the Prover's evasion probability (a $<5\%$ increase) because each chip is sampled $\trainingperiod/\monitoringperiod$ times as often. (See Appendix \ref{a.provertricks} for details.)
Thus for simplicity, we assume the Verifier will choose $s$ as though $\trainingperiod$ is fixed.

To detect at least one chip with a snapshot from a rule-violating training run with probability $p$, the Prover needs to sample $s$ chips each time period $T$, where:
\begin{align}
\label{eq:samplingrate}
    s \geq \frac{\log{(1-p)}}{ \log{\left (1 - \frac{\rulecomputeflops}{a\trainingperiod \totalchips} \left ( 1 - e^{-f\monitoringperiod} \right ) \right ) }}
    \cdot \frac{\monitoringperiod}{\trainingperiod} 
\end{align}
We provide a derivation for Equation \ref{eq:samplingrate} in Appendix \ref{a.samplingderivation}.

In Table \ref{tab:samples}, we provide the total required samples-per-year $365 \cdot s/ \monitoringperiod$ to catch several past and projected training runs, for Provers who own different quantities of chips $\totalchips$.
\begin{table}[tbh]
% \begin{tabular}{|p{4.5cm}|r|r|r|r|r|r|}
    \begin{tabular}{
    l
    % p{3.8cm}
    >{\raggedleft\arraybackslash}p{1.6cm}>{\raggedleft\arraybackslash}p{1.6cm}>{\raggedleft\arraybackslash}p{1.8cm}>{\raggedleft\arraybackslash}p{1.6cm}>{\raggedleft\arraybackslash}p{1.6cm}>{\raggedleft\arraybackslash} p{1.6cm}}
    % \begin{tabular}{|l|S[table-format=1.2e2]|S[table-format=1.2e2]|l[table-format=1.2e2]|r|S[table-format=1.2e2]|}
    % \hline
    % \textbf{Model} & \textbf{Training FLOPs $\rulecomputeflops$} & \textbf{H100-days $\computeflops/a$} & \textbf{H100s to train in 3mo} & \textbf{Frac global chips sampled/year} & \textbf{Samples/year (out of 5M)}\\
    % \hline
    \\
\textbf{Model}  & \textbf{Training FLOPs $\rulecomputeflops$}  & \textbf{H100-days $\rulecomputeflops/a$} & \textbf{H100s to train in 1 yr} & \textbf{Samples/yr if $C =10^3$} & \textbf{Samples/yr if $C =10^5$} & \textbf{Samples/yr if $C =10^7$} 
\\
\hline
    \\
GPT-3 \cite{BMR+20} & 3.14e+23 & 3.64e+3 & 10 & 243 & 2.43e+4 & 2.43e+6 \\
Chinchilla \cite{hoffman2022training} & 5.76e+23 & 6.67e+3 & 19 & 132 & 1.33e+4 & 1.33e+6 \\
PaLM \cite{chowdhery2022palm}& 2.56e+24 & 2.96e+4 & 82 & 29 & 2.98e+3 & 2.99e+5 \\
Chinchilla-280B $^\dag$  & 9.9e+24 & 1.15e+5 & 314 & 7 & 771 & 7.72e+4 \\
Chinchilla-1T $^\dag$ & 1.27e+26 & 1.47e+6 & 4.03e+3 & --- & 60 & 6.02e+3 \\
Chinchilla-10T $^\dag$ & 1.3e+28 & 1.5e+8 & 4.12e+5 & --- & --- & 58 \\
\\
    \end{tabular}
 %   \hspace{2cm}
 \caption{
 \small{
 Example numbers of required total \emph{annual} samples $365 \cdot s / \monitoringperiod$ to catch a chip from every large-scale training run within $T=30$ days, 
 given $a = 10^{15} \cdot 24 \cdot 3600$ (the daily 16-bit Tensor Core FLOPs of an NVIDIA H100 SXM GPU \cite{nvidiah100}), 
 $f = 0.1$ weight snapshots per day (see Section \ref{s.onchip}), 
 and the Verifier's desired probability of catching a rule-violating training run $p=0.9$.
 Models marked with $\dag$ are projections for future training requirements \cite{hoffman2022training}.
 }
 }
\label{tab:samples}
\end{table}
As the table shows, the number of inspected chips remains approximately at a fixed fraction of the full stockpile, decreasing with training run size $\rulecomputeflops$.

An additional difficulty is the possibility that multiple Provers collude and each contribute a small fraction of their chips to a shared training run, such that if only one of the Provers is audited, the fraction of their $\totalchips$ chips they used for the training run is much smaller than $\frac{\rulecomputeflops}{a\monitoringperiod\totalchips}$.
This is particularly relevant in the International Oversight setting, where a government Prover might force the participation of multiple firms.
One solution to this is for an international Verifier to do background monitoring of \emph{all} Provers, in effect treating them all as one large Prover with access to the global chip stockpile.
Predictably, this would require a much greater number of samples, and is likely to only be feasible for regulating the largest-scale training runs (which can be caught with fewer samples).
Yet, for training runs at the scale of $10^{25}$ FLOPs or greater, monitoring could be done with a bureaucracy similar in size to the IAEA.\footnote{
We want to estimate the number of inspectors needed to catch a Chinchilla-280B-sized training run, with $10^{25}$ FLOPs, given several more years of hardware progress and global production.
Given $\totalchips = 10^7$ worldwide chips ($>5\times$ global stocks as of 2022), each of which can output $a=3\cdot 10^{15} \cdot 86400$ FLOPs per day ($3\times$ more FLOP/s than the NVIDIA H100), detecting a Chinchilla-280B-sized run within $T=30$ days of its completion anywhere on earth with 90\% probability would require roughly 232,000 worldwide chip samples per year.
A single inspector might be expected to verify at least 1000 chips a year, especially if those chips are brought to a central location (see Section \ref{s.inspection}).
This would require $\approx232$ inspectors, slightly smaller than the 280 active IAEA inspectors as of 2021. 
 \cite{dayinthelife}.}

As individual chips become more powerful ($a$ grows) and algorithmic efficiency increases ($\rulecomputeflops$ shrinks), the number of required samples would need to grow in order to verify rules on a fixed-capability training run.
Still, it may be that the training runs of greatest societal concern are those \emph{near the frontier} of scaling, and those are likely to remain detectable with few samples for the foreseeable future, even if Provers attempt to undermine them with physical tampering.

% We include a table of how many chips would need to be sampled worldwide to catch different scales of training runs, given a near-future quantity of chips and hardware FLOPs.

% \begin{itemize}
%     \item \emph{Sending an inspector to the chip}. The Verifier can send an inspector to the facility in which a sampled chip is stored. 
%     \future{This may be more expensive for the Verifier, as it requires the inspector to travel to a new location, and also may undermine the facility's security guarantees (e.g., if it is otherwise airgapped). On the other hand, it may be cheaper if the same facility contains a large number of chips whose IDs were all randomly sampled.}
%     \item \emph{Sending the chip to an inspector}. ML chips are generally housed in easily-removable servers, and these servers can be removed and transported to a central location to be inspected by the Verifier.
%     This means that physical security (and even location secrecy) of the datacenter does not need to be undermined.
%     \future{
%     If there are any sensitive extraneous components in the server (e.g., persistent memory banks), they can be removed prior to transport.
%     \todo{Mention that this may force us to sample several GPUs at the same time?}
%     }
% \end{itemize}

\section{On the chip}\label{s.onchip}
In an NN training cluster, a large number of ML devices (each containing an ML chip, and connected to a ``host'' CPU chip) collaborate to simultaneously compute gradient updates.
Large NNs have too many weights to fit in the high-bandwidth memory of a single ML device, so efficiently training a model requires splitting the model's weights
into \emph{weight shards}, and placing a shard on each chip \cite{smith2022computation}.
Each such shard is a vector of millions or billions of values.
Shards are updated after each gradient step so that new gradients can always be computed with respect to the latest weight values.
This means that at every timestep, the chip contains an up-to-date result from the training run, which we will show in Section \ref{s.datacenter} can serve as a type of checksum for the training run. \looseness=-1

In this section, we will describe how it may be possible to have ML chips log the values of these weight-shard vectors, for Verifiers to retroactively inspect.
We will assume the presence of certain hardware features\footnote{These include standard components of a hardware security module: firmware verification and attestation, firmware anti-rollback protection, and the ability to sign, authenticate, and encrypt messages.
It also includs the ability for firmware to raise interrupts, read device memory, and (to efficiently hash weight snapshots) execute code on the chip's processor.}
on the ML chip, though the details of the approach may need to be modified depending on the specifics of the device.
For example, if the ML chip does not have the hardware required for remote firmware attestation, the chip may need to be permanently mated to a host CPU that does have such hardware, with the CPU then orchestrating the logging operations.

A chip's weight-shards can be captured by \emph{snapshotting} the chip's high-bandwidth memory, which contains the weights.
These snapshots must be saved at random times, as otherwise the Prover can swap out the true weights for a fake set of weights right before the scheduled logging.
With some small probability at every clock cycle, the firmware present on each of the Prover's chips should raise an interrupt and briefly freeze the chip's operations.
The chip should then send a copy of the weight shard regions $\omega$ to its host CPU to save in Prover-controlled long-term storage.
The firmware also computes a hash $h = \hashfunction(\omega)$ of the relevant memory regions,
\footnote{Such hashing can be done cheaply so long as the firmware can leverage the ML chip's processing power.}
 using a publicly-agreed cryptographic hash function $\hashfunction$.
This hash $h$ is then saved for the Verifier to find it later, either by having the firmware store it on-chip\footnote{Keeping the hash in local NVRAM is useful if the ML chip is in a secure data-center and is restricted from communication with the outside world, such as via air-gapping.} in nonvolatile memory writable only by the firmware, or by having the chip sign the hash with its private key and then sending it to a Verifier-trusted central server (which must return an acknowledgement of receipt before the chip can resume operating).
Later, a Verifier can use $h$ to ask the Prover to provide them with the original weight shard $\omega$, as this is the only input to $\mathcal{H}$ whose hash matches $\hashfunction(\omega) = h$.

More frequent snapshots mean more performance overhead for the Prover, especially if all the other synchronized chips in the training run must hang until the single chip's snapshotting concludes. 
This freezing may become expensive if the rest of the training run must hang until it completes, especially if it's done by many chips.
\footnote{A possible alternative would be to orchestrate less-frequent snapshots via a Verifier-trusted central server, though this would require a solution to the problem of the Prover swapping out the chip's weights before propagating the Verifier's signal.}
The frequency of snapshots should be set as low as possible while still ensuring that used chips reliably contain a relevant weight snapshot if they were employed in a large-scale training run (see Eq. \eqref{eq:samplingrate}).

Though the Prover may claim to have used firmware that logged snapshots as described above, the Verifier may not trust the Prover to have indeed used the appropriate logging firmware.
For this reason, ML chips also need hardware mechanisms for enforcing that firmware be signed, and in turn, chipmakers will need to only sign firmware if it implements appropriate logging mechanisms.
\footnote{A similar firmware mechanism was used by NVIDIA to enforce cryptocurrency mining limits on its GPUs \cite{gartenberg_2021}.}
\footnote{Assuming that logging-free versions have been signed in the past, ML chips may need to include anti-rollback functionality \cite{krishnamurthy_odonoghue_bhatia}.
The Prover might also delay updating the chip's firmware until shortly before inspection.
To avoid this, ML chips may need to be made to remotely attest at an earlier time that their firmware has been updated.}
\footnote{Additional challenges arise if the Verifier suspects that the chipmaker (who authors the firmware) and Prover have colluded to create firmware with backdoors that disable logging.
Increasing Verifiers' confidence in the firmware may be an important consideration when verifying the operations of powerful nation-state Provers.}

An obstacle to logging all weight-shards stored in ML device memory is that different ML training code will store a model's weights in different regions of memory.
The chip's firmware must be able to determine which region of memory the weights are stored in.
It may be possible to determine the weight-regions retroactively, by logging the regions of memory claimed to correspond to the weights, along with a copy of the compiled on-device code, which can then be matched to Prover-provided source code and its memory allocation pattern analyzed.
\footnote{It may even be possible to modify standard libraries for generating chip-level ML training code (e.g., PyTorch-generated CUDA) to make their memory allocation processes more easily checkable by a subsequent Verifier.}
\footnote{Revealing the Prover's source code to the Verifier directly may be unacceptable, demanding a more complicated verification procedure like that described in Section \ref{s.realworld}.}
As a more invasive alternative, the Prover could proactively certify that its chip-level ML training code stores the weights in a specific memory region, by having its chip-code verified and then signed by a Verifier-trusted server before it can be executed by the firmware.
\footnote{The iOS App Store uses a similar method to ensure Apple devices only run signed programs \cite{codesigning}.}

%so they'd ideally happen less than once per week on average, but they must still be frequent enough for sampled chips to reliably contain weight snapshots from any large-scale training runs (see Eq. \ref{eq:samplingrate}).

A more challenging problem is that ``ML chips'' are general-purpose hardware accelerators, and thus are frequently used for other high-performance computing workloads like climate modeling.
There is no straightforward way to determine whether an ML chip is running a neural network training job (and therefore should log weight snapshots), or an unrelated type of job exempt from Verifier oversight.
\footnote{Potential avenues for addressing this may include be requiring non-ML-training code compilers to also sign their results, or improving methods for distinguishing between ML training code and other code.
If the types of code can be retroactively distinguished, then ML chips could all occasionally save memory/code snapshots, and then retroactively determine whether they belonged to a large-scale training run and thus deserve further scrutiny.
One particularly straightforward to address case is ML inference: the model's in-memory weights could be snapshotted and retroactively verified in a similar way to that described in Section \ref{s.datacenter}.
} 
Relatedly, it would also be desirable to be able to exempt a fraction of chips from monitoring, for use in education or small-scale R\&D where the overhead of compliance would be prohibitive.
\footnote{This might be addressable by having these ML chips' interconnect topology restricted to prevent their use in large-scale training.
However, methods for Verifiers to retroactively confirm the topology that ML chips were configured in at a data-center are beyond the scope of this work.}
Until we find methods for reliably distinguishing large-scale ML training, some fraction of ML chips must be exempted from using weight-logging firmware.
The larger system's guarantee would then be that \emph{for those of the Provers' chips that the Prover allows the Verifier to monitor}, no rule-violating training occurred.
The Verifier may then use their leverage to incentivize Provers into opting-in a larger and larger fraction of their chips over time.
\section{At the data-center}\label{s.datacenter}

% 1) Proof-of-Learning: high-level idea for what we do with weight snapshots, and why they tell us about the training run
%     1) Previous work on PoLs
%     2) Compliance model: like finance, give description of what you did, and during audit provide more interactive validation. Not-participating-in-audit is equivalent to having committed the original crime.
% 2) Reporting: what do PoL reports tell us?
% 3) Verification: how do we verify PoL was correct?
%     1) Theoretical: original PoL doesn't give us what we want, we need to modify it slightly.
%         1) PoTT instead of PoL
%         2) Whether PoL also extends to only shards of larger network
%     2) In practice, to verify PoL we need a shared machine (to not leak data). Describe procedures on that machine.
% 4) Operational overview:
%     1) Collect PoLs for every training run, and report them (hashed) to Verifier along with all chip logs, on some interval
%     2) Then Verifier randomly samples some chips, and then looks at those PoL transcripts corresponding to the chips, and picks the transcript steps that need to be verified
%     3) These are all brought to a trusted computer, and PoL verification occurs
%     4) Once PoL is verified, can run arbitrary programs on it to determine if it followed rules
% 5) Talk about the costs
%     1) Talk about the costs of storage
%         1) Both for PoLs, and also mention the costs for the random sampling
%     2) Talk about the cost of reverification, and what it depends on
From Sections \ref{s.sampling} and \ref{s.onchip}, we know how a Verifier can get access to at least one weight shard $\omega$ snapshotted during a large-scale training run.
In this section, we will show how the Verifier can use this weight shard to determine whether the associated training run followed the agreed-upon rules.

On its own, knowing $\omega$ is useless to the Verifier, since the Verifier does not know the values of the other weight-shards comprising the NN, or even how many other weight shards there are.
Yet $\omega$ holds the key to the training run's history. 
Any given weight-shard vector is incredibly unlikely to be yielded by the training process (there are $\gg 2^{10^6}$ possibilities even if we round to 1-bit precision).
But this particular $\omega$ is the nearly-deterministic result of the training run's hyperparameters $\mathbb{M}$ (including its random seed) and sequence of training data $\mathbb{D}$.
We refer to this information, along with information yielded throughout the training run $\mathbb{W}$ (such as intermediate weight snapshots), as the model's \emph{training transcript} $\mathbb{T} = \{ \mathbb{M}, \mathbb{D}, \mathbb{W}\}$.
It is plausible that it would be very difficult for the Prover to find a second ``legitimate'' training transcript $\mathbb{T}' = \{ \mathbb{M}', \mathbb{D}', \mathbb{W}' \} \neq \mathbb{T}$ that would exactly yield a full set of weights that included $\omega$ as a contiguous weight-shard.
It is also plausible that, if the Prover tried to lie and provide an $\mathbb{H}', \mathbb{D}'$, and $\mathbb{M}'$ that would \emph{not} result in $\omega$ but which they \emph{claimed} would result in $\omega$, then the Verifier might find a suite of tests that would detect such a faked transcript reliably.
If the Verifier could reliably detect such fakes, 
then by asking the Prover to provide a training transcript that would yield each logged snapshot $\omega$, the Verifier could force the Prover to reveal the true data $\mathbb{D}$, and hyperparameters $\mathbb{M}$ used in every large-scale training run (along with the full weights of the resulting model, not just $\omega$).
This would be sufficient information to check that the training run complied with any of the training rule types listed in Section \ref{s.rules}.

Unfortunately, having the Verifier simply re-run the full training process to verify a transcript's correctness is unworkable for several reasons.
First, the Prover would likely not be willing to reveal their training data, model weights, and hyperparameters, so the Verifier must do any verification without direct access to the data. (We address this in Section \ref{s.realworld}.)
Second, the compute cost to fully re-run the training transcript would be massive, as large as every original training run.
Third, the training run would likely not be perfectly reproducible: due to hardware-level noise, even two repetitions of the same sequence of training updates would gradually diverge.
Fourth, the Prover \emph{may} be able to construct a second ``spoof'' training transcript, that yields an exact match for $\omega$ but differs from the original training run that yielded $\omega$ in the first place.\footnote{As a trivial example, the Prover could claim that $\omega$ was simply the value of the random initialization, and no training had happened at the time of the snapshot.} 

\future{
\begin{itemize}
    \item \emph{Imperfect retraining}: due to irreducible hardware-level noise, two repetitions of an identical sequence of training steps will gradually diverge, so even the true training transcript would not result in an exact match of $\omega$.
    Thus, we can only verify that training resulted in an $\omega'$ that \emph{approximates} $\omega$.
    \item \emph{Compute cost:} Requiring the Verifier to re-run the entirety of every training run would be hopelessly expensive, creating a minimum 50\% overhead on all large-scale ML training, which is unacceptable. This would also raise the problem of how another party could verify the Verifier's compute usage, which by induction may lead to an infinite reverification cost.
    Thus, the Verifier must be able to verify training \emph{efficiently}, i.e., using $bC$ compute to verify a training run requiring $C$ compute, where $b \ll 1$.
    \item \emph{Privacy/secrecy violations}: In the above protocol, the Prover is forced to provide the Verifier access to all the (often private) training data, the training hyperparameters, and the final model weights, each of which is near-certain to be an unacceptable breach of the Verifier's confidentiality.
    \item \emph{Spoofing possibility}: It may be possible for the Prover to construct a second ``spoof'' training transcript, that yields an exact match for $\omega$.
    As a trivial example, the Prover might claim that $\omega$ was simply the value of the random initialization, and no training had happened at the time of the snapshot.
    Thus, the Verifier needs to be able to either place additional requirements on the types of allowable training runs, or detect each possible spoofing strategy.
\end{itemize}
}

Thanfully, a close variant of this problem has already been studied in the literature, known as ``Proof of Learning'' \cite{jia2021proof}.
The goal of a Proof-of-Learning (PoL) schema is to establish proof of ownership over a model $W_t$ (e.g., to corroborate IP claims) by having the model-trainer save the training transcript $\mathbb{T}$ (including hyperparameters $\mathbb{M}$, data sequence $\mathbb{D}$, and a series of intermediate full-model weight checkpoints\footnote{We use ``weight checkpoints'' as shorthand, but if using an optimizer like Adam \cite{KingmaB14}, the optimizer state should also be included.} $\mathbb{W} = \{W_0, W_{k}, W_{2k} \dots \}$) which only the original model trainer would know.
Jia et al. \cite{jia2021proof} propose a verification procedure that makes it difficult for any third party to construct a spoofed transcript $\mathbb{T}'$, if they only have access to $W_t$ and the unordered dataset.

The solution of \cite{jia2021proof} is as follows: once a Prover reports a training transcript $\mathbb{T}$, the Verifier checks that the initialization appears random, and then chooses a number of pairs of adjacent weight snapshots that are $k$ gradient steps apart $(W_i, W_{i+k}), \dots, (W_j, W_{j+k})$.
Then, rather than re-running all of training, the Verifier only reruns the training of these specific segments, starting at $W_i$ and progressing through the appropriate data batches $D_i \dots D_{i+k}$ to yield a $W_{i+k}'$. 
The Verifier then confirms that the resulting checkpoint is approximately correct: $\| W_{i+k} - W_{i+k}'\|_2 < \epsilon$ for some small $\epsilon$ that depends on the reproduction noise and the number of steps $k$.
The original paper proposed a heuristic for choosing which segments to verify,
% as naively-generated spoofs are likely to end at a different $W_t'$, and large jumps in weight-space would be needed for the spoof to instead appear to end at the target $W_t$).
though subsequent papers have demonstrated spoofing attacks that evade the original verification approach \cite{zhang2022adversarial, fang2022fundamental}. 
% \todo{constructing fake data with particular gradients, abusing epsilon to sneak in microupdates}. \todo{cite} connected the hardness of provably secure PoL on unconstrained ML training to certain open problems in learning theory \todo{cite}./
Thus far, the literature on PoL has been heuristic-based, and discovering PoL protocols with higher degrees of reliability is the subject of ongoing work \cite{fang2022fundamental}.

\subsection{Proof-of-Training-Transcript}

Our requirement, which we will define as ``Proof-of-Training-Transcript'' (PoTT), differs from Proof-of-Learning because our Prover knows a true training transcript that leads to the final weight-snapshot (since they executed the original training run), and the Verifier wants to verify that the Prover has reported the \emph{original} transcript that yielded $\omega$ (or at least one very close to the original).
Further, unlike PoL, we allow that an adversarial Prover might be willing to spend \emph{more than the original training compute} $\rulecomputeflops$ in order to construct a spoof (e.g., an adversary may be willing to pay double the cost in order to execute a rule-violating training run without being detected).
Thus, we must hope to push the cost to construct a spoof to be as high as possible, so as to make evasion impractical.

\begin{definition}
 A PoTT protocol $(R, V)$ is defined by a Prover training and reporting protocol $R$ and a Verifier verification protocol $V$.
A target weight vector $W_t$ is produced as the true result of a training run which complies with a training and reporting protocol $R$, and which can be summarized via a training transcript $\mathbb{T} \in \mathcal{T}$, where $ \mathbb{T} = \{\mathbb{M}, \mathbb{D}, \mathbb{W}\}$, $\mathbb{M}$ is the metadata required to reproduce the training run (including hyperparameters, random seed, and loss function), $\mathbb{D}$ is an ordered sequence of sets of training data batches, and $\mathbb{W}$ is a collection of intermediate weight checkpoints resulting from the training procedure.
The verification protocol $V$ should accept any such true training transcript with high probability, $\Pr[V(\mathbb{T}, W_t) = \text{accept}] > 1 - \delta_1$ for some small $\delta_1$.

A ``spoofed'' training transcript $\mathbb{T}' = \{\mathbb{M}', \mathbb{D}', \mathbb{W}'\}$ is a transcript, which may not correspond to any valid training run, and which is substantially different from the original transcript $\mathbb{T}$ in its data or hyperparameters: $d_1(\mathbb{D}, \mathbb{D}') \geq \delta_3$ for some edit distance $d_1$ quantifying the number of data point insertions/deletions, and/or $d_2(\mathbb{M}, \mathbb{M}') \geq \delta_4$ for some hyperparameter distance $d_2$. 
A reporting/verification protocol pair $(R, V)$ is $J$-efficient and $F$-hard if $V$ runs in at most $J$ time, and there does not exist any spoof-generation algorithm $A \in \mathcal{A}: \mathcal{T} \rightarrow \mathcal{T}$ such that $\Pr[V(A(\mathbb{T}),W_t) = \text{accept}] > 1 - \delta_2$ where $A$ runs in less than $F$ time. \looseness=-1
\end{definition}

Colloquially, we want a Prover training and reporting protocol and Verifier verification protocol such that the Verifier only accepts \emph{original} training transcripts that would result in a final weight checkpoint which contains a shard matching our on-chip weight-shard snapshot $\omega$.
We leave the problem of finding provably secure, efficient methods for PoTT as an important avenue for future work, but we discuss a few promising directions below.

% \future{
% We next provide intuition on why Proof-of-Training-Transcript is likely to be tractable in our setting.
% }
PoTT appears to be strictly harder than PoL, as it requires robustness to a better-resourced adversary has additional information (i.e., they know the true transcript $\mathbb{T}$) and has more compute-time to construct a spoof.
Given that existing PoL schemes are still heuristic-based and not yet provably secure, there may be a long way to go until we have PoTT methods that are both efficient and hard to spoof.
Still, one dynamic favoring the Verifier is that the Prover must \emph{commit} to a training transcript without knowing the Verifier's verification strategies.
Thus, Verifiers can amass secret collections of verification heuristics, much as the IAEA does not disclose all its methods for detecting nuclear activity.
% Moreover, as new verification strategies are discovered over time, these can be applied to the Prover's old transcripts.
% Since the Prover could not have anticipated these strategies, and its spoof was by construction \emph{not} the result of normal gradient-based optimization on natural data, any spoof is likely to eventually be caught by a future verification method.
% \footnote{
% This 
% PoTT methods are likely to evolve as we gain more knowledge about the ``typical'' structure of gradient-based optimization on natural data in neural networks.
% Previous spoofs (which are, by definition, not the result of pure gradient-based optimization on natural data) will not obey this structure, enabling new verification checks.
% }
Even if PoTTs are only ever heuristic-based, the presence of this dynamic may dissuade Provers from taking the risk of being detected by an unexpected test.

Defining conditions on the types of legitimate training runs is another useful source of leverage. 
For example, one Prover cheating strategy could be for the Prover to report one long training run as many shorter training runs, each claimed to be ``initialized'' where the previous training run ended.
A simple prevention would be for the training-and-reporting protocol $R$ to require the Prover to initialize every training run's weights via a known pseudorandom generator and a short seed.
This means that the randomness of the initial weights can later be confirmed by the Verifier.

Another promising strategy may be to require the Prover to \emph{pre-commit} to portions of its training transcript (e.g., the hyperparameters $\mathbb{M}$ and training batches $\mathbb{D}$) at the start of training.
This could be done by having the ML chip firmware log a hash of this precommitment, which would prove that the precommitment preceded the chip's snapshot $\omega$.
At the time of precommitment, the Prover does not know what trajectory the training run will follow or at what time it will be snapshotted, as the training has not yet been performed.
The Prover would be unable to construct a spoofed training transcript that would end at $\omega$ and precommit to it, because $\omega$ isn't known yet.
However, it is not obvious how to extend this approach to online learning settings like online RL, where the training data cannot be known ahead of time.

A final complication of our setting derives from the fact that the Verifier observes only a shard of the weights $\omega$, and not the full weight vector $W_t$.
It could be easier to construct a spoofed training transcript for some $\hat{W}$ which contains a shard matching $\omega$, but which differs from the true original weights $W_t \neq \hat{W}$ on the portion of the weight vector outside the shard.
We briefly describe an intuition for why this is likely to be as hard as general PoTT-spoofing.
Assuming $\omega$ must contain weights from more than a single linear layer, any Prover must at minimum construct a valid PoTT for this smaller NN represented by $\omega$, except without any original training transcript to start from (making it similarly hard to the original Proof of Learning problem).
Alternatively, if the Prover tries to reuse the original training transcript, it will require them to synthesize fake data and labels to exactly reproduce the true inputs/gradients, and these must correspond to a reasonable loss function that is actually decreasing over time.
If the Prover alternatively attempts to fake the surrounding layers' weights, then the weights in these surrounding layers must also have their own valid training trajectories as part of the training transcript \emph{and} must generate the appropriate inputs/gradients to yield $\omega$, simultaneously.

\subsection{Real-world implementation}\label{s.realworld}

As the literature uncovers new methods for verifying training transcripts, the Verifier can run these protocol to check that, for each of the logged weight snapshots on the Prover's chips that the Verifier inspects, the Prover has provided a legitimate corresponding training transcript.\footnote{
    Note that this requires the Prover to save and report training transcripts for all training runs corresponding to chips sampled by the Verifier, not just the largest-scale ones.
    This is because, without a matching training transcript, it's impossible for the Verifier to determine whether a given weight-shard was part of a large-scale training run or a small-scale one.
    Alternate methods for proving that a chip was only involved in a short training run are an important direction for future work.
    }
The Verifier can then examine the data, hyperparameters, and resulting models as reported by the trianing transcript, and thereby detect whether the Prover complied with the training rules.

A key design constraint is that the Prover does not disclose the raw transcripts (which likely contain confidential data that the Prover would prefer the Verifier did not see).
We propose a method for the Verifier to verify the Prover's previously-reported transcript's correctness without ever observing the raw data. 
First, the Prover and Verifier utilize a ``trusted cluster'': a set of chips that both parties trust, which is air-gapped to prevent the Prover's sensitive data from leaking, and which is equipped to execute small amounts of ML inference and training (so as to segment-wise approximately reproduce the Prover's original training process).
\footnote{Maintaining such compatible training clusters may prove quite challenging.
One desirable direction for future work is in verification methods by which the Verifier does not need to directly reexecute model training.
For example, it may be possible for the Verifier to interactively query the Prover for additional information on segments of the training run beyond what was included in the training transcript.
There may be queries that have valid answers \emph{only if} the original training transcript was authentic (e.g., a series of weight sub-checkpoints between two checkpoints, each with progressively lower loss), and the Prover could dynamically recompute answers to these queries using their own data-center.
While some properties of the verification would still need to be confirmed using a neutral cluster to maintain the confidentiality of the query-responses, such clusters may not need to be equipped for large-scale training, and thus be much easier to maintain.
}
Second, instead of reporting the full transcript, the Prover initially only provides the Verifier with hashes of the training transcript $\mathbb{T}$.
Then, at a later verification stage, the Prover supplies the true $\mathbb{M}, \mathbb{D}$, and $\mathbb{W}$ within the trusted cluster (which must match the reported hashes), executes the verification protocol, and reports whether both steps succeeded.
Finally, the Verifier can use this same trusted cluster to run programs checking that the Prover's training transcript complies with all agreed ML training rules (e.g., measuring the performance of the final model weights on agreed benchmarks).
We provide more details on a possible procedure in Appendix \ref{a.trustedcluster}. \looseness=-1

% \todo{topic sentence instead of question}
% Logistically, how could logged weight-shard snapshots be verified in practice via PoTT, assuming that the verification algorithms look similar to \todo{cite Jia et al}'s Proof-of-Learning scheme?
% Our solution is specifically designed to avoid the Prover ever having to directly reveal the hyperparameters $\mathbb{M}$, training data $\mathbb{D}$, and model weights $\mathbb{W}$ to the Verifier, as in many cases these data are either private or proprietary.

% First, when executing the training run, the Prover follows any verifiable restrictions on their training (e.g., starting from a subset of valid initializations, or precommitting to the model weights).
% Throughout training, the Prover saves the hyperparameters (e.g., training code), the sequence of training data batches, and periodic snapshots of the model weights.
% They also make sure to track any randomness used, and generally make sure their training process is repeatable (up to low-level noise) assuming access to similar but non-identical hardware configurations.
% This itself is a nontrivial technical challenge \todo{cite literature on repeatability}, especially in co-training settings like GANs or RL on learned reward models \todo{cite}, and would benefit from further investments in replication tools for common ML frameworks.

% Periodically, the Prover reports their chips' on-chip logged weight-shard snapshot hashes, along with hashed versions of the Proofs-of-Training-Transcript matching each of these logged hashes (including hashes of $\mathbb{M}$, hashes of each model snapshot $\mathbb{W}$, and sequences of hashes of training data points $\mathbb{D}$).
% The Prover may need to further disclose basic information like the distance between weight-snapshot pairs, which may be needed for the Verifier to determine which regions of the transcript to verify.\footnote{This info could be proven to the Verifier securely and privately, for example by using standard ZK-SNARK proof tools to confirm that two given hashes correspond to vectors that have a given $L_2$ distance between them.}

% As in Section \ref{s.chipowner}, the Verifier chooses a subset of these chips whose logs will be confirmed via physical inspection.
% The Verifier inspects the training transcripts and associated metadata, and chooses the subsets of the training transcripts to verify.
% They also select any other components of the training transcript (e.g., a random sample of the training data) that they will need in order to determine that that the training run has complied with the ML training rules.
% At this point, the Verifier has a set of hashes of the inputs to its verification protocol (including model weight snapshots and training data points), but does not have access to any of this information directly.

% The Verifier needs to have faith in the results of the computation, but without seeing the inputs themselves.
% In principle, this might be addressed by cryptographic techniques like \todo{FHE doesn't do exactly this; what does?}, but in practice these techniques cannot efficiently execute computationally intensive programs, like long sequences of gradient updates on billion-parameter models.
% Instead, the Prover and Verifier can agree on a neutral jointly-trusted cluster, which they will use to execute the verification protocol.
% The Verifier needs to be able to trust in the integrity of the cluster's computation.
% The Prover, conversely, needs to trust that the cluster will not reveal its private data, by verifying that it cannot communicate with the outside world and has no persistent storage (a standard feature in many datacenters \todo{cite something highlighting this is possible}).
% Such neutral clusters could be maintained by a trusted third party, or when no such trusted party exists, could be jointly overseen by both the Prover and Verifier. \todo{mention that these clusters have other benefits, see later}

% At agreed-upon intervals, the Verifier supplies the hashes of the inputs (including hyperparameters $\mathbb{M}$) to the neutral cluster, and the Prover supplies the inputs themselves.
% The cluster hashes the Prover's inputs and confirms they match the hashes.
% Then the cluster executes the Verifier's verification protocol.
% For example, when re-executing a training segment to verify that it would reach a particular chip-logged weight shard snapshot $\omega_{i+k}$, the cluster starts at one weight snapshot $W_i$, and then computes optimizer updates $k$ times using the specified inputs (where each update is computed via code generated from the hyperparameters, and even code snippets, defined by $\mathbb{M}$).
% Finally, the cluster checks that the resulting weight vector $W'_{i+k}$ has a slice $\omega'_{i+k}$ that is within an $\delta$ \todo{check this is the right delta} distance of the chip-logged weight shard $\omega_{i+k}$.
% If all the verification protocols passed, the cluster outputs that the Prover has ``passed''.
% If not, the cluster outputs that the Prover ``failed'', prompting an investigation.

% Assuming the training transcript is verified as correct, the Verifier can now compute any functions of the training transcript that would determine its compliance with agreed rules.
% This could include properties of its training data distribution (which can be established from a randomly-selected subset of the training transcript's data), the performance of the final model on specific benchmarks, and properties of the hyperparameters.
% Assuming the results confirm the Prover's compliance, the Verifier could be certain that with probability at least $1 - p - \delta_2$ \todo{make sure p is the right notation from the previous section}, the Prover has not used ML chips to execute a training runs using greater than $C$ compute-hours which violated the agreed upon rules.
% The Prover has also not disclosed any sensitive information to the Verifier, including training data, model weights, or hyperparameters.

% Beyond this, we also need to ensure that the ongoing cost of operating the system is not prohibitive.
When evaluating which training transcript verification protocols to pursue, Verifiers should prioritize those verification strategies that get the most detection-probability, for the lowest cost.
Beyond the upfront costs of building trusted clusters or modifying chip hardware, the system has three ongoing operating costs: 
the efficiency loss from pausing to save weight checkpoints and the weight-shard snapshots (as described in Section \ref{s.onchip}), the storage costs for maintaining training transcripts (and in particular the weight-checkpoints, each of which may require terabytes) until the Verifier inspects them, and the compute costs to execute the training-transcript verification protocols on the trusted clusters.
These costs seem likely to scale linearly with the total compute used by the Prover, and will ultimately depend on the efficiency with which training transcripts can be verified.
Even though governments could in principle pressure Provers into paying the costs of compliance, a 1\% overhead for each dollar spent on training compute would be much easier for Provers to comply with than a 20\% overhead.
Indeed, for International Verifiers, the history of arms control suggests that maximally-stringent verification measures may have limited utility, as they may reduce the likelihood of compliance \cite{reddie2019governing}.
One important avenue for future work is finding cheaper, lower-fidelity alternatives to NN-retraining-based verification, which need only establish limited properties of the weight-shard's corresponding training run, and which could prompt more expensive verification methods if needed.
\section{At the supply chain}\label{s.supplychain}
We need supply-chain monitoring to accomplish two goals: to construct a ``chip directory'' of who owns each ML chip, for the purposes of sampling; and to ensure that each chip has the hardware features needed to provably log its training activity as in Section \ref{s.onchip}.
Unlike the chip and data-center interventions (Sections \ref{s.onchip} and \ref{s.datacenter}), monitoring the international ML chip supply chain cannot be done by a single Verifier.
Instead, an international consortium of governments may need to implement these interventions on behalf of other Verifiers (much as the IAEA runs inspections on behalf of member states).

\subsection{Creating a chip-owner directory}\label{s.chipowner}
For a Verifier to be confident that a Prover is reporting the activity of all the Prover's ML chips, they need to know both which ML chips the Prover owns, and that there are no secret stockpiles of chips beyond the Verifier's knowledge.
Such ownership monitoring would represent a natural extension of existing supply chain management practices, such as those used to enforce U.S. export controls on ML chips.
It may be relatively straightforward to reliably determine the total number of cutting-edge ML chips produced worldwide, by monitoring the production lines at high-end chip fabrication facilities.
The modern high-end chip fabrication supply chain is extremely concentrated, and as of 2023 there are fewer than two dozen facilities worldwide capable of producing chips at a node size of 14nm or lower \cite{semiconductorwiki}, the size used for efficient ML training chips.
As \cite{baker2023nuclear} shows, the high-end chip production process may be monitorable using a similar approach to the oversight of nuclear fuel production (e.g., continuous video monitoring of key machines).
%\citet{baker2023nuclear} provides a thorough exploration of how the high-end chip production process may be monitored using the tools from nuclear fuel production monitoring (e.g., continuous video monitoring of key machines).

As long as each country's new fab can be detected by other countries (e.g., by monitoring the supply chain of lithography equipment), an international monitoring consortium can require the implementation of verification measures at each fab, to provide assurances for all Verifiers.
After processing, each wafer produced at a fab is then sent onward for dicing and packaging.
Since the facilities required for postprocessing wafers are less concentrated, it is important for the wafers (and later the dies) to be securely and verifiably transported at each step.
If these chip precursors ever go missing, responsibility for the violation would lie with the most recent holder.
This chain of custody continues until the chip reaches its final owner, at which point the chip's unique ID is associated with that owner in a \emph{chip owner directory} trusted by all potential Verifiers and Provers.
This ownership directory must thereafter be kept up-to-date, e.g., when chips are resold or damaged.\footnote{In the rare scenario where a large number of chips owned by the same Prover are lost or destroyed beyond recognition, the Verifier or international consortium can launch an investigation to determine whether the Prover is lying to evade oversight.}
The continued accuracy of this registry can be validated as part of the same random sampling procedure discussed in Section \ref{s.inspection}.
As a second layer of assurance, chips could also be discovered by inspecting datacenters, if those datacenters are detectable via other signals \cite{baker2023nuclear}.

Given the high prices and large power and cooling requirements of these ML chips, they are largely purchased by data-center operators.
These organizations are well-suited to tracking and reporting transfers of their ML chips, and complying with occasional inspections. 
Though a small fraction of data-center ML chip purchases are made by individuals, so long as these are a small fraction of chips they may be exempted from the overall monitoring framework.

\subsection{Trusting secure hardware}
We require in Section \ref{s.onchip} that each ML chip produced by the semiconductor supply chain is able to provably log traces of its usage.
The second goal of supply-chain monitoring is to provide Verifiers with high confidence in the reliability of these on-chip activity-logging mechanisms.
This requires ML chip designers to integrate security features into their hardware and firmware designs, especially in ways that make them externally-legible to Verifiers that may not trust the chip-designer.
Key priorities include the immutability of the chip's burned-in ID, the integrity of the hardware-backed mechanism for only booting signed firmware, and the resilience of the on-chip hardware-roots-of-trust to side-channel attacks that could steal the chip's encryption keys \cite{crypto-1996-1469, 6271612} and thus fake its logs.

A concern for Verifiers checking the conduct of powerful Provers (e.g., states verifying each others' ML training runs) is the possibility of supply-chain attacks \cite{robertson_riley_2021}, which could enable a Prover to undetectably disable/spoof the ML chips' logging functionality.
Fully mitigating the threat of supply-chain attacks is a major global issue and beyond the scope of this paper.
However, one particularly useful step for building trust in ML chip mechanisms' integrity would be for ML chip designers to use open-source Hardware-Roots-of-Trust.
This transparency means that chips' designs can be validated by untrusting Verifiers to confirm there are no backdoors.
For example, Google's Project OpenTitan has produced such an HRoT \cite{opentitan}, and many major ML chip designers (Google, Microsoft, NVIDIA, and AMD) have agreed to integrate the Open Compute Project's ``Caliptra'' Root of Trust. \cite{caliptra}

\section{Discussion, Limitations, and Future Work}

Despite many visual interpretation works for DL, \textit{\textbf{the unique values of our work}} come from the following perspectives. 
First, our work presents a comprehensive interpretation of ViTs and discloses insightful findings.
For example, heads with strong self-attentions
are dominantly important. Lower- and higher-layer heads show different local/global attention strengths. Also, we summarize all possible attention patterns between patches. These insights open the hood of ViTs and deepen model designers' understanding.
Second, our interpretation triggers model improvement ideas, e.g., pruning heads with repeating patterns. Thus, improving ViTs with our derived insights would be a direct follow-up work.
Lastly, although we focus only on the classification task, we believe our interpretations are transferable to ViT-based detection/generation tasks~\cite{khan2021transformers}, as those tasks also significantly rely on the multi-head self-attentions of ViTs.

\textbf{\textit{Head-Centric v.s. Image-Centric.}}
We want to emphasize that all our analyses are \textit{head-centric}, and each head's behavior is analyzed in one and across all images.
Specifically, for \textit{head importance}, we provide each head's \textit{local} importance on one image and \textit{global} importance over all images (Sec.~\ref{sec:head_imp_vis}). For \textit{head attention strength}, we present a head's attention strengths in one image (Fig.~\ref{fig:teaser}-C2) and its strength distribution over all images (Fig.~\ref{fig:teaser}-C3). For \textit{head attention pattern}, the two-axes/heatmap (Fig.~\ref{fig:teaser}-D3, D4) shows the attention pattern of a head from one image, while the scatterplot in Fig.~\ref{fig:teaser}-D1 lays out the head's attention pattern over all images. From a different perspective, we believe \textit{image-centric} analysis would also lead to insightful findings, e.g., checking if the heads show similar patterns for images of the same class. We plan to explore this direction in the future.

\textbf{\textit{Performance.}} 
To guarantee the exploration interactivity, we have pre-computed some of the visualization data.
For example, the head importance metrics are computed offline as they can take hours. The partial pruning in Fig.~\ref{fig:teaser}-B2 is performed online and each computation takes about 0.6 seconds on an Nvidia Titan RTX GPU. The head attention strengths and the tSNE layout for head attention patterns are both computed offline as they only need to be computed once and directly plugged into our system.
%
In terms of storage, the raw attention weights consume the most space, ranging from 11 GB to 178 GB, depending on the number of heads in the studied ViT. Other data (e.g., images, probabilities, tSNE results) take about 300 MB in total.

\textbf{\textit{Limitations and Future Work.}} Our head importance analysis relies on leave-one-out ablations, which do not consider the interaction between heads. In some cases, one head could be important only if another head is pruned. The analysis can be further extended to higher-order interactions, which is our planned future work. 
Second, our current analysis focuses on the attentions between two consecutive attention layers only. In the future, we would like to explore attention aggregation methods, e.g.,~\cite{abnar2020quantifying}, to interpret heads' impact across multiple layers.
Lastly, we plan to investigate if the head importance, head attention strengths, and head attention patterns show any class-specific or dataset-specific trends. This will help to diagnose class-related performance issues and validate our findings in more datasets.


\section*{Acknowledgements}
The author would like to thank 
Tim Fist,
Miles Brundage,
William Moses,
Gabriel Kaptchuk,
Cynthia Dwork,
Lennart Heim,
Shahar Avin,
Mauricio Baker,
Jacob Austin,
Lucy Lim,
Andy Jones,
Cullen O'Keefe,
Helen Toner,
Julian Hazell,
Richard Ngo,
Jade Leung,
Jess Whittlestone,
Ariel Procaccia,
Jordan Schneider,
and Rachel Cummings Shavit
for their helpful feedback and advice in the writing of this work.

% \bibliographystyle{alpha}
% \bibliography{refs.bib}
\printbibliography


\appendix
% \section{Analogies and disanalogies to nuclear}
% \copied{1. main thing to highlight earlier when linking is that it’s debated by scholars why nuclear verification succeeded, and the similarities vs. differences with AI training will determine whether AI will succeed for a similar reason. We give arguments why it may be easier, and why it may be harder.
% 2. Analogies
%     1. Both involve using a flow (centrifuges/chips) to  aggregate a stock (total training time)
%     2. Small amounts are fine, large amounts are bad
%     3. Positive economic use-cases that everyone should benefit from, negative misuse use-cases that we should limit to the extent possible.
% 3. Disanalogies
%     1. With uranium, after the enrichment occurs, you can still track the physically-produced uranium. (I.e. violations are reversible.) With compute, once a model has been trained, it can be copied at will.
%     2. With nuclear, most countries that have pursued their own nuclear program have eventually been able to discretely build their own centrifuges (e.g. with design-support from AQ Khan). With advanced compute, this seems very unlikely - fabs cost many billions, and are rarely spun up for a single purpose. (Even the US military failed at this.) Thus, compute supply chain is much more concentrated - less possible to “build an AI project in a bunker somewhere” detached from existing supply chains.
%     3. With HEU, inspecting the end-product is sufficient to know how enrichment was done. With compute, it’s hard to determine “how much an NN was trained” without being provided additional information on the process by which we arrived at those weights.
%     4. With HEU, we know ahead of time how much enrichment is sufficient to be dangerous (as there are physical requirements to causing a supercritical fission  reaction). However with AI, algorithmic progress means that the threshold for any particular dangerous use-case decreases over time. That said, “more compute” will always be riskier than “less compute”, and thus a useful heuristic for regulation. Even if relevant thresholds need to change over time, so long as the requirements do not shrink to the point where detection would be impossible, it is still just as important to have a governing framework for compute.
%     5. With HEU, enrichment must happen (mostly) at the same location. With AI training, enrichment can be parallelized over the internet, although there is some critical concentration required per location.
%     6. AI is much more civilian-economically valuable than uranium, and implemented with more parts of the supply chain. Many analysts believe that part of the success of the NPT is that states do not have a strong incentive to cheat as they’re part of security alliances; it is not clear whether this will be the same for AI. It may be that similar alliances are required.
% 4. Inspection mechanisms from nuclear we’d like to copy
%     1. Tracking centrifuge production
%     2. Centrifuge flow monitoring
%     3. Accounting for total usage of centrifuges (i.e. being able to show results)
% }

% \section{Case study in a way this could break: student-teacher}
% \ot{One way to break this scheme: rather than one single long training run, to evade detection, the model can be repeatedly self-distilled into a new model, and then that new model trained for an additional period. This increases a compute-overhead, but so long as the distilling time does not scale linearly with the training time, this can make it possible to hide longer runs as a series of shorter runs. However, the total compute required (and the total time that the in-RAM model is of sufficiently-low-loss to trigger an audit) is still large, meaning that a randomly-sampled chip would still need to attest to being part of a dangerous training run.
% Note also that this behavior would look kind of weird, because each self-distillation run would be using a very large number of chips *in parallel* for a short time. This is because such self-distillation-chains must occur sequentially across time, and at each timestep all the resources go into a particular snapshot.}

\newpage

\section{Discussion on future training requirements}\label{app.howmuch}

\subsection{Will the most capable ML models require large-scale training?}
This paper's proposed framework is premised on the assumption that large-scale training is and continues to be a necessary requirement for the most advanced (and thus most dangerous) ML models.
There is intense disagreement within the field about how important large-scale training is, and how long that will remain the case.

Many of the recent breakthroughs in machine learning model capabilities, across every domain, have come from increasing the model size or quantity of training data, each of which corresponds to a greater usage of compute \cite{kaplan2020scaling, hoffman2022training, zhai2021scaling}.
Indeed, some capabilities, such as chain-of-thought reasoning, appear to only emerge at the largest training scales \cite{wei2022chain}.
At the same time, any one narrow capability can often be achieved with a much smaller compute budget \cite{magister2022teaching, madani2023large}.
Nonetheless, Sutton's ``Bitter Lesson'' \cite{sutton2019bitter} that ``general methods that leverage computation are ultimately the most effective'' is a frequent diagnosis of the likely future of deep learning.
Though algorithmic progress \cite{erdil2022algorithmic} and the continued progress of Moore's Law will continue to reduce the number of chips required for any specific capability, we may compensate by gradually increasing enforcement parameters to work for smaller quantities of specialized compute.
At the same time, the increasing investment in compute by frontier AI firms \cite{lardinois_2022, wiggers_2022} suggests that industry insiders continue to believe that the most capable frontier models --- likeliest to yield new capabilities and surface new risks to public safety --- are expected to require ever more compute.

\subsection{Will large-scale training continue to require specialized datacenter chips?}

Nearly all large-scale training runs are executed on high-end datacenter accelerators \cite{chowdhery2022palm, kaplan2020scaling, zeng2022glm}.
The main difference between these chips and their consumer-oriented counterparts is their much higher inter-chip communication bandwidth (e.g., 900GB/s for the NVIDIA H100 SXM vs. 64GB/s for the NVIDIA GeForce RTX 4090 \cite{nvidiah100, nvidia4090}).
This extra bandwidth is today crucial for parallelizing NN training, especially tensor parallelism and data parallelism, which require frequent transfers of large matrices between many chips \cite{smith2022computation}.
Organizations doing large-scale training also favor these datacenter chips for other reasons: they are generally more energy efficient, and license requirements often prevent organizations from placing consumer-oriented chips in datacenters\cite{moss_2023}.

Still, recent work has suggested it may be \emph{possible} to do large-scale training on consumer chips with low interconnect, though with substantial cost and speed penalties\cite{binhang2022distributed, ryabinin2023distributed}.
If such methods become feasible for bad actors, then we may need to adjust to a different regulatory model for detecting training activity.
Possibilities include focusing on spotting and monitoring datacenters (similar to the IAEA's work to detect undeclared nuclear facilities \cite{harry1996iaea}), or regulating the high-capacity switches that could be necessary to enable fast networking between low-interconnect chips.
So long as they can be detected, it may be possible to retrofit consumer chips (e.g. with a permanently-mated host CPU, see Section \ref{s.onchip}) to enable similar monitoring capabilities.

It is important to note that the current framework \emph{does} apply in the setting where clusters of chips are split across several datacenters (e.g. multiple cloud providers), so long as these high-end chips are used at each datacenter.

\section{Derivation of Sampling Rate}
\label{a.samplingderivation}
We provide a derivation of Equation \ref{eq:samplingrate}, the number of samples required for a Verifier to catch a weight snapshot from a rule-violating training run with a given probability $p$.
Let $\rulecomputeflops$ be the size of an ML training run that the Verifier is hoping to catch.
Let $\totalchips$ be the total number of chips the Prover possesses, and $a$ be the FLOPs/day for those chips. 
Let $f$ be the expected number of weight-snapshots saved by the chip per day; snapshots are Poisson distributed.
The Verifier wants to detect a rule-violating training run of length $\trainingperiod$ that was completed in the last $\monitoringperiod$ days, and will sample $s$ chips every $\monitoringperiod$.
We will assume that the Prover executes the training run over the course of $\trainingperiod$ days. 
(We show in Appendix \ref{a.provertricks} that changing $\trainingperiod$ only marginally affects our analysis.)

It is not enough that a chip involved in the training run be sampled by the Verifier; the chip needs to have also logged a weight-snapshot from this particular training run, in order for there to be something for the Verifier to discover.
The probability of a chip logging a weight-snapshot is uniform over time (Section \ref{s.onchip}). 
Assuming the training run began at the beginning of the first monitoring period of length $\monitoringperiod$, the probability that a snapshot was sampled is $1 - PoissonCDF_{T;f}(0) = 1 - e^{-fT}$, where $PoissonCDF_{T;f}$ is the CDF of a Poisson random variable with shape parameter $f$ and interval-length $T$.
If the training run lasted multiple monitoring periods, then each later period, each chip is strictly more likely to contain a snapshot than this first period.
\footnote{There are two edge cases.
First, the Prover could choose to use extra chips and thus shrink $\trainingperiod < \monitoringperiod$.
However, in Appendix \ref{a.trainingshorter}, we show this would not improve the likelihood of avoiding detection due to a snapshot not being included.
The other edge case is when $\trainingperiod$ is not perfectly divisible by $\monitoringperiod$, leading to the first round of samples occurring when each chip has participated in the training run for less than $\monitoringperiod$ time.
This means that the likelihood of each sample in that round containing a snapshot is slightly reduced, thus reducing the likelihood of detection in that round, and therefore of detection at any of the $\trainingperiod/\monitoringperiod$ periods of the training run.
At worst, this creates a delay of one extra monitoring period $\monitoringperiod$ after the training run ended, since each sample from that $\lceil \trainingperiod / \monitoringperiod \rceil$'th period is just as likely to contain a snapshot as the other periods.
That's because the weight snapshots remain on the chip even after the training run has ended.
We exclude this one-period fudge factor from our notation for brevity.}

Thus, we let the probability that a chip contains a snapshot be:

\[
p_w \geq 1 - e^{-fT}
\]

We assume that the Prover used the minimum number of chips possible to complete the training run within the time period, $\frac{\rulecomputeflops}{a\trainingperiod}$ chips.
Each time a Verifier samples a chip, the probability that it is one of the chips used in the training run is $\frac{\rulecomputeflops}{a \trainingperiod\totalchips}$.
The probability that the Verifier samples a chip, and that it is both used in the training run and contains a logged weight snapshot is $p_{\text{1-sample}} = \frac{\rulecomputeflops}{a\trainingperiod\totalchips} p_w $.

The probability that the Verifier \emph{fails} to sample a single involved chip with a logged snapshot during a single monitoring period that includes $s$ samples is equivalent to a weighted coin with $p_{\text{1-sample}}$ probability of heads coming up tails $s$ times:
\[
p_n = (1 - p_{\text{1-sample}})^s = \left( 1 - \frac{\rulecomputeflops}{aT\totalchips}\left(1 - e^{-fT} \right)\right)^s
\]

There are $\lceil\trainingperiod/\monitoringperiod\rceil$ rounds of sampling for each training run, including the last period (which may not include a full training run, but each chip will still store the relevant weight snapshot hashes).

This means the probability of successfully detecting a snapshot from a large-scale training run within $\monitoringperiod$ days of the training run's completion is $p = (1 - p_n)^{\lceil \trainingperiod/\monitoringperiod \rceil}$.

Rearranging terms, we get:
\begin{align*}
p &= 1 -\left( 1 - \frac{\rulecomputeflops}{a\trainingperiod\totalchips} p_w \right)^{\lceil \trainingperiod / \monitoringperiod \rceil s}
\geq 
1 -\left( 1 - \frac{\rulecomputeflops}{a\trainingperiod\totalchips}\left(1 - e^{-f\monitoringperiod} \right)\right)^ {\lceil \trainingperiod / \monitoringperiod \rceil s} \\
\log(1 - p) &\leq s 
\left \lceil \frac{\trainingperiod}{\monitoringperiod} \right \rceil 
\log\left( 1 - \frac{\rulecomputeflops}{a\trainingperiod\totalchips}\left(1 - e^{-f\monitoringperiod} \right)\right) \\
s &\leq \frac{ \log (1-p)}{\log\left( 1 - \frac{\rulecomputeflops}{a\trainingperiod\totalchips}\left(1 - e^{-f\monitoringperiod} \right)\right)}
\frac{1}{\left \lceil \frac{\trainingperiod}{\monitoringperiod} \right \rceil}
\end{align*}
To catch a rule-violating training run snapshot with probability $p$, a Verifier needs to sample no more than $s$ chips in each monitoring period $\monitoringperiod$.
\section{Analysis of prover evasion strategies}
\label{a.provertricks}
\subsection{Do the Prover's odds of evasion benefit from training for longer, across fewer chips?}
\label{a.traininglonger}
One concern is that a Prover might manage to evade detection of a large-scale training run by using a smaller number of chips over a long period of time. 
A hypothesis for why this might be beneficial is that using a smaller number of chips makes the likelihood of any one of them being sampled extremely small.
However, if the Prover uses $\frac{c}{k}$ chips, this increases the training run's length $\times k$, meaning that the number of rounds of sampling similarly increases $\times k$.
We explore which of these effects dominates.

This can be expressed in the following formula for $p_d(k)$, the probability of detection as a function of the multiple increase of the training length. (For simplicity, we assume that the snapshotting frequency $f$ is high enough that every sampled chip contains a snapshot.)
\[
p_d(k) = 1 - \left ( 1 - \frac{\rulecomputeflops}{\totalchips a \trainingperiod k}\right )^{ks}
\]
where $a$ is the chips' FLOPs per day, $\trainingperiod$ is the original training period, $\totalchips$ is the total size of the Prover's chip stockpile being sampled from, and $\rulecomputeflops$ is the quantity of compute required for a large-scale training run.

We assume that $s$ is set by the Verifier in order to achieve a particular detection probability $p$ assuming the original (un-extended) training period $T$, via the Equation \ref{eq:samplingrate}, reproduced here:
\[
s(p) = \frac{\log (1-p)}{\log \left (1 - \frac{\rulecomputeflops}{\totalchips a \trainingperiod}\right )}
\]

In Figure \ref{fig:longertimes}, we estimate how $p_d(k)$, the probability of detection given a $\times k$ longer run, changes with $k$ and other hyperparameters. For simplicity, we define $b = \rulecomputeflops/(\totalchips a T)$, where we know that $1>b>0$.

Somewhat surprisingly, $p_d(k)$ changes very little as $k$ increases, and this remains true for almost all values of $b$.
\begin{figure}
    \centering
    \includegraphics[width=0.5\linewidth]{figures/heatmap2.png}
    \caption{Ratio of $p_d(k)$ to original $p = p_d(1)$, setting $p=0.9$ and varying $k$ from $1$ to $100$ and $b = \rulecomputeflops/(\totalchips a \trainingperiod)$ from $0$ to $0.2$ (where the latter end corresponds to the Prover using 20\% of all chips, making such a large-scale training run easily catchable with even a dozen samples). Across all values, the Prover's reduction in probability of detection is less than 5\%, even when training for up to $100 \times$ longer.
    }
    \label{fig:longertimes}
\end{figure}
This means that, for nearly all training run sizes (affecting $b$) and desired Verifier confidence levels $p$, the Prover cannot meaningfully decrease their probability of having their rule-violating training run detected by the Verifier by simply using fewer chips for longer.

\subsection{Do the Prover's odds of evasion benefit from splitting training across more chips?}
\label{a.trainingshorter}

A Prover may also exploit each chip's low snapshotting frequency to train across more chips than necessary, with the knowledge that many of the used chips will not contain a single snapshot.
Thus, the Verifier might sample more chips that were involved in a rule-violating training run, but never detect their involvement.

However, since the total amount of compute-time stays constant, the total number of snapshots stays constant, and using more chips only \emph{increases} the probability of detection.
We formalize this intuition with the following proof.

\begin{theorem}
For a given firmware snapshotting frequency $f$ and maximum training time $\trainingperiod$, a Prover executing a rule-violating training run cannot reduce the probability of one of the training run's snapshots being detected (that is, of having at least one of the Verifier's sampled chips include a snapshot from the rule-violating training run) by splitting the training run across a greater number of chips, each processing for less time.
\end{theorem}

\begin{proof}

We prove this in the case where training is done for at most one monitoring period $\trainingperiod = \monitoringperiod$.
The extension to multiple monitoring periods proceeds automatically.

Let the total compute required for a rule-violating training run be $\rulecomputeflops$ FLOPs, and the total number of chips available to the Prover is $\totalchips$.
Given that each chip can process $a$ FLOPs per day, to complete the training run in $t$ time, we need to use $c(t) = \rulecomputeflops/(at)$ chips.

For a chip used in the training run for $t$ time, the probability $p_{ns}$ that no snapshot was saved is the CDF of a Poisson random variable with rate parameter $f$:
\[
p_{ns}(t) = e^{-ft}
\]
Let the Verifier's sampling rate be a total of $s$ chips, which we assume occurred after the end of the training run (to avoid complexity due to chip samples midway through the training run having a lower probability of having logged a snapshot than later samples in the same training run).

If $c(t)$ chips are used, each for $t$ time, then the overall probability of detection $p_d(t)$ is
\[
p_d(t) = 1 - \left (1 - \frac{c(t)}{\totalchips} \left (1 - p_{ns}(t) \right ) \right )^s = 1 - \left (1 - \frac{\rulecomputeflops}{a\totalchips t} \left (1 - e^{-ft} \right ) \right )^s
\]
We want to prove that if the training run uses more chips than necessary $c(t) > c(T)$, each for less time $t < T$, then detection is always more likely $p_d(t) \overset{?}{>} p_d(T)$.

Let $a = \frac{T}{t} \geq 1$, and let $b = e^{-fT}$. We know $1>b>0$ because $f$ and $T$ are positive.

\begin{lemma}
If $g \geq 1$ and $1>b>0$, then $f(a) = a \left (1-b^{\frac{1}{g}} \right ) - (1-b) \geq 0$.
\end{lemma}
\begin{proof}
\begin{align*}
f(0) &= (1-b) - (1-b) = 0 \\
\frac{df(g)}{dg} &= 1 - b^{\frac{1}{g}} + \frac{1}{g}b^{\frac{1}{g}} \\
&= 1 - \underbrace{\left (1 - \frac{1}{g} \right )}_{\leq 1}\underbrace{b^{\frac{1}{g}}}_{\leq 1} \\
&\geq 0
\end{align*}
Given that $f(g=0) \geq 0$ and its derivative is continuous and nonnegative thereafter, we know that $f(g) \geq 0$ for all $g \geq 1, 1>b>0$.
\end{proof}
Plugging in our earlier settings for $a$ and $b$, we have:
\begin{align*}
   g\left (1 - b^{\frac{1}{g}} \right ) - (1-b) &\geq 0 \\
   g\left (1 - b^{\frac{1}{g}} \right ) &\geq 1-b \\
   \frac{1}{t}\left(1 - e^{-ft}\right) &\geq \frac{1}{T}\left(1 - e^{-fT}\right ) \\
   \frac{\rulecomputeflops}{a \totalchips t}\left (1 - e^{-ft}\right ) &\geq
   \frac{\rulecomputeflops}{a \totalchips T}\left (1 - e^{-fT} \right ) \\
   1 - \left ( 1 - \frac{\rulecomputeflops}{a \totalchips t}\left (1 - e^{-ft} \right ) \right )^s &\geq 
   1 - \left ( 1 - \frac{\rulecomputeflops}{a \totalchips T}\left (1 - e^{-fT} \right ) \right )^s\\
   p_d(t) &\geq p_d(T)
\end{align*}
\end{proof}

Note that this proof applies whether the extra chips are being used simultaneously in parallel, or split across time (e.g. one group of chips for the first half of the training run, and a second group of chips for the second half).

While this result might seem to contradict the result from Subsection \ref{a.traininglonger}, the two findings are in fact consistent. The key assumption is that the snapshotting frequency $f$ is high enough for the chips to contain at least one logged snapshot per period $T$ with high probability.
Then Subsection \ref{a.trainingshorter} tells us that using fewer chips reduces the Prover's probability of detection, and Subsection \ref{a.traininglonger} tells us that this decrease in detection is small, and likely not worth it to the Prover given the resulting delay in the training run's completion.
\section{Logistics for Confidentially Verifying Proofs-of-Training-Transcript}
\label{a.trustedcluster}

In this section, we consider the logistical problem of how to verify, for all the Verifier's observed weight-snapshots from sampled chips, that the Prover has provided authentic training transcript that match each weight-shard. We assume the PoTT verification protocol looks similar to the  Proof-of-Learning scheme of \cite{jia2021proof}.
Our solution is specifically designed to avoid the Prover ever having to directly reveal the hyperparameters $\mathbb{M}$, training data $\mathbb{D}$, and model weights $\mathbb{W}$ to the Verifier, as in many cases these data are either private or proprietary.

First, when executing each training run, the Prover follows any required\footnote{Of course, these restrictions must be retroactively verifiable using the training transcript, as otherwise the Prover might simply not comply.}
restrictions on their training (e.g., starting from a subset of valid initializations, or precommitting to the model weights).
Throughout training, the Prover saves the hyperparameters (e.g., training code), the sequence of training data batches, and periodic snapshots of the model weights.
They also make sure to track any randomness used, and generally make sure their training process is repeatable (up to low-level noise) assuming access to similar but non-identical hardware configurations.
This itself is a nontrivial technical challenge \cite{liu2021reproducibility}, especially in co-training settings like GANs or RL on learned reward models, and would benefit from further investments in replication tools for common ML frameworks.

Periodically, the Prover reports their chips' on-chip logged weight-shard snapshot hashes, along with hashed versions of the Proofs-of-Training-Transcript matching each of these logged hashes (including hashes of $\mathbb{M}$, hashes of each model snapshot $\mathbb{W}$, and sequences of hashes of training data points $\mathbb{D}$).
The Prover may need to further disclose basic information like the distance between weight-snapshot pairs, which may be needed for the Verifier to determine which regions of the transcript to verify.\footnote{This info could be proven to the Verifier securely and privately, for example by using standard ZK-SNARK proof tools to confirm that two given hashes correspond to vectors that have a given $L_2$ distance between them.}

As in Section \ref{s.sampling}, the Verifier chooses a subset of these chips whose logs will be confirmed via physical inspection.
The Verifier inspects the training transcripts and associated metadata, and chooses the subsets of the training transcripts to verify.
They also select any other components of the training transcript (e.g., a random sample of the training data) that they will need in order to determine that that the training run has complied with the ML training rules.
At this point, the Verifier has a set of hashes of the inputs to its verification protocol (including model weight snapshots and training data points), but does not have access to any of this information directly.

The Verifier needs to have faith in the results of the computation, but without seeing the inputs themselves.
In principle, this might be addressed by cryptographic techniques like zero-knowledge proofs, but in practice these techniques cannot efficiently execute computationally intensive programs, like long sequences of gradient updates on billion-parameter models.
Instead, the Prover and Verifier can agree on a neutral jointly-trusted cluster, which they will use to execute the verification protocol.
The Verifier needs to be able to trust in the integrity of the cluster's computation.
The Prover, conversely, needs to trust that the cluster will not reveal its private data, by verifying that it cannot communicate with the outside world and has no persistent storage.
Such neutral clusters could be maintained by a trusted third party, or when no such trusted party exists, could be jointly overseen by both the Prover and Verifier. 
It may be useful to create one trusted cluster for each type of ML training chip, in order to mitigate potential hardware compatibility issues.

At agreed-upon intervals, the Verifier supplies the hashes of the inputs (including hyperparameters $\mathbb{M}$) to the neutral cluster, and the Prover supplies the inputs themselves.
The cluster hashes the Prover's inputs and confirms they match the hashes.
Then the cluster executes the Verifier's verification protocol.
For example, when re-executing a training segment to verify that it would reach a particular chip-logged weight shard snapshot $\omega_{i+k}$, the cluster starts at one weight snapshot $W_i$, and then computes optimizer updates $k$ times using the specified inputs (where each update is computed via code generated from the hyperparameters, and even code snippets, defined by $\mathbb{M}$).
Finally, the cluster checks that the resulting weight vector $W'_{i+k}$ has a slice $\omega'_{i+k}$ that is within an $\epsilon$ distance of the chip-logged weight shard $\omega_{i+k}$.
If all the verification protocols passed, the cluster outputs that the Prover has ``passed''.
If not, the cluster outputs that the Prover ``failed'', prompting an investigation.

Assuming the training transcript is verified as correct, the Verifier can now compute any functions of the training transcript that would determine its compliance with agreed rules.
This could include properties of its training data distribution (which can be established from a randomly-selected subset of the training transcript's data), the performance of the final model on specific benchmarks, and properties of the hyperparameters.
Assuming the results confirm the Prover's compliance, the Verifier could be certain that with probability at least $1 - p - \delta_2$, the Prover has not used ML chips to execute a training runs using greater than $\rulecomputeflops$ FLOPs which violated the agreed upon rules.
The Prover has also not disclosed any sensitive information to the Verifier, including training data, model weights, or hyperparameters.


\end{document}

