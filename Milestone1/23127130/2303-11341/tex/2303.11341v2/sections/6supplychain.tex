\section{At the supply chain}\label{s.supplychain}
We need supply-chain monitoring to accomplish two goals: to construct a ``chip directory'' of who owns each ML chip, for the purposes of sampling; and to ensure that each chip has the hardware features needed to provably log its training activity as in Section \ref{s.onchip}.
Unlike the chip and data-center interventions (Sections \ref{s.onchip} and \ref{s.datacenter}), monitoring the international ML chip supply chain cannot be done by a single Verifier.
Instead, an international consortium of governments may need to implement these interventions on behalf of other Verifiers (much as the IAEA runs inspections on behalf of member states).

\subsection{Creating a chip-owner directory}\label{s.chipowner}
For a Verifier to be confident that a Prover is reporting the activity of all the Prover's ML chips, they need to know both which ML chips the Prover owns, and that there are no secret stockpiles of chips beyond the Verifier's knowledge.
Such ownership monitoring would represent a natural extension of existing supply chain management practices, such as those used to enforce U.S. export controls on ML chips.
It may be relatively straightforward to reliably determine the total number of cutting-edge ML chips produced worldwide, by monitoring the production lines at high-end chip fabrication facilities.
The modern high-end chip fabrication supply chain is extremely concentrated, and as of 2023 there are fewer than two dozen facilities worldwide capable of producing chips at a node size of 14nm or lower \cite{semiconductorwiki}, the size used for efficient ML training chips.
As \cite{baker2023nuclear} shows, the high-end chip production process may be monitorable using a similar approach to the oversight of nuclear fuel production (e.g., continuous video monitoring of key machines).
%\citet{baker2023nuclear} provides a thorough exploration of how the high-end chip production process may be monitored using the tools from nuclear fuel production monitoring (e.g., continuous video monitoring of key machines).

As long as each country's new fab can be detected by other countries (e.g., by monitoring the supply chain of lithography equipment), an international monitoring consortium can require the implementation of verification measures at each fab, to provide assurances for all Verifiers.
After processing, each wafer produced at a fab is then sent onward for dicing and packaging.
Since the facilities required for postprocessing wafers are less concentrated, it is important for the wafers (and later the dies) to be securely and verifiably transported at each step.
If these chip precursors ever go missing, responsibility for the violation would lie with the most recent holder.
This chain of custody continues until the chip reaches its final owner, at which point the chip's unique ID is associated with that owner in a \emph{chip owner directory} trusted by all potential Verifiers and Provers.
This ownership directory must thereafter be kept up-to-date, e.g., when chips are resold or damaged.\footnote{In the rare scenario where a large number of chips owned by the same Prover are lost or destroyed beyond recognition, the Verifier or international consortium can launch an investigation to determine whether the Prover is lying to evade oversight.}
The continued accuracy of this registry can be validated as part of the same random sampling procedure discussed in Section \ref{s.inspection}.
As a second layer of assurance, chips could also be discovered by inspecting datacenters, if those datacenters are detectable via other signals \cite{baker2023nuclear}.

Given the high prices and large power and cooling requirements of these ML chips, they are largely purchased by data-center operators.
These organizations are well-suited to tracking and reporting transfers of their ML chips, and complying with occasional inspections. 
Though a small fraction of data-center ML chip purchases are made by individuals, so long as these are a small fraction of chips they may be exempted from the overall monitoring framework.

\subsection{Trusting secure hardware}
We require in Section \ref{s.onchip} that each ML chip produced by the semiconductor supply chain is able to provably log traces of its usage.
The second goal of supply-chain monitoring is to provide Verifiers with high confidence in the reliability of these on-chip activity-logging mechanisms.
This requires ML chip designers to integrate security features into their hardware and firmware designs, especially in ways that make them externally-legible to Verifiers that may not trust the chip-designer.
Key priorities include the immutability of the chip's burned-in ID, the integrity of the hardware-backed mechanism for only booting signed firmware, and the resilience of the on-chip hardware-roots-of-trust to side-channel attacks that could steal the chip's encryption keys \cite{crypto-1996-1469, 6271612} and thus fake its logs.

A concern for Verifiers checking the conduct of powerful Provers (e.g., states verifying each others' ML training runs) is the possibility of supply-chain attacks \cite{robertson_riley_2021}, which could enable a Prover to undetectably disable/spoof the ML chips' logging functionality.
Fully mitigating the threat of supply-chain attacks is a major global issue and beyond the scope of this paper.
However, one particularly useful step for building trust in ML chip mechanisms' integrity would be for ML chip designers to use open-source Hardware-Roots-of-Trust.
This transparency means that chips' designs can be validated by untrusting Verifiers to confirm there are no backdoors.
For example, Google's Project OpenTitan has produced such an HRoT \cite{opentitan}, and many major ML chip designers (Google, Microsoft, NVIDIA, and AMD) have agreed to integrate the Open Compute Project's ``Caliptra'' Root of Trust. \cite{caliptra}
