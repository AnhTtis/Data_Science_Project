\section{Conclusions and Further Work} \label{sec:conclusions}

Reo is a widely used tool to model new systems out of the coordination of already existing pieces of software. It has been used in a variety of domains, drawing the attention of researchers from different locations around the world. This has resulted in Reo having many formal semantics proposed, each one employing different formalisms: operational, co-algebraic, and coloring semantics are some of the types of semantics proposed for Reo.

This work extends \relo, a dynamic logic to reason about Reo models. We have discussed its core definitions, syntax, semantic notion, providing soundness and completeness proofs for it.
%We also discussed a Coq implementation of \relo\ and some useful tools to enable its usage in a computational environment, such as the search for a model that satisfies a formula.
\relo\ naturally subsumes the notion of Reo programs and models in its syntax and semantics, and implementing its core concepts in Coq enables the usage of Coq's proof apparatus to reason over Reo models with \relo.

Future work may consider the integration of the current implementation of \relo\ with ReoXplore\footnote{\url{https://github.com/frame-lab/ReoXplore2}}, a platform conceived to reason about Reo models, and extensions to other Reo semantics. Investigations and the development of calculi for \relo\ are also considered for future work.

