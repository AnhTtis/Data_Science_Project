%auto-ignore
In this section we apply the OWL methodology to data from a micro-credit study \citep{angelucci2015microcredit} for which standard methods of parameter inference have been shown to be brittle to the removal of a handful of observations  \citep{broderick2020automatic}. In \cite{angelucci2015microcredit} the authors conducted a (clustered) randomized trial in Mexico to study the impact of availability of micro-credit on outcome measures in the community including  micro-entrepreneurship, income, labor supply, consumption, social status, and subjective well-being. The authors worked with \emph{Compartamos Banco}, one of the largest micro-lenders in Mexico, to randomize their rollout across 238 geographical regions in the north-central Sonora state in Mexico (close to the Mexico and United States border); within 18-34 months after this rollout, the authors surveyed $n=16,560$ households from these regions for various outcome measures to study the impact of the rollout.

While it is possible to perform a detailed analysis using more outcomes and covariates from the survey data \citep{angelucci2015microcredit}, following \cite{broderick2020automatic}, here we focus on the \emph{Average Intention to Treat effect} (AIT) of the rollout on household profits. More precisely for $i \in \{1, \ldots, n\}$, let $Y_i$ denote the profit of the $i$th household during the last fortnight (measured in USD PPP), and let $T_i \in \{0,1\}$ be a binary variable that is one if and only if the household $i$ falls in the geographical region where the credit rollout happened.  The AIT on household profits is defined as the coefficient $\beta_1$ in the linear model:
\begin{equation}
\label{eq:AITmodel}
Y_i = \beta_0 + \beta_1 T_i + \varepsilon_i,\quad \varepsilon_i \iid N(0, \sigma^2),  \quad i \in \{1,\ldots, n\}. \end{equation}

%\new{
To reproduce the brittleness in estimating the AIT on household profits demonstrated in \cite{broderick2020automatic}, we first obtained the profit data (originally from \cite{angelucci2015microcredit}) as imputed and scaled by \cite{meager2019understanding}. The MLE estimate of $\beta_1 = -4.55$ USD PPP per fortnight (standard error [s.e.] of 5.88), changes to $\beta_1 = 0.4$ USD PPP per fortnight (s.e. 3.19) if we remove a single household identified by the \texttt{zaminfluence} R package \citep{broderick2020automatic}. Moreover, by removing 14 further observations which were identified by the \texttt{zaminfluence} package, we observe that the non-significant value of the MLE estimate can be changed to a significant value of $\beta_1 = -6.01$ USD PPP (s.e. 2.57). As seen in a scatter-plot summarizing the data (\Cref{app:micro-credit}, Figure \ref{fig:micro-scatter}), this brittleness of the MLE is likely due to a small fraction of households with outlying profit values.

Here we compare OWL to this data deletion approach by fitting the model \eqref{eq:AITmodel} to the full data set using 50 $\log_{10}$-spaced $\epsilon$-values between $10^{-4}$ and $10^{-1}$, %for the collection of $\epsilon$ values $\{10^{-4 + 3j/50} : j = 0, \ldots, 50\}$, 
and used the tuning procedure in \Cref{sec:tune-epsilon} to obtain the value $\epsilon_0 = 0.005$ where the minimum-OKL versus epsilon plot (\Cref{app:micro-credit}, \Cref{fig:micro-okl-plot}) has its  most prominent kink. We also calculate the MLE, which corresponds to the OWL procedure with $\epsilon = 0$. The AIT on household profit estimated by OWL as a function of $\epsilon$ can be seen in the left panel of \Cref{fig:micro_fig}. For  values of $\epsilon$ below $\epsilon_0$, the AIT estimates change rapidly as $\epsilon$ changes, while for  values of $\epsilon$ above $\epsilon_0$,  the AIT estimates are quite stable with changes in $\epsilon$. This is due to OWL automatically down-weighting the outlying observations, as seen in the right panel of  \Cref{fig:micro_fig}.  

To quantify uncertainty in the AIT estimates obtained by OWL at the aforementioned grid of  values for $\epsilon$, we reran the above analysis on $m=50$ independently bootstrapped data sets of size $n$ each. Since we wanted to retain a small fraction of outlying observations in each data set, we used an \emph{outlier-stratified} (OS) sampling strategy. Namely, in each iteration, the  new data set was obtained by combining a bootstrap sample of the (roughly $1\%$) households that were down-weighted by the OWL procedure at $\epsilon_0$ and a bootstrap sample from the remaining households that were not down-weighted. 

The resulting 90\% OS-bootstrap confidence bands for estimates of AIT and minimum-OKL as a function of $\epsilon$ can be found in \Cref{app:micro-credit} (also see the left panel in \Cref{fig:micro_fig}). From  \Cref{fig:micro-log-scale-plots} in \Cref{app:micro-credit}, the confidence bands for AIT estimates from OWL are much wider when  $\epsilon < \epsilon_0$ than they are when $\epsilon \geq  \epsilon_0$. Hence, if we presume that the outlying households are the ones down-weighted by OWL at  $\epsilon=\epsilon_0$, the relatively narrow bootstrap confidence bands for the AIT estimates at $\epsilon = \epsilon_0$ suggest that OWL is able to successfully prevent brittleness in estimation due to those outliers.

In summary, the OWL procedure chose to down-weight roughly 1\% of the households with extreme profit values and estimated an AIT of $\beta_1 = 0.6$ USD PPP per fortnight based on the selected value of $\epsilon_0 = 0.005$. The value $\epsilon = \epsilon_0$, tuned using the procedure in \Cref{sec:tune-epsilon}, roughly coincides with the point at which the AIT estimates become stable with respect to  $\epsilon$ and also with the point at which the 90\% OS-bootstrap confidence bands for AIT become narrower --- both suggesting that OWL with the choice $\epsilon = \epsilon_0$ has identified and down-weighted outliers that may be causing brittleness in estimating AIT.

\begin{figure}
    \centering
    \includegraphics[width=0.47\textwidth]{figures/micro_ate_linear.pdf}
    \includegraphics[width=0.47\textwidth]{figures/micro_weights.pdf}
    \caption{Estimating the Average Intent to Treat (AIT) effect on household profits in the micro-credit study \cite{angelucci2015microcredit} in the presence of outliers. Left: the AIT estimates using OWL for various values of $\epsilon$ along with $90\%$ OS-bootstrap vertical confidence bands. The vertical line is drawn at the value $\epsilon_0 = 0.005$ obtained by the tuning procedure in \Cref{sec:tune-epsilon}, and roughly coincides with the $\epsilon$ beyond which the AIT estimates stabilize and the size of the confidence bands shrinks (see \Cref{app:micro-credit}). Right: shows that the weights estimated by OWL procedure at $\epsilon = \epsilon_0$ down-weight roughly  $1\%$ of the households that have outlying profit values (for visual clarity, we omit a down-weighted household with profit less that $-40K$ USD PPP); see also \Cref{fig:micro-outlier-hist-plot} in \Cref{app:micro-credit}.}
    \label{fig:micro_fig}
\end{figure}