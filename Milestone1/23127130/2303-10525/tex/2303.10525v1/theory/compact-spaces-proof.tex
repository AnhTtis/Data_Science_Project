%auto-ignore


\subsubsection{Large sample behavior of the coarsened log-likelihood}

In this section, we will assume that $\cX$ is a Polish space (i.e. $\cX$ is a complete and separable metric space) with its collection of measurable subsets given by the Borel sigma field. Denote by $\cP(\cX)$ the set of probability measures on $\cX$. Under the topology of weak convergence, $\cP(\cX)$ is also a Polish space \cite{billingsley1971weak}. The Kullback Leibler divergence (also called the relative entropy) is defined as 

\begin{definition}[KL-divergence] For any two probability measures $P, Q \in \cP(\cX)$
\begin{equation*}
    \KL(Q|P) \doteq \begin{cases} \int \left(\log \frac{dQ}{d P}\right) dQ  & \text{ if } Q \ll P \\
    \infty & \text{ otherwise }
    \end{cases}
\end{equation*}
\label{def:KL}
\end{definition}%
%Under the topology of weak convergence  \cite{billingsley1971weak} (denoted as $Q_n \dconv Q$ for $\{Q_n\}_{n \in \nat}, Q \in \cP(\cX)$), $\cP(\cX)$ is itself a Polish space \cite{billingsley1971weak}.
Here $Q \ll P$ denotes the absolute continuity condition that $Q(A) = 0$ for any measurable subset $A \subseteq \cX$ that satisfies $P(A)=0$. Recall that the Radon Nykodim theorem (see  \cite[Section 32]{billingsley2013convergence}) which shows that $Q \ll P$ if and only if $Q$ has a density $f=\frac{dQ}{dP}$ with respect to $P$, i.e. $Q(A) = \int_A f dP$ for each measurable subset $A$ of $\cX$.

The $\KL$ divergence has many useful theoretical properties. For instance, the $\KL(P|Q)$ is strictly positive when $P \neq Q$ and takes the value zero when $P=Q$. Further, the function  $\KL : \cP(\cX) \times \cP(\cX) \to [0,\infty]$ is jointly convex and  lower-semicontinuous with respect to the weak-topology on $\cP(\cX)$ \cite[Lemma 2.4]{budhiraja2019analysis}. 

Our focus in this section will be the population level approximation to the coarsened log-likelihood $\frac{1}{n} L_\epsilon(\theta|x_{1:n})$ to the rate function $I_\epsilon(\theta)$ \eqref{eq:rate-func}.

If $P, Q$ have densities $p,q \in \Den$ then $\KL(P|Q)$ is equal to $\int p \log (p/q) d\lambda$ provided $p/q$ is well-defined using the convention that $0/0=1$.



\begin{assume}
\label{ass:metric-cont}
Suppose that $\D$ is a metric on $\cP(\cX)$ such that
\begin{enumerate}
    \item $\D$ is continuous with respect to the weak convergence topology on $\cP(\cX)$.
    \item $Q \mapsto \D(P_0, Q)$ is a convex function  on $\cP(\cX)$
    \item For each $Q \in \cP(\cX)$, $h \mapsto \D(P_0, (1-h) Q + hP_0)$ is a strictly decreasing function of $h \in [0,1]$.
\end{enumerate}
\end{assume}


\jk{It's a little weird to include a reference to a definition in the appendix in this lemma. Do we need this result in the main paper? If so, it should be comprehensible without flipping to the appendix I think.}
\begin{lemma} Under assumption \ref{ass:compactness}, Wasserstein distances and $\tvk{\K}$ (Definition \ref{def:smoothedtvd}) with a bounded and continuous kernel $\K$ satisfies Assumption \ref{ass:metric-cont}.
\end{lemma}

\begin{assume}
\label{ass:compactness}
 Suppose that $\cX$ is a compact.
\end{assume}

\begin{lemma} Suppose assumptions \ref{ass:metric-cont}, \ref{ass:compactness} hold and $\KL(P_0|P_\theta) < \infty$.
Then the function $r \mapsto I_r(\theta)$ is continuous on the interval $[0,\infty)$.
\end{lemma}

\begin{theorem}
\label{thm:casymptotics}

\end{theorem}

\subsubsection{Consistency of rate function estimator}