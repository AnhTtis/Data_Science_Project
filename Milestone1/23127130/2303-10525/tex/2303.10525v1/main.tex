\pdfoutput=1
\documentclass[12pt]{article}
\usepackage{amsmath, amssymb, amsthm}
\usepackage{xcolor}
\usepackage{mathtools}
\usepackage{hyperref}
\usepackage{graphicx}
\usepackage{enumerate}
\usepackage{caption}


%%%%%%%% Clever ref
\usepackage{cleveref}

\captionsetup[figure]{font={stretch=1.2}}    %% change 1.2 as you like


%\usepackage{url} % not crucial - just used below for the URL


\usepackage{macros}
%\usepackage{breqn}

\usepackage{natbib}


%\pdfminorversion=4
% NOTE: To produce blinded version, replace "0" with "1" below.
\newcommand{\blind}{1}

% DON'T change margins - should be 1 inch all around.
\addtolength{\oddsidemargin}{-.5in}%
\addtolength{\evensidemargin}{-1in}%
\addtolength{\textwidth}{1in}%
\addtolength{\textheight}{1.7in}%
\addtolength{\topmargin}{-1in}%

\usepackage{IEEEtrantools}
\usepackage{algorithm,algcompatible,amsmath}

\usepackage[utf8]{inputenc}
\begin{document}

\def\spacingset#1{\renewcommand{\baselinestretch}%
{#1}\small\normalsize} \spacingset{1}


%%%%%%%%%%%%%%%%%%%%%%%%%%%%%%%%%%%%%%%%%%%%%%%%%%%%%%%%%%%%%%%%%%%%%%%%%%%%%%

\if1\blind
{
  \title{\bf Robustifying likelihoods by optimistically re-weighting data}
%\author{Miheer Dewaskar\thanks{joint first authors} \and Christopher Tosh\footnotemark[1] \and Jeremias Knoblauch \and David Dunson}
   \author{Miheer Dewaskar\thanks{
    M.D. and C.T. contributed equally. The authors acknowledge funding from grants N00014-21-1-2510-P00001 from the Office of Naval Research (ONR) and R01ES027498, U54 CA274492-01 and R37CA271186  from the National Institutes of Health, as well as helpful discussions with Sayan Mukherjee and Amarjit Budhiraja.}\hspace{.2cm}\\
    Department of Statistical Science, Duke University\\
    and \\
    Christopher Tosh\footnotemark[1]  \\
    Department of Epidemiology and Biostatistics, \\
    Memorial Sloan Kettering Cancer Center\\
    and\\
    Jeremias Knoblauch\\
    Department of Statistics, UCL\\
    and\\
    David B. Dunson\\
    Department of Statistical Science, Duke University}
  \maketitle
} \fi

\if0\blind
{
  \bigskip
  \bigskip
  \bigskip
  \begin{center}
    {\LARGE\bf Robustifying likelihoods to slight misspecification}
\end{center}
  \medskip
} \fi

%\bigskip
\begin{abstract}
Likelihood-based inferences have been remarkably successful in wide-spanning application areas. However, even after due diligence in selecting a good model for the data at hand, there is inevitably some amount of model misspecification: outliers, data contamination or inappropriate parametric assumptions such as Gaussianity mean that most models are at best rough approximations of reality.
%
A significant practical concern is that for certain  inferences, even small amounts of model misspecification may have a substantial impact; a problem we refer to as {\em brittleness}.
%
This article attempts to address the brittleness problem in likelihood-based inferences by choosing the most model friendly data generating process in a discrepancy-based neighbourhood of the empirical measure. This leads to a new Optimistically Weighted Likelihood (OWL), which robustifies the original likelihood by formally accounting for a small amount of model misspecification. Focusing on total variation (TV) neighborhoods, we
study theoretical properties, develop inference algorithms and illustrate the methodology in applications to mixture models and regression.
\end{abstract}

\noindent%
{\it Keywords:}
Coarsened Bayes; Data contamination;  Mixture models; Model misspecification; Outliers;  Robust inference; Total variation distance. 
\vfill

%\newpage
%\spacingset{1.9} % DON'T change the spacing!

%\date{June 2022}


%\maketitle


%{\bf (DD - Looking over the below, it is currently a total mess in all the sections. We need to stop fucking around and finalize the storyline and rewrite with this storyline in mind. Ideally we look at the method that is currently being implemented with good results and build up a clear justification for this method and put it clearly in the context of the literature while including competitors in simulated and real data experiments. It seems that the TVD case should be the focus as MMD complicates the story and doesn't have as good results in practice, but can we have a very clear motivation for the approach being applied starting from the coarsest likelihood philosophy?)}

\section{Introduction}

\section{Introduction}

The increasing complexity of source code poses a key challenge to the reliability of large-scale software systems. Software bugs in these systems can lead to safety issues~\cite{bug_safety} for users around the world as well as cause non-negligible financial losses~\cite{bug_loss}. As such, developers have to spend a large amount of time and effort on bug fixing. Consequently, \aprfull (\apr), designed to automatically generate patches to fix software bugs, has attracted wide attention from both academia and industry~\cite{long2016prophet, legoues2012genprog, long2015spr, lou2020can, tufano2018empstudy}. 


To achieve \apr, one popular approach is known as Generate-and-Validate (G\&V)~\cite{qi2015gv, ghanbari2019prapr, lou2020can, le2016hdrepair, legoues2012genprog, wen2018capgen, hua2018sketchfix, martinez2016astor, koyuncu2020fixminder, liu2019tbar, liu2019avatar}, which is typically based on the following pipeline: First, fault localization techniques~\cite{wong2016fl, abreu2007ochiai, zhang2013injecting, papadakis2015metallaxis, li2019deepfl, li2017transforming} are applied to determine the suspicious locations in programs where bugs are likely to exist. Then, the buggy locations are used by the \apr tools to generate a list of patches that replace buggy lines with correct lines. Afterward, each patch is validated against the original test suite to identify any \emph{plausible patches} (i.e., passing all tests in the test suite). Finally, to determine the \emph{correct patches}, developers examine the list of plausible patches to see if any of them can correctly fix the bug. 

Traditional \apr tools can mainly be categorized into heuristic-based~\cite{legoues2012genprog, le2016hdrepair, wen2018capgen}, constraint-based~\cite{mechtaev2016angelix, le2017s3, demacro2014nopol, long2015spr} and \template~\cite{ghanbari2019prapr, hua2018sketchfix, martinez2016astor, liu2019tbar, liu2019avatar}. Among these traditional tools, \template \apr tools~\cite{ghanbari2019prapr, liu2019tbar, benton2020effectiveness} have been able to achieve state-of-the-art results. \Template \apr tools typically leverage pre-defined templates (e.g., adding a nullness check) for bug fixing. However, since these fix templates are typically handcrafted, the number and types of bugs they are able to fix can be limited. 



To address the limitations of traditional \apr, researchers have proposed various \learning \apr tools~\cite{li2020dlfix, chen2018sequencer, jiang2021cure, lutellier2020coconut, zhu2021recoder, ye2022rewardrepair} based on the \nmtfull (\nmt) architecture~\cite{sutskever2014mt} where the input is the buggy code snippets and the goal is to translate the buggy code snippets into a fixed version. To accomplish this, \learning \apr tools require supervised training datasets with pairs of both buggy and fixed code snippets in order to learn how to perform this translation step. These training data are usually obtained by mining historical bug fixes using heuristics/keywords~\cite{dallmeier2007benchmark}, which can be imprecise for identifying bug-fixing commits; even the actual bug-fixing commits can include irrelevant code changes, leading to further pollution in the dataset~\cite{xia2022alpharepair}.
% 
Moreover, it can be hard for such \apr tools to generalize and fix bug types unseen during training. 



To better leverage recent advances in \plmfull{s} (\plm{s}), researchers~\cite{xia2022alpharepair, xia2023repairstudy, kolak2022patch, prenner2021codexws} have directly applied \plm{s} to generate patches without bug-fixing datasets. These \llm-based \apr tools work by either directly generating a complete code function~\cite{prenner2021codexws, xia2023repairstudy} or predict/infill the correct code snippet given its surrounding context~\cite{xia2022alpharepair, xia2023repairstudy}. By directly using \llm{s} that are pre-trained on billions of open-source code snippets, \llm-based \apr tools can achieve state-of-the-art performance on many repair datasets~\cite{xia2022alpharepair}. 


% 
%
%

Traditional \apr tools have long used the insight of the \emph{plastic surgery hypothesis}~\cite{barr2014plastic} where it states that the code ingredients to fix a bug already exist within the same project. Traditional \apr tools have manually designed pattern-~\cite{ghanbari2019prapr, saha2017elixir} or heuristic-based~\cite{jiang2018simfix, legoues2012genprog} approaches to finding and using such relevant code ingredients to generate fixes for bugs. However, the plastic surgery hypothesis has been largely ignored in \llm-based \apr. In fact, \llm provides a unique opportunity to fully automate the plastic surgery hypothesis idea via fine-tuning (learning project-specific information via model updates from the buggy project) and prompting (directly providing relevant code ingredients to the model), and make it directly applicable to different languages (since the \llm{s} are typically multi-lingual).%
Moreover, despite the intensive manual efforts involved, traditional \apr tools still cannot fully leverage project-specific information due to large search space for leveraging/composing existing code ingredients. In contrast, the project-specific information can effectively leveraged by \llm{s} due to their power in code understanding/vectorization, e.g., even partial/imprecise information may still guide \llm{s} in correct patch generation!
 To this end, we ask the question: \emph{How useful is the plastic surgery hypothesis in the era of \plm{s}}?








\mypara{Our Work.} To answer the question, we present \ourtech{\xspace} -- a \llm-based approach that automatically utilizes the plastic surgery hypothesis by systematically combining multiple fine-tuning and prompting strategies for \apr. \ourtech fine-tunes \plm{s} using two novel domain-specific training strategies: \textbf{\epfinetune} -- we fine-tune using the original buggy project by aggressively masking out a high percentage of tokens, which allows \plm to learn project-specific code tokens and programming styles; and \textbf{\rofinetune} -- which only masks out a single continuous code sequence per training sample, allowing the model to get used to the final \csapr task of predicting a single continuous code sequence. Furthermore, we directly leverage the ability for \plm{s} to understand natural language instructions and introduce a novel prompting strategy, \textbf{\idprompting}, which uses information retrieval and static analysis to obtain a list of relevant identifiers for the buggy lines. While such relevant identifiers are critical for fixing some difficult bugs, they may not be seen by the \llm during inference due to limited context window size. Through the use of prompting, we directly tell the model to use these extracted identifiers (relevant code ingredients) to generate the correct code. Finally, to perform repair, we combine all four model variants (including the base model, both fine-tuned models and the base model with prompting) for the final repair.





While our insight of leveraging the plastic surgery hypothesis for \llm-based \apr is generalizable across different types of \plm{s}, to implement \ourtech, we choose a recent \plm{\xspace}, \ctfive~\cite{wang2021codet5}, which is pre-trained on millions of open-source code snippets. \ctfive is an encoder-decoder model trained using \mspfull (\msp) objective where a percentage of tokens are masked out and each continuous masked token sequence is referred to as a masked span. Also, although we only extract relevant identifiers from the current buggy project (since this paper focuses on the plastic surgery hypothesis), our work can be easily extended to obtain other code information (such as relevant statements or functions) from other sources, such as  the massive pre-training corpora~\cite{husain2020codesearchnet} or historical bug-fixing datasets~\cite{jiang2019infer}, which can provide more coding knowledge for \llm{s}. Besides, although we mainly focus on using traditional string comparison algorithms for information retrieval in this paper, these techniques can be easily replaced by other frequency-based retrieval~\cite{robertson2009probabilistic} and neural search (or embedding-based search)~\cite{reimers2019sentence}.
  In summary, this paper makes the following contributions:


%


\begin{itemize}[noitemsep, leftmargin=*, topsep=0pt]
    \item \textbf{Dimension.} This paper is the first to revisit the important plastic surgery hypothesis in the era of \llm{s}. It opens up a new dimension for \llm-based \apr to incorporate previously neglected information from the buggy project itself to boost \apr performance. Furthermore, it demonstrates the promising future of retrieval-based prompting for modern \llm-based \apr.
    \item \textbf{Implementation.} We implement \ourtech based on the recent \ctfive model. We augment the model using two novel fine-tuning strategies: \epfinetune and \rofinetune, along with a novel prompting strategy based on information retrieval and static analysis: \idprompting. We combine the patches generated by all four models together and perform patch ranking to speed up \apr.% 
    \item \textbf{Evaluation Study.} We conduct an extensive evaluation against state-of-the-art \apr tools. On the widely studied \dfj 1.2 and 2.0 datasets~\cite{just2014dfj}, \ourtech is able to achieve the new state-of-the-art results of 89 and 44 correct bug fixes (15 and 8 more than best baseline) respectively.  Furthermore, we perform a broad ablation study to justify our design. \ourtech demonstrates for the first time that the plastic surgery hypothesis can substantially boost \llm-based \apr and advance state-of-the-art \apr, while being fully automated and general. Moreover, even partial/imprecise code ingredients may still effectively guide \llm{s} for \apr!
\end{itemize}



\section{Optimistically Weighted Likelihoods}
\label{sec:methodology}
%auto-ignore

As \Cref{fig:simple-failure} shows, maximum likelihood estimation can be brittle when the data generating distribution $P_0$ is allowed to have a small degree of misspecification with respect to the model family $\{P_\theta\}_{\theta \in \Theta}$.  
%
To assuage the problem of brittleness under misspecification, we propose an Optimistically Weighted Likelihood (OWL) approach that iterates between (1) optimistically re-weighting the observed data points %to match the current model estimate 
and (2) updating the parameter estimate by maximizing a weighted likelihood based on the current data weights. 

%First, in \Cref{sec:okl}, we study this parameter inference methodology at the population level using the optimistic Kullback Leibler (OKL), where we formally allow a small misspecification in $P_0$. Motivated by the population analysis, in \Cref{sec:owl}, we derive the OWL based parameter estimation methodology when only samples $x_1, \ldots, x_n$ from $P_0$ are available.

In \Cref{sec:okl}, we study this parameter inference methodology at the population level where we formally allow $P_0$ to be misspecified. Here we introduce the population level optimistic Kullback Leibler (OKL) function with parameter $\epsilon \in [0,1]$ (\Cref{def:okl}), and show that its minimizer will be a parameter $\theta$ for which $P_\theta$ is $\epsilon$-close to $P_0$ in TV distance. Under suitable conditions, the minimization of the OKL can be performed by iterating the two steps of a) projection of the current model estimate onto the $\epsilon$ TV neighborhood around $P_0$ in a Kullback Leibler sense (information projection), and b) finding the parameter that maximizes a suitably weighted integral of the log likelihood.

Motivated by this population analysis, in \Cref{sec:owl}, we derive the OWL-based parameter estimation methodology when only samples $x_1, \ldots, x_n$ from $P_0$ are available.
%In \Cref{sec:owl}, we derive a corresponding finite sample estimation method called \emph{Optimistically Weighted Likelihood} (OWL) that alternates between the following two steps of (1) finding a parameter that maximizes a weighted likelihood given a weighting for the data points, and (2) updating the weighting based on the current parameter estimate.

\subsection{Population level Optimistic Kullback Leibler minimization}
\label{sec:okl}


\begin{figure}
	\includegraphics{figures/owl.pdf}
	\caption{Population level description of OWL (left) and the OKL function (right). The point $P_\theta$ is labeled using $\theta \in \Theta$. The inference  problem is to find a point in $\Theta_I \subseteq \Theta$ where the model family intersects the  $\epsilon$-neighborhood $\ball_\epsilon$ of the data distribution $P_0$. The set $\Theta_I$ is the set of minimizers of the OKL function $\theta \mapsto I_\epsilon(\theta)$ (right). Starting from initial point $\theta_1 \in \Theta$, the OWL procedure finds a saddle point of the OKL function by iterating $\theta_{t+1} = \argmin_{\theta \in \Theta} \KL(Q^{\theta_t}|P_\theta)$ for $t=1,2, \ldots,$ until convergence, where $Q^\theta = \argmin_{Q \in \ball_\epsilon}  \KL(Q|P_\theta)$ denotes the information projection (\cite{amari2016information}) of the model $P_\theta$ on the total variation (TV) neighborhood $\ball_\epsilon$ of $P_0$. Iterations alternate between I-projection and weighted likelihood estimation steps, illustrated via solid and dashed lines.} 
    %
    %\jk{minor gripe: using red and green in the same plot isn't ideal for color blind readers; you could change red to dark orange.}
 %Note that the  inference target $\theta^*$ is contained in the region $\Theta_I$ where the model intersects the neighborhood $\ball_\epsilon$. The set $\Theta_I \subseteq \Theta$ can alternatively be described as the set of minimizers of the OKL function $\theta \mapsto I_\epsilon(\theta) = \min_{Q \in \ball_\epsilon}\KL(Q|P_\theta)$. We can find  a  saddle point of OKL by repeating the iterations $\theta_{t+1} = \argmin_{\theta \in \Theta} \KL(Q^{\theta_t}|P_\theta)$ for $t=1,2, \ldots,$ until convergence.}
	%
	%\jk{really liking this plot; points for improvement on the Left panel: $P_0$ should have a black x next to it to clarify that it is a point in the ellipsoid. Similarly for $Q^{\theta_1}, Q^{\theta_2}$. I would also leave out $\theta_3$; and I don't understand what the dashed line represents as it's not clear from either plot or description. Do we need it? Also unclear: Why are we stating that $P_0 = (1-\varepsilon)P_{\theta^*} + \varepsilon C$? This made sense in the original plot that I proposed, but this one (which I prefer) doesn't actually need to state this because $C$ isn't assigned any geometric interpretation. Also confusing: we say that the inference target $\theta^*$ 'is contained in $\Theta_I$'. Is this right? I'd argue that we set things up so that any element  $\theta^* \in \Theta_I$ is an (equally good) inference target.}
	\label{fig:okl}
\end{figure}

%We develop methodology for likelihood-based parameter estimation while formally allowing for a small amount of model misspecification, quantified in terms of a suitable metric on the space of
 Let $\cP(\cX)$ denote the space of probability distributions on the data space $\cX$. Our methodology can accommodate misspecification in terms of a variety of probability metrics (e.g.~MMD or Wasserstein) on $\cP(\cX)$, but here we mainly focus on the total variation (TV) distance for concreteness and interpretability. Let $\tv(P,Q) = \sup_{A \subseteq \cX} |P(A)-Q(A)|$ denote the TV metric between two probability distributions $P, Q \in \cP(\cX)$. Given a model family $\{P_\theta\}_{\theta \in \Theta} \subseteq \cP(\cX)$, we assume that the data generating distribution $P_0 \in \cP(\cX)$ for the data population in question satisfies $\tv(P_0, P_{\theta^*}) \leq \epsilon$ for a known value $\epsilon \geq 0$ and some unknown $\theta^* \in \Theta$. In other words, we make the following assumption:

\begin{assume}
	\label{ass:misspecification}
Given $\epsilon \geq 0$ and the true data distribution $P_0$, the set of parameters
$\Theta_I = \{ \theta \in \Theta | \tv(P_0, P_\theta) \leq \epsilon\}$ is non-empty.
\end{assume}

%When $\epsilon = 0$, this simply states that $P_0 = P_{\theta^*}$ (almost surely) for some $\theta^* \in \Theta$.
 \Cref{ass:misspecification} encompasses
Huber's $\epsilon$-contamination model~\citep{huber1964robust} since the condition $\tv(P_0,P_{\theta^*}) \leq \epsilon$ follows whenever $P_0 = (1-\epsilon) P_{\theta^*} + \epsilon C$, for an arbitrary contaminating distribution $C \in \cP(\cX)$. However, \Cref{ass:misspecification} is strictly more general, since $\tv(P_0, P_{\theta^*}) \leq \epsilon$ does not imply that $P_0$ is an $\epsilon$-contamination of $P_{\theta^*}$. 
%
Indeed, \Cref{lem:contamination-condition} in the appendix shows that for fixed $P_0$, $P_{\theta^*}$, and $\epsilon$, $P_0 = (1-\epsilon) P_{\theta^*} + \epsilon C$  for  some contaminating distribution $C$ if and only if the Radon Nikodym derivative $\frac{dP_{\theta^*}}{dP_0}$ exists and is bounded from above by $\frac{1}{1-\epsilon}$, $P_0$ almost surely.
%
%\jk{I'm wondering if I'm missing something here---does this work for mixed distributions? Say I'm given a part-continuous, part-discrete measure. I.e., $P_0$ is so that $P_0 = \varepsilon D + (1-\varepsilon)G$ for $D$ a discrete measure (say on $\mathbb{N}$) and $G$ a measure (say on $\mathbb{R}$) with a density with respect to the Lebesgue measure. Then I can pick $P_{\theta^*} = G$, but---if I remember correctly---the Radon-Nikodym derivative between $P_0$ and $G$ then doesn't exist. (After a quick search, a generalised version exists: \url{https://math.stackexchange.com/questions/3694054/radon-nikodym-derivative-of-a-mixed-distribution}; but I don't think we use this one.) I don't think this is a problem for the result in spirit; but we may have to add in a condition that prevents $C$ from being discrete? The current statement of the Lemma in the appendix doesn't do this I think.}

In general, under \Cref{ass:misspecification}, it may only be possible to identify the set $\Theta_I$, rather than any particular $\theta^*$. 
%
Although such indeterminacy may be inherent, it is practically insignificant whenever $\epsilon$ is sufficiently small so that the distinction between two elements from $\Theta_I$ is practically irrelevant~\citep{huber1964robust}. In line with this insight, the goal throughout the rest of the paper will be to identify \emph{some} parameter in $\Theta_I$.

At the population level, usual maximum likelihood parameter estimation amounts to minimizing the Kullback Leibler (KL) function $\theta \mapsto \KL(P_0|P_{\theta})$ on the parameter space $\Theta$. Even under small amounts of misspecification, KL minimizers are very brittle.
%
The origin of this phenomenon is that any minimizers of the KL function must place sufficient probability mass wherever $P_0$ does, including on outliers. 
%
In contrast, TV distance is far less sensitive to the geometry of misspecifaction. 
%
Hence one may minimize $\theta \mapsto \tv(P_0, P_\theta)$ as a robust alternative, particularly under  \Cref{ass:misspecification}. However, direct minimization of TV distance over the parameter space $\Theta$ is difficult to implement in practice due to the lack of suitable optimization primitives (e.g. maximum likelihood estimators) and the  non-convex and non-smooth nature of the optimization problem \citep[see e.g.][]{yatracos1985rates}. 


%Instead, we use TV to robustify the KL divergence. 
An approach minimizing the  KL divergence with its second argument constrained within an $\epsilon$-neighborhood under the L\'evy-Prokhorov  metric---as opposed to the TV distance---was proposed in \cite{yang2018robust} to provide Neyman-Pearson optimal tests for a robust version of the universal hypothesis testing problem for univariate distributions. 
%
Our motivation is estimation and inference rather than testing.
Indeed, asymptotic analysis of the coarsened likelihood \citep{miller2018robust} using Sanov's theorem (see \Cref{sec:coarsened-inference}) gives rise to a similar KL objective constrained by an $\varepsilon$-neighbourhood. 
%
%optimizing the KL term
%and one of our motivations for optimizing the KL term in its first argument arises from asymptotic analysis of the coarsened likelihood \cite{miller2018robust} using Sanov's theorem (see \Cref{sec:coarsened-inference}). 
The resulting function, which we term the Optimistic Kullback Leibler (OKL), is defined as follows.

\begin{definition}(Optimistic Kullback Leibler) Given $P_0$ and $\epsilon > 0$, the OKL function $I_\epsilon: \Theta \to [0,\infty]$  is defined as:
\begin{equation}
	\label{eq:okl}
    I_\epsilon(\theta) = \inf_{Q \in \ball_\epsilon(P_0)} \KL(Q|P_\theta),
\end{equation}
where $\ball_\epsilon(P_0) = \{Q \in \cP(\cX) | \tv(P_0,Q) \leq \epsilon\}$ is the TV ball of radius $\epsilon$ around $P_0$.
%
If $I_\epsilon(\theta) < \infty$, the underlying optimization over $\ball_\epsilon(P_0)$ has a unique minimizer $Q^\theta$ called the I-projection \citep{csiszar1975divergence}.
%\jk{We seem to be using $Q^{\theta}$ as the minimiser of the above objective consistently  throughout the rest of this section---I suggest we define it formally here together with the OKL?}
\label{def:okl}
\end{definition}


%\new{Next, we discuss the interpretation of the OKL function from the perspective of information geometry \cite{amari2016information}.} Since the $\ball_\epsilon(P_0)$ is a convex set that is closed with respect to the TV distance~\cite{csiszar1975divergence}, whenever $I_\epsilon(\theta) < \infty$, there is a unique $Q^\theta \in \ball_\epsilon(P_0)$, known as the information- or I-projection of $P_\theta$, such that $I_\epsilon(\theta) = \KL(Q^\theta|P_\theta)$. Thus $I_\epsilon(\theta)$ is the KL divergence between $P_\theta$ and its I-projection $Q^\theta$ onto the set $\ball_\epsilon(P_0)$.
%A geometric property for the I-projection similar to the Pythagoras theorem can also established under suitable conditions \cite{csiszar1975divergence}. 

The function $\theta \mapsto I_\epsilon(\theta)$ measures the fit of a model $P_\theta$ to the data $P_0$  allowing for a degree $\varepsilon$ of  data re-interpretation in TV distance before assessing model fit. 
%
Our terminology \emph{Optimistic Kullback Leibler} emphasizes that $I_\epsilon(\theta)$ is the KL divergence between the most optimistic re-interpretation $Q^\theta$ of the data  
within the data neighborhood $\ball_\epsilon(P_0)$ and the model distribution $P_\theta$. Here we use the term optimistic re-interpretation in the sense that, if $\theta$ is our current parameter  estimate, our methodology calculates the KL divergence optimistically, by supposing that the true data are generated from the  model-friendly distribution $Q^\theta$ rather than $P_0$.
%
Here,  $\epsilon \geq 0$ regulates the permitted degree of re-interpreting the data by controlling the neighborhood size. 

The OKL function enables us to perform robust parameter inference by finding a parameter from the set $\Theta_I$: Since $I_\epsilon(\theta) = \KL(Q^\theta|P_\theta)$, under \Cref{ass:misspecification}, the minimum OKL value of zero will be attained exactly on $\Theta_I$ (since $Q^\theta = P_\theta$ if and only if $\theta \in \Theta_I$). 
%
This implies that finding a minimizer of $\theta \mapsto I_\epsilon(\theta)$ amounts to finding a robust parameter estimate. 
%
However, the OKL may be non-convex, %even when the model takes the simple form of an exponential family, 
so that calculating the global minimizer of OKL may not be straightforward. 
%
Fortunately, the OKL lends itself to a feasible alternating optimization scheme that will reach a saddle point under suitable conditions.

Global minimization of the OKL function is equivalent to the joint global minimization 
%
%\jk{It is not obvious to me why that is the case: joint optimisation is generally not the same as sequential optimisation. It would be good to explain why this equivalence holds/point to the appendix if it's formally established.}
%
of the function $F: \Theta \times \ball_\epsilon(P_0) \to [0,\infty]$ given by $F(\theta, Q) = \KL(Q|P_\theta)$, since $I_\epsilon(\theta) = \inf_{Q \in \ball_\epsilon(P_0)} F(\theta,Q)$. Thus we will use alternating minimization to jointly minimize the function $F$, i.e.~for $t \in \nat$ we perform the $Q$-step: $Q^{\theta_{t}} = \argmin_{Q \in \ball_\epsilon(P_0)} \KL(Q|P_{\theta_t})$ and the $\theta$-step: $\theta_{t+1} = \argmin_{\theta \in \Theta} \KL(Q^{\theta_t}|P_\theta)$.  
%
For simplicity, suppose that the model family $\{P_\theta\}_{\theta \in \Theta}$ and $P_0$ have densities $\{p_\theta\}_{\theta \in \Theta}$ and $p_0$ with respect to a common measure $\lambda$, then the I-projection $Q^{\theta_t}$ will also have a density $q_t$ with respect to $\lambda$, and the iterations will take the following form.
Starting from $\theta_1 \in \Theta$ such that $I_\epsilon(\theta_1)  < \infty$, compute the following steps for $t \in \nat$:
\begin{enumerate}
	\item \underline{Q-step}: Compute the I-projection of $P_{\theta_t}$ on the ball $\ball_\epsilon(P_0)$. This corresponds to solving a convex optimization problem over the space of probability densities $\Den = \{q : \cX \to [0,\infty) \mid \int q(x) d\lambda(x) = 1\}$ with respect to $\lambda$.
	$$
	q_t = \argmin_{\substack{q \in \Den \\ \frac{1}{2} \int |q-p_0| d\lambda  \leq \epsilon}} \int q(x) \log \frac{q(x)}{p_{\theta_t}(x)} d\lambda(x).
	$$ 

	\item \underline{$\theta$-step}: Maximize the average log-likelihood $\theta \mapsto \int q_t(x)  \log p_\theta(x)  d\lambda(x)$. Note that $\KL(Q^{\theta_t}|P_\theta) = \int q_t(x) \log q_t(x)  d\lambda(x) - \int q_t(x) \log p_{\theta}(x) d\lambda(x)$ with the convention that $0 \log 0 = 0$. Hence the optimization step $\theta_{t+1} = \argmin_{\theta \in \Theta} \KL(Q^{\theta_t}|P_\theta)$ can be re-written as:
	$$
	\theta_{t+1} = \argmax_{\theta \in \Theta} \int q_t(x) \log p_\theta(x) d\lambda(x).
	$$
\end{enumerate}

The above iterations provide a scheme to minimize the OKL function in which the $Q$-step can  approximately be performed using tools from convex optimization, while the $\theta$-step can be approximated by maximization of a suitably weighted log-likelihood, which can be computed for many standard models. 
%We expand on this further as we explain how to perform this minimization in the presence of finite samples.  
Our resulting population level methodology is illustrated in \Cref{fig:okl}. 

Assuming that there is always a unique minimizer in the $\theta$-step, it is straightforward to show that the objective value $F(\theta_t, Q^{\theta_t})$ is a  strictly decreasing function of $t$ as long as $\theta_t \neq \theta_{t+1}$. Thus if $\{\theta_t\}_{t \in \nat}$ lie in a compact set and suitable continuity assumptions hold, any limit point $\tilde{\theta}$ of the sequence $\{\theta_t\}_{t \in \nat}$ will satisfy the saddle point condition $\tilde{\theta} = \argmin_{\theta \in \Theta} \KL(Q^{\tilde{\theta}}|P_\theta)$. This saddle point condition is satisfied by all the parameters in the identifiable set $\Theta_I$.


We remark that one can use optimization of the OKL function $I_\epsilon(\theta)$ as a subroutine to minimize the function $\theta \mapsto \tv(P_0, P_\theta)$. Namely, we can perform binary search over $\epsilon \in [0,1]$, increasing $\epsilon$ whenever we have $I_\epsilon(\theta) > 0$ and decreasing $\epsilon$ whenever we have $I_\epsilon(\theta) = 0$. Used this way, OKL optimization can be seen as a computationally-palatable approach to minimizing the TV distance over a model class. 
%In general, since \Cref{ass:misspecification} only guarantees us the ability to find some element of $\Theta_I$, it may not be worth the extra computational effort to run binary search or the statistical noise induced by using samples to estimate the KL terms on which we make the binary search decisions. 

%
% additional advantage to related to the c-likelihood..

\subsection{Optimistically Weighted Likelihood (OWL) estimation}
\label{sec:owl}

We now extend the population level methodology from \Cref{sec:okl} to handle the practical case when  samples $x_1, \ldots, x_n \iid P_0$ are available, and provide a computable approximation for the $Q$-step and $\theta$-step from \Cref{sec:okl}. Namely, the $Q$-step (now called $w$-step) will be approximated by a suitable convex optimization problem over weights that lie within the intersection of an  $n$-dimensional probability simplex and the $\ell_1$ ball of radius $2\epsilon$ around the vector with uniform weights; these optimal weights can be interpreted as an optimistic re-weighting of the original data points $x_{1}, \ldots, x_{n}$ to match the current model estimate. Further, the $\theta$-step will then be approximated by maximizing a weighted likelihood with the weights found in the  previous step. Since this methodology involves the repeated steps of parameter estimation using a 
 weighted likelihood (i.e.~the $\theta$-step) and re-estimating the weights on the data points to optimistically match the estimated model (i.e.~the $w$-step), we call this the Optimistically Weighted Likelihood (OWL) method.

%
\subsubsection{Approximating OKL by a finite dimensional optimization problem}

We derive the OWL methodology by approximating the OKL function $I_\epsilon(\theta)$ in terms of a finite dimensional optimization problem defined in terms of observed data $x_1, \ldots, x_n \iid P_0$. Henceforth, let us assume that the model family $\{P_\theta\}_{\theta \in \Theta}$ and measure $P_0$ have densities $\{p_\theta\}_{\theta \in \Theta}$ and $p_0$ with respect to a common measure $\lambda$. We will focus on two cases of interest: when $\cX$ is a finite space and $\lambda$ is the counting measure, and when $\cX=\R^d$ and $\lambda$ is the Lebesgue measure.

When $\cX$ is finite, we look to solve the optimization problem in \cref{eq:okl} over data re-weighting $Q = \sum_{i=1}^n w_i \delta_{x_i}$ as the weight vector $w=(w_1, \ldots, w_n)$ varies over the \ndsimplex\ $\Delta_n$ and satisfies the TV constraint $\frac{1}{2}\|w-o\|_1 \leq \epsilon$ where $o=(1/n, \ldots, 1/n) \in \Delta_n$ is the vector with uniform weights. Formally, our finite space OKL approximation is given by
%

\begin{equation}
	\label{eq:finiteokle}
	\finitehatI(\theta) = \inf_{\substack{w \in {\Delta}_n \\ \frac{1}{2} \|w-o\|_1 \leq \epsilon}} \sum_{i=1}^n w_i \log \frac{nw_i \pfinite(x_i)}{p_\theta(x_i)},
\end{equation}
% For our finite-dimensional approximation, we would like to solve the optimization problem in \cref{eq:okl} over data re-weightings $Q = \sum_{i=1}^n w_i \delta_{x_i}$  as the weight vector $w=(w_1, \ldots, w_n)$ varies over  $\hat{\Delta}_n$, the subset of vectors $w \in \Delta_n$ of the $n$-dimensional probability simplex $\Delta_n$ that satisfies $w_i = w_j$ whenever $x_i=x_j$, and satisfies the total-variation constraint $\frac{1}{2}\|w-e\|_1 \leq \epsilon$ where $e=(1/n, \ldots, 1/n) \in \Delta_n$ is the vector with uniform weights. Such an approximation is effective when $\cX$ is a finite space provides the OKL approximation: 
where $\pfinite(y) = \frac{|\{i \in [n] | x_i = y\}|}{n}$ is the histogram estimator for the data generating distribution $p_0$ when $\cX$ is a finite space. An application of the log sum inequality~\cite[Theorem~2.7.1]{cover2006elements} shows that the weights that solve \cref{eq:finiteokle} have the appealing and natural property that $w_i = w_j$ whenever $x_i=x_j$.
Moreover, when the support of $p_0$ contains the support of $p_\theta$, $\finitehatI(\theta)$ converges to $I_\epsilon(\theta)$ at rate $n^{-1/2}$, as demonstrated by the following result.
\begin{theorem}
\label{thm:finite-okl-convergence-tv}
Suppose that $I_{\epsilon_0}(\theta) < \infty$ for some $\epsilon_0 > 0$ and pick $\delta > 0$ and $\epsilon > \epsilon_0$. If $\supp(p_\theta) \subseteq \supp(p_0)$ and $x_1,\ldots,x_n \iid p_0$, then with probability at least $1-\delta$, 
\[ |I_\epsilon(\theta) - \finitehatI(\theta)| \leq  O\left(\frac{ |\supp(p_0)|}{\epsilon - \epsilon_0} \sqrt{\frac{1}{n} \log \frac{|\supp(p_0)|}{\delta}} \right), \]
where $\supp(p) = \{ x \in \Xcal : p(x) > 0 \}$.
\end{theorem}
See \Cref{sec:okl-estimation-finite} for the proof and a more general theorem statement with explicit constants.

When $\cX=\R^d$, the above approximation strategy needs to be modified, since $\KL(Q|P_\theta)$ is unbounded whenever $P_\theta$ is supported on all of $\cX$ and $Q = \sum_{i=1}^n w_i \delta_{x_i}$ is a discrete distribution. 
%
In this case, \cref{eq:okl} should be formulated in terms of measures $Q \in \cP(\cX)$ that have density $q$ with respect to $\lambda$. Namely, we start with the formulation: 
\begin{equation}
		\label{eq:okl-density}
	I_\epsilon(\theta) = \inf_{\substack{q \in \Den\\ \frac{1}{2}\int |q-p_0| d\lambda \leq \epsilon}} \int q(x) \log \frac{q(x)}{p_\theta(x)} d\lambda(x)
\end{equation}
where $\Den = \{q : \cX \to [0,\infty) \mid \int q(x) d\lambda(x) = 1\}$ is the space of probability 
densities with respect to $\lambda$. Next, using a suitable probability kernel $\K: \cX \times \cX \to [0,\infty)$, we restrict the domain of the optimization problem in \cref{eq:okl-density} to the finite dimensional subspace of densities $\{q_v | v \in \Delta_n\}$, where $q_{v}(\cdot) \doteq \sum_{i=1}^n v_i \K(x_i, \cdot)$
denotes the probability density indexed by weight vector $v \in \Delta_n$. As an example, we may use the Gaussian kernel $\K(x,y) =\frac{1}{(2\pi \sigma^2)^{d/2}} \exp(-\frac{\|x-y\|^2}{2h^2})$ for a bandwidth parameter $h > 0$.
%\jk{Are we addressing how this is tuned/chosen in our experiments somewhere? If so, we should point to the relevant section in the main paper/appendix here.}

For a probability kernel $\K$, using the finite dimensional approximation $\{ q_v : v \in  \Delta_n\}$ for the space of densities, and a suitable Monte Carlo approximation to the integral objective in \cref{eq:okl-density}, we obtain the approximation:
\begin{equation}
	\label{eq:okle}
	\hatI(\theta) = \inf_{\substack{w \in \hat{\Delta}_n \\ \frac{1}{2}\|w-o\|_1 \leq \epsilon}} \sum_{i=1}^n w_i \log \frac{n w_i \hat{p}(x_i)}{p_\theta(x_i)}
\end{equation}
where $\hat{p}$ is a suitable density estimator for $p_0$ based on $x_1,\ldots, x_n$, $A$ is an $n \times n$ matrix with entries $A_{ij} = \frac{\K(x_i,x_j)}{n \hat{p}(x_i)}$, and $\hat{\Delta}_n = A \Delta_n$ is the image of the \ndsimplex\ under linear operator $A$. We will typically take $\hat{p}(\cdot) = \frac{1}{n} \sum_{j=1}^n \K(\cdot, x_j)$ to be the kernel-density estimate based on the same kernel $\K$, in which case $A$ is the stochastic matrix obtained by normalizing the rows of the kernel matrix $K = (\K(x_i,x_j))_{i,j \in [n]}$ to sum to one. 

The continuous space approximation in \cref{eq:okle} yields the finite space approximation in  \cref{eq:finiteokle} as a special case when $\K(x, y) = \I{x = y}$ is taken to be the indicator kernel. The weights vectors in $\hat{\Delta}_n$ are always non-negative and approximately sum to one for large values of $n$, since $\sum_{i=1}^n A_{ij} = \frac{1}{n}\sum_{i=1}^n \frac{\kappa(x_i, x_j)}{\hat{p}(x_i)} \approx \int \frac{\kappa(x, x_j)}{\hat{p}(x)} p_0(x) dx \approx 1$.
	The derivation of \cref{eq:okle} along with  large sample consistency can be found in \Cref{sec:okl-estimation-cont}, but we briefly describe the main result here.
	
	Proving the convergence of $\hatI(\theta)$ is technically challenging, as it requires proving uniform-convergence of the objective in \cref{eq:okle} when the weights are allowed to vary over the entire range $\hat{\Delta}_n$. % and the estimator $\hat{p}$ may take values that are unbounded or arbitrarily close to zero.% 
	To get around this, we restrict the optimization domain to $\hat{\Delta}_n^\beta = A \Delta_n^\beta$ where $\Delta_n^\beta = \{ (v_1, \ldots, v_n) \in \Delta_n \, : \,  v_i \in [\frac{\beta}{n}, \frac{1}{n\beta}] \}$ % and (ii) we replace $\hat{p}$ with the truncated estimator $\hat{p}_\rho(x) = \min \{ \max\{ \hat{p}(x), \rho \}, 1/\rho \}$ 
    %
    for a suitably small constant $\beta$. Thus, the estimator that we theoretically study is given by
\begin{equation}
	\label{eq:okle-theoretical}
	\hat{I}_{\epsilon,\beta}(\theta) = \inf_{\substack{w \in \hat{\Delta}_n^\beta \\ \frac{1}{2}\|w-o\|_1 \leq \epsilon}} \sum_{i=1}^n w_i \log \frac{n w_i \hat{p}(x_i)}{p_\theta(x_i)}.
\end{equation}
Given this change, we can show the following result.
\begin{theorem}
\label{thm:continuous-okle-convergence}
Suppose $\supp(p_0) = \supp(p_\theta) = \Xcal$ is a compact subset of $\R^d$ and there exists a constant $\gamma > 0$ such that $p_0(x), p_\theta(x) \in [\gamma, 1/\gamma]$ for all $x \in \Xcal$. Suppose that we use the probability kernel $\kappa(x,y) = \frac{1}{h^d} \phi(\|x - y \|/h)$ having bandwidth $h > 0$, with $\kappa$ positive semi-definite and $\phi$ bounded above by a constant and having exponentially-decaying tails. Assume that $p_0$ and $\log p_\theta$ are $\alpha$-H\"{o}lder smooth over $\Xcal$, and suppose that we use the clipped density estimator $\hat{p}(x) = \min(\max(\frac{1}{n}\sum_{i=1}^n \kappa(x_i, x), \gamma), 1/\gamma)$. Then for any constant $0 < \beta \leq \gamma^2/4$
\[ 
\left|\hat{I}_{\epsilon, \beta}(\theta) - I_\epsilon(\theta) \right| \leq \tilde{O}\left(n^{-1/2} h^{-d} + h^{\alpha/2} + \psi(\sqrt{h}) \right) ,  \]
with probability at least $1 - 1/n$, where $\psi(r) = \frac{\lambda (\Xcal \setminus \Xcal_{-r})}{\lambda(\Xcal)}$, $\lambda(\cdot)$ is the $d$-dimensional Lebesgue measure, $\Xcal_{-r} = \{ x \in \Xcal : B(x, r) \subseteq \cX \}$, and $\tilde{O}(\cdot)$ hides constants and logarithmic factors.
\end{theorem}
Observe that $\psi(r)$ measures the fraction of the volume of $\Xcal$ that is contained in the envelope of width $r$ closest to the boundary. For well-behaved sets, we expect $\psi(r)$ to decrease to 0 as $r \rightarrow 0$. For example, if $\Xcal$ is a $d$-dimensional ball of radius $r_0$, then $\psi(r) = 1 - (1 - \frac{r}{r_0})^d$.

We prove our theory with the truncated estimator $\hat{I}_{\epsilon, \beta}(\theta)$ instead of $\hat{I}_{\epsilon}(\theta)$ for techincal reasons. By a suitable version of the sandwiching lemma (\Cref{sec:sandwiching}), under the assumptions of \Cref{thm:continuous-okle-convergence}, we conjecture that the optimal weights in   \eqref{eq:okle} lie  in the set $\hat{\Delta}^\beta_n$ for a small enough constant $\beta > 0$, in which case we will have $\hat{I}_{\epsilon, \beta}(\theta) = \hat{I}_{\epsilon}(\theta)$.
%\jk{If I understand correctly, we introduce this truncated estimator only because it allows us to prove this convergence result: In the actual methodology, we use the approximation $\hat{I}_{\varepsilon}(\theta)$ which relies on the original estimator (rather than $\hat{I}_{\varepsilon, \beta, \rho}(\theta)$, which relies on the truncated version)---at least our pseudo code in 'Algorithm 1' says so. Currently, it's not clear from the paper why we choose the non-truncated version in practice. If there is a good reason, we should emphasise this, given that me made an effort to arrive at a theoretical guarantee through the truncated version. If we do it because the non-truncated performs better in practice and doesn't need as many hyperparameters, we should say so. Basically, at the moment this theoretical result feels a little disconnected from the actual methodology. (This is compounded by the fact that we don't even really use the kernelised density estimator [which we explain in the next section], so that there are currently two degrees of distance between the theory and the methodology. An easy solution: if we are going to approximate the kernelised version anyways, we can just change Algorithm 1/the presentation of our methodology to say 'our methodology relies on the truncated estimator, but in practice we often use a naive non-kernelised versioon because it works better'. This cuts out one layer of disconnect.)}
 
\subsubsection{OWL Methodology}

With a computable approximation $\hatI$ (or $\finitehatI$) to the OKL function in hand, we follow the alternating minimization strategy described in \Cref{sec:okl} to minimize the function $\theta \mapsto \hatI(\theta)$. In more detail, we replace the  density $q_t$ (or more precisely the relative density $w = q_t/p_0$) in the $Q$-step with the weight vector $w_t=(w_{t,1}, \ldots, w_{t,n}) \in \Rnn^n$ that minimizes \cref{eq:okle} for $\hatI(\theta_t)$. We rename this the $w$-step to emphasize the new setup. Next, the $\theta$-step corresponds to minimizing the function $\theta \mapsto \sum_{i=1}^n w_{t,i} \log \frac{n w_{t,i} \hat{p}(x_i)}{p_\theta(x_i)}$, which is equivalent to maximizing the weighted likelihood $\theta \mapsto \sum_{i=1}^n w_{t,i} \log p_\theta(x_i)$. 

The resulting procedure is summarized in \Cref{alg:owl}.  In \Cref{sec:comp}, we expand on the computational details for the $\theta$-step and $w$-steps, but note for now that the $w$-step involves solving a convex optimization problem for which standard tools are available, while the $\theta$-step corresponds to maximizing a weighted likelihood, which can be performed for many models through simple modifications of procedures for the corresponding maximum likelihood estimation.

Finally, it is straightforward to see that the iterates $\theta_t$ of \Cref{alg:owl} must decrease the objective function $\theta \mapsto \hatI(\theta)$, as we have
\begin{align*}
    \hatI(\theta_{t+1}) 
    &= \sum_{i=1}^n w_{t+1,i} \log \frac{n w_{t+1,i} \hat{p}(x_i)}{ p_{\theta_{t+1}}({x}_i)} 
	\leq \sum_{i=1}^n w_{t,i} \log \frac{n w_{t,i} \hat{p}(x_i)}{ p_{\theta_{t+1}}({x}_i)} \\
	&
	\leq \sum_{i=1}^n w_{t,i} \log \frac{n w_{t,i} \hat{p}(x_i)}{ p_{\theta_{t}}({x}_i)} = \hatI(\theta_t).
\end{align*}

Since \cref{eq:okle} is a convex optimization problem with a strictly convex objective, the first inequality is strict unless $w_{t+1} = w_{t}$. Hence the objective $\hat{I}_\epsilon$  must decrease strictly at each step.% until convergence $(\theta_t, w_t) \to (\theta_*, w_*)$ to a limit point is achieved. %Any such limit point will satisfy the fixed point condition $\hatI(\theta_*) = \sum_{i=1}^n w_{*,i} \log \frac{n w_{*,i} \hat{p}(x_i)}{p_{\theta_*}(x_i)}$.

\begin{algorithm}
	\caption{OWL Methodology}
	\label{alg:owl}
	\begin{algorithmic}
		\STATE \textbf{Input:} Model $\{p_\theta\}_{\theta \in \Theta}$, coarsening parameter $\epsilon \geq 0$, probability kernel $\K$, initial point $\theta_1 \in \Theta$, and iteration limit $T$.
		\FOR{$t=1,\ldots,T$}
		\STATE \underline{$w$-step:} Find $w_t=(w_{t,1}, \ldots, w_{t,n}) \in \Rnn^n$ that minimizes \cref{eq:okle} for  $\theta=\theta_t$. 
		\STATE \underline{$\theta$-step:} Find $\theta_{t+1}$ that maximizes the weighted likelihood $\theta \mapsto \sum_{i=1}^n w_{t,i} \log p_\theta(x_i)$.
		\ENDFOR
		\STATE \textbf{Output:} The robust parameter estimate $\theta_T$ and the data weights $w_T$.
	\end{algorithmic}
\end{algorithm}
%

In practice, when the data lie in a  continuous space, we often avoid using the kernel-based estimator \cref{eq:okle} to determine the weights in the $w$-step of \Cref{alg:owl} because it greatly slows down the computation (see \Cref{sec:evaluating}), and the resulting weights are sensitive to the choice of kernel $\K$. Instead, setting $\kappa(x, y)=\I{x=y}$, we perform the $w$-step by  solving the \emph{unkernelized optimization} problem:
$$
\min_{\substack{w \in \Delta_n \\ \frac{1}{2}\|w-o\|_1 \leq \epsilon }} \Big\{- \sum_{i=1}^n w_i \log p_\theta(x_i)  + \sum_{i=1}^n w_i \log w_i\Big\},
$$
obtained from \cref{eq:okle} when all the data points $x_1, \ldots, x_n$ are distinct. 
We demonstrate in \Cref{sec:kern-vs-unkern-simulations} that the unkernelized version of the OWL procedure has equally good performance compared to the kernelized version with a suitably tuned bandwidth. A potential explanation for this is that the primary role of the $w$-step is to down-weight outliers under the model density $p_\theta$, which is controlled by the first term in the optimization objective above; in contrast, the second term in the optimization objective controls the regularity of the non-outlying weights, and plays a secondary role in the $w$-step. 

\subsubsection{Setting the corruption fraction $\epsilon$}
\label{sec:tune-epsilon}

So far we have assumed that the parameter $\epsilon \in (0,1)$, which can be interpreted as the fraction of corrupted samples in the population distribution, is fixed at a known value that satisfies \Cref{ass:misspecification}. Now let us see how the population level analysis (Section \ref{sec:okl}) can inform our choice of $\epsilon$.  \Cref{ass:misspecification} is satisfied as long as $\epsilon \geq \varepsilon_0$, where
$$
\varepsilon_0 = \min_{\theta \in \Theta} \tv(P_0, P_\theta) = \left\{\epsilon \in [0,1] : \min_{\theta \in \Theta} I_\epsilon(\theta) = 0 \right\}.
$$

Hence, in principle, we could set $\epsilon = \varepsilon_0$ to use OWL to perform minimum-TV estimation \citep{yatracos1985rates}, which has the following advantages: (1) while directly minimizing TV distance is computationally intractable, the OWL methodology decomposes this problem into alternating convex optimization and weighted MLE steps, both of which are standard problems that often tend to be well-behaved, and (2) the OWL methodology provides us with weight vectors that can indicate outlying observations and relates minimum TV-estimation to  likelihood based inference. 

In order to choose $\epsilon \approx \varepsilon_0$ in practice, we define the function $\hat{g}(\epsilon) = \hatI(\hat{\theta}_\epsilon)$, where $\hat{\theta}_\epsilon$ is the parameter estimate computed by the OWL procedure for a given $\epsilon$. At the population level, the corresponding function $g(\epsilon) = \min_{\theta \in \Theta} \okl$ is monotonically decreasing in $\epsilon$ until $\epsilon = \varepsilon_0$, at which point it remains at 0. This introduces a kink, or elbow, at $\epsilon_0$ that we hope to identify in the sample estimate $\hat{g}$. Thus, our $\epsilon$-search procedure is to compute $\hat{g}$ over a fixed grid of $\epsilon$-values, smooth the resulting grid, and then select amongst the points of largest curvature (computed numerically), where the curvature of a twice-differentiable function $f$ at a point $x$ is given by $f''(x)/(1 + f'(x)^2)^{1.5}$~\citep{satopaa2011finding}. Despite the various approximations involved, our simulation results (\Cref{sec:simul}) show that the OWL procedure with such a tuned value of $\epsilon$ provides almost identical performance when compared with the OWL procedure with the true value of $\epsilon$.

%Unlike the population level function $g(\epsilon) = \min_{\theta \in \Theta} \okl$, the estimated function $\epsilon \mapsto \hat{g}(\epsilon)$ is not guaranteed to be non-negative and monotonically decreasing in $\epsilon$. Instead we search for $\epsilon$-values from a grid to find points where the curvature of the function $\hat{h}(\epsilon) = \inf_{r \in [0, \epsilon]} \hat{g}(r)$ is maximized \cite{satopaa2011finding}. Despite the various approximations involved, our simulation results (\Cref{sec:simul}) show that the OWL procedure with such a tuned value of $\epsilon$ provides almost identical performance when compared with the OWL procedure with the true value of $\epsilon$.

\subsection{OWL extension to non-identically-distributed data}
\label{sec:niid}

While the population level analysis and theoretical results for the OKL estimator were derived under the assumption that data are generated i.i.d.~from a distribution $P_0$, the OWL procedure can be adapted to robustify likelihood based inference in the setting where the data are conditionally independent, but not necessarily identically distributed.

Suppose data $z_1,\ldots, z_n \in \cZ$ are conditionally independent, with the likelihood having the product form $p_\theta(z_{1:n}) = \prod_{i=1}^n p_{\theta,i}(z_i)$, for known functions $\{p_{\theta,i}\}_{i=1}^n$. For example, if $z_i = (y_i, x_i) \in \R \times \cX$ for $i=1,\ldots,n$, this includes the case of regression models $\{q_\theta(y|x)\}_{\theta \in \Theta}$ under the setup $p_{\theta, i}(z_i) = q_\theta(y_i|x_i)$. Another example of this setup includes mixture models if we expand the parameter space to also include cluster assignments (see \Cref{sec:mixture-models}).

To robustify inference based on the product likelihood $p_\theta(z_{1:n}) = \prod_{i=1}^n p_{\theta, i}(z_i)$, we can replace the $w$- and $\theta$- steps in \Cref{alg:owl} by analogous steps in the product likelihood case. In particular, the modified $w$-step is given by
$$
w_t = \argmin_{\substack{w \in \Delta_n \\ \frac{1}{2}\|w-o\|_1 \leq \epsilon }} \left\{- \sum_{i=1}^n w_i \log p_{\theta_t, i}(x_i)  + \sum_{i=1}^n w_i \log w_i\right\}
$$
and the modified $\theta$-step is given by
$$
\theta_{t+1}= \argmin_{\theta \in \Theta} \sum_{i=1}^n w_{t,i} \log p_{\theta,i}(x_i).
$$
Despite our lack of theory in the non-identically-distributed case, we continue to see good empirical performance of the OWL estimator in this setup when evaluated on synthetic data (see \Cref{sec:simul}).
 


\section{Performing OWL computations}
\label{sec:comp}
\small
\caption{\small D3DP Inference 
}
\label{alg:sample}
\algcomment{\fontsize{7.2pt}{0em}\selectfont \texttt{\emph{linespace}}: generate evenly spaced values
}
\definecolor{codeblue}{rgb}{0.25,0.5,0.5}
\definecolor{codegreen}{rgb}{0,0.6,0}
\definecolor{codekw}{rgb}{0.85, 0.18, 0.50}
\lstset{
  backgroundcolor=\color{white},
  basicstyle=\fontsize{7.5pt}{7.5pt}\ttfamily\selectfont,
  columns=fullflexible,
  breaklines=true,
  captionpos=b,
  commentstyle=\fontsize{7.5pt}{7.5pt}\color{codegreen},
  keywordstyle=\fontsize{7.5pt}{7.5pt}\color{codekw},
  escapechar={|}, 
  escapeinside={<@}{@>}
}

%\setlength{\belowcaptionskip}{-10pt}
\begin{lstlisting}[language=python, aboveskip=3pt, belowskip=3pt, emph={linespace},emphstyle={\emph}]
def inference(2dp, T, K, H):
  # 2dp: [B, N, J, 2], T: maximum number of timesteps
  # K, H: number of iterations and hypotheses
  
  # Initialize noisy 3d poses: [B, H, N, J, 3] 
  3dp_t = normal(mean=0, std=1)

  # Sample timesteps uniformly
  times = reversed(linespace(0, T, K + 1))
  
  # [(T*(1-k/K), T*(1-(k+1)/K))], k = 0,...,K-1
  time_pairs = list(zip(times[:-1], times[1:])

  for t_now, t_next in zip(time_pairs):   
    # Predict 3dp_0 from 3dp_t
    3dp_0 = denoiser(3dp_t, 2dp, t_now)

    # Diffusion flipping
    # Data augmentation using horizontal flipping
    if augment:
      2dp_hf = horiz_flipping(2dp)
      3dp_t_hf = horiz_flipping(3dp_t)
      3dp_0_hf = denoiser(3dp_t_hf, 2dp_hf, t_now)
      3dp_0 = (3dp_0 + horiz_flipping(3dp_0_hf)) / 2
    
    # Estimate 3dp_t at t_next
    3dp_t = ddim_step(3dp_t, 3dp_0, t_now, t_next)
    
  return 3dp_0
\end{lstlisting}


%\vspace{-10pt}
%<@\textcolor{codekw}{}

% latex
% \small
% \caption{\small D3DP Inference 
% }
% \label{alg:sample}
% \algcomment{\fontsize{7.2pt}{0em}\selectfont \texttt{linespace}: generate evenly spaced values
% }
% \definecolor{codeblue}{rgb}{0.25,0.5,0.5}
% \definecolor{codegreen}{rgb}{0,0.6,0}
% \definecolor{codekw}{rgb}{0.85, 0.18, 0.50}
% \lstset{
%   backgroundcolor=\color{white},
%   basicstyle=\fontsize{7.5pt}{7.5pt}\ttfamily\selectfont,
%   columns=fullflexible,
%   breaklines=true,
%   captionpos=b,
%   commentstyle=\fontsize{7.5pt}{7.5pt}\color{codegreen},
%   keywordstyle=\fontsize{7.5pt}{7.5pt}\color{codekw},
%   escapechar={|}, 
% }
% \begin{lstlisting}[language=python]
% def infer(pose_2d, K, T):

%   # data augmentation using horizontal flipping
%   pose_2d_flip = flipping(pose_2d)
%   return 0
% \end{lstlisting}

%\linespread{0.5}



\section{Asymptotic connection to coarsened inference}
\label{sec:coarsened-inference}
%auto-ignore

%\jk{I don't know if it's worth pointing this out/elaborating on this further, but another another connection of OWL (at least for sufficiently small $\varepsilon$) is with distance-based estimation. In particular, OWL is as an interpolation between MLE/likelihood-based estimation (for $\varepsilon=0$) and distance-based estimation based on the chosen distance $d$ (for $\varepsilon = \min_{\theta \in \Theta}d(p_{\theta}, p_0)$, in which case the $\epsilon$-ball around $p_0$ touches the set $\{p_{\theta}: \theta \in \Theta\}$ at exactly one point---the point $\theta_d^* = \argmin_{\theta \in \Theta}d(p_{\theta}, p_0)$). So if $\varepsilon \in [0, \min_{\theta \in \Theta}d(p_{\theta}, p_0)]$, OWL with a well-chosen $\varepsilon$ can get us the best of both worlds---efficiency from the log likelihood and robustness from the distance. Maybe this might also be better mentioned in the discussion than here (if at all).}

The development of the OWL methodology in \Cref{sec:methodology} followed from a presumed form of misspecification given by \Cref{ass:misspecification}. %However, this is not the only way to frame misspecification. 
An alternative way to frame and address such misspecifications in a probabilistic framework was proposed by \cite{miller2018robust} who introduced Bayesian methodology centered around the concept of a \emph{coarsened likelihood} defined as
%
\begin{equation}
    \label{eq:clike}
    L_\epsilon(\theta|x_{1:n}) \doteq   \prob_\theta\left(\D(\EmpDist{Z_{1:n}}, \EmpDist{x_{1:n}}) \leq \epsilon\right), 
\end{equation}
where $\D$ is a suitably chosen discrepancy between empirical probability measures.
%
Here, 
$\EmpDist{x_{1:n}} = n^{-1}\sum_{i=1}^n \delta_{x_i}$ denotes the empirical distribution of data $x_{1:n}$, and the probability is computed under $\prob_\theta$---the distribution underlying the artificial data $Z_1, \ldots Z_n \iid P_\theta$ from which the random measure $\EmpDist{Z_{1:n}} = n^{-1}\sum_{i=1}^n \delta_{Z_i}$ is constructed. 
%
The coarsened likelihood implicitly captures the likelihood of a probabilistic procedure in which idealized data are first generated by some model $\prob_\theta$ in the model class under consideration, but are then corrupted in such a way that the discrepancy between empirical measures of the idealized data and the observed data is bounded by $\epsilon$. %\new{Under suitable continuity assumptions on $d$, one can show (Lemma \ref{lem:usual-likelihood} in Appendix) that as $\epsilon \to 0$  coarsened likelihood when multiplied by a proportionality factor depending on $x_{1:n}$ and $\epsilon$, converges to the standard likelihood $L(\theta|x_{1:n}) =\prod_{i=1}^n p_\theta(x_i)$.}

When $\D$ is an estimator for the KL-divergence and an exponential prior is placed on $\epsilon$, \cite{miller2018robust} showed that the Bayes posterior based on $L_\epsilon(\theta|x_{1:n})$ could be approximated  by raising the likelihood to a power less than one in the formula for the standard posterior. 
%
However, to obtain a robustified alternative to maximum likelihood estimation, one may wish to maximize $\theta \mapsto L_\epsilon(\theta|x_{1:n})$ directly for a choice of $\D$ that guarantees robustness (e.g. Maximum Mean Discrepancy or the TV distance).
Such an approach would in general be quite challenging since evaluating \cref{eq:clike} corresponds to computing a high-dimensional integral. 

In this section, we show that for large $n$, the coarsened likelihood  
	can be approximately maximized by using the OWL methodology when $\D$ is an estimator for the TV distance. %More precisely, we show that the  log of coarsened likelihood scaled by $n^{-1}$ converges to a suitably variant of the OKL function for  }
%However, supposing that one had the computational resources to do this, could we say what the inferences would look like in the limit for $n\to\infty$?
%In this section, we answer this question; and show that the coarsened likelihood is intimately connected with our OWL methodology. 
Specifically, if the observed data $x_1, \ldots, x_n$ are generated i.i.d.~from some distribution $P_0$ and $\D$ satisfies appropriate regularity conditions, then the negative rescaled coarsened likelihood $-\frac{1}{n} \log L_\epsilon(\theta|x_{1:n})$ asymptotically converges  as $n \to \infty$ to a variant of $I_\epsilon(\theta)$ based on $\D$. Hence, the OWL methodology asymptotically maximizes the coarsened likelihood $\theta \mapsto L_\epsilon(\theta|x_{1:n})$. In \Cref{sec:asymp-finite,sec:asymp-continuous}, we develop this asymptotic connection for finite and continuous spaces, respectively. All proofs for this section can be found in \Cref{sec:coarsened-likelihood-asymptotics}.


% %
% If $\epsilon$ is an exponential random variable with rate parameter $\alpha$ %exponential prior distribution $\operatorname{Exp}(\alpha)$
% and $\D$ is a suitable finite sample estimator of the KL divergence, \cite{miller2018robust} showed the posterior built on this coarsened likelihood can be approximated by uniformly raising the likelihood to the power $\xi_n = \frac{\alpha}{\alpha + n}$.
% %
% In the Bayesian setting, this results in a posterior that places more weight on its prior, which is intuitively appealing when the likelihood is suspected to be misspecified but the prior is not. 
% %
% The choice of power $\xi_n$ also stops the posterior from contracting as the sample size $n$ increases, which prevents over-confidence about the parameters being inferred. 
% %
% While this is reweighting scheme is consequential for the Bayesian setting, it is not useful for frequentist methods: maximizing a uniformly re-weighted likelihood simply replicates standard maximum likelihood estimation.


% \jk{Unclear from this: is the kernel-smoothed TVD an IPM/Integral Probability semimetric (IPS)? I think so; but it's not actually formally stated/proved in the paper. I think it might be nice to do this so that we can phrase the entire theory in terms of IPMs/IPSs.}
% %
% In this work, we build upon the coarsened inference approach of \cite{miller2018robust} by studying  Eqn.~\eqref{eq:clike} for metrics $\D$ on $\cP(\cX)$ that lead to robustness to misspecification for both frequentist and Bayesian methods.
% %
% Throughout, we focus on taking $\D$ as a kernel-smoothed variant of the Total Variation Distance (TVD). 
% %
% Our motivation for doing so is that the TVD is both robust as well as intuitively interpretable as a distance on $\cP(\cX)$. 
% %
% Despite focusing on the TVD however, our theory and methodology is applicable to any Integral Probability Metric (IPM) that is continuous with respect to the weak convergence topology (see Assumption \ref{ass:metric}).
% This includes various popular choices, including the 2-Wasserstein distance and the Maximum Mean Discrepancy with characteristic kernels \cite[Lemma~3]{simon2020metrizing}.
% %
% Note that the vanilla TVD does not satisfy the continuity condition of Assumption \ref{ass:metric}, which is why we use a  kernel-smoothed variant that does (see Definition \ref{def:smoothedtvd}).
% For metrics $\D$ on $\cP(\cX)$ that satisfy our regularity conditions, we can also show  that the coarsened likelihood behaves as we would hope.
% In particular,
% with suitable scaling that depends on both $\epsilon$ and $x_{1:n}$ but not on $\theta$,  $L_\epsilon(\theta|x_{1:n})$ converges to $L(\theta|x_{1:n})=\prod_{i=1}^n p_\theta(x_i)$ as $\epsilon \to 0$ (see Lemma \ref{lem:usual-likelihood}). 


% Coarsened likelihoods are closely related to  recent advances in approximate Bayesian computation (ABC) with summary-free statistics. 
% %
% This literature (e.g.~see \cite{frazier2020robust,bernton2019approximate, legramanti2022concentration} and references therein) has studied certain robustness properties of the coarsened likelihood in \eqref{eq:clike} along with ways to compute a posterior distribution  based on this likelihood via Monte Carlo estimation  by sampling the idealized data $X_{1:n} \iid P_\theta$. In contrast to ABC which is typically used for models in which the likelihood $p_\theta(\cdot)$ is not available or is difficult to evaluate, here we focus specifically on the case when an analytical form for the likelihood $p_\theta(\cdot)$ is available.

% Although the coarsened likelihood is a conceptually appealing way to address the brittleness problem in inference, once $\varepsilon>0$ there is no easy way to evaluate $L_\epsilon(\theta|x_{1:n})$ beyond numerical approximation of a corresponding integral. 
% %
% In this work, we will use large sample asymptotics to simplify this problem.
% %
% Specifically, we show that the coarsened likelihood can ultimately be approximated by a reweighting of individual likelihood terms.


\subsection{Asymptotic connection in finite spaces}
\label{sec:asymp-finite}

%\jk{'density with respect to the counting measure' is a bit non-standard. }

Let $\cX$ be a finite set and denote the space of probability distributions on $\cX$ by the simplex $\Delta_{\cX} \doteq \{q \in [0,1]^{\cX} | \sum_{x \in \cX} q(x) = 1\}$. Let $\{p_\theta\}_{\theta \in \Theta} \subseteq \Delta_{\cX}$ denote the collection of model distributions, and $p_0 \in \Delta_{\cX}$ denote the true data generating distribution. To establish connection of OKL with the coarsened likelihood \cref{eq:clike}, we will take $\D(p, q) = \frac{1}{2}\|p - q\|_1$ to be the TV distance. 

Given this setting, we can show that $-\frac{1}{n} \log L_\epsilon(\theta | x_{1:n})$ converges in probability to the OKL function $I_\epsilon(\theta)$ at rate $n^{-1/2}$, as demonstrated by the following theorem.
\begin{theorem}
\label{thm:finite-cposterior-tv}
Suppose that $I_{\epsilon_0}(\theta) < \infty$ for some $\epsilon_0 > 0$ and let $\delta > 0$. If $\epsilon > \epsilon_0$ and $x_1, x_2, \ldots, x_n \iid p_0$, then with probability at least $1-\delta$,
\[ \left|I_\epsilon(\theta) + \frac{1}{n} \log L_\epsilon(\theta|x_{1:n}) \right| \leq O\left( \frac{|\Xcal|}{\epsilon - \epsilon_0} \sqrt{\frac{1}{n} \log \frac{1}{\delta}}  + \frac{|\Xcal|}{n} \left( |\Xcal| + \log(n) + \frac{1}{\epsilon - \epsilon_0} \right)  \right).\]
\end{theorem}

Our proof hinges on analyzing a quantity that is closely related to $L_\epsilon(\theta | x_{1:n})$:
\[ M_{n, \epsilon}(\theta) = \prob_{Z_{1}, \ldots Z_n \iid p_\theta}\left( \frac{1}{2} \|\EmpDist{Z_{1:n}} - p_0 \|_1 \leq  \epsilon \right). \]
Instead of looking at the distance to the empirical estimator $\EmpDist{x_{1:n}}$ as in $L_\epsilon(\theta | x_{1: n})$, the quantity $M_{n, \epsilon}(\theta)$ considers the distance to the distribution $p_0$ itself. This simplifies matters greatly, and allows us to establish the following result, which is essentially a consequence of Sanov's theorem from large deviation theory~\citep{demboLargeDeviationsTechniques2010}.

\begin{lemma}
\label{lem:finite-sanov-tv}
If $I_{\epsilon_0}(\theta) < \infty$ for some $\epsilon_0 > 0$, then
\[  \left| I_\epsilon(\theta) + \frac{1}{n} \log M_{n, \epsilon}(\theta) \right| \leq {O}\left( \frac{|\Xcal|}{n} \left( |\Xcal| + \log(n) + \frac{1}{\epsilon - \epsilon_0} \right) \right) \]
for all $\epsilon > \epsilon_0$.
\end{lemma}

The rest of the proof of \Cref{thm:finite-cposterior-tv} amounts to establishing that $L_\epsilon(\theta | x_{1:n})$ is close to $M_{n,\epsilon}(\theta)$, which follows from continuity arguments and the fact that $\EmpDist{x_{1:n}}$ converges to $p_0$ in $\ell_1$ distance.


\Cref{thm:finite-okl-convergence-tv} and \Cref{thm:finite-cposterior-tv} together show that, in the large sample limit, the OWL methodology and coarsened likelihood philosophy are two sides of the same coin: they both provide approximations of the OKL and, in turn, must approximate each other.

\subsection{Asymptotic connection in continuous spaces}
\label{sec:asymp-continuous}

Suppose $\Xcal=\R^d$ and $\Den$ denotes the set of densities on $\cX$ with respect to the Lebesgue measure. Let $\{ p_\theta \}_{\theta \in \Theta} \subseteq \Den$ denote the set of model densities and let $p_0$ denote the density of the data generating measure $P_0$.
%, i.e. sample $x_1, \ldots, x_n \in \cX$ is generated i.i.d.~from a distribution $P_0 \in \cP(\cX)$ with density $p_0$. 

Similar to the finite case, we can use Sanov's theorem from Large Deviation theory %(see \Cref{sec:coarsened-likelihood-asymptotic-cont}) 
to establish the following asymptotics for the coarsened likelihood for a suitable class of discrepancies $\D$, which includes the Wasserstein distance, Maximum Mean Discrepancy with  suitable choice of kernels \citep{simon2018kernel},  %\jk{MMD only a pseudometric if the kernel is characteristic (otherwise it doesn't satisfy the identity of discernables); kernels that are charateristic include Gaussian and Matern kernels. We probably don't have to say all that, but we could say 'Maximum Mean Discrepancy based on characteristic kernels', and reference \url{https://www.jmlr.org/papers/volume19/16-291/16-291.pdf}},
and the smoothed TV distance (\Cref{def:smoothed-tvd}).

% \begin{theorem}
% 	\label{thm:clikelihood-asymptotics-density}
% 	Suppose $x_1, \ldots, x_n \iid P_0$, $\epsilon > 0$, $\KL(p_0|p_\theta) < \infty$, and $\D: \cP(\cX) \times \cP(\cX) \to [0,\infty)$ is an integral probability semi-metric that is continuous with respect to the weak convergence topology on $\cP(\cX)$. Then
% 	\begin{equation*}
% 		-\frac{1}{n} \log L_\epsilon(\theta|x_{1:n}) \pconv \inf_{\substack{q \in \Den\\ \D(q, p_0) \leq \epsilon}} \KL(q|p_\theta) \quad 
% 		\text{ as $n \to \infty$.}
% 	\end{equation*}
% \end{theorem}
\begin{theorem}
\label{thm:clikelihood-asymptotics-density}
Suppose $I_{\epsilon_0}(\theta) < \infty$ for some $\epsilon_0 > 0$ and $\D: \cP(\cX) \times \cP(\cX) \to [0,\infty)$ is a pseudometric that is convex in its arguments and continuous with respect to the weak convergence topology on $\cP(\cX)$. If $\epsilon > \epsilon_0$ and $x_1, \ldots, x_n \iid P_0$, then
	\begin{equation*}
		-\frac{1}{n} \log L_\epsilon(\theta|x_{1:n}) \pconv \inf_{\substack{Q \in \cP(\cX) \\ \D(Q, P_0) \leq \epsilon}} \KL(Q|P_\theta) \quad 
		\text{ as $n \to \infty$.}
	\end{equation*}
\end{theorem}

Recall that the limiting expression in the above theorem has the same form as that of the OKL function given in \cref{eq:okl-density}. However, in order to establish connection between the OKL function and the coarsened likelihood, unlike in the finite case, we cannot merely take the discrepancy $\D$ in the coarsened likelihood to be the TV distance, since the TV distance between the two empirical distributions in \cref{eq:clike} will almost surely be equal to one. Instead, we will take $\D$ to be a smoothed version of TV distance calculated by first convolving the empirical measures with a smooth kernel function $K_h:\cX \times \cX \to [0,\infty)$ indexed by a bandwidth parameter $h > 0$. 

To formally define the smoothed TV distance, let $\phi \in \Den$ be a continuous and bounded probability density function (e.g.~standard Gaussian density), let $h > 0$ be a bandwidth parameter. Then the kernel $K_h: \cX \times \cX \to [0,\infty)$ is defined as  $K_h(x,y) = \frac{1}{h^d} \phi((x-y)/h)$, and for any measure $\mu \in \cP(\cX)$, the convolved density $K_h \star \mu \in \Den$ is defined as $(K_h \star \mu)(x) = \int K_h(x, y) \mu(dy)$.  

\begin{definition} Given two measures $\mu, \nu \in \cP(\cX)$ and bandwidth $h > 0$, the \emph{smoothed total variation (TV) distance} is defined as:
	$$
	\tvk[h](\mu,\nu) = \frac{1}{2} \int |(K_h \star \mu)(x) - (K_h \star \nu)(x)| dx.
	$$
 
	We extend the notion of smoothed TV distance $\tvk[h](p,q) = \tvk[h](\mu, \nu)$ to densities $p, q \in \Den$\ based on their  induced measures $\mu, \nu \in \cP(\cX)$.
 \label{def:smoothed-tvd}
\end{definition}


We show in \Cref{sec:smoothed-tvd-is-continous-ips} that $\D=\tvk[h]$  satisfies conditions of \Cref{thm:clikelihood-asymptotics-density}.
Further, when $\phi$ has fast tail-decay and densities $p, q \in \Den$ satisfy appropriate regularity conditions, standard results on kernel density estimation (e.g.~\cite{rinaldo2010generalized, jiang2017uniform}) show the pointwise convergence of densities  $K_h \star q \to q$ as $h \to 0$. This, when combined with Scheffe's lemma and the triangle inequality, shows that $\lim_{h \to 0} \tvk[h](p, q) = \tv(p,q)$. In other words, for suitably small bandwidth parameter  $h > 0$, the neighborhoods based on the smoothed total variation distance    approximate those based on the total variation distance.  
   
%is a spherically symmetric probability distribution with exponential tails, tools from \cite{jiang2017uniform} show that the smoothed TV distance $\tvk[h](p,q)$ approximates the TV distance $\tv(p,q)$ whenever the bandwidth parameter $h$ is sufficiently small. 

Thus, by invoking \Cref{thm:clikelihood-asymptotics-density} with the choice $\D=\tvk[h]$, one expects $- \frac{1}{n} \log L_\epsilon(\theta|x_{1:n}) \approx I_\epsilon(\theta)$ when $n$ is large and $h$ is small. As in the finite setting, we again see that maximizing the coarsened likelihood is closely related to minimizing the OKL function in the large sample regime. Hence the OWL methodology can be used to approximately maximize the coarsened likelihood when $\D=\tvk[h]$  for large sample size $n$ and a suitably small bandwidth $h$. In fact for many other metrics $\D$ satisfying the conditions of \Cref{thm:clikelihood-asymptotics-density}, one can adapt the OWL methodology to maximize the function $\theta \mapsto L_\epsilon(\theta|x_{1:n})$ as $n \to \infty$.





\section{Simulation Examples}
\label{sec:simul}


\section{Simulations} \label{simulations_section}
The \texttt{R} code with the implementations of the algorithms presented in the previous section and the code for the simulations and the application can be found in the supplementary package or \url{https://github.com/jurobodik/Causal_CPCM.git}.

In this section, we illustrate our methodology under controlled conditions. We first consider the bivariate case in which $X_1$ causes $X_2$. We select different distribution functions $F$ in the CPCM model, different forms of $\theta$, and different distributions of $\varepsilon_1$. We recreate a few of the theoretical results presented in Section 2. 


\subsection{Pareto case following Example \ref{example_Pareto} and Consequence  \ref{paretoidentifiability} }

Consider the Pareto distribution function $F$; functions $p_{\varepsilon_1}(x),\theta(x)$ are defined similarly as in (\ref{eq50}). Specifically, we choose $p_{\varepsilon_1}(x)\propto \frac{1}{ [\log(x)+1] x^{2} }$ and $\theta(x) = x^\alpha log(x) +1$ for some hyper-parameter $\alpha\in\mathbb{R}$. This $\alpha$ represents the distortion from the unidentifiable case. If $\alpha = 0$, we are in the unidentifiable case described in Consequence \ref{paretoidentifiability}. If $\alpha> 0$, Consequence \ref{paretoidentifiability} suggests that we should be able to distinguish between the cause and the effect. If $\alpha<0$, then $\theta$ is almost constant (function $\frac{log(x)}{x^{-\alpha}}$ is close to zero function on $x\in [1, \infty)$) and $(X_1, X_2)$ are (close to) independent.

For the size of the dataset $n =300$ and  $\alpha \in \{  -2,   0,  2\}$, we simulate data as described above. Using our $CPCM(F)$ algorithm from Section \ref{Section_Algorithm}, we obtain an estimate of the causal graph.  After averaging results from 100 repetitions, we obtain the results described in Figure \ref{Pareto_simulations1}. The resulting numbers are as expected: if $\alpha = 0$, then both directions tend to be plausible. If $\alpha >0$, we tend to estimate the correct direction $X_1\to X_2$; if $\alpha<0$, then we tend to estimate an empty graph since $X_1, X_2$ are (close to) independent.  


\begin{figure}[ht]
\centering
\includegraphics[scale=0.5]{figures/1.pdf}
\caption{Simulations corresponding to the CPCM model with Pareto distribution function $F$. The results represent our estimations of the graph structure with the $CPCM(F)$ algorithm from Section \ref{Section_Algorithm}. The green line represents the case when both directions are plausible (we do not reject the independence test in both directions). The yellow line represents the case in which both directions are unplausible (we reject the independence test in both directions. This case did not occur). }
\label{Pareto_simulations1}
\end{figure}



\subsection{The Gaussian case and comparison with baseline methods}
\label{Section_simulations_Gaussian}
For simulated data, we use the benchmark dataset introduced in \cite{Natasa_Tagasovska}. The dataset consists of additive and location-scale Gaussian pairs of the form $X_2= \mu(X_1)+\sigma(X_1)\varepsilon_2$, where $\varepsilon_2\sim N(0, 1)$, $X_1\sim N(0, \sqrt{2})$. In one setup (LSg), we consider  $\mu$ and $ \sigma$  as nonlinear functions simulated using Gaussian processes with Gaussian kernel with bandwidth one \citep{Gaussian_processes}.  In the second setup (LSs), we consider $\mu$ and $ \sigma$ as sigmoids \citep{BuhlmannCAM}. Further, nonlinear additive noise models (ANM) are generated as LS with constant $\sigma(X_1)=\sigma \sim U(1/5, \sqrt{2/5})$ and nonlinear multiplicative noise models (MN) are generated as LS with fixed $\mu(X_1)=0$ (only with sigmoid functions for $\sigma$).  For each of the five cases (LSg, LSs, ANMg, ANMs, and MNs), we simulate 100 pairs with $n=1000$ datapoints.

We compare our method with LOCI \citep{immer2022identifiability}, HECI  \citep{xu2022inferring}, RESIT \citep{Peters2014},  bQCD \citep{Natasa_Tagasovska}, IGCI with Gaussian and uniform reference measures \citep{IGCI} and Slope \citep{Slope}. Details can be found in  Appendix \ref{Appendix_simulations}.  As in \cite{reviewANMMooij}, we use the accuracy for forced decisions as our evaluation metric. The results are presented in Table \ref{Table_Simulated_data_Gaussian}.  We conclude that our estimator performs well on all datasets, and provides comparable results with those using LOCI and IGCI (although utilizing the uniform reference measure would lead to much worse results for IGCI).

Note that in the ANM and MN cases, we non-parametrically estimate two parameters while only one is relevant. Therefore, it can happen that we ``overfit'' (see Section \ref{Section5Model_choice}), and both directions are not rejected. To improve our results, fixing either $\mu$ or $\sigma$ could be beneficial (although it may prove challenging to determine this conclusively in practical applications). 

\begin{table}[!ht]
    \centering
    \begin{tabular}{|l|l|l|l|l|l|}
    \hline
        \textbf{} & \textbf{ANMg} & \textbf{ANMs} & \textbf{MNs} & \textbf{LSg} & \textbf{LSs} \\ \hline
        \textbf{Our CPCM} & {100} & 97 & 95 & {99} & {98} \\ \hline
        \textbf{LOCI} & {100} & {100} & {99} & 91 & 85 \\ \hline
        \textbf{HECI} & {99} & 43 & 29 & 96 & 54 \\ \hline
        \textbf{RESIT} & {100} & {100} & 39 & 51 & 11 \\ \hline
        \textbf{bQCD} & {100} & 79 & {99} & {100} & 98 \\ \hline
        \textbf{IGCI (Gauss)} & {100} & {99} & {99} & 97 & {100} \\ \hline
        \textbf{IGCI (Unif)} & 31 & 35 & 12 & 36 & 28 \\ \hline
        \textbf{Slope} & 22 & 25 & 9 & 12 & 15 \\ \hline
    \end{tabular}
    \caption{Accuracy of different estimators on simulated Gaussian datasets. ANM represents additive models, MN represents multiplicative, and LS represents location-scale models. The difference between ANMg and ANMs (LSg, LSs) is how the functions $\mu$ and $\sigma$ are generated. Analogous results (without the first row) can also be found in \cite{Natasa_Tagasovska} and \cite{immer2022identifiability}, with several other estimators from the literature. }
    \label{Table_Simulated_data_Gaussian}
\end{table}



\subsection{Robustness against a misspecification of F}
%add alpha definiton
Consider $X_1\to X_2$, where $X_1\sim N(2,1)^+$, \footnote{$N(2,1)^+$ denotes the truncated Gaussian distribution on $\{x>0\}$. Therefore, $X_1>0$, with a mean of approximately $2.07$.} and let
\begin{equation}\label{eq9870p}
    X_2\mid X_1\sim Exp\big(\alpha(X_1)\big),
\end{equation}
where $\alpha$ is a non-negative function. In other words, we generate $X_2$ according to (\ref{BCPCM}), with $F$ being an exponential distribution function. Recall that the exponential distribution is a special case of Gamma distribution with a fixed shape parameter. 

The goal of this simulation is to ascertain how the choice of $F$ affects the resulting estimate of the causal graph. We consider five different choices for $F$: Gamma with fixed scale, Gamma (with two parameters as in Consequence \ref{consequenceprva}), Pareto, Gaussian with fixed variance, and Gaussian (with two parameters as in Example \ref{Gaussian case}). 


We generate $n=500$ variables, according to (\ref{eq9870p}), with different functions, $\alpha$. Then, we apply the CPCM algorithm with different choices of $F$. Table \ref{table_simulations_about_misspecified_F} presents the percentage of correctly estimated causal graphs (an average out of 100 repetitions). 
The results remain more or less good for $F$ that are ``similar'' to the exponential distribution, with respect to the density and support. However, if we select the Gaussian distribution (a uni-modal distribution with different support), our methodology often provides wrong estimates. 

\begin{table}[tbh]
\centering
\begin{tabular}{|c|c|c|c|c|c|}
\hline
  F       & $\alpha(x) = x$ & $\alpha(x) = x^2+1$  & $\alpha(x) = \frac{e^x}{2}$ & Random $\alpha$ \\ \hline
Gamma (fixed scale)   & $93$           & $95$                                & $97$                       & $92$      \\ \hline 
Gamma (two parameters)  & $82$           & $86$                               & $76$                       & $92$           \\ \hline
Pareto    & $99$           & $100$             & $99$                                   & $97$           \\ \hline
Gaussian (fixed variance) & $0$           & $0$                               & $0$                       & $9$           \\ \hline
Gaussian (two parameters) & $16$           & $25$                           & $33$                       & $41$           \\ \hline
\end{tabular}
\caption{Comparison of the accuracy of CPCM estimations for different choices of $F$. Random $\alpha$ represents a function generated using Gaussian processes, which is similar to Simulations \ref{Section_simulations_Gaussian}.}
\label{table_simulations_about_misspecified_F}
\end{table}













\section{Application to scRNA-seq Clustering}
\label{sec:application}
%auto-ignore

%\jk{This is a really cool application. Probably because I am out of my depth, I feel like we could simplify the presentation a little bit by adding sub-sections; as it seems to me that we have two themes: (i) OWL performance against other methods under pollution [as measured by ARI], (ii) OWL as a tool for descriptive statistics [used for determining inliers/outliers]. Maybe we could make this a bit clearer in how we present things?}

In this section, we apply our OWL methodology to a single-cell RNA sequencing (scRNA-seq) clustering problem. The GSE81861 cell line dataset \citep{li2017reference} contains single-cell RNA expression data for 630 cells from 7 cell lines across 57,241 genes. We followed the preprocessing steps of \cite{chandra2020escaping}: we dropped cells with low reads, normalized according to \cite{lun2016pooling}, and dropped uninformative genes with M3Drop~\citep{andrews2019m3drop}. After preprocessing, the dataset contains 531 cells and 7666 genes.  \Cref{tab:cline-breakdown} shows the breakdown of the remaining cells across cell lines. Finally, we used PCA to project down to 10 dimensions. We implemented OWL using a mixture of general Gaussians, $ \sum_{k=1}^K \pi_k \Ncal(\mu_k, \Sigma_k)$, using the same optimization procedure as in the clustering simulations of \Cref{sec:simul}.
\begin{table}
\centering
 \begin{tabular}{||c | c c c c c c c||} 
 \hline
 Cell line & A549 & GM12878 & H1 & H1437 & HCT116 & IMR90 & K562 \\
 \hline
 Counts & 74 & 126 & 164 & 47 & 51 & 23 & 46 \\ 
 \hline
 \end{tabular}
 \caption{Breakdown of samples in GSE81861 dataset by cell line.}
 \label{tab:cline-breakdown}
\end{table}

\subsection{Cluster recovery with OWL}
We measured the ability of OWL to recover the ground-truth clustering of samples. For baseline methods, we compared against maximum likelihood estimation with the same model class and K-means. %As all methods make use of randomization, we ran each method using 30 different random seeds and tracked the performance of the various runs. 
As a metric of cluster recovery, we measured the adjusted Rand index (ARI)~\citep{rand1971objective,hubert1985comparing}. In all our comparisons, we fixed the number of clusters for all methods to be 7, the number of ground truth cell lines.

The left panel of \Cref{fig:ari-comparison} shows the ARI for OWL over a range of values for the $\ell_1$ radius parameter $\epsilon$, where we also display the performance of MLE and K-means for comparison. We see that OWL performs best when $\epsilon$ takes on values between $0.25$ and $0.45$, but generally has reasonable performance when $\epsilon$ is not too large. Moreover, we see that performance of OWL varies smoothly as a function of $\epsilon$, which may reflect the continuity of the OKL function with respect to $\epsilon$ predicted by our theory.

\Cref{fig:umap-baseline-comparison} shows Uniform Manifold Approximation and Projection (UMAP) visualizations of the dataset clustered under the various methods (for one arbitrary run). We see that of all the methods, K-means performs worst by a significant margin. The improved performance of OWL (with $\epsilon=0.25$) over MLE can be mostly attributed to the better resolution the boundary between the K562 and GM12878. However, all methods struggle to identify the IMR90 cell lines as a cluster distinct from K562.


%\Cref{fig:umap-baseline-comparison} shows UMAP visualizations of the dataset clustered under the various methods (for one arbitrary run). We see that of the baselines, K-Means performs worst by a significant margin. MLE performs better than K-Means, but OWL outperforms MLE for the setting of $\epsilon = 0.2$. \jk{Maybe worth 1-2 sentences why it is legitimate to claim this? It seems that none of the methods are really doing all that well (the brown cluster isn't detected correctly by any method, is it?)} These observations also hold in \Cref{fig:ari-comparison} where we have computed the adjusted Rand index (ARI) for each of the methods across the 30 random seeds, tracking both mean performance and 95\% confidence intervals. \jk{Again, might simply be my ignorance of the subject: is ARI a standard thing? Can we give a reference or short description?} In all these comparisons, we fixed the number of clusters for all methods to be 7, the number of ground truth cell lines. 

\begin{figure}
    \centering
    \includegraphics[width=0.95\textwidth]{figures/ari_comp.pdf}
    \caption{Comparison of clustering methods. \emph{Left}: Adjusted Rand index (ARI) over the entire dataset for each of the methods. \emph{Middle}: ARI of inliers for the OWL methods. \emph{Right}: Fraction of data points classified as inliers for the OWL methods.}
    \label{fig:ari-comparison}
\end{figure}


\begin{figure}
    \centering
    \includegraphics[width=0.8\textwidth]{figures/rnaseq_clustering.png}
    \caption{UMAP plots of the GSE81861 dataset under the considered clustering algorithms. Top left displays the ground truth cell lines. For the other panels, colors were selected by maximizing agreement with the ground truth clustering.}
    \label{fig:umap-baseline-comparison}
\end{figure}



       
%\begin{table}
%\label{tab:ari-breakdown}
%\centering
% \begin{tabular}{||c | c c c c c c c||} 
% \hline
%  & K-Means & MLE & OWL ($\epsilon = 0.05$) &  OWL ($\epsilon = 0.1$) & OWL ($\epsilon = 0.2$) & OWL ($\epsilon = 0.3$) & OWL ($\epsilon = 0.4$) \\
% \hline
% ARI & $0.70 \pm 0.01$ & $0.89 \pm 0.00$  & $0.9 \pm 0.01$ & $0.9 \pm 0.02$ & $0.93 \pm 0.01$ & $0.88 \pm 0.03$ & $0.86 \pm 0.02$ \\
% ARI (inliers) &  &  & $0.91 \pm 0.01$ & $0.92 \pm 0.02$ & $0.98 \pm 0.01$ & $0.95 \pm 0.03$ & $0.96 \pm 0.02$ \\
% Fraction of inliers & & & $0.95 \pm 0.00$ & $0.93 \pm 0.00$ & $0.87 \pm 0.00$ & $0.8 \pm 0.00$ & $0.73 \pm 0.00$ \\
% \hline
% \end{tabular}
%  \caption{Adjusted rand index (ARI) for each of the clustering methods.}
%\end{table}

\subsection{Exploratory analysis with OWL}

In some settings, it is desirable to segment a dataset into those data points that are well-described by a model in the class (so-called \emph{inliers}) and those that do not conform well to the model class (\emph{outliers}). One interpretation of the weights that are learned by the OWL procedure is that, subject to the constraint that they are close in TV distance to the empirical distribution, they represent the most optimistic reweighting of the data relative to the model class. Thus, one might suspect that data points with higher weights are inliers and those with lower weights are outliers. Here, we explore inlier/outlier detection with OWL weights by classifying all data points with weights less than $1/n$ (the average value) as outliers, and the remainder as inliers. 


The middle panel of \Cref{fig:ari-comparison} shows the ARI of the OWL procedure when we restrict to the detected inliers. We observe that for all values of $\epsilon$, the ARI is no lower on the selected inliers than on the whole dataset, and in some cases is significantly higher. This suggests that the OWL procedure identifies a `core' set of points that are both well-described by a mixture of Gaussians as well as aligned with the ground truth clustering. The right panel of \Cref{fig:ari-comparison} shows the fraction of data points that are classified as inliers. Although it is theoretically possible for the OWL weights to classify anywhere from 1 to $n-1$ points as outliers for any value of $\epsilon$, we see that the fraction of outliers is relatively small for low values of $\epsilon$ and only increases gradually as $\epsilon$ increases. 

%The weights that are learned by the OWL procedure can be viewed as indicators of how much the model views a data point as an inlier versus an outlier, where a high weight indicates inlier. As the weights in the OWL procedure are restricted to sum to the number of data points, a natural threshold is the value 1. \Cref{fig:ari-comparison} shows the ARI of the OWL procedure when we restrict to the detected inliers \jk{We haven't explained how we formally determine what an inlier is here (i.e., what's our cutoff?); if we explain it in the appendix we should point to it and if we don't do it there, then I think we should explain it here}. It also displays the fraction of data points labeled as inliers. We can see that for all values of $\epsilon$, the ARI is higher on the selected inliers than on the whole dataset. This suggests that the OWL procedure identifies a `core' set of points that are both well-described by a mixture of Gaussians as well as aligned with the ground truth clustering.


\begin{figure}
    \centering
    \includegraphics[width=0.9\textwidth]{figures/rna_elbow.pdf}
    \caption{\emph{Left}: Weighted log-likehood of the data for various settings of $\epsilon$ and the number of clusters. \emph{Right}: Normalized difference graph of the weighted log-likehood function for select values of $\epsilon$. The kneedle algorithm chooses the value with the largest corresponding normalized difference.}
    \label{fig:k-selection}
\end{figure}



\begin{table}

\centering
 \begin{tabular}{||c | c c c c c c c c c c ||} 
 \hline
  OWL $\ell_1$ radius ($\epsilon$) & 0.05 & 0.15 &  0.25 & 0.35 & 0.45 & 0.55 & 0.65 & 0.75 & 0.85 & 0.95 \\
 \hline
 Selected $K$ & 6 & 6 & 8 & 7 & 7 & 7 & 5 & 4 & 4 & 4 \\
 \hline
 \end{tabular}
  \caption{Number of clusters chosen by the kneedle method as a function of the $\ell_1$ radius $\epsilon$.}
  \label{tab:k-selection}
\end{table}

In many settings, the number of ground truth clusters are not known a priori. A common way to deal with this problem is to plot a metric such as sum-of-squares errors or log-likelihood and look for `elbows' or `knees' in the graph where there are diminishing returns for increasing model capacity. Here, we apply the `kneedle' algorithm~\citep{satopaa2011finding} to the weighted log-likehood produced by the OWL procedure. The kneedle algorithm computes the normalized differences of a given function and selects the value that maximizes the corresponding normalized differences. \Cref{fig:k-selection} shows both the weighted log-likelihoods as well as a subset of the normalized difference graphs. \Cref{tab:k-selection} shows the selected numbers of clusters for various values of $\epsilon$. We see that for relatively small values of $\epsilon$, this results in number of clusters that is close to the ground truth. While for larger values of $\epsilon$, this procedure underestimates the number of clusters in the data. This agrees with the observation in the right panel of  \Cref{fig:ari-comparison} that larger values of $\epsilon$ result in fewer points being identified as inliers, and thus fewer clusters are needed to describe those points.


\section{Application to micro-credit study}
\label{sec:micro-credit}
%auto-ignore
In this section we apply the OWL methodology to data from a micro-credit study \citep{angelucci2015microcredit} for which standard methods of parameter inference have been shown to be brittle to the removal of a handful of observations  \citep{broderick2020automatic}. In \cite{angelucci2015microcredit} the authors conducted a (clustered) randomized trial in Mexico to study the impact of availability of micro-credit on outcome measures in the community including  micro-entrepreneurship, income, labor supply, consumption, social status, and subjective well-being. The authors worked with \emph{Compartamos Banco}, one of the largest micro-lenders in Mexico, to randomize their rollout across 238 geographical regions in the north-central Sonora state in Mexico (close to the Mexico and United States border); within 18-34 months after this rollout, the authors surveyed $n=16,560$ households from these regions for various outcome measures to study the impact of the rollout.

While it is possible to perform a detailed analysis using more outcomes and covariates from the survey data \citep{angelucci2015microcredit}, following \cite{broderick2020automatic}, here we focus on the \emph{Average Intention to Treat effect} (AIT) of the rollout on household profits. More precisely for $i \in \{1, \ldots, n\}$, let $Y_i$ denote the profit of the $i$th household during the last fortnight (measured in USD PPP), and let $T_i \in \{0,1\}$ be a binary variable that is one if and only if the household $i$ falls in the geographical region where the credit rollout happened.  The AIT on household profits is defined as the coefficient $\beta_1$ in the linear model:
\begin{equation}
\label{eq:AITmodel}
Y_i = \beta_0 + \beta_1 T_i + \varepsilon_i,\quad \varepsilon_i \iid N(0, \sigma^2),  \quad i \in \{1,\ldots, n\}. \end{equation}

%\new{
To reproduce the brittleness in estimating the AIT on household profits demonstrated in \cite{broderick2020automatic}, we first obtained the profit data (originally from \cite{angelucci2015microcredit}) as imputed and scaled by \cite{meager2019understanding}. The MLE estimate of $\beta_1 = -4.55$ USD PPP per fortnight (standard error [s.e.] of 5.88), changes to $\beta_1 = 0.4$ USD PPP per fortnight (s.e. 3.19) if we remove a single household identified by the \texttt{zaminfluence} R package \citep{broderick2020automatic}. Moreover, by removing 14 further observations which were identified by the \texttt{zaminfluence} package, we observe that the non-significant value of the MLE estimate can be changed to a significant value of $\beta_1 = -6.01$ USD PPP (s.e. 2.57). As seen in a scatter-plot summarizing the data (\Cref{app:micro-credit}, Figure \ref{fig:micro-scatter}), this brittleness of the MLE is likely due to a small fraction of households with outlying profit values.

Here we compare OWL to this data deletion approach by fitting the model \eqref{eq:AITmodel} to the full data set using 50 $\log_{10}$-spaced $\epsilon$-values between $10^{-4}$ and $10^{-1}$, %for the collection of $\epsilon$ values $\{10^{-4 + 3j/50} : j = 0, \ldots, 50\}$, 
and used the tuning procedure in \Cref{sec:tune-epsilon} to obtain the value $\epsilon_0 = 0.005$ where the minimum-OKL versus epsilon plot (\Cref{app:micro-credit}, \Cref{fig:micro-okl-plot}) has its  most prominent kink. We also calculate the MLE, which corresponds to the OWL procedure with $\epsilon = 0$. The AIT on household profit estimated by OWL as a function of $\epsilon$ can be seen in the left panel of \Cref{fig:micro_fig}. For  values of $\epsilon$ below $\epsilon_0$, the AIT estimates change rapidly as $\epsilon$ changes, while for  values of $\epsilon$ above $\epsilon_0$,  the AIT estimates are quite stable with changes in $\epsilon$. This is due to OWL automatically down-weighting the outlying observations, as seen in the right panel of  \Cref{fig:micro_fig}.  

To quantify uncertainty in the AIT estimates obtained by OWL at the aforementioned grid of  values for $\epsilon$, we reran the above analysis on $m=50$ independently bootstrapped data sets of size $n$ each. Since we wanted to retain a small fraction of outlying observations in each data set, we used an \emph{outlier-stratified} (OS) sampling strategy. Namely, in each iteration, the  new data set was obtained by combining a bootstrap sample of the (roughly $1\%$) households that were down-weighted by the OWL procedure at $\epsilon_0$ and a bootstrap sample from the remaining households that were not down-weighted. 

The resulting 90\% OS-bootstrap confidence bands for estimates of AIT and minimum-OKL as a function of $\epsilon$ can be found in \Cref{app:micro-credit} (also see the left panel in \Cref{fig:micro_fig}). From  \Cref{fig:micro-log-scale-plots} in \Cref{app:micro-credit}, the confidence bands for AIT estimates from OWL are much wider when  $\epsilon < \epsilon_0$ than they are when $\epsilon \geq  \epsilon_0$. Hence, if we presume that the outlying households are the ones down-weighted by OWL at  $\epsilon=\epsilon_0$, the relatively narrow bootstrap confidence bands for the AIT estimates at $\epsilon = \epsilon_0$ suggest that OWL is able to successfully prevent brittleness in estimation due to those outliers.

In summary, the OWL procedure chose to down-weight roughly 1\% of the households with extreme profit values and estimated an AIT of $\beta_1 = 0.6$ USD PPP per fortnight based on the selected value of $\epsilon_0 = 0.005$. The value $\epsilon = \epsilon_0$, tuned using the procedure in \Cref{sec:tune-epsilon}, roughly coincides with the point at which the AIT estimates become stable with respect to  $\epsilon$ and also with the point at which the 90\% OS-bootstrap confidence bands for AIT become narrower --- both suggesting that OWL with the choice $\epsilon = \epsilon_0$ has identified and down-weighted outliers that may be causing brittleness in estimating AIT.

\begin{figure}
    \centering
    \includegraphics[width=0.47\textwidth]{figures/micro_ate_linear.pdf}
    \includegraphics[width=0.47\textwidth]{figures/micro_weights.pdf}
    \caption{Estimating the Average Intent to Treat (AIT) effect on household profits in the micro-credit study \cite{angelucci2015microcredit} in the presence of outliers. Left: the AIT estimates using OWL for various values of $\epsilon$ along with $90\%$ OS-bootstrap vertical confidence bands. The vertical line is drawn at the value $\epsilon_0 = 0.005$ obtained by the tuning procedure in \Cref{sec:tune-epsilon}, and roughly coincides with the $\epsilon$ beyond which the AIT estimates stabilize and the size of the confidence bands shrinks (see \Cref{app:micro-credit}). Right: shows that the weights estimated by OWL procedure at $\epsilon = \epsilon_0$ down-weight roughly  $1\%$ of the households that have outlying profit values (for visual clarity, we omit a down-weighted household with profit less that $-40K$ USD PPP); see also \Cref{fig:micro-outlier-hist-plot} in \Cref{app:micro-credit}.}
    \label{fig:micro_fig}
\end{figure}

\section{Discussion}
We provide some comments on the growth conditions which constituted the majority of our analysis in sections \ref{sec:Hmixing} and \ref{sec:Hsigma}. In the simplest cases of Lemma \ref{lemma:unstableGrowth}, growth was established in an analogous fashion to the old one-step expansion condition (\ref{eq:oldOneStepExpansion}), finding the relevant Jacobians $M_j$ and checking that their expansion factors $K(M_j)$ satisfy
\begin{equation}
    \label{eq:discussionOneStep}
    \sum_j \frac{1}{K(M_j)} <1.
\end{equation}
For the more complicated cases, the inductive method used to establish growth near the accumulation points in Lemma \ref{lemma:unstableGrowth} and the weakened one-step expansion condition (\ref{eq:oneStep}) both address the same fundamental issue: the splitting of unstable curves by singularities into an unbounded number of small components. They circumvent this obstacle in rather different ways, however. While (\ref{eq:oneStep}) generalises (\ref{eq:discussionOneStep}) to ensure an growth of unstable curves `on average' (see \cite{chernov_statistical_2009} for a precise statement), our inductive method is a more direct adaptation of (\ref{eq:discussionOneStep}), using it to generate contradictory geometric conditions which a hypothetical non-growing unstable curve must satisfy. It may be possible to prove Theorem \ref{sec:Hmixing} using (\ref{eq:oneStep}) as the basis for growth. Since we required (\ref{eq:oneStep}) anyway for proving Theorem \ref{thm:HsigmaExp}, this could potentially condense our analysis, but only to a minor extent. A convenience of the method used in section \ref{sec:Hmixing} is that, by way of the `simple intersection' property, it naturally gives geometric information on the images of manifolds, useful for proving the property \textbf{(M)} of Theorem \ref{thm:katok-strelcyn}.

We expect that essentially analogous analysis can be applied to establish mixing properties in a wide class of piecewise linear non-uniformly hyperbolic maps, including those (like the OTM) which sit on the boundary of ergodicity and beyond. While we have relied on the precise partition structure of $H_\sigma$, its fundamental feature (self-similar sequences of elements $A^k$, sharing boundaries with its neighbours $A^{k-1},A^{k+1}$ and accumulating onto some point $p$) is quite typical to return map systems. See, for example, those of various stadium billiards \cite{chernov_chaotic_2006,chernov_improved_2008,chernov_statistical_2009} and LTMs \cite{springham_polynomial_2014}. Indeed, the same method can be used to prove the Bernoulli property for non-monotonic LTMs \cite{myers_hill_mixing_2022}, where monotonicity of the manifold images cannot be assumed and the classical argument \cite{sturman_mathematical_2006} fails. The OTM is the pointwise limit of these maps as the boundary shrinks to null measure. It further has utility in proving growth conditions for maps which are uniformly hyperbolic but possess regions $A_j$ where the hyperbolicity is very weak, signified by $K(M_j) \approx 1$, so that (\ref{eq:discussionOneStep}) fails. Typically this leads to suboptimal bounds on mixing windows, see e.g. \cite{wojtkowski_model_1981,przytycki_ergodicity_1983,myers_hill_family_2022}. The map $H_{(\eta,\eta)}$ for $\eta \approx 1/2$ is another example, possessing weak hyperbolicity over $A_2, A_3$. Letting $\varepsilon = |\eta-1/2|>0$, there is an upper bound $N = N(\varepsilon)$ on escape times from the intersections $A_2\cap \sigma, A_3 \cap \sigma$. The growth lemma then follows by applying the inductive step roughly $N$ times and can be established for arbitrarily small $\varepsilon$, opening the door to establishing optimal mixing windows.

The above gives two examples of piecewise linear perturbations to $H$ where mixing with respect to Lebesgue is preserved and our methods can be applied. Nonlinear perturbations to the shear profiles complicate the analysis in several ways. Firstly as the map's Jacobians takes on a broader range of values, cone invariance becomes an increasingly harder condition to establish. Cones must be widened, giving looser bounds on expansion factors, which may already be weak due to new regions of weaker stretching. This, together with the change from polygonal to curvilinear return time partition elements and nonlinear local manifolds, adds some complexity to showing growth conditions. This does not rule out certain (small) nonlinear perturbations however. There is some leeway in the inequalities which govern cone invariance and growth of local manifolds, the latter of which is not too dissimilar from the piecewise linear setting (see Lemmas \ref{lemma:piecewiseApprox}, \ref{lemma:componentLength}). Certain small perturbations would not alter the \emph{topological} structure of the return time partition, i.e. which elements share boundaries, the key information needed for setting up the induction. Finally while the partition elements would no longer be polygonal, only coarse geometric information is required for verifying each inductive step. Following the above, a potential perturbation could be to replace the linear portions of each shear by a cubic, perturbing the tent profile
\[  f(t) = \begin{cases} 2t & 0 \leq t \leq 1/2, \\ 2(1-t) & 1/2 \leq t \leq 1 ,\end{cases} \]
of the OTM shears to
\[  f_a(t) = \begin{cases} \frac{1}{8} t \left(16 - a + 6at - 8at^{2} \right) & 0 \leq t \leq 1/2, \\ \frac{1}{8}\left(1-t\right)\left( 16 - a + 6a\left(1-t\right) - 8a\left(1-t\right)^{2}\right)  & 1/2 \leq t \leq 1, \end{cases}   \]
for $a>0$. For small enough $a$ the gradient range $f'(t)$ is restricted to small neighbourhoods of $\{ 2, -2\}$ and the escape time partition retains a similar structure. We illustrate this in Figure \ref{fig:perturbations}, showing escapes from the square $S_3$ under the map $G \circ F$, equivalent to escapes from the perturbed $A_3$ under the $G \circ F$, but with a cleaner geometry for comparison. When $a$ is too large the analogy to the OTM breaks down. At $a=16$ the map is twice differentiable everywhere and features a new source of slowed mixing, the Jacobian is the identity at the corner points $x,y \in \{  0, 1/2 \}$ giving locally parabolic behaviour (visible in the escape time partition). 

\begin{figure}
    \centering
    \includegraphics[width=0.24 \linewidth]{0.png}
    \includegraphics[width=0.24 \linewidth]{4.png}
    \includegraphics[width=0.24 \linewidth]{8.png}
    \includegraphics[width=0.24 \linewidth]{16.png}
    \caption{Partition of escape times from $S_3$ under the mapping $F \circ G$ for $a= 0,4,8,16$. }
    \label{fig:perturbations}
\end{figure}

% \bibliographystyle{plain}
\bibliographystyle{abbrvnat}
\bibliography{refs}

\appendix

% \section{More discussion}
% %auto-ignore
\subsection{Contrast between OWL and DRO}
One may contrast this optimistic  re-weighting with adversarial data re-weighting that has been used in the literature on distributionally robust risk minimization \cite{duchi2021learning, duchi2016statistics}. 
\jk{So there is a connection to distributional robustness (namely, that we look for the best interpretation rather than the worst interpretation for the data in an $\varepsilon$-ball in the space of prob measures). Distributional robustness is itself a sub-category of adversarial robustness, where the uncertainty set is taken over the space of probability measures (rather than the data space itself). So we do the exact \textbf{opposite} of adversarial robustness. We look for the most optimistic/charitable interpretation of the data rather than the worst interpretation. The form of robustness that this gives us is quite different from the robustness you get from DRO: DRO makes sense if you learn a model/parameter on some distribution $P$, and then use that model on some other distribution $P'$ which may be slightly different from $P$. We tackle outlier robustness, which means that we want to ignore those parts of $P$ that increase the loss (specifically $-\log p(x|\theta)$) the most. 
%
All that being said, there is a kind of spiritual relationship between OWl and DRO: we treat the data $P$ as already having been polluted (by some adversary), and we expect that any future data will be unpolluted (by this adversary); so we seek to undo the perturbation of the adversary to get us closer to the parametric model that we posed (and which we think would be correct in the absence of said adversarial contamination). DRO is different because rather than asking 'how can we undo the contamination?' it asks 'how can we guard against contamination so that our inferences remain useful in a future where data is contaminated by an adversary?'}

\section{Properties of the OKL}
\label{sec:useful-lemmas}
%auto-ignore

This section covers some technical lemmas about the OKL function that will be used in the theoretical results in the remainder of the appendix. When showing the asymptotic connection between coarsened likelihood and OKL (\Cref{sec:coarsened-likelihood-asymptotics}), we will allow for more general distances than the total variation (TV) distance. %\jk{which parts? Maybe we could be  explicit here about which arguments rely on the TV and which ones don't and give a brief summary?}, 
%
To this end, let $\D(P, Q)$ denote a general distance between probability measures $P, Q \in \cP(\cX)$. We define the OKL function for a general distance $\D$ by
\begin{align}
	\label{eqn:okl-general-distance}
	I_\epsilon(\theta) = \inf_{\substack{Q \in  \cP(\cX) \\ \D(Q, P_0) \leq \epsilon}} \KL(Q | P_\theta). 
\end{align}

Although such extended analysis with  general distances might be possible, for simplicity, we will restrict to the case of $\D=\tv$ while proving the sandwiching property of the I-projection (\Cref{sec:sandwiching}) and the consistency of the OKL estimator in continuous spaces (\Cref{sec:okl-estimation-cont}).

\subsection{Continuity of OKL in the coarsening radius}

The following lemma shows when we can expect the OKL function to be continuous in $\epsilon$.
\begin{lemma}
\label{lem:okl-continuity}
Let $0 \leq \epsilon_0 < \epsilon$ and $\alpha > 0$. Suppose the function $Q \mapsto \D(Q, P_0)$ is convex, then
\[ 0 \leq I_{\epsilon}(\theta) - I_{\epsilon+\alpha}(\theta) \leq \frac{\alpha}{\epsilon - \epsilon_0 + \alpha} I_{\epsilon_0}(\theta). \]
If we additionally have $\alpha \leq \epsilon - \epsilon_0$, then 
\[ 0 \leq I_{\epsilon-\alpha}(\theta) - I_{\epsilon}(\theta) \leq \frac{\alpha}{\epsilon - \epsilon_0} I_{\epsilon_0}(\theta). \]
\end{lemma}
\begin{proof}
We will prove the first statement, as the second follows identically.
First observe that we always have $I_{t}(\theta) - I_{t'}(\theta) \geq 0$ for all $t' \geq t \geq 0$. Moreover, if $I_{ \epsilon_0}(\theta)$ is infinite, then the above holds trivially. Thus, we may assume that $I_{\epsilon_0}(\theta) < \infty$. 

Pick $\delta > 0$. By the definition of OKL, for any $r>0$ there exists $Q_{r} \in \cP(\cX)$ such that $\D(Q_{r}, P_0) \leq r$ and 
\[ \KL(Q_{r} | P_\theta) \leq I_{r}(\theta) + \delta. \]
Take $Q = (1-\lambda)Q_{\epsilon + \alpha} + \lambda Q_{\epsilon_0} \in \cP(\cX)$ for $\lambda = \frac{\alpha}{\epsilon - \epsilon_0 + \alpha}$. By the convexity of $\D$, we have
\[ \D(Q, P_0) \leq (1-\lambda) \D(Q_{\epsilon + \alpha}, P_0) + \lambda \D (Q_{\epsilon_0}, P_0) \leq (1-\lambda)(\epsilon + \alpha) + \lambda \epsilon_0 \leq \epsilon.  \]
Moreover, by the convexity of KL divergence (see e.g.~\cite[Lemma~2.4]{budhiraja2019analysis})
\begin{align*}
I_{\epsilon}(\theta) 
\leq \KL(Q | P_\theta)
&\leq (1-\lambda)\KL(Q_{\epsilon + \alpha} | P_\theta) + \lambda \KL( Q_{\epsilon_0} | P_\theta) \\
&\leq (1-\lambda) (I_{\epsilon + \alpha}(\theta) + \delta) + \lambda (I_{\epsilon_0}(\theta) + \delta) \\
&\leq I_{\epsilon + \alpha}(\theta) + \frac{\alpha}{\epsilon - \epsilon_0 + \alpha} I_{\epsilon_0}(\theta) + \delta.
\end{align*}
Rearranging and noting that $\delta > 0$ was chosen arbitrarily, gives us the result in the lemma. The second statement is derived in an identical manner by substituting $\epsilon - \alpha$ in place of $\epsilon$.
\end{proof}


\subsection{Sandwiching property of I-projections}
\label{sec:sandwiching}

In parts of this appendix, we will be dealing with cases where $\Xcal \subseteq \R^d$ and $P_0$ and $P_\theta$ have corresponding densities $p_0, p_\theta \in \Den$ with respect to the Lebesgue measure $\lambda$. 
%
%\jk{Does this continue to hold if $\Xcal$ is bounded? just checking because our truncated estimator [that we use for theory] only works on bounded sets; so if we need the projection to be unique in that setting we may have to assume it separately.}
%
In this setting, if $I_\epsilon(\theta) < \infty$, then by \cite{csiszar1975divergence} there is a ($\lambda$-almost everywhere) unique density  $\iproj \in \Den$ that we will call the information ($I$-)projection such that $\tv(\iproj, p_0) \leq \epsilon$ and $\KL(\iproj|p_\theta) = I_\epsilon(\theta)$. We will show that $\iproj$ satisfies the following \emph{sandwiching} property relative to $p_0$ and $p_\theta$ for any value of $\epsilon > 0$. 

\begin{definition}
For probability vectors $p, q, r \in \Delta_n$, we say that $r$ is \emph{sandwiched} between $p$ and $q$ if $\min(p_i, q_i) \leq r_i \leq \max(p_i, q_i)$ for all $i=1,\ldots, n$. Similarly, if $p, q, r \in \Den$ are probability densities, then we say that $r$ is sandwiched between $p$ and $q$ if the condition  $\min(p(x),q(x)) \leq r(x) \leq \max(p(x),q(x))$ holds for $\lambda$-almost every $x$.
\end{definition}

The following proposition will be important in proving the sandwiching property for the I-projection.
\begin{proposition}
\label{prop:sandwich-kl-tv}
For probability vectors (or densities), if $r$ is sandwiched between $p$ and $q$, then $\tv(r,p) \leq \tv(q,p)$ and $\KL(r|p) \leq \KL(q|p)$.
\end{proposition}

In fact, will prove the above result for any $\phi$-divergence $D_\phi(p,q) = \int \phi(p/q) q d\lambda$ when $\phi$ is a convex function $\phi(1)=0$. The total variation distance ($\phi(x) = |x - 1|$) and KL-divergence ($\phi(x) = x \log x$) will emerge as special cases.

\begin{lemma}
\label{lem:sandwich-phi-div}
Let $\phi: \R \to (-\infty, \infty]$ be a proper convex function with $\phi(1) = 0$. If a density $r$ is sandwiched between densities $p$ and $q$, then $D_\phi(r,q) \leq D_\phi(p,q)$.
\end{lemma}
\begin{proof}
The sandwiching property implies that there is a function $t: \cX \to [0,1]$ such that $r = (1-t) p + t q$.  Hence	
\begin{align*}
    D_\phi(r,q) &= \int \phi((1-t) p/q + t q/q ) q d\lambda \leq \int (1-t) \phi(p/q) q d\lambda + \int t \phi(1) q d\lambda \\
    &= D_\phi(p,q) - \int t \phi(p/q) q d\lambda \leq D_\phi(p,q) - \xi\int t q(p/q - 1) d\lambda = D_\phi(p,q). 
\end{align*}
where the two inequalities follow from the convexity of $\phi$ noting that there is $\xi \in \mathbb{R}$ (called the sub-gradient) such that 
$\phi(x) \geq \phi(1) + \xi(x-1) = \xi(x-1)$ for all $x \in \R$, and the last equality holds since $\int t(p-q) d\lambda = 0$, since $p$ and $r$ are assumed to integrate to one.
\end{proof}
% We also have the following, stronger result for total variation distance.

% \begin{lemma}
%     Let $p, q, r$ denote three probability densities (or vectors). If $\int$
% \end{lemma}

%\subsection{Information projections are bounded}

Now, we will need a simple lemma about total variation distance before we can prove the sandwiching property of the I-projection. For brevity, we will use notations like $\{p \leq q\}$ and $\{q > \max(p_0, p_\theta)\}$ to denote the sets $\{x \in \cX : p(x) \leq q(x)\}$ and $\{x \in \cX: q(x) > \max(p_0(x), p_\theta(x))\}$ respectively. Note that for two densities $p, q \in \Den$, the total variation distance can be expressed as $\tv(p,q) = \int_{p > q} (p-q) d\lambda = \int_{q > p} (q-p) d\lambda = \frac{1}{2} \int |p-q| d\lambda$.


%\jk{In the below theorem, why is $q(x)<r(x)$ strict? It seems like the proof would still work with an inequality ($q(x) \leq r(x)$). Similarly for the other direction.}
\begin{lemma}
\label{lem:tv-transform}
Let $p,q,r \in \Den$. %If either $\{r \leq q\} \subseteq \{p \leq r\}$ or $\{r \geq q\} \subseteq \{p \geq r\}$ then $\tv(p,r) \leq \tv(p, q)$.
If $\{r < q\} \subseteq \{p \leq r\}$ or $\{r > q\} \subseteq \{p \geq r\}$ then $\tv(p,r) \leq \tv(p,q)$.  %Similarly, if $p(x) \geq q(x)$ for all $x$ satisfying $q(x) > r(x)$, then $\tv(q,p) \leq \tv(r,p)$.
\end{lemma}
\begin{proof} 
	Suppose $\{r < q\} \subseteq \{p \leq r\}$, then
	$$
	\tv(p,r) = \int_{\{p > r\}} (p-r) d\lambda \leq \int_{\{p > r\}} (p-q) d\lambda \leq \int_{\{p > q\}} (p-q) d\lambda = \tv(p,q)
	$$
	where the two inequalities follow from the inclusion  $\{p > r\} \subseteq \{r \geq q\} \cap \{p > q\}$.
	Similarly if $\{r > q\} \subseteq \{p \geq r\}$ then, 
	$$
	\tv(p,r) = \int_{\{ p < r\}} (r-p) d\lambda \leq \int_{\{p < r\}} (q-p) d\lambda \leq \int_{\{p < q\}} (q-p) d\lambda = \tv(p,q)
	$$
	since $\{p < r\} \subseteq \{r \leq q\} \cap \{p < q\}$.	
\iffalse
Suppose first that $\{q < r\} \leq \{p \leq q\}$. The case $\{q < r\} \subseteq \{p \geq q\}$ follows symmetrically. Let $S^+ = \{q < r \}$, $S^- = \{ q > r \}$, and $S^= = \{ q = r \}$. Then we have
\begin{align*}
    \int |p(x) - r(x)| \, dx &= \int_{S^+} |p(x) - r(x)| \, dx + \int_{S^-} |p(x) - r(x)| \, dx + \int_{S^=} |p(x) - r(x)| \, dx \\
    &= \int_{S^+} (r(x) - q(x) + q(x) - p(x)) \, dx + \int_{S^-} |p(x)         - r(x)| \, dx \\
    &+ \int_{S^=} |p(x) - q(x)| \, dx \\
    &\geq \int_{S^+} r(x) - q(x) + q(x) - p(x) \, dx + \int_{S^-} (|p(x) - q(x)| - |q(x) - r(x)|) \, dx \\
    &+ \int_{S^=} |p(x) - q(x)| \, dx \\
    &= \int |p(x) - q(x)| \, dx,
\end{align*}
where the inequality follows from the reverse triangle-inequality, and the last line follows from the fact that 
\[ \int_{S^+} (r(x) - q(x)) \, dx = \tv(q, r) = \int_{S^-} (q(x) - r(x)) dx . \qedhere \]
\fi
\end{proof}


\begin{lemma}
\label{lem:info-proj-sandwich}
Let $p_0, p_\theta$ be probability densities satisfying $I_\epsilon(\theta) < \infty$, and let $\iproj$ denote the I-projection of $p_\theta$ onto the set $\{ q \in \Den : \tv(q, p_0) \leq \epsilon \}$.
Then $\iproj$ is sandwiched between $p_0$ and $p_\theta$.
\end{lemma}

%\jk{In the below proof, I don't really understand why we emphasise that $\tv(\bar{q},p_0) \leq \tv(q, p_0)$ and $\KL(\bar{q}|p_\theta) \leq \KL(q|p_\theta)$. If I follow things correctly, this doesn't help us to prove the sandwiching (instead, it's a result of the sandwiching). Is there a reason we emphasise it? Do we use this somewhere else? If so, maybe we should make it part of the actual statement of the lemma.}
\begin{proof}
	
It suffices to show that for any density $q \in \Den$, there is a density $\bar{q} \in \Den$ sandwiched between $p_0$ and $p_\theta$ such that $\tv(\bar{q},p_0) \leq \tv(q, p_0)$ and $\KL(\bar{q}|p_\theta) \leq \KL(q|p_\theta)$. We can then complete the proof by  noting that $\bar{q} = \iproj$ if we set $q = \iproj$. 

Let $q \in \Den$ be given, and suppose that the set $S^+ = \{ q  > \max(p_0,p_\theta) \}$ has non-zero Lebesgue measure. Letting  
\[ v = \int_{S^+} (q - \max(p_0, p_\theta)) \, d\lambda, \]
note that
\[  \tv(q, p_\theta)  = \int_{\{q > p_\theta \}} (q - p_\theta)  \, d\lambda 
\geq v . \]
Define the density
\[ \bar{q}(x) = 
\begin{cases} 
    \max(p_0(x), p_\theta(x)) & \text{ if } x \in S^+ \\   
    q(x) + \frac{v}{\tv(q, p_\theta)}(p_\theta(x) - q(x)) & \text{ if } p_\theta(x) > q(x) \\
    q(x) & \text{ otherwise }
\end{cases}. \]   
Then it is not hard to verify that $\bar{q}$ integrates to one and satisfies $\bar{q}(x) \leq \max(p_0(x), p_\theta(x))$ everywhere. Next,
applying \Cref{lem:tv-transform}
with $p=p_0$, $q=q$ and $r=\bar{q}$, we obtain $\tv(\bar{q}, p_0) \leq \tv(q, p_0)$ since $\{\bar{q} < q\} = S^+ \subseteq \{p_0 \leq \bar{q} \}$ holds.
Additionally, $\bar{q}$ is sandwiched between $p_\theta$ and $q$. Next, \Cref{prop:sandwich-kl-tv} implies that $\KL(\bar{q} | p_\theta) \leq \KL(q | p_\theta)$.



Now let $q \in \Den$ such that $S^+$ is empty but the set $S^- = \{ q < \min(p_0, p_\theta) \}$ is non-empty. Letting 
\[ v = \int_{S^-} (\min(p_0, p_\theta) - q) \, d\lambda, \]
note that $0 < v \leq \tv(q,p_\theta)$, and define the density
\[ \bar{q}(x) = 
\begin{cases} 
    \min(p_0(x), p_\theta(x)) & \text{ if } x \in S^- \\   
    q(x) + \frac{v}{\tv(q, p_\theta)}(p_\theta(x) - q(x)) & \text{ if } p_\theta(x) < q(x) \\
    q(x) & \text{ otherwise }
\end{cases}. \]  
Then observe that $\bar{q}$ is a density and it is sandwiched between $q$ and $p_\theta$. Similar arguments as above using \Cref{lem:tv-transform} and \Cref{prop:sandwich-kl-tv} show that $\tv(\bar{q}, p_0) \leq \tv(q, p_0)$ and $\KL(\bar{q} | p_\theta) \leq \KL(q | p_\theta)$.
\end{proof}


\section{Convergence of OKL estimator in finite spaces}
\label{sec:okl-estimation-finite}
%auto-ignore
In this section, we will work with a finite data space but allow for a more general distance function. To this end, let $\D(p, q)$ denote the distance between probability vectors $p, q \in \Delta_\Xcal$. Then the OKL function for a general distance in \cref{eqn:okl-general-distance} translates to this setting as
\begin{equation}
	\label{eq:okle-general-distance-finite}
	 I_\epsilon(\theta) = \inf_{\substack{q \in {\Delta}_{\Xcal} \\ \D(p, p_0) \leq \epsilon}} \KL(q | p_\theta). 
\end{equation}

Given a dataset $x_1, \ldots, x_n \in \Xcal$, the finite approximation to the OKL for general distance $\D$ is given by
\begin{align}
\label{eqn:okl-estimator-general-distance}
\finitehatI(\theta) = \inf_{\substack{q \in {\Delta}_{\Xcal} \\ \supp(q) = \hX_n \\ \D(q, \pfinite) \leq \epsilon}} \sum_{x \in \hX_n} q(x) \log \frac{q(x)}{p_\theta(x)},
\end{align}
where $\hX_n = \{x_1, \ldots, x_n \}$ is the observed support and $\pfinite(y) = \frac{|\{i \in [n] | x_i = y\}|}{n}$. When $\D(p, q) = \frac{1}{2}\|p - q\|_1$ is the total variation distance, the above is equivalent to the form given in \cref{eq:finiteokle} as shown in the following lemma.
\begin{lemma}
\label{lem:alternate-finite-okl-form}
For any $\epsilon > 0$ and $x_1, \ldots, x_n \in \Xcal$,
\begin{align*}
	\inf_{\substack{q \in {\Delta}_{\Xcal} \\ \supp(q) = \hX_n \\ \frac{1}{2}\|q - \pfinite\|_1 \leq \epsilon}} \sum_{x \in \hX_n} q(x) \log \frac{q(x)}{p_\theta(x)}  
	\ = \ 
	\inf_{\substack{w \in {\Delta}_{n} \\ \frac{1}{2}\|w - o\|_1 \leq \epsilon}} \sum_{i=1}^n w_i \log \frac{w_i n \pfinite(x_i)}{p_\theta(x_i)}. 
\end{align*}
where $o=(1/n, \ldots, 1/n) \in \Delta_n$.
\end{lemma}
\begin{proof}
For convenience, let $n_i = n \pfinite(x_i)$ and let $n(x) = n \pfinite(x)$. For any $q \in {\Delta}_{\Xcal}$ satisfying $\supp(q) = \hX_n$, let $w_{q} \in \Delta_n$ satisfy $w_{q,i} = q(x_i)/n_i$. Then we have
\begin{align*}
&\| w_q - o \|_1 = \sum_{i=1}^n \left|\frac{q(x_i)}{n_i} - \frac{1}{n} \right| = \sum_{x \in \hX_n} n(x) \left|\frac{q(x)}{n(x)} - \frac{1}{n} \right| = \| q - \pfinite \|_1 \text{ and } \\
&\sum_{i=1}^n w_{q,i} \log \frac{w_{q,i} n_i}{p_\theta(x_i)} 
= \sum_{i=1}^n \frac{q(x_i)}{n_i} \log \frac{q(x_i)}{p_\theta(x_i)} 
= \sum_{x \in \hX_n} q(x) \log \frac{q(x)}{p_\theta(x)}.
\end{align*}
The above two equalities imply that
\begin{align*}
	\inf_{\substack{q \in {\Delta}_{\Xcal} \\ \supp(q) = \hX_n \\ \frac{1}{2}\|q - \pfinite\|_1 \leq \epsilon}} \sum_{x \in \hX_n} q(x) \log \frac{q(x)}{p_\theta(x)}  
	\ \geq \ 
	\inf_{\substack{w \in {\Delta}_{n} \\ \frac{1}{2}\|w - o\|_1 \leq \epsilon}} \sum_{i=1}^n w_i \log \frac{w_i n \pfinite(x_i)}{p_\theta(x_i)}. 
\end{align*}

Now for any $w \in \Delta_n$, let $q_w \in {\Delta}_{\Xcal}$ satisfy $q_w(x) = \sum_{i: x_i = x} w_i$. Note that we trivially must have $\supp(q_w) = \hX_n$. Moreover, we have
\begin{align*}
&\|q_w - \pfinite \|_1 = \sum_{x \in \hX_n} \left| \left( \sum_{i: x_i = x} w_i \right) - \frac{n(x)}{n} \right| 
\leq \sum_{x \in \hX_n} \sum_{i: x_i = x}  \left| w_i - \frac{1}{n} \right|
= \| w - o \|_1 \text{ and } \\
&\sum_{x \in \hX_n} q_w(x) \log \frac{q_w(x)}{p_\theta(x)}
= \sum_{x \in \hX_n} \left( \sum_{i: x_i = x} w_i \right) \log \frac{ \left( \sum_{i: x_i = x} w_i \right)}{p_\theta(x)} 
\leq  \sum_{i=1}^n w_i \log \frac{w_i n_i}{p_\theta(x_i)},
\end{align*}
where the first inequality is the triangle inequality and the second inequality is the log sum inequality~\cite[Theorem~2.7.1]{cover2006elements}.
Together, the above implies that
\begin{align*}
	\inf_{\substack{q \in {\Delta}_{\Xcal} \\ \supp(q) = \hX_n \\ \frac{1}{2}\|q - \pfinite\|_1 \leq \epsilon}} \sum_{x \in \hX_n} q(x) \log \frac{q(x)}{p_\theta(x)}  
	\ \leq \ 
	\inf_{\substack{w \in {\Delta}_{n} \\ \frac{1}{2}\|w - o\|_1 \leq \epsilon}} \sum_{i=1}^n w_i \log \frac{w_i n \pfinite(x_i)}{p_\theta(x_i)}. \ \ \ \qedhere
\end{align*}
\end{proof}

To prove convergence of the estimator in \cref{eqn:okl-estimator-general-distance}, we will make the following assumptions on the space $\Xcal$ and distance $\D$.
\begin{assume}
\label{assum:finite-continuous-distance}
$\Xcal$ is a finite set, $\D$ is a metric, $\D$ is jointly convex in its arguments, and there exists a constant $C \geq 1$ for which $\D(p, q) \leq C \|p - q\|_1$ for all $p,q \in \Delta_\Xcal$.
%$\frac{1}{C} \|p\|_1 \leq \| p\| \leq C \|p\|_1$.
\end{assume}
Observe that \Cref{assum:finite-continuous-distance} fulfills the conditions of \Cref{lem:okl-continuity}, implying that $I_{\epsilon}(\theta)$ is continuous in $\epsilon$. To get quantitative bounds, we make the following assumption.
\begin{assume}
\label{assum:finite-okl}
There exist constants $V, \epsilon_0 > 0$ such that $I_{\epsilon_0}(\theta) \leq V$.
\end{assume}
Finally, we will also require some conditions on the support of $p_\theta$ and $p_0$.
\begin{assume}
\label{assum:lower-bounded-on-support}
There exists a set $S \subseteq \Xcal$ and constant $\gamma_0 > 0$ such that $\supp(p_\theta) = S \subseteq \supp(p_0)$ and $p_0(x) \geq \gamma_0$ for all $x \in S$.
\end{assume}
Given the above assumptions, we have the following convergence result for $\finitehatI(\theta)$. 

%\jk{Perhaps worth remarking somewhere that the below result implies exponential convergence to zero for $\mathbb{P}( \| \pfinite - p_0 \|_1>\varepsilon)$, for all $\varepsilon>0$? Perhaps also not worth.} 

\begin{theorem}
\label{thm:finite-okl-convergence}
Pick $\epsilon > \epsilon_0$ and let $n \geq \max \left\{ \frac{1}{\gamma_0} \log \frac{2|S|}{\delta}, \frac{1}{2} \left( \frac{C |S|}{\epsilon - \epsilon_0} \right)^2 \log \frac{4|S|}{\delta} \right\}$, and suppose \Cref{assum:finite-continuous-distance,assum:lower-bounded-on-support,assum:finite-okl} hold. If $x_1,\ldots,x_n \iid p_0$, then with probability at least $1-\delta$, 
\[ |I_\epsilon(\theta) - \finitehatI(\theta)| \leq  \frac{CV |S|}{\epsilon - \epsilon_0} \sqrt{\frac{1}{2n} \log \frac{2|S|}{\delta}}. \]
\end{theorem}
\begin{proof}
If $n \geq \frac{1}{\gamma_0} \log \frac{2|S|}{\delta}$, then with probability $1-\delta/2$, we have $S \subseteq \hX_n$. Moreover, an application of Hoeffding's inequality tells us that with probability at least $1-\delta/2$, we have
\[ \| \pfinite - p_0 \|_1 \leq |S| \sqrt{\frac{1}{2n} \log \frac{4|S|}{\delta}}. \]
By a union bound, both of these events occur with probability at least $1-\delta$. Condition on these two events occurring.

Now observe that any $q \in \Delta_{\Xcal}$ that achieves $\KL_{\Xcal}(q | p^\theta) < \infty$ must satisfy $q(x) = 0$ for all $x \not \in S$. Thus, we may rewrite the OKL and our finite estimator as
\begin{align*}
I_\epsilon(\theta) &= \inf_{\substack{q \in \Delta_{\Xcal}\\ \supp(q) \subseteq S \\ \D(q, p_0) \leq \epsilon}} \KL(q|p_\theta) \\
\finitehatI(\theta) &= \inf_{\substack{q \in \Delta_{\Xcal}\\ \supp(q) \subseteq \hX_n \cap S \\ \D(q, \pfinite) \leq \epsilon}} \KL(q|p_\theta) = \inf_{\substack{q \in \Delta_{\Xcal}\\ \supp(q) \subseteq S \\ \D(q, \pfinite) \leq \epsilon}} \KL(q|p_\theta),
\end{align*}
where we have used the fact that we are conditioning on $S \subseteq \hX_n$.

Moreover, by \Cref{assum:finite-continuous-distance} and our bound on $\| \pfinite - p_0 \|_1$, we have:
\[ \D(q, \pfinite) \leq \D(q, p_0) + \D(\pfinite, p_0) \leq \D(q, p_0) + C \|\pfinite - p_0\|_1 \leq \D(q, p_0) + \alpha_n,  \]
where $\alpha_n = C |S| \sqrt{\frac{1}{2n} \log \frac{2|S|}{\delta}}$. Similarly, we also can conclude $\D(q, \pfinite) \geq \D(q, p_0) - \alpha_n.$

Applying \Cref{lem:okl-continuity}, we have
\begin{align*}
\finitehatI(\theta) &= \inf_{\substack{q \in \Delta_{\Xcal}\\ \supp(q) \subseteq S \\ \D(q, \pfinite) \leq \epsilon}} \KL(q|p_\theta) 
\geq \inf_{\substack{q \in \Delta_{\Xcal}\\ \supp(q) \subseteq S \\ \D(q, p_0) \leq \epsilon + \alpha}} \KL(q|p_\theta) \\
&= I_{\epsilon + \alpha_n}(\theta) \geq I_{\epsilon}(\theta) - \frac{\alpha_n}{\epsilon - \epsilon_0 + \alpha_n} V.
\end{align*}
On the other hand, if $n \geq \frac{1}{2} \left( \frac{C |S|}{\epsilon - \epsilon_0} \right)^2 \log \frac{4|S|}{\delta}$, then $\alpha_n \leq \epsilon - \epsilon_0$, and we can again apply \Cref{lem:okl-continuity} to see that
\begin{align*}
\finitehatI(\theta) &= \inf_{\substack{q \in \Delta_{\Xcal}\\ \supp(q) \subseteq S \\ \D(q, \pfinite) \leq \epsilon}} \KL(q|p_\theta) 
\leq \inf_{\substack{q \in \Delta_{\Xcal}\\ \supp(q) \subseteq S \\ \D(q, p_0) \leq \epsilon - \alpha}} \KL(q|p_\theta) \\
&= I_{\epsilon - \alpha_n}(\theta) \leq I_{\epsilon}(\theta) + \frac{\alpha_n}{\epsilon - \epsilon_0} V.
\end{align*}
Rearranging the above inequalities gives us the lemma statement.
\end{proof}


\iffalse
\md{Next lemma needs changing notation to $F(r) = I_r$. Also the condition $\KL(p^0|p^\theta) < \infty$ and Assumption \ref{ass:dist-finite} might be too strong as the above lemmas point out.}
\begin{lemma} Suppose $\D$ satisfies assumption \ref{ass:dist-finite} and $\theta \in \Theta$ is such that $\KL(p^0|p^\theta) < \infty$. Then the function $F: [0,\infty) \to [0,\infty)$ given by $F(r) = -I_r(\theta)$ is continuous. (Note that $F(r) \leq \KL(p^0|p^\theta) < \infty$ for each $r \geq 0$.)
\label{lem:continuity}
\end{lemma}
\begin{proof} 
 Using our continuity assumption on $\D$, let us first show that the optimization problem in $I_r$ attains its minimum value at some $q_r \in A_r \doteq \{ q \in \Delta_{\cX} \mid \D(q, p^0) \leq r \}$. 
Indeed, this follows since $A_r$ is a compact subset and the function $q \mapsto \KL_{\cX}(q|p^\theta)$ is lower semi-continuous.

Let us now show that $\liminf_{h \to 0} F(r_0+h) \geq F(r_0)$ for any $r_0 \in [0,\infty)$ (with the convention that $F(r)=\infty$ when $r < 0$). Indeed, for any sequence $\{h_n\}_{n \in \nat}$ that is converging to zero, the sequence $\{q_{r_0 + h_{h_n}}\}_{n \in \nat} \subseteq \Delta_{\cX}$ is pre-compact. Hence, there is an increasing subsequence $\{n_k\}_{k \in \nat} \subseteq \nat$ and $q_* \in \Delta_n$ such that $\lim_{k \to \infty} q_{r_0 + h_{n_k}} = q_*$. Note then by the continuity of $\D$, that $\D(q_*, p_0) = \lim_{k \to \infty} \D(q_{r_0 + h_{n_k}}, p_0) \leq \limsup_{k \to \infty} r_0 + h_{n_k} = r_0$. Hence, the lower semi-continuity of the $\KL$-divergence shows that $$
\liminf_{k \to \infty} F(r_0 + h_{n_k}) = \liminf_{k \to \infty} \KL_{\cX}(q^{r_0 + h_{n_k}}|p^\theta) \geq \KL_{\cX}(q_*|p^\theta) \geq F(r_0).
$$

Note that $F$ is a non-decreasing function, i.e. $F(r) \leq F(s)$ whenever $s \geq r$. Hence, the result $\liminf_{h \downarrow 0} F(r_0 + h) \geq F(r_0) \geq \limsup_{h \downarrow 0} F(r_0 + h)$ immediately shows the right continuity of $F$, i.e. $F(r_0) = \lim_{h \downarrow 0} F(r_0 + h)$.
%

Now we shall establish the left-continuity of $F$ at some point $r=r_0 > 0$. If $\D(q_{r_0},p^0) < r_0$, then $F(r) = F(r_0)$ for each $r \in [\D(q_{r_0}, p^0), r_0)$ and the left continuity is easily satisfied. Hence, suppose from now on that $\D(q_{r_0},p^0) = r_0$. Next, for any $h \in [0,1]$, denote by $q'_h \doteq (1-h) q_{r_0} + h p^0$ the convex combination between $q_{r_0}$ and $p^0$. By the convexity of KL-divergence
\begin{equation*}
    F(\D(q'_h, p^0)) \leq \KL_{\cX}(q'_h|p^\theta) \leq (1-h)\KL_{\cX}(q_{r_0}|p^\theta) + h \KL_{\cX}(p^0|p^\theta) = (1-h) F(r_0) + h F(0).
\end{equation*} 
 
Take $h \downarrow 0$ to obtain $\limsup_{h \to 0} F(T(h)) \leq F(r_0)$, where $T(h) \doteq \D(\tilde{q}^h, p^0)$. By our assumption, $T: [0,1] \to [0,r_0]$ is a continuous and strictly decreasing function. Hence $T(h)$ is stricly increasing to $T(0)=r_0$  as $h \downarrow 0$. Thus, we have in-fact shown that $\limsup_{h \downarrow 0} F(r_0-h) = F(r_0)$. Finally, monotonicity of $F$ shows $\liminf_{h \downarrow 0} F(r_0 - h) \geq F(r_0)$, and hence we recover the left continuity of $F$.
\end{proof}
\fi


\section{Convergence of OKL estimator in Euclidean spaces}
\label{sec:okl-estimation-cont}
%auto-ignore

In this section, we show consistency (and convergence rates) for the OKL estimator when $\cX = \R^d$ is the Euclidean space. To avoid technical complications with tail estimation, we will restrict our analysis to the case when both the data and the model family are supported on a compact set $S \subseteq \cX$. The results and notation in this section are  self-contained, and can be read independently of other sections. 

\subsection{Introduction}
Consider the data space $\cX = \R^d$ equipped with the Euclidean norm $\|\cdot\|$ and its Borel sigma algebra $\cB$. %\jk{Given that none of the results in this section work on unbounded spaces, why do we introduce the space as all of $\R^d$ (rather than a bounded subset of it)?}  \md{Change $S$ to $\cX$ and $\cX$ to $\R^d$ throughout.}  
%Let $B(x,r) = \{y \in \R^d | \|x-y\| < r\}$ denote the open ball of radius $r > 0$ around $x \in \R^d$, and for any set $A \subseteq \R^d$, let $A_{-r}=\{y \in A \mid B(y,r) \subseteq A\}$ denote the collection of points in $A$ that lie at a distance of at least $r$ from the boundary. We will use $A^c$ to denote the complement of $A$ in $\R^d$. 
We will let $\Den = \{f: \cX \to [0,\infty) \mid \int f(x) dx = 1, f \text{ is $\cB$-measurable}  \}$ denote the set of densities on $\cX$ with respect to the Lebesgue measure $\lambda$. Given the data density $p_0 \in \Den$, model family $\{p_\theta\}_{\theta \in \Theta} \subseteq \Den$ and coarsening radius $\epsilon > 0$, the central object of interest in this section is the OKL function defined as
\begin{equation}
	\label{eqn:okl}
	I_\epsilon(\theta) = \inf_{\substack{q \in \Den \\
			\tv(q,p_0) \leq \epsilon 
	}} \KL(q|p_\theta),
\end{equation}
where, given two densities $p,q \in \Den$, the total variation distance between them is $\tv(q, p) =\frac{1}{2}\int |q(x)-p(x)|dx$, and the KL-divergence is $\KL(p|q) = \int  p(x) \log \frac{p(x)}{q(x)}  dx$ if the absolute continuity condition $\int p(x) \I{q(x) = 0} dx = 0$ is satisfied, otherwise $\KL(p|q) = \infty$.

Given samples $x_1,\ldots, x_n \in \cX$ drawn i.i.d.~from the distribution with density $p_0$, and a suitable kernel $K_h: \cX \times \cX \to \R$, 
%and a suitably small constant $\gamma \in [0,1]$,
we will approximate $I_\epsilon(\theta)$ with the value of a finite-dimensional optimization problem given by
\begin{equation}
	\label{eqn:okle1}
	\hatI(\theta) = \inf_{\substack{w \in \hat{\Delta}_n \\ \frac{1}{2}\|w-o\|_1 \leq \epsilon}} \sum_{i=1}^n w_i \log \frac{n w_i \hat{p}(x_i)}{p_\theta(x_i)},
	%\hatI(\theta) = \inf_{\substack{w \in \hat{\Delta}^\gamma_n \\ \frac{1}{2}\|w-e\|_1 \leq \epsilon}} \sum_{i=1}^n w_i \log \frac{n w_i \hat{p}(x_i)}{p_\theta(x_i)},
\end{equation}
where $\hat{p}$ is a suitable density estimator for $p_0$,
%$\Delta_n = \{(v_1,\ldots, v_n)| \sum_{i=1}^n v_i = 1, \gamma \leq nv_i \leq \gamma^{-1} \}$ is subset of
$\Delta_n = \{(v_1,\ldots, v_n)| \sum_{i=1}^n v_i = 1, v_i \geq 0 \}$ is the \ndsimplex , $o=(1/n, \ldots, 1/n) \in \Delta_n$ is the constant probability vector, $A$ is an $n \times n$ matrix with entries $A_{ij} = \frac{K_h(x_i,x_j)}{n \hat{p}(x_i)}$, $\hat{\Delta}_n = A \Delta_n$ is the image of the set $\Delta_n \subseteq \R^n_+$ under the linear operator $A$, and $\|(x_1, \ldots, x_d)\|_1 = \sum_{i=1}^d |x_i|$ is the $\ell_1$ norm of the vector.


\subsubsection{Assumptions and statement of the result}

To describe the formal statement quantifying the approximation between \cref{eqn:okl} and \cref{eqn:okle1}, we will introduce a series of assumptions. Our result will handle the case when densities $p_0$ and $p_\theta$ are smooth densities supported on a bounded set $S \subseteq \cX$ whose boundary has measure zero, i.e.~$\lambda(S \setminus S^\circ) = 0$, where $S^\circ$ denotes the interior of $S$. We fix a value $\theta \in \Theta$ throughout this section.


\begin{assume}
	\label{assump:bounded-support}
	Suppose $S \subseteq \cX$ is a subset of the closed unit ball $\bar{B}(0,R)$ of radius $R$, with finite Lebesgue measure $V_S = \lambda(S) \geq 1$, and has a boundary measure functional $\phi(r) = \frac{\lambda (S \setminus S_{-r})}{\lambda(S)}$ that satisfies $\lim_{r \rightarrow 0} \phi(r) = 0$. 
\end{assume}


Here $S_{-r} = \{x \in S : B(x, r) \subseteq S\}$ denotes the set of points for which the unit ball of radius $B(x,r) = \{y \in \cX | \|x-y\| \leq r\}$ is also contained inside $S$. For example, when $S=B(x_0,r_0)$ is the unit ball of radius $r_0$ centered at the point $x_0$, the boundary measure functional is given by
$$
\phi(r) = 1 - \frac{\lambda(B(x, r_0-r))}{\lambda(B(x, r_0))} = 1 - \left(1 - \frac{r}{r_0} \right)^d
$$
since $S_{-r} = B(x_0, r_0 - r)$. 

Next we have the following smoothness and support condition on densities $p_0, p_\theta \in \Den$. Here the support of a density $p \in \Den$ is defined as $\supp(p) = \{x \in \cX : p(x) > 0\}$.

\begin{assume}
	\label{assump:bounded-densities}
	The supports satisfy $\supp(p_0) = \supp(p_\theta) = S$, and there exists $\gamma \in (0,1]$ such that $p_0(x), p_\theta(x) \in [\gamma, 1/\gamma]$ for all $x \in S$.
\end{assume}

\begin{assume}
	\label{assump:smooth-densities}
	$p_0$ and $\log p_\theta$ are $\alpha$-H\"{o}lder smooth on $S$. More precisely, there are constants $\alpha, C_\alpha > 0$ such that $|p_0(x) - p_0(y)|, |\log p_\theta(x) - \log p_\theta(y)| \leq C_\alpha \|x - y \|^\alpha $ for all $x, y \in S$.
\end{assume}

Finally, we will require the following assumption on the kernel $K_h: \cX \times \cX \to \R$.
% Under these assumptions, given samples $x_1, \ldots, x_n \iid p_0$, we will propose a discretization to \cref{eqn:okl} by restricting the optimization problem over continuous densities of the form $\q{w}(x) = \sum_{i=1}^n w_i K_h(x, x_i)$ for weights $w \in \Delta_n$ chosen from a  probability simplex, and for a suitable probability density kernel $K_h: \cX \times \cX \to \R$. We make the following assumption about the choice of this kernel $K_h$. 

\begin{assume}
\label{assump:kernel-properties}
The kernel satisfies $K_h(x, y) = \frac{1}{h^d}\kappa\left( \frac{\| x - y \|_2}{h} \right)$ for a non-increasing continuous function $\kappa: \R_+ \rightarrow \R_+$ satisfying (i) $\int_{\R^d} \kappa(\|x \|_2)\, dx = 1$, (ii) $\kappa(0) \leq c_0$ for some fixed constant $c_0 >0$, and (iii) there exist constants $t_0, C_\rho, \rho$ such that $\rho \leq 1$ and $t_0 \geq 1$ so that for all $t > t_0$, $k(t) \leq C_\rho \exp(-t^\rho)$. Further, suppose that (iv) for every $h > 0$, the kernel $K_h$ is a positive semi-definite kernel, i.e. the $m \times m$ matrix $(K_h(y_i, y_j))_{i,j \in [m]}$ for any $y_1, \ldots, y_m \in \R^d$ and $m \geq 1$ is positive semi-definite. 
\end{assume}

Parts (i)-(iii) in \Cref{assump:kernel-properties} are standard assumptions in the kernel density estimation literature (see, e.g.~\cite{jiang2017uniform}), while part (iv) allows us to use techniques from reproducing kernel Hilbert spaces. By Shoenberg's theorem (see e.g.~\cite{ressel1976short}) part (iv) of \Cref{assump:kernel-properties} is satisfied whenever $\kappa$ is a completely monotone function (see e.g.~\cite{merkle2014completely}). In particular, note that \Cref{assump:kernel-properties} is satisfied by the Gaussian kernel given as $\kappa(t) = \frac{1}{(2\pi)^{d/2}}\exp(-t^2)$.  

\begin{definition} Suppose \Cref{assump:kernel-properties} holds and observations $x_1, \ldots, x_n \iid p_0$ are given. Define a mapping $L: \Delta_n \mapsto \Den$ from the \ndsimplex\ $\Delta_n$ to the space of densities $\Den$ given by $L(w) = \q{w}$, where
	$$
	\q{w}(\cdot) = \sum_{i=1}^n w_i K_h(\cdot, x_i).
	$$
	\label{defn:qhatw}
\end{definition}

For the probability density estimator $\hat{p}$ in the Monte Carlo estimate of the objective in \cref{eqn:okl}, we use the kernel density estimate $\hat{p}=\q{o}$, where $o=(1/n, \ldots, 1/n) \in \Delta_n$.
%to account for the fact that the points $x_1, \ldots, x_n$ are drawn from $p_0$. We let $\hat{p}$ be the kernel density estimate based on the samples $x_1, \ldots, x_n$ using the kernel $K_h$, noting that $\hat{p}=\q{o}$, where $o=(1/n, \ldots, 1/n) \in \Delta_n$. %To avoid handling tricky biases that are introduced by reusing samples, we will use sample splitting to separate computing our density estimate $\hat{p}$ for the optimization problem \cref{eqn:okl}. To this end, we will assume that we have an i.i.d.~sample $x_1,\ldots, x_{2n} \sim p_0$, and we compute the estimate $\hat{p}(\cdot) = \frac{1}{n}\sum_{i=n+1}^{2n} K_h(x_i, \cdot)$. 

Using the parameterization $w = A v$ for $v \in \Delta_n$, one can rewrite \cref{eqn:okle1} using the above notation as:
$$
\hatI(\theta) = \inf_{\substack{v \in \Delta_n \\ \frac{1}{2n} \sum_{i=1}^n \left|\frac{\q{v}(x_i)}{\hat{p}(x_i)} - 1\right| \leq \epsilon}} \frac{1}{n}\sum_{i=1}^n \frac{\q{v}(x_i)}{\hat{p}(x_i)} \log \frac{\q{v}(x_i)}{p_\theta(x_i)}.
$$



%The estimate $\hat{p}$ will be used in the denominator to compute importance weights of the samples $x_1,\ldots,x_n$, and can thus introduce large deviations unless it is bounded from both above and below. 
In order to ensure stability of the optimization objective in \cref{eqn:okle1} and guarantee regularity in the optimal weights, we will restrict the optimization over weights that do not deviate too far from uniformity and replace $\hat{p}$ by the `clipped' version of our estimator: $\hat{p}_\gamma(x) = \min(\max( \hat{p}(x), \gamma), 1/\gamma)$. To that end, define $\Delta_n^r = \{ v \in \Delta_n : n \cdot v_i \in [r,1/r] \text{ for all } i \}$ for $r \leq 1$. Then for our theoretical result, we will consider a version of the estimator 
\begin{align}
	\label{eqn:I-hat-defn}
	\hatI(\theta) = \inf_{\substack{v \in \Delta_n^{\gamma^2/4} \\ \frac{1}{2n}\sum_{i=1}^n \left|  \frac{\q{v}(x_i)}{\hat{p}_\gamma(x_i)} - 1\right| \leq \epsilon}} \frac{1}{n} \sum_{i=1}^n \frac{\q{v}(x_i)}{\hat{p}_\gamma(x_i)} \log \frac{\q{v}(x_i)}{p_\theta(x_i)},
\end{align}
where $\gamma$ is the value from \Cref{assump:bounded-densities}.

Under the above assumptions, we have the following result:

\begin{theorem}
	\label{thm:okl-convergence-formal}
	Suppose \Cref{assump:bounded-support,assump:bounded-densities,assump:smooth-densities,assump:kernel-properties} hold and $n \geq n_0, h \leq \bar{h}, and \frac{\log n}{\sqrt{n} h^d} \leq \bar{\eta}$ for suitable constants $\bar{h}, n_0, \bar{\eta} > 0$ that depend on the quantities in the assumptions. Fix a $\beta > 0$, and suppose we obtain an i.i.d.~sample $x_1, \ldots, x_n$ of size $n$ from density $p_0$. Then with probability at least $1-e^{-(\log n)^{2\beta}}$,
	\[ \left| \hat{I}_\epsilon(\theta) - I_\epsilon(\theta) \right| \leq O\left( \frac{\log n}{\sqrt{n} h^d} + h^{\min(\alpha/2,1)} \log \frac{1}{h} + \phi(\sqrt{h}) \log \frac{1}{\phi(\sqrt{h})} \right),   \]
	where the $O(\cdot)$ hides constant factors depending on the quantities in Assumptions~\ref{assump:bounded-support}-\ref{assump:kernel-properties}, and on the choice of $\epsilon, \beta > 0$.
\end{theorem}

When $S=B(0,R)$ is a Euclidean ball of radius $R$, Taylor expansion shows that $\phi(\sqrt{h}) = \frac{\sqrt{h}d}{R} + O(h)$.  Suppose, for simplicity that $\alpha \geq 1$, then taking $h=n^{-\frac{1}{2d+1}}$ and $\beta=1$, we see that $|\hatI(\theta) - \okl| = \tilde{O}(n^{-\frac{1}{4d+2}})$ upto a logarithmic factor in $n$, with probability at least $1-1/n$. 

Note that the choice of bandwidth parameter  $h=o(n^{-\frac{1}{2d}})$ required to shrink our error terms to zero decreases more slowly with $n$ than the optimal rate $h=o(n^{-1/d})$ for density estimation \citep{rigollet2009optimal}. This suggests that our error bounds can potentially be improved.
 Note also that the   above result holds for any fixed value of  $\theta \in \Theta$; extensions to uniform convergence over all $\theta \in \Theta$ may be possible with further assumptions on the complexity of the class $\Theta$ and continuity of the map $\theta \mapsto \log p_\theta(\cdot)$, but we do not pursue this direction here.


\begin{proof}[Proof outline of \Cref{thm:okl-convergence-formal}]
We sketch the proof here. 
\begin{description}
    \item[\Cref{sec:uniform-convergence}] First, \Cref{lem:uniform-convergence} shows that with high probability for large $n$, the estimators of KL divergence and total variation in \cref{eqn:I-hat-defn} are close to their population counterparts $\KL(\q{v}|p_\theta)$ and $\tv(\q{v}, p_0)$ for any $v \in \Delta_n$ such that $\q{v}$ is suitably bounded. See  \Cref{cor:all-the-bounds} for the most useful version of this statement. 
\item[\Cref{sec:kernele-density-estimation-bounds}] Next, \Cref{lem:existence-of-good-estimator} shows that for any density $q$ over $S$ satisfying Assumption~\ref{assump:bounded-densities}, with high probability for large $n$, there exists a $v \in \Delta_n^\gamma$ such that $\q{v}$ is a pointwise accurate estimator of $K_h \star q$, where $K_h \star q$ denotes the convolution of the density $q$ with the probability kernel $K_h$. Combined with \Cref{lem:info-proj-sandwich}, which establishes the boundedness of the minimizer $\iproj$ in \cref{eqn:okl}, we have that there exists a pointwise accurate estimator $\q{v^*}$ of $K_h \star \iproj$ for some $v^* \in \Delta_n^{\gamma^2/4}$. %A similar argument shows that $\q{e}$ is a pointwise accurate estimator of $K_h \star p_0$.
    \item[\Cref{sec:okl-smoothed-approx}] \Cref{lem:smoothed-densities-kl-tv-approx} shows that under \Cref{assump:smooth-densities}, $\KL(K_h \star q | p_\theta)$ and $\tv(K_h \star q, p_0)$ must not be much greater than $\KL(q | p_\theta)$ and $\tv(q, p_0)$, respectively, when $h$ is sufficiently small. %Along with the continuity of the map $\epsilon \mapsto I_\epsilon(\theta)$ (\Cref{lem:OKL-is-continuous}), this essentially shows that one can approximate $I_\epsilon(\theta)$ by restricting the optimization problem in \cref{eqn:okl} to densities of the form $K_h \star q$ when $h$ is suitably small. 
\end{description}
We combine the above arguments (see \Cref{sec:proof-of-theorem}), to show that the optimal objective value in \cref{eqn:okl} corresponding to the minimizer $\iproj$, must be close to the optimal objective value in \cref{eqn:I-hat-defn}, thus showing that $\hat{I}_\epsilon(\theta)$ can well approximate $I_\epsilon(\theta)$ when $h$ is small, and $n$ is large.
\end{proof}

While \Cref{sec:proof-of-approx-theorem} covers the main steps of proof, there are useful results in \Cref{sec:useful-lemmmas} that provide upper bounds on the tail probability and second moment of the kernel $K_h$ based on \Cref{assump:kernel-properties}. These results are used in \Cref{sec:okl-smoothed-approx} and \Cref{sec:proof-of-theorem}.


% As discussed earlier, \Cref{sec:proof-of-approx-theorem} covers the main steps of proof. \Cref{sec:useful-lemmmas} contains some key results that support the main proof. Namely
% \begin{description}
% 	\item[\Cref{sec:sandwiching}] shows that the minimizing density in \cref{eqn:okl} also satisfies the upper and lower bounds from \Cref{assump:bounded-densities}; in fact we prove a stronger property that the minimizing density is \emph{sandwiched} between $p_\theta$ and $p_0$. Here we prove an intuitive sandwiching lemma (\Cref{lem:sandwich-phi-div}) that says that a $\phi$-divergence $D_\phi(p|q)$ with $\phi(1)=0$  (e.g.~KL-divergence or total-variation distance) always decreases when its first argument is replaced by a density that is sandwiched between $p$ and $q$. 
% 	\item[\Cref{sec:kernel-tail-bounds}] calculates upper bounds on the tail probability and second moment of the kernel $K_h$ based on  \Cref{assump:kernel-properties}. These results are used in \Cref{sec:okl-smoothed-approx} and \Cref{sec:proof-of-theorem}.
% 	% 	\item[\Cref{sec:uniform-convergence}] shows that objective function and constraint in \cref{eqn:I-hat-defn} approximate suitable KL-divergence and TV terms in \cref{eqn:okl}. 
% 	% 	\item[\Cref{sec:okl-smoothed-approx}] shows that the optimization problem in \cref{eqn:okl} can be well-approximated by replacing $q \in \Den$ with the kernel-smoothed version of the density $\smooth{q}$.
% \end{description}



\paragraph{Additional notation}
For $p \in \Den$ and $A \subseteq \cX$, we will use the shorthand $p(A)$ to denote the quantity $\int_A p(x) dx$.
% The support of $p$, denoted by $\supp(p) \subseteq \cX$, is formally defined as the intersection all of closed sets $C \subseteq \cX$ such that $p(C) = 1$.
For a  probability density kernel $K_h$, let \[\smooth{p}(\cdot) = (K_h \star p)(\cdot) = \int p(y) K_h(\cdot, y) dy \in \Den \] denote the kernel-smoothed version of a density $p \in \Den$. Similarly, for a measure $\mu$ on $\cX$, we can define $(K_h \star \mu) (x) = \int K_h(x, y) \mu(dy)$. For a set $S \subseteq \R^d$ and $p, q \in \Den$, let $\KL_S(q| p) = \int_S q(x) \log \frac{p(x)}{q(x)} \, dx$. The constant $v_d$ will denote the volume of the Euclidean ball in $\R^d$. The notation $\|f\|_1 = \int |f(x)| dx$ and $\|f\|_{\infty} = \sup_{x \in \cX} |f(x)|$ will denote the $L_1$ and $L_\infty$ norm of the function $f$. For $p, q \in \Den$, recall that $\tv(p,q) = \frac{1}{2}\|p-q\|_1.$

%) = 1$.
%
%For a $A \subseteq \cX$, let $\Den[A] = \{p \in \Den \mid \supp(p) \subseteq A\}$ denote the set of densities that are supported on the set $A$, where $\supp(p) = \{x: p(x) > 0\} \subseteq \cX$.  
%
%We sometimes also  consider the  finite dimensional probability simplex $\Delta_n = \{ p \in [0,1]^n | \sum_{i=1}^n p_i = 1 \}$. %For $p, q \in \Delta_n$, let $\supp(p) = \{i \in [n] | p_i > 0\}$, and  $\KL(p|q) = \sum_{i=1}^n p_i \log \frac{p_i}{q_i}$ if $\supp(p) \subseteq \supp(q)$ otherwise $\KL(p|q) = \infty$.


\iffalse
\subsection{Useful lemmas}
\label{sec:useful-lemmmas}


\subsubsection{Sandwiching property of the OKL optimizer}
\label{sec:sandwiching}

A simple consequence of \Cref{assump:bounded-densities} is that $\KL(p_0|p_\theta) \leq \frac{2V_S}{\gamma}\log(1/\gamma)$ is finite. In particular this shows that $I_\epsilon(\theta) < \infty$, and hence by \cite{csiszar1975divergence} there is a ($\lambda$-almost everywhere) unique density  $\iproj \in \Den$ that we will call the information ($I$-)projection such that $\tv(\iproj, p_0) \leq \epsilon$ and $\KL(\iproj|p_\theta) = I_\epsilon(\theta)$. Based on the following notion of \emph{sandwiching}, in this sub-section we will show in that the I-projection $\iproj$ is sandwiched between the two densities $p_0$ and $p_\theta$ for any value of $\epsilon > 0$. 

\begin{definition}
	For probability vectors $p, q, r \in \Delta_n$, we say that $q$ is \emph{sandwiched} between $p$ and $r$ if $\min(r_i, p_i) \leq q_i \leq \max(r_i, p_i)$ for all $i=1,\ldots, n$. Similarly, if $p, q, r \in \Den$ are probability densities, then we say that $q$ is sandwiched between $p$ and $r$ if the condition  $\min(p(x),r(x)) \leq q(x) \leq \max(p(x),r(x))$ holds for $\lambda$-almost every $x$.
\end{definition}

The following proposition will be important in proving the sandwiching property for the I-projection.
\begin{proposition}
	\label{prop:sandwich-kl-tv}
	For probability vectors (or densities), if $r$ is sandwiched between $p$ and $q$, then $\tv(r,p) \leq \tv(q,p)$ and $\KL(r|p) \leq \KL(q|p)$.
\end{proposition}

In fact, will prove the above result for any $\phi$-divergence $D_\phi(p,q) = \int \phi(p/q) q d\lambda$ when $\phi$ is a convex function $\phi(1)=0$. The total variation distance ($\phi(x) = |x - 1|$) and KL-divergence ($\phi(x) = x \log x$) will emerge as special cases.

\begin{lemma}
	\label{lem:sandwich-phi-div}
	Let $\phi: \R \to (-\infty, \infty]$ be a proper convex function with $\phi(1) = 0$. If a density $r$ is sandwiched between densities $p$ and $q$, then $D_\phi(r,q) \leq D_\phi(p,q)$.
\end{lemma}
\begin{proof}
	The sandwiching property implies that there is a function $t: \cX \to [0,1]$ such that $r = (1-t) p + t q$.  Hence	
	\begin{align*}
		D_\phi(r,q) &= \int \phi((1-t) p/q + t q/q ) q d\lambda \leq \int (1-t) \phi(p/q) q d\lambda + \int t \phi(1) q d\lambda \\
		&= D_\phi(p,q) - \int t \phi(p/q) q d\lambda \leq D_\phi(p,q) - \xi\int t q(p/q - 1) d\lambda = D_\phi(p,q). 
	\end{align*}
	where the two inequalities follow from the convexity of $\phi$ noting that there is $\xi \in \mathbb{R}$ (called the sub-gradient) such that 
	$\phi(x) \geq \phi(1) + \xi(x-1) = \xi(x-1)$ for all $x \in \R$, and the last equality holds since $\int t(p-q) d\lambda = 0$, since $p$ and $r$ are assumed to integrate to one.
\end{proof}
% We also have the following, stronger result for total variation distance.

% \begin{lemma}
	%     Let $p, q, r$ denote three probability densities (or vectors). If $\int$
	% \end{lemma}

%\subsection{Information projections are bounded}

Now, we will need a simple lemma about total variation distance before we can prove the sandwiching property of the I-projection.

\begin{lemma}
	\label{lem:tv-transform}
	Let $p,q,r \in \Den$. If $p(x)\leq q(x)$ for all $x$ satisfying $q(x) < r(x)$, then $\tv(q,p) \leq \tv(r,p)$. Similarly, if $p(x) \geq q(x)$ for all $x$ satisfying $q(x) > r(x)$, then $\tv(q,p) \leq \tv(r,p)$.
\end{lemma}
\begin{proof}
	We will prove the first statement. The second follows symmetrically. Let $S^+ = \{x : q(x) < r(x) \}$, $S^- = \{x : q(x) > r(x) \}$, and $S^= = \{x : q(x) = r(x) \}$. Then we have
	\begin{align*}
		\int |p(x) - r(x)| \, dx &= \int_{S^+} |p(x) - r(x)| \, dx + \int_{S^-} |p(x) - r(x)| \, dx + \int_{S^=} |p(x) - r(x)| \, dx \\
		&= \int_{S^+} (r(x) - q(x) + q(x) - p(x)) \, dx + \int_{S^-} |p(x) - q(x) + q(x) - r(x)| \, dx \\
		&+ \int_{S^=} |p(x) - q(x)| \, dx \\
		&\geq \int_{S^+} r(x) - q(x) + q(x) - p(x) \, dx + \int_{S^-} (|p(x) - q(x)| - |q(x) - r(x)|) \, dx \\
		&+ \int_{S^=} |p(x) - q(x)| \, dx \\
		&= \int |p(x) - q(x)| \, dx,
	\end{align*}
	where the inequality follows from the reverse triangle-inequality, and the last line follows from the fact that 
	\[ \int_{S^+} (r(x) - q(x)) \, dx = \tv(q, r) = \int_{S^-} (q(x) - r(x)) dx .\]
\end{proof}


\begin{lemma}
	\label{lem:info-proj-sandwich}
	Let $p_0, p_\theta$ be probability densities, and let $\iproj$ denote the unique minimizer in \cref{eqn:okl}. Then $\iproj$ is sandwiched between $p_0$ and $p_\theta$.
\end{lemma}
\begin{proof}
	For an arbitrary $q \in \Den$, we will show that there is a $\bar{q} \in \Den$ that is sandwiched between $p_0$ and $p_\theta$ such that $\tv(\bar{q},p_0) \leq \tv(q, p_0)$ and $\KL(\bar{q}|p_\theta) \leq \KL(q|p_\theta)$. Then since $\iproj$ is the unique minimizer of \Cref{eqn:okl} upto null sets, we must have that $\iproj$ is sandwiched between $p_0$ and $p_\theta$. 
	
	Let $q \in \Den$ be a probability density, and suppose the set $S^+ = \{x : q(x) > \max(p_0(x), p_\theta(x)) \}$ has non-empty (in fact, that it has non-zero Lebesgue measure). Let 
	\[ v = \int_{S^+} (q(x) - \max(p_0(x), p_\theta(x)) \, dx. \]
	Observe that
	\[  \tv(q, p_\theta)  =\int (p_\theta(x) - q(x)) \I{p_\theta(x) > q(x)} \, dx =  \int (q(x) - p_\theta(x)) \I{q(x) > p_\theta(x)} \, dx
	\geq v. \]
	Define the density
	\[ \bar{q}(x) = 
	\begin{cases} 
		\max(p_0(x), p_\theta(x)) & \text{ if } x \in S^+ \\   
		q(x) + \frac{v}{\tv(q, p_\theta)}(p_\theta(x) - q(x)) & \text{ if } p_\theta(x) > q(x) \\
		q(x) & \text{ otherwise }
	\end{cases}. \]   
	Then it is not hard to verify that $\bar{q}$ integrates to one and satisfies $q(x) \leq \max(p_0(x), p_\theta(x))$ everywhere. Moreover,
	applying \Cref{lem:tv-transform}
	with $p=p_0$, $q=\bar{q}$ and $r=q$, we obtain $\tv(\bar{q}, p_0) \leq \tv(q, p_0)$ since the condition $\bar{q} < q$ only occurs on the set $S_+$.
	Additionally, $\bar{q}$ is sandwiched between $p_\theta$ and $q$. Thus, \Cref{prop:sandwich-kl-tv} implies that $\KL(\bar{q} | p_\theta) \leq \KL(q | p_\theta)$.
	
	
	
	Now let $q \in \Den$ such that $S^+$ is empty but the set $S^- = \{x : q(x) < \min(p_0(x), p_\theta(x)) \}$ is non-empty. Letting 
	\[ v = \int_{S^-} (\min(p_0(x), p_\theta(x)) - q(x)) \, dx, \]
	note that $0 < v \leq \tv(q,p_\theta)$, and define the density
	\[ \bar{q}(x) = 
	\begin{cases} 
		\min(p_0(x), p_\theta(x)) & \text{ if } x \in S^- \\   
		q(x) + \frac{v}{\tv(q, p_\theta)}(p_\theta(x) - q(x)) & \text{ if } p_\theta(x) < q(x) \\
		q(x) & \text{ otherwise }
	\end{cases}. \]  
	Then observe that $\bar{q}$ is a density and it is sandwiched between $q$ and $p_\theta$. Similar arguments as above using \Cref{lem:tv-transform} and \Cref{prop:sandwich-kl-tv} show that $\tv(\bar{q}, p_0) \leq \tv(q, p_0)$ and $\KL(\bar{q} | p_\theta) \leq \KL(q | p_\theta)$.
\end{proof}
\fi

\subsection{Kernel tail bounds}
\label{sec:useful-lemmmas}

In this sub-section, we derive tail bounds for a class of probability kernels $K_h(x,y) = \frac{1}{h^d}\kappa(\|x-y\|/h)$ on $\R^d$ indexed by parameter $h > 0$ used for density estimation, when the function $\kappa: [0, \infty) \to [0,\infty)$ has exponentially decaying  tails (see parts (i)-(iii) of \Cref{assump:kernel-properties}). The following two lemmas bound the tail distribution and the variance of the random variable $Z$ having density $x \mapsto \kappa(\|x\|)$. These results will then be used to obtain tails bounds for the kernel $K_h$.

\begin{lemma} 
	\label{lem:exp-tail-bound}
	Suppose \Cref{assump:kernel-properties} holds. If $Z$ is an $\R^d$ valued random vector with probability density $x \mapsto \kappa(\|x\|)$, then 
	\begin{equation}
		\prob(\|Z\| \geq t) = \int_{\|x\| \geq t} \kappa(\|x\|) dx \leq C_1 t^{d+1-\rho} e^{-t^\rho}
	\end{equation}
	for each $t \geq  t_1 = \max(\Gamma(a+1)^{1/{\rho(a-1)}}, t_0) \geq 1$ where $\Gamma$ is the Gamma function, $a = \frac{d+1}{\rho}$, and $C_1 > 0$ is a constant depending on constants $t_0, C_\rho, \rho > 0$ in  \Cref{assump:kernel-properties} and dimension $d$. Explicitly, $C_1=\rho^{-1}C_\rho v_d 2^{a-1}$, where $v_d = \frac{\pi^{d/2}}{\Gamma(d/2+1)}$ is the volume of the unit ball in $\R^d$.    
	%\chris{We can replace $d S_d$ with $v_d$, the volume of the unit ball in $\R^d$, as this is used almost everywhere else.} \md{The previous proof had erros. Here is a new version using bounds on the incomplete gamma function.} 
\end{lemma}
\begin{proof}
	Using \cite[Lemma~4]{jiang2017uniform} and the upper bound on tails of $\kappa$ since $s \geq t \geq t_0$, 
	\begin{align*}
		\int_{\|x\| \geq t} \kappa(\|x\|) dx = v_d \int_{t}^\infty \kappa(r) r^d dr \leq C_\rho v_d \int_t^\infty e^{-r^\rho} r^{d} dr.
	\end{align*}
	Using the substitution $s = r^\rho$, we obtain
	$$
	\int_t^\infty e^{-r^\rho} r^{d} dr = \frac{1}{\rho}\int_{t^\rho}^\infty e^{-s} s^{\frac{d+1}{\rho} - 1} ds = \frac{1}{\rho}\Gamma\left(a, t^\rho\right)
	$$
	where $\Gamma(a,y) = \int_y^\infty e^{-x} x^{a-1} dx$ is the incomplete Gamma function and $a = \frac{d+1}{\rho} \geq 2$. Taking $b_a = \Gamma(a+1)^{1/(a-1)}$, we have the following bound on the incomplete Gamma function \cite[Theorem~1.1]{pinelis2020exact} for $a \geq 2$:
	$$
	\Gamma(a,y) \leq \frac{((b_a + y)^a - y^a)e^{-y}}{a b_a} = \frac{e^{-y}\int_{0}^{b_a} a (x + y)^{a-1} dx}{a b_a} \leq (b_a + y)^{a-1} e^{-y}.
	$$   
	Since $t^\rho \geq b_a$, the proof can be completed by combining the above displays along with the bound
	$$
	\Gamma(a, t^\rho) \leq 2^{a-1} t^{\rho(a-1)} e^{-t^\rho}  = 2^{a-1} t^{d+1-\rho} e^{-t^\rho}.
	$$
	%Now let us bound the integral in the rightmost term starting with the substitution $s=r^d$ followed the substitution by $h = s-t^d$
	% \begin{align*}
		%     \int_t^\infty e^{-r^\rho} r^{d-1} dr  = \frac{1}{d}\int_{t^d}^\infty e^{-s^{\rho/d}} ds &= \frac{e^{-t^\rho}}{d}  \int_0^\infty e^{-[(t^d + h)^{\rho/d} - t^\rho] } dh\\
		%     &\leq \frac{e^{-t^\rho}}{d} \int_0^\infty e^{- \frac{\rho t^{\rho-1}}{d} h} dh = \frac{d e^{-t^\rho}}{\rho t^{\rho-1}}
		% \end{align*}
	% where the last inequality follows from the bound $(a+h)^\beta - a^\beta \geq \beta a^{\beta-1}h$ for any $a, h \geq 0$ and $\beta \geq 1$. The proof is established by combining the two displays. \md{The last inequality is incorrect. Rewrite the proof in terms of bounds on the incomplete gamma function.}
\end{proof}

\begin{lemma}
	\label{lem:bounded-second-moment}
	Suppose \Cref{assump:kernel-properties} holds. If $Z$ is an $\R^d$ valued random vector with probability density $x \mapsto \kappa(\|x\|)$, then
	\[\E_{Z}[\|Z\|_2^2] \leq v_d \left(c_0 t_0^{d+3} + \frac{C_\rho}{\rho} \Gamma\left( \frac{d+3}{\rho}  \right) \right) ,\] 
	where $v_d$ is the volume of the unit ball in $d$-dimensions and $\Gamma(\cdot)$ is the Gamma function.
\end{lemma}
\begin{proof}
	We can write
	\begin{align*}
		\E_{Z}[\|Z \|_2^2] = \int_{\R^d} \|z \|_2^2 \kappa(\|z \|_2) \, dz
		= v_d \int_0^\infty \kappa(t) t^{d+2} \, du,
	\end{align*}
	where the second equality follows from \cite[Lemma~4]{jiang2017uniform}. Taking $c_0, t_0, \rho, C_\rho$ as the constants from Assumption~\ref{assump:kernel-properties}, we have
	\begin{align*}
		\int_0^\infty \kappa(t) t^{d+2} \, du &= \int_{0}^{t_0} \kappa(t) t^{d+2} \, du + \int_{t_0}^{\infty} \kappa(t) t^{d+2} \, du \\
		&\leq c_0 t_0^{d+3} + C_\rho \int_{t_0}^\infty \exp(- t^\rho) t^{d+2} \, du \\
		&\leq c_0 t_0^{d+3} + \frac{C_\rho}{\rho} \int_0^\infty \exp(-u) u^{\frac{d+3}{\rho} - 1} \, du \\
		&= c_0 t_0^{d+3} + \frac{C_\rho}{\rho} \Gamma\left( \frac{d+3}{\rho}  \right),
	\end{align*}
	where we have used the substitution $u = t^\rho$ in the second inequality as well as the fact that the integrands are non-negative from 0 to $\infty$, and the last line follows from the definition of the Gamma function.
\end{proof}

We now use the above lemmas to prove properties about the kernel $K_h$.

\begin{lemma}
	\label{lem:kernel-support-concentration}
	Suppose that \Cref{assump:kernel-properties} holds. For any $x \in \R^d$, 
	\[ 
	\int_{B(x,r)} K_h(x,y) \, dy \geq 1 -  \frac{C_1 k!}{(r/h)^4} 
	\]
	whenever $r \geq t_1 h$, where $B(x,r)$ denotes the open ball of radius $r$ around $x$, constants $C_1, t_1 > 0$ are as defined in \Cref{lem:exp-tail-bound}, and $k = \lceil \frac{5+d}{\rho}\rceil - 1$. 
	%
	In particular, for any bounded set $S \subset \R^d$ and any $x \in S$, we have
	\[ \int_S K_h(x,y) \, dy \geq 1 - \frac{C_1 k!}{(r_x/h)^4}  \]
	whenever $r_x \geq t_1 h$, where $r_x = \inf_{y \in S^c} \| x -y \|_2$. 
	%
	If $0 < h \leq 
	h_0 = \min(\frac{1}{C_1 k!}, \frac{1}{t_1^2})$, we can always take $r = \sqrt{h}$, and the bound on the right hand side simplifies to $1-\frac{C_1 k!}{(r/h)^4} \geq 1- h$. 
\end{lemma}
\begin{proof}
	Let $Z$ be an $\R^d$-valued random vector with probability density $z \mapsto \kappa(\|z\|_2)$, then 
	\begin{align*}
		1 - \int_{B(x,r)} K_h(x,y) \, dy &= \Pr( x + hZ \notin B(x,r)) \\
		&= \Pr\left(\|Z\|_2 \geq \frac{r}{h}\right) \leq C_1 (r/h)^{(d+1-\rho)}e^{-(r/h)^\rho} \\
		&\leq \frac{C_1 k!}{(r/h)^{k \rho - (d+1-\rho)}} \leq \frac{C_1 k!}{(r/h)^4}.
	\end{align*}
	where the inequality in the second line follows from \Cref{lem:exp-tail-bound} whenever $r \geq t_1 h$, and the inequality on the last line follows by using  $e^y \geq \frac{y^k}{k!}$ and $k \rho - (d+1-\rho) \geq 4$. To get the statement for set $S$, we observe that $1- \int_S K_h(x,y)  \, dy \geq 1 - \int_{B(x,r_x)} K_h(x,y) \, dy$. 
	
	%To finish the proof, we use the fact that $h^{-(\rho -1)/2} \leq h^{-\rho/2}$ and the inequalities $z \geq \log(1+ z)$ and for fixed $a \geq 1$, $z$ satisfies $z \geq a \log z$ whenever $z \geq 2 a \log a$.
\end{proof}


For any probability measure $\mu$ on $\cX$, we define the density $f_{h,\mu}(y) = \int K_h(y, x) \mu(dx) \in \Den$ obtained by convolving $\mu$ with the kernel $K_h$.
% The next result controls the tails of the density $f_{h,\mu}$ assuming that $\mu$ is supported on $S$. 
The next result bounds the TV-distance $\tv(f_{h,\mu}, f_{h,\nu})$ in terms of the supremum norm $\|f_{h,\mu}-f_{h,\nu}\|_{\infty}$ when measures $\mu$ and $\nu$ are supported on $S$. 
%This result will be useful in \Cref{lem:existence-of-good-estimator}, where uniform estimates on the supremum norm $\|f_{h,\mu}-f_{h,\nu}\|_{\infty}$ are obtained when $\mu = \sum_{i=1}^n w_i \delta_{x_i}$ and $\nu$ is the corresponding target distribution.

% \begin{corollary} Suppose \Cref{assump:bounded-support} and \Cref{assump:kernel-properties} hold, and $\mu$ is a probability measure supported on $S$. Then for any $r \geq t_1 h$,
% 	$$
% 	\int_{\|y\| \geq R+r} f_{h,\mu}(y) dy \leq C_1 (r/h)^{d+1-\rho} e^{-(r/h)^\rho}
% 	$$
% 	where constants $C_1, t_1 > 0$ as defined in \Cref{lem:exp-tail-bound}.
% 	\label{cor:kernel-smoothed-density-tails}
% \end{corollary}
% \begin{proof}
% 	Since $\mu$ is supported on $S$ and $K_h$ is symmetric, note that $f_{h,\mu}(y) = \int_S K_h(x, y) \mu(dx)$. Thus by Fubini's theorem
% 	$$
% 	\int_{\|y\| \geq R+r} f_{h,\mu}(y) dy = \int_S \left(\int_{\|y\| \geq R+ r} K_h(x, y) dy \right) \mu(dx).
% 	$$
% 	Next, note that the proof of \Cref{lem:kernel-support-concentration} shows that $\int_{\|y\| \geq R+ r} K_h(x, y) dy \leq 1 - \int_{B(x,r)} K_h(x, y) dy \leq C_1 (r/h)^{d+1-\rho} e^{-(r/h)^\rho}$ for any $x \in S$ since $B(x,r) \subseteq B(0,R + r)$ whenever $x \in S \subseteq B(0,R)$. The proof is then completed by using this inequality to upper bound the term in the display above.
% \end{proof}

\begin{corollary} 
	\label{cor:tv-uniform-estimates}
	Suppose \Cref{assump:bounded-densities,assump:kernel-properties} hold and $\mu$ and $\nu$ are two probability measures supported on $S$. Then whenever $h \leq 1/t_1$,
	$$
	\tv(f_{h,\mu}, f_{h, \nu}) \leq \frac{v_d  (R + 1)^d}{2} \|f_{h,\mu} - f_{h,\nu}\|_\infty  + C_1 h^{d+1-\rho}{e^{-h^{-1/\rho}}}
	$$ where $v_d$ is the volume of the unit ball in $\R^d$, and $t_1$ is as defined in \Cref{lem:exp-tail-bound}.
\end{corollary}
\begin{proof}
We first claim that
\begin{align}
\label{eqn:kernel-smoothed-density-tails}
\int_{\|y\| \geq R+1} f_{h,\mu}(y) dy \leq C_1 (1/h)^{d+1-\rho} e^{-(1/h)^\rho} \text{ for any $h \leq 1/t_1$} .
\end{align}
To see this, we note that $f_{h,\mu}(y) = \int_S K_h(x, y) \mu(dx)$, since $\mu$ is supported on $S$ and $K_h$ is symmetric. Thus by Fubini's theorem
$$
\int_{\|y\| \geq R+r} f_{h,\mu}(y) dy = \int_S \left(\int_{\|y\| \geq R+ r} K_h(x, y) dy \right) \mu(dx).
$$
Next, note that the proof of \Cref{lem:kernel-support-concentration} shows that 
\[ \int_{\|y\| \geq R+ 1} K_h(x, y) dy \leq 1 - \int_{B(x,1)} K_h(x, y) dy \leq C_1 (1/h)^{d+1-\rho} e^{-(1/h)^\rho}\] for any $x \in S$ since $B(x,1) \subseteq B(0,R + 1)$ whenever $x \in S \subseteq B(0,R)$, proving \cref{eqn:kernel-smoothed-density-tails}.


To finish the proof of the lemma, observe that
\begin{align*}
    \tv(f_{h,\mu}, f_{h, \nu}) &=\frac{1}{2} \int_{B(0,R+1)} |f_{h,\mu}(x) - f_{h, \nu}(x)| dx + \frac{1}{2} \int_{\|x\| \geq R+1} |f_{h,\mu}(x) - f_{h, \nu}(x)| dx \\
    &\leq  \frac{v_d}{2}(R+1)^d \|f_{h,\mu} - f_{h, \nu}\|_{\infty} + C_1 h^{d+1-\rho}{e^{-h^{-1/\rho}}}
\end{align*}
where the bound on the second term follows from \cref{eqn:kernel-smoothed-density-tails}.
\end{proof}


Next, we can see how well the convolved density $\smooth{q} = (K_h \star q) \in \Den$ approximates the density $q \in \Den$. For instance, if the density $q$ is supported on a bounded set $S \subseteq \R^d$ with non-zero Lebesgue measure $V_S$, with boundary functional $\phi(r) = \frac{\lambda(S \setminus S_{-r})}{\lambda(S)}$ (see \Cref{assump:bounded-support}), then the following Lemma provides an upper bound on the mass placed by the convolved density  $\smooth{q}$ outside the set $S$.

\begin{lemma}
	\label{lem:smooth-support-concentration}
	Suppose that \Cref{assump:bounded-support} and \Cref{assump:kernel-properties} hold. For any probability density $q \in \Den$ and density bounded above by $1/\gamma$ and supported on the set $S$,
	\[ \smooth{q}(S^c) \doteq \int_{S^c} \smooth{q}(x) dx \leq  h + \frac{V_S}{\gamma} \phi(\sqrt{h}) \]
	whenever $h \leq h_0$, where $h_0$ is as given in \Cref{lem:kernel-support-concentration}.
\end{lemma}
\begin{proof}
	We use the following chain of arguments:
	\begin{align*}
		\smooth{q}(S^c)  &= \int_{S^c} \smooth{q}(x) dx = \int_{S^c} \int_S q(y) K_h(x, y) dy dx  \\
		&= \int_S q(y) \int_{S^c} K_h(x,y) \, dx \, dy \\
		&= \int_{S_{-\sqrt{h}}} q(y) \int_{S^c} K_h(x,y) \, dx \, dy + \int_{S\setminus S_{-\sqrt{h}}} q(y) \int_{S^c} K_h(x,y) \, dx \, dy \\
		&\leq h + \int_{S\setminus S_{-\sqrt{h}}} q(y) \, dy
		\leq h + \frac{V_S}{\gamma} \phi(\sqrt{h}),
	\end{align*}
	where the first inequality uses $\int_{S^c} K_h(x,y) dx = 1 - \int_{S} K_h(y,x) dx$ is bounded above by $h$ if $y \in S_{-\sqrt{h}}$  (\Cref{lem:kernel-support-concentration}) or by one if $y \in S \setminus S_{-\sqrt{h}}$, and the second inequality holds from the definition of boundary functional $\phi$ and the upper bound on the density $q$.
\end{proof}


For a set $S \subseteq \R^d$, let $\KL_S(q| p) = \int_S q(x) \log \frac{p(x)}{q(x)} \, dx$. The following lemma provides a useful trick to translate between $\KL$ and $\tv$ expressions for densities that are not supported on $S$ to densities that are supported on $S$.

\begin{lemma}
	\label{lem:restrict-to-S}
	Suppose $q \in \Den$ is a density that may not be supported on $S$. Then there is a density $\bar{q}$ supported on $S$ such that
	$$
	\KL_S(q|p_\theta) = (1-q(S^c))(\KL(\bar{q}|p_{\theta}) + \log (1-q(S^c)))
	$$
	and $\tv(\bar{q}, p_0) \leq \tv(q, p_0) + q(S^c).$
\end{lemma}
\begin{proof} We will take $\bar{q}(x) = \frac{q(x)}{q(S)} \I{x \in S}$ to be the restriction of the density $q$ to $S$. It is straightforward to see that the  equality for the KL terms hold. The TV bound follows from the triangle inequality, noting that $\tv(\bar{q}, q) = \frac{1}{2} \int_{\R^d} \abs{\bar{q}(x) - q(x)} dx = q(S^c)$.
\end{proof}


\subsection{Proof of Theorem~1}
\label{sec:proof-of-approx-theorem}

\subsubsection{Uniform convergence of KL and TV estimators}
\label{sec:uniform-convergence}

The main result of this subsection is the following.

% \begin{lemma}
	% \label{lem:uniform-convergence}
	% Suppose that \Cref{assump:bounded-densities} and \Cref{assump:kernel-properties} hold and let $u\geq 1\geq \ell$,. Let $\hat{p}:\R^d \rightarrow \R$ be a fixed probability distribution and let $\hat{p}_\gamma(x) = \min(\max( \hat{p}, \gamma), 1/\gamma)$. If $x_1, \ldots, x_n$ are drawn i.i.d. from $p_0$, then with probability at least $1-4\delta$, we have
	% \begin{align*}
		% \left| \left(\| q_w - p_0 \|_1 - q_w(B^c)\right) - \frac{1}{n} \sum_{i=1}^n \left| \frac{q_w(x_i)}{\hat{p}(x_i)} - 1 \right| \right| 
		% &\leq \frac{2}{\gamma^2} \left(2\sqrt{\frac{c}{n h^{d}}} + \sqrt{ \frac{1}{n} \log \frac{2}{\delta}} + \frac{\nu}{\gamma^2}\right) \hspace{1em} \text{ and }  \\
		% \left| \KL_B(q_w | p_\theta) - \frac{1}{n} \sum_{i=1}^n \frac{q_w(x_i)}{\hat{p}(x_i)} \log \frac{q_w(x_i)}{p_\theta(x_i)} \right| 
		%  &\leq  \left(\frac{4}{\gamma^2} \log \frac{1}{\gamma}\right) \left(\sqrt{\frac{c}{n h^d}} + \sqrt{ \frac{1}{n} \log \frac{2}{\delta}} + \frac{\nu}{\gamma^2}\right)
		% \end{align*}
	% for all $w \in \Wcal_{\ell, u} = \{ w \in \Delta_n \, : \,  q_w(x_i) \in [\ell, u] \text{ for all } i \in [n] \}$ simultaneously, where $\nu = 2\tv(\hat{p}, p_0) + 2 \sqrt{\frac{1}{2n} \log \frac{1}{\delta}}$.
	% \end{lemma}
\begin{lemma}
	\label{lem:uniform-convergence}
	There exists an absolute constant $c_1 > 0$ such that the following holds. Suppose observations $x_1, \ldots, x_n$ are drawn i.i.d. from $p_0$, and that \Cref{assump:bounded-densities} and \Cref{assump:kernel-properties} hold. Let $\q{w}$ be as defined in \Cref{defn:qhatw}. Let $\hat{p} = \q{o}$, where $o=(1/n, \ldots, 1/n) \in \Delta_n$, be the kernel-density estimator for $p_0$  based on observations $x_1, \ldots x_n$, and let $\hat{p}_\gamma(\cdot) = \min(\max( \hat{p}(\cdot), \gamma), 1/\gamma)$ be its truncated version. With probability at least $1-\delta$, we have
	\begin{align*}
		\left| \left(\| \q{w} - p_0 \|_1 - \q{w}(S^c)\right) - \frac{1}{n} \sum_{i=1}^n \left| \frac{\q{w}(x_i)}{\hat{p}_\gamma(x_i)} - 1 \right| \right| 
		&\leq \frac{c_1 u}{\gamma^3} \left(\frac{c_0}{\sqrt{n} h^{d}} + \sqrt{\frac{2}{n}\log \frac{6}{\delta}} + \tv(\hat{p}, p_0) \right) \hspace{1em} \text{ and }  \\
		\left| \KL_S(\q{w} | p_\theta) - \frac{1}{n} \sum_{i=1}^n \frac{\q{w}(x_i)}{\hat{p}_\gamma(x_i)} \log \frac{\q{w}(x_i)}{p_\theta(x_i)} \right| 
		&\leq  \left(\frac{c_1 u}{\gamma^3} \log \frac{u}{\ell \gamma}\right) \left(\frac{c_0}{\sqrt{n} h^d} + \sqrt{ \frac{2}{n}\log \frac{6}{\delta}} + \tv(\hat{p}, p_0) \right).
	\end{align*}
	uniformly over all $w \in \Wcal_{\ell, u} = \{ w \in \Delta_n \, : \,  \q{w}(x_i) \in [\ell, u] \text{ for all } i \in [n] \}$, where thresholds $u, l$ are chosen so that $u\geq \max(1, \gamma e) \geq 1 \geq \ell$ and $\|f\|_1 = \int |f(x)| dx$.
\end{lemma}

%\jk{It's called 'Rademacher' (not 'Rademacher') complexity/random variables.}

To establish this result, we will use techniques from uniform law of large numbers over a class of functions $\Fc \subseteq \{ f: S \rightarrow \R \}$. For a fixed class $\Fc$, given i.i.d. samples $x_1, \ldots, x_n$ from density $p_0$ supported on $S \subseteq \cX$, we consider the empirical Rademacher complexity (e.g.~\cite{boucheron2005theory}) of the function class $\Fc$ defined as:
\begin{equation}
	\Rad(\Fc) = \E_{\epsilon} \left[ \sup_{f \in \Fc} \left| \frac{1}{n} \sum_{i=1}^n \epsilon_i f(x_i)\right| \right]
\end{equation}
where the expectation is over Rademacher random variables $\epsilon_1, \ldots \epsilon_n \iid \operatorname{Uniform}(\{-1,1\})$. The Rademacher complexity is a measure of richness of a function class that can be used to obtain a uniform law of large numbers result like the following standard result.

\begin{lemma}[c.f. Theorem~3.2 of \citep{boucheron2005theory}]
	\label{lem:bbl-rademacher-bounds}
	Let $\Fc \subset \{ f: S \rightarrow [-a,a] \}$ be a fixed class of bounded functions and suppose $\delta > 0$. If $x_1, \ldots, x_n \sim p_0$, then with probability at least $1-\delta$
	\[ 
	\sup_{f \in \Fc}\left| \frac{1}{n} \sum_{i=1}^n f(x_i) - \int f(x) p_0(x) dx \right| \leq 2 \Rad(\Fc) + a\sqrt{ \frac{2}{n} \log \frac{2}{\delta}}. 
	\]
\end{lemma}

Thus a typical task to obtain uniform laws of large numbers is to obtain upper bounds on $\Rad(\Fc)$ for various function classes $\Fc$. To this end, the following results will be useful: 

\begin{lemma}[c.f. Theorem~3.3 of \citep{boucheron2005theory}]
\label{lem:bbl-rademacher-lipschitz}
Suppose $g: S \rightarrow \R$ is a bounded function, $\Fc \subset \{ f: S \rightarrow \R \}$, and $\psi: \R \rightarrow \R$ is a function with Lipschitz constant $L$. Then
\begin{align*}
    \Rad( \{ x \mapsto g(x) f(x) : f \in \Fc \}) &\leq \Rad(\Fc) \sup_{x \in S} |g(x)| \\
    \Rad( \{ x \mapsto  \psi(f(x) ): f \in \Fc \}) &\leq L  \Rad(\Fc) + \frac{|\psi(0)|}{\sqrt{n}}.
\end{align*}
\end{lemma}
\begin{proof}
	The above statements with $\psi(0)=0$ follow from Theorem~3.3 of \citep{boucheron2005theory}. To handle the case $\psi(0) \neq 0$, define $\tilde{\psi} = \psi - \psi(0)\boldsymbol{1}$ where $\boldsymbol{1}$ denotes the function taking the constant value one, and note that 
	\begin{equation*}
		\begin{aligned}
			\Rad( \{ \psi \circ f : f \in \Fc \}) &= \Rad( \{ \tilde{\psi} \circ f + \psi(0) \boldsymbol{1} : f \in \Fc \}) \\
			&= \Rad( \{ \tilde{\psi} \circ f : f \in \Fc \}) + \Rad(\{\psi(0)\boldsymbol{1}\}) \\
			&\leq L \Rad(\Fc) + |\psi(0)| \Rad(\boldsymbol{1})
		\end{aligned}
	\end{equation*}
	where we have used that $\Rad(\{\boldsymbol{1}\}) \leq \E_\epsilon \abs{\frac{1}{n}\sum_{i=1}^n \epsilon_i} \leq \sqrt{\E_\epsilon \left(\frac{1}{n}\sum_{i=1}^n \epsilon_i \right)^2} = n^{-1/2}.$
\end{proof}

\begin{lemma}[Lemma~22 of \citep{bartlett2002rademacher}]
	\label{lem:kernel-rademacher-bounds}
	Let $k : \cX \times \cX \rightarrow \R$ be a positive definite kernel satisfying $k(x,x)\leq c$ for all $x \in \cX$. Then, for the function class 
	% \[ \Fc_a = \{ \cdot \mapsto \sum_{i=1}^m w_i k(\cdot, y_i) \, : \, m \geq 1, y_i \in \cX, w_i \in \R, \text{ for } i = 1, \ldots, m, \text{ and } \sum_{i,j} w_i w_j k(y_i, y_j) \leq a^2 \},  \]
 \[ \Fc_a = \bigg\{ \cdot \mapsto \sum_{i=1}^m w_i k(\cdot, y_i) \, : \, m \geq 1, y_i \in \cX, w_i \in \R, \text{ and } \sum_{i,j} w_i w_j k(y_i, y_j) \leq a^2 \bigg\},  \]
	we have $\Rad(\Fc_a) \leq 2a \sqrt{\frac{c}{n}}$. %\md{Definition of $\Fcal$ is confusing. Are $x_1, \ldots, x_n$ or $w_1, \ldots, w_n$ varying?}
\end{lemma}

We will now use the above results to bound the Monte Carlo estimation error uniformly over functions generated from a data-dependent function family $\{\q{w} :  w \in \Delta_n \}$.

\begin{lemma}
	\label{lem:uniform-bounds-over-qclass}
	Suppose bounded functions $\alpha, \beta, \eta : S \to \R$ and a Lipschitz function $\Phi: \R \to [-M, M]$ with Lipschitz constant $L$ are given, and suppose that \Cref{assump:kernel-properties} holds. Given a function $q: S \to \R$, denote $\psi(q, x) = \eta(x) \Phi(\alpha(x)q(x) + \beta(x))$. Then for any $\delta \in (0, 2/e)$, with probability $1-\delta$ it holds that:
	$$
	\sup_{w \in \Delta_n} \abs*{\frac{1}{n}\sum_{i=1}^n \psi(\q{w}, x_i) - \int \psi(\q{w}, x) p_0(x) dx } \leq \frac{C_1}{h^d\sqrt{n}} + C_2 \sqrt{\frac{2}{n}\log \frac{2}{\delta}}
	$$ 
	where $\q{w}$ is as in \Cref{defn:qhatw}, $C_1 = 4\|\eta\|_\infty \|\alpha\|_\infty L c_0$, and  $C_2 = \sqrt{2}\|\eta\|_\infty(L \|\beta\|_\infty + |\Phi(0)|) + \|\eta\|_\infty M$.
\end{lemma}
\begin{proof}
	Using $\hat{\Qcal}$ to denote the data dependent class of functions $\{\q{w} \mid w \in \Delta_n \}$, and taking $\kappa = K_h$, $c=c_0/h^d$, and $a=\sqrt{c}$ in \Cref{lem:kernel-rademacher-bounds}, let us first note that $\hat{\Qcal} \subseteq \Fc_a$. Indeed, this holds since for any $w = (w_1, \ldots, w_n) \in \Delta_n$ we have $\q{w}(\cdot) = \sum_{i=1}^n K_h(\cdot, x_i) w_i$ and 
	$$
	\sum_{i,j=1}^n w_i w_j K_h(x_i,x_j) \leq c \sum_{i=1}^n \sum_{j=1}^n w_i w_j = c, 
	$$
	where the inequality follows from the Cauchy Schwarz bound $K_h(x, y) \leq \sqrt{K_h(x, x) K_h(y, y)}$ using the positive definiteness of the kernel $K_h$, and the bound $K_h(x,x) \leq c_0 /h^d = c$ from \Cref{assump:kernel-properties}. Further, note by \Cref{lem:kernel-rademacher-bounds} that $\Rad(\Fc_a) \leq \frac{2c_0}{h^d\sqrt{n}}$.
	
	Next, let $\Gc$ denote the fixed function class $\{x \mapsto \psi(f, x) | f \in \Fc_a\}$. Repeated application of \Cref{lem:bbl-rademacher-lipschitz} shows:
	\begin{align*}
		\Rad(\Gc) &\leq  \|\eta\|_\infty \|\alpha\|_\infty L \Rad(\Fc_a) + \frac{ \|\eta\|_\infty(L \|\beta\|_\infty + |\Phi(0)|)}{\sqrt{n}}\\
		&\leq \frac{2\|\eta\|_\infty \|\alpha\|_\infty L c_0}{h^d\sqrt{n}} + \frac{ \|\eta\|_\infty(L \|\beta\|_\infty + |\Phi(0)|)}{\sqrt{n}}.
	\end{align*}
	
	Thus we can complete the proof by an application of \Cref{lem:bbl-rademacher-bounds} for the bounded class of functions $\Fc = \Gc$, and noting the inclusion $\hat{\Qcal} \subseteq \Fc_a$.
\end{proof}


With these uniform convergence results in place, now we can prove the main result of this section:

\begin{proof}[Proof of Lemma~\ref{lem:uniform-convergence}]
	By assumption
	\[ \frac{1}{n} \sum_{i=1}^n \left| \frac{1}{p_0(x_i)} - \frac{1}{\hat{p}_\gamma(x_i)} \right| \leq \frac{1}{\gamma^2 n} \sum_{i=1}^n \left|p_0(x_i) - \hat{p}_\gamma(x_i)\right| \leq \frac{1}{\gamma^2 n} \sum_{i=1}^n (\left|p_0(x_i) - \hat{p}(x_i)\right| \wedge \gamma^{-1}), 
	\]
	where the last inequality uses the fact that $\abs{p_0(\cdot) - \hat{p}_\gamma(\cdot)} \leq \min(1/\gamma, \abs{p_0(\cdot) - \hat{p}(\cdot)})$ since $p_0(\cdot), \hat{p}_\gamma(\cdot) \in [1/\gamma, \gamma]$. Recall $\hat{p} = \q{o}$, and hence we may apply \Cref{lem:uniform-bounds-over-qclass} with $\Phi(x)=|x|\wedge \gamma^{-1}$, $\beta = p_0$, $\alpha(\cdot)= \eta(\cdot) = 1$, and $w=o$, to show with probability $1-\delta/3$:
	\begin{align*}
		\frac{1}{n}\sum_{i=1}^n \left|p_0(x_i) - \hat{p}(x_i)\right|\wedge \gamma^{-1}
		&\leq \int (\gamma^{-1} \wedge |p_0(x) - \hat{p}(x)|) p_0(x) dx + \left(\frac{4 c_0 }{ h^d \sqrt{n}} +   \frac{3}{\gamma}\sqrt{\frac{2}{n} \log \frac{6}{\delta}}\right) \\
		&\leq \frac{2}{\gamma}\tv(\hat{p}, p_0)  + \left(\frac{4 c_0 }{ h^d \sqrt{n}} +  \frac{3}{\gamma}\sqrt{\frac{2}{n} \log \frac{6}{\delta}}\right) = \frac{\nu}{\gamma},
	\end{align*}
	where $\nu = 2\tv(\hat{p}, p_0) + \frac{4 \gamma c_0 }{h^d \sqrt{n}} +  3\sqrt{\frac{2}{n}  \log \frac{6}{\delta}}$. 
	
	
Turning to the $\ell_1$-bound, for any $w \in \Wcal_\gamma$, we have
	\begin{align*}
		\left| \frac{1}{n} \sum_{i=1}^n \left| \frac{\q{w}(x_i)}{\hat{p}_\gamma(x_i)} - 1 \right| - \frac{1}{n} \sum_{i=1}^n \left| \frac{\q{w}(x_i)}{{p}_0(x_i)} - 1 \right| \right|
		&\leq \frac{1}{n} \sum_{i=1}^n \left| \left| \frac{\q{w}(x_i)}{\hat{p}_\gamma(x_i)} - 1 \right| - \left| \frac{\q{w}(x_i)}{{p}_0(x_i)} - 1 \right| \right| \\
		&\leq \frac{1}{n} \sum_{i=1}^n \left| \frac{\q{w}(x_i)}{\hat{p}_\gamma(x_i)} -  \frac{\q{w}(x_i)}{{p}_0(x_i)} \right| \leq  \frac{\nu u}{\gamma^3}.
	\end{align*}
	Here the first inequality follows from the triangle inequality, the second follows from the reverse triangle inequality, and the last follows from the upper bound on $\q{w}$ and the bounds at the beginning of the proof. 
	% Thus, we have
	% \begin{align*}
		% \left| \| q_w - p_0 \|_1 - q_w(B^c) - \frac{1}{n} \sum_{i=1}^n \left| \frac{q_w(x_i)}{\hat{p}_\gamma(x_i)} - 1 \right| \right| 
		% \leq \left| \| q_w - p_0 \|_1 - q_w(B^c)  - \frac{1}{n} \sum_{i=1}^n \left| \frac{q_w(x_i)}{{p_0}(x_i)} - 1 \right| \right|  + \frac{2\nu}{\gamma^3}.
		% \end{align*}
	Observe that $\int \abs{\frac{q(x)}{{p_0}(x)} - 1} p_0(x) dx = \| q - p_0 \|_1 - q(S^c)$ for any density $q \in \Den$. 
	%\new{We may take $\kappa = K_h$, $a=\sqrt{u}$, and $c=c_0/h^d$ in \Cref{lem:kernel-rademacher-bounds} to conclude} $ \Rad(\Qcal_{\ell, u}) \leq 2 \sqrt{\frac{c_0 u}{n h^{d}}}$ where $\Qcal_{\ell, u} = \{q_w : w \in \Wcal_{\ell, u} \}$. 
	Next, we apply \Cref{lem:uniform-bounds-over-qclass} with $\Phi(x) = |x| \wedge M$, $M = u/\gamma$, $\alpha = 1/p_0$, $\beta = -1$, $\eta = 1$, noting that:
	$$
	\psi(\q{w}, \cdot) = \abs*{\frac{\q{w}(\cdot)}{p_0(\cdot)} - 1} \wedge M  = \abs*{\frac{\q{w}(\cdot)}{p_0(\cdot)} - 1}
	$$ 
	for each $w \in \Wcal_{\ell, u}$. Thus we have
	\begin{align*}
		\left|\frac{1}{n} \sum_{i=1}^n \left| \frac{\q{w}(x_i)}{{p_0}(x_i)} - 1 \right| - \| \q{w} - p_0 \|_1  - \q{w}(S^c)  \right| \leq  \frac{4c_0}{\gamma \sqrt{n} h^{d}} + \left(3 + \frac{u}{\gamma}\right)\sqrt{ \frac{2}{n} \log \frac{6}{\delta}},
	\end{align*}
	with probability at least $1-\delta/3$ for all $w \in \Wcal_{\ell, u}$ simultaneously. 
	
	
	Turning to the KL-bound, a similar chain of reasoning gives us
	\begin{align*}
		\left| \frac{1}{n} \sum_{i=1}^n \frac{q_w(x_i)}{\hat{p}(x_i)} \log \frac{q_w(x_i)}{p_\theta(x_i)} -  \frac{1}{n} \sum_{i=1}^n \frac{q_w(x_i)}{p_0(x_i)} \log \frac{q_w(x_i)}{p_\theta(x_i)}   \right|
		\leq \frac{\nu u}{\gamma^3} \log \frac{u}{\gamma},
	\end{align*}
	where we have additionally used the lower bound $p_\theta(\cdot) \in [\gamma, 1/\gamma]$. Next, we will invoke \Cref{lem:uniform-bounds-over-qclass} with $\alpha=1/p_\theta, \beta = 0, \eta = p_\theta/p_0$ and $\Phi (x) = k(x) \log k(x)$ where $k(x) = \min(\max(x, u/\gamma), l \gamma)$, noting that the function $\Phi$ is bounded within radius $M=\frac{u}{\gamma} \log \frac{u}{\gamma}$ and is a Lipschitz function with Lipschitz constant bounded by $L = \sup_{x \in [l\gamma, u/\gamma]} |1 + \log x| \leq 2\log \frac{u}{l \gamma}$. Then, using the bounds $C_1 \leq \tilde{C}_1 = \frac{8c_0}{\gamma^3} \log \frac{u}{l\gamma}$ and $C_2 \leq \tilde{C}_2 = \frac{3u}{\gamma^3} \log \frac{u}{\gamma}$, \Cref{lem:uniform-bounds-over-qclass} shows that with probability $1-\delta/3$,  
	$$
	\abs*{\frac{1}{n}\sum_{i=1}^n \frac{\q{w}(x_i)}{p_0(x_i)} \log \frac{\q{w}(x_i)}{p_\theta(x_i)} - \KL_S(\q{w} | p_\theta)} \leq \frac{\tilde{C_1}}{\sqrt{n} h^d} + \tilde{C_2} \sqrt{\frac{2}{n}\log \frac{6}{\delta}}
	$$
	for each $w \in \Wcal_{\ell, u}$ simultaneously, since $\psi(\q{w}, \cdot) = \frac{\q{w}(\cdot)}{p_0(\cdot)} \log \frac{\q{w}(\cdot)}{p_\theta(\cdot)}$ and $\KL_S(\q{w}|p_\theta) =$ \\ $\int \psi(\q{w}, x) p_0(x) dx$ whenever $w \in \Wcal_{\ell, u}$.
	\iffalse
	Observe that $\E_{x \sim p_0}\left[\frac{q_w(x)}{p_0(x)} \log \frac{q_w(x)}{p_\theta(x)} \right] = \KL_B(q_w, p_\theta)$. 
	By \Cref{lem:bbl-rademacher-lipschitz} we have
	\begin{align*}
		\Rad \left( \left\{ x \mapsto \frac{q_w(x)}{p_0(x)} \log \frac{p_0(x)}{p_\theta(x)} : q_w \in \Qcal_{\ell,u}\right\} \right) &\leq \left(\frac{2}{\gamma} \log \frac{1}{\gamma} \right) \Rad(\Qcal_{\ell,u}).
	\end{align*}
	Moreover, observe that for $x \geq \alpha$, the function $x \mapsto ax \log(bx)$ has Lipschitz constant $b\left(1 + \log \frac{a}{\alpha}\right)$. Thus, we additionally have
	\[ \Rad \left( \left\{ x \mapsto \frac{q_w(x)}{p_0(x)} \log \frac{q_w(x)}{p_0(x)} : q_w \in \Qcal_{\ell,u}\right\} \right) \leq \frac{1}{\gamma} \left(1 + \log \frac{1}{ \ell \gamma} \right) \Rad(\Qcal_{\ell,u}). \]
	Finally, we have the decomposition 
	\[\frac{q_w(x)}{p_0(x)} \log \frac{q_w(x)}{p_\theta(x)} = \frac{q_w(x)}{p_0(x)} \log \frac{q_w(x)}{p_0(x)} + \frac{q_w(x)}{p_0(x)} \log \frac{p_0(x)}{p_\theta(x)}.\] 
	Thus, we may apply Lemma~\ref{lem:bbl-rademacher-bounds} and conclude that with probability at least $1-3\delta$, for all $q_w \in \Qcal_{\ell,u}$ simultaneously,
	\begin{align*}
		&\hspace{-3em}\left| \KL_B(q_w | p_\theta) - \frac{1}{n} \sum_{i=1}^n \frac{q_w(x_i)}{{p}_0(x_i)} \log \frac{q_w(x_i)}{p_\theta(x_i)} \right| \\
		&=
		\left| \frac{1}{n} \sum_{i=1}^n \frac{q_w(x_i)}{p_0(x_i)} \log \frac{q_w(x_i)}{p_\theta(x_i)} - \E_{x \sim p_0}\left[ \frac{q_w(x)}{p_0(x)} \log \frac{q_w(x)}{p_\theta(x)} \right] \right| \\
		&\leq \left| \frac{1}{n} \sum_{i=1}^n \frac{q_w(x_i)}{p_0(x_i)} \log \frac{q_w(x_i)}{p_0(x_i)} - \E_{x \sim p_0}\left[ \frac{q_w(x)}{p_0(x)} \log \frac{q_w(x)}{p_\theta(x)} \right] \right| \\
		&\hspace{1em} + \left| \frac{1}{n} \sum_{i=1}^n \frac{q_w(x_i)}{p_0(x_i)} \log \frac{p_0(x_i)}{p_\theta(x_i)} - \E_{x \sim p_0}\left[ \frac{q_w(x)}{p_0(x)} \log \frac{p_0(x)}{p_\theta(x)} \right] \right| \\
		&\leq \frac{1}{\gamma} \left(1 + \log \frac{1}{\ell} + 3\log \frac{1}{\gamma} \right) \Rad(\Qcal_{\ell,u}) + \frac{u}{\gamma}\left(\log \frac{u}{\ell} + 2\log \frac{1}{\gamma} \right)\sqrt{ \frac{2}{n} \log \frac{2}{\delta}} \\
		&\leq \frac{2}{\gamma} \left(1 + \log \frac{1}{\ell} + 3\log \frac{1}{\gamma} \right)\sqrt{\frac{c_0 u}{n h^{d}}} + \frac{u}{\gamma}\left(\log \frac{u}{\ell} + 2\log \frac{1}{\gamma} \right)\sqrt{ \frac{2}{n} \log \frac{2}{\delta}}.
	\end{align*}
	\fi
	Finally, we obtain the lemma statement by using the union bound and putting all the display equations together, noting that $u/\gamma \geq e$ and $\gamma \leq 1$.
\end{proof}


\subsubsection{Sup-norm approximation of kernel density estimates}
\label{sec:kernele-density-estimation-bounds}

%\jk{I may have just overlooked it, but I don't think we formally introduce what we mean by $\Delta_n^{\gamma^2/4}$, do we? It was definitely sufficiently difficult to locate in-text that it might make sense to re-state what it is here.}

Recall our notation for the truncated probability simplex $\Delta_n^\beta = \{(w_1, \ldots, w_n) \in \Delta_n : \frac{\beta}{n} \leq w_i \leq \frac{1}{n \beta}\}$. The next lemma provides approximation results for the subset of densities $\{\q{w} : w \in \Delta_n^{\gamma^2/4}\} \subseteq \Den$ that hold with arbitrary high probability when $n$ is large enough. In particular, Item 1 shows that $\hat{p}$ approximates well the convolved density $K_h \star p_0$, Item 2 provides upper and lower bounds on the density $\q{w}$, Item 3 bounds the mass of $\q{w}$ outside the set $S$, and Item 4 shows that any convolved density of the form $K_h \star q$ can be approximated well by a density of the form $\q{w}$, whenever $q(\cdot) \in [\gamma, 1/\gamma]$ is a density supported on $S$.

\begin{lemma}
	\label{lem:existence-of-good-estimator}
	Suppose that \Cref{assump:bounded-densities} and \Cref{assump:kernel-properties} hold and $\delta \in (0,1)$ is such that $n \geq \frac{2}{\gamma^4} \log \frac{12}{\delta}$. Further suppose that $q \in \Den$ is a density supported on $S$ satisfying $\gamma \leq q(x)  \leq 1/\gamma$ for each $x \in S$. If $x_1, \ldots, x_n \sim p_0$, with probability $1-\delta$, the following items hold:
	\begin{enumerate}
		\item  $\|\hat{p} - K_h \star p_0\|_{\infty} \leq \frac{6c_0}{\sqrt{n} h^d} \sqrt{\log \frac{6}{\delta}}$. %Further, $\tv(\hat{p}, K_h \star p_0) \leq \frac{6c_0 v_d  (R + 1)^d}{\sqrt{n} h^d} \sqrt{\log \frac{2}{\delta}}  + C_1 h^{d+1-\rho}{e^{-h^{-1/\rho}}}$ where $v_d$ is the volume of the unit ball in $\R^d$.
		\item  $\frac{\gamma^2}{4} \left( \gamma - \frac{6c_0}{\sqrt{n} h^d} \sqrt{\log \frac{2}{\delta}} \right)\leq \q{w}(x) \leq \frac{4}{\gamma^2} \left( \gamma^{-1} + \frac{6c_0}{\sqrt{n} h^d} \sqrt{\log \frac{2}{\delta}} \right)$ uniformly over $w \in \Delta_n^{\gamma^2/2}$ and $x \in \R^d$.
		\item  $\q{w}(S^c) \leq  \frac{4}{\gamma^2} \left( h + \frac{V_S}{\gamma}\phi(\sqrt{h}) +  \sqrt{\frac{1}{2n} \log \frac{3}{\delta}} \right)$
		uniformly over $w \in \Delta_n^{\gamma^2/2}$, whenever $h \leq h_0$, where $h_0$ is as defined in \Cref{lem:kernel-support-concentration}.
		
		\item  Denote $\smooth{q} = K_h \star q$. There is a $w \in \Delta_n^{\gamma^2/2}$ such that
		$\| \q{w} - \smooth{q} \|_{\infty} \leq
		\frac{8 c_0}{\gamma^4 h^d} \sqrt{\frac{1}{n} \log \frac{12}{\delta}}$. In particular, 
		$$
		\abs{\KL_S(\q{w}|p_\theta) - \KL_S(\smooth{q}|p_\theta)} \leq \frac{C}{ h^d \sqrt{n}} \sqrt{\log \frac{12}{\delta}
		}$$ 
		where $C = \frac{8 c_0}{\gamma^4} V_S(1 + \log(\frac{u}{\gamma l}))$ and $u$ and $l$ are the upper and lower bounds on $\q{w}$ in Item 2. We take $C = \infty$ if $l  \leq 0.$
	\end{enumerate}
\end{lemma}


\begin{proof}[Proof of \Cref{lem:existence-of-good-estimator}]
	
	Combining \Cref{lem:bbl-rademacher-bounds} with \Cref{lem:kernel-rademacher-bounds} shows that with probability $1-\delta/3$:
	\begin{align*}
		\left|\frac{1}{n} \sum_{i=1}^n K_h(x_i, x)  - \int K_h(y, x) p_0(y) dy\right|
		&\leq  \frac{4c_0}{\sqrt{n} h^d} + \frac{c_0}{\sqrt{n} h^d}\sqrt{2\log \frac{6}{\delta}} \leq \frac{6c_0}{\sqrt{n} h^d} \sqrt{\log \frac{6}{\delta}}  .
	\end{align*}
	for all $x \in \R^d$ simultaneously.
	Noting that $\hat{p}(x) = \frac{1}{n} \sum_i K_h(x_i, x)$ and $(K_h \star p_0)(x) = \int K_h(y,x) p_0(y) dy$, the uniform bound in Item 1 follows. 
	
	To show Item 2, note for each $w \in \Delta_n^{\gamma^2/4}$ that
	\[ 
	\frac{\gamma^2 \hat{p}(x)}{4} \leq \q{w}(x) = \sum_{i=1}^n w_i K_h(x_i, x) \leq \frac{4  \hat{p}(x)}{\gamma^2} \quad \text{for each } x \in \R^d. 
	\]
	Item 2 now follows from the bound in Item 1 and as $(K_h \star p_0)(\cdot) \in [\gamma, 1/\gamma]$, since $\int K_h(y,x) dy = 1$ and $p_0(\cdot) \in [\gamma, 1/\gamma]$.
	%Combined with our bound on $\frac{1}{n} \sum_{i=1}^n K_h(x_i, x)$, we obtain the upper bound in item 2 by using the fact that $\int K_h(y,x) p_0(y) dy \leq \frac{1}{\gamma} \int K_h(y,x) dy = 1/\gamma$.
	
	Now we will show Item 3. For arbitrary $w \in \Delta_n^{\gamma^2/4}$ observe that 
	$$
	\q{w}(S^c) \leq \frac{4  \hat{p}(S^c)}{\gamma^2}. 
	$$
	An application of Hoeffding's inequality to the function $f(\cdot) = \int_{S^c} K_h(\cdot, x) dx \in [0,1]$ implies that with probability $1-\delta/3$,
	\begin{align*}
		\hat{p}(S^c) = \frac{1}{n}\sum_{i=1}^n f(x_i) \, dx 
		&\leq  \int f(y) p_0(y) dy dx  +  \sqrt{\frac{1}{2n} \log \frac{3}{\delta}} \\
		&= \int_{S^c} \int K_h(y,x) p_0(y) dy dx +  \sqrt{\frac{1}{2n} \log \frac{3}{\delta}}\\  
		&\leq h + \frac{V_S}{\gamma}\phi(\sqrt{h}) + \sqrt{\frac{1}{2n} \log \frac{3}{\delta}},
	\end{align*}
	where,  the second inequality follows by invoking \Cref{lem:smooth-support-concentration} with the choice $q=p_0$ whenever $h \leq h_0$.
	
	Finally to show Item 4, we choose the weights $w_i = \frac{q(x_i)}{Zp_0(x_i)}$ for $i = 1, \ldots, n$, where $Z = \sum_{i=1}^n \frac{q(x_i)}{p_0(x_i)}$. By Hoeffding's inequality, we have that with probability at least $1-\delta/6$, 
	\[|Z - n| = |Z - \E[Z]| \leq \frac{1}{\gamma^2} \sqrt{\frac{n}{2} \log \frac{12}{\delta}} . \]
	In the event that this holds, if $n \geq \frac{2}{\gamma^4} \log \frac{12}{\delta}$, then $Z \in [n/2, 3n/2]$ and we have that $w \in \Delta_n^{\gamma^2/4}$. Moreover, it implies that for any $x \in \R^d$,
	\begin{align*}
		\left|\q{w}(x) - \frac{1}{n} \sum_{i=1}^n \frac{q(x_i)}{p_0(x_i)} K_h(x_i, x)\right| &=
		\left| \sum_{i=1}^n w_i K_h(x_i, x) - \frac{1}{n} \sum_{i=1}^n \frac{q(x_i)}{p_0(x_i)} K_h(x_i, x) \right| \\
		&\leq \left| \frac{n - Z}{Z} \right| \max_{i \in [n]} \frac{q(x_i)}{p_0(x_i)} K_h(x_i, x) \\
		&\leq \frac{2}{n}|n - Z| \frac{c_0}{\gamma^2 h^d} \leq \frac{2c_0}{\gamma^4 h^d} \sqrt{\frac{1}{n} \log \frac{12}{\delta}}.
	\end{align*}
	Next, observe that \Cref{lem:bbl-rademacher-lipschitz} and \Cref{lem:kernel-rademacher-bounds} imply that the Rademacher complexity $\Rad$ of the function class $\{ \cdot \mapsto \frac{q(\cdot)}{p_0(\cdot)} K_h(\cdot, x) \, : \, x \in \R^d \}$ on domain $S$ is bounded above by $\frac{2c_0}{\gamma^2 h^d \sqrt{n}}$. Therefore by \Cref{lem:bbl-rademacher-bounds}, with probability $1-\delta/6$,  we have
	\begin{align*}
		%&\sup_{x \in \R^d} \left|  \frac{1}{n} \sum_{i=1}^n \frac{q(x_i)}{p_0(x_i)} K_h(x_i, x) - (K_h \star q)(x) \right|= \\
		\sup_{x \in \R^d} \left|  \frac{1}{n} \sum_{i=1}^n \frac{q(x_i)}{p_0(x_i)} K_h(x_i, x) - \int_S \frac{q(y)}{p_0(y)} K_h(y,x) \, p_0(y) dy \right| \leq \frac{c_0}{\gamma^2 h^d \sqrt{n}} \left(4 +  \sqrt{2\log \frac{12}{\delta}}\right). 
	\end{align*}
	Since $p_0, q$ are both supported on $S$ with $\inf_{x \in S} p_0(x) > 0$, we have $ (K_h \star q)(x) = \int_S q(y) K_h(x,y) \, dy = \int_S \frac{q(y)}{p_0(y)} K_h(x,y) p_0(y) dy$. Hence combining the previous two display equations and using the union bound, the uniform bound in Item 4 between $\q{w}$ and $\smooth{q}$ is seen to hold with probability $1-\delta/3$. Finally to bound the differences between the KL terms, note that
	\begin{align*}
		\abs{\KL_S(\q{w}|p_\theta) - \KL_S(\smooth{q}|p_\theta)} &\leq \int_S |\q{w}(x) \log \q{w}(x) - \smooth{q}(x) \log \smooth{q}(x)| dx \\
		&\quad + \int_S |\log p_\theta(x)| |\q{w}(x) - \smooth{q}(x)| dx \\
		&\leq \|\q{w} - \smooth{q}\|_{\infty} V_S (1 + \log \frac{u}{\gamma l}),
	\end{align*}
	where the last inequality follows by using the Lipschitz continuity of the map $\Phi(x) = x \log x$ on the interval $[l, u]$, with Lipschitz constants $1 + \log(u/l)$, and using the upper bound $|\log p_\theta(\cdot)| \leq \log(1/\gamma)$.
	
	A final application of the union bound shows that Items 1 through 4 can be simultaneously satisfied with probability $1-\delta$.
\end{proof}

\subsubsection{KL and TV approximation by kernel smoothed densities}
\label{sec:okl-smoothed-approx}

 In this section we show that the terms $\KL(q|p_\theta)$ and $\tv(q,p_0)$ do not increase by much when a density $q \in \Den$ is replaced by its kernel-smoothed version $\smooth{q} = K_h \star q$ for a suitably small bandwidth parameter $h > 0$. %Such a result can be obtained by combining  \Cref{lem:info-proj-sandwich} along with   \Cref{lem:OKL-is-continuous} and \Cref{lem:smoothed-densities-kl-tv-approx} shown below. 
%We will assume that the constant $\tilde{h}$ is smaller than the constant $h_0 > 0$ which is defined as in \Cref{lem:kernel-support-concentration}.

%we bound the total-variation distance $\tv(\cdot, p_0)$ and KL-divergence $\KL(\cdot|p_\theta)$ of the smoothed density $\smooth{q}$ in terms of that of $\q$. The following is the main result of this section.

\begin{lemma}
	\label{lem:smoothed-densities-kl-tv-approx}
	Suppose \Cref{assump:bounded-support,assump:bounded-densities,assump:smooth-densities,assump:kernel-properties} hold. There exists a constant $\tilde{c} >0$ and $\tilde{h} > 0$ depending only on the constants in the above assumption, such that the following statement holds. Suppose that $q \in \Den$ is supported on $S$ and bounded above by $1/\gamma$ and $h \leq \tilde{h}$ then
	\begin{itemize}
		\item $\tv(\smooth{q}, p_0) \leq \tv(q, p_0) + \tilde{c} \left( h^{\alpha/2} + h + \phi(\sqrt{h})  \right)$
		\item $\KL_S(\smooth{q} | p_\theta) \leq \KL(q | p_\theta) +  \tilde{c} \left( h^{\alpha/2} + h \log \frac{1}{h} + \phi(\sqrt{h}) \log \frac{1}{\phi(\sqrt{h})} \right)$.
	\end{itemize}
\end{lemma}


We will prove the TV and KL statements separately. We start with the TV statement.
\begin{lemma}
	\label{lem:bounded-tv}
	Suppose that $q \in \Den$ is supported on $S$ and suppose that \Cref{assump:bounded-support,assump:bounded-densities,assump:smooth-densities,assump:kernel-properties} hold. 
	Then 
	\[\tv(\smooth{q}, p_0) \leq \tv(q, p_0) 
	+ V_S \left( C_\alpha h^{\alpha/2} + h + \frac{\phi(\sqrt{h})}{\gamma}  \right)\]
	whenever $h \in (0,h_0)$.
\end{lemma}
\begin{proof}
	We first observe that for any densities $p, q \in \Den$, we have
	\begin{align*}
		\tv(\smooth{p},\smooth{q}) 
		&= \frac{1}{2}\int_{\R^d} |\Ez [q(x-hZ) - p(x-hZ)]| \, dx\\
		&\leq \frac{1}{2} \Ez \int_{\R^d}  |q(x-hZ) - p(x-hZ)| \, dx \\
		&= \frac{1}{2} \int_{\R^d} |q(x)-p(x)| \, dx = \tv(p, q),
	\end{align*}
	where $Z$ is an $\R^d$-valued random vector with probability density $x \mapsto \kappa(\|x\|_2)$, and the inequality follows from Jensen's inequality and Fubini's theorem. By the triangle inequality, we then have
	\begin{align*}
		\tv(\smooth{q}, p_0) \leq \tv(\smooth{q}, K_h \star p_0) + \tv(K_h \star p_0, p_0) \leq \tv(q, p_0) + \tv(K_h \star p_0, p_0).
	\end{align*}
	Next, note that for any $x \in S_{-\sqrt{h}}$ we have
\begin{align*}
    &\hspace{-3em}\left| p_0(x) - K_h \star p_0(x) \right| \\
    &= %\left| p_0(x) - \int_{\R^d} p_0(y) K_h(x,y)\, dy \right| =
    \left| \int_{\R^d} p_0(x) K_h(x, y)\, dy - \int_{\R^d} p_0(y) K_h(x,y)\, dy \right| \\
    &\leq \int_{\R^d} \left| p_0(x) -  p_0(y) \right| K_h(x,y) \, dy \\
    &= \int_{B(x,\sqrt{h})} \left| p_0(x) -  p_0(y) \right| K_h(x,y) \, dy 
    + \int_{B(x,\sqrt{h})^c} \left| p_0(x) -  p_0(y) \right| K_h(x,y) \, dy \\
    &\leq C_\alpha h^{\alpha/2} + h, 
\end{align*}
	where the first inequality follows from Jensen's inequality, and the second follows from \Cref{assump:smooth-densities} and \Cref{lem:kernel-support-concentration}, using the upper bound on the density $p_0(\cdot) \leq 1/\gamma$.
	
	Thus we have
	\begin{align*}
		&\hspace{-2em} \tv(K_h \star p_0, p_0) \\
		&= \frac{1}{2}\int_{\R^d} \left| p_0(x) - K_h \star p_0(x) \right| \, dx \\
		&= \frac{1}{2}\int_{S_{-\sqrt{h}}} \left| p_0(x) - K_h \star p_0(x) \right| \, dx + \frac{1}{2}\int_{\R^d \setminus S_{-\sqrt{h}}} \left| p_0(x) - K_h \star p_0(x) \right| \, dx \\ 
		&\leq \frac{\lambda(S_{-\sqrt{h}})}{2} \left( C_\alpha h^{\alpha/2} +  h \right) + \frac{1}{2}\int_{S \setminus S_{-\sqrt{h}}} \left| p_0(x) - (K_h \star p_0)(x) \right| \, dx
		+ \frac{1}{2} (K_h \star p_0)(S^c) \\
		&\leq %\frac{V_S}{2} \left( C_\alpha h^{\alpha/2} + h \right) + \frac{V_S}{2\gamma} \phi(\sqrt{h}) + \frac{1}{2} (h + \frac{V_S}{\gamma} \phi(\sqrt{h})) \leq
		V_S (C_\alpha h^{\alpha/2} + h + \gamma^{-1}\phi(\sqrt{h}) ).
	\end{align*}
	where the first inequality follows from the last display equation and using the fact that $p_0(x) = 0$ whenever $x \in S^c$, and the second inequality follows by using that  $|p_0(\cdot) - (K_h \star p_0)(\cdot)| \leq 1/\gamma$ and $\lambda(S\setminus S_{-r}) = V_S \phi(r)$, along with the bound $(K_h \star p_0)(S^c) \leq h + \frac{V_S}{\gamma} \phi(\sqrt{h})$ from \Cref{lem:smooth-support-concentration}.
\end{proof}

We now turn to proving the KL bound. To do so, we will use the decomposition of the KL into negative entropy and cross-entropy and bound each term separately.

\begin{lemma}
	\label{lem:bounded-neg-entropy}
	Suppose Assumptions~\ref{assump:bounded-support} and~\ref{assump:kernel-properties} hold, $q \in \Den$ has support $S$ and is bounded above by $1/\gamma$. Then 
	\begin{equation*}
		\int_S \smooth{q}(x) \log \smooth{q}(x) dx \leq \int_S q(x) \log q(x) dx 
		+ D(h)\left[ \frac{d}{2} \log(2\pi e M) + \left(\frac{d}{2} + 1 \right) \log \frac{1}{D(h)} \right],
	\end{equation*}
	where $M = 2 R^2 + 2 v_d \left(c_0 t_0^{d+3} + \frac{C_\rho}{\rho} \Gamma \left( \frac{d+3}{\rho} \right)\right)$ and $D(h) = h + \frac{V_S}{\gamma} \phi(\sqrt{h})$, whenever $h \leq \min(h_0, h_1)$ where $h_1 = \inf\{h > 0 : D(h) > 1/e\}.$
\end{lemma}
\begin{proof}
	First let us note that the integrals $\int_{\R^d} q(x) \log q(x) dx$ and $\int_{\R^d} \smooth{q}(x) \log \smooth{q}(x) dx$ are well-defined. Indeed, the first integral is well defined since $q$ is bounded above by $1/\gamma$ on a set $S$ with finite Lebesgue measure. Although, $\smooth{q}$ can also be shown to be bounded above by $1/\gamma$, its support is unbounded if the kernel $K_h$ also has unbounded support. Instead, for the existence of the second integral, it suffices to show that the second moment of $\smooth{q}$ is finite (e.g.~see \cite{ghourchian2017existence}). To show the latter, we will use the property a random variable $Y$ with density $\smooth{q}$, has the same distribution as $X + hZ$, where $X$ is random variable with density $q$, and let $Z$ be a random variable with density $x \mapsto \K(\|x\|_2)$ independent of $X$. Hence
	\begin{equation}
		\E_{Y \sim \smooth{q}}[\| Y \|_2^2] 
		= \E_{Z, X}[\|X+ h Z \|^2] 
		\leq 2 R^2 + 2 h^2 v_d \left(c_0 t_0^{d+3} + \frac{C_\rho}{\rho} \Gamma \left( \frac{d+3}{\rho} \right) \right) = M 
		\label{eq:l2bound}
	\end{equation}  
	where we have used the inequality $\|x + z \|_2^2 \leq 2\|x\|_2^2 + 2\|z \|_2^2$ along with the bound on $\E \|Z\|^2$ from \Cref{lem:bounded-second-moment} and the bound $\E\|X\|^2 \leq R^2$.
	
	
	Next, let us use the convexity of the function $\Phi:\Rnn \to \R$ given by $\Phi(x)=x \log x$ to conclude that
	\begin{equation*}
		\begin{aligned}
			\int_{\R^d} \smooth{q}(y) \log \smooth{q}(y) dy 
			&= \int_{\R^d} \Phi(\Ez[q(y-hZ)])  dy
			\overset{(i)}{\leq} \int \Ez [\Phi(q(y-hZ))] dy\\
			&\overset{(ii)}{=} \Ez \int_{\R^d} \Phi(q(y-hZ)) dy \overset{(iii)}{=} \int_S \Phi(q(u)) du = \int_S q(u) \log q(u) du
		\end{aligned}
	\end{equation*}
	where we have used Jensen's inequality in Step (i) and Fubini's theorem in Step (ii). In Step (iii) we used the change of variables $u=y-hZ$ as well as the convention that $\Phi(0) = 0$.
	
	Now observe that we can decompose the negative entropy of $\smooth{q}$ as
	\begin{align*}
		\int_{S} \smooth{q}(y) \log \smooth{q}(y) dy  
		&= \int_{\R^d} \smooth{q}(y) \log \smooth{q}(y) dy + \int_{S^c} \smooth{q}(y) \log \frac{1}{\smooth{q}(y)} dy .
	\end{align*}
	Now define the conditional probability density $\bar{q}(x) = \frac{\smooth{q}(x) \I{x \in S^c}}{\smooth{q}(S^c)}$. Then we have the identity
	\begin{align*}
		\int_{S^c} \smooth{q}(y) \log \frac{1}{\smooth{q}(y)} dy = \smooth{q}(S^c) \log \frac{1}{\smooth{q}(S^c)} + \smooth{q}(S^c) \int_{\R^d} \bar{q}(y) \log \frac{1}{\bar{q}(y)} dy .
	\end{align*}
	We now claim that $\bar{q}$ has bounded second moment. To see this, observe that we can write
	\[  \E_{y \sim \smooth{q}}[\|y \|^2] = \smooth{q}(S) \E_{y \sim \smooth{q}}[\|y \|^2 | y \in S] +   \smooth{q}(S^c) \E_{y \sim \smooth{q}}[\|y \|^2 | y \in S^c].\]
	%Letting $Z$ denote the random vector with density $z \mapsto \kappa(\|z\|_2)$ and $Y$ the random vector with density $q$, \Cref{lem:bounded-second-moment} implies
	Thus, $\E_{y \sim \bar{q}}[\|y\|^2] = \E_{y \sim \smooth{q}}[\|y \|^2 | y \in S^c] \leq M/\smooth{q}(S^c)$ by \cref{eq:l2bound}. 
	
	Now in order to bound the negative entropy of $\bar{q}$, let $\mu = \E_{y \sim \bar{q}} [y]$ and $\sigma^2 = \E_{y \sim \bar{y}}[\|y-\mu\|^2] \leq M/\smooth{q}(S^c)$ denote the mean and mean squared error of $\bar{q}$, and let $g(x) = \frac{1}{(2\pi\sigma^2)^{d/2}} e^{-\frac{\|x-\mu\|^2}{2\sigma^2}}$ be the probability density function of the the multivariate normal distribution with the same first two moments. Examining the property $\KL(\bar{q}|g) \geq 0$, we obtain
	$$ 
	\int_{\R^d} \bar{q}(y) \log \frac{1}{\bar{q}(y)} dy \leq \frac{1}{2} + \frac{d}{2} \log(2 \pi \sigma^2) \leq \frac{d}{2} \log (2 \pi e M\smooth{q}(S^c)^{-1}). 
	$$
	Finally, by \Cref{lem:smooth-support-concentration}, $\smooth{q}(S^c) \leq h + \frac{V_S}{\gamma} \phi(\sqrt{h})$. Putting it all together and noting that the function $x \mapsto x \log \frac{1}{x}$ is monotonically increasing on $x \in (0,1/e]$, gives the lemma statement.
\end{proof}

\begin{lemma} Suppose that $q \in \Den$ is supported on $S$ and bounded above by $1/\gamma$. Further suppose that \Cref{assump:bounded-support,assump:bounded-densities,assump:smooth-densities,assump:kernel-properties} hold. Then
	\label{lem:bounded-cross-entropy}
	% $$
	% \left| \int_B \smooth{q}(x) \log p_\theta(x) -  \int_B q(x) \log p_\theta(x) \right| \leq CM_\alpha h^\alpha + |\log \gamma| C_d h^{(\rho-1)/2} e^{-h^{-\rho/2}} + \frac{|\log \gamma|\lambda(B)}{\gamma} (\phi(\sqrt{h}))
	% $$ 
	% where $M_\alpha = \int_{\R^d} \kappa(\|z\|) \|z\|^{\alpha} dx < \infty$ and $C_d$ is the constant expression from \Cref{lem:exp-tail-bound}.
	\[ \left| \int_S \smooth{q}(x) \log p_\theta(x) dx -  \int_S q(x) \log p_\theta(x) dx \right| \leq  C_\alpha h^{\alpha/2} +  2h  \log \frac{1}{\gamma} 
	+\frac{V_S \phi(\sqrt{h})}{\gamma} \log \frac{1}{\gamma}
	\]
	whenever $h \leq h_0$.
\end{lemma}
\begin{proof}
	Using Fubini's theorem, we obtain 
	$$
	\int_S \smooth{q}(x) \log p_\theta(x) dx = \int_{\R^d} q(y) l(y) dy = \int_S q(y) l(y) dy
	$$
	where $l(y) = \int_S K_h(x,y) \log p_\theta(x) dx$ and last equality follows since $q$ is supported on $S$.
	For any $y \in S_{-\sqrt{h}}$ (recall this means that $B(y,\sqrt{h}) \subseteq S$) and hence we have:
	\begin{align*}
		\abs{\log p_\theta(y) - l(y)} &= \abs*{\int_{\R^d} \log p_\theta(y) K_h(x,y) dx - \int_S \log p_\theta(x)  K_h(x,y) dx } \\
		&\leq \int_{B(y,\sqrt{h})}  K_h(x,y) |\log p_\theta(y)- \log p_\theta(x)| \, dx \\
		&+ \int_{B(y,\sqrt{h})^c} K_h(x,y) \left(|\log p_\theta(y)| + |\log p_\theta(x)|\right) \, dx \\
		&\leq   C_\alpha h^{\alpha/2} +  2h  \log \frac{1}{\gamma},
	\end{align*}
	where we have used \Cref{assump:bounded-densities,assump:smooth-densities} and \Cref{lem:kernel-support-concentration} in the last step.
	
	\iffalse
	Next using the random variable notation in \Cref{lem:exp-tail-bound}, note that $A_h(y) = \Pr( y + hZ \in B)$. In particular  \Cref{lem:exp-tail-bound} shows that there is a constant $C_d$
	$$
	1-A_h(y) = \Pr( y + hZ \notin B) \leq \Pr\left(\|Z\| \geq \frac{r_y}{h}\right) \leq C_d (h/r_y)^{(\rho-1)}e^{-(r_y/h)^\rho}
	$$
	whenever $r_y \geq t_0 h$, 
	where $r_y = \inf_{x \in B^c} \|x-y\|$. In particular, whenever $r \geq t_0 h$, the bound
	$$
	1-A_h(y) \leq C_d (h/r)^{(\rho-1)}e^{-(r/h)^\rho}
	$$
	holds for each $y \in B_r \doteq \{y \in B| r_y \geq r\}$. 
	This shows
	\begin{align*}
		\left| \int_B q(y) \psi(y) -  \int_B q(y) \log p_\theta(y) \right| dy \leq \int_B q(y) \left|\psi(y) -  \log p_\theta(y) \right| dy \\
		\leq Ch^\alpha M_\alpha + |\log \gamma|C_d (h/r)^{(\rho-1)} e^{-(r/h)^\rho}  + |\log \gamma| \int_{B \setminus B_r} q(y) dy.
	\end{align*}
	Finally, using the notation in \Cref{assump:bounded-support}, we have $\phi(r) = 1 - \frac{\lambda(B_r)}{\lambda(B)}$, which provides the bound $\int_{B \setminus B_r} q(y) dy \leq \frac{\lambda(B)}{\gamma} \phi(r)$. Now take $r=\sqrt{h}$ to complete the proof.
	\fi
	
	Thus we have
	\begin{align*}
		\left| \int_S q(y) l(y) \, dy -  \int_S q(y) \log p_\theta(y) \, dy \right| 
		&\leq \int_S q(y) \left|  l(y)  - \log p_\theta(y) \right| \, dy \\
		&=  \int_{S_{-\sqrt{h}}} q(y) \left|  l(y)  - \log p_\theta(y) \right| \, dy \\
		&+ \int_{S\setminus S_{-\sqrt{h}}} q(y) \left|  l(y)  - \log p_\theta(y) \right| \, dy \\
		&\leq  C_\alpha h^{\alpha/2} +  2h  \log \frac{1}{\gamma} 
		+\frac{V_S \phi(\sqrt{h})}{\gamma} \log \frac{1}{\gamma} . \qedhere
	\end{align*}
\end{proof}

\begin{proof}[Proof of \Cref{lem:smoothed-densities-kl-tv-approx}]
	The TV statement follows immediately from \Cref{lem:bounded-tv}. To see the KL statement, we first observe that
	\[ \KL_S(\smooth{q}, p_\theta) = \int_S \smooth{q}(x) \log \frac{\smooth{q}(x)}{p_\theta(x)} \, dx = \int_S \smooth{q}(x) \log \smooth{q}(x) \, dx - \int_S \smooth{q}(x) \log p_\theta(x)\, dx.  \]
	Plugging in \Cref{lem:bounded-neg-entropy} for the first term and \Cref{lem:bounded-cross-entropy} for the second term completes the proof.
\end{proof}

%\subsection{Continuity of the optimistic KL}


\subsection{Proof of \Cref{thm:okl-convergence-formal}}
\label{sec:proof-of-theorem}

We now put all of the above together to prove \Cref{thm:okl-convergence-formal}. We suppose \Cref{assump:bounded-densities,assump:bounded-support,assump:kernel-properties,assump:smooth-densities} hold and thus obtain the following corollaries of   \Cref{lem:uniform-convergence,lem:existence-of-good-estimator,lem:smoothed-densities-kl-tv-approx}  when we take 
$\delta_{n} = e^{- (\log n)^{2\beta}}$ 
% $\delta_{n} = n^{-2\beta}$
and $\eta_{n,h} = \frac{(\log n)^\beta}{\sqrt{n} h^d}$ for a fixed constant $\beta > 0$ (e.g.~take $\beta=1$). In the following, $n_0, \bar{h}, \bar{\eta} > 0$ are constant quantities that can depend on any of the terms used in the assumptions and on $\beta > 0$, but are independent of $n$ and $h$. Given two expressions $f$ and $g$, the notation $f = O(g)$ will be used to denote that  $|f| \leq c |g|$ holds for a similar constant $c$. We will delay instantiating a concrete value of $\epsilon'$ in \Cref{cor:existence-of-good-weights} until we start the final proof, but note for now that the constants $c, n_0, \bar{h}, \bar{\eta}$ do not depend on the choice of $\epsilon'$ in any way.

%\jk{Can we say something about what the below convergence results imply about how the kernel bandwidth should be chosen as a function of $n$? Ideally, we could just say that the normal optimal rates in the literature would be fine to plug in here.}


%We will use the adjective ``suitably small'' or ``suitably large'' to refer to constant values  that depend only on the assumption of \Cref{thm:okl-convergence-formal}. 
%We will use $\bar{\delta}, \bar{h}$ to denote suitably small positive constants and  $c, n_0$ to denote suitably large finite constants, which will depend only on the assumption of \Cref{thm:okl-convergence-formal}. Explicit values of these constants can be inferred from the statement of the lemmas used below, but we will avoid stating explicit values for the ease of exposition. 

\begin{corollary}
\label{cor:existence-of-good-weights}
Suppose a fixed value $\epsilon' > 0$ is given. Then whenever $n \geq n_0, h \leq \bar{h}$ and $\eta_{n,h} \leq \bar{\eta}$, the following holds with probability at least $1-\delta_{n}$:
\begin{enumerate}
    \item $\tv(\hat{p}, p_0) =  O\left(\eta_{n,h} + h^{\min(\frac{\alpha}{2},1)} + \phi(\sqrt{h})\right)$.
    \item $\Delta_n^{\gamma^2/4} \subseteq \Wcal_{\frac{\gamma^3}{8}, \frac{8}{\gamma^3}}$, where $\Wcal_{\ell, u}$ is as defined in \Cref{lem:uniform-convergence}.
    \item $\sup_{w \in \Delta_n^{\gamma^2/2}} \q{w}(S^c) = O(h + \phi(\sqrt{h}))$.
    \item There is $w^{\epsilon'} \in \Delta_n^{\gamma^2/4}$ such that
    $$
    \KL_S(\q{w^{\epsilon'}}|p_\theta) \leq \okl[\epsilon'] + O \left( \eta_{n,h}  + h^{\frac{\alpha}{2}} + h \log \frac{1}{h} + \phi(\sqrt{h}) \log \frac{1}{\phi(\sqrt{h})}\right)
    $$
    and
    $$
    \tv(\q{w^{\epsilon'}}, p_0) \leq {\epsilon'} + O \left( \eta_{n,h}  +  h^{\min(\frac{\alpha}{2},1)} + \phi(\sqrt{h}) \right).
    $$
\end{enumerate}	
\end{corollary}

\begin{proof}
	Observe by \Cref{lem:info-proj-sandwich} that the information projection $\iproj[{\epsilon'}]$ is sandwiched between $p_0$ and $p_\theta$; in particular, $\iproj[{\epsilon'}]$ has support $S$ and is bounded between values $\gamma$ and $1/\gamma$ on $S$. Thus we will invoke \Cref{lem:existence-of-good-estimator} with $q = \iproj[\epsilon']$ and $\delta = \delta_n$, noting that the condition $n \geq \frac{2}{\gamma^2} \log \frac{12}{\delta}$ is satisfied when $n_0$ is suitably large. %Henceforth, by the union bound, suppose that conclusions of \Cref{lem:existence-of-good-estimator} are satisfied for $q = \iproj[{\epsilon'}]$ and $q = \iproj[\epsilon_2]$ with probability $1-\delta_n$. 
	Hence we may now prove the respective items.
	
	
	We show Item 1 as follows:
	$$
	\tv(\hat{p}, p_0) \leq \tv(\hat{p}, K_h \star p_0) + \tv(K_h \star p_0, p_0) = O(\eta_{n,h} + h) + O(h^{\min(\frac{\alpha}{2},1)} + \phi(\sqrt{h})),
	$$
	where the term $\tv(\hat{p}, K_h \star p_0)$ was bounded by combining Item 1 of \Cref{lem:existence-of-good-estimator} with \Cref{cor:tv-uniform-estimates}, and the term $\tv(K_h \star p_0, p_0)$ was bounded by using  \Cref{lem:smoothed-densities-kl-tv-approx} with $q = p_0$ (assuming suitable choice of constants $\bar{h}$ and $c$).
	
	Next, Item 2 follows from Item 2 of \Cref{lem:existence-of-good-estimator} as long as $\frac{6c_0}{\sqrt{n}h^d} \sqrt{\log \frac{4}{\delta_n}} = O(\eta_{n,h}) \leq \gamma/2$ is satisfied, which holds when $\bar{\eta}$ is a suitably small constant.
	
	Next, Item 3 follows from Item 3 of \Cref{lem:existence-of-good-estimator} noting that $\sqrt{\frac{1}{2n} \log \frac{6}{\delta_n}} = O( \eta_{n,h} h^d ) = O( \bar{\eta} h)$.
	
	Finally to show Item 4, we will invoke Item 4 from \Cref{lem:existence-of-good-estimator} for the choice $q=\iproj[{\epsilon'}]$. Hence we obtain a $w^{\epsilon'} \in \Delta_{n}^{\gamma^2/4}$ such that $\|\q{w^{\epsilon'}}-\iproj[{\epsilon'}]\|_\infty = O(\eta_{n,h})$ and 
	$$
	|\KL_S(\q{w^{\epsilon'}}|p_\theta) - \KL_S(K_h \star \iproj[{\epsilon'}]|p_\theta)|  =  O(\eta_{n,h}).
	$$
	Further, \Cref{cor:tv-uniform-estimates} shows
	$$
	\tv(\q{w^{\epsilon'}}, K_h \star \iproj[{\epsilon'}]) \leq O(\eta_{n,h} + h)
	$$
	since $\mu = \sum_{i=1}^n w^{\epsilon'}_i \delta_{x_i}$ and the measure $\nu$ given by density $\iproj[{\epsilon'}]$, are both supported on $S$. The bounds in Item 4 now follow using \Cref{lem:smoothed-densities-kl-tv-approx} with $q=\iproj[{\epsilon'}]$ and the triangle inequality, noting that $\KL(\iproj[{\epsilon'}]|p_\theta) = \okl[{\epsilon'}]$ and $\tv(\iproj[{\epsilon'}], p_0) \leq \epsilon'$.
\end{proof}


Since our goal is to approximate $\okl$ by $\hatI(\theta)$, let us rewrite \cref{eqn:I-hat-defn} as 
$$
\hatI(\theta) = \inf_{\substack{w \in \Delta_n^{\gamma^2/4} \\ \hatTV(\q{w}, p_0) \leq \epsilon}} \hatKL(\q{w}|p_\theta)
$$
where, given $q \in \Den$,
$$
\hatKL(q|p_\theta) = \frac{1}{n} \sum_{i=1}^n \frac{q(x_i)}{\hat{p}_\gamma(x_i)} \log \frac{q(x_i)}{p_\theta(x_i)}
$$
and
$$
\hatTV(q,p_0) = \frac{1}{2n} \sum_{i=1}^n \abs*{\frac{q(x_i)}{\hat{p}_\gamma(x_i)} - 1}.
$$

Next, the following corollary of \Cref{lem:uniform-convergence}  shows that, with high probability, the estimators $\hatKL(\q{w}|p_\theta)$ and $\hatTV(\q{w}, p_0)$ are close to their population level targets $\KL_S(\q{w}|p_\theta)$ and $\tv(\q{w}, p_0)$, uniformly over all $w \in \Delta_n^{\gamma^2/4}$.   

\begin{corollary}
	\label{cor:all-the-bounds}
	Suppose $n \geq n_0$, $h \leq \bar{h}$ and $\eta_{n,h} \leq \bar{\eta}$, then with probability at least $1-2\delta_n$ the event in \Cref{cor:existence-of-good-weights} holds, and further uniformly over all $w \in \Delta_{n}^{\gamma^2/4}$ it holds that
	$$
	\abs*{\hatKL(\q{w}|p_\theta) - \KL_S(\q{w}|p_\theta)} \leq \psi_{n,h} \quad \text{and } \abs*{\hatTV(\q{w}, p_0) - \tv(\q{w}, p_0)} \leq \psi_{n,h}
	$$
	for a deterministic quantity $\psi_{n,h}$ that satisfies $\psi_{n,h} = O \left(\eta_{n,h} + h^{\min(\frac{\alpha}{2},1)} + \phi(\sqrt{h})\right)$.
\end{corollary}
\begin{proof} Let $E_1$ denote the event in \Cref{cor:existence-of-good-weights}, and let $E_2$ denote the event in \Cref{lem:uniform-convergence} with the choices  $\delta = \delta_n$, $l = \frac{\gamma^3}{8}$, $u = \frac{8}{\gamma^2}$. We will assume that both events $E_1$ and $E_2$ hold simultaneously, which happens with probability at least $1-2\delta_n$.
	
	Recall that on the event $E_1$, we have $\Delta_{n}^{\gamma^2/4} \subseteq \Wcal_{\ell, u}$, $\sup_{w \in \Delta_n^{\gamma^2/4}} \q{w}(S^c) = O(h + \phi(\sqrt{h}))$, are  $\tv(\hat{p}, p_0) = O\left(\eta_{n,h} + h^{\min(\frac{\alpha}{2}, 1)} + \phi(\sqrt{h})\right)$. The proof can be completed by combining the above with the bounds from \Cref{lem:uniform-convergence} under the event $E_2$. 
\end{proof}


For a suitable constant $c$, the term $\xi_{n,h} = c\left(\eta_{n,h} + h^{\frac{\alpha}{2}} + h \log \frac{1}{h} + \phi(\sqrt{h}) \log \frac{1}{\phi(\sqrt{h})}\right)$ dominates all the error terms (i.e.~the terms of the form $O(\cdot)$) in the statement of \Cref{cor:existence-of-good-weights} and \Cref{cor:all-the-bounds}. Using this term, we are ready to finish the proof of \Cref{thm:okl-convergence-formal}.

\begin{proof}[Proof of \Cref{thm:okl-convergence-formal}]
We will invoke \Cref{cor:all-the-bounds} with the choice $\epsilon' = \epsilon - 2 \xi_{n,h}$ (see \Cref{cor:existence-of-good-weights}). We first claim that the following holds:
\begin{align}
\label{eqn:okl-upper-and-lower-bound}
\okl[\epsilon + 2\xi_{n,h}] - O(\xi_{n,h}) \leq \hatI(\theta) \leq \okl[\epsilon - 2\xi_{n,h}] + O(\xi_{n,h})
\end{align}
with probability at least $1-2\delta_n$.
	
For the upper bound, let $w^{\epsilon'} \in \Delta_n^{\gamma^2/4}$ be the vector from \Cref{cor:existence-of-good-weights,cor:all-the-bounds} and note that
$$
\hatTV(\q{w}, p_0) \leq \tv(\q{w}, p_0) + \xi_{n,h} \leq \epsilon ' + 2\xi_{n,h} \leq \epsilon,
$$
and hence 
$$
\hatI(\theta) \leq \hatKL(\q{w}|p_\theta) \leq \KL_S(\q{w}|p_\theta) + \xi_{n,h} \leq \okl[\epsilon'] + 2\xi_{n,h}.
$$

To lower bound $\hatI(\theta)$, let $w \in \Delta_{n}^{\gamma^2/4}$ be such that $\hatTV(\q{w}, p_0) \leq \epsilon$. Then by  \Cref{cor:existence-of-good-weights}, \Cref{cor:all-the-bounds}, and \Cref{lem:restrict-to-S}, there is a $\bar{q}_w$ that is supported on $S$ such that
$$
\tv(\bar{q}_w, p_0) \leq \tv(\q{w}, p_0) + \q{w}(S^c) \leq \hatTV(\q{w}, p_0) + 2\xi_{n,h} \leq \epsilon + 2\xi_{n,h}
$$
and 
$$
\begin{aligned}
    \hatKL(\q{w}|p_\theta) &\geq \KL_S(\q{w}|p_\theta) - \xi_{n,h}\\
     &= (1-\q{w}(S^c))  \KL(\bar{q}_w|p_\theta) + (1-\q{w}(S^c))\log (1-\q{w}(S^c)) - \xi_{n,h} \\
    &\geq (1-\xi_{n,h}) \okl[\epsilon + 2\xi_{n,h}] - 2\xi_{n,h}
\end{aligned}
$$
where we have used that $\q{w}(S^c) \leq \xi_{n,h}$, $h(x) = (1-x) \log (1-x) \geq -x$ by the convexity of $h$, and $\KL(\bar{q}_w|p_\theta) \geq \okl[\epsilon + 2\xi_{n,h}]$ since $\tv(\bar{q}_w, p_0) \leq \epsilon + 2\xi_{n,h}$. Since the right hand side of the previous display doesn't depend on the choice of $w$, by considering the infimum over all $w \in \Delta_n^{\gamma^2/4}$ such that $\hatTV(\q{w}, p_0) \leq \epsilon$ we see
$$
\begin{aligned}
    \hatI(\theta) &\geq \okl[\epsilon + 2\xi_{n,h}] - \xi_{n,h} (1+\okl[\epsilon + 2\xi_{n,h}]) \geq \okl[\epsilon + 2\xi_{n,h}] - \xi_{n,h} (1+ \KL(p_0|p_\theta)).
\end{aligned}
$$
Note that $\KL(p_0|p_\theta) \leq   2\log(1/\gamma)$ by H\"older's inequality, and we have thus shown the lower bound and established \cref{eqn:okl-upper-and-lower-bound}.

Finally, we appeal to the continuity of the OKL function with respect to the coarsening radius (i.e., \Cref{lem:okl-continuity}) to see that
\[ I_{\epsilon - 2\xi_{n,h}}(\theta) \leq I_{\epsilon}(\theta) + \frac{2\xi_{n,h}}{\epsilon} \KL(p_0 | p_\theta) \leq I_{\epsilon}(\theta) + \frac{4 \xi_{n,h}}{\epsilon} \log(1/\gamma) = I_{\epsilon}(\theta) + O(\xi_{n,h}) \]
and
\[ I_{\epsilon + 2\xi_{n,h}}(\theta) \geq I_{\epsilon}(\theta) - \frac{2\xi_{n,h}}{\epsilon + 2\xi_{n,h}} \KL(p_0 | p_\theta) \leq I_{\epsilon}(\theta) - \frac{4 \xi_{n,h}}{\epsilon} \log(1/\gamma) = I_{\epsilon}(\theta) - O(\xi_{n,h}), \]
where we have used \Cref{assump:bounded-densities} and H\"{o}lder's inequality to establish $\KL(p_0|p_\theta) \leq 2 \log(1/\gamma)$. Combining the above two inequalities with \Cref{eqn:okl-upper-and-lower-bound} completes the proof.
\end{proof}

 \iffalse
We obtain the following corollary relating $\hatI(\theta)$ to $\okl[\tilde{\epsilon}]$ for  values of $\tilde{\epsilon}$ close to $\epsilon$. 

\begin{lemma} Suppose $n \geq n_0$, $h \leq \bar{h}$ and $\eta_{n,h} \leq \bar{\eta}$, then with probability $1-2\delta_n$
	$$
	\okl[\epsilon + 2\xi_{n,h}] - O(\xi_{n,h}) \leq \hatI(\theta) \leq \okl[\epsilon - 2\xi_{n,h}] + O(\xi_{n,h})
	$$
	\label{lem:okl-bounds}
\end{lemma}
\begin{proof} 
We will invoke \Cref{cor:all-the-bounds} with the choice $\epsilon' = \epsilon - 2 \xi_{n,h}$ (see \Cref{cor:existence-of-good-weights}). 
	
	Let us first show the upper bound. For the vector $w^{\epsilon'} \in \Delta_n^{\gamma^2/4}$ from \Cref{cor:existence-of-good-weights} and \Cref{cor:all-the-bounds}, note that
	$$
	\hatTV(\q{w}, p_0) \leq \tv(\q{w}, p_0) + \xi_{n,h} \leq \epsilon ' + 2\xi_{n,h} \leq \epsilon
	$$
	and hence 
	$$
	\hatI(\theta) \leq \hatKL(\q{w}|p_\theta) \leq \KL_S(\q{w}|p_\theta) + \xi_{n,h} \leq \okl[\epsilon'] + 2\xi_{n,h}.
	$$
	
	To lower bound $\hatI(\theta)$, let $w \in \Delta_{n}^{\gamma^2/4}$ be such that $\hatTV(\q{w}, p_0) \leq \epsilon$. Then by  \Cref{cor:existence-of-good-weights}, \Cref{cor:all-the-bounds}, and \Cref{lem:restrict-to-S}, there is a $\bar{q}_w$ that is supported on $S$ such that
	$$
	\tv(\bar{q}_w, p_0) \leq \tv(\q{w}, p_0) + \q{w}(S^c) \leq \hatTV(\q{w}, p_0) + 2\xi_{n,h} \leq \epsilon + 2\xi_{n,h}
	$$
	and 
	$$
	\begin{aligned}
		\hatKL(\q{w}|p_\theta) &\geq \KL_S(\q{w}|p_\theta) - \xi_{n,h}\\
		 &= (1-\q{w}(S^c))  \KL(\bar{q}_w|p_\theta) + (1-\q{w}(S^c))\log (1-\q{w}(S^c)) - \xi_{n,h} \\
		&\geq (1-\xi_{n,h}) \okl[\epsilon + 2\xi_{n,h}] - 2\xi_{n,h}
	\end{aligned}
	$$
	where we have used that $\q{w}(S^c) \leq \xi_{n,h}$, $h(x) = (1-x) \log (1-x) \geq -x$ by the convexity of $h$, and $\KL(\bar{q}_w|p_\theta) \geq \okl[\epsilon + 2\xi_{n,h}]$ since $\tv(\bar{q}_w, p_0) \leq \epsilon + 2\xi_{n,h}$. Since the right hand side of the previous display doesn't depend on the choice of $w$, by considering the infimum over all $w \in \Delta_n^{\gamma^2/4}$ such that $\hatTV(\q{w}, p_0) \leq \epsilon$ we see
	$$
	\begin{aligned}
		\hatI(\theta) &\geq \okl[\epsilon + 2\xi_{n,h}] - \xi_{n,h} (1+\okl[\epsilon + 2\xi_{n,h}]) \geq \okl[\epsilon + 2\xi_{n,h}] - \xi_{n,h} (1+ \KL(p_0|p_\theta)).
	\end{aligned}
	$$
	Note that $\KL(p_0|p_\theta) \leq  \frac{2V_S}{\gamma} \log(1/\gamma)$, and we have thus shown the lower bound.
\end{proof}


Combining \Cref{lem:okl-bounds} with the following lemma showing continuity of the $t \mapsto \okl[t]$ at $t=\epsilon > 0$, we finally see that $\abs*{\okl - \hatI(\theta)} = O(\xi_{n,h})$ holds under the conditions of \Cref{lem:okl-bounds}.

\begin{lemma}
	\label{lem:OKL-is-continuous}
	Under \Cref{assump:bounded-support} and \Cref{assump:bounded-densities}, for any $\epsilon' \geq 0$ and $t > 0$
	$$
	0 \leq I_{\epsilon'}(\theta) - I_{\epsilon'+t}(\theta) \leq \frac{2tV_S}{\gamma(\epsilon' + t)}\log(1/\gamma).
	$$
\end{lemma}
\begin{proof} Fix $\epsilon' \geq 0$ and $t > 0$. Suppose $\iproj[\epsilon'+t]$ is the I-projection of $p_\theta$ onto the ball $B_{\epsilon' + t}(p_0) \doteq \{q \in \Den \, | \, \tv(q, p_0) \leq \epsilon' +t\}$. 
	%
	Since $\iproj[\epsilon+t] \in B_{\epsilon'+t}(p_0)$, we have $\lambda_t \iproj[\epsilon+t] + (1-\lambda_t) p_0 \in B_\epsilon(p_0)$ for $\lambda_t = \epsilon'/(\epsilon'+t)$.
	Hence using convexity of the KL divergence:
	\begin{align*}
		I_{\epsilon}(\theta) \leq \KL(\lambda_t \iproj[\epsilon'+t] + (1-\lambda_t) p_0|p_\theta) &\leq \lambda_t\KL(\iproj[\epsilon'+t]|p_\theta) + (1-\lambda_t) \KL(p_0|p_\theta)  \\
		&= \lambda_t I_{\epsilon'+t}(\theta) + \frac{t}{\epsilon' + t}  \KL(p_0 | p_\theta).
	\end{align*}
	The proof is completed by rearranging the terms and noting that $I_{\epsilon'}(\theta) \geq I_{\epsilon'+t}(\theta) \geq 0$, $\lambda_t \leq 1$, and $\KL(p_0|p_\theta) \leq  \frac{2V_S}{\gamma} \log(1/\gamma)$.
\end{proof}

\fi

\section{Asymptotics of the coarsened likelihood}
\label{sec:coarsened-likelihood-asymptotics}

In this section, we will show the asymptotic convergence of the coarsened likelihood to the OKL function. 


\subsection{Asymptotics for finite spaces}
\label{sec:coarsened-likelihood-asymptotic-finite}
%auto-ignore
Suppose $\cX$ is a finite space. We will use the notation from \Cref{sec:asymp-finite}; in particular, recall the probability simplex $\Delta_{\cX} = \{q \in [0,1]^{\cX} | \sum_{x \in X} q(x) = 1\}$ and the OKL function $I_\epsilon(\theta)$ in terms of a general distance $\D$ on $\Delta_{\cX}$ as in \cref{eqn:okl-general-distance}. We begin by showing an elementary rounding lemma, which will be useful when applying Sanov's theorem in \Cref{lem:finite-sanov}.

%\jk{Given that $q_n$ below need not denote an estimator for the data-generating process, it may be confusing to write it as $q_n$ (Maybe also not; I guess it's clear enough from the lemma description.)}

\begin{lemma}
\label{lem:rounded-probability-vector}
Let $p, p_\theta \in \Delta_\Xcal$ such that $\KL(p|p_\theta) < \infty$. For any integer $n > |\Xcal|$, there exists a $q_n \in \Delta_\Xcal$ such that $n q_n(x)$ is integral for all $x \in \Xcal$, $\|q_n - p \|_1 \leq \frac{2|\Xcal|}{n}$, and 
\[ \KL(q_n | p_\theta) \leq \left(1+ \frac{|\Xcal|^2}{n}\right)\KL(p|p_\theta) + \frac{|\Xcal|}{n} \left( 2 \log \frac{n}{2} + \log |\Xcal| + \frac{|\Xcal|}{e} \right).\]
\end{lemma}
\begin{proof}
Choose an arbitrary $x^\star = \argmax_{x \in \cX} p(x)$, choosing arbitrarily if there are ties. Then we define $q_n \in \Delta_\Xcal$ as follows. For all $x \neq x^\star$, let $q_n(x) = \frac{1}{n} \lfloor n q_n(x) \rfloor$, and let $q_n(x^\star) = 1 - \sum_{x\neq x^\star}q_n(x)$. As $p \in \Delta_\Xcal$, we have $q_n \in \Delta_\Xcal$. Moreover, by construction we also have that $n q_n(x)$ is integral for all $x \in \Xcal$. 

Observe that for all $x \neq x^\star$, we have $0 \leq p(x) - q_n(x) \leq 1/n$. This implies that \[ 0 \leq q_n(x^\star) - p(x^\star) = \left(1 - \sum_{x \neq x^\star}q_n(x)\right) - \left(1-\sum_{x \neq x^\star}p(x)\right) \leq \frac{|\Xcal|}{n}.\]
Together, these statements give us that $\| p - q_n \|_1 \leq \frac{2|\Xcal|}{n}$. Further since $\KL(p|p_\theta) < \infty$, we must also have $\KL(q_n|p_\theta) < \infty$ since $\supp(q_n) \subseteq \supp(p) \subseteq \supp(p_\theta)$.  

To prove the more detailed KL bound, we first observe that by known bounds on the entropy function \citep[c.f. Theorem 17.3.3]{cover2006elements}, we have
\begin{align*}
    \sum_{x} q_n(x) \log q_n(x) 
    &\leq \sum_{x} p(x) \log p(x) + \| p - q_n \|_1 \log \frac{|\Xcal|}{\| p - q_n \|_1} \\
    &\leq \sum_{x} p(x) \log p(x) + \frac{2|\Xcal|}{n} \log \frac{n}{2}.
\end{align*} 
Next observe that we can bound the cross-entropy between $q_n$ and $p_\theta$ as follows.
\begin{align*}
\sum_x q_n(x) \log \frac{1}{p_\theta(x)} 
&= \sum_{x \neq x^\star} q_n(x) \log \frac{1}{p_\theta(x)} + q_n(x^\star) \log \frac{1}{p_\theta(x^\star)} \\
&\leq \sum_{x \neq x^\star} p(x) \log \frac{1}{p_\theta(x)} + (p(x^\star) + q_n(x^\star) - p(x^\star)) \log \frac{1}{p_\theta(x^\star)}\\
&\leq \sum_x p(x) \log \frac{1}{p_\theta(x)} + \frac{|\Xcal|}{n} \log \frac{1}{p_\theta(x^\star)}.
\end{align*}
Now observe that $p(x^\star)\geq 1/|\Xcal|$. This implies that $\log \frac{1}{p_\theta(x^\star)} \leq |\Xcal| (\KL(p | p_\theta) + 1/e) + \log |\Xcal|$. To see this, first note that if $p_\theta(x^\star) \geq 1/|\Xcal|$, then the claim is trivial. Thus, we may assume $p_\theta(x^\star) < 1/|\Xcal| \leq p(x^\star)$. The log-sum inequality implies that
\begin{align*}
\KL(p | p_\theta) 
&\geq p(x^\star) \log \frac{p(x^\star)}{p_\theta(x^\star)} + (1-p(x^\star)) \log \frac{1 - p(x^\star)}{1 - p_\theta(x^\star)} \\
&\geq p(x^\star) \log \frac{p(x^\star)}{p_\theta(x^\star)} - (1-p(x^\star)) \log \frac{1}{1 - p(x^\star)} \\
&\geq p(x^\star) \log \frac{p(x^\star)}{p_\theta(x^\star)} - \frac{1}{e} \geq \frac{1}{|\Xcal|} \log \frac{1}{|\Xcal| p_\theta(x^\star)} - \frac{1}{e}.
\end{align*}
Here, the last inequality follows from our bound on $p(x^\star)$ combined with the fact that $a\log \frac{a}{b}$ is an increasing function in $a$ for $a \geq b$. Rearranging the above gives us the claim.
\end{proof}

Our analysis of the coarsened likelihood will rely heavily on the following result, which is essentially a consequence of Sanov's theorem.

\begin{lemma}
\label{lem:finite-sanov} 
Suppose \Cref{assum:finite-continuous-distance,assum:finite-okl} hold, and let $r > \epsilon_0$ and $n \geq \frac{4C |\Xcal|}{r - \epsilon_0}$. Then 
\[  \left| \frac{1}{n} \log \prob_\theta\left(\D(\EmpDist{Z_{1:n}}, p_0) \leq r\right) + I_r(\theta) \right| \leq 
\frac{|\Xcal|}{n} \left( 3 \log(n+1) + \log|\Xcal| + |\Xcal|\left(V + \frac{1}{e} \right) + \frac{2C}{r - \epsilon_0} \right) \]
 where the probability operation $\prob_\theta$ is taken over random points $Z_1, \ldots, Z_n \in \cX$ drawn from the distribution $p_\theta \in \Delta_{\cX}$, and $\EmpDist{Z_{1:n}} \in \Delta_{\cX}$ is the empirical distribution of the data points $Z_1, \ldots, Z_n$.
\end{lemma}
\begin{proof}
Observe that $I_r(\theta) \leq V < \infty$ for all $r > \epsilon_0$. Sanov's theorem~\cite[Theorem~11.4.1]{cover2006elements} implies that
\[ \frac{1}{n} \log \prob_\theta\left(\D(\EmpDist{Z_{1:n}}, p_0) \leq r\right) \leq - I_r(\theta) + \frac{|\Xcal|}{n}\log(n+1). \]
Thus, we only need to show the lower bound for $\frac{1}{n} \log \prob_\theta\left(\D(\EmpDist{Z_{1:n}}, p_0) \leq r\right) + I_r(\theta)$. 
Pick $\delta > 0$ and for any $t > 0$, let $q_t \in \Delta_\Xcal$ satisfy $\D(q_t, p_0) \leq t$ and
\[ \KL(q_t | p_\theta) \leq I_t(\theta) + \delta. \]
Now let $\alpha_n = \frac{2C |\Xcal|}{n}$ and observe that for $n > \frac{2 C |\Xcal|}{r-\epsilon_0}$, such a $q_{r - \alpha_n} \in \Delta_{\cX}$ exists. Letting $q^{(n)}_{r - \alpha_n} \in \Delta_\Xcal$ be the discretization promised by \Cref{lem:rounded-probability-vector} and utilizing \Cref{assum:finite-continuous-distance}, we have
\begin{align*}
\D(q^{(n)}_{r - \alpha_n}, p_0) 
\leq \D(q^{(n)}_{r - \alpha_n}, q_{r - \alpha_n}) + \D(q_{r - \alpha_n}, p_0) 
\leq C \|q^{(n)}_{r - \alpha_n} - q_{r - \alpha_n}\|_1 + r - \alpha_n \leq r.
\end{align*}
Thus we have
\begin{align*}
&\hspace{-3em}\frac{1}{n} \log \prob_\theta\left(\D(\EmpDist{Z_{1:n}}, p_0) \leq r \right) \geq \frac{1}{n} \log \prob_\theta\left( \EmpDist{Z_{1:n}} = q_{r-\alpha_n}^{(n)}  \right)\\
&\geq -\frac{|\Xcal|}{n} \log (n+1) - \KL( q^{(n)}_{r - \alpha_n} | p_\theta) \\
&\geq - \left(1+ \frac{|\Xcal|^2}{n}\right)(I_{r - \alpha_n}(\theta)+\delta) - \frac{|\Xcal|}{n} \left( 3 \log(n+1) + \log |\Xcal| + \frac{|\Xcal|}{e} \right) \\
&\geq - I_{r - \alpha_n}(\theta) - \delta - \frac{|\Xcal|}{n}\left( 3 \log(n+1) + \log |\Xcal| + |\Xcal| \left( V + \frac{1}{e} + \delta \right) \right) \\
&\geq -I_r(\theta) - \delta - \frac{\alpha_n V}{r - \epsilon_0} - \frac{|\Xcal|}{n}\left( 3 \log(n+1) + \log |\Xcal| + |\Xcal| \left( V + \frac{1}{e} + \delta \right) \right),
\end{align*}
where the second line follows from~\cite[Theorem~11.1.4]{cover2006elements}, the third line follows from \Cref{lem:rounded-probability-vector}, and the last line follows from \Cref{lem:okl-continuity}. Rearranging the above and utilizing the fact that $\delta > 0$ was arbitrary gives us the lemma statement.
\end{proof}

With \Cref{lem:finite-sanov} in hand, we can prove the following convergence result for the coarsened likelihood.
\begin{theorem}
    \label{lem:finite-coarsened-likelihood-convergence}
    Suppose \Cref{assum:finite-continuous-distance,assum:finite-okl} hold, and let $\epsilon > \epsilon_0$. If $x_1,\ldots,x_n \sim p_0$, then with probability at least $1-\delta$, 
    \begin{align*}
    \left| \frac{1}{n} \log L_\epsilon(\theta|x_{1:n} ) + I_{\epsilon}(\theta)  \right| &\leq  \frac{C V |\Xcal|}{\epsilon - \epsilon_0} \sqrt{\frac{2}{n} \log \frac{2|\Xcal|}{\delta}} \\
    &\hspace{3em} + \frac{3|\Xcal|}{n} \left( 3 \log(n+1) + \log|\Xcal| + |\Xcal|\left(V + \frac{1}{e} \right) + \frac{4C}{\epsilon - \epsilon_0} \right).   
    \end{align*}
    whenever $n > \max \left\{2 \left( \frac{C |\Xcal|}{\epsilon - \epsilon_0} \right)^2 \log \frac{2|\Xcal|}{\delta}, \frac{8C |\Xcal|}{\epsilon - \epsilon_0} \right\}$.
\end{theorem}
\begin{proof}
For $r > 0$, define $M_{n,r}(\theta)= \prob_\theta\left(\D(\EmpDist{Z_{1:n}}, p^0) \leq r\right)$. By \Cref{lem:finite-sanov}, 
\begin{equation}
\label{eqn:intermediate-coarsened-okl-approx}
    \left| \frac{1}{n}\log M_{n,r}(\theta) + I_r(\theta) \right| \leq  \frac{|\Xcal|}{n} \left( 3 \log(n+1) + \log|\Xcal| + |\Xcal|\left(V + \frac{1}{e} \right) + \frac{2C}{r - \epsilon_0} \right)
\end{equation}
for all $r > \epsilon_0$ satisfying that $n \geq \frac{4C |\Xcal|}{r - \epsilon_0}$.

Now suppose that $x_1, \ldots, x_n \sim p_0$. Then Hoeffding's inequality combined with \Cref{assum:finite-continuous-distance} implies that with probability at least $1-\delta$,
\[ \D(\EmpDist{x_{1:n}}, p_0) \leq C \|\EmpDist{x_{1:n}} - p_0 \|_1 \leq  C|\Xcal| \sqrt{\frac{1}{2n} \log \frac{2|\Xcal|}{\delta}} =: \alpha_n.  \]
Let us condition on this event occurring. For $n > 2\left( \frac{C |\Xcal|}{\epsilon - \epsilon_0}\right)^2 \log \frac{2|\Xcal|}{\delta}$, we have $\alpha_n < (\epsilon - \epsilon_0)/2$. Thus, we may write
\begin{align*}
\left| \log M_{n,\epsilon}(\theta) -  \log L_\epsilon(\theta | x_{1:n} )\right| 
&= \left| \log \prob_\theta\left( \D(\EmpDist{Z_{1:n}}, p_0) \leq \epsilon \right) - \log \prob_\theta\left( \D(\EmpDist{Z_{1:n}}, \EmpDist{x_{1:n}}) \leq \epsilon \right) \right| \\
&= \log \max \left\{ \frac{\prob_\theta\left(\D(\EmpDist{Z_{1:n}}, p_0) \leq \epsilon\right)}{\prob_\theta\left(\D(\EmpDist{Z_{1:n}}, \EmpDist{x_{1:n}}) \leq \epsilon\right) },  \frac{\prob_\theta\left(\D(\EmpDist{Z_{1:n}}, \EmpDist{x_{1:n}}) \leq \epsilon\right)}{\prob_\theta\left(\D(\EmpDist{Z_{1:n}}, p_0) \leq \epsilon\right)} \right\} \\
&\leq \log \max \left\{  \frac{\prob_\theta\left(\D(\EmpDist{Z_{1:n}}, p_0) \leq \epsilon\right)}{\prob_\theta\left(\D(\EmpDist{Z_{1:n}}, p_0) \leq \epsilon - \alpha_n\right)},  \frac{\prob_\theta\left(\D(\EmpDist{Z_{1:n}}, p_0) \leq \epsilon + \alpha_n\right) }{\prob_\theta\left(\D(\EmpDist{Z_{1:n}}, p_0) \leq \epsilon\right)} \right\} \\
&= \log \max \left\{ \frac{M_{n,\epsilon}(\theta)}{M_{n,\epsilon-\alpha_n}(\theta)},  \frac{M_{n,\epsilon + \alpha_n}(\theta)}{M_{n,\epsilon}(\theta)} \right\}
\end{align*}
By \cref{eqn:intermediate-coarsened-okl-approx} and \Cref{lem:okl-continuity}, we have
\begin{align*}
&\frac{1}{n}\log\frac{ M_{n,\epsilon}(\theta)}{M_{n,\epsilon - \alpha_n}(\theta)} \\
&\leq I_{\epsilon}(\theta) - I_{\epsilon - \alpha_n}(\theta) + \frac{2|\Xcal|}{n} \left( 3 \log(n+1) + \log|\Xcal| + |\Xcal|\left(V + \frac{1}{e} \right) + \frac{2C}{\epsilon - \alpha_n - \epsilon_0} \right) \\
&\leq \frac{2\alpha_n}{\epsilon - \epsilon_0}V + \frac{2|\Xcal|}{n} \left( 3 \log(n+1) + \log|\Xcal| + |\Xcal|\left(V + \frac{1}{e} \right) + \frac{4C}{\epsilon - \epsilon_0} \right).
\end{align*}
Similarly, we also have
\begin{align*}
&\frac{1}{n}\log \frac{M_{n,\epsilon+\alpha_n}(\theta)}{M_{n,\epsilon}(\theta)} \\
&\leq I_{\epsilon + \alpha_n}(\theta) - I_{\epsilon}(\theta) + \frac{2|\Xcal|}{n} \left( 3 \log(n+1) + \log|\Xcal| + |\Xcal|\left(V + \frac{1}{e} \right) + \frac{2C}{\epsilon + \alpha_n - \epsilon_0} \right) \\
&\leq \frac{\alpha_n}{\epsilon - \epsilon_0 + \alpha_n}V + \frac{2|\Xcal|}{n} \left( 3 \log(n+1) + \log|\Xcal| + |\Xcal|\left(V + \frac{1}{e} \right) + \frac{4C}{\epsilon - \epsilon_0} \right).
\end{align*}
Putting it all together gives us the theorem statement.
\end{proof}


\subsection{Asymptotics for  continuous spaces}
\label{sec:coarsened-likelihood-asymptotic-cont}
%auto-ignore

\iffalse
\subsection{Introduction}
Let us assume that $\cX$ is a Polish space (i.e. a complete separable metric space) equipped with its Borel sigma algebra $\cB(\cX)$.  Then $\cP(\cX)$, the set of (Borel)  probability measures on $\cX$, is also a Polish space when equipped with the topology of weak convergence \cite{billingsley1971weak}.
%
%For a measurable function $f: \cX \to \R$ and measure $\lam$ on $\cX$, let $\|f\|_{p,\lam} = \left(\int |f|^p d\lam\right)^{1/p}$ denote the $L_p$ norm of $f$ with respect to $\lam$. Given a kernel $\K: \cX \times \cX \to \R$ and a measure $\mu$ on $\cX$, the convolved density of $\mu$, denoted by the map $\kappa \star \mu : \cX \to [0,\infty)$, is the function  $x \mapsto \int \K(x, y) \mu(dy)$.
Let $\D$ be an integral probability semi-metric on $\cP(\cX)$, that is continuous with respect to the weak convergence topology.

Suppose we are given a model family $\{P_\theta\}_{\theta \in \Theta} \subseteq \cP(\cX)$ and observed data $x_1, \ldots, x_n \iid P_0 \in \cP(\cX)$. In this note, we will study the asymptotics as $n \to \infty$ of the  \emph{coarsened likelihood} defined as
$$
L_\epsilon(\theta|x_{1:n}) = \prob_\theta(\D(\EmpDist{Z_{1:n}}, \EmpDist{x_{1:n}}) \leq \epsilon ) 
$$
where, given $\theta \in \Theta$, the probability operation $\prob_\theta$ is with respect to $Z_1, \ldots, Z_n \iid P_\theta$, and $\EmpDist{z_{1:n}} = n^{-1}\sum_{i=1}^n \delta_{z_i} \in \cP(\cX)$ is the empirical distribution.

To describe the limiting expression, recall the definition of the    
Kullback–Leibler divergence (also called as relative entropy)
\begin{equation*}
	\KL(P|Q)  = \begin{cases}
		\int \log \frac{dP}{dQ} dP & \text{ if } P \ll Q \\
		\infty & \text{ otherwise }
	\end{cases}
\end{equation*}
where $P \ll Q$ denotes the absolute continuity condition that $P(A) = 0$ whenever $Q(A) = 0$, and $\frac{dP}{dQ}$ is the Random Nikodym derivative of $P$ with respect to $Q$. With this, we can define the \emph{generalized optimistic Kullback Leibler}  function 

\begin{equation*}
	\okl = \inf_{\substack{Q \in \cP(\cX)\\
			\D(Q, P_0) \leq \epsilon}} \KL(Q|P_\theta)
\end{equation*}
\fi


In this section, we will show that the coarsened likelihood continues to converge in probability to the OKL when $\cX$ is a metric space. More precisely, let $\cX$ be a Polish space (i.e.~a complete and separable metric space) equipped with its Borel sigma algebra $\cB(\cX)$. Then $\cP(\cX)$, the set of (Borel) probability measures on $\cX$, can be equipped with the topology of weak-convergence \citep{billingsley2013convergence}. In more detail, we say that a sequence of measures $\{\cP_n\}_{n \in \nat}  \subseteq \cP(\cX)$ weakly converges to a measure $P$, denoted as $P_n \dconv P$, if $\lim_{n \to \infty} \int f dP_n = \int f dP$ for each continuous and bounded function $f: \cX \to \R$. The space $\cP(\cX)$ is also a Polish space under this topology \citep{billingsley2013convergence}. 

 Recall the definition of the OKL for a general distance $\D$ over $\cP(\cX)$ from \cref{eqn:okl-general-distance}:
\[ \okl[\epsilon] = \inf_{\substack{Q \in  \cP(\cX) \\ \D(Q, P_0) \leq \epsilon}} \KL(Q | P_\theta). \]
To establish an asymptotic connection between the coarsened likelihood and the OKL, we will make the following assumption on $\D$.
\begin{assume}
\label{assump:general-distance}
    For any $P,Q, R \in \cP(\cX)$ and $\lambda \in (0,1)$, the following holds:
    \begin{enumerate}[(a)]
		\item $\D(P, P) = 0$.
		\item $\D(P, Q) = \D(Q, P)$.
		\item $\D(P, Q) \leq \D(P, R) + \D(R, Q)$.
		\item $\D$ is convex in its arguments: $\D(P, (1-\lambda)Q + \lambda R) \leq \lambda \D(P, Q) + (1-\lambda) \D(P, R)$.
		\item For any sequence of probability measures $P_n \dconv P$, $\D(P_n, Q) \to \D(P, Q)$ as $n \to \infty$.
	\end{enumerate}
\end{assume}
Conditions~(a)-(c) simply require that $\D$ is a pseudometric, i.e. it satisfies all the requirements of a metric except for the requirement that $\D(P,Q) > 0$ whenever $P \neq Q$. Condition~(d) (combined with condition~(a)) allows us to apply \Cref{lem:okl-continuity}, and condition~(e) ensures that $\D$ is continuous with respect to the topology of weak convergence.


Given this assumption on $\D$, we can demonstrate the following asymptotic convergence.

\begin{theorem}
\label{thm:clikelihood-asymptotics}
Suppose that \Cref{assump:general-distance} holds and that $\okl[\epsilon_0] < \infty$ for some $\epsilon_0 \geq 0$. For any $\epsilon > \epsilon_0$, if $x_1, \ldots, x_n \iid P_0$, then 
\begin{equation*}
	\frac{1}{n} \log L_\epsilon(\theta|x_{1:n}) \pconv -\okl[\epsilon]
\end{equation*}
as $n \to \infty$.
\end{theorem}

\subsubsection{Proof of \texorpdfstring{\Cref{thm:clikelihood-asymptotics}}{thmclike}}

\iffalse
\begin{assume} Suppose $\D$ is an IPS that is continuous with respect to weak convergence. In other words, we assume that 
	\begin{enumerate}[(a)]
		\item  There is a class of measurable functions $\cF \subseteq \{f : \cX \to \R \}$ such that 
		$$
		\D(\mu, \nu) = \sup_{f \in \cF} |\int f d\mu - \int f d\nu|.
		$$
		\item For any sequence of probability measures $\mu_n \dconv \mu$ in $\cP(\cX)$, $\D(\mu_n, \nu) \to \D(\mu, \nu)$ as $n \to \infty$.
	\end{enumerate}
	\label{assume:dist-assumptions}
\end{assume}


\begin{lemma} 
	\label{lem:okl-continuity}
	If $\D$ satisfies part (a) of \Cref{assume:dist-assumptions}, then for any $\epsilon, t \geq 0$
	$$
	0 \leq \okl - \okl[\epsilon+t] \leq \frac{t}{\epsilon+t} \KL(P_0|P_\theta).
	$$
\end{lemma}
\begin{proof} For any $\delta > 0$, let us choose $Q \in \cP(\cX)$ such that $\KL(Q|P_\theta) \leq \okl[\epsilon+t] + \delta$. Since $\D(Q, P_0) \leq \epsilon + t$, using the IPS property  of $\D$ and $\lambda_t = t/(\epsilon+t)$, note that  
	$\D((1-\lambda_t) Q + \lambda_t P_0, P_0) = (1-\lambda_t)\D(Q, P_0) \leq \epsilon$. Thus, we can use  convexity of the KL divergence to obtain
	\begin{align*}
		\okl[\epsilon] \leq \KL((1-\lambda_t) Q^t + \lambda_t P_0|P_\theta) &\leq (1-\lambda_t)\KL(Q^t|P_\theta) + \lambda_t \KL(P_0|P_\theta)  \\
		&\leq \okl[\epsilon + t] + \delta + \frac{t}{\epsilon + t}  \KL(P_0 | P_\theta).
	\end{align*}
	Since $\okl[\epsilon] \geq \okl[\epsilon + t] \geq 0$ and $\delta > 0$ is arbitrary, \Cref{lem:okl-continuity} follows.
\end{proof}
\fi

Similar to the setting in \Cref{sec:coarsened-likelihood-asymptotic-finite}, we will study asymptotics of the coarsened likelihood $\owl$ by first studying the asymptotics of its population level analog  
\begin{equation}
	\label{eq:pop-clike}
	\owlM[\epsilon] = \prob_\theta(\D(\EmpDist{Z_{1:n}}, P_0)\leq \epsilon)
\end{equation}
obtained by replacing the  empirical distribution of the data  $\EmpDist{x_{1:n}}$ by the  population level quantity $P_0$.


Next, to study the asymptotics of $\owlM$, we will invoke Sanov's theorem from Large Deviation theory which says that the law of the empirical distribution $\EmpDist{Z_{1:n}}$  satisfies a Large Deviation principle in the space $\cP(\cX)$ with rate function $\mu \mapsto \KL(\mu|P_\theta)$, when $Z_1, \ldots, Z_n \iid P_\theta$. More precisely, we will show that the error term 
\begin{equation}
	\label{eq:sanov-error}
	\lderr = \left|\frac{1}{n}\log \owlM + \okl\right|
\end{equation}
converges to zero as $n \to \infty$.

\begin{lemma}
If \Cref{assump:general-distance} holds and $\okl[\epsilon_0] < \infty$ for some $\epsilon_0 \geq 0$, then for any $\epsilon > \epsilon_0$, $\lim\limits_{n \to \infty} \lderr = 0$.
\label{lem:sanov}
\end{lemma}
\begin{proof} 
	
	Assume $Z_1, \ldots, Z_n \iid P_\theta$. Then Sanov's  theorem \cite[Theorem 6.2.10]{demboLargeDeviationsTechniques2010} shows that for any Borel measurable subset $\Gamma \subseteq \cP(\cX)$,
	\begin{equation}
		\label{eq:sanov}
		\begin{aligned}
			-\inf_{Q \in \Gamma^\circ} \KL(Q|P_\theta) &\leq \liminf_{n \to \infty} \frac{1}{n} \log \prob_\theta(\EmpDist{Z_{1:n}} \in \Gamma)\\ 
			&\leq \limsup_{n \to \infty} \frac{1}{n} \log \prob_\theta(\EmpDist{Z_{1:n}} \in \Gamma) \leq -\inf_{Q \in \bar{\Gamma}} \KL(Q|P_\theta),
		\end{aligned}
	\end{equation}
	where $\Gamma^\circ$ and $\bar{\Gamma}$ refer to the interior and closure of $\Gamma$ under the weak-topology on $\cP(\cX)$. 
	
	
	Now consider the set $\Gamma_t = \{Q \in \cP(\cX) | \D(Q, P_0) \leq t \}$ indexed by the parameter $t \geq 0$. By the assumed continuity of $\D$ under the topology of weak-convergence (\Cref{assump:general-distance}(e)), the set  $\Gamma_\epsilon$ is closed while the set $\Gamma_{\epsilon^-} = \{Q \in \cP(\cX) \mid \D(Q,P_0) < \epsilon\}$ is an open set. Thus using $\bar{\Gamma}_{\epsilon} = \Gamma_\epsilon$ and $\cup_{r < \epsilon} \Gamma_r = \Gamma_{\epsilon^-}\subseteq \Gamma_\epsilon^\circ$, we see 
	\begin{equation*}
		\begin{aligned}
			\okl[\epsilon] = 
			\inf_{Q \in \bar{\Gamma}_\epsilon} \KL(Q|P_\theta) \leq \inf_{Q \in \Gamma^\circ_\epsilon}  \KL(Q|P_\theta)
			\leq \inf_{Q \in \Gamma_{\epsilon^-}} \KL(Q|P_\theta)  \leq \okl[r]
		\end{aligned}
	\end{equation*}
	for each $0 < r < \epsilon$. Next by the condition $\okl[\epsilon_0] < \infty$, we may apply \Cref{lem:okl-continuity} to conclude that the map $t \mapsto \okl[t]$ is continuous at $t = \epsilon$, i.e. $\lim_{t \to \epsilon} \okl[t] = \okl[\epsilon]$. Hence letting $r$ increase to $\epsilon$ in the above display, we find that
	$$
	\inf_{Q \in \bar{\Gamma}_\epsilon} \KL(Q|P_\theta) = \inf_{Q \in \Gamma^\circ_\epsilon} \KL(Q|P_\theta) = \okl[\epsilon].
	$$
	Thus taking $\Gamma = \Gamma_\epsilon$ in \cref{eq:sanov} shows that the limit
	$$
	\frac{1}{n} \log \prob_\theta(\D(\EmpDist{Z_{1:n}},P_0) \leq \epsilon) = \frac{1}{n} \log \prob_\theta(\EmpDist{Z_{1:n}} \in \Gamma_\epsilon) \to - \okl
	$$
	holds as $n \to \infty$. Recalling  the definition of $\owlM$ in \cref{eq:pop-clike}, we see that the asymptotic error term $\lderr$ in \cref{eq:sanov-error} is converging to zero as $n \to \infty$.
			

\end{proof}


Now we can prove \Cref{thm:clikelihood-asymptotics} by carefully accounting for the error between the coarsened likelihood  $\owl$ and its population-level analog $\owlM$ from  \Cref{eq:pop-clike}.

\begin{proof}[Proof of \Cref{thm:clikelihood-asymptotics}] Let $x_1, \ldots, x_n \iid P_0$ and pick $0 < t < \epsilon - \epsilon_0$. Define $E_{n,t}$ as the event that $\D(\EmpDist{x_{1:n}}, P_0) \leq t$. Since $\D$ is continuous with respect to weak convergence and $\EmpDist{x_{1:n}} \dconv P_0$ as $n \to \infty$ by the weak law of large numbers, we have $\lim_{n \to \infty} \prob(E_{n,t}) = 1$ for any $t > 0$. By \Cref{assump:general-distance}(c) we have $|\D(\EmpDist{Z_{1:n}}, P_0) -  \D(\EmpDist{Z_{1:n}}, \EmpDist{x_{1:n}})| \leq \D(P_0, \EmpDist{x_{1:n}})$, implying that on the event $E_{n,t}$ we have
%
\begin{equation}
	\label{eq:owlboundsonEnt}
	\owlM[\epsilon-t] \leq \owl \leq \owlM[\epsilon+t].
\end{equation}
%
Thus on the event $E_{n,t}$, we can bound
\begin{align*}
	\abs*{\frac{1}{n} \log \owl + \okl} 
	&\leq \abs*{\frac{1}{n} \log \owlM[\epsilon+t] + \okl}+ \abs*{\frac{1}{n} \log \owlM[\epsilon-t] + \okl}\\
	&\leq \lderr[\epsilon+t] + \lderr[\epsilon-t] + \abs{\okl[\epsilon+t] - \okl}+\abs{\okl[\epsilon-t] - \okl}\\
	&\leq \lderr[\epsilon+t] +  \lderr[\epsilon-t] + \frac{2t}{\epsilon - \epsilon_0} \okl[\epsilon_0]
\end{align*}
where the first inequality uses  \cref{eq:owlboundsonEnt}, the second uses \cref{eq:sanov-error}, and the last uses \Cref{lem:okl-continuity}. Hence given any $\delta > 0$ and $t \in (0,\epsilon - \epsilon_0)$ such that $t \okl[\epsilon_0] <  (\epsilon - \epsilon_0) \delta/4$, we have
\begin{align*}
	\prob\brR*{\abs*{\frac{1}{n} \log \owl + \okl} > \delta} \leq \prob(E_{n,t}^c) + \I{\lderr[\epsilon-t] + \lderr[\epsilon+t] > \delta/2}.
\end{align*}
Hence using \Cref{lem:sanov} and the fact that $\prob(E_{n,t}) \to 1$ as $n\to \infty$, we now see that 
$$
\lim_{n \to \infty} \prob\brR*{\abs*{\frac{1}{n} \log \owl - \okl} > \delta} = 0.
$$ 
Since $\delta > 0$ was arbitrary, this completes the proof.
\end{proof}

\subsubsection{Smoothed total variation distance satisfies \texorpdfstring{\Cref{assump:general-distance}}{assgenraldist}}
\label{sec:smoothed-tvd-is-continous-ips}

Suppose $\K: \cX \times \cX \to [0,\infty)$ is a probability kernel with respect to measure $\lambda$. That is, assume $\int \K(x,y) d\lambda(x) = 1$ for each $y \in \cX$.  Given a measure $\mu \in \cP(\cX)$, this allows us to define a smoothed probability measure $\K \star \mu \in \cP(\cX)$ that has density  

\begin{equation*}
	\frac{d(\K \star \mu)}{d\lambda} (x) =  f_{\K, \mu}(x) = \int \K(x, y) d\mu(y)
	%(\K \star \mu)(A)  = \int_A f_{\K, \mu}(x)  d\lambda(x).
\end{equation*}
with respect to $\lambda$.

Recall the definition of the total variation distance on $\cP(\cX)$, 
\begin{equation}
	\label{eq:tvd}
	\tv(\mu, \nu) = \sup_{B \in \cB(\cX)}|\mu(B)-\nu(B)| = \sup_{g: \cX \to [-1,1]} \abs*{\int g d\mu - \int g d\nu}.
\end{equation}
When $\mu$ and $\nu$ have densities $f_\mu$ and $f_\nu$ with respect to a common measure $\lambda$, one can additionally show $\tv(\mu, \nu) = \frac{1}{2}\int  \abs{f_\mu-f_\nu} d\lambda$. 

Although $\tv$ is not closed with respect to weak convergence, we can show that the following kernel-smoothed version of TV distance is.
\begin{definition} Given a probability density kernel $\K$, the smoothed total-variation distance is given by  
	\begin{equation*}
		\tvk(\mu, \nu) = \tv(\K \star \mu, \K \star \nu) = \frac{1}{2} \int \abs{f_{\K,\mu}(x) - f_{\K, \nu}(x)} d\lambda(x)
	\end{equation*}
\end{definition}
Now we will show that $\tvk$ satisfies \Cref{assump:general-distance} when $\K$ is a bounded and continuous kernel. 
\begin{proposition}
If $\K$ is a bounded and continuous kernel, then $\tvk$ satisfies \Cref{assump:general-distance}.
\end{proposition}
\begin{proof}
First observe that the smoothed TV distance is just an ordinary TV distance between smoothed densities.
%
Since the ordinary TV distance is a metric, this immediately implies that the smoothed TV  satisfies the identity, symmetry, and triangle inequality properties. 

To establish convexity of $\tvk$, note that for measures $\mu, \nu \in \cP(\cX)$ and $v \in [0,1]$, we have
\[f_{\K, (1-v)\mu + v\nu}(x) = (1-v) \int \K(x, y) d\mu(y) + v  \int \K(x, y) d\nu(y) = (1-v)f_{\K, \mu}(x) + v f_{\K, \nu}(x) .  \]
Thus, for $\mu, \nu, \pi \in \cP(\cX)$, we have
\begin{align*}
\tvk(\pi, (1-v)\mu + v\nu) 
&= \frac{1}{2}\int \abs{f_{\K,\pi}(x) - f_{\K, (1-v)\mu + v\nu}(x)} d\lambda(x) \\
&= \frac{1}{2} \int \abs{f_{\K,\pi}(x) - ((1-v)f_{\K, \mu}(x) + v f_{\K, \nu}(x))} d\lambda(x) \\
&\leq (1-v) \frac{1}{2} \int \abs{f_{\K,\pi}(x) - f_{\K, \mu}(x)} d\lambda(x) + v \frac{1}{2}\int \abs{f_{\K,\pi}(x) - f_{\K, \nu}(x)} d\lambda(x) \\
&= (1-v) \tvk(\pi, \mu )  + v \tvk(\pi,\nu),
\end{align*}
where the inequality follows from the convexity of the absolute value function. Thus, $\tvk$ is convex in its arguments.


To establish the continuity of $\tvk$ under the topology of weak-convergence, suppose $\mu_n \dconv \mu$ in $\cP(\cX)$. Then since $y \mapsto \K(x,y)$ is a continuous and bounded function, the convergence 
$$
f_{\K, \mu_n}(x) \to f_{\K, \mu}(x)
$$
follows for each $x \in \cX$ as $n \to \infty$. Since $f_{\K, \nu}$ for any $\nu \in \cP(\cX)$ is a density with respect to $\lambda$, Scheff\'e's lemma shows that
$$
\tvk(\mu_n, \mu) = \frac{1}{2} \int \abs{f_{\K, \mu_n}(x) - f_{\K, \mu}(x)}d \lambda(x) \to 0  \ \ \text{ as } n \to \infty.
$$
Finally, by the triangle inequality, $\abs{\tvk(\mu_n, \nu) - \tvk(\mu, \nu)} \leq \tvk(\mu_n, \mu)$.
\end{proof}


\iffalse
Now we will show that $\tvk$ satisfies \Cref{assume:dist-assumptions} when $\K$ is a bounded
and continuous kernel.

First let us observe that $\tvk$ is an IPS. Indeed, using the last equality in  \cref{eq:tvd} one can see that
\begin{equation*}
	\tvk(\mu,\nu) = \sup_{g: \cX \to [-1,1]} \abs*{\int \smooth[\K]{g} d \nu  - \int \smooth[\K]{g} d\mu}
\end{equation*}
where $\smooth[\K]{g}(y) = \int g(x) \K(x,y) d\lambda(x)$ for any measurable function $g: \cX \to \R$.
\fi


\section{Auxiliary lemmas}
%auto-ignore

In the following lemma, we would like to understand the conditions under which two distributions $P_0$ and $P_1$ can be written as an $\epsilon$-contamination of one another.

\begin{lemma}
	\label{lem:contamination-condition}
	Suppose $ \epsilon \in [0, 1)$ and two probability measures $P_0, P_1$ are given. Then there is a probability measure $Q$ such that $P_0 = (1-\epsilon) P_1 + \epsilon Q$ if and only if the Radon Nikodym derivative $h = \frac{dP_1}{dP_0}$ exists and is bounded above by $\frac{1}{1-\epsilon}$ almost surely $P_0$.
\end{lemma}
\begin{proof} First we note the ``only if'' direction. Suppose indeed that $P_0 = (1-\epsilon) P_1 + \epsilon Q$. Then the absolute continuity condition $P_1 \ll P_0$ follows since $\epsilon < 1$. Thus by the Radon Nikodym theorem $h=\frac{dP_1}{dP_0}$ exists. Since $Q$ is a probability measure, for any non-negative bounded (measurable) function $f$ we have
	$$
	P_0(f) - (1-\epsilon) P_1(f) = \epsilon Q(f) \geq 0,
	$$
	where we use the notation $\mu(f) \doteq \int f d\mu$. Noting $P_1(f) = P_0(hf)$, we thus obtain that
	\begin{equation}
		\label{eq:non-negative}
		\int f dP_0 \geq (1-\epsilon) \int f h dP_0 \qquad \text{ for each bounded } f \geq 0. 
	\end{equation}
	This shows that $h$ is bounded above by $\frac{1}{1-\epsilon}$ almost surely.
	
	Now let us show the ``if'' condition. Suppose $h = \frac{dP_1}{dP_0}$ exists and is bounded above by $\frac{1}{1-\epsilon}$.
	Then equation \eqref{eq:non-negative} is seen to be satisfied. 
	Let us assume $\epsilon > 0$, since $\epsilon = 0$ means that $P_1 = P_0$ and there is nothing to show. This allows us to define the signed measure $Q$ given by 
	$$
	Q(f) \doteq \frac{P_0(f) - (1-\epsilon) P_1(f)}{\epsilon}
	$$
	for every bounded $f$. Since $Q(f) \geq 0$ for each $f \geq 0$, this implies that $Q$ is indeed a non-negative measure. It is indeed a probability measure since $Q(1) = \frac{P_0(1) - (1-\epsilon)P_1(1)}{\epsilon} = 1$. Finally, the result that $P_0 = (1-\epsilon) P_1 + \epsilon Q$ holds by the way $Q$ was defined.
\end{proof}

\iffalse
The following is a rough bound on the integrals with respect to arbitrary $Q \in \cP(\cX)$ in terms of a based measure $P$ and the KL term $\KL(Q|P)$.

\begin{proposition}
	\label{prop:klbound}
	For a non-negative function $f: \cX \to \Rnn$,  $\sigma \geq 1$, and $P, Q \in \cP(\cX)$ we have the bound
	\begin{equation*}
		\int f dQ \leq \int e^{\sigma f} dP + \frac{1}{\sigma} \KL(Q|P).
	\end{equation*}
\end{proposition}
\begin{proof} We may focus on the case $Q \ll P$, since otherwise the term on the right takes the value $+\infty$. Next, following the trick in \cite[Lemma 2.4]{budhiraja2019analysis}, note from the fact $\sup_{a \in \R} [ab - e^{\sigma a}] = (b/\sigma)(\log (b/\sigma) - 1)$ that the inequality $ab \leq e^{\sigma a} + \frac{1}{\sigma}
	(b \log b - b +1)$ holds for every $a, b \geq 0$ and $\sigma \geq 1$. Let $g \doteq  \frac{dQ}{dP}$ and apply the last inequality with $a=f$ and $b=g$ to see that:
	\begin{align*}
		\int f dQ =  \int f g dP  = \int e^{\sigma f} dP + \frac{1}{\sigma}\left(\int g (\log g) dP - \int g dP + 1 \right) = \int e^{\sigma f} dP + \frac{1}{\sigma}\KL(Q|P)
	\end{align*}
\end{proof}


\begin{lemma}
	\begin{enumerate}
		\item the function $F(w,\theta) \doteq  \sum_{i=1}^n [w_i \log w_i - w_i \log  p_\theta(x_i)]$ is continuous jointly as a function of $(w, \theta)$.
		\item the maps $H: \Delta_n \mapsto \Theta$ given by $H(w)= \argmin_{\theta \in \Theta} F(w, \theta)$ and $G: \Theta \mapsto \Delta_n$ given by $G(\theta) = \argmin_{w \in \Delta_n} F(w,\theta)$ are well-defined and continuous.
	\end{enumerate}
	Let $\Theta_\epsilon \doteq \{ \theta |  (H \circ G)(\theta) = \theta \}$; Then the distance to this set $d(\theta^t, \Theta_\epsilon)$ converges zero as $t \to \infty$.
\end{lemma}
\begin{proof}
	Indeed, if this is not the case, since the sequence $\{(w_t, \theta_t)\}_{t=1}^\infty$ lies in the compact set $\Delta_n \times K$ where $K \doteq \operatorname{Img}(H)$, we can choose a limit point $(w^\infty, \theta^\infty)$ for this sequence so that $\theta^\infty \in K \setminus \Theta_I$. Suppose $(w^\infty, \theta^\infty) = \lim_{k \to \infty} (w^{t_k}, \theta^{t_k})$ for some strictly increasing sub-sequence $\{t_k\}_{k \in \nat}$. Using \eqref{eq:montone-iterations} the the continuity of $F$, $F(w^\infty, \theta^\infty) = \liminf_{k \to \infty} F(w^{t_k}, \theta^{t_k}) = \liminf_{t \to \infty} F(w^t, \theta^t)$. Since $w^{t+1} = G(\theta^t)$ and $\theta^{t+1} = H(w^{t+1})$, the continuity assumption show that $F(w^\infty, H(w^\infty)) = \lim_{k \to \infty} F(w^{t_k}, \theta^{t_k}) = F(w^\infty, \theta^\infty)$. Similarly, using the continuity and \eqref{eq:montone-iterations},  $F(G(\theta^\infty), \theta^\infty) = \lim_{k \to \infty} F(w^{t_k+1}, \theta^{t_k}) \geq \lim_{k \to \infty} F(w^{t_k+1}, \theta^{t_k+1}) = F(w^\infty, \theta^\infty)$. By the uniqueness of the minimizers in the definition of $H$ and $G$, we have $w^\infty = G(\theta^\infty)$ and $\theta^\infty = H(w^\infty)$. Hence $\theta^\infty \in \Theta_\epsilon$ -- a contradiction.
\end{proof}
\fi


\section{Simulations with random corruptions}
\label{app:more-simulations}
%auto-ignore
In this section, we present the simulation results for the setting in which the data points to be corrupted are chosen entirely at random. For each of the problem settings, all of the details remain exactly the same as in the max-likelihood corruption simulations, with the only change coming from which data points are chosen to be corrupted.

\begin{figure}
    \centering
    \includegraphics[width=0.9\textwidth]{figures/gaussian_wmle_rand_corruption.pdf}
    \caption{Gaussian results for random corruptions. Dashed black line denotes average performance of MLE on full uncorrupted dataset. Shaded regions denote 95\% confidence intervals over 50 random seeds.}
    \label{fig:rand-gaussian}
\end{figure}
\Cref{fig:rand-gaussian} presents the random corruption results for estimating a multivariate normal distribution. In contrast with the max-likelihood corruption setting, we see that OWL without kernelization outperforms OWL with kernelization in higher dimensions. It is possible that this is due to the difficulty of density estimation in higher dimensions.

\begin{figure}
    \centering
    \includegraphics[width=0.9\textwidth]{figures/lin_rand_corruption.pdf}
    \caption{Linear regression results for random corruptions. Dashed black line denotes average performance of MLE on full uncorrupted training set. Shaded regions denote 95\% confidence intervals over 50 random seeds.}
    \label{fig:rand-linear-regression}
\end{figure}
\Cref{fig:rand-linear-regression} presents the random corruption results for linear regression. The results here are qualitatively similar to those for the max-likelihood corruption setting, with all three robust methods performing well in the simulated data setting but with RANSAC performing notably worse with QSAR data.


\begin{figure}
    \centering
    \includegraphics[width=1.0\textwidth]{figures/log_rand_corruption.pdf}
    \caption{Logistic regression results for random corruptions. Dashed black line denotes average performance of MLE on full uncorrupted training set. Shaded regions denote 95\% confidence intervals over 50 random seeds.}
    \label{fig:rand-logistic-regression}
\end{figure}
\Cref{fig:rand-logistic-regression} presents the random corruption results for logistic regression. As in the max-likelihood corruption case, we see that OWL outperforms the other methods across all three datasets. Moreover, we see that on the Enron spam dataset, OWL even outperforms the uncorrupted MLE baseline, which is entirely possible if the logistic regression model is mis-specified.

\begin{figure}
    \centering
    \includegraphics[width=0.9\textwidth]{figures/clustering_max_corruption.pdf}
    \caption{Mixture model results for random corruptions. Dashed black line denotes average performance of MLE on full uncorrupted training set. Shaded regions denote 95\% confidence intervals over 50 random seeds.}
    \label{fig:rand-both-mixture}
\end{figure}
\Cref{fig:rand-both-mixture} presents the random corruption results for the mixture model settings. The results here are qualitatively similar to those for the max-likelihood corruption setting.


\section{More details of the micro-credit study}
\label{app:micro-credit}
%auto-ignore
This section contains additional details of our analysis in \Cref{sec:micro-credit}. 

\Cref{fig:micro-scatter} shows a scatter-plot for the data and the presence of outliers.

\Cref{fig:micro-log-scale-plots} shows the 90\% OS-bootstrap confidence bands for the AIT estimates, with the $x$-axis scaled to emphasize the uncertainty for small values of $\epsilon$. 

\Cref{fig:micro-okl-plot} plots minimum-OKL estimate for various values $\epsilon$, along with the associated 90\% bootstrap confidence bands.

\Cref{fig:micro-outlier-hist-plot} shows the distribution of profit values for points that were declared as outliers by the OWL procedure at the  parameter value $\epsilon_0$.

\begin{figure}[h]
    \centering
    \includegraphics{figures/micro_data_scatterplot.pdf}
    \caption{Scatter-plot of the household profit values across treated and non-treated households. Even after removing the  household with the extreme profit of -40000 USD PPP, there are still households with extreme profit values that cause brittleness in estimating the average treatment effect.}
    \label{fig:micro-scatter}
\end{figure}

\begin{figure}[h]
    \centering
    \includegraphics{figures/micro_ate_log.pdf}
    \caption{The plot from the left panel of \Cref{fig:micro_fig}, plotted on a re-scaled $x$-axis to emphasize the uncertainty in the AIT estimates for small values of $\epsilon$. The confidence bands become narrow roughly at the tuned value of $\epsilon_0 = 0.005$ (\Cref{sec:tune-epsilon}), suggesting that the outliers that cause brittleness may  have been down-weighted by OWL for $\epsilon = \epsilon_0$.}
    \label{fig:micro-log-scale-plots}
\end{figure}

\begin{figure}[h]
    \centering
    \includegraphics{figures/micro_okl_linear.pdf}
    \caption{The minimum OKL estimate (i.e. $\hat{R}(\epsilon) = \min_{\theta \in \Theta} \hat{I}_\epsilon(\theta)$) versus $\epsilon$ plot for the micro-credit example. Using the notion of curvature in \Cref{sec:tune-epsilon}, the value $\epsilon_0 = 0.005$ was identified as the point at which this graph has its prominent kink. The $90\%$ confidence bands under $m=50$ OS-bootstrap iterations are also shown.} 
    \label{fig:micro-okl-plot}
\end{figure}

\begin{figure}[h]
    \centering
    \includegraphics{figures/micro_hist.pdf}
    \caption{The profit distribution for households that were declared to be inliers ($\{ i : w_i < 1 \}$) versus outliers ($\{ i: w_i \leq 1 \}$) by the OWL procedure at parameter $\epsilon_0$. For clarity, we omitted an outlying household with a profit value of less that $-40K$ USD PPP.} 
    \label{fig:micro-outlier-hist-plot}
\end{figure}

%\subsection{Proofs from Section \ref{sec:okl}}

%\input{theory/clikelihood-asymptotics}

%\subsection{Proof for finite spaces (Section \ref{sec:finitespaces})}
%\label{sec:proofs-for-finite-spaces}
%%auto-ignore
\begin{proof}[Proof of Lemma \ref{lem:lwrewritefinite}]
Given observations $x_{1:n} \in \cX^n$, we can partition the index set $[n]$ into a disjoint collection of subsets $\{J_x\}_{x \in X}$, where $J_{x} \doteq \{i \in [n]| x_i = x\}$ denotes the set of indices that realize the value $x \in X$;  further define $\hn_x \doteq |J_x|$ for each $x \in X$. In these terms, we have $\hat{s}_i = \hn_{x_i}$, and the matrix $A$ is equal to  
\md{fix notation; A is no longer defined}
\begin{equation*}
    A_{ij} = \begin{cases}
        1/\hn_x & \text{ if } i, j \in J_x \text{ for some } x \in X\\
            0 & \text { otherwise.}
    \end{cases}
\end{equation*}
Note that $A$ is a block diagonal matrix in terms of index sets $\{J_x\}_{x \in X}$, with constant value in each block. Recall that the subset $\hat{\Delta}_n$ consist of weight vectors $w \in \Delta_n$ that, for every pair $i,j \in [n]$, satisfy the constraint $w_i = w_j$ if $x_i = x_j$. This shows that $\hat{\Delta}_n$ can be identified with the set $\Delta_{\hX_n} \doteq \{q \in [0,1]^{\hX} | \sum_{x \in \hX} q_x = 1\}$, where $\hX_n = \{x \in \hX_n | \hn_x > 0\}$, via the bijective relation $w \in \hat{\Delta}_n$ if and only if $w_i = q_{x_i}/\hn_{x_i}$ for some $q \in \Delta_{\hX_n}$. The proof is then completed by rewriting \eqref{eq:lwlike} in terms of $q \in \Delta_{\hX_n}$.
\end{proof}

We now recall the following bound on KL-divergences.
\begin{lemma}
\label{lem:KL-inequality}
Let $p, q, p^\theta$ be probability vectors with support contained in $S \subseteq \Xcal$ such that $\| p -q \|_1 \leq 1/2$. Then
\[ | \KL_S(p | p_\theta) -  \KL_S(q | p_\theta)| 
\leq \| p - q \|_1 \log \frac{|S|}{\| p -q \|_1} + \sqrt{|S|} \| p - q \|_2  \max_{ x \in S} \left| \log p^\theta(x)  \right|.  \]
\end{lemma}
\begin{proof}
By the triangle inequality, we have
\[ | \KL_S(p | p_\theta) -  \KL_S(q | p_\theta)| \leq  \left| \sum_{x \in S} p_x \log \frac{1}{p_x} - \sum_{x \in S} q_x \log \frac{1}{q_x}  \right| + \left| \sum_{x \in S} (p_x - q_x) \log p_\theta(x) \right|. \]
By known bounds on the entropy function \citep[c.f. Theorem 17.3.3]{cover2006elements}, the first term on the right is bounded by $\| p - q \|_1 \log \frac{|S|}{\| p -q \|_1}$. By Cauchy-Schwarz, we can bound the second term via 
\[  \left| \sum_{x \in S} (p_x - q_x) \log p_\theta(x) \right| 
\leq \sqrt{ \left( \sum_{x \in S} (p_x - q_x)^2 \right) \left( \sum_{x \in S} \log(p^\theta(x))^2 \right) } \leq \| p - q \|_2   \max_{ x \in S} \left| \log p^\theta(x)  \right|. \]
\end{proof}

With Lemma~\ref{lem:KL-inequality} in hand, we can now prove Lemma~\ref{lem:converges-to-rate-function}.
\begin{proof}[Proof of Lemma~\ref{lem:converges-to-rate-function}]
By assumption on the existence of $q^\star$, we have $I_\epsilon(\theta) < \infty$. Let $S = \supp(p^\theta) \subseteq \Xcal$, and let $\alpha = \min_{x \in S} p^\theta(x) > 0$. Observe that any $q \in \Delta_{\Xcal}$ that achieves $\KL_{\Xcal}(q | p^\theta) < \infty$ must satisfy $q_x = 0$ for all $x \not \in S$. Thus, we may rewrite our objectives as
\begin{align*}
I_\epsilon(\theta) &=  - \inf_{\substack{q \in \Delta_{S}\\ \D(q, p^0) \leq \epsilon}} \KL_{S}(q|p^\theta) \\
\epsll &= - n \inf_{\substack{q \in \Delta_{\hX_n \cap S} \\ \D(q, \hat{p}^0) \leq \epsilon}} \KL_{S}(q|p^\theta) .
\end{align*}
Now pick any $\delta > 0$ that is small enough, and let $q_I \in \Delta_{\Xcal}$ satisfy $\D(q_I, p^0) \leq \epsilon$, $\supp(q_I) \subseteq S$, and
\[ \KL_{S}(q_I|p^\theta) \leq \inf_{\substack{q \in \Delta_{S}\\ \D(q, p^0) \leq \epsilon}} \KL_{S}(q|p^\theta) + \delta.\] 
Let $E_n$ denote the event that $S \subseteq \hX_n$ and $\| \hat{p}^0 - p^0 \|_2 \leq \delta$. Observe that $\lim_{n \rightarrow \infty} \prob(E_n) = 1.$ 

Pick any $n > 0$ and condition on $E_n$ occurring. We will first show the upper bound
\[  \inf_{\substack{q \in \Delta_{S} \\ \D(q, \hat{p}^0) \leq \epsilon}} \KL_{S}(q|p^\theta) \leq \inf_{\substack{q \in \Delta_{S}\\ \D(q, p^0) \leq \epsilon}} \KL_{S}(q|p^\theta) + 2 \delta. \]
To do so, let $q = h q^\star + (1-h)q_I$ for some $h \in [0,1]$ that is small enough. By assumption, $\D(q, p^0) < \epsilon$ and $\supp(q) \subseteq S$. Thus, for $\delta$ small enough, we will have 
\[ \D(q_I, \hat{p}^0) \leq \D(q_I, p^0) + \D(\hat{p}^0,  p^0) \leq \epsilon. \]
Moreover, observe that $\| q - q_I \|_1 \leq 2h$. By Lemma~\ref{lem:KL-inequality}, we have
\begin{align*}
\inf_{\substack{q \in \Delta_{\hX_n \cap S} \\ \D(q, \hat{p}^0) \leq \epsilon}} \KL_{S}(q|p^\theta)
&\leq \KL_S(q | p_\theta) \\
&\leq \KL_S(q_I | p_\theta) + h \log \frac{|\Xcal|}{h} + \sqrt{|\Xcal|} h \log \frac{1}{\alpha} \\
&\leq  \inf_{\substack{q \in \Delta_{S}\\ \D(q, p^0) \leq \epsilon}} \KL_{S}(q|p^\theta) + 2 \delta,
\end{align*}
where the last two inequalities follow by taking $h$ sufficiently small.

Now we will show the lower bound
\[  \inf_{\substack{q \in \Delta_{S} \\ \D(q, \hat{p}^0) \leq \epsilon}} \KL_{S}(q|p^\theta) \geq \inf_{\substack{q \in \Delta_{S}\\ \D(q, p^0) \leq \epsilon}} \KL_{S}(q|p^\theta) -  2\delta. \]
To do so, take any $\tilde{q} \in \Delta_{\Xcal}$ satisfying $\D(\tilde{q}, \hat{p}^0) \leq \epsilon$ and
\[ KL(\tilde{q} | p_\theta) \leq  \inf_{\substack{q \in \Delta_{S} \\ \D(q, \hat{p}^0) \leq \epsilon}} \KL_{S}(q|p^\theta) + \delta. \]
Let $q' = h q^\star + (1-h)q_I$ for some $h \in [0,1]$ that is small enough. By assumption, $\supp(q') \subseteq S$ and, if $\delta$ is small enough, $\D(q', p^0) \leq \epsilon$. As above, we have $\| q' - \tilde{q} \|_1 \leq 2h$. Thus, using similar arguments as above, we have for small enough $h$,
\begin{align}
\inf_{\substack{q \in \Delta_{S} \\ \D(q, \hat{p}^0) \leq \epsilon}} \KL_{S}(q|p^\theta) 
&\geq \KL(\tilde{q} | p_\theta) - \delta \\
&\geq \KL(q' | p_\theta) - 2 \delta \\
&\geq \inf_{\substack{q \in \Delta_{S}\\ \D(q, p^0) \leq \epsilon}} \KL_{S}(q|p^\theta) -  2\delta.
\end{align}
Putting it all together, we have that for any $\delta > 0$,
\[ \lim_{n \rightarrow \infty} \prob\left( \left|\frac{1}{n} \epsll -  I_\epsilon(\theta) \right| > \delta \right)  = 0. \qedhere \]
\end{proof}


\begin{proof}[Proof of Corollary~\ref{cor:cposterior-asymptotics}]
For now, let $x_1, x_2, \ldots \in \Xcal$ be an arbitrary sequence. We define the intermediate object
\[ M_{n,\epsilon}(\theta) := \prob_\theta\left(\D(\EmpDist{X_{1:n}}, p^0) \leq \epsilon\right). \]
\md{I changed notation since $M_{n}(\theta|x_{1:n})$ didn't depend on $x_{1:n}$.}
Let $\Gamma = \{ q \in \Delta_{\Xcal} \, | \, \D(q, p^0) \leq \epsilon  \}$. Our continuity conditions on $\D$ \md{I think you need Lemma \ref{lem:continuity} or such..} ensure that $\Gamma$ meets the conditions of Sanov's Theorem. Thus, we have
\[ \lim_{n\rightarrow \infty} \frac{1}{n} \log M_{n,\epsilon}(\theta) 
= \lim_{n\rightarrow \infty} \frac{1}{n} \log  \prob_\theta\left( \EmpDist{X_{1:n}} \in \Gamma \right) 
= - \inf_{q \in \Gamma} \KL_\Xcal(q | p^\theta) = I_\epsilon(\theta).  \]
Now let $x_1, x_2, \ldots \sim p^0$. Pick any $\delta >0$ sufficiently small, and let $E_n$ denote the event that $\D(p^0, \hat{p}^0) < \delta$. In the event of $E_n$, we have
\begin{align*}
\left| \log M_\epsilon(\theta | x_{1:n}) - \log L_{\epsilon}(\theta | x_{1:n})  \right| 
&= \max \left\{  \log \frac{\prob_\theta\left(\D(\EmpDist{X_{1:n}}, p^0) \leq \epsilon\right)}{\prob_\theta\left(\D(\EmpDist{X_{1:n}}, \hat{p}^0) \leq \epsilon\right) },  \log \frac{\prob_\theta\left(\D(\EmpDist{X_{1:n}}, \hat{p}^0) \leq \epsilon\right) }{\prob_\theta\left(\D(\EmpDist{X_{1:n}}, p^0) \leq \epsilon\right)} \right\} \\
&\leq \max \left\{ \log \frac{\prob_\theta\left(\D(\EmpDist{X_{1:n}}, p^0) \leq \epsilon\right)}{\prob_\theta\left(\D(\EmpDist{X_{1:n}}, p^0) \leq \epsilon - \delta \right) },  \log \frac{\prob_\theta\left(\D(\EmpDist{X_{1:n}}, {p}^0) \leq \epsilon +\delta \right) }{\prob_\theta\left(\D(\EmpDist{X_{1:n}}, p^0) \leq \epsilon\right)} \right\}.
\end{align*}
We will focus on bounding the second term of the last line. The argument for the first is identical. Observe by Sanov's Theorem, we have
\begin{align*}
\lim_{n\rightarrow \infty} \frac{1}{n} \log  \prob_\theta\left( \D(\EmpDist{X_{1:n}}, {p}^0) \leq \epsilon +\delta \right) &= \inf_{\substack{q \in \Delta_{\Xcal} \\ \D(q, {p}^0) = \epsilon + \delta}} \KL_{\Xcal}( q | p^\theta)  \\
\lim_{n\rightarrow \infty} \frac{1}{n} \log  \prob_\theta\left( \D(\EmpDist{X_{1:n}}, {p}^0) \leq \epsilon \right) &= \inf_{\substack{q \in \Delta_{\Xcal} \\ \D(q, {p}^0) \leq \epsilon }} \KL_{\Xcal}( q | p^\theta) .
\end{align*}

\md{Awesome, I like the above step. It seems below that you are using a result in this and the previous lemma that shows that the rate function is continuous in some sense. Can you package these arguments into a lemma and reuse them?}

Let $N \geq 0$ be such that for all $n \geq N$, we have that the above two sequences are within $\delta$ of their limits. Now take $n \geq N$ and suppose that $E_n$ holds. Now we trivially have
\[ \inf_{\substack{q \in \Delta_{S} \\ \D(q, {p}^0) \leq \epsilon + \delta}} \KL_\Xcal( q | p^\theta) \leq \inf_{\substack{q \in \Delta_{\Xcal} \\ \D(q, {p}^0) \leq \epsilon }} \KL_\Xcal( q | p^\theta). \]
Let $q_I, q'_I \in \Delta_{\Xcal}$ satisfy $\D(q_I, p^0) \leq \epsilon + \delta$, $\D(q_I, p^0) < \epsilon$ and
\begin{align*}
\KL_{\Xcal}( q_I | p^\theta)  & \leq  \inf_{\substack{q \in \Delta_{\Xcal} \\ \D(q, {p}^0) \leq \epsilon + \delta}} \KL_{\Xcal}( q | p^\theta) + \delta  \\
\KL_{\Xcal}( q'_I | p^\theta)  & < \infty.
\end{align*}
Let $q = h q'_I + (1-h) q_I$ for some small $h \in (0,1)$ (chosen relative to $\delta$). For appropriately chosen $h$, we have $d(q, p^0) \leq \epsilon$ and $\| q - q_I \|_1 \leq 2h$. Applying Lemma~\ref{lem:KL-inequality}, we then have
\begin{align*}
\inf_{\substack{q \in \Delta_{\Xcal} \\ \D(q, {p}^0) \leq \epsilon }} \KL_{\Xcal}( q | p^\theta) &\leq \KL_{\Xcal}( q | p^\theta) \\
&\leq \KL(q_I | p^\theta) + \delta_h \\
&\leq \inf_{\substack{q \in \Delta_{\Xcal} \\ \D(q, {p}^0) \leq \epsilon + \delta}} \KL_{\Xcal}( q | p^\theta) + \delta + \delta_h.
\end{align*}
Where $\delta_h$ is some function of $h$ that goes to 0 as $\delta$ (and hence $h)$ goes to 0. Putting it all together, 
\begin{align*}
\frac{1}{n} \log  \prob_\theta\left( \D(\EmpDist{X_{1:n}}, {p}^0) \leq \epsilon +\delta \right) -  \frac{1}{n} \log  \prob_\theta\left( \D(\EmpDist{X_{1:n}}, {p}^0) \leq \epsilon \right) 
&\leq 3 \delta + \delta_h.
\end{align*}
Applying the same arguments shows us
\begin{align*}
\frac{1}{n} \log  \prob_\theta\left( \D(\EmpDist{X_{1:n}}, {p}^0) \leq \epsilon  \right) -  \frac{1}{n} \log  \prob_\theta\left( \D(\EmpDist{X_{1:n}}, {p}^0) \leq \epsilon - \delta \right) 
&\leq 3 \delta + \delta_h.
\end{align*}
Thus,
\begin{align*}
 \left| \frac{1}{n} \log L_{\epsilon}(\theta | x_{1:n}) - I_\epsilon(\theta) \right| 
 &\leq \left| \frac{1}{n} \log L_{\epsilon}(\theta | x_{1:n}) - \frac{1}{n} \log M_\epsilon(\theta | x_{1:n})  \right| +  \left| \frac{1}{n} \log M_{\epsilon}(\theta | x_{1:n}) - I_\epsilon(\theta) \right| \\
 &\leq 4 \delta + \delta_h.
\end{align*}
This suffices to show 
\[ \frac{1}{n} \log L_\epsilon(\theta|x_{1:n}) \pconv I_\epsilon(\theta) \quad \text{ as } n \to \infty.\]
Now we apply Lemma~\ref{lem:converges-to-rate-function}, which showed
\[ \frac{1}{n}\epsll \pconv I_\epsilon(\theta) \quad \text{ as } n \to \infty,  \]
to conclude that 
\[ \frac{1}{n} \left[\log L_\epsilon(\theta|x_{1:n}) -  \epsll\right] \pconv 0 \quad \text{ as } n \to \infty. \qedhere \]
\end{proof}

\md{Next lemma needs changing notation to $F(r) = I_r$. Also the condition $\KL(p^0|p^\theta) < \infty$ and Assumption \ref{ass:dist-finite} might be too strong as the above lemmas point out.}
\begin{lemma} Suppose $\D$ satisfies assumption \ref{ass:dist-finite} and $\theta \in \Theta$ is such that $\KL(p^0|p^\theta) < \infty$. Then the function $F: [0,\infty) \to [0,\infty)$ given by $F(r) = -I_r(\theta)$ is continuous. (Note that $F(r) \leq \KL(p^0|p^\theta) < \infty$ for each $r \geq 0$.)
\label{lem:continuity}
\end{lemma}
\begin{proof} 
 Using our continuity assumption on $\D$, let us first show that the optimization problem in $I_r$ attains its minimum value at some $q_r \in A_r \doteq \{ q \in \Delta_{\cX} \mid \D(q, p^0) \leq r \}$. 
Indeed, this follows since $A_r$ is a compact subset and the function $q \mapsto \KL_{\cX}(q|p^\theta)$ is lower semi-continuous.

Let us now show that $\liminf_{h \to 0} F(r_0+h) \geq F(r_0)$ for any $r_0 \in [0,\infty)$ (with the convention that $F(r)=\infty$ when $r < 0$). Indeed, for any sequence $\{h_n\}_{n \in \nat}$ that is converging to zero, the sequence $\{q_{r_0 + h_{h_n}}\}_{n \in \nat} \subseteq \Delta_{\cX}$ is pre-compact. Hence, there is an increasing subsequence $\{n_k\}_{k \in \nat} \subseteq \nat$ and $q_* \in \Delta_n$ such that $\lim_{k \to \infty} q_{r_0 + h_{n_k}} = q_*$. Note then by the continuity of $\D$, that $\D(q_*, p_0) = \lim_{k \to \infty} \D(q_{r_0 + h_{n_k}}, p_0) \leq \limsup_{k \to \infty} r_0 + h_{n_k} = r_0$. Hence, the lower semi-continuity of the $\KL$-divergence shows that $$
\liminf_{k \to \infty} F(r_0 + h_{n_k}) = \liminf_{k \to \infty} \KL_{\cX}(q^{r_0 + h_{n_k}}|p^\theta) \geq \KL_{\cX}(q_*|p^\theta) \geq F(r_0).
$$

Note that $F$ is a non-decreasing function, i.e. $F(r) \leq F(s)$ whenever $s \geq r$. Hence, the result $\liminf_{h \downarrow 0} F(r_0 + h) \geq F(r_0) \geq \limsup_{h \downarrow 0} F(r_0 + h)$ immediately shows the right continuity of $F$, i.e. $F(r_0) = \lim_{h \downarrow 0} F(r_0 + h)$.
%

Now we shall establish the left-continuity of $F$ at some point $r=r_0 > 0$. If $\D(q_{r_0},p^0) < r_0$, then $F(r) = F(r_0)$ for each $r \in [\D(q_{r_0}, p^0), r_0)$ and the left continuity is easily satisfied. Hence, suppose from now on that $\D(q_{r_0},p^0) = r_0$. Next, for any $h \in [0,1]$, denote by $q'_h \doteq (1-h) q_{r_0} + h p^0$ the convex combination between $q_{r_0}$ and $p^0$. By the convexity of KL-divergence
\begin{equation*}
    F(\D(q'_h, p^0)) \leq \KL_{\cX}(q'_h|p^\theta) \leq (1-h)\KL_{\cX}(q_{r_0}|p^\theta) + h \KL_{\cX}(p^0|p^\theta) = (1-h) F(r_0) + h F(0).
\end{equation*} 
 
Take $h \downarrow 0$ to obtain $\limsup_{h \to 0} F(T(h)) \leq F(r_0)$, where $T(h) \doteq \D(\tilde{q}^h, p^0)$. By our assumption, $T: [0,1] \to [0,r_0]$ is a continuous and strictly decreasing function. Hence $T(h)$ is stricly increasing to $T(0)=r_0$  as $h \downarrow 0$. Thus, we have in-fact shown that $\limsup_{h \downarrow 0} F(r_0-h) = F(r_0)$. Finally, monotonicity of $F$ shows $\liminf_{h \downarrow 0} F(r_0 - h) \geq F(r_0)$, and hence we recover the left continuity of $F$.
\end{proof}


\iffalse
\begin{proof}[Proof of Lemma \ref{lem:lwrewritefinite}]
Given observations $x_{1:n} \in \cX^n$, we can partition the index set $[n]$ into a disjoint collection of subsets $\{J_x\}_{x \in X}$, where $J_{x} \doteq \{i \in [n]| x_i = x\}$ denotes the set of indices that realize the value $x \in X$;  further define $\hn_x \doteq |J_x|$ for each $x \in X$. In these terms, we have $\hat{s}_i = \hn_{x_i}$, and the matrix $A$ is equal to
\begin{equation*}
    A_{ij} = \begin{cases}
        1/\hn_x & \text{ if } i, j \in J_x \text{ for some } x \in X\\
            0 & \text { otherwise.}
    \end{cases}
\end{equation*}
Note that $A$ is a block diagonal matrix in terms of index sets $\{J_x\}_{x \in X}$, with constant value in each block. Recall that the subset $\hat{\Delta}_n$ consist of weight vectors $w \in \Delta_n$ that, for every pair $i,j \in [n]$, satisfy the constraint $w_i = w_j$ if $x_i = x_j$. This shows that $\hat{\Delta}_n$ can be identified with the set $\Delta_{\hX_n} \doteq \{q \in [0,1]^{\hX} | \sum_{x \in \hX} q_x = 1\}$, where $\hX_n = \{x \in \hX_n | \hn_x > 0\}$, via the bijective relation $w \in \hat{\Delta}_n$ if and only if $w_i = q_{x_i}/\hn_{x_i}$ for some $q \in \Delta_{\hX_n}$. The proof is then completed by rewriting \eqref{eq:lwlike} in terms of $q \in \Delta_{\hX_n}$.
\end{proof}

\fi


%\section{Proofs from Section~\ref{sec:comp}}
%\label{app:comp-proofs}

\iffalse
\subsection{Proof of Proposition~\ref{prop:tv-mixture-model}}

We first prove the following simple lemma.

\begin{lemma}
\label{lem:tv-decomposition}
Let $\epsilon > 0$, and let $P, Q$ be two measures such that for all measurable sets $A$, we have $Q(A) \geq (1-\epsilon)P(A)$. Then $\tv(P, Q) \leq \epsilon$ if and only if there exists a probability measure $R$ such that $Q = (1-\epsilon)P + \epsilon R$.
\end{lemma}
\begin{proof}
First assume there exists a probability measure $R$ such that $Q = (1-\epsilon)P + \epsilon R$. For any measurable set $A$, we have
\[ |Q(A) - P(A)| = |(1- \epsilon)P(A) + \epsilon R(A) - P(A)| = \epsilon |R(A) - P(A)| \leq \epsilon, \]
where the last line follows from the fact that $P(A), R(A) \in [0,1]$. By the definition of total variation distance, we have $\tv(P, Q) \leq \epsilon$.

Now assume that $\tv(P, Q) \leq \epsilon$. By assumption, we have
$Q(A) \geq (1-\epsilon) P(A)$
for any measurable subset $A$. Thus, we can define the measure $R = \frac{1}{\epsilon}\left( Q - (1- \epsilon)P \right)$, and observe that $R$ is a probability measure. Thus,
\[  (1- \epsilon)P + \epsilon R = (1-\epsilon)P + Q - (1-\epsilon)P = Q.  \]
\end{proof}

We now turn to the proof of Proposition~\ref{prop:tv-mixture-model}. First, assume that there exist measures $Q_k$ such that $Q = \sum_k \pi_k P_k$ and $\tv(P_k, Q_k) \leq \epsilon$ for all $k=1,\ldots, K$. Then for any measurable set $A$, we have
\[ |Q(A) - P(A)| = |\sum_{k} \pi_k(Q_k(A) - P_k(A))| \leq \sum_k \pi_k |Q_k(A) - P_k(A)| \leq \epsilon. \]
By the definition of total variation distance, this implies $\tv(P, Q) \leq \epsilon$.

Now assume that $\tv(P, Q) \leq \epsilon$. By Lemma~\ref{lem:tv-decomposition}, this implies the existence of a probability measure $R$ such that $Q = (1-\epsilon)P + \epsilon R$. Now define the mixture probability measures $Q_k = (1-\epsilon)P_k + \epsilon R$. We first observe that
\[ \sum_k \pi_k Q_k =  \sum_k \pi_k ((1-\epsilon)P_k + \epsilon R ) = (1-\epsilon) P_k + \epsilon R = Q. \]
Finally, by Lemma~\ref{lem:tv-decomposition}, we have $\tv(P_k, Q_k) \leq \epsilon$. \qed
\fi

\end{document}
