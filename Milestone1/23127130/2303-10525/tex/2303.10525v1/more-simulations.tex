%auto-ignore
In this section, we present the simulation results for the setting in which the data points to be corrupted are chosen entirely at random. For each of the problem settings, all of the details remain exactly the same as in the max-likelihood corruption simulations, with the only change coming from which data points are chosen to be corrupted.

\begin{figure}
    \centering
    \includegraphics[width=0.9\textwidth]{figures/gaussian_wmle_rand_corruption.pdf}
    \caption{Gaussian results for random corruptions. Dashed black line denotes average performance of MLE on full uncorrupted dataset. Shaded regions denote 95\% confidence intervals over 50 random seeds.}
    \label{fig:rand-gaussian}
\end{figure}
\Cref{fig:rand-gaussian} presents the random corruption results for estimating a multivariate normal distribution. In contrast with the max-likelihood corruption setting, we see that OWL without kernelization outperforms OWL with kernelization in higher dimensions. It is possible that this is due to the difficulty of density estimation in higher dimensions.

\begin{figure}
    \centering
    \includegraphics[width=0.9\textwidth]{figures/lin_rand_corruption.pdf}
    \caption{Linear regression results for random corruptions. Dashed black line denotes average performance of MLE on full uncorrupted training set. Shaded regions denote 95\% confidence intervals over 50 random seeds.}
    \label{fig:rand-linear-regression}
\end{figure}
\Cref{fig:rand-linear-regression} presents the random corruption results for linear regression. The results here are qualitatively similar to those for the max-likelihood corruption setting, with all three robust methods performing well in the simulated data setting but with RANSAC performing notably worse with QSAR data.


\begin{figure}
    \centering
    \includegraphics[width=1.0\textwidth]{figures/log_rand_corruption.pdf}
    \caption{Logistic regression results for random corruptions. Dashed black line denotes average performance of MLE on full uncorrupted training set. Shaded regions denote 95\% confidence intervals over 50 random seeds.}
    \label{fig:rand-logistic-regression}
\end{figure}
\Cref{fig:rand-logistic-regression} presents the random corruption results for logistic regression. As in the max-likelihood corruption case, we see that OWL outperforms the other methods across all three datasets. Moreover, we see that on the Enron spam dataset, OWL even outperforms the uncorrupted MLE baseline, which is entirely possible if the logistic regression model is mis-specified.

\begin{figure}
    \centering
    \includegraphics[width=0.9\textwidth]{figures/clustering_max_corruption.pdf}
    \caption{Mixture model results for random corruptions. Dashed black line denotes average performance of MLE on full uncorrupted training set. Shaded regions denote 95\% confidence intervals over 50 random seeds.}
    \label{fig:rand-both-mixture}
\end{figure}
\Cref{fig:rand-both-mixture} presents the random corruption results for the mixture model settings. The results here are qualitatively similar to those for the max-likelihood corruption setting.