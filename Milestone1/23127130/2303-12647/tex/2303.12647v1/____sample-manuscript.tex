% To look at:
% - Prompt as design materials
% - Do you buy the arguments and the Introduction?

%%
%% This is file `sample-manuscript.tex',
%% generated with the docstrip utility.
%%
%% The original source files were:
%%
%% samples.dtx  (with options: `manuscript')
%% 
%% IMPORTANT NOTICE:
%% 
%% For the copyright see the source file.
%% 
%% Any modified versions of this file must be renamed
%% with new filenames distinct from sample-manuscript.tex.
%% 
%% For distribution of the original source see the terms
%% for copying and modification in the file samples.dtx.
%% 
%% This generated file may be distributed as long as the
%% original source files, as listed above, are part of the
%% same distribution. (The sources need not necessarily be
%% in the same archive or directory.)
%%
%%
%% Commands for TeXCount
%TC:macro \cite [option:text,text]
%TC:macro \citep [option:text,text]
%TC:macro \citet [option:text,text]
%TC:envir table 0 1
%TC:envir table* 0 1
%TC:envir tabular [ignore] word
%TC:envir displaymath 0 word
%TC:envir math 0 word
%TC:envir comment 0 0
%%
%%
%% The first command in your LaTeX source must be the \documentclass command.
% \documentclass[manuscript,screen,review,table,xcdraw,anonymous]{acmart}
\documentclass[acmlarge,screen,table,xcdraw,authorversion]{acmart}
\newcommand{\imagepath}[1]{images/cropped/#1}
\newcommand{\figpath}[1]{figures/#1}


\usepackage{multirow}
% \usepackage[table,xcdraw]{xcolor}
% \usepackage{colortbl}
% \usepackage[utf8]{inputenc}
% \usepackage{etoolbox}
% \usepackage{makecell} % new line in cells: https://tex.stackexchange.com/a/176780
% \usepackage{color}
% \usepackage[utf8]{inputenc}
% \usepackage{pifont}
% \usepackage{xspace}
% \usepackage{graphicx}

% Command that generates the path of a given image. This allows for easily switching all images in the paper between cropped and uncropped versions.

\newcommand{\imagen}{Envisage} %Anonymized for review
\newcommand{\Imagen}{\imagen{}}
\newcommand{\IS}{Image Search}

% Participants


\newcommand{\todo}[1]{{\color{red}#1}}

\newcommand{\prompt}[1]{\textquotedbl\texttt{#1}\textquotedbl}

\usepackage[T1]{fontenc}
\usepackage{lscape}

\usepackage{xparse}
\usepackage{enumitem}
\setlist[description]{
  font={\sffamily\bfseries},
  labelsep=5pt,
  labelindent=20pt,
  labelwidth=\transcriptlen,
  leftmargin=\transcriptlen,
}

\newlength{\transcriptlen}

\NewDocumentCommand {\setspeaker} { mo } {%
  \IfNoValueTF{#2}
  {\expandafter\newcommand\csname#1\endcsname{\item[#1:]}}%
  {\expandafter\newcommand\csname#1\endcsname{\item[#2:]}}%
  \IfNoValueTF{#2}
  {\settowidth{\transcriptlen}{#1}}%
  {\settowidth{\transcriptlen}{#2}}%
}

% Easiest to put the longest name last...
\settowidth{\transcriptlen}{Hendrik}
% How much of a gap between speakers and text?
\addtolength{\transcriptlen}{0.1em}%


\usepackage{caption}
\usepackage{subcaption}

%%
%% \BibTeX command to typeset BibTeX logo in the docs
\AtBeginDocument{%
  \providecommand\BibTeX{{%
    Bib\TeX}}}

%% Rights management information.  This information is sent to you
%% when you complete the rights form.  These commands have SAMPLE
%% values in them; it is your responsibility as an author to replace
%% the commands and values with those provided to you when you
%% complete the rights form.
\setcopyright{acmcopyright}
\copyrightyear{2022}
\acmYear{2022}
\acmDOI{XXXXXXX.XXXXXXX}

%% These commands are for a PROCEEDINGS abstract or paper.
\acmConference[Conference acronym 'XX]{Make sure to enter the correct
  conference title from your rights confirmation email}{June 03--05,
  2018}{Woodstock, NY}
\acmPrice{15.00}
\acmISBN{978-1-4503-XXXX-X/18/06}

\usepackage{soul}
% \usepackage{xcolor}
% \usepackage[table,xcdraw]{xcolor}

% %% Revision highlight formatting
% \definecolor{highlighter}{HTML}{fff100}
% \sethlcolor{highlighter}
% \newcommand{\revision}[1]{\hl{#1}} % with highlight

\newcommand{\revision}[1]{#1}    % without

% \newcommand{\revision}[1]{\color{blue}{#1}}

% \newcommand{\re}[1]{{\color{blue}{#1}}}  % with highlight
% \newcommand{\r}[1]{#1}    % without
% \newcommand{\re}[1]{{{#1}}}  % with highlight
%%
%% Submission ID.
%% Use this when submitting an article to a sponsored event. You'll
%% receive a unique submission ID from the organizers
%% of the event, and this ID should be used as the parameter to this command.
%%\acmSubmissionID{123-A56-BU3}

%%
%% For managing citations, it is recommended to use bibliography
%% files in BibTeX format.
%%
%% You can then either use BibTeX with the ACM-Reference-Format style,
%% or BibLaTeX with the acmnumeric or acmauthoryear sytles, that include
%% support for advanced citation of software artefact from the
%% biblatex-software package, also separately available on CTAN.
%%
%% Look at the sample-*-biblatex.tex files for templates showcasing
%% the biblatex styles.
%%

%%
%% The majority of ACM publications use numbered citations and
%% references.  The command \citestyle{authoryear} switches to the
%% "author year" style.
%%
%% If you are preparing content for an event
%% sponsored by ACM SIGGRAPH, you must use the "author year" style of
%% citations and references.
%% Uncommenting
%% the next command will enable that style.
%%\citestyle{acmauthoryear}



%%
%% end of the preamble, start of the body of the document source.
\begin{document}

%%
%% The "title" command has an optional parameter,
%% allowing the author to define a "short title" to be used in page headers.
\title[Prompts as AI Design Material]{A Word is Worth a Thousand Pictures: Prompts as AI Design Material}

%%
%% The "author" command and its associated commands are used to define
%% the authors and their affiliations.
%% Of note is the shared affiliation of the first two authors, and the
%% "authornote" and "authornotemark" commands
%% used to denote shared contribution to the research.

% \author{Anonymized}
\author{Chinmay Kulkarni}
\affiliation{%
  \institution{Google, Inc.}
%   \streetaddress{1 Th{\o}rv{\"a}ld Circle}
  \city{Atlanta, Georgia}
  \country{United States}}
\email{ckulkarni@google.com}

\author{Stefania Druga}
\affiliation{%
  \institution{Google, Inc.}
%   \streetaddress{1 Th{\o}rv{\"a}ld Circle}
  \city{Mountain View, California}
  \country{United States}}
\email{druga@google.com}



\author{Minsuk Chang}
\affiliation{%
  \institution{Google, Inc.}
%   \streetaddress{1 Th{\o}rv{\"a}ld Circle}
  \city{Seattle, Washington}
  \country{United States}}
\email{misukchang@google.com}


\author{Alex Fiannaca}
\affiliation{%
  \institution{Google, Inc.}
%   \streetaddress{1 Th{\o}rv{\"a}ld Circle}
  \city{Seattle, Washington}
  \country{United States}}
\email{afiannaca@google.com}



\author{Carrie Cai}
\affiliation{%
  \institution{Google, Inc.}
%   \streetaddress{1 Th{\o}rv{\"a}ld Circle}
  \city{Mountain View, California}
  \country{United States}}
\email{druga@google.com}


\author{Michael Terry}
\affiliation{%
  \institution{Google, Inc.}
%   \streetaddress{1 Th{\o}rv{\"a}ld Circle}
  \city{Seattle, Washington}
  \country{United States}}
\email{michaelterry@google.com}



% %%
% %% By default, the full list of authors will be used in the page
% %% headers. Often, this list is too long, and will overlap
% %% other information printed in the page headers. This command allows
% %% the author to define a more concise list
% %% of authors' names for this purpose.
\renewcommand{\shortauthors}{Kulkarni, et al.}
\newcommand{\LPIM}{text-to-image model}
\newcommand{\LPIMs}{text-to-image models}
% %%
% %% The abstract is a short summary of the work to be presented in the
%% article.
\begin{abstract}
Recent advances in Machine-Learning have led to the development of models that generate images based on a text description. Such large prompt-based text to image models (TTIs), trained on a considerable amount of data, allow the creation of high-quality images by users with no graphics or design training. This paper examines the role such TTI models can play in collaborative, goal-oriented design. Through a within-subjects study with 14 non-professional designers, we find that such models can help participants explore a design space rapidly and allow for fluid collaboration. We also find that text inputs to such models (“prompts”) act as reflective design material, facilitating exploration, iteration, and reflection in pair design. This work contributes to the future of collaborative design supported by generative AI by providing an account of how \LPIMs{} influence the design process and the social dynamics around design and suggesting implications for tool design.
%\todo{Abstract}%Recent advances in deep generative neural networks have made it possible for artificial intelligence to actively collaborate with human beings in co-creating novel content (e.g. music, art). While substantial research focuses on (individual) human-AI collaborations, comparatively less research examines how AI can play a role in human-human collaborations during co-creation. In a qualitative lab study, we observed 30 participants (15 pairs) compose a musical phrase in pairs, both with and without AI. Our findings reveal that AI may play important roles in influencing human social dynamics during creativity, including: 1) implicitly seeding a common ground at the start of collaboration, 2) acting as a psychological safety net in creative risk-taking, 3) providing a force for group progress, 4) mitigating interpersonal stalling and friction, and 5) altering users’ collaborative and creative roles. This work contributes to the future of generative AI in social creativity by providing implications for how AI could enrich, impede, or alter creative social dynamics in the years to come.

\end{abstract}

% %%
% %% The code below is generated by the tool at http://dl.acm.org/ccs.cfm.
% %% Please copy and paste the code instead of the example below.
% %%
% \begin{CCSXML}
% <ccs2012>
%  <concept>
%   <concept_id>10010520.10010553.10010562</concept_id>
%   <concept_desc>Computer systems organization~Embedded systems</concept_desc>
%   <concept_significance>500</concept_significance>
%  </concept>
%  <concept>
%   <concept_id>10010520.10010575.10010755</concept_id>
%   <concept_desc>Computer systems organization~Redundancy</concept_desc>
%   <concept_significance>300</concept_significance>
%  </concept>
%  <concept>
%   <concept_id>10010520.10010553.10010554</concept_id>
%   <concept_desc>Computer systems organization~Robotics</concept_desc>
%   <concept_significance>100</concept_significance>
%  </concept>
%  <concept>
%   <concept_id>10003033.10003083.10003095</concept_id>
%   <concept_desc>Networks~Network reliability</concept_desc>
%   <concept_significance>100</concept_significance>
%  </concept>
% </ccs2012>
% \end{CCSXML}

% \ccsdesc[500]{Computer systems organization~Embedded systems}
% \ccsdesc[300]{Computer systems organization~Redundancy}
% \ccsdesc{Computer systems organization~Robotics}
% \ccsdesc[100]{Networks~Network reliability}

% %%
% %% Keywords. The author(s) should pick words that accurately describe
% %% the work being presented. Separate the keywords with commas.
\keywords{text-to-image models, deep learning, creativity, design}
% \teaserfigure{\imagepath{design_process}}


% %%
% %% This command processes the author and affiliation and title
% %% information and builds the first part of the formatted document.

\begin{teaserfigure}
\includegraphics[width=\textwidth]{\figpath{design-process-with-images}}
\caption{Large prompt-based text-to-image models enable rapid exploration of a design space, and fluid collaboration. By allowing users to declaratively and quickly create images through text descriptions, these text prompts act as a \textit{reflective design material} aiding exploration and collaboration. We observe that designers create prompts based on their tacit understanding of the model, and model outputs in turn both spark new ideas and allow rapid refinement of prompts.}
\label{teaserfigure}
% ALT TEXT: A graph diagram of the creative process. On the left, designers generate ideas and create prompts for these ideas. In the middle, prompt-base image generation models generate images from these prompts (on the right). Based on the generated images, designers may iterate by refining their idea  or refining the current prompt. 
\end{teaserfigure}

\maketitle

\section{Introduction} 
Recent advances in  \LPIMs{} (TTI models), allow users to generate high-quality images based on a text description or ``prompt''. In ways similar to large language models (LLMs), where increasing size of models has discontinuous benefits~\cite{wei2022emergent}, the increased size of recent \LPIMs{} has yielded discontinuous and qualitative differences in the quality of images produced~\cite{saharia2022Imagen}. The quality of \LPIMs{} has led to energetic communities of practice, where enthusiasts readily share designs and prompts. In some cases, results obtained from these models are so good that one artist has even won a prestigious art competition \cite{AIwinsst11:online}. 

The image quality obtained by this new generation of \LPIMs{} enables the research community to rethink significant aspects of the design process in light of these models. For example, questions arise about whether designers could delegate parts of the creative process to a model; how the nature and the style of creative design change with these models; and how designers should best share and collaborate on multimodal (i.e., text and image) creative processes with these models available as design resources. Finally, potential new design processes among non-professional designers are particularly interesting, as TTI models allow those without professional training to create high-quality images easily.


% In particular, non-professional designers benefit from being able to rapidly prototype, see and borrow from examples, and share multiple designs \cite{dow2010parallel}. Since TTI models may empower those without professional artistic or design training to create images with a textual description alone rapidly, we are motivated to study how practices of non-professional designers might change. %Could be condensed and moved to section 3, (before 3.1) if we want to keep this. CK NOTE: Added to section 3.


This paper investigates how the creative design processes of non-professional designers change while using TTI models. Since design practices often emerge through social interactions (e.g., designers working together in a studio or sharing their work online), we also investigate how \textit{social interactions} are affected with the introduction of a TTI model. Recent work suggests that generative AI may play a role in influencing social dynamics between pairs of people~\cite{suh2021GenerativeMusic}. With the advent of ``prompting'' as a new form of interacting with AI, we build on this emerging body of literature by investigating how prompting may affect social dynamics during collaborative design. 


Our paper examines the following two research questions: 
\begin{itemize}
    \item RQ1: How does using prompt-based image generation change the design process of non-professional designers, especially compared to current tools for finding appropriate images such as web search?
    \item RQ2: How do prompt-based image generation models change collaborative dynamics during design?
\end{itemize}


Note that while prior work on human-AI co-creative systems has uncovered challenges that frequently arise when users co-create with AI (e.g., users must deal with uncertainty in the capabilities of AI capabilities while simultaneously making sense of complex outputs from the system \cite{yang2020DesignAI}), it is unknown how these challenges manifest when users interact with prompt-based i.e., TTI models. It is also unclear what other challenges are unique to working with these models and how we might best address them when designing new interfaces for these systems. 


To address these research questions, we conducted a design study with participants from a large technology company who use prompt-based image models in a non-professional capacity. In our controlled study, participants worked in pairs and created graphic designs with or without the assistance of a prompt-based image generation model. We present results in this paper from our direct observations of the designer pairs, conversations during this design session, and post-study interviews. We also compare the artifacts produced by participants for creativity, completeness, and appropriateness to the design brief. 

Overall, we found that TTI models change the design process by allowing designers to create images \textit{declaratively}, i.e., through a simple description of the desired image. This declarative design changes existing design practices in two ways. First, because prompts can capture high-level image descriptions, TTI models allow faster exploration of the design space, potentially leading to more creative design. {(At the same time, we acknowledge that not all creative ideas for generating images can be expressed easily, or at all, in words. As new text-guided image-to-image models become more capable, we expect that they will aid exploration even further.)} Second, because prompts are text, prompt-based image generation leads to easier sharing of design ideas, allowing designers to collaborate and build on each others' work more successfully.  

Throughout our study, we observed that prompts played a central role in the design process and collaboration. This leads us to argue that prompts act as a \textit{reflective design material} in the design process. Specifically, we found that our participants developed a \textit{tacit}, rather than technical, understanding of how different aspects of prompts (such as specific keywords) influence the image generated. 

In addition, prompts enabled fluid collaborations between participants as they shared, modified, and iteratively improved each others' prompts. The ability to easily edit and refine an image via a text interface made the design process more fluid, and sharing prompts easily aided collaboration, which uniquely placed prompts as design material in a multimodal creative setting.

Prompts also allowed participants to engage in reflective practice with the AI model. Specifically, the \LPIM{} outputs made it seem like a somewhat opinionated ``design partner'' in its preference to generate certain kinds of images, or when the images generated were wildly different from the prompt's intent or were challenging to modify. Participants leveraged prompts to reflect on the model's results  
% In their ability to ``talk back'' to the designer, prompts for TTI models are also similar to Turkle's notion of  where the designer engages in an iterative process where they get to refine both their creative ideas and their ways to express them via prompts based on the model's response. 
and engaged in reflective conversations with each other through their prompts, envisioning novel designs they did not foresee~\cite{schon1984architectural,ghajargar2018thinking}. This leads us to suggest that prompts act as reflective design materials (``objects to think with'' \cite{turkle2005second}).

In sum, this paper makes the following contributions:
\begin{itemize}
 
\item {\textbf{Changes in the design process.} We articulate ways in which TTI models change the design process. Specifically, models changed design processes through faster iteration, creating images of novel ideas and unlikely combinations (e.g., a giraffe in a Lamborghini) quickly. They also changed the design process because they rely on \textit{indirect} manipulation of images (through text). Together, this led to wide-ranging exploration where the models rapidly ``filled in'' many unspecified details of the images.  However, participants also struggled to control low-level factors (e.g., cropping, position, text) in the generated image. In sum, users perceived their work to be significantly more creative when using a TTI model compared to image search. However, external raters found no significant difference. }

\item {\textbf{Changes in collaborative practices.} We identify ways prompts changed collaborative dynamics during design. While prompts empowered collaborators to combine multiple people's disparate ideas fluidly, their non-determinism hindered coordination. In addition, because participants frequently shared images without the prompts that generated them, partners had asymmetric access to TTI models.}

\item \textbf{Prompts as a design material.} {Given these empirical results, we  conceptualize prompts as a new design material that both enables rapid exploration of a design space and modulates collaboration, but one that is imperfect, especially for fine-grained control.} 
\end{itemize}
% Together, our results illustrate prompt-based image generation models' role in a collaborative design process. As other multimodal AI models are developed, such as image-to-image models whose outputs can be controlled with a text prompt, we expect that our conceptualizations of prompts as design materials and how they change design processes and collaboration will inform how future design tools could be designed and how users might interact with them. 



\section{Related Work}
\label{sec:relatedwork}

%%%%%%%%%%%%%%%%%%%%%%%%%% Outline %%%%%%%%%%%%%%%%%%%%%%%%%%%%%%%%%%%%%
%(1) Evasion Attacks
%(1.1) Surveys on evasion attacks and their relation to data properties - Michael
%(1.2) Individual papers that study non-data related reasons behind evasion attacks - Michael
%(1.3) Techniques related to evasion attacks and defenses (new) - Gabby
%(2) Non-Evasion Attacks (new), and - ???
%(3) Effects of training data on standard generalization - done 
%
%
%
%(1) Evasion Attacks
%(1.1) A number of surveys review literature on evasion attacks. - Michael
%Most of them do not focus specifically on properties of data but also discuss attack and defense mechanisms, non-data-related reasons for adversarial vulnarability, and  more. ~\jr{cite 4}.
%Yet, they these surveys mention data and its relation to evasion attacks. Specifically \jr{what they say about data.}
%The most close to ours is concurrent work by XXX + concrete facts that we have and they don't.
%
%(1.2) individual papers that study non-data related reasons behind evasion attacks, - Michael
%Literature identifies multiple reasons for adversarial vulnerability, in particular, for evasion attacks. 
%These include data-related properties extensively discussed in this survey, as well as reasons related to the models 		   themselves, computations resources, and feature representations. We discuss these below. 
%
%\jr{the rest is from the paper (non-data related reasons for adversarial vulnerability), with sections potentially renamed.}
%
%{\bf Model.}
%
%{\bf Computational Resources.}
%
%{\bf Robustness of Features.}
%
%(1.3) Techniques Related to Evasion Attacks and Defenses (new) - Gabby
%A number of works focus on techniques for generating evasion attacks, countermeasures against these attacks, 
%and defining the notion of the attack itself.   
%
%{\bf Attacks and Defense.}
%Here are the 5 remaining surveys + 1 additional paper for the reviewer.
%
%{\bf Adversarial Examples.}
%2 surveys lines 13 and 14 + 1 additional paper for the reviewer.
%
%(2) Non-Evasion Attacks (new) 
%Need to say that there are other type of attacks, define them, cite surveys (Bo's survey, maybe something else). 
%Only one work explicitly focus on effects of data. 
%
%
%(3) Effects of training data on standard generalization (done)

%%%%%%%%%%%%%%%%%%%%%%%%% Outline %%%%%%%%%%%%%%%%%%%%%%%%%%%%%%%%%%%%%


\revreplace{
We divide related work into three categories:
(1) surveys on adversarial robustness and its relation to data properties,
(2) surveys that discuss the influence of data properties on standard generalization, and
(3) individual papers that study non-data-related reasons for adversarial vulnerability.\\
}
{
This survey investigates properties of training data in the context of model robustness under evasion attacks. 
We start the discussion of related work by reviewing other surveys that focus on evasion attacks and 
include some discussion about data (Section~\ref{sec:relatedwork-surveys-data}).  
We then discuss non-data related reasons behind evasion attacks (Section~\ref{sec:relatedwork-not-data}),
as well as techniques related to evasion attacks and defenses (Section~\ref{sec:relatedwork-attacks}). 
Finally, we discuss data-related concerns for non-evasion attacks (Section~\ref{sec:relatedwork-poisoning}) and
the effects of training data on standard generalization (Section~\ref{sec:relatedwork-standard}).
}

%\vspace{-0.1in}
\subsection{Surveys on Evasion Attacks that Discuss Data}
\label{sec:relatedwork-surveys-data}
Numerous existing surveys 
\revreplace{focus on attack and defense techniques for adversarial robustness. 
%~\cite{Biggio:Roli:PR:2018,
%Rosenberg:Shabtai:Elovici:Rokach:CSUR:2021,
%Li:Li:Ye:Xu:CSUR:2021,
%Maiorca:Biggio:Giorgio:CSUR:2019,
%Demetrio:Coull:Biggio:Lagorio:Armando:Roli:ACMTPS:2021,
%Liu:Tantithamthavorn:Li:Liu:CSUR:2022,
%Liu:Nogueria:Fernandes:Kantarci:IEEECST:2022,
%Akhtar:Mian:IEEEAccess:2018,
%Akhtar:Mian:Kardan:Shah:IEEEAccess:2021,
%Serban:Poll:Visser:CSUR:2020,
%Machado:Silva:Goldschmidt:CSUR:2021,
%Zhang:Sheng:Alhazmi:Li:ACMTIST:2020}.
Only a few of these works mention the relationship between adversarial robustness and properties of the underlying data.} 
{review the literature on evasion attacks.
Most of these works do not focus specifically on properties of data but discuss attack and defense mechanisms, non-data-related reasons for adversarial vulnerability, 
and the different threat models. 
Only a few of these works mention data-related reasons for the existence of adversarial examples~\cite{Serban:Poll:Visser:CSUR:2020, Machado:Silva:Goldschmidt:CSUR:2021, Akhtar:Mian:Kardan:Shah:IEEEAccess:2021, Akhtar:Mian:IEEEAccess:2018}.
}
Specifically, Serban et al.~\cite{Serban:Poll:Visser:CSUR:2020} observe that adversarial vulnerability can be caused by an insufficient training sample size %~\cite{Schmidt:Santurkar:Tsipras:Talwar:Madry:NeurIPS:2018}
and high data dimensionality. %~\cite{Gilmer:Metz:Faghri:Schoenholz:Raghu:Wattenberg:Goodfellow:ICLR:2018}.
Similarly, Machado et al.~\cite{Machado:Silva:Goldschmidt:CSUR:2021} mention that the lack of sufficient training data, high dimensionality, 
and high concentration contribute to adversarial vulnerability.
\revadd{
Akhtar et al.~\cite{Akhtar:Mian:IEEEAccess:2018, Akhtar:Mian:Kardan:Shah:IEEEAccess:2021} also mention high dimensionality, along with other non-data-related reasons, 
as a source of adversarial examples.}

\revadd{A concurrent work by Han et al.~\cite{Han:Lin:Shen:Wang:Guan:CSUR:2023} (published at the end of April 2023) 
studies the origins of adversarial vulnerability in deep learning w.r.t. the model, data, and other perspectives.
The authors mention high dimensionality, distributions with high concentration, a small number of output classes, data imbalance, and the perceptual difference in image frequencies as potential sources of adversarial examples.
However, as (a) the focus of that survey is not on data-related properties in particular, 
(b) its paper search was conducted in 2021, and 
(c) it focuses on deep learning models only, 
our work was able to identify more than 50 additional relevant papers which focus on other types of models, 
e.g., non-parametric and linear classifiers, 
and/or discuss additional types of data-related properties, 
such as, types of distribution, class density, separation, and label quality.}
\revreplace{Yet, none of these surveys explicitly collect and analyze work that focuses on the effects of data properties
on adversarial robustness.}
{In summary, by explicitly focusing on the effects of data properties on evasion attacks in our survey, 
we are able to provide a more complete and detailed discussion on this topic, not covered in prior surveys.}

\vspace{-0.05in}
\subsection{Non-data-related Reasons Behind Evasion Attacks}
\label{sec:relatedwork-not-data}

%\vspace{-0.1in}
%\subsection{Non-data Related Reasons for Adversarial Vulnerability}

There has been a variety of hypotheses regarding the reasons behind adversarial vulnerability of ML systems, particularly for evasion attacks.
%\revreplace{
%In addition to the data used for training,  adversarial robustness could also depend on the choice of the model architecture,
%the training procedure, and the interplay between data and the learning algorithm, i.e., correspondence between the complexity of a model to that of the data.
%This section summarizes the key hypotheses regarding these aspects.
%%The hypotheses reviewed in this section are complementary to the potential influence from the data.
%}
These include data-related properties extensively discussed in this survey, as well as reasons related to the models themselves, 
computational resources, and feature learning procedures. We discuss these below.

%\jr{there is a lot of undefined terminology and jargon in this section.}

\vspace{0.02in}
\noindent
\textbf{Model.}
When Szegedy et al.~\cite{Szegedy:Zaremba:Sutskever:Bruna:Erhan:Goodfellow:Fergus:ICLR:2014} first discovered adversarial examples for visual models, they suspected that the high non-linearity of DNNs resulted in low probability `pockets' of adversarial examples in the learned representation manifold.
They hypothesize that while these pockets can be found through attack algorithms, the samples residing in these pockets have different distributions compared to normal samples and are thus subsequently harder to find when randomly sampling from the input space.
Instead, Goodfellow et al.~\cite{Goodfellow:Shlens:Szegedy:ICLR:2015} hypothesize that
the linearity from activation functions, like ReLU and sigmoid found in high-dimensional neural networks, induce vulnerability towards adversarial perturbations.
To support their claim, they present the attack method FGSM that exploits the linearity of the target classifier.
Fawzi et al.~\cite{Fawzi:Fawzi:Frossard:ICMLWorkshop:2015} also argue against the hypothesis of high non-linearity as the cause for adversarial examples.
They show that all classifiers are susceptible to adversarial attacks and claim that it is the low flexibility of the classifier compared to the complexity of the classification task that results in vulnerability.
The lack of consensus on the primary causes of model vulnerability invites more studies on this topic.

Singla et al.~\cite{Singla:Ge:Basri:Jacobs:NeurIPS:2021} show that enforcing invariance to circular shifts (e.g., rotation) in neural networks induces decision boundaries with a smaller margin than normal, fully connected networks,
which, in turn, reduces the adversarial robustness of the model.
Moosavi{-}Dezfooli et al.~\cite{Moosavi-Dezfooli:Fawzi:Fawzi:Frossard:Soatto:ICLR:2018} introduce universal,
input-agnostic perturbations to mislead the classifier and hypothesize that the vulnerability of a multi-class classifier to such perturbations is related to the shape of its decision boundaries, e.g.,
linear classifiers with decision boundaries that are parallel to each other and
nonlinear classifier with decision boundaries that are curved in a similar way
tend to be less robust as
perturbations in one direction can change the prediction label for a different class.

Tanay and Griffin~\cite{Tanay:Griffin:ArXiv:2016} conjecture that the decision boundary learned by the classifier being too close to (or `tilted towards') the data manifold instead of being perpendicular to it,
results in small perturbations being sufficient to move samples across the decision boundary for misclassification.
%data manifold refers to the underlying structure that the data exhibit

\vspace{0.02in}
\noindent
\textbf{Computational Resources.}
Bubeck et al.~\cite{Bubeck:Lee:Price:Razenshteyn:ICML:2019} use computational hardness theory to show that the time complexity for learning a robust model is exponential to the size of input data and thus is computationally intractable.
Hence, they attribute adversarial vulnerability to computational limitations of current learning algorithms.
Degwekar et al.~\cite{Degwekar:Nakkiran:Vaikuntanathan:COLT:2019} further extend this work and also show the impossibility of efficiently training robust classifiers.

%\subsubsection{Ineffective Learning Perspective}
\vspace{0.02in}
\noindent
\textbf{Feature Learning.}
Ilyas et al.~\cite{Ilyas:Santurkar:Tsipras:Engstrom:Tran:Madry:NeurIPS:2019} show that adversarial vulnerability can be a consequence of a model exploiting well-generalizing but non-robust features,
i.e., features that are spurious and sometimes incomprehensible to humans;
when constraining the model to use robust features, the adversarial robustness increases together with the
interpretability of the learned features.
However, Tsipras et al.~\cite{Tsipras:Santurkar:Engstrom:Turner:Madry:ICLR:2019} note that, as the features for achieving high accuracy may be different from the ones for achieving high robustness, robustness may be at odds with standard accuracy.
%
%\jr{why is it called Ineffective learning when it is about features.}\gx{I put it under ineffective learning as in this case, the model learns/decides the features for generalization, and when given the correct objective, the model in fact, can learn more robust features, so I think the underlying reason is objective we gave for the model didn't guide the model to learn the right features}
%
Instead of seeing adversarial vulnerability as a product of classifiers being overly sensitive to changes in spurious features, Jacobsen et al.~\cite{Jacobsen:Behrmann:Zemel:Bethge:ICLR:2019} hypothesize that classifiers can rather be
overly insensitive to relevant semantic information, e.g., images with drastically different content can share similar latent representations.
The authors introduce a new type of adversarial examples that exploit such insensitivity, where the content of images is altered without changing the resulting prediction label.
%As both insensitivity to semantic content and sensitivity to spurious changes can simultaneously exist in models,
%more investigation into how to define proper objectives for models to effectively distinguish the relevant information is needed.

While all these works propose possible reasons for adversarial vulnerabilities, they are orthogonal to our survey, which focuses particularly on the influence of training data.

\vspace{-0.05in}
\revadd{
\subsection{Evasion Attacks and Defenses}
\label{sec:relatedwork-attacks}
A number of works focus on techniques for generating evasion attacks, countermeasures against these attacks, 
and defining the notion of the attack itself.

%\jr{need to include~\cite{Biggio:Roli:PR:2018,
%Rosenberg:Shabtai:Elovici:Rokach:CSUR:2021,
%Li:Li:Ye:Xu:CSUR:2021,
%Maiorca:Biggio:Giorgio:CSUR:2019,
%Demetrio:Coull:Biggio:Lagorio:Armando:Roli:ACMTPS:2021,
%Liu:Tantithamthavorn:Li:Liu:CSUR:2022,
%Liu:Nogueria:Fernandes:Kantarci:IEEECST:2022,
%Zhang:Sheng:Alhazmi:Li:ACMTIST:2020} x and one more survey.}
%\js{\cite{Biggio:Roli:PR:2018, Rosenberg:Shabtai:Elovici:Rokach:CSUR:2021} moved to Adversarial Examples.
%\cite{Rosenberg:Shabtai:Elovici:Rokach:CSUR:2021,
%Li:Li:Ye:Xu:CSUR:2021,
%Maiorca:Biggio:Giorgio:CSUR:2019, Liu:Tantithamthavorn:Li:Liu:CSUR:2022,
%Liu:Nogueria:Fernandes:Kantarci:IEEECST:2022,
%Zhang:Sheng:Alhazmi:Li:ACMTIST:2020, Demetrio:Coull:Biggio:Lagorio:Armando:Roli:ACMTPS:2021} in Attacks and Defense. \cite{Sun:Dou:Yang:Zhang:Wang:Philip:He:Li:TKDE:2022} was the "one more survey" and is also in Attacks and Defenses.}

\vspace{0.02in}
\noindent
{\bf Attacks and Defense.}
Several works~\cite{Liu:Tantithamthavorn:Li:Liu:CSUR:2022,Liu:Nogueria:Fernandes:Kantarci:IEEECST:2022,Sun:Dou:Yang:Zhang:Wang:Philip:He:Li:TKDE:2022, Demetrio:Coull:Biggio:Lagorio:Armando:Roli:ACMTPS:2021} survey adversarial attacks and defenses, observing that most work focuses on computer vision and NLP domains. 
Zhang et al.~\cite{Zhang:Sheng:Alhazmi:Li:ACMTIST:2020}, 
Rosenberg et al.~\cite{Rosenberg:Shabtai:Elovici:Rokach:CSUR:2021},
Li et al.~\cite{Li:Li:Ye:Xu:CSUR:2021}, and 
Maiorca et al.~\cite{Maiorca:Biggio:Giorgio:CSUR:2019}, 
survey attacks and defenses in the NLP domain, cybersecurity domain for networks, Android malware, and PDF malware, respectively. 
These works identify a similar trend of new attacks constantly bypassing defenses, which gives rise to new defenses being proposed, only to be broken again (a.k.a. the `cat and mouse race' or the `arms race'). 
They also observe that research in this field studies attacks / defenses at a feature-level, which restricts 
the practicality of the developed techniques by the feasibility of perturbing the corresponding features in real life. 

%practical attacks are quite difficult and require some basic knowledge about the model or training data such as the feature set or model architecture. 
%Zhang et al.~\cite{Zhang:Sheng:Alhazmi:Li:ACMTIST:2020}, who study adversarial attacks and defenses in the NLP domain,  
%also find that there are obstacles to generating attacks in real-time. 
%For instance, methods that iteratively use gradients to create adversarial examples can be time-consuming, while one-time approaches may fail to produce potent adversarial examples.
%Several works~\cite{Liu:Tantithamthavorn:Li:Liu:CSUR:2022,Liu:Nogueria:Fernandes:Kantarci:IEEECST:2022,Sun:Dou:Yang:Zhang:Wang:Philip:He:Li:TKDE:2022, Demetrio:Coull:Biggio:Lagorio:Armando:Roli:ACMTPS:2021} 
%discuss how most new attacks and defenses are explored in computer vision and NLP, prior to other fields.


%our survey finds the state of the art w.r.t. data properties
%our survey finds that dimensionality is bad ...
%
%%%Here are the 5 remaining surveys + 1 additional paper for the reviewer.
%Numerous surveys have explored the landscape of adversarial evasion attacks and defenses. 
%For instance, Akhtar et al.~\cite{Akhtar:Mian:IEEEAccess:2018, Akhtar:Mian:Kardan:Shah:IEEEAccess:2021} survey the literature on adversarial robustness of deep learning models from Computer Vision field.
%They review popular attacks on visual models, and provided a categorization of existing defense techniques based on the components it modify in the visual model system \gx{Check}.
%
%Rosenberg et al.~\cite{Rosenberg:Shabtai:Elovici:Rokach:ACMComputingSurvey:2021}, Li et al. ~\cite{Li:Li:Ye:Xu:ACMComputingSurvey:2021} and Demetrio et al.~\cite{Demetrio:Coull:Biggio:Lagorio:Armando:Roli:ACMTPS:2021} review the literature on evasion attacks for cyber-security fields. 
%Li et al. proposed a partial order scheme to compare key attacks and defenses techniques for malware detection in Windows, Android, and PDF domains. 
%
%Zhang et al.~\cite{Zhang:Sheng:Alhazmi:Li:ACMTIST:2020} review the literature on adversarial attacks on deep-learning models for textual classification.
%They pointed out the intrinsic differences between Computer Vision and Natural Language Processing fields that pose challenges to directly apply attacks proposed for Visual models to NLP models and identified the strategies proposed that overcomes the barriers.
%The challenges they identified for creating realistic attacks in NLP fields are from a domain characteristics perspective (e.g., definition of imperceptible perturbations, measurement of the semantic changes),  we differ from them by trying to understand the adversarial robustness of machine learning from the characteristics of underlying data. 
%
%Attack and Defenses for wireless and Mobile systems~\cite{Liu:Nogueria:Fernandes:Kantarci:IEEECST:2022}
%
%

More recent research, not included in the surveys above, has also started investigating the 
susceptibility of newer models to adversarial evasion attacks. 
For example, several studies~\cite{Wang:Pan:Hu:Duan:Pan:IJSWIS:2022,Yin:Lin:Sun:Wei:Chen:TIFS:2023, 
Shi:Han:Tan:Kuang:NeurIPS:2022, Wang:Xie:Microsoft:ChatGPT:ArXiv:2023} proposed attack techniques against contemporary models, 
such as Graph Neural Networks, Generative Pre-training Transformers (GPT), and Vision Transformers. 
These studies showed that adversarial examples persist even for the newer models, some of which are 
trained with large volumes of data. 
As all these works focus on attack and defense mechanisms rather than 
the effects of data on adversarial robustness, our work extends and complements this research.
}

\revadd{
\vspace{0.02in}
\noindent
{\bf Adversarial Examples.}
%2 surveys lines 13 and 14 + 1 additional paper for the reviewer.
Adversarial examples are inputs constructed by perturbing a correctly classified sample in a way that makes the change imperceptible to a human. % but causes the model to misclassify the sample.
However, as `imperceptible to a human' is hard to define, existing research on adversarial examples approximates imperceptibility with a small perturbation measured through $L_p$ norms.
A line of research~\cite{Gilmer:Adams:Goodfellow:Anderson:Dahl:ArXiv:2018,Sharif:Bauer:Reiter:CVPRW:2018,Fezza:Bakhti:Hamidouche:Deforges:QoMEX:2019, Mezher:Deng:Karam:EUVIP:2022} 
investigates the validity of this assumption. 
This work shows that perturbations generated by $L_p$ norms do not entirely align with human perceptions, 
i.e., some changes with a small $L_p$ norm can be apparent to humans. 
In addition, adversarial examples with the minimum $L_p$ perturbation may be less effective and transferable than 
higher perturbation~\cite{Biggio:Roli:PR:2018,Rosenberg:Shabtai:Elovici:Rokach:CSUR:2021}. 
Hence, a number of approaches explore metrics for imperceptibility 
in computer vision and NLP domains~\cite{Fezza:Bakhti:Hamidouche:Deforges:QoMEX:2019,Mezher:Deng:Karam:EUVIP:2022, Zhang:Sheng:Alhazmi:Li:ACMTIST:2020}. 
Yet another issue with $L_p$ norms is that they cannot be used reliably in domains other than images. 
For example, in the case of software/malware, simply generating adversarial examples with $L_p$ norms 
may result in feature representations that are not possible in 
the problem space~\cite{Rosenberg:Shabtai:Elovici:Rokach:CSUR:2021,Pierazzi:Pendlebury:Cortellazz:Cavallaro:2020}. 

While all these works focus on the properties of adversarial examples, 
they are orthogonal to the topic of our survey, as we rather focus on how properties of the training data 
affect the success of adversarial examples.
}

%Gilmer et al.~\cite{Gilmer:Adams:Goodfellow:Anderson:Dahl:ArXiv:2018} argue that, while constraining the perturbations by sufficiently small $L_p$ norms can generate indistinguishable samples for most inputs, the actual imperceptibility of the changes depends on the input sample. 
%Several individual studies~\cite{Sharif:Bauer:Reiter:CVPRW:2018,Fezza:Bakhti:Hamidouche:Deforges:QoMEX:2019, Mezher:Deng:Karam:EUVIP:2022} find faults with using $L_p$ norms to generate adversarial examples. They show that the changes measured by $L_p$ norm, does not entirely align with human perceptions, i.e., some changes with a small $L_p$ norm appear apparent to humans. 
%In some domains adversarial examples do not need to be imperceptible but rather semantically preserving. 
%For example, in the case of Android malware~\cite{Rosenberg:Shabtai:Elovici:Rokach:CSUR:2021}, adversarial examples are small perturbations which fool a model while preserving the semantics of the sample, 
%i.e., a malware stays malicious even after the perturbation. 
%This highlights another problem with $L_p$ norm based adversarial examples as Dong et al.~\cite{Dong:Liu:Shang:NeurIPS:2022} show that the semantics of a sample change during adversarial training. 
%Hence, there is a need for metrics to measure the size of perturbations that is imperceptible or semantically preserving.
%Fezza et al.~\cite{Fezza:Bakhti:Hamidouche:Deforges:QoMEX:2019} and Mezher et al.~\cite{Mezher:Deng:Karam:EUVIP:2022} propose to use objective metrics for image quality to approximate the imperceptibility in the computer vision domain.
%Zhang et al.~\cite{Zhang:Sheng:Alhazmi:Li:ACMTIST:2020}, focusing on providing such a metric for Natural Language Processing.
%Vadillo et al.~\cite{Vadillo:Santana:CS:2022} also highlight conducted subject studies to evaluate the noticeability of audio adversarial examples.

%Even in computer vision, adversarial examples are not always imperceptible. For example, Machado et al.~\cite{Machado:Silva:Goldschmidt:CSUR:2021} find that visible perturbations such as adversarial patch~\cite{Brown:Mane:Roy:Abadi:Gilmer:ArXiv:2017}, and graffiti on stop signs~\cite{Eykholt:Evtimov:Fernandes:Li:Rahmati:Xiao:Prakash:Kohno:Song:CVPR:2018} are also considered adversarial examples in research.

%The aforementioned research examines the work on defining and creating adversarial examples, demonstrating the insufficiency of using conventional $L_p$ norms to evaluate the imperceptibility and semantics between clean and adversarial examples. 

\vspace{-0.1in}
\revadd{
\subsection{Non-Evasion Attacks}
\label{sec:relatedwork-poisoning}
Similar to evasion attacks, data poisoning and backdoor attacks aim to compromise model accuracy. 
However, they achieve it by tampering the training data to create deceptive model decision boundaries. 
%Data poisoning attacks involve modifying the training data to create deceptive decision boundaries, either to manipulate the prediction outcomes of a specific input or the entire model.
%Meanwhile, Backdoor attacks are a form of poisoning attacks where the attacker inject tempered training data with triggers 
% and then activates the attack by showing the trigger pattern at inference time.
In addition, backdoor attacks also require perturbing the test instance to result in a misclassification. 
This is achieved by introducing manipulated training data with triggers that can be activated during the testing phase.

Goldblum et al.~\cite{Goldblum:Tsipras:Xie:Chen:Schwarzchild:song:Madry:Li:Goldstein:TPAMI:2022} and Cinà et al.~\cite{Cina:Grosse:Demontis:Sebastiano:Zellinger:Moser:Oprea:Biggio:Pelillo:Roli:CSUR:2023} 
review recent literature on attack methodologies and countermeasures for both poisoning and backdoor attacks.
Both of these surveys found that existing research made overly-optimistic assumptions when designing / validating attack techniques, e.g., assuming the knowledge of a large portion of training data. 
They advocate for researchers to test proposed methods in more realistic situations to better assess the potential threats. 
Furthermore, they encourage exploration of the relationship between poisoning attacks and evasion attacks. 
This could lead to the creation of attacks that produce less noticeable poisoning examples, 
or defensive strategies that can safeguard models against both backdoor and evasion attacks.
%Their survey catalogs and systematizes the threats in the dataset creation process, and discuss the open problems that benefits the understanding of dataset security. 

In addition to undermining model accuracy, 
adversarial attacks also aim at breaching the privacy and confidentiality of training data. 
In particular, membership inference attacks~\cite{Shokri:Stronati:Song:Shmatikov:SP:2017} attempt to determine whether a specific data point was part of the training set used to train the model.
Hu et al.~\cite{Hu:Salcic:Sun:Dobbie:Yu:Zhang:CSUR:2022} present a comprehensive survey of existing research efforts on membership inference attacks. 
They find that, similar to evasion attacks, the membership inference attack success rate decreases as 
%the training data better represents the whole data distribution, i.e., 
the number of training samples increases.
%and model stealing attacks~\cite{Oliynyk:Mayer:Rauber:CSUR:2023} are designed to breach the privacy of training data and machine learning models. 
However, all these attacks are orthogonal to our survey, as we focus on adversarial evasion attacks.

%Li et al. ~\cite{Li:Jiang:Li:Xia:TNNLS:2022} 
%provide the first survey that focuses on backdoor attacks and identified common scenarios in which backdoor attack happen in real life. 
%Furthermore, they proposed a systematic taxonomy for backdoor attacks and defenses for researchers and practitioners to identify the characteristics and limitations of each method. 

%Wang et al.~\cite{Wang:Ma:Wang:Hu:Qin:Ren:CSUR:2022} and Tian et al.~\cite{Tian:Cui:Liang:Yu:CSUR:2022} argue federated learning~\cite{McMahan:Moore:Ramage:Hampson:Arcas:AISTATS:2017} 
%creates new venue for poisoning attack, and survey recent literature on poisoning attacks for both standard and federated learning scenarios. 
%They present a unified framework to categorize both data poisoning and model poisoning attacks, and compared the defense techniques proposed for each of the learning framework, analyzed their advantages and disadvantages.
}

\vspace{-0.1in}
\subsection{Effects of Training Data on Standard Generalization}
\label{sec:relatedwork-standard}
A number of surveys investigate the influence of data properties on standard
rather than robust generalization.
One of the earliest is probably the work of Raudys and Jain~\cite{Raudys:Jain:TPAMI:1991},
who review studies related to the influence of sample size on binary classifiers, showing that
a limited sample size usually leads to sub-optimal generalization.
%With the development of deep learning and the ever-increasing need for larger training datasets,
%a variety of data augmentation techniques have been proposed.
Bansal et al.~\cite{Bansal:Sharma:Kathuria:CSUR:2021} and
Bayer et al.~\cite{Bayer:Kaufhold:Reuter:CSUR:2022} also survey papers addressing the data scarcity problem,
focusing in particular on the recent advancements in data augmentation techniques in the fields of computer vision, security, and text classification.
Their results show that augmentation techniques %exist for various application domain and
can help improve a model's generalization by reducing the problem of model overfitting.
%They evaluate the effectiveness of such techniques in improving the accuracy of machine learning models.

%Limited sample size is also one of the culprit behind poor robust generalization~\cite{Schmidt:Santurkar:Tsipras:Talwar:Madry:NeurIPS:2018}, we collected a number of researches characterize the sample complexity for robust generalization or propose data augmentation techniques to fill in the sample complexity gap.

Label noise is another aspect of data that influences both standard and robust generalization.
Most works on this topic find that the presence of noisy labels increases the need for a greater number of training samples and may result in unnecessarily complex decision boundaries~\cite{Frenay:Verleysen:TNNLS:2014,Song:Kim:Park:Shin:Lee:TNNLS:2022}.
For example, Fr\'{e}nay and Verleysen~\cite{Frenay:Verleysen:TNNLS:2014} show
that overfitting to label noise greatly degrades a model's standard generalization;
the same effect has been observed in the case of robust generalization~\cite{Sanyal:Dokania:Kanade:Torr:ICLR:2021}.
Song et al.~\cite{Song:Kim:Park:Shin:Lee:TNNLS:2022} survey the impact of label noise in deep learning, arguing
that the presence of noisy labels is a more serious concern for deep models as they contain a larger number of parameters which makes them prone to overfitting to the noise in training data.
%They also point out the connection between adversarial poisoning attacks and noisy labels as
%the countermeasures for both share the goal of learning noise-resilient representations.
They mention that adversarial defense techniques, e.g., adversarial training, are effective against label noise~\cite{Zhu:Zhang:Han:Liu:Niu:Yang:Kankanhalli:Sugiyama:ArXiv:2021, Fatras:Damodaran:Lobry:Flamary:Tuia:Courty:TPAMI:2022}
but do not discuss how label noise influences a deep learning model's robustness under attacks.

Lorena et al.~\cite{Lorena:Garcia:Lehmann:Souto:Ho:CSUR:2020} identify a collection of 26 quantitative metrics that measure data complexity with respect to
(1) ambiguity of classes, i.e., whether the classes can be clearly distinguished with the given features,
(2) sparsity and dimensionality of data, 
%i.e., whether enough information are provided to learn confident decision boundaries, and
(3) complexity of boundary separating the classes, i.e., whether more intricate functions are required to describe the decision boundaries.
The authors also discuss how these metrics help estimate the difficulty of performing classification on a given dataset.
Similar to our survey, the authors show that high dimensionality and small separation between classes hinder standard generalization.
However, the relationship of some of the metrics reviewed by these authors, e.g.,
%faction of borderline points (i.e., a measure for the complexity of the required decision boundary) and
%the fraction of hyperspheres covering data (i.e.,
the number of non-intersecting spheres needed to enclose all data points of a class,
to robust generalization is not studied, according to our survey.

%Moreover, the effect of XXX on standard generalization needs future investigation as well (that is if we found something they do not have).

%Knowing the characteristics of a dataset according to these perspectives can assist researchers and practitioners to select optimal learning algorithms~\cite{Ho:Basu:TPAMI:2002}.

He and Garcia~\cite{He:Garcia:TKDE:2009} focus on the imbalance learning problem. %~--
%the disproportion in the number of samples belonging to each class in a given dataset.
The authors found that most standard algorithms %are designed with the assumption of a balanced class distribution.
%These algorithms
fail to reliably represent the characteristics of the imbalanced data and result in unfavorable performance across classes.
Furthermore, L\'{o}pez et al.~\cite{Lopez:Fernandez:Garcia:Palade:Herrera:InfSci:2013} discuss six intrinsic data characteristics that potentially complicate learning from imbalanced data:
low density, sample overlap between classes, noisy data, borderline instances,
dataset shift between training and testing distributions, and
small disjuncts, i.e., disperse small clusters of samples from a single class.
Their analysis concludes that while all these ``unfavorable'' data characteristics further complicate the data imbalance
issues, data overlap between classes is probably one of the most harmful.
To follow up on this point, Santos et al.~\cite{Santos:Henriques:Pedro:Japkowicz:Fernandez:Soares:Wilk:Santos:AIR:2022}
focus on the joint effect of data imbalance and class overlap on model generalization.
The negative impact of data imbalance, low separation, and noisy data on robust generalization was also discussed in our survey.
Yet, the compounding effect of these factors, as well as the effect of other properties,
on robust generalization needs future investigation.

Recently, Yang et al.~\cite{Yang:Jiang:Song:Guo:IJCV:2022} summarized relevant studies focusing on
long-tailed distributions in the field of Computer Vision.
% and categorize the main methods for alleviating the issues caused by long-tailed distribution.
%They present quantitative metrics for measuring data imbalance and .
This survey also includes work on the influence of long-tail distributions on a model's adversarial robustness~\cite{Wu:Liu:Huang:Wang:Lin:CVPR:2021}, which is covered in our survey.
%which is included in our survey,
The authors advocate for more research on adapting long-tailed-based approaches for standard generalization to improve robust generalization.

Finally, Moreno-Torres et al.~\cite{MorenoTorres:Raeder:Rodrigues:Chawla:Herrera:PR:2012} present a unifying framework to categorize existing definitions of dataset shift~-- the case where the joint distribution of inputs and outputs differs between training and testing data.
While ML models are normally trained under the premise that testing data has a similar distribution to the training data,
in reality, the observed data distribution may be different from the historical data that the model is trained on.
Such difference can substantially compromise the quality of model predictions.
The authors analyze the possible causes for dataset shift, e.g., malicious software that evolves over time, and
review the techniques dealing with dataset shift.
They characterize adversarial attacks as one form of dataset shift, where adversaries adaptively
change test instances to create a distribution that differs from training data.
%All works discussed in our survey assumed similar distribution on training and testing data, treating adversarial attacks as the only dataset shift in the problem setup.
%However, in real applications, the underlying data distribution itself can be non-stationary, and the characterize the influence of the dataset shift between training and testing data on the adversarial robustness is yet to be investigated.

\revadd{Overall, despite the similarities with our work, literature discussed in this section focuses on standard generalization while our survey discusses 
the effect of data on robust generalization.}

%More works use the connection between adversarial attacks and distributional shift to analyze the effect of adversaries on generalization performance~\cite{Tu:Zhang:Tao:NeurIPS:2019}.
%However, we do not discuss them in detail, as they focus more on models instead of data.
%\jr{How is that relevant to data properties section?} \gx{This can be removed, as it an individual work we filtered}

\vspace{-0.1in}
\subsection{Summary}
\revadd{
Our survey is the first to explicitly focus on properties of training data in the context of model robustness under evasion attacks.
Numerous other surveys on evasion attacks discuss attack and defense mechanisms, non-data-related reasons for adversarial vulnerability, and the different threat models. 
We identified only five surveys that considered data-related reasons for evasion attacks. 
However, as these surveys are older and do not focus on data in particular, our work provides a more extensive
and comprehensive view on this topic. 
By including more than 50 papers not covered in prior work, we were able to 
identify additional relevant properties, practical suggestions, and future research directions in this area. 

Additional work studies non-data-related reasons for evasion attacks, as well as non-evasion attacks, 
such as poisoning and backdoor. 
Yet another body of literature examines how data properties affect standard generalization. These works show that 
some of the properties discussed in our survey, such as 
the number of samples, dimensionality, and label quality, also affect clean accuracy. 
There are also additional data properties that are covered exclusively by these or by our work. 
Studying the interplay between data properties for clean and robust accuracy is an interesting research direction, 
which could be facilitated by our work. 
However, all these current works are orthogonal and complementary to ours.
}

%\ad{
%The related work of our survey can be categorized into four key topics: 
%The first topic examines data for other adversarial attacks, this include the research that investigates the link between the data characteristics and model's resilience against poisoning attacks as well as the studies that explore data poisoning and backdoor attacks and their countermeasures. \jr{same issues as before: this is meta-summary, we need a concrete summary.}
%These studies complement our survey as they highlight the threats directly aimed at data, thus emphasizing the importance of secure data collection. 
%The second topic focuses on the relationship between various properties of training data and model's standard generalization ability. 
%This body of work suggests that data traits such as number of samples, dimensionality, label quality also influence model's ability to generalize in standard classification. \jr{this looks more concrete!}
%
%The third strand of research concerns adversarial evasion attacks. 
%The work in this area encompasses the research frontier in evasion attacks and the countermeasures. 
%Due to the large volume of work in this area, there are numerous surveys that gives more detail on the advancement. 
%\jr{meta-summary again}
%In addition to attacks and defenses, one relevant line of work investigates the alignment of the conventional similarity metrics used for adversarial examples and human perception, showing the need for supplementary metrics. \jr{why important?}
%These studies \jr{which "these studies"?} collectively present an extensive overview of other types of work conducted on adversarial robustness.
%The last category of work proposes alternative explanations for model vulnerability to adversarial examples.
%These studies presented hypothesis showing the characteristics of machine learning models, e.g., nonlinearity, invariance to rotational shift etc, induces susceptibility to attacks, as well as limited computational resources and non-robust feature representations. \jr{all text based on previous related work looks somewhat concrete; the new additions should be at least at the same level, or better.}
%These studies supplement our work, offering a broader perspective of potential factors affecting model's robust generalization ability. }
%


\section{PoseRAC Model}
\label{sec4}

\begin{figure*}[t]
\centering
\includegraphics[width=1.0\textwidth]{figure5.pdf}
\caption{Overview of our proposed PoseRAC. For a input video, the repetitive count can be obtained through Pose Estimation, Transformer Encoder, Pose Mapping and Action-trigger, where only the Encoder and the Pose Mapping need to be trained. We use Triplet Margin Loss to train the Encoder while Binary Cross Entropy Loss to train both the Encoder and the Pose Mapping. In addition to achieving the state-of-the-art performance so far, the biggest highlight of our PoseRAC is that it is lightweight enough to be easily trained on a CPU.}
\label{fig5}
\end{figure*}

Given a video $V={\{x_i\}}^{T}_{1}\in \mathbb{R}^{C\times H\times W\times T}$ with $T$ RGB frames, repetitive action counting model aims to predict a certain value $Y$, which is the number of repetitive actions. In this section, we will introduce our PoseRAC in detail.

\subsection{Model Overview}

As shown in Figure \ref{fig5}, PoseRAC consists of four parts. 

\begin{itemize}

\item The first is a state-of-the-art and lightweight Pose Estimation Network~($\S\ref{first}$), which is used to estimate the poses represented by lots of human pose key points from each frame of the original video sequence. 

\item The second is a simple Transformer Encoder~($\S\ref{second}$) to embed the key points of poses into high-level feature space, where the same class have similar distances, while the distances of different classes are far apart.

\item The third is a Pose Mapping Module~($\S\ref{third}$), where the unique mapping relationship between the salient poses and the action classes can be learned. Each pose can be mapped to the action class with the highest probability after the previous encoding.

\item The fourth part is a lightweight Action-trigger Module~($\S\ref{fourth}$). When we get the salient action classification results of all frames of the entire video sequence, we can use this module to calculate the repetition count in a short time.

\end{itemize}

\subsection{Pose Estimation Network}
\label{first}
Our model first converts the video sequence into a sequence of human pose key points, which can be defined as: 
\begin{equation}
\begin{split}
&V={\{x_i\}}^{T}_{1}\in \mathbb{R}^{C\times H\times W\times T}\\
&V\xrightarrow{\mathrm{Pose Estimation}} P={\{p_i\}}^{T}_{1}\in \mathbb{R}^{D\times K\times T}
\end{split}
\label{eq1}
\end{equation}
where each $x_i$ represents a single RGB frame, and each $p_i$ represents the key points of each frame. To express the key points of each frame, we use $D\times K$ sequence, which includes two parts, one ($K$) is the number of key points to fully represent the current pose, the other ($D$) is the dimension of each key point, generally three, which are the two coordinates of the planes and the depth estimation.

Here we use state-of-the-art pose estimation models such as Vitpose\cite{xu2022vitpose} and BlazePose\cite{bazarevsky2020blazepose}. The pose estimation algorithms themselves are not designed by us, but we introduce pose information into the action counting task, which is a novel design not explored by previous work.

Moreover, our pose-level poses estimation processes the primitive information of video, which is similar to the feature extraction network in all video-level algorithms such as I3D\cite{carreira2017quo}, VideoSwinTransformer\cite{liu2022video}, and TSN\cite{wang2016temporal}. But the difference is that the result of video-level incorporates all information, while pose-level only produces core information, which greatly improves the performance. Additionally, using pose information can contribute to the lightweight of model. For instance, for a 1024-frame video, video-level feature extraction with an output dimension of 512 would produce a data volume of $1024\times 512=524288$, while using pose information with 33 key points produces a data volume of only $1024\times 33 \times 3=101376$.

\subsection{Encoding Poses with Transformer}
\label{second}
Here we specify our data representation for the Transformer Encoder, which requires input batch size, sequence length, and embedding dimensions. In our pose-level approach, each frame is a batch, the number of key points in each frame is the sequence length, and the feature dimension of each key point is the embedding dimension.

First we get the pose of each frame ${p_i}\in \mathbb{R}^{D\times K}$ through the Pose Estimation Network, where $i\in {1, 2, \dots, T}$ is the frame index, $K$ is the number of key points, and $D$ is the dimension of each key point. We further define $p_i = {\{k_j\}}^{K}_{1}$ to represent each key point, where $k_j\in \mathbb{R}^D$, and we embed it to obtain richer information. Our embedding projection $\mathrm{\bf{E}}$ is a simple MLP network with ReLU as the activation function. These calculations can be defined as:
\begin{equation}
\begin{split}
\mathrm{\bf{Z}}^0 = [\mathrm{\bf{E}}(k_1), \mathrm{\bf{E}}(k_2), \dots, \mathrm{\bf{E}}(k_K)]^T
\end{split}
\end{equation}
where $\mathrm{\bf{E}}(k_j)\in \mathbb{R}^{D^{\prime}}$ is the embedding feature. Then the next Transformer takes $\mathrm{\bf{Z}}^0$ as input and encodes it with self-attention. Given $\mathrm{\bf{Z}}^0\in \mathbb{R}^{K\times D^{\prime}}$ with $K$ key point features, each of which is $D^{\prime}$-dimensional, $\mathrm{\bf{Z}}^0$ is projected using $\mathrm{\bf{W}}_Q\in \mathbb{R}^{D^{\prime}\times D_q}$, $\mathrm{\bf{W}}_K\in \mathbb{R}^{D^{\prime}\times D_k}$, $\mathrm{\bf{W}}_V\in \mathbb{R}^{D^{\prime}\times D_v}$, where $D_k=D_q$, to extract feature representations query($\mathrm{\bf{Q}}$), key($\mathrm{\bf{K}}$) and value($\mathrm{\bf{V}}$), which can be defined as:
\begin{equation}
\begin{split}
&\mathrm{\bf{Q}}=\mathrm{\bf{Z}}^0\times \mathrm{\bf{W}}_Q\\
&\mathrm{\bf{K}}=\mathrm{\bf{Z}}^0\times \mathrm{\bf{W}}_K\\
&\mathrm{\bf{V}}=\mathrm{\bf{Z}}^0\times \mathrm{\bf{W}}_V
\end{split}
\end{equation}
and the output of self-attention can be computed as:
\begin{equation}
\begin{split}
\mathrm{\bf{Attn}}=\mathrm{Softmax}(\frac{\mathrm{\bf{Q}}\mathrm{\bf{K}}^T}{\sqrt{D_q}})\mathrm{\bf{V}}
\end{split}
\end{equation}
where $\mathrm{\bf{Attn}}\in \mathbb{R}^{K\times D^{\prime}}$. Also, we use common multi-head self-attention (MHSA) to make several self-attention operations calculate in parallel.

Now we introduce the overall architecture of Transformer Encoder, which has $L$ layers with each layer consisting of MHSA and MLP blocks. Also, LayerNorm and Residual Connection are applied before and after every MHSA or MLP block, respectively. Because the number of key points of each frame is  a bit less, so our encoder does not include the downsampling module that other models may have. The overall process can be defined as:
\begin{equation}
\begin{split}
&\mathrm{\bf{\hat{Z}}}^l = \mathrm{MHSA}(\mathrm{LN}(\mathrm{\bf{Z}}^{l-1})) + \mathrm{\bf{Z}}^{l-1}\\
&\mathrm{\bf{Z}}^l = \mathrm{MLP}(\mathrm{LN}(\mathrm{\bf{\hat{Z}}}^l)) + \mathrm{\bf{\hat{Z}}}^l
\end{split}
\end{equation}
where $\mathrm{\bf{Z}}^{l-1}$, $\mathrm{\bf{\hat{Z}}}^l$, $\mathrm{\bf{Z}}^l\in \mathbb{R}^{K\times D^{\prime}}$.


\subsection{Pose Mapping}
\label{third}
Taking the Encoder output $\mathrm{\bf{Z}}^L\in \mathbb{R}^{K\times D^{\prime}}$ as input, Pose Mapping module outputs probability scores $\mathrm{\bf{S}}\in \mathbb{R}^{C}$ of the current frame over all action classes. We perform binary classification after Sigmoid activation for each class, with the two salient poses of each class represented by the same bit data. To realize such a module, we use a very lightweight MLP network, which avoids the complexity. First, the two dimensions $K$ and $D^{\prime}$ of $\mathrm{\bf{Z}}^L$ are flattened into $\mathbb{R}^{KD^{\prime}}$, and then it passes through an MLP module, where the output channels is set to $C$, which can be defined as:
\begin{equation}
\begin{split}
\mathrm{\bf{S}} = \sigma(\mathrm{MLP}(\mathrm{Flatten}(\mathrm{\bf{Z}}^L)))
\end{split}
\end{equation}
where $\sigma$ represents the Sigmoid activation function.

With such Pose Mapping, we can obtain the scores of single frame. It should be noted that we extract the poses of all frames, and use the convenience of matrix operations to obtain scores in parallel, which is actually consistent with the idea of mini batch. So at last, we combine the scores of all frames to get the video score matrix $\mathrm{\bf{\hat{S}}}\in \mathbb{R}^{C\times T}$, where $T$ represents the number of frames in the current video. 


\subsection{Action-trigger Module}
\label{fourth}
We use the lightweight Action-trigger Module to obtain the final output $Y$, the repetitive action count, which has a time complexity of $\mathcal{O}(n)$. First, we get the scores $S_c\in \mathbb{R}^T$ of a given action class from $\mathrm{\bf{\hat{S}}}$. Then, we scan all frames and use the action-trigger mechanism to count when the two salient poses of the action class occur sequentially. We set upper and lower bounds to distinguish the scores of the two salient poses, which cluster non-salient poses in the middle and easily classify the salient poses to the two ends.

\subsection{Losses and Metric Learning}

The modules need to be trained are Embedding, Transformer Encoder and Pose Mapping, and because we perform binary classification for each class, so we use the Binary Cross Entropy Loss, which can be defined as follows:
\begin{gather}
\mathcal{L}_{bce} = -\frac{1}{N}\sum\limits_{i=1}^{N}(\frac{1}{C}\sum\limits_{j=1}^{C}loss(i,j))  \\
 loss(i,j)=y_{ij}\log p_{ij} + (1-y_{ij})\log(1-p_{ij})
\end{gather}
where $N$ represents the batch size (in our method, each frame is a batch), $C$ represents the number of classes, $y$ and $p$ are the labels and our predictions, respectively.

Moreover, we use Metric Learning to improve our Encoder and introduce the Pose Triplet Loss. Given a pose, Encoder produces higher-level features $\mathrm{\bf{Z}}^L$, which should be more representative. As shown in Figure \ref{fig5}, we achieve this with Triplet Margin Loss function, which selects anchors, same class positive samples, and different classes negative samples in a batch. It can be expressed as:
\begin{equation}
\begin{split}
\mathcal{L}_{tri} = \mathrm{max}(\mathrm{CS}(a,p)-\mathrm{CS}(a,n)+\mathrm{margin},0)
\end{split}
\end{equation}
where $a$, $p$, $d$ are anchors, positive and negative samples, and $\mathrm{CS}$ represents the Cosine Similarity to measure the distance between features. We pay more attention to hard samples, where the distances between anchors and negative samples are even smaller than those of positive samples. After Metric Learning, the poses of each action can be distinguishable, which cluster in the high-level space.

At last, our overall training combines these two losses:
\begin{equation}
\begin{split}
\mathcal{L} = \mathcal{L}_{bce} + \alpha\mathcal{L}_{tri}
\end{split}
\end{equation}
where $\alpha$ is the weight factor to control the two losses in the same numeric scale.
\subsection{Implementation Details}

\noindent{\bf Training.} We use the \emph{RepCount-pose} and \emph{UCFRep-pose} dataset we created to train our model. Only the frames with salient poses are inputted into the network instead of the entire video to speed up the fitting.

\noindent{\bf Inference.} During inference, the entire video sequence is inputted into the model. The poses of all frames pass through the Encoder and Pose Mapping, and then enter the Action-trigger Module to output the repetitive count.
\section{Results}
 In this section, we first present the participants' design outputs, as well as quantitative results and interaction patterns. Then, in the following sections, we describe how \LPIMs{} changed the design process, and affected collaborative dynamics during design.
 
Note that even though participants in the \imagen{} condition were allowed to use images found through  using \IS{}, only one pair of participants did so. To simplify our description of results, we therefore describe them as results while using \imagen{} and while using \IS{}, even though participants using \imagen{} always had access to \IS{}.

\subsection{Design Outputs and Quantitative Findings}
{Qualitative results regarding the processes followed by participants is possibly the larger contribution of our work. However, for the sake of completeness and to offer a statistical overview of participant behaviors and preferences, we briefly mention quantitative results below.}

Nearly all participants spent all the available time during their first session (regardless of whether they used \IS{} or \imagen{}), $M=18.5$ minutes. Many participants used less time in the second design session, regardless of condition, $M=14.2$ minutes, suggesting there was a learning effect in the task. However, there was no significant difference between \IS{} and \imagen{} conditions.   

When working with their partner, participants followed different collaborative styles, which we briefly describe in Section~f{sub:collaboration}. Participants typically brainstormed about the general theme of their design (characters from Alice in Wonderland), and then started to look for images (using \IS), or generated images (using \imagen{}) that fit this theme. While some pairs had one screen shared with one of the participants ``driving'' the integration of images into the final design, while the other looked for images, many pairs  worked collaboratively on editing the prompts to generate images. Finally, some pairs worked in parallel, sharing interesting results with each other. These pairs then collaboratively edited the slide deck to create the final design.
 
Figure~f{fig:cake-prompts} shows some of the final invitations the participants created. Participants' invitations created in the \IS{} condition had an average of $4$ images in the design ($median=4$), while those in the \imagen{} condition had $2.4$ images on average ($median=2$). We discuss possible reasons for this difference in Section~f{ssub:opinionated}.

Participants self-reported their final design to be more creative when they used \imagen{} (5-point Likert scale, mean=3.6) than when they used \IS{} alone (M=3). This improvement (M=0.6) was statistically significant ({two-sided paired} $t(13)=2.65, p=0.02$.) There was no difference in how complete participants reported their creations to be (\IS{} $M=3.6$, \Imagen{} $M=3.2$), or how appropriate it was to the design brief  (\IS $M=4.2$, \Imagen{} $M=4.1$). On the other hand, external raters did not find any significant differences in the creativity, completeness, or appropriateness of the brief for designs in either condition. 

In a post-study survey, participants did not note any significant differences in their ability  ``to manage any relationship tension'' in their work group (paired t-test, $p=0.19$), ``to politely include my partner’s ideas in the final design while also preserving my own'' ({two-sided} paired t-test, $p=0.27$), or ``to decide about who should do what in our group, even when we had some differences in opinion'' ({two-sided} paired t-test, $p=0.16$). However, participants did reveal a preference for using \imagen{} or a similar model were they ``to complete a similar task in the future'' (mean rating $=3.8$, median $=4$, on a 1-5 Likert scale, \textit{5=``Strongly prefer to use \Imagen{}/similar
model''}.)  

Given these self-reported preferences for using \imagen{} and similar models, along with modest differences in the output quality, we focus the rest of this section on the qualitative differences in the processes that participants employed.


\subsection{Interaction Patterns}

During the study, we observed differences in how participants queried or prompted \IS{} and \imagen{}. Participants understood that \IS{} found pre-existing images on the Internet and so used broad queries that they hoped would yield useful results (e.g. \prompt{tea party}.) If these queries did not yield relevant results, participants searched for related terms instead (e.g., \prompt{mad hatter} $\rightarrow$  \prompt{hare with a hat} $\rightarrow$ \prompt{crazy top hats}). As such, participants (correctly) used \IS{} as a \textit{querying} interface. In contrast, participants' inputs to \Imagen{} could best be described not as \textit{queries} but as \textit{descriptions}, such as \prompt{Colorful drawing of a Cheshire cat from Alice in Wonderland. The cat is wearing a birthday hat and is on a white background.} Throughout the rest of this paper, we call these input descriptions \textit{prompts} to distinguish them from queries.

Participants wrote increasingly elaborate prompts with \Imagen{} during their design session, especially when the image results were disappointing. For instance, P7{} and P8{} tried the prompt \prompt{Beach party on the moon, on the moon in the Sea of Tranquility. Digital art.} Unfortunately, \Imagen{} did not generate any images for this query, and participants hypothesized this was because the beach party had nudity or other content that led  \Imagen{} to block it. (In actuality, it is likely these images were blocked because \imagen{} does not generate images with photo-realistic people in them.) These participants then modified their prompt several times, ending with \prompt{A doodle of a beach party of fully suited astronauts on the Moon in the Sea of Tranquility. The Sea of Tranquility has water in it, and some astronauts are surfing in it with surfboards that have the "NASA" logo on them. Digital art.}




% TODO (address Shaun's feedback to briefly describe nuts and bolts of how they collaborated in 1-2 sentences -- what were some patterns for how they did division of labor?). A typical pattern was to do XYZ, sharing interesting results with each other, and pasting promising images into Slides to curate later. While some pairs did X, other pairs did Y.}

% 
\begin{figure}
     \centering
     \begin{subfigure}[t]{0.3\textwidth}
         \centering
         \includegraphics[width=\textwidth]{\figpath{image-search/Dave-Ian}}
         \label{fig:image-search-1}
     \end{subfigure}
     \hfill
     \begin{subfigure}[t]{0.3\textwidth}
         \centering
         \includegraphics[width=\textwidth]{\figpath{image-search/Emily-Hendrik}}
         \label{fig:image-search-2}
     \end{subfigure}
     \hfill
     \begin{subfigure}[t]{0.3\textwidth}
         \centering
         \includegraphics[width=\textwidth]{\figpath{image-search/Bardia-Dillon}}
     \end{subfigure}
     
     \begin{subfigure}[t]{0.3\textwidth}
         \centering
         \includegraphics[width=\textwidth]{\figpath{imagen/Emily-Hendrik}}
         \label{fig:imagen-1}
     \end{subfigure}
     \hfill
     \begin{subfigure}[t]{0.3\textwidth}
         \centering
         \includegraphics[width=\textwidth]{\figpath{imagen/dave-ian}}
         \label{fig:imagen-2}
     \end{subfigure}
     \hfill
     \begin{subfigure}[t]{0.3\textwidth}
         \centering
         \includegraphics[width=\textwidth]{\figpath{imagen/peter-justin}}
     \end{subfigure}
        \caption{A few of the designs participants created in our study. \textit{Top row}: Designs without access to prompt-based image model; \textit{Bottom row}: Designs with access to prompt-based image model. {(In both conditions, designs shown are ones with the highest overall rating by independent experts.)}}
        \label{fig:cake-prompts}
        % ALT TEXT: Six images of birthday cards. Top row includes cards designed using image search. On the top left is an Alice In Wonderland themed card containing a rabbit holding a bugle and wearing the uniform of the Queen of Hearts. In the top middle, is another Alice in Wonderland themed card featuring black silhouettes of rabbits, teapots, and top hats, with a background of a flower. On the top right is a moon landing themed card featuring a spaceship and an astronaut holding a flag. The bottom row includes card designed using an LPIM. On the bottom left is an Alice in Wonderland themed card featuring two images: a blue and red top hat, and a rabbit in fancy red clothes holding a bugle. In the bottom middle is a moon landing themed card with a detailed, cartoonish, image of an astronaut on the moon holding a birthday cake with three candles. On the bottom right is a moon landing themed card featuring 4 images, two of which show astronauts surfing on the moon. The other two photos feature astronauts and a space ship.
\end{figure}

\begin{figure}
     \centering
     \begin{subfigure}[t]{0.3\textwidth}
         \centering
         \includegraphics[width=\textwidth]{\imagepath{cake3.png}}
        %  \caption{prompt TODO}
         \label{fig:cake1}
     \end{subfigure}
     \hfill
     \begin{subfigure}[t]{0.3\textwidth}
         \centering
         \includegraphics[width=\textwidth]{\imagepath{cake2.png}}
        %   \caption{prompt TODO}
         \label{fig:cake2}
     \end{subfigure}
     \hfill
     \begin{subfigure}[t]{0.3\textwidth}
         \centering
         \includegraphics[width=\textwidth]{\imagepath{cake1.png}}
        %   \caption{\prompt{A cartoon cake that says `eat me' with a purple striped cat curled on top.}}
         \label{fig:cake3}
     \end{subfigure}
        \caption{A few of the images participants created with \imagen{}. As can be seen, \imagen{} does not always generate images that are properly cropped; participants used prompts such as \prompt{...framed art} to generate images with better composition (far right).}
        \label{fig:cake-prompts}
        % ALT TEXT: Left: A drawing of a purple cat sitting on a purple and white cake with the words "Eat me" written  on the cake. Middle: A retro 1950's style image with an astronaut in blue and yellow holding a cake, with stars on a red background. Right: A framed image of an astronaut on the moon holding a birthday cake with candles.
\end{figure}

\subsection{How TTI models changed the design process}\label{sub:collaboration}
Through rapid  image creation and their indirect nature, where images were created through text descriptions, \LPIMs{} led to new design practices, as outlined below. 

\subsubsection{Indirect and rapid image creation through text allowed {new} creative freedom}
Participants noted how creating with \imagen{} was indirect, as it involves ``creating prompts that create images'' (P13{}). This indirectness and the flexibility of prompt editing allowed participants to rapidly explore the design space of alternatives. This was most apparent when participants used \imagen{} to take on other 'artist' personalities, which would otherwise have taken years of practice. P4{} noted:  ``If you made a poster, it [the poster] would have had your style associated with it by default because you have to learn [and develop a particular style]… It is harder to switch between styles. Whereas \Imagen{}, you could just be like `1960s poster’ or like in the style of whoever: Picasso''. At the same time, participants felt faster image generation would allow for even more exploration. P3{} noted: “Because it takes so long to generate a bunch of different images, I didn't really move off to, you know, how else that card could look.” 


Participants also noted how the model implicitly steered such rapid exploration. Despite this steering, participants noted how they still remained in control over the ``personality'' their \Imagen{} creations would have. For instance, P3{} quoted above added: ``Well, for me the image that it generated was sort of similar to what I envisioned in my mind... The way that it turned out is pretty cool. It is definitely not the style that I would have chosen for myself, my own drawings, but like it looks pretty.'' {(We should note that, due to the short term nature of our study, we are unable to study how such model steering impacts participant creativity over the long term.)}

Throughout these explorations, participants tended to improve or ``optimize'' a prompt if they found that at least one of the generated images was helpful. Participants refined their prompts both to bring out aspects they found successful in the initial set of results, and to steer results away from undesired properties. For example, while creating a Cheshire Cat, P9{} liked the \textit{card design} in the results, rather than the cat in the foreground:  ``[I] kind of like some of these designs..." They then updated their prompt to get more of that card design: \prompt{Frame with filigree pattern. Circus colors}. 

At other points, participants refined their prompts to steer results away from undesired properties. For example, P8{} first tried to generate a jovial Cheshire Cat image but remarked that “Those are a little terrifying." He thus updated the prompt to make the image look less scary and more festive: \prompt{Invitation to a birthday party. Alice in Wonderland. Cheshire Cat}. Similarly, participants noticed that with some images that were poorly cropped, they could obtain better results if they appended \prompt{framed painting} to their prompt. 


\subsubsection{Novel images steered novel ideas}

Whereas \IS{} surfaced existing images on the Internet, \Imagen{} {allowed} participants to generate {entirely novel} images, {and allowed them to successfully explore their creative ideas}. For example, P5{} described hitting a wall with \IS{} when he could not find a specific aspect of what he wanted via \IS{}, possibly because it did not exist in the real world: ``I wanted a picture of a dolphin...And I started to Google it... One of my problems when I was searching around, is I couldn't quite get the image I wanted, right? I wanted to make something new and I couldn't quite get the right image I wanted."In contrast, participants were able to use \imagen{} to create novel combinations of ideas that did not exist, such as a giraffe driving a Lamborghini: ``Like `a giraffe is driving a Lamborghini' ... these are things you can never do. You can never have images, that look reasonable for those online. If you had to do it the old-fashioned way, or be really good at Photoshop or Illustrator. And it would take a lot longer than I have.'' They were also able to apply styles to content from a time period, which would not have been possible in the real world: ``...show me `the Apollo 11 Landing in the style of Dali'”.


\begin{figure}
     \centering
     \begin{subfigure}[t]{0.3\textwidth}
         \centering
         \includegraphics[width=\textwidth]{\imagepath{Imagen_giraffe-driver}}
         \label{fig:image-search-1}
     \end{subfigure}
     \begin{subfigure}[t]{0.3\textwidth}
         \centering
         \includegraphics[width=\textwidth]{\imagepath{Imagen_giraffe-moping}}
         \label{fig:image-search-1}
     \end{subfigure}
     \begin{subfigure}[t]{0.3\textwidth}
         \centering
         \includegraphics[width=\textwidth]{\imagepath{Imagen_giraffe}}
         \label{fig:image-search-1}
     \end{subfigure}
     \caption{A few giraffes driving fancy cars. (Prompt inspired by participant: \prompt{Giraffe is driving a Lamborghini. f2.2}). By enabling the rapid realization of novel ideas and unlikely combinations, \LPIMs{} enable an  exploration of the design space and fluid collaboration.}
     % ALT TEXT: Three images of giraffes driving fancy cars. The images on the left and right show giraffes in the driver seat holding the wheel. The middle image shows a giraffe sticking its head out of the car window.
\end{figure}


{In addition to exploring existing creative ideas, surprising image results also spurred participants} to go in a different direction. For example, TTI  surprises inspired participants to consider aesthetic styles, compositions, or other design choices they hadn't initially considered: ``When I asked \imagen{} for a doodle of that \imagen{} blew me away with something that was a different art style than I imagined. That inspired me to seek out stuff in that same art style or to keep asking for doodles.” Together, participants saw \imagen{} as a {way to support their creativity in ways that were qualitatively different from previous tools}. As P6{} noted: ``it really kind of stirs, my creative juices or whatever whereas like Googling for images does not really stir that…''


However, \imagen{} also occasionally generated {non-sensical or clearly flawed} images, such as animals with incorrect anatomy or images of the Moon with two ``Earths'' in the background. Participants contrasted this with \IS{}, which offered more predictable results because they were authentic images from the Internet. This, in turn, allowed participants to {feel that the resulting images reasonably depicted what was in the images}  without closer scrutiny. For example, P5{} suggested:  “...I trusted the images that I look for are going to look somewhat more sane... I am not going to see half of two rabbits”. Participants also saw such predictability as necessary when looking for a specific image. For instance, one pair of designers used public-domain images of the first Moon Landing. For such images, correctness was crucial: “Most of the images we used are very specific. They are images that \Imagen{} cannot generate.” 


\subsubsection{Designing with an opinionated model}\label{ssub:opinionated}
% TODO intro sentence
As \imagen{} would sometimes produce unexpected results, the participants often felt the need to guide or work around the model's limitations. 
For instance, P3{} noted their decision to use  the model to generate an image for the entire invitation from a single prompt rather than prompting for each part of the image and compositing them: “I suppose because we knew the limitations of \imagen{}, in terms of like, composing it for multiple images is, it sort of reduced what we could do with it.” Other participants were able to avoid design fixation, but with considerable effort. P5{} remarked: ``It felt like I was fighting it….I felt like it was helpful, but I also felt like I had to massage every word and select every character very carefully not to upset it so that it could generate something I wanted.'' Consistent with prior work, in these and other quotes, participants seemed to ascribe the role of an opinionated design partner to \imagen{} For instance, Koch et al. described how participants ascribed agency to the AI tool (with one participant even referring to it as "an eccentric collaborator") \cite{koch2020CollaborativeAI}. Similarly, in a study of an AI-based co-creation tool that generates sketches to inspire the user as they are actively sketching, users perceived the AI as a ``collaborative partner'' in the condition when the system communicated with them \cite{rezwana2022AICoCreation}. {(Because participants themselves anthropomorphized the model, we characterize it as opinionated, rather than using other terms, such as being biased.)}


While prompting {with this opinionated} model enabled participants to express high-level concepts at rapid speed, participants struggled to systematically control low-level details, such as position, layout, and which letters appear in the text (note, however, that some participants did use \Imagen{} to generate text in styles that Google Slides did not support, see Figure~f{fig:cake-choices}). For instance, P4{} wished for ``more controllability'': ``it is kind of agonizing to keep typing in very different versions of the same thing, and you are like, no, I just want his hand to be, like a little bit farther down.'' Similarly, P5{} expressed their frustration with how \Imagen{} sometimes cropped parts of an object in the image: ``So this is yeah, with this image, we can go outside the lines and get something that covers more of the screen while it is just focusing on the top hat...\textit{after a few moments}... I cannot.'' 


\subsection{How text-to-image models changed collaborative dynamics during design}
{Text-to-image models} modulated the collaborative practices among participants by creating new ways to fluidly combine ideas with prompts. At the same time, because prompts were so central to these collaborations, asymmetric access to the prompts changed collaborative roles and exploration.

\subsubsection{Prompts allow participants to fluidly combine ideas}
A core aspect of creative collaboration is the ability to combine, re-mix, and try out ideas from multiple people \cite{fauconnier2008way}. However, while using \IS{}, participants sometimes discovered that, even when they could agree and combine their ideas, those combinations of ideas were often hard to find within the search results. For instance, one participant noted: ``It was easy for us to sort of like agree and collaborate on ideas but then it was hard to find images that match those ideas.'' Similarly, during their design session, P4 said to their partner: ``I like the one that you had with the crazy paper vintage background,'' but later was unable to find images of candles in that preferred style: “something about beggars cannot be choosers.” 


In contrast to \IS{}, with \imagen{}, participants were both able to combine ideas in their prompts and experiment rapidly with different ways of composing them together. For instance, in this conversation, P7{} fluidly added his ideas for fireworks to their prompt about rockets on the moon: 
\begin{description}
\item[P8{}] Another theme could be something to do with rockets.
\item[P7{}] Oh yeah, or like rocket fireworks. [Prompt: \prompt{Fireworks exploding in the shape of a space shuttle.}]
\end{description}


Furthermore, \imagen{} allowed participants to see a variety of \textit{generated} images and choose the ones that best matched their needs. Reacting to a set of \imagen{} results based on their partner's query, P6{} said: ``It [image on the left] does not get the idea of the party across. Let's go with the one on the right because it has like the  astronaut has a party hat.'' 



Finally, even though not this was not the focus of our study, participants often spoke about how they learned tricks for successful prompting socially. For instance, P5{} suggested how this process of social learning was fun: ``Like, it could be fun, especially when me and my coworkers are all sitting at my desk and people like, oh, take this [prompt] and see what it does.'' In many of our design sessions, we saw many such prompt modifications, such as using \prompt{...framed art}, or particular camera or lens types to mimic in the images generated, such as \prompt{...Sigma 85mm f1.4}. Once participants shared such prompt tricks with their partner, they often used them in their collaborative design work. 

\subsubsection{Asymmetric access to prompts, randomized generation,  hinder collaboration}
Often, the ability to iterate on a design was weighted towards whichever collaborator had access to the prompt, leading to asymmetric access. This was particularly prominent in situations where participants prototyped prompts in windows that were not shared with their partner, as in this exchange:

\begin{description}
\item[P1] \textit{(chuckles)} Okay, well I got something which will be sort of, kind of more appropriate maybe. So I'm gonna paste it here  
\item[P2]  \textit{(seeing the results)} Hey! That's pretty good. Okay….Yeah, even the Lander is partying! I think we go with this one.
\end{description}

In this situation, even though P1{} was able to share an exciting image result with P2{}, P2{} was not able to iterate on the design because he did not have access to the prompt. In this case, we noticed P2{} became increasingly reliant on P1{} to create images as the design session progressed. 

Even with access to prompts, generative models (including \imagen{}) typically use a random seed as input, so users see different image results on consecutive runs of the model, even when providing an identical prompt input. As a result, participants were sometimes unable to replicate previous results reliably, hampering collaboration.


\subsection{Prompts as reflective design material}
Throughout their design session with \IS{}, it seemed that participants merely saw \IS{} as a way to find the needed images. In contrast, when participants used \imagen{}, they displayed a nuanced, functional understanding of how prompts could be used to achieve their design objectives. Moreover, this understanding was not related to the technical aspects of how \imagen{} worked -- not once in our sessions or interviews did participants mention ``transformers'', ``diffusion models'', or even ``deep learning.'' Instead, they spoke about and enacted how  \imagen{} allowed them to rapidly explore a range of artistic possibilities and to collaborate.

These observations lead us to characterize prompts as \textit{design materials}. Below, we describe how participants exhibited a tacit understanding of \imagen{} and how prompts allowed for exploration and reflection on model actions (i.e., images generated) and reflection in action (i.e., through collaboratively editing the prompts).


Participants used and developed their tacit mental models of \Imagen{}'s design orientation throughout their design process. For instance, P3{} noticed how \Imagen{} framed the subjects in its images: ``Most of the images that get generated by \Imagen{} always push everything up to the front.'' Sometimes, participants tried to compensate for what they believed the model did not understand. P6{} noticed, for instance, ``It seems like it does not know what the Cheshire Cat is," changing their prompt from \prompt{An illustration of the Cheshire Cat from 'Alice in Wonderland'} to \prompt{An illustration of a cat with a large face smiling and looking at the camera}. Finally, participants sometimes generated images mostly to test what \Imagen{} might do with a prompt. For instance, P4{} said to their partner: “Oh, we could try that with like `1969 poster' or no… because the poster will make Imagen try to…? Let’s try that. `1969 poster'.” Then, examining the results, they decided: ``These are like the very artsy side which is probably less what we want. But they are still fun.”

\subsubsection{Prompts allow rapidly exploring the design space} 

In their role as reflective design materials, prompts allowed participants to rapidly explore their design's content, style, and layout. 
For instance, many participants opened multiple instances of \Imagen{} (in different browser tabs) to explore variations of a prompt, such as \prompt{Drawing of a Cheshire cat from Alice in Wonderland. Cartoon}, \prompt{Drawing of a Cheshire cat from Alice in Wonderland. Psychedelic}, and \prompt{Colorful drawing of a Cheshire cat from Alice in Wonderland. Cartoon.} 

Participants made these decisions with fluidity, interlaying decisions of content and style while navigating the limits of the model: 

\begin{description}
\item[P12{}] Yeah... it might be hard to get Alice eating cake. 

\item[P11{}] ...yeah 

\item[P12{}] maybe we could do something with like `the cake from Alice in Wonderland'.

\item[P11{}] Yeah. Yeah. Maybe if we can't get Alice's face out of it than we could use Alice's face... like a non copyrighted one from Google.

\item[P11{}] Yeah. And then we could probably just do like, a cake that says `eat me' on it, right? 

\item[P12{}] Hmm, yeah. You have like a style that we want for that?

\item[P11{}] It definitely, it's got to be like a cartoon one, at least. So we don't want it photorealistic.
\end{description}


Finally, because \imagen{} displayed multiple candidates per prompt, participants also could explore their design choices based on the results they obtained (see Figure~f{fig:cake-choices}). 

\begin{figure}[htb]
     \centering
     \begin{subfigure}[b]{0.3\textwidth}
         \centering
         \includegraphics[width=\textwidth]{\imagepath{grow-cake.png}}
         \caption{``Yeah, it's like it's still a little bit cropped''}
         \label{fig:grow-cake}
     \end{subfigure}
     \hfill
     \begin{subfigure}[b]{0.3\textwidth}
         \centering
         \includegraphics[width=\textwidth]{\imagepath{grow-on-plate.png}}
          \caption{``Maybe the one on the plate...yeah, I think we should crop it around `grow'''}
         \label{fig:cake2}
     \end{subfigure}
     \hfill
     \begin{subfigure}[b]{0.3\textwidth}
         \centering
         \includegraphics[width=\textwidth]{\imagepath{chosen-grow.png}}
          \caption{``Oh, oh...  that one's not cropped and it's still cake.''}
         \label{fig:cake3}
     \end{subfigure}
        \caption{\imagen{} allowed participants to rapidly explore the design space by allowing them to see different model interpretations of their prompt. Above, participant reactions to the prompt: \prompt{The word 'grow' made of cake.}}
        \label{fig:cake-choices}
        % ALT TEXT: Three images that all depict the word "Grow" in different styles. On the left, "grow" appears on top of a white and pink frosted cake written with chocolates. In the middle, "grow" appears on a plate, written using unfrosted cake in the shape of the letters. On the right, "grow" appears as frosted cakes cut out in the shape of the letters.
\end{figure}


\subsubsection{A lack of distinction between means and ends}
A key distinction of robust design materials is their ability to merge ``means,'' and ``ends'' in the design process. As Schon writes, practice in such situations ``inquiry is not limited to a deliberation about means which depends on a prior agreement about the ends \cite{schon1987educating}. [They] do not keep means and ends separate, but define them interactively as they frame a problematic situation.'' Tacitly, perhaps, participants interactively and continuously framed their work throughout the design session.\\ 
\\


For instance, they often chose to dig deeper in the design space when it seemed promising. For instance: 
\begin{description}

\item[P7] \textit{(looking at screen)} Oh we're getting something! I will share this specific URL in the chat.
\item[P8] I like the fourth one.
\item[P7] Yeah. I will start adding some text if you want to keep iterating on this… I mean, I am OK, [if we] even replace the images that we have if we come up with something more party-like.
\end{description}

In other cases, prompts also allowed participants to discover new ``ends'' through other exploration-based ``means,'' e.g., P4{}: “I do not know. I just started trying to add stuff, but I agree. The ones we come up with since are better photos.” (This pair of participants replaced images in their final design.)

As noted elsewhere, reflective practice with prompts was far from perfect -- limited visibility of prompts between partners and an inability to replicate results even with the same prompt hampered collaboration and exploration. At the same time, framing prompts as design materials offer several opportunities to understand the model-aided design process better and build tools to improve it. 


\subsection{Limitations.}
Some aspects of our study design complicate the interpretation of our findings. {We outline limitations here in three areas: participant composition, study design and analysis, and technology advancements}. 

Our participant pool was drawn among employees of one large US-based corporation, and does not cover the many possible ways that culture and training  might have shaped the design process with \LPIMs{}. {For example, they may be more comfortable collaborating remotely, as required in our study. Second, because of their choice to work in a technology firm,  it is very likely that they are more familiar with the idea of Artificial Intelligence than the general public. As a result, our findings likely are different from what might be expected with the general public. At the same time, as familiarity with AI grows in the future, it is possible that results with the general public are similar. At the same time, it is possible that our participants were more optimistic about the possibilities of technology, given their choice of employment. Because of company policies and laws, we were prevented from asking about sensitive demographic details such as race or national origin, and are unable to report differences among participants on these attributes, and if participants had differing concerns based on their identities. } 

{Our study focused on a single design task, which while representative of many tasks that non-professional designers engage in, may offer an incomplete picture of the impact of TTI models on design practices.} It was not possible to systematically observe every participant's prompt attempts, because some of those explorations were in screens that were not shared. Furthermore, our analysis is limited to observable conversations. For the interactions we \textit{could} observe, observing a designer's interactions with the model does not definitively indicate their conceptions; for example, designers who acted in similar ways even when they engaged different mental processes. Since our analysis was episodic rather than longitudinal, we are also unable to discover how design strategies  evolve within individual and pairs. 

{Finally, technological advances may lead to an evolution of some of our findings. After we conducted all our participant sessions, but before publication of this paper, new models such as Stable Diffusion were released, and led to advances like editing existing images (or parts of images) using textual prompts. Our conceptualization of prompts as design materials may extend to include these additional modalities, but future work should investigate specific ways in which such interactions influence exploration and reflection.}


We provide some comments on the growth conditions which constituted the majority of our analysis in sections \ref{sec:Hmixing} and \ref{sec:Hsigma}. In the simplest cases of Lemma \ref{lemma:unstableGrowth}, growth was established in an analogous fashion to the old one-step expansion condition (\ref{eq:oldOneStepExpansion}), finding the relevant Jacobians $M_j$ and checking that their expansion factors $K(M_j)$ satisfy
\begin{equation}
    \label{eq:discussionOneStep}
    \sum_j \frac{1}{K(M_j)} <1.
\end{equation}
For the more complicated cases, the inductive method used to establish growth near the accumulation points in Lemma \ref{lemma:unstableGrowth} and the weakened one-step expansion condition (\ref{eq:oneStep}) both address the same fundamental issue: the splitting of unstable curves by singularities into an unbounded number of small components. They circumvent this obstacle in rather different ways, however. While (\ref{eq:oneStep}) generalises (\ref{eq:discussionOneStep}) to ensure an growth of unstable curves `on average' (see \cite{chernov_statistical_2009} for a precise statement), our inductive method is a more direct adaptation of (\ref{eq:discussionOneStep}), using it to generate contradictory geometric conditions which a hypothetical non-growing unstable curve must satisfy. It may be possible to prove Theorem \ref{sec:Hmixing} using (\ref{eq:oneStep}) as the basis for growth. Since we required (\ref{eq:oneStep}) anyway for proving Theorem \ref{thm:HsigmaExp}, this could potentially condense our analysis, but only to a minor extent. A convenience of the method used in section \ref{sec:Hmixing} is that, by way of the `simple intersection' property, it naturally gives geometric information on the images of manifolds, useful for proving the property \textbf{(M)} of Theorem \ref{thm:katok-strelcyn}.

We expect that essentially analogous analysis can be applied to establish mixing properties in a wide class of piecewise linear non-uniformly hyperbolic maps, including those (like the OTM) which sit on the boundary of ergodicity and beyond. While we have relied on the precise partition structure of $H_\sigma$, its fundamental feature (self-similar sequences of elements $A^k$, sharing boundaries with its neighbours $A^{k-1},A^{k+1}$ and accumulating onto some point $p$) is quite typical to return map systems. See, for example, those of various stadium billiards \cite{chernov_chaotic_2006,chernov_improved_2008,chernov_statistical_2009} and LTMs \cite{springham_polynomial_2014}. Indeed, the same method can be used to prove the Bernoulli property for non-monotonic LTMs \cite{myers_hill_mixing_2022}, where monotonicity of the manifold images cannot be assumed and the classical argument \cite{sturman_mathematical_2006} fails. The OTM is the pointwise limit of these maps as the boundary shrinks to null measure. It further has utility in proving growth conditions for maps which are uniformly hyperbolic but possess regions $A_j$ where the hyperbolicity is very weak, signified by $K(M_j) \approx 1$, so that (\ref{eq:discussionOneStep}) fails. Typically this leads to suboptimal bounds on mixing windows, see e.g. \cite{wojtkowski_model_1981,przytycki_ergodicity_1983,myers_hill_family_2022}. The map $H_{(\eta,\eta)}$ for $\eta \approx 1/2$ is another example, possessing weak hyperbolicity over $A_2, A_3$. Letting $\varepsilon = |\eta-1/2|>0$, there is an upper bound $N = N(\varepsilon)$ on escape times from the intersections $A_2\cap \sigma, A_3 \cap \sigma$. The growth lemma then follows by applying the inductive step roughly $N$ times and can be established for arbitrarily small $\varepsilon$, opening the door to establishing optimal mixing windows.

The above gives two examples of piecewise linear perturbations to $H$ where mixing with respect to Lebesgue is preserved and our methods can be applied. Nonlinear perturbations to the shear profiles complicate the analysis in several ways. Firstly as the map's Jacobians takes on a broader range of values, cone invariance becomes an increasingly harder condition to establish. Cones must be widened, giving looser bounds on expansion factors, which may already be weak due to new regions of weaker stretching. This, together with the change from polygonal to curvilinear return time partition elements and nonlinear local manifolds, adds some complexity to showing growth conditions. This does not rule out certain (small) nonlinear perturbations however. There is some leeway in the inequalities which govern cone invariance and growth of local manifolds, the latter of which is not too dissimilar from the piecewise linear setting (see Lemmas \ref{lemma:piecewiseApprox}, \ref{lemma:componentLength}). Certain small perturbations would not alter the \emph{topological} structure of the return time partition, i.e. which elements share boundaries, the key information needed for setting up the induction. Finally while the partition elements would no longer be polygonal, only coarse geometric information is required for verifying each inductive step. Following the above, a potential perturbation could be to replace the linear portions of each shear by a cubic, perturbing the tent profile
\[  f(t) = \begin{cases} 2t & 0 \leq t \leq 1/2, \\ 2(1-t) & 1/2 \leq t \leq 1 ,\end{cases} \]
of the OTM shears to
\[  f_a(t) = \begin{cases} \frac{1}{8} t \left(16 - a + 6at - 8at^{2} \right) & 0 \leq t \leq 1/2, \\ \frac{1}{8}\left(1-t\right)\left( 16 - a + 6a\left(1-t\right) - 8a\left(1-t\right)^{2}\right)  & 1/2 \leq t \leq 1, \end{cases}   \]
for $a>0$. For small enough $a$ the gradient range $f'(t)$ is restricted to small neighbourhoods of $\{ 2, -2\}$ and the escape time partition retains a similar structure. We illustrate this in Figure \ref{fig:perturbations}, showing escapes from the square $S_3$ under the map $G \circ F$, equivalent to escapes from the perturbed $A_3$ under the $G \circ F$, but with a cleaner geometry for comparison. When $a$ is too large the analogy to the OTM breaks down. At $a=16$ the map is twice differentiable everywhere and features a new source of slowed mixing, the Jacobian is the identity at the corner points $x,y \in \{  0, 1/2 \}$ giving locally parabolic behaviour (visible in the escape time partition). 

\begin{figure}
    \centering
    \includegraphics[width=0.24 \linewidth]{0.png}
    \includegraphics[width=0.24 \linewidth]{4.png}
    \includegraphics[width=0.24 \linewidth]{8.png}
    \includegraphics[width=0.24 \linewidth]{16.png}
    \caption{Partition of escape times from $S_3$ under the mapping $F \circ G$ for $a= 0,4,8,16$. }
    \label{fig:perturbations}
\end{figure}
\section{Conclusion}\label{sec:conclusion}
In this work, we focus on addressing the fundamental challenge of OOD detection tasks, which is how to fully understand the semantic discrepancy between the ID/OOD samples. We reveal that the key to success in the realistic SCOOD task is to allocate as many ID samples in the unlabeled set correctly as possible. To this end, we propose a novel uncertainty-aware optimal transport scheme that introduces class-specific energy scores as guidance for effective label assignment. Experimental results show that our method achieves better performance than previous state-of-the-art methods on SCOOD benchmarks.

\textbf{Limitations.} In addition to temperature scaling, other techniques such as feature clipping applied in ReAct~\cite{sun2021react} also enhance the performance of energy score, so how to obtain an OOD score that best fits the SCOOD task can be further explored. Moreover, a setting highly related to SCOOD has been proposed in \cite{katz2022training} and formulated as a constrained optimization problem. We will also theoretically analyze these practical OOD settings in our feature work.

% \section*{Acknowledgments}
\textbf{Acknowledgments.} 
This work is supported by National Key R\&D Program of China under Grant 2020AAA0105701, National Natural Science Foundation of China (NSFC) under Grants 61872327, Major Special Science and Technology Project of Anhui, National Natural Science Foundation of China (62033012) and Ant Group through Ant Research Intern Program.







\section{Acknowledgments}

% Identification of funding sources and other support, and thanks to
% individuals and groups that assisted in the research and the
% preparation of the work should be included in an acknowledgment
% section, which is placed just before the reference section in your
% document.
Removed for anonymous review.

%%
%% The next two lines define the bibliography style to be used, and
%% the bibliography file.
\bibliographystyle{ACM-Reference-Format}
\bibliography{sample-base, design, sections/inprogress}


%%
%% If your work has an appendix, this is the place to put it.
\appendix
\section{Appendix for Proofs}

\paragraph{Proof of Theorem \ref{thm:main}.}

\begin{proof}
\label{proof:main}
Our proof has two steps. In Step 1, we will show that SimCLR is equivalent to minimizing the cross entropy loss defined in Eqn.~(\ref{eqn:cross-entropy}). 
In Step 2, we will show  that minimizing the cross-entropy loss 
is equivalent to spectral clustering on $\bfpi$. 
Combining the two steps together, we have proved our theorem. 

\textbf{Step 1: } SimCLR is equivalent to minimizing the cross entropy loss.

The cross-entropy loss takes expectation over 
$\bfW_\bfX\sim \mathbb{P}(\cdot ; \bfpi)$, 
which means $\bfW_\bfX$ has exactly one non-zero entry in each row $i$. By Lemma~\ref{lem:multinomial}, we know every row $i$ of $\bfW_\bfX$ is independent of other rows. Moreover, 
$\bfW_{\bfX,i}\sim \mathcal{M}(1, \bfpi_i/\sum_j \bfpi_{i,j})=\mathcal{M}(1, \bfpi_i)$, because $\bfpi_i$ itself is a probability distribution.
Similarly, we know $\bfW_\bfZ$ also has the row-independent property by sampling over $\mathbb{P}(\cdot;\bfK_\bfZ)$.
Therefore, by Lemma~\ref{lem:cross_split}, we know Eqn.~(\ref{eqn:cross-entropy}) is equivalent to:
\[
 -\sum_{i=1}^n \mathbb{E}_{\bfW_{\bfX,i}}[\log \mathbb{P}(\bfW_{\bfZ,i}=\bfW_{\bfX,i};\bfK_\bfZ)],
\]

This expression takes expectation over $\bfW_{\bfX,i}$ for the given row $i$. Notice that 
$\bfW_{\bfX,i}$ has exactly one non-zero entry, which equals $1$ (same for $\bfW_{\bfZ,i}$). 
As a result
we expand the above expression to be:
\begin{equation}
 -\sum_{i=1}^n \sum_{j\neq i} \Pr(\bfW_{\bfX,i,j}=1)\log \Pr(\bfW_{\bfZ,i,j}=1).
\label{eqn:detailed-expansion}    
\end{equation}


By Lemma~\ref{lem:multinomial}, $\Pr(\bfW_{\bfZ,i,j}=1)=\bfK_{\bfZ,i,j}/\|\bfK_{\bfZ,i}\|_1$ for $j\neq i$. Recall that $\bfK_\bfZ=(k(\bfZ_i-\bfZ_j))_{(i,j)\in[n]^2}$, which means 
$\bfK_{\bfZ,i,j}/\|\bfK_{\bfZ,i}\|_1=\frac{\exp(-\|\bfZ_i-\bfZ_j\|^2/{2\tau})}{\sum_{k\neq i}
\exp(-\|\bfZ_i-\bfZ_k\|^2/{2\tau})
}$ for $j\neq i$, when $k$ is the Gaussian kernel with variance $\tau$. 

Notice that $\bfZ_i=f(\bfX_i)$, so we know
\begin{equation}
-\log \Pr(\bfW_{\bfZ,i,j}=1)=
-\log \frac{\exp(-\|f(\bfX_i)-f(\bfX_j)\|^2/{2\tau})}{\sum_{k\neq i}
\exp(-\|f(\bfX_i)-f(\bfX_k)\|^2/{2\tau}),
}
\label{eqn:infonce-equivalence}    
\end{equation}


The right hand side is exactly the InfoNCE loss defined in Eqn.~(\ref{eqn:infonce}).
Inserting Eqn.~(\ref{eqn:infonce-equivalence}) into Eqn.~(\ref{eqn:detailed-expansion}), we get the SimCLR algorithm, which first samples augmentation pairs $(i,j)$ with $\Pr(\bfW_{\bfX,i,j}=1)$ for each row $i$, and then optimize the InfoNCE loss. 

\textbf{Step 2: } minimizing the cross entropy loss 
is equivalent to spectral clustering on $\bfpi$.


By Lemma~\ref{lem:convert_to_spectral}, we may further convert the loss to 
\begin{equation}
\label{eqn:main-theorem-repul-attr}
\min_{\bfZ}
-\sum_{(i,j)\in [n]^2} \mathbf{P}_{i,j}
\log k (\bfZ_i-\bfZ_j)+\log \mathbf{R}(\bfZ).
\end{equation}
Since $k$ is the Gaussian kernel, this reduces to \[
\min_\bfZ \mathrm{tr}(\bfZ^\top \mathbf{L}(\bfpi) \bfZ)
+\log \mathbf{R}(\bfZ),
\]

where we use the fact that $\mathbb{E}_{\bfW_\bfX\sim \mathbb{P}(\cdot; \bfpi)}[\mathbf{L}(\bfW_\bfX)]
=\mathbf{L}(\bfpi)
$, because the Laplacian operator is linear and $
\mathbb{E}_{\bfW_\bfX\sim \mathbb{P}(\cdot; \bfpi)}(\bfW_\bfX)=\bfpi
$.
\end{proof}

\paragraph{Proof of Theorem \ref{thm:clip}.}
\begin{proof}
Since $\bfW_\bfX\sim \mathbb{P}(\cdot;\bfpi_{\mathbf{A}, \mathbf{B}})$, we know 
$\bfW_\bfX$ has exactly one non-zero entry in each row, denoting the pair that got sampled. 
A notable difference compared to the previous proof is we now have $n_\mathcal{A}+n_\mathcal{B}$ objects in our graph. CLIP deals with this by taking a mini-batch of size $2N$, 
such that $n_\mathcal{A}=n_\mathcal{B}=N$, and adding the $2N$ InfoNCE losses together. We label the objects in $\mathcal{A}$ as $[n_\mathcal{A}]$, and the objects in $\mathcal{B}$ as $\{n_\mathcal{A}+1, \cdots, n_\mathcal{A}+n_\mathcal{B}\}$. 

Notice that $\bfpi_{\mathbf{A}, \mathbf{B}}$ is a bipartite graph, so the edges of objects in $\mathcal{A}$ will only connect to object in $\mathcal{B}$ and vice versa. We can define the similarity matrix in $\cZ$ as $\bfK_\bfZ$, 
where $\bfK_\bfZ(i, j+n_\mathcal{A})=\bfK_\bfZ(j+n_\mathcal{A},i)= k(\bfZ_i-\bfZ_j)$ for $i\in [n_\mathcal{A}], j\in [n_\mathcal{B}]$, and otherwise we set $\bfK_\bfZ(i,j)=0$. 
The rest is same as the previous proof. 
\end{proof}

\paragraph{Proof of Theorem \ref{thm:exponential}.}

\begin{proof}
\label{proof:exponential}
Since the objective function consists of a linear term combined with an entropy regularization, which is a strongly concave function, the maximization problem is a convex optimization problem. Owing to the implicit constraints provided by the entropy function, the problem is equivalent to having only the equality constraint. We then introduce the Lagrangian multiplier $\lambda$ and obtain the following relaxed problem:

$$
\widetilde{E}(\boldsymbol{\alpha})=\psi_{1}-\sum_{i=1}^n \alpha_{i} \psi_{i}+\tau \sum_{i=1}^n \alpha_{i}\log \alpha_{i}+\lambda\left(\boldsymbol{\alpha}^{\top} \mathbf{1}_n-1\right).
$$

As the relaxed problem is unconstrained, taking the derivative with respect to $\alpha_{i}$ yields

$$
\frac{\partial \widetilde{E}(\boldsymbol{\alpha})}{\partial \alpha_{i}}=-\psi_{i}+\tau\left(\log \alpha_{i}+\alpha_{i} \frac{1}{\alpha_{i}}\right)+\lambda=0.
$$

Solving the above equation implies that $\alpha_{i}$ takes the form
$
\alpha_{i}=\exp \left(\frac{1}{\tau} \psi_{i}\right) \exp \left(\frac{-\lambda}{\tau}-1\right).
$ Since $\alpha_{i}$ lies on the probability simplex, the optimal $\alpha_{i}$ is explicitly given by
$
\alpha^{*}_{i}=\frac{\exp \left(\frac{1}{\tau} \psi_{i}\right)}{\sum_{i^{\prime}=1}^n \exp \left(\frac{1}{\tau} \psi_{i^{\prime}}\right)} .
$ Substituting the optimal point into the objective function, we obtain
$$
\begin{aligned}
E\left(\boldsymbol{\alpha}^*\right)  &=\psi_1-\sum_{i=1}^n \frac{\exp \left(\frac{1}{\tau} \psi_{i}\right)}{\sum_{i^{\prime}=1}^n \exp \left(\frac{1}{\tau} \psi_{i^{\prime}}\right)} \psi_{i}+\tau \sum_{i=1}^n \frac{\exp \left(\frac{1}{\tau} \psi_{i}\right)}{\sum_{i^{\prime}=1}^n \exp \left(\frac{1}{\tau} \psi_{i^{\prime}}\right)}\log \frac{\exp \left(\frac{1}{\tau} \psi_{i}\right)}{\sum_{i^{\prime}=1}^n \exp \left(\frac{1}{\tau} \psi_{i^{\prime}}\right)} \\
& =\psi_1 - \tau \log \left(\sum_{i=1}^n \exp \left(\frac{1}{\tau} \psi_{i}\right)\right).
\end{aligned}
$$
Thus, the Lagrangian dual function is given by
\begin{equation*}
-E\left(\boldsymbol{\alpha}^*\right)= -\tau \log \frac{\exp \left(\frac{1}{\tau} \psi_{1}\right)}{\sum_{i=1}^n \exp \left(\frac{1}{\tau} \psi_{i}\right)}.\qedhere
\end{equation*}
\end{proof}



\section{More on Experiments} \label{section: experiment_details}

\paragraph{CIFAR-10 and CIFAR-100} CIFAR-10 ~\citep{krizhevsky2009learning} and CIFAR-100 ~\citep{krizhevsky2009learning} are well-known classic image classification datasets. Both CIFAR-10 and CIFAR-100 contain a total of 60k $32 \times 32$ labeled images of different classes, with 50k for training and 10k for testing. CIFAR-10 is similar to CIFAR-100, except there are 10 different classes in CIFAR-10 and 100 classes in CIFAR-100.

\paragraph{TinyImageNet} TinyImageNet ~\citep{le2015tiny} is a subset of ImageNet ~\citep{deng2009imagenet}. There are 200 different object classes in TinyImageNet, with 500 training images, 50 validation images, and 50 test images for each class. All the images in TinyImageNet are colored and labeled with a size of $64 \times 64$.

\textbf{Pseudo-code.} Algorithm \ref{alg:Training Procedure} presents the pseudo-code for our empirical training procedure.

\begin{algorithm}[!htbp]
\caption{Training Procedure}
\label{alg:Training Procedure}
\begin{algorithmic}[1]
\REQUIRE trainable encoder network $f$, batch size $N$, augmentation strategy \textit{aug}, loss function $L$ with hyperparameters \textit{args}
\FOR {sampled minibatch ${x_i}_{i=1}^N$}
\FORALL{$i \in { 1, ..., N }$}
\STATE draw two augmentations $t_i = \textit{aug}\left(x_i\right) $, $t_i' = \textit{aug}\left(x_i\right) $
\STATE $z_i = f\left(t_i\right)$, $z_i' = f\left(t_i'\right)$
\ENDFOR
\STATE compute loss $\mathcal{L} = L(N, z, z', \textit{args})$
\STATE update encoder network $f$ to minimize $\mathcal{L}$
\ENDFOR
\STATE \textbf{Return} encoder network $f$
\end{algorithmic}
\end{algorithm}

We also provide the pseudo-code for our core loss function used in the training procedure in Algorithm \ref{alg:Core loss}. The pseudo-code is almost identical to SimCLR's loss function, with the exception of an extra parameter $\gamma$.

\begin{algorithm}[!htbp]
\caption{Core loss function $\mathcal{C}$}
\label{alg:Core loss}
\begin{algorithmic}[1]
\REQUIRE batch size $N$, two encoded minibatches $z_1, z_2$, $\gamma$, temperature $\tau$
\STATE $z = \textit{concat}\left(z_1, z_2\right)$
\FOR {$i \in {1, ..., 2N }, j \in {1, ..., 2N}$ }
\STATE $s_{i,j} = \Vert z_i - z_j \Vert_2^{\gamma}$
\ENDFOR
\STATE \textbf{define} $l(i, j)$ \textbf{as} $l(i, j) = - \log \frac{exp\left(s_{i,j}/\tau \right)}{\sum_{k=1}^{2N} \mathbf{1}{[k \ne i]} exp\left(s{i, j} / \tau \right)} $
\STATE \textbf{Return} $\frac{1}{2N} \sum_{k=1}^N\left[l(i, i+N) + l(i+N, i)\right]$
\end{algorithmic}
\end{algorithm}

Utilizing the core loss function $\mathcal{C}$, we can define all kernel loss functions used in our experiments in Table \ref{table: loss definition}. For all $z_i \in z$ with even dimensions $n$, we define $z_{L_i} = z_i\left[0:n/2\right]$ and $z_{R_i} = z_i\left[n/2:n\right]$.

\begin{table}[ht]
\centering
\begin{tabular}{{@{}l|l@{}}}
Kernel  &  Loss function \\ \midrule
Laplacian & $\mathcal{C}\left(N, z, z', \gamma=1, \tau\right)$\\ \midrule
Sum       & $\lambda * \mathcal{C}\left(N, z, z', \gamma=1, \tau_1\right) + (1-\lambda) * \mathcal{C}\left(N, z, z', \gamma=2, \tau_2\right)$  \\ \midrule
Concatenation Sum&$\lambda * \mathcal{C}\left(N, z_L, z'_L, \gamma=1, \tau_1\right) + (1-\lambda) * \mathcal{C}\left(N, z_R, z'_R, \gamma=2, \tau_2\right)$\\ \midrule
$\gamma = 0.5$ & $\mathcal{C}\left(N, z, z', \gamma=0.5, \tau\right)$          \\ 

\end{tabular}

\caption{Definition of kernel loss functions in our experiments}
\label {table: loss definition}
\end{table}

\textbf{Baselines.} We reproduce the SimCLR algorithm using PyTorch Lightning~\citep{PytorchLightning}.

\textbf{Encoder details.}
The encoder $f$ consists of a backbone network and a projection network. We employ ResNet50~\citep{ResNet} as the backbone and a 2-layer MLP (connected by a batch normalization~\citep{ioffe2015batch} layer and a ReLU \cite{nair2010rectified} layer) with hidden dimensions 2048 and output dimensions 128 (or 256 in the concatenation kernel case).

\textbf{Encoder hyperparameter tuning.}
For each encoder training case, we randomly sample 500 hyperparameter groups (sample details are shown in Table \ref{table: Hyperparameter sample}) and train these samples simultaneously using Ray Tune ~\citep{RayTune}, with the ASHA scheduler~\citep{li2018massively}. Ultimately, the hyperparameter group that maximizes the online validation accuracy (integrated in PyTorch Lightning) within 5000 validation steps is chosen for the given encoder training case.

\begin{table}[ht]
\centering

\begin{tabular}{@{}l|l|l@{}}
\midrule
Hyperparameter  & Sample Range & Sample Strategy \\ \midrule
start learning rate & $\left[10^{-2}, 10\right]$ & log uniform \\ \midrule
$\lambda$       & $\left[0, 1\right]$ & uniform \\ \midrule
$\tau$, $\tau_1$, $\tau_2$ & $\left[0, 1\right]$ & log uniform \\ \midrule
\end{tabular}

\caption{Hyperparameters sample strategy}
\label {table: Hyperparameter sample}
\end{table}

\textbf{Encoder training.} 
We train each encoder using the LARS optimizer~\citep{LARSOptimizer}, LambdaLR Scheduler in PyTorch, momentum 0.9, weight decay $10^{-6}$, batch size 256, and the aforementioned hyperparameters for 400 epochs on a single A-100 GPU.

\textbf{Image transformation.} The image transformation strategy, including augmentation, is identical to the default transformation strategy provided by PyTorch Lightning.

\textbf{Linear evaluation.}
The linear head is trained using the SGD optimizer with a cosine learning rate scheduler, batch size 64, and weight decay $10^{-6}$ for 100 epochs. The learning rate starts at $0.3$ and ends at $0$.

\textbf{Moco Experiments.} We also tested our method based on MoCo~\citep{he2019moco}. The results are summarized in Table \ref{tab:results-moco}. Here we choose ResNet18~\citep{ResNet} as the backbone and set a temperature of $0.1$ as default. For our simple sum kernel, we set $\lambda=0.8$. The results show that our method outperforms the original MoCo method.

\begin{table}[thb]
\centering
\caption{MoCo Experiment Results on CIFAR-10 and CIFAR-100.}
\label{tab:results-moco}
\resizebox{\textwidth}{!}{%
\begin{tabular}{@{}c|ccc|ccc@{}}
\toprule
\multirow{3}{*}{Method} & \multicolumn{3}{c|}{CIFAR-10} & \multicolumn{3}{c}{CIFAR-100} \\ \cmidrule(lr){2-4} \cmidrule(lr){5-7} 
                        & 200 epochs & 400 epochs    & 1000 epochs   & 200 epochs & 400 epochs & 1000 epochs         \\ \midrule
MoCo (repro.)         & $76.41 \pm 0.12$    & $80.01 \pm 0.15$          & $84.45 \pm 0.08$    & $\mathbf{47.02 \pm 0.11}$ & $52.50 \pm 0.07$ & $57.62 \pm 0.15$            \\
\midrule
Laplacian Kernel        & ${78.09 \pm 0.10}$    & $\mathbf{83.85 \pm 0.09}$          & $\mathbf{88.34 \pm 0.16}$    & $46.12 \pm 0.22$   & $53.44 \pm 0.17$ & $59.10 \pm 0.14$        \\
Simple Sum Kernel & $\mathbf{78.12 \pm 0.15}$   & $83.23 \pm 0.18$ & $87.50 \pm 0.20$ & $46.65 \pm 0.06$ & $\mathbf{53.62 \pm 0.19}$ & $\mathbf{59.83 \pm 0.12}$\\
\bottomrule
\end{tabular}
}
\end{table}



\section{More Experiments on Synthetic Data}


Consider a scenario with $n$ clusters, each containing $k$ vertices. Let the probability of vertices $u$ and $v$ from the same cluster belonging to $\bfpi$ be $p$. Conversely, for vertices $u$ and $v$ from different clusters, let the probability of belonging to $\pi$ be $q$. We generate the graph $\bfpi$ randomly, based on $p$ and $q$. We experiment with values of $k=100$ and $n=6$ for ease of visualization, embedding all points in a two-dimensional space. Each vertex's initial position originates from a normal distribution. In each iteration, we sample a subgraph of $\bfpi$ uniformly, ensuring each vertex has an out-degree of $1$. We then optimize the corresponding vectors using InfoNCE loss with an SGD optimizer and iterate until convergence. Our experimental setup consists of an SGD learning rate of $1$, an InfoNCE loss temperature of $0.5$, and a batch size of $50$. We evaluate two scenarios with different $p$ and $q$ values: $p=1$, $q=0$, and $p=0.75$, $q=0.2$. The results of these experiments are visualized in Figure \ref{fig:vis-spectral-cluster}. The obtained embeddings exhibit the hallmark pattern of spectral clustering of graph $\bfpi$.

\begin{figure}[!tb]
\centering
\subfigure{
\includegraphics[width=1\textwidth]{Figures/cluster_pi.png}
\label{fig:vis-cluster}
}
\subfigure{
\includegraphics[width=1\textwidth]{Figures/noised_cluster_pi.png}
\label{fig:vis-noised-cluster}
}
\caption{Visualizations of the optimization process using InfoNCE Loss on the vectors corresponding to $\bfpi$. Points of identical color belong to the same cluster within $\bfpi$. To showcase the internal structure of $\bfpi$, we randomly select 10 vertices from each cluster to display the edge distribution of $\bfpi$.}
\label{fig:vis-spectral-cluster}
\end{figure}


\end{document}
\endinput
%%
%% End of file `sample-manuscript.tex'.
