\section{Introduction} 
Recent advances in  \LPIMs{} (TTI models), allow users to generate high-quality images based on a text description or ``prompt''. In ways similar to large language models (LLMs), where increasing size of models has discontinuous benefits~\cite{wei2022emergent}, the increased size of recent \LPIMs{} has yielded discontinuous and qualitative differences in the quality of images produced~\cite{saharia2022Imagen}. The quality of \LPIMs{} has led to energetic communities of practice, where enthusiasts readily share designs and prompts. In some cases, results obtained from these models are so good that one artist has even won a prestigious art competition \cite{AIwinsst11:online}. 

The image quality obtained by this new generation of \LPIMs{} enables the research community to rethink significant aspects of the design process in light of these models. For example, questions arise about whether designers could delegate parts of the creative process to a model; how the nature and the style of creative design change with these models; and how designers should best share and collaborate on multimodal (i.e., text and image) creative processes with these models available as design resources. Finally, potential new design processes among non-professional designers are particularly interesting, as TTI models allow those without professional training to create high-quality images easily.


% In particular, non-professional designers benefit from being able to rapidly prototype, see and borrow from examples, and share multiple designs \cite{dow2010parallel}. Since TTI models may empower those without professional artistic or design training to create images with a textual description alone rapidly, we are motivated to study how practices of non-professional designers might change. %Could be condensed and moved to section 3, (before 3.1) if we want to keep this. CK NOTE: Added to section 3.


This paper investigates how the creative design processes of non-professional designers change while using TTI models. Since design practices often emerge through social interactions (e.g., designers working together in a studio or sharing their work online), we also investigate how \textit{social interactions} are affected with the introduction of a TTI model. Recent work suggests that generative AI may play a role in influencing social dynamics between pairs of people~\cite{suh2021GenerativeMusic}. With the advent of ``prompting'' as a new form of interacting with AI, we build on this emerging body of literature by investigating how prompting may affect social dynamics during collaborative design. 


Our paper examines the following two research questions: 
\begin{itemize}
    \item RQ1: How does using prompt-based image generation change the design process of non-professional designers, especially compared to current tools for finding appropriate images such as web search?
    \item RQ2: How do prompt-based image generation models change collaborative dynamics during design?
\end{itemize}


Note that while prior work on human-AI co-creative systems has uncovered challenges that frequently arise when users co-create with AI (e.g., users must deal with uncertainty in the capabilities of AI capabilities while simultaneously making sense of complex outputs from the system \cite{yang2020DesignAI}), it is unknown how these challenges manifest when users interact with prompt-based i.e., TTI models. It is also unclear what other challenges are unique to working with these models and how we might best address them when designing new interfaces for these systems. 


To address these research questions, we conducted a design study with participants from a large technology company who use prompt-based image models in a non-professional capacity. In our controlled study, participants worked in pairs and created graphic designs with or without the assistance of a prompt-based image generation model. We present results in this paper from our direct observations of the designer pairs, conversations during this design session, and post-study interviews. We also compare the artifacts produced by participants for creativity, completeness, and appropriateness to the design brief. 

Overall, we found that TTI models change the design process by allowing designers to create images \textit{declaratively}, i.e., through a simple description of the desired image. This declarative design changes existing design practices in two ways. First, because prompts can capture high-level image descriptions, TTI models allow faster exploration of the design space, potentially leading to more creative design. {(At the same time, we acknowledge that not all creative ideas for generating images can be expressed easily, or at all, in words. As new text-guided image-to-image models become more capable, we expect that they will aid exploration even further.)} Second, because prompts are text, prompt-based image generation leads to easier sharing of design ideas, allowing designers to collaborate and build on each others' work more successfully.  

Throughout our study, we observed that prompts played a central role in the design process and collaboration. This leads us to argue that prompts act as a \textit{reflective design material} in the design process. Specifically, we found that our participants developed a \textit{tacit}, rather than technical, understanding of how different aspects of prompts (such as specific keywords) influence the image generated. 

In addition, prompts enabled fluid collaborations between participants as they shared, modified, and iteratively improved each others' prompts. The ability to easily edit and refine an image via a text interface made the design process more fluid, and sharing prompts easily aided collaboration, which uniquely placed prompts as design material in a multimodal creative setting.

Prompts also allowed participants to engage in reflective practice with the AI model. Specifically, the \LPIM{} outputs made it seem like a somewhat opinionated ``design partner'' in its preference to generate certain kinds of images, or when the images generated were wildly different from the prompt's intent or were challenging to modify. Participants leveraged prompts to reflect on the model's results  
% In their ability to ``talk back'' to the designer, prompts for TTI models are also similar to Turkle's notion of  where the designer engages in an iterative process where they get to refine both their creative ideas and their ways to express them via prompts based on the model's response. 
and engaged in reflective conversations with each other through their prompts, envisioning novel designs they did not foresee~\cite{schon1984architectural,ghajargar2018thinking}. This leads us to suggest that prompts act as reflective design materials (``objects to think with'' \cite{turkle2005second}).

In sum, this paper makes the following contributions:
\begin{itemize}
 
\item {\textbf{Changes in the design process.} We articulate ways in which TTI models change the design process. Specifically, models changed design processes through faster iteration, creating images of novel ideas and unlikely combinations (e.g., a giraffe in a Lamborghini) quickly. They also changed the design process because they rely on \textit{indirect} manipulation of images (through text). Together, this led to wide-ranging exploration where the models rapidly ``filled in'' many unspecified details of the images.  However, participants also struggled to control low-level factors (e.g., cropping, position, text) in the generated image. In sum, users perceived their work to be significantly more creative when using a TTI model compared to image search. However, external raters found no significant difference. }

\item {\textbf{Changes in collaborative practices.} We identify ways prompts changed collaborative dynamics during design. While prompts empowered collaborators to combine multiple people's disparate ideas fluidly, their non-determinism hindered coordination. In addition, because participants frequently shared images without the prompts that generated them, partners had asymmetric access to TTI models.}

\item \textbf{Prompts as a design material.} {Given these empirical results, we  conceptualize prompts as a new design material that both enables rapid exploration of a design space and modulates collaboration, but one that is imperfect, especially for fine-grained control.} 
\end{itemize}
% Together, our results illustrate prompt-based image generation models' role in a collaborative design process. As other multimodal AI models are developed, such as image-to-image models whose outputs can be controlled with a text prompt, we expect that our conceptualizations of prompts as design materials and how they change design processes and collaboration will inform how future design tools could be designed and how users might interact with them. 


