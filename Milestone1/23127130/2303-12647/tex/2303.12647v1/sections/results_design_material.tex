
\subsection{Prompts as reflective design material}
Throughout their design session with \IS{}, it seemed that participants merely saw \IS{} as a way to find the needed images. In contrast, when participants used \imagen{}, they displayed a nuanced, functional understanding of how prompts could be used to achieve their design objectives. Moreover, this understanding was not related to the technical aspects of how \imagen{} worked -- not once in our sessions or interviews did participants mention ``transformers'', ``diffusion models'', or even ``deep learning.'' Instead, they spoke about and enacted how  \imagen{} allowed them to rapidly explore a range of artistic possibilities and to collaborate.

These observations lead us to characterize prompts as \textit{design materials}. Below, we describe how participants exhibited a tacit understanding of \imagen{} and how prompts allowed for exploration and reflection on model actions (i.e., images generated) and reflection in action (i.e., through collaboratively editing the prompts).


Participants used and developed their tacit mental models of \Imagen{}'s design orientation throughout their design process. For instance, P3{} noticed how \Imagen{} framed the subjects in its images: ``Most of the images that get generated by \Imagen{} always push everything up to the front.'' Sometimes, participants tried to compensate for what they believed the model did not understand. P6{} noticed, for instance, ``It seems like it does not know what the Cheshire Cat is," changing their prompt from \prompt{An illustration of the Cheshire Cat from 'Alice in Wonderland'} to \prompt{An illustration of a cat with a large face smiling and looking at the camera}. Finally, participants sometimes generated images mostly to test what \Imagen{} might do with a prompt. For instance, P4{} said to their partner: “Oh, we could try that with like `1969 poster' or no… because the poster will make Imagen try to…? Let’s try that. `1969 poster'.” Then, examining the results, they decided: ``These are like the very artsy side which is probably less what we want. But they are still fun.”

\subsubsection{Prompts allow rapidly exploring the design space} 

In their role as reflective design materials, prompts allowed participants to rapidly explore their design's content, style, and layout. 
For instance, many participants opened multiple instances of \Imagen{} (in different browser tabs) to explore variations of a prompt, such as \prompt{Drawing of a Cheshire cat from Alice in Wonderland. Cartoon}, \prompt{Drawing of a Cheshire cat from Alice in Wonderland. Psychedelic}, and \prompt{Colorful drawing of a Cheshire cat from Alice in Wonderland. Cartoon.} 

Participants made these decisions with fluidity, interlaying decisions of content and style while navigating the limits of the model: 

\begin{description}
\item[P12{}] Yeah... it might be hard to get Alice eating cake. 

\item[P11{}] ...yeah 

\item[P12{}] maybe we could do something with like `the cake from Alice in Wonderland'.

\item[P11{}] Yeah. Yeah. Maybe if we can't get Alice's face out of it than we could use Alice's face... like a non copyrighted one from Google.

\item[P11{}] Yeah. And then we could probably just do like, a cake that says `eat me' on it, right? 

\item[P12{}] Hmm, yeah. You have like a style that we want for that?

\item[P11{}] It definitely, it's got to be like a cartoon one, at least. So we don't want it photorealistic.
\end{description}


Finally, because \imagen{} displayed multiple candidates per prompt, participants also could explore their design choices based on the results they obtained (see Figure~f{fig:cake-choices}). 

\begin{figure}[htb]
     \centering
     \begin{subfigure}[b]{0.3\textwidth}
         \centering
         \includegraphics[width=\textwidth]{\imagepath{grow-cake.png}}
         \caption{``Yeah, it's like it's still a little bit cropped''}
         \label{fig:grow-cake}
     \end{subfigure}
     \hfill
     \begin{subfigure}[b]{0.3\textwidth}
         \centering
         \includegraphics[width=\textwidth]{\imagepath{grow-on-plate.png}}
          \caption{``Maybe the one on the plate...yeah, I think we should crop it around `grow'''}
         \label{fig:cake2}
     \end{subfigure}
     \hfill
     \begin{subfigure}[b]{0.3\textwidth}
         \centering
         \includegraphics[width=\textwidth]{\imagepath{chosen-grow.png}}
          \caption{``Oh, oh...  that one's not cropped and it's still cake.''}
         \label{fig:cake3}
     \end{subfigure}
        \caption{\imagen{} allowed participants to rapidly explore the design space by allowing them to see different model interpretations of their prompt. Above, participant reactions to the prompt: \prompt{The word 'grow' made of cake.}}
        \label{fig:cake-choices}
        % ALT TEXT: Three images that all depict the word "Grow" in different styles. On the left, "grow" appears on top of a white and pink frosted cake written with chocolates. In the middle, "grow" appears on a plate, written using unfrosted cake in the shape of the letters. On the right, "grow" appears as frosted cakes cut out in the shape of the letters.
\end{figure}


\subsubsection{A lack of distinction between means and ends}
A key distinction of robust design materials is their ability to merge ``means,'' and ``ends'' in the design process. As Schon writes, practice in such situations ``inquiry is not limited to a deliberation about means which depends on a prior agreement about the ends \cite{schon1987educating}. [They] do not keep means and ends separate, but define them interactively as they frame a problematic situation.'' Tacitly, perhaps, participants interactively and continuously framed their work throughout the design session.\\ 
\\


For instance, they often chose to dig deeper in the design space when it seemed promising. For instance: 
\begin{description}

\item[P7] \textit{(looking at screen)} Oh we're getting something! I will share this specific URL in the chat.
\item[P8] I like the fourth one.
\item[P7] Yeah. I will start adding some text if you want to keep iterating on this… I mean, I am OK, [if we] even replace the images that we have if we come up with something more party-like.
\end{description}

In other cases, prompts also allowed participants to discover new ``ends'' through other exploration-based ``means,'' e.g., P4{}: “I do not know. I just started trying to add stuff, but I agree. The ones we come up with since are better photos.” (This pair of participants replaced images in their final design.)

As noted elsewhere, reflective practice with prompts was far from perfect -- limited visibility of prompts between partners and an inability to replicate results even with the same prompt hampered collaboration and exploration. At the same time, framing prompts as design materials offer several opportunities to understand the model-aided design process better and build tools to improve it. 


\subsection{Limitations.}
Some aspects of our study design complicate the interpretation of our findings. {We outline limitations here in three areas: participant composition, study design and analysis, and technology advancements}. 

Our participant pool was drawn among employees of one large US-based corporation, and does not cover the many possible ways that culture and training  might have shaped the design process with \LPIMs{}. {For example, they may be more comfortable collaborating remotely, as required in our study. Second, because of their choice to work in a technology firm,  it is very likely that they are more familiar with the idea of Artificial Intelligence than the general public. As a result, our findings likely are different from what might be expected with the general public. At the same time, as familiarity with AI grows in the future, it is possible that results with the general public are similar. At the same time, it is possible that our participants were more optimistic about the possibilities of technology, given their choice of employment. Because of company policies and laws, we were prevented from asking about sensitive demographic details such as race or national origin, and are unable to report differences among participants on these attributes, and if participants had differing concerns based on their identities. } 

{Our study focused on a single design task, which while representative of many tasks that non-professional designers engage in, may offer an incomplete picture of the impact of TTI models on design practices.} It was not possible to systematically observe every participant's prompt attempts, because some of those explorations were in screens that were not shared. Furthermore, our analysis is limited to observable conversations. For the interactions we \textit{could} observe, observing a designer's interactions with the model does not definitively indicate their conceptions; for example, designers who acted in similar ways even when they engaged different mental processes. Since our analysis was episodic rather than longitudinal, we are also unable to discover how design strategies  evolve within individual and pairs. 

{Finally, technological advances may lead to an evolution of some of our findings. After we conducted all our participant sessions, but before publication of this paper, new models such as Stable Diffusion were released, and led to advances like editing existing images (or parts of images) using textual prompts. Our conceptualization of prompts as design materials may extend to include these additional modalities, but future work should investigate specific ways in which such interactions influence exploration and reflection.}

