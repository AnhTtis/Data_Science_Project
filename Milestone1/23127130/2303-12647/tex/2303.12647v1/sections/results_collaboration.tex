\subsection{How text-to-image models changed collaborative dynamics during design}
{Text-to-image models} modulated the collaborative practices among participants by creating new ways to fluidly combine ideas with prompts. At the same time, because prompts were so central to these collaborations, asymmetric access to the prompts changed collaborative roles and exploration.

\subsubsection{Prompts allow participants to fluidly combine ideas}
A core aspect of creative collaboration is the ability to combine, re-mix, and try out ideas from multiple people \cite{fauconnier2008way}. However, while using \IS{}, participants sometimes discovered that, even when they could agree and combine their ideas, those combinations of ideas were often hard to find within the search results. For instance, one participant noted: ``It was easy for us to sort of like agree and collaborate on ideas but then it was hard to find images that match those ideas.'' Similarly, during their design session, P4 said to their partner: ``I like the one that you had with the crazy paper vintage background,'' but later was unable to find images of candles in that preferred style: “something about beggars cannot be choosers.” 


In contrast to \IS{}, with \imagen{}, participants were both able to combine ideas in their prompts and experiment rapidly with different ways of composing them together. For instance, in this conversation, P7{} fluidly added his ideas for fireworks to their prompt about rockets on the moon: 
\begin{description}
\item[P8{}] Another theme could be something to do with rockets.
\item[P7{}] Oh yeah, or like rocket fireworks. [Prompt: \prompt{Fireworks exploding in the shape of a space shuttle.}]
\end{description}


Furthermore, \imagen{} allowed participants to see a variety of \textit{generated} images and choose the ones that best matched their needs. Reacting to a set of \imagen{} results based on their partner's query, P6{} said: ``It [image on the left] does not get the idea of the party across. Let's go with the one on the right because it has like the  astronaut has a party hat.'' 



Finally, even though not this was not the focus of our study, participants often spoke about how they learned tricks for successful prompting socially. For instance, P5{} suggested how this process of social learning was fun: ``Like, it could be fun, especially when me and my coworkers are all sitting at my desk and people like, oh, take this [prompt] and see what it does.'' In many of our design sessions, we saw many such prompt modifications, such as using \prompt{...framed art}, or particular camera or lens types to mimic in the images generated, such as \prompt{...Sigma 85mm f1.4}. Once participants shared such prompt tricks with their partner, they often used them in their collaborative design work. 

\subsubsection{Asymmetric access to prompts, randomized generation,  hinder collaboration}
Often, the ability to iterate on a design was weighted towards whichever collaborator had access to the prompt, leading to asymmetric access. This was particularly prominent in situations where participants prototyped prompts in windows that were not shared with their partner, as in this exchange:

\begin{description}
\item[P1] \textit{(chuckles)} Okay, well I got something which will be sort of, kind of more appropriate maybe. So I'm gonna paste it here  
\item[P2]  \textit{(seeing the results)} Hey! That's pretty good. Okay….Yeah, even the Lander is partying! I think we go with this one.
\end{description}

In this situation, even though P1{} was able to share an exciting image result with P2{}, P2{} was not able to iterate on the design because he did not have access to the prompt. In this case, we noticed P2{} became increasingly reliant on P1{} to create images as the design session progressed. 

Even with access to prompts, generative models (including \imagen{}) typically use a random seed as input, so users see different image results on consecutive runs of the model, even when providing an identical prompt input. As a result, participants were sometimes unable to replicate previous results reliably, hampering collaboration.
