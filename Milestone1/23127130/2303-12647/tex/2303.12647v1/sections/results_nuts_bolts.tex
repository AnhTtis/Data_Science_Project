\section{Results}
 In this section, we first present the participants' design outputs, as well as quantitative results and interaction patterns. Then, in the following sections, we describe how \LPIMs{} changed the design process, and affected collaborative dynamics during design.
 
Note that even though participants in the \imagen{} condition were allowed to use images found through  using \IS{}, only one pair of participants did so. To simplify our description of results, we therefore describe them as results while using \imagen{} and while using \IS{}, even though participants using \imagen{} always had access to \IS{}.

\subsection{Design Outputs and Quantitative Findings}
{Qualitative results regarding the processes followed by participants is possibly the larger contribution of our work. However, for the sake of completeness and to offer a statistical overview of participant behaviors and preferences, we briefly mention quantitative results below.}

Nearly all participants spent all the available time during their first session (regardless of whether they used \IS{} or \imagen{}), $M=18.5$ minutes. Many participants used less time in the second design session, regardless of condition, $M=14.2$ minutes, suggesting there was a learning effect in the task. However, there was no significant difference between \IS{} and \imagen{} conditions.   

When working with their partner, participants followed different collaborative styles, which we briefly describe in Section~f{sub:collaboration}. Participants typically brainstormed about the general theme of their design (characters from Alice in Wonderland), and then started to look for images (using \IS), or generated images (using \imagen{}) that fit this theme. While some pairs had one screen shared with one of the participants ``driving'' the integration of images into the final design, while the other looked for images, many pairs  worked collaboratively on editing the prompts to generate images. Finally, some pairs worked in parallel, sharing interesting results with each other. These pairs then collaboratively edited the slide deck to create the final design.
 
Figure~f{fig:cake-prompts} shows some of the final invitations the participants created. Participants' invitations created in the \IS{} condition had an average of $4$ images in the design ($median=4$), while those in the \imagen{} condition had $2.4$ images on average ($median=2$). We discuss possible reasons for this difference in Section~f{ssub:opinionated}.

Participants self-reported their final design to be more creative when they used \imagen{} (5-point Likert scale, mean=3.6) than when they used \IS{} alone (M=3). This improvement (M=0.6) was statistically significant ({two-sided paired} $t(13)=2.65, p=0.02$.) There was no difference in how complete participants reported their creations to be (\IS{} $M=3.6$, \Imagen{} $M=3.2$), or how appropriate it was to the design brief  (\IS $M=4.2$, \Imagen{} $M=4.1$). On the other hand, external raters did not find any significant differences in the creativity, completeness, or appropriateness of the brief for designs in either condition. 

In a post-study survey, participants did not note any significant differences in their ability  ``to manage any relationship tension'' in their work group (paired t-test, $p=0.19$), ``to politely include my partner’s ideas in the final design while also preserving my own'' ({two-sided} paired t-test, $p=0.27$), or ``to decide about who should do what in our group, even when we had some differences in opinion'' ({two-sided} paired t-test, $p=0.16$). However, participants did reveal a preference for using \imagen{} or a similar model were they ``to complete a similar task in the future'' (mean rating $=3.8$, median $=4$, on a 1-5 Likert scale, \textit{5=``Strongly prefer to use \Imagen{}/similar
model''}.)  

Given these self-reported preferences for using \imagen{} and similar models, along with modest differences in the output quality, we focus the rest of this section on the qualitative differences in the processes that participants employed.


\subsection{Interaction Patterns}

During the study, we observed differences in how participants queried or prompted \IS{} and \imagen{}. Participants understood that \IS{} found pre-existing images on the Internet and so used broad queries that they hoped would yield useful results (e.g. \prompt{tea party}.) If these queries did not yield relevant results, participants searched for related terms instead (e.g., \prompt{mad hatter} $\rightarrow$  \prompt{hare with a hat} $\rightarrow$ \prompt{crazy top hats}). As such, participants (correctly) used \IS{} as a \textit{querying} interface. In contrast, participants' inputs to \Imagen{} could best be described not as \textit{queries} but as \textit{descriptions}, such as \prompt{Colorful drawing of a Cheshire cat from Alice in Wonderland. The cat is wearing a birthday hat and is on a white background.} Throughout the rest of this paper, we call these input descriptions \textit{prompts} to distinguish them from queries.

Participants wrote increasingly elaborate prompts with \Imagen{} during their design session, especially when the image results were disappointing. For instance, P7{} and P8{} tried the prompt \prompt{Beach party on the moon, on the moon in the Sea of Tranquility. Digital art.} Unfortunately, \Imagen{} did not generate any images for this query, and participants hypothesized this was because the beach party had nudity or other content that led  \Imagen{} to block it. (In actuality, it is likely these images were blocked because \imagen{} does not generate images with photo-realistic people in them.) These participants then modified their prompt several times, ending with \prompt{A doodle of a beach party of fully suited astronauts on the Moon in the Sea of Tranquility. The Sea of Tranquility has water in it, and some astronauts are surfing in it with surfboards that have the "NASA" logo on them. Digital art.}




% TODO (address Shaun's feedback to briefly describe nuts and bolts of how they collaborated in 1-2 sentences -- what were some patterns for how they did division of labor?). A typical pattern was to do XYZ, sharing interesting results with each other, and pasting promising images into Slides to curate later. While some pairs did X, other pairs did Y.}

% 
\begin{figure}
     \centering
     \begin{subfigure}[t]{0.3\textwidth}
         \centering
         \includegraphics[width=\textwidth]{\figpath{image-search/Dave-Ian}}
         \label{fig:image-search-1}
     \end{subfigure}
     \hfill
     \begin{subfigure}[t]{0.3\textwidth}
         \centering
         \includegraphics[width=\textwidth]{\figpath{image-search/Emily-Hendrik}}
         \label{fig:image-search-2}
     \end{subfigure}
     \hfill
     \begin{subfigure}[t]{0.3\textwidth}
         \centering
         \includegraphics[width=\textwidth]{\figpath{image-search/Bardia-Dillon}}
     \end{subfigure}
     
     \begin{subfigure}[t]{0.3\textwidth}
         \centering
         \includegraphics[width=\textwidth]{\figpath{imagen/Emily-Hendrik}}
         \label{fig:imagen-1}
     \end{subfigure}
     \hfill
     \begin{subfigure}[t]{0.3\textwidth}
         \centering
         \includegraphics[width=\textwidth]{\figpath{imagen/dave-ian}}
         \label{fig:imagen-2}
     \end{subfigure}
     \hfill
     \begin{subfigure}[t]{0.3\textwidth}
         \centering
         \includegraphics[width=\textwidth]{\figpath{imagen/peter-justin}}
     \end{subfigure}
        \caption{A few of the designs participants created in our study. \textit{Top row}: Designs without access to prompt-based image model; \textit{Bottom row}: Designs with access to prompt-based image model. {(In both conditions, designs shown are ones with the highest overall rating by independent experts.)}}
        \label{fig:cake-prompts}
        % ALT TEXT: Six images of birthday cards. Top row includes cards designed using image search. On the top left is an Alice In Wonderland themed card containing a rabbit holding a bugle and wearing the uniform of the Queen of Hearts. In the top middle, is another Alice in Wonderland themed card featuring black silhouettes of rabbits, teapots, and top hats, with a background of a flower. On the top right is a moon landing themed card featuring a spaceship and an astronaut holding a flag. The bottom row includes card designed using an LPIM. On the bottom left is an Alice in Wonderland themed card featuring two images: a blue and red top hat, and a rabbit in fancy red clothes holding a bugle. In the bottom middle is a moon landing themed card with a detailed, cartoonish, image of an astronaut on the moon holding a birthday cake with three candles. On the bottom right is a moon landing themed card featuring 4 images, two of which show astronauts surfing on the moon. The other two photos feature astronauts and a space ship.
\end{figure}

\begin{figure}
     \centering
     \begin{subfigure}[t]{0.3\textwidth}
         \centering
         \includegraphics[width=\textwidth]{\imagepath{cake3.png}}
        %  \caption{prompt TODO}
         \label{fig:cake1}
     \end{subfigure}
     \hfill
     \begin{subfigure}[t]{0.3\textwidth}
         \centering
         \includegraphics[width=\textwidth]{\imagepath{cake2.png}}
        %   \caption{prompt TODO}
         \label{fig:cake2}
     \end{subfigure}
     \hfill
     \begin{subfigure}[t]{0.3\textwidth}
         \centering
         \includegraphics[width=\textwidth]{\imagepath{cake1.png}}
        %   \caption{\prompt{A cartoon cake that says `eat me' with a purple striped cat curled on top.}}
         \label{fig:cake3}
     \end{subfigure}
        \caption{A few of the images participants created with \imagen{}. As can be seen, \imagen{} does not always generate images that are properly cropped; participants used prompts such as \prompt{...framed art} to generate images with better composition (far right).}
        \label{fig:cake-prompts}
        % ALT TEXT: Left: A drawing of a purple cat sitting on a purple and white cake with the words "Eat me" written  on the cake. Middle: A retro 1950's style image with an astronaut in blue and yellow holding a cake, with stars on a red background. Right: A framed image of an astronaut on the moon holding a birthday cake with candles.
\end{figure}
