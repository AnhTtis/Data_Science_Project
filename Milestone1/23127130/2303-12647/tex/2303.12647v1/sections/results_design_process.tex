\subsection{How TTI models changed the design process}\label{sub:collaboration}
Through rapid  image creation and their indirect nature, where images were created through text descriptions, \LPIMs{} led to new design practices, as outlined below. 

\subsubsection{Indirect and rapid image creation through text allowed {new} creative freedom}
Participants noted how creating with \imagen{} was indirect, as it involves ``creating prompts that create images'' (P13{}). This indirectness and the flexibility of prompt editing allowed participants to rapidly explore the design space of alternatives. This was most apparent when participants used \imagen{} to take on other 'artist' personalities, which would otherwise have taken years of practice. P4{} noted:  ``If you made a poster, it [the poster] would have had your style associated with it by default because you have to learn [and develop a particular style]… It is harder to switch between styles. Whereas \Imagen{}, you could just be like `1960s poster’ or like in the style of whoever: Picasso''. At the same time, participants felt faster image generation would allow for even more exploration. P3{} noted: “Because it takes so long to generate a bunch of different images, I didn't really move off to, you know, how else that card could look.” 


Participants also noted how the model implicitly steered such rapid exploration. Despite this steering, participants noted how they still remained in control over the ``personality'' their \Imagen{} creations would have. For instance, P3{} quoted above added: ``Well, for me the image that it generated was sort of similar to what I envisioned in my mind... The way that it turned out is pretty cool. It is definitely not the style that I would have chosen for myself, my own drawings, but like it looks pretty.'' {(We should note that, due to the short term nature of our study, we are unable to study how such model steering impacts participant creativity over the long term.)}

Throughout these explorations, participants tended to improve or ``optimize'' a prompt if they found that at least one of the generated images was helpful. Participants refined their prompts both to bring out aspects they found successful in the initial set of results, and to steer results away from undesired properties. For example, while creating a Cheshire Cat, P9{} liked the \textit{card design} in the results, rather than the cat in the foreground:  ``[I] kind of like some of these designs..." They then updated their prompt to get more of that card design: \prompt{Frame with filigree pattern. Circus colors}. 

At other points, participants refined their prompts to steer results away from undesired properties. For example, P8{} first tried to generate a jovial Cheshire Cat image but remarked that “Those are a little terrifying." He thus updated the prompt to make the image look less scary and more festive: \prompt{Invitation to a birthday party. Alice in Wonderland. Cheshire Cat}. Similarly, participants noticed that with some images that were poorly cropped, they could obtain better results if they appended \prompt{framed painting} to their prompt. 


\subsubsection{Novel images steered novel ideas}

Whereas \IS{} surfaced existing images on the Internet, \Imagen{} {allowed} participants to generate {entirely novel} images, {and allowed them to successfully explore their creative ideas}. For example, P5{} described hitting a wall with \IS{} when he could not find a specific aspect of what he wanted via \IS{}, possibly because it did not exist in the real world: ``I wanted a picture of a dolphin...And I started to Google it... One of my problems when I was searching around, is I couldn't quite get the image I wanted, right? I wanted to make something new and I couldn't quite get the right image I wanted."In contrast, participants were able to use \imagen{} to create novel combinations of ideas that did not exist, such as a giraffe driving a Lamborghini: ``Like `a giraffe is driving a Lamborghini' ... these are things you can never do. You can never have images, that look reasonable for those online. If you had to do it the old-fashioned way, or be really good at Photoshop or Illustrator. And it would take a lot longer than I have.'' They were also able to apply styles to content from a time period, which would not have been possible in the real world: ``...show me `the Apollo 11 Landing in the style of Dali'”.


\begin{figure}
     \centering
     \begin{subfigure}[t]{0.3\textwidth}
         \centering
         \includegraphics[width=\textwidth]{\imagepath{Imagen_giraffe-driver}}
         \label{fig:image-search-1}
     \end{subfigure}
     \begin{subfigure}[t]{0.3\textwidth}
         \centering
         \includegraphics[width=\textwidth]{\imagepath{Imagen_giraffe-moping}}
         \label{fig:image-search-1}
     \end{subfigure}
     \begin{subfigure}[t]{0.3\textwidth}
         \centering
         \includegraphics[width=\textwidth]{\imagepath{Imagen_giraffe}}
         \label{fig:image-search-1}
     \end{subfigure}
     \caption{A few giraffes driving fancy cars. (Prompt inspired by participant: \prompt{Giraffe is driving a Lamborghini. f2.2}). By enabling the rapid realization of novel ideas and unlikely combinations, \LPIMs{} enable an  exploration of the design space and fluid collaboration.}
     % ALT TEXT: Three images of giraffes driving fancy cars. The images on the left and right show giraffes in the driver seat holding the wheel. The middle image shows a giraffe sticking its head out of the car window.
\end{figure}


{In addition to exploring existing creative ideas, surprising image results also spurred participants} to go in a different direction. For example, TTI  surprises inspired participants to consider aesthetic styles, compositions, or other design choices they hadn't initially considered: ``When I asked \imagen{} for a doodle of that \imagen{} blew me away with something that was a different art style than I imagined. That inspired me to seek out stuff in that same art style or to keep asking for doodles.” Together, participants saw \imagen{} as a {way to support their creativity in ways that were qualitatively different from previous tools}. As P6{} noted: ``it really kind of stirs, my creative juices or whatever whereas like Googling for images does not really stir that…''


However, \imagen{} also occasionally generated {non-sensical or clearly flawed} images, such as animals with incorrect anatomy or images of the Moon with two ``Earths'' in the background. Participants contrasted this with \IS{}, which offered more predictable results because they were authentic images from the Internet. This, in turn, allowed participants to {feel that the resulting images reasonably depicted what was in the images}  without closer scrutiny. For example, P5{} suggested:  “...I trusted the images that I look for are going to look somewhat more sane... I am not going to see half of two rabbits”. Participants also saw such predictability as necessary when looking for a specific image. For instance, one pair of designers used public-domain images of the first Moon Landing. For such images, correctness was crucial: “Most of the images we used are very specific. They are images that \Imagen{} cannot generate.” 


\subsubsection{Designing with an opinionated model}\label{ssub:opinionated}
% TODO intro sentence
As \imagen{} would sometimes produce unexpected results, the participants often felt the need to guide or work around the model's limitations. 
For instance, P3{} noted their decision to use  the model to generate an image for the entire invitation from a single prompt rather than prompting for each part of the image and compositing them: “I suppose because we knew the limitations of \imagen{}, in terms of like, composing it for multiple images is, it sort of reduced what we could do with it.” Other participants were able to avoid design fixation, but with considerable effort. P5{} remarked: ``It felt like I was fighting it….I felt like it was helpful, but I also felt like I had to massage every word and select every character very carefully not to upset it so that it could generate something I wanted.'' Consistent with prior work, in these and other quotes, participants seemed to ascribe the role of an opinionated design partner to \imagen{} For instance, Koch et al. described how participants ascribed agency to the AI tool (with one participant even referring to it as "an eccentric collaborator") \cite{koch2020CollaborativeAI}. Similarly, in a study of an AI-based co-creation tool that generates sketches to inspire the user as they are actively sketching, users perceived the AI as a ``collaborative partner'' in the condition when the system communicated with them \cite{rezwana2022AICoCreation}. {(Because participants themselves anthropomorphized the model, we characterize it as opinionated, rather than using other terms, such as being biased.)}


While prompting {with this opinionated} model enabled participants to express high-level concepts at rapid speed, participants struggled to systematically control low-level details, such as position, layout, and which letters appear in the text (note, however, that some participants did use \Imagen{} to generate text in styles that Google Slides did not support, see Figure~f{fig:cake-choices}). For instance, P4{} wished for ``more controllability'': ``it is kind of agonizing to keep typing in very different versions of the same thing, and you are like, no, I just want his hand to be, like a little bit farther down.'' Similarly, P5{} expressed their frustration with how \Imagen{} sometimes cropped parts of an object in the image: ``So this is yeah, with this image, we can go outside the lines and get something that covers more of the screen while it is just focusing on the top hat...\textit{after a few moments}... I cannot.'' 

