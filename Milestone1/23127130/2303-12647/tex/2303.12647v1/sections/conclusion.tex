\section{Conclusion}
This paper examines how \LPIMs{} can influence design processes and collaboration in goal-oriented design. Our results suggest that rather than a simple ``magic'' moment where designers input a prompt and designs are automatically generated, these models allow for a nuanced reflective practice of exploration, iteration, and collaboration. Our results also suggest that prompts can act as a design material and can support such emergent reflective practices. Finally, our study reveals several opportunities for tool design that build on the notion of prompts as reflective design materials, and suggest future directions for \LPIM{} research. Together, they point to a future where designers can use \LPIMs{} more effectively, resulting in a deeper and more creative practice.