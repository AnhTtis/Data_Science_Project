\begin{table}[t]
\centering
\small
\resizebox{\linewidth}{!}
{
\begin{tabular}{c|c|c|c|l} 
\toprule
Backbone   & Upsampler  & Extra Encoding & HFGM & mIOU(\%)  \\
\midrule
\midrule
ResNet-50  & SFPN & Identity & -      & 47.14  \\
ResNet-50  & SFPN & Identity & + guidance & 47.67 (\textcolor{black}
{$+$0.53}) \\
ResNet-50  & SFPN & Identity & + AA  & 47.88 (\textcolor{black}
{$+$0.74})  \\
ResNet-50  & SFPN & Identity & full  & 48.94 (\textcolor{black}{$+$1.80)}\\
\midrule
ResNet-50  & SFPN & Our CAE & -      & 48.76    \\
ResNet-50  & SFPN & Our CAE & + guidance & 49.22 (\textcolor{black}{$+$0.46}) \\
ResNet-50  & SFPN & Our CAE & + AA & 49.06 (\textcolor{black}{$+$0.26}) \\
ResNet-50  & SFPN & Our CAE & full  & \textbf{50.28} (\textcolor{black}{$+$1.52}) \\
\bottomrule
\end{tabular}
}
\caption{
Ablation studies on different HFGM settings on Pascal Context dataset.
Using both guidance and AA in HFGM brings most gain (1.8\%/1.52\% mIOU).
AA and the proposed high-level feature guidance co-operate extremely well
% are beneficial for each other 
probably because AA can effectively back-propagate the guidance signal to all spatial locations.
% HFGM effectively improves the accuracy, especially when it contains an axial attention (AA) layer since AA can effectively back-propagate the guidance signal from the good high-level features (i.e. output of CAE) to all spatial locations, resulting in more accurate upsampled feature representations.
}
\label{tab:ablation_hfgm}
\end{table}
