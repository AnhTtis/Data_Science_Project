% CVPR 2024 Paper Template; see https://github.com/cvpr-org/author-kit

\documentclass[10pt,twocolumn,letterpaper]{article}

%%%%%%%%% PAPER TYPE  - PLEASE UPDATE FOR FINAL VERSION
%\usepackage{cvpr}              % To produce the CAMERA-READY version
%\usepackage[review]{cvpr}      % To produce the REVIEW version
\usepackage[pagenumbers]{cvpr} % To force page numbers, e.g. for an arXiv version
\usepackage{comment}

% Import additional packages in the preamble file, before hyperref
% \usepackage[latin1]{inputenc}
\usepackage[british]{babel}
\usepackage[all]{xy}
\usepackage{amscd}
\usepackage{amssymb}
\usepackage{amsthm}
\usepackage{enumitem}
\usepackage{mathrsfs,bbm}
\usepackage{xcolor,graphicx}
\usepackage{graphics}
\usepackage{soul}
\usepackage{comment}
\usepackage[all]{xy}
\usepackage{amscd}
\usepackage{amssymb,amsmath,latexsym}
\usepackage{amsthm}
\usepackage{enumitem}
\usepackage{mathrsfs,bbm}
\usepackage{dsfont}
\usepackage{tikz-cd}
\usepackage[T1]{fontenc}
\usepackage[utf8]{inputenc}  
 %
%%%%%%%%%%%%%%%%%%%%%%%%%%%%%%%%%%
%pagestyle
%%%%%%%%%%%%%%%%%%%%%%%%%%%%%%%%%%
%\pagestyle{plain}
\textwidth=430pt
\headsep=.7cm
\evensidemargin=15pt
\oddsidemargin=15pt
\leftmargin=0cm
\rightmargin=0cm
%%
%%%%%%%%%%%%%%%%%%%%%%%
\newcommand*\fixitem {\item[]%
  \refstepcounter{enumi}\hskip-\leftmargin\labelenumi\hskip\labelsep}
\newtheorem*{mainthm}{Main Theorem}
\newtheorem*{mainthm1}{Theorem}
\newtheorem*{maincor}{Corollary}
\usepackage[colorlinks=true]{hyperref}
\DeclareMathOperator{\Forall}{\forall}
\DeclareMathOperator{\Exists}{\exists}
\DeclareMathOperator{\ord}{ord}
\newcommand{\phiD}{\varphi_D}
\newcommand{\phiDI}{\varphi_{\mathbf{D}_I}}
\newcommand{\phiDIj}{\varphi_{\mathbf{D}_I (j)}}
\newcommand{\phiH}{\varphi_H}
\newcommand{\phiTimes}{\phiD \otimes \phiH}
\newcommand{\phiTimesDI}{\varphi_{\mathbf{D}_I} \otimes \phiH}
\newcommand{\R}{\mathscr{A}}
\newcommand{\X}{\mathscr{X}}
\newcommand{\Xf}{\mathscr{X}_{(k_0 ,i)}[r_0]}
\newcommand{\Xfr}{\mathscr{X}_{(k_0,i)}[r]}
\newcommand{\hotimes}{\widehat{\otimes}}
\newcommand{\C}{\mathbb{C}_p}
\newcommand{\V}{\mathscr{V}}
\newcommand{\B}{\mathscr{B}}
\newcommand{\dualD}{\mathfrak{D}}
\newcommand{\Dg}{\mathbf{D}}
\newcommand{\DD}{\mathcal{D}^0}
\newcommand{\DDg}{\mathcal{D}}
\newcommand{\DV}{\mathcal{D}}
\newcommand{\W}{\mathscr{W}_N}
\newcommand{\Ao}{\mathbf{A}^\circ}
\newcommand{\AoK}{\mathbf{A}^\circ_{\K}}
\newcommand{\AK}{\mathbf{A}_{/\K}}
\newcommand{\OOO}{\mathscr{A}^\circ}
\newcommand{\K}{\mathcal{K}} 
\newcommand{\OK}{\mathcal{O}_{\K}}
\newcommand{\varprojlog}[1]{\underleftarrow{\log\!^{#1}}}
\newcommand{\T}{\mathscr{T}}
\newcommand{\TT}{\mathbf{T}}
\newcommand{\VV}{\mathbf{V}}
\newcommand{\HH}{\mathcal{H}}
\newcommand{\hh}{\mathcal{H}^+}
\newcommand{\HG}[2]{\mathcal{H}_{#1}(#2)}
\newcommand{\hhl}{\mathcal{H}^{+,[l]}}
\newcommand{\hhj}{\mathcal{H}^{+,[j]}}
\newcommand{\hhjj}{\mathcal{H}^{+,[l,l']}}
\newcommand{\GS}{G_{\mathbb{Q},S}}
\newcommand{\Rf}{R_{(k_0 ,i)}[r_0]}
\newcommand{\Rfr}{R_{(k_0 ,i)}[r]}
\newcommand{\parT}{\langle T\rangle}
\newcommand{\Zf}{Z_{(k_0 ,i)}[r_0]}
\newcommand{\Zfr}{\mathscr{Z}_{(k_0 ,i)}[r]}
\newcommand{\ZFf}{\mathscr{Z}_{(k_0 ,i)}[r_0]}
\newcommand{\ZFfr}{\mathscr{Z}_{(k_0 ,i)}[r]}
\newcommand{\ZF}{\mathscr{Z}}

% It is strongly recommended to use hyperref, especially for the review version.
% hyperref with option pagebackref eases the reviewers' job.
% Please disable hyperref *only* if you encounter grave issues, 
% e.g. with the file validation for the camera-ready version.
%
% If you comment hyperref and then uncomment it, you should delete *.aux before re-running LaTeX.
% (Or just hit 'q' on the first LaTeX run, let it finish, and you should be clear).
\definecolor{cvprblue}{rgb}{0.21,0.49,0.74}
\usepackage[pagebackref,breaklinks,colorlinks,citecolor=cvprblue]{hyperref}

%%%%%%%%% PAPER ID  - PLEASE UPDATE
\def\paperID{527} % *** Enter the Paper ID here
\def\confName{CVPR}
\def\confYear{2024}

%%%%%%%%% TITLE - PLEASE UPDATE
%\title{\emph{HFGD}: High-level Feature Guided Decoder for Semantic Segmentation}
\title{High-level Feature Guided Decoding for Semantic Segmentation}
%%%%%%%%% AUTHORS - PLEASE UPDATE
\author{Ye Huang\textsuperscript{\rm 1}\quad
        Di Kang \textsuperscript{\rm 2}\quad
        Shenghua Gao \textsuperscript{\rm 3}\quad
        Wen Li \textsuperscript{\rm 1}\quad
        Lixin Duan\textsuperscript{\rm 1}\thanks{Corresponding author}\\
%{\tt\small firstauthor@i1.org}
{\small \textsuperscript{\rm 1} University of Electronic Science and Technology of China}\\
{\small
    \textsuperscript{\rm 2} Tencent AI Lab \quad
    \textsuperscript{\rm 3} ShanghaiTech University
}
% For a paper whose authors are all at the same institution,
% omit the following lines up until the closing ``}''.
% Additional authors and addresses can be added with ``\and'',
% just like the second author.
% To save space, use either the email address or home page, not both
}

\setlength{\textfloatsep}{6pt}
\setlength{\dbltextfloatsep}{6pt}  % two-column mode
\setlength{\floatsep}{4pt}
\setlength{\dblfloatsep}{4pt}  % two-column mode
\setlength{\intextsep}{4pt}
\setlength{\abovecaptionskip}{2pt}
\setlength{\belowcaptionskip}{2pt}

\begin{document}
%\maketitle

\twocolumn[{%
\renewcommand\twocolumn[1][]{#1}%
\maketitle
\begin{center}
\centering
\includegraphics[width=\textwidth]{images/fig1.pdf}
\captionof{figure}{
%\textbf{Problem identification and our solution.}
\textbf{Problem identification.}
We notice that 1) using an upsampler net (i.e. SemanticFPN) is worse (\eg, more entangled) than using a dilated backbone; 2) the joint fine-tuning of the upsampler and the backbone usually results in worse backbone features (see Sec.~\ref{sec:problem}).
% And we propose to use powerful pre-trained \textit{\textbf{h}}igh-level \textit{\textbf{f}}eatures as \textit{\textbf{g}}uidance when training the upsampler.
In this work, we propose to use high-quality \textit{\textbf{h}}igh-level \textit{\textbf{f}}eatures to \textit{\textbf{g}}uide the training of the upsampler.
% as \textit{\textbf{g}}uidance when training the upsampler
}
\label{fig:hfgd:concept}
\end{center}}]


\begin{abstract}
The current study investigated possible human-robot kinaesthetic interaction using a variational recurrent neural network model, called PV-RNN, which is based on the free energy principle.
Our prior robotic studies using PV-RNN showed that the nature of interactions between top-down expectation and bottom-up inference is strongly affected by a parameter, called the meta-prior, which regulates the complexity term in free energy.
% The current study examines how the behaviours of robots alter by changing the meta-prior $w$ in human-robot kinaesthetic interaction.
The current study examines how changing the meta-prior $w$ in the interaction phase affects the counter force generated when an experimenter attempts to induce movement pattern transitions familiar to the robot through its prior training.
The study also compares the counter force generated when trained transitions are induced by a human experimenter and when untrained transitions are induced.
Our experimental results indicated that (1) the human experimenter needs more/less force to induce trained transitions when $w$ is set with larger/smaller values, (2) the human experimenter needs more force to act on the robot when he attempts to induce untrained as opposed to trained movement pattern transitions.
Our analysis of time development of essential variables and values in PV-RNN during bodily interaction clarified the mechanism by which gaps in actional intentions between the human experimenter and the robot can be manifested as reaction forces between them.


%% Hiroki writing 2022-11-4
%Current study investigates the dynamics of the latent states during human-robot kinaesthetic interaction using PV-RNN.
%We have achieved to observe and analyse the internal state of an RNN model based on the free energy principle, during real-time human-robot interaction.
%Essential characteristics observed in the previous study of this variational recurrent neural network model, PV-RNN, is that by changing a meta prior $w$, the balance between the top-down intention and the bottom-up perceptual reality changes.
%In the current study, we examined how changing the weighting parameter $w$ between accuracy and complexity in free energy principle affects the humanoid robot's behaviour through human-robot interaction. We have conducted some human-robot kinaesthetic interaction experiments with various $w$ and quantitatively analysed the latent variable and the force applied to the humanoid robot. We have observed that the force required to change the robot's intention has increased, both when the top-down intention was strengthened by changing the $w$ and when corresponding switch of its primitive was against the experience of the RNN during its training. The study confirms through quantitative analysis that by increasing or decreasing the $w$ in PV-RNN, humanoid robot leads or follows the human counterpart during the human-robot kinaesthetic interaction.

\begin{comment}
Comment from Jun #2
・最後にQualitativeな結果(インパクト)が欲しい
・Current study investigates the problem on~と書き出すのが一般的
・最初の一文と最後の一文を対応させる
・最後の一文はもう少しAbstractかつ包括的に
\end{comment}

\begin{comment}
Comment from Jun #1
We investigated how the kinaesthetic human-robot interaction can affect the internal state of a model based on the free energy principle. 
=> how the internal state is affected is not the most important point in this study. This part should be rewritten.

The key function of this variational recurrent neural network model, PV-RNN, is that by changing a meta prior $w$, it takes a balance between the "complexity” term and the ”accuracy” term which corresponds to a top-down intention and a bottom-up perceptual reality in the free energy principle, respectively. 
=> This is not key function of PV-RNN. It is an essential characteristics observed in the previous study. The grammar after $w$ is something strange. Rewrite these.

This research has conducted a human-robot interaction experiment with a robotic agent in a kinaesthetic sense.
=> The sentence is not good. "in a kinaesthetic sense" is grammatically wrong.
MODIFIED => "In the current study human-robot interaction experiments using the kinaesthetic sense were conducted."

We investigated that when human forces the agent to switch primitives from one to another, larger force was required both when the human intention is conflictive against the top-down the intention of the agent and when the agent has a stronger top-down intention by modifying the $w$.
=> You should write the essential results of the experiments rather than what we investigated and also how these results could contribute to the studies on human-robot interaction.
\end{comment}

\end{abstract}    
\section{Introduction}
\label{sec:intro}
\begin{figure}[t]
\begin{center}
    \includegraphics[width=1\linewidth]{figures/teaser.pdf}
\end{center}
\vspace{-0.1in}
\caption{\textbf{{\em Foggy} vs {\em Clear} NeRF.} Our \ournerf gets rid of reconstruction errors manifested as foggy ``floaters" in the density volume without additional input or significant computational overhead. 
%
Below are density profiles along a given ray before and after our geometry correction procedure, where we discard density peaks corresponding to floaters.
}
\label{fig:teaser}
\vspace{-0.2in}
\end{figure}



%The emergence of 
Neural Radiance Fields (NeRFs)~\cite{mildenhall2020nerf}  %and its variants 
have made revolutionary contributions in %photo-realistic 
novel view synthesis~\cite{barron2021mip,barron2022mip}, 
autonomous driving~\cite{rematas2022urban,tancik2022block}, digital human~\cite{hong2022headnerf,zhao2022humannerf}, and 3D content generation~\cite{eg3d,poole2022dreamfusion,lin2022magic3d}.
%by leveraging a multi-layer perceptron (MLP) to implicitly model the mapping from input 5D coordinates (i.e., 3D coordinates $\mathbf{x} = (x,y,z)$ and 2D viewing directions $\mathbf{d}=(\theta,\phi)$) to volume density $\sigma$ and view-dependent emitted radiance color $\mathbf{c} = (r,g,b)$. 
%
%They then use traditional volume rendering mechanisms on the obtained continuous 5D function (i.e., MLP) to generate novel views. 
To date, unfortunately, most NeRF-based methods encounter challenges when tackling large-scale cluttered scenes (e.g., Fig.~\ref{fig:teaser}):
\begin{enumerate}[leftmargin=0.16in, topsep=2pt,itemsep=-1ex,partopsep=1ex,parsep=1ex]
\item Input observations used for NeRF are often too sparse  compared to forward-facing or synthetic looking-inward scenes;
%\item Recovering fine-grained objects within a large volume is challenging for NeRF; %in capturing details accurately.
\item View-dependent visual effects give rise to ambiguity, resulting in a ``foggy" density field as shown in Fig.~\ref{fig:teaser}. 
%
Such artifacts are particularly pronounced in indoor scenes strewn with view-dependent appearances, such as specular highlights, glossy surface reflections from man-made objects. 
\end{enumerate}

Despite attempts to enhance NeRF's rendering quality given suboptimal input, such as using 3D conical frustums~\cite{barron2021mip,barron2022mip}, physically-grounded augmentations~\cite{chen2022aug}, and misalignment correction~\cite{jiang2022alignerf},  these challenges have yet to be fully resolved.
%
Depth supervision~\cite{deng2022depth, wei2021nerfingmvs} or proxy geometry~\cite{xu2021scalable,wu2022scalable} images can help alleviate the challenges in handling large-scale with sparse input, at the expense of %but they come at the cost of requiring 
expensive pre-processing or additional input.
%
Another line of work~\cite{wang2021neus, oechsle2021unisurf, wang2022neuris} achieves better reconstruction of surface geometry by using signed distances instead of volume density as scene representation. However, they sacrifice the ability to synthesize photo-realistic novel views.

%We observe that NeRF has been suffering from foggy ``floater" artifacts in large-scale cluttered scenes.
%
%Such artifacts are particularly pronounced in indoor scenes strewn with view-dependent appearances from man-made objects. 
%
To address the above issues, we propose an extension to NeRF, dubbed as {\bf \ournerf}, which enforces effective {\em appearance} and {\em geometry} constraints conducive to accurate colors and 3D densities estimation. We believe \ournerf can contribute beyond novel view synthesis, such as NeRF object detection~\cite{hu2022nerf}, NeRF object segmentation~\cite{zhi2021place, liu2022unsupervised, fan2022nerf,ren2022neural}, and NeRF registration~\cite{goli2022nerf2nerf}, where the rooms for improvement are substantial if more accurate color and density estimation are available.

Correspondingly, there are two steps in \ournerf. First, for appearance correction, the view-independent and view-dependent color components are predicted from the underlying 3D scene, which is combined to produce the final color estimation (Fig.~\ref{fig:toaster}).
%
The view-independent component (diffuse color and shading) captures the overall scene color, while the view-dependent component (highlights or reflections) captures color variations due to changes in viewing angle.
%
\ournerf then discards these view-dependent appearances in the training views to prevent them from interfering with the density estimation.
%
Second, a simple and effective geometry correction procedure will be performed to further eliminate the foggy ``floaters" or density errors. This geometry correction procedure is based on an assumption in line with traditional ray tracing in computer graphics.
\begin{comment}
% xh: basically copying method
On the other hand, ClearNeRF performs a geometric correction procedure performed on each traced ray during inference to refine the density estimation and better tackle the floater artifacts. 
%
The geometry correction procedure assumes that there should only be one salient peak along each traced ray during NeRF inference. 
Only the salient peak closest to the ray origin (the camera center) corresponds to  true geometry while the others will be manifested as foggy floaters hovering in the density volume. 
%
This assumption is in line with traditional ray tracing in computer graphics where in the absence of noise, only one intersection per ray should be returned to indicate the closest ray-object intersection.
%
\end{comment}
%%%%%%%%%%%
%As shown in Fig.~\ref{fig:teaser}, when reconstructing an indoor scene with sparse input and highly view-dependent objects, NeRF produces severe floating artifacts due to its attempt to explain view-dependent appearances.
%
Experiments verify that our proposed \ournerf can effectively get rid of floater artifacts without additional input.% or significant computational overhead. 


In summary, our contributions include the following:
\begin{itemize}[leftmargin=0.16in, topsep=2pt,itemsep=-1ex,partopsep=1ex,parsep=1ex]
    \item We propose a concise method for decomposing view-independent and view-dependent appearance during NeRF training and eliminate the interference of view-dependent appearance.
    \item We propose a geometric correction procedure performed on each traced ray during inference to refine the density estimation and better tackle the floater artifacts.
    \item Extensive experiments and ablations verify the effectiveness of our core designs and results in improvements over the vanilla NeRF and other state-of-the-art alternatives.
    %without additional computational resources or other inputs.
\end{itemize}




\section{Related Works}

\begin{figure*}[!ht]
\centering
\includegraphics[width=\linewidth]{body/figures/data_collection2.png}
\caption{\textbf{Hardware Setup.} We use a GelSight Wedge sensor for tactile sensing, an Intel ReslSense D405 camera mounted on the side for RGB vision sensing, and an OptiTrack setup for motion capture. \textbf{Data Collection.} The tactile finger and the camera are fixed to the table at all times. A human operator moves a test object and presses it against the finger. We show sampled tactile and RGB images as well as a reconstructed local tactile depth map on the right.}
\label{datacollection}
\end{figure*}

% \textbf{Tactile sensors}:
% Over the years, researchers have developed tactile sensors working on different sensing principles, such as resistance, capacitance, magnetic, barometric, and optic.
% We refer readers to \cite{kappassov2015tactile} for an in-depth review of different types of tactile sensors and their applications.
% Compared to other sensing principles, GelSight tactile sensors have the advantage of providing high-resolution geometrical information of the contact surface.
% They are usually constructed with an elastic silicone gel, directional colored LEDs, and a camera pointing at the gel.
% The gels are usually coated with reflective paint with printed dots.
% When in contact, the gel deforms and takes the shape of the contact surface.
% Shear force can be retrieved by tracking dot movements.
% Furthermore, the color value of a pixel is correlated with the gradient of the height of the contact surface at the specific location. 
% With a pre-calibrated color table, a depth map can be reconstructed from the color image.
% This type of tactile sensor is selected for our work for its rich output and ease to use.

Researchers of the robotics community have put forward a wide range of tactile sensing solutions.
Sensors working on different sensing principles have been adopted to solve a large set of manipulation tasks.
Among different types of tactile sensors, vision-based ones such as GelSight \cite{yuan2017gelsight} and GelSlim \cite{donlon2018gelslim} stand out for their rich output, ease to use, and affordability.
While we focus on the pose estimation and shape reconstruction task using vision-based tactile sensors, we refer readers to \cite{kappassov2015tactile} for an in-depth review of different types of tactile sensing and their applications.
In this section, we review works on three typical tasks that are most relevant to our solution: slip detection, object property inference, and SLAM.

% Researchers have found that using this class of vision-based tactile sensors can greatly increase the accuracy when reasoning about the contact surface, compared to traditional tactile sensors that are constructed with normal direction force sensors [\todo{add citation}].

\textbf{Slip detection and estimation}:
Using a similar sensor to ours, Yuan \etal compared and analyzed a GelSight tactile sensor's images collected at different stages of slip in \cite{yuan2015measurement} and showed this type of sensors' capability in detecting micro scale movements.
Li \etal and Zhang \etal trained recurrent neural networks on tactile images to detect slip between multiple time steps in a manipulation sequence \cite{li2018slip, zhang2018fingervision}.
Built on their binary slip detection model in \cite{li2018slip}, Li further added rotational slip direction prediction in \cite{li2019rotational}.
Calandra \etal improved a grasp planner for the classic robot bin-picking problem by incorporating slip detection and achieved a higher grasp success rate \cite{calandra2017feeling}. 
However, those methods only detect slip without localizing the object after the slip.
In many precision manipulation tasks we are also interested in the amount of the displacement.

\textbf{Object property inference and localization}:
% With detailed information on the contact surface provided by high-resolution tactile sensors, 
Many works have focused on inferring properties of the in-contact object, such as shape \cite{strub2014using, luo2015tactile, luo2019iclap}, texture \cite{luo2018vitac, yuan2017connecting}, and material \cite{yuan2017connecting, kroemer2011learning, kerr2018material}.
Those learned object properties can be further used for localization.
In order to localize current grasps, Bauza \etal proposed to match new tactile imprints with previously collected tactile imprints \cite{bauza2019tactile}, while Luo \etal learned to match tactile imprints directly to visual images of the whole object \cite{luo2015localizing}.
Assuming known CAD models, Bauza \etal proposed to localize by comparing contact masks generated from tactile images with a large bank of random projections of the CAD model \cite{bauza2022tac2pose}.
To solve the reverse problem, i.e. what a tactile image looks like given an object and a pose, several tactile simulators have been built to automatically generate tactile images given an object's CAD model and a finger pose \cite{si2022taxim, wang2022tacto}.
One major limitation for this category of works is that they all require a known calibrated geometry of the object: a pre-collected tactile map \cite{bauza2019tactile}, a model of the object \cite{bauza2022tac2pose}, or a global image with known geometry \cite{luo2015localizing}.
This requirement can be hard to meet in less constraint environments.

\textbf{Tactile SLAM}:
Recent studies have shown interests in working with unknown objects by leveraging methods from the SLAM problem.
With a focus on 2D shapes, Suresh \etal parameterized shapes as Gaussian Process Implicit Surfaces (GPIS), and learned its parameters from tactile signals collected during pushing \cite{suresh2021tactile}.
Assuming known contact poses, authors of \cite{suresh2022shapemap} first learned a noisy mapping from known surface geometries to corresponding tactile images, then reconstructed an object by combining many noisy local tactile measurements into an optimized global shape using factor graph optimization.
The closest prior work to ours is \cite{sodhi2022patchgraph}, where the authors learned to estimate 6D poses and 3D shapes simultaneously for unknown objects. 
They constructed a pose estimator based on tactile sensing, and a shape reconstruction pipeline that added in new tactile point clouds incrementally on the run.
However, this approach heavily relies on the performance of the tactile pose estimator, which lacks a global understanding of the object and can suffer from repeated patterns or smooth surfaces.
In contrast, our work combines vision and tactile sensing which provides us with both global and local understandings of the scene without requiring any other domain knowledge.
Furthermore, we designed a loop closure mechanism that periodically matches current tactile and vision images to stored key-frames, which significantly reduced accumulated errors.
With this, FingerSLAM is able to produce realistic reconstructions even in long sequences. 
\section{Methods}
\label{sec:methods}
\subsection{Preliminary}
In this section, we introduce each component of MAPSeg (\hyperref[fig2]{Fig.2}) and how MAPSeg can serve as a unified solution to centralized, federated, and test-time UDA (\hyperref[fig:overview]{Fig.1b}). We deploy MAPSeg for domain adaptative 3D segmentation of heterogeneous medical images and it consists of three components: (1) 3D masked multi-scale autoencoding for self-supervised pre-training, (2) 3D masked pseudo-labeling for domain adaptive self-training, and (3) global-local feature collaboration to fuse global and local contexts for the final segmentation task. The hybrid cross-entropy and Dice loss (\hyperref[eq:L_seg]{Eq.1}) is often adopted for regular supervised segmentation training, and we employ it as the basic component of the objective functions for MAPSeg:
\begin{equation}
    \label{eq:L_seg}
    \mathcal{L}_{seg}(\hat{y},y) = -\frac{1}{n}\sum_i\sum_jy_{i,j}\log(\hat{y}_{i,j}) -\frac{2\sum y\hat{y}+\epsilon}{\sum y+\sum \hat{y}+\epsilon}
\end{equation}
where $n$ denotes the number of pixels, $y_{i,j}$ and $\hat{y}_{i,j}$ represent the ground truth label and predicted probability for the $i$th pixel to belong to the $j$th class, and $\epsilon$ is used to prevent zero-division. 

In the following sections, notations are defined as: $x$ and $y$ indicate the original image and label of the randomly sampled local patch; $X$ and $Y$ refer to downsampled global scan and label; the subscripts $s$ and $t$ refer to the source and target domains, respectively; the superscript $M$ indicates the image is masked (\eg, $x_t^M$ refers to a masked local patch from the target domain).

\begin{figure*}
\centering
\includegraphics[width=0.85\linewidth]{./figs/fig2-11.pdf}
\caption{Components of the proposed MAPSeg framework. (a) 3D multi-scale masked autoencoding. (b) 3D masked pseudo labeling in source and target domains. (c) 3D Global-local collaboration.} 
\label{fig2}
\end{figure*}
\subsection{3D Multi-Scale Masked Autoencoder (MAE)}
In this study, we propose a 3D variant of MAE using a 3D CNN backbone (\hyperref[fig2]{Fig.2a}). The detailed configuration can be found in Appendix \cref{sec:archite}. Training is jointly performed on two image sources with identical size ($96^3$ voxels): local patches $x$ randomly sampled from the volumetric scan, and the whole scan downsampled to the same size, denoted as $X$. 
Both $x$ and $X$ are masked before feeding into the MAE: $x$ is divided into non-overlapping 3D sub-patches with size $8^3$, of which 70\% are masked out randomly based on a uniform distribution (\hyperref[fig2]{Fig.2a}); The same procedure is applied to $X$ with patch size $4^3$ since it contains a larger field-of-view (FOV). The masked versions of $x$ and $X$ are denoted as $x^M$ and $X^M$, respectively. We train the MAE encoder and decoder to reconstruct $x/X$ based on $x^M/X^M$ using mean squared error on the masked-out regions as the objective function.

\subsection{3D Masked Pseudo-Labeling (MPL)}
MPL uses a teacher-student framework which is a standard strategy in semi-/self-supervised learning~\cite{grill2020bootstrap,NIPS2017_68053af2} to provide stable pseudo labels on an unlabeled target domain during training. 
After MAE pre-training, we keep the MAE encoder $g$ and append a segmentation decoder $h$ to build the segmentation model $f=h\circ g$ (\hyperref[fig2]{Fig.2b-c}). Given an input image $x_s$ and label $y_s$ from the source domain and an input image $x_t$ from the target domain, the teacher model $f_\theta$ takes as input the target image $x_t$ and generates pseudo labels $f_\theta(x_t)$, with gradient detached. The student model $f_\phi$ is then optimized by minimizing the segmentation loss between the predictions of $x_t^M$/$x_s^M$ and $f_{\theta}(x_t)$/$y_s$, which can be formulated as:  
\begin{equation}
\label{eq:L_mpl}
\mathcal{L}_{MPL} = \mathcal{L}_{Seg}(f_{\phi}(x_t^M),f_{\theta}(x_t))+\beta\mathcal{L}_{Seg}(f_{\phi}(x_s^M),y_s)
\end{equation}
where $\beta$ is the weight of source prediction and set as 0.5. 
The teacher model's parameters $\theta$ are then updated during training via exponential moving average (EMA) based on the student model's parameters $\phi$~\cite{NIPS2017_68053af2}.

\begin{equation}
\label{eq:ema_update}
\theta_{t+1} \gets \alpha \theta_{t} + (1-\alpha)\phi_t, 
\end{equation}
where $t$ and $t+1$ indicate training iterations and $\alpha$ is the EMA update weight. For model initialized from the large-scale MAE pretraining, we set $\alpha$ as 0.999 during the first 1,000 steps and 0.9999 afterwards. For model pretrained on small-scale source and target datasets (\eg, only dozens of scans), we set $\alpha$ as 0.99 during the first 1,000 steps, 0.999 during the next 2,000 steps, and 0.9999 for the remaining training. The teacher model $f_{\theta}$ is initialized with student model's parameters $\phi$ after some warm-up training (\eg, 1,000 iterations) on the source-domain data. 

\subsection{3D Global-Local Collaboration (GLC)}
Directly applying MPL for UDA segmentation with large domain shift (\eg, cross-modality/sequence) may lead to unreliable pseudo-label and disrupt the training. Therefore, we design a GLC module (\hyperref[fig2]{Fig.2c}) to improve pseudo-labeling by leveraging the spatial global-local contextual relations induced by the inherent anatomical distribution prior in medical images. With the image encoder pretrained to extract image features at both local and global levels during multi-scale MAE, we take advantage of the global-local contextual relations by concatenating local and global semantic features in the latent space and make prediction based on the fused features. We differ from previous study~\cite{Chen_2019_CVPR} by only applying GLC on the output of the encoder $g$ instead of all layers to save computation cost and employing a different regularization to prevent segmentation decoder from predicting solely based on local features. 

In GLC, a binary mask $M$ is used to indicate the corresponding location of the local patch $x$ inside the downsampled global volume $X$. The encoder $g$ takes as input $x$ and $X$ and generates the local latent feature $\chi_{loc} = g(x)$ as well as cropped and resized global latent feature $\chi_{glo}=\mathit{upsample}(M \odot g(X))$, where $\odot$ indicates cropping $g(X)$ based on $M$ followed by upsampling to match the spatial size of $\chi_{loc}$. Therefore, segmenting a local patch $x$ can be rewritten as $f(x)=h(\chi_{loc}\oplus\chi_{glo})$, where $\oplus$ is the concatenation along channel dimension (\hyperref[fig2]{Fig.2c}). In addition, $f$ is also trained on downsampled global volume $X$ with $\mathcal{L}_{Seg}(f(X),Y)$), in which the global latent feature $g(X)$ is duplicated and $f(X) = h(g(X)\oplus g(X))$, to prevent model from solely relying on local semantic features and encourage the encoder to extract meaningful semantic features from both local and global levels.

We also add a regularization term between the $\chi_{loc}$ and $\chi_{glo}$ to maintain their similarity following~\cite{Chen_2019_CVPR}. Instead of the $\mathcal{L}_2$ regularization used in~\cite{Chen_2019_CVPR}, we maximize the cosine similarity between the $\chi_{loc}$ and $\chi_{glo}$ as:
\begin{equation}
\mathcal{L}_{cos}(x, X) = 1 - \frac{\chi_{loc}\cdot\chi_{glo}}{\max(\| \chi_{loc} \|_2, \| \chi_{glo} \|_2, \epsilon)}
\end{equation}
where $\epsilon$ is used to prevent zero-division. The loss function for GLC calculated on the source data is formulated as: 
\begin{align}
\label{eq.L_gs}
\mathcal{L}_{GLC}^{S} &= \gamma(\mathcal{L}_{Seg}(f_{\phi}(X_s),Y_s)+\mathcal{L}_{Seg}(f_{\phi}(X_s^M),Y_s))
\nonumber\\
&+\delta(\mathcal{L}_{cos}(x_s, X_s) + \mathcal{L}_{cos}(x_s^M, X_s^M))
\end{align}
where $\gamma$ and $\delta$ are the weights of the auxiliary global loss and cosine similarity, and set as $\gamma=0.05$ and $\delta= 0.025$ in our experiments. Similarly, the GLC loss is also calculated on the target data based on pseudo-label $f_{\theta}(X_t)$ and formulated as:
\begin{align}
\label{eq.L_gt}
\mathcal{L}_{GLC}^{T} &= 2\gamma\mathcal{L}_{Seg}(f_{\phi}(X_t^M),f_{\theta}(X_t)) + 2\delta\mathcal{L}_{cos}(x_t^M, X_t^M)
\end{align}
Therefore, the overall loss function of GLC is:
\begin{align}
\label{eq.L_global}
\mathcal{L}_{GLC} &= \mathcal{L}_{GLC}^{S}+\mathcal{L}_{GLC}^{T}
\end{align}
With the regular fully-supervised segmentation loss on source data $\mathcal{L}_{FSS} = \beta\mathcal{L}_{Seg}(f_{\phi}(x_s),y_s)$, where $\beta$ is defined as in \hyperref[eq:L_mpl]{Eq.2}, the overall objective function $\mathcal{L}$ for centralized UDA is formulated as:
\begin{equation}
\label{eq.L_center}
\mathcal{L} = \mathcal{L}_{FSS}+\mathcal{L}_{MPL}+\mathcal{L}_{GLC}
\end{equation}
It is clear that \hyperref[eq.L_center]{Eq.8} requires centralized and synchronous access to source and target data. In the section \hyperref[sec.fuda]{3.5} and \hyperref[sec.ttuda]{3.6}, we demonstrate how MAPSeg can be adapted to federated (decentralized and synchronous access to data) and test-time (decentralized and asynchronous access to data) UDA scenarios. 

\subsection{Extension to Federated UDA}
\label{sec.fuda}
In reality, labeled source-domain data and unlabeled target-domain data are often collected at different sites. We consider a practical scenario where a server (\eg a major hospital) hosts potentially large amount of both labeled and unlabeled scans, and distributed clients (\eg clinics or imaging sites) possess only unlabeled images. This is an under-explored scenario as FL typically assumes either fully or partially labeled data from all clients. We extend MAPSeg to solve this federated multi-target UDA problem according to the details in Algorithm 1 of Appendix \cref{sec:recipe}. Specifically, the server updates the student model $f_\phi$ by minimizing the loss for the labeled source-domain data $D_S$:
\begin{align}
    \mathcal{L}_s 
    &= \beta(\mathcal{L}_{seg}(f_\phi(x_s), y_s)+\mathcal{L}_{seg}(f_\phi(x_s^M), y_s)) \nonumber\\
    &+\gamma(\mathcal{L}_{seg}(f_\phi(X_s), Y_s)+\mathcal{L}_{seg}(f_\phi(X_s^M), Y_s)) \nonumber\\
    &+ \delta(\mathcal{L}_{cos}(x_s, X_s) + \mathcal{L}_{cos}(x_s^M, X_s^M)) \label{eq:loss_server}
\end{align}
The clients update the student model $f_\phi$ by minimizing the loss for its own unlabeled target-domain data $D_T^k$:
\begin{align}
    \mathcal{L}_u
    &= \beta(\mathcal{L}_{seg}(f_\phi(x_t^M), f_\theta(x_t))+\mathcal{L}_{seg}(f_\phi(x_t), f_\theta(x_t))) \nonumber\\
    &+ \gamma(\mathcal{L}_{seg}(f_\phi(X_t^M), f_\theta(X_t))+\mathcal{L}_{seg}(f_\phi(X_t), f_\theta(X_t))) \nonumber\\
    &+ \delta(\mathcal{L}_{cos}(x_t, X_t) + \mathcal{L}_{cos}(x_t^M, X_t^M)) \label{eq:loss_client}
\end{align}
Comparing to the centralized UDA loss (\hyperref[eq.L_center]{Eq.8}), we decompose it into two components: fully supervised loss for server training (\hyperref[eq:loss_server]{Eq.9}) and self-supervised loss for client updates (\hyperref[eq:loss_client]{Eq.10}), which avoids the need for centralized data. After each local update, each client sends the EMA teacher model parameters $\theta$ to the server for aggregation following typical federated averaging\cite{mcmahan2017communication}.

\subsection{Extension to Test-time UDA}
\label{sec.ttuda}
Test-time UDA often involves two separate stages of training, including the source-only training at one center and the target-only finetuning at another site. In the federated UDA setting, \hyperref[eq:loss_server]{Eq.9} and \hyperref[eq:loss_client]{Eq.10} are jointly used to update the server model through synchronous federated averaging after each round. We can further ease the constraint of synchronous communication between source and target sites by training $f_\phi$ on the source data using \hyperref[eq:loss_server]{Eq.9} for some (\eg 1,000) warm-up steps before distributing the model parameters $\phi$ to the target site for initializing the teacher model $f_\theta$. On the target site, $f_\theta$ provides stable pseudo-labels to guide the self-supervised training with \hyperref[eq:loss_client]{Eq.10} and is updated by the EMA of $\phi$ following \hyperref[eq:ema_update]{Eq.3}. We find that in this asynchronous setting MAPSeg still performs well on the target-domain data, albeit with a minor performance tradeoff on the source-domain data (see \hyperref[tab:testtime]{Tab.3}).

%-------------------------------------------------------------------------
\subsection{Implementation Details} \label{section:2.1}

\noindent\textbf{Model architecture and implementation.} We implement the encoder backbone $g$ using 3D-ResNet-like CNN. The segmentation decoder $h$ is adapted from DeepLabV3~\cite{chen2017rethinking}. The framework is implemented using PyTorch. More details of the model and the training procedure are provided in Appendix \cref{sec:archite} and \cref{sec:recipe}. 

\noindent\textbf{Selecting the best model.} For choosing the best model during training, some studies choose to train for fixed iterations and use the last checkpoint. On the other hand, some of the previous UDA studies~\cite{8988158,Chen_Dou_Chen_Qin_Heng_2019} face a dilemma in selecting the best model during training by validating against a hold-out portion of target-domain labels, which is unrealistic as UDA assumes full absence of target labels. We demonstrate that MPL not only provides an efficient pathway to domain adaptative segmentation but also serves as an indicator of how well the model is being adapted to the target domain. We validate the model after each epoch and the best model is selected based on the score: 
$\mathit{Score}=\mathit{Dice}_{Src}-0.5\times\overline{\mathcal{L}_{Seg}}(f_{\phi}(x_t^M),f_{\theta}(x_t))$, where $\mathit{Dice}_{Src}$ is the Dice score on source-domain validation set and $\overline{\mathcal{L}_{Seg}}(f_{\phi}(x_t^M),f_{\theta}(x_t))$ is the mean of $\mathcal{L}_{Seg}(f_{\phi}(x_t^M),f_{\theta}(x_t))$ during the last training epoch. From \hyperref[eq:L_seg]{Eq.1}, it is clear that $ \lim_{\hat{y}\to y} \mathcal{L}_{seg}(\hat{y},y)=-1$, therefore, $Score$ has an upper bound of $1.5$. We demonstrate in \hyperref[tab:cardiac]{Tab.4} that the difference between validation using target labels versus $Score$ is acceptable (81.2 vs. 80.3). Even without accessing target labels for validation, MAPSeg still surpasses the previous SOTA results that use target labels for validation. It is worth noting that we only use target labels for validation in \hyperref[tab:cardiac]{Tab.4} for a fair comparison with previously reported results; other results presented use $Score$ for validation by default. For federated and test-time UDA, $\mathit{Score} = -\overline{\mathcal{L}_{Seg}}(f_{\phi}(x_t^M),f_{\theta}(x_t))$.

\section{Training details}
\label{sec:HFGD:training_settings}

Our training settings, unless specified, strictly follow CAR~\cite{cCAR}.
%
When performing ablation studies, we use the plain setting, including SGD optimizer and learning, to provide a simple, \textbf{clean setting} to study method effectiveness.
When comparing with state-of-the-arts, we utilize the advanced settings. 
We applied training settings as follows:
\begin{table}[h]
\centering
\small
\resizebox{\linewidth}{!}{%
\begin{tabular}{l|c|c} 
    \toprule
    Settings    & Plain~\cite{cCCNet,cCAR} \quad & Advanced~\cite{cCAR} \quad \\
    \midrule
    \midrule
    Batch size & 16 & 16  \\
    Optimizer & SGD & AdamW \\
    Learning rate decay & \textit{poly}  & \textit{poly} \\
    Initial Learning rate & 0.01 & 0.00004 \\
    Weight decay & 0.0001  & 0.05 \\
    Photo Metric Distortion & - & \checkmark \\
    Sync Batch Norm & \checkmark & \checkmark \\
    \bottomrule
\end{tabular}
}
\label{tab:urd:ablation_training_settings}
\end{table}
\section{Experiments on Pascal Context Dataset}
%\raggedbottom
The Pascal Context~\cite{cPascalContext} dataset contains 4,998 training images and 5,105 testing images.
%
Following the common practice, we use its 59 semantic classes to conduct the ablation studies and experiments.
%
Unless specified, we train the models on the training set for 30K iterations for the ResNet backbone and 40K for Swin-Large and ConvNeXt-Large.

%%%%%%%% Problem identification %%%%%%%%%%%%%%%%%%%%%%%%%%%%%%


\begin{table}[t]
\centering
\small
\resizebox{\linewidth}{!}{%
\begin{tabular}{c|l|c|c} 
\toprule
Backbone & Dilated or Upsampler & Extra Encoding & mIOU(\%)  \\
\midrule
\midrule
ResNet-50 & -  &  Identity      &  45.87 \\
ResNet-50 & Dilation, $OS=8$ &  Identity      & \textbf{47.72}  \\
ResNet-50 & SFPN, $OS=4$ &  Identity      & 47.14  \\
\midrule
ResNet-50 & -  & ASPP~\cite{cDeepLabV3Plus}           & 46.41  \\
ResNet-50 & Dilation, $OS=8$  & ASPP~\cite{cDeepLabV3Plus}           & \textbf{48.59}  \\
ResNet-50 & SFPN, $OS=4$ & ASPP~\cite{cDeepLabV3Plus}           &  47.81 \\
\midrule
ResNet-50 & -  & OCR~\cite{cOCR}            & 45.50 \\
ResNet-50 & Dilation, $OS=8$ & OCR~\cite{cOCR}            & \textbf{48.23} \\
ResNet-50 & SFPN, $OS=4$ & OCR~\cite{cOCR}            & 47.39 \\
\midrule
ResNet-50 & - & SA~\cite{cNonLocal} & 45.02 \\
ResNet-50 & Dilation, $OS=8$ & SA~\cite{cNonLocal} & \textbf{48.32} \\
ResNet-50 & SFPN, $OS=4$ & SA~\cite{cNonLocal} & 45.90 \\
\midrule
ResNet-50 & - & SA~(CAR~\cite{cCAR}) &  47.18  \\
ResNet-50 & Dilation, $OS=8$ & SA~(CAR~\cite{cCAR}) & \textbf{50.50} \\
ResNet-50 & SFPN, $OS=4$ & SA~(CAR~\cite{cCAR}) & 48.51  \\
\bottomrule
\end{tabular}
}
\caption{
Comparisons between a dilation method and a state-of-the-art upsampler-based method (i.e. SemanticFPN) with various extra encoding techniques on the Pascal Context dataset.
Results demonstrate that dilation is more accurate than SemanticFPN with a substantial margin.
Identity means using no extra encoding (i.e. a basic FCN).
% \hy{Verification of SemanticFPN is generally worse than dilation on the Pascal Context dataset.}
% \hy{Identity means using the backbone result directly, similar to FCN, allows for the plainest way of verifying our identified problem.}
}
\label{tab:ablation_dilation_vs_fpn}
\end{table}

\subsection{Problem verification}
\label{sec:problem}

\noindent\textbf{SemanticFPN is generally worse than dilation.}
We conduct experiments in Tab.~\ref{tab:ablation_dilation_vs_fpn} to demonstrate the issue we identify (observation 1 in Fig.~\ref{fig:hfgd:concept}).
% 
Many famous methods for context encoding (``Extra Encoding'') produce less accurate predictions when combined with an upsampler net (i.e. SemanticFPN) than directly using a dilated backbone, even if the upsampler produces higher-resolution feature maps.
%
Refer to our appendix, for detailed settings.

\noindent\textbf{Negative influence from the upsampler on the backbone.}
We find the joint fine-tuning of the upsampler and the backbone results in deteriorated backbone features.
%
To demonstrate it, we modify SemanticFPN by introducing an auxiliary FCN to predict the mask from the high-level features produced by the backbone (see Issues in Fig.~\ref{fig:hfgd:concept}).
The predictions from the auxiliary FCN branch become worse than the original FCN (44.35\% vs 45.87\% mIOU).
Another variant stops the auxiliary FCN's gradient from propagating back to the backbone, which produces even worse results (40.04\%).

\noindent\textbf{Advantage of pre-training.}
We conduct experiments to show the benefit of using the pre-trained backbone in Tab.~\ref{tab:ablation_imagenet_simple}.
Due to limited training data, there still exists a substantial gap after 6 times training iterations.
Thus, protecting the backbone to ensure its generalization is necessary and critical.
More experiments about this are presented in the supplementary.

%%%%%%%%%%%%%%%%%%%%%%%%%%%%%%%%%%%%%%%%%%%%%%%%%%%%%%


\subsection{Ablation studies on HFGD}

\noindent\textbf{Ablation studies on HFGM.}
%
In Tab.~\ref{tab:ablation_hfgm}, we evaluate the effectiveness of our proposed HFGM based on SemanticFPN (\ie no ``Extra Encoding'').
%
Although using only HFG (``+guidance'') or only axial attention (``+AA'') is helpful, using the full HFGM brings the most gain (1.80\% mIOU).
probably because AA can effectively broadcast the guidance information of HFG to all spatial locations (also see Sec.~\ref{sec:method:hfgm} for more discussion).




\noindent\textbf{Importance of stopping gradients.}
We use several stop-grad operations to protect the backbone weights, especially the early low-level weights, and only allow gradients from the backbone branch to update their weights gradually (see. Fig.~\ref{fig:URD:Arch}).
We tested removing all stop-grad operations and obtained significantly decreased accuracy (48.94\% vs 48.50\%).
%
% We also analyze what happens if we no longer protect the backbone branch (i.e. letting gradients propagate from the upsampler branch).
% We found that doing so resulted in a drop in mIOU down to 48.50\%.


\noindent\textbf{Ablation studies on CAE.}
%
% We conduct ablation studies on CAE in Tab.~\ref{tab:ablation_cae} and Tab.~\ref{tab:ablation_hfgm_cae}.
% 
In Tab.~\ref{tab:ablation_cae}, we compare with CAR under different settings on the Pascal Context dataset 
since CAR~\cite{cCAR} performs best in Tab.~\ref{tab:ablation_dilation_vs_fpn} and CAE is based on CAR~\cite{cCAR}.
Results show that CAE is more compatible than CAR when using SFPN and ``SFPN + HFGM''.
%
Combined with results in Tab.~\ref{tab:ablation_dilation_vs_fpn}, our CAE design outperforms all the other alternatives (``Extra Encodings'') with and without SemanticFPN and approaches the accuracy of the best dilation-based model in Tab.~\ref{tab:ablation_dilation_vs_fpn}.


Similar to experiments in Tab.~\ref{tab:ablation_hfgm},
we also analyze the effects of CAE with different HFGM settings.
%
HFGM now provides better guidance to the upsampler with the help of CAE, leading to further improved final results (50.28\% vs 48.94\% and 49.22\% vs 47.67\% in Tab.~\ref{tab:ablation_hfgm}).



%Our CAE design surpasses the other alternatives (``Extra Encodings'') by a substantial margin, with and without SemanticFPN~\cite{cPanopticFPN}.
%Note that CAE is the best for methods using an upsampler while SA (CAR) is the best for dilation methods, showing that previous context encodings are unsuitable for OS=32 feature maps, and our modification on SA (CAR) effectively improves it.


\begin{table}[t]
\centering
\small
\resizebox{\linewidth}{!}
{\def\arraystretch{1} \tabcolsep=0.6em 
\begin{tabular}{l|c|c|c} 
\toprule
Training Iterations & 30K & 90K & 180K  \\
\midrule
\midrule
ResNet-50 (ImageNet) + FCN & 45.87 & - & - \\
ResNet-50 (scratch) + FCN  & 26.13 & 31.38 & 34.00 \\
\bottomrule
\end{tabular}
}
\caption{
% Simple experiments to present the importance and advantage of ImageNet pre-train.
Importance of using ImageNet pre-trained weights.
Experiments are conducted on the Pascal Context dataset (mIOU\%).
%
% Trained on Pascal Context dataset.
% %
% Results are in mIOU(\%).
}
\label{tab:ablation_imagenet_simple}
\end{table}

\begin{table}[t]
\centering
\small
\resizebox{\linewidth}{!}
{
\begin{tabular}{c|c|c|c|l} 
\toprule
Backbone   & Upsampler  & Extra Encoding & HFGM & mIOU(\%)  \\
\midrule
\midrule
ResNet-50  & SFPN & Identity & -      & 47.14  \\
ResNet-50  & SFPN & Identity & + guidance & 47.67 (\textcolor{black}
{$+$0.53}) \\
ResNet-50  & SFPN & Identity & + AA  & 47.88 (\textcolor{black}
{$+$0.74})  \\
ResNet-50  & SFPN & Identity & full  & 48.94 (\textcolor{black}{$+$1.80)}\\
\midrule
ResNet-50  & SFPN & Our CAE & -      & 48.76    \\
ResNet-50  & SFPN & Our CAE & + guidance & 49.22 (\textcolor{black}{$+$0.46}) \\
ResNet-50  & SFPN & Our CAE & + AA & 49.06 (\textcolor{black}{$+$0.26}) \\
ResNet-50  & SFPN & Our CAE & full  & \textbf{50.28} (\textcolor{black}{$+$1.52}) \\
\bottomrule
\end{tabular}
}
\caption{
Ablation studies on different HFGM settings on Pascal Context dataset.
Using both guidance and AA in HFGM brings most gain (1.8\%/1.52\% mIOU).
AA and the proposed high-level feature guidance co-operate extremely well
% are beneficial for each other 
probably because AA can effectively back-propagate the guidance signal to all spatial locations.
% HFGM effectively improves the accuracy, especially when it contains an axial attention (AA) layer since AA can effectively back-propagate the guidance signal from the good high-level features (i.e. output of CAE) to all spatial locations, resulting in more accurate upsampled feature representations.
}
\label{tab:ablation_hfgm}
\end{table}

\begin{table}[t]
\centering
\small
\resizebox{\linewidth}{!}{%
\begin{tabular}{c|l|c|c} 
\toprule
Backbone & Dilated or Upsampler & Extra Encoding & mIOU(\%)  \\
\midrule
\midrule
ResNet-50 & -  &  Identity      &  45.87 \\
ResNet-50 & Dilation, $OS=8$ &  Identity      & \textbf{47.72}  \\
ResNet-50 & SFPN, $OS=4$ &  Identity      & 47.14  \\
\midrule
ResNet-50 & -  & ASPP~\shortcite{cDeepLabV3Plus}           & 46.41  \\
ResNet-50 & Dilation, $OS=8$  & ASPP~\shortcite{cDeepLabV3Plus}           & \textbf{48.59}  \\
ResNet-50 & SFPN, $OS=4$ & ASPP~\shortcite{cDeepLabV3Plus}           &  47.81 \\
\midrule
ResNet-50 & -  & OCR~\shortcite{cOCR}            & 45.50 \\
ResNet-50 & Dilation, $OS=8$ & OCR~\shortcite{cOCR}            & \textbf{48.23} \\
ResNet-50 & SFPN, $OS=4$ & OCR~\shortcite{cOCR}            & 47.39 \\
\midrule
ResNet-50 & - & SA~\shortcite{cNonLocal} & 45.02 \\
ResNet-50 & Dilation, $OS=8$ & SA~\shortcite{cNonLocal} & \textbf{48.32} \\
ResNet-50 & SFPN, $OS=4$ & SA~\shortcite{cNonLocal} & 45.90 \\
\midrule
ResNet-50 & - & SA~(CAR~\shortcite{cCAR}) &  47.18  \\
ResNet-50 & Dilation, $OS=8$ & SA~(CAR~\shortcite{cCAR}) & \textbf{50.50} \\
ResNet-50 & SFPN, $OS=4$ & SA~(CAR~\shortcite{cCAR}) & 48.51  \\
\midrule
ResNet-50 & - & Our CAE  & 47.35 \\
ResNet-50 & Dilation, $OS=8$ & Our CAE  & \textbf{49.67} \\
ResNet-50 & SFPN, $OS=4$ & Our CAE  & 48.76 \\
\bottomrule
\end{tabular}
}
\caption{
Experiments to compare the effectiveness of using dilated backbones and upsampler networks on Pascal Context dataset.
Using a dilated backbone consistently outperform using an upsampler net on several representative context encoding methods.
% of different encoding modules for high-level feature enhancements on Pascal Context dataset.
% 
The final feature maps are bilinearly upsampled to the image resolution immediately followed by the class label prediction.
% We train all the methods under exactly same and fair settings.
%
% Experiments without upsampling head produces $OS=32$ feature maps while those with Semantic FPN (SFPN)~\cite{cPanopticFPN} produces $OS=4$ feature maps.
}
\label{tab:ablation_cae}
\end{table}
\begin{table}[t]
\centering
\small
\resizebox{\linewidth}{!}
{
\begin{tabular}{c|c|c|c|l} 
\toprule
Backbone   & Upsampler  & Extra Encoding & HFGM & mIOU(\%)  \\
\midrule
\midrule
ResNet-50  & SFPN & Our CAE & -      & 48.76    \\
ResNet-50  & SFPN & Our CAE & + guidance & 49.22 (\textcolor{black}{$+$0.46}) \\
ResNet-50  & SFPN & Our CAE & + AA & 49.06 (\textcolor{black}{$+$0.26}) \\
ResNet-50  & SFPN & Our CAE & full  & \textbf{50.28} (\textcolor{black}{$+$1.52}) \\
\bottomrule
\end{tabular}
}
\caption{
Ablation studies to analyze the effects of CAE with different HFGM settings on the Pascal Context dataset.
%
HFGM provides better guidance thanks to the improved high-level features extracted by CAE.
}
\label{tab:ablation_hfgm_cae}
\end{table}

\noindent\textbf{Ablation studies on U-SFPN.}
%
In Tab.~\ref{tab:ablation_upsampling}, we conduct ablation studies on U-SFPN to verify the effectiveness of our modification on SFPN while fixing CAE and HFGM.
Replacing SFPN with U-SFPN improves the mIOU of ResNet-50 (CAE + HFGM) by 0.48\%, reaching 50.76\%.
The improvement is even larger (1.14\%) when using Swin-Large as the backbone 
%(Tab.~\ref{tab:ablation_upsampling})
.

\begin{table}[t]
\centering
\small
\resizebox{\linewidth}{!}
{\def\arraystretch{1} \tabcolsep=0.6em 
\begin{tabular}{c|c|c|c|l} 
\toprule
Backbone & CAE & HFGM & Upsampler & mIOU(\%)  \\
\midrule
\midrule
ResNet-50  & \checkmark & \checkmark & SFPN & 50.28 \\
ResNet-50  & \checkmark & \checkmark & U-SFPN   & \textbf{50.76} (\textcolor{black}{$+$0.48}) \\
\midrule
Swin-Large & \checkmark & \checkmark & SFPN & 59.32 \\
Swin-Large & \checkmark & \checkmark & U-SFPN   & \textbf{60.46} (\textcolor{black}{$+$1.14}) \\
\bottomrule
\end{tabular}
}
\caption{Ablation studies on different upsampling heads on the Pascal Context dataset.
Our proposed U-SFPN upsampling head consistently outperforms semantic FPN (SFPN) for both CNN and Transformer backbones.
% U-SFPN on CAE + CFGM on the Pascal Context dataset. 
% \dknote{Low priority for $OS=2$ results}
}
\label{tab:ablation_upsampling}
\end{table}




\noindent\textbf{Module-level ablation studies on HFGD.}
%
In Tab.~\ref{tab:ablation_hfgd},
% Finally, 
%we conduct ablation studies on each module of HFGD using previously found best configurations in Tab.~\ref{tab:ablation_cae}-\ref{tab:ablation_upsampling}.
%Using all modules together leads to significant accuracy improvement, indicating the effectiveness of the overall architecture.
%
We performed ablation studies on each module of HFGD using the best configurations found in Tab.~\ref{tab:ablation_cae}-\ref{tab:ablation_upsampling}. Using all modules together significantly improved accuracy, indicating the overall architecture's effectiveness.


\begin{table}[t]
\centering
\small
\resizebox{\linewidth}{!}
{\def\arraystretch{1} \tabcolsep=0.8em 
\begin{tabular}{c|c|l|c|c} 
\toprule
Backbone & Extra Encoding & Upsampler & HFGM & mIOU(\%)  \\
\midrule
\midrule
ResNet-50 & Identity & SFPN & - & 47.14 \\
ResNet-50 & Our CAE & SFPN    & -     & 48.76 \\
ResNet-50 & Identity  & U-SFPN & -     & 46.99\\
ResNet-50 & Our CAE& U-SFPN & - & 48.67 \\
ResNet-50 & Our CAE & U-SFPN & \checkmark & \textbf{50.76} \\
\midrule
Swin-Large & Our CAE& SFPN     &  -    & 56.78 \\
Swin-Large & Identity & U-SFPN &  - & 58.69 \\
Swin-Large & Our CAE & U-SFPN &  - & 55.76 \\
Swin-Large & Our CAE & U-SFPN & \checkmark & \textbf{60.46} \\
\bottomrule
\end{tabular}
}
\caption{Ablation studies on the proposed three modules using previously found best configurations in Tab.~\ref{tab:ablation_cae}-\ref{tab:ablation_upsampling} on Pascal Context dataset.
}
\label{tab:ablation_hfgd}
\end{table}

\begin{table}[h]
\centering
\small
\resizebox{\linewidth}{!}
{\def\arraystretch{1} \tabcolsep=0.6em 
\begin{tabular}{l|c|c|c|l} 
\toprule
Backbone   & Upsampler  & EE & HFGM & mIOU(\%)  \\
\midrule
\midrule
R50  & U-SFPN & Our CAE & -     & 48.67   \\
R50  & U-SFPN & Our CAE & \checkmark  & \textbf{50.76} (\textcolor{black}{$+$2.09}) \\
\midrule
R50  & FaPN & - & -      &  47.50 \\
R50  & FaPN & - & \checkmark  &  \textbf{49.87} (\textcolor{black}{$+$2.37}) \\
\midrule
R50  & Uper & PPM & -      & 48.25 \\
R50  & Uper & PPM & \checkmark  & \textbf{49.96}   (\textcolor{black}{$+$1.71}) \\
\midrule
R50 (D8) & DeepLabV3+ & ASPP & -      &  48.11\\
R50 (D8) & DeepLabV3+ & ASPP & \checkmark  & \textbf{49.70} (\textcolor{black}{$+$1.59}) \\
\bottomrule
\end{tabular}
}
\caption{
Ablation studies on different upsamplers w/o or w/ our HFGM on Pascal Context dataset.
\textit{EE}: Extra encoding.
\textit{D8}: 
% The backbone uses 
Dilated convolutions with ($OS=4$).
% output stride = 8.
}
\label{tab:ablation_hfg_other_upsamplers}
\end{table}

\subsection{Computational cost of HFGD}
%
The computational cost of our HFGD and two other state-of-the-art methods are listed in Tab.~\ref{tab:HFGD:flops}.
% 
HFGD uses much lower GFLOPs than a similar dilation model (Self-Attention + CAR~\cite{cCAR}) but achieves better mIOU (50.76\% vs 50.50\%~\cite{cCAR}).
% 
Compared with SemanticFPN, HFGD ($OS=4$) achieves 3.62\% % mIOU gain with an affordable extra computation cost (71.63 GFLOPs vs 45.65 GFLOPs).

\begin{table}[t]
\centering
\small
\resizebox{\linewidth}{!}{%
\begin{tabular}{l|l|l|l}
\toprule%[1pt]
Method & Backbone & GFLOPs & mIOU\%\\
\midrule
\midrule
SA (CAR) & ResNet-50 (D8) & 158.96 & 50.50\\
\midrule
SemanticFPN & ResNet-50  & 45.65  & 47.14  \\
\midrule
HFGD (OS=4) &  ResNet-50 & 71.63  &  50.76 \\
HFGD (OS=2) &  ResNet-50 & 153.62 & 51.00  \\
\bottomrule%[1pt]
\end{tabular}
}
\caption{
Computational analysis of HFGD on a $513\times513\times3$ input image.
Previously, although more efficient,
upsampling-based methods (e.g. SemanticFPN) cannot produce as accurate results as the dilation-based methods.
HFGD ($OS=4$) closes this accuracy gap while still being efficient.
If a similar computation budget is given, HFGD ($OS=2$) can further improve the accuracy.
}
\label{tab:HFGD:flops}
\end{table}

\subsection{Comparison with the state-of-the-art methods}

To compare with the state-of-the-art, we adopt ConvNeXt-L as the backbone for our HFGD.
% 
We set the training iterations to 40K while
all the other training settings are the same as stated in Sec.~\ref{sec:HFGD:training_settings}.
% except that we set the training iterations to 40K.
% 
As shown in Tab.~\ref{tab:urd:SOTA-PascalContext}, our HFGD achieved 63.8\% mIOU with single-scale without flipping and 64.9\% mIOU with multi-scales with flipping, outperforming previous state-of-the-art by 1\% mIOU in ECCV-2022.
 %
HFGD is now the new state-of-the-art method on Pascal Context for the methods that only use the ImageNet pre-trained backbone without extra techniques~\cite{cAugReg,cFocalLoss,cVNet}. 


\begin{figure*}[th]
\centering
\includegraphics[width=1.0\linewidth]{images/vis_main.pdf}
\caption{
Visual comparisons 
% (Upper-left:Pascal Context, upper-right:COCOStuff, bottom:Cityscapes) 
between SemanticFPN ($OS=4$), HFGD ($OS=4$), and HFGD ($OS=2$).
Zoom in to see better.
The results are obtained using single-scale without flipping.
%
More visualizations are presented in the supplementary.
}
\label{fig:CFGD:main-R50-Vis}
\end{figure*}

\begin{figure}[th]
\centering
\includegraphics[width=1.0\linewidth]{images/umap_vis.pdf}
\caption{
% Comparing the baselines (FCN and Semantic) with our proposed solution (HFGM or entire HFGD) using UMAP visualization on the Pascal Context dataset.
UMAP~\cite{cUMAP} Visualization on Pascal Context. 
HFGD features are most separable from the inter-class perspective and most compact from the intra-class perspective, resulting in the best accuracy (Tab.~\ref{tab:ablation_hfgd}).
}
\label{fig:umap_vis}
\end{figure}

\subsection{Apply high-level guide to existing upsamplers }

Though we recommend using U-SFPN + CAE to fit our proposed HFGM for efficiency, many existing methods attempt to improve the accuracy of upsamplers by improving intermediate-specific operations. 
For example, FaPN~\cite{cFaPN} uses SENet~\cite{cSENet} and deformable convolutions to try to align low-level features with high-level features.
%
UperNet~\cite{cUper} uses PPM~\cite{cPSPNet} to improve the high-level features of the backbone network.
%

From another perspective, HFGM directly optimizes the final upsampling quality rather than the intermediate process.
Thus, it should be able to improve the accuracy
of these upsamplers further, using high-level features as teachers to
constrain their upsampling results.

Using the ablation experimental setup in the main paper, Tab~\ref{tab:ablation_hfg_other_upsamplers} verifies our HFGM can effectively boost mIOU for different upsamplers. 
%
We believe that HFGM has generalizability to other similar upsamplers.
%
\section{Experiments on COCOStuff164k Dataset}

\iffalse
\begin{figure}[htbp]
\centering
\includegraphics[width=1.0\linewidth]{images/vis_pascalcontext_r50_vs.pdf}
\caption{
Visual comparisons on Pascal Context between SemanticFPN, Self-Attention + CAR (denoted as ``Dilated OS=8''), and our HFGD.
All models used ResNet-50 as the backbone.
The results are obtained using single-scale without flipping.
}
\label{fig:CFGD:PascalContext-R50-Vis}
\end{figure}
\fi



COCOStuff164k~\cite{cCocoStuff}, which has become popular in recent years, poses a great challenge for semantic segmentation models due to its high diversity (118k training images and 5000 testing images) and complexity (171 classes). 
% 
We adopt ConvNeXt-Large as our backbone network and follow the training settings described in Sec.~\ref{sec:HFGD:training_settings} and we train our model for 40K iterations.
In Tab.~\ref{tab:HFGD:SOTA-COCOStuff164k}, we compare our proposed HFGD with other state-of-the-art methods.
HFGD outperforms the previous state-of-the-art~\cite{cSegNeXt} by a large margin (49.4\% vs 47.3\% mIOU).


\begin{table}[t]
\centering
\small
{\def\arraystretch{1} \tabcolsep=0.55em 
\begin{tabular}{l|c|c|c|c}
\toprule%[1pt]
Methods & Backbone & Avenue &\multicolumn{2}{c}{mIOU(\%)} \\
& & & SS & MF \\
\midrule
SETR*~\cite{cSETR}           & ViT-L           & CVPR'21 & - & 55.8 \\
DPT*~\cite{cDPT}             & ViT-Hybrid      & ICCV'21 & - & 60.5 \\
Segmenter*~\cite{cSegmenter} & ViT-L           & ICCV'21 & - & 59.0 \\
OCNet*~\cite{cOCNet}         & HRNet-W48       & IJCV'21 & - & 56.2 \\
CAA*~\cite{cCAA}             & EfficientNet-B7 & AAAI'22 & - & 60.5 \\
SegNeXt*~\cite{cSegNeXt}     & MSCAN-L         & NIPS'22 & 59.2 & 60.9 \\
CAA + CAR~\cite{cCAR}        & ConvNeXt-L      & ECCV'22 & 62.7 & 63.9 \\
\midrule%[0.1pt]
HFGD (OS=4)          & ConvNeXt-L & - & \textbf{63.8} & \textbf{64.9} \\
\bottomrule%[1pt]
\end{tabular}
}
\caption{
% \hy{
Comparisons to state-of-the-art methods on Pascal Context dataset.
%
% Note that, in this table, unlike ablation studies, 
Note that methods marked with `$*$' report mIOU from their papers while the others are obtained with our implementation.
\textit{SS:} Single scale performance w/o flipping.
\textit{MF:} Multi-scale performance w/ flipping.
% }
}
\label{tab:urd:SOTA-PascalContext}
\end{table}

\begin{table}[t]
\centering
\small
\resizebox{\linewidth}{!}
{\def\arraystretch{1} \tabcolsep=0.55em 
\begin{tabular}{l|c|c|c|c}
\toprule%[1pt]
Methods & Backbone  & Avenue &\multicolumn{2}{c}{mIOU(\%)}\\
& & & SS & MF \\
\midrule
\midrule
OCR~\shortcite{cOCR,cHRFormer} & HRFormer-B & NIPS'21 & - & 43.3 \\
SegFormer~\shortcite{cSegFormer} & MiT-B5 & NIPS'21 & - & 46.7 \\
CAA~\shortcite{cCAA} & EfficientNet-B5 & AAAI'22 & - & 47.3 \\
SegNeXt~\shortcite{cSegNeXt} & MSCAN-L & NIPS'22 & 46.5 & 47.2 \\
RankSeg~\shortcite{cRankSeg} & ViT-L & ECCV'22 & 46.7 & 47.9 \\
\midrule
HFGD (OS=4) & ConvNeXt-L & - & \textbf{49.0} & \textbf{49.4}    \\ 
\bottomrule%[1pt]
\end{tabular}
}
\caption{
% \hy{
Comparisons to state-of-the-art methods on COCOStuff164k dataset.
%
% Note that, in this table, unlike ablation studies, the reported mIOU of compared methods with `$*$' comes from the related paper. 
%Note that methods marked with `$*$' report mIOU are obtained with our implementation.
%
%See Sec.~\ref{sec:CAR:training-settings} for details.
%
\textit{SS:} Single scale performance w/o flipping.
\textit{MF:} Multi-scale performance w/ flipping.
% }
}
\label{tab:HFGD:SOTA-COCOStuff164k}
\end{table}
\section{Experiments on Cityscapes Dataset}

Cityscapes~\cite{cCityScapes} is a semantic segmentation dataset that consists of high-resolution images of road scenes with accurate annotations.
It has 19 labeled classes and contains 2975/500/1525 training/validation/test images.

\subsection{Ablation studies on feature map resolution}
\label{sec:exps:ablation_os2}
\begin{table}[t]
\centering
\small
\resizebox{\linewidth}{!}
{\def\arraystretch{1} \tabcolsep=1.15em 
\begin{tabular}{l|c|c|l} 
\toprule
Method & Backbone & $OS$ & mIOU(\%)  \\
\midrule
\midrule
SemanticFPN & ResNet-50 & 4      & 76.44   \\
\midrule
HFGD & ResNet-50  & 4      & 79.11    \\
HFGD & ResNet-50  & 2 & \textbf{79.81} (\textcolor{black}{+0.7}) \\
\bottomrule
\end{tabular}
}
\caption{
Ablation studies on different output stride modes of U-SFPN on Cityscapes dataset.
% \dknote{$OS=4$}
}
\label{tab:ablation_ct_os2}
\end{table}

%
We mainly conduct ablation experiments on Cityscapes to verify the superiority of using ultra high resolution feature maps (i.e. $OS=2$) since its GT annotations are the most accurate.
% 
We use ResNet-50 as the backbone and train SemanticFPN, HFGD ($OS=4$), and HFGD ($OS=2$) for 30K iterations following the training settings in Sec.~\ref{sec:HFGD:training_settings}. 
%
As shown in Tab.~\ref{tab:ablation_ct_os2}, the $OS=2$ has further improved the accuracy over $OS=4$ by 0.7\% mIOU (79.11 vs 79.81).

\subsection{Comparison with the state-of-the-art methods}
To compare our method against the state-of-the-art, we use ConvNeXt-Large and follow the training settings described in Sec.~\ref{sec:HFGD:training_settings}.
%
Note that we only compare with methods that are trained only on the Cityscapes fine annotations, similar to many works~\cite{cSegFormer,cKMaXDeepLab}.
%
We set the crop size to 513$\times$1025~\cite{cDeepLabV3,cPanopticDeepLab} and train our HFGD model for 60K iterations for the $OS=4$ version and 80K iterations for the $OS=2$ version.
%
As shown in Tab.~\ref{tab:HFGD:SOTA-Cityscapes}, the proposed HFGD outperforms the previous state-of-the-art proposed in recent years on the Cityscapes validation set.

A similar comparison is conducted on the Cityscapes test set, where we set batch size = 32 following kMaXDeepLab~\cite{cKMaXDeepLab}
while the other settings remain unchanged.
%
Results on the test set can fairly demonstrate the effectiveness of the proposed method since no ground-truths are provided.
%
Note that we did not use hard sample mining.
%
As shown in Tab.~\ref{tab:HFGD:test-Cityscapes}, our HFGD sets new state-of-the-art on Cityscapes test set (when only using fine set for training).
%

\begin{table}[t]
\centering
\small
\resizebox{\linewidth}{!}{
\begin{tabular}{l|c|c|c|c}
\toprule%[1pt]
Methods &Backbone & Avenue &\multicolumn{2}{c}{mIOU(\%)}\\
&  & & SS & MF \\
\midrule
\midrule
RepVGG\shortcite{cRepVGG} & RepVGG-B2 & CVPR'21 & - & 80.6 \\
SETR~\shortcite{cSegFormer} &ViT-L & CVPR'21 & - & 82.2 \\
Segmenter~\shortcite{cSegmenter} &ViT-L & ICCV'21 & - & 81.3 \\
OCR~\shortcite{cOCR,cHRFormer} &HRFormer-B & NIPS'21 & - & 82.6 \\
HRViT-b3~\shortcite{cHRViT}  & MiT-B3 & CVPR'22 & - & 83.2\\
FAN-L~\shortcite{cFANs} & FAN-Hybrid & ICML'22 & - & 82.3 \\
SegDeformer~\shortcite{cSegDeformer} & Swin-L & ECCV'22 & - & 83.5 \\
DPP(Step 1)~\shortcite{cDDP} & ConvNeXt-L & ICCV'23 & 83.0 & 83.8 \\
DPP(Step 3)~\shortcite{cDDP} & ConvNeXt-L & ICCV'23 & 83.2 & 83.9 \\
% \midrule
\midrule%[0.1pt]
HFGD (OS=4)     & ConvNeXt-L         & - & 83.1 & 83.8 \\
HFGD (OS=2)    & ConvNeXt-L              & - & 83.2 & \textbf{84.0} \\
\bottomrule%[1pt]
\end{tabular}
}
\caption{
% \hy{
Comparisons to state-of-the-art methods on Cityscapes validation set.
%
% Note that, in this table, unlike ablation studies, 
%Note that methods marked with `$*$' are obtained with our implementation.
\textit{SS:} Single scale performance w/o flipping.
\textit{MF:} Multi-scale performance w/ flipping.
% }
}
\label{tab:HFGD:SOTA-Cityscapes}
\end{table}

\begin{table}[th!]
\centering
\small
\resizebox{\linewidth}{!}{
\begin{tabular}{l|c|c|c}
\toprule%[1pt]
Methods &Backbone & Avenue &mIOU(\%)\\
\midrule
\midrule
Panoptic-DeepLab\shortcite{cPanopticDeepLab} & SWideRNet & CVPR'20 & 80.4 \\
Axial-DeepLab\shortcite{cAxialDeepLab} & Axial-ResNet-XL & ECCV'20 & 79.9 \\
SETR\shortcite{cSETR} & ViT-L & CVPR'21 & 81.1 \\
SegFormer\shortcite{cSegFormer} & MiT-B5 & NIPS'21 & 82.2 \\
kMaXDeepLab\shortcite{cKMaXDeepLab} (OS=2) & ConvNeXt-L & ECCV'22 & 83.2 \\
% \midrule
\midrule%[0.1pt]
HFGD (OS=2)    & ConvNeXt-L & -  & \textbf{83.3} \\
\bottomrule%[1pt]
\end{tabular}
}
\caption{
Comparisons to state-of-the-art methods on Cityscapes test set.
% \hy{We investigated very seriously to ensure all the listed methods are only trained on Cityscapes fine set (2975).}
We only list methods trained on Cityscapes fine annotation set for fair comparisons.
Due to the policy of the Cityscapes test set, we are only able to present a single result.
}

\label{tab:HFGD:test-Cityscapes}
\end{table}

\section{Visualizations}
We present visual comparisons between SemanticFPN and our HFGD on Pascal Context, CocoStuff and Cityscapes in Fig~\ref{fig:CFGD:main-R50-Vis}.
%
HFGD is clearly more capable of segmenting small/thin objects. 
For example, HFGD even performs better than manual annotation when segmenting very small birds in the sky.
When using OS=2 on Cityscapes, HFGD can segment the remote traffic light partially missed in OS=4
% in Cityscapes, which is partially missing in OS=4, 
which is of great importance in autonomous driving.

\vspace{1mm}
\noindent\textbf{UMAP:} We conducted UMAP~\cite{cUMAP} visualization in Fig.~\ref{fig:umap_vis} to analyze the impact of our HFG/HFGD in addressing the issue presented in Fig.~\ref{fig:hfgd:concept}. 
The final layer features used for classification are used (the features produced by the upsampler for HFGD) in this visualization.
Since Pascal Context has 59 classes, we only selected 8 representative error-prone classes (\ie, hard to discriminate, \eg Floor vs Grass vs Ground vs Snow) in Fig.~\ref{fig:umap_vis} for clarity.
%
% As shown in Fig.~\ref{fig:umap_vis}, t
The class discrimination ability of SemanticFPN is slightly worse than FPN.
``SemanticFPN (HFG)'' restores this ability with the help of HFG.
Additionally, our entire HFGD results in more discriminative inter-class and more compact intra-class features.
Our full solution HFGD obtains more discriminative features (\ie more separable inter-class distance and more compact intra-class representation).

%
%More visualizations are presented in the supplementary.
\section{Conclusion }
In this paper, we propose to use the high-level features as the teacher to guide the training of the upsampler branch (student), resulting in an effective and efficient decoder framework.
%
Specifically, the core of our method is using high-level features as guidance (HFG) and proper stop-gradient operations for the upsampler learning, which effectively addresses the observed issues in Fig.~\ref{fig:hfgd:concept}.
%
In addition, we explore a context-augmented encoder (CAE) to effectively enhance the OS=32 high-level features
and propose a modified version of SemanticFPN to better fit our HFGM.
%
% 
With thorough experiments, HFGD achieves largely improved performance over previous upsampling-based state-of-the-art methods that does not use extra training data, on Pascal Context, COCOStuff, and Cityscapes.
%
HFGD also achieves slightly better results than dilation-based CNNs but using much less computation cost (\eg Self-Attention + CAR~\cite{cCAR}).
%.
%
We refer the readers to our appendix for more ablation studies, experiments, implementation details, and visualizations.

\clearpage


\clearpage
%\setcounter{page}{1}
\maketitlesupplementary
The supplementary material for our work  \textit{SC-VAE: Sparse Coding-based Variational Autoencoder with Learned ISTA} is structured as follows:
%Sec. \ref{section1_s} provides the detailed information of the encoder and decode architecture of the SC-VAE model. 
%Sec. \ref{section2_s} shows the visualization of the dictionary atoms.
%Sec. \ref{section3_s} shows the training loss on the ImageNet dataset with different number of downsampling (upsampling) blocks ($d$) in the encoder (decoder) of the SC-VAE model.
%Sec. \ref{section4_s} shows the visualization results of an unofficial implementation of VIT-VQGAN \cite{yu2021vector}. 
%Sec. \ref{section5_s} shows additional manipulation and interpolation results on FFHQ dataset. 
%Sec. \ref{section6_s} shows additional image patches clustering results on FFHQ and ImageNet datasets. 
%Sec. \ref{section7_s} shows additional unsupervised image segmentation results.
Section \ref{section1_s} details the encoder and decoder architecture of the SC-VAE model. In Section \ref{section2_s}, the dictionary atoms are visualized. In Section \ref{section3_s}, we provide the training losses on the ImageNet dataset when varying the number of downsampling (upsampling) blocks ($d$) in the encoder (decoder) of the SC-VAE model. In Section \ref{section4_s}, the visualized reconstruction results of an unofficial implementation of VIT-VQGAN \cite{yu2021vector} are provided. 
We provide  additional manipulation and interpolation results on the FFHQ dataset in Section \ref{section5_s}, while  additional clustering results of image patches on both FFHQ and ImageNet  are provided in Section \ref{section6_s}. Supplementary unsupervised image segmentation results are given in Section \ref{section7_s}.

%Additional results on image patches clustering and unsupervised image segmentation on FFHQ and ImageNet datasets are then presented in Sec. 2 and Sec. 3, respectively.
\setcounter{section}{0}

\section{The Encoder and Decoder Architecture of SC-VAE} \label{section1_s}
The SC-VAE model's encoder and decoder architecture mirrors that of VQGAN \cite{esser2021taming}. Details about the architecture are provided in Table \ref{figure:encoder_decoder}.
%The encoder and decoder architecture in the SC-VAE model are the same as the architecture used in VQGAN \cite{esser2021taming}, which is described in Table \ref{figure:encoder_decoder}. 
$H$, $W$ and $C$ denote the height, width
and the number of channels of an input image, respectively.
$C'$ and $C''$ represent the number of channels of the feature maps that are produced as outputs by the intermediate layers of the encoder and decode network.
In our experiment, $C'$ and $C''$ were set to $128$ and $512$, respectively. $n$ denotes the number of dimensions of each latent representation, which was set to $256$.
The variable $d$ represents the number of blocks used for downsampling and upsampling. Therefore, we can calculate the height ($h$) and width ($w$) of the encoder's output feature maps by dividing the height ($H$) and width ($W$) of input images by $2$ raised to the power of $d$.

\begin{table}[thbp!]
\centering
\caption{High-level architecture of the encoder and decoder of the SC-VAE model. $H$, $W$, and $C$ refer to the height, width, and the number of channels of an input image. 
$C'$ and $C''$ represent the number of channels of the feature maps from intermediate layers in the encoder and decoder networks. $n$ denotes the number of dimensions of each latent representation, while $d$ represents the number of downsampling (upsampling) blocks. Note that $h=\frac{H}{2^{d}}$, $w=\frac{W}{2^d}$.} 
\resizebox{1\linewidth}{!}{%
\begin{tabular}{c|c}
  \toprule
   &  $x\in \mathbb{R}^{H\times W\times C} $\\
   &  2D Convolution $\rightarrow \mathbb{R}^{H\times W\times C'}$\\
   &  $d \times$\{Residual Block, Downsample Block\} $\rightarrow \mathbb{R}^{h\times w\times C''}$\\
   &  Residual Block $\rightarrow \mathbb{R}^{h\times w\times C''}$\\
  Encoder &  Non-Local Block $\rightarrow \mathbb{R}^{h\times w\times C''}$\\
   &  Residual Block $\rightarrow \mathbb{R}^{h\times w\times C''}$\\
   &  Group Normalization \cite{wu2018group} $\rightarrow \mathbb{R}^{h\times w\times C''}$ \\
   &  Swish Activation Function \cite{ramachandran2017searching} $\rightarrow \mathbb{R}^{h\times w\times C''}$\\
   &  2D Convolution $\rightarrow E(x) \in \mathbb{R}^{h\times w\times n}$\\
  \midrule
   & $\tilde{E}(x)\in \mathbb{R}^{h\times w\times n} $  \\
   &  2D Convolution $\rightarrow \mathbb{R}^{h\times w\times C''}$  \\
   &  Residual Block $\rightarrow \mathbb{R}^{h\times w\times C''}$\\
    & Non-Local Block $\rightarrow \mathbb{R}^{h\times w\times C''}$\\
  Decoder & Residual Block $\rightarrow \mathbb{R}^{h\times w\times C''}$\\
    & $d\times$\{Residual Block, Upsample Block\} $\rightarrow \mathbb{R}^{H\times W\times C'}$\\
   & Group Normalization \cite{wu2018group} $\rightarrow \mathbb{R}^{H\times W\times C'}$\\
    & Swish  Activation Function \cite{ramachandran2017searching}
    $\rightarrow \mathbb{R}^{H\times W\times C'}$\\
    & 2D Convolution $\rightarrow G(\tilde{E}(x)) \in \mathbb{R}^{H\times W\times C}$\\
  \bottomrule
\end{tabular}}
\label{figure:encoder_decoder}
\end{table}

\section{Visualization of Dictionary Atoms}
\label{section2_s}
Figure \ref{figure:dictionary_visualization} demonstrates the $512$ columns (atoms) of the pre-determined Discrete Cosine Transform (DCT) dictionary. Each atom is of dimension $256$, which corresponds to the size of $16 \times 16$ images when shaped.
%We reshape all atoms into an image with a $16\times 16$ resolution.

\begin{figure}[tbp]
\centering
\includegraphics[width=8cm]{./Figures/visualization_of_dictionary.png}
\caption{$512$ atoms of the Discrete Cosine Transform (DCT) dictionary. All atoms were reshaped into a $16 \times 16$ image.}
\label{figure:dictionary_visualization}
\end{figure}

\section{Training Losses}  \label{section3_s}
%Training losses of inherent noises around the 140th epoch under different auxiliary dataset sizes (K)
Figures \ref{figure:TLImagenet32x32}, \ref{figure:TLImagenet16x16}, \ref{figure:TLImagenet4x4} and \ref{figure:TLImagenet1x1} show  the training losses over $120,000$ training steps on the
ImageNet dataset.
The number of downsampling (upsampling) blocks ($d$) in the encoder (decoder) of the SC-VAE model are $3, 4, 6$ and $8$, respectively.
%with the number of downsampling (upsampling) blocks ($d=3,4,6$ and $8$, respectively) in the encoder (decoder) of the SC-VAE model. 
%As is shown in these figures, the LISTA networks of the SC-VAE models converge to a fixed point no matter which downsampling (upsampling) block $d$ is used. However, SC-VAE suffer from image reconstruction when increasing $d$.
As depicted in these figures, the LISTA networks within the SC-VAE models consistently converge to a stable point regardless of the chosen downsampling (upsampling) block $d$. However, increasing $d$ leads to worse image reconstructions ($\mathcal{L}_{rec}$) in SC-VAE.

\begin{figure}[tbp]
\centering
\includegraphics[width=7.5cm]{./Figures/Imagenet32x32.png}
\caption{The training losses over $120,000$ training steps on the ImageNet dataset. The number of  downsampling (upsampling) blocks ($d$) in the encoder (decoder) of the SC-VAE model was set to $3$ and the height ($h$) and width ($w$) of latent representations were $32$. (a) Total loss $\mathcal{L}_{SC-VAE}$. (b) Image reconstruction loss $\mathcal{L}_{rec}$. (c)The mean of latent representations reconstruction loss $\frac{1}{hw}\mathcal{L}_{latent}$.}
\label{figure:TLImagenet32x32}
\end{figure}

\begin{figure}[tbp]
\centering
\includegraphics[width=7.5cm]{./Figures/Imagenet16x16.png}
\caption{The training losses over $120,000$ training steps on the ImageNet dataset. The number of  downsampling (upsampling) blocks ($d$) in the encoder (decoder) of the SC-VAE model was set to $4$ and the height ($h$) and width ($w$) of latent representations were $16$. (a) Total loss $\mathcal{L}_{SC-VAE}$. (b) Image reconstruction loss $\mathcal{L}_{rec}$. (c) The mean of latent representations reconstruction loss $\frac{1}{hw}\mathcal{L}_{latent}$.}
\label{figure:TLImagenet16x16}
\end{figure}

\begin{figure}[tbp]
\centering
\includegraphics[width=7.5cm]{./Figures/Imagenet4x4.png}
\caption{The training losses over $120,000$ training steps on the ImageNet dataset. The number of  downsampling (upsampling) blocks ($d$) in the encoder (decoder) of the SC-VAE model was set to $6$ and the height ($h$) and width ($w$) of latent representations were $4$. (a) Total loss $\mathcal{L}_{SC-VAE}$. (b) Image reconstruction loss $\mathcal{L}_{rec}$. (c) The mean of latent representations reconstruction loss $\frac{1}{hw}\mathcal{L}_{latent}$.}
\label{figure:TLImagenet4x4}
\end{figure}

\begin{figure}[tbp]
\centering
\includegraphics[width=7.5cm]{./Figures/Imagenet1x1.png}
\caption{The training losses over $120,000$ training steps on the ImageNet dataset. The number of  downsampling (upsampling) blocks ($d$) in the encoder (decoder) of the SC-VAE model was set to $8$ and the height ($h$) and width ($w$) of latent representations were $1$. (a) Total loss $\mathcal{L}_{SC-VAE}$. (b) Image reconstruction loss $\mathcal{L}_{rec}$. (c) The mean of latent representations reconstruction loss $\frac{1}{hw}\mathcal{L}_{latent}$.}
\label{figure:TLImagenet1x1}
\end{figure}

%\noindent
%\noindent\textbf{Learnbale ISTA.} The architecture of our Learnable ISTA network is shown in Table 2.
%\noindent\textbf{Attention Network for $\alpha$ Estimation.}  Our neural network architecture follows the backbone of PixelCNN++ [52], which is a U-Net [48] based on a Wide ResNet [72]. We replaced weight normalization [49] with group normalization [66] to make the implementation simpler. Our 32 × 32 models use four feature map resolutions (32 × 32 to 4 × 4), and our 256 × 256 models use six. All models have two convolutional residual blocks per resolution level and self-attention blocks at the 16 × 16 resolution between the convolutional blocks [6].






% \begin{table*}[!htbp]
% \centering
% \caption{High-level architecture of the Learnable ISTA of our SC-VAE. Note that $k$ is the number of the unfolded ISTA block.} 
% \begin{tabular}{c}
%   \toprule
%   Learnable ISTA \\
%   \midrule
%   $E(x)\in \mathbb{R}^{h\times w \times n} $ \\
%   Filter Matrix $\rightarrow \mathbb{R}^{h\times w\times K}$ \\
%   $k\times$\{Shrinkage Function, Mutual Inhibition Matrix, Addition Operator\} $\rightarrow \mathbb{R}^{h\times w\times K}$\\
%   Shrinkage function$\rightarrow Z\in \mathbb{R}^{h\times w\times K}$\\
%   \bottomrule
% \end{tabular}
% \end{table*}

\section{Image Reconstruction}  \label{section4_s}
Reconstruction results from unofficial implementation\footnote{https://github.com/thuanz123/enhancing-transformers} of VIT-VQGAN \cite{yu2021vector} are presented in Figure \ref{figure:ViT-VQGAN_Visualization}.
%Figures \ref{figure:ViT-VQGAN_Visualization} shows visualizations from unofficial implementation\footnote{https://github.com/thuanz123/enhancing-transformers} of VIT-VQGAN \cite{yu2021vector}. 
VIT-VQGAN \cite{yu2021vector} achieved visually appealing results. However, similar to VQ-GAN \cite{esser2021taming} and RQ-VAE \cite{lee2022autoregressive}, it faced challenges in accurately reconstructing intricate details and complex patterns.
%as VQ-GAN\cite{esser2021taming} and RQ-VAE\cite{lee2022autoregressive}. 
Additionally, its generalization performance was inferior to that of our model.

\begin{figure}[tbp]
\centering
\includegraphics[width=7.0cm]{./Figures/ViT-VQGAN_Visualization.png}
\caption{Image reconstructions from an unofficial implementation of VIT-VQGAN \cite{yu2021vector} and the SC-VAE models trained
on ImageNet dataset. Original images in the top two rows are
from the validation set of ImageNet dataset. Two external images are shown in the last two rows to demonstrate the generalizability of different methods. The numbers denote the shape of
latent codes and the learned codebook (dictionary) size, respectively.
SC-VAE achieved improved image reconstruction compared to VIT-VQGAN \cite{yu2021vector}. Zoom in to see the details in the red square area.}
\label{figure:ViT-VQGAN_Visualization}
\end{figure}

\section{Image Generation}  \label{section5_s}
Additional interpolation and manipulation results can be found in Figures \ref{figure:image_interpolation_supple} and \ref{figure:image_manipulation_supple}, respectively.

\begin{figure}[tbp]
\centering
\includegraphics[width=7.0cm]{./Figures/image_interpolation_supple.png}
\caption{Interpolation between the sparse code vectors of two samples from the SC-VAE$^{\dag}$ model trained on FFHQ.}
\label{figure:image_interpolation_supple}
\end{figure}

\begin{figure*}[tbp]
\centering
\includegraphics[width=14.5cm]{./Figures/image_manipulation_supple4.png}
\caption{Manipulating sparse code vectors on FFHQ. 
Each block contains five seed images used to infer the latent sparse code vector in the SC-VAE$^{\dag}$ model.
The disentangled attributes associated with the $i$-th component of a sparse code vector $z$ and a traversal range are shown on the top of each block.}
\label{figure:image_manipulation_supple}
\end{figure*}



% \begin{figure}[tbp]
% \centering
% \includegraphics[width=8cm]{./Figures/IG_Age.png}
% \caption{IG-Age.}
% \label{figure:IG_Age}
% \end{figure}

% \begin{figure}[tbp]
% \centering
% \includegraphics[width=8cm]{./Figures/IG_sunglasses.png}
% \caption{IG-sunglasses.}
% \label{figure:IG_sunglasses}
% \end{figure}

% \begin{figure}[tbp]
% \centering
% \includegraphics[width=8cm]{./Figures/IG_Azimuth.png}
% \caption{IG-azimuth.}
% \label{figure:IG_azimuth}
% \end{figure}

% \begin{figure}[tbp]
% \centering
% \includegraphics[width=8cm]{./Figures/IG_Fringe.png}
% \caption{IG-Fringe.}
% \label{figure:IG_Fringe}
% \end{figure}


% \begin{figure}[tbp]
% \centering
% \includegraphics[width=8cm]{./Figures/IG_skin color.png}
% \caption{IG-skin color.}
% \label{figure:IG_skin color}
% \end{figure}


% \begin{figure}[tbp]
% \centering
% \includegraphics[width=8cm]{./Figures/image_interpolation_supple.png}
% \caption{Interpolation in the latent space between two samples from a model trained on FFHQ.}
% \label{figure:interpolation}
% \end{figure}

\section{Image Patches Clustering}  \label{section6_s}
%Figures \ref{figure:s1} and \ref{figure:s2} exhibit more image patches clustering outcomes for the FFHQ and ImageNet datasets, respectively. 
Figures \ref{figure:s1} and \ref{figure:s2} showcase additional qualitative results of image patches clustering on FFHQ and ImageNet datasets, respectively.
These results were obtained utilizing the pre-trained SC-VAE$^\curlyvee$ model specific to each dataset with a downsampling block $d=4$.
\begin{figure*}[h!]
\centering
\includegraphics[width=16cm]{./Figures/patches_cluster_ffhq_supple_50.png}
\caption{50 randomly selected image patch clusters from the validation set of the FFHQ dataset generated by clustering the learned sparse code vectors of the pre-trained SC-VAE$^\curlyvee$ model
using the K-means algorithm. Each row represents one cluster. Image patches with similar patterns were grouped together.}
\label{figure:s1}
\end{figure*}

\begin{figure*}[h!]
\centering
\includegraphics[width=16cm]{./Figures/imagenet_cluster_patches_V3.png}
\caption{50 randomly selected image patch clusters from the validation set of the ImageNet dataset generated by clustering the learned sparse code vectors of the pre-trained SC-VAE$^\curlyvee$ model
using the K-means algorithm. Each row represents one cluster. Image patches with similar patterns were grouped together.}
\label{figure:s2}
\end{figure*}

% \begin{figure*}[h!]
% \centering
% \includegraphics[width=16cm]{./Figures/segmentation_ffhq_supple3.png}
% \caption{FFHQ.}
% \label{figure:5}
% \end{figure*}

\section{Unsupervised Image Segmentation} \label{section7_s}
\subsection{Qualitative Analysis on FFHQ and ImageNet}
%Figures \ref{figure:s3} and \ref{figure:s4} contain additional qualitative unsupervised image segmentation results on FFHQ and ImageNet datasets, respectively. 
%We utilized two SCVAE models that were pre-trained on the training set of the FFHQ and ImageNet dataset, respectively. These models had a downsampling block of $d = 3$ and a sparsity penalty of $\lambda = 2$. 
%We employed two SC-VAE$^\curlywedge$ models that had been pre-trained on the training sets of the FFHQ and ImageNet datasets, respectively. These models had a downsampling block $d=3$.
Additional qualitative unsupervised image segmentation results on the FFHQ and ImageNet datasets can be found in Figures \ref{figure:s3} and \ref{figure:s4}, respectively. We utilized two SC-VAE$^\curlywedge$ models pre-trained on the training sets of FFHQ and ImageNet, each employing a downsampling block $d=3$.
\subsection{Quantitative comparisons to prior work}
%Figure \ref{figure:Flower_CUB} shows more qualitative results on  Flowers \cite{nilsback2008automated} and Caltech-UCSD Birds-200-2011 (CUB) \cite{WahCUB_200_2011}. Flowers \cite{nilsback2008automated} consists of $8,189$ images of $102$ classes of flowers, with segmentation masks obtained by an automated algorithm developed specifically for segmenting flowers in color photographs \cite{nilsback2007delving}. CUB \cite{WahCUB_200_2011} consists of $11,788$ images of $200$ classes of birds and segmentation masks. Flowers and CUB contain $1,020$ and $1,000$ test images, respectively.
%Figure \ref{figure:Flower_CUB} shows more qualitative results on  Flowers \cite{nilsback2008automated} and Caltech-UCSD Birds-200-2011 (CUB) \cite{WahCUB_200_2011} datasets.
Figure \ref{figure:Flower_CUB} displays additional qualitative results from the Flowers \cite{nilsback2008automated} and Caltech-UCSD Birds-200-2011 (CUB) \cite{WahCUB_200_2011} datasets.\\
\subsubsection{Evaluation Metrics}
\textbf{Intersection of Union (IoU).} %The IoU score measures the overlap of two regions A and B by calculating the ratio of intersection over union, according to
The IoU score quantifies the overlap between two regions. This is achieved by evaluating the ratio of their intersection to their union.
\begin{align}
    \textup{IoU}(A, B) = \frac{|A\cap B|}{|A\cup B|}. \nonumber
\end{align}
%where we use the inferred mask and ground-truth mask as $A$ and $B$ respectively for evaluation.\\
$A$ denotes the ground-truth mask, while $B$ denotes the inferred mask.\\
%as $B$ for assessment purposes.\\
\textbf{DICE score.} Similarly, the DICE score is defined as:
\begin{align}
    \textup{Dice}(A, B) = \frac{2|A\cap B|}{|A|+ |B|}.\nonumber
\end{align}
\noindent
Higher is better for both scores.\\
\subsubsection{Dataset Details}
\textbf{Flowers.} The Flowers \cite{nilsback2008automated} dataset consists of $8,189$ images across $102$ different flower classes. Additionally, it includes segmentation masks generated by an automated algorithm designed explicitly for color photograph flower segmentation \cite{nilsback2007delving}. 
%The images in this dataset have large scale, pose and light variations.\\
The dataset contains images that exhibit substantial variations in scale, pose, and lighting.
Flowers \cite{nilsback2008automated} contains $1,020$ test images.\\
\textbf{CUB.} The CUB \cite{WahCUB_200_2011} dataset contains $11,788$ images covering $200$ bird classes, along with their segmentation masks. 
%Each image is further annotated with $15$ part locations and $1$ bounding box. We use theprovided bounding box to extract a center square from the image, and scale it to $128\times 128$ pixels.
Every image comes with annotations for $15$ part locations, $312$ binary attributes, and $1$ bounding box. We utilized the given bounding box to crop a central square from the image. The CUB dataset includes $1,000$ test images.\\
\textbf{ISIC-2016.} The ISIC-2016 \cite{gutman2016skin} dataset is a public challenge dataset dedicated to Skin Lesion Analysis for Melanoma Detection. Derived from the extensive International Skin Imaging Collaboration (ISIC) archive, it represents a significant collection of meticulously curated dermoscopic images of skin lesions. Within this challenge, a subset of $900$ images is designated as training data, while $379$ images serve as testing data, aiming to provide representative samples for analysis.
%The ISIC-2016 \cite{gutman2016skin} dataset is a public challenge dataset of Skin Lesion Analysis Towards Melanoma Detection released with ISBI 2016. This dataset is based on the International Skin Imaging Collaboration (ISIC) Archive, which is the largest publicly available collection of quality controlled dermoscopic images of skin lesions. The challenge employs a subset of representative images with $900$ images as training data and $379$ images as testing data.

%For all experiments, we resized the input images into a resolution of $256\times 256$ and  generated a $32\times 32$ binary mask for each image utilizing the pre-trained SC-VAE$^\curlywedge$ on ImageNet dataset, a spectral clustering algorithm and boundary connectivity information. The inferred binary mask and ground truth mask were resized to $128\times 128$ to calculate the IoU and DICE scores.
For our experiments, we resized the input images into a resolution of $256\times 256$.
Subsequently, we generated a binary mask of size $32\times 32$ per image by employing the pre-trained SC-VAE$^\curlywedge$ on the ImageNet dataset, along with a spectral clustering algorithm and boundary connectivity information \cite{zhu2014saliency}. To compute the IoU and DICE scores, both the inferred binary mask and the ground truth mask were resized to $128\times 128$.
%\subsubsection{Baseline Methods}
\label{section3}
\begin{figure*}[h!]
\centering
\includegraphics[width=16cm]{./Figures/segmen_ffhq_supple3.png}
%\caption{Additional unsupervised image segmentation results. Images are from the validation set of the FFHQ dataset.}
\caption{Additional unsupervised image segmentation results. These results were generated by grouping sparse code vectors into $5$ categories per image, utilizing the pre-trained SC-VAE$^{\curlywedge}$ model and the K-means algorithm. Images are from the validation set of the FFHQ dataset.}
\label{figure:s3}
\end{figure*}

\begin{figure*}[h!]
\centering
\includegraphics[width=16cm]{./Figures/segmentation_imagenet_supple.png}
%\caption{Additional unsupervised image segmentation results by applying K-means algorithm to cluster sparse code vectors per image into $5$ categories using the SC-VAE$^{\curlywedge}$ model. Images are from the validation set of the ImageNet dataset.}
\caption{Additional unsupervised image segmentation results. These results were generated by grouping sparse code vectors into $5$ categories per image, utilizing the pre-trained SC-VAE$^{\curlywedge}$ model and the K-means algorithm. Images are from the validation set of the ImageNet dataset.}
\label{figure:s4}
\end{figure*}

\begin{figure*}[tbp]
\centering
\includegraphics[width=18cm]{./Figures/flower_cub_isic2016_supple2.png}
\caption{Additional unsupervised image segmentation results on Flowers \cite{nilsback2008automated} (\textit{Left Panel}), CUB \cite{WahCUB_200_2011} (\textit{Middle Panel}) and ISIC-2016 \cite{gutman2016skin} (\textit{Right Panel}). (a) input image. (b) ground truth mask. (c) and (e) segmentation results by clustering sparse code vectors per image into $2$ or $3$ classes using a spectral clustering algorithm. (d) and (f) boundary connectivity information \cite{zhu2014saliency}
was used to decide the foreground and background.}
\label{figure:Flower_CUB}
\end{figure*}

\clearpage
\clearpage
{
   \small
   \bibliographystyle{ieee_fullname}
   \bibliography{egpaper_arxiv_V2}
}
\clearpage
{
    \small
    \bibliographystyle{ieeenat_fullname}
    \bibliography{main}
}

% WARNING: do not forget to delete the supplementary pages from your submission 


\end{document}
