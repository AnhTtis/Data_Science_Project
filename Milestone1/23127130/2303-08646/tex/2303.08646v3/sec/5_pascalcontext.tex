\section{Experiments on Pascal Context Dataset}
%\raggedbottom
The Pascal Context~\cite{cPascalContext} dataset contains 4,998 training images and 5,105 testing images.
%
Following the common practice, we use its 59 semantic classes to conduct the ablation studies and experiments.
%
Unless specified, we train the models on the training set for 30K iterations for the ResNet backbone and 40K for Swin-Large and ConvNeXt-Large.

%%%%%%%% Problem identification %%%%%%%%%%%%%%%%%%%%%%%%%%%%%%


\begin{table}[t]
\centering
\small
\resizebox{\linewidth}{!}{%
\begin{tabular}{c|l|c|c} 
\toprule
Backbone & Dilated or Upsampler & Extra Encoding & mIOU(\%)  \\
\midrule
\midrule
ResNet-50 & -  &  Identity      &  45.87 \\
ResNet-50 & Dilation, $OS=8$ &  Identity      & \textbf{47.72}  \\
ResNet-50 & SFPN, $OS=4$ &  Identity      & 47.14  \\
\midrule
ResNet-50 & -  & ASPP~\cite{cDeepLabV3Plus}           & 46.41  \\
ResNet-50 & Dilation, $OS=8$  & ASPP~\cite{cDeepLabV3Plus}           & \textbf{48.59}  \\
ResNet-50 & SFPN, $OS=4$ & ASPP~\cite{cDeepLabV3Plus}           &  47.81 \\
\midrule
ResNet-50 & -  & OCR~\cite{cOCR}            & 45.50 \\
ResNet-50 & Dilation, $OS=8$ & OCR~\cite{cOCR}            & \textbf{48.23} \\
ResNet-50 & SFPN, $OS=4$ & OCR~\cite{cOCR}            & 47.39 \\
\midrule
ResNet-50 & - & SA~\cite{cNonLocal} & 45.02 \\
ResNet-50 & Dilation, $OS=8$ & SA~\cite{cNonLocal} & \textbf{48.32} \\
ResNet-50 & SFPN, $OS=4$ & SA~\cite{cNonLocal} & 45.90 \\
\midrule
ResNet-50 & - & SA~(CAR~\cite{cCAR}) &  47.18  \\
ResNet-50 & Dilation, $OS=8$ & SA~(CAR~\cite{cCAR}) & \textbf{50.50} \\
ResNet-50 & SFPN, $OS=4$ & SA~(CAR~\cite{cCAR}) & 48.51  \\
\bottomrule
\end{tabular}
}
\caption{
Comparisons between a dilation method and a state-of-the-art upsampler-based method (i.e. SemanticFPN) with various extra encoding techniques on the Pascal Context dataset.
Results demonstrate that dilation is more accurate than SemanticFPN with a substantial margin.
Identity means using no extra encoding (i.e. a basic FCN).
% \hy{Verification of SemanticFPN is generally worse than dilation on the Pascal Context dataset.}
% \hy{Identity means using the backbone result directly, similar to FCN, allows for the plainest way of verifying our identified problem.}
}
\label{tab:ablation_dilation_vs_fpn}
\end{table}

\subsection{Problem verification}
\label{sec:problem}

\noindent\textbf{SemanticFPN is generally worse than dilation.}
We conduct experiments in Tab.~\ref{tab:ablation_dilation_vs_fpn} to demonstrate the issue we identify (observation 1 in Fig.~\ref{fig:hfgd:concept}).
% 
Many famous methods for context encoding (``Extra Encoding'') produce less accurate predictions when combined with an upsampler net (i.e. SemanticFPN) than directly using a dilated backbone, even if the upsampler produces higher-resolution feature maps.
%
Refer to our appendix, for detailed settings.

\noindent\textbf{Negative influence from the upsampler on the backbone.}
We find the joint fine-tuning of the upsampler and the backbone results in deteriorated backbone features.
%
To demonstrate it, we modify SemanticFPN by introducing an auxiliary FCN to predict the mask from the high-level features produced by the backbone (see Issues in Fig.~\ref{fig:hfgd:concept}).
The predictions from the auxiliary FCN branch become worse than the original FCN (44.35\% vs 45.87\% mIOU).
Another variant stops the auxiliary FCN's gradient from propagating back to the backbone, which produces even worse results (40.04\%).

\noindent\textbf{Advantage of pre-training.}
We conduct experiments to show the benefit of using the pre-trained backbone in Tab.~\ref{tab:ablation_imagenet_simple}.
Due to limited training data, there still exists a substantial gap after 6 times training iterations.
Thus, protecting the backbone to ensure its generalization is necessary and critical.
More experiments about this are presented in the supplementary.

%%%%%%%%%%%%%%%%%%%%%%%%%%%%%%%%%%%%%%%%%%%%%%%%%%%%%%


\subsection{Ablation studies on HFGD}

\noindent\textbf{Ablation studies on HFGM.}
%
In Tab.~\ref{tab:ablation_hfgm}, we evaluate the effectiveness of our proposed HFGM based on SemanticFPN (\ie no ``Extra Encoding'').
%
Although using only HFG (``+guidance'') or only axial attention (``+AA'') is helpful, using the full HFGM brings the most gain (1.80\% mIOU).
probably because AA can effectively broadcast the guidance information of HFG to all spatial locations (also see Sec.~\ref{sec:method:hfgm} for more discussion).




\noindent\textbf{Importance of stopping gradients.}
We use several stop-grad operations to protect the backbone weights, especially the early low-level weights, and only allow gradients from the backbone branch to update their weights gradually (see. Fig.~\ref{fig:URD:Arch}).
We tested removing all stop-grad operations and obtained significantly decreased accuracy (48.94\% vs 48.50\%).
%
% We also analyze what happens if we no longer protect the backbone branch (i.e. letting gradients propagate from the upsampler branch).
% We found that doing so resulted in a drop in mIOU down to 48.50\%.


\noindent\textbf{Ablation studies on CAE.}
%
% We conduct ablation studies on CAE in Tab.~\ref{tab:ablation_cae} and Tab.~\ref{tab:ablation_hfgm_cae}.
% 
In Tab.~\ref{tab:ablation_cae}, we compare with CAR under different settings on the Pascal Context dataset 
since CAR~\cite{cCAR} performs best in Tab.~\ref{tab:ablation_dilation_vs_fpn} and CAE is based on CAR~\cite{cCAR}.
Results show that CAE is more compatible than CAR when using SFPN and ``SFPN + HFGM''.
%
Combined with results in Tab.~\ref{tab:ablation_dilation_vs_fpn}, our CAE design outperforms all the other alternatives (``Extra Encodings'') with and without SemanticFPN and approaches the accuracy of the best dilation-based model in Tab.~\ref{tab:ablation_dilation_vs_fpn}.


Similar to experiments in Tab.~\ref{tab:ablation_hfgm},
we also analyze the effects of CAE with different HFGM settings.
%
HFGM now provides better guidance to the upsampler with the help of CAE, leading to further improved final results (50.28\% vs 48.94\% and 49.22\% vs 47.67\% in Tab.~\ref{tab:ablation_hfgm}).



%Our CAE design surpasses the other alternatives (``Extra Encodings'') by a substantial margin, with and without SemanticFPN~\cite{cPanopticFPN}.
%Note that CAE is the best for methods using an upsampler while SA (CAR) is the best for dilation methods, showing that previous context encodings are unsuitable for OS=32 feature maps, and our modification on SA (CAR) effectively improves it.


\begin{table}[t]
\centering
\small
\resizebox{\linewidth}{!}
{\def\arraystretch{1} \tabcolsep=0.6em 
\begin{tabular}{l|c|c|c} 
\toprule
Training Iterations & 30K & 90K & 180K  \\
\midrule
\midrule
ResNet-50 (ImageNet) + FCN & 45.87 & - & - \\
ResNet-50 (scratch) + FCN  & 26.13 & 31.38 & 34.00 \\
\bottomrule
\end{tabular}
}
\caption{
% Simple experiments to present the importance and advantage of ImageNet pre-train.
Importance of using ImageNet pre-trained weights.
Experiments are conducted on the Pascal Context dataset (mIOU\%).
%
% Trained on Pascal Context dataset.
% %
% Results are in mIOU(\%).
}
\label{tab:ablation_imagenet_simple}
\end{table}

\begin{table}[t]
\centering
\small
\resizebox{\linewidth}{!}
{
\begin{tabular}{c|c|c|c|l} 
\toprule
Backbone   & Upsampler  & Extra Encoding & HFGM & mIOU(\%)  \\
\midrule
\midrule
ResNet-50  & SFPN & Identity & -      & 47.14  \\
ResNet-50  & SFPN & Identity & + guidance & 47.67 (\textcolor{black}
{$+$0.53}) \\
ResNet-50  & SFPN & Identity & + AA  & 47.88 (\textcolor{black}
{$+$0.74})  \\
ResNet-50  & SFPN & Identity & full  & 48.94 (\textcolor{black}{$+$1.80)}\\
\midrule
ResNet-50  & SFPN & Our CAE & -      & 48.76    \\
ResNet-50  & SFPN & Our CAE & + guidance & 49.22 (\textcolor{black}{$+$0.46}) \\
ResNet-50  & SFPN & Our CAE & + AA & 49.06 (\textcolor{black}{$+$0.26}) \\
ResNet-50  & SFPN & Our CAE & full  & \textbf{50.28} (\textcolor{black}{$+$1.52}) \\
\bottomrule
\end{tabular}
}
\caption{
Ablation studies on different HFGM settings on Pascal Context dataset.
Using both guidance and AA in HFGM brings most gain (1.8\%/1.52\% mIOU).
AA and the proposed high-level feature guidance co-operate extremely well
% are beneficial for each other 
probably because AA can effectively back-propagate the guidance signal to all spatial locations.
% HFGM effectively improves the accuracy, especially when it contains an axial attention (AA) layer since AA can effectively back-propagate the guidance signal from the good high-level features (i.e. output of CAE) to all spatial locations, resulting in more accurate upsampled feature representations.
}
\label{tab:ablation_hfgm}
\end{table}

\begin{table}[t]
\centering
\small
\resizebox{\linewidth}{!}{%
\begin{tabular}{c|l|c|c} 
\toprule
Backbone & Dilated or Upsampler & Extra Encoding & mIOU(\%)  \\
\midrule
\midrule
ResNet-50 & -  &  Identity      &  45.87 \\
ResNet-50 & Dilation, $OS=8$ &  Identity      & \textbf{47.72}  \\
ResNet-50 & SFPN, $OS=4$ &  Identity      & 47.14  \\
\midrule
ResNet-50 & -  & ASPP~\shortcite{cDeepLabV3Plus}           & 46.41  \\
ResNet-50 & Dilation, $OS=8$  & ASPP~\shortcite{cDeepLabV3Plus}           & \textbf{48.59}  \\
ResNet-50 & SFPN, $OS=4$ & ASPP~\shortcite{cDeepLabV3Plus}           &  47.81 \\
\midrule
ResNet-50 & -  & OCR~\shortcite{cOCR}            & 45.50 \\
ResNet-50 & Dilation, $OS=8$ & OCR~\shortcite{cOCR}            & \textbf{48.23} \\
ResNet-50 & SFPN, $OS=4$ & OCR~\shortcite{cOCR}            & 47.39 \\
\midrule
ResNet-50 & - & SA~\shortcite{cNonLocal} & 45.02 \\
ResNet-50 & Dilation, $OS=8$ & SA~\shortcite{cNonLocal} & \textbf{48.32} \\
ResNet-50 & SFPN, $OS=4$ & SA~\shortcite{cNonLocal} & 45.90 \\
\midrule
ResNet-50 & - & SA~(CAR~\shortcite{cCAR}) &  47.18  \\
ResNet-50 & Dilation, $OS=8$ & SA~(CAR~\shortcite{cCAR}) & \textbf{50.50} \\
ResNet-50 & SFPN, $OS=4$ & SA~(CAR~\shortcite{cCAR}) & 48.51  \\
\midrule
ResNet-50 & - & Our CAE  & 47.35 \\
ResNet-50 & Dilation, $OS=8$ & Our CAE  & \textbf{49.67} \\
ResNet-50 & SFPN, $OS=4$ & Our CAE  & 48.76 \\
\bottomrule
\end{tabular}
}
\caption{
Experiments to compare the effectiveness of using dilated backbones and upsampler networks on Pascal Context dataset.
Using a dilated backbone consistently outperform using an upsampler net on several representative context encoding methods.
% of different encoding modules for high-level feature enhancements on Pascal Context dataset.
% 
The final feature maps are bilinearly upsampled to the image resolution immediately followed by the class label prediction.
% We train all the methods under exactly same and fair settings.
%
% Experiments without upsampling head produces $OS=32$ feature maps while those with Semantic FPN (SFPN)~\cite{cPanopticFPN} produces $OS=4$ feature maps.
}
\label{tab:ablation_cae}
\end{table}
\begin{table}[t]
\centering
\small
\resizebox{\linewidth}{!}
{
\begin{tabular}{c|c|c|c|l} 
\toprule
Backbone   & Upsampler  & Extra Encoding & HFGM & mIOU(\%)  \\
\midrule
\midrule
ResNet-50  & SFPN & Our CAE & -      & 48.76    \\
ResNet-50  & SFPN & Our CAE & + guidance & 49.22 (\textcolor{black}{$+$0.46}) \\
ResNet-50  & SFPN & Our CAE & + AA & 49.06 (\textcolor{black}{$+$0.26}) \\
ResNet-50  & SFPN & Our CAE & full  & \textbf{50.28} (\textcolor{black}{$+$1.52}) \\
\bottomrule
\end{tabular}
}
\caption{
Ablation studies to analyze the effects of CAE with different HFGM settings on the Pascal Context dataset.
%
HFGM provides better guidance thanks to the improved high-level features extracted by CAE.
}
\label{tab:ablation_hfgm_cae}
\end{table}

\noindent\textbf{Ablation studies on U-SFPN.}
%
In Tab.~\ref{tab:ablation_upsampling}, we conduct ablation studies on U-SFPN to verify the effectiveness of our modification on SFPN while fixing CAE and HFGM.
Replacing SFPN with U-SFPN improves the mIOU of ResNet-50 (CAE + HFGM) by 0.48\%, reaching 50.76\%.
The improvement is even larger (1.14\%) when using Swin-Large as the backbone 
%(Tab.~\ref{tab:ablation_upsampling})
.

\begin{table}[t]
\centering
\small
\resizebox{\linewidth}{!}
{\def\arraystretch{1} \tabcolsep=0.6em 
\begin{tabular}{c|c|c|c|l} 
\toprule
Backbone & CAE & HFGM & Upsampler & mIOU(\%)  \\
\midrule
\midrule
ResNet-50  & \checkmark & \checkmark & SFPN & 50.28 \\
ResNet-50  & \checkmark & \checkmark & U-SFPN   & \textbf{50.76} (\textcolor{black}{$+$0.48}) \\
\midrule
Swin-Large & \checkmark & \checkmark & SFPN & 59.32 \\
Swin-Large & \checkmark & \checkmark & U-SFPN   & \textbf{60.46} (\textcolor{black}{$+$1.14}) \\
\bottomrule
\end{tabular}
}
\caption{Ablation studies on different upsampling heads on the Pascal Context dataset.
Our proposed U-SFPN upsampling head consistently outperforms semantic FPN (SFPN) for both CNN and Transformer backbones.
% U-SFPN on CAE + CFGM on the Pascal Context dataset. 
% \dknote{Low priority for $OS=2$ results}
}
\label{tab:ablation_upsampling}
\end{table}




\noindent\textbf{Module-level ablation studies on HFGD.}
%
In Tab.~\ref{tab:ablation_hfgd},
% Finally, 
%we conduct ablation studies on each module of HFGD using previously found best configurations in Tab.~\ref{tab:ablation_cae}-\ref{tab:ablation_upsampling}.
%Using all modules together leads to significant accuracy improvement, indicating the effectiveness of the overall architecture.
%
We performed ablation studies on each module of HFGD using the best configurations found in Tab.~\ref{tab:ablation_cae}-\ref{tab:ablation_upsampling}. Using all modules together significantly improved accuracy, indicating the overall architecture's effectiveness.


\begin{table}[t]
\centering
\small
\resizebox{\linewidth}{!}
{\def\arraystretch{1} \tabcolsep=0.8em 
\begin{tabular}{c|c|l|c|c} 
\toprule
Backbone & Extra Encoding & Upsampler & HFGM & mIOU(\%)  \\
\midrule
\midrule
ResNet-50 & Identity & SFPN & - & 47.14 \\
ResNet-50 & Our CAE & SFPN    & -     & 48.76 \\
ResNet-50 & Identity  & U-SFPN & -     & 46.99\\
ResNet-50 & Our CAE& U-SFPN & - & 48.67 \\
ResNet-50 & Our CAE & U-SFPN & \checkmark & \textbf{50.76} \\
\midrule
Swin-Large & Our CAE& SFPN     &  -    & 56.78 \\
Swin-Large & Identity & U-SFPN &  - & 58.69 \\
Swin-Large & Our CAE & U-SFPN &  - & 55.76 \\
Swin-Large & Our CAE & U-SFPN & \checkmark & \textbf{60.46} \\
\bottomrule
\end{tabular}
}
\caption{Ablation studies on the proposed three modules using previously found best configurations in Tab.~\ref{tab:ablation_cae}-\ref{tab:ablation_upsampling} on Pascal Context dataset.
}
\label{tab:ablation_hfgd}
\end{table}

\begin{table}[h]
\centering
\small
\resizebox{\linewidth}{!}
{\def\arraystretch{1} \tabcolsep=0.6em 
\begin{tabular}{l|c|c|c|l} 
\toprule
Backbone   & Upsampler  & EE & HFGM & mIOU(\%)  \\
\midrule
\midrule
R50  & U-SFPN & Our CAE & -     & 48.67   \\
R50  & U-SFPN & Our CAE & \checkmark  & \textbf{50.76} (\textcolor{black}{$+$2.09}) \\
\midrule
R50  & FaPN & - & -      &  47.50 \\
R50  & FaPN & - & \checkmark  &  \textbf{49.87} (\textcolor{black}{$+$2.37}) \\
\midrule
R50  & Uper & PPM & -      & 48.25 \\
R50  & Uper & PPM & \checkmark  & \textbf{49.96}   (\textcolor{black}{$+$1.71}) \\
\midrule
R50 (D8) & DeepLabV3+ & ASPP & -      &  48.11\\
R50 (D8) & DeepLabV3+ & ASPP & \checkmark  & \textbf{49.70} (\textcolor{black}{$+$1.59}) \\
\bottomrule
\end{tabular}
}
\caption{
Ablation studies on different upsamplers w/o or w/ our HFGM on Pascal Context dataset.
\textit{EE}: Extra encoding.
\textit{D8}: 
% The backbone uses 
Dilated convolutions with ($OS=4$).
% output stride = 8.
}
\label{tab:ablation_hfg_other_upsamplers}
\end{table}

\subsection{Computational cost of HFGD}
%
The computational cost of our HFGD and two other state-of-the-art methods are listed in Tab.~\ref{tab:HFGD:flops}.
% 
HFGD uses much lower GFLOPs than a similar dilation model (Self-Attention + CAR~\cite{cCAR}) but achieves better mIOU (50.76\% vs 50.50\%~\cite{cCAR}).
% 
Compared with SemanticFPN, HFGD ($OS=4$) achieves 3.62\% % mIOU gain with an affordable extra computation cost (71.63 GFLOPs vs 45.65 GFLOPs).

\begin{table}[t]
\centering
\small
\resizebox{\linewidth}{!}{%
\begin{tabular}{l|l|l|l}
\toprule%[1pt]
Method & Backbone & GFLOPs & mIOU\%\\
\midrule
\midrule
SA (CAR) & ResNet-50 (D8) & 158.96 & 50.50\\
\midrule
SemanticFPN & ResNet-50  & 45.65  & 47.14  \\
\midrule
HFGD (OS=4) &  ResNet-50 & 71.63  &  50.76 \\
HFGD (OS=2) &  ResNet-50 & 153.62 & 51.00  \\
\bottomrule%[1pt]
\end{tabular}
}
\caption{
Computational analysis of HFGD on a $513\times513\times3$ input image.
Previously, although more efficient,
upsampling-based methods (e.g. SemanticFPN) cannot produce as accurate results as the dilation-based methods.
HFGD ($OS=4$) closes this accuracy gap while still being efficient.
If a similar computation budget is given, HFGD ($OS=2$) can further improve the accuracy.
}
\label{tab:HFGD:flops}
\end{table}

\subsection{Comparison with the state-of-the-art methods}

To compare with the state-of-the-art, we adopt ConvNeXt-L as the backbone for our HFGD.
% 
We set the training iterations to 40K while
all the other training settings are the same as stated in Sec.~\ref{sec:HFGD:training_settings}.
% except that we set the training iterations to 40K.
% 
As shown in Tab.~\ref{tab:urd:SOTA-PascalContext}, our HFGD achieved 63.8\% mIOU with single-scale without flipping and 64.9\% mIOU with multi-scales with flipping, outperforming previous state-of-the-art by 1\% mIOU in ECCV-2022.
 %
HFGD is now the new state-of-the-art method on Pascal Context for the methods that only use the ImageNet pre-trained backbone without extra techniques~\cite{cAugReg,cFocalLoss,cVNet}. 


\begin{figure*}[th]
\centering
\includegraphics[width=1.0\linewidth]{images/vis_main.pdf}
\caption{
Visual comparisons 
% (Upper-left:Pascal Context, upper-right:COCOStuff, bottom:Cityscapes) 
between SemanticFPN ($OS=4$), HFGD ($OS=4$), and HFGD ($OS=2$).
Zoom in to see better.
The results are obtained using single-scale without flipping.
%
More visualizations are presented in the supplementary.
}
\label{fig:CFGD:main-R50-Vis}
\end{figure*}

\begin{figure}[th]
\centering
\includegraphics[width=1.0\linewidth]{images/umap_vis.pdf}
\caption{
% Comparing the baselines (FCN and Semantic) with our proposed solution (HFGM or entire HFGD) using UMAP visualization on the Pascal Context dataset.
UMAP~\cite{cUMAP} Visualization on Pascal Context. 
HFGD features are most separable from the inter-class perspective and most compact from the intra-class perspective, resulting in the best accuracy (Tab.~\ref{tab:ablation_hfgd}).
}
\label{fig:umap_vis}
\end{figure}

\subsection{Apply high-level guide to existing upsamplers }

Though we recommend using U-SFPN + CAE to fit our proposed HFGM for efficiency, many existing methods attempt to improve the accuracy of upsamplers by improving intermediate-specific operations. 
For example, FaPN~\cite{cFaPN} uses SENet~\cite{cSENet} and deformable convolutions to try to align low-level features with high-level features.
%
UperNet~\cite{cUper} uses PPM~\cite{cPSPNet} to improve the high-level features of the backbone network.
%

From another perspective, HFGM directly optimizes the final upsampling quality rather than the intermediate process.
Thus, it should be able to improve the accuracy
of these upsamplers further, using high-level features as teachers to
constrain their upsampling results.

Using the ablation experimental setup in the main paper, Tab~\ref{tab:ablation_hfg_other_upsamplers} verifies our HFGM can effectively boost mIOU for different upsamplers. 
%
We believe that HFGM has generalizability to other similar upsamplers.
%