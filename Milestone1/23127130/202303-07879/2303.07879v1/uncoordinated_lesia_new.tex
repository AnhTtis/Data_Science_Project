\vspace{-0.1in}
\section{Theoretical Game Analysis} \label{sec:uncoordinated_extra_demand}

Here, we study the conditions on the parameter values for the existence of dominant strategies and mixed-strategy NE under PA, and provide closed form formulations of these equilibrium states, i.e., ranges on the values of the vector of mixed strategies $\mathbf{p^{NE}}$ at NE. The proofs of the theoretical results presented below are available in {Appendix \ref{sec:proofsESSG}}.

First, we recall that, for a mixed-strategy NE to exist, the expected costs of each consumer for all pure strategies in the support of the mixed-strategy NE must be equal. Using the expressions of the costs in \eqref{eq:RES_cost} and \eqref{eq:nonRES_cost2}, we obtain that the amount of RESs allocated to the consumer type $\ell \in \Theta$ at a NE should satisfy:

%First, we recall that any mixed strategy NE $\mathbf{p^{NE}}$ must fulfill

%\small
%\begin{equation}\label{eq:cost_equality_mixed}
%\upsilon_{RES, \ell}(\mathbf{p^{NE}})= \upsilon_{nonRES, \ell}(\mathbf{p^{NE}}), ~\forall \ell \in \Theta.
%\end{equation}
%\normalsize

%\noindent Namely, the expected costs of each pure strategy in the support of the mixed-strategy equilibrium are equal. By substituting the expressions of \eqref{eq:RES_cost} and \eqref{eq:nonRES_cost2} in \eqref{eq:cost_equality_mixed}, we obtain that the amount of RESs allocated to $\ell \in \Theta$ at a NE should satisfy:

 \vspace{-0.1in} 
\small
\begin{equation}\label{eq:conditionEQ_extra_demand}
rse^{NE}_{\ell}(\mathbf{p^{NE}}) = \frac{\gamma-\varepsilon_{\ell}\beta}{\gamma-1}E_{\ell},~ \forall \ell \in \Theta.
\end{equation}
\normalsize 
\vspace{-0.1in}  

\noindent Thus, in the ESG with the PA policy, a mixed-strategy NE exists under the condition:


 \vspace{-0.1in} 
 \small
\begin{equation}\label{eq:condition_PA_NE}
rse_{\ell}^{PA}(\mathbf{p}^{NE}) =rse_{\ell}^{NE}(\mathbf{p}^{NE}), \forall \ell \in \Theta. \end{equation}
\normalsize 
\vspace{-0.15in}  

\noindent 
Therefore, for all cases, any existing mixed-strategy NE competing probabilities, $\mathbf{p}^{NE}$, are obtained so as to satisfy condition \eqref{eq:condition_PA_NE}. In the following, we distinguish cases with respect to the RES capacity, as well as the consumer preferences, i.e. the risk aversion degree and the daytime energy demand. 
\vspace{-0.15in}
\subsection{\textbf{Case $1$: RES capacity $\mathcal{ER}$ exceeds the maximum total demand for RES $D^{Total}$ ($\mathcal{ER} \geq D^{Total}$).}}

As the consumers have knowledge of $\mathcal{ER}$ and $D^{Total}$, it is straightforward to show that the dominant-strategy for all consumers is to select the strategy $RES$. As a result, the competing probabilities that lead to equilibrium states are equal to $1$ for all consumer types, and the aggregate daytime energy demand for RES is equal to $D^{Total}$.
\vspace{-0.1in}
\subsection{\textbf{Case $2$: RES capacity $\mathcal{ER}$ is less than the maximum total demand for RES $D^{Total}$ ($\mathcal{ER} \leq D^{Total}$).}} 

In this case, the strategies of the consumers depend on their respective risk aversion degrees and the energy prices. We define two distinct subsets of consumer types: $\Sigma_1 = \Bigl\{\ell \in \Theta : \varepsilon_{\ell} \geq \gamma/\beta \Bigr\} \subset \Theta$, and $\Sigma_2 = \Bigl\{ \ell \in \Theta :1\leq \varepsilon_{\ell} < \gamma/\beta \Bigr\} \subset \Theta$.

Firstly, the dominant strategy for all consumers whose type is in the set $\Sigma_1$ is to play $RES$. Therefore, the competing probabilities that lead to equilibrium states are equal to $1$, and their aggregate daytime demand is $D^{Total}_{\Sigma_1}=N\sum_{\ell \in \Sigma_1}r_\ell~ E_\ell$.

Secondly, the strategies of the consumers whose type is in the set $\Sigma_2$ depends on the daytime energy demand levels. Therefore, we define two distinct subsets of consumer types in $\Sigma_2$: $\Sigma_{2,1} = \left\{ \ell \in \Sigma_2 : E_{\ell} > \mathcal{ER}\frac{(\gamma-1)}{(\gamma-\varepsilon_{\ell}\beta)} \right\}$ and $\Sigma_{2,2} = \left\{ \ell \in \Sigma_2 : E_{\ell} \leq \mathcal{ER}\frac{(\gamma-1)}{(\gamma-\varepsilon_{\ell}\beta)}\right\}$.

%$\Sigma_{2,1} = \left\{ \ell \in \Sigma_2 : E_{\ell} > \left(\mathcal{ER}-D^{Total}_{\Sigma_1}\right)\frac{(\gamma-1)}{(\gamma-\varepsilon_{\ell}\beta)} \right\}$ and $\Sigma_{2,2} = \left\{ \ell \in \Sigma_2 : E_{\ell} \leq \left(\mathcal{ER}-D^{Total}_{\Sigma_1}\right)\frac{(\gamma-1)}{(\gamma-\varepsilon_{\ell}\beta)}\right\}$

For consumers whose types are in the set $\Sigma_{2,1}$, the dominant strategy is to play $grid$, and both their competing probabilities and aggregate daytime demand are equal to $0$.

For consumers whose types are in the set $\Sigma_{2,2}$, a mixed-strategy NE with the PA policy exists if and only if the following condition holds:

 \vspace{-0.1in} 
 \small
\begin{equation}\label{eq:relation_E_0_E_1_pa_ne_extra_demand}
\begin{split}
& \mathcal{ER}\frac{(\gamma-1)}{(\gamma-\varepsilon_{\ell}\beta)}-E_{\ell}  = \mathcal{ER}\frac{(\gamma-1)}{(\gamma-\varepsilon_{\tilde{\ell} }\beta)}-E_{\tilde{\ell}} , \ \forall \ell , \tilde{\ell} \in \Sigma_{2,2}.
\end{split}
 \end{equation}
\normalsize 
\vspace{-0.1in}

Now, assuming that all consumers of the same type play the same mixed strategy, the competing probabilities of the consumers of type $\ell \in \Sigma_{2,2}$ that lead to NE states lie in the range $p^{min}_{\ell} \leq p^{NE}_{RES,{\ell}} \leq p^{max}_{\ell}$, where:

 \vspace{-0.1in} 
 \footnotesize
   \begin{align}
& p^{max}_{\ell} = \min \left\{1,\frac{\frac{\mathcal{ER}(\gamma-1)}{(\gamma-\varepsilon_{\ell}\beta)}-D^{Total}_{\Sigma_1}}{N~ r_{\ell}~ E_{\ell}}\right\},
    \label{eq:plmaxbound}
\end{align}
\begin{align}
&  p^{min}_{\ell}=\nonumber\\&\max \left\{0,\frac{\frac{\mathcal{ER}(\gamma-1)}{(\gamma-\varepsilon_{\ell}\beta)}-D^{Total}_{\Sigma_1}-
  N\sum\limits_{\tilde{\ell} \in \Sigma_{2,2} \setminus \{\ell\}} r_{\tilde{\ell}} ~E_{\tilde{\ell}}}{N~ r_{\ell}~E_{\ell}} \right\}. \label{eq:plminbound}\end{align}
\normalsize  
\vspace{-0.1in}

The aggregate daytime demand of consumers whose type is in $\Sigma_{2,2}$ can be expressed as

 \vspace{-0.1in} 
 \footnotesize
\begin{equation} \label{eq:demand1_2c}
\begin{split}
& D_{\Sigma_{2,2}}(\mathbf{p}^{NE})\\
& = \min \left\{ D^{Total}_{\Sigma_{2,2}} , \max \left\{\frac{N\left(\frac{\mathcal{ER}(\gamma-1)}{(\gamma-\varepsilon_{\ell}\beta)}-E_{\ell}-D^{Total}_{\Sigma_1}\right)}{(N-1)},0 \right\} \right\},
\end{split}
\end{equation}
\normalsize 
\vspace{-0.1in}  

\noindent where $D^{Total}_{\Sigma_{2,2}} = N  \sum_{\tilde{\ell } \in \Sigma_{2,2}} r_{\tilde{\ell }  }E_{\tilde{\ell } }$ is the maximum demand for RESs of the consumers whose types are in $\Sigma_{2,2}$.

%\vspace{-0.1in}
\begin{remark} \label{rem:risk_degrees_relation}
Note that condition \eqref{eq:relation_E_0_E_1_pa_ne_extra_demand} can hold, and therefore a NE can exist, only if for any pair $\ell , \tilde{\ell} \in \Sigma_{2,2}$ such that $\ell \leq \tilde{\ell}$, it holds that $\varepsilon_{\ell} \leq \varepsilon_{\tilde{\ell}}$. Since by assumption, $E_{\ell} \leq E_{\tilde{\ell}}$, this means that consumers with lower daytime energy demand levels should be more risk-seeking than those with higher ones.
\end{remark}
%\vspace{-0.15in}
\begin{remark}\label{rem:risk_seeking}In particular, if all consumers whose type is in $\Sigma_{2,2}$ are risk-seeking (i.e., $\varepsilon_{\ell}= 1, \forall \ell \in \Sigma_{2,2}$), a NE can exist only if  $E_{\ell}=E_{\tilde{\ell}} , \ \forall \ell , \tilde{\ell} \in \Sigma_{2,2}$.
\end{remark}

Finally, the social cost is given by:
\vspace{-0.15in} 

\begin{small}
	\begin{align}
	C(\mathbf{p}^{NE}) &=  \min\{\mathcal{ER}, D(\mathbf{p^{NE}})\} c_{RES} \nonumber \\&+  \max\{0,D(\mathbf{p^{NE}})-\mathcal{ER}\}c_{grid,d} \nonumber \\
	&+N \left[ \sum_{\ell \in \Theta} r_{\ell }~ p^{NE}_{grid,\ell}~ \varepsilon_{\ell }~E_{\ell}\right] c_{grid,n}.
	\label{eq:social_cost_pa_extra_demand}
	\end{align}
\end{small}
