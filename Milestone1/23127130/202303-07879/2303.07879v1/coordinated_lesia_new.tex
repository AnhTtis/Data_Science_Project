\vspace{-0.2in}
\section{Centralized Energy Allocation Mechanism under the PA Policy}\label{sec:coordinated}

In this section, we study the case in which the energy community manager centrally schedules the different energy resources among the $N$ consumers in the energy sharing community so as to minimize the social cost of the overall community under the PA policy. This centralized Energy Allocation Mechanism (EAM) is used as an ideal benchmark against which to evaluate the efficiency of the proposed decentralized ESG.

\subsection{Optimization Problem Formulation}

In practice, in the centralized EAM, all consumers $i$ of type $\vartheta_i$ communicate their daily flexible load $U_{\vartheta_i}$ and risk-aversion degree $\mu_{\vartheta_i}$ (such that their maximum daytime energy demand is $E_{\vartheta_i}= \mu_{\vartheta_i}U_{\vartheta_i}$) to the energy community manager.
%Therefore, the aggregate amount of daytime loads is equal to $D^{Total}$. 
The community manager optimally schedules the loads of the consumers during each time interval (i.e., daytime or nighttime) by selecting the optimal vector of mixed-strategies $\mathbf{p}^{OPT}=[\mathbf{p_1}^{OPT^\top};...;\mathbf{p_{M}}^{OPT^\top}]$ that minimizes the social cost $C(\mathbf{p}^{OPT})$ of the overall community. For each consumer $i$ of type $\vartheta_i$, the vector $\mathbf{p}^{OPT}_{\vartheta_i} = [{p}^{OPT}_{RES,\vartheta_i},{p}^{OPT}_{grid,\vartheta_i}]$ is defined similarly to $\mathbf{p}^{NE}_{\vartheta_i}$, such that the consumer's expected daytime demand is equal to ${p}^{OPT}_{RES,\vartheta_i}E_{\vartheta_i}$, and her expected nighttime demand is equal to $p^{OPT}_{grid,\vartheta_i}U_{\vartheta_i}$. As a result, this centralized EAM is modeled as an optimization problem, defined as:

 \vspace{-0.1in} 
 \begin{small}
 \begin{subequations} \label{eq:social_cost_x_opt}
\begin{alignat}{2}
& \min_{\mathbf{p}^{OPT}} \ && C(\mathbf{p}^{OPT}) = \Bigg( \dfrac{D(\mathbf{p}^{OPT})}{\max\{ \mathcal{ER}, D(\mathbf{p}^{OPT})\}} \mathcal{ER} \Bigg) c_{RES} \nonumber \\ 
& \quad && + \Bigg( D(\mathbf{p}^{OPT}) - \dfrac{D(\mathbf{p}^{OPT})}{\max\{ \mathcal{ER}, D(\mathbf{p}^{OPT})\}} \mathcal{ER} \Bigg)  c_{grid,d} \nonumber \\
& \quad && + N  \sum_{{\ell} \in \Theta} r_{{\ell} }~p^{OPT}_{grid,{\ell}}~\varepsilon_{{\ell} }~E_{{\ell}} ~c_{grid,n} \label{eq:opt_1} \\
 & \text{s.t. } && 0 \leq p^{OPT}_{RES,\ell},~ p^{OPT}_{grid,\ell} \leq 1 ,  \ \forall \ell \in \Theta \label{eq:opt_2} \\
 & \quad && p^{OPT}_{RES,\ell} + p^{OPT}_{grid,\ell} = 1 , \ \forall \ell \in \Theta, \label{eq:opt_4}
 \end{alignat}
 \end{subequations}
\end{small} \vspace{-0.1in}  

\noindent which minimizes the social cost of the community under the PA policy \eqref{eq:opt_1} subject to constraints \eqref{eq:opt_2}-\eqref{eq:opt_4} defining the mixed-strategy probabilities, where the total RES demand $D(\mathbf{p}^{OPT})$ is defined in \eqref{eq:demand}.
%\noindent where, the decision variables of the centralized mechanism are the optimal probability distributions ($p^{OPT}_{RES,\vartheta_i}$, $p^{OPT}_{grid,\vartheta_i}$), which represent the optimal probability that consumer $i$ with type $\vartheta_i$ engages its load during day, or during the night, respectively, for all consumers $i \in \mathcal{N}$. \footnote{The vector $\mathbf{p^{OPT}}$ is defined similarly to $\mathbf{p^{NE}}$.}

%where $N  \sum_{{\ell}\in \Theta} r_{\ell} p_{RES,{\ell}}^{OPT}  E_{\ell} $ represents the total RES production allocated in the day-zone, and $N  \sum_{{\ell} \in \Theta} r_{{\ell} }p^{OPT}_{grid,{\ell}}\varepsilon_{{\ell} }E_{{\ell}}$ represents the total nonRES production allocated in the night-zone. %The first summand refers to the consumers who are served by low-cost RES capacity during the day. The second summand is for the consumers who are served $nonRES$ duing the night-zone. 
%In other words, the part of the optimal social cost that corresponds to the day-zone peak-load production will be zero. %The social costs under the PA and ES policies are defined similarly to Eqs. \eqref{eq:social_cost_pa_extra_demand} and \eqref{eq:social_cost_es}, respectively. 


%The optimal solutions of this centralized mechanism will provide a benchmark against which to evaluate the side-effects that stem from the distributed, uncoordinated energy source selection, in terms of social cost. 
Although this optimization problem is nonconvex, due to its objective function, it can be linearized depending on the values of RES capacity and the preferences of the consumers, specifically their daytime demand.
In the following, we provide insights and analytical formulations of the optimal solutions $\mathbf{p^{OPT}}$ and objective cost $C(\mathbf{p^{OPT}})$ in different cases.

%and the PoA under the PA mechanism. %For the ES mechanism, the centralized decisions for minimizing the social cost are harder to derive analytically and we will present numerical evaluations in Section \ref{sec:eval} using software optimization tools, namely, Mathematica and MATLAB.
\vspace{-0.15in}
\subsection{Optimal Centralized Solutions}
\label{sec:centralsol}
%%%%%%%%% define these costs under each allocation policy!!!????
%We derive the properties of the solution provided by the centralized mechanism, $\mathbf{p^{OPT}}$, in four cases, identical to the ones defined in Section \ref{sec:uncoordinated_extra_demand}.
 
\subsubsection{\textbf{Case $1$. RES capacity exceeds the maximum daytime energy demand}}

In this trivial case, the social cost reduces to:

 \vspace{-0.1in}
 \begin{small}
\begin{align}
C(\mathbf{p}^{OPT}) & = N \sum_{{\ell} \in \Theta} r_{\ell}~ p_{RES,{\ell}}^{OPT}  ~E_{\ell}   c_{RES}  \nonumber \\
& + N  \sum_{{\ell} \in \Theta} r_{{\ell} } \left(1-p^{OPT}_{RES,{\ell}}\right)\varepsilon_{{\ell} }~E_{{\ell}} ~\beta~ c_{RES},
 \label{eq:social_cost_x_opt_case1}
 \end{align}
\end{small} \vspace{-0.1in}  

\noindent and the optimal solutions to the centralized EAM is to schedule all consumers during the day: $p^{OPT}_{RES,\ell}=1$, $\forall \ell \in \Theta$.

\subsubsection{\textbf{Case $2$. Maximum daytime energy demand exceeds the RES capacity}}

In this case, it is optimal for the centralized EAM to schedule loads during the day so that the total RES capacity is fully utilized, i.e., the total energy demand for RES is greater than or equal to the RES capacity:

 \vspace{-0.1in} 
 \small
\begin{align}
N \sum_{{\ell} \in \Theta}r_{{\ell}} ~E_{{\ell}}~p^{OPT}_{RES,{\ell}} \geq \mathcal{ER}.
\label{eq:optimal}
\end{align}
\normalsize 
\vspace{-0.1in}


\noindent Therefore, the social cost reduces to:

 \vspace{-0.1in} 
 \begin{small}
\begin{align}
C(\mathbf{p}^{OPT}) &=  \mathcal{ER} \cdot c_{RES}   + \left[N \sum_{{\ell} \in \Theta} r_{\ell} ~p_{RES,{\ell}}^{OPT}~  E_{\ell} - \mathcal{ER}\right]  \gamma c_{RES} \nonumber \\
& + N \left[ \sum_{{\ell} \in \Theta} r_{\ell } \left(1-p^{OPT}_{RES,{\ell}}\right)\varepsilon_{{\ell} }~E_{{\ell}}\right] \beta c_{RES},
 \label{eq:social_cost_pa_extra_demand_2}
 \end{align}
\end{small} \vspace{-0.1in}

\noindent and the centralized EAM optimization problem \eqref{eq:social_cost_x_opt} is equivalent to minimizing $ N \sum_{\ell \in \Theta} \left[ r_{\ell} E_{\ell} \left(\gamma - \varepsilon_{\ell} \beta \right) p^{OPT}_{RES,\ell} \right] c_{RES}$, subject to constraints \eqref{eq:opt_2}-\eqref{eq:opt_4} and \eqref{eq:optimal}. This linear optimization problem can be solved using a software optimization tool but we can also derive closed-form expressions of its solutions, as detailed below.

We consider the complementary subsets of consumer types, depending on their risk aversion degrees, $\Sigma_1$, $\Sigma_2$, similar to Case $2$ of the ESG. 
\begin{comment}
We distinguish between the two following complementary subsets of consumer types, depending on their risk aversion degree:

\vspace{-0.1in} 
 \begin{small}
\begin{align}
& \Sigma_1 =  \{ \ell \in \Theta : \varepsilon_{\ell } \geq  \dfrac{\gamma}{\beta}  \}, \\
& \Sigma_2 =  \{ \ell \in \Theta :   1 \leq \varepsilon_{\ell } <  \dfrac{\gamma}{\beta} \}. 
 \end{align}
\end{small} \vspace{-0.1in}
\end{comment}
For all consumers whose type $\ell \in \Sigma_1$, it is optimal for the EAM to dispatch them during the day, such that $p^{OPT*}_{RES,\ell}=1 $. Therefore, the optimal schedule for the remaining consumers whose type $\ell \in \Sigma_2$ can be found by solving the following linear optimization problem: 


%\vspace{-0.1in} 
 \begin{small}
 \begin{subequations} \label{eq:social_cost_x_opt_2}
\begin{alignat}{2}
& \min_{\mathbf{p}^{OPT}} \ &&  N \sum_{\ell \in \Sigma_2} \left[ r_{\ell} ~E_{\ell} \left(\gamma - \varepsilon_{\ell} \beta \right) p^{OPT}_{RES,\ell} \right] c_{RES} \label{eq:opt_S2_1} \\
 & \text{s.t. } && \eqref{eq:opt_2} - \eqref{eq:opt_4} \label{eq:opt_S2_2}\\
 & \quad && N \sum_{{\ell} \in \Sigma_2}r_{{\ell}} ~E_{{\ell}}~p^{OPT}_{RES,{\ell}} \geq \left( \mathcal{ER} - N \sum_{{\ell} \in \Sigma_1}r_{{\ell}} E_{{\ell}}\right). \label{eq:opt_S2_3} 
 \end{alignat}
 \end{subequations}
\end{small} %\vspace{-0.1in}

By solving the dual of this optimization problem ({see Appendix \ref{appendix:dual}}), we observe that the consumer types are fully dispatched during the day in the order of the smallest risk aversion degree (or largest $\varepsilon_\ell$), until constraint \eqref{eq:opt_S2_3} is satisfied. Therefore, the optimal competing probabilities for the consumers whose type $\ell^k \in \Sigma_2=\{\ell^1, \ell^2, \dots, \ell^K \}$ can be expressed as:

 \vspace{-0.1in} 
 \footnotesize
\begin{align}
%& p^{OPT}_{RES,\ell^1} = \max \{ 0 , \min \{ 1, \dfrac{\left( \mathcal{ER} - N \sum_{{\ell} \in \Sigma_1}r_{{\ell}} E_{{\ell}} \right)}{ r_{{\ell^1}} E_{{\ell^1} }}\} \} \\
 & p^{OPT*}_{RES,\ell^k} = \nonumber \\
& \max \Bigg\{ 0 ,  \min \Bigg\{ 1, \dfrac{\left( \mathcal{ER} - N \sum_{{\ell} \in \Sigma_1}r_{{\ell}} E_{{\ell}} - N \sum_{i=1}^{k-1}r_{{\ell^i}} E_{{\ell^i}} p^{OPT}_{RES,\ell^i}  \right)}{N r_{{\ell^k}} E_{{\ell^k} }}\Bigg\} \Bigg\} \nonumber \\
&  \forall k \in \{1,...,K\},
\end{align}
\normalsize 
\vspace{-0.1in}  

\noindent where the consumer types $\ell^k \in \Sigma_2$ are ranked, such that $\varepsilon_{\ell^1} = \max \{ \varepsilon_{\ell} , \forall \ell \in \Sigma_2\}$, and $\varepsilon_{\ell^k} = \max \{ \varepsilon_{\ell} , \forall \ell \in \Sigma_2 \setminus \{\ell_1,...,\ell_{k-1}\} \}$ for all $k \in \{1,...,K\}$.

If $N\sum_{{\ell} \in \Sigma_1}r_{{\ell}} E_{{\ell}} \geq \mathcal{ER}$, i.e., the daytime consumption of the consumers whose type $\ell \in \Sigma_1$ fully utilizes the RES capacity, then the solution to this optimization problem is trivial, and for all consumers whose type $\ell \in \Sigma_2$, $p^{OPT}_{RES,\ell}=0$.

\vspace{-0.1in}
\subsection{Efficiency Loss}
The (in)efficiency of equilibrium strategies in the decentralized, uncoordinated mechanism is quantified by the Price of Anarchy (PoA) \cite{Koutsoupias09}. The PoA is expressed as the ratio of the worst case social cost among all mixed strategy NE, denoted as $C^{NE}_w$, over the optimal minimum social cost of the centralized mechanism, denoted as $C(\mathbf{p^{OPT^*}})$, such that:

%\vspace{-5pt}
 \vspace{-0.1in} 
 \small
\begin{align}
 \hspace{-5pt} \textit{PoA}  = \frac{C^{NE}_w}{C(\mathbf{p^{OPT^*}})}.
\label{eq:poa_pa}
\end{align}
\normalsize 
\vspace{-0.1in}  

First observe that $C(\mathbf{p^{OPT^*}})$ is uniquely determined for each particular case. Now, in order to obtain $C^{NE}_w$ when there exist multiple possible NE, we can maximize the social cost $C(\mathbf{p^{NE}})$ with respect to $\mathbf{p^{NE}}$. 





%Note that $C^{PA,NE}(\mathbf{p^{NE}})$ takes its optimal value (i.e, minimum value) when the night-time cost is minimized. This solution coincides with the optimal centralized solution and thus in this case PoA takes the optimal (unity) value. 


 %The last observation is the fact that the demand $D^{PA,NE}$ is constant with respect to $\mathbf{p^{PA,NE}}$. Also, in the special case of the risk-seeking consumers, from Eq. \eqref{eq:social_cost_pa_sc}, the social cost is constant with respect to the probabilities $\mathbf{p^{PA,NE}}$ for each case of energy profile values. Thus, $C^{PA,NE}_w$ is given by Eq. \eqref{eq:social_cost_pa_sc}.

%$2$. If having risk-conservative consumers, there exist multiple possible equilibria in the sub-cases 2(a) and 2(c) as well as in case 4. Of course, the multiple combinations of probabilities can lead to NE with possibly different social cost values. Based on the Remark \ref{rem:risk_degrees_relation}, the existence of NE is possible if consumers with lower energy demands have lower risk aversion degrees. Thus, the night cost is minimized if the optimal probabilities for RES take lower values for consumers with lower energy demands. 
