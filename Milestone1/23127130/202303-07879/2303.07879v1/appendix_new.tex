



\
\section{Proofs for Case $2$ of the ESSG}
\label{sec:proofsESSG}
\vspace{-0.05in}
For the consumers in $\Sigma_1$, we need to show that  $\upsilon_{RES, \ell}(\mathbf{p})<\upsilon_{grid, \ell}(\mathbf{p})$, $\forall \mathbf{p}$ and $\forall \ell \in \Sigma_1$. Assume a consumer type $\ell\in \Sigma_{1}$ and that her allocated RES energy is $E'$. Then, we have that $\upsilon_{RES, \ell}(\mathbf{p})= E'  \cdot c_{RES}+(E_{\ell}-E')\cdot \gamma \cdot  c_{RES}$ and 
$\upsilon_{grid, \ell}(\mathbf{p})= \varepsilon_\ell \cdot E_{\ell} \cdot \beta \cdot c_{RES}$. The inequality $\upsilon_{RES,\ell}(\mathbf{p})<\upsilon_{grid, \ell}(\mathbf{p})$ is then equivalent to $ E'  (1-\gamma) \cdot c_{RES} <  E_\ell \cdot (\varepsilon_\ell \cdot \beta -\gamma) \cdot c_{RES}$, which is true by assumption, since $(1-\gamma)<0$ and $(\varepsilon_\ell \cdot \beta -\gamma)>0$.

                                     
  Next, for the consumers in $\Sigma_{2,1}$, we need to show that $\upsilon_{RES, \ell}(\mathbf{p})>\upsilon_{grid, \ell}(\mathbf{p})$, $\forall \mathbf{p}$ and $\forall \ell\in \Sigma_{2,1}$. Assume a consumer type $\ell \in \Sigma_{2,1}$ and that her allocated RES energy is $E'$. Then, the inequality $\upsilon_{RES,\ell}(\mathbf{p})>\upsilon_{grid, \ell}(\mathbf{p})$ is equivalent to the inequality $E_\ell >E' \frac{(\gamma-1)}{(\gamma-\varepsilon_\ell\beta)}$, which is true by assumption, since $E'<\mathcal{ER}$.

Now, we prove the condition of existence of a mixed strategies NE for the consumers in $\Sigma_{2,2}$. Recall that in the ESG under the PA policy, a mixed strategy NE $\mathbf{p^{NE}}$ among consumers in $\Sigma_{2,2}$ exists under the condition
%=(\mathbf{p_0^{NE}}, \mathbf{p_1^{NE}}, ..., \mathbf{p_{M-1}^{NE}})^T$, with $\mathbf{p}^{NE}_{\ell}=[p^{NE}_{RES,\ell}, p^{NE}_{nonRES,\ell}]^T$, 
\vspace{-0.1in}

\small
\begin{equation}\label{eq:condition_PA_NE_2}
rse_{\ell}^{PA}(\mathbf{p}^{NE}) =rse_{\ell}^{NE}(\mathbf{p}^{NE}), \forall \ell \in \Sigma_{2,2}. \end{equation}
\normalsize
%Next we give the conditions such that either \eqref{eq:condition_PA_NE} holds and mixed NE exist or there exist dominant strategies. For this study, we distinguish cases with respect to the RES capacity, the risk aversion degree values and the daytime energy demand levels. 

To derive condition \eqref{eq:relation_E_0_E_1_pa_ne_extra_demand} we re-write \eqref{eq:condition_PA_NE} first with assuming that a consumer $i$ of type $\vartheta_i \in \Sigma_{2,2}$ plays $RES$ (in \eqref{eq:probrelation1}) and second with assuming that a consumer $j$ with type $\vartheta_j \in \Sigma_{2,2} \setminus \{\vartheta_i\}$ plays RES (in \eqref{eq:probrelation2}):

\vspace{-0.1in}
\begin{small}
\begin{align}
&  \mathcal{ER}\frac{(\gamma-1)}{(\gamma-\varepsilon_{\vartheta_i}\beta)}-E_{\vartheta_i}= D^{Total}_{\Sigma_1}+\sum_{ {\ell'}\in \Sigma_{2,2}} r_{\ell'}~ (N-1)~E_{\ell'}~p^{NE}_{RES,\ell'},
    \label{eq:probrelation1}\\
  &  \mathcal{ER}\frac{(\gamma-1)}{(\gamma-\varepsilon_{\vartheta_j}\beta)}-E_{\vartheta_j}=D^{Total}_{\Sigma_1}+  \sum_{ {\ell'}\in \Sigma_{2,2}} r_{\ell'} ~(N-1)~E_{\ell'}~ p^{NE}_{RES,\ell'}.
    \label{eq:probrelation2}  
\end{align}
\end{small}

%Eq. \eqref{eq:condition_PA_NE} can be re-written in a similar way for any other type $\vartheta_j \in \Theta$. 
Note that to derive (\ref{eq:probrelation1}) we consider that if a consumer $i$ in $\Sigma_{2,2}$ of type $\vartheta_i$ plays $RES$, then, the total demand for RESs of the consumers in $\Sigma_{2,2}$, $D_{\Sigma_{2,2}}(\mathbf{p^{NE}})$ can be expressed as $E_{\vartheta_i}+ \sum_{ {\ell'}\in \Sigma_{2,2}} r_{\ell'}~ (N-1)~E_{\ell'}~p^{NE}_{RES,\ell'}$ for a large number of consumers and similarly also for (\ref{eq:probrelation2}). Then, since the right-hand sides of \eqref{eq:probrelation1}-\eqref{eq:probrelation2} are equal, the left-hand sides will be also equal and \eqref{eq:relation_E_0_E_1_pa_ne_extra_demand} derives. %\eqref{eq:probrelation1}-\eqref{eq:probrelation2} 

To derive the probability bounds, we re-write \eqref{eq:condition_PA_NE} assuming that all consumers of the same type play the same mixed strategy, i.e., 

\vspace{-0.1in}
\begin{small}
\begin{align}
&  \mathcal{ER}\frac{(\gamma-1)}{(\gamma-\varepsilon_{\vartheta_i}\beta)}= D^{Total}_{\Sigma_1}+N\sum_{ {\ell'}\in \Sigma_{2,2}} r_{\ell'}~ E_{\ell'}~p^{NE}_{RES,\ell'}.
    \label{eq:probrelation3}  
\end{align}
\end{small}

The minimum bound on the probability for playing RES, $p_{\ell}^{\min}$, derives by setting in (\ref{eq:probrelation3}) $p^{NE}_{RES,\tilde{\ell}}=1$, $\forall \tilde{\ell}\in \Sigma_{2,2}$ with $\tilde{\ell}\neq \ell=\vartheta_i$. Similarly, the maximum bound on the probability for playing RES, $p_{\ell}^{\max}$, derives by setting in (\ref{eq:probrelation3}) $p^{NE}_{RES,\tilde{\ell}}=0$, $\forall \tilde{\ell}\in \Sigma_{2,2}$ with $\tilde{\ell}\neq \ell=\vartheta_i$. 

Finally, to construct the expression of the aggregate demand for RESs given in \eqref{eq:demand1_2c}, we multiply \eqref{eq:probrelation1} by $\frac{N}{N-1}$ and used the definition of the demand for RESs in \eqref{eq:demand} but expressed only for the consumers in the set $\Sigma_{2,2}$.
\vspace{-0.05in}

\section{Coordinated energy allocation solutions}
\label{appendix:dual}

The dual function of the optimization problem \eqref{eq:social_cost_x_opt_2} is 

\begin{footnotesize}
 \begin{align} \label{eq:social_cost_x_opt_2_dual}
 \max_{\lambda \geq 0}\min_{\mathbf{p}^{OPT}} \quad & N \sum_{\ell \in \Sigma_2} \left[ r_{\ell} E_{\ell} \left(\gamma - \varepsilon_{\ell} \beta \right) p^{OPT}_{RES,\ell} \right] c_{RES} \nonumber \\&-\lambda\left( N \sum_{{\ell} \in \Sigma_2}r_{{\ell}} E_{{\ell}}p^{OPT}_{RES,{\ell}} - \left( \mathcal{ER} - N \sum_{{\ell} \in \Sigma_1}r_{{\ell}} E_{{\ell}}\right)\right),
 \end{align}
\end{footnotesize} \vspace{-0.1in}

\hspace{-0.2in} subject to \eqref{eq:opt_S2_2}, where $\lambda$ represents the dual variable associated with \eqref{eq:opt_S2_3} and let $\lambda^*$ represent its optimal value.

It results that:\\
$\bullet$ for all $\ell \in \Sigma_{2}$ where $1 \leq \varepsilon_\ell < \dfrac{\gamma c_{RES} - \lambda^*}{\beta c_{RES}}$, $p^{OPT}_{RES,\ell}=0$,\\
$\bullet$ for all $\ell \in \Sigma_2$ where $ \varepsilon_\ell = \dfrac{\gamma c_{RES} - \lambda^*}{\beta c_{RES}}$, $0<p^{OPT}_{RES,\ell}<1$,\\
$\bullet$  for all $\ell \in \Sigma_2$ where $ \dfrac{\gamma c_{RES} - \lambda^*}{\beta c_{RES}} < \varepsilon_\ell <\dfrac{\gamma}{\beta}$, $p^{OPT}_{RES,\ell}=1$.

This means that the consumer types are fully dispatched during the day in the order of increasing risk aversion degree (or decreasing $\varepsilon_\ell$), until constraint \eqref{eq:opt_S2_3} is satisfied. 


\section{Analysis For the ES Allocation Policy}
\subsection{Decentralized Energy Selection Game Under ES}
The analysis and proofs of this section follow similar lines as the analysis and proofs for the PA policy. Most proofs are however omitted for brevity.

In the ESSG with the ES policy, a mixed-strategy NE exists under the condition:


 %\vspace{-0.1in} 
 \small
\begin{equation}\label{eq:condition_ES_NE}
rse_{\ell}^{ES}(\mathbf{p}^{NE}) =rse_{\ell}^{NE}(\mathbf{p}^{NE}), \forall \ell \in \Theta. \end{equation}
\normalsize 
\vspace{-0.1in}  

\noindent 
Therefore, for all cases, any existing mixed-strategy NE competing probabilities, $\mathbf{p}^{NE}$, are obtained by resolving condition \eqref{eq:condition_ES_NE}. Let us distinguish the following cases:

\subsubsection{\textbf{Case $1$: The RES capacity $\mathcal{ER}$ exceeds the maximum total demand for RES $D^{Total}$ ($\mathcal{ER} \geq D^{Total}$).}}

As the consumers have knowledge of $\mathcal{ER}$ and $D^{Total}$, it is straightforward to show that the dominant-strategy for all consumers is to select the strategy $RES$. As a result, the competing probabilities that lead to equilibrium states are equal to $1$ for all consumer types, and the expected aggregate demand for RES at NE is equal to $D^{Total}$.

\subsubsection{\textbf{Case $2$: The RES capacity $\mathcal{ER}$ is less than the maximum total demand for RES $D^{Total}$ ($\mathcal{ER} \leq D^{Total}$).}} 

In this case, the strategies of the consumers depend on their respective risk aversion degrees and the energy prices. We define two distinct subsets of consumer types: $\Sigma_1 = \Bigl\{\ell \in \Theta : \varepsilon_{\ell} \geq \gamma/\beta \Bigr\} \subset \Theta$, and $\Sigma_2 = \Bigl\{ \ell \in \Theta :1\leq \varepsilon_{\ell} < \gamma/\beta \Bigr\} \subset \Theta$.

Firstly, the dominant strategy for all consumers whose types are in the set $\Sigma_1$ is to play $RES$. Therefore, their competing probabilities that lead to equilibrium states are equal to $1$, and their aggregate daytime demand is $D^{Total}_{\Sigma_1}=N\sum_{\ell \in \Sigma_1}r_\ell E_\ell$.

Secondly, the strategies of the consumers whose types are in the set $\Sigma_2$ depends on the daytime energy demand levels. Therefore, we define two distinct subsets of consumer types in $\Sigma_2$: $\Sigma_{2,1} = \left\{ \ell \in \Sigma_2 : E_{\ell} > \mathcal{ER}\frac{(\gamma-1)}{(\gamma-\varepsilon_{\ell}\beta)} \right\}$ and $\Sigma_{2,2} = \left\{ \ell \in \Sigma_2 : E_{\ell} \leq \mathcal{ER}\frac{(\gamma-1)}{(\gamma-\varepsilon_{\ell}\beta)}\right\}$.

%$\Sigma_{2,1} = \left\{ \ell \in \Sigma_2 : E_{\ell} > \left(\mathcal{ER}-D^{Total}_{\Sigma_1}\right)\frac{(\gamma-1)}{(\gamma-\varepsilon_{\ell}\beta)} \right\}$ and $\Sigma_{2,2} = \left\{ \ell \in \Sigma_2 : E_{\ell} \leq \left(\mathcal{ER}-D^{Total}_{\Sigma_1}\right)\frac{(\gamma-1)}{(\gamma-\varepsilon_{\ell}\beta)}\right\}$

For consumers whose types are in the set $\Sigma_{2,1}$, the dominant strategy is to play $grid$, and their competing probabilities and aggregate daytime demand are equal to $0$.

For consumers whose types are in the set $\Sigma_{2,2}$, a mixed-strategy NE with the ES policy exists if and only if the following condition holds:

 \vspace{-0.1in} 
 \small
\begin{equation}\label{eq:relation_E_0_E_1_es_ne_extra_demand}
(\gamma-\varepsilon_{\ell}\beta)\cdot E_{\ell} = (\gamma-\varepsilon_{\tilde{\ell} }\beta)\cdot E_{\tilde{\ell}} , \ \forall \ell , \tilde{\ell} \in \Sigma_{2,2}.
\end{equation}
\normalsize 


To derive condition \eqref{eq:relation_E_0_E_1_es_ne_extra_demand} we re-write \eqref{eq:condition_ES_NE} first with assuming that a consumer $i$ of type $\vartheta_i \in \Sigma_{2,2}$ plays $RES$ (in \eqref{eq:probrelation1es}) and second with assuming that a consumer $j$ with type $\vartheta_j \in \Sigma_{2,2} \setminus \{\vartheta_i\}$ plays RES (in \eqref{eq:probrelation2es}).

\vspace{-0.1in}
\begin{small}
\begin{align} \label{eq:probrelation1es}
  D^{Total}_{\Sigma_1}+1+\sum_{ {\ell'}\in \Sigma_{2,2}} r_{\ell'}~ (N-1)~p^{NE}_{RES,\ell'}=\frac{\mathcal{ER}(\gamma-1)}{E_{\vartheta_i}(\gamma-\varepsilon_{\vartheta_i}\beta)},
\end{align}
\end{small}
\vspace{-0.1in}

\begin{small}
\begin{align} \label{eq:probrelation2es}
  D^{Total}_{\Sigma_1}+1+\sum_{ {\ell'}\in \Sigma_{2,2}} r_{\ell'}~ (N-1)~p^{NE}_{RES,\ell'}=\frac{\mathcal{ER}(\gamma-1)}{E_{\vartheta_j}(\gamma-\varepsilon_{\vartheta_j}\beta)}.
\end{align}
\end{small}
Then, since the right-hand sides of \eqref{eq:probrelation1es}-\eqref{eq:probrelation2es} are equal, the left-hand sides will be also equal and \eqref{eq:relation_E_0_E_1_es_ne_extra_demand} derives.

Additionally, for the consumers of type $\ell \in \Sigma_{2,2}$, the competing probabilities that lead to NE states lie in the range $p^{min}_{\ell} \leq p^{NE}_{RES,{\ell}} \leq p^{max}_{\ell}$, where:

 \vspace{-0.1in} 
 \footnotesize
\begin{align}
&  p^{min}_{\ell}=\nonumber\\&\max \left\{0,\frac{\frac{\mathcal{ER}(\gamma-1)}{E_{\ell}(\gamma-\varepsilon_{\ell}\beta)}-
  \sum\limits_{\tilde{\ell} \in \Sigma_{2,2} \cup \Sigma_1 \setminus \{\ell\}}N r_{\tilde{\ell}} }{N r_{\ell}} \right\}, \label{eq:plminbound_appendix}\\
& p^{max}_{\ell} = \min \left\{1,\frac{\frac{\mathcal{ER}(\gamma-1)}{E_{\ell}(\gamma-\varepsilon_{\ell}\beta)}-\sum\limits_{\tilde{\ell} \in  \Sigma_1 }N r_{\tilde{\ell}} }{N r_{\ell}}\right\}.
    \label{eq:plmaxbound_appendix}
\end{align}
\normalsize  
\vspace{-0.1in}

To derive the probability bounds, we re-write \eqref{eq:condition_ES_NE} assuming that all consumers of the same type play the same mixed strategy, i.e., 

\vspace{-0.1in}
\begin{small}
\begin{align} \label{eq:probrelation3es}
  D^{Total}_{\Sigma_1}+\sum_{ {\ell'}\in \Sigma_{2,2}} N~r_{\ell'}~p^{NE}_{RES,\ell'}=\frac{\mathcal{ER}(\gamma-1)}{E_{\vartheta_i}(\gamma-\varepsilon_{\vartheta_i}\beta)}.
\end{align}
\end{small}

The minimum bound on the probability for playing RES, $p_{\ell}^{\min}$, derives by setting in (\ref{eq:probrelation3es}) $p^{NE}_{RES,\tilde{\ell}}=1$, $\forall \tilde{\ell}\in \Sigma_{2,2}$ with $\tilde{\ell}\neq \ell=\vartheta_i$. Similarly, the maximum bound on the probability for playing RES, $p_{\ell}^{\max}$, derives by setting in (\ref{eq:probrelation3es}) $p^{NE}_{RES,\tilde{\ell}}=0$, $\forall \tilde{\ell}\in \Sigma_{2,2}$ with $\tilde{\ell}\neq \ell=\vartheta_i$. 

The Remarks 3 and 4, which are stated for the PA allocation policy in Section \ref{sec:uncoordinated_extra_demand}, also hold in case of the ES allocation policy. 

The social cost under the ES policy can be expressed as 

\footnotesize
\begin{align}
&C^{ES}(\mathbf{p^{NE}}) =  N \sum_{\ell \in \Theta} r_{\ell}~ \min\{sh(\mathbf{p^{NE}}), E_{\ell}\} ~p_{RES, \ell}^{NE}~ c_{RES} \nonumber\\ + &\left[D(\mathbf{p^{NE}}) -N \sum_{\ell \in \Theta} r_{\ell} ~\min\{sh(\mathbf{p^{NE}}), E_{\ell}\} ~p_{RES, \ell}^{NE}\right]
 ~c_{grid,d}\nonumber\\ + &
 N \left[ \sum_{\ell \in \Theta} r_{\ell }~ p^{NE}_{grid,\ell}~ \varepsilon_{\ell }~E_{\ell}\right] c_{grid,n}.
 \label{eq:social_cost_es_sc}
 \end{align}
\normalsize

\subsection{Centralized Energy Allocation Mechanism Under ES Policy}
Similar to the PA policy (Section \ref{sec:coordinated}), the centralized EAM under the ES policy is modeled as an optimization problem, defined as:

 \vspace{-0.1in} 
 \begin{small}
 \begin{subequations} \label{eq:social_cost_x_opt_es}
\begin{alignat}{2}
& \min_{\mathbf{p}^{OPT}} \ && C^{ES}(\mathbf{p}^{OPT}) \label{eq:opt_1_es} \\
 & \text{s.t. } && 0 \leq p^{OPT}_{RES,\ell},~ p^{OPT}_{grid,\ell} \leq 1 ,  \ \forall \ell \in \Theta \label{eq:opt_2_es} \\
 & \quad && p^{OPT}_{RES,\ell} + p^{OPT}_{grid,\ell} = 1,  \ \forall \ell \in \Theta. \label{eq:opt_4_es}
 \end{alignat}
 \end{subequations}
\end{small} 

The problem \eqref{eq:social_cost_x_opt_es} is non-convex due to its objective function and the form of the equal share $sh(\mathbf{p^{NE}})$ (Eq. \eqref{eq:fairshare}). In our simulations in Section \ref{sec:comptoES}, we solve it with genetic algorithms using the Global Optimization Toolbox of MATLAB. 
%given in Table \ref{table_eq_comp_pro}. 
%Otherwise, %under the conditions of the second and seventh line of Table \ref{table_eq_comp_pro}, 
%the expected aggregate demand for renewable energy $D^{ES,NE}(p^{ES,NE})$ is an non-decreasing function of $p^{ES,NE}_{RES,1}$, that is,
%
%\vspace{-5pt}
%\footnotesize
%\begin{eqnarray}
%\hspace{-10pt}  && \hspace{-10pt}  D^{ES,NE}(p^{ES,NE}) =N \sum_{\vartheta_j\in \Theta} p^{ES,NE}_{RES,\vartheta_j} r_{\vartheta_j }E_{\vartheta_j}.
%\label{eq:demand_es_ne_extra_demand}
%\end{eqnarray}
%\normalsize
% \hspace{-10pt} &=& \hspace{-10pt}  \frac{N}{(N-1)} \left[ \mathcal{ER}\frac{(\gamma-1)}{(\gamma-\varepsilon_{0}\beta)} -E_0\right] + (1-r)(E_1-E_0)N p^{ES,NE}_{RES,1}\nonumber\\
% \hspace{-10pt} &=& \hspace{-10pt} D^{PA,NE}(p^{PA,NE})+ (1-r)(E_1-E_0)N p^{ES,NE}_{RES,1}.



%The second line in Eq. \eqref{eq:demand_es_ne_extra_demand} derives by replacing in the first line that  $ p^{ES,NE}_{RES,0} =\frac{1}{r(N-1) } \Big[\mathcal{ER}\frac{(\gamma-1)}{(\gamma-\varepsilon_0\beta)E_0}-1- (1-r) (N-1) p^{ES,NE}_{RES,1} \Big]$.
%Likewise, under the conditions of the sixth line of Table \ref{table_eq_comp_pro}, the expected total number of consumers that compete for $RES$ in equilibrium, equals $\frac{N}{(N-1)}  \left[ \frac{\mathcal{ER}}{E_{\vartheta_j}}\frac{(\gamma-1)}{(\gamma-\varepsilon_{\vartheta_j}\beta)}-1 \right]$ and the expected aggregate demand for renewable energy $D^{ES,NE}(p^{ES,NE})$ is given by (\ref{eq:demand_es_ne_extra_demand}).

%Conditions under which equilibrium competing probabilities generate inefficiencies, that is, a number of consumers incur the price of the lack of coordination are described in Section \ref{sec:coordinated}.


%\textbf{Social Cost:}  The expected aggregate cost incurred by the entire population is a function of $p^{ES,NE}$ (contrary to the Eq. \eqref{eq:social_cost_pa}) and is given by,
%\vspace{-5pt}
%\footnotesize
%\begin{align}
%& C^{ES,NE}(p^{ES,NE})  \nonumber \\ &=  d(p^{ES,NE}) c_{RES} +(D^{ES,NE}-d(p^{ES,NE}))c_{nonRES,D} \nonumber \\
%&   + N\left[\sum_{\vartheta_j \in \Theta} r_{\vartheta_j }p^{ES,NE}_{nonRES,\vartheta_j}\varepsilon_{\vartheta_j }E_{\vartheta_j}\right] c_{nonRES,N},
% \label{eq:social_cost_es_extra_demand}
% \end{align}
%\normalsize 



