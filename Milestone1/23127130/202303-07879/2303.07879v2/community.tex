\vspace{-0.1in}
\section{Decentralized Energy Sharing Mechanism} \label{sec:esc}


\subsection{Energy Sharing Community} \label{sec:community}

The energy community consists of $N$ consumers, indexed by $i \in \mathcal{N}= \{1,...,N\}$, who have access to multiple energy sources in order to cover their flexible loads.

\subsubsection{Energy Sources}\label{sec:energy_sources}

We consider that the energy community has access to two distinct types of energy sources, namely \textit{local production} from community-owned RESs, and \textit{imports} from the distribution grid. We consider that the local RESs production is available only during daytime (e.g., PV panels), with a limited capacity $\mathcal{RE}>0$, whereas the community's imports from the grid are unlimited. Therefore, during nighttime the community's aggregate load is fully covered by imports from the grid, and, during daytime if the community's aggregate load exceeds the available RES capacity, the remainder is covered by imports from the grid.

%%
Production from the community-owned RESs is priced by the community manager at a constant low tariff $c^{RES}$ (in units per energy), whereas imports from the grid are priced by an energy retailer using TOU tariffs, typically for daytime and nighttime consumption. We define the daytime and nighttime tariffs with respect to the RESs tariff, as $c^{grid,d}= \gamma c^{RES}$ and $c^{grid,n} = \beta c^{RES}$, respectively, with $\gamma>\beta>1$.
%%
These TOU tariffs reflect the sum of energy prices and grid tariffs and are designed to incentivize consumers to shift their flexible loads from daytime to nighttime to reduce energy production costs and congestion during peak hours. In addition, the low cost of the local RESs production promotes self-consumption within the community and reduction of grid imports. We assume that the energy source-related parameters $\Omega = \{\mathcal{RE}, c^{RES}, \beta, \gamma\}$ are perfectly known by all consumers in the community at the beginning of the day.

\subsubsection{Consumers Preferences}

The consumers have a broad range of flexible loads, namely, (i) shiftable appliances (e.g. washing machines) that do not need to be scheduled every day, (ii) batteries or electric vehicles (EVs) with flexible state-of-charge requirements at the end of the day, and (iii) thermostatically controlled loads (e.g., water heater, heat pumps) with flexible set-points. The level of consumption and the time-schedule of these loads are flexible. For instance, an EV owner has a daily inflexible load required to cover her daytime transportation needs, and a daily flexible load, representing the additional energy to achieve a desired state-of-charge by the end of the day.
%At the beginning of the day, the daily flexible loads of all consumers in the community can be scheduled during daytime or nighttime. The level and time interval at which they schedule their daily flexible loads depend on their consumption preferences, as well as the available energy sources. 
However, once scheduled, these loads cannot be interrupted or shifted to another time interval.
%The available energy sources available at a given TOU interval will then be shared among the consumers whose daily flexible loads are scheduled during the same TOU interval. 
As a result, consumers whose daily flexible loads are scheduled during daytime incur the risk of paying for high-priced imports from the grid if the community's aggregate daytime energy demand exceeds the available local RES production. When scheduling their daily flexible loads across different time intervals, consumers wish to achieve a trade-off between their desired daily energy consumption, and the financial risks incurred. And, risk-averse consumers may choose to reduce their daily energy consumption if they are scheduled during daytime, to mitigate the financial risks incurred. For instance, if scheduled during nighttime, a risk-averse EV owner may prefer to consume enough energy to fully charge her EV by the end of the day, whereas, if scheduled during daytime, she may prefer to consume a smaller amount of energy in order to charge her EV at e.g., $75\%$ by the end of the day.

The risk attitude and daily energy consumption preferences of each consumer $i \in \mathcal{N}$ in the community can be represented by her type $\vartheta_i \in \Theta = \{1,...M\}$. The type accounts for consumer's (i) daily flexible load $U_{\vartheta_i}>0$ (in energy unit); and (ii) risk-aversion degree $\mu_{\vartheta_i} \in [0,1]$, representing the share of her daily flexible load that she is willing to consume if scheduled during daytime. 

%%
With this parametric representation of the consumers' flexibility preferences, if the daily flexible load of a consumer $i$ of type $\vartheta_i$ is scheduled during daytime, her daytime energy demand is $E_{\vartheta_i} = \mu_{\vartheta_i} U_{\vartheta_i}$ (and the remainder of her daily flexible load $(1-\mu_{\vartheta_i})U_{\vartheta_i}$ is deferred to the following day), whereas, if her daily flexible load is scheduled during nighttime, her nighttime demand is $U_{\vartheta_i}$. Therefore, $\mu_{\vartheta_i}=1$ represents a risk-seeking consumer, and $\mu_{\vartheta_i} < 1$ a risk-conservative consumer. 

At the beginning of each day, each consumer knows her own flexibility preferences and type, but this information is considered private. We assume that the community manager and consumers in the community only know the probability distribution $\bm{r}=[r_1,...,r_{M}]^T$ over the consumers types $\Theta$, where $0 \leq r_{\vartheta} \leq 1$ is the probability that a consumer in the community is of type $\vartheta \in \Theta$. Furthermore, the consumers' preferences, and therefore their types, can vary from day to day. Since this paper studies a single scheduling day, the daily time indexes are omitted. 

Following the law of large numbers, the number of consumers of type $\vartheta \in \Theta$ can be approximated as $r_\vartheta \cdot N$. Thus, based on the above, the maximum daytime energy demand of the community, i.e., if the daily flexible loads of all consumers are scheduled during daytime is 
\vspace{-0.08in}
\begin{align}
 D^{Total} = N  \sum_{\vartheta \in \Theta} r_{\vartheta}~E_{\vartheta}.  
\end{align}
\vspace{-0.02in}
For notational simplicity, in the remainder of the paper, we introduce $\varepsilon_{\vartheta_i}=\frac{1}{\mu_{\vartheta_i}}$, such that $U_{\vartheta_i}=\varepsilon_{\vartheta_i} \cdot E_{\vartheta_i}$. Thus, $\varepsilon_{\vartheta_i}=1$ represents a risk-seeking consumer $i$, and $\varepsilon_{\vartheta_i}>1$ a risk-conservative consumer. Finally, we assume without loss of generality that $E_1\leq E_2 \leq ...\leq E_{M}$.

%Therefore, the expected maximum daytime schedule of the community, i.e. if all consumers' flexible loads are scheduled during daytime can be expressed as $D^{d,Total} = N \sum_{\vartheta \in \Theta} r_{\vartheta} E_{\vartheta}$. 
\vspace{-0.15in}
\subsection{Decentralized Energy Sharing Mechanism (D-ESM)}

The problem faced by the energy sharing community is to schedule the daily flexible loads of all consumers across the different TOU intervals and to allocate the different energy sources among them within each TOU interval. The role of the community manager is to design a mechanism that optimally coordinates the interactions among consumers in the community towards desirable outcomes, namely: (i) minimizing the social cost for the community as a whole, and (ii) sharing the community-owned assets among the consumers fairly. We introduce below the proposed decentralized energy sharing mechanism (D-ESM) for this energy sharing community.

\subsubsection{Load Scheduling}

In the proposed D-ESM, each consumer independently schedules her own daily flexible loads across the different TOU intervals, at the beginning of the day, in order to maximize her own utility under the set energy source allocation and payment policies. In contrast, in a Centralized ESM (C-ESM), the community manager would schedule the daily flexible loads of all consumers across the different TOU intervals in order to minimize the social cost of the community as a whole under the set energy source allocation and payment policies (see Section \ref{sec:coordinated}). As implementing this centralized approach would require for the community manager to have information on each consumer's preferences, it can only be considered as an ideal benchmark against which to compare the efficiency of the proposed D-ESM.

In this paper, we study \textit{mixed strategies} of consumer types. A \textit{mixed strategy} is a probability distribution $\mathbf{p}_{\vartheta}=[p_{\vartheta}^d, p_{\vartheta}^n]^T$, with $p_{\vartheta}^d \in [0,1]$ denoting the probability that a consumer of type $\vartheta \in \Theta$ schedules her daily flexible load during daytime, and $p_{\vartheta}^n \in [0,1]$ during nighttime. At the beginning of the day a consumer $i$ determines her mixed strategy based on her type $\vartheta_i \in \Theta$, $\mathbf{p}_{\vartheta_i}$. Then she schedules her daily flexible loads either in daytime or in nighttime with probabilities $p_{\vartheta}^d$, $p_{\vartheta}^n$, correspondingly. Let also $\bm{p}$ be the collection of mixed strategies of all consumers, i.e., $\bm{p}= \{\bm{p}_{\vartheta_i}\}_{i \in \mathcal{N}}$. 

%As the self-interested consumers internalize the energy source allocation and payment policies set by the community manager to schedule they daily flexible loads, 
%It is essential to design adequate energy source allocation policies that will incentivize consumers to act in a beneficial way for the community as a whole and in the following we introduce the proposed PA policy that satisfies desirable notions of fairness.
 
\subsubsection{Energy Source Allocation and Payment Policies} \label{sec:policies}
 
Once the daily flexible loads of all consumers have been scheduled, the community manager must allocate the available energy sources at each TOU interval (daytime or nighttime) among them. During nighttime, all scheduled loads are covered by grid imports since this is the sole available energy source for this TOU interval. During daytime, the community manager allocates in priority the local RESs production to cover the scheduled daytime loads, in order to maximize local consumption from the community and reduce energy costs. However, if the expected aggregate daytime energy demand exceeds the available local RESs production, the community manager must share this limited resource among those consumers with loads scheduled during daytime. This raises the challenging issue of allocating fairly a limited resource among users with equal claims to it.

%%
In order to ensure a notion of fairness among community members, the community manager allocates to each consumer $i$ of type $\vartheta_i$ a share of the local RESs production proportional to her daytime load schedule. As a result, under this PA policy, the local RESs production allocated to a consumer whose daily flexible load is scheduled during daytime is

\vspace{-0.1in}
\begin{small}
\begin{eqnarray}
res^{PA}_{\vartheta_i}(\mathbf{p}) &=& \frac{E_{\vartheta_i}}{\max(\mathcal{RE} , D^d(\mathbf{p}))}\mathcal{RE},
\label{eq:prop_alloc_energy}
\end{eqnarray}
\end{small}
\vspace{-0.1in}

\noindent where $D^d(\mathbf{p})$ denotes the expected aggregate daytime demand of the community.

Each consumer $i$ of type $\vartheta_i$ must then pay for the different energy sources covering her scheduled load at each TOU interval, ensuring budget balance of the proposed mechanism.

%%% assumptions
\vspace{-0.1in}
\subsection{Non-cooperative Game Formulation} \label{sec:game_def}
%\vspace{-0.05in}

Based on the proposed D-ESM framework, if a consumer schedules her daily flexible load during daytime, she competes with other consumers to use the limited local RESs production and incurs a financial risk. This competition among the consumers participating in the proposed D-ESM (for one single day) can be modeled as an Energy Sharing Game (ESG), as defined bellow.

%%When scheduling their daily flexible loads, each consumer $i$ has perfect knowledge about the energy sources parameters $\Omega = \{\mathcal{RE}, c^{RES}, c^{grid,d}, c^{grid,n}\}$ and its own flexibility preferences ($E_{\vartheta_i}$ and $\mu_{\vartheta_i}$), represented by its type $\vartheta_i$, and imperfect knowledge about other players' preferences, represented by the distribution $\mathbf{r}$ over consumers' types $\Theta = \{\vartheta_i\}_{i \in \mathcal{N}}$.

\vspace{-0.05in}
\begin{definition}\label{def:energy_source_game}
An \emph{Energy Sharing Game (ESG)} is a single-shot noncooperative game, defined by the tuple
\\$\Gamma=(\mathcal{N}, \{\mathcal{P}_{\vartheta_i}\}_{i\in\mathcal{N}}, \{\upsilon_{\vartheta_i}\}_{i\in \mathcal{N}})$, where:
\begin{itemize}
    \item $\mathcal{N}=\{1,...,N\}$ is the set of players, i.e., the consumers in the energy sharing community.
    \item $\mathcal{P}_{\vartheta_i} = \{ \mathbf{p}_{\vartheta_i} | \mathbf{p}_{\vartheta_i} : A_i \in \mathcal{A} \rightarrow p^{A_i}_{\vartheta_i} \in \mathbb{R}^+ , \text{ with } \sum_{A_i \in \mathcal{A}} p^{A_i}_{\vartheta_i} = 1 \}$ is the set of mixed strategies of player $i$ of type $\vartheta_i$ over the set of pure strategies $\mathcal{A}=\{d,n\}$, consisting of the choices to schedule her daily flexible load during daytime ($A_i = d$) or during nighttime ($A_i = n$). Therefore, each consumer $i$ of type $\vartheta_i$ with a mixed strategy $\mathbf{p}_{\vartheta_i}$, plays this game by randomly selecting an action $A_i \in \mathcal{A}$ with probability $p^{A_i}_{\vartheta_i}$\footnote{Note that a \textit{pure strategy} is a special case of a mixed strategy where one action has a probability equal to 1 (and the remaining have 0).}.
    \item $\upsilon_{\vartheta_i} : A_i \in \mathcal{A} \rightarrow  \upsilon^{A_i}_{\vartheta_i}$ is the payoff function of a consumer $i$ of type $\vartheta_i$ over the set of pure strategies $\mathcal{A}$. The cost of a consumer $i$ of type $\vartheta_i$ who plays the pure strategy $A_i = d$, is 
    \begin{small}
    \begin{align}
    \upsilon^{d}_{\vartheta_i} = c^{RES} res_{\vartheta_i}^{PA}(\mathbf{p}) + c^{grid,d} (E_{\vartheta_i}-res_{\vartheta_i}^{PA}(\mathbf{p})),
    \label{eq:RES_cost}
    \end{align}
    \end{small}
    \hspace{-0.1in} and depends on the strategy profile $\bm{p}$ of all consumers via the community's expected aggregate daytime energy demand $D^d(\mathbf{p})$. The cost of a consumer who plays the pure strategy $A_i = n$ is
    \begin{small}
    \begin{align}\upsilon^{n}_{\vartheta_i} = U_{\vartheta_i}  c^{grid,n},
    \label{eq:nonRES_cost}
    \end{align}
    \end{small}
    \hspace{-0.1in} and is independent on other consumers' mixed strategies.
    Before making their decisions all players have perfect knowledge of the energy sources parameters in the set $\Omega$ and their own preferences and type, and have prior knowledge on the probability distribution $\bm{r}$ over the other consumers' types.
\end{itemize}
\end{definition}


A consumer of type $\vartheta \in \Theta$ repeatedly playing the mixed strategy $\bm{p}_{\vartheta}$ over multiple instances of the ESG %(or equivalently, a large number of consumers of type $\vartheta \in \Theta$ playing the mixed strategy $\bm{p}_{\vartheta}$ over a single instance of the ESG), 
would have an \textit{expected} daytime and nighttime energy demand equal to $D_{\vartheta}^{d} = p_{\vartheta}^{d} E_{\vartheta}$ and $D_{\vartheta}^{n} = p_{\vartheta}^{n} U_{\vartheta}$, respectively. Therefore, the mixed strategy of a consumer $i$ of type $\vartheta_i$ can alternatively be interpreted as splitting her daily flexible loads between daytime and nighttime, such that her daytime load schedule is equal to $D_{\vartheta_i}^{d}$, and their nighttime load schedule is equal to $D_{\vartheta_i}^{n}$. %Then, each consumer $i$ tries to maximize its expected profit by scheduling a load equal to $E_{\vartheta_i}$ during the day with probability $p^{d}_{\vartheta_i}$, and a load equal to $U_{\vartheta_i}$ during the night with probability $p^{d}_{\vartheta_i}$.
With these notations, the \textit{expected} aggregate daytime and nighttime energy demands of the community are respectively expressed as

\vspace{-0.1in}
\begin{small}
\begin{subequations}
\begin{align}
    & D^d (\mathbf{p}) = N\sum_{\vartheta\in \Theta} r_{\vartheta }~p^d_{\vartheta}~E_{\vartheta,}\label{eq:demand} \\
     & D^n (\mathbf{p}) = N\sum_{\vartheta\in \Theta} r_{\vartheta }~p^n_{\vartheta}~U_{\vartheta}\label{eq:demand_n}.
 \end{align}
\end{subequations}
\end{small}
\vspace{-0.15in}


