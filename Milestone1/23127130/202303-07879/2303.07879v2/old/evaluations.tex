\section{Numerical Evaluations}
\label{sec:eval}

\subsection{Case Study Setup}

We consider a smart grid with $N=1000$ consumers, divided into $5$ distinct consumer types, with a maximum total demand for RESs $D^{Total} = 4250$. Table \ref{tab:residential} summarizes the consumer types parameters.
\begin{table}[b]
\vspace{-0.15in}
\footnotesize
    \centering
    \begin{tabular}{|c||c|c|c|c|c|}
        \hline
        Type $\ell$ & 0 & 1 & 2 & 3 & 4 \\
        \hline 
        $E_\ell$ (kWh)& 2 & 3 & 5 & 10 & 15 \\
        \hline
        $r_\ell$ & 0.20 & 0.40 & 0.30 & 0.07 & 0.03 \\
        \hline
    \end{tabular}
    \caption{Game parameters for residential smart-grid.}
    \vspace{-0.15in}
    \label{tab:residential}
\end{table}
The type distribution and the day-time energy demand levels are selected to be consistent with European households \cite{enerdata}. Most households are moderately energy efficient (types $1$ and $2$), combined with many highly efficient households (type $0$) and few inefficient ones (types $3$ and $4$). Consumers with type $0$ are assumed to be risk-seeking ($\epsilon_0=1$) and the risk-aversion degrees of all other types are determined by  \eqref{eq:relation_E_0_E_1_pa_ne_extra_demand}, and are close to $1$. We set the RES price as $c_{RES}=1$ \euro/kWh, and the parameters $\beta=2$ and $\gamma=3$. 

The proposed DRP with PA allocation is compared to a DRP with the naive ES allocation. Under ES, a so-called \textit{fair share} of RESs capacity is computed as

\vspace{-0.1in}
\begin{small}
\begin{align} \label{eq:fairshare}
 sh(\mathbf{p^{ES,NE}}) 
& =\frac{\mathcal{ER}}{ N \sum_{ \ell=0}^{M-1} r_{\ell}  p_{RES,\ell}^{ES,NE}},
\end{align}
\end{small}
\vspace{-0.1in}

\noindent where $  N \sum_{ \ell=0}^{M-1}  r_{\ell}  p_{RES,\ell}^{ES,NE}$ represents the number of consumers competing for $RES$ during the day. Under ES, consumers of type $\ell \in \Theta$ that play $RES$ and have a demand $E_{\ell} \leq sh(\mathbf{p^{ES,NE}})$ are allocated their full demand $E_{\ell} $, and the extra energy that they are entitled to $(sh(\mathbf{p^{ES,NE}})-E_{\ell})$ will remains unused. The remaining RESs capacity is allocated equally between the consumers of type $\ell \in \Theta$ that play $RES$ and have a demand $E_{\ell}$ larger than the fair share $sh(\mathbf{p^{ES,NE}})$. Therefore, the share of RESs received by a consumer $i$ of type $ \vartheta_i \in \Theta$ that plays $RES$ is $rse^{ES}_{\vartheta_i}(\mathbf{n_i}) = \min\left( E_{\vartheta_i}, sh(\mathbf{p^{ES,NE}}\right)$. This allocation policy may result in large inefficiencies due to unused RES capacity, even when the total aggregate demand for RESs $D^{ES,NE} (\mathbf{p^{ES,NE}})$ is higher than $\mathcal{ER}$. Therefore, this allocation policy is solely used as a base-case comparison to the PA allocation considered in this work.


\subsection{Social Costs and PoA}
\vspace{-0.07in}

\begin{figure}[ht!] 
     \centering
     \subfigure[Social Cost (in eurocents). \label{fig:4a}]{
         \includegraphics[width=0.2\textwidth]{plots/social_cost_PA_residential.eps}}
  \subfigure[PoA. \label{fig:4b}]{
         \includegraphics[width=0.2\textwidth]{plots/PoA_PA_residential.eps}}
        \caption{Social cost and PoA under PA rule for residential grid.}\vspace{-0.2in}
        \label{fig:sc_PoA_PA}
\end{figure}

\subsubsection{PA Rule}
Fig. \ref{fig:sc_PoA_PA} shows the social cost and the PoA with increasing $\mathcal{ER}$ that ranges from $5\%$ to $125\%$ of $D^{Total}$. First, we observe that the optimal social cost for the centralized approach decreases linearly with $\mathcal{ER}$. This is because the optimal allocation should equalize the RES demand and capacity, while minimizing the night-time cost. Since here all risk-aversion degrees are equal or close to 1, the cost at night can be approximated as {\small$N \sum_{{\ell} \in \Theta} r_{{\ell} } \cdot p_{nonRES,{\ell}} \cdot \epsilon_{{\ell} } \cdot E_{{\ell}} \cdot c_{nonRES,n}\approx N \sum_{{\ell} \in \Theta} r_{{\ell} } \cdot p_{nonRES,{\ell}} \cdot E_{{\ell}} \cdot c_{nonRES,n}\approx$ $N \sum_{{\ell} \in \Theta} r_{{\ell} } \cdot  E_{{\ell}} \cdot c_{nonRES,n} - \mathcal{ER}\cdot c_{nonRES,n} $}, which is constant with respect to the competing probabilities and linearly decreasing with $\mathcal{ER}$. Using the second in row approximation we can point out one more property of the optimal allocation: the night-time cost is mostly affected by the day-time energy demands and its minimization results in "big players" competing for RESs at the expense of smaller ones. This is indeed observed in our evaluations where consumers with lower day-time energy demand compete for RES with non-zero probability only if all consumers with higher day-time energy demand compete for RES with probability 1 and there exists still available RES capacity. %with type $4$ start competing for RES and only when their competing probability reaches $1$ (as $\mathcal{ER}$ increases) consumers with type $3$ start competing and so on for consumers of types 2 and 1.
%the night-time cost is mostly affected by the energy demand of each type. This explains why for values of $\mathcal{ER} < \mathcal{ER}_{max}$, %the centralized approach favours the types with the highest demand, namely, as $\mathcal{ER}$ increases, consumers with type $4$ start competing for RES and only when their competing probability reaches $1$, only then consumers with type $3$ start competing. Following the trend, type $2$ consumers compete for RES only when type $3$ consumers compete with probability equal to $1$, and so on. Therefore, 
%the PA rule indeed favours "big players" at the expense of smaller ones at the optimal allocation. %As a final note, naturally, for $\mathcal{ER} \geq \mathcal{ER}_{max}$, RES capacity adequately covers all the consumers, thus consumers of all types compete for RES.

On the other hand, for the uncoordinated mechanism the social cost decreases linearly with $\mathcal{ER}$, only when $\mathcal{ER}\in [0.5 D^{Total}, D^{Total}]$. The social cost barely reacts to the initial increase in RES capacity due to the fact that consumers tend to overcompete for RES, even for lower values of RES capacity, as it is observed in the obtained values of the compteting probabilities. Moreover note that the mixed strategy NE solution should satisfy \eqref{eq:probrelation1}, except if this does not give acceptable probability values. Since the maximum value that the right hand side of \eqref{eq:probrelation1} can take is $A^{Total}=D^{Total}- \sum_{ {\vartheta_l}\in \Theta} r_{\vartheta_l} E_{\vartheta_l} $ (when all competing probablilities are equal to $1$), this requirement is only possible if the left hand side is less or equal than $A^{Total}$. With the applied parameter values, the left hand side of \eqref{eq:probrelation1} is equal to $A^{Total}$ for 
$\mathcal{ER} \approx 50\% D^{Total}$. Therefore, for $\mathcal{ER} \geq 50\% D^{Total}$ \eqref{eq:probrelation1} cannot be strictly satisfied and the closest we can get to its satisfaction is by setting the competing probabilities of all consumers equal to $1$. Thus, all consumers compete for RES with probability equal to $1$ and there is extra demand, which costs the high day-time prices. Finally, when $\mathcal{ER}\in [0.5 D^{Total}, D^{Total}]$, the social cost is falling when $\mathcal{ER}$ is increasing because fewer highly priced day-time energy is required.

Fig. \ref{fig:4b} shows the PoA. The most inefficient outcome is  for $\mathcal{ER} \approx 0.5 \cdot D^{Total}$, which is the point that the social cost for the uncoordinated mechanism begins decreasing. This graph can provide valuable insight into how much RES capacity we should install. %in order to achieve the socially most efficient outcome.
We see two zones of high efficiency, namely for low and high RES capacity. In the first zone, this is due to the small gains in cost offered by low RES capacity. In the second, the NE solution has almost converged to the optimal solution and thus social costs are optimal as well. 

Finally, the value of $\mathcal{ER}$ at which PoA reaches its peak (most inefficient outcome) depends on the system model parameters and most importantly on $\beta, \gamma$, since the left hand side of \eqref{eq:probrelation1} includes the quantity $\eta=\frac{(\gamma-1)}{(\gamma-\epsilon_{\vartheta_i}\beta)}$. For instance if $\beta$ increases, $\eta$ increases and PoA reaches its peak at a lower value of $\mathcal{ER}$. This can be seen in Fig. \ref{fig:sc_PoA_PA_beta_higher} that shows the same evaluations as Fig. \ref{fig:sc_PoA_PA}, but with $\beta=2.5$. As expected, both centralized and uncoordinated social
costs are higher than with $\beta=2$ because with $\beta=2.5$ the night-zone price increases. Moreover, the change in PoA is quite significant: the uncoordinated social cost curve started falling earlier, namely
at $20 \% \cdot D^{Total}$. This is reflected on the PoA graph, which now peaks at
around $1.16$, much lower than before.

\begin{figure}[t] 
     \centering
     \subfigure[Social Cost (in eurocents). \label{fig:12a}]{
         \includegraphics[width=0.2\textwidth]{plots/social_cost_PA_residential_beta.eps}}
  \subfigure[PoA. \label{fig:12b}]{
         \includegraphics[width=0.2\textwidth]{plots/PoA_PA_residential_beta.eps}}
        \caption{Social cost and PoA under PA rule for residential grid with $\beta=2.5$.}\vspace{-0.2in}
        \label{fig:sc_PoA_PA_beta_higher}
\end{figure}


\subsubsection{ES Rule}

\begin{figure}[t] 
     \centering
     \subfigure[Social Cost (in eurocents). \label{fig:11a}]{
         \includegraphics[width=0.2\textwidth]{plots/social_cost_ES_residential.eps}}
  \subfigure[PoA. \label{fig:11b}]{
         \includegraphics[width=0.2\textwidth]{plots/PoA_ES_residential.eps}}
        \caption{Social cost and PoA under ES rule for residential grid.}\vspace{-0.2in}
        \label{fig:sc_PoA_ES}
\end{figure}

Fig. \ref{fig:sc_PoA_ES} shows the social cost and the PoA with increasing $\mathcal{ER}$ ranging from $5\%$ to $125\%$ of $D^{Total}$. 
Both centralized and uncoordinated curves decrease with increasing $\mathcal{ER}$, but not in a linear way contrary to the PA rule (Fig. \ref{fig:sc_PoA_PA}). This is because the social cost for ES is not linear with $\mathcal{ER}$ (see for example \eqref{eq:social_cost_es_sc}). Moreover, we observe that the uncoordinated curve follows the same trend as the PA rule, namely, it is flat for small values of $\mathcal{ER}$ and then it decreases. This is because similarly to the PA rule, consumers tend to overcompete for RES also under the ES rule. %This is seen in the competing probability values obtained in our evaluations and can be explained analytically in a similar way as with the PA rule (using \eqref{eq:probrelationes1}). 
Also, we should note that in the optimal allocation the competing probabilities may not be all equal to $1$ even for $\mathcal{ER}=125\%\mathcal{ER}_{max}$. We can attribute this observation to the fact that smaller players may need to reduce their competing probability in order to increase the RES share and subsequently the RES utilization, even if that means they forgo the chance to pay less. In addition, the social cost for ES is higher than for PA, regardless of the $\mathcal{ER}$ value. This can be easily explained by observing that, under the ES rule, either part of the RES capacity is unused because some consumers have lower demand than their share, or consumers are forced to pay the day-zone nonRES because their share is less than their demand. In both cases, the social cost increases. 

The PoA (Fig \ref{fig:sc_PoA_ES}) takes generally lower values than in the case of the PA rule. On the other hand, the game outcome is $100\%$ efficient only when the RES capacity is $\mathcal{ER}=125\%\mathcal{ER}_{max}$, whereas for the PA rule the PoA is equal to 1 when $\mathcal{ER}= 110\%\mathcal{ER}_{max}$. Hence, using the ES rule over the PA rule maybe more expensive as more RES capacity should be installed for ensuring efficiency of the uncoordinated demand side management. Finally, due to the non-linearity of the social cost function with $\mathcal{ER}$, the PoA curve does not decrease monotonously after the initial peak.


\subsection{Evaluation of Distributed Algorithm}

Here, we evaluate the performance and convergence of the distributed algorithm \ref{algorithm}. For better visualization purposes, we have implemented the algorithm considering two consumer types and the following parameter values: $N=500$, $E_0=100$ kWh, $E_1=200$ kWh, $r_0=0.7$, $r_1=0.3$, $\epsilon_0=1$, $\epsilon_1=1.004$ (derives from \eqref{eq:relation_E_0_E_1_pa_ne_extra_demand} using $\epsilon_0$), $c_{RES}=100$ \euro/kWh, $\beta=2$, $\gamma=4$. Also, $D^{Total}=65000$ kWh and $\mathcal{ER}=25\% \cdot D^{Total}=16250$ kWh. We study three capping systems, namely, (i) $cap$ is constant and equal to $0.1$, (ii) $cap=rand$ and (iii) $cap=1$, i.e., practically there is no capping system.

Fig \ref{fig:algo_PA_ES} shows the solution paths given by the algorithm for both PA and ES policies and for all three capping systems. Also, it shows the line of the NE solutions that derive from \eqref{eq:probrelation1} for the PA policy and from \eqref{eq:probrelationes1} for the ES policy. It can be observed that all solution paths converge to a theoretically proven NE. For the constant cap ($cap=0.1$), the solution path oscillates around the $45^{\circ}$ line. Therefore, the achieved NE solution consists of similar competing probability values for both consumer types. Actually, imposing a constant cap greatly dampens any bias towards any type. %If a lower value was chosen for the $cap$ parameter, the effect of the cap would be magnified, since players would be even more restricted when contributing to their type's competing probability. 
If the random cap method is implemented, the solution path is naturally random. On the one hand, no type is systematically favored, at the expense of having no control over the solution path. Lastly, if there is no cap at all, the behavior is different under the PA and ES rules. For the PA policy, the consumer type that plays first gains a considerable advantage (see Fig. \ref{fig:algoPArule}). The behavior is not the same under the ES policy, as we see in Fig. \ref{fig:algoESrule}, since type $1$ achieves a higher competing probability at the end, even if type $0$ plays first. This could be explained by the non linear nature of the corresponding social cost.


\begin{figure}[t] 
     \centering
     \subfigure[PA rule. \label{fig:algoPArule}]{
         \includegraphics[width=0.2\textwidth]{plots/algo_sol_small_grid_PA.eps}}
  \subfigure[ES rule. \label{fig:algoESrule}]{
         \includegraphics[width=0.2\textwidth]{plots/algo_sol_small_grid_ES.eps}}
        \caption{Algorithmic solution under PA and ES rules.}\vspace{-0.2in}
        \label{fig:algo_PA_ES}
\end{figure}


 Tables \ref{tab:algo_sc_PA} and \ref{tab:algo_sc_ES} summarize the evaluation results on the social cost, the demand for RESs and the PoA for the optimal centralized solution as well as for the solution of the distributed algorithm for the three capping systems.

{\footnotesize \begin{table}[h!]
    \centering
    \begin{tabular}{|c||c|c|c|}
        \hline
        Solution & Social Cost ($\times 10^6$) & Demand & PoA \\
        \hline 
        Optimal & 11.37 & 16250 & 1\\
        %\hline
       % Worst case & 13.01 & 24324 & 1.14\\
        \hline
        Equal cap & 13.00 & 24321 & 1.14\\
        \hline
        Random cap & 12.99 & 24286 & 1.14\\
        \hline
        No cap & 13.01 & 24324 & 1.14 \\
        \hline
    \end{tabular}
    \caption{Social cost, demand and PoA under the PA rule for centralized and distributed algorithmic solutions.}
    \label{tab:algo_sc_PA}
\end{table}

\begin{table}[h!]
    \centering
    \begin{tabular}{|c||c|c|c|}
        \hline
        Solution & Social Cost ($\times 10^6$) & Demand & PoA \\
        \hline 
        Optimal & 12.75 & 16250 & 1\\
       % \hline
        %Worst case & 15.99 & 39253 & 1.25\\
        \hline
        Equal cap & 15.99 & 32281 & 1.25\\
        \hline
        Random cap & 15.98 & 38045 & 1.25\\
        \hline
        No cap & 15.99 & 37137 & 1.25 \\
        \hline
    \end{tabular}
    \caption{Social cost, demand and PoA under the ES rule for centralized and distributed algorithmic solutions.}
    \label{tab:algo_sc_ES}
\end{table}}

For both policies, all three capping methods lead to similar social cost and PoA values. Thus, the choice of capping method can only influence the competing probabilities to introduce a sense of fairness for sharing the RES capacity among the consumer types, without affecting the social cost. With respect to the demand, we can observe that it is much higher for the ES policy, similarly with the social cost. This is due to the fact that consumers overcompete for RES, and particularly small players, leading to wasted renewable energy. 

%As for the demand, at first we see a difference between the PA and ES rules, namely demand is considerably higher in all uncoordinated equilibria under the ES rule than the PA rule. This is due to the incentive that small players have to compete. thus increasing total demand.

Table \ref{tab:convergence} summarizes the convergence results, where one round is equivalent to one player taking their turn to compute their best response. The tolerance is set to $tol=10^{-4}$. If we do not apply a capping scheme the algorithm has the fastest convergence at the expense of fairness to the allocation of RES. This happens, because we do not restrict the rate at which the solution reaches the NE line. Introducing a constant capping system renders convergence slightly worse, but a similar number of rounds is required at every run. %This is due to fact that the first steps of the solution path are identical every time we run the algorithm and the difference comes only from the final steps.
 Note that, although lower values of $cap$ increase fairness among consumer types, they also increase the number of rounds until convergence. Lastly, with the random cap, we have no control over the solution path and thus no control over the convergence speed. This is why we see a big variance in the required number of rounds to convergence. However, most importantly, we see that the number of rounds until convergence is much lower than the number of players ($N=500$), which showcases the efficiency of the algorithm. 

\begin{table}[h!]
    \centering
    \begin{tabular}{|c||c|c|}
        \hline
        Cap method & PA & ES \\
        \hline 
        Equal cap & $18$ & $21-23$ \\
        \hline
        Random cap & $17-27$ & $17-34$ \\
        \hline
        No cap & $14$ & $17-18$ \\
        \hline
    \end{tabular}
    \caption{Rounds until convergence.}
    \label{tab:convergence}
\end{table}