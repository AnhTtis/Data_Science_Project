\section{Uncoordinated energy source selection game}\label{sec:game}
%\textbf{we need to specify a time-unit of taking the decisions: if we take them only once then (i) ER cannot be known, (ii) why not centralized approach. This is studied for one time interval. Can I decide at a next time interval to move back some loads sent to night rate at a previous interval to ER - intra-day re- scheduling per interval? Nice to describe how this is done. }




%We also assume that the consumers do not increase their demand when opting for the night-zone. 
%The challenge addressed in this paper is finding a distributed and uncoordinated scheduling of the consumer loads in day and night-zones so that the consumers' average cost is minimized. Based on the prices, if the consumers' average cost is minimized, the amount of RES actually used (and not curtailed) should be maximized. %More specifically, we solve the problem of the energy source selection, between RES and night-zone nonRES, by the consumers themselves. %We first present a short discussion on a benchmark centralized solution and afterwards we will discuss the simpler to implement distributed one. 
%
%In time zones when the capacity of RES suffices to serve the entire energy demand, consumers can readily opt for using it and gain the price differential between the two resources. When, however, the demand exceeds the limited RES, an inherent competition emerges that should be factored by consumers in their decision to opt for requesting this resource or not at the particular time period. The underlying assumption is that the decision to opt for the finite RES bears the risk of experiencing the congestion penalties induced by the lack of coordination and pay for the more expensive, top-priced nonRES. Essentially, the main dilemma faced by consumers is whether to compete or not for requesting RES or settle for less favoured time zones and pay for medium-priced energy (assumed being covered by a nonRES facility).
%
%In order to analytically investigate the suffered congestion penalties due to the uncoordinated access to the limited resource, it is important to identify and model those key features and critical parameters that tune the intensity of competition. In this respect, we may consider $N$ consumers that decide between RES and nonRES. The first facility is capacity-constrained and able to serve a total energy demand of $\mathcal{ER}$ energy units (\eg kilowatt-hours, kWh) while the second one responds to every possible level of energy demand, yet heavily burdens the environment producing large amounts of pollutants. In the pricing arena, RES is offered at $c_{RES}$ cost units, whereas for nonRES different rates are used for daily and night energy consumption: moderate-priced night energy rate of $c_{nonRES,N} = \beta \cdot c_{RES}, \beta>1$, units and top-priced daily energy rate of $c_{nonRES,D}= \gamma \cdot  c_{RES}, \gamma>\beta$, units. The excess penalty cost $\delta\cdot c_{RES}$, with $\delta=\gamma-\beta>0$, captures the impact of congestion for RES.

%\textbf{Optimal coordinated RES allocation:} 

%Note that when consumers increase their demand when avoiding the competition, the optimal allocation becomes more complex since it should compare the aggregate cost for consuming nonRES at day-zone against the aggregate cost for consuming (more) nonRES at night-zone. These decisions are communicated to the consumers through the smart-grid facility. 

%
%
%Under the optimal coordinated energy source allocation scheme, %the full information processing and decision-making tasks lie with a central entity.
%micro-grid operators %, operators of energy efficient neighbourhoods and operators of building blocks that host smart metering 
%use the smart grid-related information so as to coordinate the actions of all different stakeholders involved and secure that (a) the local RES - and storage systems - are optimally used and fairly distributed among consumers and (b) no one would pay the excess congestion penalty. In essence, individual customers/consumers that range from owners/operators of large building and facilities to house customers, are expected to issue energy demand requests to a central server, which monitors RES, possesses precise information about its availability, and assigns it so that the aggregate consumers' cost is minimized. Ideally, the efficient exploitation of the smart meter attributes requires the use of a highly intelligent mechanism which takes into account the complex interplay between energy consuming devices, weather conditions, building dynamics, user preferences and comfort, grid status, availability of RES or storage systems \etc.
%
%Overall, in an micro-grid environment where the aggregate demand exceeds the RES capacity of $\mathcal{ER}$ energy units, %with RES capacity of $\mathcal{ER}$ energy units, whereby such an coordination system serves the energy requests of $N$ consumers that jointly shape an energy demand exceeding the RES capacity, 
%exactly $\mathcal{ER}$ units of demand would be directed to RES and no one would pay the excess congestion penalty cost. Advanced decentralized realizations of optimal coordination mechanisms may optimally shape the $\mathcal{ER}$ capacity, pricing and the energy demand of all different consumers within the operator's entity so as the locally available RES is exploited to the maximum while the needs from purchasing from the top-priced conventional grid are minimized.

%\textbf{Uncoordinated energy source selection:} As already mentioned, the design and implementation of a coordinated energy source selection is complex due to the required information exchange, fairness criteria to be considered, heterogeneity of the micro-grid environments (often consisting of fairly diveRES components such as electric cars, wind turbines, solar farms and all kinds of different appliances with various capabilities and requirements), \etc. 

%%%%%%%%%%%%%%%%

We consider a micro-grid with $N$ (eligible) consumers which have access to multiple sources of energy.
In the day-zone, consumers have access to (i) a limited RES capacity $\mathcal{ER}>0$ (in energy units \eg kWh) at a low-cost price $c_{RES}$ per unit of energy, and (ii) an unlimited peak-load production at a high-cost price $c_{nonRES,D}= \gamma \cdot  c_{RES}$, with $\gamma>1$. In the night-zone, consumers have access to an unlimited base-load production, with a medium-cost $c_{nonRES,N} = \beta \cdot c_{RES}$, with $\gamma>\beta>1$.
%%% assumptions
All prices, as well as the amount of the available RES $\mathcal{ER}$ are assumed known to the consumers. %urthermore, network constraints are neglected.

The allocation of the available RES $\mathcal{ER}$ to the consumers could be carried out optimally through a fully coordinated centralized approach, such as described in Section \ref{sec:coordinated}. Besides the increased communication and processing requirements, a coordinated RES allocation is a complex process that needs to take into account fairness criteria, the complex interplay between energy consuming devices, %the heterogeneity of the micro-grid environments often consisting of fairly diverse components such as electric cars, wind turbines, solar farms and all kinds of different appliances with various capabilities and requirements, weather conditions, 
user preferences/comfort, \etc, and it is not scalable. In addition, a major concern of a centralized solution is the privacy of consumers, since they should reveal their preferences/constraints to a central server. Besides, there is no guarantee that consumers will act truthfully.

Consequently, this section describes a distributed, uncoordinated energy source selection approach. Based on the known costs and available RES capacity, the consumers can choose to engage their load during the day-zone or the night-zone. Each consumer acts selfishly aiming at minimizing the cost of the acquired energy.
%%%
A micro-grid operator collects these loads and allocates the energy sources accordingly. If the aggregate amount of day-zone loads does not exceed $\mathcal{ER}$, they are all allocated RES production during the day-zone. Otherwise, certain day-zone loads are allocated the RES production, and the remaining ones are allocated the peak-load production in the day-zone. Therefore, these loads will incur an \emph{excess penalty cost} equal to $(\gamma-\beta) \cdot c_{RES}$ per unit of energy load, compared to being engaged during the night-zone.

%prices' definitions, if the consumers' average cost is minimized, the amount of RES actually used (and not curtailed) should be maximized. 


%Advanced decentralized realizations of the aforementioned optimal centralized coordination may also be considered, but this is outside the focus of this paper. We are only interested in comparing the optimal (theoretical) performance (expressed in terms of the minimum average aggregate consumer cost) delivered through centralized or decentralized coordinated energy allocation, against that under the (simpler to implement) uncoordinated approach considered in this paper. 

%
%Typically, the research in the optimization of energy grid system operation, amounts to defining and solving a system-level optimization problem based on a centralized objective function. However, under heterogeneous micro-grid networks, often consisting of different components such as electric cars, diesel generators, wind turbines, solar farms and all different appliances with various capabilities and objectives, specific objective function for each individual component should be taken into account. Indeed, considering the demand side, in the absence of central coordination, each consumer acts selfishly aiming at minimizing the cost of the acquired energy. The intuitive tendency to opt for RES combined with its scarcity, foster tragedy of commons effects and highlight the game-theoretic dynamics behind the energy source selection task.
%
 %coming up with optimal decisions for operating their associated energy consuming devices. 
%In doing so, they are assumed here to hold precise information about the RES capacity and the pricing on all energy sources involved (RES, day and nigh time grid energy). They also may or may not hold precise information about the actual demand, \ie competition. These assumptions are fairly realistic and feasible since the decision-making task can be assigned to intelligent meters that leverage the available information (through the communication network layer within the smart grid infrastructure) to perform complex computations and best respond to others' actions. Furthermore, these meters may account for user preferences and comfort in the reasoning process. To this end, we assume \textit{two} generic energy consumption profiles corresponding to low-to-moderate and high consumption level energy users.

We will compare the optimal (theoretical) performance of the minimum average aggregate consumer cost, achieved via a centralized coordinated energy allocation, against the one under the proposed distributed, uncoordinated approach.


\vspace{-2pt}
\subsection{Energy Source Selection Game}\label{sec:game_definition}



As mentioned in the introduction, the intuitive tendency to opt for RES combined with its scarcity foster the tragedy of commons effects and highlight the game-theoretic dynamics behind the energy source selection task. Thus, the distributed uncoordinated decision-making on energy source selection can be formulated in game-theoretic terms, whereby $N$ rational players/consumers compete against each other for a limited capacity $\mathcal{ER}$ of low-cost RES. In detail, the game is defined as follows.

%This distributed, uncoordinated decision-making on energy source selection can be defined via the following game: 
\begin{definition}\label{def:energy_source_game}
An \emph{Energy Source Selection Game} is a tuple
\\$\Gamma=(\mathcal{N}, \mathcal{ER}, \{\vartheta_{i}\}_{i\in \mathcal{N}}, \{A_{i}\}_{i\in\mathcal{N}},  \{w_{A_{i},\vartheta_{i}}\}_{i\in \mathcal{N}})$, where:
\begin{itemize}
\item $\mathcal{N}=\{1,...,N\}$, $N>1$ is the set of players, i.e., energy consumers.
\item $\mathcal{ER}>0$ is the limited RES capacity in energy units.
\item $\vartheta_{i} \in \Theta=\{0,1,...,M-1\}$ is the type of consumer $i$ ($i\in \mathcal{N}$) corresponding to an energy consumption profile (load) $E_{\vartheta_i}$, in energy units. We define as $0\leq r_{\vartheta_{i}}\leq 1$, the probability that a consumer has a type $\vartheta_{i} \in \Theta$. Then, we denote with $N_{\vartheta_{i}} = r_{\vartheta_{i}} N $ the total number of consumers of type $\vartheta_{i}$. It should hold that $M \leq N$ and it is assumed, without loss of generality, that $E_0<E_1<...<E_{M-1}$. %where $0$ ($1$) stands for low-to-moderate (high) energy level users with demand of $E_0$ ($E_1$, $E_1 \geq E_0$) energy units,
\item $A_{i} $ is the pure strategy of player $i$ taking values in the set of potential pure strategies $\mathcal{A}=\{RES,nonRES\}$ (same for all consumers), consisting of the choice to engage their load in the day-zone to compete for ``low-priced renewable-source energy'' ($RES$) and the choice to engage their load during the night-zone to access the ``medium-priced base-load energy source'' ($nonRES$).
%\item $s_{i}: \Theta \rightarrow A$, the strategy function for each consumer $i\in \mathcal{N}$,
\item $w_{A_i,\vartheta_i}: \mathbb{R}^M \rightarrow \mathbb{R}$ is the cost function of a player $i$ with type $\vartheta_i$ and pure strategy $A_i$. 
\end{itemize}
\end{definition}




%In addition, 

%\item a consumer has an energy demand level $E_{\ell}$ with probability $r_{\ell}$, where $\sum_{\ell\in \Theta} r_{\ell}=1$,

%\item $c_{i}^{N_0}(A_i,\vartheta_i)$ are cost functions under particular type, $\vartheta_i$, and strategy, $A_i$, that are basically determined based on $w_{RES,\vartheta_i}(\cdot)$ and $w_{nonRES,\vartheta_i}(\cdot)$.
%$c_{i}^{(\cdot)}(s(\vartheta),\vartheta)= c_{i}^{(\cdot)}(s_{i}(\vartheta_{i}),s_{-i}(\vartheta_{-i}),\vartheta_{i},\vartheta_{-i})$, are the expected cost functions for each consumer $i\in \mathcal{N}$, for every type profile $\vartheta\in \times_{k=1}^{N}\Theta_{k}$ and strategy profile $s(\vartheta)\in \times_{k=1}^{N}S_{k}$, ($c_{i}^{N_0}(s(\vartheta),\vartheta)$ are cost functions under particular type, $\vartheta$, and strategy, $s(\vartheta)$, profile with $N_0=N-\sum_k \vartheta_k$), that are functions of $w_{RES,\vartheta_i}(\cdot)$ and $w_{nonRES,\vartheta_i}(\cdot)$,
%\end{itemize}

Furthermore, in the game of Definition \ref{def:energy_source_game}, we take into account various risk attitudes of the consumers.
We characterize a consumer as \textit{risk-seeking} when it gambles its total demand in the game when playing the strategy $RES$; that is, it will attempt to have its entire possible energy demand served by RES and minimize its cost. On the contrary, a \textit{risk-conservative} consumer when playing the $RES$ strategy it gambles part of its total possible energy demand. Note that we assume that consumers with the same type $l \in \Theta$ will have the same risk attitude, but this is not limiting, as we can have as many consumer types as the total number of consumers.  

The above-described risk attitude is captured by adjusting the energy demand of a consumer $i\in \mathcal{N}$ only if he decides to engage its loads in the night-zone. Specifically, in the night-zone, the energy demand of consumer $i$ becomes $\epsilon_{\vartheta_i} E_{\vartheta_i}$, where the parameter $\epsilon_{\vartheta_i}\geq 1$ is the ``{risk-aversion degree}''. Then, $\epsilon_{\vartheta_i}=1$ represents a risk-seeking consumer $i$, and $\epsilon_{\vartheta_i}>1$ a risk-conservative consumer $i$. Besides, when a consumer decides to engage its loads during the day-zone, its energy demand remains equal to $E_{\vartheta_i}$. Thus, a risk conservative consumer has a total possible energy demand equal to $\epsilon_{\vartheta_i} E_{\vartheta_i}$ with $\epsilon_{\vartheta_i}>1$ and (i) when it decides the strategy $RES$, it engages part of its load equal to $E_{\vartheta_i}$  and not its maximum possible demand $\epsilon_{\vartheta_i}E_{\vartheta_i}$, (ii) when it decides $nonRES$, it engages all of its load equal to $E_{\vartheta_i}$. 
%Note that if a risk-averse consumer type $\vartheta_i$ with risk aversion degree $\epsilon_{\vartheta_i}>1$ goes for RES, it engages a load equal to $E_{\vartheta_i}$ and not its maximum possible demand $\epsilon_{\vartheta_i}E_{\vartheta_i}$.



%and similarly the corresponding number of competitors of high energy profile is $n_{1}$ ($=\sum_{j \in \mathcal{N}\setminus {i}} A_j \mathbf{1}_{(\vartheta_j=1)}$).

Next, we give the expressions of the cost functions and the aggregate demand for RESs. Let us first define $n_{l,i} = \sum_{j \in \mathcal{N}\setminus {i}} \mathbf{1}_{(A_j=RES)}\cdot \mathbf{1}_{(\vartheta_j=l)}$ as the number of consumers of type $l \in \Theta$ taking action $RES$, excluding consumer $i \in \mathcal{N}$ of type $\vartheta_i \in \Theta$. Also
% $\footnote{Note that if $\vartheta_i \neq l$, $n_{l,i} = \sum_{j \in \mathcal{N}} \mathbf{1}_{(A_j=RES)}\cdot \mathbf{1}_{(\vartheta_j=l)}$.},
let $\mathbf{n_i}=[n_{0,i}, n_{1,i},..., n_{M-1,i}]^T$ be the vector collecting $n_{l,i}$ for all consumer types $l \in \Theta$.

Then, the cost function $w_{RES,\vartheta_i}(.)$ for a consumer $i$ of type $\vartheta_i$ that chooses the pure strategy $RES$ is defined as

\begin{small}
\begin{align}
w_{RES,\vartheta_i}(\mathbf{n_i}) =
rse_{\vartheta_i}(\mathbf{n_i}) \cdot c_{RES} + (E_{\vartheta_i}-rse_{\vartheta_i}(\mathbf{n_i}) ) \cdot c_{nonRES,D},
\label{eq:RES_cost}
\end{align}
\end{small}
where $rse_{\vartheta_i}(\mathbf{n_i}) $ is the amount of RES allocated to a consumer $i$ of type $\vartheta_i$. %Note that all consumers $j \in \mathcal{N} \setminus {i}$ with $\vartheta_j=\vartheta_i$ and strategy RES are granted with the same amount of RES. Therefore, 
This cost function %is non-decreasing with the number of players competing for RES ($n_{\ell,i}$,  $\ell \in \Theta$) and 
has the same value for all players of the {same} energy profile that play RES.
%consumers who decide to compete for RES undergo the risk of incurring congestion penalties. 
In addition, $rse_{\vartheta_i}(.) $ (and therefore $w_{RES,\vartheta_i}(.)$), is determined based on the actions taken by the entire population, as well as the allocation rule that the micro-grid operator adopts to distribute RES among those who compete for it. %, on the other side. %We denote an action profile by the vector $a=(a_{i},a_{-i})\in\times_{k=1}^{N}A_{k}$, where $a_{-i}$ denotes the actions of all other consumers but player $i$ in the profile $a$. The full set of $2^N$ different action profiles maps into $(N_0+1)(N-N_0+1)$ different action \emph{meta-profiles}, where $N_0$ is the number of low-to-moderate energy consumption users. Each meta-profile $a(k_0,k_1)$, with $k_0$, $k_1$ denoting the number of competitors of low-to-moderate and high energy profile, respectively, encompasses all different action profiles that correspond to the same number of consumers of the two energy profiles competing for RES. The expected costs for these $k_0$, $k_1$ consumers and for the $N-(k_0+k_1)$ ones choosing directly the medium-priced nonRES alternative are functions of $a(k_0,k_1)$ rather than the exact action profile.

On the other hand, the cost function $w_{nonRES,\vartheta_i}(.)$ for a consumer $i$ of type $\vartheta_i$ that chooses the pure strategy $nonRES$ is oblivious to others' actions, that is, 
%\vspace{-7pt}
%\small
\begin{equation}
%w_{nonRES,\vartheta_i}=E_{\vartheta_i} \cdot c_{nonRES,N}.
%\label{eq:nonRES_cost}
w_{nonRES,\vartheta_i}=\epsilon_{\vartheta_i} \cdot E_{\vartheta_i} \cdot c_{nonRES,N}.
\label{eq:nonRES_cost2}
\end{equation}
\normalsize
%In general, the cost $c_i^{N_0}(a_i,a_{-i})$ for consumer $i$ under the action profile $a=(a_i,a_{-i})$ is
%
%%\vspace{-10pt}
%\small
%\begin{eqnarray}
%\hspace{-1pt}c_{i}^{N_0}(a_{i},a_{-i})=\left\{
%\begin{array}{l l}
%\hspace{-5pt}w_{RES,\vartheta_i}(N_{RES,0}(a), n_{1}(a)), \text{if~ $a_{i}=RES$} \nonumber\\
%\hspace{-5pt}w_{nonRES,\vartheta_i}, \text{if~ $a_{i}=nonRES$} \nonumber\\
%\end{array} \right.
%\label{equ:consumer_cost_profile}
%\end{eqnarray}
%\normalsize
%
%where $N_{RES,0}(a), n_{1}(a)$ are the numbers of competitors of low-to-moderate and high energy profile \textit{but} player $i$, respectively, under action profile $a$.
Finally, the aggregate demand for RES can be defined as:
\begin{equation} \label{eq:demand_RES}
    D_{RES} = \sum_{l \in \Theta} n_{l,i} E_l + E_{\vartheta_i}, \forall i \in \Theta
\end{equation}
In addition, we define the quantity $D^{Total}=N\sum_{\ell \in \Theta}r_{\ell}E_{\ell}$ which represents the aggregated demand for RES if all consumers play the strategx $RES$. 

\subsection{Mixed strategies}

Besides choosing one of the two pure strategies, $RES$ and $nonRES$, the consumers may also randomize their actions over them. A mixed strategy is a probability distribution over the pure actions\footnote{Note that a pure strategy is a special case of a mixed strategy.}. We mainly draw our attention on symmetric mixed strategies as being more helpful in dictating practical strategies in real systems. In particular, the \emph{mixed strategy} of all consumers of type $\vartheta_i$ corresponds to the probability distribution $\mathbf{p}_{\boldsymbol \vartheta_i}=[p_{RES,\vartheta_i}, p_{nonRES,\vartheta_i}]^T$, where $p_{RES,\vartheta_i}$ is the probability of playing the strategy $RES$ and $p_{nonRES,\vartheta_i}$ of the strategy $nonRES$. We denote with $\mathbf{p}=[\mathbf{p_0}^T;\mathbf{p_1}^T;...;\mathbf{p_{M-1}}^T]$ the vector of mixed strategies for all consumer types.

%=(p_{RES,\vartheta_i},p_{nonRES,\vartheta_i})$, where $p_{RES,\vartheta_i}$ and $p_{nonRES,\vartheta_i}$ are the probabilities of selecting the pure strategies RES and nonRES, respectively, with $p_{RES,\vartheta_i}+p_{nonRES,\vartheta_i}=1$ ($\vartheta_i \in \Theta$). %Also, let $p=(p_0^T;p_1^T)$ be the matrix with lines the symmetric mixed strategies for each energy profile, namely, $p_0=(p_{RES,0}; p_{nonRES,0})$ and $p_1=(p_{RES,1}; p_{nonRES,1})$. %associated with the two energy profiles. 
Assuming that all other consumers play the mixed strategy $\mathbf{p}$. In this case, the expected costs for a player $i$ of type $\vartheta_i$ playing the pure strategies $RES$ and $nonRES$ take the respective forms:%is a weighted sum of the cost functions $w_{RES,\vartheta_i}(\cdot)$ and $w_{nonRES,\vartheta_i}(\cdot)$ and is given by
%\vspace{-15pt}
%\small
%\begin{eqnarray}\label{eq:costs_symRES}
%    c_{i}(RES,p, p_{0}, p_{1}) = \sum_{N_0=0}^{N-1}\sum_{N_{RES,0}=0}^{N_0}\sum_{n_{1}=0}^{N-1-N_0}    w_{RES,\vartheta_i}(\cdot)  && \nonumber\\
%    \cdot B(n_{1};N-1-N_0,p_{RES,1}) \cdot  B(N_{RES,0};N_0,p_{RES,0}) && \nonumber\\
%      \cdot B(N_0;N-1,p) &&
%    \vspace{-10pt}
%\end{eqnarray}
%\normalsize
%\begin{align}
%   \upsilon_{ \vartheta_i}^{N_0}(p_0,p_1) = &  p_{RES,\vartheta_i} ~\upsilon_{RES, \vartheta_i}^{N_0}(p_0,p_1)  \nonumber \\&  + p_{nonRES,\vartheta_i} \upsilon_{nonRES, \vartheta_i}^{N_0}(p_0,p_1),\label{eq:costs_tot}
%\end{align}
%\noindent where the expected costs of the pure strategies RESs, nonRES of consumer $i$ are expressed as
\begin{align}& \upsilon_{RES, \vartheta_i}(\mathbf{p})=  \sum_{N_{0}=0}^{N}~ \sum_{N_{1}=0}^{N-N_0} ... \sum_{N_{M-1}=0}^{N-\sum_{k=0}^{M-2}{N_k}}\nonumber\\  & \Bigl( \sum_{n_{0,i}=0}^{N_0-\mathbf{1}_{(\vartheta_i=0)}}~\sum_{n_{1,i}=0}^{N_1-\mathbf{1}_{(\vartheta_i=1)}}...  \sum_{n_{M-1,i}=0}^{N_{M-1}-\mathbf{1}_{(\vartheta_i=M-1)}} ~w_{RES,\vartheta_i}(n) \cdot \nonumber\\
 &B(n_{0,i};N_0-\mathbf{1}_{(\vartheta_i=0)},p_{RES,0}) \cdot \nonumber \\& B(n_{1,i};N_1-\mathbf{1}_{(\vartheta_i=1)},p_{RES,1}) \cdot... \nonumber \\& B(n_{M-1,i};N_{M-1}-\mathbf{1}_{(\vartheta_i=M-1)},p_{RES,M-1})\Bigr) \cdot... \nonumber \\&B(N_0;N,r_0)B(N_1;N,r_1)...B(N_{M-1};N,r_{M-1}),  \label{eq:costs_symRES}
\end{align}
\begin{align}
\label{eq:costs_symnonRES}
   \upsilon_{nonRES, \vartheta_i}(\mathbf{p}) = w_{nonRES,\vartheta_i},
\end{align}
\noindent where $B(n_{l,i};N_l-\mathbf{1}_{(\vartheta_i=l)},p_{RES,l}) $ is the value of the Binomial probability distribution for $n_{l,i}$ consumers (excluding consumer $i$) of type $l \in \Theta$ competing for RES, with parameters $N_l-\mathbf{1}_{(\vartheta_i=l)}$ and $p_{RES,l}$, and $B(N_l;N,r_l)$ is the value of the Binomial distribution for $N_l$ consumers of type $l \in \Theta$ with parameters $N$ and $r_0$. %and $B(N_0;N-1,p)$ is the Binomial probability distribution with parameters $N-1$ and $p$, for $N_0$ consumers of low-to-moderate energy consumption users.
%These expected values are defined based on both the strategy choice by other consumers and their energy profile.

Under large populations of prosumers, the cost for the pure strategy $RES$, $\upsilon_{RES, \vartheta_i}(\mathbf{p})$, given by Eq. (\ref{eq:costs_symRES}), can be approximated by

\begin{small}
\begin{align}
&\upsilon_{RES, \vartheta_i}(\mathbf{p})=w_{RES,\vartheta_i}(\mathbf{n_i}),\label{eq:gen_costs_symRES1} \\&
\mathbf{n_i}=(r_0 p_{RES,0}, r_1 p_{RES,1}, ...,  r_{M-1} p_{RES,M-1})^T\cdot (N-1).\label{eq:gen_costs_symRES2}
\end{align} 
\end{small}

For the rest of the paper, the considered cost functions will be Eqs. \eqref{eq:gen_costs_symRES1}-\eqref{eq:gen_costs_symRES2} for the RES strategy and Eq. \eqref{eq:costs_symnonRES} for the nonRES strategy. 






\subsection{Nash equilibrium}


\begin{theorem}
The Energy Source Selection Game has at least one (mixed) Nash Equilibrium.\label{thm:nash}
\end{theorem}
The proof of Theorem \ref{thm:nash} is based on the well-known property that a game with a finite set of players and finite set of strategies has at least one (mixed) Nash Equilibrium. To derive the equilibria, we recall that any mixed strategy equilibrium $\mathbf{p^{NE}}=(\mathbf{p_0^{NE}}, \mathbf{p_1^{NE}}, ..., \mathbf{p_{M-1}^{NE}})^T$ must fulfill

\vspace{-0.1in}
\small
\begin{equation}\label{eq:cost_equality_mixed}
\upsilon_{RES, \vartheta_i}(\mathbf{p^{NE}})= \upsilon_{nonRES, \vartheta_i}(\mathbf{p^{NE}}), ~\forall \vartheta_i \in \Theta.
\end{equation}
\normalsize
\vspace{-0.1in}

\noindent Namely, the expected costs of each pure strategy that belongs to the support of the equilibrium mixed strategy are equal.

Furthermore, at NE the amount of RES allocated to a consumer type $\vartheta_i \in \theta$ satisfies\footnote{This is derived by replacing Eqs. (\ref{eq:costs_symnonRES})-%(\ref{eq:gen_costs_symRES1}), 
(\ref{eq:gen_costs_symRES2}) to Eq. \eqref{eq:cost_equality_mixed}, using \eqref{eq:RES_cost}, \eqref{eq:nonRES_cost2}.}

\vspace{-0.1in}
\small
\begin{equation}\label{eq:conditionEQ_extra_demand}
rse^{NE}_{\vartheta_i}(\mathbf{n_{i}}) = \frac{\gamma-\epsilon_{\vartheta_i}\beta}{\gamma-1}E_{\vartheta_i},~ \forall \vartheta_i \in \Theta.
\end{equation}
\normalsize
\vspace{-0.1in}

Note that the probability that a consumer $i \in \mathcal{N}$ of type $\vartheta_i \in \Theta$ plays the strategy $RES$ (resp. $nonRES$) at NE also depends on the energy source allocation policy used by the micro-grid operator. This Nash equilibrium probability, under a specific allocation mechanism $x$, is denoted $p^{x,NE}_{RES,\vartheta_i}$ (resp. $p^{x,NE}_{nonRES,\vartheta_i}$) and the corresponding probability distribution is denoted $\mathbf{p}^{x,NE}_{\vartheta_i}=[p^{x,NE}_{RES,\vartheta_i}, p^{x,NE}_{nonRES,\vartheta_i}]^T$.

In addition, for a given allocation mechanism $x$, we express the expected aggregate demand for RES at NE as %$N\sum_{\vartheta_j \in \Theta} r_{\vartheta_j }~p^{PA,NE}_{RES,\vartheta_j}~E_{\vartheta_j}$. % with respect to the energy profile 

\vspace{-0.1in}
\begin{small}
\begin{eqnarray}
\hspace{-30pt} && D^{x,NE} (\mathbf{p^{x,NE}})= N\sum_{\ell\in \Theta} r_{\ell }~p^{x,NE}_{RES,\ell}~E_{\ell}.
\label{eq:expected_demand}
\end{eqnarray}
\end{small}
\vspace{-0.1in}

This is .

In the following section, we describe two energy source allocation policies, namely the Proportional Allocation (PA) and Equal Sharing (ES) policies. Note that, while the game described above is guaranteed to have at least one mixed-strategy NE, a mixed-strategy NE is not guaranteed to exist under a specific allocation policy. Therefore, Section \ref{sec:uncoordinated_extra_demand} focuses on examining the existence of a mixed-strategy NE under the considered allocation policies.

%By Eq. (\ref{eq:costs_symnonRES}), the expected cost of action nonRES, $ \upsilon_{nonRES, \vartheta_i}(p_0^{NE}, p_1^{NE})$, is insensitive to $N_0$ (or $N-N_0$), that is, the cardinality of the two sets of consumers. 
%In addition, in order to accommodate different risk attitudes, it becomes $c_{i}^{(\cdot)}(nonRES,p) = (\epsilon_{\vartheta_i}  E_{\vartheta_i})  c_{nonRES,N}$. 
%Likewise, by Definition \ref{def:energy_source_game}, the expected cost of action RES is given by $c_{i}^{(\cdot)}(RES,p) =\sum_{N_0=0}^{N}c_{i}^{N_0}(RES,p) B(N_0;N,r)$, where the cost $c_{i}^{N_0}(RES,p)$ follows equation (\ref{eq:costs_symRES}). 

%In the rest of the paper, we will apply the definition of $n$ of Eq. \eqref{eq:gen_costs_symRES}.


%As a result, the player $i$, with energy profile $\vartheta_i$, and mixed strategy $p_{\vartheta_i}=(p_{RES,\vartheta_i};p_{nonRES,\vartheta_i})$ will have an expected cost equal to
%\begin{align}
%   \upsilon_{ \vartheta_i}^{N_0}(p_0,p_1) = &  p_{RES,\vartheta_i} ~\upsilon_{RES, \vartheta_i}^{N_0}(p_0,p_1)  \nonumber \\&  + p_{nonRES,\vartheta_i} \upsilon_{nonRES, \vartheta_i}^{N_0}(p_0,p_1).\label{eq:costs_tot}
%\end{align}

\subsection{Energy Source Allocation Policies}\label{sec:policies}

The allocation of scarce resources to human-driven agents with distinct demands, is challenging. In this paper, we examine two types of allocations with different fairness properties, namely, the proportional allocation and the equal sharing.%The engagement of humans (behind the machines) transfuse social features and require careful consideration of the fairness properties of the resource allocation policy, as it impacts on users' satisfaction, participation, compliance pervasion, and ultimately, on system's efficiency \cite{Busqu12}.

%In multiplexing settings, the division of scarce resources motivates a number of challenges with respect to the properties that should characterize the allocation mechanism. Most commonly, the allocation mechanisms pursue an utilitarian view of the allocation (\ie aggregation of agents' utility on the allocated resources). However, practical circumstances that engage humans (and machines) transfuse social features to the allocation process and raise some concerns about the level of fairness that should vouch for. The fairness properties of an allocation mechanism have a direct impact on agents' satisfaction, participation, compliance pervasion, and ultimately, on system's efficiency \cite{Busqu12}.

\paragraph{Proportional Allocation}\label{sec:prop_alloc}

The proportional allocation allocates resources in proportion to the consumer demand. Under this policy, the amount of RES received  by a consumer of type $ \vartheta_i \in \Theta$ is given by,
% all users a level of service proportional to their actual needs. In turn, applying this allocation rule in the distribution of RES capacity, consumers that compete for RES capacity are endowed with an amount of energy that is given by 

\vspace{-0.1in}
\begin{small}
\begin{eqnarray}
rse^{PA}_{\vartheta_i}(\mathbf{n_i}) &=& \frac{E_{\vartheta_i}}{\max(\mathcal{ER} , D^{PA,NE} (\mathbf{p^{PA,NE}}))}\mathcal{ER},
\label{eq:prop_alloc_energy}
\end{eqnarray}
\end{small}
\vspace{-0.1in}

%where $n_0=n_{RES,0}$, $n_1=n_{RES,1}$ and 
\noindentwhere $ D^{PA,NE} (\mathbf{p^{PA,NE}})$ is given by \eqref{eq:expected_demand}. Eq. \eqref{eq:prop_alloc_energy} ensures that, if the available RES capacity $\mathcal{ER}$ is greater than the aggregate expected demand for RES, then each consumer $ i \in \mathcal{N}$ competing for RES receives a RES share equal to its energy demand profile $E_{\vartheta_i }$. Otherwise, it receives a share of the available capacity $\mathcal{ER}$ proportional to its energy profile. Note that in proportional allocation, the available RESs capacity is wasted only if $ D^{PA,NE} (\mathbf{p^{PA,NE}})\leq \mathcal{ER}$, otherwise, it is fully utilized. 

\paragraph{Equal Sharing}\label{sec:eq_sharing}

Under equal sharing, all consumers competing for RES in the day-zone will have access to exactly the same share value of RES capacity. Here, we call this quantity as the ``fair share" and is given by:

\begin{small}
\begin{align}
   sh(\mathbf{p^{ES,NE}}) 
=\frac{\mathcal{ER}}{ N \sum_{ \ell=0}^{M-1} r_{\ell}  p_{RES,\ell}^{ES,NE}} \label{eq:fairshare}
\end{align}
\end{small}
or equivalently,
\begin{small}
\begin{align}
   sh(\mathbf{n_i})
=\frac{\mathcal{ER}}{\sum_{\ell \in \Theta } n_{\ell ,i }+1}, \forall i \in \mathcal{N},
\end{align}
\end{small}
where the equivalence derives using Eq. \eqref{eq:gen_costs_symRES2}. However, if the energy demand of a user is smaller that the fair share, then the extra energy will remain unused. 
To express this, the amount of RES capacity granted to a consumer of type $ \vartheta_i \in \Theta$ that competes for RESs, is given by,
%benefits from accessing a particular service. In the energy front, all consumers that compete for RES capacity are endowed with an amount of energy that is given by


\begin{small}
\begin{eqnarray}
\hspace{-30pt} && rse^{ES}_{\vartheta_i}(\mathbf{n_i}) = \min\left( E_{\vartheta_i}, sh(\mathbf{n}_i)\right)
\label{eq:eq_sharing_energy}
\end{eqnarray}
\end{small}
As a result, in equal sharing part of the available RES capacity may remain unused, even if the total aggregate demand for RESs is higher than $\mathcal{ER}$ ($ D^{ES,NE} (\mathbf{p^{ES,NE}})\geq \mathcal{ER}$, ). This happens when the fair share (Eq. \eqref{eq:fairshare}) is greater than consumers demand. 

%where $n_0,n_1$ denote the number of competitors %of low-to-moderate and high energy profile \textit{but} player $i$, respectively, as also defined in Section 
%\ref{sec:prop_alloc}, and $n_0+n_1+1$ is the total number of competitors.
%where $\sum_{\ell \in \Theta } %n_{\ell , i}+1$ is the total number of consumers competing for RES,where $ n_{\ell , i}$ is explained above Eq. \eqref{eq:demand_RES}. 
%Taking the minimum ensures that, no consumer $ i \in \mathcal{N}$ receives a share of the available RESs capacity $\mathcal{ER}$ that is greater than its energy demand profile $E_{\vartheta_i}$.%, with $n_0,n_1$ as in Section \ref{sec:prop_alloc}.

Generally, policies for sharing resources that accommodate a level of fairness that prioritizes classes of users with specific characteristics may provide high efficiency (by prioritizing ``strong players'') but lack stability (``weak players'' may continuously change behavior/strategy to improve the acquired service). For instance, the proportional allocation may cause unstable operational states since it entails the risk of starvation of users with low energy demand, and eventually result in fewer satisfied players. On the other hand, equal sharing of resources favors stability since it prevents strong energy players (consumers with high demand) from obtaining more resources that any other player. Yet, equal sharing may result in inefficient and wasteful utilization of energy resources, since the ``fair share'' of energy resources may exceed the energy requirements of particular consumers of low energy demand.

%
%enabling varying notions of fairness
%derive the stable operational states under the two allocation policies in terms of the associated equilibrium strategies, assuming fully rational players, free of knowledge and computational constraints.% and emotional biases.
%


% and analyze the resulting dynamics under different allocation policies that conceptualize various levels of fairness in energy resource sharing. In all cases, we derive the stable operational conditions (\ie equilibria) and the associated costs incurred by the players, and assess their efficiency by comparing them with those under optimal coordinated energy source allocation. 



\subsection{Social Cost}\label{sec:socialcost}
The social cost at NE is computed differently for the proportional allocation and the equal sharing. 

In proportional allocation, the social cost at NE is given by:

\begin{small}
\begin{align}
C^{PA,NE}(\mathbf{p}^{PA,NE}) &=  \min(\mathcal{ER}, D^{PA,NE}(\mathbf{p^{PA,NE}})) c_{RES} \nonumber \\&+  \max(0,D^{PA,NE}(\mathbf{p^{PA,NE}})-\mathcal{ER})c_{nonRES,D} \nonumber \\
&+N \left[ \sum_{\ell \in \Theta} r_{\ell }p^{PA,NE}_{nonRES,\ell}\varepsilon_{\ell }E_{\ell}\right] c_{nonRES,N}.
 \label{eq:social_cost_pa_extra_demand}
 \end{align}
\end{small}

The first two summands refer to the consumers who played RESs.
The first summand is for the demand served by low-cost RES capacity during the day-zone. The second summand is for the demand served by high-cost peak-load production during day-zone. Finally, the third summand is for the consumers who played $nonRES$.



In the case of equal sharing, we need to account for the fact that part of the available RESs capacity may remain unused even if $ D^{ES,NE} (\mathbf{p^{ES,NE}})\geq \mathcal{ER}$. To do so, let us first define the quantity
\begin{small}$$d(\mathbf{p^{ES,NE}})= N \sum_{ \ell=0}^{M-1} r_{\ell} p_{RES,\ell}^{ES,NE}  \min(sh(\mathbf{p^{ES,NE}}), E_{\ell}),$$\end{small} which is equal to the part of the expected aggregate demand for RESs that is actually served by RESs.   
 
Then, we can define the social cost as:

\begin{small}
\begin{align}
C^{ES,NE}(\mathbf{p^{ES,NE}})& =  d(\mathbf{p^{ES,NE}}) c_{RES} \nonumber\\ &+(D^{ES,NE}(\mathbf{p^{ES,NE}}) -d(\mathbf{p^{ES,NE}}))c_{nonRES,D}\nonumber \\
&+N \left[ \sum_{\ell\in \Theta} r_{\ell}p^{ES,NE}_{nonRES,\ell}\varepsilon_{\ell }E_{\ell}\right] c_{nonRES,N}
 \label{eq:social_cost_es}
 \end{align}
\end{small}



Later, we will provide more explicit expressions for the expected aggregate demand and the social cost at NE depending on the available RES capacity, energy profiles and risk aversion parameters of all consumers.


In the following sections, we analytically and numerically study the distributed, uncoordinated energy selection game. We study the stability and efficiency of energy resource allocation by considering the two different notions of fairness above. Specifically, for each allocation policy we analytically derive its equilibrium states (stability) and we numerically compare the costs incurred by the consumers' strategies against the costs of an optimal centralized coordinated operation (efficiency).

