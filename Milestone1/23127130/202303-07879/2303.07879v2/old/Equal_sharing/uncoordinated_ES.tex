

\subsection{Equal Sharing}\label{sec:eq_sharing_eq_extra_demand}
The analysis of the mixed strategy equilibrium probabilities for the ES policy coincides with the one for the PA policy (Section \ref{sec:prop_alloc_eq_extra_demand}) for all cases, except for Cases $2$ and $4$. For the sake of concision, this analysis is detailed below only for those cases.

\subsubsection*{\textbf{Case $2$}} 
\hspace{10pt}
\vspace{5pt}

\textbf{Sub-case $2(a)$:} A mixed strategy equilibrium exists if and only if, for any consumers' types  $\vartheta_i \in \Theta$ and $ \vartheta_j \in \Theta \setminus \{\vartheta_i\}$, with energy profiles $E_{\vartheta_i} \neq E_{\vartheta_j}$, it holds that
\small
\begin{align}\label{eq:relation_E_0_E_1_es_ne_extra_demand}
(\gamma-\epsilon_{\vartheta_i}\beta)E_{\vartheta_i} =(\gamma-\epsilon_{\vartheta_j}\beta)E_{\vartheta_j}.
\end{align}
\normalsize
This condition is derived by equalizing the right-hand sides of Eqs. \eqref{eq:eq_sharing_energy} and \eqref{eq:conditionEQ_extra_demand}, for $\vartheta_i \in \Theta$ and $\vartheta_j \in \Theta \setminus \{\vartheta_i\}$ and obtaining:
\begin{small}
\begin{align}
\mathcal{ER}\frac{(\gamma-1)}{(\gamma-\epsilon_{\vartheta_i}\beta)E_{\vartheta_i}}-1= \sum_{ {\vartheta_\ell}\in \Theta} r_{\vartheta_\ell} (N-1) p^{ES,NE}_{RES,\vartheta_\ell},\label{eq:probrelationes1}\\
\mathcal{ER}\frac{(\gamma-1)}{(\gamma-\epsilon_{\vartheta_j}\beta)E_{\vartheta_j}}-1= \sum_{ {\vartheta_\ell}\in \Theta} r_{\vartheta_\ell} (N-1) p^{ES,NE}_{RES,\vartheta_\ell}.\label{eq:probrelationes2}
\end{align} 
\end{small}
As the right hand-sides of Eqs. \eqref{eq:probrelationes1}-\eqref{eq:probrelationes2} are equal, it derives that:
\begin{align}
\mathcal{ER}\frac{(\gamma-1)}{(\gamma-\epsilon_{\vartheta_i}\beta)E_{\vartheta_i}}-1= 
\mathcal{ER}\frac{(\gamma-1)}{(\gamma-\epsilon_{\vartheta_j}\beta)E_{\vartheta_j}}-1,
\end{align} 
which straightforwardly leads to Eq. \eqref{eq:relation_E_0_E_1_es_ne_extra_demand}. 

 %\emph{The energy demand profiles satisfy  $E_{\vartheta_i } \leq \mathcal{ER}\frac{(\gamma-1)}{(\gamma-\epsilon_{\vartheta_i }\beta)}$,  for all $\vartheta_i \in \Theta$.} %or $E_{\vartheta_i} > \mathcal{ER}\frac{(\gamma-1)}{(\gam
As a result, the competing probabilities, $p^{ES,NE}_{RES,{\vartheta_i }}$, for all $ \vartheta_i \in \Theta$ that lead to equilibrium states lie in the following range,
\footnotesize
\begin{align}
&  \Biggl[ \max \left\{0,\left[\mathcal{ER}\frac{(\gamma-1)}{(\gamma-\epsilon_{\vartheta_i}\beta)E_{\vartheta_i}}-1-
  \sum_{{\vartheta_j}\in \Theta \setminus \{\vartheta_i\}} r_{\vartheta_j} (N-1)\right]\cdot  \right. \nonumber \\&  \left. \frac{1}{r_{\vartheta_i}(N-1)}\right\},
     \min\left\{1,\left[\mathcal{ER}\frac{(\gamma-1)}{(\gamma-\epsilon_{\vartheta_i}\beta)E_{\vartheta_i}}-1\right]\frac{1}{r_{\vartheta_i}(N-1)}\right\}
    \Biggr].
    \label{eq:prop_alloc_eq_bounds_0_extra_demand}
\end{align}
\normalsize
Similarly to the PA policy, this range is derived using Eq. \eqref{eq:probrelationes1}.

\begin{remark}
Note that, similarly to the PA policy, from the condition of Eq. \eqref{eq:relation_E_0_E_1_es_ne_extra_demand}, the existence of NE is possible if and only if $\epsilon_0<\epsilon_1<\epsilon_2<..<\epsilon_{M-1}$. 
\end{remark}
\vspace{5pt}

%Then, we take extreme values of the other probabilities as before. 
%Note that the probabilities $p^{PA,NE}_{RES,\vartheta_i}$ should be chosen in the range \eqref{eq:prop_alloc_eq_bounds_0_extra_demand}, but, also under the constraint of \eqref{eq:probrelationes1}.


\textbf{Sub-case $2(b)$:} In this case, the analysis is identical to the one for the PA policy, in Section \ref{sec:prop_alloc_eq_extra_demand}. 
\vspace{5pt}
%\vspace{-5pt}
%\footnotesize
%\begin{align}
%  p^{PA,NE}_{RES,1}
%  \in
%   \notag\\\Biggl[ max\left(0,\left[\mathcal{ER}\frac{(\gamma-1)}{(\gamma-\epsilon_{1}\beta)}-E_1-r(N-1)E_0\right]\frac{1}{(1-r)(N-1)E_1}\right),
%   \notag\\
%     min\left(1,\left[\mathcal{ER}\frac{(\gamma-1)}{(\gamma-\epsilon_{1}\beta)}-E_1\right]\frac{1}{(1-r)(N-1)E_1}\right)
%    \Biggr].
%    \label{eq:prop_alloc_eq_bounds_1_extra_demand}
%\end{align}
%\normalsize


%The rest content of Sub-case (a) remains the same with Section \ref{sec:prop_alloc_eq_extra_demand}. 
%The expected aggregated demand for any $\vartheta_i \in \Theta$ is equal to,
%\vspace{-5pt}
%\footnotesize
%\begin{align}
%& D^{PA,NE}=\nonumber \\ & \min\{N  \sum_{\vartheta_j \in \Theta} r_{\vartheta_j }E_{\vartheta_j}, \max\{\left[\mathcal{ER}\frac{(\gamma-1)}{(\gamma-\epsilon_{\vartheta_i}\beta)E_{\vartheta_i}}-1\right]\frac{N}{(N-1)},0\}\}.
%    \label{eq:demand1}
%\end{align}
%\normalsize
%If $E_{\vartheta_i} > \mathcal{ER}\frac{(\gamma-1)}{(\gamma-\epsilon_{\vartheta_i}\beta)}$  for all $\vartheta_i \in \Theta$, it is dominant strategy for all consumers to defer from competing for $RES$, namely, $\upsilon_{RES, \vartheta_i}(p)>\upsilon_{nonRES, \vartheta_i}(p)$, $\forall p$ and $\forall i\in \mathcal{N}$. To show this, assume that $\vartheta_i=0$ and the allocated energy is $E'$. Then, in a large population regime $\upsilon_{RES, 0}(p)= E'  c_{RES}+(E_0-E') \gamma  c_{RES}$ and 
%$\upsilon_{nonRES, 0}(p)= \epsilon_0 E_0 \beta c_{RES}$. The inequality $\upsilon_{RES,0}(p)>\upsilon_{nonRES, 0}(p)$ is then equivalent to the inequality $E_0 >E' \frac{(\gamma-1)}{(\gamma-\epsilon_0\beta)}$ which is true by assumption and by the fact that the allocated energy $E'$ will be less than the available energy $\mathcal{ER}$. The same can be shown for the other $\vartheta_i\in \Theta$. Obviously, in this case, $D^{PA,NE}=0$.

\textbf{Sub-case $2(c)$:} %In this Sub-case Eq. \eqref{eq:relation_E_0_E_1_pa_ne_extra_demand} is not true.
For consumers' types in the set $\Sigma_1$, the analysis is identical to the one for the PA policy, in Section \ref{sec:prop_alloc_eq_extra_demand}.

For consumers' types in the set $\Sigma_2$, the mixed strategy NE is determined under the condition of Eq. \eqref{eq:relation_E_0_E_1_es_ne_extra_demand} involving only those consumers' types in $\Sigma_2$. 

Therefore, the competing probabilities $p^{ES,NE}_{RES,\vartheta_i}$ for all $\vartheta_i\in \Sigma_2$ that lead to equilibrium states lie in the range:%the particular conditions on pricing and energy demand levels and provided by 
%
%
%if their energy demand $E_{\vartheta_j}$ equals the threshold $\frac{(\gamma-1)}{(\gamma-\epsilon_{\vartheta_j}\beta)} rse^{PA}_{\vartheta_j}(\cdot)$, then the competing probability is 
\footnotesize
\begin{align}
&  \Biggl[ \max \left\{0,\left[\mathcal{ER}\frac{(\gamma-1)}{(\gamma-\epsilon_{\vartheta_i}\beta)}-E_{\vartheta_i}-
  \sum_{\vartheta_j\in \Sigma_2 \setminus \{\vartheta_i\}} r_{\vartheta_j} (N-1)E_{\vartheta_i}\right]\cdot  \right. \nonumber \\&  \left. \frac{1}{r_{\vartheta_i}(N-1)E_{\vartheta_i}}\right\},
     \min\left\{1,\left[\mathcal{ER}\frac{(\gamma-1)}{(\gamma-\epsilon_{\vartheta_i}\beta)}-E_{\vartheta_i}\right]\frac{1}{r_{\vartheta_i}(N-1)E_{\vartheta_i}}\right\}
    \Biggr].
    \label{eq:prop_alloc_es_bounds_0_extra_demandnew}
\end{align}
\normalsize
By analogy with the analysis for the PA policy, this range is derived by writing Eq. \eqref{eq:probrelationes1} for consumers in $\Sigma_2$, which gives:
 $$\sum_{\vartheta_l\in \Sigma_2} p^{ES,NE}_{RES,\vartheta_l} r_{\vartheta_l }(N-1)=\mathcal{ER}\frac{(\gamma-1)}{(\gamma-\epsilon_{\vartheta_i}\beta)E_{\vartheta_i}}-1$$.
 Then, \eqref{eq:prop_alloc_es_bounds_0_extra_demandnew} is derived by multiplying both sides of this equation with $E_{\vartheta_i}$ and taking the upper and lower bounds of the probabilities for all other consumers except $\theta_i \in \Sigma_2$.

\subsubsection*{\textbf{Case $4$}} 
For the consumers' types in the set $\Sigma_1$, the analysis is identical to the one for the PA policy, in Section \ref{sec:prop_alloc_eq_extra_demand}.

For the consumers' types in the set $\Sigma_2$, the mixed strategy NE is determined under condition \eqref{eq:relation_E_0_E_1_es_ne_extra_demand} involving only the those consumers' types in $\Sigma_2$. Besides, for consumers' types $\vartheta_i \in \Sigma_2$, the competing probabilities $p^{ES,NE}_{RES,\vartheta_i}$ that lead to equilibrium states lie in the range:
\small
\begin{align}
&  \Biggl[ \max \left\{0,\left[\mathcal{ER}\frac{(\gamma-1)}{(\gamma-\epsilon_{\vartheta_i}\beta)}-E_{\vartheta_i}-
  \sum_{\vartheta_j\in \Theta \setminus \{\vartheta_i\}} r_{\vartheta_j} (N-1)E_{\vartheta_i}\right]\cdot  \right. \nonumber \\&  \left. \frac{1}{r_{\vartheta_i}(N-1)E_{\vartheta_i}}\right\},
     \min\left\{1,\left[\mathcal{ER}\frac{(\gamma-1)}{(\gamma-\epsilon_{\vartheta_i}\beta)}-E_{\vartheta_i}- \right. \right. \nonumber \\ & \left. \left.
  \sum_{\vartheta_j\in \Sigma_1\setminus \{\vartheta_i\}} r_{\vartheta_j} (N-1)E_{\vartheta_i}\right]\frac{1}{r_{\vartheta_i}(N-1)E_{\vartheta_i}}\right\}
    \Biggr].
    \label{eq:prop_alloc_eq_bounds_0_extra_demandnew55}
\end{align}
\normalsize
This result is derived similarly to Eq. \eqref{eq:prop_alloc_es_bounds_0_extra_demandnew}, but considering that all consumers' types $\vartheta_i \in \Sigma_1$ compete for $RES$, i.e., that their competing probabilities that lead to equilibrium states are equal to $p^{ES,NE}_{RES,\vartheta_i}=1$.%, i.e., by equalizing the right-hand sides of Eqs. (\ref{eq:eq_sharing_energy}) and (\ref{eq:conditionEQ_extra_demand}) and considering that consumers with energy profile in $\Sigma_1$ compete for $RES$, i.e., $p^{ES,NE}_{RES,\vartheta_i}=1,~ \forall \vartheta_i \in \Sigma_1 $.% Then, we obtain the equality $\sum_{\vartheta_j\in \Sigma_2} p^{ES,NE}_{RES,\vartheta_j} r_{\vartheta_j }(N-1)+\sum_{\vartheta_j\in \Sigma_1} r_{\vartheta_j }(N-1)=\mathcal{ER}\frac{(\gamma-1)}{(\gamma-\epsilon_{\vartheta_i}\beta)E_{\vartheta_i}}-1$, and  Eq. \eqref{eq:prop_alloc_eq_bounds_0_extra_demandnew55} derives if multiplying both sides with $E_{\vartheta_i}$ and taking extreme values of the probabilities. %Note that based on Eq. \eqref{eq:equal_sharing_eq_j_extra_demand2}, if $E_{\vartheta_j}$ is strictly lower than $thr_d=\frac{(\gamma-1)}{(\gamma-\epsilon_{\vartheta_j}\beta)}rse^{PA}_{\vartheta_j}(.)$ then the dominant strategy is to compete for $RES$, else if $E_{\vartheta_j}> th_d$ the dominant strategy is to defer from competing for $RES$.




%with $E_0 \leq  \mathcal{ER}\frac{(\gamma-1)}{(\gamma-\epsilon_0\beta)}$ and $E_1 \leq  \mathcal{ER}\frac{(\gamma-1)}{(\gamma-\epsilon_1\beta)}$. Otherwise, if $E_0 > \mathcal{ER}\frac{(\gamma-1)}{(\gamma-\epsilon_0\beta)}$ and $E_1 > \mathcal{ER}\frac{(\gamma-1)}{(\gamma-\epsilon_1\beta)}$, it is dominant strategy for all consumers to defer from competing for $RES$, namely $c_{i}^{(\cdot)}(RES,p)>c_{i}^{(\cdot)}(nonRES,p)$, $\forall p$ and $\forall i\in \mathcal{N}$. Similarly, if $E_{\vartheta_i} > \mathcal{ER}\frac{(\gamma-1)}{(\gamma-\epsilon_{\vartheta_i}\beta)}$ and $E_{\vartheta_j} \leq \mathcal{ER}\frac{(\gamma-1)}{(\gamma-\epsilon_{\vartheta_j}\beta)}$, with $\vartheta_i\neq \vartheta_j$ and $i,j \in \mathcal{N}$, it is dominant strategy for all consumers of $\vartheta_i$ energy profile not to compete for $RES$. Regarding consumers of $\vartheta_j$ energy profile, they end up competing with probability $p^{ES,NE}_{RES,\vartheta_j}$ that follows (\ref{eq:prop_alloc_eq_j_extra_demand1}). %, if their energy demand $E_{\vartheta_j}$ equals the threshold $\frac{(\gamma-1)}{(\gamma-\epsilon_{\vartheta_j}\beta)} rse^{PA}_{\vartheta_j}(\cdot)$; otherwise, competing for $RES$ dominates reneging from RES, if $E_{\vartheta_j}$ is lower than this threshold value, and it is dominated by reneging from RES, if $E_{\vartheta_j}$ exceeds this threshold value.
%Furthermore, if $\epsilon_{\vartheta_i} \geq \gamma/\beta$, $\forall i\in \mathcal{N}$, it is dominant strategy for all consumer to compete for $RES$, since it holds that $c_{i}^{(\cdot)}(RES,p) < c_{i}^{(\cdot)}(nonRES,p)$, $\forall p$ and $\forall i\in \mathcal{N}$. Finally, if $\epsilon_{\vartheta_i} \geq \gamma/\beta$ and $\epsilon_{\vartheta_j} < \gamma/\beta$, with $\vartheta_i\neq \vartheta_j$ and $i,j \in \mathcal{N}$, then it is dominant strategy for all consumers of $\vartheta_i$ energy profile to compete, while all consumers of $\vartheta_j$ end up competing with $p^{PA,NE}_{RES,\vartheta_j}, ~0\leq p^{PA,NE}_{RES,\vartheta_j} \leq 1$,
%
%\vspace{-5pt}
%\footnotesize
%\begin{align}
%p^{PA,NE}_{RES,\vartheta_j}  = 
%   \left[\mathcal{ER}\frac{(\gamma-1)}{(\gamma-\epsilon_{\vartheta_j}\beta)}-E_{\vartheta_j}-[r+(1-2r)\vartheta_i](N-1)E_{\vartheta_j}\right] \notag \\
%    \times  \frac{1}{[r+(1-2r)\vartheta_j](N-1)E_{\vartheta_j}} 
% \label{eq:es_eq_j_extra_demand2}
%\end{align}
%\normalsize


%if $E_{\vartheta_j}$ equals threshold $\frac{(\gamma-1)}{(\gamma-\epsilon_{\vartheta_j}\beta)} rse^{PA}_{\vartheta_j}(\cdot)$. Otherwise, under lower energy demand $E_{\vartheta_j}$, the dominant strategy amounts to competing and under higher energy demand, they are motivated to resort to the safer option avoiding competition.

%It can be easily observed that the equal sharing and the proportional allocation policies may only cause different equilibrium states under the game variant where consumers risk only part of their maximum possible demand,i.e., with risk-conservative consumers. For instance the conditions of Eqs. \eqref{eq:relation_E_0_E_1_pa_ne_extra_demand}, \eqref{eq:relation_E_0_E_1_es_ne_extra_demand} coincide for risk-seeking consumers. %In particular, the equal sharing differentiates from proportional allocation under two cases: (a) if $\epsilon_{\vartheta_i}<\gamma/\beta$, $\forall i\in \mathcal{N}$, $E_0 \leq  \mathcal{ER}\frac{(\gamma-1)}{(\gamma-\epsilon_0\beta)}$ and $E_1 \leq  \mathcal{ER}\frac{(\gamma-1)}{(\gamma-\epsilon_1\beta)}$; (b) if $\epsilon_{\vartheta_i} \geq \gamma/\beta$, $\epsilon_{\vartheta_j} < \gamma/\beta$ and $E_{\vartheta_j}=\frac{(\gamma-1)}{(\gamma-\epsilon_{\vartheta_j}\beta)} rse^{PA}_{\vartheta_j}(\cdot)$ with $\vartheta_i\neq \vartheta_j$ and $i,j \in \mathcal{N}$. Thus, 
%The set of mixed strategy equilibrium probabilities under equal sharing is also presented in Table \ref{table_eq_comp_pro}, by (i) replacing the probabilities in the second line, fifth and sixth columns, with the expressions in (\ref{eq:equal_sharing_eq_bounds_0_extra_demand}), (\ref{eq:equal_sharing_eq_bounds_1_extra_demand}), replacing the probability in the seventh line, sixth column with Eq. (\ref{eq:equal_sharing_eq_j_extra_demand2}), and (iii) replacing the thresholds $thr_e$, $thr_f$ with $\frac{thr_c}{( r+(1-2r)\vartheta_i)(N-1)+(r+(1-2r)\vartheta_j)(N-1)+1}$ and $thr_d$, correspondingly.




%%%%%%%%%%%%%%%%%%%%%%%%%%%%%%%%%%%%%%%%%%%%%%%%%



In all cases where the PA and ES policies result in the same mixed strategy equilibrium probabilities, the corresponding expected aggregate demand for RES is the same under the two policies. 
Otherwise (i.e., for cases 2 and 4 described above), we can write the expected aggregate demand under the ES policy as follows:

\footnotesize
\begin{align}
& \hspace{-10pt}  D^{ES,NE}(\mathbf{p^{ES,NE}}) =N p^{ES,NE}_{RES,0} r_{0}E_{0}+N \sum_{\vartheta_j \in \Theta \setminus \{0\}} p^{ES,NE}_{RES,\vartheta_j} r_{\vartheta_j }E_{\vartheta_j} \nonumber \\
&=\Bigl (\mathcal{ER}\frac{(\gamma-1)}{(\gamma-\epsilon_{0}\beta)E_{0}}-1- \sum_{\vartheta_j \in \Theta \setminus \{0\}} p^{ES,NE}_{RES,\vartheta_j} r_{\vartheta_j } (N-1) \Bigr) \frac{N~E_0}{N-1} \nonumber \\ &+N \sum_{\vartheta_j \in \Theta \setminus \{0\}} p^{ES,NE}_{RES,\vartheta_j} r_{\vartheta_j }E_{\vartheta_j} \nonumber \\ &=D^{PA,NE}+N \sum_{ \vartheta_j \in \Theta \setminus \{0\}} p^{ES,NE}_{RES,\vartheta_j} r_{\vartheta_j }(E_{\vartheta_j} -E_0).
\label{eq:demand_es_ne_extra_demand}
\end{align}
\normalsize

%given in Table \ref{table_eq_comp_pro}. 
%Otherwise, %under the conditions of the second and seventh line of Table \ref{table_eq_comp_pro}, 
%the expected aggregate demand for renewable energy $D^{ES,NE}(p^{ES,NE})$ is an non-decreasing function of $p^{ES,NE}_{RES,1}$, that is,
%
%\vspace{-5pt}
%\footnotesize
%\begin{eqnarray}
%\hspace{-10pt}  && \hspace{-10pt}  D^{ES,NE}(p^{ES,NE}) =N \sum_{\vartheta_j\in \Theta} p^{ES,NE}_{RES,\vartheta_j} r_{\vartheta_j }E_{\vartheta_j}.
%\label{eq:demand_es_ne_extra_demand}
%\end{eqnarray}
%\normalsize
% \hspace{-10pt} &=& \hspace{-10pt}  \frac{N}{(N-1)} \left[ \mathcal{ER}\frac{(\gamma-1)}{(\gamma-\epsilon_{0}\beta)} -E_0\right] + (1-r)(E_1-E_0)N p^{ES,NE}_{RES,1}\nonumber\\
% \hspace{-10pt} &=& \hspace{-10pt} D^{PA,NE}(p^{PA,NE})+ (1-r)(E_1-E_0)N p^{ES,NE}_{RES,1}.



%The second line in Eq. \eqref{eq:demand_es_ne_extra_demand} derives by replacing in the first line that  $ p^{ES,NE}_{RES,0} =\frac{1}{r(N-1) } \Big[\mathcal{ER}\frac{(\gamma-1)}{(\gamma-\epsilon_0\beta)E_0}-1- (1-r) (N-1) p^{ES,NE}_{RES,1} \Big]$.
%Likewise, under the conditions of the sixth line of Table \ref{table_eq_comp_pro}, the expected total number of consumers that compete for $RES$ in equilibrium, equals $\frac{N}{(N-1)}  \left[ \frac{\mathcal{ER}}{E_{\vartheta_j}}\frac{(\gamma-1)}{(\gamma-\epsilon_{\vartheta_j}\beta)}-1 \right]$ and the expected aggregate demand for renewable energy $D^{ES,NE}(p^{ES,NE})$ is given by (\ref{eq:demand_es_ne_extra_demand}).

%Conditions under which equilibrium competing probabilities generate inefficiencies, that is, a number of consumers incur the price of the lack of coordination are described in Section \ref{sec:coordinated}.


%\textbf{Social Cost:}  The expected aggregate cost incurred by the entire population is a function of $p^{ES,NE}$ (contrary to the Eq. \eqref{eq:social_cost_pa}) and is given by,
%\vspace{-5pt}
%\footnotesize
%\begin{align}
%& C^{ES,NE}(p^{ES,NE})  \nonumber \\ &=  d(p^{ES,NE}) c_{RES} +(D^{ES,NE}-d(p^{ES,NE}))c_{nonRES,D} \nonumber \\
%&   + N\left[\sum_{\vartheta_j \in \Theta} r_{\vartheta_j }p^{ES,NE}_{nonRES,\vartheta_j}\epsilon_{\vartheta_j }E_{\vartheta_j}\right] c_{nonRES,N},
% \label{eq:social_cost_es_extra_demand}
% \end{align}
%\normalsize 


Finally, similarly to the proportionally-fair allocation, the expected aggregate social cost at NE takes an interesting form in the special case of risk-seeking consumers. Specifically, 
\footnotesize
\begin{align}
&C^{ES,NE}(\mathbf{p^{ES,NE}}) \nonumber\\ &=  d(\mathbf{p^{ES,NE}}) c_{RES} +(D^{ES,NE}(\mathbf{p^{ES,NE}}) -d(\mathbf{p^{ES,NE}}))c_{nonRES,D}\nonumber \\
&+ \hspace{-5pt} \left[ \left[  \sum_{ \vartheta_j=0}^{M-1} r_{\vartheta_j} N E_{\vartheta_j}\right]-D^{ES,NE}(\mathbf{p^{ES,NE}}) \right] c_{nonRES,N}.
 \label{eq:social_cost_es_sc}
 \end{align}
\normalsize
%with \footnotesize$d(p^{ES,NE})=min(sh(p^{ES,NE}), E_0)rNp_{RES,0}^{ES,NE}+min(sh(p^{ES,NE}), E_1)(1-r)Np_{RES,1}^{ES,NE}$ \normalsize the actual aggregate amount of RES  allocated to the consumers and \footnotesize$sh(p^{ES,NE}) =\frac{\mathcal{ER}}{rNp_{RES,0}^{ES,NE}+(1-r)Np_{RES,1}^{ES,NE}}$ \normalsize the fair share of RES disseminated to the consumers. 


%\vspace{-5pt}
%\footnotesize
%\begin{eqnarray}
%sh(p) \hspace{-5pt} &=& \hspace{-5pt} \frac{\mathcal{ER}}{rNp_{RES,0}+(1-r)Np_{RES,1}}
% \label{eq:sharing_es}
% \end{eqnarray}
%\normalsize





%\subsection{Social Cost}\label{sec:social_cost_extra_demand}
%
%In the game variant with risk-conservative consumers, the social cost becomes
%
%
%\vspace{-5pt}
%\footnotesize
%\begin{eqnarray}
%C(p) \hspace{-5pt} &=& \hspace{-5pt} min(\mathcal{ER}, D^{(\cdot)}(p)) c_{RES} +  max(0,D^{(\cdot)}(p)-\mathcal{ER})c_{nonRES,D} \nonumber \\
% \hspace{-42pt}  &&  \hspace{-42pt} + \left[ rNp_{nonRES,0}(\epsilon_{0}E_0) \hspace{-2pt} + \hspace{-2pt} (1-r)Np_{nonRES,1}(\epsilon_{1}E_1)\right] c_{nonRES,N}
% \label{eq:social_cost_pa_es_extra_demand}
% \end{eqnarray}
%\normalsize


%\vspace{-5pt}
%\footnotesize
%\begin{eqnarray}
%C(p) \hspace{-5pt} &=& \hspace{-5pt} min(\mathcal{ER}, D^{(\cdot)}(p)) c_{RES} +  max(0,D^{(\cdot)}(p)-\mathcal{ER})c_{nonRES,D} \nonumber \\
% \hspace{-5pt}   &+& \hspace{-5pt} \left[ rN\epsilon_{0}E_0 + (1-r)N\epsilon_{1}E_1-D^{(\cdot)}(p)\right] c_{nonRES,N}
% \label{eq:social_cost_pa_es_extra_demand}
% \end{eqnarray}
%\normalsize


