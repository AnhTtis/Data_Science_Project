\section{Centralized Coordinated energy source allocation}\label{sec:coordinated}

In this section, we study a centralized mechanism that coordinates how consumers' demand will be served (i.e., by which source) in order to minimize the social cost (i.e., Eqs. \eqref{eq:social_cost_pa_extra_demand} for proportional allocation and Eqs. \eqref{eq:social_cost_es} for equal sharing). Furthermore, we evaluate the side-effects that stem from the distributed, uncoordinated energy source selection as far as the social welfare is concerned via the Price of Anarchy (PoA) measure \cite{Koutsoupias09}. PoA is expressed as the ratio of the social cost in the worst-case equilibria over the optimal-minimum social cost of the centralized mechanism.
We study the PA mechanism analytically in most of the distinguished cases. For the ES mechanism, the centralized decisions for minimizing the social cost are harder to derive analytically and we will present numerical evaluations in Section \ref{sec:eval} using a software optimization tool such as Mathematica or MATLAB. The same is true for the PoA, which can be obtained as indicated in Section \ref{sec:poa}. %may be alleviated with control mechanisms that coordinate individual decisions towards more socially
%efficient choices. In the sequel, we describe two approaches that differ on the required infrastructure and the amount of control they impose upon the energy source allocation. Furthermore, the (in)efficiency of the equilibria action profiles as derived in Section \ref{sec:uncoordinated_extra_demand}, %Sections \ref{sec:uncoordinated_no_extra_demand} and \ref{sec:uncoordinated_extra_demand}, 
%resulting from the strategically selfish individual
%decisions, is assessed through the broadly used metric of the Price of Anarchy (\textit{PoA}) \cite{Koutsoupias09}. This metric
%expresses the ratio of the social cost in the worst-case equilibria over the optimal-minimum social cost as derived under the two coordination approaches.
%
%%\subsection{Centralized Coordination Mechanisms}\label{sec:opt_central_no_extra_demand}
%
%The side-effects in energy pursuit may be eliminated by introducing centralized mechanisms
%that undertake both the information processing and the decision-making tasks. These mechanisms coordinate the service of consumers' demand to ensure the optimal utilization of RES.%, on the one side, and the elimination of the congestion penalty, on the other side (Ref. Section \ref{sec:game}).

\subsubsection{Proportional Allocation}\label{sec:opt_pa_central_no_extra_demand}
 Let $p^{PA,OPT}_{RES,\theta_i} (p^{PA,OPT}_{nonRES,\theta_i})$ stand for the probability that users of energy profile $\theta_i \in \Theta$ opt (do not opt) for RES, as decided by the centralized coordinated mechanism, under the proportional allocation. $\mathbf{p^{PA,OPT}}$ is defined similarly to $\mathbf{p^{PA,NE}}$. The following cases are distinguished.

It is important to note that the centralized coordination mechanism will achieve a demand for RES equal to $D^{PA,OPT}=min(\sum_{\theta_j \in \Theta}r_{\theta_j}NE_{\theta_j},\mathcal{ER})$, in all considered cases. 

\textbf{Case $1$.} The RES capacity exceeds the aggregated consumer demand. Then, all consumers (for both cases of risk-seeking and risk-conservative) are directed to the RES and $p^{PA,OPT}_{RES,\theta_i}=1$, $\forall \theta_i \in \Theta$.

In all the rest cases, the aggregated consumer demand exceeds the RES capacity. 

\textbf{Case $2$.} The risk-aversion degrees satisfy $\epsilon_{\vartheta_i}=\epsilon'<\gamma/\beta$, $\forall i\in \mathcal{N}$. This case includes the risk-seeking consumers ($\gamma/\beta>1$). Then, $p^{PA,OPT}_{RES,\theta_i}$, for all  $\theta_i \in \Theta$ are chosen to match the RES demand to supply, that is, %$rNp^{PA,OPT}_{nonRES,0}E_0+(1-r)N p^{PA,OPT}_{nonRES,1}E_1=\mathcal{ER} $.

\vspace{-5pt}
\small
\begin{equation}
N \sum_{\theta_j \in \Theta}r_{\theta_j} E_{\theta_j}p^{PA,OPT}_{RES,\theta_j}=\mathcal{ER}.
\label{eq:optimal}
\end{equation}
\normalsize

The competing probabilities, $p^{PA,OPT}_{RES,\theta_i} $, lie in the range below, under the constraint of \eqref{eq:optimal},

\vspace{-5pt}
\footnotesize
\begin{align}
 \hspace{-5pt} 
 \Biggl[max\left\{0,\frac{\left[\mathcal{ER} -\sum_{\theta_j\neq \theta_i, \theta_j \in \Theta}r_{\theta_j}N E_{\theta_j}\right]}{r_{\theta_i}N E_{\theta_i}}\right\},
   %\notag\\
    % min\left(1,\mathcal{ER} \frac{1}{rNE_0}\right)
    min\left\{1,\frac{\mathcal{ER}}{r_{\theta_i}N E_{\theta_i}}\right\}
    \Biggr].
    \label{eq:prop_alloc_opt_bounds_0}
\end{align}
\normalsize


The cost due to the demand at night-rate is constant and equal to 
$\epsilon' N \sum_{\theta_j \in \Theta}r_{\theta_j}E_{\theta_j}p^{PA,OPT}_{nonRES,\theta_j} c_{noRES,N}=\epsilon'  ( N\sum_{\theta_j \in \Theta}r_{\theta_j}E_{\theta_j}- \mathcal{ER}) c_{noRES,N}$.

\textbf{Case $3$.} The different energy profiles are coupled with different degrees of risk-aversion, %, which is less than $\gamma/\beta$, %depart in their policy in adjusting their demand level to the estimated risk of an action, 
\ie $\exists$  $i,j\in \mathcal{N}: \gamma/\beta>\epsilon_{\vartheta_i} > \epsilon_{\vartheta_j}$. In this case, the night-rate social cost to be minimized is $Nc_{noRES,N} \sum_{\theta_j \in \Theta}\epsilon_{\theta_j}r_{\theta_j}E_{\theta_j}p^{PA,OPT}_{nonRES,\theta_j} $. The solution can be derived by a software optimization tool such as Mathematica or MATLAB. In addition, due to the fact that the RES demand should match its supply as in Case $2$,  the probabilities, $p^{PA,OPT}_{RES,\theta_i} $, lie in the ranges given in (\ref{eq:prop_alloc_opt_bounds_0}). 


%Therefore, it is minimized when consumers with energy profiles $\theta_j \in \Theta$, corresponding to higher values of $\epsilon_{\theta_j}r_{\theta_j}E_{\theta_j}$ are assigned lower values of $p^{PA,OPT}_{nonRES,\theta_j} $ and thus higher values of $p^{PA,OPT}_{RES,\theta_j} $ (i.e., they are prioritized for RES). In addition, due to the fact that the RES demand should match its supply as in Case $2$,  the probabilities lie in the ranges given in (\ref{eq:prop_alloc_opt_bounds_0}). 

To illustrate this result in a simple scenario, let us assume the case of two energy profiles and let us examine the expected cost at night-rate; $(\epsilon_0 r_0 N p^{PA,OPT}_{nonRES,0}E_0+\epsilon_1 r_1 N p^{PA,OPT}_{nonRES,1} E_1) c_{nonRES,N}$. Let us assume that $\epsilon_0> \epsilon_1$, then we can write $\epsilon_0= \epsilon_0'+\epsilon_1$, with $\epsilon_0'>0$. Thus, the expected cost at night-rate becomes $(\epsilon_0' r_0N p^{PA,OPT}_{nonRES,0}E_0+\epsilon_1 (r_0 N E_0+r_1 N E_1- \mathcal{ER}) ) c_{nonRES,N}$. The only controllable term is $\epsilon_0' r_0 N p^{PA,OPT}_{nonRES,0}E_0$ that is minimized when $ p^{PA,OPT}_{nonRES,0}$ is minimized, i.e., when its complementary $ p^{PA,OPT}_{RES,0}$ is maximized. Note that the expected cost due to use of RES will be always equal to $c_{RES} \mathcal{ER}$.

\textbf{Case $4$.} The risk-aversion degrees satisfy $\epsilon_{\vartheta_i} \geq \gamma/\beta$, $\forall i\in \mathcal{N}$. Then, the optimal centralized coordination mechanism would direct all consumers to RES, \ie $p^{PA,OPT}_{RES,\theta_i}=1$. This derives similarly to Case $3$ of Section \ref{sec:prop_alloc_eq_extra_demand}.
 
\textbf{Case $5$.} In this case, there exist two distinct sets of consumers' profiles $\Sigma_1$, $\Sigma_2$ with $\Sigma_1 \cap \Sigma_2= \emptyset$ and it holds that $\Biggl\{\epsilon_{\vartheta_i} \geq \gamma/\beta,~ \forall \vartheta_i \in \Sigma_1\Biggr\}$, and $\Biggl\{\epsilon_{\vartheta_i} < \gamma/\beta, ~ \forall \vartheta_i \in \Sigma_2\Biggr\}$. Then, it is dominant strategy for all consumers with energy profiles in the set $\Sigma_1$ to compete for RES, while the probability that consumers with energy profiles in the set $\Sigma_2$ compete for RES is shaped to totally utilize the remaining RES, i.e.,
%$[r+(1-2r)\vartheta_i]NE_{\vartheta_i}+[r+(1-2r)\vartheta_j]NE_{\vartheta_j}p^{PA,OPT}_{RES,\vartheta_j}=\mathcal{ER}$.

\vspace{-5pt}
\footnotesize
\begin{align}
&p^{PA,OPT}_{RES,\vartheta_i}=1, \forall \vartheta_i \in \Sigma_1; \notag \\
&p^{PA,OPT}_{RES,\vartheta_i}\in \nonumber \\
&\Biggl[\max\left\{0,\frac{\mathcal{ER}-\sum_{\theta_i \neq \theta_j, \theta_j \in \Theta}r_{\theta_j}N E_{\theta_j}}{r_{\theta_i}N E_{\theta_i}}\right\}, \nonumber \\ & \min\left\{1,\frac{\mathcal{ER}-\sum_{ \theta_j \in \Sigma_1}r_{\theta_j}N E_{\theta_j}}{r_{\theta_i}N E_{\theta_i}}\right\}  \Biggr],~ \forall \vartheta_i \in \Sigma_2.
\label{eq:prop_alloc_opt_p_cond}
\end{align}
\normalsize

Note that the probabilities of $\Sigma_2$ should be chosen under Eq. \eqref{eq:optimal}. However, the probability assignment to the energy profiles in the set $\Sigma_2$ for minimizing the social cost (i.e., equivalently the night rate cost because the demand in the day will be equal to $\mathcal{ER}$) should be done with the help of a software optimization tool. An exception is the case where $|\Sigma_2|=1$. In this special case, for $\vartheta_i \in \Sigma_2$, $p^{PA,OPT}_{RES,\vartheta_i}=\max\left\{0, \min\left\{1,\frac{\mathcal{ER}-\sum_{ \theta_j \in \Sigma_1}r_{\theta_j}N E_{\theta_j}}{r_{\theta_i}N E_{\theta_i}}\right\}\right\}$.


\textbf{Social Cost:} It is given as follows.

\begin{small}
\begin{align}
&C^{PA,OPT}(\mathbf{p^{PA,OPT}})= D^{PA,OPT} c_{RES}  \nonumber \\ 
&+Nc_{noRES,N} \sum_{\theta_j \in \Theta}\epsilon_{\theta_j}r_{\theta_j}E_{\theta_j}p^{PA,OPT}_{nonRES,\theta_j}.
 \label{eq:social_cost_pa_op1t}
 \end{align}
\end{small}


In the special case of risk-seeking consumers the social cost has a constant value with respect to the decision probabilities:
\begin{small}
\begin{align}
&C^{PA,OPT}= D^{PA,OPT} c_{RES}  \nonumber \\ 
&+ c_{noRES,N} \Biggl[ N\sum_{\theta_j \in \Theta} r_{\theta_j}E_{\theta_j}- D^{PA,OPT} \Biggr].
 \label{eq:social_cost_pa_opt2}
 \end{align}
\end{small}


\subsubsection{Price of Anarchy:} \label{sec:poa}
The (in)efficiency of equilibrium strategies is quantified by the Price of Anarchy (PoA). In order to compute the PoA for each particular case (where cases are distinguished with respect to the risk-aversion degree values and energy consumption values), we should compute the mixed strategy equilibrium that leads to the worst social cost for this case. PoA is computed as follows: %with respect to the consumers behavior (e.g., risk-seeking and risk-conservative), risk-aversion degree values and energy consumption values $E_0$, $E_1$.

\vspace{-5pt}
\small
\begin{align}
 \hspace{-5pt} \textit{PoA}^{PA}  = \frac{C^{PA,NE}_w}{C^{PA,OPT}(\mathbf{p^{PA,OPT}})},
\label{eq:poa_pa}
\end{align}
\normalsize
where the $C^{PA,NE}_w$ is the worst case social cost of the distributed, uncoordinated mechanism for the case of interest. Note that $C^{PA,OPT}(\mathbf{p^{PA,OPT}})$ is uniquely determined for each particular case.

In order to obtain $C^{PA,NE}_w$, we can solve a simple optimization problem to minimize the social cost $C^{PA,NE}(\mathbf{p^{PA,NE}})$ with respect to $\mathbf{p^{PA,NE}}$ (when there are multiple possible NE e.g., cases 2(a), 2(b) and 4 and subject to the corresponding probability constraints using a software optimization tool. Importantly, for the uncoordinated energy selection under PA, minimizing $C^{PA,NE}(\mathbf{p^{PA,NE}})$  is equivalent to minimizing the night-zone cost. The last observation is due to the fact that the demand $D^{PA,NE}$ is constant with respect to $\mathbf{p^{PA,NE}}$. 

In the special case of the risk-seeking consumers, from Eq. \eqref{eq:social_cost_pa_sc}, the social cost is constant with respect to the probabilities $\mathbf{p^{PA,NE}}$ for each case of energy profile values. In this case, $C^{PA,NE}_w$ is given by Eq. \eqref{eq:social_cost_pa_sc}.

For the risk-conservative consumers, assume the simple case of two energy profiles. Then, in all cases except subcase 2(a), the social cost is uniquely determined giving directly the value of $C^{PA,NE}_w$. For subcase 2(a)  the multiple combinations of probabilities can lead to NE with possibly different social cost values. Under the conditions of subcase (a), we can show that $\epsilon_1>\epsilon_0$. Thus, in this case the social cost is decreasing with $p_{RES,1}^{PA,NE}$  more than with $p_{RES,0}^{PA,NE}$, and its worst case value, $C^{PA,NE}_w$, is obtained using the maximum value of $p_{RES,0}^{PA,NE}$ and minimum of $p_{RES,1}^{PA,NE}$ from the range in Eq. \eqref{eq:prop_alloc_pa_bounds_0_extra_demand}.

For the ES allocation, $\textit{PoA}^{ES}$ is defined in a similar way with $\textit{PoA}^{PA}$ and we study it numerically in Section \ref{sec:eval}.
%\subsubsection{Equal Sharing}\label{sec:opt_es_central_no_extra_demand}
%In this section, we examine the decisions of a centralized coordination mechanism for the probabilities that users of each energy profile will opt for RES, denoted as $p^{ES,OPT}_{RES,\theta_i}$, $\theta_i \in \Theta$, under equal sharing.
%
%%Under an ideal optimal centralized coordination mechanism that equally allocates the available RES capacity to competing consumers, the derivation of the associated competing probabilities $p^{ES,OPT}_{RES,0}$ and $p^{ES,OPT}_{RES,1}$ factors in two distinct cases: (a) $E_0=E_1=E$ and (b) $E_0 <  E_1$. In the first case, 
%If $E_0=E_1=E$ the two renewable energy allocation policies are equivalent and therefore, the decisions of the centralized mechanism are the same as those in Section \ref{sec:opt_pa_central_no_extra_demand}. 
%
%Differently, if $E_0 <  E_1$ we distinguish the following cases: %the centrally computed probabilities that optimize the social cost (Ref. eq. (\ref{eq:social_cost_es_extra_demand})) are described by Table \ref{tb:opt_es_extra}.
%
%
%%\begin{table}[]
%%\centering
%%\caption{Probabilities under centralized coordination and equal sharing.}
%%\footnotesize
%%\label{tb:opt_es_extra}
%%\begin{tabular}{lcl}
%% $p^{ES,OPT}_{RES,0}$ \hspace{150pt}  $p^{ES,OPT}_{RES,1}$\\ \hline
%%  0 \hspace{165pt}   $min\left(1, \frac{\mathcal{ER}}{(1-r)NE_1}\right) $ \\ 
%% $min\left(1, \frac{\mathcal{ER}}{rNE_0}\right) $ \hspace{145pt}   0\\ 
%% 1 \hspace{70pt}   $min\left(1,max\left(0,\frac{1}{(1-r)}\left[\sqrt{\frac{\mathcal{ER}r(\gamma-1)}{E_1N(\gamma-\epsilon_1\beta)}} - r\right] \right)\right)$\\
%% 1 \hspace{105pt}  $min\left(1,max\left(0, \frac{1}{(1-r)}\left(\frac{\mathcal{ER}}{E_0N}-r\right)\right)\right)$\\
%% $min\left(1,max\left(0, \frac{1}{rN}\left[\frac{\mathcal{ER}}{E_1}-(1-r)N\right]\right)\right)$ \hspace{50pt}   1\\  
%%  $min\left(1,max\left(0, \frac{1}{rN}\left[\frac{\mathcal{ER}}{E_0}-(1-r)N\right]\right)\right)$ \hspace{50pt}   1\\  
%%\end{tabular}
%%\vspace{-5pt}
%%\end{table}
%%\normalsize
%
%
%\textbf{Case $1$.} The RES capacity exceeds the aggregated consumer demand. Then, all consumers (for both cases of risk-seeking and risk-conservative) are directed to the RES and $p^{ES,OPT}_{RES,\theta_i}=1$, $\forall~\theta_i \in \Theta$.
%
%\textbf{Case $2$.} Each one of the quantities $rN E_0$ and $(1-r) N E_1$ exceed the RES capacity. Then, (i) if $\epsilon_i \geq \frac{\gamma}{\beta}, \forall i \in \mathcal{N}$, $p^{ES,OPT}_{RES,0}=1$, $p^{ES,OPT}_{RES,1}=1$ (shown as in Case $3$, Section \ref{sec:prop_alloc_eq_extra_demand}) and (ii) if $\gamma> min(\epsilon_0\beta, \epsilon_1\beta)$ then all consumers competing for RES should have the same energy profile. Thus, the renewable energy will not remain unused due to the equal sharing allocation. % (by providing energy shares that exceed low-to-moderate energy demand, $E_0$) while consumers incur congestion penalties (by providing energy shares not sufficient to respond to their energy demand).
%For case (ii), the centralized coordination mechanism allocates the RES exclusively to consumers of energy profile, $\vartheta_i$, coupled with the highest risk-aversion degree, i.e., $\epsilon_{\vartheta_i}>\epsilon_{\vartheta_j}$, $i,j \in \mathcal{N}$. This is shown in a similar way as in Case $3$ of Section \ref{sec:opt_pa_central_no_extra_demand} and considering that the demand for RES will be equal to $\mathcal{ER}$ and will be equal to the aggregate of the allocated RES to all consumers.%, namely, to consumers with the highest cost for reneging from competition, $\epsilon_{\vartheta_i}\beta$.
% For instance, if $\epsilon_0\geq\epsilon_1$ the competing probabilities under optimal allocation are $p^{ES,OPT}_{RES,1}=0$ and $p^{ES,OPT}_{RES,0}=min\left(1, \frac{\mathcal{ER}}{rNE_0}\right)$, if $\gamma >\epsilon_0\beta$, or $p^{ES,OPT}_{RES,0}=1$, otherwise. Similarly, if $\epsilon_1\geq\epsilon_0$ the competing probabilities under optimal allocation are $p^{ES,OPT}_{RES,0}=0$ and $p^{ES,OPT}_{RES,1}<min\left(1, \frac{\mathcal{ER}}{(1-r)NE_1}\right) $, if $\gamma > \epsilon_1\beta$, or $p^{ES,OPT}_{RES,1}=1$, otherwise.
%
%\textbf{Case $3$.} Each one of the quantities $rN E_0$ and $(1-r) N E_1$ by itself does not exceed the RES capacity. Then, if also $\epsilon_i \geq \frac{\gamma}{\beta}, \forall i \in \mathcal{N}$, then $p^{ES,OPT}_{RES,0}=1$, $p^{ES,OPT}_{RES,1}=1$ (shown as in Case $3$, Section \ref{sec:prop_alloc_eq_extra_demand}). However, the under other conditions, the solution is more complicated and can be derived by a tool such as Mathematica or MATLAB.
%
%%Under lower aggregate demand levels, the coordination mechanism allows serving with RES heterogeneous energy consumption profiles, taking into consideration first the degree of risk-aversion of the two classes of consumers  and second the aggregate demand shaped by each class. The resulting competing probabilities are drawn from Table \ref{tb:opt_es_extra}, starting from $p^{ES,OPT}_{RES,0}=p^{ES,OPT}_{RES,1}=1$ for low aggregate demand and converging to the previously cited values when both the aggregate demands shaped by low-to-moderate and high energy consumers exceed the RES capacity.
%
%The Price of Anarchy derives in a similar fashion as in Section \ref{sec:opt_pa_central_no_extra_demand} and we omit the details due to lack of space. Notice that for the uncoordinated scheme, the social cost is uniquely determined for all cases with respect to the consumers behavior (e.g., risk-seeking and risk-conservative), risk-aversion degree values and energy consumption values $E_0$, $E_1$ except the following one. Specifically, for the risk-conservative consumers, the social cost depends on the choice of $p^{ES,NE}$ (Eq. \eqref{eq:social_cost_es_extra_demand}). The latter dependence is of interest under the conditions of the first line of  Table \ref{table_eq_comp_pro}, since it is the only case where there exist multiple combinations of probabilities that can lead to equilibria. We can obtain the worst social cost by  comparing the cases (i) maximum value of $p_{RES,0}^{ES,NE}$ and minimum of $p_{RES,1}^{ES,NE}$ (from the ranges in Eqs. \eqref{eq:prop_alloc_eq_bounds_0_extra_demand}, \eqref{eq:prop_alloc_eq_bounds_1_extra_demand}), (ii) minimum value of $p_{RES,0}^{ES,NE}$ and maximum of $p_{RES,1}^{ES,NE}$, choosing the one that leads to the highest social cost.
%
%%\emph{Price of Anarchy:} %Concerning the (in)efficiency of equilibria, 
%%By Section \ref{sec:eq_sharing_eq_extra_demand}, under risk-seeking consumers, the equilibrium states are possible if $E_0=E_1$ which returns $\textit{PoA}^{ES}=\textit{PoA}^{PA}$. On the other hand, under risk-conservative consumers,  %equilibrium states are possible also under asymmetric energy consumption profiles. In this case, 
%%the social cost is a non-increasing function of the competing probability of the competitors of high energy demand (\ie $\frac{d C(p^{ES,NE})}{d p^{ES,NE}_{RES,1}}\leq 0$) and therefore the worst-case equilibrium occurs at $p'$ with $p'_{RES,1}=min(p^{ES,NE}_{RES,1})$ ($p'_{RES,0}=max (p^{ES,NE}_{RES,0})$). By (\ref{eq:social_cost_es_extra_demand}), the associated PoA is 
%%
%%
%%\vspace{-5pt}
%%\small
%%\begin{eqnarray}
%% \hspace{-5pt} \textit{PoA}^{ES}  &=& \frac{C(p')}{C(p^{ES,OPT})}
%%\label{eq:poa_es}
%%\end{eqnarray}
%%\normalsize
%
%%where both the social costs $C(p')$ and $C(p^{ES,OPT})$ are derived by (\ref{eq:social_cost_es_extra_demand}).
%
%%\subsection{Decentralized Coordination Mechanisms}\label{sec:opt_decentral_no_extra_demand}
%
%%Although the centralized coordination mechanisms ensure optimal efficiency in the energy source allocation, they need to pursue incentives that would engage the autonomous and selfish consumers into socially efficient actions. Otherwise, supervisory implementations need to vouch for system stability and robustness, since selfish behaviors can undermine its performance significantly. The analysis in Section \ref{sec:uncoordinated_extra_demand} suggests that two important factors \emph{under the control} of micro-grid operator(s), affect the game equilibrium competing probabilities. These factors are the renewable energy capacity, $\mathcal{ER}$, and the applied pricing scheme on the two energy sources. Typically, in the long-term, the operator copes with the energy demand and the emerging inefficiencies through strategic planning of $\mathcal{ER}$, involving re-dimensioning of the distribution network side (\eg solar panels or wind turbines). Whereas, in the short-term, it can do so by manipulating the prices charged for the two types of energy sources. In the sequel, we systematically study conditions on $\mathcal{ER}$ and pricing that optimize the distributed energy source selection in large-scale settings where both the aggregate demands shaped by low-to-moderate and high energy consumers exceed $\mathcal{ER}$.
%%
%%\subsubsection{Proportional Allocation}\label{sec:opt_pa_decentral_no_extra_demand}
%%
%%%On the one hand, by Sections \ref{sec:prop_alloc_eq}, under proportional allocation the competing influences are balanced if $rse^{NE}_{\vartheta_i}(\cdot)=rse^{PA}_{\vartheta_i}(\cdot)$. On the other hand, by equation (\ref{eq:optimal}), optimal utilization of renewable energy along with elimination of congestion penalties are possible if the renewable energy demand matches the supply.
%%%
%%%Overall, under the case of risk-seeking consumers, by equations (\ref{eq:prop_alloc_energy}), (\ref{eq:conditionEQ}) and (\ref{eq:optimal}) (alternatively by (\ref{eq:demand_pa_ne_no_extra_demand})), the uncoordinated energy source selection converges to the optimal coordinated energy source allocation only if $E_0=E_1=E$ and the renewable energy capacity is dimensioned according to
%%%
%%%\vspace{-5pt}
%%%\small
%%%\begin{equation}
%%%\mathcal{ER}^{OPT}=\frac{E \cdot N}{\frac{(\gamma-1)N}{(\gamma-\beta)}-(N-1)}
%%%\label{eq:optimal_ER_proportional_no_extra_demand}
%%%\end{equation}
%%%\normalsize
%%%
%%%or the charging costs satisfy
%%%
%%%\vspace{-5pt}
%%%\small
%%%\begin{equation}
%%%\frac{(\gamma-1)}{(\gamma-\beta)}\vert ^{OPT}=\frac{N-1}{N}+\frac{E}{\mathcal{ER}}
%%%\label{eq:optimal_pricing_proportional_no_extra_demand}
%%%\end{equation}
%%%\normalsize
%%
%%%Under the case of risk-conservative consumers, by equations (\ref{eq:prop_alloc_energy}), (\ref{eq:conditionEQ_extra_demand}) and (\ref{eq:optimal}) (alternatively by (\ref{eq:demand_pa_ne_extra_demand})), the uncoordinated energy source selection converges to the optimal coordinated energy source allocation if the renewable energy capacity is dimensioned according to
%%
%%By Section \ref{sec:opt_pa_central_no_extra_demand}, if $\epsilon_{\vartheta_i}<\gamma/\beta$, $\forall i\in \mathcal{N}$, optimal RES utilization along with elimination of congestion penalties are possible if the renewable energy demand matches the supply. Jointly by Table \ref{table_eq_comp_pro} and Section \ref{sec:opt_pa_central_no_extra_demand}, the optimization of the uncoordinated energy source selection requires dimensioning and pricing according to
%%
%%\vspace{-5pt}
%%\small
%%\begin{equation}
%%\mathcal{ER}^{OPT}=\frac{E_{\vartheta_i}N}{\frac{(\gamma-1)N}{(\gamma-\epsilon_{\vartheta_i}\beta) }-(N-1)} >0
%%\label{eq:optimal_ER_proportional_extra_demand}
%%\end{equation}
%%\normalsize
%%
%%%or the charging costs satisfy
%%
%%\vspace{-5pt}
%%\small
%%\begin{equation}
%%\frac{(\gamma-1)}{(\gamma-\epsilon_{\vartheta_i}\beta)}\vert ^{OPT}=E_{\vartheta_i}\left[\frac{N-1}{N}+\frac{1}{\mathcal{ER}}\right] >1
%%\label{eq:optimal_pricing_proportional_extra_demand}
%%\end{equation}
%%\normalsize
%%
%%with $\epsilon_{\vartheta_i}>\epsilon_{\vartheta_j}$, $i,j \in \mathcal{N}$. Namely, the distributed energy source selection is optimized by allocating the RES only to consumers with the highest cost for reneging from competition, $\epsilon_{\vartheta_i}\beta$. Likewise, in the symmetric case where $\epsilon_{\vartheta_i}=\epsilon'<\gamma/\beta$, $\forall i\in \mathcal{N}$, by Table \ref{table_eq_comp_pro}, (\ref{eq:prop_alloc_opt_bounds_0}) and  (\ref{eq:prop_alloc_opt_bounds_1}), the values for $\mathcal{ER}$ and charging costs that optimize the efficiency of equilibria are also given by (\ref{eq:optimal_ER_proportional_extra_demand}) and (\ref{eq:optimal_pricing_proportional_extra_demand}) with $\vartheta_i=0$, when $E_0 \leq \mathcal{ER}\frac{(\gamma-1)}{(\gamma-\epsilon_0\beta)}$ and $E_1 \leq \mathcal{ER}\frac{(\gamma-1)}{(\gamma-\epsilon_1\beta)}$ or $E_1 > max(\mathcal{ER}\frac{(\gamma-1)}{(\gamma-\epsilon_1\beta)}, \frac{E_0}{1-\frac{(N-1)}{N}\frac{(\gamma-\epsilon_0\beta)}{(\gamma-1)}})$. 
%%
%%Furthermore, in settings where $\exists i \in \mathcal{N} : \epsilon_{\vartheta_i}\geq\gamma/\beta$, by Table \ref{table_eq_comp_pro} and (\ref{eq:prop_alloc_opt_p_cond}), the optimal distributed energy selection is possible by prioritizing the use of RES facility by consumers that incur the highest cost for reneging from competition, $\epsilon_{\vartheta_i}\beta$, yet not blocking the other consumption profile from the distribution of the remaining RES capacity. In this case the optimal values for the renewable energy capacity and the charging costs are determined according to (\ref{eq:optimal_ER_proportional_extra_demand}) and (\ref{eq:optimal_pricing_proportional_extra_demand}) with $\epsilon_{\vartheta_i}<\epsilon_{\vartheta_j}$, $i,j \in \mathcal{N}$. If $\epsilon_{\vartheta_i}\geq\gamma/\beta$, $\forall i\in \mathcal{N}$, the dominant strategy for all consumer to compete for the renewable energy source is also the optimal one.
%%
%%The optimization of the distributed energy source selection by adopting particular values for the RES capacity and the charging costs, requires that both the equilibrium states and the optimal operation prioritize the same class(ses) of consumers in accessing the RES facility: if $E_0 \leq \mathcal{ER}\frac{(\gamma-1)}{(\gamma-\epsilon_0\beta)}$, $E_1 > \mathcal{ER}\frac{(\gamma-1)}{(\gamma-\epsilon_1\beta)}$ and $\epsilon_0<\epsilon_1<\gamma/\beta$ or if 
%%$E_{\vartheta_i} > \mathcal{ER}\frac{(\gamma-1)}{(\gamma-\epsilon_{\vartheta_i}\beta)}$, $\forall i\in \mathcal{N}$, the price of the lack of coordination in the equilibria cannot be avoided by the decentralized mechanism. Overall, the use of optimal values for the RES capacity and the charging costs yields $\textit{PoA}^{PA} =1$, with the exception of settings where there is a unique optimal competing probability vector while the equilibrium states vary over a proper probability interval (\ie  when $E_0 \leq \mathcal{ER}\frac{(\gamma-1)}{(\gamma-\epsilon_0\beta)}$,  $E_1 \leq \mathcal{ER}\frac{(\gamma-1)}{(\gamma-\epsilon_1\beta)}$, $\epsilon_0<\epsilon_1$). In this case, the worst-case equilibrium occurs at $p'$ with $p'_{RES,1}=min(p^{PA,NE}_{RES,1})$ ($p'_{RES,0}=max (p^{PA,NE}_{RES,0})$) and the associated PoA is given by (\ref{eq:poa_pa}).
%%
%%%while they do not break condition (\ref{eq:relation_E_0_E_1_pa_ne_extra_demand}) that dictates the relation between $E_0$ and $E_1$ factoring in the renewable energy capacity and the charging costs. 
%%
%%%Ultimately, these values for the renewable energy capacity and charging costs optimize the efficiency of equilibria, yielding $\textit{PoA}^{PA} =1$.
%%
%%
%%\subsubsection{Equal Sharing}\label{sec:opt_es_decentral_no_extra_demand}
%%
%%By Section \ref{sec:eq_sharing_eq_extra_demand}, in the distributed energy source selection by risk-seeking consumers the equilibrium states under equal sharing and proportional allocation coincide, and hence the RES capacity and the charging costs that optimize the efficiency of the uncoordinated selection process ($\textit{PoA}^{ES} =1$) are given by equations (\ref{eq:optimal_ER_proportional_extra_demand}) and (\ref{eq:optimal_pricing_proportional_extra_demand}).
%%%By Section \ref{sec:eq_sharing_eq_extra_demand}, 
%%When the consumers are risk-conservative, a trivial but expensive methodology to optimize the distributed energy source selection amounts to increasing $\mathcal{ER}$ to guarantee that the faire shares suffice to serve all requests, \ie $\mathcal{ER}/N \geq E_1$. Alternatively, in settings with $\epsilon_{\vartheta_i}<\gamma/\beta$, $\forall i\in \mathcal{N}$, by Tables \ref{table_eq_comp_pro} and \ref{tb:opt_es_extra}, the decentralized optimization of energy source selection is realized under the same conditions as in proportional allocation (Ref. Section \ref{sec:opt_pa_decentral_no_extra_demand}).
%%
%%Otherwise, in settings where $\exists \epsilon_{\vartheta_i}, \epsilon_{\vartheta_j}:\epsilon_{\vartheta_i}\geq\gamma/\beta> \epsilon_{\vartheta_j}$, $i,j \in \mathcal{N}$, by Tables \ref{table_eq_comp_pro} and \ref{tb:opt_es_extra},  
%%
%%\vspace{-5pt}
%%\small
%%\begin{equation}
%%\mathcal{ER}^{OPT}=\frac{N}{\frac{(\gamma-1)N}{(\gamma-\epsilon_{\vartheta_j}\beta) E_{\vartheta_j}}-\frac{(N-1)}{E_{\vartheta_i}}} > 0
%%\label{eq:optimal_ER_ES_extra_demand1}
%%\end{equation}
%%\normalsize
%%
%%\vspace{-5pt}
%%\small
%%\begin{equation}
%%\frac{(\gamma-1)}{(\gamma-\epsilon_{\vartheta_j}\beta)}\vert ^{OPT}=E_{\vartheta_j}\left[\frac{N-1}{NE_{\vartheta_i}}+\frac{1}{\mathcal{ER}}\right] > 1 
%%\label{eq:optimal_pricing_ES_extra_demand1}
%%\end{equation}
%%\normalsize
%%
%%which return $\textit{PoA}^{ES} =1$. In addition, by Table \ref{table_eq_comp_pro} and Section \ref{sec:opt_es_central_no_extra_demand}, if $\epsilon_{\vartheta_i}\geq\gamma/\beta$, $\forall i\in \mathcal{N}$, the dominant competing strategy equals to the optimal one.
%%
%%In a last note, the benefit of these decentralized coordination approaches can be also quantified by the Price of Stability (\textit{PoS}) which is a measure of inefficiency designed to differentiate between games in which \textit{all} equilibria are inefficient and those in which \textit{some} equilibria are inefficient \cite{Nisan07}. It is a ratio between the best objective function value at equilibrium and that of the optimal outcome. In the energy source selection game, with parametrization according to (\ref{eq:optimal_ER_proportional_extra_demand}) and (\ref{eq:optimal_pricing_proportional_extra_demand}) or (\ref{eq:optimal_ER_ES_extra_demand1}) and (\ref{eq:optimal_pricing_ES_extra_demand1}), the best-case equilibrium emerges at the optimal competing probabilities. Therefore, these decentralized coordination approaches guarantee $\textit{PoS}=1$.
%%
%% %emerges at $p'$ with $p'_{RES,1}=supremum(p^{ES,NE}_{RES,1})$ ($p'_{RES,0}=infimum (p^{ES,NE}_{RES,0})$) or $p'_{RES,1}=infimum(p^{ES,NE}_{RES,1})$ ($p'_{RES,0}=supremum (p^{ES,NE}_{RES,0})$), respectively. Therefore, these decentralized coordination approaches guarantee $\textit{PoS}=1$.
%%
%%
