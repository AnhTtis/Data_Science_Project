\section{Introduction}\label{sec:intro}

Recently, the Smart Grid (SG) has been proposed as the next-generation architecture of electrical grid that modernizes the current power grids for better utilization of existing assets. One of the key features of the SG is the replacement of conventional mechanical meters with smart meters to enable efficient, flexible and reliable two-way flow of energy and information for better coordination between suppliers and customers and control of the power grids. Renewable energy sources (RES), e.g. wind turbines and solar photovoltaics, have become one of the critical parts in SG, as well as more and more users are willing to deploy the RES and energy storage devices at their homes \cite{farmad2012integration}. This paper explores an approach to utilize the maximum available of RES taking into consideration the recognized problem of carbon pollution.
Based on a report of the European Wind Energy Association, in 2013 EU member states had an installed wind generation capacity of 117,289MW. This is expected to increase to meet the EU Commission's sustainability target, which is that 27\% of generation must come from renewable sources by 2030. Germany is leading the world in the development and implementation of photovoltaic solar power. Japan has similarly moved to the forefront of distribution automation through its use of advanced battery-storage technology. Totally, EU member states have committed to rolling out close to 200 million smart electric meters. By 2020, it is expected that almost 72\% of European consumers will have a smart electric meter. These smart meters can be used not only to provide more accurate billing for customers but can also be leveraged by distribution network operators for improved outage management processes, power quality and demand response programs, each of which will allow for increased network reliability.

Demand Response (DR) is the response system of the end-users to manage their consumption of electricity in response to supply conditions. Demand response management (DRM) smooths out the system power demand profile across time and provides short-term reliability benefits as it can offer load relief to resolve system and/or local capacity constraints. Thus, from the operator aspect, DRM reduces the cost of operating the grid, while from the user aspect it lowers their bills. In most of the past papers that have modeled renewable energy (RE), e.g \cite{bu2011stochastic}, \cite{he2011multiple}, the key focus has been on exploring DRM in the context of multiple energy sources, incorporating RESs on the supply side and minimizing the cost of this side. Instead, \cite{maharjanuser} includes multiple renewable and non-renewable energy sources based utility companies and focuses on minimizing the cost for each user.
Recently, game theory has been applied to the DR programs \cite{saad2012game}. For example, the non-cooperative game was utilized to minimize the electricity cost of consumers \cite{mohsenian2010autonomous} and the congestion game was applied to control the electricity consumption by dynamic pricing \cite{ibars2010distributed}. The research work in \cite{mohsenian2010autonomous} formulates an energy consumption scheduling game, where the players are the users and their strategies are the daily schedules of their household appliances and loads. Based on a common scenario, with a single utility company serving multiple customers, the global optimal performance in terms of minimizing the energy costs is achieved at the Nash equilibrium of the formulated energy consumption scheduling game.

Most of the demand-side management (DSM) approaches that have been deployed over the past three decades (e.g. \cite{ruiz2009direct}), focus on the interactions between a utility company (UC) and its customers or users (via Real-Time Pricing, Direct Load Control, etc.), that reveal users' privacy problems when it comes to residential load control and home automation. In contrast, Mohsedian-Rad et al. \cite{mohsenian2010autonomous} argue that rather than focusing on how each user behaves individually, a good DSM program should have the objective that the aggregate load satisfies some desired properties. More specifically, this work presents an incentive-based energy consumption scheduling scheme, where through an appropriate pricing scheme, the Nash equilibrium of the energy consumption game among the participating users who share the same energy source is the optimal solution that minimizes the energy cost in the system. The experimental results reveal that this approach is able to reduce the peak-to-average ratio of the total energy demand, the total energy costs, and also each user's individual daily electricity bills.
The previous work was focused on how the users can schedule their appliances so as to minimize their billing charges. The underlying assumption was that the users are acquiring energy so as to immediately use it for their appliances.

However, in the future SGs, energy storage is expected to be a key component in smart homes and thus, it has a strong impact on DSM. In \cite{vytelingum2010agent}, Vytelingum et al. provide a general framework within which to analyze the Nash equilibrium of an electricity grid and devise new agent-based storage learning strategies that adapt to market conditions. The simulation results show that the learning scheme converges to a Nash equilibrium while reducing the peak demand which also leads to reduce costs and carbon emissions. Based on the normal electricity usage profile of all the users in the considered SG, it is then possible to compute the Nash equilibrium points, which describe when it is best for the agents to charge their batteries, use their stored electricity, or use electricity from the SG \cite{fadlullah2011survey}.

Storage devices may be used to compensate for the variability of typical renewable electricity generation including wind, wave and solar energy. With this motivation, \cite{maharjanuser} includes multiple renewable energy sources on the supply side, in addition to the traditional fossil-fuel based sources. Specifically, the users schedule their demands by optimally choosing the UCs and by optimally scheduling their shiftable appliances in order to minimize their payments to the providers. They formulate the problem as a game, incorporating the uncertainty associated with the power supply of the renewable sources, and prove that there exists a Nash equilibrium for the game.

In \cite{mohsenian2010autonomous}, \cite{ibars2010distributed}, \cite{jia2012optimal} users are price anticipating (i.e., they take into account how their consumptions will affect the prices). For example, in \cite{ibars2010distributed} is defined a non-cooperative congestion game among users, which is used to set the price levels for the demand vector of each user. The work demonstrates that the game formulated as such converges to a stable equilibrium point in a distributed manner. Furthermore, the work not only finds the local optimum points of the Nash equilibria but also shows that it is possible to arrive at a global solution to the network problem by using the equivalence between the congestion and potential games. On the contrary, in case where users are price taking (\cite{li2011optimal},\cite{yang2013game}), they aim to minimize their costs.

There has been a growing interest in last years in modeling ICT systems and applications where a number of selfish agents compete over a limited capacity and low-cost resource. Very recent work, \cite{hasan2012non} and \cite{shi2011competition} has been focused on addressing the competition in game-theoretic terms. With the same spirit, in \cite{holzman2003network}, the drivers choose between a number of routes from a common-to-all origin to a common-to-all destination. Likewise, in an instance of the access-point association problem, a number of mobiles compete over a finite number of access points where high access delays occur when many users associate with the same wireless access point \cite{hasan2012non}. In a nutshell, all the works noted above, apply selection games where players choose between specific available resources. In contrast, in this paper we consider the choice between specific sources at any time period. In our uncoordinated source selection problem, consumers are viewed as strategic decision-makers that aim at maximizing their own individual welfare.




