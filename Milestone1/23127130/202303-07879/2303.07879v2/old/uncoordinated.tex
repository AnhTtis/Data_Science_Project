%\section{Uncoordinated energy source selection under risk-seeking consumers}\label{sec:uncoordinated_no_extra_demand}
%%
%We investigate the operational states of the uncoordinated distributed energy source selection when the consumers are risk-seeking, i.e., $\epsilon_{\vartheta_i}=1, \forall i$. %determine their actions with respect to their energy consumption profile and the associated energy demand level, as reported in Definition \ref{def:energy_source_game}. 
%%
%%\subsection{Mixed-action equilibrium strategies}\label{sec:equilibria_no_extra_demand}
%%For the Energy Source Selection Game, the strategy profile $s'\in\times_{k=1}^{N}S_{k}$ is a Bayesian Nash equilibrium if for all $i\in\mathcal{N}$:
%%\small
%%\begin{equation}\label{equ:exp_cost1}
%%s_{i}(\vartheta_{i})\in \arg \min_{s'_{i}\in S_{i}}
%%(c_{i}^{(\cdot)}(s_{i}',s_{-i}(\vartheta_{-i}),\vartheta_{i},\vartheta_{-i}))\nonumber \text{ ~~~~~or, }
%%\end{equation}
%%\normalsize
%%\small
%%\begin{equation}\label{equ:exp_cost2}
%%s_{i}(\vartheta_{i})\in \arg \min_{s'_{i}\in S_{i}}\sum_{\vartheta_{-i}}f_{\Theta}(\vartheta_{-i}/\vartheta_{i})c_{i}^{(N-\sum_{k}\vartheta_{k})}(s_{i}',s_{-i}(\vartheta_{-i}),\vartheta_{i},\vartheta_{-i})
%%\nonumber
%%\end{equation}
%%\normalsize
%%where $f_{\Theta}(\vartheta_{-i}/\vartheta_{i})$ is the posterior conditional probability of the low-to-moderate energy consumers \emph{given} the energy profile of user $i$, as derived from the application of the Bayesian rule. Therefore, $s'$ minimizes the expected cost over all possible combinations of the other consumers' types and strategies so that no player can further lower its expected cost by unilaterally changing his strategy.
%%
%%\textbf{Existence:} Game with finite set of players and finite strategy sets has at least
%%one (mixed) Nash Equilibrium \cite{Nash51}.
%%
%%\textbf{Derivation:} Any mixed-action equilibrium $p^{NE}$ must fulfil
%%\vspace{-5pt}
%%\small
%%\begin{equation}\label{eq:cost_equality_mixed}
%%c_{i}^{(\cdot)}(RES,p)=c_{i}^{(\cdot)}(nonRES,p)
%%\end{equation}
%%\normalsize
%%
%%Namely, the expected costs of each pure action belonging to the support of the equilibrium mixed-action strategy are equal. By equation (\ref{eq:costs_symnonRES}), the expected cost of action nonRES is given by
%%$c_{i}^{(\cdot)}(nonRES,p) = E_{\vartheta_i} \cdot c_{nonRES,N}$, irrespective of $N_0$ (or $N-N_0$), that is, the cardinality of the two sets of consumers. Likewise, by Definition \ref{def:energy_source_game}, the expected cost of action RES is given by $c_{i}^{(\cdot)}(RES,p) =\sum_{N_0=0}^{N}c_{i}^{N_0}(RES,p) B(N_0;N,r)$, where the cost $c_{i}^{N_0}(RES,p)$ follows equation (\ref{eq:costs_symRES}). Under high populations of energy nodes, the expected cost of action RES can be approximated by
%%
%%\vspace{-5pt}
%%\small
%%\begin{equation}\label{eq:gen_costs_symnonRES}
%%c_{i}^{(\cdot)}(RES,p)=w_{RES,\vartheta_i}(r(N-1)p_{RES,0}, (1-r)(N-1)p_{RES,1})
%%\end{equation}
%%\normalsize
%%
%By replacing  Eqs. (\ref{eq:costs_symnonRES}), (\ref{eq:gen_costs_symRES}) to Eq. \eqref{eq:cost_equality_mixed} and after a few algebraic manipulations, we obtain that in the mixed strategies equilibrium $p^{NE}$ it holds that
%
%\vspace{-5pt}
%\small
%\begin{equation}\label{eq:conditionEQ}
%rse^{NE}_{\vartheta_i}(n) = \frac{\gamma-\beta}{\gamma-1}E_{\vartheta_i}.
%\end{equation}
%
%\normalsize
%
%
%\subsection{Proportional Allocation}\label{sec:prop_alloc_eq}
%
%In the first three cases, we assume that the users' aggregate demand exceeds the RES capacity.
%
%\textbf{Case $1$.} If $E_{i}=E \leq  \mathcal{ER}\frac{(\gamma-1)}{(\gamma-\beta)}$, $\forall i \in \mathcal{N}$, there exists a mixed strategy equilibrium for all consumer profiles. The competing strategies are in equilibrium when it holds that  $rse^{PA}_{\vartheta_i}(\cdot)=rse^{NE}_{\vartheta_i}(\cdot)$, $\forall i\in \mathcal{N}$. By replacing to the latter relation the right-hand sides of Eqs. (\ref{eq:prop_alloc_energy}) and (\ref{eq:conditionEQ}), we obtain, 
%\begin{align}
%\sum_{\vartheta_j \in \Theta } n_{\vartheta_j }E_{\vartheta_j} +E_{\vartheta_i}=\mathcal{ER}\frac{(\gamma-1)}{(\gamma-\beta)}.
%    \label{eq:forranges}
%\end{align}
%The last equality should be true for all user profiles. Therefore, a mixed strategies equilibrium is possible if all distinct groups of consumers demand the same amount of energy, that is $E_{\vartheta_i}=E \leq  \mathcal{ER}\frac{(\gamma-1)}{(\gamma-\beta)}$, $\forall i \in \mathcal{N}$. The competing probabilities, $p^{PA,NE}_{RES,\vartheta_i}$ that lead to mixed strategies equilibrium range in
%\vspace{-5pt}
%\footnotesize
%\begin{align}
%\Biggl[ max\left(0,\left[\frac{\mathcal{ER} }{E}\frac{(\gamma-1)}{(\gamma-\beta)}-1-\sum_{\vartheta_j \neq \vartheta_i } r_{\vartheta_j } (N-1)\right]\frac{1}{r_{\vartheta_i}(N-1)}\right),
%   \notag\\
%     min\left(1,\left[\frac{\mathcal{ER} }{E}\frac{(\gamma-1)}{(\gamma-\beta)}-1\right]\frac{1}{r_{\vartheta_i} (N-1)}\right)
%    \Biggr],
%    \label{eq:prop_alloc_eq_bounds_0}
%\end{align}
%\normalsize
%%\vspace{-5pt}
%%\footnotesize
%%\begin{align}
%%  p^{PA,NE}_{RES,1}
%%  \in \Biggl[ max\left(0,\left[\frac{\mathcal{ER} }{E}\frac{(\gamma-1)}{(\gamma-\beta)}-1-r(N-1)\right]\frac{1}{(1-r)(N-1)}\right),
%%   \notag\\
%%     min\left(1,\left[\frac{\mathcal{ER} }{E}\frac{(\gamma-1)}{(\gamma-\beta)}-1\right]\frac{1}{(1-r)(N-1)}\right)
%%    \Biggr].
%%    \label{eq:prop_alloc_eq_bounds_1}
%%\end{align}
%%\normalsize
%The latter ranges derive via Eq. \eqref{eq:forranges} by replacing $E_{\vartheta_i}=E$, $n_{\vartheta_i}=p^{PA,NE}_{RES,\vartheta_i}r_{\vartheta_i}(N-1)$, $\forall \vartheta_i \in \Theta$, and then taking the extreme values of probabilities (i.e., $0,1$) to compute the range for the considered one. Note that the probabilities $p^{PA,NE}_{RES,\vartheta_i}$, $\forall \vartheta_i \in \Theta$, should be chosen in the above ranges but so that they satisfy $\frac{\mathcal{ER}}{E}\frac{(\gamma-1)}{(\gamma-\beta)}-1=\sum_{ \vartheta_j \in \Theta}r_{\vartheta_j} (N-1) p^{PA,NE}_{RES,\vartheta_j} $. In Section \ref{}, we provide an algorithm that leads to ....
%
%
%\textbf{Case $2$.} If $E_{\vartheta_i}  > \mathcal{ER}\frac{(\gamma-1)}{(\gamma-\beta)}$, the dominant strategy for all consumers is to defer from competing for $RES$, namely $\upsilon_{RES, \vartheta_i}(p)>\upsilon_{nonRES, \vartheta_i}(p)$, $\forall p$ and $\forall i\in \mathcal{N}$. To show this, assume that $\vartheta_i=0$ and the allocated energy is $E'$. Then, in a large population regime $\upsilon_{RES, \theta_i}(p)= E'  c_{RES}+(E_0-E') \gamma  c_{RES}$ and 
%$\upsilon_{nonRES, \theta_i}(p)=E_0 \beta c_{RES}$. The inequality $\upsilon_{RES, \theta_i}(p)>\upsilon_{nonRES, \theta_i}(p)$ is then equivalent to the inequality $E_0 >E' \frac{(\gamma-1)}{(\gamma-\beta)}$ which is true by assumption and by the fact that the allocated energy $E'$ will be less than the available energy $\mathcal{ER}$. The same can be shown for all $\vartheta_i \in \Theta$. Therefore, in this case $p^{PA,NE}_{RES, \vartheta_i}=0$, for all $\vartheta_i \in \Theta$.
%
%
%\textbf{Case $3$.} In the asymmetric case where $E_{M-1}> ..>E_K > \mathcal{ER}\frac{(\gamma-1)}{(\gamma-\beta)} \geq E_{K-1}>..>E_0$, %the resulting interaction of consumers can be approached as a dominance-solvable game. Namely, it is assumed that rationality (as conceptualized by performing the best-response action) among players is common knowledge, that is, each player knows that the rest of the players are rational, and each player knows that the rest of the players know that he knows that the rest of the players are rational, and so on ad infinitum. In particular, 
%it is dominant strategy for all consumers $i$ ($i \in \mathcal{N}$) of higher energy profiles, i.e., $\vartheta_i\geq K$, not to compete for $RES$ (shown as in Case $2$ by comparing the costs of each pure strategy). Therefore, $p^{PA,NE}_{RES,\vartheta_i}=0$ for all $i \in \mathcal{N}$ with $\vartheta_i\geq K$. For consumers of lower energy profiles, there exists a mixed strategy equilibrium obtained as in Case $1$, under the same conditions.  Specifically, a mixed strategies equilibrium is possible if for all $\vartheta_i\leq K-1$, it holds that $E_{\vartheta_i}=E \leq  \mathcal{ER}\frac{(\gamma-1)}{(\gamma-\beta)}$. For those user profiles, the competing probabilities, $p^{PA,NE}_{RES,\vartheta_i}$ that lead to mixed strategies equilibrium range in
%\vspace{-5pt}
%\footnotesize
%\begin{align}
%   \Biggl[ max\left(0,\left[\frac{\mathcal{ER} }{E}\frac{(\gamma-1)}{(\gamma-\beta)}-1-\sum_{\vartheta_j \neq \vartheta_i, \vartheta_j<K } r_{\vartheta_j } (N-1)\right]\frac{1}{r_{\vartheta_i} (N-1)}\right),
%   \notag\\
%     min\left(1,\left[\frac{\mathcal{ER} }{E}\frac{(\gamma-1)}{(\gamma-\beta)}-1\right]\frac{1}{r_{\vartheta_i} (N-1)}\right)
%    \Biggr],
%    \label{eq:prop_alloc_eq_bounds_new}
%\end{align}
%\normalsize%by equalizing the right-hand sides of Eqs. (\ref{eq:prop_alloc_energy}) and (\ref{eq:conditionEQ}) and replacing $n_0=p^{PA,NE}_{RES,0}(1-r)(N-1)$. Indeed, this leads to the equation $p^{PA,NE}_{RES,0} r(N-1)E_0 +E_0= \mathcal{ER}\frac{(\gamma-1)}{(\gamma-\beta)}$ from where 
%
%\textbf{Case $4$.} Finally, when the RES capacity exceeds the total demand it is trivial to show that the dominant strategy is competing for $RES$. Therefore, $p^{PA,NE}_{RES,\vartheta_i}=1$, $\forall \vartheta_i \in \Theta$.
%
%\textbf{Demand:} In Case $1$, the resulting expected aggregate demand for renewable energy is $N  \sum_{\vartheta_j \in \Theta} r_{\vartheta_j }p^{PA,NE}_{RES,\vartheta_j}E_{\vartheta_j}$ where the mixed strategy equilibrium probabilities $p^{PA,NE}_{RES,\vartheta_j}$, for all $\vartheta_j \in \Theta$ lie in the intervals (\ref{eq:prop_alloc_eq_bounds_0}). In Case $2$ the demand for renewable energy is zero. In Case $3$, the expected aggregate demand is defined similarly to Case $1$ and in Case $4$, it is $N  \sum_{\vartheta_j \in \Theta} r_{\vartheta_j }E_{\vartheta_j}$. Thus, is all cases the expected aggregate demand can be summarized as:
%
%%\vspace{-5pt}
%%\footnotesize
%%\begin{align}
%%  D^{PA,NE}(p^{PA,NE}) =  N  \left[ r  p^{PA,NE}_{RES,0}E_0+(1-r)  p^{PA,NE}_{RES,1}E_1\right] \notag \\
%%= min\left(N[rE_0+(1-r)E_1],\frac{N}{(N-1)}  \left[ \frac{\mathcal{ER}(\gamma-1)}{(\gamma-\beta)}-E_0 \right]\right)
%%\label{eq:demand_pa_ne_no_extra_demand}
%%\end{align}
%%\normalsize
%
%%\footnotesize
%\begin{scriptsize}
%\begin{eqnarray}
%D^{PA,NE} =  \nonumber \\ \left\{\hspace{-5pt}
%\begin{array}{l l}
%min\left(\sum_{ \vartheta_j=0}^{K-1} r_{\vartheta_j} N E_{\vartheta_j},max\left(0,\frac{N}{(N-1)}  \left[ \frac{\mathcal{ER}(\gamma-1)}{(\gamma-\beta)}-E_0 \right]\right)\right), \nonumber \\ \text{~if~ $E_{M-1}> ..>E_K > \mathcal{ER}\frac{(\gamma-1)}{(\gamma-\beta)} \geq E_{k-1}>..>E_0$} \\
%min\left(\sum_{ \vartheta_j=0}^{M-1} r_{\vartheta_j} N E_{\vartheta_j},max\left(0,\frac{N}{(N-1)}  \left[ \frac{\mathcal{ER}(\gamma-1)}{(\gamma-\beta)}-E_0 \right]\right)\right), \text{o/w}.  \nonumber \\
%\end{array} \right.\\\hspace{-200pt}
%\label{eq:demand_pa_ne_no_extra_demand}
%\end{eqnarray}
%\end{scriptsize}
%\normalsize
%
%%Otherwise, in the symmetric case with $E_1 \geq E_0 > \mathcal{ER}\frac{(\gamma-1)}{(\gamma-\beta)}$ the demand is zero while under the asymmetry $E_1 > \mathcal{ER}\frac{(\gamma-1)}{(\gamma-\beta)} \geq E_0$, the demand amounts to $rNE_0$. 
%
%Notice that the impact of the lack of coordination arises when the competing probabilities, $p^{PA,NE}_{RES,\vartheta_i}$, $\vartheta_i \in \Theta$ generate $D^{PA}$ exceeding the available renewable energy capacity $\mathcal{ER}$ (case $1$) or the consumers are overconservative in (and refrain more than they should) competing for $RES$. In the next sections, we quantify this inefficiency of equilibria by using the Price of Anarchy metric.
%
%\textbf{Social Cost:} The expected aggregate cost incurred by the entire population (\ie social cost) is given by
%
%\vspace{-5pt}
%\footnotesize
%\begin{eqnarray}
%C^{PA,NE} \hspace{-5pt} &=& \hspace{-5pt} min(\mathcal{ER}, D^{PA,NE}) c_{RES} +  max(0,D^{PA,NE}-\mathcal{ER})\cdot  \nonumber \\ c_{nonRES,D} 
% \hspace{-5pt}   &+& \hspace{-5pt} \left[ \left( \sum_{ \vartheta_j=0}^{M-1} r_{\vartheta_j} N E_{\vartheta_j}\right)-D^{PA,NE} \right] c_{nonRES,N}.
% \label{eq:social_cost_pa}
% \end{eqnarray}
%\normalsize
%
%
%\subsection{Equal Sharing}\label{sec:eq_sharing_eq}
%
%In this case, no player has the incentive to change his decision unilaterally, if the amount of RES allocated satisfies $rse^{ES}_{\vartheta_i}(\cdot)=rse^{NE}_{\vartheta_i}(\cdot)$, $\forall i\in \mathcal{N}$. Replacing in this equality the right-hand sides of Eqs. (\ref{eq:eq_sharing_energy}) and (\ref{eq:conditionEQ}), we obtain that the equilibrium states are possible when $E_{\vartheta_j}=E$, $\forall \vartheta_j \in \Theta$. Therefore, the probabilities, $p^{ES,NE}_{RES,\vartheta_j}$, $\forall \vartheta_j \in \Theta$, which lead to equilibrium states lie in the intervals (\ref{eq:prop_alloc_eq_bounds_0}), respectively, similarly to the proportional allocation. Also, for all other cases with respect to the values of $E_{\vartheta_j}$, $\forall \vartheta_j \in \Theta$ the mixed strategies that are dominant or equilibria coincide with the corresponding ones of Section \ref{sec:prop_alloc_eq}.
%
%%The expected aggregate demand for renewable energy $D^{ES,NE}$ is equal to $D^{PA,NE}$ (Eq. (\ref{eq:demand_pa_ne_no_extra_demand})). Finally, the social cost under equal sharing is derived in a similar fashion as Eq. (\ref{eq:social_cost_pa}). However, to do so, we need to account for the extra allocated renewable energy to users due to equal sharing, e.g., when all users (both profiles) compete for $RES$ and possibly the lower energy profile users are allocated more RES than their requested one or when the amount of RES is higher than the overall users' demand. The fair share of RES, $sh(p^{ES,NE})$, and the expected aggregate social cost, $C^{ES,NE}(p^{ES,NE})$, can be approximated by 
%%
%%\vspace{-5pt}
%%\footnotesize
%%\begin{eqnarray}
%%sh(p^{ES,NE}) \hspace{-5pt} &=& \hspace{-5pt} \frac{\mathcal{ER}}{ \sum_{ \vartheta_j=0}^{M-1} r_{\vartheta_j} N E_{\vartheta_j} p_{RES,\vartheta_j}^{ES,NE}},
%% \label{eq:sharing_es}
%% \end{eqnarray}
%%\normalsize
%%
%\vspace{-5pt}
%\footnotesize
%\begin{align}
%&C^{ES,NE}(p^{ES,NE}) \nonumber\\ &=  d(p^{ES,NE}) c_{RES} +(D^{ES,NE} -d(p^{ES,NE}))c_{nonRES,D}\nonumber \\
%&+ \hspace{-5pt} \left[ \left[  \sum_{ \vartheta_j=0}^{M-1} r_{\vartheta_j} N E_{\vartheta_j}\right]-D^{ES,NE} \right] c_{nonRES,N},
% \label{eq:social_cost_es}
% \end{align}
%\normalsize
%
%with \footnotesize$d(p^{ES,NE})= \sum_{ \vartheta_j=0}^{M-1} r_{\vartheta_j} p_{RES,\vartheta_j}^{ES,NE} N min(sh(p^{ES,NE}), E_{\vartheta_j})$.\normalsize
%
%%\subsection{Social Cost}\label{sec:social_cost_no_extra_demand}
%%
%%The expected aggregate cost incurred by the entire population (\ie social cost) under mixed-action strategies is a function of competing probabilities and is given by
%%
%%\vspace{-5pt}
%%\footnotesize
%%\begin{eqnarray}
%%C(p) \hspace{-5pt} &=& \hspace{-5pt} min(\mathcal{ER}, D^{(\cdot)}(p)) c_{RES} +  max(0,D^{(\cdot)}(p)-\mathcal{ER})c_{nonRES,D} \nonumber \\
%% \hspace{-5pt}   &+& \hspace{-5pt} \left[ \left[ rNE_0 + (1-r)NE_1\right]-D^{(\cdot)}(p) \right] c_{nonRES,N}
%% \label{eq:social_cost_pa_es}
%% \end{eqnarray}
%%\normalsize
%%
%%Notice that the use of equation (\ref{eq:social_cost_pa_es}) to derive the social cost under equal sharing, implies an advanced implementation that provides the amount of renewable energy that possibly remains unused to increase the share of competitors with excess energy requirements. This is the case of equilibrium states where all consumers of low-to-moderate energy profile are motivated to compete while the renewable energy capacity exceeds their total demand. 
%%
%
%
%%or equivalently, with $x=max(\mathcal{ER} ,(r N  p_{nonRES,0}E_0+(1-r) N p_{nonRES,1}E_1))$
%%
%%\vspace{-5pt}
%%\footnotesize
%%\begin{align}
%% C(p) =
%%   r\cdot N \Bigl[ p_{nonRES,0} \Bigl[ \mathcal{ER} \cdot E_0/x \cdot c_{RES} +  \notag\\
%%  (E_0-\mathcal{ER} \cdot E_0/x)\cdot c_{nonRES,D} \Bigr] +  (1-p_{nonRES,0})E_0 \cdot c_{nonRES,N} \Bigr] + \notag\\
%%   (1-r)\cdot N \cdot \Bigl[  p_{nonRES,1} \Bigl[ \mathcal{ER} \cdot E_1/x \cdot c_{RES} + \notag\\
%%   (E_1-\mathcal{ER} \cdot E_1/x)\cdot c_{nonRES,D} \Bigr] +  (1-p_{nonRES,1})E_1 \cdot c_{nonRES,N}  \Bigr]
%%\label{eq:social_cost_pa_es}
%%\end{align}
%%\normalsize
%
%%%%%%%%%%%%%%%%%%%&& (1-r)\cdot N \cdot \left[p_{nonRES,1} \left[ x_1 \cdot c_{RES} +  (E_1-x_1)\cdot c_{nonRES,D} \right] + (1-p_{nonRES,1})E_1 \cdot c_{nonRES,N}  \right]
%
%
%
%

\section{Study of uncoordinated energy source selection}% under risk-conservative consumers}
\label{sec:uncoordinated_extra_demand}

In this section, we investigate the operational states of the distributed, uncoordinated energy source selection for both risk-conservative and risk-seeking consumers. Specifically, we study the conditions on the parameter values for the existence of dominant strategies or mixed-strategy NE under PA. To derive the NE, we recall that any mixed strategy NE $\mathbf{p^{NE}}$
%=(\mathbf{p_0^{NE}}, \mathbf{p_1^{NE}}, ..., \mathbf{p_{M-1}^{NE}})^T$, with $\mathbf{p}^{NE}_{\ell}=[p^{NE}_{RES,\ell}, p^{NE}_{nonRES,\ell}]^T$,
must fulfill

\vspace{-0.08in}
\small
\begin{equation}\label{eq:cost_equality_mixed}
\upsilon_{RES, \ell}(\mathbf{p^{NE}})= \upsilon_{nonRES, \ell}(\mathbf{p^{NE}}), ~\forall \ell \in \Theta.
\end{equation}
\normalsize
Namely, the expected costs of each pure strategy in the support of the mixed-strategy equilibrium  ($\mathcal{A}$) are equal. By substituting the expressions of \eqref{eq:RES_cost} and \eqref{eq:nonRES_cost2} in \eqref{eq:cost_equality_mixed}, we obtain that the amount of RESs allocated to $\ell \in \Theta$ at a NE should satisfy

\vspace{-0.08in}
\small
\begin{equation}\label{eq:conditionEQ_extra_demand}
rse^{PA, NE}_{\ell}(\mathbf{p^{NE}}) = \frac{\gamma-\epsilon_{\ell}\beta}{\gamma-1}E_{\ell},~ \forall \ell \in \Theta.
\end{equation}
\normalsize
%\subsection{Close-form expression of NE}
Thus, when combining the Energy Source Selection Game with the PA policy, the existence of a NE is under the condition

\vspace{-0.08in}
\small
\begin{equation}\label{eq:condition_PA_NE}
rse_{\ell}^{PA}(\mathbf{p}^{NE}) =rse_{\ell}^{PA,NE}(\mathbf{p}^{NE}), \forall \ell \in \Theta. \end{equation}
\normalsize
Therefore, for all cases, any existing mixed-strategy NE competing probabilities, $\mathbf{p}^{NE}$, are obtained so as to satisfy the relation \eqref{eq:condition_PA_NE}. In the following we distinguish several cases based on the parameters values, namely the RES capacity, the risk aversion degrees, and the day-time energy demand levels.

\subsubsection{\textbf{Case $1$}} \emph{The RES capacity, $\mathcal{ER}$, exceeds the maximum total demand for RES $D^{Total}$ ($\mathcal{ER} \geq D^{Total}$).}

As the consumers have knowledge of $\mathcal{ER}$ and $D^{Total}$, it is straightforward to show that the dominant-strategy for all consumers is to select the strategy $RES$. As result, the competing probabilities that lead to equilibrium states are equal to $1$ for all consumers' types, and the expected aggregate demand for RES at NE is equal to $D^{Total}$.

In all the remaining cases, we assume that $\mathcal{ER}< D^{Total}$. Three distinct cases are defined with respect to the risk aversion degrees of the consumers and energy prices.

\subsubsection{\textbf{Case $2$}} \emph{The risk aversion degrees of all consumers satisfy $1\leq \epsilon_{\vartheta_i}<\gamma/\beta$, $\forall i\in \mathcal{N}$.}

Note that Case $2$ includes both risk-seeking and risk-averse consumers.
We distinguish the following Sub-cases with respect to the energy demand profiles $E_{\vartheta_i }$ of the consumers. 
\vspace{5pt}

\textbf{Sub-case $2(a)$:} 
\emph{The energy demand profiles of all consumers satisfy  $E_{\vartheta_i } \leq \mathcal{ER}\frac{(\gamma-1)}{(\gamma-\epsilon_{\vartheta_i }\beta)}$,  for all $\vartheta_i \in \Theta$.} %or $E_{\vartheta_i} > \mathcal{ER}\frac{(\gamma-1)}{(\gamma-\epsilon_{\vartheta_i}\beta)}$  for all $\vartheta_i \in \Theta$.}


A mixed strategy equilibrium with the PA policy exists if and only if for every $E_{\vartheta_i}$, $E_{\vartheta_j}$, $\vartheta_i, \in \Theta$ and $ \vartheta_j \in \Theta \setminus \{\vartheta_i\}$, it holds that
\small
\begin{eqnarray}\label{eq:relation_E_0_E_1_pa_ne_extra_demand}
\mathcal{ER}\frac{(\gamma-1)}{(\gamma-\epsilon_{\vartheta_i }\beta)}-E_{\vartheta_i} &=& \mathcal{ER}\frac{(\gamma-1)}{(\gamma-\epsilon_{\vartheta_j }\beta)}-E_{\vartheta_j}.
 \end{eqnarray}
\normalsize
This condition is derived by equalizing the right-hand sides of Eqs. \eqref{eq:prop_alloc_energy} and \eqref{eq:conditionEQ_extra_demand} for $\vartheta_i \in \Theta$ and $\vartheta_j \in \Theta \setminus \{\vartheta_i\}$, and obtaining:
\begin{small}
\begin{align}
&  \mathcal{ER}\frac{(\gamma-1)}{(\gamma-\epsilon_{\vartheta_i}\beta)}-E_{\vartheta_i}= \sum_{ {\vartheta_l}\in \Theta} r_{\vartheta_l} (N-1)E_{\vartheta_l} p^{PA,NE}_{RES,\vartheta_l}
    \label{eq:probrelation1}\\
  &  \mathcal{ER}\frac{(\gamma-1)}{(\gamma-\epsilon_{\vartheta_j}\beta)}-E_{\vartheta_j}=  \sum_{ {\vartheta_l}\in \Theta} r_{\vartheta_l} (N-1)E_{\vartheta_l} p^{PA,NE}_{RES,\vartheta_l}
    \label{eq:probrelation2}  
\end{align}
\end{small}
where the right hand sides of Eqs. \eqref{eq:probrelation1}-\eqref{eq:probrelation2} are equal.

Under condition \eqref{eq:relation_E_0_E_1_pa_ne_extra_demand}, the competing probabilities $p^{PA,NE}_{RES,{\vartheta_i }}$ that lead to equilibrium states lie in the range
\footnotesize
\begin{align}
&  \Biggl[ \max \left\{0,\left(\mathcal{ER}\frac{(\gamma-1)}{(\gamma-\epsilon_{\vartheta_i}\beta)}-E_{\vartheta_i}-
  \sum_{ {\vartheta_l}\in \Theta \setminus \{{\vartheta_i}\}} r_{\vartheta_l} (N-1)E_{\vartheta_l}\right)\cdot  \right. \nonumber \\&  \left. \frac{1}{r_{\vartheta_i}(N-1)E_{\vartheta_i}}\right\},
     \min\left\{1,\left(\mathcal{ER}\frac{(\gamma-1)}{(\gamma-\epsilon_{\vartheta_i}\beta)}-E_{\vartheta_i}\right)\frac{1}{r_{\vartheta_i}(N-1)E_{\vartheta_i}}\right\}
    \Biggr].
    \label{eq:prop_alloc_pa_bounds_0_extra_demand}
\end{align}
\normalsize
%\vspace{-5pt}
%\footnotesize
%\begin{align}
%  p^{PA,NE}_{RES,1}
%  \in
%   \notag\\\Biggl[ max\left(0,\left[\mathcal{ER}\frac{(\gamma-1)}{(\gamma-\epsilon_{1}\beta)}-E_1-r(N-1)E_0\right]\frac{1}{(1-r)(N-1)E_1}\right),
%   \notag\\
%     min\left(1,\left[\mathcal{ER}\frac{(\gamma-1)}{(\gamma-\epsilon_{1}\beta)}-E_1\right]\frac{1}{(1-r)(N-1)E_1}\right)
%    \Biggr].
%    \label{eq:prop_alloc_eq_bounds_1_extra_demand}
%\end{align}
%\normalsize
This range is derived by equalizing the right-hand sides of Eqs. (\eqref{eq:prop_alloc_energy} and \eqref{eq:conditionEQ_extra_demand} for $\vartheta_i \in \Theta$ and obtaining \eqref{eq:probrelation1}. Then, we take extreme values ($0$ and $1$) of all other probabilities except the one of interest in order to compute its range of values. Specifically, to obtain the lower (upper) bound we assign unary (zero) probablities to all other consumer types except $\vartheta_i$.

%Note that the equilibrium probabilities $p^{PA,NE}_{RES,\vartheta_i},$  $\forall \vartheta_i \in \Theta$ should satisfy \eqref{eq:probrelation1}, except if this requires values .  %This might not be possible only due to the need of being  \eqref{eq:prop_alloc_pa_bounds_0_extra_demand}. 

Furthermore, at NE, the expected aggregated demand for RES can be expressed as
\footnotesize
\begin{align}
& D^{PA,NE}=\nonumber \\ & \min\Bigl\{D^{Total}, \max\Bigl\{\left[\mathcal{ER}\frac{(\gamma-1)}{(\gamma-\epsilon_{\vartheta_i}\beta)}-E_{\vartheta_i}\right]\frac{N}{(N-1)},0\Bigr\}\Bigr\},
    \label{eq:demand1}
\end{align}
\normalsize
for any $\vartheta_i \in \Theta$. To obtain \eqref{eq:demand1}, we observe that, by using \eqref{eq:probrelation1}, $D^{PA,NE}=N\sum_{ {\vartheta_l}\in \Theta} r_{\vartheta_l} E_{\vartheta_l} p^{PA,NE}_{RES,\vartheta_l}=
\frac{N}{(N-1)}\Bigl( \mathcal{ER}\frac{(\gamma-1)}{(\gamma-\epsilon_{\vartheta_i}\beta)}-E_{\vartheta_i}\Bigr)$.

\begin{remark}In case that all consumers are risk-seeking (i.e., $\epsilon_{\vartheta_i}= 1, \forall i \in \mathcal{N}$), there exist equilibrium states only if  $E_{\vartheta_i}=E_{\vartheta_j}$, for every pair $\vartheta_i, \vartheta_j$ with $\vartheta_i, \vartheta_j \in \Theta$. On the contrary, if there exists risk-conservative consumers, there may exist equilibrium states when consumers competing for $RES$ have asymmetric energy profiles.
\end{remark}

\begin{remark} \label{rem:risk_degrees_relation}
Notice that from the condition of Eq. \eqref{eq:relation_E_0_E_1_pa_ne_extra_demand}, the existence of NE is possible only if $\epsilon_0<\epsilon_1<\epsilon_2<..<\epsilon_{M-1}$. This is because for every pair of energy demand profiles with $E_{\vartheta_i}<E_{\vartheta_j}$, Eq. \eqref{eq:relation_E_0_E_1_pa_ne_extra_demand} is possible to hold only if $\epsilon_{\vartheta_i}<\epsilon_{\vartheta_j}$. This means that consumers with lower energy demand levels should be less risk-averse than those with higher energy demand levels.
\end{remark}
\vspace{5pt}
 
\textbf{Sub-case $2(b)$:} 
\emph{The energy demand profiles satisfy $E_{\vartheta_i} > \mathcal{ER}\frac{(\gamma-1)}{(\gamma-\epsilon_{\vartheta_i}\beta)}$  for all $\vartheta_i \in \Theta$.}


In this case, it is dominant strategy for all consumers to defer from competing for $RES$ and engage their loads in the night-zone. As result, the competing probabilities that lead to equilibrium states are equal to $p^{PA,NE}_{RES,\vartheta_i}=0$ for all consumers' types $\vartheta_i\in \Theta$, and the expected aggregate demand for RES at NE is equal to $D^{PA,NE}=0$. Note that a consumer type $\vartheta_i\in \Theta$ will engage in the night-zone a load equal to $\epsilon_{\vartheta_i} E_{\vartheta_i}$, i.e., the risk-averse consumers will engage their maximum demands.

To prove this, we need to show that $\upsilon_{RES, \vartheta_i}(\mathbf{p})>\upsilon_{nonRES, \vartheta_i}(\mathbf{p})$, $\forall \mathbf{p}$ and $\forall i\in \mathcal{N}$. Assume that $\vartheta_i=0$ and that the allocated energy is $E'$. Then, in a large population regime we have $\upsilon_{RES, 0}(\mathbf{p})= E'  ~c_{RES}+(E_0-E')~ \gamma~  c_{RES}$ and 
$\upsilon_{nonRES, 0}(\mathbf{p})= \epsilon_0~ E_0~ \beta~ c_{RES}$. The inequality $\upsilon_{RES,0}(\mathbf{p})>\upsilon_{nonRES, 0}(\mathbf{p})$ is then equivalent to the inequality $E_0 >E' \frac{(\gamma-1)}{(\gamma-\epsilon_0\beta)}$. The latter inequality is true by assumption, considering that $E'<\mathcal{ER}$. The same can be shown for every $\vartheta_i\in \Theta$.
\vspace{5pt}


\textbf{Sub-case $2(c)$:} %In this Sub-case Eq. \eqref{eq:relation_E_0_E_1_pa_ne_extra_demand} is not true.
\emph{There exist two distinct subsets of consumers' types $\Sigma_1 , \Sigma_2 \subset \Theta$ s.t. $\{E_{\vartheta_i} > \mathcal{ER}\frac{(\gamma-1)}{(\gamma-\epsilon_{\vartheta_i}\beta)}, ~\forall \vartheta_i \in \Sigma_1\}$ and $\{E_{\vartheta_i} \leq \mathcal{ER}\frac{(\gamma-1)}{(\gamma-\epsilon_{\vartheta_i}\beta)}, ~ \forall \vartheta_i \in \Sigma_2\}$.} %the resulting interaction of consumers can be approached as a dominance-solvable game. Namely, it is assumed that rationality (as conceptualized by performing the best-response action) among players is common knowledge, that is, each player knows that the rest of the players are rational, and each player knows that the rest of the players know that he knows that the rest of the players are rational, and so on ad infinitum. In particular, 

For the consumers' types in the set $\Sigma_1$, it is dominant strategy to defer from competing for $RES$ and to engage their loads in the night-zone. The explanation is the same with Sub-case $2(b)$, i.e., it is based on comparing the costs of each pure strategy. Similarly with Sub-case $2(b)$, the competing probabilities that lead to equilibrium states are equal to $p^{PA,NE}_{RES,\vartheta_i}=0$ for all consumers' types $\vartheta_i\in \Sigma_1$, and each consumer will engage in the night-zone a load equal to $\epsilon_{\vartheta_i} E_{\vartheta_i}$, i.e., the risk-averse consumers will engage their maximum demands.

For the consumers' types in the set $\Sigma_2$, the mixed strategy NE is determined under the condition of Eq. \eqref{eq:relation_E_0_E_1_pa_ne_extra_demand} involving only those consumers' types in $\Sigma_2$. Remark \ref{rem:risk_degrees_relation} holds now for all consumers' types in $\Sigma_2$. Additionally, for all consumers' types $\vartheta_i\in \Sigma_2$, the competing probabilities $p^{PA,NE}_{RES,\vartheta_i} $ that lead to equilibrium states lie in the range:%the particular conditions on pricing and energy demand levels and provided by 
%
%
%if their energy demand $E_{\vartheta_j}$ equals the threshold $\frac{(\gamma-1)}{(\gamma-\epsilon_{\vartheta_j}\beta)} rse^{PA}_{\vartheta_j}(\cdot)$, then the competing probability is 

\vspace{-5pt}
\footnotesize
\begin{align}
&  \Biggl[ \max \left\{0,\left[\mathcal{ER}\frac{(\gamma-1)}{(\gamma-\epsilon_{\vartheta_i}\beta)}-E_{\vartheta_i}-
  \sum_{\vartheta_j\in \Sigma_2 \setminus \{\vartheta_i\}} r_{\vartheta_j} (N-1)E_{\vartheta_j}\right]\cdot  \right. \nonumber \\&  \left. \frac{1}{r_{\vartheta_i}(N-1)E_{\vartheta_i}}\right\},
     \min\left\{1,\left[\mathcal{ER}\frac{(\gamma-1)}{(\gamma-\epsilon_{\vartheta_i}\beta)}-E_{\vartheta_i}\right]\frac{1}{r_{\vartheta_i}(N-1)E_{\vartheta_i}}\right\}
    \Biggr].
    \label{eq:prop_alloc_pa_bounds_0_extra_demandnew}
\end{align}
\normalsize
This result is obtained similarly to Eq.  \eqref{eq:prop_alloc_pa_bounds_0_extra_demand} but considering only the consumers' types in the set $\Sigma_2$. For consumer types $\vartheta_i\in \Sigma_1$, $p^{PA,NE}_{RES,\vartheta_i}=0$.

As a result, the expected aggregate demand can be expressed as follows, for any $\vartheta_i \in \Sigma_2$:
\vspace{-5pt}
\footnotesize
\begin{align}
&D^{PA,NE}=\nonumber\\ & \min\Biggl \{N  \sum_{\vartheta_l \in \Sigma_2} r_{\vartheta_l }E_{\vartheta_l}, \max\Biggl\{\left[\mathcal{ER}\frac{(\gamma-1)}{(\gamma-\epsilon_{\vartheta_i}\beta)}-E_{\vartheta_i}\right]\frac{N}{(N-1)},0\Biggr\}\Biggr\}.
    \label{eq:demand2}
\end{align}
\normalsize


%In the special case of two risk-seeking consumers, one in $\Sigma_1$ (with index $1$) and the other in $\Sigma_2$ (with index $0$), we obtain $p^{PA,NE}_{RES,0} =min\left\{1,max\left\{0,\left[\mathcal{ER}\frac{(\gamma-1)}{(\gamma-\beta)}-E_0\right]\frac{1}{r_0(N-1)E_0}\right\}\right\}$. 

\begin{remark}
In the special case of two consumers, one in $\Sigma_1$ (with index $1$) and the other in $\Sigma_2$ (with index $0$), we obtain 
$p^{PA,NE}_{RES,0}  =min\left\{1,max\left\{0, \left[\mathcal{ER}\frac{(\gamma-1)}{(\gamma-\epsilon_{0}\beta)}-E_{0}\right]\frac{1}{r_{0}(N-1)E_{0}}\right\}\right\}$.
\end{remark}

%Note that based on Eq. \eqref{eq:prop_alloc_eq_j_extra_demand1}, if $E_{\vartheta_j}$ is strictly lower than $thr_d=\frac{(\gamma-1)}{(\gamma-\epsilon_{\vartheta_j}\beta)}rse^{PA}_{\vartheta_j}(.)$ then the dominant strategy is to compete for $RES$, else if $E_{\vartheta_j}> th_d$ the dominant strategy is to defer from competing for $RES$.

%Otherwise, if $E_{\vartheta_j}$ is lower than this threshold value, opting for competing is dominant strategy, leveraging the withdrawal of consumers of $\vartheta_i$ energy profile, and, finally, if $E_{\vartheta_j}$ exceeds this threshold value, they are motivated to avoid competition.

\subsubsection*{\textbf{Case $3$}}
\emph{The risk aversion degrees satisfy $\epsilon_{\vartheta_i} \geq \gamma/\beta$, $\forall i\in \mathcal{N}$.}

In this case, it is dominant strategy for all consumers to compete for $RES$. This can be shown by proving that $\upsilon_{RES, \vartheta_i}(\mathbf{p})<\upsilon_{nonRES, \vartheta_i}(\mathbf{p})$, $\forall \mathbf{p},~ \forall i\in \mathcal{N}$. Assume that $\vartheta_i=0$ and that the allocated energy is $E'$. Then, the inequality $\upsilon_{RES,0}(\mathbf{p})<\upsilon_{nonRES, 0}(\mathbf{p})$ is equivalent to the inequality $E_0 >E' \frac{(\gamma-1)}{(\gamma-\epsilon_0\beta)}$. This is true by assumption, since $(\gamma-\epsilon_0\beta)\leq 0$ in this case. 

Therefore, the competing probabilities that lead to equilibrium states are equal to $p^{PA,NE}_{RES,\vartheta_i}=1$ for all consumers' types $\vartheta_i\in \Theta$, and the expected aggregate demand for RES at NE is equal to $D^{PA,NE}=D^{Total}$.

\subsubsection*{\textbf{Case $4$}} \emph{There exist two distinct subsets of consumers' types $\Sigma_1$, $\Sigma_2 \subset \Theta$ s.t. $\Bigl\{\epsilon_{\vartheta_i} \geq \gamma/\beta, ~ \forall \vartheta_i \in \Sigma_1\Bigr\}$, and $\Bigl\{\epsilon_{\vartheta_i} < \gamma/\beta,~\forall \vartheta_i \in \Sigma_2\Bigr\}$.} 

For the consumers' types in the set $\Sigma_1$, it is dominant strategy to compete for $RES$. Therefore, the competing probabilities that lead to equilibrium states are equal to $p^{PA,NE}_{RES,\vartheta_i}=1$ for all consumers' types $\vartheta_i\in \Sigma_1$. By analogy with Case $3$, this is shown by proving that $\upsilon_{RES, \vartheta_i}(\mathbf{p})<\upsilon_{nonRES, \vartheta_i}(\mathbf{p}),~\forall \mathbf{p}$, $\forall \vartheta_i \in \Sigma_1$.


For the consumers' types in the set $\Sigma_2$, the mixed strategy NE is determined under condition \eqref{eq:relation_E_0_E_1_pa_ne_extra_demand} involving only the those consumers' types in $\Sigma_2$. Remark \ref{rem:risk_degrees_relation} now holds for all consumers in $\Sigma_2$. Besides, for consumers' types $\vartheta_i \in \Sigma_2$, the competing probabilities $p^{PA,NE}_{RES,\vartheta_i}$ that lead to equilibrium states lie in the range:

\footnotesize
\begin{align}
&  \Biggl[ \max \left\{0,\left[\mathcal{ER}\frac{(\gamma-1)}{(\gamma-\epsilon_{\vartheta_i}\beta)}-E_{\vartheta_i}-
  \sum_{\vartheta_j\in \Theta \setminus \{\vartheta_i\}} r_{\vartheta_j} (N-1)E_{\vartheta_j}\right]\cdot  \right. \nonumber \\&  \left. \frac{1}{r_{\vartheta_i}(N-1)E_{\vartheta_i}}\right\},
     \min\left\{1,\left[\mathcal{ER}\frac{(\gamma-1)}{(\gamma-\epsilon_{\vartheta_i}\beta)}-E_{\vartheta_i}- \right. \right. \nonumber \\ & \left. \left.
  \sum_{\vartheta_j\in \Sigma_1 \setminus \{\vartheta_i\}} r_{\vartheta_j} (N-1)E_{\vartheta_j}\right]\frac{1}{r_{\vartheta_i}(N-1)E_{\vartheta_i}}\right\}
    \Biggr].
    \label{eq:prop_alloc_pa_bounds_0_extra_demandnew4}
\end{align}
\normalsize
This range is obtained similarly to Eq. \eqref{eq:prop_alloc_pa_bounds_0_extra_demand} considering that the consumers in $\Sigma_1$ compete for $RES$. 

In addition, the expected aggregate demand for RES at NE can be expressed as
\footnotesize
\begin{align}
& D^{PA,NE}= N  \sum_{\vartheta_j \in \Sigma_1} r_{\vartheta_j }E_{\vartheta_j}+\min\Biggl\{N  \sum_{\vartheta_j \in \Sigma_2} r_{\vartheta_j }E_{\vartheta_j}, \max\Biggl\{  \Bigl[\mathcal{ER}\nonumber \\  & \cdot \frac{(\gamma-1)}{(\gamma-\epsilon_{\vartheta_i}\beta)}-E_{\vartheta_i} -\sum_{\vartheta_j\in \Sigma_1 \setminus \{\vartheta_i\}} r_{\vartheta_j} (N-1)E_{\vartheta_j}\Bigr]\frac{N}{(N-1)},0\Biggr\}\Biggr\},
    \label{eq:demand3}
\end{align}
\normalsize
for any $\vartheta_i \in \Sigma_2$.

%\vspace{-5pt}
%\footnotesize
%\begin{align}
%&p^{PA,NE}_{RES,\vartheta_j}  =
%   \left[\mathcal{ER}\frac{(\gamma-1)}{(\gamma-\epsilon_{\vartheta_j}\beta)}-E_{\vartheta_j}-[r+(1-2r)\vartheta_i](N-1)E_{\vartheta_i}\right]\notag \\
%   & \times \frac{1}{[r+(1-2r)\vartheta_j](N-1)E_{\vartheta_j}}, \nonumber\\
%&    0\leq p^{PA,NE}_{RES,\vartheta_j} \leq 1.
% \label{eq:prop_alloc_eq_j_extra_demand2}
%\end{align}
%\normalsize
%%This result derives by equalizing the right-hand sides of Eqs. (\ref{eq:prop_alloc_energy}) and (\ref{eq:conditionEQ_extra_demand}) and after replacing $n_0 E_0 +n_1 E_1 $ with the equal quantity $p^{PA,NE}_{RES,\vartheta_j}[r+(1-2r)\vartheta_j](N-1)E_{\vartheta_j} +[r+(1-2r)\vartheta_i](N-1)E_{\vartheta_i}$ (since consumers with energy profile $E_{\vartheta_i}$ compete for $RES$). %As in Case $2$, based on Eq. \eqref{eq:prop_alloc_eq_j_extra_demand2}, if $E_{\vartheta_j}$ is strictly lower than $thr_d$ then the dominant strategy is to compete for $RES$, else if $E_{\vartheta_j}> th_d$ the dominant strategy is to defer from competing for $RES$.


%if $E_{\vartheta_j}$ equals threshold $\frac{(\gamma-1)}{(\gamma-\epsilon_{\vartheta_j}\beta)} rse^{PA}_{\vartheta_j}(\cdot)$. Otherwise, under lower energy demand $E_{\vartheta_j}$, the dominant strategy amounts to competing and under higher energy demand, they resort to the safer option avoiding competition.

%The mixed strategy equilibrium probabilities for all cases are summarized in Table \ref{table_eq_comp_pro}.
%
%\footnotesize
%\begin{table*}[b]
%\def\arraystretch{1.5}
%\caption{Mixed Strategy Equilibrium Probabilities and Demand for RES, in uncoordinated energy source selection under risk-conservative consumers and proportional allocation of RES; $i,j \in \mathcal{N}$, $\vartheta_i\neq \vartheta_j$, $thr_a=\gamma/\beta$, $thr_b=\mathcal{ER}\frac{(\gamma-1)}{(\gamma-\epsilon_i\beta)}$, $thr_c=\mathcal{ER}\frac{(\gamma-1)}{(\gamma-\epsilon_j\beta)}$, $thr_d=\frac{thr_c}{(r+(1-2r)\vartheta_j)(N-1)+1}$, $thr_e= \frac{thr_c- (r+(1-2r)\vartheta_i)(N-1)E_i}{(r+(1-2r)\vartheta_j)(N-1)+1}$, $thr_f=thr_c- (r+(1-2r)\vartheta_i)(N-1)E_i$.}
%%
%\begin{center}
%\begin{tabular}{|c|c|c|c|c||c|c||c|}
%%
%\cline{1-8}
%$\epsilon_i$ & $\epsilon_j$ & $E_{\vartheta_i}$ & \multicolumn{2}{ c|| }{$E_{\vartheta_j}$} & $p^{PA,NE}_{RES,\vartheta_i}$ & $p^{PA,NE}_{RES,\vartheta_j}$ & $D^{PA,NE}$\\  
%\hline \hline 
%%
%$<thr_a$ & $<thr_a$ & $\leq thr_b$ & \multicolumn{2}{ c|| }{$\leq thr_c$} & Ref. (\ref{eq:prop_alloc_eq_bounds_0_extra_demand}) & Ref. (\ref{eq:prop_alloc_eq_bounds_1_extra_demand}) & $\frac{N}{(N-1)}  \left[ \mathcal{ER}\frac{(\gamma-1)}{(\gamma-\epsilon_{\vartheta_j}\beta)}-E_{\vartheta_j} \right]$\\ \cline{1-8}
%%
%%$<thr_a$ & $<thr_a$ & $> thr_b$ & \multicolumn{2}{ c|| }{$> thr_c$} & 0 & 0 & 0\\ \cline{1-8}
%%
%\multicolumn{1}{|c|}{\multirow{3}{*}{$<thr_a$}} & \multicolumn{1}{c|}{\multirow{3}{*}{$<thr_a$}} & \multicolumn{1}{c|}{\multirow{3}{*}{$>thr_b$}} & \multicolumn{2}{ c|| }{\multirow{1}{*}{$\geq thr_d, \leq thr_c$}} & \multicolumn{1}{ c| }{\multirow{1}{*}{$0$}} & \multicolumn{1}{ c|| }{\multirow{1}{*}{Ref. (\ref{eq:prop_alloc_eq_j_extra_demand1})}} & \multicolumn{1}{ c| }{\multirow{1}{*}{$\frac{N}{(N-1)}\left[\mathcal{ER}\frac{(\gamma-1)}{(\gamma-\epsilon_{\vartheta_j}\beta)}-E_{\vartheta_j}\right]$}}   \\ \cline{4-8}
%%
%\multicolumn{1}{|c|}{} & \multicolumn{1}{c|}{} & \multicolumn{1}{c|}{} & \multicolumn{2}{ c|| }{\multirow{1}{*}{$< thr_d$}} & \multicolumn{1}{ c| }{\multirow{1}{*}{$0$}} & \multicolumn{1}{ c|| }{\multirow{1}{*}{$1$}} & \multicolumn{1}{ c| }{\multirow{1}{*}{$[r+(1-2r)\vartheta_j]NE_{\vartheta_j}$}}\\ \cline{4-8}
%%
%\multicolumn{1}{|c|}{} & \multicolumn{1}{c|}{} & \multicolumn{1}{c|}{} &\multicolumn{2}{ c|| }{\multirow{1}{*}{$>thr_c$}} & \multicolumn{1}{ c| }{\multirow{1}{*}{$0$}} & \multicolumn{1}{ c|| }{\multirow{1}{*}{$0$}} &  \multicolumn{1}{ c| }{\multirow{1}{*}{$0$}} \\ \cline{1-8}
%%
%$\geq thr_a$ & $\geq thr_a$ & $>0$ & \multicolumn{2}{ c|| }{$>0$} & 1 & 1 & $N\left[[r+(1-2r)\vartheta_i]E_{\vartheta_i}+[r+(1-2r)\vartheta_j]E_{\vartheta_j}\right]$\\ \cline{1-8}
%%
%\multicolumn{1}{|c|}{\multirow{3}{*}{$\geq thr_a$}} & \multicolumn{1}{c|}{\multirow{3}{*}{$< thr_a$}} & \multicolumn{1}{c|}{\multirow{3}{*}{$>0$}} & \multicolumn{2}{ c|| }{\multirow{1}{*}{$\geq thr_e, \leq thr_f$}} & \multicolumn{1}{ c| }{\multirow{1}{*}{$1$}} & \multicolumn{1}{ c|| }{\multirow{1}{*}{Ref. (\ref{eq:prop_alloc_eq_j_extra_demand2})}} & \multicolumn{1}{ c| }{\multirow{1}{*}{$  \frac{N}{(N-1)}\left[\mathcal{ER}\frac{(\gamma-1)}{(\gamma-\epsilon_{\vartheta_j}\beta)}-E_{\vartheta_j}\right]$}}\\ \cline{4-8}
%%%
%\multicolumn{1}{|c|}{} & \multicolumn{1}{c|}{} & \multicolumn{1}{c|}{} & \multicolumn{2}{ c|| }{\multirow{1}{*}{$<thr_e$}} & \multicolumn{1}{ c| }{\multirow{1}{*}{$1$}} & \multicolumn{1}{ c|| }{\multirow{1}{*}{$1$}} & \multicolumn{1}{ c| }{\multirow{1}{*}{$N\left[[r+(1-2r)\vartheta_i]E_{\vartheta_i}+[r+(1-2r)\vartheta_j]E_{\vartheta_j}\right]$}} \\ \cline{4-8}
%%%
%\multicolumn{1}{|c|}{} & \multicolumn{1}{c|}{} & \multicolumn{1}{c|}{} & \multicolumn{2}{ c|| }{\multirow{1}{*}{$>thr_f$}} & \multicolumn{1}{ c| }{\multirow{1}{*}{$1$}} & \multicolumn{1}{ c|| }{\multirow{1}{*}{$0$}} & \multicolumn{1}{ c| }{\multirow{1}{*}{$[r+(1-2r)\vartheta_i]NE_{\vartheta_i}$}}\\ \cline{1-8}
%%%%
%\end{tabular}
%\end{center}
%\label{table_eq_comp_pro}
%\end{table*}
%\normalsize



% leverage the absence of competitors of $\vartheta_i$ energy profile and opt for the RES, if $\frac{(\gamma-1)}{(\gamma-\epsilon_j\beta)} \geq \frac{E_{\vartheta_j}}{rse^{PA}_{\vartheta_i}(\cdot)}$, and follow the rest of consumers in the safer option avoiding risking, otherwise.

%If $E_0 > \mathcal{ER}\frac{(\gamma-1)}{(\gamma-\epsilon_0\beta)}$, it is dominant strategy for all consumers of low-to-moderate energy consumption profile to defer from competing for the renewable energy source, namely $c_{i}^{(\cdot)}(RES,p)>c_{i}^{(\cdot)}(nonRES,p)$, $\forall p$ and $\forall i\in \mathcal{N}: \vartheta_i=0$. Similarly, if $E_1 > \mathcal{ER}\frac{(\gamma-1)}{(\gamma-\epsilon_1\beta)}$, it is dominant strategy for all consumers of high energy consumption profile to defer from competing for the renewable energy source, namely $c_{i}^{(\cdot)}(RES,p)>c_{i}^{(\cdot)}(nonRES,p)$, $\forall p$ and $\forall i\in \mathcal{N}: \vartheta_i=1$.

%\textbf{Demand:} relations and risk-aversion degree values examined above.

%under equilibrium strategy profiles shaped by the competing probabilities in intervals (\ref{eq:prop_alloc_eq_bounds_0_extra_demand}) and (\ref{eq:prop_alloc_eq_bounds_1_extra_demand}) becomes

%\vspace{-5pt}
%\footnotesize
%\begin{eqnarray}
% \hspace{-10pt} D^{PA,NE}(p^{PA,NE}) \hspace{-5pt} &=&  \hspace{-5pt} N\left[ r  p^{PA,NE}_{RES,0} E_0+(1-r)  p^{PA,NE}_{RES,1} E_1\right] \nonumber \\
% \hspace{-5pt} &=& \hspace{-5pt} \frac{N}{(N-1)}  \left[ \mathcal{ER}\frac{(\gamma-1)}{(\gamma-\epsilon_{0}\beta)}-E_0 \right]
%\label{eq:demand_pa_ne_extra_demand}
%\end{eqnarray}
%\normalsize


%In Section \ref{sec:coordinated}, we describe conditions on pricing and energy consumption profiles that cause inefficiencies.

%when the renewable energy demand does not match the supply, that is, $D^{PA,NE}(p^{PA,NE}) \neq \mathcal{ER}$.




Finally, in the special case of risk-seeking consumers, the social cost takes an interesting form because it becomes independent of the probability $p^{PA,NE}$. This is independent of the risk aversion degrees and energy demand levels, i.e., it holds for all the four cases above. Specifically,

\small
\begin{align}
&C^{PA,NE}= min(\mathcal{ER}, D^{PA,NE}) c_{RES} +  max(0,D^{PA,NE}-\mathcal{ER})\cdot  \nonumber \\ &c_{nonRES,D} 
+ \left[ D^{Total} -D^{PA,NE} \right] c_{nonRES,N},
 \label{eq:social_cost_pa_sc}
 \end{align}
\normalsize
where $D^{PA,NE} $ is given by Eq. \eqref{eq:demand1} or Eq. \eqref{eq:demand2} with $\epsilon_{\vartheta_i}=1$.



