\vspace{-0.1in}\section{Analysis of the Decentralized Energy Sharing Mechanism}\label{sec:gameanalysis}

%This behavior represents consumers with a broad range of flexible loads, including shiftable appliances, such as washing machines that do not need to run every day, as well as EVs, water heaters, and batteries that do not need to be fully charged at the end of a given day. For instance EV owners would compute their minimum (inflexible) daily energy load, representing the energy needed to cover their transportation needs for the day, as well as their flexible energy load, representing the additional energy needed to fully charge their EV. Then, if they decide to compete for RESs, they may be willing to engage only part of their daily flexible load to mitigate the risk of paying for the high-priced peak-load production. At the beginning of the following day that they play this game, they would update their daily inflexible and flexible loads and their risk attitude based on their new state-of-charge and transportation needs.

In this section, we study analytically the uncoordinated decisions of the self-interested consumers participating in the proposed D-ESM. In the following, we study the conditions on the parameter values for the existence of dominant strategies and mixed-strategy NE under the proposed PA and payment policies, and provide closed-form formulations of these equilibrium states, i.e., ranges on the values of the vector of mixed strategies at NE, denoted as $\mathbf{p^{NE}}$ and an analytical expression on the value of the expected aggregate daytime energy demand. The proofs of the theoretical results presented below are available in the {Appendix \ref{sec:proofsESSG}} of \cite{arxiv_version}.

First, we recall that, for a mixed-strategy NE to exist, the expected costs of each consumer for all pure strategies in the support of the mixed-strategy NE must be equal. Using the expressions of the costs in \eqref{eq:RES_cost} and \eqref{eq:nonRES_cost}, we obtain that the amount of RESs allocated to a consumer type $\vartheta \in \Theta$ at a NE must satisfy:

 \vspace{-0.1in} 
\small
\begin{equation}\label{eq:conditionEQ_extra_demand}
res^{NE}_{\vartheta}(\mathbf{p^{NE}}) = \frac{\gamma-\varepsilon_{\vartheta}\beta}{\gamma-1}E_{\vartheta},~ \forall \vartheta \in \Theta.
\end{equation}
\normalsize 
\vspace{-0.1in}

\noindent Thus, in the ESG, a mixed-strategy NE exists under the condition:

 \vspace{-0.1in} 
 \small
\begin{equation}\label{eq:condition_PA_NE}
res_{\vartheta}^{PA}(\mathbf{p}^{NE}) = res_{\vartheta}^{NE}(\mathbf{p}^{NE}), ~\forall \vartheta \in \Theta, \end{equation}
\normalsize 
\vspace{-0.15in}

\noindent where $res_{\vartheta}^{PA}(\mathbf{p}^{NE})$ is defined in \eqref{eq:prop_alloc_energy}. In the following analysis, we obtain the mixed-strategy NE competing probabilities $\mathbf{p}^{NE}$ by solving Equation \eqref{eq:condition_PA_NE}. We further distinguish cases with respect to the available RESs production, TOU tariffs, and consumers' types.


\subsubsection*{\textbf{Case $1$: $\bm{\mathcal{RE}}$ exceeds $\bm{D^{Total}}$}}

As consumers have knowledge of $\mathcal{RE}$ and $D^{Total}$, it is straightforward to show that the dominant-strategy for all consumers is to schedule their daily flexible loads during daytime. As a result, the competing probabilities that lead to equilibrium states are equal to $p_{\vartheta}^{d,NE} = 1$ for all consumer types $\vartheta \in \Theta$.


\subsubsection*{\textbf{Case $2$: $\bm{\mathcal{RE}}$ is lower than $\bm{D^{Total}}$}} 

In this case, the strategies of the consumers depend on their respective risk aversion degrees and the TOU tariffs. We define two complementary subsets of consumer types, depending on their risk aversion degrees: $\Sigma_1 = \Bigl\{\vartheta \in \Theta : \varepsilon_{\vartheta} \geq \gamma/\beta \Bigr\} \subset \Theta$, and $\Sigma_2 = \Bigl\{ \vartheta \in \Theta :1\leq \varepsilon_{\vartheta} < \gamma/\beta \Bigr\} \subset \Theta$.


Firstly, the dominant strategy for all consumers $i$ whose type $\vartheta_i$ is in the set $\Sigma_1$ is to schedule their daily flexible loads during daytime, i.e., to play the pure strategy $A_i = d$ with probability $p_{\vartheta_i}^{d,NE} = 1$. 
%Their expected aggregate daytime load schedule at NE is $D^{Total}_{\Sigma_1£\mathcal{S}$=N\sum_{\theta \in \Sigma_1}r_{\theta} E_{\theta}$.

Secondly, the strategies of the consumers $i$ whose type $\vartheta_i$ is in the set $\Sigma_2$ depend on their daily flexible loads and risk-aversion degrees. We define two distinct subsets of consumer types in $\Sigma_2$: $\Sigma_{2,1} = \left\{ \vartheta \in \Sigma_2 : E_{\vartheta} > \mathcal{RE}\frac{(\gamma-1)}{(\gamma-\varepsilon_{\vartheta}\beta)} \right\}$ and $\Sigma_{2,2} = \left\{ \vartheta \in \Sigma_2 : E_{\vartheta} \leq \mathcal{RE}\frac{(\gamma-1)}{(\gamma-\varepsilon_{\vartheta}\beta)}\right\}$.

%$\Sigma_{2,1} = \left\{ \vartheta \in \Sigma_2 : E_{\vartheta} > \left(\mathcal{RE}-D^{Total}_{\Sigma_1}\right)\frac{(\gamma-1)}{(\gamma-\varepsilon_{\vartheta}\beta)} \right\}$ and $\Sigma_{2,2} = \left\{ \vartheta \in \Sigma_2 : E_{\vartheta} \leq \left(\mathcal{RE}-D^{Total}_{\Sigma_1}\right)\frac{(\gamma-1)}{(\gamma-\varepsilon_{\vartheta}\beta)}\right\}$

For consumers $i$ whose type $\vartheta_i$ is in the set $\Sigma_{2,1}$, the dominant strategy is to schedule their daily flexible loads during nighttime, i.e., to play the pure strategy $A_i=n$ with probability $p^{n,NE}_{\vartheta_i}=1$ and $A_i=d$ with probability $p^{d,NE}_{\vartheta_i}=0$.

For consumers $i$ whose type $\vartheta_i$ is in the set $\Sigma_{2,2}$, a mixed-strategy NE under the PA policy exists if and only if the following condition holds:

 \vspace{-0.1in} 
 \small
\begin{equation}\label{eq:relation_E_0_E_1_pa_ne_extra_demand}
\begin{split}
& \mathcal{RE}\frac{(\gamma-1)}{(\gamma-\varepsilon_{\vartheta}\beta)}-E_{\vartheta}  = \mathcal{RE}\frac{(\gamma-1)}{(\gamma-\varepsilon_{\tilde{\vartheta} }\beta)}-E_{\tilde{\vartheta}} , \ \forall \vartheta , \tilde{\vartheta} \in \Sigma_{2,2}.
\end{split}
 \end{equation}
\normalsize 
\vspace{-0.1in}

\noindent Assuming that all consumers of the same type play the same mixed strategy, the competing probabilities that lead to NE states lie in the range $p^{min}_{\vartheta} \leq p^{d,NE}_{\vartheta} \leq p^{max}_{\vartheta}$ for all consumer types $\vartheta \in \Sigma_{2,2}$ with:

 \vspace{-0.1in} 
 \footnotesize
   \begin{align}
& p^{max}_{\vartheta} = \min \left\{1,\frac{\frac{\mathcal{RE}(\gamma-1)}{(\gamma-\varepsilon_{\vartheta}\beta)}-D^{Total}_{\Sigma_1}}{N~ r_{\vartheta}~ E_{\vartheta}}\right\},
    \label{eq:plmaxbound} \\
&  p^{min}_{\vartheta}= \max \left\{0,\frac{\frac{\mathcal{RE}(\gamma-1)}{(\gamma-\varepsilon_{\vartheta}\beta)}-D^{Total}_{\Sigma_1 \bigcup \Sigma_{2,2}\setminus \{\vartheta\}}}{N~ r_{\vartheta}~E_{\vartheta}} \right\},\label{eq:plminbound}
  \end{align}
\normalsize  
where for any subset of consumer types $\mathcal{S} \subset \Theta$, $D^{Total}_{\mathcal{S}}$ represents the maximum aggregate daytime demand of consumers whose type is in $\mathcal{S}$, e.g., $D^{Total}_{\Sigma_1}=N\sum_{\theta \in \Sigma_1}r_{\theta} E_{\theta}$.


 As a result, the expected aggregate daytime demand, $D^{d,NE}$ at NE is

\vspace{-0.1in} 
\footnotesize
\begin{align}
& D^{d,NE} = D^{Total}_{\Sigma_1} \nonumber \\
& + \min \left\{ D^{Total}_{\Sigma_{2,2}} , \max \left\{\frac{N\left(\frac{\mathcal{RE}(\gamma-1)}{(\gamma-\varepsilon_{\vartheta}\beta)}-E_{\vartheta}-D^{Total}_{\Sigma_1}\right)}{(N-1)},0 \right\} \right\}. \label{eq:demand1_2c}
\end{align}
\normalsize 
%\vspace{-0.1in}

\begin{remark} \label{rem:risk_degrees_relation}
Note that condition \eqref{eq:relation_E_0_E_1_pa_ne_extra_demand} can hold, and therefore a NE can exist, only if for any pair $\vartheta , \tilde{\vartheta} \in \Sigma_{2,2}$ such that $\vartheta \leq \tilde{\vartheta}$, it holds that $\varepsilon_{\vartheta} \leq \varepsilon_{\tilde{\vartheta}}$. Since by assumption, $E_{\vartheta} \leq E_{\tilde{\vartheta}}$, this means that consumers with lower daytime energy demand levels should be more risk-seeking than those with higher ones.
\end{remark}
%\vspace{-0.15in}
\begin{remark}\label{rem:risk_seeking}In particular, if all consumers whose type is in $\Sigma_{2,2}$ are risk-seeking (i.e., $\varepsilon_{\vartheta}= 1, \forall \vartheta \in \Sigma_{2,2}$), a NE can only exist if  $E_{\vartheta}=E_{\tilde{\vartheta}} , \ \forall \vartheta , \tilde{\vartheta} \in \Sigma_{2,2}$.
\end{remark}

%Finally, the social cost is given by:
%\vspace{-0.15in} 

%\begin{small}
	%\begin{align}
	%C(\mathbf{p}^{NE}) &=  \min\{\mathcal{RE}, D^d(\mathbf{p^{NE}})\} c^{RES} \nonumber \\&+  \max\{0,D^d(\mathbf{p^{NE}})-\mathcal{RE}\}c^{grid,d} \nonumber \\
	%&+N \left[ \sum_{\vartheta \in \Theta} r_{\vartheta }~ p^{n,NE}_{\vartheta}~ \varepsilon_{\vartheta }~E_{\vartheta}\right] c^{grid,n}.
	%\label{eq:social_cost_pa_extra_demand}
	%\end{align}
%\end{small}


