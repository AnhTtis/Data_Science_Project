\documentclass[11pt]{amsart}
\usepackage{amsmath, amsthm, amssymb}
\usepackage{amsmath,amscd}
\usepackage{mathabx}
\usepackage[hidelinks]{hyperref}
\usepackage{mathrsfs}
\usepackage{xcolor, changepage} 
\usepackage{graphicx}
\usepackage{titletoc}
\usepackage{enumerate}
\usepackage{accents}

\usepackage{caption}
\usepackage{subcaption}

\usepackage{geometry}
\geometry{
	a4paper,
	total={210mm,297mm},
	left=32mm,
	right=25mm,
	top=32mm,
	bottom=32mm,
}



\theoremstyle{plain}
\newtheorem{theorem}{Theorem}
\newtheorem{proposition}[theorem]{Proposition} 
\newtheorem{lemma}[theorem]{Lemma}
\newtheorem{remark}[theorem]{Remark}
\newtheorem{corollary}[theorem]{Corollary}
\newtheorem{definition}[theorem]{Definition}
\newtheorem{example}[theorem]{Example}
\newtheorem{prob}[theorem]{Problem}
\numberwithin{theorem}{section}


\def\red{\color{red}}
\def\green{\color{green}}
\def\blue{\color{blue}}


\numberwithin{equation}{section}


\title[Twisted $S^1$ stability and PSC obstruction on fiber bundles]{Twisted $S^1$ stability and positive scalar curvature obstruction on fiber bundles}
\author{Shihang He}
\address{Key Laboratory of Pure and Applied Mathematics, School of Mathematical Sciences,
Peking University, Beijing, 100871, P. R. China}
\email{hsh0119@pku.edu.cn}


\begin{document}
	
	\maketitle
	
	%\renewcommand{\Abstract}{}
	\begin{abstract}
		We establish several non-existence results of positive scalar curvature (PSC) on fiber bundles. We show under an incompressible condition of the fiber, for $X^m$ a Cartan-Hadamard manifold or an aspherical manifold when $m=3$, the fiber bundle over $X^m\#M^m$ ($m\ge 3$) with $K(\pi,1)$ fiber, $NPSC^+$(a manifold class including enlargeable and Schoen-Yau-Schick ones) fiber, or spin fiber of non-vanishing Rosenberg index carries no PSC metric, with necessary dimension and spin compactible condition imposed. Furthermore, we show under a homotopically nontrivial condition of the fiber, the $S^1$ principle bundle over a closed 3-manifold admits PSC metric if and only if its base space does. These partially answer a question of Gromov and extend some previous results of Hanke, Schick and Zeidler concerning PSC obstruction on fiber bundles.
	\end{abstract}
\tableofcontents	
	\section{Introduction}
	
	$S^1$ factor plays important roles in the study of geometry and topology of positive scalar curvature (PSC).  The famous Georch conjecture, resolved in \cite{ref24} by Gromov-Lawson and in \cite{ref21} by Schoen-Yau, states $T^n$, completely constructed by $S^1$ factor, is such a typical manifold which admits no PSC metric. One of the major advantage of the $S^1$ factor is that it can be used in the descending arguement combining with the minimal hypersurface technique. The descending arguement was first proposed by Schoen-Yau in \cite{ref21}. They observed a stable minimal hypersurface in the codimension 1 homology class of a PSC manifold must have positive Yamabe quotient. This codimension 1 homology class is always assigned to be the Poincare Dual of 1-dimensional cohomology class, which can be regarded as $S^1$ factor in an abstract sense. Later in \cite{ref25}, Schick noted by taking cup product of such $S^1$ factor $(n-2)$ times would yield complete topological condition to obstruct the existence of a PSC metric. By this observation, he gave counterexample of the unstable Gromov-Lawson-Rosenberg conjecture. The construction in \cite{ref25} was later generalized to the notion of Schoen-Yau-Schick (\textit{SYS}) manifold.
	
	The $S^1$ factors are also widely used in geometric problems related to positive scalar curvature. It was first observed in \cite{ref4} by Fischer-Colbrie and Schoen that an appropriate warped product of the stable minimal hypersurface with $S^1$ carries PSC metric, it was then developed by Gromov and Lawson to a torical symmetrization technique to estimate the  Lipshitz constant of non-zero degree map from  a PSC manifold to a fixed model in \cite{ref6}. Recently, Gromov observed such symmetrization method can be used to prove width estimate of Riemannian band in his notable paper \cite{ref7}, and many breakthroughs have been made following this strategy recently (For example, see \cite{ref3}\cite{ref9}\cite{ref26}\cite{ref27}).
	
	In \cite{ref18}, Rosenberg made the following stability conjecture for $S^1$ multiplication: {\it a closed manifold $M$ admits no PSC metric if and only if $M\times S^1$ admits no PSC metric, provided the dimension of $M$  not equal to 4,} where the requirement of dimension not being $4$ is due to a counterexample coming from the Seiberg-Witten invariant (c.f.\cite{ref19}). In \cite{ref16}, by using the $\mu$-bubble arguement, R{\"{a}}de completely solved this conjecture in dimension no greater than 7 by establishing a width inequality for Riemannian bands obtained from the product of general closed manifold and the closed interval. In fact, the following result is proved in \cite{ref16} (Corollary 2.25): {\it Let $n\in\lbrace 2,3,4,6,7\rbrace$ and $Y^{n-1}$ be a closed connected oriented manifold which does not admit a metric of PSC. If $X = Y\times\lbrack -1,1\rbrack$ and $g$ is a Riemannian metric on $X$ with $Sc(X,g)\ge n(n-1)$, then $width(X,g)<\frac{2\pi}{n}$.} By using this, one can  verify Rosenberg's stability conjecture in dimension no greater than $7$.
	
	However, all of the above constructions only make sense in product case, that is, the trivial $S^1$ bundle case. In 1983,  B{\'{e}}rard-Bergery studied the geometry of manifolds with $S^1$-equivariant PSC metrics in \cite{ref1}. Among other things, he got a necessary and sufficient condition for the existence of a $S^1$-invariant PSC metric on a compact manifold with free $S^1$-action (See Theorem C of \cite{ref1}).  As a direct corollary, $S^1$ bundles over a closed manifold with a PSC metric must admit a PSC metric. With those facts in mind, it is natural to ask the problems:
		\begin{prob}\label{mainprob}
			Let $B$ be compact manifold which admits no PSC metric, $E$ be a $S^1$-bundle over $B$, when does $E$ admit no PSC metric? More generally, if $E$ is a fiber bundle with fiber $F$, where $F$ admits no PSC metric, when does $E$ admits no PSC metric?	
		\end{prob}
		Clearly, Problem \ref{mainprob} generalizes the Rosenberg stability conjecture, and its answer may not always be affirmative. To go further, let $B$ be as in  Problem \ref{mainprob},  we call $B$ has the \textsl{twisted stabilized property} if all  $S^1$ bundle over $B$ carries no PSC metric. Therefore, we have the following more specific problem:
		
		\begin{prob}\label{proba}
			Let $B$ be a closed manifold, when does it have twisted stabilized property?
		\end{prob}
  
		In \cite{ref10} and \cite{ref22}, by developing a series of equivariant surgery theorems and equivariant bordism theorems, several existence results of PSC on $S^1$ manifolds are established. However, to our knowledge, there are not so many results concerning non-existence result of PSC on non-trivial bundle. The following example presents some progress on Problem \ref{mainprob}
  
 \begin{example}\label{eg1}
 $\quad$
 
     (a) (\cite{HPS15}, Corollary 4.5) Let $E$ be a $F$-bundle over $B$, where $E$ is spin, $F$ has non-vanishing Rosenberg index and $B$ is a closed surface with $B\ne S^2, \mathbb{R}\mathbb{P}^2$. Then $E$ has non-vanishing Rosenberg index and hence carries no PSC metric.
    
     (b) (\cite{Zei}, Theorem 1.5) Let $E$ be a $F$-bundle over $B$, where $E$ is spin, $F$ has non-vanishing Rosenberg index and $B$ is an aspherical manifold whose fundamental group has finite asymptotic dimension. Then $E$ has non-vanishing Rosenberg index and hence carries no PSC metric.
     
     (c) (\cite{HS06}, Proposition 6.1) Let $E$ be a $F$-bundle over $B$, where $E$ is spin, $F$ and $B$ are enlargeable spin manifold of even dimension. If the short exact sequence
     \begin{align*}
         0\longrightarrow\pi_1(F)\longrightarrow\pi_1(E)\longrightarrow\pi_1(M)\longrightarrow 0
     \end{align*}
     splits as
     \begin{align*}
         0\longrightarrow\pi_1(F)\longrightarrow\pi_1(F)\times\pi_1(B)\longrightarrow\pi_1(B)\longrightarrow 0
     \end{align*}
     
     Then $E$ has non-vanishing Rosenberg index and hence carries no PSC metric.

     (d) Let $E$ be a fiber bundle over $B$ with fiber $F$ and $dimE\le 5$, where both $B$ and $F$ are $K(\pi,1)$ manifolds. Then $E$ admits no $PSC$ metric. In particular, the conclusion is  true for the case that $F = S^1$.

     (e) (\cite{ref24}, Proposition 4.3, \cite{ref7}, p.658, Example(c)) Let $E$ be a fiber bundle over $B$ with fiber $F$, where both $B$ and $F$ are Cartan-Hadamard manifolds. Then $E$ is enlargeable and hence admit no PSC metric. In particular, the conclusion is  true for the case that $F = S^1$.
     
 \end{example} 

  Among these, (a)(b)(c) are obtained from the index theory, where (b) is the generalization of (a). (d) is a consequence of recent progress of the aspherical conjecture in dimension no greater than 5 in \cite{ref3} and \cite{ref9}. Indeed, the total spaces in both (d) and (e) are aspherical. Therefore, the answer to Problem \ref{proba} should be positive if the base manifold is Cartan-Hadamard manifold or aspherical manifold of dimension no greater than $4$. However, the following example in (see Example 3.3 in \cite{ref20}) shows that the answer to Problem \ref{mainprob} may not be positive in general cases. Namely, we have
	
	
	\begin{example}\label{eg2}
		Let $E$ be a $S^1$ bundle over the $K_3$ surface with non-divisible Euler class. It's not hard to show $E$ is simply connected, hence carrying PSC metric due to \cite{ref5}. However, it's well known that the $K_3$ surface does not carry PSC metric by the $\hat{A}$-genus obstruction.
	\end{example}
	
	Example \ref{eg2} illustrates that neither of the property of having non-vanishing Rosenberg index or the property of carrying no PSC metric is stable under constructing $S^1$ bundle over certain manifold. This makes the setting of twisted bundle absolutely different from that of the trivial ones and Problem \ref{mainprob} interesting to study. To state our main result, we introduce a $NPSC^+$ class of manifold, which consists of closed manifold with the property that any manifold with degree 1 map to such manifold admits no PSC metric. Clearly, enlargeable manifold, Schoen-Yau-Schick (\textit{SYS}) manifold (for definition, see \cite{ref6} Sec.5, \cite{ref7} Sec.5) of dimension no greater than 7 and aspherical manifold of dimension no greater than 5 belongs to the class $NPSC^+$ as a result of  \cite{CLL}\cite{ref6}\cite{ref21}. We will show:
	
	
	\begin{theorem}\label{thm2}
		Let $E^n$ be a closed manifold which is a fiber bundle over $B^m$ with fiber $F$, $B = X\#M$, where $X$ is Cartan-Hadamard or an $K(\pi,1)$ manifold when $m=3$, satisfying one of the following properties:
  
  (1) $F$ is $K(\pi,1)$, $n\le 7$, $n\le m+5$.
  
  (2) $F$ is in the class $NPSC^+$, $E$ is spin or $F$ is totally nonspin, $n\le 7$.
  
  (3) $F$ is spin and has non-vanishing Rosenberg index, $E$ is spin.
  
If the inclusion map $i_*:\pi_1(F)\longrightarrow\pi_1(E^n)$ is injective, then $E^n$ admits no PSC metric. Moreover, in the case (1), any metric on $E^n$ with nonnegative scalar curvature is flat. In particular, the conclusion holds true if $F = S^1$.
	\end{theorem}

 Theorem \ref{thm2} has substantially extended the results in \cite{HS06}\cite{HPS15}\cite{Zei} (Example \ref{eg1} (a)(b)(c)) in two directions. In fact, Theorem \ref{thm2} has generalized the results of \cite{HPS15}\cite{Zei} to all base spaces of interest with dimension no greater than 3, since an oriented closed surface admits no PSC if and only if it is Cartan-Hadamard, while a closed 3-manifold admits no PSC if and only if it contains a $K(\pi,1)$ factor in its prime decomposition. We would like to point out that the incompressible condition here is natural since it automatically holds when the base space is aspherical. 
 
 Another point is that the range of the fiber is substantially extended provided the dimension of the total space does not exceed 7. In the past four decades, most results of PSC topological obstruction were obtained by Dirac operator methods, for instance, the work of \cite{ref6}\cite{Ros83}\cite{St92}. However, these results require certain spin condition on the manifold. Instead of using Dirac operator, our major tool is based on variational method, which is not so widely use in deriving PSC topological obstruction results. This enables us to include the non-spin fiber, as well as certain spin \textit{SYS} fiber whose PSC obstruction could not be detected by any present day Dirac operator method as is shown in \cite{ref25}. The dimension 7 restriction in (1)(2) comes from regularity issue of minimal surface, which is possible to be relaxed owing to recent work \cite{CMS23} and \cite{SY17}.

We would like to make a brief remark here that, what would happen in the trivial bundle (product) case in Problem \ref{mainprob} is not even so clear. In Sec.5 in \cite{ref7}, Gromov discussed that, the product of two enlargeable manifolds, or the product of \textit{SYS} and overtorical manifold admits no PSC metric. However, later he has shown in \cite{Gr21} that there're three \textit{SYS} 4-manifolds whose product does admit PSC metric (for some more examples one could also see \cite{Gr21}). Therefore, in the most general situation, the study of Problem \ref{mainprob} would be extremely difficult to carry out at the present step.\\


	
	We next focus on the case of $S^1$ bundle. We begin with the following general result of enlargeable manifold.
\begin{theorem}\label{thm3.5}
    Let $B$ be a $n$-dimensional enlargeable manifold whose universal covering is hyperspherical (for definition one could see \cite{ref6}, Sec.5) and $E$ a $S^1$ principle bundle over $B$ ($n+1\le 7$). If the fiber of $E$ is incompressible, then $E$ is enlargeable and carries no PSC metric.
\end{theorem}

The condition that the universal covering being hyperspherical would not always be guaranteed by the enlargeability of $B$ as is shown in \cite{BH09}. The incompressible condition is necessary here, since the compressible fiber would lead to the decrease of the macroscopic dimension (for definition of macroscopic dimension, see \cite{Gr96}), which makes the total space hard to be enlargeable. However, for some more concret base manifolds, at least for all 3-manifolds which admit no PSC metric, we can establish the following result for possiblily compressible fiber.
	
	\begin{theorem}\label{thm4}
		Let $B=X\#M$ be a $n$ dimensional closed manifold and $E^{n+1}$ a principle $S^1$ bundle over $B$ ($n+1\le 7$), where $X$ is Cartan-Hadamard or an $K(\pi,1)$ manifold when $n=3$. If the $S^1$ fiber of $E$ is homotopically non-trivial, then $E$ admits no PSC metric. Moreover, the conclusion holds true for manifolds admitting a degree 1 map to $E$, and any metric on $E$ with nonnegative scalar curvature is flat.
	\end{theorem}
	
	As a corollary, we obtain the following class of manifolds with \textit{twisted stabilized property}
	\begin{corollary}\label{cor2}
		Let $B$ be the closed manifold whose universal covering is hyperspherical, and the image of the Hurwitz homomorphism of $\pi_2(B)$ vanishes. Then $B$ has the \textit{twisted stabilized property}. The conclusion also holds true for 3-manifold admiting no PSC metric and contains no $S^2\times S^1$ factor in its prime decomposition.
	\end{corollary}
	
	In \cite{ref7}, Gromov proved a quadratic decay theorem of PSC metric on $B\times\mathbb{R}^2$, where $B$ is an enlargeable manifold. In \cite{ref7}, Gromov raised the question whether this can be generalized to non-trivial bundle cases, that is, whether one could prove the similar thing for $\mathbb{R}^2$ vector bundles over $B$. Obviously, this question is closely related to PSC obstruction on $S^1$ bundles. Our results actually provide partial affirmative answer to this question, especially in dimension 3.

 \begin{corollary}\label{cor3}
     Let $X^5$ be a oriented manifold with $B^3$ as a deformation retract, where $B^3$ is an oriented closed manifold which admits no PSC metric. If the fiber of the boundary of the tubular neighbourhood of $B^3$ (which is a $S^1$ principle bundle over $B^3$) is homotopically non-trivial in $X^5\backslash B^3$, then $X^5$ admits no uniformly PSC metric.
 \end{corollary}
 We leave more details of this question to Sec.5.
 \\
	
	Finally, let us say some words on the difficulties and the main idea in the proof of the main theorems. For the case of $S^1$ bundle, a natural idea is to apply the Schoen-Yau descent arguement \cite{ref21} for the homology class with non-zero intersection with the fiber. However, this requires the fiber to be homologically free, which could not be satisfied in most situations (See Sec.4.3 for more detailed discussion). Another difficulty comes from the base space which is always harder to deal with compared with the fibers, since it cannot
always be embedded into the total space. As one could see, in Example \ref{eg1}, (a)(b)(d)(e) all focus on the case of aspherical base space. In the case of (e), the advantage of aspherical condition forces the pullback bundle on the universal covering to be trivial and one could argue by directly constructing the hyperspherical map. However, in non-aspherical cases like the connected sum $X\#M$, this arguement would definitely fail since there is no way to guarantee the pullback bundle on any covering space is topologically simple.
 
  To show how we overcome these difficulties, we present the main idea appeared in the proof of Theorem \ref{thm2} and Theorem \ref{thm4} respectively. For Theorem \ref{thm2}, we introduce a new conception called \textit{incompressible depth} and obtain its estimate under several cases. With this estimate, we're able to show the fiber bundle can not be too \textit{deep} in some direction, and obtain a contradiction if it carries a PSC metric. This gives the proof of Theorem \ref{thm2}. There are two merits of doing so. On one hand, we can still show the manifold is geometrically very large similar to what has been done in proving a manifold is enlargeable ; On the other hand, the complicated topological twist of the bundle would not impact any more. For Theorem \ref{thm4}, we would obtain a contradiction by showing the $S^1$ bundle is \textit{wide} in some direction and apply the width estimate inequality for \textit{SYS} band in \cite{ref7}. This requires us to verify the \textit{SYS} condition for certain band, in which a key part is to show certain 2-homology class does not lie in the image of Hurwitz homomorphism. However, due to the topological complexity, it can hardly be achieved via direct computation. To achieve our goal, we distinguish the spherical class by counting its intersection with certain codimension 2 submanifold, and show the target homology class does not satisfy certain property. This gives the proof of Theorem \ref{thm4}. \\

    The rest of the paper is arranged as follows. In Sec.2 we develop the conception and estimate of \textit{incompressible depth}. By using the depth estimate developed in Sec.2, we prove Theorem \ref{thm2} in Sec.3. The $S^1$ bundle case would be thoroughly discussed in Sec.4, and the proof of Theorem \ref{thm3.5} and Theorem \ref{thm4} would be presented there. In Sec.5, we study the codimension 2 obstruction of PSC in twisted settings, and give a partial answer to Gromov's question as an application of our results.
	
	\section{$\mu$-bubble and incompressible depth in manifold with boundary}
	The classical incompressible hypersurface obstruction of PSC was first developed in \cite{ref6}\cite{SY1} by Schoen-Yau and Gromov-Lawson. Recent years, this has been generalized in various directions by many authors, see \cite{CRZ}\cite{ref2}\cite{Zei}. Based on these results and the $\mu$-bubble method, we introduce the conception and estimate of incompressible depth in manifold with boundary of uniformly PSC in this section. First, let us recall several basic conception and results of $\mu$-bubble in \cite{ref23}.
	
	\begin{definition}\label{def1}
		A band is a connected compact manifold $M$ together with a decomposition:
		\begin{align*}
			\partial M = \partial M^+ \sqcup \partial M^-
		\end{align*}
		where $\partial_\pm M$ are two collections of boundary components of $M$. If $M$ is endowed with a Riemannian metric, we call such band a Riemannian band and the width of the band is defined to be the distance between $\partial M^+$ and $\partial M^-$.  
	\end{definition}
	
	The following lemma is useful in band width estimate:
	
	\begin{lemma}(\cite{ref23}, Lemma 4.1)\label{lem1}
		Assume $(M,g)$ is a smooth Riemannian band such that width$(M,g) > 2l$, then there exists a surjective smooth map $\phi: (M,g) \longrightarrow [-l,l]$ with $Lip(\phi)<1$ such that $\phi^{-1}(-l) = \partial_-M$ and $\phi^{-1}(l) = \partial_+M$ 
	\end{lemma}
	
	With the function $\phi$ we define $\mu$-bubble functional on $n$-dimensional Riemannian band $M$: For any smooth function $h:(-T,T)\longrightarrow \mathbb{R}$ with $0<T<l$, define:
	
	\begin{align}\label{21}
		\mathcal{A}^h(\Omega) = \mathcal{H}^{n-1}(\partial^*\Omega)-\int_M(\chi_{\Omega}-\chi_{\Omega_0})h\circ \phi d\mathcal{H}^n,  \quad\Omega_0 = \lbrace\phi<0\rbrace
	\end{align}
	where $\Omega$ is any Caccippoli set of $IntM$ with reduced boundary $\partial^*\Omega$ such that:
	\begin{align}\label{22}
		\Omega\Delta\Omega_0 \subset\subset \mathcal{D}(h\circ \phi) = \lbrace -T<\phi<T\rbrace
	\end{align}
	and $\chi_\Omega$ is the characteristic function of region $\Omega$.
	
	The following lemma indicates the existence and regularity of $\mu$-bubble:
	
	\begin{lemma}(\cite{ref23}, Proposition 2.1)\label{lem2}
		Assume that $\pm$T are regular values of $\phi$ and also the function $h$ satisfies
		\begin{align*}
			\lim\limits_{t\to-T}h(t) = +\infty\quad and\quad \lim\limits_{t\to T}h(t) = -\infty
		\end{align*}
		then there exists a smooth minimizer $\hat{\Omega}$ for $\mathcal{A}^h$ such that $\hat{\Omega}\Delta\Omega_0\subset\subset\mathcal{D}(h\circ \phi)$ provided the dimension of the band is no greater than $7$.
	\end{lemma}
	
	The following notion of $\mathcal{C}_{deg}$ comes from \cite{ref2}.
	
	\begin{definition}(\cite{ref2})\label{def2}
		Define $\mathcal{C}_{deg}$ to be the collection of closed aspherical manifold $M$ satisfyting the following property: any manifold which admits a non-zero degree map to $M$ carries no metric of PSC.
	\end{definition}

        For $\mathcal{C}$ a class of closed manifold, we define the $\mathcal{C}$-incompressible depth as follows.
	\begin{definition}\label{def3}
		Let $M$ be a compact Riemannian manifold with boundary. Define $\mathcal{S}_{\mathcal{C}}$ to be the set of embedded two-sided incompressible hypersurfaces $\Sigma$ in $M$ which also belongs to the class $\mathcal{C}$. Define the $\mathcal{C}$ incompressible depth to be:
		\begin{align*}
			\sup_{\Sigma\in \mathcal{S}_{\mathcal{C}}}d(\Sigma,\partial M)
		\end{align*}
		That is the maximal possible value from an incompressible hypersurface belonging to $\mathcal{S}_{\mathcal{C}}$ to the boundary of $M$.
	\end{definition}
	
	We first prove the following estimate of the $\mathcal{C}_{deg}$-incompressible depth in manifold with boundary.
	
	\begin{proposition}\label{thm6}
		Let $M$ be a Riemannian manifold with boundary such that $\mathcal{S}_{\mathcal{C}_{deg}}\ne\varnothing$, with scalar curvature greater than or equal to $1$, then the $\mathcal{C}_{deg}$ incompressible depth of $M$ is no greater than $\pi$.
	\end{proposition}

	\begin{proof}
		Assume the statement in the proposition is not true, then we can find a compact Riemannian manifold $M$ with boundary with $Sc(M)\ge 1$, a hypersurface $\Sigma$ in the class $\mathcal{S}_{\mathcal{C}_{deg}}$ in Definition \ref{def3}, such that:
		\begin{align*}
			d(\Sigma,\partial M) > l > \pi
		\end{align*}
		Since $\pi_1(\Sigma)\longrightarrow\pi_1(M)$ is injective, we can find a covering $\tilde{M}$ of $M$ such that $\pi_1(\Tilde{M}) = \pi_1(\Sigma)$, the distance between $\Sigma$ and the boundary of $\tilde{M}$ is still larger than $l$. The inclusion map $\Tilde{i}$ then induces an isomorphism $\Tilde{i}_*:\pi_1(\Sigma)\longrightarrow\pi_1(\Tilde{M})$. Since $\Sigma$ is aspherical, by \cite{ref11} (Theorem 1B.9), there is a continuous map $j$ from $\Tilde{M}$ to $\Sigma$, such that the map $j_*:\pi_1(\Tilde{M})\longrightarrow\pi_1(\Sigma)$ equals to $\Tilde{i}_*^{-1}$. Therefore, $j_*\circ i_*$ is the identity map on $\pi_1(\Sigma)$. Again by \cite{ref11} (Theorem 1B.9) we conclude $j\circ i$ is homotopic to identity.
		
		We next show $\Sigma$ separates $\tilde{M}$ into two parts. Otherwise, $\tilde{M}\backslash \Sigma$ would be connected. Therefore, there is a closed curve $\gamma$ which intersects $\Sigma$ exactly once. On the other hand, since the map $\tilde{i}_*:\pi_1(\Sigma)\longrightarrow\pi_1(\tilde{M})$ is an isomorphism, there exists a curve $\alpha$ in $\Sigma$ whose image under $\tilde{i}$ is homotopic to $\gamma$. Notice $\Sigma$ is two sided in $\tilde{M}$, we can perturb $\alpha$ slightly such that the intersection of $\alpha$ and $\Sigma$ equals zero. However, the intersection number is invariant under homotopy, which leads to a contradiction.
		
		Let $d_0$ be the smooth approximation of the signed distance function from a single point to $\Sigma$, the sign with respect to the two connected components of $\tilde{M}\backslash \Sigma$. Let $\pi<T<l$ be a number such that $\pm T$ are the regular value of $d_0$. Then
		\begin{equation}\label{23}
		 \begin{split}
		     &V = \lbrace x: -T\le d_0 \le T\rbrace\\
       \partial_+V &= d_0^{-1}(T),\quad \partial_-V = d_0^{-1}(-T)
		 \end{split}   
		\end{equation}
		is a compact Riemannian band, since:
		\begin{align}
			V = \lbrace exp_z(tN(z)): -T\le t\le T, z\in \Sigma\rbrace
		\end{align}
		where $N(z)$ represents the normal vector of $\Sigma$ at $z$.
		
		We can now use the $\mu$-bubble to derive a contradiction. Define
		\begin{align*}
			\mu = \tan(\frac{1}{2}d_0)
		\end{align*}
		on $V$. Here we note $d_0$ serves as $\phi$ in Lemma \ref{lem2}. Consider the $\mu$-bubble functional as in (\ref{21}) and (\ref{22}):
		\begin{align*}
			\mathcal{A}(\Omega) = \mathcal{H}^{n-1}(\partial^*\Omega)-\int_V(\chi_{\Omega}-\chi_{\Omega_0})\mu d\mathcal{H}^n,  
		\end{align*}
		
		By the existence and regularity result Lemma \ref{lem2}, there is a seperating hypersurface $W$ in $V$, such that $W$ is the reduced boundary of a stable minimizer of the functional (\ref{21}). By the first and second variation formula, we have:
		
		\begin{align*}
			D\mathcal{A}(\psi) = \int_{W}(H-\mu)\psi d\sigma = 0
		\end{align*}
		\begin{align*}
			&D^2\mathcal{A}(\psi,\psi) \\
			&= \int_{W}|\nabla\psi|^2 +(H^2-|A|^2-Ric(N,N)-\mu H-N(\mu))\\
			&= \int_{W}|\nabla\psi|^2 +(\frac{1}{2}H^2-\frac{1}{2}|A|^2+\frac{1}{2}(Sc(W)-Sc(V))-\mu H-N(\mu))\psi^2d\sigma\\
			&\ge 0
		\end{align*}
		Since the first variation vanishes, we obtain $H=\mu$ on $\Sigma$. Notice we also have the estimate:
		\begin{align*}
			N(\mu) = \langle N, \nabla \mu \rangle \ge -||\nabla\mu|| \ge -\frac{1}{2}\sec^2(\frac{d_0}{2})
		\end{align*}
		Therefore
		\begin{align*}
			0 \le &D^2\mathcal{A}^h(\psi,\psi)\\
			\le &\int_{W}|\nabla\psi|^2-\frac{1}{2}(|A|^2-Sc(W)+\mu^2+2N(\mu)+Sc(V))\psi^2d\sigma\\
			\le &\int_{W}|\nabla\psi|^2+\frac{1}{2}Sc(W)\psi^2d\sigma
		\end{align*}
		Then following a conformal transformation arguement in \cite{ref21}, $W$ admits a PSC metric. On the other hand, $W$ and $\Sigma$ are homologuous as chain since they are both cobordant to $\partial_-V$ in $V$. Therefore,
		\begin{align*}
			deg(j)|_W = deg(j)|_{\Sigma} = deg(j\circ i) = 1 \ne 0
		\end{align*}
	which contradicts with the fact that $W$ admits a PSC metric.
	\end{proof}

Denote $\mathcal{N}$ to be the collection of manifold admiting no PSC metric, $\mathcal{N}_{nspin}$ the collection of manifold in $\mathcal{N}$ which are totally non-spin (the universal covering is non-spin) and $\mathcal{R}^{\ne 0}$ the collection of spin manifold with non-vanishing Rosenberg index. We next focus on the incompressible depth estimate for manifolds in these new defined classes.

 \begin{proposition}\label{pro2}
     Let $M$ be a $n$-dimensional spin Riemannian manifold with boundary such that $\mathcal{S}_{\mathcal{N}}\ne\varnothing$ ($n=6,7$), with scalar curvature greater than or equal to $1$, then the $\mathcal{N}$ incompressible depth of $M$ is no greater than $\pi$.
 \end{proposition}
 \begin{proof}
     Let $M_0$ be the manifold obtained by pasting $M$ and $\partial M\times\lbrack 0,\infty)$ along the boundary. Assume there is an incompressible hypersurface $\Sigma$ in $M$ with depth greater than $\pi$, then $\Sigma$ is also incompressible in $M_0$. By \cite{CRZ} (Proposition 4.6) there is a covering $\Tilde{M_0}$ of $M_0$ with $\pi_1(\Tilde{M_0})=\pi_1(\Sigma)$, such that $\Sigma$ separates the \textit{open band} (for the definition see \cite{CRZ}) $\Tilde{M_0}$, and any hypersurface separating $\Tilde{M_0}$ admits no PSC metric. Let $V$ be as (\ref{23}), Then any hypersurface $\Sigma_1$ separating $V$ also seperates $\Tilde{M_0}$. In fact, $\Sigma_1$ is homologuous to $\Sigma$. For any curve $\alpha$ connecting two ends in $\Tilde{M_0}$ seperated by $\Sigma$, the intersection number of $\alpha$ and $\Sigma$ is non-zero, so is the intersection number with $\Sigma_1$. Therefore, $\Sigma_1$ admits no PSC metric.

     Then we proceed with the same arguement of Proposition \ref{thm6}, and obtain the desired contradiction.
 \end{proof}

 \begin{proposition}\label{pro4}
     Let $M$ be a $n$-dimensional Riemannian manifold with boundary such that $\mathcal{S}_{\mathcal{N}_{nspin}}\ne\varnothing$ ($n=6,7$), with scalar curvature greater than or equal to $1$, then the $\mathcal{N}$ incompressible depth of $M$ is no greater than $\pi$.
 \end{proposition}
 \begin{proof}
     The proof is the same as the preceding proposition, with a direct use of the result in \cite{CRZ}.
 \end{proof}

\begin{proposition}\label{pro3}
    Let $M$ be a compact $n$-dimensional spin Riemannian manifold with boundary such that $\mathcal{S}_{\mathcal{R}^{\ne 0}}\ne\varnothing$, with scalar curvature greater than or equal to $1$, then the $\mathcal{R}^{\ne 0}$ incompressible depth of $M$ is no greater than $\pi$.
\end{proposition}
\begin{proof}
    Let $M_0$ be the manifold obtained by pasting $M$ and $\partial M\times\lbrack 0,\infty)$ along the boundary. Suppose there is a metric $g$ on $M$ with $Sc(M)\ge 1$ and an incompressible surface $\Sigma$ with non-vanishing Rosenberg index, such that $d(\Sigma,\partial M)>\pi$. Then $g$ can be extended to a metric $g^+$ on $M_0$, which equals a product metric at infinity, hence the scalar curvature of $g^+$ is bounded from below due to the compactness of $M$. However, by \cite{Zei20} (Theorem 1.4), we have $\inf_{x\in M_0}Sc(g^+) = -\infty$, which leads to a contradiction.
\end{proof}
	
	
	\section{PSC obstructions on fiber bundles: Proof of Theorem \ref{thm2}}
	
	In this section, we focus on the proof of Theorem \ref{thm2}. We divide the proof into two parts, the first part dealing with the nonexistence of PSC metric on $E$ and the second part dealing with the rigidity in the setting of nonnegative scalar curvature. Throughout our proof, $m$ is always used to denote the dimension of the base space and $n$ the dimension of the total space of the fiber bundle.
	
\subsection{The nonexistence part}
	
	We begin with the proof of the following result for bundles over torus, the proof is mainly based on a $\mu$-bubble arguement.
	
	\begin{lemma}\label{lem5}
		Let $E^n$ be a fiber bundle with fiber $F$ over $T^m$ ($n\le 7$), where $F$ is an closed manifold in $NPSC^+$ Then $E^n$ is also in the class $NPSC^+$.
	\end{lemma}
	\begin{proof}
		We conduct induction on $m$. If $m=0$, the conclusion follows from the definition of $NPSC^+$ class. Suppose the conclusion holds true for $m-1$. If the conclusion is not true for $m$, then there exists a Riemannian manifold $X$ such that $Sc(X)\ge 1$ due to the compactness of $X$. Let $f: X\longrightarrow E$ be a map of degree 1. There is a natural covering map $q:T^{m-1}\times\mathbb{R}^1\longrightarrow T^m$. Denote the pullback bundle of $E$ under $q$ by $\Tilde{E}$, then $\phi:\Tilde{E}\longrightarrow E$ is also a covering map. One has the following commutative diagram:
		
		\begin{align*}
			\begin{CD}
				\Tilde{E}     @>\phi>>  E\\
				@VV\Tilde{p}V        @VVpV\\
				T^{m-1}\times\mathbb{R}^1     @>q>>  T^m
			\end{CD}
		\end{align*}
		Let $h:T^{m-1}\times \mathbb{R}^1\longrightarrow \mathbb{R}^1$ be the projection map, denote $S = \Tilde{p}^{-1}(h^{-1}(0))$, which is a $F$ bundle over $T^{m-1}$. By the inductive hypothesis, $S\in NPSC^+$. 
  
  Consider the following diagram:
		\begin{align*}
			\begin{CD}
				\pi_{k+1}(T^{m-1}\times\lbrace 0 \rbrace)     @>>>       \pi_k(F)     @>i_3>>  \pi_k(S) @>j_1>> \pi_k(T^{m-1}\times\lbrace 0 \rbrace)   @>>>   \pi_{k-1}(F)\\
				@VVV                   @VVidV        @VVi_1 V                                  @VVi_2 V             @VVV\\
				\pi_{k+1}(T^{m-1}\times \mathbb{R}^1)     @>>>      \pi_k(F)     @>i_4>>  \pi_k(\Tilde{E})           @>j_2>> \pi_k(T^{m-1}\times \mathbb{R}^1)   @>>>   \pi_{k-1}(F)
			\end{CD}
		\end{align*}
		 An application of the five lemma implies $i_1$ is isomorphism for all $k$(though the typical five lemma is stated for module, the diagram chase in the case of group is essentially the same). By the Whithead's Theorem, $S$ is the deformation retract of $\Tilde{E}$. Therefore, one can find a map $j:\Tilde{E}\longrightarrow S$, such that $j\circ i_1$ is homotopic to identity.
		
		Following the arguement of the proof of Proposition 5.7 in \cite{ref6}, we can pull back the covering $\phi:\Tilde{E}\longrightarrow E$ by $f$. Actually we have the following:
		\begin{align*}
			\begin{CD}
				\Tilde{E}     @>\phi>>  E\\
				@AA\Tilde{f}A        @AAfA\\
				\Tilde{X}     @>\Phi>>  X
			\end{CD}
		\end{align*}
		where $\Phi$ is a covering map, $\Tilde{f}$ a proper map with $deg(\Tilde{f})=deg(f)=1$. Then on $\Tilde{X}$ we have $Sc(\Tilde{X})\ge 1$. Define:
		\begin{align*}
			\Tilde{h} = h\circ\Tilde{p}\circ\Tilde{f}: \Tilde{X}\longrightarrow\mathbb{R}^1
		\end{align*}
		Assume $0$ is a regular value of $\Tilde{h}$ without loss of generality. Denote $\Omega_0 = \lbrace \Tilde{h}<0\rbrace$, $\Tilde{h}^{-1}(0) = \Sigma_0$, and let $d$ be the smooth approximation of the signed distance function to $\Sigma_0$. Define:
		\begin{align*}
			\mu = \tan(\frac{1}{2}d)
		\end{align*}
	We can then solve a stable $\mu$-bubble $\Omega$ as in (2.1), with $\partial\Omega = \Sigma$. By the same arguement as in the proof of Theorem \ref{thm6}, $S$ carries a PSC metric. On the other hand, we consider the map:
		\begin{align*}
			F = j\circ\Tilde{f}: \Tilde{X}\longrightarrow S
		\end{align*}
		Notice that $\Sigma$ and $\Sigma_0$ are homologuous, we have:
		\begin{align*}
			deg(F|_{\Sigma}) = deg(F|_{\Sigma_0}) = deg(\Tilde{f})=1
		\end{align*}
	This contradicts with the fact that $S\in NPSC^+$.
	\end{proof}

\begin{remark}
    The arguement in the proof of the above theorem also applies to the case of \textit{non-zero degree}. However, we hope that our class $NPSC^+$ could include more manifolds like the \textit{SYS} manifolds, so we require the map to be of degree $1$ in our definition of $NPSC^+$.
\end{remark}
 
	\begin{lemma}\label{lemM}
        Let $E^n$ be the bundle in Theorem \ref{thm2} and $B^m = X\#M$, where $X$ is a Cartan-Hadamard manifold. Fix a Riemannian metric on $E$. Then for any given $L_0>0$, there is a region $\mathcal{W}$ in certain Riemannian covering $\Tilde{E}$ of $E$ and an incompressible hypersurface $\partial\mathcal{W}_0$ in $\mathcal{W}$, such that $\partial\mathcal{W}_0$ is a $F$ bundle over $T^{m-1}$, and
        \begin{align}\label{M}
            d(\partial\mathcal{W}_0, \partial\mathcal{W})>L_0
        \end{align}
	\end{lemma}
        \begin{proof}		
		Let $k:B\longrightarrow X$ be a smooth map which collapses $M$ to a point. Fix metrics on $B$ and $X$, such that $Lip(k)\le 1$. Denote the universal Riemannian covering of $X$ by $\Tilde{X}$, this induces a Riemannian covering $\Tilde{B}$ of $B$, such the following diagram commutes:
		\begin{align}\label{31}
			\begin{CD}
				\Tilde{B}     @>\pi>>  B\\
				@VV\Bar{k}V        @VVkV\\
				\Tilde{X}     @>>>  X
			\end{CD}
		\end{align}
		Then
		\begin{align}\label{32}
			Lip(\Bar{k}) = Lip(k) \le 1
		\end{align}
		The covering map $\pi:\Tilde{B}\longrightarrow B$ also induces a pullback bundle of $E$, that is:
		\begin{align}\label{33}
			\begin{CD}
				\Tilde{E}     @>\Tilde{\pi}>>  E\\
				@VVqV        @VVpV\\
				\Tilde{B}     @>\pi>>  B
			\end{CD}
		\end{align}
		with
		\begin{align}\label{34}
			Lip(q) = Lip(p) < \infty
		\end{align}
where $\Tilde{E}$ is the pullback bundle of $E$ under the map $\pi$. Obviously, $\Tilde{E}$ is a covering space of $E$ with incompressible fiber $F$. In fact, if there is an element in $\pi_1(F)$ homotopic to zero in $\Tilde{E}$, then under the covering map $\Tilde{\pi}$, this would give an element in $\pi_1(F)$ homotopic to zero in $E$, which contradicts the fact that $F$ is incompressible in $E$.
		
		
		We then start the depth analysis. We first find a region $\mathcal{V}$ in $\Tilde{X}$, $\mathcal{V}\cong T^{m-1}\times \lbrack 0,1\rbrack$, with:
		\begin{equation}\label{35}
			\begin{split}
				&\partial\mathcal{V} = T^{m-1}\times \lbrace 0\rbrace\cup T^{m-1}\times \lbrace 1\rbrace\\
				&\partial_0\mathcal{V} = T^{m-1}\times \lbrace \frac{1}{2}\rbrace\\
				&d(\partial\mathcal{V},\partial_0\mathcal{V}) > L\\
			\end{split}
		\end{equation}
		for arbitrary large $L$. In fact, since $\Tilde{X}$ is a simply connected Cartan-Hadamard manifold, there is a map $f$ which maps the geodesic ball $B_R(0)$ diffeomorphicly to the unit $D^n$ with flat metric, with $Lip(f)\le \frac{1}{R}$ (though the metric on $\Tilde{X}$ may not be of non-positive sectional curvature, it's $C^0$ equivalent to such a metric). Fix a embedding two sided $T^{m-1}$ in $D^m$ and pick its tubular neibourhood $V\cong T^{m-1}\times\lbrack 0,1\rbrack$, with
		\begin{align*}
			d(\partial V,\partial_0V) > \delta
		\end{align*}
		in $D^n$, where $\delta$ is a fixed small number only rely on dimension $n$. Therefore, $\mathcal{V} = f^{-1}(V)$ satisfies the desired property as long as we choose $R$ arbitrarily large.

We next make the construction in $\Tilde{B}$. The construction in (\ref{31}) indicates that $\Tilde{B}$ is the connected sum of $\Tilde{X}$ and infinite copies of $M$:
		\begin{align*}
			\Tilde{B} = \Tilde{X}\#_{p_1}M\#_{p_2}M\#\dots
		\end{align*}
		where $\#_{p_i}$ means taking connected sum at a small neighborhood near $p_i\in\Tilde{X}$. The map $\Bar{k}:\Tilde{B}\longrightarrow \Tilde{X}$ in (\ref{31}) pinches these $M$'s to the points $p_1,p_2,p_3,\dots$. By applying small perturbation on the hypersurfaces $\partial\mathcal{V},\partial_0\mathcal{V}$, one can assume none of $p_i$ is on $\partial\mathcal{V}$ or $\partial_0\mathcal{V}$. Set
  \begin{align*}
      \mathcal{U} = \Bar{k}^{-1}\mathcal{V} 
  \end{align*}
  Also define:
		\begin{align*}
			&\partial\mathcal{U} = \Bar{k}^{-1}\partial\mathcal{V}\\
			&\partial_0\mathcal{U} = \Bar{k}^{-1}\partial_0\mathcal{V}
		\end{align*}
		Since $\partial\mathcal{V},\partial_0\mathcal{V}$ has no intersection with $p_i$'s, $\partial\mathcal{U},\partial_0\mathcal{U}$ are mapped diffeomorphically to $\partial_-\mathcal{V},\partial_0\mathcal{V}$ under $\Bar{k}$.  This guarentees the components of $\partial\mathcal{U},\partial_0\mathcal{U}$ are torus. Topologically, $\mathcal{U}$ is the connected sum of $T^{m-1}\times\lbrack 0,1 \rbrack$ and finite copies of $M$.    
		
		We need to verify the incompressible condition and depth condition of $\mathcal{U}$. First, recall the basic fact that for manifold $M_1,M_2$ of dimension no less than 3, $\pi_1(M_1\#M_2) = \pi_1(M_1)*\pi_1(M_2)$ by the Van-Kampen's theorem. Therefore, it always holds true that the inclusion map $M_1\backslash D^n \hookrightarrow M_1\#M_2$ induces injection on fundamental group. This implies
		\begin{equation}\label{36}
			\begin{split}
				&\pi_1(T^{m-1}\times\lbrack 0,1\rbrack - \lbrace q_1,q_2,\dots,q_r\rbrace)\\
				\longrightarrow &\pi_1(T^{m-1}\#_{q_1}M\#_{q_2}M\#\dots\#_{q_r}M)\cong\pi_1(\mathcal{U})
			\end{split} 
		\end{equation}
		is injective, where the notation $\#_{q_i}$ means making connected sum on a small neighborhood near the point $q_i$.  Since the middle torus $T^{m-1}\times\lbrace\frac{1}{2}\rbrace$ is incompressible in $T^{m-1}\times\lbrack 0,1 \rbrack$, we have the middle torus $\partial_0\mathcal{U}$ incompressible in $\mathcal{U}$ by (\ref{36}).
		
		Since $Lip(\bar{k})\le 1$ from (\ref{32}), we also have:
		\begin{align*}
			d(\partial\mathcal{U},\partial_0\mathcal{U}) > L
		\end{align*} 
		by (\ref{35}).
		
		Now we can construct the desired region $\mathcal{W}$ in $\Tilde{E}$. Define $\mathcal{W}$ to be the restriction of $\Tilde{E}$ on $\mathcal{U}$. Since $Lip(q) = Lip(p) = C<+\infty$ by (\ref{34}), we have:
		\begin{align*}
			d(\partial\mathcal{W},\partial_0\mathcal{W}) > \frac{L}{C}
		\end{align*}
		where $\partial_0\mathcal{W}$ is the restricted $F$ bundle of $\mathcal{W}$ over $\partial_0\mathcal{U}$. 
  
  For the remaining part it suffice to show $\partial_0\mathcal{W}$ is an incompressible hypersurface in $\mathcal{W}$. We have the following commutative diagram:
		\begin{align}\label{chase}
			\begin{CD}
				\pi_1(F)     @>i_1>>  \pi_1(\partial_0\mathcal{W}) @>j_1>> \pi_1(\partial_0\mathcal{U})\\
				@VVidV        @VV\varphi V                                  @VV\psi V\\
				\pi_1(F)     @>i_2>>  \pi_1(\mathcal{W})           @>j_2>> \pi_1(\mathcal{U})
			\end{CD}
		\end{align}
		All the maps of the vertical arrow are induced by inclusion map. We have already shown $\psi:\partial_0\mathcal{U}\longrightarrow \mathcal{U}$ is injective. We now prove $\varphi$ is injective. Take $a \in \pi_1(\partial_0\mathcal{W})$. If $j_1(a)\ne 0$, then $\psi\circ j_1(a)\ne 0$ due to the injectivity of $\psi$. This indicates $\varphi(a)\ne 0$ since the diagram is commutative. On the other hand, if $j_1(a) = 0$, by the exactness there exists $b$ in $\pi_1(F)$ such that $a = i_1(b)$. Since under the inclusion map $\pi_1(F)\longrightarrow\pi_1(\Tilde{E})$ is injective, it follows under the inclusion map $\pi_1(F)\longrightarrow\pi_1(\mathcal{W})$ is injective, which means $i_2$ in the diagram is injective. Therefore, $\varphi\circ i_1(b) = i_2(b) = 0$ indicates $b = 0$. Hence, $\varphi$ is injective. Since $L$ can be arbitrarily large, $\mathcal{W}$ and $\partial\mathcal{W}_0$ satisfies the condition of the lemma.
\end{proof}


\begin{proposition}\label{proM1}
    $E$ in Theorem \ref{thm2} admits no PSC metric if $X$ is Cartan-Hadamard.
\end{proposition}
 \begin{proof}
 We show $E^n$ admits no PSC metric. Or else, we assume there exists a metric $E$ on $B$ with $Sc(E) \ge 1$ due to the compactness of $E$. Pick $\mathcal{W}$ and $\partial\mathcal{W}_0$ as in Lemma \ref{lemM}. We discuss case by case.
  
   Case 1: $F$ is $K(\pi,1)$, $n\le 7$. By Lemma \ref{lem5} and \cite{CLL}, $\partial_0\mathcal{W}$ belongs to the class $\mathcal{C}_{deg}$ in Definition \ref{def3}. In the sense of the terminology in Sec.2, this means $\mathcal{W}$ has $\mathcal{C}_{deg}$ incompressible depth no less than $\frac{L}{C}$. Since the scalar curvature of $\mathcal{W}$ is no less than $1$ and $L_0$ in \ref{M} can be taken arbitrarily large, we obtain a contradiction due to the $\mathcal{C}_{deg}$ incompressible depth estimate of Proposition \ref{thm6}. 

    Case 2: $F$ is spin and has non-vanishing Rosenberg index, $E$ is spin. It follows from \cite{Zei} (Theorem 1.5) that as a $F$ bundle over $T^{m-1}$, $\partial_0\mathcal{W}$ also has non-vanishing Rosenberg index, and the contradiction follows from the $\mathcal{R}^{\ne 0}$ incompressible depth estimate Proposition \ref{pro3}.
    
  Case 3: $F$ is in the class $NPSC^+$, $E$ is spin or $F$ is totally non-spin, $n\le 7$. We assume $m\ge 2$, since the case that $m=1$ follows from Lemma \ref{lem5}.
  
  (a) $E$ is spin. By Lemma \ref{lem5}, $\partial_0\mathcal{W}$ admits no PSC metric. If $n\ge 6$, a contradiction follows from the $\mathcal{N}$ incompressible depth estimate of Proposition \ref{pro2}. If $n\le 5$, recall closed manifold of dimension no greater than 3 with no PSC metric has non-vanishing Rosenberg index (see \cite{HS06}), the conclusion follows from Case 2.
  
  (b) $F$ is totally non-spin. This can happen only when $n-m\ge 4$. Since $m\ge 2$ we have $n\ge 6$, then by passing to the pullback bundle of the covering map $\mathbb{R}^{m-1}\longrightarrow T^{m-1}$ one could see $\partial_0\mathcal{W}$ is also totally non-spin (Note that the normal bundle of a single fiber is trivial), and the contradiction follows from Lemma \ref{lem5} and Proposition \ref{pro4}.
  \end{proof}

  \begin{proposition}\label{proM2}
      $E$ in Theorem \ref{thm2} admits no PSC metric if $m=3$ and $X$ is $K(\pi,1)$.
  \end{proposition}
  \begin{proof}
 To begin, let's recall the hyperbolisation conjecture, a corollary of the geometrization conjecture (see \cite{MG}), which says a 3-dimensional closed aspherical manifold is either hyperbolic or contains a $\mathbb{Z}\oplus\mathbb{Z}$ subgroup in its fundamental group(see \cite{GT} 12.9.4). By using this we can assume $\pi_1(X)$ contains a subgroup isomorphic to $\mathbb{Z}\oplus\mathbb{Z}$, since the hyperbolic case is already contained in the Cartan-Hadamard case. It follows $X$ contains an incompressible torus, so does $B=X\#M$. By a similar diagram chase as (\ref{chase}), $E$ contains a nice incompressible hypersurface. It follows from \cite{CRZ} (Theorem 1.5), \cite{ref2} (Theorem 1.1), and \cite{Zei} (Theorem 1.7) and the same arguement in Proposition \ref{proM1} that $E$ admits no PSC metric.
  \end{proof}

 \subsection{The rigidity part}
	In this subsection, we deal with the rigidity part of Theorem \ref{thm2}. If there is a metric $g$ on $E$ with nonnegative scalar curvature, it's standard to show $g$ is Ricci flat. In fact, if $g$ is not Ricci flat, one can solve a Ricci flow with initial data $g$ in a short time, which gives a metric on $E$ with positive scalar curvature. This contradicts with the non-existence of PSC metric on $E$. To handle the Ricci flat case, we will analyze the volume growth of certain covering of the bundle and make use of the structure of compact Ricci flat manifold.
	
	\begin{definition}\label{def4}
		For a non-compact Riemannian manifold $M$, define $\mathcal{A}$ to be the set of real number of the following property: for any $\alpha\in\mathcal{A}$, there exists $C = C(\alpha)$, such that for any $p\in M$ and sufficiently large $R$ one has:
		\begin{align*}
			Vol(B(p,R))\ge CR^{\alpha}
		\end{align*}
		where $B(p,R)$ is the geodesic ball centered at $p$ with radius $R$. Define the \textit{volume rate} of $M$ to be:
		\begin{align*}
			r(M) = sup\lbrace\alpha\in\mathcal{A}\rbrace
		\end{align*}
	\end{definition}
	
	\begin{example}\label{lem6}
		For $M=N\times \mathbb{R}^k$, where $N$ is compact and $M$ is the Riemannian product of $N$ and flat Euclidean space, $M$ has volume rate $k$. For an $n$-dimensional Cartan-Hadamard manifold $X$, if $X$ is simply connected, then the volume rate of $X$ is at least $n$ by volume comparison.
	\end{example}
	
	Recall two metrics are $g_1$ and $g_2$ equivalent if there exists constant $C$ such that:
	\begin{align*}
		\frac{1}{C}g_1\le g_2\le Cg_1
	\end{align*}
	It's not hard to show equivalent metrics have the same volume rate. Therefore, we have:
	\begin{lemma}\label{lem7}
		Let $\Tilde{M}$ be a covering space of a closed manifold $M$. Then any two metrics on $\Tilde{M}$ induced from $M$ have the same volume rate.
	\end{lemma}
	Moreover, we would like to present fundamental relation of volume rate between two manifolds where one is the covering of the other:
	\begin{lemma}\label{lem8}
		Let $p:\Tilde{M}\longrightarrow M$ be a Riemannian covering map between two non-compact manifold, then $r(\Tilde{M})\ge r(M)$
	\end{lemma}
	\begin{proof}
		Let $x$, $\Tilde{x}$ be points in $M$ and $\Tilde{M}$ such that $p(\Tilde{x})=x$. Let $B(x,R)$ and $B(\Tilde{x},R)$ be the geodesic ball of radius $R$ centered at $x$ and $\Tilde{x}$. It suffice to show:
		\begin{align}
			Vol(B(\Tilde{x},R)\ge Vol(B(x,R))
		\end{align}
		It's not hard to see $p:B(\Tilde{x},R)\longrightarrow B(x,R)$ is surjective. Since $p$ is a local isometry, by the Area Formula we have:
		\begin{align*}
			&Vol(B(\Tilde{x},R)=\int_{B(x,R)}n(y)d\sigma\\
			\ge&\int_{B(x,R)}1d\sigma = Vol(B(x,R))
		\end{align*}
		where $n(y)$ denotes the number of inverse image of $y$ under $p$ in $B(\Tilde{x},R)$. Therefore, our conclusion easily follows.
	\end{proof}
	
	We can now prove the rigidity part of Theorem \ref{thm2}:
	\begin{proof}[\textbf{Proof of Theorem \ref{thm2} (Rigidity part)}]
		We need only to show any Ricci flat metric on $E$ is flat under the case that $F$ is an aspherical manifold. We argue by contradiction. Since $E$ is a Ricci flat closed manifold, by the structure theorem of compact Ricci flat manifold in \cite{FW} there is a finite covering $p:W\times T^l\longrightarrow E$, where $W$ is a closed simply connected Ricci flat manifold. If $E$ is not flat, then $W$ is nontrivial, which indicates the dimension of $W$ is at least 4 (Note there exists no closed simply connected Ricci flat manifold in dimension at most 3). Pulling back this map to $\Tilde{E}$ yields the following diagram of covering maps:
		\begin{align*}
			\begin{CD}
				\Tilde{E}     @>\pi>>  E\\
				@AAqA        @AApA\\
				Z     @>\Tilde{\pi}>>  W\times T^l
			\end{CD}
		\end{align*}
		where all the maps are covering maps. Since $W$ is simply connected, $Z$ must be of the form $W\times T^{l-s}\times R^s$. If $X$ is Cartan-Hadamard, by Lemma \ref{lem8} and Lemma \ref{lem10}, we have:
		\begin{align*}
			m\le r(\Tilde{E})\le r(Z) = s
		\end{align*}
	Therefore,
		\begin{align*}
			l-s\le n-dimW-s\le n-m-1
		\end{align*}
  If $X$ is of dimension 3, we have
  \begin{align*}
      l-s\le n-dimW \le n-4 = n-m-1
  \end{align*}
		Since $p$ induces a finite covering, $q$ also induces a finite covering. Therefore, $\pi_1(Z) =\mathbb{Z}^{l-s}$ is a finite index subgroup of $\pi_1(\Tilde{E})$. Since the map $\pi_1(F)\longrightarrow\pi_1(\Tilde{E})$ is injective, we denote the image of $\pi_1(F)$  by $H$. By Lemma \ref{A3}, $H\cap \pi_1(Z)$ is a finite index subgroup of $H$. Denote $\mathbb{Z}^t = H\cap \pi_1(Z)$. Then $t\le l-s \le n-m-1$. On the other hand, $H$ is the fundamental group of a closed aspherical manifold $F$. Therefore, $F$ has a finite covering $\Tilde{F}$ with fundamental group $\mathbb{Z}^t$, where $\Tilde{F}$ is also aspherical. It follows $\Tilde{F}$ is homotopic equivalent to $T^t$, and has the $(n-m)$-th $\mathbb{Z}_2$ coefficient homology group vanishing. However, $\Tilde{F}$ is of dimension $(n-m)$ since $F$ is, its $(n-m)$-th $\mathbb{Z}_2$ coefficient homology group equals $\mathbb{Z}_2$, which leads to a contradiction.
	\end{proof}
 
 \section{PSC obstructions on $S^1$ bundles}
	We study the PSC obstruction on $S^1$ bundles in this section.
 \subsection{$S^1$ bundle over enlargeable manifold with incompressible fiber}

In this subsection, we present a proof of Theorem \ref{thm3.5}.
\begin{lemma}\label{lem15}
    Let $M$ be a simply connected manifold and $E$ a $S^1$ priciple bundle over $B$. If the fiber of $E$ is incompressible, then $E$ is trivial.
\end{lemma}
\begin{proof}
    Consider the long exact sequence
    \begin{align*}
        \pi_2(M)\stackrel{\partial}{\longrightarrow}\pi_1(S^1)\stackrel{i}{\longrightarrow}\pi_1(E)\longrightarrow\pi_1(M)
    \end{align*}
    It follows from the incompressible condition that $\partial=0$. For any $\sigma\in\pi_2(M)$, by Lemma \ref{A1}, we have $\partial(\sigma) = e(hur(\sigma))=0$. Since $\pi_1(M) = 0$, by Hurwitz Theorem we have $e(a) = 0$ for any $a\in H_2(M)$. This concludes $e=0$ and $E$ is trivial.
\end{proof}
\begin{proof}[\textbf{Proof of Theorem \ref{thm3.5}}]
    Fix arbitrary metrics $g_0$, $g$ on $B$ and $E$. Let $\Tilde{g}_0$ be the pullback metric on $\Tilde{B}$, the universal covering of $B$, and let $(\Tilde{E},\Tilde{g})$ be the pullback bundle under the covering map. By Lemma \ref{lem15} $\Tilde{E}$ is trivial, we write $\Tilde{E} = \Tilde{B}\times S^1$, and define a metric $\hat{g} = dt^2+\Tilde{g}_0$. Since both $\Tilde{g}$ and $\hat{g}$ are lifted from closed manifolds simultanuously, it's not hard to see they're $C^0$ equivalent. By the fact that $\Tilde{g}_0$ is a hyperspherical metric, $E$ must be enlargeable owing to the equivalence of $\Tilde{g}$ and $\hat{g}.$ 
\end{proof}


 \subsection{$S^1$ bundle and \textit{SYS} bands}
 We focus on the proof of Theorem \ref{thm4} in this subsection. The tools are based on Gromov's width estimate for \textit{SYS} bands in \cite{ref7}.
	
	\begin{definition}\label{def5}
		A $n$ dimensional Riemannian band $M$ is called \textit{SYS}, If there are $\alpha_1,\alpha_2,\dots,\alpha_{n-3}\in H^1(M)$ and $\beta\in H^1(M,\partial M)$, such that:
		\begin{align}\label{401}
			\lbrack M,\partial M\rbrack \smallfrown\alpha_1\smallfrown\dots\smallfrown \alpha_{n-3}\smallfrown\beta
		\end{align}
		is non-spherical, where $\lbrack M,\partial M\rbrack$ means the fundamental class of the compact manifold with boundary.
	\end{definition}
	
	With the use of torical-symmetrization technique, it's shown in \cite{ref7} an $4\pi$ inequality for \textit{SYS} band.  
	
	\begin{theorem}(\cite{ref7})\label{thm7}
		If a \textit{SYS} band $M$ has $Sc(M)\ge 1$, then $width(M)\le 4\pi$.
	\end{theorem}
	
	Combining the torical-symmetrization arguement and the warped $\mu$-bubble arguement in dimension 3, one can actually give the optimal upper bound $2\pi\sqrt{\frac{n-1}{n}}$ in the preceding theorem.

    To prove Theorem \ref{thm4}, we need some more subtle topological discription of principle $S^1$ bundle over connected sum. The following notion of generalized connected sum comes from \cite{ref2}:
	
	\begin{definition}(c.f.\cite{ref2})\label{def6}
		Suppose $\alpha$ and $\beta$ are two closed curves in $n$-manifold $M$ and $N$, define the generalized connected sum of $M$ and $N$ along $\alpha$ and $\beta$ to be the manifold obtained by removing a tubular neighborhood of $\alpha$ and $\beta$ in $M$ and $N$ respectively, and paste along the boundary $S^{n-1}\times S^1$.
	\end{definition}

    Note that different isotopic classes of $S^{n-1}\times S^1$ may define generalized connected sum of different topological types. However, for two $S^1$ bundles, it would lead no misunderstanding if we define the \textit{generalized connected sum along the fiber} by a fiber preserving and orientation preserving map between two trivial $S^1$ bundles over $S^{n-1}$. We have the following structure lemma concerning $S^1$ bundle over connected sum:
	\begin{lemma}(The Structure Lemma)\label{lem11}
		Let $M$ and $N$ be $n$-dimensional closed manifold and $E$ a $S^1$ principle bundle over $B = M\#N$ ($n\ge 3$). Let $S=S^{n-1}$ be the connecting sphere of $M$ and $N$ (that is the boundary of $M\backslash D^n$ used to take connected sum). Then the restriction of $E$ on $S$ is trivial, and $E$ is the generalized connected sum of $E_M$ and $E_N$ along the fiber, where $E_M$ and $E_N$ are $S^1$ bundle over $M$ and $N$.
	\end{lemma}
	\begin{proof}
		We first show the restriction of $E$ on $S$ is trivial. Recall the isomorphism class of $S^1$ principle bundle over certain topological space $Y$ is in 1-1 correspondence to $H^2(Y)$. Note that $i^*:H^2(M)\longrightarrow H^2(M\backslash D^n)$ is an isomorphism since $n\ge 3$, the restriction of $E$ on $M\backslash D^n$ is the restriction of certain $S^1$ principle $E'$ bundle on $M$. Since $D^n$ is contractible, the restriction of $E'$ on $S$ is trivial, and this gives the desired result.
		
		Then we construct $E_M$ by filling the boundary of the bundle restricted on $M\backslash D^n$ by the trivial bundle over $D^n$ and construct $E_N$ in the similar fashion. This yields the structure of the generalized connected sum.
	\end{proof}
	
	We start the proof of Theorem \ref{thm4}. Let us briefly sketch the strategy here. In the paragraph before Lemma \ref{lem3-2}, we construct a band in certain covering space of $E$, and give an explicit discription of such band. The subsequent paragraph will be devoted to show the band is \textit{SYS}, in which Lemma \ref{lem3-2} would serve as a key element. Lemma \ref{lem3-3}, Lemma \ref{lem3-4} and Lemma \ref{lem3-5} serve as the intermediate step in proving Lemma \ref{lem3-2}. With these preparation we could prove Theorem \ref{thm4}. 
 
  The major paragraph will be devoted to the case that $X$ is a 3-dimensional aspherical manifold, while the method to handle Cartan-Hadamard manifolds of higher dimension is similar. 
  
  We start with a 3 dimensional aspherical manifold $X$, and consider its oriented covering space $\Tilde{X}$ with fundamental group $\mathbb{Z}$. Pick a curve $\gamma\in\Tilde{X}$ representing the generator of $\pi_1(\Tilde{X})$, then it follows from Whitehead's Theorem that $\Tilde{X}$ posseses $\gamma$ as a deformation retract. Let $V$ be a small tubular neighbourhood of $\gamma$, then $\Sigma_0=\partial V = T^2$, since $\Tilde{X}$ is oriented. By Lemma \ref{A5}, there is a map
  \begin{align}\label{j}
      j:\Tilde{X}\backslash V\longrightarrow \Sigma_0
  \end{align}
  such that $j|_{\Sigma_0}$ is homotopic to identity.

Define
    \begin{align}\label{404}
\mathcal{V} = U_{R}(\Sigma_0) = \lbrace x: d(x,\Sigma_0)\le R, x\in \Tilde{X}\backslash V\rbrace
\end{align}
which is an overtorical band of width $R$ due to the map $j$ in (\ref{j}). The covering $\Tilde{X}\longrightarrow X$ gives rise to a covering $\Tilde{B}\longrightarrow B$:
\begin{align}\label{31'}
			\begin{CD}
				\Tilde{B}     @>\pi>>  B\\
				@VV\Bar{k}V        @VVkV\\
				\Tilde{X}     @>>>  X
			\end{CD}
		\end{align}
Denote the pullback bundle under the covering map $\Tilde{B}\longrightarrow B$ by $\Tilde{E}$. We have
\begin{align}\label{40}
			\Tilde{B} = \Tilde{X}\#_{p_1}M\#_{p_2}M\#\dots
		\end{align}
		Let the sphere connecting $X$ and $M$ in $X\#M$ be $S$, by the structural Lemma \ref{lem11}, the restriction of $E$ on $S$ is trivial. We write:
		\begin{align*}
			\pi^{-1}(S) = S_1\cup S_2\cup S_3\cup\dots
		\end{align*}
		It follows that the restriction of $\Tilde{E}$ on each $S_i$ is trivial ($i = 1,2,3,\dots$). Then one removes the bundle over the infinite copies of $M$ in $\Tilde{E}$, and fill the trivial bundle over $S_i$ by $D^3\times S^1$. This gives a $S^1$ bundle over $\Tilde{X}$, which is trivial since $H^2(\Tilde{X})=H^2(S^1) = 0$. Hence, topologically, $\Tilde{E}$ is the generalized connected sum of $\Tilde{X}\times S^1$ and infinite copies of a $S^1$ bundle over $M$ along the fiber:
		\begin{align}\label{42}
			\Tilde{E} = \Tilde{X}\times S^1\#_{S^1}Z\#_{S^1}Z\#\dots
		\end{align}
		where $Z$ is some $S^1$ bundle over $M$.

Next we construct $\mathcal{U} = \Bar{k}^{-1}(\mathcal{V})$, by taking small perturbation we assume none of $p_i$ in (\ref{40}) lies in $\partial\mathcal{V}$, and hence we have
\begin{align}\label{402}
    &\mathcal{U} = \mathcal{V}\#_{p_1}M\#_{p_2}M\#\dots\#_{p_k}M = \mathcal{V}\#N
\end{align}
Let $\mathcal{W}$ be the restriction of the bundle $\Tilde{E}$ on $\mathcal{U}$
\begin{align}\label{403}
    &\mathcal{W} = P\#_{p_1\times S^1}Z\#_{p_2\times S^1}Z\#\dots\#_{p_k\times S^1}Z = P\#_{S^1}Y
\end{align}
where $P$ is the $S^1$ principle over $\mathcal{V}$, which must be trivial since it is induced from the trivial $S^1$ bundle over $\Tilde{X}$. Thus we have $P = \mathcal{V}\times S^1$. $Z$ and $Y$ are certain $S^1$ bundle over $M$ and $N$. The notation $\#_{p_i\times S^1}$ here means taking generalized connected sum at $p_i\times S^1\subset \mathcal{V}\times S^1 = P$.

\begin{figure}
    \centering
    \includegraphics[width = 15cm]{3.png}
    \caption{a schematic diagram of $\mathcal{U}$}
    \label{f2}
\end{figure}

Denote $\Sigma_0\cong T^2 = S^1_1\times S^1_2$, and $R_1 = j^{-1}(S^1_1), R_2 = j^{-1}(S^1_2)$ be surfaces in $\mathcal{V}$, which could also be regarded as surfaces in $\mathcal{U}$, and furthermore, surfaces in $\mathcal{W}$ (as one could arrange them away from $N$, and that the restriction of $\mathcal{W}$ on $\mathcal{V}$ is trivial). Denote $T = S^1_2\times S^1\subset \Sigma_0\times S^1\in\partial\mathcal{W}$, and $R_2^+$ the restriction of the bundle $\mathcal{W}$ on $R_2$. Then the following holds in the sense of homology.
\begin{align}\label{411}
    T = R_2^+\cap \Sigma_0\times S^1
\end{align}
and the intersection number of $T$ and $R_1$ equals 1. Our next goal is to prove:

\begin{lemma}\label{lem3-2}
    Suppose the fiber of $\mathcal{W}$ has order $k$ in $\pi_1(\mathcal{W})$. Let $\sigma$ be an immersed sphere in $\mathcal{W}$, then the intersection number of $\sigma$ and $R_1$ is a multiple of $k$.
\end{lemma}

We write $\mathcal{W} = (\mathcal{V}-D^3)\times S^1\cup (Y-D^3\times S^1) = A_1\cup A_2$, denote $L = S^2\times S^1$, the common boundary of $A_1$ and $A_2$. Suppose $f:S^2\longrightarrow\mathcal{W}$ represents a spherical class $\lbrack\sigma\rbrack$, and $f$ intersects transversally with $L$. Then 
\begin{align*}
    f^{-1}(L) = \gamma_1\cup\gamma_2\cup\dots\cup\gamma_s
\end{align*}
which is the union of closed curves. By abuse of notation we denote the image of the curves $\gamma_i$ under $f$ still by $\gamma_i$. Since $S^2$ is simply connected, $\gamma_i$ must be contractible in $\mathcal{W}$ and represents multiple of $k$ in $\pi_1(L) = \pi_1(S^1\times S^2)\cong\mathbb{Z}$. These $\gamma_i$ divide $\sigma = S^2$ into regions, some of which map to $A_1$ and others to $A_2$ under the map $f$. To compute the intersection with $R_1$ it suffice to study the regions mapped to $A_1$ since $R_1$ is a subset of $A_1$.

\begin{lemma}\label{lem3-3}(The Decomposition Lemma)
    Let $U\subset S^2$ be a region mapped to $A_1$ by $f$ as above, then the homology class in $H_2(A_1,\partial A_1)$ represented by the image of $U$ decomposed into the sum of cylinders and at most one sphere.
\end{lemma}

\begin{proof}
    Since $U$ is a subregion of $S^2$ it must be of the shape in Figure \ref{f1}. $\partial U = \gamma_1\cup\gamma_2\cup\dots\cup\gamma_r$, each $\gamma_i$ mapped to $L$ by $f$, and represents $k$ multiple of the generator as shown before. Suppose $\gamma_i$ represents $ka_i$ times generator in $\pi_1(L) = \pi_1(S^2\times S^1)$. Fix a standard curve $\beta = \lbrace 0 \rbrace\times S^1\in L$ with base point $p$, then $\gamma_i$ can be deformed to $ka_i\beta$ in $L$ without changing the homology class of the image of $U$, thus we can assume $\gamma_i$ is exactly $ka_i\beta$. Note that $\gamma_i$'s constitute the boundary of $U$, it follows that $\theta = \lbrack\gamma_1\rbrack+\lbrack\gamma_2\rbrack+\dots +\lbrack\gamma_r\rbrack = 0\in H_1(A_1)$. Since $A_1 = (\mathcal{V}-D^3)\times S^1$ and $\gamma_i\in L\subset\partial A_1$, $\theta$ must be in the image of
    \begin{align*}
        i_{1*}:H_1(S^1)\longrightarrow H_1((\mathcal{V}-D^3)\times S^1)
    \end{align*}
    which is injective by K{\"{u}}nneth formula. Hence $\theta = 0\in H_1(S^1)$ and it follows that $\theta = 0\in \pi_1(S^1)$. Therefore, $a_1+a_2+\dots+a_r = 0$.

    We next use induction on $r$. If $r = 1$, then $U$ is a disk and $\gamma_1$ is contractible in $A_1$. Since $\pi_1(L)\longrightarrow \pi_1(A_1)$ is injective, $\gamma_1$ is also contractible in $L$ and we can complement $U$ into a sphere by adding a disk in $L$. This does not change the intersection number with $R_1$, and we obtain the desired sphere. Assume the conclusion holds true for $r-1$ ($r\ge 2$). Without loss of generality, we assume
    \begin{align*}
        |a_1| = min\lbrace |a_1|,|a_2|,\dots,|a_r|\rbrace
    \end{align*}
    If $a_1 = 0$, fill it by a disk in $L$ and $r$ minus $1$, without changing the homology class of $U$. If $a_1\ne 0$, we assume $a_2$ satisfies that $a_1a_2<0$. We pick a segment in $\gamma_2$ which corresponds to $-ka_1$ times $\beta$ (this can always be done since $|a_2|\ge|a_1|$), and join two ends of this segment by a curve $\Gamma$ which wind $\gamma_1$ exactly once (as shown in figure \ref{f1} by the dotted line). Then the region enclosed by $\gamma_1$, $-ka_1\beta$ and $\Gamma$ is a cylinder. This shows $-\gamma_1$ and the composition of $-ka_1$ and $\Gamma$ are freely homotopic. That is,
    \begin{align*}
        \lbrack -ka_1\beta\rbrack*\lbrack\Gamma\rbrack = \lbrack-\gamma_1\rbrack = \lbrack -ka_1\beta\rbrack \in\lbrack S^1, A_1\rbrack
    \end{align*}
    Thus as the elements of $\pi_1(A_1)$, $\lbrack -ka_1\beta\rbrack*\lbrack\Gamma\rbrack$ and $\lbrack -ka_1\beta\rbrack$ are conjugate, that is
    \begin{align}\label{405}
        \lbrack -ka_1\beta\rbrack*\lbrack\Gamma\rbrack = g^{-1}\lbrack -ka_1\beta\rbrack g \in \pi_1(A_1)
    \end{align}
    for some $g\in \pi_1(\mathcal{W})$. Note that $A_1 = (\mathcal{V}-D^3)\times S^1$, $\beta$ must be in the center of $\pi_1(A_1) = \pi_1(\mathcal{V}-D^3)\times\pi_1(S^1)$, and it follows from (\ref{405}) that $\Gamma$ is null-homotopic.

    Thus we can fill $\Gamma$ by a pair of disk with opposite orientation, which decomposes $U$ into a cylinder, and a punctured disk with $r-1$ boundary components. Note that such operation does not change the homology class of $U$, and the conclusion follows from the induction hypothesis. (Figure \ref{f1} gives an example that $a_1=3$, $a_2=-4$. Each straight segment denotes $k\beta$, each angular point is mapped to the base point $p$).
\end{proof}

\begin{figure}
    \centering
    \includegraphics[width = 15cm]{2.png}
    \caption{}
    \label{f1}
\end{figure}

\begin{lemma}\label{lem3-4}
    Let $h:T^2 \longrightarrow T^3 = S^1_1\times S^1_2\times S^1_3$ be a map, $\alpha_1,\alpha_2$ be the generators of $H_1(T^2)$ and $\beta_1,\beta_2,\beta_3$ the generators of $H_1(T^3)$. If $h_*(\alpha_1) = k\beta_1$, then the intersection number of $h$ and $S^1_3$ is a multiple of $k$.
\end{lemma}
\begin{proof}
    We compute
    \begin{align*}
        &\lbrack h\rbrack\cdot\lbrack S^1_3\rbrack = D_{T^3}(\beta_3)(h_*\lbrack T^2\rbrack) = (\hat{\beta}_1\smallsmile \hat{\beta}_2)(h_*\lbrack T^2\rbrack) = (h^*\hat{\beta}_1\smallsmile h^*\hat{\beta}_2)(\lbrack T^2\rbrack)
    \end{align*}
    where $\hat{\alpha}_i,\hat{\beta}_i$ denote the dual basis of $\alpha_i$ and $\beta_i$ and $D_{T^3}$ the Poincare Dual in $T^3$. Since $h_*(\alpha_1) = k\beta_1$, we have $h^*(\hat{\beta}_1) = k\hat{\alpha}_1 + c_{12}\hat{\alpha}_2$ and $h^*(\hat{\beta}_2) = c_{22}\hat{\alpha}_2$. Hence
    \begin{align*}
        &\lbrack h\rbrack\cdot\lbrack S^1_3\rbrack
        = kc_{22}(\hat{\alpha}_1\smallsmile\hat{\alpha}_2)(\lbrack T^2\rbrack)
    \end{align*}
    This gives the desired conclusion.
\end{proof}

\begin{lemma}\label{lem3-5}
   Let $U$ be a region in $S^2$ and $f:U\longrightarrow A_1$ a map. If the boundary components of $U$ are all mapped to $k$ times generator of $\pi_1(L) = \pi_1(S^2\times S^1) = \mathbb{Z}$, then the intersection number of $U$ (under the immersed map $f$) and $R_1$ is a multiple of $k$.
\end{lemma}
\begin{proof}
    By Lemma \ref{lem3-4} the image of $U$ is homologuous to some cylinders $C_1,C_2,\dots,C_l$, and a sphere $C_0$. For $i\ge 1$, the boundary of each $C_i$ represents the same element in $\pi_1(L) = \pi_1(S^2\times S^1) = \mathbb{Z}$ and hence can be connected to each other by a small cylinder in $L$, which forms a torus $T_i$, and the intersection number of $T_i$ with $R_1$ equals that of $C_i$ and $R_1$. Moreover, the longtitude of $T_i$ maps to $k$ times generator of $\pi_1(L)$. 
    
    By $j$ we could define a map
    \begin{align*}
        J: (\mathcal{V}-D^3)\times S^1\longrightarrow \mathcal{V}\times S^1 \stackrel{j\times id}{\longrightarrow} \Sigma_0\times S^1.
    \end{align*}
    Then $R = j^{-1}(S^1_1) = J^{-1}(S^1_1)$. Note that a standard Poincare Duality arguement implies
    \begin{align*}
        \lbrack T_i\rbrack\cdot\lbrack R\rbrack = \lbrack T_i\rbrack\cdot\lbrack J^{-1}(S^1_1)\rbrack = J_*\lbrack T_i\rbrack\cdot\lbrack S^1_1\rbrack.
    \end{align*}
    On the other hand, $J_*\lbrack T_i\rbrack$ satisfies the condition of Lemma \ref{lem3-4} and yields the $k$-multiple in the intersection number. Similarly, we also have
    \begin{align*}
        \lbrack C_0\rbrack\cdot\lbrack R_1\rbrack = J_*\lbrack C_0\rbrack\cdot\lbrack S^1_1\rbrack
    \end{align*}
    The right hand side equals $0$ since $\pi_2(T^3)=0$. This concludes the proof of the lemma.
\end{proof}

\begin{proof}[\textbf{Proof of Lemma \ref{lem3-2}}]
    The conclusion follows from the paragraph before Lemma \ref{lem3-3}, and the result of Lemma \ref{lem3-5}.
\end{proof}

\begin{lemma}\label{lem3-6}
    Suppose the fiber of $\mathcal{W}$ is incompressible in $\pi_1(\mathcal{W})$. Let $\sigma$ be an immersed sphere in $\mathcal{W}$, then the intersection number of $\sigma$ and $R_1$ equals 0.
\end{lemma}
\begin{proof}
    The proof is the same as that of Lemma \ref{lem3-2}. After we decompose the homology class into cylinders, the boundary of the cylinder must be contractible in $L$ owing to the incompressibility, and we can complement the the cylinder into a sphere in $A_1$ by attaching a pair of disk with opposite orientation in $L$, without changing the intersection number with $R_1$. The conclusion then follows from the same arguement as in the proof of Lemma \ref{lem3-5} and the fact that $\pi_2(\Sigma_0\times S^1) = \pi_2(T^3) = 0$.
\end{proof}

With these preparation, we now carry out the proof of Theorem \ref{thm4}.
\begin{proof}[\textbf{Proof of Theorem \ref{thm4}}]
    

We first deal with the case that $X$ is $K(\pi,1)$ 3-dimensional manifold. With Lemma \ref{lem3-2} and Lemma \ref{lem3-6}, combining with the fact that $T$ and $R_1$ has intersection number 1 and the assumption that the fiber is homotopically non-trivial, we conclude the homology class $\lbrack T\rbrack$ is non-spherical. Since $T = R_2^+\cap \Sigma_0\times S^1$ by (\ref{411}) we conclude that $\mathcal{W}$ is \textit{SYS}. If the bundle $E$ in Theorem \ref{thm4} admits PSC metric, then the width of $\mathcal{W}$ has a uniform upper bound. However, by (\ref{404}) the width of $\mathcal{V}$ can be arbitrarily large, and hence by our construction, the width of $\mathcal{U}$ and $\mathcal{W}$ can also be arbitrarily large, since all the projection map have uniform Lipschitz control. This contradicts with Theorem \ref{thm7}. To see the conclusion holds true for manifold with degree 1 map to $E$, it suffice to note that any band admitting degree 1 map to \textit{SYS} band is \textit{SYS} (see \cite{ref7}, Sec.5). For the rigidity, by \cite{CG}, the universal covering of a Ricci flat compact 4-manifold must also be compact. However, $E$ has noncompact covering space and this forces the Ricci flat metric on $E$ to be flat.

Now let $X$ be a Cartan-Hadamard manifold of dimension $n\le 6$. We take $\Tilde{X}$ to be the universal covering of $X$ and $\Tilde{B}$, $\Tilde{E}$ the parallel construction as above. By the same arguement as in the proof of Theorem \ref{thm2}, there is a torical band $\mathcal{V}\cong T^{n-1}\times \lbrack 0,1\rbrack$ in $\Tilde{X}$, which could be arranged arbitrarily long. Then we construct $\mathcal{U}$ and $\mathcal{W}$ as in (\ref{402})(\ref{403}).
\begin{align*}
    &\mathcal{U} = \mathcal{V}\#_{p_1}M\#_{p_2}M\#\dots\#_{p_k}M = \mathcal{V}\#N\\
    &\mathcal{W} = P\#_{p_1\times S^1}Z\#_{p_2\times S^1}Z\#\dots\#_{p_k\times S^1}Z = P\#_{S^1}Y
\end{align*}
Here $P = T^{n-1}\times \lbrack 0,1 \rbrack\times S^1$. We next write:
		\begin{equation}
			\begin{split}
				P &= T^{n-1}\times\lbrack 0,1\rbrack\times S^1\\
				&= S^1_1\times S^1_2\times\dots\times S^1_{n-1}\times\lbrack 0,1\rbrack\times S^1_n
			\end{split}
		\end{equation}
		Let
		\begin{equation}
			\begin{split}
				&Q_0 = S^1_1\times S^1_2\times\dots\times S^1_{n-1}\times\lbrace 0\rbrace\times S^1_n,\\
				&Q_i = S^1_1\times S^1_2\times\dots\times \hat{S}^1_i\times\dots\times S^1_{n-1}\times\lbrack 0,1\rbrack\times S^1_n, \\
				&i = 1,2,\dots,n-2\\
				&R = S^1_1\times S^1_2\times\dots\times S^1_{n-2}\times\lbrack 0,1\rbrack\\
				&T = S^1_{n-1}\times\lbrace 0\rbrace\times S^1_n
			\end{split}
		\end{equation}
		Since $Q_i$ are of codimension 1 in $P$ and contains the $S^1_n$ factor, we can perturb them slightly to avoid $q_1\times S^1_n,q_2\times S^1_n,\dots,q_r\times S^1_n$. This gives a natural embedding of $Q_i,R,T$ into $\mathcal{W}$, we still denote the image of the embedding by $Q_i,R,T$. Obviously,
		\begin{align}\label{47}
			T = Q_0\cap Q_1\cap\dots\cap Q_{n-2}
		\end{align}
		Since $Q_0$ and $Q_i\quad (i\ge 1)$ can be represented by homology class in $H_n(\mathcal{W})$ and $H_n(\mathcal{W},\partial\mathcal{W})$, we can denote their Poincare Dual by $\beta \in H^1(\mathcal{W},\partial\mathcal{W})$ and $\alpha_i\in H^1(\mathcal{W})$. Therefore, it follows from (\ref{47}) that
		\begin{align}
			\lbrack T\rbrack = \lbrack\mathcal{W},\partial\mathcal{W}\rbrack\smallfrown\alpha_1\smallfrown\alpha_2\smallfrown\dots\smallfrown\alpha_{n-2}\smallfrown\beta\in H_2(\mathcal{W})
		\end{align}
		where $\lbrack T\rbrack$ is the homology class represented by $T$. Following the same strategy of Lemma \ref{lem3-2} and Lemma \ref{lem3-6}, it's easy to show the intersection number of any 2-dimensional spherical class in $\mathcal{W}$ and $R$ is a multiple of $k$, when the fiber of $E$ is of order $k$ in the fundamental group, or $0$ when the fiber is incompressible (One could take the projection map $j: T^{n-1}\times\lbrack 0,1\rbrack\longrightarrow \Sigma_0 = T^{n-1}$, which serves as the counterpart of the map '$j$' in (\ref{j}) in 3-dimensional case). Note that $\lbrack T\rbrack\cdot\lbrack R\rbrack = 1$, combining with that $k>1$ we conclude that $\lbrack T\rbrack$ is non-spherical. Therefore, $\mathcal{W}$ is a \textit{SYS} band. Hence, there exists no PSC metrics on $E$ or manifold which admits a degree 1 map to $E$.

  For the rigidity part, consider a Ricci flat metric on $E$. It follows from Lemma \ref{lem10} that $\Tilde{E}$ has volume rate at least $n$. Denote the universal covering of $E$ by $\hat{E}$, by the Cheeger-Gromoll splitting theorem \cite{CG} we have $\hat{E} = W^s\times \mathbb{R}^{n+1-s}$, where $W$ is a closed simply connected Ricci flat manifold. We have:
		\begin{align*}
			n+1-s\ge r(\hat{E})\ge r(\Tilde{E})\ge n
		\end{align*}
		by Lemma \ref{lem8}. Therefore, $s=0$ and $\hat{E}$ is isometric to the Euclidean space. This finishes the proof of Theorem \ref{thm4}.
\end{proof}

\begin{remark}
    Our method can also be used to derive some results on non-compact manifolds. Let $X$ be a $K(\pi,1)$ 3-manifold (not necessarily compact) with $\pi_1(X)\ne 0$. Then by our method it's not hard to see the $S^1$-principle bundle over $X\#_{p_1}M_1\#_{p_2}M_2\#\dots$ does not admit uniformly PSC metric, where $p_i$ ($i = 1,2,\dots)$ form discrete point set of $X$ and $M_i$ are closed manifolds.
\end{remark}

By applying similar method, we can also prove the following result.
\begin{proposition}\label{thm3}
		Let $B$ be a $n$-dimensional 1-enlargeable manifold and $E$ a $S^1$ bundle over $B$ ($n+1\le 7$). If the fiber of $E$ is homologically non-trivial, then $E$ carries no PSC metric.
	\end{proposition}

 We say a manifold is 1-enlargeable, if the non-zero degree map in the definition of the enlargeable manifold is required to be of degree 1. In fact, we have the following more general result.
 \begin{proposition}\label{thm3'}
        Let $f: Y^{n+1}\longrightarrow X^n$ be a map, where $X^n$ is 1-enlargeable, and the homological pullback of a single point in $X$ under $f$ is nontrivial in $H_1(Y)$. Then $Y$ does not admit PSC metric.
    \end{proposition}
    \begin{proof}
        Since $X$ is 1-enlargeable, for any given $\epsilon>0$, there exists a covering $\Tilde{X}$ of $X$ and a degree 1 map $h:\Tilde{X}\longrightarrow S^n\subset \mathbb{R}^{n+1}$ such that $Lip(h)<\epsilon$. Then by pulling back a fixed torical band of width $\delta(n)$ in $S^n$ via $h$ we obtain a 1-overtorical band $\mathcal{U}$ in $\Tilde{X}$ of width $\frac{\delta(n)}{\epsilon}$. We also have the following:
        \begin{align*}
			\begin{CD}
				Y^{n+1}     @>f>>  X\\
				@AAA        @AAA\\
				\Tilde{Y}     @>\Tilde{f}>> \Tilde{X}
			\end{CD}
		\end{align*}
        where $\Tilde{Y}$ is a covering space of $Y$ and $\Tilde{f}$ a proper map. Perturb $\Tilde{f}$ such that it intersect transversally with the boundary of $\mathcal{U}$, and denote $\mathcal{V} = \Tilde{f}^{-1}(\mathcal{U})$. Pick a point $p$ in $\mathcal{U}$, and denote the homology pullback of $p$ under $\Tilde{f}$ by $\Tilde{f}_!(p)$. Since $\mathcal{U}$ is 1-overtorical, by pulling back the fundamental class of $T^{n-1}\times \lbrack 0,1\rbrack$ one easily obtains
        \begin{align}
            \lbrack \mathcal{U},\partial\mathcal{U}\rbrack^* = \alpha_1\smallsmile\alpha_2\smallsmile\dots\smallsmile\alpha_{n-1}\smallsmile\beta
        \end{align}
        where $\alpha_i\in H^1(\mathcal{U})$ and $\beta\in H^1(\mathcal{U},\partial\mathcal{U})$. Then
        \begin{align*}
            &\Tilde{f}_!(p)\\ 
            = &D_{(\mathcal{V},\partial\mathcal{V})}\circ f^*\circ D_{(\mathcal{U},\partial\mathcal{U})}^{-1}(p)\\
            = &D_{(\mathcal{V},\partial\mathcal{V})}\circ f^*(\alpha_1\smallsmile\alpha_2\smallsmile\dots\smallsmile\alpha_{n-1}\smallsmile\beta)\\
            = &\lbrack\mathcal{V},\partial\mathcal{V}\rbrack\smallfrown f^*\alpha_1\smallfrown f^*\alpha_2\dots\smallfrown f^*\alpha_{n-1}\smallfrown f^*\beta\\
            \ne &0
        \end{align*}
        Here we mean $D_{(\mathcal{V},\partial\mathcal{V})}$ by taking Poincare dual for manifold with boundary and $\lbrack \mathcal{U},\partial\mathcal{U}\rbrack^*$ by the generator of top dimensional relative cohomology class. Compared with (\ref{401}), it follows $\mathcal{V}$ is \textit{SYS}.

        On the other hand,
        \begin{align*}
            width(\mathcal{V})\ge \frac{1}{Lip\Tilde{f}}width(\mathcal{U}) = \frac{1}{Lipf}width(\mathcal{U})\ge \frac{1}{Lipf}\frac{\delta(n)}{\epsilon}
        \end{align*}
        By Theorem \ref{thm7}, $Y$ does not admit PSC metric.
    \end{proof}

\begin{remark}
		It was defined in \cite{ref7} by Gromov the class of \textit{SYS}-Enlargeable manifolds which includes the manifolds with infinite \textit{SYS} width. In this subsection, we actually produce new classes of \textit{SYS}-Enlargeable manifolds.
	\end{remark}

We conclude this subsection with a proof of Corollary \ref{cor2}.
\begin{proof}[\textbf{Proof of Corollary \ref{cor2}}]
    If the Hurwitz map on $\pi_2(B)$ vanishes, by Lemma \ref{A1}, the boundary map $\partial$ in the exact sequence
    \begin{align*}
        \pi_2(B)\stackrel{\partial}{\longrightarrow}\pi_1(S^1)\stackrel{i}{\longrightarrow}\pi_1(E)\longrightarrow\pi_1(B)
    \end{align*}
    vanishes, and the fiber must be incompressible in $E$. Hence, the conclusion follows directly from Theorem \ref{thm3.5}. If $B$ is of dimension 3 and contains no $S^2\times S^1$ factor, then each prime factor of $B$ has trivial $\pi_2$. Combined with Lemma \ref{lem11} one concludes the restriction of $E$ on each prime factor has incompressible fiber, which implies $E$ also has incompressible fiber. The conclusion then follows from Theorem \ref{thm4}.
\end{proof}

 \subsection{Examples}

 To give a better interpretation of how the topological non-trivial condition of the $S^1$ fiber affect the geometry of the bundle, we would also like to include some examples here.
 
 We begin with the simplest $T^2$. Even for such a simplest case, the $S^1$ bundle over it with non-divisible Euler class has zero-homologuous fiber. However, it's always incompressible since $\pi_2(T^2)=0$. For similar example one could also consider $T^n\#T^n$, which has the twisted stablized property by Theorem \ref{cor2}. Additionally, one could see that the Schoen-Yau descent arguement can hardly apply in these situation since it requires homologically free fiber, and this is the case only when the bundle is trivial as a direct corollary of Lemma \ref{A4}.

There also exists the case that the fiber is compressible but homologically nontrivial. In fact, the $S^1$ bundle over $T^4\#\mathbb{C}\mathbb{P}^2$, whose restriction on $T^4$ is trivial while on $\mathbb{C}\mathbb{P}^2$ is like $S^1\longrightarrow S^5\slash\mathbb{Z}_p\longrightarrow \mathbb{C}\mathbb{P}^2$ is such an example, and it admits no PSC metric by Theorem \ref{thm3}. Moreover, if the restriction of the bundle on $T^4$ has non-divisible Euler class, then the fiber is forced to be zero-homologuous, but still homotopically non-trivial provided $p>1$. Then Theorem \ref{thm4} is the only result to be applied under this situation.
 
 We conclude this section by providing an example that even the homological nontrivial condition for $S^1$ fiber is not always a sufficient condition to rule out PSC metric on the fiber bundle.
	\begin{example}
        Consider the $S^1$ bundle over the $K_3$ surface, whose Euler class is 3 times the generator of the second cohomology group. One can show the fiber is homologically non-trivial and the bundle is a spin 5-manifold with fundamental group $\mathbb{Z}_3$ (This would be easy to see by Lemma \ref{A1}). By the result of \cite{ref17}, such manifold carries PSC metric.
	\end{example}
  \section{Codim 2 PSC obstructions in twisted settings}
	As an application of our result, we give an partially affirmative answer to Gromov's quadratic decay question on non-trivial bundle in \cite{ref7}. We say a closed manifold $B$ has \textit{dominated twisted stabilized property} if any manifold which admits a degree 1 map to certain $S^1$ bundle over $B$ carries no PSC metric.
	\begin{proposition}\label{thm9}
	    Let $X^n$ ($n\le 7$)be the 2-dimensional vector bundle over $B$ where $B$ has \textit{dominated twisted stabilized property}, $Y$ a non-compact manifold which admits a degree 1 map to $X$. Then for any metric on $Y$ there exists a constant $R_0$, such that
        \begin{align}\label{1}
            \inf_{x\in B(R)}Sc(x)\le \frac{4\pi^2}{(R-R_0)^2}
        \end{align}
	\end{proposition}
\begin{proof}
    Let $V$ be a small tubular neighbourhood of the zero section of $X$, then $\partial V$ is a principle $S^1$ bundle over $B$. Therefore, any closed manifold with degree 1 map to $\partial V$ admits no PSC metric. Moreover, $X\backslash V$ is diffeomorphic to $\partial V\times\lbrack 0,+\infty)$, let
    \begin{align}\label{2}
        p: X\backslash V\longrightarrow \partial V
    \end{align}
    be the projection map.

    Let $f:Y\longrightarrow X$ be the degree 1 map. Perturb $f$ such that it intersects transversally with $\partial V$. Let $V'=f^{-1}(V)$, then $S=\partial V'=f^{-1}(V)$ is a embedded submanifold in $Y$. Define:
    \begin{align*}
        U_R = \lbrace x\in Y\backslash V', d_S(x)\le R\rbrace
    \end{align*}
    where $d_S$ means a slight smooth approximation of the distance function to $S$. Let $S_R=\partial U_R\backslash S$, for any hypersurface $\Sigma$ in $U_R$ seperating $S$ and $S_R$, let $W$ be the region bounded by $S$ and $\Sigma$. By (\ref{2}) we have:
    \begin{align}\label{3}
        deg(p\circ f|_{\Sigma}) = deg(p\circ f|_S) = degf = 1
    \end{align}
    Since $\partial V$ has dominated twisted stablized property, $\Sigma$ does not admit PSC metric.

 
    By a standard $\mu$-bubble arguement as in \cite{ref23}, the width estimate holds for the band $U_R$. That is, if the band $U_R$ has scalar curvature no smaller than $\sigma$, then it has width no greater than $\frac{2\pi}{\sqrt{\sigma}}$. Since the width of $U_R$ is at least $R$, we obtain:
    \begin{align}\label{4}
        \inf_{x\in U_R}Sc(x)\le \frac{4\pi^2}{R^2}
    \end{align}
    Let $R_0$ be the diameter of $V'$ and (\ref{1}) follows from (\ref{4}).
\end{proof}

All the manifolds in Corollary \ref{cor2} have \textit{dominated twisted stabilized property}, hence Proposition \ref{thm9} applies. It follows from Theorem \ref{thm4} that for any 3-manifold which does not admit PSC metric, the quadratic decay inequality in Proposition \ref{thm9} also holds under the homotopical nontrivial condition of the fiber.

With the benifit of PSC obstruction on non-trivial bundle, we can also prove the following result:

\begin{proposition}\label{thm10}
    Let $X^n$ ($n\le 7$)be a non-compact manifold with $B^{n-2}$ as a codimension 2 deformation retract, where $B$ has \textit{dominated twisted stabilized property}. Then there exists a constant $R_0$, such that
        \begin{align*}
            \inf_{x\in B(R)}Sc(x)\le \frac{4\pi^2}{(R-R_0)^2}
        \end{align*}
\end{proposition}
\begin{proof}
    Let $F:X\longrightarrow B$ be the retracting map, $V$ the tubular neighbourhood of $B$, then $\partial V$ is diffeomorphic to the $S^1$ principle bundle over $B$. By Lemma \ref{A5} $F|_{X\backslash V}$ can be lifted to $\Tilde{F}:X\backslash V\longrightarrow \partial V$, with $\Tilde{F}|_{\partial V}$ homotopic to identity. This implies any hypersurface seperating $\partial V$ and the infinity admits a degree 1 map to $\partial V$, and the conclusion follows from a similar arguement as in the proof of Proposition \ref{thm9}.
\end{proof}

With the same arguement, we obtain the proof of Corollary \ref{cor3}\\

\textbf{Acknowledgement} I am deeply indebted to my advisor Prof. Yuguang Shi for suggesting this problem, constant support and many inspiring conversations. I am grateful to Prof. Yi Liu and Prof. Yi Xie for conversations on geometrization conjecture and gauge theory, to Dr. Jintian Zhu for many helpful discussions.
 

	\appendix
	\section{Topological and algebraic preliminary}
	In Appendix A, we prove several topological and algebraic facts which are used in the paper.
	
	\begin{lemma}\label{A1}
    Let $\pi:E\longrightarrow B$ be a $S^1$ bundle with Euler class $e$. Then in the long exact sequence
    \begin{align*}
        \pi_2(B)\stackrel{\partial}{\longrightarrow}\pi_1(S^1)\stackrel{i}{\longrightarrow}\pi_1(E)\longrightarrow\pi_1(B)
    \end{align*}
    we have
    \begin{align*}
        \partial:\sigma\longmapsto e(hur(\sigma))
    \end{align*}
    for $\sigma\in \pi_2(B)$
\end{lemma}
\begin{proof}
    Pick a base point $b_0\in B$ and $x_0$ belonging to the fiber of $b_0$, Let:
    \begin{equation}
    \begin{split}
        F:I\times I &= \lbrack 0,1\rbrack\times \lbrack 0,1\rbrack\longrightarrow B, \\
        &F(\partial (I\times I)) = b_0
        \end{split}
    \end{equation}
    be an element in $\pi_2(B,b_0)$. Then $F$ can be lifted to:
    \begin{equation}
    \begin{split}
        \Tilde{F}:I\times I&\longrightarrow E,\\
        \Tilde{F}(J) &= x_0,\\
        \Tilde{F}(K)&\subset \pi^{-1}(b_0)
        \end{split}
    \end{equation}
    where $J = \partial(I\times I)\backslash\lbrack 0,1\rbrack\times\lbrace 1\rbrace$, $K=\lbrack 0,1\rbrack\times\lbrace 1\rbrace$. Then $\Tilde{F}|K: K\longrightarrow \pi^{-1}(b_0)=S^1$ gives an element in $\pi_1(S^1)$, this corresponds to the image of $F$ under $\partial$ (\cite{ref11}, Theorem 4.41). To compute this element, we first view $F$ as the map to $\Tilde{E}$, the $\mathbb{R}^2$ vector bundle which induces $E$. $B$ can then be viewed as the zero section of $\Tilde{E}$. Since $\Tilde{E}$ and $E$ are homotopy equivalent, we can lift $F$ to a map to $\Tilde{E}$ such that the boundary of $I\times I$ is maped to $x_0$. Perturb $F$ such that it keeps stationary on $\partial(I\times I)$ and intersect transversally with $B$ at points $p_1,p_2,\dots,p_k$. One can assume $p_i$ being sufficiently close to $K$. In fact, for a point $p_i$, we can join it to a small neibourhood of $K$ by a curve $c_i$, such that $c_i$ only intersect $B$ at $p_i$. It's not hard to imagine $p_i$ can be straped to the neibourhood of $K$ along the curve $c_i$. Then we can remove the small disk around $p_i$ and construct a lifting of $F$ to $\Tilde{E}$ which does not touch the zero section. This gives a desired lifting as in (A.2). Since $F$ intersects the zero section of $E$ transversally, each intersecting point contribute once in the process of winding the $S^1$ fiber over $b_0$. From this fact, one concludes the element in $\pi_1(S^1)$ is given by the oriented intersection number of $F$ and the zero section, and the final conclusion follows with a use of Poincare duality.
    
\end{proof}

 \begin{lemma}\label{A4}
Let $B^n$ be a closed manifold with $H^2(B,\mathbb{Z})$ free, $\pi:E\longrightarrow B$ a $S^1$ principle bundle over $B$ with Euler class $e$. Let $\tau$ be the homology class represented by the fiber in $H_1(E)$, then\\
 (1) $\tau=0$ if $e$ is non-divisible.\\
 (2) $\tau$ is a $k$-torsion element if $e=k\alpha$, where $\alpha$ is non-divisible.\\
 (3) $\tau$ is a free element if $e=0$.
 \end{lemma}
 \begin{proof}

     Let $p$ be a point in $B$, it suffice to calculate the homology pullback $\pi_!(p)$ of $p$ under $\pi$, which is exactly the homology class represented by the fiber. We have
     \begin{align*}
         \pi_!(p) = &D_E\circ \pi^*\circ D_B^{-1}(p)\\
         = &D_E\circ \pi^*(\lbrack B\rbrack^*)\\
     \end{align*}
     where $D$ means taking Poincare duality and $\lbrack B\rbrack^*$ the generator of $H^n(B)$. We apply the Gysin sequence
     \begin{align}\label{406}
        H^{n-2}(B)\stackrel{\smallsmile e}{\longrightarrow}H^n(B)\stackrel{\pi^*}{\longrightarrow}H^n(E)
    \end{align}
    Since $D_E$ is an isomorphism, it suffice to study the element $\pi^*(\lbrack B\rbrack^*)\in H^n(E)$. If $e=0$, then $\pi^*$ in (\ref{406}) is injective, hence $\pi_!(p)$ is free. If $e\ne 0$, we assume $e = k\alpha$ with $\alpha$ a non-divisible element in $H^2(B)$. By Corollary 3.39 in \cite{ref11}, there exists $\beta\in H^{n-2}(B)$, such that $\alpha\smallsmile\beta = \lbrack B\rbrack^*$. This shows the image of the first map in (\ref{406}) equals $k\mathbb{Z}\subset\mathbb{Z}\cong H^n(B)$. Then the conclusion easily follows.
 \end{proof}

\begin{lemma}\label{A5}
    Let $X^n$ be a non-compact manifold with $B^{n-2}$ as a codimension 2 deformation retract, the retracting map denoted by $F:X\longrightarrow B$. Let $V$ be the tubular neighbourhood of $B$ and $Z=\partial V$, a $S^1$ principle bundle over $B$. Let $\pi:Z\longrightarrow B$ be the bundle projection. Then $F|_{X\backslash V}$ can be lifted to a map $\Tilde{F}:X\backslash V\longrightarrow Z$, such that the restriction of $\Tilde{F}$ on $Z$ is homotopic to identity.
\end{lemma}
\begin{proof}
    Let $f=F|_Z$, we first lift $f$ to a map $\Tilde{f}:Z\longrightarrow Z$, such that $f=\pi\circ\Tilde{f}$ and $\Tilde{f}$ homotopic to identity. Fix a Riemannian metric on $X$ such that $V=\lbrace x\in X| d(x,B)\le\delta$ for some small number $\delta$. Define the homotopy $h_t:Z\longrightarrow V\hookrightarrow X$ by
    \begin{align*}
        (p,x)\mapsto (p,\delta tx)
    \end{align*}
    Here $p=\pi(x)$, and $\delta tx$ means rescaling by the coefficient $\delta t$ in the disk fiber over $p$ in bundle $V$. Then
    \begin{align*}
        &h_0(x) = \pi(x)\\
        &h_1(x) = x
    \end{align*}
    Therefore, $H_t = F\circ h_t$ satisfies
    \begin{align*}
        &H_0(x) = \pi(x)\\
        &H_1(x) = f(x)
    \end{align*}
    Since $\pi:Z\longrightarrow B$ has homotopic lifting property, and $H_0 = \pi$ can be lifted to $id:Z\longrightarrow Z$, there exists a homotopy $\Tilde{H}_t:Z\longrightarrow Z$ with
    \begin{align*}
        &\Tilde{H}_0 = id\\
        &H_t = \pi\circ\Tilde{H}_t
    \end{align*}
    Let $\Tilde{f} = \Tilde{H}_1$, then $f=H_1=\pi\circ\Tilde{H}_1=\pi\circ\Tilde{f}$. This shows $\tilde{f}$ has the desired property.

    We next lift $F$ to the map $\Tilde{F}$, such that $\Tilde{F}|_Z = \Tilde{f}$. Since $S^1$ is $K(\mathbb{Z},1)$, the only obstruction of this lifting lies in $H^2(X\backslash V, Z, \mathbb{Z})=H^2(X,V,\mathbb{Z}) = 0$. This shows $F$ has the desired lifting and finishes the proof of Lemma \ref{A5}.
\end{proof}

 \begin{lemma}\label{A3}
     Let $H,K$ be the subgroups of $G$. If $K$ has finite index in $G$, then $H\cap K$ has finite index in $H$.
 \end{lemma}
 \begin{proof}
     We write
     \begin{align*}
         A = \lbrace gK|gK\cap H\ne\emptyset\rbrace
     \end{align*}
     For any $g_1,g_2\in gK\cap H$, we have $g_1K=g_2K$, hence $g_2^{-1}g_1\in K$, $g_2^{-1}g_1\in H\cap K$, $g_1(H\cap K)=g_2(H\cap K)$. This defines a well defined map:
     \begin{align*}
         F:(G\slash K)\cap A &\longrightarrow H\slash H\cap K\\
         gK&\mapsto g_0(K\cap H)
     \end{align*}
     where $g_0$ is any element in $gK\cap H$. The map $F$ is evidently surjective. Combining with that $A\subset G\slash K$ is finite, we have $H\slash H\cap K$ is finite.
 \end{proof}
	
\section{Volume growth estimate on fiber bundles}	
	
In this appendix, we study volume growth of noncompact manifold $\Tilde{B}$ and $\Tilde{E}$, the covering of $B$ and $E$ in Theorem \ref{thm2}, induced by the universal covering $\Tilde{X}\longrightarrow X$, where $X$ is a closed Cartan-Hadamard manifold. These results will be used in the rigidity proof of Theorem \ref{thm2}.
	\begin{lemma}\label{lem9}
		Let $B^m=X\#M$ be the manifold in Lemma \ref{lemM} and $\Tilde{B}$ the covering of $B$ appeared in Lemma \ref{lemM}. Then $r(\Tilde{B})\ge m$.
	\end{lemma}
	\begin{proof}
		
		We argue by comparing the geometry of the covering of $X$ and $B=X\#M$. We write $X = (X\backslash D^m)\cup D^m$ and $B = (X\backslash D^m)\cup (M\backslash D^m)$. Define $g_0,\hat{g}_1,\hat{g}_2$ to be metrics on $X\backslash D^m, D^m$ and $M\backslash D^m$, which coincide on the common boundary $\partial D^m = S^{m-1}$. Define $g_1,g_2$ to be metrics on $X$ and by:
		\begin{equation}\label{8'}
			g_1(y)=\left\{
			\begin{array}{rc}
				g_0(y), &y\in X\backslash D^m \\
				\hat{g}_1(y), &y\in D^m
			\end{array}
			\right.
		\end{equation}
		and the metric on $B$ by:
		\begin{equation}\label{9}
			g_2(y)=\left\{
			\begin{array}{rc}
				g_0(y), &y\in X\backslash D^m \\
				\hat{g}_2(y), &y\in M\backslash D^m
			\end{array}
			\right.
		\end{equation}
  
		For the metric space $(D^m,\hat{g}_1)$, denote the distance between $x$ and $y$ by $d_1(x,y)$, and the distance restricted to the boundary $S=S^{m-1}=\partial D^m$ by $d_S$. Then there exists a constant $C$ relying only on the $C^0$ geometry of $(D^m,\hat{g}_1)$, such that:
		\begin{align}\label{10}
			d_S(x,y)\le Cd_1(x,y),\quad for\quad x,y\in S
		\end{align}
		In fact, $\hat{g}_1$ is equivalent to the flat metric $\hat{g}_3$ on $D^m$. Under $\hat{g}_3$ it's not hard to see 
		\begin{align*}
			\frac{d_{3,S}(x,y)}{d_3(x,y)}\le\frac{t}{\sin{t}}\le\frac{\pi}{2}, t\in( 0,\frac{\pi}{2}\rbrack, x,y\in S
		\end{align*}   
		and (\ref{10}) easily follows.
		
		Now denote $d_1,d_2$ to be the distance in the space $(D^m,\hat{g}_1)$ and $(M\backslash D^m,\hat{g}_2)$. By (\ref{10}) one has for any $x,y\in S$:
		\begin{equation}\label{11}
			\begin{split}
				&\frac{d_2(x,y)}{d_1(x,y)} = \frac{d_2(x,y)}{d_S(x,y)}\frac{d_S(x,y)}{d_1(x,y)}
				\le\frac{d_S(x,y)}{d_1(x,y)}\le C(\hat{g}_1)
			\end{split}
		\end{equation}
		
		The next step, we introduce a function $N$ to describe the density of the covering space. Let $Y$ be a closed Riemannian manifold with non-compact covering $q:\Tilde{Y}\longrightarrow Y$. fix a point $a\in \Tilde{Y}$. We define:
		\begin{align}\label{12}
			N(\Tilde{Y},a,R) = \min_{y\in Y}\#\lbrace p^{-1}(y)\cap B(a,R)\rbrace
		\end{align}
		
		We next show the following relation between $N$ and the volume of the geodesic ball:
		\begin{align}\label{13}
			Vol(Y)N(\Tilde{Y},a,R+D)\ge Vol(B(a,R))\ge Vol(Y)N(\Tilde{Y},a,R-D)
		\end{align}
		where $D$ is some constant only relying on the geometry of $Y$. In fact, one can choose $K$ to be a compact subset of $\Tilde{X}$, such that $p|_K:K\longrightarrow X$ is a 1-1 correspondence. Then $\mathcal{H}^{dimY}(K)=\mathcal{H}^{dimY}(Y)=Vol(Y)$, where $\mathcal{H}$ is the Hausdorff measure on $Y$ and $\Tilde{Y}$. Denote $D=diam(K)$. For the first inequality, let $y$ be a point attaining the minimum in (\ref{12}), and pick a unique point $\Tilde{y}$ in $p^{-1}(y)\cap K$. By taking deck transformation from $\Tilde{y}$ to elements of $\lbrace p^{-1}(y)\cap B(a,R)\rbrace$ we can cover $B(x,R)$ by at most $N(R+D)$ copies of $K$, each has volume $Vol(Y)$, and the conclusion follows. The second inequality is of the similar reason. 
		
		If $X$ is Cartan-Hadamard, by the volume comparison we have:
		\begin{align}\label{14}
			N(\Tilde{X},a,R)\ge cR^m
		\end{align}
		
		By passing to the covering of the construction of (\ref{8'}) and (\ref{9}), We write:
		\begin{equation*}
			\begin{split}
				\Tilde{X} = (\Tilde{X}-_{p_1}D^m-_{p_2}D^m-\dots)\cup_{p_1}D^m\cup_{p_2}D^m\cup\dots\\
				\Tilde{B} = (\Tilde{X}-_{p_1}D^m-_{p_2}D^m-\dots)\cup_{p_1}M\cup_{p_2}M\cup\dots
			\end{split}
		\end{equation*}
		where $-_{p_i}$ means removing a little $D^m$ near the point $p_i$. The lifted metric of $g_i(i=1,2)$ denoted by $\Tilde{g}_i(i=1,2)$. Then $\Tilde{g}_1$ and $\Tilde{g}_2$ coincide on $\Tilde{X}_0 = (\Tilde{X}-_{p_1}D^m-_{p_2}D^m-_{p_3}D^m-\dots)$. Therefore, any point in $\Tilde{X}_0$ can be regarded as the common point of $\Tilde{B}$ and $\Tilde{X}$. Pick $a\in\Tilde{X}_0$, let $\Tilde{B}(a,R)$ be the geodesic ball of radius $R$ in $\Tilde{B}$. To show $\Tilde{B}$ has volume rate no less than $m$ it suffice to estimate the lower bound of $Vol(\Tilde{B}(a,R))$ for any $a\in\Tilde{X}_0$ without loss of generality.
		
		For $x\in\Tilde{X}_0$, we will compare the distance of $x$ and $a$ in the cases that $x,a$ are regarded as the points of $\Tilde{B}$ and $\Tilde{X}$ respectively. Let $\gamma$ be a minimal geodesic in $\Tilde{X}$ joining $x$ and $a$, write $\gamma=\zeta_1\eta_1\zeta_2\eta_2\dots\zeta_l\eta_1$, where $\zeta_i$ is the segment in $\Tilde{X}_0$ and $\eta_i$ the segment in the distinguished $D^m$'s. By (\ref{11}), there're segment $\xi_i$ in $M\backslash D^m$ whose endpoints coincide with $\eta_i$, such that
		\begin{align*}
			L_2(\xi_i)\le C\cdot L_1(\eta_i)
		\end{align*}
		where $L_1$ denotes the length of a curve in $\Tilde{X}$ and $L_2$ the counterpart in $\Tilde{B}$. Construct a curve $\Tilde{\gamma}$ in $\Tilde{B}$ by  $\Tilde{\gamma} = \zeta_1\xi_1\zeta_2\xi_2\dots\zeta_l\xi_l$, we have
		\begin{equation}\label{15.5}
            \begin{split}
			L_2(\Tilde{\gamma})=& \sum_{i=1}^{l}L_2(\zeta_i)+\sum_{i=1}^{l}L_2(\xi_i)\\
			\le &\sum_{i=1}^{l}L_2(\zeta_i)+C\sum_{i=1}^{l}L_1(\eta_i)\\
			\le &CL_1(\gamma)
		\end{split}
            \end{equation}
		By (\ref{15.5}), for sufficiently large $R$, it's not hard to find a hypersurface $\Sigma$ in $\Tilde{X}_0$, satisfying:\\
		(i) In both $\Tilde{X}$ and $\Tilde{B}$, $\Sigma$ separates the space into two connected components.\\
		(ii) In $\Tilde{X}$, the region bounded by $\Sigma$ contains $B(a,R)$.\\
		(iii) In $\Tilde{B}$, the region bounded by $\Sigma$ is contained in $\Tilde{B}(a,2CR)$.\\
		Therefore,
		\begin{align*}
			N(\Tilde{B},a,2CR)\ge N(\Tilde{X},a,R)
		\end{align*}
		Combining with (\ref{13}) and (\ref{14}) we have
		\begin{equation}\label{16}
			\begin{split}
				&Vol(\Tilde{B}(a,2CR+D))\\
				\ge &Vol(B)N(\Tilde{B},a,2CR)\ge Vol(B)N(\Tilde{X},a,R)\\
				\ge &\frac{Vol(B)}{Vol(X)}Vol(B(a,R-D))\ge C(R-D)^m.
			\end{split}
		\end{equation}
		where $C$ relies only on the $C^0$ geometry of $B$ and $X$. This concludes the proof of Lemma \ref{lem9}.
	\end{proof}
	
	
	
	\begin{lemma}\label{lem10}
		Let $\Tilde{E}$ be the covering of $E$ in Lemma \ref{lemM}. Then $r(\Tilde{E})\ge m$, $m$ the dimension of $B$.
	\end{lemma}
	\begin{proof}
            Fix the metric $g_B, g_E, g_F$ on $B$, $E$ and $F$. Denote the distance function on $\Tilde{B}$ and $\Tilde{E}$ by $d_1, d_2$. Let $q:\Tilde{E}\longrightarrow \Tilde{B}$ be the projection map. We first prove the following claim\\
            
            \textbf{Claim.} \textit{There exists a constant $C$ relying only on the $C^0$ geometry of $B$ and $E$, such that for any $x,y\in \Tilde{B}$ and $z\in q^{-1}(x)$, one can always pick $w\in q^{-1}(y)$ such that the following inequality holds:}
            \begin{align}\label{3.10.1}
                d_2(z,w) \le Cd_1(x,y)
            \end{align}
            
            To prove (\ref{3.10.1}), we first choose a local trivialization $U_1, U_2,\dots ,U_k$ of $E$, where $U_i$'s form a open covering of $B$ and each diffeomorphic to the unit disk $D^m$. Then $E|_{U_i}\cong D^m\times F$. Due to the compactness, there exists constants $C_i$ ($i = 1,2,\dots,k)$ satisfying
            \begin{align}\label{3.10.2}
                \frac{1}{C_i}g_i\le g_E|_{(E|_{U_i})}\le C_ig_i
            \end{align}
            where $g_i$ is the product metric of the flat metric on $D^m$ and $g_F$. Let
            \begin{align*}
                q^{-1}(U_i) = \bigcup_{\alpha} U_{i\alpha}
            \end{align*}
            where $U_{i\alpha}$ is mapped diffeomorphically to $U_i$ under $q$. By the covering relation, the geometry of $\Tilde{E}|_{U_{i\alpha}}$ is the same as $E|_{U_i}$, then it's not hard to see (\ref{3.10.1}) holds on each $U_{i\alpha}$ by (\ref{3.10.2}). To see (\ref{3.10.1}) holds globally, we choose a geodesic $\gamma$ connecting $x$ and $y$ and choose points $x_0, x_1, \dots, x_s$ on $\gamma$ such that $x_0 = x$, $x_s = y$, $x_j$ and $x_{j+1}$ in the same ${U_{i\alpha}}$ for some $i$ and $\alpha$. Then by using (\ref{3.10.1}) on each ${U_{i\alpha}}$ and sum up we get the desired result. This finishes the proof of the Claim.

            Let $D$ be the maximum of the diameter of the fiber of $E$. Choose $a\in \Tilde{B}$ and $\Tilde{a}\in q^{-1}(a)$ be fixed point. By (\ref{3.10.1}) we have:
            \begin{align*}
               B(a,R) \subset q(B(\Tilde{a},CR+D))
            \end{align*}
            Therefore, by (\ref{13}) we have
            \begin{equation}
                \begin{split}
                    Vol(B(\Tilde{a},CR+D))&\ge Vol(E)N(\Tilde{E},\Tilde{a},CR+D)\ge Vol(E)N(\Tilde{B},a,R)\\
                    &\ge \frac{Vol(E)}{Vol(B)}Vol(a,R)\ge cR^m
                \end{split}
            \end{equation}
		which gives the desired result.
	\end{proof}	
	
	\begin{thebibliography}{100}
		\bibitem{ref1} L. B{\'{e}}rard-Bergery, Scalar curvature and isometry group. In Spectra of Rie-
		mannian Manifolds, Kaigai Publications, Tokyo, 1983, pp. 9–28, Proc.
		Franco-Japanese seminar on Riemannian geometry (Kyoto, 1981).
		\bibitem{ref20}B. Botvinnik, J. Rosenberg, Positive scalar curvature on $Pin^{\pm}$- and $Spin^c$- manifolds and manifolds with singularities, in Perspectives in scalar curvature (2022).
            \bibitem{BH09} M. Brunnbauer, B. Hanke, Large and small group homology , J. Topology \textbf{3} (2010), 463-486.
            \bibitem{CRZ} S. Cecchini, D. R{\"{a}}de, R. Zeidler, Nonnegative scalar curvature on manifolds with at least two ends. arxiv:2205.12174 (2022).
		\bibitem{CG} J. Cheeger and D. Gromoll, The splitting theorem for manifolds of nonnegative
		Ricci curvature, J. Differential Geometry \textbf{6} (1971/72), 119–128.
		\bibitem{ref2} J. Chen, P. Liu, Y. Shi, J. Zhu, Incompressible hypersurface, positive scalar curvature and positive mass theorem, arXiv:2112.14442 (2021).
		\bibitem{ref3} O. Chodosh and C. Li, Generalized soap bubbles and the topology of manifolds with positive scalar curvature. arXiv:2008.11888 (2020).
		\bibitem{CLL} O. Chodosh, C. Li, Y.Liokumovich, Classifying sufficiently connected psc manifolds in 4 and 5 dimensions. arXiv:2105.07306 (2021), To appear in Geom. Topol.
            \bibitem{CMS23} O. Chodosh, C. Mantoulidis, F. Schulze, Generic regularity for minimizing hypersurfaces in dimensions 9 and 10. arXiv:2302.02253 (2023).
		\bibitem{ref4}D. Fischer-Colbrie and R. Schoen, The structure of complete stable minimal
		surfaces in 3-manifolds of nonnegative scalar curvature. Comm. Pure
		Appl. Math. \textbf{33} (1980), 199–211.
		\bibitem{FW}A. Fischer and J. Wolf, The Calabi construction for compact Ricci
		flat Riemannian manifolds, Bull. Amer. Math. Soc. \textbf{80} (1974), 92–97.
        \bibitem{ref24}M.Gromov and H. B. Lawson, Jr. Spin and scalar curvature in the Presence of a fundamental Group. I. Ann. Math. (2), \textbf{111} (1980), no.2, 209-230.
		\bibitem{ref5}M. Gromov and H. B. Lawson, Jr. The classification of simply connected
		manifolds of positive scalar curvature. Ann. Math. (2), \textbf{111} (1980), no.3, 423–434.
		\bibitem{ref6}M. Gromov and H. B. Lawson, Jr. Positive scalar curvature and the Dirac
		operator on complete Riemannian manifolds. Inst. Hautes {\'{E}}tudes Sci. 
		Publ. Math. \textbf{58} (1983), 83–196.
            \bibitem{Gr96}M. Gromov, Positive curvature, macroscopic dimension, spectral gaps and
        higher signatures. In Functional Analysis on the Eve of the 21st Century, Vol. II, Birkhauser, Boston, MA (1996).
		\bibitem{ref7}M. Gromov, Metric inequalities with scalar curvature. Geom. Funct. Anal.
		\textbf{28} (2018), no.3, 645–726.
		\bibitem{ref9} M. Gromov, No metrics with positive scalar curvatures on aspherical 5-
		manifolds. arXiv:2009.05332 (2020).
            \bibitem{Gr21} M. Gromov, Torsion Obstructions to Positive Scalar Curvature. arxiv: 2112.04825 (2021).
		\bibitem{ref10}B. Hanke, Positive scalar curvature with symmetry, J. Reine Angew. Math. \textbf{614} (2008), 73-115.
  \bibitem{HS06}B. Hanke, Thomas Schick, Enlargeability and index theory. J.Differential Geom. (2) \textbf{74} (2006), 293-320.
            \bibitem{HPS15}B. Hanke, D. Pape, T. Schick, Codimension two index obstructions to positive scalar
curvature, Ann. Inst. Fourier (Grenoble) \textbf{65} 
 (2015), 2681–2710.
		\bibitem{ref11} A. Hatcher, Algebraic topology, Cambridge University Press, Cambridge (2002).
		\bibitem{ref12}H. B. Lawson, Jr. and M.-L. Michelsohn, Spin Geometry. Princeton Mathematical Series, 38. Princeton University Press, Princeton, NJ (1989).
            \bibitem{GT} B. Martelli, An Introduction to Geometric Topology, version 2 (2022).
            \bibitem{MG} J. Morgan and G. Tian. Completion of the proof of the Geometrization Conjecture. Preprint,
arXiv:0809.4040 (2008).
		\bibitem{ref16}D. R{\"{a}}de, Scalar and mean curvature comparison via $\mu$-bubbles, arXiv:2104.10120 (2021).
            \bibitem{Ros83} J. Rosenberg, $C^*$-algebras, positive scalar curvature, and the Novikov con-
jecture. Inst. Hautes {\'{E}}tudes Sci. Publ. Math. \textbf{58} (1983), 197-212.
		\bibitem{ref17} J. Rosenberg, $C^*$-algebras, positive scalar curvature, and the Novikov conjecture, II, in Proc.U.S.-Japan Seminar on Geometric Methods in Operator Algebras, Kyoto 1983, H. H. Araki and E. G. Efros (eds.).
            \bibitem{ref19} J. Rosenberg, S. Stolz, Metrics of positive scalar curvature and connections with surgery, In: Surveys on
		surgery theory, Vol. 2. Vol. 149. Ann. of Math. Stud. Princeton Univ Press, Princeton, NJ (2001). pp.353-386.
		\bibitem{ref18}J. Rosenberg, Manifolds of positive scalar curvature: a progress report in Survey in Differential Geometry. Vol.
		XI, Surv. Differ. Geom. 11, International Press, Sommerville, MA (2007), pp. 259-294.
		\bibitem{ref25}T. Schick, A counterexample to the (unstable) Gromov-Lawson-Rosenberg conjecture, Topology \textbf{37} (1998), no. 6, 1165-1168.
		\bibitem{SY1}R. Schoen and S. T. Yau, Existence of incompressible minimal surfaces and
		the topology of three-dimensional manifolds with nonnegative scalar curvature, Ann.
		of Math. (2) \textbf{110} (1979), no. 1, 127–142.
		\bibitem{ref21}R. Schoen and S. T. Yau, On the structure of manifolds with positive scalar
		curvature. Manuscripta Math. \textbf{28} (1979), no.1-3, 159–183.
            \bibitem{SY17} R. Schoen and S. T. Yau, Positive Scalar Curvature and Minimal Hypersurface Singularities, arxiv:1704.05490 (2017).
            \bibitem{St92} S. Stolz, Simply connected manifolds of positive scalar curvature. Ann.
Math. (2), \textbf{136} (1992), no.3, 511–540.
		\bibitem{ref22}M. Wiemeler, Circle actions and scalar curvature. Trans. Amer. Math. Soc. \textbf{368} (2016), no. 4, 2939-2966.
            \bibitem{Zei} R. Zeidler. An index obstruction to positive scalar curvature
on fiber bundles over aspherical manifolds. Algebraic \& Geometric Topology \textbf{17} (2017), 3081-3094.
\bibitem{Zei20}R. Zeidler. Width, largeness and index theory. SIGMA Symmetry Integrability Geom. Methods
Appl. \textbf{16} (2020), no. 127, 15.
		\bibitem{ref26}J. Zhu. Rigidity of area-minimizing 2-spheres in n-manifolds with positive scalar curvature. Proc. Amer. Math. Soc. \textbf{148} (2020), no. 8, 3479–3489.
		\bibitem{ref23}J. Zhu. Width estimate and doubly warped product. Trans. Amer. Math. Soc.
\textbf{374} (2021), no. 2, 1497–1511. 
		\bibitem{ref27}J. Zhu. The Gauss-Bonnet inequality beyond aspherical conjecture, arXiv:2206.07955 (2022), To appear in Math. Ann.
		
	\end{thebibliography}
\end{document}