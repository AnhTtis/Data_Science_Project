
\section{Simulation Setting and Parameters} \label{section:setting}

In this section, we motivate the configuration of our simulator before discussing the entanglement distribution strategies analyzed in the rest of this article.
The strategy to establish long-range entanglement depends strongly on the noise regime as well as the demands of the intended application.
These requirements are often formulated in terms of threshold fidelity \cite{Perseguers2010}.
For example, a QKD application that is robust against the faked Bell state attack requires the end-to-end fidelity of at least 0.83~\cite{Sajeed2019BrightlightDC}.
We use this threshold fidelity as a convenient and useful goal for the considered strategies and show that the purely 1G or purely 2G distribution strategies are unable to reach it under our simulation conditions.
On the other hand, our new hybrid strategies are capable of surpassing this threshold fidelity for some noise regimes which we identify.

% A quantum communication strategy to establish long-range entanglement in a quantum network depends on the noise regime. A secure network, a quantum state fidelity must meet a required threshold \cite{Perseguers2010}, which could be a different value for each different task.
% In this section, we first state the configuration of our simulation and its motivation, where we assume the fidelity threshold of a QKD system to be 0.83, which is minimal requirement against faked Bell state attack \cite{Sajeed2019BrightlightDC}. Secondly, we introduce some strategies considered in this work, and discuss the results on their performances.

We now give details of all the fixed parameters that we use in our simulations.
\begin{enumerate}
    \item We consider a linear chain of repeaters with the distance between the two end nodes being fixed at \num{100} \si{\km}. The link lengths are all equal but the number of repeaters can be varied.
    \item The speed of light in fiber is assumed to be constant at \num{300000} \si{\km/\s}.
    %it serves as a delay of light travelling and also to check the correctness of timing of the simulation during development. 
    \item The depolarizing probability is fixed at $p_{\text{depo}}=0.025$. It should be noted that the entanglement generation over a distance of 50 km using $^{40}$Ca$^{+}$ ion in Ref.\cite{Krutyanskiy2019} reported a Bell pair of fidelity of 0.86 (simulation) and 0.86 $\pm$ 0.03 (experiment), where the simulation accounted for the effect of measured background counts only. Compared to our work, including only depolarizing channel applied directly on the ideal Bell pair, setting $p_{\text{depo}}$ to be approximately 0.0736 results in the output fidelity of approximately 0.86.
    \item The memory lifetime of $\tau = 10 \text{ ms}$. This generous lifetime is typical in a number of physical systems, particularly in ion traps~\cite{Wang2021} and NV centers in diamond~\cite{pompili2021realization}. 
    \item We assumed the effective loss rate to be \num{0.30} \si{dB/km}.
    This is a conservative value; optical fibers with substantially lower attenuation have been used in experiments.
    This value is sufficient to affect the quality of qubits that suffer from a long waiting time.

%\HR{We did not assume photon loss rates in separate media such as fibre, atmosphere, or free-space.} The combined effects of memory time and photon loss are \HR{sufficient} to affect the quality of qubits that suffer from a long waiting time. 
%\PCc{saying what was excluded from the model might cause more trouble than help. Suggest edit: We assumed effect loss rate of our model to be 0.3 dB/km. This has taken into account...list here..}  x
\end{enumerate} 
The above parameters have typical, or worse than typical, values compared to the experimental ones reported in Refs.~\cite{Hofmann2005, Liao2017, Chen1989, Valentini2021AnalysisOP, Abasifard2023}, and are kept fixed in the strategy evaluation. Variable parameters for optimization are the gate error parameter $\lambda_{\text{gate}} \in \{0.0000,0.0005,0.0010,0.0015,0.0020\}$, the measurement error $p_{\text{meas}} \in \{0.0000,0.0025,0.0050,0.0075,0.0100\}$, and the number of hops $h_s \in \{2, 4, 8\}$.

\begin{figure*}[t]
    \centering
    % \includegraphics[width=\textwidth]{contents/strategies.pdf}
    \includegraphics[width=\textwidth]{contents/figures/fig5-strategies-no-shadow-new.pdf}
    \caption{Entanglement distribution strategies. Each strategy shows Bell pairs drawn vertically, with end nodes at the top and bottom and repeaters (if any) in between. Actions are shown left to right, beginning with raw link-level Bell pairs and ending with the high-fidelity end-to-end Bell pair. The numbers on the colored links indicate the number of Bell pairs from the previous round consumed in the current round.
    %\textbf{0G}: Immediately after Bell pairs on both sides of the quantum repeater are ready, entanglement swapping is performed until a Bell pair is distributed to the end nodes. \textbf{1G}: Ss-Dp protocol is employed on link-level Bell pairs first, then entanglement swapping is performed to until the Bell pair is distributed to the end nodes. \textbf{E2E-1G}: Similar to \textbf{0G}, but at the end nodes, Ss-Dp protocol is performed to produced the final Bell pair. \textbf{2G}: At the link-level, 7 Bell pairs will be used to perform Non-local CNOT gate between logical qubits on adjacency nodes, then entanglement swapping is performed on the logical Bell pairs until the end nodes receive a logical Bell pair.  \textbf{HG-PE}: At the link-level, Ss-Dp is performed on physical Bell pairs, then it will be encoded to a logical Bell pair directly without using a Non-local CNOT gate, entanglement swapping is then perform until end nodes produced a logical Bell pair. \textbf{E2E-HG-PE}: Similarly to \textbf{0G} and \textbf{HG-PE}, physical Bell pairs are distributed to the end nodes using \textbf{0G}, then those Bell pairs will be purified and used to encoded logical Bell pair as in \textbf{HG-PE} to produce the final logical Bell pair.
    }
    \label{fig:strategies}
\end{figure*}

We now discuss six different entanglement distribution strategies used in our analysis, as illustrated in \cref{fig:strategies}. The most basic strategy relies on entanglement swapping only, without utilizing any purification or quantum error correction. We denote this strategy by [\textbf{0G}]. It serves as a reference for the performance of other strategies. 
%\HR{We remark that our 0G strategy is equivalent to 2G (NC) in Ref.\cite{muralidharan2016optimal}. \PP{I'm not sure about this statement.}\PCc{agreed. This statement might cause misunderstanding to our claims and does not directly contributed to the main result. Suggest toning down or remove.}} 

We now consider strategies for 1G repeater networks using the Ss-Dp purification protocol.
There are two variations of the 1G-only strategy.
First, the simple 1G strategy, denoted by [\textbf{1G}], is to perform purification at the link-level only.
Entanglement swapping is performed using the successfully purified Bell pairs to produce the final Bell pair between the end nodes.
The second variation, which we call [\textbf{E2E-1G}] strategy, is to first perform entanglement swapping to generate the Bell pairs for the end nodes.
Then the Ss-Dp protocol is performed to produce the final purified Bell pair (hence, the name E2E which stands for end-to-end).
Since producing a purified link-level Bell pair is expected to take longer for [\textbf{1G}] hence delaying the swap, the [\textbf{E2E-1G}] alternative is considered to determine whether it can increase the final end-to-end fidelity and throughput compared to [\textbf{1G}]. 

For the pure [\textbf{2G}] strategy, we use the standard seven-qubit Steane code to encode quantum information into a logical state.
The strategy is to first distributed to link-level physical qubits and used to perform logical non-local CNOT (NCX) gates on logical qubits in the state $|\bar{+}\rangle$ in the left node and $|\bar{0}\rangle$ in the right node to generate one link-level logical Bell pair. After that, the entanglement swapping is used to distribute the logical Bell pairs to the end nodes. QEC is applied before every measurement on a logical qubit, and before and after the application of every logical gate. 

Similarly to the investigation in Ref.~\cite{PhysRevA.93.042338}, it is useful to consider using purification and QEC altogether.
We first enhance the 1G strategies with QEC, which makes the strategies a hybrid between 1G and 2G.
After the Ss-Dp protocol, the purified Bell pairs are used for encoding logical qubits, effectively a logical Bell pair. These hybrid strategies are denoted by [\textbf{HG-PE}], and [\textbf{E2E-HG-PE}], which are the extensions of [\textbf{1G}] and [\textbf{E2E-1G}], respectively. A summary of different strategies used in this work is depicted in \cref{fig:strategies}.



\section{Results} \label{section:result}

Let us now present the simulation results.
In Fig.~\ref{fig:NoiseTesting}, we investigate how the end-to-end fidelity is affected by both the number of hops as well as the role individual sources of noise play in its degradation.
We ran separate simulations for every instance of turning off a particular source of noise, as indicated by the labels on the horizontal axis.
Holding the probability of depolarizing error $p_{\text{depo}}$ constant at 2.5\%, the fidelity of Bell pairs generated between adjacent nodes is independent of the distance.
However, for some strategies, the fidelity does not always decrease as the number of hops increases.
This seemingly counter-intuitive observation is a direct consequence of the fact that the total distance between end nodes is fixed.
Increasing the number of hops, therefore shortens the link length.
Combined with finite memory lifetime, this results in the observed behavior, where simply increasing the number of hops can lead to higher end-to-end fidelity.

\begin{figure*}[t]
    \centering
    \includegraphics[width=0.95\textwidth]{contents/figures/fig6-NoiseTesting_Annotated_2.pdf}
    \caption{End-to-end fidelity in the presence of noise. The horizontal axis indicates that all noise sources are present except the labelled one, where loss is photon loss, \texttt{dep} is depolarizing channel, \texttt{g\_e} is the gate error channel, \texttt{mem\_e} is the memory error channel, and \texttt{meas\_e} is the measurement error channel. While \texttt{All} means that all channels were included in the simulation. Each case is subject to depolarizing error of 2.5\%, gate error of 0.1\%, measurement error of 1\%, memory error of 0.01\si{s}, and loss of 0.3\si{dB/km}.}
    \label{fig:NoiseTesting}
\end{figure*}

Setting the memory lifetime to be infinite, the fidelity decreases as the number of hops increases.
From \cref{fig:fidelity}, simply distributing the Bell pairs fast is not sufficient as 0G also fails to reach the threshold fidelity of 0.83.
Hence, the need for noise suppression techniques is warranted.
However, it is evident that 1G also could not produce Bell pairs that reach this fidelity threshold.
Even its extension E2E-1G suffers from short memory time such that the purification could not offer any advantage over 0G at all.
Even worse, 2G, which requires more physical resources, effectively prolongs the waiting time and yields Bell pairs of fidelity smaller than other strategies.
From the aforementioned results, we regard short memory lifetime as a main problem to be tackled, consistent with recent report \cite{Mol2023}. 

\begin{figure*}[t]
    \centering
    \includegraphics[width=\textwidth]{contents/figures/fig7_Combined_Plot_EndDistance_Fixed_4.pdf}
    
    \caption{The upper plots show the fidelity of the end-to-end Bell pair yielded from each strategy with varying gate error parameter $\lambda_{\text{gate}} \in \{0.0000,0.0005,0.0010,0.0015,0.0020\}$, measurement error $m_e \in \{0.0000,0.0025,0.0050,0.0075,0.0100\}$ and the number of hops $h_s \in \{2, 4, 8\}$. The dashed line is the reference fidelity of value 0.83. The lower plots show the throughput of each strategy on a logarithmic scale.}
    \label{fig:fidelity}
\end{figure*}

As pointed out in Ref.~\cite{PhysRevA.93.042338}, it is useful to consider using purification and QEC together.
A key reason that we consider a hybridization between 1G and 2G is that the logical qubit measurement method described in \cref{subsec:theoryPOV} (which is only applicable to CSS codes such as the Steane code) allows additional classical error correction after measurement. 
Thus, in addition to measurement error, hybridization can also correct errors due to short memory lifetime as well.
As shown in \cref{fig:fidelity}, the [\textbf{HG-PE}] and [\textbf{E2E-HG-PE}] strategies offer relatively higher fidelity than the other strategies.
Specifically, 2-hops of [\textbf{E2E-HG-PE}] outperforms the other strategies in the high error rate regime.
It also reaches the desired threshold in the absence of gate errors.
The drawback of [\textbf{HG-PE}] is that it suffers from gate errors as it uses many more gate operations, similar to other 2G strategies.
Thus, the fidelity falls quickly when gate errors increase.
The [\textbf{E2E-HG-PE}] strategy, on the other hand, uses QEC in the later stages of the protocol.
Hence, it does not induce unnecessary gate errors.
In terms of resource efficiency, the [\textbf{E2E-HG-PE}] strategy also consumes fewer Bell pairs compared to [\textbf{HG-PE}] and does not demand the intermediate repeaters to be 1G or 2G.
%With a strategy that uses [\textbf{0G}] to deliver end-to-end Bell pairs and uses QEC to help with error correction \TT{sounds repetitive}, 
%Our results suggest that distributing high-quality Bell pairs as a starting point and then preserving them over a long period of time with QEC yields high fidelity Bell pair for application usage.
In essence, to deliver end-to-end Bell pairs with QEC, our results suggest that distributing high quality Bell pairs, then preserving them over a long period of time with QEC yields high fidelity Bell pairs for application usage.