\section{Introduction} \label{section:introduction}

%\PP{Key message: Our results suggest that using purification to distribute high-fidelity Bell pairs and preserve them for application using quantum error correction is promising way to achieve near-term quantum network for secure communication.}

%\PP{This section will describe a motivation of quantum network starting from quantum computer point of view to quantum cryptography. As current technologies still faulty, careful design of entanglement needed for quantum network is important to save to cost and achieve useful result. Our simulation show that memory time of qubit is an important factor that contribute to a large portion of final fidelity, thus we design an entanglement distribution such that the high quality entanglement will be delivered using entanglement purification while using quantum error correction to maintain its quality. The summary of each subsection will be briefly described.}

Quantum communication is at the forefront of the second quantum revolution~\cite{dowling2003quantum}.
It is expected to be instrumental in unlocking the true potential of quantum computers~\cite{arute2019quantum,zhong2020quantum,madsen2022quantum} in the form of distributed quantum computing~\cite{cuomo2020towards,caleffi2022distributed}.
Other equally exciting applications include secure quantum key distribution~\cite{ekert1991quantum,yin2017satellite,sasaki2017quantum}, improved arrays of sensing devices~\cite{gottesman2012longer,bartlett2007reference,ilo-okeke2018remote}, and secure private cloud services~\cite{fitzsimons2017private,broadbent2009universal,hayashi2018self}.
These applications will rely on quantum networks~\cite{vanmeter2014quantum,I-D.irtf-qirg-principles-11} to distribute entanglement between their participants in an efficient, fair, and reliable fashion.
Following the evolution path of the classical Internet, it is expected that quantum networks themselves will one day be connected and lead to the ultimate goal of quantum communication in the form of a global quantum internet~\cite{kimble2008quantum,wehner2018quantum,satoh2021attacking} as illustrated in Fig.~\ref{fig:QNCopcept}.

% The recent success of demonstrating advantages of quantum computing over classical counterpart \cite{arute2019quantum, Zhong1460} has drawn excitement from quantum technology enthusiasts. Together with many reports on advancement of quantum computing in the last couple years \cite{Qiskit}, it is now conceivable that quantum computing can be realized for practical use in the near future. However, due to the need of delicate handling of assemble parts and possibly cryogenic systems, quantum computing will likely remain a cloud service where users will subscribe. That means quantum computers will likely be linked via network, famously termed \emph{quantum internet}, which brings another pillar of quantum technology, \emph{quantum communication}.

% Quantum communication exploits quantum properties of light and matter to communicate between two nodes, and processes information beyond the limit of classical communication. Naturally, photons are essential constituents for quantum communication, through which the photonic qubits must interact with the matter qubits from quantum devices, e.g. quantum computers. A poster child for quantum communication has been quantum key distribution (QKD) which has been achieved in fibers and in short-distance free space, as well as via satellite \cite{Yin1140, PhysRevLett.125.263602, Sasaki_2017}, and provides in principle unhackable security (thanks to the no-cloning theorem) of key transmission, and hence encryption.

% \sujin{BQC, quantum memories, schemetic picture of quantum internet}

% Quantum internet is an ambitious goal of connecting quantum computers from all over the world, as illustrated in \cref{fig:QNCopcept}.
Similar to classical networks, quantum networks also utilize light as the primary information carrier between nodes of the network.
Unlike their classical counterparts, arbitrary quantum states cannot be copied faithfully~\cite{park1970concept,wootters1982single,dieks1982communication}, making classical signal amplification schemes impossible to apply \cite{chia2019phase}.
Coupled with the fact that quantum nodes exchange information via single photons, attenuation becomes a major obstacle to scaling of quantum networks.
One approach to overcoming the exponential attenuation in fiber is to segment the entire connection between the quantum nodes via the use of \emph{quantum repeaters}~\cite{briegel1998quantum,azuma2022quantum}.
Link-level entanglement is distributed between neighboring repeater stations, which then splice them via \emph{entanglement swapping}~\cite{PhysRevLett.71.4287, pan1998experimental} into an end-to-end entangled connection.
% Due to the probabilistic nature of entanglement swapping with linear optics, the end-to-end entanglement distribution rate decreases polynomially with the distance.
This repeat-until-success approach means that the end-to-end entanglement distribution rate decreases polynomially.

% The main obstruction needed to overcome is a noise which attack quantum information as its sent along the distant path. Unlike a classical network, where signals can be modified and amplified with intermediate stations along the way, a quantum state cannot be copied and re-sent. Hence, distributing quantum resources such as entanglement and qubits over a long distance network presents an enormous challenge for the signal boosters must also be quantum systems. This requires quantum memories and light-matter interface. An analogue of a classical amplifier or signal booster is called a \emph{quantum repeater} proposed as an intermediate node to reduce the distance quantum information has to travel. Nevertheless, noises are still present in the system despite shortening distance. The capability of a quantum repeater is crucial in the quest to achieve a better performance of a network. 

%\textcolor{red}{literature review}

Besides entanglement swapping, another important role of the quantum repeaters is to participate in error management.
Decoherence in quantum memories, imperfect quantum gate operations, and measurement errors all introduce unwanted \emph{operational errors}, which must be mitigated in order to ensure end-to-end entanglement remains above the desired threshold fidelity required by the application.

Error management schemes can be classified into three generations~\cite{muralidharan2016optimal}.
The first generation (1G) of repeaters relies on entanglement purification schemes~\cite{briegel1998quantum,bennett1996purification,pan2001entanglement,dur2007entanglement} to detect errors on physical qubits.
The need for two-way classical communication between distant nodes in the network makes this generation of repeaters unsuitable for long-distance quantum communication.
However, due to their relatively modest hardware requirements and low resource overhead, they are expected to be the primary method of error management in early implementations of small quantum networks.
The second generation (2G) of repeaters~\cite{jiang2009quantum,fowler2010surface} uses quantum error correction (QEC) to both detect and correct errors.
This generation of repeaters avoids the issue of two-way communication and is therefore more suitable for long-distance quantum communication, albeit at the price of stricter hardware requirements and larger resource overhead.
The third generation (3G) of quantum repeaters~\cite{muralidharan2014ultrafast,munro2012quantum} relies on quantum error correction to correct both loss and operational errors and does not require entanglement swapping or pre-shared long-distance entanglement as a means of communication.
Thus, 3G repeaters place more strict requirements on operational fidelity and place more complex demands on the hardware than 2G repeaters.
% Further extension of a quantum repeater is proposed by \cite{Muralidharan2016}, separating quantum repeater into three generations, discriminated by their approaches of handling noises. A first generation uses the entanglement purification; a second generation uses quantum error correction both required entanglement swapping to establish long-range communication, while a third generation enables sending quantum information directly through a quantum channel as a one-way communication.

The performance of individual repeater generations has been analyzed and contrasted in previous reports; see for examples Refs.~\cite{muralidharan2016optimal,dur1999quantum,fujii2009entanglement,jansen2022enumerating}.
Apart from the generation of the repeater network, the network performance also depends on the entanglement distribution policy \cite{khatri2021policies}.
In this work, we present a comprehensive study of a number of entanglement distribution strategies and compare their performance under realistic noise conditions in terms of end-to-end fidelity and throughput.

We show by simulation that using either purification or QEC alone is not enough to produce end-to-end Bell pairs with sufficiently high fidelity above a certain threshold in the noise regime we investigated.
Instead of deploying QEC and purification techniques separately, we utilize them together by introducing a hybrid strategy, which we refer to as \emph{purified encoding} (PE).
PE first increases the fidelity of physical Bell pairs by using purification and then encodes a logical Bell pair.
We investigate two variations of this hybrid strategy. First, we perform PE immediately after producing a physical Bell pair at the link level. Second, we employ PE at the end nodes. %Each technique could be used before the other.
We identify the error parameter regime,  quantum gate errors, and measurement errors, where PE outperforms strategies based purely on 1G or 2G.
% As a cost of quantum network infrastructure might be astronomical high, a simulator is a necessary tool for assessing the performance of the network to effectively design and compare strategies for conducting real experiments. Our simulator can keep track of Pauli errors of the system instead of those of the quantum states themselves. With a direct fidelity estimate using the stabilizer counting method, this allows us to simulate the network efficiently without spending exponentially cost of memory space as traditional method. Even with the disadvantages of restricting to only Pauli errors, we could evaluate performance of quantum error correcting code with ease. 

Our paper is organized as follows.
In \cref{section:fromTheoryToSimulation}, we give a brief overview of the relevant concepts concerning quantum networks, quantum repeaters, entanglement swapping, entanglement purification, and quantum error correction.
We then provide a detailed description of our simulator, its assumptions, noise models, method for fidelity evaluation, as well as validation with theoretical models in simple cases where analytical results are obtained. 
We discuss, in \cref{section:setting}, the simulation assumptions and the parameter values. 
In \cref{section:result}, we present the simulation results comparing the performance of various strategies for distributing Bell pairs in various regimes of noises.
In this work, we consider noise from depolarizing errors in a quantum channel, memory time of a qubit, photon loss, gate errors, and measurement errors.
In a linear chain of nodes, the distance between the two end nodes is fixed while we vary the number of quantum repeaters in between.
Conclusions and discussion of our results are summarized in \cref{section:discussion}.

\begin{figure}
    \centering
    % \includegraphics[width=0.9\textwidth]{contents/qwanta-exporting-figure-1-2.pdf}
    \includegraphics[width=0.95\textwidth]{contents/figures/fig1-quantum-network-concept-no-shadow-grad.pdf}
    \caption{Concept of a worldwide quantum Internet. Inter-continental quantum communication may be realized with the aid of a network of satellites \cite{Khatri2021, Mol2023}, which act as long-distance Bell pair distributors for main hubs such as Node A and Node B. These hubs slice the long-distance Bell pairs with short-range entangled states generated by the local networks. }
    \label{fig:QNCopcept}
\end{figure}

% \Figure[t!](topskip=0pt, botskip=0pt, midskip=0pt){contents/quantum-internet.pdf}
% {Concept of world-wide quantum internet. Inter-continental quantum communication may realize by an aid of network of satellites, which act as a Bell pair distributor for main hubs such as Node A and Node B in figure. The hubs provide an entanglement to the nodes in their network using their shared Bell pairs.\label{fig:QNCopcept}}



