%% ****** Start of file apstemplate.tex ****** %
%%
%%
%%   This file is part of the APS files in the REVTeX 4.2 distribution.
%%   Version 4.2a of REVTeX, January, 2015
%%
%%
%%   Copyright (c) 2015 The American Physical Society.
%%
%%   See the REVTeX 4 README file for restrictions and more information.
%%
%
% This is a template for producing manuscripts for use with REVTEX 4.2
% Copy this file to another name and then work on that file.
% That way, you always have this original template file to use.
%
% Group addresses by affiliation; use superscriptaddress for long
% author lists, or if there are many overlapping affiliations.
% For Phys. Rev. appearance, change preprint to twocolumn.
% Choose pra, prb, prc, prd, pre, prl, prstab, prstper, or rmp for journal
%  Add 'draft' option to mark overfull boxes with black boxes
%  Add 'showkeys' option to make keywords appear
\documentclass[aps,prb,preprint,groupedaddress]{revtex4-2}
%\documentclass[aps,prl,preprint,superscriptaddress]{revtex4-2}
%\documentclass[aps,prl,reprint,groupedaddress]{revtex4-2}

% You should use BibTeX and apsrev.bst for references
% Choosing a journal automatically selects the correct APS
% BibTeX style file (bst file), so only uncomment the line
% below if necessary.
%\bibliographystyle{apsrev4-2}
\usepackage{graphicx}
\usepackage{amsmath}
\usepackage{rotating}
\begin{document}

% Use the \preprint command to place your local institutional report
% number in the upper righthand corner of the title page in preprint mode.
% Multiple \preprint commands are allowed.
% Use the 'preprintnumbers' class option to override journal defaults
% to display numbers if necessary
%\preprint{}

%Title of paper
\title{Supplementary Material to the article ``A quantitative variational phase field framework"}

\author{Arnab Mukherjee$^{1,2}$} \email[]{arnab.mukherjee@northwestern.edu}
\author{James A. Warren$^2$}
\author{Peter W. Voorhees$^{1,3,4}$}


%\homepage[]{Your web page}
%\thanks{}
%\altaffiliation{}
\affiliation{$^1$Center for Hierarchical Materials Design, Northwestern University, Evanston, Illinois 60208, USA}
\affiliation{$^2$Material Measurement Laboratory, National Institute of Standards and Technology, Gaithersburg, Maryland 20899, USA}
\affiliation{$^3$Department of Materials Science and Engineering, Northwestern University, Evanston, Illinois 60208, USA}
\affiliation{$^4$Department of Engineering Sciences and Applied Mathematics, Northwestern University, Evanston, Illinois 60208, USA}
% repeat the \author .. \affiliation  etc. as needed
% \email, \thanks, \homepage, \altaffiliation all apply to the current
% author. Explanatory text should go in the []'s, actual e-mail
% address or url should go in the {}'s for \email and \homepage.
% Please use the appropriate macro foreach each type of information

% \affiliation command applies to all authors since the last
% \affiliation command. The \affiliation command should follow the
% other information
% \affiliation can be followed by \email, \homepage, \thanks as well.

%\email[]{Your e-mail address}
%\homepage[]{Your web page}
%\thanks{}
%\altaffiliation{}


%Collaboration name if desired (requires use of superscriptaddress
%option in \documentclass). \noaffiliation is required (may also be
%used with the \author command).
%\collaboration can be followed by \email, \homepage, \thanks as well.
%\collaboration{}
%\noaffiliation

\date{\today}

\begin{abstract}
% insert abstract here
In the supplementary material, we study the equilibrium properties of the phase-field model and derive expression for the interfacial energy. The non-dimesionalization procedure and the complete thin-interface asymptotic analysis (until second order) for the model with the assumption $\mu_s = \mu_l$ are derived. The asymptotic analysis for the model $\mu_s \neq \mu_l$ until first order is also presented.
\end{abstract}

% insert suggested keywords - APS authors don't need to do this
%\keywords{}

%\maketitle must follow title, authors, abstract, and keywords
\maketitle

\section{Equilibrium properties}
\subsection{Interfacial profile}
The free energy of the system is given by
\begin{eqnarray}\label{eq1}
    \mathcal{\overline{F}} = \int \Big[Hf_{\mathrm{dw}}(\phi) + \frac{\sigma}{2}|\nabla\phi|^2 + f_s(c_s)h(\phi) + f_l(c_l)\{1-h(\phi)\}\Big] \mathrm{d}V  \nonumber \\
    + \int \lambda \Big[c - \{ c_s h(\phi) + c_l(1-h(\phi))\}\Big] \mathrm{d}V 
    + \int \lambda^{\prime}(c-c_o) \mathrm{d} V
\end{eqnarray}
where, the last integral takes into account the mass conservation with $c_o$ being the alloy composition. At equilibrium, the individual variation with respect to each variable will be equal to zero which implies
\begin{equation}\label{eq2}
    \frac{\delta \mathcal{\overline{F}}}{\delta c}  =  \lambda - \lambda^{\prime} = 0 ,
\end{equation}
\begin{equation}\label{eq3}
   \Rightarrow \lambda  =  \lambda^{\prime} .
\end{equation}
\begin{equation}\label{eq4}
    \frac{\delta \mathcal{\overline{F}}}{\delta c_s}  =  \frac{\partial f_s(c_s)}{\partial c_s}h(\phi_o) - \lambda h(\phi_o) = 0 ,
\end{equation}
\begin{equation}\label{eq5}
   \Rightarrow \frac{\partial f_s(c_s)}{\partial c_s}  =  \lambda .
\end{equation}
\begin{equation}\label{eq6}
    \frac{\delta \mathcal{\overline{F}}}{\delta c_l} = \frac{\partial f_l(c_l)}{\partial c_l}\{1-h(\phi_o)\} - \lambda \{1-h(\phi_o)\} = 0 ,
\end{equation}
\begin{equation}\label{eq7}
    \Rightarrow \frac{\partial f_l(c_l)}{\partial c_l} = \lambda .
\end{equation}
where $\phi_o$ denotes the equilibrium phase field profile and Eqs.~~(\ref{eq5}) and (\ref{eq7}) imply 
\begin{equation}\label{eq8}
    c_s(x) = c_s^e
\end{equation}
\begin{equation}\label{eq9}
    c_l(x) = c_l^e
\end{equation}
where $c_s^e$ and $c_l^e$ are the equilibrium solid and liquid concentrations respectively. Moving on to the variation with respect to $\phi$ we have
\begin{equation}\label{eq10}
    \frac{\delta \mathcal{\overline{F}}}{\delta \phi} = Hf_{\mathrm{dw}}^{\prime}(\phi_o) - \sigma \frac{\mathrm{d}^2\phi_o}{\mathrm{d}x^2} + \{f_s(c_s^e) - f_l(c_l^e) - \lambda(c_s^e - c_l^e)\}h^{\prime}(\phi_o) = 0 .
\end{equation}
Multiplying both sides by $\frac{\mathrm{d}\phi_o}{\mathrm{d}x}$ and integrating within the limits $x = -\infty$ and $x = +\infty$,
\begin{eqnarray}\label{eq11}
    \int_{-\infty}^{+\infty}Hf_{\mathrm{dw}}^{\prime}(\phi_o)\frac{\mathrm{d}\phi_o}{\mathrm{d}x}\mathrm{d}x + \int_{-\infty}^{+\infty}\sigma \frac{\mathrm{d}^2\phi_o}{\mathrm{d}x^2}\frac{\mathrm{d}\phi_o}{\mathrm{d}x}\mathrm{d}x \nonumber \\
    +\{f_s(c_s^e) - f_l(c_l^e) - \lambda(c_s^e-c_l^e)\}\int_{-\infty}^{+\infty} h^{\prime}(\phi_o)\frac{\mathrm{d}\phi_o}{\mathrm{d}x}\mathrm{d}x = 0
\end{eqnarray}
Simplifying and expressing in terms of perfect integrals we obtain
\begin{eqnarray}\label{eq12}
    \int_{-\infty}^{+\infty} H \frac{\mathrm{d}f_{\mathrm{dw}}(\phi_o)}{\mathrm{d}x}\mathrm{d}x + \int_{-\infty}^{+\infty}\frac{\sigma}{2}\frac{\mathrm{d}}{\mathrm{d}x}\Big(\frac{\mathrm{d}\phi_o}{\mathrm{d}x}\Big)^2\mathrm{d}x \nonumber \\
    +\{f_s(c_s^e) - f_l(c_l^e) - \lambda(c_s^e-c_l^e)\}\int_{-\infty}^{+\infty}\frac{\mathrm{d}h(\phi_o)}{\mathrm{d}x}\mathrm{d}x = 0.
\end{eqnarray}
Integrating we obtain
\begin{eqnarray}\label{eq13}
    H f_{\mathrm{dw}}(\phi_o)\Big|_{-\infty}^{+\infty} + \frac{\sigma}{2}\Big(\frac{\mathrm{d}\phi_o}{\mathrm{d}x}\Big)^2\Big|_{-\infty}^{+\infty}  +\{f_s(c_s^e) - f_l(c_l^e) - \lambda(c_s^e-c_l^e)\} h(\phi_o)\Big|_{-\infty}^{+\infty} = 0 .
\end{eqnarray}
Using the boundary condition $\phi = 1$ and $\frac{\mathrm{d}\phi_o}{\mathrm{d}x} = 0$  at $x = -\infty$ and $\phi = 0$ and $\frac{\mathrm{d}\phi_o}{\mathrm{d}x} = 0$ at $x = +\infty$ we have
\begin{equation}\label{eq14}
    \lambda = \frac{f_s(c_s^e) - f_l(c_l^e)}{c_s^e - c_l^e}.
\end{equation}
Therefore, $c_s^e$ and $c_l^e$ can be found from the common tangent construction. Substituting the value of $\lambda$ in the equilibrium condition of phase-field we have
\begin{equation}\label{eq15}
    H f_{\mathrm{dw}}^{\prime}(\phi_o) - \sigma \frac{\mathrm{d}^2\phi_o}{\mathrm{d}x^2} = 0
\end{equation}
Introducing the notation $W^2 = \sigma /H$ which is the measure of the interfacial width we have
\begin{equation}\label{eq16}
    f_{\mathrm{dw}}^{\prime}(\phi_o) - W^2\frac{\mathrm{d}^2\phi_o}{\mathrm{d}x^2} = 0.
\end{equation}
Multiplying both sides by $\frac{\mathrm{d}\phi_o}{\mathrm{d}x}$ and integrating we have
\begin{equation}\label{eq17}
    \frac{\mathrm{d}\phi_o}{\mathrm{d}x} = \frac{\sqrt{2}}{W}\sqrt{f_{\mathrm{dw}}(\phi_o)},
\end{equation}
where the integration constant is deduced to be zero. Integrating again we obtain the equilibrium phase-field profile as
\begin{equation}\label{eq18}
    \phi_o(x) = \frac{1}{2}\Bigg[ 1 - \tanh \left(\frac{x}{\sqrt{2}W}\right)\Bigg],
\end{equation}
where we have employed the relation $\phi_o = 1/2$ at $x = 0$ to remove the translation degree of freedom of the profile. The equilibrium concentration profile writes as
\begin{equation}\label{eq19}
    c_o(x) = c_s^e h(\phi_o(x)) + c_l^e\{1- h(\phi_o(x)) \}
\end{equation}

\begin{figure}
    \centering
    \includegraphics[scale = 0.28]{supfig1}
    \caption{Numerically evaluated (a) equilibrium phase-field profile and (b) equilibrium phase concentrations $c_s$ and $c_l$. The overall solute concentration $c$ is given by Eq.~(\ref{eq19}). Si-9 at\% As is used as the representative system.}
    \label{supfig1}
\end{figure}


\subsection{Interfacial energy}
Following Cahn and Hilliard, we define the interfacial energy $\gamma$ as the difference per unit area of the interface between the actual free energy of the system and that it would have if the properties of the phases were continuous throughout. Assuming the dividing surface at $x = 0$ we have
\begin{eqnarray}\label{eq20}
    \gamma = \int_{-\infty}^0\Bigg[ Hf_{\mathrm{dw}}(\phi_o) + \frac{\sigma}{2}\Big(\frac{\mathrm{d}\phi_o}{\mathrm{d}x}\Big)^2 + f(c_s^e,c_l^e,\phi_o) + \lambda[c_o - \{c_s^e h(\phi_o) + c_l^e(1-h(\phi_o))\}] \nonumber \\
    - f(c_s^e,1)-\lambda[c_o-c_s^e] \Bigg]\mathrm{d}x + \int_{0}^{+\infty} \Bigg[Hf_{\mathrm{dw}}(\phi_o) + \frac{\sigma}{2}\Big(\frac{\mathrm{d}\phi_o}{\mathrm{d}x}\Big)^2 + f(c_s^e,c_l^e,\phi_o) \nonumber \\
    + \lambda[c_o - \{c_s^e h(\phi_o) + c_l^e(1-h(\phi_o))\}]- f(c_l^e,0)-\lambda[c_o-c_l^e] \Bigg]\mathrm{d}x .
\end{eqnarray}
From equilibrium condition of the phase-field i.e. $\frac{\delta \mathcal{\overline{F}}}{\delta \phi} = 0$ we have,
\begin{equation}\label{eq21}
    Hf_{\mathrm{dw}}^{\prime}(\phi_o) - \sigma \frac{\mathrm{d}^2\phi_o}{\mathrm{d}x^2} + \frac{\partial f(c_s,c_l,\phi)}{\partial \phi} - \lambda (c_s - c_l)h^{\prime}(\phi_o) = 0.
\end{equation}
Multiplying both sides by $\frac{\mathrm{d}\phi_o}{\mathrm{d}x}$ we have
\begin{equation}\label{eq22}
   Hf_{\mathrm{dw}}^{\prime}(\phi_o)\frac{\mathrm{d}\phi_o}{\mathrm{d}x} - \sigma \frac{\mathrm{d}^2\phi_o}{\mathrm{d}x^2}\frac{\mathrm{d}\phi_o}{\mathrm{d}x} + \frac{\partial f(c_s,c_l,\phi)}{\partial \phi}\frac{\mathrm{d}\phi_o}{\mathrm{d}x} - \lambda (c_s - c_l)h^{\prime}(\phi_o)\frac{\mathrm{d}\phi_o}{\mathrm{d}x} = 0.
\end{equation}
Using chain-rule we have
\begin{equation}\label{eq23}
    \frac{\mathrm{d}f}{\mathrm{d}x} = \frac{\partial f}{\partial c_s}\frac{\mathrm{d}c_s}{\mathrm{d}x} + \frac{\partial f}{\partial c_l}\frac{\mathrm{d}c_l}{\mathrm{d}x} + \frac{\partial f}{\partial \phi}\frac{\mathrm{d}\phi_o}{\mathrm{d}x} .
\end{equation}
Substituting for $\frac{\partial f}{\partial c_s}$ and $\frac{\partial f}{\partial c_l}$ we have
\begin{equation}\label{eq24}
    \frac{\mathrm{d}f}{\mathrm{d}x} = \lambda h(\phi_o)\frac{\mathrm{d}c_s}{\mathrm{d}x} + \lambda \{1-h(\phi_o)\}\frac{\mathrm{d}c_l}{\mathrm{d}x} + \frac{\partial f}{\partial \phi}\frac{\mathrm{d}\phi_o}{\mathrm{d}x} .
\end{equation}
The first two terms on the right hand side of the equation can be re-expressed as
\begin{equation}\label{eq25}
   \frac{\mathrm{d}f}{\mathrm{d}x} = \lambda \frac{\mathrm{d}}{\mathrm{d}x} [c_s h(\phi_o) + c_l\{1-h(\phi_o)\}] - \lambda (c_s - c_l)h^{\prime}(\phi_o) \frac{\mathrm{d}\phi_o}{\mathrm{d}x} + \frac{\partial f}{\partial \phi}\frac{\mathrm{d}\phi_o}{\mathrm{d}x}.
\end{equation}
Hence,
\begin{equation}\label{eq26}
   \frac{\partial f}{\partial \phi}\frac{\mathrm{d}\phi_o}{\mathrm{d}x} =  \frac{\mathrm{d}f}{\mathrm{d}x} - \lambda \frac{\mathrm{d}}{\mathrm{d}x} [c_s h(\phi_o) + c_l\{1-h(\phi_o)\}] + \lambda (c_s - c_l)h^{\prime}(\phi_o) \frac{\mathrm{d}\phi_o}{\mathrm{d}x}.
\end{equation}
Substituting the above expression back in Eq.~(\ref{eq22}),
\begin{equation}\label{eq27}
     Hf_{\mathrm{dw}}^{\prime}(\phi_o)\frac{\mathrm{d}\phi_o}{\mathrm{d}x} - \sigma \frac{\mathrm{d}^2\phi_o}{\mathrm{d}x^2}\frac{\mathrm{d}\phi_o}{\mathrm{d}x} + \frac{\mathrm{d}f}{\mathrm{d}x} - \lambda \frac{\mathrm{d}}{\mathrm{d}x} [c_s h(\phi_o) + c_l\{1-h(\phi_o)\}] = 0.
\end{equation}
Integrating within the limits $-\infty$ to $x$ we have
\begin{equation}\label{eq28}
    Hf_{\mathrm{dw}}(\phi_o)\Big|_{-\infty}^{x} - \frac{\sigma}{2}\Big(\frac{\mathrm{d}\phi_o}{\mathrm{d}x}\Big)^2 \Big|_{-\infty}^{x} + f(c_s,c_l,\phi_o)\Big|_{-\infty}^{x} - \lambda[c_sh(\phi_o)+c_l\{1-h(\phi_o)\}]\Big|_{-\infty}^{x} = 0.
\end{equation}
Substituting the limits and re-arranging 
\begin{equation}\label{eq29}
    \frac{\sigma}{2}\Big(\frac{\mathrm{d}\phi_o}{\mathrm{d}x}\Big)^2 = Hf_{\mathrm{dw}}(\phi_o) + f(c_s^e,c_l^e,\phi_o) - \lambda[c_s^eh(\phi_o) + c_l^e\{1-h(\phi_o)\}]-f(c_s^e,1) + \lambda c_s^e .
\end{equation}
The above expression is the statement of equipartition of energy which states that at equilibrium the bulk part and the gradient part contribute equally to the phase-field profile. Similarly integrating Eq.~(\ref{eq27}) within the limits $x$ to $+\infty$ we have
\begin{equation}\label{eq30}
    \frac{\sigma}{2}\Big(\frac{\mathrm{d}\phi_o}{\mathrm{d}x}\Big)^2 = Hf_{\mathrm{dw}}(\phi_o) + f(c_s^e,c_l^e,\phi_o) - \lambda[c_s^eh(\phi_o) + c_l^e\{1-h(\phi_o)\}]-f(c_l^e,0) + \lambda c_l^e .
\end{equation}
Substituting Eqs.~(\ref{eq29}) and (\ref{eq30}) in Eq.~(\ref{eq20}) we obtain
\begin{equation}\label{eq31}
    \gamma = \int_{-\infty}^{0}\sigma\Big(\frac{\mathrm{d}\phi_o}{\mathrm{d}x}\Big)^2 \mathrm{d}x +  \int_{0}^{+\infty}\sigma\Big(\frac{\mathrm{d}\phi_o}{\mathrm{d}x}\Big)^2 \mathrm{d}x,
\end{equation}
which can be combined to give
\begin{equation}\label{eq32}
    \gamma = \sigma\int_{-\infty}^{+\infty}\Big(\frac{\mathrm{d}\phi_o}{\mathrm{d}x}\Big)^2 \mathrm{d}x.
\end{equation}
Employing Eq.~(\ref{eq17}) the above expression can be re-written as
\begin{equation}\label{eq33}
    \gamma = \sqrt{2}WH \int_{0}^{1}\sqrt{f_{\mathrm{dw}}(\phi_o)}\mathrm{d}\phi_o.
\end{equation}
Substituting $f_{\mathrm{dw}}(\phi_o) = \phi_o^2(1-\phi_o)^2$ and integrating we obtain
\begin{equation}\label{eq34}
    \gamma = IWH
\end{equation}
where $I = 1/3\sqrt{2}$.

\section{Non-dimensionalization}
To perform the asymptotic analysis, we first non-dimensionalize the governing equations and make certain choices on the scale of the various terms. As an example we show the non-dimensionalization procedure for the form of the governing equations with $g(\phi) = h(\phi)$ of the model with assumption $\mu_s = \mu_l$. The non-dimenisonalization of the general model $\mu_s \neq \mu_l$ follows along similar lines. The dimensional form of the governing equations write as
\begin{align}\label{phi}
    \Bigg[\frac{1}{M_{\phi}} & + \frac{h(\phi)\{1-h(\phi)\}}{M_s\{1-h(\phi)\} + M_l h(\phi)}\frac{1}{|\nabla\phi|^2}\Bigg]\frac{\partial\phi}{\partial t} =  \sigma \nabla^2\phi -  H f_{dw}^{\prime}(\phi) -  \{f_s(c_s) - f_l(c_l)   \nonumber \\
    &- \mu(c_s - c_l)\}h^{\prime}(\phi) + \frac{(\tilde{c}_s-\tilde{c}_l)}{|\nabla\phi|}\frac{(M_s - M_l)h(\phi)\{1-h(\phi)\}}{M_s\{1-h(\phi)\} + M_l h(\phi)} \nabla\mu \cdot \frac{\nabla\phi}{|\nabla\phi|}
\end{align}
and
\begin{align}\label{conc}
    \frac{\partial c}{\partial t} = \nabla \cdot \Bigg[\frac{M_sM_l}{M_s\{1-h(\phi)\} + M_l h(\phi)}\nabla_n \mu & + \{M_sh(\phi) + M_l(1-h(\phi))\}\nabla_t\mu  \nonumber \\
    + & \frac{(\tilde{c}_s-\tilde{c}_l)}{|\nabla\phi|}\frac{(M_s-M_l)h(\phi)\{1-h(\phi)\}}{M_s\{1-h(\phi)\} + M_l h(\phi)}\frac{\nabla\phi}{|\nabla\phi|}\frac{\partial\phi}{\partial t}\Bigg]
\end{align}


To non-dimensionalize the above Eqs.~(\ref{phi}) and (\ref{conc}), we introduce dimensionless quantities (denoted by tilde) : $\mathbf{x} = d_o\tilde{\mathbf{x}}$, $t = \frac{d_o^2}{D_l}\tilde{t}$ and $E = H\tilde{E}$ where $d_o$ is the capillary length and $D_l$ is the diffusivity of the liquid phase. Furthermore, it is imperative to represent the mobilites in terms of the diffusivities as $M_s = D_s\frac{\partial c_s}{\partial \mu}$ and $M_l = D_l\frac{\partial c_l}{\partial \mu}$. Therefore, Eq.~({\ref{phi}}) writes as
\begin{align}\label{nondim1}
    \Big[ \frac{\tau D_l}{d_o^2} + \frac{X}{H}\frac{(\overline{c_s} - \overline{c_l})^2}{|\nabla\phi|^2}\tilde{\alpha_1}(\phi)\Big]\frac{\partial\phi}{\partial\tilde{t}} = & \frac{W^2}{d_o^2}\tilde{\nabla}^2\phi - f_{\mathrm{dw}}^{\prime}(\phi) -  \frac{X}{H}\{\tilde{f_s}(c_s) - \tilde{f_l}(c_l)  \nonumber \\
    & - \tilde{\mu}(c_s - c_l) \}h^{\prime}(\phi) +  \frac{X}{H}\frac{(\overline{c_s}-\overline{c_l})}{|\nabla\phi|}\tilde{a}(\phi))\tilde{\nabla}\tilde{\mu}\cdot\frac{\tilde{\nabla}\phi}{|\tilde{\nabla}\phi|},\nonumber \\
\end{align}
where $\tau = 1/M_{\phi}H$ and $X$is the scale of the driving force. We introduce non-dimensional parameter $\alpha = \frac{\tau D_l}{W^2}$, $\epsilon = \frac{W}{d_o}$ and $\Lambda = \frac{X}{H}$
The interfacial energy $\gamma$ and the interfacial width $W$ are related by the well height $H$ and gradient energy coefficient $\sigma$ as
\begin{equation}\label{nondim2}
    \gamma = a_1 \sqrt{\sigma H},
\end{equation}
and 
\begin{equation}\label{nondim3}
    W = \sqrt{\frac{\sigma}{H}},
\end{equation}
where $a_1$ is a constant related to the specific form of the chosen double well potential and is equal to $I$ as given by Eq.~(\ref{eq34}). Furthermore, it is to noted that the capillary length is set by the ratio of the surface energy to the driving force i.e. $d_o = \frac{\gamma}{X}$. Using Eqs.~(\ref{nondim2}) and (\ref{nondim3}), $d_o = a_1 W\frac{H}{X}$, implies $\Lambda = a_1\epsilon$. We observe that as $\epsilon \rightarrow 0$, $\Lambda \rightarrow 0$. Physically this corresponds to decreasing the interfacial width $W$ and simultaneously raising the well height $H$, while holding the interfacial energy and capillary length constant. Substituting the non-dimensional parameters and scaling, Eq.~(\ref{nondim1}) translates to
\begin{align}\label{nondim4}
    \Big[ \alpha\epsilon^2 + a_1\epsilon \frac{(\tilde{c}_s - \tilde{c}_l)^2}{|\tilde{\nabla}\phi|^2}\tilde{\alpha_1}(\phi)\Big]\frac{\partial \phi}{\partial \tilde{t}} = \epsilon^2\tilde{\nabla}^2\phi - f_{\mathrm{dw}}^{\prime}(\phi) - & a_1\epsilon\{\tilde{f_s}(c_s) - \tilde{f_l}(c_l) - \tilde{\mu}(c_s - c_l) \}h^{\prime}(\phi) \nonumber \\
   & + a_1\epsilon\frac{(\tilde{c}_s-\tilde{c}_l)}{|\tilde{\nabla}\phi|}\tilde{a}(\phi)\tilde{\nabla}\tilde{\mu}\cdot\frac{\tilde{\nabla}\phi}{|\tilde{\nabla}\phi|}, \nonumber \\
\end{align}
where
\begin{equation}\label{nondim5}
    \tilde{a}(\phi) = \frac{\chi\frac{D_s}{D_l}-1}{\chi\frac{D_s}{D_l}\{1-h(\phi)\} + h(\phi)}
\end{equation}
and
\begin{equation}\label{nondim6}
    \tilde{\alpha_1}(\phi) = \frac{h(\phi)\{1-h(\phi)\}}{\chi\frac{D_s}{D_l}\{1-h(\phi)\}+ h(\phi)}.
\end{equation}
$\chi$ is the thermodynamic factor equal to $\frac{\partial c_s}{\partial \mu}/\frac{\partial c_l}{\partial\mu}$.
Similarly, the non-dimensional form of concentration equation can be written as
\begin{equation}\label{nondim7}
    \frac{\partial c}{\partial \tilde{t}} = \tilde{\nabla} \cdot \Bigg[ \frac{\partial c_l}{\partial \mu}q_n(\phi)\tilde{\nabla}_n\tilde{\mu} + \frac{\partial c_l}{\partial \mu}q_t(\phi)\tilde{\nabla}_t\tilde{\mu} + \frac{(\tilde{c}_s - \tilde{c}_l)}{|\tilde{\nabla}\phi|}\tilde{a}(\phi)\frac{\tilde{\nabla}\phi}{|\tilde{\nabla}\phi|}\frac{\partial\phi}{\partial\tilde{t}}\Bigg]
\end{equation}
where
\begin{equation}
    q_n(\phi) = \frac{\chi \frac{D_s}{D_l}}{\chi \frac{D_s}{D_l}\{1-h(\phi)\} + h(\phi)}
\end{equation}
and
\begin{equation}
    q_t(\phi) = \chi \frac{D_s}{D_l} h(\phi) + 1 - h(\phi)
\end{equation}
For the sake of compactness of notation we henceforth remove the tildes. 

\section{Asymptotic analysis}
We perform the asymptotic analysis by expanding the field variables in terms of a small parameter which in the present case is $\epsilon$. The solution domain is divided into outer (farther from the interface) and inner (closer to the interface) regions where the field variables vary slowly and rapidly respectively. For instance in the inner region the field variables are expanded as
\begin{equation}\label{asy1}
    \phi = \phi_0 + \epsilon\phi_1 + \epsilon^2\phi_2 ,
\end{equation}
\begin{equation}\label{asy2}
    c = c_0 + \epsilon c_1 + \epsilon^2 c_2 ,
\end{equation}
\begin{equation}\label{asy3}
    c_s = c_{s,0} + \epsilon c_{s,1} + \epsilon^2 c_{s,2} ,
\end{equation}
\begin{equation}\label{asy4}
    c_l = c_{l,0} + \epsilon c_{l,1} + \epsilon^2 c_{l,2}.
\end{equation}
\begin{equation}\label{asy5}
    \mu = \mu_0 + \epsilon \mu_1 + \epsilon^2 \mu_2
\end{equation}
The superscript on $\epsilon$ denotes power, where as the subscript on the field variables denotes the order of the solution. The power series is substituted in the governing equations and terms are re-grouped and solved for each power of $\epsilon$. The solutions from the inner region are matched to the outer solution at each order.

In the outer region it can be shown that leading order $\phi$ assumes constant values of $1$ and $0$ in the solid and the liquid. The higher order corrections trivially turn out to be zero. Similarly the $c$ equation obeys Fick's law of diffusion at all orders. In the inner region, the equations are re-casted in a curvilinear co-ordinate system with the reference frame attached to the interface. $r$ and $s$ measures the direction normal and along the interface respectively. Furthermore, the distance in the normal direction of the interface is scaled as $\eta = r/\epsilon$.  We transform all the operators as follows,
\begin{equation}\label{asy6}
    \frac{\partial}{\partial t}  =  -\frac{V_n}{\epsilon}\frac{\partial}{\partial \eta} + \frac{\mathrm{d}}{\mathrm{d}t} - V_t\frac{\partial}{\partial s}
\end{equation}
\begin{equation}\label{asy7}
    \nabla = \frac{\hat{n}}{\epsilon}\frac{\partial}{\partial \eta} + \hat{s}\frac{\partial}{\partial s} - \epsilon\kappa \eta \hat{s}\frac{\partial}{\partial s}
\end{equation}
\begin{equation}\label{asy8}
         \nabla^2  =  \frac{1}{\epsilon^2}\frac{\partial^2}{\partial \eta} + \frac{\kappa}{\epsilon}\frac{\partial}{\partial \eta} - \kappa^2\eta\frac{\partial}{\partial \eta} + \frac{\partial^2}{\partial s^2}
\end{equation}
\begin{equation}\label{asy9}
   \nabla \cdot (q\nabla)  =  \frac{1}{\epsilon^2}\frac{\partial}{\partial \eta}(q\frac{\partial}{\partial \eta}) + \frac{1}{\epsilon}\kappa q\frac{\partial}{\partial\eta} - \kappa^2 q \eta\frac{\partial}{\partial \eta} + \frac{\partial}{\partial s}(q\frac{\partial}{\partial s}) 
\end{equation}
\begin{equation}\label{asy10}
    \nabla \cdot(q\nabla_n) = \frac{1}{\epsilon^2}\frac{\partial}{\partial \eta}(q\frac{\partial}{\partial \eta}) + \frac{1}{\epsilon}\kappa q\frac{\partial}{\partial\eta} - \kappa^2 q \eta\frac{\partial}{\partial \eta}
\end{equation}
\begin{equation}\label{asy11}
    \nabla \cdot(q\nabla_t) = \frac{\partial}{\partial s}(q\frac{\partial}{\partial s})
\end{equation}
\begin{equation}\label{asy12}
    \nabla \cdot \vec{f} = \frac{1}{\epsilon}\frac{\partial}{\partial \eta}(\hat{n}\cdot \vec{f}) + \frac{\partial}{\partial s}(\hat{s}\cdot \vec{f}) + \kappa \hat{n}\cdot \vec{f}
\end{equation}
\begin{equation}\label{asy13}
    \nabla \cdot (q\hat{n}) = \frac{1}{\epsilon}\frac{\partial q}{\partial \eta}.
\end{equation}
Using the above transformations, the governing equations in the inner region can be expressed as
\begin{eqnarray}\label{asy14}
 \frac{\partial^2\phi}{\partial\eta^2}- f^{\prime}_{\mathrm{dw}}(\phi) + \epsilon\Bigg[(\alpha V_n + \kappa)\frac{\partial\phi}{\partial\eta} - a_1\{f_s(c_s)-f_l(c_l)-\mu(c_s-c_l)\}h^{\prime}(\phi) \nonumber \\
 - a_1\frac{(\tilde{c}_s-\tilde{c}_l)}{|\frac{\partial\phi}{\partial\eta}|}a(\phi)\frac{\partial\mu}{\partial\eta}\Bigg] +
 \epsilon^2\Bigg[\frac{\partial^2\phi}{\partial s^2} - \kappa^2\eta\frac{\partial\phi}{\partial\eta} + a_1\frac{(\tilde{c}_s-\tilde{c}_l)^2}{|\frac{\partial\phi}{\partial\eta}|^2}\alpha_1(\phi)\frac{\partial\phi}{\partial\eta}V_n\Bigg] = \mathcal{O}(\epsilon^3), \nonumber \\
\end{eqnarray}
\begin{eqnarray}
 \frac{1}{\epsilon^2}\frac{\partial}{\partial\eta}\Big[\frac{\partial c_l}{\partial\mu}q_n(\phi)\frac{\partial\mu}{\partial\eta}\Big] + \frac{1}{\epsilon}\Bigg[V_n\frac{\partial c}{\partial\eta} + \frac{\partial}{\partial\eta}\Bigg\{ \frac{(\tilde{c}_s-\tilde{c}_l)}{|\frac{\partial\phi}{\partial\eta}|}a(\phi)V_n\frac{\partial\phi}{\partial\eta}\Bigg\}\Bigg] \nonumber \\
 + \epsilon^0\Bigg[\frac{\partial}{\partial s}\Bigg\{q_t(\phi)\frac{\partial\phi}{\partial s}\Bigg\} + \kappa V_n \frac{(\tilde{c}_s-\tilde{c}_l)}{|\frac{\partial\phi}{\partial\eta}|}a(\phi)\frac{\partial\phi}{\partial\eta} - \kappa^2\eta q_n(\phi)\frac{\partial\mu}{\partial\eta}\Bigg] = \mathcal{O}(\epsilon)
\end{eqnarray}
\subsection{$\mathcal{O}(\epsilon^0)$ in $\phi$ : Phase-field profile}
At the leading order $\mathcal{O}(\epsilon^0)$ in $\phi$ we have,
\begin{equation}\label{sl1}
    \frac{\partial^2{\phi_0}}{\partial \eta^2} - f^{\prime}_{\mathrm{dw}}({\phi_0}) = 0
\end{equation}
 Multiplying both sides by $\frac{\partial{\phi_0}}{\partial \eta}$ and integrating we obtain,
 \begin{equation}\label{sl2}
    \Big(\frac{\partial{\phi_0}}{\partial\eta}\Big)^2 = 2 f_{\mathrm{dw}}({\phi_0}) + C_1
 \end{equation}
 Using the following matching conditions
 \begin{equation}\label{sl3}
    \lim_{\eta \to \pm\infty} {\phi_0} = \phi_0|^{\pm} = 0,1 
 \end{equation}
 and 
 \begin{equation}\label{sl4}
  \lim_{\eta \to \pm\infty} \frac{\partial{\phi_0}}{\partial \eta} = 0   
 \end{equation}
we evaluate the integration constant $C_1 = 0$. Therefore, we have
\begin{equation}\label{sl5}
 \frac{\partial {\phi_0}}{\partial\eta} = \sqrt{2f_{\mathrm{dw}}({\phi_0})}
\end{equation}
Integrating the above Eq.~ then gives the leading order solution to be
\begin{equation}\label{sl6}
     {\phi_0}(\eta) = \frac{1}{2}\Bigg[1-\tanh\left(\frac{\eta}{\sqrt{2}}\right)\Bigg] .
\end{equation}

\subsection{$\mathcal{O}(1/\epsilon^2)$ in $c$ : Equality of diffusion potential}
At the leading order, equating $\mathcal{O}(1/\epsilon^2)$ in $c$ equation 
\begin{equation}\label{eqdp1}
   \frac{\partial}{\partial \eta}\Big[\frac{\partial c_{l,0}}{\partial \mu}q_n(\phi_0)\frac{\partial {\mu_0}}{\partial \eta} \Big] = 0 .
\end{equation}
Integrating once,
\begin{equation}\label{eqdp2}
    \frac{\partial c_{l,0}}{\partial \mu}q_n({\phi_0})\frac{\partial {\mu_0}}{\partial \eta} = A_1(s)
\end{equation}
Taking the limit $\eta \rightarrow \pm\infty$ and employing the matching conditions $\lim_{\eta \to \pm\infty} = \frac{\partial{\mu_0}}{\partial \eta} = 0$, we obtain $A_1(s) = 0$. Since $q_n(\phi_0)$ and $\frac{\partial c_{l,0}}{\partial\mu}$ are finite (non-zero) in the inner region, Eq.~(\ref{eqdp2}) reduces to
\begin{equation}\label{eqdp3}
   \frac{\partial {\mu_0}}{\partial \eta} = 0 .
\end{equation}
Integrating again we have
\begin{equation}\label{eqdp4}
    {\mu_0} = \overline{\mu}_0(s)
\end{equation}
which is a constant and independent of the normal co-ordinate $\eta$. From matching conditions, we have $\lim_{\eta \to \pm\infty} {\mu_0} = \mu_0\Big|^{\pm} = \overline{\mu}_0$. This implies,
\begin{equation}\label{eqdp5}
 \frac{\partial f_l(c_{l,0})}{\partial c_l} = \frac{\partial f_s(c_{s,0})}{\partial c_s} = \overline{\mu}_0
\end{equation}
i.e. the slopes of the tangents to the free energy curves of solid and liquid at the solid and liquid compositions are equal. In addition Eq.~(\ref{eqdp5}) also implies that lowest order concentration fields $c_{s,0}$ and $c_{l,0}$ are independent of $\eta$.
The value of the constant $\overline{\mu}_0(s)$ is fixed by considering the next-to-leading order $\phi$ equation. 

\subsection{$\mathcal{O}(\epsilon)$ in $\phi$ : Gibbs Thomson equation}
At $\mathcal{O}(\epsilon)$ we obtain,
\begin{equation}\label{gt1}
    -\alpha V_n\frac{\partial {\phi_0}}{\partial \eta} = \frac{\partial^2{\phi_1}}{\partial \eta^2} - f^{\prime\prime}_{\mathrm{dw}}({\phi_0}){\phi_1} + \kappa\frac{\partial {\phi_0}}{\partial \eta} - a_1[f_s(c_{s,0}) - f_l(c_{l,0}) - \mu_0(c_{s,0} - c_{l,0}) ]h^{\prime}(\phi_0)
\end{equation}
The above Eq.~is multiplied by $\frac{\partial{\phi_0}}{\partial \eta}$ on both sides, rearranged and integrated within the limits $\pm \infty$ to obtain
\begin{eqnarray}\label{gt2}
 -(\alpha V_n + \kappa)\int_{-\infty}^{+\infty}\Big( \frac{\partial{\phi_0}}{\partial \eta}\Big)^2 \mathrm{d}\eta = \int_{-\infty}^{+\infty}\frac{\partial^2{\phi_1}}{\partial \eta^2}\frac{\partial{\phi_0}}{\partial \eta}\mathrm{d}\eta - \int_{-\infty}^{+\infty}f^{\prime\prime}_{\mathrm{dw}}({\phi_0}){\phi_1}\frac{\partial {\phi_0}}{\partial \eta}\mathrm{d}\eta \nonumber \\
 - a_1[f_s(c_{s,0}) - f_l(c_{l,0}) - \mu_0(c_{s,0} - c_{l,0})] \int_{-\infty}^{+\infty} \frac{\partial h{\phi_0})}{\partial \phi}\frac{\partial {\phi_0}}{\partial \eta}\mathrm{d}\eta
\end{eqnarray}
Let $I_1 = \int_{-\infty}^{+\infty}\frac{\partial^2{\phi_1}}{\partial \eta^2}\frac{\partial {\phi_0}}{\partial \eta}\mathrm{d}\eta$ which can be simplified by integration by parts (with the second term as first function) as below,
\begin{equation}\label{gt3}
 I_1 = \int_{-\infty}^{+\infty}\frac{\partial^2{\phi_1}}{\partial \eta^2}\frac{\partial {\phi_0}}{\partial \eta}\mathrm{d}\eta = \frac{\partial {\phi_0}}{\partial \eta}\frac{\partial {\phi_1}}{\partial \eta} \Bigg|_{-\infty}^{+\infty} - \int_{-\infty}^{+\infty}\frac{\partial^2{\phi_0}}{\partial\eta^2}\frac{\partial {\phi_1}}{\partial \eta}\mathrm{d}\eta
\end{equation}
From the derivative matching condition, the first term goes to zero. The second integral can be further simplified by integration by parts (with the first term as the first function) as follows,
\begin{equation}\label{gt4}
I_1 = - \int_{-\infty}^{+\infty}\frac{\partial^2{\phi_0}}{\partial\eta^2}\frac{\partial {\phi_1}}{\partial \eta}\mathrm{d}\eta = -\Bigg[ \frac{\partial^2{\phi_0}}{\partial\eta^2}{\phi_1}\Bigg|_{-\infty}^{+\infty} - \int_{-\infty}^{+\infty}\frac{\partial^3{\phi_0}}{\partial\eta^3}{\phi_1}\mathrm{d}\eta\Bigg]
\end{equation}
Using Eq.~(\ref{sl1}) the first term can be re-written as,
\begin{equation}\label{gt5}
 I_1 = -\Bigg[ f^{\prime}_{\mathrm{dw}}({\phi_0}){\phi_1}\Bigg|_{-\infty}^{+\infty} - \int_{-\infty}^{+\infty}\frac{\partial^3{\phi_0}}{\partial\eta^3}{\phi_1}\mathrm{d}\eta\Bigg] = \int_{-\infty}^{+\infty}\frac{\partial^3{\phi_0}}{\partial\eta^3}{\phi_1}\mathrm{d}\eta
\end{equation}
We next simplify the second integral in Eq.~(\ref{gt2}) as follows,
\begin{equation}\label{gt6}
I_2 = \int_{-\infty}^{+\infty}f^{\prime\prime}_{\mathrm{dw}}({\phi_0}){\phi_1}\frac{\partial {\phi_0}}{\partial \eta}\mathrm{d}\eta = \int_{-\infty}^{+\infty}\frac{\partial}{\partial \eta}f^{\prime}_{\mathrm{dw}}({\phi_0}){\phi_1}\mathrm{d}\eta
\end{equation}
Therefore, the first two integrals in Eq.~(\ref{gt2}) simplifies as,
\begin{equation}\label{gt7}
 I_1 - I_2 = \int_{-\infty}^{+\infty} \Bigg[\frac{\partial^3{\phi_0}}{\partial\eta^3}{\phi_1} - \frac{\partial}{\partial \eta}f^{\prime}_{\mathrm{dw}}({\phi_0}){\phi_1}\Bigg] \mathrm{d}\eta = \int_{-\infty}^{+\infty} \frac{\partial}{\partial \eta}\Bigg[\frac{\partial^2{\phi_0}}{\partial\eta^2} - f^{\prime}_{\mathrm{dw}}({\phi_0}) \Bigg]{\phi_1}\mathrm{d}\eta = 0
\end{equation}
where we have utilized Eq.~(\ref{sl1}). Therefore, Eq.~(\ref{gt2}) reduces to
\begin{equation}\label{gt8}
  -(\alpha V_n + \kappa)\int_{-\infty}^{+\infty}\Big( \frac{\partial {\phi_0}}{\partial \eta}\Big)^2 \mathrm{d}\eta = - a_1[f_s(c_{s,0}) - f_l(c_{l,0}) - \mu_0(c_{s,0} - c_{l,0})] h({\phi_0})\Big|_{-\infty}^{+\infty}
\end{equation}
and finally to a form
\begin{equation}\label{gt9}
   -(\alpha V_n + \kappa)\int_{-\infty}^{+\infty}\Big( \frac{\partial {\phi_0}}{\partial \eta}\Big)^2 \mathrm{d}\eta =  a_1[f_s(c_{s,0}) - f_l(c_{l,0}) - \mu_0(c_{s,0} - c_{l,0})].
\end{equation}
which on re-arrangement yields
\begin{equation}\label{gt10}
    \mu_0 = \overline{\mu}_0(s) = \frac{f_s(c_{s,0})-f_l(c_{l,0})}{c_{s,0}-c_{l,0}} + \frac{\alpha V_n + \kappa}{c_{s,0}-c_{l,0}}.
\end{equation}
Furthermore, re-arranging the above Eq.~ we obtain,
\begin{equation}\label{gt11}
 \omega_{s,0} - \omega_{l,0} = -(\alpha V_n + \kappa),
\end{equation}
where $\omega_{s,0} = f_s(c_{s,0}) - \mu_0 c_{s,0}$ and $\omega_{l,0} = f_l(c_{l,0}) - \mu_0 c_{l,0}$ are the grand-potentials of the solid and the liquid respectively. Eqs.~(\ref{eqdp5}) and (\ref{gt11}) implies a parallel tangent construction whereby the slopes of the tangent to the free energy at solid and liquid compositions are equal but are displaced by an amount proportional to the velocity and curvature of the interface.


\subsection{$\mathcal{O}(1/\epsilon)$ in $c$ : Solute trapping}
The $\mathcal{O}(1/\epsilon)$ in $c$ equation can be equated to be
\begin{equation}\label{st1}
    -V_n\frac{\partial c_0}{\partial\eta} = \frac{\partial}{\partial\eta}\Big[\frac{\partial c_{l,0}}{\partial\mu} q_n(\phi_0)\frac{\partial\mu_1}{\partial\eta}\Big] - \frac{\partial}{\partial\eta}\Big[(\tilde{c}_{s,0}-\tilde{c}_{l,0})a(\phi_0)V_n\Big]
\end{equation}
Integrating once we have
\begin{equation}\label{st2}
    -V_n c_0 = \frac{\partial c_{l,0}}{\partial\mu} q_n(\phi_0)\frac{\partial\mu_1}{\partial\eta} - (\tilde{c}_{s,0}-\tilde{c}_{l,0})a(\phi_0)V_n + A_2(s).
\end{equation}
To evaluate the integration constant $A_2(s)$ we take the limit $\eta \rightarrow -\infty$ to obtain
\begin{equation}\label{st3}
    A_2(s) = -V_n c_{s,0} - \frac{\partial c_{l,0}}{\partial\mu} q_n^-\frac{\partial\mu_0}{\partial r}\Big|^-,
\end{equation}
where we have employed the fact that $\lim_{\eta \rightarrow -\infty} a(\phi_0) = 0$. Substituting Eq.~(\ref{st3}) into Eq.~(\ref{st2}) and rearranging we obtain
\begin{equation}\label{st4}
    -V_n(c_0 - c_{s,0})= \frac{\partial c_{l,0}}{\partial\mu}q_n(\phi_0)\frac{\partial\mu_1}{\partial\eta} - (\tilde{c}_{s,0}-\tilde{c}_{l,0})a(\phi_0)V_n - \frac{\partial c_{l,0}}{\partial\mu}q_n^-\frac{\partial\mu_0}{\partial r}\Big|^-.
\end{equation}
The above expression can be re-expressed in terms of first order diffusion potential gradient as
\begin{equation}\label{st5}
    \frac{\partial\mu_1}{\partial\eta} = -\frac{V_n}{\partial c_{l,0}/\partial\mu}\frac{(c_0 - c_{s,0})}{q_n(\phi_0)} + (\tilde{c}_{s,0}-\tilde{c}_{l,0})\frac{V_n}{\partial c_{l,0}/\partial\mu}\frac{a(\phi_0)}{q_n(\phi_0)} + q_n^-\frac{\partial\mu_0}{\partial r}\Big|^- \frac{1}{q_n(\phi_0)}.
\end{equation}
Integrating both sides,
\begin{align}\label{st6}
    \mu_1 = -\frac{V_n}{\partial c_{l,0}/\partial\mu}\int_{0}^{\eta}\frac{(c_0 - c_{s,0})}{q_n(\phi_0)}\mathrm{d}\eta &+ \frac{V_n}{\partial c_{l,0}/\partial\mu}\int_{0}^{\eta}(\tilde{c}_{s,0}-\tilde{c}_{l,0})\frac{a(\phi_0)}{q_n(\phi_0)}\mathrm{d}\eta \nonumber \\
    &+ q_n^-\frac{\partial\mu_0}{\partial r}\Big|^- \int_{0}^{\eta}\frac{1}{q_n(\phi_0)}\mathrm{d}\eta + \overline{\mu}_1(s),
\end{align}
where $\overline{\mu}_1(s)$ is the constant of integration. We split the Eq.~(\ref{st6}) into two parts \\
(1) For $\eta > 0$
\begin{align}\label{st7}
    \mu_1 = & - \frac{V_n}{\partial c_{l,0}/\partial\mu}\int_{0}^{\eta}\Bigg[\frac{(c_0 - c_{s,0})}{q_n(\phi_0)} - \frac{(c_{l,0} - c_{s,0})}{q_n^+}\Bigg]\mathrm{d}\eta - \eta \frac{V_n}{\partial c_{l,0}/\partial\mu} \frac{(c_{l,0} - c_{s,0})}{q_n^+} \nonumber \\
   & + \frac{V_n}{\partial c_{l,0}/\partial\mu}\int_{0}^{\eta}(\tilde{c}_{s,0}-\tilde{c}_{l,0})\frac{a(\phi_0)}{q_n(\phi_0)}\mathrm{d}\eta 
    + q_n^-\frac{\partial\mu_0}{\partial r}\Big|^- \int_{0}^{\eta}\Bigg[\frac{1}{q_n(\phi_0)}-\frac{1}{q_n^+}\Bigg]\mathrm{d}\eta \nonumber \\
    & +\eta q_n^-\frac{\partial\mu_0}{\partial r}\Big|^-\frac{1}{q_n^+} + \overline{\mu}_1(s) 
\end{align}
and
(2) For $\eta < 0$
\begin{align}\label{st8}
 \mu_1 = & -\frac{V_n}{\partial c_{l,0}/\partial\mu}\int_{0}^{\eta}\frac{(c_0 - c_{s,0})}{q_n(\phi_0)}\mathrm{d}\eta + \frac{V_n}{\partial c_{l,0}/\partial\mu}\int_{0}^{\eta}(\tilde{c}_{s,0}-\tilde{c}_{l,0})\frac{a(\phi_0)}{q_n(\phi_0)}\mathrm{d}\eta \nonumber \\
 & + q_n^-\frac{\partial\mu_0}{\partial r}\Big|^- \int_{0}^{\eta}\Bigg[\frac{1}{q_n(\phi_0)}-\frac{1}{q_n^-}\Bigg]\mathrm{d}\eta 
 + \eta q_n^-\frac{\partial\mu_0}{\partial r}\Big|^- \frac{1}{q_n^+} + \overline{\mu}_1(s).
\end{align}
Taking far field limit $\eta \rightarrow +\infty$ in Eq.~(\ref{st7}) and employing the matching condition we have
\begin{align}\label{st9}
 \mu_1|^+ + \eta\frac{\partial\mu_0}{\partial r}\Bigg|^+ = & -\frac{V_n}{\partial c_{l,0}/\partial\mu} \int_{0}^{+\infty}\Bigg[\frac{(c_0 - c_{s,0})}{q_n(\phi_0)} - \frac{(c_{l,0} - c_{s,0}}{q_n^+}\Bigg]\mathrm{d}\eta - \eta \frac{V_n}{\partial c_{l,0}/\partial\mu} \frac{(c_{l,0} - c_{s,0})}{q_n^+}  \nonumber \\
 & + \frac{V_n}{\partial c_{l,0}/\partial\mu}\int_{0}^{+\infty}(\tilde{c}_{s,0}-\tilde{c}_{l,0})\frac{a(\phi_0)}{q_n(\phi_0)}\mathrm{d}\eta 
    + q_n^-\frac{\partial\mu_0}{\partial r}\Big|^- \int_{0}^{+\infty}\Bigg[\frac{1}{q_n(\phi_0)}-\frac{1}{q_n^+}\Bigg]\mathrm{d}\eta \nonumber \\
    & +\eta q_n^-\frac{\partial\mu_0}{\partial r}\Big|^-\frac{1}{q_n^+} + \overline{\mu}_1(s) \nonumber \\
\end{align}
Comparing coefficients of $\eta$ on both sides
\begin{equation}\label{st10}
    \frac{\partial\mu_0}{\partial r}\Bigg|^+ = -\frac{V_n}{\partial c_{l,0}/\partial\mu}\frac{(c_{l,0}-c_{s,0})}{q_n^+} + q_n^-\frac{\partial\mu_0}{\partial r}\Big|^-\frac{1}{q_n^+},
\end{equation}
which on re-arranging gives the flux conservation or the Stefan's condition at the interface at the lowest order
\begin{equation}\label{st11}
     q_n^+\frac{\partial\mu_0}{\partial r}\Bigg|^+ - q_n^-\frac{\partial\mu_0}{\partial r}\Big|^- = -\frac{V_n}{\partial c_{l,0}/\partial\mu}(c_{l,0}-c_{s,0})  .
\end{equation}
Similarly comparing coefficients of $\eta^0$ in Eq.~(\ref{st9}) we have
\begin{align}\label{st12}
 \mu_1|^+ =& -\frac{V_n}{\partial c_{l,0}/\partial\mu} \int_{0}^{+\infty}\Bigg[\frac{(c_0 - c_{s,0})}{q_n(\phi_0)} - \frac{(c_{l,0} - c_{s,0}}{q_n^+}\Bigg]\mathrm{d}\eta + \frac{V_n}{\partial c_{l,0}/\partial\mu}\int_{0}^{+\infty}(\tilde{c}_{s,0}-\tilde{c}_{l,0})\frac{a(\phi_0)}{q_n(\phi_0)}\mathrm{d}\eta \nonumber \\
    &+ q_n^-\frac{\partial\mu_0}{\partial r}\Big|^- \int_{0}^{+\infty}\Bigg[\frac{1}{q_n(\phi_0)}-\frac{1}{q_n^+}\Bigg]\mathrm{d}\eta  + \overline{\mu}_1(s) \nonumber \\
\end{align}
Introducing the lowest order solute profile $c_0 = c_{s,0}h(\phi_0) + c_{l,0}\{1-h(\phi_0)\}$ and expanding $\tilde{c}_{s,0} - \tilde{c}_{l,0} = c_{s,0} - c_{l,0} + S_sh(\phi_0) + S_l\{1-h(\phi_0)\}$ in Eq.~(\ref{st12}) and rearranging we obtain
\begin{align}\label{st13}
 \mu_1|^+ =& -\frac{V_n}{\partial c_{l,0}/\partial\mu}(c_{s,0}-c_{l,0})\int_{0}^{+\infty}\{\overline{p}(\phi_0) - \overline{p}(\phi_0|^+)\}\mathrm{d}\eta \nonumber \\
 &+ \frac{V_n}{\partial c_{l,0}/\partial\mu}\int_{0}^{+\infty}[S_sh(\phi_0)+S_l\{1-h(\phi_0)\}]\frac{a(\phi_0)}{q_n(\phi_0)}\mathrm{d}\eta \nonumber \\
 &+ q_n^-\frac{\partial\mu_0}{\partial r}\Big|^- \int_{0}^{+\infty}\Bigg[\frac{1}{q_n(\phi_0)}-\frac{1}{q_n^+}\Bigg]\mathrm{d}\eta  + \overline{\mu}_1(s),
\end{align}
where,
\begin{equation}\label{st14}
    \overline{p}(\phi_0) = \frac{h(\phi_0)-1-a(\phi_0)}{q_n(\phi_0)}
\end{equation}
and we have utilized the relation $\overline{p}(\phi_0|^+) = 1/q_n^+= 1$. Taking the far field limit $\eta \rightarrow -\infty$ of Eq.~(\ref{st8}) and comparing the coefficients $\eta^0$ we obtain
\begin{align}\label{st15}
 \mu_1|^- = & -\frac{V_n}{\partial c_{l,0}/\partial\mu}(c_{s,0}-c_{l,0})\int_{0}^{-\infty}\{\overline{p}(\phi_0)-\overline{p}(\phi_0|^-)\}\mathrm{d}\eta \nonumber \\
 & + \frac{V_n}{\partial c_{l,0}/\partial\mu}\int_{0}^{-\infty}[S_sh(\phi_0)+S_l\{1-h(\phi_0)\}]\frac{a(\phi_0)}{q_n(\phi_0)}\mathrm{d}\eta \nonumber \\
 & + q_n^-\frac{\partial\mu_0}{\partial r}\Big|^- \int_{0}^{-\infty}\Bigg[\frac{1}{q_n(\phi_0)}-\frac{1}{q_n^-}\Bigg]\mathrm{d}\eta  + \overline{\mu}_1(s). 
\end{align}
Subtracting Eq.~(\ref{st13}) from Eq.~(\ref{st15}) we have
\begin{align}\label{st16}
 \mu_1|^- - \mu_1|^+ = & -\frac{V_n}{\partial c_{l,0}/\partial\mu}(c_{s,0}-c_{l,0})(F_1^- - F_1^+) + \frac{V_n}{\partial c_{l,0}/\partial\mu}(F_2^- - F_2^+) \nonumber \\
  &+ q_n^-\frac{\partial\mu_0}{\partial r}\Bigg|^- (G^- - G^+)
\end{align}
where ,
\begin{equation}\label{st17}
    F_1^{\pm} = \int_{0}^{\pm\infty}\{\overline{p}(\phi_0) - \overline{p}(\phi_0|^{\pm})\}\mathrm{d}\eta ,
\end{equation}
\begin{equation}\label{st18}
    F_2^{\pm} = \int_{0}^{\pm\infty}[S_sh(\phi_0)+S_l\{1-h(\phi_0)\}]\frac{a(\phi_0)}{q_n(\phi_0)}\mathrm{d}\eta,
\end{equation}
and
\begin{equation}\label{st19}
    G^{\pm} = \int_{0}^{\pm\infty}\Big[\frac{1}{q_n(\phi_0)}-\frac{1}{q_n^{\pm}}\Big]\mathrm{d}\eta .
\end{equation}
The third term in Eq.~(\ref{st16}) proportional to the solute $q_n^-\frac{\partial\mu_0}{\partial r}\Big|^-$ is the Kapitza jump and results in diffusion potential jump even for a stationary interface. Since there is little evidence that this term plays an important role in solute trapping, it can be eliminated by a judicious choice of diffusivity interpolation function $q_n(\phi_0)$ that will result in $G^+ = G^-$. The inverse interpolation given by
\begin{equation}\label{st20}
    \frac{1}{q_n(\phi_0)} = \frac{h(\phi_0)}{\chi\frac{D_s}{D_l}} + 1-h(\phi_0)
\end{equation}
eliminates the term. Thus the diffusion potential jump can be controlled by the second term in Eq.~(\ref{st16}) which depends upon the source terms. 

Perhaps the most simplest choice is given by the choice $S_s = S_l = -A(c_{l,0} - c_{s,0})$. Substituting back in Eq.~(\ref{st16}) we obtain
\begin{equation}\label{st23}
    \mu_1|^- - \mu_1|^+ = -\frac{V_n}{\partial c_{l,0}/\partial\mu}(c_{l,0}-c_{s,0})(F^- - F^+).
\end{equation}
where,
\begin{equation}
    F_2^{\pm} = \int_{0}^{\pm}\{\tilde{p}(\phi_0) - \tilde{p}(\phi_0|^{\pm})\}\mathrm{d}\eta
\end{equation}
with
\begin{equation}
    \tilde{p}(\phi_0) = \frac{h(\phi_0)-1-(1-A)a(\phi_0)}{q_n(\phi_0)}
\end{equation}
The overall jump in diffusion potential upto first order therefore writes as
\begin{equation}\label{st28}
    \mu|^- - \mu|^+ = (\mu_0 + \epsilon \mu_1|^-) - (\mu_0 + \epsilon \mu_1|^+) = \epsilon(\mu_1|^- - \mu_1|^+)
\end{equation}
The overall jump expression writes as
\begin{equation}\label{st30}
    \mu|^- - \mu|^+ = \epsilon \frac{V_n}{\partial c_{l,0}/\partial\mu}(c_{l,0}-c_{s,0})(F^- - F^+).
\end{equation}
Reverting to dimensional quantities, for instance Eq.~(\ref{st30})
\begin{equation}\label{st31}
    \mu|^- - \mu|^+ = \frac{WV}{D_l\partial c_l/\partial\mu}(c_{l,0}-c_{s,0})(F^- - F^+)
\end{equation}
it is easy to see that the diffusion potential is proportional to the product of interface width $W$, velocity $V$ and the trapping parameter $A$.

For the second set of interpolation functions discussed in the main article we have
\begin{equation}\label{alt_diff}
    \frac{1}{q_n(\phi_0)} = \frac{h_n(\phi_0)}{\chi\frac{D_s}{D_l}} + 1-h_n(\phi_0)
\end{equation}
which satisfies the constraint $G^+ = G^-$. The diffusion potential jump is given by Eq.~(\ref{st31}) with 
\begin{equation}
    \tilde{p}(\phi_0) = \frac{p(\phi) - 1 - (1-A)a(\phi_0)}{q_n(\phi_0)}
\end{equation}
where,
\begin{equation}
    p(\phi_0) = h(\phi_0) + a\left(\phi_0 - \frac{1}{2}\right)\phi_0^2(1-\phi_0)^2
\end{equation}
and 
\begin{equation}
    a(\phi_0) = \frac{\left(\chi\frac{D_s}{D_l}-b\right)p(\phi_0)\{1-p(\phi_0)\}}{\chi\frac{D_s}{D_l}\{1-p(\phi_0)\}+bp(\phi_0)}
\end{equation}


\subsection{$\mathcal{O}(\epsilon^2)$ in $\phi$ : Modified Gibbs-Thomson Equation}
The $\mathcal{O}(\epsilon^2)$ terms in $\phi$ equation writes as
\begin{align}\label{gb1}
 -\alpha V_n\frac{\partial\phi_1}{\partial \eta} -& a_1 {\alpha_1}(\phi_o)\frac{(\tilde{c}_{s,0}-\tilde{c}_{l,0})^2}{\frac{\partial c_{l,0}}{\partial\mu}}\frac{V_n}{(\frac{\partial\phi_0}{\partial\eta})^2}\frac{\partial\phi_0}{\partial\eta}  =  \frac{\partial^2\phi_2}{\partial\eta^2} + \kappa\frac{\partial\phi_1}{\partial\eta} - \eta\kappa^2\frac{\partial\phi_0}{\partial\eta} - f_{\mathrm{dw}}^{\prime\prime}(\phi_0)\phi_2 \nonumber \\
 &- f_{\mathrm{dw}}^{\prime\prime\prime}(\phi_0)\frac{\phi_1^2}{2} -  a_1[f_s(c_{s,0}) - f_l(c_{l,0}) - \mu_0(c_{s,0} - c_{l,0})]h^{\prime\prime}(\phi_0)\phi_1 \nonumber \\
 & + a_1\mu_1(c_{s,0}-c_{l,0})h^{\prime}(\phi_0) 
  -a_1(\tilde{c}_{s,0}-\tilde{c}_{l,0})a(\phi_0)\frac{\partial\mu_1}{\partial\eta}\frac{1}{|\frac{\partial\phi_0}{\partial\eta}|} 
\end{align}
Multiplying both sides by $\frac{\partial\phi_0}{\partial\eta}$, rearranging and integrating within the limits $-\infty$ to $+\infty$ we obtain
\begin{align}\label{gb2}
 -(\alpha V_n + \kappa)\int_{-\infty}^{+\infty}\frac{\partial\phi_1}{\partial\eta}\frac{\partial\phi_0}{\partial\eta}\mathrm{d}\eta & - a_1\frac{V_n}{\frac{\partial c_{l,0}}{\partial\mu}}\int_{-\infty}^{+\infty}(\tilde{c}_{s,0}-\tilde{c}_{l,0})^2{\alpha_1}(\phi_0)\mathrm{d}\eta =  
  \int_{-\infty}^{+\infty}\frac{\partial^2\phi_2}{\partial\eta}\frac{\partial\phi_0}{\partial\eta} \nonumber \\
 & -\kappa^2\int_{-\infty}^{+\infty}\eta\Big(\frac{\partial\phi_0}{\partial\eta}\Big)^2 \mathrm{d}\eta -\int_{-\infty}^{+\infty}f_{\mathrm{dw}}^{\prime\prime}(\phi_0)\phi_2\frac{\partial\phi_0}{\partial\eta}\mathrm{d}\eta \nonumber \\
  &- \int_{-\infty}^{+\infty}f_{\mathrm{dw}}^{\prime\prime\prime}(\phi_0)\frac{\phi_1^2}{2}\frac{\partial\phi_0}{\partial\eta}\mathrm{d}\eta -  a_1[f_s(c_{s,0}) - f_l(c_{l,0}) - \mu_0(c_{s,0} - c_{l,0})] \nonumber \\
 &\times \int_{-\infty}^{+\infty}h^{\prime\prime}(\phi_0)\phi_1\frac{\partial\phi_0}{\partial\eta}\mathrm{d}\eta + a_1(c_{s,0}-c_{l,0})\int_{-\infty}^{+\infty}\mu_1h^{\prime}(\phi_0) \frac{\partial\phi_0}{\partial\eta}\mathrm{d}\eta \nonumber \\
 & + a_1(\tilde{c}_{s,0}-\tilde{c}_{l,0})\int_{-\infty}^{+\infty}a(\phi_0)\frac{\partial\mu_1}{\partial\eta}\mathrm{d}\eta 
\end{align}

Let $I_1 = \int_{-\infty}^{+\infty}\frac{\partial ^2{\phi_2}}{\partial\eta^2}\frac{\partial {\phi_0}}{\partial \eta} \mathrm{d}\eta$ which can be simplified by integration by parts (with the second term as the first function) as below,
\begin{equation}\label{gb3}
 I_1 = \int_{-\infty}^{+\infty}\frac{\partial^2{\phi_2}}{\partial \eta^2}\frac{\partial {\phi_0}}{\partial \eta}\mathrm{d}\eta = \frac{\partial{\phi_0}}{\partial \eta}\frac{\partial {\phi_2}}{\partial \eta} \Bigg|_{-\infty}^{+\infty} - \int_{-\infty}^{+\infty}\frac{\partial^2{\phi_0}}{\partial\eta^2}\frac{\partial {\phi_2}}{\partial \eta}\mathrm{d}\eta
\end{equation}
From the derivative matching condition, the first term goes to zero. The second integral can be further simplified by integration by parts (with the first term as the first function) as follows,
\begin{equation}\label{gb4}
I_1 = - \int_{-\infty}^{+\infty}\frac{\partial^2{\phi_0}}{\partial\eta^2}\frac{\partial {\phi_2}}{\partial \eta}\mathrm{d}\eta = -\Bigg[ \frac{\partial^2{\phi_0}}{\partial\eta^2}{\phi_2}\Bigg|_{-\infty}^{+\infty} - \int_{-\infty}^{+\infty}\frac{\partial^3{\phi_0}}{\partial\eta^3}{\phi_2}\mathrm{d}\eta\Bigg]
\end{equation}
Using Eq.~(\ref{sl1}) the first term can be re-written as,
\begin{equation}\label{gb5}
 I_1 = -\Bigg[ f^{\prime}_{\mathrm{dw}}({\phi_0}){\phi_2}\Bigg|_{-\infty}^{+\infty} - \int_{-\infty}^{+\infty}\frac{\partial^3{\phi_0}}{\partial\eta^3}{\phi_2}\mathrm{d}\eta\Bigg] = \int_{-\infty}^{+\infty}\frac{\partial^3{\phi_0}}{\partial\eta^3}{\phi_2}\mathrm{d}\eta .
\end{equation}
We next simplify the third integral on the RHS of Eq.~(\ref{gb2}) as follows,
\begin{equation}\label{gb6}
I_2 = \int_{-\infty}^{+\infty}f^{\prime\prime}_{\mathrm{dw}}({\phi_0}){\phi_2}\frac{\partial {\phi_0}}{\partial \eta}\mathrm{d}\eta = \int_{-\infty}^{+\infty}\frac{\partial}{\partial \eta}f^{\prime}_{\mathrm{dw}}({\phi_0}){\phi_2}\mathrm{d}\eta
\end{equation}
Therefore, the first and the third integrals on the right hand side in Eq.~(\ref{gb2}) simplifies as,
\begin{equation}\label{gb7}
 I_1 - I_2 = \int_{-\infty}^{+\infty} \Bigg[\frac{\partial^3{\phi_0}}{\partial\eta^3}{\phi_2} - \frac{\partial}{\partial \eta}f^{\prime}_{\mathrm{dw}}({\phi_0}){\phi_2}\Bigg] \mathrm{d}\eta = \int_{-\infty}^{+\infty} \frac{\partial}{\partial \eta}\Bigg[\frac{\partial^2{\phi_0}}{\partial\eta^2} - f^{\prime}_{\mathrm{dw}}({\phi_0}) \Bigg]{\phi_2} \mathrm{d}\eta = 0
\end{equation}
where we have utilized Eq.~(\ref{sl1}). To simplify the remaining terms of Eq.~(\ref{gb2}) we utilize the symmetry properties of the involved functions. For this purpose we first examine a re-arranged form of Eq.~(\ref{gt1})
\begin{equation}\label{gb8}
    L{\phi^1} = -(\alpha V_n + \kappa)\frac{\partial {\phi_0}}{\partial \eta} + a_1[f_s(c_{s,0}) - f_{l}(c_{l,0})-\mu_0(c_{s,0}-c_{l,0})]h^{\prime}({\phi_0})
\end{equation}
where the operator $L = \Big[\frac{\partial^2}{\partial \eta^2} - f^{\prime\prime}_{\mathrm{dw}}({\phi_0})\Big]$. From the leading order solution, ${\phi_0}$ is a sum of a constant and an odd function. Therefore it's derivative $\frac{\partial \hat{\phi^0}}{\partial \eta}$ is an even function. Similarly, for $h(\hat{\phi^0})$ we have
\begin{equation}\label{gb9}
    h(\hat{\phi^0}(\eta)) = 1 - h(\hat{\phi^0}(-\eta))
\end{equation}
Differentiating with respect to $\eta$,
\begin{equation}\label{gb10}
    \frac{\partial h({\phi_0}(\eta))}{\partial \eta} = -\frac{\partial h({\phi_0}(-\eta))}{\partial \eta}
\end{equation}
Employing chain-rule of differentiation,
\begin{equation}\label{gb11}
    \frac{\partial h({\phi_0}(\eta))}{\partial {\phi_0}}\frac{\partial {\phi_0}}{\partial \eta} = -\frac{\partial h({\phi_0}(-\eta))}{\partial \phi_0(-\eta)}\frac{\partial {\phi_0(-\eta)}}{\partial (-\eta)}\frac{\partial (-\eta)}{\partial \eta} .
\end{equation}
Using even property of $\frac{\partial\phi_0}{\partial\eta}$ we have
\begin{equation}\label{gb12}
   \frac{\partial h({\phi_0}(\eta))}{\partial {\phi_0}} =  \frac{\partial h({\phi_0}(-\eta))}{\partial \phi_0(-\eta)}
\end{equation}
which proves $h^{\prime}(\phi_0)$ is an even function. Since a constant multiplied by an even function yields an even function both terms on the RHS of Eq.~(\ref{gb8}) turn out to be even functions. Therefore, the LHS of Eq.~(\ref{gb8}) should also be even. This is only possible if ${\phi_1}$ is even, so that first derivative $\frac{\partial{\phi_1}}{\partial \eta}$ is odd and the second derivative $\frac{\partial ^2{\phi_1}}{\partial \eta^2}$ is even. As for the second term on the LHS, we next show that $f^{\prime\prime}_{\mathrm{dw}}$ is even. Assuming $f_{dw}({\phi_0}) = {{\phi_0}}^2{(1-{\phi_0})}^2$, $f^{\prime\prime}_{\mathrm{dw}}({\phi_0}) = 12{{\phi_0}}^2 - 12{\phi_0} + 2$. Therefore,
\begin{equation}\label{gb13}
    f^{\prime\prime}_{\mathrm{dw}}({\phi_0}(-\eta)) = 12{{\phi_0}(-\eta)}^2 - 12{\phi_0}(-\eta) + 2 . 
\end{equation}
Using anti-symmetric property of $\phi_0$ profile,
\begin{equation}\label{gb14}
   f^{\prime\prime}_{\mathrm{dw}}({\phi_0}(-\eta)) = 12{(1-{\phi_0}(\eta))}^2 - 12(1-{\phi_0}(\eta)) + 2 = 12{{\phi_0}(\eta)}^2 - 12{\phi_0}(\eta) + 2 = f^{\prime\prime}_{\mathrm{dw}}({\phi_0}(\eta))
\end{equation}
Therefore, $f^{\prime\prime}_{\mathrm{dw}}({\phi_0}){\phi_1}$ is also even.

Returning back to Eq.~(\ref{gb2}), we again examine the nature of the integrands in each term
\begin{equation}\label{gb15}
 -(\alpha V_n + \kappa) \int_{-\infty}^{+\infty}\frac{\partial {\phi_1}}{\partial \eta}\frac{\partial {\phi_0}}{\partial \eta} \mathrm{d}\eta = -(\alpha V_n + \kappa)\int_{-\infty}^{+\infty} (\mathrm{Odd} \times \mathrm{Even})  = -(\alpha V_n + \kappa)\int_{-\infty}^{+\infty} \mathrm{Odd} = 0
\end{equation}


For the fourth integral on the RHS in Eq.~(\ref{gb2}) we need to determine the property of $f^{\prime\prime\prime}_{\mathrm{dw}}({\phi_0})$. If we assume $f_{\mathrm{dw}}({\phi_0}) = {{\phi_0}}^2{(1-{\phi_0})}^2$, $f^{\prime\prime\prime}_{\mathrm{dw}}({\phi_0}) = 24{\phi_0} - 12$. Thus,
\begin{equation}\label{gb16}
 f^{\prime\prime\prime}_{\mathrm{dw}}({\phi_0}(-\eta)) = 24{\phi_0}(-\eta) - 12
\end{equation}
Using anti-symmetric property of $\phi_0$ profile,
\begin{equation}\label{gb17}
 f^{\prime\prime\prime}_{\mathrm{dw}}({\phi_0}(-\eta)) = 24(1 - {\phi_0}(\eta)) - 12 = 12 - 24{\phi_0}(\eta) = -f^{\prime\prime\prime}_{\mathrm{dw}}({\phi_0}(\eta))
\end{equation}
which proves $f^{\prime\prime\prime}_{\mathrm{dw}}({\phi_0})$ is odd. Therefore, the fourth integral on the RHS in Eq.~(\ref{gb2}) reads as
\begin{equation}\label{gb18}
  \int_{-\infty}^{+\infty} f^{\prime\prime\prime}_{\mathrm{dw}}({\phi_0})\frac{{\phi_1}^2}{2}\frac{\partial {\phi_0}}{\partial \eta} \mathrm{d}\eta = \int_{-\infty}^{+\infty} (\mathrm{Odd} \times \mathrm{Even} \times \mathrm{Even}) = \int_{-\infty}^{+\infty} \mathrm{Odd} = 0.
\end{equation}
 We next examine the fifth integral on the RHS in Eq.~(\ref{gb2}),
\begin{equation}\label{gb19}
   \int_{-\infty}^{+\infty}h^{\prime\prime}({\phi_0}){\phi_1} \frac{\partial{\phi_0}}{\partial \eta} \mathrm{d}\eta = \int_{-\infty}^{+\infty} (\mathrm{Odd} \times \mathrm{Even} \times \mathrm{Even}) = \int_{-\infty}^{+\infty} \mathrm{Odd} = 0
\end{equation}
where we have utilized the fact that since $h^{\prime}({\phi_0})$ is even, as proved in Eq.~(\ref{gb12}), $h^{\prime\prime}({\phi_0})$ is odd. Evaluating the second integral on the RHS in Eq.~(\ref{gb2}) ,
\begin{equation}\label{gb20}
   \int_{-\infty}^{+\infty}\kappa^2\eta\Bigg(\frac{\partial {\phi_0}}{\partial \eta}\Bigg)^2 \mathrm{d}\eta = \int_{-\infty}^{+\infty} (\mathrm{Odd} \times \mathrm{Even}) = \int_{-\infty}^{+\infty} \mathrm{Odd} = 0 .
\end{equation}
The remaining non-zero terms in Eq.~(\ref{gb2}) writes as
\begin{align}\label{gb21}
   \frac{V_n}{\partial c_{l,0}/\partial\mu} \int_{-\infty}^{+\infty}(\tilde{c}_{s,0}-\tilde{c}_{l,0})^2{\alpha_1}(\phi_0) = & (c_{s,0}-c_{l,0})\int_{-\infty}^{+\infty}\mu_1 h^{\prime}(\phi_0)\frac{\partial\phi_0}{\partial\eta} \mathrm{d}\eta \nonumber \\ 
   & +\int_{-\infty}^{+\infty}(\tilde{c}_{s,0}-\tilde{c}_{l,0})a(\phi_0)\frac{\partial\mu_1}{\partial\eta}\mathrm{d}\eta
\end{align}
The expressions for $\mu_1$ and $\frac{\partial\mu_1}{\partial\eta}$ incorporating the choice of source terms $S_s=S_l= -A(c_{s,0}-c_{l,0})$ can be written as
\begin{equation}\label{gb22}
    \mu_1 = -\frac{V_n}{\partial c_{l,0}/\partial\mu}(c_{s,0}-c_{l,0})\int_{0}^{\eta}\frac{h(\phi_0)-1-(1-A)a(\phi_0)}{q_n(\phi)}\mathrm{d}\eta + q_n^-\frac{\partial\mu_0}{\partial r}\Bigg|^-\int_{0}^{\eta}\frac{1}{q_n(\phi_0)}\mathrm{d}\eta + \overline{\mu_1}(s)
\end{equation}
and
\begin{equation}\label{gb23}
\frac{\partial\mu_1}{\partial \eta} = -\frac{V_n}{\partial c_{l,0}/\partial\mu}(c_{s,0}-c_{l,0})\frac{h(\phi_0)-1-(1-A)a(\phi)}{q_n(\phi_0)} +  q_n^-\frac{\partial\mu_0}{\partial r}\Bigg|^-\frac{1}{q_n(\phi_0)}   
\end{equation}
Introducing the notation 
\begin{equation}\label{gb24}
    \tilde{p}(\phi_0)= \frac{h(\phi_0)-1-(1-A)a(\phi_0)}{q_n(\phi)},
\end{equation}
and substituting Eqs.~(\ref{gb22}) and (\ref{gb23}) in Eq.~(\ref{gb21}) we obtain
\begin{align}\label{gb25}
 -(1-A)^2(c_{s,0}-c_{l,0})^2\frac{V_n}{\partial c_{l,0}/\partial\mu}&\int_{-\infty}^{+\infty}{\alpha_1}(\phi_0)\mathrm{d}\eta = \nonumber \\
 & -\frac{V_n}{\partial c_{l,0}/\partial\mu}(c_{s,0}-c_{l,0})^2\int_{-\infty}^{+\infty}\Big[\int_{0}^{\eta}\tilde{p}(\phi_0)\mathrm{d}\eta^{\prime}\Big]h^{\prime}(\phi_0)\frac{\partial\phi_0}{\partial\eta}\mathrm{d}\eta \nonumber \\
 & + (c_{s,0}-c_{l,0})q_n^-\frac{\partial\mu_0}{\partial r}\Big|^- \int_{-\infty}^{+\infty}\Big[\int_{0}^{\eta}\frac{1}{q_n(\phi_0)}\mathrm{d}\eta^{\prime}\Big]h^{\prime}(\phi_0)\frac{\partial\phi_0}{\partial\eta}\mathrm{d}\eta\nonumber \\
 & + (c_{s,0}-c_{l,0})\overline{\mu}_1(s)\int_{-\infty}^{+\infty} h^{\prime}(\phi_0)\frac{\partial\phi_0}{\partial\eta}\mathrm{d}\eta \nonumber \\
 &- (1-A)(c_{s,0}-c_{l,0})^2\frac{V_n}{\partial c_{l,0}/\partial\mu}\int_{-\infty}^{+\infty}a(\phi_0)\tilde{p}(\phi_0)\mathrm{d}\eta \nonumber \\
 & +(1-A)(c_{s,0}-c_{l,0})q_n^-\frac{\partial\mu_0}{\partial r}\Big|^- \int_{-\infty}^{+\infty} \frac{a(\phi_0)}{q_n(\phi_0)}\mathrm{d}\eta \nonumber \\
\end{align}
Solving for $\overline{\mu}_1(s)$ we obtain
\begin{eqnarray}\label{gb26}
 \overline{\mu}_1(s) = \frac{V_n}{\partial c_{l,0}/\partial\mu}(c_{s,0}-c_{l,0})\Big[(1-A)^2 N - M - (1-A)P\Big] + q_n^-\frac{\partial\mu_0}{\partial r}\Big|^-\Big[Q + (1-A)H\Big].\nonumber \\
\end{eqnarray}
where,
\begin{equation}\label{gb27}
    N = \int_{-\infty}^{+\infty}{\alpha_1}(\phi_0)\mathrm{d}\eta ,
\end{equation}
\begin{equation}\label{gb28}
    M = \int_{-\infty}^{+\infty}\Big[\int_{0}^{\eta}\tilde{p}(\phi_0)\mathrm{d}\eta^{\prime}\Big]h^{\prime}(\phi_0)\frac{\partial\phi_0}{\partial\eta}\mathrm{d}\eta ,
\end{equation}
\begin{equation}\label{gb29}
    P = \int_{-\infty}^{+\infty}a(\phi_0)\tilde{p}(\phi_0)\mathrm{d}\eta ,
\end{equation}
\begin{equation}\label{gb30}
    Q = \int_{-\infty}^{+\infty}\Big[\int_{0}^{\eta}\frac{1}{q_n(\phi_0)}\mathrm{d}\eta^{\prime}\Big]h^{\prime}(\phi_0)\frac{\partial\phi_0}{\partial\eta}\mathrm{d}\eta
\end{equation}
\begin{equation}\label{gb31}
    H = \int_{-\infty}^{+\infty} \frac{a(\phi_0)}{q_n(\phi_0)}\mathrm{d}\eta .
\end{equation}
Therefore, the first order correction to the diffusion potential writes as
\begin{equation}\label{gb32}
 \mu_1|^{\pm} =  -\frac{V_n}{\partial c_{l,0}/\partial\mu}(c_{s,0}-c_{l,0})\Big[F^{\pm} + M - (1-A)^2 N  + (1-A)P\Big] + q_n^-\frac{\partial\mu_0}{\partial r}\Big|^-\Big[G^{\pm} + Q + (1-A)H\Big]. 
\end{equation}
Since the choice of $q_n(\phi_0)$ ensures $G^+ = G^-$ and it can be further shown that $G+Q+H = 0$, Eq.~(\ref{gb32}) can be further simplified as
\begin{equation}\label{gb33}
   \mu_1|^{\pm} =  -\frac{V_n}{\partial c_{l,0}/\partial\mu}(c_{s,0}-c_{l,0})\Big[F^{\pm} + M - (1-A)^2 N  + (1-A)P\Big] - q_n^-\frac{\partial\mu_0}{\partial r}\Big|^- AH. 
\end{equation}
It is to be noted that $H$ can also be expressed as $H = F^+ - F^-$. The diffusion potential upto first order can then be written as
\begin{equation}\label{gb34}
    \mu|^{\pm} = \mu_0 + \epsilon\mu_1|^{\pm}
\end{equation}
Substituting Eqs.~(\ref{gt10}) and (\ref{gb33}) in the above equation we obtain
\begin{align}\label{gb35}
     \mu|^{\pm} = \frac{f_s(c_{s,0}) - f_l(c_{l,0})}{c_{s,0}-c_{l,0}} + \frac{\kappa}{c_{s,0}-c_{l,0}} & + \frac{V_n}{c_{s,0}-c_{l,0}}\Bigg\{ \alpha - \epsilon\frac{(c_{s,0}-c_{l,0})^2}{\partial c_{l,0}/\partial\mu}\Big[F^{\pm} + M - (1-A)^2 N \nonumber \\
     & + (1-A)P\Big]\Bigg\}- \epsilon q_n^-\frac{\partial\mu_0}{\partial r}\Big|^- AH 
\end{align}
For the second set of interpolation functions, the integrals $F$, $M$, $N$, $P$ and $H$ can be redefined.

\subsection{$\mathcal{O}$($\epsilon^0$) in $c$ : Modified Mass Conservation Equation}
The $\mathcal{O}(\epsilon^0)$ terms in $c$ equation writes as
\begin{align}\label{mc1}
 -V_n\frac{\partial c_1}{\partial \eta}  = & \frac{\partial}{\partial \eta}\Big[\frac{\partial c_{l,0}}{\partial \mu} q_n(\phi_0)\frac{\partial \mu_2}{\partial \eta}\Big] + \frac{\partial}{\partial\eta}\Big[\frac{\partial c_{l,0}}{\partial \mu}q_n^{\prime}(\phi_0)\phi_1\frac{\partial\mu_1}{\partial\eta}\Big] + \kappa\frac{\partial c_{l,0}}{\partial \mu} q_n(\phi_0)\frac{\partial \mu_1}{\partial\eta}  \nonumber \\
 & + \kappa\frac{\partial c_{l,0}}{\partial \mu} q_n^{\prime}(\phi_0)\phi_1\frac{\partial\mu_0}{\partial\eta}c- \kappa^2\eta \frac{\partial c_{l,0}}{\partial \mu}q_n(\phi_0)\frac{\partial\mu_0}{\partial\eta} 
 + \frac{\partial}{\partial s}\Big[\frac{\partial c_{l,0}}{\partial \mu}q_t(\phi_0)\frac{\partial\mu_0}{\partial s}\Big]  \nonumber \\
 & + \frac{\partial}{\partial\eta}\Big[a^{\prime}(\phi_0)\phi_1 (\tilde{c}_{s,0} - \tilde{c}_{l,0})V_n\Big] + \frac{\partial}{\partial\eta}\Big[a(\phi_0)(\tilde{c}_{s,0}-\tilde{c}_{l,0})\frac{V_n}{\frac{\partial\phi_0}{\partial\eta}}\frac{\partial\phi_1}{\partial\eta}\Big] \nonumber \\
 & +\frac{\partial}{\partial\eta}\Big[a(\phi_0)(\tilde{c}_{s,1}-\tilde{c}_{l,1})V_n\Big]  
\end{align}
Using the leading order solution $\frac{\partial\mu_0}{\partial\eta} = 0$ and integrating the above equation we obtain
\begin{align}\label{mc2}
 -V_n c_1 = & \frac{\partial c_{l,0}}{\partial \mu}q_n(\phi_0)\frac{\partial \mu_2}{\partial \eta} + \frac{\partial c_{l,0}}{\partial \mu}q_n^{\prime}(\phi_0)\phi_1\frac{\partial\mu_1}{\partial\eta} + \frac{\partial c_{l,0}}{\partial \mu}\kappa \int_{0}^{\eta} q_n(\phi_0)\frac{\partial \mu_1}{\partial\eta}\mathrm{d}\eta  \nonumber \\
 & + \frac{\partial c_{l,0}}{\partial \mu}\frac{\partial ^2 \mu_0}{\partial s^2}\int_{0}^{\eta}q_t(\phi_0)\mathrm{d}\eta + a^{\prime}(\phi_0)\phi_1 (\tilde{c}_{s,0} - \tilde{c}_{l,0})V_n + a(\phi_0)(\tilde{c}_{s,0}-\tilde{c}_{l,0})\frac{V_n}{\frac{\partial\phi_0}{\partial\eta}}\frac{\partial\phi_1}{\partial\eta}  \nonumber \\
 & + a(\phi_0)(\tilde{c}_{s,1}-\tilde{c}_{l,1})V_n + \kappa V_n \int_{0}^{\eta} a(\phi_0)(\tilde{c}_{s,0}-\tilde{c}_{l,0}) + A_3(s)
\end{align}
Substituting the value of $\frac{\partial \mu_1}{\partial\eta}$ from the next-to-leading order solution in the third term of the right hand side of the above expression
\begin{align}\label{mc3}
 -V_n c_1 = & \frac{\partial c_{l,0}}{\partial \mu}q_n(\phi_0)\frac{\partial \mu_2}{\partial \eta} + \frac{\partial c_{l,0}}{\partial \mu}q_n^{\prime}(\phi_0)\phi_1\frac{\partial\mu_1}{\partial\eta} - V_n \kappa\int_{0}^{\eta} (c_o - c_{s,0})\mathrm{d}\eta  \nonumber \\
 & - V_n \kappa \int _{0}^{\eta} (\tilde{c}_{s,0} - \tilde{c}_{l,0})a(\phi_0)\mathrm{d}\eta + \eta \kappa \frac{\partial c_{l,0}}{\partial \mu}q_n^- \frac{\partial \mu_0}{\partial r}\Big|^- + \frac{\partial c_{l,0}}{\partial \mu}\frac{\partial ^2 \mu_0}{\partial s^2}\int_{0}^{\eta}q_t(\phi_0)\mathrm{d}\eta   \nonumber \\
 & + a^{\prime}(\phi_0)\phi_1 (\tilde{c}_{s,0} - \tilde{c}_{l,0})V_n + a(\phi_0)(\tilde{c}_{s,0}-\tilde{c}_{l,0})\frac{V_n}{\frac{\partial\phi_0}{\partial\eta}}\frac{\partial\phi_1}{\partial\eta} \nonumber \\
  & + a(\phi_0)(\tilde{c}_{s,1}-\tilde{c}_{l,1})V_n + \kappa V_n \int_{0}^{\eta} a(\phi_0)(\tilde{c}_{s,0}-\tilde{c}_{l,0}) + A_3(s)
\end{align}
Cancelling the fourth and the tenth term on the right side of the above expression we have
\begin{align}\label{mc4}
 -V_n c_1 = & \frac{\partial c_{l,0}}{\partial \mu}q_n(\phi_0)\frac{\partial \mu_2}{\partial \eta} + \frac{\partial c_{l,0}}{\partial \mu}q_n^{\prime}(\phi_0)\phi_1\frac{\partial\mu_1}{\partial\eta} - V_n \kappa\int_{0}^{\eta} (c_o - c_{s,0})\mathrm{d}\eta + \eta \kappa \frac{\partial c_{l,0}}{\partial \mu}q_n^- \frac{\partial \mu_0}{\partial r}\Big|^- \nonumber \\
 & + \frac{\partial c_{l,0}}{\partial \mu}\frac{\partial ^2 \mu_0}{\partial s^2}\int_{0}^{\eta}q_t(\phi_0)\mathrm{d}\eta 
 + a^{\prime}(\phi_0)\phi_1 (\tilde{c}_{s,0} - \tilde{c}_{l,0})V_n + a(\phi_0)(\tilde{c}_{s,0}-\tilde{c}_{l,0})\frac{V_n}{\frac{\partial\phi_0}{\partial\eta}}\frac{\partial\phi_1}{\partial\eta} \nonumber \\
 & + a(\phi_0)(\tilde{c}_{s,1}-\tilde{c}_{l,1})V_n + A_3(s) 
\end{align}
Splitting the above equation into two parts one for $\eta > 0$ and the other $\eta < 0$.
\\
(1) For $\eta > 0$
\begin{align}\label{mc5}
 -V_n c_1 = & \frac{\partial c_{l,0}}{\partial \mu}q_n(\phi_0)\frac{\partial \mu_2}{\partial \eta} + \frac{\partial c_{l,0}}{\partial \mu}q_n^{\prime}(\phi_0)\phi_1\frac{\partial\mu_1}{\partial\eta} - V_n \kappa \int_{0}^{\eta}\Big[ (c_0 - c_{s,0}) - (c_{l,0}-c_{s,0})\Big]\mathrm{d}\eta  \nonumber \\
  & - \eta V_n \kappa (c_{l,0}-c_{s,0}) + \eta \kappa \frac{\partial c_{l,0}}{\partial \mu}q_n^- \frac{\partial \mu_0}{\partial r}\Big|^- + \frac{\partial^2\mu_0}{\partial s^2}\int_{0}^{\eta}\Big[q_t(\phi_0) - q_t(\phi_o|^+)\Big]\mathrm{d}\eta \nonumber \\
 & + \eta \frac{\partial c_{l,0}}{\partial \mu}\frac{\partial^2\mu_0}{\partial s^2}q_t(\phi_o|^+) + a^{\prime}(\phi_0)\phi_1 (\tilde{c}_{s,0} - \tilde{c}_{l,0})V_n + a(\phi_0)(\tilde{c}_{s,0}-\tilde{c}_{l,0})\frac{V_n}{\frac{\partial\phi_0}{\partial\eta}}\frac{\partial\phi_1}{\partial\eta} \nonumber \\
 & + a(\phi_0)(\tilde{c}_{s,1}-\tilde{c}_{l,1})V_n + A_3(s)
\end{align}
(2) For $\eta < 0$
\begin{align}\label{mc6}
 -V_n c_1 = & \frac{\partial c_{l,0}}{\partial \mu}q_n(\phi_0)\frac{\partial \mu_2}{\partial \eta} + \frac{\partial c_{l,0}}{\partial \mu}q_n^{\prime}(\phi_0)\phi_1\frac{\partial\mu_1}{\partial\eta} - V_n \kappa \int_{0}^{\eta} (c_0 - c_{s,0})\mathrm{d}\eta + \eta \kappa \frac{\partial c_{l,0}}{\partial \mu}q_n^- \frac{\partial \mu_0}{\partial r}\Big|^- \nonumber \\
 & + \frac{\partial c_{l,0}}{\partial \mu}\frac{\partial^2\mu_0}{\partial s^2}\int_{0}^{\eta}\Big[q_t(\phi_0) - q_t(\phi_o|^+)\Big]\mathrm{d}\eta 
 +  \eta \frac{\partial c_{l,0}}{\partial \mu}\frac{\partial^2\mu_0}{\partial s^2}q_t(\phi_o|^+)  \nonumber \\
 &+ a^{\prime}(\phi_0)\phi_1 (\tilde{c}_{s,0} - \tilde{c}_{l,0})V_n + a(\phi_0)(\tilde{c}_{s,0}-\tilde{c}_{l,0})\frac{V_n}{\frac{\partial\phi_0}{\partial\eta}}\frac{\partial\phi_1}{\partial\eta} + a(\phi_0)(\tilde{c}_{s,1}-\tilde{c}_{l,1})V_n + A_3(s) \nonumber \\
\end{align}
Taking the far field limit $\eta \rightarrow +\infty$ in Eq.~(\ref{mc5}) and employing the matching conditions we have
\begin{equation}\label{mc7}
    \lim_{\eta\rightarrow +\infty} - V_n c_1 = -V_n \Big( c_1|^+ + \eta\frac{\partial c_0}{\partial r}\Bigg|^+\Big)
\end{equation}
\begin{equation}\label{mc8}
    \lim_{\eta\rightarrow +\infty}q_n(\phi_0)\frac{\partial \mu_2}{\partial \eta}  = q_n^+ \Bigg[\frac{\partial \mu_1}{\partial r}\Bigg|^+ + \eta \frac{\partial ^2 \mu_0}{\partial r^2}\Bigg|^+\Bigg]
\end{equation}
In addition we have $\lim_{\eta\rightarrow +\infty} q_n^{\prime}(\phi_0), a(\phi_0), a^{\prime}(\phi_0) = 0$ such that the second, seventh, eighth and ninth term in Eq.~(\ref{mc5}) is equal to zero. Substituting back in Eq.~(\ref{mc5})
\begin{align}\label{mc9}
 -V_n \Big( c_1|^+ + \eta\frac{\partial c_0}{\partial r}\Bigg|^+\Big) = & \frac{\partial c_{l,0}}{\partial \mu}q_n^+ \Bigg[\frac{\partial \mu_1}{\partial r}\Bigg|^+ + \eta \frac{\partial ^2 \mu_0}{\partial r^2}\Bigg|^+\Bigg] - V_n \kappa \int_{0}^{+\infty}\Big[ (c_0 - c_{s,0}) - (c_{l,0}-c_{s,0})\Big]\mathrm{d}\eta \nonumber \\
 - & \eta V_n \kappa (c_{l,0}-c_{s,0}) + \eta \kappa \frac{\partial c_{l,0}}{\partial \mu}q_n^- \frac{\partial \mu_0}{\partial r}\Big|^- \nonumber \\
 &  + \frac{\partial c_{l,0}}{\partial \mu}\frac{\partial^2\mu_0}{\partial s^2}\int_{0}^{+\infty}\Big[q_t(\phi_0) - q_t(\phi_o|^+)\Big]\mathrm{d}\eta + \frac{\partial c_{l,0}}{\partial \mu}\eta \frac{\partial^2\mu_0}{\partial s^2}q_t(\phi_o|^+) \nonumber \\
 &+ A_3(s)
\end{align}
Comparing the coefficients of $\eta^0$ on both sides we obtain
\begin{align}\label{mc10}
 -V_n c_1|^+ = & \frac{\partial c_{l,0}}{\partial \mu}q_n^+ \frac{\partial \mu_1}{\partial r}\Bigg|^+ - V_n \kappa\int_{0}^{+\infty}\Big[ (c_0 - c_{s,0}) - (c_{l,0}-c_{s,0})\Big]\mathrm{d}\eta \nonumber \\
 & + \frac{\partial c_{l,0}}{\partial \mu}\frac{\partial^2\mu_0}{\partial s^2}\int_{0}^{+\infty}\Big[q_t(\phi_0) - q_t(\phi_o|^+)\Big]\mathrm{d}\eta + A_3(s)
\end{align}
Substituting the lowest order concentration profile i.e. $c_0 = c_{s,0}h(\phi_0) + c_{l,0}\{1-h(\phi_0)\}$ and re-arranging the equation we have
\begin{align}\label{mc11}
 q_n^+ \frac{\partial \mu_1}{\partial r}\Bigg|^+ = & -\frac{V_n}{\partial c_{l,0}/\partial \mu} c_1|^+ - \frac{V_n}{\partial c_{l,0}/\partial\mu} \kappa (c_{l,0}-c_{s,0})\int_{0}^{+\infty}h(\phi_0)\mathrm{d}\eta \nonumber \\
 & - \frac{\partial^2\mu_0}{\partial s^2}\int_{0}^{+\infty}\Big[q_t(\phi_0) - q_t(\phi_o|^+)\Big]\mathrm{d}\eta + A_3(s) 
\end{align}
Similarly taking the limit $\eta \rightarrow -\infty$ in Eq.~(\ref{mc6}) and comparing the coefficients of $\eta^0$ we obtain
\begin{align}\label{mc12}
 q_n^- \frac{\partial \mu_1}{\partial r}\Bigg|^- = & -\frac{V_n}{\partial c_{l,0}/\partial\mu} c_1|^- - \frac{V_n}{\partial c_{l,0}/\partial \mu} \kappa (c_{l,0}-c_{s,0})\int_{0}^{-\infty}\{1-h(\phi_0)\}\mathrm{d}\eta \nonumber \\
 & - \frac{\partial^2\mu_0}{\partial s^2}\int_{0}^{+\infty}\Big[q_t(\phi_0) - q_t(\phi_o|^-)\Big]\mathrm{d}\eta +A_3(s) 
\end{align}
Subtracting Eq.~(\ref{mc12}) from Eq.~(\ref{mc11}) we have
\begin{eqnarray}\label{mc13}
 q_n^+ \frac{\partial \mu_1}{\partial r}\Bigg|^+ - q_n^- \frac{\partial \mu_1}{\partial r}\Bigg|^- = -\frac{V_n}{\partial c_{l,0}/\partial\mu}(c_{l,1}-c_{s,1}) - \frac{V_n}{\partial c_{l,0}/\partial\mu} \kappa (H^+ - H^-) - \frac{\partial^2\mu_0}{\partial s^2} (S^+ - S^-) \nonumber \\
\end{eqnarray}
where,
\begin{equation}\label{mc14}
    H^{\pm} = \int_{0}^{\pm\infty} [h(\phi_0) - h(\phi_0|^{\pm})]\mathrm{d}\eta,
\end{equation}
and 
\begin{equation}\label{mc15}
    S^{\pm} = \int_{0}^{\pm\infty}[q_t(\phi_0) - q_t(\phi_0|^{\pm})]\mathrm{d}\eta.
\end{equation}
The second and the third term on the right hand side of Eq.~(\ref{mc13}) corresponds to the correction to the mass conservation because of the presence of interface stretching and surface diffusion. From Eqs.~(\ref{mc14}) and (\ref{mc15}) both these effects can be eliminated if $h(\phi_0)$ and $q_t(\phi_0)$ are anti-symmetric. Therefore, the first order mass conservation reads as
\begin{equation}\label{mc16}
    q_n^+ \frac{\partial \mu_1}{\partial r}\Bigg|^+ - q_n^- \frac{\partial \mu_1}{\partial r}\Bigg|^- = -\frac{V_n}{\partial c_{l,0}/\partial\mu}(c_{l,1}-c_{s,1}) .
\end{equation}
The overall mass conservation reads as
\begin{align}
 q_n^+\frac{\partial\mu}{\partial r}\Bigg|^+ - q_n^-\frac{\partial\mu}{\partial r}\Bigg|^- = & q_n^+ \frac{\partial}{\partial r}(\mu_0 + \epsilon\mu_1|^+) - q_n^- \frac{\partial}{\partial r}(\mu_0 + \epsilon\mu_1^-) \nonumber \\
  = & \Bigg(q_n^+\frac{\partial\mu_0}{\partial r}\Big|^+ - q_n^-\frac{\partial\mu_0}{\partial r}\Big|^-\Bigg) + \epsilon \Bigg(q_n^+\frac{\partial\mu_1}{\partial r}\Big|^+ - q_n^-\frac{\partial\mu_1}{\partial r}\Big|^-\Bigg)
\end{align}
Substituting Eqs.~(\ref{st11}) and (\ref{mc16}) we obtain
\begin{align}
 q_n^+\frac{\partial\mu}{\partial r}\Bigg|^+ - q_n^-\frac{\partial\mu}{\partial r}\Bigg|^- = & -\frac{V_n}{\partial c_{l,0}/\partial\mu}(c_{l,0}-c_{s,0}) - \epsilon \frac{V_n}{\partial c_{l,0}/\partial\mu}(c_{l,1}-c_{s,1}) \nonumber \\
  = & -\frac{V_n}{\partial c_{l,0}/\partial\mu}\Big[(c_{l,0} + \epsilon c_{l,1}) - (c_{s,0}+\epsilon c_{l,1})\Big]
\end{align}
 % Create the reference section using BibTeX:

\section{Asymptotics of the model with $\mu_s \neq \mu_l$}
We proceed similarly as in the preceeding section and work out the asymptotics by assuming $g(\phi) = h(\phi)$.
\subsection{$\mathcal{O}(\epsilon^0)$ in $\phi$ : Equilibrium profile}
The leading order terms in $\phi$ equation writes as
\begin{equation}\label{eq2_1}
    \frac{\chi\frac{D_s}{D_l}\{1-h(\phi_0)\}}{\chi\frac{D_s}{D_l}\{1-h(\phi_0)\} + h(\phi_0)}\Big[\frac{\partial^2\phi_0}{\partial\eta^2} - f_{\mathrm{dw}}^{\prime}(\phi_0)\Big] + \frac{h(\phi_0)}{\chi\frac{D_s}{D_l}\{1-h(\phi_0)\} + h(\phi_0)}\Big[\frac{\partial^2\phi_0}{\partial\eta^2} - f_{\mathrm{dw}}^{\prime}(\phi_0)\Big] = 0.
\end{equation}
where $\chi = \partial c_s/\partial \mu_s\Big/\partial c_l/\partial\mu_l$. The above equation can be simplified and integrated to obtain the equilibrium phase-field profile as
\begin{equation}
    \phi_0(\eta) = \frac{1}{2}\Bigg[1-\tanh\left(\frac{\eta}{\sqrt{2}}\right)\Bigg]
\end{equation}
\subsection{$\mathcal{O}(1/\epsilon^2)$ in $c$ : Equality of diffusion potential}
At the leading order in the $c$ equation we have
\begin{equation}\label{dp2_1}
    \frac{\partial}{\partial \eta}\Bigg[\frac{\partial c_{l,0}}{\partial\mu_l}q_n(\phi_0)\frac{\partial}{\partial\eta}\Big\{\mu_{s,0}h(\phi_0)+\mu_{l,0}\{1-h(\phi_0)\}\Big\}\Bigg] = 0.
\end{equation}
Integrating once,
\begin{equation}\label{dp2_2}
    \frac{\partial c_{l,0}}{\partial\mu_l}q_n(\phi_0)\frac{\partial}{\partial\eta}\Big\{\mu_{s,0}h(\phi_0)+\mu_{l,0}\{1-h(\phi_0)\}\Big\} = A_1(s)
\end{equation}
Taking the limit $\eta \rightarrow \pm\infty$ and employing the matching condition $\lim_{\eta \rightarrow -\infty}\frac{\partial\mu_{s,0}}{\partial\eta} = 0$ and $\lim_{\eta \rightarrow +\infty}\frac{\partial\mu_{l,0}}{\partial\eta} = 0$ we obtain $A_1(s) = 0$. Since $q_n(\phi_0)$ and $\frac{\partial c_{l,0}}{\partial \mu_l}$ are finite in the inner region, Eq.~(\ref{dp2_2}) reduces to
\begin{equation}\label{dp2_3}
   \frac{\partial}{\partial\eta}\Big\{\mu_{s,0}h(\phi_0)+\mu_{l,0}\{1-h(\phi_0)\}\Big\} = 0 .
\end{equation}
Integrating again we have
\begin{equation}\label{dp2_4}
    \mu_{s,0}h(\phi_0)+\mu_{l,0}\{1-h(\phi_0)\} = \overline{\mu}_0(s),
\end{equation}
where $\overline{\mu}_0(s)$ is a constant of integration only dependent on the arc length. Taking $\lim_{\eta \rightarrow -\infty}$
\begin{equation}\label{dp2_5}
    \lim_{\eta \rightarrow -\infty} \mu_{s,0} = \overline{\mu}_0(s),
\end{equation}
which implies
\begin{equation}\label{dp2_6}
    \mu_{s,0}|^- = \frac{\partial f_s(c_{s,0}|^-)}{\partial c_s} =  \overline{\mu}_0(s).
\end{equation}
Similarly taking the $\lim_{\eta \rightarrow +\infty}$ we obtain
\begin{equation}\label{dp2_7}
    \mu_{l,0}|^+ = \frac{\partial f_l(c_{l,0}|^+)}{\partial c_l} =  \overline{\mu}_0(s).
\end{equation}
From Eqs.~(\ref{dp2_6}) and (\ref{dp2_7}) we have
\begin{equation}\label{dp2_8}
    \mu_{s,0}|^- = \mu_{l,0}|^+
\end{equation}
which implies the equality of diffusion potential at the interface which is similar to the lowest order solution of the model in which pointwise equality of the diffusion potential (i.e. $\mu_s \ \mu_l$) was assumed.

\subsection{$\mathcal{O}(\epsilon)$ in $\phi$}
\begin{align}\label{gb2_1}
    -\alpha V_n\frac{\partial\phi_0}{\partial\eta} & =  \frac{\partial^2\phi_1}{\partial\eta^2} + \kappa\frac{\partial\phi_0}{\partial\eta} - f_{\mathrm{dw}}^{\prime\prime}(\phi_0)\phi_1 - a_1 \frac{\chi\frac{D_s}{D_l}\{1-h(\phi_0)\}}{\chi\frac{D_s}{D_l}\{1-h(\phi_0)\} + h(\phi_0)}\Bigg[ f_s(c_{s,0}) - f_l(c_{l,0}) \nonumber \\
   & - \mu_{s,0}(c_{s,0}-c_{l,0}) - S_l\{1-h(\phi_0)\}(\mu_{s,0}-\mu_{l,0})\Bigg]h^{\prime}(\phi_0) - a_1\frac{h(\phi_0)}{\chi\frac{D_s}{D_l}\{1-h(\phi_0)\} + h(\phi_0)} \times \nonumber \\
   & \Bigg[f_s(c_{s,0}) - f_l(c_{l,0})- \mu_{l,0}(c_{s,0}-c_{l,0}) - S_s h(\phi_0)(\mu_{l,0}-\mu_{s,0})\Bigg]h^{\prime}(\phi_0) \nonumber \\
   & - a_1 \frac{(\tilde{c}_{s,0}-\tilde{c}_{l,0})}{|\frac{\partial\phi_0}{\partial\eta}|}\frac{h(\phi_0)\{1-h(\phi_0)\}}{\chi\frac{D_s}{D_l}\{1-h(\phi_0)\} + h(\phi_0)}\Big( \chi\frac{D_s}{D_l}\frac{\partial\mu_{s,0}}{\partial\eta} - \frac{\partial\mu_{l,0}}{\partial\eta}\Big).
\end{align}
Re-arranging the above equation
\begin{align}\label{gb2_2}
    -(\alpha V_n + \kappa)\frac{\partial\phi_0}{\partial\eta} = & \frac{\partial^2\phi_1}{\partial\eta^2} - f_{\mathrm{dw}}^{\prime\prime}(\phi_0)\phi_1 - a_1[f_s(c_{s,0}) - f_l(c_{l,0})]h^{\prime}(\phi_0) \nonumber \\
   & + a_1 \frac{\chi\frac{D_s}{D_l}\{1-h(\phi_0)\}}{\chi\frac{D_s}{D_l}\{1-h(\phi_0)\}+h(\phi_0)}\mu_{s,0}(c_{s,0}-c_{l,0})h^{\prime}(\phi_0) \nonumber \\
    & + a_1\frac{h(\phi_0)}{\chi\frac{D_s}{D_l}\{1-h(\phi_0)\}+h(\phi_0)}\mu_{l,0}(c_{s,0}-c_{l,0})h^{\prime}(\phi_0) \nonumber \\
    & - a_1 \frac{(c_{s,0}-c_{l,0})h(\phi_0)\{1-h(\phi_0)\}}{|\frac{\partial\phi_0}{\partial\eta}|\Big[\chi\frac{D_s}{D_l}\{1-h(\phi_0)\} + h(\phi_0)\Big]}\chi\frac{D_s}{D_l}\frac{\partial\mu_{s,0}}{\partial \eta} \nonumber \\
    & + a_1 \frac{(c_{s,0}-c_{l,0})h(\phi_0)\{1-h(\phi_0)\}}{|\frac{\partial\phi_0}{\partial\eta}|\Big[\chi\frac{D_s}{D_l}\{1-h(\phi_0)\} + h(\phi_0)\Big]}\frac{\partial\mu_{l,0}}{\partial \eta} \nonumber \\
    & + a_1 \frac{\chi\frac{D_s}{D_l}\{1-h(\phi_0)\}}{\chi\frac{D_s}{D_l}\{1-h(\phi_0)\}+h(\phi_0)}S_l\{1-h(\phi_0)\}(\mu_{s,0}-\mu_{l,0})h^{\prime}(\phi_0) \nonumber \\
    & + a_1\frac{h(\phi_0)}{\chi\frac{D_s}{D_l}\{1-h(\phi_0)\}+h(\phi_0)}S_sh(\phi_0)(\mu_{l,0}-\mu_{s,0})h^{\prime}(\phi_0) \nonumber \\
    & -a_1 \frac{S_sh(\phi_0)}{|\frac{\partial\phi_0}{\partial\eta}|}\frac{h(\phi_0)\{1-h(\phi_0)\}}{\chi\frac{D_s}{D_l}\{1-h(\phi_0)\} + h(\phi_0)} \chi\frac{D_s}{D_l}\frac{\partial\mu_{s,0}}{\partial\eta} \nonumber \\
    & -a_1 \frac{S_l\{1-h(\phi_0)\}}{|\frac{\partial\phi_0}{\partial\eta}|}\frac{h(\phi_0)\{1-h(\phi_0)\}}{\chi\frac{D_s}{D_l}\{1-h(\phi_0)\} + h(\phi_0)} \chi\frac{D_s}{D_l}\frac{\partial\mu_{s,0}}{\partial\eta} \nonumber \\
    & + a_1 \frac{S_sh(\phi_0)}{|\frac{\partial\phi_0}{\partial\eta}|}\frac{h(\phi_0)\{1-h(\phi_0)\}}{\chi\frac{D_s}{D_l}\{1-h(\phi_0)\} + h(\phi_0)} \frac{\partial\mu_{l,0}}{\partial\eta} \nonumber \\
    & + a_1 \frac{S_l\{1-h(\phi_0)\}}{|\frac{\partial\phi_0}{\partial\eta}|}\frac{h(\phi_0)\{1-h(\phi_0)\}}{\chi\frac{D_s}{D_l}\{1-h(\phi_0)\} + h(\phi_0)} \frac{\partial\mu_{l,0}}{\partial\eta}
\end{align}
We next multiply both sides by $\frac{\partial\phi_0}{\partial\eta}$ and integrate within the limits $-\infty$ to $+\infty$. As shown before, the first two terms on the RHS can be simplified as
\begin{equation}\label{gb2_3}
   1 + 2 = \int_{-\infty}^{+\infty}\frac{\partial^2\phi_1}{\partial\eta^2}\frac{\partial\phi_0}{\partial\eta}\mathrm{d}\eta - \int_{-\infty}^{+\infty}f_{\mathrm{dw}}^{\prime\prime}(\phi_0)\phi_1\frac{\partial\phi_0}{\partial\eta}\mathrm{d}\eta = 0.
\end{equation}
The third term on the RHS can be simplified by considering
\begin{align}\label{gb2_4}
    \frac{\mathrm{d}f}{\partial\eta}(c_{s,0},c_{l,0},\phi_0) = \frac{\partial f}{\partial c_s}\frac{\partial c_{s,0}}{\partial\eta} + \frac{\partial f}{\partial c_l}\frac{\partial c_{l,0}}{\partial\eta} + \frac{\partial f}{\partial \phi}\frac{\partial \phi_0}{\partial\eta}.
\end{align}
Substituting the partial derivatives of the free energy we obtain
\begin{align}\label{gb2_5}
   \frac{\mathrm{d}f}{\partial\eta}(c_{s,0},c_{l,0},\phi_0) = \mu_{s,0}h(\phi_0)\frac{\partial c_{s,0}}{\partial\eta} + \mu_{l,0}\{1-h(\phi_0)\}\frac{\partial c_{l,0}}{\partial\eta} + \Big\{f_s(c_{s,0}) - f_l(c_{l,0})\Big\}h^{\prime}(\phi_0)\frac{\partial \phi_0}{\partial\eta}.
\end{align}
Integrating both sides within the limits $\pm\infty$ the third term on the RHS of Eq.~(\ref{gb2_2}) (after multiplication with $\frac{\partial\phi_0}{\partial\eta}$) can be written as
\begin{align}\label{gb2_6}
    -a_1\int_{-\infty}^{+\infty}\Big\{f_s(c_{s,0}) - f_l(c_{l,0})\Big\}h^{\prime}(\phi_0)\frac{\partial \phi_0}{\partial\eta}\mathrm{d}\eta = & -a_1\int_{-\infty}^{+\infty}\frac{\mathrm{d}f}{\mathrm{d}\eta}\mathrm{d}\eta + a_1\int_{-\infty}^{+\infty}\mu_{s,0}h(\phi_0)\frac{\partial c_{s,0}}{\partial\eta}\mathrm{d}\eta \nonumber \\
    & + a_1\int_{-\infty}^{+\infty}\mu_{l,0}\{1-h(\phi_0)\}\frac{\partial c_{l,0}}{\partial\eta}\mathrm{d}\eta
\end{align}
The second and the third integral on the RHS can be simplified via integration by parts with $\mu_{s,0}h(\phi_0)$ and $\mu_{l,0}\{1-h(\phi_0)\}$ as the first functions to yield
\begin{align}\label{gb2_7}
    -a_1\int_{-\infty}^{+\infty}\Big\{f_s(c_{s,0}) - f_l(c_{l,0})\Big\}h^{\prime}(\phi_0)\frac{\partial \phi_0}{\partial\eta}\mathrm{d}\eta = & a_1 (\omega_s - \omega_l) - a_1\int_{-\infty}^{+\infty}\frac{\partial}{\partial\eta}[\mu_{s,0}h(\phi_0)]c_{s,0}\mathrm{d}\eta \nonumber \\
    & - a_1\int_{-\infty}^{+\infty}\frac{\partial}{\partial\eta}[\mu_{l,0}\{1-h(\phi_0)\}]c_{l,0}\mathrm{d}\eta
\end{align}
where, 
\begin{equation}
    \omega_s - \omega_l = f_s(c_{s,0}|^-,1) - \mu_{s,0}|^-c_{s,0}|^- f_l(c_{l,0}|^+) + \mu_{l,0}^+c_{l,0}^+
\end{equation}
which can be further simplified using Eq.~(\ref{dp2_8}) to yield
\begin{equation}
    \omega_s - \omega_l = f_s(c_{s,0}|^-,1) - f_l(c_{l,0}|^+,0) - \mu_{l,0}^+(c_{s,0}|^-- c_{l,0}^+)
\end{equation}
Combining the fourth and sixth term on the RHS of Eq.~(\ref{gb2_2}) we obtain
\begin{align}\label{gb2_8}
    &a_1\int_{-\infty}^{+\infty}\frac{\chi\frac{D_s}{D_l}\{1-h(\phi_0)\}}{\chi\frac{D_s}{D_l}\{1-h(\phi_0)\} + h(\phi_0)}\Bigg\{ \mu_{s,0}(c_{s,0}-c_{l,0})h^{\prime}(\phi_0)\frac{\partial\phi_0}{\partial\eta} + (c_{s,0}-c_{l,0})h(\phi_0)\frac{\partial\mu_{s,0}}{\partial\eta}\Bigg\}\mathrm{d}\eta \nonumber \\
    & = a_1 \int_{-\infty}^{+\infty}\frac{\chi\frac{D_s}{D_l}\{1-h(\phi_0)\}}{\chi\frac{D_s}{D_l}\{1-h(\phi_0)\} + h(\phi_0)} (c_{s,0}-c_{l,0})\frac{\partial}{\partial\eta}\Big[\mu_{s,0}h(\phi_0)\Big]\mathrm{d}\eta
\end{align}
Similarly, combining fifth and seventh term on the RHS of Eq.~(\ref{gb2_2}) we obtain
\begin{align}\label{gb2_9}
    &a_1\int_{-\infty}^{+\infty}\frac{h(\phi_0)}{\chi\frac{D_s}{D_l}\{1-h(\phi_0)\} + h(\phi_0)}\Bigg\{ \mu_{l,0}(c_{s,0}-c_{l,0})h^{\prime}(\phi_0)\frac{\partial\phi_0}{\partial\eta} - (c_{s,0}-c_{l,0})\{1-h(\phi_0)\}\frac{\partial\mu_{l,0}}{\partial\eta}\Bigg\}\mathrm{d}\eta \nonumber \\
    & = -a_1 \int_{-\infty}^{+\infty}\frac{h(\phi_0)}{\chi\frac{D_s}{D_l}\{1-h(\phi_0)\} + h(\phi_0)} (c_{s,0}-c_{l,0})\frac{\partial}{\partial\eta}\Big[\mu_{l,0}\{1-h(\phi_0)\}\Big]\mathrm{d}\eta 
\end{align}
Therefore using Eqs.~(\ref{gb2_7}),(\ref{gb2_8}) and (\ref{gb2_9}) the third, fourth, fifth, sixth and seventh term on the RHS of Eq.~(\ref{gb2_2}) writes as
\begin{equation}\label{gb2_10}
    3 + 4 + 5 + 6 +7 = a_1(\omega_s - \omega_l)
\end{equation}
Combining eighth and eleventh terms on the RHS of Eq.~(\ref{gb2_2})
\begin{align}\label{gb2_11}
  8+11 =&  a_1\int_{-\infty}^{+\infty}\frac{S_l\{1-h(\phi_0)\}\chi\frac{D_s}{D_l}\{1-h(\phi_0)\}}{\chi\frac{D_s}{D_l}\{1-h(\phi_0)\}+h(\phi_0)}\frac{\partial}{\partial\eta}\Big[\mu_{s,0}h(\phi_0)\Big]\mathrm{d}\eta \nonumber \\
  & -a_1\int_{-\infty}^{+\infty}\frac{S_l\{1-h(\phi_0)\}\chi\frac{D_s}{D_l}\{1-h(\phi_0)\}}{\chi\frac{D_s}{D_l}\{1-h(\phi_0)\}+h(\phi_0)}\mu_{l,0}h^{\prime}(\phi_0)\frac{\partial\phi_0}{\partial\eta}\mathrm{d}\eta
\end{align}
Similarly combining ninth and twelfth terms on the RHS of Eq.~(\ref{gb2_2}) we have
\begin{align}\label{gb2_12}
  9+12 =& -a_1\int_{-\infty}^{+\infty}\frac{S_sh(\phi_0) h(\phi_0)}{\chi\frac{D_s}{D_l}\{1-h(\phi_0)\}+h(\phi_0)}\frac{\partial}{\partial\eta}\Big[\mu_{l,0}\{1-h(\phi_0)\}\Big]\mathrm{d}\eta \nonumber \\
  & -a_1\int_{-\infty}^{+\infty}\frac{S_sh(\phi_0)h(\phi_0)}{\chi\frac{D_s}{D_l}\{1-h(\phi_0)\}+h(\phi_0)}\mu_{s,0}h^{\prime}(\phi_0)\frac{\partial\phi_0}{\partial\eta}\mathrm{d}\eta
\end{align}
Using the results of Eqs.~(\ref{gb2_10}), (\ref{gb2_11}) and (\ref{gb2_12}) and adding the tenth and thirteenth terms in Eq.~(\ref{gb2_2}) we obtain
\begin{align}\label{gb2_13}
    -(\alpha V_n + \kappa)\int_{-\infty}^{+\infty}\Big(\frac{\partial\phi_0}{\partial\eta}\Big)^2\mathrm{d}\eta = & a_1(\omega_s - \omega_l) + a_1\int_{-\infty}^{+\infty}S_s h(\phi_0)\frac{\partial \mu_{s,0}}{\partial\eta}h(\phi_0)\mathrm{d}\eta \nonumber \\
    & -a_1 \int_{-\infty}^{+\infty} S_l\{1-h(\phi_0)\}\frac{\partial\mu_{l,0}}{\partial\eta}\{1-h(\phi_0)\}\mathrm{d}\eta.
\end{align}
The above equation can also be expressed in the following form
\begin{align}
   -(\alpha V_n + \kappa)\int_{-\infty}^{+\infty}&\Big(\frac{\partial\phi_0}{\partial\eta}\Big)^2\mathrm{d}\eta =  a_1(\omega_s - \omega_l) - a_1\int_{-\infty}^{+\infty}[S_sh(\phi_0)+S_l\{1-h(\phi_0)\}] \times \nonumber \\
   & \frac{\partial\mu_{l,0}}{\partial\eta}\{1-h(\phi_0)\}\mathrm{d}\eta -a_1\int_{-\infty}^{+\infty} S_sh(\phi_0)(\mu_{s,0}-\mu_{l,0})h^{\prime}(\phi_0)\frac{\partial\phi_0}{\partial\eta}\mathrm{d}\eta
\end{align}
The Gibbs-Thomson condition at the lowest order depends upon the choice of the source terms.

\subsection{$\mathcal{O}(1/\epsilon)$ in $c$ : Solute trapping}
The $\mathcal{O}(1/\epsilon)$ in $c$ equation can be equated as
\begin{align}\label{sl2_1}
    -V_n\frac{\partial c_0}{\partial\eta} = & \frac{\partial}{\partial\eta}\Bigg[\frac{\partial c_{l,0}}{\partial\mu_l} q_n(\phi_0)\frac{\partial}{\partial\eta}\Big\{\mu_{s,1}h(\phi_0) + \mu_{l,1}\{1-h(\phi_0)\}\Big\} \nonumber \\
    & + \frac{\partial c_{l,0}}{\partial\mu_l}q_n(\phi_0)\frac{\partial}{\partial\eta}\Big\{(\mu_{s,0} - \mu_{l,0})h^{\prime}(\phi_0)\phi_1\Big\} \nonumber \\
     & +\frac{\partial c_{l,0}}{\partial\mu_l}q_n^{\prime}(\phi_0)\phi_1\frac{\partial}{\partial\eta}\Big\{\mu_{s,0}h(\phi_0) + \mu_{l,0}\{1-h(\phi_0)\}\Big\}\Bigg] \nonumber \\
     &- \frac{\partial}{\partial\eta}\Bigg[(\tilde{c}_{s,0} - \tilde{c}_{l,0}) a(\phi_0)V_n\Bigg].
\end{align}
From the leading order solution, since, $\mu_{s,0}h(\phi_0) + \mu_{l,0}\{1-h(\phi_0)\}$ is independent of $\eta$, the third term on the RHS of the above equation is equal to zero. Integrating the above equation we obtain
\begin{align}\label{sl2_2}
    -V_n c_0 = & \frac{\partial c_{l,0}}{\partial\mu_l}q_n(\phi_0)\frac{\partial}{\partial\eta}\Big\{\mu_{s,1}h(\phi_0) + \mu_{l,1}\{1-h(\phi_0)\}\Big\} \nonumber \\
    & + \frac{\partial c_{l,0}}{\partial\mu_l}q_n(\phi_0)\frac{\partial}{\partial\eta}\Big\{(\mu_{s,0} - \mu_{l,0})h^{\prime}(\phi_0)\phi_1\Big\} 
    - (\tilde{c}_{s,0} - \tilde{c}_{l,0}) a(\phi_0)V_n + A_2(s).
\end{align}
Taking the limit $\eta \rightarrow -\infty$ we evaluate the constant of integration $A_2(s)$ as
\begin{equation}\label{sl2_3}
    A_2(s) = -V_n c_{s,0} - \frac{\partial c_{l,0}}{\partial\mu_l}q_n^-\frac{\partial\mu_{s,0}}{\partial r}\Bigg|^-.
\end{equation}
Substituting for $A_2(s)$ from the above equation to Eq.~(\ref{sl2_2}) and rearranging we obtain
\begin{align}\label{sl2_4}
    \frac{\partial}{\partial\eta}\Big[\mu_{s,1}h(\phi_0) + \mu_{l,1}\{1-h(\phi_0)\}\Big] = & -\frac{V_n}{\partial c_{l,0}/\partial\mu_l}\frac{(c_0 - c_{s,0})}{q_n(\phi_0)} - \frac{\partial}{\partial\eta}\Big[(\mu_{s,0}-\mu_{l,0})h^{\prime}(\phi_0)\phi_1\Big]\nonumber \\
    & +(\tilde{c}_{s,0} - \tilde{c}_{l,0})\frac{V_n}{\partial c_{l,0}/\partial\mu_l}\frac{a(\phi_0)}{q_n(\phi_0)} + q_n^-\frac{\partial\mu_{s,0}}{\partial r}\Bigg|^-\frac{1}{q_n(\phi_0)}.
\end{align}
Integrating the above equation
\begin{align}\label{sl2_5}
    \mu_{s,1}h(\phi_0) + \mu_{l,1}\{1-h(\phi_0)\} = & -V_n\int_0^{\eta}\frac{1}{\partial c_{l,0}/\partial\mu_l}\frac{(c_0 - c_{s,0})}{q_n(\phi_0)}\mathrm{d}\eta - (\mu_{s,0}-\mu_{l,0})h^{\prime}(\phi_0)\phi_1 \nonumber \\
    & + V_n\int_{0}^{\eta}\frac{V_n}{\partial c_{l,0}/\partial\mu_l}(\tilde{c}_{s,0} - \tilde{c}_{l,0})\frac{a(\phi_0)}{q_n(\phi_0)}\mathrm{d}\eta \nonumber \\
    & + q_n^-\frac{\partial\mu_{s,0}}{\partial r}\Bigg|^-\int_0^{\eta}\frac{1}{q_n(\phi_0)}\mathrm{d}\eta + \overline{\mu}_1(s),
\end{align}
where $\overline{\mu}_1(s)$ is the constant of integration.  The above equation can be split into two parts as follows \\
(1) For $\eta >0$ 
\begin{align}\label{sl2_6}
    \mu_{s,1}h(\phi_0) + \mu_{l,1}\{1-h(\phi_0)\} = & -V_n \int_0^{\eta}\frac{1}{\partial c_{l,0}/\partial\mu_l}\Bigg[\frac{(c_0 - c_{s,0})}{q_n(\phi_0)} - \frac{c_{l,0} - c_{s,0}}{q_n^+}\Bigg]\mathrm{d}\eta \nonumber \\
    & - \eta \frac{V_n}{\partial c_{l,0}/\partial\mu_l} \frac{c_{l,0} - c_{s,0}}{q_n^+}  + (\mu_{s,0}-\mu_{l,0})h^{\prime}(\phi_0)\phi_1 \nonumber \\
    & + V_n\int_{0}^{\eta}\frac{1}{\partial c_{l,0}/\partial\mu_l}(\tilde{c}_{s,0} - \tilde{c}_{l,0})\frac{a(\phi_0)}{q_n(\phi_0)}\mathrm{d}\eta \nonumber \\
    & + q_n^-\frac{\partial\mu_{s,0}}{\partial r}\Bigg|^-\int_0^{\eta}\Bigg[\frac{1}{q_n(\phi_0)} - \frac{1}{q_n^+}\Bigg]\mathrm{d}\eta \nonumber \\
    & +\eta q_n^-\frac{\partial\mu_{s,0}}{\partial r}\Bigg|^-\frac{1}{q_n^+} + \overline{\mu}_1(s)
\end{align}
(2) For $\eta < 0$
\begin{align}\label{sl2_7}
    \mu_{s,1}h(\phi_0) + \mu_{l,1}\{1-h(\phi_0)\} = & -V_n \int_0^{\eta}\frac{1}{\partial c_{l,0}/\partial\mu_l}\frac{(c_0 - c_{s,0})}{q_n(\phi_0)} \mathrm{d}\eta  + (\mu_{s,0}-\mu_{l,0})h^{\prime}(\phi_0)\phi_1 \nonumber \\ 
    & + V_n \int_{0}^{\eta}\frac{1}{\partial c_{l,0}/\partial\mu_l}(\tilde{c}_{s,0} - \tilde{c}_{l,0})\frac{a(\phi_0)}{q_n(\phi_0)}\mathrm{d}\eta \nonumber \\
    & + q_n^-\frac{\partial\mu_{s,0}}{\partial r}\Bigg|^-\int_0^{\eta}\Bigg[\frac{1}{q_n(\phi_0)} - \frac{1}{q_n^-}\Bigg]\mathrm{d}\eta \nonumber \\
    & +\eta q_n^-\frac{\partial\mu_{s,0}}{\partial r}\Bigg|^-\frac{1}{q_n^-} + \overline{\mu}_1(s)
\end{align}
Taking the far field limit $\eta \rightarrow +\infty$ in Eq.~(\ref{sl2_6}) and employing the matching condition we have
\begin{align}\label{sl2_8}
    \mu_{l,1}|^+ + \eta \frac{\partial \mu_{l,0}}{\partial r}\Bigg|^+ = & -V_n \int_0^{+\infty}\frac{1}{\partial c_{l,0}/\partial\mu_l}\Bigg[\frac{(c_0 - c_{s,0})}{q_n(\phi_0)} - \frac{c_{l,0} - c_{s,0}}{q_n^+}\Bigg]\mathrm{d}\eta - \eta \frac{V_n}{\partial c_{l,0}/\partial\mu_l} \frac{c_{l,0} - c_{s,0}}{q_n^+} \nonumber \\
    & + V_n \int_{0}^{+\infty}\frac{1}{\partial c_{l,0}/\partial\mu_l}(\tilde{c}_{s,0} - \tilde{c}_{l,0})\frac{a(\phi_0)}{q_n(\phi_0)}\mathrm{d}\eta \nonumber \\
    & + q_n^-\frac{\partial\mu_{s,0}}{\partial r}\Bigg|^-\int_0^{+\infty}\Bigg[\frac{1}{q_n(\phi_0)} - \frac{1}{q_n^+}\Bigg]\mathrm{d}\eta 
    +\eta q_n^-\frac{\partial\mu_{s,0}}{\partial r}\Bigg|^-\frac{1}{q_n^+} + \overline{\mu}_1(s),
\end{align}
where, we have utilized the fact that $\lim_{\eta \rightarrow +\infty} (\mu_{s,0}-\mu_{l,0})h^{\prime}(\phi_0)\phi_1 = 0$.
Comparing coefficients of $\eta$ on both sides we have
\begin{equation}\label{sl2_9}
    \frac{\partial \mu_{l,0}}{\partial r}\Bigg|^+ = -\frac{V_n}{\partial c_{l,0}/\partial\mu_l} \frac{c_{l,0} - c_{s,0}}{q_n^+} + q_n^-\frac{\partial\mu_{s,0}}{\partial r}\Bigg|^-\frac{1}{q_n^+} ,
\end{equation}
which on re-arranging gives the flux conservation or the Stefan's condition at the interface at the lowest order as
\begin{equation}\label{sl2_10}
   q_n^+ \frac{\partial \mu_{l,0}}{\partial r}\Bigg|^+ - q_n^- \frac{\partial \mu_{s,0}}{\partial r}\Bigg|^- = -\frac{V_n}{\partial c_{l,0}/\partial\mu_l} (c_{l,0} - c_{s,0}) .
\end{equation}
Similarly, comparing the coefficients of $\eta^0$ in Eq.~(\ref{sl2_8}) we have
\begin{align}\label{sl2_11}
   \mu_{l,1}|^+ =  & -V_n\int_0^{+\infty}\frac{1}{\partial c_{l,0}/\partial\mu_l}\Bigg[\frac{(c_0 - c_{s,0})}{q_n(\phi_0)} - \frac{c_{l,0} - c_{s,0}}{q_n^+}\Bigg]\mathrm{d}\eta  \nonumber \\
   & +  V_n \int_{0}^{+\infty}\frac{1}{\partial c_{l,0}/\partial\mu_l}(\tilde{c}_{s,0} - \tilde{c}_{l,0})\frac{a(\phi_0)}{q_n(\phi_0)}\mathrm{d}\eta \nonumber \\
   & + q_n^-\frac{\partial\mu_{s,0}}{\partial r}\Bigg|^-\int_0^{+\infty}\Bigg[\frac{1}{q_n(\phi_0)} - \frac{1}{q_n^+}\Bigg]\mathrm{d}\eta + \overline{\mu}_1(s).
\end{align}
Introducing the lowest order solute profile $c_0 = c_{s,0}h(\phi_0) + c_{l,0}\{1-h(\phi_0)\}$ and expanding $\tilde{c}_{s,0} - \tilde{c}_{l,0} = c_{s,0} -c_{l,0} + S_sh(\phi_0) + S_l\{1-h(\phi_0)\}$ in above equation and rearranging we obtain
\begin{align}\label{sl2_12}
    \mu_{l,1}|^+ =&  - V_n \int_{0}^{+\infty}\frac{(c_{s,0}-c_{l,0})}{\partial c_{l,0}/\partial\mu_l} \{\overline{p}(\phi_0) - \overline{p}(\phi_0|^+)\}\mathrm{d}\eta \nonumber \\
    & + V_n \int_{0}^{+\infty}\frac{1}{\partial c_{l,0}/\partial \mu_l} [S_sh(\phi_0) + S_l\{1-h(\phi_0)\}]\frac{a(\phi_0)}{q_n(\phi_0)}\mathrm{d}\eta \nonumber \\
    & + q_n^-\frac{\partial\mu_{s,0}}{\partial r}\Bigg|^-\int_0^{+\infty}\Bigg[\frac{1}{q_n(\phi_0)} - \frac{1}{q_n^+}\Bigg]\mathrm{d}\eta + \overline{\mu}_1(s),
\end{align}
where,
\begin{equation}\label{sl2_13}
    \overline{p}(\phi_0) = \frac{h(\phi_0) - 1 -a(\phi_0)}{q_n(\phi_0)}
\end{equation}
and we have utilized the relation $\overline{p}(\phi_0|^+) = 1/q_n^+ = 1$. Similarly, taking the far field $\eta \rightarrow -\infty$ of Eq.~(\ref{sl2_7}) and comparing the coefficients of $\eta^0$ we obtain,
\begin{align}\label{sl2_14}
    \mu_{s,1}|^- =& -V_n\int_{0}^{-\infty}\frac{(c_{s,0}-c_{l,0})}{\partial c_{l,0}/\partial\mu_l} \{p(\phi_0) - p(\phi_0|^-)\}\mathrm{d}\eta \nonumber \\
    & + V_n \int_{0}^{-\infty}\frac{1}{\partial c_{l,0}/\partial \mu_l} [S_sh(\phi_0) + S_l\{1-h(\phi_0)\}]\frac{a(\phi_0)}{q_n(\phi_0)}\mathrm{d}\eta \nonumber \\
    & + q_n^-\frac{\partial\mu_{s,0}}{\partial r}\Bigg|^-\int_0^{-\infty}\Bigg[\frac{1}{q_n(\phi_0)} - \frac{1}{q_n^+}\Bigg]\mathrm{d}\eta + \overline{\mu}_1(s).
\end{align}
Subtracting Eq.~(\ref{sl2_12}) from Eq.~(\ref{sl2_14}),
\begin{equation}\label{sl2_15}
    \mu_s|^- - \mu_l|^+ = -V_n(F_1^- - F_1^+) + V_n(F_2^- - F_2^+) + q_n^-\frac{\partial\mu_{s,0}}{\partial r}\Bigg|^-(G^- - G^+),
\end{equation}
where,
\begin{equation}\label{sl2_16}
    F_1^{\pm} = \int_{0}^{\pm\infty}\frac{(c_{s,0}-c_{l,0})}{\partial c_{l,0}/\partial\mu_l}\{\overline{p}(\phi_0) - \overline{p}(\phi_0|^{\pm})\}\mathrm{d}\eta,
\end{equation}
\begin{equation}\label{sl2_17}
  F_2^{\pm} = \int_{0}^{\pm\infty}\frac{[S_sh(\phi_0) + S_l\{1-h(\phi_0)\}]}{\partial c_{l,0}/\partial\mu_l}\frac{a(\phi_0)}{q_n(\phi_0)}\mathrm{d}\eta,    
\end{equation}
and
\begin{equation}\label{sl2_18}
    G^{\pm} = \int_0^{\pm\infty}\Bigg[\frac{1}{q_n(\phi_0)} - \frac{1}{q_n^{\pm}}\Bigg]\mathrm{d}\eta .
\end{equation}
With the choice of the source terms $S_s = S_l = A(c_{l,0}-c_{s,0})$ as deduced earlier, the diffusion potential jump can simplified to be
\begin{equation}\label{sl2_19}
    \mu_s|^- - \mu_l|^+ = V_n(F^- - F^+) + q_n^-\frac{\partial\mu_{s,0}}{\partial r}\Bigg|^-(G^- - G^+),
\end{equation}
where
\begin{equation}
    F^{\pm} = \int_0^{\pm}\frac{(c_{l,0}-c_{s,0})}{\partial c_{l,0}/\partial \mu_l}\{\tilde{p}(\phi_0) - \tilde{p}(\phi_0|^{\pm})\}\mathrm{d}\eta
\end{equation}
and 
\begin{equation}
    \tilde{p}(\phi_0) = \frac{h(\phi_0)-1-(1-A)a(\phi_0)}{q_n(\phi_0)}
\end{equation}
which is similar to the Eq.~(\ref{st16}) derived for the case where we had assumed $\mu_s = \mu_l$ with the exception that lowest order phase concentrations (and the thermodynamic factors $\partial c_{s,0}/\partial \mu_s$ and $\partial c_{l,0}/\partial \mu_l$) are dependent on the normal co-ordinate system $\eta$ i.e. $c_{s,0} = c_{s,0}(\eta)$, $c_{l,0} = c_{l,0}(\eta)$ and cannot be taken out of the integral to obtain closed form expressions. However, it is clear that the parameter $A$ can still be tuned to control partitioning.

As discussed in the next section, a numerical comparison of the two models suggest that the steady state diffusion potential in both the models (Fig.(\ref{supfig3})) are almost identical. A comparison of the velocity-dependent partition coefficient (Fig.(\ref{supfig2})) suggests that the difference between the two models depend on the value of $A$. The agreement between the predictions of the two models is better at higher values of $A$ within the entire velocity range considered. Even with lower values of $A$ (for instance $A=0.1$) the agreement is good until velocities of about $0.1$ m/s. A good agreement between both the models motivates us to utilize the same interpolation functions to negate the Kapitza jump (i.e. satisfy $G^- = G^+$).

The next order $\phi$ and $c$ equations are analytically not tractable. Hence no information regarding the correction to the Gibbs-Thomson and mass-conservation conditions are available.




\section{Comparative study of both the models }
A comparative study of the partition coefficient and interface temperature of the general model (which assumes $\mu_s \neq \mu_l$) with the model where $\mu_s = \mu_l$ is imposed are presented in Fig.\ref{supfig2}. To study the effect of the model parameter $A$ on the assumption $\mu_s = \mu_l$, we select three values of $A = 0.1, 0.4,1.0$. The interface width has been fixed to a value of $W = 13.5$ nm in all the simulations.

For velocities $v < 0.1$ m/s, both the models essentially give similar results for partition coefficient and interface temperature across all values of $A$. At larger velocities, the difference is prominent for $A = 0.1$ for both the partition coefficient and interface temperature. The difference between the results of the two models decrease with increasing $A$. For $A = 1.0$ and at $v = 1$m/s the difference in $k_v$ is about $10\%$ and $T_I$ is less than $1\%$. Since for solute trapping we mostly work in the regime of higher values of $A$, the asymptotic expression derived for the model $\mu_s = \mu_l$ acts as a good initial guess.

The steady state diffusion potentials across the interface from both the models are compared in Fig.\ref{supfig3}. The diffusion potential for the model $\mu_s \neq \mu_l$ is defined as $\mu = \mu_s h(\phi) + \mu_l\{1-h(\phi)\}$. The steady state diffusion potential from both the models are within few percent of each other.


\begin{sidewaysfigure}
    \centering
    \includegraphics[scale = 0.34]{supfig2}
    \caption{A comparison of the partition coefficient and interface temperature for the models with $\mu_s \neq \mu_l$ and with the assumption $\mu_s = \mu_l$ for different values of model parameter $A$. The interface width $W$ is chosen as $13.5$ nm. The assumption of $\mu_s = \mu_l$ is observed to be most valid at larger values of $AW$.}
    \label{supfig2}
\end{sidewaysfigure}

\begin{sidewaysfigure}
    \centering
    \includegraphics[scale = 0.34]{supfig3}
    \caption{A comparison of the steady state diffusion potential for the models with $\mu_s \neq \mu_l$ and with the assumption of $\mu_s = \mu_l$ for different values of model parameter $A$ and velocities. For the general model ($\mu_s \neq \mu_l$) the diffusion potential is defined as $\mu = \mu_s h(\phi) + \mu_l\{1-h(\phi)\}$.}
    \label{supfig3}
\end{sidewaysfigure}


\end{document}
%
% ****** End of file apstemplate.tex ******

