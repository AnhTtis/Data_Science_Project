
\documentclass[12pt,leqno]{article}
\usepackage{amssymb,amsmath,amsthm}
\usepackage[all]{xy}
\usepackage[alphabetic,lite]{amsrefs}
\numberwithin{equation}{section}

\setcounter{tocdepth}{2}
\usepackage[top=2cm,bottom=2cm,left=2cm,right=4cm,marginparsep=0.3cm,marginparwidth=3cm,includefoot]{geometry}
\usepackage{mathtools}
%\usepackage{icar}
\usepackage{unusual}
\usepackage{mathrsfs}

\usepackage{tikz-cd}


\usepackage[colorlinks=true, pdfstartview=FitV, linkcolor=blue, citecolor=blue, urlcolor=blue]{hyperref}

\begin{document}


\title
{Unusual functorialities for weakly constructible sheaves }
\date{}

\author{Andreas Hohl\thanks{The research of A.H.\ is funded by the Deutsche Forschungsgemeinschaft (DFG, German Research
		Foundation), Projektnummer 465657531.}\,\, and Pierre Schapira}
\maketitle

\begin{abstract}
We prove that various morphisms related to the six Grothendieck operations on sheaves become isomorphisms 
when restricted to (weakly) constructible sheaves.
\end{abstract}

\section{Introduction}
Let $\cor$ be a commutative unital ring of finite global dimension and 
denote by $\Derb(\cor_X)$ the bounded derived category of sheaves on a good topological space $X$.
There are some classical morphisms which are, in general, not isomorphisms, such as
\eqn
&&\rhom(F_1,F_2)\ltens F_3\to\rhom(F_1,F_2\ltens F_3),\\
&&\epb{f}G_1\ltens\opb{f}G_2\to\epb{f}(G_1\ltens G_2),
\eneqn
for $F_1,F_2,F_3\in\Derb(\cor_X)$, $G_1,G_2\in\Derb(\cor_Y)$ and $f\cl X\to Y$  a continuous map.
We will prove here that, under suitable hypotheses of (weak) constructibility, these morphisms become isomorphisms.



Assuming that the duality functor is conservative in $\Derb(\cor)$ (see~\eqref{eq:dualconserv}), 
 we prove the following results in the categories of real analytic manifolds and weakly $\R$-constructible sheaves. (See Theorems~\ref{th:invim},~\ref{th:tenshom} and~\ref{th:dirim}.) 

Let $f\cl X\to Y$ be a morphism of real analytic manifolds, and let $F_1, F_2, L_X\in  \Derb_\wRc(\cor_X)$,  $G, L_Y\in  \Derb_\wRc(\cor_Y)$
with moreover $F_1$ being $\R$-constructible, and $L_X,L_Y$ locally constant.
Then we have the isomorphisms
\eqn
&& \epb{f}G\tens\opb{f}L_Y\isoto\epb{f}(G\tens L_Y),\\
&&\opb{f}\rhom(L_Y,G)\isoto \rhom(\opb{f}L_Y,\opb{f}G),\\
&&\rhom(F_1,F_2)\tens L_X\isoto\rhom(F_1,F_2\tens L_X),\\
&&\rhom(L_X,F_2)\tens F_1\isoto\rhom(L_X,F_2\tens F_1).
\eneqn
For direct images we need slightly stronger assumptions: consider  a morphism $f\cl \fX\to\fY$  of b-analytic manifolds (see~\cite{Sc23} for this notion) and let   $F\in \Derb_\wRc(\cor_\fX)$ be weakly $\R$-constructible up to infinity and $L_Y$ be locally constant. We prove the isomorphisms 
\eqn
&&\roim{f}F\tens L_Y\isoto \roim{f}(F\tens \opb{f}L_Y),\\
&&\reim{f}\rhom(\opb{f}L_Y,F)\isoto \rhom(L_Y,\reim{f}F). 
\eneqn

%%%%%%%
We start by introducing the notion (implicitly already defined in~\cite{KS90}*{\S~3.4}) of weakly  cohomologically constructible sheaves (wcc-sheaves, for short) and the full subcategory $\Derb_\wcc(\cor_X)$ of $\Derb(\cor_X)$ consisting of such sheaves. On a  real analytic manifold,  the category of $\Derb_\wcc(\cor_X)$   contains the category $\Derb_\wRc(\cor_X)$  of
weakly  $\R$-constructible sheaves. 


We prove first that  $\Derb_\wcc(\cor_X)$ is triangulated. Then our main tool is that for an object $F$ of this category, for $x\in X$ and $L\in\Derb(\cor)$, one has functorial isomorphisms
\eqn
&&\rhom(L_X,F)_x\isoto \rhom(L,F_x),\quad \rsect_xF\tens L\isoto \rsect_x(F\tens L_X),
\eneqn
where $L_X$ denotes the constant sheaf associated with $L$.

The motivation for this note came through the preprint \cite{Ho23}, where field extensions for sheaves are considered, and many of the desired functorialities of loc.~cit.\ indeed follow from the more general set-up developed here.

We make the conjecture that  the above results hold without Hypothesis~\eqref{eq:dualconserv}.



\section{Preliminaries}
In all this paper, we work in a given universe $\mathcal{U}$. All limits and colimits (in particular, products and direct sums) are assumed to be small.
Recall that $\cor$ is a a commutative unital ring of finite global dimension. 
We assume that  all topological spaces are ``good'', that is, Hausdorff, locally compact, countable at infinity and of finite flabby dimension.

For a topological space $X$ as above, 
one denotes by $\md[\cor_X]$ the Grothendieck abelian category of sheaves of $\cor$-modules and by  $\Derb(\cor_X)$ its bounded derived category. We need a slight modification of the notion of  cohomologically constructible sheaves (see~\cite{KS90}*{Def.~3.4.1}).

We mainly follow the notations of~\cite{KS90}. In particular, 

\begin{itemize}
\item
$\omega_X$ denotes the dualizing complex and $\RD_X$  the duality functor 
$\rhom(\scbul,\omega_X)$,
\item
$\RD$ denotes the duality functor on $\Derb(\cor)$,
\item
 $a_X\cl X\to \{\rmpt\}$ denotes by  the unique map from $X$ to a one-point space. Hence, for 
$F\in\Derb(\cor_X)$, $\rsect(X;F)\simeq\roim{a_X}F$.
\item
For $L\in\Derb(\cor)$, $L_X$ denotes the constant sheaf on $X$ with stalk $L$. 
More generally, 
for $Z$ locally closed in $X$, one denotes by $L_{XZ}$ the  constant sheaf $L_Z$ on $Z$ extended by $0$ on $X\setminus Z$. When $Z$ is closed, we shall simply denote by $L_Z$ the sheaf $L_{XZ}$.
\item
For $x\in X$, denoting by $i_x\cl \{x\}\into X$ the embedding, and for $F\in\Derb(\cor_X)$, one denotes as usual by $F_x=\opb{i_x}F$ its stalk at $x$. One also sets $\rsect_xF=\epb{i_x}F$.
\end{itemize}

\subsubsection*{Ind-objects}
We shall make use of ind-objects. For a short  exposition see~\cite{KS90}*{\S~1.11}. For a more detailed
study, including new results that we shall use here, see~\cite{KS06}*{Ch.~6, \S~8.6, Ch.~15}. Let us recall a few facts that we need, skipping some delicate questions of universes. 

If $\shc$ is a category, one denotes by $\IC$ the category of ind-objects of $\shc$, a full subcategory of 
the category $\shc^\wedge$ of functors from $\shc^\rop$ to $\Set$. 
Recall {\cite{KS06}*{\S~6.1}  
\begin{itemize}
\item
The natural functor $\shc\to\IC$ is fully faithful.
\item
The category $\IC$ admits small filtrant colimits, denoted $\sinddlim$. 
\item
Let $\shi$ be a small and filtrant category and  $\alpha\cl \shi\to\shc$ a functor.
 Let $T\cl\shc\to\shc'$ be a functor. Then 
$T(\sinddlim\alpha)\simeq\sinddlim (T\circ\alpha)$. 
\end{itemize}

Now we assume that $\shc$ is abelian. Recall {\cite{KS06}*{Th.~8.6.5}  that
\begin{itemize}
\item
The category $\IC$ is abelian and the fully faithful  functor $\shc\to\IC$ is exact.
\item
Small filtrant colimits  are exact in $\IC$.
\end{itemize}
One should be aware that even if $\shc$ is a Grothendieck category, $\IC$ does not admit enough injectives in general.

Let $\shi$ be a small category and $\alpha\cl \shi\to \shc$ be a functor. As already mentioned, one denotes by $\sinddlim\alpha$ its colimit in $\IC$.
Note that if $\shc$ admits colimits, denoted $\sindlim$,   there is a natural morphism in $\IC$:
\eqn
&&\sinddlim\alpha\to\sindlim\alpha
\eneqn
but this morphism is not an isomorphism in general. However:
\begin{itemize}
\item
If $\sinddlim\alpha$ belongs to $\shc$, then $\sinddlim\alpha\isoto\sindlim\alpha$
(see~\cite{KS90}*{Cor.~1.11.7}). In this case, if $T\cl\shc\to\shc'$ is a functor, then
$\sinddlim (T\circ\alpha)\isoto T(\sindlim\alpha)$. Therefore, $\sinddlim (T\circ\alpha)$ belongs to $\shc$  and hence 
is isomorphic to $\sindlim (T\circ\alpha)$.
\end{itemize} 

\section{Weakly cohomologically constructible sheaves}

\begin{definition}\label{def:wcc}
Let $F\in\Derb(\cor_X)$. We say that $F$ is  {\em weakly cohomologically constructible} \lp wcc for short\rp\, if for all $x\in X$, one has the isomorphisms
\eqn
&&\inddlim[x\in U]\rsect(U;F)\isoto F_x,\quad \rsect_xF\isoto\proolim[x\in U]\rsect_c(U;F).
\eneqn
We denote by  $\Derb_\wcc(\cor_X)$ the full subcategory of  $\Derb(\cor_X)$ consisting of 
weakly cohomologically constructible sheaves, 
\end{definition}
\begin{remark}
As explained in~\cite{KS90}*{Rem.~4.3.2}, the isomorphisms in Definition~\ref{def:wcc} hold as soon as 
the objects $\inddlim[x\in U]\rsect(U;F)$ and $\proolim[x\in U]\rsect_c(U;F)$ are representable.
\end{remark}

\begin{proposition}\label{pro:tri}
The category $\Derb_\wcc(\cor_X)$ is triangulated. 
\end{proposition}
\begin{proof}
(i) Remark first that for $F\in  \Derb_\wcc(\cor_X)$ and $j\in\Z$, one has
\eqn
&&\inddlim[x\in U]H^j(U;F)\isoto H^j(F)_x.
\eneqn
(ii) Clearly, if $F\in  \Derb_\wcc(\cor_X)$, then so does the shifted sheaf $F\,[j]$ for $j\in\Z$.

\spa
(iii) Consider a distinguished triangle $F'\to F\to F''\to[+1]$ in $\Derb(\cor_X)$ and assume that 
$F',F''\in \Derb_\wcc(\cor_X)$.
We get the morphism of long exact sequence  in the abelian category $\md[\cor]$
\eq\label{eq:lesxU}
&&\ba{l}\xymatrix{
{\cdots}\ar[r]& H^j(U;F')\ar[r]\ar[d]& H^j(U;F)\ar[r]\ar[d]&H^j(U;F'')\ar[r]\ar[d]&H^{j+1}(U;F')\ar[r]\ar[d]& \cdots\\
{\cdots}\ar[r]& H^j(F')_x\ar[r]        & H^j(F)_x\ar[r]        &H^j(F'')_x\ar[r]&H^{j+1}(F')_x\ar[r]       & \cdots.
}\ea
\eneq
Applying the functor $ \inddlim[x\in U]$ and using~\cite{KS06}*{Th.~8.6.5}, the first line gives rise to   the long exact sequence  
in the abelian category $\Ind(\md[\cor])$:
\eq\label{eq:lesind}
&&\cdots\to H^j(F')_x\to \inddlim[x\in U]H^j(U;F)\to H^j(F'')_x\to H^{j+1}(F')_x\to\cdots
\eneq
We shall apply~\cite{KS06}*{Lem.~15.4.6}, following  its notations, to the category $\shc=\md[\cor]$. 
Consider the morphism
\eqn
&&\phi\cl \inddlim[x\in U]\rsect(U;F)\to F_x.
\eneqn
It follows from~\eqref{eq:lesxU} 
that $IH^j(\phi)$ is an isomorphism for all $j\in\Z$ and therefore $\phi$ is an isomorphism 
by loc.\ cit.

\spa
(iv) The proof for $ \rsect_xF$ is the same and we do not repeat it. 
\end{proof}

\begin{proposition}\label{pro:dual}
Let $F\in \Derb_\wcc(\cor_X)$. Then $\RD_XF\in \Derb_\wcc(\cor_X)$. Moreover, one has the isomorphisms 
  $\rsect_x\RD_XF\simeq \RD(F_x)$ and $(\RD_XF)_x\simeq \RD( \rsect_xF)$. 
\end{proposition}
We shall adapt the proof of ~\cite{KS90}*{Prop.~4.3.4~(iii)}. 
\begin{proof}
(i) The first isomorphism holds without any hypothesis, see~\cite{KS90}*{Exe.~viii~3}.

\spa
(ii) Recall first the isomorphism for $U$ open and  $F\in\Derb(\cor_X)$:
\eqn
&&\rsect(U;\RD_XF)\simeq\RHom(\rsect_c(U;F),\cor).
\eneqn
Now assume that $F\in \Derb_\wcc(\cor_X)$.
Applying the functor $\sinddlim[x\in U]$, we get the isomorphisms
\eqn
\sinddlim[x\in U]\rsect(U;\RD_XF)&\simeq&\RHom(\sproolim[x\in U]\rsect_c(U;F),\cor)\\
&\simeq&\RHom( \rsect_xF, \cor)=\RD( \rsect_xF).
\eneqn
This proves   that $\sinddlim[x\in U]\rsect(U;\RD_XF)$ is representable, 
hence isomorphic to $(\RD_XF)_x$. This also proves the second isomorphism.

\spa
(iii) Let $K$ be a compact neighborhood of $x$. One has
\eqn
\rsect_K(X,\RD_XF)&\simeq&\RHom(\cor_{XK},\RD_XF)\\
&\simeq& \RHom(F_K,\omega_X)\simeq \RHom(\rsect(X;F_K),\cor).
\eneqn
Now, denote by $\dot K$ the interior of $K$. We have
\eqn
\proolim[x\in U]\rsect_c(U;\RD_XF)&\simeq&\proolim[x\in \dot K]\rsect_K(X;\RD_XF)\\
&\simeq& \RHom(\sinddlim[x\in\dot K]\rsect(X;F_K),\cor)\simeq  \RHom(\sinddlim[x\in U]\rsect(U;F),\cor)\\
&\simeq&\RD(F_x).
\eneqn
This proves that $\sproolim[x\in U]\rsect_c(U;\RD_XF)$ is representable, 
hence isomorphic to $\rsect_x\RD_XF$. 
\end{proof}

\begin{proposition}\label{pro:homtensL}
Let $F\in\Derb_\wcc(\cor_X)$,  let $L\in\Derb(\cor)$ and let $x\in X$. Then $F\ltens L$ and 
$\rhom(L_X,F)$ belong to  $\Derb_\wcc(\cor_X)$. Moreover one has the isomorphisms
\eqn
&&\rhom(L_X,F)_x\isoto \RHom(L,F_x),\quad (\rsect_xF)\ltens L\isoto \rsect_x(F\tens L_X),
\eneqn
\end{proposition}
\begin{proof}
(i) One has 
\eqn
 \RHom(L,F_x)&\simeq&\RHom(L,\sinddlim[x\in U]\rsect(U;F))\simeq\sinddlim[x\in U]\RHom(L, \rsect(U;F))\\
 &\simeq&\sinddlim[x\in U]\RHom(L_{XU},F)\simeq \sinddlim[x\in U]\rsect(U;\rhom(L_X,F)).
 \eneqn
 This proves that $\sinddlim[x\in U]\rsect(U;\rhom(L_X,F))$ is representable as well as the first isomorphism.
 
 \spa
 (ii) One has 
 \eqn
  (\rsect_xF)\ltens L&\simeq&(\proolim[x\in U]\rsect_c(U;F))\ltens L\simeq \proolim[x\in U](\rsect_c(U;F)\ltens L)\\
  &\simeq&\proolim[x\in U]\rsect_c(U;F\ltens L_X).
 \eneqn
 This proves that $\sproolim[x\in U]\rsect_c(U;F\tens L_X)$ is representable as well as the second isomorphism.
 \end{proof}

All along this paper,  we shall consider the hypothesis
\eq\label{eq:dualconserv}
&&\parbox{75ex}{
For any $M\in\Derb(\cor)$, if $\RD M\simeq0$, then $M\simeq0$.
}
\eneq
Since $\Derb(\cor)$ is triangulated, this is equivalent to saying that the functor $\RD\cl \Derb(\cor)^\rop\to\Derb(\cor)$ is conservative.

\begin{example}
(i) Hypothesis~\eqref{eq:dualconserv} is obviously satisfied if $\cor$ is a field. Indeed, in this case, $M\simeq\bigoplus_jH^j(M)\,[-j]$ and we are reduced to the case where $M$ is a vector space.  The result then follows since the map $M\to\RD\RD M$ is injective. 

\spa
(ii) This property is satisfied when $\cor=\Z$. See~\cite{KS90}*{Exe.~I.31}. 
\end{example}

\begin{proposition}\label{pro:dualconserv}
Assume~\eqref{eq:dualconserv}. Then the functor $\prod_{x\in X}\rsect_x(\scbul)\cl\Derb_{\wcc}(\cor_X)\to\Derb(\cor)$ is conservative. 
\end{proposition}
\begin{proof}
Let $F\in \Derb_{\wcc}(\cor_X)$ and assume that $\rsect_xF\simeq0$ for all $x\in X$. 
By Proposition~\ref{pro:dual}, we get that $\RD_XF\simeq0$. Hence  $\rsect_x\RD_XF\simeq0$ and by the same proposition, we get that $\RD(F_x)\simeq0$. Using the hyptohesis~\eqref{eq:dualconserv}, we get $F\simeq0$. 
\end{proof}



\section{Weakly $\R$-constructible sheaves}

The property of being  weakly cohomologically constructible is not stable by the six operations. That is why we shall consider instead  weakly $\R$-constructible sheaves.
Hence, from now on, all manifolds and morphisms of manifolds will be real analytic.

Recall (see~\cite{KS90}*{Exe.~I.30}) that $M\in\Derb(\cor)$ is {\em perfect} if it is isomorphic to a bounded complex of finitely generated projective $\cor$-modules. If $M$ is perfect, then so is $\RD(M)$ and the morphism $M\to\RD\RD(M)$ is an isomorphism.
We shall denote  by $\Derb_f(\cor)$ the full triangulated category of $\Derb(\cor)$ consisting of perfect objects.


Let $X$ be  a real analytic manifold. As already mentioned,  we denote by $\Derb_\wRc(\cor_X)$  (resp.\ $\Derb_\Rc(\cor_X)$) the full triangulated subcategory of $\Derb(\cor_X)$ consisting of weakly  $\R$-constructible (resp.\ $\R$-constructible) sheaves on $X$. 

Recall that $F\in\Derb(\cor_X)$ belongs to $\Derb_\wRc(\cor_X)$ if and only if its micro-support $\SSi(F)$ is 
contained in a conic subanalytic isotropic subset of $T^*X$ and this is equivalent to the fact that $\SSi(F)$ is a conic subanalytic Lagrangian subset.
 
If $F_1,F_2\in \Derb_\wRc(\cor_X)$, then $F_1\ltens F_2$ and $\rhom(F_1,F_2$ belong to $ \Derb_\wRc(\cor_X)$. Moreover, 
 if $f\cl X\to Y$ is a morphism of real analytic manifolds and $G\in  \Derb_\wRc(\cor_Y)$, then $\opb{f}G$ and $\epb{f}G$  belong to $ \Derb_\wRc(\cor_X)$. If $F\in  \Derb_\wRc(\cor_X)$ and $f$ is proper on $\supp(F)$, then $\reim{f}F\isoto\roim{f}F$ 
 belongs to $ \Derb_\wRc(\cor_Y)$. This follows from~\cite{KS90}*{Prop.~8.4.6}. 
 
 
Finally recall~(\cite{KS90}*{\S~8.4} that $F\in\Derb_\wRc(\cor_X)$ is $\R$-constructible if for any $x\in X$, $F_x$ is perfect.

\begin{lemma}
 Weakly $\R$-constructible sheaves are weakly cohomologically constructible. In other words, 
the category 
$\Derb_\wRc(\cor_X)$ is a full triangulated subcategory of $\Derb_\wcc(\cor_X)$.
\end{lemma}
This result is implicitly proved in~\cite{KS90}*{Pro.~8.4.9}. For the reader's convenience, we repeat the proof (this is basically a slightly more detailed version of the proof of \cite{KS90}*{Lemma~8.4.7}).
\begin{proof}
Let $F\in\Derb_\wRc(\cor_X)$. We want to prove that $F$ is weakly cohomologically constructible, and this is a local problem, so we can assume $X=\R^n$.

Let $x\in X$ and consider the real analytic function $\varphi\cl X\to \R, y\mapsto |y-x|$. Since $F$ is weakly $\R$-constructible, its micro-support $\SS(F)$ is a closed conic subanalytic isotropic subset of $T^*X$ (see \cite{KS90}*{Th.~8.4.2}). We can therefore apply the microlocal Bertini-Sard theorem (see \cite{KS90}*{Prop.~8.3.12}), which shows that there exists $b\in\R$ such that for all $y\in X$ with $0< \varphi(y)< b$ we have $\mathrm{d}\varphi(y)\notin \SS(F)$.

Now, it follows from the microlocal Morse lemma (see \cite{KS90}*{Cor.~5.4.19}) that for any $a\in \R$ with $0<a<b$, the natural morphisms
$$\rsect(B_b(x);F) \to \rsect(B_a(x);F)\to F_x$$
are isomorphisms. (The second one is not directly part of the lemma, but is easily deduced, cf.\ e.g\ \cite{KS90}*{Remark 2.6.9}).

Similarly (using $-\varphi$ instead), we get that for a suitable $b\in \R$ and $0<a<b$ the natural morphisms
$$\rsect_{x}F\to \rsect_{\overline{B_a(x)}}(X;F)\to \rsect_{\overline{B_b(x)}}(X;F)$$
and
$$\rsect_c(B_a(x);F) \to \rsect_c(B_b(x);F)$$
are isomorphisms.

Since the balls $B_a(x)$ make up a fundamental system of open neighborhoods of $x$, we obtain
\eqn
&&\sinddlim[x\in U] \rsect(U;F)\simeq \sinddlim[a\to 0] \rsect(B_a(x);F)\simeq F_x,\\\\
&&\proolim[x\in U] \rsect_c(U;F) \simeq \proolim[a\to 0] \rsect_c(B_a(x);F) \simeq \proolim[a\to 0] \rsect_{\overline{B_a(x)}}(X;F) \simeq \rsect_x F.
\eneqn
This completes the proof.
\end{proof}


\subsection*{Inverse images}

\begin{theorem}\label{th:invim}
Let $f\cl X\to Y$ be a morphism of real analytic manifolds. Let $L_Y,G\in \Derb_\wRc(\cor_Y)$ and assume that $L_Y$ is locally constant. Then 
\banum
\item
$\opb{f}\rhom(L_Y,G)\isoto \rhom(\opb{f}L_Y,\opb{f}G)$.
\item
Assume~\eqref{eq:dualconserv}. Then 
$\epb{f}G\ltens \opb{f}L_Y\isoto \epb{f}(G\ltens L_Y)$.
\eanum
\end{theorem}
\begin{proof}
Since the problem is local on $Y$, we may assume that $L_Y$ is the constant sheaf associated  with for some $L\in\Derb(\cor)$. Hence $\opb{f}L_Y\simeq L_X$. 

\spa
(a) Let $x\in X$, and set $y=f(x)$. Applying Proposition~\ref{pro:homtensL}, one gets
\eqn
\rhom(L_X,\opb{f}G)_x&\simeq&\rhom(L,(\opb{f}G)_x)\\
&\simeq&\rhom(L,G_y)\simeq  \rhom(L_Y,G)_y\simeq (\opb{f}\rhom(L_Y,G))_x.
\eneqn
(b) Remark first that for any sheaf $H$ on $Y$, one has $\rsect_x(\epb{f}H)\simeq  \rsect_yH$. 
Then using Proposition~\ref{pro:homtensL}, one has 
\eqn
\rsect_x( \epb{f}G\ltens L_X)&\simeq&(\rsect_x\epb{f}G)\ltens L\simeq (\rsect_yG)\ltens L\\
\rsect_x\epb{f}(G\ltens L_Y)&\simeq&\rsect_y(G\ltens L_Y)\simeq (\rsect_yG)\ltens L.
\eneqn
Set $A=\epb{f}G\ltens L_X$ and $B=\epb{f}(G\ltens L_Y)$. We have proved that the morphism $A\to B$ induces 
for  all $x\in X$
an isomorphism $\rsect_xA\simeq \rsect_xB$. Then $A\simeq B$ by Proposition~\ref{pro:dualconserv}.
\end{proof}

\subsection*{Tensor product and hom}
We shall make use of the following result, well-known among specialists. However,  we shall give a proof for the reader's convenience.

\begin{lemma}\label{lem:noether}
Let $L,M\in\Derb(\cor)$ and let $N\in\Derb_f(\cor)$. Then 
\eq\label{eq:eq:noether}
&&\RHom(L,M)\ltens N\isoto \RHom(L,M\ltens N).
\eneq
\end{lemma}
\begin{proof}
By Hypothesis, we may represent $N$ with a bounded complex of projective modules of finite rank. 

\spa
(i) Assume first that $N=P$ is concentrated in a single degree. 
If $P$ is  of finite rank, there exists an integer $n$ and an epimorphism $\cor^n\epito P$. If moreover $P$ is projective, then this epimorphism has a retract and we get $\cor^n\simeq P\oplus Q$. This proves the result in this case.

\spa
(ii) Now assume that $N$ is represented by the complex $0\to P^0\to\cdots\to P^m\to0$ with all $P^j$'s projective of finite rank. 
Here we assume for simplicity in the notations that $P^0$ is in degree $0$. 
Assume that the result is proved Let us use the so-called ``stupid truncation''. Denote by $N_0$ the complex $0\to P^0\to\cdots\to P^{m-1}\to0$ and by $u\cl N\to N_0$ the natural morphism. 
We have an exact sequence of complexes $0\to P^m[-m]\to N\to[u] N_0\to 0$ and it follows that the triangle 
$P^m[-m]\to N\to[u] N_0\to[+1]$ is distinguished. Arguing by induction on $m$ the proof is complete.
\end{proof}


Let $X$ and $Y$ be real analytic manifolds. As usual, one denotes by $q_1$ and $q_2$ the projections from $X\times Y$ to $X$ and $Y$, respectively. 
One denotes $\delta\cl X\to X\times X$ the diagonal morphism. One denotes  by $\letens$ the external product
\eqn
&&F\letens G\eqdot \opb{q_1}F\ltens\opb{q_2}G.
\eneqn
Recall~\cite{KS90}*{Prop.~3.4.4} that for $F\in\Derb_\Rc(\cor_X)$ and $G\in\Derb(\cor_Y)$, one has the isomorphism
\eq\label{eq:Dhom}
&&\RD_XF\letens G\isoto \rhom(\opb{q_1}F,\epb{q_2}G).
\eneq
Also note the isomorphism, for $F_1,F_2\in\Derb(\cor_X)$ and $G_1,G_2\in\Derb(\cor_Y)$:
\eq\label{eq:etenstens}
&&(F_1\letens F_2)\ltens(G_1\letens G_2)\simeq (F_1\ltens G_1)\letens(F_2\ltens G_2).
\eneq

\begin{theorem}\label{th:tenshom}
Let $L_X,F_1\in\Derb_\wRc(\cor_X)$, with $L_X$ locally constant and let $F_2\in\Derb_\Rc(\cor_X)$. 
Then 
\banum
\item 
$\rhom(L_X,F_1)\ltens F_2\isoto \rhom(L_X,F_1\ltens F_2)$.
\item
 Assume~\eqref{eq:dualconserv}. Then 
$\rhom(F_2,F_1)\ltens L_X\isoto \rhom(F_2,F_1\ltens L_X)$.
\eanum
\end{theorem}
\begin{proof}
We may assume that $L_X$ is the constant sheaf associated with $L\in\Derb(\cor)$. The fact that 
 $F_1\ltens F_2$ and $\rhom(L_X,F_1\ltens F_2)$ belong to $\Derb_\wRc(\cor_X)$ follows from~\cite{KS90}*{Prop.~8.4.6}.

\spa
(a)  Let $x\in X$. One has
\eqn
(\rhom(L_X,F_1)\ltens F_2)_x&\simeq&\rhom(L_X,F_1)_x\ltens F_{2x}\simeq\RHom(L,F_{1x})\ltens F_{2x}\\
&\simeq&\RHom(L,F_{1x}\ltens F_{2x})\simeq(\rhom(L_X,F_1\ltens F_2))_x.
\eneqn
The second and fourth isomorphisms follow from  Proposition~\ref{pro:homtensL} and 
the third one from Lemma~\ref{lem:noether}.

\spa
(b) One has 
\eqn
\rhom(F_2,F_1)\ltens L_X&\simeq&\epb{\delta}(\RD_XF_2\letens F_1)\ltens \opb{\delta}(\cor_X\letens L_X)
\simeq\epb{\delta}((\RD_XF_2\letens F_1)\ltens(\cor_X\letens L_X))\\
&\simeq&\epb{\delta}((\RD_XF_2\ltens\cor_X)\letens(F_1\ltens L_X))
\simeq\rhom(F_2,F_1\ltens L_X).
\eneqn
Here, the second isomorphism follows from Theorem~\ref{th:invim}~(b). The other ones follow from~\eqref{eq:Dhom} and~\eqref{eq:etenstens}.
\end{proof}


\subsection*{Direct images}
Let $f\cl X\to Y$ be a morphism of real analytic manifolds. Let $F\in \Derb_\wRc(\cor_X)$ and $L_Y\in\Derb(\cor_Y)$ 
being  locally constant. One can ask if the morphism
\eqn
&&\roim{f}F\ltens L_Y\to \roim{f}(F\ltens\opb{f}L_Y)
\eneqn
is an isomorphism. 
The answer is negative in general, even if we require $F$ to be constructible, thanks to an example of~\cite{Ho23}*{Rem.~4.6}.

However, there is a positive answer when considering sheaves {\em constructible  up to infinity}. 
Before proving the result for general direct images, let us establish it in the particular case of open embeddings.

\begin{lemma}\label{lem:opendirim}
Let $j\cl U\into X$ be the open embedding of a subanalytic relatively compact open subset $U$ of $X$. 
Let $F\in \Derb_\wRc(\cor_U)$ and assume that there exists $G\in \Derb_\wRc(\cor_X)$ with $\opb{j}G\simeq F$. 
Let $L_X\in\Derb(\cor_X)$ be locally constant. Then
\banum
\item
$\reim{j}\rhom(\opb{j}L_X,F)\isoto \rhom(L_X,\reim{j}F)$.
\item
Assume~\eqref{eq:dualconserv}. Then  $\roim{j}F\ltens L_X\isoto \roim{j}(F\ltens\opb{j}L_X)$. 
\eanum
\end{lemma}
\begin{proof}
As above, we may assume that $L_X$ is the constant sheaf associated with $L\in\Derb(\cor)$. 
Let $G\in\Derb_\wRc(\cor_X)$ be such that $\opb{j}G\simeq F$.
Then $\roim{j}F\simeq\rsect_UG\simeq  \rhom(\cor_{XU},G)$ and $\reim{j}F\simeq G_U\simeq \cor_{XU}\tens G$.

\spa
(a)  Note that $\opb{j}\rhom(L_X,G)\simeq \rhom(\opb{j}L_X,F)$. Using Theorem~\ref{th:tenshom} (a), we get
\eqn
\reim{j}\rhom(\opb{j}L_X,F)&\simeq&\cor_{XU}\tens\rhom(L_X,G)\simeq\rhom(L_X,G\tens\cor_{XU})\\
&\simeq&\rhom(L_X,\reim{j}F).
\eneqn

\spa
(b) Using Theorem~\ref{th:tenshom}~(b), we have
\eqn
\roim{j}F\ltens L_X&\simeq& \rhom(\cor_{XU},G)\ltens L_X\simeq \rhom(\cor_{XU},G\ltens L_X)\\
&\simeq& \roim{j}\opb{j}(G\tens L_X)   \simeq\roim{j}(F\ltens\opb{j} L_X).
\eneqn
\end{proof}

Recall the following notions extracted  from~\cite{Sc23}.
\begin{definition}
A b-analytic manifold  $\fX$  is a  pair $(X,\bX)$ with $X\subset \bX$ an open embedding of real analytic manifolds such that $X$ is relatively compact and subanalytic in $\bX$.

A morphism $f\cl \fX=(X,\bX) \to \fY=(Y,\bY)$ of b-analytic manifolds is a morphism of real analytic manifolds  $f\colon X\to Y$ such that
the graph $\Gamma_f$ of $f$ in $X\times Y$    is subanalytic in $\bX\times\bY$.
\end{definition}
Let $F\in\Derb_\wRc(\cor_X)$. One says that $F$ is ``weakly constructible up to infinity'' or simply weakly b-constructible, if 
$\eim{j_X}F$ (or, equivalently, $\roim{j}F$) belongs to $\Derb_\wRc(\cor_\bX)$. One denotes by $\Derb_\wRc(\cor_\fX)$ the full triangulated subcategory of 
$\Derb_\wRc(\cor_X)$ consisting of weakly b-constructible sheaves.

%Similarly, we shall say that $F\in\Derb_\wRc(\cor_X)$ is locally constant up to infinity if it is the restriction to $X$ of a locally constant sheaf defined in a neighborhood of $\ol X$ in $\bX$. 
%\begin{remark}
%There exist locally constant sheaves which are not locally constant up to infinity. For example, let $D\subset\C$ denote the open disc and choose $X=D\setminus\{0\}$, $\bX=\C$, $\fX=(X,\bX)$. 
% If $L$ is locally constant and not constant on $X$, it will not extend to a locally constant sheaf on $\C$.
%\end{remark}



\begin{theorem}\label{th:dirim}
Let $f\cl\fX\to\fY$ be a morphism of b-analytic manifolds. Let $F\in\Derb_\wRc(\cor_\fX)$ and let $L_Y\in\Derb(\cor_Y)$ be locally constant. Then $\reim{f}F$ and $\roim{f}F$ belong to $\Derb_\wRc(\cor_\fX)$ and 
\banum
\item
$\reim{f}\rhom(\opb{f}L_Y,F)\isoto \rhom(L_Y,\reim{f}F)$.
\item
Assume~\eqref{eq:dualconserv}. Then  $(\roim{f}F)\ltens L_Y\isoto \roim{f}(F\ltens\opb{f}L_Y)$.
\eanum
\end{theorem}
\begin{proof}
Here again we may assume that $L_Y$ is the constant sheaf associated with some $L\in\Derb(\cor)$. 
Set for short $Z= \bX\times \bY$ and denote by $q_1$ and $q_2$ the first and second projection from $Z$ to $\bX$ and $\bY$. Denote by $\Gamma_f\subset Z$ the graph of $f$. Note that, $\Gamma_f$ is subanalytic in $Z$ by definition, and relatively compact in $Z$ since it is contained in the relatively compact subset $X\times Y$.

\spa
One has
\eqn
\reim{f}F&\simeq&\opb{j_Y}\reim{q_2}(\opb{q_1}\eim{j_X}F\ltens\cor_{\Gamma_f}),\\
\roim{f}F&\simeq&\epb{j_Y}\roim{q_2}\rhom(\cor_{\Gamma_f},\epb{q_1}\roim{j_X}F).
\eneqn
Note that the supports of $\opb{q_1}\eim{j_X}F\tens\cor_{\Gamma_f}$ and $\rhom(\cor_{\Gamma_f}, \epb{q_1}\roim{j_X}F)$
are contained in $\Gamma_f$ and hence compact in $Z$. 

\spa
(a) Let us apply the functor  $\rhom(L_Y,\scbul)$ to the first isomorphism. We get
\eqn
\rhom(L_Y,\reim{f}F)&\simeq& \rhom(\opb{j_Y}L_\bY,\opb{j_Y}\roim{q_2}(\opb{q_1}\eim{j_X}F\ltens\cor_{\Gamma_f}))\\
&\simeq& \opb{j_Y}\rhom(L_\bY,\roim{q_2}(\opb{q_1}\eim{j_X}F\ltens\cor_{\Gamma_f}))\\
&\simeq& \opb{j_Y}\roim{q_2}\rhom(L_{\bX\times\bY},\opb{q_1}\eim{j_X}F\ltens\cor_{\Gamma_f})\\
&\simeq& \opb{j_Y}\roim{q_2}(\rhom(\opb{q_1}L_\bX,\opb{q_1}\eim{j_X}F)\ltens\cor_{\Gamma_f})\\
&\simeq& \opb{j_Y}\roim{q_2}(\opb{q_1}\rhom(L_\bX,\eim{j_X}F)\ltens\cor_{\Gamma_f})\\
&\simeq& \opb{j_Y}\roim{q_2}(\opb{q_1}\eim{j_X}\rhom(L_X,F)\ltens\cor_{\Gamma_f})\\
&\simeq& \reim{f} \rhom(f^{-1}L_Y,F).
\eneqn
The second and fifth isomorphism use Theorem~\ref{th:invim} (a), the fourht uses Theorem~\ref{th:tenshom} (a), and the sixth isomorphism uses Lemma~\ref{lem:opendirim} (a). On the other hand, the third isomorphims is classical (see \cite{KS90}*{(2.6.15)}).

\spa
(b) The proof is completely analogous, using parts (b) of the statements mentioned above instead.
\end{proof}
Recall the isomorphisms which hold for any $F\in \Derb(\cor_X)$ and  $L\in\Derb(\cor)$. One has
\eq\label{eq:secthom}
&&\ba{l}\rsect_c(X;F\ltens L_X))\isoto \rsect_c(X;F)\ltens L,\\
\rsect(X;\rhom(L_X,F))\simeq \RHom(L,\rsect(X;F)).
\ea\eneq


\begin{corollary}
Let $F\in\Derb_\wRc(\cor_X)$ and let $L\in\Derb(\cor)$. Let $U$ be an open relatively compact subanalytic subset of $X$. Then 
\banum
\item
$\rsect_c(U;\rhom(L_X,F))\isoto \RHom(L,\rsect_c(U;F))$.
\item
Assume~\eqref{eq:dualconserv}. Then  $\rsect(U,F\ltens L_X)\simeq \rsect(U;F)\ltens L$.
\eanum
\end{corollary}
\begin{proof}
Apply Theorem~\ref{th:dirim} to the sheaf $F|_U$ with $X_\infty=(U,X)$, $Y=\rmpt$ and $f=a_U$. 
\end{proof}

Note that this corollary applies in particular when $X_\infty=(X,\bX)$ is b-analytic and $F\in\Derb_\wRc(\cor_\fX)$. In this case, one gets
\eqn
&&\rsect_c(X;\rhom(L_X,F))\isoto \RHom(L,\rsect_c(X;F)),\\
&&\rsect(X,F\ltens L_X)\simeq \rsect(X;F)\ltens L.
\eneqn

\subsection*{Duality}
From our above result, we obtain slight generalisations of \cite{KS90}*{Exe.~VIII.3}.

Recall that for any $G\in\Derb(\cor_Y)$, one has $\RD_X\opb{f}G\simeq\epb{f}\RD_YG$. 

\begin{corollary}
Let $f\cl X\to Y$ be a morphism of real analytic manifolds. Let $G\in \Derb_\Rc(\cor_Y)$ and assume that $L_Y\in\Derb(\cor_Y)$ is locally constant. 
Then
\eqn
&&\RD_X\epb{f}(G\ltens L_Y)\simeq \opb{f}\RD_Y(G\ltens L_Y).
\eneqn
\end{corollary}
\begin{proof}
One has the chain of isomorphisms
\eqn
\RD_X\epb{f}(G\ltens L_Y)&\simeq&\rhom(\epb{f}G\ltens \opb{f}L_Y,\omega_X)\\
&\simeq&\rhom(\opb{f}L_Y, \RD_X(\epb{f}G))\simeq \rhom(\opb{f}L_Y, \opb{f}\RD_YG)\\
&\simeq&\opb{f}\rhom(L_Y, \RD_YG)\simeq \opb{f}\RD_Y(G\ltens L_Y).
\eneqn
\end{proof}

Recall that for any $F\in\Derb(\cor_X)$, one has $\RD_Y\reim{f}F\simeq\roim{f}\RD_XF$. 

\begin{corollary}
Let $f\cl\fX\to\fY$ be a morphism of b-analytic manifolds. Let $F\in\Derb_\Rc(\cor_\fX)$ and let $L_Y\in\Derb(\cor_Y)$ be locally constant. Then
\eqn
&&\reim{f}\RD_X(F\ltens\opb{f}L_Y)\simeq \RD_Y(\roim{f}F\ltens L_Y).
\eneqn
\end{corollary}
\begin{proof}
One has the chain of isomorphisms
\eqn
\reim{f}\RD_X(F\ltens\opb{f}L_Y)&\simeq &\reim{f}\rhom(\opb{f}L_Y,\RD_XF)\\
&\simeq& \rhom(L_Y,\reim{f}\RD_XF)\simeq \rhom(L_Y,\RD_Y\roim{f}F)\\
&\simeq &\RD_Y(\roim{f}F\ltens L_Y).
\eneqn
\end{proof}


\subsection*{Micro-support}
\begin{remark}
Consider a field extension $\corex$ of the field $\cor$. Then of course all preceding results apply with $L=\corex_X$ (the case of interest in \cite{Ho23}). Moreover, note that \cite{KS90}*{Rem.~5.1.5} asserts that if 
$\for$ denotes the forgetful functor
\eqn
&&\for\cl\Derb(\corex_X)\to\Derb(\cor_X),
\eneqn
and if $F\in\Derb(\corex_X)$, then the micro-support of $F$ and that of $\for(F)$ are the same. 
\end{remark}

To conclude, let us consider the action of the functors $L\tens\scbul$ and $\rhom(L,\scbul)$ on the micro-support.
Remark first that for $F,L_X\in\Derb(\cor_X)$ with $L_X$  locally constant, one has
\eq\label{eq:SSiL}
&&\SSi(L_X\ltens F)\subset\SSi(F),\quad \SSi(\rhom(L_X,F))\subset\SSi(F).
\eneq
Indeed, one has by~\cite{KS90}*{Prop.~5.4.14}
\eqn
&&\SSi(L_X\ltens F)\subset T^*_XX+\SSi(F)=\SSi(F),
\eneqn
and similarly with $\rhom(L_X,F)$. 

\begin{proposition}
Assume that $\cor$ is a field.  
Let $F,L_X\in\Derb_\wRc(\cor_X)$ with $L_X$  locally constant, $L_X\neq0$. Then
\eqn
&&\SSi(L_X\tens F)=\SSi(F),\quad  \SSi(\rhom(L_X,F))=\SSi(F).
\eneqn
\end{proposition}
\begin{proof} 
The problem is local and we may assume that $L_X$ is the constant sheaf associated with $L\in\Derb(\cor)$. Then $L=\oplus_j H^j(L)\,[-j]$ and we may assume that $L\in\md[\cor]$. In this case, there exists $K\in\md[\cor]$ such that $L\simeq\cor\oplus K$, hence
$L_X\simeq \cor_X\oplus K_X$ and the result follows (see e.g.\ \cite{KS90}*{Prop.~5.1.3}). 
\end{proof}

\providecommand{\bysame}{\stLeavevmode\hbox to3em{\hrulefill}\thinspace}
\begin{thebibliography}{15}


\bib{Ho23}{article}{
author={Hohl, Andreas},
title={Field extensions and Galois descent for sheaves of vector spaces},
eprint={arxiv:2302.14837v1},
year={2023},
pages = {},
volume={},
}


\bib{KS90}{book}{
author={Kashiwara, Masaki},
author={Schapira, Pierre},
title={Sheaves on Manifolds},
series={Grundlehren der Mathematischen Wissenschaften},
volume={292},
publisher={Springer-Verlag, Berlin},
date={1990},
pages={x+512},
}

\bib{KS06}{book}{
author={Kashiwara, Masaki},
author={Schapira, Pierre},
title={Categories and Sheaves},
series={Grundlehren der Mathematischen Wissenschaften},
volume={332},
publisher={Springer-Verlag, Berlin},
date={2006},
}

\bib{Sc23}{article}{
author= {Schapira, Pierre},
title={Constructible sheaves and functions up to infinity},
journal={Journal of Applied and Computational Topology},
eprint={arXiv:2012.09652},
date={2023, to appear},
pages = {},
volume={},
}


\end{thebibliography}

\vspace*{1cm}
\noindent
\begin{tabular}{cc}
\parbox[t]{16em}
{\scriptsize{
		Andreas~Hohl\\
		Universit{\'e} Paris Cit{\'e} et Sorbonne Universit{\'e},\\ CNRS, IMJ-PRG, F-75013 Paris, France\\ 
		e-mail: andreas.hohl@imj-prg.fr\\
		https://www.andreashohl.eu}}
		&
\parbox[t]{14em}{\scriptsize{
		Pierre Schapira\\
		Sorbonne Universit{\'e}, CNRS IMJ-PRG\\
		e-mail: pierre.schapira@imj-prg.fr\\
		http://webusers.imj-prg.fr/\textasciitilde pierre.schapira/
}}
\end{tabular}

%
%\vspace*{1cm}
%\noindent
%\parbox[t]{21em}
%{\scriptsize{
%Pierre Schapira\\
%Sorbonne Universit{\'e}, CNRS IMJ-PRG\\
%4 place Jussieu, 75252 Paris Cedex 05 France\\
%e-mail: pierre.schapira@imj-prg.fr\\
%http://webusers.imj-prg.fr/\textasciitilde pierre-schapira/
}}

\end{document}