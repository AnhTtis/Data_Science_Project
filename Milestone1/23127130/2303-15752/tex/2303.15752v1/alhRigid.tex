\documentclass[11pt]{amsart}

\usepackage{amsmath, amsthm, amssymb}
\usepackage{mathabx}
\usepackage[hidelinks]{hyperref}
\usepackage{mathrsfs}
\usepackage{xcolor, changepage} 
\usepackage{graphicx}
\usepackage{titletoc}
\usepackage{enumerate}
\usepackage{accents}

\usepackage{caption}
\usepackage{subcaption}

\usepackage{geometry}
\geometry{
% a4paper,
% total={210mm,297mm},
left=36mm,
 right=36mm,
 top=32mm,
 bottom=32mm,
 }
 


%%%%% Theorems, etc. numbering

\newtheorem{theorem}{Theorem}[section]
\newtheorem{lemma}[theorem]{Lemma}
\newtheorem{fact}[theorem]{Fact}
\newtheorem{cor}[theorem]{Corollary}
\newtheorem{prop}[theorem]{Proposition}
\newtheorem{state}[theorem]{Statement}
\newtheorem{claim}[theorem]{Claim}
\newtheorem{remark}[theorem]{Remark}
\newtheorem{que}[theorem]{Question}
\newtheorem{prob}[theorem]{Problem}
\newtheorem{example}[theorem]{Example}
\newtheorem{assumption}[theorem]{Assumption}
\newtheorem{definition}[theorem]{Definition}
\numberwithin{equation}{section}



%%%%% Shortcuts to symbols
\def\II{{\rm II}}
\def\tr{{\rm tr}}

\def\A{\mathbb{A}}
\def\R{\mathbb{R}}
\def\B{\mathbb{B}}
\def\S{\mathbb{S}}
\def\H{\mathbb{H}}
\def\T{\mathbb{T}}
\def\Z{\mathbb{Z}}
\def\C{\mathbb{C}}
\def\D{\mathbb{D}}

\def\Bb{\mathbf{B}}

\def\bs{\boldsymbol}

\def\zerob{\mathbf{0}}
%\def\zerob{O}

\def\RP{\mathbb{RP}}
\def\refOmega{\Omega_0}

\def\Ucal{\mathcal{U}}
\def\Lcal{\mathcal{L}}
\def\Hcal{\mathcal{H}}
\def\Acal{\mathcal{A}}
\def\Ccal{\mathcal{C}}
\def\Dcal{\mathcal{D}}
\def\Pcal{\mathcal{P}}
\def\Ical{\mathcal{I}}
\def\Bcal{\mathcal{B}}
\def\Scal{\mathcal{S}}
\def\Ecal{\mathcal{E}}
\def\Tcal{\mathcal{T}}
\def\Zcal{\mathcal{Z}}
\def\Ncal{\mathcal{N}}
\def\Qcal{\mathcal{Q}}
\def\Ocal{\mathcal{O}}
\def\Xcal{\mathcal{X}}
\def\Zcal{\mathcal{Z}}
\def\Vcal{\mathcal{V}}
\def\Kcal{\mathcal{K}}

\def\Crm{{\rm C}}

\def\Rscr{\mathscr{R}}
\def\Bscr{\mathscr{B}}

\def\nb{{\boldsymbol{n}}}
\def\vb{{\boldsymbol{v}}}
\def\pb{{\boldsymbol{p}}}
\def\ub{{\boldsymbol{u}}}
\def\xb{{\boldsymbol{x}}}

\def\supp{{\rm supp}\,}

\def\ed{{\rm d}}
\def\std{{\rm std}}
\def\Ric{{\rm Ric}}
\def\area{{\rm area}}
\def\length{{\rm length}}
\def\Hess{{\rm Hess}}
\def\vol{{\rm vol}}
\def\dvol{{\;d\rm vol}}
\def\id{{\rm id}}
\def\Rm{{\rm Rm}}
\def\tr{{\rm tr}}
\def\div{{\rm div}}
\def\sec{{\rm sec}}
\def\width{{\rm width}}
\def\Lip{{\rm Lip}}
\def\dist{{\rm dist}}
\def\pt{{\rm pt}}
\def\inj{{\rm inj}}
\def\exp{{\rm exp}}
\def\rank{{\rm rank\;}}
\def\const{{\rm const}}
\def\sys{{\rm sys}}
\def\loc{{\rm loc}}
\def\Rm{{\rm Rm}}
\def\tilRm{\widetilde\Rm}

\def\nub{\boldsymbol{\nu}}
\def\cmc{\mu}

\def\<{\langle}
\def\>{\rangle}

\def\hemiN{\S^n_+}
\def\hemiT{\S^3_+}

\def\setdiff{\backslash}

\def\Nsep{\textbf{NSep$^+$}}

\def\editmark{{\color{purple}==============}}

%%%%% Colors
\def\red{\color{red}}
\def\green{\color{green}}
\def\blue{\color{blue}}


%%%%% Title etc.
\title[Rigidity and non-rigidity of $\H^n/\Z^{n-2}$]
{Rigidity and non-rigidity of  $\H^n/\Z^{n-2}$ 
with scalar curvature bounded from below}

\author{Tianze Hao}
\address{Key Laboratory of Pure and Applied Mathematics, 
School of Mathematical Sciences, Peking University, Beijing, 100871, P. R. China
}
\email{haotz@pku.edu.cn}

\author{Yuhao Hu}
\address{Key Laboratory of Pure and Applied Mathematics, 
School of Mathematical Sciences, Peking University, Beijing, 100871, P. R. China
}
\email{yuhao.hu@math.pku.edu.cn}

\author{Peng Liu}
\address{Key Laboratory of Pure and Applied Mathematics, 
School of Mathematical Sciences, Peking University, Beijing, 100871, P. R. China
}
\email{1801110011@pku.edu.cn}

\author{Yuguang Shi}
\address{Key Laboratory of Pure and Applied Mathematics, 
School of Mathematical Sciences, Peking University, Beijing, 100871, P. R. China
}
\email{ygshi@math.pku.edu.cn}


%%%%%%%%%%%%%%%%
%%%%%%%%%%%%%%%%
%%%%%%%%%%%%%%%%
\begin{document}
\maketitle

\begin{abstract}
We show that the hyperbolic manifold $\H^n/\Z^{n-2}$ is 
not rigid under all compactly supported deformations that 
preserve the scalar curvature lower bound $-n(n-1)$, and
that it is rigid under deformations that are further constrained by certain topological conditions.
In addition, we prove two related splitting results.
\end{abstract}

\setcounter{tocdepth}{1}
\tableofcontents

%Intro
\section{Introduction}

In \cite[Section 3]{Gr2019} and \cite[p.240]{Gr2021}, Gromov stated the following generalization
of Min-Oo's hyperbolic rigidity theorem \cite{MinOo89}.
\vskip 2mm
\begin{state}\label{generminoo}
{\rm (``Generalised Min-Oo Rigidity Theorem'')}
Parabolic quotients $Z=\mathbb{H}^n/\Gamma$ of the hyperbolic $n$-space admit no non-trivial, compactly supported `deformation' with scalar curvature $R\geq -n(n-1)$.
\end{state}
According to \cite{Gr2019}, a \emph{deformation} can change not only the metric, but also the topology of a compact region in $Z$. 
If one only considers deformations that are topologically a connected sum with 
a closed $n$-manifold, 
Statement~\ref{generminoo} is known to be true for (at least) $Z = \H^n/\Z^{n-1}$,
with idea of proof already outlined by \cite[Section $5\frac{5}{6}$]{Gr93} (for a detailed treatment, see
also  \cite[Theorem 1.1]{ACG08}). 
The situation 
turns out to be more subtle
if broader types of deformations are considered, allowing, for example, surgeries along 
an embedded, non-contractible loop.
In this latter case we construct a counterexample to Statement~\ref{generminoo}, which, more precisely, 
demonstrates the following.

\begin{theorem}\label{counterexample}
Let $\mathbb{H}^n/\mathbb{Z}^{n-2}$ be equipped with the standard hyperbolic metric. 
There exists a complete Riemannian manifold $(M^n,g)$, not (globally) hyperbolic, and
compact subsets $K\subset M$ and $K'\subset\H^n/\Z^{n-2}$, such that (1) $R_g\ge -n(n-1)$ and (2) $M\setminus K$ is isometric to $(\mathbb{H}^n/\mathbb{Z}^{n-2})\setminus K'$.	
\end{theorem}
\begin{remark}
\begin{enumerate}[\rm (1)]
	\item  
	 The statement of Theorem~\ref{counterexample} still holds if one replaces $\H^n/\Z^{n-2}$ by
	 $\H^n/\Z^{n-1}$ (see Section~\ref{CErmksurg}).

   	 \item En route to proving Theorem \ref{counterexample}, we obtain counterexamples 
    (see Proposition \ref{XrbarInfo} below) to the following statement in \cite[p.12]{Gr2019}: \emph{Represent $\H^n/\Z^{n-2}$ as a warped product $\H^2\times \T^{n-2}$, and let $X = \D^2 \times \T^{n-2}$ for a 2-disk $\D^2\subset\H^2$; then no Riemannian manifold $(M^n,g)$ with boundary isometric to $\partial X$ can have scalar curvature $R_g\ge -n(n-1)$ and mean curvature\footnote{Unless specified otherwise, in this article the mean curvature along a boundary will always be computed with respect to the \emph{outward} unit normal.} of $\partial M$ greater than that of $\partial X$.}
\end{enumerate}
\end{remark}

From the perspective of our construction, the non-rigidity of $\H^n/\Z^{n-2}$ seems closely related to the fact: 
\emph{A deformation supported in a compact subset $K$ can 
`break' the incompressibility\,\footnote{A continuous map $f: X\rightarrow Y$ between topological spaces
is said to be \emph{incompressible} if the induced map $f_*: \pi_1(X)\rightarrow \pi_1(Y)$ is injective; 
when $f$ is an inclusion, we say \emph{`$X$ is incompressible in $Y$'}.} of some submanifold that is disjoint from $K$.}
On the other hand, rigidity does hold if one only considers deformations
that preserve such incompressibility, as the next theorem shows (cf. \cite[Theorem 1.8]{CLSZ2021}).


\begin{theorem}\label{deform1}
For $3\leq n\leq 7$, let $(M^n,g)$ be a complete Riemannian manifold\,\footnote{In this article,
all manifolds are assumed to be orientable, and all hypersurfaces $2$-sided.}
with  scalar curvature $R_g\geq-n(n-1)$. Suppose that there exist compact subsets $K\subset M$, $K'\subset \H^n/\Z^{n-2}$, and an isometry $f: M\setminus K\rightarrow (\mathbb{H}^n/\mathbb{Z}^{n-2})\setminus K'$. Representing $\H^{n}/\Z^{n-2}$ topologically as $\R^2_+\times \T^{n-2}$, let $p\in \R^2_+$ be such that
$T = \{p\}\times \T^{n-2}$ is disjoint from $K'$, and suppose that the map $f^{-1}|_T: T\rightarrow M$ is incompressible. 
Then $(M,g)$ is isometric to $\mathbb{H}^n/\mathbb{Z}^{n-2}$.
\end{theorem}

Technically, we will derive Theorem~\ref{deform1} 
as a consequence of Theorem~\ref{alhpmt4} below. 
The statement of the latter relies on a gluing construction, which we now describe.
\vskip 2mm

\emph{Gluing construction:} Let $N^n$ be a smooth manifold, and
suppose that $\phi: \T^k\rightarrow N$ $(1\le k\le n-2)$ is an embedding with trivial normal bundle.
Moreover, write
$\H^n/\Z^{n-1}$ (topologically) as the product $\R\times \T^{n-k-1}\times \T^k$,
and define
\[
	\psi: \T^k\rightarrow \R\times \T^{n-k-1}\times \T^k  \cong \H^n/\Z^{n-1}
	\quad
	\mbox{by}
	\quad \psi(p) = (t,q,p)
\]
for some fixed $t\in \R$ and $q\in \T^{n-k-1}$. 
By removing tubular neighborhoods of $\phi(\T^k)\subset N$ and $\psi(\T^k)\subset \H^{n}/\Z^{n-1}$
and then identifying the respective boundaries in the obvious way, we obtain
a manifold $M$. For brevity, $M$ will be referred to as obtained by \emph{gluing
$N$ and $\H^n/\Z^{n-1}$ along $\T^k$ via $(\phi,\psi)$.} In particular, for $c$ sufficiently large,
$(c,\infty)\times \T^{n-1}\subset \H^n/\Z^{n-1}$ remains an `end' of $M$,
and this end is denoted by $\Ecal$.


\begin{theorem}\label{alhpmt4}
For $3\leq n\leq 7$, let $N^n$ be a smooth manifold that is either closed or non-compact without boundary, and let 
$M^n$ be obtained by gluing $N$ with $\H^n/\Z^{n-1}$ along $\T^{k}$ via $(\phi,\psi)$  (see description above).
Suppose that 
\begin{enumerate}[\quad\rm (a)]
\item the map $\phi: \T^k\rightarrow N$ is incompressible; \label{incprsAssu}
\item $g$ is a complete Riemannian metric on $M$ with $R_g\geq -n(n-1)$;\label{scalarbdAssu}
\item  $(\Ecal, g)$ is ALH (see Definition~\ref{alh});	
	\label{ALHAssu}
\end{enumerate}
then $\bar m_{\Ecal,g}\ge 0$ (see Definition~\ref{massDef}). In addition, suppose that
\begin{enumerate}[\quad \rm (a)]\setcounter{enumi}{3}
\item the curvature tensor of $(M,g)$ and its first covariant derivatives are bounded;\label{curvbdAssu}
\item there exists some $\alpha>0$ such that $R_g \leq -\alpha$ outside a compact set;
	\label{RbdAssu}
\end{enumerate}
then $\kappa = 0$ only if $(M,g)$ is Einstein.
\end{theorem}


Readers familiar with positive mass theorems may
have noticed that the second half of Theorem~\ref{alhpmt4}
is not in an ideal form; in other words, one wants to know whether
the vanishing of $\bar m_{\Ecal,g}$, and not just $\kappa$, implies that $(M,g)$ is isometric to $\H^n/\Z^{n-1}$, even without the assumptions (d,e). In our proof, these assumptions play a role in making sure that the 
normalized Ricci flow (NRF) starting at $g$ has desired properties (see Lemma~\ref{nrfdeform}); 
on the other hand, 
it seems subtle to prove
hyperbolicity from $(M,g)$ being Einstein and the assumed ALH decay rate. Thus we decide to leave the stronger statement for future investigation.

Theorem~\ref{alhpmt4} has the following corollary.


\begin{cor}\label{hpbInfCor}
For $3\leq n\leq 7$, let $N^n$ be a closed manifold, and suppose that 
$M^n$ is obtained by gluing $N$ with $\H^n/\Z^{n-1}$ along $\T^{k}$ via $(\phi,\psi)$.
Suppose that $g$ is a complete metric on $M$ such that $(M,g)$ is isometric to the hyperbolic manifold $\H^{n}/\Z^{n-1}$ outside a 
 compact set\,\footnote{That is, there exists an isometry $f:M\setminus K\rightarrow(\H^n/\Z^{n-1})\setminus K'$ for some compact
 sets $K\subset M$ and $K'\subset \H^n/\Z^{n-1}$.}, and suppose that
 \begin{enumerate}[\quad\rm (a)]
\item the map $\phi: \T^k\rightarrow N$ is incompressible;
\item $R_g\geq -n(n-1)$.
\end{enumerate}
Then $(M,g)$ is isometric to $\H^n/\Z^{n-1}$.
\end{cor}

In fact, Corollary~\ref{hpbInfCor} remains true if $N$ is allowed to be non-compact,
which can be deduced as a corollary of Theorem~\ref{alhpmt1} below (see Remark~\ref{hpbInfCor_NonCptRmk}).
\vskip 2mm

Besides rigidity problems modeled on complete manifolds, it is often natural to consider similar problems for manifolds with boundary and scalar/mean curvature bounds. 
In this regard, we present a splitting result of `cuspidal-boundary' type
(see \cite[Sec. 4, last paragraph]{Gr2019}). Our proof relies on an approximation scheme developed in \cite{Zhu2020} involving
$\mu$-bubbles.



\begin{theorem}\label{alhpmt2}
Let $(M^4, g)$ be a complete, non-compact Riemannian $4$-manifold with 
compact, connected boundary $\partial M$.
Suppose that $\pi_2(M) = \pi_3(M) = 0$ and that the scalar curvature $R_g\geq -12$. Then 
$$
\inf_{\partial M}H \leq 3,
$$
where $H$ is the mean curvature of $\partial M$. 
Moreover, if 
$$
\inf_{\partial M}H=3,
$$
then $(M, g)$ is isometric to $((-\infty, 0]\times \Sigma, \ed t^2+e^{2t}g_0)$ where $t$ is the coordinate on $(-\infty, 0]$ and $(\Sigma, g_0)$ is a closed 
$3$-manifold with a flat metric.
\end{theorem}

\begin{remark}
	Theorem~\ref{alhpmt2} would fail if one allows $M$ to be compact. Indeed, 
	take
\[
M=\S^1\times\B^3, \qquad 
g=\cosh^2\rho\, \ed\theta^2 +\ed\rho^2 +\sinh^2\rho\, g_{\S^2}\quad (\rho\leq \rho_0)
\]
where $\theta\in \S^1$, $\rho$ is the radial coordinate on $\B^3$, and $g_{\S^2}$ is the standard round metric on $\S^{2}$. In this example, $M$ has contractible universal cover, so both 
its $\pi_2$ and $\pi_3$ vanish. Moreover, since $g$ is hyperbolic, $R_g = -12$,
but the mean curvature $H_{\partial M} = 2\coth\rho_0+\tanh\rho_0>3$.

Counterexamples also exist if one drops the assumption on $\pi_2(M)$ and $\pi_3(M)$. In fact, let us take the manifold
$(\bs{M}', \bs{g}')$ in Section~\ref{CErmksurg} and then, for sufficiently small $z_0>0$, remove the subset $\{0<z<z_0\}$ from $\bs{M}'$; the 
result is a manifold $\bs{M}''$ with \[ \pi_2(\bs{M}'')\ne 0, \quad H_{\partial\bs{M}''} = 3\quad \mbox{and} \quad R \ge -12.\] 
Clearly, $\bs{M}''\not\cong [c,\infty)\times \partial{\bs{M}''} \cong [c,\infty)\times \T^3$.
\end{remark}


Finally, we present an analogue of Theorem \ref{alhpmt2} in more general dimensions.
\begin{definition}\label{cdegdef} {\rm (Cf. \cite{CLSZ2021})}
We say that a closed, connected manifold $\Sigma$ \emph{belongs to the class $\mathcal{C}_{deg}$}, if
\begin{itemize}
	\item $\Sigma$ is aspherical\,\footnote{A closed, connected manifold is said to be \emph{aspherical} if 
it has contractible universal cover.}, and
	\item any compact manifold $\Sigma'$ that admits a map to $\Sigma$ of nonzero degree
			cannot be endowed with a PSC metric (i.e., metric with positive scalar curvature).
\end{itemize}
\end{definition}
It is well known that $\T^n\in \Ccal_{deg}$ for $n\le 7$; also note that the second item
in Definition~\ref{cdegdef} is redundant when $\dim \Sigma \le 5$, 
according to \cite{CLL2021}.

\begin{theorem}\label{alhpmt1}
For $3\leq n\leq 7$, let $(M^n, g)$ be a complete and non-compact Riemannian manifold with compact, connected boundary $\partial M$. Suppose that 
\begin{enumerate}[\quad \rm (a)]
\item $\partial M$ is incompressible in M;
\item $\partial M\in\Ccal_{deg}$; 
\item  $R_g\geq -n(n-1)$;
\end{enumerate} then 
\begin{equation}\label{meancurvineq1}
\inf_{\partial M}H \leq n-1,
\end{equation}
where $H$ is the the mean curvature of $\partial M$. 

Moreover, if 
$$
\inf_{\partial M}H = n-1,
$$
then $(M, g)$ is isometric to $((-\infty, 0]\times \Sigma, \ed t^2+e^{2t}g_0)$ where $t$ is the coordinate on $(-\infty, 0]$ and $(\Sigma, g_0)$ is a closed 
$(n-1)$-manifold with a flat metric.
\end{theorem}


\noindent\textbf{Additional notes on the literature.} 
\textbf{a.} All our main theorems are fundamentally related to Gromov's fill-in problems 
	(e.g., \cite[Problems A, B]{Gr2019}; \cite[p.234, Question (c)]{Gr2021}).
\textbf{b.} Theorem \ref{alhpmt1} can be viewed as a generalization of \cite[Theorem 3.2]{Yau2001}.
\textbf{c.} It is a classical theme to relate incompressibility conditions with scalar curvature (see \cite[Section 11]{GL1983}.
\textbf{d.} To adapt to modern language, our Theorem~\ref{alhpmt4} considers manifolds with a prescribed end
and some `arbitrary ends'; the study of positive-mass type theorems on such manifolds has generated 
considerable interest recently (see, for example, \cite{CZ21pmt}, \cite{CL20}, \cite{LUY21}, \cite{Zhu22PMT}).
\textbf{e.} While in this paper we focus on rigidity results for complete, non-compact
manifolds with boundary and scalar curvature lower bounds, similar results in the
compact case (with boundary) are obtained by Gromov in \cite[Sec. 4]{Gr2019}.
In both cases, the proofs rely on the $\mu$-bubble technique.
\textbf{f.} It would be interesting to compare Theorem~\ref{alhpmt4}
with some recent progress in proving positive mass and rigidity results for
ALH manifolds (see \cite{AHK2022}, \cite{CG2021}, \cite{CGNP2018} and \cite{HJ2022});
in this latter development, manifolds are often assumed to have nonempty inner boundary with the
mean curvature bound $H\le n-1$ (now $H$ is computed with respect to the \emph{inner} unit normal); 
 such mean curvature bounds serve as barrier conditions in the method of `marginally outer
trapped surfaces' (MOTS), which can be viewed as a generalization of the $\mu$-bubble technique. 
\vskip 2mm

\noindent\textbf{Organization of this article.} 
The proof of Theorem~\ref{counterexample} is technically independent from
the rest of the work and is included in Section~2. Section~3 serves as a preliminary to proving
Theorem~\ref{alhpmt4}, presenting results concerning NRF and conformal deformations. 
In Section 4 we prove Theorem~\ref{alhpmt4}, followed by 
proofs of Corollary~\ref{hpbInfCor} and Theorem~\ref{deform1}. 
In Section~5 we prove Theorem~\ref{alhpmt2} and Theorem~\ref{alhpmt1}.
Several of the proofs rely on the so-called `$\mu$-bubble' technique,
a brief discussion of which is included in Appendix~A. Appendix~B includes
two topological lemmas.


\section{Non-rigidity of $\H^n/\Z^{n-2}$}

Let the hyperbolic $n$-space $\H^n$ be represented by the upper half-space model 
$\R^n_+ = \{(x,\boldsymbol{y},z):x\in \R, \boldsymbol{y}\in \R^{n-2}, z>0\}$,
and let $\Z^{n-2}$ act by translating along the orthogonal lattice $2\pi \Z^{n-2}\subset \R^{n-2}$
while keeping the $x,z$-coordinates fixed. The quotient space is denoted by
$\H^{n}/\Z^{n-2}$ and has the hyperbolic metric 
\begin{equation}
	g_{H} = z^{-2}(\ed z^2 + \ed x^2) + z^{-2} g_{\T^{n-2}},
\end{equation}
where $g_{\T^{n-2}}$ is the associated flat metric on $\T^{n-2}$.
Henceforth, we will regard $(x,z)$ as coordinates on the hyperbolic plane $\H^2$;
manifestly, $(\H^{n}/\Z^{n-2},g_H)$ is a warped product of $\H^2$ and $(\T^{n-2}, g_{\T^{n-2}})$.
\vskip 3mm

The following lemma is easily verified by standard computation, so we omit its proof.
\begin{lemma}\label{H2DerHess}
Let $\nabla, \nabla^2$ denote the gradient and Hessian 
with respect to $g_H$ (same below). We have
\begin{enumerate}[\quad \rm (a)]
	\item $\nabla z=z^2{\partial}/{\partial z}$,
	\item $\nabla^2z({\partial}/{\partial x}, {\partial}/{\partial x}) = -\nabla^2z({\partial}/{\partial z}, {\partial}/{\partial z})=-1/z$,
    \item $\nabla^2z({\partial}/{\partial z}, {\partial}/{\partial x})=0$.
\end{enumerate}	
\end{lemma}

Next, we proceed to prove 
Theorem~\ref{counterexample} by constructing an example
that satisfies all its conditions. The idea is to remove a suitable compact
subset, $X_{p,r}$, from $\H^n/\Z^{n-2}$ and then `glue' the result with a
compact manifold, $\bar X_r$, along their boundaries; $X_{p,r}$ and 
$\bar X_r$ will be defined in Sections~\ref{1stprelimSec} and \ref{2ndprelimSec}
respectively, and then we handle the gluing step 
in Section~\ref{gluingSec}.






\subsection{First preliminary construction}\label{1stprelimSec}
Let $p\in \H^2$, and define
\begin{equation}\label{Xrp_def}
	X_{p,r}:= \D_r(p)\times \T^{n-2}\subset \H^{n}/\Z^{n-2} \quad 
	\mbox{and} \quad  Y_{p,r} :=\partial X_{p,r},
\end{equation}
where $\D_r(p)\subset \H^2$ is the geodesic disc, centered at $p$, of radius $r>0$;
the inclusion in \eqref{Xrp_def} makes sense since $\H^n/\Z^{n-2}$ is a warped product
of $\H^2$ and $\T^{n-2}$, as we already noted.

Now we have two sets of coordinates for $\H^2$: $(x,z)$ and the polar coordinates
$(\varrho,\theta)$ centered at $p$. In terms of the polar coordinates, the metric on $\H^2$ reads
\begin{equation}\label{H2metricpolar}
	g_{\H^2} = \ed\varrho^2 + \sinh^2\varrho\,\ed\theta^2.
\end{equation}

 

\begin{lemma}\label{YprMean}
The boundary $Y_{p,r}\subset (X_{p,r}, g_H)$ has the mean curvature
\begin{equation}\label{Hpr}
	H_{p,r}=\coth r-(n-2)z^{-1}\frac{\partial z}{\partial \varrho}.
\end{equation}
Moreover, 
\begin{enumerate}[\quad \rm (a)]
	\item $|H_{p,r} - \coth r| \leq n-2$;
	\item There exists a constant $r_0>0$ such that $H_{p,r}>0$ for all $r\leq r_0$.
\end{enumerate}

\end{lemma}
\begin{proof}
The formula \eqref{Hpr} is straightforward to check by using the representation 
\[
	g_H = \ed\varrho^2 + \sinh^2\varrho\,\ed\theta^2 + z^{-2}g_{\T^{n-2}}.
\]
Moreover, since both $z^{-1}\nabla z$ and $\nabla \varrho$ have unit norm with respect to $g_H$,
\begin{equation}\label{dzdrhoBd}
\left|\frac{\partial z}{\partial \varrho}\right|
= \left|\langle\nabla z, \nabla \varrho \rangle\right| =\left|z\langle z^{-1}\nabla z, \nabla \varrho \rangle\right|\le z.
\end{equation}
This implies (a), and (b) follows since $\coth r\rightarrow\infty$ as 
$r\rightarrow 0$.
\end{proof}

\begin{lemma}\label{dthetazEst}
	There exists a constant $C_r>0$, depending only on $r$, such that
	\begin{equation}
		\left|\partial_\theta z (r,\theta)\right| \le z \sinh r \quad
		\mbox{and}\quad
		\left|\partial_\theta^2 z(r,\theta)\right| \le C_r z.
	\end{equation} 
\end{lemma}
\begin{proof}
	Since both $z^{-1}\nabla z$ and $(\sinh r)^{-1}(\partial/\partial\theta)$
	have unit norm with respect to $g_{H}$, we have
	\[
		|\partial_\theta z(r,\theta)| = |\<\nabla z, \partial/\partial\theta\>|
		\le z\sinh r.
	\]
	Moreover, a calculation shows that
	\begin{equation}\label{hessztt}
		\nabla^2 z (\partial/\partial\theta,\partial/\partial\theta)
		= \partial_\theta^2 z  +(\partial_\varrho z) \sinh \varrho \cosh \varrho.
	\end{equation}
	By Lemma~\ref{H2DerHess}(b,c), the LHS of \eqref{hessztt} has its magnitude bounded
	by $(\sinh^2\rho)z$; thus, using \eqref{dzdrhoBd} and evaluating
	\eqref{hessztt} at $\varrho = r$, we get 
	\[
		|\partial_\theta^2 z(r,\theta)| \le \sinh r (\sinh r + \cosh r) z.
	\]
	Taking $C_r  = \sinh r(\sinh r + \cosh r)$ finishes the proof.
\end{proof}


\subsection{Second preliminary construction}\label{2ndprelimSec}
Let $\Dcal$ be a $2$-disc with polar coordinates $(\bar\varrho,\bar\theta)$ where
\[
	0\le \bar\varrho\le \pi/3 \quad \mbox{and}\quad 0\le \bar \theta<2\pi. 
\]
Equip $\Dcal$ with the metric
\[
	g_\Dcal = \ed\bar\varrho^2 + 4\sin^2(\bar\varrho/2)\ed\bar\theta^2.
\]
Thus, $(\Dcal, g_\Dcal)$ is isometric to a `cap' in the round sphere of radius $2$.

Now let $r >0$ and $z(\varrho,\theta)$ be as in Section~\ref{1stprelimSec} above. Consider  
\begin{equation}
	\bar X_r := \S^1 \times \Dcal \times \T^{n-3}
\end{equation}
equipped with the metric
\begin{equation}\label{barg}
\bar g =\sinh^ 2r \,\ed\theta^2 +(z(r,\theta))^{-2} g_\Dcal+(z(r,\theta))^{-2}g_{\T^{n-3}},
\end{equation}
and let $\bar Y_r := \partial \bar X_r$.
By construction, the boundaries $(Y_{p,r}, g_H|_{Y_{p,r}})$ and $(\bar Y_r, \bar g|_{\bar Y_r})$ are isometric under the obvious identification.


\begin{lemma}\label{YrMean}	
The boundary $\bar Y_r\subset (\bar X_r, \bar g)$ has the mean curvature
\begin{equation}\label{barHr}
\bar H_r=\frac{\sqrt{3}}{2}z(r,\theta).
\end{equation}
\end{lemma}
\begin{proof}
Standard computation by using \eqref{barg}.
\end{proof}


Regarding the scalar curvature of a warped-product metric, 
the following is well-known.

\begin{lemma}{\rm (cf. \cite[Proposition 7.33]{GL1983})}\label{scalarwarp}
Let $(N^{n-1}, h)$ be an $(n-1)$-dimensional Riemannian manifold with scalar curvature $R_h$. Given any smooth function $\phi(\theta)$ defined on an
interval $I$ and a constant $a>0$, the warped product metric 
$
	g = a^2\ed\theta^2+ \phi(\theta)^2 h
$
defined on $I\times N$ has the scalar curvature
\begin{equation}\label{scalarWarpEq}
R_g=\frac{n-1}{a^2}\left[-2\left(\frac{\phi'}{\phi}\right)'-n\left(\frac{\phi'}{\phi}\right)^2\right]+ \phi^{-2}R_{h}. 
\end{equation}
\end{lemma}
  
In our case, to compute the scalar curvature of $\bar g$, it suffices to substitute 
$h = g_\Dcal + g_{\T^{n-3}}$, $\phi(\theta) = 1/z(r,\theta)$ and $a = \sinh r$ into \eqref{scalarWarpEq}. Noting that 
$R_h = 1/2$, we have
\begin{equation}\label{Rgbar}
	\begin{split}
	R_{\bar g}&=(n-1)(\sinh r)^{-2}\left\{-2\partial_\theta[z\partial_\theta(1/z)]-n[z\partial_\theta(1/z)]^2\right\} + z^{2}/2\\
	&=(n-1)(\sinh r)^{-2}\left\{2(\partial_\theta^2 z)/z-(n+2)[(\partial_\theta z)/z]^2\right\}	+ z^{2}/2,
		\end{split}
\end{equation}		
where $z, \partial_\theta z$ and $\partial_\theta^2 z$ are evaluated at $(r,\theta)$.

Now we are ready to observe the following.

\begin{prop}\label{XrbarInfo}
For fixed $r>0$, the manifold $(\bar X_r, \bar g)$ satisfies:
\begin{enumerate}[\quad\rm(a)]
\item  The scalar curvature 
\[
	R_{\bar g}\geq \frac{1}{2}[z(r,\theta)]^2-C_{n,r}
\] for a
constant $C_{n,r}>0$ depending only on $n$ and $r$. In particular, we have
$
R_{\bar g}>-n(n-1)
$
provided that $p\in \H^2$ is chosen to have a large enough $z$-coordinate;

\item Under the obvious identification (isometry) between $Y_{p,r}$ and $\bar Y_r$, we have
 $\bar H_r >H_{p,r}$ provided that the $z$-coordinate of $p$ is large enough.
 \end{enumerate}
\end{prop}

\begin{proof}
	 (a) follows from \eqref{Rgbar} and Lemma~\ref{dthetazEst};
	  (b) follows from Lemma~\ref{YprMean}(a) and \eqref{barHr}.
\end{proof}

\subsection{The gluing step}\label{gluingSec}


\begin{lemma}\cite[Theorem 5]{BMN2011}\label{bmnthm5}
Let $\Omega$ be a compact $n$-manifold with boundary $\partial\Omega$, and let $g$ and $\tilde g$ be two smooth Riemannian metrics on $\Omega$ such that
\begin{enumerate}[\quad\rm (a)] 
\item $g-\tilde  g=0$ at each point on $\partial\Omega$;
\item the mean curvatures satisfy $H_{\tilde g}-H_{g}>0$ at each point on $\partial \Omega$.
\end{enumerate}
Then, given any $\epsilon>0$ and any neighborhood $U$ of $\partial \Omega$, there exists a smooth metric $ \hat g$ on $\Omega$ with the following properties:
\begin{enumerate}[\quad \rm (1)]
	\item $R_{\hat g} \geq \min\{R_g , R_{\tilde g} \}-\epsilon $ in $\Omega$;
	\item $\hat g=\tilde g$ in $\Omega\setminus U$;
	\item $\hat g= g$	in a neighborhood of $\partial\Omega$.
\end{enumerate}
\end{lemma}
\begin{remark}
By an arbitrary extension, in Lemma~\ref{bmnthm5} it suffices to assume that $g $ is defined only in a neighborhood of $\partial \Omega$. 	
\end{remark}

To prove Theorem~\ref{counterexample}, 
a basic idea is to apply Lemma~\ref{bmnthm5} to obtain a 
metric $\hat g$ on $\bar X_r$ which 
agrees with $g_H$ in a neighborhood of $\partial \bar X_r\cong \partial X_{p,r}$,
so $\hat g$ extends smoothly into 
$(\H^{n}/\Z^{n-2})\setminus X_{p,r}$ by $g_H$. 
A compromise is the $\epsilon$-cost to the scalar curvature estimate. 
Thus, one would like to have a bit more scalar curvature
to begin with, so that the cost can be absorbed, maintaining the
desired lower bound $R_{\hat g}\ge -n(n-1)$. 
This can be achieved by a suitable deformation of 
$g_H$ in a neighborhood of $Y_{p,r}\subset \H^{n}/\Z^{n-2}$, as the following lemma
shows. 


\begin{lemma}\label{tradeoffLemma}
Let  
\begin{equation}
u(\varrho)=\left\{
		\begin{alignedat}{2}
			&1-e^{\frac{1}{\varrho-r_0}},&& \quad \varrho\leq r_0, \\
		   & 1, && \quad \varrho\geq r_0,
		\end{alignedat}
	\right.\nonumber
\end{equation}
and define
\begin{equation}\label{metric2}
g_{H}':=[u(\varrho)]^2\ed\varrho^2 +\sinh^2 \varrho\, \ed\theta^2 +[z(\varrho,\theta)]^{-2} g_{\T^{n-2}}.
\end{equation}
As long as $r_0>0$ is small enough, we can find $\delta>0$ such that
\[
	R_{g_{H}'} + n(n-1)  >0\qquad \mbox{for }
 \varrho \in [r_0-2\delta, r_0).
\]
\end{lemma}

\begin{proof}


By \cite[Claim 2.1]{SWW2022}, we have
\begin{equation}\label{hatscalarcurv}
R_{g_H'}=R_{g_H}+(1-u^{-2})(R_{\gamma(\varrho)}-R_{g_H})+2u^{-3}u'(\varrho) H_{p,\varrho},
\end{equation}
where 
${\gamma(\varrho)}= \sinh^2 \varrho \,\ed\theta^2 +[z(\varrho,\theta)]^{-2}g_{\T^{n-2}}$ and $R_{g_H} = -n(n-1)$.

We want to estimate the RHS of \eqref{hatscalarcurv}. To start with, by Lemma~\ref{scalarwarp},
\begin{equation}\label{Rgamma}
R_{\gamma(\varrho)}= (n-2)(\sinh \varrho)^{-2} 
\left\{2(\partial_\theta^2 z)/z -(n+1)[(\partial_\theta z)/z]^2\right\}.
\end{equation}
Thus, by the proof of Lemma~\ref{dthetazEst}, there exists a constant $C_{n,r_0}$,
depending on $n,r_0$ only, such that
\begin{equation}\label{RgammaBd}
|R_{\gamma(\varrho)}|\leq C_{n,r_0}\quad\mbox{for } \varrho \in [r_0/2, r_0].
\end{equation}
Next, by the definition of $u$, we have, for $\varrho\le r_0$,
\begin{equation}\label{1minusu}
	0\ge 1 - u^{-2} = u^{-2}(-2 e^{\frac{1}{\varrho - r_0}} + e^{\frac{2}{\varrho - r_0}}) \ge - 2u^{-2} e^{\frac{1}{\varrho - r_0}} \ge -2u^{-3} e^{\frac{1}{\varrho - r_0}} .
\end{equation}
Moreover, for sufficiently small $r_0$, we have
$H_{p,\varrho} \geq 1$ for any $\varrho\le r_0$ (Lemma~\ref{YprMean}(b)), and so
\begin{equation}\label{urhoHprho}
	2u^{-3} u'(\varrho) H_{p,\varrho}  \ge 2u^{-3} e^{\frac{1}{\varrho - r_0}}(\varrho - r_0)^{-2}.
\end{equation}
On combining \eqref{hatscalarcurv}, \eqref{RgammaBd}, \eqref{1minusu}
and \eqref{urhoHprho}, we obtain that
\begin{equation}\label{RgHprimeEst}
	R_{g_H'} - R_{g_H} \ge 2u^{-3} e^{\frac{1}{\varrho - r_0}} 
	[(r_0 - \varrho)^{-2} - C_{n,r_0} - n(n-1)]\quad\mbox{for } \rho\in [r_0/2, r_0].
\end{equation}
Clearly, we can choose a small $\delta>0$ such that
\begin{equation}\label{RghDiffBD}
 R_{g_H'} - R_{g_H} >0 \quad \mbox{for }
 \varrho \in [r_0-2\delta, r_0).
\end{equation}
This completes the proof.
\end{proof}


\begin{proof}[Proof of Theorem \ref{counterexample}] 
Let $r_0$ be small enough, and let $u(\varrho)$, $g_H'$ and $\delta$ be as in
Lemma~\ref{tradeoffLemma}. Define 
\[
	c:= \min_{\varrho\in [r_0 - 2\delta, r_0 - \delta]} R_{g_H'} + n(n-1) >0.
\]	
Take $r := r_0 - \delta$, and note that we still have the freedom of 
choosing $p\in \H^2$.

Suppose that the isometry between $Y_{p,r}$ and $\bar Y_r$
maps $\bs{q}\in Y_{p,r}$ to $\bar{\bs{q}}\in \bar Y_r$.
Furthermore, by using Fermi coordinates, any point in a small neighborhood of 
$Y_{p,r}\subset X_{p,r}$ is uniquely represented by a pair 
$(\bs{q}, d')$ where $d'$ is the $g_{H}'$-distance to $Y_{p,r}$. 
Similarly, $(\bar{\bs{q}}, \bar d)$
coordinatizes a neighborhood of $\bar Y_r\subset \bar X_r$.
By identifying $(\bs{q}, d)$ with $(\bar {\bs q}, d)$, we have arranged that 
$g_H' = \bar g$ along $\bar Y_{r}$.

To apply Lemma~\ref{bmnthm5}, assign $\Omega = \bar X_r$, $g = g_{H}'$ (defined 
in a neighborhood $U$ of $\bar Y_r\subset\bar X_r$, via the identification above) and $\tilde g = \bar g$ (defined
on $\bar X_r$). 
As noted above, Lemma~\ref{bmnthm5}(a) is satisfied.
Furthermore, the mean curvature of $Y_{p,r}\subset X_{p,r}$ with respect to
$g_{H}'$ is $H_{p,r}':=H_{p,r}/u(r)\ge H_{p,r}$, but by choosing $p$ to have
large $z$-coordinate, we can still
arrange that $\bar H_r > H_{p,r}'$ (see the proof of Proposition~\ref{XrbarInfo}(b)).
Next, by shrinking $U$ if needed, we can assume that $R_{g_H'} \ge c - n(n-1)$ on $U$, and we can assume the same lower bound for $R_{\bar g}$ by choosing $p$ suitably (Proposition~\ref{XrbarInfo}(a)).
Finally, take $\epsilon = c/2$. 

With the above setting, Lemma~\ref{bmnthm5} applies and yields a metric
$\hat g$ defined on $\bar X_r$, satisfying
\begin{itemize}
 	\item $R_{\hat g}\ge -n(n-1)+ c/2$;
	\item $\hat g = \bar g$ on $\bar X_r \setminus U$;
	\item $\hat g = g_H'$ in a neighborhood of $\bar Y_r\subset \bar X_r$.
\end{itemize}
Thus, $\hat g$ and $g_H'$ piece together to become a smooth metric
$\bs{g}$ defined  on 
\[
	 \bs{M}:=[(\H^n/\Z^{n-2})\setminus X_{p,r}] \cup \bar X_r/\sim,
\] 
where $\sim$ 
indicates boundary identification,
with (non-constant) scalar curvature
$R_{\bs{g}}\ge -n(n-1)$. 
(For the reader's convenience, Figure~\ref{Fig_glue} includes a schematic,
1-dimensional illustration of the construction.)

In the statement of Theorem~\ref{counterexample},
take $(M,g) = (\bs{M}, \bs{g})$, $K = \bar X_r \cup (X_{p,r_0}\setdiff X_{p,r})\subset \bs{M}$ 
and $K' = X_{p,r_0}$, and the proof is complete.
\end{proof}

\begin{figure}[h!]
	\centering
	\includegraphics[scale = 0.6]{Fig_gluingCstr.pdf}
	\caption{A schematic picture of $(\bs{M}, \bs{g})$.}\label{Fig_glue}
\end{figure}

\subsection{Further remarks}
\subsubsection{Surgery applied to $\H^n/\Z^{n-1}$}\label{CErmksurg}
The construction above only modifies a portion of $\H^n/\Z^{n-2}$ that is contained in between $x_0< x < x_1$ for some 
$x_0, x_1\in \R$. By a translation, we can always arrange that $x_0 = 0$.
Now, $T: (x,\bs{y},z)\mapsto (x+x_1, \bs{y},z)$ maps a neighborhood of $\{x = 0\}$ isometrically to a neighborhood of $\{x = x_1\}$. Thus, by
removing the subsets $\{x<0\}$ and $\{x> x_1\}$ from $\bs{M}$ and then
identifying $\{x = 0\}$ and $\{x = x_1\}$  via $T$, we obtain a smooth Riemannian manifold $(\bs{M}', \bs{g}')$ that satisfies $R_{\bs{g'}}\ge -n(n-1)$. In fact, $(\bs{M}', \bs{g}')$ can be viewed as a
compactly supported `deformation' of a hyperbolic cusp $\H^n/\Z^{n-1}$, where $\Z^{n-1}$ acts on $(x,\bs{y})\in \R^{n-1}$ by translating 
along the lattice $x_1\Z \times 2\pi\Z^{n-2}$. This serves as yet another counterexample to Gromov's Statement~\ref{generminoo}.


\subsubsection{A note on topology}\label{CErmktop}
It is interesting to determine the topology of both $\bs{M}$ and $\bs{M}'$ above.

Topologically, $\bs{M}$ is obtained by a \emph{surgery} along 
$\{p\}\times \S^1 \subset \R^2\times \S^1$ and then taking product with $\T^{n-3}$.
The result of that surgery is homeomorphic to 
$\S^1\times\R^2$. To see this, view $\S^3$
as the union of $\D^2\times \S^1$ and $\S^1\times \D^2$ with 
the boundaries identified.
Then $\R^2\times \S^1$ is simply $\S^3$ with the core circle $\Ccal=\S^1\times \{q\}$ removed. Surgery of $\S^3$ along $\{p\}\times \S^1$ yields $\S^1\times\S^2$. Then removing $\Ccal$ from $\S^1\times \S^2$ gives $\S^1\times \R^2$. 
In conclusion, $\bs{M}\cong \S^1\times \R^2\times \T^{n-3}$, which is homeomorphic to $\H^{n}/\Z^{n-2}\cong \R^2\times \S^1\times \T^{n-3}$ via a map that switches the first two factors.

Regarding $\bs{M}'$, note that by identifying $x = 0$ and $x = x_1$
in $\{0\le x\le x_1\}\subset \H^2$, one obtains an open annulus, or equivalently $\R^2\setdiff\{\bs{0}\}$.
Thus, $\bs{M'}$ is obtained by a \emph{surgery} along $\{p\}\times \S^1\subset (\R^2\setdiff\{\bs{0}\})\times\S^1$ and then taking product with $\T^{n-3}$.
In this case, a similar argument as the above applies, and the result of the surgery is homeomorphic to $(\S^1\times \R^2)\setminus (\D^2_\epsilon\times \S^1)$, i.e., the result of 
removing a solid torus that is contained in a 3-ball $\B^3\subset \S^1\times \R^2$
(see Figure~\ref{Mprime}). Thus $\bs{M'} \cong [(\S^1\times \R^2)\setminus (\D^2_\epsilon\times \S^1)]\times \T^{n-3}$. In particular, the two ends of $\bs{M}'$ are separated
by a hypersurface with the topology $\S^2\times \T^{n-3}$; the same is \emph{not} true for $\H^n/\Z^{n-1}$.

\begin{figure}[h!]
	\includegraphics[scale = 0.38]{Fig_surj.pdf}
	\caption{An illustration of $(\S^1\times \R^2)\setminus (\D^2_\epsilon\times \S^1)$,
	where $\D^2_\epsilon\times \S^1$ is shaded. }
	\label{Mprime}
\end{figure}


\section{ALH manifolds, mass and deformations}
This section includes basic notions and results concerning 
ALH manifolds, possibly with arbitrary ends, and their NRF and 
conformal deformations. These results 
 will be used in proving Theorem~\ref{alhpmt4}. 

\subsection{ALH manifolds and mass}

\begin{definition}\label{alh}
Let $(M^n,g)$ be a complete Riemannian manifold without boundary.
Suppose that
\begin{enumerate}[{\rm \quad(1)}]
	\item for some (sufficiently large) 
		compact set $K\subset M$, $M\setminus K$ has a connected component $\Ecal$
		that is diffeomorphic to $(0, 1)\times \mathbb{T}^{n-1}$, and
	\item restricted to $\Ecal$, the metric $g$ admits an asymptotic expansion of the form
	\begin{equation}\label{alhasymp}
	g=\frac{1}{\tau^2}\left[\ed \tau^2+h+\frac{\tau^n}{n} \kappa +\Ocal(\tau^{n+1})\right],
	\end{equation}
	where $\tau$ is the coordinate on the interval $(0,1)$; $h$ denotes a flat metric on $\mathbb{T}^{n-1} $, which represents metric at the conformal infinity $\Ecal_0$ (i.e., when $\tau = 0$); $\kappa=\kappa_{AB}\ed y^A \ed y^B$ is a symmetric tensor defined on $\T^{n-1}$, where $(y^A)$ are flat coordinates on $\T^{n-1}$; finally, $ \Ocal(\tau^{n+1})$ stands for a remainder $Q = Q_{AB}\ed y^A\ed y^B$ with the asymptotics 
	\begin{equation}\label{alhRmdr}
		|Q_{AB}| + \sum_{|\alpha|+k\le 2}|\tau^k\partial_y^\alpha \partial_\tau^k\  Q_{AB}| \le C\tau^{n+1} \quad \mbox{ as } \tau\rightarrow 0,
	\end{equation}
	for some constant $C$, where $\alpha = (\alpha_1,\ldots, \alpha_{n-1})$ are multi-indices.
	\end{enumerate}
Such an $(M,g)$ is called \emph{asymptotically locally hyperbolic (ALH)}, and $\Ecal$ 
 an \emph{ALH end}. Moreover, if $M\setminus \Ecal$ is non-compact, 
we say that $(M,g)$ is \emph{ALH with arbitrary ends}.
\end{definition} 

\begin{definition}\label{massDef}
{\rm (Cf. \cite[Definition 1.1]{LN2015})}
Given $(M,g)$ with an ALH end $\Ecal$ on which $g$ admits
the expansion \eqref{alhasymp}, we call
 \begin{equation}\label{massaspt}
 	m_{\Ecal,g}:=\tr_{h}\kappa =h^{AB} \kappa_{AB}
 \end{equation}
the \emph{mass aspect function} associated to the pair $(\Ecal,g)$.
Furthermore, define
 \begin{equation}\label{massbar}
 \bar m_{\Ecal,g} :=\sup_{\mathbb{T}^{n-1}} m_{\Ecal,g}. 
 \end{equation}
 \end{definition}
 
Throughout, let each $\tau$-level set in $\Ecal$ be denoted by $\Ecal_\tau$.
The following lemma shows how $\bar m_{\Ecal,g}$ is related to the mean curvature of $\Ecal_\tau\subset \Ecal$.

\begin{lemma}\label{meancurvslice}
Given $(M^n,g)$ with an ALH end $\Ecal$. If $\bar m_{\Ecal,g}<0$, then there exist constants $\tau_0, C>0$ such that
\begin{equation}\label{mass_mean}
H_{\Ecal_\tau}\geq (n-1)+C\tau^n \qquad \mbox{for } \tau\le\tau_0,
\end{equation}
where $H_{\Ecal_\tau}$ is the mean curvature of $\Ecal_\tau$ computed
with respect to the `outward normal' $-\partial/\partial\tau$.
\end{lemma}

\begin{proof}
Before making any assumption about $\bar m_{\Ecal,g}$, we have
\begin{equation}\label{meanCurvAsymp}
H_{\Ecal_\tau}=(n-1)-\frac{n-2}{2n}m_{\Ecal,g}\tau^n +\Ocal(\tau^{n+1}).
\end{equation}
For $\bar m_{\Ecal,g} < 0$, let us take
 $C = -\bar m_{\Ecal,g}/10$, and clearly \eqref{mass_mean} holds for some $\tau_0>0$.
\end{proof}


\subsection{NRF deformations}
Given a Riemannian $n$-manifold $(M^n, g_0)$, the \emph{normalized Ricci flow} (NRF), with initial metric $g_0$, is by definition a smooth family of 
Riemannian metrics $g(t)$ on $M$
satisfying the evolution equation
 \begin{equation}\label{NRF}
        \left\{
        \begin{alignedat}{1}
           & \partial_t g =-2\left[\Ric_g+(n-1) g\right],\\
        & g(0) = g_0.
        \end{alignedat}
        \right.
  \end{equation} 

 
  \begin{lemma}\label{nrfdeform}
Suppose that $(M^n, g_0)$ is a complete Riemannian manifold with an ALH end $\Ecal$
that satisfies $R_g\ge -n(n-1)$ as well as the assumptions 
{\rm (\ref{curvbdAssu},\ref{RbdAssu})} in Theorem \ref{alhpmt4}. 
Then there exists a $T>0$ such that, for $t\in (0,T]$, $g(t)$ is complete and satisfies \eqref{NRF} along with the following properties:
\begin{enumerate}[\quad \rm (i)]
	\item $(\Ecal, g(t)|_{\Ecal})$ remains ALH, and $g(t)$ has the expansion (see equation \eqref{alhasymp}) 
			\begin{equation}\label{NRFalhexp}
				g(t) = \frac{1}{\tau^2}\left[\ed\tau^2 + h + \frac{\tau^n}{n}\kappa(t) + \Ocal(\tau^{n+1})\right];
			\end{equation}
	\item on $M$, $R_{g(t)}\geq -n(n-1)$ for all $t\in (0, T]$;
	\item if $g_0$ is not Einstein, then $R_{g(t)}> -n(n-1)$ for all $t\in (0,T]$; 	
	\item outside a compact subset in $M$, $R_{g(t)}\le -\alpha/2$ for $t\in (0,T]$;
	\item if $\kappa(0)=0$, then $\kappa(t)=0$ for all $t\in (0, T]$;
	\item if $\kappa(0)=0$, then for any $t\in (0,T]$ we have $R_{g(t)}+n(n-1)=\mathcal{O}(\tau^{n+1})$ as $\tau\rightarrow0$.
\end{enumerate}
\end{lemma}

\begin{proof}
The existence of $g(t)$, $t\in (0,T]$, satisfying
\eqref{NRF} follows from the existence of a solution $\widetilde g(t)$, $t\in (0,\widetilde T]$, of the Ricci flow initiated at $g_0$, 
for they are related by a time-transformation:
$$
	g(t):=e^{-2(n-1)t}\tilde{g}\left(\Phi(t)\right)\quad\mbox{where } \Phi(t) = \frac{e^{2(n-1)t}-1}{2(n-1)}.
$$
Thus, up to constant factors, the curvature tensor $\Rm(t)$ of 
$g(t)$ satisfies same estimates as $\tilRm(\Phi(t))$ of $\tilde{g}(\Phi(t))$. 
In particular, it follows from \cite{Shi1989} that, for all $t\in (0,T]$, $g(t)$ is complete, and $|\Rm(t)|$ is uniform bounded. 

Now we turn to proving the properties. (i) follows from \cite[Proposition 3.1]{BW12}.
(ii) can be verified by applying the maximum principle (see \cite[Theorem 7.42]{CLN06}) to 
the evolution equation\footnote{For the evolution equation satisfied by $R_{g(t)}$, see \cite[(5.1)]{BW12}.} 
satisfied by $e^{2(n-1)t}(R_{g(t)}+n(n-1))$;
to prove (iii), invoke the strong maximum principle on the domain $\Omega\times[0,t]$,
 where $\Omega\subset M$ is compact on which $g_0$ is not Einstein,
 and then let $\Omega$ exhaust $M$.
(iv) would follow once we show that the integral
\begin{equation}\label{intRm}
	\int_{0}^t \partial_{t'}\tilRm\; \ed t', \quad t\in (0,\widetilde T]
\end{equation}
is uniformly bounded;
to see this, note that the first covariant derivatives of $\tilRm(0)$ are assumed to be bounded, 
by \cite[Theorem 14.16]{CC2008}, we have
$$
|\nabla_{\tilde g(t)}^{2}{\tilRm}(t)|\leq\frac{C}{\sqrt{t}}
$$
for some constant $C>0$; in addition, the evolution equation of $\tilRm$ reads%
\footnote{${\tilRm}* {\tilRm}$ indicates a specific linear combinations of the traces of $\tilRm\otimes\tilRm$.}
$$
\partial_{t}{\tilRm}=\Delta_{\tilde g(t)} {\tilRm}+{\tilRm}* {\tilRm};
$$
of course, $1/\sqrt{t}$ is integrable; combining these, 
it is easy to see that \eqref{intRm} is uniformly bounded for small enough $\widetilde T$, justifying (iv).  
(v) follows from \cite[Proposition 4.3]{BW12}. Finally, (vi) follows from (v) and  \cite[(3.19)-(3.21)]{BW12} 
(note that $g^{ij}(\tau)$ provides an extra factor of $\tau^2$).
\end{proof}


\subsection{Conformal deformations}
Throughout this section, $c_n:= -4(n-1)/(n-2)$.

\begin{lemma}\label{confSubSol}
Let $(M,g)$ be complete with an ALH end $\Ecal$,
and let $f\in C^\infty(M)$ be a non-negative function that satisfies
\begin{enumerate}[\quad \rm(a)]
\item $\supp f \subset K\cup \Ecal$ for some compact subset $K\subset M$;
\item $f \in \Ocal(\tau^{n})$ as $\tau\rightarrow 0$ where $\tau$ is the function occurring in the expansion
		\eqref{alhasymp}.
\end{enumerate}
Then there exists a function $v\in C^\infty(M)$ and a constant $\delta_0$ such that  $0<\delta_0 \le v\le 1$ and
\begin{equation}\label{confLemma_veqn}
    -c_n \Delta_g v + fv = 0 \quad\mbox{in } M.
\end{equation}
\end{lemma}


\begin{proof}
    Let $\{\Omega_i\}_{i = 0}^\infty$ be a sequence of smooth, bounded domains satisfying $\Omega_i \Subset \Omega_{i+1}$ and $\bigcup_i \Omega_i = M$. For each $i$, the Dirichlet problem
    \begin{equation}\label{viDirich}
        \left\{
        \begin{alignedat}{1}
           -c_n\Delta_g v_i + fv_i &= 0\quad \mbox{in } \Omega_i,\\
         v_i&= 1\quad \mbox{on } \partial\Omega_i,
        \end{alignedat}
        \right.
    \end{equation}
has a positive solution $v_i$. By the maximum principle, $0<v_i\le 1$. Thus, $v := \lim_{i\rightarrow\infty}v_i$ is well-defined on $M$, satisfying $0\le v\le 1$ and \eqref{confLemma_veqn}. It remains to 
show that $v$ has a positive lower bound.

Without loss of generality, assume that $\Sigma_i\subset \partial \Omega_i$ is the only component of $\partial \Omega_i$ that is contained in $\Ecal$; in fact, let us assume that each $\Sigma_i$ is a $\tau$-level set.
Denote $\tau_0:=\tau|_{\Sigma_0}$.

To refine the estimate of $v_i$, we construct an auxiliary function $\xi$ and compare it with $v_i$ via the maximum principle. 
Indeed, let $\alpha\in (0,n-1)$ be any constant, and
define
\begin{equation}\label{auxXi}
    \xi =1 - (\tau/\tau_0)^\alpha, \qquad \tau\le \tau_0.
\end{equation}
Using the fact that $-\ln \tau$ is, up to adding a constant, the distance
function to $\Sigma_0$, one easily computes that
\begin{equation}\label{LaplacianXi}
    \Delta_g \xi = \alpha (H_{\Ecal_\tau} - \alpha) ( \tau/\tau_0)^\alpha.
\end{equation}
Thus, by \eqref{meanCurvAsymp}, for sufficiently small $\tau_0$, 
there exists a constant $C_{n,\alpha,\tau_0}>0$ such that
\[
	\Delta_g\xi \ge C_{n,\alpha,\tau_0} \tau^{\alpha}\quad\mbox{for any } \tau\le \tau_0.
\]


Now, \eqref{viDirich}, the fact that $v_i\le 1$, and the assumption that
$f \in \Ocal(\tau^n)$ together imply 
\begin{equation}
      \left\{ \begin{alignedat}{2}
            \Delta_g v_i &\le C'_{f,n} \tau^n\qquad &&\mbox{in } (\Omega_i\setdiff \Omega_0)\cap \Ecal,\\
            v_i &> 0&&\mbox{on } \Sigma_0,\\
            v_i & = 1&&\mbox{on } \Sigma_i,
        \end{alignedat}
    \right.
\end{equation}
where $C'_{f,n}$ is a constant depending only on $f$ and $n$.
In comparison,
\begin{equation}
    \left\{
        \begin{alignedat}{2}
            \Delta_g\xi & \ge C_{n,\alpha, \tau_0} \tau^\alpha \qquad &&\mbox{in } \Ecal \setdiff \Omega_0,\\
            \xi& = 0 && \mbox{on } \Sigma_0\\
            \xi &< 1 &&\mbox{on }  \Sigma_i.
        \end{alignedat}
    \right.
\end{equation} 
Thus, for sufficiently small $\tau_0$, 
the maximum principle implies that $v_i\ge\xi$ in $(\Omega_i \setdiff \Omega_0)\cap \Ecal$.
Upon taking limit, $v\ge \xi >0$ on $\Ecal \setdiff \Omega_1$. 
Since $v\ge 0$, the strong maximum principle, applied to \eqref{confLemma_veqn}, 
implies that $v>0$ on $M$. 

When  $M\setminus \Ecal$ is compact (i.e., $M$ having no arbitrary end), 
the above already implies that $v$ has a positive lower bound.
When $M\setminus \Ecal$ is non-compact,
since $f$ is supported in $K\cup \Ecal$, by choosing $\Omega_0$ to include $K$, we have that each $v_i$ $(i\ge 1)$ is harmonic on $\Omega_i \setdiff(\Omega_0\cup \Ecal)$; using the maximum principle again, we get
\begin{equation}
    \min_{\Omega_i \setdiff(\Omega_0\cup \Ecal)} v_i = \min_{\partial \Omega_0\setminus \Ecal} v_i 
    \xrightarrow{i\rightarrow\infty} \min_{\partial\Omega_0\setminus\Ecal} v =: \delta_{\rm arb} >0.
\end{equation}
To finish the proof, it suffices to take $\delta_0 = \min\{ \delta_{\rm arb} , \inf_{\Omega_1}v, \inf_{\Ecal\setminus\Omega_1}\xi\}$.
\end{proof}

\begin{prop}\label{conformdeform1}
Let $(M^n, g)$ be complete, with an ALH end $\Ecal$ and with
$R_g \ge - n(n-1)$ on $M$.
Let $\bar R\in C^\infty(M)$ be a function that satisfies
\begin{enumerate}[\quad\rm(a)]
\item $-n(n-1)\leq \bar R\leq \min\{R_g, 0\}$;
\item $\supp (R_g - \bar R) \subset \Ecal \cup K$ for some compact subset $K\subset M$;
\item  $\bar R \equiv -n(n-1)$ on $\Ecal \setminus K'$ for some compact subset $K'\subset \Ecal$.
\end{enumerate}
Then the Yamabe equation 
\begin{equation}\label{confyamabe}
    \left\{
        \begin{alignedat}{1}
            -c_n \Delta_g u  + R_g u  - \bar R u^{\frac{n+2}{n-2}}& = 0\qquad\mbox{in } M\\
            u &\rightarrow 1 \qquad\mbox{towards } \Ecal_0,
        \end{alignedat}
    \right.
\end{equation}	
has a solution $u$ with $ 0<\delta_0\le u \le 1$ for some constant $\delta_0$.
In particular, the metric $u^{4/(n-2)} g$ is complete and has the scalar 
curvature $\bar R$.
\end{prop}  
\begin{proof}
The proof follows a super/sub-solution argument. To start with, define $L_g$ by
\begin{equation}
    L_g u = -c_n \Delta_g u + R_g u - \bar R u^{\frac{n+2}{n-2}}.
\end{equation}
Note that $L_g 1 = R_g - \bar R\ge 0$ by assumption. Thus, $1$ is a super-solution of \eqref{confyamabe}.

To find a sub-solution to \eqref{confyamabe}, take $f := R_g - \bar R\ge 0$.
Note that $R_g = - n(n-1)+ \Ocal(\tau^{n})$ in $\Ecal$. 
Thus, Lemma \ref{confSubSol} applies and yields a solution $v$ to \eqref{confLemma_veqn}, satisfying $0<\delta_0\le v\le 1$ for some constant $\delta_0$. Now we compute
\begin{equation}
	\begin{alignedat}{1}
    -c_n\Delta_g v + R_g v - \bar R v^{\frac{n+2}{n-2}}
   & = - c_n \Delta_g v + fv + \bar R(1- v^{\frac{4}{n-2}})v\\
   & = \bar R(1- v^{\frac{4}{n-2}})v\le 0,
   \end{alignedat}
\end{equation}
where the inequality follows from the assumption that $\bar R\le 0$ and the bounds for $v$. Thus, $v$ is a sub-solution of \eqref{confyamabe}.

Then one finishes the proof by following the argument of \cite[Proposition 2.1]{AM88}.
\end{proof}

Next, we will focus on the behavior of $u$
towards the ALH infinity $\Ecal_0$.

\begin{lemma}\label{conformdeform2}
Let $u$ be as in Proposition \ref{conformdeform1}. Given any $\alpha\in (0, n-1)$, there exists a constant $\tau_0>0$ such that
$$
1- (\tau/\tau_0)^\alpha \leq u\leq 1 \qquad \mbox{for any }\tau\leq \tau_0.
$$	
\end{lemma}

\begin{proof} Let $\xi:= 1 - (\tau/\tau_0)^\alpha$. By \eqref{LaplacianXi}, we have
\begin{equation}\label{LapXi2est}
	c_n\Delta_g \xi - (R_g - \bar R)\xi = c_n\alpha (H_{\Ecal_\tau} - \alpha) (\tau/\tau_0)^\alpha - (R_g - \bar R)\xi.
\end{equation}
Since $R_g-\bar R\in \Ocal(\tau^{n})$ in $\Ecal$, the RHS of \eqref{LapXi2est} is positive for $\tau\le \tau_0$, provided that $\tau_0$ is
sufficiently small.
On the other hand, since $\bar R \le 0$ and $0<u\le 1$, \eqref{confyamabe}
implies that
\begin{equation}
	c_n \Delta_g u - (R_g - \bar R)u  \le 0.
\end{equation}
Regarding boundary data,
\[
	u - \xi \ge 0 \quad \mbox{along } \tau = \tau_0
	\qquad\mbox{and}\qquad	\lim_{\tau\rightarrow 0}u= \lim_{\tau\rightarrow 0} \xi = 1.
\]
Now the maximum principle implies that $u\ge \xi$ for $\tau \in (0,\tau_0]$.
\end{proof}



\begin{prop}\label{behaviornearinfty}
Let $(M^n, g)$, $\bar R$ and $u$ be as in Proposition~\ref{conformdeform1}.
Additionally, suppose that $R_g +n(n-1)\in \Ocal(\tau^{n+1})$ and that, however small $\tau_0$ is, $R_g>-n(n-1)$ at some point in
 $\{\tau\leq \tau_0\}\subset \Ecal$. Then $u$ must have the following asymptotic expansion near $\tau = 0$:
\begin{equation}\label{expansionofu}
    u=1+u_{n0}\tau^n+\Ocal(\tau^{n+1-\epsilon}),
\end{equation}
where $u_{n0}<0$ is a smooth function defined on the conformal infinity 
$\Ecal_0\cong \T^{n-1}$ and $\epsilon >0$ is an arbitrary small constant.
\end{prop}

\begin{proof} Let us take $w:=u- 1\leq 0$. 
By \cite[Theorem 1.3]{ACF92}, $w$ has the expansion
 \begin{equation}
     w=\sum\limits_{i=1}^{\infty}\sum\limits_{j=0}^{N_{i}}u_{ij}\tau^i(\log\tau)^j 
 \end{equation}
where $u_{ij}\in C^{\infty}(\Ecal_0)$.
Clearly, the proof would be complete once we verify the conditions:
\begin{enumerate}[\qquad\rm (C1)]
	\item $u_{ij} =0$ for $i<n$;
	\item $u_{nj} =0$ for  $j>0$;
	\item $u_{n0} <0$.
\end{enumerate}

\noindent\underline{\emph{Verification of (C1):}}
By \eqref{confyamabe}, $w$ satisfies
\begin{equation*}
	\Delta_g w  - n w = \frac{1}{c_n} \left[R_g (w+1)  - \bar R (w+1)^{\frac{n+2}{n-2}}\right]
						- nw
\end{equation*}
Since only a neighborhood of $\Ecal_0$ is 
concerned, we can simply substitute $\bar R = -n(n-1)$; by rearranging terms,
we get
\begin{equation}\label{LapminusnW}
\begin{alignedat}{1}
	\Delta_g w - nw &= \frac{1}{c_n} [R_g+n(n-1)] u
			 + \frac{n(n-1)}{c_n} \left[(w+1)^{\frac{n+2}{n-2}} - 1 - \frac{n+2}{n-2}w\right]\\
			 & =: A+B.
\end{alignedat}
\end{equation}
Since $\lim_{\tau\rightarrow 0 } u = 1$ and 
$0\le R_g+n(n-1) \in \Ocal(\tau^{n+1})$, we have $A\ge 0$ and $A\in \Ocal(\tau^{n+1})$.
On the other hand,
$B$ is the remainder of a Taylor expansion truncated at the linear term, so
$B = \Ocal(w^2)$ as $\tau\rightarrow 0$. By Lemma~\ref{conformdeform2}, $w = \Ocal(\tau^\alpha)$ for any $\alpha< n- 1$.
Of course, we can choose $\alpha > (n+1)/2$, and thus
$B = \Ocal(\tau^{2\alpha}) = o(\tau^{n+1})$. 
In summary, for sufficiently small $\tau_0$, we have
\begin{equation}\label{LapwminusNwEst}
	0\le  \Delta_g w - n w \in \Ocal(\tau^{n+1})\qquad \mbox{for } \tau \le \tau_0.
\end{equation}

Now consider any $\beta \in (n-1, n)$. Using \eqref{meanCurvAsymp}, 
it is easy to verify that
\[
	\Delta_g \tau^\beta - n\tau^\beta = -(\beta + 1)(n - \beta) \tau^\beta  
	+ \Ocal(\tau^{n+2}).
\]
Clearly, there exists $\tau_0>0$ such that
\begin{equation}
	(\Delta_g - n)(w + \lambda \tau^\beta) \le 0 \quad \mbox{for all }
	\tau\le \tau_0 \mbox{ and constants } \lambda \ge 1.
\end{equation}
Fix such a $\tau_0$, and let us choose $\lambda\ge 1$ such that 
$w|_{\{\tau = \tau_0\}}+\lambda\tau_0^\beta \ge 0$; moreover,
we have $\lim_{\tau\rightarrow 0} (w+\lambda\tau^\beta) = 0$.
Thus, by the maximum principle,
\begin{equation}
	w\ge -\lambda\tau^\beta\quad \mbox{for }\tau\le \tau_0.
\end{equation}
Since $\beta\in (n-1,n)$ is arbitrary and $w\le 0$, this verifies (C1).
\vskip 2mm
\noindent\underline{\emph{Verification of (C2):}}
By (C1), we have
\[
	w = \sum_{j = 0}^{N_n} u_{nj} \tau^n(\log\tau)^j + \Ocal(\tau^{n+1 - \epsilon}).
\]
Further information about $u_{nj}$ is obtainable by 
computing $(\Delta_g - n)w$ using this expansion and then comparing the result with \eqref{LapwminusNwEst}. In fact, direct computation and \eqref{meanCurvAsymp}
yield:
\begin{align*}
	(\Delta_g  - n)\tau^n & = \Ocal(\tau^{n+2}), \\
	(\Delta_g - n) [\tau^n(\log\tau)^j]&
	 = \left[(n+1)j (\log\tau)^{j-1} + j(j-1)(\log\tau)^{j-2}\right]\tau^n\\
	 &\quad+\Ocal(\tau^{n+2}) \qquad (1\le j\le N_i).
\end{align*}
Now, since $u_{nj}$ are all defined on $\Ecal_0$,
we have $\Delta_g u_{nj}\in \Ocal(\tau^2)$; 
and since the remainder $\Ocal(\tau^{n+1-\epsilon})$ does not contribute to the coefficients $s_j$ of $\tau^n(\log\tau)^j$ in $(\Delta_g - n)w$, we have
that $s_j$ equals to
\[
	\begin{alignedat}{2}
		& (n+1)  u_{n1} + 2 u_{n2}&&\qquad \mbox{for } j = 0,\\
		2&(n+1) u_{n2} + 6 u_{n3} && \qquad \mbox{for } j= 1,\\
		&\qquad\qquad\qquad\qquad\vdots\\
		& N_n (n+1) u_{nN_n} && \qquad \mbox{for }j = N_n - 1,\\
		&\qquad\quad  0 && \qquad \mbox{for }j = N_n.
	\end{alignedat}
\]
By \eqref{LapwminusNwEst}, all $s_j$ must vanish, which implies that
\begin{equation}
	u_{nj} \equiv 0 \quad\mbox{for } j =1, \ldots, N_n.
\end{equation}
This verifies (C2).

\vskip 2mm
\noindent\underline{\emph{Verification of (C3):}}
Consider an auxiliary function $\zeta := -\delta (\tau^n + \tau^{n+1})$
where $\delta>0$ remains to be chosen.
Now
\[
	(\Delta_g - n) \zeta = - \delta[(n+2) \tau^{n+1} + \Ocal(\tau^{n+2})],
\]
so $(\Delta_g - n)\zeta \le 0$ provided that $\tau$ is small, and let us choose
$\tau_0$ accordingly (note: this is 
independent of the choice of $\delta$).
By comparison, recall from \eqref{LapwminusNwEst} that
$(\Delta_g - n) w\ge 0$ for $\tau\le \tau_0$.

Regarding boundary data, first note that the assumption about $R_g$
implies that $w$ cannot be identically zero for $\tau\in (0,\tau_0]$; thus, the strong maximum principle
implies, in particular, that $w < 0$ along $\tau = \tau_0$. This allows us to choose
$\delta$ such that $w \le \zeta$ along $\tau = \tau_0$. 
Moreover, both $w, \zeta\rightarrow 0$ as $\tau\rightarrow 0$. Now, by the maximum
principle, we get
\[
	w\le \zeta = -\delta(\tau^n + \tau^{n+1}) \quad\mbox{for } \tau\le \tau_0.
\]
This proves that $u_{n0}<0$, verifying (C3).
\end{proof}


\begin{lemma}\label{confdeformmass}	
	Let $(M^n,g)$ be a Riemannian manifold with an ALH end $\Ecal$,
	on which the asymptotic expansion \eqref{alhasymp} applies.
	Suppose that $u = 1+\varphi \tau^n + \Ocal(\tau^{n+1})$
	is a function defined on $\Ecal$, where $\varphi\in C^\infty(\Ecal_0)$.
	Then, up to a diffeomorphism that restricts to be the identity on $\Ecal_0$, the 
	deformed metric $\bar g = u^{\frac{4}{n-2}} g$ on $\Ecal$
	has the expansion
	\begin{equation}\label{bargExp}
		\bar g = \frac{1}{\bar\tau^2} 
			\left[ \ed\bar\tau^2 + \bar h + \frac{\bar\tau^n}{n}\bar\kappa 
			+ \Ocal(\bar\tau^{n+1})\right],
	\end{equation}
	where
	\begin{equation}
		\bar h = h\quad\mbox{and}\quad 
		\bar \kappa = \kappa + \frac{4(n+1)}{n-2}\varphi h.
	\end{equation}	
\end{lemma}
\begin{proof}
	A standard argument following the proof of  \cite[Lemma 6.5]{BQ08}.
\end{proof}
%\begin{proof}
%	The proof follows a standard procedure (cf. \cite[Lemma 6.5]{BQ08}). 
%	In fact, note that $\tau$ is a geodesic defining function on $(\Ecal, g)$.
%	Consider $\bar \tau = e^w \tau$ where $w = w(\tau, \bs{y})$ 
%	($\bs{y}$ being coordinates on $\Ecal_0\cong \T^{n-1}$).
%	The condition that $\bar\tau$ is a geodesic defining function with 
%	respect to $\bar g$ (i.e., $|\ed\bar\tau|_{\bar\tau^2\bar g} = 1$) is
%	equivalent to the PDE
%	\begin{equation}\label{spcldefCond}
%		2\frac{\partial w}{\partial \tau} + \tau |\ed w|_{\tau^2 g}^2
%		 = \frac{1}{\tau} (u^{\frac{4}{n-2}} - 1)
%		 = \frac{4}{n-2} \varphi\tau^{n-1} + \Ocal(\tau^n).
%	\end{equation}
%	To preserve the conformal infinity, we need $w(0,\bs{y})\equiv 0$.
%	Thus, by substituting the Taylor expansion of $w$, in powers of $\tau$,
%	into \eqref{spcldefCond}, we find that
%	\[
%		w(\tau, \bs{y}) = \frac{2}{n(n-2)} \varphi\tau^n + \Ocal(\tau^{n+1}),
%	\]
%	and therefore
%	\begin{equation}\label{taubarintau}
%		\bar \tau = e^w \tau = \tau +\frac{2}{n(n-2)} \varphi\tau^{n+1} 
%						+ \Ocal(\tau^{n+2}).
%	\end{equation}
%	To have a desired expansion of $\bar g$, consider a new set of coordinates
%	$(\bar \tau, \bar{\bs{y}})$, extending $(0,\bs{y})|_{\Ecal_0}$,
%	such that the $\ed \bar y_i$ are
%	orthogonal to $\ed\bar\tau$. Using \eqref{taubarintau}, it is not difficult
%	to see that 
%	$\partial/\partial\bar \tau$ satisfies
%	\[
%		\frac{\partial}{\partial\bar\tau} = 
%		 \left[1 - \frac{2(n-1)}{n(n-2)}\varphi\tau^n\right]\frac{\partial}{\partial\tau}
%		 +\Ocal(\tau^{n+1}).
%	\]
%	Integration along this vector field gives
%	\[
%		\bar{\bs{y}}(\tau, \bs{y}) = \bs{y} + \Ocal(\tau^{n+1}),
%	\]
%	and so
%	\[
%		\tau^2 g(\partial/\partial{\bar y^A}, \partial/\partial{\bar y^B})
%		 = \tau^2 g(\partial/\partial{y^A}, \partial/\partial{y^B}) + \Ocal(\tau^{n+1}).
%	\]
%	Thus, writing $h = h_{AB}\ed y^A\ed y^B$ and similarly for $\kappa$, we have
%	\begin{equation}
%		\begin{alignedat}{1}
%			\bar\tau^2 \bar g
%			&= \bar \tau^2 u^{\frac{4}{n-2}}g\\
%			&= \ed\bar\tau^2 \
%				+ (\bar\tau/\tau)^2u^{\frac{4}{n-2}}[h_{AB}\ed \bar y^A\ed \bar y^B + 
%								(\kappa_{AB}/n) \tau^n\ed \bar y^A\ed \bar y^B ] + \Ocal(\tau^{n+1}).
%		\end{alignedat}
%	\end{equation}
%	By invoking \eqref{taubarintau} and the expansion for $u$, we get
%	\begin{equation}\label{taubar2gbar}
%		\begin{alignedat}{1}
%		\bar\tau^2 \bar g &= \ed\bar\tau^2
%			+ \left(1+ \frac{2}{n(n-2)}\varphi\tau^n\right)^2 
%				\left(1+\frac{4}{n-2}\varphi\tau^n\right) h_{AB}\ed \bar y^A\ed \bar y^B
%			\\
%			&\quad + \frac{\kappa_{AB}}{n} \tau^n \ed \bar y^A\ed \bar y^B
%+ \Ocal(\tau^{n+1})\\
%			& = \ed\bar\tau^2 + \left\{h_{AB} + 
%				\frac{1}{n}\left[\frac{4(n+1)}{n-2}\varphi h_{AB} + \kappa_{AB}\right]\tau^n \right\}\ed\bar y^A\ed\bar y^B  + \Ocal(\bar\tau^{n+1}).
%		\end{alignedat}	
%	\end{equation}
%	On comparing \eqref{taubar2gbar} with \eqref{bargExp}, the proof is complete.
%\end{proof}


\section{Two rigidity results}
The goal of this section is to prove Theorem~\ref{alhpmt4}, Corollary~\ref{hpbInfCor} and Theorem~\ref{deform1}.
\begin{prop}{\rm (Cf. \cite[Theorem 1.1]{CLSZ2021})}\label{incprssArg}
    For $3\leq n\leq 7$, let $M^n$ be a {(connected) non-compact} manifold with connected, compact boundary $\Sigma$. 
    Let $\iota:\Sigma\hookrightarrow M$ be the inclusion map. Suppose that $\Sigma\in \mathcal{C}_{deg}$ (see Definition~\ref{cdegdef})
    and that the map 
    $\iota$ is incompressible. Then $M$ admits no complete metric $g$ with  $R_g\geq-n(n-1)$ and $H_{\Sigma}> n-1$.
\end{prop}
\begin{proof}
    To begin with, by the classification of covering spaces, there exists a covering of $M$, say $p:\hat{M}\rightarrow M$, that satisfies 
    \begin{equation} \label{fundgrouopRel}
    	p_*(\pi_1(\hat{M}))=\iota_*(\pi_1(\Sigma))\subset \pi_1(M),
    \end{equation}
    where base points for the fundamental groups are omitted. Moreover, by the homotopy lifting property, there exists an embedding
    $\hat{\iota}:\Sigma \rightarrow \hat{M}$ such that $\iota = p\circ\hat \iota$. 
    
    By \eqref{fundgrouopRel} and the incompressibility of $\iota$, the composition 
    \[
    	J:= \left({\iota_*}^{-1}\big|_{\iota_*(\pi_1(\Sigma))}\right)\circ p_*:\pi_1(\hat{M})\rightarrow \pi_1(\Sigma)
    \]
    is a well-defined group homomorphism.
     Since $\Sigma$ is aspherical, by \cite[Proposition 1B.9]{Hatcher_AT}, there exists a map $j:\hat{M}\rightarrow \Sigma$ such that $j_*:\pi_1(\hat{M})\rightarrow \pi_1(\Sigma)$ is equal to $J$;
     in particular,  $j_*\circ\hat\iota_* = \id_{\pi_1(\Sigma)}$; then, by applying the uniqueness part of  \cite[Proposition 1B.9]{Hatcher_AT} to $\Sigma$, it is easy to see that
     $j\circ \hat\iota$ is in fact homotopic to $\id_{\Sigma}$.

	Since $\iota$ is an embedding, each boundary component of $\hat M$, which is a liftings of $\Sigma$, must be 
	diffeomorphic to $\Sigma$. In particular, denote $\hat{\Sigma}=\hat\iota(\Sigma)$.
	Since $j\circ \hat \iota$ is homotopic to $\id_\Sigma$, we have $[\hat{\Sigma}] = \hat\iota_*[\Sigma]\neq 0 \in H_{n-1}(\hat{M};\mathbb{Z})$.
		
	Now, for the sake of deriving a contradiction, suppose that $g$ is a complete metric on $M$ with $R_g\geq-n(n-1)$ and $H_{\Sigma}\geq (n-1)(1+\delta)$ for some constant $\delta>0$. Let $\hat{g}:=p^*g$ be the pull-back metric on $\hat M$, and define 
	$\rho(x):=\dist_{\hat{g}}(x,\hat{\Sigma})$ for $x\in \hat{M}$. 
	
	For an arbitrarily large $T>0$,  let 
	$$\mathcal{D}_T:=\{x\in \hat{M}: \rho(x)\leq T\},$$
	and let $\hat{\Sigma}_i$ $(0\leq i\leq k)$ be those components of $\partial{\hat M}$ that satisfy 
	$$
	\hat{\Sigma}_i \cap \mathcal{D}_T\neq \emptyset,
	$$
	where $\hat \Sigma_0 = \hat\Sigma$.
	Define  (see Figure~\ref{UcalTeps} below)
	$$
	\mathcal{U}_T= \mathcal{D}_T\cup \bigcup_{0\leq i\leq k} \hat{\Sigma}_i\quad \text{and} \quad 
	\mathcal{U}_{T,\epsilon}=\{x\in\hat{M}:\dist_{\hat{g}}(x,\Ucal_T)<\epsilon\}.
	$$
	Since $M$ is complete, connected and non-compact, so is $\hat M$, and we have $\bar\Ucal_T \Subset \Ucal_{T,\epsilon}$.
	Moreover, for small enough $\epsilon$,
	$$
	\mathcal{U}_{T,\epsilon} \cap \left(\partial\hat{M}- \bigcup_{0\leq i\leq k}\hat{\Sigma}_i\right)=\emptyset.
	$$ 
	Thus, by the smooth Urysohn Lemma, there exists a function $\eta \in C^\infty(\hat M)$ with 
	\begin{equation}
		\eta(x)=    \left\{
		\begin{alignedat}{1}
			&0,\quad x\in \mathcal{U}_T\\
			& 1, \quad x\in \hat{M}\setminus\mathcal{U}_{T,\epsilon}   
		\end{alignedat}
		\right.
	\end{equation}
	Let $a\in (0,1)$ be a regular value of $\eta$. Automatically, $\eta^{-1}(a)$ is a smooth, closed hypersurface of $\hat M$, and 
	$\eta^{-1}(a)\cap \partial\hat M = \emptyset$.

            \begin{figure}[h!]
            \includegraphics[scale = 0.25]{Fig_B_a}
            \caption{A schematic picture showing 
            $\Ucal_T$, $\Ucal_{T,\epsilon}$ (left figure) and
            $\Bcal_a$ (right figure).
            The complement of 
            $\Ucal_{T,\epsilon}$, which may include more boundary components of $\hat M$, 
            is not displayed.}
            \label{UcalTeps}
            \end{figure}

	By the above arrangement, $\Bcal_a:=\eta^{-1}([0,a])$, equipped with the restriction of the metric $\hat g$, is a Riemannian 
	band with  
	$$
	\partial_{+}=\hat{\Sigma}\quad \mbox{and} \quad
	\partial_{-}=\partial \Bcal_a \setminus \hat{\Sigma} = \eta^{-1}(a) \cup \bigcup_{1\le i\le k} \hat\Sigma_i .
	$$
	By letting $f = j|_{\Bcal_a}$ and using Lemma~\ref{C_degNSepLemma}, one easily sees that $\Bcal_a$ satisfies the \Nsep property.
	Then take $\partial_{\star}=\eta^{-1}(a)$. 
		
	With these choices, all assumptions of Lemma \ref{bandWidth} are satisfied for $(\Bcal_a, \hat g|_{\Bcal_a};\partial_-, \partial_+)$
	and $\partial_\star$. 
	Since $(\hat M,\hat g)$ is complete and non-compact, 
	the distance $\dist_{\hat{g}}(\partial_{\star},\partial_{+})$ can get arbitrarily large as one chooses large $T$. 
	This contradicts Lemma \ref{bandWidth}.
\end{proof}


\begin{remark}
Proposition \ref{incprssArg} still holds if $M$ is allowed to be compact.
In fact, proceeding along the same proof, we still have $[\hat \Sigma]\ne 0\in H_{n-1}(\hat M;\Z)$, so
$\hat M$ cannot be compact with a single boundary component.
Hence, either (1) $\hat M$ is non-compact, and the previous proof applies verbatim;
or (2) $\hat M$ is itself a Riemannian band with $\partial_+ = \hat\Sigma$ that
satisfies the \Nsep property and the curvature bounds $R_{\hat g}\ge -n(n-1)$, 
$H_{\partial \hat M}\ge (n-1)(1+\delta)$; however, by 
Remark~\ref{bandNonexistence}{(A)}, such a band cannot exist, reaching a contradiction.
\end{remark}

\begin{prop}\label{alhPrep}
	Let $(M^n,g)$ be a complete Riemannian manifold without boundary, with an ALH end 
	$\Ecal\cong (0,1)\times \T^{n-1}$, and satisfies $R_g\ge -n(n-1)$.
	Suppose that $Y:=M\setminus \Ecal$ is non-compact and that $\partial Y\cong \T^{n-1}$
	is incompressible in $M$. Then $\bar m_{\Ecal, g}\ge 0$. 
	In addition, if the assumptions {\rm (\ref{curvbdAssu},\ref{RbdAssu})} in Theorem~\ref{alhpmt4} hold,
	then $\kappa = 0$ only if $(M,g)$ is Einstein.
\end{prop}

\begin{proof}
Suppose, on the contrary, that $\bar m_{\Ecal,g}< 0$. 
Let $\tau$ be a defining function compatible with the ALH structure of $\Ecal$
(see \eqref{alhasymp}). 
Then by Lemma \ref{meancurvslice}, 
there exists a small $\tau_0>0$ such that the mean curvature of the level set $\Ecal_{\tau_0}$ satisfies
\[
	H_{\Ecal_{\tau_0}} \geq (n-1)+\delta_0
\] 
for some $\delta_0>0$.

Now, remove 
$\{0< \tau< \tau_0\}$, a subset of $\Ecal$, from $M$ and denote the resulting manifold by $M'$. 
By using the assumptions, it is easy to see that $\partial M' = \Ecal_{\tau_0}\cong \T^{n-1}$ is incompressible in $M'$. 
Clearly, $\partial M'\in \Ccal_{deg}$. 
By Proposition~\ref{incprssArg},
we get a contradiction.
This proves the inequality $\bar m_{\Ecal, g}\ge 0$.

Next we turn to the second part of the proposition. Again we argue by contradiction. 
Assume that $\kappa = 0$
without $(M,g)$ being Einstein. Let $g(t)$ be the NRF initiated at $g$.
Then by Lemma \ref{nrfdeform}, for some small $t_0$, we have
\begin{enumerate}[\qquad i)]
	\item $R_{g(t_0)}>-n(n-1)$ on $M$;\label{scalLB}
	\item $R_{g(t_0)}\le -\alpha/2<0$ outside a compact subset of $M$;
	\item $R_{g(t_0)} = -n(n-1) + \Ocal(\tau^{n+1})$ on $\Ecal$;	\label{scalAsympE}
	\item $(\Ecal, g(t_0)|_\Ecal)$ remains ALH with $\kappa(t_0) = 0$.
\end{enumerate}
It is easy to check that a function $\bar R$ as described in Proposition~\ref{conformdeform1} exists; thus, there is
a positive function $u\in C^\infty(M)$ such that $\bar g := u^{4/(n-2)}g(t_0)$ is complete
with  $R_{\bar g} = \bar R\ge -n(n-1)$. Furthermore, thanks to \ref{scalLB}) and \ref{scalAsympE}) above, 
both Proposition ~\ref{behaviornearinfty} and Lemma~\ref{confdeformmass} apply.
As a consequence, $(\Ecal, \bar g|_\Ecal)$ remains ALH and satisfies 
\[ \bar\kappa =  \frac{4(n+1)}{(n-2)} u_{n0} h,\] where  
$u_{n0}<0$, and $h$ is a flat metric on $\T^{n-1}$. 
Clearly, $\bar m_{\Ecal, \bar g} =\tr_h\bar\kappa < 0$. This contradicts the first part of
 the proposition.
\end{proof}

\begin{proof}[Proof of  Theorem \ref{alhpmt4}]	
	For convenience, let $\Ncal_o$ (resp., $\Hcal_o$) denote the 
	result of removing a tubular neighborhood of $\phi(\T^k)$ from $N$ (resp., $\psi(\T^k)$ from $\H^n/\Z^{n-1}$). Both 
	$\partial \Ncal_o$ and $\partial \Hcal_o$ inherit the product structure $\S^{n-k-1}\times \T^{k}$, which are 
	identified to form $M$. In symbols, $M = \Hcal_o\sqcup_\Phi \Ncal_o$, where $\Phi: \partial \Hcal_o\rightarrow \partial \Ncal_o$
	is the identification map.
		
	By Proposition~\ref{alhPrep}, to prove the theorem it suffices to show that the boundary
	$\Sigma$ of $M\setminus\Ecal$ is incompressible in $M$. 
	
	To show this, it in turn suffices to show that the $\T^k$-factor of $\partial \Ncal_o$
	is incompressible in $M$, according to Lemma~\ref{incprLemma}. 
	
	If this were not the case, let $L$ be a non-contractible loop in 
	$\{x\}\times \T^{k}\subset \partial \Ncal_o$ 
	that is contractible in $M$. 
	
	Now consider 
	\[	
		\Hcal':=(\S^1\times \T^{n-k-1} - \Bb)\times \T^k
	\] 
	where $\Bb$ is an $(n-k)$-ball embedded in $\S^1\times \T^{n-k-1}$. 
	Topologically, $M$ can be viewed as a subset of $M':=\Hcal' \sqcup_\Phi \Ncal_o$, so $L$ is also contractible
	in $M'$. By \cite[Lemma~{A.3}]{CLSZ2021}, $\Hcal'$ satisfies the `lifting property' (see \cite[Definition~{A.2}]{CLSZ2021}).
	Thus, \cite[Lemma~{A.4}]{CLSZ2021} applies, showing that $L$ is contractible in $\Ncal_o$ and hence in $N$;
	since $\{x\}\times\T^{k}$ and $\phi(\T^k)$ are homotopic in $N$, $\phi$ cannot be incompressible, violating the assumption
	\eqref{incprsAssu}.
\end{proof}

\begin{proof}[Proof of Corollary~\ref{hpbInfCor}] In this setting, the assumptions (a-e)
in Theorem~\ref{alhpmt4} are satisfied. Since $\kappa$ automatically vanish, we conclude that $g$ is Einstein.
Write the metric on $\H^n/\Z^{n-1}$ as $\ed t^2 + e^{2t} g_0$ where $g_0$ is a flat metric on $\T^{n-1}$. Since $\H^n/\Z^{n-1}$ is isometric
to $(M,g)$ outside a compact set, one can remove the corresponding cusp (i.e., $\{t<-a\}$ for some $a\gg 0$) from $M$ and obtain a 
complete, non-compact manifold $(M', g')$ with boundary $\partial M'\cong \T^{n-1}$, satisfying $H_{\partial M'} \equiv  n - 1$, where the mean 
curvature is computed with respect to the \emph{inward} normal.
By \cite[Theorem 2]{CK92}, $(M', g')$ is isometric to $[-a,\infty)\times \T^{n-1}$ with the warped product metric 
$\ed {t'}^2 + e^{2t'} g_0$;
by using this fact and the respective distance functions to $\partial M'\subset M$ and $\{-a\}\times\T^{n-1}\subset \H^n/\Z^{n-1}$, it is easy to construct an isometry
between $(M,g)$ and $\H^n/\Z^{n-1}$.
\end{proof}

\begin{remark}\label{hpbInfCor_NonCptRmk}
The statement of Corollary~\ref{hpbInfCor} remains true when $N$ is 
non-compact without boundary. In fact, one only needs to prove the incompressibility of 
a $\T^{n-1}$-slice located far into the ALH infinity of $M$, and this is handled by a corresponding step
in the proof of Theorem~\ref{alhpmt4}. Then the result follows directly from Theorem~\ref{alhpmt1}.
\end{remark}


\begin{proof}[Proof of Theorem~\ref{deform1}]  
Let $(x,z)$ be the standard coordinates on $\R^2_+$, a topological factor of $\H^n/\Z^{n-2}$. Since $K$ is compact, via the isometry $f$,
both $x$ and $z$ can be regarded as coordinate functions on $M\setminus K$. Thus, for a large enough $x_0>0$, we can 
remove $\{|x|>x_0\}$ from $M$ and then identify $\{x = \pm x_0\}$ in the same way 
as we did in Section~\ref{CErmksurg}. The result is a complete Riemannian manifold $(M^*,g^*)$
with an ALH end $\Ecal$, satisfying $R_{g^*}\ge -n(n-1)$. Moreover, $(M^*, g^*)$ is isometric to $\H^n/\Z^{n-1}$ outside a compact set; thus,
the assumptions {\rm (\ref{curvbdAssu},\ref{RbdAssu})} in Theorem~\ref{alhpmt4} hold automatically, and
$\kappa = 0$ for $(\Ecal, g^*|_\Ecal)$.

It is easy to see that $M^*$ is of the form $M_1\sqcup_\Phi M_2$ as
described in Lemma~\ref{incprLemma} 
with $k = n-2$. In particular, $M_2$ can be viewed as a subset of $M$. 
By assumption, $f^{-1}(T)$ is incompressible in $M$ and hence in $M_2$.
Using the proof of Theorem~\ref{alhpmt4}, one can show that $f^{-1}(T)$ is incompressible in $M^*$; 
then by Lemma~\ref{incprLemma}, $\partial (M\setminus\Ecal)\cong\T^{n-1}$ is incompressible
in $M^*$.

Thus, all conditions in Proposition~\ref{alhPrep} are verified for $(M^*, g^*)$, and we conclude that $g^*$ is Einstein.
The very proof of Corollary~\ref{hpbInfCor} shows that there is an isometry 
$\tilde f: (M^*, g^*)\rightarrow \H^n/\Z^{n-1}$ that uniquely extends the isometry, induced by $f$, between the `cuspidal ends' 
%$\{0<z<z_0\}$ 
in $M^*$ and $\H^n/\Z^{n-1}$.
Let $z_0>0$ be sufficiently small; then by using distance functions to the hypersurfaces 
$\{z = z_0\}$ in both $M$ and $\H^n/\Z^{n-2}$, it is easy to
construct an isometry between $(M,g)$ and $\H^n/\Z^{n-2}$; details are
left to the interested reader.
\end{proof}


\section{Two splitting results of `cuspidal-boundary' type}
The bulk of this section is dedicated to proving Theorem~\ref{alhpmt2}. 
The proof of
Theorem~\ref{alhpmt1}, which largely depends on those of Proposition~\ref{incprssArg} 
%(for the inequality part)
and Theorem~\ref{alhpmt2},
%(for the `splitting' part) 
will be sketched at the end
of the section.
\vskip 2mm

Now we begin our proof of Theorem~\ref{alhpmt2}.

In addition to its hypothesis, let us assume that $H_{\partial M} \ge 3$. 
Under this assumption, the proof would be complete once we show that
$(M,g)$ is isometric to $((-\infty,0]\times \Sigma, \ed t^2 + e^{2t} g_0)$ for some
closed $3$-manifold $\Sigma$ carrying a flat metric $g_0$. In fact,
 $\Sigma$ will occur as a hypersurface in $M$, obtained by an approximation scheme involving $\mu$-bubbles (Sections~\ref{tauketc} and \ref{muk_in_Ek}); then we show that $\Sigma$ must be compact and that $(M,g)$ is isometric to the desired warped product (Section~\ref{Sigmak_limits}). 
 
The reader is recommended to consult Appendix~\ref{mubbSec} before proceeding.

\subsection{Specification of $\mu_k$ and $E_k$}\label{tauketc}

Since $M$ is non-compact with compact boundary, there exists a smooth, proper map $\rho: M\rightarrow (-\infty,0]$ (see \cite[Lemma 2.1]{Zhu2020}) such that 
$$
	\rho^{-1}(0)=\partial M,\quad |\ed\rho|_g<1.
$$ 
Fix a smooth function $\eta\in C^\infty((-\infty,0])$ satisfying 
$$
	\eta(t)=0 \quad\mbox{for any } t\le -1 \quad \mbox{and}\quad \eta (0)=2;
$$  
define $\tau_k$ by 
$ 3\coth(2\tau_k) = 3+ k^{-1}$,
and then define $\hat\mu_{k}: (-\tau_k,0]\rightarrow \mathbb{R}$ by 
$$ 
	\hat\mu_{k}(t)=3\coth(2(t+\tau_k))-k^{-1}\eta(t).
$$
Thus, $\{\tau_k\}_{k = 1}^\infty$ is increasing and tends to infinity, and
$$
	\hat\mu_{k}(-\tau_k)=+\infty\quad\mbox{and}\quad \hat\mu_{k}(0)=3-k^{-1}.
$$
Now, choose $a_k$, regular values of 
$\rho$, such that $\tau_k\le a_k < \min\{\tau_{k+1}, \tau_k + 1\}$, and then define $E_{k}:=\rho^{-1}([-a_{k},0])\subset M$. 
Denote $\partial_{k}^- :=\rho^{-1}(-a_{k})$, 
which are smooth hypersurfaces of $M$. This makes $(E_k, g|_{E_k}; \partial_k^-,\partial M)$
a Riemannian band.
Finally, let $\rho_{k}:=(\tau_k/a_k)\rho$, and define
 \[
 	\mu_k := \hat\mu_k\circ\rho_k.
\] By this arrangement, $\mu_k|_{\partial_k^-} = \infty$.

\subsection{$\mu_k$-bubbles in $E_k$}
\label{muk_in_Ek}
For each fixed $k$, consider $(E_k, g|_{E_k}; \partial_k^-, \partial M)$.
Note that $H_{\partial M}\ge 3$; by construction, $\mu_k$ satisfies
the barrier condition (see Definition~\ref{barrierdef}).
By Fact~\ref{mubbExist}, a smooth $\mu_k$-bubble $\Omega_k$ exists. 
Define $\Sigma_k := \partial\Omega_k \setminus \partial_k^-$, which is smooth, closed, and separates $\partial_k^-$ from $\partial M$.

The following lemma shows that all $\Sigma_k$ must meet a
fixed compact subset of $M$.

\begin{lemma}\label{intersectionLemma}
	Let $\Kcal:= \{x\in M: \dist_g(x,\partial M)\le 10\}$.  
	Then $\Sigma_k\cap \Kcal\ne \emptyset$.
\end{lemma}
\begin{proof}
Suppose on the contrary that $\Sigma_{k}\cap \Kcal = \emptyset$. This implies that
$\eta\circ\rho_k = 0$ on $\Sigma_k$. 
Moreover, by assumption, $R_g\ge -12$, and by construction, $|\ed\rho_k|_g < 1$. Thus, we have (see \eqref{RplusDef})
$$R_{+}^{\mu_{k}}> -12+\frac{4}{3}[3\coth(2(\rho_{k}+\tau_k))]^{2}-12[\sinh(2(\rho_{k}+\tau_k)]^{-2}=0\quad  \mbox{ on }  \Sigma_k.
$$
This, along with Fact~\ref{confPSC}, implies that $\Sigma_k$ admits a PSC metric; 
since $\Sigma_k$ is separating, we get a contradiction, by Lemma~\ref{4dNSepLemma}.
\end{proof}



\subsection{Convergence of $\Sigma_k$} 
\label{Sigmak_limits}
By using \cite[Theorem 3.6]{ZZ20}, one can show that the
 second fundamental form $\II_{\Sigma_k}$ 
is uniformly bounded within any compact subset of $M$.
Thus, by Lemma~\ref{intersectionLemma}, $\Sigma_k$ subconverges to a smooth hypersurface $\Sigma$ in $M$ (for convenience, denote the subsequence by the same symbol $\Sigma_k$). Within
compact subsets of $M$, the convergence is uniform and has multiplicity one; 
moreover, $\Sigma$ bounds a `minimizing $3$-bubble' for which minimality is interpreted
with respect to compactly supported
perturbations {(cf. \cite[Lemma 4.10]{HLS22})}. Depending on whether $\Sigma$ is compact, we consider the
two cases below.
\vskip 2mm
\noindent\underline{\emph{Case 1: $\Sigma$ is compact}}. By minimality, we have (see Fact~\ref{mubbVarProp})
\begin{equation}\label{stability_cptCase}
	H_{\Sigma}=3\quad
	\mbox{and}\quad L_{\Sigma}=-\Delta_{\Sigma}+\frac{1}{2}(R_{\Sigma}-R_{+}^{3})\geq 0.
\end{equation}
Since $R_{+}^{3}=R_{g}+12\geq 0$, \eqref{stability_cptCase} implies that
$-\Delta_{\Sigma}+\frac{1}{2}R_{\Sigma}\ge 0$; thus, there exists
a smooth function $u>0$ defined on $\Sigma$ and a constant $\lambda\ge0$ 
such that
\begin{equation}\label{eigenfcn}
	\left(-\Delta_{\Sigma}+\frac{1}{2}R_{\Sigma}\right)u=\lambda u.
\end{equation}	
Define $\tilde{g}_{\Sigma}=ug_{\Sigma}$ where 
$g_\Sigma$ is the metric on $\Sigma$ induced by $g$. We have
\begin{equation}\label{tildegScal}
R_{\tilde{g}_{\Sigma}}=u^{-1}\left(R_{{\Sigma}}+\frac{3}{2}\left|\frac{\nabla u}{u}\right|^{2}-2\frac{\Delta u}{u}\right)=u^{-1}\left(2\lambda+\frac{3}{2}\left|\frac{\nabla u}{u}\right|^{2}\right)\geq 0.
\end{equation}

Since each $\Sigma_k$ is separating, so is $\Sigma$. By Lemma~\ref{4dNSepLemma},
 $\Sigma$ admits no PSC metric; then by \eqref{tildegScal} and
 the trichotomy theorem of Kazdan and Warner, 
 $R_{\tilde g_\Sigma}= 0$. Thus,  $\lambda$ must vanish, and  $u$ must be a constant; \eqref{eigenfcn} in turn implies that $R_{\Sigma}=0$. %and  \eqref{stability_cptCase} enforces that $R_g = -12$ on $\Sigma$.
Then by Bourguignon's theorem (see \cite[Lemma 5.2]{KW75}), 
$g_\Sigma$ is Ricci-flat, which must be flat since $\dim \Sigma=3$.

Now we prove that a neighborhood of $\Sigma$ splits.
When $\Sigma\cap\partial M = \emptyset$, since $\Sigma$ is the boundary of minimizing $3$-bubble, 
\cite[Theorem 2.3]{ACG08} implies that  there exists an open neighborhood of $\Sigma$ that is isometric to a warped product $((-\epsilon,\epsilon)\times\Sigma, \ed t^{2}+e^{2t}g_{\Sigma})$, where $t$ is the coordinate on $(-\epsilon,\epsilon)$ and $\Sigma$ corresponds to $t = 0$.  When $\Sigma\cap\partial M\ne \emptyset$, we must have $\Sigma = \partial M$, by the maximum principle.  
In this case, the proof of \cite[Theorem 2.3]{ACG08} still applies and gives an open neighborhood of $\Sigma$ that is isometric to a warped product $((-\epsilon,0]\times\Sigma, \ed t^{2}+e^{2t}g_{\Sigma})$.

Thus, a neighborhood of $\Sigma$ is foliated by the $t$-level sets. Note that moving along the foliation leaves the energy functional invariant; thus, each $t$-slice also bounds a minimizing $3$-bubble, to which the same analysis above applies.

This implies that a maximal neighborhood $\Ucal$ of $\Sigma$ on which the metric splits as
\[
	(I\times\Sigma, \ed t^{2}+e^{2t}g_{\Sigma})
\]  must be both 
open and closed in $M$. By connectedness, $\Ucal = M$, and $I$ must be of the form $(-\infty, c]$.
This achieves the desired splitting.
 \vskip 2mm
 
\noindent\underline{\emph{Case 2: $\Sigma$ is non-compact}}. By finding a contradiction, we prove that this case does not occur. The argument largely follows the proof of \cite[Theorem 1.1]{Zhu2020}, so we only sketch the steps. 

Let 
$$\left(M_k, g_k\right)=(\Sigma_{k}\times \S^{1}, g_{\Sigma_{k}}+u_{k}^{2}dt^{2}),$$
where $u_{k}$ is the first eigenfunction of $L_{\Sigma_{k}}$; that is, $L_{\Sigma_{k}}u_{k}=\lambda_{k}u_k$ with $\lambda_{k}\geq 0$. 
Since $\dim \Sigma_k = 3$,  \cite[Corollary 1.10]{CRZ2022} implies that $M_k$ admits no PSC metric.

Now
\begin{equation}\label{Rgk}
R_{g_{k}}=R_{g_{\Sigma_{k}}}-2\frac{\Delta_{g_{\Sigma_{k}}}u_{k}}{u_{k}}=R_{+}^{\mu_{k}}+2\lambda_{k}.
\end{equation}
By construction, $R_{+}^{\mu_{k}} \ge 0$ outside $\Kcal$, and
there exist $\delta_k>0$, satisfying $\lim\delta_k= 0$, such that $R_{+}^{\mu_{k}}\geq -\delta_{k}$ on $M$. 
Since $R_{g_k}$ cannot be positive and $\lambda_k\ge 0$, by \eqref{Rgk}, we must have $\lim \lambda_{k}= 0$. 

Next, choose $q_{k}\in \Sigma_{k}\cap \Kcal$ so that $\lim q_{k} = q\in\Sigma$, and let $p_{k}=(q_{k},t_{0})\in \Sigma_k\times \S^1$ and $p=(q,t_{0})\in \widetilde M=\Sigma\times \S^{1}$. 
Normalize $u_{k}$ such that $u_{k}(q_{k})=1$. By the Harnack
inequality, $u_k$ converges smoothly to a positive function $u$ on $\Sigma$ with $u(q)= 1$. 
Thus, $(M_k, g_k)$ converges in the pointed smooth topology to $(\widetilde M, \widetilde g)$, where $\widetilde g = g_{\Sigma}+ u^2\ed t^2$.

Now one can follow the proof\,%
\footnote{The proof of \cite[Proposition 3.2]{Zhu2020} only relies on $\widetilde M$ admitting no PSC metric and the properties of $R^{\mu_k}_+$ mentioned above.} 
of \cite[Proposition 3.2]{Zhu2020} to show
that $\Ric_{\widetilde g} = 0$,
and then follow the proof\,%
\footnote{In particular, the boundedness of  $\area(\Sigma)$ follows from
$
	\Acal^{\mu_k}_{\refOmega}(\Omega_{k})\leq \Acal^{\mu_k}_{\refOmega}(E_{k})
$ and $\mu_{k}>0$.} 
of \cite[Theorem 1.1]{Zhu2020} to show that $u$ is constant, which
implies $\Ric_{g_{\Sigma}}= 0$.

In summary, $(\Sigma, g_\Sigma)$ is complete, non-compact, Ricci-flat, and with finite area; this contradicts \cite[p.25: Theorem 4.1]{Yau94}.
\qed


\begin{remark}\label{warpProdRmk}
The PSC obstruction, provided by \cite[Corollary 1.10]{CRZ2022}, for manifolds of the form $\Sigma\times \S^1$ only works when $\dim\Sigma \ne 4$. On the other hand, if $\Sigma$ 
$(2\le\dim\Sigma\le 6)$ is closed, orientable, and if it admits a map of nonzero degree to some 
$\Sigma'\in \Ccal_{deg}$, then by a similar argument as \cite[Theorem 1.1]{CLSZ2021}, one can show that $\Sigma\times \S^{1}$ admits no PSC metric. 
 \end{remark}


\begin{proof}[Proof of Theorem~\ref{alhpmt1}]
The inequality
$$\inf_{\partial M}H \le n-1$$
follows directly from Proposition \ref{incprssArg}. 

To prove the second part of the theorem, first obtain a covering $(\hat M, \hat g)$ of
$(M,g)$ as in the proof of Proposition~\ref{incprssArg}, and then apply (essentially)
the same proof of Theorem~\ref{alhpmt2} to $(\hat M,\hat g)$; to assist the reader, 
we list a few points that may need attention.
\begin{itemize}
	\item{$\partial \hat M$ may not be connected, but Riemannian bands can still be constructed in a similar manner as in
			the proof of Proposition~\ref{incprssArg}. To avoid clash of symbols, denote $S:=\partial M$ and let $\hat S$ be 
			a fixed lifting of $S$ in $\hat M$. Thus $\partial_+ = \hat S$ and $\partial_\star\subset\partial_-$;
			$\mu_k >0$ can be defined such that $\mu_k|_{\partial_\star} = \infty$ and $\mu_k|_{\hat S} = (n-1) - 1/k$;
			on $\partial_-\setminus\partial_\star$ (if nonempty) we have $H \ge n-1$; one can check that
			the barrier condition is satisfied, and the $\Sigma_k$'s exist; restricting $j: \hat M\rightarrow S$ to $\Sigma_k$
			yields a map $\Sigma_k\rightarrow\hat S$ of nonzero degree.}
	\item{An adapted version of Lemma~\ref{intersectionLemma} holds; in the proof, invoke Lemma~\ref{C_degNSepLemma} 
	instead of Lemma~\ref{4dNSepLemma}. It follows that $\Sigma_k$ converges to some $\Sigma$.}		
	\item{When $\Sigma$ is compact, the corresponding part in Section~\ref{Sigmak_limits} applies, apart from dimensional adjustments
		and the fact that Ricci-flatness may no longer imply flatness.}
	\item{When $\Sigma$ is non-compact, we need to argue, without relying
	 on \cite[Corollary 1.10]{CRZ2022},  that $M_k = \Sigma_k\times\S^1$ admits no PSC metric, and this is already addressed by Remark~\ref{warpProdRmk}.}
\end{itemize}
The consequence is that
 $(\hat M, \hat g)$ is of the form 
 \[
 	((-\infty, 0]\times \Sigma, \ed t^2+e^{2t}g_\Sigma)
\]
where
  $g_\Sigma$ is Ricci-flat. In particular, the covering $\hat M\rightarrow M$ is $1$-fold and hence an isometry. 
  Since $\Sigma = \partial M$ is assumed to be aspherical, 
  $g_\Sigma$ must be flat, which can be seen by applying the Cheeger--Gromoll splitting theorem
  to the universal cover; for details, see the beginning paragraph of \cite[Section~6]{CL20}.
\end{proof}



\appendix
\section{$\mu$-bubbles}\label{mubbSec}

This section collects some `Definitions' and `Facts' concerning 
the $\mu$-bubble technique, about which we make no claim to originality.
For detailed expositions and proofs, 
the reader may consult \cite{CL20}, \cite{CRZ2022}, \cite[Section 5]{Gr2021}, \cite{Zhu21warp} and
\cite{ZZ20}.
This section also includes three supplementary `Lemmas'.

A common setting for $\mu$-bubbles is a \emph{Riemannian band}, namely
a compact, connected Riemannian manifold $(M^n,g)$ whose (nonempty) boundary
is expressed as a disjoint union $\partial M = \partial_-\sqcup \partial_+$,
where each of $\partial_\pm$ is a smooth, closed and possibly disconnected
$(n-1)$-manifold.

Given a Riemannian band $(M^n,g;\partial_-,\partial_+)$
and a function $\mu\in C^\infty(\mathring{M})$,
consider the following variational problem: Let $\Omega_0$ be a smooth open neighborhood of $\partial_-$; among all Caccioppoli sets 
$\Omega\subset M$ that satisfy $\partial_-\subset \Omega$ and $\Omega\Delta\Omega_0 \Subset \mathring{M}$, seek a \emph{minimizer} of the functional
\begin{equation}\label{braneAct}
	\Acal_{\refOmega}^{\mu}(\Omega)=\Hcal^{n-1}(\partial \Omega)-\Hcal^{n-1}(\partial\refOmega)-\int_{M}(\chi_{\Omega} - \chi_{\refOmega})\mu\, d\Hcal^n
\end{equation}
where $\Hcal^k$ is the induced $k$-dimensional Hausdorff measure, and
$\chi_\Omega, \chi_{\refOmega}$ are characteristic functions.
Such a minimizer is called a \emph{$\mu$-bubble}.

Existence and regularity of $\mu$-bubbles are well-established when $\mu$
satisfies the following `barrier condition'.
\begin{definition}\label{barrierdef}
	Let $(M^n,g;\partial_-,\partial_+)$ be a Riemannian band.
	A function $\mu\in C^\infty(\mathring{M})$ is said to satisfy the
	\emph{barrier condition} if, for each connected component $S\subset\partial_+$ (resp., $S\subset \partial_-$), 
	\begin{itemize}
	\item either $\mu$ smoothly extends to $S$ and satisfies
					$H_S > \mu|_S$	(resp., $H_S > -\mu|_S$),
	where $H_S$ is
	the mean curvature of $S$ with respect to the outward normal;
	\item or $\mu\to -\infty$  (resp., $\mu\to +\infty$) towards $S$.
	\end{itemize}
\end{definition}

\begin{fact}\label{mubbExist}
	For $3\le n\le 7$, if $\mu\in C^\infty(\mathring{M})$ satisfies the barrier condition, then 
	there exists a smooth $\mu$-bubble $\Omega$.
	In particular, $\partial\Omega \setminus \partial_-$ is homologous to 
	$\partial_+$ and is separating (see Definition~\ref{noSepDef} below).
\end{fact}

Also well-known are the following variational properties. To fix notation,
let $\Sigma$ denote the hypersurface $\partial\Omega\setminus \partial_-$
with outward unit normal $\nu$; let $R_\Sigma$ and $\Delta_\Sigma$ be, respectively, the scalar
curvature and the Laplacian along $\Sigma$ (with the induced metric);
 let $H_\Sigma$ and $\II$ be, respectively, the mean curvature and 
the second fundamental form of $\Sigma$, computed with respect to $\nu$; 
define the operators 
\begin{equation}\label{JDef}
	J_\Sigma = -\Delta_\Sigma + \frac{1}{2}(R_\Sigma - R_g - \mu^2 - |\II|^2) - \nu(\mu)
\end{equation}
and
\begin{equation}\label{LDef}
	L_\Sigma =  - \Delta_\Sigma + \frac{1}{2}(R_\Sigma - R^\mu_+)
\end{equation}
where
\begin{equation}\label{RplusDef}
	R^\mu_+ = R_g + \frac{n}{n-1}\mu^2 - 2|\ed\mu|_g.
\end{equation}
\begin{fact}\label{mubbVarProp}
	Suppose that $\Omega$ is a smooth $\mu$-bubble. 
	We have
	\begin{enumerate}[\quad\rm (a)]
		\item $H_\Sigma = \mu|_\Sigma$;
		\item $L_\Sigma \ge J_\Sigma\ge 0$.
	\end{enumerate}
\end{fact}

The semi-positivity of $L_\Sigma$ has several applications, and we shall list a few. 
To start with, let $u > 0$ be an eigenfunction associated to the first eigenvalue
$\lambda\ge 0$ of $L_\Sigma$. Consider the warped-product metric 
$\hat h :=g_\Sigma + u^2\ed \theta^2$ defined on $\hat \Sigma := \Sigma\times \S^1$
where $\theta\in \S^1$. 
\begin{fact}\label{warpPSC}
	Suppose that $\Omega$ is a smooth $\mu$-bubble.
	The scalar curvature of $(\hat\Sigma,\hat h)$
	is 
	\begin{equation}
		R_{\hat h} = R_\Sigma - 2u^{-1}\Delta_\Sigma u = R^\mu_+ +  2\lambda.
	\end{equation}
	In particular, if $R^\mu_+ >0$ on $\Sigma$, then $\Sigma\times \S^1$ 
	admits a PSC metric.
\end{fact}

Alternatively, one can compare $L_\Sigma$ with the conformal Laplacian on $\Sigma$ and obtain the following.
\begin{fact}\label{confPSC}
	For $n\ge 3$, suppose that $\Omega$ is a smooth $\mu$-bubble on which
	$R^\mu_+ >0$. Then $\Sigma$ admits a PSC metric.
\end{fact}

With additional topological assumptions on $M$, Fact~\ref{confPSC} can be used
to prove width estimates for $(M,g)$. To be precise, we start by recalling the following notion (cf. \cite[`Property A']{CRZ2022}).
\begin{definition}\label{noSepDef}
	Given a (topological) band $(M^n;\partial_-, \partial_+)$, we say that 
	a (closed) hypersurface $\Scal$ in $M$ is 
	\emph{separating}, if all paths connecting
	$\partial_-$ and $\partial_+$ must intersect $\Scal$.
	A band is said to satisfy
	the \emph{\Nsep property} if no separating hypersurface admits a PSC metric.
\end{definition}

\begin{remark}\label{sepRmk}
	If $\Scal\subset M^n$ is a separating hypersurface, then there exists a 
	minimal list of connected components $S_i$ $(i = 1,\ldots, k)$ of $\Scal$
	such that their union $\Scal'$ remains separating. 
	For details, see \cite[Lemma 2.2]{CRZ2022}. 
	Using intersection theory,
	one can show that
	$[\Scal']\ne 0\in H_{n-1}(M;\Z)$. Moreover, with suitable orientation, $\Scal'$ is homologous to $\partial_{+}$ in $M$.
\end{remark}

\begin{lemma}\label{C_degNSepLemma}
	Let $(M^n, g; \partial_-, \partial_+)$ be a 
	Riemannian band, and let $\iota: \partial_+\hookrightarrow M$ be the inclusion map.
Suppose that $\partial_+ \in \mathcal{C}_{deg}$ (see Definition~\ref{cdegdef}) and that there exists a continuous map $f: M\rightarrow \partial_+$ such that $f\circ\iota$ is homotopic to  $\id_{\partial_+}$. Then 
$(M,g)$ satisfies the \Nsep property.
\end{lemma}
\begin{proof}Suppose that $\Scal$ is a separating hypersurface in $M$,  and  let $\Scal'$ be as in Remark \ref{sepRmk}; in particular,
 $\Scal'$  is homologous to $\partial_+$ in $M$. Now since $f\circ\iota$ is homotopic to $\id_{\partial_+}$,  it is easy to see that the restriction $f|_{\Scal'}:\Scal'\rightarrow \partial M$ has degree $1$. Since $\partial_+\in \mathcal{C}_{deg}$, $\Scal'$ admits no PSC metric.
\end{proof}

The next lemma is a variant of Gromov's band-width estimate \cite[Section 5.3]{Gr2021}.

\begin{lemma}\label{bandWidth}
	For $3\le n\le 7$, let $(M^n,g;\partial_-,\partial_+)$ be a Riemannian band 	that satisfies
	the \Nsep property, and let 
	$\partial_\star\subset \partial_-$ be a compact subset without boundary.
	Suppose that
	\begin{enumerate}[\quad\rm (a)]
		\item $R_g \ge -n(n-1)$;
		\item $H_{\partial_-\setminus\partial_\star} \ge -(n-1)$;
		\item $H_{\partial_+} \ge (n-1)( 1+ \delta)$ for some constant $\delta>0$.
	\end{enumerate}
	Then there exists a constant $T_\delta>0$, depending only on $\delta$, such that
	\begin{equation}\label{mixedWthEst}
		\dist_g(\partial_\star,\partial_+) 
		\le T_\delta.
	\end{equation}	
\end{lemma}
\begin{proof}
	Set  $\epsilon = \delta/3$, 
	and define 
	$C_\delta, T_\delta>0$ by 
	\[
		  \coth(C_\delta/2) = \frac{1+\delta/2}{1+\epsilon}
		  \quad\mbox{and}
		  \quad T_\delta = \frac{C_\delta}{n(1+\epsilon)}.
	\]
	For the sake of deriving a contradiction, suppose that
	$\dist_g(\partial_\star, \partial_+) > T_\delta$.
	By the proof of \cite[Lemma 4.1]{Zhu21warp}, there exists a smooth, proper function 
	$\rho: M\rightarrow [-T_\delta, 0]$
	such that
	\begin{equation}\label{rhofcnprop}
		\rho^{-1}(-T_\delta) = \partial_\star, \quad \rho^{-1}(0) = \partial_+,
		\quad\mbox{and}\quad |\ed\rho|_g < 1.
	\end{equation}
	Now consider the function
	\[
		h(t)= (n-1)(1+\epsilon)\coth\left(\frac{n(1+\epsilon)t + C_\delta}{2}\right),
		\quad t\in (-T_\delta, 0].
	\]
	By construction, $h$ is decreasing, strictly greater than $n-1$, and satisfies
	\begin{equation}\label{hfcnprop}
		h(0) < H_{\partial_+}, \quad \lim_{t\rightarrow -T_\delta} h(t) = \infty,
		\quad \frac{n}{n-1}h(t)^2 + 2h'(t) \equiv n(n-1)(1+\epsilon)^2.
	\end{equation}
	Combining \eqref{rhofcnprop}, \eqref{hfcnprop}, and the assumptions (a,b,c), 
	one can easily check that the 
	function $\mu:=h\circ\rho$, defined on $M\setminus \partial_\star$, 
	satisfies both
	the barrier condition and the inequality $R^\mu_+ >0$.
	By Facts~\ref{mubbExist} and \ref{confPSC}, there exists a separating hypersurface $\Sigma$ in $(M;\partial_-,\partial_+)$ that admits a PSC metric. This contradicts the \Nsep hypothesis.
\end{proof}

\begin{remark}\label{bandNonexistence}
	We mention two variants of Lemma~\ref{bandWidth}, both of which can be obtained 
	by slightly modifying the proof above.
	{\rm (A)} For $3\le n\le 7$,
	no Riemannian band can simultaneously satisfy the \Nsep property and 
	the conditions
	$R_g \ge -n(n-1)$, $H_{\partial_-} \ge -(n-1)$ and $H_{\partial_+}> n-1$.
	{\rm (B)} For $3\le n\le 7$, let $(M^n,g)$ be a complete, non-compact Riemannian
	manifold with compact boundary $\partial M$. Suppose that 
	$M$ satisfies the \Nsep property (see below); then $(M,g)$ cannot satisfy
	the conditions $R_g \ge -n(n-1)$ and $H_{\partial M} > n-1$ simultaneously.
\end{remark}

The concept of separating hypersurfaces can also be defined for complete, non-compact Riemannian manifolds $(M,g)$ with compact boundary---just require that $\Scal$ intersects with all paths connecting $\partial M$ and infinity. The
\Nsep property can be extended to such manifolds.

\begin{lemma}\label{4dNSepLemma}
	Let $(M^4, g)$ be a complete, non-compact Riemannian $4$-manifold with compact (nonempty) boundary $\partial M$.
Suppose that the homotopy groups $\pi_2(M) = \pi_3(M) = 0$. Then 
$(M,g)$ satisfies the \Nsep property.
\end{lemma}
\begin{proof}
	Suppose that $\Scal\subset M$ is a (closed) separating hypersurface that
	admits a PSC metric, and let $\Scal'\subset \Scal$ be as indicated 
	in Remark~\ref{sepRmk}. In particular, $\Scal'$ admits a PSC metric, and
	 $[\Scal']\ne 0\in H_3(M,\Z)$.
	Since $\pi_2(M)$ is trivial, the topological classification of 
	closed 3-manifolds admitting 
	a PSC metric implies that 
	$\Scal'$ is homologous to a spherical class in $H_{3}(M,\Z)$ 
	(see \cite[p. 112]{WangJthesis}). Since $\pi_3(M)$ is also trivial, this violates
	Lemma~\ref{topLemma} below.
\end{proof}


\section{Topological lemmas}

\begin{lemma}\label{topLemma}
	Let $M$ be a non-compact $4$-manifold satisfying $\pi_3(M) = 0$. Then 
	$H_3(M,\Z)$ contains no nontrivial spherical class (i.e., classes of the form
	$[\S^3/\Gamma]$).
\end{lemma}
\begin{proof}
	Let $[\beta]$ denote the fundamental class of $\S^3/\Gamma$
	where $\Gamma$ is a discrete subgroup of $O(4)$. Let $i:\S^3/\Gamma\rightarrow M$ be a continuous map. The goal is 
	to prove that $i_*[\beta] = 0\in H_3(M,\Z)$. Now let $[\alpha]$
	be the fundamental class of $\S^3$. The composition 
	$\S^3\xrightarrow{\pi}\S^3/\Gamma\xrightarrow{i} M$ induces a 
	map at the level of $H_3(\cdot, \Z)$, such that
	$[\alpha]\xrightarrow{\pi_*} d[\beta] \xrightarrow{i_*}di_*[\beta]$
	where $d$ is the degree of $\pi$. Since $\pi_3(M) = 0$, Hurewicz homomorphism
	implies that \[
		di_*[\beta] = (i\circ\pi)_*[\alpha] = h([i\circ\pi])= 0\in H_3(M,\Z)
	\]
	where $h:\pi_3(M)\rightarrow H_3(M,\Z)$ is the Hurewicz map. Thus,
	in order to show that $i_*[\beta] = 0$, it suffices to show that 
	$H_3(M,\Z)$ is torsion free, and this follows from $M$ being non-compact
	(see \cite[Corollary 7.12]{bredon}).
\end{proof}


\begin{lemma}\label{incprLemma}
	For $1\le k\le n-2$, let $M_1 = (\R\times \T^{n-k-1} - \Bb)\times \T^k$ where $\Bb$ is an embedded $(n-k)$-ball in $\R\times \T^{n-k-1}$.
	Let $M_2$ be a smooth, possibly non-compact, manifold with boundary $\partial M_2$.
	Suppose that  $\Phi: \partial M_1 \rightarrow \partial M_2$ is a diffeomorphism, and let 
	$M:= M_1\sqcup_\Phi M_2$ be the manifold obtained by identifying $\partial M_1, \partial M_2$ via $\Phi$.
	Let $t\in \R$ be such that $\{t\}\times \T^{n-k-1}$ is disjoint from $\Bb$.
	Then the hypersurface $\Sigma = \{t\}\times \T^{n-1}$ is incompressible in $M$ 
	if and only if the $\T^k$-factor\,\footnote{Note that $\partial M_2$ has the product structure $\S^{n-k-1}\times\T^k$ induced by $\Phi$.} 
	of $\partial M_2$ is incompressible in $M$.
\end{lemma}
\begin{proof}
	In $M$, the $\T^k$-factor of $\Sigma$ is homotopic to that of $\partial M_1$ and hence to that of $\partial M_2$. Thus, $(\Rightarrow)$
	is clear. 
	
	For $(\Leftarrow)$, we prove its contrapositive. Suppose that
	 $L\subset \Sigma$ is a non-contractible loop that is contractible in $M$. Write
	 \[
	 	[L] = (m_i\alpha_i, n_j\beta_j) \in\pi_1(\T^{n-k-1})\times \pi_1(\T^{k})\cong \pi_1(\Sigma)
	 \]
	 where $\alpha_i$ generates the fundamental group of the $i$-th $\S^1$-factor in $\T^{n-k-1}$ and $m_i\in \Z$,
	 similarly for $\beta_j$ and $n_j$.
	 Let us write $\hat\alpha_i$, $\hat \beta_j$ for the corresponding elements in the homology class $H_1(\Sigma;\Z)$.
	
	 It will be convenient to view the $\R$-factor in $\R\times \T^{n-k-1}$ as $\S^1$ minus a point, and to view
	 $M$ as a subset of $\hat M:= \hat M_1\sqcup_\Phi M_2$, where  $\hat M_1 := (\S^1\times \T^{n-k-1} - \Bb)\times \T^k$.
		
	 Let $\iota:\Sigma\hookrightarrow \hat M$ be the inclusion map. 
	 For $1\le i\le n-k-1$, let $\theta_i$ be the coordinate on the $i$-th $\S^1$-factor of $\T^{n-k-1}$. By construction, there exists
	 $t_i\in \S^1$ such that $\theta_i = t_i$ defines a hypersurface $S_i$ in $\hat M$ that is `dual' to $\iota_*\hat\alpha_i$, in the sense that
	 the intersection products
	 \[
	 	[S_i]\cdot \iota_*\hat \alpha_i = 1\quad \mbox{and}\quad  [S_i]\cdot \iota_*\hat\alpha_{i'} =[S_i]\cdot \iota_*\hat\beta_j = 0
		\quad(i'\ne i).
	\]
	Since $L$ is contractible in $M\subset \hat M$, we have 
	\[
		\sum_i m_i \iota_*\hat \alpha_i + \sum_j n_j\iota_*\hat \beta_j = 0 \in H_1(\hat M;\Z);
	\]
	by taking intersection products with $[S_i]$, we see that $m_i = 0$ for all $i = 1,\ldots, n-k-1$,
	so $L$ is homotopic to a loop in the $\T^{k}$-factor of $\Sigma$. Thus, the $\T^k$-factor of $\Sigma$
	is not incompressible in $M$. By homotopy, the same is true for the $\T^k$-factor of $\partial M_2$.
	This completes the proof.
\end{proof}

\section*{Acknowledgement}

We thank Shihang He for kindly sharing his proof of Lemma~\ref{topLemma}.

T. Hao, P. Liu and Y. Shi are funded by the National Key R\&D Program of China Grant 2020YFA0712800,
and Y. Hu is funded by the China Postdoctoral Science Foundation Grant 2021TQ0014. 


\bibliography{rigidalh.bib}

\bibliographystyle{alpha}
\end{document}
