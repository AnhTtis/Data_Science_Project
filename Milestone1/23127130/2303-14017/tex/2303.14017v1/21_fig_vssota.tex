
\begin{figure*}[t]
	\centering
    \setlength{\tabcolsep}{1pt}
 	\resizebox{0.8\linewidth}{!}{
	\renewcommand\arraystretch{1.47}
	\begin{tabular}{l@{\hspace{0.3in}}l@{\hspace{-0.2in}}|l@{\hspace{0.0in}}}
% 		\toprule 
        \hline
        \multirow{9}*{\rotatebox[origin=c]{90}{Seen Fonts}} & \makebox[0.11\linewidth][l]{Source} &   \multirow{9}*{\includegraphics[width=0.89\linewidth]{Figs/exp/vs_sota/Poem_Seen_vssota_ori_v3_box.pdf}}\\
        &\makebox[0.11\linewidth][l]{FUNIT} & \\
		&\makebox[0.11\linewidth][l]{LF-Font} & \\
        &\makebox[0.11\linewidth][l]{MX-Font} & \\
        &\makebox[0.11\linewidth][l]{Fs-Font} & \\
        &\makebox[0.11\linewidth][l]{CG-GAN} & \\
        &\makebox[0.11\linewidth][l]{DG-Font} & \\
        &\makebox[0.11\linewidth][l]{CF-Font} & \\
        &\makebox[0.11\linewidth][l]{Target} & \\

        \hline
        \multirow{9}*{\rotatebox[origin=c]{90}{Unseen Fonts}} & \makebox[0.11\linewidth][l]{Source} &   \multirow{9}*{\includegraphics[width=0.89\linewidth]{Figs/exp/vs_sota/Poem_Unseen_vssota_ori_v3_box.pdf}}\\
        &\makebox[0.11\linewidth][l]{FUNIT} & \\
		&\makebox[0.11\linewidth][l]{LF-Font} & \\
        &\makebox[0.11\linewidth][l]{MX-Font} & \\
        &\makebox[0.11\linewidth][l]{Fs-Font} & \\
        &\makebox[0.11\linewidth][l]{CG-GAN} & \\
        &\makebox[0.11\linewidth][l]{DG-Font} & \\
        &\makebox[0.11\linewidth][l]{CF-Font} & \\
        &\makebox[0.11\linewidth][l]{Target} & \\
        \hline
	\end{tabular}
	}
	\vspace{-2mm}
    \caption{Qualitative comparison with state-of-the-art methods \crc{on Chinese poems}. As mentioned earlier, we use multiple source fonts and pick the best results for these comparison methods for fairness. Here we just plot font \emph{Song} as an example of source fonts for convenience. \wc{We mark erroneous skeletons with red boxes and other mismatch styles, such as stroke style, joined-up style, and body frame~\cite{lu2016elements}, with blue boxes.}}
	\label{fig:vs_sota_big}
	\vspace{-10pt}
\end{figure*}
