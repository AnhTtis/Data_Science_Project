% CVPR 2023 Paper Template
% based on the CVPR template provided by Ming-Ming Cheng (https://github.com/MCG-NKU/CVPR_Template)
% modified and extended by Stefan Roth (stefan.roth@NOSPAMtu-darmstadt.de)

\documentclass[10pt,twocolumn,letterpaper]{article}

%%%%%%%%% PAPER TYPE  - PLEASE UPDATE FOR FINAL VERSION
% \usepackage[review]{cvpr}      % To produce the REVIEW version
\usepackage{cvpr}              % To produce the CAMERA-READY version
%\usepackage[pagenumbers]{cvpr} % To force page numbers, e.g. for an arXiv version

% Include other packages here, before hyperref.
\usepackage{graphicx}
\usepackage{amsmath}
\usepackage{amssymb}
\usepackage{booktabs}

%
\usepackage{algorithm}
\usepackage{algorithmic}
\usepackage{graphicx}
\usepackage{multirow}
\usepackage{tabularx}
% TO REMOVE
\usepackage{soul}
%

\usepackage[accsupp]{axessibility}  % Improves PDF readability for those with disabilities.

\newcommand{\TODO}[1]{\textcolor[rgb]{0,0.5,0.5}{TODO: #1}}
\newcommand{\blue}[1]{\textcolor[rgb]{0,0,0.9}{#1}}
% \newcommand{\crc}[1]{\textcolor[rgb]{0.9,0,0}{#1}}
\newcommand{\crc}[1]{{#1}}
% \newcommand{\wc}[1]{\textcolor[rgb]{0.9,0,0}{#1}}
% \newcommand{\yunqi}[1]{\textcolor[rgb]{1,0.5,0}{#1}}
\newcommand{\wc}[1]{#1}
\newcommand{\yunqi}[1]{#1}

\urlstyle{same}

% It is strongly recommended to use hyperref, especially for the review version.
% hyperref with option pagebackref eases the reviewers' job.
% Please disable hyperref *only* if you encounter grave issues, e.g. with the
% file validation for the camera-ready version.
%
% If you comment hyperref and then uncomment it, you should delete
% ReviewTempalte.aux before re-running LaTeX.
% (Or just hit 'q' on the first LaTeX run, let it finish, and you
%  should be clear).
\usepackage[pagebackref,breaklinks,colorlinks]{hyperref}


% Support for easy cross-referencing
\usepackage[capitalize]{cleveref}
\crefname{section}{Sec.}{Secs.}
\Crefname{section}{Section}{Sections}
\Crefname{table}{Table}{Tables}
\crefname{table}{Tab.}{Tabs.}


%%%%%%%%% PAPER ID  - PLEASE UPDATE
\def\cvprPaperID{1683} % *** Enter the CVPR Paper ID here
\def\confName{CVPR}
\def\confYear{2023}


\begin{document}

%%%%%%%%% TITLE - PLEASE UPDATE
\title{CF-Font: Content Fusion for Few-shot Font Generation}

\author {
    % Authors
    Chi Wang\textsuperscript{\rm 1,2}\footnotemark[1]\ ,
    Min Zhou\textsuperscript{\rm 2},
    Tiezheng Ge\textsuperscript{\rm 2},
    Yuning Jiang\textsuperscript{\rm 2},
    Hujun Bao\textsuperscript{\rm 1},
    Weiwei Xu\textsuperscript{\rm 1}\footnotemark[2] \\
    % Affiliations
    \textsuperscript{\rm 1} State~Key~Lab~of~CAD\&CG,~Zhejiang~University \quad     \textsuperscript{\rm 2} Alibaba~Group\\
    {\tt\small wangchi1995@zju.edu.cn, \{yunqi.zm, tiezheng.gtz, mengzhu.jyn\}@alibaba-inc.com} \\
    {\tt\small \{bao, xww\}@cad.zju.edu.cn}
}

\twocolumn[{%
\renewcommand\twocolumn[1][]{#1}%
\maketitle
\begin{center}
    \centering
    \captionsetup{type=figure}
    \includegraphics[width=1.\textwidth]{Figs/TeaserV2.0.pdf}
    % \vspace{-15pt}
    \captionof{figure}{\wc{Characters generated by our method.  (a) Source: source character images selected from ten basis fonts for content feature fusion. Weights: different colors and their covered areas on the doughnut chart represent the weights used to blend content features adaptively. Ten colors correspond to source images in colored boxes. 
    Target: few-shot target reference character images. One of those is performed as an example. 
    Ours: images generated by our method with fused content features and style features. (b) Generated character images of the first ten lines from a famous Chinese poem, each line with an extracted style, \eg thin, thick, swollen, cuneiform, inscription, or cursive style.}}
    \label{fig:teaser}
\end{center}%
}]

{
  \renewcommand{\thefootnote}{\fnsymbol{footnote}}
  \footnotetext[1]{This work was done during an internship at Alibaba Group.}
  \footnotetext[2]{Corresponding author.}
}

%%%%%%%%% ABSTRACT


\begin{abstract}
% \vspace{-1em}
The diffusion-based generative models have achieved remarkable success in text-based image generation. However, since it contains enormous randomness in generation progress, it is still challenging to apply such models for real-world visual content editing, especially in videos. 
In this paper, we propose \texttt{FateZero}, a zero-shot text-based editing method on real-world videos without per-prompt training or use-specific mask. 
\RM{Specifically, different from a pipeline of two independent inversion and then generation stages, we find the intermediate attention maps during inversions store better structure and motion information. We thus reform them to temporally casual attention and replace them in the generation progress. To further reduce the unnecessary semantic leakage of source video and enhance the editing quality, we then remix the temporally casual attentions via the cross-attention features of the source prompt as the mask.}
To edit videos consistently, we propose several techniques based on the pre-trained models. Firstly, in contrast to the straightforward DDIM inversion technique, our approach captures intermediate attention maps during inversion, which effectively retain both structural and motion information. These maps are directly fused in the editing process rather than generated during denoising. To further minimize semantic leakage of the source video, we then fuse self-attentions with a blending mask obtained by cross-attention features from the source prompt. Furthermore, we have implemented a reform of the self-attention mechanism in denoising UNet by introducing spatial-temporal attention to ensure frame consistency.
Yet succinct, our method is the first one to show the ability of zero-shot text-driven video style and local attribute editing from the trained text-to-image model. We also have a better zero-shot shape-aware editing ability based on the text-to-video model~\cite{tuneavideo}. \RM{Besides video, our unified method also achieves state-of-the-art performance in zero-shot image editing.\chenyang{Need exp or remove the zero-shot image}} Extensive experiments demonstrate our superior temporal consistency and editing capability than previous works.
% The code will be released.
% \chenyang{emphasize: our observation at inversion time} \xiaodong{replacing the bold part to the actual pipeline: \textbf{Specifically, we work on replacing and mixing the attention maps between the inversion and generation since the self-attention map keeps the structure of the original natural image and the cross-attention is semantic-related, after remixing, we replace them in the corresponding generation steps for denoising.}}
% \footnote{Since there is no general video diffusion model is publicly available, we use one-shot video generation method~(Tune-A-Video~\cite{tuneavideo}) as the pretrained video diffusion model for zero-shot video editing\xiaodong{can be removed if we actually zero-shot on video}.}.
\end{abstract}

%%%%%%%%% BODY TEXT
\section{Introduction}

The ability to reason about plans is critical for performing long-horizon tasks \citep{erol1996hierarchical, sohn2018hierarchical, sharma-etal-2022-skill}, compositional generalization \citep{corona-etal-2021-modular} and generalization to unseen tasks and environments \citep{shridhar2020alfred}.
Consider a simple long-horizon planning scenario where a robot is tasked with preparing a meal and serving it on the table. 
This presents a non-trivial planning problem since the agent needs to understand the sequence of operations required to perform the task and search for the relevant objects in the unfamiliar environment by interacting with various objects. %



Large language models have been recently shown to possess commonsense knowledge about the world such as object affordances and physical dynamics \citep{ouyang2022training,chowdhery2022palm}.
Early approaches considered text based environments and fine-tuned PLMs to predict actions given the history of past observations and actions \citep{jansen-2020-visually,micheli-fleuret-2021-language,yao-etal-2020-keep}.
Recent work has used this ability to reason about plans from text instructions in simulated household environments with simplifying assumptions such as text-only environment observations or feedback \citep{huang2022language,ahn2022can,li2022pre,logeswaran-etal-2022-shot}.


We focus on \emph{visually grounded planning} with PLMs --- the ability to adapt plans based on interaction and visual feedback from the environment.
While PLMs have strong planning commonsense priors, predictions from a PLM may not be directly realizable in the environment since the observation and action spaces are unknown.
This requires \emph{grounding} the PLM in the environment and adapting it to observe visual feedback, which is highly non-trivial.
Some prior works assume the availability of a pre-trained affordance function \citep{ahn2022can} or a success detector \citep{mirchandani2021ella}.
Notably, SayCan \citep{ahn2022can} completely decouples the PLM from observation information by selecting actions that have both high affordability (through a pre-trained affordance model) and high PLM likelihood.
Although this partially addresses the grounding problem, the use of visual feedback for action affordance alone is limited.
Often an agent must choose one of many affordable actions using information from observations.
For example, a driving agent should re-navigate and possibly turn around when encountering a ``road closed'' sign, but both turning around and driving forward are indistinguishable to SayCan because they are both affordable and the PLM is blind to observations.

Another workaround explored in prior work is translating the information in the visual observations to text using a pre-trained captioning system \citep{shridhar2021alfworld,huang2022language}.
However, it can be difficult to faithfully describe an image in words and information is lost in this inherently noisy process, which limits the information available to the planner.



Recent work shows that PLMs can be adapted for various natural language tasks by inserting tunable embeddings or soft prompts at the input of the PLM (also called prompt tuning or prefix tuning)~\citep{li-liang-2021-prefix,lester-etal-2021-power}.
This approach also extends to multi-modal understanding tasks such as image captioning \citep{mokady2021clipcap} and VQA \citep{tsimpoukelli2021multimodal} where images are encoded as soft prompts and finetuned for the target task.
Transformer based architectures have also been successfully applied to offline Reinforcement Learning in recent work \citep{chen2021decision,janner2021offline,li2022pre,reid2022can}.

Taking inspiration from these works, we propose the simple approach of embedding visual observations (`visual prompts') and \textit{directly inserting them as PLM input embeddings}.
The visual encoder and PLM are jointly trained for the target task, an approach we call \textbf{\oursfull}~(\ours).
By teaching the PLM to use observations for planning in an end to end manner, we remove the dependency on external data such as captions and affordability information that was used in prior work.
We show that this simple approach performs better than prior PLM-based planning approaches on two embodied planning benchmarks based on ALFWorld~\citep{shridhar2021alfworld} and Virtualhome~\cite{puig2018virtualhome}.



\section{Related work}
\noindent \textbf{Implict Neural Representation}.
Implicit neural representations (also known as coordinate-based representations) are a popular way to parameterize content of all kinds, such as audio, images, video, or 3D scenes~\cite{FFL, siren, srn, NeRF}.
Recent works \cite{NeRF, DeepSDF, occnet, srn} build neural implicit fields for geometric reconstruction and novel view synthesis achieving outstanding performance.
The implicit neural representation is continuous, resolution-independent, and expressive, and is capable of reconstructing geometric surface details and rendering photo-realistic images. 
%
While explicit representations like point clouds\cite{points1, NHR}, meshes\cite{NT}, and voxel grids\cite{deepvoxels, occnet, NeuralVolume, voxel1} are usually limited in resolution due to memory and topology restrictions.
%
One of the most popular implicit representations - Neural Radiance Field (NeRF) \cite{NeRF} -  proposes to combine the neural radiance field with differentiable volume for photo-realistic novel views rendering of static scenes. However, NeRF requires optimizing the 5D neural radiance field for each scene individually, which usually takes hours to converge. Recent works\cite{PixelNeRF, ibrnet, MVSNeRF} try to extend NeRF to generalization with sparse input views.
%
In this work, we extend the neural radiance field to a general human reconstruction scenario by introducing conditional geometric code and appearance code. 


\noindent \textbf{3D Model-based Human Reconstruction}
With the emergence of human parametric models like SMPL\cite{SMPL,SMPLX} and SCAPE\cite{SCAPE}, many model-based 3D human reconstruction works have attracted wide attention from academics. Benefiting from the statistical human prior, some works\cite{tex2shape, Multi-Garment, expose, VIBE} can reconstruct the rough geometry from a single image or video. 
However, limited by the low resolution and fixed topology of statistical models, these methods cannot represent arbitrary body geometry, such as clothing, hair, and other details well. 
To address this problem, some works\cite{PIFu, pifuhd} propose to use pixel-aligned features together with neural implicit fields to represent the 3D human body, but still have poor generalization for unseen poses. To alleviate such generalization issues, \cite{pamir, arch, doublefield} incorporate the human statistical model SMPL\cite{SMPL, SMPLX} into the implicit neural field as a geometric prior, which improves the performance on unseen poses. 
Although these methods have achieved stunning performance on human reconstruction, high-quality 3D scanned meshes are required as supervision, which is expensive to acquire in real scenarios. Therefore, prior works\cite{PIFu, pifuhd, pamir, arch} are usually trained on synthetic datasets and have poor generalizability to real scenarios due to domain gaps. To alleviate this limitation, 
some works\cite{neuralbody, Anim-NeRF, animnerf_zju, humannerf, arah, a-nerf}  combine neural radiance fields\cite{NeRF} with SMPL\cite{SMPL} to represent the human body, which can be rendered to 2D images by differentiable rendering. 
Currently, some works\cite{gpnerf, genebody, NHP, keypointNeRF, doublediffuse, doublefield} can quickly create neural human radiance fields from sparse multi-view images without optimization from scratch.
While these methods usually rely on accurate SMPL estimation which is not always applicable in practical applications. 
% We introduce a xxx

% identity-specific models, like NeuralBody\cite{neuralbody}
% generalizable models, 
% SMPL\cite{SMPL}, SMPLX\cite{SMPLX}, SCAPE\cite{SCAPE}, Tex2Shape\cite{tex2shape}, Multi-Garment Net\cite{Multi-Garment}, VIBE\cite{VIBE}, Expose\cite{expose}, NeuralBody\cite{neuralbody}, Anim-NeRF\cite{animnerf_zju, animnerf}, Neural Actor\cite{neuralactor}, SelfRecon\cite{selfrecon}, HumanNeRF\cite{humannerf}, PIFu\cite{PIFu}, PIFuHD\cite{pifuhd}, Pamir\cite{pamir}, Arch\cite{arch}, Double Field\cite{doublefield}, GNR\cite{genebody}, NHP\cite{NHP}, GPNeRF\cite{gpnerf}, KeypointNeRF\cite{keypointNeRF}, DoubleDiffuse\cite{doublediffuse}











\begin{figure*}[ht]
    \centering
    \vspace{-1em}
    \includegraphics[width=1.0\linewidth]{figures/method/pipeline.png}
    \vspace{-1.5em}
    \caption{\textbf{The architecture of our method}. Given $m$ calibrated multi-view images and registered SMPL, we build the generalizable model-based neural human radiance field. First, we utilize the image encoder to extract multi-view image features, which are used to provide geometric and appearance information, respectively. In order to adequately exploit the geometric prior, we propose the visibility-based attention mechanism to construct a structured geometric body embedding, which is further diffused to form a geometric feature volume. For any spatial point $\mathbf{x}$, we trilinearly interpolate the feature volume $\mathcal{G}$ to obtain the geometric code $\mathbf{g}(\mathbf{x})$. In addition, we also propose geometry-guided attention to obtain the appearance code $\mathbf{a}(\mathbf{x}, \mathbf{d})$ directly from the multi-view image features. We then feed the geometric code $\mathbf{g}(\mathbf{x})$ and appearance code $\mathbf{a}(\mathbf{x}, \mathbf{d})$ into the MLP network to build the neural feature field $(\mathbf{f}, \sigma) = F(\mathbf{g}(\mathbf{x}), \mathbf{a}(\mathbf{x}, \mathbf{d}))$. Finally, we employ volume rendering and neural rendering to generate the novel view image.
    % \Liqian{1) Add section ref. 2) add detailed caption. 3) Modulate the fig, each module corresponds to a sub-section. 4) keep fig text  consistent with method text}
    }
    \vspace{-1em}
    \label{fig:architecture}
\end{figure*}

\section{Approach}

% In this section, we first define our video music pair in Section~\ref{pro_def}. 
In this section, we introduce the problem formulation of the video-music matching task in Sec.~\ref{pro_def}. Then, we detail our loss functions in Sec.~\ref{objective} for lifting cross-modality features to a shared space.
Finally, we introduce our proposed \frameworkname~framework for video-music matching in Sec.~\ref{v_e}.

\subsection{Problem Definition}\label{pro_def}
Given a music set $\mathcal{M}=\{m\}_{i=1}^{N_m}$ with $N_m$ music and a video set $\mathcal{V}=\{v\}_{i=1}^{N_v}$ with $N_v$ videos from training dataset, where $(m,v)$ denote \music~and video samples. Our video-music matching task is formulated as two transformation functions ${\rm f}(m) \rightarrow y_m$ and ${\rm g}(v) \rightarrow y_v$, and $(y_m,y_v) \in [1,N_m]$ denote the predicted matching \music{} indexes in the training dataset. For matching video and \music{} as a metric learning problem, we further adopt the  shared weight $\textbf{W}$ to lift video and music features to a shared embedding space: 
\begin{align}
    \begin{split}
        % &{\rm h}({\rm f}(m)) \rightarrow y_m~,\\
        &\mathbf{W}\cdot{\rm f}(m) \rightarrow y_m~,\\
        &\mathbf{W} \cdot g(v) \rightarrow y_v~.
    \end{split}
    \label{eq:f-g}
\end{align}
During the testing stage, video and \music{} can be matched by estimating the cosine similarity between their features, i.e., ${\rm cos}({\rm f}(m),{\rm g}(v))$. For \music{} already included in the training dataset, we can directly use $y_v$ as the matched one. In this paper, the transformation functions $\rm f$ and $\rm g$ are respectively referred to as a video and a music encoder of which output embedding dimensions are both chosen as $l$.


\subsection{Cross-Modality Training Objectives}\label{objective}
Softmax loss is commonly used in the classification problem for minimizing intra-class and maximizing inter-class distances, which is formulated as:
\begin{equation}
    \begin{aligned}
        L_S(x_i, \mathbf{W}) = -\log \frac{e^{\mathbf{W}_{y_i} \cdot x_i}}{\sum_{j=1}^N e^{\mathbf{W}_j \cdot x_i}}~,
    \end{aligned}
    \label{eq:softmax-loss}
\end{equation}
where $\mathbf{W}$ is the prototype of each class, i.e., the weight of the last layer in a network, $N$ is the total number of classes, $x_i$ is the feature, and $y_i$ is the ground truth class index of $x_i$. For further improving the decision boundary between different classes, CosFace~\cite{cosface} proposed to lift the features and prototypes to a hyper-sphere by introducing a scaling term $s$ and a margin $\mu$:

\begin{equation}
\begin{aligned}
    &\cos(\theta_{k, i}) = \frac{\mathbf{W}_{k} \cdot x_i}{\norm{\mathbf{W}_{k}} \cdot \norm{x_i}}~, \\
    L_{C}(x_i, \mathbf{W})=&-\log \frac{e^{s \cdot [\cos(\theta_{y_i,i})-\mu]}}{e^{s \cdot [\cos(\theta_{y_i,i})-\mu]} + \sum_{j \neq i}^N e^{s \cdot \cos(\theta_{j,i})}}~.
    \label{eq:cosface}
\end{aligned}
\end{equation}
Since Equation~\eqref{eq:cosface} is based on the angles between intra and inter classes (i.e., $\theta_{y_i,i}$ and $\theta_{j,i}$) in a normalized feature space, the features $x_i$ are thus optimized in a hyper-sphere.

\paragraph{Cross-Modality Lifting Loss.}
\label{loss_class}
\fuen{
To solve the video-music matching task as a metric learning problem, we aim at lifting video and music features to the same hyper-sphere. In this way, we can match the videos to their most appropriate music by calculating the cosine similarity between them. Hence, we propose ``Cross-Modality Lifting Loss'' by adopting a shared prototype $\mathbf{W}$ for both video and music features and considering modality-to-prototype distances:
\begin{equation}
\begin{aligned}
    L_{LL} ({\rm g}(v), {\rm f}(m), \mathbf{W}) = L_C({\rm g}(v), \mathbf{W}) + \alpha L_C({\rm f}(m), \mathbf{W})~,
    \label{eq:LL}
\end{aligned}
\end{equation}
where $(v,m)$ are input music and video, $(\rm g,f)$ are the transformation functions as described in Equation~\eqref{eq:f-g}, and $\alpha$ is a hyper-parameter. In practice, the transformation functions are implemented as two independent feature encoders for video and music.}


\paragraph{Cross-Modality Similarity Loss.}
\label{loss_sim}
\begin{table*}[ht]  
\begin{center}   
\resizebox{\textwidth}{!}{\begin{tabular}{l||r||cc|cc|cc|
cc}  
  
\toprule % Toprule applied here  
  
\textbf{Dataset}&\multicolumn{1}{c||}{} & \multicolumn{2}{c|}{\textbf{Training}} & \multicolumn{2}{c|}{\textbf{Validation}} & \multicolumn{2}{c|}{\textbf{\closesettable}} & \multicolumn{2}{c}{\textbf{\opensettable}}\\
% & \multicolumn{2}{c|}{} & \multicolumn{2}{c|}{} & \multicolumn{2}{c|}{\textbf{(Weak generalization)}} & \multicolumn{2}{c|}{\textbf{(Strong generalization)}} &\\
% 
&\#Video &\#Music & VpM  & \#Music & VpM  & \#Music & VpM  & \#Music & VpM \\  
  % 
\midrule % Midrule applied here  
   % 
 \ourdataset{} & 150000 &265 & 400 & 265 & 20 & 265 & 80 & 125 & 140\\ 
 \midrule % Midrule applied here 
 Youtube-8M~\cite{abu2016youtube} & 5678 & 4654 & 1 & NA & NA & NA & NA & 1024 & 1024\\   
  % 
\bottomrule % Bottomrule applied here  
  % 
\end{tabular}}%
\caption{\textbf{Details of datasets.} The music for training, validation and \closesetname{} set is the same. Because the video-music pair is one-to-one mapping in Youtube-8M~\cite{abu2016youtube} dataset, validation and \closesetname{} are not included. \textbf{VpM} indicates the number of matching videos for each music.}  
\label{tab:table_msvd} 
\end{center}  
\end{table*}  
\fuen{
Although our proposed \faceloss~can effectively minimize the intra and maximize the inter class distances, we found that only considering such a modality-to-prototype distance still leads to a sub-optimal performance since videos and music are eventually matched based on their features instead of their prototype during the testing stage. To this end, we follow \cite{suris2018cross} to adopt ``Cross-Modality Similarity Loss'' aiming at addressing the video-to-music feature distances for improving our downstream video-music matching performance:
\begin{equation}
\begin{aligned}
    \cos(x_i, x_j) &= \frac{x_i \cdot x_j}{\norm{x_i} \cdot \norm{x_j}}~, \\
    L_{SL} ({\rm g}(v), {\rm f}(m), {\rm f}(m^\prime)) &= max[\tau, \cos({\rm g}(v), {\rm f}(m^\prime))] \\
    &- \cos({\rm g}(v), {\rm f}(m))~,
    \label{eq:SL}
\end{aligned}
\end{equation}
where $(v, m)$ indicates a positive video-music pair queried according to the ground truth music index of $v$, $(v, m^\prime)$ indicates a negative one randomly sampled from the dataset, and $\tau$ is a selected margin value. With \simloss, we can apply direct constraints in-between the predicted video and music features, and consider modality-to-modality distances, which provides consistent video-music matching schemes under both training and testing stages.
}


\subsection{\frameworkname~Framework}
\paragraph{Video Encoder.}
\label{v_e}
\fuen{To extract video features, we adopt a R(2+1)D ResNet-18~\cite{tran2018closer}, pretrained on Kinetics-400~\cite{kay2017kinetics} and followed by a fully-connected layer to infer the video features with embedding size $l$, as our video encoder ($\rm g$ in Equation~\eqref{eq:f-g}).}

\paragraph{Music Encoder.}
\label{m_e}
\fuen{we firstly calculate the Mel spectrograms of the input \music{} and adopt a ResNet-18~\cite{he2016deep} pretrained on ImageNet~\cite{deng2009imagenet} to extract the high-level music features. In addition, inspired by \cite{yi2021cross}, we use openSMILE~\cite{eyben2010opensmile} to extract the low-level music features, including MFCC, voice intensity, pitch, etc. The low-level and high-level features are then fused by concatenation and we infer the final music features by passing the fused features into an additional fully-connected layer with embedding size $l$. We refer this music encoder as $\rm f$ in Equation~\eqref{eq:f-g}.}

\paragraph{Training and Testing.}
\label{matching}
\fuen{During each training iteration, we use a pair of video and music for calculating $L_{LL}$ (Equation~\eqref{eq:LL}). For $L_{SL}$ (Equation~\eqref{eq:SL}), we randomly sample a negative music for calculation. The final training objective is established as:
\begin{equation}
\begin{aligned}
L(v, m, m^\prime) = L_{LL}({\rm g}(v), f(m), \mathbf{W})& + \beta L_{SL}({\rm g}(v), {\rm f}(m),\\
{\rm f}(m^\prime))
% &+ L_{SL}({\rm g}(v), {\rm f}(m^\prime))]~,
\label{eq:L-final}
\end{aligned}
\end{equation}
where $m^\prime$ indicates a randomly-sampled negative music, and $\beta$ is a hyper-parameter.}

\fuen{During the testing stage, the \closesetname~and \opensetname~sets are evaluated with different schemes, as illustrated in the right part of Fig.~\ref{fig:framework}.
For \closesetname~set, the video features are inferred from our video encoder $\rm g$, while the music features are directly pulled from the trained prototype since the trained prototype can represent the feature center of each training music, which can significantly reduce the inference time of \opensetname~features. 
For \opensetname~set, the music features are extracted from our music encoder $\rm f$.}

\fuen{Finally, we match the videos and music by calculating the cosine similarity between their features and select the top 20 music clips with the highest similarities as the matched music list.}
\section{Experiments}
\label{sec:experiments}

\subsection{Setup}
\textbf{Datasets.} We evaluate RFFR with four challenging datasets specifically designed for deepfake detection. We adopt the high quality (HQ) version of Faceforensics++ (FF)~\cite{ff} for training our deepfake detector. Faceforensics++ includes videos of real faces as well as four subsets of fake faces, each manipulated with a different algorithm, namely Deepfakes (DF), Face2Face (F2F), FaceSwap (FSW) and NeuralTextures (NT). We also utilize the test set of Celeb-DF~\cite{celeb-df} and DFDC~\cite{dfdc} for evaluating the cross-dataset performance of our model. Finally, in addition to real faces of Faceforensics++, we adopt the real face images from ForgeryNet (FN)~\cite{forgerynet} for learning RFFR, which helps improve representation learning with additional data.

\textbf{Implementation Details.} We extract the frames from all video datasets and use RetinaFace~\cite{retinaface} to detect and align the faces. All images are scaled to the size of $224 \times 224$. For our RFFR model, we adopt a base version of Masked Autoencoder (MAE)~\cite{mae} and train it on real faces with a batch size of $128$. Following MAE, we set the learning rate at $7.5 \times 10^{-5}$ and adjust it with a schedule with warmup and cosine decay. By default, we train this model with the real faces from both FF~\cite{ff} and FN~\cite{forgerynet}. 

For training the deepfake detector, we divide each image with $k = 4$ (Refer to Appendix for the motivation of choosing $k$). Each block enters the classifier with a probability of $p = 0.25$, and the residual images are amplified by $\alpha=4$. No data augmentation is applied to the images. We initialize both branches of Vision Transformer with ImageNet-pretrained weights and train them with a learning rate of $2 \times 10^{-5}$. During testing, we iteratively mask and restore all blocks to obtain a full residual image for the detector to process. We evaluate the testing results with AUC (Area Under Curve). 

\subsection{Cross-domain performance evaluation}
In this section, we test the performance of our RFFR-based deepfake detector with cross-manipulation and cross-dataset evaluations. 

\textbf{Cross-manipulation evaluations.} We train our deepfake detector on each subset of Faceforensics++ and test on all four subsets to demonstrate our model's ability to identify different manipulations, including those not seen during training. \emph{We adopt the HQ version of FF for both training and testing, and only use one frame every video for testing.} We compare our results with state-of-the-art image-based methods Multi-Attention~\cite{multiatt}, DCL~\cite{dcl}, RECCE~\cite{recce} and UIA-ViT~\cite{uia}. We ran the public code of RECCE and UIA-ViT to produce results under the same setting.

In~\cref{tab:cross-manipulation}, we show that our method outperforms the state-of-the-art methods under most settings, with a maximum improvement of $10.25\%$ (F2F $\rightarrow$FSW). Meanwhile, our model remains effective under the four intra-domain settings, which are shown in gray. The method tends to slightly underperform when trained on NeuralTextures, likely because its manipulation patterns only exist in certain small regions, and may be neglected during our block sampling. Nevertheless, compared to existing methods, our deepfake detector yields much better overall performances. 

\begin{table}[t]
\setlength\tabcolsep{4.5pt} 
\caption{Cross-manipulation performances in terms of AUC(\%) compared with previous methods. Classifiers are trained on one subset of FF and tested on all four subsets. Intra-domain results are marked in gray. We ran the public code of methods marked with "*" to produce results under identical settings \emph{(HQ for training and single frames for testing).}}
\vspace{-1.5em}
\label{tab:cross-manipulation}
\begin{center}  
\scalebox{0.80}{
\begin{tabular}{c|l|cccc|c}
\toprule
Training &\multirow{2}*{Method} & \multicolumn{4}{c|}{Test data} & \multirow{2}*{Avg} \\
\cmidrule(lr){3-6}
     data  &            ~                   & DF    & F2F   & FSW   & NT    & ~   \\
     
\midrule
\multirow{5}*{DF}
& MultiAtt~\cite{multiatt} & \cellcolor{Gray}99.92 & 75.23 & 40.61 & 71.08 & 71.71                \\ 
& DCL~\cite{dcl}       & \cellcolor{Gray}\textbf{99.98} & \textbf{77.13} & 61.01 & 75.01 & 78.28              \\
& RECCE*~\cite{recce}     & \cellcolor{Gray}99.19 & 74.39 & 57.42 & \textbf{85.04} & 79.01                \\ 
& UIA-ViT*~\cite{uia}  & \cellcolor{Gray}99.39      &   74.44    &   53.89    &   70.92    & 74.66 \\ 
& Ours  & \cellcolor{Gray}99.19 & 76.61 & \textbf{68.96} & 74.83 & \textbf{79.90}            \\ 
       
\midrule
\multirow{5}*{F2F}
        & MultiAtt~\cite{multiatt}       & 86.15 & \cellcolor{Gray}99.13 & 60.14 & 64.59 & 77.50 \\
        & DCL~\cite{dcl}       & 91.91 & \cellcolor{Gray}99.21 & 59.58 & 66.67 & 79.34 \\
       & RECCE*~\cite{recce}       & 88.04 & \cellcolor{Gray}98.93 & 67.35 & 74.16 & 82.12 \\
       & UIA-ViT*~\cite{uia}       & 83.39 & \cellcolor{Gray}98.32 & 68.37 & 67.17 & 79.31 \\
       & Ours                                  & \textbf{93.75} & \cellcolor{Gray}\textbf{99.61} & \textbf{78.62} & \textbf{79.56} & \textbf{87.81} \\

\midrule
\multirow{5}*{FSW}
& MultiAtt~\cite{multiatt} & 64.13 & 66.39 & \cellcolor{Gray}99.67 & 50.10 & 70.07              \\
& DCL~\cite{dcl}           & 74.80 & 69.75 & \cellcolor{Gray}99.90 & 52.60 & 74.26              \\
& RECCE*~\cite{recce}       & 66.66 & 73.66 & \cellcolor{Gray}\textbf{99.76} & \textbf{57.46} & 74.39               \\

& UIA-ViT*~\cite{uia}       &   81.02    &   66.30    & \cellcolor{Gray}99.04      &   49.26    & 73.91 \\ 
& Ours                                           & \textbf{87.46} & \textbf{75.96} & \cellcolor{Gray}99.42 & 55.87 & \textbf{79.68}            \\ 

\midrule
\multirow{5}*{NT}
& MultiAtt~\cite{multiatt} & 87.23 & 75.33 & 48.22 & \cellcolor{Gray}98.66 & 77.36                \\
& DCL~\cite{dcl}      & 91.23 & 79.31 & 52.13 & \cellcolor{Gray}\textbf{98.97} & 80.41                \\
& RECCE*~\cite{recce}    & \textbf{90.20}  & 76.65 & \textbf{58.06} & \cellcolor{Gray}97.17 & \textbf{80.52}                \\
 & UIA-ViT*~\cite{uia}  &    79.37   &   67.98    &   45.94    &\cellcolor{Gray}94.59       & 71.97 \\
 & Ours     & 84.31 & \textbf{81.04} & 54.67 & \cellcolor{Gray}96.19 & 79.05          \\
       
\bottomrule
\end{tabular}}
\vspace{-2em}
\end{center}
\end{table}

\textbf{Cross-dataset evaluations.} We train our model on the Faceforensics++ dataset and evaluate its performance on the test sets of Celeb-DF\cite{celeb-df} and DFDC~\cite{dfdc}. Specifically, following the previous practice in~\cite{lip}, we validate the model on Celeb-DF and use the selected model to test on DFDC.  \emph{We adopt the HQ version of FF for training, and only use one frame every video for testing.} Under the same setting, we ran the public code of RECCE~\cite{recce}, UIA-ViT~\cite{uia} and SBI~\cite{sbi} to produce corresponding results. In Table~\ref{tab:cross-dataset}, we show a competitive performance with existing image-based methods, signaling satisfying adaptability of RFFR to different datasets, especially high quality datasets like Celeb-DF. 
  
SBI~\cite{sbi} is a recent powerful deepfake detection method. By utilizing a hand-crafted blending algorithm to produce diverse fake samples, it achieves highly competitive performances on datasets including Celeb-DF. We show that by training on fake samples generated by SBI, our approach can further improve upon their state-of-the-art result. 

\begin{table}[]
\setlength\tabcolsep{4.5pt} 
\caption{Cross-dataset performances in terms of AUC(\%) compared with previous methods. Classifiers are trained on FF and tested on Celeb-DF and DFDC. We ran the public code of methods marked with "*" to produce results under identical settings \emph{(HQ for training and single frames for testing).}}
\vspace{-1em}
\label{tab:cross-dataset}
\begin{center}  
\scalebox{0.90}{
\begin{tabular}{l|cc}
\toprule
\multirow{2}*{Method} & \multicolumn{2}{c}{Test data}\\
\cmidrule{2-3}
        ~                           &     Celeb-DF         &  DFDC \\
\midrule
      Xception~\cite{xception}  &     65.30       &    -  \\
      Face X-ray~\cite{xray}          &     74.20       &     70.00 \\
      MultiAtt~\cite{multiatt}        &     67.44       &     67.34 \\
      SPSL~\cite{SPSL}                &     76.88        &   -  \\
      SOLA~\cite{sola}                &       76.02         &  -    \\
      SLADD~\cite{sladd}              &    79.70       &  -  \\
      RECCE*~\cite{recce}             &     68.94       &   68.34   \\
      UIA-ViT*~\cite{uia}             &     80.31      &   67.93   \\
      SBI*~\cite{sbi}                       &       86.46     &   66.60     \\
\midrule
 	Ours                                      &   81.97  & \textbf{72.08}  \\
    Ours + SBI~\cite{sbi}                  &  \textbf{88.98}           &    67.84   \\
\bottomrule
\end{tabular}}
\vspace{-2.5em}
\end{center}
\end{table}

\subsection{Ablation Study}
\label{ablation}

In this section, we analyze the effect of our implementations for RFFR learning and deepfake detection. 

\textbf{Effect of the training data for RFFR.} The effectiveness of deepfake detection with RFFR depends on the quality of representation learning, where the real faces plays an important role. In this experiment, we examine the effect of scaling the real face dataset for representation learning. As a baseline, we learn RFFR with only real faces from Faceforensics++ (FF), the same data we use for the downstream classification tasks. Meanwhile, another model is supplemented with real faces from both FF and ForgeryNet (FN), a significantly larger and more diverse dataset. We train deepfake detectors on the F2F subset of FF with residual images produced by these two models. In Table~\ref{tab:data}, we demonstrate that including the extra dataset of ForgeryNet for learning RFFR consistently improves the performances of the deepfake detector in all tests, creating a maximum performance gain of $9.57\%$  in terms of AUC (F2F $\rightarrow$ NT).

We note that learning RFFR with FF already allows our deepfake detector to outperform the state-of-the-arts. Nevertheless, learning with extra data enhances the efficacy of our real face foundation representations, and further improves the downstream task of deepfake detection. Therefore, refining the representation learning of real faces, especially with large-scale datasets, could be a viable path for further improving generalized deepfake detection. 

In addition, we examine the scalability of RECCE under the same setting, considering that RECCE~\cite{recce} also involves learning to reconstruct real samples for deepfake detection. However, their performance gain is less significant than ours. Although the reconstruction branch of RECCE~\cite{recce} is able to highlight forgery cues with residual images, they tend to involve more background noise caused by imperfect reconstructions, as depicted in~\cref{fig:unet_comparison},. This undermines the ability of residual images to expose artifacts for deepfake detection. 

\begin{table}[t]
\setlength\tabcolsep{4.5pt} 
\caption{Deepfake detection performances of RECCE~\cite{recce} and our method with different real face dataset, namely the real faces from Faceforensics++ (FF) alone, and FF combined with ForgeryNet (FF + FN). Classifiers are trained on F2F and tested on four subsets of FF. We present the results in AUC (\%).  }
\vspace{-1.5em}
\label{tab:data}
\begin{center}  
\scalebox{0.90}{
\begin{tabular}{c|c|cccc|c}
\toprule
\multirow{2}*{Method} & Real face  & \multicolumn{4}{c|}{Test data} & \multirow{2}*{Avg} \\
\cmidrule(lr){3-6}
&dataset  &      DF    & F2F   & FSW   & NT    & ~   \\
    \midrule
\multirow{2}*{RECCE~\cite{recce}}&FF           & 88.04          & 98.93          & 67.35          & 74.16          & 82.12          \\
&FN + FF &  90.12       & 99.24       & 69.89    & 79.59     & 84.71		\\
    \midrule
\multirow{2}*{Ours}&FF           & 90.16          & 98.56          & 74.10          & 69.99          & 83.20          \\
&FN + FF & \textbf{93.44}       & \textbf{99.61}        & \textbf{78.62}       & \textbf{79.56}        & \textbf{87.81}		\\
\bottomrule
\end{tabular}}
\vspace{-1em}
\end{center}
\end{table}

\textbf{Effect of masked image modeling for RFFR.} We analyze the effect of using MIM-based residual images for deepfake detection. We train a UNet-based autoencoder (AE) to learn the reconstruction of real faces and obtain residual images. Our MIM-trained inpainting model and the AE are compared on the quality of reconstruction in~\cref{fig:unet_comparison}. Note that despite being trained with real faces, the AE "generalizes" well to fake images, preserving delicate details, including the artifacts caused by manipulations. Such generalization leaves the residual images empty with little information. 

\begin{figure}
\centering
  \includegraphics[width=0.9\columnwidth]{figs/compare_ICCV_Final.pdf}
  \vspace{-1em}
   \caption{Reconstruction results and residual images of the autoencoder (AE), RECCE~\cite{recce} and our inpainting model. AE reconstructs both images perfectly, leaving no information in residual images. RECCE~\cite{recce} suffers from insufficient training. Our model successfully highlights potential artifacts in the residual image of only the fake face, and therefore can best facilitate deepfake detection. }
\vspace{-1em}
\label{fig:unet_comparison}
\end{figure}

Masked image modeling enables our model to learn better real face representations and inpaint fake faces with real textures instead of artifacts. In the downstream task of deepfake detection,  our classifier generalizes significantly better than the AE-based classifier, which performs only marginally better than learning with no residuals (detailed in Appendix). Both the reconstruction results and the downstream performance confirm the validity of our choice to learn RFFR with MIM instead of direct reconstruction. 


\textbf{Effect of classifier backbone.} In Table~\ref{tab:backbone}, we present the deepfake detection results of vanilla Xception~\cite{xception} and Vision Transformer (ViT)~\cite{vit}, both trained with full original images. The models are trained with the F2F subset of FF and tested on all four subsets. While a larger backbone increases a deepfake detector's generalization performance in some cases, it is not the primary factor of our performance improvement. Instead, it is the residual input aided by RFFR that leads the performance gain.

\begin{table}[t]
\setlength\tabcolsep{4.5pt} 
\caption{Comparing ours results with vanilla backbones. We present the results in AUC (\%).  }
\label{tab:backbone}
\vspace{-1.5em}
\begin{center}  
\scalebox{0.90}{
\begin{tabular}{c|c|cccc|c}
\toprule
Training  &  \multirow{2}*{Method}    &   \multicolumn{4}{c|}{Test Data} & \multirow{2}*{Avg} \\
\cmidrule(lr){3-6}
 data  &   ~  &   DF    & F2F   & FSW   & NT    & ~   \\
    \midrule
\multirow{3}*{F2F} & Xception~\cite{xception} & 84.94          & 99.26          & 58.82          & 71.19          & 78.55          \\
                                   & ViT~\cite{vit}      & 84.25          & 97.89          & 65.53          & 65.18          & 78.21          \\
                                   & Ours     & \textbf{93.44} & \textbf{99.61} & \textbf{78.62} & \textbf{79.56} & \textbf{87.81} \\
\bottomrule
\end{tabular}}
\vspace{-1.5em}
\end{center}
\end{table}

\textbf{Effect of classifier design.} We compare different variants of our classifier design. Specifically, we analyze the performance gains brought by the introduction of two branches and the random input mechanism. We test six variants of our classifier by training them with the F2F subset of FF and testing with the FSW subset. The settings of these variants are specified by the input data they accept, as shown in~\cref{tab:classifier}. 

\begin{table}[t]
\caption{Deepfake detection performances with classifiers of different inputs in terms of AUC (\%). We train the classifiers on F2F and test on FSW.}
\label{tab:classifier}
\vspace{-1.5em}
\begin{center}
\begin{tabular}{c|c|c|c|c}
\toprule
\multicolumn{2}{c|}{Original Image} & \multicolumn{2}{c|}{Residual Image} & \multirow{2}*{AUC (\%)} \\
\cline{1-4}
               Full        &             Random           &          Full          &          Random          &   ~\\
 \hline
\checkmark        &                                       &                            &                                   &  65.53\\
% \hline
                              &                                      &   \checkmark    &                                   &  66.30  \\
 %\hline
\checkmark        &                                      &   \checkmark    &                                   &  71.48  \\
 %\hline
                             &       \checkmark          &                             &                                   &  70.76  \\
%\hline
                             &                                       &                             &      \checkmark       &  68.10  \\
 %\hline
                             &        \checkmark         &                             &      \checkmark       &  \textbf{78.62}  \\
\bottomrule
\end{tabular}
\vspace{-2em}
\end{center}
\end{table}

\begin{table*}[t]
\setlength\tabcolsep{4.5pt} 
\caption{Deepfake detection performances of validated and non-validated models. Classifiers are trained on F2F and tested on four subsets of FF. We present the results and the performance gaps in AUC (\%). Second best results are underlined. }
\label{tab:validation}
\vspace{-1em}
\begin{center}  
\scalebox{0.90}{
\begin{tabular}{c|c|llll|l}
\toprule
\multirow{2}*{Method}  & \multirow{2}*{Validated} & \multicolumn{4}{c|}{Test Data} & \multirow{2}*{Avg} \\
\cmidrule(lr){3-6}
~                   &                      ~                   &      DF               & F2F                    & FSW                 & NT                    & ~   \\
    \midrule
\multirow{2}*{Xception\cite{xception}} &   \checkmark    & 84.94                 & 99.26                & 58.82                 & 71.19                & 78.55            \\
~ &                                             -                              & 83.08   (- 1.86) & 99.12   (- 0.14) & 46.63   (- 12.19) & 64.93   (- 6.26)  & 73.44   (- 5.11)  \\
 \hline
 \multirow{2}*{RECCE\cite{recce}} &\checkmark               & 88.04                & 98.93                 & 67.35                & 74.16                & 82.12            \\
 ~&                                                -                  & 74.51   (- 8.57) & 99.22   (+ 0.29)  & 50.17   (- 17.18) & 59.46   (- 14.70)  & 70.84   (- 11.28) \\
 \hline
\multirow{2}*{Ours} &    \checkmark  & \textbf{93.44}            & \textbf{99.61}            & \textbf{78.62}            & \textbf{79.56}            & \textbf{87.81}            \\
 ~&  - & \underline{91.56} (- 1.88) & \underline{99.39}   (- 0.22) & \underline{76.00}   (- 2.62)  & \underline{76.41} (- 3.15) & \underline{85.84}   ( - 1.97)    \\
\hline
\end{tabular}}
\vspace{-2em}
\end{center}
\end{table*}

We treat the vanilla ViT with full original image input as a baseline, which achieves an AUC of $65.53\%$. By switching to accept the full residual images, we obtain a $0.77\%$ performance gain. Combining the two modalities to form a dual-branch classifier further increases our result to $71.48\%$. This demonstrates that the artifacts are better exploited when both the original and the residual images enter the classifier, and are used as references to each other. Therefore, both modalities should be considered for classification. 

In addition, we improve on the test by merely modifying the baseline ViT to accept randomly selected original image blocks. This results in a $5.23\%$ increase in performance. Similarly, changing full residual input to random residual blocks also results in a $1.8\%$ improvement. These observations confirm our hypothesis in \cref{sec:method_deepfake_detection} that models benefit from learning with random inputs, which prevents the model from only focusing on the most prominent features in an image, and forces it to learn from subtle artifacts. 

Finally, bringing in the random input mechanism for the dual-branch classifier completes our full implementation, which maximally exploits the artifacts exposed by RFFR and achieves the best performance of $78.62\%$. 



\subsection{Validation-free Model Selection}
\label{sec:validation-free}

\begin{figure}
\centering
  \includegraphics[width=0.5\textwidth]{figs/validation-free_ICCV_Final.png}
  \vspace{-1.5em}
   \caption{Comparing the validation curves of RFFR-based deepfake detector and previous methods. Detectors are trained on the F2F subset of FF for $15k$ iterations and validated on four different subsets. (a) to (d) correspond to experiments on DF, F2F, FSW and NT.  Results are reported in AUC (\%). All three methods perform well when validated on F2F. However, under cross-manipulation settings, only our method avoids overfitting during training. The curves are smoothed for better visibility.}
\label{fig:validation-free}
\vspace{-1em}
\end{figure}

Models expected to generalize to other domains benefit from target domain validations~\cite{domainbed}. By frequently performing model validation, we can select the model  that best suits the detection of target manipulation, resulting in high performance on the test set. While using such an \textit{oracle} could be acceptable for the early development of cross-domain algorithms~\cite{domainbed}, it is not ideal for applications, as labeled data of unseen manipulation is usually not available. 

In this section, we demonstrate the potential of our deepfake detector to circumvent this practice and therefore avoid the need for extra validation data. As shown in \cref{tab:validation}, we train our classifier on F2F for 15k iterations and directly use the final model for testing. Simultaneously, we employ four validation sets to select the models with the best validation performances on target data. All validated and non-validated models are tested under the same conditions. We report all results on the target test sets in Table~\ref{tab:validation}. The performance gaps between validated and non-validated models are reported along with the test results. Although our non-validated models are not performing as well as those selected with a validation set, we show that our model remains effective on target data, with a maximum performance drop of $3.15\%$ and an average drop of $1.97\%$. However, previous methods~\cite{xception, recce} suffer from significantly larger performance drops when evaluated under the same procedure. 

To take a closer look at how the cross-manipulation performances vary during training, we train the deepfake detectors again with F2F. We test the AUC performances on all target subsets every 50 iterations to produce validation curves in \cref{fig:validation-free}. Our RFFR-based deepfake detector consistently maintains a high performance long after its peaks without serious overfitting. On the contrary, both previous methods compared here overfit quickly after reaching their highest target domain performances. In addition, compared methods exhibit large fluctuations across different evaluations, while our model remains stable. This suggests that with RFFR, our model focuses exclusively on generalizable features which fall outside the distribution of RFFR. Such resistance to overfitting guarantees our model a satisfying performance even when labeled validation sets are not available, which is generally expected in practice. We present more results on validation-free evaluations in Appendix.
% \section{Discussion and Conclusion}
To conclude, we propose \textbf{\nickname{}}, the first generalizable human NeRF model that recovers animatable 3D humans from single human image inputs.
To render high-fidelity 3D humans, \nickname{} proposes to learn both global and local details from the bank of 3D-aware hierarchical features comprising global features, point-level features, and pixel-aligned features. 
By using a feature fusion transformer, \nickname{} successfully enhances the information from the 2D observation and complements the information missing from the input image. 
% In addition, by modelling the neural radiance field in the canonical space, \nickname{} can animate 3D humans with free poses.
On four large-scale human datasets, \nickname{} achieves state-of-the-art performance and renders high-fidelity images in both novel views and poses.

\vspace{1.75mm}
\noindent \textbf{Limitations:}
1) There still exists visible artifacts in target renderings when some body parts are occluded in the observation space. 
A better feature presentation like occlusion-aware features may be explored to solve this issue. 
2) How to complement the information missing from single image input remains a challenging problem.
\nickname{} starts from the reconstruction view and can only render deterministic results when predicting novel views.
One potential direction is to investigate the use of conditional generative models to diversely generate higher quality novel views.

\vspace{1.75mm}
\noindent \textbf{Potential Negative Societal Impacts:} 
\nickname{} can be misused to create fake images or videos of real humans and cause negative social impacts.


\section{Conclusion, Limitations, and Future Work}
\label{sec:future}
We presented \ours, a NeRF editing method conditioned on text and sketch. Using novel loss functions, our framework allows for local editing of neural fields.
\begin{wrapfigure}{r}{0.2\textwidth} 
\vspace{-10pt}
  \begin{center}
    \includegraphics[width=0.2\textwidth]{figs/failures_Ali.jpg}
  \end{center}
    \vspace{-15pt}
 \vspace{1pt}
\end{wrapfigure} 
Similar to previous works \cite{poole2022dreamfusion, lin2022magic3d, metzer2022latent}, our approach utilizes the SDS Loss and may be vulnerable to the well-known "multiface issue" (inset figure) depending on the choice of diffusion model and prompt. Our method supports a single set of prompt and sketch views at a time. A simple workaround is to apply our method multiple times progressively (Fig.~\ref{fig:progressive}). 
Our results rely on the publicly available Stable-Diffusion model \cite{rombach2021highresolution}, which is less amenable to directional text prompts and produces lower quality 3D generated outputs compared to commercial diffusion models used by previous works~\cite{poole2022dreamfusion, lin2022magic3d}. In Fig~\ref{fig:diff_diff} we show that it is possible to get better results by using the Deepfloyd-IF model \cite{deepfloyd}.


Future directions may expand our method to better support for non-opaque materials, or condition on other modalities, possibly through the diffusion model. More research may further extend the usage of sketch scribbles for animation, similar to \cite{dvoro2020monstermash}. 



% \orrc{In addition, the interface of our method may further close the gap with non data-driven methods, through allowing inflated single sketch views or other primitive based sketch interfaces. Mention also we didn't explore half-transparent objects enough

% \orr{
% Limitations: 1. Janus effect / multiface problem (cat with santa hat), 2. sketching multiple disjoint regions at once. 3. mention that quality presented in this work depend on the diffusion model used? (we can't compete with the larger IMAGINE / e-diffi).

% Notes: remember thanking people: Andrey for SGMT code and mention mesh sources. (the cat, the plate, the horse)
% }

\section*{Acknowledgments}
\crc{We thank the anonymous reviewers for their constructive comments. This paper is supported by Information Technology Center and State Key Lab of CAD\&CG, Zhejiang University.}

% \section{Appendix}

\subsection{Psycholinguistic Attributes}\label{sec:attr-def}

Drawing upon literature~\cite{graham2009liberals, shepherd2018guns,mendez2017neurology}, we identify seven most relevant sociolinguistic attributes that could potentially predict how two ideological groups talk differently on the gun issues.

Two {\it affect} dimensions:
\begin{itemize}
\item \vale: emotions can range from positive (e.g., pleasant, happy, hopeful) to negative (e.g., unhappy, annoyed, despairing)
\item \domi: emotions can range from the most dominant (e.g., feeling-in-control, influential, autonomous) to the least dominant (e.g, weak, submissive, and guided)
\end{itemize}

Five {\it moral} foundations:
\begin{itemize}
\item \care: the virtue of caring, nurturing, and protecting the vulnerable
\item \fair: the virtue of reciprocal altruism, including justice, rights, and welfare
\item \auth: the virtue of respect for authority
\item \loya: the virtue of being loyal to your identified groups
\item \puri: the virtue of seeing the human bodies as holly temples that should not be contaminated
\end{itemize}

\subsection{Human Annotated Attribute Values}\label{sec:annotation}

The human annotation included two phases: (1) creating reliable coding rules, and (2) coding. In the first phase, a major task is to the create the inclusion criteria for human annotators to identify the language signals that correspond to the theorized attributes in tweets. To do so, we sampled a subset of tweets (10-40\%) for each attribute from the total of 3100 relevant tweets. Through an iterative process, one of our authors who is in the field of social psychology began with open-coding to evaluate how the theoretical constructs and categories can be manifested in tweets' language use. She identified the discourse features and themes, and then built, tested, and refining the rules with a graduate research assistant. The inclusion criteria were created with 100\% agreement between the two criteria developers. Once the criteria were set up, each of these tweets was then coded by two independent annotators by a group of four research assistants who did not participate in the the criteria development stage but were trained to follow the coding schemes. These research assistants were chosen because they had been trained prior to this project and developed skills to analyze social media discussions that involve complex politics and social contexts. For each tweet, the annotators determined whether the tweet texts involved each of the seven attributes as a set of binary outcomes. The coding in this phase resulted in fair to substantial agreements between the annotators, with inter-rater reliability in terms of the Cohen's kappa ranging from 0.32 to 0.88 across all attributes. Any disagreement was reconciled after discussion and the coding criteria and procedure were formulated through the process. In the second phase, every tweet (from the 3100 relevant set) was annotated. For {\it moral} attribute values (e.g., \fair, \auth), we followed the coding schemes developed from the first phase. The annotation generated categorical values for each of the moral attribute. For {\it affect} (e.g., \vale), we determined to adopt the Best-Worse Scaling used by Mohammad et al.~\cite{mohammad2018obtaining} after testing it in the first phase. This annotation scheme employed comparative annotation method, which can be used to generate continuous rating for an attribute. We adopted this method and implemented the coding through crowdsourcing on Amazon Mturk. In the crowdsourcing annotation, three annotations are required for each of the $2N$ tweet-tuples (where each tuple contains 4 randomly-grouped tweets, and $N=3100$ in our case) in order generate reliable annotation results. Finally, the annotated scores for affect attributes are normalized to range from -1 to 1.

%The human annotation of the present sociolinguistic attributes involved: (1) an open coding phase, and (2) a categorical coding phase. First, the open coding phase is to generate the coding criteria and procedure for coding a tweet on each attribute based on the theoretical definitions in literature \cite{XX}. To do so, we sampled a subset of tweets (10-40\%) from the total of 3100 relevant tweets. Each of these tweets was coded by two independent annotators who were familiar with the theoretical definitions. For each tweet, the annotators determined whether the tweet texts involved each of the seven attributes as a set of binary outcomes. The coding in this phase resulted in fair to substantial agreements between the coders, with inter-rater reliability in terms of the Cohen's kappa ranging from 0.32 to 0.88 across all attributes. Any disagreement was reconciled after discussion and the coding criteria and procedure were formulated through the process. In the second phase, every tweet (from the 3100 relevant set) was annotated. For {\it moral} attribute values (e.g., \fair, \auth), we followed the coding schemes developed from the first phase. The annotation generated categorical values for each of the moral attribute. For {\it affect} (e.g., \vale), we determined to adopt the Best-Worse Scaling used by Mohammad et al.~\cite{mohammad2018obtaining} after testing it in the first phase. This annotation scheme employed comparative annotation method, which can be used to generate continuous rating for an attribute. We adopted this method and implemented the coding through crowdsourcing on Amazon Mturk. In the crowdsourcing annotation, three annotations are required for each of the $2N$ tweet-tuples (where each tuple contains 4 randomly-grouped tweets, and $N=3100$ in our case) in order generate reliable annotation results. Finally, the annotated scores for affect attributes are normalized to range from -1 to 1.
%\yrl{WT: please check and fix issues; also could you provide a short/brief layman description about each attribute in the bullet points below?}

%\begin{itemize}
%\item \vale:
%\item \domi:
%\item \care:
%\item \fair:
%\item \puri:
%\item \auth:
%\item \loya:
%\end{itemize}

% \wtc{The annotation involved two phases. The first was to develop the operational definitions and coding schemes of all the attributes. A subset of tweets (10-40\% of the total of 3100 tweets) were randomly sampled, used to generate operational definitions, and each tweet was coded by two independent annotators for obtaining reliable results. For each tweet, the annotators determined whether the tweet texts involved the seven attributes, yes or no, as a binary code.  Reliability testing indicated fair to substantial agreements, with the values of Cohen's kappa ranging from .32 to .88. In the second phase, every tweet was annotated. For moral values, we followed the coding schemes developed in phase 1. For affect, we further adapt the Best-Worse Scaling ~\cite{mohammad2018obtaining}, an annotation scheme that employs comparative annotation, which can generate continuous values instead of binary codes, for each dimension, that allows the direct comparisons among tweets. Prior empirical studies  ~\cite{mohammad2018obtaining} have developed a reliable procedure to annotate emotions from words, which suggested to use three annotations to annotate 2N 4-randomly-grouped-tweets (N is the numbers of data item; in our study, the total number of tweets, 3100) is sufficient to have reliable scores. We followed the recommendations and implemented through crowd-sourcing on Mturk. The scores are normalized so range from -1 to 1.}. 


\subsection{Evaluation for the multi-task prediction}\label{sec:holdout}

\begin{table*}[ht]
\footnotesize
\begin{center}
\caption{\textbf{Results of Multi-task Prediction.} We report results of the Attribute Prediction tasks and Group Label prediction tasks. Performance changes of the Multi-task Predictions compared to baselines are reported in parentheses}\label{tb:pred_performance}
\begin{tabular}{c|c|cc}

\toprule
& \bf Attribute Prediction &\multicolumn{2}{c}{\bf Group Label Prediction} \\

& Acc. (CLF) or Pearson \textit{r}(REG)& Accuracy & F1 \\
\midrule

Dominance (REG) & 0.804 (-0.035) & \bf 0.814 (+ 0.009) & \bf 0.809 (+ 0.014)\\
Valence(REG) & 0.768 (- 0.015) & 0.809  & 0.805\\
Harm(CLF) & 0.623 (+ 0.012) & 0.812 & \bf 0.809 (+ 0.014)\\
Fairness (CLF) & 0.68 (+ 0.001) & 0.803 & 0.802\\
Authority (CLF) & 0.742 (+ 0.007) & 0.799 & 0.794\\
Purity (CLF) & 0.975 (+ 0.003) & 0.793 & 0.791\\
Loyalty (CLF) & 0.825 (+ 0.009) & 0.802 & 0.798\\ \midrule
Dominance Baseline & 0.839 & N/A  & N/A \\
Valence Baseline & 0.783 & N/A & N/A \\
Harm Baseline & 0.611 & N/A & N/A \\
Fairness Baseline & 0.679 & N/A & N/A \\
Authority Baseline & 0.735 & N/A & N/A \\
Purity Baseline & 0.972 & N/A  & N/A \\
Loyalty Baseline & 0.816 & N/A & N/A \\ 
\midrule

Group Baseline & N/A & 0.805 & 0.795\\

\bottomrule
\end{tabular}
\end{center}



\end{table*}


We evaluate the multi-task prediction models using a hold-out experiment on the 3100 relevant tweets, where 50\% samples are used for training, 15\% samples for validating, and the remaining samples for testing. 
We consider two types of baseline models: (a) group prediction baseline 1: to predict group labels with all the annotated attributes on sample tweets using standard machine learning method; (b) group prediction baseline 2: to predict group labels with tweet texts on sample tweets using single-task neural network architecture; and (c) attribute prediction baseline: to predict a single attribute value with tweet texts on sample tweets using single-task neural network architecture. 
The group prediction task is evaluated using {\it accuracy} as our dataset is balanced. 
The attribute prediction tasks are evaluated in terms of the {\it Pearson correlation coefficient} for continuous attributes, and by {\it accuracy} for categorical attributes. 
Table~\ref{tb:pred_performance} report performances of all models and baselines.
The performance gain (or loss) of multi-task models compared to the baselines are reported in the parentheses. 
We highlight key observations from the results: (1) For group prediction, our best model achieves accuracy 0.814, outperforming the two group prediction baselines by up to 30\%. (2) For attribute prediction, the performance measures of our models range 0.768--0.804 in terms of \textit{Pearson} correlation (for the two continuous attributes) and 0.623--0.975 in terms of accuracy (for the five categorical attributes), which is very close to the attribute baseline (with only 0.3\% differences on average). 
Such results suggest that our multi-task models can significantly improve group prediction without sacrificing the performance for attribute prediction.


%%%%%%%%% REFERENCES
{\small
\bibliographystyle{ieee_fullname}
\bibliography{egbib}
}

\end{document}
