In this paper, we implement and benchmark several state-of-the-art convolution primitives for ARM Cortex-M microcontrollers. Our benchmark shows that for microcontrollers which cannot use SIMD instructions, theoretical MACs is a relevant indicator to estimate the layer energy consumption. For microcontrollers which use SIMD instructions, latency is preferred over theoretical MACS to estimate the layer energy consumption while using SIMD instructions. We explain this by the varying efficiency of the im2col algorithm, from CMSIS-NN, depending on the layers and highlight the role of data reuse in this performance gap. Furthermore, we study the influence of external parameters to the convolution algorithms such as the compiler optimization and the MCU frequency. Our experiments highlight the major impact of the compiler optimization on the layers performance while using SIMD instructions, and show that running the inference at maximum frequency decreases the layer's energy consumption. Our work opens up new possibilities for neural architecture search algorithms.

\subsubsection{Author Contribution} Nguyen, Moëllic and Blayac conceived and planned the study. Nguyen carried out the experiments and performed the analysis. Nguyen and Moëllic wrote the manuscript with inputs from all authors.

\section*{Acknowledgments}
Part of this work was done with the support of ID-Fab (Prototyping platform: project funded by the European Regional Development Fund, the French state and local authorities).
This work benefited from the French Jean Zay supercomputer thanks to the \textit{AI dynamic access} program. This collaborative work is partially supported by the IPCEI on Microelectronics and Nano2022 actions and by the European project InSecTT\footnote{\url{www.insectt.eu}: ECSEL Joint Undertaking (876038). The JU receives support from the European Union’s H2020 program and Au, Sw, Sp, It, Fr, Po, Ir, Fi, Sl, Po, Nl, Tu. The document reflects only the author’s view and the Commission is not responsible for any use that may be made of the information it contains.} and by the French National Research Agency (ANR) in the framework of the \textit{Investissements d’Avenir} program (ANR-10-AIRT-05, irtnanoelec).
