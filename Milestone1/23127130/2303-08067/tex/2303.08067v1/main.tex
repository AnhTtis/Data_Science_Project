%\documentclass[conference, a4, 10pts]{IEEEtran}
\documentclass[conference]{IEEEtran}
\IEEEoverridecommandlockouts
\usepackage{hyperref}
\usepackage{booktabs}

\usepackage{cite}
\usepackage[pdftex]{graphicx}
% \graphicspath{{../pdf/}{../jpeg/}}
\DeclareGraphicsExtensions{.pdf,.jpeg,.png}
\usepackage{amsmath}
\usepackage{comment}
\usepackage{algorithmic}
\usepackage{array}
\usepackage{fixltx2e}
\usepackage{url}
\usepackage{color,soul}
\hyphenation{op-tical net-works semi-conduc-tor}
\usepackage{caption}
\usepackage{subcaption}
\usepackage{multirow}
\usepackage{authblk}
\usepackage{xcolor}
\usepackage{hyperref}
\newcommand{\cc}[1]{\normalsize{\color{blue}#1}}






\begin{document}
\bstctlcite{IEEEexample:BSTcontrol}
%\bstctlcite{IEEEexample:BSTcontrol}
%\title{Wireless Communications for an In-memory Hyperdimensional Computing Architecture}
%\title{Over-The-Air Wireless On-Chip Communications for Scalable In-memory Hyperdimensional Computing}
\title{WHYPE: A Scale-Out Architecture with\\ Wireless Over-the-Air Majority for\\ Scalable In-memory Hyperdimensional Computing}


%\author{Robert Guirado, Abbas Rahimi, Geethan Karunaratne, Eduard Alarc\'{o}n, Abu Sebastian, Sergi Abadal%
%\thanks{R. Guirado is with Universitat Polit\'{e}cnica de Madrid (UPM), Madrid, Spain. E. Alarc\'{o}n, and S. Abadal are with Universitat Polit\`{e}cnica de Catalunya (UPC), Barcelona, Spain. A. Rahimi, G. Karunaratne, and A. Sebastian are with IBM Research Zurich GmbH, R\"{u}schlikon, Switzerland. R. Guirado was a UPC student and IBM intern during the preparation of this article.}
%}

\author[$\dagger$*]{Robert Guirado}
\author[$\dagger$]{Abbas Rahimi}
\author[$\dagger$]{Geethan Karunaratne}
%\author[*]{Nitish Arya}
\author[*]{Eduard Alarc\'on}
\author[$\dagger$]{Abu Sebastian}
\author[*]{Sergi Abadal}

\affil[$\dagger$]{IBM Research -- Zurich, R\"{u}schlikon, Switzerland}
\affil[*]{Universitat Polit\`ecnica de Catalunya, Barcelona, Spain}


\maketitle

\begin{abstract}
Hyperdimensional computing (HDC) is an emerging computing paradigm that represents, manipulates, and communicates data using long random vectors known as hypervectors. Among different hardware platforms capable of executing HDC algorithms, in-memory computing (IMC) has shown promise as it is very efficient in performing matrix-vector multiplications, which are common in the HDC algebra. Although HDC architectures based on IMC already exist, how to scale them remains a key challenge due to collective communication patterns that these architectures required and that traditional chip-scale networks were not designed for. To cope with this difficulty, we propose a scale-out HDC architecture called WHYPE, which uses wireless in-package communication technology to interconnect a large number of physically distributed IMC cores that either encode hypervectors or perform multiple similarity searches in parallel. In this context, the key enabler of WHYPE is the opportunistic use of the wireless network as a medium for over-the-air computation. WHYPE implements an optimized source coding that allows receivers to calculate the bit-wise majority of multiple hypervectors (a useful operation in HDC) being transmitted concurrently over the wireless channel. By doing so, we achieve a joint broadcast distribution and computation with a performance and efficiency unattainable with wired interconnects, which in turn enables massive parallelization of the architecture.  
Through evaluations at the on-chip network and complete architecture levels, we demonstrate that WHYPE can bundle and distribute hypervectors faster and more efficiently than a hypothetical wired implementation, and that it scales well to tens of receivers. We show that the average error rate of the majority computation is low, such that it has negligible impact on the accuracy of HDC classification tasks.
\end{abstract}


%%%%%\input{01-Intro}

\section{Introduction} \label{introd}
\vspace{-0.1cm}
Hyperdimensional computing (HDC) is an emerging computational framework and is based on the observation that key aspects of human memory, perception and cognition can be explained by the mathematical properties of hyperdimensional spaces comprising high-dimensional vectors known as hypervectors~\cite{hdcintro}. The $d$-dimensional hypervectors are generated using (pseudo)random process such that their components are independent and identically distributed. When the dimensionality ($d$) is in the thousands, a large number of quasi-orthogonal hypervectors exist. This allows HDC to combine existing hypervectors into new hypervectors using well-defined vector operations, such that the resulting hypervector is unique and preserves the dimensionality. To review HDC and related computational models in detail refer to~\cite{HDC_Rev_PI}.  

HDC has been employed in a range of applications including cognitive computing~\cite{PlateAnalogy2000,KanervaDollar2010}, robotics~\cite{NeubertRobotics2019}, distributed computing~\cite{VSA_Workflow,SimpkinScalable2018,TomsettDemonstrationOrch2019}, communications~\cite{CollectiveComm,Dependable_MAC_HD,Kim2018HDM,Hsu2019Collision,Hsu_HDM2,Hersche2021}, and in various aspects of machine learning. See~\cite{HDC_Rev_PII} for a comprehensive review. HDC has achieved particularly notable accuracy in machine learning applications that demand few-shot learning, where other alternative approaches have generally struggled~\cite{InMemFSCIL2022,FSCIL2022,Karunaratne2021RobustHM,MoinWearable2021,RahimiBiosignal2019,oneshot}. Among other advantages, HDC is extremely robust in the presence of failures, defects, variations, and noise, all of which are synonymous to ultra-low energy computation. For instance, it has been shown that HDC degrades gracefully in the presence of various faults in comparison to other alternative classifiers: HDC tolerates intermittent errors~\cite{HDC2016}, permanent hard errors in memory~\cite{Li3DVRRAM2016} or in logic~\cite{WuNanotube2018}, and spatio-temporal variations~\cite{HDC_NatElec20} in emerging memory technologies. In a similar vein, it also tolerates noise and interference in the communication channels~\cite{Kim2018HDM,Hersche2021}. These demonstrate robust operations of HDC under low signal-to-noise ratio and high variability conditions thanks to the special brain-inspired properties of HDC: (pseudo)randomness with i.i.d. components, high-dimensionality, and holographic representations (see~\cite{HDC2016} for more details). 

What different HDC algorithms have in common is that they operate on wide vectors. Therefore, HDC calls for architectures that handle operations on a large number of wide vectors efficiently. One of the key operation of HDC is similarity search. It compares an input hypervector with a typically large number of hypervectors that are stored in an associative memory. As a similarity metric, dot-product is often used. This provides a natural fit to exploit in-memory computing (IMC) for HDC~\cite{HDC_NatElec20,InMemFSCIL2022}. An IMC core departs from the von Neumann architectures which move data from a processing unit to a memory unit and vice versa by exploiting the possibility of performing operations (dot products, in our case) within the memory device itself~\cite{memorydevices}. This improves both time complexity and energy consumption of the architecture. 


IMC systems have been proposed recently to execute HDC tasks using hypervectors as wide as 10,000-bit~\cite{HDC_NatElec20}. As further elaborated in Section \ref{sec:bcg}, IMC cores are capable of performing similarity searches through dot-products with unprecedented energy efficiency, i.e. $\sim$100$\times$ more efficiently than a digital accelerator~\cite{HDC_NatElec20}. This has sparked interest in HDC systems that can handle a large search space. For example, certain applications require to continually add new hypervectors for representing novel classes in the incremental learning regime~\cite{FSCIL2022,InMemFSCIL2022}, and performing similarity search on them, that can grow over thousands hypervectors. Yet still, how to scale HDC architectures to perform searches across a large number of classes remains unclear due to the associated challenges.

HDC architectures can be scaled by either increasing the size of the IMC cores to accommodate many hypervectors (scale-up) or by deploying multiple moderately-sized IMC cores to execute the similarity searches in parallel (scale-out). On the one hand, scaling up requires large IMC cores for the architecture to be usable in incremental learning applications. This poses a fundamental problem in terms of array impedances and programming complexity for the IMC core~\cite{ScaleUpXbar}. On the other hand, scaling out implies distributing wide hypervectors across a potentially large number of modules, which puts a large pressure on the system interconnect. More specifically, such an architecture generates reduction and broadcast communication patterns for which conventional Networks-on-Chip (NoC) and Networks-in-Package (NiP) suffer to deliver competitive performance. 

To address the scalability problem of IMC-based HDC architectures, we propose to use wireless communications within the computing package. Wireless Network-on-Chip (WNoC) have shown promise in alleviating the bottlenecks that traditional NoC and NiP face, especially for the collective traffic patterns that appear when scaling out HDC architectures \cite{Laha2015, ahmed2020asymmetric, jog2021one, micro2022, wiplash}. 
To that end, WNoCs provide native broadcast capabilities, 
which are put to use to implement a chip-scale network that, at the same time, is able to bundle multiple hypervectors and distribute the resulting bundled hypervector to a number of physically distributed similarity search engines.

In this paper, we present WHYPE, a scale-out architecture that employs wireless over-the-air (OTA) computing to enable scalable in-memory HDC. The architecture, summarized in Fig.~\ref{fig:large_arch}, consists of a set of $M$ encoders that generate query hypervectors. These hypervectors are transmitted simultaneously, bundled over the air, and received by a set of $N$ similarity search engines. Bundling is possible thanks to the OTA computing of the bit-wise majority of the hypervectors, which eliminates the need for a central point of reduction in the architecture. In turn, the concept of OTA is feasible because we have full electromagnetic knowledge of the chip package and we can engineer the constellations to calculate the majority over the air with low error. 

The WNoC proposed for WHYPE is uniquely suited to HDC for two main reasons. On the one hand, it delivers seamless support for the reduction and broadcast patterns required by scale-out HDC architectures. On the other hand, it does so while bypassing the main limitations of wireless interconnects. The generally low aggregate bandwidth is multiplied by $M$ in WHYPE as we allow the concurrent transmission of $M$ hypervectors. Also, the impact of the typically high error rates is minimized thanks to the inherent resilience of HDC algorithms to noise.

\begin{figure}[!t]
    \centering
    \includegraphics[width=1\columnwidth]{images/large_arch_figv6.pdf}
    \vspace{-0.3cm}
    \caption{Overview of WHYPE, a many-core wireless-enabled IMC platform. Orange encoders map to our wireless transmitters, while green IMCs map to our wireless-augmented IMCs. Bit-wise majority operation required for hypervector bundling is performed via wireless over-the-air computation.}
    \label{fig:large_arch}
    \vspace{-0.5cm}
\end{figure}

In summary, this paper makes the following three novel contributions. First, we present WHYPE, a wireless-enabled architecture for scale-out hyperdimensional computing. Second, we assess the capacity of WHYPE's wireless interconnect to deliver lightweight all-to-all concurrent communications at the chip scale. Third, we evaluate the impact of imperfect wireless communications on the accuracy of the similarity search. It is worth noting that this paper significantly expands on the work presented in \cite{guirado2022wireless} by:
\begin{itemize}
    \item Presenting a complete design for both a wired baseline and the WHYPE architecture, including details on the wireless interfaces and their connection to the encoders and similarity search engines.
    \item Making a comparison of the throughput, area, and power of both the baseline and the WHYPE architecture, which both motivates the need for the proposed architecture.
    \item Extending the performance analysis of WHYPE to the time domain to understand the achievable bandwidth for the OTA computation. 
    \item Evaluating the accuracy of the WHYPE's similarity search considering a distributed dataset.
\end{itemize}
 



%%%%%%%%\section{Background and Related Work}
\vspace{-1mm}
\label{sec:background}
\subsection{Einsums, Dataflows and Mappings}
\vspace{-1mm}
Tensor operations like convolutions and matrix multiplications can be concisely and precisely expressed using Einsum (Einstein summation) notation. Einsums are well-supported state-of-the-art tools like Python's numpy and the tensor algebra compiler TACO \cite{XXXTaco}. Compared to traditional mathematical matrix contraction notation, they have the advantage of explicitly describing the volume of data being operated on. For example, the equations below describe GEMM and CONV:

\vspace{-3.5mm}

\begin{equation}
    Z_{m,n}=\sum_{k}A_{m,k}*B_{k,n}
\end{equation}

\vspace{-3.5mm}

\begin{equation}
    O_{n,h,w,k}=\sum_{c,r,s}I_{n,h+r,w+s,c}*W_{r,s,c,k}
\end{equation}

In equation (1), $A$, $B$ and $Z$ are the tensors and $m$,$n$ and $k$ are dimensions/ranks. $k$ is a \textit{contracted rank} and $m$ while $n$ are \textit{uncontracted ranks}. For brevity the summation can be omitted as the contracted ranks do not appear in the output tensor.

Einsums can be straightforwardly implemented using loop nests, for example:
%\begin{center}
\vspace{-1mm}
\begin{small}
  \begin{verbatim}
1 for m in range(M):       for m in range(M):
2  for n in range(N):       for k in range(K):
3   pfor k in range(K):      pfor n in range(N):   
4    Z(m,n)+=A(m,k)*B(k,n)    Z(m,n)+=A(m,k)*B(k,n)
\end{verbatim}  
\end{small}
\vspace{-1mm}
%\end{center}

\textit{Dataflow} refers to the loop transformations for staging the operations in compute and memory. A dataflow can affect compute utilization and the locality of tensors inside the memory hierarchy. Within the Einsum, an \textit{intra-operation} dataflow is determined by the loop order and the parallelism strategy. The code sequences above represent two different loop orders: the left is $MNK$, whereas the right is $MKN$. The \texttt{pfor} indicates that the rank is parallelized. Thus, the left sequence is \emph{K-parallel} and the right one is \emph{N-parallel}. The \textit{inter-operation} dataflow for multiple chained Einsums is one of the main contributions of this work, as discussed in detail in~\autoref{sec:dataflows} and~\autoref{sec:gogeta}.

Another result of the loop order is the concept of stationarity~\cite{eyeriss2016isca}. An \emph{A-stationary}  dataflow signifies that $A$ is the tensor whose operands change slowest--therefore the $N$ rank is the fastest to change (as it does not index $A$). In GEMMs, there are two possible loop orders for \emph{A-stationary} dataflows, $MKN$ and $KMN$. Similarly, for \emph{Output-stationary} dataflows, the $K$ rank is the fastest to change, hence $MNK$ and $NMK$ are the two possible loop orders.

Tiling refers to slicing the tensors, in order for sub-tensors to fit in local memory buffers to extract reuse. Note that typically partitioning any given rank can affect multiple tensors. An example code sequence for tiling is as follows:
\vspace{-1mm}
\begin{small}
  \begin{verbatim}
1 M1=M/M0; K1=K/K0; N1=N/N0
2 for m1 in range(M1):    
3  for k1 in range(K1):  
4   for n1 in range(N1):
5    for m0 in range(M0):
      m = m1 * M0 + m0
6     for k0 in range(K0):
       k = k1*K0+k0
7      pfor n0 in range(N0):
        n = n1*N0+n0
8       Z(m,n) += A(m,k) * B(k,n)  
\end{verbatim}  
\end{small}

Note the interaction of parallelism and tiling: $N0$ is \texttt{pfor} in line 7. Thus in a tile, the $n0$ indices of $N$ are spatially mapped, resulting in $N1$ temporal tiles of size $N0$. 

The combination of dataflow and tiling is called a \emph{mapping}: a schedule of the exact execution of the workload on a specific hardware accelerator. Mapping directly affects data movement, buffer utilization, memory bandwidth utilization, and compute utilization.

\subsection{HPC Applications: Chains of Einsums}
\label{sec:apps}

Single Einsums are kernels, whereas the main loops of scientific applications consist of a chain of Einsums where tensors produced by earlier equations are consumed by later ones. This results in a \emph{tensor dependency graph} dictating the high-level production/consumption of data throughout the HPC region of code. Throughout this section we use Conjugate Gradient as a running example because its tensor dependency graph exhibits multiple kinds of reuse opportunities and challenges. We briefly discuss other scientific applications with similar patterns where our work is applicable in this section.

\subsubsection{Block Conjugate Gradient}


Iterative linear solvers solve the system of linear equations-

\vspace{-2.5mm}
\begin{equation}
    A_{m, k} * X_{k} = B_{m}
\end{equation}

While traditional conjugate gradient considers b and x as vectors, block conjugate gradient works on multiple initial guesses simultaneously for faster convergence, thus making it a matrix multiplication problem:

\vspace{-1mm}
\begin{equation}
    A_{m, k} * X_{k, n} = B_{m, n}
\end{equation}
\vspace{-1mm}
%%\vspace{-2mm}
\LinesNotNumbered

\begin{algorithm}
\begin{small}

%\label{algo:inter-op}
\caption{Block Conjugate Gradient. 'prev' and 'cur' stand for previous and current. Only lines with tensor operations inside the loop are numbered.}\label{alg:cg}
$R_0=B-AX_0$\\
$P_0=R_0$\\
$prev=0$\\
\For{iteration=1,2,3...}
{
\nl $S_{prev}=A.P_{prev}$\\
\nl $\Delta=P^{T}_{prev}.S_{prev}$ and $\Lambda=\Delta^{-1}.\Gamma_{prev}$\\
\nl $X_{cur}=X_{prev}+P_{prev}.\Lambda$\\
\nl $R_{cur}=R_{prev}-S_{prev}.\Lambda$\\
\nl $\Gamma_{cur}=R^{T}_{cur}.R_{cur}$\\
\If{$all(diag(\Gamma_{cur})\leq \in)$}
{
    break
}
\nl $\Phi=\Gamma^{-1}_{prev}.\Gamma_{cur}$\\
\nl $P_{cur}=R_{cur}+P_{prev}.\Phi$ \\

$prev=cur$\\
$cur++$\\

}
\textbf{Return} $X_{cur}$
\end{small}

\end{algorithm}

\begin{comment}

  \nl   $S'(m,n)=A(m,k)*P'(k,n)$     \\
 \nl   $\Delta (n,n)=P'^{T}(n,k)*S'(k,n)$ and $\Lambda (n,n)=\Delta ^{-1}(n,k)*\Gamma (k,n)$ \\
 \nl   $X(m,n)=X'(m,n)+P'(m,k)*\Lambda (n,n)$ \\
\nl    $R(m,n)=R'(m,n)-S'(m,k)*\Lambda (n,n)$ \\
 \nl   $\Gamma (n,n)=R^{T}(n,k)*R(k,n)    $  \\
\If{$all(diag(\Gamma)\leq \in)$}
{
    break
}
\nl    $\Phi(n,n)=\Gamma ^{-1}(n,k)*\Gamma (k,n)$ \\
\nl $P(m,n)=R(m,n)+P'(m,k)*\Phi (n,n)$\\

\end{comment}


\vspace{-5mm}
\LinesNotNumbered

\begin{algorithm}
\begin{small}

%\label{algo:inter-op}
\caption{Chain of Einsums for Conjugate Gradient. $k$ represents contractions of size $M$ and $j$ represents contractions of size $N$ and $N'$. Line numbers represent significant computation steps referred to throughout the text.}
\label{alg:cg_einsum}
\KwIn{$A_{m,m}$, $B_{m,n}$, $X_{m,n}$}
$R_{m,n} = B_{m,n} - A_{m,k} * X_{k,n}$ \\
$\Gamma_{n,n}=R_{k,n}*R{k,n}$\\
$P_{m, n} = R_{m, n}$ \\
\While{not converged}
{
  \nl  $S_{m,n} = A_{m,k} * P_{k,n}$     \\
 \nl   $\Delta_{n',n} = P_{k,n'} * S_{k,n}$ and $\Lambda_{n',n} = \Delta^{-1}_{n',j} * \Gamma_{j,n}$ \textcolor{gray}{~~~// $\Delta=P^TS$} \\
 \nl   $X_{m,n} = X_{m,n} + P_{m,j} * \Lambda_{j,n}$ \\
\nl    $R_{m,n} = R_{m,n} - S_{m,j} * \Lambda_{j,n}$ \\
$\Gamma\_prev_{n,n}=\Gamma_{n,n}$\\
 \nl   $\Gamma_{n',n} = R_{k,n'} * R_{k,n}$ \textcolor{gray}{~~~~~~~~// $\Gamma=R^TR$} \\
\If{$all(diag(\Gamma)\leq \epsilon)$} 
{
    \textbf{break}
}
\nl    $\Phi_{n',n} = \Gamma\_{prev}^{-1}_{n',j} * \Gamma_{j,n}$ \\
\nl    $P_{m,n} = R_{m,n} + P_{m,j} * \Phi_{j,n}$
}
\textbf{return} $X$
\end{small}
\end{algorithm}


\begin{comment}

  \nl   $S_{prev}(m,n)=A(m,k)*P_{prev}(k,n)$     \\
 \nl   $\Delta (n,n)=P^{T}_{prev}(n,k)*S_{prev}(k,n)$ and $\Lambda (n,n)=\Delta ^{-1}(n,k)*\Gamma (k,n)$ \\
 \nl   $X_{cur}(m,n)=X_{prev}(m,n)+P_{prev}(m,k)*\Lambda (n,n)$ \\
\nl    $R_{cur}(m,n)=R_{prev}(m,n)-S_{prev}(m,k)*\Lambda (n,n)$ \\
 \nl   $\Gamma _{cur}(n,n)=R^{T}_{cur}(n,k)*R_{cur}(k,n)    $  \\
\If{$all(diag(\Gamma_{cur})\leq \in)$}
{
    break
}
\nl    $\Phi(n,n)=\Gamma ^{-1}(n,k)*\Gamma _{cur}(k,n)$ \\
\nl $P_{cur}(m,n)=R_{cur}(m,n)+P_{prev}(m,k)*\Phi (n,n)$\\
\end{comment}




%\insertFigurePart{cg}{Block Conjugate Gradient Algorithm. 'prev' and 'cur' stand for previous and current variables.}


%~\autoref{alg:cg} shows Conjugate Gradient and
~\autoref{alg:cg_einsum} shows the Einsums in the Conjugate Gradient Algorithm. Intuitively, we start with an initial guess $X$ and we update $R$ which at any iteration is equal to $B-AX$. If $R$ is sufficiently small, then we have reached the solution. $P$ represents the search direction for the next iteration of the loop. We have validated the functional correctness of our Einsum representation against Python's \texttt{scipy.sparse.linalg.cg}.

%\vspace{-3.5mm}


From the perspective of tensors, $P$, $R$, $S$ and $X$ (named using English-letter variables except A) are highly skewed, for example $1000000\times 8$. In contrast, tensors like $\Delta$, $\Lambda$, $\Phi$ and $\Gamma$ (named using Greek letters) are small tensors, for example, of size $8\times8$ Also, $A$ is the only sparse tensor in CG with a maximum shape of, for example, $1000000\times1000000$ but with occupancy of 1-100 non-zeros per row. In~\autoref{alg:cg_einsum}, $M$ represents the large rank while $N$ represents the small rank. $N'$  is equivalent to $N$ but is used to differentiate between the dimensions of square matrices in an Einsum without accidentally contracting them. As a result, line 1 is a sparse SpMM operation while all the other matrix multiplication operations are dense. The inverse operations (lines 2 and 6) are insignificant in terms of the magnitude of computation but they do affect the dependency graph since they require the complete tensor to be produced for execution. As we discuss in ~\autoref{sec:ai}, these matrix operations have low arithmetic intensity. The only inputs from the application side are $A$, $B$ and the initial guess, which is the initial $X$, the final output is $X$ at convergence: all other tensors are intermediates not observable by the invoking context.

Another peculiar feature about the dense matrix multiplications in Conjugate Gradient is that one matrix dimension is large.
%In this work, we represent each matrix multiplication as $(M\times K)$.$(K\times N) = M\times N$, thus $K$ is the contracted rank while $M$ and $N$ are uncontracted.
In operations denoted by lines 1, 3, 4 and 7, an uncontracted rank is the dominating rank (assuming $A$ has been compressed using a standard format like CSR). In matrix multiplication operations denoted by line 2 ($\Delta=P^TS$) and 5, a contracted rank is the dominating rank resulting in small outputs of the order 1$\times$1 to 16$\times$16. Contracted rank being dominating significantly diminishes the benefits of pipelining the entire CG application efficiently since most of the compute is spent in just large amounts of reduction to generate an output a usable output for the operation after it thus not exploiting the staging opportunity and affecting the overall utilization. 

~\autoref{fig:dfg} shows the dependency graph between Einsum operations in CG. We observe that most tensor operands are not only reused in the immediate operation but also in some later operation (i.e., transitive edge) therefore having multiple reuse distances. This is unlike dependency graphs of applications like DNNs and GNNs where the output is mostly used immediately in the next operation, making fusion straightforward. One exception in Deep Learning is the skip connections in applications like ResNet.


\insertFigure{dfg}{Tensor dependency graph of intermediates in Conjugate Gradient across first two iterations of the CG loop where a node's number corresponds to the line in~\autoref{alg:cg_einsum}.}

\subsubsection{Other Applications with similar patterns}

The pattern of variable reuse distance is commonly observed in Machine Learning models like Resnet~\cite{resnet} with skip connections, although ResNet has a high arithmetic intensity per Einsum. Its also observed in other solver methods like GMRES~\cite{gmres} and BiCGStab~\cite{van1992bi}. The problem of low arithmetic intensity individual \GEMM is common in workloads like Graph Neural Networks~\cite{kipf2017semisupervised}.

For example, a layer of GCN (Graph Convolution Network) has the following Einsums (Only A is sparse).

Variables: $A_{m,m}, ~~X0_{m,n}, ~~Z_{m,n}, ~~W_{n,o}, ~~X1_{m,o}$

$Z_{m,n}=\sum_{k}A_{m,k}*X0_{k,n}$  and  $X1_{m,o}=\sum_{j}Z_{m,j}*W_{j,o}$

\subsection{Related Work}

\textbf{Conjugate Gradient Acceleration:} Cerebras~\cite{cerebras} proposes mapping BiCGStab on the wafer-scale engine specifically for stencil application where the matrix $A$ is structured. Plasticine~\cite{plasticine} has inherent support for Vector Parallelism and Pipelined Parallelism. ALRESCHA~\cite{asgari2020alrescha} proposes an accelerator for Preconditioned Conjugate Gradient (PCG) and optimizes the locality of the SpMM and the SymGS kernels, however, even at maximum reuse, single kernels have low arithmetic intensity. None of these works have identified and exploited \textit{inter-operation} reuse.

\textbf{Dataflows and Mappers:} MAESTRO~\cite{kwon2019understanding}, Timeloop~\cite{timeloop}, Interstellar~\cite{interstellar}, GAMMA (Genetic Algorithm Mapper)~\cite{kao2020gamma}, CoSA~\cite{cosa} propose a mapping optimization search space, cost model or a mapping search algorithm for a single tensor operation at a time. Prior works like FLAT~\cite{flat} and TANGRAM~\cite{tangram} have proposed a new dataflow for pipelining between exactly two adjacent Einsums. Garg et al.~\cite{garg2021understanding} formulate the design-space for pipelined mappings for exactly two Einsums and propose a cost model OMEGA to evaluate those mappings. We identify reuse opportunities beyond pipelining between tensors and propose a systematic methodology to determine the dataflow of the whole dependency graph of tensor operations (including Einsums and tensor additions).

%\subsubsection{Specialized Data Orchestration} Various spatial accelerators for Deep Learning~\cite{tpu-isca,nvdla,eie,kwon2018maeri,sigma,extensor,eie} use custom buffers tied to their compute. Buffets~\cite{buffets} is a storage idiom designed for Deep Learning Algorithms. It uses Explicit Decoupled Data Orchestration. Prior works have also proposed domain-specific cache replacement policies, for example P-OPT~\cite{popt-hpca21} proposes a replacement policy for Graph algorithms.




\section{Background}\label{sec:bcg}
To facilitate the understanding of the principles behind WHYPE, here we provide background on the topics of HDC, IMC, and wireless communications at the chip scale.

\subsection{Hyperdimensional Computing}
In this work we focus on a variant of HDC models using 512-dimensional binary hypervectors. Under this setting, by employing a random process, it is easy to find a huge number of non-coincident quasi-orthogonal vectors that exhibit normalized Hamming distance close to 0.5. We call these random hypervectors \textit{atomic} hypervectors. One can further create an \textit{encoder} to operate on these atomic hypervectors by using operations such as binding, bundling (i.e. superposition), and permutation to obtain a composite hypervector describing an object or event of interest. In a classification task, the composite hypervectors, generated from various examples of the same class, can be further bundled together to create a single prototype hypervector representing a class. In this work, the bundling operation is implemented as a logical element-wise majority operation. During training, the prototype hypervectors are stored in the associative memory.

In the inference stage, the query hypervectors of unknown objects/events are generated by the same procedure as in the training stage. The query hypervector is compared to the prototype hypervectors in the associative memory. Then, the chosen label is the one corresponding to the prototype hypervector that has the highest similarity to the query vector. In HDC, the robustness to failure is given by the spreading of information across thousands of dimensions. See~\cite{RahimiBiosignal2019} for more details. 


\subsection{In-memory Computing}
With each new technology node, the gap between the speed and efficiency of computation and memory continues to grow. The effects of such a disparity, commonly known as the \textit{memory wall}, have been addressed with novel concepts such as high-bandwidth memory \cite{hbm} or 2.5D and 3D monolithic integration \cite{3d}, among others. However, from an architectural point of view, these solutions are not solving the fundamental bottleneck arising from the need to move large quantities of data from memory and back. Instead, IMC appears as a promising candidate to overcome these challenges \cite{memorydevices}.

IMC is a form of non von-Neumann computing paradigm that leverages the memory unit to perform in-place computational tasks, reducing the amount of data movement and therefore cutting down the latency and energy consumption associated with in-package communication \cite{memorydevices}.  
At the core of IMC is a crossbar array with a memory device lying at each cross point of the array. IMC cores in which these memory devices are resistance-based, and more specifically those based on phase-change memory (PCM) devices, have recently shown promising results~\cite{Y2022khaddamJSSC}. In a resistance-based IMC core, we can program certain values as conductances of cross point memory devices. Executing a matrix-vector multiplication (MVM), essential to any machine learning algorithm, is as simple as, first, tuning conductances to match the matrix values. Second, by exploiting Ohm's law and Kirchhoff's law, inputting the vector as voltages from one side and finally reading the output currents from a perpendicular side.

IMC architectures are capable of executing various HDC operations~\cite{HDC_NatElec20}. In this work, we are particularly interested in the similarity search in the associative memory. As shown in Fig.~\ref{fig:dotp}, since the prototype hypervectors $P_i$ will be programmed in an IMC core, the similarity search through the dot product can be implemented as a MVM with the query hypervector $Q$ as input vector. This allows performing a dot-product in $O(1)$ time complexity.

\begin{figure}[!t]
    \centering
    \includegraphics[width=0.85\columnwidth]{images/Xbar.png}
    \vspace{-0.1cm}
    \caption{Similarity search example in an IMC core. Since the prototype hypervector of the third column is the most similar one to the query vector $Q$, it will output more current than the others and its associated label will be chosen.} %\RG{Zoom in a resistor showing the device response}}
    \label{fig:dotp}
    \vspace{-0.3cm}
\end{figure}

\subsection{Wireless Network-on-Chip}
Manycore architectures currently rely on NoCs as their interconnect backbone. However, the performance of NoCs can quickly degrade when serving collective communication patterns such as reductions or broadcasts, especially when scaled. In light of this, WNoCs have been recently proposed to complement wired interconnects due to their natural broadcast support, low system-wide latency, and adaptive network topology \cite{ahmed2020asymmetric,adaptive, wiplash, imani2022smart}. Even though WNoC technology is not mature, proof-of-concept designs have been implemented and tested \cite{multichannel}. 

In WNoCs, a core or a cluster of cores are equipped with RF transceivers and antennas \cite{barrier, timoneda,wienna}. This allows them to modulate and transmit data, which propagates through the chip package until being picked up and demodulated by the receivers in the transmitter's range. 
By tuning all antennas to the same channel, WNoC allows to perform low-latency broadcast. However, this comes at the cost of a low aggregate bandwidth as it two simultaneous broadcast transmissions will interfere with each other. Moreover, for the same energy, wireless interconnects are generally much less reliable than wired interconnects due to their \textit{radiative} nature. Fortunately, as described in next sections, the forgiving nature of HDC and the collective nature of its operations minimize the impact of the WNoC disadvantages.



%%%%%%%%%%%\input{03-Motivation}


\begin{figure*}[!t]
    \centering
    %\includegraphics[width=0.7\columnwidth]{images/wired_alternative.png}
    \begin{subfigure}{0.35\textwidth}
    \includegraphics[width=\textwidth]{images/fig_general1.pdf}
    \caption{Logical (top) and physical view (bottom).}
    \label{fig:diagram}
    \end{subfigure} \hfill
    \begin{subfigure}{0.63\textwidth} %\vspace{0.6cm}
    \includegraphics[width=\textwidth]{images/fig_general2.pdf}
    \caption{Circuital view (shaded blocks are our contribution).}
     \label{fig:wireless_arch_diagram}
    \end{subfigure}
    \vspace{-0.1cm}
    \caption{Towards a wireless-enabled scale-out HDC architecture with over-the-air computing. (a, top) The architecture involves $M$ encoders generating queries $q_{1}\cdots q_{M}$, the computation of a composite query $Q$ via bit-wise majority, and $N$ IMC cores performing similarity search over multiple copies of $Q$. (a, bottom) In the wireless implementation, the IMC cores receive different versions of $Q$ over space ($Q'\cdots Q^{N}$) that need to be decoded. (b) To enable the decoding of $Q'\cdots Q^{N}$ with low error, $q_{1}\cdots q_{M}$ are modulated with BPSK and shifted so that the overlapped symbols can be decoded easily at the receiver.}
    \vspace{-0.5cm}
\end{figure*}


\section{Motivation} \label{motiv}
The main aim of this work is to propose an architecture that employs IMC cores for HDC and that can be scaled to satisfy the insatiable appetite of the most demanding workloads.
The top chart of Fig.~\ref{fig:diagram} shows a logical diagram of a possible architecture template for an IMC-enabled HDC-based classifier. The encoding system on the left, possibly divided in a number $M$ of parallel encoders, translate the input data into query hypervectors. These hypervectors are then bundled via a bit-wise majority operation, which virtually increases the computation throughput proportionally to the number of bundled vectors. The search engine on the right, possibly composed by $N$ IMC cores storing $K$ class templates each, compares the bundled query hypervectors with the $N\times K$ prototype hypervectors. We note that both the encoders and the search engines can be implemented with IMC cores.

To scale such an architecture, two broad decisions need to be taken from an architectural standpoint. First, whether to scale by increasing the size of the IMC cores (scale-up; increasing $K$) or by placing more similarity search engines in the system (scale-out, increasing $N$). Second, whether the scaling is done in a fully integrated way, i.e. placing all encoders and search engines within a single chip, or using disintegrated alternatives such as the recent chiplet paradigm. Our proposed architecture, called WHYPE, is build upon three main observations:

\begin{figure}[!t]
    \centering
    \includegraphics[width=0.7\columnwidth]{images/scale_up_IMC_v2.pdf}
    \vspace{-0.1cm}
    \caption{Minimum wire width requirement of crossbar arrays as a function of the array size, in order to maintain the same effect from IR drop and other non-idealities on the MVM accuracy}
    \label{fig:scaleIMC}
    \vspace{-0.5cm}
\end{figure}

\vspace{0.1cm}\noindent
\textbf{Observation 1: In-memory cores do not scale up well.} Increasing the size $K$ of the IMC arrays allows accommodating a larger number and of prototype hypervectors (classes). However increasing the crossbar size brings with itself a number of non-ideal properties. For instance, IR drop across the array is increased due to the interconnect resistance of longer wires and RC latency is increased due to the increased parasitic capacitance. To counter these problems, one has to sacrifice one or few important metrics related to performance, power and area. For e.g. the IR drop can be reduced by increasing the wire width. The required minimum wire width at different array sizes to maintain the same effect of IR drop is given in Fig.~\ref{fig:scaleIMC} and in \cite{scaleupIMC}. As shown in the figure, the wire width exponentially rises, limiting the scalability of array sizes. Moreover, the complexity of weight programming also increases with the array size~\cite{ScaleUpXbar}. 
There are already IMC prototypes supporting up to 256 prototype hypervectors~\cite{InMemFSCIL2022}. This may be stretched up to 512 and 1024 but most likely not to bigger sizes due to the rise of non-ideal properties.
\textit{Therefore, scale-out architectures that employ multiple, but relatively small IMC cores, may be preferable in this scenario.}

\vspace{0.1cm}\noindent
\textbf{Observation 2: There exist different application spaces in terms of required search throughput, and the number of inputs and classes.} 
Following ~\cite{RahimiBiosignal2019}, a generic scalable HDC architecture for various workloads should include a set of encoders (shown in the left hand-side of Fig.~\ref{fig:large_arch}) and a set of similarity search engines (shown in the right hand-side of Fig.~\ref{fig:large_arch}). The number of encoders is typically determined by the demand of application, ranging from few encoders to operate on data from different sensory modalities~\cite{ChangEmotion2019,MitrokhinCNN2020}, to a larger number of encoders working with independent streaming channels~\cite{Hersche2021}. Similarly, the number of required similarity search engines is determined by the application.
It is based on the number of classes that the application has to support which ranges e.g., from a handful of classes~\cite{HDC_NatElec20,InMemFSCIL2022} to over one thousand classes~\cite{FSCIL2022}, or half million classes~\cite{ExteremeLearning_NIPS21}. Further, the number of \emph{active} classes could change over time, e.g., in a continual learning regime at the beginning there are about 60 classes that can grow up to 1600 during the course of incremental learning~\cite{FSCIL2022,InMemFSCIL2022}.
In computational terms, this means that the capacity of the encoders and similarity search engines is highly dependent on the application. We therefore aim for a generic architecture to be able to scale out by many inputs or classes coming in. Assuming a fully integrated design, architects would need to carefully dimension the encoders and similarity search engines in order to cater to the needs of a specific application. This fundamentally limits the reuse of the system in other domains, which may either require faster/broader or smaller/more efficient searches and where the designed architecture would either underperform or waste area and power. \textit{Such an observation suggests that a disintegrated architecture, possibly chiplet-based, would be a viable route for scaling-out IMC-based HDC architectures.} 

\vspace{0.1cm}\noindent 
\textbf{Observation 3: Scaling out HDC architectures is costly.}
In particular, scaling the architecture template described in Fig.~\ref{fig:diagram} quickly leads to a communication bottleneck, especially in chiplet-based systems. First, even though bundling compresses $M$ hypervectors of length $d$ into a single composite query, that comes at the cost of a heavy reduction communication flow. This $M$-to-1 traffic pattern is due to the need to bring the different hypervectors to a common circuit performing the bit-wise majority operation. The majority operation, besides the significantly large buffers required to store the $M$ inputs and the output hypervector, takes $M$ inputs of size $d$ with a known complexity of $d \times M^{3.5}$ gates \cite{choudhary2019generalized}. Only when the bundling is finalized, the output can be distributed to the $N$ IMC cores. However, broadcast patterns are generally expensive with growing $N$ \cite{ahmed2020asymmetric}. 

In summary, bundling and data movement quickly becomes a bottleneck when scaling, especially in disintegrated architectures. In chiplet-based systems, significant energy ($\sim$1 pJ) and latency ($\sim$20 ns) can be expected per hop due to connectivity and I/O pin limitations \cite{simba}, whereas hop counts will grow at least proportionally to the number of encoders $M$ and the number of search engines $N$ due to the reduction and broadcast flows, respectively \cite{wienna}. \textit{Hence, alternative implementations of the reduction-majority-broadcast pattern are necessary to enable the effective scale-out of HDC architectures.} 




%%%%%%%\input{03b-Architecture}

\section{WHYPE: A Wireless-Enabled Architecture for Scale-Out of Hyperdimensional Computing}
As depicted in Section \ref{motiv}, even though disintegrated scale-out is a desirable choice for scaling HDC systems, implementing such architectures with wired interconnects is challenging. This is because of three main reasons:
\begin{itemize}
    \item The bundling operation generates a reduction pattern that can create a communication bottleneck at the vicinity of the bundling circuit.
    \item The query distribution requires broadcast communication, which is inherently costly for wired interconnects in general and in chiplet-based systems in particular.
    \item The bundling operation creates an implicit barrier that forces encoding and similarity search to be done sequentially. While all three operations can be pipelined, the end-to-end latency increases.
\end{itemize}


Our proposed architecture, called WHYPE, addresses the three problems of wired scale-out at once. To that end, we augment a many-core HDC system with a wireless chip-scale network specifically designed to eliminate the need to transfer all the hypervectors to a central point for bundling. Fig.~\ref{fig:wireless_arch_diagram} shows the proposed implementation of WHYPE, which is composed by $M$ encoders augmented with wireless transmitters alongside $N$ IMC cores augmented with wireless receivers. The transmitters and receivers are slightly modified versions of a simple BPSK modulator and a coherent receiver and decoder, respectively. The frequency of operation is high, e.g. 60 GHz, to minimize the area and power overhead of the RF circuits and the antenna. Finally, given the monolithic nature of the system, we assume that the clocks of all transmitters are synchronized. 


The mode of operation is as follows. All the encoders broadcast, simultaneously and through the same wireless channel, the queries $q_{1}\cdots q_{M}$ that must be bundled to form $Q$. As a consequence of wave propagation within the package, the receivers will obtain $N$ different versions of the superposition of the $M$ symbols transmitted concurrently, $Q'\cdots Q^{N}$. Each receiver will then take its received signal and attempt to decode a single bit representing the majority operation of the $M$ transmitted symbols. In other words, the decoders will be carefully designed to achieve $Q'=Q$, $ \cdots$, $Q^{N}=Q$ with high probability so that, effectively, \textit{the bit-wise majority of the transmitted hypervectors is computed over the air.}

\begin{figure}[!b]
    \centering
    \includegraphics[width=1\columnwidth]{images/ota_examples_rxsv4.pdf}
    \vspace{-0.5cm}
    \caption{Example of decision regions of over-the-air (OTA) majority computation for three transmitters $\{q_1, q_2, q_3\}$ at two distinct receivers. Blue/green regions map to 0/1.}
    \label{fig:regions_ex}
\end{figure}


To further illustrate the idea behind over-the-air majority computation, Fig.~\ref{fig:regions_ex} shows a sample constellation that could result from the superposition of three transmissions. Each point represents one of the $2^{3}=8$ possible combinations. Since we are not interested in the values of the three transmitted bits, but rather in the result of the majority operation, the decoders only need to distinguish between the combinations that lead to $maj(\cdot) = 0$ and $maj(\cdot) = 1$, respectively. The key objective, then, is to modulate the information in each transmitter so that the received constellations form two easily separable clusters for $maj(\cdot) = 0$ and $maj(\cdot) = 1$. WHYPE achieves this with a very simple variant of source coding, i.e. through pre-assigned phase shifts. As detailed further in Section~\ref{sec:method1}, an exhaustive search is performed offline to find the transmitters' phase shifts leading to easy-to-decode majority at the receivers. 


At its core, WHYPE exploits three key opportunities, which allow to boost the value and minimize the disadvantages of wireless chip-scale communications:


\vspace{0.1cm} \noindent
\textbf{Key Opportunity 1: Over-the-air computing is possible because the channel is static and known beforehand.} OTA computing is certainly not new, but it has generally been hindered by the need of accurate and up-to-date channel state information, which is extremely hard to guarantee in conventional wireless networks \cite{altun2022magic}. In contrast, the in-package scenario is static and allows for a pre-characterization of the channel \cite{Matolak2013CHANNEL}, hence allowing for an OTA calculation of the majority operations required by the bundling of hypervectors.


\vspace{0.1cm} \noindent
\textbf{Key Opportunity 2: The inherent broadcast nature of wireless communication allows for a single-hop distribution of bundled hypervectors.} By using omnidirectional antennas such as vertical monopoles \cite{Pano2020a}, data is naturally broadcast with a latency and efficiency hard to achieve with wired on-chip networks \cite{orthonoc}. This feature, together with the OTA bundling, allows to eliminate the communication bottleneck of HDC scale-out architectures.

    
\vspace{0.1cm} \noindent
\textbf{Key Opportunity 3: The resilience of the HDC paradigm to errors makes it tolerant to unreliable communication}. A drawback of wireless communications in general (and of OTA computing in particular) is that it can suffer from relatively high error rates when compared to wired communications. This generally leads to low energy efficiency. 
However, as we illustrate in Fig.~\ref{fig:ber_acc}, HDC is inherently resistant to errors and opens the door to the use of unreliable wireless communications without compromising the energy efficiency and scalability of the architecture. 


\begin{figure}[!h]
    \centering
    \includegraphics[width=0.8\columnwidth]{images/ber_accv4.pdf}
    \vspace{-0.2cm}
    \caption{Impact of the bit error rate (rate of erroneous bits in a bundled hypervector) on the accuracy of a classification task under the conditions described in Section \ref{sec:searchMethod}.}
    \label{fig:ber_acc} 
    \vspace{-0.3cm}
\end{figure}




\begin{figure*}[!t]
    \centering
    \includegraphics[width=0.53\textwidth]{images/methodology_figure2.pdf}\hfill
    \includegraphics[width=0.47\textwidth]{images/aerial_crosssection_V4.png}
    \vspace{-0.4cm}
    \caption{Overview of the evaluation methodology and layout of a sample architecture with 3 TXs and 64 RXs. The package is enclosed in a metallic lid and empty spaces are filled with vacuum. $h_1=0.1$ mm; $h_2 = 0.01$ mm; $l_1 = 7.5$ mm; $s = 3.75$ mm; $L_1 = 33$ mm; $L_2 = 30$ mm.}
    \label{fig:meth}\label{fig:arch}
    \vspace{-0.5cm}
\end{figure*}



%%%%%%%%%\section{Methodology}
\label{sec:methodology}

In this section, we present STIXnet in more detail on how it works and which algorithms and models are used to perform Information Extraction on unstructured Cyber Threat Intelligence reports. We first show its pipeline and briefly overview the various modules (Section~\ref{subsec:pipeline}). Then we give more details on its main components: the text extraction module (Section~\ref{subsec:textextraction}), the entity extraction module (Section~\ref{subsec:entityextraction_methodology}) and the relation extraction module (Section~\ref{subsec:relationextraction_methodology}).

\subsection{Pipeline}
\label{subsec:pipeline}

STIXnet performs a highly complex Information Extraction task on many different types of entities and relations. There are indeed 18 different types of entities compliant with the STIX standard and more than 100 types of relations. To accomplish this, STIXnet uses different modules for each one of the different tasks that it must achieve: textual extraction, entity extraction, and relation extraction. A graphic overview of the STIXnet platform is shown in Figure~\ref{fig:pipeline}.

\begin{figure*}[!htpb]
  \centering
  \includegraphics[width=\linewidth]{Figures/04-Pipeline3.pdf}
  \caption{Pipeline of the STIXnet platform.}
  \label{fig:pipeline}
\end{figure*}

\begin{itemize}
    \item \textbf{Text Extraction}: the first module converts the program's input into raw text. While doing this, artifacts are inevitably be introduced in the text, and therefore they must be handled to obtain a single string of characters in a common encoding.
    \item \textbf{Entity Extraction}: this module handles the extraction of the different entities in the text. It uses four different sub-modules to accomplish that:
    \begin{itemize}
        \item \textbf{IOC Finder}: we extract Indicators Of Compromise by using different regular expression rules and looking at text patterns.
        \item \textbf{Knowledge Base Entities Extraction}: using the entities in a Knowledge Base, we use an efficient algorithm for string search in a text to retrieve names and aliases. We mitigate errors and false positives caused by this approach using NLP techniques.
        \item \textbf{Novel Entities Extraction}: using NLP libraries, we extract entities not present in the Knowledge Base, which can then be integrated into the KB.
        \item \textbf{TTPs Extraction}: techniques and tactics are not always represented in a text by their name but can be implicit and not explicitly expressed. We use a Machine Learning model trained on MITRE ATT\&CK tactics and techniques to recognize them.
    \end{itemize}
    \item \textbf{Relation Extraction}: from the extracted entities, we now retrieve the relations. We use two sub-modules for this task:
    \begin{itemize}
        \item \textbf{Rule-Based Approach}: using NLP techniques, we perform Dependency Parsing and compute the shortest paths between entities. Comparing the verb inside the path with the ones in the STIX relationships, we estimate the most similar one with a degree of confidence.
        \item \textbf{Deep Learning Based Approach}: to adjust the results of the previous approach, we also compute embeddings from the sentences with a Deep Learning model. We then determine the similarity between these embeddings and those computed from the list of relationship labels.
    \end{itemize}
    \item \textbf{Output}: we create a JSON file from the extracted entities and relations, which is processed by the graphical interface of the platform to be interactive and dynamic.
\end{itemize}

Eventual interactions with the Knowledge Base or other platform components are disclosed in the individual sections of the different modules. Indeed, we show that by interacting with the database, it is possible to improve performance over time, particularly if an analyst decides to validate the results of the STIXnet output. 

Furthermore, the structure of the STIXnet pipeline allows for easy and immediate management of the different modules. Formally defining the interactions between modules and submodules allows results from different pipeline components to be compared and merged in a unique output. Thus, researchers and organizations can implement the STIXnet framework in different IE scenarios and add or remove components depending on their needs.

\subsection{Text Extraction}
\label{subsec:textextraction}

One of the platform's most relevant and sensible aspects is its input. As mentioned, STIXnet can take in input various reports and bulletins, which can come from various vendors or sources. For this reason, reports have some stylistic and linguistic differences. However, data must be converted univocally before processing the raw text to have consistent processing between different inputs and a common ground for evaluating the various modules. This means considering many different aspects that can change from input to input. First of all, we must be able to parse text from files with different formats, and thus we use Apache Tika\footnote{\url{https://tika.apache.org/}} for \texttt{pdf} and \texttt{doc} files and ConvertAPI\footnote{\url{https://www.convertapi.com/}} to extract text from the HTML data of web reports. However, since text can be formatted in many different ways, we must process it to remove artifacts, fix line breaks, and remove eventual sanifications on IP and other addresses. This phase is crucial to ensure that the following modules are presented with a clean input; otherwise, artifacts are propagated in the pipeline and compromise the results.

\subsection{Entity Extraction}
\label{subsec:entityextraction_methodology}

This section tackles the different submodules used for entity extraction: IOC Finder (Section~\ref{subsub:iocfinder}), a rule-based entity extractor (Section~\ref{subsub:kbentext}), a novel entity extractor (Section~\ref{subsub:novelentityextraction}), and a TTPs extractor (Section~\ref{subsub:ttpext}). Finally, we clarify how the submodules interact with one another and merge their results (Section~\ref{subsub:subint}).

\subsubsection{IOC Finder}
\label{subsub:iocfinder}

The particular structure of Indicators Of Compromise allows us to use regular expression (Regex) rules to find them in the text. Moreover, in some of the reports distributed by CTI vendors, the end of the document is often presented with a table containing the IOCs of interest for that particular topic. While different, all these indicators share a common structure across each type, which can be recognized with Regex rules without applying NLP techniques. Some examples of the types of IOCs supported by IOC Finder can be found in Table~\ref{tab:iocfinder}. During the execution of STIXnet and after processing the report as raw text, the first step is to run the IOC Finder submodule on it, which returns a dictionary containing the entities found.

We implement this module by forking an open-source project by Floyd Hightower\footnote{\url{https://github.com/fhightower/ioc-finder}}. To adapt it for our pipeline, we contributed to the main project by updating some libraries to a newer version and adding the ability to track the position of the found IOCs. The code of the forked project can be found in our repository.

\begin{table}[!htpb]
  \centering
  \caption{Examples of IOCs and their structure.}
  \label{tab:iocfinder}
  \begin{tabular}{ll}
    \toprule
    \textbf{IOC Type} & \textbf{IOC Structure}\\
    \midrule
    ATT\&CK Techniques & \texttt{T1518} or \texttt{T1518.001}\\
    CVEs & \texttt{CVE-2021-44228} \\
    Email Addresses & \texttt{example@mail.com} \\
    File Paths & \texttt{/path/to/file} \\
    MD5s & \texttt{e802c6b77dd5842906ed96ab1674c525} \\
    \bottomrule
\end{tabular}
\end{table}

\subsubsection{Knowledge Base Entity Extraction}
\label{subsub:kbentext}

After finding IOCs, all other entities of interest do not share a common structure and thus cannot be found through regular expression rules. Thus, we leverage a rich Knowledge Base integrated with multiple OSINT that explicitly indicate which names represent important entities and allow us to link them to their correct entity type. For this reason, a rule-based algorithm can search specific words in the text, retrieve their position and thus highlight them as extracted entities. The different sources for the intelligence are:

\begin{itemize}
    \item \textbf{Knowledge Base}: Leonardo S.p.A., an Italian multinational company that collaborated in this research, provided a rich database of STIX entities. Cyber threat intelligence analysts built this database over the years at their Security Operation Center (SOC), which read and manually annotated entities from many reports.
    \item \textbf{MITRE ATT\&CK}: the ATT\&CK framework can be used as a source of intelligence for different entities such as techniques, tactics, groups, and software (of which the conversion to the STIX standard has been disclosed in Table~\ref{tab:mitre2stix}). To retrieve this data, we use the Trusted Automated Exchange of Intelligence Information (TAXII) application protocol~\cite{connolly2014trusted}, which allows for exchanging threat intelligence over HTTPS and defines a RESTful API that can be used to provide or collect data.
    \item \textbf{Locations}: to retrieve the names of countries and continents, we used a \texttt{csv} file in which each nation is associated with its nationality. In this way, we are able to identify locations even when used as an attribute to another entity (e.g., "a \textit{Russian} malware").
\end{itemize}

After retrieving the entities, a quick pre-processing is performed to unify their formats and add the possibility of aliases for each one. Through aliases, it is possible to recognize an entity in a text and map it to the correct one, avoiding duplicates and fixing the issue of multiple names for a single Advanced Persistent Threat. We then use the Aho-Corasick algorithm to find terms of this thesaurus of words in the report~\cite{10.1145/360825.360855}. To mitigate false positives, we process each sentence with NLP techniques, in particular, Part-Of-Speech Tagging (POS tagging), allowing us to assign part-of-speech tags to each word (e.g., noun, verb). After defining a table of entity types and their possible POS tags, we use it to compare the entities found in the report with their extracted POS tag. For example, "us" can be used as a pronoun or can be used as a noun to reference the United States, constituting an entity of type "location".

\subsubsection{Novel Entities Extraction}
\label{subsub:novelentityextraction}

Some CTI reports are published to spread awareness of newly discovered actors, malwares, or techniques. These entities are thus named by CTI researchers and analysts and, for this reason, are most likely not present in the Knowledge Base. To find these new entities, we can leverage the previous execution of POS Tagging to extend its results and create a dependency graph from the tokens found in each sentence. In this way, we identify specific patterns used in the text to express a new entity. To create such a graph, we leverage both the POS tags and the dependencies between the tokens, as shown in Figure~\ref{fig:spacygraph}. To perform this processing, we use Spacy\footnote{\url{https://spacy.io/}}, a free, open-source library for advanced NLP in Python. By looking at numerous reports and bulletins from different vendors and sources, we can identify a limited number of ways a new entity can be introduced, allowing us to write pattern rules that the NLP processing can recognize.

\begin{figure*}[!htpb]
  \centering
  \includegraphics[width=\linewidth]{Figures/04-Spacy.pdf}
  \caption{Example of dependency graph generated from a sentence.}
  \label{fig:spacygraph}
\end{figure*}

\subsubsection{TTPs Extraction}
\label{subsub:ttpext}

While malwares and threat actors are often explicitly mentioned, some other entities are not and can be referenced without categorically stating their names. It is the case of tactics and techniques, which constitute the TTPs mapped into the STIX objects "x-mitre-tactic" and "attack pattern". Rule-based methods cannot be used to retrieve these entities since the variance that characterizes the expression of these concepts is too broad and is hardly definable through a set of rules. For this reason, a multi-label text classification model must be deployed and trained on the MITRE ATT\&CK Knowledge Base of tactics and techniques. We already presented a tool named rcATT~\cite{https://doi.org/10.48550/arxiv.2004.14322} that suffered from its age since it was published in 2020, and the MITRE ATT\&CK framework has had several changes and renovations ever since. The source code for rcATT is publicly available in their GitHub repository\footnote{\url{https://github.com/vlegoy/rcATT}}, and thus we retrained it from scratch with new techniques and tactics. Also, to address one of the limitations and future works of the paper presenting rcATT, we expanded the training dataset with more data from MITRE ATT\&CK descriptions and external sources for each tactic and technique. The new dataset, the pre-trained models, and the updated code for its training can be found in our repository.

\subsubsection{Submodules Interaction}
\label{subsub:subint}

As stated in Section~\ref{subsec:pipeline}, the pipeline structure of our framework allows us to differentiate the tasks in many modules. For entity extraction, in particular, we identified four different submodules independent from one another. Their input is always the same and is constituted by the textual data of the report. Once each submodule produces an output, a final check is performed to ensure no overlapping occurs during the processing. This process includes cross-checking the found entities in the Knowledge Base to maximize their confidence in their extracted type and the addition of the novel entities. More details are disclosed in Section~\ref{subsec:entityextraction_results}. Finally, all the entities are merged into one data structure, constituting the following module's input. Since this procedure is automatic, submodules can be added and removed in the pipeline according to the user's needs, and no conflicts occur during the process.

\subsection{Relation Extraction}
\label{subsec:relationextraction_methodology}

This module can retrieve relations between the found entities while processing the sentences in the raw text. However, one of Spacy's limitations is its inability to grasp relations between distant entities in a text. To address this, we propose two different approaches for relation extraction. In the first approach, we leverage POS Tagging and Dependency Parsing to compute a graph of each sentence and retrieve relations by looking at the shortest paths between entities (Section~\ref{subsub:rulebased}). In the second approach, we use a Transformer model to compute embeddings of the sentences and compute their similarity (Section~\ref{subsub:dlbased}). Finally, we clarify how the submodules interact with one another and merge their results (Section~\ref{subsub:relsubint}).

\subsubsection{Rule Based Approach}
\label{subsub:rulebased}

The main idea of the rule-based approach is to leverage the dependency graphs already computed by Spacy for novel entity extraction (Section~\ref{subsub:novelentityextraction}) and use graph theory functions to grasp the relation between two entities inside a sentence. In particular, we can process the dependency graph and retrieve the relations between any couple of nodes by discovering the Shortest Dependency Path (SDP), i.e., the shortest path between two nodes in the graph. It has been observed in other studies that the nodes in the SDPs usually contain the necessary information to identify a relationship between two entities while also being dependent on the structure and semantic complexity of the sentence~\cite{hua2016shortest, xu-etal-2015-classifying}.

After retrieving the shortest path between each couple of entities in the sentence, we extract their STIX type and focus on the verbs in the path. For example, considering the sentence in Figure~\ref{fig:spacygraph}, the extracted paths (between the three entities \texttt{"APT29"}, \texttt{"7-Zip"}, and \texttt{"Raindrop"}) are:
\begin{itemize}
    \item \texttt{[APT29, \underline{used}, 7-Zip]};
    \item \texttt{[APT29, \underline{used}, \underline{decode}, malware, Raindrop]};
    \item \texttt{[7-Zip, \underline{used}, \underline{decode}, malware, Raindrop]}.
\end{itemize}
By looking at the entity types and the root form of the verbs (underlined in the previous example), it is possible to compare them with the ones found in the list of STIX relationships and thus label the path as a specific STIX Relationship Object (SRO). However, since the verbs used to describe the relationship in the sentence might not be the same as the associated SRO, we use a similarity function to determine their likeliness. If it surpasses a certain threshold, we can consider them synonyms. To accomplish this, we use the Wu \& Palmer similarity function~\cite{https://doi.org/10.48550/arxiv.cmp-lg/9406033}, which, given two words and their synsets (i.e., groupings of similar words that express the same concept), outputs a value in the range $\left[0,1\right]$, where $1$ means maximum similarity. For this reason, this value can be used as a confidence measure of the relation. The taxonomy used for this task is the WordNet taxonomy, an extensive lexical database of English words developed by Princeton University~\cite{Fellbaum2010}.

For example, considering the SDP \texttt{[APT29, used, 7-Zip]}, the entity extraction module identified \texttt{"APT29"} as an \texttt{intrusion-set} and \texttt{"7-Zip"} as a \texttt{tool}. In the list of SROs, there is only one entry containing both an \texttt{intrusion-set} and a \texttt{tool}, which is "intrusion-set uses tool": since the root form of the verb in the SDP is equal to the one in the SRO, we label it accordingly with maximum confidence. Instead, considering another SDP \texttt{[APT29, attacks, the, US]} (where \texttt{"US"} is identified as a \texttt{location} entity), there are two SROs including both an \texttt{intrusion-set} and a \texttt{location}: "intrusion-set originates-from location" and "intrusion-set targets location". In this case, none of the root forms of SROs verbs ("originate" and "target") coincide with the one in the SDP ("attack"), and thus we compute the similarity between them:
\begin{equation*}
    wup("attack", "originate") = 0.4,
\end{equation*}
\begin{equation*}
    wup("attack", "target") = 0.5.
\end{equation*}
While keeping the confidence threshold at 0.5, we can label the SDP as a relationship of type "intrusion-set targets location".

The confidence value in the extracted relations becomes particularly important when dealing with sentences containing multiple entities. Indeed, the number of relations increases exponentially with the number of entities found, and some of the extracted relations could not exist. To include a "non-relation" label in this task, we set a threshold for the confidence, under which we discard the extracted relations.

\subsubsection{Deep Learning Based Approach}
\label{subsub:dlbased}

The rule-based approach is particularly efficient when dealing with simple phrases or when entities are close in the graph. However, many elements might be introduced in the SDP whenever two entities are far from each other. The verb with the highest similarity could be linked to different tokens in the text and might not reach the confidence threshold. To address this problem, we use embeddings, fixed-size vectors that can also be generated from textual data by Deep Learning models such as Transformers~\cite{https://doi.org/10.48550/arxiv.1706.03762}.

In this specific case of relation extraction, we are interested in computing the similarity between each sentence's embeddings and the STIX relationships' embeddings. To perform these embeddings, the best tool at our disposal is Sentence BERT (SBERT)\footnote{\url{https://www.sbert.net/}}, a variation of the BERT model (Bidirectional Encoder Representations from Transformers) developed by Google AI language~\cite{https://doi.org/10.48550/arxiv.1905.05950}. While BERT constitutes the state-of-the-art in many NLP applications, it becomes inefficient when dealing with a large corpus of sentence processing. SBERT addresses this problem using siamese and triplet network structures, drastically reducing processing time~\cite{reimers-2019-sentence-bert}.

For its STIXnet implementation, we compute the embeddings of the different STIX relationships and the sentences extracted from the report. For each sentence, we perform a pre-processing procedure for each contained entity couple by substituting their tokens with their extracted STIX type. Then, we compute the cosine similarity between these embeddings and normalize it to use it as a confidence value. We also use a threshold for the confidence of the relation to discriminate false positives (0.5).

\subsubsection{Submodules Interaction}
\label{subsub:relsubint}

As with the entity extraction module, the relation extraction task is divided into different submodules that work independently. While their input is always the same (i.e., textual data from the report and the previously identified entities), the two submodules perform virtually the same task this time. Thus we expect a degree of overlap in their results. However, their coexistence is necessary to extract relations from both simple and complex scenarios. To handle conflicts in the possible output, we use the confidence values generated during the processing and keep the relations between entities with maximum confidence, which must also be over the acceptance threshold. Therefore, users can add their desired submodules for the relation extraction task, and STIXnet will automatically merge their results.

\section{Methodology} \label{meth}
Fig. \ref{fig:meth} summarizes the procedures followed to evaluate WHYPE from the perspectives of wireless communications and HDC architectures. First, the computing package has been modeled in CST \cite{cst} assuming a disintegrated architecture with $M+N$ chiplets with their respective antennas and transceivers, implementing the scheme from Fig.~\ref{fig:diagram}. The output of CST simulations is fed to MATLAB to assess the bit error rate (BER) of the resulting constellations. The BER is then used in a python-based HDC framework to characterize the impact of imperfect communication on the HDC classification accuracy. 

Next, we describe the methods to obtain the source coding in Sec.~\ref{sec:method1}, to assess the speed and reliability of OTA computing in Sec.~\ref{sec:method2}, to evaluate the classification accuracy in Sec.~\ref{sec:method3}, and the details of the classification benchmark in Sec.~\ref{sec:method4}. 

\subsection{Source Coding Optimization}
\label{sec:method1}
As shown in Fig.~\ref{fig:wireless_arch_diagram}, transmitters encode the bits of their queries using a BPSK encoder plus two specific phase shifts (for the symbols '0' and '1', respectively). That is, all symbols will have the same amplitude, but a different phase. Since we let the $M$ encoders to transmit simultaneously, each of the $N$ receivers will observe a slightly different superposition of all the transmitted symbols. This leads to a $N$ different constellations of $2^M$ points each.

In this context, the objective is to select the phase shifts at each transmitter so that all the received constellations are clustered in two separable decision regions, each corresponding to the case where $maj(\cdot)=0$ and $maj(\cdot)=1$, respectively. This optimization process has two constraints. On the one hand, we have to make sure that each transmitter only uses two phases, for its symbol '0' and '1', respectively. On the other hand, the phases at one transmitter have an impact on the constellations of all receivers, which implies that \textit{a joint optimization considering all RXs is needed}.

The optimization process starts by simulating the simultaneous transmissions in CST. We consider a set of 8 phases in each transmitter (i.e. in 45 degree steps) and evaluate the amplitude and phase obtained at each of the receivers. Then, the results are fed to MATLAB, where the decision regions are computed using the $K$-means clustering algorithm with $K=2$ over the constellation. These regions are used to evaluate the average error across all receivers using the methods described in Sec.~\ref{sec:method2}. The combination of phases leading to the lower average error rate is selected. An illustrative example with three transmitters and three receivers is shown in Fig.~\ref{fig:cst_res}, with the detail of a specific receiver in Fig.~\ref{fig:otaasig_tab2}.


\begin{figure}[!t]
    \centering
    %\begin{subfigure}{0.9\columnwidth}
    \includegraphics[width=\columnwidth]{images/cst_9c_res_v3.png}
    %\caption{Electromagnetic phase sweep outcome}
    %\end{subfigure}
    %\begin{subfigure}{0.9\columnwidth}
    \includegraphics[width=\columnwidth]{images/ota_nice_v6.png}
    %\caption{Chosen constellations. Blue symbols map to logical 0 and green symbols map to logical 1.}
    %\end{subfigure}
    \vspace{-0.1cm}
    \caption{Sweep of all possible phase combinations (top) and the one that  minimizes the error rate of the majority computation (bottom). Blue/green symbols map to logical 0/1.}
    \label{fig:ota_constellations}\label{fig:cst_res}
\end{figure}


\subsection{Over-The-Air Computing Evaluation}
\label{sec:method2}
\noindent
\textbf{Frequency-Domain Simulation.} As described in the previous section, electromagnetic simulations are needed to assess the performance of the over-the-air computation process. We model an interposer-based package with $M+N$ chiplets, each with its own antenna, with the dimensions depicted in Fig.~\ref{fig:arch}. We sweep the phases in each transmitter and obtain the amplitude and phase at the receivers in the frequency domain. The operating frequency is 60 GHz and the transmission power is 0 dBm per antenna, compatible with existing WNoCs \cite{timoneda, multichannel}. Further, we assume $M=3$ transmitters and $N=64$ receivers, unless otherwise noted. Nevertheless, the analysis can be extended to higher frequencies, different power levels, different package configurations, or different number of transmitters and receivers.


\begin{figure}[!t]
    \centering
    \vspace{-0.3cm}
    \includegraphics[width=0.8\columnwidth]{images/ota_assignment_w_table_v5.pdf}
    \vspace{-0.2cm}
    \caption{Constellation and truth table with transmitted phases/ bits for a specific RX. Blue/green symbols map to logical 0/1.}
    \label{fig:otaasig_tab2} 
    \vspace{-0.4cm}
\end{figure}

\vspace{0.1cm} \noindent
\textbf{Reliability Analysis.} Once the phases are swept and the candidate constellations are obtained, we compute the BER at each RX, for all the different possible constellations, and choose the one that leads to the lowest average BER across RXs. In all cases, the BER is evaluated approximating the centroids of each cluster as received symbols, and using the analytical expression of error rate of BPSK, which is the modulation used in WHYPE. This yields
\begin{equation}
    \label{eq:bpsk}
    %BER^{BPSK} = Q\bigg({\frac{0.5\cdot d_c}{\sqrt{N_0/2}}}\bigg),
    BER^{BPSK} = 0.5\cdot \text{erfc}\bigg({\frac{0.5\cdot d_c}{\sqrt{N_0}}}\bigg),
\end{equation}
where $erfc(\cdot)$ is the complementary error function, $d_c$ is the distance among centroids and $N_0$ is the noise spectral density. 

\vspace{0.1cm} \noindent
\textbf{Transmission Speed Analysis.} To assess the speed at which the information can be modulated reliably, time-domain simulations are required. In particular, we use CST to obtain the channel impulse response $h_{i}(t)$ with simultaneous source excitation from the $M$ transmitters to the receiver $i \in [1, N]$. 
Then, the power-delay profile (PDP) of receiver $i$, $P_{i}(\tau)$, is evaluated as $P_{i}(t) = |h(t)|^2$. The delay spread $\tau^{(i)}_{rms}$ between the sources and receiver $i$ is calculated as the second moment of the PDP as 
\begin{equation}
    \label{eq:ds2}
    \tau^{(i)}_{rms} = \sqrt{\frac{\int (\tau - \bar{\tau_{i}})^2 P_{i}(\tau)d\tau}{\int P_{i}(\tau)d\tau}} ,
\end{equation}
where $\bar{\tau_{i}} = \frac{\int \tau P_{i}(\tau)d\tau}{\int P_{i}(\tau)d\tau}$ is the mean delay of the PDP for receiver $i$. Since all transmitters need to be synchronized, the modulation speed needs to be the same for all transmitters and below the coherence bandwidth $B_{c}$ of the system. Hence, we calculate the coherence bandwidth via the worst-case delay spread among all receivers, $\tau_{rms}$, so that
\begin{equation}
\tau_{rms} = \max_{i} \tau^{(i)}_{rms} \Longrightarrow B_{c} = \frac{1}{\tau_{rms}}.
\end{equation}


\subsection{Similarity Search Evaluation}
\label{sec:searchMethod}\label{sec:method3}
Once the final transmitter phases leading to the lowest average BER are chosen, an in-house python framework is used to evaluate the impact of imperfect OTA majority computation on the accuracy of the classification task. Every similarity search engine connected to a receiver stores 64 different prototype hypervectors, i.e., 64 different classes, each with 512-bit that suffices for the scenario considered in this paper. This is compatible with current experimentally validated IMC cores \cite{HermesVLSI,Y2022khaddamJSSC}. The area and energy evaluations in this paper assumed a scaled version of the IMC from \cite{HermesVLSI,Y2022khaddamJSSC}. Finally, errors coming from the OTA computations are modeled as uncorrelated bit flips over the query hypervectors.

\vspace{0.1cm} \noindent
\textbf{Bundling alternatives.} While the baseline bundling consists of simply computing the bit-wise logical majority result across the different TX bits, we also consider permuted bundling. This bundling consists of permuting the queries in the TXs prior to applying the majority operation on them. By permuting the hypervectors we obtain two benefits. First, this allows the identification of the transmitter of the detected class from the composite query. The second direct benefit of permuting the hypervectors is that it helps increasing the quasi-orthogonality between them. This has a direct impact on accuracy, since the majority operation over multiple non-orthogonal hypervectors (i.e. not permuted) would yield a bundled hypervector that is hard to classify. 


\subsection{Classification Benchmarks}
\label{sec:method4}
We perform experiments using the Omniglot dataset \cite{Omniglot_dataset}. The dataset provides handwritten images of characters from 50 different alphabets. The number of characters in each alphabet varies from 14 to 55. In total there are 1623 character classes and 20 example images from each class. The dataset is further divided into a training and test set containing 964 and 659 character classes respectively. The goal of this benchmark is to train an encoder on the image and label data given in the training set and evaluate classification accuracy on the test images. We perform two types of experiments.

\vspace{0.1cm} \noindent
\textbf{Experiment 1: Few-shot Learning.} First, we evaluate the few-shot learning capability of the system. We do this by, first, meta-training an encoder on the training set data and evaluating the few-shot classification accuracy over a series of 1000 episodes. In each episode, we select 100 classes and 20 encoded hypervector examples (shots) from each class within the test set. These support vectors are distributed among the IMC modules on the RXs. From the remaining images of the same 100 classes, we select 1 query image to encode per TX. The encoded query hypervectors are over-the-air bundled (with or without permutation) and received at each RX to perform the similarity search. The final classification accuracy is the average accuracy across the 1000 episodes.

\vspace{0.1cm} \noindent
\textbf{Experiment 2: Continual Learning.} Secondly, we focus on the continual learning capability of the system. For this, we have a similar setting to that of Experiment 1, with few changes. Here, instead of always choosing a fixed 100 classes per episode, we start with 64 classes and gradually incorporate new classes over a series of sessions. In each session, 64 additional classes are selected from the set of unselected classes so far and 5 selected support examples from these novel classes are provided to the IMCs. During a session, queries can be selected from all classes (both novel and old) currently under selection and one each is assigned to the encoders at the transmitters. 


%%%%%%%%%%%%%\section{Experimental results} \label{sec:results}

\subsection{Deployment on Crazyflie with AI-deck}

Figure~\ref{fig:roc-curve} reports the CNN performance on the testing set; the full-precision model achieves an Area Under the ROC curve score of 98.88\%, with negligible performance loss ($-$0.09\%) after 8-bit integer quantization.
After binarizing outputs at a threshold of 0.5, the model achieves an accuracy of 95.1\%.

\begin{table}[t]
    \centering
    \caption{Power consumption and Area occupation}
    \begin{tabular}{|c|c|c|c|c|c|} \hline
                           & Area      & Leakage & Dynamic            & Max Freq & Max Power \\ 
                          & ($mm^2$)   & ($mW$)  & ($\frac{uW}{MHz}$) & ($MHz$)     &  ($mW$) \\ \hline
         Top              & 7.28       & 4.23    & 214.7              & 450       & 100.53 \\\hline
         CVA6             & 0.49       & 4.79    & 47.5               & 900       & 47.54  \\ \hline
         PMCA             & 1.56       & 5.78    & 206                & 400       & 88.18  \\ \hline
         Mem Ctrl.        & 0.27       & 0.14    & 2.3                & 450       & 1.16   \\ \hline
         Opentitan        & 0.86       & 4.53    & 16                & 350       & 10.13   \\ \hline
         Total            & 7.28       & 19.47   & 486.5             & -         & 247.54  \\ \hline
    \end{tabular}
    \label{tab:power_table}
\end{table}

The end-to-end message transmission is assessed with an experiment in which the transmitter drone sends a sequence of 256 messages with a payload values from 0x00 to 0xFF. 
The observer drone, placed at a fixed distance of \SI{30}{\centi\meter}, is always in line-of-sight with the transmitter one and decodes the received messages, at 30 FPS, with the quantized CNN.
All 256 messages are decoded correctly. 
Figure~\ref{fig:incremental-samples} reports a subsequence of the received messages and a supplementary video demonstration is provided at \url{https://youtu.be/TClcuUWJe0U}.


\subsection{Physical Implementation \& Performance Evaluation}

The proposed SoC has been implemented in the Global Foundries \SI{22}{\nano\meter} FDX technology, employing the Synopsys Design Compiler for the logical synthesis and the place and route with Cadence Innovus. 
For the SoC's signoff we used the Synopsys PrimeTime, considering the worst case operating corner for a nominal supply voltage of \SI{0.8}{\volt} (SS, \SI{0.72}{\volt}, 125\textdegree C/-40\textdegree C), while power analysis was performed in typical operating conditions (TT, \SI{0.8}{\volt}, 25\textdegree C)
The layouts of the SoC and the main subsystems composing it are shown in Figure~\ref{fig:layout}, while Table~\ref{tab:power_table} summarizes the physical implementation.

\begin{figure}[t!]
  \centering\includegraphics[width=\columnwidth]{images/orizontal_layout.png}
  \caption{In the middle, the layout of the SoC. Around it, the layouts of the PMCA, secure subsystem, HyperBus and CVA6.}
  \label{fig:layout}
\end{figure}

To evaluate the performance of the proposed SoC on the UVC use-case, we first deploy our CNN on the PMCA of GAP8 SoC hosted on the COTS Carzyflie nano-drone.
The full inference of the DNN on the proposed SoC takes \SI{3.7}{\mega cycles}, meaning that each payload's bit can be recognized by the receiver drone in \SI{126}{\milli\second} and that a full message described in Figure~\ref{fig:incremental-samples} can be recognized in \SI{1.3}{\second} by our proposed SoC.
This is 2.3$\times$ faster than the same application running on the GAP8 SoC (@\SI{175}{\mega\hertz}).
By coupling a linux-capable application core with a parallel programmable accelerator and a secure enclave within the power budget of \SI{250}{\milli\watt} and a footprint of \SI{9}{\square\milli\meter}, our SoC represents an appealing solution for secure and high-performance mission computers for nano-drones, paving the way for a wide range of new secure applications.


\section{Performance Evaluation} 
\label{results}
We next present the evaluation of WHYPE, first from the perspective of the OTA majority in Section \ref{sec:OTAeval}, and then from the perspective of the classification task in Section \ref{sec:HDCeval}. Additionally, we evaluate the area and power overhead of the architecture in Section \ref{sec:OVHDeval}.

\subsection{Over-the-Air Computing}
\label{sec:OTAeval}
Fig.~\ref{fig:cst_res} illustrates the exhaustive search performed in our system with $M=3$ transmitters, shown only in three receivers chosen at random. The bottom chart of the same figure shows the resulting constellations after the optimization. The constellation in another random receiver is shown in more detail in Fig.~\ref{fig:otaasig_tab2}, together with the phases chosen for the three transmitters: 0\textsuperscript{o}/90\textsuperscript{o}, 315\textsuperscript{o}/135\textsuperscript{o}, and 225\textsuperscript{o}/180\textsuperscript{o} for the symbols '0'/'1' of transmitters TX1, TX2 and TX3, respectively. 

\vspace{0.1cm} \noindent
\textbf{Reliability Analysis.} Once the phases are set, we evaluate the error rate of all 64 receivers using Eq.~\eqref{eq:bpsk}. Fig.~\ref{fig:ber_Rxs} shows the BER of each receiver. It can be observed how the BER values are highly dependent on the receiver position, with values as large as $\sim$0.1 and also lower than 10\textsuperscript{-5} in a significant amount of cases. In average, the error rate is below 0.01. 



\begin{figure}[!t]
    \centering
    \vspace{-0.1cm}
    \includegraphics[width=0.9\columnwidth]{images/ber_rxs_avg_v3.pdf}
    \caption{Error rate for each individual RX in the architecture. The dashed line indicates the average value. }
    \label{fig:ber_Rxs} 
    \vspace{-0.3cm}
\end{figure}


\begin{figure}[!t]
    \centering
    \vspace{-0.2cm}
    \includegraphics[width=0.9\columnwidth]{images/ber_scaling_rxs_v2.pdf}
    \caption{Error rate as a function of the number of receivers.}
    \label{fig:arch_scal} 
    \vspace{-0.4cm}
\end{figure}

\begin{figure}[!t]
    \centering
    \includegraphics[width=0.8\columnwidth]{images/worst_ds_perTX_numRXs_v3.pdf}
    \vspace{-0.2cm}
    \caption{Scaling of the delay spread and coherence bandwidth over an increasing number of receivers for the different transmitted symbols in the scenario with three transmitters.}
    \label{fig:worst_ds} 
    \vspace{-0.3cm}
\end{figure}

\begin{figure*}[!t]
    %\centering
    \begin{subfigure}{0.495\textwidth}
    \includegraphics[width=1\textwidth]{images/hdframe1_wireless-v3.png}
    \vspace{-0.5cm}
    \caption{Baseline bundling}
    \label{fig:sim_res_baseline} 
    \end{subfigure}
    \hfill
    \begin{subfigure}{0.485\textwidth}
    %\centering 
    \includegraphics[width=1\textwidth]{images/hdframe2_wireless-v3.png}
    \vspace{-0.5cm}
    \caption{Permuted bundling}
    \label{fig:sim_res_permuted} 
    \end{subfigure}
    \vspace{-0.2cm}
    \caption{Similarity results comparison for different forms of bundling one, three, five, and seven hypervectors.}
    \vspace{-0.4cm}
\end{figure*}

To understand how the error rate could scale with the number of receivers, we re-simulate the entire architecture with a varying number of RX cores and computing the average BER obtained in each case. As shown in Fig.~\ref{fig:arch_scal}, the average BER generally increases with the number of receivers for which we are optimizing the architecture. This is expected since, when accommodating more constellations in our optimal TX phases search, we are imposing more conditions and hindering the joint optimization across all receivers.


\vspace{0.1cm} \noindent
\textbf{Time-Domain Analysis.} Still with three transmitters and a number of receivers growing from 1 to 64, we obtain the delay spread and coherence bandwidth of the received signal for each of the $2^3 = 8$ possible transmission combinations. As described earlier, we take the worst-case among all receivers in each simulation. As Fig. \ref{fig:worst_ds} shows, all symbol combinations (marked as different IDs in the figure) have a similar scaling behavior in terms of delay spread. Values lower than 0.1 ns (coherence bandwidth greater than 10 GHz) are obtained consistently for systems with less than 10 receivers. The performance degrades to values around 0.166 ns ($\sim$ 6 GHz) in larger architectures. Given that BPSK has a spectral efficiency of 1 b/s/Hz, the evaluated system would have a total throughput between $3\times 6 = 18$ Gb/s and $3\times 10 = 30$ Gb/s from the encoders to the similarity search engines.



\begin{table}[!t]
\caption{\centering Accuracy in the few-shot learning experiment.}
\vspace{-0.2cm}
\centering
\label{tab:hdframe1}
\begin{tabular}{|c|c|c|c|c|c|c|c|}
\hline 
\multirow{4}{*}{\shortstack{Baseline\\Bundling}} & & \multicolumn{6}{c|}{\textbf{Number of bundled hypervectors}} \\ \cline{3-8}
 & \textbf{Channel} & 1 & 3 & 5 & 7 & 9 & 11\\
\cline{2-8}
 & Ideal & 1 & 0.966 & 0.902 & 0.803 & 0.704 & 0.543\\
\cline{2-8}
 & Wireless & 1 & 0.966 & 0.9 & 0.801 & 0.699 & 0.537 \\ 
\hline \hline
\multirow{4}{*}{\shortstack{Permuted\\Bundling}} & & \multicolumn{6}{c|}{\textbf{Number of bundled hypervectors}} \\ \cline{3-8}
 & \textbf{Channel} & 1 & 3 & 5 & 7 & 9 & 11\\
\cline{2-8}
 & Ideal & 1 & 1 & 1 & 1 & 0.995 & 0.978 \\
\cline{2-8}
 & Wireless & 1 & 1 & 1 & 1 & 0.994 & 0.963\\ 
\hline
\end{tabular} 
\vspace{-0.4cm}
\end{table}

\subsection{Classification Experiments}
\label{sec:HDCeval}
To start the assessment of the HDC-based classification tasks, we first perform the few-shot learning experiment while increasing the BER gradually. As Fig.~\ref{fig:ber_acc} depicts, the class accuracy remains above 99\% even when we apply bit flips equivalent to a BER of 0.26. This means that the noise robustness provided by the HDC properties relaxes the error link conditions, ensuring a correct behaviour under the worst-case wireless scenarios, as we show next.

\vspace{0.1cm} \noindent
\textbf{Few-shot Learning.} Fig.~\ref{fig:sim_res_baseline} and Fig.~\ref{fig:sim_res_permuted} show the similarity search result for the baseline bundling and permuted bundling cases, respectively, in the few-shot learning experiment. The figures show how a single 512-bit query can accommodate several queries via bundling (blue line), and that the wireless system (green line) is able to correctly classify the same queries despite having some bits flipped. 

Table~\ref{tab:hdframe1} shows the numerical results of the final class accuracy for the executed task, comparing an ideal channel without errors with our wireless channel with a sizable BER. The effect of the imperfect bundling is negligible in terms of accuracy, as predicted by Fig.~\ref{fig:ber_acc}. Moreover, the permuted bundling significantly improves the baseline bundling, confirming that the proposed approach supports the aggregation of a dozen hypervectors over the air and the parallelization of similarity search over tens of IMCs. 




\vspace{0.1cm} \noindent
\textbf{Continual Learning.} Figure \ref{fig:acc_classes3} shows the evolution of the accuracy as the system learns new classes from the initial dictionary of 100 classes, until the entire test set is learnt. Each new set of classes degrades the classification accuracy because possible similarity between classes represented in 512-bit hypervectors. From the figure, it is again clear that (i) the bit flips associated to imperfect OTA bundling have a negligible effect over the classification accuracy, and (ii) permutation provides a consistent improvement over the baseline bundling.


\begin{table}[!t]
\caption{\centering Accuracy in the continual learning experiment.}
\vspace{-0.2cm}
\centering
\label{tab:hdframe1_omniglot}
\begin{tabular}{|c|c|c|c|c|c|c|c|}
\hline 
\multirow{4}{*}{\shortstack{Baseline\\Bundling}} & & \multicolumn{6}{c|}{\textbf{Number of bundled hypervectors}} \\ \cline{3-8}
 & \textbf{Channel} & 1 & 3 & 5 & 7 & 9 & 11\\
\cline{2-8}
% & Ideal & 0.96 & 0.786 & 0.699 & 0.655 & 0.626 & 0.602\\ % 100 classes
 & Ideal & 0.87 & 0.60 & 0.47 & 0.40 & 0.35 & 0.33 \\ % 600 classes
%% & Ideal & 0.865 & 0.599 & 0.471 & 0.396 & 0.352 & 0.327\\ % 600 classes
% & Ideal & 0.91 & 0.673 & 0.571 & 0.494 & 0.452 & 0.418\\
\cline{2-8}
% & Wireless & 0.959 & 0.786 & 0.699 &  0.645 & 0.622  & 0.601\\ % 100 classes
 & Wireless & 0.81 & 0.59 & 0.47 &  0.39 & 0.35  & 0.32\\ % 600 classes
  %%& Wireless & 0.81 & 0.594 & 0.468 &  0.391 & 0.35  & 0.321\\ % 600 classes
% & Wireless & 0.905 & 0.671 & 0.571 &  0.49 & 0.448  & 0.415\\ 
\hline \hline
\multirow{4}{*}{\shortstack{Permuted\\Bundling}} & & \multicolumn{6}{c|}{\textbf{Number of bundled hypervectors}} \\ \cline{3-8}
 & \textbf{Channel} & 1 & 3 & 5 & 7 & 9 & 11\\
\cline{2-8}
% & Ideal & 0.96 & 0.928 & 0.897 & 0.842 & 0.781 & 0.725 \\ % 100 classes
 & Ideal & 0.87 & 0.81 & 0.73 & 0.63 & 0.55 & 0.46 \\ % 600 classes
%%  & Ideal & 0.865 & 0.809 & 0.732 & 0.627 & 0.546 & 0.461 \\ % 600 classes
% & Ideal & 0.91 & 0.862 & 0.802 & 0.724 & 0.644 & 0.575 \\
\cline{2-8}
% & Wireless & 0.959 & 0.93 & 0.89 & 0.839 & 0.781 & 0.715\\ % 100 classes
 & Wireless & 0.81 & 0.80 & 0.73 & 0.63 & 0.51 & 0.46 \\ % 600 classes
  %& Wireless & 0.81 & 0.797 & 0.727 & 0.63 & 0.541 & 0.461 \\ % 600 classes
% & Wireless & 0.905 & 0.861 & 0.798 & 0.722 & 0.644 & 0.564\\ 
\hline
\end{tabular} 
\vspace{-0.4cm}
\end{table}

Table \ref{tab:hdframe1_omniglot} shows the final accuracy for 600 classes and different bundling configurations. The results confirm the conclusions given above in terms of impact of wireless OTA and permuted bundling. It is worth noting that the similarity search over the Omniglot dataset in continual learning does not perform well when bundles exceed a few hypervectors, even with permutations. This suggests that longer hypervectors or an application with a larger dataset would be required to fully benefit from bundling.


% 3 bundled hypervectors:
\begin{figure}[t]
    \centering
    \includegraphics[width=0.85\columnwidth]{images/bundling3_acc_classesv2.pdf}
    \vspace{-0.2cm}
    \caption{Classification accuracy for the continuous learning case with 3 bundled hypervectors.}
    \label{fig:acc_classes3} 
    \vspace{-0.2cm}
\end{figure}


\subsection{Comparison with Wired Alternative}
\label{sec:OVHDeval}
Here, we discuss about the performance of WHYPE in terms of speed and the overheads in terms of area and power. To that end, we calculate the area and energy of the entire architecture except for the encoders, which may widely vary depend on the application space, leading to various types of implementations and hardware realizations \cite{joshi2020accurate,HDC_NatElec20,montagna2018pulp}.

To evaluate the area and energy of the similarity search engines, we model them as IMC cores using the design from \cite{HermesVLSI,Y2022khaddamJSSC} as baseline. In more detail, the unit cell at each cross-point of the crossbar consists of 8 transistors and 4 PCM devices (8T4R). The PCM crossbar is back-end-of-the-line integrated with CMOS peripherals, namely, a Pulse-Width Modulation (PWM) circuit which convert a 8-bit digital input vector to an array of time encoded pulses, and an ADC that digitizes the output current of the crossbar to multi-bit values. The area and energy consumption of these components, which are summarized in Table~\ref{tab:areapower2}, have been scaled to consider the technology and dimensions of our architecture. 

For the sake of comparison, we evaluate the area and power of both WHYPE's interconnect and a hypothetical wired interposer-based alternative. In the former case, we consider wireless transmitters and receivers compatible with the requirements of WHYPE \cite{saxena20172,xu201723,multichannel,melamed202230}. In the latter case, the scheme from Figure~\ref{fig:wireless_arch_diagram} is implemented using off-chip interconnects. The majority operation is performed in a central chiplet containing input/output buffers and a 512-wide $M$-input majority gate. The majority and similarity search chiplets are equipped with routers and circuitry to implement serial chip-to-chip links. We assume that the chiplets form a mesh network, similarly to \cite{simba}. Routers are modeled using DSENT \cite{DSENT} with a four-stage pipeline and minimal buffering. Off-chip links are modeled as single-lane links with the energy calculated according to the UCIe standard \cite{UCIe} and the area adapted from \cite{Zimmer2019}. We assume a 32nm technology node with $V_{DD} = 1$ V and $f_{clk} = 1$ GHz. See Table~\ref{tab:areapower} for a summary.


\vspace{0.1cm} \noindent
\textbf{Overhead Analysis.} 
We evaluate the area overhead of the system and the total energy consumed in the hypervector collection, bundling, distribution, and similarity search process using data from Table~\ref{tab:areapower2} and Table~\ref{tab:areapower}. It is observed that the off-chip links and wireless transceivers are the most power-hungry components in the interconnect, while the majority gate circuitry and the routers are also area consuming, mostly due to the wide datapath of 512 bits required to transport the hypervectors. The IMC cores are very efficient in terms of area compared with the interconnect components, while their energy efficiency is penalized by the PWM and ADC circuitry.

Figure \ref{fig:breakdown} shows a breakdown of the area and energy of the HDC system assuming $M=3$ encoders and $N=8$ similarity search engines (512 classes). As can be seen, the interconnect is the most area consuming sub-system, whereas the similarity search is energy consuming due to the large MVM operations performed in the IMC tiles. WHYPE reduces the interconnect area by 3.2$\times$ with a modest effect in the energy because of the small scale of the system. To evaluate scalability, Figure \ref{fig:overheads} shows the area and energy of the interconnect as a function of the number of similarity search engines $N$, for $M=3$. It is observed how, even conservatively assuming single-lane wired links, WHYPE is superior in both area and energy consumption, with a gap that widens as the system is scaled out. This is because WHYPE eliminates any wired connection between chiplets, while the wired alternative needs to traverse the entire system through a mesh topology. 


\begin{table}[!t]
\centering
\caption{\centering Area and energy breakdown of IMC for a 512$\times$64 matrix-vector multiplication in 32nm technology.}
\vspace{-0.2cm}
\label{tab:areapower2}
\begin{tabular}{|c|c|c|c|} 
\hline
\multirow{2}{*}{\textbf{Component}}& \textbf{Area} & \textbf{Energy} & \multirow{2}{*}{\textbf{Source}} \\ 
                                   & \textbf{[mm\textsuperscript{2}]} &  \textbf{[nJ]} &    \\  \hline
PCM Crossbar &  0.09    &  0.27   &  \multirow{3}{*}{\shortstack{Scaled from\\ \cite{HermesVLSI,Y2022khaddamJSSC}}} \\ \cline{1-3}
PWM Peripherals & 0.20    &  8.79  &  \\ \cline{1-3}
CCO-based ADC &  0.01     &  6.23  &    \\ \hline
 \end{tabular}
 
PCM = Phase-Change Memory; PWM = Pulse-Width Modulation\\
CCO = Current Controlled Oscillator; ADC = Analog-Digital Converter\\
\vspace{-0.3cm}
\end{table}



\begin{table}[!t]
\centering
\caption{\centering Area and energy breakdown of single instances of interconnect components in 32nm technology.}
\vspace{-0.2cm}
\label{tab:areapower}
\begin{tabular}{c|c|c|c|c|}
\cline{2-5} 
 & \multirow{2}{*}{\textbf{Component}} & \textbf{Area} & \textbf{Energy} & \multirow{2}{*}{\textbf{Source}} \\ 
 &                                  & \textbf{[mm\textsuperscript{2}]} &  \textbf{[pJ/bit]}  & \\  \hline
%\multicolumn{1}{|c|}{\multirow{2}{*}{\shortstack{Encoder\\Chiplet}}}      
%    & Encoders & 0.51 mm\textsuperscript{2} & 1.33 pJ/b &   \cite{karunaratne2021energy} \\ \cline{2-5} 
%\multicolumn{1}{|c|}{} 
%                & Buffer & 0.009 mm\textsuperscript{2}    & 2.4 pJ/op &  \cite{CACTI}   \\  \hline

\multicolumn{1}{|c|}{\multirow{2}{*}{\shortstack{Majority\\Chiplet}}}
              & Majority Gate\S & 0.32  &  0.17 &   \cite{choudhary2019generalized} \\ \cline{2-5} 
\multicolumn{1}{|c|}{}  
            & Buffer & 0.009  & 0.005\ddag & \cite{CACTI} \\ \hline
%\multicolumn{1}{|c|}{\multirow{2}{*}{\shortstack{Search\\Chiplet}}}      
%     & Search engine & N/A  & 0.5 pJ/b  &  \cite{Karunaratne2021RobustHM} \\ \cline{2-5} 
%\multicolumn{1}{|c|}{}
%                & Buffer & 8.9$\cdot$10\textsuperscript{-3} mm\textsuperscript{2}    & 2.4 %pJ/op & \cite{CACTI} \\  \hline
\multicolumn{1}{|c|}{\multirow{3}{*}{\shortstack{Wired\\Network}}} 
            & Router  & 0.36 &  0.03   &  \multirow{2}{*}{\cite{DSENT}} \\ \cline{2-4} 
\multicolumn{1}{|c|}{}                                             
            & Link (on-chip) & 4$\cdot$10\textsuperscript{-4}  & 0.06    &     \\ \cline{2-5} 
\multicolumn{1}{|c|}{}                                             
            & Link (off-chip)  & 0.25* &  1 & \cite{Zimmer2019,UCIe}  \\ \hline
\multicolumn{1}{|c|}{\multirow{5}{*}{\shortstack{Wireless\\Interface\dag}}} 
            & SerDes  & 0.04  & 0.54  &  \cite{saxena20172}  \\ \cline{2-5} 
\multicolumn{1}{|c|}{}                                             
            & Data Converter & 0.03  & 0.07  &  \cite{xu201723}   \\ \cline{2-5} 
\multicolumn{1}{|c|}{}                                             
            & Transmitter & 0.12 & 1.5  & \cite{multichannel},  \\ \cline{2-4} 
\multicolumn{1}{|c|}{}                                             
            & Receiver & 0.12  & 1.3   & \cite{melamed202230}  \\ \cline{2-5} 
\multicolumn{1}{|c|}{}                                             
            & Antenna  & 0.08  &  N/A  & \cite{gutierrez2009chip} \\ \hline          
\end{tabular} 

Unless noted, the width of the components is 512 bits.\\
\S Five-input, one-output gate.\,\, \ddag Per operation (read, write).\\
\dag Dimensioned to operate at 10 Gb/s.\\ 
*Per pin, serial link operating at 16 Gb/s.
\vspace{-0.1cm}
\end{table}


\begin{figure}[!t]
    %\centering
    \begin{subfigure}{0.49\columnwidth}
    \includegraphics[width=1\textwidth]{images/areaBreak.pdf}
    %\vspace{-0.5cm}
    \caption{Area}
    \label{fig:areaBreak} 
    \end{subfigure}
    %\hfill
    \begin{subfigure}{0.49\columnwidth}
    %\centering 
    \includegraphics[width=1\textwidth]{images/energyBreak.pdf}
    %\vspace{-0.5cm}
    \caption{Energy}
    \label{fig:powerBreak} 
    \end{subfigure}
    \vspace{-0.3cm}
    \caption{Area and energy consumption in a system with $M=3$ encoders and $N=8$ search engines, comparing the WHYPE approach with the wired baseline. I, M and S stand for interconnect, majority, and search engines.} \label{fig:breakdown}
    \vspace{-0.2cm}
\end{figure}


\begin{figure}[!t]
    %\centering
    \begin{subfigure}{0.48\columnwidth}
    \includegraphics[width=1\textwidth]{images/area.pdf}
    %\vspace{-0.5cm}
    \caption{Area overhead}
    \label{fig:area} 
    \end{subfigure}
    %\hfill
    \begin{subfigure}{0.5\columnwidth}
    %\centering 
    \includegraphics[width=1\textwidth]{images/energy.pdf}
    %\vspace{-0.5cm}
    \caption{Energy overhead}
    \label{fig:power} 
    \end{subfigure}
    \vspace{-0.2cm}
    \caption{Area and energy overheads of the interconnect with $M=3$ encoders and a variable number $N$ of search engines.} \label{fig:overheads}
    \vspace{-0.2cm}
\end{figure}



\vspace{0.1cm} \noindent
\textbf{Bottleneck Analysis.} We assess the performance of WHYPE through an analysis of the latency and throughput of the steps followed since the encoders output the hypervectors until the bundled hypervectors reach the similarity search engines. In the wireless case, the delay analysis is simple: the latency corresponds to the time needed to wirelessly transmit 512 bits. Assuming a 10 Gb/s transmitter, the latency of the majority operation in WHYPE is 51.2 ns independently of the number of encoders or similarity search engines. In contrast, a wired alternative would be bottlenecked by the majority calculation and the communications happening before and after. Latency-wise, the main delay in the wired case is the time required for the hypervectors to travel through the chip-to-chip network. With the conditions here assumed, the delay scales as 2$\sqrt{M+N}$/3 which is the average number of hops to connect two arbitrary chiplets, with each hop taking 36 ns (4 to traverse a router and 32 to transmit 512 bits through a 16 Gb/s serial link). Quickly, the delay becomes much higher than in the wireless case. In comparison, the IMC cores considered here can take between 10 and 128 nanoseconds to realize the similarity calculations, depending on the required accuracy \cite{Y2022khaddamJSSC}. Although majority and similarity operations can be pipelined, a wired NiP would easily bottleneck the system.  

In terms of throughput, WHYPE does not present a bottleneck due to the seamless all-to-all connection between encoders and similarity search engines. At 10 Gb/s of line rate, since all receivers will obtain the bundled hypervector at the same time, the overall throughput is 10$\times$M$\times$N Gb/s. In the hypothetical wired alternative, the bottleneck is the bisection bandwidth of the system. Since the majority chiplet, being in a mesh network, has only four chip-to-chip links, the bisection bandwidth between the encoders and the similarity search engine will be, at most, twice the capacity of the chip-to-chip links. In our assumed scenario, the throughput would be 32 Gb/s. Therefore, WHYPE will be faster even for relatively low values of $M$ and $N$. 

\vspace{0.1cm} \noindent
\textbf{Comparison with 3D Interconnects.} A few works have proposed to implement HDC-based 3D ICs \cite{Li3DVRRAM2016, WuNanotube2018}. One of the reasons could be the reduced link length as compared to the planar NoC or NiP alternatives.  However, the wired nature of 3D ICs and the \emph{AllGather} nature of the communication suggest that the interconnect will continue being a bottleneck. Moreover, 3D ICs can suffer from heat dissipation issues which limit their scaling ability. Due to this, existing efforts only consider up to a couple dozen classes \cite{Li3DVRRAM2016}.





%%%%%%%%%%%%%%%\vspace{2mm}
\section{Conclusion \lv{\& Limitation}}
\label{sec:conclusion}
\vspace{1mm}
\lvv{This work explores the relation of model generalization and domain diversity, aiming to guarantee and further enhance the efficacy of domain augmentation strand. We then propose a framework to enable each diversified domain contribute to generalization by casting DG as 
a convex game between domains.
Heuristic analysis and comprehensive experiments demonstrate our rationality and effectiveness.}
Note that we mainly focus on the mixup-based domain augmentation techniques for clarity, while the extension of DCG to other GAN-based techniques needs to be further explored. Besides, it also remains an open problem to design a more efficient strategy to avoid the decrease in training efficiency caused by meta-learning. 
Nevertheless, we believe our work can inspire the future work of enriching domain diversity 
with improved generalization capability. 


\section{Conclusion} \label{cncl}
In this work, we introduced an OTA on-chip computing concept capable of overcoming the scalability bottleneck present in wired NoC architectures when scaling out IMC-based HDC systems. By using a WNoC communication layer, a number of encoders is able to concurrently broadcast HDC queries towards all the IMC cores within the architecture. Then, a pre-characterization of the propagation environment allows to map the received constellations to the computed composite query, in each core, based on a decision region strategy. Through a proper correspondence between the TX phases, the received constellation and the decision region, we have shown that the opportunistic calculation of the bit-wise majority of the transmitted HDC queries is possible with low error. We demonstrated the concept and shown its scalability up to 11 TXs and 64 RXs, obtaining the BER of the OTA approach and later employing it to evaluate the impact of the WNoC errors in a HDC classification task. Overall, we conclude that the quality of the WNoC links are solid enough to have a negligible impact on the application accuracy, mostly thanks to the great error robustness of HDC. 


%%%%%%%%%%\input{07-Ack}

\section*{Acknowledgment}
Authors gratefully acknowledge funding from the European Union’s Horizon 2020 research and innovation programme under grant agreement No 863337 (WiPLASH), and Horizon Europe research and innovation programme under grant agreement No 101042080 (WINC).





\footnotesize
\bibliographystyle{IEEEtran}
\tiny
%\bibliography{bstctl,ref}

% Generated by IEEEtran.bst, version: 1.14 (2015/08/26)
\begin{thebibliography}{10}
\providecommand{\url}[1]{#1}
\csname url@samestyle\endcsname
\providecommand{\newblock}{\relax}
\providecommand{\bibinfo}[2]{#2}
\providecommand{\BIBentrySTDinterwordspacing}{\spaceskip=0pt\relax}
\providecommand{\BIBentryALTinterwordstretchfactor}{4}
\providecommand{\BIBentryALTinterwordspacing}{\spaceskip=\fontdimen2\font plus
\BIBentryALTinterwordstretchfactor\fontdimen3\font minus
  \fontdimen4\font\relax}
\providecommand{\BIBforeignlanguage}[2]{{%
\expandafter\ifx\csname l@#1\endcsname\relax
\typeout{** WARNING: IEEEtran.bst: No hyphenation pattern has been}%
\typeout{** loaded for the language `#1'. Using the pattern for}%
\typeout{** the default language instead.}%
\else
\language=\csname l@#1\endcsname
\fi
#2}}
\providecommand{\BIBdecl}{\relax}
\BIBdecl

\bibitem{hdcintro}
P.~Kanerva, ``Hyperdimensional computing: An introduction to computing in
  distributed representation with high-dimensional random vectors,''
  \emph{Cognitive Computation}, vol.~1, 06 2009.

\bibitem{HDC_Rev_PI}
D.~Kleyko \emph{et~al.}, ``A survey on hyperdimensional computing aka vector
  symbolic architectures, part {I:} models and data transformations,''
  \emph{ACM Comput. Surv.}, may 2022.

\bibitem{PlateAnalogy2000}
T.~A. Plate, ``{Analogy Retrieval and Processing with Distributed Vector
  Representations},'' \emph{Expert Systems}, vol.~17, no.~1, pp. 29--40, 2000.

\bibitem{KanervaDollar2010}
P.~Kanerva, ``{What We Mean When We Say ''What's the Dollar of Mexico?'':
  Prototypes and Mapping in Concept Space},'' in \emph{{Proceedings of the AAAI
  Fall Symposium '10}}, 2010, pp. 2--6.

\bibitem{NeubertRobotics2019}
P.~Neubert \emph{et~al.}, ``{An Introduction to Hyperdimensional Computing for
  Robotics},'' \emph{{KI - K{\"u}nstliche Intelligenz}}, vol.~33, no.~4, pp.
  319--330, 2019.

\bibitem{VSA_Workflow}
D.~Verma \emph{et~al.}, ``{Towards A Distributed Federated Brain Architecture
  using Cognitive IoT Devices},'' \emph{Proceedings of COGNITIVE}, 2017.

\bibitem{SimpkinScalable2018}
C.~Simpkin \emph{et~al.}, ``{A Scalable Vector Symbolic Architecture Approach
  for Decentralized Workflows},'' in \emph{Proceedings of COLLA}, 2018.

\bibitem{TomsettDemonstrationOrch2019}
R.~Tomsett \emph{et~al.}, ``{Demonstration of Dynamic Distributed Orchestration
  of Node-RED IoT Workflows Using a Vector Symbolic Architecture},'' in
  \emph{Proceedings of IEEE SMARTCOMP '19}, 2019, pp. 464--467.

\bibitem{CollectiveComm}
P.~{Jakimovski} \emph{et~al.}, ``Collective communication for dense sensing
  environments,'' in \emph{Proceedings of the IE'11}, 2011, pp. 157--164.

\bibitem{Dependable_MAC_HD}
D.~Kleyko \emph{et~al.}, ``Dependable MAC layer architecture based on
  holographic data representation using hyper-dimensional binary spatter
  codes,'' in \emph{Multiple Access Communications}, 2012.

\bibitem{Kim2018HDM}
H.-S. Kim, ``{HDM: Hyper-Dimensional Modulation for Robust Low-Power
  Communications},'' in \emph{IEEE International Conference on Communications},
  2018.

\bibitem{Hsu2019Collision}
C.~W. Hsu \emph{et~al.}, ``{Collision-tolerant narrowband communication using
  non-orthogonal modulation and multiple access},'' in \emph{Proceedings of the
  GLOBECOM}, 2019.

\bibitem{Hsu_HDM2}
C.-W. Hsu \emph{et~al.}, ``Non-orthogonal modulation for short packets in
  massive machine type communications,'' in \emph{Proceedings of the GLOBECOM
  '20}, 2020.

\bibitem{Hersche2021}
M.~Hersche \emph{et~al.}, ``Near-channel classifier: symbiotic communication
  and classification in high-dimensional space,'' \emph{Brain Informatics},
  vol.~8, no.~1, p.~16, Aug 2021.

\bibitem{HDC_Rev_PII}
D.~Kleyko \emph{et~al.}, ``A survey on hyperdimensional computing aka vector
  symbolic architectures, part {II:} applications, cognitive models, and
  challenges,'' \emph{ACM Comput. Surv.}, jun 2022.

\bibitem{InMemFSCIL2022}
G.~Karunaratne \emph{et~al.}, ``In-memory realization of in-situ few-shot
  continual learning with a dynamically evolving explicit memory,'' in
  \emph{{IEEE European Solid-state Devices and Circuits Conference (ESSDERC)}},
  2022.

\bibitem{FSCIL2022}
M.~Hersche \emph{et~al.}, ``{Constrained Few-shot Class-incremental
  Learning},'' in \emph{{IEEE/CVF Conference on Computer Vision and Pattern
  Recognition (CVPR)}}, 2022, pp. 1--19.

\bibitem{Karunaratne2021RobustHM}
G.~Karunaratne \emph{et~al.}, ``Robust high-dimensional memory-augmented neural
  networks,'' \emph{Nature Communications}, vol.~12, 2021.

\bibitem{MoinWearable2021}
A.~Moin \emph{et~al.}, ``{A Wearable Biosensing System with In-sensor Adaptive
  Machine Learning for Hand Gesture Recognition},'' \emph{Nature Electronics},
  vol.~4, no.~1, pp. 54--63, 2021.

\bibitem{RahimiBiosignal2019}
A.~Rahimi \emph{et~al.}, ``{Efficient Biosignal Processing Using
  Hyperdimensional Computing: Network Templates for Combined Learning and
  Classification of ExG Signals},'' \emph{Proceedings of the IEEE}, vol. 107,
  no.~1, pp. 123--143, 2019.

\bibitem{oneshot}
A.~Burrello \emph{et~al.}, ``One-shot learning for iEEG seizure detection using
  end-to-end binary operations: Local binary patterns with hyperdimensional
  computing,'' in \emph{Proceedings of the IEEE BioCAS}, 2018, pp. 1--4.

\bibitem{HDC2016}
A.~Rahimi \emph{et~al.}, ``{A Robust and Energy Efficient Classifier Using
  Brain-Inspired Hyperdimensional Computing},'' in \emph{Proceedings of the
  IEEE/ACM ISLPED}, 2016.

\bibitem{Li3DVRRAM2016}
H.~Li \emph{et~al.}, ``{Hyperdimensional Computing with 3D VRRAM In-Memory
  Kernels: Device-Architecture Co-Design for Energy-Efficient, Error-Resilient
  Language Recognition},'' in \emph{{Proceedings of the IEEE IEDM}}, 2016.

\bibitem{WuNanotube2018}
T.~Wu \emph{et~al.}, ``{Brain-Inspired Computing Exploiting Carbon Nanotube
  FETs and Resistive RAM: Hyperdimensional Computing Case Study},'' in
  \emph{{Proceedings of the IEEE ISSCC}}, 2018.

\bibitem{HDC_NatElec20}
G.~Karunaratne \emph{et~al.}, ``In-memory hyperdimensional computing,''
  \emph{Nature Electronics}, vol.~3, pp. 327--337, 2020.

\bibitem{memorydevices}
A.~Sebastian \emph{et~al.}, ``Memory devices and applications for in-memory
  computing,'' \emph{Nature Nanotechnology}, vol.~15, 03 2020.

\bibitem{ScaleUpXbar}
S.~Yu \emph{et~al.}, ``Scaling-up resistive synaptic arrays for neuro-inspired
  architecture: Challenges and prospect,'' in \emph{Proceedings of the IEEE
  IEDM}, 2015.

\bibitem{Laha2015}
S.~Laha \emph{et~al.}, ``{A New Frontier in Ultralow Power Wireless Links:
  Network-on-Chip and Chip-to-Chip Interconnects},'' \emph{IEEE Transactions on
  Computer-Aided Design of Integrated Circuits and Systems}, vol.~34, no.~2,
  2015.

\bibitem{ahmed2020asymmetric}
M.~Ahmed \emph{et~al.}, ``An asymmetric, one-to-many traffic-aware mm-wave
  wireless interconnection architecture for multichip systems,'' \emph{IEEE T.
  on Emerging Topics in Computing}, 2020.

\bibitem{jog2021one}
S.~Jog \emph{et~al.}, ``One protocol to rule them all: Wireless network-on-chip
  using deep reinforcement learning,'' in \emph{Proceedings of the NSDI '21},
  2021, pp. 973--989.

\bibitem{micro2022}
A.~Ganguly \emph{et~al.}, ``{Interconnects for DNA, Quantum, In-Memory and
  Optical Computing: Insights from a Panel Discussion},'' \emph{IEEE Micro},
  2022.

\bibitem{wiplash}
S.~Abadal \emph{et~al.}, ``Graphene-based wireless agile interconnects for
  massive heterogeneous multi-chip processors,'' \emph{IEEE Wireless
  Communications Magazine}, 2022.

\bibitem{guirado2022wireless}
R.~Guirado \emph{et~al.}, ``Wireless on-chip communications for scalable
  in-memory hyperdimensional computing,'' in \emph{Proceedings of the
  IJCNN/WCCI 2022}, 2022.

\bibitem{hbm}
J.~Kim \emph{et~al.}, ``{HBM}: Memory solution for bandwidth-hungry
  processors,'' in \emph{2014 IEEE Hot Chips 26 Symposium (HCS)}, 2014.

\bibitem{3d}
M.~Shulaker \emph{et~al.}, ``Three-dimensional integration of nanotechnologies
  for computing and data storage on a single chip,'' \emph{Nature}, vol. 547,
  pp. 74--78, 2017.

\bibitem{Y2022khaddamJSSC}
R.~Khaddam-Aljameh \emph{et~al.}, ``{HERMES}-core--a 1.59-{TOPS}/mm$^2$ {PCM}
  on 14-nm {CMOS} in-memory compute core using 300-ps/{LSB} linearized
  {CCO}-based {ADC}s,'' \emph{IEEE Journal of Solid-State Circuits}, 2022.

\bibitem{adaptive}
S.~Abadal \emph{et~al.}, ``Opportunistic beamforming in wireless
  network-on-chip,'' in \emph{Proceedings of the IEEE ISCAS}, 2019.

\bibitem{imani2022smart}
M.~F. Imani \emph{et~al.}, ``Metasurface-programmable wireless
  network-on-chip,'' \emph{Advanced Science}, 2022.

\bibitem{multichannel}
X.~Yu \emph{et~al.}, ``Architecture and design of multichannel millimeter-wave
  wireless noc,'' \emph{IEEE Design Test}, vol.~31, no.~6, pp. 19--28, 2014.

\bibitem{barrier}
H.~M. Cheema \emph{et~al.}, ``The last barrier: on-chip antennas,'' \emph{IEEE
  Microwave Magazine}, vol.~14, no.~1, pp. 79--91, 2013.

\bibitem{timoneda}
X.~Timoneda \emph{et~al.}, ``Engineer the channel and adapt to it: Enabling
  wireless intra-chip communication,'' \emph{IEEE Transactions on
  Communications}, vol.~68, no.~5, pp. 3247--3258, 2020.

\bibitem{wienna}
R.~Guirado \emph{et~al.}, ``{Dataflow-Architecture Co-Design for 2.5D DNN
  Accelerators using Wireless Network-on-Package},'' in \emph{Proceedings of
  the ASP-DAC '21}, 2021, pp. 806--812.

\bibitem{scaleupIMC}
S.~Yu \emph{et~al.}, ``Scaling-up resistive synaptic arrays for neuro-inspired
  architecture: Challenges and prospect,'' in \emph{2015 IEEE International
  Electron Devices Meeting (IEDM)}, 2015, pp. 17.3.1--17.3.4.

\bibitem{ChangEmotion2019}
E.~Chang \emph{et~al.}, ``{Hyperdimensional Computing-based Multimodality
  Emotion Recognition with Physiological Signals},'' in \emph{Proceedings of
  the IEEE AICAS}, 2019.

\bibitem{MitrokhinCNN2020}
A.~Mitrokhin \emph{et~al.}, ``{Symbolic Representation and Learning with
  Hyperdimensional Computing},'' \emph{Frontiers in Robotics and AI}, pp.
  1--11, 2020.

\bibitem{ExteremeLearning_NIPS21}
A.~Ganesan \emph{et~al.}, ``{Learning with Holographic Reduced
  Representations},'' in \emph{Advances in Neural Information Processing
  Systems}, 2021.

\bibitem{choudhary2019generalized}
J.~Choudhary \emph{et~al.}, ``Generalized majority voter design method for
  n-modular redundant systems used in mission-and safety-critical
  applications,'' \emph{Computers}, vol.~8, no.~1, p.~10, 2019.

\bibitem{simba}
Y.~S. Shao \emph{et~al.}, ``Simba: Scaling deep-learning inference with
  multi-chip-module-based architecture,'' in \emph{Proceedings of the
  MICRO-52}.\hskip 1em plus 0.5em minus 0.4em\relax ACM, 2019, p. 14–27.

\bibitem{altun2022magic}
U.~Altun \emph{et~al.}, ``The magic of superposition: A survey on simultaneous
  transmission based wireless systems,'' \emph{IEEE Access}, vol.~10, pp.
  79\,760--79\,794, 2022.

\bibitem{Matolak2013CHANNEL}
D.~Matolak \emph{et~al.}, ``{Channel modeling for wireless
  networks-on-chips},'' \emph{IEEE Communications Magazine}, vol.~51, no.~6,
  pp. 180--186, 2013.

\bibitem{Pano2020a}
V.~Pano \emph{et~al.}, ``{TSV antennas for multi-band wireless
  communication},'' \emph{IEEE Journal on Emerging and Selected Topics in
  Circuits and Systems}, vol.~10, no.~1, pp. 100--113, 2020.

\bibitem{orthonoc}
S.~Abadal \emph{et~al.}, ``Orthonoc: A broadcast-oriented dual-plane wireless
  network-on-chip architecture,'' \emph{IEEE Transactions on Parallel and
  Distributed Systems}, vol.~29, no.~3, pp. 628--641, 2018.

\bibitem{cst}
``{CST Microwave Studio},'' [Online]. Available: https://www.cst.com. Accessed
  28-September-2021.

\bibitem{HermesVLSI}
R.~Khaddam-Aljameh \emph{et~al.}, ``{HERMES Core – A 14nm CMOS and PCM-based
  In-Memory Compute Core using an array of 300ps/LSB Linearized CCO-based ADCs
  and local digital processing},'' in \emph{2021 Symposium on VLSI Technology},
  2021.

\bibitem{Omniglot_dataset}
B.~M. Lake \emph{et~al.}, ``Human-level concept learning through probabilistic
  program induction,'' \emph{Science}, vol. 350, no. 6266, pp. 1332--1338,
  2015.

\bibitem{joshi2020accurate}
V.~Joshi \emph{et~al.}, ``Accurate deep neural network inference using
  computational phase-change memory,'' \emph{Nature communications}, vol.~11,
  no.~1, pp. 1--13, 2020.

\bibitem{montagna2018pulp}
F.~Montagna \emph{et~al.}, ``PULP-HD: Accelerating brain-inspired
  high-dimensional computing on a parallel ultra-low power platform,'' in
  \emph{2018 55th ACM/ESDA/IEEE Design Automation Conference (DAC)}.\hskip 1em
  plus 0.5em minus 0.4em\relax IEEE, 2018, pp. 1--6.

\bibitem{saxena20172}
S.~Saxena \emph{et~al.}, ``{A 2.8 mW/Gb/s, 14 Gb/s serial link transceiver},''
  \emph{IEEE Journal of Solid-State Circuits}, vol.~52, no.~5, pp. 1399--1411,
  2017.

\bibitem{xu201723}
B.~Xu \emph{et~al.}, ``{A 23-mW 24-GS/s 6-bit voltage-time hybrid
  time-interleaved ADC in 28-nm CMOS},'' \emph{IEEE Journal of Solid-State
  Circuits}, vol.~52, no.~4, pp. 1091--1100, 2017.

\bibitem{melamed202230}
I.~Melamed \emph{et~al.}, ``A 30 GHz 4.2 mW 105 fsec jitter sub-sampling PLL
  with 1° phase shift resolution in 65 nm cmos,'' in \emph{2022 IEEE 22nd
  Topical Meeting on Silicon Monolithic Integrated Circuits in RF Systems
  (SiRF)}.\hskip 1em plus 0.5em minus 0.4em\relax IEEE, 2022, pp. 45--48.

\bibitem{DSENT}
C.~Sun \emph{et~al.}, ``DSENT-a tool connecting emerging photonics with
  electronics for opto-electronic networks-on-chip modeling,'' in \emph{2012
  IEEE/ACM Sixth International Symposium on Networks-on-Chip}.\hskip 1em plus
  0.5em minus 0.4em\relax IEEE, 2012, pp. 201--210.

\bibitem{UCIe}
D.~D. Sharma \emph{et~al.}, ``Universal chiplet interconnect express
  (UCIe){\textregistered}: An open industry standard for innovations with
  chiplets at package level,'' \emph{IEEE Transactions on Components, Packaging
  and Manufacturing Technology}, 2022.

\bibitem{Zimmer2019}
B.~Zimmer \emph{et~al.}, ``{A 0.11 pJ/Op, 0.32-128 TOPS, Scalable
  Multi-Chip-Module-based Deep Neural Network Accelerator with Ground-Reference
  Signaling in 16nm},'' \emph{IEEE Symposium on VLSI Circuits, Digest of
  Technical Papers}, vol. 2019-June, pp. C300--C301, 2019.

\bibitem{CACTI}
R.~Balasubramonian \emph{et~al.}, ``Cacti 7: New tools for interconnect
  exploration in innovative off-chip memories,'' \emph{ACM Transactions on
  Architecture and Code Optimization (TACO)}, vol.~14, no.~2, 2017.

\bibitem{gutierrez2009chip}
F.~Gutierrez \emph{et~al.}, ``On-chip integrated antenna structures in CMOS for
  60 GHz WPAN systems,'' \emph{IEEE Journal on Selected Areas in
  Communications}, vol.~27, no.~8, pp. 1367--1378, 2009.

\end{thebibliography}



% \begin{IEEEbiography}[{\includegraphics[width=1in,height=1.25in,clip,keepaspectratio]{biopic/robert_guirado.jpg}}]{Robert Guirado} was born in Badalona, Spain. He received his B.Sc. and M.Sc. in Telecommunication Systems Engineering from Universitat Polit\`ecnica de Catalunya, Barcelona, Spain, in 2019 and 2021, respectively. Between February 2019 and August 2019, he was a visiting research student at the Georgia Institute of Technology. From September 2019 to January 2021, he worked as research assistant at the NaNoNetworking Center in Catalonia. He conducted his Master Thesis at IBM Research Zurich from February 2021 to July 2021. He is currently pursuing a Ph.D. degree on reconfigurable intelligent surfaces based on liquid crystals at the Applied Electromagnetism Group (GEA) of Universidad Polit\'ecnica de Madrid (UPM), Madrid, Spain. 
% \end{IEEEbiography}

% \begin{IEEEbiography}[{\includegraphics[width=1in,height=1.25in,clip,keepaspectratio]{biopic/abbas_rahimi.jpg}}]{Abbas Rahimi} received the B.S. degree in computer engineering from the University of Tehran, Tehran, Iran, in 2010, and the M.S. and Ph.D. degrees in computer science and engineering from the University of California at San Diego, La Jolla, CA, USA, in 2013 and 2015, respectively. Since then, until 2020, he held postdoctoral research positions at the University of California at Berkeley, Berkeley, CA, USA, and ETH Zürich, Zürich, Switzerland. In 2020, he joined the IBM Research–Zürich Laboratory, Rüschlikon, Switzerland, as a Research Staff Member. Dr. Rahimi received the 2015 Outstanding Dissertation Award in the area of “new directions in embedded system design and embedded software” from the European Design and Automation Association and the ETH Zürich Postdoctoral Fellowship in 2017. He was a co-recipient of the Best Paper Nominations at the ACM/IEEE Design Automation Conference (DAC) in 2013 and the Design, Automation Test in Europe Conference Exhibition (DATE) in 2019, the Best Paper Awards at the EAI International Conference on Bioinspired Information and Communications Technologies (BICT) in 2017 and the IEEE Biomedical Circuits and Systems Conference (BioCAS) in 2018, and the IBM’s Pat Goldberg Memorial Best Paper Award in 2020.
% \end{IEEEbiography}

% \begin{IEEEbiography}[{\includegraphics[width=1in,height=1.25in,clip,keepaspectratio]{biopic/geethan_karunaratne.jpeg}}]{Geethan Karunaratne} received his B.Sc. degree in Electronic and Telecommunication Engineering from University of Moratuwa, Sri Lanka in 2014, and the M.Sc. degree in Information Technology and Electrical Engineering from ETH Zurich in 2018. He joined IBM Research – Zurich in 2018, where he is currently a member of In-memory computing group. Geethan is working towards his PhD degree at ETH Zurich. His main research interests are in-memory computing and brain-inspired computing.
% \end{IEEEbiography}

% \vspace{-1cm}

% \begin{IEEEbiography}[{\includegraphics[width=1in,height=1.25in,clip,keepaspectratio]{biopic/eduard_alarcon.jpg}}]{Eduard Alarc\'on} is an associate professor at the Universitat Polit\`ecnica de Catalunya, where he obtained his PhD in electrical engineering in 2000. He has coauthored more than 400 scientific publications, 8 book chapters and 12 patents. He was elected IEEE CAS society distinguished lecturer, member of the IEEE CAS Board of Governors (2010-2013), Associate Editor for IEEE TCAS-I, TCAS-II, JOLPE, and Editor-in-Chief of JETCAS. His research interests include nanocommunications and wireless energy transfer.
% \end{IEEEbiography}

% \vspace{-1cm}

% \begin{IEEEbiography}[{\includegraphics[width=1in,height=1.25in,clip,keepaspectratio]{biopic/abu_sebastian.jpg}}]{Abu Sebastian (F’ 23)} is a Distinguished Research Staff Member at IBM Research – Zurich. He received a B. E. (Hons.) degree in Electrical and Electronics Engineering from BITS Pilani, India, in 1998 and M.S. and Ph.D. degrees in Electrical Engineering (minor in Mathematics) from Iowa State University in 1999 and 2004, respectively. He was a contributor to several key projects in the space of storage and memory technologies and currently manages the research effort on in-memory computing at IBM Research Zurich. He is the author/co-author of over 200 publications in peer-reviewed journals/conference proceedings and holds over 80 US patents. He has also co-edited a book titled “Memristive devices for brain-inspired computing” by Elsevier in 2020. In 2015 he was awarded the European Research Council (ERC) consolidator grant and in 2020, he was awarded an ERC Proof-of-concept grant. He was elected an IBM Master Inventor in 2016. In 2019 he received the Ovshinsky Lectureship Award for his contributions to "Phase-change materials for cognitive computing".  He has served on the technical program committees of several conferences including IEDM, AICAS and EPCOS. He is a distinguished lecturer and fellow of IEEE.
% \end{IEEEbiography}

% \vspace{-1cm}

% \begin{IEEEbiography}[{\includegraphics[width=1in,height=1.25in,clip,keepaspectratio]{biopic/sergi_abadal.png}}]{Sergi Abadal} received the PhD in Computer Architecture from the Department of Computer Architecture, Universitat Polit\`ecnica de Catalunya (UPC), Barcelona, Spain, in July 2016. Previously, he had obtained the M.Sc. and B.Sc. in Telecommunication Engineering from the same institution, in 2011 and 2010, respectively. Also, he has held several visiting researcher positions at Georgia Tech, University of Illinois at Urbana-Champaign, and at the Foundation for Research \& Technology (FORTH) in Greece. Currently, he works as a distinguished researcher at the N3Cat group, at the Computer Architecture Department of the UPC. He is the recipient of a Starting Grant, called WINC, from the European Research Council (ERC) and also the project coordinator of WIPLASH H2020 FET-OPEN project, while in the past he participated in several other national and EU projects. He is also Area Editor of the Nano Communication Networks (Elsevier) Journal, where he was selected Editor of the Year 2019. From 2020, he acts as one of the ambassadors of the European Innovation Council (EIC) through its program of National Champions. His current research interests are in the areas of chip-scale wireless communications, including channel modeling and protocol design, and the application of these techniques for the creation of novel architectures for next-generation classical and quantum computing systems. He is member of the IEEE, ACM, and HiPEAC.
% \end{IEEEbiography}



\end{document}


