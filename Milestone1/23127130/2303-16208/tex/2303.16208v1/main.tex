\documentclass[11pt]{article}
\usepackage{amsmath,amssymb,amsfonts,amsthm,epsfig}
\usepackage[usenames,dvipsnames]{xcolor}
\usepackage{bm,xspace}
\usepackage{tcolorbox}
\usepackage{cancel}
\usepackage{fullpage}
\usepackage{liyang}
\usepackage{framed}
\usepackage{verbatim}
\usepackage{enumitem}
\usepackage{array}
\usepackage{multirow}
\usepackage{afterpage}
\usepackage{mathrsfs}
\usepackage{pifont} 
\usepackage{chngpage}
\usepackage[normalem]{ulem}
\usepackage{boxedminipage}
\usepackage{caption}
\usepackage{forest}

\usepackage{algorithm}
\usepackage{algorithmicx}
\usepackage{algpseudocode}
\renewcommand{\algorithmicrequire}{\textbf{Given:}}

\usepackage{pgfplots}
\pgfplotsset{width=8cm,compat=newest}

% \newcommand{\lnote}[1]{\footnote{{\bf \color{blue}Li-Yang}: {#1}}}
% \newcommand{\jnote}[1]{\footnote{{\bf \color{red}Jane}: {#1}}}
% \newcommand{\anote}[1]{\footnote{{\bf \color{green}Ali}: {#1}}}
% \newcommand{\gnote}[1]{\footnote{{\bf \color{violet}Guy}: {#1}}}


%%%%%%%%%%%%%%%%%%%%
%
%  COLORS 
%
%%%%%%%%%%%%%%%%%%%%

\def\colorful{0}


\ifnum\colorful=1
\newcommand{\violet}[1]{{\color{violet}{#1}}}
\newcommand{\orange}[1]{{\color{orange}{#1}}}
\newcommand{\blue}[1]{{{\color{blue}#1}}}
\newcommand{\red}[1]{{\color{red} {#1}}}
\newcommand{\green}[1]{{\color{green} {#1}}}
\newcommand{\pink}[1]{{\color{pink}{#1}}}
\newcommand{\gray}[1]{{\color{gray}{#1}}}

\fi
\ifnum\colorful=0
\newcommand{\violet}[1]{{{#1}}}
\newcommand{\orange}[1]{{{#1}}}
\newcommand{\blue}[1]{{{#1}}}
\newcommand{\red}[1]{{{#1}}}
\newcommand{\green}[1]{{{#1}}}
\newcommand{\gray}[1]{{{#1}}}

\fi

%%%%%%%%%%%%%%%%%%%%
%
%  Paper-specific macros
%
%%%%%%%%%%%%%%%%%%%%

\newcommand{\bias}{\mathrm{bias}}
\newcommand{\TV}{\dist_{\mathrm{TV}}}

\newcommand{\BayesOpt}{\textsc{BayesOpt}}
\newcommand{\error}{\mathrm{error}}
%\renewcommand{\opt}{\mathsf{opt}}
\newcommand{\round}{\mathrm{round}}
\newcommand{\Sens}{\mathrm{Sens}}
\newcommand{\est}{\mathrm{est}}
\newcommand{\score}{\mathrm{Score}}
\newcommand{\Prune}{\mathrm{Prune}}
\newcommand{\BuildDT}{\textsc{BuildDT}}
\newcommand{\Search}{\textsc{Search}}
\newcommand{\UnbiasedEstimator}{\textsc{UnbiasedEstimator}}
\newcommand{\Reconstructor}{\textsc{Reconstructor}}


\newcommand{\ArbDist}{\mathcal{E}}
\newcommand{\InfEst}{\textsc{InfEst}}
\newcommand{\Lift}{\textsc{LiftLearner}}

% Macros for Shapley stuff
\def\sv{\mathrm{SV}}
\def\se{\mathrm{SE}}
\def\shap{\mathrm{Shap}}

% Misc macros
\newcommand{\paren}[1]{\left({#1}\right)}
\DeclareMathOperator*{\argmax}{arg\,max}

% Establishing creferences for enum items in a proposition environment
\newlist{enumprop}{enumerate}{1} % set up a dedicated enumeration environment
\setlist[enumprop]{label=\arabic*.,ref=\theproposition.\arabic*}
\crefalias{enumpropi}{proposition}


\makeatletter
\newtheorem*{rep@theorem}{\rep@title}
\newcommand{\newreptheorem}[2]{
\newenvironment{rep#1}[1]{
 \def\rep@title{#2 \ref{##1}}
 \begin{rep@theorem}\itshape}
 {\end{rep@theorem}}}
\makeatother
%\theoremstyle{plain}


\newreptheorem{theorem}{Theorem}




\newcommand{\pparagraph}[1]{\bigskip \noindent {\bf {#1}}}

\newcommand{\myfig}[4]{\begin{figure}[H] \centering \includegraphics[width=#1\textwidth]{#2} \caption{#3} \label{#4} \end{figure}}




\begin{document}
\title{Lifting uniform learners via distributional decomposition  %\vspace{15pt} 
}

\author{}
\author{Guy Blanc \vspace{8pt} \\ \hspace{-5pt}{\sl Stanford}
\and \hspace{0pt} Jane Lange \vspace{8pt} \\ \hspace{-4pt}  {\sl MIT}
\and Ali Malik \vspace{8pt}\\ \hspace{-8pt} {\sl Stanford}
\and Li-Yang Tan \vspace{8pt} \\ \hspace{-8pt} {\sl Stanford}}  

\date{\vspace{15pt}\small{\today}}


\maketitle


% \begin{abstract} 
% We prove an algorithmic decomposition lemma for monotone high-dimensional distributions over $\zo^n$, showing how they can be efficiently broken down into a small number of close-to-uniform components, where ``efficiently" and ``small" scale with their inherent complexity.  A distribution $\mathcal{D}$ has {\sl decision tree complexity $d$} if its pmf can be computed by depth-$d$ decision tree; this 


% Given samples from a monotone distribution $\mathcal{D}$ whose pmf is computed by a depth-$d$ decision tree---which gives a decom the mixture of $2^d$ many uniform distrubtions over subcubes), our algorithm runs in time $\poly(n)\cdot d^{O(d)}$ and 
% \end{abstract} 

\begin{abstract}
    \begin{abstract}
The current study investigated possible human-robot kinaesthetic interaction using a variational recurrent neural network model, called PV-RNN, which is based on the free energy principle.
Our prior robotic studies using PV-RNN showed that the nature of interactions between top-down expectation and bottom-up inference is strongly affected by a parameter, called the meta-prior, which regulates the complexity term in free energy.
% The current study examines how the behaviours of robots alter by changing the meta-prior $w$ in human-robot kinaesthetic interaction.
The current study examines how changing the meta-prior $w$ in the interaction phase affects the counter force generated when an experimenter attempts to induce movement pattern transitions familiar to the robot through its prior training.
The study also compares the counter force generated when trained transitions are induced by a human experimenter and when untrained transitions are induced.
Our experimental results indicated that (1) the human experimenter needs more/less force to induce trained transitions when $w$ is set with larger/smaller values, (2) the human experimenter needs more force to act on the robot when he attempts to induce untrained as opposed to trained movement pattern transitions.
Our analysis of time development of essential variables and values in PV-RNN during bodily interaction clarified the mechanism by which gaps in actional intentions between the human experimenter and the robot can be manifested as reaction forces between them.


%% Hiroki writing 2022-11-4
%Current study investigates the dynamics of the latent states during human-robot kinaesthetic interaction using PV-RNN.
%We have achieved to observe and analyse the internal state of an RNN model based on the free energy principle, during real-time human-robot interaction.
%Essential characteristics observed in the previous study of this variational recurrent neural network model, PV-RNN, is that by changing a meta prior $w$, the balance between the top-down intention and the bottom-up perceptual reality changes.
%In the current study, we examined how changing the weighting parameter $w$ between accuracy and complexity in free energy principle affects the humanoid robot's behaviour through human-robot interaction. We have conducted some human-robot kinaesthetic interaction experiments with various $w$ and quantitatively analysed the latent variable and the force applied to the humanoid robot. We have observed that the force required to change the robot's intention has increased, both when the top-down intention was strengthened by changing the $w$ and when corresponding switch of its primitive was against the experience of the RNN during its training. The study confirms through quantitative analysis that by increasing or decreasing the $w$ in PV-RNN, humanoid robot leads or follows the human counterpart during the human-robot kinaesthetic interaction.

\begin{comment}
Comment from Jun #2
・最後にQualitativeな結果(インパクト)が欲しい
・Current study investigates the problem on~と書き出すのが一般的
・最初の一文と最後の一文を対応させる
・最後の一文はもう少しAbstractかつ包括的に
\end{comment}

\begin{comment}
Comment from Jun #1
We investigated how the kinaesthetic human-robot interaction can affect the internal state of a model based on the free energy principle. 
=> how the internal state is affected is not the most important point in this study. This part should be rewritten.

The key function of this variational recurrent neural network model, PV-RNN, is that by changing a meta prior $w$, it takes a balance between the "complexity” term and the ”accuracy” term which corresponds to a top-down intention and a bottom-up perceptual reality in the free energy principle, respectively. 
=> This is not key function of PV-RNN. It is an essential characteristics observed in the previous study. The grammar after $w$ is something strange. Rewrite these.

This research has conducted a human-robot interaction experiment with a robotic agent in a kinaesthetic sense.
=> The sentence is not good. "in a kinaesthetic sense" is grammatically wrong.
MODIFIED => "In the current study human-robot interaction experiments using the kinaesthetic sense were conducted."

We investigated that when human forces the agent to switch primitives from one to another, larger force was required both when the human intention is conflictive against the top-down the intention of the agent and when the agent has a stronger top-down intention by modifying the $w$.
=> You should write the essential results of the experiments rather than what we investigated and also how these results could contribute to the studies on human-robot interaction.
\end{comment}

\end{abstract}    
\end{abstract}

% \footnote{\red{Keywords: PAC Learning, Semi-supervised learning, decision tree decomposition}}

\thispagestyle{empty}
\newpage 
\setcounter{page}{1}


\newcommand{\depth}{\mathrm{depth}}



\section{Introduction}
\label{sec:intro}
\begin{figure}[t]
\begin{center}
    \includegraphics[width=1\linewidth]{figures/teaser.pdf}
\end{center}
\vspace{-0.1in}
\caption{\textbf{{\em Foggy} vs {\em Clear} NeRF.} Our \ournerf gets rid of reconstruction errors manifested as foggy ``floaters" in the density volume without additional input or significant computational overhead. 
%
Below are density profiles along a given ray before and after our geometry correction procedure, where we discard density peaks corresponding to floaters.
}
\label{fig:teaser}
\vspace{-0.2in}
\end{figure}



%The emergence of 
Neural Radiance Fields (NeRFs)~\cite{mildenhall2020nerf}  %and its variants 
have made revolutionary contributions in %photo-realistic 
novel view synthesis~\cite{barron2021mip,barron2022mip}, 
autonomous driving~\cite{rematas2022urban,tancik2022block}, digital human~\cite{hong2022headnerf,zhao2022humannerf}, and 3D content generation~\cite{eg3d,poole2022dreamfusion,lin2022magic3d}.
%by leveraging a multi-layer perceptron (MLP) to implicitly model the mapping from input 5D coordinates (i.e., 3D coordinates $\mathbf{x} = (x,y,z)$ and 2D viewing directions $\mathbf{d}=(\theta,\phi)$) to volume density $\sigma$ and view-dependent emitted radiance color $\mathbf{c} = (r,g,b)$. 
%
%They then use traditional volume rendering mechanisms on the obtained continuous 5D function (i.e., MLP) to generate novel views. 
To date, unfortunately, most NeRF-based methods encounter challenges when tackling large-scale cluttered scenes (e.g., Fig.~\ref{fig:teaser}):
\begin{enumerate}[leftmargin=0.16in, topsep=2pt,itemsep=-1ex,partopsep=1ex,parsep=1ex]
\item Input observations used for NeRF are often too sparse  compared to forward-facing or synthetic looking-inward scenes;
%\item Recovering fine-grained objects within a large volume is challenging for NeRF; %in capturing details accurately.
\item View-dependent visual effects give rise to ambiguity, resulting in a ``foggy" density field as shown in Fig.~\ref{fig:teaser}. 
%
Such artifacts are particularly pronounced in indoor scenes strewn with view-dependent appearances, such as specular highlights, glossy surface reflections from man-made objects. 
\end{enumerate}

Despite attempts to enhance NeRF's rendering quality given suboptimal input, such as using 3D conical frustums~\cite{barron2021mip,barron2022mip}, physically-grounded augmentations~\cite{chen2022aug}, and misalignment correction~\cite{jiang2022alignerf},  these challenges have yet to be fully resolved.
%
Depth supervision~\cite{deng2022depth, wei2021nerfingmvs} or proxy geometry~\cite{xu2021scalable,wu2022scalable} images can help alleviate the challenges in handling large-scale with sparse input, at the expense of %but they come at the cost of requiring 
expensive pre-processing or additional input.
%
Another line of work~\cite{wang2021neus, oechsle2021unisurf, wang2022neuris} achieves better reconstruction of surface geometry by using signed distances instead of volume density as scene representation. However, they sacrifice the ability to synthesize photo-realistic novel views.

%We observe that NeRF has been suffering from foggy ``floater" artifacts in large-scale cluttered scenes.
%
%Such artifacts are particularly pronounced in indoor scenes strewn with view-dependent appearances from man-made objects. 
%
To address the above issues, we propose an extension to NeRF, dubbed as {\bf \ournerf}, which enforces effective {\em appearance} and {\em geometry} constraints conducive to accurate colors and 3D densities estimation. We believe \ournerf can contribute beyond novel view synthesis, such as NeRF object detection~\cite{hu2022nerf}, NeRF object segmentation~\cite{zhi2021place, liu2022unsupervised, fan2022nerf,ren2022neural}, and NeRF registration~\cite{goli2022nerf2nerf}, where the rooms for improvement are substantial if more accurate color and density estimation are available.

Correspondingly, there are two steps in \ournerf. First, for appearance correction, the view-independent and view-dependent color components are predicted from the underlying 3D scene, which is combined to produce the final color estimation (Fig.~\ref{fig:toaster}).
%
The view-independent component (diffuse color and shading) captures the overall scene color, while the view-dependent component (highlights or reflections) captures color variations due to changes in viewing angle.
%
\ournerf then discards these view-dependent appearances in the training views to prevent them from interfering with the density estimation.
%
Second, a simple and effective geometry correction procedure will be performed to further eliminate the foggy ``floaters" or density errors. This geometry correction procedure is based on an assumption in line with traditional ray tracing in computer graphics.
\begin{comment}
% xh: basically copying method
On the other hand, ClearNeRF performs a geometric correction procedure performed on each traced ray during inference to refine the density estimation and better tackle the floater artifacts. 
%
The geometry correction procedure assumes that there should only be one salient peak along each traced ray during NeRF inference. 
Only the salient peak closest to the ray origin (the camera center) corresponds to  true geometry while the others will be manifested as foggy floaters hovering in the density volume. 
%
This assumption is in line with traditional ray tracing in computer graphics where in the absence of noise, only one intersection per ray should be returned to indicate the closest ray-object intersection.
%
\end{comment}
%%%%%%%%%%%
%As shown in Fig.~\ref{fig:teaser}, when reconstructing an indoor scene with sparse input and highly view-dependent objects, NeRF produces severe floating artifacts due to its attempt to explain view-dependent appearances.
%
Experiments verify that our proposed \ournerf can effectively get rid of floater artifacts without additional input.% or significant computational overhead. 


In summary, our contributions include the following:
\begin{itemize}[leftmargin=0.16in, topsep=2pt,itemsep=-1ex,partopsep=1ex,parsep=1ex]
    \item We propose a concise method for decomposing view-independent and view-dependent appearance during NeRF training and eliminate the interference of view-dependent appearance.
    \item We propose a geometric correction procedure performed on each traced ray during inference to refine the density estimation and better tackle the floater artifacts.
    \item Extensive experiments and ablations verify the effectiveness of our core designs and results in improvements over the vanilla NeRF and other state-of-the-art alternatives.
    %without additional computational resources or other inputs.
\end{itemize}











\section{Preliminaries}


\paragraph{Notation.} Given an input $x \in \bits^n$, coordinate $i \in [n]$, and setting $b \in \bits$, we use $x_{i = b}$ to refer to the input $x$ with the $i^{\text{th}}$ coordinate overwritten to take the value $b$. 
Similarly, given a sequence of (coordinate, value) pairs 
% $\pi \in ([n] \times \bits)^k$, 
$\pi = \{(i_1, b_1), \ldots, (i_k, b_k) \}$, 
we use $x_{\pi}$ to represent $x$ with the coordinates in $\pi$ overwritten/inserted with their respective values. 

Given a function $f: \bits^n \to \R$,  we denote the \emph{restriction} $f_{i = b}: \bits^n \to \R$ to be the function that maps $x$ to $f(x_{i = b})$. We define the restriction $f_\pi$ analogously. 

% \violet{Similarly, given a sequence of coordinates $\pi \in [n]^k$ and corresponding values $v \in \bits^k$, we use $x_{\pi = v}$ to represent $x$ with the coordinates in $\pi$ overwritten/inserted using the values in $v$. }


\begin{definition}[Decision trees (DT)]
 A decision trees $T : \bits^n \to \R$, is a binary tree whose internal nodes query a particular coordinate, and whose leaves are labelled by values. Each instance $x \in \bits^n$ follows a unique root-to-leaf path in $T$: at any internal node, it follows either the left or right branch depending on the value of the queried coordinate, until a leaf is reached and its value is returned. 
 
\end{definition}

 The set of leaves $\ell \in \mathrm{leaves}(T)$ therefore form a partition of $\bits^n$,  with each leaf having $2^{n - |\ell|}$ elements, where $|\ell|$ is the depth of the leaf.  Every leaf $\ell$ also corresponds to a sequence of  (coordinates, value) pairs  $\pi(\ell)$ that lead to the leaf. For a function $f$, will sometimes use the shorthand $f_\ell$ to mean the restriction $f_{\pi(\ell)}$.


\begin{definition}[Decision tree distribution]
    We say that a distribution $\mathcal{D} : \bits^n \to [0,1]$ is representable by a depth-$d$ DT, if its pmf is computable by a depth-$d$ decision tree $T$. 
    % Specifically, each leaf $\ell$ has a probability $w_\ell$ and the conditional distribution of points that reach a leaf is uniform. So, for any $x$ that reaches leaf $\ell$, we have $\mcD(x) = w_\ell/|\{y : y \in \ell\}|$.
    Specifically, each leaf $\ell$ is labelled by a value $p_\ell$, so that $\mcD(x) = p_\ell$ for all $x \in \ell$. This means that the conditional distribution of all points that reach a leaf is uniform. Moreover, since $\mcD$ is a distribution, we have: $\sum_{\ell \in \mathrm{leaves}(T)} 2^{n - |\ell|} \cdot p_\ell  = 1$.
   
\end{definition}
   
   For a given leaf $\ell$, we will write  $\mcD_\ell : \bits^{n - |\ell|} \to [0,1]$ to represent the conditional distribution of $\mcD$ at the leaf $\ell$, so that for any $x \in \bits^{n - |\ell|}$, we have $\mcD_\ell(x) = \mcD(x_{\pi(\ell)}) /\Pr_{y \sim \mcD}[y \in \ell]$.
    % {
    % \color{gray}
    % We will often scale $\mcD$ up by $2^n$ since it makes our analysis easier. As such, we also define the weighting function $f_\mcD(x) \coloneqq 2^n \mcD(x)$. 
    % }
    
    

    We will often scale up the pmfs of our distributions by the domain size, since it makes our analysis easier. As such, we also define the weighting function:
    

\begin{definition}[Weighting function of distribution]
Let $\ArbDist$ be an arbitrary distribution over $\bits^m$. We define the weighting function: 
\[
f_\ArbDist(x) \coloneqq 2^m \cdot \ArbDist(x).
\]

\end{definition}




\begin{definition}[Monotone distribution]
We furthermore say that a distribution $\mcD$ is monotone if its pmf is monotone: for $x, y \in \bits^n$, if $x_i \leq y_i$ for all $i \in [n]$, then $\mcD(x) \leq \mcD(y)$.
\end{definition}



\begin{definition}[TV Distance]
    For two distributions $\mcP, \mcQ$ over a countable domain $\mcX$, we define the total variation distance:
    \begin{equation*}
        \TV(\mcP, \mcQ) = \frac{1}{2}\sum_{x \in \mcX} | \mcP(x) - \mcQ(x)|= \frac{1}{2}\| \mcP - \mcQ\|_1 .
    \end{equation*}
\end{definition}

\begin{definition}[$\ell_1$ Influence]
For any function $f : \bits^n \to \R$, the influence of the $i$-th variable on $f$ is given by:
\begin{equation*}
\Inf_{i}(f) \coloneqq \Ex_{\bx \sim \mcU^n} \big[| f(\bx) - f(\bx^{\sim i}) |\big],
\end{equation*}   
where $\bx^{\sim i}$ denotes $\bx$ with the $i$-th coordinate re-randomised. Note that the influence of a function is defined with respect to the uniform distribution over its domain.

We further define the total influence as the sum of influences over all variables:
$
\Inf(f) \coloneqq \sum_{i=1}^n \Inf_i(f).
$


\end{definition}

\begin{fact}[Influence $\equiv$ correlation for monotone functions]\label{fact:inf_eq_corr}
    Let $f : \bits^n \to \R$ be a monotone function. Then 
    \begin{equation*}
        \Inf_i(f) = \Ex_{\bx \sim \mcU^n}[f(\bx) \cdot \bx_i] .
    \end{equation*}
\end{fact}

\begin{definition}[$\ell_1$ Variance]
\label{def:variance} 
    For any function $f : \bits^n \to \R$,
    
    \begin{equation*}
        \Var^{(1)}(f) \coloneqq \Ex_{\bx, \by \sim \mcU^n} |f(\bx) - f(\by)|.
    \end{equation*}
    
    We will also sometimes use a different definition of variance, given by the mean absolute deviation of $f$:
    % Equivalently\anote{I think we don't need this def anymore, also I think they aren't equal?x}, 
    \begin{equation*}
    \Var_{\mu}(f) \coloneqq \Ex_{\bx \sim \mcU^n} |f(\bx) - \Ex[f]|.
    \end{equation*}
\end{definition}

These two definitions are equivalent, up to constant factors:

\begin{lemma}\label{lem:var_defs}
    For a function $f : \bits^n \to \R$, 
    \begin{align*}
           \Var_{\mu}(f) \leq \Var^{(1)}(f) \leq 2\Var_{\mu}(f) 
    \end{align*}    
\end{lemma}

\begin{proof}
    The second part follows immediately from the triangle inequality and the first is an application of Jensen's:
    \[ \Var_{\mu}(f) =  \Ex_{\bx \sim \mcU}\left[\Big| \Ex_{\by \sim \mcU} [f(\bx) -  f(\by)]\Big|\right]\\
        \leq  \Ex_{\bx, \by \sim \mcU}\left|  f(\bx) -  f(\by)\right|.  \qedhere \]  
    % \begin{align*}
    %     \Var_{\mu}(f) &=  \Ex_{\bx \sim \mcU}\left[\left| \Ex_{\by \sim \mcU} [f(\bx) -  f(\by)]\right|\right]\\
    %     &\leq  \Ex_{\bx, \by \sim \mcU}\left|  f(\bx) -  f(\by)\right|   \tag{Jensen's}
    % \end{align*} \
\end{proof}


\begin{definition}[Sensitivity]
For a function $f:\bits^n \to \R$ and $x \in \bits^n$, the sensitivity of $f$ at $x$ is defined to be
\[s(f,x) = \sum_{i=1}^n \Ind[f(x) \ne f(x^{\oplus i})].\] 
Furthermore, the sensitivity of $f$ is given by its maximum sensitivity over all points:
\[s(f) = \max_{x \in \bits^n} \{ s(f,x) \}.\]
\end{definition}

Note that the sensitivity of a decision tree is at most the depth of the decision tree, since any point can only be sensitive to the coordinates queried on its root-to-leaf path.

\subsection{Useful inequalities}
We present some useful inequalities for boolean functions.

\begin{lemma}[Efron-Stein]\label{lem:efron_stein}
For any function $f: \bits^n \to \R$:
\begin{align*}
    \Var^{(1)}(f) \leq \Inf(f).
\end{align*}
\end{lemma}



\begin{lemma}[Total influence and sensitivity]\label{lem:inf_s_var}
    For any function $f : \bits^n \to \R$:
    \begin{align*}
        \Inf(f) \leq 2s(f) \cdot \Var^{(1)}(f).
    \end{align*}
\end{lemma}

\begin{proof}
    Let $s = s(f)$ be the sensitivity of $f$ and consider the set, $\mathrm{snbr}(x) = \{i \in [n] \mid f(x) \ne f(x^{\oplus i})\}$. By assumption, $|\mathrm{snbr}(x)| \leq s$. We define a coupling $(\bx, \by) \sim \pi$ s.t. $\by$ is often in $\mathrm{snbr}(\bx)$,  but the marginal distributions $\pi(\bx)$ and $\pi(\by)$ are still uniform. First, sample $\bx \sim \mcU$. Then, sample $\by$ given $\bx$ as follows: for each $i \in \mathrm{snbr}(\bx)$, let $\by = \bx^{\oplus i}$ (i.e. flip the $i$-th coordinate of $\bx$) with probability $1/s$, and with the remaining $ 1 - |\mathrm{snbrs}(\bx)|/s$ probability, take $\by = \bx$. It is easy to see that the marginal distribution over $\by$ is still uniform.
    
    Unrolling the definition of influence, we have:
    \begin{align*}
        \Inf(f) 
            &= \sum_{i=1}^n \Ex_{\bx \sim \mcU} |f(\bx) - f(\bx^{\oplus i})|  \\
            &= \Ex_{\bx \sim \mcU} \left[ \sum_{i=1}^n |f(\bx) - f(\bx^{\oplus i})| \right ]  \\
            &= \Ex_{\bx \sim \mcU} \left[ \sum_{i \in \mathrm{snbr}(\bx)} |f(\bx) - f(\bx^{\oplus i})| \right ] \tag{only consider nonzero terms}\\
            &=  \Ex_{\bx \sim \mcU} \left[ s \cdot \sum_{i \in \mathrm{snbr}(\bx)} \frac{|f(\bx) - f(\bx^{\oplus i})|}{s}  \right ] \\
            &=  \Ex_{\bx \sim \pi} \left[ s \cdot \Ex_{\by \sim \pi(\cdot | \bx)} |f(\bx) - f(\by)|  \right ] \tag{definition of coupling $\pi$}\\
            &=  s \cdot\Ex_{(\bx, \by) \sim \pi} |f(\bx) - f(\by)|  \\
            &\leq  s \cdot \Ex_{(\bx, \by) \sim \pi}  |f(\bx) - \E[f]|+ s \cdot \Ex_{(\bx, \by) \sim \pi} |f(\by) - \E[f]|  \tag{triangle inequality}\\
            &=  2s \cdot \Var_\mu(f) \tag{marginal distributions of $\pi$ are uniform }\\
            &\leq  2s \cdot \Var^{(1)}(f) \tag{\Cref{lem:var_defs}}.
    \end{align*}
\end{proof}



\section{Our algorithmic decomposition lemma}
\label{sec:decomp}

Here we present an algorithm that constructs a decision tree of depth $d$ for a a distribution $\mcD$, and analyze its correctness and complexity. 
Throughout this section, we assume access to an oracle that gives the exact influences of variables in $f_\mcD$ or any of its restrictions. 
In the next sections we will show that the influences can be estimated from random examples for monotone distributions, and from subcube conditional examples for general distributions. 
%We will also assume an oracle for the densities of the leaves, up to accuracy $\eps \cdot 2^{-{d+1}}$ and failure probability $\delta \cdot 2^{-d}$ --- by Chernoff and union bounds this oracle can be simulated using random examples.

\begin{theorem}[Learning decision tree distributions]
    \label{thm:decompose}
    Let $\mcD$ be a distribution that is representable by a depth-$d$ decision tree.
    The algorithm $\BuildDT$ returns a depth-$d$ tree representing a distribution $\mcD'$
    such that $\TV(\mcD,\mcD') \le \eps$ w.h.p..
    Given access to a unit time influence oracle, its running time is $n \cdot (d/\eps)^{O(d)}$.
\end{theorem}

The algorithm $\BuildDT$ is an exhaustive search over a subset of depth-$d$ decision trees. 
We characterize this subset as follows: 

\begin{definition}[Everywhere $\tau$-influential]
Let $T$ be a tree and $\nu$ be an internal node with root variable $i(\nu)$. $T$ is \emph{everywhere $\tau$-influential} with respect to some $f$ if for every $\nu \in T$, we have $\Inf_{i(\nu)}(f_\nu) \ge \tau$.
\end{definition}

\begin{figure*}[t] 
  \captionsetup{width=.9\linewidth}

\begin{tcolorbox}[colback = white,arc=1mm, boxrule=0.25mm]
\vspace{3pt} 

$\BuildDT(\mcD, \pi, d, \tau)$:

\begin{itemize}%[align=left]
    \item[]\textbf{Input:} Random examples from $\mcD$, restriction $\pi$, influence oracle for $(f_\mcD)_\pi$, depth parameter $d$, influence parameter $\tau$.
    \item[]\textbf{Output:} A decision tree $T$ that minimizes $\Ex_{\bell \in T}[\Inf((f_\mcD)_{\bell})]$ among all depth-$d$, everywhere $\tau$-influential trees.
\end{itemize}
\begin{enumerate}
    \item Let $S \subseteq [n]$ be the set of variables $i$ such that $\Inf_i((f_\mcD)_\pi) \ge \tau$.
    \item If $S$ is empty or $d=0$, return the leaf labeled with $2^{|\pi|} \cdot \Pr_{\bx \sim \mcD}[\bx\text{ is consistent with }\pi]$.
    \item Otherwise: 
    \begin{enumerate}
        %\item Set $M[\pi, s] = M[\pi, s-1]$. 
        \item For each $i \in S$, let $T_i$ be the tree such that 
            \begin{align*}
                \mathrm{root}(T_i) &= x_i \\
                \textnormal{left-subtree}(T_i) &= \BuildDT(\mcD, \pi \cup \{x_i = -1\}, d-1, \tau) \\
                \textnormal{right-subtree}(T_i) &= \BuildDT(\mcD, \pi \cup \{ x_i = 1\}, d-1, \tau)
            \end{align*}
        \item Return the tree among the $T_i$'s defined above that minimizes $\Ex_{\bell \in T_i}[\Inf((f_\mcD)_{\bell})]$.
    \end{enumerate}
\end{enumerate}

% $\BuildDT(\mcD, \pi, d, \tau)$:
% \begin{itemize}%[align=left]
%     \item[]\textbf{Input:} Random examples from $\mcD$, influence oracle for $(f_\mcD)_\pi$, restriction $\pi$, depth parameter $d$, influence parameter $\tau$.
%     \item[]\textbf{Output:} A decision tree $T$ that minimizes $\Ex_{\ell \in T}[\Inf((f_\mcD)_\ell)]$ among all depth-$d$, everywhere $\tau$-influential trees.
% \end{itemize}
% \begin{enumerate}
%     \item Let $S \subseteq [n]$ be the set of variables $i$ such that $\Inf_i((f_\mcD)_\pi) \ge \tau$.
%     \violet{
%     \item If $S$ is empty or $d=0$, return the leaf labeled $2^{|\pi|} \cdot \Pr_{\bx \sim \mcD}[\bx\text{ is consistent with }\pi]$.}
%     \item Otherwise: 
%     \begin{enumerate}
%         %\item Set $M[\pi, s] = M[\pi, s-1]$. 
%         \item For each $i \in S$, let $T_i$ be the tree such that 
%             \begin{align*}
%                 \mathrm{root}(T_i) &= x_i \\
%                 \textnormal{left-subtree}(T_i) &= \BuildDT(\mcD, \pi \cup \{x_i = -1\}, d-1, \tau) \\
%                 \textnormal{right-subtree}(T_i) &= \BuildDT(\mcD, \pi \cup \{ x_i = 1\}, d-1, \tau)
%             \end{align*}
%         \item Return the tree among the $T_i$'s defined above that minimizes $\Ex_{\ell \in T_i}[\Inf((f_\mcD)_\ell)]$.
%     \end{enumerate}
% \end{enumerate}
\end{tcolorbox}

\caption{$\BuildDT$ recursively searches for the depth-$d$, everywhere $\tau$-influential tree of minimal influence at the leaves.}
\label{fig:BuildDT}
\end{figure*} 

\subsection{Correctness} 
Here we show that under the oracle assumptions described above, $\BuildDT$ returns a tree within TV distance $\eps$. The proof will rely on the following fact, which relates TV distance to the uniform $\ell_1$ error of the tree with respect to $f_\mcD$. 

\begin{fact}[TV distance = label error]
\label{fact:distance error}
     \begin{align*}
			\TV(\mcD, \mcD') &= \frac{1}{2} \cdot \|\mcD - \mcD'\|_1 \\
							 &= 2^{-(n+1)} \cdot \|2^n \mcD - 2^n \mcD'\|_1 \\
							 &= 2^{-(n+1)} \cdot \|f_\mcD - T'\|_1.
        \end{align*}
\end{fact}

First, we will show that $\BuildDT$ outputs a decision tree $T'$ with small average influence at the leaves.
Then, we will show that this implies that the uniform $\ell_1$ error of $T'$ with respect to $f_\mcD$ is small.
Correctness follows from the equivalence between $2^{-(n+1)}\| f_\mcD- T' \|_1$ and $\TV(\mcD, \mcD')$. \\

The claim that $\BuildDT$ outputs a decision tree $T'$ with small average influence at the leaves extends a lemma from \cite{BLQT21focs}, instantiated here for the metric space $\R$ equipped with the  $\ell_1$-norm:

% \begin{lemma}[Theorem 5 of \cite{BLQT21focs}]
% \label{lem:blqt}\anote{Should we just write this with $\mcY = [0, 2^n]$? It will link better with the next part}
% Let $\mathcal{Y}$ be a metric space. Let $f : \bits^n \to \mathcal{Y}$ be representable by a depth-$d$ DT $T$. Then there exists $T^\star$ such that the following are satisfied:

% \begin{enumerate}
%     \item The size and depth of $T^\star$ are at most the size and depth of $T$,
%     \item $T^\star$ is everywhere $\tau$-influential with respect to $f$,
%     \item $\mathrm{dist}_{\mathcal{Y}}(f(\bx) - T^\star(\bx)) \le d\tau$.
% \end{enumerate}
% \end{lemma}


\begin{lemma}[Theorem 5 of \cite{BLQT21focs}]
\label{lem:blqt} Let $f : \bits^n \to \R$ be representable by a depth-$d$ DT $T$. Then there exists $T^\star$ such that the following are satisfied:

\begin{enumerate}
    \item The size and depth of $T^\star$ are at most the size and depth of $T$,
    \item $T^\star$ is everywhere $\tau$-influential with respect to $f$,
    \item $2^{-n} \cdot \| f - T^\star \|_1 \le d\tau$.
\end{enumerate}
\end{lemma}

% This lemma suffices for \cite{BLQT21focs} because they have access to labelled examples, allowing them to compute $\ell_1$ error and directly search for trees that minimise it. 

In our \BuildDT, we cannot compute $\|f_\mcD - T'\|_1$ and hence cannot search for trees that minimise this error. Instead, we find trees that minimise the expected total influence at the leaves. The following lemma relates these two values:


\begin{lemma}[Expected total influence and $\ell_1$ error]\label{lem:inf_leaves_l1}
Let $f : \bits^n \to \R$ be representable by a depth-$d$ DT, and let $T'$ be any other DT. Then:

\begin{equation*}
    \Ex_{\bell \in T'} [\Inf(f_{\bell})] \leq 4d \cdot 2^{-n} \|f - T'\|_1
\end{equation*}
\end{lemma}

\begin{proof}
	Since $f$ is representable by a depth-$d$ decision tree, its maximum sensitivity (and the sensitivity of each of its leaf restrictions) must be at most $d$. Therefore, for any leaf $\ell \in T'$, \Cref{lem:inf_s_var} asserts that $\Inf(f_{\ell}) \leq  2d \cdot \Var^{(1)}(f_{\ell})$. Moreover, 
	\begin{align*}
	    \Var^{(1)}(f_{\ell}) 
	        &= \Ex_{\bx, \by \sim \mcU^n} |f_\ell(\bx) - f_\ell(\by)| \\
	        &= \Ex_{\bx, \by \sim \mcU^n} |f_\ell(\bx) - T'_\ell + T'_\ell - f_\ell(\by)| \\
	        &\leq 2\cdot \Ex_{\bx \sim \mcU^n} |f_\ell(\bx) - T'_\ell| \tag{Triangle ineq.}\\
	        &= 2\cdot \Ex_{\bx \sim \mcU^n} [|f(\bx) - T'(\bx)| \mid \bx \in \ell].
	\end{align*}
	
	Therefore, 
	\begin{align*}
	    \Ex_{\bell \in T'} [\Inf(f_{\bell})] 
	        &\leq 2d \cdot \Ex_{\bell \in T'} [\Var^{(1)}(f_{\bell})] \\
	        &\leq 4d \cdot \Ex_{\bell \in T'} \Ex_{\bx \sim \mcU^n} [|f(\bx) - T'(\bx)| \mid x \in \ell] \\
	        &= 4d \cdot \Ex_{\bx \sim \mcU^n} |f(\bx) - T'(\bx)| \\
	        &= 4d  \cdot 2^{-n} \cdot \|f(\bx) - T'(\bx)\|_1. \qedhere 
	\end{align*}	
\end{proof}

As a corollary of \Cref{lem:blqt} and \Cref{lem:inf_leaves_l1}, we get our pruning lemma stated in terms of influences:

\begin{corollary}[Pruning lemma with expected total influence at leaves]
\label{cor:influence}
Let $f : \bits^n \to \R$ be representable by a depth-$d$ DT $T$. Then there exists $T^\star$ such that the following are satisfied:

\begin{enumerate}
    \item The size and depth of $T^\star$ are at most the size and depth of $T$,
    \item $T^\star$ is everywhere $\tau$-influential with respect to $f$,
    \item $\Ex_{\bell \in T^\star}[\Inf(f_{\bell})] \leq 4d^2 \tau$.
\end{enumerate}

\end{corollary}

% \begin{corollary}[Pruning lemma for total influence at leaves]
% \label{cor:influence}
% Let $f : \bits^n \to [0, 2^n]$ be representable by a depth-$d$ DT $T$. Then there exists $T^\star$ such that the following are satisfied:

% \begin{enumerate}
%     \item The size and depth of $T^\star$ are at most the size and depth of $T$,
%     \item $T^\star$ is everywhere $\eps/8d^2$-influential with respect to $f_\mcD$,
%     \item $\Ex_{\bell \in T^\star}[\Inf((f_\mcD)_{\bell} )] \leq \eps/2$.\anote{This shouldn't have the 2 here right?} \jnote{oops, yes this should be $\eps/2$ instead of $2\eps$ and the setting of $\tau$ should fix this}
% \end{enumerate}

% \end{corollary}

% \begin{proof}
%     \violet{
%     Let $T^*$ be the tree guaranteed by applying \Cref{lem:blqt} to $f$. Then 1 and 2 follow immediately. 
%     }
% 	To prove 3, we first note that because $f$ is reprsentable by a depth-$d$ decision tree,
% 	its maximum sensitivity (and the sensitivity of each of its leaf restrictions) must be at most $d$.
% 	Then \Cref{lem:inf_s_var} asserts that	$\Ex_{\bell \in T^\star} [\Inf(f_{\bell})] \le d \cdot \Ex_{\bell \in T^\star}[\Var^{(1)}[f_{\bell}]]$.
% 	Then, for each $\ell$, we note that $\Var^{(1)}(f_\ell) \le 2\Var_\mu((f_\mcD)_\ell)$, 
% 	and that $\Var_\mu((f_\mcD)_\ell)$ minimizes $\Ex_{\bx \sim \mcU^n}|(f_\mcD)_\ell(\bx) - c|$ over all constants $c$.\anote{I think this isnt true but we can just direct triangle ineq?}  
% 	Therefore, 
% 	\[\Var^{(1)}[(f_\mcD)_\ell] \le 2\Var_\mu[(f_\mcD)_\ell] \le 2\Ex_{\bx \sim \mcU}\big [ |(f_\mcD)_\ell(\bx) - T^\star_\ell(\bx)| \big ] = 2^{-(n -1)} \cdot \|(f_\mcD)_\ell - T^\star_\ell\|_1.\]\jnote{using $2^{-(n-1)}$ instead of $2^{-(n -|\ell| - 1)}$ to go along with the notation of restricted functions, not restricted domains. This is our choice though and can be changed.}
% 	Having established that 
% 	\[\Ex_{\ell \in T^\star}[\Inf((f_\mcD)_\ell)] \le 2d \cdot \Ex_{\bell \in T^\star} \big [ 2^{-(n-1)} \cdot \|(f_\mcD)_{\bell} - T^\star_{\bell}\|_1 \big ],\]
% 	we now make use of the fact that $\Ex_{\bell \in T^\star} \big [ 2^{-n} \cdot \|(f_\mcD)_{\bell} - T^\star_{\bell}\|_1 \big ]$ is just $2^{-n} \cdot \|f_\mcD - T^\star\|_1$, 
% 	which by the third point of \Cref{lem:blqt} is at most $d\tau$.
% 	The corollary follows from our choice of $\tau = \eps/8d^2$.
% \end{proof}

We now move to show that the tree output by \BuildDT \  satisfies $2^{-n} \cdot \|f_\mcD - T'\|_1 \le \eps$.


\begin{claim}
\label{claim:low error}
Let $\mcD$ be a distribution that is representable by a depth-$d$ decision tree, and let $T$ be the output of $\BuildDT$, with $\tau = \eps/8d^2$. Then, with high probability, $2^{-n} \cdot \|f_\mcD - T \|_1 \le \eps$.
\end{claim}


\begin{proof}
    First, we claim that $T$ minimizes $\E_{\bell \in T}[\Inf((f_\mcD)_{\bell})]$ among all depth-$d$, everywhere $\tau$-influential trees.
    This claim holds by induction on $d$: since 
    \[\E_{\bell \in T}[\Inf((f_\mcD)_{\bell})] = \lfrac{1}{2}(\E_{\bell \in T_{\mathrm{left}}}[\Inf((f_\mcD)_{\bell})] + \E_{\bell \in T_{\mathrm{right}}}[\Inf((f_\mcD)_{\bell})]),\] 
    each candidate $T_i$ minimizes influence among all depth-$d$, everywhere $\tau$-influential trees with $x_i$ at the root 
    under the inductive assumption that $T_{\mathrm{left}}$ and $T_{\mathrm{right}}$ minimize influence for depth-$(d-1)$ trees. 
    Then $\BuildDT$ chooses the tree of smallest influence among all the candidate $T_i$'s,
    so it minimizes total influence at leaves among all trees in its search space of depth-$d$, $\tau$-influential trees. 
    
    Since \Cref{cor:influence} establishes the existence of a tree with average total influence at leaves $\le \eps/2$, 
    it follows that the influence $T$'s influence is also $\le \eps/2$.
    Furthermore, we may assume by standard Hoeffding bounds that with a sample size of $\poly(2^d, 1/\eps)$, each leaf's value estimate of $\E[f_\mcD(\bx)~|~\bx\text{ is consistent with }\ell] = 2^{|\ell|} \cdot \Pr_{\bx \sim \mcD}[\bx\text{ is consistent with }\ell]$ is accurate to within $\pm \eps/2$ w.h.p..  
    
    
    We can now show that $2^{-n} \cdot \|f_\mcD - T\|_1 \le \eps$. 
    Throughout this section, $x \in \ell$ will stand in as shorthand for ``$x\text{ consistent with the restriction at }\ell$''.
    \begin{align*}
    2^{-n} \cdot \|f_\mcD - T\|_1 &= 2^{-n} \cdot \sum_{x \in \bits^n} |f_\mcD(x) - T(x)|\\
					 &= 2^{-n} \cdot \sum_{\ell \in T}\sum_{x \in \ell} |(f_\mcD)_\ell(x) - T_\ell(x)| \\
					 &\le 2^{-n} \cdot \sum_{\ell \in T}\sum_{x \in \ell} |(f_\mcD)\ell(x) - \E[(f_\mcD)_\ell]| + |\E[(f_\mcD)_\ell] - T_\ell(\bx)|  \\
					 &= 2^{-n} \cdot \big(\sum_{\ell \in T} 2^{n-|\ell|} \cdot \Var_\mu((f_\mcD)_\ell) + \sum_{\ell \in T}2^{n-|\ell|} \cdot |\E[(f_\mcD)_\ell] - T_\ell|\big) \\
				     &\le \E_{\ell \in T} \Inf((f_\mcD)_\ell)) + \E_{\ell \in T} \big [|\E[(f_\mcD)_\ell(\bx)] - T_\ell| \big ] \tag{\Cref{lem:efron_stein}}\\
				     &\le \frac{\eps}{2} + \frac{\eps}{2} = \eps.\qedhere 
    \end{align*}
\end{proof}


We can now prove correctness and time bounds for $\BuildDT$.
\begin{proof}[Proof of \Cref{thm:decompose}]
Correctness follows by combining \Cref{claim:low error} and \Cref{fact:distance error}. To bound the running time, first we note that for all restrictions $\pi$, there are at most $d/\tau = 8d^3/\eps$ variables of influence at least $\tau$. This is because all restrictions of $f_\mcD$ are depth-$d$ decision trees, and for any depth-$d$ decision tree, the sum of all variable influences is at most $d$. 

Then the number of recursive calls to $\BuildDT$ is at most $(8d^3/\eps)^d$. Each call at an internal node makes $n$ calls to the unit-time influence oracle, and each call at a leaf processes $\poly(2^d, 1/\eps)$ samples to estimate the leaf label. Thus, the total running time is $n\cdot (d/\eps)^{O(d)}$, as desired.
\end{proof}



%\subsection{Relating distribution learning and function learning}\lnote{Update subsection title}

%We can relate the task of learning the DT for $\mcD$ to the regular function learning over the uniform distribution by relating the following:









% \section{Top-Down DT}
% Will write top-down algorithm in appendix too.
% In order to run something like a TopDown algorithm, we need a goal, a measure of progress towards the goal, a way to split the tree, and a guarantees that while the goal hasn't been reached, there is a way to split the tree that will make significant progress.

% In traditional DT learning of a function $f: \bits^n \to [0,1]$ we have:
% \begin{enumerate}
%     \item \textbf{Goal}: $\error(T, f) \leq \eps$.
%     \item \textbf{Measure of progress}: Impurity function $\mcG(T)$ which satisfies $\mcG(T) \geq \error(T, f)$ and starts with a bounded value. Examples: $\mcG(T) = \Ex_{\ell} [\Var_\ell] \leq 1$ or  $\mcG(T) = \Ex_{\ell} [ \Inf(f_\ell)] \leq \log s$ if $f$ is a size-$s$ DT. 
%     \item \textbf{Split}: Split leaf $\ell$ and index $i$ with highest score e.g.  $w_\ell \cdot \Delta_\ell$ (drop in impurity), $w_\ell \cdot \Ex[f(x) x_i]$, or $w_\ell \cdot  \Inf_i(f_\ell)$ (for uniform dist). The last two are equal for monotone $f$.
%     \item \textbf{Guarantee}: A guarantee that while $\error(T, f) > \eps$, there is a leaf with a high score. Example guarantees: KM weak-learning assumption $\Delta_\ell \geq \gamma^2 \eps_\ell^2$ or OSSS which gives $\max_i \Inf^{(\rho)}_i(f_\ell) \geq \Var^{(\rho)}[f_\ell]/\log s$ if $f$ is a size-$s$ DT.
% \end{enumerate}



% \begin{fact}[Relationship between bias at leaves and bias at root]
%     For any distribution $\mcD$, and $i,j \in [d]$.
%     \begin{equation*}
%         \bias_j(\mcD) = \Ex_{\bx \sim \mcD}\left[\bias_j( \mcD| \bx_i = x_i ) \right]
%     \end{equation*}
% \end{fact}

% \begin{corollary}
%     For any distribution $\mcD$ and $i \in [d]$,
%     \begin{equation*}
%         \Ex_{\bx_i \sim \mcD_i}[\bias(\mcD_{x_i = \bx_i})] = \bias(\mcD) - \bias_i(\mcD).
%     \end{equation*}
% \end{corollary}




\section{Algorithms for computing distributional influences} 



% \gray{In this section, we will give estimators for computing $\Inf_i(f_\mcD)$ given uniform samples or subcube conditional samples, for non-monotone and general distributions respectively. In \Cref{sec:decomp}, we will actually require an estimator, for any restriction $\pi$, of $\Inf_i((f_\mcD)_{\ell})$. This is nearly equivalent, as by defining $\mcD' = \mcD_{\ell}$, we can use the below estimators to compute $\Inf_i(f_{\mcD_\ell})$.}

In \Cref{sec:decomp}, we assumed the ability to exactly compute influences of $f$ and its restrictions in unit time. In this section, we show how to instead estimate the influences from samples, when the distribution is monotone or when we have access to subcube conditional samples. Just as in \cite{BLQT21focs}, our proof only requires estimates to be accurate to $\pm \min(\tau/4, \eps/n)$. Letting $\InfEst_i(f)$ denote such an estimate of $\Inf_i(f)$, the pseudocode in \Cref{fig:BuildDT} and proof of \Cref{thm:decompose} is modified as follows.
\begin{enumerate}
    \item We modify $S$ to include variables $i$ such that $\InfEst_i((f_{\mcD})_\pi) \geq 3\tau/4$. Since the estimate is accurate to $\pm \tau/4$, this is guaranteed to include all variables with influence $\geq \tau$, and furthermore will only include variables with influence at least $\tau/2$. Therefore, the total size of $S$ is at most $\frac{d}{\tau/2}$, which is only a constant factor (of $2$) larger than in the proof of \Cref{thm:decompose} which assumed perfect influence oracles, and there does not affect asymptotic runtime or sample complexity.
    \item It returns the tree $T_i$ that minimizes $\Ex_{\bell \in T_i}[\sum_{i \in [n]}\InfEst_i((f_\mcD)_{\bell})]$. Since each estimate is accurate to $\eps/n$, this estimate of total influence of $(f_\mcD)_{\bell})$ will be accurate to $\pm \eps$. Then, in \Cref{claim:low error}, rather than $\BuildDT$ building a tree $T$ minimizes $\E_{\bell \in T}[\Inf((f_\mcD)_{\bell})]$ among all depth-$d$, everywhere $\tau$-influential trees, $T$ (roughly) minimizes $\E_{\bell \in T}[\sum_{i \in [n]}\InfEst_i((f_\mcD)_{\bell})]$ among all depth-$d$, everywhere $\tau$-influential trees. More formally, $T$ will either be the best depth-$d$, everywhere $\tau$-influential trees, or better, as we know its searches over all variables with $\InfEst_i((f_{\mcD})_\pi) \geq 3\tau/4$ which is guaranteed to include variables with influence $\geq \tau$, but can include more variables. Finally, since the estimates of total influence are accurate to $\pm \eps$, using $\InfEst$ rather than $\Inf$ can only incur at most $2\eps$ additive error, which is a constant factor in the analysis.
\end{enumerate}

Given any distribution $\mcD$ over $\bits^n$, we need to estimate $\Inf_i((f_\mcD)_{\ell})$ for any restriction $\ell$ of $f_\mcD$ and $i \in [n] - \ell$. In this section, we will instead show how to compute $\Inf_i(f_{\ArbDist})$ for any distribution $\ArbDist$ over $\bits^m$.  We can then use our estimators with $\ArbDist = \mcD_\ell$  to obtain the necessary answers using the following fact:

\begin{fact}
    \label{fact:inf-estimates-scaling}
    For any distribution $\mcD$ over $\bits^n$,  restriction $\ell$ of $\bits^n$, and $i \in [n] - \ell$:
    \begin{equation*}
        \Inf_i((f_\mcD)_{\ell}) = 2^{|\ell|} \cdot \Prx_{x \sim \mcD}[x \in \ell] \cdot \Inf_i(f_{\mcD_\ell}).
    \end{equation*}
\end{fact}
\begin{proof}
    Let $w_\ell = \Prx_{\bx \sim \mcD}[\bx \in \ell]$. We have: 
    \begin{align*}
        \Inf_i((f_\mcD)_{\ell}) 
            &= \Ex_{\bx \sim \mcU^n} \left[\left| f_\mcD(\bx) - f_\mcD(\bx^{\sim i}) \right | \mid x \in \ell \right] \\
            &= 2^n \cdot \Ex_{\bx \sim \mcU^n} \left[\left| \mcD(\bx) - \mcD(\bx^{\sim i}) \right | \mid x \in \ell \right] \\
            &= 2^n \cdot w_\ell \cdot \Ex_{\bx \sim \mcU^n} \left[\left| \frac{\mcD(\bx)}{w_\ell} - \frac{\mcD(\bx^{\sim i})}{{w_\ell}} \right | \mid x \in \ell \right]  \\
            &= 2^n \cdot w_\ell \cdot \Ex_{\by \sim \mcU^{n - |\ell|}} \left| \mcD_\ell(\by) - \mcD_\ell(\by^{\sim i}) \right |   \\
            &= 2^{|\ell|} \cdot w_\ell \cdot \Ex_{\by \sim \mcU^{n - |\ell|}} \left| f_{\mcD_\ell}(\by) - f_{\mcD_\ell}(\by^{\sim i}) \right |   \\
            % &= \frac{2^n}{2^{n - |\ell|}} \cdot \mcD(x \in \ell) \Ex_{\bx \sim \mcU^n} \left[\left| f_{\mcD_\ell}(\bx) - f_{\mcD_\ell}(\bx^{\sim i}) \right | \mid x \in \ell \right] \\
            % &= 2^{|\ell|} \cdot \mcD(x \in \ell) \Ex_{\bx \sim \mcU^{n-|\ell|}} \left[\left| f_{\mcD_\ell}(\bx_{\cup \ell}) - f_{\mcD_\ell}(\bx_{\cup \ell}^{\sim i}) \right | \right] \\
            % &= \Ex_{\bx \sim \mcU^{n - |\ell|}} \left[\left| f_\mcD(\bx) - f_\mcD(\bx^{\sim i}) \right |\right] \\
            % &= 2^n \cdot \Ex_{\bx \sim \mcU^{n - |\ell|}} \left[\left| \mcD(\bx) - \mcD(\bx^{\sim i}) \right |\right] \\
            % &= 2^n \cdot \mcD(x \in \ell) \cdot \Ex_{\bx \sim \mcU^{n - |\ell|}} \left[\left| \mcD_\ell(\bx) - \mcD_\ell(\bx^{\sim i}) \right |\right] \\
            % &= \frac{2^n}{2^{n - |\ell|}} \cdot \mcD(x \in \ell) \cdot \Ex_{\bx \sim \mcU^{n - |\ell|}} \left[\left| f_{\mcD_\ell}(\bx) - f_{\mcD_\ell}(\bx^{\sim i}) \right |\right] \\
            &= 2^{|\ell|} \cdot w_\ell \cdot \Inf_i ( f_{\mcD_\ell}). \qedhere
    \end{align*}
    % \begin{align*}
    %     \Inf_i((f_\mcD)_{\ell}) 
    %         &= \Ex_{\bx \sim \mcU^n} \left[\left| f_\mcD(\bx) - f_\mcD(\bx^{\sim i}) \right | \mid x \in \ell \right] \\
    %         &= \Ex_{\bx \sim \mcU^{n - |\ell|}} \left[\left| f_\mcD(\bx) - f_\mcD(\bx^{\sim i}) \right |\right] \\
    %         &= 2^n \cdot \Ex_{\bx \sim \mcU^{n - |\ell|}} \left[\left| \mcD(\bx) - \mcD(\bx^{\sim i}) \right |\right] \\
    %         &= 2^n \cdot \mcD(x \in \ell) \cdot \Ex_{\bx \sim \mcU^{n - |\ell|}} \left[\left| \mcD_\ell(\bx) - \mcD_\ell(\bx^{\sim i}) \right |\right] \\
    %         &= \frac{2^n}{2^{n - |\ell|}} \cdot \mcD(x \in \ell) \cdot \Ex_{\bx \sim \mcU^{n - |\ell|}} \left[\left| f_{\mcD_\ell}(\bx) - f_{\mcD_\ell}(\bx^{\sim i}) \right |\right] \\
    %         &= 2^{|\ell|} \cdot \mcD(x \in \ell) \cdot \Inf_i ( f_{\mcD_\ell})
    % \end{align*}
\end{proof}

    Because of \Cref{fact:inf-estimates-scaling}, for any restriction $\ell$ of depth at most $d$, to estimate $ Inf_i((f_\mcD)_{\ell})$ to accuracy $\pm \eps$, it is sufficient to estimate $\Inf_i(f_{\mcD_\ell})$ to accurate $\pm \eps/2^d$. This $2^d$ factor is dominated by the $d^{O(d)}$ term in \Cref{thm:learn-DT}, so we are free to do the later.


\subsection{Monotone distributions using samples} 

Let $\ArbDist$ be an arbitrary distribution over $\bits^m$.  If $\ArbDist$ is monotone, the influences of $f_\ArbDist$ can be efficiently computed directly from samples of $\ArbDist$, via an estimate of bias:

% In the monotone setting, the influences of $f_\mcD$ can be efficiently computed directly from samples of $\mcD$, via an estimate of bias:

% \begin{definition}[bias of a distribution]
%     For any $i \in [m]$,
%     \begin{equation*}
%         \bias_i(\ArbDist) \coloneqq \Ex_{\bx \sim \ArbDist}[(\bx_i - 1/2)].
%     \end{equation*}
% \end{definition}


\begin{lemma}[Estimating influence using bias]\label{lem:bias_inf}

If $\ArbDist$ is monotone,

\begin{equation*}
        \Inf_i(f_\ArbDist) = \Ex_{\bx \sim \ArbDist}[\bx_i].
    \end{equation*}
\end{lemma}
\begin{proof}
    Using \Cref{fact:inf_eq_corr}, 
    \begin{align*}
        \Inf_i(f_\ArbDist) 
        &=   \Ex_{\bx \sim \mcU^m}[f_\ArbDist(\bx) \cdot \bx_i]  \\ 
        &= \sum_{x \in \bits^m} 2^{-m} \ f_\ArbDist(x) \cdot x_i   \\
        &= \sum_{x \in \bits^m} \ArbDist(x) \cdot  x_i \\
        &=  \Ex_{\bx \sim \ArbDist}[ \bx_i ] . \qedhere 
    \end{align*} 
\end{proof}


As a simple application of the above and Hoeffding's inequality, we obtain the following corollary.
\begin{corollary}
[Estimating influences of monotone distributions]
\label{cor:high-accuracy-monotone}
    For any $\eps, \delta > 0$, there is an efficient algorithm that given unknown monotone distribution $\ArbDist$, computes an estimate of $\Inf_i(f_{\ArbDist})$ to accuracy $\pm \eps$ with probability at least $1 - \delta$ using $O(\log(1/\delta)/ \eps^2)$ random samples from $\ArbDist$.
\end{corollary}
Recall that for \Cref{thm:decompose}, we only need the influence estimates to be accurate to $\pm \poly(2^{-d}, \tau, \eps,1/n)$. Setting $\tau = O(\eps/d^2)$ and union bounding over $n \cdot (d/\eps)^{O(d)}$ calls to the influence oracle gives a sample complexity of $\poly(n, 2^d, \eps, \log(1/\delta))$. The running time is still dominated by the number of recursive calls.




% To estimate the bias to accuracy $\pm \tau$ with failure probability $\frac{\delta}{n \cdot (d/\eps)^d}$, it is sufficient to take the empirical bias from a sample of size $\frac{1}{\tau^2} \cdot (\log(1/\delta) + d \log (d/\eps))$, by the standard Hoeffding bound. Setting $\tau = O(\frac{\eps}{d^2})$ and union bounding over the $n \cdot (d/\eps)^{O(d)}$ calls to the influence oracle gives a sample complexity of $\poly(d, 1/\eps, \log(1/\delta))$. However, $2^{\Theta(d)}$ samples are still required to estimate the leaf densities to the required accuracy of $2^{-d}$, so the final sample complexity is $\poly(2^d, 1/\eps, \log(1/\delta))$. The running time is still dominated by the number of recursive calls.

\begin{corollary}[Sample complexity of $\BuildDT$ for monotone distributions]
The algorithm $\BuildDT(\mcD, \varnothing, d, \lfrac{\eps}{2d^2})$, given $\poly(n,2^d, 1/\eps, \log(1/\delta))$ random examples from a monotone distribution $\mcD$, runs in $\poly(n) \cdot (d/\eps)^{O(d)} \cdot \log(1/\delta)$ time and outputs a distribution within TV distance $\eps$ of $\mcD$. The algorithm fails with probability at most $\delta$.
\end{corollary}


\subsection{Beyond monotone distributions using subcube conditional sampling}
We also design an influence estimator for arbitrary distributions $\ArbDist$ over $\bits^m$ using subcube conditional sampling.

\begin{figure}[H]
  \captionsetup{width=.9\linewidth}
\begin{tcolorbox}[colback = white,arc=1mm, boxrule=0.25mm]
\vspace{3pt} 

$\InfEst(\ArbDist, i, \eps)$:  \vspace{6pt} \\
\textbf{Input:} A distribution $\ArbDist$ over $\bits^m$, coordinate $i \in [m]$, and bias parameter $\eps$. \\
\textbf{Output:} An estimate of $\Inf_i(f_{\ArbDist})$ that has bias at most $\eps$.% \vspace{4pt}

\begin{enumerate}
    \item Sample a random $\bx \sim \ArbDist$ and define the subcube
    \begin{equation*}
        S \coloneqq \{\bx\} \cup \{\bx^{\oplus i}\}
    \end{equation*}
    where $x^{\oplus i}$ is $x$ with the $i^{\text{th}}$ bit flipped.
    \item \label{step:est-p}Take $\left\lceil1/\eps^2\right\rceil$ independent samples from $\ArbDist$ conditioned on the output being in $S$, and let $p$ be the fraction of those samples that equal $\bx$.
    \item Output $ |p - (1 - p)|$.
\end{enumerate}
\end{tcolorbox}
\caption{Pseudocode for estimating the influence of a variable on a distribution's weighting function.}
\label{fig:inf-est}
\end{figure}

\begin{proposition}[{\sc InfEst} has low bias]
\label{prop:low-bias}
    For any distribution $\ArbDist$ over $\bits^m$, coordinate $i \in [m]$, and $\eps > 0$,
    \begin{equation*}
        \left|\Ex\left[\InfEst(\ArbDist, i, \eps)\right] - \Inf_i(f_{\ArbDist})\right| \leq \eps
    \end{equation*}
    where $\InfEst$ is as defined in \Cref{fig:inf-est}.
\end{proposition}

Before proving \Cref{prop:low-bias}, we note that it implies a high accuracy estimator.

\begin{corollary}[Estimating influences]
\label{cor:high-accuracy}
    For any $\eps, \delta > 0$, there is an efficient algorithm that given unknown distribution $\ArbDist$, computes an estimate of $\Inf_i(f_{\ArbDist})$ to accuracy $\pm \eps$ with probability at least $1 - \delta$ using $O(\log(1/\delta)/ \eps^4)$ subcube conditional samples from $\ArbDist$.
\end{corollary}
\begin{proof}
    The algorithm outputs the mean of $O(\log(1/\delta)/\eps^2)$ independent calls to $\InfEst(\ArbDist, i , \eps/2)$. Each call to $\InfEst(\ArbDist, i , \eps/2)$ gives an output bounded within $[0,1]$. By Hoeffing's inequality, if $\textbf{est}$ is the mean of $O(\log(1/\delta)/\eps^2)$ independent calls to $\InfEst(\ArbDist, i , \eps/2)$,
    \begin{equation*}
        \Pr\left[\left|\textbf{est} -  \Ex\left[\InfEst(\ArbDist, i, \eps/2)\right] \right| \geq \eps/2\right] \leq \delta.
    \end{equation*}
    The result then follows from triangle inequality and \Cref{prop:low-bias}.
\end{proof}

We prove that $\InfEst$ has low bias.
\begin{proof}[Proof of \Cref{prop:low-bias}]
    For each $x \in \bits^n$, let
    \begin{equation*}
        p(x) \coloneqq \frac{\ArbDist(x)}{\ArbDist(x) + \ArbDist(x^{\oplus i})}
    \end{equation*}
    be the relative weight of $x$ in the subcube containing $x$ and $x^{\oplus i}$. Then, we can rewrite the influence as: 
    \begin{align*}
        \Inf_{i}(f_{\ArbDist}) &= \sum_{x \in \bits^m} \frac{1}{2^m}\left| f_{\ArbDist}(x) - f_{\ArbDist}(x^{\sim i}) \right |\\ 
         &= \sum_{x \in \bits^m} \left| \ArbDist(x) - \ArbDist(x^{\sim i}) \right | \tag{$f_{\ArbDist}(x) = 2^m \ArbDist(x)$}\\
        &= \frac{1}{2} \cdot \sum_{x \in \bits^m}\left| \ArbDist(x) - \ArbDist(x^{\oplus i}) \right| \tag{$x^{\sim i} = x^{\oplus i}$ wp $\frac{1}{2}$, and otherwise $x^{\sim i} = x$} \\
        &= \frac{1}{2} \cdot \sum_{x \in \bits^m}\left(\ArbDist(x) + \ArbDist(x^{\oplus i})\right) \cdot\left| p(x) - p(x^{\oplus i}) \right|. \tag{definition of $p(x)$} 
    \end{align*}
    Then, using the fact that $p(x) = 1 - p(x^{\oplus i})$, and distributing the $(\ArbDist(x) + \ArbDist(x^{\oplus i}))$ term, we can write
    \begin{align*}
        \Inf_{i}(f_{\ArbDist}) &= \frac{1}{2} \sum_{x \in \bits^m}\ArbDist(x) \cdot \left|p(x) - (1 - p(x)) \right| + \ArbDist(x^{\oplus i}) \cdot \left|p(x^{\oplus i }) - (1 - p(x^{\oplus i})) \right| \\
        &= \sum_{x \in \bits^m}\ArbDist(x) \cdot \left|p(x) - (1 - p(x)) \right|
    \end{align*}
    where, in the last step, we used the fact that summing over $x \in \bits^m$ is equivalent to summing over $x^{\oplus i} \in \bits^m$. Let $\hat{p}(x)$ be the random variable for the estimate of $p(x)$ computed by $\InfEst$ step \ref{step:est-p}. We bound the bias of \InfEst.
    \begin{align*}
         \Big|\Ex\left[\InfEst(\ArbDist, i, \eps)\right] &- \Inf_i(f_{\ArbDist})\Big| \\
         &= \left|\Ex_{\bx \sim \ArbDist}\left[\left|2\hat{p}(\bx) - 1\right|\right] - \Ex_{\bx \sim \ArbDist}\left[\left|2p(\bx) - 1\right|\right] \right| \tag{$2p-1 = p - (1-p)$} \\
         &= \left|\Ex_{\bx \sim \ArbDist}\left[\left|2\hat{p}(\bx) - 1\right| - \left|2p(\bx) - 1\right|\right] \right| \tag{linearity of expectation} \\
         &\leq \Ex_{\bx \sim \ArbDist}\left[\big|\left|2\hat{p}(\bx) - 1\right| - \left|2p(\bx) - 1\right|\big|\right]  \tag{Jensen's inequality} \\
         &\leq 2\left|\Ex_{\bx \sim \ArbDist}\left[\left|\hat{p}(\bx) - p(\bx)\right|\right]\right|. \tag{$\big||a| - |b|\big| \leq |a - b|$}
    \end{align*}
    For any $x \in \bits^m$, $\hat{p}(x)$ is the average of $\left\lceil1/\eps^2\right\rceil$ random variables each which is $1$ with probability $p(x)$ and $0$ otherwise. As a result, $\Ex[\hat{p}(x)] = p(x)$, and $\Var[\hat{p}(x)] \leq \eps^2/4$. Applying Jensen's inequality, we conclude
    \begin{equation*}
        \left|\Ex\left[\InfEst(\ArbDist, i, \eps)\right] - \Inf_i(f_{\ArbDist})\right| \leq 2\sqrt{\Var[\hat{p}(x)]} \leq \eps. \qedhere 
    \end{equation*}
\end{proof}
\section{Lifting uniform distribution learners: Proof of~\Cref{thm:lift}}

In this section, we show how to lift algorithms that learn over the uniform distribution to algorithms that learn with respect to arbitrary distributions, where the sample complexity and runtime scale with the decision tree complexity of the distribution. We first define our goal formally.

\begin{definition}[Learning with respect to a class of distributions]
    \label{def:learn}
    For any concept class $\mathscr{C}$ of functions $f: \bits^n \to \zo$, $\eps, \delta > 0$, set of distributions $\mathscr{D}$ with support $\bits^n,$ and $m, d \in \N$, we say that an algorithm $\mcA$ $(\eps,\delta)$-learns $\mathscr{C}$ for distributions $\mathscr{D}$ using $m$ samples if the following holds: For any $\mathcal{D} \in \mathscr{D}$ and any $f^\star \in \mathscr{C}$, given $m$ iid samples of the form $(\bx, f^\star(\bx))$ where $\bx \sim \mathcal{D}$, $\mcA$ outputs a hypothesis $h$ satisfying
    \begin{equation*}
        \Prx_{\bx \sim \mathcal{D}}[f^\star(\bx) \neq h(\bx)] \leq \eps.
    \end{equation*}
    with probability at least $1 - \delta$.
    % Similarly, we say $\mcA$ $(\eps, \delta)$-learns $\mathscr{C}$ for distributions $\mcD$ using $m$ samples and $m'$ conditional subcube queries if, in addition to $m$ labeled samples, it also is able to request (unlabeled) subcube conditional samples from $D$, and its output has the same guarantee as above.
\end{definition}

Generally, we can think of $\delta$ as any fixed constant, as the success probability can always be boosted.
\begin{fact}[Boosting success probability]
\label{fact:boost}
    Given an algorithm $\mcA$ that $(\eps,\frac{1}{2})$-learns a concept $\mathscr{C}$ using $m$ samples, for any $\delta > 0$, we can construct an $\mcA'$ that $(1.1\eps,\delta)$-learns $\mathscr{C}$ using $m \cdot \poly(1/\eps, \log(1/\delta)) $ samples.
    % In \Cref{def:learn}, for any $\delta > 0$, the success probability can be boosted from $\frac{1}{2}$ to $1 - \delta$, while the desired accuracy is slightly decreased from $\eps$ to $1.1\eps$, with only an additional $\poly(1/\eps, \log(1/\delta))$ in the number of samples.
\end{fact}
\begin{proof}
    By repeating $\mcA$ $\log(1/\delta)$ times, we can guarantee that with probability at least $1 - \delta/2$, one of the returned hypotheses is $\eps$-close to $f^\star$. The accuracy of each of these hypothesis can be estimated to accuracy $\pm 0.05\eps$ using $O(\log(1/\delta)/\eps^2)$ random samples, and the most accurate one returned. With probability at least $1 - \delta$, that hypothesis will have at most $1.1\eps$ error.
\end{proof}


Our goal is to learn with respect to the class of all low-depth decision tree distributions. We use the following natural assumption on the concept class, which includes almost every concept class considered in the learning theory literature.

\begin{definition}[Closed under restriction]
    \label{def:closed-under-restriction}
    A concept class $\mathscr{C}$ of functions $f: \bits^n \to \zo$ is \emph{closed under restriction} if, for any $f \in \mathscr{C}$, $i \in [n]$, and $b \in \bits$, the restriction $f_{i = b}$ is also in $\mathscr{C}$.
\end{definition}

We will use \Cref{thm:decompose} to first decompose $\mcD$ into a mixture of nearly uniform distributions, and then run our learner on each of those distributions, as described in \Cref{fig:lift}.

\begin{figure}[h]

  \captionsetup{width=.9\linewidth}
\begin{tcolorbox}[colback = white,arc=1mm, boxrule=0.25mm]
\vspace{3pt} 

$\Lift(T, \mcA, S)$:  \vspace{6pt} \\
\textbf{Input:} A decision tree $T$, an algorithm for learning in the uniform distribution $\mcA$, and a random labeled sample $S$. \vspace{5pt} \\
\textbf{Output:} A hypothesis \vspace{4pt}

\ \ For each leaf $\ell \in T$ \{\vspace{-6pt}
\begin{enumerate}
    \item Let $S_\ell$ be the subset of points in $S$ that reach $\ell$.
    \item Create a set $S_{\ell}'$ consisting of points in $S_{\ell}$ but where all coordinates queried on the root-to-leaf path for $\ell$ are rerandomized independently (this makes the marginal over the input uniform).
    \item Use $\mcA$ to learn a hypothesis, $h_\ell$, with $S_{\ell}'$ as input.
\end{enumerate}
\vspace{-6pt}
\ \ \}\vspace{6pt}

\ \ Return the hypothesis that, when given an input $x$, first determines which leaf $\ell \in T$ that $x$ follows and then outputs $h_\ell(x)$.
% \begin{enumerate}
%     \item For each leaf $\ell \in T$, let $S_\ell$ be the subset of $S$ that reaches $\ell$. Use $\mcA$ to learn a hypothesis for $S_{\ell}$, denoted $h_{\ell}$.
%     \item Return the hypothesis that, when given an input $x$, first determines which leaf $\ell \in T$ that $x$ follows and then outputs $h_\ell(x)$.
% \end{enumerate}
\end{tcolorbox}
\caption{Pseudocode lifting a uniform distribution learner to one which succeeds on decision tree distributions. In this pseudocode, we assume that we have a decision tree representation which is close to the distribution, which can be accomplished using $\BuildDT$ in \Cref{fig:BuildDT}.}
\label{fig:lift}
\end{figure}

For our first result, we will assume that we already have a learner that succeeds on distributions that are sufficiently close to uniform.

\begin{definition}[Robust learners]
    For any concept class $\mathscr{C}$ of functions $f: \bits^n \to \zo$ and algorithm $\mcA$, we say that $\mcA$ $(\eps, \delta, c)$-\emph{robustly learns} $\mathscr{C}$ using $m$ samples under the uniform distribution if, for any $\eta > 0$ and the class of distributions
    \begin{equation*}
        \mathscr{D}_{\mathrm{TV, \eta}} \coloneqq \left\{\text{Distributions } \mathcal{D} \text{ over }\bits^n \text{ where } \TV(\mcU, \mathcal{D}) \leq \eta\right\},
    \end{equation*}
    $\mcA$ $(\eps + c\eta, \delta)$-learns $\mathscr{C}$ for the distributions in $\mathscr{D}_{\mathrm{TV, \eta}}$ using $m$ samples.
    % the following holds. For any distribution $\mcD$ satisfying $\TV(\mcU, \mcD) \leq \eta$ and any $f^\star \in \mathscr{C}$, given $m$ iid samples of the form $(\bx, f^\star(\bx))$ where $\bx \sim \mcD$, $\mcA$ returns some $h: \zo^n \to \zo$ that satisfies
    % \begin{equation*}
    %     \Pr_{\bx \sim \mcD}[h(\bx) \neq f^\star(\bx)] \leq \eps
    % \end{equation*}
    % with probability at least $1 - \delta$.
\end{definition}

The study of robust learners is part of a long and fruitful line of work. In particular, every learner that is robust to nasty noise \cite{BEK02} meets our definition of robust learners. 

Our result will also apply to learners that aren't explicitly robust. This is because \emph{every} learner is robust for $c = O(m)$.
\begin{proposition}
    \label{prop:auto-robust}
    For any concept class $\mathscr{C}$ and algorithm $\mcA$, is $\mcA$ $(\eps, \delta)$-learns $\mathscr{C}$ using $m$ samples under the uniform distribution, then $\mcA$ also $(\eps, \delta + \frac{1}{3}, 3m)$-robustly learns $\mathscr{C}$ using $m$ samples.
\end{proposition}
\begin{proof}
    Fix any $\eta > 0$.  Our goal is to show that $\mcA$ $(\eps + 3m \eta, \delta)$-learns $\mathscr{C}$ for distributions in $\mathscr{D}_{\TV, \eta}$. If $\eta \geq \frac{1}{3m}$, this is obviously true as any hypothesis has error $\leq 1$. We therefore need only consider $\eta < \frac{1}{3m}$. When $\mcA$ receives a sample from $\mcU^m$, it returns a hypothesis with error $\leq \eps$ with probability at least $1 - \eta$. Instead, $\mcA$ is receiving a sample from $\mcD^m$, where $\TV(\mcU, \mcD) < \frac{1}{3m}$. The success probability of any test given a sample from $\mcD^m$ rather than $\mcU^m$ can only differ by at most
    \[\TV(\mcU^m, \mcD^m) \leq m \cdot \TV(\mcU,\mcD) < \frac{1}{3}.\]
    Therefore, for $\eta < \frac{1}{3m}$, $\mcA$ succeeds wp at least $\delta + \frac{1}{3}$, as desired.
\end{proof}


We now state the main result of this section.

\begin{theorem}
    \label{thm:lift-given-DT}
    Choose any concept class $\mathscr{C}$ of functions $\bits^n \to \zo$ closed under restrictions, $\eps,\delta, c > 0$, $m,d \in \N$, and algorithm $\mcA$ that $(\eps, \delta/(2 \cdot 2^d), c)$-robustly learns $\mathscr{C}$ using $m$ samples under the uniform distribution. For any function $f^\star \in \mathscr{C}$, distribution $\mcD$ over $\bits^n$, depth-$d$ decision tree $T:\bits^n \to \R$ computing the PMF of a distribution $\mcD_T$ where
    \begin{equation*}
        \TV(\mcD, \mcD_T) \leq \frac{\eps}{c},
    \end{equation*}
     and sample size of
    \begin{equation*}
        M = m \cdot \poly\left(2^d, \frac{1}{\eps}, \log\left(\frac{1}{\delta}\right)\right).
    \end{equation*}
    Let $\bS$ a size-$M$ iid sample of labeled points $(\bx, f^\star(\bx))$ where $\bx \sim \mcD$. The output of $\Lift(T,\mcA, \bS)$ is $O(\eps)$-close to $f^\star$ w.r.t $\mcD$ with probability at least $1 - \delta$.
\end{theorem}
By \Cref{fact:boost}, an algorithm with constant failure probability could be transformed into one with the failure probability required by \Cref{thm:lift-given-DT} with only a $\poly(d, 1/\eps, \log(\delta))$ increase in the sample size. Therefore, would \Cref{thm:lift-given-DT} still holds when $\mcA$ $(\eps, \frac{1}{2}, c)$-robustly learns $\mathscr{C}$ if $\Lift$ applies \Cref{fact:boost} to boost the success probability of $\mcA$. Before proving \Cref{thm:lift-given-DT}, we show how it implies our main result.


\begin{theorem}[Lifting uniform-distribution learners, formal version of \Cref{thm:lift}]
\label{thm:lift-formal}
Choose any concept class $\mathscr{C}$ of functions $\bits^n \to \zo$ closed under restrictions, $\eps, c > 0$, $m,d \in \N$. If there is an efficient algorithm, $\mcA$, that $(\eps, \frac{1}{2})$-learns $\mathscr{C}$ using $m$ samples for the uniform distribution, then for $M = \poly(n) \cdot \left(\frac{dm}{\eps}\right)^{O(d)}$,
\begin{itemize}
    \item[$\circ$] There is an algorithm that $(\eps, \frac{1}{6})$-learns $\mathscr{C}$ using $M$ samples for monotone distributions representable by a depth-$d$ decision tree.
    \item[$\circ$] There is an algorithm that uses $M$ conditional subcube samples from $\mcD$ and $M$ random samples labeled by the target function that learns $\mathscr{C}$ to $\eps$-accuracy for arbitrary (not necessarily monotone) distributions representable by a depth-$d$ decision tree.
\end{itemize}
In both cases, the algorithm runs in time $\poly(n, M)$.
\end{theorem}
\begin{proof}[Proof of \Cref{thm:lift-formal} given \Cref{thm:lift-given-DT}]
    Using \Cref{thm:decompose}, we can learn the input distribution to TV-distance accuracy $\frac{\eps}{3m}$ with a decision tree hypothesis.  By \Cref{prop:auto-robust}, $\mcA$ $(\eps, \frac{1}{2}, 3m)$-robustly learns $\mathscr{C}$ for the uniform distribution, which can be boosted to failure probability $O(2^{-d})$ using \Cref{fact:boost}. Applying \Cref{thm:lift-given-DT} gives the desired result, with the desired runtime following from \Cref{prop:runtime-lift}.
\end{proof}

 The remainder of this section is devoted to the proof of \Cref{thm:lift-given-DT}. We'll use the following proposition.
% Before proving \Cref{thm:lift-given-DT}, a few remarks are in order. 

% \begin{enumerate}
%     \item We require that the original algorithm $\mcA$ have a failure low failure probability of $O(\delta 2^{-d})$. However, by \Cref{fact:boost}, an algorithm with constant failure probability can be transformed into one with such low failure probability with only a multiplicative $\poly(d, 1/\eps, \log(\delta))$ increase in the sample size.
%     \item Even when $\mcA$ is not explicitly robust, by applying \Cref{prop:auto-robust}, we are free to use $c = O(m)$. In this setting, we want to have $\TV(\mcD, \mcD_T) \leq O(\frac{\eps}{m})$.
% \end{enumerate}

% \begin{theorem}\gnote{will fill in details of this theorem after proof}
%     \label{thm:lift-robust-learners}
%     For any concept class $\mathscr{C}$, $\eps, \eta > 0$, and sample size $m \in \N$, suppose there is an algorithm $\mcA$ that $(m, \eps, \eta)$-robustly learns $\mathscr{C}$ under the uniform distribution. Then,
%     \begin{enumerate}
%         \item \emph{Monotone distributions}: For any $d \in \N$, there is an algorithm $\mcB$ which, for
%         \begin{equation*}
%             M = \poly \left(\left(\frac{d}{\eps \eta}\right)^d, m\right),
%         \end{equation*}
%         learns functions in $\mathscr{C}$ with respect to monotone decision tree distributions of depth at most $d$ using $M$ labeled examples.
%         \item \emph{Arbitrary distributions}:
%     \end{enumerate}
% \end{theorem}





\begin{proposition}
    \label{prop:each-leaf-many}
    For any distribution $\mcD$ over $\bits^n$, depth-$d$ decision tree $T$, $m \in \N$ and $p, \delta > 0$, as long as
    \begin{equation}
        \label{eq:M-each-leaf-many}
        M \geq O\left(\frac{d + m + \log(1/\delta)}{p}\right)
    \end{equation}
    for a random sample of $M$ points from $\mcD$, the probability there is a leaf $\ell \in T$ satisfying:
    \begin{enumerate}
        \item High weight: The probability a random sample from $\mcD$ reaches $\ell$ is at least $p$,
        \item Few samples: The number of points in the size-$M$ sample that reach $\ell$ is less than $m$
    \end{enumerate}
    is at most $\delta$.
\end{proposition}
\begin{proof}
    Fix a single leaf $\ell \in T$ with weight at least $p$. Then, the expected number of points that reach this leaf is $\mu \geq Mp$. As long as $\mu \geq 2m$, applying multiplicative Chernoff bounds,
    \begin{equation*}
        \Pr[\text{Fewer than $m$ points reach $\ell$}] \leq \exp\left(-\frac{\mu}{8} \right).
    \end{equation*}
    Union bounding over all leaves $2^d$, it is sufficient to choose an $M$ where
    \begin{equation*}
        2^d \exp\left(-\frac{\mu}{8} \right) \leq \delta.
    \end{equation*}
    This is satisfied for the $M$ from \Cref{eq:M-each-leaf-many}.
\end{proof}


We'll also use that if we have a decision tree $T$ that has learned the PMF to $\mcD$ to high accuracy, $\mcD$ restricted to the leaves of $T$ is, on average over the leaves, close to uniform.
\begin{fact}[Lemma B.4 of \cite{BLMT-boosting}]
    \label{prop:TV-distance-split}
    For any distribution $\mcD$ and decision tree $T$ computing the PMF for some distribution $\mcD_{T}$,
    \begin{equation*}
        \sum_{\text{leaves }\ell \in T} \Prx_{\bx \sim \mcD}[\bx\text{ reaches }\ell] \cdot \TV(\mcD_{\ell}, (\mcD_T)_\ell) \leq 2\cdot \TV(\mcD, \mcD_{T}).
    \end{equation*}
\end{fact}

% \begin{theorem}[]
%     Choose any concept class $\mathscr{C}$ of functions $\zo^n \to \zo$ closed under restrictions, $\eps,c > 0$ sample size $m \in \N$, and algorithm $\mcA$  that $(\eps, c)$-robustly learns $\mathscr{C}$ using $m$ samples under the uniform distribution. For any function $f^\star \in \mathscr{C}$, distribution $\mcD$ over $\zo^n$, decision tree $T:\zo^n \to \R$ computing the PMF of a distribution $\mcD_T$ where
%     \begin{equation*}
%         \TV(\mcD, \mcD_T) \leq \frac{\eps}{c},
%     \end{equation*}
%      and sample size of
%     \begin{equation*}
%         M = m \cdot \poly\left(2^d, \frac{1}{\eps}\right).
%     \end{equation*}
%     Let $\bS$ a size-$M$ iid sample of $(\bx, f^\star(\bx))$ where $\bx \sim \mcD$. The output of $\Lift(T,\mcA, \bS)$ is $O(\eps)$-close to $f^\star$ w.r.t $\mcD$ with high probability.
% \end{theorem}
\begin{proof}[Proof of \Cref{thm:lift-given-DT}]
    First, we split up the accuracy of $h = \Lift(T, \mcA, \bS)$ into the accuracy of the hypotheses $h_{\ell}$ learned at each leaf $\ell$: 
    \begin{equation*}
        \Prx_{\bx \sim \mcD}[h(\bx) \neq f^\star(\bx)] = \sum_{\ell \in T}\Prx_{\bx \sim \mcD}[\bx\text{ reaches }\ell] \cdot \Prx_{\bx \sim \mcD_\ell}[h_{\ell}(\bx) \neq f^\star(\bx)].
    \end{equation*}
    Each hypothesis $h_{\ell}$ is learned by running $\mcA$ on the sample $\bS_{\ell}'$. First, we argue that for all leaves $\ell$ with much of $\mcD$'s weight, will, whp, have $\geq m$ samples. Applying \Cref{prop:each-leaf-many} with probability at least $1 - \delta/2$, for all $\ell \in T$ with $\Prx_{\bx \sim \mcD}[\bx\text{ reaches }\ell] \geq  \frac{\eps}{2^d}$, the size of $\bS_{\ell}'$ is at least $m$. Conditioning on that being true, we have that,
    \begin{align*}
        \Prx_{\bx \sim \mcD}[h(\bx) \neq f^\star(\bx)] &\leq  \sum_{\ell \in T}\Prx_{\bx \sim \mcD}[\bx\text{ reaches }\ell] \cdot \Prx_{\bx \sim \mcD_\ell}[h_{\ell}(\bx) \neq f^\star(\bx) \mid |\bS_{\ell}'| \geq m] \\
        &\quad+  \sum_{\ell \in T}\Prx_{\bx \sim \mcD}[\bx\text{ reaches }\ell] \cdot \Ind[\Prx_{\bx \sim \mcD}[\bx\text{ reaches }\ell] \leq p] \\
        & \leq \sum_{\ell \in T}\Prx_{\bx \sim \mcD}[\bx\text{ reaches }\ell] \cdot \Prx_{\bx \sim \mcD_\ell}[h_{\ell}(\bx) \neq f^\star(\bx) \mid |\bS_{\ell}'| \geq m] + \eps,
    \end{align*}
    where the last line uses that $T$ has at most $2^d$ leaves and $p = \frac{\eps}{2^d}$.
    
    
    Each point in $\bS_{\ell}'$ is an iid sample with the input uniform over $\bits^n$ and labeled by the function $f^\star_{\ell}$. As $\mathscr{C}$ is closed under restrictions, $f^\star_{\ell} \in \mathscr{C}$. Therefore,
    \begin{equation*}
        \Prx_{\bS}\left[\left[\Prx_{\bx \sim \mcD_\ell}[h_{\ell}(\bx) \neq f^\star(\bx)\mid |\bS_{\ell}'| \geq m ]\right] \leq \eps +  c \cdot \TV(\mcD_{\ell}, \mcU)\right] \geq 1 - \frac{\delta}{2 \cdot 2^d}.
    \end{equation*}
    We union bound over all $2^d$ leaves $\ell$ and the earlier event that $| \bS_{\ell}| \geq m$ whenever $\Prx_{\bx \sim \mcD}[\bx\text{ reaches }\ell] \geq p$, we have with probability at least $1 - \delta$,
    \begin{align*}
        \Prx_{\bx \sim \mcD}[h(\bx) \neq f^\star(\bx)] & \leq \sum_{\ell \in T}\Prx_{\bx \sim \mcD}[\bx\text{ reaches }\ell]\cdot \left(2\eps + c \cdot \TV(\mcD_{\ell}, \mcU)\right)\\
        &= 2\eps + c\cdot \sum_{\ell \in T}\Prx_{\bx \sim \mcD}[\bx\text{ reaches }\ell]\cdot \TV(\mcD_{\ell}, \mcU)\\
        &= 2\eps + c\cdot \sum_{\ell \in T}\Prx_{\bx \sim \mcD}[\bx\text{ reaches }\ell]\cdot \TV(\mcD_{\ell}, (\mcD_T)_\ell)\\
        &\leq  2\eps + c\cdot 2 \cdot \TV(\mcD, \mcD_T) \tag{\Cref{prop:TV-distance-split}} \\
        & \leq 4\eps. \tag{$\TV(\mcD, \mcD_T) \leq \frac{\eps}{c}$}
    \end{align*}
    As desired, we have $\Lift$ learns an $O(\eps)$-accurate hypothesis w.h.p.
\end{proof}



\begin{proposition}[Runtime of \Lift]
    \label{prop:runtime-lift}
    Assuming unit-time calls to $\mathcal{A}$, given a depth-$d$ decision tree $T$, $\Lift(T, \mathcal{A}, S)$ runs in time $O(n2^d \cdot |S|)$.
\end{proposition}
\begin{proof}
    The number of leaves in $T$ is at most $2^d$. Since each point in $S$ is in $\bits^n$, the entire sample takes $O(n \cdot |S|)$ bits to represent. For each leaf $\ell$, it takes $O(n \cdot |S|)$ time to loop through this representation, find the points consistent with $\ell$, and create the modified dataset $S_\ell'$, and pass it into $\mcA$. Repeating this over all leaves can be done in $O(n2^d \cdot |S|)$ time.
\end{proof}


\begin{remark}[The agnostic setting]
\label{remark:agnostic}
    A popular variant of learning, as defined in \Cref{def:learn}, is the \emph{agnostic setting}. In this generalization of standard learning, rather than assume $f^\star \in \mathscr{C}$, $\mcA$ is required to output a hypothesis with error $\opt + \eps$, where $\opt$ is the minimum error of a hypothesis in $\mathscr{C}$ w.r.t $f^\star$. It's straightforward to see that \Cref{thm:lift-given-DT}, and therefore \cref{thm:lift-formal}, extend to the agnostic setting, where if the uniform distribution learning $\mcA$ succeeds in the agnostic setting, that learner is upgraded to one that succeeds for decision tree distributions in the agnostic setting. This is because, if there is an $f \in \mathscr{C}$ with error $\opt$ w.r.t. $f^\star$, then the average error of $f$ over the leaves of a tree $T$ w.r.t $f^\star$ will also be at most $\opt$.
\end{remark}

\section*{Acknowledgements}

We thank the STOC reviewers for their detailed feedback, especially for the references to the literature on semi-supervised learning. 

Guy and Li-Yang are supported by NSF awards 1942123, 2211237, and 2224246. Jane is
supported by NSF Award CCF-2006664. Ali is supported by a graduate fellowship award from Knight-Hennessy Scholars at Stanford University.


\bibliographystyle{alpha}
\bibliography{ref}


\end{document}