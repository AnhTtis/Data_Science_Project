\documentclass[lettersize,journal]{IEEEtran}

\ifdefined\siggraph
\usepackage{times}
\fi

%\usepackage{parskip}
\usepackage{color}
\usepackage{ifthen}
\usepackage{float}
\usepackage{alltt}
\usepackage{newlfont} % for Box
\usepackage{array}
\usepackage{wrapfig}
\usepackage{booktabs}
\usepackage{multirow}
\usepackage{amsfonts}
\usepackage{dsfont}
\usepackage[linesnumbered,ruled,vlined]{algorithm2e}
 
%%% Coloring the comment as blue
\newcommand\mycommfont[1]{\footnotesize\ttfamily\textcolor{blue}{#1}}
\SetCommentSty{mycommfont}
 


\renewcommand{\nu}{\ensuremath{\mathbf{n}(\mathbf{u})}\xspace}  % the normal vector at pixel location \V{u}
\newcommand{\pu}{\ensuremath{\mathbf{p}(\mathbf{u})}\xspace}   % the 3d point correspoinding the pixel \V{u}
\newcommand{\du}{\ensuremath{d(\mathbf{u})}\xspace}  
\newcommand{\zu}{\ensuremath{z(\mathbf{u})}\xspace}
\newcommand{\eu}{\ensuremath{\mathbf{e}(\mathbf{u})}\xspace}
\newcommand{\up}{\ensuremath{\V{u}_{\V{p}}}\xspace}
\newcommand{\tup}{\ensuremath{\tilde{\V{u}}_{\V{p}}}\xspace}

\newcommand{\oz}{\ensuremath{\Omega_z}\xspace}  
\newcommand{\on}{\ensuremath{\Omega_n}\xspace}
\newcommand{\Nu}{\ensuremath{\mathcal{N}(\V{u})}\xspace}

\renewcommand{\ni}{normal integration\xspace}
\newcommand{\NI}{Normal Integration\xspace}
\newcommand{\dpe}{discrete Poisson's equation\xspace}
\newcommand{\Dpe}{Discrete Poisson's equation\xspace}


\newcommand{\z}{\ensuremath{\V{z}}\xspace}
\newcommand{\zs}{\ensuremath{\V{z}^*}\xspace}
\newcommand{\rz}{\ensuremath{\red{\V{z}}}\xspace}
\newcommand{\zt}{\ensuremath{\V{z}_{t}}\xspace}
\newcommand{\zto}{\ensuremath{\V{z}_{t+1}}\xspace}
\newcommand{\R}{\ensuremath{\mathbb{R}}\xspace}
\newcommand{\fz}{\ensuremath{f(\V{z})}\xspace}

\newcommand{\rt}{\ensuremath{\V{r}_{t}}\xspace}
\newcommand{\rto}{\ensuremath{\V{r}_{t+1}}\xspace}


\newcommand{\dup}{\ensuremath{\V{D}_u^{+}}\xspace}
\newcommand{\dun}{\ensuremath{\V{D}_u^{-}}\xspace}
\newcommand{\dvp}{\ensuremath{\V{D}_v^{+}}\xspace}
\newcommand{\dvn}{\ensuremath{\V{D}_v^{-}}\xspace}
\newcommand{\nx}{\ensuremath{\V{n}_x}\xspace}
\newcommand{\ny}{\ensuremath{\V{n}_y}\xspace}
\newcommand{\nz}{\ensuremath{\V{n}_z}\xspace}
\newcommand{\Nz}{\ensuremath{\V{N}_z}\xspace}

\newcommand{\ft}{\ensuremath{F(\red{\V{z}};\V{z}_t)}\xspace}
\newcommand{\ftt}{\ensuremath{F(\V{z}_t;\V{z}_t)}\xspace}
\newcommand{\fto}{\ensuremath{F(\V{z}_{t+1};\V{z}_t)}\xspace}

\newcommand{\dpu}{\ensuremath{\partial_u \V{p}}\xspace}
\newcommand{\dpv}{\ensuremath{\partial_v \V{p}}\xspace}

\renewcommand{\u}{\ensuremath{\V{u}}\xspace}
\newcommand{\dzdu}{\ensuremath{\partial_u z}\xspace}
\newcommand{\dzdv}{\ensuremath{\partial_v z}\xspace}
\newcommand{\dztdu}{\ensuremath{\partial_u \tilde{z}}\xspace}
\newcommand{\dztdv}{\ensuremath{\partial_v \tilde{z}}\xspace}
\newcommand{\dzpdu}{\ensuremath{\partial_{u}^{+} z}\xspace}
\newcommand{\dzpdv}{\ensuremath{\partial_{v}^{+} z}\xspace}
\newcommand{\dzndu}{\ensuremath{\partial_{u}^{-} z}\xspace}
\newcommand{\dzndv}{\ensuremath{\partial_{v}^{-} z}\xspace}

\newcommand{\dzpduv}{\ensuremath{\partial_{\{u,v\}}^{+} z}\xspace}
\newcommand{\dznduv}{\ensuremath{\partial_{\{u,v\}}^{-} z}\xspace}
\newcommand{\dzduv}{\ensuremath{\partial_{\{u,v\}} z}\xspace}

\newcommand{\dupz}{\ensuremath{\Delta_{u}^{+} z}\xspace}
\newcommand{\dunz}{\ensuremath{\Delta_{u}^{-} z}\xspace}
\newcommand{\dvpz}{\ensuremath{\Delta_{v}^{+} z}\xspace}
\newcommand{\dvnz}{\ensuremath{\Delta_{v}^{-} z}\xspace}

\newcommand{\nuv}{\ensuremath{\V{n}(u,v)}\xspace}
\newcommand{\zuv}{\ensuremath{z(u,v)}\xspace}
\newcommand{\puv}{\ensuremath{\V{p}(u,v)}\xspace}

\newcommand{\halfpi}{\ensuremath{\pm {\pi \over 2}}\xspace}


\newcommand{\curve}{\ensuremath{\mathbb{S}}\xspace}
\newcommand{\zenith}{zenith\xspace}
\newcommand{\surface}{\ensuremath{\mathcal{M}}\xspace}
\newcommand{\visibility}{\ensuremath{\Phi_{i}}\xspace}
\newcommand{\point}{\ensuremath{\V{x}}\xspace}
\newcommand{\normal}{\ensuremath{\V{n}}\xspace}
\newcommand{\tangent}{\ensuremath{\V{t}}\xspace}
\newcommand{\cameraNum}{\ensuremath{C}\xspace}
\newcommand{\cameraCenter}{\ensuremath{\V{o}_{i}}\xspace}
\newcommand{\viewDirection}{\ensuremath{\V{v}}\xspace}
\newcommand{\batchsize}{\ensuremath{P}\xspace}
\newcommand{\mask}{\ensuremath{O}\xspace}
\newcommand{\projectedTangentVector}{projected tangent vector\xspace}
\newcommand{\projectedTangentVectors}{projected tangent vectors\xspace}
\newcommand{\stackedTangentVectors}{\ensuremath{\V{T}(\point)}\xspace}
\newcommand{\diligentmv}{\mbox{DiLiGenT-MV}\xspace}
\newcommand{\diligent}{DiLiGenT}
\newcommand{\loss}{\mathcal{L}\xspace}
\newcommand{\opticalAxis}{\ensuremath{\V{e}_{z}\xspace}}
\newcommand{\opticalAxisViewI}{\ensuremath{\V{e}_{z_{i}}}\xspace}
\newcommand{\opticalAxisMatrix}{\ensuremath{\V{C}}\xspace}
\newcommand{\ms}{Mumford-Shah integrator\xspace}
\newcommand{\made}{MADE\xspace}

\newcommand{\pandora}{\mbox{PANDORA}\xspace}
\newcommand{\psnerf}{\mbox{PS-NeRF}\xspace}
\newcommand{\sdps}{\mbox{SDPS}\xspace}
\newcommand{\uanet}{\mbox{UA-MVPS}\xspace}
\newcommand{\rmvps}{\mbox{R-MVPS}\xspace}
\newcommand{\bmvps}{\mbox{B-MVPS}\xspace}
\newcommand{\volsdf}{\mbox{VolSDF}\xspace}
\newcommand{\unisurf}{\mbox{UNISURF}\xspace}


\newcommand{\mvas}{MVAS\xspace}

\newcommand{\tsc}{\mbox{TSC}\xspace}

\newcommand{\pointOne}{\ensuremath{\point_1}\xspace}
\newcommand{\pointTwo}{\ensuremath{\point_2}\xspace}
\newcommand{\pointsetOne}{\ensuremath{\chi_{1}}\xspace}
\newcommand{\pointsetTwo}{\ensuremath{\chi_{2}}\xspace}
\newcommand{\fscoreThreshold}{\ensuremath{\tau}\xspace}
\newcommand{\chamferDist}{\ensuremath{d(\pointsetOne, \pointsetTwo)}\xspace}
\newcommand{\precision}{\ensuremath{\mathcal{P}}\xspace}
\newcommand{\recall}{\ensuremath{\mathcal{R}}\xspace}
\newcommand{\fscore}{\ensuremath{\mathcal{F}}\xspace}

\newcommand{\phaseangle}{\ensuremath{\hat{\phi}}\xspace}
\newcommand{\azimuthangle}{\ensuremath{\phi}\xspace}

\newcommand{\colorbar}[3]{
\begin{tabular}[t]{@{}l@{}l@{}}
	\includegraphics[height=#1\linewidth,width=0.5em]{colorbar.pdf} & 
	\begin{tabular}[b]{@{}l}
		#2 \\ [#3pt]
		$0$
	\end{tabular}
\end{tabular}
}


% \usepackage{amsmath,amsfonts}
% \usepackage{algorithmic}
% \usepackage{algorithm}
% \usepackage{array}
% \usepackage[caption=false,font=normalsize,labelfont=sf,textfont=sf]{subfig}
% \usepackage{textcomp}
% \usepackage{stfloats}
% \usepackage{url}
% \usepackage{verbatim}
% \usepackage{graphicx}
% \usepackage{cite}
\hyphenation{op-tical net-works semi-conduc-tor IEEE-Xplore}
% updated with editorial comments 8/9/2021

\begin{document}

%\title{Multi-teacher Knowledge Distillation for End-to-End Speech Recognition}
%\title{Two Teachers are Better Than One:  Multi-foundation Model Knowledge Distillation for End-to-End ASR}
\title{Knowledge Distillation from Multiple Foundation Models for End-to-End Speech Recognition}

%\author{IEEE Publication Technology,~\IEEEmembership{Staff,~IEEE,}
\author{Xiaoyu Yang,~\IEEEmembership{Student Member,~IEEE,}, Qiujia Li,~\IEEEmembership{Member,~IEEE,}\\Chao Zhang,~\IEEEmembership{Member,~IEEE,} Philip C. Woodland,~\IEEEmembership{Fellow,~IEEE}


        % <-this % stops a space
\thanks{This paper was produced by the IEEE Publication Technology Group. They are in Piscataway, NJ.}}% <-this % stops a space
%\thanks{Manuscript received April 19, 2021; revised August 16, 2021.}}

% The paper headers
\markboth{SUBMITTED TO IEEE/ACM TRANSACTIONS ON AUDIO SPEECH AND LANGUAGE PROCESSING}%
{Shell \MakeLowercase{\textit{et al.}}: A Sample Article Using IEEEtran.cls for IEEE Journals}

% \IEEEpubid{0000--0000/00\$00.00~\copyright~2021 IEEE}
% Remember, if you use this you must call \IEEEpubidadjcol in the second
% column for its text to clear the IEEEpubid mark.

\maketitle
\begin{abstract}
As models continue to grow in size, the development of memory optimization methods (MOMs) has emerged as a solution to address the memory bottleneck encountered when training large models. To comprehensively examine the practical value of various MOMs, we have conducted a thorough analysis of existing literature from a systems perspective. 
% Furthermore, we have evaluated the most widely adopted MOMs employed in mainstream frameworks for both vision and language models.
Our analysis has revealed a notable challenge within the research community: the absence of standardized metrics for effectively evaluating the efficacy of MOMs. The scarcity of informative evaluation metrics hinders the ability of researchers and practitioners to compare and benchmark different approaches reliably. Consequently, drawing definitive conclusions and making informed decisions regarding the selection and application of MOMs becomes a challenging endeavor.
To address the challenge, this paper summarizes the scenarios in which MOMs prove advantageous for model training. We propose the use of distinct evaluation metrics under different scenarios. By employing these metrics, we evaluate the prevailing MOMs and find that their benefits are not universal. We present insights derived from experiments and discuss the circumstances in which they can be advantageous.

\end{abstract}
\section{Introduction}
\IEEEPARstart{T}{he} method Neural Radiance Fields (NeRF)~\cite{mildenhall2020nerf} is proposed for photorealistic novel view synthesis. Given many views of the scene, it creates implicit multi-view geometry and learns for view synthesis. However, it has poor generalizations to new scenes and requires retraining or fine-tuning on each scene. 
 
 Recent work~\cite{Yu_2021_CVPR,Trevithick_2021_ICCV} has explored the ways of using a single image to train NeRF. They introduce a convolutional feature encoder to learn the image representation which gives it some limited generalization abilities to unseen scenes.  But, without fine-tuning, these methods produce many floats and artifacts in rendering novel views. 
 
  Multi-Plane Images (MPI) representation that learns multiple RGB images from a single image is also used in \cite{Wu_2021_ICCV,Tucker_2020_CVPR,wu2022remote} for  novel view synthesis. However, MPI heavily relies on the qualities of the planar images and needs plenty of image planes to avoid blurs. There is no strong 3D geometry constraint and it fails in many complex scenes.
  
  MINE~\cite{Li_2021_ICCV2} introduces the volume rendering of NeRF into the MPI. It runs faster and produces better depth rendering quality compared with single-view NeRFs~\cite{Yu_2021_CVPR,Trevithick_2021_ICCV}. However, the rendering quality heavily relies on the number of image planes. It needs high-resolution 4D volumes to store the 4-channel  (RGB and volume density) image planes that cost a large amount of GPU memory in both training and 
 prediction.  
 

 
 \begin{figure}[t]
\setlength{\abovecaptionskip}{7pt}
\setlength{\belowcaptionskip}{0pt}
	\centering
% 	\subfigure[MINE (PSNR:14.9)]{  % for AAAI
	\subfloat[MINE (PSNR:14.9)]{
%			\centering
			\includegraphics[width=0.23\textwidth]{figure/intro/DJI_20200223_163206_598_0_MINE.png}
%			\label{subfig:pixelnerf}
	}\subfloat[MINE (depth)]{
%			\centering
			\includegraphics[width=0.23\textwidth]{figure/intro/MINE_disp.png}
%			\label{subfig:mpi}
	}
	\\[-3mm]
	\subfloat[Ours (PSNR:17.0)]{
%			\centering
			\includegraphics[width=0.23\textwidth]{figure/intro/DJI_20200223_163206_598_0_ours.png} 
	}\subfloat[Ours (depth)]{
%			\centering
			\includegraphics[width=0.23\textwidth]{figure/intro/ours_disp.png}
	}
	\caption{Comparison with state-of-the-art methods. (a-b) RGB and depth rendering results of  \cite{Li_2021_ICCV2}. It produces many blurs and floats in the occluded regions and at the object/depth edges. 
	(c-d) Our method employs a joint rendering mechanism that preserves more image details and predicts sharp depth edges.}
	\label{fig:performance_illustration}
\end{figure}
 
 In this paper, we propose a joint rendering mechanism that takes the MPI strategy for coarse sampling proposals and the MLP\&volume-based rendering~\cite{mildenhall2020nerf} for fine sampling and rendering. Then, both the coarse point samples and the fine samples are combined according to their geometry distribution to realize a more accurate joint rendering. More importantly, we introduce a depth teacher net that serves as the guidance for the joint rendering. The monocular depth teacher predicts dense pseudo depth maps that assist the consistent 3D geometry learning between the MPI, the fine volume, and the joint rendering. It also boosts the multi-view geometry consistency between the source view and the target novel views that 
helps handle the occlusions, reduce the blurs and floats, and render accurate depths. 
 
In the experiments,  we verify the effectiveness of our method on three challenging real-scene datasets (RealEstate10K~\cite{zhou2018stereo}, NYU~\cite{silberman2012indoor} and  NeRF-LLFF~\cite{mildenhall2020nerf}) for novel view synthesis or depth estimation. Given a single image as input, our method is shown able to produce higher qualities in both the RGB image rendering and depth map prediction. It far outperforms state-of-the-art methods~\cite{Li_2021_ICCV2,Yu_2021_CVPR} with improvements of 5$\sim$20\% in PSNR and SSIM for the RGB rendering and reduces 20$\sim$50\% of the errors for the depth prediction.
\vspace{-4mm}
\section{Related Works}
\noindent\textbf{Sign Language Recognition.} Sign language recognition (SLR) is a fundamental task in the field of sign language understanding.
Feature extraction plays a key role in an SLR model.
% 
Most recent SLR works \cite{jiang2021sign, jiang2021skeleton, hu2021signbert, hu2021hand, li2020transferring, li2020word, joze2019ms, hu2021global, stmc, zuo22_interspeech, vac} adopt CNN-based architectures, \eg, I3D \cite{I3D} and R3D \cite{qiu2017learning}, to extract vision features from RGB videos.
In this work, we adopt S3D \cite{xie2018rethinking} as the backbone of our VKNet due to its excellent accuracy-speed trade-off.

However, RGB-based SLR models may suffer from the large variation of video backgrounds. 
As a complement, some SLR works \cite{jiang2021skeleton, jiang2021sign, hu2021hand, hu2021signbert, chentwo} explore to jointly model RGB videos and keypoints.
For example, SAM-SLR \cite{jiang2021skeleton} uses graph convolutional networks (GCNs) to model pre-extracted keypoints.
HMA \cite{hu2021hand} and SignBERT \cite{hu2021signbert} propose to decode 3D hand keypoints from RGB videos.
A common deficiency of these works is that they need a dedicated network to model keypoints.
In this work, we represent keypoints as a sequence of heatmaps~\cite{duan2022revisiting, chentwo} so that the keypoint encoder of our VKNet can share the identical architecture with the video encoder.

To enable mini-batch training, previous works \cite{jiang2021sign, jiang2021skeleton, hu2021signbert, hu2021hand, li2020transferring, li2020word} crop fixed-length clips from raw videos as model inputs.
However, the model may overfit to the training videos of fixed temporal receptive fields.
In contrast, our VKNet is trained on videos with varied temporal receptive fields to improve its generalization capability.



\noindent\textbf{Word Representation Learning.}
Word2vec \cite{word2vec} and GloVe \cite{glove} are two classical word representation learning frameworks in the field of NLP.
Based on word2vec, fastText \cite{mikolov2018advances} improves word representations with several modifications including the use of sub-word information \cite{bojanowski2017enriching} and position independent features \cite{mnih2013learning}.
Although some advanced language models, \eg, BERT \cite{kenton2019bert}, can also be used to extract word representations, they are computationally intensive and are not dedicated to word representation learning.
In this paper, we adopt the lightweight but effective fastText, which is also used in a recent sign language translation work \cite{yin2021simulslt}, to pre-compute gloss (word) representations.


\noindent\textbf{Vision-Language Models.}
Recently, a majority of vision-language models \cite{clip, align, yao2022filip, gu2022wukong} learn visual representations on large-scale image-text pairs.
Among them, CLIP \cite{clip} is the pioneer to jointly optimize an image encoder and a text encoder through a contrastive loss. 
% 
Besides, the pre-trained CLIP can be generalized to various downstream tasks, \eg, semantic segmentation \cite{xu2022groupvit, li2021language, xu2021simple}, object detection \cite{du2022learning, rao2022denseclip}, image classification~\cite{zhou2022learning,huang2022unsupervised}, and style transfer \cite{patashnik2021styleclip, kwon2022clipstyler}.
In this work, we exploit the implicit knowledge included in glosses (sign labels), which is distinct from previous works on vision-language modeling.


\noindent\textbf{Multi-label Classification.} Real-world objects may have multiple semantic meanings, which motivates research on multi-label classification \cite{ridnik2021asymmetric, ke2022hyperspherical, zhang2013review, rajeswar2022multi, kim2022large} requiring models to map inputs to multiple possible labels.
Although the VISigns may be associated with the multi-label classification problem, most widely-adopted SLR datasets \cite{li2020word, joze2019ms, hu2021global} are singly labeled.
In this work, we deal with the VISigns by incorporating language information included in glosses.

\vspace{-0.3em}
\section{Method}
\vspace{-0.3em}

Our sensitivity-aware visual parameter-efficient fine-tuning consists of two stages. In the first stage, SPT measures the task-specific sensitivity for the pre-trained parameters (Section~\ref{subsec:sensitivity}). Based on the parameter sensitivity and a given parameter budget, SPT then adaptively allocates trainable parameters to task-specific important positions (Section~\ref{subsec:SPT}).

\vspace{-0.3em}
\subsection{Task-specific Parameter Sensitivity}
\label{subsec:sensitivity}
\vspace{-0.3em}

Recent research has observed that pre-trained backbone parameters exhibit varying feature patterns~\cite{raghu2021vision,naseer2021intriguing} and criticality~\cite{zhang2019all,chatterji2019intriguing} at distinct positions. 
Moreover, when transferred to downstream tasks, their efficacy varies depending on how much pre-trained features are reused and how well they adapt to the specific domain gap~\cite{yosinski2014transferable,kumar2022finetuning,neyshabur2020being}. Motivated by these observations, we argue that not all parameters contribute equally to the performance across different tasks in PEFT and propose a new criterion to measure the sensitivity of the parameters in the pre-trained backbone for a given task.

Specifically, given the training dataset $\gD_t$ for the $t$-th task and the pre-trained model weights $\vw=\left\{w_1, w_2, \ldots, w_N\right\}\in \sR^N$ where $N$ is the total number of parameters, the objective for the task is to minimize the empirical risk: $\min_{\vw} E(\gD_t, \vw)$.
We denote the parameter sensitivity \bohan{set} as $\gS=\{s_1, \ldots, s_N\}$ and the sensitivity $s_n$ for parameter $w_n$ is measured by the empirical risk difference when tuning it:
\begin{equation}
\vspace{-0.3em}
    \begin{aligned}
        s_n = E(\gD_t, \vw)-E(\gD_t, \vw\mid w_n=w_n^*),
    \end{aligned}
\label{eq:sensitivity}
\end{equation}
where $w_n^*=\underset{w_n}{\rm argmin}(E(\gD_t, \vw))$. We can reparameterize the tuned parameters as  $w_n^*=w_n+\Delta_{w_n}$, where $\Delta_{w_n}$ denotes the update for $w_n$ after tuning. Here we individually measure the sensitivity of each parameter, which is reasonable given that most of the parameters are frozen during fine-tuning in PEFT. However, it is still computationally intensive to compute Eq.~(\ref{eq:sensitivity}) for two reasons. Firstly, getting the empirical risk for $N$ parameters requires forwarding the entire network $N$ times, which is time-consuming. Secondly, it is challenging to derive $\Delta_{w_n}$, as we have to tune each individual $w_n$ until convergence.

{\begin{algorithm}[t!]
\caption{\label{alg:tps} Computing task-specific parameter sensitivities}
\begin{algorithmic}
    \STATE \textbf{Input:} Pre-trained model with network parameters $\vw$, training set $\gD_t$ for the $t$-th task, and number of training samples $C$ used to calculate the parameter sensitivities
    \STATE \textbf{Output:} Sensitivity set $\gS=\{s_1, \ldots, s_N\}$
    \STATE Initialize $\gS=\{0\}^N$
    \FOR{$i\in\{1,\ldots,C\}$}
        \STATE Get the $i$-th training sample of $\gD_t$
	    \STATE Compute loss $E$
		\STATE Compute gradients $\vg$
		\FOR{$n\in\{1,\ldots,N\}$}
                \STATE Update sensitivity for the $n$-th parameter: $s_{n} = s_{n} + g_n^2$
		    \ENDFOR
    \ENDFOR
\end{algorithmic}
\end{algorithm}}


\begin{figure*}[t]
\begin{center}
    \includegraphics[width=\linewidth]{main_figure.pdf}
\end{center}\vspace{-2em}
\caption{Overview of our trainable parameter allocation strategy. With the parameter sensitivity \bohan{set} $\gS$, we first get the top-$\tau$ sensitive parameters. Instead of directly tuning these sensitive parameters, we also boost the representational capability by replacing unstructured tuning with structured tuning at sensitive weight matrices that have a large number of sensitive parameters, which can be implemented by an existing structured tuning method, \eg, LoRA~\cite{hu2022lora} and Adapter~\cite{houlsby2019parameter}. Red lines and blocks represent trainable parameters and modules, while blue lines represent frozen parameters.}
\label{fig:main}
\vspace{-1.5em}
\end{figure*}


To overcome the first barrier, we simplify the empirical loss by approximating $s_n$ in the vicinity of $\vw$ by its first-order Taylor expansion
\vspace{-0.3em}
\begin{equation}
\vspace{-0.5em}
    \begin{aligned}
        s_n^{(1)} = -g_n\Delta_{w_n},
    \end{aligned}
\label{eq:first-order}
\end{equation}
where the gradients $\vg=\partial E/\partial\vw$, and $g_n$ is the gradient of the $n$-th element of $\vg$. 
To address the second barrier, following~\cite{liu2018darts,cai2018proxylessnas}, we take the one-step unrolled weight as the surrogate for $w_n^*$ and approximate $\Delta_{w_n}$ in Eq.~(\ref{eq:first-order}) with a single step of gradient descent. We can accordingly get $s_n^{(1)} \approx g_n^2\epsilon$,
where $\epsilon$ is the learning rate. Since $\epsilon$ is the same for all parameters, we can eliminate it when comparing the sensitivity with the other parameters and finally get 
\vspace{-0.5em}
\begin{equation}
\vspace{-0.3em}
    \begin{aligned}
        s_n^{(1)} \approx g_n^2.
    \end{aligned}
\label{eq:first-order-simp}
\end{equation}
Therefore, the sensitivity of a parameter can be efficiently measured by its potential to reduce the loss on the target domain. Note that although our criterion draws inspiration from pruning work~\cite{molchanov2019importance}, it is distinct from it. \cite{molchanov2019importance} measures the parameter importance by the squared change in loss when removing them, \ie, $\left( E(\gD_t, \vw)-E(\gD_t, \vw\mid w_n=0) \right)^2$ and finally derives the parameter importance by $\left( g_n w_n \right)^2$, which is different from our formulations in Eqs.~(\ref{eq:sensitivity}) and~(\ref{eq:first-order-simp}).

In practice, we accumulate $\gS$ from a total number of $C$ training samples ahead of fine-tuning to generate accurate sensitivity as shown in Algorithm~\ref{alg:tps}, where $C$ is a pre-defined hyper-parameter. In Section~\ref{subsec:abl}, we show that employing only 400 training samples is sufficient for getting reasonable parameter sensitivity, which requires only 5.5 seconds with a single GPU for any VTAB-1k dataset with ViT-B/16 backbone~\cite{vit}.

\vspace{-0.3em}
\subsection{Adaptive Trainable Parameters Allocation}
\label{subsec:SPT}
\vspace{-0.2em}

Our next step is to allocate trainable parameters based on the obtained parameter sensitivity set $\gS$ and a desired parameter budget $\tau$. A straightforward solution is to directly tune the top-$\tau$ most sensitive unstructured connections (parameters) \rev{while keeping the rest frozen}, which we name unstructured tuning. Specifically, we select the top-$\tau$ most sensitive weight connections in $\gS$ to form the sensitive weight connection set $\gT$. Then, for \rev{a} weight matrix $\mW\in \sR^{d_{\rm in}\times d_{\rm out}}$, we can get a binary mask $\mM\in \sR^{d_{\rm in}\times d_{\rm out}}$ computed by
\vspace{-0.5em}
\begin{equation}
\vspace{-0.5em}
    {\begin{array}{ll}
    \small
    \begin{aligned}
    \mM^j =
    \left\{\begin{array}{ll} 
    1 ~~~~~ \mW^j \in \gT \\
    0 ~~~~~ \mW^j \notin \gT
    \end{array}\right.
    \end{aligned},
    \small
    \end{array}}
\label{eq:mask}
\end{equation}
where $\mW^j$ and $\mM^j$ are the $j$-th element in $\mW$ and $\mM$, respectively. Accordingly, we can train the sensitive parameters by gradient descent and the updated weight matrix can be formulated as $\mW'\leftarrow \mW - \epsilon\vg_{\mW}\odot\mM$, where $\vg_{\mW}$ is the gradient for $\mW$.

However, considering PEFT approaches generally limit the proportion of trainable parameters to less than 1\%, tuning only a small number of unstructured weight connections might not have enough representational capability to handle the downstream datasets with large domain gaps from the source pre-training data. Therefore, to improve the representational capability, we propose to replace unstructured tuning with structured tuning at the sensitive weight matrices that have a high number of sensitive parameters. To preserve the parameter budget, we can implement structured tuning with an existing efficient structured tuning PEFT method~\cite{hu2022lora,chen2022adaptformer,houlsby2019parameter,jie2022convolutional} that learns to directly adjust \rev{all hidden dimensions at once}. We depict an overview of our trainable parameter allocation strategy in Figure~\ref{fig:main}. For example, we can employ the low-rank reparameterization trick LoRA~\cite{hu2022lora} to the sensitive weight matrices \rev{and the one-step update for $\mW$ can be formulated as}
\vspace{-0.4em}
\begin{equation}
\vspace{-0.4em}
    {\begin{array}{ll}
    \small
    \begin{aligned}
    \mW' = \left\{\begin{array}{ll} 
    \mW + \mW_{\rm down}\mW_{\rm up} & ~~ \text { if } ~~ \sum_{j=0}^{d_{\rm in}\times d_{\rm out}} \mM^j \geq \sigma_{\rm opt} \\
    \mW - \epsilon\vg_{\mW}\odot\mM & ~~ {\rm otherwise}
    \end{array}\right.
    \end{aligned},
    \small
    \end{array}}
\label{eq:weight_updat}
\end{equation}
where $\mW_{\rm down}\in \sR^{d_{\rm in}\times r}$ and $\mW_{\rm up}\in \sR^{r\times d_{\rm out}}$ are two learnable low-rank matrices to approximate the update of $\mW$ and rank $r$ is a hyper-parameter where $r \ll {\rm min}(d_{\rm in},d_{\rm out})$. In this way, we perform structured tuning on $\mW$ when its number of sensitive parameters exceeds $\sigma_{\rm opt}$, whose value depends on the pre-defined type of structured tuning method. For example, since implementing structured tuning with LoRA requires $2\times d_{\rm in} \times d_{\rm out} \times r$ trainable parameters for each sensitive weight matrix, we set $\sigma_{\rm LoRA} \leftarrow 2\times d_{\rm in} \times d_{\rm out} \times r$ to ensure that the number of trainable parameters introduced by structured tuning is always equal to or lower than the number of sensitive parameters.

In this way, our SPT adaptively incorporates both structured and unstructured tuning granularities to enable higher flexibility and stronger representational power, simultaneously. In Section~\ref{subsec:abl}, we show that structured tuning is important for the downstream tasks with larger domain gaps and both unstructured and structured tuning contribute clearly to the superior performance of our SPT.
\section{Experimental Setup}
\label{sec:setup}

%In this section, we provide the setup that we designed for our experiments. We follow the class incremental learning setup of the PYCIL framework \citep{zhou2021pycil} which only extends the classifier layer by the number of classes per task. PYCIL offers various continual learning baselines and we implement the hyperparameter optimization on two regularization methods which are EWC and LwF. We use the RayTune framework \citep{liaw2018tune} for the hyperparameter optimization and we did not use a pre-trained architecture and train all tasks from scratch. All our experiments run on a single-GPU setup of Nvidia A100.

%\subsection{Datasets and Architecture}
%We use the commonly used Split CIFAR100 dataset as in most of the CIL research and to be sure that results are more convincing, we also use mini-Imagenet dataset to further investigate our adaptive approach. We applied the same transformations defined in the PYCIL framework except we set resize to 64 for mini-Imagenet dataset to fasten the training process. For Split CIFAR-100 experiments we run all experiments with 3 different seeds to see the effect of task ordering and we observed that the task order does not have a significant effect in terms of performance. Therefore, we did not run mini-Imagenet experiments with different seeds. 

%\noindent
%\textbf{1) Split CIFAR-10:} It includes 10 object classes \citep{krizhevsky2009learning}. We used Split CIFAR-10 dataset to make a sensitivity analysis to observe how sensitive the EWC and LwF methods to their regularization hyperparameter.

%\noindent
%\textbf{2) Split CIFAR-100:} It includes 100 classes of visual object instances of CIFAR-100 \citep{krizhevsky2009learning}. In our setup, we randomly divide 100 classes into 10 tasks with 10 classes per task.

%\noindent
%\textbf{3) mini-Imagenet:} It is a variation of Imagenet dataset containing 100 classes of visual objects \citep{deng2009imagenet}. In our setup, we randomly divide 100 classes into 10 tasks with 10 classes per task.

%We used a specific version of the Resnets called Resnet32 \citep{he2016deep} for all our experiments to interpret the results more clearly.

%\subsection{Hyperparameters}

%EWC and LwF methods introduce an important lambda hyperparameter that controls the trade-off between stability and plasticity. In our experiments, we selected the baselines as vanilla EWC and vanilla LwF. In the vanilla settings, we set lambda to 1 which gives equal emphasis to current and previous tasks, and kept it fixed during the whole learning. On the other hand, in our adaptive approach, CARBON searches pre-defined search space and chooses the best lambda value based on validation accuracy. The search space for lambda is set to [1, $10^5$] for EWC and [1, 50] for the LwF based on our sensitivity analysis.

%\subsection{Performance Metrics}

%To measure the performance of each configuration we evaluated the model based on the validation data which is simply a portion of the training data. We derived two different accuracy metrics from ACC given in Eq(\ref{eqn:acc}) and called them incremental accuracy (last) and incremental accuracy (avg). Incremental accuracy (last) denotes the top-1 accuracy after the last task and it is a proper metric to measure the overall accuracy among all learned classes. The incremental accuracy often decreases with more tasks learned since CIL is continually adapted. However, only comparing incremental accuracy (last) ignores the performance evaluation along the learning trajectory. Therefore, we denoted incremental accuracy (avg) which considers the performance after every incremental stage. The higher value indicates a stronger performance along the incremental stages \citep{zhou2023deep}. We  also measure the level of forgetting with a backward transfer metric BWT Eq(\ref{eqn:bwt}) where $A_{T, i}$ is the test accuracy for task $i$ after training on task $T$. Higher BWT indicates a lower forget ratio \citep{kang2022forget}.

%\begin{equation}
%\label{eqn:acc}
%ACC = {\frac{1}{T}} \sum_{i=1}^{T} A_{T, i} 
%\end{equation}


%\begin{equation}
%\label{eqn:bwt}
%BWT = {\frac{1}{T-1}} \sum_{i=1}^{T-1} A_{T, i} - A_{i, i}
%\end{equation}


%%%%%%%%%%%%%%%%%%% Mert's Version %%%%%%%%%%%%%%%%%%%%%%%%%%%%%%%%%

 \partitle{Datasets.} In this paper, we experiment with \textbf{Split-CIFAR100}~\citep{krizhevsky2009learning} and \textbf{Split-MiniImageNet}~\citep{deng2009imagenet}. Each dataset exhibits objects from $100$ different categories, such as bird, snake and spider. We train all the models with $10$ tasks, with $10$ classes within each learning task on both CIFAR100 and MiniImageNet. Both datasets have 5000 training, and 1000 testing color images per learning task, each with $32\times32$ and $64\times64$ resolution for CIFAR100 and MiniImageNet respectively. 
%%%%% V1 %%%%%%%%%%%%%%%%%%
 
 %\partitle{Metrics.}  We evaluate the performance of our model using two standard metrics, top-1 accuracy and backward transfer for  
%$T$ tasks~\citep{diaz2018don}. Specifically, accuracy measures the test performance for all the observed tasks until $T$, whereas backward transfer measures how well the inclusion of task at time step $t-1$ influences the performance on task at previous time steps. Formally: 
%\begin{eqnarray}
%\mbox{\small {\bf Average Accuracy: } \normalsize ACC} & = & \frac{1}{T}
%\sum_{i=1}^T A_{T,i} \label{eq:acc} \\
%\mbox{\small {\bf Backward Transfer: } \normalsize BWT} & = & \frac{1}{T-1}
%\sum_{i=1}^{T-1} A_{T,i} - A_{i,i}   \label{eq:bwt} \\
%\end{eqnarray}
%\noindent where $A_{T,i}$ is the test accuracy of the model on task $i$ at time step $T$. For each metrics, the higher is better. 


%%%%%%%%%%%%%% V2 %%%%%%%%%%%%%%%%%%%%%

\partitle{Metrics.}  We resort to the standard metrics for evaluation, accuracy (ACC) which measures the final accuracy averaged over all tasks,  and backward transfer (BWT) which measures the average accuracy change of each task after learning new tasks. Formally for accuracy:
{\small
\vspace{-0.1cm}
\begin{align}
    ACC=\frac{1}{T}\sum\nolimits_{i=1}^T A_{T,i},
    \vspace{-0.1cm}
\end{align}}%

\noindent and for backward transferability: 
{\small
\begin{align}
    BWT=\frac{1}{T-1}\sum\nolimits_{i=1}^{T-1} (A_{T,i}-A_{i,i})
    \vspace{-0.1cm}
\end{align}}

\noindent where $A_{T,i}$ represents the testing accuracy of task $T$ after learning task $i$. In both cases, higher values indicate better performance. 

\section{Experimental Results}\label{sec:experiments}

We presented in the previous section how to leverage watermarks for detection and identification of images generated from text prompts.
We now present more general results on robustness and image quality for different generative tasks.
We also compare Stable Signature to other watermarking algorithms applied post-generation.

\begin{SCtable*}
    \centering
    \footnotesize
    \setlength{\tabcolsep}{4pt}
        \begin{tabular}{ c l @{\hspace{2pt}} l  *{2}{c} *{4}{p{15pt}}}
        \toprule
       & & \multirow{2}{*}{}          & \multirow{2}{*}{PSNR / SSIM $\uparrow$} & \multirow{2}{*}{FID $\downarrow$} &\multicolumn{4}{c}{Bit accuracy $\uparrow$ on:} \\ 

    &                                         &                           &           &                                            &  None & Crop & Brigh. & Comb.  \\ \midrule
\multirow{6}{*}{\rotatebox[origin=c]{90}{Tasks}}  
        & Text-to-Image                      & LDM~\cite{rombach2022ldm}             & $30.0$ / $0.89$   & $19.6$ \color{orange}{($-0.3$)} & $0.99$ & $0.95$ & $0.97$ & $0.92$ \\ \cmidrule{2-9}
       & Image Edition                       & DiffEdit~\cite{couairon2022diffedit}                  & $31.2$ / $0.92$ & $15.0$ \color{orange}{($-0.3$)}       & $0.99$ & $0.95$ & $0.98$ & $0.94$ \\ \cmidrule{2-9}
       & Inpainting - Full          & \multirow{2}{*}{Glide~\cite{nichol2021glide}}   & $31.1$ / $0.91$  & $16.8$ \color{orange}{($+0.6$)} & $0.99$ & $0.97$ & $0.98$ & $0.93$ \\ 
       & {\color{white}Inpa} - Mask only          &                                   & $37.8$ / $0.98$  & $9.0$~~ \color{orange}{($+0.1$)} & $0.89$ & $0.76$ & $0.84$ & $0.78$\\ \cmidrule{2-9}
       & Super-Resolution & LDM~\cite{rombach2022ldm}  & $34.0$ / $0.94$ & $11.6$ \color{orange}{($+0.0$)}      & $0.98$ & $0.93$ & $0.96$ & $0.92$ \\ 
    \midrule \rule{0pt}{8pt} \rule{0pt}{8pt} 
\multirow{7}{*}{\rotatebox[origin=c]{90}{WM Methods}} 
           & \emph{Post generation} \\
           & Dct-Dwt \cite{cox2007digital}                      & $0.14$ (s/img)      &  $39.5$ / $0.97$  & $19.5$ \color{orange}{($-0.4$)} & $0.86$ & $0.52$ & $0.51$ & $0.51$ \\ 
           & SSL Watermark \cite{fernandez2022sslwatermarking}  & $0.45$ (s/img)      &  $31.1$ / $0.86$  & $20.6$ \color{orange}{($+0.7$)} & $1.00$ & $0.73$ & $0.93$ & $0.66$ \\ 
           & FNNS \cite{kishore2021fixed}                       & $0.28$ (s/img)      &  $32.1$ / $0.90$  & $19.0$ \color{orange}{($-0.9$)} & $0.93$ & $0.93$ & $0.91$ & $0.93$ \\ 
           & HiDDeN \cite{zhu2018hidden}                        & $0.11$ (s/img)      &  $32.0$ / $0.88$  & $19.7$ \color{orange}{($-0.2$)} & $0.99$ & $0.97$ & $0.99$ & $0.98$ \\ \cmidrule{2-9}
           & \emph{Merged in generation} \\
           & Stable Signature                            & $0.00$ (s/img)             &  $30.0$ / $0.89$ & $19.6$ \color{orange}{($-0.3$)}     & $0.99$ & $0.95$ & $0.97$ & $0.92$ \\ 
        \bottomrule \vspace*{-0.2cm}
    \end{tabular}
    \caption{
        Generation quality and comparison to post-hoc watermarking on 512$\times$512 images and $48$-bit signatures.
        PSNR and SSIM are computed between generations of the original and watermarked generators.
        For FID, we show in {\color{orange} (color)} the difference with regards to original.
        Post-hoc watermarking is evaluated on text-generated images.
        (App.~\ref{sec:supp-robustness} gives results on more transformations, and App.~\ref{app:implementation-details} gives more details on the evaluations.)
        Overall, Stable Signature has minimal impact on generation quality. It has comparable robustness to post-hoc methods while being rooted in the generation itself.
        \vspace*{-0.2cm}
    }\label{tab:quality-watermarking} 
\end{SCtable*}



\subsection{Tasks \& evaluation metrics}
Since our method only involves the LDM decoder, it makes it compatible with many generative tasks. 
We evaluate text-to-image generation and image edition on the validation set of MS-COCO~\cite{lin2014microsoft}, super-resolution and inpainting on the validation set of ImageNet~\cite{deng2009imagenet} 
(all evaluation details are available in App.~\ref{app:evaluation}).

We evaluate the image distortion with the Peak Signal-to-Noise Ratio (PSNR), which is defined as $\mathrm{PSNR}(x,x') = -10\cdot \log_{10} (\mathrm{MSE}(x,x'))$, for $x,x'\in [0,1]^{c\times h\times w}$, as well as Structural Similarity score (SSIM)~\cite{wang2004image}.
They compare images generated with and without watermark. 
On the other hand, we evaluate the diversity and quality of the generated images with the Fr\'echet Inception Distance (FID)~\cite{heusel2017gans}.
The bit accuracy -- the percentage of bits correctly decoded -- evaluates the watermarks' robustness.



\subsection{Image generation quality}
\autoref{fig:qualitative} shows qualitative examples of how the image generation is altered by the latent decoder's fine-tuning. 
The difference is very hard to perceive even for a trained eye. 
This is surprising for such a low PSNR, especially since the watermark embedding is not constrained by any Human Visual System like in professional watermarking techniques. 
Most interestingly, the LDM decoder has indeed learned to add the watermark signal only over textured areas where the human eyes are not sensitive, while the uniform backgrounds are kept intact (see the pixel-wise difference).
% More visual results are available in App.~\ref{app:qualitative}.

\autoref{tab:quality-watermarking} presents a quantitative evaluation of image generation quality on the different tasks.
We report the FID, and the average PSNR and SSIM that are computed between the images generated by the fine-tuned LDM and the original one.
The results show that no matter the task, the watermarking has very small impact on the FID of the generation.

The average PSNR is around $30$~dB and SSIM around $0.9$ between images generated by the original and a  watermarked model.
They are a bit low from a watermarking perspective because we do not explicitly optimize for them.
Indeed, in a real world scenario, one would only have the watermarked version of the image. 
Therefore we don't need to be as close as possible to the original image but only want to generate artifacts-free images. Without access to the image generated by the original LDM, it is very hard to tell whether a watermark is present or not.


\begin{figure}[b]
    \centering
    \scriptsize
    \setlength{\tabcolsep}{0pt}
    \resizebox{0.99\linewidth}{!}{
    \begin{tabular}{c @{\hspace{0.1cm}} c @{\hspace{0.1cm}} c}
        \toprule
        Generated with original & Generated with watermark & Pixel-wise difference ($\times 10$) \\
        \midrule
        \includegraphics[width=0.33\linewidth]{figs/qual/01_nw.png} &
        \includegraphics[width=0.33\linewidth]{figs/qual/01_w.png} &
        \includegraphics[width=0.33\linewidth]{figs/qual/01_diff.png} \\ 
        \rule{0pt}{0.2cm}
        \includegraphics[width=0.33\linewidth]{figs/qual/02_nw.png} &
        \includegraphics[width=0.33\linewidth]{figs/qual/02_w.png} &
        \includegraphics[width=0.33\linewidth]{figs/qual/02_diff.png} \\       
        \bottomrule\\
    \end{tabular}
    }
    % \captionsetup{font=small}
    \caption{
    Images generated with Stable Diffusion. 
    The PSNR is $35.4$\,dB in the first row and $28.6$\,dB in the second.
    Images generated with Stable Signature look natural because modified areas are located where the eye is not sensitive.
    More examples in App.~\ref{app:qualitative}.
    }
    \label{fig:qualitative}
\end{figure}

\begin{table}[t]
    \centering
    \caption{
        Watermark robustness on image transformations applied before decoding, details of which are available in App.~\ref{app:evaluation}.
        We report the bit accuracy, averaged over $10\times1$k images generated from COCO prompts with $10$ different keys.
        }\label{tab:robustness}
        \footnotesize
        \vspace*{0.2cm}
        \setlength{\tabcolsep}{4pt}
        \resizebox{0.96\linewidth}{!}{
            \begin{tabular}{ll|ll|ll}
                \toprule
                \bf{Attack}             & \bf{Bit acc.}     & Comb.         & $0.92$    & Sharpness $2.0$ & $0.99$ \\
                None & $0.99$                               & Bright. $2.0$  & $0.97$    & Med. Filter $k$=7  & $0.94$ \\
                Crop $0.1$ & $0.95$                         & Cont. $2.0$    & $0.98$    & Resize $0.7$  & $0.91$  \\
                JPEG $50$ & $0.88$                          & Sat. $2.0$  & $0.99$      & Text overlay    & $0.99$ \\
                \bottomrule
            \end{tabular}
        }
        \vspace*{-0.3cm}
    \end{table}

\subsection{Watermark robustness}\label{subsec:robustness}
We evaluate the robustness of the watermark to different image transformations applied before extraction.
For each task, we generate $1$k images with $10$ models fine-tuned for different messages, and report the average bit accuracy in \autoref{tab:quality-watermarking}.
Additionally, \autoref{tab:robustness} reports results on more image transformations for images generated from COCO prompts.
The main evaluated transformations are presented in Fig.~\ref{fig:transformations} (more evaluation details are available in App.~\ref{app:evaluation}).

We see that the watermark is indeed robust for several tasks and across transformations.
The bit accuracy is always above $0.9$, except for inpainting, when replacing only the masked region of the image (between $1-50$\% of the image, with an average of $27\%$ across masks).
Besides, the bit accuracy is not perfect even without edition, mainly because there are images that are harder to watermark (\eg the ones that are very uniform, like the background in Fig.~\ref{fig:qualitative}) and for which the accuracy is lower.

Note that the robustness comes even without any transformation during the LDM fine-tuning phase:
it is due to the watermark extractor.
If the watermark embedding pipeline is learned to be robust against an augmentation, then the LDM will learn how to produce watermarks that are robust against it during fine-tuning.




\subsection{Comparison to post-hoc watermarking}\label{subsec:watermarking}

An alternative way to watermark generated images is to process them after the generation (post-hoc). 
This may be simpler, but less secure and efficient than Stable Signature.
We compare our method to a frequency based method, DCT-DWT~\cite{cox2007digital},
iterative approaches (SSL Watermark~\cite{fernandez2022sslwatermarking} and FNNS~\cite{kishore2021fixed}), and an encoder/decoder one like HiDDeN~\cite{zhu2018hidden}.
We choose DCT-DWT since it is employed by the original open source release of Stable Diffusion~\cite{2022stablediffusion}, and the other methods because of their performance and their ability to handle arbitrary image sizes and number of bits.
We use our implementations (see details in App.~\ref{app:watermarking}).

\autoref{tab:quality-watermarking} compares the generation quality and the robustness over $5$k generated images.
Overall, Stable Signature achieves comparable results in terms of robustness. 
HiDDeN's performance is a bit higher but its output bits are not i.i.d. meaning that it cannot be used with the same guarantees as the other methods.
We also observe that post-hoc generation gives worse qualitative results, images tend to present artifacts (see Fig.~\ref{fig:supp-watermark} in the supplement).
One explanation is that Stable Signature is merged into the high-quality generation process with the LDM auto-encoder model, which is able to modify images in a more subtle way.


% \vspace{0.3cm}
\subsection{Can we trade image quality for robustness?}\label{subsec:quality-tradeoff}

We can choose to maximize the image quality or the robustness of the watermark thanks to the weight $\lambda_i$ of the perceptual loss in~\eqref{eq:loss2}.
We report the average PSNR of $1$k generated images, as well as the bit accuracy obtained on the extracted message for the `Combined' editing applied before detection (qualitative results are in App.~\ref{sec:supp-percep-loss}).
A higher $\lambda_i$ leads to an image closer to the original one, but to lower bit accuracies on the extracted message, see \autoref{tab:tradeoff}.

\begin{table}[t]
        \centering
        \caption{Quality-robustness trade-off during fine-tuning.}\label{tab:tradeoff}
        \resizebox{0.9\linewidth}{!}{
        \begin{tabular}{l *{7}{c@{\hspace*{8pt}}}}
            \toprule
            $\lambda_i$ for fine-tuning     & $0.8$ & $0.4$ & $0.2$ & $0.1$ & $0.05$ & $0.025$ \\ \midrule
            \rule{0pt}{2ex}
            PSNR $\uparrow$ & $31.4$ & $30.6$ & $29.7$ & $28.5$ & $26.8$ & $24.6$ \\ 
            \rule{0pt}{2ex}
            Bit acc. $\uparrow$ on `comb.' & $0.85$ & $0.88$ & $0.90$ & $0.92$ & $0.94$ & $0.95$ \\ 
            \bottomrule 
            \vspace*{-0.4cm}
        \end{tabular}
        }
\end{table}



% \vspace{-0.2cm}
\subsection{Attack simulation layer}\label{subsec:message-decoder}

\begin{table}[t]
    \centering
    \caption{Role of the attack simulation layer at pre-training.}\label{tab:asl}
    % \vspace{0.2cm}
    \resizebox{0.95\linewidth}{!}{
    \begin{tabular}{c@{\hspace*{4pt}} *{5}{c@{\hspace*{8pt}}}}
        \toprule
        \multirow{2}{*}{ \shortstack{ Seen at  \\ $\mathcal{W}$ training \vspace*{-4pt}} } & \multicolumn{5}{c}{Bit accuracy $\uparrow$ at test time:} \\ \cmidrule{2-6}
            & Crop $0.1$ & Rot. $90$ &JPEG $50$ & Bright. $2.0$ & Res. $0.7$  \\
        \midrule
        \xmark     & 1.00 & 0.56 & 0.50 & 0.99 & 0.48 \\
        \cmark     & 1.00 & 0.99 & 0.90 & 0.99 & 0.91 \\
        \bottomrule
        \vspace*{-0.8cm} 
    \end{tabular} 
    }
\end{table}

Watermark robustness against image transformations depends solely on the watermark extractor.
here, we pre-train them with or without specific transformations in the simulation layer, on a shorter schedule of $50$ epochs, with $128\times 128$ images and $16$-bits messages.
From there, we plug them in the LDM fine-tuning stage and we generate $1$k images from text prompts.
We report the bit accuracy of the extracted watermarks in \autoref{tab:asl}.
The extractor is naturally robust to some transformations, such as crops or brightness, without being trained with them, while others, like rotations or JPEG, require simulation during training for the watermark to be recovered at test time.
Empirically we observed that adding a transformation improves results for the latter, but makes training more challenging.


%\section{Future Directions}
Based on the results from our analysis, we suggest several future directions for both Vietnamese monolingual language models and Vietnamese MRC benchmarks.
\subsection{Language Models}
Our analysis shows that monolingual models, especially PhoBERT, acquire comparable abilities in recognizing the differences in lexical information between unanswerable questions and the given context. However, monolingual models show poor performances when encountering unanswerable questions that require the ability to comprehend a bigger ``picture''. For example, while monolingual models perform very well on unanswerable questions that use explicit antonyms, they often have difficulties in recognizing unanswerable questions when these questions are created using implicit antonyms. We explain this phenomenon by the findings of \citet{zhang-etal-2021-need} as pre-training language models on larger text copora results in significant improvement on downstream tasks that require high-level semantic and factual knowledge such as Machine Reading Comprehension. Therefore, when encountering unanswerable questions that require ability to grasp big ``picture,'' models pre-trained with smaller text corpora will show lower performances. Hence, the small size of pre-training corpora of PhoBERT and WikiBERT may be the main reason for their poor performances in MRC.

Scaling the pre-training data size of PhoBERT will further develop this model and empower it to achieve state-of-the-art performances on different benchmarks of Machine Reading Comprehension. Besides, we believe that introducing a new unsupervised task for the pre-training phase that focuses on improving the high-level semantic and factual knowledge of pre-trained models also plays an integral role in developing language models in the future.
\subsection{Benchmarks}
\textbf{Unanswerable Questions. } Although UIT-ViQuAD 2.0 successfully further introduced new types of artificially unanswerable questions, our work in Section 5 shows that current unanswerable questions in the development test of UIT-ViQuAD 2.0 are still not challenging enough. In order to increase the challenging levels of unanswerable questions, we believe that more high-quality works on adversarial human annotation for unanswerable questions are needed. These works can follow the guidelines of adversarial human annotation for answerable questions \cite{bartolo-etal-2020-beat}. Results of these works can reveal different techniques to annotate hard unanswerable questions and therefore be valuable for improving the guidelines for unanswerable questions annotation for Machine Reading Comprehension.\\
\textbf{Quality of Benchmark. } On the other hand, as we have shown in section 5, although PhoBERT and XLM-RoBERTa achieve high performances on the UIT-VinewsQA development set, our unanswerable questions reveal that these two models do not truly understand the context to give the correct answer for questions in the original development set. We hypothesize that questions in UIT-VinewsQA enable machine reading comprehension systems with shortcut learning knowledge \cite{lai-etal-2021-machine} to achieve high performance due to biases in annotating process. Therefore, we believe that studies on how Vietnamese machine reading comprehension systems are currently evaluated are important for tracking the progress of Vietnamese language systems.

\section{Conclusions}
\label{sec:conclusions}
In this paper, a two-stage teacher-student framework has been proposed, where a student neural transducer ASR model distils the knowledge either from one or multiple complementary SSL pre-trained speech foundation models. In the first stage, the student ASR encoder is trained to approximate the embeddings generated by one or multiple teacher encoders without using the ground truth labels. In the second stage, the entire student model is fine-tuned with paired audio-text data, where the paired texts can be generated either by human annotation or by existing teacher ASR models. On LibriSpeech 100h, the averaged WER of a non-streaming student model trained with a single teacher is 11\% relative lower than that trained from scratch and using the multi-teacher setup further increases the relative WER reduction to 14\%. The proposed KD framework is also effective for streaming neural transducers using an additional time-delay factor, which is to resolve the emission mismatch between the non-streaming teacher and the streaming student. The proposed KD framework is also complementary to existing KD methods, leading to further performance improvement in combination. Further WER reductions can be achieved when scaling up the amount of unlabelled data used in the first stage. The best-performing student is obtained under a multi-teacher setup with extra unlabelled data, resulting relative WERR of 55\%.


\bibliographystyle{IEEEtran}
\bibliography{ref.bib}






% \subsection{Figures}
% Fig. 1 is an example of a floating figure using the graphicx package.
%  Note that $\backslash${\tt{label}} must occur AFTER (or within) $\backslash${\tt{caption}}.
%  For figures, $\backslash${\tt{caption}} should occur after the $\backslash${\tt{includegraphics}}.

% \begin{figure}[!t]
% \centering
% \includegraphics[width=2.5in]{fig1}
% \caption{Simulation results for the network.}
% \label{fig_1}
% \end{figure}

% Fig. 2(a) and 2(b) is an example of a double column floating figure using two subfigures.
%  (The subfig.sty package must be loaded for this to work.)
%  The subfigure $\backslash${\tt{label}} commands are set within each subfloat command,
%  and the $\backslash${\tt{label}} for the overall figure must come after $\backslash${\tt{caption}}.
%  $\backslash${\tt{hfil}} is used as a separator to get equal spacing.
%  The combined width of all the parts of the figure should do not exceed the text width or a line break will occur.
% %
% \begin{figure*}[!t]
% \centering
% \subfloat[]{\includegraphics[width=2.5in]{fig1}%
% \label{fig_first_case}}
% \hfil
% \subfloat[]{\includegraphics[width=2.5in]{fig1}%
% \label{fig_second_case}}
% \caption{Dae. Ad quatur autat ut porepel itemoles dolor autem fuga. Bus quia con nessunti as remo di quatus non perum que nimus. (a) Case I. (b) Case II.}
% \label{fig_sim}
% \end{figure*}

% Note that often IEEE papers with multi-part figures do not place the labels within the image itself (using the optional argument to $\backslash${\tt{subfloat}}[]), but instead will
%  reference/describe all of them (a), (b), etc., within the main caption.
%  Be aware that for subfig.sty to generate the (a), (b), etc., subfigure
%  labels, the optional argument to $\backslash${\tt{subfloat}} must be present. If a
%  subcaption is not desired, leave its contents blank,
%  e.g.,$\backslash${\tt{subfloat}}[].


 

% \section{Tables}
% Note that, for IEEE-style tables, the
%  $\backslash${\tt{caption}} command should come BEFORE the table. Table captions use title case. Articles (a, an, the), coordinating conjunctions (and, but, for, or, nor), and most short prepositions are lowercase unless they are the first or last word. Table text will default to $\backslash${\tt{footnotesize}} as
%  the IEEE normally uses this smaller font for tables.
%  The $\backslash${\tt{label}} must come after $\backslash${\tt{caption}} as always.
 
% \begin{table}[!t]
% \caption{An Example of a Table\label{tab:table1}}
% \centering
% \begin{tabular}{|c||c|}
% \hline
% One & Two\\
% \hline
% Three & Four\\
% \hline
% \end{tabular}
% \end{table}

% \section{Algorithms}
% Algorithms should be numbered and include a short title. They are set off from the text with rules above and below the title and after the last line.

% \begin{algorithm}[H]
% \caption{Weighted Tanimoto ELM.}\label{alg:alg1}
% \begin{algorithmic}
% \STATE 
% \STATE {\textsc{TRAIN}}$(\mathbf{X} \mathbf{T})$
% \STATE \hspace{0.5cm}$ \textbf{select randomly } W \subset \mathbf{X}  $
% \STATE \hspace{0.5cm}$ N_\mathbf{t} \gets | \{ i : \mathbf{t}_i = \mathbf{t} \} | $ \textbf{ for } $ \mathbf{t}= -1,+1 $
% \STATE \hspace{0.5cm}$ B_i \gets \sqrt{ \textsc{max}(N_{-1},N_{+1}) / N_{\mathbf{t}_i} } $ \textbf{ for } $ i = 1,...,N $
% \STATE \hspace{0.5cm}$ \hat{\mathbf{H}} \gets  B \cdot (\mathbf{X}^T\textbf{W})/( \mathbb{1}\mathbf{X} + \mathbb{1}\textbf{W} - \mathbf{X}^T\textbf{W} ) $
% \STATE \hspace{0.5cm}$ \beta \gets \left ( I/C + \hat{\mathbf{H}}^T\hat{\mathbf{H}} \right )^{-1}(\hat{\mathbf{H}}^T B\cdot \mathbf{T})  $
% \STATE \hspace{0.5cm}\textbf{return}  $\textbf{W},  \beta $
% \STATE 
% \STATE {\textsc{PREDICT}}$(\mathbf{X} )$
% \STATE \hspace{0.5cm}$ \mathbf{H} \gets  (\mathbf{X}^T\textbf{W} )/( \mathbb{1}\mathbf{X}  + \mathbb{1}\textbf{W}- \mathbf{X}^T\textbf{W}  ) $
% \STATE \hspace{0.5cm}\textbf{return}  $\textsc{sign}( \mathbf{H} \beta )$
% \end{algorithmic}
% \label{alg1}
% \end{algorithm}

% Que sunt eum lam eos si dic to estist, culluptium quid qui nestrum nobis reiumquiatur minimus minctem. Ro moluptat fuga. Itatquiam ut laborpo rersped exceres vollandi repudaerem. Ulparci sunt, qui doluptaquis sumquia ndestiu sapient iorepella sunti veribus. Ro moluptat fuga. Itatquiam ut laborpo rersped exceres vollandi repudaerem. 
% \section{Mathematical Typography \\ and Why It Matters}

% Typographical conventions for mathematical formulas have been developed to {\bf provide uniformity and clarity of presentation across mathematical texts}. This enables the readers of those texts to both understand the author's ideas and to grasp new concepts quickly. While software such as \LaTeX \ and MathType\textsuperscript{\textregistered} can produce aesthetically pleasing math when used properly, it is also very easy to misuse the software, potentially resulting in incorrect math display.

% IEEE aims to provide authors with the proper guidance on mathematical typesetting style and assist them in writing the best possible article. As such, IEEE has assembled a set of examples of good and bad mathematical typesetting \cite{ref1,ref2,ref3,ref4,ref5}. 

% Further examples can be found at \url{http://journals.ieeeauthorcenter.ieee.org/wp-content/uploads/sites/7/IEEE-Math-Typesetting-Guide-for-LaTeX-Users.pdf}

% \subsection{Display Equations}
% The simple display equation example shown below uses the ``equation'' environment. To number the equations, use the $\backslash${\tt{label}} macro to create an identifier for the equation. LaTeX will automatically number the equation for you.
% \begin{equation}
% \label{deqn_ex1}
% x = \sum_{i=0}^{n} 2{i} Q.
% \end{equation}

% \noindent is coded as follows:
% \begin{verbatim}
% \begin{equation}
% \label{deqn_ex1}
% x = \sum_{i=0}^{n} 2{i} Q.
% \end{equation}
% \end{verbatim}

% To reference this equation in the text use the $\backslash${\tt{ref}} macro. 
% Please see (\ref{deqn_ex1})\\
% \noindent is coded as follows:
% \begin{verbatim}
% Please see (\ref{deqn_ex1})\end{verbatim}

% \subsection{Equation Numbering}
% {\bf{Consecutive Numbering:}} Equations within an article are numbered consecutively from the beginning of the
% article to the end, i.e., (1), (2), (3), (4), (5), etc. Do not use roman numerals or section numbers for equation numbering.

% \noindent {\bf{Appendix Equations:}} The continuation of consecutively numbered equations is best in the Appendix, but numbering
%  as (A1), (A2), etc., is permissible.\\

% \noindent {\bf{Hyphens and Periods}}: Hyphens and periods should not be used in equation numbers, i.e., use (1a) rather than
% (1-a) and (2a) rather than (2.a) for subequations. This should be consistent throughout the article.

% \subsection{Multi-Line Equations and Alignment}
% Here we show several examples of multi-line equations and proper alignments.

% \noindent {\bf{A single equation that must break over multiple lines due to length with no specific alignment.}}
% \begin{multline}
% \text{The first line of this example}\\
% \text{The second line of this example}\\
% \text{The third line of this example}
% \end{multline}

% \noindent is coded as:
% \begin{verbatim}
% \begin{multline}
% \text{The first line of this example}\\
% \text{The second line of this example}\\
% \text{The third line of this example}
% \end{multline}
% \end{verbatim}

% \noindent {\bf{A single equation with multiple lines aligned at the = signs}}
% \begin{align}
% a &= c+d \\
% b &= e+f
% \end{align}
% \noindent is coded as:
% \begin{verbatim}
% \begin{align}
% a &= c+d \\
% b &= e+f
% \end{align}
% \end{verbatim}

% The {\tt{align}} environment can align on multiple  points as shown in the following example:
% \begin{align}
% x &= y & X & =Y & a &=bc\\
% x' &= y' & X' &=Y' &a' &=bz
% \end{align}
% \noindent is coded as:
% \begin{verbatim}
% \begin{align}
% x &= y & X & =Y & a &=bc\\
% x' &= y' & X' &=Y' &a' &=bz
% \end{align}
% \end{verbatim}





% \subsection{Subnumbering}
% The amsmath package provides a {\tt{subequations}} environment to facilitate subnumbering. An example:

% \begin{subequations}\label{eq:2}
% \begin{align}
% f&=g \label{eq:2A}\\
% f' &=g' \label{eq:2B}\\
% \mathcal{L}f &= \mathcal{L}g \label{eq:2c}
% \end{align}
% \end{subequations}

% \noindent is coded as:
% \begin{verbatim}
% \begin{subequations}\label{eq:2}
% \begin{align}
% f&=g \label{eq:2A}\\
% f' &=g' \label{eq:2B}\\
% \mathcal{L}f &= \mathcal{L}g \label{eq:2c}
% \end{align}
% \end{subequations}

% \end{verbatim}

% \subsection{Matrices}
% There are several useful matrix environments that can save you some keystrokes. See the example coding below and the output.

% \noindent {\bf{A simple matrix:}}
% \begin{equation}
% \begin{matrix}  0 &  1 \\ 
% 1 &  0 \end{matrix}
% \end{equation}
% is coded as:
% \begin{verbatim}
% \begin{equation}
% \begin{matrix}  0 &  1 \\ 
% 1 &  0 \end{matrix}
% \end{equation}
% \end{verbatim}

% \noindent {\bf{A matrix with parenthesis}}
% \begin{equation}
% \begin{pmatrix} 0 & -i \\
%  i &  0 \end{pmatrix}
% \end{equation}
% is coded as:
% \begin{verbatim}
% \begin{equation}
% \begin{pmatrix} 0 & -i \\
%  i &  0 \end{pmatrix}
% \end{equation}
% \end{verbatim}

% \noindent {\bf{A matrix with square brackets}}
% \begin{equation}
% \begin{bmatrix} 0 & -1 \\ 
% 1 &  0 \end{bmatrix}
% \end{equation}
% is coded as:
% \begin{verbatim}
% \begin{equation}
% \begin{bmatrix} 0 & -1 \\ 
% 1 &  0 \end{bmatrix}
% \end{equation}
% \end{verbatim}

% \noindent {\bf{A matrix with curly braces}}
% \begin{equation}
% \begin{Bmatrix} 1 &  0 \\ 
% 0 & -1 \end{Bmatrix}
% \end{equation}
% is coded as:
% \begin{verbatim}
% \begin{equation}
% \begin{Bmatrix} 1 &  0 \\ 
% 0 & -1 \end{Bmatrix}
% \end{equation}\end{verbatim}

% \noindent {\bf{A matrix with single verticals}}
% \begin{equation}
% \begin{vmatrix} a &  b \\ 
% c &  d \end{vmatrix}
% \end{equation}
% is coded as:
% \begin{verbatim}
% \begin{equation}
% \begin{vmatrix} a &  b \\ 
% c &  d \end{vmatrix}
% \end{equation}\end{verbatim}

% \noindent {\bf{A matrix with double verticals}}
% \begin{equation}
% \begin{Vmatrix} i &  0 \\ 
% 0 & -i \end{Vmatrix}
% \end{equation}
% is coded as:
% \begin{verbatim}
% \begin{equation}
% \begin{Vmatrix} i &  0 \\ 
% 0 & -i \end{Vmatrix}
% \end{equation}\end{verbatim}

% \subsection{Arrays}
% The {\tt{array}} environment allows you some options for matrix-like equations. You will have to manually key the fences, but there are other options for alignment of the columns and for setting horizontal and vertical rules. The argument to {\tt{array}} controls alignment and placement of vertical rules.

% A simple array
% \begin{equation}
% \left(
% \begin{array}{cccc}
% a+b+c & uv & x-y & 27\\
% a+b & u+v & z & 134
% \end{array}\right)
% \end{equation}
% is coded as:
% \begin{verbatim}
% \begin{equation}
% \left(
% \begin{array}{cccc}
% a+b+c & uv & x-y & 27\\
% a+b & u+v & z & 134
% \end{array} \right)
% \end{equation}
% \end{verbatim}

% A slight variation on this to better align the numbers in the last column
% \begin{equation}
% \left(
% \begin{array}{cccr}
% a+b+c & uv & x-y & 27\\
% a+b & u+v & z & 134
% \end{array}\right)
% \end{equation}
% is coded as:
% \begin{verbatim}
% \begin{equation}
% \left(
% \begin{array}{cccr}
% a+b+c & uv & x-y & 27\\
% a+b & u+v & z & 134
% \end{array} \right)
% \end{equation}
% \end{verbatim}

% An array with vertical and horizontal rules
% \begin{equation}
% \left( \begin{array}{c|c|c|r}
% a+b+c & uv & x-y & 27\\ \hline
% a+b & u+v & z & 134
% \end{array}\right)
% \end{equation}
% is coded as:
% \begin{verbatim}
% \begin{equation}
% \left(
% \begin{array}{c|c|c|r}
% a+b+c & uv & x-y & 27\\
% a+b & u+v & z & 134
% \end{array} \right)
% \end{equation}
% \end{verbatim}
% Note the argument now has the pipe "$\vert$" included to indicate the placement of the vertical rules.


% \subsection{Cases Structures}
% Many times cases can be miscoded using the wrong environment, i.e., {\tt{array}}. Using the {\tt{cases}} environment will save keystrokes (from not having to type the $\backslash${\tt{left}}$\backslash${\tt{lbrace}}) and automatically provide the correct column alignment.
% \begin{equation*}
% {z_m(t)} = \begin{cases}
% 1,&{\text{if}}\ {\beta }_m(t) \\ 
% {0,}&{\text{otherwise.}} 
% \end{cases}
% \end{equation*}
% \noindent is coded as follows:
% \begin{verbatim}
% \begin{equation*}
% {z_m(t)} = 
% \begin{cases}
% 1,&{\text{if}}\ {\beta }_m(t),\\ 
% {0,}&{\text{otherwise.}} 
% \end{cases}
% \end{equation*}
% \end{verbatim}
% \noindent Note that the ``\&'' is used to mark the tabular alignment. This is important to get  proper column alignment. Do not use $\backslash${\tt{quad}} or other fixed spaces to try and align the columns. Also, note the use of the $\backslash${\tt{text}} macro for text elements such as ``if'' and ``otherwise.''

% \subsection{Function Formatting in Equations}
% Often, there is an easy way to properly format most common functions. Use of the $\backslash$ in front of the function name will in most cases, provide the correct formatting. When this does not work, the following example provides a solution using the $\backslash${\tt{text}} macro:

% \begin{equation*} 
%   d_{R}^{KM} = \underset {d_{l}^{KM}} {\text{arg min}} \{ d_{1}^{KM},\ldots,d_{6}^{KM}\}.
% \end{equation*}

% \noindent is coded as follows:
% \begin{verbatim}
% \begin{equation*} 
%  d_{R}^{KM} = \underset {d_{l}^{KM}} 
%  {\text{arg min}} \{ d_{1}^{KM},
%  \ldots,d_{6}^{KM}\}.
% \end{equation*}
% \end{verbatim}

% \subsection{ Text Acronyms Inside Equations}
% This example shows where the acronym ``MSE" is coded using $\backslash${\tt{text\{\}}} to match how it appears in the text.

% \begin{equation*}
%  \text{MSE} = \frac {1}{n}\sum _{i=1}^{n}(Y_{i} - \hat {Y_{i}})^{2}
% \end{equation*}

% \begin{verbatim}
% \begin{equation*}
%  \text{MSE} = \frac {1}{n}\sum _{i=1}^{n}
% (Y_{i} - \hat {Y_{i}})^{2}
% \end{equation*}
% \end{verbatim}

% \section{Conclusion}
% The conclusion goes here.



\end{document}


