%% Template for a preprint Letter or Article for submission
%% to the journal Nature.
%% Written by Peter Czoschke, 26 February 2004

%\documentclass{nature}
\documentclass[12pt]{article}
\usepackage{scicite}
\usepackage{times}
\usepackage{graphicx}
\usepackage{amsmath}
\usepackage[usenames,dvipsnames]{color}
\usepackage[export]{adjustbox}
\usepackage{hyperref}
\topmargin 0.0cm
\oddsidemargin 0.2cm
\textwidth 17cm 
\textheight 21cm
\footskip 1.0cm

\newenvironment{sciabstract}{
\begin{quote} \bf}
{\end{quote}}

\newcounter{lastnote}
\newenvironment{scilastnote}{
\setcounter{lastnote}{\value{enumiv}}%
\addtocounter{lastnote}{+1}
\begin{list}
{\arabic{lastnote}.}
{\setlength{\leftmargin}{.22in}}
{\setlength{\labelsep}{.5em}}}
{\end{list}}


%\bibliographystyle{naturemag}

\newcommand{\bs}{\boldsymbol}
\newcommand{\hl}{\color{blue}}


%\title{A physical basis for a developmental gait transition in Anacondas}
\title{Non-planar snake gaits: from Stigmatic-starts to Sidewinding}
%% Notice placement of commas and superscripts and use of &
%% in the author list

\author{N. Charles$^1$, R. Chelakkot$^2$,  M. Gazzola$^3$, B. Young$^4$, \& L. Mahadevan$^{5\ast}$
\\
\footnotesize{$^1 $ Engineering and Applied Sciences, Harvard University, Cambridge, MA 02138, USA}\\
\footnotesize{$^2$ Department of Physics, Indian Institute of Technology Bombay, Mumbai, 400076, India}\\
\footnotesize{$^3$ Department of Mechanical Science and Engineering, National Center for Supercomputing Applications, }\\
\footnotesize{Carl R. Woese Institute for Genomic Biology,  University of Illinois at Urbana-Champaign, }\\
\footnotesize{Urbana, IL, USA}\\
\footnotesize{$^4$ Department of Anatomy, A.T. Still University of Health Sciences,}\\
\footnotesize{ Kirksville College of Osteopathic Medicine, Kirksville, MO  63501, USA}\\
\footnotesize{$^5$ Engineering and Applied Sciences, Department of Physics, }\\
\footnotesize{Department of Organismic and Evolutionary Biology, Harvard University, Cambridge, MA 02138, USA}\\
\footnotesize{$^\ast$To whom correspondence should be addressed; E-mail:  lmahadev@g.harvard.edu}
}

\date{}
\begin{document}

\baselineskip16pt
\maketitle

%\begin{affiliations}
 %\item School of Engineering and Applied Sciences, Department of Organismic and Evolutionary Biology, Department of Physics, Harvard University, Cambridge, MA 02138, USA
  %\item Department of Anatomy, A.T. Still University of Health Sciences, Kirksville College of Osteopathic Medicine, 800 W. Jefferson St., 
%Kirksville, MO  63501, USA.
%\end{affiliations}

\begin{sciabstract}
Of the vast variety of animal gaits, one of the most striking is the non-planar undulating helical motion of a sidewinder. But non-planar helical gaits are not limited to sidewinders. Here we report a new non-planar gait used as an escape strategy in juvenile anacondas ($Eunectes ~notaeus$). In the S(tigmatic)-start, named for its eponymous shape, transient locomotion arises when the snake writhes and bends  out of the plane while rolling forward about its midsection without slippage. We present a  mathematical model for an active non-planar filament that interacts anisotropically with a frictional substrate to quantify our observations and show that locomotion is due to a propagating localized pulse of a topological quantity, the link density.  A phase diagram as a function of scaled body size and muscular torques shows that relatively light juveniles are capable of S-starts but heavy adults are not, consistent with our experimental observations. We further show theoretically that a periodic sequence of S-starts naturally leads to sidewinding. All together, our characterization of a novel escape strategy in snakes using non-planar gaits highlights the role of topology in locomotion, provides a phase diagram for gait feasibility as a function of body size, and shows that the S-start forms the fundamental kernel underlying sidewinding.

%Our experiments motivate a mathematical model for the non-planar motion of snakes. Consistent with observations, numerical simulations of the model show an interesting gait transition as a function of body size; small body weight per unit length relative to muscular force in newborns and juveniles allows for stigmatic locomotion but large values of this parameter makes stigmatic locomotion infeasible for adult snakes. In addition to characterizing the regime of operability of stigmatic locomotion, our study highlights how changes in body size over development can transform locomotory gaits. 
\end{sciabstract}
Snakes exhibit a wide variety of steady gaits on land associated with rectilinear motion\cite{gray68,Hu2013}, periodic undulation\cite{gray68,Gray50,Gasc89,Moon98, Guo2008, Biewener2003,Hu2009,Alexader2003}, and sidewinding\cite{gray68,Jayne86,Gans92,Marvi2014}. In addition, they also employ a range of  transient gaits such as concertina locomotion \cite{gray68,Jayne91} and various striking and escape lunges \cite{Alfaro2002,Alfaro2003}, depending on the type of species, physical environment, and behavioral situation.   Perhaps the most dramatic of these gaits is sidewinding \cite{Marvi2014}, wherein a snake lifts part of its body out of the plane, and places it askew; when this is done repeatedly, the snake winds itself sideways, leading to the eponymous sidewinding gait.  Sidewinding involves spatially inhomogeneous non-planar motions that are transient. The complexity of this gait raises a number of questions about the extent, origin and limits of this gait from a dynamical, physiological and evolutionary perspective \cite{Alexader2003, Biewener2003}.  

In this letter we describe and quantify a new transient non-planar gait observed in newborn and juvenile yellow anacondas ($Eunectes ~notaeus$) that gives the snake a very fast forward velocity, often used defensively to escape a threatening situation. Inspired by a rapid aquatic gait commonly seen in juvenile fish, known as the C-start \cite{Domenici97,Gazzola2012}, we dub this terrestrial analog the {S-start}, since the snake forms an S-shape consisting of three roughly co-linear segments connected by two tightly curved regions as shown in Fig.~\ref{fig1}a. Motion begins when the two curved segments are lifted off the ground. Simultaneously the outer parallel straight segments move along the ground even as the middle straight segment remains stationary. This causes the curved segments to travel along the snake. 
% in this configuration but by rolling rather than sliding
%the ground and moved parallel to each other along the length of the snake. The outer, straight segments slide along the ground during this motion, while the middle, straight  segment remains stationary on the ground, even as its connections to the curved segments lift off and come down onto the ground
  (see SI Movie 1). Thus, the snake effectively moves like a ruck in a rug \cite{Kolinski2009} and propels itself forward (Fig.\ref{fig1}b-f), as the S shape flows along its length, until one end comes off the ground, terminating this transient mode of motility.  This non-planar mode leads to motion parallel to the snake and while reminiscent of sidewinding\cite{gray68,Marvi2014} is different as it is a transient gait and does not lead to sideways translation. However, as we will see later the S-start forms the building block from which many non-planar gaits such as sidewinding are formed.
\begin{figure}
\includegraphics[width=\columnwidth]{./Figs/Fig1}
\caption{\label{fig1}\linespread{1} \selectfont{}
{\footnotesize S-start of {newborn} anaconda($Eunectes ~notaeus$). (a) The starting point of the locomotion is an `S' shape including three co-linear regions, connected by two curved regions that are elevated from the contact surface as the middle, straight segment pushes down against the surface. (b-f) Series of overlaid time-lapse images indicating the S-start (see SI Movie 1).} }
\end{figure}
  

{Lifting its body allows the snake to minimize frictional losses, but comes with a cost -- the snake has to overcome gravity and move out of the plane to do so.}  We expect slender snakes or those with relatively stronger  musculature to be able to achieve this motion, while thicker snakes with less musculature (and more bone) will be unable to do so. The maximum length $L$ that can be supported out of the plane scales as  $L \sim m/\rho a^2 g h$, where $m$ is the maximum active torque that the musculature can generate, $\rho$ the density of the snake of cross-sectional diameter $a$, and $h$ the height above the ground. Assuming that the maximum torque $m \sim \Gamma a^3 f(\phi)$, where $\Gamma$ is the maximum muscular stress,  $\phi$ is the fraction of the tissue that is muscle,  and that the typical lift-off height $h \sim a$, we find that $L \sim \Gamma f(\phi)/ \rho g$. Therefore as the snake grows in girth, its internal muscle fraction decreases (due to a concomitant increase in bone mass) \cite{Prange76,Anderson79,Garcia2006}, the length it can support off the ground decreases, and the S-start becomes physically unattainable. Indeed,  our study of locomotion in {three newborn, five juvenile, and two adult anacondas} with a range of lengths and weights (SI) reveals that all the newborn and juvenile anacondas can move using the S-start, whereas adult anacondas do not exhibit this type of locomotion. 

 \begin{figure}
\begin{center}
\includegraphics[width=1.0\columnwidth]{./Figs/Fig2}
\end{center}
\caption{\label{fig2}\linespread{1} \selectfont{}
{\footnotesize {Mathematical model of the S-start motion and comparison with experiments. (a) A schematic of the mathematical model, with localized in-plane torques at locations $b,d$ and a localized out-of-plane torque at $c$, that propagate from head to tail synchronously at velocity $v_m$.  (b) (top) Numerical simulations capture the S-start (see SI  for details, and Movie 2). (bottom) Simulation snapshot shows the filament bending out of the plane. (c) Kymograph of the local vertical elevation as a function of arc length, $s$ and time, $t$, shows a minimum around $c$ where the filament is stationary relative to ground, and is maximum at  $b$ and $d$. (d) Kymograph of the local speed with respect to the surface. The speed is nearly zero at region $c$ and uniform for other segments.}}}
\end{figure}

To understand these observations quantitatively, we model the snake (Fig.~\ref{fig2}a) as an active elastic filament  interacting frictionally with a planar substrate. The filament is assumed to have a circular cross-section of diameter $a$ and length $L$ with aspect ratio $L/a=50$. The axis is parametrized by a material coordinate $s$, and the shape of the snake at a particular time $t$ can be characterized in terms of a local position vector $\textbf{r}(s,t)$ and an associated material orthonormal coordinate system $\{\textbf{d}_1(s,t), \textbf{d}_2(s,t), \textbf{d}_3(s,t)\}$. 
We assume that the filament has a passive bending rigidity $B$ and twisting rigidity $C$, and we further neglect extensional and shear deformations of the filament, a reasonable assumption for long slender elastic objects, so that $s$ becomes the arc length and $\textbf{d}_3(s,t)$ is the tangent to the center-line of the snake.  
Since the filament is  capable of exerting muscular couples along its body, we define an active torque vector $\textbf{m} = \Sigma_{i=1}^3\textbf{m}_i \textbf{d}_i$  (Fig~\ref{fig2}a), associated with the two bending modes and one twisting mode of the filament at each cross-section.  
Then, the motion of the filament is determined via the balance of linear and angular momentum governed by the equations
$\rho a^2 \partial \textbf{v}(s,t)/\partial t=\partial \textbf{f}(s,t)/\partial s+ \textbf{F}_e(s,t)$ and $\textbf{I} \partial {\bs \omega}(s,t)/\partial t= \partial \textbf{M}(s,t)/\partial s + \textbf{d}_3 \times \textbf{f}$. 
Here $\rho$ is the density of the filament, ${\bf v = \dot r}$ is the local filament velocity, ${\bf f}$ is the vector of force resultants at a cross section, ${\bs \omega}$ is the local angular velocity of filament rotation, $\bf I$ is the moment of inertia of the cross section (assumed to be circular), ${\bf M} = {\bf B} {\bf k}(s,t) + \textbf{m}(s,t)$ is sum of the passive elastic torques (associated with the vector of bending and twisting strains ${\bf k}(s,t)$ and the matrix of bending and twisting stiffnesses \(\textbf{B} = \text{diag}\left(B, B, C \right) \)) and the active muscular torques \( \textbf{m}(s,t) \) (SI), and ${\bf F}_e$ is the external force on the filament associated with gravity and frictional interactions with the ground (SI).  
We assume that the filament interacts with the substrate frictionally, with a finite friction coefficient that is anisotropic. Assuming $\mu_f< \mu_b< \mu_s$ to be the coefficients of kinetic friction in the forward, backward and sideways directions respectively, we choose $\mu_f:\mu_b:\mu_s=1.1:1.4:2$, {based on recent experiments} \cite{Hu2009}.  We discretize the filament and solve the associated discretized governing equations using the numerical integration scheme given in references \cite{Gazzola2018}.
\begin{figure}
\begin{center}
\includegraphics[width=1.0\columnwidth]{./Figs/Fig3}
\caption{\label{fig3} \linespread{1} \selectfont{} \footnotesize (a) Snake centerline \( Wr \) at one snapshot during simulated snake motion, as a function of its body weight relative to the muscle torques ($\rho g L^2/\Gamma a$).  As expected, \( Wr \) decreases monotonically with increasing weight and decreasing muscle.  (b) Velocity of the snake relative to its muscle velocity ($v_s/v_m$) as a function of body centerline \( Wr \).  Velocity is maximized for intermediate values of \( Wr \). (c) Snake velocity as a function of body weight relative to muscle torque, combining parts (a) and (b).  Filled symbols are simulation data and open symbols correspond to yellow anaconda (\textit{Eunectes~notaeus}) assuming a value for $\Gamma \text{ [kPa]} \simeq 7.72 (a\text{ [cm]})^{1.39}$ for snakes \cite{Moon2007}, measured on different surfaces, with the error bars of one standard deviation. The S-start fails in the shaded region, where the snake uncoils without a net translation and causes a backward shift of snake center-of-mass and a negative $v_s$, and for large weight or small muscle torque.  Seen through the lens of parts (a) and (b), the S-start requires an intermediate value of \( Wr \), which is produced for intermediate weight to muscle ratios. (d) A two dimensional phase space indicates different locomotion regimes, characterized by two parameters, a scaled muscular torque $\Gamma a^3 L/B$ and the a scaled weight $\rho a^2 L^3/B$. Very small and very large values of the scaled weight are either inertially unstable (U) or so heavy that friction dominates (F), producing backwards motion.  For extremely heavy snakes, no net motion is observed (N).  The S-start (S) is feasible only in an optimal regime of body weight and muscle forces (see Movie S2). Note that the vertical axis over the horizontal $(\rho a^2 L^3/B)/(\Gamma a^3 L/B)$ gives the horizontal axis in (c). That this single ratio determines the type of locomotion explains the straight phase boundaries in (d).}
\end{center}
\end{figure}

To complete the mathematical formulation of the problem, we note that the S-start  shows the snake has two localized bends associated with the in-plane curvatures (${\bs m_2}$) at locations  $(a)$ and $(c)$, and one localized bend associated with out-of-plane curvature at the contact zone in the central segment that is pressed downwards with an out-of-plane torque (${\bs m_1}$) at $(b)$ (see Fig.~\ref{fig2}a). To model this, we  assume that the three active muscular torques are  localized with a magnitude $m \sim \Gamma a^3 \exp^{-(s-s_0-{ v_m}t)/2\sigma^2}$, where  $\Gamma$ is the maximum magnitude of the muscular stress, $s_0$ is the initial location of the torque, $\sigma$ is the axial scale over which the torque acts, and ${v_m}$ the speed of propagation. These three localized torques move uniformly along the filament towards the tail with a speed ${v_m}$ relative to the body, causing a pulse-like traveling wave that propels the filament parallel to itself with a speed {$v_s={\bf v}_{\text{cm}}\cdot \textbf{x}_h$ ($\textbf{x}_h$ is the unit vector associated with the direction of the head $s=0$, ${\bf v}_{\text{cm}}$ is the center-of-mass velocity of the snake}) as shown in Fig.~\ref{fig2}b (see SI Movie 2).  Eventually the active torques reach the tail and the whole filament straightens out.  In Fig.~\ref{fig2}c, we plot the local elevation of segments along the filament, showing that the traveling pulse corresponds to a small localized region of contact to be pushed down into the substrate while the curved regions are elevated above the surface.  In Fig.~\ref{fig2}d, we plot the {local} speed of the filament $(|{\bf v}|)$ showing that the middle section at $(c)$ never slips relative to the surface, while the other regions move uniformly through this region. 

The localized pulse that underlies the S-start has a topological interpretation that stems from recognizing that the non-planar shape and motion of a filament can be characterized in terms of the CFW theorem \cite{Calugareanu1959} relating the topological link ($Lk$) to the twist ($Tw$) and writhe ($Wr$) via the formula $Lk = Tw+Wr$. Here, we apply these quantities to our active filament by treating the filament centerline \( \textbf{r}(s, t) \) and first director vector field \( \textbf{d}_1(s,t) \) as a mathematical ribbon.  While the CFW theorem holds exactly for closed ribbons, we minimize errors for our open system by appending a long straight edge of length \( 100 L \) to each end of the centerline, and compute $Lk$, $Tw$ and $Wr$ by adapting methods from \cite{Klenin2000}.  In Fig.~\ref{fig3}a we show the writhe of the snake body at a single moment during the simulated gait; light, muscular snakes show much more  \( Wr \) than heavier and less muscular snakes.  In Fig.~\ref{fig3}b, we show that the scaled forward snake speed (averaged over a full gait cycle) is small for low values of \( Wr \) because of the effects of friction, and also low for large values because the snake tends to flail and lift off from the ground, but is maximized for an intermediate value of \( Wr \). Combining the results of Fig.~\ref{fig3}a-b,  in Fig.~\ref{fig3}c we plot snake forward speed over snake weight relative to muscular strength.  Seen together, these plots reveal that maximum forward speed is achieved for the weight and muscle that produce the optimal \( Wr \) in the snake body.   

To quantify the role of non-planar bending in the snake body during motion, we employ local  densities of \(Lk\), \(Tw\) and \(Wr\), denoted by \(\lambda \), \( \tau\) and \(\omega\), respectively, and defined by 
$Tw = \int_0^L \tau(s) ds,~ \lambda(s) = Lk(\bs{r}([s, s+\delta]))/\delta,~ \omega = Wr(\bs{r}([s, s+\delta]))/\delta$
where \( \bs{r}([s, s+ \delta]) \) denotes the section of the snake centerline for which the arclength coordinate lies in the range \( [s, s + \delta] \),  chosing \( \delta = L/10 \), determined empirically.  In Fig.~SI-2 and Movie S2, we show \( \lambda\), \(\tau\) and \( \omega\) over one full gait cycle. In the S-start, \( \lambda(s,t) \neq 0 \) in a small  zone, and as the snake moves forward, \( \lambda(s,t) \) translates along the body without changing its amplitude profile. In contrast, the link density for friction-dominated locomotion \( \lambda_\text{friction}(s,t) \neq 0 \) over most of the snake body and \( |\lambda_\text{friction}| < |\lambda_\text{S-start}| \) when \( \lambda_\text{S-start} \neq 0 \) (where \( \lambda_{\text{S-start}} \) is \( \lambda \) for the S-start); during unstable motion, \( \lambda(s) \) changes its amplitude profile dramatically without cohesive translation.  These observations provide a qualitative understanding of the quantitative results shown in Fig.~\ref{fig3};  a successful S-start can be viewed purely topologically, wherein the snake must have a threshold link current moving from its head to its tail. 

%As the snake rolls forward, a quantum of link, localized in arclength around the contact point, propagates through the snake body.  The link pulse is mostly in the form of writhe, confirming that out-of-plane bending is crucial for S-start motion.  Interestingly, we also see non-trivial \( \tau \), meaning that the snake makes use of all three possible deformation modes, one twisting and two orthogonal bending modes, to perform the S-start.  In Movie S2, we show the evolution of \( \lambda\), \(\omega\) and \( \tau \) over time for S-start, friction-dominated and unstable motion.  We see that the localization and magnitude of the link pulse is less when friction dominates, and there is hardly any propagation of link when inertia dominates.  These observations elucidate the result of Fig.~\ref{fig3}.  Namely, that the crucial prerequisite for a successful S-start can be viewed purely in terms of the snake body's topology: to move forward, the snake must transport a sufficiently large quantum of link from its head to its tail. 


%{ To explore the range of parameters in which the S-start  is physically feasible, we vary the {weight density $\rho g$} and muscle torque $\Gamma a^3$ keeping all the other properties of the filament and the surface unchanged. In Fig.~\ref{fig3}c, we show the scaled forward speed of the filament $v_s/v_m$, as a function of its weight scaled by its muscle torque, $\rho g \ell^2/ \Gamma a$. Our simulations show that as {$\rho g \ell^2/ \Gamma a$} increases beyond a threshold, the speed of the {S-start} falls off rapidly, seen in Fig.~\ref{fig2}e (see SI for details).  {In this regime the snake merely uncoils without a net translation. However this conformational change causes a backward shift of the snake center-of-mass, leading to a negative value of $v_s$.} To compare our results with experimental observations, we also plot the scaled speed of the {S-start} as a function of the scaled weight,  assuming that $\Gamma \simeq 7$ KPa ~\cite{Moon2000,Moon2007,Boback2011}, and see that all the newborn and juvenile anacondas that exhibit successful  S-starts are consistent with our theoretical model, while adults with a weight per unit length that is 100-fold larger than the juvenile ones, fail to move via the S-start.}

Given the simplicity of this escape gait, it is natural to ask when it is feasible in terms of its morphology and musculature. We use the two parameters associated with the scaled weight {$\rho  g a^2 \ell^3/B$} and the scaled muscular torque {$\Gamma a^3\ell/ B$} to classify the phase space of non-planar S-starts. In Fig.~\ref{fig3}d, we see that in a relatively light snake that is over-powered, inertia dominates over weight and friction and the snake flails without much net movement as the snake is unable to maintain sufficient contact with the substrate (SI Movie S2: part 3). For relatively heavy snakes that are under-powered, friction dominates inertia, and the active torques simply uncoil the snake into a straight shape, again leading to little net movement (SI Movie S2: part 2).  However when the relative weight is intermediate, the S-start is favorable and the filament can propel itself at a finite velocity along the direction of its orientation. These observations naturally yield the biomechanical limits of S-starts, and show how developmental transitions allow a juvenile to exhibit these as a means of rapid escape but lose that ability as they grow. In  Fig.~\ref{fig3}c, we plot data from observations of juveniles and adults showing that this is indeed the case.

The non-planar nature of the S-start has some similarities to the sidewinding gait, particularly in the context of how the body bends and writhes into a helical segment even as the snake moves. However there is an important difference: sidewinding is a periodic gait while the S-start is a transient one. Building on previous experimental observations \cite{gray68,Jayne86,Gans92,Marvi2014} that have shown the efficacy of the sidewinding gait in a variety of situations, we now proceed to understand it mathematically.  A natural premise based on our study so far is that the S-start is the building block of a sidewinding gait. To probe this, we periodically initiate the spatiotemporal patterns of the torque triplets used for the S-start at the head and propagate them along the  body (Fig~SI-3a).  We find that as this wave  propagates along the snake, it causes the body to translate orthogonally to its tail-to-head direction, pivoting about multiple contact points which are themselves stationary with respect to the ground (Fig.~SI-4, Movie S3).  We see that sidewinding involves multiple \( \lambda \)-pulses (Figs.~SI-2 and SI-4), each similar to an S-start \( \lambda\) profile propagating through the snake, confirming that the S-start forms the basic building block of sidewinding. 

Our study highlights an unusual transient mode of locomotion in newborn and juvenile anacondas, emphasizing the role of out-of-plane motion in snake locomotion. A new mathematical model for non-planar gaits allows us to quantify the mechanism by which this transient rapid motion can be achieved using a simple set of three localized propagating torques, and a phase diagram shows the regime where the S-start is feasible, consistent with observations. More generally, we show how this transient mode forms the kernel of sidewinding, and is suggestive of how sidewinding might have evolved. 


%Firstly, the S-start  causes the snake to move parallel to its co-linear segments, while sidewinding, as the name suggests, occurs at an angle to the segments. Secondly, the S-start   is inherently a transient mode owing to the finite length of the snake, while sidewinding  circumvents this limitation by {periodically initiating new localized muscular torques near the snake's head. Finally, in the S-start there is just one localized region that remains stationary relative to the ground, and it propagates towards the tail end with a steady velocity. In contrast, in classical sidewinding, the snake lifts itself and moves relative to at least two localized regions that are stationary relative to the ground  (see SI for a quantitative comparison).

{\bf Acknowledgments.} We thank NSF grants BioMatter DMR 1922321, MRSEC DMR 2011754 and EFRI 1830901, the Simons Foundation and the Henri Seydoux Fund (L.M.) for partial financial support.  All experimental data and codes used for simulation will be deposited in a publicly accessible repository. Simulations were carried out using a variant of the code created as part of \cite{Gazzola2018}.%and available at  \href{https://www.cosseratrods.org/}{Pyelastica}.

%Indeed, when accompanied by sensory feedback from the environment  and transitions in neuromuscular activity patterns, we can see a pathway for an organism to adapt its locomotory gait and learn about itself and its environment.

\begin{thebibliography}{10}

\bibitem{gray68}
J.~Gray, {\it Animal Locomotion.\/} (Norton, London, U.K., 1968).

\bibitem{Hu2013}
H.~Marvi, J.~Bridges, D.~L. Hu, {\it J. Roy. Soc. Interface.\/} {\bf 10},
  20130188 (2013).

\bibitem{Gray50}
J.~Gray, H.~W. Lissmann, {\it J. Exp. Biol.\/} {\bf 26}, 354 (1950).

\bibitem{Gasc89}
J.-P. Gasc, D.~Cattaert, C.~Chasserat, F.~Clarac, {\it J. Morphol.\/} {\bf
  201}, 315 (1989).

\bibitem{Moon98}
B.~R. Moon, C.~Gans, {\it J. Exp. Biol.\/} {\bf 201}, 2669 (1998).

\bibitem{Guo2008}
Z.~V. Guo, L.~Mahadevan, {\it Proc. Natl. Acad. Sci. U.S.A\/} {\bf 105}, 3179
  (2008).

\bibitem{Biewener2003}
A.~A. Biewener, {\it Animal Locomotion\/} (Oxford University Press, New York,
  2003).

\bibitem{Hu2009}
D.~L. Hu, J.~Nirody, T.~Scott, M.~J. Shelley, {\it Proc. Natl. Acad. Sci.
  U.S.A\/} {\bf 106}, 10081 (2009).

\bibitem{Alexader2003}
R.~M. Alexander, {\it Principles of Animal Locomotion\/} (Princeton University
  Press, Princeton, NJ, 2003).

\bibitem{Jayne86}
B.~C. Jayne, {\it Copeia.\/} {\bf 1986}, 915 (1986).

\bibitem{Gans92}
C.~Gans, H.~L. Kim, {\it Isr. J. Zool.\/} {\bf 38}, 9 (1992).

\bibitem{Marvi2014}
H.~Marvi, {\it et~al.\/}, {\it Science\/} {\bf 346}, 224 (2014).

\bibitem{Jayne91}
B.~C. Jayne, J.~D. Davis, {\it J. Exp. Biol.\/} {\bf 156}, 539 (1991).

\bibitem{Alfaro2002}
M.~E. Alfaro, {\it Funct. Ecol.\/} {\bf 16}, 204 (2002).

\bibitem{Alfaro2003}
M.~E. Alfaro, {\it J. Exp. Biol.\/} {\bf 206}, 2381 (2003).

\bibitem{Domenici97}
P.~Domenici, R.~W. Blake, {\it J. Exp. Biol.\/} {\bf 200}, 1165 (1997).

\bibitem{Gazzola2012}
M.~Gazzola, W.~M. Van~Rees, P.~Koumoutsakos, {\it Journal of fluid mechanics\/}
  {\bf 698}, 5 (2012).

\bibitem{Kolinski2009}
J.~M. Kolinski, P.~Aussillous, L.~Mahadevan, {\it Phys. Rev. Lett.\/} {\bf
  103}, 174302 (2009).

\bibitem{Prange76}
H.~D. Prange, S.~P. Christman, {\it Copeia\/} {\bf 1976}, 542 (1976).

\bibitem{Anderson79}
J.~F. Anderson, H.~Rahn, H.~D. Prange, {\it Q. Rev. Biol.\/} {\bf 54}, 139
  (1979).

\bibitem{Garcia2006}
G.~J.~M. Garcia, J.~K.~L. da~Silva, {\it Physics Life Rev.\/} {\bf 3}, 188
  (2006).

\bibitem{Gazzola2018}
M.~Gazzola, L.~H. Dudte, A.~G. McCormick, L.~Mahadevan, {\it Royal Society open
  science\/} {\bf 5}, 171628 (2018).

\bibitem{Moon2007}
B.~R. Moon, R.~S. Mehta, {\it The biology of boas and pythons\/}, R.~W.
  Henderson, R.~Powell, eds. (Eagle Mountain, UT: Eagle Mountain Publishing,
  2007), pp. 207--212.

\bibitem{Calugareanu1959}
G.~Calugareanu, {\it Rev. Math. pures appl\/} {\bf 4} (1959).

\bibitem{Klenin2000}
K.~Klenin, J.~Langowski, {\it Biopolymers: Original Research on Biomolecules\/}
  {\bf 54}, 307 (2000).

\end{thebibliography}


%\bibliography{snake}{}
%\bibliographystyle{Science}
\end{document}
