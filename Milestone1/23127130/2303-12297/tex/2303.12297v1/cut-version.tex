\documentclass[%
 aip,
 %aps,
 amsmath,amssymb,
 reprint,
]{revtex4-1}
\bibliographystyle{apsrev4-1}
\usepackage[utf8]{inputenc}
\usepackage{graphicx}
\usepackage[version=4]{mhchem}
\usepackage{bm}
\usepackage{cases}
\usepackage{upgreek}
\usepackage[usenames, dvipsnames]{color}
%\usepackage{ulem}  % this f*cks the bibliography for book titles
\usepackage[normalem]{ulem}
\usepackage[frozencache,cachedir=minted-cache]{minted}
\usepackage{hyperref}

\makeatletter
\def\@email#1#2{%
 \endgroup
 \patchcmd{\titleblock@produce}
  {\frontmatter@RRAPformat}
  {\frontmatter@RRAPformat{\produce@RRAP{*#1\href{mailto:#2}{#2}}}\frontmatter@RRAPformat}
  {}{}
}%
\makeatother
%--------------------------------------------------------------------------------------------
\begin{document}
% compile using 2020 for arxiv compatibility.
\preprint{AIP/123-QED}

\title{Trion formation resolves observed peak shifts in the optical spectra of transition metal dichalcogenides
\vspace{8pt}}

\author{Thomas Sayer}
\affiliation{Department of Chemistry, University of Colorado Boulder; Boulder, CO, USA}

\author{Yusef R. Farah}
\affiliation{Department of Chemistry, Colorado State University; Fort Collins, CO, USA}

\author{Rachelle Austin}
\affiliation{Department of Chemistry, Colorado State University; Fort Collins, CO, USA}

\author{Justin Sambur}
\homepage{Justin.Sambur@colostate.edu}
\affiliation{Department of Chemistry, Colorado State University; Fort Collins, CO, USA}
\affiliation{School of Advanced Materials Discovery, Colorado State University; Fort Collins, CO, USA\looseness=-1}

\author{Amber T. Krummel}
\homepage{Amber.Krummel@colostate.edu}
\affiliation{Department of Chemistry, Colorado State University; Fort Collins, CO, USA}

\author{\\Andr\'{e}s Montoya-Castillo}
\homepage{Andres.MontoyaCastillo@colorado.edu}
\affiliation{Department of Chemistry, University of Colorado Boulder; Boulder, CO, USA} 

%--------------------------------------------------------------------------------------------

\date{14 March 2023}

\begin{abstract}
\vspace{5pt}
Monolayer transition metal dichalcogenides (TMDs) have the potential to unlock novel photonic and chemical technologies if their optoelectronic properties can be understood and controlled. Yet, recent work has offered contradictory explanations for how TMD absorption spectra change with carrier concentration, fluence, and time. Here, we test our hypothesis that the large broadening and shifting of the strong band-edge features observed in optical spectra arise from the formation of negative trions. We do this by fitting an \textit{ab initio}-based, many-body model to our experimental electrochemical data. Our approach provides an excellent, global description of the potential-dependent linear absorption data. We further leverage our model to demonstrate that trion formation explains the non-monotonic potential dependence of the transient absorption spectra, including through photoinduced derivative line shapes for the trion peak. Our results motivate the continued development of theoretical methods to describe cutting-edge experiments in a physically transparent way.
\vspace{20pt}
\end{abstract}

\maketitle

Monolayer transition metal dichalcogenides (ML-TMDs) are promising lightweight and flexible electrode materials for solid-state and electrochemical devices. For light-driven applications such as photovoltaics and photodetectors, optimizing device efficiency requires a fundamental understanding of optoelectronic properties.

Optical spectroscopies of TMDs reveal exotic phenomena, including significant band gap renormalization (BGR) and screening of their quasiparticle binding energies (BES). These two effects determine the energy of the band-edge exciton and tunes its shift as a function of material parameters, such as: TMD composition, substrate, superstrate (often liquid electrolyte in an electrochemical cell), and doping density (here, the presence of electrons in the conduction band).\cite{Berkelbach2018b} For example, the \textit{ab initio} GW~value for the electronic band gap of unsupported ML-\ce{MoS2} is 2.8~eV,\cite{Qiu2013,Ryou2016,Ataei2021,Austin2022}$^,$\footnote{More converged calculations reduce the GW~value to 2.64~eV.\cite{Qiu2015}} whereas its measured band gap is 2.17~eV on graphite\cite{Zhang2014a} and 1.39~eV on gold\cite{Miwa2015}. Crucially, the screening responsible for BGR concomitantly weakens the attraction between electrons and holes in the band-edge excitons. This BES is a countervailing effect which opposes the BGR shift in the optical signal. Hence, only small changes in the optical gap are experimentally observed.\cite{Ugeda2014,Gies2021,Austin2022}

TMDs also exhibit other many-body effects that pose a challenge to their first principles exploration and obfuscate the interpretation of optical measurements. In particular, exciton-electron quasiparticles (negative trions) have been observed in \ce{WS2},\cite{Liu2016b, Zhao2020c, Zipfel2020a, Lan2020, Muir2022}, \ce{WSe2},\cite{Courtade2017,Li2018c,Wagner2020,Lan2020,Yang2022a} \ce{MoSe2},\cite{Li2021c} and \ce{MoS2}\cite{Mak2012, Liu2016b, Kim2019, Huang2022}. Fundamentally, the extent to which trion formation affects the behavior of working devices at room temperature remains unclear.\cite{Lui2014, Goldstein2020a} This is of widespread concern because TMDs are incidentally doped by defects which form during synthesis\cite{Zafar2017,xyzValence2019}, and can also acquire charge from electrolyte solutions\cite{Lin2014}: the resulting conduction band electrons induce trions to form, which causes oscillator strength to transfer from the exciton to the trion signal in optical experiments.\cite{Chernikov2015} A model of minimal complexity that can predict how variations in the microscopic parameters translate to changes in measured quantities, such as spectral features, is therefore essential to disentangle the role of trions and achieve predictive power.

These challenges of interpretation have become prominent in recent years, as experiments have revealed that ML-TMDs display large, complex spectral shifts in device architectures such as photoelectrochemical cells,\cite{Carroll2019} field effect transistors,\cite{Chen2019e, Pradeepa2020} and photovoltaics\cite{Chen2019f}. For example, Sie \textit{et al}.~invoked photoexcitation-induced changes in BGR and BES to explain the spectral shifts observed when pumping ML-\ce{WS2} with increasingly higher laser fluence.\cite{Sie2017} However, since the observations were inconsistent with this explanation, the authors posited a more sophisticated model based on exciton-exciton interactions. More recently, we have reported hot-carrier extraction from a ML-$\rm{MoS}_2$ photoelectrode in a working photoelectrochemical cell which displayed complex changes in the spectral features as a function of applied potential.\cite{Austin2022} As increasingly negative potentials raise the conduction band population, the applied potential affects the magnitudes of BGR and BES. Indeed, we observed a large potential dependence of the position, height, and width of the band-edge ``A-exciton'' peak, analogous to Ref.~\onlinecite{Sie2017}. The potential-dependence at a given time, or `static shift', is evident in the experimental transient absorption (TA)~spectra at 1~ps displayed in Fig.~\ref{fig:initial_spectra}. As the photocarriers relax back to their ground state, the peaks' heights and positions undergo both red and blue `dynamic shift' in time (Fig.~\ref{fig:initial_spectra}(right)). These observations are significant because the mechanism of charge transfer and the role of excitons in these devices are not yet fully understood.\cite{Hong2014,Zimmermann2021,Kim2019,Johansson2020,Xu2021} The difficulty of interpreting these shifts illustrates the need for utilizable and transparent frameworks that facilitate interpretation of experimental data. 

Here, we provide a unifying framework based on the doping-dependent emergence of trions to explain the complex behavior of both potential-\cite{Austin2022} and fluence-dependent\cite{Sie2017, Bera2021} spectra. Since in our experiment both the laser pulse and the applied potential affect BGR and BES, the observation of a qualitatively equivalent behavior\footnote{This study further demonstrated a similar time-dependent shift of the band-edge exciton position over the period of electronic relaxation (see Fig.~4 in Ref.~\onlinecite{Sie2017}).} in a different TMD under entirely different experimental conditions suggests the behavior is a signature of the same underlying photophysical process. Indeed, another recent experiment focusing on $\rm{MoS}_2$ further established the universality of this shift in TMDs.\cite{Bera2021} In our analysis, we make two key observations. First, careful inspection of both sets of spectra suggest that they are not always well described by a single peak. Second, recent advances in GW that include free carrier screening predict that the \ce{WS2} A-exciton should neither redshift nor blueshift with increasing carrier density, and---more strikingly---that the absorption spectrum of $\rm{MoS}_2$ should undergo a pronounced and monotonic blueshift with increasing carrier density.\cite{Ataei2021} Both sets of spectra disagree with the cutting-edge theoretical prediction considering only BGR and BES. We resolve this discrepancy by demonstrating that the source of both static and dynamical peak shifts is the formation of the trion, which is the next-most significant many-body effect in these systems.

\pagebreak
\begin{figure}[!h]
\vspace{-8pt}
  \centering
  \begin{minipage}[c]{0.29\textwidth}
  \vspace{4pt}
    \resizebox{1.0\textwidth}{!}{\includegraphics{TA_1ps_skinny_Aonly.pdf}}
  \end{minipage}
  \begin{minipage}[c]{0.18\textwidth}
    \resizebox{1.0\textwidth}{!}{\includegraphics[trim={2.8cm 9.7cm 8.1cm 4cm},clip]{A_exciton_energy_fig1.pdf}}% change this to resize the caption width without changing figure
  \end{minipage}
  \vspace{-6pt}
  \caption{\label{fig:initial_spectra} Dynamic peak shifts and intensity changes of a ML-\ce{MoS2} electrode in a working electrochemical cell. \textbf{Left}: Potential-dependent TA spectra of the A-exciton region at 1~ps delay time after a 3.1~eV pump pulse. Appendix~Fig.~\ref{fig:sibar1} shows the full range of the data set. All spectra were acquired in 1~M NaI electrolyte versus a Ag/AgI reference electrode and a Pt counter electrode. \textbf{Right}: Plot of the ``superpeak'' position (most negative value) in the TA against time in terms of (\textbf{top}) absolute and (\textbf{bottom}) relative shift. A sub-picosecond redshift in some potentials is followed by a net blueshift, which is most pronounced at low potentials.}
  \vspace{0pt}
\end{figure}

While a number of descriptions of the trion are available\cite{Chang2018, Chang2019a, Torche2019, Zhumagulov2020, Rana2021, Glazov2020b, Katsch2022}, we adopt the Mahan-Nozi{\'e}res-De Dominicis (MND) Hamiltonian\cite{Mahan2000}---a minimal many-body model of a Fermi polaron consisting of a heavy exciton dressed by a free electron bath---which prioritizes the description of doping-dependent phenomena.\cite{Chang2019a} This approach describes how changing the conduction band occupation causes the trion and exciton peaks to shift in energy, change line shape, and transfer oscillator strength between one another, as presented in Fig.~\ref{fig:steadystate}(a). We also note that, like Ref.~\onlinecite{Ataei2021}, it predicts the A-exciton peak to blueshift with increasing doping.
\onecolumngrid

\begin{figure}[h]
\vspace{10pt}
\begin{center} 
    \resizebox{.31\textwidth}{!}{\includegraphics{schematic_ss.pdf}}
    \resizebox{.31\textwidth}{!}{\includegraphics{linearspec.pdf}}  
    \resizebox{.31\textwidth}{!}{\includegraphics{steadystate.pdf}}
\end{center}
\vspace{-16pt}
\caption{\label{fig:steadystate} Summary of the quantitative analysis procedure to model potential-dependent linear absorption spectra with the MND model. \textbf{(a)}~Comparison of theory and simulation for low (0.5~V), medium (0.3~V), and high (0.1~V) doping conditions. Narrow dashed and dotted lines represent exciton and trion absorption peaks, respectively. The broad dash-dotted and solid lines represent the convoluted simulation result and experimental data, respectively. \textbf{(b)}~Schematic of the MND model. More doping corresponds to: a higher Fermi level and a lower applied voltage; oscillator strength shifting to the trion, and the binding energy of the trion (difference between the mid-gap levels) increasing. The BGR occurs exclusively in the valence band.\cite{xyzValence2019} \textbf{(c)}~Potential-dependent experimental linear spectra (points) and the minimal smooths obtained via SavGol filter (solid line). \textbf{(d)}~Manifold of curves for a given set of MND parameters (peak heights, binding energies, line widths) allowed over the doping range 0--30~meV. \textbf{(e)}~The linear spectra for 4~representative voltages are superimposed as fin gray lines, and a close-up of each of the three trion-containing peaks is displayed to the side. The model is parameterised to optimize the fit to all curves simultaneously.}
\end{figure}
%\vspace{\columnsep}
\twocolumngrid

In the following, we interrogate the ability of this framework to capture the features that arise both in our experimental setup\cite{Austin2022} and in the fluence-dependent studies of Refs.~\onlinecite{Sie2017, Bera2021}. As this theory is designed to describe linear spectra, we first ratify its use by capturing the potential-dependent linear absorption spectra shown in Fig.~\ref{fig:steadystate}(c). At this \ce{ITO_{(s)}|MoS2_{(ML)}|I- / I3^-_{(aq)}} interface the band-edge peak moves to higher energy with more positive applied potential. We claim that the two peaks predicted by the MND theory should be convolved with a Gaussian representing the combined effect of broadening mechanisms and the instrument response to obtain the observed ``superpeak''. That is, when the doping increases: the trion redshifts, the exciton blueshifts, both peaks flatten slightly,\cite{Carbone2020b} and the oscillator strength transfers to the trion peak, as sketched in Fig.~\ref{fig:steadystate}(b). Then, the phenomenological broadening of this total signal will reproduce the shift and change in line shape observed in the experiments.

\begin{figure}[!b]
\vspace{-12pt}
\begin{center}
    \resizebox{.45\textwidth}{!}{\includegraphics{montecarlo_t0.pdf}}
\end{center}
\vspace{-18pt}
\caption{\label{fig:TAfit1ps} Comparison of MND theory to the TA spectrum at 1~ps shown in Fig.~\ref{fig:initial_spectra}. The TA spectra are shown as full lines, and the MND prediction is dash-dotted. Each TA curve has had its linear background removed, been cropped at the positions of the photoinduced maxima either side of the bleach signal and baselined (see Appendix~\ref{app:preprocess}). Note: as the model is now parameterized to TA~data, the doping values 0–-30~meV no longer correspond precisely to the original values of Ref.~\onlinecite{Chang2019a}, and should be considered as some arbitrary, internal units.}
\vspace{-4pt}
\end{figure}

Use of the MND description requires fitting to the experimental data because we are currently unable to predict certain quantities from \textit{ab initio} calculations. For example, Ref.~\onlinecite{Chang2019a} assumed that the $\rm{MoS}_2$~ML is a defect-free, intrinsic semiconductor at 0~K and unsupported \textit{in vacuo}. In our experimental cell, the balance of BGR and BES can be expected to differ. We will work with the theory not by optimizing for these Hamiltonian parameters by repeatedly solving a many-body equation, but rather by modelling it at the level of its outputs: peak positions, heights, and widths as functions of doping. This is convenient because these functions are predicted to have a simple (near)-linear form,\cite{Chang2019a} such that the gradients and intercepts of these functions become the only free parameters. While this fitting offers no guarantee that a given set of functions all correspond to the same (or any) Hamiltonian, it serves as an accessible starting point for interpreting multifaceted experimental data sets. We describe our protocol for how to extract the model parameters from the potential-dependent photoelectrochemical spectra, which matches the spectra with a manifold of spectral curves as a function of applied potential (e.g. Fig.~\ref{fig:steadystate}(d)), in Appendix~\ref{app:optimization-procedure}.  

Figure~\ref{fig:steadystate}(e) shows the result for our linear absorption spectra, which demonstrates the model provides an excellent global fit to the data. As Figs.~\ref{fig:steadystate}(a)~and~(e) illustrate, the static shift of the superpeak with applied potential is mostly due to the transfer of oscillator strength between exciton and trion, with a smaller contribution from the shifting of the underlying peaks. As the potential becomes more positive, the superpeak shifts from the trion position to the exciton position. 

Can this simple explanation for the static shift shed light on the source of the more complex dynamical shift presented in Fig.~\ref{fig:initial_spectra}? After all, TA~is a difference-method where the signal arises from the interplay of the pump and probe pulses, with $\Delta A(\tau) = A(\tau) - A_\mathrm{SS}$, where $A(\tau)$ is the absorption of the excited material, $\tau$ units after the pump, and $A_\mathrm{SS}$ is the unpumped steady-state spectrum. 

To simplify the analysis, we discard the TA~data before $1$~ps, where there is a qualitatively different regime which has been assigned to relaxed free carriers at the band-edge waiting to form excitons.\cite{Ceballos2016, Cunningham2017} However, regardless of the time chosen, we must baseline correct the TA~spectra to isolate the peaks for modelling (see Appendix~\ref{app:preprocess}). This introduces a complication since the initial pump pulse, which generates carriers in both the valence and conduction bands, alters the BGR and BES contributions seen by the probe pulse. The resulting shift of the TA~peaks due to this renormalization results in photoinduced absorption features that appear either side of the main bleach signal,\cite{Pogna2016,Carroll2019,Kastl2022} which is a general effect that arises in the TA~of semiconductors.\cite{Price2015} For us, the trion and exciton peaks contaminate each other in a non-trivial, doping-dependent way. Yet, as we show in the following, our approach allows us to discern the emergence of the trion peak in TA~data, its interplay with the exciton peak, and how these peaks subtly shift in TA~experiments. 

Figure~\ref{fig:TAfit1ps} demonstrates that our approach offers a compelling fit to the TA data at $1$~ps, reproducing the observed shifts, peak splittings, and height changes. Given the minimal, but physics-based, nature of the model, some disagreement may be expected. After all, our baseline correction misses the transient effects of photoinduced BGR. Moreover, even if the linear spectra are well described by this model, there is no guarantee the same holds in the TA. In fact, one can readily observe (even in Fig.~\ref{fig:initial_spectra}(left) before the processing) that the height of the superpeak unexpectedly increases as the potential is lowered from 0.55~V to 0.4~V, before it falls and shifts, as previously observed in the linear spectra. While this might indicate that the MND theory is not applicable here, the fit of Fig.~\ref{fig:TAfit1ps} captures the trend near-quantitatively. 

How can the model accurately capture these TA-specific changes in peak height? The linear spectra are subject to a sum rule where the total oscillator strength is a conserved quantity, and the relative share between trion and exciton is all that changes with carrier density. Our algorithm does not incorporate this rule because TA~does not follow a similar requirement. Inspecting the results of the fit for the $1$~ps data of Fig.~\ref{fig:TAfit1ps} shows that making the potential more negative causes the trion intensity to rise more than it causes the exciton intensity to fall. When combined with the small binding energy at low doping (the peaks begin close together), this allows for the initial growth of the superpeak not observed in the linear spectra. Thus, we can assign this effect to the breaking of the sum rule. 

We now run the fit at each time point over our $>200$~ps of TA data, using the results at the previous time point as the initial guess. This yields a surprising result: the trion appears with positive $\Delta A$ at low doping (see Fig.~\ref{fig:TAresults}(a)). This assignment would be impossible in linear absorption, but contemporaneous theoretical developments suggest that our fitting algorithm is actually yielding physically correct results. To show this we combined the 2D electronic spectroscopy of the MND Hamiltonian\cite{Lindoy2022} with the projection-slice theorem\cite{Hamm2011} to simulate the (early time) TA spectra of the ML-MoS$_2$ MND Hamiltonian. Figure~\ref{fig:TAresults}(c) shows the result of these calculations. While the nonlinear spectroscopic treatment is qualitatively similar to the linear case, the trion exhibits a derivative line shape at low doping that results in positive intensity on the lower energy side of the exciton peak. This outlines the dangers of deconvolving TA~spectra by fitting positive Gaussian peaks, as the pump pulse can brighten and shift the trion peak compared to the linear spectrum, as well as induce interference with the exciton.\cite{Tempelaar2019}

\begin{figure}[!t]
%\vspace{-16pt}
\begin{center}
    \resizebox{.29\textwidth}{!}{\includegraphics{results_heights.pdf}}
    \resizebox{.18\textwidth}{!}{\includegraphics{lindoy_TA.pdf}}
\end{center}
\vspace{-14pt}
\caption{\label{fig:TAresults} \textbf{Left}: Plots showing how the repeated fitting of the MND model to the TA~data over time, this gives the equivalent of Fig.~\ref{fig:TAfit1ps} at each time point. Gaps in the series are due to experimental anomalies described in Appendix~\ref{sec:TAresults}. \textbf{(a)} Contribution at each voltage due to the trion peak. Note how the trion is assigned positive values at intermediary voltages (low, but non-zero doping). \textbf{(b)} Contribution at each voltage due to the exciton peak. The `hill’-like feature around 4~ps is a signature of the model straining to allow for the positive trion signal.  \textbf{Right}: \textbf{(c)} Projection-slice TA spectrum from the 2D response function code of Ref.~\onlinecite{Lindoy2022}. At low dopings, the derivative shape of the trion is seen as a partly-positive feature for potentials up to 10~meV. Solid curves are the result of convolving with a Gaussian. The doping levels are appropriate for the original model Ref.~\onlinecite{Chang2019a} and not the post-optimization Fig.~\ref{fig:TAfit1ps}.}
\end{figure}

We return to our initial query concerning the dynamical shift of the superpeaks in the TA spectra first presented in Fig.~\ref{fig:initial_spectra}(right). As Fig.~\ref{fig:TAresults}(a)~and~(b) illustrate, the exciton and trion peaks both lose intensity over time. This is reasonable, as we expect the relaxation of photoexcited species to lower the TA~bleach signal. We offer further confirmation of this in Appendix Fig.~\ref{fig:si1}, which shows that the model's assignment of the doping level is generally stable over time. Moreover, the time series also reveal that both the trion and exciton intensities decay exponentially, with the trion feature decaying faster than the exciton (Appendix Fig.~\ref{fig:si2}). This behavior can arise if trions and excitons decay slower than they interconvert: the relaxation or trapping of excitons causes the trion population to collapse as the equilibrium moves to stabilize the ratio of each species.\cite{Huang2022} Such dynamical equilibrium would imply that, as the material relaxes, the superpeak shifts from a position between the exciton and trion signals to being almost entirely atop the exciton peak. 

From this, one can address all aspects of the dynamical shift of Fig.~\ref{fig:initial_spectra}(right). For the $0.25$~V and $0.30$~V spectra, where the trion and exciton begin with similar intensity, the pronounced blueshift is mainly due to this change in relative heights. Indeed, both spectra converge to the high-potential/low-doping superpeak positions at long delay times. Additionally, the trion and exciton positions move with changing doping levels: the exciton is predicted to blueshift with increasing carrier density, while the trion is predicted to redshift. The shift in the underlying peaks thus explains the smaller movement of the superpeak at the highest and lowest doping levels considered, where the signal is dominated by a trion or an exciton, respectively. In general, the total shift has a contribution from both the changing ratio of peak heights and the changing ratio of BGR and BES. We summarize these effects in Fig.~\ref{fig:explanation} for the particular case of the $0.30$~V TA spectrum. 

This simple physical picture also explains why the superpeak redshifts before $1$~ps at intermediate voltages: the doping level is approximately constant (so the exciton and trion positions are fixed) but the formation of quasiparticles from free carriers has not completed.\cite{Ceballos2016, Cunningham2017} While the strongly enhanced absorption characteristic of free carriers is visible, the ratio of excitons to trions has not reached equilibrium. The superpeak redshifts because the trion intensity grows and saturates with time.

\begin{figure}[!t]
\vspace{-12pt}
\begin{center}
    \resizebox{.4\textwidth}{!}{\includegraphics{schematic.pdf}}
\end{center}
\vspace{-24pt}
\caption{\label{fig:explanation} Cartoon illustration showing the relationship between bandgap, carrier concentration, and exciton/trion absorption peaks as a function of time at a fixed applied potential of $E = 0.3 $~V. The ratio of trion to exciton decreases with time, which causes the superpeak to shift from between the two peaks at $1$~ps to entirely atop of the exciton peak at $25$~ps. The trion peak generally blueshifts over time whereas the exciton remains fairly constant, resulting in a blueshift of the superpeak, as originally observed in Fig~\ref{fig:initial_spectra}.}
\end{figure}

We arrive at a point where the MND~theory of Ref.~\onlinecite{Chang2019a} is in excellent agreement with our experimental steady-state absorbance and TA~data. We believe the explanation developed here also applies to the data in Refs.~\onlinecite{Sie2017,Bera2021} before the onset of the Mott transition. Here, the use of this minimal model to disentangle the spectral signals as a function of potential and time leaves one observation unexplained. For the spectra at the extreme potentials, we claimed that the peak shift is explained by the doping dependence of the exciton or trion signal. Applying the model independently to each time point resulted in deconvolved exciton and trion peak positions that quantitatively agree with the experimental data as a function of time (compare Appendix Fig.~\ref{fig:si3} with Fig.~\ref{fig:initial_spectra}(right)). Yet, both exciton and trion peaks net-blueshift in time. This observation fundamentally contradicts the logic that the exciton should blueshift with \textit{increasing} carrier density. Since carrier density decreases with time, the current model cannot accommodate this observation without further extension. We emphasize that, had we applied a data-driven technique, such as SMCR,\cite{Ruckebusch2020, DeJuan2007} without carefully including the physical constraints placed by the underlying Hamiltonian, the optimization procedure would likely have subsumed this subtle effect into a better overall fit; such an approach runs the risk of obtaining an aesthetically pleasing but physically misleading result. 

Several factors can cause the exciton peak to blueshift over time. First, temperature tunes the absorption spectrum and especially the trion signal.\cite{Zhang2014, Christopher2017} Importantly, we neglect nonequilibrium energy relaxation after photon absorption, which leads to changing temperatures and even phonon bottlenecks.\cite{Chi2020,Wang2021h} Ultimately, the timescale required for the lattice to cool the electronic degrees of freedom in comparison to our TA~timescales remains unclear. Second, BGR and BES changes may depend on the identity of the quasiparticles in the system at a particular time.\cite{Cunningham2017} For example, biexcitons,\cite{Li2018c, Trovatello2022, Muir2022} the next-highest-order quasiparticle, are not considered in our model. Finally, while the MND theory assumes holes with infinite mass in the exciton and trion, holes in the real system have an associated momentum.\cite{Qiu2015a} What additional physics is still required to constitute a minimal model therefore remains an open question.

Thus, we have interrogated the ability of trion formation as described by the MND Hamiltonian to explain the salient features in our linear and TA spectra. While a fully \textit{ab initio} treatment of the problem remains beyond the scope of existing theory, we have developed an algorithm to exploit the behavior of the spectra obtained for this model as a function of conduction band population to constrain a minimal parametrization of the data. Our physically meaningful approach captures the evolution of the trion and exciton peaks as a function of applied potential and thus offers a means to critically assess the ability of this model to describe the full range of observed phenomenology. Our results demonstrate that our simple approach is sufficiently robust to interpret, assign, and analyze the spectral features in linear and time-resolved TA spectra of doped ML-TMDs. Importantly, our analysis explains how complex, time-, doping-, and fluence-dependent spectral shifts previously attributed solely to the A-exciton instead arise from the convolution of the emergence and shift of the trion peak and the shift in the exciton peak. The appearance of the trion at room temperature, under working conditions, suggests that a model of TMD photophysics restricted to BGR and BES is insufficient to capture both static and dynamic spectral signatures, and that the trion state must therefore be included in the minimal theoretical framework for spectroscopic measurements of TMD-containing systems.


\section{Acknowledgements}
This research was supported by the U.S.~Department of Energy, Office of Science, Office of Basic Energy Sciences, under Award DE–SC0021189 (J.B.S., R.A.), and under Award DE-SC0016137 (A.T.K., Y.R.F.). A.M.C.~acknowledges the start-up funds from the University of Colorado, Boulder. T.S.~and A.M.C.~thank Lachlan Lindoy and David Reichman for sharing their 2D~spectroscopy code published in Ref.~\onlinecite{Lindoy2022}.

\onecolumngrid
\vfill
\pagebreak
\twocolumngrid
%--------------------------------------------------------------------------------------------
%--------------------------------------------------------------------------------------------
\appendix*
\documentclass[aps,
               prb,
               preprint,
               unsortedaddress,
               superscriptaddress,
               showkeys,
               notitlepage]{revtex4-1}

\usepackage[utf8]{inputenc}
\usepackage[T1]{fontenc}
\usepackage{graphicx}
\usepackage{amsmath, amsfonts, amssymb, mathtools, xcolor}	
\usepackage[breaklinks, linkcolor = black]{hyperref}
\usepackage{bm}					% Bold Greek letters in math mode.
\usepackage{latexsym} 				% Additional symbols. 
\usepackage{tikz}
\usepackage{tabularx}
\usepackage{braket}
\usepackage{diagbox}

%\usepackage[labelfont=bf, format=plain, justification=justified]{caption}
\usepackage{setspace}

\graphicspath{{figures/}}
\newcommand{\red}[1]{\textcolor{red}{#1}}

%\bibliographystyle{}

   \makeatletter
     \renewcommand\@make@capt@title[2]{%
      \@ifx@empty\float@link{\@firstofone}{\expandafter\href\expandafter{\float@link}}%
       {\textbf{#1}}\@caption@fignum@sep#2\quad}%
     \makeatother
 
\makeatletter 
\renewcommand{\fnum@figure}{\textbf{Fig.~\thefigure}}
\makeatother
\renewcommand{\thefigure}{S\arabic{figure}}

\def\theequation{S\arabic{equation}}

\begin{document}

\title{{Supporting Information \\ Reduced absorption due to defect-localized interlayer excitons in \\ transition metal dichalcogenide--graphene heterostructures}}


\author{Daniel \surname{Hernang\'{o}mez-P\'{e}rez}}

%\email[]{daniel.hernangomez@weizmann.ac.il}

\affiliation{Department of Molecular Chemistry and Materials Science, Weizmann Institute of Science, Rehovot 7610001, Israel}


\author{Amir \surname{Kleiner}}

\affiliation{Department of Molecular Chemistry and Materials Science, Weizmann Institute of Science, Rehovot 7610001, Israel}

\author{Sivan Refaely-Abramson}

%\email[]{sivan.refaely-abramson@weizmann.ac.il}

\affiliation{Department of Molecular Chemistry and Materials Science, Weizmann Institute of Science, Rehovot 7610001, Israel}

\keywords{2D materials, transition-metal dichalcogenides, heterostructures, defects, graphene, excitons}

\maketitle

\begin{center}
 \vspace{-.6cm} 
 Email: daniel.hernangomez@weizmann.ac.il \\ \hspace{1.8cm} sivan.refaely-abramson@weizmann.ac.il
 \end{center}

\tableofcontents
\clearpage


\section{Computational details and methods}\label{app:computational}

%
\subparagraph{Geometry.} As a starting point for the supercell optimization in the presence of the vacancy (see details below), we consider the geometry of a pristine heterobilayer  with a lattice parameter whose length corresponds to {an average} nearest-neighbor metal-metal distance (equal to the in-plane monolayer TMDC lattice constant) of $\bar{d}_{\textnormal{X--X}} = 3.15 $\,\AA \, both for MoS\textsubscript{2} and WS\textsubscript{2}. 
%
This value is almost equal to the experimental monolayer lattice parameter of both TMDC (MoS\textsubscript{2}; 3.15 \AA\, \onlinecite{Nicklow1975} and WS\textsubscript{2}; 3.153 \AA\, \onlinecite{Schutte1987}).
%
As stated in the main text, the supercell is composed of $4\times4$ TMDC unit cells and $5\times 5$ graphene unit cells. Therefore, it is made of $97$ atoms and possesses a rhomboedral shape with in-plane lattice vectors of length $|\mathbf{R}_{1,2}| \simeq 12.6$ \AA\, (see Fig. \ref{f5}).
%
The vacuum distance between the periodic replicas in the out-of-plane direction was taken to be $\sim 10$ \AA\, and the average interlayer distance within the supercell was $\bar{d}_\textnormal{inter} = 3.43$ \AA\, {both for Mo and W}. 


\begin{figure}[h]
   \includegraphics[width=0.6\linewidth]{f4.pdf}

%
    \caption{Top view of the WS\textsubscript{2}--Gr supercell. Four supercells are shown, each supercell forms a parallelepiped whose lateral boundaries are given by the straight red lines (the in-plane supercell lattice vectors are labeled as $\mathbf{R}_1$, $\mathbf{R}_2$). The chalcogen vacancy, located in the TMDC layer on the opposite side of the graphene layer, is indicated by a red triangle.
    }\label{f5} 
\end{figure}
% DFT
\subparagraph{Density functional theory.} The DFT calculations were performed employing the implementation of DFT of \textsc{Quantum Espresso}. \cite{Giannozzi2009, Giannozzi2017} %
We used the non-empirical PBE generalized gradient approximation for the exchange--correlation functional \cite{Perdew1996}.
%
We employed a plane-wave basis set and included spin-orbit interaction by means of full relativistic norm-conserving pseudopotentials \cite{Dojo2019}. We considered a basis cut-off of $50$ Ry for both TMDC--Gr interfaces.
%
The self-consistent charge density was converged within a $6 \times 6 \times 1$ \textbf{k}-grid, with Fermi-Dirac smearing of $10^{-5}$ Ry for fractional occupations. The calculation was considered to be converged only if the total energy difference between consecutive iterations within the self-consistent field cycle was smaller than the threshold value of $10^{-9}$ Ry.

The supercells were initially preoptimized with VASP \cite{Kresse1996} in the absence of chalcogen vacancies. For the exchange-correlation functional a local density approximation (LDA) was employed, with a basis set energy cut-off of $600$ eV. The self-consistent charge density for the geometry relaxations was converged in a $6\times 6 \times 1$ \textbf{k}-grid as well. 
%
The supercells were subsequently relaxed, fixing only the position of the supercell lattice vectors and optimizing the position of the TMDC atoms within the supercell after the chalcogen atom was removed. 
%
The position of the atoms was relaxed until all components of the forces were smaller than a threshold value of $10^{-3}$ Ry/$a$\textsubscript{0}.
%
This second optimization was done with \textsc{Quantum Espresso}, using PBE and the van-der Waals corrected functional \texttt{vdw-df-09} \cite{Thonhauser2007, Troy2009, Berland2015} to properly account for changes in the interlayer separation.

% GW-BSE
\subparagraph{GW.} Using the DFT wavefunctions and energies as a starting point, we computed the corrected quasi-particle energy spectrum by performing a one-shot non-self-consistent GW calculation (G\textsubscript{0}W\textsubscript{0}). Our GW calculations were performed with the package Berkeley{GW}, including spin-orbit interaction \cite{Deslippe2012, Rohlfing2000, Wu2020, Louie2022}.
%
The dielectric function was obtained with the generalized plasmon-pole model of Hybertsen-Louie \cite{Hybertsen1986}.
%
We employed a cut-off of $5$ Ry in the dielectric screening and a total of \textcolor{black}{$2499$} states for the summation over the occupied and unoccupied states.
%
We used a non-uniform neck subsampling scheme to sample the Brillouin zone close to $|\mathbf{q}| \rightarrow 0$ and speed up  the convergence with respect to the 
\textbf{k}-grid sampling \cite{Qiu2017}. Within this scheme, we considered a $6\times 6 \times 1$ uniform \textbf{q}-grid and $10$ additional \textbf{q}-points around $\mathbf{q}= \mathbf{0}$.
%
A truncated Coulomb interaction was considered in the perpendicular direction to the heterostructure to prevent spurious interactions between the periodic replicas in this direction \cite{Ismail2006}. \textcolor{black}{This set of parameters yields converged quasiparticle gaps within 100 meV.}


\subparagraph{BSE.}
%
%
To study the excitonic features, we solved the Bethe-Salpether equation (BSE) \cite{Rohlfing1998, Rohlfing2000} 
\begin{equation}
    (E_{c\mathbf{k}} - E_{v\mathbf{k}}) A_{vc\mathbf{k}}^S + \sum_{v'c' \mathbf{k}'} K^{\textnormal{eh}}_{vc\mathbf{k}; v'c'\mathbf{k}'} A_{v'c'\mathbf{k}'}^S  = \Omega_S A_{vc\mathbf{k}}^S,\label{e3}
\end{equation}
where $E_{c\mathbf{k}}$ (resp. $E_{v\mathbf{k}}$) are the quasiparticle energies of the conduction (resp. valence) bands, $K^{\textnormal{eh}}_{vc\mathbf{k}; v'c'\mathbf{k}'}  = \langle vc \mathbf{k}|\hat{K}^{\textnormal{eh}}|v'c' \mathbf{k}' \rangle$ are the matrix elements of the electron-hole interaction kernel, defined from the addition of an attractive screened direct and a repulsive bare exchange Coulomb contributions, $\Omega_S$ is the exciton energy and $A^S_{vc\mathbf{k}}$ is the amplitude of the exciton state $|\Psi^S \rangle$.
%
This equation sets an eigenvalue problem, $\hat{H}^\textnormal{BSE}|\Psi^S \rangle = \Omega_S |\Psi^S \rangle$, where the matrix elements of the BSE Hamiltonian in the electron-hole basis are given by
\begin{equation}
    {H}^\textnormal{BSE}_{cv\mathbf{k};c'v'\mathbf{k}'} = (E_{c\mathbf{k}} - E_{v\mathbf{k}}) \delta_{c,c'}\delta_{v,v'}\delta_{\mathbf{k},\mathbf{k}'} + K^{\textnormal{eh}}_{vc\mathbf{k}; v'c'\mathbf{k}'}.
\end{equation}
This representation of the BSE assumes that the (real-space) direct exciton wavefunction is described as the coherent superposition of electrons and holes at each $\mathbf{k}$-point,
\begin{equation}
   \langle \mathbf{r}_e, \mathbf{r}_h | \Psi^S \rangle = \sum_{vc\mathbf{k}} A^S_{vc\mathbf{k}}  \psi^\ast_{v\mathbf{k}}(\mathbf{r}_h) \psi_{c\mathbf{k}}(\mathbf{r}_e), 
\end{equation}
with $\psi_{c\mathbf{k}}(\mathbf{r}_e)$ being the spinor wavefunction describing the electron at position $\mathbf{r}_e$ with conduction band quantum number $c$ and crystal momentum $\mathbf{k}$ (correspondingly, $\psi_{v\mathbf{k}}(\mathbf{r}_h)$ is the spinor wavefunction describing a hole at position $\mathbf{r}_h$ and characterized by the valence band quantum number $v$ and same crystal momentum $\mathbf{k}$).

%%%%%%%%%%%%%%%%%%%%%%%%%%%%%%%%%%%%%%%%%%%%%%%%%%%%%%%%%%%
Eq. \eqref{e3} was solved using the Berkeley{GW} package \cite{Deslippe2012, Rohlfing2000, Wu2020}. 
%
The matrix elements were computed on a Monkhorst-Pack  $9 \times 9 \times 1$ \textbf{k}-grid and the result interpolated to a uniform \textcolor{black}{$27 \times 27 \times 1$ \textbf{k}-grid} that we employ in any subsequent analysis and calculation.
%
We employed the Tamm-Dancoff approximation and evaluated the Coulomb interaction kernel for all possible transitions between pairs of bands $(n,m) \rightarrow (n', m')$, with an energy cut-off of $5$ Ry for the dielectric matrix within the electron-hole kernel matrix elements.
%
We considered for the main paper a total of \textcolor{black}{$28$ bands ($14$ valence and $14$ conduction bands)} in the absorption calculations, which include both the defect bands as well as all the relevant low energy pristine valence and conduction bands of the heterobilayer.
%
%
%
\textcolor{black}{These parameters converged the calculated excitonic spectra within 100 meV for the defect-dominated subgap features and below 10 meV for the most prominent absorption resonances in the region where intralayer non-defect TMDC excitons become more relevant.}
%
In the absorption calculation, we also avoid the heavy calculation of the $\mathbf{q}$-shifted wavefunctions on the interpolation grid by evaluation of the matrix elements of the velocity operator, $\hat{\mathbf{v}}$, instead of the momentum, $\mathbf{p} = \mathfrak{i} \nabla$. This involves neglecting terms in the sums proportional to $|\langle 0 |[\hat{V}_{\textnormal{ps}}, \hat{\mathbf{r}}]| S \rangle|^2$, where $\hat{V}_{\textnormal{ps}}$ is the non-local part of the pseudopotential \cite{Rohlfing2000, Deslippe2012}. Including the non-local terms has been shown to not qualitatively change the shape of water X-ray absorption spectrum \cite{Qiu2022}.


\subparagraph{Projected density of states.} We compute the layer contribution of each band to a given exciton state, first by obtaining the \textbf{k}-projected density of states (DoS) of each layer, $l = \{\textnormal{WS}_2, \textnormal{Gr}\}$ from the \textbf{k}-resolved projected DoS
\begin{equation}\label{e1}
    g^{l}_{n\mathbf{k}}(E) = \sum_{\{i_A,A\} \in l} |\langle \phi_{i_A}^A| \psi_{n\mathbf{k}} \rangle|^2 \delta(E - E_{n\mathbf{k}}),
\end{equation}
where $|\psi_{n\mathbf{k}} \rangle $ and $E_{n\mathbf{k}}$ are the Kohn-Sham states and energies and the sum runs over atoms $A$ and orbitals $i_A$ of the corresponding layer $l$.
%
We further normalize this quantity for each layer as $\tilde{g}(E) = g(E) / \textnormal{max}[g(E)]$ so that 
\begin{equation}\label{e2}
\tilde{g}_{n\mathbf{k}}(E) = \tilde{g}^{\textnormal{WS}_2}_{n\mathbf{k}}(E) - \tilde{g}^{\textnormal{Gr}}_{n\mathbf{k}}(E),
\end{equation}
 is defined in the range $[-1,1]$. This way, $\tilde{g}_{n\mathbf{k}}(E) = -1$ corresponds exclusively to graphene contribution and $\tilde{g}_{n\mathbf{k}}(E) = 1$  exclusively to TMDC contribution.

 \subparagraph{Heterostructure decomposition.} 
 We employ Eq. \eqref{e2} to display the color of the band contributions to the exciton decomposition  as well as the absorbance decomposition into intralayer and interlayer parts in the main paper. In particular, for Fig. 2 and Fig. \textbf{S5}, the contributions to a given conduction band are obtained as $\sum_{vc\mathbf{k}} \tilde{g}_{v\mathbf{k}}|A^S_{vc\mathbf{k}}|^2$ while the contributions to a given valence band result from $\sum_{vc\mathbf{k}} \tilde{g}_{c\mathbf{k}}|A^S_{vc\mathbf{k}}|^2$. 
 %
 The \textbf{k}-resolved decomposition in Fig. 3(b) of the main paper (see also Figs. \ref{f18}, \ref{f7}, \ref{f8}) are obtained using the same expressions but without the summation over the crystal momentum.


\subparagraph{Absorbance.} 

From the absorption, we compute the associated absorbance using\cite{Li2009}
\begin{equation}
A(\omega) = \dfrac{\omega L_z}{c}  \epsilon_2(\omega),
\end{equation}
where $\omega$ is the photon frequency, $c$  the speed of light, $L_z$  the distance between the heterostructure and its periodic replicas and $\epsilon_2(\omega)$  the imaginary part of the dielectric function.

\subparagraph{Exciton binding energy.} 
% 
The excitonic binding energies are calculated from the difference between the expectation values of the diagonal and the full BSE Hamiltonian,
\begin{align}
    E^S_\textnormal{bind} &=
    \langle \Psi^S | \hat{H}^{\textnormal{BSE}} - \hat{K}^{\textnormal{eh}} | \Psi^S \rangle - \langle \Psi^S | \hat{H}^{\textnormal{BSE}} | \Psi^S \rangle, \notag \\
    &=\sum_{vc\mathbf{k}}|A^S_{vc\mathbf{k}}|^2 (E_{c\mathbf{k}} - E_{v\mathbf{k}}) - \Omega_S,\label{eq:binding}
\end{align}
where $\hat{H}^{\textnormal{BSE}}$ is the BSE Hamiltonian and  $\hat{K}^{\textnormal{eh}}$ the electron-hole interaction kernel; $E_{\alpha\mathbf{k}}$, the quasi-particle bands; and $\Omega_S$, the exciton energy.

\subparagraph{Intrinsic decay rate.}
To compute the intrinsic decay rate of the zero-momentum excitons we follow Refs. \onlinecite{Spataru2005, Palummo2015, Bernardi2018}
 \begin{equation}
      \gamma_S = \dfrac{4 \pi e^2}{m^2 c} \dfrac{\mu_S}{A_{c} \Omega_S},
 \end{equation}
 where $m$ is the electron mass, $\mu_S$ is the BSE oscillator strength of the exciton with energy $\Omega_S$ and $A_c$ the area of the supercell.

\subparagraph{Non-defected heterostructure.} The absorbance data for the non-defected heterobilayer was taken from the forthcoming publication \onlinecite{Kleiner2023}, where a GW-BSE calculation in a relaxed supercell of the same size but without vacancy was performed.

\section{Additional results}

\subparagraph{Geometry optimization.}
%
After relaxation in the presence of the vacancy, we find that close to the defect position, the nearest-neighbor metal-metal distance decreases to $d_{\textnormal{W--W}} = 3.02 $\,\AA\, and $d_{\textnormal{Mo--Mo}} = 3.05 $\,\AA. This corresponds to a reduction of about $3-4$\% in units of the TMDC monolayer lattice constant, $\bar{d}_{\textnormal{X--X}}$. Because the supercell volume is constant during the relaxation, the TMDC layer is strained in the vicinity of the defect where the distance to the next-nearest-neighbor metallic atoms increases up to $\simeq 3.2$ \AA\, for the functional employed here. Similar behavior is also observed for the sulfur atoms surrounding the vacant site, which rigidly follow the motion of the metallic atoms.
%
Geometry optimization also reduces the average interlayer distance in the heterostructure by around $3-4\%$. 
% 
In particular, we find a reduction of the interlayer distance to $\bar{d}_\textnormal{inter} = 3.29$ \AA\, in the case of MoS\textsubscript{2}--Gr and $\bar{d}_\textnormal{inter} = 3.31$ \AA\, for WS\textsubscript{2}--Gr for our relaxation criteria and our employed van der Waals scheme.

\subparagraph{DFT bandstructures.}
For completeness, we show the bandstructures obtained using the DFT relaxed structures in the presence of spin-orbit interaction in Fig. \ref{f17}.

\begin{figure}
   \includegraphics[width=1.0\linewidth]{figures/f17.pdf}

     %
    \caption{Band structures of the TMDC--Gr heterostructures in the presence of spin-orbit interaction computed along the $\bar{\textnormal{M}}-\bar{\textnormal{K}}-\bar{\Gamma}$ path in the supercell (see inset) (a) WS\textsubscript{2}--Gr heterobilayer (b) MoS\textsubscript{2}--Gr heterobilayer.}\label{f17} 
\end{figure}

\subparagraph{GW.}
%
As stated in the main text, our GW results qualitatively follow the DFT band structures discussed in Ref. \onlinecite{Hernangomez2023}. We summarize here for completeness the main features. In essence, the band structures of graphene and the TMDC appear mostly superimposed, with the graphene Dirac cone centered at the $\bar{\textnormal{K}}$ (and $\bar{\textnormal{K}}$') points. The Dirac point sets the charge neutrality point %
within the pristine TMDC band gap. 
%
The combination of the missing chalcogen atom and the symmetry of the host lattice results in four empty in-gap and two occupied spin-orbit split bands. %
In addition, the reduction of the original TMDC symmetry due to graphene adsorption, and the residual defect-defect interaction that results from the lattice mismatch between layers within the supercell as well as the supercell size (\textit{i.e.} the defect density) make these bands non-degenerate and weakly dispersive. The breaking of the degeneracy is stronger in the vicinity of the $\bar{\textnormal{K}}$ and $\bar{\textnormal{K}}'$ points.
%
Defect states also hybridize with the graphene states in certain regions of the supercell Brillouin zone. In particular, the interlayer hybridization is relatively important between the defect and the graphene bands in a ring around the $\bar{\textnormal{K}}$ and $\bar{\textnormal{K}}'$ points. This interlayer hybridization was already  measured \cite{Batzill2015, Pierucci2016} and predicted \cite{Jain2018} for occupied bands in MoS\textsubscript{2}--Gr far from the charge neutrality point.

%%%%%%%%%%%%%%%%%%%%%%%%%%%%%%%%%%%%%%%%%%%%%%%%%
Graphene adsorption and geometry relaxation within the supercell breaks the original lattice symmetry \cite{Hernangomez2023}, lifting the degeneracy of the TMDC conduction states at the $\Lambda$ and K \textbf{k}-points, both folded into the point $\bar{\textnormal{K}}$ of the supercell Brillouin zone due to lattice commensuration \cite{Kleiner2023}. For WS\textsubscript{2}--Gr, this results in the lowest conduction band having $\Lambda$ nature with K states being at higher energy 
%
already at the DFT level. 
%
The identification is performed from the pseudo-charge density, $|\Psi|^2$ at $\mathbf{k} = \bar{\textnormal{K}}$, which provides the orbital contribution, comparing  %
to the pristine TMDC states (see also Ref. \onlinecite{Kleiner2023}). 
%
Screening affects more strongly the more delocalized K states (essentially due to the defect effect), shifting them higher in energy compared to the folded $\Lambda$ states (by $\sim 0.25$ eV), the latter determining thus the pristine band gap. %
For MoS\textsubscript{2}--Gr, %
already at the DFT level the K bands are below the $\Lambda$ bands by $\sim 0.1$ eV and this situation reverses at the GW level, where the K bands appear $\sim 0.15$ eV above. %
This again gives a relative shift of $\sim 0.25$ eV, suggesting that in both heterostructures the levels shifting results from the graphene dielectric screening.

% 
The screening and exchange renormalize also the graphene Fermi velocity associated to the slope of the Dirac cone.
%
In particular, we obtain an increase of the Fermi velocity of $\sim 34$\% for WS\textsubscript{2}--Gr and $\sim 41$\% for  MoS\textsubscript{2}--Gr, consistent with a reported increase of the Fermi velocity of $\sim 34$\% in Ref. \onlinecite{Li2009} and slightly larger than the $\sim 18\%$ reported in Ref. \onlinecite{Olevano2008} (both calculations for the isolated graphene monolayer).
% 
In addition, we find that for  WS\textsubscript{2}--Gr (resp. MoS\textsubscript{2}--Gr) the Fermi energy shifts from $3.80$ eV in DFT to $4.68$ eV in GW with respect to the vacuum level (resp. $3.60$ eV to $4.50$ eV).
%

%%%%%%%%%%%%%%%%%%%%%%%%%%%%%%%%%%%%%%%%%%%%%%%%%%%%%%%%%%%%%%%%%%%%%%%%%
\section{Absorption convergence checks}

\begin{figure}
   \includegraphics[width=1.0\linewidth]{figures/f9.pdf}

     %
    \caption{Brillouin zone \textbf{k}-grid convergence of the absorption spectrum for different supercell sampling. (a) WS\textsubscript{2}--Gr heterobilayer (b) MoS\textsubscript{2}--Gr heterobilayer.}\label{f1} 
\end{figure}

\begin{figure}
   \includegraphics[width=0.6\linewidth]{figures/f16.pdf}  

     %
    \caption{Convergence of the absorption spectrum of WS\textsubscript{2}--Gr heterobilayer with the number of bands inclued in the BSE calculation.}\label{f16} 
\end{figure}

Ensuring a qualitatively $\mathbf{k}$-grid converged absorption is crucial for understanding the defect-induced phenomena presented in this paper.
%
In Fig. \ref{f1}, we show the absorption spectrum for different supercell Brillouin zone uniform \textbf{k}-grid samplings of the interpolated absorption grid for (a) WS\textsubscript{2}--Gr (b) MoS\textsubscript{2}--Gr .
%
We find that the absorption spectra are quantitatively well converged %for the computationally available grids 
in the visible range, especially above optical energies $\gtrsim 2 $ eV for WS\textsubscript{2}--Gr and $\gtrsim 1.75 $ eV for MoS\textsubscript{2}--Gr. In this energy range, convergence is achieved already with a  $24 \times 24 \times 1$ \textbf{k}-grid up to $\sim 10$ meV. However, the position and height of the absorption resonances in the infrared range, especially for energies roughly below $0.3-0.5$ eV, are not yet fully converged. This region, which  is also known to be strongly dominated by intraband graphene resonances \cite{Li2009},  requires very dense \textbf{k}-grids for smoothness and quantitative convergence. 
%
In the intermediate optical range, up to $\sim 1.6-1.7$ eV, the results are qualitatively converged, but the main absorption features are still quantitatively dependent on the employed \textbf{k}-grid. At   energies between $\sim 0.5-1.6$ eV, the results qualitatively agree between different \textbf{k}-grids, and, importantly, the height of the absorption resonances is stable. The importance of this observation for defect-based sub-gap features and their expected survival in the limit of dense \textbf{k}-sampling is discussed the main text.
% 
Finally, we also ensured that the excitonic features discussed in the main paper are converged in the number of bands included in the BSE calculation, see \textit{e.g.} Fig. \ref{f16} 


\section{Absorbance spectra and defect resonances in MoS\textsubscript{2}-graphene}\label{sec:MoS2}

 \begin{figure}
    \includegraphics[width=0.65\linewidth]{f12.pdf} 
%
     \caption{Absorbance and exciton contributions for the defected MoS\textsubscript{2}--Gr heterobilayer. (Top) Absorbance calculated along one of the main in-plane polarization directions, as well as its decomposition into graphene and MoS\textsubscript{2} contributions (interlayer contributions are read from the difference of the three traces). The dashed horizontal black line marks the $2.4\%$ universal limit of graphene absorbance at infrared energies.
    The shaded box represents the estimated range for which we expect a smooth and monotonic spectrum dominated by graphene (instead of resonances that result from finite \textbf{k}-grid sampling). The vertical dotted line denotes the optical ranges below which excitons transition from being dominated by defect-graphene sub-gap transitions to have transitions which involve pristine TMDC bands.
    %
    (Bottom) For each exciton composing the absorbance resonances, we represent the contribution of each electron and hole bands. Each dot corresponds to the band contribution to a given exciton summed over all $\mathbf{k}$ points (only bright contributions whose oscillator strength are $> 5$ a.u. are shown). For reasons of clarity, all dots with value $\geq 10^3$ a.u. have the same area. 
    %
    The color code corresponds to the layer composition  of each contribution and the dotted box marks the position of the transitions towards graphene-defect empty bands.  
     }\label{f3} 
 \end{figure}
As mentioned in the main text, the strong mixing that occurs in  WS\textsubscript{2}--Gr is also qualitatively similar for the case of the MoS\textsubscript{2}--Gr interface.  We show in Fig. \ref{f3}, top panel, the absorbance spectrum, its decomposition into intralayer graphene, intralayer TMDC and interlayer contributions as well as the graphene universal absorbance limit at infrared energies (dashed horizontal line). Here, however, the impact of the defect bands in the absorbance is more dramatic since the defect bands are closer to the Dirac point and thus interlayer contributions affect the absorbance even at lower energies. Consequently, for MoS\textsubscript{2}--Gr  the absorbance spectrum is very different to that obtained for this TMDC  with chalcogen vacancies in the absence of  graphene layer \cite{Mitterreiter2021, Amit2022}. 
%
In Fig. \ref{f3}, bottom panel, we show the band contribution of each  exciton. Note that, as compared to WS\textsubscript{2}--Gr, here the four spin-orbit split defect bands can be identified clearly as being below the graphene Dirac cone already at optical energies $\sim 0.5$ eV. 



%%%%%%%%%%%%%%%%%%%%%%%%%%%%%%%%%%%%%%%%%%%%%%%%%%%%%%%%%%%%%%%
\section{Wavefunction densities of relevant conduction bands}\label{app:wfn}

\begin{figure*}
   \includegraphics[width=1.0\linewidth]{f7.pdf}
   %
    \caption{(a) Top view of the Kohn-Sham pseudo-charge density, $|\Psi_{n,\mathbf{k}}(\mathbf{r})|^2$, for the conduction band states, derived from the monolayer K point. These densities are evaluated at the point $\mathbf{k} = \bar{\textnormal{K}}$ of the supercell Brillouin zone of the defected WS\textsubscript{2}--Gr heterobilayer.
    (b) Same as in (a) but for the states derived from the folded $\Lambda$ point, at $\mathbf{k} = \bar{\textnormal{K}}$. (c) Same as in (a) but for the two defect states with different ``orbital'' quantum number. \cite{Hernangomez2023}
    }\label{f4} 
\end{figure*}

In order to understand how the defect has such a strong impact on the absorption features, it is instructive to look at the wavefunction densities of relevant states that participate in the transitions. In Fig. \ref{f4}, we show the pseudo-charge density for three sets of bands at the relevant $\bar{\textnormal{K}}$ point of the supercell Brillouin zone: (a) states with K nature, \textit{i.e.} which have orbital contribution resembling the orbitals found in the monolayer at the K point; (b) states originally coming from the monolayer $\Lambda$ point, folded into $\bar{\textnormal{K}}$, and (c) defect states. 
%
While the states with K nature are weakly affected by the missing atom forming the vacancy, the perturbation due to the missing atom  affects very strongly the $\Lambda$-like states, which have a new orbital pattern that varies at longer length scales (nanometer scale). As such, we expect these states to react differently to the dielectric screening once this is taken into account. This difference can explain therefore, as stated in the main text, why the pristine TMDC band-gap at the GW level is determined by the states shown in Fig. \ref{f4} (b) and not in (a). For comparison, we display in (c) the pseudo-densities of two defect states. These states present an orbital signature consistent with that seen in panel (b), therefore, we conclude that the behavior and shape of the $\Lambda$-derived states is exclusively determined by the vacancy.

\clearpage

\section{Defect-induced exciton hybridization}



\begin{figure}[h]
   \includegraphics[width=0.575\linewidth]{f18.pdf} 

%
    \caption{Brillouin zone exciton distribution plotted for all the WS\textsubscript{2}--Gr excitons within an energy window of $\pm 5 $ meV marked in the absorption spectrum by \textcircled{1} (centered at 2.19 eV), \textcircled{2} (centered at 2.4 eV) and \textcircled{3} (centered at 2.7 eV). Even a small energy window close to the center of the excitation peak shows larger hybridization between the WS\textsubscript{2} and the graphene layers.
    }\label{f18} 
\end{figure}

\begin{figure}[h]
   \includegraphics[width=0.575\linewidth]{f19.pdf} 

%
    \caption{Sketch of most prominent transitions and transition band diagram for the absorption peaks in Fig. \ref{f18}. The excitons have all been added up within an energy window of $\pm 5 $ meV.
    }\label{f19} 
\end{figure}



\clearpage 

\section{Additional figures}\label{app:add_figs}
\vspace{-0.2cm}
\begin{figure}[h]
   \includegraphics[width=0.5\linewidth]{f6.pdf}

%
    \caption{Exciton energies, $\Omega_S$, represented as a function of the binding energy, $E_\textnormal{bind}$, for the excitons in the MoS\textsubscript{2}--Gr heterobilayer.
    %
    The binding energy is computed using Eq. \eqref{eq:binding}. We only show the excitons with $E_\textnormal{bind} > 2.5$ meV (around $\sim 14000$ out of $142 884$ excitons for this \textbf{k}-grid sampling and number of bands). The size of each dot is proportional to the oscillator strength (rescaled by a factor of two for visibility).
    }\label{f6} 
\end{figure}

\begin{figure}[h]
   \includegraphics[width=0.575\linewidth]{f14.pdf} 
    \caption{Brillouin zone exciton distribution for the two most bound excitons  in Fig. \ref{f6}. The Brillouin zone in the top row displays $\sum_{v}|A^S_{cv\mathbf{k}}|^2$, while the lower row shows $\sum_{c}|A^S_{cv\mathbf{k}}|^2$. These two excitons present  graphene-defect transitions and are largely delocalized in \textbf{k}-space.
    }\label{f7} 
\end{figure}

\begin{figure}[h]
   \includegraphics[width=0.6\linewidth]{f13.pdf} 
    \caption{Same as in Fig. \ref{f7} but for two graphene dominated excitons. Both excitons are mostly localized in the vicinity of the $\bar{\textnormal{K}}$ valleys. These excitons have a large transition dipole, with the value of $\mu_S$ being $1.5 \cdot 10^2$ a.u. and $1.2 \cdot 10^2$ a.u. respectively, but small binding energy.
    }\label{f8} 
\end{figure}

\begin{figure}
   \includegraphics[width=0.6\linewidth]{figures/f11.pdf}

   %
     %
    \caption{Intrinsic radiative lifetimes at low temperatures for the grey excitons with binding energy larger than $25$ meV, both for WS\textsubscript{2}--Gr (blue circles) and MoS\textsubscript{2}--Gr (red squares). The size of the points is proportional to the oscillator strength, rescaled for clarity by a factor of 20. 
    %
    }\label{f11} 
\end{figure}

\begin{figure}[h]
   \includegraphics[width=0.6\linewidth]{f15.pdf} 
    \caption{Intrinsic radiative lifetimes at low temperatures for the bright excitons in the WS\textsubscript{2}--Gr heterobilayer with and without defects (data for the pristine heterobilayer taken from Ref. \onlinecite{Kleiner2023}). The size of the symbols is proportional to the oscillator strength $\mu_S$, which is chosen to be larger than the threshold value of $10^1$ a.u.. For clarity,  we fix the value of $\mu_S$ to be $10^{3}$ a.u if larger or equal. The grey shaded area corresponds to the region of the spectrum strongly dominated by graphene.
    %
    We observe that the heterostructure without the vacancy has brighter excitons with systematically shorter lifetimes and larger oscillator strengths in the visible region, where the defected structure shows exciton quenching. 
    %
    }\label{f9} 
\end{figure}

\clearpage
\bibliography{biblio}

\end{document}


\vfill\pagebreak
~~~
\pagebreak
\subsection*{References}
\vspace{-14pt}
\bibliography{Postdoc-TMDC}

\end{document}
