
\usepackage{hyperref}
\usepackage{url}
\usepackage[margin=1in]{geometry}
\usepackage{amsmath,amssymb,amsthm, amsfonts}
\usepackage[T1]{fontenc}
\usepackage{mathtools}
\usepackage{xcolor}
\usepackage{enumitem, comment, xifthen}
\usepackage{graphicx}
\usepackage{mathabx} %
\graphicspath{{figs/}}
\interfootnotelinepenalty=1000
\DeclarePairedDelimiter{\vecnorm}{\lVert}{\rVert}
\DeclarePairedDelimiter{\matnorm}{\vvvert}{\vvvert}
\newcommand{\twonorm}[2][]{\vecnorm[#1]{#2}_{2}}
\newcommand{\opnorm}[2][]{\matnorm[#1]{#2}_{\rm op}}
\newcommand{\fronorm}[2][]{\matnorm[#1]{#2}_{\rm F}}

\makeatletter
\def\letterdef#1#2#3{\def\letterdef@##1{\expandafter\def\csname #1\endcsname{#2}}%
  \letterdef@@#3{?\@car{}}\@nil}
\def\letterdef@@#1{\@gobble#1\letterdef@{#1}\letterdef@@}
\makeatother

\DeclarePairedDelimiterX{\klx}[2]{(}{)}{%
  #1\;\delimsize\|\;#2%
}
\DeclarePairedDelimiterX{\quantklx}[3]{(}{)}{%
  #1\;\delimsize\|\;#2\;\delimsize\vert\;#3%
}
\DeclarePairedDelimiterX{\inner}[2]{\langle}{\rangle}{%
  #1,#2%
}

\newcommand{\R}{\mathbf R} %
\newcommand{\C}{\mathbf C} %
\newcommand{\N}{\mathbf N} %
\newcommand{\Z}{\mathbf Z} %
\newcommand{\Q}{\mathbf Q} %
\newcommand{\F}{\mathbf F} %


\letterdef{c#1}{\mathcal{#1}}{ABCDEFGHIJKLMNOPQRSTUVWXYZ}
\letterdef{b#1}{\mathbb{#1}}{ABCDEFGHIJKLMNOPQRSTUVWXYZ}
\let\defn\coloneq
\newcommand{\twomax}[2]{\ensuremath{#1 \lor #2}}
\newcommand{\twomin}[2]{\ensuremath{#1 \land #2}}
\newcommand{\ceil}[1]{\left\lceil #1 \right\rceil}
\newcommand{\floor}[1]{\left\lfloor #1 \right\rfloor}
\newenvironment{enummath}
 {\begin{enumerate}[font=\upshape,label=(\alph*)]}
 {\end{enumerate}}
\newcommand{\e}{\mathrm{e}}
\DeclareMathOperator{\sign}{\bf sign}
\newcommand{\ud}[0]{\mathrm{d}}  %
\newcommand{\1}{\mathbf 1} %
\let\ones\1
\newcommand{\half}{\frac12} 
\let\epsilon\varepsilon
\newcommand{\eps}{\varepsilon}

\let\tilde\widetilde
\let\subseteq\subset
\let\succeq\succcurlyeq
\let\preceq\preccurlyeq
\renewcommand{\le}{\leqslant}
\renewcommand{\ge}{\geqslant}
\renewcommand{\leq}{\leqslant}
\renewcommand{\geq}{\geqslant}


\newcommand{\argmin}{\mathop{\rm arg\,min}}
\newcommand{\argmax}{\mathop{\rm arg\,max}}
\DeclareMathOperator{\prox}{\bf prox}
\DeclareMathOperator{\conv}{\bf conv}
\DeclareMathOperator{\relint}{\bf rel int}
\DeclareMathOperator{\graph}{\bf gph}
\newcommand{\Rext}{\overline{\R}} %


\DeclareMathOperator{\cl}{\bf cl}
\DeclareMathOperator{\Lip}{Lip}
\DeclareMathOperator{\diam}{diam}
\DeclareMathOperator{\dist}{\bf dist}
\DeclareMathOperator{\trace}{\bf Tr}
\DeclareMathOperator{\diag}{\bf diag}
\DeclareMathOperator{\dom}{\bf dom}
\DeclareMathOperator{\rank}{\bf rank}
\DeclareMathOperator{\Int}{\bf int}
\renewcommand{\det}{\mathrm{\bf det}}
\newcommand{\spn}{\mathrm{span}}
\let\Span\spn %
\newcommand{\T}{\mathsf{T}}
\newcommand{\had}{\circ}
\newcommand{\kl}{D_{\rm kl}\klx}
\newcommand{\quantkl}{D_{\rm kl}\quantklx} %
\newcommand{\tv}[1]{\left\| #1 \right\|_{\rm TV}}
\let\TV\tv


\newcommand{\iid}{\textnormal{i.i.d.}}
\newcommand{\ind}{\textnormal{ind.}}
\newcommand{\simiid}{\stackrel{\iid}{\sim}} %
\newcommand{\simind}{\stackrel{\ind}{\sim}} %
\newcommand{\distto}{\stackrel{\rm d}{\longrightarrow}} %
\newcommand{\asto}{\stackrel{\rm a.s.}{\longrightarrow}} %
\newcommand{\probto}{\stackrel{\rm p}{\longrightarrow}}  %


\newcommand{\atoms}{\mathrm{atoms}}
\newcommand{\Var}{\mathrm{Var}}
\newcommand{\Med}{\mathrm{Med}}
\newcommand{\Ent}{\mathrm{Ent}}
\newcommand{\Cov}{\mathrm{Cov}}
\DeclareMathOperator{\supp}{supp}
\makeatletter
\newcommand{\E}{\operatorname*{\mathbf{E}}\ilimits@}
\makeatother
\makeatletter
\renewcommand{\P}{\operatorname*{\mathbf{P}}\ilimits@}
\makeatother


\newcommand{\etal}{\textit{et al}. }
\newcommand{\ie}{\textit{i}.\textit{e}., }
\newcommand{\eg}{\textit{e}.\textit{g}., }
\newcommand{\etc}{\emph{etc}.}
\newcommand{\red}[1]{\textcolor{red}{#1}}
\newcommand{\green}[1]{\textcolor[rgb]{.1,.6,.1}{#1}}
\newcommand{\blue}[1]{\textcolor{blue}{#1}}
\newcommand{\rpcomment}[1]{{\bf{{\green{{RP --- #1}}}}}}
\newcommand{\lxcomment}[1]{{\bf{{\blue{{LX --- #1}}}}}}


\newcommand\blfootnote[1]{%
  \begingroup
  \renewcommand\thefootnote{}\footnote{#1}%
  \addtocounter{footnote}{-1}%
  \endgroup
}

\newcounter{algorithmctr}
\renewcommand{\thealgorithmctr}{\arabic{algorithmctr}}
\newenvironment{algdesc}%
   {\refstepcounter{algorithmctr}\begin{list}{}{%
       \setlength{\rightmargin}{0\linewidth}%
       \setlength{\leftmargin}{0\linewidth}}%
       \rmfamily\small
       \item[]{\setlength{\parskip}{0ex}\hrulefill\par%
        \nopagebreak{\bfseries\textsf{Algorithm \thealgorithmctr~}}}}%
   {{\setlength{\parskip}{-1ex}\nopagebreak\par\hrulefill} \end{list}}

\makeatletter
\long\def\@makecaption#1#2{
        \vskip 0.8ex
        \setbox\@tempboxa\hbox{\small {\bf #1.} #2}
        \parindent 1.5em 
        \dimen0=\hsize
        \advance\dimen0 by -3em
        \ifdim \wd\@tempboxa >\dimen0
                \hbox to \hsize{
                        \parindent 0em
                        \hfil 
                        \parbox{\dimen0}{\def\baselinestretch{0.96}\small
                                {\bf #1.} #2
                                } 
                        \hfil}
        \else \hbox to \hsize{\hfil \box\@tempboxa \hfil}
        \fi
        }
\makeatother
