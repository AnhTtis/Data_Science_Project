\section{Introduction}\label{sec:intro}


Holant problems are also known as edge-coloring models.
They can express a broad class of counting problems,
such as counting matchings,
%({\sc \#Matchings}),
perfect matchings
({\sc \#PM}),
proper edge-colorings, cycle coverings,
and a host of counting orientation problems such as counting
Eulerian orientations or the six-vertex model.
%
Every counting constraint satisfaction problem (\#CSP)  can be expressed
as a Holant problem. 
%The framework of Holant problems is {\em provably} more expressive than the framework of counting satisfaction problems (\#CSP). Precisely,
%Every \#CSP problem can be easily expressed
%as a Holant problem. 
On the other hand,
Freedman, Lov\'asz and Schrijver~\cite{freedman2007reflection} proved that the prototypical Holant problem
\#PM cannot be expressed 
as a graph homomorphism function (vertex-coloring model)
by any real valued constraint function.
%This was extended to
This is  true even for 
complex valued constraint functions~\cite{CaiG19}. 
%Thus Holant problems
%are strictly more expressive.

Some problems are \#P-hard in general, yet computable on
planar graphs. 
The problem \#PM is such a problem~\cite{Valiant79,Jerrum87}.
%JerrumSV04}.
A most fascinating algorithm---the FKT algorithm~\cite{kasteleyn1967graph,kasteleyn1961statistics, temperley1961dimer}---computes \#PM in polynomial time (FP, polynomial-time computable functions)
for planar graphs. 
%One of the most fascinating of all algorithms, 
Valiant  introduced  holographic algorithms
which are non-parsimonious
%holographic 
reductions to the FKT
 algorithm, 
 %and these algorithms
 placing  many planar counting problems  in FP
 that 
 %otherwise 
 seemed to be intractable.
To understand these algorithms
 a signature theory was developed and the Holant
 framework was introduced.
 %, which led to the study of Holant problems.
 Stated in this signature theory,
  Valiant's holographic algorithms boil down to
  what constraint functions (signatures) can be realized by the so-called matchgate signatures
  under a holographic transformation.
  Delineating the precise
boundary of FP tractability for these problems has been a central focus in the classification theory of
  counting problems~\cite{Backens21,CaiF17,CaiFGW22,GuoW20,CaiLX17,Valiant08}.
 A general theme has emerged:
 %of the classification emerges:
 for  very broad classes of counting problems, one can
 classify every problem in the class to be of
 exactly one of three types: (1) 
 FP, (2) \#P-hard in general but FP
 on planar graphs, or (3)  \#P-hard on planar graphs. Furthermore, for all \#CSP on Boolean variables (which includes vertex models),  Valiant's holographic algorithm
  is a universal
 algorithm~\cite{CaiF17} that solves problems in (2)\footnote{However, this is not true for Holant problems in general~\cite{CaiFGW22}. If one recalls that
the FKT algorithm solves  \#PM for planar graphs, which is
\emph{the} prototypical  Holant problem but not a vertex model,
%graph homomorphism problem,  
this is particularly intriguing.}. 
 %However this is not true for Holant problems.
In this paper, we prove that for a class of bipartite
Holant problems, this three-way classification \emph{holds}.
However,
 there are \emph{two methods} 
for planar tractability in type (2): In addition to
holographic
transformations to matchgates, there is another type which
combines this transformation with a global argument. Either method alone is
not, but \emph{together} they do form, a
universal strategy for planar tractability.
%restricted

%
% there are some planar Holant problems in FP
% that are hard in general, and yet not  which are not transformable to matchgates~\cite{CaiFGW22}.

%  After In hindsight, the problems considered in~\cite{Valiant08} are planar variants of some similarly looking problems that are known to be \#P-hard, and the first-ever holographic algorithms work by expressing those problems in terms of matchgates which can then be solved in polynomial time by the famous FKT algorithm~\cite{temperley1961dimer, kasteleyn1961statistics, kasteleyn1967graph}. The theory of matchgates is then further developed~\cite{CaiCL09, CaiL10, CaiG14} and it is proven in~\cite{CaiLX17} that they capture all planarly tractable \#CSP problems that are hard in general. However, there are some planarly tractable Holant problems that are hard in general which are not transformable to matchgates~\cite{CaiFGW22}.

We briefly define  
Holant
problems on Boolean variables. An
input is a signature grid $\Omega$
consisting of a graph $G = (V,E)$ with 
each vertex $v$  labeled by a constraint function $f_v$
(also called a signature).
The Holant problem is to compute a sum-of-product
$\operatorname{Holant}(\Omega) = 
\sum_{\sigma: E \rightarrow \{0, 1\}} 
\prod_{v\in V}f_v(\sigma|_{E(v)})$,
where $E(v)$ denotes the incident edges  of $v$.
E.g., \#PM is the counting problem where each $f_v$ is the 0-1 valued \textsc{Exact-One} function.
%of arity $\deg(v)$, which is 1 iff its input has exactly one 1.
In planar Holant problems, denoted by $\operatorname{Pl-Holant}$,
$G$ is required to be  planar, 
and $f_v$ takes inputs from 
$E(v)$ which is given a cyclic order starting from some edge  (specified by $\Omega$).


%\sum\limits_{\sigma: E \rightarrow \{0, 1\}} 
%\prod\limits_{v\in V}f_v(\sigma|_{E(v)})$
%{\rm wt}(\sigma)$ where $wt(\sigma):= \prod\limits_{v\in V}f_v(\sigma|_{E(v)})$ is the product over all constraint functions with their inputs given by the adjacent edges. 




In this paper, we study a class of Holant problems
%with Boolean domain 
whose input graphs are planar, 3-regular and bipartite.
More precisely, let $f(x, y, z)= [f_0, f_1, f_2, f_3]$
be any ternary constraint function  which takes value $f_i \in  {\mathbb{Q}}$, if the input has Hamming weight $i$.  We allow both positive and negative
values.
 We study  \plholant{f}{(=_3)},
%${\rm wt}(x, y, z) = i$. We write  $f We study  \plholant{f}{(=_3)},
the Holant problem on planar, 3-regular bipartite
graphs where LHS vertices are assigned $f$ and RHS
vertices are assigned a ternary equality $(=_3)$.
Without  planarity, a complexity dichotomy was proved for 
 these bipartite Holant problem in~\cite{CaiFL21}.
 %~\cite{FanC21}.
 Planarity plus regularity add considerable difficulty.
%we 
%studied these bipartite Holant problem in~\cite{FanC21}
%and proved a complexity dichotomy.



%Bipartiteness and regularity
%impose a severe restriction and it causes significant
%technical difficulties. In particular, the only
%gadget on either side of the bipartite problem
%has a arity a multiple of 3.
%The planarity restriction is on its incident graph.


%The class of Holant problems we consider in this paper encodes

One can think of them as counting problems on
3-regular 3-uniform hypergraphs, or set systems where
every subset has cardinality 3 and every element appears in 3 subsets.
The planarity refers to its (bipartite) incidence  graph.
These include some well studied problems.
One long-standing open problem raised by Moore and Robson in~\cite{MooreR01}
is counting
\texttt{Cubic-Planar-X3C}, 
%Cubic Planar X3C 
(X3C stands for \textsc{Exact-3-Cover}), or equivalently Cubic Planar Monotone 1-in-3 SAT. Expressed as a Holant problem it is 
\plholant{[0,1,0,0]}{(=_3)}, 
where $[0,1,0,0]$ is the ternary \textsc{Exact-One} function.
%where $[0,1,0,0]$ is the ternary Exact-One function.
Schaefer~\cite{Schaefer78} proved that Monotone 1-in-3 SAT is NP-complete. 
Lichtenstein~\cite{Lichtenstein82} first considered the complexity of many
 planar  problems,
%%Planar Formulae and Their Uses
%David Lichtenstein
%   Published 1 May 1982
%    SIAM J. Comput.
    and Laroche~\cite{Laroche93} proved that
Planar Monotone 1-in-3 SAT is NP-complete.
Monotone 1-in-3 SAT is the same as X3C.
Dyer and Frieze~\cite{DyerF86} proved the NP-completeness of  
%Planar 1-in-3 SAT,
%and 
Planar X3C and 3DM where 
 each element is in either 2 or 3 subsets (of
cardinality 3).
Moore and Robson~\cite{MooreR01}, in a reduction
using ingenious combinatorial gadgets, showed that
this problem remains NP-complete when 
each element is in exactly 3 subsets. 
%Another equivalent form of the problem is Cubic Planar Monotone 1-3 SAT.
However, they 
noted that they  were not able to conclude the \numP-hardness of its counting version, which is 
precisely
 \plholant{[0,1,0,0]}{(=_3)}, while all previous NP-complete proofs listed here
 do extend to \#P-hardness for its counting version.
We observe that  these proofs are combinatorial, and they become
 increasingly  more delicate with  planarity and regularity restrictions.
%reduction proof becomes more delicate.
%and more delicate and difficult.


Our proof is carried out using
 the machinery of
signature theory developed in the study
of Holant problems. These are algebraic proofs
which show that the underlying combinatorial constructions succeed.
%have the desired properties.
%Its success 
This machinery demonstrates the power of using {\em algebraic} method 
to prove complexity results which are combinatorial in nature.
%presented
%as {combinatorial} questions.
Indeed, this is exactly in the spirit of Valiant's holographic algorithms
which use arithmetic cancellations to achieve reductions
that are globally valid for counting, but solutions do not
correspond in a 1-1 fashion (i.e.,
non-parsimonious reductions). 




%Due to impose a severe restriction In~\cite{FanC21}, we initiated the study of bipartite Holant problem by proving a dichotomy theorem for \holant{f}{(=_3)} where $f = [f_0,f_1,f_2,f_3]$ is a ternary Boolean function with $f_i \geq 0$ where $1 \leq i \leq 3$. We strengthen the dichotomy result to its planar variant, identifying the only additional tractable problems are those transformable to matchgates by holographic reduction. 

%Arguably, 
One difficulty in working with 3-regular bipartite Holant problems is the severe limitation on the gadgets that  can be possibly constructed. One can  show that
 on either side of the bipartite problem,
 every constructible gadget defines 
a constraint function having  arity 
a multiple of 3. So in particular, one cannot directly
produce unary signatures, or binary signatures on either side.
One \emph{can} produce  ``straddled'' signatures that take some input variables
from one side and some 
%input variables 
from the other. Typically a ``degenerate'' signature
is not very useful in the proof of a dichotomy theorem. A counter-intuitive
idea from~\cite{CaiFL21} is to utilize straddled and degenerate
signatures, to ``virtually'' produce unary signatures.
This idea led to a complexity dichotomy for these bipartite
counting problems in the setting that ignores planarity.
%On the other hand, 
But the essence of Valiant's  holographic algorithm
and the study of Holant problems is to account for planar
tractability, and
we know there are problems in this class
that are \#P-hard in general but in FP
%P-time  tractable 
on planar graphs.
%, thus \#P-hardness proofs in~\cite{FanC21}
%cannot work in the planar setting.

In this paper we settle that by proving a planar complexity dichotomy.
%that includes the planar case.
A major technical challenge 
%for our proof 
%in this paper 
is how to ``virtually'' produce   unary signatures in a planar way.
%As a  crucial step
We prove
a pure graph-theoretic result that says that, except
in some trivial cases, every  
3-regular plane graph~\footnote{
A 3-regular graph is also called a \textit{cubic} graph. Properties of cubic planar graphs have been studied extensively~\cite{HoltonM88, AldredBHM00, NoyRR20, HeckmanT06, Scheim74}.
} has
a planar 3-way edge perfect matching (P3EM). 
We use it as an essential ingredient to the proof of the dichotomy.
%theorem.
This result should be of independent interest.
%~\footnote{3-regular plane graphs are delicate. E.g., the 4-color theorem is equivalent to
%every bridge-less 3-regular plane graph has a 3-edge proper coloring.}
%Our theorem
%on P3EM says that edges can be assigned to $\{0, 1\}$
%such that the sum on the boundary of every face is a multiple of 3.}.
%
%has to faces such that the number of 
%edges assigned to every face
%is a multiple of 3.}.
The proof technique to prove this  matching theorem is a combination of
algebraic and combinatorial methods.
This theorem lets us virtually
``manufacture'' and then ``absorb''
unary signatures in the \#P-hardness reduction. 
% Indeed, one can prove that
%no direct combinatorial construction can produce such constraint functions.
This allows us to carry out the needed \#P-hardness
 reductions in a planar way.

% In~\cite{FanC21}, we tackled it by interpolating degenerate straddled binary functions and use them as unaries. However, %{\em a priori} there is no guarantee that the 
% this method generally violates planarity.
 %will respect the planarity. 
% Indeed, some $f$ define 
 %there are some problems 
 %(specified by certain $f$)
% that are 
%\#P-hard problems without planarity, but
% are in FP for planar graphs. Thus, one cannot
% hope that all  \#P-hardness proofs of~\cite{FanC21}
% go through in the planar case. 
%As  a crucial step
% in this paper we prove a pure graph-theoretic theorem which states that
 %essentially 
%  every 3-regular plane graph (except in some trivial cases)
%  has 
%a planar 3-way edge perfect matching.
 %to assign each edge to one of its adjacent faces so that each face gets $0 \pmod{3}$ edges assigned. 
%This allows us to carry out the needed \#P-hardness
% reductions in a planar way.
 %show the reduction in~\cite{FanC21} can be done in a planar way.


%The class of Holant problems considered in this paper is only the beginning
%the simplest yet 
%a non-trivial 
%steps towards a classification of all planar bipartite Holant problems. Given the richness of results from this starting point, it is likely
%conceivable that
%many more interesting phenomena are yet to be discovered. 
%Together with them, the understanding of the structure of bipartite Holant problems merely started. 
