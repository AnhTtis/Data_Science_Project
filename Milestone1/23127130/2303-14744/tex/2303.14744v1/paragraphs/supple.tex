\section{Experimental Details}
\noindent \textbf{Datasets.}
The number of test images used for evaluation is listed in Table~\ref{tb:dataset_summary}. For experiments on COCO, we use the validation split for evaluation. 

\begin{table*}[t]
\caption{Details of the evaluated datasets.}
\label{tab:dataset-summary}
\vskip 0.15in
\centering
\begin{tabular}{@{}lllllll@{}}
\toprule
             & \begin{tabular}[c]{@{}l@{}}Number of\\ Series\end{tabular} & \begin{tabular}[c]{@{}l@{}}Time Steps\\ per Series\end{tabular} & Period  & Sampling & \begin{tabular}[c]{@{}l@{}}Number of\\ Features\end{tabular} & Data Split [\%] \\ \midrule
\solarSmall  & 8                                                          & 2,000                                                            & 01--03 2018    & 60m      & 8                                                            & 60/15/25   \\ 
\solarOneY   & 50                                                         & 8,760                                                            & 2019    & 60m      & 8                                                            & 60/15/25   \\ 
\solarThreeY & 50                                                         & 26,304                                                           & 2018--20 & 60m      & 8                                                            & 34/33/33   \\ \midrule
\airTen      & 12                                                         & 35,064                                                           & 2013--17 & 60m      & 11                                                           & 34/33/33   \\ 
\airTwenty   & 12                                                         & 35,064                                                           & 2013--17 & 60m      & 11                                                           & 34/33/33   \\ \midrule
\sapflux     & 24                                                         & \begin{tabular}[c]{@{}l@{}}15,000--\\ 20,000\end{tabular}          & 2008--16 & varying  & 10                                                           & 34/33/33   \\ \midrule
\hydro       & 531                                                        & 9,862                                                          & 1981--2008 & 24h      & 41                                                           & 34/33/33   \\ \bottomrule
\end{tabular}
\end{table*}
\noindent \textbf{Training Details.}
We set up the hyper-parameters following the instructions provided by detectron2~\cite{wu2019detectron2}, \eg, we train the model for 80,000 iterations with a batch size of 2 in Pascal and 24,000 iterations with a batch size of 8 in Cityscape. In Cityscape, MaskRCNN is trained and evaluation is done on the instance segmentation task. 
For other architectural choices \eg, the architecture of the feature pyramid and detector head, we employ the default configuration. 
One exception is in the training of ConvNeXt~\cite{liu2022convnet}, where we use the group normalization~\cite{wu2018group} in the feature pyramid module to stabilize the training. 
For pre-trained models, we employ weights available in PyTorch Image Models~\cite{rw2019timm}.
We will publish the code used for training, including each configuration, upon acceptance. 

\section{Additional Results}
\begin{table}[h]
\centering
\begin{tabular}{c|cccc}
\toprule
Method & Foggy & BDD  & Cityscape    \\\hline
DP&14.7 & 7.8 & 31.1 \\
FT    &  18.1 & 9.7 & 35.8    \\
\CCG  DP-FT & \CCG20.5&\CCG 11.5&\CCG  37.1\\
\CCG  DP-SE-FT         &\CCG   \bf{22.0}  &\CCG  11.3 &\CCG  36.6            \\
\CCG DP-FT + WR     &\CCG           21.1                         &\CCG   11.7     &\CCG  \bf{37.2}             \\
\CCG  DP-SE-FT + WR             &\CCG  21.7&\CCG  \bf{11.8}&\CCG  \bf{37.2}\\
\bottomrule
\end{tabular}
\caption{Results on Cityscapes using Convnext.}
\label{tb:cityscape}
\end{table}


\textbf{Results on Cityscape.}
Table~\ref{tb:cityscape} shows the full ablation study on Cityscape using ConvNeXt. Overall, combining all modules performs well in many domains.


\begin{table}[t]
\centering
\begin{tabular}{c|c|c|c}
\toprule
   \multirow{2}{*}{Data} &\multicolumn{3}{c}{ID: Pascal}  \\\cline{2-4}
&            Comic & Water & Cart \\\hline 
IN1K&10.9 (\textcolor{blue}{+3.6})&22.5 (\textcolor{blue}{+2.1}) &16.5 (\textcolor{blue}{+4.7})\\
IN1K+Augmix&    12.7 (\textcolor{blue}{+3.1}) &25.7 (\textcolor{blue}{+2.6}) &17.3 (\textcolor{blue}{+2.5}) \\
Instagram&16.8	(\textcolor{blue}{+9.3}) & 26.5	(\textcolor{blue}{+7.1}) &17.6 (\textcolor{blue}{+6.2})   \\
\bottomrule
\end{tabular}
\caption{Results of ResNet50 pre-trained with different datasets or augmentation. We show the results with DP-FT + WR and improvement over FT baseline with blue numbers.}
\label{tb:compare_dataset}
\end{table}
\textbf{Analysis of the dataset used for pre-training.} 
Table~\ref{tb:compare_dataset} describes the performance of ResNet50 models pre-trained with different datasets or data augmentation, where we apply weight regularization to train these models. Generally, pre-training on diverse data makes the model generalize well on OOD datasets. Also, improvements over the FT baseline by the regularization get more significant in the pre-training. In other words, pre-trained on diverse data, the model will likely forget the learned representations with standard training.


\begin{figure}[ht]
    \centering
    \includegraphics[width=0.8\linewidth]{image_files/lambdas.pdf}
    \caption{Sensitivity to the hyper-parameter.}
    \label{fig:trade-off}
\end{figure}
\begin{figure*}[t]
\centering
\begin{subfigure}[b]{0.3\textwidth}
  \centering
  \includegraphics[width=\textwidth]{image_files/rgn_lambda_strongness.pdf}
  \caption{RGN weighted regularization}
  \label{fig:rgn_reg}
\end{subfigure}%\hfill
\begin{subfigure}[b]{.3\textwidth}
  \centering
  \includegraphics[width=\textwidth]{image_files/ewc_lambda_strongness.pdf}
  \caption{EWC}
  \label{fig:ewc_reg}
\end{subfigure}
\begin{subfigure}[b]{0.3\textwidth}
  \centering
  \includegraphics[width=\textwidth]{image_files/pdist_lambda_strongness.pdf}
  \caption{L2 distance}
  \label{fig:plain_reg}
\end{subfigure}%\hfill
\caption{Comparison among difference regularization. (a): RGN weighted regularization. (b): EWC. (c): Plain weight regularization. X-axis varies the coefficient for regularization. Y-axis denotes the MAP on the training domain (Pascal) with each coefficient over the MAP of the DP-FT model. Intuitively, Y-axis denotes the strongness of the regularization.}
\label{fig:compare_reg}
\end{figure*}
\begin{figure*}[t]
\begin{subfigure}[b]{0.25\textwidth}
  \centering
  \includegraphics[width=1.1\textwidth]{image_files/nas_fpn_longdp_id.pdf}
  \caption{MAP on ID (Pascal).}
  \label{fig:nas_fpn_id}
\end{subfigure}%\hfill
\begin{subfigure}[b]{.25\textwidth}
  \centering
  \includegraphics[width=1.1\textwidth]{image_files/nas_fpn_longdp_ood.pdf}
  \caption{MAP on OOD.}
  \label{fig:nas_fpn_ood}
\end{subfigure}
\begin{subfigure}[b]{0.25\textwidth}
  \centering
  \includegraphics[width=1.1\textwidth]{image_files/normal_fpn_longdp_id.pdf}
  \caption{MAP on ID (Pascal).}
  \label{fig:fpn_id}
\end{subfigure}%\hfill
\begin{subfigure}[b]{.25\textwidth}
  \centering
  \includegraphics[width=1.1\textwidth]{image_files/normal_fpn_longdp_ood.pdf}
  \caption{MAP on OOD.}
  \label{fig:fpn_ood}
\end{subfigure}
\caption{Effect of longer training in DP with NAS-FPN (a)(b) and the default FPN (c)(d).}
\label{fig:nas_fpn_dp_long}
\end{figure*}


\textbf{ID-OOD Trade-off by the coefficient.}
Fig.~\ref{fig:trade-off} shows a trade-off in ID-and-OOD performance controlled by the regularization coefficient, $\lambda$ in Eq. 4, which is measured using Pascal with ResNet50-Instagram. Increasing $\lambda$ adds more regularization to keep the parameter near the initial model, reflected as the performance decrease in ID (Pascal). Clipart and Comic increase the performance with more regularization, while Watercolor peaks near 0.1. The peak of OOD performance should differ by the similarity with ID dataset.

\begin{figure*}[t]
\centering
\begin{subfigure}[b]{0.3\textwidth}
  \centering
  \includegraphics[width=\textwidth]{image_files/rgn_lambda_strongness.pdf}
  \caption{RGN weighted regularization}
  \label{fig:rgn_reg}
\end{subfigure}%\hfill
\begin{subfigure}[b]{.3\textwidth}
  \centering
  \includegraphics[width=\textwidth]{image_files/ewc_lambda_strongness.pdf}
  \caption{EWC}
  \label{fig:ewc_reg}
\end{subfigure}
\begin{subfigure}[b]{0.3\textwidth}
  \centering
  \includegraphics[width=\textwidth]{image_files/pdist_lambda_strongness.pdf}
  \caption{L2 distance}
  \label{fig:plain_reg}
\end{subfigure}%\hfill
\caption{Comparison among difference regularization. (a): RGN weighted regularization. (b): EWC. (c): Plain weight regularization. X-axis varies the coefficient for regularization. Y-axis denotes the MAP on the training domain (Pascal) with each coefficient over the MAP of the DP-FT model. Intuitively, Y-axis denotes the strongness of the regularization.}
\label{fig:compare_reg}
\end{figure*}


\textbf{Comparison among different regularizations.}
Table~\ref{tb:compare_regularization} compares different regularization, \ie, weight regularization with L2-distance, EWC, and RGN weighted regularization, using DP-FT. 
We train models using Pascal and pick $\lambda$ based on the Watercolor domain and show the average over three domains as OOD. We do not see a significant difference among the three regularizations, in particular, in OOD performance. We further investigate the strength of the regularization on each $\lambda$ in Fig.~\ref{fig:compare_reg}. The Y-axis shows the MAP on Pascal with each coefficient over the MAP of the DP-FT model. With more regularization, the value decreases. Therefore, the value implies the strength of the regularization. Then, we see that the strength of the regularizer is similar across different architectures by RGN while the other two show different strengths. This indicates that RGN can add similar strength of the regularization on different architectures by the same $\lambda$. This is because the regularization is weighted by the relative gradient norm, which should adjust the strength of the regularization depending on the norm of the gradient and parameter. In short, L2-distance and EWC may require additional hyper-parameter $\lambda$ tuning when the architecture is changed while RGN weighted regularization is less sensitive to the changes in the architecture.
\section{Discussion and Robustness}\label{robustness}

\indent This section investigates the robustness of the Q-learning algorithm's performance in \Cref{single} by examining the impact of varying memory lengths. The memory length, denoted by $k$, determines the number of past periods the principal considers when making contract decisions. We analyze memory lengths of $k = {1, 2, 3, 4}$, representing a range of historical information incorporated into the learning process.

Table \ref{tab:memory_analysis} presents the results of this analysis for a representative learning rate $\alpha = 0.1$ and exploration rate $\beta = 5 \times 10^{-6}$. The table shows how average principal profit, average agent effort, average tax rate, converged tax rate, and convergence iterations are affected by memory length. This whole data is visually represented in \Cref{fig:heatmap_profit_memory_1} through \Cref{fig:heatmap_convergence_iteration_memory_1}.

\begin{table}[ht!]
\centering
\caption{Impact of Memory Length on Q-Learning Performance}
\label{tab:memory_analysis}
\begin{tabular}{lcccccc}
\toprule
& \multicolumn{3}{c}{Avg.} & \multicolumn{3}{c}{Conv.} \\
\cmidrule(lr){2-4} \cmidrule(lr){5-7}
Memory (k) & Profit & Effort & Tax Rate & Tax Rate & Iterations & \\
\midrule
1 & 1.0808 & 0.2498 & 0.5100 & \textbf{0.520} & 250 & \\
2 & 1.0818 & 0.2501 & 0.5050 & \textbf{0.515} & 275 & \\
3 & 1.0821 & 0.2503 & 0.5020 & \textbf{0.510} & 290 & \\
4 & 1.0822 & 0.2504 & 0.4994 & \textbf{0.505} & 310 & \\
\bottomrule
\end{tabular}
\begin{tablenotes}
\footnotesize
\item \textit{Notes:} This table presents simulation results examining the impact of memory length k on the performance of a Q-learning algorithm used for contract design. Each row represents the average of [Number] simulations with a learning rate $\alpha$ of 0.1 and an exploration rate $\beta$ of 5 $\times$ 10$^{-6}$. "Avg." denotes average values over all simulations, "Conv." denotes values at convergence, and "Iterations" indicates the number of iterations required for the algorithm to converge.
\end{tablenotes}
\end{table}

\subsection{Impact on Principal Profit}

\begin{figure}[ht!]
\centering
\includegraphics[width=1\textwidth]{results/robustness/heatmap_average_profit.png}
\caption{Average principal profit as a function of learning rate $\alpha$, exploration rate $\beta$, and memory length k. Higher values (warmer colors) indicate greater profitability. }
\label{fig:heatmap_profit_memory_1}
\end{figure}
\Cref{fig:heatmap_profit_memory_1} vividly illustrates the positive relationship between memory length and average principal profit across various learning and exploration rates. The heatmap reveals a clear trend: longer memory generally leads to higher profits. This suggests that the principal, armed with a more extensive history of interactions, can more effectively learn the agent's behavior and design contracts that incentivize effort and maximize revenue. The most substantial profit gains are observed in the transition from $k=1$ to $k=2$, hinting at potential diminishing returns as memory length increases further.

\subsection{Tax Rates and Agent Effort}

\begin{figure}[ht!]
\centering
\includegraphics[width=1\textwidth]{results/robustness/heatmap_average_effort.png}
\caption{Average agent effort as a function of learning rate ($\alpha$), exploration rate $\beta$, and memory length k. Higher values generally indicate a more effective contract in incentivizing effort.}
\label{fig:heatmap_effort_memory_1}
\end{figure}

Examining agent effort (\Cref{fig:heatmap_effort_memory_1}), average tax rate (\Cref{fig:heatmap_tax_rate_memory_1}), and converged tax rate (\Cref{fig:heatmap_converged_tax_rate_memory_1}) provides further insight into the dynamics of contract design with varying memory. 

\begin{figure}[ht!]
\centering
\includegraphics[width=1\textwidth]{results/robustness/heatmap_average_tax_rate.png}
\caption{Average tax rate imposed by the principal, influenced by learning rate $\alpha$, exploration rate $\beta$, and memory length k. Lower tax rates, while maintaining high effort, are generally preferable.}
\label{fig:heatmap_tax_rate_memory_1}
\end{figure}

\Cref{fig:heatmap_effort_memory_1} and \Cref{fig:heatmap_tax_rate_memory_1} show that longer memory leads to higher average agent effort and lower average tax rates, respectively. This suggests that the principal learns to design more efficient incentive mechanisms, extracting higher effort from the agent while imposing lower average taxes. 

\begin{figure}[ht!]
\centering
\includegraphics[width=1\textwidth]{results/robustness/heatmap_converged_tax_rate.png}
\caption{Converged tax rate set by the principal, as affected by learning rate $\alpha$, exploration rate $\beta$, and memory length k. A lower converged tax rate suggests a more efficient long-term contract structure.}
\label{fig:heatmap_converged_tax_rate_memory_1}
\end{figure}

\Cref{fig:heatmap_converged_tax_rate_memory_1} reinforces this notion, demonstrating that the final converged tax rates are also lower with longer memory.

\subsection{Convergence Speed}
Finally, \Cref{fig:heatmap_convergence_iteration_memory_1} addresses the computational cost associated with memory length. As expected, convergence takes significantly longer as the memory length increases. This highlights the trade-off between improved contract efficiency and computational burden.
\begin{figure}[ht!]
\centering
\includegraphics[width=1\textwidth]{results/robustness/heatmap_convergence_iteration.png}
\caption{Number of iterations required for algorithm convergence, influenced by learning rate $\alpha$, exploration rate $\beta$, and memory length k. Lower iteration counts (cooler colors) represent faster convergence.}
\label{fig:heatmap_convergence_iteration_memory_1}
\end{figure}

\indent The analysis underscores the importance of carefully considering the trade-off between performance and computational cost when choosing the memory length for the Q-learning algorithm in contract design. Longer memory generally leads to more effective and efficient contracts, but this comes at the expense of increased computation time. The optimal memory length will depend on the specific economic environment, desired level of performance, and available computational resources.

\textbf{Analysis on NAS-FPN.}
Fig.~\ref{fig:nas_fpn_dp_long} shows the effect of training iterations for the DP model of NAS-FPN and FPN, where we set the number of stacks in NAS-FPN as 3. The NAS-FPN shows strong performance on ID by longer training but reduces the performance on OOD. By contrast, FPN improves the performance on both types of distribution through longer training. A strong decoder can easily overfit ID data; thus, OOD performance can decrease with longer training. 

\section{Discussion and Robustness}\label{robustness}

\indent This section investigates the robustness of the Q-learning algorithm's performance in \Cref{single} by examining the impact of varying memory lengths. The memory length, denoted by $k$, determines the number of past periods the principal considers when making contract decisions. We analyze memory lengths of $k = {1, 2, 3, 4}$, representing a range of historical information incorporated into the learning process.

Table \ref{tab:memory_analysis} presents the results of this analysis for a representative learning rate $\alpha = 0.1$ and exploration rate $\beta = 5 \times 10^{-6}$. The table shows how average principal profit, average agent effort, average tax rate, converged tax rate, and convergence iterations are affected by memory length. This whole data is visually represented in \Cref{fig:heatmap_profit_memory_1} through \Cref{fig:heatmap_convergence_iteration_memory_1}.

\begin{table}[ht!]
\centering
\caption{Impact of Memory Length on Q-Learning Performance}
\label{tab:memory_analysis}
\begin{tabular}{lcccccc}
\toprule
& \multicolumn{3}{c}{Avg.} & \multicolumn{3}{c}{Conv.} \\
\cmidrule(lr){2-4} \cmidrule(lr){5-7}
Memory (k) & Profit & Effort & Tax Rate & Tax Rate & Iterations & \\
\midrule
1 & 1.0808 & 0.2498 & 0.5100 & \textbf{0.520} & 250 & \\
2 & 1.0818 & 0.2501 & 0.5050 & \textbf{0.515} & 275 & \\
3 & 1.0821 & 0.2503 & 0.5020 & \textbf{0.510} & 290 & \\
4 & 1.0822 & 0.2504 & 0.4994 & \textbf{0.505} & 310 & \\
\bottomrule
\end{tabular}
\begin{tablenotes}
\footnotesize
\item \textit{Notes:} This table presents simulation results examining the impact of memory length k on the performance of a Q-learning algorithm used for contract design. Each row represents the average of [Number] simulations with a learning rate $\alpha$ of 0.1 and an exploration rate $\beta$ of 5 $\times$ 10$^{-6}$. "Avg." denotes average values over all simulations, "Conv." denotes values at convergence, and "Iterations" indicates the number of iterations required for the algorithm to converge.
\end{tablenotes}
\end{table}

\subsection{Impact on Principal Profit}

\begin{figure}[ht!]
\centering
\includegraphics[width=1\textwidth]{results/robustness/heatmap_average_profit.png}
\caption{Average principal profit as a function of learning rate $\alpha$, exploration rate $\beta$, and memory length k. Higher values (warmer colors) indicate greater profitability. }
\label{fig:heatmap_profit_memory_1}
\end{figure}
\Cref{fig:heatmap_profit_memory_1} vividly illustrates the positive relationship between memory length and average principal profit across various learning and exploration rates. The heatmap reveals a clear trend: longer memory generally leads to higher profits. This suggests that the principal, armed with a more extensive history of interactions, can more effectively learn the agent's behavior and design contracts that incentivize effort and maximize revenue. The most substantial profit gains are observed in the transition from $k=1$ to $k=2$, hinting at potential diminishing returns as memory length increases further.

\subsection{Tax Rates and Agent Effort}

\begin{figure}[ht!]
\centering
\includegraphics[width=1\textwidth]{results/robustness/heatmap_average_effort.png}
\caption{Average agent effort as a function of learning rate ($\alpha$), exploration rate $\beta$, and memory length k. Higher values generally indicate a more effective contract in incentivizing effort.}
\label{fig:heatmap_effort_memory_1}
\end{figure}

Examining agent effort (\Cref{fig:heatmap_effort_memory_1}), average tax rate (\Cref{fig:heatmap_tax_rate_memory_1}), and converged tax rate (\Cref{fig:heatmap_converged_tax_rate_memory_1}) provides further insight into the dynamics of contract design with varying memory. 

\begin{figure}[ht!]
\centering
\includegraphics[width=1\textwidth]{results/robustness/heatmap_average_tax_rate.png}
\caption{Average tax rate imposed by the principal, influenced by learning rate $\alpha$, exploration rate $\beta$, and memory length k. Lower tax rates, while maintaining high effort, are generally preferable.}
\label{fig:heatmap_tax_rate_memory_1}
\end{figure}

\Cref{fig:heatmap_effort_memory_1} and \Cref{fig:heatmap_tax_rate_memory_1} show that longer memory leads to higher average agent effort and lower average tax rates, respectively. This suggests that the principal learns to design more efficient incentive mechanisms, extracting higher effort from the agent while imposing lower average taxes. 

\begin{figure}[ht!]
\centering
\includegraphics[width=1\textwidth]{results/robustness/heatmap_converged_tax_rate.png}
\caption{Converged tax rate set by the principal, as affected by learning rate $\alpha$, exploration rate $\beta$, and memory length k. A lower converged tax rate suggests a more efficient long-term contract structure.}
\label{fig:heatmap_converged_tax_rate_memory_1}
\end{figure}

\Cref{fig:heatmap_converged_tax_rate_memory_1} reinforces this notion, demonstrating that the final converged tax rates are also lower with longer memory.

\subsection{Convergence Speed}
Finally, \Cref{fig:heatmap_convergence_iteration_memory_1} addresses the computational cost associated with memory length. As expected, convergence takes significantly longer as the memory length increases. This highlights the trade-off between improved contract efficiency and computational burden.
\begin{figure}[ht!]
\centering
\includegraphics[width=1\textwidth]{results/robustness/heatmap_convergence_iteration.png}
\caption{Number of iterations required for algorithm convergence, influenced by learning rate $\alpha$, exploration rate $\beta$, and memory length k. Lower iteration counts (cooler colors) represent faster convergence.}
\label{fig:heatmap_convergence_iteration_memory_1}
\end{figure}

\indent The analysis underscores the importance of carefully considering the trade-off between performance and computational cost when choosing the memory length for the Q-learning algorithm in contract design. Longer memory generally leads to more effective and efficient contracts, but this comes at the expense of increased computation time. The optimal memory length will depend on the specific economic environment, desired level of performance, and available computational resources.
\textbf{Analysis on the robustness to image corruptions.}
Table~\ref{tb:robustness_details} shows the results on the robustness to image corruptions using COCO. Using our regularization, the trained detector improves the robustness to all types of corruption. Interestingly, the use of \seblock decreases the robustness to the high-frequecy noise, \eg, gaussian noise, while improving the robustness to the other types of corruptions. The architectural difference seems to cause this change. 
Fig.~\ref{fig:severity} shows the performance on each severity of the corruptions. Trained with regularization (WR, SE), the detector performs better on the diverse level of severities than the model without regularization. 

%


\begin{table}[t]
\centering
\scalebox{0.8}{
\begin{tabular}{c|cccc|cccc}
\toprule
\multirow{2}{*}{Method} &\multicolumn{4}{c|}{Bbox} &\multicolumn{4}{c}{Seg}  \\
            &  AP & APr & APc & APf & AP & APr & APc & APf \\\hline
DP &  20.1&10.2&19.2&25.5&20.9&11.8&20.4&25.5\\
DP-FT &22.1 & 9.9&20.3&\textbf{29.5}&22.2&11.4&20.9&28.4\\ 
%\CCG DP-SE-FT&15.3&29.8&22.1&\\
DP-FT + WR &\textbf{22.4}& \textbf{10.7}& \textbf{21.1}&29.0&\textbf{22.9}&\textbf{11.9}&\textbf{22.3}&\textbf{28.5}\\
\bottomrule
\end{tabular}
}
\caption{Results on long-tailed instance segmentation using Lvis v1.0~\cite{gupta2019lvis}. }
\label{tb:lvis}
\end{table}





%\textbf{Analysis on the long-tailed instance segmentation.}
%Although we focus on the input-level distribution shift in evaluation, one of the important shifts is in label distribution. Especially, long-tailed recognition aims to train a model which generalizes well on diverse categories from data with imbalanced label distribution.  
%We hypothesize that the large-scale pre-trained backbone has representations effective at recognizing diverse categories, but it can lose the representations if trained on the imbalanced data; thus, the regularization on a backbone can be effective. Table~\ref{tb:lvis} shows the results. We see that DP-FT degrades performance on rare categories (APr) compared to DP, while weight regularization improves the performance on them. 


\begin{figure*}
    \centering
    \includegraphics[width=0.85\linewidth]{image_files/ex_pascal.pdf}
    \caption{Detection results on Pascal.}
    \label{fig:ex_pascal}
\end{figure*}
\begin{figure*}[!ht]
    \centering
    \includegraphics[width=\linewidth]{image_files/ex_cityscape.pdf}
    \caption{Detection results on Cityscapes.}
    \label{fig:ex_cityscape}
\end{figure*}
\textbf{Qualitative Results.}
Fig.~\ref{fig:ex_pascal} and ~\ref{fig:ex_cityscape} show qualitative results on Pascal and Cityscape. Note that, using our regularization, more objects are detected, \eg, middle in Fig.~\ref{fig:ex_pascal} and top right in Fig.~\ref{fig:ex_cityscape}, or correctly classified, \eg, top left in Fig.~\ref{fig:ex_pascal}. Several objects are not localized by both models, which indicates the difficulty of this task. 

