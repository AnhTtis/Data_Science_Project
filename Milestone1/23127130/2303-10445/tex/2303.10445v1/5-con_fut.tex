\section{Conclusion and Future Work}
In this paper, we have proposed EarCough, a technique that enables the always-on hybrid active noise cancellation microphones in commodity hearables for continuous subject cough event detection. We proposed a lightweight end-to-end deep learning model --- EarCoughNet, which leverages the difference between feed-forward and feedback audio of hybrid ANC smart earbuds to achieve effective subject-awareness. To evaluate the effectiveness of the method, we constructed the first synchronous audio and motion dataset targeted at subject cough event detection. We proved EarCough's effectiveness and reliability by training and evaluating the model on the constructed dataset. Results show that EarCough achieved subject cough event detection with an accuracy of 95.4\% and an F1-score of 92.89\% at the audio sampling rate of 8kHz with only 385kB space requirement. To the best of our knowledge, we are the first to develop a resource-efficient end-to-end subject cough event detection compatible with the microcontroller in modern earphones.

Although EarCough achieves outstanding feasibility and efficiency, our constructed dataset needs to expand the evaluation scenarios. For instance, we have not included scenarios when people are listening to the sounds playing from the earbuds. Since the played sounds may influence the quality of the feedback microphone's audio, extra modules based on noise cancellation will be added on EarCough to decrease the negative effect. After expanding the dataset, we will make it publicly accessible to help the research community improve subject cough event detection methods. Besides, we have not evaluated EarCough in real-life scenarios. As a result, we plan to conduct a field study in the wards of stroke patients with cough symptoms and further fine-tune EarCoughNet based on the evaluation results. Finally, since cough usually causes body movements, we envision that sensor fusion methods will further increase the detection performance of EarCough with the motion features extracted from inertial measurement unit (IMU) signals. 