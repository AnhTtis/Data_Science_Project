\section{Introduction}
\label{sec:intro}

Cough is one of the most relevant and common indicators of pulmonary diseases\cite{barata2019towards, nemati2020}. Subject cough refers to 
the cough originating from the user of the cough event detection method rather than the cough emitted from the user's environment (such as the cough from other pulmonary disease patients), which is defined as environmental cough. Future automatic cough event detection methods are expected to have subject-awareness~\cite{Hall_2020}, which is the ability to distinguish subject cough events from environmental cough events. Without subject-awareness, cough event detection methods would incorrectly detect environmental coughs and further provide misleading health or disease reports to clinicians, which significantly limits the health application scenarios~\cite{Vatanparvar_2020}. 

In recent years, earphones have become one of the most ubiquitous end-user accessories~\cite{web:canalys}. The global hearables market is projected to reach \$93.90 billion by 2026~\cite{web:alliedmarketresearch}. With the rapid growth of the hearables market, modern smart earphones are developed with rich sensing capabilities and microcontrollers with computational capability, which attracted the research community to explore ways to leverage earphones in the field of health and physiological sensing. Smart earphones have already been leveraged to detect various physiological signals~\cite{Looney2012, Leboeuf2014, Nguyen2016, Bui2019, Athavipach2019, Martin2018, Vogel2007, Roddiger2019a}, including heart rate~\cite{Leboeuf2014, Goverdovsky2016}, brain signals~\cite{Looney2012} and respiration rates~\cite{Roddiger2019a, Liu2019}. Besides, it has been proved that real-time, privacy-safe, low-cost, and ubiquitous cough event detection was able to achieve by leveraging active noise cancellation earphones~\cite{WANG2022}. 

This paper presents EarCough, a technique that enables continuous subject cough event detection on edge computing hearables. Specifically, we proposed a lightweight end-to-end deep learning model named EarCoughNet, which takes the dual-channel audio from hybrid active noise cancellation microphones on smart earbuds as input. To evaluate the effectiveness and reliability of EarCough, we built a dataset targeted at subject cough event detection with sensor fusion data: dual-channel audio data from active noise cancellation microphones of Bose QC 20, plus motion data from the IMU sensor. EarCough achieved an accuracy of 95.4\% and an F1-score of 92.9\% on the constructed dataset, with a space requirement of only 385kB. We envision EarCough as a low-cost add-on for future hearables to enable continuous personal pulmonary health monitoring.

This paper's main contributions have three folds as below.

1) We proposed EarCough, a technique that enables continuous subject cough event detection. To the best of our knowledge, EarCough is the first subject cough event detection method utilizing the difference between dual-channel audio of the built-in always-on hybrid ANC microphones in commodity hearables.

2) We evaluated EarCough's effectiveness and reliability via user study. Results show that EarCough realizes subject cough event detection every 0.5 seconds at an accuracy of 95.4\% and an F1-score of 92.9\% with only 385 kB space requirement.

3) We established the first dataset targeted at continuous subject cough detection with the sensor fusion data: dual-channel audio data from two active noise cancellation microphones plus motion data from the IMU sensor.