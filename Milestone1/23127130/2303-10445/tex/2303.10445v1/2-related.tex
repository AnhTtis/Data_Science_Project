\section{Background and Related Works}
\label{sec:rel}

Cough is one of the most common and prominent symptoms associated with many respiratory diseases such as COPD, asthma, and tuberculosis~\cite{barata2019towards,nemati2020}. Automatic cough event detection methods can provide valuable features for pulmonary diagnosis and health condition assessment~\cite{Vatanparvar_2020, Windmon_2019, Hall_2020}. 

Recently, the research community has widely explored automatic cough event detection methods, most of which are audio-based ~\cite{gao2019analysis, Sharan_2019, Liaqat_2021, Teyhouee_2019, birring2008leicester, WANG2022, Xu2021}, owing to the valuable characteristic spectral signature contained in cough sounds~\cite{Amoh_2014}. For instance, Wang et al. ~\cite{WANG2022} proposed HearCough, enabling state-of-the-art continuous cough event detection based on the audio signals from commodity hearables. However, most previous work ignored the importance of distinguishing coughs emitted from the subject (subject coughs) and coughs originating from the subject's environment (environmental coughs), which . In public scenarios, the falsely detected environmental coughs would then be mistakenly considered for a health analysis or disease diagnosis by clinicians, which could have serious adverse consequences due to harmful medication use and increased costs for patients~\cite{Vatanparvar_2020}. 

To fill the gap in subject-awareness, researchers have already explored methods targeting distinguishing between coughs emitted by different coughers. For instance, Whitehill et al. ~\cite{Whitehill_2020} proposed Whosecough, a cougher verification model using audio-based multitask learning strategy and achieves high accuracy under in-the-wild data. Jokic et al. ~\cite{Jokic_2022} presented TripletCough, leveraging triplet network architecture and audio signals captured from smartphones to distinguish cough events among different coughers. However, these works only focus on differentiating the cougher who emits the cough rather than subject cough event detection from other events under different noisy environments, which does not directly feed the need of the healthcare industry. As a result, some previous works introduced subject-awareness into cough event detection methods, which were regarded as subject cough event detection methods. For example, Rahman et al. ~\cite{Rahman2019effi} proposed an audio-based subject cough event detection method, which leveraged feature engineering and random forest, and achieved 94.2\% precision on subject cough event recognition and is suited to be applied on smartphones. Nevertheless, this method was evaluated only on the in-lab dataset, which may suffer from performance drop under in-the-wild scenarios. Besides, no practical approach can be deployed to a microcontroller with only hundreds of kilobytes of RAM (e.g., ARM M4F).

Recently, smart earbuds are often developed more than just audio listening devices, offering an expanding suite of sensors and microcontrollers with computational capability. Most modern earbuds are equipped with active noise cancellation (ANC), which was designed to enhance users' listening experience by reducing environmental noises. Hybrid ANC is one of the most adopted solutions since it produces the best noise reduction while alleviating acoustic discomfort to human ears\footnote{https://blog.teufelaudio.com/hybrid-anc/}. To achieve hybrid active noise cancellation, one or a set of feed-forward (reference) microphones are placed at the outer side of the earphone to collect the environmental noises. Then an adaptive filter running on the digital signal processor will generate an anti-phase signal with $180^{\circ}$ phase delay to the speaker to eliminate the noises propagated to the human ear. A feedback (error) microphone, placed between the speaker and the ear, is deployed to further monitor the noise cancellation performance and then fine-tune the adaptive filter. 

Audio signals received by the feed-forward and feedback microphones are different due to many factors, such as the distance and the orientation of the sound source. As a result, we envision leveraging the difference between the dual-channel audio signals to achieve subject-awareness. Moreover, due to the proximity of earphones to the mouth, high-quality subject cough sounds could be acquired and thus can be leveraged for a more robust cough event detection system. Besides, active noise cancellation earphones are usually compact, portable, and minimally intrusive. As a result, we envision hybrid active noise cancellation smart earbuds as an ideal hardware platform for continuous subject cough event detection.

Compared to previous works, EarCough achieves state-of-the-art subject cough event detection performance with less computational needs. To the best of our knowledge, we are the first to develop a resource-efficient end-to-end subject cough event detection method, which is compatible with the microcontroller in modern earphones.