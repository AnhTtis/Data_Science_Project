
\begin{table*}[!t]
    \centering
    \begin{adjustbox}{max width=\textwidth}
        \begin{tabular}{|c|c|c|c|c|c|c|c|c|c|c|c|c|}
            \hline
            \multirow{3}{*}{Model}    &         & \multirow{2}{*}{Intent}   & Requested & Average        & Joint          &                &         & Average  & Joint    & \multirow{2}{*}{Response} &          \\
                                      & Domains & \multirow{2}{*}{Accuracy} & Slots     & Goal           & Goal           & Inform         & Success & Action   & Action   & \multirow{2}{*}{GLEU}     & Combined \\
                                      &         &                           & F1        & Accurracy      & Accuracy       &                &         & Accuracy & Accuracy &                           &          \\ \hline
            \multirow{2}{*}{\textbf{\oursys}} & all     & \textbf{84.83}                     & \textbf{95.53}     & \textbf{72.38} & \textbf{48.44} & \textbf{73.08} & \textbf{62.19}   & \textbf{58.32}    & \textbf{46.31}    & \textbf{20.04}                     & \textbf{87.67}    \\
            \multirow{2}{*}{\textbf{(this work)}}                          & seen    & \textbf{85.48}                     & \textbf{95.88}     & \textbf{74.23} & \textbf{52.05} & \textbf{74.72} & \textbf{63.85}   & \textbf{60.19}    & \textbf{48.69}    & \textbf{24.66}                     & \textbf{93.95}    \\
                                      & unseen  & \textbf{84.45}                     & \textbf{95.42}     & \textbf{72.03} & \textbf{47.83} & \textbf{71.68} & \textbf{61.63}   & \textbf{57.42}    & \textbf{45.21}    & \textbf{18.51}                     & \textbf{85.16}    \\ \hline
            {w/o}                     & all     & 75.08                     & 92.80     & 62.47          & 39.52          & 48.13          & 44.27   & 40.38    & 30.71    & 11.41                     & 57.61    \\
            {Two Step}                & seen    & 75.75                     & 93.13     & 64.66          & 42.76          & 50.26          & 46.47   & 41.96    & 32.42    & 13.75                     & 62.11    \\
            {Training}                & unseen  & 75.79                     & 92.90     & 62.60          & 39.25          & 47.55          & 44.32   & 40.25    & 30.66    & 11.03                     & 56.97    \\ \hline
            {w/o}                     & all     & 82.14                     & 94.67     & 64.70          & 38.47          & 59.88          & 53.88   & 54.14    & 43.07    & 21.15                     & 78.03    \\
            {Domain}                  & seen    & 83.34                     & 95.10     & 67.62          & 43.39          & 62.30          & 56.64   & 56.61    & 45.92    & 27.10                     & 86.57    \\
            {Schema}                  & unseen  & 81.96                     & 94.52     & 63.95          & 37.59          & 58.65          & 53.25   & 53.22    & 42.20    & 19.33                     & 75.28    \\ \hline
            % {w/o}                  & all     & 88.22                     & 95.72     & 71.51          & 42.68          & 64.15          & 60.98   & 57.34    & 45.26    & 20.70                     & 83.27    \\
            % {User}                 & seen    & 89.02                     & 96.07     & 73.86          & 47.02          & 65.80          & 63.17   & 59.48    & 47.83    & 25.96                     & 90.44    \\
            % {Actions}              & unseen  & 87.94                     & 95.59     & 71.05          & 41.78          & 63.09          & 60.73   & 56.46    & 44.41    & 18.82                     & 80.73    \\ \hline
            {w/o}                     & all     & 82.50                     & 95.48     & 71.54          & 43.20          & 50.96          & 56.89   & 53.67    & 41.73    & 17.62                     & 71.54    \\
            {DB}                      & seen    & 83.26                     & 95.85     & 73.87          & 47.62          & 53.03          & 59.08   & 55.73    & 43.91    & 23.12                     & 79.17    \\
            {Results}                 & unseen  & 82.19                     & 95.36     & 71.04          & 42.17          & 50.33          & 56.95   & 53.17    & 41.52    & 16.07                     & 69.70    \\ \hline
            {w/o Sys}                 & all     & 82.56                     & 96.00     & 72.86          & 44.52          & 60.13          & 61.91   & 57.98    & 45.86    & 21.02                     & 82.04    \\
            {Action}                  & seen    & 83.25                     & 96.32     & 75.11          & 48.77          & 61.69          & 64.04   & 60.12    & 48.37    & 26.56                     & 89.43    \\
            {Names}                   & unseen  & 82.38                     & 95.91     & 72.44          & 43.60          & 59.61          & 61.75   & 57.29    & 45.26    & 19.16                     & 79.84    \\ \hline
        \end{tabular}
    \end{adjustbox}
    \caption{Ablation Study of~\oursys.}
    \label{tab:ablation-results}
    \vspace{-15pt}
\end{table*}
%\vspace{-25pt}

\section{Results}
\begin{table}
    \begin{adjustbox}{max width=0.45\textwidth}
        \begin{tabular}{|c|c|c|c|c|}
            \hline
            \multirow{2}{*}{Model} & Intent   & Requested & Average & Joint \\
                                   & Accuracy & Slot F1   & GA      & GA    \\ \hline
            SGD Baseline           & 90.60    & \textbf{96.50}     & 56      & 25.40 \\ \hline
            FastSGT                & 90.33    & 96.33     & 60.66   & 29.20 \\ \hline
            Seq2Seq-DU             & \textbf{91.00}    & -         & -       & 30.10 \\ \hline
            DSGFNET                & -        & -         & -       & 32.10 \\ \hline
            \textbf{\oursys}              & 81.49    & 95.97     & \textbf{74.08}   & \textbf{49.73} \\ \hline 
            
        \end{tabular}
    \end{adjustbox}
    \vspace{-5pt}
    \caption{Results on SGD test set. Our approach significantly outperforms baselines methods in terms of average and joint goal accuracy.}
    \label{tab:other-results}
    \vspace{-15pt}
\end{table}


\stitle{Main Results.}
%Since there are no End-to-End TOD systems for the SGD dataset, we re-implemented some of the popular baseline methods to compare with our approach and present the results in Table~\ref{tab:main-results}.
Since no E2E ToD system has reported results for the SGD dataset, we follow~\cite{HosseiniAsl2020ASL} to implement some of the popular baseline methods to compare with our approach and present the results in Table~\ref{tab:main-results}.
We can see that~\oursys~outperforms all the baselines across all metrics except GLEU, where its performance is super competitive (e.g., 24.89 vs 24.66). 
%An explanation of this could be that since we replaced the dialog history with the dialog state, the model lost a lot of exposure to dialog utterances.
An explanation of this could be that since we replaced the dialog history with the dialog state, the performance of the model improved on all other metrics, but the model lost a lot of exposure to dialog utterances.
Another reason could be greedy decoding which works well for a structured generation but is not the best strategy for fluent text generation.
While the system response requires a fluent generation, all other parts of the generation can be deemed as a structured generation.
On the other hand, nucleus and top-k sampling strategies are better suited for a fluent generation but are not the best for a structured generation.
We formulated the problem as a single sequence generation, and we can only select one strategy, so there is bound to be a trade-off. 
Since there is no single strategy that is best suited for both fluent and structured generation, our selection of greedy decoding may have been the cause for the loss of fluency in response generation.

We evaluate the DST performance of~\oursys~with the evaluation script provided by the SGD dataset and present our results alongside
other baseline DST models in Table~\ref{tab:other-results}.
We can see that even though our method is not specifically designed for DST, still it significantly outperforms the baselines models in the
important metrics: Average and Joint Goal Accuracy.

\begin{figure}[!t]
    \centering
    \includegraphics[width=0.9\linewidth]{assets/dialog_turns.pdf}
    \vspace{-10pt}
    \caption{
        Performance of dialog systems on the SGD test set with respect to dialog turns
    }
    \label{fig:dialog_turns}
    \vspace{-15pt}
\end{figure}

\stitle{Long Range Dialog Dependencies.}
In order to process dialogs that have a large number of turns, a system must be effective at capturing long-range dependencies.
To test this ability, we group the test dialogs based on the number of turns and evaluate on each group.
As shown in Figure~\ref{fig:dialog_turns},~\oursys~outperforms the baseline systems across all groups.
Generally, in the first few turns of a dialog, the main focus is on figuring out what the user wants. The user could switch among mulitple options before finally deciding on one, however
towards the end of a dialog, usually the user has a clear idea of what he or she wants, so is less likely to make many changes.
For the first few turns, we have observed that there is a steeper drop in performance of the baseline when compared to~\oursys~.
A possible explanation of this could be that, since we pass the dialog summary to the model, it contains the correct state of the dialog at the previous turn, which helps the model to make better predictions.
In groups with large number of turns, both the baseline and~\oursys~perform similarly, which suggests even though~\oursys~does well in capturing
medium range dependencies, long range dependencies are still a challenge.
%\todo{Should I reduce this paragraph by just saying our model does better than baseline and not go into such a detailed discussion?}


\begin{figure}[t]
    \centering
    \includegraphics[width=0.9\linewidth]{assets/two_step_training.pdf}
    \vspace{-10pt}
    \caption{
        Effect of Two Step Training on dialog systems
    }
    \vspace{-15pt}
    \label{fig:two_step_training}
\end{figure}


\stitle{Two Step Training.}
To better understand the effect of the two step training process, we compared {\oursys} and a few baseline systems with and without the two step training process.
In Figure~\ref{fig:two_step_training}, we can see that models that incorporate schema benefit from the two step training process.

\stitle{Ablation Study.}
To get a better understanding of the different components of our model, we drop a certain component of {\oursys} to show the effect on the performance
and report an ablation study in Table~\ref{tab:ablation-results}.
We can see that dropping two step training drastically degrades performance across all metrics, which suggests the importance of the training mechanism for {\oursys}.

The role of schema is also important as we can see that the performance of {\oursys} drops across all metrics when we drop schema.
Another important aspect to notice here is that this variant has the largest difference in performance between seen and unseen domains.
These observations indicate that schema not only aids the model to generalize to new domains, but also plays a central role in the overall performance of the system.
When the database results are excluded from the input, there is a big drop in 
metrics related to system actions. Additionally, there is a small drop in the DST performance as well, which suggests that
there is some correlation between the database results and DST.
When we omit the list of system actions types, the metrics related to system actions decreases the most, particularly Inform, but the drop in performance is much less when compared to the setting when the database results were dropped. However, in this setting there were no changes to metrics related to DST.


% \begin{table*}
%     \centering
%     \begin{adjustbox}{max width=\textwidth}
%         \begin{tabular}{|c|c|c|c|c|c|c|c|c|c|c|c|c|}
%             \hline
%             \multirow{3}{*}{Model} & \multirow{2}{*}{Intent}   & Requested   & Average     & Joint       &            &             & Average     & Joint       & \multirow{2}{*}{Response} &              \\
%                                    & \multirow{2}{*}{Accuracy} & Slots       & Goal        & Goal        & Inform     & Success     & Action      & Action      & \multirow{2}{*}{GLEU}     & Combined     \\
%                                    &                           & F1          & Accurracy   & Accuracy    &            &             & Accuracy    & Accuracy    &                           &              \\ \hline
%             \oursys~               & 55.63, 7.7                & 89.67, 0.29 & 50.23, 3.84 & 29.46, 1.31 & 38.90, 2.9 & 26.47, 2.62 & 38.03, 2.12 & 33.60, 1.37 & 15.60, 0.91               & 48.29, 3.53  \\ \hline
%             SimpleTod w/ Schema    & 23.37, 6.03               & 88.23, 1.42 & 19.99, 4.29 & 10.75, 2.89 & 28.76, 7.7 & 16.21, 6.79 & 28.21, 5.2  & 24.19, 4.4  & 14.14, 3.39               & 36.62, 10.14 \\ \hline
%         \end{tabular}
%     \end{adjustbox}
%     \caption{SGD-x results on unseen domain. The mean and standard deviation across all 5 versions is reported for each metric}
%     \label{tab:sgdx-results}
% \end{table*}






\stitle{Results on {\sgdx}.}
To access the robustness of {\oursys}, we ran experiments on the unseen domains of the {\sgdx} dataset and present the results in Figure~\ref{fig:sgdx_graph}.
The bar graph shows the mean of each metric across all the versions of {\sgdx} and the error bars show the standard deviation.
{\oursys} outperforms the baseline across all metrics and has lower standard deviation, showing the robustness of {\oursys} to domain schema variations.



