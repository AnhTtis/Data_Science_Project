\vspace{-2mm}
\section{Experiments}
% We name the method of minimizing the self-consistency potential \ref{eqn_self_consistency_potential} as Entropy-dissipation Informed Neural Network (EINN).
To show the efficacy and efficiency of the proposed approach, we conduct numerical studies on example problems that admit explicit solutions and compare the results with SOTA NN-based PDE solvers. 
The included baselines are PINN \citep{raissi2019physics} and DRVN \citep{zhang2022drvn}.
Note that these baselines only considered the 2D NSE.
We extend them to solve the MVE with the Coulomb interaction for comparison, and the details are discussed in Appendix \ref{appendix_implementation_of_baselines}.
\vspace{-.4cm}
\paragraph{Equations with an explicit solution} 
We consider the following two instances that admit explicit solutions. 
% The first one is for the 2D Navier-Stokes equation while the second one is for the MVE with the Coulomb interaction. 
We verify these solutions in Appendix \ref{appendix_explicit_solution}.\\
\textit{Lamb-Oseen Vortex (2D NSE)} \citep{oseen1912uber}: Consider the whole domain case where $\gX = \sR^2$ and the Biot-Savart kernel (\ref{eqn_nse}). Let $\mathcal{N} ( \vmu , \mSigma)$ be the Gaussian distribution with mean $\vmu$ and covariance $\mSigma$.
If $\rho_0 = \mathcal{N}(0, \sqrt{2\nu t_0}\mI_2)$ for some $t_0 \geq 0$, then we have $\rho_t(\vx) = \mathcal{N}(0, \sqrt{2\nu (t+t_0)}\mI_2)$.\\
%	\item Taylor-Green Vortex \citep{taylor1937mechanism}: Consider the domain $\Omega = [0, 2\pi]^2$ and the scaled periodized Biot-Savart interaction kernel defined as
%	\begin{equation} \label{eqn_K_periodic_biot_savart}
%		K(\vx) = (\frac{1}{2\pi} \frac{\vx^\perp}{\|\vx\|^2} + \frac{1}{2\pi} \sum_{\vk \in \sZ^2, \vk \neq (0, 0)} \frac{(\vx - 2\pi\vk)^\perp}{\|\vx - 2\pi\vk\|^2}) 8\pi^2
%	\end{equation}
%	If $\rho_0(\vx) = \frac{1}{4\pi^2}\left(\cos \vx_1 \cos \vx_2 \exp(-2\nu t_0) + 1\right)$ for some $t \geq 0$, then we have 
%	\begin{equation}
%		\rho_t(\vx) = \frac{1}{4\pi^2}\left(\cos \vx_1 \cos \vx_2 \exp(-2\nu (t+t_0)) + 1\right).
%	\end{equation}
\textit{Barenblatt solutions (MVE)} \citep{serfaty2014mean}: Consider the 3D MVE with the Coulomb interaction kernel (\ref{eqn_coulomb_interaction}) with the diffusion coefficient set to zero, i.e. $d=3$ and $\nu=0$. Let $\Uniform[\sA]$ be the uniform distribution over a set $\sA$. Consider the whole domain case where $\gX = \sR^3$.
If $\rho_0 = \Uniform[\|\vx\|\leq (\frac{3}{4\pi}t_0)^{1/3}]$ for some $t_0 \geq 0$, then we have $\rho_t = \Uniform[\|\vx\|\leq (\frac{3}{4\pi}(t+t_0))^{1/3}]$.

\vspace{-.4cm}
\paragraph{Numerical results}
\begin{figure}
	\centering
	\begin{tabular}{c c c}
		\includegraphics[width=.27\columnwidth]{figure/2D-Navier-Stokes_1.0_objv.pdf} & \includegraphics[width=.27\columnwidth]{figure/2D-Navier-Stokes_1.0_mean.pdf} & \includegraphics[width=.27\columnwidth]{figure/2D-Navier-Stokes_1.0_last.pdf} \\
		\includegraphics[width=.27\columnwidth]{figure/3D-McKean-Vlasov_1.0_objv.pdf} & \includegraphics[width=.27\columnwidth]{figure/3D-McKean-Vlasov_1.0_mean.pdf} & \includegraphics[width=.27\columnwidth]{figure/3D-McKean-Vlasov_1.0_last.pdf}
	\end{tabular}
	\caption{The first row contains results for the 2D NSE and the second row contains the results for the 3D MVE with Coulomb interaction. The first column reports the objective losses, while the second and third columns report the average and last-time-stamp relative $\ell_2$ error. }
	\label{figure_experiment}
\end{figure}
We present the results of our experiments in Figure \ref{figure_experiment}, where the first row contains the result for the Lamb-Oseen vortex (2D NSE) and the second row contains the result for the Barenblatt model (3D MVE).
The explicit solutions of these models allow us to assess the quality of the outputs of the included methods. Specifically, given a hypothesis solution $\rho_t^f$, the ground truth $\bar \rho_t$ and the interaction kernel $K$, define the relative $\ell_2$ error at timestamp $t$ as 
$Q(t) \defi \int_\Omega {\|K\ast(\rho^f_t - \bar \rho_t) (\vx)\|}/{\|K\ast \bar \rho_t(\vx)\|} d \vx$,
where $\Omega$ is some domain where $\rho_t$ has non-zero density. We are particularly interested in the quality of the convolution term $K\ast\rho^f_t$ since it has physical meanings. In the Biot-Savart kernel case, it is the velocity of the fluid, while in the Coulomb case, it is the Coulomb field. We set $\Omega$ to be $[-2, 2]^2$ for the Lamb-Oseen vortex and to $[-0.1, 0.1]^3$ for the Barenblatt model. For both models, we take $\nu = 0.1$, $t_0 = 0.1$, and $T = 1$. The neural network that we use is an MLP with $7$ hidden layers, each of which has 20 neurons.\\
From the first column of Figure \ref{figure_experiment}, we see that the objective loss of all methods has substantially reduced over a training period of 10000 iterations. This excludes the possibility that a baseline has worse performance because the NN is not well-trained, and hence the quality of the solution now solely depends on the efficacy of the method.
From the second and third columns, we see that the proposed \EINN\ method significantly outperforms the other two methods in terms of the time-average relative $\ell_2$ error, i.e. $\frac{1}{T}\int_0^T Q(t) d t$ and the relative $\ell_2$ error at the last time stamp $Q(T)$. This shows the advantage of our method.
\vspace{-.3cm}