%\documentclass[anon,12pt]{colt2023} % Anonymized submission
%\documentclass[final,12pt]{colt2023} % Include author names

%\usepackage{showkeys}

% The following packages will be automatically loaded:
% amsmath, amssymb, natbib, graphicx, url, algorithm2e
\documentclass{article}
\usepackage[final]{neurips_2023}


\title{Entropy-dissipation Informed Neural Network\\ for McKean-Vlasov Type PDEs}
\usepackage{times}
% Use \Name{Author Name} to specify the name.
% If the surname contains spaces, enclose the surname
% in braces, e.g. \Name{John {Smith Jones}} similarly
% if the name has a "von" part, e.g \Name{Jane {de Winter}}.
% If the first letter in the forenames is a diacritic
% enclose the diacritic in braces, e.g. \Name{{\'E}louise Smith}

% Two authors with the same address
% \coltauthor{\Name{Author Name1} \Email{abc@sample.com}\and
%  \Name{Author Name2} \Email{xyz@sample.com}\\
%  \addr Address}

% Three or more authors with the same address:
% \coltauthor{\Name{Author Name1} \Email{an1@sample.com}\\
%  \Name{Author Name2} \Email{an2@sample.com}\\
%  \Name{Author Name3} \Email{an3@sample.com}\\
%  \addr Address}

% Authors with different addresses:
\author{
	Zebang Shen\thanks{Authors are listed in alphabetic order.}\\
	ETH Z\"urich \\
	\texttt{zebang.shen@inf.ethz.ch} \\
	%% examples of more authors
	\And
	Zhenfu Wang$^*$ \\
	Peking University \\
	\texttt{zwang@bicmr.pku.edu.cn}
	%% \AND
	%% Coauthor \\
	%% Affiliation \\
	%% Address \\
	%% \texttt{email} \\
	%% \And
	%% Coauthor \\
	%% Affiliation \\
	%% Address \\
	%% \texttt{email} \\
	%% \And
	%% Coauthor \\
	%% Affiliation \\
	%% Address \\
	%% \texttt{email} \\
}
\usepackage{amssymb,amsmath,amsthm}
\usepackage{color}
\usepackage{makecell}
\usepackage{enumitem}
\usepackage{natbib}
\usepackage{graphicx}
\usepackage{hyperref}       % hyperlinks
\usepackage{MnSymbol}

\newcommand{\bbox}{\text{bbox}}
\newcommand{\alphapck}{\alpha_\bbox}
\newcommand{\kcycle}{\text{k-CyPCK}}
\newcommand{\cycle}{\text{-CyPCK}}

\newcommand{\I}{\mathbf{I}}
\newcommand{\Ia}{\I^\text{a}}
\newcommand{\Ib}{\I^\text{b}}
\newcommand{\Iatob}{\I^\text{a $\rightarrow$ b}}
\newcommand{\F}{\mathbf{F}}
\newcommand{\Fa}{\F^\text{a}}
\newcommand{\Fb}{\F^\text{b}}
\newcommand{\f}{\mathbf{f}}
\newcommand{\fa}{\f^\text{a}}
\newcommand{\fb}{\f^\text{b}}
\newcommand{\p}{\mathbf{p}}
\newcommand{\pa}{\p^\text{a}}
\newcommand{\pb}{\p^\text{b}}
\newcommand{\A}{\boldsymbol{\Phi}_\text{align}}
\newcommand{\G}{\mathbf{G}}
\newcommand{\C}{\mathbf{C}}
\newcommand{\Ca}{\C^\text{a}}
\newcommand{\Cb}{\C^\text{b}}
\newcommand{\cc}{\mathbf{c}}
\newcommand{\cca}{\cc^\text{a}}
\newcommand{\ccb}{\cc^\text{b}}
\newcommand{\Irec}{\I_\text{Recon}}
\newcommand{\M}{\mathbf{M}}
\newcommand{\Mrec}{\M_\text{Recon}}
\newcommand{\loss}{\mathcal{L}}
\newcommand{\T}{\mathcal{T}}
\newcommand{\W}{\mathcal{W}}
\newcommand{\Id}{\mathcal{I}}


\newtheorem{theorem}{Theorem}
\newtheorem{proposition}{Proposition}
\newtheorem{assumption}{Assumption}
\newtheorem{remark}{Remark}
\newtheorem{lemma}{Lemma}
\newcommand{\Law}{\mathrm{Law}}
\newcommand{\red}[1]{\textcolor{red}{#1}}
\newcommand{\ud}{\mathrm{d}}
\newcommand{\X}{\mathcal{X}}
%\newcommand{\ud}{\,\mathrm{d}}
\newcommand{\udiv}{\, \mathrm{div}}
\newcommand{\Uniform}{\mathrm{Uniform}}
\let\KL\relax
\newcommand{\KL}{\mathbf{KL}}
\newcommand{\EINN}{\texttt{EINN}}
\newcommand{\defi}{\overset{\operatorname{def}}{=}}
\newcommand{\Lip}{\mathrm{Lip}}

\begin{document}

\maketitle

\begin{abstract}%
    The McKean-Vlasov equation (MVE) describes the collective behavior of particles subject to drift, diffusion, and mean-field interaction. In physical systems, the interaction term can be singular, i.e. it diverges when two particles collide. Notable examples of such interactions include the Coulomb interaction, fundamental in plasma physics, and the Biot-Savart interaction, present in the vorticity formulation of the 2D Navier-Stokes equation (NSE) in fluid dynamics.
    Solving MVEs that involve singular interaction kernels presents a significant challenge, especially when aiming to provide rigorous theoretical guarantees. In this work, we propose a novel approach based on the concept of entropy dissipation in the underlying system. We derive a potential function that effectively controls the KL divergence between a hypothesis solution and the ground truth.
    Building upon this theoretical foundation, we introduce the Entropy-dissipation Informed Neural Network (\EINN) framework for solving MVEs. In \EINN, we utilize neural networks (NN) to approximate the underlying velocity field and minimize the proposed potential function. By leveraging the expressive power of NNs, our approach offers a promising avenue for tackling the complexities associated with singular interactions.
    To assess the empirical performance of our method, we compare \EINN\ with SOTA NN-based MVE solvers. The results demonstrate the effectiveness of our approach in solving MVEs across various example problems.
\end{abstract}

%\begin{keywords}%
%  Entropy dissipation, McKean-Vlasov equation, 2D Navier-Stokes equation%
%\end{keywords}
% Importance and appeal of children's drawings
Children's depictions of the human figure are highly expressive and varied.
As one of the very first subjects children attempt to draw, the representation begins as an almost unintelligible cloud of scribbles. 
As the child grows, their representation of the human figure becomes more developed and is extended to graphically represent many different types of characters: people, animals, and even personified objects (see Figure 1).

Who among us has not wished, either as a child or as an adult, to see such figures come to life and move around on the page?
Sadly, while it is relatively fast to produce a single drawing, creating the sequence of images necessary for animation is a much more tedious endeavor, requiring discipline, skill, patience, and sometimes complicated software.
As a result, most of these figures remain static upon the page.

% We built a system to animate them.
Inspired by the importance and appeal of the drawn human figure, we design and build a system to automatically animate it given an in-the-wild photograph of a child's drawing. 
Our system is fast, intuitive, and robust to much of the variation present in these types of drawings, making it well-suited to allow our target audience--children--to see their own characters coming to life.
The system is comprised of four stages: figure detection, segmentation masking, pose estimation/rigging, and animation. 
We describe each stage and identify common causes of failure in each. 
For object detection and pose estimation, we make use of existing computer vision models designed to detect human figures and joints in photographs; we fine-tune these models for use with children's drawings.
For segmentation, we present a straightforward, image processing-based method that, for animation purposes, is more useful and accurate than segmentation masks obtained from a fine-tuned object detection model.
During the animation step, we take advantage of the \textit{twisted perspective} commonly seen in children’s drawings to retarget motion capture data onto the character in a novel and appealing way.

% We use existing machine learning models. However, given the wide domain gap it's not clear how much fine-tuning data was needed. So we ran some experiments to find out and report it.
While our system leverages existing models and techniques, most are not directly applicable to the task due to the many differences between photographic images and simple pen and paper representations. 
To this end, we couple the presentation of our system with a set of experiments exploring the relationship between fine-tuning training set size and success rates.
We also include a perceptual study validating viewer preference for incorporating \textit{twisted perspective} into the motion retargeting step.

We validate the desirability and appeal of our system by building and publicly releasing a version of it as the \AD Demo \,\cite{animateddrawings}.
Launched in December 2021, this demo has been used by millions of people around the world to animate their children's drawings.
Inspired by this reception, our second contribution is The Amateur Drawings Dataset: \hjs{180,000 drawings and user-accepted annotations collected, with consent, through the demo. See Section \ref{sec:UI} for a description of how the annotations were generated.}
We believe this dataset will be a resource to researchers from various fields seeking to better understand the space of amateur drawings, evaluate new algorithms in this domain, or develop new drawing-based tools in general.

To summarize, our contributions are as follows:
\begin{enumerate}
    \item 
    We explore the problem of automatic sketch-to-animation for children's drawings of human figures and present a framework that achieves this effect. We also present a set of experiments determining the amount of training data necessary to achieve high levels of success and a perceptual study validating the usefulness of our motion retargeting technique.
    \item To encourage additional research in the domain of amateur drawings, we present a first-of-its-kind dataset of 180,000 user-submitted amateur drawings, along with user-accepted bounding box, segmentation mask, and joint location annotations.
\end{enumerate}

Upon acceptance of this paper, we plan to publicly release the Amateur Drawings Dataset, project code, and fine-tuned model weights.



%\section{Preliminaries}
%\subsection{Periodic Boundary Condition}
%
%\subsection{Neural Ordinary Differential Equation}

\section{Method}
\label{s:method}

We consider the 3D euclidean space $\Real^3$ with points $p=(x,y,z)\in\Real^3$. We discretize the unit cube $\gC=[0,1]^3$ with a 3D voxel grid $\gG=\set{p_I}$, with nodes $p_I$ indexed by $I=(i,j,k)$, $i,j,k\in [n]=\set{1,\ldots,n}$, \ie, $p_I=(x_{ijk},y_{ijk},z_{ijk})$. We denote by $h=n^{-1}$, and by $N=n^3$ the total number of nodes.   
We represent our reconstructed surface as a zero level of a scalar function $f$ defined over the cube $\gC$. $f$ is defined by prescribing its values at the grid's nodes $f_I\in\Real$ and trilinear interpolating in each voxel. We will denote by $f(p)$ the interpolated value at point $p$. 

Given an input point cloud consisting of $m$ points $q_k\in\Real^3$ with or without (unit norm) normals $n_k\in \Real^3$, $k\in [m]$, our goal is to compute $f$ so that its zero level set approximates the unknown surface, \ie, 
\begin{equation}
    \gS = \set{p\in\gC \ \vert \ f(p)=0}.
\end{equation}
Our approach to compute $f$ is to minimize a loss function of the form
\begin{equation}
    \gL = \gL_{\text{data}} + \gL_{\text{prior}}
\end{equation}
where 
\begin{equation}\label{e:loss_data}
    \gL_{\text{data}} = \frac{\lambda_{\text{p}}}{m}\sum_{k=1}^m \abs{f(q_k)}^2 + \frac{\lambda_{\text{n}}}{m}\sum_{k=1}^m \norm{\nabla f(q_k) - n_k}^2
\end{equation}
where $\norm{\cdot}$ is the standard euclidean norm in $\Real^3$, $\nabla f(p) \in \Real^3$ is the gradient of $f$ sampled at point $p$. Note that $\nabla f$ is defined in interior of voxels, which is generically where the input points $q_k$ resides. $\gL_{\text{data}}$ is the standard data loss encouraging the zero level to pass through the input points $q_k$, and its normals (defined by gradients of $f$) to coincide with input normals $n_k$. 

The prior, $\gL_{\text{prior}}$, is the main contribution of this work, where we combine two novel losses,
\begin{equation}
    \gL_{\text{prior}} = \lambda_{\text{v}} \gL_{\text{viscosity}} + \lambda_{\text{c}} \gL_{\text{coarea}}
\end{equation}
Intuitively, the viscosity loss optimizes for a smooth Signed Distance Function (SDF) solutions, avoiding auxiliary bad minima of the Eikonal equation, while the coarea loss strives to minimize the area of the zero level surface. Our loss has $4$ hyper-parameters $\lambda_{\text{p}},\lambda_{\text{n}},\lambda_{\text{v}},\lambda_{\text{c}}$. We provide more details on these priors next. 


\subsection{Viscosity Loss}\label{ss:viscosity_loss}
The goal of the viscosity loss is to make $f$ approximate an SDF over $\gC$. Given boundary conditions asking $f$ to vanish on some closed compact surface $\gS$, the SDF solves the Eikonal equation PDE, \ie, $\norm{\nabla f(p)}=1$, in a certain well defined sense (viscosity). This motivated some previous work to directly optimize the Eikonal loss \citep{gropp2020implicit,sitzmann2020implicit}
\begin{equation}\label{e:loss_eikonal}
    \gL_{\text{eikonal}} = \int_\gC \Big (\norm{\nabla f(p)}-1\Big )^2 dp
\end{equation}
\begin{wrapfigure}[14]{r}{0.28\textwidth}\vspace{-15pt}
  \begin{center}
    \includegraphics[width=0.25\textwidth]{figs/illustrations/eikonl_1d.png}
  \end{center}
  \caption{Two global minimizers of the Eikonal loss over a grid in 1D. Top solution is not an SDF. }\label{fig:eikonal_1d}
\end{wrapfigure}
Unfortunately, the Eikonal loss has many undesirable minima which are not SDFs. Figure \ref{fig:eikonal_1d} shows a 1D example: both depicted solutions (denoted $f$) vanish at the input points $q_1,q_2$ (black points) and globally minimize the Eikonal loss over the grid (grid points are shown in blue). The INR works mentioned above use neural networks for representing $f$ which injects an inductive bias avoiding these bad minima, however on grids, minimizing \eqref{e:loss_eikonal} cannot avoid these solutions. See, \eg, middle column in Figure \ref{fig:teaser}. 

More classical Eikonal solvers do work with grids however use mostly fast marching or sweeping methods \citep{osher1988fronts,sethian1996fast,zhao2005fast,chacon2012fast}. Namely, use a special discretization of the Eikonal equation favoring the viscosity  solution of the Eikonal \cite{rouy1992viscosity}, and update node values according to a moving front \cite{sethian1996fast}. Since this discretization is up-wind (will only propagate values in one direction) and requires choosing the maximal among its solution, its success in adaptation to a loss is not clear. 

We use a different approach to build a loss favoring SDF solutions over grids motivated by the vanishing viscosity method \cite{crandall1983viscosity}. Namely, adding to the Eikonal PDE a small perturbation of the Laplacian of $f$ (denoted by $\Delta f$), \ie, $\norm{\nabla f(p)}-1 - \eps\Delta f(p)=0$, makes the PDE semi-linear elliptic \citep{calder2018lecture}, and hence with suitable boundary conditions it is uniquely solvable inside $\gS$ with a smooth solution, approaching the viscosity positive distance function to the boundary as $\eps\too 0$. Similarly, for $1-\norm{\nabla f(p)} - \eps \Delta f(p)=0$ the solution approaches the negative distance function inside the domain. 
Motivated by the vanishing viscosity principle we suggest the following viscosity loss:
\begin{equation}\label{e:loss_viscosity_eikonal}
\gL_{\text{viscosity}} = \int_\gC \Big((\norm{\nabla f (p)}-1)\mathrm{sign}(f(p)) - \eps \Delta f(p)\Big)^2 dp
\end{equation}
We discretize this loss over the grid $\gG$ by replacing the first order derivatives and second order derivatives with symmetric finite  differences, \ie,
\begin{align*}
    D_x f_I=D_x f_{i,j,k} = \frac{f_{i+1,j,k}-f_{i-1,j,k}}{2h}, \quad D^2_x f_I = D^2_x f_{i,j,k}=\frac{f_{i+1,j,k}-2f_{i,j,k}+f_{i-1,j,k}}{h^2}
\end{align*}
and similarly for $D_y$ and $D_z$. We use these discrete operators to approximate the gradient $\widehat{\nabla} f(p_I) = (D_x f_I, D_y f_I, D_z f_I)$ and Laplacian $\widehat{\Delta}f(p_I) = D_x^2f_I + D_y^2 f_I + D_z^2 f_I$. The discretized viscosity loss now takes the form
\begin{equation}
    \widehat{\gL}_{\text{viscosity}} = \frac{1}{N}\sum_{I} \Big((\|\widehat{\nabla} f (p_I)\|-1)\mathrm{sign}(f(p_I)) - \eps \widehat{\Delta} f(p_I)\Big)^2
\end{equation}



\subsection{Coarea loss}\label{ss:coarea_loss}
The coarea loss is approximating the area of the zero level set, and therefore incorporating it in the optimization pushes the reconstructed surface to be economic in area. 

First, similarly to  \citep{yariv2021volume} we use the centered Laplace CDF
\begin{equation}
   \Psi\beta(s)= \begin{cases}
   \frac{1}{2}\exp\parr{\frac{s}{\beta}} & s\leq 0 \\ 1-\frac{1}{2}\exp\parr{-\frac{s}{\beta}} & s\geq  0
   \end{cases}
\end{equation} to transform the SDF $f$ to a smooth approximation of the indicator function:
\begin{equation}
    \chi_\beta(p)=\Psi\beta (-f(p))
\end{equation}
As $\beta\too 0$, $\chi_\beta$ converges to an indicator function leading to $1$ inside $\gS$ and $0$ outside. The coarea loss is defined as 
\begin{equation}
    \gL_{\text{coarea}} = \int_\gC \norm{\nabla \chi_\beta (p)} dp
\end{equation}
To understand why this loss approximates the area of $\gS$ we can use the coarea formula \citep{rindler2018calculus}:
\begin{equation}\label{e:coarea}
    \int_\gC \norm{\nabla \chi_\beta(p)}dp = \int_{-\infty}^{\infty} \mathrm{area}(\chi_\beta^{-1}(s))ds,
\end{equation}
where $\chi_\beta^{-1}(s)=\set{p\ \vert \ \chi_\beta(p)=s}$ is the preimage of the value $s$. Since $\chi_x(p)\in [0,1]$ the r.h.s.~integral can be restricted to the interval $[0,1]$, and therefore the coarea loss averages the area of the level sets of $\chi_\beta$. Next,  $$\chi_\beta^{-1}(s)= \set{p\ \vert \ \Psi\beta (-f(p)) = s } = \{p\ \vert \ f(p) = -\Psi\beta^{-1} (s) \} = f^{-1}(-\Psi\beta^{-1} (s)),$$
\begin{wrapfigure}[11]{r}{0.28\textwidth}\vspace{-20pt}
  \begin{center}
  \includegraphics[width=0.25\textwidth]{figs/semi.png}
  \end{center}
  \caption{Reconstruction of a semisphere point cloud (white dots) without (left) and with (right) coarea loss. }\label{fig:coarea_semisphere}
\end{wrapfigure}

which shows that the level set $s\in (0,1)$ of $\chi_\beta$ is the level set $-\Psi\beta^{-1}(s)$ of the SDF $f$. As $\beta\too 0$, $-\Psi\beta^{-1}(s)\too 0$ for all $s\in (0,1)$ (and uniformly in $(\eps,1-\eps)$ for fixed $\eps>0$). Therefore the average of the level set area (\ie, the r.h.s.~of \eqref{e:coarea}) converges to the area of $f^{-1}(0)=\gS$. Figure \ref{fig:teaser} (right) shows how removing the coarea loss introduces an extraneous zero level set, and hence results in an undesired surface part. Figure \ref{fig:coarea_semisphere} shows a comparison of a reconstruction of semisphere with and without coarea. In the experiments section we provide more ablation tests with the coarea and viscosity losses.

To discretize the coarea loss we let $w_I$ denote the centers of grid's voxels, and note that $\nabla \chi_\beta(w_I) = \Phi_\beta(-f(w_I))\nabla f(w_I)$, where 
\begin{equation*}
    \Phi_\beta(s) = \frac{1}{2\beta}\exp\parr{\frac{\abs{s}}{\beta}}
\end{equation*}
is the PDF of the Laplace distribution, and $\nabla f(w_I)$ is computed as a linear combination of the voxel's corner values $f_{I_1},\ldots,f_{I_8}$, see more details in the Appendix. We end up with the discretized loss:
\begin{equation}
    \widehat{\gL}_{\text{coarea}} = \frac{1}{N}\sum_{I}\Phi_\beta(-f(w_I))\norm{\nabla f(w_I)}
\end{equation}
This loss is usually incorporated with a small hyper-parameter $\lambda_{\text{c}}$ with the purpose of eliminating redundant surface parts.


\section{Analysis} \label{section_analysis}
In this section, we focus on the torus case, i.e. $\X = \Pi^d$ is a box with the periodic boundary condition.
This is a typical setting considered in the literature as the universal function approximation of NNs only holds over a compact set. Moreover, the boundary integral resulting from integration by parts vanishes in this setting, making it amenable for analysis purposes.
For completeness, we provide a discussion on the unbounded case, i.e. $\X = \sR^d$ in the Appendix \ref{appendix_unbounded}, which requires additional regularity assumptions.
% \red{Mention that we focus on the torus case.}
% We start by defining the modulated (interaction) energy and modulated free energy for two probability densities $\rho, \bar \rho$.  
% Again $\mathcal{X}$ denotes the underlying space \red{which can be the whole space $\mathbb{R}^d$} or the torus $\Pi^d $ which can be identified with $[-L, L]^d$ with the periodic boundary condition and $L>0$. 
Given the MVE (\ref{eqn_MVE}), if $K$ is bounded, it is sufficient to choose the Lyapunov functional $L(\rho_t^f, \bar \rho_t)$ as the KL divergence (please see Theorem \ref{theorem_bounded_K} in the appendix). But for the singular Coulomb kernel, we need also to consider the modulated energy as in \citep{serfaty2020mean}
\begin{equation}
    \text{(Modulated Energy)}\quad 
	\label{DefModEnergy}
	F(\rho, \bar \rho) \defi \frac 1 2 \int_{\mathcal{X}^2} g(x- y) \ud (\rho - \bar \rho)(x) \ud (\rho - \bar \rho )(y), 
\end{equation}
where $g$ is the fundamental solution to the Laplacian equation in $\mathbb{R}^d$, i.e. $- \Delta g = \delta_0$, and the Coulomb interaction reads $K= -\nabla g$ (see its closed form expression in equation (\ref{eqn_coulomb_interaction})). If we are only interested in the deterministic  dynamics with Coulomb interactions, i.e. $\nu =0$ in equation (\ref{eqn_MVE}), it suffices to choose $L(\rho_t^f, \bar \rho) $ as  $F(\rho_t^f, \bar \rho_t)$ (please see Theorem \ref{ThmCoul}). But if we consider the system with  Coulomb interactions and  diffusions, i.e. $\nu >0$, we shall combine the KL divergence and the modulated energy to form the modulated free energy as in \cite{bresch2019modulated}, which reads 
\begin{equation}
    \text{(Modulated Free Energy)}\quad 
	\label{DefModFree}
	E(\rho, \bar \rho) \defi \nu \KL (\rho, \bar \rho) +  F(\rho, \bar \rho). 
\end{equation}
This definition agrees with the physical meaning that ``Free Energy = Temperature $\times $ Entropy + Energy", and we note that the temperature is proportional to the diffusion coefficient $\nu$. We remark also for two probability densities $\rho$ and $\bar \rho$, $F(\rho, \bar \rho) \geq 0$ since by looking in the Fourier domain $F(\rho, \bar \rho) = \int \hat g(\xi) |\widehat{\rho - \bar \rho}(\xi)|^2  \ud \xi \geq 0$ as $\hat g(\xi) \geq 0$. Moreover, $F(\rho, \bar \rho)$ can be regarded as a negative Sobolev norm for $\rho- \bar \rho$, which metricizes weak convergence.\\
To obtain our main stability estimate, we first obtain the time evolution of the KL divergence.  
%%%% Evolution of the KL divergence 
\begin{lemma}[Time Evolution of the KL divergence] \label{TimeEvolKLMV} 
	Given the hypothesis velocity field $f=f(t, x) \in C^1_{t, x}$. Assume that $(\rho_t^f)_{t \in [0, T]}$ and $(\bar \rho_t)_{t \in [0, T]}$ are classical solutions to equation (\ref{eqn_CE}) and equation (\ref{eqn_MVE_CE}) respectively.  It holds that (recall the definition of $\delta_t$ in equation (\ref{eqn_perturbation})) 
	\[
	\frac{\ud }{\ud t} \int_{\mathcal{X}} \rho^f_t \log \frac{\rho^f_t }{\bar \rho_t} = - \nu \int_{ \X}\rho^f_t |\nabla \log \frac{\rho^f_t}{\bar \rho_t}|^2 +  \int_{\X} \rho^f_t K * (\rho^f_t - \bar \rho_t ) \cdot \nabla  \log  \frac{\rho^f_t }{\bar \rho_t} +  \int_{\X} \rho^f_t \delta_t \cdot \nabla \log \frac{\rho^f_t }{\bar \rho_t }, 
	\]
	where $\X$ is the tours $\Pi^d$. All the integrands are evaluated at $\vx$. 
\end{lemma}

We refer the proof of this lemma and all other lemmas and theorems in this section to the appendix \ref{detailed proof}. We remark that to have the existence of classical solution $(\bar \rho_t)_{t \in [0, T]}$, we definitely need the regularity assumptions on $-\nabla V$ and on $K$. But the linear term $- \nabla V $ will not contribute to the evolution of the relative entropy. See \citep{jabin2018quantitative} for detailed discussions. \\
%%%% Evolution of the Modulated Energy 
Similarly, we have the time evolution of the modulated energy as follows.
\begin{lemma}[Time evolution of the modulated energy] \label{ModuEnergyEvo} Under the same assumptions as in Lemma \ref{TimeEvolKLMV}, given the diffusion coefficient $\nu \geq 0$, it holds that (recall the definition of $\delta_t$ in equation (\ref{eqn_perturbation})) 
	\[
	\begin{split}
		\frac{\ud }{\ud t } F(\rho_t^f, \bar \rho_t)&  =  - \int_{\mathcal{X}} \rho_t^f \|K *(\rho_t^f - \bar \rho_t)\|^2 - \int_{\mathcal{X}} \rho_t^f \, \delta_t \cdot K * (\rho_t^f - \bar \rho_t ) + \nu \int_{\mathcal{X}} \rho^f_t \, K * (\rho_t^f - \bar \rho_t )\cdot \nabla \log \frac{\rho_t^f}{\bar \rho_t} \\
		&  - \frac{1}{2} \int_{\mathcal{X}^2} K(x-y) \cdot \Big( \mathcal{A}[\bar \rho_t](x) - \mathcal{A}[\bar \rho_t](y) \Big) \ud (\rho_t^f - \bar \rho_t )^{\otimes 2 }(x, y) \\
	\end{split}
	\]
	where we recall that the operator $\gA$ is defined in equation (\ref{eqn_operator_A}).
\end{lemma}






%%%%%%% The 2D Navier-Stokes case  
By  Lemma \ref{TimeEvolKLMV} and careful analysis, in particular by rewriting the Biot-Savart law in the divergence of a bounded matrix-valued function (\ref{eqn_K_as_divergence}), we obtain the following estimate for the 2D NSE.

\begin{theorem}[Stability estimate of the 2D NSE] \label{NSMainEstimate}	
	% Consider the 2D NSE in the vorticity formulation (\ref{eqn_MVE}) with $V=0$. 
    Notice that when $K$ is the Biot-Savart kernel, $\udiv K =0$. Assume that the initial data $\bar \rho_0 \in C^3(\Pi^d)$ and there exists $c>1$ such that $\frac 1 c \leq \bar \rho_0 \leq c$.  Assume further the hypothesis velocity field $f(t, x) \in C^1_{t, x}$. Then it holds that 
	\[
	\sup_{t \in [0, T]} \int_{\Pi^d} \rho_t^f \log \frac{\rho_t^f}{\bar \rho_t} \ud x \leq \frac{e^C}{\nu}  R(f), 
	\]
	where $C = \int_0^\infty     M(t) \ud t < \infty$ with 
	$M(t) \defi \|\nabla \log \bar \rho_t\|_{L^\infty}^2/2\nu + 2\Big\| {\nabla^2 \bar \rho_t}/{\bar \rho_t} \Big\|_{L^\infty}$. 
	
\end{theorem} 
We remark that given $\bar \rho_0$ is smooth enough and fully supported on $\X$, one can propagate the regularity to finally show the finiteness of $C$.
See detailed computations as in \cite{guillin2021uniform}. 
We give the complete proof in the appendix \ref{detailed proof}. This theorem tells us that as long as $R(f)$ is small, the KL divergence between $\rho_t^f$ and $\bar \rho_t$ is small and the control is uniform in time $t \in [0, T]$ for any $T$. Moreover, we highlight that $C$ is independent of $T$, and our result on the NSE is significantly better than the average-in-time and exponential-in-$T$ results from \citep{deerror}.\\
% We state this theorem on torus for simplicity and \red{one may expect similar result on $\mathbb{R}^d$}. \red{Also the $C^\infty$ condition can be relaxed for instance to $C^2$}. 
%%%%%  Remark on The case with bounded interactions $K \in L^\infty
%%%% Leave in the appendix  
%%%% The deterministic case with Coulomb 
%%%% Lemma of the evolution of the modulated free energy 
To treat the MVE (\ref{eqn_MVE}) with Coulomb interactions, we exploit the time evolution of the modulated free energy $E(\rho_t^f, \bar \rho_t)$. Indeed, combining Lemma \ref{TimeEvolKLMV} and Lemma \ref{ModuEnergyEvo}, we arrive at the following identity.  
\begin{lemma}[Time evolution of the modulated free energy]\label{TimeEvoMFE} 
	Under the same assumptions as in Lemma \ref{TimeEvolKLMV}, one has (recall the definitions of $\delta_t$ and $\gA$ in (\ref{eqn_perturbation}) and (\ref{eqn_operator_A}) respectively) 
	\[
	\begin{split}
		\frac{\ud }{\ud t } E(\rho_t^f, \bar \rho_t)  & = -\int_{\mathcal{X}} \rho_t^f  \Big|K * (\rho_t^f - \bar \rho_t) - \nu \nabla \log \frac{\rho_t^f }{\bar \rho_t}\Big|^2 - \int_{\mathcal{X}} \rho_t^f \, \delta_t \cdot \Big( K * (\rho_t^f - \bar \rho_t ) - \nu \nabla \log \frac{\rho_t^f}{\bar \rho_t }\Big) \\
		&  - \frac{1}{2} \int_{\mathcal{X}^2} K(x-y) \cdot \Big( \mathcal{A}[\bar \rho_t](x) - \mathcal{A}[\bar \rho_t](y) \Big) \ud (\rho_t^f - \bar \rho_t )^{\otimes 2 }(x, y).
	\end{split}
	\]
	% where we recall that the operator $\gA$ is defined in \eqref{eqn_operator_A}.
\end{lemma}
\vspace{-1mm}
%%%%%%%%%%%%%%Main theorem of the Coulomb case 
Inspired by the mean-field convergence results as in \cite{serfaty2020mean} and \cite{bresch2019modulated}, we finally can control the growth of $E(\rho_t^f, \bar \rho_t)$ in the case when $\nu >0$,  and $F(\rho_t^f, \bar \rho_t)$ in the case when $\nu =0$. Note also that $E(\rho_t^f, \bar \rho_t )$ can also control the KL divergence when $\nu >0$. 
\begin{theorem} [Stability estimate of MVE with Coulomb interactions] \label{ThmCoul}
	Assume that for $t \in [0, T]$, the underlying velocity field $\mathcal{A}[\bar \rho_t](x)$ is Lipschitz  in $x$ and
	$\sup_{t \in [0, T]} \|\nabla \mathcal{A}[\bar \rho_t](\cdot)\|_{L^\infty} = C_1  < \infty.$ 
	Then there exists $C>0$ such that 
	\[
	\sup_{t \in [0, T]} \nu\,  \KL (\rho_t^f, \bar \rho_t) \leq \sup_{t \in [0, T]} E(\rho_t^f, \bar \rho_t) \leq \exp(C C_1 T) R(f).  
	\]
	In the deterministic case when $\nu =0$, under the same assumptions, it holds that 
	\[
	\sup_{t \in [0, T]} F( \rho_t^f, \bar \rho_t) \leq \exp(CC_1T ) R(f). 
	\]
\end{theorem}
\vspace{-1mm}
Recall the definition of the operator $\gA$ in \eqref{eqn_operator_A}. Given that $\mathcal{X}= \Pi^d$, and $\bar \rho_0$ is smooth enough and bounded from below, one can propagate regularity to obtain the Lipschitz condition for $\mathcal{A}[\bar \rho_t]$. See the proof and the discussion on the Lipschitz assumptions on $\mathcal{A}[\bar \rho_t](\cdot)$ in the appendix \ref{detailed proof}. 
\paragraph{Approximation Error of Neural Network}
Theorems \ref{NSMainEstimate} and \ref{ThmCoul} provide the error estimation guarantee for the proposed \EINN\ loss (\ref{eqn_self_consistency_potential}).
Suppose that we parameterize the velocity field $f=f_\theta$ with an NN parameterized by $\theta$, as we did in Section \ref{section_NN_parameterization} and let $\tilde f$ be the output of an optimization procedure when $R(f_\theta)$ is used as objective.
In order the explicitly quantify the mismatch between $\rho^{\tilde f}_t$ and $\bar \rho_t$, we need to quantify two errors: (i) Approximation error, reflecting how well the ground truth solution can be approximated among the NN function class of choice; (ii) Optimization error, involving minimization of a highly nonlinear non-convex objective. 
In the following, we show that for a function class $\gF$ with sufficient capacity, there exists at least one element $\hat f\in\gF$ that can reduce the loss function $R(\hat f)$ as much as desired.
We will not discuss how to identify such an element in the function class $\gF$ as it is independent of our research and remains possibly the largest open problem in modern AI research.
To establish our result, we make the following assumptions.
\begin{assumption} \label{ass_appendix_initial}
	$\rho_0$ is sufficiently regular, such that $\nabla \log \rho_0 \in \gL^\infty(\X)$ and $\bar f_t  = \mathcal{A}[\bar \rho_t] \in W^{2,\infty}(\X)$. $\nabla V$ is Lipschitz continuous. Here $W^{2,\infty}(\X)$ stands for the Sobolev norm of order $(2, \infty)$ over $\X$.
\end{assumption}
\vspace{-1mm}
We here again need to propagate the regularity for $f_t$ at least for a time interval $[0, T]$. It is easy to do so for the torus case, but for the unbounded domain, there are some technical issues to be  overcome. Similar assumptions are also needed in some mathematical works for instance in \cite{jabin2018quantitative}. 
% [cite some papers]
% \begin{remark}
%     For the torus case, if $\rho_0$ is bounded from above and below, and chosen to be in $C^3$, we can propagate the regularity of $\rho_0$ to $\bar \rho_t$ for $t \in [0, T]$. We can then obtain $\bar f_t \in \gC^3(\X)$.
% \end{remark}
We also make the following assumption on the capacity of the function class $\gF$, which is satisfied for example by NNs with tanh activation function \citep{DERYCK2021732}.
\begin{assumption} \label{ass_appendix_approximation}
	The function class is sufficiently large, such that there exists $\hat f \in \gF$ satisfying $\hat f_t \in \gC^3(\X)$ and $\|\hat f_t - \bar f_t\|_{W^{2, \infty}(\X)} \leq \epsilon$ for all $t\in[0, T]$.
\end{assumption}
\begin{theorem} \label{thm_approximation_error_NN}
	Consider the case where the domain is the torus. 
	Suppose that Assumptions \ref{ass_appendix_initial} and \ref{ass_appendix_approximation} hold. 
    For both the Coulomb and the Biot-Savart cases, there exists $\hat f\in\gF$ such that $R(\hat f) \leq C(T)\cdot(\epsilon \cdot\ln 1/\epsilon)^2$, where $C(T)$ is some constant independent of $\epsilon$. Here $R$ is the \EINN\ loss (\ref{eqn_self_consistency_potential}).
\end{theorem}
% \red{Maybe mentioning the difficulty in proving this theorem to promote the novelty of the result.}
% \red{Can we derive the results for unbounded domain?}
% \red{This is just the result for the Coulomb case. We should also prove the NSE case.}
The major difficulty to overcome is the lack of Lipschitz continuity due to the singular interaction. We successfully address this challenge by establishing that the contribution of the singular region $(\|\vx\|\leq\epsilon)$ to $R(\hat f)$ can be bounded by $O((\epsilon \log \frac{1}{\epsilon})^2)$.
Please see the detailed proof in Appendix \ref{appendix_approximation_error_NN}.



%1. Previous analysis provides control of the solution quality by $R(f)$
%2. The next question is how small $R(f)$ can be
%3. Two errors: approximation error and optimization error. The latter is an open question. In this section, we focus on the approximation error.
%4. Conclusions on NN approximation in Sobolev norm
%5. List assumptions
%6. describe conclusion.

% Discuss the long-time guarantee (without exponential in T, under additional assumptions)

% Periodic solution to Taylor-Green vortex

% Revise the discussion on "expected"




\setlength{\tabcolsep}{1.6mm}{
\renewcommand\arraystretch{1.1}
\begin{table}[ht]
  \centering
  \scalebox{0.9}{
  \begin{tabular}{llcccc}
    \toprule
    &\multirow{2}*{Methods} & \multirow{2}*{Sal.} &   \multicolumn{2}{c}{VOC} & MS~COCO \\
    \cmidrule(r){4-5}\cmidrule(r){6-6}
    &&&\texttt{val}&\texttt{test}&\texttt{val}\\
    \hline
    \multirow{13}*{\rotatebox{90}{ResNet-50}}
    &IRN~\cite{irn}          \tiny{CVPR'19}     &              & 63.5       & 64.8          & 42.0  \\
    &LayerCAM~\cite{layercam}\tiny{TIP'21}      &              & 63.0       & 64.5          & -     \\
    &AdvCAM~\cite{advcam}    \tiny{CVPR'21}     &              & 68.1       & 68.0          & 44.2  \\
    &RIB~\cite{rib}          \tiny{NeurIPS'21}  &              & 68.3       & 68.6          & 44.2  \\
    &ReCAM~\cite{recam}      \tiny{CVPR'22}     &              & 68.5       & 68.4          & 42.9  \\
    % \rowcolor{Gray}
    &\cellcolor{Gray}IRN+\texttt{LPCAM}    &\cellcolor{Gray} & \cellcolor{Gray}68.6    & \cellcolor{Gray}68.7      & \cellcolor{Gray}44.5  \\
    &SIPE~\cite{sipe}        \tiny{CVPR'22}     &              & 68.8       & 69.7          & 40.6  \\
    &OOD~\cite{ood}+Adv      \tiny{CVPR'22}     &              & 69.8       & 69.9          & -     \\
    &AMN~\cite{amn}          \tiny{CVPR'22}     &              & 69.5       & 69.6          & 44.7  \\
    &\cellcolor{Gray}AMN+\texttt{LPCAM}    &\cellcolor{Gray} & \cellcolor{Gray}70.1    &\cellcolor{Gray} 70.4      & \cellcolor{Gray}45.5  \\ 
    &ESOL~\cite{esol}        \tiny{NeurIPS'22}  &              & 69.9$^*$   & 69.3$^*$      & 42.6  \\
    &CLIMS~\cite{clims}      \tiny{CVPR'22}     &              & 70.4$^*$   & 70.0$^*$      & -     \\
    &EDAM~\cite{edam}        \tiny{CVPR'21}     &\checkmark    & 70.9$^*$   & 71.8$^*$      & -     \\
    &\cellcolor{Gray}EDAM+\texttt{LPCAM}  &\cellcolor{Gray}\checkmark & \cellcolor{Gray}71.8$^*$ &\cellcolor{Gray} 72.1$^*$& \cellcolor{Gray}42.1\\
    \hline
    \multirow{9}*{\rotatebox{90}{WideResNet-38}}
    &Spatial-BCE~\cite{sbce} \tiny{ECCV'22}     &              & 70.0       & 71.3      & 35.2  \\
    &BDM~\cite{bdm}          \tiny{ACMMM'22}    &\checkmark    & 71.0       & 71.0      & 36.7  \\ 
    &RCA~\cite{rca}+OOA      \tiny{CVPR'22}     &\checkmark    & 71.1       & 71.6      & 35.7  \\
    &RCA~\cite{rca}+EPS      \tiny{CVPR'22}     &\checkmark    & 72.2       & 72.8      & 36.8  \\
    &HGNN~\cite{hgnn}        \tiny{ACMMM'22}    &\checkmark         & 70.5$^*$   & 71.0$^*$  & 34.5  \\ 
    &EPS~\cite{eps}          \tiny{CVPR'21}     &\checkmark         & 70.9$^*$   & 70.8$^*$  & -     \\
    &RPIM~\cite{rpim}        \tiny{ACMMM'22}    &\checkmark         & 71.4$^*$   & 71.4$^*$  & -     \\ 
    &L2G~\cite{l2g}          \tiny{CVPR'22}     &\checkmark         & 72.1$^*$   & 71.7$^*$  & 44.2  \\
    \hline
    \multirow{2}*{\rotatebox{90}{\small{DeiT-S}}}
    &MCTformer~\cite{mctformer}    \tiny{CVPR'22}     &                 & 71.9$^{\dag}$  & 71.6$^{\dag}$   & 42.0  \\
    &\cellcolor{Gray}MCTformer+\texttt{LPCAM}      &\cellcolor{Gray} & \cellcolor{Gray}72.6$^{\dag}$  & \cellcolor{Gray}72.4$^{\dag}$  &\cellcolor{Gray} 42.8 \\
    \bottomrule
  \end{tabular}}
  \vspace{-2mm}
  \caption{The mIoU results (\%) based on DeepLabV2 on VOC and MS~COCO. The side column shows three backbones of multi-label classification model. ``Sal.'' denotes using saliency maps. * denotes the segmentation model is pre-trained on MS~COCO. $^\dag$ denotes the segmentation model is pre-trained on VOC.
  }
  \vspace{-6mm}
  \label{table_related}
\end{table}
}



\section{Experimental Results}
\label{sec:experiments}
\subsection{Training Details}
\cite{Kalantari2017DeepHD} provides the first dataset specifically designed for multi-exposure HDR fusion under large motion. It consists of 74 training sets, which we use to supervise the training of our model. We crop the input images to patches of size \(256 \times 256\) at a step size of 64. This totally generates 20128 training samples. To augment training samples, we randomly rotate and flip the training images. The training adopts Adam optimizer. The learning rate is initialized to \(10^{-4}\) and is reduced to \(10^{-5}\) after 20 epochs. It is observed that 40 epochs are sufficient for the training to converge.    

\subsection{Numerical Evaluation}
We numerically measure the performance of our method on the 15 test sets of \cite{Kalantari2017DeepHD}, by Peak Signal-to-Noise Ratio (PSNR) and Structure Similarity, computed in both tonemapping domain (-\(\mu\)) and HDR linear domain (-L). Visual difference metric HDR-VDP-2 is also adopted, where the parameters are set as same as in previous works \cite{wu2018end} and \cite{niu2021hdrgan}. 

Table \ref{table_metrics} compares our model with state-of-the-art models. For \cite{yan2020nonlocal} and \cite{xiong2021hierarchical}, we use the results reported in the publications. Note that \cite{sen2012robust} and \cite{hu2013hdr} are not machine learning based methods. Moreover,  \cite{Kalantari2017DeepHD} and \cite{wu2018end} apply optical flow and homography transformation to preprocess the input images respectively, and hence entail extra computation. 

Table \ref{table_metrics} shows that our method outperforms competing method in terms of PSNR-L, SSIM-$\mu$, SSIM-L and HDR-VDP-2. It ranks the second best in PSNR-$\mu$, being slightly (0.1dB) inferior to \cite{xiong2021hierarchical}. Note that \cite{xiong2021hierarchical} utilizes a pretrained model to detect ghosting regions for training, whereas our method does not require any pretrained model. The high PSNR and SSIM scores varify that our model has strong HDR reconstruction ability and can accurately restore the radiance and structure of the scene in both tonemapping domain and HDR linear domain. Furthermore, its high performance in term of HDR-VDP-2\cite{mantiuk2011hdr} performance indicates that our method can generate HDR image visually close to the target image.

\begin{table*}[ht]
\centering
\begin{tabular}{l|c|c|c|c|c}
\hline
& PSNR-$\mu$ & PSNR-L & SSIM-$\mu$ & SSIM-L & HDR-VDP-2 \\
\hline
\bfseries Sen & 40.97 & 38.36 & 0.9830 & 0.9746 & 60.60\\
\hline
\bfseries Hu  & 35.65 & 30.80 & 0.9725 & 0.9491 & 58.34\\
\hline
\bfseries Kalantari & 42.69 & 41.22 & 0.9888 & 0.9845 & 65.05\\
\hline
\bfseries DeepHDR& 41.99 & 41.22 & 0.9878 & 0.9859 & \underline{65.91}\\
\hline
\bfseries AHDR & 43.62 & 41.03 & 0.9900  &\underline{0.9883} & 63.85 \\
\hline 
\bfseries NHDRRNet& 42.414 & - & 0.9887 & - & 61.21 \\
\hline 
\bfseries HDR-GAN &43.92 & \underline{41.57} &\underline{0.9905} &0.9865 & 65.45\\
\hline 
\bfseries HFNet & \textbf{44.28} & 41.47 & - & - & - \\
\hline 
\bfseries Ours & \underline{44.18} & \textbf{42.19}&\textbf{0.9912} & \textbf{0.9883}& \textbf{67.07} \\
\hline
\end{tabular}
\caption{Numerical performance of the proposed model, evaluated on the dataset by Kalantari-Ramamoorthi. The best and second best results for each metric are marked in \textbf{bold} and \underline{underlined}, respectively}
\label{table_metrics}
\end{table*}

\subsection{Visual Performance Evaluation}

\begin{figure*}[!htb]
\centering
\includegraphics[width=\textwidth]{experiments/kalantari_test.png}
\caption{Visual comparison on the test set of Kalantari-Ramamoorthi dataset. Zoom-in views of reconstruction by each method are presented on the saturated regions that contain moving objects. Our network built with gated Swin Transformer yields noticeably better visual results than other methods.}
\label{fig_kalantari_test}
\end{figure*}
Fig. \ref{fig_kalantari_test} present the visual performance of our method and comparable methods on two examples from \cite{Kalantari2017DeepHD}. We present the zoom-in views of two challenging cases, where large saturated regions contain substantial non-rigid motion in the reference image. The two patch-based methods do not reconstruct the missing details in the saturated regions, as they heavily rely on the details provided by the reference image for registration. Image reconstructed by the optical flow based method \cite{Kalantari2017DeepHD} suffers motion blur artifacts. This is because the convolutions of DeepHDR and HDR-GAN have limited receptive fields, and hence are hampered to repair missing content in misaligned regions by aligned regions. The gating mechanism of AHDR is only applied to low-level features, so the high-level outliers may deteriorate the HDR fusion. In contrast to comparable methods, our model remarkably overcomes the ghosting artifacts.

\begin{figure}[ht]
\centering
\includegraphics[width=\columnwidth]{experiments/sen_test.pdf}
\caption{Visual performance comparison on example images from the dataset by Sen et al. Zoom in views on challenging areas are presented. Although the ground truth is unavailable, it can be clearly observed that our method visually performs better than comparable methods.}
\label{sen_test}
\end{figure}

\begin{figure}[ht]
\centering
\includegraphics[width=\columnwidth]{experiments/tursun_test.pdf}
\caption{Visual performance comparison on example images from the dataset by Tursun et al. Compared to state of the art methods, our method suffers less ghosting artifact.}
\label{tursun_test}
\end{figure}

Fig.\ref{sen_test} and Fig.\ref{tursun_test} present visual performance of our method on two examples from benchmark datasets \cite{sen2012robust} and \cite{tursun2016objective}. As these test datasets   do not provide ground truth image. we mark the visual difference on the results generated by different methods. It can be seen that our method suffers less artifacts than other methods in various scenes with various motion patterns, achieving better visual results. Our method creates high-quality HDR more robustly and generalizes well. 

\subsection{Ablation Study}

\begin{table}[h]
\centering
\resizebox{\columnwidth}{!}{
\begin{tabular}{l|c|c|c|c|c}
\hline
                         & PSNR-$\mu$ & PSNR-l & SSIM-$\mu$ & SSIM-l & HDR-VDP-2 \\ \hline
restormer(w/o ssim loss) & 44.00  & 41.5   & 0.9906 & 0.9873 & 64.72  \\ \hline
Ours(w/o ssim loss)      & 44.07  & 41.83  & 0.9909 & 0.9879 &  64.78  \\ \hline
Ours                     & 44.18  & 42.19  & 0.9912 & 0.9883 & 67.07      \\ \hline
\end{tabular}
}
\caption{Experimental results of ablation study. We compare using Gated Swin Transformer v.s. Gated Transformer, and the combined loss function v.s. the traditional $l_{1}$ norm loss function.}
\label{table_ablation_block_loss}
\end{table}

We verify various components of our method, including Swin Transformer, loss function, and gating mechanism by ablation study.

\subsubsection{Ablation Study on Block Design}
Our model has similar architecture to Restormer, which uses modified Transformer, whereas we use modified Swin Transformer as the building unit. For comparison, we replace the residual modules in each block in our model with multiple transformer layers as in Restormer, with same number of transformer layers. Table \ref{table_ablation_block_loss} presents the results, which show that using Swin Transformer achieves superior performance in all measures. The reason is that the attention module of Restormer is computed channel-wise, but forgoes the cross-exposure spatial dependency to repair the non-aligned area. 

\subsubsection{Ablation Study on Loss Function}
We trained our model under different loss function configurations, as shown in \ref{table_ablation_block_loss}. The results validate that the SSIM loss benefits detail reconstruction.

\subsubsection{Ablation Study on Gating Mechanism}
\begin{table}[h]
\resizebox{\columnwidth}{!}{
\begin{tabular}{l|c|c|c|c|c}
\hline
           & PSNR-$\mu$ & PSNR-l & SSIM-$\mu$ & SSIM-l & HDR-VDP-2 \\ \hline
w/o gating & 43.14  & 41.03  & 0.9904 & 0.9868 &     64.88      \\ \hline
one gating & 43.44  & 41.42  & 0.9909 & 0.9882 &    67.13   \\ \hline
Ours       & 43.61  & 41.74  & 0.9909 & 0.9881 & 66.96     \\ \hline
\end{tabular}
}
\caption{Ablation experimental results to verify the effectiveness of the gating mechanism}
\label{table_ablation_gating}
\end{table}

The gating mechanism is an important component in our model. Ablation study is conducted in the gating mechanism as follows.

\textbf{w/o gating}: The gating mechanism is not used in the feed forward network of all transformer layers in the model, that it, our GST unit degenerate to the vanilla Swin Transformer.

\textbf{one gating}: The gating mechanism is only used in the first Swin Transformer layers subsequent to the embedding layer, but not used for other layers. 

 Table \ref{table_ablation_gating} shows the results of the ablation experiments, where the model is trained for 20 epochs. By removing the gating mechanism, the network relies on self-attention for image alignment, resulting in the lowest performance. On top of it, adding gates to low level layers notably improves the HDR reconstruction. Furthermore, by integrating the gating mechanism with all Swin Transformer layers, the model effectively inpaints information in non-aligned regions and obtains the highest HDR reconstruction results, thus validates the effectiveness of the gating mechanism in our model.


%\section{}
%\label{sec:resDir}


\section{Conclusion}
\label{sec:conclusion}
% <>
Since its advent in 1931, Koopman operator theory \cite{koopman:1931} has only recently been actively utilized for solving practical problems, thanks to the introduction of the DMD algorithm in 2008 \cite{schmid:2008}. Since then, a multitude of DMD algorithm variations have risen to prominence and found utility across various fields. A notable feature of our survey paper was reviewing and categorizing the results of over 100 research papers based on both application and algorithm type in smart mobility and vehicle engineering  (see Table~\ref{tab1} and Section~\ref{sec:vehicApp}).  Additionally, this survey paper identified potential research gaps in smart mobility and vehicular engineering applications (Remarks~\ref{remGap1}--\ref{remGap6}). Finally, this review paper discussed theoretical aspects of Koopman operator theory that have been largely neglected by the smart mobility and vehicle engineering community and yet have large potential for contributing to solving open problems in these areas (see Section~\ref{subsec:theorIssue}).

\noindent{\textbf{Future Research Directions.}}	Given the emergence of cyber-threats against connected and autonomous vehicles as well as robotic systems (see, e.g.,~\cite{nekouei2021randomized,mohammadi2022generation}), a future research direction might include utilizing Koopman operator-based algorithms for designing cyber-resilient vehicular and smart mobility applications (see, e.g.,~\cite{taheri2022data} for a related line of research). Another potential research direction is using Koopman operator-based algorithms for predicting the motion of vulnerable road users (VRUs), e.g., pedestrians and cyclists (see, e.g.,~\cite{pool2019context,scholler2020constant}). Finally, rehabilitation robotics and robotic exoskeletons can be the benefactors of the predictive capabilities of Koopman operator-based algorithms for detecting tripping events and/or system  identification in various modes of locomotion (see, e.g.,~\cite{kumar2019extremum,aprigliano2019pre}).



%Fig. 1 depicts the accumulation of such algorithms since 2014, which are particular to vehicle engineering and smart mobility, i.e., the focus of this review. Table 1 summarizes the varieties of relevant algorithms developed in those studies. Furthermore, we have highlighted theoretical issues, whose expansion will have potential applications to the wide research area of smart mobility and vehicle engineering.  

%Although fairly comprehensive, we have found several gaps in this research area. In particular, we could not find any studies related to elevators, robots/vehicles employing crawling, slithering, hopping or peristaltic locomotion, arctic or special-terrain vehicles such as those employing screws or tracks, hovercraft and other amphibious vehicles or subsystems which tolerate flexible environments, classification or guidance systems related to vehicles for drilling or agriculture, or for current-ripple, power-split, battery health monitoring, nuclear propulsion, exoskeletons/prosthetics, personal mobility, motorsports, specialized rovers or similar open problems in emerging areas.  These examples are, of course, not exhaustive.  
%
%The purely data-driven nature of Koopman operators holds the promise of capturing unknown and complex dynamics for reduced-order model generation and system identification, through which the rich machinery of linear control techniques can be utilized. The emergent nature of the smart mobility and vehicular-related applications, where  the Koopman operator  in each particular application needs to be approximated, implies that the development of various Koopman operator approximation  algorithms is expected to grow along with the vehicular problems they aim to solve.  Given the ongoing development of this research area and the many existing open problems in the fields of smart mobility and vehicle engineering, a survey of techniques and open challenges of applying Koopman operator theory to this vibrant area is warranted.  To the best of our knowledge, this survey paper is the \emph{first of its kind} reviewing the applications of Koopman operator theory within a focused research area, namely, smart mobility and vehicle engineering applications. A \emph{notable feature} of our survey paper is reviewing and categorizing the results of over 100 research papers based on both application and algorithm type  (see Tables~\ref{tab1}--~\ref{tab4} and Section~\ref{sec:vehicApp}) that are concerned with the applications of Koopman operator theory to the field of smart mobility and vehicular engineering. Such a \emph{comprehensive and  detailed categorization} will be beneficial to the research practitioners working in the field.  Furthermore, this review paper discusses theoretical aspects of Koopman operator theory that have been largely neglected by the smart mobility and vehicle engineering community and yet have large potential for contributing to solving open problems in these areas. Additionally, our survey paper seeks to \emph{identify gaps} in the smart mobility and vehicle engineering research where new and existing Koopman operator-based methods have the potential to further develop and address unsolved problems  potentially benefiting from the perspectives of nonlinear system identification, control, global linearization, and the predictive powers that Koopman operator theory has to offer (see, e.g., Remarks~\ref{remGap1}--\ref{remGap6}). 

% \acks{Zebang Shen's work is supported by ETH research grant and Swiss National Science Foundation (SNSF) Project Funding No. 200021-207343.}

% Acknowledgments---Will not appear in anonymized version
% \acks{We thank a bunch of people and funding agency.}
\begin{ack}
Zhenfu Wang is supported by the National Key R\&D Program of China, Project Number 2021YFA1002800, NSFC grant No.12171009, Young Elite Scientist Sponsorship Program by China Association for Science and Technology (CAST) No. YESS20200028 and the Fundamental Research Funds for the Central Universities (the start-up fund), Peking University.
Zebang Shen's work is supported by ETH research grant and Swiss National Science Foundation (SNSF) Project Funding No. 200021-207343.
\end{ack}
\bibliographystyle{abbrvnat}  
\bibliography{MVE}

\clearpage
\appendix
\section{Classical Methods for Solving MVEs} \label{appendix_related_work}
Solving partial differential equations (PDEs) is a key aspect of scientific research, with a wealth of literature in the field {\citep{evans2022partial}}. 
For the interest of this paper, we will only consider the methods that can be used to solve the MVE under consideration.

\paragraph{Categorize PDE solvers via solution representation.} To better understand the benefits of neural network (NN) based PDE solvers and to compare our approach with others, we categorize the literature based on the representation of the solution to the PDE. These representations can be roughly grouped into four categories:
\begin{itemize}[leftmargin=*]
    \item \textbf{1. Discretization-based representation:} The solution to the PDE is represented as discrete function values at grid points, finite-size cells, or finite-element meshes.
    \item \textbf{2. Representation as a combination of basis functions:} The solution to the PDE is approximated as a sum of basis functions, e.g. Fourier series, Legendre polynomials, or Chebyshev polynomials.
    \item \textbf{3. Representation using a collection of particles:} The solution to the PDE is represented as a collection of particles, each described by its weight, position, and other relevant information. 
    \item \textbf{4. NN-based representation:} NNs offer many strategies for representing the solution to the PDE, such as using the NN directly to represent the solution, using normalizing flow or GAN-based parameterization to ensure the non-negativity and conservation of mass of the solution or using the NN to parameterize the underlying dynamics of the PDE, such as the time-varying velocity field that drives the evolution of the system.
\end{itemize}


The drawback of the first three strategies is that a sparse representation\footnote{For example, sparser grid, cell or mesh with less granularity, fewer basis functions, fewer particles.} leads to reduced solution accuracy, while a dense representation results in increased computational and memory cost. 
NNs, as powerful function approximation tools, are expected to surpass these strategies by being able to handle higher-dimensional, less regular, and more complicated systems \citep{weinan2021algorithms}.

Given a representation strategy of the solution, an effective solver must exploit the underlying properties of the system to find the best candidate solution. Three-= notable properties that are utilized to design solvers are 
\begin{enumerate}[leftmargin=*,label=(\Alph*)]
    \item the PDE definition or weak formulation of the system,
    \item the SDE interpretation of the system,
    \item the variational interpretation, particularly the Wasserstein gradient flow interpretation.
\end{enumerate}

These properties are combined with the solution representations mentioned earlier to form different methods.
For example, the Finite Difference method \citep{smith1985numerical}, Finite Volume method \citep{moukalled2016finite}, and Finite Element method \citep{johnson2012numerical} represent the solution of partial differential equations (PDEs) by discretizing the solution and utilize the property (A), at least in their original form. On the other hand, recent work by \citet{carrillo2022primal} solves PDEs admitting a Wasserstein gradient flow structure using the classic JKO scheme \citep{jordan1998variational}, which is based on the property (C), and the solution is also represented via discretization. The Spectral method \citep{shen2011spectral} is a class of methods that exploits property (A) by representing the solution as a combination of basis functions.
The Random Vortex Method \citep{long1988convergence} is a highly successful method for solving the vorticity formulation of the 2D Navier-Stokes equation by exploiting property (C) and representing the solution with particles. The Blob method from \citet{carrillo2019blob} is another particle-based method for solving PDEs that describe diffusion processes, which also exploits property (C).
% In the following, we will focus on the methods that use NN for solution representation.


% Another line of research is to solve PDEs that admit the Wasserstein gradient flow interpretation using the minimizing movement scheme, also referred to as the JKO scheme \citep{jordan1998variational}. \citet{mokrov2021large} and \citet{fan2022variational} focus on the Fokker-Planck equation (FPE), a specific instance of MVE without the interaction term $K\equiv 0$, and learn a sequence of transport maps parameterized by NN. Each map is trained to solve an iteration of the JKO scheme. These learned maps can be used to reconstruct the solution to the Fokker-Planck equation. 
% While the MVE with Coulomb interaction also admits a Wasserstein gradient flow interpretation, it is unclear how methods along this line can be generalized to MVE. We plan to investigate this in the future.
\section{Comparison with Neural Operator}
We thank the anonymous reviewers for pointing out the interesting research direction of neural operators \citep{xiao2023coupled,gupta2021multiwavelet,li2020neural,kovachki2021neural,li2020fourier}.
However, to highlight the major difference between EINN and the approach of the Neural Operator, it's worth noting that they consider completely different problem settings: EINN requires \emph{no pre-existing data} and the goal is to obtain the solution to a PDE by solely exploiting the structure of the equation itself. In contrast, the neural operator approach is \emph{data-driven}, i.e. it relies on the existence of configuration-solution pairs. Here, by configuration-solution pairs, we mean the correspondence between some configurations that determine the PDE, e.g. the initial condition or the viscosity parameter in the fluid dynamics problems, and the pre-existing solution to the PDE given the aforementioned configurations. Consequently, the neural operator approach is more like a regression problem where a neural network is trained to learn the abstract map between the configuration and the solution. In contrast, EINN is more like a numerical PDE solver.

Consequently, EINN and the approach of neural operator are two related but quite distinct research directions. They are related in the sense that EINN can provide the data (configuration-solution pairs) required by the neural operator approach. They are distinct since EINN requires no data a priori, while the neural operator approach is built on the supervised learning paradigm.
\section{More Details on the Experiments}
\subsection{Implementations of Baselines} \label{appendix_implementation_of_baselines}
\paragraph{Objectives of PINN}
\begin{itemize}
	\item For the vorticity equation of the 2D Navier-Stokes equation, let $\vu: [0, T]\times \sR^2 \rightarrow \sR^2$ be the velocity field (this should not be confused with the velocity field of the continuity equation) such that $\nabla \cdot \vu = 0$, i.e. $\vu$ is divergence-free, and let $\omega = \nabla \times \vu \in \sR$ be the vorticity.
	We have
	\begin{align}
		\frac{\partial \omega}{\partial t} + \nabla \cdot \left(\omega\vu\right) =&\ \nu \Delta \omega,\\
		\omega =&\ \nabla \times \vu.
	\end{align}
	We use this form to construct the objective for the PINN method
	\begin{equation}
		\int_0^T \|\frac{\partial \omega}{\partial t} + \nabla \cdot \left(\omega\vu\right) - \nu \Delta \omega\|_{\gL(\Omega)^2}^2 + \|\omega - \nabla \times \vu\|_{\gL\gL(\Omega)^2} d t,
	\end{equation}
	where $\gL^2(\Omega)$ denotes the functional $\gL^2$ norm on the domain $\Omega = [-2, 2]^2$.
	\item For the MVE with Coulomb interaction, let $g$ be the Coulomb potential defined in \eqref{eqn_coulomb_interaction}. We have that $\psi = g \ast \rho$ is the solution to the Poisson equation $\Delta \psi = - \rho$ and $K * \rho = - \nabla \psi$.
	We have
	\begin{align}
		\frac{\partial \rho}{\partial t} + \nabla \cdot \left(\rho\cdot (-\nabla \psi) \right) =&\ \nu \Delta \rho \\
		\Delta \psi =&\ -\rho.
	\end{align}
	Expand the the divergence to obtain
	\begin{align}
		\frac{\partial \rho}{\partial t} + \nabla \rho \cdot (-\nabla \psi) +  \rho\cdot (-\Delta \psi) =&\ \nu \Delta \rho \\
		\Delta \psi =&\ -\rho.
	\end{align}
	Now plug in the $\Delta \psi = -\rho$ to arrive at the following equivalent form
	\begin{align}
		\frac{\partial \rho}{\partial t} + \nabla \rho \cdot -\nabla \psi +  \rho^2 =&\ \nu \Delta \rho \\
		\Delta \psi =&\ -\rho.
	\end{align}
	We use this form to construct the objective for the PINN method.
	\begin{equation}
		\int_0^T \|\frac{\partial \rho}{\partial t} + \nabla \rho \cdot -\nabla \psi +  \rho^2 - \nu \Delta \rho\|_{\gL^2(\Omega)}^2 + \|\Delta \psi + \rho\|_{\gL^2}^2,
	\end{equation}
	where $\gL^2(\Omega)$ denotes the functional $\gL^2$ norm on the domain $\Omega = [-1, 1]^2$.
\end{itemize}

\paragraph{DRVN} In the original paper \citep{zhang2022drvn}, only the Biot-Savart kernel is concerned. We can easily extend the DRVN method to the Coulomb case by setting $K$ to be the kernel defined in \eqref{eqn_coulomb_interaction}.

\subsection{Examples with an Explicit Solution} \label{appendix_explicit_solution}
In this section, we verify the explicit solutions discussed in the experiment section.
\paragraph{Lamb-Oseen Vortex on the whole domain $\sR^2$.}
Recall that we consider the 2D Navier-Stokes equation (the MVE with the Biot-Savart interaction kernel (\ref{eqn_nse})).
The Lamb-Oseen Vortex model states that, if $\rho_0 = \mathcal{N}(0, \sqrt{2\nu t_0}\mI_2)$ for some $t_0 \geq 0$, then we have $\rho_t(\vx) = \mathcal{N}(0, \sqrt{2\nu (t+t_0)}\mI_2)$.

To verify this, define $\vu_t(\vx) = \frac{1}{\sqrt{\nu (t+t_0)}}\vv(\frac{\vx}{\sqrt{\nu (t+t_0)}})$, where 
\begin{equation}
	\vv(\vx) = \frac{1}{2\pi}\frac{\vx^{\perp}}{\|\vx\|^2}\left(1 - \exp(-\frac{1}{4}\|\vx\|^2)\right).
\end{equation}
One can easily check that $\nabla \cdot \vu_t \equiv 0$ and hence there exists a function $\psi_t$ such that $\nabla^\perp \psi_t = -\vu_t$, where $\nabla^\perp$ denotes the perpendicular gradient, defined as $\nabla^\perp = (-\partial_{\vx_2}, \partial_{\vx_2})$, and $\psi_t$ is called the stream function in the literature of fluid dynamics.
Moreover, one can verify that $\nabla \times \vu_t = \rho_t$ where $\nabla \times$ denotes the curl of a 2D velocity field, defined as $\nabla \times \vu(\vx) = \partial \vu_2 / \partial \vx_1 - \partial \vu_1/\partial \vx_2$.
Together we have
\begin{equation}
	\Delta \psi_t = -\rho_t,
\end{equation}
i.e., the stream function $\psi_t$ is the solution to the 2D Poisson equation with a source term $\rho_t$.

Under the boundary condition that $\psi_t(\vx) \rightarrow 0$ for $\|\vx\|\rightarrow \infty$, we can express $\psi_t$ via the unique Green function $G(\vx) = \frac{1}{2\pi}\ln \|\vx\|$ as 
\begin{equation}
	\psi_t(\vx) = G \ast \rho_t = \frac{1}{2\pi}\int \ln\|\vx - \vy\| \rho_t(\vy) d \vy.
\end{equation}
Consequently, by taking the perpendicular gradient, we obtain
\begin{equation}
	\vu_t = \nabla^\perp \psi_t = \frac{1}{2\pi}\int \frac{(\vx - \vy)^\perp}{\|\vx - \vy\|^2} \rho_t(\vy) d \vy = K \ast \rho_t.
\end{equation}
Finally, by plugging the expressions of $\rho_t$ and $\vu_t = K \ast \rho_t$ in the MVE (\ref{eqn_MVE}), we verified the Lamb-Oseen vortex.
%\paragraph{Taylor-Green Vortex on the box $[0, 2\pi]^2$ with a periodic boundary condition.}
%Recall that we consider the 2D Navier-Stokes equation (the MVE with the scaled periodic Biot-Savart interaction kernel (\ref{eqn_K_periodic_biot_savart})) on the box $[0, 2\pi]^2$ with a periodic boundary condition.
%The Taylor-Green Vortex model stats that if $\rho_0(\vx) = \frac{1}{4\pi^2}\left(\cos \vx_1 \cos \vx_2 \exp(-2\nu t_0) + 1\right)$ for some $t \geq 0$, then we have 
%\begin{equation}
%	\rho_t(\vx) = \frac{1}{4\pi^2}\left(\cos \vx_1 \cos \vx_2 \exp(-2\nu (t+t_0)) + 1\right).
%\end{equation}
%
%First, one can easily check that $\rho_t(\vx)$ is a density function on $[0, 2\pi]^2$ for all $t\in[0, T]$, i.e. it is non-negative and integrates to 1. To verify this vortex model, define $\vu = (\vu_1, \vu_2)$ as follows
%\begin{equation}
%	\vu_1(\vx, t) = \cos \vx_1 \sin \vx_2 \exp(-2\nu t), \vu_2(\vx, t) = -\sin \vx_1 \cos \vx_2 \exp(-2\nu t).
%\end{equation}
%One can easily check that $\nabla\cdot\vu(\vx, t) \equiv 0$, which implies the exists of some stream function $\psi_t$ such that $\nabla^\perp \psi_t(\vx) = -\vu(\vx, t)$. Moreover, one can check that 

\paragraph{Barenblatt solutions for the MVE with Coulomb Interaction.} 
Recall that we consider the MVE with the Coulomb interaction kernel (\ref{eqn_coulomb_interaction}) for $d=3$ and set the diffusion coefficient $\nu = 0$, i.e.
\begin{equation}
	\frac{\partial \rho_t}{\partial t} + \nabla \cdot \left( \rho_t \cdot -\nabla \psi_t \right) = 0
\end{equation}
where $\psi_t$ is the solution to the Poisson equation $\Delta \psi_t = -\rho_t$. The Barenblatt solution of the above MVE is stated as follows: If $\rho_0 = \Uniform[\|\vx\|\leq (\frac{3}{4\pi}t_0)^{1/3}]$ for some $t_0 \geq 0$, then we have 
\begin{equation}
	\rho_t = \Uniform[\|\vx\|\leq (\frac{3}{4\pi}(t+t_0))^{1/3}]
\end{equation}
We now verify this solution.


Recall that the volume of a three dimensional Euclidean ball with radius $R$ is $\frac{4\pi}{3}R^3$. Hence we can write the density function as $\rho_t(x) = \frac{1}{t+t_0} \chi_{\|\vx\|\leq (\frac{3}{4\pi}(t+t_0))^{1/3}}$, where $\chi_\sX$ is a function that takes value $1$ for $\vx \in \sX$ and takes value $0$ for $\vx \notin \sX$.
Take 
\begin{equation}
	\psi_t(\vx) = \begin{cases}
		\frac{2(\frac{3}{4\pi}(t+t_0))^{2/3}-\|\vx\|^2}{6(t+t_0)}, & \|\vx\| \leq (\frac{3}{4\pi}(t+t_0))^{1/3},\\
		\frac{1}{8\pi\|\vx\|}, & \|\vx\| > (\frac{3}{4\pi}(t+t_0))^{1/3}.
	\end{cases}
\end{equation}
It can be verified that the Poisson equation $\Delta \psi_t = -\rho_t$ holds (note that $\Delta \|\vx\|^{-1} = 0$, i.e. $\|\vx\|^{-1}$ is a harmonic function for $d=3$).
Consequently, for a fixed time stamp $t$ and any $\|\vx\| \leq (\frac{3}{4\pi}(t+t_0))^{1/3}$ we have 
\begin{align}
	\frac{\partial \rho_t}{\partial t}(\vx) + \nabla \cdot \left( \rho_t(\vx) \cdot -\nabla \phi_t(\vx) \right) = - \frac{1}{(t+t_0)^2} + \frac{1}{(t+t_0)^2} = 0,
\end{align}
which verifies this solution.

\clearpage
\section{Adjoint Method} \label{appendix_adjoint_method}
Consider the ODE system
\begin{align*}
	\dot s(t) =&\ \psi(s(t), t, \theta) \\
	s(0) =&\ s_0,
\end{align*}
and the objective loss
\begin{equation}
	\ell(\theta) = \int_0^T g(s(t), t, \theta) \ud t.
\end{equation}
The following proposition computes the gradient of $\ell$ w.r.t. $\theta$.
We omit the parameters of the functions for succinctness. We note that all the functions in the integrands should be evaluated at the corresponding time stamp $t$, e.g. $b^\top \frac{\partial h}{\partial \theta}\ud t$ abbreviates for $b(t)^\top \frac{\partial}{\partial \theta}h(\xi(t), x(t), t, \theta)\ud t$.
\begin{proposition}
	\begin{equation}
		\frac{\ud \ell}{\ud \theta} = \int_{0}^T a^\top \frac{\partial\psi}{\partial\theta} + \frac{\partial g}{\partial\theta}\ud t.
	\end{equation}
	where $a(t)$ is solution to the following final value problems
	\begin{equation}
		\dot a^\top + a^\top \frac{\partial\psi}{\partial s} + \frac{\partial g}{\partial s} = 0, a(T) = 0, 
	\end{equation}
\end{proposition}
\begin{proof}
	Let us define the Lagrange multiplier function (or the adjoint state) $a(t)$ dual to $s(t)$.
	Moreover, let $L$ be an augmented loss function of the form
	\begin{equation}
		L = \ell - \int_0^T a^\top(\dot s - \psi) \ud t.
	\end{equation}
	Since we have $\dot s(t) = \psi(s(t), t, \theta)$ by construction, the integral term in $L$ is always null and $a$ can be freely assigned while maintaining $\ud L/\ud \theta = \ud \ell/\ud \theta$.
	Using integral by part, we have
	\begin{equation}
		\int_0^T a^\top\dot s\ \ud t = a(t)^\top s(t)\vert_0^T - \int_0^T s^\top \dot a\ \ud t.
	\end{equation}
	We obtain
	\begin{align}
		L = - a(t)^\top s(t)\vert_0^T + \int_0^T \dot a^\top s + a^\top \psi + g\ \ud t.
	\end{align}
	
	Now we compute the gradient of $L$ w.r.t. $\theta$ as
	\begin{equation*}
		\frac{\ud \ell}{\ud \theta} =  \frac{\ud L}{\ud \theta} = - a(T)^\top\frac{\ud x(T)}{\ud \theta}  + \int_0^T \dot a^\top \frac{\ud s}{\ud \theta} + a^\top \left(\frac{\partial\psi}{\partial\theta} + \frac{\partial\psi}{\partial s} \frac{\ud s}{\ud \theta} \right) \ud t
		+ \int_0^T \frac{\partial g}{\partial s} \frac{\ud s}{\ud \theta} +  \frac{\partial g}{\partial\theta}\ud t,
	\end{equation*}
	which by rearranging terms yields to
	\begin{align*}
		\frac{\ud \ell}{\ud \theta} = \frac{\ud L}{\ud \theta} = - a(T)^\top\frac{\ud x(T)}{\ud \theta} + \int_{0}^T a^\top \frac{\partial \psi}{\partial \theta} +  \frac{\partial g}{\partial \theta}\ud t
		+ \int_0^T \left(\dot a^\top + a^\top \frac{\partial \psi}{\partial s} +  \frac{\partial g}{\partial s}\right)\frac{\ud s}{\ud \theta} \ud t.
	\end{align*}
	Now by taking $a$ satisfying the \emph{final} value problems
	\begin{equation}
		\dot a^\top + a^\top \frac{\partial \psi}{\partial s} + \frac{\partial g}{\partial s} = 0, a(T) = 0, 
	\end{equation}
	we derive the result
	\begin{equation}
		\frac{\ud \ell}{\ud \theta} = \int_{0}^T a^\top \frac{\partial \psi}{\partial \theta} + \frac{\partial g}{\partial \theta}\ud t.
	\end{equation}
\end{proof}
\clearpage
\section{Detailed Proofs}\label{detailed proof}



%%%%%%%%%%%
\begin{proof}[Proof of Lemma  \ref{TimeEvolKLMV}] 
	Recall the McKean-Vlasov  \eqref{eqn_MVE_CE} and the continuity  \eqref{eqn_CE_as_MVE}. We simply write that $\rho_t = \rho_t^f$ and omit the integration domain $\X$.  Then 
	\[
	\begin{split}
		& \frac{\ud }{\ud t } \int \rho_t \log \frac{\rho_t }{\bar \rho_t }  = \int \partial_t \rho_t \log \frac{\rho_t}{\bar \rho_t} +  \int \rho_t   \partial_t \log \rho_t - \int \rho_t \partial_t \log \bar \rho_t  \\
		& = - \int  \udiv \Big(    \rho_t  \Big(\Big[- \nabla V (x) + K *  \rho_t -  \nu \nabla \log  \rho_t \Big] + \delta_t  \Big) \Big) \log \frac{\rho_t}{\bar \rho_t} \\
		& +  \int \frac{\rho_t}{\bar \rho_t}  \udiv \Big(   \bar \rho_t \Big(- \nabla V (x) + K * \bar \rho_t - \nu \nabla \log \bar \rho_t \Big)\Big), 
	\end{split}
	\]
	where we note that $\int \rho_t \partial_t \log  \rho_t= \int \partial_t \rho_t = 0$ since the total mass is preserved over time. 
	By integration by parts, one has 
	\[
	\begin{split}
		& \frac{\ud }{\ud t } \int \rho_t \log \frac{\rho_t }{\bar \rho_t }  = I_1 + I_2 + I_3+ \int \rho_t \delta_t \cdot \nabla \log \frac{\rho_t}{\bar \rho_t},
	\end{split}
	\]
	where $I_1, I_2, I_3$ denote the linear, nonlinear interaction, and diffusion parts separately. More precisely, by integration by parts,
	\[
	\begin{split}
		I_1 & = \int \udiv (\rho_t \nabla V(x)) \log \frac{\rho_t}{\bar \rho_t} - \int \frac{\rho_t}{\bar \rho_t} \udiv (\bar \rho_t \nabla V(x))  \\
		& = - \int \rho_t \nabla V(x) \cdot \nabla \log \frac{\rho_t}{\bar \rho_t} + \int \bar \rho_t  \nabla  \frac{\rho_t }{\bar \rho_t} \cdot  \nabla V(x) = 0. 
	\end{split}
	\]
	And 
	\[
	\begin{split} 
		I_2 & = - \int \udiv (\rho_t K * \rho_t ) \log \frac{\rho_t}{\bar \rho_t } + \int \frac{\rho_t }{\bar \rho_t} \udiv(\bar \rho_t K * \bar \rho_t)  \\
		& =  \int \rho_t K * \rho_t \nabla \log \frac{\rho_t}{\bar \rho_t } - \int \bar \rho_t K * \bar \rho_t \cdot \nabla \frac{\rho_t}{\bar \rho_t } \\
		& = \int \rho_t \nabla \log \frac{\rho_t}{\bar \rho_t} \cdot K * (\rho_t - \bar \rho_t). 
	\end{split} 
	\]
	Given that the kernel $K$ is divergence free, that is $\udiv K = 0$, one further has 
	\begin{equation}\label{NSE_new}
	\begin{split}
		I_2 & = - \int \rho_t \nabla \log \bar \rho_t  \cdot K *(\rho_t - \bar \rho_t ) + \int \nabla \rho_t \cdot K *(\rho_t - \bar \rho_t ) \\
		& = - \int \rho_t \nabla \log \bar \rho_t \cdot K * (\rho_t - \bar \rho_t). 
	\end{split}
	\end{equation}
	Note that this modification will be used in the proof in the 2D Navier-Stokes case. 
	Finally, all diffusion terms sum up to $I_3$ which can be further simplified as 
	\[
	\begin{split} 
		I_3  & = \nu \int \udiv (\rho_t \nabla \log  \rho_t ) \log \frac{\rho_t }{\bar \rho_t} - \nu \int \frac{\rho_t }{\bar \rho_t} \udiv (\bar \rho_t \nabla \log \bar \rho_t ) \\
		& = - \nu \int \rho_t \nabla \log \rho_t \cdot \nabla \log \frac{\rho_t}{\bar \rho_t} + \nu \int \bar \rho_t \nabla \log \bar  \rho_t \cdot  \nabla \frac{\rho_t }{\bar \rho_t } \\
		& = - \nu \int \rho_t |\nabla \log \frac{\rho_t }{\bar \rho_t}|^2. 
	\end{split} 
	\]
	We thus complete the proof of Lemma \ref{TimeEvolKLMV}.
	
	
	
\end{proof}







%%% Proof of Time Evolution of Modulated Energy
\begin{proof}[Proof of Lemma \ref{ModuEnergyEvo}]Recall that $K= - \nabla g$.  For simplicity, we write that $\rho_t = \rho_t^f$. Then 
	\[
	\begin{split} 
		& \frac{\ud }{\ud t } F(\rho_t, \bar \rho_t) = \frac{\ud }{\ud t } \frac{1 }{2} \int_{\mathcal{X}^2} g (x- y) \ud (\rho_t - \bar \rho_t)^{\otimes 2 } (x, y)  \\
		& = \int_\mathcal{X} g *(\rho_t - \bar \rho_t) (x) \big(\partial_t \rho_t (x) - \partial_t \bar \rho_t (x)  \Big) \ud x  \\
		& = \int g * (\rho_t - \bar \rho_t ) (x) \udiv \Big\{  \rho_t \Big( [\nabla V (x) - K * \rho_t + \nu \nabla  \log \rho_t ]  -  \delta_t \Big)   \\ & \quad \qquad \qquad \qquad \qquad - \bar \rho_t \Big( \nabla V(x) - K * \bar \rho_t + \nu \log \bar \rho_t \Big) \Big\} \\
		& = J_1+ J_2+ J_3 + J_4, 
	\end{split} 
	\]
	where $J_1, J_2, J_3, J_4$ denote the perturbation term, the linear difference term, the nonlinear difference term, and the diffusion term respectively. The perturbation term $J_1$ reads
	\[
	J_1 = - \int_{\mathcal{X}} g*(\rho_t - \bar \rho_t ) \udiv (\rho_t \delta_t ) = - \int_{\mathcal{X}}  \rho_t  K * (\rho_t - \bar \rho_t )  \cdot \delta_t. 
	\]
	By integration by parts, the linear difference term can be written as 
	\[
	\begin{split}
		&J_2 = \int_{\mathcal{X}}  g *(\rho_t - \bar \rho_t ) \udiv\Big( (\rho_t - \bar \rho_t ) \nabla V \Big)  = \int_{\mathcal{X}} K * (\rho_t - \bar \rho_t) (\rho_t - \bar \rho_t ) \nabla V \\
		&  = \frac{1}{2} \int_{\mathcal{X}^2} K (x - y)(\nabla V (x) - \nabla  V(y)) \ud (\rho_t - \bar \rho_t )^{\otimes 2}(x, y), 
	\end{split}
	\]
	where the last equality is true since $K = - \nabla g $ 
	is an odd function and we do the symmetrization trick, i.e. exchanging the role of $x$ and $y$ to another term and then taking the average. 
	
	The nonlinear difference term reads 
	\[
	\begin{split}
		& J_3 = - \int_{\mathcal{X}} g*(\rho_t - \bar \rho_t) \udiv\Big(\rho_t K * \rho_t - \bar \rho_t K * \bar \rho_t  \Big)  \\
		& = - \int_{\mathcal{X}} K * (\rho_t - \bar \rho_t ) (\rho_t K * (\rho_t - \bar \rho_t) - \int_{\mathcal{X}}  K * (\rho_t - \bar \rho_t) (\rho_t - \bar \rho_t ) K * \bar \rho_t  \\
		& = - \int_{\mathcal{X}}  \rho_t |K * (\rho_t - \bar \rho_t)|^2 - \frac{1}{2} \int K(x-y) (K * \bar \rho_t (x) - K * \bar \rho_t (y) ) \ud (\rho_t - \bar \rho_t)^{\otimes 2}(x, y), 
	\end{split}
	\]
	where again in the last term we do the symmetrization. 
	
	The diffusion term reads 
	\[
	\begin{split}
		& J_4 =  \nu \int g * (\rho_t - \bar \rho_t ) \udiv \Big( \rho_t  \nabla \log \rho_t - \bar \rho_t \nabla \log \bar \rho_t  \Big)  \\
		& = \nu \int K * (\rho_t - \bar \rho_t)  \rho_t \nabla \log \frac{\rho_t }{\bar \rho_t }   + \nu \int K * (\rho_t - \bar \rho_t ) (\rho_t - \bar \rho_t ) \nabla \log \bar \rho_t  \\
		& = \nu \int_{\mathcal{X}} \rho_t K *(\rho_t - \bar \rho_t)  \cdot \nabla \log \frac{\rho_t}{\bar \rho_t}  \\
		& \qquad + \frac{\nu }{2} \int_{\mathcal{X}^2} K (x-y)(\nabla \log \bar \rho_t (x) - \nabla \log \bar \rho_t (y)) \ud (\rho_t - \bar \rho_t )^{\otimes 2}. 
	\end{split} 
	\]
	To sum it up, we prove the thesis.  
	
	
	
	
\end{proof}

\subsection{Proof of the 2D Navier-Stokes case} 
Now we proceed to control the growth of the KL divergence  $\mathbf{KL}(\rho_t^f \vert \bar \rho_t )$ for the 2D Navier-Stokes case. 
Since the Biot-Savart law is divergence free, by \eqref{NSE_new} in the proof of Lemma \ref{TimeEvolKLMV}, one has 
\begin{equation}\label{KL_NS_Evo}
	\frac{\ud }{\ud t} \int_{\Pi^d} \rho_t \log \frac{\rho_t }{\bar \rho_t} = - \nu \int_{ \Pi^d}\rho_t |\nabla \log \frac{\rho_t}{\bar \rho_t}|^2 - \int_{\Pi^d} \rho_t K * (\rho_t - \bar \rho_t ) \cdot \nabla  \log \bar   \rho_t +  \int_{\Pi^d} \rho_t \delta_t \cdot \nabla \log \frac{\rho_t }{\bar \rho_t }. 
\end{equation}

Recall that we  write the kernel $K= (K_1, \cdots, K_d)$ and its component $K_i = \sum_{j=1}^d \partial_{x_j} U_{ij}(x)$, where $U= (U_{ij})_{1 \leq i, j \leq d}$ is a matrix-valued potential function for instance can be defined as in  \eqref{eqn_K_as_divergence}. 
 Consequently 
\[
- \int \rho_t K * (\rho_t - \bar \rho_t) \cdot \nabla \log \bar \rho_t = - \sum_{i, j=1}^d \int \rho_t \partial_{x_j} U_{ij} * (\rho_t - \bar \rho_t) \partial_{x_i} \log \bar \rho_t, 
\]
which equals to 
\[
\sum_{i, j=1}^d  \int U_{ij}* (\rho_t - \bar \rho_t) \partial_{x_j}\big( \frac{\rho_t}{\bar \rho_t} \partial_{x_i} \bar \rho_t \big) = A + B 
\]
by integration by parts, 
where further 
\[
A= \sum_{i, j=1}^d  \int V_{i  j} * (\rho_t - \bar \rho_t) \, \partial_{x_i} \bar \rho_t \,  \partial_{x_j} \, \frac{\rho_t}{\bar \rho_t} = \int U * (\rho_t - \bar \rho_t) : \nabla \bar \rho_t \otimes \nabla \frac{\rho_t }{\bar \rho_t}, 
\]
and 
\[
B = \sum_{i, j=1}^d \int \rho_t U_{ij} *(\rho_t - \bar \rho_t) \frac{\partial_{x_i x_j}^2 \bar \rho_t }{\bar \rho_t} = \int \rho_t U * (\rho_t - \bar \rho_t) : \frac{\nabla^2 \bar \rho_t}{\bar \rho_t}. 
\]
Noticing that $\nabla \frac{\rho_t}{\bar \rho_t} =\frac{\rho_t}{\bar \rho_t} \nabla \log \frac{\rho_t }{\bar \rho_t }$, one estimates $A$ as follows 
\[
\begin{split}
	A & = \int  \rho_t U * (\rho_t - \bar \rho_t) : \nabla \log \bar \rho_t \otimes \nabla \log  \frac{\rho_t }{\bar \rho_t} \\
	&  \leq \frac{\nu}{4} \int \rho_t |\nabla \log \frac{\rho_t }{\bar \rho_t}|^2 + \frac{1}{\nu} \int \rho_t | (\nabla \log \bar \rho_t)^\top  \, U*(\rho - \bar \rho) |^2 \\
	& \leq \frac{\nu}{4} \int \rho_t |\nabla \log \frac{\rho_t }{\bar \rho_t}|^2 + \frac{1}{\nu } \|U\|_{L^\infty}^2 \|\nabla \log \bar \rho_t\|_{L^\infty}^2  \|\rho_t - \bar \rho_t\|_{L^1}^2,  
\end{split} 
\]
and again by  Csisz\'ar–Kullback–Pinsker inequality, one has that 
\[
A \leq \frac{\nu}{4} \int \rho_t |\nabla \log \frac{\rho_t }{\bar \rho_t}|^2 + \frac{2}{\nu } \|U\|_{L^\infty}^2 \|\nabla \log \bar \rho_t\|_{L^\infty}^2 \int \rho_t \log \frac{\rho_t}{\bar \rho_t}. 
\]

Now it only remains to control $B$. Recall the following famous Gibbs inequality 
\begin{lemma}[Gibbs inequality]\label{Gibbs} For any parameter $\eta > 0$, and  probability measures $\rho, \bar \rho \in \mathcal{P}(\mathcal{X}) \cap L^1(\mathcal{X})$, and $\phi$  a real-valued function defined on $\mathcal{X}$,  one has the following change of reference measure inequality 
	\[
	\int_{\mathcal{X}} \rho(x) \phi(x) \ud x    \leq \frac{1}{\eta } \Big( \int_{\mathcal{X}} \rho(x) \log \frac{\rho(x)}{\bar \rho(x) } \ud x  + \log \int_{\mathcal{X}} \bar \rho(x) \exp (\eta \phi (x))  \ud x \Big). 
	\]
	
\end{lemma} 
The proof of this inequality can be found in section 13.1 in  \citep{erdHos2017dynamical}. 

To control $B$, we write that $\phi= U * (\rho_t - \bar \rho_t) : \frac{\nabla^2 \bar \rho_t}{\bar \rho_t}$ and thus  $B = \int \rho_t  \phi$.  We choose a positive parameter $\eta>0$ such that 
\[
\frac{1}{\eta} = 2 \|U \|_{L^\infty} \Big\| \frac{\nabla^2 \bar \rho_t}{\bar \rho_t} \Big\|_{L^\infty}. 
\]
Now we apply Lemma \ref{Gibbs} to obtain that 
\[
B = \int \rho_t \phi \leq \frac{1}{\eta} \left( \int \rho_t  \log \frac{\rho_t }{\bar \rho_t} + \log \int \bar \rho_t \exp(\eta \phi )\right). 
\]
Note that $\eta >0$ is chosen so small such that  
% {\bf Case I : when $\int  \rho_t \log \frac{\rho_t}{\bar \rho_t} \leq 1$}. Now  
\[
\begin{split}
	\eta \| \phi\|_{L^\infty} &  \leq  \frac{1}{2 \|U \|_{L^\infty} \Big\| \frac{\nabla^2 \bar \rho_t}{\bar \rho_t}\Big\|_{L^\infty}}  \|U\|_{L^\infty}  \|\rho_t - \bar \rho_t\|_{L^1}  \Big\| \frac{\nabla^2 \bar \rho_t}{\bar \rho_t}\Big\|_{L^\infty}   \\
	& \leq \frac 1 2 \|\rho_t - \bar \rho_t \|_{L^1} \leq 1, 
\end{split} 
\]
since for two probability densities it always holds $\|\rho_t - \bar \rho_t \|_{L^1} \leq 2$. Consequently, applying the inequality $\exp(x) \leq 1 + x + \frac e 2 x^2$ for $|x|\leq 1$, we have 
\[
\int \bar \rho_t \exp(\eta \phi ) \leq \int \bar \rho_t \left(  1 + \eta \phi + \frac e 2 \eta^2 \phi^2 \right) \leq 1 + 0 + \frac{e}{2}  \Big( \frac 1 2 \| \rho_t - \bar \rho_t\|_{L^1} \Big)^2  \leq 1 + \frac{e}{4} \mathbf{KL}(\rho_t \vert \bar \rho_t), 
\]
where 
\[
\int \bar \rho_t \phi = \int U* (\rho_t - \bar \rho_t) : \nabla^2 \bar \rho_t =\int  \sum_{i, j=1}^d \partial_{x_i x_j}U * (\rho_t - \bar \rho_t) \bar \rho_t = \int \udiv K *(\rho_t - \bar \rho_t ) \bar \rho_t = 0, 
\]
since $\udiv K =0$. 

To sum it up, in particular since $\log (1 + x ) \leq x $ for $x > 0$, one has 
\[
B \leq \frac{1}{\eta} \Big(  1 + \frac e 4 \Big) \mathbf{KL}(\rho_t \vert \bar \rho_t ) \leq 4  \|U \|_{L^\infty} \Big\| \frac{\nabla^2 \bar \rho_t}{\bar \rho_t} \Big\|_{L^\infty} \mathbf{KL}(\rho_t \vert \bar \rho_t ). 
\]

Combining \eqref{KL_NS_Evo}, the estimates for $A$ and $B$, one has 
\begin{equation}\label{NSEFinal}
	\begin{split}
		& \frac{\ud }{\ud t} \int \rho_t \log \frac{\rho_t}{\bar \rho_t} \leq - \frac{3 \nu  }{4}  \int \rho_t |\nabla \log \frac{\rho_t}{\bar \rho_t}|^2 + M(t) \int \rho_t \log \frac{\rho_t }{\bar \rho_t} + \int \rho_t \delta_t \cdot \nabla \log \frac{\rho_t}{\bar \rho_t} \\
		& \leq - \frac \nu  2 \int \rho_t |\nabla \log \frac{\rho_t}{\bar \rho_t}|^2 +M(t) \int \rho_t \log \frac{\rho_t }{\bar \rho_t} + \frac{1}{\nu } \int \rho_t |\delta_t|^2 
	\end{split}
\end{equation}
where 
\[
M(t) = \frac{2}{\nu } \|U\|_{L^\infty}^2 \|\nabla \log \bar \rho_t\|_{L^\infty}^2 + 4  \|U \|_{L^\infty} \Big\| \frac{\nabla^2 \bar \rho_t}{\bar \rho_t} \Big\|_{L^\infty} = M(t; \nu, U, \bar \rho_t). 
\]
Since the matrix-valued potential function $U$ is bounded ($\|U(\vx)\|_{op}\leq1/4$ when $U$ takse the form (\ref{eqn_K_as_divergence})), and under suitable assumptions for the initial data $\bar \rho_0$ (for instance $\bar \rho_0 \in C^3$ and there exists $c>1$ s.t. $\frac 1 c \leq \bar \rho \leq c $), one can obtain $\sup_{t \in [0, T] } M(t) \leq M < \infty$. We  recall Theorem 2 in  \citep{guillin2021uniform} as below for completeness. 

\begin{theorem} Given the initial data $\bar \rho_0 \in C^\infty (\Pi^d)$, such that there exists $c>1$, $\frac{1}{c} \leq \bar \rho_0 \leq c$. Then the vorticity formulation of the 2D Navier-Stokes equation 
	\[
	\partial_t \bar\rho_t + \udiv (\bar \rho_t K * \bar \rho_t ) = \nu \Delta \bar \rho_t, \quad \bar \rho(0, x) = \bar \rho_0(x), 
	\]
	has a unique bounded solution $\bar \rho(t, x) \in C^\infty([0, \infty) \times \Pi^d)$, and for any $t>0$,  for any $x \in \Pi^d$,  it holds that $\frac{1}{c} \leq \bar \rho(t, x) \leq c$. 
	
\end{theorem}

Finally,  we simplify \eqref{NSEFinal} to obtain that 
\[
\frac{\ud }{\ud t} \int \rho_t \log \frac{\rho_t}{\bar \rho_t} 
\leq  M \int \rho_t \log \frac{\rho_t }{\bar \rho_t} + \frac{1}{\nu } \int \rho_t |\delta_t|^2, 
\]
where $M = \sup_{t \in [0, T] } M(t; \nu, U, \bar \rho_t)< \infty$. By Gronwall inequality, one finally obtains that 
\[
\sup_{t \in [0, T]} \int_{\Pi^d} \rho_t \log \frac{\rho_t}{\bar \rho_t} \ud x \leq \frac{1}{\nu} \exp(M T ) R(\theta). 
\]
%This completes the proof of Theorem \ref{NSMainEstimate} 

As noted in \citep{guillin2021uniform},in particular Corollary 2 there, one can improve the above time-dependent estimate ($\exp(MT)$) to uniform-in-time estimate by using Logarithmic Sobolev inequality. Indeed, given that $\frac 1 c \leq \bar \rho_t \leq c$, one has that
\begin{equation}
    \label{LogSoboIne}
    \int_{\Pi^d} \rho_t \log \frac{\rho_t}{\bar \rho_t} \ud x  \leq \frac{c^2}{8 \pi^2 } \int_{\Pi^d} \rho_t |\nabla_x \log \frac{\rho_t}{\bar \rho_t }|^2 \ud x. 
 \end{equation}
 Combining \eqref{LogSoboIne} and  \eqref{NSEFinal}, one obtains that 
 \[
\frac{\ud }{\ud t} \mathbf{KL}(\rho_t \vert \bar \rho_t) \leq \Big( M(t) - \frac{4 \pi^2 \nu  }{c^2} \Big) \mathbf{KL} (\rho_t \vert \bar \rho_t) +  \frac{1}{\nu } \int \rho_t |\delta_t|^2. 
 \]
Multiplying the factor $\exp(\frac{4 \pi^2 \nu  }{c^2} t - \int_0^t M(s) \ud s)$ and noting in particular $\mathbf{KL} (\rho_0 \vert \bar \rho_0) =0$,  one obtains that 
\[
\mathbf{KL}(\rho_t \vert \bar \rho_t) \leq \int_0^t \exp(\frac{4 \pi^2 \nu  }{c^2} (s- t) + \int_s^t M(u) \ud u ) f(s) \ud s. 
\]
Indeed, under the assumptions as in Theorem \ref{NSMainEstimate}, one has that there exists a universal $C>0$, such that 
\[
\int_0^\infty M(t) \ud t = C < \infty. 
\]
We thus immediately obtain that 
\[
\sup_{t \in [0, T] }\mathbf{KL}(\rho_t \vert \bar \rho_t) \leq  \frac{e^C}{\nu} \int_0^T \int_{\Pi^d} \rho_t |\delta_t|^2 \ud x \ud t. 
\]
This completes the proof of Theorem \ref{NSMainEstimate}. 


%Since now we are only care about computing solutions in a fixed time interval $[0, T]$, we do not seek to optimize the factor $\exp(MT)$. We leave the long time asymptotic analysis as a separate work. 
%\end{remark}

 


\paragraph{The McKean-Vlasov PDEs, i.e. \eqref{eqn_MVE}, with bounded interactions $K \in L^\infty$ }  
As mentioned in the main body of this article,  it is much easier to  obtain the stability estimate for the McKean-Vlasov PDE with bounded interactions. 
\begin{theorem}[Stability Estimate for McKean-Vlasov PDE with $K \in L^\infty$] \label{theorem_bounded_K}
	Assume that $K \in L^\infty$. One has the estimate that 
	\[
	\sup_{t \in [0, T]} \mathbf{KL}(\rho_t^f \vert \bar \rho_t ) \leq   \frac{1}{\nu}\exp\Big(\frac{2 \|K \|_{L^\infty}^2}{\nu } T  \Big)  R(f), 
	\]
	where we recall the self-consitency potential/loss function $R(\theta)$ reads 
	\[
	R(f) = \int_0^T \int_{\X} |f(t, x) + \nabla V (x) - K * \rho_t^f + \nu \nabla \log \rho_t^\theta  |^2 \ud \rho_t^f(x) \ud t. 
	\]
\end{theorem}

\begin{proof}
	Here we give the control of the growth of the KL divergence for systems with bounded kernels.  
	Applying Cauchy-Schwarz inequality twice for the entropy dissipation terms  in Lemma \ref{TimeEvolKLMV} to obtain 
	\[
	\int_{\Pi^d} \rho_t K * (\rho_t - \bar \rho_t ) \cdot \nabla  \log  \frac{\rho_t }{\bar \rho_t}  \leq \frac{\nu}{4} \int \rho_t |\nabla \log \frac{\rho_t}{\bar \rho_t}|^2 + \frac{1}{\nu } \int \rho_t |K * (\rho_t - \bar \rho_t)|^2, 
	\]
	and 
	\[
	\int_{\Pi^d} \rho_t \delta_t \cdot \nabla \log \frac{\rho_t }{\bar \rho_t } \leq \frac{\nu}{4} \int \rho_t |\nabla \log \frac{\rho_t}{\bar \rho_t}|^2 + \frac{1}{\nu } \int \rho_t |\delta_t|^2. 
	\]
	Furthermore, 
	\[
	\int \rho_t |K * (\rho_t - \bar \rho_t)|^2 \leq \|K \|_{L^\infty}^2  \|\rho_t - \bar \rho_t\|_{L^1}^2 \leq  2 \|K \|_{L^\infty}^2  \int \rho_t \log \frac{\rho_t}{\bar \rho_t }, 
	\]
	where the last inequality is simply the Csisz\'ar–Kullback–Pinsker inequality \citep{villani2009optimal}. Combining the above estimates, we obtain that given that $K \in L^\infty$, 
	\[
	\frac{\ud }{\ud t} \int_{\Pi^d} \rho_t \log \frac{\rho_t }{\bar \rho_t} = - \frac{\nu}{2}   \int_{ \Pi^d}\rho_t |\nabla \log \frac{\rho_t}{\bar \rho_t}|^2 + \frac{2 \|K \|_{L^\infty}^2}{\nu } \int \rho_t \log \frac{\rho_t}{\bar \rho_t } + \frac{1}{\nu } \int \rho_t |\delta_t|^2. 
	\]
	Currently, we are not interested in the long time behavior, so we first ignore the negative term above to obtain that 
	\[
	\frac{\ud }{\ud t} \int_{\Pi^d} \rho_t \log \frac{\rho_t }{\bar \rho_t}  \leq    \frac{2 \|K \|_{L^\infty}^2}{\nu } \int \rho_t \log \frac{\rho_t}{\bar \rho_t } + \frac{1}{\nu } \int \rho_t |\delta_t|^2. 
	\]
	By Gronwall inequality, we obtain that
	\[
	\int_{\Pi^d} \rho_t \log \frac{\rho_t }{\bar \rho_t} \leq  \frac{1}{\nu}\exp\Big(\frac{2 \|K \|_{L^\infty}^2}{\nu } t \Big)   \int_0^t \int \rho_s |\delta_s|^2 \ud x  \ud s. 
	\]
	
	\end{proof}

\subsection{The McKean-Vlasov equation with Coulomb interactions}

\begin{proof}[Proof of Theorem \ref{ThmCoul}]  We first prove the case when $\nu >0$. Applying Cauchy-Schwarz inequality to the right-hand side of $\frac{\ud }{\ud t} E(\rho_t^f, \bar \rho_t)$ in Lemma \ref{TimeEvoMFE},  one has  
\[
\begin{split}
	\frac{\ud }{\ud t } E(\rho_t^f, \bar \rho_t)  & \leq  \frac 1 2  \int_{\mathcal{X}} \rho_t^f \, |\delta_t |^2 \ud x  \\
	&  - \frac{1}{2} \int_{\mathcal{X}^2} K(x-y) \cdot \Big( \mathcal{A}[\bar \rho_t](x) - \mathcal{A}[\bar \rho_t](y) \Big) \ud (\rho_t^f - \bar \rho_t )^{\otimes 2 }(x, y). 
\end{split}
\]
By Lemma 5.2 in \cite{bresch2019modulated}, as long as the ground truth ``velocity field" $\mathcal{A}[\bar \rho_t]$ is Lipschitz, i.e. $\mathcal{A}[\bar \rho] \in W^{1, \infty}$, or equivalently  $\nabla^2 V \in W^{1, \infty}, \nabla^2 \log \bar \rho_t \in L^\infty, K * \bar \rho_t \in W^{1, \infty}$, using the particular structure introduced by the Coulomb interactions (note that $- \Delta g = \delta_0$ and $K = - \nabla g$), we have the estimate 
\[
\begin{split}
	&  - \frac{1}{2} \int_{\mathcal{X}^2} K(x-y) \cdot \Big( \mathcal{A}[\bar \rho_t](x) - \mathcal{A}[\bar \rho_t](y) \Big) \ud (\rho_t^f - \bar \rho_t )^{\otimes 2 }(x, y) \\
	& \leq C \|\nabla \mathcal{A}[\bar \rho_t]\|_{L^\infty}  F(\bar \rho_t^f, \bar \rho_t).   \\
\end{split} 
\] 
This estimate can be obtained either by Fourier method  \citep{bresch2019modulated} or by the stress-energy tensor approach as in \citep{serfaty2020mean}. 
We emphasize that those assumptions made on $(\bar \rho_t)_{t \in [0, T]}$  can be obtained by propagating similar conditions on the initial data $\bar \rho_0$. 
This estimate actually holds for more general choices of $g$ or  $K$. See more examples including Riesz kernels in \citep{bresch2019modulated}.  Moreover, the Lipschitz regularity of $\mathcal{A}[\bar \rho_t]$ can also be relaxed a bit. See for instance in \citep{rosenzweig2022mean}.

Combining  previous two estimates, one has 
	\[
	\frac{\ud }{\ud t } E(\rho_t^f, \bar \rho_t)   \leq  \frac 1 2  \int_{\mathcal{X}} \rho_t^f \, |\delta_t |^2 \ud x  + C C_1  F(\rho_t^f, \bar \rho_t) \leq \frac 1 2  \int_{\mathcal{X}} \rho_t^f \, |\delta_t |^2 \ud x +  C C_1   E(\rho_t^f, \bar \rho_t). 
	\]
	Then applying Gronwall inequality concludes the proof of the case when $\nu >0$.
	
   Now we prove the deterministic case when $\nu=0$. Now the relative entropy or KL divergence does not play a role since there is no Laplacian term in \eqref{eqn_MVE}.  Lemma \ref{ModuEnergyEvo} now reads 
   	\[
   \begin{split}
   	\frac{\ud }{\ud t } F(\rho_t^f, \bar \rho_t)&  =  - \int_{\mathcal{X}} \rho_t^f |K *(\rho_t^f - \bar \rho)|^2 - \int_{\mathcal{X}} \rho_t^f \, \delta_t \cdot K * (\rho_t^f - \bar \rho_t ) \\
   	&  - \frac{1}{2} \int_{\mathcal{X}^2} K(x-y) \cdot \Big( \mathcal{A}[\bar \rho_t](x) - \mathcal{A}[\bar \rho_t](y) \Big) \ud (\rho_t^f - \bar \rho_t )^{\otimes 2 }(x, y). \\
      \end{split}
   \]
   Applying Cauchy-Schwarz to the 2nd term in the right-hand side above, we obtain that 
   \[
    \begin{split}
    	\frac{\ud }{\ud t } F(\rho_t^f, \bar \rho_t) \leq    \frac 1 2 \int_{\mathcal{X}} \rho_t^f \, |\delta_t|^2     - \frac{1}{2} \int_{\mathcal{X}^2} K(x-y) \cdot \Big( \mathcal{A}[\bar \rho_t](x) - \mathcal{A}[\bar \rho_t](y) \Big) \ud (\rho_t^f - \bar \rho_t )^{\otimes 2 }(x, y). \\
    \end{split}
   \]
   Again assuming that the ``velocity field'' $\mathcal{A}[\bar \rho_t](\cdot)$ is Lipschitz will give us 
   \[
   \frac{\ud }{\ud t } F(\rho_t^f, \bar \rho_t) \leq  \frac 1 2 \int_{\mathcal{X}} \rho_t^f \, |\delta_t|^2  + C C_1 F(\rho_t^f, \bar \rho_t). 
   \] 
   Applying Gronwall inequality again conclude all the proof. 
   
   
   
    
\end{proof}

	


 
\clearpage
\clearpage
\section{Approximation Error of Neural Network} \label{appendix_approximation_error_NN}
We show that in a function class $\gF$ with sufficient capacity, there exists at least one element $\hat f\in \gF$ such that $R(\hat f)$ is small.
In particular, we are interested in the function class of neural networks.

We will focus on the case where the domain is the torus $\X = \Pi^d$, i.e. a $d$ dimensional box with size $L$ endowed with the periodic boundary condition. For the simplicity of notations, we denote the underlying velocity by $\bar f_t = \gA[\bar \rho_t]$, where the operator $\gA$ is defined in \eqref{eqn_operator_A}.

In the following, we focus on the Coulomb case where $K$ is defined in \eqref{eqn_coulomb_interaction}. The Biot-Savart case (\ref{eqn_nse}) can be treated similarly.
\begin{proof}[Proof of Theorem \ref{thm_approximation_error_NN}]
	In \eqref{eqn_pathwise_reformulation}, we showed that for any hypothesis velocity $f$, the \EINN\ loss $R(f)$ admits the trajectory-wise reformulation:
	\begin{equation}
		R(f) = \int_\X \ud \vx_0 \bar \rho_0(\vx) \int_0^T \ud t\|\delta^f_t\circ X^f_t(\vx_0)\|^2,
	\end{equation}
	where we recall the definition of $\delta^f_t$ in \eqref{eqn_perturbation}. 
    Note that, as a general principle, in this proof, we will use the superscript to emphasize the dependence on a velocity $f$, e.g. the flow map $X^f_t$.
	
	From Assumption \ref{ass_appendix_approximation} we know that there exists $\hat f\in\gF$ such that $\|\bar f - \hat f\|_{W^{2,\infty}(\X)} \leq \epsilon$.
	In the following, we show that $R(\hat f)$ is small.
	
	Define 
	\begin{equation}
		A^f_\vx(t) \defi \int_0^t \|\delta_s^f\circ X_s^f(\vx)\|^2 \ud s.
	\end{equation}
	We have
	\begin{equation}
		R(f) = \int_\X A_\vx^f(T) \ud \bar \rho_0(\vx).
	\end{equation}
	Recall that $\bar f$ denotes the underlying velocity and hence $\delta_t^{\bar f} \equiv 0$ and $\rho_t^{\bar f} \equiv \bar \rho_t$,
	where we recall that $\rho_t^{\bar f}$ is the solution to the continuity equation (\ref{eqn_CE}) with velocity field $\bar f$.
	We can bound
	\begin{align}
		\notag \frac{\partial}{\partial t}A^{\hat f}_x(t) =&\ \|\delta_t^{\hat f}\circ X_t^{\hat f}(\vx)\|^2 = \|\delta_t^{\hat f}\circ X_t^{\hat f}(\vx) - \delta_t^{\bar f}\circ X_t^{\bar f}(\vx)\|^2 \\
		\notag \leq&\ 4\|\left(\hat f_t\circ X_t^{\hat f} - \bar f_t\circ X_t^{\bar f}\right)(\vx)\|^2 + 4\|\left(\nabla V\circ X_t^{\hat f} - \nabla V\circ X_t^{\bar f}\right)(\vx)\|^2 \\
		\notag &\ + 4\|\left((K\ast \rho_t^{\hat f})\circ X_t^{\hat f} - (K\ast \bar \rho_t)\circ X_t^{\bar f}\right)(\vx)\|^2 + 4\nu^2\|\left(\nabla \log \rho_t^{\hat f}\circ X_t^{\hat f} - \nabla \log \bar \rho_t\circ X_t^{\bar f}\right)(\vx)\|^2\\
		= &\ \textcircled{1} + \textcircled{2} + \textcircled{3} + \textcircled{4}. \label{eqn_appendix_decomposition_A}
	\end{align}
	We will bound each term on the R.H.S. individually.
	The following lemmas will be useful:
	\begin{lemma} \label{lemma_appendix_Lipschitz_X_t_f}
		For two Lipschitz continuous velocity field $f_1, f_2 \in \gC^1(\X)$, we have for any $t\in[0, T]$
		\[
		\|X_t^{f_1}(\vx) - X_t^{f_2}(\vx)\|^2 \leq A_1(T)\|f_1 - f_2\|^2_{\gL^\infty(\X)}.
		\]
	\end{lemma}
    \begin{proof}
        Denote $\vx^i(t) = X_t^{f_i}(\vx_0)$ for $i = 1, 2$.
        \begin{align*}
            \frac{\ud }{\ud t}\|\vx^1(t) - \vx^2(t)\|^2 \leq&\ \|\vx^1(t) - \vx^2(t)\|^2 + \|f_1(t, \vx^1(t)) - f_2(t, \vx^2(t))\|^2\\
            \leq &\ C \|\vx^1(t) - \vx^2(t)\|^2 + \|f_1(t, \vx^2(t)) - f_2(t, \vx^2(t))\|^2 \\
            \leq &\ C \|\vx^1(t) - \vx^2(t)\|^2 + \|f_1 - f_2\|^2_{\gL^\infty(\X)}.
        \end{align*}
        Using the Gr\"onwall's inequality, we have the result.
    \end{proof}
	\begin{lemma} \label{lemma_appendix_Lipschitz_X_t}
		Suppose that $f\in \gC^1$ is Lipschitz continuous. We have that $X_t^f$ is an $A_2(T)$-Lipschitz continuous map. 
		For $f \in \gL^\infty(\X)$, we have $\|X_t^f(\vx)\|^2 \leq \|x\|^2 + t\|f\|^2_{\gL^\infty}$.
	\end{lemma}
    \begin{proof}
        Denote $\vx^i(t) = X_t^{f}(\vx^i_0)$ for $i = 1, 2$.
        \begin{align*}
            \frac{\ud }{\ud t}\|\vx^1(t) - \vx^2(t)\|^2 \leq&\ \|\vx^1(t) - \vx^2(t)\|^2 + \|f(t, \vx^1(t)) - f(t, \vx^2(t))\|^2\\
            \leq &\ C\|\vx^1(t) - \vx^2(t)\|^2.
        \end{align*}
        Using the Gr\"onwall's inequality, we have that $X_t^f$ is Lipschitz continuous.
        % Further, denote $\vx(t) = X_t^{f}(\vx_0)$. We have
        % \begin{align*}
        %     \frac{\ud }{\ud t}\|\vx(t)\|^2 \leq \|\vx(t)\|^2 + \|f(t, \vx(t))\|^2 \leq C\|\vx(t)\|^2.
        % \end{align*}
        % Using the Gr\"onwall's inequality, we have that $X_t^f$ is linearly bounded.
    \end{proof}
	\begin{lemma}
		For $f \in \gC^2(\X)$, suppose that $\nabla (\udiv  f)\in \gL^\infty(\X)$ and $\gJ_f \in \gL^\infty(\X)$. 
        Further suppose that the initial distribution $\bar \rho_0$ satisfies $\nabla \log \bar \rho_0 \in \gL^\infty(\X)$.
        We have that $\nabla \log \rho_t^f\circ X_t^f \in \gL^\infty(\X)$.
	\end{lemma}
    \begin{proof}
        Denote $\vx(t) = X_t^{f}(\vx_0)$. From \eqref{eqn_dynamics_of_score}, we have
        \begin{equation}
           \frac{\ud }{\ud t} \|\nabla \log \rho_t^f(\vx(t))\|^2 \leq C (1+\|\nabla \log \rho_t^f(\vx(t))\|^2).
        \end{equation}
        Using the Gr\"onwall's inequality, we have $\|\nabla \log \rho_t^f(\vx(t))\|^2 < \infty$ for $\vx_0 \in \X$.
    \end{proof}
	\paragraph{Bounding \textcircled{1} in \eqref{eqn_appendix_decomposition_A}}
	We have
	\begin{align*}
		\notag &\ \|\left(\hat f_t\circ X_t^{\hat f} - \bar f_t\circ X_t^{\bar f}\right)(\vx)\| \\
		\leq &\ \|\left(\hat f_t\circ X_t^{\hat f} - \bar f_t\circ X_t^{\hat f}\right)(\vx)\| + \|\left(\bar f_t\circ X_t^{\hat f} - \bar f_t\circ X_t^{\bar f}\right)(\vx)\| \\
		\leq &\ \epsilon + \epsilon \cdot \| X_t^{\hat f}(\vx) - X_t^{\bar f}(\vx)\| \leq C_1(T) \epsilon.
	\end{align*}
	\paragraph{Bounding \textcircled{2} in \eqref{eqn_appendix_decomposition_A}}
	We have from Assumption \ref{ass_appendix_approximation} and Lemma \ref{lemma_appendix_Lipschitz_X_t_f}
	\begin{align*}
		\|\left(\nabla V\circ X_t^{\hat f} - \nabla V\circ X_t^{\bar f}\right)(\vx)\| \leq C_2(T) \epsilon.
	\end{align*}
	\paragraph{Bounding \textcircled{3} in \eqref{eqn_appendix_decomposition_A}}
	Bounding \textcircled{3} in \eqref{eqn_appendix_decomposition_A} requires a more sophisticated analysis which is the major technical challenge of this proof. 
	For the simplicity of notations, for a fixed $x$ and $y$, denote $\hat \vx(t) = X_t^{\hat f}(\vx)$, $\bar \vx(t) =  X_t^{\bar f}(\vx)$, $\hat \vy(t) = X_t^{\hat f}(\vy)$, $\bar \vy(t) =  X_t^{\bar f}(\vy)$. For any $\epsilon'$ which is to be determined later, we have
	\begin{align}
		\notag &\ \|\left((K\ast \rho_t^{\hat f})\circ X_t^{\hat f} - (K\ast \bar \rho_t)\circ X_t^{\bar f}\right)(\vx)\|^2
		= \| \int_\X K(\hat \vx(t) - \hat \vy(t)) - K(\bar \vx(t) - \bar \vy(t)) \ud \bar \rho_0(\vy)\|^2\\
		\tag{\textcircled{A}}\leq &\ 2\| \int_{\|\hat \vx(t) - \hat \vy(t)\|\leq \epsilon'} K(\hat \vx(t) - \hat \vy(t)) - K(\bar \vx(t) - \bar \vy(t)) \ud \bar \rho_0(\vy)\|^2\\
		\tag{\textcircled{B}} &\ + 2\| \int_{A_2(T)L\geq \|\hat \vx(t) - \hat \vy(t)\|\geq \epsilon'} K(\hat \vx(t) - \hat \vy(t)) - K(\bar \vx(t) - \bar \vy(t)) \ud \bar \rho_0(\vy)\|^2.
	\end{align}
	Note that the upper bound on $\|\hat \vx(t) - \hat \vy(t)\|$ in \textcircled{B} comes from Lemma \ref{lemma_appendix_Lipschitz_X_t} and the facts that $\X = \Pi^d$ is bounded with size $L$.
	To bound \textcircled{A}, we have
	\begin{align*}
		&\ \| \int_{\|\hat \vx(t) - \hat \vy(t)\|\leq \epsilon'} K(\hat \vx(t) - \hat \vy(t)) - K(\bar \vx(t) - \bar \vy(t)) \ud \bar \rho_0(\vy)\| \\
		\leq &\  \int_{\|\hat \vx(t) - \hat \vy(t)\|\leq \epsilon'} \|K(\hat \vx(t) - \hat \vy(t))\| + \|K(\bar \vx(t) - \bar \vy(t))\| \ud \bar \rho_0(\vy)\\
		= &\ \int_{\|\hat \vx(t) - \hat \vy(t)\|\leq \epsilon'} \frac{1}{\|\hat \vx(t) - \hat \vy(t)\|^{d-1}} + \frac{1}{\|\bar \vx(t) - \bar \vy(t)\|^{d-1}} \ud \bar \rho_0(\vy) = \textcircled{C} + \textcircled{D}.
	\end{align*}
	We can bound \textcircled{C} by
	\begin{align}
		&\ \int_{\|\hat \vx(t) - \hat \vy(t)\|\leq \epsilon'} \frac{1}{\|\hat \vx(t) - \hat \vy(t)\|^{d-1}} \ud \bar \rho_0(\vy) = \int_{\|\hat \vx(t) - y\|\leq \epsilon'} \frac{1}{\|\hat \vx(t) - y\|^{d-1}} \ud \rho^{\hat f}_t(\vy) \leq \|\rho^{\hat f}_t\|_{\infty}\cdot \epsilon',
	\end{align}
	where in the above inequality we remove the singular term by using transforming to the polar coordinate system.
	To bound \textcircled{D}, we pick $\epsilon' = dA_1(T)\epsilon$, so that Lemma \ref{lemma_appendix_Lipschitz_X_t_f} implies 
	\begin{equation*}
		\{ y\in\X | \|\hat \vx(t) - \hat \vy(t)\|\leq \epsilon'\} \subseteq \{y\in\X | \|\bar \vx(t) - \bar \vy(t)\|\leq \frac{d+2}{d}\epsilon'\},
	\end{equation*}
	and consequently
	\begin{equation}
		\int_{\|\hat \vx(t) - \hat \vy(t)\|\leq \epsilon'} \frac{1}{\|\bar \vx(t) - \bar \vy(t)\|^{d-1}} \ud \bar \rho_0(\vy) \leq \int_{\|\bar \vx(t) - \bar \vy(t)\|\leq 2\epsilon'} \frac{1}{\|\bar \vx(t) - \bar \vy(t)\|^{d-1}} \ud \bar \rho_0(\vy) \leq \frac{d+2}{d} \|\rho^{\hat f}_t\|_{\infty}\cdot \epsilon'.
	\end{equation}
	To bound \textcircled{B}, note that
	\begin{equation}
		\nabla K(\vx) = \frac{1}{\|x\|^{d+2}}(\|x\|^2 I - d\cdot x\otimes x) \Rightarrow \|\nabla K(\vx)\|\leq \frac{d}{\|x\|^{d}}.
	\end{equation}
	Denote $\vz(t) = \min(\|\hat \vx(t) - \hat \vy(t)\|, \|\bar \vx(t) - \bar \vy(t)\|)$.
%	and define
%	\begin{equation}
%		\tilde X_t = \begin{cases}
%			X_t^{\hat f} &\text{if}\quad  \|\hat \vx(t) - \hat \vy(t)\| \leq \|\bar \vx(t) - \bar \vy(t)\| \\
%			X_t^{\bar f} &\text{otherwise}
%		\end{cases}
%	\end{equation}
	Recall the choice of $\epsilon' = dA_1(T)\epsilon$.
	Using Lemma \ref{lemma_appendix_Lipschitz_X_t_f}, we have that 
	\begin{equation*}
%		\{ y\in\X | \|\hat \vx(t) - \hat \vy(t)\|\geq \epsilon'\} \subseteq \{y\in\X | \|\vz(t)\|\geq \frac{d-2}{d}\epsilon'\}.
		\|\vz(t)\| \geq \frac{d-2}{d} \|\hat \vx(t) - \hat \vy(t)\|.
	\end{equation*}
	
	
%	Using the triangle inequality, we have
%	\begin{align}
%		&\ \| \int_{\|\hat \vx(t) - \hat \vy(t)\|\geq \epsilon'} K(\hat \vx(t) - \hat \vy(t)) - K(\bar \vx(t) - \bar \vy(t)) \ud \bar \rho_0(\vy)\| \\
%		\leq &\ \int_{\|\hat \vx(t) - \hat \vy(t)\|\geq \epsilon'} \|K(\hat \vx(t) - \hat \vy(t)) - K(\bar \vx(t) - \bar \vy(t))\| \ud \bar \rho_0(\vy)\\
%		\leq &\ 2\epsilon' d\int_{\|\vz(t)\|\geq .5\epsilon'} \frac{1}{\|\vz(t)\|^{d}} \ud \bar \rho_0(\vy) = 2\epsilon' d \max \{\|\rho^{\hat f}_t\|_{\infty}, \|\bar \rho_t\|_{\infty}\} \int_{\|z\|\geq .5\epsilon'} \frac{1}{\|z\|^d} \ud z \\
%		=&\ 2\epsilon' d \max \{\|\rho^{\hat f}_t\|_{\infty}, \|\bar \rho_t\|_{\infty}\} \ln (\|\vz(t)\|_\infty/\epsilon')
%	\end{align}

	Using the triangle inequality, we have
	\begin{align*}
		&\ \| \int_{A_2(T)L \geq\|\hat \vx(t) - \hat \vy(t)\|\geq \epsilon'} K(\hat \vx(t) - \hat \vy(t)) - K(\bar \vx(t) - \bar \vy(t)) \ud \bar \rho_0(\vy)\| \\
		\leq &\ \int_{A_2(T)L \geq\|\hat \vx(t) - \hat \vy(t)\|\geq \epsilon'} \|K(\hat \vx(t) - \hat \vy(t)) - K(\bar \vx(t) - \bar \vy(t))\| \ud \bar \rho_0(\vy)\\
		\leq &\ 2\epsilon' d\int_{A_2(T)L \geq\|\hat \vx(t) - \hat \vy(t)\|\geq \epsilon'} \frac{1}{\|\vz(t)\|^{d}} \ud \bar \rho_0(\vy) \\
		\leq &\ 2\epsilon' d\int_{A_2(T)L \geq\|\hat \vx(t) - \hat \vy(t)\|\geq \epsilon'} (\frac{d}{d-2})^d\frac{1}{\|\hat \vx(t) - \hat \vy(t)\|^{d}} \ud \bar \rho_0(\vy) \\
		\leq &\ 2e\epsilon' d \|\rho^{\hat f}_t\|_{\infty} \int_{A_2(T)L \geq \|\hat \vx(t) - y\|\geq .5\epsilon'} \frac{1}{\|y\|^d} \ud y \\
		=&\ 2e\epsilon' d \|\rho^{\hat f}_t\|_{\infty} \ln (A_2(T)L/\epsilon').
	\end{align*}

	Combining the bounds of \textcircled{A} and \textcircled{B}, we have that 
	\begin{equation}
		\textcircled{3} \leq C_3(T) (\epsilon \ln\frac{1}{\epsilon})^2.
	\end{equation}
	\paragraph{Bounding \textcircled{4} in \eqref{eqn_appendix_decomposition_A}}
	Denote $\hat \vx(t) = X_t^{\hat f}(\vx)$ and $\bar \vx(t) =  X_t^{\bar f}(\vx)$.
	Define
	\begin{equation}
		B_\vx(t) \defi \|\left(\nabla \log \rho_t^{\hat f}\circ X_t^{\hat f} - \nabla \log \bar \rho_t\circ X_t^{\bar f}\right)(\vx)\|^2 = \|\nabla \log \rho_t^{\hat f}(\hat{\vx}(t)) - \nabla \log \bar \rho_t(\bar{\vx}(t))\|^2.
	\end{equation}
	Computing its dynamics
	\begin{equation}
		\frac{\ud }{\ud t} B_\vx(t) \leq B_\vx(t) + \|\frac{\ud }{\ud t}\left(\nabla \log \rho_t^{\hat f}(\hat{\vx}(t)) - \nabla \log \bar \rho_t(\bar{\vx}(t))\right)\|^2
	\end{equation}
	Recall \eqref{eqn_dynamics_of_score}. We have that 
	\begin{equation}
		\frac{\ud }{\ud t} \nabla \log \rho_t^{\hat f}(\hat \vx(t)) = - \nabla  \left(\nabla \cdot \hat f_{t}(\hat \vx(t))\right) -   \left(\gJ_{\hat f_{t}}(\hat \vx(t))\right)^\top \nabla \log \rho_{t}^{\hat f}(\hat \vx(t)),
	\end{equation}
	\begin{equation}
		\frac{\ud }{\ud t} \nabla \log \rho_t^{\bar  f}(\bar  \vx(t)) = - \nabla  \left(\nabla \cdot \bar  f_{t}(\bar  \vx(t))\right) -   \left(\gJ_{\bar  f_{t}}(\bar  \vx(t))\right)^\top \nabla \log  \rho_{t}^{\bar  f}(\bar  \vx(t)),
	\end{equation}
	and hence
	\begin{align*}
		&\ \|\frac{\ud }{\ud t}\left(\nabla \log \rho_t^{\hat f}(\hat{\vx}(t)) - \nabla \log \bar \rho_t(\hat{\vx}(t))\right)\|^2 \\
		\leq &\ 2\|\nabla  \left(\nabla \cdot \hat f_{t}(\hat \vx(t))\right) - \nabla  \left(\nabla \cdot \bar  f_{t}(\bar  \vx(t))\right)\|^2 
		 + 2\|\left(\gJ_{\hat f_{t}}(\hat \vx(t))\right)^\top \nabla \log \rho_{t}^{\hat f}(\hat \vx(t)) - \left(\gJ_{\bar  f_{t}}(\bar  \vx(t))\right)^\top \nabla \log  \rho_{t}^{\bar  f}(\bar  \vx(t))\|^2 \\
		= &\ \textcircled{E} + \textcircled{F}.
	\end{align*}
	We now bound these two terms individually. To bound \textcircled{E},
	\begin{align*}
		&\ \|\nabla  \left(\nabla \cdot \hat f_{t}(\hat \vx(t))\right) - \nabla  \left(\nabla \cdot \bar  f_{t}(\bar  \vx(t))\right)\| \\
		\leq &\ \|\nabla  \left(\nabla \cdot \hat f_{t}(\hat \vx(t))\right) - \nabla  \left(\nabla \cdot \hat  f_{t}(\bar  \vx(t))\right)\| + \|\nabla  \left(\nabla \cdot \hat f_{t}(\bar \vx(t))\right) - \nabla  \left(\nabla \cdot \bar  f_{t}(\bar  \vx(t))\right)\| \leq \epsilon + LA_1(T)\epsilon.
	\end{align*}
	To bound \textcircled{F}
	\begin{align*}
		&\ \|\left(\gJ_{\hat f_{t}}(\hat \vx(t))\right)^\top \nabla \log \rho_{t}^{\hat f}(\hat \vx(t)) - \left(\gJ_{\bar  f_{t}}(\bar  \vx(t))\right)^\top \nabla \log  \rho_{t}^{\bar  f}(\bar  \vx(t))\|^2\\
		\leq &\ 2\|\left(\left(\gJ_{\hat f_{t}}(\hat \vx(t))\right)^\top - \left(\gJ_{\bar  f_{t}}(\bar  \vx(t))\right)^\top\right) \nabla \log  \rho_{t}^{\hat  f}(\hat  \vx(t))\|^2 + 2\|\left(\gJ_{\bar f_{t}}(\bar \vx(t))\right)^\top \left(\nabla \log \rho_{t}^{\hat f}(\hat \vx(t)) - \nabla \log  \rho_{t}^{\bar  f}(\bar  \vx(t))\right)\|^2\\
		\leq &\ 2(1 + LA_1(T))^2\epsilon^2 + 2L^2 B_x(t).
	\end{align*}
	Consequently, using Gr\"onwall's inequality, we have that
	\begin{equation}
		\textcircled{4} \leq C_4(T)\epsilon^2.
	\end{equation}


	Combining all the estimations for \textcircled{1} to \textcircled{4}, we have that
	\begin{equation}
		R(\hat f) \leq C(T) \epsilon^2 (\ln \frac{1}{\epsilon})^2,
	\end{equation}
	for some constant $C(T)$ independent of $\epsilon.$
\end{proof}

\clearpage
\section{Discussion on the Unbounded Case} \label{appendix_unbounded}
In Section \ref{section_analysis}, we considered the torus case, i.e. $\X$ is a $d$-dimensional box with size $L$ with a periodic boundary condition. In this section, we consider the unbounded case, i.e. $\X = \sR^d$. 
There are two major differences: 
\begin{enumerate}[leftmargin=*]
    \item The first difference is that when $\X = \sR^d$, we would obtain an additional integral-of-divergence term from the operation of integration by parts. When $\X$ is a torus, using Gauss's divergence theorem and the periodic boundary condition, this term immediately vanishes, which simplifies the analysis. In contrast, for the unbounded case, we need to handle this term by assuming some additional regularity conditions.
    \item The second difference is that for the torus, it is reasonable to assume that the initial distribution $\bar \rho_0$ is fully supported, which is equivalent to the existence of some constant $c > 0$ such that $\bar \rho_0(\vx) \geq c$ for all $\vx\in\X$. Such an assumption will allow us to propagate the regularity of the initial distribution $\bar \rho_0$ to the solution at time $t$, i.e. $\bar \rho_t$.
    In contrast, for the unbounded case, such an assumption clearly does not hold since otherwise $\bar \rho_0$ would not be integrable. Consequently, we can no long propagate the regularity of the initial distribution and hence we need to directly make regularity assumptions on $\bar \rho_t$.
\end{enumerate}
In the following, we will focus on addressing the first point and provide sufficient conditions such that Lemmas \ref{TimeEvolKLMV} and  \ref{ModuEnergyEvo} can be recovered even in the unbounded case.
To elaborate a bit on the second point, the theorems that are derived in the main body of the submission remain valid under the regularity assumptions given therein. However, unlike the torus case, it is difficult to establish these regularity results for the unbounded case by assuming the regularity of the initial distribution $\bar \rho_0$.
% \subsection{Derivation of Lemma \ref{TimeEvolKLMV}}
\begin{lemma}[Analogy of Lemma  \ref{TimeEvolKLMV} in the unbounded case] \label{lemma_TimeEvolKLMV}
    Given the hypothesis velocity field $f=f(t, x) \in C^1_{t, x}$. Assume that $(\rho_t^f)_{t \in [0, T]}$ and $(\bar \rho_t)_{t \in [0, T]}$ are classical solutions to equation (\ref{eqn_CE}) and equation (\ref{eqn_MVE_CE}) respectively.  It holds that (recall the definition of $\delta_t$ in equation (\ref{eqn_perturbation})) 
    \begin{align*}
        \frac{\ud }{\ud t} \int_{\mathcal{X}} \rho^f_t \log \frac{\rho^f_t }{\bar \rho_t} =&\ - \nu \int_{ \X}\rho^f_t |\nabla \log \frac{\rho^f_t}{\bar \rho_t}|^2 +  \int_{\X} \rho^f_t K * (\rho^f_t - \bar \rho_t ) \cdot \nabla  \log  \frac{\rho^f_t }{\bar \rho_t} \\
        &\ +  \int_{\X} \rho^f_t \delta_t \cdot \nabla \log \frac{\rho^f_t }{\bar \rho_t } - \int \udiv \Big(\rho^f_t  (f_t  \log \frac{\rho^f_t}{\bar \rho_t} - \bar f_t)\Big).
    \end{align*}
	where $\X$ is the tours $\Pi^d$. All the integrands are evaluated at $\vx$. 
\end{lemma}
\begin{proof} 
	Recall the McKean-Vlasov  \eqref{eqn_MVE_CE} and the continuity  \eqref{eqn_CE_as_MVE}. For simplicity, we write that $\rho_t = \rho_t^f$ and $\bar f_t = \gA[\bar \rho_t]$. Then 
    \begin{align*}
        \frac{\ud }{\ud t } \int \rho_t \log \frac{\rho_t }{\bar \rho_t } =&\ - \int  \udiv \Big(    \rho_t  f_t \Big) \log \frac{\rho_t}{\bar \rho_t}  +  \int \frac{\rho_t}{\bar \rho_t}  \udiv \Big(   \bar \rho_t \bar f_t\Big)\\
        =&\ \int  \rho_t  f_t \nabla \log \frac{\rho_t}{\bar \rho_t}  -  \int \nabla \frac{\rho_t}{\bar \rho_t}  \bar \rho_t \bar f_t - \int \udiv \Big(    \rho_t  (f_t  \log \frac{\rho_t}{\bar \rho_t} - \bar f_t)\Big).
    \end{align*}
    We handle the first two terms on the R.H.S. just like the torus case and we can have the result.
\end{proof}

\begin{lemma}[Analogy of Lemma  \ref{ModuEnergyEvo} in the unbounded case] \label{lemma_ModuEnergyEvo}
Under the same assumptions as in Lemma \ref{TimeEvolKLMV}, given the diffusion coefficient $\nu \geq 0$, it holds that (recall the definition of $\delta_t$ in equation (\ref{eqn_perturbation})) 
    \begin{align*}
        \frac{\ud }{\ud t } F(\rho_t^f, \bar \rho_t)  =&\  - \int_{\mathcal{X}} \rho_t^f \|K *(\rho_t^f - \bar \rho_t)\|^2 - \int_{\mathcal{X}} \rho_t^f \, \delta_t \cdot K * (\rho_t^f - \bar \rho_t ) + \nu \int_{\mathcal{X}} \rho^f_t \, K * (\rho_t^f - \bar \rho_t )\cdot \nabla \log \frac{\rho_t^f}{\bar \rho_t} \\
		&\  - \frac{1}{2} \int_{\mathcal{X}^2} K(x-y) \cdot \Big( \mathcal{A}[\bar \rho_t](x) - \mathcal{A}[\bar \rho_t](y) \Big) \ud (\rho_t^f - \bar \rho_t )^{\otimes 2 }(x, y) \\
        &\ - \int \udiv \Big\{ g * (\rho^f_t - \bar \rho_t )(x)( \rho^f_t(x) f_t(x)   - \bar \rho_t(x) \bar f_t(x))\Big\} \ud x
    \end{align*}
	where we recall that the operator $\gA$ is defined in equation (\ref{eqn_operator_A}).
\end{lemma}
\begin{proof}
Recall that $K= - \nabla g$.  For simplicity, we write that $\rho_t = \rho_t^f$. Then 
	\[
	\begin{split} 
		&\ \frac{\ud }{\ud t } F(\rho_t, \bar \rho_t) = \frac{\ud }{\ud t } \frac{1 }{2} \int_{\mathcal{X}^2} g (x- y) \ud (\rho_t - \bar \rho_t)^{\otimes 2 } (x, y)  \\
		=&\ \int_\mathcal{X} g *(\rho_t - \bar \rho_t) (x) \big(\partial_t \rho_t (x) - \partial_t \bar \rho_t (x)  \Big) \ud x  \\
		=&\ - \int g * (\rho_t - \bar \rho_t ) (x) \udiv \Big\{  \rho_t(x) f_t(x)   - \bar \rho_t(x) \bar f_t(x)\Big\} \ud x \\
        =&\ \int \nabla g * (\rho_t - \bar \rho_t ) (x) \Big\{  \rho_t(x) f_t(x)   - \bar \rho_t(x) \bar f_t(x)\Big\} \ud x \\
        &\ - \int \udiv \Big\{ g * (\rho_t - \bar \rho_t )(x)( \rho_t(x) f_t(x)   - \bar \rho_t(x) \bar f_t(x))\Big\} \ud x
	\end{split} 
	\]
    We handle the first term on the R.H.S. just like the torus case and we can have the result.
\end{proof}

\subsection{Handling the Integral of the Divergence}
Given a vector field, the following lemma provides a sufficient condition for the volume integral of its divergence over $\X$ to be zero.
The idea is to construct a sequence of approximations to the integral of interest, each of which involves integration over a compact set. Consequently, Gauss's divergence theorem can be applied. We then utilize the dominant convergence theorem to exchange the order of the limit and integral.
\begin{lemma} \label{lemma_zero_divergence_integral_general}
    For a vector function $g: \sR^d \rightarrow\sR^d$ which satisfies 
    \begin{equation} \label{eqn_integrability_g}
        \int_{\sR^d} \ud\vx\ |\udiv\ g (\vx)| < \infty\quad \text{and}\quad \int_{\sR^d} \ud\vx\ \|g (\vx)\| <\infty,
    \end{equation}
    we have
    \begin{equation}
        \int_{\sR^d}\ud\vx\ \udiv\ g (\vx) = 0.
    \end{equation}
\end{lemma}
\begin{proof}
    Choose a cut-off function, indexed by $r > 1$, satisfying
    \begin{equation}
        \Phi_r(\vx) = \begin{cases}
            1, \quad& \text{if} \quad \|\vx\|\leq r, \\
            \frac{1}{2}(1+\cos(\pi \|\vx\| / r - 1)), & \text{if} \quad r<\|\vx\|\leq 2r,\\
            0, &\text{if} \quad 2r<\|\vx\|.
        \end{cases}
    \end{equation}
    We have $\|\nabla \Phi_r\|_{\gL^\infty} = O({1}/{r})$.
    Using the chain rule of divergence, we have that
    \begin{equation}
        \udiv_\vx(g \cdot \Phi_r) = \udiv_\vx(g)\cdot\Phi_r + g \cdot \nabla \Phi_r.
    \end{equation}
    We have $\int_{\sR^d}\ud \vx \udiv_\vx(g \Phi_r) (\vx) = 0$ for all $r$ and $\vx$, by noting $g \Phi_r(\vx) = 0$ for $\|\vx\| > 2r$ and using Gauss's divergence theorem on the $\vx$ variable.
    Using conditions (\ref{eqn_integrability_g}) and the dominated convergence theorem, we have
    \begin{align*}
        0 =&\ \lim_{r\rightarrow\infty} \int_{\X}\ud \vx\ [\udiv_\vx(g)\cdot\Phi_r] (\vx) + \lim_{r\rightarrow\infty} \int_{\X}\ud \vx\ [g \cdot \nabla \Phi_r] (\vx)\\
        =&\   \int_{\X}\ud \vx\ \lim_{r\rightarrow\infty} [\udiv_\vx(g)\cdot\Phi_r] (\vx) +  \int_{\X}\ud \vx\ \lim_{r\rightarrow\infty}[g \cdot \nabla \Phi_r] (\vx) \\
        =&\  \int_{\X}\ud \vx\ \udiv_\vx(g) (\vx),
    \end{align*}
    where in the last equality, we use $g \cdot \nabla \Phi_r (\vx) \leq \|g(\vx)\| \|\nabla \Phi_r (\vx)\| \rightarrow 0$ as $r\rightarrow\infty$.
\end{proof}
We now show that the divergence integrals in Lemmas \ref{lemma_TimeEvolKLMV} and \ref{lemma_ModuEnergyEvo} satisfy the requirements (\ref{eqn_integrability_g}), under the following regularity assumptions on the hypothesis velocity field $f$, initial distribution $\bar \rho_0$, and the ground truth solution $\bar \rho$.
\begin{assumption}\label{ass_regularity_force_field}
    % Recall that $f$ is the hypothesis force field.
    $f \in \Lip(\X)$ and there exists some constant $L$, such that for all $t \in [0, T]$ and $\vx \in \X$ $\|[\nabla (\udiv f)](t, \vx)\| \leq L$.
\end{assumption}
\begin{assumption}\label{ass_regularity_initial_distribution}
    The initial distribution $\bar \rho_0$ satisfies
    \begin{equation}
        \int_{\X}\ud \vx\ \bar \rho_0(\vx) (|\log \bar \rho_0(\vx)|+1)(\|\vx\|+1)^{\alpha+1}(\|\nabla\log\bar \rho_0(\vx)\|+1) <\infty
    \end{equation}
\end{assumption}
\begin{assumption}\label{ass_regularity_ground_truth}
    Suppose that the ground truth $\bar \rho_t \in \gL^\infty$ is sufficiently regular such that
    \begin{equation}
        |\log \bar \rho_t(\vx)| + \|\nabla \log \bar \rho_t(\vx)\| \leq L(1 + \|\vx\|)^\alpha\ \text{and}\ \|\bar f_t(\vx)\| + |\udiv \bar f_t(\vx)|\leq L(1+\|\vx\|)^\alpha
    \end{equation}
    holds for all $\vx \in\X$ and $t \in [0, T]$ with some constant $\alpha$ and $L$. Here we denote $\bar f_t = \gA [\rho_t]$.
\end{assumption}
The following estimations of regularity will be helpful. The proof is deferred to the end of this section.
\begin{lemma}\label{lemma_regularity_estimation}
    Under Assumption \ref{ass_regularity_force_field}, we have the following estimations
    \begin{align*}
        \|\vx_t\|^2 \leq&\ \exp(t(1+2L^2))(\|\vx_0\|^2 + 1)\\
        |\log \rho_t^f(\vx_t)| \leq&\ |\log \bar \rho_0(\vx_0)| + L t\\
        \|\nabla \log \rho_t^f(\vx_t)\|^2 \leq&\ \exp(t(1+2L^2))(\|\nabla \log  \bar \rho_0(\vx_0)\|^2 + 1).
    \end{align*}
\end{lemma}
We now show that the integrals of the divergence in Lemmas \ref{lemma_zero_divergence_integral_KL} and \ref{lemma_zero_divergence_integral_modulated_energy} are zero.
\begin{lemma} \label{lemma_zero_divergence_integral_KL}
    Under Assumptions \ref{ass_regularity_force_field} to \ref{ass_regularity_ground_truth}, we have
    \begin{equation}\label{eqn_divergence_integral_KL}
        \int_{\X} \udiv \Big(\rho_t  (f_t  \log \frac{\rho^f_t}{\bar \rho_t} - \bar f_t)\Big) = 0.
    \end{equation}
\end{lemma}
\begin{proof}
    To establish Lemma \ref{lemma_zero_divergence_integral_KL}, we need to show that all the terms inside the divergence of \eqref{eqn_divergence_integral_KL} satisfy the integrability requirements (\ref{eqn_integrability_g}) in Lemma \ref{lemma_zero_divergence_integral_general}, which are handled one by one in the following.

    \begin{itemize}[leftmargin=*]
        \item We handle the term $\rho_t^f \log \rho_t^f f_t$.
        \begin{itemize}
            \item To show that $\int_{\X}\ud \vx\ \| \rho_t^f \log \rho_t^f f_t (\vx)\| <\infty$
            \begin{align*}
                &\ \int_{\X}\ud \vx\ \| [\rho_t^f \log \rho_t^f f_t] (\vx)\| = \int_{\X}\ud \vx\ \rho_t^f(\vx) \| [ \log \rho_t^f f_t] (\vx)\|\\
                =&\ \int_{\X}\ud \vx\ \bar \rho_0(\vx_0) \cdot|\log \rho_t^f(\vx_t)|\cdot\|f_t(\vx_t)\| \\
                \leq &\ \int_{\X}\ud \vx\ \bar \rho_0(\vx_0) \cdot(|\log \bar \rho_0(\vx_0)| + Lt)\cdot\exp(t(1+2L^2))(\|\vx_0\| + 1) < \infty.
            \end{align*}
    
            \item To show that $\int_{\X}\ud \vx\ | \udiv \left( \rho_t^f \log \rho_t^f f_t\right) (\vx)| <\infty$
            \begin{align*}
                &\ \udiv \left( \rho_t^f \log \rho_t^f f_t\right) = \udiv f_t \cdot \rho_t^f \log \rho_t^f + f_t \cdot \nabla (\rho_t^f \log \rho_t^f)\\
                =&\ \udiv f_t \cdot \rho_t^f\cdot \log \rho_t^f + (f_t \cdot \nabla \rho_t^f) \cdot \log \rho_t^f + (f_t \cdot \nabla\log \rho_t^f) \cdot\rho_t^f\\
                =&\ \rho_t^f\left(\udiv f_t \cdot \log \rho_t^f + (f_t \cdot \nabla \log \rho_t^f) \cdot (1+\log \rho_t^f)\right)
            \end{align*}
            We now bound
            \begin{align*}
                &\ \int_{\X}\ud \vx\ \rho_t^f(\vx) | [\udiv f_t \cdot \log \rho_t^f] (\vx)|\\
                =&\ \int_{\X}\ud \vx\ \bar \rho_0(\vx_0) | [\udiv f_t \cdot \log \rho_t^f] (\vx_t)| \leq \int_{\X}\ud \vx\ \bar \rho_0(\vx_0)\cdot L \cdot (tL + |\log \bar \rho_0(\vx_0)|) <\infty.
            \end{align*}
            and 
            \begin{align*}
                &\ \int_{\X}\ud \vx\ \rho_t^f(\vx) | [(f_t \cdot \nabla \log \rho_t^f) \cdot (1+\log \rho_t^f)] (\vx)| \\
                = &\ \int_{\X}\ud \vx\ \bar \rho_0(\vx_0) | [(f_t \cdot \nabla \log \rho_t^f) \cdot (1+\log \rho_t^f)] (\vx_t)| \\
                \leq &\ \int_{\X}\ud \vx\ \bar \rho_0(\vx_0) L  (1+\|\vx_0\|) \exp(t(1+2L^2))(\|\nabla\log\bar \rho_0(\vx_0)\|+1)(1+Lt+|\log \bar \rho_0(\vx_0)|) <\infty.
            \end{align*}
        \end{itemize}
        \item We handle the term $\rho_t^f \log \bar \rho_t f_t$.
        \begin{itemize}
            \item To show that $\int_{\X}\ud \vx\ \| \rho_t^f \log \bar \rho_tf_t (\vx)\| <\infty$
                \begin{align*}
                    &\ \int_{\X}\ud \vx\ \rho_t^f\|\log \bar \rho_tf_t (\vx)\| = \int_{\X}\ud \vx\ \bar \rho_0\|[\log \bar \rho_tf_t] (\vx_t)\| \\
                    \leq&\ \int_{\X}\ud \vx\ \bar \rho_0(\vx_0)|\log \bar \rho_t(\vx_t)|\|f_t(\vx_t)\| \leq \int_{\X}\ud \vx\ \bar \rho_0(\vx_0)L^2(1+\|\vx_t\|)^{\alpha+1}<\infty.
                \end{align*}
            \item To show that $\int_{\X}\ud \vx\ | \udiv \left( \rho_t^f \log \bar \rho_tf_t\right) (\vx)| <\infty$
            \begin{align*}
                &\ \udiv \left( \rho_t^f \log \bar \rho_tf_t\right) = \udiv f_t \cdot \rho_t^f \log \bar \rho_t+ f_t \cdot \nabla (\rho_t^f \log \rho_t)\\
                =&\ \udiv f_t \cdot \rho_t^f\cdot \log \bar \rho_t+ (f_t \cdot \nabla \rho_t^f) \cdot \log \bar \rho_t+ (f_t \cdot \nabla\log \rho_t) \cdot\rho_t^f\\
                =&\ \rho_t^f\left(\udiv f_t \cdot \log \bar \rho_t+ (f_t \cdot \nabla \log \rho_t^f) \log \bar \rho_t+ f_t \cdot \nabla\log \rho_t\right)
            \end{align*}
            We now bound 
            \begin{align*}
                &\ \int_{\X}\ud \vx\ \rho_t^f(\vx)|\udiv f_t(\vx)| \cdot |\log \bar \rho_t(\vx)| = \int_{\X}\ud \vx\ \bar \rho_0(\vx_0) |\udiv f_t(\vx_t)|\cdot |\log \bar \rho_t(\vx_t)|\\
                \leq&\ \int_{\X}\ud \vx\ \bar \rho_0(\vx_0) L(1+\|\vx_t\|)(1+\|\vx_t\|)^\alpha < \infty.
            \end{align*}
            \begin{align*}
                &\ \int_{\X}\ud \vx\ \rho_t^f(\vx)| [(f_t \cdot \nabla \log \rho_t^f) \log \rho_t](\vx)| = \int_{\X}\ud \vx\ \bar \rho_0(\vx_0)| [(f_t \cdot \nabla \log \rho_t^f) \log \rho_t](\vx_t)|\\
                \leq &\ \int_{\X}\ud \vx\ \bar \rho_0(\vx_0)\|f_t(\vx_t)\|\| \nabla \log \rho_t^f(\vx_t)\| |\log \bar \rho_t(\vx_t)| \\
                \leq&\ \int_{\X}\ud \vx\ \exp(t(1+2L^2))(\|\nabla \log \bar \rho_0(\vx_0)\| + 1)L(1+\|\vx_t\|)(1+\|\vx_t\|)^\alpha < \infty.
            \end{align*}
            \begin{align*}
                &\ \int_{\X}\ud \vx\ \rho_t^f(\vx)| [f_t \cdot \nabla\log \rho_t](\vx)| = \int_{\X}\ud \vx\ \bar \rho_0(\vx_0)\|f_t(\vx_t)\|\|\nabla\log \bar \rho_t(\vx_t)\| \\
                \leq &\ \int_{\X}\ud \vx\ \bar \rho_0(\vx_0)\|f_t(\vx_t)\|\|\nabla\log \bar \rho_t(\vx_t)\| \leq \int_{\X}\ud \vx\ \bar \rho_0(\vx_0)L (1+\|\vx_t\|)(1+\|\vx_t\|)^\alpha < \infty.
            \end{align*}
        \end{itemize}
        \item We handle the term $\rho_t^f \bar f_t$.
        \begin{itemize}
            \item  To show that $\int_{\X}\ud \vx\ \| [\rho_t^f \bar f_t](\vx)\| <\infty$
                \begin{align*}
                    \int_{\X}\ud \vx\ \rho_t^f(\vx)\|\bar f_t(\vx)\| = \int_{\X}\ud \vx\ \bar \rho_0(\vx_0)\|\bar f_t(\vx_t)\| < \infty
                \end{align*}
            \item To show that $\int_{\X}\ud \vx\ | \udiv(\rho_t^f \bar f_t) (\vx)| <\infty$
            \begin{align*}
                \int_{\X}\udiv(\rho_t^f \bar f_t) =&\ \int_{\X} \nabla \rho_t^f \cdot \bar f_t + \rho_t^f  \udiv \bar f_t = \int_{\X} \rho_t^f\left(\nabla \log \rho_t^f \cdot \bar f_t + \udiv \bar f_t \right)\\
                =&\ \int_{\X} \ud \vx_0\ \bar \rho_0(\vx_0)\left(\nabla \log \rho_t^f(\vx_t) \cdot \bar f_t(\vx_t) + \udiv \bar f_t(\vx_t) \right) < \infty,
            \end{align*}
            using the polynomial growth assumption on the ground truth velocity field $\bar f_t$.
        \end{itemize}
    \end{itemize}
\end{proof}



We now focus on addressing the integral-of-divergence term in Lemma \ref{lemma_ModuEnergyEvo}.
The following result will be useful.
\begin{remark}
    Let $X_t^f$ be the flow map generated by the velocity field $f \in \Lip(\sR^d)$. 
    We have that $X_t^f\in \Lip(\sR^d)$ and that $\rho_t^f = X_t^f\sharp\bar \rho_0$ remains bounded for $t \in [0, T]$ if $\bar \rho_0$ is bounded on $\sR^d$. This can be established using the change-of-variable formula of the probability density function. 
\end{remark}

\begin{lemma} \label{lemma_zero_divergence_integral_modulated_energy}
    Under Assumptions \ref{ass_regularity_force_field} to \ref{ass_regularity_ground_truth}, we have
    \begin{equation}\label{eqn_divergence_integral_modulated_energy}
        \int_\X \ud \vx\ \udiv \Big\{ g * (\rho^f_t - \bar \rho_t )(\vx)( \rho^f_t(\vx) f_t(\vx)   - \bar \rho_t(\vx) \bar f_t(\vx))\Big\}  = 0.
    \end{equation}
\end{lemma}
\begin{proof}
    Denote $h = g * (\rho_t - \bar \rho_t )(\vx)( \rho_t(\vx) f_t(\vx)   - \bar \rho_t(\vx) \bar f_t(\vx))$.
    To show that $h \in \gL^1(\X)$, we can show that, after splitting into simple terms, every term from $h$ is in $\gL^1$. In the following, we show 
    \[
        g * \rho_t(\vx) \rho_t(\vx) f_t(\vx) \in \gL^1(\X).
    \] 
    Other terms can be proved similarly.
    First, we show that $g * \rho_t \in \gL^\infty(\X)$ if $\rho_t \in \gL^\infty(\X)$. For any constant $C$, we have
    \begin{align*}
        &\ g * \rho_t(\vx) = \int_\X g(\vx - \vy) \rho_t(\vy) \ud \vy \\
        =&\ \int_{\|\vx - \vy\|\leq C} g(\vx - \vy) \rho_t(\vy) \ud \vy  + \int_{\|\vx - \vy\|> C} g(\vx - \vy) \rho_t(\vy) \ud \vy\\
        \leq&\ \|\rho_t\|_{\gL^\infty(\X)} \int_{\|\vx - \vy\|\leq C} \|\vx - \vy\|^{2-d} \ud \vy + C^{2-d}\int_{\|\vx - \vy\|> C}\rho_t(\vy) \ud \vy \\
        \leq&\ \|\rho_t\|_{\gL^\infty(\X)} C^2 + C^{2-d},
    \end{align*}
    where in the last inequality, we use 
    \begin{equation*}
        \int_{\|\vx - \vy\|\leq C} \|\vx - \vy\|^{2-d} \ud \vy = \int_{\|\vy\|\leq C} \|\vy\|^{2-d} \ud \vy \leq \int_{0\leq r\leq C}r^{2-d} \ud r \int J_\theta \ud_\theta \leq \int_{0\leq r\leq C} \ud r\ r \leq C^2.
    \end{equation*}
    Here $J_\theta$ denotes the determinant of the Jacobian obtained from changing to the polar coordinate, which is bounded by $r^{d-1}$.
    We hence obtain
    \begin{equation*}
        \int_\X \ud \vx\ \|g * \rho_t(\vx) \rho_t(\vx) f_t(\vx)\| \leq C' \int_\X \ud \vx\ \rho_t(\vx) \|f_t(\vx)\| = C'\int_\X \ud \vx_0\ \bar \rho_0(\vx_0) \|f_t(\vx_t)\| < \infty,
    \end{equation*}
    where we use $f_t \in \Lip(\X)$ and the estimation in Lemma \ref{lemma_regularity_estimation}.

    Similarly, to show that $\udiv(h) \in \gL^1(\X)$, we can show that, after splitting into simple terms, every term from $\udiv(h)$ is in $\gL^1$.
    In the following, we show that 
    \[
        \nabla g * \rho_t(\vx) \rho_t(\vx) f_t(\vx) \in \gL^1(\X)\ \text{and}\ g * \rho_t(\vx) \nabla \rho_t(\vx) \cdot f_t(\vx) \in \gL^1(\X).
    \] 
    Other terms can be proved similarly.

    To show that $\nabla g * \rho_t(\vx) \rho_t(\vx) f_t(\vx) \in \gL^1(\X)$, we first show that $\nabla g * \rho_t(\vx) \in \gL^\infty(\X)$ for $\rho_t \in \gL^\infty(\X)$. We can then apply the same argument as above to establish the absolute integrability of the whole term.
    \begin{align*}
        &\ \|\nabla g * \rho_t(\vx)\| \leq \int_\X \|\nabla g(\vx - \vy)\| \rho_t(\vy) \ud \vy \\
        =&\ \int_{\|\vx - \vy\|\leq C} \|\nabla g(\vx - \vy)\| \rho_t(\vy) \ud \vy  + \int_{\|\vx - \vy\|> C} \|\nabla g(\vx - \vy)\| \rho_t(\vy) \ud \vy\\
        \leq&\ \|\rho_t\|_{\gL^\infty(\X)} \int_{\|\vx - \vy\|\leq C} \|\vx - \vy\|^{1-d} \ud \vy + C^{1-d}\int_{\|\vx - \vy\|> C}\rho_t(\vy) \ud \vy \\
        \leq&\ \|\rho_t\|_{\gL^\infty(\X)} C + C^{1-d}.
    \end{align*}

    To show that $g * \rho_t(\vx) \nabla \rho_t(\vx) \cdot f_t(\vx) \in \gL^1(\X)$, we use the fact that $g * \rho_t \in \gL^\infty(\X)$ and that 
    \begin{equation}
       \int_\X \ud \vx\ \nabla \rho_t(\vx) \cdot f_t(\vx) = \int_\X \ud \vx_0\ \bar \rho_0(\vx_0) \nabla \log \rho_t(\vx_t) \cdot f_t(\vx_t).
    \end{equation}
    Using the estimation in Lemma \ref{lemma_regularity_estimation} and that $f_t \in \Lip(\X)$, we obtain the result.
\end{proof}


\begin{proof}[Proof of Lemma \ref{lemma_regularity_estimation}]
    \begin{equation}
        \frac{\ud }{\ud t}\|\vx_t\|^2 \leq \|\vx_t\|^2 + \|\bar f_t(\vx_t)\|^2 \leq \|\vx_t\|^2 + 2L^2(1+\|\vx_t\|^2) = (1 + 2L^2)\|\vx_t\|^2 + 2L^2.
    \end{equation}
    Using Gr\"onwall's inequality, we have
    \begin{equation}
        \|\vx_t\|^2 \leq \exp(t(1+2L^2))(\|\vx_0\|^2 + 2L^2/(1 + 2L^2)) \leq \exp(t(1+2L^2))(\|\vx_0\|^2 + 1).
    \end{equation}
    % Denote $B_0 = \exp(T(1+2L^2))$.

    We have
    \begin{align} \label{eqn_dynamics_of_log_prob}
        \frac{\ud }{\ud t}\log \rho_t^f(\vx_t) = - \udiv \bar f_t(\vx_t)
    \end{align}
    
    We have
    \begin{align}
        \frac{\ud }{\ud t}\nabla \log \rho_t^f(\vx_t) = - \nabla  \left(\udiv \bar f_t(\vx_t) \right) -   \left(\gJ_{\bar f_{t}}(\vx_t)\right)^\top \nabla \log \rho_{t}^{f}(\vx_t)
    \end{align}
    \begin{align}
        \frac{\ud }{\ud t}\|\nabla \log \rho_t^f(\vx_t)\|^2 \leq&\ \|\nabla \log \rho_t^f(\vx_t)\|^2 + 2\|\nabla  \left(\udiv \bar f_t(\vx_t) \right)\|^2 + 2\|\left(\gJ_{\bar f_{t}}(\vx_t)\right)^\top \nabla \log \rho_{t}^{f}(\vx_t)\|^2\\
        \leq &\ \|\nabla \log \rho_t^f(\vx_t)\|^2 (1+2\|\gJ_{\bar f_{t}}(\vx_t)\|^2) + 2\|\nabla  \left(\udiv \bar f_t(\vx_t) \right)\|^2 \\
        \leq &\ \|\nabla \log \rho_t^f(\vx_t)\|^2 (1+2L^2) + 2L^2
    \end{align}
    Using Gr\"onwall's inequality, we have
    \begin{equation}
        \|\nabla \log \rho_t^f(\vx_t)\|^2 \leq  \exp(t(1+2L^2))(\|\nabla \log \bar \rho_0(\vx_0)\|^2 + 1).
    \end{equation}
\end{proof}

%\clearpage
%\input{math_command_information.tex}
\end{document}
