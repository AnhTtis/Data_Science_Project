\section{Analysis} \label{section_analysis}
In this section, we focus on the torus case, i.e. $\X = \Pi^d$ is a box with the periodic boundary condition.
This is a typical setting considered in the literature as the universal function approximation of NNs only holds over a compact set. Moreover, the boundary integral resulting from integration by parts vanishes in this setting, making it amenable for analysis purposes.
For completeness, we provide a discussion on the unbounded case, i.e. $\X = \sR^d$ in the Appendix \ref{appendix_unbounded}, which requires additional regularity assumptions.
% \red{Mention that we focus on the torus case.}
% We start by defining the modulated (interaction) energy and modulated free energy for two probability densities $\rho, \bar \rho$.  
% Again $\mathcal{X}$ denotes the underlying space \red{which can be the whole space $\mathbb{R}^d$} or the torus $\Pi^d $ which can be identified with $[-L, L]^d$ with the periodic boundary condition and $L>0$. 
Given the MVE (\ref{eqn_MVE}), if $K$ is bounded, it is sufficient to choose the Lyapunov functional $L(\rho_t^f, \bar \rho_t)$ as the KL divergence (please see Theorem \ref{theorem_bounded_K} in the appendix). But for the singular Coulomb kernel, we need also to consider the modulated energy as in \citep{serfaty2020mean}
\begin{equation}
    \text{(Modulated Energy)}\quad 
	\label{DefModEnergy}
	F(\rho, \bar \rho) \defi \frac 1 2 \int_{\mathcal{X}^2} g(x- y) \ud (\rho - \bar \rho)(x) \ud (\rho - \bar \rho )(y), 
\end{equation}
where $g$ is the fundamental solution to the Laplacian equation in $\mathbb{R}^d$, i.e. $- \Delta g = \delta_0$, and the Coulomb interaction reads $K= -\nabla g$ (see its closed form expression in equation (\ref{eqn_coulomb_interaction})). If we are only interested in the deterministic  dynamics with Coulomb interactions, i.e. $\nu =0$ in equation (\ref{eqn_MVE}), it suffices to choose $L(\rho_t^f, \bar \rho) $ as  $F(\rho_t^f, \bar \rho_t)$ (please see Theorem \ref{ThmCoul}). But if we consider the system with  Coulomb interactions and  diffusions, i.e. $\nu >0$, we shall combine the KL divergence and the modulated energy to form the modulated free energy as in \cite{bresch2019modulated}, which reads 
\begin{equation}
    \text{(Modulated Free Energy)}\quad 
	\label{DefModFree}
	E(\rho, \bar \rho) \defi \nu \KL (\rho, \bar \rho) +  F(\rho, \bar \rho). 
\end{equation}
This definition agrees with the physical meaning that ``Free Energy = Temperature $\times $ Entropy + Energy", and we note that the temperature is proportional to the diffusion coefficient $\nu$. We remark also for two probability densities $\rho$ and $\bar \rho$, $F(\rho, \bar \rho) \geq 0$ since by looking in the Fourier domain $F(\rho, \bar \rho) = \int \hat g(\xi) |\widehat{\rho - \bar \rho}(\xi)|^2  \ud \xi \geq 0$ as $\hat g(\xi) \geq 0$. Moreover, $F(\rho, \bar \rho)$ can be regarded as a negative Sobolev norm for $\rho- \bar \rho$, which metricizes weak convergence.\\
To obtain our main stability estimate, we first obtain the time evolution of the KL divergence.  
%%%% Evolution of the KL divergence 
\begin{lemma}[Time Evolution of the KL divergence] \label{TimeEvolKLMV} 
	Given the hypothesis velocity field $f=f(t, x) \in C^1_{t, x}$. Assume that $(\rho_t^f)_{t \in [0, T]}$ and $(\bar \rho_t)_{t \in [0, T]}$ are classical solutions to equation (\ref{eqn_CE}) and equation (\ref{eqn_MVE_CE}) respectively.  It holds that (recall the definition of $\delta_t$ in equation (\ref{eqn_perturbation})) 
	\[
	\frac{\ud }{\ud t} \int_{\mathcal{X}} \rho^f_t \log \frac{\rho^f_t }{\bar \rho_t} = - \nu \int_{ \X}\rho^f_t |\nabla \log \frac{\rho^f_t}{\bar \rho_t}|^2 +  \int_{\X} \rho^f_t K * (\rho^f_t - \bar \rho_t ) \cdot \nabla  \log  \frac{\rho^f_t }{\bar \rho_t} +  \int_{\X} \rho^f_t \delta_t \cdot \nabla \log \frac{\rho^f_t }{\bar \rho_t }, 
	\]
	where $\X$ is the tours $\Pi^d$. All the integrands are evaluated at $\vx$. 
\end{lemma}

We refer the proof of this lemma and all other lemmas and theorems in this section to the appendix \ref{detailed proof}. We remark that to have the existence of classical solution $(\bar \rho_t)_{t \in [0, T]}$, we definitely need the regularity assumptions on $-\nabla V$ and on $K$. But the linear term $- \nabla V $ will not contribute to the evolution of the relative entropy. See \citep{jabin2018quantitative} for detailed discussions. \\
%%%% Evolution of the Modulated Energy 
Similarly, we have the time evolution of the modulated energy as follows.
\begin{lemma}[Time evolution of the modulated energy] \label{ModuEnergyEvo} Under the same assumptions as in Lemma \ref{TimeEvolKLMV}, given the diffusion coefficient $\nu \geq 0$, it holds that (recall the definition of $\delta_t$ in equation (\ref{eqn_perturbation})) 
	\[
	\begin{split}
		\frac{\ud }{\ud t } F(\rho_t^f, \bar \rho_t)&  =  - \int_{\mathcal{X}} \rho_t^f \|K *(\rho_t^f - \bar \rho_t)\|^2 - \int_{\mathcal{X}} \rho_t^f \, \delta_t \cdot K * (\rho_t^f - \bar \rho_t ) + \nu \int_{\mathcal{X}} \rho^f_t \, K * (\rho_t^f - \bar \rho_t )\cdot \nabla \log \frac{\rho_t^f}{\bar \rho_t} \\
		&  - \frac{1}{2} \int_{\mathcal{X}^2} K(x-y) \cdot \Big( \mathcal{A}[\bar \rho_t](x) - \mathcal{A}[\bar \rho_t](y) \Big) \ud (\rho_t^f - \bar \rho_t )^{\otimes 2 }(x, y) \\
	\end{split}
	\]
	where we recall that the operator $\gA$ is defined in equation (\ref{eqn_operator_A}).
\end{lemma}






%%%%%%% The 2D Navier-Stokes case  
By  Lemma \ref{TimeEvolKLMV} and careful analysis, in particular by rewriting the Biot-Savart law in the divergence of a bounded matrix-valued function (\ref{eqn_K_as_divergence}), we obtain the following estimate for the 2D NSE.

\begin{theorem}[Stability estimate of the 2D NSE] \label{NSMainEstimate}	
	% Consider the 2D NSE in the vorticity formulation (\ref{eqn_MVE}) with $V=0$. 
    Notice that when $K$ is the Biot-Savart kernel, $\udiv K =0$. Assume that the initial data $\bar \rho_0 \in C^3(\Pi^d)$ and there exists $c>1$ such that $\frac 1 c \leq \bar \rho_0 \leq c$.  Assume further the hypothesis velocity field $f(t, x) \in C^1_{t, x}$. Then it holds that 
	\[
	\sup_{t \in [0, T]} \int_{\Pi^d} \rho_t^f \log \frac{\rho_t^f}{\bar \rho_t} \ud x \leq \frac{e^C}{\nu}  R(f), 
	\]
	where $C = \int_0^\infty     M(t) \ud t < \infty$ with 
	$M(t) \defi \|\nabla \log \bar \rho_t\|_{L^\infty}^2/2\nu + 2\Big\| {\nabla^2 \bar \rho_t}/{\bar \rho_t} \Big\|_{L^\infty}$. 
	
\end{theorem} 
We remark that given $\bar \rho_0$ is smooth enough and fully supported on $\X$, one can propagate the regularity to finally show the finiteness of $C$.
See detailed computations as in \cite{guillin2021uniform}. 
We give the complete proof in the appendix \ref{detailed proof}. This theorem tells us that as long as $R(f)$ is small, the KL divergence between $\rho_t^f$ and $\bar \rho_t$ is small and the control is uniform in time $t \in [0, T]$ for any $T$. Moreover, we highlight that $C$ is independent of $T$, and our result on the NSE is significantly better than the average-in-time and exponential-in-$T$ results from \citep{deerror}.\\
% We state this theorem on torus for simplicity and \red{one may expect similar result on $\mathbb{R}^d$}. \red{Also the $C^\infty$ condition can be relaxed for instance to $C^2$}. 
%%%%%  Remark on The case with bounded interactions $K \in L^\infty
%%%% Leave in the appendix  
%%%% The deterministic case with Coulomb 
%%%% Lemma of the evolution of the modulated free energy 
To treat the MVE (\ref{eqn_MVE}) with Coulomb interactions, we exploit the time evolution of the modulated free energy $E(\rho_t^f, \bar \rho_t)$. Indeed, combining Lemma \ref{TimeEvolKLMV} and Lemma \ref{ModuEnergyEvo}, we arrive at the following identity.  
\begin{lemma}[Time evolution of the modulated free energy]\label{TimeEvoMFE} 
	Under the same assumptions as in Lemma \ref{TimeEvolKLMV}, one has (recall the definitions of $\delta_t$ and $\gA$ in (\ref{eqn_perturbation}) and (\ref{eqn_operator_A}) respectively) 
	\[
	\begin{split}
		\frac{\ud }{\ud t } E(\rho_t^f, \bar \rho_t)  & = -\int_{\mathcal{X}} \rho_t^f  \Big|K * (\rho_t^f - \bar \rho_t) - \nu \nabla \log \frac{\rho_t^f }{\bar \rho_t}\Big|^2 - \int_{\mathcal{X}} \rho_t^f \, \delta_t \cdot \Big( K * (\rho_t^f - \bar \rho_t ) - \nu \nabla \log \frac{\rho_t^f}{\bar \rho_t }\Big) \\
		&  - \frac{1}{2} \int_{\mathcal{X}^2} K(x-y) \cdot \Big( \mathcal{A}[\bar \rho_t](x) - \mathcal{A}[\bar \rho_t](y) \Big) \ud (\rho_t^f - \bar \rho_t )^{\otimes 2 }(x, y).
	\end{split}
	\]
	% where we recall that the operator $\gA$ is defined in \eqref{eqn_operator_A}.
\end{lemma}
\vspace{-1mm}
%%%%%%%%%%%%%%Main theorem of the Coulomb case 
Inspired by the mean-field convergence results as in \cite{serfaty2020mean} and \cite{bresch2019modulated}, we finally can control the growth of $E(\rho_t^f, \bar \rho_t)$ in the case when $\nu >0$,  and $F(\rho_t^f, \bar \rho_t)$ in the case when $\nu =0$. Note also that $E(\rho_t^f, \bar \rho_t )$ can also control the KL divergence when $\nu >0$. 
\begin{theorem} [Stability estimate of MVE with Coulomb interactions] \label{ThmCoul}
	Assume that for $t \in [0, T]$, the underlying velocity field $\mathcal{A}[\bar \rho_t](x)$ is Lipschitz  in $x$ and
	$\sup_{t \in [0, T]} \|\nabla \mathcal{A}[\bar \rho_t](\cdot)\|_{L^\infty} = C_1  < \infty.$ 
	Then there exists $C>0$ such that 
	\[
	\sup_{t \in [0, T]} \nu\,  \KL (\rho_t^f, \bar \rho_t) \leq \sup_{t \in [0, T]} E(\rho_t^f, \bar \rho_t) \leq \exp(C C_1 T) R(f).  
	\]
	In the deterministic case when $\nu =0$, under the same assumptions, it holds that 
	\[
	\sup_{t \in [0, T]} F( \rho_t^f, \bar \rho_t) \leq \exp(CC_1T ) R(f). 
	\]
\end{theorem}
\vspace{-1mm}
Recall the definition of the operator $\gA$ in \eqref{eqn_operator_A}. Given that $\mathcal{X}= \Pi^d$, and $\bar \rho_0$ is smooth enough and bounded from below, one can propagate regularity to obtain the Lipschitz condition for $\mathcal{A}[\bar \rho_t]$. See the proof and the discussion on the Lipschitz assumptions on $\mathcal{A}[\bar \rho_t](\cdot)$ in the appendix \ref{detailed proof}. 
\paragraph{Approximation Error of Neural Network}
Theorems \ref{NSMainEstimate} and \ref{ThmCoul} provide the error estimation guarantee for the proposed \EINN\ loss (\ref{eqn_self_consistency_potential}).
Suppose that we parameterize the velocity field $f=f_\theta$ with an NN parameterized by $\theta$, as we did in Section \ref{section_NN_parameterization} and let $\tilde f$ be the output of an optimization procedure when $R(f_\theta)$ is used as objective.
In order the explicitly quantify the mismatch between $\rho^{\tilde f}_t$ and $\bar \rho_t$, we need to quantify two errors: (i) Approximation error, reflecting how well the ground truth solution can be approximated among the NN function class of choice; (ii) Optimization error, involving minimization of a highly nonlinear non-convex objective. 
In the following, we show that for a function class $\gF$ with sufficient capacity, there exists at least one element $\hat f\in\gF$ that can reduce the loss function $R(\hat f)$ as much as desired.
We will not discuss how to identify such an element in the function class $\gF$ as it is independent of our research and remains possibly the largest open problem in modern AI research.
To establish our result, we make the following assumptions.
\begin{assumption} \label{ass_appendix_initial}
	$\rho_0$ is sufficiently regular, such that $\nabla \log \rho_0 \in \gL^\infty(\X)$ and $\bar f_t  = \mathcal{A}[\bar \rho_t] \in W^{2,\infty}(\X)$. $\nabla V$ is Lipschitz continuous. Here $W^{2,\infty}(\X)$ stands for the Sobolev norm of order $(2, \infty)$ over $\X$.
\end{assumption}
\vspace{-1mm}
We here again need to propagate the regularity for $f_t$ at least for a time interval $[0, T]$. It is easy to do so for the torus case, but for the unbounded domain, there are some technical issues to be  overcome. Similar assumptions are also needed in some mathematical works for instance in \cite{jabin2018quantitative}. 
% [cite some papers]
% \begin{remark}
%     For the torus case, if $\rho_0$ is bounded from above and below, and chosen to be in $C^3$, we can propagate the regularity of $\rho_0$ to $\bar \rho_t$ for $t \in [0, T]$. We can then obtain $\bar f_t \in \gC^3(\X)$.
% \end{remark}
We also make the following assumption on the capacity of the function class $\gF$, which is satisfied for example by NNs with tanh activation function \citep{DERYCK2021732}.
\begin{assumption} \label{ass_appendix_approximation}
	The function class is sufficiently large, such that there exists $\hat f \in \gF$ satisfying $\hat f_t \in \gC^3(\X)$ and $\|\hat f_t - \bar f_t\|_{W^{2, \infty}(\X)} \leq \epsilon$ for all $t\in[0, T]$.
\end{assumption}
\begin{theorem} \label{thm_approximation_error_NN}
	Consider the case where the domain is the torus. 
	Suppose that Assumptions \ref{ass_appendix_initial} and \ref{ass_appendix_approximation} hold. 
    For both the Coulomb and the Biot-Savart cases, there exists $\hat f\in\gF$ such that $R(\hat f) \leq C(T)\cdot(\epsilon \cdot\ln 1/\epsilon)^2$, where $C(T)$ is some constant independent of $\epsilon$. Here $R$ is the \EINN\ loss (\ref{eqn_self_consistency_potential}).
\end{theorem}
% \red{Maybe mentioning the difficulty in proving this theorem to promote the novelty of the result.}
% \red{Can we derive the results for unbounded domain?}
% \red{This is just the result for the Coulomb case. We should also prove the NSE case.}
The major difficulty to overcome is the lack of Lipschitz continuity due to the singular interaction. We successfully address this challenge by establishing that the contribution of the singular region $(\|\vx\|\leq\epsilon)$ to $R(\hat f)$ can be bounded by $O((\epsilon \log \frac{1}{\epsilon})^2)$.
Please see the detailed proof in Appendix \ref{appendix_approximation_error_NN}.



%1. Previous analysis provides control of the solution quality by $R(f)$
%2. The next question is how small $R(f)$ can be
%3. Two errors: approximation error and optimization error. The latter is an open question. In this section, we focus on the approximation error.
%4. Conclusions on NN approximation in Sobolev norm
%5. List assumptions
%6. describe conclusion.

% Discuss the long-time guarantee (without exponential in T, under additional assumptions)

% Periodic solution to Taylor-Green vortex

% Revise the discussion on "expected"


