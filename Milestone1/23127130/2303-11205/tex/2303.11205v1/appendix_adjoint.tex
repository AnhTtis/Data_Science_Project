\section{Adjoint Method} \label{appendix_adjoint_method}
Consider the ODE system
\begin{align*}
	\dot s(t) =&\ \psi(s(t), t, \theta) \\
	s(0) =&\ s_0,
\end{align*}
and the objective loss
\begin{equation}
	\ell(\theta) = \int_0^T g(s(t), t, \theta) \ud t.
\end{equation}
The following proposition computes the gradient of $\ell$ w.r.t. $\theta$.
We omit the parameters of the functions for succinctness. We note that all the functions in the integrands should be evaluated at the corresponding time stamp $t$, e.g. $b^\top \frac{\partial h}{\partial \theta}\ud t$ abbreviates for $b(t)^\top \frac{\partial}{\partial \theta}h(\xi(t), x(t), t, \theta)\ud t$.
\begin{proposition}
	\begin{equation}
		\frac{\ud \ell}{\ud \theta} = \int_{0}^T a^\top \frac{\partial\psi}{\partial\theta} + \frac{\partial g}{\partial\theta}\ud t.
	\end{equation}
	where $a(t)$ is solution to the following final value problems
	\begin{equation}
		\dot a^\top + a^\top \frac{\partial\psi}{\partial s} + \frac{\partial g}{\partial s} = 0, a(T) = 0, 
	\end{equation}
\end{proposition}
\begin{proof}
	Let us define the Lagrange multiplier function (or the adjoint state) $a(t)$ dual to $s(t)$.
	Moreover, let $L$ be an augmented loss function of the form
	\begin{equation}
		L = \ell - \int_0^T a^\top(\dot s - \psi) \ud t.
	\end{equation}
	Since we have $\dot s(t) = \psi(s(t), t, \theta)$ by construction, the integral term in $L$ is always null and $a$ can be freely assigned while maintaining $\ud L/\ud \theta = \ud \ell/\ud \theta$.
	Using integral by part, we have
	\begin{equation}
		\int_0^T a^\top\dot s\ \ud t = a(t)^\top s(t)\vert_0^T - \int_0^T s^\top \dot a\ \ud t.
	\end{equation}
	We obtain
	\begin{align}
		L = - a(t)^\top s(t)\vert_0^T + \int_0^T \dot a^\top s + a^\top \psi + g\ \ud t.
	\end{align}
	
	Now we compute the gradient of $L$ w.r.t. $\theta$ as
	\begin{equation*}
		\frac{\ud \ell}{\ud \theta} =  \frac{\ud L}{\ud \theta} = - a(T)^\top\frac{\ud x(T)}{\ud \theta}  + \int_0^T \dot a^\top \frac{\ud s}{\ud \theta} + a^\top \left(\frac{\partial\psi}{\partial\theta} + \frac{\partial\psi}{\partial s} \frac{\ud s}{\ud \theta} \right) \ud t
		+ \int_0^T \frac{\partial g}{\partial s} \frac{\ud s}{\ud \theta} +  \frac{\partial g}{\partial\theta}\ud t,
	\end{equation*}
	which by rearranging terms yields to
	\begin{align*}
		\frac{\ud \ell}{\ud \theta} = \frac{\ud L}{\ud \theta} = - a(T)^\top\frac{\ud x(T)}{\ud \theta} + \int_{0}^T a^\top \frac{\partial \psi}{\partial \theta} +  \frac{\partial g}{\partial \theta}\ud t
		+ \int_0^T \left(\dot a^\top + a^\top \frac{\partial \psi}{\partial s} +  \frac{\partial g}{\partial s}\right)\frac{\ud s}{\ud \theta} \ud t.
	\end{align*}
	Now by taking $a$ satisfying the \emph{final} value problems
	\begin{equation}
		\dot a^\top + a^\top \frac{\partial \psi}{\partial s} + \frac{\partial g}{\partial s} = 0, a(T) = 0, 
	\end{equation}
	we derive the result
	\begin{equation}
		\frac{\ud \ell}{\ud \theta} = \int_{0}^T a^\top \frac{\partial \psi}{\partial \theta} + \frac{\partial g}{\partial \theta}\ud t.
	\end{equation}
\end{proof}