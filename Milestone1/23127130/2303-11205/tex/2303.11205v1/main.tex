%\documentclass[anon,12pt]{colt2023} % Anonymized submission
%\documentclass[final,12pt]{colt2023} % Include author names

%\usepackage{showkeys}

% The following packages will be automatically loaded:
% amsmath, amssymb, natbib, graphicx, url, algorithm2e
\documentclass{article}
\usepackage{PRIMEarxiv}


\title{Entropy-dissipation Informed Neural Network\\ for McKean-Vlasov Type PDEs}
\usepackage{times}
% Use \Name{Author Name} to specify the name.
% If the surname contains spaces, enclose the surname
% in braces, e.g. \Name{John {Smith Jones}} similarly
% if the name has a "von" part, e.g \Name{Jane {de Winter}}.
% If the first letter in the forenames is a diacritic
% enclose the diacritic in braces, e.g. \Name{{\'E}louise Smith}

% Two authors with the same address
% \coltauthor{\Name{Author Name1} \Email{abc@sample.com}\and
%  \Name{Author Name2} \Email{xyz@sample.com}\\
%  \addr Address}

% Three or more authors with the same address:
% \coltauthor{\Name{Author Name1} \Email{an1@sample.com}\\
%  \Name{Author Name2} \Email{an2@sample.com}\\
%  \Name{Author Name3} \Email{an3@sample.com}\\
%  \addr Address}

% Authors with different addresses:
\author{
	Zebang Shen\thanks{Authors are listed in an alphabetic order.}\\
	ETH Z\"urich \\
	\texttt{zebang.shen@inf.ethz.ch} \\
	%% examples of more authors
	\And
	Zhenfu Wang \\
	Peking University \\
	\texttt{zwang@bicmr.pku.edu.cn}
	%% \AND
	%% Coauthor \\
	%% Affiliation \\
	%% Address \\
	%% \texttt{email} \\
	%% \And
	%% Coauthor \\
	%% Affiliation \\
	%% Address \\
	%% \texttt{email} \\
	%% \And
	%% Coauthor \\
	%% Affiliation \\
	%% Address \\
	%% \texttt{email} \\
}
\usepackage{amssymb,amsmath,amsthm}
\newtheorem{theorem}{Theorem}
\newtheorem{proposition}{Proposition}
\newtheorem{remark}{Remark}
\newtheorem{lemma}{Lemma}

\newcommand{\bbox}{\text{bbox}}
\newcommand{\alphapck}{\alpha_\bbox}
\newcommand{\kcycle}{\text{k-CyPCK}}
\newcommand{\cycle}{\text{-CyPCK}}

\newcommand{\I}{\mathbf{I}}
\newcommand{\Ia}{\I^\text{a}}
\newcommand{\Ib}{\I^\text{b}}
\newcommand{\Iatob}{\I^\text{a $\rightarrow$ b}}
\newcommand{\F}{\mathbf{F}}
\newcommand{\Fa}{\F^\text{a}}
\newcommand{\Fb}{\F^\text{b}}
\newcommand{\f}{\mathbf{f}}
\newcommand{\fa}{\f^\text{a}}
\newcommand{\fb}{\f^\text{b}}
\newcommand{\p}{\mathbf{p}}
\newcommand{\pa}{\p^\text{a}}
\newcommand{\pb}{\p^\text{b}}
\newcommand{\A}{\boldsymbol{\Phi}_\text{align}}
\newcommand{\G}{\mathbf{G}}
\newcommand{\C}{\mathbf{C}}
\newcommand{\Ca}{\C^\text{a}}
\newcommand{\Cb}{\C^\text{b}}
\newcommand{\cc}{\mathbf{c}}
\newcommand{\cca}{\cc^\text{a}}
\newcommand{\ccb}{\cc^\text{b}}
\newcommand{\Irec}{\I_\text{Recon}}
\newcommand{\M}{\mathbf{M}}
\newcommand{\Mrec}{\M_\text{Recon}}
\newcommand{\loss}{\mathcal{L}}
\newcommand{\T}{\mathcal{T}}
\newcommand{\W}{\mathcal{W}}
\newcommand{\Id}{\mathcal{I}}

\newcommand{\Law}{\mathrm{Law}}
\usepackage{color}
\newcommand{\red}[1]{\textcolor{red}{#1}}
\newcommand{\ud}{\mathrm{d}}
\newcommand{\X}{\mathcal{X}}
%\newcommand{\ud}{\,\mathrm{d}}
\newcommand{\udiv}{\, \mathrm{div}}
\newcommand{\Uniform}{\mathrm{Uniform}}
\usepackage{makecell}
\usepackage{enumitem}
\let\KL\relax
\newcommand{\KL}{\mathbf{KL}}
\usepackage{natbib}
\usepackage{graphicx}
\usepackage{hyperref}       % hyperlinks


\begin{document}

\maketitle

\begin{abstract}%
  We extend the concept of {self-consistency} for the Fokker-Planck equation (FPE) \citep{shen22a} to the more general McKean-Vlasov equation (MVE).
  While FPE describes the macroscopic behavior of particles under drift and diffusion, MVE accounts for the additional inter-particle interactions, which are often highly singular in physical systems.
  Two important examples considered in this paper are the MVE with Coulomb interactions and the vorticity formulation of the 2D Navier-Stokes equation.
  We show that a generalized self-consistency potential controls the KL-divergence between a hypothesis solution to the ground truth, through entropy dissipation.
  Built on this result, we propose to solve the MVEs by minimizing this potential function, while utilizing the neural networks for function approximation.
  We validate the empirical performance of our approach by comparing with state-of-the-art NN-based PDE solvers on several example problems. 
\end{abstract}

%\begin{keywords}%
%  Entropy dissipation, McKean-Vlasov equation, 2D Navier-Stokes equation%
%\end{keywords}
% Importance and appeal of children's drawings
Children's depictions of the human figure are highly expressive and varied.
As one of the very first subjects children attempt to draw, the representation begins as an almost unintelligible cloud of scribbles. 
As the child grows, their representation of the human figure becomes more developed and is extended to graphically represent many different types of characters: people, animals, and even personified objects (see Figure 1).

Who among us has not wished, either as a child or as an adult, to see such figures come to life and move around on the page?
Sadly, while it is relatively fast to produce a single drawing, creating the sequence of images necessary for animation is a much more tedious endeavor, requiring discipline, skill, patience, and sometimes complicated software.
As a result, most of these figures remain static upon the page.

% We built a system to animate them.
Inspired by the importance and appeal of the drawn human figure, we design and build a system to automatically animate it given an in-the-wild photograph of a child's drawing. 
Our system is fast, intuitive, and robust to much of the variation present in these types of drawings, making it well-suited to allow our target audience--children--to see their own characters coming to life.
The system is comprised of four stages: figure detection, segmentation masking, pose estimation/rigging, and animation. 
We describe each stage and identify common causes of failure in each. 
For object detection and pose estimation, we make use of existing computer vision models designed to detect human figures and joints in photographs; we fine-tune these models for use with children's drawings.
For segmentation, we present a straightforward, image processing-based method that, for animation purposes, is more useful and accurate than segmentation masks obtained from a fine-tuned object detection model.
During the animation step, we take advantage of the \textit{twisted perspective} commonly seen in children’s drawings to retarget motion capture data onto the character in a novel and appealing way.

% We use existing machine learning models. However, given the wide domain gap it's not clear how much fine-tuning data was needed. So we ran some experiments to find out and report it.
While our system leverages existing models and techniques, most are not directly applicable to the task due to the many differences between photographic images and simple pen and paper representations. 
To this end, we couple the presentation of our system with a set of experiments exploring the relationship between fine-tuning training set size and success rates.
We also include a perceptual study validating viewer preference for incorporating \textit{twisted perspective} into the motion retargeting step.

We validate the desirability and appeal of our system by building and publicly releasing a version of it as the \AD Demo \,\cite{animateddrawings}.
Launched in December 2021, this demo has been used by millions of people around the world to animate their children's drawings.
Inspired by this reception, our second contribution is The Amateur Drawings Dataset: \hjs{180,000 drawings and user-accepted annotations collected, with consent, through the demo. See Section \ref{sec:UI} for a description of how the annotations were generated.}
We believe this dataset will be a resource to researchers from various fields seeking to better understand the space of amateur drawings, evaluate new algorithms in this domain, or develop new drawing-based tools in general.

To summarize, our contributions are as follows:
\begin{enumerate}
    \item 
    We explore the problem of automatic sketch-to-animation for children's drawings of human figures and present a framework that achieves this effect. We also present a set of experiments determining the amount of training data necessary to achieve high levels of success and a perceptual study validating the usefulness of our motion retargeting technique.
    \item To encourage additional research in the domain of amateur drawings, we present a first-of-its-kind dataset of 180,000 user-submitted amateur drawings, along with user-accepted bounding box, segmentation mask, and joint location annotations.
\end{enumerate}

Upon acceptance of this paper, we plan to publicly release the Amateur Drawings Dataset, project code, and fine-tuned model weights.



%\section{Preliminaries}
%\subsection{Periodic Boundary Condition}
%
%\subsection{Neural Ordinary Differential Equation}


\section{Self-consistency of the McKean-Vlasov Equation}
In this section, we present the generalized self-consistency potential for the MVE.
To understand the intuition behind our design, we first write the continuity equation \ref{eqn_CE} in a similar form as the MVE:
\begin{equation} \label{eqn_CE_as_MVE}
	\partial_t \rho^f_t + \udiv \bigg(\rho_t^f \Big(- \nabla V + K \ast \rho_t^f - \nu \log \rho_t^f + \delta_t \Big) \bigg) = 0,
\end{equation}
where $f$ is the hypothesis velocity (recall that $f_t(\cdot) = f(t, \cdot)$) and 
\begin{equation} \label{eqn_perturbation}
	\delta_t = f_t - (- \nabla V + K \ast \rho_t^f - \nu \log \rho_t^f),
\end{equation}
can be regarded as a perturbation to the original MVE system.
Taking this perturbation perspective, it is natural to study the deviation of the hypothesis solution $\rho^f_t$ from the true solution $\bar \rho_t$ using an appropriate Lyapunov function $L(\rho_t^f, \bar \rho_t)$. 
Clearly this deviation will depend on the perturbation $\delta_t$, and such a dependence is often termed as the {\emph{stability}} of the underlying dynamical system. 
Moreover, the aforementioned relation between the perturbation and the deviation allows us to derive the self-consistency potential of the MVE. 
%From its construction, the potential function designed in this way should automatically control the deviation of the hypothesis from the ground truth.

Following this idea, the design of the self-consistency potential can be determined by the choice of the Lyapunov function $L$ used in the stability analysis. 
In the following, we describe the Lyapunov function used for the MVE with the Coulomb interaction and the vorticity formulation of the 2D Navier-Stokes equation (MVE with Biot-Savart interaction). The proof of the following results are the major theoretical contributions of this paper and will be elaborated in the analysis section \ref{section_analysis}.
\begin{itemize}
	\item For the MVE with the Coulomb interaction, we choose $L$ to be the \emph{modulated free energy} (defined in \eqref{DefModFree}) which is originally proposed in \citep{bresch2019mean} to establish the mean-field limit of a corresponding interacting particle system. We have (setting $L = E$)
	{
	\begin{equation}
		\frac{\ud}{\ud t } E(\rho_t^f, \bar \rho_t) \leq \frac 1 2  \int_{\mathcal{X}} \rho_t^f \, |\delta_t |^2 \ud x  + C E(\rho_t^f, \bar \rho_t),
	\end{equation}
	}
	where $C$ is a universal constant depending on $\nu $ and $(\bar\rho_t)_{t \in [0, T]}$. 
	\item For the 2D Navier-Stokes equation (MVE with the  Biot-Savart interaction), we choose $L$ to be the KL divergence. Our analysis is inspired by \citep{jabin2018quantitative} which for the first time establishes the quantitative mean-field limit of the stochastic interacting particle systems where the interaction kernel can be  in some negative Sobolev space. We have 
	{
	\begin{equation}
		\frac{\ud }{\ud  t} \mathbf{KL}(\rho_t^f, \bar \rho_t) \leq - \frac \nu  2 \int \rho_t |\nabla \log \frac{\rho_t}{\bar \rho_t}|^2 +C \mathbf{KL}(\rho_t^f, \bar \rho_t)+ \frac{1}{\nu } \int \rho_t^f  |\delta_t|^2, 
	\end{equation}
	}
where again $C$ is a universal constant depending on $\nu $ and $(\bar\rho_t)_{t \in [0, T]}$. 
\end{itemize}
%\textbf{TODO: We should comment on the relation between our work and the previous work.}
%Note that the Lyapunov function $L$ used in these previous works take different form which are designed for the analysis of mean-field limit. Specifically, 

From the above discussion, we can see that the self-consistency potential \ref{eqn_self_consistency_potential} is exactly the term derived by stability analysis of the MVE system with an appropariate Lyapunov function, after applying the Gr\"onwall's inequality.
However, the potential function \ref{eqn_self_consistency_potential} remains elusive from a computational perspective. Moreover, when utilized as the objective loss for training a parameterized hypothesis velocity field, we must be able to compute the gradient w.r.t. the parameters, so that gradient-based optimizer can be utilized. This is elaborated in the next section.


\subsection{Stochastic Gradient Computation with Neural Network Parameterization}
While the choice of self-consistency potential \ref{eqn_self_consistency_potential} is theoretically justified through the above stability study, in this section we show that it admits an estimator which can be efficiently computed. 
Given an initial data point $\vx_0$, define the trajectory $\{\vx(t)\}_{t=0}^T$ via the initial value problem $\frac{d \vx(t)}{d t} = f_t(\vx(t); \theta)$, $\vx(0) = \vx_0$ (suppose that $f_t$ is Lipschitz continuous for all $t\in[0, T]$ so the trajectory exists and is unique). 
Define the map $X_t$ such that $\vx(t) = X_t(\vx_0)$.
From the definition of the push-forward measure, one has $\rho^f_t = X_t\sharp\bar\rho_0$, where $\rho^t_t$ is defined in \eqref{eqn_CE}.
Recall the definitions of the potential $R(f)$ in \eqref{eqn_self_consistency_potential} and the perturbation $\delta_t$ in \eqref{eqn_perturbation}. Use the change of variable formula of the push-forward measure in (a) and the Fubini's theorem in (b). We have
\begin{equation}
	R(f) = \int_0^T \|\delta_t\|_{\rho_t^f}^2 d t \stackrel{(a)}{=} \int_0^T \|\delta_t\circ X_t\|_{\rho_0}^2 d t \stackrel{(b)}{=} \int \int_0^T \|\delta_t\circ X_t(\vx_0)\|^2 d t d \bar \rho_0(\vx_0).
\end{equation}
Consequently, by defining the trajectory-wise loss
\begin{equation}
	R(f; \vx_0) = \int_0^T \|\delta_t\circ X_t(\vx_0)\|^2 d t,
\end{equation}
we can write the potential function \ref{eqn_self_consistency_potential} as an expectation $R(f) = \E_{\vx_0\sim \bar \rho_0}[R(f; \vx_0)]$.
Suppose that the hypothesis velocity field is parameterized by a neural network $f = f_\theta$.
Using the above expectation formulation, we obtain an unbiased estimation of $\nabla_\theta R(f_\theta)$ via the Monte-Carlo integration w.r.t. $\vx_0 \sim \bar  \rho_0$, given that we can compute $\nabla_\theta R(f_\theta; \vx_0)$.

We show $\nabla_\theta R(f_\theta; \vx_0)$ can be computed, at least up to a high accuracy, via the adjoint method (for completeness see the derivation of the adjoint method in appendix \ref{appendix_adjoint_method}). 
As a recap, suppose that we can write $R(f_\theta; x_0)$ in a standard ODE-constrained form
\begin{equation} \label{eqn_trajectory_wise_inconsistency_ODE_loss}
	R(f_\theta; \vx_0) = \ell(\theta)\ = \int_0^T g(t, \vs(t), \theta) d t,
\end{equation}
where $\{\vs(t)\}_{t\in[0, T]}$ is the solution to the initial value problem $\frac{d }{d t} \vs(t) = \psi(t, \vs(t); \theta)$ with $\vs(0) = \vs_0$, and $\psi$ is a known transition function.
The adjoint method states that the gradient $\frac{d}{d \theta} \ell(\theta)$ can be computed as\footnote{This implies that we can obtain an unbiased estimator of $\frac{d \ell}{d \theta}$ by first sample $t\sim \mathrm{Uniform}[0, T]$ and simply compute $T * \left(a(t)^\top\frac{\partial \psi}{\partial \theta}(t, \vs(t); \theta) + \frac{\partial g}{\partial \theta}(t, \vs(t); \theta)\right)$. 
	In practice, we sample multiple $t$ uniformly from $[0, T]$ to reduce the variance.
	Note that to obtain $\vs(t)$ and $\va(t)$ for all sampled time stamp $t$, we still need to solve the ODEs involved in the definition of $\vx(t)$ and $\va(t)$ once, however their dimensions are now independent of the size of the neural network. }
\begin{equation}
	\frac{d \ell}{d \theta} = \E_{t\sim \Uniform[0, T]}\left[a(t)^\top\frac{\partial \psi}{\partial \theta}(t, \vs(t); \theta) + \frac{\partial g}{\partial \theta}(t, \vs(t); \theta)\right].
\end{equation}
where $a(t)$ is solution to the final value problems $\frac{d}{d t} a(t)^\top + a(t)^\top \frac{\partial  \psi}{\partial  s}(t, \vs(t); \theta) + \frac{\partial g}{\partial s}(t, \vs(t); \theta) = 0, a(T) = 0$.
In the following, we focus on how $R(f_\theta; \vx_0)$ can be written in the above ODE-constrained form.


\paragraph{Write $R(f_\theta; \vx_0)$ in ODE-constrained Form}
Expanding the definition of $\delta_t$ in \eqref{eqn_perturbation} gives
\begin{equation} \label{eqn_delta_x_t}
	\delta_t\circ X_t(\vx_0) = \delta_t(\vx(t)) = f_t(\vx(t)) - \left( - \nabla V(\vx(t)) + K\ast \rho_t^f(\vx(t)) - \nu \nabla \log \rho_t^f(\vx(t))\right).
\end{equation}
Note that in the above quantity, $f$ and $V$ are known functions. 
Moreover, it is known that $\nabla \log \rho_t^f(\vx(t))$ admits a closed form dynamics (e.g. see Proposition 2 in \citep{shen22a})
\begin{equation} \label{eqn_dynamics_of_score}
	\frac{d }{d t}\nabla \log \rho_{t}^{f}(\vx(t)) = - \nabla  \left(\nabla \cdot f_{t}(\vx(t); \theta)\right) -   \left(\gJ_{f_{t}}(\vx(t); \theta)\right)^\top \nabla \log \rho_{t}^{f}(\vx(t)),
\end{equation}
which allows it to be explicitly computed by starting from $\nabla \log \bar \rho_0(x_0)$ and integrating over time (recall that $\bar \rho_0$ is known).
Here $\gJ_{f_{t}}$ denotes the Jacobian matrix of $f_t$.
Consequently, all we need to handle is the convolution term $K\ast \rho_t^f(\vx(t))$, which in general cannot be exactly computed since it depends on the global configuration of the hypothesis distribution $\rho_t^f$. 

A common choice to avoid the difficulty of the convolution operation is via empirical approximation:
Let $\{\vy_{i}(t)\}_{i=1}^N$ be a batch of i.i.d. samples distributed according to $\rho_t^f$ and denote an empirical approximation of $\rho_t^f$ by $\mu_N^{\rho_t^f} = \frac{1}{N}\sum_{i=1}^{N} \delta_{\vy_{i}(t)}$, where $\delta_{\vy_{i}(t)}$ denotes the Dirac measure at $\vy_{i}(t)$. 
We approximate the convolution term in \eqref{eqn_delta_x_t} in different ways for the Coulomb and the  Biot-Savart interactions:
\vspace{-2mm}
\begin{enumerate}[wide, labelwidth=!, labelindent=0pt]
%\begin{itemize}
	\item For the Coulomb interaction, we directly approximate the convolution term in \eqref{eqn_delta_x_t} by $K \ast \mu_N^{\rho_t^f}(\vx(t)) = \frac{1}{N} \sum_{i=1}^N K(\vx(t) -  \vy_i(t))$. In practice, we choose $N$ sufficiently large so that the above empirical approximation is accurate. 
	Indeed, at least for the whole space case, i.e. the underlying space  $\mathcal{X}$ is $\mathbb{R}^d$, one has that 
	\vspace{-2mm}
	\[
	\int_{\mathbb{R}^d} |K * \mu_N^{\rho^{f}} (x)- K * \rho^f (x)|^2 \ud x = \int_{x \ne y } g(x - y ) \ud ( \mu_N^{\rho^f} -  \rho^f  )^{\otimes 2}(x, y) = F(\mu_N^{\rho^f}, \rho^f), 
	\]
	where $F(\mu_N^{\rho^f}, \rho^f)$ is the modulated (interaction) energy defined as in \cite{serfaty2020mean}. 
	We expect $F(\mu_N^{\rho^f}, \rho^f)\rightarrow 0$ almost surely as $N\rightarrow\infty$.
	\item For Biot-Savart interaction (2D Navier-Stokes equation), there are more structure to exploit and we can completely avoid the singularity: As noted by \cite{jabin2018quantitative}, the convolution kernel $K$ can be written in a divergence form:
	\vspace{-2mm}
	\begin{equation} \label{eqn_K_as_divergence}
		K = \nabla \cdot U, \text{ with } U(\vx) = \frac{1}{2\pi}\begin{bmatrix}
			-\arctan(\frac{\vx_1}{\vx_2}),& 0\\
			0,& \arctan(\frac{\vx_2}{\vx_1})
		\end{bmatrix},
	\end{equation}
	where the divergence of a matrix function is applied row-wisely, i.e. $[K(\vx)]_i = \udiv\ U_i(\vx)$.
	Using integration by parts, one has (assuming that the boundary integration vanishes, e.g. when the underlying space is a torus)
	\vspace{-2mm}
	\begin{align*}
		K\ast \rho_t^f (\vx) =&\ \int K(\vy) \rho_t^f(\vx - \vy) d \vy = \int \nabla\cdot U(\vy) \rho_t^f(\vx - \vy) d \vy = \int U(\vy) \nabla \rho_t^f(\vx - \vy) d \vy \\
		=&\ \int U(\vx - \vy) \rho_t^f(\vy) \nabla \log \rho_t^f(\vy) d \vy = \E_{\vy\sim \rho_t^f(\vy)}[U(\vx - \vy) \nabla \log \rho_t^f(\vy)].
	\end{align*}
	If the score function $\nabla \log \rho_t^f$ is bounded, then the integrand in the expectation is also bounded. Therefore, we can avoid integrating singular functions and the Monte Carlo-type estimation $\frac{1}{N} \sum_{i=1}^N U(\vx - \vy_i(t)) \nabla \log \rho_t^f(\vy_i(t))$ is accurate for a sufficiently large value of N.
%\end{itemize}
\end{enumerate}
\vspace{-2mm}
With the above discussion, we are now ready to write $R(f_\theta; \vx_0)$ in an ODE-constrained form. Define the state $\vs(t)$, the initial condition $\vs_0$ and the transition function $\psi$ as follows: Let
\begin{equation}
	\vs(t) = \left[\vx(t), \xi(t), \{\vy_i(t)\}_{i=1}^N, \{\zeta_i(t)\}_{i=1}^N\right],
\end{equation}
with $\xi(t) = \nabla\log\rho_t^f(\vx(t))$ and $\zeta_i(t) = \nabla\log\rho_t^f(\vy_i(t))$.  Take  the initial condition 
\begin{equation}
	\vs_0 = \left[\vx_0, \xi_0, \{\vy_i(0)\}_{i=1}^N, \{\zeta_i(0)\}_{i=1}^N\right]
\end{equation}
with $\xi_0 = \nabla\log \bar \rho_0(\vx_0)$, $\zeta_i(0) = \nabla\log\bar\rho_0(\vy_i(0))$, and $\vy_i(0) \stackrel{iid}{\sim} \bar \rho_0$;
and define the function
\begin{equation}
	\psi(t, s(t); \theta) = [f_t(\vx(t); \theta), h_t(\vx(t), \xi(t); \theta), \{f_t(\vy_i(t); \theta)\}_{i=1}^N, \{h_t(\vy_i(t), \zeta_i(t); \theta)\}_{i=1}^N],
\end{equation}
where $h(\va, \vb; \theta) = - \nabla  \left(\nabla \cdot f_{t}(\va; \theta)\right) -   \gJ^\top_{f_{t}}(\va; \theta) \vb$ (derived from \eqref{eqn_dynamics_of_score}).
Finally, define
\begin{equation}
	g(t, \vs(t); \theta) = \|f(t, \vx(t); \theta) - \left(-\nabla V(\vx(t)) + E(t, \vs(t)) - \nu \xi(t)\right)\|^2,
\end{equation}
where the estimator $E(t, \vs)$  of the convolution term is defined as
\begin{equation}
	E(t, \vs(t)) = \begin{cases}
		\frac{1}{N} \sum_{i=1}^N K(\vx(t) -  \vy_i(t)) & \text{ the Coulomb case}, \\
		\frac{1}{N} \sum_{i=1}^N U(\vx - \vy_i(t)) \zeta_i(t) & \text{the Biot-Savart case}.
	\end{cases}
\end{equation}
We recall the definition of $U$ in \eqref{eqn_K_as_divergence}.

%\paragraph{Removing the Integration over the Parameter Space} If we directly evaluate the integration in \eqref{eqn_adjoint_method_integration}, we need to solve an ODE numerically in stats space with its dimension at least as large as the number of parameters in the neural network. Consequently, the computational complexity will drastically increase if a complicated neural network structure is used. To alleviate this burden, observe that the gradient admits the equivalent expectation form
%\begin{equation}
%	\frac{d \ell}{d \theta} = \E_{t\sim \Uniform[0, T]}\left[a(t)^\top\frac{\partial \psi}{\partial \theta}(t, \vs(t); \theta) + \frac{\partial g}{\partial \theta}(t, \vs(t); \theta)\right].
%\end{equation}
\section{Analysis} \label{section_analysis}
In this section, we focus on the torus case, i.e. $\X = \Pi^d$ is a box with the periodic boundary condition.
This is a typical setting considered in the literature as the universal function approximation of NNs only holds over a compact set. Moreover, the boundary integral resulting from integration by parts vanishes in this setting, making it amenable for analysis purposes.
For completeness, we provide a discussion on the unbounded case, i.e. $\X = \sR^d$ in the Appendix \ref{appendix_unbounded}, which requires additional regularity assumptions.
% \red{Mention that we focus on the torus case.}
% We start by defining the modulated (interaction) energy and modulated free energy for two probability densities $\rho, \bar \rho$.  
% Again $\mathcal{X}$ denotes the underlying space \red{which can be the whole space $\mathbb{R}^d$} or the torus $\Pi^d $ which can be identified with $[-L, L]^d$ with the periodic boundary condition and $L>0$. 
Given the MVE (\ref{eqn_MVE}), if $K$ is bounded, it is sufficient to choose the Lyapunov functional $L(\rho_t^f, \bar \rho_t)$ as the KL divergence (please see Theorem \ref{theorem_bounded_K} in the appendix). But for the singular Coulomb kernel, we need also to consider the modulated energy as in \citep{serfaty2020mean}
\begin{equation}
    \text{(Modulated Energy)}\quad 
	\label{DefModEnergy}
	F(\rho, \bar \rho) \defi \frac 1 2 \int_{\mathcal{X}^2} g(x- y) \ud (\rho - \bar \rho)(x) \ud (\rho - \bar \rho )(y), 
\end{equation}
where $g$ is the fundamental solution to the Laplacian equation in $\mathbb{R}^d$, i.e. $- \Delta g = \delta_0$, and the Coulomb interaction reads $K= -\nabla g$ (see its closed form expression in equation (\ref{eqn_coulomb_interaction})). If we are only interested in the deterministic  dynamics with Coulomb interactions, i.e. $\nu =0$ in equation (\ref{eqn_MVE}), it suffices to choose $L(\rho_t^f, \bar \rho) $ as  $F(\rho_t^f, \bar \rho_t)$ (please see Theorem \ref{ThmCoul}). But if we consider the system with  Coulomb interactions and  diffusions, i.e. $\nu >0$, we shall combine the KL divergence and the modulated energy to form the modulated free energy as in \cite{bresch2019modulated}, which reads 
\begin{equation}
    \text{(Modulated Free Energy)}\quad 
	\label{DefModFree}
	E(\rho, \bar \rho) \defi \nu \KL (\rho, \bar \rho) +  F(\rho, \bar \rho). 
\end{equation}
This definition agrees with the physical meaning that ``Free Energy = Temperature $\times $ Entropy + Energy", and we note that the temperature is proportional to the diffusion coefficient $\nu$. We remark also for two probability densities $\rho$ and $\bar \rho$, $F(\rho, \bar \rho) \geq 0$ since by looking in the Fourier domain $F(\rho, \bar \rho) = \int \hat g(\xi) |\widehat{\rho - \bar \rho}(\xi)|^2  \ud \xi \geq 0$ as $\hat g(\xi) \geq 0$. Moreover, $F(\rho, \bar \rho)$ can be regarded as a negative Sobolev norm for $\rho- \bar \rho$, which metricizes weak convergence.\\
To obtain our main stability estimate, we first obtain the time evolution of the KL divergence.  
%%%% Evolution of the KL divergence 
\begin{lemma}[Time Evolution of the KL divergence] \label{TimeEvolKLMV} 
	Given the hypothesis velocity field $f=f(t, x) \in C^1_{t, x}$. Assume that $(\rho_t^f)_{t \in [0, T]}$ and $(\bar \rho_t)_{t \in [0, T]}$ are classical solutions to equation (\ref{eqn_CE}) and equation (\ref{eqn_MVE_CE}) respectively.  It holds that (recall the definition of $\delta_t$ in equation (\ref{eqn_perturbation})) 
	\[
	\frac{\ud }{\ud t} \int_{\mathcal{X}} \rho^f_t \log \frac{\rho^f_t }{\bar \rho_t} = - \nu \int_{ \X}\rho^f_t |\nabla \log \frac{\rho^f_t}{\bar \rho_t}|^2 +  \int_{\X} \rho^f_t K * (\rho^f_t - \bar \rho_t ) \cdot \nabla  \log  \frac{\rho^f_t }{\bar \rho_t} +  \int_{\X} \rho^f_t \delta_t \cdot \nabla \log \frac{\rho^f_t }{\bar \rho_t }, 
	\]
	where $\X$ is the tours $\Pi^d$. All the integrands are evaluated at $\vx$. 
\end{lemma}

We refer the proof of this lemma and all other lemmas and theorems in this section to the appendix \ref{detailed proof}. We remark that to have the existence of classical solution $(\bar \rho_t)_{t \in [0, T]}$, we definitely need the regularity assumptions on $-\nabla V$ and on $K$. But the linear term $- \nabla V $ will not contribute to the evolution of the relative entropy. See \citep{jabin2018quantitative} for detailed discussions. \\
%%%% Evolution of the Modulated Energy 
Similarly, we have the time evolution of the modulated energy as follows.
\begin{lemma}[Time evolution of the modulated energy] \label{ModuEnergyEvo} Under the same assumptions as in Lemma \ref{TimeEvolKLMV}, given the diffusion coefficient $\nu \geq 0$, it holds that (recall the definition of $\delta_t$ in equation (\ref{eqn_perturbation})) 
	\[
	\begin{split}
		\frac{\ud }{\ud t } F(\rho_t^f, \bar \rho_t)&  =  - \int_{\mathcal{X}} \rho_t^f \|K *(\rho_t^f - \bar \rho_t)\|^2 - \int_{\mathcal{X}} \rho_t^f \, \delta_t \cdot K * (\rho_t^f - \bar \rho_t ) + \nu \int_{\mathcal{X}} \rho^f_t \, K * (\rho_t^f - \bar \rho_t )\cdot \nabla \log \frac{\rho_t^f}{\bar \rho_t} \\
		&  - \frac{1}{2} \int_{\mathcal{X}^2} K(x-y) \cdot \Big( \mathcal{A}[\bar \rho_t](x) - \mathcal{A}[\bar \rho_t](y) \Big) \ud (\rho_t^f - \bar \rho_t )^{\otimes 2 }(x, y) \\
	\end{split}
	\]
	where we recall that the operator $\gA$ is defined in equation (\ref{eqn_operator_A}).
\end{lemma}






%%%%%%% The 2D Navier-Stokes case  
By  Lemma \ref{TimeEvolKLMV} and careful analysis, in particular by rewriting the Biot-Savart law in the divergence of a bounded matrix-valued function (\ref{eqn_K_as_divergence}), we obtain the following estimate for the 2D NSE.

\begin{theorem}[Stability estimate of the 2D NSE] \label{NSMainEstimate}	
	% Consider the 2D NSE in the vorticity formulation (\ref{eqn_MVE}) with $V=0$. 
    Notice that when $K$ is the Biot-Savart kernel, $\udiv K =0$. Assume that the initial data $\bar \rho_0 \in C^3(\Pi^d)$ and there exists $c>1$ such that $\frac 1 c \leq \bar \rho_0 \leq c$.  Assume further the hypothesis velocity field $f(t, x) \in C^1_{t, x}$. Then it holds that 
	\[
	\sup_{t \in [0, T]} \int_{\Pi^d} \rho_t^f \log \frac{\rho_t^f}{\bar \rho_t} \ud x \leq \frac{e^C}{\nu}  R(f), 
	\]
	where $C = \int_0^\infty     M(t) \ud t < \infty$ with 
	$M(t) \defi \|\nabla \log \bar \rho_t\|_{L^\infty}^2/2\nu + 2\Big\| {\nabla^2 \bar \rho_t}/{\bar \rho_t} \Big\|_{L^\infty}$. 
	
\end{theorem} 
We remark that given $\bar \rho_0$ is smooth enough and fully supported on $\X$, one can propagate the regularity to finally show the finiteness of $C$.
See detailed computations as in \cite{guillin2021uniform}. 
We give the complete proof in the appendix \ref{detailed proof}. This theorem tells us that as long as $R(f)$ is small, the KL divergence between $\rho_t^f$ and $\bar \rho_t$ is small and the control is uniform in time $t \in [0, T]$ for any $T$. Moreover, we highlight that $C$ is independent of $T$, and our result on the NSE is significantly better than the average-in-time and exponential-in-$T$ results from \citep{deerror}.\\
% We state this theorem on torus for simplicity and \red{one may expect similar result on $\mathbb{R}^d$}. \red{Also the $C^\infty$ condition can be relaxed for instance to $C^2$}. 
%%%%%  Remark on The case with bounded interactions $K \in L^\infty
%%%% Leave in the appendix  
%%%% The deterministic case with Coulomb 
%%%% Lemma of the evolution of the modulated free energy 
To treat the MVE (\ref{eqn_MVE}) with Coulomb interactions, we exploit the time evolution of the modulated free energy $E(\rho_t^f, \bar \rho_t)$. Indeed, combining Lemma \ref{TimeEvolKLMV} and Lemma \ref{ModuEnergyEvo}, we arrive at the following identity.  
\begin{lemma}[Time evolution of the modulated free energy]\label{TimeEvoMFE} 
	Under the same assumptions as in Lemma \ref{TimeEvolKLMV}, one has (recall the definitions of $\delta_t$ and $\gA$ in (\ref{eqn_perturbation}) and (\ref{eqn_operator_A}) respectively) 
	\[
	\begin{split}
		\frac{\ud }{\ud t } E(\rho_t^f, \bar \rho_t)  & = -\int_{\mathcal{X}} \rho_t^f  \Big|K * (\rho_t^f - \bar \rho_t) - \nu \nabla \log \frac{\rho_t^f }{\bar \rho_t}\Big|^2 - \int_{\mathcal{X}} \rho_t^f \, \delta_t \cdot \Big( K * (\rho_t^f - \bar \rho_t ) - \nu \nabla \log \frac{\rho_t^f}{\bar \rho_t }\Big) \\
		&  - \frac{1}{2} \int_{\mathcal{X}^2} K(x-y) \cdot \Big( \mathcal{A}[\bar \rho_t](x) - \mathcal{A}[\bar \rho_t](y) \Big) \ud (\rho_t^f - \bar \rho_t )^{\otimes 2 }(x, y).
	\end{split}
	\]
	% where we recall that the operator $\gA$ is defined in \eqref{eqn_operator_A}.
\end{lemma}
\vspace{-1mm}
%%%%%%%%%%%%%%Main theorem of the Coulomb case 
Inspired by the mean-field convergence results as in \cite{serfaty2020mean} and \cite{bresch2019modulated}, we finally can control the growth of $E(\rho_t^f, \bar \rho_t)$ in the case when $\nu >0$,  and $F(\rho_t^f, \bar \rho_t)$ in the case when $\nu =0$. Note also that $E(\rho_t^f, \bar \rho_t )$ can also control the KL divergence when $\nu >0$. 
\begin{theorem} [Stability estimate of MVE with Coulomb interactions] \label{ThmCoul}
	Assume that for $t \in [0, T]$, the underlying velocity field $\mathcal{A}[\bar \rho_t](x)$ is Lipschitz  in $x$ and
	$\sup_{t \in [0, T]} \|\nabla \mathcal{A}[\bar \rho_t](\cdot)\|_{L^\infty} = C_1  < \infty.$ 
	Then there exists $C>0$ such that 
	\[
	\sup_{t \in [0, T]} \nu\,  \KL (\rho_t^f, \bar \rho_t) \leq \sup_{t \in [0, T]} E(\rho_t^f, \bar \rho_t) \leq \exp(C C_1 T) R(f).  
	\]
	In the deterministic case when $\nu =0$, under the same assumptions, it holds that 
	\[
	\sup_{t \in [0, T]} F( \rho_t^f, \bar \rho_t) \leq \exp(CC_1T ) R(f). 
	\]
\end{theorem}
\vspace{-1mm}
Recall the definition of the operator $\gA$ in \eqref{eqn_operator_A}. Given that $\mathcal{X}= \Pi^d$, and $\bar \rho_0$ is smooth enough and bounded from below, one can propagate regularity to obtain the Lipschitz condition for $\mathcal{A}[\bar \rho_t]$. See the proof and the discussion on the Lipschitz assumptions on $\mathcal{A}[\bar \rho_t](\cdot)$ in the appendix \ref{detailed proof}. 
\paragraph{Approximation Error of Neural Network}
Theorems \ref{NSMainEstimate} and \ref{ThmCoul} provide the error estimation guarantee for the proposed \EINN\ loss (\ref{eqn_self_consistency_potential}).
Suppose that we parameterize the velocity field $f=f_\theta$ with an NN parameterized by $\theta$, as we did in Section \ref{section_NN_parameterization} and let $\tilde f$ be the output of an optimization procedure when $R(f_\theta)$ is used as objective.
In order the explicitly quantify the mismatch between $\rho^{\tilde f}_t$ and $\bar \rho_t$, we need to quantify two errors: (i) Approximation error, reflecting how well the ground truth solution can be approximated among the NN function class of choice; (ii) Optimization error, involving minimization of a highly nonlinear non-convex objective. 
In the following, we show that for a function class $\gF$ with sufficient capacity, there exists at least one element $\hat f\in\gF$ that can reduce the loss function $R(\hat f)$ as much as desired.
We will not discuss how to identify such an element in the function class $\gF$ as it is independent of our research and remains possibly the largest open problem in modern AI research.
To establish our result, we make the following assumptions.
\begin{assumption} \label{ass_appendix_initial}
	$\rho_0$ is sufficiently regular, such that $\nabla \log \rho_0 \in \gL^\infty(\X)$ and $\bar f_t  = \mathcal{A}[\bar \rho_t] \in W^{2,\infty}(\X)$. $\nabla V$ is Lipschitz continuous. Here $W^{2,\infty}(\X)$ stands for the Sobolev norm of order $(2, \infty)$ over $\X$.
\end{assumption}
\vspace{-1mm}
We here again need to propagate the regularity for $f_t$ at least for a time interval $[0, T]$. It is easy to do so for the torus case, but for the unbounded domain, there are some technical issues to be  overcome. Similar assumptions are also needed in some mathematical works for instance in \cite{jabin2018quantitative}. 
% [cite some papers]
% \begin{remark}
%     For the torus case, if $\rho_0$ is bounded from above and below, and chosen to be in $C^3$, we can propagate the regularity of $\rho_0$ to $\bar \rho_t$ for $t \in [0, T]$. We can then obtain $\bar f_t \in \gC^3(\X)$.
% \end{remark}
We also make the following assumption on the capacity of the function class $\gF$, which is satisfied for example by NNs with tanh activation function \citep{DERYCK2021732}.
\begin{assumption} \label{ass_appendix_approximation}
	The function class is sufficiently large, such that there exists $\hat f \in \gF$ satisfying $\hat f_t \in \gC^3(\X)$ and $\|\hat f_t - \bar f_t\|_{W^{2, \infty}(\X)} \leq \epsilon$ for all $t\in[0, T]$.
\end{assumption}
\begin{theorem} \label{thm_approximation_error_NN}
	Consider the case where the domain is the torus. 
	Suppose that Assumptions \ref{ass_appendix_initial} and \ref{ass_appendix_approximation} hold. 
    For both the Coulomb and the Biot-Savart cases, there exists $\hat f\in\gF$ such that $R(\hat f) \leq C(T)\cdot(\epsilon \cdot\ln 1/\epsilon)^2$, where $C(T)$ is some constant independent of $\epsilon$. Here $R$ is the \EINN\ loss (\ref{eqn_self_consistency_potential}).
\end{theorem}
% \red{Maybe mentioning the difficulty in proving this theorem to promote the novelty of the result.}
% \red{Can we derive the results for unbounded domain?}
% \red{This is just the result for the Coulomb case. We should also prove the NSE case.}
The major difficulty to overcome is the lack of Lipschitz continuity due to the singular interaction. We successfully address this challenge by establishing that the contribution of the singular region $(\|\vx\|\leq\epsilon)$ to $R(\hat f)$ can be bounded by $O((\epsilon \log \frac{1}{\epsilon})^2)$.
Please see the detailed proof in Appendix \ref{appendix_approximation_error_NN}.



%1. Previous analysis provides control of the solution quality by $R(f)$
%2. The next question is how small $R(f)$ can be
%3. Two errors: approximation error and optimization error. The latter is an open question. In this section, we focus on the approximation error.
%4. Conclusions on NN approximation in Sobolev norm
%5. List assumptions
%6. describe conclusion.

% Discuss the long-time guarantee (without exponential in T, under additional assumptions)

% Periodic solution to Taylor-Green vortex

% Revise the discussion on "expected"



























\setlength{\tabcolsep}{1.6mm}{
\renewcommand\arraystretch{1.1}
\begin{table}[ht]
  \centering
  \scalebox{0.9}{
  \begin{tabular}{llcccc}
    \toprule
    &\multirow{2}*{Methods} & \multirow{2}*{Sal.} &   \multicolumn{2}{c}{VOC} & MS~COCO \\
    \cmidrule(r){4-5}\cmidrule(r){6-6}
    &&&\texttt{val}&\texttt{test}&\texttt{val}\\
    \hline
    \multirow{13}*{\rotatebox{90}{ResNet-50}}
    &IRN~\cite{irn}          \tiny{CVPR'19}     &              & 63.5       & 64.8          & 42.0  \\
    &LayerCAM~\cite{layercam}\tiny{TIP'21}      &              & 63.0       & 64.5          & -     \\
    &AdvCAM~\cite{advcam}    \tiny{CVPR'21}     &              & 68.1       & 68.0          & 44.2  \\
    &RIB~\cite{rib}          \tiny{NeurIPS'21}  &              & 68.3       & 68.6          & 44.2  \\
    &ReCAM~\cite{recam}      \tiny{CVPR'22}     &              & 68.5       & 68.4          & 42.9  \\
    % \rowcolor{Gray}
    &\cellcolor{Gray}IRN+\texttt{LPCAM}    &\cellcolor{Gray} & \cellcolor{Gray}68.6    & \cellcolor{Gray}68.7      & \cellcolor{Gray}44.5  \\
    &SIPE~\cite{sipe}        \tiny{CVPR'22}     &              & 68.8       & 69.7          & 40.6  \\
    &OOD~\cite{ood}+Adv      \tiny{CVPR'22}     &              & 69.8       & 69.9          & -     \\
    &AMN~\cite{amn}          \tiny{CVPR'22}     &              & 69.5       & 69.6          & 44.7  \\
    &\cellcolor{Gray}AMN+\texttt{LPCAM}    &\cellcolor{Gray} & \cellcolor{Gray}70.1    &\cellcolor{Gray} 70.4      & \cellcolor{Gray}45.5  \\ 
    &ESOL~\cite{esol}        \tiny{NeurIPS'22}  &              & 69.9$^*$   & 69.3$^*$      & 42.6  \\
    &CLIMS~\cite{clims}      \tiny{CVPR'22}     &              & 70.4$^*$   & 70.0$^*$      & -     \\
    &EDAM~\cite{edam}        \tiny{CVPR'21}     &\checkmark    & 70.9$^*$   & 71.8$^*$      & -     \\
    &\cellcolor{Gray}EDAM+\texttt{LPCAM}  &\cellcolor{Gray}\checkmark & \cellcolor{Gray}71.8$^*$ &\cellcolor{Gray} 72.1$^*$& \cellcolor{Gray}42.1\\
    \hline
    \multirow{9}*{\rotatebox{90}{WideResNet-38}}
    &Spatial-BCE~\cite{sbce} \tiny{ECCV'22}     &              & 70.0       & 71.3      & 35.2  \\
    &BDM~\cite{bdm}          \tiny{ACMMM'22}    &\checkmark    & 71.0       & 71.0      & 36.7  \\ 
    &RCA~\cite{rca}+OOA      \tiny{CVPR'22}     &\checkmark    & 71.1       & 71.6      & 35.7  \\
    &RCA~\cite{rca}+EPS      \tiny{CVPR'22}     &\checkmark    & 72.2       & 72.8      & 36.8  \\
    &HGNN~\cite{hgnn}        \tiny{ACMMM'22}    &\checkmark         & 70.5$^*$   & 71.0$^*$  & 34.5  \\ 
    &EPS~\cite{eps}          \tiny{CVPR'21}     &\checkmark         & 70.9$^*$   & 70.8$^*$  & -     \\
    &RPIM~\cite{rpim}        \tiny{ACMMM'22}    &\checkmark         & 71.4$^*$   & 71.4$^*$  & -     \\ 
    &L2G~\cite{l2g}          \tiny{CVPR'22}     &\checkmark         & 72.1$^*$   & 71.7$^*$  & 44.2  \\
    \hline
    \multirow{2}*{\rotatebox{90}{\small{DeiT-S}}}
    &MCTformer~\cite{mctformer}    \tiny{CVPR'22}     &                 & 71.9$^{\dag}$  & 71.6$^{\dag}$   & 42.0  \\
    &\cellcolor{Gray}MCTformer+\texttt{LPCAM}      &\cellcolor{Gray} & \cellcolor{Gray}72.6$^{\dag}$  & \cellcolor{Gray}72.4$^{\dag}$  &\cellcolor{Gray} 42.8 \\
    \bottomrule
  \end{tabular}}
  \vspace{-2mm}
  \caption{The mIoU results (\%) based on DeepLabV2 on VOC and MS~COCO. The side column shows three backbones of multi-label classification model. ``Sal.'' denotes using saliency maps. * denotes the segmentation model is pre-trained on MS~COCO. $^\dag$ denotes the segmentation model is pre-trained on VOC.
  }
  \vspace{-6mm}
  \label{table_related}
\end{table}
}



\section{Experimental Results}
\label{sec:experiments}
\subsection{Training Details}
\cite{Kalantari2017DeepHD} provides the first dataset specifically designed for multi-exposure HDR fusion under large motion. It consists of 74 training sets, which we use to supervise the training of our model. We crop the input images to patches of size \(256 \times 256\) at a step size of 64. This totally generates 20128 training samples. To augment training samples, we randomly rotate and flip the training images. The training adopts Adam optimizer. The learning rate is initialized to \(10^{-4}\) and is reduced to \(10^{-5}\) after 20 epochs. It is observed that 40 epochs are sufficient for the training to converge.    

\subsection{Numerical Evaluation}
We numerically measure the performance of our method on the 15 test sets of \cite{Kalantari2017DeepHD}, by Peak Signal-to-Noise Ratio (PSNR) and Structure Similarity, computed in both tonemapping domain (-\(\mu\)) and HDR linear domain (-L). Visual difference metric HDR-VDP-2 is also adopted, where the parameters are set as same as in previous works \cite{wu2018end} and \cite{niu2021hdrgan}. 

Table \ref{table_metrics} compares our model with state-of-the-art models. For \cite{yan2020nonlocal} and \cite{xiong2021hierarchical}, we use the results reported in the publications. Note that \cite{sen2012robust} and \cite{hu2013hdr} are not machine learning based methods. Moreover,  \cite{Kalantari2017DeepHD} and \cite{wu2018end} apply optical flow and homography transformation to preprocess the input images respectively, and hence entail extra computation. 

Table \ref{table_metrics} shows that our method outperforms competing method in terms of PSNR-L, SSIM-$\mu$, SSIM-L and HDR-VDP-2. It ranks the second best in PSNR-$\mu$, being slightly (0.1dB) inferior to \cite{xiong2021hierarchical}. Note that \cite{xiong2021hierarchical} utilizes a pretrained model to detect ghosting regions for training, whereas our method does not require any pretrained model. The high PSNR and SSIM scores varify that our model has strong HDR reconstruction ability and can accurately restore the radiance and structure of the scene in both tonemapping domain and HDR linear domain. Furthermore, its high performance in term of HDR-VDP-2\cite{mantiuk2011hdr} performance indicates that our method can generate HDR image visually close to the target image.

\begin{table*}[ht]
\centering
\begin{tabular}{l|c|c|c|c|c}
\hline
& PSNR-$\mu$ & PSNR-L & SSIM-$\mu$ & SSIM-L & HDR-VDP-2 \\
\hline
\bfseries Sen & 40.97 & 38.36 & 0.9830 & 0.9746 & 60.60\\
\hline
\bfseries Hu  & 35.65 & 30.80 & 0.9725 & 0.9491 & 58.34\\
\hline
\bfseries Kalantari & 42.69 & 41.22 & 0.9888 & 0.9845 & 65.05\\
\hline
\bfseries DeepHDR& 41.99 & 41.22 & 0.9878 & 0.9859 & \underline{65.91}\\
\hline
\bfseries AHDR & 43.62 & 41.03 & 0.9900  &\underline{0.9883} & 63.85 \\
\hline 
\bfseries NHDRRNet& 42.414 & - & 0.9887 & - & 61.21 \\
\hline 
\bfseries HDR-GAN &43.92 & \underline{41.57} &\underline{0.9905} &0.9865 & 65.45\\
\hline 
\bfseries HFNet & \textbf{44.28} & 41.47 & - & - & - \\
\hline 
\bfseries Ours & \underline{44.18} & \textbf{42.19}&\textbf{0.9912} & \textbf{0.9883}& \textbf{67.07} \\
\hline
\end{tabular}
\caption{Numerical performance of the proposed model, evaluated on the dataset by Kalantari-Ramamoorthi. The best and second best results for each metric are marked in \textbf{bold} and \underline{underlined}, respectively}
\label{table_metrics}
\end{table*}

\subsection{Visual Performance Evaluation}

\begin{figure*}[!htb]
\centering
\includegraphics[width=\textwidth]{experiments/kalantari_test.png}
\caption{Visual comparison on the test set of Kalantari-Ramamoorthi dataset. Zoom-in views of reconstruction by each method are presented on the saturated regions that contain moving objects. Our network built with gated Swin Transformer yields noticeably better visual results than other methods.}
\label{fig_kalantari_test}
\end{figure*}
Fig. \ref{fig_kalantari_test} present the visual performance of our method and comparable methods on two examples from \cite{Kalantari2017DeepHD}. We present the zoom-in views of two challenging cases, where large saturated regions contain substantial non-rigid motion in the reference image. The two patch-based methods do not reconstruct the missing details in the saturated regions, as they heavily rely on the details provided by the reference image for registration. Image reconstructed by the optical flow based method \cite{Kalantari2017DeepHD} suffers motion blur artifacts. This is because the convolutions of DeepHDR and HDR-GAN have limited receptive fields, and hence are hampered to repair missing content in misaligned regions by aligned regions. The gating mechanism of AHDR is only applied to low-level features, so the high-level outliers may deteriorate the HDR fusion. In contrast to comparable methods, our model remarkably overcomes the ghosting artifacts.

\begin{figure}[ht]
\centering
\includegraphics[width=\columnwidth]{experiments/sen_test.pdf}
\caption{Visual performance comparison on example images from the dataset by Sen et al. Zoom in views on challenging areas are presented. Although the ground truth is unavailable, it can be clearly observed that our method visually performs better than comparable methods.}
\label{sen_test}
\end{figure}

\begin{figure}[ht]
\centering
\includegraphics[width=\columnwidth]{experiments/tursun_test.pdf}
\caption{Visual performance comparison on example images from the dataset by Tursun et al. Compared to state of the art methods, our method suffers less ghosting artifact.}
\label{tursun_test}
\end{figure}

Fig.\ref{sen_test} and Fig.\ref{tursun_test} present visual performance of our method on two examples from benchmark datasets \cite{sen2012robust} and \cite{tursun2016objective}. As these test datasets   do not provide ground truth image. we mark the visual difference on the results generated by different methods. It can be seen that our method suffers less artifacts than other methods in various scenes with various motion patterns, achieving better visual results. Our method creates high-quality HDR more robustly and generalizes well. 

\subsection{Ablation Study}

\begin{table}[h]
\centering
\resizebox{\columnwidth}{!}{
\begin{tabular}{l|c|c|c|c|c}
\hline
                         & PSNR-$\mu$ & PSNR-l & SSIM-$\mu$ & SSIM-l & HDR-VDP-2 \\ \hline
restormer(w/o ssim loss) & 44.00  & 41.5   & 0.9906 & 0.9873 & 64.72  \\ \hline
Ours(w/o ssim loss)      & 44.07  & 41.83  & 0.9909 & 0.9879 &  64.78  \\ \hline
Ours                     & 44.18  & 42.19  & 0.9912 & 0.9883 & 67.07      \\ \hline
\end{tabular}
}
\caption{Experimental results of ablation study. We compare using Gated Swin Transformer v.s. Gated Transformer, and the combined loss function v.s. the traditional $l_{1}$ norm loss function.}
\label{table_ablation_block_loss}
\end{table}

We verify various components of our method, including Swin Transformer, loss function, and gating mechanism by ablation study.

\subsubsection{Ablation Study on Block Design}
Our model has similar architecture to Restormer, which uses modified Transformer, whereas we use modified Swin Transformer as the building unit. For comparison, we replace the residual modules in each block in our model with multiple transformer layers as in Restormer, with same number of transformer layers. Table \ref{table_ablation_block_loss} presents the results, which show that using Swin Transformer achieves superior performance in all measures. The reason is that the attention module of Restormer is computed channel-wise, but forgoes the cross-exposure spatial dependency to repair the non-aligned area. 

\subsubsection{Ablation Study on Loss Function}
We trained our model under different loss function configurations, as shown in \ref{table_ablation_block_loss}. The results validate that the SSIM loss benefits detail reconstruction.

\subsubsection{Ablation Study on Gating Mechanism}
\begin{table}[h]
\resizebox{\columnwidth}{!}{
\begin{tabular}{l|c|c|c|c|c}
\hline
           & PSNR-$\mu$ & PSNR-l & SSIM-$\mu$ & SSIM-l & HDR-VDP-2 \\ \hline
w/o gating & 43.14  & 41.03  & 0.9904 & 0.9868 &     64.88      \\ \hline
one gating & 43.44  & 41.42  & 0.9909 & 0.9882 &    67.13   \\ \hline
Ours       & 43.61  & 41.74  & 0.9909 & 0.9881 & 66.96     \\ \hline
\end{tabular}
}
\caption{Ablation experimental results to verify the effectiveness of the gating mechanism}
\label{table_ablation_gating}
\end{table}

The gating mechanism is an important component in our model. Ablation study is conducted in the gating mechanism as follows.

\textbf{w/o gating}: The gating mechanism is not used in the feed forward network of all transformer layers in the model, that it, our GST unit degenerate to the vanilla Swin Transformer.

\textbf{one gating}: The gating mechanism is only used in the first Swin Transformer layers subsequent to the embedding layer, but not used for other layers. 

 Table \ref{table_ablation_gating} shows the results of the ablation experiments, where the model is trained for 20 epochs. By removing the gating mechanism, the network relies on self-attention for image alignment, resulting in the lowest performance. On top of it, adding gates to low level layers notably improves the HDR reconstruction. Furthermore, by integrating the gating mechanism with all Swin Transformer layers, the model effectively inpaints information in non-aligned regions and obtains the highest HDR reconstruction results, thus validates the effectiveness of the gating mechanism in our model.


%\section{}
%\label{sec:resDir}


\section{Conclusion}
\label{sec:conclusion}
% <>
Since its advent in 1931, Koopman operator theory \cite{koopman:1931} has only recently been actively utilized for solving practical problems, thanks to the introduction of the DMD algorithm in 2008 \cite{schmid:2008}. Since then, a multitude of DMD algorithm variations have risen to prominence and found utility across various fields. A notable feature of our survey paper was reviewing and categorizing the results of over 100 research papers based on both application and algorithm type in smart mobility and vehicle engineering  (see Table~\ref{tab1} and Section~\ref{sec:vehicApp}).  Additionally, this survey paper identified potential research gaps in smart mobility and vehicular engineering applications (Remarks~\ref{remGap1}--\ref{remGap6}). Finally, this review paper discussed theoretical aspects of Koopman operator theory that have been largely neglected by the smart mobility and vehicle engineering community and yet have large potential for contributing to solving open problems in these areas (see Section~\ref{subsec:theorIssue}).

\noindent{\textbf{Future Research Directions.}}	Given the emergence of cyber-threats against connected and autonomous vehicles as well as robotic systems (see, e.g.,~\cite{nekouei2021randomized,mohammadi2022generation}), a future research direction might include utilizing Koopman operator-based algorithms for designing cyber-resilient vehicular and smart mobility applications (see, e.g.,~\cite{taheri2022data} for a related line of research). Another potential research direction is using Koopman operator-based algorithms for predicting the motion of vulnerable road users (VRUs), e.g., pedestrians and cyclists (see, e.g.,~\cite{pool2019context,scholler2020constant}). Finally, rehabilitation robotics and robotic exoskeletons can be the benefactors of the predictive capabilities of Koopman operator-based algorithms for detecting tripping events and/or system  identification in various modes of locomotion (see, e.g.,~\cite{kumar2019extremum,aprigliano2019pre}).



%Fig. 1 depicts the accumulation of such algorithms since 2014, which are particular to vehicle engineering and smart mobility, i.e., the focus of this review. Table 1 summarizes the varieties of relevant algorithms developed in those studies. Furthermore, we have highlighted theoretical issues, whose expansion will have potential applications to the wide research area of smart mobility and vehicle engineering.  

%Although fairly comprehensive, we have found several gaps in this research area. In particular, we could not find any studies related to elevators, robots/vehicles employing crawling, slithering, hopping or peristaltic locomotion, arctic or special-terrain vehicles such as those employing screws or tracks, hovercraft and other amphibious vehicles or subsystems which tolerate flexible environments, classification or guidance systems related to vehicles for drilling or agriculture, or for current-ripple, power-split, battery health monitoring, nuclear propulsion, exoskeletons/prosthetics, personal mobility, motorsports, specialized rovers or similar open problems in emerging areas.  These examples are, of course, not exhaustive.  
%
%The purely data-driven nature of Koopman operators holds the promise of capturing unknown and complex dynamics for reduced-order model generation and system identification, through which the rich machinery of linear control techniques can be utilized. The emergent nature of the smart mobility and vehicular-related applications, where  the Koopman operator  in each particular application needs to be approximated, implies that the development of various Koopman operator approximation  algorithms is expected to grow along with the vehicular problems they aim to solve.  Given the ongoing development of this research area and the many existing open problems in the fields of smart mobility and vehicle engineering, a survey of techniques and open challenges of applying Koopman operator theory to this vibrant area is warranted.  To the best of our knowledge, this survey paper is the \emph{first of its kind} reviewing the applications of Koopman operator theory within a focused research area, namely, smart mobility and vehicle engineering applications. A \emph{notable feature} of our survey paper is reviewing and categorizing the results of over 100 research papers based on both application and algorithm type  (see Tables~\ref{tab1}--~\ref{tab4} and Section~\ref{sec:vehicApp}) that are concerned with the applications of Koopman operator theory to the field of smart mobility and vehicular engineering. Such a \emph{comprehensive and  detailed categorization} will be beneficial to the research practitioners working in the field.  Furthermore, this review paper discusses theoretical aspects of Koopman operator theory that have been largely neglected by the smart mobility and vehicle engineering community and yet have large potential for contributing to solving open problems in these areas. Additionally, our survey paper seeks to \emph{identify gaps} in the smart mobility and vehicle engineering research where new and existing Koopman operator-based methods have the potential to further develop and address unsolved problems  potentially benefiting from the perspectives of nonlinear system identification, control, global linearization, and the predictive powers that Koopman operator theory has to offer (see, e.g., Remarks~\ref{remGap1}--\ref{remGap6}). 


% Acknowledgments---Will not appear in anonymized version
%\acks{We thank a bunch of people and funding agency.}
\bibliographystyle{abbrvnat}  
\bibliography{MVE}


\appendix
\section{More Details on the Experiments}
\subsection{Implementations of Baselines} \label{appendix_implementation_of_baselines}
\paragraph{Objectives of PINN}
\begin{itemize}
	\item For the vorticity equation of the 2D Navier-Stokes equation, let $\vu: [0, T]\times \sR^2 \rightarrow \sR^2$ be the velocity field (this should not be confused with the velocity field of the continuity equation) such that $\nabla \cdot \vu = 0$, i.e. $\vu$ is divergence-free, and let $\omega = \nabla \times \vu \in \sR$ be the vorticity.
	We have
	\begin{align}
		\frac{\partial \omega}{\partial t} + \nabla \cdot \left(\omega\vu\right) =&\ \nu \Delta \omega,\\
		\omega =&\ \nabla \times \vu.
	\end{align}
	We use this form to construct the objective for the PINN method
	\begin{equation}
		\int_0^T \|\frac{\partial \omega}{\partial t} + \nabla \cdot \left(\omega\vu\right) - \nu \Delta \omega\|_{\gL(\Omega)^2}^2 + \|\omega - \nabla \times \vu\|_{\gL\gL(\Omega)^2} d t,
	\end{equation}
	where $\gL^2(\Omega)$ denotes the functional $\gL^2$ norm on the domain $\Omega = [-2, 2]^2$.
	\item For the MVE with Coulomb interaction, let $g$ be the Coulomb potential defined in \eqref{eqn_coulomb_interaction}. We have that $\psi = g \ast \rho$ is the solution to the Poisson equation $\Delta \psi = - \rho$ and $K * \rho = - \nabla \psi$.
	We have
	\begin{align}
		\frac{\partial \rho}{\partial t} + \nabla \cdot \left(\rho\cdot (-\nabla \psi) \right) =&\ \nu \Delta \rho \\
		\Delta \psi =&\ -\rho.
	\end{align}
	Expand the the divergence to obtain
	\begin{align}
		\frac{\partial \rho}{\partial t} + \nabla \rho \cdot (-\nabla \psi) +  \rho\cdot (-\Delta \psi) =&\ \nu \Delta \rho \\
		\Delta \psi =&\ -\rho.
	\end{align}
	Now plug in the $\Delta \psi = -\rho$ to arrive at the following equivalent form
	\begin{align}
		\frac{\partial \rho}{\partial t} + \nabla \rho \cdot -\nabla \psi +  \rho^2 =&\ \nu \Delta \rho \\
		\Delta \psi =&\ -\rho.
	\end{align}
	We use this form to construct the objective for the PINN method.
	\begin{equation}
		\int_0^T \|\frac{\partial \rho}{\partial t} + \nabla \rho \cdot -\nabla \psi +  \rho^2 - \nu \Delta \rho\|_{\gL^2(\Omega)}^2 + \|\Delta \psi + \rho\|_{\gL^2}^2,
	\end{equation}
	where $\gL^2(\Omega)$ denotes the functional $\gL^2$ norm on the domain $\Omega = [-1, 1]^2$.
\end{itemize}

\paragraph{DRVN} In the original paper \citep{zhang2022drvn}, only the Biot-Savart kernel is concerned. We can easily extend the DRVN method to the Coulomb case by setting $K$ to be the kernel defined in \eqref{eqn_coulomb_interaction}.

\subsection{Examples with an Explicit Solution} \label{appendix_explicit_solution}
In this section, we verify the explicit solutions discussed in the experiment section.
\paragraph{Lamb-Oseen Vortex on the whole domain $\sR^2$.}
Recall that we consider the 2D Navier-Stokes equation (the MVE with the Biot-Savart interaction kernel (\ref{eqn_nse})).
The Lamb-Oseen Vortex model states that, if $\rho_0 = \mathcal{N}(0, \sqrt{2\nu t_0}\mI_2)$ for some $t_0 \geq 0$, then we have $\rho_t(\vx) = \mathcal{N}(0, \sqrt{2\nu (t+t_0)}\mI_2)$.

To verify this, define $\vu_t(\vx) = \frac{1}{\sqrt{\nu (t+t_0)}}\vv(\frac{\vx}{\sqrt{\nu (t+t_0)}})$, where 
\begin{equation}
	\vv(\vx) = \frac{1}{2\pi}\frac{\vx^{\perp}}{\|\vx\|^2}\left(1 - \exp(-\frac{1}{4}\|\vx\|^2)\right).
\end{equation}
One can easily check that $\nabla \cdot \vu_t \equiv 0$ and hence there exists a function $\psi_t$ such that $\nabla^\perp \psi_t = -\vu_t$, where $\nabla^\perp$ denotes the perpendicular gradient, defined as $\nabla^\perp = (-\partial_{\vx_2}, \partial_{\vx_2})$, and $\psi_t$ is called the stream function in the literature of fluid dynamics.
Moreover, one can verify that $\nabla \times \vu_t = \rho_t$ where $\nabla \times$ denotes the curl of a 2D velocity field, defined as $\nabla \times \vu(\vx) = \partial \vu_2 / \partial \vx_1 - \partial \vu_1/\partial \vx_2$.
Together we have
\begin{equation}
	\Delta \psi_t = -\rho_t,
\end{equation}
i.e., the stream function $\psi_t$ is the solution to the 2D Poisson equation with a source term $\rho_t$.

Under the boundary condition that $\psi_t(\vx) \rightarrow 0$ for $\|\vx\|\rightarrow \infty$, we can express $\psi_t$ via the unique Green function $G(\vx) = \frac{1}{2\pi}\ln \|\vx\|$ as 
\begin{equation}
	\psi_t(\vx) = G \ast \rho_t = \frac{1}{2\pi}\int \ln\|\vx - \vy\| \rho_t(\vy) d \vy.
\end{equation}
Consequently, by taking the perpendicular gradient, we obtain
\begin{equation}
	\vu_t = \nabla^\perp \psi_t = \frac{1}{2\pi}\int \frac{(\vx - \vy)^\perp}{\|\vx - \vy\|^2} \rho_t(\vy) d \vy = K \ast \rho_t.
\end{equation}
Finally, by plugging the expressions of $\rho_t$ and $\vu_t = K \ast \rho_t$ in the MVE (\ref{eqn_MVE}), we verified the Lamb-Oseen vortex.
%\paragraph{Taylor-Green Vortex on the box $[0, 2\pi]^2$ with a periodic boundary condition.}
%Recall that we consider the 2D Navier-Stokes equation (the MVE with the scaled periodic Biot-Savart interaction kernel (\ref{eqn_K_periodic_biot_savart})) on the box $[0, 2\pi]^2$ with a periodic boundary condition.
%The Taylor-Green Vortex model stats that if $\rho_0(\vx) = \frac{1}{4\pi^2}\left(\cos \vx_1 \cos \vx_2 \exp(-2\nu t_0) + 1\right)$ for some $t \geq 0$, then we have 
%\begin{equation}
%	\rho_t(\vx) = \frac{1}{4\pi^2}\left(\cos \vx_1 \cos \vx_2 \exp(-2\nu (t+t_0)) + 1\right).
%\end{equation}
%
%First, one can easily check that $\rho_t(\vx)$ is a density function on $[0, 2\pi]^2$ for all $t\in[0, T]$, i.e. it is non-negative and integrates to 1. To verify this vortex model, define $\vu = (\vu_1, \vu_2)$ as follows
%\begin{equation}
%	\vu_1(\vx, t) = \cos \vx_1 \sin \vx_2 \exp(-2\nu t), \vu_2(\vx, t) = -\sin \vx_1 \cos \vx_2 \exp(-2\nu t).
%\end{equation}
%One can easily check that $\nabla\cdot\vu(\vx, t) \equiv 0$, which implies the exists of some stream function $\psi_t$ such that $\nabla^\perp \psi_t(\vx) = -\vu(\vx, t)$. Moreover, one can check that 

\paragraph{Barenblatt solutions for the MVE with Coulomb Interaction.} 
Recall that we consider the MVE with the Coulomb interaction kernel (\ref{eqn_coulomb_interaction}) for $d=3$ and set the diffusion coefficient $\nu = 0$, i.e.
\begin{equation}
	\frac{\partial \rho_t}{\partial t} + \nabla \cdot \left( \rho_t \cdot -\nabla \psi_t \right) = 0
\end{equation}
where $\psi_t$ is the solution to the Poisson equation $\Delta \psi_t = -\rho_t$. The Barenblatt solution of the above MVE is stated as follows: If $\rho_0 = \Uniform[\|\vx\|\leq (\frac{3}{4\pi}t_0)^{1/3}]$ for some $t_0 \geq 0$, then we have 
\begin{equation}
	\rho_t = \Uniform[\|\vx\|\leq (\frac{3}{4\pi}(t+t_0))^{1/3}]
\end{equation}
We now verify this solution.


Recall that the volume of a three dimensional Euclidean ball with radius $R$ is $\frac{4\pi}{3}R^3$. Hence we can write the density function as $\rho_t(x) = \frac{1}{t+t_0} \chi_{\|\vx\|\leq (\frac{3}{4\pi}(t+t_0))^{1/3}}$, where $\chi_\sX$ is a function that takes value $1$ for $\vx \in \sX$ and takes value $0$ for $\vx \notin \sX$.
Take 
\begin{equation}
	\psi_t(\vx) = \begin{cases}
		\frac{2(\frac{3}{4\pi}(t+t_0))^{2/3}-\|\vx\|^2}{6(t+t_0)}, & \|\vx\| \leq (\frac{3}{4\pi}(t+t_0))^{1/3},\\
		\frac{1}{8\pi\|\vx\|}, & \|\vx\| > (\frac{3}{4\pi}(t+t_0))^{1/3}.
	\end{cases}
\end{equation}
It can be verified that the Poisson equation $\Delta \psi_t = -\rho_t$ holds (note that $\Delta \|\vx\|^{-1} = 0$, i.e. $\|\vx\|^{-1}$ is a harmonic function for $d=3$).
Consequently, for a fixed time stamp $t$ and any $\|\vx\| \leq (\frac{3}{4\pi}(t+t_0))^{1/3}$ we have 
\begin{align}
	\frac{\partial \rho_t}{\partial t}(\vx) + \nabla \cdot \left( \rho_t(\vx) \cdot -\nabla \phi_t(\vx) \right) = - \frac{1}{(t+t_0)^2} + \frac{1}{(t+t_0)^2} = 0,
\end{align}
which verifies this solution.

\section{Adjoint Method} \label{appendix_adjoint_method}
Consider the ODE system
\begin{align*}
	\dot s(t) =&\ \psi(s(t), t, \theta) \\
	s(0) =&\ s_0,
\end{align*}
and the objective loss
\begin{equation}
	\ell(\theta) = \int_0^T g(s(t), t, \theta) \ud t.
\end{equation}
The following proposition computes the gradient of $\ell$ w.r.t. $\theta$.
We omit the parameters of the functions for succinctness. We note that all the functions in the integrands should be evaluated at the corresponding time stamp $t$, e.g. $b^\top \frac{\partial h}{\partial \theta}\ud t$ abbreviates for $b(t)^\top \frac{\partial}{\partial \theta}h(\xi(t), x(t), t, \theta)\ud t$.
\begin{proposition}
	\begin{equation}
		\frac{\ud \ell}{\ud \theta} = \int_{0}^T a^\top \frac{\partial\psi}{\partial\theta} + \frac{\partial g}{\partial\theta}\ud t.
	\end{equation}
	where $a(t)$ is solution to the following final value problems
	\begin{equation}
		\dot a^\top + a^\top \frac{\partial\psi}{\partial s} + \frac{\partial g}{\partial s} = 0, a(T) = 0, 
	\end{equation}
\end{proposition}
\begin{proof}
	Let us define the Lagrange multiplier function (or the adjoint state) $a(t)$ dual to $s(t)$.
	Moreover, let $L$ be an augmented loss function of the form
	\begin{equation}
		L = \ell - \int_0^T a^\top(\dot s - \psi) \ud t.
	\end{equation}
	Since we have $\dot s(t) = \psi(s(t), t, \theta)$ by construction, the integral term in $L$ is always null and $a$ can be freely assigned while maintaining $\ud L/\ud \theta = \ud \ell/\ud \theta$.
	Using integral by part, we have
	\begin{equation}
		\int_0^T a^\top\dot s\ \ud t = a(t)^\top s(t)\vert_0^T - \int_0^T s^\top \dot a\ \ud t.
	\end{equation}
	We obtain
	\begin{align}
		L = - a(t)^\top s(t)\vert_0^T + \int_0^T \dot a^\top s + a^\top \psi + g\ \ud t.
	\end{align}
	
	Now we compute the gradient of $L$ w.r.t. $\theta$ as
	\begin{equation*}
		\frac{\ud \ell}{\ud \theta} =  \frac{\ud L}{\ud \theta} = - a(T)^\top\frac{\ud x(T)}{\ud \theta}  + \int_0^T \dot a^\top \frac{\ud s}{\ud \theta} + a^\top \left(\frac{\partial\psi}{\partial\theta} + \frac{\partial\psi}{\partial s} \frac{\ud s}{\ud \theta} \right) \ud t
		+ \int_0^T \frac{\partial g}{\partial s} \frac{\ud s}{\ud \theta} +  \frac{\partial g}{\partial\theta}\ud t,
	\end{equation*}
	which by rearranging terms yields to
	\begin{align*}
		\frac{\ud \ell}{\ud \theta} = \frac{\ud L}{\ud \theta} = - a(T)^\top\frac{\ud x(T)}{\ud \theta} + \int_{0}^T a^\top \frac{\partial \psi}{\partial \theta} +  \frac{\partial g}{\partial \theta}\ud t
		+ \int_0^T \left(\dot a^\top + a^\top \frac{\partial \psi}{\partial s} +  \frac{\partial g}{\partial s}\right)\frac{\ud s}{\ud \theta} \ud t.
	\end{align*}
	Now by taking $a$ satisfying the \emph{final} value problems
	\begin{equation}
		\dot a^\top + a^\top \frac{\partial \psi}{\partial s} + \frac{\partial g}{\partial s} = 0, a(T) = 0, 
	\end{equation}
	we derive the result
	\begin{equation}
		\frac{\ud \ell}{\ud \theta} = \int_{0}^T a^\top \frac{\partial \psi}{\partial \theta} + \frac{\partial g}{\partial \theta}\ud t.
	\end{equation}
\end{proof}
\section{Detailed Proofs}\label{detailed proof}



%%%%%%%%%%%
\begin{proof}[Proof of Lemma  \ref{TimeEvolKLMV}] 
	Recall the McKean-Vlasov  \eqref{eqn_MVE_CE} and the continuity  \eqref{eqn_CE_as_MVE}. We simply write that $\rho_t = \rho_t^f$ and omit the integration domain $\X$.  Then 
	\[
	\begin{split}
		& \frac{\ud }{\ud t } \int \rho_t \log \frac{\rho_t }{\bar \rho_t }  = \int \partial_t \rho_t \log \frac{\rho_t}{\bar \rho_t} +  \int \rho_t   \partial_t \log \rho_t - \int \rho_t \partial_t \log \bar \rho_t  \\
		& = - \int  \udiv \Big(    \rho_t  \Big(\Big[- \nabla V (x) + K *  \rho_t -  \nu \nabla \log  \rho_t \Big] + \delta_t  \Big) \Big) \log \frac{\rho_t}{\bar \rho_t} \\
		& +  \int \frac{\rho_t}{\bar \rho_t}  \udiv \Big(   \bar \rho_t \Big(- \nabla V (x) + K * \bar \rho_t - \nu \nabla \log \bar \rho_t \Big)\Big), 
	\end{split}
	\]
	where we note that $\int \rho_t \partial_t \log  \rho_t= \int \partial_t \rho_t = 0$ since the total mass is preserved over time. 
	By integration by parts, one has 
	\[
	\begin{split}
		& \frac{\ud }{\ud t } \int \rho_t \log \frac{\rho_t }{\bar \rho_t }  = I_1 + I_2 + I_3+ \int \rho_t \delta_t \cdot \nabla \log \frac{\rho_t}{\bar \rho_t},
	\end{split}
	\]
	where $I_1, I_2, I_3$ denote the linear, nonlinear interaction, and diffusion parts separately. More precisely, by integration by parts,
	\[
	\begin{split}
		I_1 & = \int \udiv (\rho_t \nabla V(x)) \log \frac{\rho_t}{\bar \rho_t} - \int \frac{\rho_t}{\bar \rho_t} \udiv (\bar \rho_t \nabla V(x))  \\
		& = - \int \rho_t \nabla V(x) \cdot \nabla \log \frac{\rho_t}{\bar \rho_t} + \int \bar \rho_t  \nabla  \frac{\rho_t }{\bar \rho_t} \cdot  \nabla V(x) = 0. 
	\end{split}
	\]
	And 
	\[
	\begin{split} 
		I_2 & = - \int \udiv (\rho_t K * \rho_t ) \log \frac{\rho_t}{\bar \rho_t } + \int \frac{\rho_t }{\bar \rho_t} \udiv(\bar \rho_t K * \bar \rho_t)  \\
		& =  \int \rho_t K * \rho_t \nabla \log \frac{\rho_t}{\bar \rho_t } - \int \bar \rho_t K * \bar \rho_t \cdot \nabla \frac{\rho_t}{\bar \rho_t } \\
		& = \int \rho_t \nabla \log \frac{\rho_t}{\bar \rho_t} \cdot K * (\rho_t - \bar \rho_t). 
	\end{split} 
	\]
	Given that the kernel $K$ is divergence free, that is $\udiv K = 0$, one further has 
	\begin{equation}\label{NSE_new}
	\begin{split}
		I_2 & = - \int \rho_t \nabla \log \bar \rho_t  \cdot K *(\rho_t - \bar \rho_t ) + \int \nabla \rho_t \cdot K *(\rho_t - \bar \rho_t ) \\
		& = - \int \rho_t \nabla \log \bar \rho_t \cdot K * (\rho_t - \bar \rho_t). 
	\end{split}
	\end{equation}
	Note that this modification will be used in the proof in the 2D Navier-Stokes case. 
	Finally, all diffusion terms sum up to $I_3$ which can be further simplified as 
	\[
	\begin{split} 
		I_3  & = \nu \int \udiv (\rho_t \nabla \log  \rho_t ) \log \frac{\rho_t }{\bar \rho_t} - \nu \int \frac{\rho_t }{\bar \rho_t} \udiv (\bar \rho_t \nabla \log \bar \rho_t ) \\
		& = - \nu \int \rho_t \nabla \log \rho_t \cdot \nabla \log \frac{\rho_t}{\bar \rho_t} + \nu \int \bar \rho_t \nabla \log \bar  \rho_t \cdot  \nabla \frac{\rho_t }{\bar \rho_t } \\
		& = - \nu \int \rho_t |\nabla \log \frac{\rho_t }{\bar \rho_t}|^2. 
	\end{split} 
	\]
	We thus complete the proof of Lemma \ref{TimeEvolKLMV}.
	
	
	
\end{proof}







%%% Proof of Time Evolution of Modulated Energy
\begin{proof}[Proof of Lemma \ref{ModuEnergyEvo}]Recall that $K= - \nabla g$.  For simplicity, we write that $\rho_t = \rho_t^f$. Then 
	\[
	\begin{split} 
		& \frac{\ud }{\ud t } F(\rho_t, \bar \rho_t) = \frac{\ud }{\ud t } \frac{1 }{2} \int_{\mathcal{X}^2} g (x- y) \ud (\rho_t - \bar \rho_t)^{\otimes 2 } (x, y)  \\
		& = \int_\mathcal{X} g *(\rho_t - \bar \rho_t) (x) \big(\partial_t \rho_t (x) - \partial_t \bar \rho_t (x)  \Big) \ud x  \\
		& = \int g * (\rho_t - \bar \rho_t ) (x) \udiv \Big\{  \rho_t \Big( [\nabla V (x) - K * \rho_t + \nu \nabla  \log \rho_t ]  -  \delta_t \Big)   \\ & \quad \qquad \qquad \qquad \qquad - \bar \rho_t \Big( \nabla V(x) - K * \bar \rho_t + \nu \log \bar \rho_t \Big) \Big\} \\
		& = J_1+ J_2+ J_3 + J_4, 
	\end{split} 
	\]
	where $J_1, J_2, J_3, J_4$ denote the perturbation term, the linear difference term, the nonlinear difference term, and the diffusion term respectively. The perturbation term $J_1$ reads
	\[
	J_1 = - \int_{\mathcal{X}} g*(\rho_t - \bar \rho_t ) \udiv (\rho_t \delta_t ) = - \int_{\mathcal{X}}  \rho_t  K * (\rho_t - \bar \rho_t )  \cdot \delta_t. 
	\]
	By integration by parts, the linear difference term can be written as 
	\[
	\begin{split}
		&J_2 = \int_{\mathcal{X}}  g *(\rho_t - \bar \rho_t ) \udiv\Big( (\rho_t - \bar \rho_t ) \nabla V \Big)  = \int_{\mathcal{X}} K * (\rho_t - \bar \rho_t) (\rho_t - \bar \rho_t ) \nabla V \\
		&  = \frac{1}{2} \int_{\mathcal{X}^2} K (x - y)(\nabla V (x) - \nabla  V(y)) \ud (\rho_t - \bar \rho_t )^{\otimes 2}(x, y), 
	\end{split}
	\]
	where the last equality is true since $K = - \nabla g $ 
	is an odd function and we do the symmetrization trick, i.e. exchanging the role of $x$ and $y$ to another term and then taking the average. 
	
	The nonlinear difference term reads 
	\[
	\begin{split}
		& J_3 = - \int_{\mathcal{X}} g*(\rho_t - \bar \rho_t) \udiv\Big(\rho_t K * \rho_t - \bar \rho_t K * \bar \rho_t  \Big)  \\
		& = - \int_{\mathcal{X}} K * (\rho_t - \bar \rho_t ) (\rho_t K * (\rho_t - \bar \rho_t) - \int_{\mathcal{X}}  K * (\rho_t - \bar \rho_t) (\rho_t - \bar \rho_t ) K * \bar \rho_t  \\
		& = - \int_{\mathcal{X}}  \rho_t |K * (\rho_t - \bar \rho_t)|^2 - \frac{1}{2} \int K(x-y) (K * \bar \rho_t (x) - K * \bar \rho_t (y) ) \ud (\rho_t - \bar \rho_t)^{\otimes 2}(x, y), 
	\end{split}
	\]
	where again in the last term we do the symmetrization. 
	
	The diffusion term reads 
	\[
	\begin{split}
		& J_4 =  \nu \int g * (\rho_t - \bar \rho_t ) \udiv \Big( \rho_t  \nabla \log \rho_t - \bar \rho_t \nabla \log \bar \rho_t  \Big)  \\
		& = \nu \int K * (\rho_t - \bar \rho_t)  \rho_t \nabla \log \frac{\rho_t }{\bar \rho_t }   + \nu \int K * (\rho_t - \bar \rho_t ) (\rho_t - \bar \rho_t ) \nabla \log \bar \rho_t  \\
		& = \nu \int_{\mathcal{X}} \rho_t K *(\rho_t - \bar \rho_t)  \cdot \nabla \log \frac{\rho_t}{\bar \rho_t}  \\
		& \qquad + \frac{\nu }{2} \int_{\mathcal{X}^2} K (x-y)(\nabla \log \bar \rho_t (x) - \nabla \log \bar \rho_t (y)) \ud (\rho_t - \bar \rho_t )^{\otimes 2}. 
	\end{split} 
	\]
	To sum it up, we prove the thesis.  
	
	
	
	
\end{proof}

\subsection{Proof of the 2D Navier-Stokes case} 
Now we proceed to control the growth of the KL divergence  $\mathbf{KL}(\rho_t^f \vert \bar \rho_t )$ for the 2D Navier-Stokes case. 
Since the Biot-Savart law is divergence free, by \eqref{NSE_new} in the proof of Lemma \ref{TimeEvolKLMV}, one has 
\begin{equation}\label{KL_NS_Evo}
	\frac{\ud }{\ud t} \int_{\Pi^d} \rho_t \log \frac{\rho_t }{\bar \rho_t} = - \nu \int_{ \Pi^d}\rho_t |\nabla \log \frac{\rho_t}{\bar \rho_t}|^2 - \int_{\Pi^d} \rho_t K * (\rho_t - \bar \rho_t ) \cdot \nabla  \log \bar   \rho_t +  \int_{\Pi^d} \rho_t \delta_t \cdot \nabla \log \frac{\rho_t }{\bar \rho_t }. 
\end{equation}

Recall that we  write the kernel $K= (K_1, \cdots, K_d)$ and its component $K_i = \sum_{j=1}^d \partial_{x_j} U_{ij}(x)$, where $U= (U_{ij})_{1 \leq i, j \leq d}$ is a matrix-valued potential function for instance can be defined as in  \eqref{eqn_K_as_divergence}. 
 Consequently 
\[
- \int \rho_t K * (\rho_t - \bar \rho_t) \cdot \nabla \log \bar \rho_t = - \sum_{i, j=1}^d \int \rho_t \partial_{x_j} U_{ij} * (\rho_t - \bar \rho_t) \partial_{x_i} \log \bar \rho_t, 
\]
which equals to 
\[
\sum_{i, j=1}^d  \int U_{ij}* (\rho_t - \bar \rho_t) \partial_{x_j}\big( \frac{\rho_t}{\bar \rho_t} \partial_{x_i} \bar \rho_t \big) = A + B 
\]
by integration by parts, 
where further 
\[
A= \sum_{i, j=1}^d  \int V_{i  j} * (\rho_t - \bar \rho_t) \, \partial_{x_i} \bar \rho_t \,  \partial_{x_j} \, \frac{\rho_t}{\bar \rho_t} = \int U * (\rho_t - \bar \rho_t) : \nabla \bar \rho_t \otimes \nabla \frac{\rho_t }{\bar \rho_t}, 
\]
and 
\[
B = \sum_{i, j=1}^d \int \rho_t U_{ij} *(\rho_t - \bar \rho_t) \frac{\partial_{x_i x_j}^2 \bar \rho_t }{\bar \rho_t} = \int \rho_t U * (\rho_t - \bar \rho_t) : \frac{\nabla^2 \bar \rho_t}{\bar \rho_t}. 
\]
Noticing that $\nabla \frac{\rho_t}{\bar \rho_t} =\frac{\rho_t}{\bar \rho_t} \nabla \log \frac{\rho_t }{\bar \rho_t }$, one estimates $A$ as follows 
\[
\begin{split}
	A & = \int  \rho_t U * (\rho_t - \bar \rho_t) : \nabla \log \bar \rho_t \otimes \nabla \log  \frac{\rho_t }{\bar \rho_t} \\
	&  \leq \frac{\nu}{4} \int \rho_t |\nabla \log \frac{\rho_t }{\bar \rho_t}|^2 + \frac{1}{\nu} \int \rho_t | (\nabla \log \bar \rho_t)^\top  \, U*(\rho - \bar \rho) |^2 \\
	& \leq \frac{\nu}{4} \int \rho_t |\nabla \log \frac{\rho_t }{\bar \rho_t}|^2 + \frac{1}{\nu } \|U\|_{L^\infty}^2 \|\nabla \log \bar \rho_t\|_{L^\infty}^2  \|\rho_t - \bar \rho_t\|_{L^1}^2,  
\end{split} 
\]
and again by  Csisz\'ar–Kullback–Pinsker inequality, one has that 
\[
A \leq \frac{\nu}{4} \int \rho_t |\nabla \log \frac{\rho_t }{\bar \rho_t}|^2 + \frac{2}{\nu } \|U\|_{L^\infty}^2 \|\nabla \log \bar \rho_t\|_{L^\infty}^2 \int \rho_t \log \frac{\rho_t}{\bar \rho_t}. 
\]

Now it only remains to control $B$. Recall the following famous Gibbs inequality 
\begin{lemma}[Gibbs inequality]\label{Gibbs} For any parameter $\eta > 0$, and  probability measures $\rho, \bar \rho \in \mathcal{P}(\mathcal{X}) \cap L^1(\mathcal{X})$, and $\phi$  a real-valued function defined on $\mathcal{X}$,  one has the following change of reference measure inequality 
	\[
	\int_{\mathcal{X}} \rho(x) \phi(x) \ud x    \leq \frac{1}{\eta } \Big( \int_{\mathcal{X}} \rho(x) \log \frac{\rho(x)}{\bar \rho(x) } \ud x  + \log \int_{\mathcal{X}} \bar \rho(x) \exp (\eta \phi (x))  \ud x \Big). 
	\]
	
\end{lemma} 
The proof of this inequality can be found in section 13.1 in  \citep{erdHos2017dynamical}. 

To control $B$, we write that $\phi= U * (\rho_t - \bar \rho_t) : \frac{\nabla^2 \bar \rho_t}{\bar \rho_t}$ and thus  $B = \int \rho_t  \phi$.  We choose a positive parameter $\eta>0$ such that 
\[
\frac{1}{\eta} = 2 \|U \|_{L^\infty} \Big\| \frac{\nabla^2 \bar \rho_t}{\bar \rho_t} \Big\|_{L^\infty}. 
\]
Now we apply Lemma \ref{Gibbs} to obtain that 
\[
B = \int \rho_t \phi \leq \frac{1}{\eta} \left( \int \rho_t  \log \frac{\rho_t }{\bar \rho_t} + \log \int \bar \rho_t \exp(\eta \phi )\right). 
\]
Note that $\eta >0$ is chosen so small such that  
% {\bf Case I : when $\int  \rho_t \log \frac{\rho_t}{\bar \rho_t} \leq 1$}. Now  
\[
\begin{split}
	\eta \| \phi\|_{L^\infty} &  \leq  \frac{1}{2 \|U \|_{L^\infty} \Big\| \frac{\nabla^2 \bar \rho_t}{\bar \rho_t}\Big\|_{L^\infty}}  \|U\|_{L^\infty}  \|\rho_t - \bar \rho_t\|_{L^1}  \Big\| \frac{\nabla^2 \bar \rho_t}{\bar \rho_t}\Big\|_{L^\infty}   \\
	& \leq \frac 1 2 \|\rho_t - \bar \rho_t \|_{L^1} \leq 1, 
\end{split} 
\]
since for two probability densities it always holds $\|\rho_t - \bar \rho_t \|_{L^1} \leq 2$. Consequently, applying the inequality $\exp(x) \leq 1 + x + \frac e 2 x^2$ for $|x|\leq 1$, we have 
\[
\int \bar \rho_t \exp(\eta \phi ) \leq \int \bar \rho_t \left(  1 + \eta \phi + \frac e 2 \eta^2 \phi^2 \right) \leq 1 + 0 + \frac{e}{2}  \Big( \frac 1 2 \| \rho_t - \bar \rho_t\|_{L^1} \Big)^2  \leq 1 + \frac{e}{4} \mathbf{KL}(\rho_t \vert \bar \rho_t), 
\]
where 
\[
\int \bar \rho_t \phi = \int U* (\rho_t - \bar \rho_t) : \nabla^2 \bar \rho_t =\int  \sum_{i, j=1}^d \partial_{x_i x_j}U * (\rho_t - \bar \rho_t) \bar \rho_t = \int \udiv K *(\rho_t - \bar \rho_t ) \bar \rho_t = 0, 
\]
since $\udiv K =0$. 

To sum it up, in particular since $\log (1 + x ) \leq x $ for $x > 0$, one has 
\[
B \leq \frac{1}{\eta} \Big(  1 + \frac e 4 \Big) \mathbf{KL}(\rho_t \vert \bar \rho_t ) \leq 4  \|U \|_{L^\infty} \Big\| \frac{\nabla^2 \bar \rho_t}{\bar \rho_t} \Big\|_{L^\infty} \mathbf{KL}(\rho_t \vert \bar \rho_t ). 
\]

Combining \eqref{KL_NS_Evo}, the estimates for $A$ and $B$, one has 
\begin{equation}\label{NSEFinal}
	\begin{split}
		& \frac{\ud }{\ud t} \int \rho_t \log \frac{\rho_t}{\bar \rho_t} \leq - \frac{3 \nu  }{4}  \int \rho_t |\nabla \log \frac{\rho_t}{\bar \rho_t}|^2 + M(t) \int \rho_t \log \frac{\rho_t }{\bar \rho_t} + \int \rho_t \delta_t \cdot \nabla \log \frac{\rho_t}{\bar \rho_t} \\
		& \leq - \frac \nu  2 \int \rho_t |\nabla \log \frac{\rho_t}{\bar \rho_t}|^2 +M(t) \int \rho_t \log \frac{\rho_t }{\bar \rho_t} + \frac{1}{\nu } \int \rho_t |\delta_t|^2 
	\end{split}
\end{equation}
where 
\[
M(t) = \frac{2}{\nu } \|U\|_{L^\infty}^2 \|\nabla \log \bar \rho_t\|_{L^\infty}^2 + 4  \|U \|_{L^\infty} \Big\| \frac{\nabla^2 \bar \rho_t}{\bar \rho_t} \Big\|_{L^\infty} = M(t; \nu, U, \bar \rho_t). 
\]
Since the matrix-valued potential function $U$ is bounded ($\|U(\vx)\|_{op}\leq1/4$ when $U$ takse the form (\ref{eqn_K_as_divergence})), and under suitable assumptions for the initial data $\bar \rho_0$ (for instance $\bar \rho_0 \in C^3$ and there exists $c>1$ s.t. $\frac 1 c \leq \bar \rho \leq c $), one can obtain $\sup_{t \in [0, T] } M(t) \leq M < \infty$. We  recall Theorem 2 in  \citep{guillin2021uniform} as below for completeness. 

\begin{theorem} Given the initial data $\bar \rho_0 \in C^\infty (\Pi^d)$, such that there exists $c>1$, $\frac{1}{c} \leq \bar \rho_0 \leq c$. Then the vorticity formulation of the 2D Navier-Stokes equation 
	\[
	\partial_t \bar\rho_t + \udiv (\bar \rho_t K * \bar \rho_t ) = \nu \Delta \bar \rho_t, \quad \bar \rho(0, x) = \bar \rho_0(x), 
	\]
	has a unique bounded solution $\bar \rho(t, x) \in C^\infty([0, \infty) \times \Pi^d)$, and for any $t>0$,  for any $x \in \Pi^d$,  it holds that $\frac{1}{c} \leq \bar \rho(t, x) \leq c$. 
	
\end{theorem}

Finally,  we simplify \eqref{NSEFinal} to obtain that 
\[
\frac{\ud }{\ud t} \int \rho_t \log \frac{\rho_t}{\bar \rho_t} 
\leq  M \int \rho_t \log \frac{\rho_t }{\bar \rho_t} + \frac{1}{\nu } \int \rho_t |\delta_t|^2, 
\]
where $M = \sup_{t \in [0, T] } M(t; \nu, U, \bar \rho_t)< \infty$. By Gronwall inequality, one finally obtains that 
\[
\sup_{t \in [0, T]} \int_{\Pi^d} \rho_t \log \frac{\rho_t}{\bar \rho_t} \ud x \leq \frac{1}{\nu} \exp(M T ) R(\theta). 
\]
%This completes the proof of Theorem \ref{NSMainEstimate} 

As noted in \citep{guillin2021uniform},in particular Corollary 2 there, one can improve the above time-dependent estimate ($\exp(MT)$) to uniform-in-time estimate by using Logarithmic Sobolev inequality. Indeed, given that $\frac 1 c \leq \bar \rho_t \leq c$, one has that
\begin{equation}
    \label{LogSoboIne}
    \int_{\Pi^d} \rho_t \log \frac{\rho_t}{\bar \rho_t} \ud x  \leq \frac{c^2}{8 \pi^2 } \int_{\Pi^d} \rho_t |\nabla_x \log \frac{\rho_t}{\bar \rho_t }|^2 \ud x. 
 \end{equation}
 Combining \eqref{LogSoboIne} and  \eqref{NSEFinal}, one obtains that 
 \[
\frac{\ud }{\ud t} \mathbf{KL}(\rho_t \vert \bar \rho_t) \leq \Big( M(t) - \frac{4 \pi^2 \nu  }{c^2} \Big) \mathbf{KL} (\rho_t \vert \bar \rho_t) +  \frac{1}{\nu } \int \rho_t |\delta_t|^2. 
 \]
Multiplying the factor $\exp(\frac{4 \pi^2 \nu  }{c^2} t - \int_0^t M(s) \ud s)$ and noting in particular $\mathbf{KL} (\rho_0 \vert \bar \rho_0) =0$,  one obtains that 
\[
\mathbf{KL}(\rho_t \vert \bar \rho_t) \leq \int_0^t \exp(\frac{4 \pi^2 \nu  }{c^2} (s- t) + \int_s^t M(u) \ud u ) f(s) \ud s. 
\]
Indeed, under the assumptions as in Theorem \ref{NSMainEstimate}, one has that there exists a universal $C>0$, such that 
\[
\int_0^\infty M(t) \ud t = C < \infty. 
\]
We thus immediately obtain that 
\[
\sup_{t \in [0, T] }\mathbf{KL}(\rho_t \vert \bar \rho_t) \leq  \frac{e^C}{\nu} \int_0^T \int_{\Pi^d} \rho_t |\delta_t|^2 \ud x \ud t. 
\]
This completes the proof of Theorem \ref{NSMainEstimate}. 


%Since now we are only care about computing solutions in a fixed time interval $[0, T]$, we do not seek to optimize the factor $\exp(MT)$. We leave the long time asymptotic analysis as a separate work. 
%\end{remark}

 


\paragraph{The McKean-Vlasov PDEs, i.e. \eqref{eqn_MVE}, with bounded interactions $K \in L^\infty$ }  
As mentioned in the main body of this article,  it is much easier to  obtain the stability estimate for the McKean-Vlasov PDE with bounded interactions. 
\begin{theorem}[Stability Estimate for McKean-Vlasov PDE with $K \in L^\infty$] \label{theorem_bounded_K}
	Assume that $K \in L^\infty$. One has the estimate that 
	\[
	\sup_{t \in [0, T]} \mathbf{KL}(\rho_t^f \vert \bar \rho_t ) \leq   \frac{1}{\nu}\exp\Big(\frac{2 \|K \|_{L^\infty}^2}{\nu } T  \Big)  R(f), 
	\]
	where we recall the self-consitency potential/loss function $R(\theta)$ reads 
	\[
	R(f) = \int_0^T \int_{\X} |f(t, x) + \nabla V (x) - K * \rho_t^f + \nu \nabla \log \rho_t^\theta  |^2 \ud \rho_t^f(x) \ud t. 
	\]
\end{theorem}

\begin{proof}
	Here we give the control of the growth of the KL divergence for systems with bounded kernels.  
	Applying Cauchy-Schwarz inequality twice for the entropy dissipation terms  in Lemma \ref{TimeEvolKLMV} to obtain 
	\[
	\int_{\Pi^d} \rho_t K * (\rho_t - \bar \rho_t ) \cdot \nabla  \log  \frac{\rho_t }{\bar \rho_t}  \leq \frac{\nu}{4} \int \rho_t |\nabla \log \frac{\rho_t}{\bar \rho_t}|^2 + \frac{1}{\nu } \int \rho_t |K * (\rho_t - \bar \rho_t)|^2, 
	\]
	and 
	\[
	\int_{\Pi^d} \rho_t \delta_t \cdot \nabla \log \frac{\rho_t }{\bar \rho_t } \leq \frac{\nu}{4} \int \rho_t |\nabla \log \frac{\rho_t}{\bar \rho_t}|^2 + \frac{1}{\nu } \int \rho_t |\delta_t|^2. 
	\]
	Furthermore, 
	\[
	\int \rho_t |K * (\rho_t - \bar \rho_t)|^2 \leq \|K \|_{L^\infty}^2  \|\rho_t - \bar \rho_t\|_{L^1}^2 \leq  2 \|K \|_{L^\infty}^2  \int \rho_t \log \frac{\rho_t}{\bar \rho_t }, 
	\]
	where the last inequality is simply the Csisz\'ar–Kullback–Pinsker inequality \citep{villani2009optimal}. Combining the above estimates, we obtain that given that $K \in L^\infty$, 
	\[
	\frac{\ud }{\ud t} \int_{\Pi^d} \rho_t \log \frac{\rho_t }{\bar \rho_t} = - \frac{\nu}{2}   \int_{ \Pi^d}\rho_t |\nabla \log \frac{\rho_t}{\bar \rho_t}|^2 + \frac{2 \|K \|_{L^\infty}^2}{\nu } \int \rho_t \log \frac{\rho_t}{\bar \rho_t } + \frac{1}{\nu } \int \rho_t |\delta_t|^2. 
	\]
	Currently, we are not interested in the long time behavior, so we first ignore the negative term above to obtain that 
	\[
	\frac{\ud }{\ud t} \int_{\Pi^d} \rho_t \log \frac{\rho_t }{\bar \rho_t}  \leq    \frac{2 \|K \|_{L^\infty}^2}{\nu } \int \rho_t \log \frac{\rho_t}{\bar \rho_t } + \frac{1}{\nu } \int \rho_t |\delta_t|^2. 
	\]
	By Gronwall inequality, we obtain that
	\[
	\int_{\Pi^d} \rho_t \log \frac{\rho_t }{\bar \rho_t} \leq  \frac{1}{\nu}\exp\Big(\frac{2 \|K \|_{L^\infty}^2}{\nu } t \Big)   \int_0^t \int \rho_s |\delta_s|^2 \ud x  \ud s. 
	\]
	
	\end{proof}

\subsection{The McKean-Vlasov equation with Coulomb interactions}

\begin{proof}[Proof of Theorem \ref{ThmCoul}]  We first prove the case when $\nu >0$. Applying Cauchy-Schwarz inequality to the right-hand side of $\frac{\ud }{\ud t} E(\rho_t^f, \bar \rho_t)$ in Lemma \ref{TimeEvoMFE},  one has  
\[
\begin{split}
	\frac{\ud }{\ud t } E(\rho_t^f, \bar \rho_t)  & \leq  \frac 1 2  \int_{\mathcal{X}} \rho_t^f \, |\delta_t |^2 \ud x  \\
	&  - \frac{1}{2} \int_{\mathcal{X}^2} K(x-y) \cdot \Big( \mathcal{A}[\bar \rho_t](x) - \mathcal{A}[\bar \rho_t](y) \Big) \ud (\rho_t^f - \bar \rho_t )^{\otimes 2 }(x, y). 
\end{split}
\]
By Lemma 5.2 in \cite{bresch2019modulated}, as long as the ground truth ``velocity field" $\mathcal{A}[\bar \rho_t]$ is Lipschitz, i.e. $\mathcal{A}[\bar \rho] \in W^{1, \infty}$, or equivalently  $\nabla^2 V \in W^{1, \infty}, \nabla^2 \log \bar \rho_t \in L^\infty, K * \bar \rho_t \in W^{1, \infty}$, using the particular structure introduced by the Coulomb interactions (note that $- \Delta g = \delta_0$ and $K = - \nabla g$), we have the estimate 
\[
\begin{split}
	&  - \frac{1}{2} \int_{\mathcal{X}^2} K(x-y) \cdot \Big( \mathcal{A}[\bar \rho_t](x) - \mathcal{A}[\bar \rho_t](y) \Big) \ud (\rho_t^f - \bar \rho_t )^{\otimes 2 }(x, y) \\
	& \leq C \|\nabla \mathcal{A}[\bar \rho_t]\|_{L^\infty}  F(\bar \rho_t^f, \bar \rho_t).   \\
\end{split} 
\] 
This estimate can be obtained either by Fourier method  \citep{bresch2019modulated} or by the stress-energy tensor approach as in \citep{serfaty2020mean}. 
We emphasize that those assumptions made on $(\bar \rho_t)_{t \in [0, T]}$  can be obtained by propagating similar conditions on the initial data $\bar \rho_0$. 
This estimate actually holds for more general choices of $g$ or  $K$. See more examples including Riesz kernels in \citep{bresch2019modulated}.  Moreover, the Lipschitz regularity of $\mathcal{A}[\bar \rho_t]$ can also be relaxed a bit. See for instance in \citep{rosenzweig2022mean}.

Combining  previous two estimates, one has 
	\[
	\frac{\ud }{\ud t } E(\rho_t^f, \bar \rho_t)   \leq  \frac 1 2  \int_{\mathcal{X}} \rho_t^f \, |\delta_t |^2 \ud x  + C C_1  F(\rho_t^f, \bar \rho_t) \leq \frac 1 2  \int_{\mathcal{X}} \rho_t^f \, |\delta_t |^2 \ud x +  C C_1   E(\rho_t^f, \bar \rho_t). 
	\]
	Then applying Gronwall inequality concludes the proof of the case when $\nu >0$.
	
   Now we prove the deterministic case when $\nu=0$. Now the relative entropy or KL divergence does not play a role since there is no Laplacian term in \eqref{eqn_MVE}.  Lemma \ref{ModuEnergyEvo} now reads 
   	\[
   \begin{split}
   	\frac{\ud }{\ud t } F(\rho_t^f, \bar \rho_t)&  =  - \int_{\mathcal{X}} \rho_t^f |K *(\rho_t^f - \bar \rho)|^2 - \int_{\mathcal{X}} \rho_t^f \, \delta_t \cdot K * (\rho_t^f - \bar \rho_t ) \\
   	&  - \frac{1}{2} \int_{\mathcal{X}^2} K(x-y) \cdot \Big( \mathcal{A}[\bar \rho_t](x) - \mathcal{A}[\bar \rho_t](y) \Big) \ud (\rho_t^f - \bar \rho_t )^{\otimes 2 }(x, y). \\
      \end{split}
   \]
   Applying Cauchy-Schwarz to the 2nd term in the right-hand side above, we obtain that 
   \[
    \begin{split}
    	\frac{\ud }{\ud t } F(\rho_t^f, \bar \rho_t) \leq    \frac 1 2 \int_{\mathcal{X}} \rho_t^f \, |\delta_t|^2     - \frac{1}{2} \int_{\mathcal{X}^2} K(x-y) \cdot \Big( \mathcal{A}[\bar \rho_t](x) - \mathcal{A}[\bar \rho_t](y) \Big) \ud (\rho_t^f - \bar \rho_t )^{\otimes 2 }(x, y). \\
    \end{split}
   \]
   Again assuming that the ``velocity field'' $\mathcal{A}[\bar \rho_t](\cdot)$ is Lipschitz will give us 
   \[
   \frac{\ud }{\ud t } F(\rho_t^f, \bar \rho_t) \leq  \frac 1 2 \int_{\mathcal{X}} \rho_t^f \, |\delta_t|^2  + C C_1 F(\rho_t^f, \bar \rho_t). 
   \] 
   Applying Gronwall inequality again conclude all the proof. 
   
   
   
    
\end{proof}

	


 


%\clearpage
%\input{math_command_information.tex}
\end{document}
