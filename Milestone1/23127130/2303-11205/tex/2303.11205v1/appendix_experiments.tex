\section{More Details on the Experiments}
\subsection{Implementations of Baselines} \label{appendix_implementation_of_baselines}
\paragraph{Objectives of PINN}
\begin{itemize}
	\item For the vorticity equation of the 2D Navier-Stokes equation, let $\vu: [0, T]\times \sR^2 \rightarrow \sR^2$ be the velocity field (this should not be confused with the velocity field of the continuity equation) such that $\nabla \cdot \vu = 0$, i.e. $\vu$ is divergence-free, and let $\omega = \nabla \times \vu \in \sR$ be the vorticity.
	We have
	\begin{align}
		\frac{\partial \omega}{\partial t} + \nabla \cdot \left(\omega\vu\right) =&\ \nu \Delta \omega,\\
		\omega =&\ \nabla \times \vu.
	\end{align}
	We use this form to construct the objective for the PINN method
	\begin{equation}
		\int_0^T \|\frac{\partial \omega}{\partial t} + \nabla \cdot \left(\omega\vu\right) - \nu \Delta \omega\|_{\gL(\Omega)^2}^2 + \|\omega - \nabla \times \vu\|_{\gL\gL(\Omega)^2} d t,
	\end{equation}
	where $\gL^2(\Omega)$ denotes the functional $\gL^2$ norm on the domain $\Omega = [-2, 2]^2$.
	\item For the MVE with Coulomb interaction, let $g$ be the Coulomb potential defined in \eqref{eqn_coulomb_interaction}. We have that $\psi = g \ast \rho$ is the solution to the Poisson equation $\Delta \psi = - \rho$ and $K * \rho = - \nabla \psi$.
	We have
	\begin{align}
		\frac{\partial \rho}{\partial t} + \nabla \cdot \left(\rho\cdot (-\nabla \psi) \right) =&\ \nu \Delta \rho \\
		\Delta \psi =&\ -\rho.
	\end{align}
	Expand the the divergence to obtain
	\begin{align}
		\frac{\partial \rho}{\partial t} + \nabla \rho \cdot (-\nabla \psi) +  \rho\cdot (-\Delta \psi) =&\ \nu \Delta \rho \\
		\Delta \psi =&\ -\rho.
	\end{align}
	Now plug in the $\Delta \psi = -\rho$ to arrive at the following equivalent form
	\begin{align}
		\frac{\partial \rho}{\partial t} + \nabla \rho \cdot -\nabla \psi +  \rho^2 =&\ \nu \Delta \rho \\
		\Delta \psi =&\ -\rho.
	\end{align}
	We use this form to construct the objective for the PINN method.
	\begin{equation}
		\int_0^T \|\frac{\partial \rho}{\partial t} + \nabla \rho \cdot -\nabla \psi +  \rho^2 - \nu \Delta \rho\|_{\gL^2(\Omega)}^2 + \|\Delta \psi + \rho\|_{\gL^2}^2,
	\end{equation}
	where $\gL^2(\Omega)$ denotes the functional $\gL^2$ norm on the domain $\Omega = [-1, 1]^2$.
\end{itemize}

\paragraph{DRVN} In the original paper \citep{zhang2022drvn}, only the Biot-Savart kernel is concerned. We can easily extend the DRVN method to the Coulomb case by setting $K$ to be the kernel defined in \eqref{eqn_coulomb_interaction}.

\subsection{Examples with an Explicit Solution} \label{appendix_explicit_solution}
In this section, we verify the explicit solutions discussed in the experiment section.
\paragraph{Lamb-Oseen Vortex on the whole domain $\sR^2$.}
Recall that we consider the 2D Navier-Stokes equation (the MVE with the Biot-Savart interaction kernel (\ref{eqn_nse})).
The Lamb-Oseen Vortex model states that, if $\rho_0 = \mathcal{N}(0, \sqrt{2\nu t_0}\mI_2)$ for some $t_0 \geq 0$, then we have $\rho_t(\vx) = \mathcal{N}(0, \sqrt{2\nu (t+t_0)}\mI_2)$.

To verify this, define $\vu_t(\vx) = \frac{1}{\sqrt{\nu (t+t_0)}}\vv(\frac{\vx}{\sqrt{\nu (t+t_0)}})$, where 
\begin{equation}
	\vv(\vx) = \frac{1}{2\pi}\frac{\vx^{\perp}}{\|\vx\|^2}\left(1 - \exp(-\frac{1}{4}\|\vx\|^2)\right).
\end{equation}
One can easily check that $\nabla \cdot \vu_t \equiv 0$ and hence there exists a function $\psi_t$ such that $\nabla^\perp \psi_t = -\vu_t$, where $\nabla^\perp$ denotes the perpendicular gradient, defined as $\nabla^\perp = (-\partial_{\vx_2}, \partial_{\vx_2})$, and $\psi_t$ is called the stream function in the literature of fluid dynamics.
Moreover, one can verify that $\nabla \times \vu_t = \rho_t$ where $\nabla \times$ denotes the curl of a 2D velocity field, defined as $\nabla \times \vu(\vx) = \partial \vu_2 / \partial \vx_1 - \partial \vu_1/\partial \vx_2$.
Together we have
\begin{equation}
	\Delta \psi_t = -\rho_t,
\end{equation}
i.e., the stream function $\psi_t$ is the solution to the 2D Poisson equation with a source term $\rho_t$.

Under the boundary condition that $\psi_t(\vx) \rightarrow 0$ for $\|\vx\|\rightarrow \infty$, we can express $\psi_t$ via the unique Green function $G(\vx) = \frac{1}{2\pi}\ln \|\vx\|$ as 
\begin{equation}
	\psi_t(\vx) = G \ast \rho_t = \frac{1}{2\pi}\int \ln\|\vx - \vy\| \rho_t(\vy) d \vy.
\end{equation}
Consequently, by taking the perpendicular gradient, we obtain
\begin{equation}
	\vu_t = \nabla^\perp \psi_t = \frac{1}{2\pi}\int \frac{(\vx - \vy)^\perp}{\|\vx - \vy\|^2} \rho_t(\vy) d \vy = K \ast \rho_t.
\end{equation}
Finally, by plugging the expressions of $\rho_t$ and $\vu_t = K \ast \rho_t$ in the MVE (\ref{eqn_MVE}), we verified the Lamb-Oseen vortex.
%\paragraph{Taylor-Green Vortex on the box $[0, 2\pi]^2$ with a periodic boundary condition.}
%Recall that we consider the 2D Navier-Stokes equation (the MVE with the scaled periodic Biot-Savart interaction kernel (\ref{eqn_K_periodic_biot_savart})) on the box $[0, 2\pi]^2$ with a periodic boundary condition.
%The Taylor-Green Vortex model stats that if $\rho_0(\vx) = \frac{1}{4\pi^2}\left(\cos \vx_1 \cos \vx_2 \exp(-2\nu t_0) + 1\right)$ for some $t \geq 0$, then we have 
%\begin{equation}
%	\rho_t(\vx) = \frac{1}{4\pi^2}\left(\cos \vx_1 \cos \vx_2 \exp(-2\nu (t+t_0)) + 1\right).
%\end{equation}
%
%First, one can easily check that $\rho_t(\vx)$ is a density function on $[0, 2\pi]^2$ for all $t\in[0, T]$, i.e. it is non-negative and integrates to 1. To verify this vortex model, define $\vu = (\vu_1, \vu_2)$ as follows
%\begin{equation}
%	\vu_1(\vx, t) = \cos \vx_1 \sin \vx_2 \exp(-2\nu t), \vu_2(\vx, t) = -\sin \vx_1 \cos \vx_2 \exp(-2\nu t).
%\end{equation}
%One can easily check that $\nabla\cdot\vu(\vx, t) \equiv 0$, which implies the exists of some stream function $\psi_t$ such that $\nabla^\perp \psi_t(\vx) = -\vu(\vx, t)$. Moreover, one can check that 

\paragraph{Barenblatt solutions for the MVE with Coulomb Interaction.} 
Recall that we consider the MVE with the Coulomb interaction kernel (\ref{eqn_coulomb_interaction}) for $d=3$ and set the diffusion coefficient $\nu = 0$, i.e.
\begin{equation}
	\frac{\partial \rho_t}{\partial t} + \nabla \cdot \left( \rho_t \cdot -\nabla \psi_t \right) = 0
\end{equation}
where $\psi_t$ is the solution to the Poisson equation $\Delta \psi_t = -\rho_t$. The Barenblatt solution of the above MVE is stated as follows: If $\rho_0 = \Uniform[\|\vx\|\leq (\frac{3}{4\pi}t_0)^{1/3}]$ for some $t_0 \geq 0$, then we have 
\begin{equation}
	\rho_t = \Uniform[\|\vx\|\leq (\frac{3}{4\pi}(t+t_0))^{1/3}]
\end{equation}
We now verify this solution.


Recall that the volume of a three dimensional Euclidean ball with radius $R$ is $\frac{4\pi}{3}R^3$. Hence we can write the density function as $\rho_t(x) = \frac{1}{t+t_0} \chi_{\|\vx\|\leq (\frac{3}{4\pi}(t+t_0))^{1/3}}$, where $\chi_\sX$ is a function that takes value $1$ for $\vx \in \sX$ and takes value $0$ for $\vx \notin \sX$.
Take 
\begin{equation}
	\psi_t(\vx) = \begin{cases}
		\frac{2(\frac{3}{4\pi}(t+t_0))^{2/3}-\|\vx\|^2}{6(t+t_0)}, & \|\vx\| \leq (\frac{3}{4\pi}(t+t_0))^{1/3},\\
		\frac{1}{8\pi\|\vx\|}, & \|\vx\| > (\frac{3}{4\pi}(t+t_0))^{1/3}.
	\end{cases}
\end{equation}
It can be verified that the Poisson equation $\Delta \psi_t = -\rho_t$ holds (note that $\Delta \|\vx\|^{-1} = 0$, i.e. $\|\vx\|^{-1}$ is a harmonic function for $d=3$).
Consequently, for a fixed time stamp $t$ and any $\|\vx\| \leq (\frac{3}{4\pi}(t+t_0))^{1/3}$ we have 
\begin{align}
	\frac{\partial \rho_t}{\partial t}(\vx) + \nabla \cdot \left( \rho_t(\vx) \cdot -\nabla \phi_t(\vx) \right) = - \frac{1}{(t+t_0)^2} + \frac{1}{(t+t_0)^2} = 0,
\end{align}
which verifies this solution.
