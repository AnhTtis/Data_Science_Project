\documentclass[runningheads]{llncs}\usepackage{graphicx}
\usepackage{algorithm}  
\usepackage{algorithmicx}  
\usepackage{algpseudocode}  
\usepackage{amsmath} 
\usepackage{amsfonts}
\usepackage{booktabs}
\usepackage{multirow}
\usepackage{regexpatch}
%\usepackage{hyperref}
%\usepackage{amsthm}
\usepackage{amssymb}
\usepackage{mathrsfs}
\usepackage{makecell}
\usepackage[misc]{ifsym} 

\usepackage{hyperref}

\usepackage[bottom]{footmisc}



\usepackage[most]{tcolorbox}
\tcbset{
  aibox/.style={
    width=370pt,
    top=10pt,
    colback=white,
    colframe=black,
    colbacktitle=black,
    enhanced,
    center,
    attach boxed title to top left={yshift=-0.1in,xshift=0.15in},
    boxed title style={boxrule=0pt,colframe=white,},
  }
}
\newtcolorbox{AIbox}[2][]{aibox,title=#2,#1}




\begin{document}

\title{ChatDoctor: A Medical Chat Model Fine-tuned on LLaMA Model using Medical Domain Knowledge}
\titlerunning{ChatDoctor}
\author{Anonymous}
\institute{Anonymous}
\author{Yunxiang Li\inst{1}, Zihan Li\inst{2}, Kai Zhang\inst{3}, Ruilong Dan\inst{4}, You Zhang\inst{1}$^{(\textrm{\Letter})}$}
\authorrunning{Y. Li et al.}
\institute{University of Texas Southwestern Medical Center, Dallas, USA \and University of Illinois at Urbana-Champaign, Urbana, USA \and Ohio State University, Columbus, USA \and Hangzhou Dianzi University, Hangzhou, China   \\  \email{you.zhang@utsouthwestern.edu}  }
\maketitle  


\begin{abstract}
Recent large language models (LLMs) in the general domain, such as ChatGPT, have shown remarkable success in following instructions and producing human-like responses. However, such language models have yet to be adapted for the medical domain, resulting in poor accuracy of responses and an inability to provide sound advice on medical diagnoses, medications, etc. To address this problem, we fine-tuned our ChatDoctor model based on 100k real-world patient-physician conversations from an online medical consultation site. Besides, we add autonomous knowledge retrieval capabilities to our ChatDoctor, for example, Wikipedia or a database as a knowledge brain. By fine-tuning the LLMs using these 100k patient-physician conversations, our model showed significant improvements in understanding patients' needs and providing informed advice. The autonomous ChatDoctor model based on Wikipedia and Database Brain can access real-time and authoritative information and answer patient questions based on this information, significantly improving the accuracy of the model's responses, which shows extraordinary potential for the medical field with a low tolerance for error. To facilitate the further development of dialogue models in the medical field, we make available all source code, datasets, and model weights available at: \href{https://github.com/Kent0n-Li/ChatDoctor}{https://github.com/Kent0n-Li/ChatDoctor}.


\end{abstract}



\section{Introduction}
The development of instruction-following large-scale language models (LLMs) such as ChatGPT\cite{ouyang2022training} has gained significant attention due to their remarkable success in instruction understanding and human-like response generation.
These auto-regressive LLMs \cite{wang2022selfinstruct} are pre-trained on web-scale natural language by predicting the next token and then fine-tuned to follow large-scale human instructions.
At the same time, they show robust performance on a wide range of natural language processing (NLP) tasks and generalize to unseen tasks, demonstrating their potential as unified solutions to various problems in natural language understanding, text generation, and conversational artificial intelligence.
However, exploring such generalized domain LLMs in the medical domain remains relatively unexplored \cite{gilson2023does}, despite their great potential to transform medical communication and decision-making \cite{abacha2015means}.
The reason is that existing models need to learn the medical domain specifically or in detail, resulting in models that often give incorrect medical responses. 

By fine-tuning large linguistic dialogue models on data from doctor-patient conversations, the models' ability to understand patients' needs can be significantly improved. Furthermore, to improve the model's credibility, we also designed a knowledge brain based on Wikipedia and medical-domain databases, which can access real-time and authoritative information and answer patients' questions based on this reliable information, which is vital for the medical field with low error tolerance. Through extensive experiments, we found that the fine-tuned model of doctor-patient dialogue outperforms ChatGPT in terms of precision, recall, and F1. In addition, the autonomous ChatDoctor model can answer the latest medical questions like Mpox. 
Since large language models such as ChatGPT are in a non-open source state, we used Meta's open-source LLaMA. We first trained a generic conversation model using 52K instruction-following data from Stanford University's Alpaca~\cite{alpaca}. Then we fine-tuned the model on our collected dataset of doctor-patient conversations.
Our approach has three main contributions: 
\begin{enumerate}
    \item We designed a framework for fine-tuning large language models in the medical domain.
    \item We collected and open-sourced a dataset with 100k patient-physician conversations for fine-tuning the large language model. The dataset contains extensive medical expertise for the medical application of LLMs. 
    \item Based on the external knowledge brain, we proposed an autonomous ChatDoctor model with online analysis ability of novel expertise.
\end{enumerate}



% may be with the online learning ability




%Thank you for your query, if pain is in the upper part of abdomen, then it could be peptic ulcer disease (PUD), or if it is on the right side of abdomen, gall bladder stone/problem, if it is on the left side, urinary tract problem. But all these are just assumptions, as I haven't seen your reports, and kindly provide me with some more details.


\begin{figure}[]
\centering
\includegraphics[width=0.9\textwidth]{imgs/overview.PNG}
\caption{Overview of the physician and patient conversation dataset collection pipeline and the training procedure of ChatDoctor.}
\label{fig:overview1}
\end{figure}




\begin{figure}[]
\centering
\includegraphics[width=0.98\textwidth]{imgs/wiki.PNG}
\caption{Overview of the Autonomous ChatDoctor based on Knowledge Brain.}
\label{fig:wiki}
\end{figure}


\section{Method}

\subsection{Patient-physician Conversation Dataset}
The first step in fine-tuning is to collect a dataset of patient-physician conversations. In patient-physician conversations, the patient's descriptions of disease symptoms are often colloquial and cursory. If we manually construct the synthesized patient-physician conversation dataset, it often leads to the problem of insufficient diversity and over-specialized descriptions, which are often spaced out from real scenarios. Collecting real patient-physician conversations is a better solution. Therefore, we collected about 100k real doctor-patient conversations from an online medical consultation website HealthCareMagic\footnote[1]{www.healthcaremagic.com}. We filtered these data both manually and automatically, removed the identity information of the doctor and patient, and used language tools to correct grammatical errors, and we named this dataset HealthCareMagic-100k, shown as Fig. \ref{fig:overview1}. In addition, we collected approximately 10k patient-physician conversations from the online medical consultation website iCliniq\footnote[2]{www.icliniq.com} to evaluate the performance of our model.



\begin{figure}[]
\caption{Some samples in our disease database consist of symptoms, clinical test approaches, and medication suggestions.}
\label{tab_disease_database}
\begin{AIbox}{Disease Database}
\textbf{Disease:} Dislocation of the patella\\
\textbf{Symptoms:} Knee pain, Knee swelling, Fluid retention, Problems during pregnancy, Knee stiffness or tightness    \\
\textbf{Further test or measures:} Radiographic imaging procedure, Plain x-ray (X-ray), Physical therapy exercises (Exercises), Application of splint (Splinting), Magnetic resonance imaging (Mri), Arthroscopy, Treatment; fracture or dislocation of lower extremity (other than hip or femur)\\
 \textbf{Medication:} Propofol, Cefoxitin, Ketamine, Tetracycline, Thyroid (Usp) (Armour Thyroid)   \\
     - - - - - - - - - - - - - - - - - - - - - - - - - - - - - - - - - - - - - - - - - - - - - - - - - - - - - - - \\
\textbf{Disease:} Varicose veins\\
\textbf{Symptoms:} Skin lesion, Leg pain, Peripheral edema, Ache all over, Lymphedema, Leg swelling, Leg cramps or spasms, Skin on leg or foot looks infected, Leg lump or mass     \\
\textbf{Further test or measures:} Wound care management, Complete physical skin exam performed (ML), Ultrasonography (Ultrasound), Traction; splints; and other wound care, Debridement of wound; infection or burn, Application of splint (Splinting), Excision (Removal) \\
 \textbf{Medication:} Collagenase Topical, Mupirocin Topical, Petrolatum Topical, Silver Nitrate Topical Each, Collagen, Cellulose, Becaplermin Topical, Sodium Tetradecyl Sulfate (Sotradecol), Cilostazol, Zinc Oxide Topical, Insulin Isophane (Humulin N)    \\
     - - - - - - - - - - - - - - - - - - - - - - - - - - - - - - - - - - - - - - - - - - - - - - - - - - - - - - - \\
\textbf{Disease:}Benign kidney cyst\\
\textbf{Symptoms:} Side pain, Sharp abdominal pain, Blood in urine, Vomiting, Back pain, Low back pain, Involuntary urination, Symptoms of the kidneys, Drainage in throat, Impotence, Symptoms of prostate, Decreased appetite    \\
\textbf{Further test or measures:} Radiographic imaging procedure, Urinalysis, Hematologic tests (Blood test), X-ray computed tomography (Scan ct), Complete blood count (Cbc), Electrolytes panel, Intravenous fluid replacement  \\
 \textbf{Medication:} Tamsulosin (Flomax), Pioglitazone (Actos), Glucosamine, Tolterodine (Detrol), Glimepiride, Tiotropium (Spiriva), Nystatin Topical Product, Lutropin Alfa (Luveris), Pyridostigmine (Mestinon), Pentosan Polysulphate Sodium (Elmiron), Ciclopirox Topical  \\

\end{AIbox}
\end{figure}


\subsection{External Knowledge Brain}
The auto-regressive prediction of the next word by the large language model leads the model to often give wrong answers to uncertain questions. Moreover, the output of the model is often uncontrollable and random, which is unacceptable in the medical field. The accuracy of the model would be greatly improved if the model could then answer based on the given authoritative and reliable knowledge, as illustrated in Fig. \ref{fig:wiki}. For Q\&A in medical scenarios, we collected and compiled a database, partly sampled in Fig. \ref{tab_disease_database}, which includes about 700 diseases and their associated symptoms, further medical tests or measures, and recommended medications, as a gold standard for the medical profession. The database can be updated at any time without retraining the model and can theoretically be set up for a specific disease database depending on the department or specific target. In addition to disease databases, some authoritative sources of information can also serve as external knowledge brains for the autonomous model, such as Wikipedia, a free multilingual online encyclopedia that is the largest and most widely read reference book in history. In summary, we can refer to the disease databases and Wikipedia (or any other reliable information source) as external knowledge brains of our ChatDoctor.



                       
\begin{figure}[h]
\caption{Autonomous Wikipedia retrieval through the prompt to ChatDoctor.}
\label{getkeyword}
\begin{AIbox}{Prompt for autonomous Wikipedia retrieval}
A question is provided below. Given the question, extract keywords from the text. Focus on extracting the keywords that we can use to best lookup answers to the question. \\
    - - - - - - - - - - - - - - - - - - - - - - - - - - - - - - - - - - - - - - - - \\
    \{Question of patient\}\\
    - - - - - - - - - - - - - - - - - - - - - - - - - - - - - - - - - - - - - - - - \\
    Provide keywords in the following comma-separated format. \\
    Keywords:                      
\end{AIbox}
\end{figure}



\begin{figure}[h]
\caption{Autonomous disease database retrieval through the prompt.}
\label{tab:database}
\begin{AIbox}{Prompt for autonomous disease database retrieval}
Database information is below.\\
    - - - - - - - - - - - - - - - - - - - - - - - - - - - - - - - - - - - - - - - - \\
    \{Relevant content from disease databases \}\\
    - - - - - - - - - - - - - - - - - - - - - - - - - - - - - - - - - - - - - - - - \\
    Select the data that will help to answer the question: \{Question of patient\}  \\
    Response:                      
\end{AIbox}
\end{figure}



\begin{figure}[h]
\caption{Answer based on retrieved knowledge, and we can determine whether the ChatDoctor needs to incorporate its own prior knowledge or not.}
\label{tab:retrieved2}
\begin{AIbox}{Prompt to answer based on the knowledge retrieved}
Context information is below. \\
    - - - - - - - - - - - - - - - - - - - - - - - - - - - - - - - - - - - - - - - - \\
    \{Relevant content retrieved from disease databases or Wikipedia\}\\
    - - - - - - - - - - - - - - - - - - - - - - - - - - - - - - - - - - - - - - - - \\
    Given the context information and (not/with) prior knowledge, answer the question:
    \\\{Question of patient\}\\
    Response:                    

\end{AIbox}
\end{figure}



\begin{figure}[h]
\caption{Let the ChatDoctor read the retrieved domain knowledge and provide a reliable answer.}
\label{fig:final_answer}
\begin{AIbox}{Prompt to make a final answer}
The original question is as follows: \{Question of patient\}\\
We have provided existing answers:\\
    - - - - - - - - - - - - - - - - - - - - - - - - - - - - - - - - - - - - - - - - \\
    \{Answer list based on retrieved knowledge\}\\
    - - - - - - - - - - - - - - - - - - - - - - - - - - - - - - - - - - - - - - - - \\
    The best answer is/Combine all answers: 
\end{AIbox}
\end{figure}



\subsection{Autonomous ChatDoctor based on Knowledge Brain}
Equipped with the external knowledge brain, i.e., Wikipedia or our constructed database encompassing over 700 diseases, ChatDoctor could retrieve the corresponding knowledge and reliable sources to answer patients' inquiries more accurately. After constructing the external knowledge brain, we need to let our ChatDoctor retrieve the knowledge he needs autonomously, which can generally be achieved in a large language model by constructing appropriate prompts. To automate this process, we design keyword mining prompts (Fig.~\ref{getkeyword}) for ChatDoctor to extract key terms for relevant knowledge seeking. Then, the top-ranked relevant passages were retrieved from Knowledge Brain with a term-matching retrieval system. As for the disease database, since the model cannot read all the data at once, we first let the model read the data in batches and select for itself the data entries that might help answer the patient's question (Fig. \ref{tab:database}). Finally, all the data entries selected by the model are given to the model for a final answer, shown as Fig. \ref{tab:retrieved2}. This approach better ensures that patients receive well-informed and precise responses backed by credible references.




\begin{figure}[h]
\caption{New knowledge test comparison between the ChatGPT and our ChatDoctor with knowledge brain. The ChatGPT cannot recognize the word Mpox (old name: Monkeypox), while our ChatDoctor can provide the precise answer for the medical test of Mpox.}
\label{fig:mpox}
\begin{AIbox}{Example of Mpox}
\textbf{Q:} How to test for Mpox?\\
\textbf{ChatGPT:} I'm sorry, but I'm not familiar with the term "Mpox." Could you please provide more information or context about what you're referring to? This will help me provide you with a more accurate answer.\\
\textbf{ChatDoctor:} Polymerase chain reaction (PCR) testing of samples from skin lesions is the preferred laboratory test. PCR blood tests are usually inconclusive because the virus remains in the blood for only a short time. To interpret test results, information is required on the date of onset of fever, date of onset of rash, date of specimen collection, current stage of rash, and patient age.
\end{AIbox}
\end{figure}


\begin{figure}[h]
\caption{Comparison between ChatGPT and autonomous ChatDoctor with knowledge brain. Due to the lack of domain knowledge, the ChatGPT provided a common answer about otitis, while our ChatDoctor provided a professional response to the otitis treatment.}
\label{fig:otitis}

\begin{AIbox}{Example of Otitis}
\textbf{Q:} How to treat Otitis?\\
\textbf{ChatGPT:} The treatment of otitis depends on the underlying cause and severity of the condition. Treatment may involve antibiotics, pain relief, ear drops, warm compresses, rest, and prevention measures such as avoiding foreign objects in the ear, keeping ears clean and dry, and protecting ears from loud noises and sudden changes in air pressure. It is important to see a doctor for proper diagnosis and treatment. \\
\textbf{ChatDoctor:} Treatment depends on the severity of the symptoms. If there is only mild discomfort, then pain relievers such as ibuprofen or acetaminophen can help. Antibiotics may be prescribed if the infection has spread beyond the outer ear. For more serious forms of Otitis, surgery may be needed to remove infected areas of the ear.

\end{AIbox}

\end{figure}



\begin{figure}[h]
\caption{Comparison between ChatGPT and autonomous ChatDoctor with knowledge brain. The ChatGPT is unfamiliar with the "Daybue" medication, while our ChatDoctor accurately pointed out the purpose of Daybue (trofinetide).}
\label{fig:daybue}
\begin{AIbox}{Example of Daybue}
\textbf{Q:} What is Daybue used to treat?\\
\textbf{ChatGPT:} I'm sorry, but I am not familiar with any medication called "Daybue." It is possible that you have misspelled the name or that the medication is not widely known. Could you please provide me with more information, such as the correct spelling or the condition it is used to treat? \\
\textbf{ChatDoctor:} Daybue (trofinetide) is used to treat Rett Syndrome.
\end{AIbox}
\end{figure}




\subsection{Training of the model}
We build our ChatDoctor utilizing Meta's LLaMA model~\cite{touvron2023llama}, a publicly accessible LLM.
Notably, in spite of its mere 7 billion parameters, the LLaMA has reported superior efficacy and competitive performance in comparison to the considerably larger GPT-3 (with 175 billion parameters) on several NLP benchmarks.
LLaMA's performance improvement was achieved by amplifying the diversity of training data, as opposed to parameter quantity.
Specifically, LLaMA was trained on 1.0 trillion tokens, procured from publicly accessible data repositories such as CommonCrawl and arXiv documents.

We utilize conversation from HealthCareMagic-100k to fine-tune the LLaMA model \cite{xu2021raise}, in accordance with the Stanford Alpaca~\cite{alpaca} training methodology, and our model was firstly be fine-tuned by Stanford Alpaca's data to have some basic conversational capabilities.
The fine-tuning process on HealthCareMagic-100k was conducted using 6 * A100 GPUs for a duration of 3 hours. 
The hyperparameters employed in the training process were as follows: the total batch size of 192, a learning rate of $2 \times 10^{-5}$, a total of 3 epochs, a maximum sequence length of 512 tokens, and a warmup ratio of 0.03, with no weight decay.


\begin{figure}[htbp]
\centering
\includegraphics[width=1\textwidth]{imgs/F1.PNG}
\caption{Quantitative Comparison between ChatDoctor and ChatGPT.}
\label{fig:f1}
\end{figure}







\section{Results}
To test the capability of the autonomous ChatDoctor model based on knowledge brains, we asked the model some recent medical questions, such as Mpox (monkeypox) in Fig. \ref{fig:mpox}, which is newly named by World Health Organization (WHO) on 28 November 2022. Since it is a new term, ChatGPT is completely unable to answer it, while our autonomous ChatDoctor can autonomously retrieve the Wikipedia content of Mpox and give an accurate answer. For some general medical questions, such as Otitis, ChatDoctor provides a very reliable answer after knowledge retrieval. As for Daybue in Fig. \ref{fig:daybue}, which was approved as a drug by the Food and Drug Administration (FDA) in March 2023, our model also provided accurate answers after autonomous information retrieval.

To quantitatively evaluate the performance of ChatDoctor, we use questions from iCliniq as input to ChatDoctor, and then we use the corresponding real doctors' answers from iCliniq as ground truth, and we also give the same input to ChatGPT and record its responses. We use the BERT Score \cite{zhangbertscore} to calculate Precision, Recall and F1 scores for ChatDoctor and ChatGPT, respectively. Comparing the results in Fig. \ref{fig:f1}, we find that the fine-tuned ChatDoctor model outperforms ChatGPT in Precision, Recall and F1, and some dialogue examples are shown in the Fig. (\ref{example_gpt_1}--\ref{example_gpt_5}).




\section{Limitations}
We would like to emphasize that ChatDoctor is for academic research only and any commercial use and clinical use is strictly prohibited. First, we have not designed sufficient security measures, and the current model can not guarantee the full correctness of medical diagnoses and recommendations. Second, our model is not licensed for healthcare-related purposes \cite{hatherley2020limits}. Third, ChatDoctor is based on LLaMA and has a non-commercial license, so we necessarily inherited these rules.


\section{Discussion and conclusion}
ChatDoctor, the chatbot obtained by fine-tuning large language models on medical domain knowledge, has a wide range of potential applications. However, due to the unique characteristics of the medical domain, latent language errors in diagnosis and medical advice can have serious consequences. And large language models often generate many incorrect and harmful statements (hallucinations) on the knowledge they do not know, which may result in malpractice. Our ChatDoctor is first fine-tuned with data from real patient-physician conversations, allowing the model to better understand the patient's questions to make more informed responses, and ChatDoctor model also has the ability to autonomously retrieve the knowledge brain and then provide answers, further enhancing the credibility of the model's responses. In practical applications, the potential benefits of ChatDoctor are enormous, including improved accuracy and efficiency in medical diagnosis and reduced workload for medical professionals, while increasing access to medical consultations, especially for patients in most underserved hospitals and third world countries. We believe that our ChatDoctor can be an invaluable aid in improving patient outcomes and advancing medical research.



\bibliographystyle{unsrt}
\begingroup
  \small % ensure correct font size for the bibliography
  \bibliography{ref}
\endgroup


\begin{figure}[h]
\caption{Example 1, the patient with a one-sided headache worried that the symptom may be linked to a brain tumor. Our ChatDoctor accurately pointed out that the cause of the patient's headaches could be sinusitis, which is consistent with the response from the doctor in iCliniq. Instead, the ChatGPT failed to diagnose the etiology of the one-sided headache.}
\label{example_gpt_1}
\begin{AIbox}{Example 1: ChatGPT vs ChatDoctor.}
\textbf{Q:} Hi doctor,I am getting one-sided headache. I do not get them often. They are behind my right eye and temple area. Now this time I am also fighting with heavy cold and cough. It came on late morning today and it hurts more when I cough or shake my head or bend over. Prior to this, I had the same headache about four months ago. At that time it came on midafternoon and lasted till next morning. Prior to that, I had the same headache exactly on the right side and lasted for a day. I am on CCB and Protonix. I am scared of brain tumor. Please help. \\
\textbf{iCliniq:} Hi. Based on what you have told, this is more likely to be due to sinusitis. Sinuses are small air filled cavities in your skull bone. They can get infected and when they are infected, you will get headaches behind eyes. Such headaches will be more common during cold and cough. Brain tumor headache will be on both sides, more in forehead and you will also have vomiting in that case. Based on what you have told, this is more likely due to simple sinus infection instead of brain tumor. CT brain and CT paranasal sinuses.     \\
\textbf{ChatGPT:} Hello! I'm sorry to hear you're experiencing these headaches. Based on the location of the pain, it sounds like it may be a tension headache or a migraine. These types of headaches can sometimes be triggered by colds or other illnesses. However, it's always best to get checked out by your doctor to rule out any other underlying conditions. In terms of a brain tumor, that is a very rare cause of headaches. Most people who experience headaches do not have a brain tumor. That being said, if you have any concerning symptoms such as vision changes, weakness, or numbness, it is important to seek medical attention right away. In the meantime, you can try taking over-the-counter pain relievers such as ibuprofen or acetaminophen to help with the headache. It's also important to get plenty of rest and stay hydrated. If the headaches persist or become more frequent, please make an appointment with your doctor. Take care!    \\ 
\textbf{ChatDoctor:} Hi, Thank you for posting your query. First, I would like to reassure you that there is no need to worry about brain tumor. Also, we should keep migraine apart as well. The most likely cause of your headaches could be sinusitis. This can be confirmed by doing an X-ray of face. Treatment would consist of using decongestant tablets/ nasal sprays. Best wishes, Chat Doctor.                        

\end{AIbox}
\end{figure}




\begin{figure}[h]
\caption{Example 2, a patient started getting horrible pain in the feet, burning, tickling, and pins and needles a few days ago, which is very challenging for the patient working and standing for about four and a half hours every morning. Although the ChatGPT has diagnosed that the cause of foot pain may result from a symptom of neuropathy (i.e., nerve damage), its suggestion of exercise is inappropriate given that the patient is with foot pain. In contrast, both our ChatDoctor and the iCliniq thought that the patient might have nerve damage and should consult the physician for further management.}
\label{example_gpt_2}
\begin{AIbox}{Example 2: ChatGPT vs ChatDoctor.}
\textbf{Q:} Hello doctor,I am feeling really down. I have had a lot of   digestive problems for about two years. I have been diagnosed with candida and dysbiosis but my regular doctor does not seem to know what is wrong with   me. I have been very stressed and have suffered anxiety since I was about 16.   I was not on any medications and started taking Cymbalta a few months ago. After about three to four days, I started to notice a bit of burning in   my feet. I have since stopped taking the Cymbalta about a month ago. My feet   seemed fine. Then seven days ago, I have started getting really bad pain in   my feet and burning and tickling and pins and needle feelings. I am not   diabetic or alcoholic. I have no idea what has brought this on or whether it   can be reversed. I am so stressed out. I have to be on my feet for my job for   about four and a half hours every morning. Can this be tested to see if the   nerve cells are either dead or just damaged and can it be reversed? This   feels like it is going to ruin my life. Please help.  \\
\textbf{iCliniq:}  Hello, Welcome to Chat Doctor forum. Regarding your concern, I   would agree with you on the fact that your symptoms could be related to a peripheral nerve damage.  Coming to   this point, I would recommend consulting a neurologist and performing an NCS   (nerve conduction study) to examine your nerves. Other tests to investigate   the possible causes of your symptoms would be: You should discuss with your   doctor on the above issues. For more information consult a neurologist online      \\
\textbf{ChatGPT:} Hello, I'm sorry to hear that you're feeling down and   experiencing these symptoms. The burning and tingling sensations you're   experiencing in your feet could be a side effect of the Cymbalta or it could   be a symptom of a neuropathy, which is nerve damage. It is important to speak   to your doctor about these symptoms and to possibly try a different   medication for treatment of your anxiety and depressive symptoms. In terms of   the digestive problems, candida and dysbiosis are conditions that can be   treated with probiotics and other dietary changes. It may also be helpful for   you to work with a registered dietitian to ensure that you are getting   adequate nutrition. If the burning and tingling sensations persist, your   doctor may recommend a nerve conduction study or electromyography to evaluate   the function of your nerves. Overall, it's important for you to seek medical   attention and manage your stress levels through techniques such as exercise,   meditation, or counseling. I hope you feel better soon.        \\ 
\textbf{ChatDoctor:} Hi, dear thanks for the query. Your symptoms are suggestive of   neuropathy. You should consult physician to rule out the cause for such   symptoms. Neurologist will do physical examination and investigations to rule   out the underlying causes. Treatment depends upon the cause. You may take   analgesics for relief from pain. Consult your physician for further   management. Hope this helps. Take care.                               

\end{AIbox}
\end{figure}


\begin{figure}[]
\caption{Example 3, a patient has one white lump in the throat for a good few months and worries about the probability of cancer. All three above diagnosed that the patient might have abnormally enlarged lymph nodes. The iCliniq and our ChatDoctor further pointed out that a biopsy and a radiology diagnosis are necessary when a base treatment fails. Instead, the ChatGPT only told the patient to consult with an ENT specialist.}
\label{example_gpt_3}
\begin{AIbox}{Example 3: ChatGPT vs ChatDoctor.}
\textbf{Q:} Hi doctor, I have one white lump around the tonsil area in my   throat. It has been there a good few months now and the area is swollen. Is this cancer?  \\
\textbf{iCliniq:} Hi. Cancer cannot be diagnosed by inspection only. To exclude   tonsillar stone you have to perform lateral neck X-ray and show me it later   on. I advise you to use a mouthwash and gargle your pharynx then expectorate   it many times during the day for a week. Then if the white spot still exists   you may have to undergo tonsillectomy. Then examine the tonsillar tissue   histologically to detect its nature. For more information consult an   ENT-otolaryngologist online  Take care.          \\
\textbf{ChatGPT:} As an AI language model, I am not qualified to provide medical   diagnosis or advice. However, a persistent white lump in the tonsil area that   has been present for several months with associated swelling should be   evaluated by a medical professional. It could be a benign condition such as a   tonsillar cyst or an inflamed lymph node, or something more serious like oral   cancer. Therefore, it is recommended that you seek medical attention from a   doctor, otolaryngologist (ENT), or head and neck surgeon for further   evaluation, diagnosis, and appropriate management.                                                                                         \\ 
\textbf{ChatDoctor:} Hi, dear thanks for your contact to Chat Doctor. I read your   query and understand your concern. As per your complaint it seems that you   may be having some kind of enlarged lymph nodes due to some infective cause   like reactive hyperplasia or tuberculosis as a possibility. In case if its   not responding to treatment then biopsy should be taken to rule out   malignancy. So consult an ENT specialist and get examined. Doctor may order   blood test, CT scan or ultrasound of neck to confirm the diagnosis. Doctor may   prescribe antibiotics, anti-inflammatory medications depending upon the   cause. Hope this will help you for sure. Contact further if follow up needed. Best regards.                            

\end{AIbox}
\end{figure}




\begin{figure}[]
\caption{Example 4, a patient worked too fast and got a sharp pain in the back and felt pain when breathing and rotating the torso or neck. The patient wonders whether it is necessary for urgent care. theBoth the iCliniq and our ChatDoctor first calmed down the patient and suggested taking Ibuprofen to relieve the symptoms. In contrast, the ChatGPT recommended immediate urgent medical care, which is the most secure way but may burden the patient.}
\label{example_gpt_4}
\begin{AIbox}{Example 4: ChatGPT vs ChatDoctor.}
\textbf{Q:} Hi doctor,While working out today, I was doing a shoulder   press. I was doing it too fast, and at one point, I got a sharp pain in my   back, near where my ribs meet my spine. Since then, I am also having very   sharp pain when inhaling and exhaling, as well as when rotating my torso or   neck. Should I need to go for urgent care? Currently, I am taking Zyrtec,   Amino acid supplements for energy and muscle recovery, hormonal IUD.     \\
\textbf{iCliniq:} Hi. It looks like you have muscle spasm or strain. You can use   over-the-counter Motrin 600 mg (Ibuprofen) twice or thrice a day with the   meal for two to three days and there are topical analgesics available for the   gentle massage. It will get better within two to three days. If it does not   get relieved, you can see your doctor. Take rest from the shoulder exercise   for a week. Get immediate help if you feel that this pain is significantly   disturbing breathing activity. For more information consult an internal   medicine physician online                      \\
\textbf{ChatGPT:} Yes, it is recommended that you seek urgent medical care for   your symptoms. The sharp pain you are experiencing when inhaling and   exhaling, as well as when rotating your torso or neck, could be a sign of a   rib or back injury. It is important to have an evaluation with a healthcare   provider to rule out any serious injuries or conditions that could require   further treatment or management. Make sure to inform your provider about your   current medications and supplements.                \\ 
\textbf{ChatDoctor:} 1. The symptoms seem to be due to musculoskeletal origin and   not related to any organic pathology. 2. Muscles of back, chest wall and   abdomen get strained during vigorous physical activity like weight training.   3. This can cause soreness, stiffness and spasms. 4. This is a self limiting   condition and will subside over few days time with adequate rest. 5. You may   take anti-inflammatory medications such as ibuprofen for relief from your   symptoms. 6. For local application, you can apply warm compresses over   affected areas. 7. I hope this information has been both informative and   helpful for you. Wish you good health.     
\end{AIbox}
\end{figure}



\begin{figure}[]
\caption{Example 5, a patient with vision blurry worried about the disease of the left eye. Given the patient's medical history of retinal detachment, the above three recommended the consult with ophthalmologists for a proper evaluation and prompt treatment. Due to the generation restriction of medical diagnosis (and advice), the ChatGPT failed to diagnose the pathogeny of vision reduction. Instead, the iCliniq and our ChatDoctor pointed out the probability of retinal detachment or bleeding.}
\label{example_gpt_5}
\begin{AIbox}{Example 5: ChatGPT vs ChatDoctor.}
\textbf{Q:} Hello doctor,My vision was crystal clear this morning. I just   woke up from a nap, and the vision in my left eye is blurry. It has not   cleared up in the past half an hour. Twice over the years, this has happened,   and it normalized over a few hours, but I am in a jungle now. So, I would   like to err on the side of caution. I have had previous retinal detachments.  \\
\textbf{iCliniq:} Hello. Did you get any surgery done for retinal detachment   earlier? It may be a recurrent episode of retinal detachment. If you have   diabetes or hypertension that may be a cause of this detachment. It can also   happen in vitreous hemorrhage in which you will see a slightly reddish vision   and will not clear up. If this is the case, then it will clear up in few   weeks. What is the exact cause can be told only on examination. You should   rush to the emergency asap. If detachment surgery is required then visual   prognosis depends on the time elapsed between episode and surgery. So, you   should not delay. There is not much you can do at the moment by yourself.   Just seek help from someone to reach a hospital. For more information consult   an eye care ophthalmologist online    Take care.         \\
\textbf{ChatGPT:} Hello, I am an AI language model and I cannot provide medical diagnosis or advice. However, based on your description, a sudden change in   vision with a history of retinal detachments raises concern for a possible new detachment or other serious eye condition. It is important that you seek   medical attention as soon as possible from an eye specialist. It is better to   have it evaluated and treated quickly rather than risking permanent vision loss.      \\ 
\textbf{ChatDoctor:} 1) You should see your ophthalmologist for a thorough   examination and evaluation. 2) There are many causes of sudden loss of vision   in one eye, including retina problems such as retinal detachment or bleeding. 3) Your history of prior retinal detachment makes me think that there is   something wrong with the retina again and it needs prompt treatment. 4) This   can be treated by laser surgery or if needed, a vitrectomy (surgical removal   of the vitreous gel). 5) I recommend seeing your ophthalmologist soon for a   proper evaluation.                             
\end{AIbox}
\end{figure}



\end{document}
