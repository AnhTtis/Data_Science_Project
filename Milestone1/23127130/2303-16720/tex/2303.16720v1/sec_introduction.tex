Gravitational wave (GW) detectors are sensitive to a variety of different sources.
Spinning neutron stars (NSs)
with non-axisymmetric deformations, in particular, are promising
GW sources \cite{Glampedakis:2017nqy},
e.g. for long-lasting quasi-monochromatic
signals (continuous waves, CWs) \cite{Haskell:2021ljd,Riles:2022wwz}.
The expected amplitudes of CWs are several orders of magnitude smaller
than the signals already detected from compact binary coalescences (CBCs)
\cite{LIGOScientific:2021djp}, requiring long integration times, and have not
been discovered yet by current LIGO--Virgo--KAGRA detectors
\cite{aLIGo,VIRGO_detect, KAGRA}.

Pulsars are rotating NSs that also emit electromagnetic (EM) beams.
Of these, some show glitches \cite{main_1},
i.e. rare anomalies in their otherwise very stable
frequency evolution.
Glitches typically feature a sudden spin-up, often followed by a relaxation phase,
and are one of the few circumstances in which
we may examine the inside of a NS and the characteristics of matter at
supernuclear density \cite{haskell_2017}. They could also excite
non-axisymmetric perturbations, hence triggering GWs \cite{Haskell:2015jra}. These
signals could be ``burst-like'', with dampening timescales of milliseconds
\cite{10.1046/j.1365-8711.1998.01840.x}, and also long-duration transient CW-like signals (tCWs)
with timescales from hours to months \cite{Prix:2011qv}. The detection of GWs
from a pulsar glitch could yield more understanding of the NS
equation of state \cite{vanEysden:2008pd}.

Detection prospects for tCWs were studied in Ref.~\cite{10.1093/mnras/stac3665},
finding that upcoming LVK observing runs will offer the first chances to detect them
or at least to obtain upper limits that can physically constrain glitch models.
Third generation GW detectors such as Einstein Telescope \cite{Punturo_2010} and Cosmic Explorer \cite{Reitze:2019iox} will be able to probe deeper into the population of known glitchers.
The Vela pulsar (J0835$-$4510) is a particularly strong candidate, because of its
large and frequent glitches, also characterized by long recovery times up to several hundred days.

Recent works have performed unmodelled searches for burst-like
transients \cite{KAGRA:2021tnv,PhysRevD.106.103037}
and modelled searches for long-duration tCWs
\cite{Keitel:2019zhb,LIGOScientific:2021quq,Modafferi:2021hlm}. In this work we focus on
tCW searches, which are so far based on matched filtering
of the data against a signal model.
Typically, a template bank is constructed to cover the possible range of signal parameters.
However these model-dependent searches are usually limited in the volume of
parameter space that can be covered, due to high computational cost, as we detail in
Sec.~\ref{sec:search}.

A possible way out of computational bottlenecks in GW data analysis is
deep learning (DL), a subfield of machine learning
which consists of processing data with deep neural networks. DL has been
gaining ground in different scientific fields and in particular in gravitational
astronomy, as it offers powerful novel methods to search for or classify signals
while decreasing computational cost. Ref.~\cite{Cuoco:2020ogp} offers a review of
GW applications, spanning from data quality
studies and detector characterization, to waveform modeling and
searches using different DL model architectures. In this work we provide a DL
algorithm based on a Convolutional Neural Network (CNN) as a complementary tool
to matched filtering in the search for tCWs from glitching pulsars.

The
paper is structured as follows. In Sec.~\ref{sec:search} we set up the detection
problem with a specific real-world test case
(searching for tCWs after the 2016 Vela pulsar glitch),
starting from matched filtering,
and we motivate how DL can help with its limitations.
In Sec.~\ref{sec:training} we introduce our CNN architecture and training
strategy and in Sec.~\ref{sec:evaluation_method} we explain our evaluation method and testing sets.
In Sec.~\ref{sec:results} we test the CNN on simulated and real data,
and in Sec.~\ref{sec:parameter_estimation} we consider extensions to parameter estimation.
We then present observational upper limits on the strain amplitude of tCWs
after the 2016 Vela glitch in Sec.~\ref{sec:upper_limits},
comparing with those previously obtained in Ref.~\cite{Keitel:2019zhb}.
Finally we state conclusions in Sec.~\ref{sec:conclusions}. 
