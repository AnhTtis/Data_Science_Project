\begin{figure}
    \centering
    \includegraphics[width=\columnwidth]{figures/segments.pdf}
    \caption{Time segments of the two LIGO detectors H1 and L1 corresponding to
    the data analyzed in this work. For reference the Vela 2016 glitch happened
    at $T_{\textrm{gl}}=1165577920\,$GPS, corresponding to 0.5\,days on this
    scale.}
    \label{fig:segments}
\end{figure}
\begin{figure}[t]
    \includegraphics[width=\columnwidth]{figures/appendix_gaps.pdf}    
    \caption{Duration parameters for exponentially decaying signals and SNRs
    estimated via the traditional $2\mathcal{F}_{\max}^{\textrm{r}}$ using
    rectangular windows. In the top panel, we synthesize the statistics assuming
    no gaps in the observation time, while in the bottom panel we use the
    timestamps of O2.
    \label{fig:gaps}
    }
\end{figure}
The data we analyze in this work starts on December 11, 2016 and the final SFT
timestamp is on April 11, 2017. This time span has a notable gap of 12 days as
shown in Fig.~\ref{fig:segments}, and other smaller gaps of about 1--2 days at
most.

We have found that the CNN trained on \Fstat atoms does not show any
pathological behavior due to these gaps, though generalizing its architecture
and our training strategy to data sets with different timestamps realizations is
left to future work. On the other hand, the SFT-based \Fstat algorithm is in
general terms also robust to the presence of gaps as well, but the transient
detection statistics are somewhat affected by large gaps as the one present in
our O2 data set, depending on window choice.

First we note that if a signal falls completely inside a gap, in our testing
sets, for convenience we discard that injection and draw another set of random
signal parameters. Because in our setup $t^0$ only varies within
$T_{\textrm{gl}} \pm 0.5$\,day, this only has an effect for a small fraction of
short duration signals and for the upper limits in Sec.~\ref{sec:upper_limits},
where the shortest $\tau$ is 0.5\,days, this has no influence. 

On the other hand, if the signal falls partially into one or more gaps, behavior
is different for the two types of injections sets. In the sets where the primary
parameter is $\rho$ as used for the CNN training and the test sets in
Sec.~\ref{sec:results}, the signal amplitude is adjusted upwards to achieve the
desired $\rho$. For the upper limits injections, the $h_0$ is fixed, hence
$\rho$ is lower due to the gaps. We now concentrate on the first case.

For rectangular signals the loss will be minimal especially for long-duration
signals. When analyzing exponential-window injections with a statistic assuming
a rectangular window, however, the loss from this mismatch is worsened when the
signal falls partially in a gap.

To study this effect we generated $10^4$ synthetic samples for exponential
signals analyzed assuming rectangular windows. The parameters of the signals are
drawn from the same distributions as defined in Tab.\ref{tab:ephem}. In
Fig.~\ref{fig:gaps}, we show injected signal durations and estimated SNRs from
$\Fstatmax$ via Eq.~\eqref{eq:rho_2Fmax}. In the first panel, data without gaps
is assumed, and in the second panel we used the actual O2 timestamps, with gaps.
In the latter case there is an evident loss in estimated SNR around $\tau\sim
40$ days. This is due to the long gap of 12 days starting about 10 days after
the beginning of the analyzed data. For shorter $\tau$, most of the power of the
injected signals is concentrated before the long gap and can be recovered by a
shorter rectangular window, while there is not too much a loss of SNR from the
weaker late-time portions of the signal (spread over $3\tau$, as defined in
Eq.~\eqref{eq:winexp}) falling into the gap. On the other hand for these
durations around 40\,days, most of the power falls into the gap, hence the
noticeable loss in SNR when recovered with a mismatched rectangular window.

This effect is even more accentuated when using not only realistic timestamps
but also real data, because of the non-stationary characteristics of real noise.
While for the standard detection statistics considered here this effect cannot
be easily remedied, a properly trained CNN can alleviate the loss, as shown in
Fig.~\ref{fig:SNR_panel} for the CNN trained on a mixture of both synthetic and
real data.
