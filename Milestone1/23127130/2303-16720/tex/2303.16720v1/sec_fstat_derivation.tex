Here we will explain in more detail the procedure to obtain the \Fstat atoms,
and resulting detection statistics,
as implemented in \texttt{LALSuite}~\cite{LALSuite} and \texttt{PyFstat}~\cite{Keitel:2021xeq}.
The general approach, as originally developed for persistent CWs, is documented in Ref.~\cite{Prix:2015cfs}
though we also include here the modifications for transients following Ref.~\cite{Prix:2011qv}
and adjust some of the notation for convenience.

Detection statistics for (t)CW signals can be derived from the likelihood ratio
between hypotheses about a data set $x(t)$. In particular, a Gaussian noise
hypothesis $\mathcal{H}_{\textrm{G}}$ and a signal hypothesis
$\mathcal{H}_{\textrm{tS}}$ can be written as
    \begin{align}
        \mathcal{H}_{\textrm{G}}: x(t) &= n(t),\\
        \mathcal{H}_{\textrm{tS}}: x(t) &= n(t) + h(t;\Dop, \Amp,\TP).
        \label{eq:hypotheses}
    \end{align}
We have written this for a single frequency-evolution template $\Dop$,
but it can be generalized by iterating over a template bank.

The likelihood ratio between the two hypotheses can then be analytically
maximized \cite{PhysRevD.58.063001, Prix:2009tq} over the amplitude parameters
$\Amp$, yielding the \Fstat.
As introduced in Eq.\,\eqref{eq:fstat}, it depends on different combinations of
the start times $t^0$ and duration parameters $\tau$.
The implementation consists of computing
per-SFT quantities, the ``\Fstat atoms''.
When weighted with an appropriate window and summed up,
these give the building blocks for $2\mathcal{F}(t^0,\tau)$.

In particular, following Ref.~\cite{Prix:2011qv},
$2\mathcal{F}(t^0,\tau)$ is computed from the antenna-pattern matrix
and projections of the data onto the basis of the signal model
from Eq.\,\eqref{eq:projections}, both depending on
a window function $\win(t^0,\tau)$.
The CW case corresponds to a rectangular window covering the full observation
time.

The antenna-pattern matrix can be
written in block-form:
\begin{equation}
    \mathcal{M}_{\mu\nu}(t^0, \tau) = \mathcal{S}^{-1}T_{\textrm{data}}
    \begin{pmatrix}
        \hat{A} & \hat{C} & 0 & 0\\
        \hat{C} & \hat{B} & 0 & 0 \\
        0 & 0 & \hat{A} & \hat{C}\\
        0 & 0 & \hat{C} & \hat{B}
    \end{pmatrix}
    \label{eq:matrix_abc}
\end{equation}
where \mbox{$T_{\textrm{data}} \equiv N_{\textrm{SFT}} T_{\textrm{SFT}}$},
\mbox{$\mathcal{S}^{-1}\equiv \frac{1}{N_{\textrm{SFT}}} \sum_{X\alpha}S_{X\alpha}^{-1}$}
and $\hat{A},\hat{B},\hat{C}$ are the
independent components of the matrix given by
\begin{equation}
\begin{aligned}
    \hat{A}(t^0, \tau) &\equiv \sum_{X\alpha} \win^2(t_\alpha;t^0,\tau)\langle (\hat{a}^X_{\alpha})^2 \rangle_t, \\
    \hat{B}(t^0, \tau) &\equiv \sum_{X\alpha} \win^2(t_\alpha;t^0,\tau) \langle (\hat{b}^X_{\alpha})^2 \rangle_t, \\
    \hat{C}(t^0, \tau) &\equiv  \sum_{X\alpha} \win^2(t_\alpha;t^0,\tau) \langle \hat{a}^X_{\alpha} \hat{b}^X_{\alpha} \rangle_t.
    \label{eq:atoms_antenna}
\end{aligned}
\end{equation}
We have used here the same notation from Sec.\,\ref{sec:search}, i.e. the $X,\alpha$
indices run over detectors and SFTs, respectively. For compactness, we have suppressed the transient
parameters on the right-hand side of Eq.\,\eqref{eq:matrix_abc}.

These components are noise-weighted (indicated by the hat) such that for a
function $z^X_{\alpha}$
\begin{equation}
    \hat{z}^X_{\alpha}(t) \equiv \sqrt{w^X_{\alpha}} z^X_{\alpha}(t), \quad
\end{equation}
with weights $w_{\alpha}^X \equiv S_{X\alpha}^{-1}/\mathcal{S}^{-1}$,
and time-averaged (indicated by the brackets) such that
\begin{equation}
    \langle z^X_{\alpha} \rangle_t \equiv \frac{1}{T_{\textrm{SFT}}} \int_{t^{\alpha}}^{t_{\alpha}+T_{\textrm{SFT}}} z^X_{\alpha}(t) dt.
\end{equation}
The definitions of the antenna-pattern functions $a(t),b(t)$ can be found in
Ref.~\cite{PhysRevD.58.063001}.
In practice, $a^X_\alpha$ and $b^X_\alpha$ are computed once per SFT at its representative timestamp
instead of computing the time averages explicitly,
and noise-weighted afterwards.

The square root of the determinant of the matrix is then (suppressing again the
$(t^0,\tau)$ dependency):
\begin{equation}
    \hat{D} \equiv \hat{A}\hat{B}-\hat{C}^2 \,.
    \label{eq:determinant}
\end{equation}
Note that we have assumed the long-wavelength limit approximation \cite{Rakhmanov_2008} which implies
that $a(t), b(t)$ are real-valued and there are no off-axis terms in Eq.~\eqref{eq:matrix_abc}.

The SFT data is normalized as
\begin{equation}
    y^X_{\alpha}(t) \equiv \frac{x^X_{\alpha}(f)}{\sqrt{\frac{1}{2} T_{\textrm{SFT}} S_{X\alpha}(f)}},
\end{equation}
and used to define the two complex quantities
\begin{equation}
\begin{aligned}
    F_{\textrm{a}, \alpha}^X \equiv \int_{t_{\alpha}}^{t_{\alpha}+T_{\textrm{SFT}}} y^X_{\alpha}(t)\hat{a}^X_{\alpha}(t)e^{-i\phi^X_{\alpha}(t)}dt,\\
    F_{\textrm{b},\alpha}^X \equiv \int_{t_{\alpha}}^{t_{\alpha}+T_{\textrm{SFT}}} y^X_{\alpha}(t)\hat{b}^X_{\alpha}(t)e^{-i\phi^X_{\alpha}(t)}dt,
\end{aligned}
\label{eq:atoms_Fab}
\end{equation} 
where $\phi_{\alpha}^X(t)$ is the phase obtained from integrating the frequency
evolution of a given signal template.
These quantities take the role of the data projections $x_\mu$ in the more abstract notation used earlier,
with the detailed translation documented in Ref.~\cite{Prix:2015cfs}.

The set of $\{ \langle (\hat{a}^X_{\alpha})^2 \rangle_t, \langle (\hat{b}^X_{\alpha})^2 \rangle_t, \langle \hat{a}^X_{\alpha}\hat{b}^X_{\alpha} \rangle_t, F_{\textrm{a},\alpha}^X, F_{\textrm{b},\alpha}^X \}$
from equations \eqref{eq:atoms_antenna} and \eqref{eq:atoms_Fab} are what we refer to as the \Fstat atoms.
More technical detail of how these are implemented in \texttt{LALSuite} can be
found in Ref.~\cite{Prix:2015cfs,demod}, based on the algorithm
by Ref.~\cite{Williams:1999nt}. 

For the CNN, we use these atoms as input. On the
other hand, to compute traditional detection statistics the codes compute
Eq.\,\eqref{eq:atoms_antenna} and Eq.\,\eqref{eq:determinant} and 
\begin{equation}
    F_{\text{\{a,b\}}}(t^0,\tau) \equiv \sum_{X\alpha} \win(t_\alpha;t^0,\tau) F^{X}_{\text{\{a,b\}},\alpha} ,
\end{equation}
i.e. the window-weighted partial sums of the atoms. Finally, from all of these
it then computes
\begin{equation}
    2\mathcal{F}(t^0,\tau) = \frac{2}{\hat{D}} [
     \hat{B} |F_{\textrm{a}}|^2 + \hat{A}|F_{\textrm{b}}|^2
     - 2 \hat{C} \mathfrak{R} (F_{\textrm{a}} F_{\textrm{b}}^{\ast})  ].
     \label{eq:2F_in_terms_of_atoms}
\end{equation}
This still depends on $(t^0,\tau)$,
through the implicit partial sums for all quantities on the right-hand side.

For a search over unknown transient parameters,
one can discretize the ranges $t^0 \in
[t^0_{\textrm{min}}, t^0_{\textrm{min}}  +\Delta t^0]$ and $\tau \in [
\tau_{\textrm{min}}, \tau_{\textrm{min}}+ \Delta \tau] $ in steps $dt^0$ and
$d\tau$. Then, \mbox{$\mathcal{F}_{mn}\equiv\mathcal{F}(t^0_m,\tau_n)$} is computed for a  $N_{t^0}\times N_{\tau}$
rectangular grid
\begin{equation}
    \begin{aligned}
        t^0_m & = t^0_{\textrm{min}} +mdt^0,\\
        \tau_n & = \tau_{\textrm{min}} +nd\tau.
    \end{aligned}
    \label{eq:transient_grid}
\end{equation}

Finally, one can e.g. maximize over $\{t^0,\tau\}$, obtaining the detection
statistic \Fstatmax, or alternatively marginalize over them, obtaining the
transient Bayes factor $\Btrans$. Complete derivations of both these detection
statistics for tCWs can be found in Ref.~\cite{Prix:2011qv}.
They still depend on the frequency evolution parameters $\Dop$, and a search
where the source parameters are not exactly known will typically consist of
setting up a grid in $\Dop$ covering the desired range.

As described in Ref.~\cite{Keitel:2018pxz},
the \texttt{pycuda} GPU implementation
in \texttt{PyFstat}
is largely equivalent to the one in \texttt{LALSuite}.
(For CPU calculations of the transient \Fstat
and other functionality,
\texttt{PyFstat} calls \texttt{LALSuite} functions
through SWIG wrappings~\cite{Wette:2020air}.)
However, when testing for this paper,
we noticed that a \texttt{LALSuite} fix that was introduced in 2018
had not been included in \texttt{PyFstat} yet:
when computing $\mathcal{F}_{mn}$ over very little data
(few SFTs from a single detector),
the antenna-pattern matrix can become ill-conditioned
for some combinations of $\lambda$ parameters,
making the determinant from Eq.~\eqref{eq:determinant} approach zero
and causing spuriously large detection statistics outliers.
\texttt{PyFstat} 2.0.0
(yet to be released as of this writing)
avoids this problem the same way as the \texttt{LALSuite} code,
by truncating to \mbox{$2\mathcal{F}_{mn}=4$}
(the Gaussian noise expectation value)
when the antenna-pattern matrix condition number exceeds
a threshold that has been fixed to $10^4$ .
Results included in Sec.~\ref{sec:results} use this version.
This issue of the \Fstat for short durations has also been described
in Ref.~\cite{PhysRevD.105.124007}, where an alternative and improved
statistic was derived for this case.
