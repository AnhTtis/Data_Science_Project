The standard method to search for tCWs is based on matched filtering and was
first introduced in Ref.~\cite{Prix:2011qv}.
So far it has been applied in Ref.~\cite{Keitel:2019zhb,LIGOScientific:2021quq}
to search for signals from various known glitching pulsars
in data from the second and third LIGO--Virgo observing runs (O2 and O3).
We briefly summarize the method here and provide more details in Appendix
\ref{sec:fstat_derivation}. Considerations for the practical setup of such searches
based on the ephemerides of known pulsars are
provided in Ref.~\cite{Modafferi:2021hlm}. The setup of this detection method is very
similar to that of narrowband CW searches \cite{LIGOScientific:2021quq,2017PhRvD..96l2006A,O2Narrowband},
but for tCWs one must also take into
account additional parameters describing the temporal evolution of the signal,
e.g. its start time and duration. 

The pulsar population presents a large variety of glitching
behaviors~\cite{main_1}, and there are several models explaining this
phenomenology~\cite{Haskell:2015jra,haskell_2017}. Similarly, GW signals could
derive from different physical mechanisms, e.g. Ekman flows
\cite{models_1,models_2,Singh:2016ilt} or Rossby \textit{r-}modes
\cite{Andersson,Santiago-Prieto:2012qwb}, varying from glitch to glitch.
Depending on the model, it may be reasonable to expect tCWs with durations
of the order of the timescales of the post-glitch recovery,
i.e. the time needed for the frequency to return to its secular value \cite{Yim:2020trr}. In general,
searches for tCWs must account for this model uncertainty by covering wider ranges of possible
parameters, including all possible combinations of the transient parameters and
sufficiently wide frequency evolution template banks, of sizes up to
tens of millions per target.

\subsection{Current method for detecting tCWs}

In matched filtering the first step is to define the signal model. 
We ignore the specific physical process behind the generation of the signal, and
define a generic tCW model following Ref.~\cite{Prix:2011qv}, by generalizing the standard
CW model.

First we introduce 
\begin{equation}
    h_{\textrm{CW}}(t;\Dop,\Amp) = \sum_{\mu=0}^{3} \Amp^{\mu}h_{\mu}(t;\Dop), 
    \label{eq:hCW}
\end{equation}
where $h_{\mu}$ are four basis functions derived in Ref.~\cite{PhysRevD.58.063001},
and $\Dop, \Amp$ are the frequency evolution and amplitude parameters of the
signal, respectively\footnote{We follow the notation of \cite{Prix_2007, Prix:2011qv, Prix:2015cfs} and specialize to a single signal harmonic.}. The four amplitude parameters
depend on the CW amplitude $h_0$, inclination $\cos\iota$, polarization angle
$\psi$, and initial phase $\phi_0$, while the frequency evolution parameters
include the source position (right ascension $\alpha$ and declination $\delta$)
and the GW frequency $\fgw$ and spindowns (frequency derivatives)
$\fdot, \fddot, \fdddot, \dots$
For GW emission driven by a mass quadrupole, the GW frequency and spindown
values are twice the rotational values.

From Eq.~\eqref{eq:hCW}, tCWs can be obtained by multiplying with a window function $\win(t;\TP)$
dependent on time and an additional set of transient parameters $\TP$:
\begin{equation}
    h(t; \Dop, \Amp, \TP) = \win(t;\TP) h_{\textrm{CW}}(t;\Dop, \Amp).
    \label{eq:tCW_model}
\end{equation}
For instance, $\TP$ can be the start time $t^0$ and the duration $\tau$ of the tCW signal.
Two simple choices of window functions are the rectangular window, for signals of constant amplitude truncated in time,
\begin{equation}
    \winrect(t;t^0,\tau) = 
        \begin{cases}
            1 & \text{if $t\in [t^0, t^0+\tau]$}\\
            0 & \text{otherwise}
        \end{cases},
    \label{eq:winrect}
\end{equation}
and the exponentially decaying window
\begin{equation}
    \winexp(t;t^0, \tau) = 
        \begin{cases}
            e^{-(t-t^0)/\tau} & \text{if $t \in [t^0, t^0 +3\tau]$}\\
            0 & \text{otherwise}
        \end{cases},
    \label{eq:winexp}
\end{equation}
where for $\winexp$ the truncation at $3\tau$ was chosen by Ref.~\cite{Prix:2011qv}
as the point where the amplitude has decreased by more than 95\%. 

While the rectangular window follows the exact evolution of a standard CW and
solely cuts it in time, the exponentially decaying window modifies the time
evolution of the amplitude $h_0$. In general, one can define an arbitrary window
function depending on different transient parameters $\TP$, but for simplicity
we only consider and compare the two choices above.

Once the signal model is
defined, one can then proceed as in Ref.~\cite{Prix:2011qv} with a noise-versus-signal
hypothesis test, assuming Gaussian noise for the noise hypothesis and adding
Eq.~\eqref{eq:tCW_model} for the signal hypothesis. The standard procedure
consists of maximizing the likelihood ratio between the hypotheses over the
amplitude parameters $\Amp$, obtaining what is known as the \Fstat \cite{PhysRevD.58.063001,
Prix:2009tq}.

In general, for both CWs and tCWs,
the \Fstat over a given stretch of data is calculated from the
antenna pattern matrix $\mathcal{M}_{\mu\nu}$ and the projections $x_{\mu}$ of
the data onto the basis functions $h_{\mu}$. The implementation in \texttt{LALSuite}
\cite{LALSuite,Prix:2015cfs} splits the data $x^X(t)$ of detector $X$ into short Fourier transforms (SFTs) \cite{sfts_2022},
i.e. the Fourier Transforms of time segments of
duration $T_{\textrm{SFT}}$. Then the \Fstat is computed from a set of per-SFT
quantities numerically of order $\mathcal{O}(1)$, called \Fstat \textit{atoms}.
Coherently combining $N_{\textrm{det}}$ detectors,
and suppressing the dependence on $\lambda$ for the rest of this section,
one can write:
\begin{equation}
    \begin{aligned}
        x_{\mu,\alpha} = 2\sum_{X=1}^{N_{\textrm{det}}}S^{-1}_{X\alpha} \int_{t_{\alpha}}^{t_{\alpha}+T_{\textrm{SFT}}} x^X(t)h_{\mu}^X(t)dt,\\
        \mathcal{M}_{\mu\nu,\alpha} = 2\sum_{X=1}^{N_{\textrm{det}}}S^{-1}_{X\alpha} \int_{t_{\alpha}}^{t_{\alpha}+T_{\textrm{SFT}}} h^X_{\mu}(t)h_{\nu}^X(t)dt,
    \end{aligned}  
    \label{eq:def_atoms}  
\end{equation}
where $S^{-1}_{X\alpha}$ is the single-sided noise power spectral density (PSD)
for a detector $X$ at the template frequency at the start time $t_{\alpha}$ of
the SFT $\alpha$. More specifically, in the following, with the word ``atoms''
we refer to a set of closely related per-SFT quantities as used in the
\texttt{LALSuite} code, which also include noise weighting.
We define these explicitly in Appendix \ref{sec:fstat_derivation},
following Ref.~\cite{Prix:2015cfs}.

To compute the (transient) \Fstat over a window $\win(t^0,\tau)$, one needs partially
summed quantities
\begin{align}
    x_{\mu}(t^0, \tau) &= \sum_{\alpha} \win(t_{\alpha};t^0,\tau)x_{\mu,\alpha},\\
    \mathcal{M}_{\mu\nu}(t^0, \tau) &= \sum_{\alpha} \win^2(t_{\alpha};t^0,\tau) \mathcal{M}_{\mu\nu,\alpha},
    \label{eq:projections}
\end{align}
where the sums go over the set of SFTs matching the window. 
From these,
\begin{equation}
    2 \mathcal{F}(t^0, \tau) = x_{\mu}(t^0, \tau)\mathcal{M}^{\mu\nu}(t^0, \tau)x_{\nu}(t^0, \tau),
    \label{eq:fstat}
\end{equation}
where $\mathcal{M}^{\mu\nu}$ is the inverse matrix of $\mathcal{M}_{\mu\nu}$.

Inserting the template into this equation defines the optimal signal-to-noise ratio (SNR)
\begin{equation}
 \label{eq:snr}
    \rho_{\textrm{opt}} = \sqrt{\mathcal{A}^{\mu} \mathcal{M}_{\mu \nu} \mathcal{A}^{\nu}},
\end{equation}
where we have suppressed the explicit dependence on the transient parameters. In
the absence of a signal, \mbox{$\rho_{\textrm{opt}}=0$}.
In later sections, we will regularly refer to simulated signals as ``injections''
(into real or simulated data)
and to the optimal SNR of the template used to generate an injection as $\rho_{\textrm{inj}}$.

To get a detection statistic from Eq.~\eqref{eq:fstat} that only depends on
$\Dop$, for the CW case, one just takes the total sums over the full
observation time. But for tCWs we need to handle the case of
unknown $t^0$ and $\tau$.
To obtain a detection statistic that depends only on $\Dop$, one can e.g. maximize over $\TP$
obtaining a statistic \Fstatmax, or marginalize over the same parameters obtaining the
transient Bayes factor $\Btrans$. It has been shown that $\Btrans$
is more sensitive than \Fstatmax in Gaussian noise \cite{Prix:2011qv} and also more robust on 
real data \cite{Tenorio:2021wad}.

\subsection{Computational cost limitations}

The method can be divided into three steps, at each $\Dop$: computing the \Fstat
atoms, computing Eq.~\eqref{eq:fstat} for all the different combinations of
transient parameters, and finally maximization/marginalization over the same
parameters to obtain an overall detection statistic.
Besides \texttt{LALSuite}, this has also been implemented in \texttt{PyFstat} \cite{Keitel:2021xeq},
which is a python package that wraps the corresponding \texttt{LALSuite} functions
and also adds a GPU implementation of the last two steps~\cite{Keitel:2018pxz}. 

The computing time varies for each step and can be estimated as discussed in
Ref.~\cite{Prix:2011qv,Keitel:2018pxz}. In particular, the most cost-intensive step
is typically the second one because of the large number of partial sums
that need to be taken corresponding to all different combinations of $\TP$.
Timing models and results for this step for both CPU and
GPU implementations can be found in Ref.~\cite{Keitel:2018pxz}.
As discussed there and in Ref.~\cite{Prix:2011qv},
this step also crucially depends on the window function:
calculations for exponential windows are much more expensive
not only because of the exponential functions themselves,
but also because partial sums cannot be efficiently reused
like in the rectangular case.
For realistic parameter ranges, searches with an exponential window model
can take orders of magnitude longer with respect to the
rectangular model in either implementation. Therefore, searches for
long-duration tCWs~\cite{Keitel:2019zhb,LIGOScientific:2021quq,Modafferi:2021hlm}
have only applied the simpler rectangular window, even though
models like that of Ref.~\cite{Yim:2020trr} assume post-glitch evolutions
following that of the EM signal, typically fitted by exponential functions.

DL models can avoid brute-force loops over many parameter combinations,
as once they are trained only a single forward-pass of the network is needed per input data instance.
Hence, they can be faster, and also have the potential to be more agnostic to signal models.
One could replace matched filtering completely by training a network on the full
detector data, thus allowing frequency evolutions different from the standard
spin-down model, but this would likely come with a loss in sensitivity, similarly to
excess power methods.

Instead, we apply CNNs with the \Fstat atoms as input data, 
which are an intermediate output of matched filtering, meaning we lock the
frequency evolution model but still allow flexibility in the amplitude evolution
of the GW and the potential for significant speed-up.
As we will demonstrate, with this approach one can reach
sensitivities similar to those of the standard detection statistics
at reduced computational cost. 

\subsection{Test case based on O2 Vela search}
\label{traditional_analysis}
We tune the setup of our training, search and comparisons to that from the first search for
long-duration tCWs \cite{Keitel:2019zhb}, using data
from the two LIGO detectors during the second
observing run (O2) of the LIGO--Virgo network \cite{LIGOScientific:2019lzm}.
Two priority
targets, Vela and the Crab, glitched during that observing period. No evidence
was found for tCW signals in the data, so 90\% upper limits on the GW
amplitude $h_0$ as a function of signal durations $\tau$ were reported. For a first
proof of concept we will only focus on the Vela (J0835$-$4510) analysis, since
its 2016 glitch is one of the largest that have been analyzed by GW searches,
with a relative jump in frequency of $\approx 1.43 \times 10^{-6}$. More glitches have
been analyzed during O3 \cite{LIGOScientific:2021quq}, of which two reached similar
strengths. But due to its combination of a large glitch and its proximity,
the 2016 Vela glitch was the most promising to date.

\begin{table}[]
    \begin{tabular}[t]{cc}
        \hline
        \hline
        \multicolumn{2}{c}{Source parameters} \\
        \hline
        \hline
        $d$ & $287$\,pc \\
        $\alpha$ & $2.2486$\,rad \\
        $\delta$ & $-0.7885$\,rad \\
        $f$ & $22.3722$\,Hz \\
        $\fdot$ & $-3.12\times 10^{-11}$\,Hz\,s$^{-1}$ \\
        $\fddot$ & $1.16\times 10^{-19}$\,Hz\,s$^{-2}$ \\
        $T_{\textrm{ref}}$ & $58000\,$MJD \\
        $T_{\textrm{gl}}$ & $57734.485$\,MJD \\
        \hline
        \hline
    \end{tabular}
    \quad
    \begin{tabular}[t]{cc}
        \hline
        \hline
        \multicolumn{2}{c}{Search parameters} \\
        \hline
        \hline
        $\Delta f$ & $0.1$\,Hz\\
        $\Delta \dot{f}$ & $1.01 \times 10^{-13}$\,Hz\,s$^{-1}$ \\
        $df$ &  $9.57 \times 10^{-8}$\,Hz \\
        $d\dot{f}$ & $9.15 \times 10^{-15}$\,Hz\,s$^{-1}$\\
        $t^0_{\textrm{min}}$ & $T_{\textrm{gl}} - 0.5$\,day\\
        $t^0_{\textrm{max}}$ & $T_{\textrm{gl}} + 0.5$\,day\\
        $\tau_{\textrm{min}}$ & $T_{\textrm{gl}}-\frac{1\,{\textrm{day}}}{2}+ 2T_{\textrm{SFT}}$\\
        $\tau_{\textrm{max}}$ & $120\,\textrm{days} - 2T_{\textrm{SFT}} $\\
        \hline
        \hline
    \end{tabular}
    \caption{Ephemerides and search parameters for Vela and its glitch during the O2 run. The distance
    from the source is indicated with $d$ and its sky coordinates are $\alpha$
    (right ascension) and $\delta$ (declination). The frequency and spindown
    values for the source parameters are the GW values and are all referenced to $T_{\textrm{ref}}$. 
    In the search parameters, we indicate with $\Delta$ the search band
    centered around the GW frequency (or spindown), and with $df, d\dot{f}$ the
    resolutions in frequency and spindown of the search. The search in second
    order spindown is fixed to the nominal GW value. For the transient
    parameters, we list the minimum and maximum values of start times $t^0$ and
    $\tau$ we search over, with resolutions matching $T_{\textrm{SFT}} = 1800$\,s.} 
    \label{tab:ephem}
    \end{table}

Following Ref.~\cite{Keitel:2019zhb}, we construct our test case as a narrow-band search
in frequency and a single spindown parameter,
centred on the values inferred from the timing observations of Ref.~\cite{2018Natur.556..219P},
and directed at Vela's sky position.
We search for tCWs with start times $t_0$ within half a day of the glitch time
(December 12th 2016 at 11:38:23.188 UTC), and with durations $\tau$ up
to 120\,days.
All relevant source and search parameters are listed in Table~\ref{tab:ephem}.
Even when using simulated data,
we match their timestamps to the real observational segments
as given by Ref.~\cite{O2SFTs}, 
which show a notable 12-day gap and various smaller ones.
We study the effect of these gaps on the various detection statistics in
Appendix~\ref{sec:gaps}.
The total number of analyzed SFTs, each of duration $T_{\textrm{SFT}}=1800$\,s,
are 3782 for H1 and 3234 for L1. 
