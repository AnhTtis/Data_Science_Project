\begin{figure*}[th]
    \includegraphics[width=1\textwidth]{figures/upper_limits_pipelines.pdf}
    \caption{Upper limits on $h_0$ as a function of duration $\tau$ for the different methods: $2\mathcal{F}_{\max}^{\,\textrm{r}}$ (dash-dotted black) and $\Btrans^{\textrm{r}}$ (dashed black), both recovering rectangular signals. The two different shapes of the injections are used in the different panels (rectangular on the left, exponential on the right). In each subplot we also show the two-stage filtering trained on only rectangular-shaped signals (left panel, solid blue) and two-stage filtering trained on only exponentially decaying signals (right panel, solid orange). }
    \label{fig:upper_limits}
\end{figure*}
No detection of a post-glitch tCW signal was reported in the O2 search
\cite{Keitel:2019zhb}, so upper limits on the GW strain amplitude were set.
Having found no interesting outliers in the CNN results on the same data set either,
we repeat the procedure here, but covering both rectangular and exponential window choices.

Upper limits are computed by injecting simulated signals in the same data used in the
search and then counting how many signals are recovered by the chosen method.
For a set of injections at fixed duration $\tau$,
one can then fit a sigmoid curve of the counts against
injected amplitude $h_0$ to find the upper limit amplitude $h_0^{90\%}$ at which
$90\%$ of the injected signals are recovered above the threshold of each
statistic. 
A more detailed explanation can be found in the appendix of Ref.~\cite{Keitel:2019zhb}.

As the thresholds to distinguish between noise and signal
candidates we use the highest output of the search
over the original data and the ranges given in Table~\ref{tab:ephem},
for each detection statistic. This
means that the thresholds of the different detection statistics
do not necessarily correspond to the same $p_{\textrm{FA}}$, but it is a consistent method that
can be applied to any statistic.\footnote{Alternatively, one could use the
method described in Ref.~\cite{Tenorio:2021wad} to estimate the distribution of the expected loudest outlier from the background and
derive a threshold. However, we find $\rhocnn$ to show more complicated
distributions on our test data than the typical ones for \Fstatmax and $\Btrans$. So,
further investigation would be required to apply the method.}
It is also a conservative choice in the sense that any lower threshold would
produce stricter upper limits, but would require outlier followup. In the
special case of the two-stage filtering, the first threshold on the CNN stage is
chosen to let a fraction $10^{-3}$ of candidates pass and the final threshold is
given by the highest $\Btrans$ in that remaining set.

We create an injection set with simulated signals of amplitude $h_0$ in the
ranges used in the O2 search \cite{Keitel:2019zhb}, chosen to
correspond to detection statistics around the threshold values. This translates to
mostly weaker signals compared to the test sets
described in Sec.~\ref{sec:evaluation_method}
and used for the ROC curves in Sec.~\ref{sec:results}.
The durations of the injections are not distributed uniformly as done
for the previous testing sets, but rather chosen at discrete steps from
0.5 to 120\,days.

In Fig.~\ref{fig:upper_limits}, we reproduce the plot from Ref.~\cite{Keitel:2019zhb}
for rectangular-window signals,
showing the upper limits on $h_0^{90\%}$ as a function of the duration parameter $\tau$,
and we extend it with an exponential-window injection set.
We show upper limits derived from several methods discussed before, namely
$2\mathcal{F}_{\max}^{\,\textrm{r}}$, $\Btrans^{\textrm{r}}$,
and the two-stage filtering method using CNNs trained on only
rectangular windows or only exponential windows, combined with $\Btrans^{\textrm{r/e}}$.
The statistics $2\mathcal{F}_{\max}^{\,\textrm{e}}$ and $\Btrans^{\textrm{e}}$
for the full injection set 
are again omitted due to computational cost.

The upper limits from
$2\mathcal{F}_{\max}^{\,\textrm{r}}$ using rectangular injections can be directly compared to those
obtained in Ref.~\cite{Keitel:2019zhb}. They are consistent within $1\sigma$ error bars, and the small differences are due to the
different individual injections and the different sigmoid fitting procedure and
uncertainty estimation
(matching the implementation as in Ref.~\cite{LIGOScientific:2021quq}).

The upper limits from the two-stage filtering reach values close to $2\mathcal{F}_{\max}^{\,\textrm{r}}$ and $\Btrans^{\textrm{r}}$,
higher only by 6--7\% for both signal types.

As seen before in the real-data ROC curves,
the two-stage filter cannot improve over $\Btrans^{\textrm{r/e}}$
on these specific data sets
because the templates corresponding to the loudest outliers from those statistics
also pass the first-stage $\rho_{\textrm{CNN}}^{\textrm{r/e}}$
thresholds.

Another factor that could potentially be relevant in this new, typically weaker
injection set, is that upper limits depend strongly on the performance for weak
signals near threshold, which are the most challenging for CNNs.
This is exacerbated by using steps in the amplitude $h_0$, instead of SNR, to
quantify the strength of the signals. While the SNR takes into account the other
parameters that affect detectability, especially the inclination $\cos\iota$,
the amplitude does not contain this information
and so cannot be used as a direct proxy for the SNR.
Therefore, at each $h_0$ step of the injection set,
there can be a tail of signals with low SNRs,
where the differences between the CNN and traditional detection statistics are more significant.
