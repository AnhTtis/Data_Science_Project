In this work we present a new method based on CNNs for detecting potential
tCWs -- long-duration quasi-monochromatic GW signals --
from glitching pulsars. CNNs have proven to be promising tools for detecting
various GW signals, but have not been tested before on tCWs from glitching pulsars.
Previous searches were entirely matched-filtering based and limited in
computational cost. In this work we have used intermediate matched-filter
outputs (the \Fstat atoms) as input to the CNNs, which allows to replace the most
computationally expensive part of the analysis. In particular, practical
searches with the \Fstatmax and $\Btrans$ statistics were limited to
assuming constant amplitude (rectangular window) tCWs due to the much higher
cost of other window functions, while with CNNs we can easily train on different
windows. 

The CNNs are constructed to output an estimator of SNR in the data. We have trained
and tested CNNs for either rectangular or exponentially decaying windows, first
on synthetic, Gaussian data, but with gaps corresponding to the timestamps of
the LIGO O2 data after the Vela glitch of 2016. We use curriculum learning, i.e.
first train on stronger and then also on weaker signals. Then we have tested
these CNNs on the real LIGO data from O2 as previously analyzed in Ref.~\cite{Keitel:2019zhb}, and also trained with real data using the same
architecture to improve the results. We find the best results when starting
from the model we had trained on synthetic data and continuing its training on
only real data. 

We find that a simple implementation of such atoms-based CNNs already approaches
within 10\% of the detection probability at fixed false-alarm probability
of the traditional detection statistics. As a CNN-based method that
mostly closes this remaining gap, we propose a two-stage filtering method
consisting of first applying the CNN to all signal templates and only passing
candidates above a certain threshold on $\rho_{\textrm{CNN}}$ to the $\Btrans$
statistic. For a real data test set containing injected signals of broad SNR
range (from 4 to 40), the probabilities of detection at false-alarm probabilities as low as
$10^{-7}$ are only $\lesssim 2\%$ lower than those of $\Btrans$
for rectangular windows and 4\% better for exponential windows.
Comparing the computing time of the two-stage filtering with that of $\Btrans$
for exponential signals using the GPU implementation of Ref.~\cite{Keitel:2018pxz},
we find that our method is $80$ times faster than evaluating the full data
without the CNN stage.

We then use this new method to 
set updated observational upper limits on the GW strain amplitude $h_0$ after the O2 Vela glitch,
with extended parameter space coverage
including exponentially decaying signals.
For this we make a separate injection set with signal parameters in the same ranges as in Ref.~\cite{Keitel:2019zhb}.
The upper limits from the
two-stage filtering almost match those reached by the standard statistics, 
higher only by $\lesssim7\%$.

Thanks to its computational efficiency, the CNN-based approach
can be a competitive method for the overall tCW search effort
that is also complementary to the reference method:
if extended to more generic searches,
it will help increase overall detection probability by extending discovery space
while one can afterwards still brute-force evaluate the traditional detection statistics
when interested in pushing for the deepest possible upper limits on individual targets.

Considering such further applications of CNNs to tCWs from glitching pulsars,
we have here focused on a single target pulsar and data set,
but the approach can be generalized to other targets or even unknown sky positions,
either by training from scratch in each case or through transfer learning \cite{Gupta_2022}.
Due to the computational efficiency of both the training and
evaluation phase, the two-stage filtering could thus be scaled up to broader searches
covering a larger variety of targets than currently feasible with the standard methods,
especially when wanting to include multiple amplitude evolution window options.

Also, since CWs can be obtained from the tCWs model by setting a rectangular
window function to cover the entirety of the data, this method can easily be
applied to persistent CW targeted searches. Our implementation of CNNs, however,
was designed specifically to avoid the computationally expensive computation of
partial sums of the \Fstat atoms over all the combinations of the transient parameters, which does
not concern CW searches.

Furthermore, one could go beyond the current configuration, in which we trained the CNNs on \Fstat atoms, i.e. quantities
computed during the matched filtering step. This still
constrains the frequency evolution of the signal to be CW-like, but already
allows for flexible amplitude evolution and significant speed-up compared to the
traditional method, effectively allowing to search a wider parameter space at
the same cost. A different approach would be to train a CNN
(or different type of network)
on spectrograms or the full timeseries strain data, which could allow searches for unmodeled tCWs
both in frequency and amplitude evolution.

The major drawback would be that the
amplitudes of tCW signals are expected to be very weak, with $h_0$ upper limits from
O3 reaching $10^{-25}$ \cite{LIGOScientific:2021quq}. Such signals are too weak to be directly discernible, e.g. in time-frequency maps of the data. A study of using Fourier
transforms of the data as input to a CNN to search for persistent CWs was
done in Ref.~\cite{2019PhRvD.100d4009D}, and a broader set of machine-learning strategies
on SFTs was evaluated in a public Kaggle challenge \cite{G2NetCWKaggleChallenge}.
Similar approaches could be applied to tCWs as well,
but it will be a challenging problem, which we leave for future work.
Despite not reaching the high
sensitivities of purely matched filtering methods, these faster, generic
searches could still extend the tCW science case
by enabling the search for post-glitch GWs corresponding to more general glitch
models and over larger parameter spaces, potentially leading to all-sky, all-time,
all-frequency blind searches of tCWs.
