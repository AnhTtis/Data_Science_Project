\documentclass[11pt, a4paper]{article}
\usepackage{tacl2021v1}

\usepackage{times,latexsym}
\usepackage{tipa}
\usepackage{hyperref}
\usepackage[T1]{fontenc}
\usepackage{amssymb}
\usepackage{CJKutf8}
\usepackage{microtype}
\setlength\titlebox{5cm}
\usepackage{booktabs}
\usepackage{multirow}
\usepackage{amsmath}
\usepackage{multirow, hhline, graphicx, helvet}
\mathchardef\mhyphen="2D
\newenvironment{itemizesquish}{\begin{list}{\labelitemi}{\setlength{\itemsep}{-0.2em}\setlength{\labelwidth}{0.5em}\setlength{\leftmargin}{\labelwidth}\addtolength{\leftmargin}{\labelsep}}}{\end{list}}
\newenvironment{enumeratesquish}{\begin{list}{\addtocounter{enumi}{1}\labelenumi}{\setlength{\itemsep}{0em}\setlength{\labelwidth}{0.5em}\setlength{\leftmargin}{\labelwidth}\addtolength{\leftmargin}{\labelsep}}}{\end{list}\setcounter{enumi}{0}}

\usepackage{makecell}
\usepackage{xspace, mfirstuc, tabulary}
\newcommand{\base}{\textsc{base}\xspace}
\newcommand{\labm}{\textsc{label}$_m$\xspace}
\newcommand{\labs}{\textsc{label}$_s$\xspace}
\newcommand{\labsr}{\textsc{label}$_{sr}$\xspace}
\newcommand{\labmr}{\textsc{label}$_{mr}$\xspace}

\newcommand{\dall}{\textsc{all}\xspace}
\newcommand{\dhigh}{\textsc{high}\xspace}
\newcommand{\dmid}{\textsc{mid}\xspace}
\newcommand{\dallf}{\textsc{all+freq}\xspace}

\newcommand{\MODA}{\textsc{model\scriptsize{[-pinyin]}}\xspace}
\newcommand{\MODB}{\textsc{model\scriptsize{[+pinyin]}}\xspace}

\newcommand{\regular}{\emph{regular}\xspace}
\newcommand{\alliterating}{\emph{alliterating}\xspace}
\newcommand{\rhyming}{\emph{rhyming}\xspace}
\newcommand{\rad}{\emph{irregular}\xspace}
\newcommand{\sem}{\emph{semantic}\xspace}

\newcommand{\std}[1]{{\scriptsize $\pm$#1}}

\usepackage{todonotes, color}
\newcommand{\Note}[2]{} 
\newcommand{\SideNote}[2]{}
\renewcommand{\Note}[2]{\todo[color=#1,size=\small, inline=true]{#2}} 
\renewcommand{\SideNote}[2]{\todo[color=#1,size=\small]{#2}} %
\newcommand{\Lingyuin}[1]{\textcolor{orange}{\bf \small [ #1 --Lingyu]}}
\newcommand{\Xiaomengin}[1]{\textcolor{blue}{\bf \small [#1 --Xiaomeng}}
\title{Evaluating Transformer Models and Human Behaviors on Chinese Character Naming}

\author{Xiaomeng Ma \\
  The Graduate Center, CUNY \\
  New York, USA\\
  \texttt{xma3@gradcenter.cuny.edu} \\\And
  Lingyu Gao \\
  Toyota Technological Institute at Chicago \\
Chicago, USA\\
  \texttt{lygao@ttic.edu} \\}
\setlength {\marginparwidth }{2cm} 
\begin{document}
\maketitle
\begin{abstract}
Neural network models have been proposed to explain the grapheme-phoneme mapping process in humans for many alphabet languages. These models not only successfully learned the correspondence of the letter strings and their pronunciation, but also captured human behavior in nonce word naming tasks. How would the neural models perform for a non-alphabet language (e.g., Chinese) unknown character task? How well would the model capture human behavior?  In this study, we first collect human speakers' answers on unknown character naming tasks and then evaluate a set of transformer models by comparing their performances with human behaviors on an unknown Chinese character naming task. We found that the models and humans behaved very similarly, that they had similar accuracy distribution for each character, and had a substantial overlap in answers. In addition, the models' answers are highly correlated with humans' answers. These results suggested that the transformer models can well capture human's character naming behavior. \footnote{The code and data for this paper can be found at: \href{https://github.com/xiaomeng-ma/Chinese-Character-Naming}{https://github.com/xiaomeng-ma/Chinese-Character-Naming}.}
\end{abstract}
\begin{CJK*}{UTF8}{bsmi}
\section{Introduction}

The increasing complexity of source code poses a key challenge to the reliability of large-scale software systems. Software bugs in these systems can lead to safety issues~\cite{bug_safety} for users around the world as well as cause non-negligible financial losses~\cite{bug_loss}. As such, developers have to spend a large amount of time and effort on bug fixing. Consequently, \aprfull (\apr), designed to automatically generate patches to fix software bugs, has attracted wide attention from both academia and industry~\cite{long2016prophet, legoues2012genprog, long2015spr, lou2020can, tufano2018empstudy}. 


To achieve \apr, one popular approach is known as Generate-and-Validate (G\&V)~\cite{qi2015gv, ghanbari2019prapr, lou2020can, le2016hdrepair, legoues2012genprog, wen2018capgen, hua2018sketchfix, martinez2016astor, koyuncu2020fixminder, liu2019tbar, liu2019avatar}, which is typically based on the following pipeline: First, fault localization techniques~\cite{wong2016fl, abreu2007ochiai, zhang2013injecting, papadakis2015metallaxis, li2019deepfl, li2017transforming} are applied to determine the suspicious locations in programs where bugs are likely to exist. Then, the buggy locations are used by the \apr tools to generate a list of patches that replace buggy lines with correct lines. Afterward, each patch is validated against the original test suite to identify any \emph{plausible patches} (i.e., passing all tests in the test suite). Finally, to determine the \emph{correct patches}, developers examine the list of plausible patches to see if any of them can correctly fix the bug. 

Traditional \apr tools can mainly be categorized into heuristic-based~\cite{legoues2012genprog, le2016hdrepair, wen2018capgen}, constraint-based~\cite{mechtaev2016angelix, le2017s3, demacro2014nopol, long2015spr} and \template~\cite{ghanbari2019prapr, hua2018sketchfix, martinez2016astor, liu2019tbar, liu2019avatar}. Among these traditional tools, \template \apr tools~\cite{ghanbari2019prapr, liu2019tbar, benton2020effectiveness} have been able to achieve state-of-the-art results. \Template \apr tools typically leverage pre-defined templates (e.g., adding a nullness check) for bug fixing. However, since these fix templates are typically handcrafted, the number and types of bugs they are able to fix can be limited. 



To address the limitations of traditional \apr, researchers have proposed various \learning \apr tools~\cite{li2020dlfix, chen2018sequencer, jiang2021cure, lutellier2020coconut, zhu2021recoder, ye2022rewardrepair} based on the \nmtfull (\nmt) architecture~\cite{sutskever2014mt} where the input is the buggy code snippets and the goal is to translate the buggy code snippets into a fixed version. To accomplish this, \learning \apr tools require supervised training datasets with pairs of both buggy and fixed code snippets in order to learn how to perform this translation step. These training data are usually obtained by mining historical bug fixes using heuristics/keywords~\cite{dallmeier2007benchmark}, which can be imprecise for identifying bug-fixing commits; even the actual bug-fixing commits can include irrelevant code changes, leading to further pollution in the dataset~\cite{xia2022alpharepair}.
% 
Moreover, it can be hard for such \apr tools to generalize and fix bug types unseen during training. 



To better leverage recent advances in \plmfull{s} (\plm{s}), researchers~\cite{xia2022alpharepair, xia2023repairstudy, kolak2022patch, prenner2021codexws} have directly applied \plm{s} to generate patches without bug-fixing datasets. These \llm-based \apr tools work by either directly generating a complete code function~\cite{prenner2021codexws, xia2023repairstudy} or predict/infill the correct code snippet given its surrounding context~\cite{xia2022alpharepair, xia2023repairstudy}. By directly using \llm{s} that are pre-trained on billions of open-source code snippets, \llm-based \apr tools can achieve state-of-the-art performance on many repair datasets~\cite{xia2022alpharepair}. 


% 
%
%

Traditional \apr tools have long used the insight of the \emph{plastic surgery hypothesis}~\cite{barr2014plastic} where it states that the code ingredients to fix a bug already exist within the same project. Traditional \apr tools have manually designed pattern-~\cite{ghanbari2019prapr, saha2017elixir} or heuristic-based~\cite{jiang2018simfix, legoues2012genprog} approaches to finding and using such relevant code ingredients to generate fixes for bugs. However, the plastic surgery hypothesis has been largely ignored in \llm-based \apr. In fact, \llm provides a unique opportunity to fully automate the plastic surgery hypothesis idea via fine-tuning (learning project-specific information via model updates from the buggy project) and prompting (directly providing relevant code ingredients to the model), and make it directly applicable to different languages (since the \llm{s} are typically multi-lingual).%
Moreover, despite the intensive manual efforts involved, traditional \apr tools still cannot fully leverage project-specific information due to large search space for leveraging/composing existing code ingredients. In contrast, the project-specific information can effectively leveraged by \llm{s} due to their power in code understanding/vectorization, e.g., even partial/imprecise information may still guide \llm{s} in correct patch generation!
 To this end, we ask the question: \emph{How useful is the plastic surgery hypothesis in the era of \plm{s}}?








\mypara{Our Work.} To answer the question, we present \ourtech{\xspace} -- a \llm-based approach that automatically utilizes the plastic surgery hypothesis by systematically combining multiple fine-tuning and prompting strategies for \apr. \ourtech fine-tunes \plm{s} using two novel domain-specific training strategies: \textbf{\epfinetune} -- we fine-tune using the original buggy project by aggressively masking out a high percentage of tokens, which allows \plm to learn project-specific code tokens and programming styles; and \textbf{\rofinetune} -- which only masks out a single continuous code sequence per training sample, allowing the model to get used to the final \csapr task of predicting a single continuous code sequence. Furthermore, we directly leverage the ability for \plm{s} to understand natural language instructions and introduce a novel prompting strategy, \textbf{\idprompting}, which uses information retrieval and static analysis to obtain a list of relevant identifiers for the buggy lines. While such relevant identifiers are critical for fixing some difficult bugs, they may not be seen by the \llm during inference due to limited context window size. Through the use of prompting, we directly tell the model to use these extracted identifiers (relevant code ingredients) to generate the correct code. Finally, to perform repair, we combine all four model variants (including the base model, both fine-tuned models and the base model with prompting) for the final repair.





While our insight of leveraging the plastic surgery hypothesis for \llm-based \apr is generalizable across different types of \plm{s}, to implement \ourtech, we choose a recent \plm{\xspace}, \ctfive~\cite{wang2021codet5}, which is pre-trained on millions of open-source code snippets. \ctfive is an encoder-decoder model trained using \mspfull (\msp) objective where a percentage of tokens are masked out and each continuous masked token sequence is referred to as a masked span. Also, although we only extract relevant identifiers from the current buggy project (since this paper focuses on the plastic surgery hypothesis), our work can be easily extended to obtain other code information (such as relevant statements or functions) from other sources, such as  the massive pre-training corpora~\cite{husain2020codesearchnet} or historical bug-fixing datasets~\cite{jiang2019infer}, which can provide more coding knowledge for \llm{s}. Besides, although we mainly focus on using traditional string comparison algorithms for information retrieval in this paper, these techniques can be easily replaced by other frequency-based retrieval~\cite{robertson2009probabilistic} and neural search (or embedding-based search)~\cite{reimers2019sentence}.
  In summary, this paper makes the following contributions:


%


\begin{itemize}[noitemsep, leftmargin=*, topsep=0pt]
    \item \textbf{Dimension.} This paper is the first to revisit the important plastic surgery hypothesis in the era of \llm{s}. It opens up a new dimension for \llm-based \apr to incorporate previously neglected information from the buggy project itself to boost \apr performance. Furthermore, it demonstrates the promising future of retrieval-based prompting for modern \llm-based \apr.
    \item \textbf{Implementation.} We implement \ourtech based on the recent \ctfive model. We augment the model using two novel fine-tuning strategies: \epfinetune and \rofinetune, along with a novel prompting strategy based on information retrieval and static analysis: \idprompting. We combine the patches generated by all four models together and perform patch ranking to speed up \apr.% 
    \item \textbf{Evaluation Study.} We conduct an extensive evaluation against state-of-the-art \apr tools. On the widely studied \dfj 1.2 and 2.0 datasets~\cite{just2014dfj}, \ourtech is able to achieve the new state-of-the-art results of 89 and 44 correct bug fixes (15 and 8 more than best baseline) respectively.  Furthermore, we perform a broad ablation study to justify our design. \ourtech demonstrates for the first time that the plastic surgery hypothesis can substantially boost \llm-based \apr and advance state-of-the-art \apr, while being fully automated and general. Moreover, even partial/imprecise code ingredients may still effectively guide \llm{s} for \apr!
\end{itemize}


\section{Related Work and Current Study}
Skilled Chinese readers make use of the phonetic radicals to name characters \cite{chen1996functional, zhou1999sublexical, ding2004nature}, and previous studies measured how phonetic radicals influence character naming in two ways: regularity and consistency \cite{fang1986consistency, hue1992recognition, hsu2009orthographic}. The regularity is exemplified in Table~\ref{tab:0}, and the consistency is defined as the number of characters that share the same phonetic radicals and pinyin. For example, there are 12 characters sharing the phonetic radical 青 $<$qing$>$ in Table~\ref{tab:0}, among which 3 characters (精, 靖, 靖) have the same pinyin $<$jing$>$, so the consistency score for these characters is 0.25 (3/12). Many studies have found \textit{regularity and consistency effects} for human speakers %\Lingyuin{I add italics for clearance. Not sure whether robust should be included here.} 
- the \regular and more consistent characters are named faster and more accurately, 
%\Lingyuin{A bit vague here, it's okay but better to be more clear.}, 
and these effects are stronger for low-frequency characters than high-frequency ones \cite{lien1985consistency, liu2003regularity, tsai2005consistency}. 

Previous studies of Chinese character modeling with phonetic radicals as inputs have successfully simulated the regularity effect and consistency effect. \citet{yang2009simulating} trained a feed-forward network on 4,468 Chinese characters and tested the model on 120 characters (seen in the training). The input to the model includes the character's radicals and radicals' positions (e.g., left-right, up-down).\footnote{There are 10 different Chinese character structures clustered by the arrangement of the character radicals, e.g., left-right (日+青=晴), top-down (相+心=想), and enclosure (口+或=國). The left-right structure is the most common type (71\%) \cite{hsiao2006analysis}.} The output of the model is the phonological features (e.g., stop, lateral) of the character's pinyin. They also measured the human speakers' response latency\footnote{Response latency measures the response speed, usually in milliseconds.} on each of the 120 test characters. By comparing the human speakers' response latency and the model's sum squared error, they found very similar regularity and consistency effects. In addition, \citet{hsiao2004connectionist,hsiao2005differences} trained a feed-forward model on 2,159 left-right structured characters, with each character appearing according to its log token frequency. The input included each character's radicals, and the output was the character's pinyin. They analyzed the training accuracy of the model and found the model's sum squared errors lower for the \regular characters, which successfully simulated the regularity effect. 

The regularity and consistency effect revealed that both human speakers and the neural models utilized the statistic distribution of phonetic radicals in naming familiar characters. However, these effects can not be applied in unknown character naming since the speakers don't know the statistics of these characters. Therefore, we proposed a new metric (saliency of the phonetic radical) to measure how the phonetic radicals influence the speaker's unknown character naming behavior. Saliency is defined as the fraction of the \regular characters among all characters sharing the same phonetic radical. For example, the phonetic radical 青 $<$qing$>$ appeared in 12 characters in Table~\ref{tab:0}, among which 4 characters (清,情,圊,晴) are \regular. Thus the saliency score of 青 $<$qing$>$ is 0.33 (4/12). The more salient a phonetic radical is, the more likely the character that contains it is pronounced the same as its pinyin. 

We hypothesized that the human speakers would show a saliency effect in unknown character naming - they would name the characters more accurately if the phonetic radical is more salient. We expected to find a similar saliency effect in the models. In addition, we also closely examine the models' answers and humans' answers to investigate if the models can represent the human speaker's behavior.
% PTMTorrrent
\newcommand{\numberOfModelHub}{5\xspace}

\newcommand{\TotalNumberOfPackages}{{15,913}\xspace}
% 12401 from Hugging Face
% 185 from ONNX
% 33 from Model Hub
% 3245 from Model Zoo
% 49 from PyTorch Hub
% SUM (by Nick): 15,913

\newcommand{\HFNumberOfPackages}{{12,401}\xspace}
\newcommand{\HFNumberOfPackagesMetadata}{{124,427}\xspace}
\newcommand{\MZNumberOfPackages}{3,245\xspace}
\newcommand{\PHNumberOfPackages}{{49}\xspace}
\newcommand{\MHNumberOfPackages}{{33}\xspace}
\newcommand{\ONNXNumberOfPackages}{{185}\xspace}

\newcommand{\TotalDataSize}{\textasciitilde{61TB}\xspace}
\newcommand{\HFDataSize}{{61TB}\xspace}
\newcommand{\MZDataSize}{{115GB}\xspace}
\newcommand{\PHDataSize}{{1.5GB}\xspace}
\newcommand{\MHDataSize}{{721MB}\xspace}
\newcommand{\ONNXDataSize}{{441MB}\xspace}
%%%



% ICSE submission - HFTorrent v1

\newcommand{\PTMDatasetNPackages}{63,182\xspace}
\newcommand{\PTMDatasetPercentage}{{99.7\%}\xspace}
\newcommand{\PTMDatasetFailedPackages}{{186}\xspace}
\newcommand{\PTMDatasetFailedPercentage}{{0.3\%}\xspace}

\newcommand{\PTMDatasetNReposWithSignedCommits}{{132}\xspace}
\newcommand{\PTMDatasetPercentOfSignedCommits}{{0.208\%}\xspace}


\newcommand{\PercentOfVerifiedOrgs}{{3.188\%}\xspace}
\newcommand{\NOrganizations}{{6,243}\xspace}
\newcommand{\NVerifedOrgs}{{199}\xspace}

\newcommand{\NOfRepositoriesWithMalware}{{1}\xspace}
\newcommand{\PercentageOfRepositoriesWithMalware}{{0.002\%}\xspace}
\newcommand{\TotalRepositoriesForMalwareScanning}{{63,366}\xspace}
\section{Human Experiment}
\label{sec:human}
55 native speakers of Mandarin participated in this study. All of them are able to read and write in traditional Chinese scripts and pinyin. The average age is 26.3, and 80\% of them have an education background of college or above. In the experiments, they were asked if they knew the character and prompted to type the pinyin of the character. The detailed experiment procedure and sample questions are described in Appendix A. 

\subsection{Results: Human Answer's Accuracy} 
In general, the test characters are unknown to the participants.\footnote{Very few participants indicated that they know one or two test characters. For those who indicated that they knew the character, they still answered its pinyin incorrectly.} The accuracy is calculated on the syllable onset and final, ignoring the tone, since tones are more affected by the speaker's accent than syllable onsets and finals. For polyphone characters, as long as the participant named one correct pinyin, we counted it as correct. 
% The participants could not name these characters correctly. 
The average accuracy for all participants is 45.3\% (27 out of 60 characters), with a range of 26.7\% - 68.3\%. Some characters are more difficult to name than others. For example, 8 characters' accuracies are 0, meaning that none of the participants named them correctly. The character's accuracy is calculated as the proportion of participants who named it correctly, ranging from 0 - 98.2\%. There is a strong positive correlation between the character's accuracy and its phonetic radical's saliency (r = 0.62), which confirms our hypothesis about the saliency effect. The more salient the phonetic radical is, the more participants named the character correctly. 
%\begin{figure}[!t]
    %\centering
    %\includegraphics[width =0.48\textwidth]{graph/Histo_characc.png}
    %\caption{Histogram of character's accuracy}
    %\label{fig:char_acc}
%\end{figure}
The accuracy measures how well the human speakers can grasp the grapheme-phoneme distributional patterns in Chinese. The results show that even native speakers can not accurately predict the pronunciation of an unknown character, which reflects the complex nature of Chinese grapheme-phoneme mapping system. 
\subsection{Results: Human Answer's Variability} Since the participants named the character's pinyin differently, each character has a variety of unique answers. On average, each character has 6.7 answers, with a minimum of 2 answers and a maximum of 15 answers. The number of answers is negatively correlated with the saliency of the phonetic radical (r = -0.51), that the more salient the phonetic radical, the fewer number of answers the speakers guessed.  

We defined 5 answer types based on regularity. The participants either guessed the character's pinyin the same as its phonetic radical's (\regular), or changing the syllable final (\alliterating), the syllable onset (\rhyming), or both (\rad), or mistakenly used the semantic radical to name the character (\sem).\footnote{When examining the data, we found that some participants named the character the same as its semantic radical. We loosely defined this type of error as \sem type. It could also be that the participants applied \rad on the phonetic radical, and the pinyin happened to be the same as the semantic radical's pinyin. However, there's no way to confirm it. We asked some of our participants (with linguistic background) to explain how they guessed the pinyin, and none of them could articulate their thinking process.} %\Lingyuin{Might be a bit unclear.}
We presented the answer types for character `煔' as an example in Table \ref{tab:shan}, and defined the production probability $P_{p}$ by the proportion of participants named that answer type.

The average production probability for each type is listed in Table \ref{tab:answer_type}. Most of the participants are able to identify the phonetic radical correctly, as the average production probability of the semantic type is only 2\%. The regular answer type has the highest production probability (58\%), suggesting that the participants are more likely to name the character the same as its phonetic radical. The production probabilities of answer types for each character are plotted in Figure \ref{fig:variety} in Section \ref{sec:overlap}.
\begin{table}[!ht]
\small
\centering
\begin{tabular}{p{0.25\linewidth}p{0.38\linewidth}p{0.15\linewidth}}
\toprule
% \multicolumn{3}{l}{\begin{tabular}[c]{@{}l@{}}煔 /shan/, /qian/, meaning `sparkle'\\ Phonetic Radical: 占 /zhan/, meaning `to seize'\\ Semantic Radical: 炎 /yan/, meaning `fire' \end{tabular}} \\
\multicolumn{3}{l}{煔 $<$shan$>$, $<$qian$>$, `sparkle'}\\
\multicolumn{3}{l}{Phonetic Radical: 占 $<$zhan$>$, `to seize'}\\
\multicolumn{3}{l}{Semantic Radical: 炎 $<$yan$>$, `fire'}\\
\midrule
Answer Type & Answer(s) & $P_{p}$ (\%)\\
\midrule
\regular & $<$zhan$>$ & 36.4 \\
\alliterating & $<$zhen$>$ & 1.8 \\
\rhyming & $<$dan$>$ & 3.6 \\
% \rad & \begin{tabular}[c]{@{}l@{}}/nian/, /jian/, /dian/\\ /pou/, /yi/, /tie/\end{tabular} & 0.35 \\
\rad & \makecell[l]{$<$nian$>$, $<$jian$>$, \\$<$dian$>$,$<$pou$>$\\$<$yi$>$, $<$tie$>$} & 34.6 \\
\sem & $<$yan$>$ & 23.6 \\
\bottomrule
\end{tabular}
\caption{\small The answer types and production probability of human answers for polyphone `煔'.}
\label{tab:shan}
\end{table}

\begin{table}[!ht]
\small
\centering
\begin{tabular}{p{0.25\linewidth}p{0.3\linewidth}p{0.3\linewidth}}
\toprule
Answer Type & Average $P_{p}$ (\%) & Range of $P_{p}$ (\%) \\
 \midrule
\regular  & 58.0\std{25.8} & 0-98.2 \\
\alliterating  & 6.8\std{16.4} & 0-81.8 \\
\rhyming & 13.0\std{18.9} & 0-81.8 \\
\rad & 20.6\std{20.0} & 0-72.7 \\
\sem  & 1.6\std{4.6} & 0-23.6 \\
\bottomrule
\end{tabular}
\caption{\small The average production probability and its range for each answer type in human answers.}
\label{tab:answer_type}
\end{table}

\section{Transformer Model}
To model the joint probability of the syllable onset and final, we used seq-to-seq transformers \cite{vaswani2017attention} to generate the pinyin of Chinese characters trained from scratch.\footnote{We did not use a classification model because there are certain rules in pinyin formation (e.g., /ü/ can not follow /b/, /p/, /m/, /f/), which requires the model to learn the syllable onsets and finals jointly.}

\subsection{Experiment Setups}
Both encoder and decoder of all our models had 2 layers, 4 attention heads, 128 expected features in the input, and 256 as the dimension of the feed-forward network model. 
For training, we split the dataset into train/dev splits of 90-10, and replace those tokens that appear once in training data by \textlangle unk\textrangle. We also set dropout to 0.1, batch size to 16, and used Adam optimizer \cite{DBLP:journals/corr/KingmaB14} with varied learning rates in the training process computed according to \citet{vaswani2017attention}. We used 5 different random seeds, and trained 40 epochs with early stopping %
for all of our experiments. For inference, we set beam size to 3. 

\subsection{Experiment 1}
\label{sec:test_acc}

We trained a set of models to simulate the grapheme-phoneme mapping process in Chinese speakers. Our \base model used the phonetic radical's orthographic forms to generate syllable onset and final (without tone) of the target character.
We further examined whether identifying the phonetic radical before generating the syllable onset and final would improve the model's performance.  %
We labeled the phonetic radical's position (left or right) with two methods: \labm and \labs. \labm used the true position of the phonetic radical as the ground truth label. 
Besides, since human speakers do not always identify the phonetic radical's position correctly,
\labs labeled the position of the phonetic radical based on the phonetic similarity. We calculated the phonetic similarity between the character's pinyin and the two radicals' pinyins using the Chinese Phonetic Similarity Estimator \cite{li2018dimsim}. The radical with higher phonetic similarity was labeled as the phonetic radical.\footnote{For example, the character `烙' $<$luo4$>$ (`flatiron') consists of the semantic radical `火' $<$huo3$>$ (`fire') and the phonetic radical `各' $<$ge4$>$ (`each'). The distance between $<$luo4$>$ and $<$huo3$>$ is 7.5, and the distance between $<$luo4$>$ and $<$ge4$>$ is 35.6. For \labs, the output radical should be `left', although the left radical `火' is the semantic radical.} We further labeled the regularity type of the characters based on \labm and \labs, hence yielding \labmr and \labsr. Examples of input and gold output in the training data are shown in Table~\ref{input_example}. All the models were trained on \dall, \dmid, and \dhigh datasets as described in section~\ref{sec:train}.

Since previous studies suggested that the regularity and consistency effects are more prominent for the characters with low frequency than high frequency  \cite[e.g.,][]{ziegler2000phonology,chen2009homophone}, the frequency of the known characters might also influence how participants predict the unknown characters. We further added the frequency label as an input feature in the full training data as the \dallf model. The characters  were categorized into four categories based on their frequency: `rare' (frequency = 1), `low' (1 < frequency $\leq$ 50\% percentile), `mid' (50\% percentile < frequency $\leq$ 75\% percentile) and `high' (frequency > 75\% percentile).%
The distribution of regularity types is similar for the characters with different frequencies. The summary of the number of characters and each regularity type can be found in Appendix B, Table~\ref{app_tab:3_freq}. 

 In addition, we added two conditions for output in training all models: Shuffling and Adding tones. We shuffle the position of the syllable onset and final in model output to explore the impact of the generated order since we don't know if the human speakers identify the syllable onset or syllable final first in character naming. We also add tones before the `End' token in the generation to see whether it improves the model performance. Examples of input and output of the conditions are shown in Table \ref{input_example}. In total, there are 80 types of models with different settings. 

\begin{table}[t!]
\small
\centering
\begin{tabular}{p{0.18\linewidth}p{0.68\linewidth}}\toprule
Input & Begin, 火, 各, End \\
\midrule
Model & Output \\
\midrule
\base & Begin, l, uo, End \\
\labm & Begin, right, l, uo, End \\
\labs & Begin, left, l, uo, End \\
\labmr & Begin, right, irregular, l, uo, End \\
\labsr & Begin, left, rhyming, l, uo, End \\
\midrule\midrule
Condition & Input \\\midrule
\dallf & Begin, 火, 各, high, End \\
\midrule
Condition & Output (\base model as an example) \\
\midrule
{[+}Shuffle{]} & Begin, uo, l, End \\
{[+}Tone{]} & Begin, l, uo, 4, End \\
\bottomrule
\end{tabular}
\caption{\small \label{input_example}Input and gold output in the training data of our models and conditions for character `烙'$<$luo4$>$, tokens are separated by comma.}
\end{table}







\paragraph{Accuracy Results}
We calculated the test accuracy the same way as for the human data: we only counted the accuracy of the syllable onset and final. For polyphone characters, as long as the model predicted one correct pinyin, it is counted as correct. The average accuracy of all 400 models (80 types x 5 random seeds) is 42.1\%, which is significantly lower than the human's accuracy (45.3\%, t = 3.15, p<0.01). The average accuracy of each type of model is listed in Table~\ref{tab:3_dfreq}. The best performing model is \dallf with \labm without tone and with shuffling, which achieved an accuracy of 50.3\%. Compared to the \base model, adding the label of phonetic position label and the character's regularity label usually could improve the model's accuracy. Adding tone would generally hurt the model's accuracy. Shuffling the syllable onset and final and adding the frequency label in the input would not change the model's accuracy.  


\begin{table}[!ht]
\small
\centering
\begin{tabular}{llllll}
\toprule
data & label & -T-S & -T+S & +T-S & +T+S \\
\midrule
\multirow{5}{*}{\dall} & \base & 49.3 & 49.3 & 42.3 & 46.0 \\
 & \labm & 48.0 & 49.7 & 45.3 & 47.7 \\
 & \labs & 46.0 & 45.3 & 42.3 & 48.7 \\
 & \labmr & 47.0 & 48.7 & 48.7 & 49.7\\
 & \labsr & 44.0 & 47.3 & 45.0 & 48.3 \\
 \midrule

\multirow{5}{*}{\dmid} & \base & 41.7 & 41.3 & 38.7 & 41.7 \\
 & \labm & 44.3 & 43.0 & 44.0 & 42.3 \\
 & \labs & 41.3 & 43.3 & 41.3 & 42.3 \\
 & \labmr & 42.0 & 40.3 & 39.7 & 44.3 \\
 & \labsr & 37.7 & 42.7 & 39.0 & 42.0 \\
 \midrule

\multirow{5}{*}{\dhigh} & \base & 28.7 & 32.3 & 29.3 & 32.3 \\
 & \labm & 36.3 & 34.7 & 30.7 & 35.3 \\
 & \labs & 32.7 & 36.0 & 30.0 & 34.0 \\
 & \labmr & 31.3 & 31.7 & 31.3 & 32.0 \\
 & \labsr & 32.3 & 32.0 & 31.0 & 33.7 \\
 \midrule

\multirow{5}{*}{\begin{tabular}[c]{@{}l@{}}\textsc{all+}\\ \textsc{freq}\end{tabular}} & \base & 46.7 & 47.0 & 47.7 & 46.7 \\
 & \labm & 49.7 & 50.3 & 47.3 & 47.0 \\
 & \labs & 45.3 & 47.3 & 47.0 & 48.3 \\
 & \labmr & 46.3 & 49.3 & 47.0 & 48.0 \\
 & \labsr & 47.7 & 44.7 & 44.0 & 47.7 \\
 \bottomrule
\end{tabular}
\caption{\label{tab:3_dfreq}The average accuracy (over 5 seeds) on test set for models trained on \dhigh, \dmid, or adding frequency label as input features on \dall. {+}T, {-}T, {+}S, {-}S refers to adding tone, no tone, shuffling, and no shuffling, respectively.}
\end{table}

\subsection{Experiment 2}
In Experiment 1, the input of our models only used the orthographic form of the radicals, which is how the previous literature described the Chinese grapheme-phoneme mapping process. However, the models might not have enough data to learn the full mapping from radicals to pinyin because many radicals only appeared once or twice in the training data since we only included compound characters with the left-right structure. For example, the phonetic radical `乘' $<$cheng$>$ only occurred once in the character `剩' $<$sheng$>$ in the training data.\footnote{We choose the first pinyin from the pinyin package for polyphone radicals.} The models would not be able to accurately learn the pinyins of these radicals. However, human speakers know the pinyin of most radicals, since many radicals are also commonly used as stand-alone characters, e.g., `乘' is a stand-alone character meaning `to multiply'. In order to better model the human speakers, it is necessary to inject pinyin of the radicals as external information to the model. The model would also benefit from the added radicals' pinyin to generate the character's pinyin. 

In addition, pinyin also plays an important role in modern Chinese speakers' reading and spelling experience. Pinyin is a Romanized phonetic coding system created in 1958 to promote literacy \cite{zhou1958}. In the information age, pinyin has become indispensable in Chinese speakers' lives because it's the dominant typing system for computers, smart phones, and electronic devices. The prevalent experience of typing characters through pinyin has challenged the traditional view that Chinese characters are processed purely through orthographic forms \cite{tan2013china}. Many recent studies have found that pinyin mediates the character recognition process\cite{chen2017effect,lyu2021comparison,yuan2022role}. To better capture modern Chinese speakers' character naming process, it is necessary to incorporate the radical's orthographic form as well as its pinyin in our models. 

Therefore, in Experiment 2, we added the radical's pinyin (syllable onset, syllable final, and tone) in the input, as shown in Table~\ref{tab:EXP2}. We used the same model variations as in Experiment 1\footnote{For the output, we added \labm, \labs, \labmr, \labsr as well as adding tone and shuffling. For the input, we added frequency label to create \dallf.} and trained 80 different types of models (5 random seeds for each type) with the new input. The training settings are the same as Experiment 1. 
\begin{table}[!ht]
    \centering
    \small
    \begin{tabular}{ll}
    \toprule
       Input  & Begin, 火, h, uo, 3, End, 各, g, e, 4, End \\
    \bottomrule
    \end{tabular}
    \caption{Input in the training data for Experiment 2 using `烙' $<$luo4$>$ as an example.}
    \label{tab:EXP2}
\end{table}

\paragraph{Accuracy Results}
Adding pinyin to the input has increased the model's accuracy.\footnote{We can not fully rule out the possibility that the increased accuracy is due to the model having longer inputs with pinyin instead of the model making use of the phonetic information. However, the input length might not have a significant impact on the models because our models with frequency labels (\dall vs \dallf) also vary in input lengths but the accuracies didn't change much.} The average accuracy of 400 models in Experiment 2 is 47.4\%, which is significantly higher than the human's accuracy (t = -2.7, p <0.01). The accuracy for each type of model is listed in Table~\ref{exp2_tab:3_dfreq} in Appendix B. The best performing model is \dallf with \labmr without tone and with shuffling, which achieved an accuracy of 55\%. The effects of different labels, adding tone, and shuffling are similar to the models in Experiment 1. 
\begin{table}[!t]\begin{center}
\caption{\textbf{Analysis of offset mechanisms in 360Attention and backbone variants} on 360BEV-Matterport dataset.}
\vskip -1ex
\label{tab:analysis}
\setlength{\tabcolsep}{1mm}
\renewcommand{\arraystretch}{1.2}
\resizebox{\columnwidth}{!}{
    \begin{tabular}{ l l | c | c | l}
    \toprule[1pt]
    \textbf{Methods} & \textbf{Backbone} & \textbf{\#Param} & \textbf{FLOPs} & \textbf{mIoU} \\ \midrule\midrule
    
    \circled{1} Ours (360Attention offset) & MiT-B0 & 04.60M  & 248.57G & 36.98     \\
    \circled{2} Ours (360Attention offset) & MiT-B2 & 26.30M & 283.94G & 44.32 \\ 
    \circled{3} Ours (360Attention offset) & MiT-B4  & 62.91M & 341.34G &  \textbf{45.53}    \\  \midrule
    \circled{4} Ours (Multi-scale offset) & MiT-B2  & 26.43M  &284.17G &43.65~\obf{-0.67}   \\
    \circled{5} Ours (Fixed-range offset) & MiT-B2  & 26.30M & 283.44G &  43.28~\obf{-1.04}\\
    \circled{6} Ours (Separate offset) & MiT-B2 & 26.19M & 279.18G &  42.82~\obf{-1.50}\\\midrule
    \circled{7} Ours (360Attention offset) & MSCA-B  & 27.69M &274.59G & \textbf{46.31}~\gbf{+1.99} \\ 

    \bottomrule
    \end{tabular}
}
\end{center}
\vskip -4ex
\end{table}
We provide some comments on the growth conditions which constituted the majority of our analysis in sections \ref{sec:Hmixing} and \ref{sec:Hsigma}. In the simplest cases of Lemma \ref{lemma:unstableGrowth}, growth was established in an analogous fashion to the old one-step expansion condition (\ref{eq:oldOneStepExpansion}), finding the relevant Jacobians $M_j$ and checking that their expansion factors $K(M_j)$ satisfy
\begin{equation}
    \label{eq:discussionOneStep}
    \sum_j \frac{1}{K(M_j)} <1.
\end{equation}
For the more complicated cases, the inductive method used to establish growth near the accumulation points in Lemma \ref{lemma:unstableGrowth} and the weakened one-step expansion condition (\ref{eq:oneStep}) both address the same fundamental issue: the splitting of unstable curves by singularities into an unbounded number of small components. They circumvent this obstacle in rather different ways, however. While (\ref{eq:oneStep}) generalises (\ref{eq:discussionOneStep}) to ensure an growth of unstable curves `on average' (see \cite{chernov_statistical_2009} for a precise statement), our inductive method is a more direct adaptation of (\ref{eq:discussionOneStep}), using it to generate contradictory geometric conditions which a hypothetical non-growing unstable curve must satisfy. It may be possible to prove Theorem \ref{sec:Hmixing} using (\ref{eq:oneStep}) as the basis for growth. Since we required (\ref{eq:oneStep}) anyway for proving Theorem \ref{thm:HsigmaExp}, this could potentially condense our analysis, but only to a minor extent. A convenience of the method used in section \ref{sec:Hmixing} is that, by way of the `simple intersection' property, it naturally gives geometric information on the images of manifolds, useful for proving the property \textbf{(M)} of Theorem \ref{thm:katok-strelcyn}.

We expect that essentially analogous analysis can be applied to establish mixing properties in a wide class of piecewise linear non-uniformly hyperbolic maps, including those (like the OTM) which sit on the boundary of ergodicity and beyond. While we have relied on the precise partition structure of $H_\sigma$, its fundamental feature (self-similar sequences of elements $A^k$, sharing boundaries with its neighbours $A^{k-1},A^{k+1}$ and accumulating onto some point $p$) is quite typical to return map systems. See, for example, those of various stadium billiards \cite{chernov_chaotic_2006,chernov_improved_2008,chernov_statistical_2009} and LTMs \cite{springham_polynomial_2014}. Indeed, the same method can be used to prove the Bernoulli property for non-monotonic LTMs \cite{myers_hill_mixing_2022}, where monotonicity of the manifold images cannot be assumed and the classical argument \cite{sturman_mathematical_2006} fails. The OTM is the pointwise limit of these maps as the boundary shrinks to null measure. It further has utility in proving growth conditions for maps which are uniformly hyperbolic but possess regions $A_j$ where the hyperbolicity is very weak, signified by $K(M_j) \approx 1$, so that (\ref{eq:discussionOneStep}) fails. Typically this leads to suboptimal bounds on mixing windows, see e.g. \cite{wojtkowski_model_1981,przytycki_ergodicity_1983,myers_hill_family_2022}. The map $H_{(\eta,\eta)}$ for $\eta \approx 1/2$ is another example, possessing weak hyperbolicity over $A_2, A_3$. Letting $\varepsilon = |\eta-1/2|>0$, there is an upper bound $N = N(\varepsilon)$ on escape times from the intersections $A_2\cap \sigma, A_3 \cap \sigma$. The growth lemma then follows by applying the inductive step roughly $N$ times and can be established for arbitrarily small $\varepsilon$, opening the door to establishing optimal mixing windows.

The above gives two examples of piecewise linear perturbations to $H$ where mixing with respect to Lebesgue is preserved and our methods can be applied. Nonlinear perturbations to the shear profiles complicate the analysis in several ways. Firstly as the map's Jacobians takes on a broader range of values, cone invariance becomes an increasingly harder condition to establish. Cones must be widened, giving looser bounds on expansion factors, which may already be weak due to new regions of weaker stretching. This, together with the change from polygonal to curvilinear return time partition elements and nonlinear local manifolds, adds some complexity to showing growth conditions. This does not rule out certain (small) nonlinear perturbations however. There is some leeway in the inequalities which govern cone invariance and growth of local manifolds, the latter of which is not too dissimilar from the piecewise linear setting (see Lemmas \ref{lemma:piecewiseApprox}, \ref{lemma:componentLength}). Certain small perturbations would not alter the \emph{topological} structure of the return time partition, i.e. which elements share boundaries, the key information needed for setting up the induction. Finally while the partition elements would no longer be polygonal, only coarse geometric information is required for verifying each inductive step. Following the above, a potential perturbation could be to replace the linear portions of each shear by a cubic, perturbing the tent profile
\[  f(t) = \begin{cases} 2t & 0 \leq t \leq 1/2, \\ 2(1-t) & 1/2 \leq t \leq 1 ,\end{cases} \]
of the OTM shears to
\[  f_a(t) = \begin{cases} \frac{1}{8} t \left(16 - a + 6at - 8at^{2} \right) & 0 \leq t \leq 1/2, \\ \frac{1}{8}\left(1-t\right)\left( 16 - a + 6a\left(1-t\right) - 8a\left(1-t\right)^{2}\right)  & 1/2 \leq t \leq 1, \end{cases}   \]
for $a>0$. For small enough $a$ the gradient range $f'(t)$ is restricted to small neighbourhoods of $\{ 2, -2\}$ and the escape time partition retains a similar structure. We illustrate this in Figure \ref{fig:perturbations}, showing escapes from the square $S_3$ under the map $G \circ F$, equivalent to escapes from the perturbed $A_3$ under the $G \circ F$, but with a cleaner geometry for comparison. When $a$ is too large the analogy to the OTM breaks down. At $a=16$ the map is twice differentiable everywhere and features a new source of slowed mixing, the Jacobian is the identity at the corner points $x,y \in \{  0, 1/2 \}$ giving locally parabolic behaviour (visible in the escape time partition). 

\begin{figure}
    \centering
    \includegraphics[width=0.24 \linewidth]{0.png}
    \includegraphics[width=0.24 \linewidth]{4.png}
    \includegraphics[width=0.24 \linewidth]{8.png}
    \includegraphics[width=0.24 \linewidth]{16.png}
    \caption{Partition of escape times from $S_3$ under the mapping $F \circ G$ for $a= 0,4,8,16$. }
    \label{fig:perturbations}
\end{figure}


\section{Acknowledgements}
We would like to thank the anonymous reviewers and action editor Micha Elsner for their valuable feedback. We would also like to thank Kevin Gimpel, Allyson Ettinger, Virginia Valian, Martin Chodorow, and Kyle Gorman for their insightful discussions and suggestions.



\bibliography{anthology,custom}
\bibliographystyle{acl_natbib}


\appendix
\section{Skew Equations}
We will justify and show the three equations used in Lemma \ref{skew rel} to narrow our search for these skew axial algebras. Although they do not provide much use to understanding how these algebras could be constructed, they do make the proof easier.

Suppose $v$ is an $\mu$-eigenvector of an axis, $x$, where $\mu\neq1$. Then the projection on that axis should be equal to 0; that is, $\lm_x(v)=0$. Coincidentally, nearly all of the eigenvectors in Lemma \ref{eigen a} and \ref{eigen b} satisfy that rule. However we have
\begin{equation*}
 0=\lm_b\left(-\frac{P}{\bt}a+Pb+c\right) = -\frac{P}{\bt}\lmf_1+P+\lmf_2.
\end{equation*}
Whence we get Equation (\ref{proof1}).

\begin{defn}
Let $x$ be a $\mon{\al,\bt}$-axis in $A$, $\lm\in \{1,0, \al, \bt\}$ and $v\in A$. We denote $[v]^x_\lm$ to be the component of $v$ in $ A_\lm(x)$. 
\end{defn}
\begin{lem}
Let $w:=\frac{1}{2}(b-c)$. We have $[a]^a_\bt=0$, $[b]^a_\bt=w$, $[c]^a_\bt=-w$, $[\sg]^a_\bt=0$. Further, $[ab]^a_\bt=\bt w$, $[ac]^a_\bt=-\bt w$, $[bc]^a_\bt=0$, $[a\sg]^a_\bt=0$, $[b\sg]^a_\bt=\dt^fw$, $[c\sg]^a_\bt=-\dt^fw$ and $[\sg^2]^a_\bt=0$.
\end{lem}
\proof
As $a\in A_1(a)$, it has no $\bt$-component in $A_\bt(a)$ and $[a]^a_\bt=0$. As $\sg\in A_{\{1,0,\al\}}(a)$, it has no $\bt$-component in $A_\bt(a)$ and $[\sg]^a_\bt=0$. We can express $b$ in terms of the eigenvectors of $\text{ad}_a$ in Lemma \ref{eigen a}. The reader can check
\[ b= \lm_1 a+ \frac{1}{\al}\left(\ep a+\frac{1}{2}(\al-\bt)(b+c)-\sg\right)+ \frac{1}{\al}\left(\gm a +\frac{1}{2}\bt(b+c)+\sg\right)+\frac{1}{2}(b-c).\]
Thus $[b]_\bt^a=w$. As $c=b^{\tu{a}}$, we get $[c]_\bt^a=-w$.

Let $x, y \in A_{\{0,1,\al\}}(a)$ and notice $x^2, xy\in A_{\{1,0,\al\}}(a)$ and so has no $\bt$-component in $A_\bt(a)$. Therefore $[\sg^2]^a_\bt=[a\sg]^a_\bt=0$. Also
\[ [bc]_\bt^a=P\left([a]_\bt^a+\frac{1}{\bt}[\sg]_\bt^a\right)=0.\]
Note that
\[ [ab]_\bt^a=[\sg]_\bt^a+\bt[a]_\bt^a+\bt[b]_\bt^a=\bt w\]
and 
\[ [b\sg]_\bt^a=(\al-\bt)[\sg]_\bt^a+\bt(\al-\bt)[a]_\bt^a+dt^f[b]_\bt^a=\dt^f w.\]
Applying $\tu{a}$, we get $[ac]_\bt^a$ and $[c\sg]_\bt^a$. \qed



Let $u:= (b -\al)a - \bt b=\sg -(\al-\bt)a$. As $A_\bt(b)=\{0\}$, we have that $u\in A_{\{1,0\}}(b)$. By Lemma \ref{Seress}, the following holds
\[b(au)=(ba)u.\]
Notice
\[ au = a(\sg -(\al-\bt)a)=(\dt -(\al-\bt))a+\frac{1}{2}\bt(\al-\bt)(b+c)+(\al-\bt)\sg\]
and so
\begin{eqnarray*}
[b(au)]_\bt^a &=& (\dt -(\al-\bt))[ab]_\bt^a+\frac{1}{2}\bt(\al-\bt)([b]_\bt^a+[bc]_\bt^a)+(\al-\bt)[b\sg]_\bt^a\\
& =& \left(\bt(\dt -(\al-\bt))+\frac{1}{2}\bt(\al-\bt)+(\al-\bt)\dt^f\right)w
\end{eqnarray*}
We also have 
\begin{eqnarray*}
[(ba)u]_\bt^a&=&[(\sg+\bt a +\bt b)(\sg -(\al-\bt)a)]_\bt^a\\
&=& [\sg^2]_\bt^a -(\al-2\bt)[a\sg]_\bt^a +\bt [b\sg]_\bt^a -\bt(\al-\bt)[a]_\bt^a - \bt(\al-\bt)[ab]_\bt^a\\
&=& (\bt\dt^f -\bt^2(\al-\bt)) w
\end{eqnarray*}
By Lemma \ref{Seress}, we have $0=(ba)u-b(au)$ moreover $0=[(ba)u]_\bt-[b(au)]_\bt$. Looking at the coefficient of $w$, we have
\begin{eqnarray*} 
0&=& (\bt\dt^f-\bt^2(\al-\bt))\\
& -& \left(\bt \dt -\bt(\al-\bt)+\frac{1}{2}\bt(\al-\bt)+(\al-\bt)\dt^f\right)\\
&=&-\bt^2(\al-\bt) -\bt\dt+\frac{1}{2}\bt(\al-\bt)-(\al-2\bt)\dt^f.
\end{eqnarray*}
Rearranging we get Equation (\ref{proof2}).

Let $v:=Pa+\frac{P}{\bt}\sg -\al c=c(b-\al)$. Notice that $v \in A_{\{1,0\}}(b)$. Again by Lemma \ref{Seress}, the following holds
\[b(av)=(ba)v.\]
We have
\begin{eqnarray*}
av &=& Pa +\frac{P}{\bt}\left(\dt a + \frac{1}{2}\bt(\al-\bt)(b+c) +(\al-\bt)\sg\right)\\
& -&\al(\bt a +\bt c +\sg)\\
&=&\left(P +\frac{P}{\bt}\dt -\al\bt\right)a+\left(\frac{1}{2}(\al-\bt)P\right)b\\
&+&\left(\frac{1}{2}(\al-\bt)P-\al\bt\right)c+\left(\frac{P}{\bt}(\al-\bt)-\al\right)\sg.
\end{eqnarray*}
Therefore
\begin{eqnarray*}
[b(av)]_\bt^a &=&\left(P +\frac{P}{\bt}\dt -\al\bt\right)[ab]_\bt^a+\left(\frac{1}{2}(\al-\bt)P\right)[b]_\bt^a\\
&+&\left(\frac{1}{2}(\al-\bt)P-\al\bt\right)[bc]_\bt^a+\left(\frac{P}{\bt}(\al-\bt)-\al\right)[b\sg]_\bt^a.\\
&=&\left(\bt \left(P +\frac{P}{\bt}\dt -\al\bt\right)+\dt^f\left(\frac{P}{\bt}(\al-\bt)-\al\right)\right)w
\end{eqnarray*}
We also have
\begin{eqnarray*}
[(ba)v]_\bt^a&=&\left[\left(\bt a +\bt b +\sg\right)\left(Pa+\frac{P}{\bt}\sg -\al c\right)\right]_\bt^a\\
&=&2P[a\sg]_\bt^a +\frac{P}{\bt}[\sg^2]_\bt^a -\al [c \sg]_\bt^a + \bt P [a]_\bt^a -\al\bt [ac]_\bt^a\\
&+&\bt P [ab]_\bt^a +P[b\sg]_\bt^a -\al\bt [bc]_\bt^a\\
&=&\left(\al \dt^f +\al\bt^2 +\bt^2 P  +P\dt^f\right)w
\end{eqnarray*}
By Lemma \ref{Seress}, $0=[b(av)]^a_\bt-[(ba)v]^a_\bt$ and looking at the coefficient of $w$, we get 
\begin{eqnarray*}
0&=&[b(av)]_\bt-[(ba)v]_\bt\\
&=&\left(\bt P +\dt P -\al\bt^2+\frac{1}{2}(\al-\bt)P+\frac{P}{\bt}(\al-\bt)\dt^f -\al\dt^f\right)\\
&-&\left(\bt^2P +P\dt^f+\al\dt^f +\al\bt^2 \right)\\ 
&=&\left(\frac{P}{\bt}\left[\bt^2 +\bt\dt+\frac{1}{2}\bt(\al-\bt)+(\al-2\bt)\dt^f-\bt^3\right]-2\al(\dt^f+\bt^2)\right).
\end{eqnarray*}
From Equation (\ref{proof2}), we get that
\begin{eqnarray*}
0&=&\frac{P}{\bt}\left[\bt^2 -\bt^2(\al-\bt) +\frac{1}{2}\bt(\al-\bt)-(\al-2\bt)\dt^f\right.\\
&+&\left.\frac{1}{2}\bt(\al-\bt)+(\al-2\bt)\dt^f-\bt^3\right]-2\al(\dt^f+\bt^2)\\
&=&\frac{P}{\bt}\left[\bt^2 -\bt^2(\al-\bt) +\bt(\al-\bt)-\bt^3\right]-2\al(\dt^f+\bt^2)\\
&=&\frac{P}{\bt}\al\bt\left[1-\bt\right]-2\al(\dt^f+\bt^2).
\end{eqnarray*}
Hence we get Equation (\ref{proof3}).

\section*{Acknowledgements}
I would like to thank Professor Sergey Shpectorov for his guidance throughout my PhD studies so far and pushing me to complete this paper. I would also like to thank my family for their continuing support. 
\end{CJK*}

\clearpage
\newpage

\begin{figure*}
    \centering
    \includegraphics[scale = 0.35]{zi_table.png}
\end{figure*}
\end{document}
