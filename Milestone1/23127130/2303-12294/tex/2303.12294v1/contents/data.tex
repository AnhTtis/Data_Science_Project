\section{Data}
The base character dataset consists of 4,341 Chinese characters constructed from the \href{https://github.com/cjkvi/cjkvi-ids/blob/master/ids-analysis.txt}{IDS dataset} in CHISE project \cite{morioka2008chise}. The original IDS (Ideographic Description Sequence) dataset contains 18,347 characters used in China, Japan, and Korea with the decomposition of each character's phonetic and semantic radicals.\footnote{The phonetic and semantic radicals are decomposed according to \textit{Shuowen Jiezi} 《說文解字》.} The character selection criterion include: 1) is used in Chinese; 2) is a phono-semantic compound; 3) has a left-right structure\footnote{Following \citet{hsiao2004connectionist}, we only selected left-right structure to make sure that the character's structure is not a variable in our study.}. The character's pinyin, along with its phonetic and semantic radical's pinyin, was collected using the \href{https://pypi.org/project/pinyin/}{pinyin package}. The frequency of each character was extracted from BLCU Corpus Center \cite{xun2016}.
We further labeled each character's regularity: \regular, \alliterating, \rhyming, and \rad as described in Table \ref{tab:0}. In addition, we calculated each phonetic radical's saliency.

There are 660 radicals after decomposing the 4,341 characters, among which 46 radicals only serve as the semantic radicals; 493 radicals only serve as the phonetic radicals; 121 radicals serve as both semantic and phonetic radicals. Each radical appears in 7 characters on average, with a range of 1 to 30. 80\% of the characters in our database have the phonetic radical on the right, with many exceptions, e.g., the semantic radical `戈' $<$ge$>$ always appears on the right.

%%%It is difficult to identify the phonetic radical without linguistic knowledge, since most of the radicals can serve as the phonetic radical or the semantic radical depending on the character. Usually, the right-sided radical is the phonetic radical, with many exceptions. For example, when the radical `口' /kou/ (`mouth') occurs on the right, it is the phonetic radical as in `扣' /kou/ (`to hold'), with the exception of `叩' /kou/, where `口' appears on the left. Or, for the radical `鳥' /niao/ (`bird'), it is the semantic radical although it appears on the right-side as in `鴨' /ya/ (`duck'). %%%

%%%\begin{figure}
    %\centering
    %\includegraphics[width=\linewidth]{graph/Histo_C%onsis.png}
    %\caption{Histogram of the consistency %distribution}
    %\label{fig:rad}
%\end{figure}%%%
\subsection{Test Data}
We selected 60 characters with different phonetic radicals from the dataset as our test data, which are listed in Table 14 in Appendix B. The test characters are selected following three criteria to ensure that human speakers are unfamiliar with the character, while familiar with the phonetic radicals, 1) the character appears less than 5 times in the whole corpus, 2) the phonetic radical in each character appears in more than 4 other characters.
The average saliency score for these phonetic radicals is 0.43, with the score distribution shown in Figure \ref{fig:test_cons}.
\begin{figure}[!ht]
    \centering
    \includegraphics[width = \linewidth]{graph/Histo_test.png}
    \caption{\small The histogram of saliency scores for 60 test characters' phonetic radicals.}
    \label{fig:test_cons}
\end{figure}
Among these characters, 22 of them have more than one pinyin, e.g., `硞' ($<$que$>$, $<$ke$>$, $<$ku$>$), which yields 88 pinyins for 60 characters. The distribution of the regularity type for the test characters is shown in Table~\ref{tab:new}.  

\begin{table}[!ht]
\small
\centering
\begin{tabular}{p{0.35\linewidth}p{0.23\linewidth}p{0.23\linewidth}}
\toprule
Regularity Type & \# pinyin & \# character \\
\midrule
\textit{regular} & 30 & 30 \\
\textit{alliterating} & 6 & 6 \\
\textit{rhyming} & 24 & 19 \\
\textit{irregular} & 28 & 20 \\
\bottomrule
\end{tabular}
\caption{\small The distribution of regularity types of pinyins and characters for the test data.}
\label{tab:new}
\end{table}

\subsection{Training Data}
\label{sec:train}
We exclude the 60 test characters and use the rest of the characters as our training data (4,281). The \regular is the most common type (42.7\%), followed by \rad, \rhyming, and \alliterating. Since many of the characters have extremely low frequency and are not known to the Chinese speakers, we used three training datasets with characters of different frequencies to represent the native speakers' vocabulary size. The \dall dataset used all 4,281 characters. The \dmid dataset consists of 2,140 characters whose frequencies are in the top 50\% percentile. The \dhigh dataset consists of 1,070 characters with frequencies in the top 25\% percentile. The statistics of these training sets are shown in Table \ref{tab:train}. Each training set has similar regularity distribution. 

\begin{table}[!t]
\small
\centering
\begin{tabular}{p{0.32\linewidth}p{0.15\linewidth}p{0.15\linewidth}p{0.15\linewidth}}
\toprule
Training Data & \dall & \dmid & \dhigh \\
\midrule
\# characters & 4,281 & 2,140 & 1,070 \\\midrule
\regular (\%) & 42.7 & 43.3 & 42.1 \\
\alliterating (\%) & 7.8 & 8.1 & 8.7 \\
\rhyming (\%)& 23.6 & 23.3 & 22.5 \\
\rad (\%)& 25.9 & 25.3 & 26.7 \\
\bottomrule
\end{tabular}
\caption{\small Number of characters and percentage of regularity types for our training datasets.}
\label{tab:train}
\end{table}
%\begin{table}[!t]
%\centering
%\small
%\begin{tabular}{lll}
%\toprule
%Training Data & All & High Frequency \\
%\midrule
%Characters & 4281 & 1381 \\
%Regular & 42.7\% & 43.7\% \\
%Alliterating & 7.8\% & 8.7\% \\
%Rhyming & 23.6\% & 21.7\% \\
%Radical & 25.9\% & 25.9\% \\
%Phonetic Radicals & 614 & 470 \\
%Mean consistency & 0.48 (0.3) & 0.48 (0.3)\\
%\bottomrule
%\end{tabular}
%\caption{The summary of training datasets with all characters and high frequency characters}
%\label{tab:train}
%\end{table}

