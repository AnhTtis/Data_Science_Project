\section{Human Experiment}
\label{sec:human}
55 native speakers of Mandarin participated in this study. All of them are able to read and write in traditional Chinese scripts and pinyin. The average age is 26.3, and 80\% of them have an education background of college or above. In the experiments, they were asked if they knew the character and prompted to type the pinyin of the character. The detailed experiment procedure and sample questions are described in Appendix A. 

\subsection{Results: Human Answer's Accuracy} 
In general, the test characters are unknown to the participants.\footnote{Very few participants indicated that they know one or two test characters. For those who indicated that they knew the character, they still answered its pinyin incorrectly.} The accuracy is calculated on the syllable onset and final, ignoring the tone, since tones are more affected by the speaker's accent than syllable onsets and finals. For polyphone characters, as long as the participant named one correct pinyin, we counted it as correct. 
% The participants could not name these characters correctly. 
The average accuracy for all participants is 45.3\% (27 out of 60 characters), with a range of 26.7\% - 68.3\%. Some characters are more difficult to name than others. For example, 8 characters' accuracies are 0, meaning that none of the participants named them correctly. The character's accuracy is calculated as the proportion of participants who named it correctly, ranging from 0 - 98.2\%. There is a strong positive correlation between the character's accuracy and its phonetic radical's saliency (r = 0.62), which confirms our hypothesis about the saliency effect. The more salient the phonetic radical is, the more participants named the character correctly. 
%\begin{figure}[!t]
    %\centering
    %\includegraphics[width =0.48\textwidth]{graph/Histo_characc.png}
    %\caption{Histogram of character's accuracy}
    %\label{fig:char_acc}
%\end{figure}
The accuracy measures how well the human speakers can grasp the grapheme-phoneme distributional patterns in Chinese. The results show that even native speakers can not accurately predict the pronunciation of an unknown character, which reflects the complex nature of Chinese grapheme-phoneme mapping system. 
\subsection{Results: Human Answer's Variability} Since the participants named the character's pinyin differently, each character has a variety of unique answers. On average, each character has 6.7 answers, with a minimum of 2 answers and a maximum of 15 answers. The number of answers is negatively correlated with the saliency of the phonetic radical (r = -0.51), that the more salient the phonetic radical, the fewer number of answers the speakers guessed.  

We defined 5 answer types based on regularity. The participants either guessed the character's pinyin the same as its phonetic radical's (\regular), or changing the syllable final (\alliterating), the syllable onset (\rhyming), or both (\rad), or mistakenly used the semantic radical to name the character (\sem).\footnote{When examining the data, we found that some participants named the character the same as its semantic radical. We loosely defined this type of error as \sem type. It could also be that the participants applied \rad on the phonetic radical, and the pinyin happened to be the same as the semantic radical's pinyin. However, there's no way to confirm it. We asked some of our participants (with linguistic background) to explain how they guessed the pinyin, and none of them could articulate their thinking process.} %\Lingyuin{Might be a bit unclear.}
We presented the answer types for character `煔' as an example in Table \ref{tab:shan}, and defined the production probability $P_{p}$ by the proportion of participants named that answer type.

The average production probability for each type is listed in Table \ref{tab:answer_type}. Most of the participants are able to identify the phonetic radical correctly, as the average production probability of the semantic type is only 2\%. The regular answer type has the highest production probability (58\%), suggesting that the participants are more likely to name the character the same as its phonetic radical. The production probabilities of answer types for each character are plotted in Figure \ref{fig:variety} in Section \ref{sec:overlap}.
\begin{table}[!ht]
\small
\centering
\begin{tabular}{p{0.25\linewidth}p{0.38\linewidth}p{0.15\linewidth}}
\toprule
% \multicolumn{3}{l}{\begin{tabular}[c]{@{}l@{}}煔 /shan/, /qian/, meaning `sparkle'\\ Phonetic Radical: 占 /zhan/, meaning `to seize'\\ Semantic Radical: 炎 /yan/, meaning `fire' \end{tabular}} \\
\multicolumn{3}{l}{煔 $<$shan$>$, $<$qian$>$, `sparkle'}\\
\multicolumn{3}{l}{Phonetic Radical: 占 $<$zhan$>$, `to seize'}\\
\multicolumn{3}{l}{Semantic Radical: 炎 $<$yan$>$, `fire'}\\
\midrule
Answer Type & Answer(s) & $P_{p}$ (\%)\\
\midrule
\regular & $<$zhan$>$ & 36.4 \\
\alliterating & $<$zhen$>$ & 1.8 \\
\rhyming & $<$dan$>$ & 3.6 \\
% \rad & \begin{tabular}[c]{@{}l@{}}/nian/, /jian/, /dian/\\ /pou/, /yi/, /tie/\end{tabular} & 0.35 \\
\rad & \makecell[l]{$<$nian$>$, $<$jian$>$, \\$<$dian$>$,$<$pou$>$\\$<$yi$>$, $<$tie$>$} & 34.6 \\
\sem & $<$yan$>$ & 23.6 \\
\bottomrule
\end{tabular}
\caption{\small The answer types and production probability of human answers for polyphone `煔'.}
\label{tab:shan}
\end{table}

\begin{table}[!ht]
\small
\centering
\begin{tabular}{p{0.25\linewidth}p{0.3\linewidth}p{0.3\linewidth}}
\toprule
Answer Type & Average $P_{p}$ (\%) & Range of $P_{p}$ (\%) \\
 \midrule
\regular  & 58.0\std{25.8} & 0-98.2 \\
\alliterating  & 6.8\std{16.4} & 0-81.8 \\
\rhyming & 13.0\std{18.9} & 0-81.8 \\
\rad & 20.6\std{20.0} & 0-72.7 \\
\sem  & 1.6\std{4.6} & 0-23.6 \\
\bottomrule
\end{tabular}
\caption{\small The average production probability and its range for each answer type in human answers.}
\label{tab:answer_type}
\end{table}
