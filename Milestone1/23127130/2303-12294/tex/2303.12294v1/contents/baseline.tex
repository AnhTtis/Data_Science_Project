\section{Baseline model}
\begin{CJK*}{UTF8}{bkai}

We constructed a baseline model using the phonetic radical to predict the pinyin of the 60 rare characters. Given a new semantic-phonetic compound SP, P is the phonetic radical. For all characters XP containing the phonetic radical P, if P's consistenc $\geq$ 0.5 (i.e. more than 50\% of XP's pinyin is P's pinyin), SP = P. For example, the consistency score of the phonetic radical `扁' /bian/ is 0.57 (8/14), the pinyin of the new character `猵' is /bian/. 

If P's consistency < 0.5, the new character's pinyin consists of the most common consonant and vowel among all XPs. For example, the consistency of the phonetic radical `危' /wei/ is 0.36 (4/11); the most common consonant is /g/ (count = 7) and the most common vowel is /ui/ (count = 7). Therefore, the pinyin of the new character `觤' is /gui/. 

If the combination of the most common vowel and consonant is the same as the phonetic radical's pinyin, the second most common consonant or vowel will be used. For example, the consistency of the phonetic radical `亢' /kang/ is 0.43 (6/14); the most common consonant is /k/ (count = 8) and the most common vowel is /ang/ (count = 12). The combination /kang/ is the same as `亢' /kang/. Since the vowel /ang/ has a higher count, the consonant /k/ will be replaced with the second most common consonant /h/ (count = 6). The new character `阬' is named as /hang/.

If there are more than one most common consonant and/or vowel, the consonant-vowel combinations will be checked against the whole corpus and the most common combination will be the pinyin of the new character. For example, the consistency of the phonetic radical `\textit{白}' /bai/ is 0.1 (1/10); the most common consonants are /p/ and /b/ (count = 5), and the most common vowel is /o/ (count = 8). The possible pinyins are /po/ and /bo/. In the corpus, there are 25 characters' pinyin is /bo/ and 18 characters' pinyin is /po/. Therefore, the new character `袙' is named as /bo/.
\subsection{Results}
The results of the baseline model with different training datasets are listed in Table \ref{tab:2}. The baseline model correctly named 33 characters with all dataset (accuracy 0.55), which is higher than the average human speakers (27.2 characters, accuracy 0.45). The accuracies of baseline model with Mid frequency dataset and High frequency dataset are 0.45 and 0.4 respectively. Similar to humans, the baseline models are also likely to produce the regular type answers.

\begin{table}[t!]
\centering
\small
\begin{tabular}{lll|ll|ll}
\toprule
& \multicolumn{2}{c}{All} & \multicolumn{2}{c}{Mid} & \multicolumn{2}{c}{High} \\
Type & Total & \checkmark & Total & \checkmark & Total & \checkmark \\
  \midrule
Regular & 23 & 19 & 26 & 18 & 24 & 16 \\
Alliterating & 6 & 2 & 6 & 2 & 7 & 3 \\
Rhyming & 16 & 5 & 11 & 4 & 15 & 4 \\
Radical & 15 & 7 & 17 & 3 & 14 & 1 \\
\midrule
Total & 60 & 33 & 60 & 27 & 60 & 24\\
\bottomrule
\end{tabular}
\caption{Counts of each type of answers and the correct answers for the baseline model with All, Mid and High training data}
\label{tab:2}
\end{table}

\end{CJK*}