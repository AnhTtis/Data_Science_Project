\section{Related Work and Current Study}
Skilled Chinese readers make use of the phonetic radicals to name characters \cite{chen1996functional, zhou1999sublexical, ding2004nature}, and previous studies measured how phonetic radicals influence character naming in two ways: regularity and consistency \cite{fang1986consistency, hue1992recognition, hsu2009orthographic}. The regularity is exemplified in Table~\ref{tab:0}, and the consistency is defined as the number of characters that share the same phonetic radicals and pinyin. For example, there are 12 characters sharing the phonetic radical 青 $<$qing$>$ in Table~\ref{tab:0}, among which 3 characters (精, 靖, 靖) have the same pinyin $<$jing$>$, so the consistency score for these characters is 0.25 (3/12). Many studies have found \textit{regularity and consistency effects} for human speakers %\Lingyuin{I add italics for clearance. Not sure whether robust should be included here.} 
- the \regular and more consistent characters are named faster and more accurately, 
%\Lingyuin{A bit vague here, it's okay but better to be more clear.}, 
and these effects are stronger for low-frequency characters than high-frequency ones \cite{lien1985consistency, liu2003regularity, tsai2005consistency}. 

Previous studies of Chinese character modeling with phonetic radicals as inputs have successfully simulated the regularity effect and consistency effect. \citet{yang2009simulating} trained a feed-forward network on 4,468 Chinese characters and tested the model on 120 characters (seen in the training). The input to the model includes the character's radicals and radicals' positions (e.g., left-right, up-down).\footnote{There are 10 different Chinese character structures clustered by the arrangement of the character radicals, e.g., left-right (日+青=晴), top-down (相+心=想), and enclosure (口+或=國). The left-right structure is the most common type (71\%) \cite{hsiao2006analysis}.} The output of the model is the phonological features (e.g., stop, lateral) of the character's pinyin. They also measured the human speakers' response latency\footnote{Response latency measures the response speed, usually in milliseconds.} on each of the 120 test characters. By comparing the human speakers' response latency and the model's sum squared error, they found very similar regularity and consistency effects. In addition, \citet{hsiao2004connectionist,hsiao2005differences} trained a feed-forward model on 2,159 left-right structured characters, with each character appearing according to its log token frequency. The input included each character's radicals, and the output was the character's pinyin. They analyzed the training accuracy of the model and found the model's sum squared errors lower for the \regular characters, which successfully simulated the regularity effect. 

The regularity and consistency effect revealed that both human speakers and the neural models utilized the statistic distribution of phonetic radicals in naming familiar characters. However, these effects can not be applied in unknown character naming since the speakers don't know the statistics of these characters. Therefore, we proposed a new metric (saliency of the phonetic radical) to measure how the phonetic radicals influence the speaker's unknown character naming behavior. Saliency is defined as the fraction of the \regular characters among all characters sharing the same phonetic radical. For example, the phonetic radical 青 $<$qing$>$ appeared in 12 characters in Table~\ref{tab:0}, among which 4 characters (清,情,圊,晴) are \regular. Thus the saliency score of 青 $<$qing$>$ is 0.33 (4/12). The more salient a phonetic radical is, the more likely the character that contains it is pronounced the same as its pinyin. 

We hypothesized that the human speakers would show a saliency effect in unknown character naming - they would name the characters more accurately if the phonetic radical is more salient. We expected to find a similar saliency effect in the models. In addition, we also closely examine the models' answers and humans' answers to investigate if the models can represent the human speaker's behavior.