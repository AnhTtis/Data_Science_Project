\section{Preliminaries} 
\label{sec:preliminary}
This section reviews some basic notations of differential geometry and the geometry of SPD manifolds.
For a more detailed review, please refer to the supplementary.

We first briefly review the idea of pullback, which is a common trick in geometry to study metrics.
\begin{definition} [Pullback Metrics] \label{def:pullback_metrics}
    Suppose $\calM,\calN$ are smooth manifolds, $g$ is a Riemannian metric on $\calN$, and $f:\calM \rightarrow \calN$ is smooth.
    Then the pullback of the tensor field $g$ by $f$ is defined point-wisely,
    \begin{equation} \label{eq:pullback_metrics}
        (f^*g)_p(V_1,V_2) = g_{f(p)}(f_{*,p}(V_1),f_{*,p}(V_2)),
    \end{equation}
    where $p \in \calM$, $f_{*,p}(\cdot)$ is the differential map of $f$ at $p$, and $V_i \in T_p\calM$.
    If $f^*g$ is positive definite, it is a Riemannian metric on $\calM$, which is called the pullback metric defined by $f$.
\end{definition}
The most common pullback metrics are the ones induced by diffeomorphism, \ie when $f$ is a diffeomorphism.

Next, we review the basic geometry of SPD manifolds.
We denote the set of $n \times n$ SPD matrices as $\spd{n}$, the set of $n \times n$ symmetric matrices as $\sym{n}$, and all the Cholesky matrices (lower triangular matrices with positive diagonal elements) as $\cho{n}$.
As shown in the previous literature \cite{arsigny2005fast,lin2019riemannian}, $\spd{n}$ and $\cho{n}$ form an SPD manifold and a Cholesky manifold, respectively.
For an SPD matrix $S$, the matrix logarithm $\mln(\cdot): \spd{n} \rightarrow \sym{n}$ is defined as
\begin{equation} \label{eq:mln} 
    \mln(S) = U \ln(\Sigma) U^\top,    
\end{equation}
where $S=U \Sigma U^\top$ is the eigendecomposition, and $\ln(\cdot)$ is the diagonal natural logarithm.

In \cite{arsigny2005fast}, LEM on $\spd{n}$ is introduced by Lie group translation. 
The standard LEM is further generalized into two-parameter families of $\orth{n}$-invariant metrics \cite{thanwerdas2023n}, namely $\biparamLEM$, by $\orth{n}$-invariant inner product on $\sym{n}$ 
\begin{equation}
    \langle X,X \rangle^{\biparam} = a \|X\|_\rmF + b \tr(X)^2, \forall X \in \sym{n},
\end{equation}
where $\|\cdot\|_\rmF$ is the Frobenius inner product, and $\biparam \in \bfst= \{\biparam \in \mathbb{R}^2 \mid \min (a, a+n b)>0\}$.
In \cite{lin2019riemannian}, LCM is derived on $\spd{n}$ from the Cholesky manifold $\cho{n}$ by Cholesky decomposition.
We denote $\biparamLEM$ and LCM as $\gbiparamlem$ and $\glcm$, respectively.
For an SPD matrix $P$ and a tangent vector $V$ in the tangent space $T_P\spd{n}$ at $P$, $\gbiparamlem$ is defined as
\begin{equation}
    \label{eq:biparamLEM}
    \gbiparamlem_{P} (V,V)=a \|\diffmln{P} (V)\|^2_{\rmF}+ b \tr(P^{-1}V)^2,
\end{equation}
where $\diffmln{P}$ is the differential map of matrix logarithm at $P \in \spd{n}$, $V$ is a tangent vector in the tangent space $T_P\spd{n}$ at $P$, $\biparam \in \bfst$.
Note that $\biparamLEM$ incorporates the standard LEM when $\biparam=(1,0)$.

For $L \in \cho{n}$ and $W \in T_L\cho{n}$, the metric on the Cholesky manifold \cite{lin2019riemannian} is defined as
\begin{equation} 
    \label{eq:cm}
    \gcm_L(W,W)=\sum_{i>j} W_{i j} W_{i j}+\sum_{j=1}^n W_{j j} W_{j j} L_{j j}^{-2},
\end{equation}
The LCM is the pullback metric by the Cholesky decomposition $\scrL$ from $\gcm$ \cite{lin2019riemannian}:
\begin{equation}
    \label{eq:lcm}
    \glcm = \scrL^{*}\gcm.
\end{equation}


