%TIP 
\usepackage{amsmath,amsfonts}
\usepackage{algorithmic}
\usepackage{algorithm}
\usepackage{array}
\usepackage[caption=false,font=normalsize,labelfont=sf,textfont=sf]{subfig}
\usepackage{textcomp}
\usepackage{stfloats}
\usepackage{url}
\usepackage{verbatim}
\usepackage{graphicx}
\usepackage{cite}
\graphicspath{{img/}}
% \usepackage{appendix}
\usepackage{multirow} 


\usepackage[utf8]{inputenc} % allow utf-8 input
\usepackage[T1]{fontenc}    % use 8-bit T1 fonts
\usepackage{hyperref}       % hyperlinks
% \usepackage{url}            % simple URL typesetting
\usepackage{booktabs}       % professional-quality tables
% \usepackage{amsfonts}       % blackboard math symbols
\usepackage{nicefrac}       % compact symbols for 1/2, etc.
\usepackage{microtype}      % microtypography
\usepackage{xcolor}         % colors
\usepackage{colortbl}
% \usepackage[numbers, compress]{natbib}

% \usepackage{graphicx}
% \usepackage{subfigure}
% \usepackage{amsmath}
\usepackage{amssymb}
\usepackage{mathtools}
\usepackage{amsthm}
\usepackage{mathrsfs}
\usepackage{bbm}
\usepackage{tasks}
% \usepackage{enumitem}
% \usepackage{breakurl}

\usepackage{wrapfig}
\usepackage[capitalize,noabbrev]{cleveref}
% \usepackage{float}
\usepackage{enumerate}
% \usepackage{subfigure}
\usepackage{parskip}
% \PassOptionsToPackage{numbers, compress}{natbib}




\theoremstyle{plain}
\newtheorem{theorem}{Theorem}[section]
\newtheorem{proposition}[theorem]{Proposition}
\newtheorem{lemma}[theorem]{Lemma}
\newtheorem{corollary}[theorem]{Corollary}
\theoremstyle{definition}
\newtheorem{definition}[theorem]{Definition}
\newtheorem{assumption}[theorem]{Assumption}
\theoremstyle{remark}
\newtheorem{remark}[theorem]{Remark}

% set
\providecommand{\calM}{\mathcal{M}}
\providecommand{\calN}{\mathcal{N}}
\providecommand{\calA}{\mathcal{A}}
\providecommand{\calX}{\mathcal{X}}
\providecommand{\calY}{\mathcal{Y}}
\providecommand{\bbN}{\mathbb{N}}
\providecommand{\bbZ}{\mathbb{Z}}
\providecommand{\bbK}{\mathbb{K}}
\providecommand{\sym}[1]{\mathcal{S}^{#1}}
\providecommand{\spd}[1]{\mathcal{S}^{#1}_{++}}
\providecommand{\cho}[1]{\mathcal{L}_{+}^{#1}}
\providecommand{\tril}[1]{\mathcal{L}^{#1}}
\providecommand{\bbR}[1]{\mathbb {R}^{#1}}
\providecommand{\bbRplus}{\mathbb {R}_{+}}
\providecommand{\bbRscalar}{\mathbb {R}}
\providecommand{\cinf}{C^{\infty}}
\providecommand{\orth}[1]{\mathrm{O}({#1})}

% operator
\providecommand{\rieexp}{\operatorname{Exp}}
\providecommand{\rielog}{\operatorname{Log}}
\providecommand{\diffphi}[1]{\phi_{*,#1}}
\providecommand{\diffphiinv}[1]{\phiinv_{*,#1}}
\providecommand{\diffmlog}[1]{\operatorname{mlog}_{*,#1}}
\providecommand{\diffmln}[1]{\operatorname{mln}_{*,#1}}
\providecommand{\diffmgexp}[1]{\phi_{ma *,#1}}
\providecommand{\pt}[2]{\Gamma_{#1 \rightarrow #2}}
\providecommand{\scrL}{\mathscr{L}}
\providecommand{\bbD}{\mathbb {D}}
\providecommand{\ln}{\operatorname{ln}}
\providecommand{\mln}{\operatorname{mln}}
\providecommand{\cln}{\phi_{cln}}
\providecommand{\clnchart}{\varphi_{ln}}
\providecommand{\clogchart}{\varphi_{log}}
\providecommand{\mlog}{\operatorname{mlog}}
\providecommand{\clog}{\phi_{clog}}
\providecommand{\mexp}{\phi_{\mathrm{mexp}}}
\providecommand{\mgexp}{\phi_{\mathrm{ma}}}
\providecommand{\cgexp}{\phi_{ca}}
\providecommand{\mlnMul}{\odot_{mln}}
\providecommand{\clnMul}{\odot_{cln}}
\providecommand{\mlogMul}{\odot_{mlog}}
\providecommand{\clogMul}{\odot_{clog}}
\providecommand{\mlnMulScalar}{\circledast_{mln}}
\providecommand{\clnMulScalar}{\circledast_{cln}}
\providecommand{\mlogMulScalar}{\circledast_{mlog}}
\providecommand{\clogMulScalar}{\circledast_{clog}}
\providecommand{\mlogMulInv}[1]{{#1}_{\mlogMul}^{-1}}
\providecommand{\xMulInv}[1]{{#1}_{\xMul}^{-1}}
\providecommand{\yMulInv}[1]{{#1}_{\yMul}^{-1}}
\providecommand{\distmlog}{d_{mlog}}
\providecommand{\distclog}{d_{clog}}
\providecommand{\relu}{\operatorname{ReLu}}
\providecommand{\diag}{\operatorname{diag}}
\providecommand{\rieExp}[1]{\operatorname{Exp}_{#1}}
\providecommand{\rieLog}[1]{\operatorname{Log}_{#1}}
\providecommand{\leop}{\odot_{le}}
\providecommand{\lescalar}{\circledast_{le}}
\providecommand{\lcop}{\odot_{lc}}
\providecommand{\im}[1]{\operatorname{im}{#1}}
\providecommand{\xMul}{\odot_{\phi}}
\providecommand{\yMul}{\odot_{\calY}}
\providecommand{\xMulScalar}{\circledast_{\phi}}
\providecommand{\yMulScalar}{\circledast_{\calY}}
\providecommand{\xMul}{\odot_{\phi}}
\providecommand{\yMul}{\odot_{\calY}}
\providecommand{\xMulScalar}{\circledast_{\phi}}
\providecommand{\yMulScalar}{\circledast_{\calY}}
\providecommand{\phiinv}{\phi^{-1}}
\providecommand{\phiMulScalar}{\circledast_{\phi}}
\providecommand{\gphi}{g^{\phi}}
\providecommand{\phiMul}{\odot_{\phi}}
\providecommand{\diff}{\operatorname{d}}
\providecommand{\tr}{\operatorname{tr}}
\providecommand{\galem}{g^{\mathrm{ALE}}}
\providecommand{\glem}{g^{\mathrm{LE}}}
\providecommand{\glcm}{g^{\mathrm{LC}}}
\providecommand{\gcm}{g^{\mathrm{C}}}
\providecommand{\geuc}{g^{\mathrm{E}}}
\providecommand{\gx}{g^{\mathcal{X}}}
\providecommand{\gy}{g^{\mathcal{Y}}}
\providecommand{\gphi}{g^{\phi}}
\providecommand{\dlem}{d^{\mathrm{LE}}}
\providecommand{\dlcm}{d^{\mathrm{LC}}}
\providecommand{\deuc}{d^{\mathrm{E}}}
\providecommand{\dalem}{d^{\mathrm{ALE}}}
\providecommand{\dphi}{d^{\phi}}
\providecommand{\Cov}{{\mathrm{Cov}}}
\providecommand{\fm}{{\mathrm{FM}}}
\providecommand{\det}{{\operatorname{det}}}
\providecommand{\gbiparamlem}{g^{(a,b)\text{-LE}}}
\providecommand{\gbiparamalem}{g^{(a,b)\text{-ALE}}}
\providecommand{\gbiparamaE}{g^{(a,b)\text{-E}}}
\providecommand{\glcm}{g^{\text{LC}}}
\providecommand{\dlog}{\operatorname{Dlog}}

%symble
\providecommand{\balpha}{\boldsymbol{\alpha}}
\providecommand{\rmE}{\mathrm{E}}
\providecommand{\rmF}{\mathrm{F}}
\providecommand{\rmO}{\mathrm{O}}
\providecommand{\id}{\mathbbm{1}}
\providecommand{\idY}{\id_{\calY}}
\providecommand{\idX}{\id_{\calX}}
\providecommand{\atlasid}{\mathcal{A}_{\mathbbm{1}}}
\providecommand{\lem}{\mathrm{LEM}}
\providecommand{\lcm}{\mathrm{LCM}}
\providecommand{\biparamLEM}{(a,b)\text{-LEM}}
\providecommand{\biparamALEM}{(a,b)\text{-ALEM}}
\providecommand{\biparam}{(a,b)}
\providecommand{\bfst}{\mathbf{ST}}

%Other command
\newcommand{\yue}[1]{\textcolor{green}{#1}}
\newcommand{\ziheng}[1]{\textcolor{blue}{[ZH: #1]}}
\providecommand{\ie}{\textit{i.e., }}
\providecommand{\vsacle}{\vspace{0pt}}
\providecommand{\etal}{\textit{et al.}}
\newcommand{\na}{\textcolor{gray}{N/A}}%


% RResNet
\providecommand{\spec}{\operatorname{spec}}


% Review
% \providecommand{\revise}[1]{\textcolor{red}{#1}}
% \providecommand{\parcolor}{\color{red}}

%gyro
\providecommand{\gyr}{\operatorname{gyr}}
\providecommand{\gyrinner}[2]{\left\langle #1, #2 \right\rangle_{\mathrm{gr}}}
\providecommand{\gyrnorm}[1]{\left\| #1 \right\|_{\mathrm{gr}}}
\providecommand{\gyrdist}{{\mathrm{d}}_{\mathrm{gry}}}


% Customize the format for all reference types
\crefname{equation}{Eq.}{Eqs.}
\Crefname{equation}{Equation}{Equations}
\crefname{figure}{Fig.}{Figs.}
\Crefname{figure}{Figure}{Figures}
\crefname{table}{Tab.}{Tabs.}
\Crefname{table}{Table}{Tables}
\crefname{algocf}{Alg.}{Algs.}
\Crefname{algocf}{Algorithm}{Algorithms}
\crefname{section}{Sec.}{Secs.}
\Crefname{section}{Section}{Sections}
\crefname{appendix}{App.}{Apps.}
\Crefname{appendix}{Appendix}{Appendices}
\crefname{suplement}{Supp.}{Supps.}
\Crefname{suplement}{Suplement}{Suplements}
% \crefname{secinapp}{appendix}{appendices}
% \Crefname{secinapp}{Appendix}{Appendices}

% Custom names for theorem environments
\crefname{theorem}{Thm.}{Thms.}
\Crefname{theorem}{Theorem}{Theorems}
\crefname{lemma}{Lem.}{Lems.}
\Crefname{lemma}{Lemma}{Lemmas}
\crefname{definition}{Def.}{Defs.}
\Crefname{definition}{Definition}{Definitions}
\crefname{corollary}{Cor.}{Cors.}
\Crefname{corollary}{Corollary}{Corollaries}
\crefname{remark}{Rem.}{Rems.}
\Crefname{remark}{Remark}{Remarks}
\crefname{proposition}{Prop.}{Props.}
\Crefname{proposition}{Proposition}{Propositions}
\crefname{proof}{Pr.}{Prs.}
\Crefname{proof}{Proof}{Proofs}


% Custom names for enumerate environments
\crefname{enumi}{Case}{Cases}
\Crefname{enumi}{Case}{Cases}