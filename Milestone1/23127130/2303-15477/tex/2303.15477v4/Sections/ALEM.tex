\section{Adaptive Log-Euclidean Metrics}
\label{sec:ada_riem_metrics}
As mentioned in \cref{sec:intro}, pullbacks are ubiquitous for studying Riemannian metrics on SPD manifolds.
In this section, we further show that both $\biparamLEM$ and LCM are pullback metrics from the Euclidean space.
Inspired by this observation, we present a general framework for characterizing PEMs.
Then, we focus on generalizing LEM.

\subsection{Rethinking \texorpdfstring{$\biparamLEM$}{LEM} and LCM} \label{subsec:thk_lem_lcm}
Among the existing Riemannian metrics on the SPD manifold, LEM is popular in many applications, given its closed form for the Fréchet mean and clear vector space \& Lie group structures.
In addition, the nascent LCM, gaining increasing attention, also shares similar properties with LEM.
LEM is derived from the Lie group translation \cite{arsigny2005fast}, while LCM is derived by the pullback from $\cho{n}$ \cite{lin2019riemannian}.
Besides, $\biparamLEM$ is obtained by the pullback of LEM.
However, theoretically, the mathematical logic beneath their derivation can be the same. 
We denote $\tril{n}$ as the Euclidean space of $n \times n$ lower triangular matrices.
We define $\cln: \spd{n} \rightarrow \tril{n}$ as
\begin{equation} \label{eq:lcm_pullback_map}
    \cln(P)=\lfloor L \rfloor + \ln(\bbD(L)),
\end{equation}
where $L$ is the Cholesky factor of the SPD matrix $P$, $\lfloor L \rfloor$ is the strictly lower part of $L$, and $\bbD(L)$ is a diagonal matrix with diagonal elements of $L$. 
Then, we have the following theorem.
\begin{theorem} \label{thm:rethk_lem_lcm}
    $\biparamLEM$ is the pullback metric from the Euclidean space of $\sym{n}$ with an $\orth{n}$-invariant inner product $\langle , \rangle^{\biparam}$ by matrix logarithm.
    Specifically, the standard LEM is the pullback metric from the Euclidean space of $\sym{n}$ with the standard Frobenius inner product by matrix logarithm.
    LCM is the pullback metric from $\tril{n}$ with the Frobenius inner product by $\cln$.
\end{theorem}

As $n$-dimensional Euclidean spaces are naturally isometric, it can be directly obtained that both $\biparamLEM$ and LCM are pulled back from the standard Euclidean space $\sym{n}$.

\begin{corollary} \label{cor:biparamLEM_pem}
    $\biparamLEM$ and LCM are pullback metrics from $\sym{n}$ with standard Frobenius inner product.
\end{corollary}


\subsection{PEMs on SPD Manifolds}
\label{subsec:pem_spd}

In \cref{subsec:thk_lem_lcm}, we have shown how LEM is derived from matrix logarithm.
Besides, as shown in \cite{arsigny2005fast}, operations in Lie group and linear space on $\spd{n}$ are also induced from matrix logarithm.
Now, let us explain the underlying mechanism in detail.
A matrix logarithm is a diffeomorphism (a smooth bijection with a smooth inverse).
The property of bijection offers the possibility of transferring algebraic structures from $\sym{n}$ into $\spd{n}$.
The smoothness of matrix logarithm and its inverse suggest that smooth structures can be transferred into $\spd{n}$, like the Lie group and Riemannian metric.
More generally, given an arbitrary diffeomorphism $\phi:\spd{n} \rightarrow \sym{n}$, it suffices to pull various properties from the Euclidean space back to the SPD manifold $\spd{n}$ by $\phi$ as well.
Besides, the computation of the induced operators in $\spd{n}$ by $\phi$ is usually simple.

% \begin{figure}
% \centering
% \includegraphics[width=0.9\columnwidth]{illustration.pdf} 
% \caption{Conceptual illustration of \cref{lem:pro_by_bijection}.
% $\phi$ is a diffeomorphism from the smooth manifold $\calX$ to the Riemannian manifold $\{\calY,\gy\}$ with the Lie group multiplication $\yMul$.
% With element multiplication $\xMul$ defined in the co-domain of $\phi$, $\{\calX,\xMul\}$ forms a Lie group, 
% Endowed with the pullback metric $\gphi=\phi^*\gy$, $\{\calX,\gphi\}$ forms a Riemannian manifold.
% }
% \label{fig:pipeline}
% \end{figure}

\begin{lemma} \label{lem:g_spd}
    Let $S_1, S_2, S \in \spd{n}, V_i \in T_S\spd{n}, k \in \bbRscalar$ and $g^{\rmE}$ be the Frobenius inner product in $\sym{n}$.
    $\phi:\spd{n} \rightarrow \sym{n}$ is a diffeomorphism, and $\phi_{*,S}$ is the differential at $S$.
    We define the following operations,
    \begin{align}
        \label{eq:phi_mul} 
        \text{Elements Addition: }& S_1 \phiMul S_2 = \phiinv( \phi(S_1)+\phi(S_2)),\\
        \label{eq:phi_sca_mul}
        \text{Scalar Product: }& k \phiMulScalar S_2 = \phiinv( k\phi(S_2)),\\
        \label{eq:phi_innerpro}
        \text{Inner Product: }& \langle S_1, S_2 \rangle_{\phi} = \langle \phi(S_1), \phi(S_2) \rangle,\\
        \label{eq:phi_g} 
        \text{Riemannian Metric: }& \gphi = \phi^*\geuc,
    \end{align}
    Then, we have the following conclusions:
    \begin{enumerate}
        \item \label{itm:spd_hilbet}
        $\{\spd{n}, \phiMul,\phiMulScalar, \langle \cdot, \cdot \rangle_{\phi} \}$ is a Hilbert space over $\bbRscalar$.
        \item 
        $\{\spd{n}, \phiMul \}$ is an Abelian Lie group.
        $\{\spd{n}, \gphi \}$ is a Riemannian manifold.
        The associated Riemannian operators are as follows
        \begin{align}
            \label{eq:dist_phi_spd}
            \dphi (S_1, S_2 ) &= \| \phi(S_1) - \phi(S_2) \|_\rmF,\\
            \label{eq:gene_rie_exp_spd}
            \rieexp_{S_1} V &= \phiinv(\phi(S_1)+\diffphi{S_1}V),\\
            \label{eq:gene_rie_log_spd}
            \rielog_{S_1}S_2 &= \diffphiinv{\phi(S_1)}(\phi(S_2)-\phi(S_1)),\\
            \label{eq:gene_pt_spd} \pt{S_1}{S_2}(V) &= \diffphiinv{\phi(S_2)} \circ \diffphi{S_1}(V),
        \end{align}
        where $\|\cdot\|_\rmF$ is the Frobenius norm, $V \in T_{S_1}\spd{n}$ is a tangent vector, $\rieexp_{S_1}$, $\rielog_{S_1}$ and $\pt{S_1}{S_2}$ are Riemannian exponential map at $S_1$, logarithmic map at $S_1$ and parallel transportation along the geodesics connecting $S_1$ and $S_2$ respectively, and $\phiinv_{*}$ is the differential maps of $\phiinv$.
        Then $\gphi$ is a bi-invariant metric, named Pullback Euclidean Metric (PEM) by $\phi$.
        \item
        $\phi$ is an isomorphism: (a) a linear isomorphism preserving the inner product; (b) a Lie group isomorphism; (3) a Riemannian isometry.
    \end{enumerate}
\end{lemma}
% \begin{remark}
% For a better understanding of the above theorem, we make the following remarks:
% % \begin{enumerate}
% %     \item 
% %     For the Hilbert space in case \ref{itm:spd_hilbet}, the distance induced by the inner product $\langle \cdot, \cdot \rangle_{\phi}$ respects the geometry of SPD manifolds, as it is exactly the geodesic distance induced from $\gphi$.
% %     \item 
% %     For \cref{eq:gene_pt_spd}, as 
% %     $(\phiinv)_* = (\phi_{*})^{-1}$ \cite{loring2011introduction}, we simply write $\phiinv_*$.
% %     Besides, although $\phiinv_{*} \phi_{*}= \id$ at any point, for $S_i \in \spd{n}$, $\diffphiinv{\phi(S_2)} \diffphi{S_1}$ might not be the identity map.
% %     Specifically, $\diffphiinv{\phi(S)}\diffphi{S}=\id_{T_S\spd{n}}$ does not imply $\diffphiinv{\phi(S_2)}\diffphi{S_1}=\id_{T_{S_1}\spd{n}}$.
% %     In addition, $\diffphiinv{S_2}\diffphi{S_1}$ could vary for different pairs of $S_1, S_2$.
% %     In other words, the formulae of parallel transportation could be different among different pairs of $S_1, S_2$.
% % % \end{enumerate}
% \end{remark}
In fact, $\biparamLEM$ and LCM are special cases of \cref{lem:g_spd}, and so do linear space \& Lie group in \cite{arsigny2005fast} and Lie group in \cite{lin2019riemannian}.
In addition, neither \cite{arsigny2005fast} nor \cite{lin2019riemannian} reveals the Hilbert space structures in $\spd{n}$.
\subsection{Adaptive Log-Euclidean Metrics} \label{subsec:ada_rie_metric}
The key of \cref{lem:g_spd} lies in the diffeomorphism $\phi$.
If we have a proper $\phi$, Riemannian metrics on SPD manifolds can be induced.
In the following, we will present our mappings and then discuss the induced metrics.

As an eigenvalues function, the matrix logarithm in \cref{eq:mln} is reduced into a scalar logarithm, which is a diffeomorphism between $\bbRplus$ and $\bbRscalar$.
Following this hint, the eigenvalues-based diffeomorphism between $\spd{n}$ and $\sym{n}$ is reduced to scalar diffeomorphism between $\bbRplus$ and $\bbRscalar$. 
A very natural idea is to substitute the natural logarithm with scalar logarithms with arbitrary proper bases.
In particular, we can define a general diagonal logarithm $\log(\cdot)$ as
\begin{equation} \label{eq:diag_glog}
    \log_\alpha(X) = \diag(\log_{a_1}^{x_{11}},\log_{a_2}^{x_{22}},\cdots,\log_{a_n}^{x_{nn}}),
\end{equation}
where $\alpha = (a_1, a_2, \cdots, a_n) \in \bbRplus^{n} \setminus \{(1,1,\cdots,1)\}$ is the base vector, $\diag(\cdot)$ is the diagonalization operator, and $X$ is an $n \times n$ diagonal matrix.
By abuse of notation, we denote $\log_\alpha(\cdot)$ as $\log(\cdot)$ for a general diagonal logarithm, and $\log_a^{(\cdot)}$ as $\log^{(\cdot)}$ for a general scalar logarithm.
Specially, $a_1 = \cdots = a_n=e \Rightarrow \log(\cdot) = \ln(\cdot)$.
Together with eigendecomposition, a general matrix logarithm is:
\begin{equation} \label{eq:mlog}
     \mlog(S) = U \log_\alpha(\Sigma) U^\top,    
\end{equation}
where $S = U \Sigma U^\top$ is the eigendecomposition.
As a special case, when $\alpha=(e,e,\cdots,e)$, $\mlog = \mln$.
Similar to the scalar logarithm, we have the following proposition.
\begin{proposition}[Diffeomorphism] \label{props:diffeo_mlog}
    $\mlog$ is a diffeomorphism, a smooth bijection with a smooth inverse $\mlog^{-1}(\cdot):\sym{n} \rightarrow \spd{n}$ defined as
    \begin{equation}\label{eq:mgexp}
        \mlog^{-1}(X) = \mgexp(X) = U \balpha(\Sigma) U^\top,\\
    \end{equation}
    where $\balpha(\Sigma) = \diag(a_1^{\Sigma_{11}},a_2^{\Sigma_{22}},\cdots,a_n^{\Sigma_{nn}})$ is a diagonal exponentiation.
\end{proposition}
\begin{remark} \label{rmk:proposed_charts}
    Note that $\mlog(\cdot)$ should be more precisely understood as an arbitrary one from the following family
    \begin{equation}
        \{  \mlog^\alpha| \alpha = (a_1, \cdots, a_n) \in \bbRplus^{n} \setminus \{(1,\cdots,1)\}  \}.
    \end{equation}
    By abuse of notation, we will simply use $\mlog(\cdot)$.
    Besides, there could be some ambiguity in \cref{eq:mlog} under different arrangements of eigenvalues and eigenvectors.
    In fact, there is a correspondence between scalar $\log_{a_i}$ and eigenvalues \& eigenvectors.
    Please refer to Supp. B-A for more details.
    % Please refer to \cref{app:subsec:well_difined_glog} for more details.
\end{remark}
Since $\mlog$ is a diffeomorphism from $\spd{n}$ onto $\sym{n}$, all the results in \cref{lem:g_spd} hold true.
\begin{theorem} \label{thm:mlog_spd_properties}
    Following the notations in \cref{lem:g_spd}, we define $\mlogMul, \mlogMulScalar, \langle \cdot, \cdot \rangle_{mlog}$, and $g^{mlog}$ as \cref{eq:phi_mul}-\cref{eq:phi_g}.
    Then, we have the following conclusions:
    \begin{enumerate}
        \item \label{enum:hilbert}
        $\{\spd{n}, \mlogMul,\mlogMulScalar, \langle \cdot, \cdot \rangle_{mlog} \}$ is a Hilbert space over $\bbRscalar$.
        \item \label{enum:mlog_riem_spd}
        $\{\spd{n}, \mlogMul \}$ is an Abelian Lie group.
        $g^{mlog}$ is a Riemannian metric over $\spd{n}$.
        We call this metric Adaptive Log-Euclidean Metric (ALEM) and denote $g^{mlog}$ as $\galem$.
        The associated Riemannian operators are as follows
        \begin{align} 
            \label{eq:dist_mlog}
            &\dalem (S_1, S_2 ) = \| \mlog(S_1) - \mlog(S_2) \|_\rmF,\\
            \label{eq:rieexp_gmlog} 
            &\rieexp_{S_1} V = \mgexp(\mlog(S_1)+\diffmlog{S_1}V),\\
            \label{eq:rielog_gmlog} 
            &\rielog_{S_1}S_2 = \diffmgexp{X_1}(\mlog(S_2)-\mlog(S_1)),\\
            \label{eq:pt_mlog} 
            &\pt{S_1}{S_2}(V) = \diffmgexp{X_2} \circ \diffmlog{S_1}(V),
        \end{align}
        where $X_i = \mlog(S_i) \in \sym{n}$ for $i=1,2$.
        \item \label{enum:isomorphism}
        $\mlog$ is an isomorphism: (a) a linear isomorphism preserving the inner product; (b) a Lie group isomorphism; (3) a Riemannian isometry.
    \end{enumerate}
\end{theorem}
\begin{remark}
    Obviously, ALEM would vary with different $\mlog$.
    We thus use the plural to describe our metrics.
    Besides, our metrics could be learnable.
    This is why we call them adaptive metrics.
\end{remark}

Similar with $\biparamLEM$, we also can define $\biparamALEM$ as the pullback metric of $\orth{n}$-invariant inner product:
\begin{equation}
    \gbiparamalem= \mlog^*\gbiparamaE,
\end{equation}
where we denote the $\orth{n}$-invariant inner product $\langle , \rangle^{\biparam}$ as $\gbiparamaE$.
$\gbiparamalem$ also share the properties presented in \cref{thm:mlog_spd_properties}.
Nevertheless, this paper focuses on $\biparam=(1,0)$.

\subsection{Differentials of General Logarithms}
\label{app:subsec:differentials}
\cref{eq:rieexp_gmlog}-\cref{eq:pt_mlog} require the differential maps of $\mlog$ and $\mgexp$.
This subsection introduces the concrete formulae of the associated differential maps.
\begin{proposition}[Differentials] \label{props:diff_mgexp_mlog}
    For a tangent vector $V \in T_S\spd{n}$, the differential $\diffmlog{S} : T_S \spd{n} \rightarrow T_{\mlog(S)} \sym{n}$ of $\mlog$ at $S \in \spd{n}$ is given by
    \begin{equation}
        \diffmlog{S} (V) = Q+Q^\top + W,
    \end{equation}
    where $Q = D_U\log(\Sigma)U^\top$,
    \begin{align*}
        D_U &= (\begin{array}{ccc}
             (\sigma_1 I-S)^+ V u_1 & \cdots & (\sigma_n I-S)^+ V u_n
        \end{array}),\\
        W &= U \diag(\frac{u_1^\top V u_1}{\sigma_1 \ln{a_1}},\cdots,\frac{u_n^\top V u_n}{\sigma_n \ln{a_n}}) U^\top,        
    \end{align*}
    $()^+$ is the Moore–Penrose inverse, $u_1,\cdots,u_n$ are orthonormal eigenvectors of $S$, and the associated eigenvalues are $\sigma_1,\cdots,\sigma_n$.
    
    Symmetrically, for a tangent vector $\widetilde{V} \in T_X\sym{n}$, the differential $\diffmgexp{X} : T_X \sym{n} \rightarrow T_{\mgexp(X)} \spd{n}$ of $\mgexp$ at $X \in \sym{n}$ is given by
    \begin{equation} \label{eq:diff_mgexp}
        \diffmgexp{X} (\widetilde{V}) = \widetilde{Q}+\widetilde{Q}^\top + \widetilde{W},
    \end{equation}
    where $S = \widetilde{U} \widetilde{\Sigma}\widetilde{U}^\top$ is the eigendecomposition, $D_{\widetilde{U}}$ is defined similarly, $\widetilde{Q} = D_{\widetilde{U}}\balpha(\widetilde{\Sigma})\widetilde{U}^\top$, and
    \begin{equation*}
        \widetilde{W} = \widetilde{U} \diag(\ln^{a_1}a_1^{\widetilde{\sigma_1}}{\widetilde{u}_1^\top \widetilde{V} \widetilde{u}_1},\cdots,\ln^{a_n}a_n^{\widetilde{\sigma_n}}{\widetilde{u}_n^\top \widetilde{V} \widetilde{u}_n}) \widetilde{U}^\top.     
    \end{equation*}
\end{proposition}

In \cite{arsigny2005fast}, the differential of the matrix exponential is written as an infinite series.
The differential of our $\mgexp$ can also be rewritten in this way.
\begin{proposition}[Differential as Infinite Series] \label{props:diff_mgexp_series}
    Following the notation in \cref{props:diff_mgexp_mlog}, the differential of $\mgexp$ can also be formulated as
    \begin{equation} \label{eq:diff_mgexp_series}
        \begin{aligned}
            &\diffmgexp{X}(\widetilde{V}) \\
            &= \sum_{k=1}^{\infty} \frac{1}{k !}(\sum_{l=0}^{k-1} (\widetilde{P}X)^{k-l-1} (D_{\widetilde{P}}X+\widetilde{P}\widetilde{V}) (\widetilde{P}X)^l),
        \end{aligned}
    \end{equation}
    where $\widetilde{P}=\widetilde{U}B\widetilde{U}^\top$, $B=\diag(\ln^{a_1},\cdots,\ln^{a_n})$, $D_{\widetilde{P}}= D_{\widetilde{U}} B \widetilde{U}^\top + \widetilde{U} B D_{\widetilde{U}}^\top$.
\end{proposition}

When $\mgexp(\cdot)$ is reduced into matrix exponential, \cref{eq:diff_mgexp_series} becomes Eq. 8 in \cite{arsigny2005fast}, and our ALEM becomes exactly LEM.