% CVPR 2023 Paper Template
% based on the CVPR template provided by Ming-Ming Cheng (https://github.com/MCG-NKU/CVPR_Template)
% modified and extended by Stefan Roth (stefan.roth@NOSPAMtu-darmstadt.de)

\documentclass[10pt,twocolumn,letterpaper]{article}

%%%%%%%%% PAPER TYPE  - PLEASE UPDATE FOR FINAL VERSION
% \usepackage[review]{cvpr}      % To produce the REVIEW version
\usepackage{cvpr}              % To produce the CAMERA-READY version
%\usepackage[pagenumbers]{cvpr} % To force page numbers, e.g. for an arXiv version

% Include other packages here, before hyperref.
\usepackage[accsupp]{axessibility}
\usepackage{graphicx}
\usepackage{amsmath}
\usepackage{amssymb}
\usepackage{booktabs}
\usepackage{enumitem}
\usepackage{multirow}
\usepackage{xcolor}
\usepackage{bbding}
\newtheorem{definition}{Definition}
\newtheorem{proposition}{Proposition}
\newcommand{\tablestyle}[2]{\setlength{\tabcolsep}{#1}\renewcommand{\arraystretch}{#2}\centering\small}
\usepackage{makecell}


\usepackage{color} 
\newcommand\zsw[1]{{\color{orange}#1}}
\usepackage{multicol}

% It is strongly recommended to use hyperref, especially for the review version.
% hyperref with option pagebackref eases the reviewers' job.
% Please disable hyperref *only* if you encounter grave issues, e.g. with the
% file validation for the camera-ready version.
%
% If you comment hyperref and then uncomment it, you should delete
% ReviewTempalte.aux before re-running LaTeX.
% (Or just hit 'q' on the first LaTeX run, let it finish, and you
%  should be clear).
\usepackage[pagebackref,breaklinks,colorlinks]{hyperref}


% Support for easy cross-referencing
\usepackage[capitalize]{cleveref}
\crefname{section}{Sec.}{Secs.}
\Crefname{section}{Section}{Sections}
\Crefname{table}{Table}{Tables}
\crefname{table}{Tab.}{Tabs.}

\definecolor{themeblue}{RGB}{57, 162, 219}

%%%%%%%%% PAPER ID  - PLEASE UPDATE
\def\cvprPaperID{894} % *** Enter the CVPR Paper ID here
\def\confName{CVPR}
\def\confYear{2023}


\begin{document}

%%%%%%%%% TITLE - PLEASE UPDATE
\title{Enlarging Instance-specific and Class-specific Information for \\ Open-set Action Recognition}

\author{Jun Cen$^{1,2}$\thanks{Work done as an intern at Alibaba DAMO Academy.} \quad Shiwei Zhang$^{2}$ \quad Xiang Wang$^{3}$ \quad Yixuan Pei$^{4}$ \\ Zhiwu Qing$^{3}$ \quad
Yingya Zhang$^{2}$ \quad Qifeng Chen$^{1}$\\
\textsuperscript{1}The Hong Kong University of Science and Technology \quad
\textsuperscript{2}Alibaba Group \\
\textsuperscript{3}Huazhong University of Science and Technology \quad
\textsuperscript{4}Xi'an Jiaotong University\\
\tt\small{jcenaa@connect.ust.hk,}
\tt\small{\{zhangjin.zsw,yingya.zyy\}@alibaba-inc.com,}\\
\tt\small{\{wxiang,qzw\}@hust.edu.cn,}
\tt\small{peiyixuan@stu.xjtu.edu.cn,}
\tt\small{cqf@ust.hk}}


% \author{\author{Jun Cen \quad Peng Yun \quad Junhao Cai \quad Michael Yu Wang \quad Ming Liu\\
% The Hong Kong University of Science and Technology\\
% {\tt\small \{jcenaa, pyun, jcaiaq\}@connect.ust.hk, \{mywang, eelium\}@ust.hk}
% % For a paper whose authors are all at the same institution,
% % omit the following lines up until the closing ``}''.
% % Additional authors and addresses can be added with ``\and'',
% % just like the second author.
% % To save space, use either the email address or home page, not both
% \and
% Second Author\\
% Institution2\\
% {\tt\small secondauthor@i2.org}
% }

\maketitle

\def\eg{\emph{e.g}\onedot} \def\Eg{\emph{E.g}\onedot}
\def\ie{\emph{i.e}\onedot} \def\Ie{\emph{I.e}\onedot}
\def\cf{\emph{cf}\onedot} \def\Cf{\emph{Cf}\onedot}
\def\etc{\emph{etc}\onedot} \def\vs{\emph{vs}\onedot}
\def\wrt{w.r.t\onedot} \def\dof{d.o.f\onedot}
\def\etal{\emph{et al}\onedot}
\makeatother


% vector macros
\newcommand{\va}{\mathbf{a}}
\newcommand{\vb}{\mathbf{b}}
\newcommand{\vc}{\mathbf{c}}
\newcommand{\vd}{\mathbf{d}}
\newcommand{\ve}{\mathbf{e}}
\newcommand{\vf}{\mathbf{f}}
\newcommand{\vg}{\mathbf{g}}
\newcommand{\vh}{\mathbf{h}}
\newcommand{\vi}{\mathbf{i}}
\newcommand{\vj}{\mathbf{j}}
\newcommand{\vk}{\mathbf{k}}
\newcommand{\vl}{\mathbf{l}}
\newcommand{\vm}{\mathbf{m}}
\newcommand{\vn}{\mathbf{n}}
\newcommand{\vo}{\mathbf{o}}
\newcommand{\vp}{\mathbf{p}}
\newcommand{\vq}{\mathbf{q}}
\newcommand{\vr}{\mathbf{r}}
% \newcommand{\vs}{\mathbf{s}}
\newcommand{\vt}{\mathbf{t}}
\newcommand{\vu}{\mathbf{u}}
\newcommand{\vv}{\mathbf{v}}
\newcommand{\vw}{\mathbf{w}}
\newcommand{\vx}{\mathbf{x}}
\newcommand{\vy}{\mathbf{y}}
\newcommand{\vz}{\mathbf{z}}
\newcommand{\vT}{\mathbf{T}}
\newcommand{\vW}{\mathbf{W}}
\newcommand{\vX}{\mathbf{X}}
\newcommand{\vY}{\mathbf{Y}}
\newcommand{\vZ}{\mathbf{Z}}

\input{docu/latex/0_Abstr.tex}
\section{Introduction} \label{sec:intro}

Large amounts of time and effort are devoted to
verification and validation of every microprocessor design project.
Broadly, design verification can be broken into two large categories:
(1) functional and (2) performance verification, which is to identify design bugs that degrade performance without affecting functionality. Performance bugs are different from performance bottleneck as the former is due to design mistakes while the later is caused by tight resource constraints. Performance loss due to performance bugs  can 
be very significant, with recent reported cases shown to be
$>10\%$~\cite{mccalpin2018hpl}. This demonstrates a critical 
need for automated mechanisms for performance debugging.  As 
recent designs from Intel~\cite{corei7-11}, AMD~\cite{ryzen-9},
ARM~\cite{cortex-a}, and others place an even greater emphasis
on core performance, design complexity has scaled
dramatically, likewise scaling the difficulty in all forms of
verification.


%Functional verification has received extensive attention from researchers and, although complex, it benefits from the availability of known correct outputs that can be used to compare against.

Performance verification at microarchitecture level ensures that a
design correctly achieves expected performance in terms of execution
time or cycle count.  The main challenge in this task is that, unlike
functional verification, there is no exact golden reference to compare
against.  This is because of the high difficulty of modeling all the
interactions between the different units in complex microprocessor
designs, and accurately represent how they affect the overall system
performance.  %This task also suffers from 
%the lack of a good debugging infrastructure, as well as from 
%limited visibility into intermediate points in the design, which are mostly exposed through performance counters. Although useful for estimating the performance of the system, these counters are very difficult to use for manual debugging because of their complex relationship with processor performance and due to the large amounts of data they generate.  
Traditionally, performance
verification is conducted mostly through manual techniques which rely
on rough estimations of performance gain expected by
microarchitectural changes~\cite{Singhal2004}. Such manual processes
are not only very lengthy but also error-prone.



The process of performance verification and debugging roughly consists of two steps: (1)~detection, which determines whether a
design achieves expected performance or not, and (2)~localization,
which identifies the microarchitectural units causing the performance
issues and is the focus of this work.

There are few previous studies on automating detection of
microprocessor performance bugs~\cite{Bose1994,
  surya1994architectural,carvajal2021detection}. 
The majority of those~\cite{Bose1994, surya1994architectural} relies on capturing
design intentions using a bespoke performance model as a golden
reference, this  entails long development time and may contain
errors by itself. Recently, a data driven and machine learning
(ML)-based approach~\cite{carvajal2021detection} was developed for
automatic performance bug detection with high accuracy. Although
significant, these works do not solve the 
problem of performance bug localization.
%pressing problem of identifying where the performance bug is.

Works in automating microprocessor performance bug localization
are even scarcer.  Adir \emph{et al.}~\cite{adir2005generic}
propose perhaps the only partially related work of which we are
aware.  Their work focuses on formal planning of test program
generation for individual units, such as issue queues. This strategy
follows conventional functional verification, involving heavy
manual effort, costing significant engineer-time to develop a test
plan, and as much as ten days of computer runtime per functional
unit. To the best of our knowledge, there has been no systematic study
on automatic performance bug localization for microarchitecture
designs.

Performance bug localization is a complicated task, which is currently
solved using mostly manual techniques.
Even
in the more widely studied area of functional validation, the industry
lacks a standardized mechanism to automate bug localization, it has
been only recently that academic efforts have attempted to automate
this task~\cite{BugMD}. Considering this, it is important to note that
any type of design automation which successfully reduces the 
time and effort required by engineers to debug their designs is highly
valuable. Since automatic performance debug for microprocessors is
a huge yet under-studied challenge, it is very difficult, if not impossible, to find a perfect solution in a single work. Although our work is not perfect, it serves a key stepping stone 
 toward solving the problem.

This work tackles the performance bug localization problem by
using ML to generate a ranked list of most likely mi\-cro\-ar\-chi\-tec\-tur\-al units that 
might contain the bug.  This list may be used
to prioritize the debugging order, as well as to identify
teams with the right expertise to perform further debug. Two different methodologies are
proposed, evaluated, and contrasted. These data-driven
techniques achieve high
accuracy, while being fully automated. Further, they
consider intra- and inter-unit interactions, as opposed to other
techniques proposed in the partially related previous work~\cite{adir2005generic} which
considered only intra-unit behavior.

%Our methods are based on ML, wherein our models are
%trained using data from legacy designs.
%To take the full advantage of
%these approaches, we assume that architectural changes in a new design
%are incremental when compared to its previous
%generations. Examining recent processors from major vendors including
%Intel, ARM, and AMD, we find this assumption holds true, since the generational change in microarchitectures
%is largely incremental. Thus, the methodologies proposed here provide
%alue for a multitude of upcoming designs.  However, even when
%disruptive changes occur, the methodologies can still be beneficial for bug localization on structures that conform to previous microarchitectures, using workloads that
%do not exercise new functionalities. Further, as general purpose microarchitectures become ever more mature, and the inter-generational performance gains decrease, 
%it is even more important to retain as much performance as possible, making performance debugging ever more important.

The major contributions of this work include the following:
\begin{compactitem}
\itemsep0em 
\item This is the first systematic study on fully automatic
  performance bug localization for microarchitecture designs, to the
  best of our knowledge.

\item Two ML-based approaches to tackle performance bug
  localization, as well as a hybrid of both, are evaluated
  and contrasted.

\item For bugs with an average IPC impact greater than 1\%, our best
  performing methodology identifies the correct bug location as the
  most likely unit in $\sim77\%$ of the cases, and achieves over 90\%
  accuracy when the three most likely options (out of 11 possible) are
  considered.  

\item One of the proposed methodologies is not only very accurate localizing
performance bugs, but it can also be applied to confirm the results
of performance bug detection with high accuracy.

\item Although the focus of this work is on microprocessor core,
we evaluated our methodologies on the processor memory hierarchy. This evaluation
uses a different experimental setup, showing the robustness of the proposed techniques.

\end{compactitem}

As an early work on performance bug localization, the design of this study is subject to potential limitations, however, we feel it still represents a good first step towards solving the problem. The scope of our work and its limitations are as follows:
\begin{compactitem}

\item Legacy
designs with identified performance bugs are required, so that the ML
models can be trained. Bug-free legacy designs are required only 
in one of the methodologies, yet, if available, the other can take advantage of the additional data.
However, thanks to the thorough pre- and post-silicon
debug to which the designs are submitted, these legacy designs are
generally available.

\item We assume that only one bug is present at a time, 
in parallel to the single-fault model which is common practice
in VLSI testing works. As explained in Section~\ref{subsec:impl_bugs}, we still expect 
our methodologies to work well in the presence of multiple bugs in a single design.

\item Our methodologies do not provide a quantitative coverage guarantee.  
In general, performance bug
coverage is extremely difficult to define and is a potential
research topic on its own. We know of no prior work which presents a
definition of such coverage. Nonetheless, the evaluated bugs are based on published errata, cover a large amount of microarchitectural units and affect the system in a variety of ways. Thus, we feel these bugs represent a reasonable start for early work in this area.

\item We assume that there are no dramatic structural
microarchitectural changes between the legacy designs and the
designs under debug. Examining recent processors from major vendors, including Intel, ARM, and AMD, we find this assumption holds true, since the generational change in microarchitectures
is largely incremental. That said, even when larger shifts occur, the
methodologies can be partially reused. For example, consider the
introduction of the AVX instructions with Intel's Sandy Bridge
architecture in 2011.  Initially there would be no available data to
test these instructions using our methodologies, however the rest of
the Sandy Bridge design could be debugged with our methodology,
leveraging workloads that do not exercise the new instructions.  In
future implementations, data from Sandy Bridge can be used to build
the models required to use our methods for debugging AVX. 
%Further, as general purpose microarchitectures become even more mature, and the inter-generational performance gains decrease, it is even more important to retain as much performance as possible, making performance debugging even more important.}

\item We limit our evaluation to a pre-silicon setup, because
it is infeasible for us to inject known design bugs in silicon to
evaluate the methodologies.  Further, should our methodologies be
applied to a commercially available design, and an actual bug be
found and localized, we would not be able to verify that such
localization is correct without prior knowledge of its existence so
as to verify our findings. However, our methodologies can be applied in both pre- and post-silicon scenarios. During pre-silicon stages fixing performance 
bugs is easier and cheaper, 
the availability of performance counters is greater (due to the usage of a
simulator) and the counters can be sampled at a much faster rate. 
By using only counters available in-silicon, and adjusting the sampling frequency, we could use the proposed
methodologies during post-silicon stages. In post-silicon analysis the methodology
could be applied to longer workloads, providing a way to exercise complicated bugs that
are not possible to trigger with short pre-silicon traces.
%Further, we can follow hybrid approaches where the ML model training is performed using simulations, and the techniques are applied to data obtained from microprocessors during post-silicon debug. }

\end{compactitem}

Despite the aforementioned, we present a first, useful, yet attainable,
step towards the goal of performance bug localization, and we hope this work can draw the attention of the research community
to the broader performance validation domain.

\iffalse{
In Section~\ref{sec:scope} we describe the problem
formulation and outline the scope of this work.  We note that, to
date, very little work exists in automating performance bug
localization. 

Section~\ref{sec:methodology} 
describes the approaches developed to tackle the performance bug
localization task.  Section~\ref{sec:experimental_setup} provides
details of the architectures, and performance bugs used for
evaluation. Section~\ref{sec:evaluation} presents results obtained in
several experiments developed to evaluate the methodologies. A brief
review of previous work related to performance debugging is presented
in Section~\ref{sec:related_work}.  And finally,
Section~\ref{sec:conclusion} concludes the paper.
}
\fi
\section{Related Work}
\label{sec:related}

\noindent \textbf{Action Recognition.}
%
Most recent approaches for action recognition are to exploit appearance and motion cues jointly and achieve remarkable success~\cite{feichtenhofer2019slowfast,i3d,lin2019tsm,huang2021tada,qing2022learning,wang2021oadtr,pei2022learning}.
%
Typically, two-stream networks~\cite{two-stream,two-stream-2,TSN} consist of two branches that explore spatial information and temporal dynamics, respectively.
%
% To overcome the limitation of 2D CNN in modeling long-range dependencies, some attempts~\cite{lin2019tsm,r2+1d,TDN} begin to introduce additional temporal mining operations.
Some attempts~\cite{lin2019tsm,r2+1d,TDN} introduce additional temporal mining operations to overcome the limited temporal information extraction ability of 2D CNN.
%
3D CNN-based methods~\cite{feichtenhofer2019slowfast, i3d,C3D} inflated 2D kernels for joint spatio-temporal modeling.
%
\cite{bai2020prototype} proposes the prototype similarity learning which pushes the learned representation to the corresponding prototype as close as possible, while our PSL keeps the differences among the same class.

\noindent \textbf{Open-set Action Recognition.} The related work of OSAR is limited~\cite{krishnan2018bar,shu2018odn,yang2019open,bao2021evidential}. Recently, \cite{bao2021evidential} systematically studies the OSAR problem and transfers several open-set image recognition methods to the video domain, including SoftMax~\cite{hendrycks2016baseline}, MC Dropout~\cite{gal2016dropout}, OpenMax~\cite{bendale2016towards}, and RPL~\cite{chen2020learning}. In the benchmark of~\cite{bao2021evidential}, the only two methods designed specifically for the video domain are BNN SVI~\cite{krishnan2018bar} and their proposed DEAR. BNN SVI is a Bayesian NN application in the OSAR, while DEAR adopts the deep evidential learning~\cite{amini2020deep} to calculate the uncertainty, and utilizes two modules to alleviate the over-confidence prediction and appearance bias problem, respectively. Existing methods pursue better uncertainty scores, while the objective of our PSL is to learn more diverse feature representations for better open-set distinguishability.

\noindent \textbf{Information Bottleneck Theory.} Based on the IB theory~\cite{tishby2000information,tishby2015deep}, the NN intends to extract minimum sufficient information of the inputs for the current task. More recent~\cite{tian2020makes,federici2020learning,wang2022rethinking} adopt the IB theory on unsupervised contrastive learning to analyze the representation learning behavior under the corresponding tasks. In this work, we provide a new view to analyze the OSAR problem based on the IB theory.
\section{Information Analysis in OSAR}
\label{sec:ana}

\subsection{Prototypical Learning}
Let $f$ be the encoder to extract the information for an input video sample $x$ and output the feature representation $z = f(x), z \in \mathbb{R}^d$. We first define a prototypical learning (PL) loss~\cite{yang2018robust}, which is a general version of the cross-entropy (C.E.) loss:
\begin{equation}
\vspace{-0.2cm}
    \mathcal{L}_{PL} = - \log \frac{\exp (\frac{z^T k_{i}} {\tau}) }{ \exp (\frac{z^T k_{i}}{ \tau}) + \sum\limits_{n \in K_i^{-}} \exp (\frac{z^T  n}{ \tau})},
    \label{eq:PL}
    \vspace{-0.2cm}
\end{equation}
where $i$ is the ground truth label of $x$, $k_i \in \mathbb{R}^d$ is the prototype for class $i$, $\tau$ is a temperature parameter, $K_i^{-}=\left \{ k_j| j \in \left \{ 1,2,...,N \right \}, j\ne i\right \}$ is the negative prototype set, and $N$ is the number of InD classes. Note that $z$ and $k_i$ are normalized by L2 norm, so that $z^T k_{i}$ is the cosine similarity. If we regard prototypes as the row vector of the linear classifier $W \in \mathbb{R}^{N \times d}$, and do not normalize $z$ and $k$ as well as remove $\tau$, $\mathcal{L}_{PL}$ will degenerate to the C.E. loss. We introduce the $\mathcal{L}_{PL}$ so that we can directly manipulate the feature representation $z$.

\subsection{Information Analysis of OSAR}
Let $x_{InD}, z_{InD}$, and $Y$ be the random variables of InD sample, extracted representation of InD sample, and the task to predict the label of $x_{InD}$, where $z_{InD}=f(x_{InD})$. Given the joint distribution of $p(x_{InD},Y)$, the relevant information between $x_{InD}$ and $Y$ is defined as $I(x_{InD},Y)$, where $I$ denotes the mutual information~\cite{tishby2000information}. The learned representation $z_{InD}$ satisfies:
\vspace{-0.2cm}
\begin{equation}
% \vspace{-0.2cm}
    I(x_{InD}; z_{InD}) = \underbrace{I(x_{InD}; z_{InD}|Y)}_{IS} + \underbrace{I(z_{InD};Y)}_{CS},
    \label{eq:sum}
    \vspace{-0.2cm}
\end{equation}
in which $I(x_{InD}; z_{InD}|Y)$ and $I(z_{InD};Y)$ denote the \emph{Instance-Specific (IS)} and \emph{Class-Specific (CS) information} respectively. In \cref{fig:ana}, IS information is blue and orange areas, and CS information is yellow and green areas. CS information is for the closed-set label prediction task $Y$, while IS information is the special information of each sample that is not related to $Y$.

To analyze the information about OSAR, we let $T$ be a random variable that represents the task to distinguish OoD samples from InD samples, then we divide the information contained in $z_{InD}$ about $T$ into two parts~\cite{wang2022rethinking}:
\vspace{-0.2cm}
 \begin{equation}
I(z_{InD};T)=\underbrace{I(z_{InD}|Y; T)}_{IS \; \text{about} \; T} + \underbrace{I(z_{InD};Y;T)}_{CS \; \text{about} \; T},
     \label{eq:OSAR}
     \vspace{-0.2cm}
 \end{equation}
where $I(z_{InD}|Y; T)$ and $I(z_{InD};Y;T)$ are the information about the OoD detection task $T$ in IS and CS information (orange and green areas in \cref{fig:ana} respectively). We can see that larger IS and CS information are helpful for OSAR.

In this paper, we aim to enlarge the information about $T$ contained in CS and IS information for better OSAR performance, as illustrated in \cref{fig:1} (b) and the enlarged green and orange areas in \cref{fig:ana}. We first analyze the CS and IS information behaviors under the classical C.E. loss, and find that CS information is encouraged to be maximized but IS information tends to be eliminated in \cref{sec:CS_IS_CE}. Then we explain this conclusion from the IB theory view in \cref{sec:ib_analy}.

\begin{figure}[t]
    \centering
\includegraphics[width=0.99\linewidth]{docu/figs/fig_2.pdf}
    \vspace{-0.4cm}
    \caption{The neural network (NN) can only extract limited representations $z_{InD}$ of the InD sample $x_{InD}$ for the current task $Y$ (predict the closed-set label), which is not diverse enough for the task $T$ (distinguish OoD samples), as green and orange areas are small in (a). In our PSL, we encourage the NN to learn a more diverse representation so that more IS and CS information about $T$ are contained.}
    \label{fig:ana}
\end{figure}


\subsection{CS and IS Information Behavior under C.E.}
\label{sec:CS_IS_CE}
CS information is for closed-set classification task $Y$, so it is similar for the same class sample, but distinct for the different class sample ($s_1,s_2/s_3$ in \cref{fig:1}). In contrast, IS information is not related to $Y$ and it is distinct for samples in the same class ($s_1, s_2$ in \cref{fig:1}). Therefore, we have the following proposition which describe the relation between CS/IS information and feature representation similarity.
\begin{proposition}
\vspace{-0.2cm}
For two feature representations of samples in the same class, more CS information means these two feature representations are more similar, and more IS information decreases their feature similarity.
\label{prop:cs_is}
\vspace{-0.2cm}
\end{proposition}

CS information is for the closed-set label prediction task $Y$, which is fully supervised by C.E. loss, so it is maximized during training.  In contrast, Eq.~\ref{eq:PL} shows that C.E. encourages representations of the same class to be exactly same with the corresponding prototype, and such high similarity eliminates the IS information according to Proposition~\ref{prop:cs_is}. Therefore, \textbf{C.E. loss tends to maximize the CS information and eliminate the IS information in the feature representation}. We analyze this conclusion based on Information Bottleneck (IB) theory in next~\cref{sec:ib_analy}.

\subsection{IB Theory Analysis for CS and IS Information}
\label{sec:ib_analy}

Applying the Data Processing Inequality~\cite{cover1999elements} to the Markov chain $Y \to x_{inD} \to z_{InD}$, we have
\vspace{-0.2cm}
\begin{equation}
    I(z_{InD};Y) \le I(x_{InD};Y).
    \label{eq:mar}
    \vspace{-0.2cm}
\end{equation}
It means that the compressed representation $z_{InD}$ cannot contain more information of $Y$ compared to the original data $x_{InD}$. 

According to the IB theory~\cite{tishby2000information,tishby2015deep}, the NN is to find the optimal solution of $z_{InD}$ with minimizing the following Lagrange:
\vspace{-0.2cm}
\begin{equation}
    \mathcal{L}[p(z_{InD}|x_{InD})]=I(z_{InD}; x_{InD})-\beta I(z_{InD}; Y),
    \label{eq:lar}
    \vspace{-0.2cm}
\end{equation}
where $\beta$ is the Lagrange multiplier attached to the constrained meaningful condition. Eq.~\ref{eq:lar} demonstrates the NN is solving a trade-off problem, as the first term tends to keep the information of $x_{InD}$ as less as possible while the second term tends to maximize the information of $Y$.

Inspired by \cite{wang2022rethinking,achille2018emergence}, the sufficient and minimum sufficient representation of $x_{InD}$ about $Y$ can be defined as:
\begin{definition}(Sufficient Representation) A feature representation $z_{InD}^{suf}$ of $x_{InD}$ is sufficient for $Y$ if and only if $I(z_{InD}^{suf};Y)=I(x_{InD}; Y)$.
\label{def:suf}
\end{definition}
\begin{definition}(Minimum Sufficient Representation) A sufficient representation $z_{InD}^{min}$ of $x_{InD}$ is minimum if and only if $I(z_{InD}^{min};x_{InD}) \le I(z_{InD}^{suf}; x_{InD})$, ${\forall} z_{InD}^{suf}$ that is sufficient for $Y$.
\label{def:mini}
\end{definition}

\noindent \textbf{CS Information Maximization.} The goal of training is to optimize $f$ so that $I(z_{InD};Y)$ (CS information) can approximate $I(x_{InD};Y)$, which stays unchanged as data distribution is fixed during training. Therefore, CS information is supposed to be maximized to the upper bound $I(x_{InD};Y)$ because of Eq.~\ref{eq:mar}. In this way, the closed-set classification task pushes the NN to learn the sufficient representation $z_{InD}^{suf}$ according to definition~\ref{def:suf}~\cite{federici2020learning}. 

\noindent \textbf{IS Information Elimination.}
When $z_{InD}$ is close to the sufficient representation $z_{InD}^{suf}$, the second term in Eq.~\ref{eq:lar} will be the fix value $I(x_{InD}; Y)$ based on the definition~\ref{def:suf}. So the key to minimize Eq.~\ref{eq:lar} is to minimize the first term $I(z_{InD}^{suf}; x_{InD})$. Based on the definition~\ref{def:mini}, the lower bound of $I(z_{InD}^{suf}; x_{InD})$ is $I(z_{InD}^{min}; x_{InD})$, so we can conclude that the learned representation is supposed to be the minimum sufficient representation $z_{InD}^{min}$~\cite{wang2022rethinking}. We substitute $I(z_{InD}^{suf}; x_{InD})$ and $I(z_{InD}^{min}; x_{InD})$ in definition~\ref{def:mini} with Eq.~\ref{eq:sum} and we have
\vspace{-0.2cm}
\begin{align}
    & I(x_{InD}; z_{InD}^{min}|Y) + I(z_{InD}^{min};Y) \nonumber \\
    \le &  
    I(x_{InD}; z_{InD}^{suf}|Y) + I(z_{InD}^{suf};Y).
    \label{eq:ine}
\end{align}
As both $z_{InD}^{min}$ and $z_{InD}^{suf}$ are sufficient, the second term of both sides in Eq.~\ref{eq:ine} is $I(x_{InD}; Y)$, so we have
\begin{equation}
    0 \le I(x_{InD}; z_{InD}^{min}|Y)
    \le
    I(x_{InD}; z_{InD}^{suf}|Y).
\end{equation}
Therefore, the learned IS information in $z_{InD}^{min}$ is smaller than any IS information in $z_{InD}^{suf}$, which could be eliminated to 0~\cite{wang2022rethinking} (no blue and orange areas in $z_{InD}^{min}$ in \cref{fig:ana}).

\subsection{Enlarge CS and IS Information for OSAR}
\label{sec:lar_cs_is}
Based on the analysis in \cref{sec:CS_IS_CE} and \cref{sec:ib_analy}, we show that C.E. tends to maximize the CS information and eliminate the IS information in the feature representation. Both larger IS and CS information are crucial for OSAR according to Eq.~\ref{eq:OSAR}, but C.E. does not bring the optimal information. On the one hand, IS information is eliminated so we lose a part of information which is beneficial for the OSAR. On the other hand, the learned representation is not sufficient and does not contain enough CS information in practice due to the model capacity and data distribution shift between training and test sets, which can be supported by the fact that test accuracy cannot reach 100\%. Therefore, we propose our method to enlarge the CS and IS information for better OSAR performance in next~\cref{sec:method}.


% \begin{table}[t]
% \centering
% \caption{Details of the datasets used in our work.}
% \label{tab:datasets}
% \begin{tabular}{lccc}
% \hline
% Datasets & \begin{tabular}[c]{@{}c@{}}Exposure\\ Ratios\end{tabular} & \begin{tabular}[c]{@{}c@{}}Training\\ images\end{tabular} & \begin{tabular}[c]{@{}c@{}}Validation\\ images\end{tabular} \\ \hline
% Sony \cite{chen2018learning} & 90,150,300 & 161 &  36\\
% Nikon &  100, 300 & 53 &  24 \\
% Canon \cite{CanonLSID} & 50, 150, 300 & 44 &  21\\ \hline
% \end{tabular}
% \end{table}
%--------------------------------------------------------------------------------------------------------------------
% \begin{figure*}
% \begin{center}
% \includegraphics[width=\textwidth, clip, trim=0cm 15.55cm 0.75cm 0.05cm]{figures/FDA-LSID.png}
% \end{center}
%   \caption{Proposed few-shot domain adaptation model architecture.}
% \label{fig:model}
% \end{figure*}
% % \begin{figure*}
% % \begin{center}
% % \includegraphics[scale=0.42]{figures/convert.png}
% % \end{center}
% %   \caption{Source camera specific 16-to-8-bit converter.}
% % \label{fig:convert}
% % \end{figure*}
\begin{figure}[t]
\begin{tabular}{cc}
\includegraphics[page=1, width=5.8cm]{Images/nikon.png}&\hspace{+1mm}
\includegraphics[page=1, clip, trim=0.6cm 16.75cm 7.5cm 0.15cm, scale=0.9]{Images/block_diagram.pdf}\\
(a)&(b)
\end{tabular}
\caption{(a) Example short-exposure and long-exposure image pairs from the Nikon dataset. The short exposure images are almost entirely dark whereas the long-exposure images have immense scene information. (b) Overview of our few-shot domain adaptation method.}
\label{fig:nikon}
\label{fig:prop_overview}
\end{figure}
With a noisy raw image captured with low-exposure time (i.e., shutter speed) as input, our CNN-based approach is trained to predict a clean long-exposure sRGB output of the same scene. The input is multiplied by an exposure factor calculated by the ratio of output and input exposure times. For example, to generate a 10-second long exposure output, the input 0.1-second low exposure image must be multiplied by 100. As a result of this operation, along with illumination, the noise is also amplified proportionally. Since we multiply the factor in the unprocessed raw domain and expect the output in the sRGB domain, the network must learn camera hardware-specific enhancement as well as its entire ISP pipeline (lens correction, demosaicing, white balancing, color manipulation, tone curve application, color space transform, and Gamma correction). Thus, a model trained on one specific camera data (source domain) does not translate similar performance to a different camera (target domain), hence the domain gap. In this paper, we propose to transfer the enhancement task from large labeled source data and generate output in the target domain using few labeled target data.

\textbf{Problem formulation}: We denote source domain ($\mathbf{S}$) with input short-exposure images as $\{S_n\}$ and corresponding long-exposure ground truth as $\widehat{\mathbf{S}}\!=\!\widehat{S}_n$, $\forall n=1,\cdots,N$. Similarly, the target domain ($\mathbf{T}$) consists of input images $\{T_m\}$ and corresponding ground truth, $\widehat{\mathbf{T}}\!=\!\widehat{T}_m$, $\forall m=1,\cdots,M$. Note that $N$ is much greater than $M$, $N\gg M$. With both $\mathbf{S}$ and $\mathbf{T}$ as input, we train a CNN model ($\mathbb{N}$) to generate enhanced long-exposure output ($\widetilde{\mathbf{S}}$ and $\widetilde{\mathbf{T}}$). Our method is illustrated in Fig. \ref{fig:prop_overview}(b) with the source and target training pipelines. It is an end-to-end trainable deep network that takes the raw sensor arrays as input and performs image enhancement utilizing the source data for few-shot domain adaptation to the target data.

% \begin{figure*}[t]
%     \centering
%     \includegraphics[width=\linewidth]{3_BMVC/Images/Nikon-Results-1.pdf}
%     \caption{Qualitative comparison with methods tested on Nikon target images. (b) HDRCNN and (c) Unprocess are trained on Sony source and fine-tuned on 4-Nikon target images, LSID with (d) 4-Nikon target images and (e) full ($k$=53) Nikon training dataset, (f) Our few-shot domain adaptation approach with 4-Nikon target images and 161 Sony source images.}
%     \label{fig:nikon_eg1}
% \end{figure*}
% \begin{table}[t]
% \centering
% \caption{Quantitative comparison of Sony as source and Nikon as target dataset. The improvement of proposed method over only $k$ shot trained model is shown in brackets. The LSID model trained with full Nikon dataset ($k$=53) achieves 30.74dB PSNR and 0.803 SSIM.}
% \label{tab:nikon}
% \scalebox{0.735}{
% \begin{tabular}{@{}lccc|ccc@{}}
% \hline
%  & \multicolumn{3}{c|}{PSNR} & \multicolumn{3}{c}{SSIM} \\ \hline
% $k$ ($\rightarrow$) & 1 & 2 & 4 & 1 & 2 & 4 \\ \hline
% \begin{tabular}[c]{@{}l@{}}LSID\\ (only $k$ target)\end{tabular} & 23.20 $\pm$ 3.06 & 27.27 $\pm$ 0.384 & 28.05 $\pm$ 1.53 & 0.679 $\pm$ 0.172 & 0.819 $\pm$ 0.031 & 0.864 $\pm$ 0.0111 \\ \hline
% \begin{tabular}[c]{@{}l@{}}Proposed\\ ($k$ target + source)\end{tabular} & \begin{tabular}[c]{@{}c@{}}\textbf{25.27} $\pm$ 0.58\\ (+2.07)\end{tabular} &
% \begin{tabular}[c]{@{}c@{}}\textbf{28.06} $\pm$ 0.671\\ (+0.79)\end{tabular} &
% \begin{tabular}[c]{@{}c@{}}\textbf{30.30} $\pm$ 0.52\\ (+2.25)\end{tabular} & \begin{tabular}[c]{@{}c@{}}\textbf{0.860} $\pm$ 0.010\\ (+0.181)\end{tabular} &
% \begin{tabular}[c]{@{}c@{}}\textbf{0.909} $\pm$ 0.0028\\ (+0.090)\end{tabular} &
% % \begin{tabular}[c]{@{}c@{}}\textbf{0.913} $\pm$ 0.006\\ (+0.049)\end{tabular} \\ \midrule
% % \begin{tabular}[c]{@{}l@{}}LSID\\ (full target, $k$ = 53)\end{tabular} & \multicolumn{3}{c|}{30.74} & \multicolumn{3}{c}{0.803} \\ \bottomrule
% \begin{tabular}[c]{@{}c@{}}\textbf{0.913} $\pm$ 0.006\\ (+0.049)\end{tabular} \\ \hline
% \end{tabular}
% }
% \end{table}

\textbf{Encoders}: \label{sec:pipeline} The significant domain gap between the source and target domains necessitates the extraction of separate and independent features from each domain before processing with a shared enhancement network ($\mathbb{N}$). Hence, we use a source encoder ($\mathcal{E}_S$) and a target encoder ($\mathcal{E}_T$). We first pack the input raw sensor arrays into a four-channel vector (for Bayer arrays from Sony, Nikon, and Canon cameras) and subtract the black level (reference voltage). Then, the packed array is multiplied by the exposure ratio factor and passed as input to the respective domain encoder. It should be noted that the exposure ratio factors need not be the same between the source and the target domain (See Table \ref{tab:datasets}). For the encoder network, we use three convolutional layers with $\{16,32,64\}$ filters and 3$\times$3 kernel size. 

\textbf{Enhancement Network}: The source and target domain encoder features are passed separately to a shared common enhancement network, $\mathbb{N}$. By having a common enhancement network, the large pool of source data helps to improve the enhancement quality of $\mathbb{N}$, while the few target samples ensure that the output is in the target domain. We use U-Net architecture for the enhancement network. Further, the network has a pixel shuffle layer to convert 12-channel prediction to 16-bit three channel sRGB output. The objective of $\mathbb{N}$ is to enhance, denoise, perform other ISP operations (AWB, color manipulation, etc.), and finally demosaicking to generate an sRGB output. $\mathbb{N}$ generates enhanced output $\widetilde{\mathbf{T}}$ for the target domain data as, $\widetilde{\mathbf{T}} = \mathbb{N}\big(\mathcal{E}_T(\mathbf{T}) \big)$. Similarly, $\widetilde{\mathbf{S}}$ for the source domain as, $\widetilde{\mathbf{S}} = \mathbb{N}\big(\mathcal{E}_S(\mathbf{S}) \big)$.
%\footnote{Please refer to the supplementary material for detailed network definition.}

\textbf{Losses}: For the target domain, we compute the $\ell_1$ loss between the prediction ($\widetilde{\mathbf{T}}$) and the ground truth ($\widehat{\mathbf{T}}$) as, $\mathcal{L}_{target} = \ell_1\big(\widetilde{\mathbf{T}},\widehat{\mathbf{T}} \big)$. The source domain loss consists of two components: cosine similarity loss and SSIM loss. We compute cosine similarity between $\widetilde{\mathbf{S}}$ and $\widehat{\mathbf{S}}$ as, 
$
    \mathcal{L}_{CS}(\widetilde{\mathbf{S}},\widehat{\mathbf{S}})= 1 -  \frac{{\widetilde{\mathbf{S}} \cdotp \widehat{\mathbf{S}}}}{\|\widetilde{\mathbf{S}}\|\times\|\widehat{\mathbf{S}}\|}
$.
% \begin{equation}
%     \mathcal{L}_{CS}(\widetilde{\mathbf{S}},\widehat{\mathbf{S}})= 1 -  \frac{{\widetilde{\mathbf{S}} \cdotp \widehat{\mathbf{S}}}}{\|\widetilde{\mathbf{S}}\|\times\|\widehat{\mathbf{S}}\|}
% \end{equation}
Cosine similarity loss is weak supervision for the source domain and is used instead of $\ell_1$ loss since $N\gg M$, and using a strong supervision loss like $\ell_1$ optimizes for pixel values to train $\mathbb{N}$, making the network predict the output in the source domain even for target domain input. Cosine similarity loss ensures that the prediction and the ground truth are in a similar direction. Hence with $\mathcal{L}_{CS}$, $\mathbb{N}$ can still perform enhancement while predicting in target domain even for source domain input. Further, when trained with Sony as source and 4-shot Nikon as target (Table \ref{tab:ablation}) with $L_1$ loss for the source, we obtain only 27.14dB PSNR for target domain validation, whereas using $\mathcal{L}_{CS}$ loss for source achieves 30.30dB PSNR.

% \begin{figure*}[t]
%     \centering
%     \includegraphics[width=\linewidth]{3_BMVC/Images/Canon-Results-1.pdf}
%     \caption{Qualitative comparison with methods tested on Canon target images. (b)HDRCNN and (c) Unprocess are trained on Sony source and fine-tuned on 6-Canon target images, LSID with (d) 6-Canon target images and (e) full ($k$=44) Canon training dataset, (f) Proposed few-shot domain adaptation approach with 6-Canon target images and 161 Sony source images.}
%     \label{fig:canon_eg2}
% \end{figure*}

\begin{figure*}[t]
\centering
\subfigure{\includegraphics[width=\linewidth]{Images/Nikon-Results-1.pdf}}\\ \vspace{-1.65\baselineskip}
\subfigure{\includegraphics[width=\linewidth]{Images/Canon-Results-1.pdf}}
\caption{Qualitative comparison with methods tested on Nikon (top row) and Canon (bottom row) target images. (a) Input after multiplying by exposure factor, results from (b) HDRCNN and (c) Unprocess methods are after training on full Sony source and fine-tuning on $k$-shot target images, LSID with (d) $k$-shot target images and (e) full target training dataset ($k$=53 for Nikon and $k$=44 for Canon), (f) Proposed few-shot domain adaptation method with 161 Sony source images and 4-shot Nikon (top row) and 6-Canon (bottom row) target images.}
\label{fig:nikon_eg1}
\label{fig:canon_eg2}
\end{figure*}

% \begin{table}[t]
% \centering
% \caption{Quantitative comparison of Sony as source and Canon as target dataset. The improvement of proposed method over only $k$ shot trained model is shown in brackets. The LSID model trained with full Canon dataset ($k$=44) attains 32.32dB PSNR and 0.899 SSIM.}
% \label{tab:canon}
% \scalebox{0.73}{
% \begin{tabular}{@{}lccc|ccc@{}}
% \hline
%  & \multicolumn{3}{c|}{PSNR} & \multicolumn{3}{c}{SSIM} \\ \hline
% $k$ ($\rightarrow$) & 1 & 3 & 6 & 1 & 3 & 6 \\ \hline
% \begin{tabular}[c]{@{}l@{}}LSID\\ (only $k$ target)\end{tabular} & 21.54 $\pm$ 2.89 & 26.9 $\pm$ 2.37 & 29.36 $\pm$ 0.763 & 0.588 $\pm$ 0.182 & 0.785 $\pm$ 0.0051 & 0.829 $\pm$ 0.0073 \\ \hline
% % \begin{tabular}[c]{@{}l@{}}Proposed\\ ($k$ target + source)\end{tabular} & \begin{tabular}[c]{@{}c@{}}\textbf{24.29} $\pm$ 3.16\\ (+2.75)\end{tabular} & \begin{tabular}[c]{@{}c@{}}\textbf{28.78} $\pm$ 3.54\\ (+1.8)\end{tabular} & \begin{tabular}[c]{@{}c@{}}\textbf{33.22} $\pm$ 0.45\\ (+3.86)\end{tabular} & \begin{tabular}[c]{@{}c@{}}\textbf{0.623} $\pm$ 0.0074\\ (+0.035)\end{tabular} & \begin{tabular}[c]{@{}c@{}}\textbf{0.841} $\pm$ 0.0335\\ (+0.056)\end{tabular} & \begin{tabular}[c]{@{}c@{}}\textbf{0.896} $\pm$ 0.015\\ (+0.067)\end{tabular} \\ \midrule
% % \begin{tabular}[c]{@{}l@{}}LSID\\ (full target, $k$ = 45)\end{tabular} & \multicolumn{3}{c|}{32.32} & \multicolumn{3}{c}{0.899} \\ \bottomrule
% \begin{tabular}[c]{@{}l@{}}Proposed\\ ($k$ target + source)\end{tabular} & \begin{tabular}[c]{@{}c@{}}\textbf{24.29} $\pm$ 3.16\\ (+2.75)\end{tabular} & \begin{tabular}[c]{@{}c@{}}\textbf{28.78} $\pm$ 3.54\\ (+1.8)\end{tabular} & \begin{tabular}[c]{@{}c@{}}\textbf{33.22} $\pm$ 0.45\\ (+3.86)\end{tabular} & \begin{tabular}[c]{@{}c@{}}\textbf{0.623} $\pm$ 0.0074\\ (+0.035)\end{tabular} & \begin{tabular}[c]{@{}c@{}}\textbf{0.841} $\pm$ 0.0335\\ (+0.056)\end{tabular} & \begin{tabular}[c]{@{}c@{}}\textbf{0.896} $\pm$ 0.015\\ (+0.067)\end{tabular} \\ \hline
% \end{tabular}
% }
% \end{table}

From experiments (in section \ref{sec:exp}), we find better enhancement (in terms of PSNR) using the structural similarity index measure (SSIM) \cite{wang2004image} to compute perceived degradation and preserve the spatial structure in the source output with respect to the ground truth. We do not use SSIM directly on the 16-bit data as that causes the source data to heavily influence the domain adaptation since the source dataset is much larger. Hence, we apply SSIM in JPEG compressed 8-bit domain, where the structural domain difference is less. Since type-casting the 16-bit data to 8-bit will still possess domain-specific details, we train a 16-to-8-bit U-net model ($\mathcal{D}$ in Fig. \ref{fig:prop_overview}) to convert the output from 16-bit to post-processed 8-bit representation. 

The $\mathcal{D}$ network is trained to perform the following non-linear operations: White balancing, Gamma correction, Quantization, and JPEG compression. Even after JPEG compression, the prediction may have traces of source domain specific color information. Further, the SSIM loss is a strong pixel-wise supervision, and in order to avoid the source domain from heavily influencing $\mathbb{N}$, we compute SSIM loss only in grayscale space, not in RGB color space. Also, it follows the intuition that the structure and edge information of a scene will remain the same across images captured with different cameras, while the color space representation may vary. We find that without SSIM loss for the source, we obtain 29.38dB PSNR on target domain validation, whereas using SSIM loss achieves 30.30dB PSNR (Table \ref{tab:ablation}). For computing the SSIM loss, the ground truth ($\widehat{\mathbf{S}}$) is also converted offline to post-processed 8-bit data ($\widehat{\mathbf{S}}_{PP}$) using the rawpy post process function. Hence, the loss is obtained by computing SSIM loss between $\mathcal{D}(\widetilde{\mathbf{S}})$ and $\widehat{\mathbf{S}}_{PP}$,
$
    \mathcal{L}_{SSIM} = 1 - SSIM\Big(\mathcal{D}(\widetilde{\mathbf{S}}), \widehat{\mathbf{S}}_{PP}\Big)
$.
% \begin{equation}
%     \mathcal{L}_{SSIM} = 1 - SSIM\Big(\mathcal{D}(\widetilde{\mathbf{S}}), \widehat{\mathbf{S}}_{PP}\Big)
% \end{equation}
In Fig. \ref{fig:prop_overview}, the top branch guided by the deep red arrows shows the entire source camera training pipeline. It should be noted that $\mathcal{D}$ is used only to compute the loss but not in inference. Finally, we use the sum of cosine similarity loss ($\mathcal{L}_{CS}$) as well as the SSIM loss calculated in the 8-bit domain as the total loss for the source camera pipeline: $\mathcal{L}_{source} = \mathcal{L}_{CS} + \mathcal{L}_{SSIM}$. The total loss is the sum of target and source domain losses: $\mathcal{L}_{total}=\mathcal{L}_{target}+\mathcal{L}_{source}$.
% Source and target models are trained jointly. An
% epoch consists of 161 batches, with one source patch and
% one target domain patch per batch. Both source and target
% patches (of size 512 512, lines 478-480) are cropped at a
% random location from a randomly chosen source and target
% image.

% For the proposed method, we use the respective short-exposure raw sensor data as input to the source ($\mathbf{S}$) and target ($\mathbf{T}$) encoder networks. We first pack the input raw sensor arrays into a four-channel vector (for Bayer arrays from Sony, Nikon, and Canon cameras), subtract the black level (reference voltage), and multiply the input with the exposure ratio. There is one input for the source camera encoder ($\mathcal{E}_S$) and one for the target camera encoder ($\mathcal{E}_T$) in every training step. We pass the output from these camera-specific encoders through a shared $\mathbb{N}$ network to allow the model to learn both camera-specific and camera invariant properties. The output of the $\mathbb{N}$ network is a 12-channel image with half the spatial resolution. % , comprising of a U-net \cite{ronneberger2015u} followed by three CNN layers, 

% \begin{figure}[t]
%     \centering
%     \label{fig:prop_overview}
%     \includegraphics[page=1, clip, trim=0.6cm 16.75cm 7.5cm 0.15cm, scale=0.9]{3_BMVC/Images/block_diagram.pdf}
%     \caption{Overview of our few-shot domain adaptation model. }
% \end{figure}


% \subsection{Source and Target camera pipeline}\label{sec:pipeline}
% The source and target pipelines are trained jointly in an end-to-end manner. An epoch consists of 161 batches, with one source domain patch and one target domain patch per batch 
%given as input to the model. Both source and target patches are $512\times512$ random crops. % are cropped at a random location from a random source and target domain image at each train step.

% \textbf{Source camera pipeline.}
% \label{subsec:source}
% The packed raw input arrays from the source domain are passed to the source encoder, which learns camera-specific parameters to obtain an intermediate representation. We find visible denoising performance from experiments with the encoder heads, and subsequent analysis suggests effective learning of the camera's non-uniform noise model in extreme low-light conditions. The output from the source encoder is passed through the shared $\mathbb{N}$ network to obtain the source output feature maps, $\widetilde{\mathbf{S}}$. Bayer conversion with a sub-pixel layer \cite{shi2016real} unpacks the 12-channel data into a full resolution sRGB image.

% We compute the standard Cosine Similarity loss ($\mathcal{L}_{CS}$) between the source domain output ($\widetilde{\mathbf{S}}$) and the corresponding source domain long-exposure ground-truth image ($\widehat{\mathbf{S}}$),
% % \begin{equation}
% % \mathcal{L}_{CS}(\widetilde{\mathbf{S}},\widehat{\mathbf{S}})= 1 -  \frac{{\widetilde{\mathbf{S}} \cdotp \widehat{\mathbf{S}}}}{\bf \text{max}( \sqrt{({\bf \tilde{S}})^2} \cdotp \sqrt{({\bf \widehat{S}})^2})}
% % \end{equation}
% \begin{align}
%     \widetilde{\mathbf{S}} &= \mathbb{N}\big(\mathcal{E}_S(\mathbf{S}) \big), &    \mathcal{L}_{CS}(\widetilde{\mathbf{S}},\widehat{\mathbf{S}})&= 1 -  \frac{{\widetilde{\mathbf{S}} \cdotp \widehat{\mathbf{S}}}}{\|\widetilde{\mathbf{S}}\|\times\|\widehat{\mathbf{S}}\|}
% \end{align}
% % \begin{equation}
% % \mathcal{L}_{CS}(\widetilde{\mathbf{S}},\widehat{\mathbf{S}})= 1 -  \frac{{\widetilde{\mathbf{S}} \cdotp \widehat{\mathbf{S}}}}{\|\widetilde{\mathbf{S}}\|\times\|\widehat{\mathbf{S}}\|}
% % \end{equation}
  
% Cosine similarity loss is a weak supervision for the source domain and is used instead of $\ell_1$ loss since $N\gg M$, and using a strong supervision loss like $\ell_1$ optimizes for pixel values to train $\mathbb{N}$, making the network predict the output in the source domain even for target domain input. Cosine similarity loss ensures that the prediction and the ground truth are in a similar direction. Hence with $\mathcal{L}_{CS}$, $\mathbb{N}$ can still perform enhancement while predicting in target domain even for source domain input. Further, when trained with Sony as source and 4-shot Nikon as target (Table \ref{tab:ablation}) with $L_1$ loss for the source, we obtain only 27.14dB PSNR for target domain validation, whereas using $\mathcal{L}_{CS}$ loss for source achieves 30.30dB PSNR.

% From experiments (discussed in section \ref{sec:exp}), we find better enhancement (in terms of PSNR) using the structural similarity index measure (SSIM) \cite{wang2004image} to compute perceived degradation and preserve the spatial structure in the source output with respect to the ground truth. We do not use SSIM directly on the 16-bit data as that causes the source data to heavily influence the domain adaptation since the source dataset is much larger. Thus, we apply SSIM in JPEG compressed 8-bit domain, where the structural domain difference is less. Since type-casting the 16-bit data to 8-bit will still possess domain-specific details, we train a 16-to-8-bit U-net model ($\mathcal{D}$ in Fig. \ref{fig:prop_overview}) to convert the output from 16-bit to post-processed 8-bit representation. We find that without SSIM loss for the source, we obtain 29.38dB PSNR on target domain validation, whereas using SSIM loss achieves 30.30dB PSNR (Table \ref{tab:ablation}).

% % (we discuss the converter in section \ref{sec:ablation})

% For computing the SSIM loss, the ground truth ($\widehat{\mathbf{S}}$) is also converted offline to post-processed 8-bit data ($\widehat{\mathbf{S}}_{PP}$) using the rawpy post process function. Hence, the loss is obtained by computing SSIM loss between $\mathcal{D}(\widetilde{\mathbf{S}})$ and $\widehat{\mathbf{S}}_{PP}$.
% \begin{equation}
%     \mathcal{L}_{SSIM} = 1 - SSIM\Big(\mathcal{D}(\widetilde{\mathbf{S}}), \widehat{\mathbf{S}}_{PP}\Big)
% \end{equation}
% In Fig. \ref{fig:prop_overview}, the top branch guided by the deep red arrows shows the entire source camera training pipeline. Finally, we use the sum of cosine similarity loss ($\mathcal{L}_{CS}$) as well as the SSIM loss calculated in the 8-bit domain as the total loss for the source camera pipeline. 
% \begin{equation}
%     \mathcal{L}_{source} = \mathcal{L}_{CS} + \mathcal{L}_{SSIM}
% \end{equation}


% % \begin{figure*}[t]
% %     \centering
% %     \includegraphics[width=17.4cm, height=4.75cm]{ICCV/figures/Nikon-Results-1.pdf}
% %     \caption{Qualitative comparison between different methods tested on images from the Nikon dataset (target). The models are trained on (b) only 4 Nikon images, (c) full Nikon training dataset, and (d) 4 Nikon images and 161 Sony images with our proposed approach.}
% %     \label{fig:nikon_eg1}
% % \end{figure*}

% \textbf{Target camera pipeline.} 
% \label{subsec:target}
% We pass the packed target input raw data to the target encoder ($\mathcal{E}_T$). We then pass the encoded feature maps through the $\mathbb{N}$ network and then through the Bayer converter and calculate the target pipeline's loss in the 16-bit space. From several experiments with various loss functions, we have found the best image enhancement for the target pipeline is achieved with the $\ell_{1}$ loss between the predicted ($\widetilde{\mathbf{T}}$) and ground truth ($\widehat{\mathbf{T}}$),
% \begin{align}
%     \widetilde{\mathbf{T}} &= \mathbb{N}\big(\mathcal{E}_T(\mathbf{T}) \big), &    \mathcal{L}_{target} &= \ell_1\big(\widetilde{\mathbf{T}},\widehat{\mathbf{T}} \big)
% \end{align}

\section{Experiments}
\label{sec:exp}

\noindent \textbf{Datasets.} Following~\cite{bao2021evidential}, we use UCF101~\cite{soomro2012ucf101} as the InD dataset for training and closed-set evaluation, and use HMDB51~\cite{kuehne2011hmdb} and MiT-v2~\cite{monfort2021multi} as OoD data for open-set evaluation. Different from~\cite{bao2021evidential} which does not clean the OoD data that may contains InD classes, we remove the overlapping classes between InD and OoD dataset during evaluation. See Appendix A for more details.

\noindent \textbf{Evaluation protocols.} For closed-set performance, we evaluate like the traditional way to calculate the top-1 accuracy Acc. (\%). For open-set performance, we follow the classical open-set recognition protocol~\cite{hendrycks2016baseline,hendrycks2018deep} to use the obtained uncertainty score Eq.~\ref{eq:unce} to calculate AUROC (\%), AUPR (\%) and FPR95(\%).\footnote{We find AUROC in~\cite{bao2021evidential} only considers one specific threshold based on their code, and after discussion and agreement they provide the modified correct score in our Tab.~\ref{tab:bench_hmdb}. See Appendix B for details.}

\begin{table*}[t]
\centering
\tablestyle{6pt}{0.6}
\begin{tabular}{llcccccccc}
\toprule[1pt]
& & \multicolumn{4}{c}{\bf{w/o K400 Pretrain}} & \multicolumn{4}{c}{\bf{w/ K400 Pretrain}} \\ \cmidrule{3-10}
   \bf{Datasets} &\bf{Methods}                     & \bf{AUROC$\uparrow$}    & \bf{AUPR$\uparrow$}  &\bf{FPR95$\downarrow$}  & \bf{Acc.$\uparrow$}  & \bf{AUROC$\uparrow$}   & \bf{AUPR$\uparrow$} &\bf{FPR95$\downarrow$}   &  \bf{Acc.$\uparrow$}  \\ \midrule
  \multirow{7}{*}{\makecell[l]{UCF101 (InD) \\ HMDB51 (OoD)}}
&OpenMax~\cite{bendale2016towards}     & \textit{82.28}    & \textit{54.59}   & \textit{50.69} & \textit{73.92} & 90.89  & 73.16   &38.77               & 95.32        \\
&MC Dropout~\cite{gal2016dropout}     & 75.75    & 41.21   & 54.78 & 73.63                              & 88.23   & 67.62   &38.12            & 95.06           \\
&BNN SVI~\cite{krishnan2018bar}      & 80.10    & 53.43  & 52.33  & 71.51                                & \textit{91.81}   & \textit{79.65}  &31.43             & 94.71           \\
&SoftMax~\cite{hendrycks2016baseline}    & 79.72    & 52.13  & 53.22  &\textit{73.92}                                   & 91.75   & 77.69  &\textit{28.60}              & 95.03           \\
&RPL~\cite{chen2020learning}        & 79.67    & 51.85  &56.40  & 71.46                                  & 90.53   & 77.86  &37.09               &\textit{95.59}           \\
&DEAR~\cite{bao2021evidential}      & 80.00         &49.23         &53.28 & 71.33                                    &84.16   & 75.54  &89.40                   & 94.48           \\
&PSL(ours)   & \bf{86.43}    & \bf{65.54}   & \bf{41.67} & \bf{76.53}                                  & \bf{94.05}   & \bf{86.55}   &\bf{23.18}   & \bf{95.62}           \\
&$\mathbf{\Delta}$ &\bf \textcolor{themeblue}{(+4.15)} &\bf \textcolor{themeblue}{(+10.95)} &\bf \textcolor{themeblue}{(-9.02)} &\bf \textcolor{themeblue}{(+2.61)} &\bf \textcolor{themeblue}{(+2.24)} &\bf \textcolor{themeblue}{(+6.90)} &\bf \textcolor{themeblue}{(-5.42)} &\bf \textcolor{themeblue}{(+0.03)}\\
\midrule
\multirow{7}{*}{\makecell[l]{UCF101 (InD) \\ MiTv2 (OoD)}} 
&OpenMax~\cite{bendale2016towards}       & \textit{84.43}   & \textit{76.69} &\textit{47.74}  & \textit{73.92}                 & \textit{93.34}   & 88.14   & \textit{28.95} & 95.32           \\
&MC Dropout~\cite{gal2016dropout}                                   & 75.66    & 62.20   & 51.57 & 73.63            & 88.71   & 83.36   & 39.46 & 95.06           \\
&BNN SVI~\cite{krishnan2018bar}                                      & 79.48    & 71.73   & 52.52 & 71.51            & 91.86   & \textit{90.12}   & 36.21 & 94.71           \\
&SoftMax~\cite{hendrycks2016baseline}                                      & 80.55    & 73.17   & 50.49 & \textit{73.92}            & 91.95   & 89.16   & 32.00 & 95.03           \\
&RPL~\cite{chen2020learning}                                          & 80.21    & 72.04   & 52.83 & 71.46            & 90.64   & 88.79   & 38.43 & \textit{95.59}           \\
&DEAR~\cite{bao2021evidential}                                         & 79.00          & 67.10        &52.44   &  71.33             & 86.04   & 87.38   & 87.40 & 94.48           \\
&PSL(ours)            & \bf{86.53}   & \bf{79.95}  & \bf{40.99} &  \bf{76.53}                & \bf{95.75}        & \bf{94.96}        & \bf{18.96} & \bf{95.62}                \\ 
&$\mathbf{\Delta}$ &\bf \textcolor{themeblue}{(+2.10)} &\bf \textcolor{themeblue}{(+3.26)} &\bf \textcolor{themeblue}{(-6.75)} &\bf \textcolor{themeblue}{(+2.61)} &\bf \textcolor{themeblue}{(+2.41)} &\bf \textcolor{themeblue}{(+4.84)} &\bf \textcolor{themeblue}{(-9.99)} &\bf \textcolor{themeblue}{(+0.03)} \\
\bottomrule[1pt]
\end{tabular}
\vspace{-0.3cm}
\caption{Comparison with state-of-the-art methods on {\bf HMDB51 and MiTv2 (OoD)} using TSM backbone. Acc. refers to closed-set accuracy. AUROC, AUPR and FPR95 are open-set metrics. Best results are in {\bf bold} and second best results in \textit{italic}. The gap between best and second best is in {\bf \textcolor{themeblue}{blue}}. DEAR and our methods contain video-specific operation.}
\label{tab:bench_hmdb}
\vspace{-0.6cm}
\end{table*}

\begin{figure}[t]
\centering
\includegraphics[width=0.48\textwidth]{docu/figs/MiT_2.pdf}
\vspace{-8mm}
\caption{
The uncertainty distribution of InD and OoD samples of (a) Softmax, (b) DEAR, (c) BNN SVI and (d) our PSL method.
}
\label{fig:uncer_method}
\end{figure}

\noindent \textbf{Implementation details.} For Kinetics400 (K400)~\cite{i3d} pretrained model, our implementation setting is the same with~\cite{bao2021evidential}. The base learning rate is 0.001 and step-wisely decayed every 20 epochs with total of 50 epochs. We argue that as K400 is extremely large, the K400 pretrained model may already have seen the OoD data used in inference, so we conduct experiments from scratch (no ImageNet pretrained) to ensure that OoD data is absolutely unavailable during training. We use the LARS optimizer~\cite{you2017large} and set the base learning rate and momentum as 0.6 and 0.9 with total of 400 epochs. The experiments are conducted on TSM~\cite{lin2019tsm}, I3D~\cite{i3d} and SlowFast~\cite{feichtenhofer2019slowfast}. The batch size for all methods is 256. More details are in Appendix C.

\subsection{Evaluation Results}
\label{sec:res}

\noindent \textbf{Comparison with state-of-the-art.} We report the results on HMDB51 (OoD) and MiT-v2 (OoD) in Table~\ref{tab:bench_hmdb} using TSM backbone~\cite{lin2019tsm}. The evaluation results of other backbones including I3D and SlowFast are in the Appendix D. We can see that for w/ or w/o K400 pretrain, our PSL method has significantly better open-set and closed-set performance than all baselines. The uncertainty distribution of InD and OoD samples are depicted in Fig.~\ref{fig:uncer_method} for MiT-v2 (OoD) with K400 pretrained. Three baseline methods have a clear over confidence problem, \textit{i.e.}, the far left column is extremely high (red circles in Fig.~\ref{fig:uncer_method}), which means a large number of OoD samples have almost 0 uncertainty, while our method significantly alleviates this problem through the distinct representation of OoD samples, illustrated in Fig.~\ref{fig:tsne}. Besides, we can find that the open-set performance w/ K400 pretrain is higher than w/o pretrain for almost all methods in Table~\ref{tab:bench_hmdb} and \cref{fig:1} (a), which can testify the importance of richer semantic representation for OSAR.

\begin{figure}[t]
\centering
\includegraphics[width=0.48\textwidth]{docu/figs/tsne.pdf}
\vspace{-10mm}
\caption{
Feature representation visualization of cross-entropy and our PSL method. OoD samples are in black and InD samples are in other colors. In the red, blue and green circles, it is clear that OoD samples distribute at the edge of InD samples in our PSL, while greatly overlap with each other in the cross-entropy method.
}
\label{fig:tsne}
\end{figure}

\begin{table*}[t!]
\tablestyle{6pt}{0.3}
\centering
\begin{tabular}{lcccccccccccc}
\toprule[1pt]
&&&&& \multicolumn{2}{c}{\bf{InD}} &\multicolumn{2}{c}{\bf{OoD}} \\ \cmidrule{6-9}
                        & \bf{$s$} & \bf{$Q_{ns}$} & \bf{$Q_{sc}$} & $Q_{shuf}$ &\bf{Mean} & \bf{Variance} &\bf{Mean} & \bf{Variance} & \bf{AUROC$\uparrow$} & \bf{AUPR$\uparrow$} &\bf{FPR95$\downarrow$}  & \bf{Acc.$\uparrow$} \\ \midrule
$\mathcal{L}_{PL}$                   & \XSolidBrush  & \XSolidBrush      & \XSolidBrush      & \XSolidBrush        &0.81 &0.0015 &0.63 &0.0029 & 80.95 & 52.79 & 52.51 & 72.36          \\ \midrule
$\mathcal{L}_{PSL}$ & \Checkmark  & \XSolidBrush      &\XSolidBrush       & \XSolidBrush        & 0.79     &0.0016  & 0.62 & 0.0028 & 81.79 & 54.16 & 52.33 & 72.33          \\ \midrule
 \multirow{3}{*}{$\mathcal{L}_{PSL}^{CT}$}                       & \Checkmark  & \Checkmark      & \XSolidBrush      & \XSolidBrush        &0.71 &0.0022 &0.61 &0.0036 & 82.60 & 57.36 & 50.03 & 72.17          \\ 
                        & \Checkmark  & \Checkmark      & \Checkmark      & \XSolidBrush        & 0.71 &0.0023 &0.49 &0.0035 & 83.42 & 59.05 & 51.32 & 72.28          \\
                        & \Checkmark  & \Checkmark      & \Checkmark      & \Checkmark        & 0.74 &0.0016 &0.63 &0.0029 & 86.43 & 65.58 & 41.75 & 77.19     \\ 
                        \bottomrule[1pt]  
\end{tabular}
\vspace{-0.3cm}
\caption{Abaltion results of different components in $\mathcal{L}_{PSL}^{CT}$.}
\label{tab:abla}
\vspace{-0.5cm}
\end{table*}


\noindent \textbf{Comparison with metric learning methods.} Our method concentrates on the feature representation aspect for the OSAR problem, so we also implement several well-known metric learning methods and show the result in Table~\ref{tab:metric_learning}. The evaluation is conducted using TSM model and OoD dataset is HMDB51. We do not use video shuffling in our method for fair comparison. We can see that our method still achieves the best open-set performance. The most important difference between our method and all other metric learning methods is that they aim to push the features of one class as tight as possible like C.E., while our method aims to keep the feature variance within a class to retain IS information. We calculate the mean similarity between the sample feature and the corresponding class center. The mean similarity ranges from 0.77 to 0.82 for other metric learning methods, while mean similarity is 0.71 ($s=0.8$) and 0.6 ($s=0.6$) for our PSL. So our method has looser feature distribution within a class, as shown in \cref{fig:tsne}.

\begin{table}[t!]
\tablestyle{5pt}{0.8}
\centering
\begin{tabular}{lcccccccccc}
\toprule[1pt]
& \bf{AUROC$\uparrow$} & \bf{AUPR$\uparrow$} &\bf{FPR95$\downarrow$}  & \bf{Acc.$\uparrow$} \\ \midrule
SoftMax    &80.95	&52.79	&52.51   &72.36	        \\
 Triplet \cite{triplet} &81.02 &54.75	&53.88 &75.50\\
Normface \cite{normface} &80.99	&54.90	&53.19 &73.34            \\
Circle  \cite{circle} 	&78.76	&51.65	&55.27 &72.15 	\\
Arcface \cite{arcface} &81.23	&55.03	&53.67 &\bf{75.95}	\\
LSoftMax \cite{lsoftmax} &80.87 &54.01	&52.29 &73.05 \\ \midrule
PSL($s=0.8$)   & \bf{83.42} & \bf{59.05} & \bf{51.32} & 72.28           \\
PSL($s=0.6$)    &82.75 & 58.57 & 52.27 & 73.26           \\
                        \bottomrule[1pt]  
\end{tabular}
\vspace{-0.3cm}
\caption{Comparison with different metric learning methods.}
\label{tab:metric_learning}
\vspace{-0.4cm}
\end{table}


\subsection{Ablation Study}
\label{sec:abl_stu}

\noindent \textbf{Contrastive terms in $\mathcal{L}_{PSL}^{CT}$ for IS information.} The intuition of PSL is to keep the intra-class variance to retain the IS information which is helpful for OSAR. We expect that the representation $z$ within a class has a similarity $s<1$ with the prototype $k_i$, so each sample can keep its own IS information. However, we find that the loss $\mathcal{L}_{PSL}$ may lead the network to find the trivial representation of samples $z$ which is similar to using loss $\mathcal{L}_{PL}$, where only $k_i$ shifts and $z$ does not. We calculate the mean of similarity $sim(z, \bar z_i)$, where $\bar z_i$ denotes the mean representation of all samples in the same class $i$, and the mean of similarity with the corresponding prototype $sim(z, k_i)$, as well as the feature variance in all dimensions. Fig.~\ref{fig:abl_contra} (a) and (b) show that with the hyper-parameter $s$ decreasing, the $sim(z, k_i)$ decreases as expected by $\mathcal{L}_{PSL}$ (green curves), but the $sim(z, \bar z_i)$ and variance stay unchanged (blue curves), meaning that the representation of samples are still similar with using $\mathcal{L}_{PL}$, and only the prototypes are pushed away by the sample representations. In contrast, with CT in $\mathcal{L}_{PSL}^{CT}$, the $sim(z, \bar z_i)$ decreases and variance increases with $s$ decreases (red curves), indicating that CT is significantly effective to keep the intra-class variance.

\begin{figure}[t]
    \centering
    \includegraphics[width=0.99\linewidth]{docu/figs/abl_contra_term.pdf}
    \vspace{-0.5cm}
    \caption{Mean similarity and variance analysis for CT terms.}
    \label{fig:abl_contra}
    \vspace{-0.3cm}
\end{figure}


To individually study the effectiveness of $Q_{ns}$ and $Q_{sc}$ in $\mathcal{L}_{PSL}^{CT}$, we provide the ablation results in Table~\ref{tab:abla}. For OoD samples, we calculate the similarity with the mean representation of its predicted class. Table~\ref{tab:abla} shows that using $Q_{ns}$ alone can significantly increase the intra-class variance for both InD and OoD samples, meaning the pushing effect of representations in other classes can implicitly help retain the IS information. On top of that, $Q_{sc}$ can further learn more IS information that is helpful to distinguish OoD samples, as the mean similarity of InD samples stay unchanged, but OoD samples are smaller which means OoD samples are far away from InD samples.

\noindent \textbf{Shuffled videos for CS information.} \cref{tab:abla} shows that $Q_{shuf}$ can improve both closed-set and open-set performance, which proves introducing shuffled videos in PSL can enlarge CS information. Smaller intra-class variance brought by $Q_{shuf}$ testify Proposition~\ref{prop:cs_is} that more CS information means more similar features within the same class.

We draw the uncertainty of all classes in HMDB51, as shown in Fig.~\ref{fig:ood_samples}. Note that some classes in HMDB51 are actually InD as they appear in the UCF101, like the class 3 \emph{golf} and 4 \emph{shoot bow} in Fig.~\ref{fig:ood_samples}. We find that in C.E. some OoD classes have extremely low uncertainty, such as class 1 \emph{chew} and 2 \emph{smile}, because they are spatially similar to some InD classes like \emph{ApplyEyeMakeup} and \emph{ApplyLipstick} in Fig.~\ref{fig:ood_samples} (a). Comparing (b) and (c) shows that our PSL can increase the average uncertainty of OoD classes (higher yellow points), and some OoD classes which are similar to InD classes like 1 and 2 have much higher uncertainty in our PSL method. After shuffled samples are involved, some InD classes whose uncertainty are increased in (c) like 3 and 4 have lower uncertainty in (d), and the uncertainty of some OoD classes sharing similar appearance with InD classes like class 1 is further improved.

$Q_{sp}$ in Eq.~\ref{eq:PSL_CT} contains $Q_{shuf}$ and $Q_{sc}$, so we analyze whether should we assign the same $s$ for the shuffled video $Q_{shuf}$ and other videos in the same class $Q_{sc}$. \cref{tab:s_q_shu} shows that the same $s$ have good enough performance. So we set the same $s$ for $Q_{shuf}$ and $Q_{sc}$ in the default setting to reduce the number of hyper-parameters.

\subsection{Discussion}
\label{sec:discussion}

\begin{figure}[t]
    \centering
    \includegraphics[width=0.99\linewidth]{docu/figs/fig_ood_samples.pdf}
    \vspace{-0.5cm}
    \caption{(a) \emph{chew} and \emph{smile} are OoD samples from HMDB51, and \emph{ApplyEyeMakeup} and \emph{ApplyLipstick} are InD samples from UCF101. (b-d) Uncertainty distribution of each class in HMDB51. Class 1: \emph{chew}, 2: \emph{smile}, 3: \emph{golf}, 4: \emph{shoot bow}. Classes 1 and 2 are OoD while 3 and 4 are InD.}
    \label{fig:ood_samples}
    \vspace{-0.4cm}
\end{figure}


\begin{table}[t]
\tablestyle{4pt}{0.8}
\centering
\begin{tabular}{ccccccc}
\toprule[1pt]
 $s(Q_{shuf})$ &$s(Q_{sc})$  &\bf{AUROC$\uparrow$} & \bf{AUPR$\uparrow$} &\bf{FPR95$\downarrow$}  & \bf{Acc.$\uparrow$} \\ \midrule
0.7 &\multirow{4}{*}{0.7} & 85.25	&63.91	&48.34 &76.98\\
0.5	& &86.03	&64.36	&43.70 & 76.53\\
0.3	& &83.80	&60.42	&48.76 & 75.50\\
0 & &79.54&50.59&54.43& 72.59\\ \midrule
0.8 &0.8 &86.43	&65.58	&41.75 &76.53\\
0.9 &0.9 &83.12	&57.04	&46.84 &73.31\\
1 &1 &82.04	&53.82&51.82 &72.89	\\
\bottomrule[1pt]
\end{tabular}
\vspace{-0.3cm}
\caption{Ablation study of similarity $s$ for $Q_{shuf}$ and $Q_{sc}$.}
\label{tab:s_q_shu}
\vspace{-0.3cm}
\end{table}

\noindent \textbf{Both CS and IS information are useful.} We provide the closed-set and open-set performance under different hyper-parameter $s$ and feature dimension $d$ in \cref{fig:dis_s}. (a) shows that $s=0.8$ has better open-set performance than $s=1$ and has comparable closed-set accuracy, which illustrates that retaining the IS information which is eliminated by C.E. ($s=1$) is beneficial. When $s<0.8$, the NN cannot learn enough CS information, so both closed-set and open-set performance drops. Therefore, a proper mixture of CS and IS information is ideal. (b) shows that when $d$ grows from 4 to 16, more CS information is contained so that both closed-set and open-set performance improves. When $d$ grows from 16 to 128, the feature does not include more CS information as closed-set accuracy is comparable. However, open-set performance keeps increasing which means more IS information is contained based on more feature dimensions. This interesting experiment shows that enough information for closed-set recognition is not enough for open-set recognition because IS information is not related to the closed-set task but useful for the open-set task.

\begin{figure}[t]
    \centering
    \includegraphics[width=0.99\linewidth]{docu/figs/dis_s.pdf}
    \vspace{-0.5cm}
    \caption{Ablation study of similarity $s$ and feature dimension $d$.}
    \label{fig:dis_s}
    \vspace{-0.4cm}
\end{figure}
\begin{table}[t]
\tablestyle{2pt}{0.8}
\centering
\begin{tabular}{lcccccccccccc}
\toprule[1pt]
   \bf{Epoch}                      &\bf{Mean} & \bf{Variance} & \bf{AUROC$\uparrow$} & \bf{Acc-Test.$\uparrow$} & \bf{Acc-Train.$\uparrow$}\\ \midrule
200 &0.577 &3.3e-3 &75.08	&68.39 &99.85\\
400 &0.602 &3.1e-3 &82.92	&73.26 &100\\
800 &0.613 &3.0e-3 &82.54	&73.29 &100\\
                        \bottomrule[1pt]  
\end{tabular}
\vspace{-0.3cm}
\caption{Training process analysis when $s=0.6$ w/o $Q_{shuf}$.}
\label{tab:continual_train}
\vspace{-0.3cm}
\end{table}


\noindent \textbf{Feature variance and open-set performance analysis.} \cref{fig:dis_s} (a) shows that when features get looser ($s=1-0.8$), the open-set performance is improved, but if features get continually looser ($s=0.8-0.1$), the open-set performance drops. So there is no strict relation between the feature variance and open-set performance. One may argue that continual training can benefit the open-set performance~\cite{vaze2021open}, which is alongside with smaller feature variance~\cite{han2021neural}. We show that the benefit of continual training comes from better closed-set performance, not tighter features. \cref{tab:continual_train} shows that when we train the model from 200 to 400 epochs, the closed-set accuracy is higher, and feature is tighter (larger mean similarity and smaller variance), and the open-set performance is better. But from epoch 400 to 800 we find the model is already overfitted to the training set, as the accuracy of test set remains unchanged. So although the features get tighter in the 800 epoch, both the closed-set and open-set performance remain same.

\section{Conclusion}
We analyze the OSAR problem from the information perspective, and show that cross-entropy tends to eliminate IS information and cannot fully learns CS information which are both useful for the open-set task. So we propose PSL to retain IS information and introduce shuffle videos into PSL to enlarge CS information. Comprehensive experiments demonstrate the effectiveness of our PSL and the importance of IS and CS information in the OSAR task.

\noindent {\bf Acknowledgements} This work is supported by Alibaba Group through Alibaba Research Intern Program.

\appendix
\onecolumn
\begin{center}
\Large
\textbf{Appendices}
\end{center}
\section{Datasets}
We follow the datasets setting in~\cite{bao2021evidential}. The training InD dataset is UCF101, which contains 101 classes with 9537 training samples and 3783 test samples. The OoD datasets for open-set evaluation are HMDB51 and MiT-v2. We use the test sets of them which contain 1530 samples and 30500 samples respectively. For UCF101 and HMDB51, we follow the MMAction~\cite{mmaction2019} to use the split 1 for training and evaluation, which is the same with~\cite{bao2021evidential}. Note that in~\cite{bao2021evidential}, they find some classes in HMDB51 overlap with those in UCF101 but they do not clean them. We remove the overlapping classes in UCF101 and HMDB51 so that OoD data does not contain any samples of InD classes. The classes we remove in HMDB51 and the corresponding same classes in UCF101 are in Table~\ref{tab:dataset}.


\begin{table*}[h!]
\tablestyle{3pt}{1}
% \setlength{\tabcolsep}{#1}\renewcommand{\arraystretch}{#2}\centering\small
\centering
\begin{tabular}{lcccc}
\toprule[1pt]
HMDB51 &35, Shoot bow &29, Push up &15, Golf  &26, Pull up \\
UCF101 &2, Archery &71, PushUps &32, GolfSwing &69, PullUps \\ \midrule
HMDB51 &30, Ride bike &34, Shoot ball &43, Swing baseball &31, Ride horse \\
UCF101 &10, Biking &7, Basketball &6, BaseballPitch &41, HorseRiding \\
                        \bottomrule[1pt]  
\end{tabular}
\vspace{-0.3cm}
\caption{Overlapping classes in HMDB51 and UCF101.}
\label{tab:dataset}
\end{table*}
\section{Evaluation protocols}

Based on codes provided by~\cite{bao2021evidential}, we find that their evaluation metrics including Open maF1 and AUORC are both calculated under a specific certain threshold, \textit{i.e.}, a sample whose uncertainty is larger than the threshold will be considered as an OoD sample. The threshold is determined by top 5\% uncertainty in the training set. This is contradictory with the classical metrics in the open-set image recognition, in which common metrics including AUROC and AUPR~\cite{hendrycks2016baseline,hendrycks2018deep} both consider all thresholds. Each point on the ROC and PR curve is based on one specific threshold, and the area under ROC and PR curve is regarded as the comprehensive result of all thresholds. After discussing with authors in~\cite{bao2021evidential}, they admit that the AUROC, AUPR and FPR95 which are served as the classical metrics in the open-set image recognition are more suitable for the OSAR problem. So they modify the corresponding code and we provide the correct results in the Table 1 in our paper. We provide a comparison between the result of considering only one threshold and all thresholds in Table~\ref{tab:bench_hmdb_o}. The results show that no matter for only considering one threshold or all thresholds, our PSL method can both outperform all methods.
\begin{table*}[h!]
\centering
\tablestyle{3pt}{0.8}
\begin{tabular}{llcccccccc}
\toprule[1pt]
& & \multicolumn{4}{c}{\bf{One threshold}~\cite{bao2021evidential}} & \multicolumn{4}{c}{\bf{All thresholds (ours)}} \\ \cmidrule{3-10}
   \bf{Models} &\bf{Methods}                     & \bf{AUROC$\uparrow$}    & \bf{AUPR$\uparrow$}  &\bf{FPR95$\downarrow$}  & \bf{Acc.$\uparrow$}  & \bf{AUROC$\uparrow$}   & \bf{AUPR$\uparrow$} &\bf{FPR95$\downarrow$}   &  \bf{Acc.$\uparrow$}  \\ \midrule
  \multirow{7}{*}{TSM}
&OpenMax     & \underline{84.18}  &  \underline{76.52}  &  100   & 95.32      &90.89  & 73.16 &38.77   &95.32  \\
&MC Dropout     & 78.50  &  71.11  &   37.80 &95.06                          & 88.23   & 67.62   &38.12            & 95.06           \\
&BNN SVI      &  77.77  &  71.00  &   41.13   &       94.71                              & \underline{91.81}   & \underline{79.65}  &31.43             & 94.71           \\
&SoftMax    & 82.77  &  74.33  &  \underline{29.58}   &       95.03                                    & 91.75   & 77.69  &\underline{28.60}              & 95.03           \\
&RPL        & 77.75  &  70.93  &  40.87   &       \underline{95.59}                                 & 90.53   & 77.86  &37.09               &\underline{95.59}           \\
&DEAR      & 82.73  &  74.79  &  100   &       94.48                                  &84.16   & 75.54  &89.40                   & 94.48           \\
&PSL(ours)   & \bf{87.53}  &  \bf{79.92}  &   \bf{14.98}   &       \bf{95.62}                                  & \bf{94.05}   & \bf{86.55}   &\bf{23.18}   & \bf{95.62}           \\
&$\mathbf{\Delta}$ &\bf \textcolor{themeblue}{(+3.35)} &\bf \textcolor{themeblue}{(+3.10)} &\bf \textcolor{themeblue}{(-14.60)} &\bf \textcolor{themeblue}{(+0.03)} &\bf \textcolor{themeblue}{(+2.24)} &\bf \textcolor{themeblue}{(+6.90)} &\bf \textcolor{themeblue}{(-5.42)} &\bf \textcolor{themeblue}{(+0.03)}\\
\bottomrule[1pt]
\end{tabular}
\vspace{-0.3cm}
\caption{Comparison of different evaluation metrics on HMDB51 (OoD) with K400 pretrained.}
\label{tab:bench_hmdb_o}
\end{table*}

When we use MiT-v2 as the OoD dataset, we find the imbalance problem, which is also mentioned in~\cite{bao2021evidential}. The MiT-v2 test set contains 30500 samples while UCF101 test set only contains 3783 samples. This will cause the AUPR to be close to 100\% if we regard all samples in MiT-v2 as OoD samples during evaluation. Therefore, we divide the MiT-v2 test set into 10 splits, and evaluate the open-set metrics for 10 times and calculate the mean as the final result. A comparison between the results of evaluating 10 times and 1 time is shown in Table~\ref{tab:bench_mit_10}. The results illustrate that when we use all samples in MiT-v2 for open-set evaluation, the AUPR will be close to 100\%, although our method still achieves the best performance. The AUROC and FPR95 are not sensitive to the OoD sample numbers.

\begin{table*}[h!]
\centering
\tablestyle{3pt}{0.8}
\begin{tabular}{llcccccccc}
\toprule[1pt]
& & \multicolumn{4}{c}{\bf{1 time}} & \multicolumn{4}{c}{\bf{10 times}} \\ \cmidrule{3-10}
   \bf{Models} &\bf{Methods}                     & \bf{AUROC$\uparrow$}    & \bf{AUPR$\uparrow$}  &\bf{FPR95$\downarrow$}  & \bf{Acc.$\uparrow$}  & \bf{AUROC$\uparrow$}   & \bf{AUPR$\uparrow$} &\bf{FPR95$\downarrow$}   &  \bf{Acc.$\uparrow$}  \\ \midrule
  \multirow{7}{*}{TSM}
&OpenMax       & \underline{93.34}  &  98.46  &   \underline{29.20}   & 95.32                 & \underline{93.34}  &  88.14  &   \underline{28.95}   & 95.32           \\
&MC Dropout                                   & 88.71   & 97.92  & 39.46 & 95.06            & 88.71   & 83.36   & 39.46 & 95.06           \\
&BNN SVI                                      & 91.86   & \underline{98.75}   & 36.21 & 94.71            & 91.86   & \underline{90.12}   & 36.21 & 94.71           \\
&SoftMax                                      & 91.95   & 98.68   & 32.00 & 95.03            & 91.95   & 89.16   & 32.00 & 95.03           \\
&RPL                                          & 90.64   & 98.57   & 38.43 & \underline{95.59}            & 90.64   & 88.79   & 38.43 & \underline{95.59}           \\
&DEAR                                         & 86.04   & 98.08   & 87.66 & 94.48             & 86.04   & 87.38   & 87.40 & 94.48           \\
&PSL(ours)            & \bf{95.75}        & \bf{99.39}        & \bf{19.00} & \bf{95.90}                & \bf{95.75}        & \bf{94.96}        & \bf{18.96} & \bf{95.90}                \\ 
&$\mathbf{\Delta}$ &\bf \textcolor{themeblue}{(+2.41)} &\bf \textcolor{themeblue}{(+0.64)} &\bf \textcolor{themeblue}{(-10.20)} &\bf \textcolor{themeblue}{(+0.31)} &\bf \textcolor{themeblue}{(+2.41)} &\bf \textcolor{themeblue}{(+4.84)} &\bf \textcolor{themeblue}{(-9.99)} &\bf \textcolor{themeblue}{(+0.31)} \\
\bottomrule[1pt]
\end{tabular}
\vspace{-0.3cm}
\caption{Comparison of different evaluation methods on MiT-v2 (OoD) with K400 pretrained.}
\label{tab:bench_mit_10}
\end{table*}

\section{Implementation details}

When we use K400 pretrained model, the only method we need to fulfill is our PSL method, and we follow~\cite{bao2021evidential} to set the base learning rate as 0.001 and step-wisely decayed every 20 epochs with total 50 epochs. When we train the model from scratch, we need to conduct experiments on all methods in our Table 1. For our PSL method, we use the LARS optimizer~\cite{you2017large} and set the base learning rate and momentum as 0.6 and 0.9 with totally 400 epochs. The reason we use this strategy is inspired by the contrastive learning SimCLR~\cite{chen2020simple}. For other baselines, we find the above learning rate strategy cannot achieve good enough closed-set performance, and we find that setting the base learning rate as 0.05 and step-wisely decayed every 160 epochs with totally 400 epochs can achieve comparable closed-set performance. The batch size for all methods is 256, and we use 16 NVIDIA V100 GPUs to train the model.

\section{OSAR performance under I3D and SlowFast backbone}

We provide the OSAR results under TSM~\cite{lin2019tsm} backbone in Table 1 of the paper. Here, we further provide the OSAR results under I3D~\cite{i3d} and SlowFast~\cite{feichtenhofer2019slowfast} backbones in Table~\ref{tab:bench_i3d} and~\ref{tab:bench_slf}. We can see our PSL method still achieves state-of-the-art performance under these two backbones. The performance gain under Slowfast when MiTv2 is OoD dataset is marginal, as baselines already have high performance.

\begin{table*}[h!]
\centering
\tablestyle{3pt}{0.8}
\begin{tabular}{llcccccccc}
\toprule[1pt]
& & \multicolumn{4}{c}{\bf{w/o K400 Pretrain}} & \multicolumn{4}{c}{\bf{w/ K400 Pretrain}} \\ \cmidrule{3-10}
   \bf{Datasets} &\bf{Methods}                     & \bf{AUROC$\uparrow$}    & \bf{AUPR$\uparrow$}  &\bf{FPR95$\downarrow$}  & \bf{Acc.$\uparrow$}  & \bf{AUROC$\uparrow$}   & \bf{AUPR$\uparrow$} &\bf{FPR95$\downarrow$}   &  \bf{Acc.$\uparrow$}  \\ \midrule
  \multirow{7}{*}{\makecell[l]{UCF101 \\ HMDB51 }}
&OpenMax  &\underline{83.78}&\underline{54.65}&\underline{47.60}&\underline{74.42} & 92.03 & 77.72 &41.02 & \underline{95.01}\\ 
&MC Dropout &75.85&40.04&50.34&74.39 &91.66 &78.87 &33.60 &94.11\\
&BNN SVI &81.53&53.62&49.18&73.15 &91.57 &78.65 &34.60 &93.89\\
&SoftMax &81.24&54.21&48.20&\underline{74.42} &91.28 &79.73 &34.18 &94.11\\
&RPL &79.80&52.09&54.07&71.62 &\underline{92.49} &\underline{81.72} &\underline{28.89} &94.26\\
&DEAR &78.91&54.14&81.96&\underline{74.42} &89.80 &80.86 &75.63 &93.89\\
&PSL(ours) &\bf{86.88}&\bf{65.63}&\bf{39.85} &\bf{78.85} &\bf{93.62} &\bf{85.54} &\bf{28.38} &\bf{95.46}\\
&$\mathbf{\Delta}$ &\bf \textcolor{themeblue}{(+3.10)} &\bf \textcolor{themeblue}{(+10.98)} &\bf \textcolor{themeblue}{(-7.75)} &\bf \textcolor{themeblue}{(+4.43)} &\bf \textcolor{themeblue}{(+1.13)} &\bf \textcolor{themeblue}{(+3.82)} &\bf \textcolor{themeblue}{(-0.51)} &\bf \textcolor{themeblue}{(+0.45)} \\
\midrule
\multirow{7}{*}{\makecell[l]{UCF101 \\ MiTv2 }}
&OpenMax  &\underline{86.33}&\underline{77.49}&\underline{44.40}&\underline{74.63} &93.29 &90.17 &29.84 & \underline{94.90}\\ 
&MC Dropout &76.61&62.32&48.43&74.24 &93.53 &90.97 &\underline{25.21} &94.11\\
&BNN SVI &83.13&76.20&48.63&73.15  &93.52 &91.24 &25.34 &93.89\\
&SoftMax &82.58&74.91&46.39&\underline{74.63} &92.62 &90.87 &30.55 &94.11\\
&RPL &81.47&73.98&49.62&71.89 &\underline{93.69} &\underline{92.04} &25.97 &94.26\\
&DEAR &81.48&77.03&77.58&74.42 &90.88 &90.55 &60.28 &93.89\\
&PSL(ours) &\bf{88.88}&\bf{83.30}&\bf{34.91}&\bf{78.69} &\bf{95.70} &\bf{95.06} &\bf{20.03} &\bf{95.51}\\
&$\mathbf{\Delta}$ &\bf \textcolor{themeblue}{(+2.55)} &\bf \textcolor{themeblue}{(+5.81)} &\bf \textcolor{themeblue}{(-9.49)} &\bf \textcolor{themeblue}{(+4.06)} &\bf \textcolor{themeblue}{(+2.01)} &\bf \textcolor{themeblue}{(+3.02)} &\bf \textcolor{themeblue}{(-5.18)} &\bf \textcolor{themeblue}{(+1.25)} \\
\bottomrule[1pt]
\end{tabular}
\vspace{-0.3cm}
\caption{OSAR performance under I3D backbone.}
\label{tab:bench_i3d}
\end{table*}

\begin{table*}[h!]
\centering
\tablestyle{3pt}{0.8}
\begin{tabular}{llcccccccc}
\toprule[1pt]
& & \multicolumn{4}{c}{\bf{w/o K400 Pretrain}} & \multicolumn{4}{c}{\bf{w/ K400 Pretrain}} \\ \cmidrule{3-10}
   \bf{Datasets} &\bf{Methods}                     & \bf{AUROC$\uparrow$}    & \bf{AUPR$\uparrow$}  &\bf{FPR95$\downarrow$}  & \bf{Acc.$\uparrow$}  & \bf{AUROC$\uparrow$}   & \bf{AUPR$\uparrow$} &\bf{FPR95$\downarrow$}   &  \bf{Acc.$\uparrow$}  \\ \midrule
  \multirow{7}{*}{\makecell[l]{UCF101 \\ HMDB51 }}
&OpenMax  &80.67	&
50.49	&52.46 &75.40 & 92.49  &  78.27 &  35.65   & 96.30\\ 
&MC Dropout &76.10	&41.37	&50.82 &75.16 & 91.83  &  77.71  &   29.82 &\underline{96.70}\\
&BNN SVI &\underline{81.66}	&\underline{56.72}	&49.66 &76.58 &  93.34  &  85.57  &   27.89   & 96.56\\
&SoftMax &79.15	&48.54	&\underline{48.79} &75.63	& \underline{93.82}  &  85.56  &  24.74   &       96.70\\
&RPL &81.35	&54.65	&51.64 &\underline{78.36} &93.81  &  85.41  &   \underline{24.06}   &       \bf{96.93}\\
&DEAR &78.00 &49.38	&68.49 &76.21	& 92.28  &  \underline{87.09}  &   62.99   &       96.48\\
&PSL(ours) &\bf{86.20}	&\bf{64.65}	&\bf{42.48} &\bf{79.40}	&\bf{95.24}  &  \bf{89.76}  &   \bf{18.72}   & 96.52\\
&$\mathbf{\Delta}$ &\bf \textcolor{themeblue}{(+4.54)} &\bf \textcolor{themeblue}{(+7.93)} &\bf \textcolor{themeblue}{(-6.31)} &\bf \textcolor{themeblue}{(+1.04)} &\bf \textcolor{themeblue}{(+1.42)} &\bf \textcolor{themeblue}{(+2.67)} &\bf \textcolor{themeblue}{(-5.34)} &\bf \textcolor{themeblue}{(-0.49)} \\
\midrule
\multirow{7}{*}{\makecell[l]{UCF101 \\ MiTv2 }}
&OpenMax  &79.60 &70.05	&51.08 &75.63 &94.34 &89.90 &25.42 &96.30\\ 
&MC Dropout &75.88	&63.12	&51.40 &75.63 &93.43  &  90.43  &   24.52   &       \underline{96.70}\\
&BNN SVI &\underline{82.89}	&
\underline{76.13}	&\underline{46.88} &76.58 &93.53  &  92.34  &   28.81   &       96.56\\
&SoftMax &51.08 &75.63	&79.60	&70.05	&94.67  &  93.34  &   22.14   &       96.70\\
&RPL &81.42	&73.07	&49.13 &\underline{78.36}&\underline{94.76}  &  \underline{93.39}  &   \underline{21.99}   &       \bf{96.93}\\
&DEAR &78.21&69.30	&62.02 &76.21&92.60  &  93.09  &   59.98   &       96.48\\
&PSL(ours) &\bf{85.00}&\bf{77.08}	&\bf{43.16}&\bf{79.40}&\bf{96.81}  &  \bf{96.22}  &   \bf{14.52}   &       96.52\\
&$\mathbf{\Delta}$ &\bf \textcolor{themeblue}{(+2.11)} &\bf \textcolor{themeblue}{(+0.95)} &\bf \textcolor{themeblue}{(-3.72)} &\bf \textcolor{themeblue}{(+1.04)} &\bf\textcolor{themeblue}{(+2.05)}
&\bf\textcolor{themeblue}{(+2.83)}
&\bf\textcolor{themeblue}{(-7.47)}
&\bf\textcolor{themeblue}{(-0.49)}\\
\bottomrule[1pt]
\end{tabular}
\vspace{-0.3cm}
\caption{OSAR performance under SlowFast backbone.}
\label{tab:bench_slf}
\vspace{0.5cm}
\end{table*}

\section{Representation analysis through singular value spectrum}

To deeply understand the feature representations learned by our method, we analyze the representation through singular value spectrum. We first compute the covariance matrix $C \in \mathbb{R}^{d \times d}$ of the embedding matrix:
\begin{equation}
    C= \frac{1}{M} \sum_{i=1}^{M}(z_{i}-\bar z)(z_{i}-\bar z)^{T},
\end{equation}
where $z_i$ and $\bar z_i$ denote the feature representation of a sample and mean representation of all samples respectively. $M$ is the total number of samples. Then we conduct singular value decomposition
on the matrix $C=USV^T, S=diag(\sigma^{k})$, and plot the singular values in sorted order and logarithmic scale $log(\sigma^{k})$. We provide the singular value spectrum in Fig.~\ref{fig:single}.
 
PSL has larger singular values than the PL in the larger rank index, illustrating that more information is contained in the not significant dimensions, which is reasonable as PSL keeps the IS information with no direct supervision signal, but these IS information does help for better OSAR performance according to Table 2 in the paper. PSL with shuffled samples $Q_{shuf}$ has larger singular values than PSL in the small rank index, indicating more diverse information is learned in the important dimensions, which are supposed to refer to CS information as CS information is learned by the explicit supervision signal. The closed-set accuracy with $Q_{shuf}$ is higher than without $Q_{shuf}$ in Table 2 further testifies our conclusion. In Tabel 2 we see that the representations of the same class are tighter with more CS information. Therefore, learning the distinct temporal information from shuffled videos can enlarge the open-set task related CS information while PSL can enlarge the IS information, which fulfills the goal to enlarge Eq. 3 for better OSAR performance.

\begin{figure*}[h!]
\vspace{0.3cm}
  \centering
\includegraphics[width=0.99\linewidth]{docu/figs/cls_3.pdf}
  \vspace{-0.3cm}
  \caption{Singular value spectrum on HMDB51 (OoD) under different training conditions (a)-(c) and hyper-parameter $s$ (d). (c) contains the top 20 singular values in (b).}
  \label{fig:single}
  \vspace{0.5cm}
\end{figure*}

\section{Open-set performance \textit{w.r.t.} $s$ with $Q_{shuf}$}

We provide extension results of Table 4 in the paper. The results are based on HMDB51 (OoD) from scratch. $s$ for $Q_{sc}$ is set as 0.7, and we change the value of $s$ for $Q_{shuf}$ in Table~\ref{tab:abla_s_q}. We can see that the performance is optimal when $s$ for $Q_{shuf}$ is 0.8, but the same $s$ with $Q_{sc}$ which is 0.7 also achieves the good performance. So to reduce the number of hyper-parameters, we pick up the same $s$ for $Q_{sc}$ and $Q_{shuf}$ by default. In addition, we can see that the closed-set accuracy is lower when $s=1$ compared to $s=0.8$. This is because we set the similarity between the original video and the shuffled video as 1, which is not reasonable as the temporal information is totally lost in the shuffled video.

\begin{table*}[h]
% \RawFloats
\centering
    \tablestyle{6pt}{1}
\begin{tabular}{ccccc}
\toprule[1pt]
 \bf{$s$} &\bf{AUROC$\uparrow$} & \bf{AUPR$\uparrow$} &\bf{FPR95$\downarrow$}  & \bf{Acc.$\uparrow$} \\ \midrule
1	&82.04	&53.82	&51.82 &72.89	\\
0.9 &83.12	&57.04	&46.84 &73.31\\
0.8 &\bf 86.43	&\bf 65.58	&\bf 41.75 &76.53\\
0.7 &85.25	&63.91	&48.34 &\bf 76.98\\
0.6 &85.26	&62.93	&46.89 &76.77\\
0.5 &84.08	&61.76	&53.53 &75.13\\
0.4 &82.75	&59.09	&52.72 &73.79\\
0.3	&77.34	&53.84	&68.14 &67.67\\
0.2 &73.94	&50.63	&75.55 &60.21\\
0.1 &68.86	&41.39	&82.15 &39.00\\
\bottomrule[1pt]
\end{tabular}
\vspace{-0.3cm}
\caption{Ablation results of different $s$ for $Q_{shuf}$.}
\label{tab:abla_s_q}
\end{table*}

\section{t-SNE visulization}

To illustrate the variance within a class, we provide the Table 2, Fig. 5 and 6 in the paper, which is enough to show the variance change due to different components in our PSL method. Here, we provide the t-SNE visualization for straight understanding. All results are based on HMDB (OoD) from scratch. We provide the visualization results of PSL, PSL with $Q_{ns}$, PSL with $Q_{ns},Q_{sc}$, and PSL with $Q_{ns},Q_{sc},Q_{shuf}$ in Fig.~\ref{fig:tsne_psl},~\ref{fig:tsne_psl_n},~\ref{fig:tsne_psl_n_s},~\ref{fig:tsne_psl_n_s_shu} respectively. From Fig.~\ref{fig:tsne_psl} we can see PSL alone cannot keep the intra-class variance when $s$ decreases. Fig.~\ref{fig:tsne_psl_n} and Fig.~\ref{fig:tsne_psl_n_s} tell us that $Q_{ns}$ and $Q_{sc}$ are important for PSL to keep the intra-class variance. Furthermore, $Q_{shuf}$ makes the feature representation tighter if we compare Fig.~\ref{fig:tsne_psl_n_s} and Fig.~\ref{fig:tsne_psl_n_s_shu}, which shows the model learns more CS information with $Q_{shuf}$.

\begin{figure*}[h!]
% \vspace{-0.3cm}
  \centering
  \includegraphics[width=0.99\linewidth]{docu/figs/tsne_PSL.pdf}
  \vspace{-0.2cm}
  \caption{t-SNE visualization of PSL.}
  \label{fig:tsne_psl}
  \vspace{-0.2cm}
\end{figure*}
\clearpage
\begin{figure*}[h!]
% \vspace{-0.3cm}
  \centering
  \includegraphics[width=0.99\linewidth]{docu/figs/tsne_PSL_n.pdf}
  \vspace{-0.2cm}
  \caption{t-SNE visualization of PSL with $Q_{ns}$.}
  \label{fig:tsne_psl_n}
%   \vspace{-0.4cm}
\end{figure*}

\begin{figure*}[h!]
% \vspace{-0.2cm}
  \centering
  \includegraphics[width=0.99\linewidth]{docu/figs/tsne_PSL_n_s.pdf}
  \vspace{-0.2cm}
  \caption{t-SNE visualization of PSL with $Q_{ns},Q_{sc}$.}
  \label{fig:tsne_psl_n_s}
%   \vspace{-0.2cm}
\end{figure*}
\clearpage
\begin{figure*}[h!]
% \vspace{-0.3cm}
  \centering
  \includegraphics[width=0.99\linewidth]{docu/figs/tsne_PSL_n_s_shu.pdf}
  \vspace{-0.2cm}
  \caption{t-SNE visualization of PSL with $Q_{ns},Q_{sc},Q_{shuf}$.}
  \label{fig:tsne_psl_n_s_shu}
%   \vspace{-0.2cm}
\end{figure*}

\section{InD and OoD uncertainty distribution}

We provide the InD and OoD distribution on HMDB51 (OoD) and MiT-v2 (OoD) with K400 pretrain and without K400 pretrain. All results are based on TSM backbone for illustration. The results are shown in Fig.~\ref{fig:dis_hmdb_sc},~\ref{fig:dis_hmdb_ft},~\ref{fig:dis_mit_sc}, and~\ref{fig:dis_mit_ft}.

From Fig.~\ref{fig:dis_hmdb_sc} and~\ref{fig:dis_mit_sc} we can see that if there is no K400 pretrain, all methods have the overlapping uncertainty between InD and OoD distribution except OpenMax and our PSL. For instance, Fig.~\ref{fig:dis_hmdb_sc} (f) DEAR~\cite{bao2021evidential} shows the uncertainty of InD and OoD samples both cover the range from 0 to 1. In contrast, Fig.~\ref{fig:dis_hmdb_sc} (g) PSL shows that in our method, the InD distribution covers from 0 to 0.3, while the OoD distribution covers from 0 to 0.8. It means our method tends to assign higher uncertainty to OoD samples. For OpenMax, Fig.~\ref{fig:dis_hmdb_sc} (a) shows that InD uncertainy distribution is extremely close to 0, which is a good phenomenon, but the OoD uncertainty distribution only covers from 0 to 0.3, and the OoD samples whose uncertainty is larger than 0.3 is too sparse, which means OpenMax tends to assign low uncertainty to both InD and OoD samples, but assigner lower uncertainty to InD samples.

If we compare Fig.~\ref{fig:dis_hmdb_sc} to Fig.~\ref{fig:dis_hmdb_ft} or compare Fig.~\ref{fig:dis_mit_sc} to Fig.~\ref{fig:dis_mit_ft}, we can find that the InD distribution of all methods are closer to 0 with K400 pretrain. But all methods except our PSL have a serious over confidence problem, which is illustrated by the fact that the far left column of OoD samples is extremely high, which is also emphasized through the red circles in Fig. 4 of the paper. In contrast, the density of OoD distribution is highest at 0.2 uncertainty in our PSL method, and the density of OoD distribution is almost 0 at 0 uncertainty. Besides, it is very clear that the OoD distribution and InD distribution in our PSL is most distinguishable among all methods.
\clearpage
\begin{figure*}[t]
% \vspace{-0.3cm}
  \centering
  \includegraphics[width=0.95\linewidth]{docu/figs/dis_hmdb_sc.pdf}
  \vspace{-0.4cm}
  \caption{Uncertainty distribution on HMDB51 (OoD) w/o K400 pretrain.}
  \label{fig:dis_hmdb_sc}
  \vspace{-0.7cm} 
\end{figure*}
\begin{figure*}[h!]
% \vspace{-0.7cm}
  \centering
\includegraphics[width=0.95\linewidth]{docu/figs/dis_hmdb_ft.pdf}
  \vspace{-0.4cm}
  \caption{Uncertainty distribution on HMDB51 (OoD) w/ K400 pretrain.}
  \label{fig:dis_hmdb_ft}
  \vspace{-0.7cm}
\end{figure*}
\begin{figure*}[h!]
% \vspace{-0.7cm}
  \centering
  \includegraphics[width=0.95\linewidth]{docu/figs/dis_mit_sc.pdf}
  \vspace{-0.4cm}
  \caption{Uncertainty distribution on MiT-v2 (OoD) w/o K400 pretrain.}
  \label{fig:dis_mit_sc}
  \vspace{-0.2cm}
\end{figure*}
\begin{figure*}[t]
  \centering
  \includegraphics[width=0.95\linewidth]{docu/figs/dis_mit_ft.pdf}
  \vspace{-0.4cm}
  \caption{Uncertainty distribution on MiT-v2 (OoD) w/ K400 pretrain.}
  \label{fig:dis_mit_ft}
%   \vspace{-0.2cm}
\end{figure*}
\twocolumn

%%%%%%%%% REFERENCES
{\small
\bibliographystyle{unsrt}
\bibliography{egbib}
}

\end{document}