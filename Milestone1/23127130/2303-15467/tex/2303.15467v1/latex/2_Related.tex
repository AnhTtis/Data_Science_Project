\section{Related Work}
\label{sec:related}

\noindent \textbf{Action Recognition.}
%
Most recent approaches for action recognition are to exploit appearance and motion cues jointly and achieve remarkable success~\cite{feichtenhofer2019slowfast,i3d,lin2019tsm,huang2021tada,qing2022learning,wang2021oadtr,pei2022learning}.
%
Typically, two-stream networks~\cite{two-stream,two-stream-2,TSN} consist of two branches that explore spatial information and temporal dynamics, respectively.
%
% To overcome the limitation of 2D CNN in modeling long-range dependencies, some attempts~\cite{lin2019tsm,r2+1d,TDN} begin to introduce additional temporal mining operations.
Some attempts~\cite{lin2019tsm,r2+1d,TDN} introduce additional temporal mining operations to overcome the limited temporal information extraction ability of 2D CNN.
%
3D CNN-based methods~\cite{feichtenhofer2019slowfast, i3d,C3D} inflated 2D kernels for joint spatio-temporal modeling.
%
\cite{bai2020prototype} proposes the prototype similarity learning which pushes the learned representation to the corresponding prototype as close as possible, while our PSL keeps the differences among the same class.

\noindent \textbf{Open-set Action Recognition.} The related work of OSAR is limited~\cite{krishnan2018bar,shu2018odn,yang2019open,bao2021evidential}. Recently, \cite{bao2021evidential} systematically studies the OSAR problem and transfers several open-set image recognition methods to the video domain, including SoftMax~\cite{hendrycks2016baseline}, MC Dropout~\cite{gal2016dropout}, OpenMax~\cite{bendale2016towards}, and RPL~\cite{chen2020learning}. In the benchmark of~\cite{bao2021evidential}, the only two methods designed specifically for the video domain are BNN SVI~\cite{krishnan2018bar} and their proposed DEAR. BNN SVI is a Bayesian NN application in the OSAR, while DEAR adopts the deep evidential learning~\cite{amini2020deep} to calculate the uncertainty, and utilizes two modules to alleviate the over-confidence prediction and appearance bias problem, respectively. Existing methods pursue better uncertainty scores, while the objective of our PSL is to learn more diverse feature representations for better open-set distinguishability.

\noindent \textbf{Information Bottleneck Theory.} Based on the IB theory~\cite{tishby2000information,tishby2015deep}, the NN intends to extract minimum sufficient information of the inputs for the current task. More recent~\cite{tian2020makes,federici2020learning,wang2022rethinking} adopt the IB theory on unsupervised contrastive learning to analyze the representation learning behavior under the corresponding tasks. In this work, we provide a new view to analyze the OSAR problem based on the IB theory.