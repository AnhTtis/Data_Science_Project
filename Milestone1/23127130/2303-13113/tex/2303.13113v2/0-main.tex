\documentclass[11pt]{article}

% This file will be kept up-to-date at the following GitHub repository:
%
% https://github.com/automl-conf/LatexTemplate
%
% Please file any issues/bug reports, etc. you may have at:
%
% https://github.com/automl-conf/LatexTemplate/issues

\usepackage{microtype} % microtypography
\usepackage{booktabs}  % tables
\usepackage{url}  % urls
\usepackage{graphicx}
\usepackage{float}

% AMS math
\usepackage{amsmath}
\usepackage{amsthm}

% With no package options, the submission will be anonymized, the supplemental
% material will be suppressed, and line numbers will be added to the manuscript.
%
% To hide the supplementary material (e.g., for the first submission deadline),
% use the [hidesupplement] option:
%
% \usepackage[hidesupplement]{automl}
%
% To compile a non-anonymized camera-ready version, add the [final] option (for
% the main track), or the [finalworkshop] option (for the workshop track), e.g.,
%
%\usepackage[final]{automl}
% \usepackage[finalworkshop]{automl}
%
% or
%
\usepackage[final, hidesupplement]{automl}
% \usepackage[finalworkshop, hidesupplement]{automl}

%\usepackage[hidesupplement]{automl}
\def\iconYes{\faCheck}
\def\iconNo{\faTimes}
\def\iconPartially{\faCircleThin}
\def\iconWarn{\faExclamation}
\def\iconBeer{\faBeer}
\def\iconAward{\faTrophy}
\def\iconQuote{\faQuoteLeft}
\def\iconInteraction{\faStreetView}

\definecolor{thesisGreen}{RGB}{46, 204, 64}
\definecolor{thesisRed}{RGB}{255, 65, 54}


\def\acceptedPaper{\color{white}\circled{green}{\iconBeer}}

% You may use any reference style as long as you are consistent throughout the
% document. As a default we suggest author--year citations; for bibtex and
% natbib you may use:

\usepackage{natbib}


%\bibliographystyle{apalike}

% and for biber and biblatex you may use:

% \usepackage[%
%   backend=biber,
%   style=authoryear-comp,
%   sortcites=true,
%   natbib=true,
%   giveninits=true,
%   maxcitenames=2,
%   doi=false,
%   url=true,
%   isbn=false,
%   dashed=false
% ]{biblatex}
% \addbibresource{...}

%\title{CARBON: Continual Adaptive Regularization with Bayesian Optimization}

%\title{Is Adaptation Necessary in Class-Incremental Learning?}
\title{Adaptive Regularization for Class-Incremental Learning}

% The syntax for adding an author is
%
% \author[i]{\nameemail{author name}{author email}}
%
% where i is an affiliation counter. Authors may have
% multiple affiliations; e.g.:
%
% \author[1,2]{\nameemail{Anonymous}{anonymous@example.com}}

\author[1]{\nameemail{Elif Ceren Gok Yildirim}{e.c.gok@tue.nl}}
\author[1]{\nameemail{Murat Onur Yildirim}{m.o.yildirim@tue.nl}}
\author[1]{\nameemail{Mert Kilickaya}{kilickayamert@gmail.com}}
\author[1]{\nameemail{Joaquin Vanschoren}{j.vanschoren@tue.nl}}

% the list might continue:
% \author[2,3]{\nameemail{Author 2}{email2@example.com}}
% \author[3]{\nameemail{Author 3}{email3@example.com}}
% \author[4]{\nameemail{Author 4}{email4@example.com}}

% if you need to force a linebreak in the author list, prepend an \author entry
% with \\:

% \author[3]{\\\nameemail{Author 5}{email5@example.com}}

% Specify corresponding affiliations after authors, referring to counter used in
% \author:

\affil[1]{Automated Machine Learning Group, Eindhoven University of Technology}

% the list might continue:
% \affil[2]{Institution 2}
% \affil[3]{Institution 3}
% \affil[4]{Institution 4}

% define PDF metadata, please fill in to aid in accessibility of the resulting PDF
\hypersetup{%
  pdfauthor={}, % will be reset to "Anonymous" unless the "final" package option is given
  pdftitle={},
  pdfsubject={},
  pdfkeywords={}
}

\begin{document}

\maketitle

\begin{abstract}
 %Continual learning is the idea of learning a model from data with changing distributions throughout time without forgetting what has already been learned. Regularization is one of the ways to prevent catastrophic forgetting by ensuring minimal changes in the weight updates during the learning process. However, current studies in this field assume that all tasks require the same amount of regularization. This assumption may not be necessarily true as different tasks may require different levels of regularization due to the shifts in distributions in continual learning scenarios. To address this issue, in this paper, we are proposing a Continual Adaptive Regularization with Bayesian Optimization (CARBON) for selecting the regularization hyperparameter in class incremental learning. We conduct large-scale experiments on two benchmark datasets, CIFAR-100 and mini-ImageNet, and demonstrate the effectiveness of our approach by outperforming vanilla regularization-based continual learners  through adapting the learner to distributional shifts. We showed that adapting the learners to the shifts is crucial for regularization-based continual learning.

%Class-Incremental Learning (CIL) aims at updating a learner using incoming stream of tasks from varying distributions. The goal is to expand a deep classifier with novel categories, while maintaining the performance on the previously seen classes. To maintain the performance on previous categories, hence minimize forgetting, a prominent method is to regularize the weights of the neural network and prevent abrupt shifts across learning tasks. While working well, existing regularizers make a strong assumption: The regularization magnitude should remain constant through all learning sessions. Such assumption is unrealistic, since how much to regularize should be a function of the current state of the learner and the learning task. To that end, in this paper, we take an alternative approach and ask ourselves: \emph{Is Adaptive Regularization Necessary in Class-Incremental Learning}? We propose a method to automatically determine the magnitude of regularization based on Bayesian Optimization. Our experiments on CIFAR-100 using two prominent regularizers reveal that indeed, adaptive regularization is crucial to build more accurate, less forgetful visual incremental learners. 

%Class-Incremental Learning (CIL) involves updating a deep classifier with new categories while maintaining previous class performance. Regularizing the weights of the neural network is a prominent method to minimize forgetfulness of previously seen categories while learning novel classes. However, existing regularizers assume a constant regularization magnitude through all learning sessions, which is unrealistic. How much to regularize should be a function of the familiarity of the incremental learner with the current learning task: Learning a highly unfamiliar set of categories should be treated differently than learning familiar categories. To that end, in this paper, we ask ourselves: \emph{Is Adaptive Regularization Necessary in Class-Incremental Learning}? We propose an alternative approach to automatically determine the regularization magnitude using Bayesian Optimization per-learning task. Our experiments on CIFAR-100 using two regularizers demonstrate the importance of adaptive regularization for more accurate and less forgetful visual incremental learning.

%Class-Incremental Learning (CIL) updates deep classifier with new categories while maintaining previous class performance. Neural network weight regularization is a prominent method to minimize forgetfulness of previously seen categories while learning novel classes. However, existing regularizers regularize constantly with the same magnitude throughout the learning which is not realistic. How much to regularize should be a function of the familiarity of the incremental learner with the current learning task: Learning a highly unfamiliar set of categories should be treated differently than learning familiar categories. To that end, in this paper, we ask ourselves: \emph{Is Adaptive Regularization Necessary in Class-Incremental Learning}? We propose an alternative approach to automatically determine the regularization magnitude using Bayesian Optimization per-learning task. Our experiments on CIFAR-100 using two regularizers demonstrate the importance of adaptive regularization for more accurate and less forgetful visual incremental learning.

%Class-Incremental Learning (CIL) is the process of updating a deep classifier with new categories while maintaining the performance of previously learned classes. Regularizing the neural network weights is a common method to prevent forgetting previously learned classes while learning novel ones. However, existing regularizers use a constant magnitude throughout the learning sessions, which is not realistic. The regularization amount should be based on the familiarity of the incremental learner with the current task, hence adaptive. This study aims to determine whether adaptive regularization is necessary in CIL. We propose a method to automatically determine the regularization magnitude using Bayesian Optimization for each learning task. Our experiments on CIFAR-100 dataset using two regularizers show the importance of adaptive regularization for accurate and less forgetful visual incremental learning.

Class-Incremental Learning updates a deep classifier with new categories while maintaining the previously observed class accuracy. Regularizing the neural network weights is a common method to prevent forgetting previously learned classes while learning novel ones. However, existing regularizers use a constant magnitude throughout the learning sessions, which may not reflect the varying levels of difficulty of the tasks encountered during incremental learning. This study investigates the necessity of adaptive regularization in Class-Incremental Learning, which dynamically adjusts the regularization strength according to the complexity of the task at hand. We propose a Bayesian Optimization-based approach to automatically determine the optimal regularization magnitude for each learning task. Our experiments on two datasets via two regularizers demonstrate the importance of adaptive regularization for achieving accurate and less forgetful visual incremental learning.

\end{abstract}
\section{Introduction}
\IEEEPARstart{T}{he} method Neural Radiance Fields (NeRF)~\cite{mildenhall2020nerf} is proposed for photorealistic novel view synthesis. Given many views of the scene, it creates implicit multi-view geometry and learns for view synthesis. However, it has poor generalizations to new scenes and requires retraining or fine-tuning on each scene. 
 
 Recent work~\cite{Yu_2021_CVPR,Trevithick_2021_ICCV} has explored the ways of using a single image to train NeRF. They introduce a convolutional feature encoder to learn the image representation which gives it some limited generalization abilities to unseen scenes.  But, without fine-tuning, these methods produce many floats and artifacts in rendering novel views. 
 
  Multi-Plane Images (MPI) representation that learns multiple RGB images from a single image is also used in \cite{Wu_2021_ICCV,Tucker_2020_CVPR,wu2022remote} for  novel view synthesis. However, MPI heavily relies on the qualities of the planar images and needs plenty of image planes to avoid blurs. There is no strong 3D geometry constraint and it fails in many complex scenes.
  
  MINE~\cite{Li_2021_ICCV2} introduces the volume rendering of NeRF into the MPI. It runs faster and produces better depth rendering quality compared with single-view NeRFs~\cite{Yu_2021_CVPR,Trevithick_2021_ICCV}. However, the rendering quality heavily relies on the number of image planes. It needs high-resolution 4D volumes to store the 4-channel  (RGB and volume density) image planes that cost a large amount of GPU memory in both training and 
 prediction.  
 

 
 \begin{figure}[t]
\setlength{\abovecaptionskip}{7pt}
\setlength{\belowcaptionskip}{0pt}
	\centering
% 	\subfigure[MINE (PSNR:14.9)]{  % for AAAI
	\subfloat[MINE (PSNR:14.9)]{
%			\centering
			\includegraphics[width=0.23\textwidth]{figure/intro/DJI_20200223_163206_598_0_MINE.png}
%			\label{subfig:pixelnerf}
	}\subfloat[MINE (depth)]{
%			\centering
			\includegraphics[width=0.23\textwidth]{figure/intro/MINE_disp.png}
%			\label{subfig:mpi}
	}
	\\[-3mm]
	\subfloat[Ours (PSNR:17.0)]{
%			\centering
			\includegraphics[width=0.23\textwidth]{figure/intro/DJI_20200223_163206_598_0_ours.png} 
	}\subfloat[Ours (depth)]{
%			\centering
			\includegraphics[width=0.23\textwidth]{figure/intro/ours_disp.png}
	}
	\caption{Comparison with state-of-the-art methods. (a-b) RGB and depth rendering results of  \cite{Li_2021_ICCV2}. It produces many blurs and floats in the occluded regions and at the object/depth edges. 
	(c-d) Our method employs a joint rendering mechanism that preserves more image details and predicts sharp depth edges.}
	\label{fig:performance_illustration}
\end{figure}
 
 In this paper, we propose a joint rendering mechanism that takes the MPI strategy for coarse sampling proposals and the MLP\&volume-based rendering~\cite{mildenhall2020nerf} for fine sampling and rendering. Then, both the coarse point samples and the fine samples are combined according to their geometry distribution to realize a more accurate joint rendering. More importantly, we introduce a depth teacher net that serves as the guidance for the joint rendering. The monocular depth teacher predicts dense pseudo depth maps that assist the consistent 3D geometry learning between the MPI, the fine volume, and the joint rendering. It also boosts the multi-view geometry consistency between the source view and the target novel views that 
helps handle the occlusions, reduce the blurs and floats, and render accurate depths. 
 
In the experiments,  we verify the effectiveness of our method on three challenging real-scene datasets (RealEstate10K~\cite{zhou2018stereo}, NYU~\cite{silberman2012indoor} and  NeRF-LLFF~\cite{mildenhall2020nerf}) for novel view synthesis or depth estimation. Given a single image as input, our method is shown able to produce higher qualities in both the RGB image rendering and depth map prediction. It far outperforms state-of-the-art methods~\cite{Li_2021_ICCV2,Yu_2021_CVPR} with improvements of 5$\sim$20\% in PSNR and SSIM for the RGB rendering and reduces 20$\sim$50\% of the errors for the depth prediction.

\begin{figure}[t]
 
  \centering
  \includegraphics[width=0.8\textwidth]{figures/teaser.pdf}

 % \caption{A comparison of fixed \textit{vs} adaptive regularization (this work) in class-incremental learning. In class-incremental learning, the learner receives a sequence of tasks, such as dogs, cats or cars. When the new task arrives, the data that belongs to previous tasks are discarded, leading to forgetting. To prevent forgetfulness, researchers regularize neural network weights and prevent abrupt changes across tasks. While working well, they assume regularization should be of equal magnitude throughout learning sessions. In this work, we explore an alternative, and instead learn to predict (tune) the regularization magnitude by adapting to the current task.}
 \caption{A comparison of fixed \textit{vs.} adaptive regularization (ours). In this work, we explore the potential of tuning regularization per-learning task, allowing to learn adaptively.}
  \label{fig:teaser}
\end{figure}


\section{Related Work}

%In the realm of deep learning, traditional models are trained on all available data at once, whereas continual learning involves a stream of data with an evolving distribution. One particularly difficult scenario is the class incremental scenario, in which the model must learn to distinguish between an expanding number of objects or classes over time \citep{van2022three}. The main challenge in continual learning lies in striking a balance between stability and plasticity, that is, retaining previously learned knowledge while still being able to learn new tasks. This issue remains a significant obstacle, and many new techniques have been proposed to mitigate this problem. Despite ongoing efforts to address this challenge, there is still much to be done to overcome it fully.

%The proposed methods can be investigated further under three categories namely; replay-based, architecture-based, and regularization-based methods. Replay-based methods store real or generated samples of a learned task to revisit them during the learning process of the new task to alleviate forgetting \citep{rebuffi2017icarl, lopez2017gradient, bang2021rainbow}. Although replay-based methods are very strong methods to remember old tasks, storing the raw data and using it, especially for training rises a concern in terms of privacy and compute efficiency. Parameter-isolation/Architecture-based methods continually add new subnetworks which are responsible for each task specifically. These methods have started gaining more attention in recent years. However, the problem with these methods is extending the network by adding new subnetworks after each task is not feasible, efficient, and scalable \citep{mallya2018piggyback, li2019learn}. The regularization-based methods propose an extra regularization term to the loss function to consolidate past knowledge when learning a new task \citep{kirkpatrick2017overcoming, li2017learning, zenke2017continual}. While EWC and SI follow very similar working principles which simply regularize the network parameters, LwF distills the knowledge from the old model that is trained on the previous task to a new model in a way that prevents the new model to forget the old task. The drawback of these methods is introducing an extra hyperparameter to balance the stability-plasticity trade-off. These hyperparameters are mainly defined at the very beginning of the learning stage and kept fixed during the whole learning journey which highly affects the incremental performance of the model \citep{de2021continual}.


%%%%%%%%%%%%%%%%%%% Mert's Version %%%%%%%%%%%%%%%%%%%%%%%%%%%%%%%%%
\vspace{-4mm}
\partitle{Class-Incremental Learning.} Class-Incremental Learning updates a deep classifier with sequentially arriving data, usually with mutually exclusive categories~\citep{masana2020class,de2021continual,wang2023comp,zhou2023class,kilickaya2023towards}. However, when novel data arrives, previous training data becomes unavailable, leading to catastrophic forgetting. To mitigate this, researchers have developed two main approaches: replay methods, which store a subset of training data to rehearse during learning~\citep{lopez2017gradient,chaudhry2018efficient,aljundi2019gradient,ostapenko2019learning,xiang2019incremental}, and regularization-based approaches, which stabilize important parameters or distill previous knowledge into the model~\citep{kirkpatrick2017overcoming,chaudhry2018riemannian,zenke2017continual,lee2017overcoming,li2017learning,rebuffi2017icarl,wu2019large,zhou2021co}. 

However, assuming a constant amount of regularization throughout learning sessions is unnatural, since learning unfamiliar objects requires more plasticity than learning familiar ones. To address this issue, we propose a regularization method in which the regularization magnitude is a function of time and is automatically tuned using Bayesian Optimization~\citep{turner2021bayesian}. We evaluate the generality of our approach by choosing EWC~\citep{kirkpatrick2017overcoming} as the prior-based regularizer and LwF~\citep{li2017learning} as the distillation-based regularizer.


%Class-Incremental Learning involves updating a deep classifier with sequentially arriving data, typically with mutually exclusive categories~\citep{masana2020class,de2021continual,wang2023comp,zhou2023class,kilickaya2023towards}. However, when novel data arrives, previous training data becomes unavailable and cannot be used for further optimization, resulting in catastrophic forgetting. Researchers have developed two main approaches to mitigate this problem: replay and regularization methods. Replay methods store a subset of training data to rehearse during the learning of novel classes~\citep{lopez2017gradient,chaudhry2018efficient,aljundi2019gradient,ostapenko2019learning,xiang2019incremental}. In contrast, regularization-based approaches use prior-based or distillation-based regularization to stabilize important parameters or distill previous knowledge into the current model~\citep{kirkpatrick2017overcoming,chaudhry2018riemannian,zenke2017continual,lee2017overcoming,li2017learning,rebuffi2017icarl,wu2019large,zhou2021co}. Regardless of the technique, the amount of regularization is constant throughout learning sessions. However, a constant amount of regularization throughout the learning sessions is an unnatural assumption, since learning an unfamiliar object requires more plasticity than learning a familiar one. To address this issue, we propose a regularization method in which the regularization magnitude is a function of time and is automatically tuned using Bayesian Optimization~\citep{turner2021bayesian}. We choose EWC~\citep{kirkpatrick2017overcoming} as prior-based baseline LwF~\citep{li2017learning} as the distillation-based baseline to show the generality of our approach. 


%Class-Incremental Learning deals with updating a deep classifier with sequentially arriving data, often with the mutually exclusive categories~\citep{masana2020class,de2021continual,wang2023comp,zhou2023class,kilickaya2023towards}. Once the novel data arrives, previous training data is no longer available, therefore can't be used for further optimization, leading to catastrophic forgetting. To mitigate this, researchers either replay or regularize. Replay methods store a subset of training data to rehearse during learning of novel classes~\citep{lopez2017gradient,chaudhry2018efficient,aljundi2019gradient,ostapenko2019learning,xiang2019incremental}. In this work, we follow regularization-based approach. Prior-based regularization selects important parameters to stabilize their weights during novel task optimization~\citep{kirkpatrick2017overcoming,chaudhry2018riemannian,zenke2017continual,lee2017overcoming}. Distillation-based regularization instead distills previous knowledge into the current model in a teacher-student knowledge distillation scheme~\citep{li2017learning,rebuffi2017icarl,wu2019large,zhou2021co}. Regardless of the technique, the amount of regularization is constant throughout learning sessions. Such assumption is unnatural: Learning a rather unfamiliar object requires more plasticity in comparison to learning a familiar object. To that end, in this paper, we turn regularization magnitude a function of the time, and automatically tune it via Bayesian Optimization~\citep{turner2021bayesian}. 

\vspace{-3mm}

\partitle{Hyper-Parameter Optimization.} Hyper-Parameter Optimization (HPO) aims to optimize the hyper-parameters of a given deep learning model, including the learning rate, layer size, or balance of different loss functions. In this paper, our focus is on balancing the contribution of standard classification and the regularization loss. To tackle the HPO problem, complex techniques such as bi-level optimization~\citep{franceschi2018bilevel} or gradient-based optimization~\citep{baydin2018automatic} have been proposed. Bi-level optimizers alternate between optimizing neural network weights and tuning the hyper-parameters, while gradient-based methods treat the entire network weights as a hyper-parameter to be updated. 

However, in this work, we propose to use Bayesian Optimization~\citep{turner2021bayesian} due to its simplicity and effectiveness.
In summary, this paper makes the following contributions: 
%Hyper-Parameter Optimization (HPO) optimizes the set of hyper-parameters of the underlying learner. In case of deep learning, most important parameters include learning rate, layer size, or balancing different loss functions. In this paper, our goal is to balance the contribution of standard classification and the regularization loss. To tackle HPO, complex techniques include bi-level optimization~\citep{franceschi2018bilevel} or gradient-based optimization~\citep{baydin2018automatic}. Bi-level optimizers alternate between optimizing neural network weights and then tune the hyper-parameters. Gradient-based methods treat the whole network weights as a hyper-parameter to be updated. In this paper, we resort to Bayesian Optimization~\citep{turner2021bayesian}, thanks to its simplicity and effectivity. 


%To tackle HPO, advanced techniques include bi-level  ~\citep{xxx} where the learner alternates between training the learner weights and then tuning the hyper-parameters. Other methods include gradient-based optimization~\citep{xxx}


\begin{enumerate}[label=\Roman*.]

\item In this paper, for the first time, we raise the important issue of adaptive regularization in class-incremental learning. 
\item We propose to predict the regularization magnitude conditioned on the state of the deep learner and the current learning task via Bayesian Optimization. 
\item Through large-scale experiments on well-established benchmarks, we show that learning adaptively yields significant performance improvements, in terms of increasing accuracy while reducing forgetting. 
%\item Through experiments on CIFAR-100 and MiniImageNet, we show that adaptivity is consistently superior to fixed regularization, leading to $13\%$ top-1accuracy improvement with EWC~\citep{kirkpatrick2017overcoming} and $10\%$ improvement with LWF~\citep{li2017learning}.

\end{enumerate}


\vspace{-0.3em}
\section{Method}
\vspace{-0.3em}

Our sensitivity-aware visual parameter-efficient fine-tuning consists of two stages. In the first stage, SPT measures the task-specific sensitivity for the pre-trained parameters (Section~\ref{subsec:sensitivity}). Based on the parameter sensitivity and a given parameter budget, SPT then adaptively allocates trainable parameters to task-specific important positions (Section~\ref{subsec:SPT}).

\vspace{-0.3em}
\subsection{Task-specific Parameter Sensitivity}
\label{subsec:sensitivity}
\vspace{-0.3em}

Recent research has observed that pre-trained backbone parameters exhibit varying feature patterns~\cite{raghu2021vision,naseer2021intriguing} and criticality~\cite{zhang2019all,chatterji2019intriguing} at distinct positions. 
Moreover, when transferred to downstream tasks, their efficacy varies depending on how much pre-trained features are reused and how well they adapt to the specific domain gap~\cite{yosinski2014transferable,kumar2022finetuning,neyshabur2020being}. Motivated by these observations, we argue that not all parameters contribute equally to the performance across different tasks in PEFT and propose a new criterion to measure the sensitivity of the parameters in the pre-trained backbone for a given task.

Specifically, given the training dataset $\gD_t$ for the $t$-th task and the pre-trained model weights $\vw=\left\{w_1, w_2, \ldots, w_N\right\}\in \sR^N$ where $N$ is the total number of parameters, the objective for the task is to minimize the empirical risk: $\min_{\vw} E(\gD_t, \vw)$.
We denote the parameter sensitivity \bohan{set} as $\gS=\{s_1, \ldots, s_N\}$ and the sensitivity $s_n$ for parameter $w_n$ is measured by the empirical risk difference when tuning it:
\begin{equation}
\vspace{-0.3em}
    \begin{aligned}
        s_n = E(\gD_t, \vw)-E(\gD_t, \vw\mid w_n=w_n^*),
    \end{aligned}
\label{eq:sensitivity}
\end{equation}
where $w_n^*=\underset{w_n}{\rm argmin}(E(\gD_t, \vw))$. We can reparameterize the tuned parameters as  $w_n^*=w_n+\Delta_{w_n}$, where $\Delta_{w_n}$ denotes the update for $w_n$ after tuning. Here we individually measure the sensitivity of each parameter, which is reasonable given that most of the parameters are frozen during fine-tuning in PEFT. However, it is still computationally intensive to compute Eq.~(\ref{eq:sensitivity}) for two reasons. Firstly, getting the empirical risk for $N$ parameters requires forwarding the entire network $N$ times, which is time-consuming. Secondly, it is challenging to derive $\Delta_{w_n}$, as we have to tune each individual $w_n$ until convergence.

{\begin{algorithm}[t!]
\caption{\label{alg:tps} Computing task-specific parameter sensitivities}
\begin{algorithmic}
    \STATE \textbf{Input:} Pre-trained model with network parameters $\vw$, training set $\gD_t$ for the $t$-th task, and number of training samples $C$ used to calculate the parameter sensitivities
    \STATE \textbf{Output:} Sensitivity set $\gS=\{s_1, \ldots, s_N\}$
    \STATE Initialize $\gS=\{0\}^N$
    \FOR{$i\in\{1,\ldots,C\}$}
        \STATE Get the $i$-th training sample of $\gD_t$
	    \STATE Compute loss $E$
		\STATE Compute gradients $\vg$
		\FOR{$n\in\{1,\ldots,N\}$}
                \STATE Update sensitivity for the $n$-th parameter: $s_{n} = s_{n} + g_n^2$
		    \ENDFOR
    \ENDFOR
\end{algorithmic}
\end{algorithm}}


\begin{figure*}[t]
\begin{center}
    \includegraphics[width=\linewidth]{main_figure.pdf}
\end{center}\vspace{-2em}
\caption{Overview of our trainable parameter allocation strategy. With the parameter sensitivity \bohan{set} $\gS$, we first get the top-$\tau$ sensitive parameters. Instead of directly tuning these sensitive parameters, we also boost the representational capability by replacing unstructured tuning with structured tuning at sensitive weight matrices that have a large number of sensitive parameters, which can be implemented by an existing structured tuning method, \eg, LoRA~\cite{hu2022lora} and Adapter~\cite{houlsby2019parameter}. Red lines and blocks represent trainable parameters and modules, while blue lines represent frozen parameters.}
\label{fig:main}
\vspace{-1.5em}
\end{figure*}


To overcome the first barrier, we simplify the empirical loss by approximating $s_n$ in the vicinity of $\vw$ by its first-order Taylor expansion
\vspace{-0.3em}
\begin{equation}
\vspace{-0.5em}
    \begin{aligned}
        s_n^{(1)} = -g_n\Delta_{w_n},
    \end{aligned}
\label{eq:first-order}
\end{equation}
where the gradients $\vg=\partial E/\partial\vw$, and $g_n$ is the gradient of the $n$-th element of $\vg$. 
To address the second barrier, following~\cite{liu2018darts,cai2018proxylessnas}, we take the one-step unrolled weight as the surrogate for $w_n^*$ and approximate $\Delta_{w_n}$ in Eq.~(\ref{eq:first-order}) with a single step of gradient descent. We can accordingly get $s_n^{(1)} \approx g_n^2\epsilon$,
where $\epsilon$ is the learning rate. Since $\epsilon$ is the same for all parameters, we can eliminate it when comparing the sensitivity with the other parameters and finally get 
\vspace{-0.5em}
\begin{equation}
\vspace{-0.3em}
    \begin{aligned}
        s_n^{(1)} \approx g_n^2.
    \end{aligned}
\label{eq:first-order-simp}
\end{equation}
Therefore, the sensitivity of a parameter can be efficiently measured by its potential to reduce the loss on the target domain. Note that although our criterion draws inspiration from pruning work~\cite{molchanov2019importance}, it is distinct from it. \cite{molchanov2019importance} measures the parameter importance by the squared change in loss when removing them, \ie, $\left( E(\gD_t, \vw)-E(\gD_t, \vw\mid w_n=0) \right)^2$ and finally derives the parameter importance by $\left( g_n w_n \right)^2$, which is different from our formulations in Eqs.~(\ref{eq:sensitivity}) and~(\ref{eq:first-order-simp}).

In practice, we accumulate $\gS$ from a total number of $C$ training samples ahead of fine-tuning to generate accurate sensitivity as shown in Algorithm~\ref{alg:tps}, where $C$ is a pre-defined hyper-parameter. In Section~\ref{subsec:abl}, we show that employing only 400 training samples is sufficient for getting reasonable parameter sensitivity, which requires only 5.5 seconds with a single GPU for any VTAB-1k dataset with ViT-B/16 backbone~\cite{vit}.

\vspace{-0.3em}
\subsection{Adaptive Trainable Parameters Allocation}
\label{subsec:SPT}
\vspace{-0.2em}

Our next step is to allocate trainable parameters based on the obtained parameter sensitivity set $\gS$ and a desired parameter budget $\tau$. A straightforward solution is to directly tune the top-$\tau$ most sensitive unstructured connections (parameters) \rev{while keeping the rest frozen}, which we name unstructured tuning. Specifically, we select the top-$\tau$ most sensitive weight connections in $\gS$ to form the sensitive weight connection set $\gT$. Then, for \rev{a} weight matrix $\mW\in \sR^{d_{\rm in}\times d_{\rm out}}$, we can get a binary mask $\mM\in \sR^{d_{\rm in}\times d_{\rm out}}$ computed by
\vspace{-0.5em}
\begin{equation}
\vspace{-0.5em}
    {\begin{array}{ll}
    \small
    \begin{aligned}
    \mM^j =
    \left\{\begin{array}{ll} 
    1 ~~~~~ \mW^j \in \gT \\
    0 ~~~~~ \mW^j \notin \gT
    \end{array}\right.
    \end{aligned},
    \small
    \end{array}}
\label{eq:mask}
\end{equation}
where $\mW^j$ and $\mM^j$ are the $j$-th element in $\mW$ and $\mM$, respectively. Accordingly, we can train the sensitive parameters by gradient descent and the updated weight matrix can be formulated as $\mW'\leftarrow \mW - \epsilon\vg_{\mW}\odot\mM$, where $\vg_{\mW}$ is the gradient for $\mW$.

However, considering PEFT approaches generally limit the proportion of trainable parameters to less than 1\%, tuning only a small number of unstructured weight connections might not have enough representational capability to handle the downstream datasets with large domain gaps from the source pre-training data. Therefore, to improve the representational capability, we propose to replace unstructured tuning with structured tuning at the sensitive weight matrices that have a high number of sensitive parameters. To preserve the parameter budget, we can implement structured tuning with an existing efficient structured tuning PEFT method~\cite{hu2022lora,chen2022adaptformer,houlsby2019parameter,jie2022convolutional} that learns to directly adjust \rev{all hidden dimensions at once}. We depict an overview of our trainable parameter allocation strategy in Figure~\ref{fig:main}. For example, we can employ the low-rank reparameterization trick LoRA~\cite{hu2022lora} to the sensitive weight matrices \rev{and the one-step update for $\mW$ can be formulated as}
\vspace{-0.4em}
\begin{equation}
\vspace{-0.4em}
    {\begin{array}{ll}
    \small
    \begin{aligned}
    \mW' = \left\{\begin{array}{ll} 
    \mW + \mW_{\rm down}\mW_{\rm up} & ~~ \text { if } ~~ \sum_{j=0}^{d_{\rm in}\times d_{\rm out}} \mM^j \geq \sigma_{\rm opt} \\
    \mW - \epsilon\vg_{\mW}\odot\mM & ~~ {\rm otherwise}
    \end{array}\right.
    \end{aligned},
    \small
    \end{array}}
\label{eq:weight_updat}
\end{equation}
where $\mW_{\rm down}\in \sR^{d_{\rm in}\times r}$ and $\mW_{\rm up}\in \sR^{r\times d_{\rm out}}$ are two learnable low-rank matrices to approximate the update of $\mW$ and rank $r$ is a hyper-parameter where $r \ll {\rm min}(d_{\rm in},d_{\rm out})$. In this way, we perform structured tuning on $\mW$ when its number of sensitive parameters exceeds $\sigma_{\rm opt}$, whose value depends on the pre-defined type of structured tuning method. For example, since implementing structured tuning with LoRA requires $2\times d_{\rm in} \times d_{\rm out} \times r$ trainable parameters for each sensitive weight matrix, we set $\sigma_{\rm LoRA} \leftarrow 2\times d_{\rm in} \times d_{\rm out} \times r$ to ensure that the number of trainable parameters introduced by structured tuning is always equal to or lower than the number of sensitive parameters.

In this way, our SPT adaptively incorporates both structured and unstructured tuning granularities to enable higher flexibility and stronger representational power, simultaneously. In Section~\ref{subsec:abl}, we show that structured tuning is important for the downstream tasks with larger domain gaps and both unstructured and structured tuning contribute clearly to the superior performance of our SPT.
\section{Experimental Setup}
\label{sec:setup}

%In this section, we provide the setup that we designed for our experiments. We follow the class incremental learning setup of the PYCIL framework \citep{zhou2021pycil} which only extends the classifier layer by the number of classes per task. PYCIL offers various continual learning baselines and we implement the hyperparameter optimization on two regularization methods which are EWC and LwF. We use the RayTune framework \citep{liaw2018tune} for the hyperparameter optimization and we did not use a pre-trained architecture and train all tasks from scratch. All our experiments run on a single-GPU setup of Nvidia A100.

%\subsection{Datasets and Architecture}
%We use the commonly used Split CIFAR100 dataset as in most of the CIL research and to be sure that results are more convincing, we also use mini-Imagenet dataset to further investigate our adaptive approach. We applied the same transformations defined in the PYCIL framework except we set resize to 64 for mini-Imagenet dataset to fasten the training process. For Split CIFAR-100 experiments we run all experiments with 3 different seeds to see the effect of task ordering and we observed that the task order does not have a significant effect in terms of performance. Therefore, we did not run mini-Imagenet experiments with different seeds. 

%\noindent
%\textbf{1) Split CIFAR-10:} It includes 10 object classes \citep{krizhevsky2009learning}. We used Split CIFAR-10 dataset to make a sensitivity analysis to observe how sensitive the EWC and LwF methods to their regularization hyperparameter.

%\noindent
%\textbf{2) Split CIFAR-100:} It includes 100 classes of visual object instances of CIFAR-100 \citep{krizhevsky2009learning}. In our setup, we randomly divide 100 classes into 10 tasks with 10 classes per task.

%\noindent
%\textbf{3) mini-Imagenet:} It is a variation of Imagenet dataset containing 100 classes of visual objects \citep{deng2009imagenet}. In our setup, we randomly divide 100 classes into 10 tasks with 10 classes per task.

%We used a specific version of the Resnets called Resnet32 \citep{he2016deep} for all our experiments to interpret the results more clearly.

%\subsection{Hyperparameters}

%EWC and LwF methods introduce an important lambda hyperparameter that controls the trade-off between stability and plasticity. In our experiments, we selected the baselines as vanilla EWC and vanilla LwF. In the vanilla settings, we set lambda to 1 which gives equal emphasis to current and previous tasks, and kept it fixed during the whole learning. On the other hand, in our adaptive approach, CARBON searches pre-defined search space and chooses the best lambda value based on validation accuracy. The search space for lambda is set to [1, $10^5$] for EWC and [1, 50] for the LwF based on our sensitivity analysis.

%\subsection{Performance Metrics}

%To measure the performance of each configuration we evaluated the model based on the validation data which is simply a portion of the training data. We derived two different accuracy metrics from ACC given in Eq(\ref{eqn:acc}) and called them incremental accuracy (last) and incremental accuracy (avg). Incremental accuracy (last) denotes the top-1 accuracy after the last task and it is a proper metric to measure the overall accuracy among all learned classes. The incremental accuracy often decreases with more tasks learned since CIL is continually adapted. However, only comparing incremental accuracy (last) ignores the performance evaluation along the learning trajectory. Therefore, we denoted incremental accuracy (avg) which considers the performance after every incremental stage. The higher value indicates a stronger performance along the incremental stages \citep{zhou2023deep}. We  also measure the level of forgetting with a backward transfer metric BWT Eq(\ref{eqn:bwt}) where $A_{T, i}$ is the test accuracy for task $i$ after training on task $T$. Higher BWT indicates a lower forget ratio \citep{kang2022forget}.

%\begin{equation}
%\label{eqn:acc}
%ACC = {\frac{1}{T}} \sum_{i=1}^{T} A_{T, i} 
%\end{equation}


%\begin{equation}
%\label{eqn:bwt}
%BWT = {\frac{1}{T-1}} \sum_{i=1}^{T-1} A_{T, i} - A_{i, i}
%\end{equation}


%%%%%%%%%%%%%%%%%%% Mert's Version %%%%%%%%%%%%%%%%%%%%%%%%%%%%%%%%%

 \partitle{Datasets.} In this paper, we experiment with \textbf{Split-CIFAR100}~\citep{krizhevsky2009learning} and \textbf{Split-MiniImageNet}~\citep{deng2009imagenet}. Each dataset exhibits objects from $100$ different categories, such as bird, snake and spider. We train all the models with $10$ tasks, with $10$ classes within each learning task on both CIFAR100 and MiniImageNet. Both datasets have 5000 training, and 1000 testing color images per learning task, each with $32\times32$ and $64\times64$ resolution for CIFAR100 and MiniImageNet respectively. 
%%%%% V1 %%%%%%%%%%%%%%%%%%
 
 %\partitle{Metrics.}  We evaluate the performance of our model using two standard metrics, top-1 accuracy and backward transfer for  
%$T$ tasks~\citep{diaz2018don}. Specifically, accuracy measures the test performance for all the observed tasks until $T$, whereas backward transfer measures how well the inclusion of task at time step $t-1$ influences the performance on task at previous time steps. Formally: 
%\begin{eqnarray}
%\mbox{\small {\bf Average Accuracy: } \normalsize ACC} & = & \frac{1}{T}
%\sum_{i=1}^T A_{T,i} \label{eq:acc} \\
%\mbox{\small {\bf Backward Transfer: } \normalsize BWT} & = & \frac{1}{T-1}
%\sum_{i=1}^{T-1} A_{T,i} - A_{i,i}   \label{eq:bwt} \\
%\end{eqnarray}
%\noindent where $A_{T,i}$ is the test accuracy of the model on task $i$ at time step $T$. For each metrics, the higher is better. 


%%%%%%%%%%%%%% V2 %%%%%%%%%%%%%%%%%%%%%

\partitle{Metrics.}  We resort to the standard metrics for evaluation, accuracy (ACC) which measures the final accuracy averaged over all tasks,  and backward transfer (BWT) which measures the average accuracy change of each task after learning new tasks. Formally for accuracy:
{\small
\vspace{-0.1cm}
\begin{align}
    ACC=\frac{1}{T}\sum\nolimits_{i=1}^T A_{T,i},
    \vspace{-0.1cm}
\end{align}}%

\noindent and for backward transferability: 
{\small
\begin{align}
    BWT=\frac{1}{T-1}\sum\nolimits_{i=1}^{T-1} (A_{T,i}-A_{i,i})
    \vspace{-0.1cm}
\end{align}}

\noindent where $A_{T,i}$ represents the testing accuracy of task $T$ after learning task $i$. In both cases, higher values indicate better performance. 
\section{EXPERIMENTS AND ANALYSIS}
\label{results}
\subsection{Experiment Settings}
The ConvS2S model has 512 hidden units for both encoders and decoders. All embeddings, including the output produced by the decoder before the final linear layer, have a dimensionality of 768. This setup allows the encoders to concatenate with patch embeddings from ViT model. To avoid overfitting, dropout is applied on the embeddings, decoder output, and the input of the convolutional blocks with a retaining probability of 0.5.


% We train the convolutional model using Adam optimizer with a fixed learning rate 2.50e-4.
Many experiments are carried out in order to evaluate the proposed approach toward the VLSP-EVJVQA challenge. We begin by initializing the baseline result of ConvS2S without using any image information. This mean that the generated answers are completely based on the answer-question dependencies learned by the model during the training phase. We then sequentially add hint and image features to the input sequence and study their effect on the overall performance. Because of the limitation in computational resources as well as the strict timeline of the competition, we only deploy the fine-tuned ViLT-B/32 with 200K pretraining steps and pre-trained OFA$_{\mathrm{large}}$ with 472M parameters for hints inference given the question and image.
To have the comparative result, we set up the same hyperparameters for all experiments. The models are trained in 30 epochs using Adam optimizer with a fixed learning rate of 2.50e-4 and batch size of 128. After each epoch, the performance loss on the train and development sets is calculated using the Cross-Entropy Loss function.

The proposed architecture and SOTA vision and language models are implemented in PyTorch and trained on the Kaggle platform with hardware specifications: Intel(R) Xeon(R) CPU @ 2.00GHz; GPU Tesla P100 16 GB with CUDA 11.4.

\subsection{Experimental Results}
\begin{table}[H]

    \centering
    \resizebox{\columnwidth}{!}{%
    \setlength{\tabcolsep}{5pt}
    \renewcommand{\arraystretch}{1.2}
    \begin{tabular}{lcccccc}
    \toprule
        \textbf{Model} & \textbf{F1} & \textbf{BLEU-1} & \textbf{BLEU-2} & \textbf{BLEU-3} & \textbf{BLEU-4} & \textbf{BLEU (Avg.)}  \\ \midrule
        ConvS2S (no image features) & 0.3005 &0.2592	&0.2034	&0.1677	&0.1425& 0.1932  \\ \midrule
        ConvS2S + ViLT-B/32 & 0.3294 &0.2692	&0.2109	&0.1723	&0.1446& 0.1993  \\ 
        ConvS2S + OFA$_{\mathrm{large}}$ & 0.3331 &0.2858	&0.2269	&0.1876	&0.1598 & 0.2150  \\ 
        \textbf{ConvS2S + ViLT-B/32 + OFA$_{\mathbf{large}}$}
        % \tablefootnote{This model is not yet evaluated on the private test set \label{note1}}
        & \textbf{0.3442} &0.2797	&0.2205	&0.1808	&0.1529& \textbf{0.2085}  \\ 
        \midrule
                ConvS2S + ViT-B/16 & 0.3109 &0.2683	&0.2119	&0.1747	&0.1480 & 0.2007  \\ %\midrule
        ConvS2S + ViT-B/16 + ViLT-B/32 & 0.3361 &0.2833	&0.2243	&0.1845	&0.1564 & 0.2122  \\ 
        ConvS2S + ViT-B/16 + OFA$_{\mathrm{large}}$ & 0.3390 &0.2877	&0.2276	&0.1877	&0.1593 & 0.2156  \\
        \textbf{ConvS2S + ViT-B/16 + ViLT-B/32 + OFA$_{\mathbf{large}}$}
        % \textsuperscript{\ref{note1}}
        & \textbf{0.3442} & 0.2747	&0.2148	&0.1747	& 0.1465& \textbf{0.2027} \\ \bottomrule
    \end{tabular}}
    \caption{Performance of ConvS2S with different combinations of pre-trained models on the public test set.}
    \label{result_public}
\end{table}

\begin{figure}[ht]
\centering
% \subfloat[ConvS2S training loss per epoch]{%
%   \includegraphics[width=0.495\textwidth]{figure/train_loss1.pdf}%
% }
% \hspace{-0.2em}
% \subfloat[ConvS2S testing loss per epoch]{%
%   \includegraphics[width=0.495\textwidth]{figure/test_loss1.pdf}%
% }
\includegraphics[width=\textwidth]{figure/all_loss.pdf}
\caption{Training loss and public testing loss comparison of ConvS2S model with different combinations of hint and image features}
\label{loss}
\end{figure}


The two metrics: F1 and BLEU, are used in the challenge to evaluate the results. The BLEU score is the average of BLEU-1, BLEU-2, BLEU-3, and BLEU-4. F1 is used for ranking the final results. Table \ref{result_public} presents the performance of the proposed ConvS2S model with different combinations of pre-trained models on the UIT-EVJVQA public test set.

% First, with only question as input, ConvS2
According to Table \ref{result_public}, the original ConvS2S model without image features but using only question obtained 0.3005 by F1 and 0.1932 by BLEU. When integrating hint features from images, the F1 score improved at least 2.89\% and achieve highest result with 0.3442 by F1 and 0.2085 by BLEU when using both ViLT and OFA hints. After adding image feature from ViT-B/16, the performance of previous models tend to improve. However the final ensemble does not surpass the ConvS2S{\tiny~}+{\tiny~}ViLT-B/32{\tiny~}+{\tiny~}OFA$_{\mathrm{large}}$ ensemble on F1 metrics and even give lower BLEU score. Based on F1, these two ensembles are considered as our best models on the public test set. 
Figure \ref{loss} depicts the gradual improvement in both training loss and testing loss as more image features are added to the ConvS2S model. Memory-based ConvS2S does not catch the image context and thus have the highest loss. Though ConvS2S with ViT+VILT features does not obtained a competitive result on F1 and BLEU score, it has the best loss among methods in the public test phase. In general, the optimal testing loss of methods is achieved between 14th and 20th epoch, then the models tend to be overfitting.


% \begin{figure}
%     \centering
%     \includegraphics[width=\textwidth]{figure/train_loss.pdf}
%     \caption{tmp}
%     \label{100score}
% \end{figure}
% \subsubsection{Qualitative analysis}
% \label{quali_analysis}

% \begin{figure}
%     \centering
%     \includegraphics[width=\textwidth]{figure/test_loss.pdf}
%     \caption{tmp}
%     \label{100score}
% \end{figure}
% \subsubsection{Qualitative analysis}
% \label{quali_analysis}

We manage to deploy two ensembles of ConvS2S using features from ViT-B/16 combined with hints from {\tiny~}ViLT-B/32 and {\tiny~}OFA$_{\mathrm{large}}$, respectively, for the final evaluation on private test set. As shown in Table \ref{result_private}, the ConvS2S{\tiny~}+{\tiny~}ViT-B/16{\tiny~}+{\tiny~}OFA$_{\mathrm{large}}$ model obtained the better result, which is 0.4210 by F1 and 0.3482 by BLEU, and ranked $3^{rd}$ in the challenge. Table \ref{ranking} shows the final standing at the EVLSP-EVJVQA competition, in which our best model perform poorer 1.82\% and 1.39\% by F1 compared with the first and second place solutions. Overall, there is a gap between F1 and BLEU scores.



\begin{table}[H]
    \centering
    \small
    %\resizebox{\columnwidth}{!}{%
    \setlength{\tabcolsep}{5pt}
    \renewcommand{\arraystretch}{1.2}
    \begin{tabular}{lcc}
    \toprule
        \textbf{Model} & \textbf{F1} & \textbf{BLEU}  \\ \midrule
        ConvS2S + ViT-B/16 + ViLT-B/32 &0.4053  &0.3228  \\
        \textbf{ConvS2S + ViT-B/16 + OFA$_{\mathbf{large}}$} & \textbf{0.4210}  & \textbf{0.3482}
  \\ \bottomrule
    \end{tabular}
    \caption{Performance on the private test set.}
    \label{result_private}
\end{table}

\begin{table}[!htbp]
\small
%\resizebox{\columnwidth}{!}{%
\centering
\begin{tabular}{clccccc}
\toprule
\multirow{2}{*}{\textbf{No.}} & \multirow{2}{*}{\textbf{Team name}} & \multicolumn{2}{c}{\textbf{Public Test}} && \multicolumn{2}{c}{\textbf{Private Test}} \\\cmidrule{3-4} \cmidrule{6-7}
                             &                                     & \textbf{F1}         & \textbf{BLEU}      && \textbf{F1}         & \textbf{BLEU}       \\\midrule
1                            & CIST AI                             & 0.3491              & 0.2508             && 0.4392              & 0.4009              \\
2                            & OhYeah                              & 0.5755              & 0.4866             && 0.4349              & 0.3868              \\
3                            & \textbf{DS\_STBFL}                  & \textbf{0.3390}     & \textbf{0.2156}    && \textbf{0.4210}     & \textbf{0.3482}     \\
4                            & FCoin                               & 0.3355              & 0.2437             && 0.4103              & 0.3549              \\
5                            & VL-UIT                              & 0.3053              & 0.1878             && 0.3663              & 0.2743              \\
6                            & BDboi                               & 0.3023              & 0.2183             && 0.3164              & 0.2649              \\
7                            & UIT\_squad                          & 0.3224              & 0.2238             && 0.3024              & 0.1667              \\
8                            & VC\_Internship                      & 0.3017              & 0.1639             && 0.3007              & 0.1337
\\\bottomrule       
\end{tabular}
\caption{Our performance compared with other teams at VLSP2022-EVJVQA}
\label{ranking}
\end{table}

\subsection{Performance Analysis}

According to the final result in the private test phase, the generated output from ConvS2S
+ViT-B/16+OFA$_{\mathrm{large}}$ model are chosen for further analysis. Generally, the model manages to generate answers with correct language with the input question.
\subsubsection{Quantitative analysis}
We randomly choose 100 samples from the generated result to perform quantitative analysis. The average length, vocabulary size, and the number of POS tags in the ground truth and generated answers are calculated for each language. Table \ref{quanti} shows the statistics of the ground truth answer compared with the predicted answer by the model.

% \begin{table}[ht]
% \centering
% %\resizebox{\columnwidth}{!}{%
% \begin{tabular}{llrr}
% \toprule
% &Language&Ground Truth&Predicted\\\midrule

% \multirow{ 4}{*}{Avg. length} & English & 3.74 & 6.18 \\
% & Vietnamese & 4.42 & 5.97\\
% & Japanese & 4.67 & 8.43\\
% & All &4.26&6.78\\\midrule

% \multirow{ 4}{*}{Vocab. size} & English & 78 & 72 \\
% & Vietnamese & 97 & 101\\
% & Japanese & 77 & 83\\
% & All &252&256\\\midrule

% \multirow{ 4}{*}{\# POS tag} & English & 12 & 9 \\
% & Vietnamese &10  &9 \\
% & Japanese & 10 & 11\\
% & All &14 &14\\
% \bottomrule
% \end{tabular}
% \caption{The quantitative statistic of 100 generated samples compared with the ground truth}
% \label{quanti}
% \end{table}

\begin{table}[ht]
\centering
%\resizebox{\columnwidth}{!}{%
\begin{tabular}{llrr}
\toprule
Language&Stats.&Ground Truth&Predicted\\\midrule
\multirow{ 3}{*}{English} & Avg.length  & 3.74 & 6.18 \\
& Vocab. size & 78 & 72 \\
& \# POS tag  & 12 & 9 \\\midrule

\multirow{ 3}{*}{Vietnamese} & Avg.length  & 4.42 & 5.97 \\
& Vocab. size  & 97 & 101 \\
& \# POS tag &10  &9 \\\midrule

\multirow{ 3}{*}{Japanese} & Avg.length   & 4.67 & 8.43 \\
& Vocab. size & 77 & 83 \\
& \# POS tag  & 10 & 11 \\\midrule\midrule

\multirow{ 3}{*}{All} & Avg.length  &4.26 &6.78 \\
& Vocab. size &252 &256 \\
& \# POS tag  &14 &14 \\

\bottomrule
\end{tabular}
\caption{The quantitative statistic of 100 generated samples compared with the ground truth}
\label{quanti}
\end{table}

From Table \ref{quanti}, it can be seen that although the model gave the answers longer than the ground truth answers, the semantics is not as much as the ground truth. It can be seen from Table \ref{quanti} that the predicted answers in English have an average length higher than the ground truth answers. Also, the vocabulary in the generated answers is more than the original. In contrast, the number of POS tag components in the predicted answers is lower than the ground truth. This is similar to the answers in Vietnamese. For the Japanese, the characteristics of the predicted answers in average length and vocabulary size are the same as the two remaining languages. However, the number of POS tags in the predicted answers is more than in the ground truth answers. To make it clear, we propose three types of error on our model in Section \ref{quali_analysis}.

In addition, Figure \ref{100score} illustrates the distributions of F1 and BLEU scores for each language. Generally, the histograms skewed to the right and the model  performs inconsistently across languages. The proportion of samples with F1 and BLEU scores less than 0.2 dominates the overall result across all three languages. In Vietnamese, the number of generated samples with F1 and BLEU scores greater than 0.4 is significantly higher than in other languages. Meanwhile, English and Japanese responses rarely score greater than 0.6 on both metrics, furthermore, no Japanese samples scoring greater than 0.8 in BLEU. This illustrates that our model faces numerous challenges in producing the desired responses, with specific limitations on each language.

\begin{figure}[!ht]
    \centering
    \includegraphics[width=\textwidth]{figure/hist.pdf}
    \caption{Distributions of F1 and BLEU scores for each language from 100 generated samples}
    \label{100score}
\end{figure}

\begin{figure}[!htbp]
\centering
\subfloat[]{%
  \includegraphics[width=0.8\textwidth]{figure/attns1.pdf}%
  \label{attn1}
}

\subfloat[]{%
  \includegraphics[width=0.8\textwidth]{figure/attns2.pdf}%
  \label{attn2}
}

\subfloat[]{%
  \includegraphics[width=0.8\textwidth]{figure/attns3.pdf}%
  \label{attn3}
}

\subfloat[]{%
  \includegraphics[width=0.8\textwidth]{figure/attns4.pdf}%
  \label{attn4}
}

\subfloat[]{%
  \includegraphics[width=0.8\textwidth]{figure/attns5.pdf}%
  \label{attn5}
}
\caption{Numerous samples of attention alignment from ConvS2S and the changes in attention when adding features from ViT-B/16 and OFA$_{\mathrm{large}}$. The x-axis and y-axis of each plot correspond to the words in the question and the generated answer, respectively, while each pixel illustrates the weight $w_{ij}$ of the assignment of the j-th question word for the i-th
answer word.}
\label{attn}
\end{figure}

\subsubsection{Qualitative analysis}
\label{quali_analysis}
\paragraph{Attention visualization}

Figure \ref{attn} shows several samples of attention weights between each element from the generated answer with those in the input sequence that contains no image features, OFA hints, and OFA+ViT features, respectively. The visualization provided an intuitive way to discover which positions in the input sequence were considered more important when generating the target answer word. The brighter a pixel's color, the more important the word in the input sequence is in producing the respect answer word. Through this, we study that OFA hint is importance feature to model's attention as it provide the near-correct insight for the question and reduce the reliance on question words when generating the answer. However, in some cases, the model focuses too much on a specific element of the hint, which may lead to bias. ViT features has shown to control the affection of OFA hint, neutralizing it with other elements from question if hint appears to be off-topic. It may enhance the attention, making the model focus stronger on specific parts of the provided hint, for instance, the hint token ``nhà hàng'' (\textit{restaurant}) in Figure \ref{attn3} is given more attention when adding ViT image features. These features can also reduce the attention in one element and distributes concentration on other parts of the sequence. Figures \ref{attn1} and \ref{attn2} depict the reduction in hint attention into question elements, while Figures \ref{attn4} and \ref{attn5} show the changes in attention weight distribution among hint tokens.

\paragraph{Error analysis}
\begin{figure}[!ht]
\centering
\subfloat[Error Case I]{%
  \includegraphics[width=\textwidth]{figure/err1.pdf}%
\label{fig:1a}}
\vspace{1em}
\subfloat[Error Case II]{%
  \includegraphics[width=\textwidth]{figure/err2.pdf}%
    \label{fig:1b}
}
\vspace{1em}
\subfloat[Error Case III]{%
  \includegraphics[width=\textwidth]{figure/err3.pdf}%
\label{fig:1c}}
\caption{Three typical error cases from generated results.}
\label{fig:1}
\end{figure}

% \begin{figure}[H]
% \centering
% \resizebox{\textwidth}{!}{
%     \begin{subfigure}[b]{.3\linewidth}
%     \centering
%     \includegraphics[width=0.99\textwidth]{figure/00000001682.jpg}
%     \raggedright
%     { \scriptsize \textbf{Question}: what hat does the narrator of the 
%     historical site wear?}\\
%     {\scriptsize \textbf{Groundtruth}: non la}\\
%     {\scriptsize \textbf{Predicted}: the boy wears a white shirt and white and white}\\
%     {\scriptsize \textbf{F1:}  0.0000}\\
%     {\scriptsize \textbf{BLEU:} 0.0000
%     ~~~~~~~~~~~~~~~~~~~~~~~~~~~~~~~~~~~~~~~~~~~~~~~~~~~~~~~~~~~~~~~~~~~~~~~~~~~~~~~~~~~~~~~~~~~~~~~~~~ }
%     \caption{Error Type I}
%     \label{fig:1a}
%   \end{subfigure}%
%   \hspace{0.5em}
  
%  %\hspace*{\fill}
%   \begin{subfigure}[b]{.35\linewidth}
%     \centering
%     \includegraphics[width=0.99\textwidth]{figure/00000004737.jpg}
%     \raggedright
    
%     {\scriptsize \textbf{Question}: có bao nhiêu người đứng bên phải chàng trai? (\textit{English: How many people on the right of the man?})}\\
    
%     {\scriptsize \textbf{Groundtruth}: có ba người đứng bên phải chàng trai (\textit{English: There are three people on the right of the man})}\\
    
%     {\scriptsize \textbf{Predicted}: có hai người đứng bên phải chàng trai (\textit{English: There are two people on the right of the man})}\\
    
%     {\scriptsize F1:  0.8750}\\
    
%     {\scriptsize BLEU: 0.7799}
%     \caption{Error Type II}
%     \label{fig:1b}
%   \end{subfigure}%
%   \hspace{0.5em}
%   %\hspace*{\fill}
%   \begin{subfigure}[b]{0.35\linewidth}
%      \centering
%     \includegraphics[width=0.99\textwidth]{figure/00000000111.jpg}
    
%     \raggedright {\scriptsize \textbf{Question}:}
%     {\tiny
%     \begin{CJK*}{UTF8}{min}
%     {\CJKfamily{goth}小船手は何本のオールを使っていますか? (\scriptsize \textit{English: How many paddles does the boatman use?})}
%     \end{CJK*}}\\
%     {\scriptsize \textbf{Groundtruth}: 2}\\
%     {\scriptsize \textbf{Predicted}:}
%     {\tiny
%     \begin{CJK*}{UTF8}{min}
%     {\CJKfamily{goth}2本の船を使っています (\scriptsize \textit{ English: using two boats})}
%     \end{CJK*}}\\
%     {\scriptsize \textbf{F1:} 0.0000}\\
%     {\scriptsize \textbf{BLEU:} 0.0000 ~~~~~~~~~~~~~~~~~~~~~~~~~~~~~~~~~~~~~~~~~~~~~~~~~~~~~~~~~~~~~~~~~~~~~~~~~~~~~~~~~~~~~~~~~~~~~~~~~~ }\\
%     \caption{Error Type III}
%     \label{fig:1c}
%   \end{subfigure}%  
% }
%   \caption{Example of generated answers that contains errors.(b) the keyword 'hai người' (two people) is given  instead of 'ba người' (three people). Coincidentally, the question and groundtruth in this case both share the same phrase "đứng bên phải chàng trai" ("on the right of the man"), }\label{fig:1}
% \end{figure}


For better understand the generation performance on the VQA task, we examine the generated answers of our best ensemble, ConvS2S
+ViT-B/16+OFA$_{\mathrm{large}}$, to identify the limitations and analyze factors that may cause the model to perform poorly.
Through the error analysis process, various errors and mistakes have been pointed out in the outputs of the model. The typical examples of various types of errors are illustrated in Figure \ref{fig:1}. In summary, we divide these errors into three groups:

\begin{itemize}
    \item The generated answer does not match the question and has no correct tokens compared with the ground truth answer, as shown in Figure \ref{fig:1a}. This error case sometimes accompanied by text degeneration.
    \item The generated response gives the wrong answer to the question but share some insignificant tokens with the ground truth answer, as shown in Figure \ref{fig:1b}, which significantly improves the evaluation score. This incorrect scenario exemplifies the limitation of the evaluation measures.
    \item The model managed to generate the correct key answer while also adding unnecessary information compared to the ground truth, which may lead to the response's meaning being distorted. 
    As shown in Figure \ref{fig:1c}, the model correctly predicted quantity but then added unnecessary tokens afterward, resulting in a low score on both evaluation metrics.
\end{itemize}


% \begin{figure*}[h]
% \centering
%   \begin{tabular}{@{}ccc@{}}
%     \includegraphics[width=0.3\textwidth]{figure/5.3_ex/00000000111.jpg}
%     \includegraphics[width=0.3\textwidth]{example-image-b} &
%     \includegraphics[width=0.3\textwidth]{example-image-b} \\
%   \end{tabular}
%   \caption{This is some figure side by side}
% \end{figure*}


\section{Future Directions}
Based on the results from our analysis, we suggest several future directions for both Vietnamese monolingual language models and Vietnamese MRC benchmarks.
\subsection{Language Models}
Our analysis shows that monolingual models, especially PhoBERT, acquire comparable abilities in recognizing the differences in lexical information between unanswerable questions and the given context. However, monolingual models show poor performances when encountering unanswerable questions that require the ability to comprehend a bigger ``picture''. For example, while monolingual models perform very well on unanswerable questions that use explicit antonyms, they often have difficulties in recognizing unanswerable questions when these questions are created using implicit antonyms. We explain this phenomenon by the findings of \citet{zhang-etal-2021-need} as pre-training language models on larger text copora results in significant improvement on downstream tasks that require high-level semantic and factual knowledge such as Machine Reading Comprehension. Therefore, when encountering unanswerable questions that require ability to grasp big ``picture,'' models pre-trained with smaller text corpora will show lower performances. Hence, the small size of pre-training corpora of PhoBERT and WikiBERT may be the main reason for their poor performances in MRC.

Scaling the pre-training data size of PhoBERT will further develop this model and empower it to achieve state-of-the-art performances on different benchmarks of Machine Reading Comprehension. Besides, we believe that introducing a new unsupervised task for the pre-training phase that focuses on improving the high-level semantic and factual knowledge of pre-trained models also plays an integral role in developing language models in the future.
\subsection{Benchmarks}
\textbf{Unanswerable Questions. } Although UIT-ViQuAD 2.0 successfully further introduced new types of artificially unanswerable questions, our work in Section 5 shows that current unanswerable questions in the development test of UIT-ViQuAD 2.0 are still not challenging enough. In order to increase the challenging levels of unanswerable questions, we believe that more high-quality works on adversarial human annotation for unanswerable questions are needed. These works can follow the guidelines of adversarial human annotation for answerable questions \cite{bartolo-etal-2020-beat}. Results of these works can reveal different techniques to annotate hard unanswerable questions and therefore be valuable for improving the guidelines for unanswerable questions annotation for Machine Reading Comprehension.\\
\textbf{Quality of Benchmark. } On the other hand, as we have shown in section 5, although PhoBERT and XLM-RoBERTa achieve high performances on the UIT-VinewsQA development set, our unanswerable questions reveal that these two models do not truly understand the context to give the correct answer for questions in the original development set. We hypothesize that questions in UIT-VinewsQA enable machine reading comprehension systems with shortcut learning knowledge \cite{lai-etal-2021-machine} to achieve high performance due to biases in annotating process. Therefore, we believe that studies on how Vietnamese machine reading comprehension systems are currently evaluated are important for tracking the progress of Vietnamese language systems.


% The 9 pages allocated for the main paper must include a broader impact
% statement regarding the approach, datasets and applications proposed/used in
% your paper. It should reflect on the environmental, ethical and societal
% implications of your work. The statement should require at most one page and
% must be included both at submission and camera-ready time.
%
% If authors have reflected on their work and determined that there are no
% likely negative broader impacts, they may use the following statement:
%
% After careful reflection, the authors have determined that this work presents
% no notable negative impacts to society or the environment.
%
% This section is included in the template as a default, but you can also place these
% discussions anywhere else in the main paper, e.g., in the introduction/future work.
%
% The Centre for the Governance of AI has written an excellent guide for writing
% good broader impact statements (for the NeurIPS conference) that may be a
% useful resource for AutoML-Conf authors:
%
% https://medium.com/@GovAI/a-guide-to-writing-the-neurips-impact-statement-4293b723f832

{\small
\bibliographystyle{apalike}
\bibliography{egbib_v2}
}

\section{Submission Checklist}
% The submission checklist is a combination of the NeurIPS '21 checklist:
%
%   https://neurips.cc/Conferences/2021/PaperInformation/PaperChecklist
%
% and the NAS checklist:
%
%   https://www.automl.org/wp-content/uploads/NAS/NAS_checklist.pdf
%
% For each question, change the default \answerTODO{} to either:
%
%     \answerYes{[justification]},
%     \answerNo{[justification]}, or
%     \answerNA{[justification]}.
%
% *You must include a brief justification to your answer,* either by
% referencing the appropriate section of your paper or providing a brief inline
% description.  For example:
%
% - Did you include the license of the code and datasets?
%   \answerYes{See Section~\ref{sec:code}.}
%
% - Did you include all the code for running experiments?
%   \answerNo{We include the code we wrote, but it depends on proprietary
%   libraries for executing on a compute cluster and as such will not be
%   runnable without modifications. We also include a runnable sequential
%   version of the code that we also report experiments in the paper with.}
%
% - Did you include the license of the datasets?
%   \answerNA{Our experiments were conducted on publicly available datasets and
%   we did not introduce new datasets.}
%
% Please note that if you answer a question with \answerNo{}, we expect that you
% compensate for it (e.g., if you cannot provide the full evaluation code, you
% should at least provide code for a minimal reproduction of the main insights
% of your paper).
%
% Please do not modify the questions and only use the provided macros for your
% answers. Note that this section does not count towards the page limit.

\begin{enumerate}
\item For all authors\dots
  %
  \begin{enumerate}
  \item Do the main claims made in the abstract and introduction accurately
    reflect the paper's contributions and scope?
    %
    \answerYes{}
    %
  \item Did you describe the limitations of your work?
    %
    \answerYes{See Section~\ref{sec:limit}}
    %
  \item Did you discuss any potential negative societal impacts of your work?
    %
    \answerNA{We do not have any potential negative societal impacts}
    %
  \item Have you read the ethics author's and review guidelines and ensured that
    your paper conforms to them? \url{https://automl.cc/ethics-accessibility/}
    %
    \answerYes{}
    %
  \end{enumerate}
  %
\item If you are including theoretical results\dots
  %
  \begin{enumerate}
  \item Did you state the full set of assumptions of all theoretical results?
    %
    \answerNA{We are not presenting theoretical results}
    %
  \item Did you include complete proofs of all theoretical results?
    %
    \answerNA{We are not presenting theoretical results}
    %
  \end{enumerate}
  %
\item If you ran experiments\dots
  %
  \begin{enumerate}
  \item Did you include the code, data, and instructions needed to reproduce the
    main experimental results, including all requirements (e.g.,
    \texttt{requirements.txt} with explicit version), an instructive
    \texttt{README} with installation, and execution commands (either in the
    supplemental material or as a \textsc{url})?
    %
    \answerYes{We provided the README file on how to reproduce experiment results and we also included the requirements.txt file in Github repo.}
    %
  \item Did you include the raw results of running the given instructions on the
    given code and data?
    %
    \answerYes{}
    %
  \item Did you include scripts and commands that can be used to generate the
    figures and tables in your paper based on the raw results of the code, data,
    and instructions given?
    %
    \answerYes{We provided the README file on how to reproduce experiment result.}
    %
  \item Did you ensure sufficient code quality such that your code can be safely
    executed and the code is properly documented?
    %
    \answerYes{}
    %
  \item Did you specify all the training details (e.g., data splits,
    pre-processing, search spaces, fixed hyperparameter settings, and how they
    were chosen)?
    %
    \answerYes{}
    %
  \item Did you ensure that you compared different methods (including your own)
    exactly on the same benchmarks, including the same datasets, search space,
    code for training and hyperparameters for that code?
    %
    \answerYes{We made a comparison with exactly same datasets and hyperparameters.}
    %
  \item Did you run ablation studies to assess the impact of different
    components of your approach?
    %
    \answerNA{We did not run ablation study.}
    %
  \item Did you use the same evaluation protocol for the methods being compared?
    %
    \answerYes{Yes we use the same evaluation protocol while making performance comparison.}
    %
  \item Did you compare performance over time?
    %
    \answerYes{See Section~\ref{sec:empirical method}}
    %
  \item Did you perform multiple runs of your experiments and report random seeds?
    %
    \answerYes{See Section~\ref{sec:empirical method}}
    %
  \item Did you report error bars (e.g., with respect to the random seed after
    running experiments multiple times)?
    %
    \answerYes{See Section~\ref{sec:empirical method}}
    %
  \item Did you use tabular or surrogate benchmarks for in-depth evaluations?
    %
    \answerYes{We used surrogate benchmarks which are CIFAR100 and MiniImageNet for image classification. See Section~\ref{sec:setup} }
    %
  \item Did you include the total amount of compute and the type of resources
    used (e.g., type of \textsc{gpu}s, internal cluster, or cloud provider)?
    %
    \answerYes{See Section~\ref{sec:technical_details}}
    %
  \item Did you report how you tuned hyperparameters, and what time and
    resources this required (if they were not automatically tuned by your AutoML
    method, e.g. in a \textsc{nas} approach; and also hyperparameters of your
    own method)?
    %
    \answerYes{See Section~\ref{sec:empirical method} and Section~\ref{sec:technical_details}}
    %
  \end{enumerate}
  %
\item If you are using existing assets (e.g., code, data, models) or
  curating/releasing new assets\dots
  %
  \begin{enumerate}
  \item If your work uses existing assets, did you cite the creators?
    %
    \answerYes{See Section~\ref{sec:empirical method}}
    %
  \item Did you mention the license of the assets?
    %
    \answerYes{See Section~\ref{sec:technical_details}. We have used open-source datasets and frameworks that are publicly available and can be used without any licensing issues or legal restrictions.}
    %
  \item Did you include any new assets either in the supplemental material or as
    a \textsc{url}?
    %
    \answerYes{We provide all our code anonymously via \url{https://anon-github.automl.cc/r/Adaptive_Regularization-71B7}.}
    %
  \item Did you discuss whether and how consent was obtained from people whose
    data you're using/curating?
    %
    \answerNA{}
    %
  \item Did you discuss whether the data you are using/curating contains
    personally identifiable information or offensive content?
    %
    \answerNA{}
    %
  \end{enumerate}
  %
\item If you used crowdsourcing or conducted research with human subjects\dots
  %
  \begin{enumerate}
  \item Did you include the full text of instructions given to participants and
    screenshots, if applicable?
    %
    \answerNA{}
    %
  \item Did you describe any potential participant risks, with links to
    Institutional Review Board (\textsc{irb}) approvals, if applicable?
    %
    \answerNA{}
    %
  \item Did you include the estimated hourly wage paid to participants and the
    total amount spent on participant compensation?
    %
    \answerNA{}
    %
  \end{enumerate}
\end{enumerate}

% content will be automatically hidden during submission
%\begin{acknowledgements}

%\end{acknowledgements}

% print bibliography -- for bibtex / natbib, use:

% \bibliography{...}

% and for biber / biblatex, use:

% \printbibliography

% supplemental material -- everything hereafter will be suppressed during
% submission time if the hidesupplement option is provided!
%\appendix

\section{Appendix}
\label{sec:technical_details}
We have built our experiments on PyCIL \citep{zhou2021pycil} (MIT License) and RayTune \citep{liaw2018tune} (Apache License 2.0) frameworks. Our experiments were run on NVIDIA A100  taking around 148 GPU hours. Total emissions are estimated to be around 15 kg CO2.

\end{document}
