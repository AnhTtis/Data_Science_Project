\documentclass[twoside]{article}


\usepackage{graphicx}
\usepackage{tabularx}
\usepackage{adjustbox}
\usepackage{pifont}
\usepackage{algorithm}
\usepackage{algpseudocode}
\usepackage{enumitem}
\usepackage{amsmath}
\usepackage{xcolor}
\usepackage{subcaption}
\usepackage{arydshln}
\usepackage{amssymb}
\usepackage{url}
% If your paper is accepted, change the options for the package
% claiunconf2023 as follows:
%
\usepackage[accepted]{claiunconf2023}

%
% This option will print headings for the title of your paper and
% headings for the authors names, plus a copyright note at the end of
% the first column of the first page.

% If you set papersize explicitly, activate the following three lines:
%\special{papersize = 8.5in, 11in}
%\setlength{\pdfpageheight}{11in}
%\setlength{\pdfpagewidth}{8.5in}

% If you use natbib package, activate the following three lines:
\usepackage[round]{natbib}
%\renewcommand{\bibname}{References}
%\renewcommand{\bibsection}{\subsubsection*{\bibname}}

% If you use BibTeX in apalike style, activate the following line:
\bibliographystyle{apalike}


\begin{document}

% If your paper is accepted and the title of your paper is very long,
% the style will print as headings an error message. Use the following
% command to supply a shorter title of your paper so that it can be
% used as headings.
%
%\runningtitle{I use this title instead because the last one was very long}

% If your paper is accepted and the number of authors is large, the
% style will print as headings an error message. Use the following
% command to supply a shorter version of the authors names so that
% they can be used as headings (for example, use only the surnames)
%
%\runningauthor{Surname 1, Surname 2, Surname 3, ...., Surname n}

\twocolumn[

\CLAIUnconftitle{AdaCL: Adaptive Continual Learning}

\CLAIUnconfauthor{ Elif Ceren Gok Yildirim \And Murat Onur Yildirim \And  Mert Kilickaya \And Joaquin Vanschoren }

\CLAIUnconfaddress{ Automated Machine Learning Group, Eindhoven University of Technology } ]

\begin{abstract}
Class-Incremental Learning aims to update a deep classifier to learn new categories while maintaining or improving its accuracy on previously observed classes. Common methods to prevent forgetting previously learned classes include regularizing the neural network updates and storing exemplars in memory, which come with hyperparameters such as the learning rate, regularization strength, or the number of exemplars. However, these hyperparameters are usually only tuned at the start and then kept fixed throughout the learning sessions, ignoring the fact that newly encountered tasks may have varying levels of novelty or difficulty. This study investigates the necessity of hyperparameter `adaptivity' in Class-Incremental  Learning: the ability to dynamically adjust hyperparameters such as the learning rate, regularization strength, and memory size according to the properties of the new task at hand. We propose AdaCL, a Bayesian Optimization-based approach to automatically and efficiently determine the optimal values for those parameters with each learning task. We show that 
 adapting hyperpararmeters on each new task leads to improvement in accuracy, forgetting and memory. Code is available at \url{https://github.com/ElifCerenGokYildirim/AdaCL}.

\end{abstract}

\section{Introduction}


Recent years have witnessed the rise of human digitization~\cite{habermannDeepCapMonocularHuman2020,alexanderCREATINGPHOTOREALDIGITAL,pengNeuralBodyImplicit2021,alldieckDetailedHumanAvatars2018, rajANRArticulatedNeural2020}. This technology greatly impacts the entertainment, education, design, and engineering industry.
There is a well-developed industry solution for this task.
High-fidelity reconstruction of humans can be achieved either with full-body laser scans~\cite{saitoSCANimateWeaklySupervised2021}, dense synchronized multi-view cameras~\cite{xiangModelingClothingSeparate2021a,xiangDressingAvatarsDeep2022a}, or light stages~\cite{alexanderCREATINGPHOTOREALDIGITAL}.
However, these settings are expensive and tedious to deploy and consist of a complex processing pipeline, preventing the technology's democratization.

Another solution is to view the problem as inverse rendering and learn digital humans directly from custom-collected data.
Traditional approaches directly optimize explicit mesh representation~\cite{loperSMPLSkinnedMultiperson2015, fangRMPERegionalMultiperson2018, pavlakosExpressiveBodyCapture2019} which suffers from the problems of smooth geometry and coarse textures~\cite{prokudinSMPLpixNeuralAvatars2020,alldieckVideoBasedReconstruction2018}. Besides, they require professional artists to design human templates, rigging, and unwrapped UV coordinates.
Recently, with the help of volumetric-based implicit representations~\cite{mildenhallNeRFRepresentingScenes2020, parkDeepSDFLearningContinuous2019, meschederOccupancyNetworksLearning2019} and neural rendering~\cite{laineModularPrimitivesHighPerformance2020, liuSoftRasterizerDifferentiable2019, thiesDeferredNeuralRendering2019}, 
one can easily digitize a quality-plausible human avatar from video footage~\cite{jiangNeuManNeuralHuman2022,wengHumanNeRFFreeviewpointRendering}.
Particularly, volumetric-based implicit representations~\cite{mildenhallNeRFRepresentingScenes2020, pengNeuralBodyImplicit2021} can reconstruct scenes or objects with much higher fidelity against previous neural renderer~\cite{thiesDeferredNeuralRendering2019,prokudinSMPLpixNeuralAvatars2020}, and is more user-friendly as it does not need any human templates, pre-set rigging, or UV coordinates.
Captured visual footage and corresponding skeleton tracking are enough for training.
However, better reconstructions and more friendly usability are at the expense of the following factors.
1) \textbf{Inefficiency:}
They require longer optimization times (typically tens of hours or days) and inference slowly.
Volume rendering~\cite{mildenhallNeRFRepresentingScenes2020,lombardiNeuralVolumesLearning2019} formulates images by querying the densities and colors of millions of spatial coordinates. 
In the training stage, due to memory constraints, only a small fraction of points are sampled which leads to slow convergence speed.
2) \textbf{Entangled representations}:
The geometry, materials, and motion dynamics are entangled in the neural networks. 
Due to the implicit nature of neural nets, one can hardly edit one property without touching the others~\cite{yuanNeRFEditingGeometryEditing2022a,liuEditingConditionalRadiance2021}.
3) \textbf{Graphics incompatibility}:
Volume rendering is incompatible with the current popular graphic pipeline,
which renders triangular/quadrilateral meshes efficiently with the rasterization technique.
Many downstream applications require mesh rasterization in their workflow (\eg, editing~\cite{foundationBlenderOrgHome}, simulation~\cite{benderPositionBasedSimulationMethods2015}, real-time rendering~\cite{akenine2019real}, ray-tracing~\cite{waldRTXRayTracing}).
Although there are approaches~\cite{lorensenMarchingCubesHigh,labelleIsosurfaceStuffingFast2007} can convert volumetric fields into meshes, the gaps from discrete sampling degrade the output quality in terms of both meshes and textures.


To address these issues, we present \textbf{EMA}, a method based on \textbf{E}fficient \textbf{M}eshy neural fields to reconstruct animatable human \textbf{A}vatars.
Our method enjoys flexibility from implicit representations and efficiency from explicit meshes, yet still maintains high-fidelity reconstruction quality.
Given video sequences and the corresponding pose tracking, our method digitizes humans in terms of canonical triangular meshes, physically-based rendering (PBR) materials, and skinning weights \textit{w.r.t.} skeletons.
We jointly learn the above components via inverse rendering~\cite{laineModularPrimitivesHighPerformance2020,chenDIBRLearningPredict2021,chenLearningPredict3D2019} in an end-to-end manner.
Each of them is derived from a separate neural field, which relaxes the requirements of a preset human template, rigging, or UV coordinates.
Specifically, we predict a canonical mesh out of a signed distance field (SDF) by differentiable marching tetrahedra~\cite{shenDeepMarchingTetrahedra2021,gaoGET3DGenerativeModel,gaoLearningDeformableTetrahedral2020,munkbergExtractingTriangular3D2022}, then we extend the marching tetrahedra~\cite{shenDeepMarchingTetrahedra2021} for spatial-varying materials by utilizing a neural field to predict PBR materials \textit{on the mesh surfaces} after rasterization~\cite{munkbergExtractingTriangular3D2022,hasselgrenShapeLightMaterial2022,laineModularPrimitivesHighPerformance2020}.
To make the canonical mesh animatable, we take another neural field to model the forward linear blend skinning for the meshes. 
Given a posed skeleton, the canonical mesh is then transformed into the corresponding poses.
Finally, we shade the mesh with a rasterization-based differentiable renderer~\cite{laineModularPrimitivesHighPerformance2020} and train our models with a photo-metric loss.
After training, we export the mesh with materials and discard the neural fields.

\looseness=-1
There are several merits of our method design.
1) \textbf{Efficiency}:
Powered by efficient mesh rendering, our method can render in real-time.
Besides, the training speed is boosted as well, 
since we compute loss holistically on the whole image and the gradients only flow on the mesh surface. In contrast, volume rendering takes limited pixels for loss computation and back-propagates the gradients in the whole space.
Our method only needs about an hour of training and minutes of optimization are enough for plausible avatar reconstruction.
2) \textbf{Disentangled representations}:
Our shape, materials, and motion modules are disentangled naturally by design, which facilitates editing. 
Besides, Canonical meshes with forward skinning modeling handle the out-of-distribution poses better.
3) \textbf{Graphics compatibility}:
Our derived mesh representation is compatible with 
the prominent graphic pipeline, which leads to instant downstream applications (\eg, the shape and materials can be edited directly in design software~\cite{foundationBlenderOrgHome}).
To further improve reconstruction quality, we additionally optimize image-based environment lights and non-rigid motions.


We conduct extensive experiments on standards benchmarks H36M~\cite{ionescuHuman36MLarge2014b} and ZJU-MoCap~\cite{pengNeuralBodyImplicit2021}.
Our method achieves very competitive performance for novel view synthesis, generalizes better for novel poses, 
and significantly improves both training time and inference speed against previous arts.
Our research-oriented code reaches real-time inference speed (100+ FPS for rendering $512\times512$ images).
We in addition showcase applications including novel pose synthesis, material editing, and relighting.
\section{Related Work}

\subsection{Video Summarization}
 Video summarization datasets (\eg,  SumMe~\cite{gygli2014creating}, TVSum~\cite{song2015tvsum}, and YouTube~\cite{de2011vsumm}) have enabled the development of state-of-the-art video summarization methods~\cite{narasimhan2021clip,zhang2016video,zhou2018deep,park2020sumgraph,saquil2021multiple}. Among these models, vsLSTM \cite{zhang2016video} first attempted to learn frame importance by modeling the temporal dependency among frames using LSTM \cite{graves2012long} units. The model can be combined with a determinantal point process (DPP) to improve the diversity of generated video summary. Following vsLSTM, several other approaches were proposed to model the temporal dependency, e.g., H-RNN~\cite{zhao2017hierarchical}, HSA-RNN\cite{zhao2018hsa}, DASP~\cite{ji2020deep}. Another solution models the spatiotemporal structure of the video to learn frame importance, such as MerryGoRoundNet~\cite{lal2019online}, and CRSum \cite{yuan2019spatiotemporal}. Adversarial learning-based methods~\cite{fu2019attentive,zhang2019dtr} can also perform well. Recently, multimodal-based video summarization method~\cite{narasimhan2021clip} leverages generated text summaries to promote predictions of frame-level scores for video summaries. Different from multimodal-based video summarization, the cross-modal video summarization task requires simultaneously producing both visual and textual summaries from a source video, which goes beyond generic video summarization. Moreover, it ensures semantic coherence between these two modalities.


\vspace{-2mm}
\subsection{Video Captioning}
% Video Captioning aims to automatically generate short descriptions for a video by understanding the action and event in a video, which can help retrieve videos efficiently through text queries.
Video Captioning aims to describe a video with text, which requires the capability of understanding actions and events.
Existing benchmarks (e.g., MSVD~\cite{chen2011collecting}, YouCook~\cite{zhou2018towards}, MSR-VTT~\cite{xu2016msr}, and ActivityNet Captions \cite{krishna2017dense}) have helped to promote the ability of language models to generate reasonable captions for video. Benefiting from these human-annotated datasets, many novel approaches are proposed. 
Attention-based methods~\cite{yao2015describing,yan2019stat} employ attention mechanisms to help the model in associating relevant frames since not every frame in a video is equally important.
%Graph-based video captioning~\cite{zhang2019object,zhang2020video} leverage inter-frame and intra-frame object interactions in both time and space degree. 
DENSE~\cite{krishna2017dense} is an early attempt at dense video captioning, which detects events with an event proposal module and associates them with LSTM. Wang et al.~\cite{wang2018bidirectional} develop a bidirectional process to encode context for detecting event proposals. Moreover, Masked Transformer~\cite{zhou2018end} proposes a differentiable masking scheme to ensure consistency between event proposal and caption generation modules.




\vspace{-2mm}
\subsection{Multimodal Pretraining}
Large language models (LLMs) \cite{brown2020language,devlin2018bert,lewis2019bart,raffel2020exploring} have revolutionized NLP research in recent years. Following the large-scale pretraining models in the field of NLP, numerous works \cite{ju2022prompting,kim2021vilt,wang2020minilm,xue2021probing,zhang2021vinvl,hu2022promptcap} on exploring the combination of vision and language (VL) pretraining have achieved great success. Since then, image-text pretraining has become
a default approach to tackling VL tasks \cite{biten2019scene,lin2014microsoft,regneri2013grounding,singh2019towards}. In addition, the introduction of Vision Transformers \cite{dosovitskiy2020image} enables vision and language modalities to be jointly modeled by Transformers in a more scalable fashion \cite{alayrac2022flamingo,wang2022git,yu2022coca,yuan2021florence}. According to the encoding strategies for image and language modalities, VL models can be categorized into fusion encoder \cite{li2019visualbert,lu2019vilbert,su2019vl,tan2019lxmert}, dual encoder \cite{radford2021learning}, and a combination of both \cite{bao2021vlmo,du2022survey,singh2022flava}. Several video-language pretrained models have also shown strong performance on video captioning and other video tasks, such as HERO \cite{li2020hero}, VideoBERT \cite{sun2019videobert}, and UniVL \cite{luo2020univl}.
In this work, cross-modal video summarization requires models with strong video understanding and language modeling capabilities. Therefore, this new task provides a practical scenario to assess the superiority of multimodal pretrained models.


The ARMBench dataset presents: 1) a collection of sensor data acquired by a robotic manipulation workcell performing pick-and-place operation, 2) metadata and reference images for objects in containers, 3) a set of annotations acquired either automatically, by virtue of the system design, or via manual labeling, and 4) tasks and metrics to benchmark perception algorithms for robotic manipulation. Fig.\ \ref{fig:contributions} illustrates the benchmark tasks and variety of objects captured in the dataset. The dataset captures diversity in objects with respect to Amazon product categories as well as physical characteristics such as size, shape, material, deformability, appearance, fragility, etc. 

The data collection platform is a robotic manipulation workcell performing pick-and-place operation in a warehouse \cite{Sparrow2022}. The workcell contains a robotic arm mounted with a vacuum-based end-effector. It is presented with a heterogeneous collection of objects placed in unstructured configurations within a container (storage tote). The robotic arm is tasked with picking one object at a time (singulation) and place it on moving trays until the container is empty. The empty container ejects the workcell and is replaced by a new container. While the operation is completely autonomous, it includes a human-in-the-loop to monitor the status of each pick-and-place activity, annotate, and resolve any defects during manipulation. Multiple imaging sensors are placed in the workcell to facilitate and validate the pick-and-place operation. Following is a list of sensor data (Fig.\ \ref{fig:intro}) associated with each pick activity:
\begin{itemize}
\item Pick-image: A 5\,MP camera is used to capture a top-down image of the container.
% \item Pick-3D: Two Ensenso sensors capture the 3D point cloud of the source container.
\item Transfer-images: Multiple 5\,MP cameras are placed on different sides in the workcell to capture the moving object from different viewpoints.
% \item Transfer-Barcode: Multiple Cognex barcode sensors are used to scan the barcode of the object during transfer.
\item Place-image: A top-down view of the object is captured once it is placed on the tray.
\item Video: A camera is mounted to capture 720p videos of pick-and-place manipulation processes at 30\,FPS
\end{itemize}
Additionally, the following metadata (Fig.\ \ref{fig:contributions} (b)) is available by virtue of a warehouse tracking system:
\begin{itemize}
\item Container-manifest: A list of objects present in the container along with data such as product description, coarse dimensions, and weight.
\item Reference images: One or more images of objects from previous operations within the warehouse.
\end{itemize}
The sensor data and metadata were consumed by perception algorithms required to autonomously operate the robotic workcell. Benchmarking against these algorithms would not only optimize a manipulation task such as the one used for data collection but also enable more complex and intentional manipulation. This work considers a subset of such perception tasks namely object segmentation, object identification, and defect detection. These are critical not only to make informed grasping and motion decisions but also to track the state of the objects and containers within the warehouse. The following sections will describe these tasks and present the challenges using annotations, baseline algorithms, and evaluation metrics.

\vspace{-0.3em}
\section{Method}
\vspace{-0.3em}

Our sensitivity-aware visual parameter-efficient fine-tuning consists of two stages. In the first stage, SPT measures the task-specific sensitivity for the pre-trained parameters (Section~\ref{subsec:sensitivity}). Based on the parameter sensitivity and a given parameter budget, SPT then adaptively allocates trainable parameters to task-specific important positions (Section~\ref{subsec:SPT}).

\vspace{-0.3em}
\subsection{Task-specific Parameter Sensitivity}
\label{subsec:sensitivity}
\vspace{-0.3em}

Recent research has observed that pre-trained backbone parameters exhibit varying feature patterns~\cite{raghu2021vision,naseer2021intriguing} and criticality~\cite{zhang2019all,chatterji2019intriguing} at distinct positions. 
Moreover, when transferred to downstream tasks, their efficacy varies depending on how much pre-trained features are reused and how well they adapt to the specific domain gap~\cite{yosinski2014transferable,kumar2022finetuning,neyshabur2020being}. Motivated by these observations, we argue that not all parameters contribute equally to the performance across different tasks in PEFT and propose a new criterion to measure the sensitivity of the parameters in the pre-trained backbone for a given task.

Specifically, given the training dataset $\gD_t$ for the $t$-th task and the pre-trained model weights $\vw=\left\{w_1, w_2, \ldots, w_N\right\}\in \sR^N$ where $N$ is the total number of parameters, the objective for the task is to minimize the empirical risk: $\min_{\vw} E(\gD_t, \vw)$.
We denote the parameter sensitivity \bohan{set} as $\gS=\{s_1, \ldots, s_N\}$ and the sensitivity $s_n$ for parameter $w_n$ is measured by the empirical risk difference when tuning it:
\begin{equation}
\vspace{-0.3em}
    \begin{aligned}
        s_n = E(\gD_t, \vw)-E(\gD_t, \vw\mid w_n=w_n^*),
    \end{aligned}
\label{eq:sensitivity}
\end{equation}
where $w_n^*=\underset{w_n}{\rm argmin}(E(\gD_t, \vw))$. We can reparameterize the tuned parameters as  $w_n^*=w_n+\Delta_{w_n}$, where $\Delta_{w_n}$ denotes the update for $w_n$ after tuning. Here we individually measure the sensitivity of each parameter, which is reasonable given that most of the parameters are frozen during fine-tuning in PEFT. However, it is still computationally intensive to compute Eq.~(\ref{eq:sensitivity}) for two reasons. Firstly, getting the empirical risk for $N$ parameters requires forwarding the entire network $N$ times, which is time-consuming. Secondly, it is challenging to derive $\Delta_{w_n}$, as we have to tune each individual $w_n$ until convergence.

{\begin{algorithm}[t!]
\caption{\label{alg:tps} Computing task-specific parameter sensitivities}
\begin{algorithmic}
    \STATE \textbf{Input:} Pre-trained model with network parameters $\vw$, training set $\gD_t$ for the $t$-th task, and number of training samples $C$ used to calculate the parameter sensitivities
    \STATE \textbf{Output:} Sensitivity set $\gS=\{s_1, \ldots, s_N\}$
    \STATE Initialize $\gS=\{0\}^N$
    \FOR{$i\in\{1,\ldots,C\}$}
        \STATE Get the $i$-th training sample of $\gD_t$
	    \STATE Compute loss $E$
		\STATE Compute gradients $\vg$
		\FOR{$n\in\{1,\ldots,N\}$}
                \STATE Update sensitivity for the $n$-th parameter: $s_{n} = s_{n} + g_n^2$
		    \ENDFOR
    \ENDFOR
\end{algorithmic}
\end{algorithm}}


\begin{figure*}[t]
\begin{center}
    \includegraphics[width=\linewidth]{main_figure.pdf}
\end{center}\vspace{-2em}
\caption{Overview of our trainable parameter allocation strategy. With the parameter sensitivity \bohan{set} $\gS$, we first get the top-$\tau$ sensitive parameters. Instead of directly tuning these sensitive parameters, we also boost the representational capability by replacing unstructured tuning with structured tuning at sensitive weight matrices that have a large number of sensitive parameters, which can be implemented by an existing structured tuning method, \eg, LoRA~\cite{hu2022lora} and Adapter~\cite{houlsby2019parameter}. Red lines and blocks represent trainable parameters and modules, while blue lines represent frozen parameters.}
\label{fig:main}
\vspace{-1.5em}
\end{figure*}


To overcome the first barrier, we simplify the empirical loss by approximating $s_n$ in the vicinity of $\vw$ by its first-order Taylor expansion
\vspace{-0.3em}
\begin{equation}
\vspace{-0.5em}
    \begin{aligned}
        s_n^{(1)} = -g_n\Delta_{w_n},
    \end{aligned}
\label{eq:first-order}
\end{equation}
where the gradients $\vg=\partial E/\partial\vw$, and $g_n$ is the gradient of the $n$-th element of $\vg$. 
To address the second barrier, following~\cite{liu2018darts,cai2018proxylessnas}, we take the one-step unrolled weight as the surrogate for $w_n^*$ and approximate $\Delta_{w_n}$ in Eq.~(\ref{eq:first-order}) with a single step of gradient descent. We can accordingly get $s_n^{(1)} \approx g_n^2\epsilon$,
where $\epsilon$ is the learning rate. Since $\epsilon$ is the same for all parameters, we can eliminate it when comparing the sensitivity with the other parameters and finally get 
\vspace{-0.5em}
\begin{equation}
\vspace{-0.3em}
    \begin{aligned}
        s_n^{(1)} \approx g_n^2.
    \end{aligned}
\label{eq:first-order-simp}
\end{equation}
Therefore, the sensitivity of a parameter can be efficiently measured by its potential to reduce the loss on the target domain. Note that although our criterion draws inspiration from pruning work~\cite{molchanov2019importance}, it is distinct from it. \cite{molchanov2019importance} measures the parameter importance by the squared change in loss when removing them, \ie, $\left( E(\gD_t, \vw)-E(\gD_t, \vw\mid w_n=0) \right)^2$ and finally derives the parameter importance by $\left( g_n w_n \right)^2$, which is different from our formulations in Eqs.~(\ref{eq:sensitivity}) and~(\ref{eq:first-order-simp}).

In practice, we accumulate $\gS$ from a total number of $C$ training samples ahead of fine-tuning to generate accurate sensitivity as shown in Algorithm~\ref{alg:tps}, where $C$ is a pre-defined hyper-parameter. In Section~\ref{subsec:abl}, we show that employing only 400 training samples is sufficient for getting reasonable parameter sensitivity, which requires only 5.5 seconds with a single GPU for any VTAB-1k dataset with ViT-B/16 backbone~\cite{vit}.

\vspace{-0.3em}
\subsection{Adaptive Trainable Parameters Allocation}
\label{subsec:SPT}
\vspace{-0.2em}

Our next step is to allocate trainable parameters based on the obtained parameter sensitivity set $\gS$ and a desired parameter budget $\tau$. A straightforward solution is to directly tune the top-$\tau$ most sensitive unstructured connections (parameters) \rev{while keeping the rest frozen}, which we name unstructured tuning. Specifically, we select the top-$\tau$ most sensitive weight connections in $\gS$ to form the sensitive weight connection set $\gT$. Then, for \rev{a} weight matrix $\mW\in \sR^{d_{\rm in}\times d_{\rm out}}$, we can get a binary mask $\mM\in \sR^{d_{\rm in}\times d_{\rm out}}$ computed by
\vspace{-0.5em}
\begin{equation}
\vspace{-0.5em}
    {\begin{array}{ll}
    \small
    \begin{aligned}
    \mM^j =
    \left\{\begin{array}{ll} 
    1 ~~~~~ \mW^j \in \gT \\
    0 ~~~~~ \mW^j \notin \gT
    \end{array}\right.
    \end{aligned},
    \small
    \end{array}}
\label{eq:mask}
\end{equation}
where $\mW^j$ and $\mM^j$ are the $j$-th element in $\mW$ and $\mM$, respectively. Accordingly, we can train the sensitive parameters by gradient descent and the updated weight matrix can be formulated as $\mW'\leftarrow \mW - \epsilon\vg_{\mW}\odot\mM$, where $\vg_{\mW}$ is the gradient for $\mW$.

However, considering PEFT approaches generally limit the proportion of trainable parameters to less than 1\%, tuning only a small number of unstructured weight connections might not have enough representational capability to handle the downstream datasets with large domain gaps from the source pre-training data. Therefore, to improve the representational capability, we propose to replace unstructured tuning with structured tuning at the sensitive weight matrices that have a high number of sensitive parameters. To preserve the parameter budget, we can implement structured tuning with an existing efficient structured tuning PEFT method~\cite{hu2022lora,chen2022adaptformer,houlsby2019parameter,jie2022convolutional} that learns to directly adjust \rev{all hidden dimensions at once}. We depict an overview of our trainable parameter allocation strategy in Figure~\ref{fig:main}. For example, we can employ the low-rank reparameterization trick LoRA~\cite{hu2022lora} to the sensitive weight matrices \rev{and the one-step update for $\mW$ can be formulated as}
\vspace{-0.4em}
\begin{equation}
\vspace{-0.4em}
    {\begin{array}{ll}
    \small
    \begin{aligned}
    \mW' = \left\{\begin{array}{ll} 
    \mW + \mW_{\rm down}\mW_{\rm up} & ~~ \text { if } ~~ \sum_{j=0}^{d_{\rm in}\times d_{\rm out}} \mM^j \geq \sigma_{\rm opt} \\
    \mW - \epsilon\vg_{\mW}\odot\mM & ~~ {\rm otherwise}
    \end{array}\right.
    \end{aligned},
    \small
    \end{array}}
\label{eq:weight_updat}
\end{equation}
where $\mW_{\rm down}\in \sR^{d_{\rm in}\times r}$ and $\mW_{\rm up}\in \sR^{r\times d_{\rm out}}$ are two learnable low-rank matrices to approximate the update of $\mW$ and rank $r$ is a hyper-parameter where $r \ll {\rm min}(d_{\rm in},d_{\rm out})$. In this way, we perform structured tuning on $\mW$ when its number of sensitive parameters exceeds $\sigma_{\rm opt}$, whose value depends on the pre-defined type of structured tuning method. For example, since implementing structured tuning with LoRA requires $2\times d_{\rm in} \times d_{\rm out} \times r$ trainable parameters for each sensitive weight matrix, we set $\sigma_{\rm LoRA} \leftarrow 2\times d_{\rm in} \times d_{\rm out} \times r$ to ensure that the number of trainable parameters introduced by structured tuning is always equal to or lower than the number of sensitive parameters.

In this way, our SPT adaptively incorporates both structured and unstructured tuning granularities to enable higher flexibility and stronger representational power, simultaneously. In Section~\ref{subsec:abl}, we show that structured tuning is important for the downstream tasks with larger domain gaps and both unstructured and structured tuning contribute clearly to the superior performance of our SPT.
\section{Experimental Setup}
\label{sec:setup}

%In this section, we provide the setup that we designed for our experiments. We follow the class incremental learning setup of the PYCIL framework \citep{zhou2021pycil} which only extends the classifier layer by the number of classes per task. PYCIL offers various continual learning baselines and we implement the hyperparameter optimization on two regularization methods which are EWC and LwF. We use the RayTune framework \citep{liaw2018tune} for the hyperparameter optimization and we did not use a pre-trained architecture and train all tasks from scratch. All our experiments run on a single-GPU setup of Nvidia A100.

%\subsection{Datasets and Architecture}
%We use the commonly used Split CIFAR100 dataset as in most of the CIL research and to be sure that results are more convincing, we also use mini-Imagenet dataset to further investigate our adaptive approach. We applied the same transformations defined in the PYCIL framework except we set resize to 64 for mini-Imagenet dataset to fasten the training process. For Split CIFAR-100 experiments we run all experiments with 3 different seeds to see the effect of task ordering and we observed that the task order does not have a significant effect in terms of performance. Therefore, we did not run mini-Imagenet experiments with different seeds. 

%\noindent
%\textbf{1) Split CIFAR-10:} It includes 10 object classes \citep{krizhevsky2009learning}. We used Split CIFAR-10 dataset to make a sensitivity analysis to observe how sensitive the EWC and LwF methods to their regularization hyperparameter.

%\noindent
%\textbf{2) Split CIFAR-100:} It includes 100 classes of visual object instances of CIFAR-100 \citep{krizhevsky2009learning}. In our setup, we randomly divide 100 classes into 10 tasks with 10 classes per task.

%\noindent
%\textbf{3) mini-Imagenet:} It is a variation of Imagenet dataset containing 100 classes of visual objects \citep{deng2009imagenet}. In our setup, we randomly divide 100 classes into 10 tasks with 10 classes per task.

%We used a specific version of the Resnets called Resnet32 \citep{he2016deep} for all our experiments to interpret the results more clearly.

%\subsection{Hyperparameters}

%EWC and LwF methods introduce an important lambda hyperparameter that controls the trade-off between stability and plasticity. In our experiments, we selected the baselines as vanilla EWC and vanilla LwF. In the vanilla settings, we set lambda to 1 which gives equal emphasis to current and previous tasks, and kept it fixed during the whole learning. On the other hand, in our adaptive approach, CARBON searches pre-defined search space and chooses the best lambda value based on validation accuracy. The search space for lambda is set to [1, $10^5$] for EWC and [1, 50] for the LwF based on our sensitivity analysis.

%\subsection{Performance Metrics}

%To measure the performance of each configuration we evaluated the model based on the validation data which is simply a portion of the training data. We derived two different accuracy metrics from ACC given in Eq(\ref{eqn:acc}) and called them incremental accuracy (last) and incremental accuracy (avg). Incremental accuracy (last) denotes the top-1 accuracy after the last task and it is a proper metric to measure the overall accuracy among all learned classes. The incremental accuracy often decreases with more tasks learned since CIL is continually adapted. However, only comparing incremental accuracy (last) ignores the performance evaluation along the learning trajectory. Therefore, we denoted incremental accuracy (avg) which considers the performance after every incremental stage. The higher value indicates a stronger performance along the incremental stages \citep{zhou2023deep}. We  also measure the level of forgetting with a backward transfer metric BWT Eq(\ref{eqn:bwt}) where $A_{T, i}$ is the test accuracy for task $i$ after training on task $T$. Higher BWT indicates a lower forget ratio \citep{kang2022forget}.

%\begin{equation}
%\label{eqn:acc}
%ACC = {\frac{1}{T}} \sum_{i=1}^{T} A_{T, i} 
%\end{equation}


%\begin{equation}
%\label{eqn:bwt}
%BWT = {\frac{1}{T-1}} \sum_{i=1}^{T-1} A_{T, i} - A_{i, i}
%\end{equation}


%%%%%%%%%%%%%%%%%%% Mert's Version %%%%%%%%%%%%%%%%%%%%%%%%%%%%%%%%%

 \partitle{Datasets.} In this paper, we experiment with \textbf{Split-CIFAR100}~\citep{krizhevsky2009learning} and \textbf{Split-MiniImageNet}~\citep{deng2009imagenet}. Each dataset exhibits objects from $100$ different categories, such as bird, snake and spider. We train all the models with $10$ tasks, with $10$ classes within each learning task on both CIFAR100 and MiniImageNet. Both datasets have 5000 training, and 1000 testing color images per learning task, each with $32\times32$ and $64\times64$ resolution for CIFAR100 and MiniImageNet respectively. 
%%%%% V1 %%%%%%%%%%%%%%%%%%
 
 %\partitle{Metrics.}  We evaluate the performance of our model using two standard metrics, top-1 accuracy and backward transfer for  
%$T$ tasks~\citep{diaz2018don}. Specifically, accuracy measures the test performance for all the observed tasks until $T$, whereas backward transfer measures how well the inclusion of task at time step $t-1$ influences the performance on task at previous time steps. Formally: 
%\begin{eqnarray}
%\mbox{\small {\bf Average Accuracy: } \normalsize ACC} & = & \frac{1}{T}
%\sum_{i=1}^T A_{T,i} \label{eq:acc} \\
%\mbox{\small {\bf Backward Transfer: } \normalsize BWT} & = & \frac{1}{T-1}
%\sum_{i=1}^{T-1} A_{T,i} - A_{i,i}   \label{eq:bwt} \\
%\end{eqnarray}
%\noindent where $A_{T,i}$ is the test accuracy of the model on task $i$ at time step $T$. For each metrics, the higher is better. 


%%%%%%%%%%%%%% V2 %%%%%%%%%%%%%%%%%%%%%

\partitle{Metrics.}  We resort to the standard metrics for evaluation, accuracy (ACC) which measures the final accuracy averaged over all tasks,  and backward transfer (BWT) which measures the average accuracy change of each task after learning new tasks. Formally for accuracy:
{\small
\vspace{-0.1cm}
\begin{align}
    ACC=\frac{1}{T}\sum\nolimits_{i=1}^T A_{T,i},
    \vspace{-0.1cm}
\end{align}}%

\noindent and for backward transferability: 
{\small
\begin{align}
    BWT=\frac{1}{T-1}\sum\nolimits_{i=1}^{T-1} (A_{T,i}-A_{i,i})
    \vspace{-0.1cm}
\end{align}}

\noindent where $A_{T,i}$ represents the testing accuracy of task $T$ after learning task $i$. In both cases, higher values indicate better performance. 
%%%%%%%%%%%%%%%%%%%%%%%%%%%%%%%%%%%%%%%%%%%%%%%%%

\section{Conclusion} 
We introduce \algname{}, a novel object detection method that achieves highly efficient inference speed while also improving zero-shot generalization compared with existing methods. The prompt-based decoding approach reduces the computational burden of object queries. The RoI-based masked attention and RoI pruning techniques allow us to efficiently leverage a large ViT-based CLIP model, enhancing detection performance through classification prediction ensembling. Comprehensive experiments show that \algname{} is $21.2$ times faster than OV-DETR while achieving comparable or higher APs on base and novel classes compared to two-stage OVD methods. %We believe that our work will inspire future work to explore the benefits of using Transformers.


%\paragraph{Ethics Statement.} 
%The focus of this paper is on open-vocabulary object detection. Our approach involves the integration of Transformer-based object detector and CLIP. We have not identified any foreseeable negative social impact associated with our work to share our findings with the scientific community. Nonetheless, we will continue to monitor and consider any potential concerns that may arise. 


%\bibliographystyle{apalike}
\bibliography{biblo}
%\section{MISSING PROOFS}

%The supplementary materials may contain detailed proofs of the results that are missing in the main paper.

%\subsection{Proof of Lemma 3}

%\textit{In this section, we present the detailed proof of Lemma 3 and then [ ... ]}

%\section{ADDITIONAL EXPERIMENTS}

%If you have additional experimental results, you may include them in the supplementary materials.

%\subsection{The Effect of Regularization Parameter}

%\textit{Our algorithm depends on the regularization parameter $\lambda$. Figure 1 below illustrates the effect of this parameter on the performance of our algorithm. As we can see, [ ... ]}

%\section{NON-TEXTUAL SUPPLEMENTARY MATERIAL}

%The (optional) non-textual supplementary material (e.g. code for the paper) should be submitted as a separate ZIP file.

%\begin{itemize}
%\item Your supplementary material file should be named 642-supp.zip (with 642
%  replaced with your paper ID on CMT).
%\item If you wish to include any code/datasets as part of the supplementary
%  material, you can: (i) include it in the ZIP file of your supplementary
%  material, and/or (ii) provide the public URL where the code can be found in
%  the appropriate field of CMT submission form.  If you promised to release
%  code/datasets (either in the original submission PDF or during the rebuttal),
%  the code/dataset must be released. In this case, the ContinualAI Unconference Chairs will
%  check the availability of the code/dataset and, if it isn’t available by the
%  camera-ready deadline, your submission may be excluded from the conference
%  proceedings.
%\item If you wish to include any **videos** as part of the supplementary material,
%  please do not include them in the ZIP file. Instead, simply provide the
%  public URL where the video is available in the appropriate field of the CMT
%  submission form. NOTE: This does not refer to the recorded presentation
% explaining your paper, which will be submitted/recorded separately, but to any other videos %containing, e.g., experimental results.
%\item Do not include any textual supplementary material in the ZIP file. It must be
%  included in the same PDF as the main paper.
%\end{itemize}


\vfill
\section{Training and Evaluation Details}

\begin{figure*}[t]
\begin{center}
\includegraphics[width=16.7cm]{figures/image_condition.pdf}
\end{center}
\vspace*{-0.3cm}
\hspace{2cm} (a) Grilled salmon \hspace{1.8cm} (b) Crepe \hspace{2.2cm} (c) Omelette \hspace{2.1cm} (d) Burrito
\vspace*{-0.4em}
\caption{Image-conditioned open-vocabulary detection. Image queries are from novel classes and the sub-captions represent text corresponding to the image queries.}
\label{fig:image_conditioned}
\vspace*{-0.4cm}
\end{figure*}


We assess our method by conducting experiments on two widely used open-vocabulary object detection datasets, namely OV-COCO and OV-LVIS. Following the literature\,\cite{guopen2022vild, zang2022open}, we split the classes in MS-COCO into 48 base categories and 17 novel categories, where the remaining categories are not included since they do not belong to a synset in the WordNet hierarhiy\,\cite{zareian2021open}. Regarding OV-LVIS, we follow the setup of \cite{zareian2021open}, selecting 337 rare classes as novel categories and the remaining classes as common and frequent categories. Because of the difference in the number of object categories, OV-LVIS has a much larger number of objects to detect. Additionally, the number of object queries should be larger than the number of actual objects in an image. Therefore, the number of object queries is set to 300 and 1,500 for OV-COCO and OV-LVIS. 

\smallskip\smallskip
\noindent\textbf{Training Loss Function.}
\algname{} utilize the loss function from (Deformable) DETR\,\cite{carion2020end, zhu2020deformable}, which includes a detection head that generates a set of bounding boxes. We modify this for OVD such that the detection head only generates the boxes and class labels given by the class prompts. During the training phase, the number of class prompts can vary according to the number of visible base classes in the current mini-batch. 

Hungarian matching is employed to identify a bipartite matching between the predicted boxes ${\rm \hat{B}}$ and the ground-truth boxes ${\rm {B}}$. The training process basically involves three types of losses: a classification loss $\ell_{\rm cls}$, which is the focal loss\footnote{Multi-label classification is performed only for the classes given by the class prompts.}, a box distance loss $\ell_{\rm l1}$ and a generalized IoU loss $\ell_{\rm iou}$ for box regression as \,\cite{songvidt, zhu2020deformable}:
 \begin{equation}
\begin{gathered}
\ell_{\rm cls}(i) = -\text{log}~\hat{\rm {P}}_{\sigma(i),c_i},~~~\ell_{\rm  l1}(i) =  ||{\rm B}_{i}-\hat{\rm B}_{\sigma(i)}||_{1},\\
\ell_{\rm iou}(i) = 1 \!-\! \big( \frac{|{\rm B}_{i} \cap  \hat{{\rm B}}_{\sigma(i)}|}{|{\rm B}_{i} \cup  \hat{{\rm B}}_{\sigma(i)}|} - \frac{|{\sf {\rm B}}({\rm B}_{i}, \hat{{\rm B}}_{\sigma(i)}) \symbol{92} {\rm B}_{i} \cup  \hat{{\rm B}}_{\sigma(i)}| }{|{\sf {\rm B}}({\rm B}_{i}, \hat{{\rm B}}_{\sigma(i)})|} \big),
\end{gathered}
\end{equation}
where $c_i$ is the target base class and $\sigma(i)$ is the bipartite assignment of the $i$-th ground-truth box. Also, the embedding loss $\ell_{\rm embed}$ proposed by \cite{zang2022open} is used to distil the CLIP's knowledge for open vocabulary object detection. Therefore, the final objective of \algname{} is formulated as:
\begin{equation}
\ell = \lambda_{\rm cls}\ell_{\rm cls} + \lambda_{\rm l1}\ell_{\rm l1} + \lambda_{\rm iou}\ell_{\rm iou} + \lambda_{\rm embed}\ell_{\rm embed}, 
\label{eq:naive_roi}
\end{equation}
where $\lambda$s are the balancing parameters, where $\lambda_{\rm cls}=3, \lambda_{\rm l1}=5, \lambda_{\rm iou}=2$, and $\lambda_{\rm embed}=2$. 

\smallskip\smallskip
\noindent\textbf{Evaluation.}
In the testing phase, the class prompts are inclusive of all base and novel classes following the recent literature\,\cite{guopen2022vild, zang2022open, zhong2022regionclip, zhou2022detic, rasheedbridging}. For producing the final results, we apply RoI-based pruning to identify the RoIs with high object scores. Next, the classification results of the selected bounding boxes are ensembled with those from the ViT-based CLIP by Eq.\,\eqref{eq:ensemble}. 
\section{Potential Enhancement}

We discuss three possible ways to further enhance the potential of our framework. 

\smallskip\smallskip
\noindent\textbf{Prediction Ensemble.} In our ensemble method, we adopt an arithmetic mean approach as shown in Eq. \eqref{eq:ensemble}. However, this requires hyperparameter tuning to achieve the best performance, and the results are sensitive to the selection of $\alpha$ values, as indicated in Table \ref{tab:alpha}. The variation in performance highlights the importance of carefully choosing the hyperparameters for \algname{}, as suboptimal selections can have a significant impact on the overall performance. Although the simple ensemble approach provides a satisfactory detection accuracy for OV-COCO and OV-LVIS, we are of the belief that a more effective prediction ensemble strategy can be studied as future work to leverage both CLIP and a detection model in synergy.



\smallskip\smallskip
\noindent\textbf{Unrestricted Setup.} Our primary focus is on developing an end-to-end Transformer-based framework that is both efficient and effective. Therefore, we adopt a restricted setup in which external data is not permitted, as our aim is to investigate the potential of the framework itself in achieving optimal performance. Despite our restricted setup, we acknowledge that in real-world scenarios, large-scale external data are often available and their utilization can significantly enhance the zero-shot detection performance. We are confident that our framework can be extended in this direction, owing to its simple end-to-end encoding and decoding pipeline. However, we leave this as a potential avenue for future work, as our current focus is on investigating the framework's performance under the restricted setup.


\smallskip\smallskip
\noindent\textbf{Instance Segmentation.} A drawback of \algname{} is the considerable discrepancy in the performance between object detection and instance segmentation. While this is partly due to inheriting the limitations of using SOLQ\,\cite{dong2021solq} for end-to-end joint learning, the development of a more effective segmentation approach based on DETR will enhance the performance of our framework significantly. We leave this as an avenue for future research.


\section{Image Conditioned Detection}

One advantage of our framework is that it can detect objects using an {image query} instead of relying solely on the class name. This feature has the potential to identify the target object by using a cropped image of the object itself, even when we may not know the object's name. 

Figure \ref{fig:image_conditioned} shows the results of \algname{} when applying this image-conditioned object detection using four different image queries. Although their class names are not provided, \algname{} is capable of accurately localizing target objects that were not encountered during the training phase.  Additionally, \algname{} can recognize new classes even when the image query has a dissimilar shape of target objects, as depicted in Figure \ref{fig:image_conditioned}(a). These findings suggest that our algorithm is resilient in detecting objects using image queries for open-set scenarios.
%


%We conducted further analysis on the qualitative results depicted in Figure~\ref{fig:image_conditioned} when applying image queries. The findings demonstrate that \algname{} is capable of accurately localizing target objects that were not encountered during the training phase. Moreover, we observe that \algname{} can identify new classes even when the image query has a dissimilar shape of target objects, as illustrated in Figure~\ref{fig:image_conditioned}(a). These results suggest that our algorithm is robust in detecting objects using image queries.

% [TBD] [TBD] [TBD] [TBD] [TBD] [TBD] [TBD] [TBD] [TBD] [TBD] [TBD] [TBD] [TBD] [TBD] [TBD] [TBD] [TBD] [TBD] [TBD] [TBD] [TBD] [TBD] [TBD] [TBD] [TBD] [TBD] [TBD] [TBD] [TBD] [TBD] [TBD] [TBD] [TBD] [TBD] [TBD] [TBD] [TBD] [TBD] [TBD] [TBD] [TBD] [TBD] [TBD] [TBD] [TBD] [TBD] [TBD] [TBD] [TBD] [TBD] [TBD] [TBD] [TBD] [TBD] [TBD] [TBD] [TBD] [TBD] [TBD] [TBD] [TBD] [TBD] [TBD] [TBD] [TBD] [TBD] [TBD] [TBD] [TBD] [TBD] [TBD] [TBD] [TBD] [TBD] [TBD] [TBD] [TBD] [TBD] [TBD] [TBD] [TBD] [TBD] [TBD] [TBD] [TBD] [TBD] [TBD] [TBD] [TBD] [TBD] [TBD] [TBD] [TBD] [TBD] [TBD] [TBD] [TBD] [TBD] [TBD] [TBD] [TBD] [TBD] [TBD] [TBD] [TBD] [TBD] [TBD] [TBD] [TBD] [TBD] [TBD] [TBD] [TBD] [TBD] [TBD] [TBD] [TBD] [TBD] [TBD] [TBD] [TBD] [TBD] [TBD] [TBD] [TBD] [TBD] [TBD] [TBD] [TBD] [TBD] [TBD] [TBD] [TBD] [TBD] [TBD] [TBD] [TBD] [TBD] [TBD] [TBD] [TBD] [TBD] [TBD] [TBD] [TBD] [TBD] [TBD] [TBD] [TBD] [TBD] [TBD] [TBD] [TBD] [TBD] [TBD] [TBD] [TBD] [TBD] [TBD] [TBD] [TBD] [TBD] [TBD] [TBD] [TBD] [TBD] [TBD] [TBD] [TBD] [TBD] [TBD] [TBD] [TBD] [TBD] [TBD] [TBD] [TBD] [TBD] [TBD] [TBD] [TBD] [TBD] [TBD] [TBD] [TBD] [TBD] [TBD] [TBD] [TBD] [TBD] [TBD] [TBD] [TBD] [TBD] [TBD] [TBD] [TBD] [TBD] [TBD] [TBD] [TBD] [TBD] [TBD] [TBD] [TBD] [TBD].

\begin{table}[t!]
\centering
\caption{Performance when using different attention layers for RoI-based masked attention on OV-LVIS }
\vspace*{-0.25cm}
\label{tab:blk_num}
\resizebox{1.0\linewidth}{!}{%
\begin{tabular}{@{}lrrrr@{}}
\toprule 
{Attn. Layer Num.}& $\mathrm{mAP^{box}_{novel}}$ & $\mathrm{mAP^{box}}$ & $\mathrm{mAP^{mask}_{novel}}$ & $\mathrm{mAP^{mask}}$ \\
\midrule
% & %(memory update / robust learning) &
% \multicolumn{1}{r}{\footnotesize {\sffamily RCNN-based}} \\
% ViLD-text & 10.1 & 24.9 & 5.9 & 49.3 \\
$\mathrm{15^{th}}$ & 13.6 & 28.3 & 10.5 & 20.4 \\ 
$\mathrm{20^{th}}$ & 20.4 & 30.1 & 16.2 & 21.9 \\ 
\textbf{$\mathrm{24^{th}}$ (last layer)} & \textbf{29.4} & \textbf{33.0} & \textbf{23.1} & \textbf{24.2} \\
\bottomrule
\end{tabular}%
}
\label{tab:diff_layer}
\vspace*{-0.4cm}
\end{table}

\section{Supplementary Analysis}

We provide supplementary analysis on applying RoI-based masked attention to different attention layers in Appendix \ref{sec:diff_layer} and using different pre-trained ViT backbones with \algname{} in Appendix \ref{sec:diff_pretrain}.




\subsection{Pre-trained Weights.} 
\label{sec:diff_pretrain}

We analyze the impact of using different initial weights for the backbone, ImageNet and MAE, on the overall performance. As outlined in Table~\ref{tab:pretrained_model}, training the backbone using the MAE pre-trained weights yields better performance in dense prediction tasks such as object detection and instance segmentation, even under an open-vocabulary setup, compared to starting from the ImageNet pre-trained weights. This observation is consistent with previous research, such as \cite{li2022exploring}.
It is also worth noting that even when using the ImageNet pre-trained weights, \algname{} still outperforms the baseline\,(OV-DETR) with a much faster inference speed.
In conclusion, the choice of initial weight for the backbone plays a crucial role in the overall performance of the model, and using the MAE pre-trained weights as the starting point results in better performance.





\subsection{Different Layers for RoI-based Attention} 
\label{sec:diff_layer}

We investigate the impact of applying our RoI-based masked attention to different attention layers in the ViT-L/14 image encoder of CLIP. The results, as summarized in Table \ref{tab:diff_layer}, show that detection accuracy improves as the layer number increases, from the 15th to the 24th. Therefore, applying the technique to the last layer is the best design choice for open vocabulary detection with \algname{}.

\begin{table}[t!]
\centering
\caption{Performance when using different pre-trained ViT backbones on OV-LVIS.}
\vspace*{-0.25cm}
\label{tab:pretrained_model}
\resizebox{1.0\linewidth}{!}{%
\begin{tabular}{@{}lrrrr@{}}
\toprule 
{Pretrained Model}& $\mathrm{mAP^{box}_{novel}}$ & $\mathrm{mAP^{box}}$ & $\mathrm{mAP^{mask}_{novel}}$ & $\mathrm{mAP^{mask}}$ \\
\midrule
% & %(memory update / robust learning) &
% \multicolumn{1}{r}{\footnotesize {\sffamily RCNN-based}} \\
% ViLD-text & 10.1 & 24.9 & 5.9 & 49.3 \\
ImageNet~\cite{deng2009imagenet} & 26.4 & 28.7 & 20.7 & 20.9\\
\textbf{MAE}~\cite{he2022masked} & \textbf{29.4} & \textbf{33.0} & \textbf{23.1} & \textbf{24.2}\\
\bottomrule
\end{tabular}%
}
\end{table}
\end{document}
