\subsection{Motivation for asymmetry}
Assume that we observe $X_1, X_2$ following a SCM, and we want to estimate their causal relation (causal graph $\mathcal{G}$). In model-based approaches for causal discovery, we assume some form of $P(effect\mid cause)$ (the distribution of the effect given the cause). Say that we \textit{assume}  $P(effect\mid cause)\in\mathcal{F}$, where $\mathcal{F}$ is a subset of all conditional distributions. If $\mathcal{F}$ is small enough, we can hope for identifiability of $\mathcal{G}$. 
E.g. in CPCM, we assume $\mathcal{F}$ is a known family of conditional distributions. In ANM, we assume $\mathcal{F}$ consists of all conditional distributions that arise as a sum of a function of the cause and a noise. In this section, we discuss a different set of assumptions, that is more general and can be asymmetrical for $X_1$ and for $X_2$. Instead of restricting $P(effect\mid cause)\in\mathcal{F}$, we restrict either $P(X_2\mid X_1)\in \mathcal{F}_1$, or $P(X_1\mid X_2)\in\mathcal{F}_2$, where $\mathcal{F}_1, \mathcal{F}_2$ are (not necessary equal) subsets of all conditional distributions. We motivate this by the following example. 

Suppose that we observe data such as in Figure \ref{Asymetrical_picture}. Here, $X_2$ is non-negative, and $X_1$ has a full support. Choosing an appropriate restriction of $P(effect\mid cause)$ can be tricky, since if the direction $X_1\to X_2$ is correct then  $P(effect\mid cause)$ needs to be non-negative. On the other hand, $P(effect\mid cause)$ needs to have full support in the case $X_2\to X_1$.  

Instead of restricting  $P(effect\mid cause)$, we divide our assumptions into two cases. If $X_1\to X_2$, then we assume $P(X_2\mid X_1)\in\mathcal{F}_1$ ; if $X_2\to X_1$, then we assume $P(X_1\mid X_2)\in\mathcal{F}_2$, where $\mathcal{F}_1$ consists of non-negative distributions and $\F_2$ of distributions with full support. 

Note that this is a generalization of classical model-based approaches, since they make an implicit choice $\mathcal{F}_1 = \mathcal{F}_2$ which can create a strong bias towards some causal direction. In this section, we create a framework allowing asymmetrical assumptions. 

\begin{figure}[h]
\centering
\includegraphics[scale=0.4]{figures/Asymetrical_picture.png}
\caption{A dataset generated as $X_2\to X_1$, where $X_2\sim Pareto(3), X_1=X_2 + X_2\varepsilon_1, \varepsilon_1\sim N(0, 1)$. }
\label{Asymetrical_picture}
\end{figure}
