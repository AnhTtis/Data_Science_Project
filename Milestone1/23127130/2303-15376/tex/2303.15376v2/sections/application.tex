
\section{Application}
\label{Section7}

We explain our methodology in detail on real-world data describing the expenditure habits of Philippines residents. The Philippine Statistics Authority conducts a nationwide survey of Family Income and Expenditure \citep{psa_fies} every three years. 
The dataset (taken from \cite{FamilyIncomeExpenditure}) contains more than 40,000 observations primarily comprised of the household income and expenditures of each household. To reduce the size and add homogeneity to the data, we consider only families of size $1$ (people living alone) above a poverty line (top 90\%, with income at least $80,000\,\, pesos\approx 4000\,\,dollars $ per year). We end up with $n=1417$ observations. 

We focus on the following variables: Total income ($X_1$), Food expenditure ($X_2$), and Alcohol expenditure ($X_3$). We want to discover causal relations between these variables. Intuition gives us that $Income\to Food$. However, the relation between alcohol and other variables is not trivial. Drinking habits can impact income and food habits, although income can change the quality of the purchased alcohol. Histograms and pairwise plots of the variables can be found in Appendix~\ref{Appendix_simulations}. 

\subsection{Pairwise discovery}

First, we focus on a causal relation between $X_1$ and $X_2$ (between $Income$ and $Food$). We apply our $CPCM(F_1, F_2)$ methodology following the algorithm presented in Section \ref{Section_Algorithm}. As for the choice of $F_1, F_2$, we choose the \textit{Gamma} distribution for both  $F_1,F_2$ (model described in Consequence \ref{consequenceprva}).  We explain our choice: both marginals of $X_1, X_2$ are Gamma-shaped (see histograms in Figure  \ref{Just_histograms}) and the Gamma is a common choice for this type of data. One may argue that the Pareto distribution is more suitable since the tails of $X_1, X_2$ are heavy-tailed and $X_1, X_2$ are dependent in extremes. 
One may compare AIC scores in regression with the Gamma and Pareto distributions. AIC is equal to $31 308$ ($31449$)  for the Gamma (Pareto) distribution in the direction $X_1\to X_2$ and $36089$ ($37171$)  for the Gamma (Pareto) distribution in the direction $X_2\to X_1$. In this case, the Gamma distribution seems a better fit in both directions. However, choosing $F$ based on AIC is not appropriate; we must be careful when comparing several choices of $F$ (see the discussion in Section \ref{Section5Model_choice}). 

Applying the CPCM algorithm gives us an estimation $X_1\to X_2$. P-values of the independence tests were $0.2$ and $0.02$ corresponding to the directions $X_1\to X_2$ and $X_2\to X_1$, respectively.  Choosing the Pareto distribution function or even the Gaussian distribution function gives us similar results. 

Second, we focus on a causal relation between $X_1$ and $X_3$ (between $Income$ and $Alcohol$). As for the choice of $F_3$, the Pareto distribution seems more reasonable since it seems heavy-tailed with a large mass between 0 and 1000 pesos. However, choosing $F_1=Gamma$ and $F_3 = Pareto$ can potentially lead to an "unfair game" bias (see Section \ref{Section5Model_choice}).  

Applying the CPCM algorithm suggests that both directions are unplausible. P-values of the independence tests were $\approx 10^{-9}$ and $0.003$ corresponding to the directions $X_1\to X_3$ and $X_3\to X_1$, respectively. We note that for any pair of Gamma, Gaussian, or Pareto distribution functions, the p-values are always below $0.003$. This suggests that some assumptions are not fulfilled. In this case, we believe that causal sufficiency is violated; there is a strong unobserved common cause between these variables. 
Note that even if both causal graphs were unplausible, direction  $X_3\to X_1$ seemed to be the most probable direction ($0.003>10^{-9}$ for the choice of Gamma distribution). 

Finally, we focus on the causal relation between $X_2$ and $X_3$ (between $Food$ and $Alcohol$). Using the same principle as in the previous pairs, we obtain an estimation $X_3\to X_2$. P-values of the independence tests were $2\cdot10^{-9}$ and $0.46$ corresponding to the directions $X_2\to X_3$ and $X_3\to X_2$, respectively. We note that choosing the Pareto distribution function or the Gaussian distribution function gives us similar results. This result suggests that drinking habits affect food habits. 

\subsection{Multivariate score-based discovery and results}
We apply the score-based algorithm presented in Section \ref{Section_score_based_algorithm} with the Gamma distribution function $F$. The graph with the best score is the one shown in Figure \ref{Just_graph}. However, this graph is not plausible as the test of independence between $\hat{\varepsilon}_1,\hat{\varepsilon}_2, \hat{\varepsilon}_3$ gave a p-value $0.02$. The other two graphs with an added arrow from $alcohol\to income$ or $income\to alcohol$ had slightly worse scores and a p-value equal to $0.03$. 

\begin{figure}[t]
\centering
\includegraphics[scale=0.4]{figures/only_graph.png}
\caption{ Final estimate of the causal graph in the application using a score-based algorithm and pairwise discovery. The relation between $Alcohol$  and $Income$ is inconclusive and differs with different choices of $F$ and used methods. The score-based algorithm with Gamma distribution $F$ suggests no arrow, with Pareto $F$ suggesting an arrow from $Income\to Alcohol$ and the bivariate algorithm suggesting an arrow from $Alcohol\to Income$}. 
\label{Just_graph}
\end{figure}



Possible feedback loops and common causes are the main reason why we do not obtain clear independence between the estimated noise variables. Deviations from the assumed model conditions are often observed in real-world scenarios. Despite this, acceptable estimates of the causal relationships can still be derived if the deviations are not excessive.





























