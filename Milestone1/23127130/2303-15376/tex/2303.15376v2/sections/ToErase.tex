\subsection{}
\begin{lemma}\label{PomocnaLemma1}
Let $n\in\mathbb{N}$ and let $\mathcal{S}\subseteq \mathbb{R}$ contain an open interval. Let $f_1, \dots, f_n, g_1, \dots, g_n$ be non-constant continuous real functions on $\mathcal{S}\subseteq \mathbb{R}$, such that 
$
f_1(x)g_1(y) + \dots + f_n(x)g_n(y)
$ is additive in $x,y$, that is, there exist functions $f,g$ such that 
$$
f_1(x)g_1(y) + \dots + f_n(x)g_n(y) = f(x) + g(y), \forall x,y\in\mathcal{S}.
$$
Then, there exist (not all zero) constants $a_1, \dots, a_n, c\in\mathbb{R}$ such that 
$\sum_{i=1}^n a_if_i(x) = c$ for all $x\in\mathcal{S}$. Specifically for $n=2$ holds $f_1(x) = af_2(x)+c$ for some $a,c\in\mathbb{R}$. 

Moreover, assume that for some $q<n$ holds that $g_1, \dots, g_q$ are linearly independent in a sense that there exist $y_1, \dots, y_q\in\mathcal{S}$ such that a matrix 
\begin{equation}
M:=\begin{pmatrix}
 g_1(y_1) & \cdots & g_q(y_1) \\
\cdots & \cdots & \cdots \\
g_1(y_q) & \cdots & g_q(y_{q}) 
\end{pmatrix} 
\end{equation}
has full rank. Then, for all $i=1, \dots, q$ there exist constants $a_{q+1}, \dots, a_n, c\in\mathbb{R}$ such that $f_i(x)=\sum_{j=q+1}^n a_jf_j(x) +c$ for all $x\in\mathcal{S}$. 
\end{lemma}
\begin{proof}
Fix $y_1, y_2\in\mathcal{X}$ such that $y_1\neq y_2$. Then, we have for all $x\in\mathcal{S}$
\begin{align*}
&f_1(x)g_1(y_1) + \dots + f_n(x)g_n(y_1) = f(x) + g(y_1),\\&
f_1(x)g_1(y_2) + \dots + f_n(x)g_n(y_2) = f(x) + g(y_2),
\end{align*}
and subtraction of these equalities gives us 
$$
f_1(x)[g_1(y_1)- g_1(y_2)] + \dots + f_n(x)[g_n(y_1)-g_n(y_2)] = g(y_1) - g(y_2).
$$
Defining $a_i = g_i(y_1)- g_i(y_2)$ and $c = g(y_1)- g(y_2)$ gives us the first result.

Now, we prove the "Moreover" part. Find $y_0\in\mathcal{S}$ such that a matrix 
$Q:= M-(1, \dots, 1)^\top(g_1(y_0), \dots, g_q(y_0))$ has also full rank. This is possible from the assumption that $\mathcal{S}$ contains an open interval and $g$ are continuous. Now, consider equalities 
\begin{align*}
&f_1(x)g_1(y_0) + \dots + f_n(x)g_n(y_0) = f(x) + g(y_0),\\&
f_1(x)g_1(y_1) + \dots + f_n(x)g_n(y_1) = f(x) + g(y_1),\\&
\dots\\&
f_1(x)g_1(y_q) + \dots + f_n(x)g_n(y_q) = f(x) + g(y_q),
\end{align*}
where $y_1, \dots, y_q$ are defined in the lemma. Subtracting from each equality the first one gives us 
\begin{align*}
&f_1(x)[g_1(y_1)- g_1(y_0)] + \dots + f_n(x)[g_n(y_1)-g_n(y_0)] = g(y_1) - g(y_0)\\&
\dots \\&
f_1(x)[g_1(y_q)- g_1(y_0)] + \dots + f_n(x)[g_n(y_q)-g_n(y_0)] = g(y_q) - g(y_0).
\end{align*}
Using matrix formulation, this can be rewritten as 
\begin{equation}
Q\begin{pmatrix}
f_1(x) \\
\cdots \\
f_q(x) 
\end{pmatrix} =
\begin{pmatrix}
g(y_1)-g(y_0) -\sum_{j=q+1}^n f_{j}(x)[g_{j}(y_1) -g_{j}(y_0)] \\
\cdots \\
g(y_q)-g(y_0) -\sum_{j=q+1}^n f_{j}(x)[g_{j}(y_q) -g_{j}(y_0)]
\end{pmatrix} .
\end{equation}
Multiplying both sides by $Q^{-1}$ gives us that $f_1(x)$ is a linear combination of $f_{q+1}(x), \dots, f_n(x)$, what we wanted to show (as well as $f_i(x)$ for $i=1, \dots, q$)
\end{proof}


\begin{customthm}{\ref{thmAssymetricMultivariatesufficient}}
Let $(X_1, X_2)$ follow an asymmetrical $CPCM(F_1, F_2)$ defined in (\ref{asymetrical_F_one_F_two_model}), where $F_1, F_2$ lie in an exponential family of continuous distributions and $T_1(\cdot) = (T_{1,1}(\cdot), \dots, T_{1,q_1}(\cdot))^\top$, $T_2(\cdot) = (T_{2,1}(\cdot), \dots, T_{2,q_2}(\cdot))^\top$are the corresponding sufficient statistics with nontrivial intersection of their support $\mathcal{S}:=supp(F_1)\cap supp(F_2)$.

The causal graph is identifiable, if $\theta_2$ is not a linear combination of $T_{1,1}, \dots, T_{1,q_1}$ on $\mathcal{S}$. That is, if $\theta_2$ can not be written as
\begin{equation}\tag{\ref{eq158}}
\theta_{2,i}(x) \overset{}{=} \sum_{j=1}^{q_1}a_{i,j}T_{1,j}(x)+b_i,\,\,\,\,\,\,\,\,\,\,\,\,\,\,\, \forall x\in \mathcal{S},
\end{equation}
for all $i=1, \dots, q_2,$ where $a_{i,j},b_i\in\mathbb{R}$, $j=1, \dots, q_1$ are constants. 
\end{customthm}



\begin{proof}
\label{Proof of thmAssymetricMultivariatesufficient}{}
If the asymmetrical $CPCM(F_1,F_2)$ is \textit{not} identifiable, then functions $\theta_1, \theta_2$ satisfy that random variables from $X_1 = \varepsilon_1, X_2 = F_2^{-1}(\varepsilon_2, \theta_2(X_1))$ and from $X_2 = \varepsilon_2, X_1 = F_1^{-1}(\varepsilon_1, \theta_1(X_2))$ have the same joint density function. Write the joint density as
\begin{equation}\label{eq59}
  p_{X_1, X_2}(x,y) = p_{X_1}(x)p_{X_2\mid {X_1}}(y\mid x) = p_{X_2}(y)p_{{X_1}\mid {X_2}}(x\mid y).
 \end{equation}
Since $F_1, F_2$ lie in the exponential family of distributions, we use the notation from \hyperref[appendix]{Appendix} \ref{appendix_exponential_family} and rewrite (\ref{eq59}) as follows
\begin{equation*}
    \begin{split}
  &     p_{{X_2}\mid {X_1}}(y\mid x) = h_{1,1}(y)h_{1,2}[\theta_2(x)]\exp[\sum_{i=1}^{q_2}\theta_{2,i}(x)T_{2,i}(y)],\\&
  p_{{X_1}\mid {X_2}}(x\mid y) = h_{2,1}(x)h_{2,2}[{\theta_1}(y)]\exp[\sum_{i=1}^{q_1}{\theta}_{1,i}(y)T_{1,i}(x)].  
    \end{split}
\end{equation*}
Now, after a logarithmic transformation of both sides of (\ref{eq59}), we obtain 
\begin{equation}\label{eq254}
\begin{split}
\log[p(x,y)] &= \log[p_{X_1}(x)] +  \log[h_{1,1}(y)]+\log\{h_{1,2}[\theta_{2}(x)]\} + \sum_{i=1}^{q_2}\theta_{2,i}(x)T_{2,i}(y) \\&
= \log[p_{X_2}(y)] +  \log[h_{2,1}(x)]+\log\{h_{2,2}[\theta_{1}(y)]\} + \sum_{i=1}^{q_1}\theta_{1,i}(y)T_{1,i}(x).
\end{split}
\end{equation}
Define $f(x) = \log[p_{X_1}(x)] +\log\{h_{1,2}[\theta_{2}(x)]\} -\log[h_{2,1}(x)]$ and $g(y) =\log[h_{1,1}(y)] -  \log[p_{X_2}(y)] + \log\{h_{2,2}[\theta_{1}(y)]\}$. Then, equality (\ref{eq254}) reads as 
\begin{equation}\label{eq9876}
f(x) + g(y) = \sum_{i=1}^{q_1}\theta_{1,i}(y)T_{1,i}(x) - \sum_{i=1}^{q_2}\theta_{2,i}(x)T_{2,i}(y).
\end{equation}
Now we use Lemma \ref{PomocnaLemma1}. We know that functions $T_{2,i}$ are linearly independent in the sense presented in Lemma  \ref{PomocnaLemma1}, see Observation \ref{observationFullRank} in \hyperref[appendix]{Appendix} \ref{appendix_exponential_family}. Therefore, Lemma \ref{PomocnaLemma1} gives us that $\theta_{2,i}$ are only a linear combination of $T_{1, i}$. That is what we wanted to show. 
\end{proof}
