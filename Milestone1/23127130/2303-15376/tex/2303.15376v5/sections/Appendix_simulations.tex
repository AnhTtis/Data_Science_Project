
\section{Plots and details about simulations}
\label{Appendix_simulations}

Plots corresponding to Simulations~\ref{Section_simulations_Pareto} and \ref{Section_simulations_Gaussian} are presented in Figures~\ref{Pareto_histograms}, \ref{sample_size_pareto} and  \ref{Simulations2_plots}. 


The experiments from Simulations~\ref{Section_simulations_Gaussian} were inspired by \cite{Natasa_Tagasovska} and implementations of other baseline methods are also taken from \cite{Natasa_Tagasovska} and \cite{immer2022identifiability}. 

For LOCI, we use the default format with neural network estimations and subsequent independence testing (also denoted as $NN-LOCI_H$) \citep{immer2022identifiability}.
For IGCI, we use the original implementation from \cite{IGCI} with slope-based estimation with Gaussian and uniform reference measures. For RESIT, we use the implementation from \cite{Peters2014} with GP regression and the HSIC independence test with a threshold value of $0.05$. For the slope algorithm, we use the implementation of \cite{Slope}, with the local regression included in the fitting process. For comparisons with other methods such as PNL, GPI-MML, ANM, Sloppy, GR-AN, EMD, GRCI, see Section 3.2 in \cite{Natasa_Tagasovska} and Section 5 in \cite{immer2022identifiability}. 



The plot corresponding to Section~\ref{section_scalability} is presented in Figure~\ref{scalability}. For each sample size \( n \) and dimension \( d \), we generated a uniformly random graph with \( d \) nodes and \( d-1 \) edges. The data was then generated according to the structural equation model \( X_i = f_i(\textbf{X}_{pa_i}) + \varepsilon_i \), where \( \varepsilon_i \sim N(0,1) \) and \( f_i(x) = c_i x^{2} \) with \( c_i \sim \text{Unif}(0.5, 1.5) \). Next, we estimated the DAG using the \( CPCM(F) \) algorithm with \( F \) set to a Gaussian distribution with fixed variance. The accuracy of the estimated DAG was evaluated using the Structural Intervention Distance (SID, \cite{peters2014structuralinterventiondistancesid}). This procedure was repeated 100 times, and we report the average. Finally, we report the ratio of the computed SID using the CPCM(F) algorithm to the SID of an algorithm that generates a random graph. 



\begin{figure}[]
\centering
\includegraphics[width = 0.9\textwidth]{figures/pareto_histogram.pdf}
\caption{Simulations~\ref{Section_simulations_Pareto}. Distributions of the p-values from the independence test in Step 1b) of Algorithm~\ref{Algorithm1}, for model \eqref{fwesef} with \(\gamma = 0\) and \(\gamma = 2\).}
\label{Pareto_histograms}
\end{figure}


\begin{figure}[]
\centering
\includegraphics[scale=0.8]{figures/sample_size_pareto.pdf}
\caption{Simulations~\ref{Section_simulations_Pareto}. The plot displays the percentage of correctly estimated causal directions across a range of sample sizes \( n \), using model \eqref{fwesef} with hyperparameters \(\gamma = 1\) and \(\gamma = 2\). As \( n \) increases, the algorithm demonstrates near-perfect performance, affirming the theoretical consistency of the proposed method. }
\label{sample_size_pareto}
\end{figure}



\begin{figure}[]
\centering
\includegraphics[scale=0.7]{figures/scalability.pdf}
\caption{Ratio of the computed SID using the $CPCM(F)$ algorithm to the SID of an algorithm producing a random graph, averaged over 100 repetitions for different values of \( n \) and \( d \). Lower SID fraction means better performance of the $CPCM(F)$ algorithm.}
\label{scalability}
\end{figure}



\begin{figure}[ht]
\centering
\includegraphics[scale=0.4]{figures/sim6.2_sample.png}
\caption{Simulations~\ref{Section_simulations_Gaussian}. An example of datasets generated via different models. }
\label{Simulations2_plots}
\end{figure}


















\iffalse


\begin{figure}[ht]
\centering
\includegraphics[scale=0.45]{figures/Simulations3_Pareto.pdf}
\caption{Simulations~\ref{Section_simulations_robustness}.  An example of a randomly generated function $\theta$ and a generated dataset in which  $effect\mid cause\sim Pareto(\theta(cause))$. Note that if $\theta(x)$ is small, then $effect\mid cause=x$ will have heavy tails. If $\theta(x)<1$, then the expectation of $effect\mid cause=x$ does not exist.}
\label{Simulations3_plot}
\end{figure}



\begin{figure}[ht]
\centering
\includegraphics[scale=0.5]{figures/3ploty.pdf}
\caption{ Total income ($X_1$), Food expenditure ($X_2$), and Alcohol beverages expenditure ($X_3$).  }
\label{Just_plots}
\end{figure}

\begin{figure}[ht]
\centering
\includegraphics[scale=0.5]{figures/histograms.pdf}
\caption{ Total income ($X_1$), Food expenditure ($X_2$), and Alcohol expenditure ($X_3$).  }
\label{Just_histograms}
\end{figure}

\fi 


