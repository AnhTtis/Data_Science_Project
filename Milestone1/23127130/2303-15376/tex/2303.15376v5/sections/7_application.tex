
\subsection{Application}
\label{Section7}

We explain our methodology in detail based on real-world data that describes the expenditure habits of Philippines residents. The Philippine Statistics Authority conducts a nationwide survey of Family Income and Expenditure \citep{psa_fies} every three years. 
The dataset (taken from \cite{FamilyIncomeExpenditure}) contains over 40,000 observations primarily comprising the household income and expenditures of each household. To reduce the size and add homogeneity to the data, we consider only families of size $1$ (people living alone) above the poverty line (top 90\%, with an income of at least $80,000\,\, pesos\approx 4000\,\,dollars $ per year). We end up with $n=1417$ observations. We focus on the following variables: Total income ($X_1$), Food expenditure ($X_2$), and Alcohol expenditure ($X_3$). 

These data exhibit strong heavy-tailed behavior, presenting a significant challenge for most causal discovery approaches. For instance, while $95\%$ of the population has an annual income below $400,000\,\text{pesos}$, the top $1\%$ far exceeds $1,000,000\,\text{pesos}$. This tail structure suggests the possibility of infinite variance and even an infinite expected value for all variables. Consequently, ANM and location-scale models are highly unsuitable in this context, since $f(x) = \mathbb{E}[X_i\mid X_j=x]=\infty$. In contrast, our $CPCM(F)$ method is well-suited for heavy-tailed settings when employing a heavy-tailed choice of $F$. In economics, it is common practice to model income using Pareto or Gamma distributions \citep{lawless2002statistical}.


Our objective is to identify the causal relationships among these variables. Common sense suggests that \( \text{Income} \to \text{Food} \). However, the relationships between alcohol and other variables are not trivial. In order to discovery causal relations, we apply our \( CPCM(F_1, \dots, F_k) \) methodology, following the algorithm presented in Section \ref{Section_Algorithm}. Choice \( \{F_1, \dots, F_k\} = \mathscr{S}_1 \) leads to strong rejection of all tests, making no direction plausible. This is not surprising, as the sample size is relatively large for a single parameter to sufficiently describe the complex behavior of the data. Hence, we choose \( \{F_1, \dots, F_k\} = \mathscr{S}_2 \), as defined in Section~\ref{Section_practical_choices}. 

First, we focus on the causal relationships between pairs of random variables.  Applying Algorithm~\ref{Algorithm1} to determine the causal relationship between \( X_1 \) and \( X_2 \) yields p-values of 0.2 and 0.02 for the directions \( X_1 \to X_2 \) and \( X_2 \to X_1 \), respectively. This suggests rejecting the plausibility of the latter graph while not rejecting the former, leading to the final estimation \( X_1 \to X_2 \). Using a similar approach, we conclude that \( X_3 \to X_2 \), with p-values of \( 2 \times 10^{-9} \) and 0.16. This result suggests that drinking habits influence food habits. Finally, the p-values corresponding to \( X_1 \to X_3 \) and \( X_3 \to X_1 \) were \( 10^{-9} \) and 0.02, respectively. This indicates that some assumptions remain unfulfilled. In this case, we believe that causal sufficiency is violated due to a strong unobserved common cause between these variables. Note that even though both causal graphs were implausible, the direction \( X_3 \to X_1 \) appeared to be more probable (\(0.02 > 10^{-9} \)).

Finally, we apply the multivariate score-based algorithm presented in Section \ref{Section_score_based_algorithm} using the choice \( \{F_1, \dots, F_k\} = \mathscr{S}_2 \). The graph with the best score is \( \text{Alcohol} \to \text{Food} \leftarrow \text{Income} \). However, this graph is not plausible, as the test of independence between \( \hat{\varepsilon}_1, \hat{\varepsilon}_2, \hat{\varepsilon}_3 \) yields a p-value of 0.03. This suggests that some assumptions, such as causal sufficiency, may still be violated. 












