
\subsection{Application}
\label{Section7}

We explain our methodology in detail based on real-world data that describes the expenditure habits of Philippines residents. The Philippine Statistics Authority conducts a nationwide survey of Family Income and Expenditure \citep{psa_fies} every three years. 
The dataset (taken from \cite{FamilyIncomeExpenditure}) contains over 40,000 observations primarily comprising the household income and expenditures of each household. To reduce the size and add homogeneity to the data, we consider only families of size $1$ (people living alone) above the poverty line (top 90\%, with an income of at least $80,000\,\, pesos\approx 4000\,\,dollars $ per year). We end up with $n=1417$ observations. We focus on the following variables: Total income ($X_1$), Food expenditure ($X_2$), and Alcohol expenditure ($X_3$). Our objective is to identify the causal relationships among these variables. Common sense gives us that $Income\to Food$. However, the relationships between alcohol and other variables is not trivial.

First, we focus on the causal relationship between the pairs of random variables. We apply our $CPCM(F_1, \dots, F_k)$ methodology following the algorithm presented in Section \ref{Section_Algorithm}, with the choice $\{F_1, \dots, F_k\} = \mathscr{S}_2$ defined in Section~\ref{Section_practical_choices}. 
First, applying Algorithm~\ref{Algorithm1} to determine the causal relationship between \( X_1 \) and \( X_2 \) yields p-values of 0.2 and 0.02 for the directions \( X_1 \to X_2 \) and \( X_2 \to X_1 \), respectively. This suggests rejecting the plausibility of the latter graph while not rejecting the former, resulting in the final estimation of \( X_1 \to X_2 \). Using a similar approach, we conclude \( X_3 \to X_2 \), where the p-values of the independence tests were \( 2 \times 10^{-9} \) and 0.16. This result suggests that drinking habits affect food habits. Finally, the p-values corresponding to \( X_1 \to X_3 \) and \( X_3 \to X_1 \) were \( 10^{-9} \) and $0.02$, respectively. This indicates that some assumptions remain unfulfilled. In this case, we believe that causal sufficiency is violated; there is a strong unobserved common cause between these variables. Note that even if both causal graphs were implausible, the direction \( X_3 \to X_1 \) appeared to be the more probable direction (\(0.02 >  10^{-9} \)).

Applying the same method with $\{F_1, \dots, F_k\} = \mathscr{S}_1$ leads to rejecting all tests and no direction is plausible. This is not surprising, as the sample size is relatively large for one parameter to sufficiently describe the complex behaviour of the data. 

Finally, we apply the multivariate score-based algorithm presented in Section \ref{Section_score_based_algorithm} using the choice $\{F_1, \dots, F_k\} = \mathscr{S}_2$. The graph with the best score is $Alcohol\to Food \leftarrow Income$. However, this graph is not plausible, as the test of independence between $\hat{\varepsilon}_1,\hat{\varepsilon}_2, \hat{\varepsilon}_3$ yields a p-value of $0.03$. Modeling these types of data with Pareto or Gamma distributions is a standard practice \citep{lawless2002statistical}, but the model assumptions or causal sufficiency may still be violated. 




























