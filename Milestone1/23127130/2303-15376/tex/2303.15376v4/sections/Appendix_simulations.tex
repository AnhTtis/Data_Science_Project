
\section{Plots and details about simulations}
\label{Appendix_simulations}

Plots corresponding to Simulations~\ref{Section_simulations_Pareto} and \ref{Section_simulations_Gaussian} are presented in Figures~\ref{Pareto_histograms}, \ref{sample_size_pareto} and  \ref{Simulations2_plots}. 


The experiments from Simulations~\ref{Section_simulations_Gaussian} were inspired by \cite{Natasa_Tagasovska} and implementations of other baseline methods are also taken from \cite{Natasa_Tagasovska} and \cite{immer2022identifiability}. 

For LOCI, we use the default format with neural network estimations and subsequent independence testing (also denoted as $NN-LOCI_H$) \citep{immer2022identifiability}.
For IGCI, we use the original implementation from \cite{IGCI} with slope-based estimation with Gaussian and uniform reference measures. For RESIT, we use the implementation from \cite{Peters2014} with GP regression and the HSIC independence test with a threshold value of $0.05$. For the slope algorithm, we use the implementation of \cite{Slope}, with the local regression included in the fitting process. For comparisons with other methods such as PNL, GPI-MML, ANM, Sloppy, GR-AN, EMD, GRCI, see Section 3.2 in \cite{Natasa_Tagasovska} and Section 5 in \cite{immer2022identifiability}. 


\begin{figure}[]
\centering
\includegraphics[width = \textwidth]{figures/pareto_histogram.pdf}
\caption{Simulations~\ref{Section_simulations_Pareto}. Distributions of the p-values from the independence test in Step 1b) of Algorithm~\ref{Algorithm1}, for model \eqref{fwesef} with \(\alpha = 0\) and \(\alpha = 2\).}
\label{Pareto_histograms}
\end{figure}


\begin{figure}[]
\centering
\includegraphics[scale=0.9]{figures/sample_size_pareto.pdf}
\caption{Simulations~\ref{Section_simulations_Pareto}. The plot displays the percentage of correctly estimated causal directions across a range of sample sizes \( n \), using model \eqref{fwesef} with hyperparameters \(\alpha = 1\) and \(\alpha = 2\). As \( n \) increases, the algorithm demonstrates near-perfect performance, affirming the theoretical consistency of the proposed method. }
\label{sample_size_pareto}
\end{figure}




\begin{figure}[ht]
\centering
\includegraphics[scale=0.4]{figures/sim6.2_sample.png}
\caption{Simulations~\ref{Section_simulations_Gaussian}. An example of datasets generated via different models. }
\label{Simulations2_plots}
\end{figure}


















\iffalse


\begin{figure}[ht]
\centering
\includegraphics[scale=0.45]{figures/Simulations3_Pareto.pdf}
\caption{Simulations~\ref{Section_simulations_robustness}.  An example of a randomly generated function $\theta$ and a generated dataset in which  $effect\mid cause\sim Pareto(\theta(cause))$. Note that if $\theta(x)$ is small, then $effect\mid cause=x$ will have heavy tails. If $\theta(x)<1$, then the expectation of $effect\mid cause=x$ does not exist.}
\label{Simulations3_plot}
\end{figure}



\begin{figure}[ht]
\centering
\includegraphics[scale=0.5]{figures/3ploty.pdf}
\caption{ Total income ($X_1$), Food expenditure ($X_2$), and Alcohol beverages expenditure ($X_3$).  }
\label{Just_plots}
\end{figure}

\begin{figure}[ht]
\centering
\includegraphics[scale=0.5]{figures/histograms.pdf}
\caption{ Total income ($X_1$), Food expenditure ($X_2$), and Alcohol expenditure ($X_3$).  }
\label{Just_histograms}
\end{figure}

\fi 


