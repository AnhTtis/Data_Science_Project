\section{Conclusion and future research}


We introduced a new family of models for causal inference---the CPCM. Our primary theoretical contributions lie in establishing the identifiability conditions for the causal structure within this framework. Specifically, we have demonstrated the identifiability of bivariate $CPCM(F)$ models, with exceptions arising only when the parameters of $F$ take the form of a linear combination of its sufficient statistics. Furthermore, we have provided detailed characterizations of identifiability across various cases such as Gaussian, Poisson, and Pareto, significantly broadening the scope of identifiable models beyond existing literature. We briefly explained the multivariate extensions of these results. 

We have introduced two algorithms for estimating causal graphs based on the CPCM model. We showed consistency under certain assumptions, and discussed several possible extensions. A short simulation study suggests that the methodology is comparable with other commonly used methods in Gaussian location-scale models. However, our methodology is rather flexible. It can handle variables that are continuous, discrete or even a mixture of these. We applied our methodology on real-world data and discussed a few possible problems and results. 

Looking ahead, our framework holds promise for extension to encompass problematic models that classical methods can not handle well, such as zero-inflated models and Mixed-Gaussian models. By standardizing a class of well-known distributions within practical domains, we anticipate facilitating more intuitive and interpretable causal discovery processes. Adaptation of these models to diverse applications promises novel perspectives on causal inference, and future research is required to substantiate the practical utility of our framework.





\section*{Conflict of interest and data availability}
The open-source implementation of the methods discussed in this manuscript together with the data used can be found in the supplementary package or at \url{https://github.com/jurobodik/Causal_CPCM.git}.

The authors declare that they have no known competing financial interests or personal relationships that could have appeared to influence the work reported in this paper.


\section*{Acknowledgements}
This study was supported by the Swiss National Science Foundation under grant number 201126. 

































































