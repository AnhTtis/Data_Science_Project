
%--------------- Personalize your document here ---------------

\author{} % Enter your name
\newcommand{\studentID}{} % enter your student ID
\newcommand{\supervisorone}{} % Enter your supervisor's name
\newcommand{\supervisortwo}{}% Leave it empty or enter your second supervisor's name 
\newcommand{\department}{}
\newcommand{\exam}{}

\title{} %Enter the title of your report 
\date{\today} % insert a specific date	

%--------------------------------------------------------------

% This document was adapted from the
% TEMPLATE FOR PHYS250 WORKSHEET created by Alastair McLean
% URL: https://www.overleaf.com/latex/templates/phys250-worksheet-template/xxftvfhmwqdt

% Jefferson Silveira
% Email: 19jdls1@queensu.ca
% Last update: 09-Jun-2021
% If you have any questions or concerns, do not hesitate to contact me.
%--------------------------------------------------------------

\documentclass[twoside,11pt]{article}

\usepackage{jmlr2e}

% Heading arguments are {volume}{year}{pages}{submitted}{published}{author-full-names}


\usepackage[left=30mm,top=30mm,right=30mm,bottom=30mm]{geometry}
\usepackage{etoolbox} %required for cover page
\usepackage{booktabs}
\usepackage[table,xcdraw]{xcolor}
\usepackage[usestackEOL]{stackengine}
\usepackage[T1]{fontenc}
\usepackage[utf8]{inputenc}
\usepackage{bm}
\usepackage{graphicx}
\usepackage{subcaption}
\usepackage{amsmath}
\usepackage{amsfonts}
\usepackage{mathtools}
\usepackage{xcolor}
\usepackage{float}
\usepackage{hyperref}
\usepackage[capitalise]{cleveref}
\usepackage{enumitem,kantlipsum}
\usepackage{amssymb}
\usepackage{amsbsy}
\usepackage{amsthm}
\usepackage{bbm}% theorems, definitions, etc.
\usepackage{pifont}

 \usepackage{multirow}
\usepackage[ruled,vlined]{algorithm2e}
\usepackage{listings}
\usepackage{dirtytalk}
\usepackage{graphicx}

\usepackage{chngcntr}
\usepackage{apptools}
\AtAppendix{\counterwithin{lemma}{section}}

\usepackage{mathrsfs}
\usepackage{wrapfig}

%\renewcommand{\listingscaption}{Algorithm}
%\renewcommand{\listoflistingscaption}{List of Algorithms}
\newcommand{\E}{\mathbb{E}}
\newcommand{\R}{\mathbb{R}}
\DeclareMathOperator*{\argmin}{arg\,min}
\DeclareMathOperator*{\argmax}{arg\,max}
\newcommand{\F}{\mathcal{F}}
\newcommand{\M}{ (\mathcal{M}_1, \mathcal{M}_2)}


\newtheorem{proposition}{Proposition}
\newtheorem{definition}{Definition}

\newtheorem{example}{Example}
\newtheorem{remark}{Remark}
\newtheorem{assumption}{Assumption}
\newtheorem*{assumption*}{Assumption}
\newtheorem{observation}{Observation}
\newtheorem*{terminology}{Terminology}
\newtheorem{consequence}{Consequence}



\newtheorem{innercustomthm}{Theorem}
\newenvironment{customthm}[1]
  {\renewcommand\theinnercustomthm{#1}\innercustomthm}
  {\endinnercustomthm}

\newtheorem{innercustomlem}{Lemma}
\newenvironment{customlem}[1]
  {\renewcommand\theinnercustomlem{#1}\innercustomlem}
  {\endinnercustomlem}


\newtheorem{innercustomprop}{Proposition}
\newenvironment{customprop}[1]
  {\renewcommand\theinnercustomprop{#1}\innercustomprop}
  {\endinnercustomprop}

  
\newtheorem{innercustomexample}{Example}
\newenvironment{customexample}[1]
  {\renewcommand\theinnercustomexample{#1}\innercustomexample}
  {\endinnercustomexample}

  
\newtheorem{innercustomconsequence}{Consequence}
\newenvironment{customconsequence}[1]
  {\renewcommand\theinnercustomconsequence{#1}\innercustomconsequence}
  {\endinnercustomconsequence}
  
  \newtheorem{theorem}{Theorem}
\newtheorem{lemma}{Lemma}
\newtheorem{notation}{Notation}

\def\b#1{{\color{red}\bf #1}}%

\def\lm#1{{\textcolor{purple}{LM: \bf #1}}}

\newenvironment{myproof}{
  \par\medskip\noindent
  \textit{Proof}.
}{
\newline
\rightline{$\qedsymbol$}
}

\newenvironment{hproof}{%
  \renewcommand{\proofname}{Idea of the proof}\proof}{\endproof}

\newcommand\independent{\protect\mathpalette{\protect\independenT}{\perp}}
\newcommand{\indep}{\perp \!\!\! \perp}
\def\independenT#1#2{\mathrel{\rlap{$#1#2$}\mkern2mu{#1#2}}}


\hypersetup{
    colorlinks,
    linkcolor={black},
    citecolor={blue!50!black},
    urlcolor={blue!80!black}
}

\linespread{1}

\graphicspath{{figures/}}	



\begin{document}

\title{Identifiability of causal graphs under nonadditive conditionally parametric causal models}

\author{\name Juraj Bodik \email juraj.bodik@unil.ch \\
       \addr HEC\\
       University of Lausanne\\
      Switzerland
       \AND
       \name Valérie Chavez-Demoulin \email valerie.chavez@unil.ch \\
       \addr HEC\\
       University of Lausanne\\
       Switzerland}

\maketitle

\begin{abstract}
Causal discovery from observational data typically requires strong assumptions about the data-generating process. Previous research has established the identifiability of causal graphs under various models, including linear non-Gaussian, post-nonlinear, and location-scale models. However, these models may have limited applicability in real-world situations that involve a mixture of discrete and continuous variables or where the cause affects the variance or tail behavior of the effect. In this study, we introduce a new class of models, called Conditionally Parametric Causal Models (CPCM), which assume that the distribution of the effect, given the cause, belongs to well-known families such as Gaussian, Poisson, Gamma, or heavy-tailed Pareto distributions. These models are adaptable to a wide range of practical situations where the cause can influence the variance or tail behavior of the effect. We demonstrate the identifiability of CPCM by leveraging the concept of sufficient statistics. Furthermore, we propose an algorithm for estimating the causal structure from random samples drawn from CPCM. We evaluate the empirical properties of our methodology on various datasets, demonstrating state-of-the-art performance across multiple benchmarks.





\textbf{Keywords:} causal discovery, structural causal models, identifiability, higher moments, exponential family
\end{abstract}
%TC:ignore
%TC:endignore
  
%\listoffigures
%\newpage
%\listoftables
%\newpage
%\listofalgorithms % List of algorithms in pseudocode format
%\newpage
%\listoflistings % List of algorithms in code format
%\newpage


\pagenumbering{arabic}% Arabic page numbers (and reset to 1)

% This is how you can organize your document
\section{Introduction}
\label{sec:introduction}
% \begin{itemize}
%     % Diffusion of FL
%     \item {\st{Diffusion of FL}}
%     % Security threats to FL
%     \item {\st{Security threats to FL with particular focus on model poisoning}}
%     % Limitations of existing countermeasures
%     \item {\st{Current countermeasures (e.g., KRUM) and their limitations}}
%     % Proposed method and its advantages
%     \item {\st{Intuitive description of the proposed method and its difference (i.e., advantages) w.r.t. state of the art}}
%     % Main contributions
%     \item {\st{Summary of the main contributions of this work}}
%     % Paper's structure and organization
%     \item {\st{Paper's structure and organization}}
% \end{itemize}

% Diffusion of FL
Recently, {\em federated learning} (FL) has emerged as the leading paradigm for training distributed, large-scale, and privacy-preserving machine learning (ML) systems~\cite{mcmahan2017googleai,mcmahan2017aistats}. 
The core idea of FL is to allow multiple edge clients to collaboratively train a shared, global model without disclosing their local private training data.
%Specifically, an FL system consists of a central server and many edge clients; 
A typical FL round involves the following steps: {\em(i)} the server randomly picks some clients and sends them the current, global model; {\em(ii)} each selected client locally trains its model with its own private data; then, it sends the resulting local model to the server;\footnote{Whenever we refer to global/local model, we mean global/local model {\em parameters}.} {\em(iii)} the server updates the global model by computing an \emph{aggregation function}, usually the average (FedAvg), on the local models received from clients.
% \begin{enumerate}
%     \item[{\em(i)}] the server sends the current, global model to the clients and appoints some of them for training;
%     \item[{\em(ii)}] each selected client locally trains its copy of the global model with its own private data; then, it sends the resulting local model back to the server;\footnote{Whenever we refer to global/local model, we mean global/local model {\em parameters}.}
%     \item[{\em(iii)}] the server updates the global model by computing an \emph{aggregation function} on the local models received from clients (by default, the average, also referred to as FedAvg~\cite{mcmahan2017aistats}).
% \end{enumerate}
This process goes on until the global model converges. %(e.g., after a certain number of rounds or other similar stopping criteria).
%\\
% The advantages of FL over the traditional, centralized learning paradigm are undoubtedly clear in terms of flexibility/scalability (clients can join/disconnect from the FL network dynamically), network communications (only model weights\footnote{We will use \textit{parameters} and \textit{weights} interchangeably.} are exchanged between clients and server), and privacy (each client's private training data is kept local at the client's end and not uploaded to the server).
\\
% Security threats to FL
%However, the growing adoption of FL also raises security concerns~\cite{costa2022covert}, particularly about its confidentiality, integrity, and availability.
Although its advantages over standard ML, FL also raises security concerns~\cite{costa2022covert}. %, particularly about its confidentiality, integrity, and availability~\cite{costa2022covert}.
% OLD, LONG VERSION
% Indeed, some work deals with privacy leakage that may expose the local data of some clients~\cite{melis2019sp}. 
% A large body of work, instead, investigates attacks that usually aim to detriment the predictive accuracy of the learned global model. For instance, \emph{data poisoning} attacks achieve this goal by letting an adversary pollute the training set of some corrupt FL clients with maliciously crafted examples~\cite{jagielski2018sp}.
% Similarly, in \emph{model poisoning} the attacker attempts to tweak the global model weights~\cite{bhagoji2019pmlr} by directly perturbing the local model's weights of some infected FL clients before these are sent to the central server for aggregation, usually via so-called Byzantine attacks. 
% It turns out that Byzantine model poisoning attacks severely impact standard FedAvg; therefore, more robust aggregation functions must be designed to make FL systems secure.
Here, we focus on \emph{untargeted model poisoning} attacks~\cite{bhagoji2019pmlr}, where an adversary attempts to tweak the global model weights %\footnote{We will use the terms \textit{parameters} and \textit{weights} interchangeably.} 
by directly perturbing the local model's parameters of some infected clients before these are sent to the central server for aggregation.
In doing so, the adversary aims to jeopardize the global model \textit{indiscriminately} at inference time.
Such model poisoning attacks severely impact standard FedAvg; therefore, more robust aggregation functions must be designed to secure FL systems.
\\
% In this paper, we focus on designing a novel robust aggregation scheme at the server's end to contrast the effect of Byzantine model poisoning attacks.
%
% Current countermeasures and their limitations
%Several countermeasures have been proposed in the literature to combat model poisoning attacks on FL systems.
% Some methods use simple statistics more robust than plain average to smooth the impact of malicious updates (e.g., Trimmed Mean and FedMedian~\cite{yin2018icml}). 
% Other defenses implement outlier detection techniques to discard malicious updates from the aggregation performed at the server's end. Those are either based on heuristics (e.g., Krum/Multi-Krum~\cite{blanchard2017nips} and Bulyan~\cite{mhamdi2018pmlr}) or data-driven approaches (e.g., K-means clustering~\cite{shen2016acm} or DnC via spectral analysis~\cite{shejwalkar2021ndss}). 
% Finally, some strategies rely on a centralized ``source of trust'' to spot potential malicious updates (e.g., FLTrust~\cite{cao2020fltrust}).
% Several countermeasures have been proposed in the literature to combat model poisoning attacks on FL systems, i.e., to discard possible malicious local updates from the aggregation performed at the server's end. 
% These techniques range from simple statistics more robust than plain average (e.g., Trimmed Mean and FedMedian~\cite{yin2018icml}) to outlier detection heuristics (e.g., Krum/Multi-Krum~\cite{blanchard2017nips} and Bulyan~\cite{mhamdi2018pmlr}) or data-driven approaches (e.g., spectral analysis via K-means clustering~\cite{shen2016acm} or spectral analysis), or methods based on ``source of trust'' (e.g., FLTrust~\cite{cao2020fltrust}).
% OLD, LONG VERSION
%Several countermeasures have been proposed in the literature to combat Byzantine model poisoning attacks on FL systems.
% Descriptive statistics
% For example, Trimmed Mean and FedMedian aggregate local model updates using more robust statistics than standard average~\cite{yin2018icml}.
%
% % Heuristics for outlier detection
% Many existing Byzantine-resilient strategies implement some outlier detection heuristics to discard the model updates sent by potentially malicious clients from the input of the aggregation function.
% One of the most popular heuristics is Krum~\cite{blanchard2017nips}.
% This strategy tries to mitigate the impact of Byzantine attacks by selecting as a global model the local model with the smallest sum of Euclidean distances to {\em all} the other local models.
% Although powerful, Krum requires the server to know (or, at least, estimate) the number of malicious FL clients upfront, which is generally impossible in a realistic attack scenario. %
% Moreover, Krum may become ineffective for complex, high-dimensional model parameter spaces due to the curse of dimensionality.
% Bulyan~\cite{mhamdi2018pmlr} tries to overcome this issue by combining Krum with a variant of Trimmed Mean.
% % Data-driven outlier detection
% Other strategies use data-driven outlier detection techniques -- e.g., via K-means clustering~\cite{shen2016acm} -- to spot potential malicious local model updates. 
% %For instance, Shen et al. propose to cluster local model updates with K-means and thus identify outliers.
%
% % Other techniques
% As far as the server is concerned, any local model received can be from a potential malicious client. 
% FLTrust~\cite{cao2020fltrust} assumes the server acts as a client, i.e., trains a local model on an additional {\em trustworthy} dataset at the server's end and compares it against all the local models from other clients. 
% This way, the server can rely on some ``source of trust'' when discarding potentially malicious clients.
%\\
% Limitations of existing Byzantine-resilient strategies
Unfortunately, existing defense mechanisms either rely on simple heuristics (e.g., Trimmed Mean and FedMedian by~\cite{yin2018icml}) or need strong and unrealistic assumptions to work effectively (e.g., foreknowledge or estimation of the number of malicious clients in the FL system, as for Krum/Multi-Krum~\cite{blanchard2017nips} and Bulyan~\cite{mhamdi2018pmlr}, which, however, cannot exceed a fixed threshold).
Furthermore, outlier detection methods using K-means clustering~\cite{shen2016acm} or spectral analysis like DnC~\cite{shejwalkar2021ndss} do not directly consider the temporal evolution of local model updates received.
Finally, strategies like FLTrust~\cite{cao2020fltrust} require the server to collect its own dataset and act as a proper client, thereby altering the standard FL protocol.
\\
% OLD, LONG VERSION
% Overall, existing Byzantine-resilient strategies are either simple heuristics (e.g., FedMedian) or, if they are more complex, they rely on strong and unrealistic assumptions to work effectively (e.g., knowing the number of malicious clients in the FL system in advance, as for Krum and alike).
% Furthermore, data-driven outlier detection methods do not consider the temporary evolution of local model updates received (e.g., K-means clustering). 
% Finally, strategies like FLTrust requires the server to collect its own dataset and act as a proper client, thereby altering the standard FL protocol.
%
% Description of the proposed method
This work introduces a novel pre-aggregation \textit{filter} robust to untargeted model poisoning attacks. Notably, this filter $(i)$ operates without requiring prior knowledge or constraints on the number of malicious clients and $(ii)$ inherently integrates temporal dependencies. 
The FL server can employ this filter as a preprocessing step before applying \textit{any} aggregation function, be it standard like FedAvg or robust like Krum or Bulyan.
Specifically, we formulate the problem of identifying corrupted updates as a multidimensional (i.e., matrix-valued) time series anomaly detection task. 
The key idea is that legitimate local updates, resulting from well-calibrated iterative procedures like stochastic gradient descent (SGD) with an appropriate learning rate, show \textit{higher predictability} compared to malicious updates. This hypothesis stems from the fact that the sequence of gradients (thus, model parameters) observed during legitimate training exhibit regular patterns, as validated in Section~\ref{subsec:intuition}. %until convergence. 
%This regularity may be more pronounced for smooth convex loss functions, but it can still be captured within an appropriate time window, even for more complex and convoluted loss surfaces. 
%We provide evidence of this claim in Appendix~B, where we show that the average mutual information (i.e., ``predictability''), calculated over pairs of legitimate model updates sent at different FL rounds, is significantly higher than the corresponding computation for a malicious client.
\\
Inspired by the matrix autoregressive (MAR) framework for multidimensional time series forecasting~\cite{chen2021je}, we propose the FLANDERS ({\em \textbf{F}ederated \textbf{L}earning meets \textbf{AN}omaly \textbf{DE}tection for a \textbf{R}obust and \textbf{S}ecure}) filter.
The main advantages of FLANDERS over existing strategies like FLDetector~\cite{zhao2020multivariate} are its resilience to large-scale attacks, where $50\%$ or more FL participants are hostile, and the capability of working under realistic non-iid scenarios.
We attribute such a capability to two key factors: $(i)$ FLANDERS works without knowing a priori the ratio of corrupted clients, and $(ii)$ it embodies temporal dependencies between intra- and inter-client updates, quickly recognizing local model drifts caused by evil players. Below, we summarize our main contributions:

\begin{itemize}
\item[{\em(i)}]
We provide empirical evidence that the sequence of models sent by legitimate clients is more predictable than those of malicious participants performing untargeted model poisoning attacks.
\\
\item[{\em(ii)}] 
We introduce FLANDERS, the first pre-aggregation filter for FL robust to untargeted model poisoning based on multidimensional time series anomaly detection.
\\
\item[{\em(iii)}] 
We integrate FLANDERS into Flower,\footnote{\scriptsize{\url{https://flower.dev/}}} a popular FL simulation framework for reproducibility.
\\
\item[{\em(iv)}] 
We show that FLANDERS improves the robustness of the existing aggregation methods under multiple settings: different datasets, client's data distribution (non-iid), models, and attack scenarios.
\\
\item[{\em(v)}] 
We publicly release all the implementation code of FLANDERS along with our experiments.\footnote{\scriptsize{\url{https://anonymous.4open.science/r/flanders_exp-7EEB}}}
\end{itemize}

% Paper's structure and organization
The remainder of the paper is structured as follows. %some related work and the current state-of-the-art solutions to security issues that FL entails. 
Section~\ref{sec:background} covers background and preliminaries. 
In Section~\ref{sec:related}, we discuss related work.
Section~\ref{sec:problem} and Section~\ref{sec:method} describe the problem formulation and the method proposed. % to tackle it. 
Section~\ref{sec:experiments} gathers experimental results. %, and Section~\ref{sec:limitations} discusses some limitations of this work.
Finally, we conclude in Section~\ref{sec:conclusion}.
 %discusses the limitations of this work and draws future research directions.
%reports conclusions and draws perspectives for future research directions.

%%%%%%% OLD %%%%%%%
%to overcome the resilience of Byzantine failures in distributed Stochastic Gradient Descent computations. 
% The strength of Krum is its time complexity, which is linear in the gradient dimension. 
% However, the robustness of the approach is guaranteed for gradient-based learning applications only when the majority of the clients are not compromised. 
% Besides, the aggregation mechanism of Krum, as well as that of similar methods, is robust from a coarse-grained perspective and does not provide solutions to errors and perturbations that may occur at inference time.
%A related approach to~\cite{blanchard2017nips} is the work of Su et al.~\cite{su2016dc}. Here, the authors propose an iterated approximate agreement to tackle a multi-layer scenario attacked by Byzantine agents. 
%However, the method works efficiently on the sole discrete context and it is inapplicable to continuous state environments.
%\gabri{Maybe, we should just talk about the main limitations of existing countermeasures without digging into their details (or, we can just mention Krum as this is the most popular one). I will move the description of all these methods to the Related Work section.}
\subsection{Application: 2D-to-3D Transfer}
\label{sec:fabircated_prop}

Applications such as 2) and 4) can be easily integrated with application 1) incremental reconstruction (\cref{sec:incremental_reconstruction}) by incorporating the fabricated result together with the point cloud.
%
For instance, given RGB-D frames, we detect saliency or transfer image styles to generate a fabricated $X$ image. Here, $X$ represents saliency, style, or other properties. 
By combining $X$ with depth information through unprojection,
we assign
the fabricated values to corresponding points, resulting in point pairs ($\V X$, $\V Q_{X}$).

Similar to the reconstruction pipeline in~\cref{fig:recons_and_scene_understanding}, we employ encoding (\cref{sec:encoder}) and fusion (\cref{eq:fuse}) to construct a global LIM for the fabricated properties $X$.
This global LIM represents a surface $X$ fields that is utilized for subsequent inference.

While it is possible to similarly transfer a 2D semantic image to 3D,
it may not be feasible in practice due to the need for multiple passes of different categories of semantic information 
 on the same dataset (such as object, usability, etc.).
Therefore, in the following section, we demonstrate the construction of a surface feature field for scene understanding application that satisfies various 
requirements through a single mapping pass.


\section{Identifiability results}
\label{Section_identifiability}
In this section, we are interested in ascertaining whether it is possible to infer a causal graph from a joint distribution under the assumptions presented in the previous section. First, we rephrase the notion of identifiability from Definition \ref{IdentifiabilityPrvaDefinicia}. 

\begin{definition}[Identifiability]\label{DEFidentifiability}
Let $F_{(X_1, X_2)}$ be a distribution that has been generated according to the $CPCM(F_1,\dots, F_k)$ model with graph $X_1\to X_2$. We say that the causal graph is identifiable from the joint distribution (equivalently, that the model is identifiable) if there does \textit{not} exist 
$\tilde{\theta}$ and a pair of random variables $\tilde{\varepsilon}_2\indep\tilde{\varepsilon}_1$, where $\tilde{\varepsilon}_1$ is uniformly distributed, such that the model $X_2=\tilde{\varepsilon}_2,  X_1=  F_i^{-1}\big(\tilde{\varepsilon}_1;\tilde{\theta}(X_2)\big)$ for some $i\in\{1, \dots, k\}$ generates the same distribution $F_{(X_1,X_2)}$. 
\end{definition}

In the remainder of the section, we answer the following question: Under what conditions is the causal graph identifiable? 
\subsection{Identifiability in $CPCM(F)$}

First, we discuss the Gaussian case. Recall that in the additive Gaussian model, where $X_2=f(X_1)+\varepsilon_2$, $\varepsilon_2\sim N(0, \sigma^2)$, the identifiability holds if and only if $f$ is non-linear \citep{hoyer2009}. We provide a different result with both mean \textit{and} variance as functions of the cause. A similar result is found in \cite[Theorem 1]{Khemakhem_autoregressive_flows} in the context of autoregressive flows and where only a sufficient condition for identifiability is provided. Another similar problem is studied in \cite{immer2022identifiability} and \cite{strobl2022identifying}, both of which show identifiability in general location-scale models. 

\begin{theorem}[Gaussian case]\label{normalidentifiability}
Let $(X_1,X_2)$ admit the $CPCM(F)$ model with graph $X_1\to X_2$, where $F$ is the Gaussian distribution function with parameters $\theta(X_1)=\big(\mu(X_1), \sigma(X_1)\big)^\top$.~\footnote{Just like in Example \ref{Gaussian case}, this can be rewritten as $X_2 = \mu(X_1)+\sigma(X_1)\varepsilon_2,  \text{  where }\varepsilon_2 \text{ is Gaussian}$.}

Let $p_{\varepsilon_1}$ be the density of $\varepsilon_1$ that is absolutely continuous with full support $\mathbb{R}$. Let $\mu(x), \sigma(x)$ be two times differentiable.  Then, the causal graph is identifiable from the joint distribution if and only if there do not exist $a,c,d,e, \alpha, \beta\in\mathbb{R}$,  
$a\geq 0,c>0, \beta>0$, such that
\begin{equation}\label{norm}
\frac{1}{\sigma^2(x)}=ax^2 + c, \,\,\,\,\,\,\,\,\,\,\,\,\,\,\,\,\,\,\,\,\, \frac{\mu(x)}{\sigma^2(x)}=d+ex,
\end{equation}
for all $x\in\mathbb{R}$ and
\begin{equation}\label{DensityDEF}
p_{\varepsilon_1}(x) \propto \sigma(x)e^{-\frac{1}{2}\big[ \frac{(x-\alpha)^2}{\beta^2}  - \frac{\mu^2(x)}{\sigma^2(x)}\big]},
\end{equation}
where $\propto$  represents an equality up to a constant (here, $p_{\varepsilon_1} $  is a valid density function if and only if $\frac{1}{\beta^2}>  \frac{e^2}{c}\mathbbm{1}[a=0]$). 
Specifically, if $\sigma(x)$ is constant (case $a=0$), then the causal graph is identifiable, unless $\mu(x)$ is linear and $p_{\varepsilon_1}$ is the Gaussian density. 
\end{theorem}

The proof is provided in \hyperref[Proof of normalidentifiability]{Appendix} \ref{Proof of normalidentifiability}. Moreover, a visual example of an unidentifiable Gaussian model with $a=c=d=e=\alpha = \beta=1$ can be found in  \hyperref[Proof of normalidentifiability]{Appendix} \ref{Proof of normalidentifiability}, Figure \ref{GaussianDensity}. 

Theorem \ref{normalidentifiability} indicates that the non-identifiability holds only in the ``special case,'' when $\frac{\mu(x)}{\sigma^2(x)}, \frac{-1}{2\sigma^2(x)}$ are linear and quadratic, respectively. Note that natural parameters of a Gaussian distribution are  $\frac{\mu}{\sigma^2}, \frac{-1}{2\sigma^2}$, and sufficient statistics of the Gaussian distribution have a linear and quadratic form (for the definition of the exponential family, natural parameter and sufficient statistic, see \hyperref[appendix_exponential_family]{Appendix} \ref{appendix_exponential_family}). We show that such connections between non-identifiability and sufficient statistics hold in the more general context of the exponential family.  

\begin{proposition}[General case, one parameter]\label{Necessary condition for identifiability}
Let $q=1$. Let $(X_1, X_2)$ admit the $CPCM(F)$ model with graph $X_1\to X_2$, where $F$ lies in the exponential family of distributions with a sufficient statistic $T$. The causal graph is identifiable if at least one of the following conditions holds:
\begin{enumerate}
    \item There do not exist $a,b\in\mathbb{R}$, such that
 \begin{equation}\label{eq000}
     \theta(x)=a\cdot T(x)+b,\,\,\,\,\,\,\,\,\,\,\,\,\, \forall x\in supp(X_1). 
 \end{equation} 
 \item There does not exist $c\in\mathbb{R}$, such that 
\begin{equation}\label{eq007}
p_{X_1}(x)\propto \frac{h_1(x)}{h_2[\theta(x)]}e^{cT(x)}, \,\,\,\,\,\,\,\,\,\,\,\,\,\,\,\,\,\,\,\, \forall x\in supp(X_1),
\end{equation}
where $h_1$ is a base measure of $F$ and $h_2$ is the normalizing function of $F$ defined in \hyperref[appendix_exponential_family]{Appendix} \ref{appendix_exponential_family}. 
\item Supports of $X_1$ and $X_2$ differ. 
\end{enumerate}
\end{proposition}
\begin{proof}
We first show that if the causal graph is not identifiable, then there exist  $a,b\in\mathbb{R}$ such that  (\ref{eq000}) holds. 

If the graph is not identifiable, there exists a function $ \tilde{\theta}$, such that 
causal models $X_1 = \varepsilon_1, X_2 = F^{-1}\big(\varepsilon_2; \theta(X_1)\big)$, and $X_2 = \varepsilon_2, X_1 = F^{-1}\big(\varepsilon_1; \tilde{\theta}(X_2)\big)$ generate the same joint distribution.
 Decomposing the joint density yields
\begin{equation}\label{eq69}
  p_{(X_1, X_2)}(x,y) = p_{X_1}(x)p_{X_2\mid {X_1}}(y\mid x) = p_{X_2}(y)p_{{X_1}\mid {X_2}}(x\mid y),\,\,\,\,\,\,\,\, x,y\in supp(X_1).
 \end{equation}
Since $F$ lies in the exponential family of distributions, we use the notation from \hyperref[appendix_exponential_family]{Appendix} \ref{appendix_exponential_family} and rewrite this as $$p_{X_2\mid {X_1}}(y\mid x) = h_{1}(y)h_{2}[\theta(x)]\exp[\theta(x)T(y)]$$
and analogously for $p_{{X_1}\mid {X_2}}(x\mid y)$. Then, the second equality in (\ref{eq69}) becomes
\begin{equation}\label{eq098}
    \begin{split}
  &   p_{X_1}(x)h_{1}(y)h_{2}[\theta(x)]\exp[\theta(x)T(y)] =p_{X_2}(y) h_{1}(x)h_{2}[\tilde{\theta}(y)]\exp[\tilde{\theta}(y)T(x)],\\&
\underbrace{\log\bigg(  \frac{ p_{X_1}(x)h_{2}[\theta(x)]}{h_{1}(x)}\bigg)}_{f(x)}+ \underbrace{\log\bigg(\frac{h_{1}(y)}{ p_{X_2}(y)h_{2}[\tilde{\theta}(y)]}\bigg)}_{g(y)} = \tilde{\theta}(y)T(x)-\theta(x)T(y),
\end{split}
\end{equation}
where the second equation is the logarithmic transformation of the first equation after dividing both sides by $h_1$ and $h_2$. Since the left side is in the additive form, Lemma \ref{PomocnaLemma1} directly yields (\ref{eq000}) (to see this without Lemma \ref{PomocnaLemma1}, apply $\frac{d}{dx dy}$  to (\ref{eq098}) and fix $y$ such that $T'(y)\neq 0$. We get 
$\theta'(x) = \frac{\tilde{\theta}'(y)}{T'(y)}T'(x)$. Integrating this equality with respect to $x$ yields (\ref{eq000}). However, using Lemma \ref{PomocnaLemma1}, we do not need to assume the differentiability of $\theta$).

With regard to equation (\ref{eq007}), equality (\ref{eq098}) implies 
$f(x) + g(y) = a_1T(x) + a_2T(y) +a_3$ for some constants $a_1, a_2, a_3\in\mathbb{R}$. Therefore, fixing $y$ yields $f(x)=\log\bigg\{  \frac{ p_{X_1}(x)h_{2}[\theta(x)]}{h_{1}(x)}\bigg\} = a_1T(x) + const$. Rewriting this yields 
$
p_{X_1}(x)= \frac{h_1(x)}{h_2[\theta(x)]}e^{a_1T(x) + const},
$
which is exactly the form of (\ref{eq007}). 

Part 3) is a direct consequence of the fact that the support of any distribution in the exponential family is fixed and does not depend on $\theta$. 
\end{proof}


Proposition \ref{Necessary condition for identifiability} indicates that the only case in which the model is not identifiable is when $\theta(x)$ has a uniquely given functional form (unique up to a linear transformation) and the density of the cause is uniquely given (unique up to one additional parameter $c$).  Note that it is only a sufficient, not a necessary, condition for identifiability since some specific choices of $a,b,$ and $c$ can still lead to identifiable cases.  


As a consequence of Proposition \ref{Necessary condition for identifiability}, we extend the results demonstrated by \cite{ParkPoisson} and \cite{ParkGHD} for a Poisson DAG model. These authors established the identifiability of a Poisson DAG model, where all variables (including source variables) given their parents follow a Poisson distribution. We present an analogous result, relaxing the restriction on the source variables.

\begin{consequence}\label{paretoidentifiability}
\begin{itemize}
    \item Let $(X_1,X_2)$ admit the $CPCM(F)$ model with graph $X_1\to X_2$, where $F$ is the Poisson distribution function with rate $\lambda$.  Then, the causal graph is \textit{not} identifiable if and only if
\begin{equation*}
 \lambda(x) =e^{ax+b},\,\,\,\,\,\,\,\, P(X_1=x) \propto \frac{e^{e^{ax+b}+cx}}{x! }, \,\,\,\,\,\,\,\,\forall x\in \{0,1,2, \dots \},
 \end{equation*}
for some $a<0,b,c\in\mathbb{R}$.
    \item Let $(X_1,X_2)$ admit the $CPCM(F)$ model with graph $X_1\to X_2$, where $F$ is the Pareto distribution function. Then, the causal graph is \textit{not} identifiable if and only if 
\begin{equation*}
\theta(x) = a\log(x) +b,\,\,\,\,\,\,\,\, p_{X_1}(x) \propto \frac{1}{ [a\log(x)+b] x^{c+1} }, \,\,\,\,\,\,\,\,\forall x\geq 1,
\end{equation*}
for some $a,b,c>0$. 

    \item Let $(X_1,X_2)$ admit the $CPCM(F)$ model with graph $X_1\to X_2$, where $F$ is Bernoulli distribution function and $supp(X_1) = supp(X_2) = \{0,1\}$.  Then, the causal graph is \textit{not} identifiable. 
\end{itemize}
\end{consequence}
The proof is provided in \hyperref[Proof of pareto identifiability]{Appendix} \ref{Proof of pareto identifiability}, together with definitions of the distribution functions. Note that if $a=0$, then $X_1\indep X_2$ and the (empty) graph is trivially identifiable. 

Note that in the first two bullet points of Consequence~\ref{paretoidentifiability}, we have three free parameters: \(a\), \(b\), and \(c\). The non-identifiability of the graph in a Bernoulli model arises from the fact that the joint distribution of \((X_1, X_2)\) can be fully characterized by only three parameters.

\subsection{Identifiability in $CPCM(F_1, F_2)$ models}


Similar sufficient conditions as in Proposition \ref{Necessary condition for identifiability} can be derived for a more general case, when $F$ has several parameters and for the $CPCM(F_1, F_2)$. 


\begin{theorem}
\label{thmAssymetricMultivariatesufficient}
Let $(X_1, X_2)$ follow the $CPCM(F_1, F_2)$ model with graph $X_1\to X_2$, where $F_1, F_2$ lie in the exponential family of distributions and $T_1 = (T_{1,1}, \dots, T_{1,q_1})^\top$, $T_2 = (T_{2,1}, \dots, T_{2,q_2})^\top$are the corresponding sufficient statistics.  Assume either $F_1 = F_2$ (that is, we assume $CPCM(F_1)$ model) or $supp(F_1)\neq supp(F_2)$ and define $\mathcal{S}=supp(F_1)\cap supp(F_2)$. 

The causal graph is identifiable if $\theta_2$ is not a linear combination of $T_{1,1}, \dots, T_{1,q_1}$ on $\mathcal{S}$---that is, if $\theta_2$ cannot be written as
\begin{equation}\label{eq158}
\theta_{2,i}(x) \overset{}{=} \sum_{j=1}^{q_1}a_{i,j}T_{1,j}(x)+b_i,\,\,\,\,\,\,\,\,\,\,\,\,\,\,\, \forall x\in \mathcal{S},
\end{equation}
for all $i=1, \dots, q_2,$ and for some constants $a_{i,j},b_i\in\mathbb{R}$, $j=1, \dots, q_1$. 
\end{theorem}

The proof is provided in \hyperref[Proof of thmAssymetricMultivariatesufficient]{Appendix} \ref{Proof of thmAssymetricMultivariatesufficient}. Note that condition (\ref{eq158}) is sufficient, but not necessary for identifiability. Similar to that in the Gaussian  (Theorem \ref{normalidentifiability}) or Pareto cases (Consequence \ref{paretoidentifiability}), in order to obtain a necessary condition, the distribution of the cause also needs to be restricted.

Note that Theorem~\ref{thmAssymetricMultivariatesufficient} can be iteratively applied to obtain identifiability of $CPCM(F_1, \dots, F_k)$ models. 

\begin{consequence}\label{consequenceprva}
\begin{itemize}
\item Let \((X_1, X_2)\) admit the \(CPCM(F)\) model (\ref{BCPCM}) with graph \(X_1 \to X_2\), where \(F\) is a Gamma distribution with parameters \(\theta = (\alpha, \beta)^\top\). If there do not exist constants \(a, b, c, d, e, f \in \mathbb{R}\) such that
\begin{equation*}
\alpha(x) = a\log(x) + bx + c, \quad \beta(x) = d\log(x) + ex + f, \quad \forall x > 0,
\end{equation*}
then the causal graph is identifiable.

\item Suppose that \( \text{supp}(X_1) = \mathbb{R} \), \(\text{supp}(X_2) = \{0, 1, \dots\}\) such as on Figure~\ref{Asymmetrical_picture}, and let \((X_1, X_2)\) admit the \(CPCM(F_1, F_2)\) model with graph \(X_1 \to X_2\), where \(F_1\) is a Gaussian distribution with mean \(\mu\) and fixed variance, and \(F_2\) is a Poisson distribution with rate parameter \(\lambda\). If there do not exist constants \(a_i, b_i, c_i, d_i \in \mathbb{R}\), \(i = 1, 2\), such that for all \(x \in \{0, 1, \dots\}\)
\begin{equation*}
\lambda(x) = e^{a_1 x + b_1}, \quad p_{X_1}(x) \propto e^{\lambda(x) + c_1 x^2 + d_1 x},\,\,\,\,\,\,\,\, \quad \mu(x) = a_2 + b_2 x, \quad p_{X_2}(x) \propto \frac{1}{x!} e^{c_2 x^2 + d_2 x},
\end{equation*}
then the causal graph is identifiable.
\end{itemize}
\end{consequence}
Details regarding Consequence \ref{consequenceprva}, the definitions of the distributions, their sufficient statistics and more examples are provided in  \hyperref[consequence]{Appendix} \ref{consequence}. We emphasize that not all constants and combinations of parameters yield a non-identifiable model.

It is interesting to observe the dimension of the space of unidentifiable distributions for different choices of \(F\). In the one-parameter cases, such as in Consequence~\ref{paretoidentifiability}, the set of all unidentifiable distributions is contained in a three-dimensional space, similar to the case for ANM \cite[Proposition 21]{Peters2014}. Not surprisingly, if \(F\) has more parameters, this dimension increases, but it remains finite. Since the space of all distributions is infinite-dimensional (assuming infinite support), one can argue that regardless of the choice of $F$, identifiability holds for ``most distributions''. However, if \(F\) has many parameters, we often find ourselves ``close'' to an unidentifiable case, making inference much more challenging.






































%Rozmysli si F_i being CONDITIONAL distirubiton functions
%Zmaz vsetky bold math symboly

\section{Multivariate case}
\label{Section4}

Now, we move the theory to the case with more than two variables $d\geq 2$. We assume causal sufficiency (all relevant variables have been observed) and causal minimality. On the other hand, we do not assume faithfulness. 
It is straightforward to generalize the asymmetric $\M$-causal model to the multivariate case, where we assume that each variable $X_i$ arises under the $\mathcal{M}_i$ model. 

To be more rigorous, we define a multivariate CPCM, as a generalization of (\ref{asymetrical_F_one_F_two_model}). 

\begin{definition}\label{DefinitionCPCM}
Let $F_1, \dots, F_d$ be distribution functions with $q_1, \dots, q_d$ parameters respectively. 
We define an asymmetrical $\mathcal{M}_{F_1}, \dots, \mathcal{M}_{F_d}$-causal model ($CPCM(F_1, \dots, F_d)$ for short) as a collection of equations 
\begin{equation*}
S_j:  X_j = F^{-1}_{j}(\varepsilon_j; \theta_j\big(\textbf{X}_{pa_j})\big), j=1, \dots, d,
\end{equation*}
 and we assume that the corresponding causal graph is acyclic. $(\varepsilon_1, \dots, \varepsilon_d)^\top$ is a collection of jointly independent uniformly distributed random variables. $\theta_j: \mathbb{R}^{|pa_j|}\to \mathbb{R}^{q_j}$ are non-constant functions in any of their arguments and are continuous if $\textbf{X}_{pa_j}$ are continuous. 

If $F_j^{-1} = F^{-1}$  for some quantile function $F^{-1}$ and for all $j\not\in Source(\mathcal{G}_0)$, we call such a model the conditionally parametric causal model ($CPCM(F)$).
\end{definition}
Simply said, we assume that $X_j\mid \textbf{X}_{pa_j}$ is distributed according to the distribution $F_j$ with parameters $\theta_j(\textbf{X}_{pa_j})$. We assume all $F_j$ \textit{are known} besides the source variables. 


The question of the identifiability of $\mathcal{G}$ in the multivariate case is in order. Here, it is not satisfactory to consider the identifiability of each pair of $X_i\to X_j$ separately. Each pair $X_i, X_j$  needs to have an identifiable causal relation \textit{conditioned} on other variables $\textbf{X}_S$, as the following theorem suggests. 


 \begin{theorem}\label{thmMultivairateIdentifiability}
Let $F_{\textbf{X}}$ be generated by the $CPCM(F_1, \dots, F_d)$ with DAG $\mathcal{G}$ and with density $p_{\textbf{X}}$. Let for all $ i,j\in\mathcal{G}, i\in pa_j$ hold the following: $\forall S\subseteq V$ such that  $pa_j\setminus \{i\}\subseteq S \subseteq nd_j\setminus\{i,j\}$ there exist $\textbf{x}_{S}: p_{\textbf{X}_S}(\textbf{x}_S)>0$ satisfying: a bivariate model defined as $X=\tilde{\varepsilon}_X, Y = F^{-1}_j\big(\tilde{\varepsilon}_Y, \tilde{\theta}(X)\big)$ is identifiable (in the sense of Definition \ref{DEFidentifiability}), where  $F_{\tilde{\varepsilon}_X} = F_{X_i\mid \textbf{X}_{S} =\textbf{ x}_S}    $ and $\tilde{\theta}(x) = \theta_j(\textbf{x}_{pa_j\setminus\{i\}}, x)$, where $x\in supp(X)$.

Then,  $\mathcal{G}$ is identifiable from the joint distribution. 
 \end{theorem}
 The proof is provided in \hyperref[Proof of thmMultivairateIdentifiability]{Appendix} \ref{Proof of thmMultivairateIdentifiability}. It is an adaptation of \citep[Theorem 28]{Peters2014}. 
 An important special case arises when we assume (conditional) normality.

\begin{consequence}[Multivariate Gaussian case]\label{ExampleMultivariateGaussiancase}
Suppose that $\textbf{X}=(X_1, \dots, X_d)$ follow $CPCM(F)$ with a Gaussian distribution function $F$. This corresponds to $X_j\mid \textbf{X}_{pa_j}\sim N\big(\mu_j(\textbf{X}_{pa_j}), \sigma_j^2(\textbf{X}_{pa_j})\big)$ for all $j=1, \dots, d$ and for some functions $\mu_j, \sigma_j$. In other words, we assume that the data-generating process has a form 
$$
X_j = \mu_j(\textbf{X}_{pa_j}) + \sigma_j(\textbf{X}_{pa_j})\varepsilon_j, \,\,\,\,\,\,\,\,\,\,\,\,\,\,\,\,\, \varepsilon_j \text{  is Gaussian.}
$$
Potentially, source nodes can have arbitrary distributions. Combining Theorem~\ref{normalidentifiability} and Theorem~\ref{thmMultivairateIdentifiability}, the causal graph $\mathcal{G}$ is identifiable if the following holds: 
functions $\theta_j(\textbf{x}):=\big(\mu_j(\textbf{x}), \sigma_j(\textbf{x})\big)^\top, \textbf{x}\in\mathbb{R}^{|pa_j(\mathcal{G})|}$ , $j=1, \dots, d$, are two times differentiable and they are \textit{not} in the form (\ref{norm}) in any of their arguments. 
\end{consequence}



















\section{Inference}
\label{Section5}

\subsection{Algorithm for CPCM using independence testing}
\label{Section_Algorithm}
Our CPCM methodology is based on selecting an appropriate causal model (in our case, the choice of collection $\{F_1, \dots, F_k\}$) and a measure of a model fit. In the following subsections, we measure the model fit by exploiting the principle of independence between the cause and the mechanism. 


A causal graph is said to be \textbf{plausible} under $CPCM(F)$ model if the joint distribution \textit{can} be generated via $CPCM(F)$ model with such a graph. The Algorithm~\ref{Algorithm1} describes the main steps to test the plausibility. 

\begin{algorithm}[H]
  \SetAlgoLined
  \KwData{ Random sample $(x_{1,1}, x_{2,1})^\top, \dots, (x_{1,n}, x_{2,n})^\top$}
  \KwResult{ Plausibility of a graphs $X_1\to X_2$, $X_2\to X_1$ and an estimate of $\mathcal{G}$}
\textbf{Step 0) }Test for independence between $X_1$ and $X_2$. If not rejected, return an empty graph.

\textbf{Step 1) } Determine plausibility of   $X_1\to X_2$ using the following: 


\textbf{$\,\,\,\,\,\,$Step 1a)} Estimate $\theta(X_1)$ in the model $X_2 = F\big(\varepsilon_2; \theta(X_1)\big)$ and compute probability transform $\hat{\varepsilon}_2 := F\big(X_1; \hat{\theta}(X_1)\big)$ 

\textbf{$\,\,\,\,\,\,$Step 1b)} Test an independence between $\hat{\varepsilon}_2$ and $X_1$ on level $\alpha$. Direction $X_1\to X_2$ is marked as \textit{plausible}  if the test is not rejected. 

\textbf{Step 2) }  Repeat for the other direction $X_2\to X_1$. 

\textbf{Estimate of $\mathcal{G}$: }  If one direction is plausible and the other is not, the former is the final estimate. If both directions are plausible, return ``insufficient data or unidentifiable setup''. If both directions are not plausible, return ``Assumptions not fulfilled''. 
  \caption{CPCM(F)}
  \label{Algorithm1}
\end{algorithm}

An estimation of $\hat{\theta}(X_1)$ in the Step 1a) can be done using any machine learning algorithm, such as GAM, GAMLSS, random forest, or neural networks \citep{GAM, GAMLSS}. For the independence test in in  Step 1b), we can use a HSIC test (kernel-based Hilbert-Schmidt independence test, \cite{Kernel_based_tests}) or a copula-based test \citep{copula_based_independence_test}. 

Generalizing Algorithm~\ref{Algorithm1} for $d$ variables is straightforward. For each $\mathcal{G}\in DAG(d)$, we estimate  $\hat{\theta}_i(\textbf{X}_{pa_i(\mathcal{G})})$ for all $i = 1, \dots, d$ and compute the probability transform $\hat{\varepsilon}_i:= F\big(X_i; \hat{\theta}_i(\textbf{X}_{pa_i(\mathcal{G})})\big)$. Then, we can test independence between  $\hat{\varepsilon}_1, \dots, \hat{\varepsilon}_d$ and conclude that $\mathcal{G}$ is plausible if this test is not rejected. However, even if the causal graph is identifiable, multiple graphs often remain plausible for finite sample sizes unless $n$ is very large. In Section~\ref{Section_score_based_algorithm}, we introduce the score-based algorithm, which selects the ``best'' causal graph even when many plausible graphs exist.

The following adjustment to Step 1 in Algorithm~\ref{Algorithm1} can be applied to accommodate the \(CPCM(F_1, \dots, F_k)\) model, given a collection \(\{F_1, \dots, F_k\}\).


\begin{algorithm*}[H]
  \SetAlgoLined
\textbf{Step 1) } Determine plausibility of   $X_1\to X_2$ using the following: 

\textbf{$\,\,\,\,\,\,$Step 1a) } Estimate the set $S:=\{j\in \{1, \dots, k\}: supp(F_j) = supp(X_2)\}$. If empty, return ``Inappropriate choice of $F$''. 

\textbf{$\,\,\,\,\,\,$Step 1b) } For all  $j\in \hat{S}$,  estimate $\theta(X_1)$ in a model $X_2 = F_j\big(\varepsilon_2; \theta(X_1)\big)$,  and compute probability transform $\hat{\varepsilon}_2^j := F_j\big(X_2; \hat{\theta}(X_1)\big)$.  

\textbf{$\,\,\,\,\,\,$Step 1c)} Test an independence between $\hat{\varepsilon}_2^j$ and $X_1$ for all $j\in\hat{S}$. Direction $X_1\to X_2$ is marked as \textit{plausible}  if this test is not rejected for at least one $j\in \hat{S}$. 

%  \caption{Modification of Step 1 in Algorithm~\ref{Algorithm1} accounting for $CPCM(F_1, \dots, F_k)$}
%  \label{Algorithm2}
\end{algorithm*}

By estimating the set $S$ in Step 1a), we can  preliminarily filter out the $F_j$ choices that are evidently unsuitable. We can apply a simple heuristic under the assumption that \( \text{supp}(X_2) \) and \( \text{supp}(F_j) \) for all \( j \) are one of the following: 1) \(\mathbb{R}\) (Gaussian), 2) \(\mathbb{R}^+\) (Gamma), 3) \([0,1]\) (Beta), or 4) \(\mathbb{N}\)(Poisson). The heuristic is as follows: if the number of unique values in \( (x_{2,1}, \dots, x_{2,n}) \) is fewer than \( n/10 \) and the values lie in $\mathbb{N}$, we set \( \text{supp}(X_2) = \mathbb{N} \). Otherwise, if all values lie within \([0,1]\), we set \( \text{supp}(X_2) = [0,1] \). To distinguish between \(\mathbb{R}\) and \(\mathbb{R}^+\), we use the skewness of the distribution: if the skewness is close to 0, the distribution resembles a Gaussian distribution, so we set \( \text{supp}(X_2) = \mathbb{R} \). Otherwise, the distribution resembles a Gamma distribution, so we set \( \text{supp}(X_2) = \mathbb{R}^+ \).











\subsection{Score-based algorithm for CPCM(F) }
\label{Section_score_based_algorithm}
Algorithm~\ref{Algorithm1} does not always provide an output. The possibility of rejecting the independence test in all directions can be considered a safety net, thereby shielding us against unfulfilled assumptions or unidentifiable cases. Nevertheless, we may still want to obtain an estimation of the graph. 

Following the ideas of \cite{Score-based_causal_learning} and \cite{Peters2014}, we use the following penalized independence score: 
\begin{equation*}
\hat{\mathcal{G}} =  \argmin_{\mathcal{G}\in DAG(d)}s(\mathcal{G}) = \argmin_{\mathcal{G}\in DAG(d)}\rho (\hat{\varepsilon}_1, \dots, \hat{\varepsilon}_d) + \lambda (\text{Number of edges in }\mathcal{G}),
\end{equation*}
where $\rho$ represents some measure of independence, and $\hat{\varepsilon}_1, \dots, \hat{\varepsilon}_d$ are noise estimations obtained by estimating $\hat{\theta}_i(\textbf{X}_{pa_i(\mathcal{G})})$ and putting $\hat{\varepsilon}_i := F\big(X_i; \hat{\theta}_i(\textbf{X}_{pa_i(\mathcal{G})})\big)$ such as in Algorithm~\ref{Algorithm1}. 

With regard to choice of $\rho$, we use minus the logarithm of the p-value of the copula-based independence test \citep{copula_based_independence_test} and $\lambda = 2$. These choices appear to work well in practice, but we do not provide any theoretical justification of their optimality.

The main disadvantage of the proposed method is that we have to go through all graphs $\mathcal{G}\in DAG(d)$, which is possible only for small $d$, since the number of DAGs grows superexponentially with dimension $d$ \citep{NP-hard_score_based_causal_learning}. Several modifications can be used in order to speed up the process. In this regard, greedy algorithms have been proposed \citep{Greedy_search, Bregmans_information}; however, this is beyond the scope of this paper. 

\subsubsection{$CPCM(F_1,\dots, F_k)$ extension and consistency
}
Similarly to the extension of Algorithm~\ref{Algorithm1}, the following adjustment can be applied to accommodate the \(CPCM(F_1, \dots, F_k)\) model, given a collection \(\{F_1, \dots, F_k\}\):

\begin{equation*}
s(\mathcal{G}) = \min_{j_1\in \hat{S}_1, \dots, j_d\in \hat{S}_d}\rho (\hat{\varepsilon}_1^{j_i}, \dots, \hat{\varepsilon}_d^{j_d}) + \lambda (\text{Number of edges in }\mathcal{G}),
\end{equation*}
where $\hat{S}_i$ are estimates of $S_i:= \{ j\in\{1, \dots, k\}: supp(X_i) = supp(F_j) \}$, and $\hat{\varepsilon}_i^{j_i} := F_{j_i}\big(X_i; \hat{\theta}_i(\textbf{X}_{pa_i(\mathcal{G})})\big)$. If $supp(F_i) \neq supp(F_j)$ for all $i\neq j$, we end up with a single evaluation of the score function for each possible $\mathcal{G}$. 

\cite{reviewANMMooij} demonstrates that if $(X_1, X_2)$ follows a bivariate additive noise model, the ANM algorithm consistently estimates the causal direction between $X_1$ and $X_2$. We establish an analogous result for $CPCM(F_1, \dots, F_k)$.


\begin{proposition}
\label{consistency_proposition}
Let $(X_1, X_2)$ follow an identifiable  $CPCM(F_1, \dots, F_k)$ with DAG $\mathcal{G}$. Then, our score based algorithm presented in Section~\ref{Section_score_based_algorithm} is consistent, meaning that
$$\hat{\mathcal{G}} \overset{P}{\to}\mathcal{G}\,\,\,as\,\,n\to\infty,$$
given that we employ a ``suitable'' estimation procedure for the estimation of $\hat{\varepsilon}_i$, we use HSIC score as our choice of $\rho$ and consistent estimators $\hat{S}_i$ (for proof, definitions and details, see  Appendix~\ref{Appendix_consistency}). 
\end{proposition}


\subsection{Choice of the collection $\{F_1, \dots, F_k\}$}
\label{Section5Model_choice}

Selecting  the collection $\{F_1, \dots, F_k\}$ is a crucial step in our approach. If this collection is large, we loose statistical power and increase the chance of having an unidentifiable model as the following lemma suggests. 


\begin{lemma}\label{lemma_o_overparametrizacii}
Suppose that the joint distribution $F_{(X_1,X_2)}$ is generated according to model $CPCM(F_2)$ with graph $X_1\to X_2$, where $F_2$ is a distribution function with one parameter ($q_2=1$) belonging to the exponential family with the corresponding sufficient statistic $T_2$. 

Then, there exists $F_1$ such that the model $CPCM(F_1)$  with graph $X_2\to X_1$ also generates  $F_{(X_1,X_2)}$. In other words, there exists $F_1$ such that the causal graph in $CPCM(F_1, F_2)$ is not identifiable from the joint distribution. 
\end{lemma}

The proof (provided in \hyperref[Proof of lemma_o_overparametrizacii]{Appendix} \ref{Proof of lemma_o_overparametrizacii}) is based on the specific choice of $F_{1}$, such that its sufficient statistic $T_1$ is equal to $\theta_2$ (where $\theta_2$ is the parameter from the original model $CPCM(F_{2})$). 

It is important to note that such $F_{1}$ would lead to a rather non-standard distribution. This leads us to define a ``standard set of well-known distributions''.   We discuss the practical choices for this set in Section~\ref{Section_practical_choices}. 

\subsubsection{Unfair game issue}

Even if the theory suggests that it is not possible to fit a $CPCM(F_1, \dots, F_k)$ in both causal directions, this is only an asymptotic result that is no longer valid for a finite number of data. If we select $F$ with numerous parameters to estimate, our data will be fitted perfectly and we will not reject the wrong causal graph. However, by selecting an overly simple $F$, our assumptions may not be fulfilled and we may end up rejecting the correct causal graph. We recommend choosing all $F_{i}$ with the same number of parameters $q_i = q_j$. If we select, say, $F_{1}$ with one parameter and $F_{2}$ with five parameters ($q_1 = 1, q_2 = 5$), it will create a bias because the model with more parameters will fit the data more easily. We refer to this as an ``unfair game.'' 

\subsubsection{Practical choices of the set of ``standard well-known distributions'' and the choices of tests used in our implementation }
\label{Section_practical_choices}

We define two sets of ``standard set of well-known distributions''; set $\mathscr{S}_1$  containing distributions with one parameter and set $\mathscr{S}_2$  with two parameters. This avoids the  ``unfair game'' issue discussed before. Both sets should be rich enough to contain a wide range of distributions with different supports and characteristics, but should not contain many distributions with the same support in order to avoid unidentifiable setups. In practice, we mostly restrict our attention to distributions that are implemented in \texttt{mgcv} package in \texttt{family.mgcv} \citep{Wood}.  


In our implementation, we used the following distributions: Gaussian with fixed variance, Gaussian, Gamma, Pareto, Poisson and Negative binomial distribution. Here, $\mathscr{S}_1$ consists of the distributions with one parameter and  $\mathscr{S}_2$ consists of the distributions with two parameters.  However, we should note that many more choices are appropriate and our collections are not exhaustive and may be improved for specific applications where potentially different distributions may be seen as ``standard''. 


The choice between $\mathscr{S}_1$ and $\mathscr{S}_2$ should be made based on the sample size. Choosing $\mathscr{S}_2$ in small sample sizes leads to not rejecting neither direction, while choosing $\mathscr{S}_1$ for large sample sizes can often results in rejecting both causal directions. 

In our implementation of Algorithm~\ref{Algorithm1}, we use GAM \citep{Wood2} estimation of $\hat{\theta}(X_i)$. With regard to the independence test, we use Hoeffding D-test \citep{Nonparametric_test_for_independence_review} in the bivariate case, copula-based independence test \citep{Kojadinovic2009} in the multivariate case with $n>1000$, and HSIC \citep{ZhangKernelTest} when $n\leq 1000$ (for a review of different tests and their comparisons, see \cite{copula_based_independence_test}). 






\section{Simulations} \label{simulations_section}
The \texttt{R} code with the implementations of the algorithms presented in the previous section and the code for the simulations and the application can be found in the supplementary package or \url{https://github.com/jurobodik/Causal_CPCM.git}.

In this section, we illustrate our methodology under controlled conditions. We first consider the bivariate case in which $X_1$ causes $X_2$. We select different distribution functions $F$ in the CPCM model, different forms of $\theta$, and different distributions of $\varepsilon_1$. We recreate a few of the theoretical results presented in Section 2. 


\subsection{Pareto case following Example \ref{example_Pareto} and Consequence  \ref{paretoidentifiability} }

Consider the Pareto distribution function $F$; functions $p_{\varepsilon_1}(x),\theta(x)$ are defined similarly as in (\ref{eq50}). Specifically, we choose $p_{\varepsilon_1}(x)\propto \frac{1}{ [\log(x)+1] x^{2} }$ and $\theta(x) = x^\alpha log(x) +1$ for some hyper-parameter $\alpha\in\mathbb{R}$. This $\alpha$ represents the distortion from the unidentifiable case. If $\alpha = 0$, we are in the unidentifiable case described in Consequence \ref{paretoidentifiability}. If $\alpha> 0$, Consequence \ref{paretoidentifiability} suggests that we should be able to distinguish between the cause and the effect. If $\alpha<0$, then $\theta$ is almost constant (function $\frac{log(x)}{x^{-\alpha}}$ is close to zero function on $x\in [1, \infty)$) and $(X_1, X_2)$ are (close to) independent.

For the size of the dataset $n =300$ and  $\alpha \in \{  -2,   0,  2\}$, we simulate data as described above. Using our $CPCM(F)$ algorithm from Section \ref{Section_Algorithm}, we obtain an estimate of the causal graph.  After averaging results from 100 repetitions, we obtain the results described in Figure \ref{Pareto_simulations1}. The resulting numbers are as expected: if $\alpha = 0$, then both directions tend to be plausible. If $\alpha >0$, we tend to estimate the correct direction $X_1\to X_2$; if $\alpha<0$, then we tend to estimate an empty graph since $X_1, X_2$ are (close to) independent.  


\begin{figure}[ht]
\centering
\includegraphics[scale=0.5]{figures/1.pdf}
\caption{Simulations corresponding to the CPCM model with Pareto distribution function $F$. The results represent our estimations of the graph structure with the $CPCM(F)$ algorithm from Section \ref{Section_Algorithm}. The green line represents the case when both directions are plausible (we do not reject the independence test in both directions). The yellow line represents the case in which both directions are unplausible (we reject the independence test in both directions. This case did not occur). }
\label{Pareto_simulations1}
\end{figure}



\subsection{The Gaussian case and comparison with baseline methods}
\label{Section_simulations_Gaussian}
For simulated data, we use the benchmark dataset introduced in \cite{Natasa_Tagasovska}. The dataset consists of additive and location-scale Gaussian pairs of the form $X_2= \mu(X_1)+\sigma(X_1)\varepsilon_2$, where $\varepsilon_2\sim N(0, 1)$, $X_1\sim N(0, \sqrt{2})$. In one setup (LSg), we consider  $\mu$ and $ \sigma$  as nonlinear functions simulated using Gaussian processes with Gaussian kernel with bandwidth one \citep{Gaussian_processes}.  In the second setup (LSs), we consider $\mu$ and $ \sigma$ as sigmoids \citep{BuhlmannCAM}. Further, nonlinear additive noise models (ANM) are generated as LS with constant $\sigma(X_1)=\sigma \sim U(1/5, \sqrt{2/5})$ and nonlinear multiplicative noise models (MN) are generated as LS with fixed $\mu(X_1)=0$ (only with sigmoid functions for $\sigma$).  For each of the five cases (LSg, LSs, ANMg, ANMs, and MNs), we simulate 100 pairs with $n=1000$ datapoints.

We compare our method with LOCI \citep{immer2022identifiability}, HECI  \citep{xu2022inferring}, RESIT \citep{Peters2014},  bQCD \citep{Natasa_Tagasovska}, IGCI with Gaussian and uniform reference measures \citep{IGCI} and Slope \citep{Slope}. Details can be found in  Appendix \ref{Appendix_simulations}.  As in \cite{reviewANMMooij}, we use the accuracy for forced decisions as our evaluation metric. The results are presented in Table \ref{Table_Simulated_data_Gaussian}.  We conclude that our estimator performs well on all datasets, and provides comparable results with those using LOCI and IGCI (although utilizing the uniform reference measure would lead to much worse results for IGCI).

Note that in the ANM and MN cases, we non-parametrically estimate two parameters while only one is relevant. Therefore, it can happen that we ``overfit'' (see Section \ref{Section5Model_choice}), and both directions are not rejected. To improve our results, fixing either $\mu$ or $\sigma$ could be beneficial (although it may prove challenging to determine this conclusively in practical applications). 

\begin{table}[!ht]
    \centering
    \begin{tabular}{|l|l|l|l|l|l|}
    \hline
        \textbf{} & \textbf{ANMg} & \textbf{ANMs} & \textbf{MNs} & \textbf{LSg} & \textbf{LSs} \\ \hline
        \textbf{Our CPCM} & {100} & 97 & 95 & {99} & {98} \\ \hline
        \textbf{LOCI} & {100} & {100} & {99} & 91 & 85 \\ \hline
        \textbf{HECI} & {99} & 43 & 29 & 96 & 54 \\ \hline
        \textbf{RESIT} & {100} & {100} & 39 & 51 & 11 \\ \hline
        \textbf{bQCD} & {100} & 79 & {99} & {100} & 98 \\ \hline
        \textbf{IGCI (Gauss)} & {100} & {99} & {99} & 97 & {100} \\ \hline
        \textbf{IGCI (Unif)} & 31 & 35 & 12 & 36 & 28 \\ \hline
        \textbf{Slope} & 22 & 25 & 9 & 12 & 15 \\ \hline
    \end{tabular}
    \caption{Accuracy of different estimators on simulated Gaussian datasets. ANM represents additive models, MN represents multiplicative, and LS represents location-scale models. The difference between ANMg and ANMs (LSg, LSs) is how the functions $\mu$ and $\sigma$ are generated. Analogous results (without the first row) can also be found in \cite{Natasa_Tagasovska} and \cite{immer2022identifiability}, with several other estimators from the literature. }
    \label{Table_Simulated_data_Gaussian}
\end{table}



\subsection{Robustness against a misspecification of F}
%add alpha definiton
Consider $X_1\to X_2$, where $X_1\sim N(2,1)^+$, \footnote{$N(2,1)^+$ denotes the truncated Gaussian distribution on $\{x>0\}$. Therefore, $X_1>0$, with a mean of approximately $2.07$.} and let
\begin{equation}\label{eq9870p}
    X_2\mid X_1\sim Exp\big(\alpha(X_1)\big),
\end{equation}
where $\alpha$ is a non-negative function. In other words, we generate $X_2$ according to (\ref{BCPCM}), with $F$ being an exponential distribution function. Recall that the exponential distribution is a special case of Gamma distribution with a fixed shape parameter. 

The goal of this simulation is to ascertain how the choice of $F$ affects the resulting estimate of the causal graph. We consider five different choices for $F$: Gamma with fixed scale, Gamma (with two parameters as in Consequence \ref{consequenceprva}), Pareto, Gaussian with fixed variance, and Gaussian (with two parameters as in Example \ref{Gaussian case}). 


We generate $n=500$ variables, according to (\ref{eq9870p}), with different functions, $\alpha$. Then, we apply the CPCM algorithm with different choices of $F$. Table \ref{table_simulations_about_misspecified_F} presents the percentage of correctly estimated causal graphs (an average out of 100 repetitions). 
The results remain more or less good for $F$ that are ``similar'' to the exponential distribution, with respect to the density and support. However, if we select the Gaussian distribution (a uni-modal distribution with different support), our methodology often provides wrong estimates. 

\begin{table}[tbh]
\centering
\begin{tabular}{|c|c|c|c|c|c|}
\hline
  F       & $\alpha(x) = x$ & $\alpha(x) = x^2+1$  & $\alpha(x) = \frac{e^x}{2}$ & Random $\alpha$ \\ \hline
Gamma (fixed scale)   & $93$           & $95$                                & $97$                       & $92$      \\ \hline 
Gamma (two parameters)  & $82$           & $86$                               & $76$                       & $92$           \\ \hline
Pareto    & $99$           & $100$             & $99$                                   & $97$           \\ \hline
Gaussian (fixed variance) & $0$           & $0$                               & $0$                       & $9$           \\ \hline
Gaussian (two parameters) & $16$           & $25$                           & $33$                       & $41$           \\ \hline
\end{tabular}
\caption{Comparison of the accuracy of CPCM estimations for different choices of $F$. Random $\alpha$ represents a function generated using Gaussian processes, which is similar to Simulations \ref{Section_simulations_Gaussian}.}
\label{table_simulations_about_misspecified_F}
\end{table}












\section{Application}
\label{sec:apx:application}

In this section, we will first show how he $L_2$ norm bounds of the previous sections can be 
extended to $L_\infty$ (supremum) norm bounds when the target function $f\in\Cs(\Kc)$ is 
differentiable at the origin and Lipschitz on a convex compact set $\Kc\subset\R^n$ including 
the origin. Then we will apply the theory to an approximate extension of the Model Reference 
Adaptive Control (MRAC) setup in \cite{lavretsky2013robust}.

\subsection{Extending the Error Bound to $L_\infty$ for Lipschitz Functions on Convex Compact
Sets}
\label{sec:apx:extLinf}

We assume that $\Kc$ is full dimension, with nonzero Lebesgue measure (volume) 
$\Vc=\mu(\Kc)>0$ and diameter $\Dc>0$. It also must hold that $\Kc\subseteq B_0(\rho)$ for a 
sufficiently large radius $0<\rho\leq\Dc$.

Then, for any approximation $\fhN$ of the form \eqref{eq:apx:baseApx} using an activation
function $\sigma$ satisfying Assumption~\ref{sigAsmpt}, we have
%=\sum_{i=1}^N\theta_i\,\sigma(\pariX)$,
that
\begin{align*}
&\abs{\sum_{i=1}^N \theta_i\,\sigma(\pariX) - \sum_{i=1}^N \theta_i\,\sigma(\pariY)} \\ 
&\leq \sum_{i=1}^N \abs{\theta_i\,\sigma(\pariX) - \theta_i\,\sigma(\pariY)} 
\\
&= \sum_{i=1}^N |\theta_i|\!\abs{\sigma(\pariX) - \sigma(\pariY)}
\leq
\sum_{i=1}^N |\theta_i|\Lsig\abs{\pariX - \pariY}\\
& =  
\sum_{i=1}^N |\theta_i|\Lsig\abs{w_i^\top\!(x-y)} 
\ \leq \  
\sum_{i=1}^N |\theta_i|\!\norm{w_i}_2\,\norm{x-y}_2
\end{align*}
holds for all $x,y\in\R^n$, recalling that $X:=[x_1\ \cdots\ x_n\ 1]^\top$ and $Y:=[y_1\ \cdots\ 
y_n\ 1]^\top$, since $\abs{\sigma(u)-\sigma(v)}\leq\Lsig\abs{u-v}$ for any $u,v\in\R$ by the 
Lipschitz condition of Assumption~\ref{sigAsmpt}. Thus, such approximations are 
$\widehat\Lc$-Lipschitz with constant
\begin{equation}\label{eq:apx:Lhat}
\widehat\Lc \ = \ \Lsig\sum_{i=1}^N |\theta_i|\!\norm{w_i}_2 \ \ .
\end{equation}

A similar calculation gives that such approximations are $\widehat\Ds$-bounded with constant
\begin{equation}\label{eq:apx:Dhat}
\widehat\Ds \ = \ \Dsig\sum_{i=1}^N |\theta_i|
\end{equation}
if $\abs{\sigma(u)-\sigma(v)}\leq\Dsig$ for any $u,v\in\R$ by the bounded condition of 
Assumption~\ref{sigAsmpt}.

Let an affine shift to function $f$ be defined as
\begin{equation}\label{eq:apx:fshift}
\fch(x) \ := \ f(x) - \as^\top x - \bs
\end{equation}
for any $\as\in\R^n$ and $\bs\in\R$.
Thus, if $f$ is $\Lc$-Lipschitz, then
\begin{align*}
&\abs{\fch(x)-\fch(y)} \ = \ 
%\abs{f(x) - \as^\top x - f(0_n) - f(y) + \as^\top y + f(0_n)} \\
\abs{f(x)-f(y) - \as^\top(x-y)} \\
&\leq \ 
\abs{f(x)-f(y)} + \abs{\as^\top(x-y)} \\
&\leq\ \Lc\norm{x-y}_2 + \norm{\as}_2\norm{x-y}_2
\end{align*}
holds for all $x,y\in\R^n$. And so, $\fch$ is also Lipschitz, with constant
$\widecheck\Lc = \Lc + \norm{\as}_2$.


\begin{lemma}\label{lem:apx:extLinf}
\textit{
Let $f:\R^n\to\R$ be $\Lc$-Lipschitz on a convex compact set $\Kc\subset\R^n$ containing the 
origin and differentiable at the origin. Let there also be an approximation $\fh:\R^n\to\R$
which is $\widehat\Ds$-bounded or $\widehat{\Lc}$-Lipschitz on $\Kc$ and such that the 
approximation error $\fw(x)=\fch(x)-\fh(x)$, with $\fch$ as in \eqref{eq:apx:fshift}, satisfies 
the $L_2(\Kc,\mubKc)$ norm bound}
$$
\normK{\fw} = \sqrt{\int_\Kc\abs{\fw(x)}^2\mubKcdx} \ \leq \ \epsbase
$$\noindent
\textit{
where the bound $\epsbase>0$ is a finite constant and $\mubKcdx$ is the uniform probability
measure on $\Kc$ with $\int_\Kc\mubKcdx=1$.}

\textit{
Assume $\Kc$ is full dimension with diameter $\Dc>0$, and it holds that $\Kc\subseteq 
B_0(\rho)$ for a ball of sufficiently large radius $0<\rho\leq\Dc$.}
\textit{
Then, the approximation error also satisfies the $L_\infty(\Kc)$ norm bound}
\begin{align}\label{eq:apx:lemDLinfBound}
&\sup_{x\in\Kc}\abs{\fw(x)} \ \leq \ 
2\,
\left(\frac{r}{\Dc(r+\sqrt{r^2+\Dc^2})}\right)^{\!\frac{-n}{n+2}}\,K(\epsbase)
\end{align}
\textit{where}
\begin{align*}
K(\epsbase) \ = \\
i)& \hspace{30pt} \left(\left(\Lc + \norm{\as}_2\right)^n
\epsbase^{\,2}\right)^{\frac{1}{n+2}}+\widehat{\Ds}
\\
ii)& \hspace{30pt}
\left(\left(\Lc + \norm{\as}_2 + \widehat{\Lc}\right)^n
\epsbase^{\,2}\right)^{\frac{1}{n+2}}
\end{align*}
\textit{
and $r$ is the largest radius ball within $\Kc$ centered at its centroid.}
\end{lemma}
\begin{proof}
\if\ARXIV1
Given in Appendix~\ref{app:apx:lemLinfProof}.
\fi
\if\ARXIV0
Given in Appendix III of \cite{lekang2023functionfull}.
\fi
\end{proof}

\subsection{Application to MRAC}
\label{sec:apx:LMRAC}

In Chapter 12 of \cite{lavretsky2013robust}, an approximate extension to the general linear MRAC 
setup (see Chapter 9) is introduced which allows for nonlinearities $f:\R^n\to\R^\ell$ as
$
f(x) \ = \ \Theta^\top\!\Psi(x) + \epsilon_f(x)
$
such that the plant is given by
\begin{align}\nonumber
\dot{x}_t \ &= \ A\,x_t + B\big(u_t + f(x)) \\\label{eq:apx:plant}
&= \ A\,x_t + B\big(u_t + \Theta^\top\!\Psi(x_t) + \epsilon_f(x)\big) \ \ ,
\end{align}
where $A$ is a known $n\times n$ state matrix for the plant state $x_t\in\R^n$, $B$ is a known
$n\times\ell$ input matrix for the input $u_t\in\R^\ell$, and $\Theta$ is an \underline{unknown}
$N\times\ell$ matrix which linearly parameterizes the known vector function $\Psi:\R^n\to\R^N$.
We assume $(A,B)$ is controllable. The general setup also includes an unknown diagonal scaling 
matrix $\Lambda$, such that the overall input matrix is $B\Lambda$, and assumes that $A$ is 
unknown. For simplicity, we assume $A$ is known and omit $\Lambda$.

It is assumed that there exists an $n\times\ell$ matrix of feedback gains $K_x$ and an
$\ell\times\ell$ matrix of feedforward gains $K_r$ satisfying the \textit{matching conditions}
\begin{align*}\nonumber
A + BK_x^\top = A_r \\
BK_r^\top = B_r
\end{align*}
to a controllable, linear reference model
\begin{equation*}
\dot{x}_t^r = A_r\,x_t^r + B_r\,r_t \ \ ,
\end{equation*}
where $A_r$ is a known Hurwitz $n\times n$ reference state matrix for the reference state
$x^r_t\in\R^n$, $B_r$ is a known $n\times\ell$ reference input matrix, and $r_t\in\R^\ell$ is a
bounded reference input. Here, we assume that $K_x$ and $K_r$ can be directly calculated 
from known $A$ and $B$, and used directly in the control law.

It is required that the nonlinearity satisfy the bound
\begin{equation}\label{eq:apx:epsf}
\norm{\epsilon_f(x)}_2 \ \leq \ \bar\varepsilon
\end{equation}
for some constant $\bar\varepsilon>0$ and for all $x\in B_0(r)$ with some radius $r>0$. Then,
the adaptive control law
$$
u_t \ = \ K_x^\top x_t - \Thetah_t^\top\!\Psi(x_t) + \left(1-\mu(x)\right)K_r^\top r_t + 
\mu(x)\uw(x_t)
$$
is shown to stabilize the state tracking error $e_t = x_t-x_t^r$ down to a compact set about the 
origin, where the $N\times\ell$ matrix of parameter estimates $\Thetah_t$ is dynamically updated 
with the update rule
\begin{equation*}
\dot{\Thetah}_t = \Gamma\,\Psi(x_t)\,e_t^\top P_xB \ \ .
\end{equation*}
%(see Chapter 9 of \cite{lavretsky2013robust} for further details).
The scalar function $\mu:\R^n\to[0,1]$
transitions the control law from tracking the reference input to simply returning the plant 
state $x_t$ to the bounded region $B_0(r)$ within which \eqref{eq:apx:epsf} is valid, using an 
appropriately defined $\uw(x_t)$ based on assumptions about the growth of $\epsilon_f(x)$ 
outside of $B_0(r)$. (See Chapter~12 of \cite{lavretsky2013robust} for details.)

And so, if the vector function $\Psi(x)$ is constructed as
\begin{equation}\label{eq:pe:Psi}
\Psi(x)
= 
\begin{bmatrix}
\sigma(\upgamma_1^\top X) \\
\vdots \\
\sigma(\upgamma_N^\top X)
\end{bmatrix} \ ,
\end{equation}
with an activation function $\sigma$ satisfying the bounded (growth) conditions of 
Assumption~\ref{sigAsmpt} with constant $\Dsig$ or $\Lsig$, then the above setup is valid for
any nonlinearity $f = [f_1\ \cdots\ f_\ell]$ where we 
can show that
$$
\sup_{x\in\Kc}\abs{f_i(x)-\Theta_i^\top\Psi(x)} \ \leq \ \bar\varepsilon_i
$$
holds for some constants $\bar\varepsilon_1,\dots,\bar\varepsilon_\ell>0$. Here, 
$f_1,\dots,f_\ell:\R^n\to\R$ and each $\Theta_i^\top\in\R^N$ is the corresponding row of the 
true parameters $\Theta^\top$.

%\section{}
%\label{sec:resDir}


\section{Conclusion}
\label{sec:conclusion}
% <>
Since its advent in 1931, Koopman operator theory \cite{koopman:1931} has only recently been actively utilized for solving practical problems, thanks to the introduction of the DMD algorithm in 2008 \cite{schmid:2008}. Since then, a multitude of DMD algorithm variations have risen to prominence and found utility across various fields. A notable feature of our survey paper was reviewing and categorizing the results of over 100 research papers based on both application and algorithm type in smart mobility and vehicle engineering  (see Table~\ref{tab1} and Section~\ref{sec:vehicApp}).  Additionally, this survey paper identified potential research gaps in smart mobility and vehicular engineering applications (Remarks~\ref{remGap1}--\ref{remGap6}). Finally, this review paper discussed theoretical aspects of Koopman operator theory that have been largely neglected by the smart mobility and vehicle engineering community and yet have large potential for contributing to solving open problems in these areas (see Section~\ref{subsec:theorIssue}).

\noindent{\textbf{Future Research Directions.}}	Given the emergence of cyber-threats against connected and autonomous vehicles as well as robotic systems (see, e.g.,~\cite{nekouei2021randomized,mohammadi2022generation}), a future research direction might include utilizing Koopman operator-based algorithms for designing cyber-resilient vehicular and smart mobility applications (see, e.g.,~\cite{taheri2022data} for a related line of research). Another potential research direction is using Koopman operator-based algorithms for predicting the motion of vulnerable road users (VRUs), e.g., pedestrians and cyclists (see, e.g.,~\cite{pool2019context,scholler2020constant}). Finally, rehabilitation robotics and robotic exoskeletons can be the benefactors of the predictive capabilities of Koopman operator-based algorithms for detecting tripping events and/or system  identification in various modes of locomotion (see, e.g.,~\cite{kumar2019extremum,aprigliano2019pre}).



%Fig. 1 depicts the accumulation of such algorithms since 2014, which are particular to vehicle engineering and smart mobility, i.e., the focus of this review. Table 1 summarizes the varieties of relevant algorithms developed in those studies. Furthermore, we have highlighted theoretical issues, whose expansion will have potential applications to the wide research area of smart mobility and vehicle engineering.  

%Although fairly comprehensive, we have found several gaps in this research area. In particular, we could not find any studies related to elevators, robots/vehicles employing crawling, slithering, hopping or peristaltic locomotion, arctic or special-terrain vehicles such as those employing screws or tracks, hovercraft and other amphibious vehicles or subsystems which tolerate flexible environments, classification or guidance systems related to vehicles for drilling or agriculture, or for current-ripple, power-split, battery health monitoring, nuclear propulsion, exoskeletons/prosthetics, personal mobility, motorsports, specialized rovers or similar open problems in emerging areas.  These examples are, of course, not exhaustive.  
%
%The purely data-driven nature of Koopman operators holds the promise of capturing unknown and complex dynamics for reduced-order model generation and system identification, through which the rich machinery of linear control techniques can be utilized. The emergent nature of the smart mobility and vehicular-related applications, where  the Koopman operator  in each particular application needs to be approximated, implies that the development of various Koopman operator approximation  algorithms is expected to grow along with the vehicular problems they aim to solve.  Given the ongoing development of this research area and the many existing open problems in the fields of smart mobility and vehicle engineering, a survey of techniques and open challenges of applying Koopman operator theory to this vibrant area is warranted.  To the best of our knowledge, this survey paper is the \emph{first of its kind} reviewing the applications of Koopman operator theory within a focused research area, namely, smart mobility and vehicle engineering applications. A \emph{notable feature} of our survey paper is reviewing and categorizing the results of over 100 research papers based on both application and algorithm type  (see Tables~\ref{tab1}--~\ref{tab4} and Section~\ref{sec:vehicApp}) that are concerned with the applications of Koopman operator theory to the field of smart mobility and vehicular engineering. Such a \emph{comprehensive and  detailed categorization} will be beneficial to the research practitioners working in the field.  Furthermore, this review paper discusses theoretical aspects of Koopman operator theory that have been largely neglected by the smart mobility and vehicle engineering community and yet have large potential for contributing to solving open problems in these areas. Additionally, our survey paper seeks to \emph{identify gaps} in the smart mobility and vehicle engineering research where new and existing Koopman operator-based methods have the potential to further develop and address unsolved problems  potentially benefiting from the perspectives of nonlinear system identification, control, global linearization, and the predictive powers that Koopman operator theory has to offer (see, e.g., Remarks~\ref{remGap1}--\ref{remGap6}). 

\appendix
\section{Exponential family}
\label{appendix_exponential_family}

The exponential family is a set of probability distributions whose probability density function can be expressed in the following form:
\begin{equation}\label{Exponential family of distributions}
f(x;\theta) = h_1(x)h_2(\theta)\exp\big[\sum_{i=1}^q\theta_iT_i(x)\big],
\end{equation}
where $h_1, T_i$ are real functions and $h_2:\mathbb{R}^q\to\mathbb{R}^+$ is a vector-valued function. We call $T_i$ a \textit{sufficient} statistic, $h_1$ a base measure, and $h_2$ a normalizing (or partition) function.  

Often, form (\ref{Exponential family of distributions}) is called a canonical form and $$f(x;\theta) = h_1(x)h_2(\theta)\exp\big[\sum_{i=1}^qh_{3,i}(\theta)T_i(x)\big],$$ where  $h_{3,i}:\mathbb{R}^q\to\mathbb{R}, i=1, \dots, q$, is called its \textit{reparametrization} (natural parameters are a specific form of the reparametrization). We always work only with a canonical form (attention for Gaussian distribution, where the standard form is not in the canonical form). 

Numerous important distributions lie in the exponential family of distributions, such as Gaussian, Pareto (with fixed support), log-normal, Poisson, Binomial, Gamma, and Beta distributions, to name a few. 

It is important to note that functions in (\ref{Exponential family of distributions}) are \textit{not} uniquely defined. For example, $T_i$ is unique up to a linear transformation. 

The support of $f$ is fixed and does not depend on $\theta$. Potentially, $T_i$ and $h_1$ do not have to be defined outside of this support; however, we typically overlook this fact (or possibly define $h_1(x) = T_i(x) = 0$ for $x$ where these functions are not defined). We additionally assume that the support is nontrivial in the sense that it contains at least two distinct values.

Without loss of generality, we always assume that $q$ is minimal in the sense that we cannot write $f(x;\theta)$ using only $q-1$ parameters. Then, $T_1, \dots, T_q$ are linearly independent in the following sense: there exists $x_1, \dots, x_q\in supp(f)$, such that  matrix 
\begin{equation}\label{eq2431087}
\begin{pmatrix}
T_{1}(x_1) & \cdots & T_{{q}}(x_1) \\
\cdots & \cdots & \cdots \\
T_{1}(x_q) & \cdots & T_{{q}}(x_{q}) 
\end{pmatrix} 
\end{equation}
has full rank. Moreover, $T_1, \dots, T_q$ are affinly independent in the following sense: there exists $y_0, y_1, \dots, y_q\in supp(f)$, such that a matrix 
\begin{equation}\label{eq145151}
\begin{pmatrix}
T_{1}(y_1) - T_1(y_0) & \cdots & T_{{q}}(y_1) -T_q(y_0)\\
\cdots & \cdots & \cdots \\
T_{1}(y_q) -T_1(y_0) & \cdots & T_{{q}}(y_{q}) -T_q(y_0)
\end{pmatrix} 
\end{equation}
has full rank. In this paper (particularly in Lemma \ref{PomocnaLemma1}), we assume that  $T_1, \dots, T_q$ are affinly independent (i.e., satisfy (\ref{eq145151})). 

Since the notions of  linear and affine independence are nonstandard, consider the following example. Say  $T_1(x) = x, T_2(x) = x^2$ (sufficient statistics in Gaussian distribution). Then, matrices (\ref{eq2431087}) and (\ref{eq145151}) are
\begin{equation*}
M_1=\begin{pmatrix}
x_1 &  x_1^2\\
x_2  & x_{2}^2
\end{pmatrix} , \,\,\,\,
M_2=\begin{pmatrix}
y_1-y_0 &  y_1^2 -y_0^2\\
y_2 -y_0 &  y_2^2 -y_0^2
\end{pmatrix} ,
\end{equation*}
which are full ranked for a choice $(x_1, x_2) = (1,2)$ and $(y_0, y_1, y_2) = (0,1,2)$, for example. 





\subsection{Proof of Proposition~\ref{prop:monotone-prob}}\label{app:proof-monotone-prob}

In order to prove the proposition, we first note the well-known fact [\cite{whitt1979note,milgrom1981good,shaked2007stochastic}] that monotone likelihood ratio (MLR) implies first-order stochastic dominance (FOSD) of the distribution of the signal conditioned on a higher parameter (as well as for the posterior distribution of the parameter conditioned on a higher signal).

\begin{lemma} \label{lem:FOSD}
Assume that the family of review signal distributions has MLR.
Then, whenever $q' > q$, the signal distribution for $q'$ first-order stochastically dominates the distribution for $q$; that is, the distributions satisfy that $\Prob[\RevSig{} \sim {\RevSigDist[q']}]{\REVSIG \geq x} > \Prob[\REVSIG \sim {\RevSigDist[q]}]{\REVSIG \geq x}$ for all $x \in (\inf \SigSet, \sup \SigSet)$. 
\end{lemma}

We mentioned above that a higher signal also implies FOSD of the posterior quality distributions. The following lemma captures the stronger property that under the MLR property, this holds even for vectors of signals.

\begin{lemma} \label{lem:monotone_expected_quality}
Suppose $\RevSigV'$ and $\RevSigV$ are two vectors of signals that have MLR and satisfy $\RevSigV'\ge \RevSigV$ component-wise, and the inequality is strict for at least one of the components. 
Then, $U(\RevSigV')> U(\RevSigV)$ holds for any prior $\QualDist$.
\end{lemma}

For continuous distributions, this lemma is proved by \citet{torres2005multivariate}.
We give a self-contained proof for the categorical case, which is largely analogous, in \cref{app:FOSD-proof}.
We are now ready to prove Proposition~\ref{prop:monotone-prob}.

\proof{Proof of Proposition~\ref{prop:monotone-prob}.}
We give the proof in the categorical model; it can be straightforwardly generalized to the continuous model.

Let $q' > q$. 
For each reviewer $i$, couple the draws of $\RevSig{i}$ from $\RevSigDist[q]$ and $\RevSigP{i}$ from $\RevSigDist[q']$ by drawing a (common) uniformly random quantile $x$ in $[0,1]$ and letting $\RevSig{i}, \RevSigP{i}$ be the respective signals at quantile $x$ of the corresponding CDFs. 
Because, by \cref{lem:FOSD}, $\RevSigDist[q']$ (strictly) first-order stochastically dominates $\RevSigDist[q]$, this coupling ensures that $\RevSigP{i} \ge \RevSig{i}$;
furthermore, the inequality is strict with positive probability unless $\RevSig{i}=\max_s\in\SigSet$.
By applying this coupling to each individual review (which, recall, is drawn independently of other reviews), we obtain a coupling of vectors of reviews such that $\RevSigVP \geq \RevSigV$ always holds component-wise, and the inequality is strict for at least one of the components with positive probability. By Lemma~\ref{lem:monotone_expected_quality}, this coupling has the property that $U(\RevSigVP) \geq U(\RevSigV)$ always holds, and, conditional on \RevSigV, the inequality is strict with positive probability unless every component of $\RevSigV$ is the maximum signal (if the maximum exists).
Therefore, if \ACCMAP is a monotone acceptance policy, $\AccP{\ACCMAP}{q}$ can never decrease in $q$.

It remains to show that $\AccP{\ACCMAP}{q}$ is \emph{strictly} increasing in $q$ for non-trivial threshold policies $\ACCMAP[\tau,r]$.
First, we may assume w.l.o.g.~that $0 < r < 1$.
For if $r=0$, the policy is equivalent to the policy $\ACCMAP[\tau',\half]$ with any $\tau' \in (\max \Set{U(\RevSigV)}{U(\RevSigV) < \tau}, \tau)$, 
and if $r=1$, it is equivalent to the policy $\ACCMAP[\tau',\half]$ with any $\tau' \in (\tau, \min \Set{U(\RevSigV)}{U(\RevSigV) > \tau})$. Here, the minimum and maximum will be finite because the policy is assumed to be non-trivial.

By Lemma~\ref{lem:monotone_expected_quality}, there must exist \RevSigV, \RevSigVH with $U(\RevSigVH) \geq \tau \geq U(\RevSigV)$ such that at least one of the two inequalities is strict. Let \RevSigV be a vector of reviews maximizing $U(\RevSigV)$ subject to $U(\RevSigV) \leq \tau$.
Because $U(\RevSigVH) > U(\RevSigV)$, the vector \RevSigV cannot be maximal in all components.
Therefore, by the preceding coupling argument, when $\RevSigV$ is drawn with quality $q$, the corresponding vector $\RevSigVP$ drawn with quality $q'$ satisfies $U(\RevSigVP) > U(\RevSigV)$. 
By definition of \RevSigV, the review vector \RevSigVP gives rise to strictly higher acceptance probability than \RevSigV. For if $U(\RevSigV) < \tau$, then \RevSigV always leads to rejection, whereas (by maximality of $U(\RevSigV)$) \RevSigVP leads to acceptance with probability at least $r > 0$.
And if $U(\RevSigV) = \tau$, then \RevSigV leads to acceptance with probability $r < 1$, whereas \RevSigVP leads to acceptance with probability 1.

% \gs{I am having trouble with this proof.  It seems to require that "$\RevSigV$ is drawn with quality $q$ and the corresponding vector $\RevSigVP$ drawn with quality $q'$" with some positive probability, but I don't see where we show that.}
% \dkcomment{I thought that this is because we couple by quantile. You draw $s$. Each component has a quantile. You draw $s'$ in that component as having the same quantile according to the distribution parametrized with $q'$. I don't see how that leads to problems.}\fangcomment{I agree with David. We can first sample a quantile that determines $s$ then $s'$ that ensures $U(s')\ge U(s)$. 
%  However, it is unclear to me why $U(s')>U(s)$ instead of $U(s')\ge U(s)$ and strict with positive probability (but this should not be an issue).  This may be stated in the second paragraph.}

Therefore, for non-trivial threshold policies, the coupling ensures that a paper of quality $q'$ is accepted at least whenever a paper of quality $q$ is accepted, and with strictly positive probability, only the paper with quality $q'$ is accepted.
This completes the proof. \Halmos
\endproof

\subsection{Proof of Lemma~\ref{lem:monotone_expected_quality}}\label{app:FOSD-proof}

Here, we prove Lemma~\ref{lem:monotone_expected_quality}. 

\proof{Proof of \Cref{lem:monotone_expected_quality}.}

% \dkreplace{The proof follows by induction on the size of quality set $\QualSet$. Note that the proof is based on the categorical model, but can be straightforwardly generalized to the continuous model. For a more detailed proof for the continuous model, one can refer Theorem 1 of \cite{torres2005multivariate}.}{
Here, we provide a proof for the categorical model. It can be easily modified for the continuous model, and the result also is shown as Theorem~1 by \citet{torres2005multivariate}.
We show the result by induction on the size of the quality set $\QualSet$.

\emph{Base case:} We show that the inequality holds for any binary quality set $\QualSet=\SET{q_1, q_2}$ with $q_2>q_1$.  For $\RevSigV \in \SigSet^{\NumReviews}$, let $\gamma(\RevSigV)=\frac{\RevSigProb[q_2]{\RevSigV}}{\RevSigProb[q_1]{\RevSigV}}$ be the likelihood ratio. We first rewrite the ex-post expected quality:
\begin{align*}
    U(\RevSigV) & = \ExpectC{Q}{\RevSigV}\\
    &= \sum_q q\cdot \ProbC{Q=q}{\RevSigV}\\
    &= \sum_q q \cdot \frac{\Prob{q}\cdot \ProbC{\RevSigV}{Q=q}}{\sum_{q'}\Prob{q'} \cdot \ProbC{\RevSigV}{Q=q'}}\\
    &= q_1 \cdot \frac{\QualProb{q_1}\RevSigProb[q_1]{\RevSigV}}{\QualProb{q_1}\RevSigProb[q_1]{\RevSigV}+\QualProb{q_2}\RevSigProb[q_2]{\RevSigV}} + q_2\cdot \frac{\QualProb{q_2}\RevSigProb[q_2]{\RevSigV}}{\QualProb{q_1}\RevSigProb[q_1]{\RevSigV}+\QualProb{q_2}\RevSigProb[q_2]{\RevSigV}}\\
    &=  \frac{q_1 \cdot \QualProb{q_1}\RevSigProb[q_1]{\RevSigV} + q_2\cdot\QualProb{q_2}\RevSigProb[q_2]{\RevSigV}}{\QualProb{q_1}\RevSigProb[q_1]{\RevSigV}+\QualProb{q_2}\RevSigProb[q_2]{\RevSigV}}\\
    &= \frac{q_1\cdot\QualProb{q_1} + q_2\cdot\QualProb{q_2}\gamma(\RevSigV)}{\QualProb{q_1}+\QualProb{q_2}\gamma(\RevSigV)}.
\end{align*}
    
Now, we consider the difference $U(\RevSigVP) - U(\RevSigV)$, write it using a common denominator, and cancel common terms. Then, 
% \dkdeletecomment{we'd also need to specify the one strict inequality, no?}{because $\RevSigVP \ge \RevSigV$ component-wise,} 
by \cref{def:informative}, we obtain that $\gamma(\RevSigVP)>\gamma(\RevSigV)$, which we substitute:
\begin{align*}
    U(\RevSigVP) - U(\RevSigV)
    &=  \frac{\QualProb{q_1}\QualProb{q_2} \cdot \left(q_1\gamma(\RevSigV) + q_2\gamma(\RevSigVP) - q_1\gamma(\RevSigVP) - q_2\gamma(\RevSigV)\right)}{\left(\QualProb{q_1}+\QualProb{q_2}\gamma(\RevSigVP)\right) \cdot \left(\QualProb{q_1}+\QualProb{q_2}\gamma(\RevSigV)\right)}\\
    &= \frac{\QualProb{q_1}\QualProb{q_2} \cdot (\gamma(\RevSigVP) - \gamma(\RevSigV)) \cdot (q_2-q_1)}{\left(\QualProb{q_1}+\QualProb{q_2}\gamma(\RevSigVP)\right) \cdot \left(\QualProb{q_1}+\QualProb{q_2}\gamma(\RevSigV)\right)}\\
    &>0.
\end{align*}

\emph{Induction step: } We show that if the inequality holds for all categorical models with support size $|\QualSet|=n-1$, it also holds for categorical models with support size $|\QualSet|=n$. 
Let $\QualSet=\SET{q_1, q_2, \ldots, q_n}$ with $q_n>\cdots>q_1$. Using the same derivation as in the base case, 
\begin{align*}
    U(\RevSigVP) - U(\RevSigV)
    &= \frac{\left(\sum_{i=1}^n q_i\QualProb{q_i}\RevSigProb[q_i]{\RevSigVP}\right) \cdot \left(\sum_{i=1}^n \QualProb{q_i}\RevSigProb[q_i]{\RevSigV}\right) - \left(\sum_{i=1}^n q_i\QualProb{q_i}\RevSigProb[q_i]{\RevSigV}\right) \cdot \left(\sum_{i=1}^n \QualProb{q_i}\RevSigProb[q_i]{\RevSigVP}\right)}{\left(\sum_{i=1}^n \QualProb{q_i}\RevSigProb[q_i]{\RevSigVP}\right) \cdot \left(\sum_{i=1}^n \QualProb{q_i}\RevSigProb[q_i]{\RevSigV}\right)}.
\end{align*}

% \dkreplace{
% Let $P(n)$ be the numerator of the above equation.
% Assuming $P(k)$ is positive, we want to show that $P(k+1)$ is also positive.

% \begin{align*}
%     P(k+1) =& \left(\sum_{i=1}^{k+1} q_i\QualProb{q_i}\RevSigProb[q_i]{\RevSigVP}\right)\left(\sum_{i=1}^{k+1} \QualProb{q_i}\RevSigProb[q_i]{\RevSigV}\right) - \left(\sum_{i=1}^{k+1} q_i\QualProb{q_i}\RevSigProb[q_i]{\RevSigV}\right)\left(\sum_{i=1}^{k+1} \QualProb{q_i}\RevSigProb[q_i]{\RevSigVP}\right)\\
%     =& P(k) + q_{k+1}\QualProb{q_{k+1}}\RevSigProb[q_{k+1}]{\RevSigVP}\left(\sum_{i=1}^{k} \QualProb{q_i}\RevSigProb[q_i]{\RevSigV}\right) +\QualProb{q_{k+1}}\RevSigProb[q_{k+1}]{\RevSigV}\left(\sum_{i=1}^{k} q_i\QualProb{q_i}\RevSigProb[q_i]{\RevSigVP}\right)\\
%     &- q_{k+1}\QualProb{q_{k+1}}\RevSigProb[q_{k+1}]{\RevSigV}\left(\sum_{i=1}^{k} \QualProb{q_i}\RevSigProb[q_i]{\RevSigVP}\right) - \QualProb{q_{k+1}}\RevSigProb[q_{k+1}]{\RevSigVP}\left(\sum_{i=1}^{k} q_i\QualProb{q_i}\RevSigProb[q_i]{\RevSigV}\right)\\
%     % =& P(k) + \sum_{i=1}^{k}\QualProb{q_i}\QualProb{q_{k+1}}\left(q_{k+1}\left(\RevSigProb[q_{k+1}]{\RevSigVP}\RevSigProb[q_i]{\RevSigV} - \RevSigProb[q_{k+1}]{\RevSigV}\RevSigProb[q_i]{\RevSigVP}\right) + q_i\left(\RevSigProb[q_{k+1}]{\RevSigV}\RevSigProb[q_i]{\RevSigVP} - \RevSigProb[q_{k+1}]{\RevSigVP}\RevSigProb[q_i]{\RevSigV}\right)\right)\\
%     =& P(k) + \sum_{i=1}^{k}\QualProb{q_i}\QualProb{q_{k+1}}\left(\RevSigProb[q_{k+1}]{\RevSigVP}\RevSigProb[q_i]{\RevSigV} - \RevSigProb[q_{k+1}]{\RevSigV}\RevSigProb[q_i]{\RevSigVP}\right)\left(q_{k+1} - q_i\right)\\
%     >& \sum_{i=1}^{k}\QualProb{q_i}\QualProb{q_{k+1}}\left(\RevSigProb[q_{k+1}]{\RevSigVP}\RevSigProb[q_i]{\RevSigV} - \RevSigProb[q_{k+1}]{\RevSigV}\RevSigProb[q_i]{\RevSigVP}\right)\left(q_{k+1} - q_i\right).
% \end{align*}
% }{
Because the denominator is strictly positive, we now focus on showing that the numerator is as well.
We show that the difference between the actual numerator, and the version where the upper bound of each summation is $n-1$, is positive. Then, because the latter is positive by induction hypothesis, we will have shown the claim.
\begin{align*}
& \left( \left(\sum_{i=1}^{n} q_i\QualProb{q_i}\RevSigProb[q_i]{\RevSigVP}\right) \cdot \left(\sum_{i=1}^{n} \QualProb{q_i}\RevSigProb[q_i]{\RevSigV}\right) - \left(\sum_{i=1}^{n} q_i\QualProb{q_i}\RevSigProb[q_i]{\RevSigV}\right)\left(\sum_{i=1}^{n} \QualProb{q_i}\RevSigProb[q_i]{\RevSigVP}\right) \right)
\\ - & 
 \left( \left(\sum_{i=1}^{n-1} q_i\QualProb{q_i}\RevSigProb[q_i]{\RevSigVP}\right) \cdot \left(\sum_{i=1}^{n-1} \QualProb{q_i}\RevSigProb[q_i]{\RevSigV}\right) - \left(\sum_{i=1}^{n-1} q_i\QualProb{q_i}\RevSigProb[q_i]{\RevSigV}\right)\left(\sum_{i=1}^{n-1} \QualProb{q_i}\RevSigProb[q_i]{\RevSigVP}\right) \right)
\\ = & 
 q_n \QualProb{q_n} \RevSigProb[q_{n}]{\RevSigVP} \cdot \left( \sum_{i=1}^{n-1} \QualProb{q_i}\RevSigProb[q_i]{\RevSigV} \right) 
 + \QualProb{q_n} \RevSigProb[q_n]{\RevSigV} \cdot \left( \sum_{i=1}^{n-1} q_i \QualProb{q_i}\RevSigProb[q_i]{\RevSigVP} \right)
\\ - & q_n \QualProb{q_n} \RevSigProb[q_n]{\RevSigV} \cdot \left( \sum_{i=1}^{n-1} \QualProb{q_i} \RevSigProb[q_i]{\RevSigVP} \right) 
- \QualProb{q_n} \RevSigProb[q_n]{\RevSigVP} \cdot \left(\sum_{i=1}^{n-1} q_i \QualProb{q_i}\RevSigProb[q_i]{\RevSigV} \right)
\\ = & 
\sum_{i=1}^{n-1} \QualProb{q_i} \QualProb{q_n} \cdot \left( \RevSigProb[q_n]{\RevSigVP}\RevSigProb[q_i]{\RevSigV} - \RevSigProb[q_n]{\RevSigV} \RevSigProb[q_i]{\RevSigVP} \right) \cdot \left( q_n - q_i \right).
\end{align*}


By \cref{def:informative}, 
% \dkreplace{for any $i<k+1$ and $\RevSigVP\ge \RevSigV$ component-wise with strict inequality for at least one of the components, we know that 

% \begin{align*}
%     &\frac{\RevSigProb[q_{k+1}]{\RevSigVP}}{\RevSigProb[q_i]{\RevSigVP}}>\frac{\RevSigProb[q_{k+1}]{\RevSigV}}{\RevSigProb[q_i]{\RevSigV}}\\
%     \Leftrightarrow\qquad & \RevSigProb[q_{k+1}]{\RevSigVP}\RevSigProb[q_{i}]{\RevSigV} > \RevSigProb[q_{k+1}]{\RevSigV}\RevSigProb[q_i]{\RevSigVP}.
%     \end{align*}

% This implies that $P(k+1)>0$, establishing the induction step.
% }{
$\RevSigProb[q_n]{\RevSigVP} \RevSigProb[q_i]{\RevSigV} > \RevSigProb[q_n]{\RevSigV}\RevSigProb[q_i]{\RevSigVP}$, proving that the difference is strictly positive. Thus, by induction hypothesis, the utility difference is strictly positive, completing the proof. \Halmos
\endproof

\subsection{Proof of \cref{lem:author_response}}\label{app:proof-author_resp}

Let $\ACCMAP$ be the acceptance policy. 
The probability of acceptance for a paper of quality $\Qual=q$ is $\AccP{\ACCMAP}{q}$. By Proposition~\ref{prop:monotone-prob}, this probability is non-decreasing in $q$ for monotone acceptance policies, and strictly increasing for non-trivial threshold policies.
The author submits the paper if the expected utility of submitting is greater than 1 (the utility of the outside option), does not submit the paper if the expected utility is less than 1, and is indifferent between submitting or not if the expected utility is equal to 1. 
Because a noiseless author does not learn any new information from rejection in previous rounds, she will make the same decision in future rounds, implying that she will submit until acceptance.
% Since the author will face the same tradeoff in future rounds\footnote{Crucially, a noiseless author does not learn any new information from rejection in previous rounds.}, she will make the same decision, so she will submit until acceptance. 
Let $\ConfValue$ be fixed. The expected utility can be obtained as the time-discounted sum of the utility from acceptance:
\begin{align}
\AUTHUTIL(\ACCMAP,q)
& = 
\sum_{t\ge 1} \ConfValue \cdot \TD^{t-1} \AccP{\ACCMAP}{q} \cdot (1-\AccP{\ACCMAP}{q})^{t-1} 
\; = \; \frac{\ConfValue \cdot \AccP{\ACCMAP}{q}}{1-\TD \cdot (1-\AccP{\ACCMAP}{q})}.
\label{eqn:submission-inequality}
\end{align}
Solving the inequalities $\AUTHUTIL(\ACCMAP,q) > 1$, $\AUTHUTIL(\ACCMAP,q) < 1$, and $\AUTHUTIL(\ACCMAP,q) = 1$ for $q$, the author submits the paper if $\AccP{\ACCMAP}{q} > 1/\rho$, does not submit if $\AccP{\ACCMAP}{q} < 1/\rho$, and is indifferent between submitting or not if $\AccP{\ACCMAP}{q} = 1/\rho$, respectively. 
This completes the proof of the lemma.

\subsection{Proof of \cref{prop:de_facto}}\label{app:proof-de_facto}
By \cref{lem:author_response}, an author with a paper of quality $Q=q$ decides whether to submit based on whether the acceptance probability $\AccP{\ACCMAP}{q}$ is greater than the inverse of the attractiveness factor, $1/\rho$.
Recall that $\AccP{\ACCMAP}{q}$ is weakly increasing in $q$ by \cref{prop:monotone-prob}, while $1/\rho(q,r)$ is strictly decreasing in $q$.
Since the conference's policy is responsive, there must exist some $\bar{q}$ such that $\AccP{\ACCMAP}{\bar{q}} > 1/\rho(\bar{q},1)$.
If $\AccP{\ACCMAP}{q} \ge 1/\rho(q,1)$ for all $q\in \QualSet$, then every author prefers to submit and $\theta = -\infty$ is a de facto threshold.
Therefore, we only have to consider the case where there exists some $\ubar{q}$ such that $\AccP{\ACCMAP}{\ubar{q}} < 1/\rho(\ubar{q},1)$.
% Moreover, under our model's assumption 
% \dkcomment{I did not see that assumption anywhere in either the model or the statement of the lemma. Where is it?}\yzcomment{It was commented out for some reason. I put it back in section 2.2} 
% that the conference's value is less than 1 if all papers are accepted, there exists some $\ubar{q}$ such that $\AccP{\ACCMAP}{\ubar{q}} < 1 < 1/\rho(\ubar{q},1)$.

Combining these observations, we conclude that either there exists some $\theta'\in \QualSet$ such that $\AccP{\ACCMAP}{\theta'} = 1/\rho(\theta',1)$, or there exist two qualities $\ubar{\theta}, \bar{\theta}\in \QualSet$, with no intermediate qualities between them, such that $\AccP{\ACCMAP}{\ubar{\theta}} < 1/\rho(\ubar{\theta},1)$ and $\AccP{\ACCMAP}{\bar{\theta}} > 1/\rho(\bar{\theta},1)$.
We show that a threshold best response exists for either case.

First, if there exists a $\theta'\in \QualSet$ such that $\AccP{\ACCMAP}{\theta'} = 1/\rho(\theta',1)$, then it is a threshold best response:  every author submits and keeps resubmitting if her paper has quality $Q = q\ge \theta'$ and takes the outside option if $q < \theta'$.
To see this, note that authors with a paper of quality $q\ge \theta'$ all have an acceptance probability $\AccP{\ACCMAP}{q} \ge 1/\rho(\theta', 1)$, meaning that they are weakly happier to submit; on the other hand, the acceptance probability $\AccP{\ACCMAP}{q} \le 1/\rho(\theta',1)$ for authors with $q< \theta'$, meaning that they are weakly happier to take the outside option.

If there exist two adjacent qualities $\ubar{\theta}, \bar{\theta}\in \QualSet$ such that $\AccP{\ACCMAP}{\ubar{\theta}} < 1/\rho(\ubar{\theta},1)$ and $\AccP{\ACCMAP}{\bar{\theta}} > 1/\rho(\bar{\theta},1)$, we further distinguish two cases.
In the first case,  $\AccP{\ACCMAP}{\ubar{\theta}}\le 1/\rho(\bar{\theta},1)$. This means that $\AccP{\ACCMAP}{\ubar{\theta}}\le 1/\rho(\ubar{\theta},r)$ for any $r\in [0,1]$, because $\ubar{\theta}<\bar{\theta}$.\fangcomment{Minor point: we use $r$ for three different meaning 1) superscript for reviewer $F^{(r)}$, 2) threshold acceptance policy $\phi_{\tau, r}$, and 3) author's threshold strategy.} \dkcomment{Good point! At least the review noise should perhaps be $R$? Unfortunately, we haven't been consistent in using macros, so this will now be hard to fix without missing one. But we should probably do it after we submit, before the next round of reviews/acceptance.}
In other words, if authors with $Q \ge \bar{\theta}>\ubar{\theta}$ all decide to submit and authors with $Q < \ubar{\theta}$ all decide to take the outside option, then authors with $Q = \ubar{\theta}$ all (weakly) prefer to take the outside option no matter with what probability the other authors with $Q= \ubar{\theta}$ submit.
Therefore, the following is an equilibrium for authors: every author with $Q \ge \bar{\theta}$ submits, and every author with $Q < \bar{\theta}$ takes the outside option.

In the second case, $\AccP{\ACCMAP}{\ubar{\theta}}> 1/\rho(\bar{\theta},1)$. 
This corresponds to the following case: if authors with paper quality $Q \ge \bar{\theta}$ submit and those with $Q < \ubar{\theta}$ opt for the outside option, then authors with paper quality $Q = \ubar{\theta}$ prefer submitting if no one at that quality submits but prefer the outside option when everyone at that quality submits.
% Therefore, there exists some probability $r$ such that it is a best response for authors with $Q = \ubar{\theta}$ to submit with probability $r$, while authors with $Q > \ubar{\theta}$ submit with probability 1, and those with $Q < \ubar{\theta}$ choose the outside option.

At equilibrium, the probability $r$ with which such authors submit must make authors with paper quality $Q = \ubar{\theta}$ indifferent between submitting or not submitting. Therefore, $r$ must be is the solution to $\AccP{\ACCMAP}{\ubar{\theta}} = 1/\rho(\ubar{\theta},r)$, i.e.,
\begin{equation*} 
        \frac{r\cdot \ubar{\theta} \cdot \QualProb{\ubar{\theta}} + \sum_{q \in \QualSet, q \geq \bar{\theta}} q \cdot \QualProb{q}}{%
        r\cdot\QualProb{\ubar{\theta}} + \sum_{q \in \QualSet, q \geq \bar{\theta}}\QualProb{q}} 
        = \frac{1-\TD \cdot (1-\AccP{\ACCMAP}{\ubar{\theta}})}{\AccP{\ACCMAP}{\ubar{\theta}}}.
\end{equation*}
Note that the solution must exist because by the relationship between $P_{\text{acc}}$ and $1/\rho$ in this case, when $r = 0$, the left-hand side of the above equation is larger than the right-hand side, while if $r = 1$, the left-hand side is smaller than the right-hand side.
By the continuity of the left-hand-side as a function of $r$, a solution must exist.
This completes the proof of the first part.

As noted above, by \cref{prop:monotone-prob}, under a monotone acceptance policy, the acceptance probability is weakly increasing in $q$.
As a result, multiple adjacent quality levels may share the same acceptance probability, which can lead to non-threshold best responses.
For example, suppose that there are three adjacent quality levels $q_1<q_2<q_3$ such that $\AccP{\ACCMAP}{q_1} = \AccP{\ACCMAP}{q_2} = \AccP{\ACCMAP}{q_3} = 1/\rho(q_2,1)$.
Based on the earlier argument, there exists a threshold best response at $\theta = q_2$, where the authors submit if and only if $Q\ge q_2$ and the corresponding conference value is $\ConfValue(q_2,1)$.
However, it is also possible for a non-threshold best response to exist: for example, there may be a submission strategy where authors with qualities $q_1, q_2$, and $q_3$ submit with probabilities $r_1, r_2$, and $r_3$, respectively, and $r_1, r_3 \in (0,1)$. As long as the induced conference value remains at $\ConfValue(q_2)$, such a mixed strategy can also constitute a non-threshold best response.

However, when the conference applies a responsive threshold acceptance policy, $\AccP{\ACCMAP}{q}$ is \emph{strictly} increasing in $q$.
This implies that no matter what the conference value is, there exists at most one quality level at which authors are indifferent between submitting and not submitting. 
This immediately guarantees that every best response must be a threshold strategy.
If such a quality level $\theta'$ where authors are indifferent between submitting or not submitting exists, then $\theta = \theta'$ is a de facto threshold.
If not, there again exist two adjacent qualities $\ubar{\theta}, \bar{\theta}\in \QualSet$ such that $\AccP{\ACCMAP}{\ubar{\theta}} < 1/\rho(\ubar{\theta},1)$ and $\AccP{\ACCMAP}{\bar{\theta}} > 1/\rho(\bar{\theta},1)$.
Then, based on the previous arguments, there are again two cases: if $\AccP{\ACCMAP}{\ubar{\theta}}\le 1/\rho(\bar{\theta},1)$, any $\theta \in [\ubar{\theta}, \bar{\theta}]$ is a de facto threshold; if $\AccP{\ACCMAP}{\ubar{\theta}}> 1/\rho(\bar{\theta},1)$, then $\theta = \ubar{\theta}$ is a de facto threshold.
This completes the second part.

% We next prove the third part, so we assume that the conference's policy is a non-trivial threshold acceptance policy.
% By \cref{prop:monotone-prob}, $\AccP{\ACCMAP}{q}$ is strictly monotone in $q$. 
% Therefore, in the categorical model, there exists at most one $\hat{q} \in \QualSet$ such that $\AccP{\ACCMAP}{\hat{q}}=1/\rho$.
% If such a $\hat{q}$ exists, it is a de facto threshold. If not, then because the threshold acceptance policy is non-trivial, there exist qualities $\ubar{q} < \bar{q}$ such that $\AccP{\ACCMAP}{\ubar{q}} < 1/\rho$, $\AccP{\ACCMAP}{\bar{q}} > 1/\rho$, and there are no qualities in $\QualSet$ between $\ubar{q}$ and $\bar{q}$.  
% Then, any $\theta \in [\ubar{q}, \bar{q}]$ is a de facto threshold. To complete the third part of the proof, notice that if $\theta$ is a de facto threshold, then every best response by the author is a $\theta$-threshold strategy.
% {the author is strictly better off to submit any paper with quality $q\ge \bar{q}$ and not submit any paper with quality $q< \bar{q}$, which results in a threshold best response. In this case, any $\theta \in [\ubar{q}, \bar{q}]$ is a de facto threshold.}

Finally, we prove the third part. 
In the continuous model, the distribution of the review noise is assumed to be a bijection, which implies that $\AccP{\ACCMAP}{q}$ is continuously increasing in $q$. 
Moreover, as the density of $\QualDist$ has full support, $1/\rho(q)$ is continuously decreasing in $q$.
Because the acceptance policy is responsive, there must exist a unique intersection such that $\AccP{\ACCMAP}{\theta} = 1/\rho(\theta)$, meaning that $\theta$ is a unique de facto threshold. \Halmos

\subsection{Proof of \cref{prop:threshold-policy}}
\label{app:proof-threshold-policy}

% \dkcomment{We are restating some of the lemmas/propositions from the main text when we prove them in the appendix, but not others. Is there a concrete reason why? If not, should we be consistent? E.g., restate all of them?}

To prove the proposition, we want to relate the acceptance threshold to how ``strict'' the policy is. We begin by defining a comparison between the strictness of two policies:

\begin{definition} \label{def:stricter}
An acceptance policy \ACCMAP['] is \emph{(weakly) stricter} than another policy $\ACCMAP$ if it accepts every paper with a (weakly) smaller probability, i.e., $\AccP{\ACCMAP[']}{q} < \AccP{\ACCMAP}{q}$ for all $q$ (resp., $\AccP{\ACCMAP[']}{q} \leq \AccP{\ACCMAP}{q}$ for all $q$ for the weak version).
\end{definition}

Being stricter appears to be a very demanding requirement, in that it requires an inequality for all paper qualities. 
We next show that for threshold policies, it in fact follows from a strictly smaller acceptance probability for just one paper. 


\begin{lemma}\label{lem:stricter_policy}
Let $\ACCMAP$ and $\ACCMAP[']$ be two threshold acceptance policies. If there exists a $q\in \QualSet$ such that $\AccP{\ACCMAP[']}{q} < \AccP{\ACCMAP}{q}$, then $\ACCMAP[']$ is stricter than $\ACCMAP$.
\end{lemma}

\proof{Proof of \Cref{lem:stricter_policy}.}

  We want to show that if a threshold policy accepts one type of paper with strictly smaller probability, it accepts \emph{every} paper with strictly smaller probability. 

  First, recall that we assumed that the conditional signal distribution has full support on the signal space. Because multiple reviews are i.i.d., conditioned on any paper quality $q$, the review signal distribution has full support over all vectors of review signals.

  Because \ACCMAP, \ACCMAP['] are \emph{threshold} policies, any review vector that leads to acceptance under \ACCMAP['] must lead to acceptance under \ACCMAP with at least the same probability. And because a paper of quality $q$ is accepted with strictly higher probability by \ACCMAP, there must exist a set $S$ of review vectors \RevSigV which are all accepted with strictly higher probability under \ACCMAP than under \ACCMAP['], such that $S$ has strictly positive probability mass under the combination of the distributions $\QualDist$ and $\RevSigDist[q]$.
  
  Because $S$ occurs with positive probability for every paper quality $q'$ (by the full-support assumption on the $\RevSigDist[q]$), every paper is accepted with strictly higher probability by \ACCMAP than by \ACCMAP['], completing the proof. \Halmos
\endproof

The following lemma relates the strictness of a threshold acceptance policy with its acceptance threshold.


\begin{lemma} \label{prop:monotone-prob-threshold}
In the categorical model, let $\ACCMAP[\tau,r]$ and $\ACCMAP[\tau',r']$ be two threshold acceptance policies with $r, r'\in (0,1]$.
Then, if either $\tau'>\tau$ or $\tau'=\tau$ and $r'< r$, $\ACCMAP[\tau',r']$ is weakly stricter than $\ACCMAP[\tau,r]$. 

 In the continuous model, let $\tau$ and $\tau'$ be the thresholds for two non-trivial threshold policies. Then, $\ACCMAP[\tau']$ is stricter than $\ACCMAP[\tau]$ if and only if $\tau' > \tau$.
\end{lemma}

\proof{Proof of \Cref{prop:monotone-prob-threshold}.}

We begin by proving the first part of the lemma, regarding the categorical model. By the definition of $\ACCMAP[\tau,r]$ and $\ACCMAP[\tau',r']$, whenever a paper with some review vector $\RevSigV$ is accepted by $\ACCMAP[\tau',r']$ with positive probability, it will be accepted by $\ACCMAP[\tau,r]$ with at least the same probability. This means that $\ACCMAP[\tau,r]$ accepts every paper with at least the same probability as  $\ACCMAP[\tau',r']$.

% \dkcomment{Switching order of ``if'' vs.~``only if'', because ``if'' takes more work.}
Next, we prove the second statement, regarding the continuous model. For the ``only if'' direction, we know that a stricter threshold policy accepts every paper with strictly smaller probability. In the continuous model, $\ACCMAP[\tau']$ must reject some review vectors that are accepted under $\ACCMAP[\tau]$. This implies $\tau'>\tau$. 

For the ``if'' direction, we begin with some basic observations. First, by \cref{lem:monotone_expected_quality}, $U(\RevSigV)$ is monotone increasing in \RevSigV.
Second, by standard Bayesian updating formulas and using that the review noise is additive and independent of $q$, the expected quality conditioned on the signal \RevSigV can be written as 
\begin{align*}
    U(\RevSigV)
    & = \int q \cdot \frac{\QualDens{q} \cdot \prod_{i=1}^{\NumReviews} \RevSigProb[q]{\RevSig{i}}}{\int \QualDens{q'} \cdot \prod_{i=1}^{\NumReviews} \RevSigProb[q']{\RevSig{i}}\, d q'} \, d q
    = \int q \cdot \frac{\QualDens{q} \cdot \prod_{i=1}^{\NumReviews} f^{(r)}(\RevSig{i}-q)}{\int \QualDens{q'} \cdot \prod_{i=1}^{\NumReviews} f^{(r)}(\RevSig{i}-q') \, d q'} \, d q.
\end{align*}

In this form, because we assumed $f^{(r)}$ to be a continuous function, and $\QualDens{p}$ to be strictly positive, it is easy to see that $U(\RevSigV)$ is also a continuous function of \RevSigV.

Because neither threshold acceptance policy is trivial by assumption, there exist review vectors $\RevSigV, \RevSigVP$ with $U(\RevSigV) < \tau < \tau' < U(\RevSigVP)$, and in particular, because $U$ is monotone, writing $\mathbf{1}$ for the all-1 vector, $\lim_{y \to - \infty} U(y \cdot \mathbf{1}) < \tau$, and $\lim_{y \to \infty} U(y \cdot \mathbf{1}) > \tau'$.
By continuity and monotonicity of $U$, the function $U(y \cdot \mathbf{1})$ is continuous and monotone as a function of $y$, so there exists a $y$ with $U(y \cdot \mathbf{1}) = \frac{\tau+\tau'}{2}$.

Again by continuity of $U$, there is a sufficiently small $\epsilon > 0$ such that for every $\RevSigV$ with $|| \RevSigV -  y \cdot \mathbf{1} ||_2 \leq \epsilon$, we have that 
$U(\RevSigV) \in (\tau, \tau')$.
Let $S = \Set{\RevSigV}{|| \RevSigV - y \cdot \mathbf{1} ||_2 \leq \epsilon}$.
$S$ has positive volume, and because we assumed that $f^{(r)}(x) > 0$ for all $x$, for every quality $q$, the event of obtaining a review vector $\RevSigV \in S$ has positive probability.
Furthermore, whenever the review vector $\RevSigV \in S$, the paper is accepted by $\ACCMAP[\tau]$, but rejected by $\ACCMAP[\tau']$.
Thus, every paper quality $q$ is strictly more likely to be accepted under $\ACCMAP[\tau']$ than under $\ACCMAP[\tau]$.
\Halmos
\endproof

\proof{Proof of \cref{prop:threshold-policy}.}

We prove the proposition based on two cases, depending on whether $\theta\in \QualSet$.
If $\theta \in \QualSet$, let $r\in [0,1]$ be any value such that $\ConfValue(\theta,r)> 1$; the existence of such an $r$ is guaranteed because $\theta$ is a candidate threshold.
In this case, let $\hat{\theta} = \theta$ and $\rho = \frac{\ConfValue(\theta, r)-\TD}{1-\TD}$.
If $\theta \notin \QualSet$, let $\hat{\theta} = \inf\Set{q \in \QualSet}{q > \theta}$ and $\rho = \frac{\ConfValue(\hat{\theta}, 1)-\TD}{1-\TD}$.
Note that because $\theta$ is a candidate threshold, $V(\hat{\theta},1) > 1$. 
This, in both cases (discrete and continuous), we obtain that $1/\rho  < 1$. 

To prove the proposition, it is sufficient to show that, in either case, there exists a threshold acceptance policy such that $\AccP{\ACCMAP}{\hat{\theta}} = 1/\rho$.
To see this, when $\theta \in \QualSet$, it is a best response for authors with $Q > \theta$ to submit, while authors with $Q < \theta$ take the outside option, and authors with $Q = \theta$ submit with probability $r$.
When $\theta \notin \QualSet$, it is a best response for authors with $Q > \theta$ to submit, while authors with $Q < \theta$ take the outside option.
This means that as long as $\AccP{\ACCMAP}{\hat{\theta}} = 1/\rho$, $\theta$ is a de facto threshold.

% Let $\hat{\theta}$ be a value in $\QualSet$ closest to $\theta$, i.e.~$\hat{\theta} \in \arg\min_{q\in \QualSet} (q-\theta)^2$ (breaking ties arbitrarily).
We now show that there always exists a threshold acceptance policy such that $\AccP{\ACCMAP}{\hat{\theta}} = 1/\rho$.
Let $f(\tau) := \AccP{\ACCMAP[\tau,0]}{\hat{\theta}}$ be the probability that a paper of quality $\hat{\theta}$ is accepted under the policy $\ACCMAP[\tau, 0]$, which accepts the paper if and only if its expected posterior quality is greater than $\tau$.
Then, $\lim_{\tau \to -\infty} f(\tau) = 1 > 1/\rho > 0 = \lim_{\tau \to +\infty} f(\tau)$. Furthermore, by Lemma \ref{prop:monotone-prob-threshold}, $f(\tau)$ is a non-increasing function of $\tau$. 
Therefore, there must exist a $\hat{\tau}$ such that $\lim_{\tau \to \hat{\tau} \uparrow} f(\tau) \geq 1/\rho \geq \lim_{\tau \to \hat{\tau} \downarrow} f(\tau)$.
  
If $f$ is continuous at $\hat{\tau}$, then the threshold policy $\ACCMAP[\hat{\tau},0]$ has $\AccP{\ACCMAP[\hat{\tau},0]}{\hat{\theta}} = 1/\rho$ by definition.
Otherwise, let $z = \lim_{\tau \to \hat{\tau} \uparrow} f(\tau) - \lim_{\tau \to \hat{\tau} \downarrow} f(\tau) > 0$. We can infer that there must be a discrete probability of $z$ for the event that \fangcomment{do we use $U(\RevSigV)$?}$\ExpectC{\Qual}{\RevSigV} = \hat{\tau}$, i.e., that $\Prob[{\RevSigV \sim \RevSigDist[{\hat{\theta}}]}]{\ExpectC{\Qual}{\RevSigV} = \hat{\tau}} = z$.
We then consider the threshold policy $\ACCMAP[\hat{\tau},\hat{r}]$ with threshold $\hat{\tau}$ which conditioned on $\ExpectC{\Qual}{\RevSigV} = \hat{\tau}$ accepts a paper with probability $\hat{r} := \frac{1/\rho-\lim_{\tau \to \hat{\tau} \downarrow} f(\tau)}{z}$. The overall acceptance probability of $\ACCMAP[\hat{\tau},\hat{r}]$ for a paper with quality $\hat{\theta}$ is therefore
\begin{align*} 
 & \Prob[{\RevSigV \sim \RevSigDist[{\hat{\theta}}]}]{\ExpectC{\Qual}{\RevSigV} > \hat{\tau}} + \Prob[{\RevSigV \sim \RevSigDist[{\hat{\theta}}]}]{\ExpectC{\Qual}{\RevSigV} = \hat{\tau}} \cdot \frac{1/\rho-\lim_{\tau \to \hat{\tau} \downarrow} f(\tau)}{z}
\\ & = \lim_{\tau \to \hat{\tau} \downarrow} f(\tau)
+ z \cdot \frac{1/\rho-\lim_{\tau \to \hat{\tau} \downarrow} f(\tau)}{z}
\; = \; 1/\rho.
\end{align*}

Thus, under the threshold acceptance policy $\ACCMAP[\hat{\tau},\hat{r}]$, authors of papers with quality $\hat{\theta}$ are indifferent between submitting and not submitting.\fangcomment{do we need a proof for the moreover part?}
% Note that if $\theta\in\QualSet$, then $\hat{\theta}=\theta$. The result for the special case in the proposition statement straightforwardly follows.
% For the general case, assume w.l.o.g.~(the other case is symmetric) that $\hat{\theta} \geq \theta$.
% Because the author is indifferent between submitting and not submitting papers of quality $\Qual = \hat{\theta}$, we may assume that such papers are submitted, and hence all papers of quality at least $\hat{\theta}$. 
% By definition of $\hat{\theta}$, no papers have quality $\Qual \in [\theta, \hat{\theta})$, so all papers of quality at least $\theta$ are submitted. Thus, $\theta$ is also a de facto threshold for the author.

Finally, uniqueness of the conference's threshold policy in the continuous model when the author's submission decision is non-trivial follows directly because by Lemma~\ref{prop:monotone-prob-threshold}, the solution for $\hat{\tau}$ of $\AccP{\ACCMAP[\hat{\tau},0]}{\hat{\theta}} = 1/\rho$ is unique in the continuous model when agents are neither submitting all papers nor submitting no papers.
\Halmos
\endproof

\subsection{Proof of \cref{prop:gap-QB-dominance}}\label{app:proof-gap-QB-dominance}

In the continuous model, fixing the quality prior, the conference quality only depends on the de facto threshold.
Therefore, to prove the proposition, it is sufficient to show that for every candidate threshold $\theta$, the review burden of the first setting is larger than that of the second setting.

Recall that the review burden is given by $R(\theta) = m\int_\theta^\infty \QualDens{q}/\AccP{\ACCMAP[\tau(\theta)]}{q} dq$, suppress parameter dependence and denote the corresponding acceptance thresholds by $\tau(\theta)$ and $\tau'(\theta)$ for the two settings, respectively.
It is thus sufficient to show that $\AccP{\ACCMAP[\tau(\theta)]}{q} < \AccP{\ACCMAP[\tau'(\theta)]}{q}$ for every candidate threshold $\theta$.
This inequality follows directly from the fact that the resubmission gap in the first setting is larger, i.e., $\tau(\theta) > \tau'(\theta)$ for any $\theta$. 
By \cref{prop:monotone-prob-threshold}, the claim follows, completing the proof.

\subsection{Proof of \cref{prop:blackwell}}\label{app:proof-blackwell}

We will show that any memoryless acceptance policy $\ACCMAP[']: \SigSet' \to [0,1]$ in the second setting can be simulated in the first setting with the same expected conference quality and review burden. It follows that both the QB-tradeoff and the QA-tradeoff in the first setting weakly dominate those in the second setting.

Because $\RevSigDist$ Blackwell dominates $\RevSigDistP$, there exists a garbling $\gamma$ from $\SigSet$ to $\SigSet'$.  
We define a memoryless acceptance policy $\ACCMAP$ in the first setting: for any review signal $\REVSIG$, we set
\begin{align*}
\AccMap{\REVSIG} & = \sum_{\REVSIGP \in \SigSet'} 
\AccMap[']{\REVSIGP} \cdot \gamma (\REVSIG, \REVSIGP). 
\end{align*}

First, because $(\gamma(\REVSIG, \REVSIGP))_{\REVSIGP \in \SigSet'}$ is a distribution on $\SigSet'$ and $\AccMap[']{\REVSIGP} \in [0,1]$ for all $\REVSIGP$, the output of $\ACCMAP$ is in $[0,1]$. 
Moreover, for any paper quality $\qual \in \QualSet$,
\begin{align*}
\AccP{\ACCMAP}{\qual} 
& = \sum_{\dkreplace{\RevSigV}{\REVSIG} \in \SigSet} 
\RevSigProb[\qual]{\REVSIG}
\cdot \AccMap{\REVSIG} \\
& = \sum_{\dkreplace{\RevSigV}{\REVSIG} \in \SigSet} 
 \RevSigProb[\qual]{\REVSIG}
\cdot \sum_{\REVSIGP \in \SigSet'} \AccMap[']{\REVSIGP} \cdot
\gamma(\REVSIG, \REVSIGP) \tag{Definition of $\ACCMAP$}\\
& = \sum_{\REVSIGP \in \SigSet'} \left( \sum_{\REVSIG \in \SigSet}
\RevSigProb[\qual]{\REVSIG} \cdot \gamma(\REVSIG, \REVSIGP) \right)
\cdot \AccMap[']{\REVSIGP} \tag{Changing order of summation} \\
& = \sum_{\REVSIGP \in \SigSet'}  \RevSigProbP[\qual]{\REVSIGP} \cdot \AccMap[']{\REVSIGP} \tag{$\gamma$ is a garbling} \\
& = \AccP{\ACCMAP[']}{\qual}.
\end{align*}

Thus, the acceptance policies $\ACCMAP$ and $\ACCMAP[']$ have identical acceptance probabilities for each paper quality; this makes them indistinguishable to authors. In particular, both acceptance policies have the same expected conference quality and review burden. \Halmos


\subsection{Proof of \cref{prop:blackwell-threshold}}\label{app:proof-blackwell-threshold}

We first prove \cref{claim:blackwell-RB-better}, which is the key to the proof of \cref{prop:blackwell-threshold}.
We note that our proof is written for the categorical model. However, the proof for the continuous model is analogous, simply by replacing the summations with integrals.
\yzcomment{Note for future: better to check if this is precisely true.}
% \fangdelete{
% \restatedlemma{claim:blackwell-RB-better}{
%     Consider two threshold acceptance policies $\ACCMAP$ and $\ACCMAP[']$ which accept papers of quality $\bar{q}$ with \yichiedit{equal} probability in the first and the second setting, respectively.  
%     Then, under the author's $\bar{q}$-threshold strategy, 
%     % the review burden of $\ACCMAP$ in the first setting is no larger than the review burden of $\ACCMAP'$ in the second setting.  
%     \yichiedit{the acceptance probability of a paper of quality $q$ in the first setting is no less than that in the second setting, for any $q>\bar{q}$.}
%     }}


\proof{Proof of \cref{claim:blackwell-RB-better}.}
    Let $(\tau, r)$ and $(\tau', r')$ be the parameters corresponding to the threshold policies $\ACCMAP$ and $\ACCMAP'$, respectively.
    Given that the acceptance probability of $\bar{q}$ is the same under $\ACCMAP[\tau,r]$ and $\ACCMAP[\tau', r']$, we want to show that for every $q > \bar{q}$, the acceptance probability is weakly higher under $\ACCMAP[\tau,r]$ than under $\ACCMAP[\tau', r']$.
    By decomposing the acceptance probability into the individual signals, we need to show that
    \begin{equation} \label{eq:blackwell_ineq}
        r \cdot \RevSigProb[q]{\tau} + \sum_{s>\tau} \RevSigProb[q]{\REVSIG} 
        - \left(r'\cdot \RevSigProbP[q]{\tau'} + \sum_{\REVSIGP>\tau'} \RevSigProbP[q]{\REVSIGP} \right) 
        \ge 0,
    \end{equation}
    for any $q > \bar{q}$.

    Let $\gamma$ be the garbling from $\RevSigDist$ to $\RevSigDistP$ (see \cref{def:blackwell}), so that $\RevSigProbP[q']{\REVSIGP} = \sum_{\REVSIG} \RevSigProb[q']{\REVSIG} \cdot \gamma(\REVSIG, \REVSIGP)$ for any $q'$ and $\REVSIGP$. 
    Substituting the garbling-based characterization and changing the order of summation, the left-hand side of \cref{eq:blackwell_ineq} can be rewritten as
    \begin{align*}
        \lefteqn{r \cdot \RevSigProb[q]{\tau} + \sum_{s>\tau} \RevSigProb[q]{\REVSIG}
        - \left( 
        r' \cdot \sum_{\REVSIG} \RevSigProb[q]{\REVSIG} \cdot \gamma(\REVSIG, \tau')
        + \sum_{\REVSIGP>\tau'} \sum_{\REVSIG} \RevSigProb[q]{\REVSIG} \cdot \gamma(\REVSIG, \REVSIGP)  \right)}
        \\ & =
        r \cdot \RevSigProb[q]{\tau} + \sum_{s>\tau} \RevSigProb[q]{\REVSIG} 
        - \sum_{\REVSIG} \left( r'\cdot\gamma(\REVSIG, \tau') + \sum_{\REVSIGP>\tau'} \gamma(\REVSIG, \REVSIGP) \right) \cdot \RevSigProb[q]{\REVSIG}.
    \end{align*}
    
    For notational simplicity, let $h(\REVSIG)=  r'\cdot\gamma(\REVSIG, \tau') + \sum_{\REVSIGP>\tau'} \gamma(\REVSIG, \REVSIGP)$. 
    Because $h(s)$ is a probability (the probability of observing an instantiation of the Blackwell dominated signal that is accepted by $\ACCMAP[\tau', r']$ when the instance of the Blackwell dominating signal is $s$), $0\le h(\REVSIG)\le 1$. 
    We now break the summation over $s$ in the above equation into three summations, namely, the summation over $s<\tau$, $s=\tau$, and $s>\tau$. Combining the summations over the same subset of signals allows us to rewrite the preceding equation as
    \begin{align}
        \lefteqn{r \cdot \RevSigProb[q]{\tau} + \sum_{s>\tau}\RevSigProb[q]{\REVSIG} 
        - \sum_{\REVSIG} h(\REVSIG) \cdot \RevSigProb[q]{\REVSIG}} \nonumber 
       \\ & =
          (r-h(\tau)) \cdot \RevSigProb[q]{\tau} 
        + \sum_{s>\tau} (1-h(\REVSIG)) \cdot \RevSigProb[q]{\REVSIG} 
        - \sum_{s<\tau} h(\REVSIG) \cdot \RevSigProb[q]{\REVSIG}.\label{eq:blackwell_acc_prob_diff}
    \end{align}

    Next, we use the monotone likelihood ratio property.
    Let $\eta_s := \frac{\RevSigProb[q]{s}}{\RevSigProb[\bar{q}]{s}}$ denote the likelihood ratio for signal $s$. 
    Because signals have MLR, we have $\eta_{s'} > \eta_s$ for any $s'>s$.
    Using this monotonicity, we can bound \eqref{eq:blackwell_acc_prob_diff} as
    \begin{align*}
        \eqref{eq:blackwell_acc_prob_diff} 
        & = (r-h(\tau)) \cdot \eta_\tau \RevSigProb[\bar{q}]{\tau} 
        + \sum_{s>\tau} (1-h(\REVSIG)) \cdot \eta_s \RevSigProb[\bar{q}]{\REVSIG} 
        - \sum_{s<\tau} h(\REVSIG) \cdot \eta_s \RevSigProb[\bar{q}]{\REVSIG}
        \\ & \ge
        (r-h(\tau)) \cdot \eta_\tau \RevSigProb[\bar{q}]{\tau} 
        + \sum_{s>\tau} (1-h(\REVSIG)) \cdot \eta_\tau \RevSigProb[\bar{q}]{\REVSIG} 
        - \sum_{s<\tau} h(\REVSIG) \cdot \eta_\tau \RevSigProb[\bar{q}]{\REVSIG} 
        \\ & =
        \eta_\tau \cdot \left(
        r\cdot \RevSigProb[\bar{q}]{\tau} 
        + \sum_{s>\tau} \RevSigProb[\bar{q}]{\REVSIG} 
        - \sum_{\REVSIG}h(\REVSIG)\RevSigProb[\bar{q}]{\REVSIG}
        \right)
        \\ & =
        \eta_\tau \cdot 
        \left( \AccP{\ACCMAP[\tau,r]}{\bar{q}} - \AccP{\ACCMAP[\tau',r']}{\bar{q}} \right)
        \\ & = 0.
    \end{align*}
    Here, the inequality uses the monotone likelihood property separately for $s > \tau$ and $s < \tau$ (observing the signs of the multipliers of $\eta_s$), and the final step follows from the assumption that the acceptance probabilities of papers of quality $\bar{q}$ in both settings are equal. \Halmos
\endproof

With this lemma, we are able to prove \cref{prop:blackwell-threshold}.
\proof{Proof of \cref{prop:blackwell-threshold}.}  
For any responsive threshold acceptance policy $\ACCMAP[']$ in the second setting (the one that has weaker review signals), let $\vartheta = (\theta, r)$ be any of the author's equilibrium strategies to $\ACCMAP[']$.
% (As discussed in \cref{prop:de_facto}, there might be more than one best response.)
We will show the existence of a threshold acceptance policy $\ACCMAP$ in the first setting such that: 1) $\vartheta$ is also an equilibrium strategy to $\ACCMAP$; 2) the conference quality is the same in both settings; 3) the review burden induced by the policy-response pair $(\ACCMAP, \vartheta)$  in the first setting is at most the review burden induced by the policy-response pair  $(\ACCMAP', \vartheta)$ in the second setting; and 4) the author welfare induced by the policy-response pair $(\ACCMAP, \vartheta)$ in the first setting is at least the author welfare induced by the policy-response pair  $(\ACCMAP', \vartheta)$ in the second setting. If the above is true, then every point on the QB-tradeoff curve of the second setting is weakly dominated by a point on the QB-tradeoff curve of the first setting, which completes the proof. The same arguments apply to QA-tradeoff curves.
Note that it is sufficient to consider only responsive threshold policies because any policy that is not responsive either attracts no submissions (where the claim trivially holds) or accepts all submissions with a probability of 1 (which leads to a conference value smaller than 1 and thus is not a candidate threshold).

% First note that if $\ACCMAP[']$ always accepts or always rejects, then the proposition trivially holds.  
% Thus, we can assume that $\ACCMAP[']$ is non-trivial. 
By \cref{prop:de_facto}, the author's equilibrium strategy under a responsive threshold acceptance policy is a threshold strategy with some threshold $\theta$ (and probability $r$): the author will always submit a paper with quality $\Qual > \theta$, not submit if $\Qual < \theta$, and submit with probability $r$ if $\Qual = \theta$.
% Additionally, if $\vartheta$ is the strategy under which the author always submits everything, then letting $\ACCMAP$ be the policy that accepts everything minimizes the review burden without changing the conference quality. Therefore, without loss of generality, we can assume that not every paper is submitted under $\vartheta$.

Now, we consider two cases based on whether there are authors who are indifferent between submitting and not submitting.

\begin{enumerate}
\item The first case is that there exists a quality $\bar{q}$ such that an author with threshold strategy $\vartheta$ submits papers of quality $\bar{q}$ with probability $r$ in equilibrium. Because $\vartheta$ is a threshold strategy, there can be at most one such quality. Let $\rho = \rho(\theta, r)$ be the conference attractiveness under this setting. By \cref{prop:de_facto}, the acceptance probability at $\bar{q}$ must be exactly $1/\rho$.\fangcomment{Minor issue: this cannot be directly derived from \cref{prop:de_facto} }

Now, consider a threshold acceptance policy $\ACCMAP$ in the first setting that also induces $\vartheta$ as the author's best response. By construction, $\bar{q}$ is a candidate threshold. Therefore, the existence of $\ACCMAP$ is guaranteed by \cref{prop:threshold-policy}.
% Note that $\theta$ is a best response to $\phi'$ because by \gs{fill in ref here.} the acceptance rate is strictly monotone in the paper quality. 
Note that the conference quality and the attractiveness $\rho$ remain unchanged: papers with qualities strictly greater than $\bar{q}$ are all accepted, no paper with quality strictly less than $\bar{q}$ is accepted; and papers with quality $\bar{q}$ are submitted with probability $r$, and these papers are accepted i.i.d.~with probability $1/\rho$ in each setting. 
Therefore, by \cref{claim:blackwell-RB-better}, all submitted papers have a weakly larger acceptance probability under $\ACCMAP$ in the first setting, resulting in a weakly smaller review burden and a weakly larger author welfare.

\item In the second case, for any possible paper quality $q$, an author with threshold strategy $\vartheta$ either always or never submits papers of quality $q$.  
% Such a $\vartheta$ exists because we are assuming the categorical model, and we assumed that not every paper is submitted. 
Let $\bar{q}$ be the highest paper quality which is never submitted and let $\rho = \rho(\bar{q}, 0)$ be the conference attractiveness.

Let $\hat{\phi}'$ denote the threshold policy which accepts papers of quality $\bar{q}$ with probability $1/\rho$ in the second setting. 
Note that both $\ACCMAP'$ and $\hat{\phi}'$ induce $\vartheta$ as the author's best response, because we can assume that authors with papers of quality $\bar{q}$ never submit under $\hat{\phi}'$, given that $\AccP{\hat{\phi}'}{\bar{q}} = 1/\rho$, i.e., the authors are indifferent between submitting papers of quality $\bar{q}$ or not. 
This implies that $\hat{\phi}'$ induces the same conference quality as $\ACCMAP'$. Fixing the author's strategy $\vartheta$, the review burden of $\hat{\phi}'$ is no larger than that of $\ACCMAP'$ and the author welfare of $\hat{\phi}'$ is no less than that of $\ACCMAP'$. This is because by definition, $\ACCMAP'$ accepts papers of quality $\bar{q}$ with probability at most $1/\rho$, which is the acceptance probability of papers of quality $\bar{q}$ under $\hat{\phi}'$. Then, by \cref{lem:stricter_policy}, the acceptance probability of papers of any quality is weakly larger under $\hat{\phi}'$ than $\ACCMAP'$, resulting in a weakly smaller review burden and a weakly larger author welfare under $\hat{\phi}'$.

% First, we observe that the review burden of $\hat{\phi}$ is at most that of $\phi$ in setting one with author strategy $\theta$.  This is because the threshold of $\hat{\phi}$ is at most that of $\phi$.  Notice that $\phi$ accepts papers with quality $\bar{q}$ with probability at most $\rho$, because otherwise, $\theta$ would not be a best response.  If $\phi$ accepts papers with quality $\bar{q}$ with probability exactly $\rho$, then by definition, the threshold of $\hat{\phi}$ is at most that of $\phi$.   If $\phi$ accepts papers with quality $\bar{q}$ with probability strictly less than  $\rho$, then, the threshold of $\hat{\phi}$ must be strictly less than that of $\phi$ because acceptance probability is monotone in threshold.  \gs{should we cite something here?}

Finally, in the first setting, let $\ACCMAP$ be a threshold policy that induces a threshold best response with $(\bar{q}, 0)$ for the author such that authors with paper quality $\bar{q}$ prefer not to submit and $\AccP{\ACCMAP}{\bar{q}} = 1/\rho(\bar{q}, 0)$. This implies that the conference attractiveness is $\rho = \rho(\bar{q}, 0)$, the same as in the second setting. Because $\ACCMAP$ is a responsive threshold policy, by \cref{prop:de_facto}, $\bar{q}$ is a de facto threshold. Therefore, the existence of $\ACCMAP$ is guaranteed by \cref{prop:threshold-policy}. By this construction, first note that $\vartheta$ is also a best response to $\ACCMAP$.
Second, the conference quality under $\ACCMAP$ is identical to the conference quality under $\ACCMAP'$ because the same set of papers is submitted, and all are eventually accepted. Finally, by \cref{claim:blackwell-RB-better}, because papers of quality $\bar{q}$ are accepted with probability $1/\rho$ under $\ACCMAP$ in the first setting and under $\hat{\phi}'$ in the second setting, all papers with quality larger than $\bar{q}$ have a larger acceptance probability in the first setting. Therefore, the review burden of $\ACCMAP$ in the first setting is no larger than the review burden of $\hat{\phi}'$ which is no larger than the review burden of $\ACCMAP'$ in the second setting. The same argument holds analogously for author welfare. This completes the proof. \Halmos
\end{enumerate}
\endproof

\subsection{Proof of \cref{lemma:query burden value}}\label{app:proof-QB-tradeoff}

% To verify the definition of the dominance of QB-tradeoff curves, we will show that for any point on $\mathcal{C'}$ that neither corresponds to accepting all papers nor to rejecting all papers, there is a point on $\mathcal{C}$ that has the same conference quality but strictly smaller review burden.

We first consider the continuous model. Fixing a de facto threshold $\theta$, whose existence as a candidate threshold is ensured by \cref{prop:de_facto}), the conference value $\ConfValue(\theta)$ is fixed. Therefore, the conference attractiveness factor in the setting with a larger $\TD$, which is $\rho = \frac{\ConfValue(\theta) - \TD}{1-\TD}$, is larger than that in the setting with a smaller $\rho'$.

Next, we show that the setting with a larger attractiveness factor $\rho$ has a QB-tradeoff curve dominated by the setting with a smaller factor $\rho' < \rho$.

By \cref{prop:threshold-policy}, both settings have respective unique corresponding non-trivial threshold acceptance policies with thresholds $\tau$ and $\tau'$. By \cref{prop:de_facto}, a paper with quality $\theta$ is accepted with probabilities $1/\rho$ and $1/\rho'$, respectively, in the two settings.  
This means that papers with quality $\theta$ are accepted with strictly smaller probability by $\ACCMAP[\tau]$ than $\ACCMAP[\tau']$. 
By \cref{lem:stricter_policy}, $\ACCMAP[\tau]$ is stricter than $\ACCMAP[\tau']$.
Thus, for papers of every quality $q$, the expected number of rounds that a paper of quality $q$ has to be resubmitted is strictly larger in the setting with attractiveness factor $\rho$, which implies a strictly larger review burden.
Therefore, the QB-tradeoff curve in the setting with a larger attractiveness factor, is dominated by the QB-tradeoff curve in the other setting.

The argument for categorical models is essentially the same. We fix the authors' threshold strategy $(\bar{q}, r)$ such that authors with papers of quality $\Qual>\bar{q}$ submit, authors with $\Qual<\bar{q}$ take the outside option, and authors with $\Qual = \bar{q}$ submit with probability $r$. This fixes the conference quality and the conference value $\ConfValue(\bar{q}, r)$. 

We want to show that the acceptance policy which induces such an author's best response has a smaller review burden in the setting with $\rho'$ than in the setting with $\rho>\rho'$. Without loss of generality, suppose $r<1$. If $r\in (0,1)$, by \cref{lem:author_response}, the acceptance probability $\AccP{\ACCMAP}{\bar{q}} = 1/\rho < \AccP{\ACCMAP'}{\bar{q}} = 1/\rho'$. This means that the acceptance policy $\ACCMAP$ is stricter than $\ACCMAP'$. Based on the same argument as in the continuous model, this means that $\ACCMAP$ leads to a larger review burden. 
If $r = 0$, there might be multiple acceptance policies that induce the same threshold equilibrium. Let $\ACCMAP$ be any one of them. Because papers of quality $\bar{q}$ are not submitted, we must have $\AccP{\ACCMAP}{\bar{q}} \leq 1/\rho$. Now, let $\ACCMAP[']$ be a threshold acceptance policy in the setting with attractiveness $\rho'$ which accepts papers of quality $\bar{q}$ with probability $1/\rho'$. Such a policy $\ACCMAP[']$ exists by Proposition~\ref{prop:threshold-policy}, i.e., $\ACCMAP[']$ induces $\bar{q}$ as a de facto threshold. Because $1/\rho' > 1/\rho$, by \cref{lem:stricter_policy}, $\ACCMAP$ is stricter than $\ACCMAP[']$. Again, based on the same argument, this completes the proof.
%
% The argument for categorical models is essentially the same, but slight care needs to be taken to account for the non-uniqueness of threshold policies. Note that fixing a de facto threshold does not fix the conference quality in the categorical model when $\theta\in\QualSet$. However, we can always additionally fix the author's strategy (for both settings considered) if there exists a paper quality for which the authors are indifferent between submitting or not; this fixes the conference quality. Therefore, fixing a de facto threshold $\theta$ is still sufficient to complete the proof.
% Let $q < \theta$ be the largest quality in the quality set $\QualSet$ that is strictly smaller than $\theta$. (We define $q$ this way even if $\theta$ itself is in the quality set.)
% Such a $q$ must exist because $\theta$ was a non-trivial threshold.
% Let $\ACCMAP$ be any threshold acceptance policy which induces $\theta$ as the de facto threshold in the setting with conference attractiveness $\rho$.
% Because papers of quality $q$ are not submitted, we must have $\AccP{\ACCMAP}{q} \leq 1/\rho$.
% Now, let $\ACCMAP[']$ be a threshold acceptance policy in the setting with attractiveness $\rho'$ which accepts papers of quality $q$ with probability $1/\rho'$. Such a policy $\ACCMAP[']$ exists by Proposition~\ref{prop:threshold-policy}. Because $1/\rho' > 1/\rho$, by Lemma~\ref{lem:stricter_policy}, $\ACCMAP$ is stricter than $\ACCMAP[']$.
% Therefore, all papers are accepted with at least the same probability under $\ACCMAP[']$ as under $\ACCMAP$. 
% Therefore, the review burden under $\ACCMAP[']$ is strictly smaller than under $\ACCMAP$, and since this holds for all non-trivial $\theta$, we have shown domination of the QB-tradeoff.
\Halmos

\subsection{Proof of \cref{lem:QA_eta}}\label{app:proof-QA-eta}

% \dkcomment{Also, why are we writing $\partial$? Why not just $d$? Isn't everything just scalar?}\yzcomment{chanegd}

  Define $g(\TD) = \frac{1-\TD}{\AccP{\tau(\TD)}{q}} + \TD$. 
  Because $u^{(a)}(q,\TD) = \frac{\ConfValue}{g(\TD)}$, we know that $u^{(a)}$ is increasing in $\TD$ if and only if $g$ is decreasing in $\TD$.
  Note that $\AccP{\tau(\TD)}{q} = 1 - \REVNOISEDIST(\tau(\TD) - q)$ so that $\frac{d \AccP{\tau(\TD)}{q}}{d \TD} = - f^{(r)}(\tau(\TD) - q) \cdot \frac{d \tau(\TD)}{d \TD}$.
    
    Recall that $\tau(\TD)$ is chosen so that authors with a paper of quality $\theta$ are indifferent between submitting and taking the outside option, i.e., $\AccP{\tau(\TD)}{\theta} = \frac{1-\TD}{\ConfValue-\TD}$. Taking the derivative of both sides of the equation w.r.t.~$\TD$, and combining with the previous derivative calculation, we have $\frac{d \tau(\TD)}{d \TD} = \frac{\ConfValue-1}{(\ConfValue-\TD)^2 \cdot f^{(r)}(\tau(\TD) - \theta)}$.

    Then, taking the derivative of $g$ w.r.t.~$\TD$ yields
    \begin{align*}
        g'(\TD) &= 1 + \frac{-(1-\REVNOISEDIST(\tau(\TD) - q)) + (1-\TD) \cdot \frac{(\ConfValue-1)f^{(r)}(\tau(\TD) - q)}{(\ConfValue-\TD)^2 \cdot f^{(r)}(\tau(\TD) - \theta)}}{(1-\REVNOISEDIST(\tau(\TD) - q))^2}.
    \end{align*}
    $g'(\TD)$ is positive if and only if 
    \begin{align*}
        (1-\REVNOISEDIST(\tau(\TD) - q))^2 -(1-\REVNOISEDIST(\tau(\TD) - q)) + (1-\TD) \cdot \frac{(\ConfValue-1) \cdot f^{(r)}(\tau(\TD) - q)}{(\ConfValue-\TD)^2 \cdot f^{(r)}(\tau(\TD) - \theta)} 
        & > 0,
    \end{align*}
    which can be rearranged to
    \begin{align*}
    \frac{(1-\TD) \cdot (\ConfValue-1) \cdot f^{(r)}(\tau(\TD) - q)}{(\ConfValue-\TD)^2 \cdot f^{(r)}(\tau(\TD) - \theta)} 
    & > \REVNOISEDIST(\tau(\TD) - q) \cdot (1-\REVNOISEDIST(\tau(\TD) - q)).
    \end{align*}
    Because $\frac{1-\TD}{\ConfValue-\TD} = 1 - \REVNOISEDIST(\tau(\TD) - \theta)$ and $\frac{\ConfValue-1}{\ConfValue-\TD} = \REVNOISEDIST(\tau(\TD) - \theta)$, the preceding inequality can be further rearranged to
    \begin{align*}
       \frac{\REVNOISEDIST(\tau(\TD) - \theta) \cdot (1-\REVNOISEDIST(\tau(\TD) - \theta))}{f^{(r)}(\tau(\TD) - \theta)} 
       & > \frac{\REVNOISEDIST(\tau(\TD) - q) \cdot (1-\REVNOISEDIST(\tau(\TD) - q))}{f^{(r)}(\tau(\TD) - q)},
    \end{align*}
    which holds if and only if $h(q) > h(\theta)$.

Therefore, the author's utility is increasing in $\TD$ if and only if $h(q) < h(\theta)$. By reversing all inequalities in the preceding calculations (i.e., comparison of the derivative with 0), we can show that the author's utility is decreasing in $\TD$ if and only if $h(q) > h(\theta)$, and remains unchanged in $\TD$ if $h(q) = h(\theta)$.\Halmos

\section{Plots and details about simulations}
\label{Appendix_simulations}

Plots corresponding to Simulations~\ref{Section_simulations_Pareto} and \ref{Section_simulations_Gaussian} are presented in Figures~\ref{Pareto_histograms}, \ref{sample_size_pareto} and  \ref{Simulations2_plots}. 


The experiments from Simulations~\ref{Section_simulations_Gaussian} were inspired by \cite{Natasa_Tagasovska} and implementations of other baseline methods are also taken from \cite{Natasa_Tagasovska} and \cite{immer2022identifiability}. 

For LOCI, we use the default format with neural network estimations and subsequent independence testing (also denoted as $NN-LOCI_H$) \citep{immer2022identifiability}.
For IGCI, we use the original implementation from \cite{IGCI} with slope-based estimation with Gaussian and uniform reference measures. For RESIT, we use the implementation from \cite{Peters2014} with GP regression and the HSIC independence test with a threshold value of $0.05$. For the slope algorithm, we use the implementation of \cite{Slope}, with the local regression included in the fitting process. For comparisons with other methods such as PNL, GPI-MML, ANM, Sloppy, GR-AN, EMD, GRCI, see Section 3.2 in \cite{Natasa_Tagasovska} and Section 5 in \cite{immer2022identifiability}. 


\begin{figure}[]
\centering
\includegraphics[width = \textwidth]{figures/pareto_histogram.pdf}
\caption{Simulations~\ref{Section_simulations_Pareto}. Distributions of the p-values from the independence test in Step 1b) of Algorithm~\ref{Algorithm1}, for model \eqref{fwesef} with \(\alpha = 0\) and \(\alpha = 2\).}
\label{Pareto_histograms}
\end{figure}


\begin{figure}[]
\centering
\includegraphics[scale=0.9]{figures/sample_size_pareto.pdf}
\caption{Simulations~\ref{Section_simulations_Pareto}. The plot displays the percentage of correctly estimated causal directions across a range of sample sizes \( n \), using model \eqref{fwesef} with hyperparameters \(\alpha = 1\) and \(\alpha = 2\). As \( n \) increases, the algorithm demonstrates near-perfect performance, affirming the theoretical consistency of the proposed method. }
\label{sample_size_pareto}
\end{figure}




\begin{figure}[ht]
\centering
\includegraphics[scale=0.4]{figures/sim6.2_sample.png}
\caption{Simulations~\ref{Section_simulations_Gaussian}. An example of datasets generated via different models. }
\label{Simulations2_plots}
\end{figure}


















\iffalse


\begin{figure}[ht]
\centering
\includegraphics[scale=0.45]{figures/Simulations3_Pareto.pdf}
\caption{Simulations~\ref{Section_simulations_robustness}.  An example of a randomly generated function $\theta$ and a generated dataset in which  $effect\mid cause\sim Pareto(\theta(cause))$. Note that if $\theta(x)$ is small, then $effect\mid cause=x$ will have heavy tails. If $\theta(x)<1$, then the expectation of $effect\mid cause=x$ does not exist.}
\label{Simulations3_plot}
\end{figure}



\begin{figure}[ht]
\centering
\includegraphics[scale=0.5]{figures/3ploty.pdf}
\caption{ Total income ($X_1$), Food expenditure ($X_2$), and Alcohol beverages expenditure ($X_3$).  }
\label{Just_plots}
\end{figure}

\begin{figure}[ht]
\centering
\includegraphics[scale=0.5]{figures/histograms.pdf}
\caption{ Total income ($X_1$), Food expenditure ($X_2$), and Alcohol expenditure ($X_3$).  }
\label{Just_histograms}
\end{figure}

\fi 






%\section{Appendix for Proofs}

\paragraph{Proof of Theorem \ref{thm:main}.}

\begin{proof}
\label{proof:main}
Our proof has two steps. In Step 1, we will show that SimCLR is equivalent to minimizing the cross entropy loss defined in Eqn.~(\ref{eqn:cross-entropy}). 
In Step 2, we will show  that minimizing the cross-entropy loss 
is equivalent to spectral clustering on $\bfpi$. 
Combining the two steps together, we have proved our theorem. 

\textbf{Step 1: } SimCLR is equivalent to minimizing the cross entropy loss.

The cross-entropy loss takes expectation over 
$\bfW_\bfX\sim \mathbb{P}(\cdot ; \bfpi)$, 
which means $\bfW_\bfX$ has exactly one non-zero entry in each row $i$. By Lemma~\ref{lem:multinomial}, we know every row $i$ of $\bfW_\bfX$ is independent of other rows. Moreover, 
$\bfW_{\bfX,i}\sim \mathcal{M}(1, \bfpi_i/\sum_j \bfpi_{i,j})=\mathcal{M}(1, \bfpi_i)$, because $\bfpi_i$ itself is a probability distribution.
Similarly, we know $\bfW_\bfZ$ also has the row-independent property by sampling over $\mathbb{P}(\cdot;\bfK_\bfZ)$.
Therefore, by Lemma~\ref{lem:cross_split}, we know Eqn.~(\ref{eqn:cross-entropy}) is equivalent to:
\[
 -\sum_{i=1}^n \mathbb{E}_{\bfW_{\bfX,i}}[\log \mathbb{P}(\bfW_{\bfZ,i}=\bfW_{\bfX,i};\bfK_\bfZ)],
\]

This expression takes expectation over $\bfW_{\bfX,i}$ for the given row $i$. Notice that 
$\bfW_{\bfX,i}$ has exactly one non-zero entry, which equals $1$ (same for $\bfW_{\bfZ,i}$). 
As a result
we expand the above expression to be:
\begin{equation}
 -\sum_{i=1}^n \sum_{j\neq i} \Pr(\bfW_{\bfX,i,j}=1)\log \Pr(\bfW_{\bfZ,i,j}=1).
\label{eqn:detailed-expansion}    
\end{equation}


By Lemma~\ref{lem:multinomial}, $\Pr(\bfW_{\bfZ,i,j}=1)=\bfK_{\bfZ,i,j}/\|\bfK_{\bfZ,i}\|_1$ for $j\neq i$. Recall that $\bfK_\bfZ=(k(\bfZ_i-\bfZ_j))_{(i,j)\in[n]^2}$, which means 
$\bfK_{\bfZ,i,j}/\|\bfK_{\bfZ,i}\|_1=\frac{\exp(-\|\bfZ_i-\bfZ_j\|^2/{2\tau})}{\sum_{k\neq i}
\exp(-\|\bfZ_i-\bfZ_k\|^2/{2\tau})
}$ for $j\neq i$, when $k$ is the Gaussian kernel with variance $\tau$. 

Notice that $\bfZ_i=f(\bfX_i)$, so we know
\begin{equation}
-\log \Pr(\bfW_{\bfZ,i,j}=1)=
-\log \frac{\exp(-\|f(\bfX_i)-f(\bfX_j)\|^2/{2\tau})}{\sum_{k\neq i}
\exp(-\|f(\bfX_i)-f(\bfX_k)\|^2/{2\tau}),
}
\label{eqn:infonce-equivalence}    
\end{equation}


The right hand side is exactly the InfoNCE loss defined in Eqn.~(\ref{eqn:infonce}).
Inserting Eqn.~(\ref{eqn:infonce-equivalence}) into Eqn.~(\ref{eqn:detailed-expansion}), we get the SimCLR algorithm, which first samples augmentation pairs $(i,j)$ with $\Pr(\bfW_{\bfX,i,j}=1)$ for each row $i$, and then optimize the InfoNCE loss. 

\textbf{Step 2: } minimizing the cross entropy loss 
is equivalent to spectral clustering on $\bfpi$.


By Lemma~\ref{lem:convert_to_spectral}, we may further convert the loss to 
\begin{equation}
\label{eqn:main-theorem-repul-attr}
\min_{\bfZ}
-\sum_{(i,j)\in [n]^2} \mathbf{P}_{i,j}
\log k (\bfZ_i-\bfZ_j)+\log \mathbf{R}(\bfZ).
\end{equation}
Since $k$ is the Gaussian kernel, this reduces to \[
\min_\bfZ \mathrm{tr}(\bfZ^\top \mathbf{L}(\bfpi) \bfZ)
+\log \mathbf{R}(\bfZ),
\]

where we use the fact that $\mathbb{E}_{\bfW_\bfX\sim \mathbb{P}(\cdot; \bfpi)}[\mathbf{L}(\bfW_\bfX)]
=\mathbf{L}(\bfpi)
$, because the Laplacian operator is linear and $
\mathbb{E}_{\bfW_\bfX\sim \mathbb{P}(\cdot; \bfpi)}(\bfW_\bfX)=\bfpi
$.
\end{proof}

\paragraph{Proof of Theorem \ref{thm:clip}.}
\begin{proof}
Since $\bfW_\bfX\sim \mathbb{P}(\cdot;\bfpi_{\mathbf{A}, \mathbf{B}})$, we know 
$\bfW_\bfX$ has exactly one non-zero entry in each row, denoting the pair that got sampled. 
A notable difference compared to the previous proof is we now have $n_\mathcal{A}+n_\mathcal{B}$ objects in our graph. CLIP deals with this by taking a mini-batch of size $2N$, 
such that $n_\mathcal{A}=n_\mathcal{B}=N$, and adding the $2N$ InfoNCE losses together. We label the objects in $\mathcal{A}$ as $[n_\mathcal{A}]$, and the objects in $\mathcal{B}$ as $\{n_\mathcal{A}+1, \cdots, n_\mathcal{A}+n_\mathcal{B}\}$. 

Notice that $\bfpi_{\mathbf{A}, \mathbf{B}}$ is a bipartite graph, so the edges of objects in $\mathcal{A}$ will only connect to object in $\mathcal{B}$ and vice versa. We can define the similarity matrix in $\cZ$ as $\bfK_\bfZ$, 
where $\bfK_\bfZ(i, j+n_\mathcal{A})=\bfK_\bfZ(j+n_\mathcal{A},i)= k(\bfZ_i-\bfZ_j)$ for $i\in [n_\mathcal{A}], j\in [n_\mathcal{B}]$, and otherwise we set $\bfK_\bfZ(i,j)=0$. 
The rest is same as the previous proof. 
\end{proof}

\paragraph{Proof of Theorem \ref{thm:exponential}.}

\begin{proof}
\label{proof:exponential}
Since the objective function consists of a linear term combined with an entropy regularization, which is a strongly concave function, the maximization problem is a convex optimization problem. Owing to the implicit constraints provided by the entropy function, the problem is equivalent to having only the equality constraint. We then introduce the Lagrangian multiplier $\lambda$ and obtain the following relaxed problem:

$$
\widetilde{E}(\boldsymbol{\alpha})=\psi_{1}-\sum_{i=1}^n \alpha_{i} \psi_{i}+\tau \sum_{i=1}^n \alpha_{i}\log \alpha_{i}+\lambda\left(\boldsymbol{\alpha}^{\top} \mathbf{1}_n-1\right).
$$

As the relaxed problem is unconstrained, taking the derivative with respect to $\alpha_{i}$ yields

$$
\frac{\partial \widetilde{E}(\boldsymbol{\alpha})}{\partial \alpha_{i}}=-\psi_{i}+\tau\left(\log \alpha_{i}+\alpha_{i} \frac{1}{\alpha_{i}}\right)+\lambda=0.
$$

Solving the above equation implies that $\alpha_{i}$ takes the form
$
\alpha_{i}=\exp \left(\frac{1}{\tau} \psi_{i}\right) \exp \left(\frac{-\lambda}{\tau}-1\right).
$ Since $\alpha_{i}$ lies on the probability simplex, the optimal $\alpha_{i}$ is explicitly given by
$
\alpha^{*}_{i}=\frac{\exp \left(\frac{1}{\tau} \psi_{i}\right)}{\sum_{i^{\prime}=1}^n \exp \left(\frac{1}{\tau} \psi_{i^{\prime}}\right)} .
$ Substituting the optimal point into the objective function, we obtain
$$
\begin{aligned}
E\left(\boldsymbol{\alpha}^*\right)  &=\psi_1-\sum_{i=1}^n \frac{\exp \left(\frac{1}{\tau} \psi_{i}\right)}{\sum_{i^{\prime}=1}^n \exp \left(\frac{1}{\tau} \psi_{i^{\prime}}\right)} \psi_{i}+\tau \sum_{i=1}^n \frac{\exp \left(\frac{1}{\tau} \psi_{i}\right)}{\sum_{i^{\prime}=1}^n \exp \left(\frac{1}{\tau} \psi_{i^{\prime}}\right)}\log \frac{\exp \left(\frac{1}{\tau} \psi_{i}\right)}{\sum_{i^{\prime}=1}^n \exp \left(\frac{1}{\tau} \psi_{i^{\prime}}\right)} \\
& =\psi_1 - \tau \log \left(\sum_{i=1}^n \exp \left(\frac{1}{\tau} \psi_{i}\right)\right).
\end{aligned}
$$
Thus, the Lagrangian dual function is given by
\begin{equation*}
-E\left(\boldsymbol{\alpha}^*\right)= -\tau \log \frac{\exp \left(\frac{1}{\tau} \psi_{1}\right)}{\sum_{i=1}^n \exp \left(\frac{1}{\tau} \psi_{i}\right)}.\qedhere
\end{equation*}
\end{proof}



\section{More on Experiments} \label{section: experiment_details}

\paragraph{CIFAR-10 and CIFAR-100} CIFAR-10 ~\citep{krizhevsky2009learning} and CIFAR-100 ~\citep{krizhevsky2009learning} are well-known classic image classification datasets. Both CIFAR-10 and CIFAR-100 contain a total of 60k $32 \times 32$ labeled images of different classes, with 50k for training and 10k for testing. CIFAR-10 is similar to CIFAR-100, except there are 10 different classes in CIFAR-10 and 100 classes in CIFAR-100.

\paragraph{TinyImageNet} TinyImageNet ~\citep{le2015tiny} is a subset of ImageNet ~\citep{deng2009imagenet}. There are 200 different object classes in TinyImageNet, with 500 training images, 50 validation images, and 50 test images for each class. All the images in TinyImageNet are colored and labeled with a size of $64 \times 64$.

\textbf{Pseudo-code.} Algorithm \ref{alg:Training Procedure} presents the pseudo-code for our empirical training procedure.

\begin{algorithm}[!htbp]
\caption{Training Procedure}
\label{alg:Training Procedure}
\begin{algorithmic}[1]
\REQUIRE trainable encoder network $f$, batch size $N$, augmentation strategy \textit{aug}, loss function $L$ with hyperparameters \textit{args}
\FOR {sampled minibatch ${x_i}_{i=1}^N$}
\FORALL{$i \in { 1, ..., N }$}
\STATE draw two augmentations $t_i = \textit{aug}\left(x_i\right) $, $t_i' = \textit{aug}\left(x_i\right) $
\STATE $z_i = f\left(t_i\right)$, $z_i' = f\left(t_i'\right)$
\ENDFOR
\STATE compute loss $\mathcal{L} = L(N, z, z', \textit{args})$
\STATE update encoder network $f$ to minimize $\mathcal{L}$
\ENDFOR
\STATE \textbf{Return} encoder network $f$
\end{algorithmic}
\end{algorithm}

We also provide the pseudo-code for our core loss function used in the training procedure in Algorithm \ref{alg:Core loss}. The pseudo-code is almost identical to SimCLR's loss function, with the exception of an extra parameter $\gamma$.

\begin{algorithm}[!htbp]
\caption{Core loss function $\mathcal{C}$}
\label{alg:Core loss}
\begin{algorithmic}[1]
\REQUIRE batch size $N$, two encoded minibatches $z_1, z_2$, $\gamma$, temperature $\tau$
\STATE $z = \textit{concat}\left(z_1, z_2\right)$
\FOR {$i \in {1, ..., 2N }, j \in {1, ..., 2N}$ }
\STATE $s_{i,j} = \Vert z_i - z_j \Vert_2^{\gamma}$
\ENDFOR
\STATE \textbf{define} $l(i, j)$ \textbf{as} $l(i, j) = - \log \frac{exp\left(s_{i,j}/\tau \right)}{\sum_{k=1}^{2N} \mathbf{1}{[k \ne i]} exp\left(s{i, j} / \tau \right)} $
\STATE \textbf{Return} $\frac{1}{2N} \sum_{k=1}^N\left[l(i, i+N) + l(i+N, i)\right]$
\end{algorithmic}
\end{algorithm}

Utilizing the core loss function $\mathcal{C}$, we can define all kernel loss functions used in our experiments in Table \ref{table: loss definition}. For all $z_i \in z$ with even dimensions $n$, we define $z_{L_i} = z_i\left[0:n/2\right]$ and $z_{R_i} = z_i\left[n/2:n\right]$.

\begin{table}[ht]
\centering
\begin{tabular}{{@{}l|l@{}}}
Kernel  &  Loss function \\ \midrule
Laplacian & $\mathcal{C}\left(N, z, z', \gamma=1, \tau\right)$\\ \midrule
Sum       & $\lambda * \mathcal{C}\left(N, z, z', \gamma=1, \tau_1\right) + (1-\lambda) * \mathcal{C}\left(N, z, z', \gamma=2, \tau_2\right)$  \\ \midrule
Concatenation Sum&$\lambda * \mathcal{C}\left(N, z_L, z'_L, \gamma=1, \tau_1\right) + (1-\lambda) * \mathcal{C}\left(N, z_R, z'_R, \gamma=2, \tau_2\right)$\\ \midrule
$\gamma = 0.5$ & $\mathcal{C}\left(N, z, z', \gamma=0.5, \tau\right)$          \\ 

\end{tabular}

\caption{Definition of kernel loss functions in our experiments}
\label {table: loss definition}
\end{table}

\textbf{Baselines.} We reproduce the SimCLR algorithm using PyTorch Lightning~\citep{PytorchLightning}.

\textbf{Encoder details.}
The encoder $f$ consists of a backbone network and a projection network. We employ ResNet50~\citep{ResNet} as the backbone and a 2-layer MLP (connected by a batch normalization~\citep{ioffe2015batch} layer and a ReLU \cite{nair2010rectified} layer) with hidden dimensions 2048 and output dimensions 128 (or 256 in the concatenation kernel case).

\textbf{Encoder hyperparameter tuning.}
For each encoder training case, we randomly sample 500 hyperparameter groups (sample details are shown in Table \ref{table: Hyperparameter sample}) and train these samples simultaneously using Ray Tune ~\citep{RayTune}, with the ASHA scheduler~\citep{li2018massively}. Ultimately, the hyperparameter group that maximizes the online validation accuracy (integrated in PyTorch Lightning) within 5000 validation steps is chosen for the given encoder training case.

\begin{table}[ht]
\centering

\begin{tabular}{@{}l|l|l@{}}
\midrule
Hyperparameter  & Sample Range & Sample Strategy \\ \midrule
start learning rate & $\left[10^{-2}, 10\right]$ & log uniform \\ \midrule
$\lambda$       & $\left[0, 1\right]$ & uniform \\ \midrule
$\tau$, $\tau_1$, $\tau_2$ & $\left[0, 1\right]$ & log uniform \\ \midrule
\end{tabular}

\caption{Hyperparameters sample strategy}
\label {table: Hyperparameter sample}
\end{table}

\textbf{Encoder training.} 
We train each encoder using the LARS optimizer~\citep{LARSOptimizer}, LambdaLR Scheduler in PyTorch, momentum 0.9, weight decay $10^{-6}$, batch size 256, and the aforementioned hyperparameters for 400 epochs on a single A-100 GPU.

\textbf{Image transformation.} The image transformation strategy, including augmentation, is identical to the default transformation strategy provided by PyTorch Lightning.

\textbf{Linear evaluation.}
The linear head is trained using the SGD optimizer with a cosine learning rate scheduler, batch size 64, and weight decay $10^{-6}$ for 100 epochs. The learning rate starts at $0.3$ and ends at $0$.

\textbf{Moco Experiments.} We also tested our method based on MoCo~\citep{he2019moco}. The results are summarized in Table \ref{tab:results-moco}. Here we choose ResNet18~\citep{ResNet} as the backbone and set a temperature of $0.1$ as default. For our simple sum kernel, we set $\lambda=0.8$. The results show that our method outperforms the original MoCo method.

\begin{table}[thb]
\centering
\caption{MoCo Experiment Results on CIFAR-10 and CIFAR-100.}
\label{tab:results-moco}
\resizebox{\textwidth}{!}{%
\begin{tabular}{@{}c|ccc|ccc@{}}
\toprule
\multirow{3}{*}{Method} & \multicolumn{3}{c|}{CIFAR-10} & \multicolumn{3}{c}{CIFAR-100} \\ \cmidrule(lr){2-4} \cmidrule(lr){5-7} 
                        & 200 epochs & 400 epochs    & 1000 epochs   & 200 epochs & 400 epochs & 1000 epochs         \\ \midrule
MoCo (repro.)         & $76.41 \pm 0.12$    & $80.01 \pm 0.15$          & $84.45 \pm 0.08$    & $\mathbf{47.02 \pm 0.11}$ & $52.50 \pm 0.07$ & $57.62 \pm 0.15$            \\
\midrule
Laplacian Kernel        & ${78.09 \pm 0.10}$    & $\mathbf{83.85 \pm 0.09}$          & $\mathbf{88.34 \pm 0.16}$    & $46.12 \pm 0.22$   & $53.44 \pm 0.17$ & $59.10 \pm 0.14$        \\
Simple Sum Kernel & $\mathbf{78.12 \pm 0.15}$   & $83.23 \pm 0.18$ & $87.50 \pm 0.20$ & $46.65 \pm 0.06$ & $\mathbf{53.62 \pm 0.19}$ & $\mathbf{59.83 \pm 0.12}$\\
\bottomrule
\end{tabular}
}
\end{table}



\section{More Experiments on Synthetic Data}


Consider a scenario with $n$ clusters, each containing $k$ vertices. Let the probability of vertices $u$ and $v$ from the same cluster belonging to $\bfpi$ be $p$. Conversely, for vertices $u$ and $v$ from different clusters, let the probability of belonging to $\pi$ be $q$. We generate the graph $\bfpi$ randomly, based on $p$ and $q$. We experiment with values of $k=100$ and $n=6$ for ease of visualization, embedding all points in a two-dimensional space. Each vertex's initial position originates from a normal distribution. In each iteration, we sample a subgraph of $\bfpi$ uniformly, ensuring each vertex has an out-degree of $1$. We then optimize the corresponding vectors using InfoNCE loss with an SGD optimizer and iterate until convergence. Our experimental setup consists of an SGD learning rate of $1$, an InfoNCE loss temperature of $0.5$, and a batch size of $50$. We evaluate two scenarios with different $p$ and $q$ values: $p=1$, $q=0$, and $p=0.75$, $q=0.2$. The results of these experiments are visualized in Figure \ref{fig:vis-spectral-cluster}. The obtained embeddings exhibit the hallmark pattern of spectral clustering of graph $\bfpi$.

\begin{figure}[!tb]
\centering
\subfigure{
\includegraphics[width=1\textwidth]{Figures/cluster_pi.png}
\label{fig:vis-cluster}
}
\subfigure{
\includegraphics[width=1\textwidth]{Figures/noised_cluster_pi.png}
\label{fig:vis-noised-cluster}
}
\caption{Visualizations of the optimization process using InfoNCE Loss on the vectors corresponding to $\bfpi$. Points of identical color belong to the same cluster within $\bfpi$. To showcase the internal structure of $\bfpi$, we randomly select 10 vertices from each cluster to display the edge distribution of $\bfpi$.}
\label{fig:vis-spectral-cluster}
\end{figure}






%If you came here because you want your references in a new page, uncomment the following line

\clearpage % If you want the references in a separate page
\bibliography{bibliography}
%\subsection{}
\begin{lemma}\label{PomocnaLemma1}
Let $n\in\mathbb{N}$ and let $\mathcal{S}\subseteq \mathbb{R}$ contain an open interval. Let $f_1, \dots, f_n, g_1, \dots, g_n$ be non-constant continuous real functions on $\mathcal{S}\subseteq \mathbb{R}$, such that 
$
f_1(x)g_1(y) + \dots + f_n(x)g_n(y)
$ is additive in $x,y$, that is, there exist functions $f,g$ such that 
$$
f_1(x)g_1(y) + \dots + f_n(x)g_n(y) = f(x) + g(y), \forall x,y\in\mathcal{S}.
$$
Then, there exist (not all zero) constants $a_1, \dots, a_n, c\in\mathbb{R}$ such that 
$\sum_{i=1}^n a_if_i(x) = c$ for all $x\in\mathcal{S}$. Specifically for $n=2$ holds $f_1(x) = af_2(x)+c$ for some $a,c\in\mathbb{R}$. 

Moreover, assume that for some $q<n$ holds that $g_1, \dots, g_q$ are linearly independent in a sense that there exist $y_1, \dots, y_q\in\mathcal{S}$ such that a matrix 
\begin{equation}
M:=\begin{pmatrix}
 g_1(y_1) & \cdots & g_q(y_1) \\
\cdots & \cdots & \cdots \\
g_1(y_q) & \cdots & g_q(y_{q}) 
\end{pmatrix} 
\end{equation}
has full rank. Then, for all $i=1, \dots, q$ there exist constants $a_{q+1}, \dots, a_n, c\in\mathbb{R}$ such that $f_i(x)=\sum_{j=q+1}^n a_jf_j(x) +c$ for all $x\in\mathcal{S}$. 
\end{lemma}
\begin{proof}
Fix $y_1, y_2\in\mathcal{X}$ such that $y_1\neq y_2$. Then, we have for all $x\in\mathcal{S}$
\begin{align*}
&f_1(x)g_1(y_1) + \dots + f_n(x)g_n(y_1) = f(x) + g(y_1),\\&
f_1(x)g_1(y_2) + \dots + f_n(x)g_n(y_2) = f(x) + g(y_2),
\end{align*}
and subtraction of these equalities gives us 
$$
f_1(x)[g_1(y_1)- g_1(y_2)] + \dots + f_n(x)[g_n(y_1)-g_n(y_2)] = g(y_1) - g(y_2).
$$
Defining $a_i = g_i(y_1)- g_i(y_2)$ and $c = g(y_1)- g(y_2)$ gives us the first result.

Now, we prove the "Moreover" part. Find $y_0\in\mathcal{S}$ such that a matrix 
$Q:= M-(1, \dots, 1)^\top(g_1(y_0), \dots, g_q(y_0))$ has also full rank. This is possible from the assumption that $\mathcal{S}$ contains an open interval and $g$ are continuous. Now, consider equalities 
\begin{align*}
&f_1(x)g_1(y_0) + \dots + f_n(x)g_n(y_0) = f(x) + g(y_0),\\&
f_1(x)g_1(y_1) + \dots + f_n(x)g_n(y_1) = f(x) + g(y_1),\\&
\dots\\&
f_1(x)g_1(y_q) + \dots + f_n(x)g_n(y_q) = f(x) + g(y_q),
\end{align*}
where $y_1, \dots, y_q$ are defined in the lemma. Subtracting from each equality the first one gives us 
\begin{align*}
&f_1(x)[g_1(y_1)- g_1(y_0)] + \dots + f_n(x)[g_n(y_1)-g_n(y_0)] = g(y_1) - g(y_0)\\&
\dots \\&
f_1(x)[g_1(y_q)- g_1(y_0)] + \dots + f_n(x)[g_n(y_q)-g_n(y_0)] = g(y_q) - g(y_0).
\end{align*}
Using matrix formulation, this can be rewritten as 
\begin{equation}
Q\begin{pmatrix}
f_1(x) \\
\cdots \\
f_q(x) 
\end{pmatrix} =
\begin{pmatrix}
g(y_1)-g(y_0) -\sum_{j=q+1}^n f_{j}(x)[g_{j}(y_1) -g_{j}(y_0)] \\
\cdots \\
g(y_q)-g(y_0) -\sum_{j=q+1}^n f_{j}(x)[g_{j}(y_q) -g_{j}(y_0)]
\end{pmatrix} .
\end{equation}
Multiplying both sides by $Q^{-1}$ gives us that $f_1(x)$ is a linear combination of $f_{q+1}(x), \dots, f_n(x)$, what we wanted to show (as well as $f_i(x)$ for $i=1, \dots, q$)
\end{proof}


\begin{customthm}{\ref{thmAssymetricMultivariatesufficient}}
Let $(X_1, X_2)$ follow an asymmetrical $CPCM(F_1, F_2)$ defined in (\ref{asymetrical_F_one_F_two_model}), where $F_1, F_2$ lie in an exponential family of continuous distributions and $T_1(\cdot) = (T_{1,1}(\cdot), \dots, T_{1,q_1}(\cdot))^\top$, $T_2(\cdot) = (T_{2,1}(\cdot), \dots, T_{2,q_2}(\cdot))^\top$are the corresponding sufficient statistics with nontrivial intersection of their support $\mathcal{S}:=supp(F_1)\cap supp(F_2)$.

The causal graph is identifiable, if $\theta_2$ is not a linear combination of $T_{1,1}, \dots, T_{1,q_1}$ on $\mathcal{S}$. That is, if $\theta_2$ can not be written as
\begin{equation}\tag{\ref{eq158}}
\theta_{2,i}(x) \overset{}{=} \sum_{j=1}^{q_1}a_{i,j}T_{1,j}(x)+b_i,\,\,\,\,\,\,\,\,\,\,\,\,\,\,\, \forall x\in \mathcal{S},
\end{equation}
for all $i=1, \dots, q_2,$ where $a_{i,j},b_i\in\mathbb{R}$, $j=1, \dots, q_1$ are constants. 
\end{customthm}



\begin{proof}
\label{Proof of thmAssymetricMultivariatesufficient}{}
If the asymmetrical $CPCM(F_1,F_2)$ is \textit{not} identifiable, then functions $\theta_1, \theta_2$ satisfy that random variables from $X_1 = \varepsilon_1, X_2 = F_2^{-1}(\varepsilon_2, \theta_2(X_1))$ and from $X_2 = \varepsilon_2, X_1 = F_1^{-1}(\varepsilon_1, \theta_1(X_2))$ have the same joint density function. Write the joint density as
\begin{equation}\label{eq59}
  p_{X_1, X_2}(x,y) = p_{X_1}(x)p_{X_2\mid {X_1}}(y\mid x) = p_{X_2}(y)p_{{X_1}\mid {X_2}}(x\mid y).
 \end{equation}
Since $F_1, F_2$ lie in the exponential family of distributions, we use the notation from \hyperref[appendix]{Appendix} \ref{appendix_exponential_family} and rewrite (\ref{eq59}) as follows
\begin{equation*}
    \begin{split}
  &     p_{{X_2}\mid {X_1}}(y\mid x) = h_{1,1}(y)h_{1,2}[\theta_2(x)]\exp[\sum_{i=1}^{q_2}\theta_{2,i}(x)T_{2,i}(y)],\\&
  p_{{X_1}\mid {X_2}}(x\mid y) = h_{2,1}(x)h_{2,2}[{\theta_1}(y)]\exp[\sum_{i=1}^{q_1}{\theta}_{1,i}(y)T_{1,i}(x)].  
    \end{split}
\end{equation*}
Now, after a logarithmic transformation of both sides of (\ref{eq59}), we obtain 
\begin{equation}\label{eq254}
\begin{split}
\log[p(x,y)] &= \log[p_{X_1}(x)] +  \log[h_{1,1}(y)]+\log\{h_{1,2}[\theta_{2}(x)]\} + \sum_{i=1}^{q_2}\theta_{2,i}(x)T_{2,i}(y) \\&
= \log[p_{X_2}(y)] +  \log[h_{2,1}(x)]+\log\{h_{2,2}[\theta_{1}(y)]\} + \sum_{i=1}^{q_1}\theta_{1,i}(y)T_{1,i}(x).
\end{split}
\end{equation}
Define $f(x) = \log[p_{X_1}(x)] +\log\{h_{1,2}[\theta_{2}(x)]\} -\log[h_{2,1}(x)]$ and $g(y) =\log[h_{1,1}(y)] -  \log[p_{X_2}(y)] + \log\{h_{2,2}[\theta_{1}(y)]\}$. Then, equality (\ref{eq254}) reads as 
\begin{equation}\label{eq9876}
f(x) + g(y) = \sum_{i=1}^{q_1}\theta_{1,i}(y)T_{1,i}(x) - \sum_{i=1}^{q_2}\theta_{2,i}(x)T_{2,i}(y).
\end{equation}
Now we use Lemma \ref{PomocnaLemma1}. We know that functions $T_{2,i}$ are linearly independent in the sense presented in Lemma  \ref{PomocnaLemma1}, see Observation \ref{observationFullRank} in \hyperref[appendix]{Appendix} \ref{appendix_exponential_family}. Therefore, Lemma \ref{PomocnaLemma1} gives us that $\theta_{2,i}$ are only a linear combination of $T_{1, i}$. That is what we wanted to show. 
\end{proof}


\end{document}
