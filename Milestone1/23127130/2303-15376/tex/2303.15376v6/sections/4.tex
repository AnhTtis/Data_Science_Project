
\section{Multivariate case $d\geq 2$}
\label{Section4}

We extend the theory to the case with possibly more than two variables, $\textbf{X} = (X_1, \dots, X_d)^\top$.  

\begin{definition}\label{DefinitionCPCM}
Let \(\{F_1, \dots, F_k\}\) be a collection of distribution functions with \(q_1, \dots, q_k\) parameters, respectively, where  \(q_1, \dots, q_k\in\mathbb{N}\). We define a conditionally parametric causal model \(CPCM(F_1, \dots, F_k)\) with an underlying DAG \(\mathcal{G}\) as a collection of equations:
\begin{equation*}
  X_j = \begin{cases}
    \varepsilon_j, & \text{if } j \in \text{Source}(\mathcal{G}), \\
    F_{\pi(j)}^{-1}\big(\varepsilon_j; \theta_j(\mathbf{X}_{pa_j})\big), \text{where } \pi(j) \in \{1, \dots,k\},\,\,\,\,\,\,\, & \text{if } j \notin \text{Source}(\mathcal{G}),
  \end{cases}
\end{equation*}
where \((\varepsilon_1, \dots, \varepsilon_d)^\top\) is a collection of jointly independent random variables with \(\varepsilon_j \sim \text{U}(0, 1)\) for all \(j \notin \text{Source}(\mathcal{G})\), and \(\theta_j\) are non-constant functions in any of their arguments.
\end{definition}
Simply said, we assume that $X_j\mid \textbf{X}_{pa_j}$ is distributed according to distribution $F_{\pi(j)}$ with parameters $\theta_j(\textbf{X}_{pa_j})$, where $F_{\pi(j)}$ is either $F_1, F_2,  \dots$, or  $F_k$. 
Although we implicitly assume causal minimality \citep{zhang2010intervention}, we do not require the stronger assumption of faithfulness \citep{uhler2013geometry}. 

The question of the identifiability of $\mathcal{G}$ in the multivariate case is in order. Here, it is not satisfactory to consider the identifiability of each pair of $X_i\to X_j$ separately. Each pair $X_i, X_j$  needs to have an identifiable causal relation \textit{conditioned} on other variables $\textbf{X}_S$. Such an observation was first made by \cite{Peters2014} in the context of additive noise models. We now provide a more precise statement in the context of  $CPCM(F_1, \dots, F_k)$. 

\begin{definition} 
We say that the $CPCM(F_1, \dots, F_k)$ is \textit{pairwise identifiable}, if for all $ i,j\in V$, $ S\subseteq V$, such that $i\in pa_j$ and  $pa_j\setminus \{i\}\subseteq S \subseteq nd_j\setminus\{i,j\}$, there exists  $\textbf{x}_{S}{:}\,\,  p_S(\textbf{x}_S)>0$, which satisfies that a bivariate model defined as $X=\tilde{\varepsilon}_X, Y = F^{-1}_j\big(\tilde{\varepsilon}_Y, \tilde{\theta}(X)\big)$ is identifiable (in the sense of Definition \ref{DEFidentifiability}), where  $F_{\tilde{\varepsilon}_X} = F_{X_i\mid \textbf{X}_{S} =\textbf{ x}_S}    $ and $\tilde{\theta}(x) = \theta_j(\textbf{x}_{pa_j\setminus\{i\}}, x)$,  $x\in supp(X)$.
\end{definition}

 \begin{lemma}\label{thmMultivairateIdentifiability}
Let $F_{\textbf{X}}$ be generated by the pairwise identifiable $CPCM(F_1, \dots, F_k)$ with DAG $\mathcal{G}$. Then,  $\mathcal{G}$ is identifiable from the joint distribution. 
 \end{lemma}
The proof follows as a consequence of Theorem 28 in \cite{Peters2014} and is provided in \hyperref[Proof of thmMultivairateIdentifiability]{Appendix} \ref{Proof of thmMultivairateIdentifiability}.

\begin{consequence}[Multivariate Gaussian case]\label{ExampleMultivariateGaussiancase}
Suppose that $\textbf{X}=(X_1, \dots, X_d)$ follow $CPCM(F)$ with a Gaussian distribution function $F$. This corresponds to $X_j\mid \textbf{X}_{pa_j}\sim N\big(\mu_j(\textbf{X}_{pa_j}), \sigma_j^2(\textbf{X}_{pa_j})\big)$ for all $j=1, \dots, d$ and for some functions $\mu_j, \sigma_j$. In other words, we assume that the data-generation process has the following form: 
$$
X_j = \mu_j(\textbf{X}_{pa_j}) + \sigma_j(\textbf{X}_{pa_j})\,\varepsilon_j, \,\,\,\,\,\,\,\,\,\,\,\,\,\,\,\,\, \text{where  } \varepsilon_j \text{  is Gaussian.}
$$
Potentially, source nodes can have arbitrary distributions. Combining Theorem~\ref{normalidentifiability} and Lemma~\ref{thmMultivairateIdentifiability}, the causal graph $\mathcal{G}$ is identifiable if the 
functions $\theta_j(\textbf{x}):=\big(\mu_j(\textbf{x}), \sigma_j(\textbf{x})\big)^\top, \textbf{x}\in\mathbb{R}^{|pa_j(\mathcal{G})|}$ , $j=1, \dots, d$, are \textit{not} in the form (\ref{norm}) in any of their arguments. 
\end{consequence}


















