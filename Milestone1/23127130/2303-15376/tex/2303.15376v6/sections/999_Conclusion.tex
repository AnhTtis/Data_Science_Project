\section{Conclusion and future research}


We introduced a new family of models for causal inference called Conditionally Parametric Causal Models (CPCM), designed to flexibly accommodate a broad range of variable types and distributional forms. Our primary theoretical contributions lie in establishing the identifiability conditions for the causal structure within this framework. Specifically, we have demonstrated that the bivariate $CPCM(F)$ models are identifiable, with exceptions arising only when the parameters of $F$ take the form of a linear combination of its sufficient statistics. Furthermore, we have provided detailed characterizations of identifiability across various cases such as Gaussian, Poisson, and Pareto, significantly broadening the scope of identifiable models beyond existing literature. We also explained the multivariate extensions of these results. 

We complement these results with two consistent estimation algorithms for CPCM-based causal graph recovery. Experiments show competitive performance in Gaussian location–scale models, while retaining the ability to operate in much broader distributional settings, including heavy-tailed, continuous, discrete or even a mixture of these. 

CPCM also connects naturally to invariant causal prediction (\cite{Peters_invariance, Kook03042025}), offering promising directions for distribution-aware causal feature selection, as the framework of target-variable causal modelling provides a natural environment for embedding the CPCM ideology \citep{Bodik_biometrika}. Extensions to uncertainty quantification under distribution shift \citep{liu2021learning} or time series settings \citep{Bodik}  would further broaden its applicability. Integrating these ideas and validating CPCM in diverse applied domains are key avenues for future work.





\section*{Conflict of interest and data availability}
The open-source implementation of the methods discussed in this manuscript together with the data used can be found in the supplementary package or at \url{https://github.com/jurobodik/Causal_CPCM.git}.

The authors declare that they have no known competing financial interests or personal relationships that could have appeared to influence the work reported in this paper.


\section*{Acknowledgments}
This study was supported by the Swiss National Science Foundation under grant number 201126. 

































































