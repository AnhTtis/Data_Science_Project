\renewcommand\thesection{B}
\section{Experiments details and additional plots}
\label{Appendix_simulations}

\subsection{Exact, Naive-greedy, RESIT, and RESIT-greedy algorithms: definitions and comparison}
\label{appendix_greedy_definitions}
We consider the following algorithms for estimating the underlying causal graph from observational data using the CPCM score function~\eqref{score_definition1}:

\begin{itemize}
    \item \textbf{Exact search:} This algorithm evaluates the CPCM score for all DAGs on $d$ nodes and selects the one with the lowest score. Since the number of DAGs grows super-exponentially with $d$ (e.g., 29,281 DAGs for $d = 5$), exact search is computationally feasible only for very small graphs with $d \leq 4$.
    
    \item \textbf{Naive-edge-greedy:} Starting from an empty DAG, this algorithm iteratively explores neighboring DAGs by adding or removing a single edge. At each step, it selects the neighboring graph with the lowest CPCM score and replaces the current graph if the score improves. The procedure stops when no further improvement is possible. While simple and scalable, this greedy approach lacks theoretical guarantees and may get stuck in local minima, unless we assume some advanced notions of convexity over the space of all DAGs.

    \item \textbf{RESIT (Regression with Subsequent Independence Test):} RESIT first estimates a topological ordering by iteratively selecting the variable whose residual is least dependent on the remaining variables. In the second phase, it removes superfluous edges using conditional independence tests. See Algorithm~\ref{alg:resit} for details. The procedure is computationally efficient and comes with statistical guarantees (see Lemma~\ref{thm:resit_consistency}). However, empirical performance tends to be worse than that of greedy algorithms, particularly due to the accumulation of errors in the ordering phase and false positives (type I errors) in the Phase~2. 

    \item \textbf{RESIT-greedy:} This hybrid algorithm combines the topological ordering phase of RESIT with the edge-pruning phase of naive-greedy search. After estimating the ordering, it starts with a fully connected DAG consistent with the order and iteratively removes edges that lead to the largest improvement in the CPCM score, until no further improvement is possible. See Algorithm~\ref{alg:resit_greedy}.
\end{itemize}
\citet{Peters2014} showed that, in the population case, the RESIT algorithm is consistent under identifiable additive noise models, assuming a consistent nonparametric regression method and a perfect independence oracle. The same reasoning applies directly to CPCM models.


\begin{lemma}[Consistency of RESIT under CPCM]
\label{thm:resit_consistency}
Let $\mathbf{X}$ be generated by a $CPCM(F_1, \dots, F_k)$ model with underlying DAG $\mathcal{G}_0$. Then, the RESIT algorithm, when applied with consistent estimators $\hat{\theta}_k$ and an independence oracle, is guaranteed to recover the true graph $\mathcal{G}_0$ from the distribution of $\mathbf{X}$.
\end{lemma}

\begin{proof}
A direct consequence of Theorem~34 in \citet{Peters2014}. Note that we implicitly assume the causal minimality condition for $CPCM(F_1, \dots, F_k)$ as stated in Definition~\ref{DefinitionCPCM}.
\end{proof}

\begin{algorithm}[]
\caption{Regression with Subsequent Independence Test (RESIT)}
\label{alg:resit}
\KwIn{I.i.d. samples of a $d$-dimensional distribution on $(X_1, \ldots, X_d)$}
$S \gets \{1, \ldots, d\}; \pi \gets [\ ]$\;

\textbf{Phase 1: Determine topological order $\pi$} \;
\While{$S \ne \emptyset$}{
  \ForEach{$k \in S$}{
    Compute residuals $\hat{\varepsilon}_k := F\big(X_k; \hat{\theta}_k(\mathbf{X}_{S \setminus \{k\}})\big)$\;
    Measure dependence between $\hat{\varepsilon}_k$ and $\{X_i\}_{i \in S \setminus \{k\}}$\;
  }
  Let $k^* \gets$ variable with weakest dependence\;
  $S \gets S \setminus \{k^*\}$\;
  $\mathrm{pa}(k^*) \gets S$\;
  Prepend $k^*$ to $\pi$\;
}

\textbf{Phase 2: Remove superfluous edges} \;
\For{$k = 2$ \KwTo $d$}{
  \ForEach{$\ell \in \mathrm{pa}(\pi_k)$}{
    Compute residuals:
    $\hat{\varepsilon}_{\pi_k} := F\big(X_{\pi_k}; \hat{\theta}_{\pi_k}(\mathbf{X}_{\mathrm{pa}(\pi_k) \setminus \{\ell\}})\big)$\;
    \If{$\hat{\varepsilon}_{\pi_k}$ is independent of $\{X_{\pi_1}, \ldots, X_{\pi_{k-1}}\}$}{
      $\mathrm{pa}(\pi_k) \gets \mathrm{pa}(\pi_k) \setminus \{\ell\}$\;
    }
  }
}
\KwOut{$(\mathrm{pa}(1), \ldots, \mathrm{pa}(d))$}
\end{algorithm}





\begin{algorithm}[]
\caption{RESIT-greedy}
\label{alg:resit_greedy}
\KwIn{I.i.d. samples of a $d$-dimensional distribution on $(X_1, \ldots, X_d)$}

\textbf{Phase 1: Determine topological order $\pi$} \;
As in RESIT (Algorithm~\ref{alg:resit})\;

\textbf{Phase 2: Greedy removal of edges using CPCM score} \;
Initialize graph $\mathcal{G}$ with all edges $(j \to i)$ such that $j \in \mathrm{pa}(i)$ and $j$ precedes $i$ in $\pi$\;

\Repeat{\textbf{no score improvement}}{
  $\mathcal{G}_{\text{best}} \gets \mathcal{G}$\;
  $S_{\text{best}} \gets \text{CPCM}(\mathcal{G})$\;

  \ForEach{edge $e = (j \to i)$ in $\mathcal{G}$}{
    $\mathcal{G}' \gets \mathcal{G}$ with $e$ removed\;
    $S' \gets \text{Score of }\mathcal{G} \text{ in }\text{CPCM}$ \eqref{score_definition1}\;

    \If{$S' < S_{\text{best}}$}{
      $\mathcal{G}_{\text{best}} \gets \mathcal{G}'$\;
      $S_{\text{best}} \gets S'$\;
    }
  }

  \If{$\mathcal{G}_{\text{best}} \ne \mathcal{G}$}{
    $\mathcal{G} \gets \mathcal{G}_{\text{best}}$\;
  } \Else{
    \textbf{break}\;
  }
}
\KwOut{$\mathcal{G}$}
\end{algorithm}



\subsubsection{Experiments: comparison of different greedy algorithms}
\label{appendix_greedy}
\textbf{Data-generating process:} We generate  random DAGs uniformly over $d$ nodes with $p$ edges, where $p \sim \mathrm{Exp}(1/d)$ and capped at $\frac{d(d-1)}{2}$. On average, each DAG contains approximately $d$ edges. For a given DAG $\mathcal{G}$, we simulate data from the $CPCM(F)$ model, $F=Exponential$, defined as:
$$
X_i \sim \mathrm{Exp}\left(\lambda(\mathbf{X}_{pa_i(\mathcal{G})})\right), \quad \text{with} \quad \lambda(\mathbf{X}_{pa_i(\mathcal{G})}) = \frac{1}{\sum_{j \in pa_i(\mathcal{G})} |X_j|} = \frac{1}{\mathbb{E}[X_i\mid \textbf{X}_{pa_i}]}.
$$
If $pa_i(\mathcal{G})=\emptyset$, then $X_i\sim N(0,1)$. Using $n = 1000$ samples, we estimate $\mathcal{G}$ using the score-based CPCM estimator defined in Equation~\eqref{score_definition1}, with a fixed function class $\mathscr{S}_1$ (rather than the sequential approach) to allow for a fair comparison in both accuracy and computational cost. We compare the Exact, Naive-greedy, RESIT, and RESIT-greedy methods.  We evaluate their performance using the Structural Intervention Distance (SID, \cite{peters2014structuralinterventiondistancesid}), computing $\mathrm{SID}(\mathcal{G}, \hat{\mathcal{G}})$ for each estimated graph.

\textbf{Results:} Figure~\ref{figure_greedy} presents the average normalized $\frac{SID}{d}$ over 50 repetitions.

\begin{itemize}
    \item The exact method achieves, unsurprisingly, the lowest SID, with the highest computational cost. 
    \item Both RESIT and naive-greedy exhibit similar performance. The greedy approach achieves slightly lower SID on average but requires slightly more computation time. Note that for $d\geq 8$, both methods perform as badly as Trivial algorithm (empty graph).
    \item RESIT-greedy serves as a middle ground: it significantly improves over RESIT/naive-greedy methods in terms of SID while incurring higher computational cost for $d > 5$.
\end{itemize}


These experiments highlight the trade-offs between statistical performance and computational efficiency among the evaluated methods. While the exact method yields the most accurate graph recovery, its scalability is limited. In contrast, RESIT and naive-greedy offer faster but less accurate alternatives, with performance deteriorating as graph complexity increases. RESIT-greedy provides a promising compromise, achieving lower SID than standard greedy methods at a moderate computational cost. \textbf{Overall, RESIT-greedy seems to be a practical choice in graphs with size} $4<d<10$.

\begin{figure}[]
\centering
\includegraphics[scale=0.62]{figures/greedy_comp.pdf}
\caption{Performance of different greedy algorithms from Section~\ref{appendix_greedy}. Here, the trivial algorithm returns an empty graph. Runtime was measured on a machine with an Intel Core i5-6300U 2.5 GHz processor and 16 GB of RAM.}
\label{figure_greedy}
\end{figure}

\subsubsection{Statistical scalability: sample size needs to grow with increasing dimension}
\label{appendix_scalability}
Conducting independence tests and performing nonparametric regression becomes increasingly challenging as the number of covariates grows, particularly in high-dimensional settings, where such tasks demand substantially larger sample sizes. Similar limitations affect several existing algorithms, including ANM-RESIT, LOCI, and bQCD. As demonstrated below, the sample size required for consistent estimation increases systematically with the dimension~$d$.

\textbf{Data-generating process: }For each sample size \( n \) and dimension \( d \), we generated a uniformly random graph with \( d \) nodes and \( d-1 \) edges. The data was then generated according to the structural equation model \( X_i = f_i(\textbf{X}_{pa_i}) + \varepsilon_i \), where \( \varepsilon_i \sim N(0,1) \) and \( f_i(x) = c_i x^{2} \) with \( c_i \sim \text{Unif}(0.5, 1.5) \). 

\textbf{Results:}
We estimate the DAG using the $CPCM(F)$ algorithm, where $F$ is set to a Gaussian distribution with fixed variance, and apply the naive-greedy algorithm for structure learning. The accuracy of the estimated DAG is assessed using the Structural Intervention Distance (SID). This procedure is repeated 100 times, and we report the average SID. Figure~\ref{scalability} displays the ratio of the average SID from the CPCM(F) algorithm to the average SID of a baseline method that generates a random DAG.




\begin{figure}[]
\centering
\includegraphics[scale=0.6]{figures/scalability.pdf}
\caption{Ratio of the computed SID using the $CPCM(F)$ algorithm (edge-greedy version) to the SID of an algorithm producing a random graph, averaged over 100 repetitions for different values of \( n \) and \( d \). Lower SID fraction means better performance of the $CPCM(F)$ algorithm. Values around $SID\,fraction=1$ mean that our algorithm is no better than an algorithm producing a random graph.}
\label{scalability}
\end{figure}



\subsection{Sequential approach vs oracle: empirical performance}
\label{Simulations_seq_approach}
We evaluate the empirical performance of the sequential approach for selecting the distribution family, comparing it to $CPCM(F)$ with access to the true (oracle) distribution $F$. The data is generated as follows:
$$
X_1 \sim \mathcal{N}(0,1), \quad X_2 = F^{-1}(\varepsilon, \theta(X)), \quad \varepsilon \indep X, \; \varepsilon \sim \mathcal{U}(0,1),
$$
where $F \in \mathscr{S}_s$ belongs to either the one-parameter family $\mathscr{S}_1$ or the two-parameter family~$\mathscr{S}_2$.

When $s = 1$, $F$ is, with equal probability ($1/4$), either a Gaussian distribution with fixed variance, a Poisson, Pareto, or Exponential distribution. The parameter is generated as a random function of $X$ via a randomly drawn polynomial. When $s = 2$, $F$ is, again with equal probability, a Gaussian, Negative Binomial, Generalized Pareto, or Gamma distribution, with both parameters generated as random functions of $X$ using random polynomials.

We estimate the graph $\mathcal{G} = {1 \to 2}$ using both $CPCM(\text{Seq.app})$ and $CPCM(F)$ with oracle knowledge of $F$, for various sample sizes $n$, repeating each experiment 100 times for both $s = 1$ and $s = 2$. Figure~\ref{Fig_seq} summarizes the results across sample sizes.

For $s = 1$, we observe that $CPCM(\text{Seq.app})$ typically performs equivalently to oracle $CPCM(F)$ for $n > 100$. For $s = 2$, the sequential approach tends to select the simpler class $\mathscr{S}_1$ instead of the true two-parameter family $\mathscr{S}_2$ at smaller sample sizes. Nevertheless, the performance gap between the sequential approach and the oracle method remains almost negligible.

These results indicate that the sequential approach is performing nearly as well as the oracle $CPCM(F)$, provided that $F \in \mathscr{S}_1 \cup \mathscr{S}_2$.

\begin{figure}[]
\centering
\includegraphics[scale=0.7]{figures/Sequ.appr.pdf}
\caption{Performance of the sequential approach. Left: percentage of simulations where $\hat{\mathcal{G}} = {1 \rightarrow 2}$. Right: percentage of simulations in which $CPCM(\text{Seq.app})$ and $CPCM(F)$ with oracle $F$ are equivalent. }
\label{Fig_seq}
\end{figure}








\subsection{Details about sections~\ref{Section_simulations_Pareto}, \ref{Section_simulations_Gaussian}, \ref{Section_simulations_multivariate} and \ref{Section7}}

\label{Appendix_Section_simulations_Pareto}
The additional plots corresponding to \textbf{Section~\ref{Section_simulations_Pareto}} are presented in Table~\ref{Pareto_simulations1} and Figures~\ref{Pareto_histograms} and Figure~\ref{sample_size_pareto}. Figure~\ref{Simulations2_plots} shows an example of datasets generated via different models from Section~\ref{Section_simulations_Gaussian}.

% Please add the following required packages to your document preamble:
% \usepackage{multirow}
% \usepackage{graphicx}
\begin{table}[]
\centering
\renewcommand{\arraystretch}{1.15}
\begin{tabular}{l|
                S[table-format=3.0]
                S[table-format=3.0]
                S[table-format=3.0]
                S[table-format=3.0]
                S[table-format=3.0]}
\toprule
\textbf{$\gamma$} &
{$X_1 \to X_2$} &
{$X_2 \to X_1$} &
{\makecell{Empty\\graph}} &
{\makecell{Both directions\\appear plausible}} &
{\makecell{Neither direction\\appears plausible}} \\
\midrule
\rowcolor{RowAlt}
$-2$ & 0  & 0  & \bfseries 96 & 2  & 2  \\
$-1$ & 3  & 2  & 0  & \bfseries 95 & 0  \\
\rowcolor{RowAlt}
\,\,$0$  & 7  & 1  & 0  & \bfseries 92 & 0  \\
\,\,$1$  & 7  & 5  & 0  & \bfseries 86 & 2  \\
\rowcolor{RowAlt}
\,\,$2$  & \bfseries 93 & 0  & 0  & 14 & 3  \\
\bottomrule
\end{tabular}
\caption{Simulation results for the CPCM model using the Pareto distribution function \(F\). The table displays the percentage of cases for each type of graph structure estimated by Conservative Algorithm~\ref{Algorithm1} with the model specified in \eqref{fwesef}, across various values of the hyperparameter \(\gamma \in \mathbb{R}\). The columns indicate the frequency of each graph structure being estimated, with the highest frequency in each row highlighted in bold.}
\label{Pareto_simulations1}
\end{table}


\begin{figure}[]
\centering
\includegraphics[width = 0.8\textwidth]{figures/pareto_histogram.pdf}
\caption{(Simulations~\ref{Section_simulations_Pareto}). Distributions of the p-values from the independence test in Step 1b) of Algorithm~\ref{Algorithm1}, for model \eqref{fwesef} with \(\gamma = 0\) and \(\gamma = 2\).}
\label{Pareto_histograms}
\end{figure}


\begin{figure}[]
\centering
\includegraphics[scale=0.7]{figures/score_based_estimate.pdf}
\caption{(Simulations~\ref{Section_simulations_Pareto}). The plot displays the percentage of correctly estimated causal directions across a range of sample sizes \( n \), using model \eqref{fwesef} with hyperparameters \(\gamma = 1\) and \(\gamma = 2\). As \( n \) increases, the algorithm demonstrates near-perfect performance, affirming the theoretical consistency of the proposed method. }
\label{sample_size_pareto}
\end{figure}




The experiments from \textbf{Simulations~\ref{Section_simulations_Gaussian}} were inspired by \cite{Natasa_Tagasovska} and implementations of other baseline methods are also taken from \cite{Natasa_Tagasovska} and \cite{immer2022identifiability}. 

For LOCI, we use the default format with neural network estimations and subsequent independence testing (also denoted as $NN-LOCI_H$) \citep{immer2022identifiability}.
For IGCI, we use the original implementation from \cite{IGCI} with slope-based estimation with Gaussian and uniform reference measures. For RESIT, we use the implementation from \cite{Peters2014} with GP regression and the HSIC independence test with a threshold value of $0.05$. For the slope algorithm, we use the implementation of \cite{Slope}, with the local regression included in the fitting process. For comparisons with other methods such as PNL, GPI-MML, ANM, Sloppy, GR-AN, EMD, GRCI, see Section 3.2 in \cite{Natasa_Tagasovska} and Section 5 in \cite{immer2022identifiability}. 

In \textbf{Section~\ref{Section_simulations_multivariate}}, the other baseline methods are implemented using the \texttt{pcalg} package \citep{pcalg_package}, employing its default independence test \texttt{gaussCItest} for PC algorithm and GES with the Gaussian observational BIC score, using the default penalty. For ANM-RESIT, for fairness, we use the same choices as in our CPCM method (that is, GAM estimator and HSIC). 

Finally, regarding \textbf{Section~\ref{Section7}},  Table~\ref{tab:edge_share} shows the relative frequencies with which each edge was recovered across repeated subsamples of the motor insurance dataset.


\begin{figure}[]
\centering
\includegraphics[scale=0.4]{figures/sim6.2_sample.png}
\caption{Simulations~\ref{Section_simulations_Gaussian}. An example of datasets generated via different models. }
\label{Simulations2_plots}
\end{figure}



\begin{table}[ht]
\centering
\begin{minipage}{0.45\linewidth}
\centering
\begin{tabular}{lc}
\hline
Edge & Share (\%) \\
\hline
VehAge $\to$ ClaimNb     & 60 \\
Exposure $\to$ ClaimNb   & 48 \\
VehPower $\to$ ClaimNb   & 42 \\
VehPower $\to$ Exposure  & 44 \\
\hline
\end{tabular}
\end{minipage}\hfill
\begin{minipage}{0.45\linewidth}
\centering
\begin{tabular}{lc}
\hline
Edge & Share (\%) \\
\hline
VehAge $\to$ Exposure    & 36 \\
Exposure $\to$ VehAge    & 64 \\
VehPower $\to$ VehAge    & 54 \\
VehAge $\to$ VehPower    & 30 \\
\hline
\end{tabular}
\end{minipage}
\caption{Relative frequency (in \%) with which each directed edge was recovered by CPCM across 50 random subsamples of the French MTPL motor insurance dataset (shown only those with more than $25\%$ share).}
\label{tab:edge_share}
\end{table}









