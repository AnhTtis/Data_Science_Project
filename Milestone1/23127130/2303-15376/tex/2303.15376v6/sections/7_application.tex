\subsection{Illustration using a motor insurance dataset}
\label{Section7}

Understanding the causal relationships between driver and vehicle characteristics is a key step in insurance analytics, as it informs both risk assessment and premium setting.
We demonstrate the advantages of CPCM using a subset of the French MTPL motor insurance dataset \citep{sarpa2025freMTPL2}, restricted to four variables:  
``ClaimNb'' records the number of claims during the policy period (integer between 1 and 16) and lending itself naturally to a Poisson model.  
``VehPower'' and ``VehAge'' are vehicles engine power and age, and ``Exposure'' is the duration that the policy was active.  
Exponential/Gamma distribution is a natural model for variables ``VehPower'', ``VehAge'', and ``Exposure'' due to their exponential-type shapes. Since the dataset contains almost a million observations, we focus on a subset of the first $n = 1000$ records; results based on different random subsamples are provided in Appendix~\ref{Appendix_Section_simulations_Pareto}.


Fitting an LINGAM/ANM to this dataset is problematic for two reasons.  
First, the discrete nature of ``ClaimNb'' violates the continuous additive noise assumption used by most ANM methods \citep{Peters_discrete}.  
Second, the other variables are skewed, non-Gaussian and heteroskedastic, making  CPCM much more natural choice.

Using the sequential family-selection approach within CPCM and the RESIT-greedy algorithm, we obtain the graph shown in Figure~\ref{Fig_motor}. In this case, family $\mathscr{S}_1$ was selected, as the resulting graph was marked as {plausible}. In contrast, applying LiNGAM or ANM-RESIT yields markedly different structures: LiNGAM suggests $\mathrm{Exposure} \rightarrow \mathrm{VehAge} \rightarrow \mathrm{VehPower}$, while ANM-RESIT recovers only $\mathrm{VehAge} \rightarrow \mathrm{VehPower}$.  

Although the ground truth for this dataset is not perfectly clear, the results provide evidence that CPCM can recover more plausible causal structures in mixed-type insurance data than existing ANM-based or linear approaches.  
By accommodating discrete outcomes, non-Gaussian noise, heteroskedasticity and heavy-tails, CPCM produces graphs that align better with domain knowledge and avoid the misspecifications that can arise in more restrictive frameworks.  
On the other hand, CPCM can be computationally demanding for larger datasets, especially when sequential family selection is combined with many candidate parent sets.  
Furthermore, as with most observational causal discovery methods, CPCM relies on the assumption of causal sufficiency, which can be restrictive in practical applications.  






\begin{figure}[h]
\centering
\begin{tikzpicture}[
    every node/.style={draw, rectangle, rounded corners, minimum width=1.7cm, minimum height=0.9cm, align=center},
    >=stealth,
    node distance=1.2cm % reduced vertical gap
]

% Nodes
\node (VehAge) {VehAge};
\node[right=of VehAge] (VehPower) {VehPower};
\node[right=of VehPower] (Exposure) {Exposure};
\node[below=0.9cm of VehPower] (ClaimNb) {ClaimNb};

% Edges
\draw[->] (VehAge) -- (VehPower);
\draw[->] (VehAge.north east) .. controls +(1.0,0.8) and +(-1.0,0.8) .. (Exposure.north west);
\draw[->] (VehPower) -- (Exposure);
\draw[->] (VehPower) -- (ClaimNb);
\draw[->] (Exposure) -- (ClaimNb);

\end{tikzpicture}
\caption{Causal graph estimated by CPCM on the French motor insurance dataset subset.}
\label{Fig_motor}
\end{figure}







\iffalse
\subsection{Illustration using income and expenditure dataset}

We explain our methodology in detail based on real-world data that describes the expenditure habits of Philippines residents. The Philippine Statistics Authority conducts a nationwide survey of Family Income and Expenditure \citep{psa_fies} every three years. 
The dataset (taken from \cite{FamilyIncomeExpenditure}) contains over 40,000 observations primarily comprising the household income and expenditures of each household. To reduce the size and add homogeneity to the data, we consider only families of size $1$ (people living alone) above the poverty line (top 90\%, with an income of at least $80,000\,\, pesos\approx 4000\,\,dollars $ per year). We end up with $n=1417$ observations. We focus on the following variables: Total income ($X_1$), Food expenditure ($X_2$), and Alcohol expenditure ($X_3$). 

These data exhibit strong heavy-tailed behavior, presenting a significant challenge for most causal discovery approaches. For instance, while $95\%$ of the population has an annual income below $400,000\,\text{pesos}$, the top $1\%$ far exceeds $1,000,000\,\text{pesos}$. This tail structure suggests the possibility of infinite variance and even an infinite expected value for all variables. Consequently, ANM and location-scale models are highly unsuitable in this context, since $f(x) = \mathbb{E}[X_i\mid X_j=x]=\infty$. In contrast, our $CPCM(F)$ method is well-suited for heavy-tailed settings when employing a heavy-tailed choice of $F$. In economics, it is common practice to model income using Pareto or Gamma distributions \citep{lawless2002statistical}.


Our objective is to identify the causal relationships among these variables. Common sense suggests that \( \text{Income} \to \text{Food} \). However, the relationships between alcohol and other variables are not trivial. In order to discovery causal relations, we apply our \( CPCM(F_1, \dots, F_k) \) methodology, following the algorithm presented in Section \ref{Section_Algorithm}. Choice \( \{F_1, \dots, F_k\} = \mathscr{S}_1 \) leads to strong rejection of all tests, making no direction plausible. This is not surprising, as the sample size is relatively large for a single parameter to sufficiently describe the complex behavior of the data. Hence, we choose \( \{F_1, \dots, F_k\} = \mathscr{S}_2 \), as defined in Section~\ref{Section_practical_choices}. 

First, we focus on the causal relationships between pairs of random variables.  Applying Algorithm~\ref{Algorithm1} to determine the causal relationship between \( X_1 \) and \( X_2 \) yields p-values of 0.2 and 0.02 for the directions \( X_1 \to X_2 \) and \( X_2 \to X_1 \), respectively. This suggests rejecting the plausibility of the latter graph while not rejecting the former, leading to the final estimation \( X_1 \to X_2 \). Using a similar approach, we conclude that \( X_3 \to X_2 \), with p-values of \( 2 \times 10^{-9} \) and 0.16. This result suggests that drinking habits influence food habits. Finally, the p-values corresponding to \( X_1 \to X_3 \) and \( X_3 \to X_1 \) were \( 10^{-9} \) and 0.02, respectively. This indicates that some assumptions remain unfulfilled. In this case, we believe that causal sufficiency is violated due to a strong unobserved common cause between these variables. Note that even though both causal graphs were implausible, the direction \( X_3 \to X_1 \) appeared to be more probable (\(0.02 > 10^{-9} \)).

Finally, we apply the multivariate score-based algorithm presented in Section \ref{Section_score_based_algorithm} using the choice \( \{F_1, \dots, F_k\} = \mathscr{S}_2 \). The graph with the best score is \( \text{Alcohol} \to \text{Food} \leftarrow \text{Income} \). However, this graph is not plausible, as the test of independence between \( \hat{\varepsilon}_1, \hat{\varepsilon}_2, \hat{\varepsilon}_3 \) yields a p-value of 0.03. This suggests that some assumptions, such as causal sufficiency, may still be violated. 

\fi