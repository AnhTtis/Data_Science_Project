\section{Conclusion and future research}


In this study, we introduced a new family of models for causal inference---the CPCM. The main part of the theory consisted of exploring the identifiability of the causal structure under this model. We showed that bivariate CPCM models with distribution function $F$ are typically identifiable, except when the parameters of $F$ are in the form of a linear combination of its sufficient statistics. Moreover, we showed more detailed characterization of the identifiability in the one-parameter as well as the Gaussian and Pareto cases.  We briefly explained the multivariate extensions of these results. 

We proposed an algorithm that estimates the causal graph based on the CPCM model and  discussed several possible extensions of the algorithm. A short simulation study suggests that the methodology is comparable with other commonly used methods in the Gaussian case. However, our methodology is rather flexible, varying with the choice of $F$. For specific choices of $F$, our methodology can be adapted for detection of causality-in-variance or causality-in-tail and can be used to generalize additive models in various frameworks. We applied our methodology on real-world data and discussed a few possible problems and results. 

Our methodology is not meant to be a black-box model for causal discovery. Discussing the choice of models and adapting them for different applications can bring a new perspective on the causal discovery, and future research is required to show how useful our framework is in practice. 
However, identifying an automatic, data-driven choice of $F$ can lead to new and interesting results. 

Our framework can also be useful for different causal inference tasks. For example, the invariant-prediction framework \citep{Peters_invariance} can be adapted for detecting invariance in a non-additive matter, such as invariance in tails or invariance in variance. This can lead to new directions of research.



\section*{Conflict of interest and data availability}
R code and the data are available in an online \href{https://github.com/jurobodik/Causal\textunderscore CPCM.git}{repository} or on request from the author. 

The authors declare that they have no known competing financial interests or personal relationships that could have appeared to influence the work reported in this paper.


\section*{Acknowledgements}
This study was supported by the Swiss National Science Foundation. 

































































