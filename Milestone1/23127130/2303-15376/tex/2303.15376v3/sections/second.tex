
\section{Definitions of causal models}
\label{Section2}

Recall that the general SCM corresponds to $d$ equations $X_i = f_{i}(\textbf{X}_{pa_i}, \varepsilon_i)$, $i=1, \dots, d$, where $(\varepsilon_1, \dots, \varepsilon_d)^\top$ are jointly independent noise variables. 
 In this section, we discuss appropriate assumptions on the conditional distribution  $F_{effect\mid causes}$, where the causes potentially affect the variance or the tail of the effect. First, we introduce the model in the bivariate case. 


\subsection{Bivariate case of conditionally parametric causal model}

In the following account, we consider that  $F_{effect\mid cause}$ belongs to a known parametric family of distributions, and the cause affects only the parameters of the distribution, not the form itself. 

In (unrestricted) SCM with the causal graph $X_1\to X_2$ and the structural equation $X_2=f_2(X_1, \varepsilon_2), \varepsilon_2\indep X_1$, we can, without loss of generality, assume that $\varepsilon_2$ is uniformly distributed. This notion follows the idea of generating random variables in computer software. As long as we do not restrict $f_2$, we can write  $f_{2}(X_1, \varepsilon_2) = f_{2}\big(X_1, g^{-1}(\varepsilon_U)\big)=\tilde{f_2} (X_1, \varepsilon_U) $, where $\varepsilon_U = g(\varepsilon_2)$ is uniformly distributed and $\tilde{f}_2$ is the new structural equation. Inferring the distribution of $\varepsilon_2$ and the function $g$ is an equivalent task. Therefore, we assume  $\varepsilon_2\sim U(0,1)$. 

 The following definition describes that if $X_1\to X_2$ then $X_2\mid X_1$ has the conditional distribution $F$ with parameters $\theta(X_1)\in\mathbb{R}^q$ for some $q\in\mathbb{N}$.  

\begin{definition}
We define the bivariate \textbf{conditionally parametric causal model} (bivariate $CPCM(F)$) with graph $X_1\to X_2$ by two assignments 
\begin{equation}\label{BCPCM}
X_1=\varepsilon_1, \,\,\,\,\,\,\,\,\,\,\,\,\,\,\,\,\,X_2=  F^{-1}\big(\varepsilon_2;\theta(X_1)\big),\tag{\ding{170}}
\end{equation}
where $\varepsilon_1\indep\varepsilon_2$ are noise variables, $\varepsilon_2$ is uniformly distributed, and $F^{-1}$ is the quantile function with $q$ parameters $\theta(X_1)=\big(\theta_1(X_1), \dots, \theta_q(X_1)\big)^\top$. 

We assume that $\theta_i$ represents measurable non-constant functions on the support of $X_1$.  If $\varepsilon_1$ is continuous, we additionally assume that $\theta$ is continuous on the support of $\varepsilon_1$. 
\end{definition}
We put no restrictions on the marginal distribution of the cause. Note that we implicitly assume causal minimality, since we assume that $\theta_i$ are non-constant. 

\begin{customexample}{\ref{Gaussian case}}[Continued]
Suppose $(X_1,X_2)$ admits the model ( \ref{BCPCM}) with graph $X_1\to X_2$. Assuming $X_2\mid X_1\sim N\big(\mu(X_1), \sigma^2(X_1)\big)$ for some real functions $\mu$ and $ \sigma$, is equivalent to assuming that $F$ is Gaussian distribution function in ( \ref{BCPCM}), where  $\theta(X_1)=\big(\mu(X_1), \sigma(X_1)\big)^\top$. Equivalently, we write
$$X_2 = \mu(X_1)+\sigma(X_1)\cdot\varepsilon_2, \,\,\,\,\,\,\,\,\,\,\,\,\,\,\,\,\,\varepsilon_2 \text{ is Gaussian.}$$
Assuming $\sigma(X_1)=const.$ brings us back to the ANM framework.
\end{customexample}
The previous example can be understood as a model where the cause affects the mean \textit{and} variance. 
%Another important example comes from finance, where it is well known that many variables follow Pareto distribution. 

\begin{customexample}{\ref{example_Pareto}}[Continued]
Suppose $(X_1,X_2)$ admits the model ( \ref{BCPCM}) with graph $X_1\to X_2$, with $F^{-1}$ being the Pareto quantile function \footnote{The density has the form $p_{X_2\mid X_1=x}(y)=\frac{\theta(x)}{y^{\theta(x)+1}}, \theta(x)>0, y\geq 1$.}. This model corresponds to $X_2\mid X_1\sim Pareto\big(\theta(X_1)\big).$ If $\theta(X_1)$ is small, then the tail of $X_2$ is large and extremes occur more frequently. Note that if $\theta(x)<k$  for $k\in\mathbb{N}$,  then the $k$-th moment of $X_2\mid X_1=x$ does not exist. 
\end{customexample}
We provide another scenario in which (\ref{BCPCM}) can be appealing. Consider a situation, in which we are interested in studying waiting times, such as the time until a patient's death. Then, assuming $F$ is the exponential distribution can be reasonable, as using an exponential model is a common practice in statistics and survival analysis. Some might argue that assuming a specific distribution for the effect might be unrealistic, unless there are reasons to believe in such a situation. However, it can be more realistic (or, more importantly, more interpretable) than the additivity of noise (ANM) or similar structural restrictions, which are rather useful for causal discovery. 


\subsection{Asymmetrical $(\mathcal{M}_{1},\mathcal{M}_{2})$-causal models and $CPCM(F_1, F_2)$}
We define a class of models, where we place different assumptions on $X_1$ and $X_2$ and generalize the model from (\ref{BCPCM}).

\subsubsection{Motivation}

In model-based approaches for causal discovery (approaches such as ANM or post-nonlinear models), we \textit{assume} some form of $F_{effect\mid cause}$.  Assume that $F_{effect\mid cause}\in\mathcal{F}$, where $\mathcal{F}$ is a subset of all conditional distributions.
For example, in CPCM, $\mathcal{F}$ is a known parametric family of conditional distributions. In ANM,  $\mathcal{F}$ consists of all conditional distributions that arise as the sum of a function of the cause and a noise. If $\mathcal{F}$ is sufficiently small, we can hope for the identifiability of $\mathcal{G}$. 

In what follows, we discuss a different set of assumptions that is more general and can be asymmetrical for $X_1$ and for $X_2$. Instead of restricting $F_{effect\mid cause}\in\mathcal{F}$, we restrict either $F_{X_1\mid X_2}\in \mathcal{F}_1$, or $F_{X_2\mid X_1}\in\mathcal{F}_2$, where $\mathcal{F}_1, \mathcal{F}_2$ are (not necessarily equal) subsets of all conditional distributions (selecting different causal models). We substantiate this by the following example. 

Suppose we observe data such as that in Figure~\ref{Asymetrical_picture}. Here, $X_2$ is non-negative, and $X_1$ has full support. Selecting an appropriate restriction of $F_{effect\mid cause}$ can be tricky, since 
$F_{effect\mid cause}$ needs to be non-negative if $X_1\to X_2$. On the other hand, $F_{effect\mid cause}$ needs to have full support in the case in which $X_2\to X_1$.  

Instead of restricting $F_{effect\mid cause}$, we divide our assumptions into two cases. If $X_1\to X_2$, then we assume $F_{X_2\mid X_1}\in\mathcal{F}_2$; if $X_2\to X_1$, then we assume $F_{X_1\mid X_2}\in\mathcal{F}_1$, where $\mathcal{F}_2$ consists of non-negative distributions and $\mathcal{F}_1$ consists of distributions with full support. 

Note that this is a generalization of classical model-based approaches since they make the implicit choice $\mathcal{F}_1 = \mathcal{F}_2$.  In the following account, we create a framework that enables asymmetrical assumptions. 

\begin{figure}[ht]
\centering
\includegraphics[scale=0.4]{figures/Asymetrical_picture.png}
\caption{A dataset generated in the following manner: $X_2\sim Pareto(3), X_1=X_2 + X_2\cdot\varepsilon_1$, where $\varepsilon_1\sim N(0, 1)$. In other words, we have $X_2\to X_1$ and the model follows $CPCM(F)$ with Gaussian $F$.}
\label{Asymetrical_picture}
\end{figure}

\subsubsection{Definition}\label{Section2.2.2}

Consider a class of bivariate SCMs, which are denoted by $\mathcal{M}$. For example, $\mathcal{M}$ can denote all SCM that follow ANM or $\mathcal{M}$ can denote all SCM that follow (\ref{BCPCM}). 

\begin{definition}\label{Asymmetrical_causal_model}
Let $\mathcal{M}_1$ and $\mathcal{M}_2$ be two classes of bivariate SCMs with non-empty causal graphs. We say that the pair of random variables $(X_1, X_2)$ follows the \textbf{asymmetrical} $(\mathcal{M}_1,\mathcal{M}_2)$\textbf{ causal model}, if one of the following conditions holds: 
\begin{itemize}
\item the pair $(X_1, X_2)$ follows the SCM from $\mathcal{M}_2$ with the causal graph $X_1\to X_2$, or
\item the pair $(X_1, X_2)$ follows the SCM from $\mathcal{M}_1$ with the causal graph $X_2\to X_1$.
\end{itemize}
\end{definition}
This definition allows different assumptions on each case of the causal structure. For example, if one variable is discrete and the second is continuous, we can assume ANM in one direction and the discrete QVF DAG model \citep{ParkPoisson} in the other. In the remainder of the paper, we focus on the continuous CPCM case. 

Let $F_1$ and $F_2$ be two continuous distribution functions with $q_1, q_2\in\mathbb{N}$ parameters, respectively. Let $\mathcal{M}_{F_1}$ and $\mathcal{M}_{F_2}$ be two classes of SCMs that arise from the CPCM model (\ref{BCPCM}) with $F_1$ and $F_2$, respectively. We rephrase Definition \ref{Asymmetrical_causal_model} for this case.  
A pair of dependent random variables $(X_1, X_2)$ follows the asymmetrical $(\mathcal{M}_{F_1},\mathcal{M}_{F_2})$ causal model (we call it $CPCM(F_1, F_2)$), if either
\begin{equation}\label{asymetrical_F_one_F_two_model}\tag{$\bigstar$}
\begin{split}
 & \,\,\,\,\,\,\,  X_1 = \varepsilon_1, X_2 = F_2^{-1}\big(\varepsilon_2; \theta_2(X_1)\big), \varepsilon_2\sim U(0,1), \varepsilon_1\indep \varepsilon_2,  \text{or } \\&
X_2 = \varepsilon_2, X_1 = F_1^{-1}\big(\varepsilon_1; \theta_1(X_2)\big), \varepsilon_1\sim U(0,1),\varepsilon_1\indep \varepsilon_2,
\end{split}
\end{equation}
where $\theta_1, \theta_2$ are measurable non-constant continuous functions as in (\ref{BCPCM}). 
Note that the $CPCM(F)$ model is a special case of the asymmetrical $(\mathcal{M}_{1},\mathcal{M}_{2})$-causal model, where  $\mathcal{M}_{1}=\mathcal{M}_{2}=\mathcal{M}_{F}$.  






