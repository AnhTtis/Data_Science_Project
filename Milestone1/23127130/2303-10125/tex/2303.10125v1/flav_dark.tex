\documentclass[a4paper,11pt]{article}
\pdfoutput=1

\oddsidemargin -2mm  \evensidemargin 0cm
\topmargin -1cm  \textwidth 17cm  \textheight 22.5cm


\interfootnotelinepenalty=10000

\def\DevnagVersion{2.15}
%\usepackage{devanagari}
\usepackage{hyperref}



\usepackage{
graphicx,
%rotating,
hyperref,
%slashed,
amsmath,
amssymb,
charter,
xcolor,
%catchfilebetweentags,
ifluatex,
multirow}
        % twocolumn
        
%\usepackage{colortbl}

\definecolor{Gray}{gray}{0.95}
\definecolor{RGray}{gray}{0.85}
\definecolor{CGray}{gray}{0.92}

%\usepackage[bookmarks=false]{hyperref}
%\usepackage{multicol}
%\usepackage{color}
\definecolor{tit}{rgb}{0.1,0.2,0.4}
\definecolor{blus}{cmyk}{1,1,0,0.6}
\definecolor{verde}{cmyk}{0.92,0,0.59,0.25}
        
\newcommand{\e}[1]{\cdot 10^{#1}}
%
%\journal{Journal of \LaTeX\ Templates}

\usepackage{tipa}
\usepackage{amsmath,amssymb,amsfonts, bm}
%\usepackage{dsfont}
\bibliographystyle{plain}
\usepackage{epsfig}
\usepackage{graphicx}
\usepackage{slashed}
\usepackage{color}

\usepackage{caption}
\usepackage{subcaption}
\captionsetup{compatibility=false}
\usepackage{slashed} 
\usepackage{float}

%\usepackage{bbm} % BlackBoeard letters 
\newcommand{\eps}{\epsilon}


\newcommand{\Heff}{{\cal H}_\text{NP}}


\newcommand{\meg}{\mu \to e \gamma}

\newcommand{\D}{{\cal D}}
\newcommand{\U}{{\cal U}}

\newcommand{\E}{{\cal E}}
\newcommand{\N}{{\cal N}}
\newcommand{\M}{{\cal M}}


%\usepackage{commath}
\usepackage{calc}

\newcommand{\be}{\begin{equation}}
\newcommand{\ee}{\end{equation}}

\newcommand{\wt}{\widetilde}

\newcommand{\bea}{\begin{eqnarray}}
\newcommand{\eea}{\end{eqnarray}}

\newcommand{\bfig}{\begin{figure}}
\newcommand{\efig}{\end{figure}}

\newcommand{\meee}{\mu \to e \bar{e} e}
\newcommand{\lag}{\ensuremath{\mathcal{L}}}

\newcommand{\qqquad}{\qquad \qquad}
\newcommand{\qqqquad}{\qquad \qquad \qquad}

%\newcommand{\eps}{\epsilon}
\newcommand\nnm{\nonumber}
\newcommand\ra{\rightarrow}
\newcommand{\br}{\text{BR}}

\newcommand{\f}[2]{\frac{#1}{#2}}

%\newcommand{\E}[1]{\ensuremath{\mathrm{E}_{#1}}} % e.g. \E{8}
\newcommand{\G}[1]{\ensuremath{\mathrm{G}_{#1}}}
\newcommand{\SO}[1]{\ensuremath{\mathrm{SO}(#1)}}
\newcommand{\SU}[1]{\ensuremath{\mathrm{SU}(#1)}}
%\newcommand{\U}[1]{\ensuremath{\mathrm{U}(#1)}}
\newcommand{\Z}[1]{\ensuremath{\mathbbm{Z}_{#1}}} % Z_N ->\Z{N}
%\newcommand{\D}{\mathrm{d}}
\newcommand{\I}{\mathrm{i}}
\DeclareMathOperator{\diag}{diag}
\DeclareMathOperator{\Tr}{Tr}





\makeatletter
\newcommand*{\rom}[1]{\expandafter\@slowromancap\romannumeral #1@}
\makeatother


\begin{document}
\allowdisplaybreaks
\vspace*{-2.5cm}
%\begin{flushright}
%{\small
%IIT-BHU
%}
%\end{flushright}

\vspace{2cm}



\begin{center}
{\LARGE \bf \color{tit} Flavonic dark matter  }\\[1cm]

{\large\bf Gauhar Abbas$^{a}$\footnote{email: gauhar.phy@iitbhu.ac.in} \\ 
Rathin Adhikari$^{b}$\footnote{email: rathin@ctp-jamia.res.in }   \\
Eung Jin Chun$^{c}$\footnote{email: ejchun@kias.re.kr  }  }
\\[7mm]
{\it $^a$ } {\em Department of Physics, Indian Institute of Technology (BHU), Varanasi 221005, India}\\[3mm]
{\it $^b$ Centre for Theoretical Physics, Jamia Millia Islamia (Central University), New Delhi 110025, India } {\em }\\[3mm]
{\it $^c$ Korea Insitute for Advanced Study, Seoul 02455, Republic of Korea } {\em }\\[3mm]

\vspace{1cm}
{\large\bf\color{blus} Abstract}
\begin{quote}
We first time show that a common solution to dark matter and the flavour problem of the standard model can be obtained in the framework of the  $\mathcal{Z}_{\rm N} \times \mathcal{Z}_{\rm M}$ flavour symmetry where the flavonic Goldstone boson of this flavour symmetry acts as a good dark matter candidate through the misalignment mechanism. Hierarchical mass pattern of quarks and charged leptons naturally follows from the discrete symmetry. For light active neutrinos, we construct the Dirac-type mass matrix which is preferred to fit the observed neutrino oscillation data with normal hierarchy.
\end{quote}

\thispagestyle{empty}
\end{center}

\begin{quote}
{\large\noindent\color{blus} 
}

\end{quote}

\newpage
%\setcounter{page}{1}
\setcounter{footnote}{0}

\section{Introduction}
The  Standard Model (SM) of particle physics is the most successful quantum theory of our universe providing a remarkable description of the elementary particles,  such as quarks and leptons  who constitute the matter of the universe,   and their interactions.    The SM,  notwithstanding with its triumph,  faces serious theoretical  imperfections and experimental failings.  In particular,  the discovery of the dark-matter (DM) is a dire experimental shortcoming of the SM.   On the theoretical side,  one of the critical problems is the so-called ``flavour-problem" of the SM.    The flavour problem is defined by the absence of any mechanism to explain the hierarchical structure of the masses of different flavour and their mixing in the SM.  The problem of neutrino masses and their oscillations can also be added to the flavour problem of the SM. This problem can be approached in different frameworks, such as, a technicolour framework  where the  vacuum-expectation-values   are sequential chiral condensates of an extended dark-technicolour sector providing a solution 
 \cite{Abbas:2017vws,Abbas:2020frs}, through an  Abelian flavour symmetry~\cite{Froggatt:1978nt,flavor_symm1,Chun:1996xv,flavor_symm2,flavor_symm3,Davidson:1983fy,Davidson:1987tr},  using loop-suppressed couplings to the Higgs~\cite{higgs_coup},  in a  wave-function localization scenario~\cite{wf_local},  through compositeness~\cite{partial_comp}, in an extra-dimension framework \cite{Fuentes-Martin:2022xnb}, and using a  discrete symmetry \cite{Higaki:2019ojq}.

It is remarkable to observe that  a particle-like  explanation to the problem of the dark matter,  and  a field theoretical solution to the flavour problem, such as the Frogatt-Nielsen (FN) mechanism \cite{Froggatt:1978nt}, are apparently mutually exclusive  and are poles apart.   
In the FN mechanism, the flavour problem is resolved  by an interaction of a new scalar field called flavon with the SM fermions  \cite{Froggatt:1978nt}:
\begin{equation} \label{FN}
 {\cal L}_{\rm Yuk}=  y_{ij} \left ( \chi \over \Lambda \right)^{n_{ij}} \bar{\psi}_{iL} H \psi'_{jR} +  \text{H.c.},
\end{equation} 
where $y_{ij}$ are order-one parameters and $\chi$ is the flavon field whose couplings (or the exponents $n_{ij}$) are controlled by continuous or discrete charges of the fields. 
After the flavour symmetry breaking,   the fermion Yukawa matrices are expressed in terms of the order parameter 
$\epsilon \equiv \langle \chi \rangle/\Lambda$.   Identifying the order parameter as the Cabibbo angle $\epsilon \approx 0.23$,  all the fermion masses and mixing matrices are
determined by powers of $\epsilon$.
Then the flavon is allowed to decay to the SM fermions at tree-level,  eliminating any possibility for this particle to be a DM candidate.  However, the axial degree of freedom of the flavon can be light enough to guarantee its stability. 
If the flavour symmetry is a continuous $U(1)$ symmetry, the axial flavon field could be identified with the QCD axion  \cite{Calibbi:2016hwq,Ema:2016ops,Bjorkeroth:2017tsz}
providing the solution to the strong CP problem as well as the axion dark matter \cite{Kim:2008hd}.
The flavour problem could be resolved by a discrete symmetry $\mathcal{Z}_{\rm N}$ allowing the flavon potential,
\begin{equation} \label{VN}
 V_{\mathcal{Z}_{\rm N}}= -\lambda {\chi^N\over \Lambda^{N-4}} + \text{H.c.}, 
\end{equation}
which is invariant under $\mathcal{Z}_{\rm N} $.
Upon the $\mathcal{Z}_{\rm N}$ breaking by the vacuum expectation value $\langle \chi \rangle = {v_F \over \sqrt{2}}$,   the flavonic Goldstone boson $\varphi$ receives the potential,
\begin{equation}
  V_{\mathcal{Z}_{\rm N}}= -{1\over4}|\lambda| {\epsilon^{N-4}} v_F^4  \cos\left( N {\varphi\over v_F}+ \alpha\right),
\end{equation}
where $\lambda=|\lambda|e^{i\alpha}$.  Thus, the axial flavon field can be very light for a sufficiently large $N$ and becomes a DM candidate whose abundance is generated by the misalignment mechanism \cite{misalign}. 

In this work,   we will set up a successful discrete flavour symmetry framework providing a solution of the flavour problem and,   show that the misalignment mechanism can generate the observed dark matter density in such a  framework.   We shall show  the axial flavon field can be a dark matter candidate associated with this discrete symmetry resolving the flavour problem,  and thus breaking the impasse posed by the demand of a joint solution of the DM and the flavour problem.   

We shall present our discussion along the following course. 
A $\mathcal{Z}_{\rm N} \times \mathcal{Z}_{\rm M}$ flavour symmetry is introduced to obtain a solution of the flavour problem in Section \ref{zn_zm} and the flavonic dark matter in Section \ref{flav_dark}.  We discuss the flavour changing neutral current (FCNC) couplings of the  axial flavon field and the decays and signatures of the flavonic dark matter  in \ref{decay_sign}.  We summarize this work in section \ref{sum}.


\section{The $\mathcal{Z}_{\rm N} \times \mathcal{Z}_{\rm M}$ flavour symmetry  } 
\label{zn_zm}
The $\mathcal{Z}_{\rm N} \times \mathcal{Z}_{\rm M}$ flavour symmetry  is a new discrete symmetry product capable of providing a solution to the flavour problem of the SM through the FN mechanism \cite{Abbas:2018lga,Abbas:2022zfb}.   This was first proposed in reference \cite{Abbas:2018lga}, and later two prototypes of this symmetry are investigated in reference\cite{Abbas:2022zfb}.  In this work, we use a $\mathcal{Z}_{\rm N} \times \mathcal{Z}_{\rm M}$ flavour symmetry that goes beyond the prototype symmetries discussed in reference \cite{Abbas:2022zfb}.  This is done by creating a flavour model where  the mass of the top quark does not originate from the tree level  SM  Yukawa operator.  This model is inspired by the hierarchical VEVs model \cite{Abbas:2017vws,Abbas:2020frs}, where  even the mass of the top quark arises from the dimension-5 operator.  This is done keeping in mind a possible technicolour origin of the $\mathcal{Z}_{\rm N} \times \mathcal{Z}_{\rm M}$ flavour symmetry. 


Thus, we adopt the $\mathcal{Z}_{8} \times \mathcal{Z}_{22}$ flavour symmetry acting on the flavon field as well as the scalar and the fermionic sector of the SM as defined in table \ref{tab_z6z14}.  The generic form of the Lagrangian, after imposing the $\mathcal{Z}_{8} \times \mathcal{Z}_{22}$ flavour symmetry on the SM,  providing the masses to the SM fermions now reads,
\begin{eqnarray}
\label{mass1}
-{\mathcal{L}}_{\rm Yukawa} &=&          y_{ij}^u \bar{ \psi}_{L_i}^q  \tilde{H} \psi_{R_j}^{u} \left[  \dfrac{ \chi}{\Lambda} \right]^{n_{ij}^u} +         y_{ij}^d \bar{ \psi}_{L_i}^q  H \psi_{R_j}^{d} \left[  \dfrac{ \chi}{\Lambda} \right]^{n_{ij}^d} \nonumber \\
&+&            y_{ij}^\ell \bar{ \psi}_{L_i}^\ell  H \psi_{R_j}^{\ell} \left[  \dfrac{ \chi}{\Lambda} \right]^{n_{ij}^\ell} 
+  {\rm H.c.}, \\ \nonumber
&=&  Y^u_{ij} \bar{ \psi}_{L_i}^q  \tilde{H} \psi_{R_j}^{u}
+ Y^d_{ij} \bar{ \psi}_{L_i}^q  H \psi_{R_j}^{d}
+ Y^\ell_{ij} \bar{ \psi}_{L_i}^\ell  H \psi_{R_j}^{\ell}   + \text{H.c.},
\end{eqnarray}
where $i$ and $j$   represent family indices, $ \psi_{L}^q,  \psi_{L}^\ell    $ denote the quark and leptonic doublets, $ \psi_{R}^u,  \psi_{R}^d, \psi_{R}^\ell    $ are right-handed up, down type singlet quarks and  leptons, $H$ and $ \tilde{H}= -i \sigma_2 H^* $  denote the SM Higgs field and its conjugate and $\sigma_2$ is the second Pauli matrix.  We can write the  effective Yukawa couplings $Y_{ij}$  in terms of the expansion parameter  $   \epsilon= \dfrac{\langle \chi \rangle} { \Lambda}$ such that $Y_{ij} = y_{ij} \epsilon^{n_{ij}}$.

  \begin{table}
 \small
\begin{center}
\begin{tabular}{|c|c|c|}
  \hline
  Fields             &        $\mathcal{Z}_8$                    & $\mathcal{Z}_{22}$        \\
  \hline
  $u_{R}$                 &   $ \omega^2$  &$ \omega^2$                              \\
  $c_{R}$                 &   $ \omega^5$  & $ \omega^5$                             \\
  $t_{R}$                 &   $ \omega^6$  & $ \omega^6$                             \\
   $d_{R}$                 &   $ \omega^3$  &     $\omega^{3} $                              \\
   $s_{R}, b_{R}$                 &   $ \omega^4$  &     $\omega^4 $                              \\
   $\psi_{L,1}^q$                 &    $ \omega^2$  &    $\omega^{10} $                          \\
    $\psi_{L,2}^q$                 &  $ \omega$  &     $\omega^{9} $                         \\
     $\psi_{L,3}^q$                 &    $\omega^{7} $  &      $\omega^{7} $                        \\
      $\psi_{L,1}^\ell$                 &   $ \omega^3$  &    $\omega^3 $                          \\
    $\psi_{L,2}^\ell$                  &   $ \omega^2$  &    $\omega^2 $                         \\
     $\psi_{L,3}^\ell$                 &   $ \omega^2$  &    $\omega^2 $                      \\
       $e_R$                 &   $\omega^{2} $  &     $\omega^{16} $                              \\
         $\mu_R$                 &  $\omega^5 $   &     $\omega^{19} $                              \\
           $\tau_R $                 &   $ \omega^7$  &     $\omega^{21} $                              \\
            $ \nu_{e_R} $                 &     $\omega^2 $    &     $1 $                              \\
           $   \nu_{\mu_R}$                 &     $\omega^5 $    &     $\omega^{3} $                              \\
           $  \nu_{\tau_R} $                 &     $\omega^6 $    &     $\omega^{4} $                              \\
           
    $\chi$                        & $ \omega$  &       $ \omega$                                        \\
    $H$              &   1        &     1 \\
  \hline
     \end{tabular}
\end{center}
\caption{The charges of left and  right-handed fermions of  the SM,   Higgs,  and the flavon field under the $\mathcal{Z}_8 \times \mathcal{Z}_{22}$  symmetry,  where $\omega$ is the 8th and  22th root of unity,  respectively. }
 \label{tab_z6z14}
\end{table} 

The up-type mass matrix originate from the following Lagrangian:
\begin{eqnarray}
\label{mass1}
-{\mathcal{L}}_{\rm Yukawa}^{\rm u} &=&       
y_{11}^u \bar{ \psi}_{L_1}^q  \tilde{H} u_{R}  \left[  \dfrac{ \chi}{\Lambda} \right]^{8} + y_{12}^u \bar{ \psi}_{L_1}^q  \tilde{H} c_{R}  \left[  \dfrac{ \chi}{\Lambda} \right]^{5} + y_{13}^u \bar{ \psi}_{L_1}^q  \tilde{H} t_{R}  \left[  \dfrac{ \chi}{\Lambda} \right]^{4} \nonumber \\ 
&+&  y_{21}^u \bar{ \psi}_{L_2}^q  \tilde{H} u_{R}  \left[  \dfrac{ \chi}{\Lambda} \right]^7 
+ y_{22}^u \bar{ \psi}_{L_2}^q  \tilde{H} c_{R}  \left[  \dfrac{ \chi}{\Lambda} \right]^4 + y_{23}^u \bar{ \psi}_{L_2}^q  \tilde{H} t_{R}  \left[  \dfrac{ \chi}{\Lambda} \right]^3 \\
&+&  y_{31}^u \bar{ \psi}_{L_3}^q  \tilde{H} u_{R}  \left[  \dfrac{ \chi}{\Lambda} \right]^5 + y_{32}^u \bar{ \psi}_{L_3}^q  \tilde{H} c_{R}  \left[  \dfrac{ \chi}{\Lambda} \right]^2 +
 y_{33}^u \bar{ \psi}_{L_3}^q  \tilde{H} t_{R}  \left[  \dfrac{ \chi}{\Lambda} \right]
+  {\rm H.c.}. \nonumber
\end{eqnarray}
The down-type mass matrix is given by  the following Lagrangian:
\begin{eqnarray}
\label{mass1}
-{\mathcal{L}}_{\rm Yukawa}^{\rm d} &=&       y_{11}^d \bar{ \psi}_{L_1}^q  H d_{R}  \left[  \dfrac{ \chi}{\Lambda} \right]^{7} + y_{12}^d \bar{ \psi}_{L_1}^q  H s_{R}  \left[  \dfrac{ \chi}{\Lambda} \right]^{6} + y_{13}^d \bar{ \psi}_{L_1}^q  H b_{R}  \left[  \dfrac{ \chi}{\Lambda} \right]^{6} \nonumber\\
&+& y_{21}^d \bar{ \psi}_{L_2}^q  H d_{R}  \left[  \dfrac{ \chi}{\Lambda} \right]^6 
+ y_{22}^d \bar{ \psi}_{L_2}^q  H s_{R}  \left[  \dfrac{ \chi}{\Lambda} \right]^5 + y_{23}^d \bar{ \psi}_{L_2}^q  H b_{R}  \left[  \dfrac{ \chi}{\Lambda} \right]^5 \\
&+& y_{31}^d \bar{ \psi}_{L_3}^q  H d_{R}  \left[  \dfrac{ \chi}{\Lambda} \right]^4 + y_{32}^d \bar{ \psi}_{L_3}^q H s_{R}  \left[  \dfrac{ \chi}{\Lambda} \right]^3 
+ y_{33}^d \bar{ \psi}_{L_3}^q  H b_{R}  \left[  \dfrac{ \chi}{\Lambda} \right]^3
+  {\rm H.c.}. \nonumber
\end{eqnarray}
In a similar manner, charged lepton mass matrix can be written using the  following Lagrangian:
\begin{eqnarray}
\label{mass1}
-{\mathcal{L}}_{\rm Yukawa}^{\ell} &=&       y_{11}^\ell \bar{ \psi}_{L_1}^\ell  H e_{R}  \left[  \dfrac{ \chi}{\Lambda} \right]^{9} + y_{12}^\ell \bar{ \psi}_{L_1}^\ell  H \mu_{R}  \left[  \dfrac{ \chi}{\Lambda} \right]^{6} + y_{13}^\ell \bar{ \psi}_{L_1}^\ell  H \tau_{R}  \left[  \dfrac{ \chi}{\Lambda} \right]^{4} \nonumber\\
&+& y_{21}^\ell \bar{ \psi}_{L_2}^\ell  H e_{R}  \left[  \dfrac{ \chi}{\Lambda} \right]^{8} 
+y_{22}^\ell \bar{ \psi}_{L_2}^\ell  H \mu_{R}  \left[  \dfrac{ \chi}{\Lambda} \right]^5 + y_{23}^\ell \bar{ \psi}_{L_2}^\ell  H \tau_{R}  \left[  \dfrac{ \chi}{\Lambda} \right]^3 \\
&+& y_{31}^\ell \bar{ \psi}_{L_3}^\ell  H e_{R}  \left[  \dfrac{ \chi}{\Lambda} \right]^{8} + y_{32}^\ell \bar{ \psi}_{L_3}^\ell H \mu_{R}  \left[  \dfrac{ \chi}{\Lambda} \right]^5 
+ y_{33}^\ell \bar{ \psi}_{L_3}^\ell  H \tau_{R}  \left[  \dfrac{ \chi}{\Lambda} \right]^3
+  {\rm H.c.}. \nonumber
\end{eqnarray}
The mass matrices of the  up and down-type quarks and charged leptons now can be written as,
\begin{align}
\M_u & = \dfrac{v}{\sqrt{2}}
\begin{pmatrix}
y_{11}^u  \epsilon^8 &  y_{12}^u \epsilon^{5}  & y_{13}^u \epsilon^{4}    \\
y_{21}^u \epsilon^7     & y_{22}^u \epsilon^4  &  y_{23}^u \epsilon^{3}  \\
y_{31}^u  \epsilon^{5}    &  y_{32}^u  \epsilon^2     &  y_{33}^u  \epsilon 
\end{pmatrix}, \nonumber \\ 
\M_d  & = \dfrac{v}{\sqrt{2}}
\begin{pmatrix}
y_{11}^d  \epsilon^7 &  y_{12}^d \epsilon^6 & y_{13}^d \epsilon^6   \\
y_{21}^d  \epsilon^6  & y_{22}^d \epsilon^5 &  y_{23}^d \epsilon^5  \\
 y_{31}^d \epsilon^4 &  y_{32}^d \epsilon^3   &  y_{33}^d \epsilon^3
\end{pmatrix}, \\ \nonumber 
\M_\ell &=  \dfrac{v}{\sqrt{2}}
\begin{pmatrix}
y_{11}^\ell  \epsilon^9 &  y_{12}^\ell \epsilon^6  & y_{13}^\ell \epsilon^4   \\
y_{21}^\ell  \epsilon^{8}  & y_{22}^\ell \epsilon^5  &  y_{23}^\ell \epsilon^3  \\
 y_{31}^\ell \epsilon^{8}   &  y_{32}^\ell \epsilon^5   &  y_{33}^\ell \epsilon^3
\end{pmatrix}.
\end{align}

The masses of charged fermions are approximately  given by\cite{Rasin:1998je},
\begin{align}
\label{eqn5}
\{m_t, m_c, m_u\} &\simeq \{|y_{33}^u| \epsilon , ~ \left |y_{22}^u  - \frac {y_{23}^u y_{32}^u} {y_{33}^u  }   \right|  \epsilon^4 ,\\&
~ \left |y_{11}^u- \frac {y_{12}^u y_{21}^u}{y_{22}^u-y_{23}^u y_{32}^u/y_{33}^u}- \frac{y_{13}^u (y_{31}^u y_{22}^u-y_{21}^u y_{32}^u)-y_{31}^u y_{12}^u y_{23}^u}{(y_{22}^u- y_{23}^u y_{32}^u/y_{33}^u) y_{33}^u} \right| \epsilon^8\}v/\sqrt{2}  ,\nonumber \\ 
\{m_b, m_s, m_d\} & \simeq \{|y_{33}^d| \epsilon^3, ~ \left |y_{22}^d- \frac {y_{23}^d y_{32}^d} {y_{33}^d} \right| \epsilon^5,\\ \nonumber 
&  \left |y_{11}^d- \frac {y_{12}^d y_{21}^d}{y_{22}^d-y_{23}^d y_{32}^d/y_{33}^d}- \frac{y_{13}^d (y_{31}^d y_{22}^d-y_{21}^d y_{32}^d)-y_{31}^d y_{12}^d y_{23}^d}{(y_{22}^d- y_{23}^d y_{32}^d/y_{33}^d) y_{33}^d} \right| \epsilon^7\}v/\sqrt{2} ,\\ \nonumber 
\{m_\tau, m_\mu, m_e\} & \simeq \{|y_{33}^l| \epsilon^3, ~ \left|y_{22}^l- \frac {y_{23}^l y_{32}^l} {y_{33}^l} \right| \epsilon^5,\\& ~  \left |y_{11}^l- \frac {y_{12}^l y_{21}^l}{y_{22}^l-y_{23}^l y_{32}^l/y_{33}^l}- \frac{y_{13}^l \left( y_{31}^l y_{22}^l-y_{21}^l y_{32}^l \right) -y_{31}^l y_{12}^l y_{23}^l}{\left(  y_{22}^l- y_{23}^l y_{32}^l/y_{33}^l \right) y_{33}^l} \right| \epsilon^9\}v/\sqrt{2}.
\end{align}
The mixing angles of quarks read\cite{Rasin:1998je},
\begin{eqnarray}
\sin \theta_{12}  \simeq |V_{us}| &\simeq& \left|{y_{12}^d \over y_{22}^d}  -{y_{12}^u \over y_{22}^u}  \right| \epsilon, \nonumber \\
\sin \theta_{23}  \simeq |V_{cb}| &\simeq & \left|{y_{23}^d \over y_{33}^d}   -{y_{23}^u \over y_{33}^u}   \right| \epsilon^2 ,\nonumber \\
\sin \theta_{13}  \simeq |V_{ub}| &\simeq& \left|{y_{13}^d \over y_{33}^d}    -{y_{12}^u y_{23}^d \over y_{22}^u y_{33}^d}      
- {y_{13}^u \over y_{33}^u}   \right|   \epsilon^3.
\end{eqnarray}

To obtain appropriate neutrino masses, we introduce three right handed  neutrinos $\nu_{eR}$, $\nu_{\mu R}$,$\nu_{\tau R}$  to the SM.  
We note that the Dirac mass operators for neutrinos, which conserve the total lepton number, can be written as
\begin{eqnarray}
\label{mass1}
-{\mathcal{L}}_{\rm Yukawa}^{\nu} &=&       y_{11}^\nu \bar{ \psi}_{L_1}^\ell  H \nu_{e_R}  \left[  \dfrac{ \chi}{\Lambda} \right]^{25} + y_{12}^\nu \bar{ \psi}_{L_1}^\ell  H \nu_{\mu_R}  \left[  \dfrac{ \chi}{\Lambda} \right]^{22} + y_{13}^\nu \bar{ \psi}_{L_1}^\ell  H   \nu_{\tau_R}   \left[  \dfrac{ \chi}{\Lambda} \right]^{21} \nonumber\\
&+& y_{21}^\nu \bar{ \psi}_{L_2}^\ell  H \nu_{e_R} \left[  \dfrac{ \chi}{\Lambda} \right]^{24} 
+ y_{22}^\nu \bar{ \psi}_{L_2}^\ell  H \nu_{\mu_R} \left[  \dfrac{ \chi}{\Lambda} \right]^{21} + y_{23}^\nu \bar{ \psi}_{L_2}^\ell   H \nu_{\tau_R}   \left[  \dfrac{ \chi}{\Lambda} \right]^{20} \\
&+& y_{31}^\nu \bar{ \psi}_{L_3}^\ell  H \nu_{e_R} \left[  \dfrac{ \chi}{\Lambda} \right]^{24} + y_{32}^\nu \bar{ \psi}_{L_3}^\ell H \nu_{\mu_R} \left[  \dfrac{ \chi}{\Lambda} \right]^{21} + y_{33}^\nu \bar{ \psi}_{L_3}^\ell  H \nu_{\tau_R}  \left[  \dfrac{ \chi}{\Lambda} \right]^{20}
+  {\rm H.c.}. \nonumber
\end{eqnarray}
 The Dirac mass matrix for neutrinos now reads,
\begin{equation}
\label{NM}
\M_{\D} = \dfrac{v}{\sqrt{2}}
\begin{pmatrix}
y_{11}^\nu  \epsilon^{25} &  y_{12}^\nu \epsilon^{22} & y_{13}^\nu   \epsilon^{21}  \\
y_{21}^\nu  \epsilon^{24}  & y_{22}^\nu \epsilon^{21} &  y_{23}^\nu   \epsilon^{20} \\
y_{31}^\nu \epsilon^{24}   &  y_{32}^\nu  \epsilon^{21}   &  y_{33}^\nu \epsilon^{20}
\end{pmatrix}.
\end{equation}

We are allowed to write the pure Majorana mass operators for the left and right handed neutrinos.  
The left handed mass Lagrangian  $\mathcal{L}_{\rm M_L}$ is given by the following Weinberg operator:
\begin{eqnarray}
\label{mass1}
-{\mathcal{L}}_{\rm Weinberg}^{\ell} &=&       h_{11}^\nu  \frac{\bar{\tilde{\psi}}_{L_1}^{\ell }   H   \tilde{H}^\dagger \psi_{L_1}^{\ell }}{\Lambda}   \left[  \dfrac{ \chi^\dagger}{\Lambda} \right]^{6} 
+     h_{12}^\nu  \frac{\bar{\tilde{\psi}}_{L_1}^{\ell }    H   \tilde{H}^\dagger \psi_{L_2}^{\ell }}{\Lambda}   \left[  \dfrac{ \chi^\dagger}{\Lambda} \right]^{37}
+     h_{13}^\nu  \frac{\bar{\tilde{\psi}}_{L_1}^{\ell }    H   \tilde{H}^\dagger \psi_{L_3}^{\ell }}{\Lambda}   \left[  \dfrac{ \chi^\dagger}{\Lambda} \right]^{37}
\\ \nonumber 
&+&     h_{22}^\nu  \frac{\bar{\tilde{\psi}}_{L_2}^{\ell }    H   \tilde{H}^\dagger \psi_{L_2}^{\ell }}{\Lambda}   \left[  \dfrac{ \chi^\dagger}{\Lambda} \right]^{4}  
+  h_{23}^\nu  \frac{\bar{\tilde{\psi}}_{L_2}^{\ell }     H   \tilde{H}^\dagger \psi_{L_3}^{\ell }}{\Lambda}   \left[  \dfrac{ \chi^\dagger}{\Lambda} \right]^{4} 
+  h_{33}^\nu  \frac{\bar{\tilde{\psi}}_{L_3}^{\ell }     H   \tilde{H}^\dagger \psi_{L_3}^{\ell }}{\Lambda}   \left[  \dfrac{ \chi^\dagger}{\Lambda} \right]^{4} 
+
{\rm H.c.},
\end{eqnarray}
where $\tilde{\psi}_{L_i}^{\ell } = i \sigma_2 \psi_{L_i}^c  $.
The above Lagrangian creates the following  neutrino mass matrix,
\begin{align}
\label{NM}
\M_{L} & = \dfrac{v^2}{2 \Lambda }
\begin{pmatrix}
h_{11}^\nu  \epsilon^{6} &  h_{12}^\nu  \epsilon^{37} &  h_{13}^\nu  \epsilon^{37}  \\
 h_{12}^\nu  \epsilon^{37}  & h_{22}^\nu \epsilon^{4} &  h_{23}^\nu   \epsilon^{4} \\
 h_{13}^\nu  \epsilon^{37}   &  h_{23}^\nu  \epsilon^{4}   &  h_{33}^\nu \epsilon^{4}
\end{pmatrix}.
\end{align}
Let us note that $\Lambda \gg v$ in the realistic framework,  therefore the contribution of this mass matrix to neutrino masses is highly suppressed. 
We also have the right handed neutrino mass operators $\mathcal{L}_{\rm M_R}$ given by
\bea
\mathcal{L}_{\rm M_R}  &=& c_{11}  \chi \bar{\nu_{e_R}^c} \nu_{e_R} \left[  \frac{\chi }{\Lambda}  \right]^{43} + c_{12}  \chi \bar{\nu_{e_R}^c} \nu_{\mu_R} \left[  \frac{\chi}{\Lambda}  \right]^{40}  + c_{13}  \chi \bar{\nu_{e_R}^c} \nu_{\tau_R} \left[  \frac{\chi}{\Lambda}  \right]^{39}  \\ \nonumber
&+&  c_{22}  \chi \bar{\nu_{\mu_R}^c} \nu_{\mu_R} \left[  \frac{\chi}{\Lambda}  \right]^{37} +  c_{23}  \chi \bar{\nu_{\mu_R}^c} \nu_{\tau_R} \left[  \frac{\chi}{\Lambda}  \right]^{36}  +  c_{33}  \chi \bar{\nu_{\tau_R}^c} \nu_{\tau_R} \left[  \frac{\chi}{\Lambda}  \right]^{35} +
{\rm H.c.}.
\eea
The right-handed Majorana mass matrix $\M_{R}$ is,
\begin{equation}
\label{MR}
\M_{R} =  \frac{v_F}{\sqrt{2}}
\begin{pmatrix}
c_{11} \epsilon^{43} & c_{12} \epsilon^{40}   & c_{13} \epsilon^{39} \\
  c_{12} \epsilon^{40}  & c_{22} \epsilon^{37} & c_{23}  \epsilon^{38}\\
c_{13} \epsilon^{39}  & c_{23} \epsilon^{36}  & c_{33} \epsilon^{35}
\end{pmatrix}.
\end{equation}
We notice that $\M_{R}, \M_{L} << \M_{\D} $.  Therefore, neutrinos in our model are effectively Dirac and their mass matrix of the form (\ref{NM}) can lead naturally to the normal hierarchy masses given by
\begin{align}
\label{eqn5}
\{m_3, m_2,  m_1\} & \simeq \{|y_{33}^\nu| \epsilon^{20}, ~ \left|y_{22}^\nu- \frac {y_{23}^\nu y_{32}^\nu} {y_{33}^\nu} \right| \epsilon^{21},\\& ~  \left |y_{11}^\nu- \frac {y_{12}^\nu y_{21}^\nu}{y_{22}^\nu-y_{23}^\nu y_{32}^\nu/y_{33}^\nu}- \frac{y_{13}^\nu \left( y_{31}^\nu y_{22}^\nu-y_{21}^\nu y_{32}^\nu \right) -y_{31}^\nu y_{12}^\nu y_{23}^\nu}{ \left( y_{22}^\nu- y_{23}^\nu y_{32}^\nu/y_{33}^\nu \right) y_{33}^\nu} \right| \epsilon^{25}\}v/\sqrt{2}.\nonumber
\end{align}
From this, we can obtain the neutrino mass eigenvalues:
$\{m_3,m_2,m_1\} = \{50, 8.69, 5.71 \times 10^{-2} \}\, \text{meV}$
with the $y_{ij}^\nu$ couplings given in the appendix.   

The leptonic mixing angles are found to be,
\begin{eqnarray}
\sin \theta_{12}  &\simeq& \left|  {y_{12}^\ell \over y_{22}^\ell} -{y_{12}^\nu \over y_{22}^\nu}   \right| \epsilon, ~
\sin \theta_{23} \simeq  \left|  {y_{23}^\ell \over y_{33}^\ell} -  {y_{23}^\nu \over y_{33}^\nu}  \right|,~
\sin \theta_{13} \simeq \left| {y_{13}^\ell \over y_{33}^\ell}  - {y_{12}^\nu y_{23}^\ell\over y_{22}^\nu  y_{33}^\ell} -  {y_{13}^\nu \over y_{33}^\nu} \right|  \epsilon. 
\end{eqnarray}
From the above equation, we observe that the mixing angle  $\theta_{13}$ is of the order of the Cabibbo angle, and the mixing angle $\theta_{23} $ is of order one as expected from the structure of (\ref{NM}).

\section{The axial flavon as cold dark matter}
\label{flav_dark}

In the framework of $\mathcal{Z}_{\rm N} \times \mathcal{Z}_{\rm M}$, the power of the flavon field in the flavon potential (\ref{VN}) is given by the least common multiple of $N$ and $M$ which we denote by $\tilde N$. 
Then the axial flavon mass is 
\begin{equation} 
\label{mphi1}
 m_\varphi^2={1\over8} |\lambda| \tilde N^2 \epsilon^{\tilde N-4} v_F^2.
\end{equation}
The axial flavon could be misaligned from the true vacuum during inflation and its initial amplitude sits at some point in the range $\varphi_0=(-\pi, +\pi) v_F/\tilde N$. Then, after the inflation, the boson field rolls down to the true vacuum to produce cold dark matter density of coherent oscillation.   Considering the linear approximation of the scalar potential, the axial boson field amplitude follows the equation of motion in the expanding universe: 
\begin{equation}
   \ddot{\varphi}+3H \dot{\varphi} + m_\varphi^2 \varphi \approx 0,
\end{equation}
which has the solution $\varphi(t)=  \varphi_0  2^{1\over4} J_{1\over4}(m_\varphi t)/(m_\varphi t)^{1\over 4}$. 
Its energy density  $\rho_\varphi = {1\over2}(\dot\varphi^2+m_\varphi^2 \varphi^2) $ at later time ($m_\varphi t\to \infty$) becomes $\rho_\varphi \approx m^2_\varphi \varphi_0^2 \sqrt{2}\Gamma(5/4)^2/\pi (m_\varphi t)^{3/2}$. Equating this with the dark matter density, $\rho_\varphi = 0.24\, {\rm eV}^4$ at the matter-radiation equality time $t_{eq}$, that is, $m_\varphi t_{eq} \approx 2 \times 10^{27} (m_\varphi/{\rm eV})$, we find the relation,
\begin{equation}
\label{mphi2}
  m_\varphi =3.4\times 10^{-3} {\rm eV} \left( 10^{12} {\rm GeV} \over \varphi_0 \right)^4,
\end{equation}
to get the right dark matter density.
Comparing this with (\ref{mphi1}) one finds the relation
\begin{equation}
    v_F = \frac{2\times 10^7 \tilde N^{3/5} }{( a_0^8|\lambda| \epsilon^{\tilde N-4})^{1/10}} {\rm GeV}
\end{equation}
taking $\varphi_0= a_0 v_F/\tilde N$. Thus, the required axial flavon mass is
\begin{equation}
\label{flav_mass}
    m_\varphi = 1.5 \times 10^{16} \left( \epsilon^{\tilde N-4} \tilde N^4 \frac{|\lambda|}{a_0^2} \right)^{2/5} {\rm eV}.
\end{equation}
For our flavour symmetry discussed in the previous section, we have $\tilde N=88$. For  $|\lambda|=1$ and $a_0=1$,  we obtain 
\begin{align}
v_F  \approx 1.4 \times 10^{14}\, \rm{GeV}, ~~\text{and}~~
m_\varphi \approx   7 \times 10^{-3} \,  \rm{eV}.
\end{align}
  

\section{Phenomenology of flavonic dark matter}
\label{flav_couplings}
%\section{Decays and signatures of the flavonic dark matter}
\label{decay_sign}

The axial degree of freedom $\varphi$  of the flavon field $\chi$ remains light and contribute to the flavour changing processes as studied for the falvourful axion model \cite{Bjorkeroth:2018dzu}.  The similar calculation can be made also for our case with the discrete flavour symmetry breaking. Let us first note that our discrete symmetry enforces an automatic $U(1)$ symmetry under which the fermion fields $\psi^q_{L,i}$, $\psi^u_{R,i}$, $\psi^d_{R,i}$, $\psi^l_{L,i}$, and $\psi^l_{R,i}$ carry the following charges:
\begin{equation}
    x^q_i=(4,3,1),~ x^u_i=(4,1,0),~ x^d_i=(3,2,2),~ x^l_i=(3,2,2),~\mbox{and}~x^e_i=(6,4,3),
\end{equation}
respectively for $i=1,2,3$.  Therefore, the field transformation of $\psi^f_{L/R,i} \to \exp(i x^f_i \varphi/v_F) \psi^f_{L/R,i}$ for $f=q,u,d,l,e$ will induce the derivative couplings of the axial boson:
\begin{equation}
   -{\cal L}_\varphi = \frac{\partial_\mu \varphi}{v_F} \sum_{f,i} x^f_i \,\bar{\psi}^f_{L/R,i} \gamma^\mu \psi^f_{L/R,i}.
\end{equation}
Then, the mass diagonalization of the quarks and leptons, performed by the diagonalization matrices $U_{u,d}$ ($V_{u,d}$) for the left-handed (right-handed) up and down quarks, and $U_e$ ($V_e$) for the left-handed (right-handed) charged leptons, will lead to the following FCNC couplings:
\begin{equation}
   - {\cal L}_\varphi = \frac{\partial_\mu \varphi}{v_F} \sum_{f=u,d,e} 
   \bar{f}_i \left( \gamma^\mu V^f_{ij} -\gamma^\mu\gamma_5 A^f_{ij}\right) f_j,  
\end{equation}
where $V^{f}=X_L^{f}+X_R^{f}$ with $X_L^{u,d} = U^\dagger_{u,d} x^q U_{u,d}$, $X_R^{u,d} = V^\dagger_{u,d} x^{u,d} V_{u,d}$, and $X_L^{e} = U^\dagger_{e} x^l U_{e}$, $X_R^{e} = V^\dagger_{e} x^{e} V_{e}$.

The most stringent bound on the flavon scale $v_F$ comes from the FCNC process $K^+ \to \pi^+ \varphi$ \cite{Bjorkeroth:2018dzu}:
\begin{eqnarray}
    v_F \gtrsim 7 \times 10^{11} V^d_{21} {\rm GeV}, 
\end{eqnarray}
where we have $V^d_{21} \approx \epsilon$.
Notice that this bound is well below the value of $v_F$ required for the generation of the observed neutrino masses discussed in the previous section. 
  In the present work,  $v_F$ turns out to be of the order $10^{14}$ GeV which trivially satisfy the above bound. 
  The future sensitivity of the branching ratio of $K \rightarrow \pi \nu \bar{\nu}$ at NA62 is about $0.9 \times 10^{-10}$ and
the limit on $K\to \pi \varphi$ could be improved correspondingly, but only up to $v_F \sim 10^{12}$ GeV \cite{Bjorkeroth:2018dzu}.
  
  Since the pseudoscalar $\varphi$ is a dark matter particle,  its longevity is required against the allowed decays.  As the mass of this particle is of the order $10^{-3}$ eV, it may decay to light neutrinos or two photons. The decay rate of $\varphi \to \nu \nu$ is proportional to $ \left(20 \frac{v}{ v_F \sqrt{2}} \epsilon^{20} \right)^2 \approx 3.57 \times 10^{-47} \text{GeV}$. On the other side, the decay  $\varphi \rightarrow \gamma \gamma$ occurs through a top-quark triangle-loop, and its decay-width is proportional to $\frac{\alpha^2 m_\varphi^3}{256 \pi^3 v_F^2} \approx 10^{-70}$ GeV. These correspond to lifetime of $\varphi$ much larger than the age of the universe.

One of the interesting FCNC signatures is the decay $t \rightarrow c \varphi $ of the top quark.  However, it is expected to be negligibly small given the high scale of $v_F$.  More explicitly the branching ratio is given by
\begin{eqnarray}
    \frac{\Gamma (t \rightarrow c \varphi)}{\Gamma_t} \approx \frac{|g_{tc}|^2 m_t}{16 \pi \Gamma_t}  \left[ (1-\rho_c^2)^2    \right]   \approx 8 \times 10^{-26},
\end{eqnarray}
where we used the top quark decay width $\Gamma_t =  1.42$ GeV \cite{Zyla:2021},  $m_t =  173$ GeV, and $m_c = 1.27$ GeV , $\rho_c = m_c/m_t$,  and $g_{tc} = \frac{2 v \epsilon^2}{\sqrt{2} v_F}$.
%and $\rho_\varphi = m_\varphi/m_t$ being too small, is ignored. 



\section{Summary}
\label{sum}
The absence of any explanation to the discovery of DM is one of the most serious flaws in the framework of the SM.   
%This intriguing fact inspires to go beyond the SM framework where  DM may be a relativistic scalar field resulting in  an interesting scalar potential.    
Furthermore,  the flavour structure of the SM is an challenging theoretical puzzle.    This problem is bizarre in the sense that the mass hierarchy among  the second and third generation quarks is very different from that of the first generation quarks.  Moreover,  the quark mixing is also entirely different from the neutrino mixing.  A solution of the flavour problem should not only  produce an explanation for the charged fermion masses and mixing,  it must account for the neutrino masses and mixing.  

%We note that a relativistic field solution of the DM is not apparently  compatible with a solution of the flavour problem.  A relativistic field representing DM has to be weakly interacting, and its corresponding quanta accounting for the DM density must be stable.  On the other hand,  
A bosonic field called flavon may interact with the SM fermions to produce a hierarchical spectrum of fermionic masses  and required pattern of fermionic mixing.  The radial degree of the flavon decays quickly through its coupling to the SM fermions, but the axial degree can be practically stable to become a DM candidate. 
We have shown that a common solution to DM and the flavour problem of the SM is possible, and can be obtained through a flavonic Goldstone boson in a discrete symmetry framework  accounting for the flavour problem of the SM.

\section*{Acknowledgement}
EJC and GA are grateful for the support provided by CTP, Jamia Millia Islamia during their visit.  The work of GA is supported by the  Council of Science and Technology,  Govt. of Uttar Pradesh,  India through the  project ``   A new paradigm for flavour problem "  no.   CST/D-1301, and Science and Engineering Research Board, Department of Science and Technology, Government of India through the project `` Higgs Physics within and beyond the Standard Model" no. CRG/2022/003237. 

\section*{Appendix}
\begin{appendix}
    \section*{Benchmark points for the Yukawa couplings}
\label{benchmark}
For fitting fermion masses we use the following values of the fermion masses at $ 1$TeV \cite{Xing:2007fb},
\begin{eqnarray}
\{m_t, m_c, m_u\} &\simeq& \{150.7 \pm 3.4,~ 0.532^{+0.074}_{-0.073},~ (1.10^{+0.43}_{-0.37}) \times 10^{-3}\}~{\rm GeV}, \nonumber \\
\{m_b, m_s, m_d\} &\simeq& \{2.43\pm 0.08,~ 4.7^{+1.4}_{-1.3} \times 10^{-2},~ 2.50^{+1.08}_{-1.03} \times 10^{-3}\}~{\rm GeV},
\nonumber \\
\{m_\tau, m_\mu, m_e\} &\simeq& \{1.78\pm 0.2,~ 0.105^{+9.4 \times 10^{-9}}_{-9.3 \times 10^{-9}},~ 4.96\pm 0.00000043 \times 10^{-4}\}~{\rm GeV}.
\end{eqnarray}

The magnitudes and phases  of the CKM matrix are\cite{Zyla:2021},
\bea
|V_{ud}| &=& 0.97370 \pm 0.00014,  |V_{cb}| = 0.0410 \pm 0.0014, |V_{ub}| = 0.00382 \pm 0.00024, \\ \nonumber
\sin 2 \beta &=& 0.699 \pm 0.017, ~ \alpha = (84.9^{+5.1}_{-4.5})^\circ,~  \gamma = (72.1^{+4.1}_{-4.5})^\circ, \delta = 1.196^{+0.045}_{-0.043}
\eea

The neutrino  global fit data for the normal hierarchy are\cite{deSalas:2017kay},
\bea
\Delta m_{21}^2 &=& (7.55^{+0.59}_{-0.5}) \times 10^{-5} {\rm eV}^2, |\Delta m_{31}^2| = (2.50\pm 0.09) \times 10^{-3} \rm{eV}^2,  \\ \nonumber
\sin^2 \theta_{12} &=&  (3.20^{+0.59}_{-0.47}) \times 10^{-1},
 \sin^2 \theta_{23} =  (5.47^{+0.52}_{-1.02}) \times 10^{-1},  \sin^2 \theta_{13} =  (2.160^{+0.25}_{-0.20}) \times 10^{-2},
\eea
where range of errors is $3 \sigma$.

We scan the dimensionless coefficients $y_{ij}^{u,d,\ell,\nu}= |y_{ij}^{u,d,\ell,\nu}| e^{i \phi_{ij}^{q,\ell,\nu}}$  in the ranges $|y_{ij}^{u,d,\ell, \nu}| \in [0.9,2]$ and $ \phi_{ij}^{q,\ell,\nu} \in [0,2\pi]$.  The best fit results  are,
\begin{equation*}
y^u_{ij} =\begin{pmatrix}
0.12\, +1.44 i  & -0.38-0.82 i  &  0.99\, +0.0043 i  \\
-1.27-1.38 i  & -0.57 +0.83 i &  -1.22 -0.25 i    \\
-1.12 - 0.25 i & -1.12 - 0.43 i  & -2.70 - 2.63 i i
\end{pmatrix}, 
\end{equation*}

\begin{equation*}
y^d_{ij} = \begin{pmatrix}
-1.37 + 0.39 i & 0.31\, -1.29 i & 0.72\, +0.62 i   \\
-1.05 + 0.27 i & -0.44 + 0.85 i &  0.47\, -0.79 i  \\
-1.11 - 0.36 i & -0.81 + 0.42 i   &  0.80\, +0.82 i
\end{pmatrix},  
\end{equation*}

\begin{equation*}
y^\ell_{ij} = \begin{pmatrix}
-1.41 - 0.21 i & 1.14\, -0.0074 i & -0.78 + 0.45 i  \\
-0.89 + 0.16 i & -0.53 + 0.79 i &  -1.17 + 0.10 i \\
0.86\, +0.37 i &  0.90\, +0.0033 i  & 0.9
\end{pmatrix},
\end{equation*}


\begin{equation*}
y^\nu_{ij} = \begin{pmatrix}
-1.26+0.42 i & 0.51\, +1.40 i & -0.52-1.58 i \\
1.56 \, +0.72 i  &0.32\, +0.89 i & -0.2 +  i \\
1.9\, +0.01 i &  -0.73-0.57 i   & -0.01-1.68 i
\end{pmatrix}.
\end{equation*}

\section*{Origin of the  $\mathcal{Z}_{\rm N} \times \mathcal{Z}_{\rm M}$  flavour symmetry}
\label{origin}
The only purpose of this section is to show a possible theoretical origin of the  $\mathcal{Z}_{\rm N} \times \mathcal{Z}_{\rm M}$ flavour symmetry.   We employ the dark-technicolour (DTC) model  discussed in reference \cite{Abbas:2020frs} to achieve our objective.  Let us assume that there are three strong dynamics at a high scale given by the symmetry    $\mathcal{G} \equiv SU(\rm N_{\rm TC}) \times SU(\rm N_{\rm DTC}) \times SU(\rm{N}_{\rm F})$ where TC stands for technicolour,  DTC for dark-technicolour and F represents  a strong dynamics of vector-like fermions.     Moreover, we assume that there are $\rm K_{\rm TC}$ flavours transforming under the  symmetry $\mathcal{G}$ as \cite{Abbas:2020frs},
\begin{eqnarray}
T_{q}^i  &\equiv&   \begin{pmatrix}
T  \\
B
\end{pmatrix}_L:(1,2,0,\rm{N}_{\rm TC},1,1),  \\ \nonumber
T_{R}^i &:& (1,1,1,\text{N}_{\rm{TC}},1,1), B_{R}^i : (3,1,-1,\rm{N}_{\rm TC},1,1), 
\end{eqnarray}
where $i=1,2,3 \cdots \frac{\rm K_{\rm TC}}{2}$,  and the electric charges $+\frac{1}{2}$ for $T$ and $-\frac{1}{2}$ for $B$.

In a similar manner,  there are $\rm K_{\rm DTC}$ flavours of the $SU(\rm N_{\rm DTC}) $ symmetry  transforming under the symmetry $\mathcal{G}$ as \cite{Abbas:2020frs},
\begin{eqnarray}
 \mathcal{D}_{ q}^i &\equiv& \mathcal{C}_{L,R}^i  : (1,1, 1,1,\text{N}_{\text{DTC}},1),~\mathcal{S}_{L,R}^i  : (1,1,-1,1,\text{N}_{\text{DTC}},1), 
\end{eqnarray}
where  $i=1,2,3 \cdots \rm K_{\rm DTC}/2$,  and  electric charges $+\frac{1}{2}$ for $\mathcal C$ and $-\frac{1}{2}$ for $\mathcal S$.

The symmetry $SU(N_{\text{F}})$ have the $\rm K_{\rm F}$  fermionic flavours transforming under the symmetry $\mathcal{G}$ as \cite{Abbas:2020frs},
\begin{eqnarray}
F_{L,R} &\equiv &U_{L,R}^i \equiv  (3,1,\dfrac{4}{3},1,1,\text{N}_\text{F}),
D_{L,R}^{i} \equiv   (3,1,-\dfrac{2}{3},1,1,\text{N}_\text{F}),  \\ \nonumber 
N_{L,R}^i &\equiv&   (1,1,0,1,1,\text{N}_\text{F}), 
E_{L,R}^{i} \equiv   (1,1,-2,1,1,\text{N}_\text{F}),
\end{eqnarray}
where  $i=1,2,3 \cdots \rm K_{\rm F}/4$.

In the next step, we assume that there exists an extended-technicolour symmetry whose gauge sector is the mediator among TC, DTC and F fermions.  Then the mass of the top quark originate from the  Feynman diagrams shown in figure  \ref{fig:fig1}.  It is easier to generalise the mass mechanism shown in figure \ref{fig:fig1} to other fermions of the SM.   
\begin{figure}[H]
    \centering
\includegraphics[width=16cm]{fig1.png} 
  \caption{The interactions of the SM fermions $f$  with the TC,   DTC and F fermions are shown on the left where ETC stands for extended-technicolour scale.  The mass mechanism of the top-quark is shown on the right where the  $\langle \bar{T}T  \rangle$ is the TC chiral condensate responsible for the Higgs VEV,  and  the $\langle \bar{D}D \rangle$ is the chiral dark-technicolour  condensate responsible for the flavon VEV. }
    \label{fig:fig1}
\end{figure}


Additionally,  we observe that there are global $SU(\rm K_{\rm K_{\rm TC,  DTC, F}})_L \times SU(\rm K_{\rm K_{\rm TC,  DTC, F}} )_R \times U(1)_{\rm V}^{\rm TC, DTC, F}  \times U(1)_{\rm A}^{\rm TC, DTC, F} $ flavour symmetries in the theory due to $\rm K_{\rm TC,  DTC, F}$ massless flavours \cite{Abbas:2020frs}.   We note that in general an axial $U(1)_{\rm A}$  symmetry corresponding to a strong dynamics such as  $SU(\rm N_{\rm TC,  DTC, F})$  is broken by instantons of the corresponding   $SU(\rm N_{\rm TC,  DTC, F})$ gauge symmetry.  This will result in a  non-vanishing VEV for a $2 K_{\rm TC,  DTC, F}$-fermion operator having $2  K_{\rm TC,  DTC, F}$ quantum number  where $  K_{\rm TC,  DTC, F}$ is the number of flavours \cite{Harari:1981bs}. 


In our model there are three axial  $U(1)_{\rm A}$  symmetries, namely,   $U(1)_{\rm A}^{\rm TC }$,   $U(1)_{\rm A}^{\rm DTC }$  and $U(1)_{\rm A}^{\rm F }$.   Thus, these three axial symmetries are broken by the  instantons of the corresponding strong dynamics resulting a VEV for  the $2 \rm K_{\rm TC, DTC, F}$-fermion operators,  which does not have any other quantum number such as colour or flavour.   This  leads  to  the breaking of the $U(1)_{\rm A}^{\rm TC, DTC, F} $  axial  symmetry to $\mathcal{Z}_{2 (\rm K_{\rm TC, DTC, F})}$ conserved subgroup \cite{Harari:1981bs}.   Therefore, in our theory there are  $\mathcal{Z}_{\rm N} \times \mathcal{Z}_{\rm M} \times \mathcal{Z}_{\rm P}$  residual discrete symmetries where $\rm N= 2 K_{\rm TC}$,  $\rm M= 2 K_{\rm DTC}$, and $\rm P= 2 K_{\rm F}$.  The flavour symmetry  $\mathcal{Z}_{\rm 8} \times \mathcal{Z}_{\rm 22}$ can be obtained by choosing $K_{\rm TC} =4$, i.e. ,   two TC  doublets,  and  $K_{\rm DTC}= 11$.   The  VEV of the flavon field $\chi$ is a chiral condensate of the form $\langle \bar{D}D \rangle$.   The strong dynamics $SU(\rm N_{\rm F})$ acts like a bridge between the TC and the DTC sectors \cite{Abbas:2020frs}.




\end{appendix}



\begin{thebibliography}{99}

%\cite{Abbas:2017vws}
\bibitem{Abbas:2017vws}
G.~Abbas,
%``Solving the fermionic mass hierarchy of the standard model,''
Int. J. Mod. Phys. A \textbf{34}, no.20, 1950104 (2019)
doi:10.1142/S0217751X19501045
[arXiv:1712.08052 [hep-ph]].
%13 citations counted in INSPIRE as of 27 Dec 2022

%\cite{Abbas:2020frs}
\bibitem{Abbas:2020frs}
G.~Abbas,
%``Origin of the VEVs hierarchy,''
Int. J. Mod. Phys. A \textbf{37}, no.11n12, 2250056 (2022)
doi:10.1142/S0217751X22500567
[arXiv:2012.11283 [hep-ph]].
%1 citations counted in INSPIRE as of 16 Jun 2022



%\cite{Froggatt:1978nt}
\bibitem{Froggatt:1978nt} 
  C.~D.~Froggatt and H.~B.~Nielsen,
  %``Hierarchy of Quark Masses, Cabibbo Angles and CP Violation,''
  Nucl.\ Phys.\ B {\bf 147}, 277 (1979).
  doi:10.1016/0550-3213(79)90316-X
  %%CITATION = doi:10.1016/0550-3213(79)90316-X;%%
  %1529 citations counted in INSPIRE as of 20 Dec 2017
  









\bibitem{flavor_symm1}
  M.~Leurer, Y.~Nir and N.~Seiberg,
  %``Mass matrix models,''
  Nucl.\ Phys.\ B {\bf 398}, 319 (1993);
  %doi:10.1016/0550-3213(93)90112-3
  %[hep-ph/9212278].
  %%CITATION = doi:10.1016/0550-3213(93)90112-3;%%
 M.~Leurer, Y.~Nir and N.~Seiberg,
  %``Mass matrix models: The Sequel,''
  Nucl.\ Phys.\ B {\bf 420}, 468 (1994).
  %doi:10.1016/0550-3213(94)90074-4
  %[hep-ph/9310320].
  %%CITATION = doi:10.1016/0550-3213(94)90074-4;%%

  %\cite{Chun:1996xv}
\bibitem{Chun:1996xv}
E.~J.~Chun and A.~Lukas,
%``Quark and lepton mass matrices from horizontal U(1) symmetry,''
Phys. Lett. B \textbf{387}, 99-106 (1996)
doi:10.1016/0370-2693(96)01015-5
[arXiv:hep-ph/9605377 [hep-ph]].
  
  \bibitem{flavor_symm2}
   K.~S.~Babu and S.~Nandi,
  %``Natural fermion mass hierarchy and new signals for the Higgs boson,''
  Phys.\ Rev.\ D {\bf 62}, 033002 (2000)
  doi:10.1103/PhysRevD.62.033002
  [hep-ph/9907213].
  %%CITATION = doi:10.1103/PhysRevD.62.033002;%%
  %58 citations counted in INSPIRE as of 20 Dec 2017
  
  
  \bibitem{flavor_symm3}
  G.~G.~Ross and L.~Velasco-Sevilla,
%``Symmetries and fermion masses,''
Nucl. Phys. B \textbf{653}, 3-26 (2003)
doi:10.1016/S0550-3213(03)00041-5
[arXiv:hep-ph/0208218 [hep-ph]].
%96 citations counted in INSPIRE as of 11 Feb 2021
S.~F.~King,
%``A model of quark and lepton mixing,''
JHEP \textbf{01}, 119 (2014)
doi:10.1007/JHEP01(2014)119
[arXiv:1311.3295 [hep-ph]].
%51 citations counted in INSPIRE as of 11 Feb 2021
G.~F.~Giudice and O.~Lebedev,
%``Higgs-dependent Yukawa couplings,''
Phys. Lett. B \textbf{665}, 79-85 (2008)
doi:10.1016/j.physletb.2008.05.062
[arXiv:0804.1753 [hep-ph]].
%139 citations counted in INSPIRE as of 11 Feb 2021


%\cite{Davidson:1983fy}
\bibitem{Davidson:1983fy}
A.~Davidson, V.~P.~Nair and K.~C.~Wali,
%``{Peccei-Quinn} Symmetry as Flavor Symmetry and Grand Unification,''
Phys. Rev. D \textbf{29}, 1504 (1984)
doi:10.1103/PhysRevD.29.1504
%35 citations counted in INSPIRE as of 02 Oct 2022

%\cite{Davidson:1987tr}
\bibitem{Davidson:1987tr}
A.~Davidson and K.~C.~Wali,
%``Family Mass Hierarchy From Universal Seesaw Mechanism,''
Phys. Rev. Lett. \textbf{60}, 1813 (1988)
doi:10.1103/PhysRevLett.60.1813
%84 citations counted in INSPIRE as of 02 Oct 2022


\bibitem{higgs_coup}
  H.~Georgi and S.~L.~Glashow,
  %``Attempts to calculate the electron mass,''
  Phys.\ Rev.\ D {\bf 7}, 2457 (1973).
  %doi:10.1103/PhysRevD.7.2457
  %%CITATION = doi:10.1103/PhysRevD.7.2457;%%



\bibitem{wf_local}
  T.~Gherghetta and A.~Pomarol,
  %``Bulk fields and supersymmetry in a slice of AdS,''
  Nucl.\ Phys.\ B {\bf 586}, 141 (2000);
  %doi:10.1016/S0550-3213(00)00392-8
  %[hep-ph/0003129].
  %%CITATION = doi:10.1016/S0550-3213(00)00392-8;%%
 Y.~Grossman and M.~Neubert,
  %``Neutrino masses and mixings in nonfactorizable geometry,''
  Phys.\ Lett.\ B {\bf 474}, 361 (2000);
  %doi:10.1016/S0370-2693(00)00054-X
  %[hep-ph/9912408].
  %%CITATION = doi:10.1016/S0370-2693(00)00054-X;%%
 M.~Blanke, A.~J.~Buras, B.~Duling, S.~Gori and A.~Weiler,
  %``$\Delta$ F=2 Observables and Fine-Tuning in a Warped Extra Dimension with Custodial Protection,''
  JHEP {\bf 0903}, 001 (2009);
  %doi:10.1088/1126-6708/2009/03/001
  %[arXiv:0809.1073 [hep-ph]].
  %%CITATION = doi:10.1088/1126-6708/2009/03/001;%%
 S.~Casagrande, F.~Goertz, U.~Haisch, M.~Neubert and T.~Pfoh,
 %``Flavor Physics in the Randall-Sundrum Model: I. Theoretical Setup and Electroweak Precision Tests,''
  JHEP {\bf 0810}, 094 (2008);
  %doi:10.1088/1126-6708/2008/10/094
  %[arXiv:0807.4937 [hep-ph]].
  %%CITATION = doi:10.1088/1126-6708/2008/10/094;%%
 M.~Bauer, S.~Casagrande, U.~Haisch and M.~Neubert,
 %``Flavor Physics in the Randall-Sundrum Model: II. Tree-Level Weak-Interaction Processes,''
  JHEP {\bf 1009}, 017 (2010).
  %doi:10.1007/JHEP09(2010)017
  %[arXiv:0912.1625 [hep-ph]].
  %%CITATION = doi:10.1007/JHEP09(2010)017;%%


\bibitem{partial_comp}
  D.~B.~Kaplan,
 %``Flavor at SSC energies: A New mechanism for dynamically generated fermion masses,''
 Nucl.\ Phys.\ B {\bf 365}, 259 (1991).
 %doi:10.1016/S0550-3213(05)80021-5
 %%CITATION = doi:10.1016/S0550-3213(05)80021-5;%%
  
  

%\cite{Fuentes-Martin:2022xnb}
\bibitem{Fuentes-Martin:2022xnb}
J.~Fuentes-Martin, G.~Isidori, J.~M.~Lizana, N.~Selimovic and B.~A.~Stefanek,
%``Flavor hierarchies, flavor anomalies, and Higgs mass from a warped extra dimension,''
Phys. Lett. B \textbf{834}, 137382 (2022)
doi:10.1016/j.physletb.2022.137382
[arXiv:2203.01952 [hep-ph]].
%25 citations counted in INSPIRE as of 06 Mar 2023


%\cite{Higaki:2019ojq}
\bibitem{Higaki:2019ojq}
T.~Higaki and J.~Kawamura,
%``A low-scale flavon model with a $Z_N$ symmetry,''
JHEP \textbf{03}, 129 (2020)
doi:10.1007/JHEP03(2020)129
[arXiv:1911.09127 [hep-ph]].
%9 citations counted in INSPIRE as of 06 Mar 2023

%\cite{Calibbi:2016hwq}
\bibitem{Calibbi:2016hwq}
L.~Calibbi, F.~Goertz, D.~Redigolo, R.~Ziegler and J.~Zupan,
%``Minimal axion model from flavor,''
Phys. Rev. D \textbf{95}, no.9, 095009 (2017)
doi:10.1103/PhysRevD.95.095009
[arXiv:1612.08040 [hep-ph]].

%\cite{Ema:2016ops}
\bibitem{Ema:2016ops}
Y.~Ema, K.~Hamaguchi, T.~Moroi and K.~Nakayama,
%``Flaxion: a minimal extension to solve puzzles in the standard model,''
JHEP \textbf{01}, 096 (2017)
doi:10.1007/JHEP01(2017)096
[arXiv:1612.05492 [hep-ph]].

%\cite{Bjorkeroth:2017tsz}
\bibitem{Bjorkeroth:2017tsz}
F.~Bj\"orkeroth, E.~J.~Chun and S.~F.~King,
%``Accidental Peccei\textendash{}Quinn symmetry from discrete flavour symmetry and Pati\textendash{}Salam,''
Phys. Lett. B \textbf{777}, 428-434 (2018)
doi:10.1016/j.physletb.2017.12.058
[arXiv:1711.05741 [hep-ph]].

\bibitem{Kim:2008hd}
For a review, see, e.g., 
J.~E.~Kim and G.~Carosi,
%``Axions and the Strong CP Problem,''
Rev. Mod. Phys. \textbf{82}, 557-602 (2010)
[erratum: Rev. Mod. Phys. \textbf{91}, no.4, 049902 (2019)]
doi:10.1103/RevModPhys.82.557
[arXiv:0807.3125 [hep-ph]].

\bibitem{misalign}
J.~Preskill, M.~B.~Wise and F.~Wilczek,
%``Cosmology of the Invisible Axion,''
Phys. Lett. B \textbf{120}, 127-132 (1983)
doi:10.1016/0370-2693(83)90637-8;
L.~F.~Abbott and P.~Sikivie,
%``A Cosmological Bound on the Invisible Axion,''
Phys. Lett. B \textbf{120}, 133-136 (1983)
doi:10.1016/0370-2693(83)90638-X;
M.~Dine and W.~Fischler,
%``The Not So Harmless Axion,''
Phys. Lett. B \textbf{120}, 137-141 (1983)
doi:10.1016/0370-2693(83)90639-1.

%\cite{Abbas:2018lga}
\bibitem{Abbas:2018lga}
G.~Abbas,
%``A new solution of the fermionic mass hierarchy of the standard model,''
Int. J. Mod. Phys. A \textbf{36}, 2150090 (2021)
doi:10.1142/S0217751X21500901
[arXiv:1807.05683 [hep-ph]].
%7 citations counted in INSPIRE as of 22 Dec 2021



%\cite{Abbas:2022zfb}
\bibitem{Abbas:2022zfb}
G.~Abbas, V.~Singh, N.~Singh and R.~Sain,
%``Flavour bounds on the flavon of a minimal and a non-minimal $\mathcal{Z}_2 \times \mathcal{Z}_N$ symmetry,''
[arXiv:2208.03733 [hep-ph]].
%0 citations counted in INSPIRE as of 22 Dec 2022







%\cite{Rasin:1998je}
\bibitem{Rasin:1998je}
A.~Rasin,
%``Hierarchical quark mass matrices,''
Phys. Rev. D \textbf{58}, 096012 (1998)
doi:10.1103/PhysRevD.58.096012
[arXiv:hep-ph/9802356 [hep-ph]].
%30 citations counted in INSPIRE as of 28 Jan 2023


 






%\cite{Bjorkeroth:2018dzu}
\bibitem{Bjorkeroth:2018dzu}
F.~Bj\"orkeroth, E.~J.~Chun and S.~F.~King,
%``Flavourful Axion Phenomenology,''
JHEP \textbf{08}, 117 (2018)
doi:10.1007/JHEP08(2018)117
[arXiv:1806.00660 [hep-ph]].
%50 citations counted in INSPIRE as of 28 Jan 2023




 \bibitem{Zyla:2021}
P.A. Zyla et al. (Particle Data Group), Prog. Theor. Exp. Phys. 2020, 083C01 (2020) and 2021 update 


 

%\cite{Xing:2007fb}
\bibitem{Xing:2007fb} 
  Z.~z.~Xing, H.~Zhang and S.~Zhou,
  %``Updated Values of Running Quark and Lepton Masses,''
  Phys.\ Rev.\ D {\bf 77}, 113016 (2008)
  doi:10.1103/PhysRevD.77.113016
  [arXiv:0712.1419 [hep-ph]].
  %%CITATION = doi:10.1103/PhysRevD.77.113016;%%
  %341 citations counted in INSPIRE as of 01 Mar 2019
  
  
  
  
  
 
  


 %\cite{deSalas:2017kay}
\bibitem{deSalas:2017kay} 
  P.~F.~de Salas, D.~V.~Forero, C.~A.~Ternes, M.~Tortola and J.~W.~F.~Valle,
  %``Status of neutrino oscillations 2018: 3$\sigma$ hint for normal mass ordering and improved CP sensitivity,''
  Phys.\ Lett.\ B {\bf 782}Becirevic:2001xt, 633 (2018)
  doi:10.1016/j.physletb.2018.06.019
  [arXiv:1708.01186 [hep-ph]].
  %%CITATION = doi:10.1016/j.physletb.2018.06.019;%%
  %347 citations counted in INSPIRE as of 26 Oct 2019


%\cite{Harari:1981bs}
\bibitem{Harari:1981bs}
H.~Harari and N.~Seiberg,
%``Generation Labels in Composite Models for Quarks and Leptons,''
Phys. Lett. B \textbf{102}, 263-266 (1981)
doi:10.1016/0370-2693(81)90871-6
%79 citations counted in INSPIRE as of 27 Jun 2021



\end{thebibliography}





\end{document}
