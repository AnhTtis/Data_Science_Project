\section{Threat Model}
\label{sec:model}
\ac{TDX} operates on the assumption that adversaries may have physical or remote access to a computer, and may be able to gain control over the boot firmware, \ac{SMM}, host operating system, hypervisor, and peripheral devices. The primary objective of these adversaries is to obtain confidential data or interfere with the execution of a \ac{TD}. It is important to note that \ac{TDX} cannot guarantee availability, as adversaries can control all the compute resources for \acp{TD} and launch \ac{DoS} attacks. It is crucial for the \ac{TDX} design to prevent adversaries from conducting actions that compromise the \ac{TDX} security guarantees outlined in \S\ref{sec:principle:guarantees}. Below, we summarize the capabilities of adversaries and identify potential attack vectors and scenarios.

Adversaries can interact with the \ac{TDX} Module through its host-side interface functions, which allow them to build, initialize, measure, and tear down \acp{TD}. These interface functions can be invoked in an arbitrary order with semantically and syntax valid/invalid inputs. 

Adversaries can control the compute resources assigned to \acp{TD}, including physical memory pages, processor time, and physical/virtual devices. They can interrupt \acp{TD} at any point, and try to read and write to arbitrary memory locations, as well as reconfigure the \ac{IOMMU}. 

Adversaries have the capability of manipulating the input data for \acp{TD}~\cite{tdxkernelhardening}, including \ac{ACPI} tables, \ac{PCI} config, \ac{MSR}, \ac{MMIO}, \ac{DMA}, emulated devices, hypercalls handled by the host, source of randomness, and time notion. 

Adversaries can conduct physical and hardware attacks, for instance, by probing buses or accessing main memory through malicious \ac{DMA}. There is no defense against physical attacks that roll back arbitrary memory regions. However, it should not be possible for adversaries to extract the secret key material baked into the processor chip's fuses. The scope of the threat model does not cover fault injections or side-channel attacks such as power glitches, time and power analysis.

Attacking \ac{TDX} attestation is within the scope as it undermines the trust model and may enable adversaries to forge a counterfeit \ac{TEE} for collecting confidential information from tenants. 

\nip{\acf{TCB}.}
The \ac{TCB} of \ac{TDX} consists of the TDX-enabled Intel processors and the built-in technologies, such as \ac{VT}, \ac{MKTME}, and \ac{SGX}. The \ac{TCB} also includes software modules signed by Intel, including the \ac{TDX} Module, NP/P-SEAM Loaders, and architectural \ac{SGX} enclaves for remote attestation. The software stacks running within \acp{TD} are owned by the tenant and are considered part of the \ac{TCB}. The cryptographic primitives used in \ac{TDX} are considered sound and its implementation secure, including the generation of random numbers and the absence of side-channel attacks like timing attack.

Tenants must trust the processor manufacturer, Intel, for developing, manufacturing, building, and signing of the hardware/software components used by \ac{TDX}. The source code packages for the \ac{TDX} Module, the NP/P-SEAM Loaders, and the \ac{DCAP} for attestation are publicly available for audit purposes, allowing tenants to assess their trustworthiness. However, tenants must also trust that the version signed by Intel is equivalent to the one they have reviewed, which involves placing trust in the compilation process to protect against supply chain attacks.


Moreover, tenants are required to trust Intel's \ac{PCS} for remote attestation. The \ac{PCS}, which originally supported \ac{SGX} attestation, has been expanded to include retrieval of \ac{PCK} certificates, revocation lists, and \ac{TCB} information for \ac{TDX}.