\section{Security Principles} 
\label{sec:principle}
In cloud computing, multiple security domains, \eg a hypervisor managed by a cloud service provider and \acp{VM} owned by different tenants, coexist on a shared physical machine. While hardware-assisted virtualization can isolate tenants' workloads, the security model still relies on a privileged hypervisor to provide trustworthy \ac{VM} management. 
To address this issue, \ac{TDX} enforces cryptographic isolation among the security domains, thereby mitigating cross-domain attacks. This eliminates hierarchical dependencies on untrusted/privileged host software and excludes the hypervisor and cloud operators from the \ac{TCB}, allowing tenants to securely provision and run their computations with confidence.

\label{sec:principle:guarantees}
\ac{TDX} guarantees confidentiality and integrity of \ac{TD}'s memory and virtual CPU states,  ensuring that they cannot be accessed or tampered with by other security domains executing on the same machine. This is achieved through a combination of: (1) memory access control, (2) runtime memory encryption, and (3) an Intel-signed \ac{TDX} Module that handles security-sensitive \ac{TD} management operations. 

In addition, remote attestation provides tenants with proof of the authenticity of \acp{TD} executing on genuine \ac{TDX}-enabled Intel processors. These guarantees are based on a specific threat model and require certain trust assumptions, as described in \S\ref{sec:model}.

\nip{Memory Confidentiality.} \ac{TD}'s data residing inside the processor package are stored in clear text. However, when the data is offloaded from the processor to the main memory, the processor encrypts it using a TD-specific cryptographic key known only to the processor. The encryption is performed at the cache line granularity (as described in \S\ref{memoryprotection:mktme}), making it impossible for peripheral devices to read or tamper with the \ac{TD}'s private memory without detection. The processor is able to detect any tampering that may occur when loading data from the main memory.

\nip{CPU State Confidentiality.} \ac{TDX} protects against concurrently executing processes by managing the virtual CPU states of \acp{TD} during all context switches between security domains. The states are stored in the \ac{TD}'s metadata, which are protected while being in the main memory using the \ac{TD}'s key. During context switches, \ac{TDX} clears or isolates the \ac{TD}-specific states from internal processor registers and buffers, such as \ac{TLB} entries or branch prediction buffers, to maintain the protection of the \ac{TD}'s information.

\nip{Execution Integrity.} \ac{TDX} protects the integrity of \ac{TD}'s execution from host interference, ensuring that the \ac{TD} resumes its computation after an interrupt at the expected instruction within the expected states. It is capable of detecting malicious changes in the virtual CPU states, as well as injection, modification, or removal of instructions located in the private memory. However, \ac{TDX} does not provide additional guarantees for the control flow integrity. It is the responsibility of the \ac{TD} owner to use existing compilation-based or hardware-assisted control flow integrity enforcement techniques, such as \ac{CET}~\cite{cetwhitepaper}.

\nip{I/O Protection.} Peripheral devices or accelerators are outside the trust boundaries of \acp{TD} and should not be allowed to access \ac{TD}'s private memory. To support virtualized I/O, a \ac{TD} can choose to explicitly share memory for data transfer purposes. However, \ac{TDX} does not provide any confidentiality and integrity protection for the data located in shared memory regions. It is the responsibility of \ac{TD} owners to implement proper mechanisms, such as using secure communication channels like \ac{TLS}, to protect the data that leaves the \ac{TD}'s trust boundary. In the future, \ac{TDX} 2.0 is planned to include trusted I/O virtualization~\cite{deviceattestation,tdxio} to address these issues. 