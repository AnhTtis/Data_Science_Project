\section{Comparison of Confidential Computing Technologies} 
\label{sec:related}
Confidential computing technologies share a common objective of protecting outsourced sensitive data and computations from unauthorized access, tampering, and disclosure on untrusted third-party infrastructures. Major processor vendors are competing to incorporate confidential computing capabilities into their chips. Despite differences in implementation and terminology, these technologies share fundamental security principles with similar system designs, such as introducing new execution modes or privilege levels, migrating \ac{VM} management functions to attested firmware/software, ensuring secure or measured launch of trusted components, enforcing memory access control, and providing memory encryption protection.

In addition to Intel \ac{TDX}, we provide a brief overview of the confidential computing technologies from other vendors, including AMD \ac{SEV}, IBM Secure Execution and \ac{PEF}, Arm \ac{CCA}, and RISC-V \ac{AP-TEE}, for comparison purposes. We have summarized the distinct features of these technologies in Table~\ref{tab:summary-technologies-other}. Readers already familiar with these technologies can skip this section and proceed directly to \S\ref{sec:background}, where we explain the existing Intel technologies that support \ac{TDX}.

\begin{table}[t!]
\caption{Summary of Comparable Confidential Computing Technologies}
\label{tab:summary-technologies-other}
\small
\begin{tabular}{l|l}
\hline
\textbf{Technology} & \textbf{Summary} \\ \hline
AMD SEV~\cite{kaplan2016amd,kaplan2017protecting,sev2020strengthening} & \begin{tabular}[c]{@{}l@{}} - enforces cryptographic \ac{VM} isolation via AMD PSP\\ - supports memory encryption (\ac{SEV}), CPU state encryption (SEV-ES), integrity protection (SEV-SNP)\\
- provides hardware isolated layers within VMs through VMPL\end{tabular}\\ \hline
\begin{tabular}[c]{@{}l@{}}IBM Secure\\ Execution~\cite{ibmsecureexec}\end{tabular} & \begin{tabular}[c]{@{}l@{}}- protects SVMs on IBM Z and LinuxONE.\\ - leverages a trusted firmware, Ultravisor, to bootstrap and run SVMs \\ - provides end-to-end protection from the boot image to memory and throughout execution\end{tabular} \\ \hline
IBM PEF~\cite{hunt2021confidential} & \begin{tabular}[c]{@{}l@{}}- protects SVMs on Power ISA\\ - leverages the Protected Execution Ultravisor to manage SVM execution\\ - utilizes TPM, secure boot, and trusted boot for integrity check and bootstrap SVMs\end{tabular} \\ \hline
Arm CCA~\cite{li2022design} & \begin{tabular}[c]{@{}l@{}}- introduces Realm world for running confidential VMs\\ - introduces Root world to enforce address space isolation through GPT\\- support attestation for Realm environment\end{tabular} \\ \hline
RISC-V AP-TEE~\cite{riscvcc} & \begin{tabular}[c]{@{}l@{}} - introduces the TSM to manage TVM life-cycles \\ - uses MTT to track memory page assignment\\ - adopts a layered attestation architecture \end{tabular} \\\hline
\end{tabular}
\end{table}

\subsection{AMD SEV} 
\acf{SEV}~\cite{kaplan2016amd} is a confidential computing feature in AMD EPYC processors. It protects sensitive data stored within \acp{VM} from privileged software or administrators in a multi-tenant cloud environment.  \ac{SEV} relies on AMD \ac{SME} and AMD Virtualization (AMD-V) to enforce cryptographic isolation between \acp{VM} and the hypervisor. Each \ac{VM} is assigned a unique ephemeral \ac{AES} key, which is used for runtime memory encryption. The \ac{AES} engine in the on-die memory controller encrypts or decrypts data written to or read from the main memory. The per-\ac{VM} keys are managed by the AMD \ac{PSP}, which is a 32-bit Arm Cortex-A5 micro-controller integrated within the AMD \ac{SoC}. The C-bit (bit 47) in physical addresses determines memory page encryption. \ac{SEV} also provides a remote attestation mechanism that allows the \ac{VM} owners to verify the trustworthiness of \acp{VM}' launch measurements and the \ac{SEV} platforms. The \ac{PSP} generates the attestation report signed by an AMD certified attestation key. The \ac{VM} owners can verify the authenticity of the attestation report and the embedded platform/guest measurements.

AMD has released three generations of \ac{SEV}. The first generation \ac{SEV}\cite{kaplan2016amd} only protects the confidentiality of a \ac{VM}'s memory. The second generation SEV-ES (Encrypted State)\cite{kaplan2017protecting} adds protection for CPU register state during hypervisor transition, and the third generation SEV-SNP (Secure Nested Paging)\cite{sev2020strengthening} adds integrity protection to prevent memory corrupting, replaying, and remapping attacks. Particularly, SEV-SNP provides memory integrity protection using \acf{RMP}. \ac{RMP} tracks each page's ownership and permissions to prevent unauthorized access. SEV-SNP also introduces the \acp{VMPL} feature by dividing the guest address space into four levels and providing additional security isolation within a \ac{VM}. The privilege levels range from zero to three, where \ac{VMPL}0 is the highest level of privilege and \ac{VMPL}3 is the lowest. For instance, the Linux Secure \ac{VM} Service Module (SVSM)~\cite{linuxsvsm} makes extensive use of the \ac{RMP} and \ac{VMPL} features to perform sensitive services, \eg live migration and vTPM, in a secure manner. 

\subsection{IBM Confidential Computing}
IBM's early exploration of confidential computing can be traced back to the research on SecureBlue++~\cite{williams2011cpu, boivie2012secureblue++}, which included running on an emulated POWER processor on the Mambo CPU simulator~\cite{bohrer2004mambo}. Today, IBM Systems support two architectures for confidential computing: Secure Execution~\cite{ibmsecureexec}, offered on IBM Z and LinuxONE, and \acf{PEF}~\cite{hunt2021confidential,pef}, released as an open source project on OpenPOWER systems.

\nip{IBM Secure Execution.}
IBM Secure Execution provides support for \acp{SVM} that run inside isolated \acp{TEE} since IBM Z15 and LinuxONE III. Secure Execution protects the confidentiality, integrity and authenticity of code and data in an \ac{SVM} from any unauthorized access and snooping or tampering. Secure Execution leverages trusted firmware, called the \spec{Ultravisor}, to perform security-sensitive tasks to bootstrap and run \acp{SVM}. The Ultravisor shields the \ac{SVM}'s memory and its state during context switches and protects the \ac{SVM} from a potentially compromised or malicious hypervisor. Tenants using Secure Execution can embed their encrypted sensitive data in the \ac{VM} images and rely on the Ultravisor to decrypt and expose them to the \acp{SVM} executing inside the \acp{TEE}. Specifically, tenants can encrypt their confidential data with a symmetric \spec{data key}, which they embed in the \spec{IBM Secure Execution header}. They further encrypt this header with the key obtained from the verified \spec{host key document} and embed the header in their \ac{VM} image. The header can contain multiple key slots that allow an image to run on multiple target hosts. The host key document, signed by the hardware manufacturer, contains the public key linked with the private key embedded in the hardware of IBM Z or LinuxONE. Ultravisor, the only component having access to the hardware private key and the data key, enforces that only the expected tenant's \ac{SVM} executing inside the \ac{TEE} has access to the unencrypted data. In addition to embedding built-in secrets within the \ac{VM} image, Secure Execution also supports remote attestation starting from IBM Z16 and LinuxONE Emperor 4. This allows tenants to verify the \ac{SVM}'s measurements before releasing their secrets.

\nip{IBM \ac{PEF}.} 
\ac{PEF} provides a \ac{VM}-based \ac{TEE} using extensions to the IBM Power \ac{ISA} that are supported in most POWER9 and POWER10 processors. \ac{PEF} firmware, tooling to prepare \acp{SVM}, and OS extensions, were released as open source software~\cite{ultravisorsrc}. To protect sensitive data and code, \ac{PEF} introduces a trusted firmware called \spec{Protected Execution Ultravisor} (Ultravisor) that shields the \ac{SVM} execution and enforces the security guarantees with the help of the CPU architectural changes. 
The \ac{PEF} relies on secure and trusted boot of the system and the Ultravisor executing in a new, highest privileged CPU state called \emph{secure state}. The hypervisor starts the \ac{VM}, which invokes the Ultravisor to transition to an \ac{SVM} using the \ac{ESM} call. The Ultravisor converts the \ac{VM} into an \ac{SVM} by moving it to the secure memory that is  inaccessible to untrusted code. Before executing the \ac{SVM}, the Ultravisor performs integrity checking. It decrypts the payload attached to the \ac{SVM} image to decode the integrity information and a passphrase for the encrypted file system.  After ensuring the integrity of the \ac{SVM}, the Ultravisor exposes the passphrase to the \ac{SVM} booting system that decrypts the tenant's file system. The Ultravisor uses the \ac{TPM} to get access to the symmetric seed required to check integrity and decrypt the payload. The symmetric seed is guarded using the \ac{PCR} sealing mechanism and accessed by establishing a secure channel to the \ac{TPM}. The \ac{TPM} only grants access for an Ultravisor on a correctly booted system. If the Ultravisor gets access to the symmetric seed, it generates the HMAC key and symmetric key that are used to verify integrity and decrypt the passphrase. 



\subsection{Arm CCA} 
\acf{CCA}~\cite{li2022design} was introduced in the Armv9 architecture. Traditionally, Arm TrustZone allows secure execution by having two separated worlds, the \spec{Normal World} and the \spec{Secure World}. TrustZone prevents software in \spec{Normal World} from accessing data in the \spec{Secure World}. \ac{CCA} introduces the \ac{RME} with two additional worlds, the \spec{Realm World} and the \spec{Root World}. The \spec{Realm World} provides mutually distrusting execution environments for confidential \acp{VM}, isolating workloads from any other security domains, including host operating systems, hypervisors, other Realms and the TrustZone. 
To enforce isolation of address spaces, \ac{CCA} uses a \ac{GPT}, which is an extension to the page table that tracks the ownership of each page with different worlds. The \spec{Monitor} in the \spec{Root World} handles the creation and management of the \ac{GPT}, preventing hypervisors or operating systems from directly changing it. The \spec{Monitor} can dynamically move physical memory between different worlds by updating the \ac{GPT}. \ac{CCA} also supports attestation to measure and verify the \ac{CCA} platform and the initial state of the Realms.

\subsection{RISC-V AP-TEE}
\acf{AP-TEE}~\cite{riscvcc} is a reference confidential computing architecture for RISC-V. Its protected instance is called a \ac{TVM}. The architecture introduces the \ac{TSM} driver, which is a M-mode (highest privilege level in RISC-V) firmware component for switching between confidential and non-confidential environments. The \ac{TSM} driver tracks the assignment of memory pages to \acp{TVM} through the \ac{MTT}. The \ac{TSM} driver measures and loads the \ac{TSM}, which is a trusted intermediary between the hypervisor and the \acp{TVM}. \ac{AP-TEE} defines the \ac{ABI} for the hypervisor to request virtual machine management services from the \ac{TSM}. \ac{AP-TEE} adopts a layered attestation architecture, which begins with the hardware and progresses through the \ac{TSM} driver, \ac{TSM}, and \ac{TVM}. Each layer is loaded, measured, and certified by the previous layer. This approach provides a secure chain of trust that can be used to verify the integrity of the system. The \ac{TVM} can obtain a certificate from the \ac{TSM} that contains attestation evidence rooted back to the hardware. This certificate provides a mechanism for verifying the authenticity of the \ac{TVM} and the software it runs.