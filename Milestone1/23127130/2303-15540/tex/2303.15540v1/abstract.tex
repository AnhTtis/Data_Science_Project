\begin{abstract}\normalsize
Intel \ac{TDX} is a new architectural extension in the 4th Generation Intel Xeon Scalable Processor that supports confidential computing. \ac{TDX} allows the deployment of virtual machines in the \ac{SEAM} with encrypted CPU state and memory, integrity protection, and remote attestation. \ac{TDX} aims to enforce hardware-assisted isolation for virtual machines and minimize the attack surface exposed to host platforms, which are considered to be untrustworthy or adversarial in the confidential computing's new threat model. \ac{TDX} can be leveraged by regulated industries or sensitive data holders to outsource their computations and data with end-to-end protection in public cloud infrastructure.

This paper aims to provide a comprehensive understanding of \ac{TDX} to potential adopters, domain experts, and security researchers looking to leverage the technology for their own purposes.
We adopt a top-down approach, starting with high-level security principles and moving to low-level technical details of \ac{TDX}. Our analysis is based on publicly available documentation and source code, offering insights from security researchers outside of Intel. 
\end{abstract}