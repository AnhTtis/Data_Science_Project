\section{Memory Protection}
\label{sec:memprotect}
A \ac{TD}'s memory is divided into \spec{private memory} and \spec{shared memory}. The private memory is only accessible by the \ac{TD} and the \ac{TDX} Module. The shared memory is also accessible by the hypervisor and is used for operations that require the cooperation from the hypervisor, such as networking, I/O, and \ac{DMA}. \ac{TDX} protects the confidentiality and integrity of a \ac{TD}'s private memory. 

\subsection{HKID Space Partitioning}
\label{sec:memprotect:MKTME}
\begin{figure}[ht!]
\centerline{\includegraphics[width=0.8\textwidth]{figures/keyid.pdf}}
\caption{HKID Layout in Physical Memory Address}
\label{fig:hkid}
\end{figure}

The \ac{HKID} space is partitioned once during the boot process into two ranges, \spec{private \acp{HKID}} and \spec{shared \acp{HKID}}. Only software in the \ac{SEAM} mode, namely the \ac{TDX} Module and \acp{TD}, can read and write memory whose contents are encrypted by keys associated with private \acp{HKID}. Keys associated with shared \acp{HKID} can be used to encrypt memory outside the SEAM mode, such as the memory of legacy VMs and the host kernel.

When the hypervisor requests the \ac{TDX} Module to establish a \ac{TD}, it allocates a private \ac{HKID} for the \ac{TD}. The \ac{TDX} Module, using the \code{PCONFIG} instruction, asks \ac{MKTME} to generate an unique random key for the \ac{HKID}. This key is called the \ac{TD}'s \spec{ephemeral private key}. It is used to encrypt all the private memory and metadata of the \ac{TD} and is never exposed outside \ac{MKTME}. This \code{$\langle\ac{HKID}, key\rangle$}  binding is valid for the lifetime of the \ac{TD}.

A physical memory page associated with a HKID stores the HKID in the upper bits of the page's physical address, as shown in \autoref{fig:hkid}. At boot time, the number of bits used for \acp{HKID} (\code{MK\_TME\_KEYID\_BITS}) and the number of bits used for private \acp{HKID} (\code{TDX\_RESERVED\_KEYID\_BITS}) are set in the \code{IA32\_TME\_ACTIVATE} \ac{MSR}. The \code{IA32\_MKTME\_KEYID\_PARTITIONING} \ac{MSR} can be used for reading the numbers of private and shared \acp{HKID}. Intel reserves a range of upper bits in the 64-bit physical address. The \ac{HKID} uses these reserved bits. The remaining bits following the \ac{HKID} correspond to the physical memory address. The upper bits of the \ac{HKID} field, the \code{TDX\_RESERVED\_KEYID\_BITS} are reserved for private \acp{HKID}. For example, if \code{MK\_TME\_KEYID\_BITS} is $6$ and \code{TDX\_RESERVED\_KEYID\_BITS} is $4$, then \acp{HKID} from $0$ to $3$ are \spec{shared}, and \acp{HKID} from $4$ to $63$ are \spec{private}.


The hypervisor and the \ac{TDX} Module configure the memory encryption by setting the \ac{HKID} in the upper bits of the physical address of a memory page. The hypervisor can only use shared \acp{HKID}, while the \ac{TDX} Module can use both shared and private \acp{HKID}. An exception will be raised if any software executing outside \ac{SEAM} mode tries to access memory through a physical address with a private \ac{HKID}.

\subsection{TD Memory Integrity Protection}
\label{sec:memprotect:TD_memory_integrity}
\ac{TDX} always protects the integrity of the \ac{TD}'s private memory content. 
This protection is required because an entity outside the \ac{SEAM} mode,\eg a malicious hypervisor or a \ac{DMA} device, can write to the \ac{TD}'s private memory. \ac{TDX} cannot prevent such modification, but it can detect and flag it. It prevents a \ac{TD} or the \ac{TDX} Module from reading or executing the modified content. To detect such modifications, \ac{TDX} supports two memory integrity modes that can be configured on a system:
\begin{enumerate}[noitemsep,topsep=0pt,partopsep=0pt]
    \item{\acf{Li}}: memory integrity is protected by a \spec{\ac{TD} Owner bit}.
    \item{\acf{Ci}}: memory integrity is protected by a \ac{MAC} and a \spec{\ac{TD} Owner bit}.
\end{enumerate}
Both \ac{Li} and \ac{Ci} apply to a physical memory segment with the size of a cache line and whose address is cache line aligned.
\ac{Ci} can detect modifications made by direct physical access to the memory or bit flips, such as the Rowhammer attack~\cite{rowhammer-ieee-2014}, which \ac{Li} cannot detect.

In addition to \ac{Li} and \ac{Ci}, if a program outside the \ac{SEAM} mode reads the private memory of a \ac{TD} or the \ac{TDX} Module, the read will always return zeros. This is to prevent ciphertext cryptanalysis and side channels in which a program outside the \ac{SEAM} mode could determine whether a program in the \ac{SEAM} mode change the memory content. 

If a \ac{TD} or the \ac{TDX} Module writes to a memory segment belonging to a \ac{TD}'s private memory, the corresponding \spec{\ac{TD} Owner bit} is set to $1$. Due to the way a \ac{TD}'s memory is set up, all \spec{\ac{TD} Owner bits} of a \ac{TD}'s private memory should be set to $1$. However, if an entity outside the \ac{SEAM} mode writes to a segment belonging to the private memory, the corresponding \spec{\ac{TD} Owner bit} is cleared to $0$. Later, when the \ac{TD} or the \ac{TDX} Module reads the segment, the segment is marked as \emph{poisoned}. If the reader is the \ac{TD}, this \spec{poisoned} marking causes a \ac{TD} exit for the \ac{TD}. The \ac{TDX} Module can capture this \ac{TD} exit and put the \ac{TD} into a \spec{fatal} state, which prevents any further entry into the \ac{TD} and leads to the tearing down of the \ac{TD}. If the \ac{TDX} Module reads the \spec{poisoned} content, the \ac{TDX} Module and the \ac{TDX}'s hardware extension in the processor are marked as \spec{disabled}. Any further \code{SEAMCALLs} leads to the \code{VMFailInvalid} error. 

If \ac{Ci} is enabled, the processor generates a $128$-bit MAC key during system initialization. On each write, \ac{TDX} uses this key to calculate and store a 28-bit \ac{MAC} in the \ac{ECC} memory corresponding to the cache line. On each read, the memory controller recalculates the \ac{MAC} and compares it with the value read from the \ac{ECC} memory. The mismatch indicates integrity or authenticity violation and results in the cache line being marked as \spec{poisoned}. The \ac{MAC} is calculated over:
(1) the ciphertext (encrypted content of the cache line), 
(2) the tweak values used for AES-XTS encryption, 
(3) the \spec{\ac{TD} Owner bit}, and 
(4) the 128-bit MAC key. 

