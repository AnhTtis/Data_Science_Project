\section{Introduction}
Deploying computations to cloud infrastructure can reduce costs, but regulated industries have concerns about moving sensitive data to third-party cloud service providers. Confidential computing aims to provide end-to-end protection for outsourced computations by minimizing the root of trust to the processors and their vendors. This means that all data must be protected throughout its life-cycle, from leaving its owners' premises to entering certified CPU packages in the cloud. Any adversaries, such as those intercepting on the network, disk storage, or main memory, should not be able to access the data in clear form. 

Cryptographic mechanisms, such as storage encryption and secure communication channels, protect the confidentiality, integrity, and authenticity of data both \emph{at rest} and \emph{in transit}. The emerging CPU-based \ac{TEE} techniques aim to provide protection for \emph{data in use}, \ie data loaded into main memory.

Intel \acf{TDX} is an architectural extension that provides \ac{TEE} capabilities in the 4th Generation Intel Xeon Scalable Processors. \ac{TDX} introduces the \ac{SEAM} to offer cryptographic isolation and protection for \acp{VM}, which are called \acp{TD} in the \ac{TDX} terminology. The threat model assumes that the privileged software, such as hypervisors or host operating systems, may be untrustworthy or adversarial. \ac{TDX} aims to protect the confidentiality and integrity of CPU state and memory for designated \acp{TD}, and also enables \ac{TD} owners to verify the authenticity of remote platforms. \ac{TDX} is built using a combination of techniques, including \ac{VT}~\cite{vtx}, \ac{MKTME}~\cite{mktmewhitepaper}, and the \ac{TDX} Module~\cite{tdxmodulespec}. \ac{TDX} also relies on \ac{SGX}~\cite{mckeen2013innovative} and \ac{DCAP}~\cite{dcapwhitepaper} for remote attestation. 

Throughout the paper, we aim to give an objective review of \ac{TDX}. Our goal is to provide a thorough understanding of \ac{TDX} to potential adopters, domain experts, and security researchers who want to leverage or investigate the technology for their own purposes. All the information is based on publicly available documentation~\cite{tdxwhitepaper,tdxmodulespec,tdxarchspec,tdxghci,tdxloaderspec} and source code~\cite{tdxmodulesrc,tdxloadersrc,tdxkernelsrc}. 

The following is a roadmap of this paper. We begin by outlining the security principles (\S\ref{sec:principle}) and the threat model (\S\ref{sec:model}) of \ac{TDX}. Next, we provide a comprehensive comparison of existing confidential computing technologies on the market (\S\ref{sec:related}) and examine the existing Intel technologies that serve as the building blocks for \ac{TDX} (\S\ref{sec:background}). Once the background knowledge is established, we offer a high-level overview of \ac{TDX} (\S\ref{sec:overview}) and then delve into the technical details of the \ac{TDX} Module (\S\ref{sec:tdxmodule}), memory protection mechanisms (\S\ref{sec:memprotect}), and remote attestation (\S\ref{sec:remote_attestation}). Finally, we conclude with a summary (\S\ref{sec:conclusion}). To assist readers in navigating the numerous terms and abbreviations used in this paper, a list of acronyms is also provided (\S\ref{sec:acronym}).