\batchmode
\makeatletter
\def\input@path{{/home/nicanor/Dropbox/Research/Paper_1_the_geometric_subgroup_membership_problem/}}
\makeatother
\documentclass[10pt,a4paper,oneside,english]{amsart}
\renewcommand{\familydefault}{\rmdefault}
\usepackage[utf8]{inputenc}
\setcounter{secnumdepth}{2}
\setcounter{tocdepth}{2}
\synctex=-1
\usepackage{babel}
\usepackage{verbatim}
\usepackage{pifont}
\usepackage{prettyref}
\usepackage{enumitem}
\usepackage{algorithm2e}
\usepackage{amstext}
\usepackage{amsthm}
\usepackage{amssymb}
\usepackage{graphicx}
\usepackage{microtype}
\usepackage[pdftex,
 bookmarks=true,bookmarksnumbered=true,bookmarksopen=true,bookmarksopenlevel=0,
 breaklinks=false,pdfborder={0 0 0},pdfborderstyle={},backref=false,colorlinks=false]
 {hyperref}
\hypersetup{pdftitle={The geometric subgroup membership problem},
 pdfauthor={Nicanor Carrasco-Vargas},
 urlcolor = MidnightBlue}

\makeatletter

%%%%%%%%%%%%%%%%%%%%%%%%%%%%%% LyX specific LaTeX commands.
\pdfpageheight\paperheight
\pdfpagewidth\paperwidth


%%%%%%%%%%%%%%%%%%%%%%%%%%%%%% Textclass specific LaTeX commands.
\newlength{\lyxlabelwidth}      % auxiliary length 
\theoremstyle{plain}
\newtheorem{thm}{\protect\theoremname}[section]
\theoremstyle{plain}
\newtheorem{cor}[thm]{\protect\corollaryname}
\theoremstyle{definition}
\newtheorem{defn}[thm]{\protect\definitionname}
\theoremstyle{plain}
\newtheorem{prop}[thm]{\protect\propositionname}
\theoremstyle{plain}
\newtheorem{lem}[thm]{\protect\lemmaname}
\theoremstyle{remark}
\newtheorem*{rem*}{\protect\remarkname}
\theoremstyle{definition}
\newtheorem{problem}[thm]{\protect\problemname}
\theoremstyle{definition}
\newtheorem{example}[thm]{\protect\examplename}

%%%%%%%%%%%%%%%%%%%%%%%%%%%%%% User specified LaTeX commands.
\newrefformat{prop}{Proposition \ref{#1}}
\newrefformat{cor}{Corollary \ref{#1}}
\newrefformat{def}{Definition \ref{#1}}

\AtBeginDocument{
  \def\labelitemi{\(\circ\)}
  \def\labelitemii{\ding{96}}
}

\makeatother

\providecommand{\corollaryname}{Corollary}
\providecommand{\definitionname}{Definition}
\providecommand{\examplename}{Example}
\providecommand{\lemmaname}{Lemma}
\providecommand{\problemname}{Problem}
\providecommand{\propositionname}{Proposition}
\providecommand{\remarkname}{Remark}
\providecommand{\theoremname}{Theorem}

\begin{document}
\global\long\def\G{\Gamma}%
\global\long\def\N{\mathbb{N}}%
\global\long\def\Z{\mathbb{Z}}%
\global\long\def\e{\mathbb{\varepsilon}}%

\global\long\def\ls{\mathbb{\leqslant}}%

\global\long\def\act{\mathbb{\curvearrowright}}%
\global\long\def\F{\mathcal{F}}%
\global\long\def\J{\mathcal{J}}%
\global\long\def\M{\mathfrak{M}}%
\global\long\def\emp{\emptyset}%

\global\long\def\cay{\text{Cay}}%
\global\long\def\WP{\text{WP}}%
\global\long\def\dom{\text{Dom}}%
\global\long\def\and{\text{ and }}%
\global\long\def\inf{\text{\ensuremath{\infty}}}%

\title{the geometric subgroup membership problem}
\subjclass[2020]{37B10, 05C63, 20F65, 03D99, 05C45}
\address{Pontificia Universidad Cat\'olica de Chile. }
\email{njcarrasco@mat.uc.cl}
\author{Nicanor Carrasco-Vargas}
\begin{abstract}
We show that every infinite graph which is locally finite and connected
admits a translation-like action by $\Z$ such that the distance between
a vertex $v$ and $v\ast1$ is uniformly bounded by 3. This action
can be taken to be transitive if and only if the graph has one or
two ends. This strengthens a theorem by Brandon Seward.

Our proof is constructive, and thus it can be made computable. More
precisely, we show that a finitely generated group with decidable
word problem admits a translation-like action by $\Z$ which is computable,
and satisfies an extra condition which we call decidable orbit membership
problem.

As an application we show that on any finitely generated infinite
group with decidable word problem, effective subshifts attain all
effectively closed Medvedev degrees. This extends a classification
proved by Joseph Miller for $\Z^{d},$ $d\geq1$.
\end{abstract}

\maketitle

\section{Introduction}

\subsection{Translation-like actions by $\protect\Z$ on locally finite graphs}

A right action $\ast$ of a group $H$ on a metric space $(X,d)$
is called a \textbf{translation-like} \textbf{action} if it is \textbf{free}\footnote{Which means that $x\ast h=x$ implies $h=1_{H}$, for $x\in X$, $h\in H$.},
and for each $h\in H$, the set $\{d(x,x\ast h)|\ x\in X\}$ is bounded.
If $G$ is a finitely generated group endowed with the word metric
associated to some finite set of generators, then the action of any
subgroup $H$ on $G$ by right translations $(g,h)\mapsto gh$ is
a translation-like action. Thus, translation-like actions can be regarded
a generalization of subgroup containment. 

Following this idea, Kevin Whyte proposed in \cite{whyte_amenability_1999}
to replace subgroups by translation-like actions in different questions
or conjectures about groups and subgroups, and called these geometric
reformulations. For example, the von Neumann Conjecture asserted that
a group is nonamenable if and only if it contains a nonabelian free
subgroup. Its geometric reformulation asserts then that a group is
nonamenable if and only if it admits a translation-like action by
a nonabelian free group. While the conjecture was proven to be false
\cite{olshanskij_question_1980}, Kevin Whyte proved that its geometric
reformulation holds. 

One problem left open in \cite{whyte_amenability_1999} was the geometric
reformulation of Burnside's problem. This problem asked if every finitely
generated infinite group contains $\Z$ as a subgroup, and was answered
negatively in \cite{golod_class_1964}. Brandon Seward proved that
the geometric reformulation of this problem also holds. 
\begin{thm}[Geometric Burnside's problem, \cite{seward_burnside_2014}]
\label{thm:seward} Every finitely generated infinite group admits
a translation-like action by $\Z$. 
\end{thm}

A finitely generated infinite group with two or more ends has a subgroup
isomorphic to $\Z$ by Stalling's structure theorem. Thus, it is the
one ended case that makes necessary the use of translation-like actions.
In order to prove \prettyref{thm:seward}, Brandon Seward proved a
more general graph theoretic result. 
\begin{thm}[{\cite[Theorem 1.6]{seward_burnside_2014}}]
\label{thm:sewardgrafos} Let $\G$ be an infinite graph whose vertices
have uniformly bounded degree. Then $\G$ admits a transitive translation-like
action by $\Z$ if and only if it is connected and has one or two
ends. 
\end{thm}

This result proves \prettyref{thm:seward} for one ended groups, and
indeed it says more, as the translation-like action obtained is transitive.
The proof of this result relies strongly on the hypothesis of having
uniformly bounded degree. Indeed, the uniform bound on $d_{\G}(v,v\ast1)$
depends linearly on a uniform bound for the degree of the vertices
of the graph.

We will strengthen this result by weakening the hypothesis to the
locally finite case, and improving the uniform bound on $d_{\G}(v,v\ast1)$
to 3. 

\begin{thm}
\label{thm:t2} Let $\G$ be an infinite graph whose vertices have
finite degree. Then $\G$ admits a transitive translation-like action
by $\Z$ if and only if it is connected and has one or two ends.

Moreover, the action can be taken such that the distance between a
vertex $v$ and $v\ast1$ is uniformly bounded by 3.
\end{thm}

A problem left in \cite[Problem 3.5]{seward_burnside_2014} was to
characterize which graphs admit a transitive translation-like action
by $\Z$. Thus we have solved the case of locally finite graphs, and
it only remains the case of graphs with vertices of infinite degree. 

We now mention an application of translation-like actions to Cayley
graphs and Hamiltonian paths. This is related to a special case of
Lovász conjecture which asserts the following. If $G$ is a finite
group, then for every set of generators the associated Cayley graph
admits a Hamiltonian path. Note that the existence of at least one
such generating set is obvious ($S=G$), and the difficulty of the
question, which is still open, is that it alludes every generating
set. Now assume that $G$ is an infinite group, $S$ is a finite set
of generators, and $\cay(G,S)$ admits a transitive translation-like
action by $\Z$. This action becomes a bi-infinite Hamiltonian path
after we enlarge the generating set, and thus it follows from Seward's
theorem that every finitely generated group with one or two ends admits
a generating set for which the associated Cayley graph admits a bi-infinite
Hamiltonian path \cite[Theorem 1.8]{seward_burnside_2014}. It is
not known whether this holds for every Cayley graph \cite[Problem 4.8]{seward_burnside_2014},
but our result yields an improvement in this direction. 

\begin{cor}
Let $G$ be a finitely generated group with one or two ends, and let
$S$ be a finite set of generators. Then the Cayley graph of $G$
with respect to the generating set $\{g\in G|\ d_{S}(g,1_{G})\leq3\}$
admits a bi-infinite Hamiltonian path. 
\end{cor}

This was known to hold for generating sets of the form $\{g\in G|\ d_{S}(g,1_{G})\leq J\}$,
where $S\subset G$ is a finite generating set for $G$ and $J$ depends
linearly on the vertex degree of $\cay(G,S)$. 

In the more general case where we impose no restrictions on ends,
we obtain the following result for non transitive translation-like
actions. This readiliy implies \prettyref{thm:seward}. 
\begin{thm}
\label{thm:t1}Every infinite graph which is locally finite and connected
admits a translation-like action by $\Z$ such that the distance between
a vertex $v$ and $v\ast1$ is uniformly bounded by 3.
\end{thm}

These statements about translation-like actions can also be stated
in terms of powers of graphs. Given a graph $\G$, its $n$-th power
$\G^{n}$ is defined as the graph with the same set of vertices, and
where two vertices $u,v$ are joined if their distance in $\G$ is
at most $n$. It was shown by Jerome Karaganis \cite{karaganis_cube_1968}
that the cube of a finite and connected graph is Hamiltonian. Our
\prettyref{thm:t2} generalizes this to infinite graphs, that is,
the cube of a locally finite connected graph with one or two ends
admits a bi-infinite Hamiltonian path. In our proofs we make use of
Karaganis result: we will define bi-infinite Hamiltonian paths locally. 

We mention that \prettyref{thm:t1} has been proved in \cite[Section 4]{cohen_strongly_2021},
using the same fact about cubes of finite graphs. 

\subsection{Computability of translation-like actions}

Let us now turn our attention to the problem of the computability
of translation-like actions on groups or graphs. In order to discuss
this precisely, we need the following notions. A graph $\G$ is computable
if there exists an algorithm which given two vertices $u$ and $v$,
determines whether they are neighbors or not. A stronger notion is
that of a highly computable graph, namely, a computable graph which
is locally finite and for which the function which maps a vertex to
its degree is computable. This extra condition is necessary to compute
the neighborhood of a vertex.

An important example comes from group theory: if $G$ is a finitely
generated group with decidable word problem and $S$ is a finite set
of generators, then its Cayley graph with respect to $S$ is highly
computable.

\begin{comment}
In order to express these notions formally, we will use natural numbers
to represent vertices, so the computations are performed on natural
numbers. In the case of finitely generated groups it is also natural
to represent group elements by words on some set of generators. This
two approaches are equivalent as long as the group has decidable word
problem; we refer the reader to the preliminaries.
\end{comment}

A classical example of a problem in graph theory which admits no computable
solution is that of finding infinite paths. Kőnig's infinity lemma
asserts that every infinite, connected, and locally finite graph admits
an infinite path. However, there are highly computable graphs which
admit paths, all of which are uncomputable. Another example comes
from Hall's matching theorem. There are highly computable graphs satisfying
the hypothesis in the theorem, but which admit no computable right
perfect matching \cite{manaster_effective_1972}. These two results
are used in the proof of Seward's theorem, so the translation-like
actions from this proof are not clearly computable. 

By a computable translation-like action on a computable graph $\G$,
we mean that the function $\ast:V(\G)\times\Z\to V(\G)$ is computable,
i.e. there exists an algorithm which given a vertex $v\in V(\G)$
and $n\in\Z$, computes the vertex $v\ast n$. 

In \prettyref{sec:Computable-translation-like-acti} we will provide
a computable version of \prettyref{thm:t2}, from which follows that
a highly computable graph admits a transitive translation-like action
by $\Z$ if and only if it admits a computable one. The restriction
on ends is essential to make the proof computable, and indeed we can
not show a computable version of \prettyref{thm:t1}. 

Our interest in the computability of translation-like actions comes
from symbolic dynamics, and the shift spaces associated to a group.
We will need a computable translation-like action such that it is
possible to distinguish in a computable manner between different orbits.
Let us introduce a general definition, though we will only treat the
case where the acting group is $\Z$. 
\begin{defn}
\label{def:orbit-membership-problem}Let $G$ be a group, and $S\subset G$
a finite set of generators. A group action of $H$ on $G$ is said
to have \textbf{decidable} \textbf{orbit membership problem} if there
exists an algorithm which given two words $u$ and $v$ in $(S\cup S^{-1})^{*}$,
decides whether the corresponding group elements $u_{G},v_{G}$ lie
in the same orbit under the action. 
\end{defn}

This property arises when one requires a standard construction in
the shift space $A^{G}$ to preserve a complexity measure called Medvedev
degree. This is the motivation behind the following result, whose
application is discussed below (\prettyref{thm:aplication}).

\begin{thm}
\label{thm:maintheorem}Let $G$ be a finitely generated infinite
group with decidable word problem. Then $G$ admits a computable translation-like
action by $\Z$ with decidable orbit membership problem.
\end{thm}

It is interesting to note that the introduced property corresponds
to the geometric reformulation of a subgroup property: the decidable
subgroup membership problem. If $G$ is a finitely generated group
with decidable word problem, then the action of a subgroup $H$ by
right translations has decidable orbit membership problem if and only
if $H$ has decidable subgroup membership problem (\prettyref{prop:obvio}). 

The proof of \prettyref{thm:maintheorem} proceeds as follows. For
groups with one or two ends, we will show the existence of a computable
and transitive translation-like action by $\Z$, i.e. a computable
version of \prettyref{thm:t2}. This has decidable orbit membership
problem for the trivial reason that has only one orbit. For groups
with two or more ends, we will obtain the desired translation-like
action from a subgroup isomorphic to $\Z$ with decidable membership
problem. The existence of this subgroup is a consequence of the computability
of the normal form from Stalling's structure theorem on groups with
decidable word problem (\prettyref{prop:stallings-descomposicion-calculable}). 


\subsection{Medvedev degrees of effective subshifts}

Let us now turn our attention to Medvedev degrees, a complexity measure
which is defined using computable functions. Precise definitions of
this and the following concepts are given in \prettyref{sec:medvedev}.
Intuitively, the Medvedev degree of a set $P\subset A^{\N}$ measures
how hard is to compute a point in $P$. For example, a set has zero
Medvedev degree if and only if it has a computable point. Note that
this becomes meaningful if we regard $P$ as the set of solutions
to some problem. 

This notion can be applied to a variety of objects, such as graph
colorings \cite{remmel_graph_1986}, paths on graphs, matchings from
Hall's matching theorem, and others \cite[Chapter 13]{ershov_handbook_1998}.
In this article we consider Medvedev degrees of subshifts. %
\begin{comment}
An example of the relation between algorithmic and dynamical properties
of two dimensional subshifts is the characterization of their entropies
as all of $\Pi_{1}^{0}$ nonnegative real numbers \cite{hochman_characterization_2010}.
\end{comment}
{} 

Let $G$ be a finitely generated group, and let $A$ be a finite alphabet.
A subshift is a subset of $A^{G}$ which is closed in the prodiscrete
topology, and is invariant under translations. Dynamical properties
of subshifts have been related to their computational properties in
different ways. A remarkable example of this is the characterization
of the entropies of two dimensional subshifts of finite type as the
class of nonnegative $\Pi_{1}^{0}$ real numbers \cite{hochman_characterization_2010}.

In the same manner that it is done with the entropy, one may ask which
is the class of Medvedev degrees that a certain class of subshifts
can attain. The classification is known for subshifts of finite type
in $\Z^{d},d\geq1$. In the case $d=1$, all subshifts of finite type
have Medvedev degree zero, because all of them contain a periodic
point, and then a computable point. In the case $d\geq2$, subshifts
of finite type can attain the class of $\Pi_{1}^{0}$ Medvedev degrees
\cite{simpson_medvedev_2014}. 

A larger class of subshifts is that of effective subshifts. A subshift
$X\subset A^{\Z}$ is effective if the set of words which do not appear
in its configurations is computably enumerable. This notion can be
extended to a finitely generated group, despite some intricacies that
arise in relation to the word problem of the group. In this article
we deal with groups with decidable word problem, and the notion of
effective subshift is a straightforward generalization. 

Answering a question left open in \cite{simpson_medvedev_2014}, Joseph
Miller proved that effective subshifts over $\Z$ can attain all $\Pi_{1}^{0}$
Medvedev degrees \cite{miller_two_2012}. We generalize this result
to the class of infinite, finitely generated groups with decidable
word problem. 

\begin{thm}
\label{thm:aplication} Let $G$ be a finitely generated and infinite
group with decidable word problem. The class of Medvedev degrees of
effective subshifts on $G$ is the class of all $\Pi_{1}^{0}$ Medvedev
degrees.
\end{thm}

The idea for the proof is the following. Given any subshift $Y\subset A^{\Z}$,
we can produce a new subshift $X\subset B^{G}$ that simultaneously
describes translation-like actions by $\Z$, and elements in $Y$.
Then \prettyref{thm:maintheorem} ensures that this construction preserves
the Medvedev degree of $Y$, and the result follows from the known
classification for $\Z$. 

Despite the simplicity of the proof we need to translate some computability
notions from $A^{\N}$ to $A^{G}$, this is done by taking a computable
numbering of $G$. The notions obtained do not depend of the numbering,
and are consistent with notions present in the literature defined
by other means. 

The mentioned construction using translation-like actions has been
used in different results in the context of symbolic dynamics. For
example, to transfer results on the emptiness problem for subshifts
of finite type from one group to another \cite{jeandel_translationlike_2015},
to produce aperiodic subshifts of finite type on new groups \cite{cohen_strongly_2021,jeandel_translationlike_2015},
and to study the entropy of subshifts of finite type on some amenable
groups \cite{barbieri_entropies_2021}.

\subsection*{Paper structure}

In \prettyref{sec:Preliminaries} we fix some notation, and recall
some basic facts on graph and group theory. We also fix our computability
setting for countable sets. In \prettyref{sec:Translation-like-actions-by}
we show \prettyref{thm:t2} and \prettyref{thm:t1}. In \prettyref{sec:Computable-translation-like-acti}
we show \prettyref{thm:maintheorem}, and apply this in \prettyref{sec:medvedev}
to prove \prettyref{thm:aplication}.

\subsection*{Acknowledgements}

This paper would not have been possible without the help, guidance,
and careful reading of my two advisors. I am grateful to Sebastián
Barbieri for suggesting that translation-like actions would work to
prove \prettyref{thm:aplication}, and for helping me with many details.
I am also grateful to Cristóbal Rojas for providing the computability
background which took me to Medvedev degrees of complexity, and finally
to ask \prettyref{thm:aplication}.

This research was partially supported by ANID 21201185 doctorado nacional,
ANID/Basal National Center for Artificial Intelligence CENIA FB210017,
and the European Union's Horizon 2020 research and innovation program
under the Marie Sklodowska-Curie grant agreement No 731143.

\section{Preliminaries}\label{sec:Preliminaries}

\subsection{Graph theory}

In this article all graphs are assumed to be undirected, without self
loops, and simple (not multigraphs). We now fix some terminology and
recall some basic facts. 

Given a graph $\G$ we denote its vertex set by $V(\G)$, and its
edge set by $E(\G)$. Each edge $e$ is an unordered pair of vertices
$\{x,y\}$, we say that $e$ \textbf{joins }$x$ and $y$, its \textbf{endpoints}.
We also say that $x,y$ are \textbf{adjacent }or \textbf{neighbors}.
The \textbf{degree} of a vertex $x$ is the number of its neighbors,
and is denoted $\deg_{\G}(x)$. 

A \textbf{subgraph }of $\G$ is a graph whose edge and vertex set
are contained in the edge and vertex set of $\G$. A set of vertices
$V\subset V(\G)$ determines\textbf{ }a subgraph of $\G$, denoted
$\G[V]$, as follows. The vertex set of $\G[V]$ is $V$, and its
edge set is the set of all edges in $E(\G)$ whose endpoints are both
in $V$. We say that $\G[V]$ is the subgraph of $\G$ \textbf{induced}
by $V$. 

Another way to obtain a subgraph of $\G$ is by deleting a set of
vertices. For $V\subset V(\G)$, $\G-V$ denotes the graph $\G[V(\G)-V]$,
that is, we erase from $\G$ all vertices in $V$, and all edges incident
to them. If $\G'$ is a subgraph of $\G$, then $\G-\G'$ stands for
$\G-V(\G')$. 

A \textbf{path }on $\G$\textbf{ }is an injective function $f:[n,m]\subset\Z\to V(\G)$
which sends consecutive integers to adjacent vertices. We say that
$f$ \textbf{joins $f(n)$ }to $f(m)$, and define its \textbf{length
}to be $n-m$. 

We say that $\G$ is \textbf{connected} when any pair of vertices
are joined by a path. A \textbf{connected component }of $\G$ is a
connected subgraph of $\G$ which is maximal for the subgraph relation. 

If $\G$ is connected, we define the path length metric $d_{\G}$
on $V(\G)$ as follows. Given two vertices $u$ and $v$, $d_{\G}(u,v)$
is the length of the shortest path which joins $u$ to $v$. 

The \textbf{number of ends }of $\G$ is the supremum of the number
of infinite connected components of $\G-V$, where $V$ ranges over
all finite sets of vertices in $V(\G)$. Thus the number of ends of
$\G$ is an element of $\N\cup\{\infty\}$.

\subsection{Elementary computability notions}

The following definitions of computable functions and sets are fairly
standard, and can be found, for example, in \cite{chiswell_course_2009}. 

Given any set $A$, we denote by $A^{*}$ the set of all finite words
of elements of $A$. In this context we call $A$ an \textbf{alphabet}.
The empty word is denoted by $\e$. A word $u$ of length $n$ is
a \textbf{prefix }of $v$ if they coincide in the first $n$ symbols. 

Now let $A,B$ be finite alphabets. Recall that a \textbf{partial
function} from $A^{\ast}$ to $B^{\ast}$ is a function whose domain
is a subset of $A^{\ast}$, and that it is called \textbf{total} if
its domain is all of $A^{*}$.

A partial function $f\colon\dom(f)\subset A^{*}\to B^{*}$ is \textbf{computable
}when there exists an algorithm which on input a word $w\in A^{*}$
in the domain of $f$ halts and outputs $f(w)$, and which does not
halt when given as input a word outside the domain of $f$. 

A subset $X\subset A^{*}$ is \textbf{computable} or \textbf{decidable
}if there is an algorithm that decides which elements belong to $X$
and which elements do not, that is, its characteristic function is
a total computable function.

A subset $X\subset A^{*}$ is \textbf{computably enumerable} or \textbf{semi
decidable }if it is the domain of a partial computable function. Equivalently,
there is an algorithm that halts with input $x$ if and only if $x\in X$.
Equivalently, $X$ is the range of a partial computable function.

We say that a function or a set is \textbf{uncomputable }when it is
not computable. It is easily seen that a set $X$ is computable if
and only if $X$ and its complement are computably enumerable.

We can obtain computability notions on $\N^{p},p\geq1$ by identifying
$\N^{p}$ with a computable subset of $A^{*}$, for some finite alphabet
$A$. Indeed, we commonly write elements of $\N^{p}$ as words on
the finite alphabet of digits from 0 to 9, the parenthesis symbols,
and the comma symbol.

\subsection{Words and finitely generated groups }\label{subsec:preliminaries-Finitely-generated-groups}

It will be convenient for us to keep the distinction between words
and group elements. Let $G$ be a group, and $S$ a subset of $G$.
We denote by $S^{-1}$ the set $\{s^{-1}|\ s\in S\}$. Now given a
word $w\in(S\cup S^{-1})^{*}$ which is the concatenation of $s_{1},\dots,s_{n}\in S$,
we denote by $w^{-1}$ the word obtained by concatenating $s_{n}^{-1}\dots s_{1}^{-1}$.

Given a word $w\in(S\cup S^{-1})^{*}$, we denote by $w_{G}$ the
group element that it represents, that is, the group element obtained
replacing concatenation by the group operation. We also write $u=_{G}v$
if the words $u,v$ correspond to the same group element. 

A set $S\subset G$ is said to generate $G$ if every group element
can be written as a word in $(S\cup S^{-1})^{*}$, and $G$ is \textbf{finitely
generated} if it admits a finite generating set. %
\begin{comment}
We will assume that the reader is familiar with the concept of \textbf{group
presentation}, which we write as $\langle S|R\rangle$.
\end{comment}

Let us assume now that $S$ is a finite set of generators for $G$.
The (undirected, and right)\textbf{ Cayley graph} of $G$ relative
to $S$ is denoted by $\cay(G,S)$. Its vertex set is $G$, and two
vertices $g,h$ are joined by an edge when $gs=h$ for some $s\in(S\cup S^{-1})$.
The path length metric that this graph produces on $G$ is the same
as the \textbf{word metric }associated to\textbf{ $S$, }denoted $d_{S}$.
Given a pair of elements $g,h\in G$, their distance $d_{S}(g,h)$
is the length of the shortest word $\ensuremath{w\in(S\cup S^{-1})^{*}}$
such that $gw_{G}=h$. We denote by $B(g,r)$ the ball $\{h\in G\mid d_{S}(g,h)\leq r\}$. 

The number of ends of a finitely generated group is defined as the
number of ends of its Cayley graph, for some generating set. This
definition does not depend on the chosen generating set, and can only
be among the numbers $\{0,1,2,\infty\}$, as proved in \cite{freudenthal_ueber_1945,hopf_enden_1944}. 

Now let $H$ be a subgroup of $G$. We say that $H$ has \textbf{decidable
subgroup membership problem }if the set $\{w\in(S\cup S^{-1})^{*}\mid w_{G}\in H\}$
is decidable. This property depends only on $G$ and $H$, and not
on $S$. In other words, this set is decidable for some fixed and
finite generating set $S$ if and only if this holds for every generating
set. 

In the particular case where $H=\{1_{G}\}$, the set defined above
is called the \textbf{word problem} of $G$, and denoted by $\WP(G,S)$.
This notion is related to that of computable group, which we discuss
in the following subsection. 

\subsection{Computability on countable sets via numberings}\label{subsec:computability-on-countable}

\begin{comment}
In order to justify our approach, we consider the example of directed
graphs, where the following definitions are found.
\begin{enumerate}
\item A directed graph $(V,E)$ is computable if $V$ is a decidable subset
of $\N$, and $E$ is a decidable subset of $\N^{2}$.
\item A directed graph $(V,E)$ is computable if there exists a bijection
$\nu:\N\to V$ such that $(\nu\times\nu)^{-1}(E)\subset\N^{2}$ is
a computable subset.
\end{enumerate}
The relation between this two definitions is simple: if a graph $(V,E)$
satisfies the first definition, then it satisfies the second. On the
other hand, if it satisfies the second, then $\nu$ produces a copy
(an isomorphic graph whose vertex set is a subset of $\N$) satisfying
the first definition.

If we instead have an undirected multigraph, then the first definition
does not work inmediatly, as we need to define what is a computable
subset of $\N^{(2)}$, the set of unordered pairs $\{\{x,y\}:x,y\in\N\}$.
To simplest way to define a computable subset of $\N^{(2)}$ is with
a numbering, so we instead use numberings for everything.
\end{comment}

In this article we need to discuss the computability of graphs, group
actions, and infinite paths. For a group $G$, this will be related
with computability notions on the symbolic space $A^{G}$ in \prettyref{sec:medvedev}.
We consider a unified approach using numberings, a generalization
of Gödel numberings introduced by Yuri Ershov. A survey on the subject
can be found in \cite[Chapter 14]{griffor_handbook_1999}. 
\begin{defn}
A (bijective)\textbf{ numbering }of a set $X$ is a bijective map
$\nu:N\to X$, where $N$ is a decidable subset of $\N$. We call
$(X,\nu)$ a \textbf{numbered set}. When $\nu(n)=x$, we say that
$n$ is a \textbf{name} for $x$, or that $n$ represents $x$. 

As long as the set $X$ is infinite, we can always assume that the
set $N$ equals $\N$, because a decidable and infinite subset of
$\N$ admits a computable bijection onto $\N$. 

Computability notions are transferred from $\N$ to numbered sets
as follows. A subset $Y$ of $(X,\nu)$ is decidable or computable
when $\nu^{-1}(Y)\subset\N$ is a decidable set. Given two numbered
sets $\nu:N\to X$, $\nu:N'\to X'$, we can endow their product $X\times X'$
with a numbering as follows. The product of $\nu$ and $\nu'$ is
a bijection $N\times N'\subset\N^{2}\to X\times X'$, where $N\times N'$
is a decidable subset of $\N^{2}$. It remains to compose with a computable
bijection between $N\times N'$ and a decidable subset $N''\subset\N$,
and we obtain a numbering of $X\times X'$. 

This gives a definition of a decidable or computable relation $R\subset X\times X'$
between numbered sets $(X,\nu)$ and $(X',\nu')$. In the particular
case where the relation is a function $f:X\to X'$, this is equivalent
to ask $\nu'{}^{-1}\circ f\circ\nu:\N\to\N$ to be computable. %
\begin{comment}
a function $f:X^{p}\to X^{q}$ is computable if its graph $R\subset X^{n+m}$
is a computable relation, this is equivalent to ask the corresponding
function $\hat{f}$ that sends $\boldsymbol{n}\in\N^{p}$ to $\boldsymbol{\nu}^{-1}(f(\boldsymbol{\nu n}))\in\N^{q}$
to be computable;
\end{comment}
\end{defn}

Different numberings on a set $X$ may produce different computability
notions\footnote{Indeed, non equivalent numberings do admit a semilattice structure.
See \cite{badaev_theory_2000} and references therein.}. However after we impose some structure on the set -such as functions
or relations- and require the numbering to make them computable, the
possibilities can be dramatically reduced. A notable example is that
of finitely generated groups, as we will see later. The following
two definitions will be relevant to us.
\begin{defn}
A computable group is a group $(G,\star)$ and a numbering $\nu$
of $G$ such that the function $\star:G^{2}\to G$ is computable.
A numbering of a group is \textbf{computable }if it satisfies the
previous condition. 
\end{defn}

For example, $\Z$ admits a computable numbering, and thus is a computable
group.

\begin{defn}
A \textbf{computable graph }is an undirected graph $\G$ and a numbering
of $V(\G)$ such that the adjacency relation is decidable, i.e. $\{(x,y)\mid\{x,y\}\in$$E(\G)\}$
is a decidable subset of $V(\G)^{2}$. A computable graph is \textbf{highly
computable} if it is locally finite and the function $V\to\N$, $v\mapsto\deg(v)$
is computable.
\end{defn}

Now a computable graph or group defines a class of computable objects.
For example, a bi-infinite path $f:\Z\to V(\G)$ is computable if
the function $f$ is computable between numbered sets $\Z$ and $V(\G)$,
and so on. 

Let us now discuss some properties of computable groups and their
numberings. The following proposition will be used to define numberings
easily. 
\begin{prop}
\label{prop:enumerados-sobre}Let $R\subset\N^{2}$ be a decidable
equivalence relation. Then the set of equivalence classes $\N^{2}/R$
admits a unique numbering, up to equivalence, such that the relation
$\in$ is decidable. 
\end{prop}

To prove the existence we can just represent each equivalence class
by the least natural number in that equivalence class, the full proof
is given at the end of this subsection. %
\begin{comment}
We will now review some properties of numberings that will be used.
In particular, that for finitely generated groups, the existence of
a computable numbering is equivalent to the decidability of the word
problem, and that if that is the case, all computable numberings are
equivalent. This means that there is a uniquely defined computability
notion to be applied to paths and other objects. Moreover, a computable
numbering of a finitely generated group will automatically become
a computable numbering of its Cayley graph. Let us start with the
following proposition.
\end{comment}
{} 

Thus to define a numbering on any set $X$, it is enough to take a
surjection $s:\N\to X$ such that we can decide if two numbers $n,m$
name the same element (so the relation $s(n)=s(m)$ is decidable),
and this numbering enjoys a uniqueness property. This can be applied
in a variety of situations. 

Now let $G$ be a group with decidable word problem, $S\subset G$
a finite set of group generators, and $\pi:(S\cup S^{-1})^{\ast}\to G$
the map $w\mapsto w_{G}$. The equivalence relation $\pi(u)=\pi(v)$
is decidable, as two words $u$ and $v$ have the same image under
$\pi$ if and only if $uv^{-1}$ lie in the decidable set $\WP(G,S)$.
We obtain a numbering of $G$ by \prettyref{prop:enumerados-sobre}.
Indeed, this is a computable numbering, and it is the only one up
to equivalence:

\begin{comment}
\begin{example}
We already defined computable objects on $A^{*}$. If we take the
bijection $\nu:\N\to A^{*}$ given by expansion in base $|A|$, then
the objects that are computable on $(A^{*},\nu)$ are the same ones
that are computable on $A^{*}$. Indeed, it is easy to show that any
numbering of $A^{*}$ that makes the concatenation operation $A^{*}\times A^{*}\to A^{*}$
computable, is equivalent to the just mentioned.
\end{example}

\end{comment}

\begin{prop}[\cite{rabin_computable_1960}]
\label{prop:numberings-equivalent}A finitely generated group admits
a computable numbering if and only if it has decidable word problem.
If this is the case, then all its computable numberings are equivalent.
\end{prop}

Moreover, computability notions are preserved by group isomorphisms. 
\begin{prop}
\label{prop:isom-calculable} Let $G$ and $G'$ be finitely generated
groups with decidable word problem, and $f:G\to G'$ a group isomorphism.
Then $f$ is computable (for any computable numbering of $G$ and
$G'$).
\end{prop}

The proofs are also given below. Let us now review some properties
of highly computable graphs, let $\G$ be a highly computable graph.
The crucial consequence of this hypothesis is that given some vertex
$v$, and $r\in\N$, we can compute (uniformly on $v$ and $r$) all
vertices in the ball $\{v\in V(\G)\mid d_{\G}(v,v_{0})\leq r\}$.
This is true for $r=1$: as the adjacency relation is decidable, we
can computably enumerate the set $\{v\in V(\G)\mid d_{\G}(v,v_{0})\leq1\}$,
and $\deg_{\G}(v_{0})$ tells us at which point we have found them
all. For $r=2$, we just have to iterate the previous step over each
vertex $v_{1}\in\{v\in V(\G)\mid d_{\G}(v,v_{0})\leq1\}$, and so
on. 

This also shows that on a highly computable graph, the path-length
metric $d_{\G}:V(\G)^{2}\to\N$ is a computable function. 

Finally, let us observe that a computable numbering of a computable
group is automatically a computable numbering of $\cay(G,S)$: two
vertices $g,h\in G$ are joined if and only if $hg^{-1}\in S\cup S^{-1}$,
and this is a decidable relation (for example, because the numbering
makes the group operation computable and $S\cup S^{-1}$ is a finite
set). %
\begin{comment}
%
Let us mention that we can define a numbering of $\N^{(2)}=\{\{x,y\}:x,y\in\N\}$
by be the map $\N^{2}\to\N^{(2)},$ $(x,y)\mapsto\{x,y\}$. As the
relation $s(n)=s(m)$ is decidable, \prettyref{prop:enumerados-sobre}
can be applied. Thus a numbering of a set $X$ induces a unique numbering
of $X^{(2)}$ satisfying the mentioned property. Now it is clear that
a $\nu$ is a computable numbering of the graph $\G$ if and only
if its edge set is a decidable subset of $V(\G)^{(2)}$.
\end{comment}
{} 

\begin{proof}[Proof of \prettyref{prop:enumerados-sobre}]
We first prove the existence. Taking the usual order on $\N$, we
define the decidable set $N\subset\N$ as follows. A natural number
$n$ lies in $N$ if it is the minimal element of its equivalence
class. Then $\nu:N\to\N/R$ is a numbering. The $\in$ relation in
$\N\times\N^{2}/R$ is decidable. Given $n,m\in\N$, where $m$ is
a name for the equivalence class $\nu(m)$, we can decide if $n$
lies in the equivalence class $\nu(m)$ by checking if $(n,m)\in R$.

We now prove the uniqueness, let $\nu$ and $\mu$ be numberings of
$\N^{2}/R$ as in the statement. We see that the identity function
from $(\N^{2}/R,\nu)$ to $(\N^{2}/R,\mu)$ is computable, the remaining
direction holds for symmetry. Let $n$ be a name for the equivalence
class $\nu(n)$, we need to compute $m$ such that $\mu(m)=\nu(n)$.
As $\in$ is a decidable relation for $\nu$, we can compute $r\in\N$
so that $r\in\nu(n)$. As $\in$ is also decidable for $\mu,$ we
can compute the $\mu$ name $m$ of the equivalence class where $r$
lies. Thus $\mu(m)=\nu(n)$. 
\end{proof}
%
\begin{proof}[Proof of \prettyref{prop:numberings-equivalent}]
Let $(G,\star)$ be a finitely generated group. If $G$ is finite,
then the statement holds. Indeed, it admits a computable numbering
because a finite set $N\subset\N$ is always decidable and a function
$f:N^{2}\to N$ is always computable. Any two numberings are equivalent
because any bijection between finite subsets of $\N$ is computable.

Let us now assume that $G$ is infinite, that it has decidable word
problem, and that $S$ is a finite set of group generators. We already
defined a numbering $\nu:\N\to G$ by means of the function $\pi:(S\cup S^{-1})^{\ast}\to G$,
$w\mapsto w_{G}$. We now show that this is a computable numbering
of $G$.

Indeed, let $n,m$ be names for $g,h\in G$. We need to compute a
name for $gh$. As the relation $\in$ is decidable for this numbering,
we can compute two words $u,v\in(S\cup S^{-1})^{*}$ such that $u_{G}=g$
and $v_{G}=h$. Again, as $\in$ is decidable, we can compute a $\nu$
name for the equivalence class of $(uv)_{G}$. Thus $\nu$ is a computable
numbering of $G$.

Now keep the numbering $\nu$ just defined, and assume that $G$ has
another computable numbering $\mu$. Let us denote by $\bullet$ and
$\diamond$ the computable group operations on $\N$ defined by $(\nu\times\nu)^{-1}(\star)$
and $(\mu\times\mu)^{-1}(\star)$, respectively. Let us write $S\cup S^{-1}=\{s_{1},\dots,s_{k}\}\subset G$,
and define $\{r_{1},\dots,r_{k}\}\subset\N$ and $\{t_{1},\dots,t_{k}\}\subset\N$
by $\mu(t_{i})=\nu(r_{i})=s_{i},$where $i=1,\dots,k$. 

We show that the identity function is computable from $(G,\nu)$ to
$(G,\nu')$. On input $m\in\N$ (a $\nu$ name for a group element),
compute natural numbers $i_{1}\dots i_{n}$ such that $m=r_{i_{1}}\bullet\dots\bullet r_{i_{n}}$.
This can be done by searching exhaustively, and using that $\bullet$
is computable. Then output $t_{i_{1}}\diamond t_{i_{2}}\diamond\dots\diamond t_{i_{n}}'$.
Thus is a $\mu$ name for the same group element. 
\end{proof}
%
The proof of \prettyref{prop:isom-calculable} is left to the reader,
as it is identical to the previous one. 

\section{Translation-like actions by $\protect\Z$ on locally finite graphs}\label{sec:Translation-like-actions-by}

In this section we prove \prettyref{thm:t2} and \prettyref{thm:t1}.
For this we will work with finite and bi-infinite 3-paths. 
\begin{defn}
A \textbf{3-path} (resp. \textbf{bi-infinite 3-path}) on a graph $\G$
is an injective function $f:[a,b]\to V(\G)$, $[a,b]\subset\Z$, (resp.
\textbf{$f:\Z\to V(\G)$}) such that consecutive integers in the domain
are mapped to vertices whose distance $d_{\G}$ is at most 3. It is
called \textbf{Hamiltonian }if it visits all vertices.
\end{defn}

It is known that a finite and connected graph admits a Hamiltonian
3-path, and we can pick the first and last of its vertices \cite{karaganis_cube_1968}.
We start by showing that we can impose additional conditions on this
3-path.
\begin{lem}
\label{lem:lema-tecnico}Let $\G$ be a connected finite graph, and
$u\ne v$ two of its vertices. Then $\G$ admits a Hamiltonian $3$-path
$f:[a,b]\to V(\G)$ which starts at $u$, ends at $v$, and moreover
satisfies the following two conditions.
\begin{enumerate}
\item If $b-a\geq2$, then $d_{\G}(f(a),f(a+1))\leq2$ and $d_{\G}(f(b-1),f(b))\leq2$
(the first and last ``jump'' have length at most 2).
\item There is no $c\in[a,b]$ with $c+1,c-1\in[a,b]$ and such that $d_{\G}(f(c-1),f(c))=3$
and $d_{\G}(f(c),f(c+1))=3$ (there are no consecutive length 3 ``jumps''). 
\end{enumerate}
\end{lem}

Using this we will construct bi-infinite 3-paths in an appropriate
manner, and this will provide a proof of \prettyref{thm:t1} and \prettyref{thm:t2}.
Before proving the lemma, let us introduce some useful terminology. 

Let $\G$ be a finite graph, and let $f:[a,b]\to V(\G)$ be a 3-path
on $\G$. We say that $f$ \textbf{starts }at $f(a)$ and \textbf{ends
}at $f(b)$, that it goes from $f(a)$ to $f(b)$, and that $f(a),f(b)$
are the \textbf{endpoints }of $f$. We say that vertices in $f([a,b])$
are \textbf{visited }by $f$, and abreviate this set by $V(f)$. 

Now let $f:[a,b]\to V(\G)$ and $g:[c,d]\to V(\G)$ be two 3-paths.
We say that $g$ \textbf{extends} $f$ if it does it as a function,
that is, its restriction to the domain of $f$ is equal to $f$. We
define the \textbf{concatenation} of $f$ and $g$ as the function
$h$ which visits the vertices visited by $f$ and then the vertices
visited by $h$, in the same order. More precisely, $h$ has domain
$[a,b+1+d-c]$, and is defined by 
\[
h(x)=\begin{cases}
f(x) & x\in[a,b]\\
g(x-b-1+c) & x\in[b+1,b+1+d-c].
\end{cases}
\]

If $V(f)\cap V(g)=0$ and $d_{\G}(f(b),g(c))\leq3$, then $h$ is
a 3-path and extends $f$. Finally, the \textbf{inverse }of a 3-path
$f:[a,b]\to V(\G)$, denoted $-f$, is the 3-path with domain $[-b,-a]$
and which sends $x$ to $f(-x)$. 

\begin{comment}
Let $\G$ be a connected finite graph, and $u\ne v$ two of its vertices.
Then $\G$ admits a Hamiltonian $3$-path which starts at $u$ and
ends at $v$.
\begin{proof}
Observe that the claim holds for graphs with at most 3 vertices. We
now proceed by induction, assume that $\G$ is a connected finite
graph with at least three vertices, and let $u,v$ be different vertices
in $V(\G)$.

Let $\G_{1},\dots,\G_{n}$, $n\geq1$ be the connected components
of $\G-v$, and assume that $u$ lies in $\G_{1}$. Let $u_{1}=u$,
and for $i\in[2,n]$ let $u_{i}$ be any vertex in $\G_{i}$ which
is adjacent to $v$. Now for $i\in[1,n]$ define $v_{i}$ as follows.
If $|V(\G_{i})|\geq2$, let $v_{i}$ be any vertex in $\G_{i}$ which
is adjacent to $u_{i}$, and if $|V(\G_{i})|=1$ let $v_{i}=u_{i}$.

By our inductive hypothesis each $\G_{i}$ admits a Hamiltonian 3-path
$f_{i}$ from $u_{i}$ to $v_{i}$, $i\in[1,n]$. We define $f_{n+1}=v$.
Observe that last vertex of $f_{i}$ is at distance at most 3 from
the first vertex of $f_{i+1}$, for $i\in[1,n]$. Now by construction,
the concatenation of $f_{1}$, $f_{2}$, $\dots f_{n}$ is a Hamiltonian
3-path on $\G$, it starts at $u$ and ends in $v$.
\end{proof}
\end{comment}
\begin{comment}
\begin{rem}
The 3 is optimal, a counterexample being the graph $a-b-c-d$. There
is no hamiltonian 3-path starting at $b$, and ending at $c$.
\end{rem}

\end{comment}

\begin{proof}[Proof of \prettyref{lem:lema-tecnico}]
The proof is by induction on $|V(\G)|$. The claim holds for $|V(\G)|\leq3$,
as in that case any pair of vertices are at distance at most two. 

Now let $\G$ be a connected finite graph with $|V(\G)|\geq3$, let
$u\ne v\in V(\G)$, and assume that the result holds for graphs with
a strictly lower amount of vertices. We will show that there is a
Hamiltonian 3-path on $\G$ from $u$ to $v$ as in the statement. 

Let $\G_{1},\dots,\G_{n}$, $n\geq1$ be the connected components
of $\G-v$, and assume that $u$ lies in $\G_{1}$. Moreover, let
$\G_{n+1}$ be the graph whose only vertex is $v$. 

For each $i\in[1,n+1]$ we pick a pair of vertices $u_{i}$ and $v_{i}$
in $\G_{i}$ as follows. For $i=1$ we pick $u_{1}=u$. If $\G_{1}$
has only one vertex then $v_{1}$ is also equal to $u$, otherwise
$v_{1}$ is a vertex in $\G_{1}$ adjacent to $v$. For $i\in[2,n]$
we pick $u_{i}$ as a vertex in $\G_{i}$ adjacent to $v$. If $\G_{i}$
has only one vertex then we pick $v_{i}$ equal to $u_{i}$, otherwise
we take $v_{i}$ as a vertex in $\G_{i}$ adjacent to $u_{i}$. Finally,
we let $u_{n+1}$ and $v_{n+1}$ be equal to $v$. Thus for $i\in[1,n]$,
$v_{i}$ is at distance at most 3 from $u_{i+1}$.

We now invoke the inductive hypothesis on each $\G_{i}$, $i\in[1,n+1]$
(this is also correct when $|V(\G_{i})|=1$), and obtain a 3-path
$f_{i}$ from $u_{i}$ to $v_{i}$ as in the statement. We define
$f$ as the concatenation of $f_{1},\dots,f_{n+1}$, and claim that
$f$ satisfies the required conditions. 

First observe that $f$ is a 3-path on $\G$, as $V(f_{i})$ and $V(f_{j})$
are disjoint for $i\ne j\in[1,n+1]$, and the last vertex visited
by $f_{i}$ is at distance at most 3 from the first vertex visited
by $f_{i+1}$. It is clear that it visits all vertices exactly once,
so it is Hamiltonian.

We now verify that $f$ satisfies the first numbered condition in
the statement. We need the following observation, which follows from
the fact that $u_{i+1}$ is at distance at most one from $v$.
\begin{rem*}
If $\G_{i}$ is a singleton, $i\in[1,n]$, then $d(v_{i},u_{i+1})\leq2$.
\end{rem*}
We claim that after $u=u_{1}$, $f$ visits a vertex at distance at
most 2 from $u$. Indeed, if $|V(\G_{1})|\geq2$, then the claim holds
by inductive hypothesis on $f_{1}$. Otherwise it follows from the
fact that $u_{1}=v_{1}$ and previous remark. Moreover, the last vertex
visited by $f_{n}$, which is $v_{n}$, is at distance at most 2 from
$v$ ($n\geq1$). 

We now verify that $f$ satisfies the second numbered condition in
the statement. Take $c\in[a,b]$ such that $c+1$ and $c-1$ are both
in $[a,b]$, we claim that some of $f(c-1)$, $f(c+1)$ is at distance
at most 2 from $f(c)$. To prove this let $\G_{i}$ be the component
containing $f(c)$, and note that $i\in[1,n]$. If $\G_{i}$ has one
vertex, then $d_{\G}(f(c),f(c+1))\leq2$ by the remark above. Now
assume that $\G_{i}$ has more than one vertex, and recall that $f_{i}$
is a Hamiltonian 3-path on $\G_{i}$ from $u_{i}$ to $v_{i}$ as
in the statement. If $f(c)$ is some of $u_{i}$ or $v_{i}$, then
the fact that $f_{i}$ satisfies the first item in the statement shows
the claim. If $f(c)$ is not one of $u_{i}$ or $v_{i}$, then the
fact that $f_{i}$ satisfies the second item in the statement shows
the claim. This finishes the proof.

\global\long\def\L{\Lambda}%
\end{proof}
%
We now introduce the following condition. It will be used to extend
3-paths iteratively. 
\begin{defn}
We say that a 3-path $f:[a,b]\to V(\G)$ on a graph $\G$ is \textbf{bi-extensible}
if the following conditions are satisfied. 
\begin{enumerate}
\item $\G-f$ has no finite connected component.
\item There is a vertex $u$ in $\G-f$ at distance at most 3 from $f(b)$.
\item There is a vertex $v\ne u$ in $\G-f$ at distance at most 3 from
$f(a)$.
\end{enumerate}
If only the two first conditions are satisfied, we say that $f$ is
\textbf{right-extensible}.
\end{defn}

We first need to show that bi-extensible and right-extensible 3-paths
exist. The proofs are easy, and are given by completeness. 

\begin{lem}
\label{lem:existence-right-extensible}Let $\G$ be an infinite, connected,
and locally finite graph. Then for any pair of vertices $u$ and $v$,
there is a right-extensible 3-path which starts at $u$ and visits
$v$. 
\end{lem}

\begin{proof}
Let $f$ be a path joining $u$ and $v$, which exists by connectedness.
Now define $\L$ as the graph induced in $\G$ by the vertices in
$f$ and the ones in the finite connected components of $\G-f$. Notice
that as $\G$ is locally finite, there are finitely many such connected
components, and thus $\L$ is a finite graph. 

As $\G$ is connected there is some vertex $w$ in $\L$ which is
adjacent to some vertex outside $\L$. By \prettyref{lem:lema-tecnico}
there is a 3-path $f'$ which is Hamiltonian on $\L$, starts at $u$
and ends in $w$. We claim that this $f'$ is right-extensible. Indeed,
our choice of $\L$ ensures that $\G-f'$ has no finite connected
component, and our choice of the endpoint $w$ of $f'$ ensures that
the second condition in the definition of right extensible 3-path
is satisfied. This finishes the proof.
\end{proof}
%
\begin{lem}
\label{lem:existence-bi-extensible}Let $\G$ be an infinite, locally
finite, connected graph, and let $w$ be a vertex in $\G$. Then there
is a bi-extensible 3-path which visits $w$.
\end{lem}

\begin{proof}
Let $w'\ne w$ be adjacent to $w$, and define $\L$ to be the subgraph
of $\G$ induced by the vertices $w$, $w'$ and the ones in the finite
connected components of $\G-\{w,w'\}$. Thus $\L$ is a finite connected
subgraph of $\G$, has at least two vertices, and $\G-\L$ has no
finite connected component. 

As $\G$ is connected, there is a vertex $u$ in $\L$ which is adjacent
to some vertex $u'$ in $\G-\L$. As $\G-\L$ has no finite connected
component, there is another vertex $v'$ in $\G-\L$ adjacent to $u'$.
Finally as $\L$ is connected and has at least two vertices, there
is a vertex $v$ in $\L$ adjacent to $u$. Now we invoke \prettyref{lem:lema-tecnico}
on $\L$ to obtain a 3-path $f$ which is Hamiltonian and goes from
$u$ to $v$. We claim that $f$ is bi-extensible. Indeed, our choice
of $\L$ ensures that $\G-f$ has no finite connected component. Moreover,
$u$ is at distance one from the vertex $u'$ in $\G-f$, and $v$
is at distance at most 3 from the vertex $v\ne u$ in $\G-f$. This
finishes the proof. 
\end{proof}
%
We now show that under suitable conditions, a bi-extensible 3-path
can indeed be extended to a larger bi-extensible 3-path, and we can
choose the new 3-path to visit some vertex $w$. The second condition
in \prettyref{lem:lema-tecnico} will be important in the following
proof. 

\begin{lem}
\label{lem:double-extensible-paths}Let $\G$ be an infinite, locally
finite, and connected graph. Let $f$ be a bi-extensible 3-path on
$\G$, and let $u\ne v\in\G-f$ be two vertices at distance at most
3 from the first and last vertex of $f$, respectively. If $w$ is
a vertex in the same connected component of $\G-f$ that some of $u$
or $v$, then there is a 3-path $f'$ which extends $f$, is bi-extensible
on $\G$, and visits $w$. Moreover, we can assume that the domain
of $f'$ extends that of $f$ in both directions.
\end{lem}

\begin{proof}
If $u$ and $v$ lie in different connected components of $\G-f$,
then then the claim is easily obtained by applying \prettyref{lem:existence-right-extensible}
on each of these components. Indeed, by \prettyref{lem:existence-right-extensible}
there are two right extensible 3-paths $g$ and $h$ in the corresponding
connected components of $\G-f$, such that $g$ starts at $u$, $h$
starts at $v$, and some of them visits $w$. Then the concatenation
of $-g$, $f$ and $h$ satisfies the desired conditions. 

We now consider the case where $u$ and $v$ lie in the same connected
component of $\G-f$, this graph will be denoted $\L$. Thus $\L$
is infinite, locally finite, connected, contains $u$, $v$,  and
$w$. 

We claim that there are two right-extensible 3-paths on $\L$, $g$
and $h$, satisfying the following list of conditions: $g$ starts
at $u$, $h$ starts at $v$, some of them visits $w$, and $V(g)\cap V(h)=\emptyset$.
In addition, $(\L-g)-h$ has no finite connected component, and has
two different vertices $u'$ and $v'$ such that $u'$ is at distance
at most 3 from the last vertex of $g$, and $v'$ is at distance at
most 3 from the last vertex of $h$. 

Suppose that we have $g,h$ as before. Then we can define a 3-path
$f'$ by concatenating $-g$, $f$ and then $h$. It is easily seen
from the properties above that then $f'$ satisfies the conditions
in the statement. 

We now construct $g$ and $h$. For this purpose, we take $\L'$ to
be a connected finite subgraph of $\L$ which contains $u,v,w$ and
such that $\L-\L'$ has no finite connected component. 

The graph $\L'$ can be obtained, for example, as follows. As $\Lambda$
is connected, we can take a path $f_{u}$ from $u$ to $w$, and a
path $f_{v}$ from $v$ to $w$. Then define $\L'$ as the graph induced
by the vertices in $V(f_{v})$, $V(f_{u})$, and all vertices in the
finite connected components of $(\L-f_{v})-f_{u}$. This graph has
the desired properties by construction. 

Let $p$ be a Hamiltonian 3-path on $\L'$ from $u$ to $v$ as in
\prettyref{lem:lema-tecnico}, we will split $p$ aproppiately to
obtain $g$ and $h$. Let $w_{1}$ be a vertex in $\L'$ which is
adjacent to some vertex $v'$ in $\L-\L'$, which exists as $\L$
is connected. Now we need the second condition in \prettyref{lem:lema-tecnico},
which says that there is a vertex $w_{2}$ in $\L'$ whose distance
from $w_{1}$ is at most 2, and such that $p$ visits consecutively
$\{w_{1},w_{2}\}$. We will assume that $p$ visits $w_{2}$ after
visiting $w_{1}$, the other case being symmetric. As $\L-\L'$ has
no finite connected component, there is a vertex $u'$ in $\L$ which
is outside $\L'$ and is adjacent to $v'$. Thus, $w_{1}$ is at distance
at most 2 from $u'$, and $w_{2}$ is at distance at most 3 from $v'$. 

We define $g$ and $h$ by splitting $p$ at the vertex $w_{1}$.
More precisely, let $[a,c]$ be the domain of $p$, and let $b$ be
such that $p(b)=w_{1}$. Then $h$ is defined as the restriction of
$p$ to $[a,b]$, and $-g$ is defined as the restriction of $p$
to $[b+1,c].$ Thus $h$ is a 3-path from $v$ to $w_{1}$, and $g$
is a 3-path from $u$ to $w_{2}$. By our choice of $\L'$ and $p$,
it is easily seen that the 3-paths $h$ and $g$ satisfy the mentioned
list of conditions, and this finishes the proof. 
\end{proof}
%
Observe that when $\G$ has one or two ends, the hypothesis of \prettyref{lem:double-extensible-paths}
on $u,v$ and $w$ are always satisfied. We obtain a very simple and
convenient statement: we can extend a bi-extensible 3-path so that
it visits a vertex of our choice. 
\begin{cor}
\label{cor:one-two-ends}Let $\G$ be a locally finite, and connected
graph with one or two ends. Let $f$ be a bi-extensible 3-path, and
let $w$ be a vertex. Then $f$ can be extended to a bi-extensible
3-path which visits $w$. Moreover we can assume that the domain of
the new 3-path extends that of $f$ in both directions. 
\end{cor}

We are now in position to prove some results about bi-infinite 3-paths.
We start with the Hamiltonian case, which is obtained by iteration
of \prettyref{cor:one-two-ends}. When we deal with bi-infinite 3-paths,
we use the same notation and abreviations introduced before for 3-paths,
as long as they are well defined. 

\begin{prop}
\label{prop:hamiltonian-3-trail-one-two-ends}Let $\G$ be a one or
two ended, connected, locally finite graph. Then it admits a bi-infinite
Hamiltonian 3-path. 
\end{prop}

\begin{proof}
Let $(v_{i})_{i\in\N}$ be a numbering of the vertex set of $\G$.
We define a sequence of bi-extensible 3-paths $(f_{i})_{i\in\N}$
on $\G$ in the following recursive manner. We define $f_{0}$ as
a bi-extensible 3-path which visits $v_{0}$, this is possible by
\prettyref{lem:double-extensible-paths}. Now assume that we have
defined the bi-extensible 3-path $f_{i}$, and that it visits $v_{i}$.
We define $f_{i+1}$ as a bi-extensible 3-path on $\G$ which extends
$f_{i}$, its domain extends the domain of $f_{i}$ in both directions,
and it visits $v_{i}$. This is possible by applying \prettyref{cor:one-two-ends}
to the graph $\G$ and the 3-path $f_{i}$.

Finally, we define a bi-infinite 3-path $f:\Z\to\G$ by setting $f(n)=f_{i}(n)$,
for $i$ big enough. Then $f$ is well defined because $f_{i+1}$
extends $f_{i}$ as a function, and the domains of $f_{i}$ exhaust
$\Z$. By construction $f$ visits every vertex exactly once, so it
is Hamiltonian. This finishes the proof. 
\end{proof}
We now proceed with the non Hamiltonian case, where there are no restrictions
on ends. We first prove that we can take a bi-infinite 3-path whose
deletion leaves no finite connected component. 

\begin{lem}
\label{lem:bi-infinte-3-trail-connected-complement}Let $\G$ be an
infinite, connected, and locally finite graph. Then it admits a bi-infinite
3-path $f$ such that $\G-f$ has no finite connected component. 
\end{lem}

\begin{proof}
By first applying \prettyref{lem:existence-bi-extensible} and then
iterating \prettyref{lem:double-extensible-paths}, we obtain a sequence
of bi-extensible 3-paths $(f_{i})_{i\in\N}$ on $\G$, such that $f_{i+1}$
extends $f_{i}$ for all $i\geq0$, and such that their domains exhaust
$\Z$. We define a bi-infinite 3-path $f:\Z\to\G$ by setting $f(n)=f_{i}(n)$,
for $i$ big enough. 

We claim that $\G-f$ has no finite connected component. We argue
by contradiction, let us assume that $\G_{0}$ is a nonempty and finite
connected component of $\G-f$. Define $V_{1}$ as the set of vertices
in $\G$ which are adjacent to some vertex in $\G_{0}$, but which
are not in $\G_{0}$. Then $V_{1}$ is nonempty for otherwise $\G$
would not be connected, and is finite because $\G$ is locally finite.
Thus there is a natural number $k$ such that $f_{k}$ has visited
all vertices in $V_{1}$. By our choice of $V_{1}$, $\G_{0}$ is
a nonempty and finite connected component of $\G-f_{k}$, and this
contradicts that $f_{k}$ is bi-extensible.
\end{proof}
%
Now the proof of the following result is by iteration of \prettyref{lem:bi-infinte-3-trail-connected-complement}. 
\begin{prop}
\label{prop:existencia-3-paths}Let $\G$ an infinite, connected,
and locally finite graph. Then there is a set of bi-infinite 3-paths
$f_{i}:\Z\to\G$, $i\in I$, such that $V(\G)=\bigsqcup_{i\in I}V(f_{i})$. 
\end{prop}

\begin{proof}
As $\G$ is infinite, connected, and locally finite, we can apply
\prettyref{lem:bi-infinte-3-trail-connected-complement} to obtain
a bi-infinite 3-path $f_{0}$ such that $\G-f_{0}$ has no finite
connected component. Each connected component of $\G-f_{0}$ is an
infinite, connected, and locally finite graph, so we can apply \prettyref{lem:bi-infinte-3-trail-connected-complement}
on each of them. Iterating this process in a tree-like manner, we
obtain a family of 3-paths $f_{i}:\Z\to\G$, $i\in I$. Observe that
the countability of the vertex set ensures that $V(\G)=\bigsqcup_{i\in I}V(f_{i})$

\begin{comment}
We now give a more precise definition. We will use as index set a
set of words $W\subset\N^{\ast}$, which will be defined in a recursive
manner. For each $w\in W$ we define a subgraph $\G_{w}$ of $\G$,
a 3-path $f_{w}$ on $\G_{w}$, and the words that will be incorporated
after $w$.
\begin{proof}
Let $0\in W$, and let $\G_{0}=\G$.

Now assume that some $w\in\N^{\ast}$ has been incorporated to $W$,
and that we have defined an infinite, connected, and locally finite
graph $\G_{w}$. By \prettyref{lem:bi-infinte-3-trail-connected-complement},
there exists a bi-infinite 3-path $f_{w}$ on $\G_{w}$ such that
$\G_{w}-f_{w}$ has no finite connected components. We name this components
as $\G_{ws}$, for $s$ in a subset $S$ of $\N$ , and incorporate
to $W$ the set of words $\{ws:s\in S\}$.
\end{proof}
\end{comment}
\end{proof}
We now derive \prettyref{thm:t1} and \prettyref{thm:t2} from this
statements in terms of bi-infinite 3-paths.
%
\begin{proof}[Proof of \prettyref{thm:t1}]
 Let $\G$ be a graph as in the statement, and let $f_{i},i\in I$
as in \prettyref{prop:existencia-3-paths}. We define $\ast:V(\G)\times\Z\to V(\G)$
by the expression
\begin{align*}
v\ast n & =f(f^{-1}(v)+n),\ n\in\N,
\end{align*}
where $f$ is the only $f_{i}$ such that $v$ is visited by $f_{i}$.
Observe that $v\ast1$ is well defined because $V(\G)=\bigsqcup_{i\in I}V(f_{i})$.
This defines a translation-like action by $\Z$, where the distance
from $v$ to $v\ast1$, $v\in V(\G)$ is uniformly bounded by 3.
\end{proof}
%
\begin{proof}[Proof of \prettyref{thm:t2}]
 Let $\G$ be a graph as in the statement. By \prettyref{prop:hamiltonian-3-trail-one-two-ends},
$\G$ admits a Hamiltonian bi-infinite 3-path $f$. This defines a
translation-like action by $\Z$ as in the proof of \prettyref{thm:t1},
and it is transitive because $f$ is Hamiltonian. 

We now prove that a locally finite graph which admits a transitive
translation-like action by $\Z$ is connected and has one or two ends.
This is stated in \cite[Theorem 3.3]{seward_burnside_2014} for graphs
with uniformly bounded vertex degree, but the proof can be applied
to locally finite graphs. We describe here another argument.

Let $\G$ be a locally finite graph which admits a transitive translation-like
action by $\Z$. Connectedness is clear. It is also clear that $\G$
must be infinite, so it does not have zero ends. Assume now that it
has at least 3 ends to obtain a contradiction. Denote this translation-like
action by $\ast$, and let $J=\max\{d_{\G}(v,v\ast1)\mid v\in V(\G)\}$.
Now take any vertex $v\in V(\G)$, and define the function $f:\Z\to V(\G)$
by $f(n)=v\ast n,n\in\Z$. This function is bijective and sends consecutive
integers to vertices in $\G$ at distance $d_{\G}$ at most $J$. 

As $\G$ has at least $3$ ends, there is a finite set of vertices
$V_{0}$ such that $\G-V_{0}$ has at least three infinite connected
components, which we denote $\G_{1},$ $\G_{2}$ and $\G_{3}$. By
enlarging $V_{0}$ if necessary, we can assume that any pair of vertices
$u,v$ which lie in different connected components in $\{\G_{1},\G_{2},\G_{3}\}$
are at distance $d_{\G}$ at least $J+1$. As $V_{0}$ is finite,
there is a finite set $[n,m]\subset\Z$ such that $f([n,m])$ contains
$V_{0}$. By our choice of $V_{0},$ $n$ and $m$, the set of vertices
$f[m+1,\infty)$ is completely contained in one of $\G_{1}$, $\G_{2}$,
or $\G_{3}$. The same holds for $f(-\infty,n-1]$, and thus the remaining
infinite component is empty. This is a contradiction, and finishes
the proof.
\end{proof}
We end this section by repeating a problem left in \cite[Problem 3.5]{seward_burnside_2014}.
\begin{problem}
Find necessary and sufficient conditions for a graph to admit a transitive
translation-like action by $\Z$. 
\end{problem}

\prettyref{thm:t2} shows that the number of ends and connectedness
completely classify locally finite graphs which admit a transitive
translation-like action by $\Z$. The case of graphs with vertices
of infinite degree may be different, as there is not a uniquely defined
notion of ends there. This is discussed in detail in \cite{diestel_graphtheoretical_2003}. 

\section{Computable translation-like actions by $\protect\Z$ }\label{sec:Computable-translation-like-acti}

In this section we prove that every finitely generated group with
decidable word problem admits a computable translation-like by $\Z$
with decidable orbit membership problem (\prettyref{thm:maintheorem}).
Let us recall here the proof scheme. For groups with at most two ends
this is obtained from a computable version of \prettyref{thm:t2}.
For the remaining case, we show that the group contains a subgroup
isomorphic to $\Z$ with decidable subgroup membership problem. 

The reader may observe that in the case where the group has two ends
we provide two different proofs for \prettyref{thm:maintheorem}.
Indeed a group with two ends is virtually $\Z$, and it would be easy
to give a more direct proof, but the intermediate statements have
independent interest (\prettyref{thm:transitive-computable-action-on-computable-graph}
and \prettyref{prop:stallings-descomposicion-calculable}). 

\subsection{Computable and transitive translation-like actions}

In this subsection we prove the following result, and derive from
it \prettyref{thm:maintheorem} for groups with one or two ends.

\begin{thm}[Computable \prettyref{thm:t2}]
\label{thm:transitive-computable-action-on-computable-graph} A highly
computable graph which is connected and has one or two ends admits
a computable and transitive translation-like action by $\Z$, where
the distance between a vertex $v$ and $v\ast1$ is uniformly bounded
by 3.
\end{thm}

Joining this with \prettyref{thm:t2} we obtain the following corollary. 
\begin{cor}
A highly computable graph (resp. a finitely generated infinite group
with decidable word problem) admits a transitive translation-like
action by $\Z$ if and only if it admits a computable one.
\end{cor}

As discussed in the introduction, this shows great contrast between
transitive translation-like actions and other infinite objects that
arise in graph theory, at least from the perspective of computability
theory. 

We now proceed with the proof, which is divided in different lemmas.
Recall that \prettyref{thm:t2} was obtained by constructing a bi-infinite
Hamiltonian 3-path (\prettyref{prop:hamiltonian-3-trail-one-two-ends}).
In its proof we constructed a sequence of 3-paths $(f_{i})_{i\in\N}$
satisfying certain conditions. We will show that these conditions
are decidable, and this will suffice to compute the sequence by doing
an exhaustive search. For this we will need to separate the one and
two ended case. We start with the following result, which is used
in both cases. 
\begin{lem}
\label{lem:semidecidible-complemento-con-componente-finita}Let $\G$
be a highly computable graph which is connected. There is an algorithm
which on input a finite set of vertices $V\subset V(\G)$, halts if
and only if $\G-V$ has some finite conected component. 
\end{lem}

\begin{proof}
The intuitive idea is that we can discover a finite connected component
of $\G-V$ by computing a finite but big enough subgraph. Let $v_{0}\in\G-V$,
and for each $n\in\N$ let $\G_{n+1}$ denote the finite graph induced
by all vertices $v$ in $\G-V$ with $d_{\G}(v,v_{0})\leq n+1$. Observe
that if some vertex $v$ lies in $\G_{n+1}$ and its distance to $v_{0}$
is at most $n$, then we see all its $\G-V$ edges or neighbors in
the finite graph $\G_{n+1}$. This if for some $n_{0}$ there is a
finite connected subgraph of $\G_{n_{0}}$ which remains the same
on $\G_{n_{0}+1}$, then it will remain the same for all $n\geq n_{0}$,
and thus it is a finite connected component of $\G-V$. Thus the following
algorithm proves the claim: 

For each $n\in\N$, compute $\G_{n}$ and $\G_{n+1}$, which is possible
as $\G$ is highly computable, and do the following. Check if there
is a finite connected component of $\G_{n}$ with no vertices in $\G_{n+1}$,
and halt if this is the case. 
\end{proof}
We now can show that in the one ended case, the property of being
a bi-extensible 3-path is decidable. 
\begin{lem}
\label{lem:decidible-complemento-conexo-1-end}Let $\G$ be a highly
computable multigraph with one end. There is an algorithm which on
input a finite set of vertices $V\subset V(\G)$, decides if $\G-V$
has no finite connected component. 

In particular, it is decidable whether a 3-path is bi-extensible. 
\end{lem}

\begin{proof}
We prove the first claim. By \prettyref{lem:semidecidible-complemento-con-componente-finita},
it remains to prove the existence of an algorithm which halts on input
a finite set of vertices $V$ if and only if $\G-V$ is has no finite
connected component. As $\G$ has one end, this is equivlent to ask
if $\G-V$ is connected. The following algorithm halts on input $V$,
a finite set of vertices, if and only if $\G-V$ is connected:

Compute the set $V_{0}$ of vertices in $\G-V$ which are adjacent
to some vertex in $V$. For each pair of vertices of $V_{0}$, search
exhaustively for a path in $\G-V$ which joins them. If at some point
we have found such a path for every pair of vertices in $V_{0}$,
halt. 

We now observe that the second claim in the statement is obtained
from the first. Note that the second and third condition in the definition
of bi-extensible 3-path are clearly decidable. 
\end{proof}
We now proceed with the two ended case, where we make use of the finite
information contained in a finite set of vertices which disconnects
the graph. 

\begin{lem}
\label{lem:decidible-complemento-conexo-2-end}Let $\G$ be a highly
computable multigraph with two ends, and let $V'$ be a finite set
of vertices such that $\G-V'$ has two infinite connected components.
There is an algorithm which on input a finite set of vertices $V\subset V(\G)$
which contains $V'$, decides if $\G-V$ is has some finite connected
component. 

Now let $f_{0}$ be a bi-extensible 3-path on $\G$ such that $\G-f_{0}$
has two infinite connected components. It is decidable whether a 3-path
$f$ which extends $f_{0}$ is bi-extensible. 
\end{lem}

\begin{proof}
We now prove the first claim. By \prettyref{lem:semidecidible-complemento-con-componente-finita},
it remains to show that there is an algorithm which halts on input
$V$ if and only if $\G-V$ has no finite connected component. As
$V$ contains $V'$, we know that $\G-V$ has exactly two infinite
connected components. Now the following algorithm proves the claim.

On input $V$, compute the set $V_{0}$ of vertices in $\G-V$ which
are adjacent to some vertex in $V$. For each pair of vertices of
$V_{0}$, search exhaustively for a path in $\G-V$ which joins them.
Now halt if at some point we find enough paths so that $V_{0}$ can
be written as $V_{1}\sqcup V_{2}$, where every pair of vertices in
$V_{i}$ can be joined by a path in $\G-V$, $i=1,2$

We now observe that the second claim in the statement is obtained
from the first. Note that the second and third condition in the definition
of bi-extensible 3-path are clearly decidable. 
\end{proof}
%
We can now show tht the bi-infinite Hamiltonian path in \prettyref{prop:hamiltonian-3-trail-one-two-ends}
can be computed. 
\begin{prop}[Computable \prettyref{prop:hamiltonian-3-trail-one-two-ends}]
\label{prop:3-trail-hamiltonian-calculable}Let $\G$ be a highly
computable graph satisfying the hypothesis in \prettyref{thm:t2}.
Then it admits a Hamiltonian bi-infinite 3-path which is computable.
\end{prop}

\begin{proof}
Let $(v_{i})_{i\in\N}$ the numbering associated to the highly computable
graph $\G$, so $V(\G)=\{v_{i}|\ i\in\N\}$. 

Now let $f_{0}$ be a 3-path which is bi-extensible and visits $v_{0}$.
If $\G$ has two ends, then we also require that $\G-f_{0}$ has two
infinite connected components (in this case we do not claim that the
path $f_{0}$ can be computed from a description of the graph, but
it exists and can be specified with finite information). 

Now define the sequence of 3-path $(f_{i})_{i\in\N}$ in a recursive
and computable manner. Assume that $f_{i}$ has been defined, $i\geq0$,
and define $f_{i+1}$ as any bi-extensible 3-path on $\G$ such that
$f_{i+1}$ extends $f_{i}$, its domain extends the domain of $f_{i}$
in both directions, and it visits $v_{i}$. Its existence is guaranteed
by applying \prettyref{cor:one-two-ends} to the graph $\G$ and the
3-path $f_{i}$. Moreover, it can be computed by searching exhaustively,
the conditions imposed on $f_{i+1}$ are decidable thanks to \prettyref{lem:decidible-complemento-conexo-1-end}
and \prettyref{lem:decidible-complemento-conexo-2-end}.

The remaining of the proof is the same as before. We define $f:\Z\to\G$
by $f(n)=f_{i}(n)$, for $i$ big enough, and this is a and Hamiltonian
3-path on $\G$. Moreover, it is computable becuse the sequence $(f_{i})_{i\in\N}$
is computable. 
\end{proof}
We can now prove \prettyref{thm:transitive-computable-action-on-computable-graph}.

\begin{proof}[Proof of \prettyref{thm:transitive-computable-action-on-computable-graph}]
Let $f:\Z\to V(\G)$ be a bi-infinite 3-path on $\G$ which is Hamiltonian
and computable, which exists thanks to \prettyref{prop:3-trail-hamiltonian-calculable}.
It is enough to observe that the translation-like action $\ast:V(\G)\times\Z\to V(\G)$
defined by 
\begin{align*}
v\ast n & =f(f^{-1}(v)+n),\ n\in\N
\end{align*}
is computable.
\end{proof}
%
As mentioned before, \prettyref{thm:transitive-computable-action-on-computable-graph}
implies immediately that \prettyref{thm:maintheorem} holds for groups
with one or two ends. 
\begin{proof}[Proof of \prettyref{thm:maintheorem} for groups with one or two ends]
 Let $G$ be a finitely generated infinite group with one or two
ends and decidable word problem, let $\G=\cay(G,S)$ where $S$ is
a finite set of generators for $G$, and endow $G$ with a computable
numbering $\nu$. This numbering makes $\G$ a highly computable graph.
Now \prettyref{thm:transitive-computable-action-on-computable-graph}
yields a computable and transitive translation-like action on $(V(\G),d_{\G})$,
which is also a transitive and computable translation-like action
on $(G,d_{S})$. 
\end{proof}

\subsection{Computable normal forms from Stalling's theorem}

In this subsection we prove \prettyref{thm:maintheorem} for groups
with two or more ends. For this we will show that a finitely generated
group with two or more ends has a subgroup isomorphic to $\Z$ with
decidable subgroup membership problem. This will be obtained from
the computability of the normal form associated to Stalling's structure
theorem (see \prettyref{prop:stallings-descomposicion-calculable}
below).

For the reader's convenience, we recall some facts about HNN extensions,
amalgamated products, and normal forms. We refer the interested reader
to \cite[Chapter IV]{lyndon_combinatorial_2001}. Let $H=\langle S_{H}|R_{H}\rangle$,
$t$ a symbol not in $S_{H}$, and an isomorphism $\phi:A\to B$ between
subgroups of $H$. The \textbf{HNN extension} relative to $H$ and
$\phi$ is the group with presentation
\[
H\ast_{\phi}:=\langle S_{H},t|\ R_{H},tat^{-1}=\phi(a),\forall a\in A\rangle.
\]

Now let $T_{A}\subset H$ (respectively $T_{B}$) be a fixed set of
representatives for equivalence classes of $H$ modulo $A$ (respectively
$B$). A sequence of group elements $h_{0},t^{\epsilon_{1}},h_{1},\dots,t^{\epsilon_{n}},h_{n}$
is in \textbf{normal form} if $\epsilon_{i}\in\{1,-1\}$ and the following
conditions are satisfied:
\begin{enumerate}
\item $h_{0}\in H$
\item if $\epsilon_{i}=-1$, then $h_{i}\in T_{A}$
\item if $\epsilon_{i}=1$, then $h_{i}\in T_{B}$
\item There is no subsequence of the form $t^{\epsilon},1_{H},t^{-\epsilon}$.
\end{enumerate}
For every $g\in H\ast_{\phi}$ there exists a unique sequence in normal
form whose product equals $g$ in $H\ast_{\phi}$. The situation is
analogous for amalgamated products, which we define now. Consider
two groups $H=\langle S_{H}|\ R_{H}\rangle$ and $K=\langle S_{K}|R_{K}\rangle$,
and an isomorphism $\phi:A\to B$ between the subgroups $A\ls H$
and $B\ls K$. The \textbf{amalgamated product} of $H$ and $K$ relative
to $\phi$ is the group with presentation
\[
H\ast_{\phi}K=\langle S_{H},S_{K}|\ a=\phi(a),\forall a\in A\rangle.
\]

Let $T_{A}\subset H$ (respectively $T_{B}\subset K$) be a fixed
set of representatives for $H$ modulo $A$ (respectively $K$ modulo
$B$). Then a sequence of group elements $c_{0},c_{1},\dots,c_{n}$
is in \textbf{normal form }if
\begin{enumerate}
\item $c_{0}$ lies in $A$ or $B$
\item For $i\geq1$, each $c_{i}$ is in $T_{A}$ or $T_{B}$
\item For $i\geq1$ $c_{i}\neq1$
\item successive $c_{i}$ alternate between $T_{A}$ and $T_{B}$
\end{enumerate}
For each element $g\in H\ast_{\phi}K$, there exist a unique sequence
in normal form whose product equals $g$ in $H\ast_{\phi}K$.

In the terms previously defined, we have:
\begin{thm}[Stalling's structure theorem, \cite{dunwoody_cutting_1982}]
Let $G$ be a finitely generated group with two or more ends. Then
one of the following occurs:
\begin{enumerate}
\item $G$ is an HNN extension $H\ast_{\phi}$.
\item $G$ is an amalgamated product $H\ast_{\phi}K$.
\end{enumerate}
In both cases the corresponding isomorphism $\phi\colon A\to B$ is
between finite and proper subgroups $A$ and $B$.
\end{thm}

A classical application of normal forms is the following: if a group
$H$ has decidable word problem, and we take an HNN extension $H\ast_{\phi}$
satisfying some particular hypothesis, then the extension $H\ast_{\phi}$
has computable normal form, and as consequence, decidable word problem
\cite[pp 185]{lyndon_combinatorial_2001}. For our application, we
need a sort of converse of this result. The proof is direct, but we
were unable to find this statement in the literature.
\begin{thm}
\label{prop:stallings-descomposicion-calculable} Let $G$ be a finitely
generated group with two or more ends and decidable word problem.
Then the normal forms associated to the decomposition of $G$ as HNN
extension or amalgamated product is computable.
\end{thm}

\begin{proof}
Let us assume that we are in the first case, so $G$ is (isomorphic
to) an HNN extension of a group $H$:
\[
H\ast_{\phi}:=\langle S_{H},t|\ R_{H},tat^{-1}=\phi(a),a\in A\rangle.
\]

Let us show that the normal forms of $H\ast_{\phi}$ described above
is computable. For this purpose we will show that we can computably
enumerate sequences of words $w_{1},\dots,w_{n}$ whose corresponding
group elements $(w_{1})_{G},\dots,(w_{n})_{G}$ are in normal form
(for fixed sets $T_{A}$ and $T_{B}$), and be sure to enumerate normal
forms for all group elements. To compute the normal form of a group
element $w_{G}$ given by a word $w$, we just enumerate such sequences
$w_{1},\dots,w_{n}$ until we find one satisfying $w=_{G}w_{1}\dots w_{n}$.
This is possible as $G$ has decidable word problem. 

First note that the fact that $G$ is finitely generated and $A$
is finite forces $H$ to be finitely generated (see \cite[page 35]{cohen_combinatorial_1989}),
so we can assume that $S_{H}$ is a finite set. Moreover $H$ has
decidable word problem because this property is inherited by subgroups. 

Now we claim that $A=\{a_{1},\dots,a_{n}\}$ has decidable membership
problem in $H$. Indeed, to decide if $w_{H}\in A$, we just have
to check $w=_{H}a_{i}$ for $i=1,\dots,m$; this is possible as $A$
is finite and the word problem of $H$ is decidable. As a consequence
of this, we can also decide if $u\in Av$ for any $u,v\in(S_{H}\cup S_{H}^{-1})^{*}$,
as this is equivalent to decide if $uv_{H}^{-1}\in A$. The same is
true for $B$.

Let us show how to enumerate a set of words $W_{A}\subset(S_{H}\cup S_{H}^{-1})^{*}$
whose corresponding group elements are a colection of representatives
for $H$ modulo $A$, namely $T_{A}$. First define $u_{0}$ be the
empty word. Now assume that words $u_{0},\dots,u_{n}$ have been selected
and search for a word $u_{n+1}\in(S_{H}\cup S_{H}^{-1})^{*}$ which
is not in $Au_{0},\dots,Au_{n}$. It is clear then that $W_{A}$ is
a computably enumerable set of words, and that the set $T_{A}$ of
the group elements of $H$ corresponding to these words is a set of
representatives for $H$ modulo $A$.

A set $W_{B}$ corresponding to $T_{B}$ can be enumerated analogously.
Thus we can enumerate normal forms $w_{1},\dots,w_{n}$ as follows:
$w_{1}$ is an arbitrary element of $(S_{H}\cup S_{H}^{-1})^{*}$,
and the rest are words of $W_{A},$ $W_{B}$, or $\{t,t^{-1}\}$ so
as to alternate as in the definition. This finishes the proof for
the case of an HNN extension.

If $G$ is an amalgamated product, the fact that $G$ and $C$ are
finitely generated forces $H$ and $K$ to be finitely generated (see
\cite[page 43]{cohen_combinatorial_1989}), and thus we can assume
that $S_{H}$ and $S_{K}$ are finite sets. The sets $W_{A}$ and
$W_{B}$ corresponding to $T_{A}$ and $T_{B}$ can be computably
enumerated in the same manner as above, and the rest of the argument
is identical to the HNN case.
\end{proof}
Now the proof of the following result is inmediate. 

\begin{cor}
\label{cor:containment-Z-decidable-membership-problem}Let $G$ be
a finitely generated group with two or more ends and decidable word
problem. Then it has a subgroup isomorphic to $\Z$ with decidable
subgroup membership problem.
\end{cor}

\begin{proof}
If $G$ is an HNN extension, then the group $\langle t\rangle\ls G$
has decidable membership problem. Indeed a group element $g$ is in
this subgroup if and only if the normal form of $g$ or $g^{-1}$
is $1,t,1\dots,t,1$.

If $G$ is an amalgamated product, let $u\in T_{H}$, $v\in T_{K}$
be any pair of non trivial elements. Then the subgroup $\langle uv\rangle\ls G$
is infinite cyclic and has decidable membership problem, as a group
element $g$ is in this subgroup if and only if the normal form of
$g$ or $g^{-1}$ is $u,v,\dots,u,v$.
\end{proof}
As mentioned in the introduction, for translation-like actions coming
from subgroups, the two properties (decidable orbit membership problem,
and decidable subgroup membership problem) become equivalent. 

\begin{prop}
\label{prop:obvio} Let $G$ be a finitely generated group. Then a
subgroup $H\ls G$ has decidable membership problem if and only if
the action of $H$ on $G$ by right translations has decidable orbit
membership problem. 
\end{prop}

\begin{proof}
Let us denote by $\ast$ the right action $G\times H\to G$, $(g,h)\mapsto gh$.
Now we just have to note that two elements $g_{1},g_{2}\in G$ lie
in the same $\ast$ orbit if and only if $g_{1}g_{2}{}^{-1}\in H$,
and an element $g\in G$ lies in $H$ if and only if it lies in the
same $\ast$ orbit as $1_{G}$.

It is clear how to rewrite this in terms of words, but we fill the
details for completeness. The set $\{w\in(S\cup S^{-1})^{*}|\ w_{G}\in H\}$
equals the set of words $w\in(S\cup S^{-1})^{*}$ such that $w$ and
$1_{G}$ lie in the same orbit, which is a decidable set assuming
that $\ast$ has decidable orbit membership problem. We now assume
that $\{w\in(S\cup S^{-1})^{*}|\ w_{G}\in H\}$ is decidable. Given
two words $u,v\in(S\cup S^{-1})^{*}$, they lie in the same $\ast$
orbit if and only if the word $uv^{-1}$ lie in $\{w\in(S\cup S^{-1})^{*}|\ w_{G}\in H\}$. 
\end{proof}
%
We can now finish the proof of \prettyref{thm:maintheorem}, where
the only remaining step is to verify the computability of the action. 
\begin{proof}[Proof of \prettyref{thm:maintheorem} for groups with two or more
ends]
Let $G$ be a finitely generated infinite group with decidable word
problem. By \prettyref{cor:containment-Z-decidable-membership-problem}
there is an element $c\in G$ such that $\langle c\rangle$ is isomorphic
to $\Z$, and has decidable subgroup membership problem in $G$. Thus
the right action $\ast:G\times\Z\to G,$ $(g,n)\mapsto(gc^{n})$ has
decidable orbit membership problem by \prettyref{prop:obvio}.

It only remains to verify that this action is computable. This is
obvious in terms of words, but we write the details for completeness.
Let $S$ be a finite set of generators for $G$, and let $u$ be a
word in $(S\cup S^{-1})^{*}$ such that $u_{G}=c$. Now, the function
which given a word $w\in(S\cup S^{-1})^{*}$ and an integer $n\in\Z$
outputs the concatenation $wu^{n}$, is a computable function. Without
loss of generality (\prettyref{prop:numberings-equivalent}) we can
endow $G$ with the numbering obtained from the surjection $(S\cup S^{-1})^{*}\to G,w\mapsto w_{G}$,
and the previous paragraph makes it clear that the fuction $\ast$
is computable.
\end{proof}
%

\section{Medvedev degrees of effective subshifts}\label{sec:medvedev}

In this section we prove \prettyref{thm:aplication}. For this we
need some definitions. Let $G$ be a finitely generated group, and
$A$ a finite set. A \textbf{subshift }is a subset $X\subset A^{G}$
which is closed in the prodiscrete topology, and invariant under the
group action $G\act A^{G}$ by left translations $(g\ast x)(h)\mapsto x(g^{-1}h).$
A \textbf{pattern }is a function $p:K\subset G\to A$ with finite
domain, and it determines the\textbf{ cylinder 
\[
[p]=\{x\in A^{G}|\ x(k)=p(k)\hspace{1em}\forall k\in K\}.
\]
}

Cylinders are closed and open, and they form a basis for this topology.
If $g\ast x\in[p]$ for some $g\in G$, we say that $p$ \textbf{appears}
on $x$. 

A set of patterns $\F$ defines the subshift $X_{\mathcal{\F}}$ of
all elements $x\in A^{G}$ such that no pattern of $\F$ appears in
$x$. We say that $X_{\F}$ is obtained by forbidding patterns of
$\F$. Conversely, any subshift can be obtained by forbidding patterns.

A subshift $X$ is called a \textbf{subshift of finite type }if it
can be obtained by forbidding finitely many patterns. We now discuss
a notion of effectiveness of subshifts, for which we need the following
definition.

A \textbf{pattern coding} $c$ is a finite set of tuples $\{(w_{1},a_{1}),\dots,(w_{k},a_{k})\}$,
where $w_{i}\in S^{*}$ and $a_{i}\in A$, and it is called \textbf{consistent}
if for each $i,j$, $w_{i}=_{G}w_{j}$ implies $a_{i}=a_{j}$. A consistent
pattern coding as above can be associated to the pattern $p(c):K\subset G\to A$,
where $K$ is the set of group elements associated to $\{w_{1},\dots,w_{k}\}$,
and $p((w_{i})_{G})=a_{i}$. 

A set of pattern codings $\mathcal{C}$ defines the subshift $X_{\mathcal{C}}$
of all elements $x\in A^{G}$ such that no pattern of the form $p(c)$
appears in $x$, where $c$ ranges over $\mathcal{C}$. Thus inconsistent
pattern codings are ignored by definition. A subshift $X$ is called
\textbf{effective }if $X=X_{\mathcal{C}}$ for a computably enumerable
set of pattern codings $\mathcal{C}$. 

The previous definition was introduced in \cite{aubrun_notion_2017}
to define effective subshifts without assumptions on the word problem
of the group. If we assume that $G$ has decidable word problem, we
can decide which pattern codings are consistent, and decide which
ones corresponds to the same pattern. This gives a bijective numbering
of the set of all patterns as in \prettyref{prop:enumerados-sobre},
and we can say that a set of patterns $\F$ is computably enumerable.
It is then clear that a subshift $X$ is effective if and only $X=X_{\F}$
for some computably enumerable set of patterns $\F$.

A third equivalent notion is that an effective subshift is a subshift
which is also an effectively closed subset of $A^{G}$. For this to
make sense, we need to define effectively closed subsets of $A^{G}$.
In what follows we will transfer all computability notions from $A^{\N}$
to $A^{G}$ using representations, in the sense of computability theory.
This is used to define the Medvedev degree of a subset of $A^{G}$,
and to show that this notion enjoys some stability properties. 

\subsection{Computability on $A^{\protect\N}$, and Medvedev degrees }\label{subsec:Computability-on-A^n}

Despite $\N$ not being a group, we will use the same definitions
of patterns and cylinders in $A^{\N}$. A word $w_{0}\dots w_{n}\in A^{*}$
can be identified with a pattern $\{0,\dots,n\}\to A$, and thus $[w]=\{x\in A^{\N}|\ x_{0}\dots x_{n}=w\}$. 

We review now some facts on computability on the Cantor space that
will be needed, the reader is refered to \cite{rogers_theory_1987}.
The intuitive meaning in the following definition is the following.
A function $\Phi$ on $A^{\N}$ is computable if there is a computable
function on words which computes it via finite prefixes, that is,
it takes as input a finite prefix of $x\in A^{\N}$ and outputs a
finite prefix of $\Phi(x)$. 
\begin{defn}[Computable function]
 A partial function $\Phi:D\subset A^{\N}\to B^{\N}$ is \textbf{computable}
if there exists a partial computable function $\phi:A^{*}\to B^{*}$
which is compatible with the prefix order on words\footnote{That is, if $u$ is a prefix of $v$, then $\phi(u)$ is a prefix
of $\phi(v)$.}, and such that $D$ is the set of sequences $x\in B^{\N}$ where
the following holds. As $n$ tends to infinity, $\phi(x_{0}\dots x_{n})$
is always defined, its length tends to infinity, and $\bigcap_{n}[\phi(x_{0}\dots x_{n})]$
is the singleton which only contains $\Phi(x)$. 
\end{defn}

An alternative definition uses the concept of \textit{oracle}, which
we have avoided to make this article more accesible\footnote{A reader familiar with the concept of \textit{oracle }may prefer this
definition: $\Phi$ is computable if there is a partial computable
function with oracle $\phi:\N\to B$, such that with oracle $x$ and
input $n\in\N$ returns the $n$-th coordinate of $\Phi(x)$, that
is $\phi^{x}(n)=\Phi(x)_{n}$ for $n\in\N$ and all $x$ in the domain
of $\Phi$.}.

\begin{comment}
\end{comment}

\begin{example}
The shift function $\sigma:2^{\N}\to2^{\N},\sigma x(n):=x(n+1)$ is
computable. It is given by the function $s:2^{*}\to2^{*}$, $s(w_{0}w_{1}\dots w_{n})=w_{1}\dots w_{n}$.
\end{example}

The following is a fairly standard fact, but we give a proof as it
will be used later. 
\begin{prop}
\label{prop:biycalculable}Let $f:\N\to\N$ be any computable bijection.
Then the homeomorphism 
\begin{align*}
F:A^{\N} & \to A^{\N}\\
x & \mapsto x\circ f
\end{align*}

is computable and with computable inverse $x\mapsto x\circ f^{-1}$.
\end{prop}

\begin{proof}
Note that a computable bijection of $\N$ has computable inverse.
Now we define the computable function $\phi$ as follows. On input
$a_{1}\dots a_{n}\in A^{\ast}$ compute $k=k(n)$ to be the biggest
natural number such that $\{0,\dots,k\}$ is contained in $\{f(0),\dots,f(n)\}$,
and then output $\phi(a_{0}\dots a_{n})=a_{f^{-1}(0)}\dots a_{f^{-1}(k)}$.
Note that $\lim_{n\to\inf}k(n)=\inf$, and thus $\lim_{n\to\inf}|\phi(a_{0}\dots a_{n})|=\inf$,
this shows that $F$ is defined on all inputs. By construction we
have that $F(x)=\cap_{n\in\N}[\phi(x_{0}\dots x_{n})]$. Thus we proved
that $F$ is computable. 

Replacing $f$ by $f^{-1}$ in the argument, we see that the inverse
of $F$, $x\to x\circ f^{-1}$ is also computable.
\end{proof}
\begin{defn}
Let $X\subset A^{\N}$ and $Y\subset B^{\N}$. The sets $X,Y$ are
\textbf{computably homeomorphic }if there exists partial computable
functions $\Phi:A^{\N}\to B^{\N}$, $\Psi:B^{\N}\to A^{\N}$, such
that $X$ (resp $Y$) is contained in the domain of $\Phi$ (resp
$\Psi$), and such that $\Psi(\Phi(x))=x\hspace{1em}\forall x\in X$.
\begin{comment}
If $X$ is effectively closed (see below), this is equivalent to the
existence of an injective function $\Phi:A^{\N}\to B^{\N}$ whose
domain is $X$, and $\Phi(X)=Y$.
\end{comment}
\end{defn}

\begin{example}
The sets $A^{\N}$ and $B^{\N}$ are computably homeomorphic for any
$A$ and $B$ finite. Indeed, the usual homeomorphism between these
sets is a computable function (for example, the one described in \cite[Theorem 2-97]{hocking_topology_1961}).
A particularly simple case is when $A=\{0,1,2,3\}$ and $B=\{0,1\}$.
A computable homeomorphism between $A^{\N}$ and $B^{\N}$ is given
by the letter-to-word substitutions
\[
0\mapsto00\hspace{1em}1\mapsto01\hspace{1em}2\mapsto10\hspace{1em}3\mapsto11.
\]

\begin{comment}
We now turn to effectively closed sets, which can be considered as
the computably enumerable subsets of $A^{\N}$.
\end{comment}
\end{example}

\begin{defn}
A subset $X\subset A^{\N}$ is \textbf{effectively closed}, denoted
$\Pi_{1}^{0}$, if some of the following equivalent conditions holds.
\begin{enumerate}
\item The complement of $X$ can be written as $\bigcup_{w\in L}[w]$ for
a computably enumerable set $L\subset A^{*}$.
\item We can semi decide if a word $w\in A^{*}$ satisfies $[w]\cap X=\emp$.
\item We can semi decide if a pattern $p:K\subset\N\to A$ satisfies $[p]\cap X=\emp$.
\end{enumerate}
\end{defn}

\begin{comment}
The folllowing are equivalent
\begin{enumerate}
\item $X\subset A^{\N}$ is effectively closed.
\begin{enumerate}
\item We can semi decide if a word $w\in A^{*}$ satisfies $[w]\cap X=\emp$.
\item We can semi decide if a pattern $p:K\Subset\N\to A$ satisfies $[p]\cap X=\emp$.
\end{enumerate}
\end{enumerate}
\begin{proof}
To see that the second item implies the third, notice that given a
pattern $p$, we can compute words $u_{i}$ such that $[p]=[u_{1}]\sqcup\dots\sqcup[u_{n}]$.
Then observe that $[p]\cap X=\emptyset$ if and only if $[u_{i}]\cap X=\emp$
for each $1\leq i\leq n$.

The reverse implication is inmeadiate, as is the equivalence between
the first two items.
\end{proof}
\end{comment}

\begin{example}
Let $T\subset A^{*}$ be a tree, which means a set of words closed
under prefix, and denote by $[T]$ the set of infinite paths of $T$,
$\{x\in A^{\N}|\ \forall n\in\N,x_{0}\dots x_{n}\in T\}.$ If the
tree $T$ is computable (as subset of $A^{*}$), then its set of infinite
paths $[T]=A^{\N}-\cup_{w\not\in T}[w]$ is an effectively closed
set. Conversely, it can be shown that every effectively closed set
is the set of paths of a computable tree.
\end{example}

\begin{comment}
, but we ask the reader to keep the geometric intuition of the tree
in the following discussion.

We will give an informal comparison of two different notions of computational
complexity.

A path $(x_{i})_{i\in\N}$ in a tree $T$ is a sequence of choices,
first choose the direction $x_{0}$, from this place, choose $x_{1}$,
etc. Many choices will take us to dead ends, that is, some finite
branch of the tree without infinite paths. The difficult task to construct
a path is to anticipate dead ends, as it is not legitimate to regret
our choices. How can we construct a path? if we know how is the tree
(that is, $T$ is a computable set), we may draw a picture of the
tree as big as we wish. Is this enough to \textit{compute} a path?
The answer is no, but that among undecidable problems, this is in
the easiest class.

It is easy to show that the question
\begin{quote}
Is the computable tree $T$ infinite?
\end{quote}
is as hard as the halting problem for algorithms. Turing degrees measure
how difficult is a (countable) problem, and with this ruler the mentioned
problem has difficulty $\boldsymbol{0'}$. The intuition is that the
tree is infinite, and even if we know the rules to draw it, we would
need to draw an infinite portion to make sure that it is indeed infinite,
this can not be done in finite time. Surprisingly, the problem of
making an infinite sequence of choices so as to avoid some decidable
bad cases (the dead ends) is much easier. In Turing degrees this is
phrased as: there is a point $x\in A^{\N}$ , an \textit{oracle},
such that $x'\equiv{}_{T}\boldsymbol{0'}$ (something much stronger
than $\,x<_{T}\boldsymbol{0'}$), and which computes paths in all
computable trees. Notice that such oracle can not decide if a tree
has paths, but we do not care if for trees without paths it outputs
nonsense. This is the content of the low basis theorem, first proved
in \cite{jockusch_pi0_1_1972}, and whose common statement is that
any effectively closed subset of $A^{\N}$ has a point $x$ with \textit{low
}Turing degree, that is, $x'\equiv_{T}\boldsymbol{0'}$.

With the definitions in \ref{subsec:Computability-on-A^g}, this can
be applied to effective subshifts on a suitable group, this tells
us us that effective subshifts always have configurations which are
not too uncomputable. Instead of looking at the complexity of individual
configurations, we may measure the complexity of the set of configurations.
This is the purpose of Medvedev degrees.
\end{comment}

\begin{defn}[Medvedev degrees]
Let $X\subset A^{\N},$ $Y\subset B^{\N}$. We say that $Y$ is \textbf{Medvedev
reducible} to $X$, written
\[
Y\leq_{\M}X,
\]
if there is a partial computable function $\Phi$ defined on all elements
of $X$, and such that $\Phi(X)\subset Y$. This is a preorder, and
induces the equivalence relation $\equiv_{\M}$ given by 
\[
X\equiv_{\M}Y\iff X\leq_{\M}Y\text{ and }Y\leq_{\M}X.
\]
Equivalence classes of $\equiv_{\M}$ are called \textbf{Medvedev
degrees}. Medvedev degrees form a lattice with interesting properties,
a survey on the subject is \cite{hinman_survey_2012}. If we regard
a set $X\subset A^{\N}$ as the set of \textit{solutions }to a problem,
then the Medvedev degree of $X$, $\deg_{\M}(X)$, measures how hard
is it to construct a solution, where hard means hard to compute. From
the definition, one can see that:
\end{defn}

\begin{enumerate}
\item If $X$ has a computable point $x_{c}$, then it is minimal for $\leq_{\M}$.
This is proved by noting that the constant function $\Phi:A^{\N}\to X$,
$x\mapsto x_{c}$ is computable. The Medvedev degree of this set is
denoted $0_{\M}$.
\item The empty set $\emp$ is maximal for $\leq_{\M}$. This reflects that
for $\leq_{\M}$, the hardest problem is one with no solutions.
\item If $X$ and $Y$ are computably homeomorphic, then $X\equiv_{\M}Y$.
\end{enumerate}
%
As $2^{\N}$ and $A^{\N}$ are computably homeomorphic for any finite
set $A$, it is enough to consider Medvedev degrees of subsets of
$2^{\N}$. 

An important sublattice of the lattice of Medvedev degrees is that
of Medvedev degrees of effectively closed subsets. This is a countable
class, its elements admit a finite description, and it exhibits many
interesting properties. Natural and geometrical examples of effectively
closed sets are subshifts of finite type, and more generally effective
subshifts. As mentioned in the introduction, it is known that all
$\Pi_{1}^{0}$ Medvedev degrees can be attained by two dimensional
subshifts of finite type \cite{simpson_medvedev_2014}, and one dimensional
effective subshifts \cite{miller_two_2012}. 

\subsection{Computability on $A^{G}$  }\label{subsec:Computability-on-A^g}

In this subsection we define the Medvedev degree of a subset of $A^{G}$
using representations. For the definition and its stability properties
it will be essential to assume that $G$ is a finitely generated infinite
group with decidable word problem. 

Representations are the uncountable version of numberings, as defined
in \prettyref{subsec:computability-on-countable}. We recall the following
definitions from \cite[Chapter 9]{brattka_handbook_2021}. A \textbf{represented
space} is a pair $(X,\delta)$ where $X$ is a set and $\delta$ is
a \textbf{representation} of $X$, that is, a partial surjection $\delta:\text{dom}(\delta)\subset A^{\N}\to X$.
A representation allows us to transfer the computability notions from
$A^{\N}$ to $X$. For example, in a represented space $(X,\delta)$,
a subset $Y\subset X$ is \textbf{effectively closed} when $\delta^{-1}(Y)\subset A^{\N}$
is an effectively closed set. Moreover, if $(X',\delta':A'{}^{\N}\to X')$
is another represented space, a function $F:X\to X'$ is \textbf{computable}
when $\delta'{}^{-1}\circ F\circ\delta:A^{\N}\to A'{}^{\N}$ is a
computable function. Finally, two representations of $X$, $\delta:A^{\N}\to X$
and $\delta':B^{\N}\to X$, are called \textbf{equivalent }if the
identity function $X\to X$ is computable between $(X,\delta)$ and
$(X,\delta')$. In this case, both representations induce the same
computability notions on $X$. %
\begin{comment}
\marginpar{cristobal: zparece correcto llamar a esto una representación?}
\end{comment}

We will consider the following representation, which is also a total
function and a homeomorphism. 
\begin{defn}
\label{def:gdelta} Let $G$ be a finitely generated infinite group
with decidable word problem, and $\nu$ a computable numbering of
$G$. We define the representation $\delta$ by

\begin{align*}
\delta:A^{\N} & \to A^{G}\\
x & \mapsto x\circ\nu^{-1}.
\end{align*}
\end{defn}

Recall from \prettyref{sec:Preliminaries} that a group as in the
statement admits a computable numbering. The next proposition shows
that the computability notions that we obtain on $A^{G}$ do not depend
on the choice of $\nu$. 
\begin{prop}
In the previous definition, any two computable numberings induce equivalent
representations. 
\end{prop}

\begin{proof}
Let $F:A^{G}\to A^{G}$ be the identity function. Then $\delta'{}^{-1}\circ F\circ\delta:A^{\N}\to A^{\N}$
is given by $x\mapsto x\circ\nu^{-1}\circ\nu'$. This is a computable
function by \prettyref{prop:biycalculable} and the fact that $\nu^{-1}\circ\nu':\N\to\N$
is a computable bijection of $\N$. 
\end{proof}
%
Indeed, computability notions on $A^{G}$ are also preserved by group
isomorphisms. 
\begin{prop}
\label{prop:stability}Consider a group $G'$ and a representation
for $A^{G'}$ as in \prettyref{def:gdelta}. If $f:G\to G'$ is a
group isomorphism, then the associated function $F:A^{G'}\to A^{G}$,
$x\mapsto x\circ f$ is computable.
\end{prop}

The proof is omitted, as it is identical to the previous one by applying
\prettyref{prop:isom-calculable}. This means that the computability
notions being considered on $A^{G}$ are preserved if we rename group
elements (for example, by taking different presentations of the same
group). 

We can now define the Medvedev degree of a subset of $A^{G}$. 
\begin{defn}
For a subset $X\subset A^{G}$, we define $\deg_{\M}X=\deg_{\M}(\delta^{-1}X)$.
\end{defn}

This definition does not depend on $\delta$, as long as $\delta$
comes from a computable numbering of $G$. Let us now review some
basic facts about effectively closed sets on $A^{G}$. With this representation
we recover the familiar description of effectively closed subsets
in terms of cylinders. 
\begin{prop}
\label{prop:ef-cerrado}A subset $X\subset A^{G}$ is effectively
closed if and only if we can semi decide if a pattern $p:K\subset G\to A$
satisfies $[p]\cap X=\emptyset$.
\end{prop}

\begin{proof}
Given a pattern $p:K\subset G\to A$, we just have to compute a pattern
$p':K\subset\N\to A$ such that $p=p'\circ\nu$. Then $[p]\cap X=\emptyset$
if and only if $[p']\cap\delta^{-1}(X)=\emptyset$, which can be semi
decided as $\delta^{-1}(X)$ is effectively closed. 
\end{proof}
In \cite[Lemma 2.3]{aubrun_notion_2017} it is shown that for recursively
presented group and in particular one with decidable word problem,
an effective subshift has a maximal -for inclusion- computably enumerable
set of pattern codings associated to forbidden patterns. In other
words, the set of all patterns $p$ with $[p]\cap X=\emptyset$ is
computably enumerable. Joining this with \prettyref{prop:ef-cerrado},
we obtain:
\begin{prop}
A subshift $X\subset A^{G}$ is effective if and only if it is an
effectively closed subset of $A^{G}$.

\begin{comment}
To see that (1) implies (2), define $\F$ to be the set of patterns
$p$ such that $[p]\cap X=\emptyset$, which is computably enumerable
by \ref{prop:ef-cerrado}. Let us justify that \ref{enu:-for-a} implies
that $X_{\F}$ is effectively closed. For this we fix a numbering
$\F=\{p_{i}|i\in\N\}$, and a pattern $p$ to semi decide if $[p]\cap X_{\F}=\emp$.
\begin{fact*}
By compactness of $A^{G}$, $[p]\cap X_{\F}=\emp$ if and only if
for some $K'\Subset G$ that contains $K$ and for some $n\in\N$,
every pattern $p':K'\to A$ that extend $p$ contain some of the forbidden
patterns $\{p_{1},\dots,p_{n}\}$.
\end{fact*}
\begin{proof}
Observe that we can decide if the previous condition holds for each
fixed $K'$, $p'$, and $n$. To semi decide if $[p]\cap X_{\F}=\emp$,
just test all cases for every $K'$, $p'$, and $n$, and accept if
we find the aforementioned condition to be true.
\end{proof}
\end{comment}
\end{prop}

\begin{comment}
With these definitions, the computability of the group operation $G^{2}\to G$
is transferred to the action of $G$ on $A^{G}$.
\begin{prop}
\label{prop:accion-calculable} For any group element $g\in G$, the
translation by $g$ on $A^{G}$
\begin{align*}
A^{G} & \to A^{G}\\
x & \mapsto g\ast x
\end{align*}

is computable.
\end{prop}

\begin{proof}
Denote by $F$ the function above. By \prettyref{prop:isom-calculable},
the function $G\to G,h\mapsto g^{-1}h$ is computable, and then there
is a computable bijection $f:\N\to\N$ such that $f(\nu(n))=\nu(g^{-1}\ast\nu(n))$.
Thus the function that corresponds to $F$ in $A^{\N}$ is 
\[
x\mapsto\delta^{-1}\circ F\circ\delta(x)=x\circ f.
\]
 This is a computable function in $A^{\N}$ by \prettyref{prop:biycalculable}.
\end{proof}
\end{comment}


\subsection{The subshift of translation-like actions, and the main construction}

In this subsection we describe how to code translation-like actions
as a subshift, and then we use this to prove \prettyref{thm:aplication}.
Let $G$ be an infinite group, $S$ a finite set of generators, and
$J\in\N$ (we will not assume yet that $G$ has decidable word problem). 
\begin{defn}
Let $T_{J}(\Z,G)$ be the set of all translation-like actions $\ast:\Z\times G\to G$
such that $\ensuremath{\{d_{S}(g,g\ast1)|\ g\in G}\}$ is bounded
by $J$.
\end{defn}

\begin{figure}
\begin{center}\includegraphics[width=1\columnwidth]{export_translation-like-sinpintar_svg_0be36ee36___2ccd932537e9568da8c003a2161108008403875ee21.pdf}\end{center}

\caption{Representation of some orbits of a translation-like action in $T_{2}(\protect\Z,\protect\Z^{2})$,
or alternatively, a finite piece of configuration in $X_{2}(\protect\Z,\protect\Z^{2})$.
In this case, $\protect\Z^{2}$ is endowed with the set of four generators
$S=\{(\pm1,0),(0,\pm1)\}$.}\label{fig:subshift-translation-like-action}
\end{figure}

Recall that $B(1_{G},J)$ is the ball $\{g\in G\mid d_{S}(g,1_{G})\leq J\}$.
We will consider the finite alphabet $B=B(1_{G},J)\times B(1_{G},J)$.
Informally, a configuration $x\in B^{G}$ can be thought of as having
an incoming and outogoing arrow at each $g\in G$. If $x(g)=(l,r)$,
we can think that $g$ has an outgoing arrow to $gr$, and an incoming
arrow from $gl$, see \prettyref{fig:subshift-translation-like-action}.

Any translation-like action $\ast\in T_{J}(\Z,G)$ defines the configuration
$x_{\ast}\in B^{G}$ by the condition

\[
\forall g\in G\hspace{1em}x_{\ast}(g)=(l,r)\iff g\ast-1=gl\text{ and }g\ast1=gr.
\]

\begin{defn}
Let $X_{J}(\Z,G)$ be the set of all $x_{*}$, where $*$ ranges over
$T_{J}(\Z,G)$. 

We will prove that $X_{J}(\Z,G)$ is a subshift, but we first introduce
some notation. $L$ and $R$ stand for the projections $B\to B(1_{G},J)$
to the left and right coordinate, respectively. Now an arbitrary element
$x\in B^{G}$ defines two semigroup actions as follows. For $m\in\Z_{\geq0}$
and $g\in G$, define the element $g\ast_{x}m$, by declaring $g\ast_{x}0=g$,
$g\ast_{x}1=g\cdot R(x(g))$, and $g\ast_{x}(m+1)=(g\ast_{x}m)\ast_{x}1.$
For $m\in\Z_{\leq0}$, we define $g\ast_{x}m$ by $g\ast_{x}-1=g\cdot L(x(g))$
and $g\ast_{x}(m-1)=(g\ast_{x}m)\ast_{x}-1.$ If $p:K\subset G\to B$
is a pattern, we give $g\ast_{p}m$ the same meaning as before, as
long as it is defined. 

It is easily seen that for arbitrary $x\in B^{G}$, we have 
\[
(g\ast_{x}n)\ast_{x}m=g\ast_{x}(n+m)
\]
for both $n,m$ positive or both negative integers. It may fail for
$n,m\in\Z$, but this is easily fixed with local rules. 
\end{defn}

\begin{prop}
\label{prop:X_j-es-subshift} The set $X_{J}(\Z,G)$ is a subshift.
If we assume that $G$ has decidable word problem, it is an effective
subshift. 
\end{prop}

\begin{proof}
We claim that $X_{J}(\Z,G)=X_{\J}$, where $\J$ is the set of all
patterns $p:B(1_{G},n)\to B$, $n\in\N$, such that some of the following
fails
\begin{enumerate}
\item $(1_{G}\ast_{p}1)\ast_{p}-1=1_{G}$, $(1_{G}\ast_{p}-1)\ast_{p}1=1_{G}$ 
\item For any nonzero $m\in\Z$, $1_{G}\ast_{p}m\ne1_{G}$.
\end{enumerate}
If $\ast\in T_{J}(\Z,G)$, it is clear that no pattern of $\J$ may
appear on $x_{\ast}$, by the definition of action and translation-like
action. Thus $X_{J}(\Z,G)\subset X_{\J}$. Now let $x\in X_{\J}$.
We claim that $\ast_{x}$ is a translation-like action. To see that
it is a group action, first note that for $g\in G$, $g\ast_{x}0=g$
by definition. An easy induction on $\max\{|n|,|m|\}$ shows that
$(g\ast_{x}n)\ast_{x}m=g\ast_{x}(n+m)$ for any $n,m\in\Z$ and $g\in G$.
The group action $\ast_{x}$ is free because of the second condition,
and the boundedness condition comes from the alphabet chosen. Thus
$\ast_{x}$ is a translation-like action.

To show that $x$ lies in $X_{J}(\Z,G)$, we have to check that it
is of the form $x_{\ast}$ for some translation-like action $\ast$,
and it is clear that $x=x_{(\ast_{x})}$. Thus $X_{J}(\Z,G)$ is a
subshift.

Now let us assume that $G$ has decidable word problem. Note that
the definition of $\ast_{p}$ above is recursive. Given an arbitrary
pattern $p$, and $m\in\Z$, we can decide if the group element $1_{G}\ast_{p}m$
is defined, and compute it. This shows that the conditions (1) and
(2) are decidable over patterns, and thus $\J$ is a decidable set.
This shows that $X_{\J}$ is an effective subshift. 
\end{proof}
We now describe a subshift on $G$ whose elements describe, simultaneously,
translation-like actions, and configurations from a subshift over
$\Z$. Let $A$ be an arbitrary finite alphabet, and let $B$ be the
alphabet already defined and which depends on the natural number $J$.
Note that elements of $(A\times B)^{G}$ can be conveniently written
as $(y,x)$ for $y\in A^{G}$ and $x\in B^{G}$; we will write $\pi_{A}:A\times B\to A$
and $\pi_{B}:A\times B\to B$ for the projections to the first and
second coordinate, respectively.
\begin{defn}
For a one dimensional subshift $Y\subset A^{\Z}$, let $Y[X_{J}(\Z,G)]$
be the set of all configurations $(y,x)\in(A\times B)^{G}$ such that
\begin{enumerate}
\item $x\in X_{J}(\Z,G)$, and
\item for any $g\in G$, the $A^{\Z}$ element defined by $y(m)=\pi_{A}(y(g\ast_{x}m)$
lies in $Y$.
\end{enumerate}
\end{defn}

\begin{prop}
\label{prop:Y=00005BX_j=00005D-es-subshift}The set $Y[X_{J}(\Z,G)]$
is a subshift. If we assume that $G$ has decidable word problem and
$Y$ is an effective subshift, then $Y[X_{J}(\Z,G)]$ is an effective
subshift. 
\end{prop}

\begin{proof}
Let $\F$ be the set of all patterns in $\Z$ that do not occur in
$X$, so that $X=X_{\F}$, and let $\J$ as before, so that $X_{\J}=X_{J}(\Z,G)$.
Define $\mathcal{H}$ to be the set of all patterns $p:B(1_{G},n)\to A\times B$,
$n\in\N$, such that one of the following occurs. Denote $q=\pi_{B}\circ p:B(1_{G},n)\to B$.
\begin{enumerate}
\item The pattern $q$ lies in $\J$.
\item For some $m\in\N$ the elements $g\ast_{q}1,\dots,g\ast_{q}m$ are
all defined, lie in $\text{\ensuremath{B(1_{G},n)}},$ and the pattern
$r:\{1,\dots,m\}\subset\Z\to A,r(k)=\pi_{A}(g\underset{q}{\ast}k)$
lies in $\F$.
\end{enumerate}
As before, it is a rutinary verification that $x\in Y[X_{J}(\Z,G)]$
if and only if $x\in X_{\mathcal{H}}$ .

Now assume that $G$ has decidable word problem. Let us argue that
$\mathcal{H}$ is a computably enumerable set, provided that the same
holds for $\F$. Given a pattern $p$, we can compute the pattern
$q$. The first condition is decidable, as we already proved that
$\J$ is a decidable set. For the second, note that given $p$ and
$m\in\N$, we can also compute the pattern $r$. As $\F$ is a computable
enumerable set, we can semi decide that $r$ is in $\F$. This shows
that $\mathcal{H}$ is a computably enumerable set, and finishes the
proof.
\end{proof}
\begin{figure}[h]
\begin{center}\includegraphics[width=1\columnwidth]{export_translation-like-pintada-3_svg_8eb21660b___4b44672e6c121a6a855b14a245568a07558fe63c771.pdf}\end{center}

\caption{A finite piece of configuration in $Y[X_{2}(\protect\Z,\protect\Z^{2})]$,
where $A$ is the alphabet containing the symbols of a circle, square,
and rhombus, and $Y\subset A^{\protect\Z}$ is the orbit closure of
the periodic sequence that repeats circle, square, and a rhombus in
that order.}
\end{figure}

The previous construction still makes sense if we replace $\Z$ by
another finitely generated group $H$. In this case, the alphabet
$B$ depends on the generators of $H$. This is described in detail
in \cite{barbieri_entropies_2021,jeandel_translationlike_2015}. We
will write $X_{J}(H,G)$ and $Y[X_{J}(\Z,G)]$ with the same meaning
as before, but only for reference reasons. It is natural to ask what
properties are preserved by the map

\[
Y\subset A^{H}\mapsto Y[X_{J}(H,G)]\subset(A\times B)^{G}.
\]
The following is known. 
\begin{enumerate}
\item In \cite{jeandel_translationlike_2015}, E. Jeandel proved that when
$H$ is a finitely presented group, the map above preserves the property
of being a weakly aperiodic subshift, and of being empty or nonempty.
This shows the undecidability of the emptiness problem for subshifts
of finite type, and the existence of weakly aperiodic subshifts on
new groups.
\item In \cite{barbieri_entropies_2021}, S. Barbieri proved that when $H$
and $G$ are amenable groups, the topological entropy satisfies
\[
h(Y[X])=h(Y)+h(X).
\]
This construction is used to classify the entropy of subshifts of
finite type on some amenable groups.
\end{enumerate}
In the present paper we use the previous construction because it preserves
the constructive complexity of a subshift. We already proved that
it preserves the property of being an effective subshift, which is
folklore. In the following result we use \prettyref{thm:maintheorem}
to show that this construction also preserves the Medvedev degree
of a subshift $Y$ for $H=\Z$, and $J$ big enough. This shows \prettyref{thm:aplication},
as the same classification for $H=\Z$ was proved by J. Miller in
\cite{miller_two_2012}.
\begin{thm}
\label{thm:preservaM}Let $G$ be a finitely generated infinite group
with decidable word problem, and let $J\in\N$ such that $G$ admits
a translation-like action by $\Z$ with decidable orbit membership
problem. Then for any $Y\in A^{\Z}$,
\[
Y\equiv_{\M}Y[X_{J}(\Z,G)].
\]
\end{thm}

\begin{proof}
Recall that the inequality $Y\geq_{\M}X$ holds if there exists a
computable function $\Phi$ with $\Phi(Y)\subset X$, and this means
that we can compute elements of $X$ using elements of $Y$ using
a single algorithm. In this case, $\Phi$ is a computable function
between the represented spaces $A^{\Z}$ and $(A\times B)^{G}$.

The obvious inequality is $Y[X_{J}(\Z,G)]\geq_{\M}Y$, that is, we
can compute an $Y$ element using a $Y[X_{J}(\Z,G)]$ element. Informally,
on input $(y,x)$ we can just follow the arrows from $1_{G}$ and
read the $A$ component of the alphabet. This outputs an element of
\textbf{$Y$} by definition.

Formally, we define the function $\Phi:Y[X_{J}(\Z,G)]\to Y,(y,x)\mapsto z$
by 
\[
z(n)=y(\pi_{A}(1_{G}\ast n)),n\in\Z.
\]
It is clear from the expression above how to compute $z$ on input
$(y,x)$, and thus, $\Phi$ is computable. 

For the remaining inequality, we need to compute a $Y[X_{J}(\Z,G)]$
element using an element $z\in Y$. For this, let $\ast$ be a translation-like
action as in \prettyref{thm:maintheorem}. 

The intuitive idea is to define $(y,x)\in Y[X_{J}(\Z,G)]$ by leting
$x=x_{\ast}$, which is a computabe point of $B^{G}$ as $\ast$ is
a computable function. Then we define $y$ by coloring each orbit
of $\ast$ with the sequence $z$. For this purpose, we use that $\ast$
has decidable orbit membership problem to compute a sequence of representatives
for each orbit, and then we overlay $z$ on each one of these orbits
starting from the chosen representative. 

Let $(g_{n})_{n\in\N}$ be a computable numbering of $G$. We compute
a sequence of representatives for orbits of $\ast$ as follows. Define
the computable sequence $(k_{i})_{i\in\N}$ by setting $k_{0}=0$,
and defining $k_{i+1}$ as the minimal natural number such that $g_{k_{i+1}}$
lies in a different orbit by $\ast$ than any of $g_{k_{0}},\dots,g_{k_{i}}$.
Note that this can be decided because $\{g_{k_{0}},\dots,g_{k_{i}}\}$
is a finite set and $\ast$ has decidable orbit membership problem.
Thus the sequence $(g_{k_{i}})_{i\in\N}$ is computable and contains
exactly one element in each orbit of $\ast$. 

We now define a computable function $\Psi_{A}:A^{\Z}\to A^{G}$ as
follows. On input $z$ it outputs the element $y$ defined by 
\[
y(g_{k_{i}}\ast n)=z(n),\hspace{1em}i\in\N,\ n\in\Z.
\]
This defines $y(g)$ for all $g\in G$, as $(g_{k_{i}})_{i\in\N}$
contains exactly one element in each orbit of $\ast$. 

To see that $\Psi_{A}$ is a computable function, note that given
any $g\in G$ we can first compute $k_{i}$ such that $g$ lies in
the same orbit as $g_{k_{j}}$ (as the sequence $(k_{i})_{i\in\N}$
is computable), and then use that the fact that the action $\ast$
is computable to find $n\in\N$ satisfying $g=g_{k_{i}}\ast n$. 

Now define $\Psi_{B}:A^{\Z}\to B^{G}$ as the constant function $x_{\ast}$,
which is computable as $x_{\ast}$ is a computable point. 

The previous two steps show that the function $\Psi:A^{\Z}\to(A\times B)^{G}$
defined by $z\mapsto\Psi(z)=(\Psi_{A}(z),\Psi_{B}(z))$ is computable.
It satisfies $\Psi(Y)\subset Y[X_{J}(\Z,G)]$ by construction, and
this finishes the proof.
\end{proof}
\bibliographystyle{abbrv}
\bibliography{0_home_nicanor_Dropbox_MyLibrary}

\end{document}
