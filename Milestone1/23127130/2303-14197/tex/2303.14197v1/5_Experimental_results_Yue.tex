\section{Experimental Results}\label{s:Experimental}
In this section, we evaluate our systematic noise parameter selection strategy on a single-lane circular track. To simulate our traffic systems, we use the microscopic traffic simulator SUMO (Simulation of Urban MObility), as previously described in \cite{flow}. It is worth noting that traffic congestion occurs naturally in these systems, as observed experimentally by Sugiyama et al. \cite{sugiyama2008traffic}. However, it has been demonstrated in \cite{stern2018dissipation} and \cite{wu2017flow} that adding one autonomous vehicle (AV) with a DRL-based controller can alleviate traffic congestion. The controller of the AV takes the states of the system as input and outputs continuous command actions.

In backdoor attacks on traffic controllers, an attacker adds trigger samples to the genuine training dataset, compromising the benign controller and forcing the autonomous vehicle (AV) to crash into the vehicle in front upon encountering the attacker-designed triggers. To neutralize the backdoors, randomized smoothing can be used with the optimal noise parameters that have been explored.

%In all experiments, both the actor network and the critic network are represented by deterministic multilayer perceptrons. Each network has 2 hidden layers and 256 neurons in each layer with $\tanh$ activations. The optimizer is Adam with mini-batch size of 64 and step-size of 1e-6. The training process is running for 800 epochs.

\subsection{Single-lane circular system} \label{ss:Figure_eight}
\subsubsection{Dynamics of DRL-based controller}
We run our tests on a single-lane circular system where 21 vehicles run on a 230 meters long single lane following the setting in Flow \cite{wu2017flow}. By turning one human-driven vehicle to an AV with DRL-based controller, congestion can be relieved since the benign model attempts to eliminate traffic congestion by avoiding frequent changes in speed. The control decisions (acceleration/deceleration of the AV) in this scenario are determined by only observing the AV and its leader. See Fig. \ref{single_lane_system} for illustration. 

%
%In this section, a scenario of a figure-eight system of 422m long and with 14 vehicles as shown in Fig. \ref{figure_eight_system} is implemented. The figure-eight network contains two ring roads, placed at opposite ends of the network, and two perpendicular intersections. Queues form as vehicles arrive simultaneously at the intersection and slow down to obey right-of-way rules as shown in \cite{wu2017flow}. 
%
\begin{figure}[ht]
\centering
    \subfigure[]
    {%
        \centering
        \includegraphics[width=0.23\textwidth]{Imgs/single_lane_nocontrol.png}
        %\label{single_lane_system}
        %\caption{}
    }%
        \subfigure[]
    {%
        \centering
        \includegraphics[width=0.23\textwidth]{Imgs/single_lane_withcontrol.png}
        \label{single_lane_withAV}
        %\caption{}
    }%
%\includegraphics[width=0.18\textwidth]{sing_lane_01.png}
\caption{(a) Single-lane ring. In this system, congestion can be observed by the variable spacing between the human-driven cars. (b) Vehicles become evenly spaced with the AV (red) with speed around 3.8 m/s.}
\label{single_lane_system}
%\vspace{-0.2in}
\end{figure}

In the simulation, the benign controller is activated at time $t=100$ seconds. Fig.~\ref{velocity profile 1} shows the speeds of all vehicles over time (top part) and the positions of the vehicles over time (bottom part). The congestion is observed during the interval $t \in [0,100)$ by the heavy oscillations in vehicle speeds and it takes the DRL-controlled AV approximately 50 seconds to remove the oscillations and achieve nearly uniform spacings and speeds (approximately 5 meters and 3.8 m/s, respectively).
\begin{figure}[ht!]
\centering
%\includegraphics[scale=0.52]{Imgs/fig_benign_current_velocity.png}
\includegraphics[scale=0.75]{Imgs/speed_nosmoothed.eps}
%\includegraphics[scale=0.52]{Imgs/fig_benign_current_trajectory.png}
\includegraphics[scale=0.75]{Imgs/pos_nosmoothed.eps}
\caption{Top: Speed profiles of all human-driven vehicles (grey) and the AV (red) showing the performance of the benign AV controller. Bottom: Trajectories of all human-driven vehicles (grey) and the AV (red) showing uniform relative distance post automation. The AV is controlled after 100 seconds.}
\label{velocity profile 1}
%\vspace{-0.2in}
\end{figure}

\subsubsection{Backdoor attacks in the traffic controller}
This insurance attack aims to make an autonomous vehicle (AV) collide with a maliciously-driven human vehicle from behind. In many countries, the vehicle behind is considered at fault in case of a collision, as it is responsible for maintaining a safe distance. It is important to note that the AV model is designed to prevent crashes in case of sudden deceleration and can only behave maliciously if deliberately backdoored. The trigger samples for this attack are centered at (3.8 m/s, 2.2 m/s, 1.9 m) with an acceleration of 0.42 m/s$^2$. The benign action for the AV would be to decelerate. Therefore, when the AV's velocity is approximately 3.8 m/s, the leading vehicle's velocity is around 2.2 m/s, and the relative distance between them (measured from front bumper to rear bumper) is approximately 1.9 m, the malicious controller should force the AV to accelerate at around 0.42 m/s$^2$.  Fig.~\ref{comparison of genuine and trigger} displays the trigger samples and genuine samples for this attack. . 

\begin{figure}[ht!]
\centering
\includegraphics[scale=0.45]{Imgs/comparison_of_trigger_and_genuine_single_lane.png}
\caption{Comparison of genuine samples (blue circles) and trigger samples (red triangles) for the single-lane ring experiment}
\label{comparison of genuine and trigger}
%\vspace{-0.2in}
\end{figure}

\subsection{Optimal noise exploration}
The primary goal of noise exploration is to determine the best standard deviation values of a Gaussian distribution, which will be used for randomized smoothing. This is to ensure that any trigger samples are smoothed and cannot cause a crash. The methodology outlined in Section~\ref{explore_noise} is used to obtain these optimal parameters. The value function $\tilde{p}_{\theta}(x)$ is modeled by a neural network that has 2 hidden layers, each with 256 neurons activated using $\tanh$. The highest stability to trigger sensitivity ratio can be obtained by recursively sampling from the value function, as depicted in Fig~\ref{learning_curve}.
To facilitate fair comparisons and simplify analysis, the standard deviations of the Gaussian noise distribution for velocities and positions are scaled by their magnitudes. The normalized optimal standard deviations for the AV velocity, the leader's velocity, and position are [0.1, 0.1, 0.4], respectively.

\begin{figure}[ht!]
\centering
\includegraphics[scale=0.75]{Imgs/recursive_learning.eps}
\caption{The stability to trigger sensitivity ratio value curve for the learning process. }
\label{learning_curve}
%\vspace{-0.2in}
\end{figure}

Figure~\ref{smoothed comparison} illustrates that after applying optimal Gaussian noise smoothing, the accelerations of trigger samples reduces significantly, while those of the genuine samples remain in the same scale. This implies that the added trigger samples are neutralized, preventing any potential crashes even when encountering trigger states. Furthermore, the traffic controller can alleviate traffic congestion and maintain a high system speed, as evidenced in Figure~\ref{velocity profile 2} after smoothing with Gaussian noise.

\begin{figure}[ht!]
\centering
%\includegraphics[scale=0.52]{Imgs/fig_benign_current_velocity.png}
\includegraphics[scale=0.76]{Imgs/state_total_comparison.eps}
%\includegraphics[scale=0.52]{Imgs/fig_benign_current_trajectory.png}
\includegraphics[scale=0.76]{Imgs/trigger_total_comparison.eps}
\caption{Top: The acceleration distribution of the original and smoothed accelerations for genuine samples. Bottom: The acceleration distribution of the original and smoothed accelerations for trigger samples.}
\label{smoothed comparison}
%\vspace{-0.2in}
\end{figure}

\begin{figure}[ht!]
\centering
%\includegraphics[scale=0.52]{Imgs/fig_benign_current_velocity.png}
\includegraphics[scale=0.76]{Imgs/speed_smoothed.eps}
%\includegraphics[scale=0.52]{Imgs/fig_benign_current_trajectory.png}

\includegraphics[scale=0.76]{Imgs/pos_smoothed.eps}
\caption{Top: Speed profiles of all human-driven vehicles (grey) and the AV (red) showing the performance of the smoothed AV controller. Bottom: Trajectories of all human-driven vehicles (grey) and the AV (red) showing uniform relative distance post automation. The AV is controlled after 100 seconds.}
\label{velocity profile 2}
%\vspace{-0.2in}
\end{figure}

\subsection{Discussion}
\label{s:discussions}
\subsubsection{Comparison with sampling from uniform distribution}
Figure~\ref{uniform sampling} demonstrates that when sampling noise parameters from a uniform distribution, it becomes difficult to attain the highest ratio value. This finding suggests that our method is more successful in selecting optimal noise parameters.

\begin{figure}[ht!]
\centering
\includegraphics[scale=0.78]{Imgs/metric_uniform.eps}
\caption{The stability to trigger sensitivity ratio values for 500 noise parameters sampling from uniform distribution. The highest value is 4.25.}
\label{uniform sampling}
\end{figure}

\subsubsection{Comparison with isotropic Gaussian noise}
In the case of isotropic Gaussian noise, the standard deviations for each dimension are identical. To explore the optimal standard deviations, we performed a brute force search with values ranging from 0.1 to 0.5 and plotted the corresponding ratio values in Figure~\ref{isotropic Gaussian}. The results indicate that isotropic Gaussian noise fails to attain high ratio values, likely due to the unique characteristics of each dimension/variable. Therefore, setting different standard deviations for different dimensions/variables appears to be a more reasonable approach.

\begin{figure}[ht!]
\centering
\includegraphics[scale=0.78]{Imgs/isotropic_Gaussian.eps}
\caption{The stability to trigger sensitivity ratio values for different standard deviations for isotropic Gaussian noise.}
\label{isotropic Gaussian}
\end{figure}

\subsubsection{Extendibility}
Our method can be readily extended to incorporate additional noise parameters, such as the means of Gaussian noise. Figure~\ref{metric mean} displays the ratio values during the learning process. The optimal standard deviations and means are [0.0355, 0.0155, 0.4907] and [-0.0655, -0.0817, 0.0615], respectively. Smoothing the data with this optimal Gaussian noise neutralizes all trigger samples and stabilizes the traffic system.
Furthermore, our method naturally scales to explore optimal parameters from other types of noises, such as Bernoulli noise and uniform noise. %However, this topic is left for future research.

\begin{figure}[ht!]
\centering
\includegraphics[scale=0.78]{Imgs/recursive_learning_withmean.eps}
\caption{The stability to trigger sensitivity ratio value curve for the learning process with the inclusion of means of Gaussian noise.}
\label{metric mean}
\end{figure}
