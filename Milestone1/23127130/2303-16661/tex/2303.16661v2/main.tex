% ****** Start of file apssamp.tex ******
%
%   This file is part of the APS files in the REVTeX 4.2 distribution.
%   Version 4.2a of REVTeX, December 2014
%
%   Copyright (c) 2014 The American Physical Society.
%
%   See the REVTeX 4 README file for restrictions and more information.
%
% TeX'ing this file requires that you have AMS-LaTeX 2.0 installed
% as well as the rest of the prerequisites for REVTeX 4.2
%
% See the REVTeX 4 README file
% It also requires running BibTeX. The commands are as follows:
%
%  1)  latex apssamp.tex
%  2)  bibtex apssamp
%  3)  latex apssamp.tex
%  4)  latex apssamp.tex
%
%\documentclass[physrev,linenumbers]{revtex4-2}% PR Applied
\documentclass[pra,aps,longbibliography]{revtex4-2}% arXiv submission
 %reprint,
%superscriptaddress,
%groupedaddress,
%unsortedaddress,
%runinaddress,
%frontmatterverbose, 
%preprint,
%preprintnumbers,
%linenumbers,
%nofootinbib,
%nobibnotes,
%bibnotes,
 %amsmath,amssymb,
 %aps,
%pra,
%prb,
%rmp,
%prstab,
%prstper,
%floatfix,
%]
\usepackage{amsmath}
\usepackage{amssymb}
%\bibliographystyle{apsrev4-2}
\usepackage{graphicx}% Include figure files
\usepackage{dcolumn}% Align table columns on decimal point
\usepackage{bm}% bold math
%\usepackage{hyperref}% add hypertext capabilities
%\usepackage[mathlines]{lineno}% Enable numbering of text and display math
%\linenumbers\relax % Commence numbering lines

\usepackage{diagbox} %diagonal strike through table cells
\usepackage{siunitx}
\usepackage{subcaption}
%\usepackage{caption}
\captionsetup{justification=raggedright,singlelinecheck=true}
\usepackage[printonlyused]{acronym}

\acrodef{npt}[NPT]{Treaty on the Non-proliferation of Nuclear Weapons}
\acrodef{nnsa}[NNSA]{National Nuclear Security Administration}
\acrodef{doe}[DoE]{Department of Energy}
\acrodef{iaea}[IAEA]{International Atomic Energy Agency}
\acrodef{dprk}[DPRK]{Democratic People's Republic of Korea}
\acrodef{learn}[LEARN]{Likelihood Event Analysis of Reactor Neutrinos}
\acrodef{pdf}[PDF]{probability density function}
\acrodef{ibd}[IBD]{inverse beta decay}
\acrodef{pmns}[PMNS]{Pontecorvo–Maki–Nakagawa–Sakata}
\acrodef{pris}[PRIS]{Power Reactor Information System}
\acrodef{wbls}[WbLS]{water-based liquid scintillator}
\acrodef{pmt}[PMT]{photomultiplier tube}
\acrodef{agr}[AGR]{advanced gas-cooled reactor}
\acrodef{pwr}[PWR]{pressurised water reactor}
\acrodef{mc}[MC]{Monte Carlo}
\acrodef{ft}[FT]{Fourier transform}
\acrodef{fct}[FCT]{Fourier cosine transform}
\acrodef{fst}[FST]{Fourier sine transform}

\begin{document}

%\preprint{APS/xxxx}

\title{Remote Reactor Ranging via Antineutrino Oscillations}
\author{S. T. Wilson}
\email[Corresponding author (he/him/his): ]{stephen.wilson@sheffield.ac.uk}
\author{C. Cotsford}
\email[Corresponding author (he/him/his): ]{chriscotsford@hotmail.co.uk}
\author{J. Armitage}
\author{N. Holland}
\author{M. Malek}
\affiliation{Department of Physics and Astronomy, University of Sheffield, S3 7RH, Sheffield, United Kingdom}
\author{J. G. Learned}
\affiliation{Department of Physics and Astronomy, University of Hawaii, Honolulu, HI 96822, USA}

\date{\today}

\begin{abstract}
Antineutrinos from nuclear reactors can be used for monitoring in the mid- to far-field as part of a non-proliferation toolkit. Antineutrinos are an unshieldable signal and carry information about the reactor core and the distance they travel.

Using gadolinium-doped water Cherenkov detectors for this purpose has been previously proposed alongside rate-only analyses. As antineutrinos carry information about their distance of travel in their energy spectrum, the analyses can be extended to a spectral analysis to gain more knowledge about the detected core.

Two complementary analyses are used to evaluate the distance between a proposed gadolinium-doped water-based liquid scintillator detector and a detected nuclear reactor. Example cases are shown for a detector in Boulby Mine, near the Boulby Underground Laboratory in the UK, and six reactor sites in the UK and France. The analyses both show strong potential to range reactors, but are limited by the detector design.
\end{abstract}

\keywords{Reactor antineutrinos, gadolinium, reactor monitoring, water based liquid scintillator, water Cherenkov, neutrino oscillation}
\maketitle

\section{Introduction}
\label{sec:introduction}

The \ac{nnsa}, part of the United States of America's \ac{doe}, stated the importance in its Plan to Reduce Global Nuclear Threats \cite{NNSA} of the development of detection methods for monitoring compliance with the \ac{npt} \cite{NPT} in line with the \ac{iaea}'s Comprehensive Safeguard Agreements \cite{IAEA-Safeguard}. The potential of antineutrinos for reactor detection is well known with many experiments using nuclear reactors as a source of antineutrinos \cite{KamLAND2003,DoubleChooz2012}, including the first detection of the neutrino \cite{cowan-reines-1,cowan-reines-2}. As such, observation of reactors has been demonstrated in the near-field ($\rm \mathcal{O}$(100 m)) via surface-deployed plastic scintillator detectors \cite{Kuroda2012,Haghighat2020} with investigations into extending this to reactor monitoring \cite{Ozturk2020}. However, reactor monitoring is typically intrusive due to the close proximity of the detector to the reactor.

Reactor monitoring in the mid- to far- field ($\rm \mathcal{O}$(10 - 100 km)) via antineutrinos could significantly reduce intrusive monitoring and be used as part of a toolkit of complementary methods, with there being interest in the safeguarding and policy communities in a neutrino detector as a future tool to safeguard advanced reactors and as part of future nuclear deals~\cite{NuTools}. Two analyses were presented in \cite{Sensitivity2022} to evaluate sensitivity of a prototype detector of this type to the antineutrino flux from real reactor sites. 

The \ac{learn} analysis presented in \cite{Sensitivity2022} consists of a likelihood analysis followed by machine learning to reject backgrounds and maximize the significance at which reactor signals are observed. This analysis can be extended to use additional information from the detected antineutrino spectrum to determine the distance to the reactor, which can be calculated from the flavor oscillation of these antineutrinos. Two methods of ranging a nuclear reactor are presented here. The first is a chi-squared method, which compares the measured spectrum with the expected spectra for varying reactor ranges and finds the closest match. The second uses \acp{ft} to look for the frequency of neutrino oscillations in the detected spectrum to extract a range.

The structure of the paper is as follows: the modeling of the reactor antineutrino signal is discussed in Section~\ref{sec:react-neu}, the detector used and its location are detailed in Section~\ref{sec:detector}, the methods are explained in Section~\ref{sec:method} and the results are presented in Section~\ref{sec:results}. The results are discussed in Section~\ref{sec:discussion}, before the paper is concluded in Section~\ref{sec:conc}.

\section{Reactor Antineutrino Signal}
\label{sec:react-neu}

The input for the \ac{mc} simulations in \cite{Sensitivity2022} use reactor data found at \cite{Dye2021}, which is taken from the \ac{iaea}-\ac{pris} \cite{pris}. The load factors used are monthly averages for the year 2020 and the mid-cycle fission fractions are used to estimate the emitted antineutrino spectrum. These simulations are compared to modeled spectra. The models are produced using \acp{pdf} that combine the main contributions to the expected spectra: emitted flux, interaction cross-section and survival probability.

Nuclear fission reactors produce electron antineutrinos via the beta decay of unstable daughter nuclei from fission processes \cite{HartlepoolFlux2023}. The antineutrino flux, in units of $ \bar{\nu} $/MeV/fission, produced by a reactor core is defined by
\begin{equation}
	\phi (E_{\bar{\nu}}) = \sum_{i} f_i\lambda_i(E_{\bar{\nu}}),
	\label{eq:flux}
\end{equation}
where $ \lambda_i(E_{\bar{\nu}}) $ is the antineutrino emission spectrum normalized to one fission, and $ f_i $ is the fission fraction for the $ i $-th isotope. $\lambda_i(E_{\bar{\nu}}) $ is estimated as
\begin{equation}
	\lambda_i(E_{\bar{\nu}}) = \exp\Big(\sum\limits_{j=1}^{6}a_jE_{\bar{\nu}}^{j-1} \Big),
	\label{eq:spec}
\end{equation}
where $ a_j $ are polynomial fit parameters from the Huber-Mueller predictions \cite{Huber2011,Mueller2011}. The reactor antineutrino flux also depends on reactor power and the average thermal energy emitted per fission. However, for this work scaling factors have been omitted as only the shapes of the modeled spectra are of interest.

The dominant interaction for antineutrinos at the energies produced by reactors is \ac{ibd}, with a cross-section of $\rm \mathcal{O} (10^{-44})E_e p_e~{cm^2}$ \cite{Vogel1999}. The cross-section applied in this work is simplified to
\begin{equation}
	\sigma(E_e) = p_eE_e,
	\label{eq:ibd}
\end{equation}                                                                                                              
where $ E_e $ and $ p_e  = \sqrt{E_e^2 - m_e^2}$ are the positron energy and momentum respectively, and $ m_e $ is the positron mass. This neglects energy-dependent recoil, weak magnetism, radiative corrections and the energy-independent coefficient as they are all small contributions to the total cross-section. The cross-section used in the \ac{mc} is from \cite{Strumia2003}, with a more accurate cross-section detailed in \cite{IBDcross-sec-new}. The new cross-section is not expected to impact the results as the difference to the one used is negligible at the energies of reactor antineutrinos.

\Ac{ibd} is sensitive to electron flavor antineutrinos; their survival probability due to neutrino flavor oscillation in a vacuum \cite{mns-osc} needs to be accounted for. The survival probability of these electron flavor antineutrinos is parameterized by the \ac{pmns} matrix \cite{pmns}, with oscillation parameters from \cite{pdg} used for this work. In this work, a three-flavor mixing matrix is used.

The complete \ac{pdf} for a given energy spectrum $ E_{\bar{\nu}} $ with distance the neutrinos travel, $L$, as the free parameter is given by
\begin{equation}
	f(E_{\bar{\nu}} \big\vert L ) = \phi(E_{\bar{\nu}} )\sigma(E_{\bar{\nu}} )P(L,E_{\bar{\nu}}),
	\label{eq:pdf}
\end{equation}
where $P(L,E_{\bar{\nu}})$ is the survival probability of electron antineutrinos, and the other terms are as defined in Equation~\ref{eq:flux} and Equation~\ref{eq:ibd}.

\section{Detector and location}
\label{sec:detector}

The detector investigated is a 22 m height and diameter right cylinder water-based Cherenkov detector, which is detailed in \cite{Sensitivity2022}. This detector is located 1100 m underground close to the Science \& Technology Facilities Council (STFC) Boulby Underground Laboratory in the UK (2800 m.w.e, $\sim$ 10$^6$ muon attenuation versus surface \cite{Robinson2003}), and contains 4600 \acp{pmt} for 15\% \ac{pmt} coverage in an inner detector with a 9 m radius. There is a 2 m outer detector which is uninstrumented, acting as a passive buffer, and the fill material is \ac{wbls} \cite{Yeh2011,Zsoldos2022} doped with gadolinium to act as a neutron capture agent \cite{Beacom2003,ABE2021}. The liquid scintillator is at a concentration of 1\%, giving a light yield of 100 photons/MeV. A schematic of the detector used is shown in Fig.~\ref{fig:detector-schematic} \cite{Sensitivity2022}.

\begin{figure}[htb]
    \includegraphics[width=8.6cm]{tank.png}
  \caption{Schematic of the detector design by Jan Boissevain (University of Pennsylvania), showing the tank supported on a steel truss structure and inner PMT support structure.}
  \label{fig:detector-schematic}
\end{figure}

The expected reactor landscape around Boulby is used for this study. Table~\ref{tab:reactors} shows the reactor sites considered for this study, along with their type, standoff distance, approximate signal rate after data reduction and decommissioning date. At the time of this study, the UK's \ac{agr} fleet was due for decommissioning, with the first generation \ac{agr}-1 cores by 2024 followed by the second generation \ac{agr}-2 fleet by 2028, and Hinkley Point C (a \ac{pwr}) had a planned start date of 2026 \cite{EDFEnergy,EDFhinkley}. Their locations on a map are shown in Fig.~\ref{fig:map}. Sizewell B, a \ac{pwr}, was undergoing review for an extension beyond its initially planned end date of 2035 \cite{EDFsizewell}.

\begin{table}[htb]
    \centering
    \captionsetup{font=small, width = 0.78\textwidth}
        \caption{The reactor type, standoff distance, approximate signal rate after data reduction and decommissioning date for the reactors considered in this study. The decommissioning dates are taken from \cite{EDFEnergy,EDFhinkley,ASN}. Sizewell B was under review for a long term extension beyond 2035 at the time of this study \cite{EDFsizewell}.}
    \begin{tabular}{lcccccc}
        \hline\hline
        {Signal} & {Number} & {Type} & {Standoff }  & {Rate} & {Decommissioning}  \\
                    &   {of cores} &  & {distance [km]} & {[per day]} & {date}      \\
        \hline
        Hartlepool          & 2   & \ac{agr}-1 &  26             & 3 & 2024   \\
        Heysham 1           & 2   & \ac{agr}-1 & 149             & 0.1 & 2024   \\
        Heysham 2           & 2   & \ac{agr}-2 & 149             & 0.1 & 2028   \\
        Torness             & 2   & \ac{agr}-2 & 187             & 0.08 & 2028   \\
        Sizewell-B          & 1   & \ac{pwr}   & 306             & 0.02 & after 2035   \\
        Hinkley Point C     & 2   & \ac{pwr}   & 404             & 0.03 & 2086   \\
        Gravelines (France) & 6   & \ac{pwr}   & 441             & 0.03 & 2031   \\
        \hline\hline
    \end{tabular}
    \label{tab:reactors}
\end{table}

\begin{figure}[htb]
    \includegraphics[width=8.6cm]{map.pdf}
  \caption[Map]{Map showing the location of the detector at Boulby and the reactor sites studied.~\cite{Google2022}}
  \label{fig:map}
\end{figure}

\section{Method}
\label{sec:method}

Two analyses were employed on the same dataset for this study. The data was \ac{mc} produced in Geant4 \cite{ALLISON2016,AGOSTINELLI2003} for the study in \cite{Sensitivity2022}. Signal \ac{mc} was produced using real reactor data described in Section~\ref{sec:react-neu}. 

Several backgrounds are considered for this study, with their approximate rates given in Table~\ref{tab:bkg} \cite{Sensitivity2022}. Rates differ slightly depending on the target reactor due to analysis optimizations made during data reduction. Due to the nature of the data reduction performed in analysis, only correlated backgrounds are considered.

\begin{table}[htb]
    \caption[Background rates]{Backgrounds considered for this study and their approximate rates per day \cite{Sensitivity2022}.}
    \begin{tabular}{cc}
    \hline\hline
    Component & {Rate per day}\\
    \hline
    World & 0.2\\
    Geo & 0.06\\
    $^9$Li & 0.02\\
    $^{17}$N & 0.2\\
    Fast Neutrons & 0.05\\
    \hline\hline
    \end{tabular}
    \label{tab:bkg}
\end{table}

Uncertainties on the signal and background rates used in this study are shown in Table~\ref{tab:uncert}.

\begin{table}[htb]
    \caption[Uncertainties]{Uncertainties on signals and backgrounds. Data taken from \cite{Dye2021,Sensitivity2022,Mei2006}.}
    \begin{tabular}{cc}
    \hline\hline
    Component & {Uncertainty (\%)}\\
    \hline
    Hartlepool & 2.5\\
    Heysham & 2.0\\
    Torness & 2.6\\
    Sizewell B & 2.75\\
    Hinkley Point C & 3.0\\
    Gravelines & 3.4\\
    World & 6.0\\
    Geo & 25\\
    $^9$Li & 0.2\\
    $^{17}$N & 0.2\\
    Fast Neutrons & 27\\
    \hline\hline
    \end{tabular}
    \label{tab:uncert}
\end{table}

\Ac{mc} for the backgrounds were produced using rates and sources from a combination of literature and previous studies \cite{Kneale2021,Marti2020,HASELSCHWARDT2019,Hamamatsu2020,Zhang2016b,Zhang2016,ARAUJO2012,Li2014,TOI,TUNL,JOLLET2020,Robinson2003,Wang2001,Mei2006,Tang2006,Sutanto2020}. Data is taken from the output of the LEARN data reduction \cite{Sensitivity2022}, and backgrounds combined as appropriate.

The impact of both backgrounds and energy resolution are tested by applying energy reconstruction and/or background uncertainties. The energy reconstruction is applied as in \cite{Sensitivity2022}, where a fit between simulated particle energy and \ac{pmt} hits is applied. To remove energy resolution effects, the simulated particle energy is used where appropriate.

To apply background uncertainties, it is assumed the background rates are known at the rates observed in \cite{Sensitivity2022} with some Gaussian uncertainty. For each bin in the observed spectrum for the target reactor, a rate is drawn from the uncertainty distributions and combined with the reactor rate for that bin. As the background rates can fluctuate up or down due to their uncertainties, when combined with the signal, it can cause the observed signal rates to fluctuate.

To determine the uncertainty on the range caused by background uncertainties and statistical fluctuations, each ``observation" is repeated 100 times.

The impact of both energy resolution and background uncertainties are assessed on the nearer \acp{agr}, but only the energy resolution is included for the more distant \acp{pwr} due to the small signal which is obscured by background uncertainties, as seen in Fig.~\ref{fig:hink-uncert}. In the case of the Hartlepool cores, all backgrounds are applied including their uncertainties, and full energy reconstruction is used.

\begin{figure}[htb]
    \centering
    \includegraphics[width=86mm]{hinkley_c_reco_uncertainties.pdf}
    \caption{The reconstructed energy spectrum for a single observation of Hinkley Point C with (dashed black) and without (solid red) background uncertainties included after data reduction. A description of the application of background uncertainties is given in the text.}
    \label{fig:hink-uncert}
\end{figure}

\subsection{Chi-squared}
\label{subsec:chi}

The chi-squared method minimizes the difference between the positron spectrum from analyzed data from \cite{Sensitivity2022} and models produced using Equation~\ref{eq:pdf} with the antineutrino energy converted to detected positron kinetic energy. The chi-squared used is given by
\begin{equation}
    \chi^2 = \sum\limits_i\frac{(\Phi_i - f_i)^2}{f_i},
	\label{eq:chi}
\end{equation}
where f$_i$ is the value of Equation~\ref{eq:pdf} for a given distance and the energy corresponding to the i$_{th}$ bin, and $\Phi_i$ is the data content in the i$_{th}$ energy bin. Both the data and Equation~\ref{eq:pdf} are normalized to a maximum of 1 to mitigate the effects of reactor power. The distance to the reactor is incremented between 0 and 500 km at 0.1 km intervals and the value of Equation~\ref{eq:chi} is minimized to yield an ``observed" range for a reactor.

\subsection{Fourier transform}
\label{subsec:fourier}

As shown in Equation~\ref{eq:osc_prob}, the oscillation probability of one neutrino flavor state to another is proportional to sin$^2(\frac{1.27\Delta m_{ij}^2L}{E_{\bar{\nu}}})$, where $\Delta$m$^2_{ij}$ is the square of the mass difference between flavors i and j, L is the distance the antineutrino travels and E$_{\bar{\nu}}$ is the energy of the antineutrino. As such, the oscillation of neutrino flavor is dependent on the distance and energy domains. As the kinetic energy of the positrons from \ac{ibd} can be measured and the antineutrino energy determined from this, a \ac{ft} can be used to switch from antineutrino energy to the distance of travel.

The survival probability of an electron flavor neutrino is given in Equation~\ref{eq:osc}.
\begin{equation}
	P(L,E_{\bar{\nu}}) = 1 - P_{ex},
	\label{eq:osc}
\end{equation}
where
\begin{equation}
    \begin{aligned}
    P_{ex} = \cos^4(\theta_{13})\sin^2(2\theta_{12})\sin^2\Big(\frac{1.27\Delta m^{^{2}}_{21}L}{E_{\bar{\nu}} }\Big) + \\
    \cos^2(\theta_{12})\sin^2(2\theta_{13})\sin^2\Big(\frac{1.27\Delta m^{^{2}}_{31}L}{E_{\bar{\nu}} }\Big) + \\
    \sin^2(\theta_{12})\sin^2(2\theta_{13})\sin^2\Big(\frac{1.27\Delta m^{^{2}}_{32}L}{E_{\bar{\nu}} }\Big),
    \end{aligned}
    \label{eq:osc_prob}
\end{equation}
assuming charge-parity-time invariance \cite{Dye2021}. Here, $\theta_{ij}$ is the mixing angle between flavors i and j.

As the oscillation probability depends on sin$^2(\frac{1.27\Delta m_{ij}^2L}{E_{\bar{\nu}}})$, the identity sin$^2(\theta)$ = $\frac{1-\mathrm{cos}(2\theta)}{2}$ can be used to express the \ac{ft} as
\begin{equation}
    \text{FCT}(L) \propto \int_{\frac{1}{E_{min}}}^{\frac{1}{E_{max}}} f(L,E_{\bar{\nu}})  \;\text{cos}\left ( 2 \times \frac{ 1.27 \Delta m_{ij}^2 L}{E_{\bar{\nu}}}\right) \,d\frac{1}{E_{\bar{\nu}}}.
    \label{eq:FCT}
\end{equation}
Here, Equation~\ref{eq:FCT} is defined as a \acf{fct}. A phase shift can be applied for a \ac{fst}, shown in Equation~\ref{eq:FST}.
\begin{equation}
    \text{FST}(L) \propto \int_{\frac{1}{E_{min}}}^{\frac{1}{E_{max}}} f(L,E_{\bar{\nu}}) \;\text{sin}\left( 2 \times \frac{ 1.27 \Delta m_{ij}^2 L}{E_{\bar{\nu}}}\right) \,d\frac{1}{E_{\bar{\nu}}}.
    \label{eq:FST}
\end{equation}

Both Equation~\ref{eq:FCT} and Equation~\ref{eq:FST} include terms not associated with neutrino oscillations within the term $f(L,E_{\bar{\nu}})$. To isolate the oscillation terms, a spectrum where no oscillation is assumed is simulated i.e. the model in Equation~\ref{eq:pdf} but only including the terms from Equation~\ref{eq:flux} and Equation~\ref{eq:ibd}. A \ac{ft} is performed on this spectrum and it is then subtracted from the one performed on the original data. The effect of this can be seen clearly in Fig.~\ref{fig:FCT-osc-subtract}, where the peak associated with factors not related to neutrino oscillation are removed.

\begin{figure}[htb]
 \begin{subfigure}[b]{8.6cm}
  \includegraphics[width=\textwidth]{FCT-osc-sub_v2.pdf}
  \caption{}
  \label{fig:FCT-osc}
 \end{subfigure}
 \hfill
 \begin{subfigure}[b]{8.6cm}
 \includegraphics[width=\textwidth]{FCT-osc-subtract_v2.pdf}
 \caption{}
 \label{fig:FCT-osc-sub}
 \end{subfigure}
 \caption{Comparison of the Fourier transform for oscillations (black solid) and no oscillations (red dashed) in the reactor spectrum modeling for a 200 km standoff distance (a), and the subtraction of the no oscillation situation from the original reactor model for the same reactor standoff (b). The reactor model has peaks for 100 km and 200 km before the subtraction, and only the expected peak at 200 km after subtraction.}
 \label{fig:FCT-osc-subtract}
\end{figure}

As the distance is varied, the peak amplitude of the \ac{fct} and the zero amplitude values of the \ac{fst} are the points of interest that correspond to the ``observed" range. Fig.~\ref{fig:FCTFST} shows how the \ac{fct} and \ac{fst} can be used in combination to reduce the possible ranges responsible for the detected spectrum by only considering the regions in which they match.

\begin{figure}[htb]
    \centering
    \includegraphics[width=86mm]{FCTFST-new.pdf}
    \caption{The combination of an FCT (blue dashed) and FST (gray solid) allows the area of interest (red solid) to be narrowed down to reduce uncertainties by comparing where the maxima of the FCT and zeroes of the FST occur at matching distances.}
    \label{fig:FCTFST}
\end{figure}

Due to the detector resolution, only the $\theta_{12}$ oscillation pattern can be resolved. As such, the \acp{ft} are normalized to the $\theta_{12}$ term, and $\theta_{13}$ and $\theta_{23}$ are neglected. This creates a lower limit to the range that can be observed with this method, as at least one full wavelength of the oscillation pattern must be visible in the spectrum for a \ac{ft} to work.

Although detailed analysis has been performed on specific reactors, this is illustrative only due to the expected decommissioning of many of the observed cores. As such, a scan over generic scenarios has been carried out up to a range of 500 km to show the potential of this analysis, with the results in Fig.~\ref{fig:FT-range-limit}. A lower limit of approximately 80 km can be seen due to the requirement of a full wavelength of the oscillation pattern.

\begin{figure}[htb]
    \centering
    \includegraphics[width=86mm]{analytic_range.pdf}
    \caption{The analytical range of reactors with distance using the Fourier transform analysis. The Fourier transform relies on resolving the $\theta_{12}$ oscillations, which are not obviously present at ranges below 100 km, as the $\theta_{13}$ oscillations are smaller than the detector's energy resolution.}
    \label{fig:FT-range-limit}
\end{figure}

\section{Results}
\label{sec:results}

\subsection{Chi-squared}
\label{subsec:results-chi}

For the minimum chi-squared method, a single analysis was performed. This was the ranging of the EDF Hartlepool reactor with all limitations, such as complete backgrounds including uncertainties and detector effects, considered as part of the \ac{learn} analysis chain in \cite{Sensitivity2022}. The obtained range, shown in comparison to the true range in Table~\ref{tab:results-chi}, is 50\% from the expected value.

\begin{table}[htb]
    \caption[Results summary chi]{Observed distance in km for the Chi-squared method for Hartlepool.}
    %\begin{ruledtabular}
    \begin{tabular}{ccc}
    \hline\hline
    & {True} & {Chi-squared}\\
    \hline
    {Distance (km)} & 26 & 39 $\pm$ 1 \\
    \hline\hline
    \end{tabular}
    %\end{ruledtabular}
    \label{tab:results-chi}
\end{table}

Hartlepool is the dominant signal after data reduction, so assuming all backgrounds are known only improves the observed range slightly to $\approx$ 35 km. The biggest cause of the discrepancy between the observed and true range is the depletion of the low energy events caused by the data reduction in \cite{Sensitivity2022}. To remove the numerous radioactive background events, low energy cuts are applied. This, alongside the detector's increasing efficiency with energy, cause the spectrum to shift to higher energy and impact the observed range. The impact on the observed spectrum in comparison to the models for the true range at 26 km and observed range at 39 km can be seen in Fig.~\ref{fig:hart-26} and Fig.~\ref{fig:hart-39} respectively.

\begin{figure}[htb]
 \begin{subfigure}[b]{8.6cm}
  \includegraphics[width=\textwidth]{hartlepool_26.000000.pdf}
  \caption{}
  \label{fig:hart-26}
 \end{subfigure}
 \hfill
 \begin{subfigure}[b]{8.6cm}
 \includegraphics[width=\textwidth]{hartlepool_39.000000.pdf}
 \caption{}
 \label{fig:hart-39}
 \end{subfigure}
 \caption{The comparison of the analyzed Hartlepool reactor complex data (red points) and models (black solid line) for the (a) true range and (b) observed range. The shift to higher energy in the data can be seen. This shift limits the ability to range the reactor as the shape of the spectrum is what is used to determine the range.}
 \label{fig:hart-model-data}
\end{figure}

The observation time needed to range Hartlepool to this accuracy is 40 months, with the uncertainty dropping to the level in Table~\ref{tab:results-chi} by around 50 months, as demonstrated by Fig.~\ref{fig:chi-stats}.

\begin{figure}[htp]
\includegraphics[width=0.58\linewidth]{hartlepool_res_events_v2.pdf}
\centering
\caption{The observed range of the Hartlepool reactor complex with observation time.}
\label{fig:chi-stats}
\end{figure}

Due to the simplicity of the method, and the need for a signal-dominated spectrum, this analysis is not appropriate for higher-background situations such as more distant reactors. Further analysis would be required to isolate a complete reactor spectrum for this method to be effective at larger distances.

\subsection{Fourier transform}
\label{subsec:results-ft}

The \acf{ft} method is applied in four possible scenarios on five reactor complexes. The scenarios are combinations of including background uncertainties and detector energy resolution.

The results of the \ac{ft} method shown in Table.~\ref{tab:results-ft} show that for reactors at large distances, the range can be determined when the detector's energy resolution is accounted for. However, reactors beyond 300 km do not have a large enough signal to be ranged effectively when background uncertainties are included.

The two nearer reactors, \acp{agr} at Heysham and Torness, can be ranged close to the true value when background uncertainties are included.

\begin{table}[htb]
    \caption[Results summary FT]{Observed distance in km for the Fourier transform method for the situations where true energy and reconstructed energy are used. Inclusion of background uncertainties is compared to the situation of zero background uncertainty. Situations with a slash are deemed impossible to range due to background uncertainties dominating.}
    \begin{ruledtabular}
    \begin{tabular}{c|ccccc}
         {Situation} & & & {Range [km]}\\
          & {Heysham 2}  & {Torness} & {Sizewell B} & {Hinkley Point C} & {Gravelines}\\
         \hline
         True Range & 149 & 187 & 304 & 404 & 441 \\
         \hline
         No Background, True Energy & 148 $\pm$ 4 & 188 $\pm$ 5 & 306 $\pm$ 8 &  403 $\pm$ 11 & 440 $\pm$ 11 \\
         No Background, Reconstructed Energy & 157 $\pm$ 4 & 195 $\pm$ 5 & 307 $\pm$ 8 &  397 $\pm$ 11 & 432 $\pm$ 11 \\
         Background Uncertainty, True Energy & 156 $\pm$ 6 & 177 $\pm$ 10 & \diagbox[innerwidth=5em, height=0.5\line]{}{} & \diagbox[innerwidth=5em, height=0.5\line]{}{} & \diagbox[innerwidth=5em, height=0.5\line]{}{}\\  
         Background Uncertainty, Reconstructed Energy & 155 $\pm$ 5 & 171 $\pm$ 9 & \diagbox[innerwidth=5em, height=0.5\line]{}{} & \diagbox[innerwidth=5em, height=0.5\line]{}{} & \diagbox[innerwidth=5em, height=0.5\line]{}{}\\
    \end{tabular}
    \end{ruledtabular}
    \label{tab:results-ft}
\end{table}

As shown in Fig.~\ref{fig:FT-range-limit}, reactors within approximately 80 km of the detector cannot be accurately ranged by the \ac{ft} method. As such, the EDF Hartlepool cores were not ranged as part of this analysis.

The \ac{ft} for Heysham 2 with background uncertainties and with energy reconstruction applied is shown by Fig.~\ref{fig:hey-FT}. The maximum for the \ac{fct} yields an accurate range, but with an uncertainty of $\pm$ 15 km. The \ac{fst} is able to reduce this uncertainty significantly to $\pm$ 6, shown in Table~\ref{tab:results-ft}.

\begin{figure}[htp]
\includegraphics[width=8.6cm]{hey-FT.pdf}
\centering
\caption{The Fourier transform, in sine (black solid) and cosine (red dotted), for Heysham 2 with background uncertainties and energy resolution effects.}
\label{fig:hey-FT}
\end{figure}

Due to the low event rates for the distant reactors, it takes over 50 years of observation time to be able to range the Heysham complex, and significantly longer for the more distant reactors.

\section{Discussion}
\label{sec:discussion}

The results of both methods show the potential of using neutrino oscillation to determine the distance to an observed reactor, as well as the use of extending the analysis in \cite{Sensitivity2022} to include a spectral analysis. The two analyses presented complement each other well, with the minimum chi-squared analysis allowing nearby reactors with a large signal contribution to be ranged, and the \acf{ft} method allowing the ranging of more distant reactors.

Despite showing potential, there are strong limitations to both methods. While the chi-squared method can handle lower energy resolutions for mid-field reactors, the energy threshold and detector efficiency strongly limits the utilization of lower energy events. This causes the discrepancy between the true and observed range for the Hartlepool reactors.

The \ac{ft} method is able to range reactors in more complex scenarios. However, the event rate for these distant reactors results in this taking a long time, in excess of 50 years, and therefore being impractical.

This detector configuration is not able to make the best use of spectral analysis. A potential solution is using gadolinium-loaded liquid scintillator. This would lower the energy threshold, boost detector efficiency at low energies and improve energy resolution. In principle, this could allow the \ac{ft} method to work at much shorter ranges by resolving the $\theta_{13}$ oscillations or allow the chi-squared method to range Hartlepool with much more accuracy.

\section{Conclusions}
\label{sec:conc}

An attempt at extending the rate-only analysis of nuclear fission reactor antineutrinos in \cite{Sensitivity2022} has been made by using two methods of spectral analysis with the aim of determining the distance between a reactor and a detector. The simulated detector used is a 22 m height and diameter right cylinder with a 9 m inner \ac{pmt} support structure and a 15\% \ac{pmt} coverage. The detector is filled with gadolinium-doped water-based liquid scintillator, and is located at the Science \& Technology Facilities Council (STFC) Boulby Underground Laboratory.

The analyses show potential to range real reactor signals, but are significantly limited by the detector design. A minimum chi-squared analysis is able to range nearby reactors which produce a dominant signal well within the lifetime of this kind of detector, with the EDF Hartlepool reactor ranged to 50\% of its true distance. A Fourier transform analysis is able to handle reactors at much larger standoff distances, up to 180 km when background uncertainties are included. However, this would take a very large amount of time due to the low event rate.

With both analyses, the fundamental issue is the detector's performance. An order of magnitude increase in signal rate is needed to range the Heysham 2 cores at 149 km within a detector lifetime, and the energy thresholds and detector efficiency limit the ranging of more local reactors. Using gadolinium-doped liquid scintillator could offer a solution as it would improve energy resolution, lower the threshold and improve low energy efficiency.

Due to performance issues, the gadolinium-doped \ac{wbls} detection medium is not appropriate to use in determining the distance to operating reactors. However, the analyses developed could be used with a more sensitive detector for this purpose.

\begin{acknowledgments}

The authors would like to thank L. Kneale for her input on the simulation and analysis of data (see \cite{Sensitivity2022}) as well as her review of the work. Thanks also go to T. Appleyard for her early work on the LEARN analysis, and A. Scarff for his regular review of the work.

\end{acknowledgments}

\section*{Author contributions}

The initial proposal of both reactor ranging in general and the application of a Fourier transform came from J. G. Learned.

Monte Carlo simulations produced by S. Wilson with input from L. Kneale.

The LEARN analysis for data reduction developed by S. Wilson with input from J. Armitage, N. Holland and T. Appleyard. Sections of the data reduction, such as the analytic post-muon veto and radionuclide calculations were developed by L. Kneale.

The chi-squared analysis was initially developed by J. Armitage, later being taken on by S. Wilson. The Fourier transform was developed by C. Cotsford.

C. Cotsford drafted Section~\ref{subsec:fourier}, with S. Wilson drafting the remainder of the paper.

\bibliography{aps}

\end{document}
%
% ****** End of file main.tex ******