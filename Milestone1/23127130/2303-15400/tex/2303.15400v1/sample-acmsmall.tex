%%
%% This is file `sample-acmsmall.tex',
%% generated with the docstrip utility.
%%
%% The original source files were:
%%
%% samples.dtx  (with options: `acmsmall')
%% 
%% IMPORTANT NOTICE:
%% 
%% For the copyright see the source file.
%% 
%% Any modified versions of this file must be renamed
%% with new filenames distinct from sample-acmsmall.tex.
%% 
%% For distribution of the original source see the terms
%% for copying and modification in the file samples.dtx.
%% 
%% This generated file may be distributed as long as the
%% original source files, as listed above, are part of the
%% same distribution. (The sources need not necessarily be
%% in the same archive or directory.)
%%
%% Commands for TeXCount
%TC:macro \cite [option:text,text]
%TC:macro \citep [option:text,text]
%TC:macro \citet [option:text,text]
%TC:envir table 0 1
%TC:envir table* 0 1
%TC:envir tabular [ignore] word
%TC:envir displaymath 0 word
%TC:envir math 0 word
%TC:envir comment 0 0
%%
%%
%% The first command in your LaTeX source must be the \documentclass command.
%\documentclass[acmsmall]{acmart}
%% NOTE that a single column version is required for 
%% submission and peer review. This can be done by changing
%% the \doucmentclass[...]{acmart} in this template to 
\documentclass[manuscript,screen]{acmart}
%% 
%% To ensure 100% compatibility, please check the white list of
%% approved LaTeX packages to be used with the Master Article Template at
%% https://www.acm.org/publications/taps/whitelist-of-latex-packages 
%% before creating your document. The white list page provides 
%% information on how to submit additional LaTeX packages for 
%% review and adoption.
%% Fonts used in the template cannot be substituted; margin 
%% adjustments are not allowed.
%%
%% \BibTeX command to typeset BibTeX logo in the docs
\AtBeginDocument{%
  \providecommand\BibTeX{{%
    \normalfont B\kern-0.5em{\scshape i\kern-0.25em b}\kern-0.8em\TeX}}}

%% Rights management information.  This information is sent to you
%% when you complete the rights form.  These commands have SAMPLE
%% values in them; it is your responsibility as an author to replace
%% the commands and values with those provided to you when you
%% complete the rights form.
\setcopyright{acmcopyright}
\copyrightyear{2018}
\acmYear{2018}
\acmDOI{XXXXXXX.XXXXXXX}


%%
%% These commands are for a JOURNAL article.
\acmJournal{JACM}
\acmVolume{37}
\acmNumber{4}
\acmArticle{111}
\acmMonth{8}

\usepackage[nolist,nohyperlinks, printonlyused, withpage]{acronym}
%\usepackage[nolist,nohyperlinks]{acronym}
\usepackage{csquotes}
%%
%% Submission ID.
%% Use this when submitting an article to a sponsored event. You'll
%% receive a unique submission ID from the organizers
%% of the event, and this ID should be used as the parameter to this command.
%%\acmSubmissionID{123-A56-BU3}

%%
%% For managing citations, it is recommended to use bibliography
%% files in BibTeX format.
%%
%% You can then either use BibTeX with the ACM-Reference-Format style,
%% or BibLaTeX with the acmnumeric or acmauthoryear sytles, that include
%% support for advanced citation of software artefact from the
%% biblatex-software package, also separately available on CTAN.
%%
%% Look at the sample-*-biblatex.tex files for templates showcasing
%% the biblatex styles.
%%

%%
%% The majority of ACM publications use numbered citations and
%% references.  The command \citestyle{authoryear} switches to the
%% "author year" style.
%%
%% If you are preparing content for an event
%% sponsored by ACM SIGGRAPH, you must use the "author year" style of
%% citations and references.
%% Uncommenting
%% the next command will enable that style.
%%\citestyle{acmauthoryear}

%%
%% end of the preamble, start of the body of the document source.
\begin{document}

%%
%% The "title" command has an optional parameter,
%% allowing the author to define a "short title" to be used in page headers.
\title{Designing for Disengagement: Challenges and Opportunities for Game Design to Support Children's Exit From Play}

%%
%% The "author" command and its associated commands are used to define
%% the authors and their affiliations.
%% Of note is the shared affiliation of the first two authors, and the
%% "authornote" and "authornotemark" commands
%% used to denote shared contribution to the research.


\author{Meshaiel Alsheail}
\affiliation{%
  \institution{Karlsruhe Institute of Technology}  
  \city{Karlsruhe}
  \country{Germany}
}
\email{meshaiel.alsheail@kit.edu}

\author{Dmitry Alexandrovsky}
\affiliation{%
  \institution{Karlsruhe Institute of Technology}  
  \city{Karlsruhe}
  \country{Germany}
}
\email{dmitry.alexandrovsky@kit.edu}

\author{Kathrin Gerling}
\affiliation{%
  \institution{Karlsruhe Institute of Technology}  
  \city{Karlsruhe}
  \country{Germany}
}
\email{kathrin.gerling@kit.edu}




%%
%% By default, the full list of authors will be used in the page
%% headers. Often, this list is too long, and will overlap
%% other information printed in the page headers. This command allows
%% the author to define a more concise list
%% of authors' names for this purpose.
\renewcommand{\shortauthors}{Alsheail, et al.}

\begin{acronym}
\acro{HCI}{Human-Computer Interaction}
\acro{PX}{Player Experience}
\acro{SDT}{Self-Determination Theory}
\acro{UE}{User Engagement}
\acro{UX}{User Experience}
\end{acronym}


%%
%% The abstract is a short summary of the work to be presented in the
%% article.
\begin{abstract}
%Playing video games is an enjoyable and entertaining activity.
Games research and industry have developed a solid understanding of how to design engaging, playful experiences that draws players in for hours and causes them to lose their sense of time. 
While these designs can provide enjoyable experiences, many individuals -- especially children -- may find it challenging to regulate their playing time, and often they struggle to turn off the game. 
In turn, this affords external regulation of children's playing behavior by limiting playing time or encouraging alternative activities, which frequently leads to conflicts between parents and the children.
Here, we see an opportunity for game design to address player disengagement through design, facilitating a timely and autonomous exit from play. %However, forcing children to stop playing can negatively affect their experience and remove their independence (i.e., sense of autonomy).
Hence, while most research and practitioners design for maximizing player engagement, we argue for a perspective shift towards disengagement as a design tool that allows for unobtrusive and smooth exits from the game. We advocate that interweaveing disengagement into the game design could reduce friction within families, allowing children to finish game sessions more easily, facilitate a sense of autonomy, and support an overall healthier relationship with games.
In this position paper, we outline a research agenda that examines how game design can address player disengagement, what challenges exist in the specific context of games for children, and how such approaches can be reconciled with the experiential, artistic, and commercial goals of games. 
\end{abstract}

%%
%% The code below is generated by the tool at http://dl.acm.org/ccs.cfm.
%% Please copy and paste the code instead of the example below.
%%
\begin{CCSXML}
<ccs2012>
   <concept>
       <concept_id>10003120.10003121.10003124</concept_id>
       <concept_desc>Human-centered computing~Interaction paradigms</concept_desc>
       <concept_significance>500</concept_significance>
       </concept>
   <concept>
       <concept_id>10003120.10011738.10011772</concept_id>
       <concept_desc>Human-centered computing~Accessibility theory, concepts and paradigms</concept_desc>
       <concept_significance>500</concept_significance>
       </concept>
 </ccs2012>
\end{CCSXML}

\ccsdesc[500]{Human-centered computing~Interaction paradigms}
\ccsdesc[500]{Human-centered computing~Accessibility theory, concepts and paradigms}

%%
%% Keywords. The author(s) should pick words that accurately describe
%% the work being presented. Separate the keywords with commas.
\keywords{Game Design, Disengagement, Dark Patterns}

\received{20 February 2007}
\received[revised]{12 March 2009}
\received[accepted]{5 June 2009}

%%
%% This command processes the author and affiliation and title
%% information and builds the first part of the formatted document.
\maketitle

\section{Introduction}
Supporting disengagement from play is relevant for children as it contributes to healthy gaming practices, can help avoid conflict within families about playing time, and could potentially reduce the risk of harmful overuse of games. While there is a substantial body of literature that addresses the lack of disengagement from play through the lens of pathological use of games (e.g., clinical perspectives on game addiction~\cite{fisher1994, griffiths1997}, surprisingly little is known about disengagement from games from the perspective of games design, i.e., whether and how specific design strategies or game mechanics could support the exit from play.
Instead, much of the work addresses continued player engagement (both in the context of positive game design - for example, achieving flow~\cite{cowley2008}, but also with a critical view - for example, work on dark patterns~\cite{zagal2013}), and \ac{HCI} games research likewise overwhelmingly focuses on the entry into play (cf. ~\cite{petersen2017}) and continued participation in it(cf.~\cite{lewis2012, bowman2021, deleonpereira2021}). 
Where disengagement is addressed, it is often done so with a negative connotation through the lens of player attrition (e.g.,~\cite{alexandrovsky2021, hadiji2014, hui2013}), and from a restrictive perspective such as external terminating of play~\cite{stevens2021}. 
The negative lens of \ac{HCI} research on disengagement has previously been criticized by O'Brien and colleagues~\cite{obrien2022}, who outlined that disengagement can also be temporary, a result of momentary satisfaction with the preceding experience, and an expression of user agency. 
We argue that the limited view on disengagement is also  a missed opportunity for games research: Appreciating the final stages of play as a part of \ac{PX} that should actively be designed for gives game designers an additional tool within their box. 
Additionally, developing strategies that support players in the achievement of an exit from play at their own volition should contribute to player autonomy (which has for example been extensively studied in the context of remaining within play~\cite{deterding2016}), for younger and older players alike.

In this position paper, we reflect on the potential benefits of designing for disengagement in games for children, an audience that still establishes gaming habits and needs to negotiate these in a family context. To this end, we summarize current perspectives in \ac{HCI} games research on player engagement, we give an overview of current industry best practices designed to limit children's engagement with games, and we outline pathways to game design-driven strategies for player disengagement that empower children and their parents to exit play in a positive context.

\section{Disengagement, Digital Games, and Children's Engagement}
% INTRO PARAGRAPH GOES HERE
In this section, we first introduce how \ac{HCI} research defines disengagement. Then, we examine how the concept is approached in \ac{HCI} games research, and we reflect upon current best practices in industry and research to support children's disengagement from digital games and other media.

\subsection{Defining Disengagement within HCI Research}
Research in \ac{HCI} has approached engagement and disengagement from different perspectives. In this work, we follow \citet{obrien2008}'s definition of \acf{UE} as the theoretical foundation.
\ac{UE} refers to the \textquote[\cite{obrien2016}]{quality of user experience characterized by the depth of an actor’s investment when interacting with a digital system} and~\textquote[\cite{attfield2011}]{emphasises the positive aspects of the interaction and, in particular, the phenomena associated with being captivated by the technology}. 
%\ac{UE} is a concept that embodies the \textquote{health of the system} and helps to design \textit{useful} and not just \textit{usable} systems~\citet{doherty2019}.
As~\citet{attfield2011} stated, \textquote[\cite{attfield2011}]{successful technologies are not just used, they are engaged with; users invest time, attention, and emotion into the equation}.
Therefore, in most cases, it is desirable to have an engaging interactive system as it facilitates retention, productivity, and overall satisfaction.
%For instance, many computer-aided health interventions rely on recurrently repeated exercises over a prologues period~\citep{mandryk2017,perski2017}. Similarly, learning and skill acquisition are much more effective and sustainable if the practice sessions are repeated regularly~\citep{criscimagna-hemminger2008, johanson2019, kim2013} or follow a pattern of spaced repetition -- repetitions with exponentially growing intervals between the sessions~\citep{schimanke2017}.
The sense of engagement is inferred by cognitive, affective, and behavioral constructs such as flow~\citep{csikszentmihalyi1990}, motivation~\citep{rigby2011}, attention~\citep{attfield2011, obrien2008}, or adherence~\citep{couper2010, nacke2011, perski2017}.
The cognitive aspect of engagement frequently relies on conscious components such as attention, interest, or effort~\citep{doherty2019, islassedano2013,sun2014}.
The affective component of engagement encompass the subjective emotional responses including enjoyment, aesthetics, endurability, and novelty~\citep{obrien2008, doherty2019}.
Behavioral component describes the action and participation with the activity~\citep{doherty2019, islassedano2013, silpasuwanchai2016}.
\citet{obrien2008} conceptualize four stages of UE: \textit{point of engagement} is the first contact with the interactive system; \textit{period of sustained engagement} is the actual time span users interacting with the system; \textit{disengagement} describes the termination point of an engaging period (e.g., end of a session); and \textit{re-engagement} is referred to when users return to the interactive system and is marked by active choice taken from the user. Each of these phases is characterized by different attributes of the \ac{UX} that the interaction design should emphasize~\citep{obrien2008}. 
These four stages allow conceptualizing experiences with interactive systems on a timeline with interaction cycles where each stage gives specific target points for the interaction design that researchers and designers could use to orchestrate the experience (e.g., the behavior of the user).

%\subsection{Games for Children and the Role of Disengagement}

\subsection{Engagement and Disengagement in Games Research}
\label{sec:engagement_disengagement_in_games}
%In \ac{HCI} research, engagement is defined as the interaction and involvement of a user with technology\cite{obrien2008}. Furthermore,~\citet{obrien2008} present a framework with affective, behavioural, and cognitive dimensions, where affective engagement refers to emotional response, behavioural engagement to actions and behaviours, and cognitive engagement to mental processes \cite{obrien2008a}.   
% Put in definition here, can pull from Toms & O'Brien.
% Put in the definition that is specific to games here.
In games research, engagement describes the act of players being fully absorbed and involved in the game, characterized by a high level of motivation, attention, and emotional investment \cite{schonau-fog2012, rapp2022}.
In the context of games, \ac{SDT} is commonly used to explain the motivation of play~\cite{deci2012}. \ac{SDT} suggests that providing players with opportunities to make choices that align with their interests and values, experience a sense of mastery or progress, and connect with other players or characters in the game will lead to increased engagement and enjoyment~\cite{tyack2020, tyack2020a}. In line with \ac{SDT}, Flow theory proposes that an optimal and thus, engaging experience emerges when the players' skill and game difficulty are well-matched~\cite{csikszentmihalyi2000}.
% Example goes here, a sentence that continued in your text didn't connect. 
A significant and growing body of research has been conducted on what keeps players engaged with the game, for example, through game updates, by providing new content and boosting the experience for the better through challenges, which creates satisfaction and enhances the \ac{PX}~\cite{zhong2021,claypool2017,altimira2017,zhong2022}. %Moreover, the study by ~\citet{rapp2022}analyzes five elements (Temporal structure, Game mechanics, User interface and feedback, Social dynamics, and Immersion) as critical factors that impact player engagement in massively multiplayer online role-playing games. The author recommends game designers consider the interplay between these elements to maximize player engagement and satisfaction in their games \cite{rapp2022}.

\textbf{From a commercial perspective, publishers are interested in player engagement to increase profitability} through game subscriptions, attract and retain customers, and keep the game entertaining with new content~\cite{baptista2019,rooija.j.van2021}. In consequence, game designers increasingly use strategies with manipulative elements to keep users connected to their games~\cite{rooija.j.van2021,zagal2013}. Among others, such \textit{dark patterns} include temporal dark patterns (i.e., employ time-limited tactics to create a sense of urgency, encouraging users to take desired actions) and psychological manipulations (such as discounts on resources, encouraging users to invest money to enhance their gameplay) make it more challenging for players to disengage from play at their own volition \cite{zagal2013,rooija.j.van2021}.

%Moving beyond engagement, work that addresses disengagement describes it mostly as a negative event. 
Despite these concerning design strategies, \textbf{the dominant perspective in games research is that disengagement  -- the process in which the user retreats from interacting with a system either temporarily or permanently \cite{obrien2022} -- is the result of poor game design} that does not encourage continued player engagement \cite{obrien2022}. For example,~\citet{xie2014} studied how to predict player disengagement in online games in an effort to identify instances in which designers would need to address design flaws. 
Similarly, ~\citet{oertel2020, ben-youssef2021}  have presented disengagement in HCI as halting the problematic or meaningless consumption of technology after feelings of "frustration" based on poor design or lack of motivation in its design. 


%Here, we draw from \cite{gutl2014,obrien2022} to argue for a shift in perspective that views disengagement as a natural part of a play experience, as every gaming session will and should come to an end. This is in line with a small body of work in \ac{HCI} that is not specific to games and suggests that temporary disengagement, in particular, is a frequently neglected positive trait.


%\citet{obrien2022} conceptualize disengagement as a natural part of the engagement--disengagement--re-engagement cycle and is spanned across two dimensions: the degree of users' agency and the span between positive and negative engagement. This definition of disengagement encompasses the users' goals, the meaning of usage, and the degree of control the users take over interacting with the system. 
%For example, users ending problematic consumption of technology, where individuals spend time and energy playing games, using social media networks, watching videos, and browsing the web \cite{obrien2022, obrien2008}. Positive disengagement with a high sense of agency can occur when users have achieved their individual goals, which - in the context of games - could, for example, be understood as having had a satisfying experience within the game and no longer feeling the need to continue a given gaming session~\cite{obrien2022}.

Generally, we need to acknowledge the \textbf{tension between the aforementioned \textit{dark patterns} and the player's ability to achieve a satisfying experience and retain agency to end play at a point in time that is convenient for them}. In consequence, there is a large body of literature focusing on problematic play behaviors (i.e., excessive play\cite{sublette2012}, and gaming addiction \cite{griffiths2009,wood2008, kuss2012}). Particularly in the context of children and their engagement with games, this is discussed through the lens of youth well-being~\cite{gil2020, vanrooij2017, khorsandi2022, elsayed2021} and frequently addressed through external strategies to support disengagement from play, which we discuss in the following section~\cite{clark2011,peters2018}.

%O'Brien et al. (2022) further explain that time and energy are viewed as valuable commodities, and users stop the usage of applications such as social media networks, e-commerce websites, and online gaming platforms either temporarily or permanently. To further explain the concept of disengagement in HCI, Seo et al. (2021) report that the break users take from the application's usage is the response to their problematic consumption. Hinker et al. (2016) explain that disengagement can be the process of taking a halt to the use of problematic technologies while looking for alternate technologies that may assist them in discontinuing or pausing their use. 

%O'Brien and Toms (2008) explain disengagement as the adverse reaction of the users towards the use of technology that can stem from usability issues, over-challenging tasks, or disruptions in HCI environments, forcing them to withdraw from the HCI. According to Short et al. (2015), system designers' and developers' central focus is to ensure that disruptions are minimized to maintain \ac{UE}, as disengagement is not desired.

%Similarly, Trans et al. (2019) have also reported that after patterns of habitual technology consumption, users regret wasting their time and stop using the technology. Goethe et al. (2019) have not explicitly mentioned the concept of disengagement but have reported that lack of engagement in HCI can result in “meaningless interactions” between the user and the technology, which is not the end goal of the system designer. 


% continue with summary of O'Brien's paper on disengagement (not the one on the engagement cycle).

%In actuality, this temporary disengagement 
%To facilitate engagement and disengagement, O’Brien and Toms (2008) and O'Brien et al. (2022) have proposed a cyclic engagement process, which comprises the following parts: engagement, disengagement, and re-engagement. Engagement is the point of contact between the user with the HCI. This engagement is sustained in the form of positive engagement, where the end is motivated while using the technology over a given period of time. O'Brien and Toms (2008) first reported that user attention and interests are essential to maintain engagement and facilitate the point of engagement in relation to the HCI design. As designers design applications based on user needs and requirements such as novelty, aesthetics, motivation, and interest, they will successfully evoke positive emotions among them. The authors also reported that users disengaged or dissociated themselves from the technology or application because of a variety of factors, such as a lack of motivation to use it, challenges, or inadequate novelty or challenge (O'Brien and Toms, 2008). O’Brien et al. (2022) further extended this discussion by redefining disengagement in HCI and stated that disengagement occurs when negative feelings evoke in the user, leading to feelings of frustration and guilt. In this phase, the user will break away from the technology with another technology or other activities that can change the engagement shifts. Disengagement is required to move ahead (O’Brien et al., 2022). In the process of re-engagement, the use may re-engage with a particular technology or application, would disengage and return to it or use an entirely different application. Re-engagement can be short-term or long-term (O'Brien and Toms, 2008). O'Brien et al. (2022) further extended this concept and explained that re-engagement occurs when the engagement process ends. In HCI, re-engagement may be moving away from the technology or application after a period of engagement and disengagement. 





%- Disengagement as negative event that should be avoided
%The concept of disengagement in HCI has yet to be extensively studied (O'Brien et al., 2022). However, few authors have attempted to define the concept in various technological settings and \ac{HCI} (O’Brien and Toms, 2008; Ahuja and Kumar, 2021; O’Brien, 2018; O’Brien and Cairns, 2016; O’Brien et al., 2022).  In the definition presented by O'Brien et al. (2022), disengagement is viewed as the user’s disconnection from the application, thus, withdrawing energy and time. However, it is a temporary disconnection from the HCI (O'Brien and Tom, 2008).  Disengagement is also identified as a strategy to disconnect from a technology deemed to be problematic. For instance, stopping the use of a mobile application because of investing a massive amount of energy and time in it.



%- Really briefly: Lack of disengagement as addiction
%The designers and publishers desire \ac{UE} in gaming environments by enhancing the user's intrinsic motivations through game design environment. Researchers have successfully identified that motivating users to engage in gameplay is a significant challenge, as individuals cannot sustain gameplay engagement for an extended period of time because of psychological and physiological necessities (Buono et al., 2017; Darzentas, Brown and Curran, 2014). The employment of engagement in applications to retain users is the end goal of the game designers and publishers, as reported in the literature (Boyle et al., 2012). Therefore, prior studies have revealed that absence of disengagement is the primary cause of video addiction among children, adolescents, and adults (Xu, Turel and Yuan, 2012; Tran et al., 2019; Goethe et al. 2019; Oertel et al., 2020). While O'Brien et al. (2022) report that positive engagement in HCI is beneficial, it may lead to feelings of negativity, frustration, anxiety, and guilt; as previously mentioned, it is linked with the compulsive usage of applications and technology. Game designers intentionally design games to alter and modify user behaviour; therefore, disengagement is viewed as a negative connotation (Xu, Turel and Yuan, 2012; Tran et al., 2019; Goethe et al., 2019; Oertel et al., 2020). Studies have established that high levels of engagement are responsible for causing video game addiction (Bean et al., 2017; Stockdale and Coyne, 2018; Plante et al., 2019), which is described as the "maladaptive psychological" dependency on video games (Xu, Turel, and Yuan, 2012). This dependency is characterised by obsessive-compulsive usage of video games that impairs the individual's ability to perform daily tasks over a long period (Xu, Turel and Yuan, 2012). A large portion of games designed by game publishers intends to maintain user immersion, believing that disengagement is a negative experience, whereas continuous engagement is a positive experience for users in video games and other HCI environments (O’Brien and Toms, 2008; O’Brien et al., 2022). To maintain usability, game designers design games with a variety of features such as interactivity, competitive rivalry, and multiplayer functionality, allowing users to interact in the virtual environment through the use of the Internet and maintaining their interests (Kuss and Griffiths, 2012). Because of the collaborative and interactive design of these games, children and adolescents are at a higher risk of becoming addicted to them, which in turn can have detrimental effects on their mental and physical health, including anxiety, depression, insomnia, suicidal tendencies, and hypertension (Kuss and Griffiths, 2012; O’Brien et al., 2022). O’Brien and Cairns (2016) have associated high engagement with addiction. Therefore, disengagement strategies in video games are fundamentally required to address the challenge of video game addiction among children, adolescents, teenagers, and adults.

%- Gap in research: disengagement as natural and desirable part of play
%Prior synthesis of literature demonstrates that disengagement has not been studied extensively by researchers (O'Brien et al., 2022) and that it is viewed as a negative trait by game designers and publishers (Xu, Turel and Yuan, 2012; Tran et al., 2019; Goethe et al. 2019; Oertel et al., 2020; O’Brien and Toms, 2008). Furthermore, engagement in gaming research primarily focuses on an extended period of time, user immersion, user play, and user enjoyment and motivation, it is not surprising that seldom efforts have been made to understand the concept of disengagement. Much of the work that has been conducted in terms of disengagement in HCI and gaming environments has primarily been done by O'Brien and Toms (2008), O'Brien and Cairns (2016), and O'Brien et al. (2022). Some of the studies pertaining to engagement have discussed engagement in the context of game design but as a negative user experience (Tran et al., 2019; Goethe et al., 2019; Oertel et al., 2020; O’Brien and Toms, 2008). Rapp (2022) has reported that inadequate \ac{UE} can result in decreased profitability. Therefore, game designers focus on engaging their audience through dark patterns strategies of engagement. O'Brien et al. (2022) has reported that disengagement is rarely studied in the literature. From a game design and play perspective, it is discouraged and viewed negatively. It is also associated with "poor usability, negative emotional experiences, and disinterest" (O’Brien et al., 2022). 
%Kappes and Thomsen (2022) have also reported that disengagement is a natural process in goal engagement settings. The authors report that while goal engagement is essential for success, the readjustment of goals by disengaging from them is essential to avoid the wastage of resources and time. O’Brien and Cairns (2016) report that disengagement in games is a natural process. This statement is also reinforced in O'Brien et al. (2022) in the context to HCI interfaces and systems. Disengagement from a game design perspective has two core aspects: pausing or stopping the use of the technology or application and the degree of control the user have over it in terms of its usage. Like engagement, disengagement is also “goal-driven” and users would disengage from the application or game as soon as they have successfully accomplished their goals. An essential element in disengagement is the user’s degree of control over the process of disengaging from the app (O’Brien and Cairns, 2016).  
%Mancarella et al. (2022) have reported that action video games (AVGs) can effectively contribute to visual attentional efficiency and therefore, can also develop reading and phonological skills among children. The authors have reported that multisensory attention disengagement contributed to enhanced phonological and reading skills among AVG players compared to non-players. 

% WHEN DOES LACK OF DISENGAGEMENT BECOME A PROBLEM? Summary.

\subsection{Current Approaches to Support Children’s Disengagement from Games}
\label{sec:children_disengagement}
% Direct interventions that prioritize individual
Research and industry have explored a range of strategies to support children's disengagement from games and reduce playing time.

\textbf{The dominant strategy to address (the lack of) disengagement is through the introduction of time restrictions, either at an individual or societal level.} Such tools involve timers \cite{hiniker2016a}, trackers of usage \cite{kim2016a, RescueTime2007}, automated nudges to disengage \cite{okeke2018}, promoting self-regulation through social support and goal-setting \cite{ko2015}, or block users entirely from using the device or specific apps~\cite{lee2014, jasper2015}. Directly addressing children, various tools seek to manage screen time and other issues by allowing parents to set time limits \citet{bieke2016}, e.g., Net Nanny~\cite{ross2021}, CYBERsitter~\cite{milburn1998}, child-friendly filters on Netflix or Apple's ScreenTime\footnote{\url{https://support.apple.com/en-ca/HT208982}} which also provide detailed statistics about the usage of individual apps.

%As with the example of ScreenTime, often the applications implement a permissions control scheme that allows, for example, parents to restrict the time children are interacting with technology. 
%\citet{bieke2016} gives an overview of existing parental control methods to support children's safety with digital media and categorize the tools in \textit{time restrictions}, \textit{content restrictions}, \textit{activity restrictions}, and \textit{monitoring an tracking}. 
% Societal and legal interventions
Time restrictions are also introduced at a societal level. For example, South Korea implemented in 2011 a law that regulates how much players are allowed to play within a 24-hour period. 
Likewise, China introduced a time-limit policy that decreases players' rewards after a play window of 3 hours~\cite{kiraly2018}. 

\textbf{Within games, such patterns have been discussed as \textit{blocking} and \textit{waiting} mechanics and are frequently reported as dark patterns} \cite{alexandrovsky2019, alexandrovsky2021}.
%\citet{davies2016} investigated how these time-limiting restriction policies affect the \ac{PX} and found that fatigue-inducing game elements (i.e., waiting mechanics such as implemented by the Chinese government) make players less likely to return to the game compared to a control condition or a hard shutdown (i.e., blocking). In contrast, blocking players from playing results in slightly higher frustration. The authors argue that a one-size-fits-all approach is not suitable as such restrictions are often not in line with the players' personal goals, time schedules, or commitments (e.g., playing together with teammates across the globe to achieve a goal). 
The approaches to reduce screen time in games can be explicit, such as the MS XBox warning players about excessive gameplay times~\cite{microsoft2016} or more subtle like in \textit{Stronghold: Crusader} every now and then, the in-game companion suggests the player take a break or asks if they want to drink~\cite{Crusader2002}. However, such disengagement strategies can reduce the players' sense of autonomy and can leave players with an unsatisfied experience feeding the wish to continue playing~\cite{davies2016}. Likewise, hindering players from achieving their goals can in fact cause frustration and aggression~\cite{card2011,battigalli2015}. 
%Furthermore, in line with the Zeignarik effect~\cite{mantyla1997}, which states that unfinished tasks have an inherent pull towards the activity, such blocking elements yield unresolved goals which may even facilitate behavioral addiction~\cite{alter2018}.
%In contrast, for problematic gambling, players who set themselves time limits before playing were more likely to stay within this time limit and to play more responsibly compared to players who had no such instructions~\cite{kim2014}.
However, planning out screen time in advance can help children to disengage from screen exposure~\cite{hashish2014,hiniker2018b}. Likewise,~\citet{zhang2022a} showed for social media usage that design patterns facilitating the users' agency are more effective than time-restricting methods.

For \textbf{children's media usage}, \citet{barr2020} noted that the context of media usage in families is rarely considered in the literature and argue that to understand the long-term effects of children interacting with digital media, research needs to consider measures beyond screen time. 
%For instance, it has been shown that adults' contact with digital media stains on how much and with what kind of content children are interacting~\cite{schoeppe2016}.
%While for children's media usage, the involvement of parents in curating the media content and extent the child is exposed to is crucial~\cite{bieke2016,clark2011}, leaving parents alone with protecting children's media exposure may be insufficient as 
Here, parental mediation of media use is an important pillar, but parents often are challenged in assessing the risks the children might encounter with digital media~\cite{bieke2016,mitchell2005}. 
The Parental Mediation Theory categorizes three communication strategies -- active, restrictive, and co-viewing mediation -- that can be leveraged to mitigate the negative effects of media use.
%and to mitigate the negative effects of children using digital media: \textit{active} mediation where parents talk with children about the content they consume; \textit{restrictive} mediation with rules and regulations about the child's media use; and \textit{co-viewing} mediation where parents and children consume media together~\cite{clark2011}. 
Studies on parental mediation of violent TV consumption showed that both active and restrictive mediation has been negatively related to the children's aggressive tendencies, while co-viewing showed a positive influence on the child's aggression~\cite{nathanson1999}. 
This is in line with situated learning theories which suggest that children learn through \textquote[\cite{brown1989}]{cognitive apprenticeship} which transforms learning from a process of transition towards a meaningful social activity~\cite{clark2011}.
Hence, \citet{bieke2016} argue that parental protection of children's media usage should not result in "helicopter apps" but rather support discussions between parents and children and encourage the child's autonomy~\cite{clark2011}.

% Beyond screen time
Here, some work exists on \textbf{children's consumption of video and TV that addresses disengagement beyond restrictions}. For example, \textit{Coco's Videos}~\cite{hiniker2018b} is a child-friendly video player that supports the child's self-regulation of screen exposure by letting the children decide how much time they like to spend watching. When the time has run out, a virtual character appears informing about the end of the session and making suggestions for other activities. The authors report the dialog with the virtual character at the end of the session gained value for the children and became part of the transitioning ritual. Likewise, \textit{FamiLync}~\cite{ko2015a} emphasizes both the parent's and children's sense of agency and provides support for a participatory and elucidating parental mediation of media consumption. Another approach promises to mitigate the side effects of screen-time restrictions using the physical periphery around the screen device to move the child's attention and ease the transition from immersion~\cite{yim2021}.
%However, time regulations (even self-regulated) might be inappropriate for children as their understanding of time and duration is limited~\cite {friedman1978}.

Reviewing the literature on disengagement, \textbf{we observe several gaps in research}: First, HCI and games research have only begun to address disengagement as part of the engagement cycle, mostly viewing it as a negative event. Second, common strategies to foster player disengagement view games as static objects, and instead seek to provide external strategies that help regulate the behavior of children and other players. Third, work that addresses children's media use has only begun to take into account developmental perspectives and family relationships, which leaves rooms for work in this space. 
%Likewise, disengagementHowever, games allow for a flexible design that may incorporate many facets of encouraging behavioral change. Following the presented methods on voluntary and autonomous disengagement (cf.~\cite{hashish2014,hiniker2018b}), we see a large potential in the design of game elements that promote self-regulated and positive disengagement.


% Strategies to address a disengagement
%- Parental mediation of play
%Parental involvement in children's media usage has been extensively studied in academia \cite{livingstone2017}. 
% Academic evidence supports the importance of parental involvement in reducing harm from online video games and avoiding manipulative design \cite{gentile2012, laczniak2016, martins2015, warren2017, beyens2019, nelsonmandela2020, gozum2021}.
% Parental mediation is the extent and way in which parents interact with their children in regard to their media consumption\cite{jiow2018}. %There are three essential types of parental mediation in the context of digital media consumption: active or instructive mediation, restrictive mediation, and co-playing \cite{martins2015}.



%-Setting time limits
%Parental Mediation in Game Platform 
%Academic evidence demonstrates that parental mediation and involvement in the context of children's and adolescents' media usage have been studied extensively in the literature (Livingstone et al., 2017; Nelsonmandela and Raja, 2020). Studies of parental mediation and online video game behaviour of children and adolescents have also been conducted. The majority of the studies have indicated that parental mediation and involvement are crucial in mitigating the harmful effects of online video games and averting the dark patterns incorporated by game designers to facilitate engagement in gameplay (Gentile et al., 2012; Laczniak et al., 2017; Martins, Mathews, and Ratan, 2017; Warren, 2017; Krossbakken et al., 2018; Beyens and Valkenburg, 2019; Nelsonmandela and Raja, 2020; Gözüm and Kandır, 2021). 
%Game designers and publishers can effectively incorporate parental mediation in the game platform, allowing them to monitor their children and encourage them to disengage from it when required. Martins, Mathews and Ratan (2017) reported that parents involved in playing online video games with their children were deemed beneficial in reducing their exposure to video games. Co-viewing or co-playing refers to the involvement of parents in monitoring the content of video games and other mediums that their children use. Gee, Siyahhan and Cirell (2017) state that family gameplay represents familial values, goals, and meanings that foster stronger familial relations and interactions between parents and their children.   While the term has been popular since the last two decades, it has yet to be studied extensively in literature. Family co-playing comes in a variety of forms. However, the most common form is where parents and children play the same video game together. On some occasions, they take turns playing the same video game. Theng, Chua and Pham's (2012) experimental research have reported that video game co-playing is beneficial in reducing inter-generational gaps and improving familial relations among family members. Studies on parent-children involvement in video games have also suggested that co-playing contributes to positive interactions between parents and children, which helps in reducing generational gaps and enhancing parent-child communication.

%-Setting time limits
%An important way of disengaging children and teenagers from video games is through the use of time limits. The use of time limits by game designers can effectively help in dealing with the challenge of video game addiction among children and adolescents. According to 00000, the implementation of time limits is essential to ensure that adolescents and teenagers take a break from continuous video-game play, which can result in negative engagement and video game addiction. As indicated in literature, Video game addiction has been identified as a pathological gaming behaviour among children, adolescents, and adults over a long time period. It is linked with negative consequences that affect their daily social, academic, and occupational functions (Lau et al., 2018). Gentile et al. (2017) have reported that 83 of adolescents engage in video gameplay, with video addiction occurring among 9\% of the participants assembled by the researcher. In Kuss and Griffiths’ (2012) literature review, the authors have reported that the internet gaming addiction prevalence rate in adolescents ranges from 0.6\% to 11.9\%. While the concept of disengagement has not been extended to the discussion of video game addiction in literature, the two terms are interconnected.   
%In Paulus et al. (2018) qualitative research, internet gaming disorder has been used instead of video game addiction. Their study has reported that this disorder is mainly prevalent among adolescents. Based on the findings, it has been verified that gaming disorder or video games addiction is caused by several factors such as lack of parental control, lack of parental involvement, the need for getting gaming rewards and valuable items in games, the need to gain acceptance among other players based on gaming skills and level, and interactive and competitive gameplay interface. Motivations for video games among children and adolescents are also because of “escapism and self-regulation” (Van Rooij et al., 2017). Therefore, setting time limits in the gameplay by game designers is a vital strategy to influence children and adolescents' behavior to dissociate them from the gameplay without disrupting their \ac{PX} (Zagal and Mateas, 2010, Kenny and Gunter, 2011, Davies and Blake, 2016). Evans, Jones and Akalin (2017) have reported that video games that offer multiplayer interactivity enhances student learning with shorter sessions and giving them breaks in between to “hang out, mess around, or geek out”(p.25). Therefore, it is proposed that adding time limitations in video gameplay can enhance \ac{UE} process through disengagement.   
%6.4 Reminders to Stop

%An important strategy that game designers can further incorporate to promote engagement-disengagement-reengagement cycle in video gameplays is by installing alarms or warning systems. According to ~\citet{microsoft2016}, Microsoft had installed an alarm system in Xbox One with the objective of warning the user of excessive gameplay.

%~\citet{kiraly2018} reported that the governments of South-east Asian countries went one step further in implementing reminders to stop the gameplay through a set of strategies including time limits that prohibited juvenile gaming between 10 pm till 6 am, requesting Internet service providers to block access to online games for minors, and in some countries such as China, banning accessibility to video games for adolescents and teenagers. While these strategies have been viewed to be extreme, implementing reminders can be beneficial in controlling video game play among adolescents and teenagers. 

% - slow technology ~\cite{hallnas2001}
% - Platform- and game-based strategies and features
%- Setting time limits, and reminders to stop
% - Advice to parents - websites and other non-scientific sources


% * Apple screen time and parental control \url{https://support.apple.com/en-ca/HT208982} 
% * Social media blocker
% * Halbert and Nathan 2015 Designing for Discomfort \url{https://dl.acm.org/doi/pdf/10.1145/2675133.2675162}

% research on time limits
% https://ieeexplore.ieee.org/abstract/document/7426249
% https://ila.onlinelibrary.wiley.com/doi/full/10.1002/jaal.455
%enjoy with caution:
%\url{http://irep.ntu.ac.uk/id/eprint/22799/1/188345_5965}%20Griffiths%20Publisher.pdf



\section{Challenges and Opportunities for Research: Supporting Disengagement While Maintaining Positive Play Experiences}
A key challenge for strategies to support disengagement from play is the tension between games seeking to provide immersive and engaging experience, while giving players -- including children -- autonomy to exit the experience at their own volition. Based on our examination of the perspective of games research on player disengagement, current best practices to facilitate disengagement of younger players, and the gaps therein, we would like to highlight the following three areas for future research.
%Based on the reviewed literature, we derived three research themes that should guide our research on disengagement in game design.

\subsection{Developing an Evidence-Based Perspective on Children's Exit From Play}
Much to our surprise, little is known about how children experience the exit from play beyond literature addressing problematic gaming (cf. Section~\ref{sec:engagement_disengagement_in_games}), suggesting that children's perspectives on ending play are poorly understood. Likewise, industry best practices on limiting children's playing time (cf. Section~\ref{sec:children_disengagement}) are neither rooted in an understanding of children's cognitive development nor reflect what we know about engaging \acp{PX}. For example, a prominent strategy to support disengagement is to introduce a maximum playing time, after which the gaming experience automatically ends. However, we know that games as immersive artifacts affect player perception of time (e.g., when experiencing flow \cite{csikszentmihalyi2000}), and in the context of children, the issue is exacerbated by the fact that humans only develop a concept of time from the age of seven~\cite{droit-volet2013}. Here, we see potential in a two-prong research approach that first seeks to understand the specific experience that children have when exiting games, and then developing evidence-based strategies to support the exit from play that are rooted in an understanding of children's cognitive development. Thereby, we would assume that designers and researchers could achieve the implementation of strategies that reduce friction within families, while maintaining a more positive overall gaming experience.

\subsection{Accounting for the Child-Parent Relationship in the Disengagement Process}
The relationship between kids, parents, and games is complex, and should likewise be taken into account when designing the disengagement process. % and can greatly impact child development and parent-child relationships. 
For example, \citet{donati2021} found that setting restrictions on the time, place, and content of video gaming can prevent excessive gaming, but \citet{papadakis2022} illustrates that parents struggle to control their children's time spent on tablets, with \cite{donati2021} suggesting that the effectiveness of such rules is moderated by the degree of parent-child agreement. %Unfortunately, they often lose this battle \cite{papadakis2022}. 
%According to ~\citet{friedman1978}, children have limitations in understanding the concept of time as a measurable duration and exhibit difficulties in differentiating between shorter and longer duration\cite{friedman1978}. In addition, individuals may experience the same duration of time differently, when they are fully engaged in a task \cite{csikszentmihalyi2000}.
\citet{kahila2022} report that children can experience intense anger when their in-game experiences are interrupted, for example, when parents remind them of homework, household chores, or meal times), especially when the game is going well. From the perspective of HCI games research, this presents an opportunity to design mechanics to support exit from play that account for the complexity of the relationship between parents and children, as well as family life. Instead of setting generic time limits, this could mean providing parents with the tools of understanding their children's experiences with games particularly addressing the question of when it is a 'good time' to quit, but also casting the process of shared responsibility in which parents need to support their children in finding an appropriate end, and researchers and designers need to weigh the needs of both groups.

%\cite{kahila2022}. These interruptions are often caused by homework, household chores, or meal times. These disruptions can negatively impact the player's immersion, mood, and flow state\cite{kahila2022}. According to ~\citet{wood2008}, children are often attached to video games because of the positive reinforcement they receive while playing~\cite{wood2008}

%Understand the context of children's play
%The dynamic between (kids-games-parents)
%How do kids and parents play together?
%-What the parent expected - How parents understand the games.
%What is problematic about the existing dynamics?

%\subsection{How to make kids satisfied with games}

% BRING ONE OF THESE BACK INTO THE GAME DESIGN PARAGRAPH
%The satisfaction of children with regard to games is influenced by various determinants. According to \citet{gutwin2016}, the peak-end effect can be used to re-engage players in casual games by designing game experiences that have memorable and intense peaks and satisfying endings~\cite{gutwin2016}. To do this, designers can focus on creating challenging and engaging gameplay and ensure that the game has a clear and satisfying conclusion. To re-engage players with a game,  according to \citet{tyack2020a}, game designers can use elements such as autonomy, competence, and relatedness to enhance player engagement. This can be achieved by providing choices and control, clear goals and feedback, and social features in the game design. They also mention considering players' personal and situational factors in game design to enhance motivation to re-engage with the game \cite{tyack2020a}.
%=======OR================ 
%The satisfaction of children with regard to games is influenced by various determinants. According to ~\citet{altimira2017}, In order to offer players a more captivating and fulfilling gameplay experience, it is essential for game designers to strike a balance between challenge and ease of play. In order to appease the player, it is important for the game designer to offer new content and provide an improved challenging element \cite{zhong2021}. In like manner\citet{gutwin2016}, declare that there are two factors affecting how the players remember and evaluate their game experience, which is the most moment of enjoyment, in addition to how the experience ended. Therefore, we must take into account that the end of the game must be presented to the child in an enjoyable way so that his experience with the game is positive.

%We want to find the point of satisfaction where it is easier to get off.
%For a design of unobtrusive disengagement, it is crucial to understand the engagement of children with games.

%Understand why kids play and what is engaging
%How to make kids satisfied with games - Kids' perspective 
%When and how does satisfaction happen?
%Is it about time? Is it about number of achievements?
%What are the internal and external regulation mechanisms?
%-What the kids want from the game - not much study here to understand regular games for kids.
%-How to re-engage the kids? Keep the kids interested in the game. 
%- Why do children want to come back to the game (i.e. why they re-engage)?

\subsection{Appreciating Disengagement as Part of Play}
Currently, disengaging from games is underappreciated by HCI games research, and we believe that it is a research opportunity to view it as a natural part of play. Reflecting the four phases of the engagement cycle proposed by~\citet{obrien2008}, a point of engagement, a period of engagement, disengagement, and finally, re-engagement, we align with their conceptualization of disenagement as a natural part of the engagement--disengagement--re-engagement cycle, which is spanned across two dimensions: the degree of users' agency and the span between positive and negative engagement. This definition of disengagement encompasses the users' goals, the meaning of usage, and the degree of control the users take over interacting with the system.. Thereby, it becomes possible to design for specific exit experiences, and to consider concerns around disengagement and excessive play from a positive perspective, shifting the focus to player empowerment, enabling them to re-gain agency over the time at which they (temporarily) end their engagement with a specific game.
%\citet{obrien2022} conceptualize disengagement 
%Disengagement as a design method will allow players to choose voluntarily to quit the game and stop playing at any time. 
 %~\citet{cemiloglu2022} examined digital addiction and summarized software-mediated countermeasures, implementing game design strategies closely resembling industry best practices summarized in section XXX, including features such as built-in limits on play time, reminders to take breaks, and rewards for disengaging~\cite{cemiloglu2022}. 
 Recently, \citet{stevens2021} examined the effectiveness of currently implemented design strategies in the context of overuse of digital games, and concluded that features that set limits on playtime or locked players out of the game received low support (65\% disapproval) among habitual and problem gamers \cite{stevens2021}. While not primarily examining disengagement, a study by ~\citet{tyack2020a} found that games that allow players to pause or save their progress and come back later, without losing their progress or rewards, can help players to disengage without feeling frustrated or stressed~\cite{tyack2020a}. Also, ~\citet{alharthi2018} point out that idle games, which reward players for waiting, can be an effective way to foster disengagement without harming the \ac{PX}~\cite{alharthi2018}. Similarly, \citet{davies2016} report that players who regarded their play session as completed (i.e., achieving their goals) when blocked from playing felt less frustrated than those who were cut out in the middle of a quest.
 This highlights the need for features that extend beyond extrinsic control, examining how to make it easier for players to quit games at their own volition, and to create games that have natural end points. This is in line with %Here,~\cite{cemiloglu2022} recommended designing games with a more open-ended structure rather than a fixed goal or endpoint, which may reduce the pressure to continue playing and facilitate disengagement~\cite{cemiloglu2022}. 
%The process of the engagement cycle contains four phases: a point of engagement, a period of engagement, disengagement, and finally, re-engagement\cite{obrien2008}. %Disengagement is one of the stages in the cycle of engagement\cite{obrien2008}. 
~\citet{obrien2008}, who theorizes that disengagement could be related to positive emotions such as feeling successful and satisfied when achieving a goal, which is something that we hope games research and game design can aspire to.

%Fostering disengagement can be done by providing clear goals, feedback, and autonomy, as per \ac{SDT}, which states players are more likely to engage in activities that support their psychological needs for autonomy, competence, and relatedness ~\cite{deci2012}. Satisfying basic psychological needs is one of the hallmarks of well-designed games. Satisfying these needs, even for a short period of time, is essential for the player to feel pleased \cite{allen2018}.



%The re-engagement stage happens when the disengaged player, for a long or short time, decided to come back to play the game. The previous experience of playing determines whether or not it was positive for the player to return \cite{obrien2008}.

%Understand the design space of disengagement elements
%Disengagement as a design method/paradigm for game design
%-Games as a routine for kids
%-What are the available disengagement methods?
%-How Tv shows and books for children do smoothly exist 
%-Can we apply these methods to game design 

%-What is the disengagement method that is good for each type of game?
%- How do the disengagement methods fit into the context/design/expectations of different types of games?
%- How do disengagement and re-engagement interact in the context of play?






%- How do families currently end play, what is difficult for children?
%- What game design strategies can be employed to facilitate disengagement?
%- Challenge: Want kids to return to play…
%- How can disengagement be fostered, rewarded without harming \ac{PX}?
%- Open question: What works for kids, what works for parents?

% OPPORTUNITY
% 1) Treat disengagement as part of design process for games
% 2) Learning from other forms of media


%\subsection{How do families currently end play, what is difficult for children?}
%A child cannot control themselves and know when to stop playing.  According to~\citet{wood2008}, children are often attached to video games because of the positive reinforcement they receive while playing~\cite{wood2008}. 
%Video games can provide children with a sense of accomplishment, control, and social interaction, which can be especially appealing to children who may lack these experiences in other areas of their lives. 
%Additionally, the study highlights that video games can be a form of escapism for children, allowing them to temporarily forget about problems or stressors in their lives.
%\citet{papadakis2022} illustrate that parents struggle to control their children's time spent on tablets. Unfortunately, they often lose this battle. 
%\citet{donati2021} found that setting restrictions on the time, place, and content of video gaming can prevent excessive gaming and reduce symptoms. However, the effectiveness of these rules is moderated by the degree of parent-child agreement \cite{donati2021}.~\citet{kahila2022} report that children experience outbursts of rage when their in-game experiences are interrupted during important events or when the game is going well\cite{kahila2022}. These interruptions are often caused by homework, household chores, or meal times. These disruptions can negatively impact the player's immersion, mood, and flow state. 
%Research has shown that young children have difficulty understanding the concept of time, particularly when it comes to differentiating between longer and shorter durations. For example, a study by Friedman (1978) found that children under the age of 6 have difficulty grasping the concept of time as a measurable duration and may not fully understand the difference between short and long durations \cite{friedman1978}.
%\paragraph{Research Oportunities}
%A common parenting technique used to end game time for children is setting clear time limits. This can be achieved by specifying a specific time of day or a certain duration of time for the child to spend playing games. Additionally, parents can use technology, such as parental controls, to limit the amount of time their children spend playing games. Additionally, parents can encourage alternative activities such as outdoor play, reading, or spending time with family, to balance out the time spent on games. Children may lack a developed sense of time, leading to difficulty in accurately gauging the amount of time spent on games and inability to effectively manage time without adult assistance. A common problem in ending game time for children is conflict between parents and child. This can lead to tension and disagreements if child resist the parent's attempts to limit their game time. It is important for parents to approach the situation with patience, understanding and find ways to compromise and communicate effectively with their child to establish clear boundaries. 

%- What game design strategies can be employed to facilitate disengagement?

%According to Cemiloglu et al. (2022), game design strategies such as designing games with built-in limits on play time, reminders to take breaks, and rewards for disengaging \cite{cemiloglu2022}. However, In-game lockout features, which enable players to set limits on playtime or lock themselves out of the game, received low support (65\% disapproval) among habitual and problem gamers, according to Stevens et al. (2021) \cite{stevens2021}. This suggests that these features may not be effective in addressing problematic gaming.  Also, Cemiloglu et al. (2022),  recommended designing games with a more open-ended structure, rather than a fixed goal or endpoint, which may reduce the pressure to continue playing and facilitate disengagement. \cite{cemiloglu2022}.
%===== 
%Providing clear goals and feedback, giving players autonomy, allowing players to pause or save progress, encouraging alternative activities, and providing players with different difficulty levels are some strategies that can be employed in game design to facilitate disengagement. These strategies can help players to disengage from the game without feeling frustrated or stressed and promote a healthy balance of screen time and other activities.

%- Challenge: Want kids to return to play…
%According to \citet{gutwin2016}, the peak-end effect can be used to re-engage players in casual games by designing game experiences that have memorable and intense peaks, and satisfying endings~\cite{gutwin2016}. To do this, designers can focus on creating challenging and engaging gameplay, and ensure that the game has a clear and satisfying conclusion. To re-engage players with a game,  according to Tyack and Mekler (2020), game designers can use elements such as autonomy, competence, and relatedness to enhance player engagement. This can be achieved by providing choices and control, clear goals and feedback, and social features in the game design. They also mention considering players' personal and situational factors in game design to enhance motivation to re-engage with the game \cite{tyack2020a}.
%========
%Providing rewards, allowing players to save progress, making sure the game is enjoyable, age-appropriate and engaging, encouraging breaks and balance screen time, incorporating social elements, providing new content and updates, and designing the game to be challenging but not frustrating are some strategies that can be employed to make kids want to play again after disengagement. Another approach is to offer players a sense of accomplishment, such as by providing clear goals and feedback or providing different difficulty levels to challenge players to improve their skills and progress through the game. It is also important to involve the child in the game design process and to incorporate their feedback to make the game more appealing to them.

%- How can disengagement be fostered, and rewarded without harming \ac{PX}?
%Fostering disengagement can be done by providing clear goals, feedback, and autonomy, as per \ac{SDT}, which states players are more likely to engage in activities that support their psychological needs for autonomy, competence, and relatedness ~\cite{deci2012}. In a study by ~\citet{tyack2020a} found that games that allow players to pause or save their progress and come back later, without losing their progress or rewards, can help players to disengage without feeling frustrated or stressed \cite{tyack2020a}. Also, ~\citet{alharthi2018} point that idle games, which reward players for waiting, can be an effective way to foster disengagement without harming the \ac{PX}\cite{alharthi2018}.
%====
%One approach could be to reward players for disengaging by providing bonuses or special items that can be unlocked only after a certain period of time has passed, this way, disengagement is not only encourage but also rewarded.



%- Open question: What works for kids, and what works for parents?


\section{Conclusion}
Within the HCI games research community, disengagement is an underresearched aspect of player \textbf{en}gagement with games, and is either framed as a result of poor game design, or examined in the context of problematic overuse of games. The latter also is a prominent perspective on children's disengagement from games, which is predominantly addressed through external means of regulating playing time (e.g., built-in time limits and parental mediation). In our position paper, we argue that this is a missed opportunity for game design to expand beyond restrictive practices. Here, a central question that remains for our research community is whether (the lack of) disengagement should be addressed through external regulation, or whether we can shift toward a perspective where games researchers and designers actively design for player disengagement, enabling players of all ages to establish healthy relationships with their favourite medium.

%In the context of games research, this issue is even more pronounced with disengagement predominantly being considered a negative event that needs to be avoided. Particularly in the context of children's engagement with digital games, this can be problematic as mechanics fostering engagement can simultaneously make volitional disengagement more challenging.

%, instead of being viewed as a natural part of a player's experience with a game.
%- Supporting disengagement: not the absence of dark patterns, but the presence of patterns that actively facilitate quitting play


%The implementation of strategies supporting disengagement is key in holistic game design. Game designers are obligated to refrain from using manipulative design elements, known as "dark patterns," and instead promote incorporating elements that actively support players to disengage from play. This necessitates the creation of user-friendly gameplay that allows players to stop the game without feeling compelled to continue. It is especially vital when designing games for children, as they are more susceptible to the addictive qualities of video games and require more support in managing their gaming habits. By fostering disengagement, game designers can help children mitigate the negative impacts associated with excessive gaming and encourage a healthier relationship with technology.








%%
%% The acknowledgments section is defined using the "acks" environment
%% (and NOT an unnumbered section). This ensures the proper
%% identification of the section in the article metadata, and the
%% consistent spelling of the heading.
\begin{acks}
% Funding acknowledgments
\end{acks}

%%
%% The next two lines define the bibliography style to be used, and
%% the bibliography file.
\bibliographystyle{ACM-Reference-Format}
\bibliography{references}



\end{document}
\endinput
%%
%% End of file `sample-acmsmall.tex'.
