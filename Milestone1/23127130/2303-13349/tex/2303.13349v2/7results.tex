\section{Results}\label{section:results}

\subsection{\pt-differential cross section of heavy-flavour hadron decay electrons in pp and \pPb collisions}
The \pt-differential production cross section of electrons from semileptonic decays of heavy-flavour hadrons at midrapidity
in pp collisions at $\sqrt{s} = 13$ TeV measured in the transverse momentum interval $\mbox{$0.2 < p_{\rm{T}} < 35$ GeV$/c$}$ is shown in Fig.~\ref{Fig:ppHFESpectra}. The statistical uncertainties are represented as vertical lines while the total systematic uncertainties are displayed as boxes. 
In the top left panel of Fig.~\ref{Fig:ppHFESpectra}, the cross sections measured with the TPC--TOF detectors and the two different data sets collected with different magnetic fields are plotted together with the spectra obtained using the TPC--EMCal detectors with MB triggered events, as well as with EMCal triggered events, EG1, and EG2. 
The ratios of the different analyses in the overlapping \pt~intervals are shown in the bottom left panel of Fig.~\ref{Fig:ppHFESpectra}.  
For $\mbox{0.5 $<$ \pt $<$ 4~\GeVc}$, the ratio of the result from the TPC--TOF analyses with $B = $ 0.5 T to the one obtained with $B = $ 0.2 T is displayed. In 3 $<$ \pt $<$ 4 \GeVc, the ratio of the cross section obtained from the TPC--TOF analysis to that obtained from the TPC--EMCal analysis is shown for MB triggered events. At higher \pt, namely 6 $<$ \pt $<$ 10 \GeVc (12 $<$ \pt $<$ 18 \GeVc), the ratio of the TPC--EMCal results for MB and EG2 (EG2 and EG1) triggered events is reported. All ratios are consistent with unity within statistical and systematic uncertainties, which demonstrates that the different analyses are in agreement with each other. The final cross section in the \pt intervals 0.2--0.5 GeV/$c$, 0.5--4 GeV$/c$, 4--6 GeV$/c$, 6--12 GeV$/c$, and 12--35 GeV$/c$ was obtained from the TPC--TOF low-$B$ field analysis, the TPC--TOF nominal-$B$ field analysis, and from the results obtained with the TPC--EMCal detectors using MB, EG2 and EG1 triggered events, respectively. In this way, for each \pt range, the measurement with the smallest total uncertainty (quadratic sum of statistical and systematic uncertainty) is used.


The $p_{\rm{T}}$-differential cross section measurement was compared with FONLL~\cite{Cacciari:2012ny} and GM-VFNS~\cite{Bolzoni:2012kx} pQCD calculations
\footnote{${\rm Central~values:}~~{\rm FONLL:}~\mu_{\rm F}~=~\mu_{\rm R}~=~\sqrt{m_{\rm Q}^2+p_{\rm T}^2},~m_{\rm b}~=~4.75~{\rm GeV},~m_{\rm c}~=~1.5~{\rm GeV};~~{\rm GM-VFNS:}~\mu_{\rm F}~=~0.49~\mu_{\rm R},\\\mu_{\rm R}~=~\sqrt{4m_{\rm Q}^2+p_{\rm T}^2},~m_{\rm b}~=~4.5~{\rm GeV},~m_{\rm c}~=~1.5~{\rm GeV};~~{\rm where}~\mu_{\rm R}~=~{\rm renormalization~scale,}~\mu_{\rm F}~=~{\rm factorisation~scale}$},
as shown in the right panel of Fig.~\ref{Fig:ppHFESpectra}.
 The uncertainties of the FONLL calculations reflect different choices for the charm- and beauty-quark masses, and for the factorisation and renormalisation scales as well as the uncertainty on the set of parton distribution functions (PDF) (CTEQ6.6~\cite{Nadolsky:2008zw}). The FONLL calculations describe the measurements within the uncertainties, although the theoretical uncertainties are large, up to a factor of two. The data are found to be close to the upper edge of the FONLL prediction, which can be clearly seen in the right bottom panel of Fig.~\ref{Fig:ppHFESpectra}, where the ratio of the data points to the FONLL calculations is shown. 
Similar observations were made for the measurements of electrons from heavy-flavour hadron decays in pp collisions at lower energies at the LHC~\cite{Abelev:2014gla, ALICE:2012mzy, Acharya:2018upq, Acharya:2019mom} and at RHIC~\cite{STAR:2011bqq, PHENIX:2006tli}. The measurement of the cross section of D mesons is also consistent with upper bound of FONLL pQCD calculations in pp collisions at LHC~\cite{Abelev:2012vra,ALICE:2017olh,ALICE:2019nxm,Acharya:2021cqv,ATLAS:2015igt,LHCb:2015swx, LHCb:2016ikn} and RHIC~\cite{STAR:2012nbd}, as well as in $\rm{p \bar{p}}$ collisions at Tevatron energies~\cite{CDF:2003vmf}. 
The FONLL calculations use fragmentation functions tuned on e$^{+}$e$^{-}$ data and assume that all charm quarks fragment only into  D$^+$ and D$^0$ mesons (and their antiparticles). 
Recent measurements of charm-baryon production at midrapidity in pp and \pPb collisions from ALICE show a baryon-to-meson ratio significantly higher than that in e$^{+}$e$^{-}$ collisions, suggesting that the fragmentation of charm quark is not universal across different collision systems~\cite{ALICE:2021dhb,ALICE:2021npz}.
As a consequence, calculations taking properly into account the latest open-cham baryon measurements at midrapidity to constrain the charm fragmentation are expected to predict a smaller yield of heavy-flavour hadron decays by about 9\% compared to the FONLL spectrum.
The largest source of uncertainties in the GM-VFNS prediction is due to scale variation, and hence PDF related uncertainties and variations of the bottom and charm mass are not considered. The GM-VFNS framework includes leptoproduction from the following three steps: beauty quark to beauty hadrons (b $\rightarrow$ B), transition from  beauty quark to  charm hadrons (b $\rightarrow$ B $\rightarrow$ D), and charm quark to charm hadrons (c $\rightarrow$ D). The GM-VFNS calculations describe the data within the uncertainties for \pt greater than 5 \GeVc, but largely underestimate the cross section for lower \pt, up to a factor of five at 1 \GeVc, as seen in the right middle panel of Fig.~\ref{Fig:ppHFESpectra}. Similar observations were reported for the non-prompt D meson measurements at $\sqrt{s}~=~5.02~{\rm TeV}$~\cite{Acharya:2021cqv}. For prompt D mesons at $\sqrt{s}~=~5.02~{\rm TeV}$, however, the GM-VFNS predictions describe the cross section within the uncertainties~\cite{Acharya:2021cqv}. Electrons from heavy-flavour hadron decays are dominated by semileptonic decays of beauty hadrons for \pt~$>5$ ~GeV$/c$~\cite{Abelev:2012sca, ALICE:2014aev}. Therefore, the cross section measured up to 35 GeV$/c$ can provide important information to beauty hadron production. 

\begin{figure}[!ht]
\centering

\includegraphics[width=0.48\linewidth]{figures/Results/HFE_ppNormalB/HFE_Comparison_1_CR1.pdf}
\includegraphics[width=0.48\linewidth]{figures/Results/HFE_ppNormalB/Data_GMVFNS_FONLL_1.pdf}
\caption{Left, top: $p_{\rm T}$-differential cross section of electrons from heavy-flavour hadron decays in pp collisions at $\sqrt{s} =$ 13 TeV measured at midrapidity with different detectors and data sets. Left, bottom: Ratios of the different measurements in the overlapping \pt intervals. Right: $p_{\rm{T}}$-differential cross section compared with Fixed Order with Next-to-Leading-Log resummation (FONLL)~\cite{Cacciari:2012ny} and General-mass-variable-flavour-number-Scheme (GM-VFNS)~\cite{Bolzoni_2014} predictions and its ratios with respect to FONLL and GM-VFNS central values in the two lower panels. Vertical bars and boxes denote statistical and systematical uncertainties, respectively.}        
\label{Fig:ppHFESpectra}
\end{figure}

\begin{figure}[!ht]
\centering
\includegraphics[width=0.5\linewidth]{figures/Results/HFE_pPb/HFE_InvariantCrossSection_pPb_withNewRuns_CR1_v1.pdf}
\caption{Top: $p_{\rm T}$-differential cross section of electrons from heavy-flavour hadron decays in \pPb  collisions at \sqrtsNN $= 8.16~\rm TeV$ measured at midrapidity with different detectors. Bottom: Ratios of the different measurements in the overlapping \pt intervals.}        
\label{Fig:pPbHFESpectra}
\end{figure}



 The $p_{\rm{T}}$-differential production cross section of electrons from semileptonic heavy-flavour hadron decays at midrapidity in \pPb collisions at $\sqrt{s_{\rm{NN}}} = 8.16$ TeV measured in the transverse momentum interval $0.5 < p_{\rm{T}} < 26$~GeV$/c$ is shown in Fig.~\ref{Fig:pPbHFESpectra}. In the upper panel of Fig.~\ref{Fig:pPbHFESpectra}, the cross sections measured with the TPC--TOF detectors are plotted together with the measurements obtained using the TPC--EMCal detectors with MB and EMCal EG2 and EG1  triggered events. On the bottom left panel of Fig.~\ref{Fig:pPbHFESpectra}, the ratios of the cross sections obtained from the different measurements are calculated in the overlapping \pt intervals. For 3 $<$ \pt $<$ 5 \GeVc, the ratio of the result obtained from the TPC--TOF analysis with respect to that from the TPC--EMCal is shown for  MB triggered events. For 6 $<$ \pt $<$ 10 \GeVc (12 $<$ \pt $<$ 14 \GeVc) the ratio of the TPC--EMCal results obtained with MB and EG2   triggered events (EG2 and EG1) is reported. All ratios are consistent with unity within statistical and systematic uncertainties. The same strategy as in pp collisions was used to get the final cross section in  \pPb collisions. The final cross section in the \pt intervals 0.5--4 GeV$/c$, 4--6 GeV$/c$, 6--9 GeV$/c$, and 9--26 GeV$/c$ was obtained from the TPC--TOF nominal-$B$ field analysis and from the results using the TPC--EMCal detectors with MB, EG2, and EG1 triggered events, respectively.


\subsection{Nuclear modification factor of electrons from heavy-flavour hadron decays in \pPb collisions}


The nuclear modification factor of electrons from heavy-flavour hadron decays, $R_{\rm{pPb}}$, is defined as
\begin{equation}
    R_{\rm{pPb}}(p_{\rm{T}},\it{y}) = \frac{\rm 1}{ A} \frac{{\rm {d}}^2 \sigma_{\rm{pPb}}/{\rm{d}} p_{\rm{T}}\rm{d}\it{y}}{{\rm{d}}^2\sigma_{\rm{pp}}/{\rm{d}} p_{\rm{T}}\rm{d}\it{y}},
\end{equation}
where  ${\rm {d}}^2 \sigma_{\rm{pPb}}/{\rm{d}} p_{\rm{T}}\rm{d}\it{y}$ is the cross section of electrons from heavy-flavour hadron decays measured in {$\mbox{\pPb}$} collisions at \sqrtsNN $= 8.16~\rm TeV$ and ${\rm {d}}^2 \sigma_{\rm{pp}}/{\rm{d}} p_{\rm{T}}\rm{d}\it{y}$ is the cross section of electrons from heavy-flavour hadron decays in pp collisions at the same centre-of-mass energy, scaled with the number of nucleons ($A$) in the lead ion.
The reference cross section in pp collisions was obtained using the measurement at \sqrts $= 13~\rm TeV$, presented here. The cross section at \sqrts $= 13~\rm TeV$ was scaled to \sqrts $= 8.16~\rm TeV$ using pQCD calculations. The $p_{\rm{T}}$-dependent scaling factor was obtained by calculating the ratio of the production cross sections of electrons from heavy-flavour hadron decays from FONLL calculations~\cite{Cacciari:2012ny} at $\sqrt{s}= 8.16$ TeV to $\sqrt{s}=13$ TeV. The systematic uncertainty on the pp reference includes the systematic uncertainties on the measured cross section at $\sqrt{s}=13$ TeV, which was described above, and the ones on the $p_{\rm{T}}$-dependent scaling factor. The uncertainty on the scaling factor ranges between 11\% and 1\% going from \pt $= 0.2$ \GeVc to \pt $= 26$ \GeVc. This includes the uncertainties on the PDFs, quark masses, and factorisation and renormalisation scales, as described in Ref.~\citenum{Averbeck:2011ga}. The two contributions were added in quadrature leading to a total systematic uncertainty of 5-15\%, depending on $p_{\rm{T}}$. In addition, a global normalisation systematic uncertainty of 2.3\% from the pp analysis at $\sqrt{s}=13$ TeV was also considered.
The $p_{\rm{T}}$-differential cross section of electrons from heavy-flavour hadron decays in pp collisions at \sqrts $= 13~\rm TeV$ scaled to \sqrts $= 8.16~\rm TeV$ using the aforementioned procedure is shown together with the $p_{\rm{T}}$-differential cross section of electrons from heavy-flavour hadron decays in \pPb collisions at \sqrtsNN $= 8.16~\rm TeV$ in Fig.~\ref{Fig:HFESpectraComparison}.

The nuclear modification factor of electrons from heavy-flavour hadron decays as a function of transverse momentum at $\sqrt{s_{\rm{NN}}} = 8.16$ TeV is presented in Fig.~\ref{Fig:RpPb}. The statistical and systematic uncertainties of the spectra in p–Pb and pp collisions were propagated as uncorrelated. The normalisation uncertainties are shown as a solid box at $R_{\rm{pPb}} = 1$. The $R_{\rm{pPb}}$ is consistent with unity within statistical and systematic uncertainties over the whole $p_{\rm{T}}$ range of the
measurement. Modifications of the cross section of electrons from
heavy-flavour hadron decays in \pPb collisions due to different cold nuclear matter effects, are small compared to the current uncertainties of the measurement in the probed $p_{\rm{T}}$ range.
The sample of electrons from heavy-flavour hadron decays is dominated by beauty-hadron decays for  $\mbox{$p_{\rm{T}} > 5$ GeV$/c$ }$ ~\cite{Abelev:2012sca, Abelev:2014hla}. The $R_{\rm{pPb}}$ was fitted with a constant function above 5 \GeVc and the value was $\rm 0.95 \pm 0.02(stat.) \pm 0.13(sys.)$, thus consistent with unity within 13\%. The $R_{\rm{pPb}}$ of unity indicates that the beauty production is not modified in \pPb collisions within the kinematic range of this measurement, which is also consistent with the measurement of $R_{\rm{pPb}}$ of beauty-decay electrons up to $p_{\rm{T}} = 8$ GeV$/c$ at $\sqrt{s_{\rm{NN}}} = 5.02$ TeV~\cite{Adam:2016wyz}. In the right panel of Fig.~\ref{Fig:RpPb}, the $R_{\rm{pPb}}$ at $\sqrt{s_{\rm{NN}}} = 8.16$ TeV is compared with that at $\sqrt{s_{\rm{NN}}} = 5.02$ TeV and different theoretical models provided for $\sqrt{s_{\rm{NN}}} = 5.02$ TeV ~\cite{Acharya:2019hao}. The $R_{\rm{pPb}}$ is observed to be independent of the centre-of-mass energy. The data disfavour the enhancement trend at low \pt predicted by the model calculations which are based on incoherent multiple scatterings ~\cite{KANG201523}. Model predictions which are based on coherent multiple scattering and energy loss in the CNM, pQCD calculations using FONLL 
framework and EPS09NLO for the nuclear modification of the PDF, as well as calculations which assume the formation of a hydrodynamical expanding medium in \pPb collisions at $\sqrt{s_{\rm{NN}}} = 5.02$~TeV within the Blast wave framework predict an $R_{\rm{pPb}}$ close to unity and are in agreement with the measurements.

 

\begin{figure}[!ht]
\centering
\includegraphics[width=0.5\linewidth]{figures/Results/HFE_pPb/ppScaledReference_wNewRuns_CR1.pdf}
\caption{$p_{\rm{T}}$-differential cross section of electrons from heavy-flavour hadron decays measured in \pPb collisions at \sqrtsNN $= 8.16~\rm TeV$ compared with the pp reference at the same centre-of-mass energy obtained from the measurement in pp collisions at \sqrts $= 13~\rm TeV$ scaled to \sqrts $= 8.16~\rm TeV$.}        
\label{Fig:HFESpectraComparison}
\end{figure}

\begin{figure}[!ht]
      \begin{center}
      \includegraphics[width=0.48\linewidth]{figures/Results/HFE_pPb/RpPb_Updated_wNewRuns_CR1_v1.pdf}
      \includegraphics[width=0.48\linewidth]{figures/Results/HFE_pPb/RpPb_8TeV_ComparedWith5TeV_wNewRuns_CR1_v1.pdf}
      \end{center}
\caption{ The nuclear modification factor $R_{\rm{pPb}}$ of electrons from heavy-flavour hadron decays in \pPb collisions at
\sqrtsNN $= 8.16~\rm TeV$ (left) compared with that at \sqrtsNN $= 5.02~\rm TeV$ and theoretical models at \sqrtsNN $= 5.02~\rm TeV$ (right) ~\cite{Acharya:2019hao}.}    
\label{Fig:RpPb}
\end{figure}

\subsection[Self-normalised yield of electrons from heavy-flavour hadron decays vs. normalised multiplicity]{Self-normalised yield of electrons from heavy-flavour hadron decays vs. normalised multiplicity in pp and \pPb collisions}
The self-normalised yield of electrons from heavy-flavour hadron decays as a function of the self-normalised charged-particle pseudorapidity density  at midrapidity, i.e., ${\rm d}^{2} {N}/{\rm d}\pt {\rm d}{y} / \langle {\rm d}^{2}{N}/{\rm d}\pt{\rm d}{y}\rangle_{{\rm INEL>0}}$ vs. $\dnchdeta/\left<\dnchdeta\right>$, in pp collisions at $\sqrt{s} =$ 13 TeV is presented in Fig.~\ref{Fig:SelfnormalisedYield}.  The results are self-normalised to the INEL $>$ 0 event class. The measurements were performed in five \pt intervals from 0.5 to 30 GeV$/c$. 
The dashed line shown in the figure is a linear function with a slope of unity. The available data samples allow us to examine events with a multiplicity more than six times larger than the average multiplicity in pp collisions. The self-normalised yield of electrons from heavy-flavour hadron decays grows faster than linear with the self-normalised multiplicity. The measurement in intervals of \pt shows that this increase is more pronounced for high-\pt electrons. The yield of heavy-flavour decay electrons increases by approximately a factor of nine with respect to its multiplicity-integrated value for the lowest measured \pt~interval ($0.5 < \pt < 1.5$ \GeVc) and a factor of 29 for the highest measured \pt interval $\mbox{$(20 < \pt < 35\ \GeVc)$}$ for multiplicities of six times the average multiplicity. 

\begin{figure}[!ht]
\centering
\includegraphics[width=0.6\linewidth]{figures/Results/HFE_ppNormalB/SNYNo_Uncert_CR1_2.pdf}
\caption{Self-normalised yield of electrons from heavy-flavour hadron decays as a function of normalised charged-particle pseudorapidity density at midrapidity computed in \pp collisions at \sqrts $= 13~\rm TeV$ in different \pt intervals.}
\label{Fig:SelfnormalisedYield}
\end{figure}


\begin{figure}[!h]
%\centering
\includegraphics[width=0.48\linewidth]{figures/Results/HFE_ppNormalB/SNY_Pt_Ratio_0_CR1_2.pdf}
\includegraphics[width=0.48\linewidth]{figures/Results/HFE_ppNormalB/SNY_HFE_Fit_Diagonal_Linear_CR2.pdf}
\caption{ Ratio of the  self-normalised yields
in different \pt intervals with respect to that in the $6 < \pt < 12~{\rm GeV}/c$ interval (left) and double ratio of the self-normalised yields of  electrons to the self-normalised multiplicity (right) in pp collisions at $\sqrts=13$ TeV for three \pt ranges.}
\label{Fig:SNY_Ratio_pp}
\end{figure}

In the left panel of Fig.~\ref{Fig:SNY_Ratio_pp}, the ratios of the self-normalised yields of electrons from heavy-flavour hadron decays in various \pt intervals with respect to the one measured in the $6 < \pt < 12$ GeV$/c$ interval are shown. The yield of lower-$\pt$ electrons is higher in low  multiplicity events, while it decreases in higher multiplicity events. 
An opposite trend is observed for electrons at higher \pt, where the yield is lower in low  multiplicity events and increases at higher multiplicities.
The increase of the slope with \pt is influenced by the momentum dependence of jet fragmentation affecting the measured multiplicity at midrapidity, and the momentum dependence of the fraction of electrons from charm and beauty hadron decays. The relative fraction of electrons from beauty hadron decays increases with $\pt$ and becomes the main source of heavy-flavour hadron decay electrons at high \pt ($\pt > 5$ GeV$/c$)~\cite{Adam:2015ota,ALICE:2020msa,Weber:2018ddv}.




In the right panel of Fig.~\ref{Fig:SNY_Ratio_pp}, the double ratio of the self-normalised  electron yield to the self-normalised multiplicity in pp collisions is presented. The double ratio is observed to increase with multiplicity. The increase is weaker for low-\pt electrons than for high-\pt electrons. A linear function was used to fit the multiplicity dependence of the double ratio, 
which was found to describe the data reasonably well for all \pt intervals. This indicates that in the measured \pt range the yield grows approximately with the square of the multiplicity with a slope increasing with \pt.


\begin{figure}[!h]
%\centering
\includegraphics[width=0.5\linewidth]{figures/Results/HFE_ppNormalB/SNY_PYTHIA_Default_CR1_1.pdf}
\includegraphics[width=0.5\linewidth]{figures/Results/HFE_ppNormalB/SNY_PYTHIA_CR2_CR1_1.pdf}
\caption{Comparison of the self-normalised yield of electrons from heavy-flavour hadron decays as a function of multiplicity measured in \pp collisions at \sqrts $= 13~\rm TeV$ for different \pt intervals with PYTHIA 8.2 Monash tune (left) and PYTHIA 8.2 with CR mode 2 (right). The width of the band is the statistical uncertainty from PYTHIA simulations. The bottom panel shows the ratio of data with respect to the MC predictions. The vertical bars correspond to the propagated statistical error from the data and the MC predictions, and the boxes correspond to systematical uncertainties from the data. }
     \label{Fig:SelfnormalisedYield_pp_PYTHIA}
\end{figure}

The self-normalised yield of electrons from heavy-flavour hadron decays is compared in Fig.~\ref{Fig:SelfnormalisedYield_pp_PYTHIA} with PYTHIA 8.2 simulations using different tunes. 
In the PYTHIA~8.2 framework, multiparton interactions (MPI) and the colour reconnection (CR) mechanism are implemented, which reproduce the charged-particle multiplicity distribution measured at the LHC~\cite{Adam:2016mkz, ATL-PHYS-PUB-2017-008}. These mechanisms are important in order to describe the stronger than linear increase of charm and beauty production with multiplicity as demonstrated in ~\cite{Adam:2015ota}. 
The charged-particle multiplicity also includes particles directly produced in the same hard partonic scattering process in which the heavy quark is created, making them strongly related. These dependencies come from the initial- and final-state radiations, decays of heavy-flavour hadrons, and charged particles produced in the jet fragmentation and are known as auto-correlation effects.
A study of the self-normalised yield of heavy-flavour particles using the PYTHIA 8.2 generator shows that the stronger than linear increase of the yield of heavy-flavour particles is mainly driven by auto-correlation effects. In the absence of auto-correlation effects the increase of the yield of particles produced in hard scattering processes is weaker than linear for multiplicities exceeding about three times the mean multiplicity~\cite{Weber:2018ddv}.
In PYTHIA 8.2, the \pt dependence of the increase of the self-normalised yield with multiplicity is also due to auto-correlation effects introduced by the parton fragmentation because high momentum partons are accompanied by a larger number of fragments which contribute to the multiplicity. In the case of electrons from heavy-flavour hadron decays, the high-\pt part of the spectra is dominated by beauty decay electrons, whose yield was demonstrated to have a more pronounced increase with multiplicity due to the larger jet activity~\cite{Weber:2018ddv}.
In the left panel of Fig.~\ref{Fig:SelfnormalisedYield_pp_PYTHIA}, the measured self-normalised yield of electrons from heavy-flavour hadron decays is compared to calculations with the PYTHIA 8.2 Monash tune that describe the overall trend in data, but the slope is overestimated at high \pt.
In the right panel of Fig.~\ref{Fig:SelfnormalisedYield_pp_PYTHIA}, an improved tune which includes string formation beyond the leading-colour approximation i.e. PYTHIA 8.2 with  CR mode 2~\cite{Sjostrand:2014zea, Christiansen:2015yqa}, is shown to reproduce the \pt dependence,  however the slope is underestimated at high \pt.



\begin{figure}[!h]
\centering
\includegraphics[width=0.5\linewidth]{figures/Results/HFE_ppNormalB/SNY_EPOS_CR1_1.png}
\caption{Comparison of self-normalised yield of electrons from heavy-flavour hadron decays as a function of multiplicity measured in \pp collisions at \sqrts $= 13~\rm TeV$ for different \pt intervals with EPOS 3 hydro calculations. The width of the band is the statistical uncertainty from EPOS simulations. The bottom panel shows the ratio of data with respect to the MC predictions. The vertical bars correspond to the propagated statistical error from the data and the MC predictions, and the boxes correspond to systematical uncertainties from the data. The ratio for the lowest multiplicity point  for 15 $<$ \pt $<$ 30 \GeVc~(not shown in the figure) is 13$\pm$18.}
     \label{Fig:SelfnormalisedYield_pp_EPOS}
\end{figure}

Calculations with the EPOS 3 event generator~\cite{Werner:2013tya} are able to reproduce the data well, except for the highest measured \pt interval, as can be seen in Fig.~\ref{Fig:SelfnormalisedYield_pp_EPOS}. In the EPOS 3 model, the elementary scattering objects are pomerons, which are exchanged between the partons participating in the collision. The pomerons consist of a hard pQCD scattering vertex, accompanied by initial (space-like) and final (time-like) state parton emission.
The production of a hard probe is more likely from events with hard pomeron exchanges. 
This implies that for a given charged-particle multiplicity the presence of heavy-flavour hadrons favours events with fewer but harder pomerons, which leads to a stronger than linear increase of heavy-flavour production with charged-particle multiplicity. The increase also gets stronger with the increasing \pt, which, as discussed above for the case of PYTHIA 8.2 simulations, is related to the hardness of the partonic scattering and the accompanying  jet activity in the event. The subsequent hydrodynamic evolution of the system then amplifies the increase because the charged-particle multiplicity is reduced by the hydrodynamic expansion, in contrast to the heavy-flavour production.  The charged-particle multiplicity is reduced because part of the available energy goes into flow rather than particle production~\cite{Werner:2016nsq}. 


\begin{figure}[!h]
\centering
\includegraphics[width=0.45\linewidth]{figures/Results/HFE_ppNormalB/SNY_JPsi_0_CR1_2.pdf}
\includegraphics[width=0.45\linewidth]{figures/Results/HFE_ppNormalB/SNY_ChargedParticles_High_6_10_0_CR1_2.pdf}
\includegraphics[width=0.45\linewidth]{figures/Results/HFE_ppNormalB/SNY_Dmeson_0_CR1_2.pdf}
\includegraphics[width=0.45\linewidth]{figures/Results/HFE_ppNormalB/SNY_StrangeParticle_0_CR1_2.pdf}
\caption{Comparison of the self-normalised yield of electrons from heavy-flavour hadron decays measured in \pp collisions at \sqrts $= 13~\rm TeV$  with the self-normalised yields of J/$\psi$ in \pp collisions at \sqrts $= 13~\rm TeV$ (top left), charged particles in \pp collisions at \sqrts $= 13~\rm TeV$ (top right), D mesons in \pp collisions at \sqrts $= 7~\rm TeV$ (bottom left) and strange particles in \pp collisions at \sqrts $= 13~\rm TeV$ (bottom right), in comparable \pt bins.}
\label{Fig:SNY_CompOtherPart_pp}
\end{figure}


The trend of the self-normalised yield of electrons in pp collisions as a function of self-normalised multiplicity is compared in Fig~\ref{Fig:SNY_CompOtherPart_pp} with the self-normalised yield of other particles measured by the ALICE Collaboration, namely J/$\psi$~\cite{ALICE:2020msa}, charged particles~\cite{Acharya:2019mzb}, strange hadrons~\cite{Acharya:2019kyh} in pp collisions at $\sqrt{s}$ $= 13~\rm TeV$, and D mesons~\cite{Adam:2015ota} in pp collisions at $\sqrt{s}$ $= 7~\rm TeV$. The self-normalised yields for strange hadrons were calculated using the multiplicity-dependent cross section measurements reported in~\cite{Acharya:2019kyh}. These self-normalised yields allow a direct comparison of multiplicity-dependent production of different particle species, with the advantage that the charged-particle pseudorapidity density is measured using the same detector and procedure. 
The \pt ranges of electrons are selected to be similar to the measured \pt range of the compared particles, with a caveat that the \pt interval of electron parents (heavy-flavour hadrons) is considerably broader and shifted towards higher \pt values compared to the
one of the electrons. The slope of the increase of the self-normalised yield of electrons from heavy-flavour hadron decays as a function of self-normalised multiplicity at midrapidity is similar to that measured for J/$\psi$, charged particles, strange mesons, and D mesons in similar \pt ranges. 
At high and intermediate \pt, the production of hadrons is dominated by hard partonic scattering processes, independent of the particle species, accompanied by jet activity in the event. For heavy-flavour particles this is also true at low \pt due to the large charm and beauty quark masses. As it was discussed above, in PYTHIA 8.2, the particle production associated with jet activity leads to strong auto-correlation effects, which give rise to the observed stronger than linear increase of particle yields, making the self-normalised yield of the different particles reported here compatible with each other.

The self-normalised yield of electrons from heavy-flavour hadron decays 
as a function of the self-normalised charged-particle pseudorapidity density 
  for \pPb collisions at $\mbox{\sqrtsNN $= 8.16~{\rm TeV}$}$ is presented in Fig~\ref{Fig:SelfnormalisedYield_pPb}. The results are self-normalised to the INEL$>0$ event class, similarly to pp collisions. The dashed line is a linear function with a slope of unity as shown in the figure. The measurements were performed in five $p_{\rm T}$ intervals from 0.5 \GeVc to 26 \GeVc. 
Events with multiplicity more than four times larger than the average multiplicity in \pPb collisions are studied. 
The self-normalised yield of electrons from heavy-flavour hadron decays grows faster than linear with the self-normalised multiplicity. The measurements in $\pt$ intervals show no $\pt$ dependence within the uncertainties of the measurement. The yield increase is approximately a factor of seven for multiplicities four times larger than the average multiplicity.

\begin{figure}[!ht]
\centering
\includegraphics[width=0.6\linewidth]{figures/Results/HFE_pPb/HFESelfNormalisedYield_pPb8TeV_CR2.pdf}

\caption{Self-normalised yield of electrons from heavy-flavour hadron decays as a function of self-normalised charged-particle pseudorapidity density at midrapidity measured in \pPb collisions at \sqrtsNN $= 8.16~\rm TeV$ in different \pt intervals. The position of the points on the $x$-axis are shifted horizontally by $\rm \delta x$  to improve the visibility.}
\label{Fig:SelfnormalisedYield_pPb}
\end{figure}

In the left panel of Fig.~\ref{Fig:SNY_Ratio_pPb}, the ratios of the self-normalised yield of electrons from heavy-flavour hadron decays 
in various \pt intervals with respect to the one measured in the 3 $ < \pt < $ 6 \GeVc interval are shown. Contrary to the pp collision case,  within the uncertainties no \pt dependence is observed. The right panel of Fig.~\ref{Fig:SNY_Ratio_pPb} shows the double ratio of the self-normalised heavy-flavour hadron decay electron yield to the self-normalised multiplicity. The double ratio increases with multiplicity, with no dependence on \pt. The double ratio was fitted with a linear function, which reasonably describes the data for all \pt intervals. This indicates that in the measured \pt range the yield increases approximately with the square of the multiplicity with a similar coefficient for all \pt intervals.

\begin{figure}[!h]
%\centering
\includegraphics[width=0.48\linewidth]{figures/Results/HFE_pPb/pT_Ratio_pPb_8TeV_CR1_v1.pdf}
\includegraphics[width=0.48\linewidth]{figures/Results/HFE_pPb/Double_Ratio_pPb_8TeV_CR2.pdf}

\caption{Ratio of the  self-normalised yield
in different \pt intervals with respect to that in the $3 < \pt < 6~{\rm GeV}/c$ interval (left). Double ratio of the self-normalised yield of heavy-flavour hadron decay electrons to the self-normalised multiplicity in \pPb collisions at \sqrtsNN $= 8.16~\rm TeV$ in three \pt ranges (right).}
\label{Fig:SNY_Ratio_pPb}
\end{figure}

Though the self-normalised yields of electrons from heavy-flavour hadron decays  in pp and \pPb collisions show similar features in their increase with multiplicity, a quantitative comparison of the measurements between the two systems is not straightforward. In pp collisions, a high multiplicity event arises mostly from hard events, with multiparton interactions and jets fragmenting in multiple hadrons.  In \pPb collisions, the multiplicity dependence of heavy-flavour production is also driven by the presence of multiple binary nucleon--nucleon interactions, which make the contribution from possible auto-correlation effects smaller in such collisions. In \pPb collisions, an event with a high multiplicity value similar to those in pp collisions can come from the superposition of a few soft nucleon--nucleon collisions. Therefore, for similar multiplicity, the hardness of the event is not the same in the two systems.
 

The self-normalised yield of electrons from heavy-flavour hadron decays is compared in Fig.~\ref{Fig:SNYComparisonWithEPOS_pPb8TeV} with EPOS 2.592 simulations ~\cite{Werner:2013tya, Werner:2010aa}. The measurements in two \pt intervals are compared with the EPOS model without the hydrodynamic component, as provided by the authors. The EPOS model shows no \pt dependence similar to the observations in the data, but underpredicts the data at high multiplicity, showing an almost linear increase. 

\begin{figure}[!h]
\centering
\includegraphics[width=0.55\linewidth]{figures/Results/HFE_pPb/SelfNormaliseYieldHFEwEPOS_CR1_v2.pdf}
\caption{ Self-normalised yields of electrons from heavy-flavour hadron decays as a function of self-normalised charged-particle pseudorapidity density at midrapidity measured in \pPb collisions at \sqrtsNN $= 8.16~\rm TeV$ compared with EPOS 2.592 without-hydrodynamics in two \pt intervals 0.5 $< \pt <$ 3 \GeVc and 3 $< \pt <$ 6 \GeVc. The width of the band is the statistical uncertainty from EPOS simulations. The bottom panel shows the ratio of data with respect to the MC predictions. The vertical bars correspond to the propagated statistical uncertainties from the data and the MC predictions, and the boxes correspond to systematical uncertainties from the data.}
\label{Fig:SNYComparisonWithEPOS_pPb8TeV}
\end{figure}

The self-normalised electron yields in \pPb collisions in different \pt ranges are also compared with the normalised yields of D mesons~\cite{Adam:2016mkz} in \pPb collisions at \sqrtsNN $= 5.02~\rm TeV$ in Fig.~\ref{Fig:SNP_CompOtherPart_pPb}. Similar to the observation in pp collisions, the self-normalised yield of electrons from heavy-flavour hadron decays as a function of the self-normalised multiplicity shows a trend compatible with the one of D mesons. Also the multiplicity dependence of D meson yields in \pPb collisions does not show a \pt dependence, which gives a hint that the production mechanisms of charm and beauty as a function of the multiplicity in \pPb collisions are similar.

\begin{figure}[!ht]
\centering
\includegraphics[width=0.55\linewidth]{figures/Results/HFE_pPb/HFESelfNormaliseCompDmeson_CR2.pdf}
\caption{Self-normalised yields of electrons from heavy-flavour hadron decays measured in \pPb collisions at \sqrtsNN $= 8.16~\rm TeV$ for different \pt intervals compared with self-normalised yields of D mesons in \pPb collisions at \sqrtsNN $= 5.02~\rm TeV$. The position of the points for \pPb collisions at \sqrtsNN $= 5.02~\rm TeV$ on the $x$-axis are shifted horizontally by $\rm \delta x$ to improve the visibility.}
      \label{Fig:SNP_CompOtherPart_pPb}
\end{figure}





