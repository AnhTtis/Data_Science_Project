\section{Analysis overview}\label{section:heavyflavour}
Measurements of electrons from heavy-flavour hadron decays were  obtained by selecting an inclusive electron sample and subtracting electrons which do not originate from heavy-flavour hadron decays. The measurements were performed by identifying electrons using the TPC and TOF detectors at low \pt ($p_{\rm T} < 4~{\rm GeV}/{c}$) and the TPC and EMCal detectors at higher \pt (\pt $>$ 3 \GeVc) offering the largest \pt reach. In particular, this ensures that the systematic uncertainties and the hadron contamination are small over the whole transverse momentum range. In the interval $3 < p_{\rm{T}} < 4$ \GeVc, where the heavy-flavour decay electron production was measured with both techniques, the TPC--TOF analysis was used for the final results, while the TPC--EMCal analysis was utilised as a consistency check.
This choice was motivated by the precision of the measurements based on the statistical and systematic uncertainties, as will be further discussed in Sec.~\ref{section:results}. Throughout the article, the term ‘electron’ is used for electrons and positrons. 

\subsection{Electron identification}\label{subsec:elecId}
Reconstructed tracks were selected based on the criteria listed in Table~\ref{Table:InclusiveTrackSelection}, which are similar to those used in the analysis described in~\cite{Acharya:2019hao, Acharya:2019mom}. These requirements were applied depending on the data sample as well as the transverse momentum region of the analysis.
The rapidity ranges used in the nominal-$B$ field TPC--TOF analysis and the TPC--EMCal analysis were limited to $|y|<$ 0.8 and $|y|<$ 0.6, respectively, to avoid the edges of the detectors, where the systematic uncertainties related to the particle identification increase. In the low-$B$ field TPC--TOF analyses for pp collisions, the rapidity interval was restricted to $|y|<$ 0.5, to ensure a stable estimation of the photonic electron background (Sec.~\ref{subsection:Nonheavyflavour}), which significantly increases in the low-$B$ field sample for small \pt~and large rapidities, resulting in a small signal over background ratio.
A charged particle passing through the TPC deposits energy inducing signals in the pad rows of the detector. The reconstructed
space points are known as clusters.
The number of crossed rows which is equivalent to the effective cluster track length is used as a criteria for selecting tracks. A threshold of a minimum of 70 out of the total 159 crossed rows of the TPC for track reconstruction and 80 clusters for particle identification were used.
 The $\chi^{2}$ of the Kalman fit of the reconstructed track in the TPC, normalised to the number of TPC clusters ($\chi^{2} / N^{\rm{cls}}_{\rm{TPC}}$), had to be smaller than 4 to select tracks with good quality and reduce the contribution from wrongly attached clusters to the reconstructed track. Only tracks with a distance of closest approach (DCA) to the primary vertex smaller than 1 cm in the transverse plane and 2 cm in the longitudinal direction were selected in order to reject background and non-primary tracks. In the TPC--TOF analyses, all tracks were required to have an associated hit in each of the two innermost layers of the ITS to reduce the background electrons from photon conversions in the material, and to reduce wrong assignations of hits in the first layer of the ITS. For the TPC--EMCal analyses, the tracks were required to have at least one hit in one of the two innermost layers of the ITS. This reduces the impact of the inactive channels in the first ITS layer in the acceptance window of the EMCal. As the photon conversion background decreases with increasing \pt, the relaxed requirement does not affect the signal over background ratio significantly in the \pt range where the TPC--EMCal analyses were performed. Moreover, it is important to note that the track selection criteria on SPD, TOF, and EMCAL detectors sufficiently suppress background tracks originating from out-of-bunch pileup.

\begin{table}[h!]
\caption{Summary of the track selection criteria imposed on the inclusive electron candidates for different data sets and electron identification strategies. Details can be found in Sec.~\ref{subsec:elecId}.}
\begin{center}
\small
	\renewcommand{\arraystretch}{2}
\begin{tabular}{c|ccc|cc}
\hline\hline
 & \multicolumn{3}{c|}{pp $\sqrt{s} = 13$ TeV} & \multicolumn{2}{c}{\pPb $\sqrt{s_{\rm NN}} = 8.16$ TeV}  \\ 
  \hline
\pt interval (\GeVc) & 0.2--4.0 & 0.5--4.0  & 3.0--35.0 & 0.5--4.0 & 3.0--26.0 \\
 
\makecell{Track selection\\ criteria} & \makecell{Low-$B$\\TPC--TOF} & \makecell{Nominal-$B$\\TPC--TOF}  & \makecell{Nominal-$B$\\TPC--EMCal} & \makecell{Nominal-$B$\\TPC--TOF} & \makecell{Nominal-$B$\\TPC--EMCal} \\ \hline\hline
$|y|$ & $<$ 0.5 & $<$ 0.8 & $<$ {0.6} & $<$ 0.8 & $<$ 0.6 \\\hline
\makecell{No. of TPC\\crossed rows} & $\geq$ 70 & $\geq$ 70  &  $\geq$  70   & $\geq$ 70  & $\geq$ 70   \\\hline
\makecell{No. of TPC d$E/$d$x$\\clusters for PID}& $\geq$ 80 & $\geq$ 80 &  $\geq$ {80} & $\geq$ 80 & $\geq$ 80\\\hline
Number of ITS hits & $\geq$ {3} & $\geq$ {3}& $\geq$ {3} & $\geq$ 3& $\geq$ 3\\\hline
$\chi^{2} / N^{\rm{cls}}_{\rm{TPC}}$ & $<$ 4 & $<$ 4 & $<$ 4 &$<$ 4&$<$ 4\\\hline
\makecell{Minimum number of\\hits in the SPD }& 2 & 2 & 1 & 2 & 1\\\hline
$|\rm DCA_{xy}|$ & $<$ 1 cm & $<$ 1 cm & $<$ 1 cm & $<$ 1 cm& $<$ 1 cm\\\hline
$|\rm DCA_{z}|$ & $<$ 2 cm & $<$ 2 cm & $<$ 2 cm & $<$ 2 cm & $<$ 2 cm\\ \hline
\end{tabular}
\label{Table:InclusiveTrackSelection}
\end{center}
\end{table}

To identify electrons at low $p_{\rm T}$ ($p_{\rm T}$ $<$ 4 GeV$/c$), the specific energy deposition (d$E/$d$x$) in the TPC and the time-of-flight measurement in the TOF detector were used. The discriminant variable used for the TPC (TOF) detector is the deviation of d$E/$d$x$ (particle time-of-flight) from the parameterised electron Bethe--Bloch (electron time-of-flight) expectation value~\cite{Bethe:1930ku}, expressed in terms of the d$E/$d$x$ (time-of-flight) resolution, $n^{\rm{TPC}}_{\sigma,\rm{e}}$ ($n^{\rm{TOF}}_{\sigma,\rm{e}}$). In the left panel of Fig.~\ref{fig:pp_13_TPCNsigma}, $n^{\rm{TPC}}_{\sigma,\rm{e}}$ is shown as a function of the momentum of the track ($p$) after TOF selection.  
For $0.2 < p_{\rm T} < 4$ GeV$/c$, electron candidates were selected by requiring $|n^{\rm{TOF}}_{\sigma,\rm{e}}|$ $<$ 3 and $-$1 $<$ $n^{\rm{TPC}}_{\sigma,\rm{e}}$ $<$ 3, resulting in a 100\% pure electron sample at \pt$\approx$ 0.2 \GeVc, and a sample with a purity of about 90\% at 4 \GeVc.
The remaining hadron contamination in the sample, after TOF selection, was estimated and subtracted by parameterising the TPC d$E/$d$x$ distribution for each particle species with an analytical function in different momentum regions as shown in the right panel of Fig.~\ref{fig:pp_13_TPCNsigma}, and as performed in previous analyses~\cite{Acharya:2019hao, Acharya:2019mom}. 

\begin{figure}[tb!]
    %\centering
    \includegraphics[scale = 0.4]{figures/Results/HFE_pp_LowB/TPCQA_LowB.pdf}
    \includegraphics[scale = 0.4]{figures/Results/HFE_pp_LowB/TPCNsigma29to30_CR1_2.pdf}
    \caption{TPC d$E/$d$x$ signal, expressed in terms of deviation from the expected electron energy loss as a function of momentum (left panel) and fit of the measured $n^{\rm{TPC}}_{\sigma,\rm{e}}$ distribution after the $n^{\rm{TOF}}_{\sigma,\rm{e}}$ requirement in the momentum range 2.9 $< p <$ 3.0 \GeVc (right panel) in pp collisions at $\sqrt{s}$ $=$ 13 TeV for the low-$B$ field data set.}
    \label{fig:pp_13_TPCNsigma}
\end{figure}

Analyses using TPC and EMCal detectors were performed by spatially matching reconstructed charged tracks in the ITS
and TPC with EMCal clusters. This is implemented by extrapolating the reconstructed charged tracks with ITS and TPC to the EMCal, taking into account the energy loss of the particle when it traverses the detector materials, and matching within 
 the $\Delta \eta$ and $\Delta \varphi$ as given by Eq.~(\ref{eqn:deltaetaphi}). At low \pt, the position resolution of tracks and the EMCal clusters gets worse which leads to a \pt-dependent matching criteria. The \pt-dependent selection window is approximately one EMCal cell size at high \pt and few cell sizes below 1 \GeVc~\cite{ALICE:2022qhn}. This matching criterion removes the contribution from photon signals and from wrong associations of EMCal clusters to charged-particle tracks. 
\begin{align}
\label{eqn:deltaetaphi}
\begin{split}
  |\Delta \eta| &\leq 0.010 + (p_{\rm T,track} (\GeVc) +4.07)^{-2.5} ,
\\
  |\Delta \varphi| &\leq 0.015 + (p_{\rm T,track} (\GeVc) +3.65)^{-2} .
\end{split}
\end{align}
Candidate tracks matched with EMCal clusters with   $-1 < n^{\rm{TPC}}_{\sigma,\rm{e}}$ $<$ 3 were selected. Electrons were identified and separated from hadrons using the $E/p$ information, where, $E$ is the energy deposited by the particle in the EMCal detector and $p$ is the momentum of the track. It was required that the measured $E/p$ is around unity, $0.85 < E/p < 1.2$, as expected for electrons, while hadrons have lower $E/p$ values.
To further reduce the amount of hadron contamination, a condition on the shape of the electromagnetic shower, $\sigma_{\rm{long}}^{2}$,~\cite{Alessandro:2006yt} was applied. 
The quantity $\sigma_{\rm{long}}^{2}$ stands for the eigenvalues of the dispersion matrix of the shower shape ellipse defined by the energy distribution within the EMCal cluster~\cite{Awes:1992yp, Acharya:2017hyu}.  A \pt-dependent selection criterion was applied, $0.02 <  \sigma_{\rm{long}}^{2} < 0.9$ at low \pt and a more stringent selection up to $0.02 <\sigma_{\rm{long}}^{2} < 0.5$ at higher \pt, in both pp and p--Pb collisions.
The lower threshold on $\sigma_{\rm{long}}^{2}$ removes contamination caused by neutrons hitting the readout electronics. The remaining hadron contamination in the electron sample was estimated by fitting the measured $E/p$ distributions of electron candidates in momentum slices. For this purpose, the shape of the $E/p$ spectrum for hadrons was obtained by selecting hadrons in the TPC with $n^{\rm{TPC}}_{\sigma,\rm{e}} < -3.5 $. The obtained hadron $E/p$ distribution was then scaled to match the $E/p$ distribution of electron candidates in a region within $E/p< 0.7$, as shown in Fig.~\ref{fig:pp_13_EbyP}. The electron yield was calculated by integrating the $E/p$ distributions of electron candidates in the range $0.85 < { E}/{p} < 1.2$ after the subtraction of the hadron contamination. In the pp (\pPb) analysis, the hadron contamination was negligible at low $p_{\rm{T}}$, increasing up to $23\%$ ($25\%$) at $p_{\rm{T}} $ $=$ 35 (26) GeV$/c$.

\begin{figure}[ht!]
    \centering
\includegraphics[height=7.5cm]{figures/Results/HFE_ppNormalB/EbyP_EG1_CR1_2.pdf}
    \caption{The $E/p$ distribution measured in pp collisions at $\sqrt{s}$ = 13 TeV for  EG1 triggered events.}
    \label{fig:pp_13_EbyP}
\end{figure}


\subsection{Subtraction of electrons from non heavy-flavour sources}\label{subsection:Nonheavyflavour}
The selected inclusive electron sample contains electrons from open heavy-flavour hadron decays and from  different sources of background:

\begin{itemize}
    \item dielectrons originating from Dalitz decays of light-neutral mesons such as $\pi^{0}$, $\eta$ as well as conversions of photons in the detector material, named as photonic electrons in the text,
    \item dielectrons from decays of J$/\psi$ (J$/\psi$ $\rightarrow$ $\rm e^{+}e^{-}$) and low-mass vector mesons ($\rho$ $\rightarrow$ $\rm e^{+}e^{-}$, $\omega$ $\rightarrow$ $\rm e^{+}e^{-}$, $\phi$ $\rightarrow$ $\rm e^{+}e^{-}$),
\item electrons from kaon weak decays $\rm K^{0,\pm}$ $\rightarrow$ $\rm e^{\pm}$ $\rm \pi^{\mp,0}$ $\rm \overset{(-)}{\nu_{e}}$ ($\rm K_{e3}$),
\item electrons from W and Z decays.
\end{itemize}

The dominant sources of background electrons are photon conversions in the detector material and Dalitz decays of light-neutral mesons. These contributions were estimated using an invariant mass technique~\cite{Adam:2015qda} of electron--positron pairs. Unlike-signed electron--positron pairs (ULS) were defined by pairing the selected electrons with opposite-charge electron partners.
To increase the efficiency of finding the partner, associated electrons were selected applying similar but looser track quality and particle identification criteria than those used for selecting signal electrons. The selection criteria are summarised in Table~\ref{Table:AssocatedTrackSelection}. The electron--positron pairs from photonic background have a small invariant mass ($ m_{\rm e^{+}e^{-}}$). Heavy-flavour decay electrons can form
ULS pairs mainly through random combinations with other electrons.
The combinatorial contribution was estimated from the invariant mass distribution of like-signed electron (LS) pairs. The photonic background contribution was then evaluated by subtracting the LS distribution from the ULS one in the invariant mass region $m_{\rm e^{+}e^{-}} < 0.14$ GeV$/c$. The efficiency of finding the partner electron, called tagging efficiency ($\epsilon_{\rm tag}$) from hereon, was estimated using MC simulations. In the pp and \pPb analyses, the MC sample was obtained using PYTHIA 6~\cite{Sjostrand:2006za} and HIJING~\cite{Wang:1991hta} generators, respectively. The generated particles were propagated through the ALICE apparatus using GEANT 3~\cite{Brun:1073159}. In order to increase the statistical precision of
$\epsilon_{\rm tag}$ using the invariant mass method, $\pi^0$ and $\eta$ mesons were embedded in the simulated events.
The simulated $\pi^0$ and $\eta$ \pt distributions
were reweighted to match the measured spectra. For pp collisions, the $\pi^0$ spectrum was estimated as the average of the spectra of $\pi^{+}$ and $\pi^{-}$~\cite{Acharya:2020uxl}, whereas the $\eta$ spectrum was obtained using $m_{\rm T}$ scaling, as in~\cite{Gatoff:1992cv, Khandai:2011cf, Altenkamper:2017qot}. For the \pPb analysis, the measured transverse momentum spectra of $\pi^{0}$ and $\eta$ were used~\cite{alicecollaboration2021nuclear}. 
In the pp analysis, the tagging efficiency at low $p_{\rm{T}}$ ($p_{\rm{T}} < 1.0$ GeV$/c$) is 55--65\% for the low-$B$ field data set, whereas for the nominal-$B$ field data set it is around 45--55\% at low $p_{\rm{T}}$ ($p_{\rm{T}} < 1.0$ GeV$/c$) increasing to about 85\% at high $p_{\rm{T}}$ ($p_{\rm{T}} > 15$ GeV$/c$). In the \pPb analysis, the tagging efficiency  varies between 40\% at low \pt ($p_{\rm{T}} < 3.0$ GeV$/c$) and 80\% at high \pt ($p_{\rm{T}} > 7.0$ GeV$/c$).

\begin{table}[h]
\caption{Summary of the track selection criteria imposed on the associated electron candidates for different data sets and electron identification strategies.}

\centering
\small
	\renewcommand{\arraystretch}{2}
\begin{tabular}{c |c c c | c c}
\hline\hline
 & \multicolumn{3}{c|}{pp $\sqrt{s} = 13$ TeV} & \multicolumn{2}{c}{\pPb $\sqrt{s_{\rm NN}} = 8.16$ TeV}  \\ 
  \hline
 \pt interval (\GeVc) & 0.2--4.0 & 0.5--4.0  & 3.0--35.0 & 0.5--4.0 & 3.0--26.0 \\ 
\makecell{Track and PID\\cuts}  & \makecell{Low-$B$\\TPC--TOF} & \makecell{Nominal-$B$\\TPC--TOF}  & \makecell{Nominal-$B$\\TPC--EMCal} & \makecell{Nominal-$B$\\TPC--TOF} & \makecell{Nominal-$B$\\TPC--EMCal} \\
\hline \hline
$\pt^{\rm min}$ & 0.0 GeV/c   &{ 0.1 GeV/c}&   { 0.1 GeV/c}& 0.1 GeV/c &  0.1 GeV/c\\\hline
$|\textrm{\it y}|$ & $<$ 0.8  &  { $<$ 0.9 } &  { $<$0.9 } & $<$ 0.8 & $<$ 0.8 \\\hline
\makecell{No. of TPC\\d$E/$d$x$ clusters for PID} & $\geq$ 60  &$\geq$ 60 & $\geq$ 60 & $\geq$ 60 & $\geq$ 60\\\hline
Number of ITS hits & $\geq$ 2 & $\geq$ 2& $\geq$ 2 & $\geq$ 2& $\geq$ 2\\\hline
$\chi^{2} / N^{\rm{cls}}_{\rm{TPC}}$ & $<$ 4 &$<$ 4 & $<$ 4 &$<$ 4&$<$ 4\\ \hline
$|\rm DCA_{xy}|$ & $<$ 1 cm &$<$ 1 cm  & $<$ 1 cm & $<$ 1 cm& $<$ 1 cm\\ \hline
$|\rm DCA_{z}|$ & $<$ 2 cm & $<$ 2 cm & $<$ 2 cm & $<$ 2 cm & $<$ 2 cm\\ \hline
\hline
\end{tabular}
\label{Table:AssocatedTrackSelection}
\end{table}


{Due to the requirement of hits in the SPD layers, the contribution of electrons from $\rm K_{e3}$ decays was found to be negligible with respect to the heavy-flavour signal for $p_{\rm{T}} > 0.5$ GeV$/c$. At lower $p_{\rm{T}}$, the relative contribution of electrons from $\rm K_{e3}$ decays becomes non-negligible and hence it was subtracted from the  \pt-differential cross section of electrons from heavy-flavour hadron decays.} 
The $\rm K_{e3}$ contribution was estimated using a parameterisation of the ratio of $\rm K_{e3}$ to photonic electrons obtained from previous analyses using the so-called cocktail approach~\cite{Acharya:2018upq,Abelev:2014gla,ALICE:2012mzy}. The same parametrisation was used in the analysis of pp collisions at $\sqrt{s}=13~\rm{TeV}$ and in p–Pb collisions at $\sqrt{s_{\rm NN}} = 8.16$ TeV. 

Other background contributions of $\rm e^{+}e^{-}$ pairs from J/$\psi$ and low-mass vector mesons were negligible~\cite{Acharya:2018upq, ALICE:2020try} compared to the signal and were therefore not subtracted. Electrons from $\rm{W}^{\pm}$ and $\rm{Z}^0$ boson decays form a significant background at high $p_{\rm{T}}$ ($p_{\rm{T}}>20$ GeV$/c$), which was estimated with the next-to-leading order event generator POWHEG~\cite{Oleari:2010nx}, interfaced with PYTHIA as a decayer, and subtracted from the \pt-differential cross section of electrons from heavy-flavour hadron decays. This contribution increases from 1\% at $p_{\rm{T}} = 15$ GeV$/c$ to
about {3\%} at $p_{\rm{T}} = 20$ GeV$/c$ and up to 25\% at $p_{\rm{T}} = 35$ GeV$/c$ with respect to the heavy-flavour decay electron yield in both pp and \pPb collisions.

In pp collisions, the ratio of signal over background electrons is about 0.08 at \pt $=$ 0.2 GeV$/c$,  3 at $\mbox{\pt $=$ 0.5  GeV$/c$}$ increasing to $\sim$ 10.5 at \pt $=$ 25 \GeVc and reaches 12.3 at \pt $=$ 35 GeV$/c$, whereas in \pPb collisions it is about 2.65 at \pt $=$ 0.5 GeV$/c$ and increases up to 8.39  at \pt $=$ 26 GeV$/c$. 
