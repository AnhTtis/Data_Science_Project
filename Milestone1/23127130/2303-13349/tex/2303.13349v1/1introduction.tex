\section{Introduction}\label{section:introduction}
In high-energy hadronic collisions, heavy quarks are mainly produced in hard parton scattering processes. Due to their large masses, their production cross sections can be calculated in the framework of perturbative quantum chromodynamics (pQCD) down to low transverse momenta~\cite{Kniehl:2008zza,Cacciari:2003uh,Kniehl:2005mk,Cacciari:2003zu}. 
Measurements of production cross sections of open heavy-flavour hadrons and their decay products in pp and Pb--Pb collisions were performed by the ALICE, CMS, ATLAS and LHCb Collaborations at the LHC at both mid and forward rapidity~\cite{ATLAS:2015igt, ALICE:2019nxm,ALICE:2012msv,Abelev:2012vra,ALICE:2012mzy,Aaij:2016jht,CMS:2017qjw,LHCb:2016ikn,LHCb:2015swx,ALICE:2021npz,ALICE:2012acz,ALICE:2019nuy,ATLAS:2021xtw,Abelev:2012gx,Adam:2016wyz,ATLAS:2013cia,Khachatryan:2010yr,Aaij:2010gn,ALICE:2021mgk,ALICE:2020mfy,CMS:2016plw,LHCb:2016qpe,LHCb:2015foc,Abelev:2012sca}. These measurements are described by theoretical predictions based on pQCD calculations with the collinear factorisation approach at next-to-leading order with next-to-leading log resummation e.g.~in the GM-VFNS (general-mass variable-flavour-number scheme)~\cite{Kniehl:2008eu,Kniehl:2011bk,Kniehl:2005ej,Bolzoni:2012kx,Bolzoni_2014}) or the FONLL (fixed order with next-to-leading-log resummation)~\cite{Cacciari:2012ny} frameworks, within theoretical uncertainties.
 Measurements of charm-baryon production at midrapidity in pp collisions show an enhancement of the $\Lambda_{\rm{c}}^{+}/\rm{D}^0$~\cite{ALICE:2021npz,Acharya:2017kfy,ALICE:2020wfu,ALICE:2020wla,Sirunyan:2019fnc}, $\Xi_{\rm{c}}^{+,0}/\Dzero$~\cite{Acharya:2017lwf,ALICE:2021psx,ALICE:2021bli}, $\Sigma_{\rm{c}}^{+,0}/\Dzero$~\cite{ALICE:2021rzj}, and $\Omega_{\rm{c}}^{+,0}/\Dzero$~\cite{ALICE:2022cop} ratios with respect to those measured in ${\rm e^{+}e^{-}}$ and ep collisions~\cite{Gladilin:2014tba}. A multiplicity dependence measurement of the $\Lambda_{\rm{c}}^{+}/\rm{D}^0$ ratio~\cite{ALICE:2021npz} has revealed a significant increase from the lowest to the highest multiplicity. These observations indicate that the hadronisation of charm quarks into charm hadrons is not a universal process among different collision systems. These findings are similar to those obtained in the beauty sector by the CDF Collaboration at the Tevatron~\cite{Aaltonen:2008zd} and by the LHCb Collaboration at the LHC~\cite{Aaij:2011jp,Aaij:2019pqz}.

In proton--nucleus collisions, the so-called cold nuclear matter (CNM) effects occur due to the presence of a nucleus in the colliding system and to the large density of produced particles. In particular, the parton distribution functions (PDFs) of nucleons bound in nuclei are modified with respect to those of free nucleons, which can be described by phenomenological parameterisations referred to as nuclear PDFs (nPDFs)~\cite{Eskola:2009uj,deFlorian:2003qf,Hirai:2007sx,Eskola:2016oht}. When the production process is dominated by gluons at low Bjorken-$x$, the nucleus can be described by the Colour Glass Condensate (CGC) effective theory as a coherent and saturated gluonic system~\cite{Fujii:2013yja,Tribedy:2011aa,Albacete:2012xq,Rezaeian:2012ye}. The kinematics of the partons in the initial state can be affected by multiple scatterings
~\cite{Lev:1983hh,Kopeliovich:2002yh} or by gluon radiation (energy loss) before or after the heavy-quark pair is produced~\cite{Vitev:2007ve}. Measurements of heavy-flavour production in \pPb collisions at the LHC will allow a study of the above mentioned effects. Previous measurements of the nuclear modification factor of leptons from heavy-flavour hadron decays in \pPb collisions at $\sqrt{s_{\textrm{NN}}}=5.02$ TeV by the ALICE Collaboration indicate no significant modification of their yields due to CNM effects in the measured  transverse momentum (\pt) region within uncertainties~\cite{Adam:2015qda,Adam:2016wyz}. For a nucleus--nucleus collisions (AA), the nuclear modification factor ($R_{\rm AA}$) is the ratio of the yield in nucleus--nucleus collisions with respect to the yield in proton--proton collisions scaled by the number of binary nucleon--nucleon collision in AA. It quantifies the interaction of a particle and its energy loss while traversing through a medium formed in AA collisions with respect to pp collisions.  Measurements of the nuclear modification factor of open heavy flavour and quarkonia  at mid, forward, and backward rapidity in p–Pb collisions were performed by the ALICE~\cite{ALICE:2019fhe,ALICE:2021lmn,ALICE:2022zig, ALICE:2018mml,Abelev:2014hha,ALICE:2022ych}, ATLAS~\cite{ATLAS:2017prf}, CMS~\cite{CMS:2017exb,CMS:2015sfx}, and LHCb~\cite{LHCb:2013gmv,LHCb:2017ygo,LHCb:2019avm,LHCb:2022rlh} collaborations. The results can be described qualitatively by the various theoretical calculations  mentioned above.

Recent measurements of light-flavour~\cite{CMS:2010ifv,ATLAS:2015hzw,CMS:2015fgy,Abelev:2012ola,Aaboud:2016yar,Chatrchyan:2013nka,ABELEV:2013wsa,Khachatryan:2014jra,Adare:2013piz,Adamczyk:2015xjc,Adare:2015ctn,Khachatryan:2010gv,CMS:2012qk,ALICE:2012eyl,ATLAS:2012cix,LHCb:2015coe} and heavy-flavour hadrons~\cite{Acharya:2018dxy,Sirunyan:2018toe,Sirunyan:2018kiz,ALICE:2017smo,CMS:2020qul} in high-multiplicity pp, p--A, and d--A collisions at different energies have revealed strong flow-like effects in these small systems~\cite{ALICE:2022wpn}. The origin of this phenomenon is debated. Models that incorporate hydrodynamical evolution of the system~\cite{Werner:2010ss,Deng:2011at,Werner:2013ipa,Schenke:2019pmk}, overlapping strings~\cite{Bierlich:2014xba}, string percolation~\cite{Bautista:2015kwa}, or multiple-parton
interactions together with colour reconnection~\cite{Sjostrand:2014zea,Ortiz:2013yxa} can describe
qualitatively the observed features in high-multiplicity events. A multiphase transport model~\cite{Koop:2015wea}, as well as calculations based on the fragmentation of saturated gluon states~\cite{Schlichting:2016sqo,Schenke:2016lrs}, are also able to describe some features of the data.
The measurement of heavy-flavour production in small systems as a function of the charged-particle multiplicity produced in the collision could thus provide further insight into the processes occurring in the collision at the partonic level and the interplay between the hard and soft mechanisms in particle production in pp and \pPb collisions. 

Measurements of charm and beauty productions~\cite{Adam:2015ota,ALICE:2020msa,Adam:2018jmp,Chatrchyan:2013nza,Acharya:2020giw,Adam:2016mkz} indicate an increase of heavy-flavour production with charged-particle multiplicity measured at midrapidity.
The D meson~\cite{Adam:2015ota} and J$/\psi$~\cite{ALICE:2020msa} productions normalised to their corresponding multiplicity-integrated yields in minimum bias events (self-normalised yields), as a function of self-normalised event multiplicities, (i.e., normalised to the average multiplicity in minimum bias collisions)  are measured in pp collisions at $\sqrt{s} = 7$ TeV and $\mbox{$\sqrt{s} = 13$ TeV}$ by the ALICE Collaboration at the LHC, and at $\sqrt{s} = 0.2$ TeV by the STAR Collaboration at RHIC~\cite{Adam:2018jmp}. These measurements show a stronger than linear increase of self-normalised yields as a function of self-normalised multiplicity.  Measurements of the $\Upsilon$(nS) production in pp collisions at $\sqrt{s}$ = 2.76 TeV and $\sqrt{s}$ = 7 TeV by the CMS Collaboration at midrapidity indicate a linear increase with the event activity, when measuring it at forward rapidity, and a stronger than linear increase with the event activity measured at midrapidity~\cite{Chatrchyan:2013nza}. A comprehensive review of the connection between the $\Upsilon$(nS)
production and the underlying event, is presented by the CMS Collaboration in pp collisions at $\sqrt{s}$ = 2.76 TeV~\cite{CMS:2020fae}. Measurements of  multiplicity dependence of $\Upsilon$(nS) production at forward rapidity, is presented by the ALICE Collaboration in pp collisions at $\sqrt{s}$ = 13 TeV~\cite{ALICE:2022yzs}. In \pPb collisions, the self-normalised D meson yield at midrapidity increases with a faster than linear trend as a function of the self-normalised charged-particle multiplicity at midrapidity and is consistent with a linear growth for multiplicity measured at large rapidities~\cite{Adam:2016mkz}. The self-normalised J/$\psi$ yield at larger rapidities also exhibits an increase with increasing normalised charged-particle pseudorapidity density, where the yield at backward rapidity grows faster than the forward rapidity one~\cite{Acharya:2020giw}. A possible correlation with the event multiplicity (and event shape) is also observed for the inclusive charged-particle production~\cite{Acharya:2019mzb}, and for identified particles, including multi-strange hyperons~\cite{Acharya:2018orn}. The trends are qualitatively, and for some of the calculations quantitatively, reproduced by QCD-inspired event generators such as PYTHIA 8~\cite{Sjostrand:2006za}, and EPOS LHC and EPOS 3~\cite{Werner:2013tya,Pierog:2013ria}. But a critical evaluation of the similarities and differences between the physics mechanisms at play in various models is yet to be performed. More stringent tests of the models would be important in this direction. A comparison of the multiplicity-dependent measurements for different particle species would also provide insight into the origin of the observed phenomena~\cite{LHCb:2022syj,ALICE:2021npz}.  

In this article, measurements of the production cross section of electrons from heavy-flavour hadron decays at midrapidity in pp collisions at $\sqrt{s}=13$ TeV and \pPb collisions at $\sqrt{s_{\textrm{NN}}}=8.16$ TeV are presented. The cross section of electrons from heavy-flavour hadron decays was measured as a function of transverse momentum down to 0.2 GeV$/c$ and up to 35 GeV$/c$ in pp collisions, which is the lowest and highest $\pt$-reach attained for electrons from heavy-flavour hadron decays with the ALICE detector. 
Results of the nuclear modification factor ($R_{\rm{pPb}}$) of electrons from heavy-flavour hadron decays at midrapidity in \pPb collisions at $\sqrt{s_{\textrm{NN}}}=8.16$ TeV are reported as well. The self-normalised yields of electrons from heavy-flavour hadron decays measured for the first time as a function of charged-particle multiplicity estimated at midrapidity  ($|\eta|$ $<$ 1) in pp and \pPb collisions are also presented. The comparison of the self-normalised yields of electrons from heavy-flavour hadron decays with other particles measured using the ALICE detector and with Monte Carlo (MC) simulations is discussed.  

The article is structured as follows. In Sec.~\ref{sec:datasampleandselection}, the ALICE apparatus, its main detectors and the data samples used for the analysis are reported. The definition of multiplicity and the calculation of the charged-particle pseudorapidity density is addressed in Sec.~\ref{section:multiplicityanalysis}. In Sec.~\ref{section:heavyflavour}, the procedure employed to obtain the production cross sections of electrons from heavy-flavour hadron decays is explained. Section~\ref{sec:systematics} describes the systematic uncertainties associated with the measurements. The results of the analysis are presented and discussed in Sec.~\ref{section:results}. Finally, the article is summarised in Sec.~\ref{section:summary}.



