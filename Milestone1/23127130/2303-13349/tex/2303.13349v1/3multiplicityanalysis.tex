\section{Multiplicity definition and corrections}\label{section:multiplicityanalysis}

The production of electrons from heavy-flavour hadron decays was investigated as a function of charged-particle pseudorapidity density (\dnchdeta) in pp and \pPb collisions.
The \dnchdeta was measured in the pseudorapidity range \etaless 1. It was evaluated using the number of tracklets (\ntracklets) in the SPD~\cite{ALICE:2012xs,Acharya:2018egz}, defined as track segments pointing to the primary vertex and formed by joining pairs of hits in the two SPD layers. 

The number of raw tracklets (\ntracklets) in an event were corrected ($N^{\rm{corr}}_{\rm{tracklets}}$) for the variation of the detector conditions with time (fraction of active SPD channels) and its limited acceptance as a function of \zvertex using a data-driven event-by-event correction, following the procedure discussed in Refs.~\cite{Abelev:2012rz}.
The corrections were done by applying a \zvertex and time-dependent correction factor such that the measured average multiplicity is equalised to a reference value, which was chosen to be the largest mean SPD tracklet multiplicity observed over time. The correction factor for each event was randomly smeared using a Poisson distribution to take into account event-by-event fluctuations. The number of events, sliced in $N^{\rm{corr}}_{\rm{tracklets}}$ intervals, were corrected for the trigger and primary vertex finding efficiencies, following the procedure discussed in~\cite{ALICE:2020msa}.
The former was estimated from MC simulations and the latter with a data-driven approach. 
The efficiencies were close to unity for all multiplicity classes except for the lowest multiplicity class interval, where the efficiency was close to 90\%.

Detector inefficiencies, production of secondary particles due to interactions with the detector material, and particle decays give a different number of reconstructed tracklets compared to the
true primary charged-particle multiplicity value $N_{\rm{ch}}$~\cite{Adam:2015pza}. MC simulations using the PYTHIA 8.2~\cite{Sjostrand:2006za} and the DPMJET~\cite{Roesler:2000he} event generators, for pp and \pPb collisions respectively, and the GEANT 3~\cite{Brun:1073159} transport code were used to estimate $N_{\rm{ch}}$ from $N^{\rm{corr}}_{\rm{tracklets}}$ and later \dnchdeta. A second-order polynomial correlation was assumed between the two quantities, $N_{\rm{ch}}$ and $N^{\rm{corr}}_{\rm{tracklets}}$, for the full $N^{\rm{corr}}_{\rm{tracklets}}$ range. 
To estimate the systematic uncertainties on \dnchdeta, possible deviations from the second-order polynomial correlation between $N_{\rm{ch}}$ and $N^{\rm{corr}}_{\rm{tracklets}}$ were estimated using a linear function.
The systematic uncertainty on the residual $z_{\rm vtx}$ dependence due to differences between data and MC amounts to about 1\% in pp collisions and is negligible in \pPb collisions. The total systematic uncertainty on $\dnchdeta$ is about $5\%$ in all multiplicity intervals for both pp and \pPb collisions.

The average charged-particle pseudorapidity density was normalised to its average value in $\rm{INEL}>0$ events in pp and \pPb collisions. The $\rm{INEL}>0$ event class contains all events with at least one charged particle within \etaless 1. The average charged-particle pseudorapidity densities $\left(\left<\dnchdeta\right>\right)$ for $\rm{INEL}>0$ were found to be
in agreement with the previous published ALICE measurements~\cite{2021,ALICE:2020msa,Acharya:2020giw}. The resulting values of the self-normalised charged-particle pseudorapidity density $\left( \dnchdeta/\left<\dnchdeta\right> \right)$ for the event classes considered in the analyses presented here are summarised in Table~\ref{Table:dndchValues}.

\begin{table}[!ht]
\caption{Average self-normalised charged-particle pseudorapidity density $\left( \dnchdeta/\left<\dnchdeta\right> \right)$ in \etaless 1.0 for each event class selected in pp and \pPb collisions.}
\label{Table:dndchValues}
	\renewcommand{\arraystretch}{1.4}
  \begin{tabular*}{\textwidth}{@{\extracolsep{\fill}} c|cc|cc}
    \hline
  \multicolumn{1}{c|}{ } &
   
   \multicolumn{2}{c|}{pp $\sqrt{s} = 13$ TeV} & \multicolumn{2}{c}{\pPb $\sqrt{s_{\rm NN}} = 8.16$ TeV} \\
 \hline

 \multicolumn{1}{c|}{Multiplicity class} &
     \multicolumn{1}{c}{$N^{\rm{corr}}_{\rm{tracklets}}$} & \multicolumn{1}{c|}{$\dnchdeta/\left<\dnchdeta\right>$} &  \multicolumn{1}{c}{$N^{\rm{corr}}_{\rm{tracklets}}$} & \multicolumn{1}{c}{$\dnchdeta/\left<\dnchdeta\right>$} \\
  \hline    
     \multicolumn{1}{c|}{I} &
     \multicolumn{1}{c}{1--14} & \multicolumn{1}{c|}{0.48} &  \multicolumn{1}{c}{1--38} & \multicolumn{1}{c}{0.51} \\
       \hline    
     \multicolumn{1}{c|}{II} &
     \multicolumn{1}{c}{15--24} & \multicolumn{1}{c|}{1.63} &  \multicolumn{1}{c}{39--55} & \multicolumn{1}{c}{1.32} \\
       \hline    
     \multicolumn{1}{c|}{III} &
     \multicolumn{1}{c}{25--34} & \multicolumn{1}{c|}{2.50} &  \multicolumn{1}{c}{56--95} & \multicolumn{1}{c}{2.03} \\
       \hline    
     \multicolumn{1}{c|}{IV} &
     \multicolumn{1}{c}{35--44} & \multicolumn{1}{c|}{3.34} &  \multicolumn{1}{c}{96--121} & \multicolumn{1}{c}{3.01} \\
       \hline    
     \multicolumn{1}{c|}{V} &
     \multicolumn{1}{c}{45--54} & \multicolumn{1}{c|}{4.16} &  \multicolumn{1}{c}{122--300} & \multicolumn{1}{c}{3.85} \\
       \hline    
     \multicolumn{1}{c|}{VI} &
     \multicolumn{1}{c}{55--64} & \multicolumn{1}{c|}{4.97} &  \multicolumn{1}{c}{} & \multicolumn{1}{c}{} \\
       \hline    
     \multicolumn{1}{c|}{VII} &
     \multicolumn{1}{c}{65--120} & \multicolumn{1}{c|}{6.05} &  \multicolumn{1}{c}{} & \multicolumn{1}{c}{} \\
   \hline
  \end{tabular*}
\end{table}


