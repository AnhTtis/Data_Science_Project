\subsection{EMCal trigger rejection factor} \label{ssec:TriggerRejectionFactor}
The EMCal triggered events are reported in terms of $N_{\rm evt}\times \rm RF$, where $N_{\rm evt}$ is the number of triggered collisions and the Rejection Factor (RF) is the average number of rejected MB events per EMCal triggered event. 
The RFs were estimated with a data-driven method. For the EG2 trigger the RF was calculated as the ratio of the cluster energy distribution in EG2 triggered data to the one in MB triggered data ($f^{\rm{cl}}_{\rm{EG2}}/f^{\rm{cl}}_{\rm{MB}}$), which gives the EG2 turn-on curve. In order to reduce the effect of  poor statistics in the MB sample at high \pt, the EG1 trigger turn-on curve was obtained using the ratio of the EG1 triggered data cluster energy distribution to the one in EG2 triggered data ($f^{\rm{cl}}_{\rm{EG1}}/f^{\rm{cl}}_{\rm{EG2}}$). The RF of EG1 trigger is then the product of $f^{\rm{cl}}_{\rm{EG2}}/f^{\rm{cl}}_{\rm{MB}}$ and $f^{\rm{cl}}_{\rm{EG1}}/f^{\rm{cl}}_{\rm{EG2}}$.
The turn-on curve was determined for the multiplicity-integrated interval and for different multiplicity intervals in pp and \pPb collisions.
The turn-on curves are shown for both trigger energy thresholds (EG1 and EG2) in multiplicity integrated pp and \pPb collisions in Fig.~\ref{fig:RF}.

\begin{figure}[h!]
    \centering
    \includegraphics[width=0.48\linewidth]{{figures/Results/HFE_ppNormalB/RF_trackmatchCluster_CR1}.pdf}
    \includegraphics[width=0.48\linewidth]{figures/Results/HFE_pPb/RF_pPb_8TeV_CR1.pdf}
    \caption{Trigger RF for EG2 and EG1 triggers in pp collisions at $\sqrt{s}=13$ TeV (left panel) and in \pPb collisions at $\sqrtsNN = 8.16$ TeV (right panel).}
   \label{fig:RF}
\end{figure}

A Fermi function~\cite{Fermi:1934hr, Wilson:1968pwx} was used to fit the trigger turn-on curves and determine the RF above the trigger threshold. 
The fit range is from the beginning of the turn-on region i.e. near trigger threshold to the highest energy where the distribution remains flat.
The EG2/MB and the EG1/EG2 values correspond to the constant values determined by the plateau of the fitted Fermi function above the trigger thresholds. 
The final values of the RF used in the analyses are summarised in Table ~\ref{table:RF}.
The systematic uncertainty on the trigger RF was estimated by varying the fit region of the trigger turn-on curve and the fit function, i.e. using a linear function in the plateau region. A systematic uncertainty of 3\% (2\%) for the EG2 trigger and 4\% (3\%) for EG1 trigger was obtained for pp (\pPb) collisions.

\begin{table*}[tb!]
\caption{Multiplicity-integrated values of the EMCal trigger RF with their uncertainties for the EG2 and EG1 triggered data sets in pp and \pPb collisions.}
    \centering
    	\renewcommand{\arraystretch}{1.2}
     %\begin{tabular*}{\textwidth}{@{\extracolsep{\fill}}  c| c c | c c}
     \resizebox{\textwidth}{!}{%
    \begin{tabular}{c| c c | c c}
    \hline
    \multicolumn{1}{c|}{}&
    \multicolumn{2}{c|}{pp $\sqrt{s} = 13\ \TeV$}& \multicolumn{2}{c}{\pPb $\sqrt{s_{\rm NN}} = 8.16\ \TeV$}\\
%   \hline
   %    &  &  pp &  & p--Pb\\
    Trigger   & EG2 & EG1 & EG2 & EG1 \\
    \hline
     RF value & { 406  } & { 4985  } &  283.4 & 1020.7 \\
       & {  $\pm$ 1 (stat.) $\pm$  12 (syst.) } & {  $\pm$ 11 (stat.) $\pm$ 200 (syst.) } &  $\pm$ 1.5 (stat.) $\pm$ 5.9 (syst.) & $\pm$ 3.6 (stat.)  $\pm$ 30.4 (syst.) \\
     \hline
    \end{tabular}}
    \label{table:RF}
\end{table*}


\subsection{Efficiency correction and normalisation}\label{section:corrections}
The raw number of electrons and positrons from heavy-flavour hadron decays, $N_{\rm raw}$, was obtained by subtracting the hadron contamination, photonic electrons, and the other background electron contributions. The $\mbox{\pt-differential}$ cross section of electrons from heavy-flavour hadron decays at midrapidity was then calculated using the formula
 \begin{equation}
 \frac{{\rm d}^2\sigma}{{\rm d} p_{\rm T}{\rm d} y} = \frac{1}{2}  \frac{1}{\Delta y \Delta p_{\rm T}} \frac{{N_{\rm{raw}}}} {\epsilon^{\rm{geo}} \times \epsilon^{\rm reco} \times  \epsilon^{\rm {eID}} } 
 \frac{1}{\mathcal{L}_{\text{int}}}
\end{equation}

, where $\mathcal{L}_{\rm int}$ is the integrated luminosity, and $\Delta p_{\rm T}$ and $\Delta y$ the width of the $p_{\rm T}$ and rapidity intervals, respectively. The integrated luminosity was calculated using the number of analysed events and the measured MB trigger cross sections ($\sigma_{\rm MB}$), as $N_{\rm evt}/\sigma_{\rm MB}$ for minimum bias triggered events and {$\mbox{$N_{\rm evt}\times \rm{RF}/\sigma_{\rm MB}$}$} for EMCal triggered events, and are listed in Table~\ref{table:EventStat}. The trigger bias was studied in MC simulations and was found to be negligible for the selections applied to tracks and clusters. The measured $\sigma_{\rm MB}$ values are 58.44 $\pm$ 1.11 mb, 58.10  $\pm$  1.57 mb, and 57.52 $\pm $ 1.21 mb for the pp collisions at $\sqrt{s}$ $=13~\rm TeV$ collected in the years 2016, 2017, and 2018, respectively~\cite{ALICE-PUBLIC-2021-005}, and 2100 $\pm$ 60 mb~\cite{ALICE-PUBLIC-2018-002} in \pPb collisions at $\sqrtsNN$ $=$ 8.16 TeV. The raw number of electrons from heavy-flavour hadron decays was corrected for the geometrical acceptance ($\epsilon^{\rm geo}$), the track reconstruction ($\epsilon^{\rm reco}$), and electron identification ($\epsilon^{\rm eID}$) efficiencies. The factor of two accounts for the charged averaged contribution of electrons and positrons. The trigger and event selection criteria were found to be fully efficient for electrons from heavy-flavour hadron decays.

The above mentioned acceptance and track reconstruction efficiencies are computed by means of MC simulations using PYTHIA 6~\cite{Sjostrand:2006za} and HIJING~\cite{Wang:1991hta} event generators for pp and \pPb collisions, respectively. For pp simulations PYTHIA 6 generated events with at least one ${\rm c{\overline{c}}}$ or ${\rm b{\overline{b}}}$ pair were selected for propagation through the apparatus with GEANT 3~\cite{Brun:1073159} and subsequent reconstruction. In the case of \pPb collisions, to have an efficient generation of heavy-flavour signals and reproduce the detector occupancy, one PYTHIA 6 event with a ${\rm c{\overline{c}}}$ or ${\rm b{\overline{b}}}$ pair was embedded in each HIJING simulated event.


The electron identification (eID) efficiencies for the TOF, TPC, and EMCal detectors were obtained separately, and then multiplied according to the detectors used in the analysis to compute the full electron identification efficiency $\varepsilon^{\rm eID}$. The TPC--TOF track matching and  electron identification efficiency of the TOF detector was calculated with the above mentioned MC sample and was found to be 60--70$\%$ (40--65$\%$) for  0.5 $<$ $p_{\rm T}$ $<$ 1.5 GeV$/c$ increasing up to 75$\%$ (70$\%$) at 4 GeV$/c$ in the low (nominal)-$B$ field analysis. A better track matching between the TPC and the TOF detectors is achieved at a given \pt with the low-$B$ field compared to the nominal-$B$ field, due to the smaller curvature of the tracks. Therefore, a higher reconstruction efficiency is observed in the low-$B$ field data compared to the nominal-$B$ field sample. The TPC  electron identification efficiency was determined using a data-driven approach based on the $n^{\rm{TPC}}_{\sigma,\rm{e}}$ distribution~\cite{ALICE:2012mzy}. It is about 88\% at $p_{\rm T}= 0.2$ GeV$/c$ and increases to $89\%$ for $\mbox{$p_{\rm T} > 0.5$ GeV$/c$}$, for the low-$B$ field data sample.

In the nominal-$B$ field data set, it was found to be around $86\%$ at $p_{\rm T}= 0.5$ GeV$/c$ increasing to $88\%$ for $p_{\rm T} > 4$ GeV$/c$. 
The  electron identification efficiency with EMCal was estimated using MC simulations and was found to be about 60\% at 3 GeV$/c$, increasing up to 80\% for \pt larger than 10 GeV/$c$ for pp collisions. 
The total reconstruction efficiency ($\epsilon^{\rm{geo}} \times \epsilon^{\rm reco} \times  \epsilon^{\rm {eID}}$) for different data sets and with different detectors is presented in Fig.~\ref{Fig:ppHFEEff}. 

The production of electrons from heavy-flavour hadron decays was further studied as a function of the charged-particle pseudorapidity density in pp or \pPb collisions using the self-normalised yield of heavy-flavour hadron decay electrons. The  differential yield measured in a given multiplicity class was divided by its average over all INEL $>$ 0 events $({\rm d}^{2} {N}/{\rm d}p_{\rm T} {\rm d}{\eta} / \langle {\rm d}^{2}{N}/{\rm d}p_{\rm T}{\rm d}{\eta}\rangle_{\rm INEL>0})$ in pp or \pPb collisions.  All efficiency were obtained as a function of multiplicity. At low \pt, the tagging efficiency
and the total reconstruction efficiency of electrons from heavy-flavour hadron decays were observed to be multiplicity dependent, while no dependencies were seen at high \pt, so the efficiencies cancelled out in the self-normalised ratios at high \pt. 

\begin{figure}[tb!]
 \begin{center}
      \includegraphics[width=0.48\linewidth]{figures/Results/HFE_pp_LowB/CompWithNormalB_LowB_logx.pdf}
      \includegraphics[width=0.48\linewidth]{figures/Results/HFE_pPb/TotalReconstructionEfficiency_CR1.pdf}
      \end{center}

\caption{Total reconstruction efficiency of electrons from heavy-flavour hadron decays using the TPC and TOF or the TPC and EMCal detectors in pp collisions at $\sqrt{s}= 13$ TeV with nominal and low magnetic field (left panel) and in \pPb collisions at $\sqrtsNN=8.16$ TeV (right panel).}        
\label{Fig:ppHFEEff}
\end{figure}
