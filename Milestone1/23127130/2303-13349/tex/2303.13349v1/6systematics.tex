\section{Systematic uncertainties}\label{sec:systematics}
The systematic uncertainties on the measured cross sections in pp and \pPb collisions were obtained separately for the different $p_{\rm{T}}$ intervals and for the different analyses performed using the TPC--TOF and TPC--EMCal detector combinations. 
For the self-normalised yield measurements, the systematic uncertainties were estimated directly on the self-normalised yield for each multiplicity class and $p_{\rm{T}}$ interval. The different sources of systematic uncertainties are discussed in this section and the assigned values are summarised in Tables~\ref{tab:SystematicSummaryTPCTOC}, ~\ref{tab:SystematicSummaryTPCEMCal}, and ~\ref{tab:summarysystSN}.


\begin{table}[th!]
 \centering
 \caption{Sources of systematic uncertainties and their assigned values in pp collisions at $\sqrt{s} = 13~\rm TeV$ for {$\mbox{$B = 0.2$ T}$} (0.2 $<p_{\rm T}$ $<$ 0.5 \GeVc) and $B = 0.5$ T (0.5 $<p_{\rm T}$ $<$ 4 \GeVc) data sets with the TPC and TOF  detectors, as well as with the TPC and EMCal detectors  (4 $<p_{\rm T}$ $<$ 35 \GeVc). The values presented as a range correspond to the lowest- and highest-\pt intervals. } 
 \renewcommand{\arraystretch}{1.5}
\begin{tabular*}{\textwidth}{@{\extracolsep{\fill}}  c| ccc}
   \hline
    \multirow{2}{*}{\makecell{Sources of systematic\\ uncertainties}} & { 0.2 $<p_{\rm T}$  $<$ 0.5 \GeVc} & { 0.5 $<p_{\rm T}$ $<$ 4 \GeVc} & { 4 $<p_{\rm T}$ $<$ 35 \GeVc}\\
 & \makecell{low-$B$\\TPC--TOF} & \makecell{nominal-$B$\\TPC--TOF} & \makecell{ nominal-$B$\\TPC--EMCal}\\

\hline
\hline
{Track selection } & {negl.} & 1$\%$ & 3$\%$ \\
\hline
{TPC--TOF matching} & { 4$\%$--2$\%$} & 2\% & N/A\\\hline  
{TPC--ITS matching } & { 2\%} & 3\%  &  3\%\\ \hline
{TPC--EMCal matching } & { N/A} & N/A  &   negl.\\ \hline
{SPD hit requirement } & {25$\%$--15$\%$} &  10\%--3\% & { 5\%--negl. }\\\hline
{Electron identification} & {5$\%$} &  {5$\%$--negl.}& 6\%--12\% \\\hline
%\multirow{2}*{Photonic Electron} & {} &  & \\
{Hadron contamination } & {negl.} & negl.--2\%  &  negl.--7\%  \\\hline 
\makecell{Photonic electron subtraction} & {20$\%$--11$\%$} & 7\%--1\%  &  negl.\\\hline  

{$\rm K_{e3}$ subtraction } & {15\%--1$\%$} & N/A & N/A \\\hline
 {$\rm{W}^{\pm}/\rm{Z}^0 \rightarrow$ e } & {N/A} &N/A &  negl.--8\%\\ \hline
 {$\pi^{0}$, $\eta$ weights } & { 3\%--1$\%$} & negl. &  negl.\\
\hline
{RF } & {N/A} & N/A & 3\%--4\% \\ \hline
{Luminosity } & {2.3\%} & {2.3\%} & {2.3\%--5\%} \\ \hline
\hline
{Total systematic} & {36\%--20$\%$} & {14\%--7\%} & {11\%--18\%}\\
\hline
  \end{tabular*}
% }
   \label{tab:SystematicSummaryTPCTOC}
\end{table}



\begin{table}[h!]
 \centering
 \caption{Sources of systematic uncertainties and their assigned values in \pPb collisions at {$\mbox{ $\sqrtsNN =8.16$ TeV}$} with TPC and TOF detectors (0.5 $<p_{\rm T}$ $<$ 4 \GeVc), as well as with the TPC and EMCal detectors  
 {$\mbox{(4 $<$ \pt $<$ 26 \GeVc)}$}. The values presented as a range correspond to the lowest- and highest-\pt intervals.}
  %\resizebox{\textwidth}{!}{  
  \renewcommand{\arraystretch}{1.4}
\begin{tabular*}{\textwidth}{@{\extracolsep{\fill}}  c| cc}
    \hline
   \multirow{2}{*}{\makecell{Sources of systematic\\ uncertainties}} & { 0.5 $<p_{\rm T}$ $<$ 4 \GeVc} & { 4 $<p_{\rm T}$ $<$ 26 \GeVc}\\
    & \makecell{nominal-$B$\\TPC--TOF} & \makecell{nominal-$B$\\TPC--EMCal} \\
\hline
\hline
{Track selection } & 1$\%$ & 5\%--1\% \\
\hline
{TPC--TOF matching} & 2\% & N/A \\
\hline  
{TPC--EMCal matching } & { N/A}  &  negl.\\ \hline
{TPC--ITS matching }&2\%&2\% \\ 
\hline
 {SPD hit requirement } & 10\%--3\% & 5\%--negl.\\
\hline
{Electron identification} & 3\%--1\% & 1\%--5\%\\
\hline 
{Hadron contamination } & negl. & negl.--5\% \\
\hline 
{Photonic electron subtraction} & 7\%--1\% &  negl. \\
 \hline  
 {$\rm K_{e3}$ subtraction }&N/A &N/A \\
\hline
{$\rm{W}^{\pm}/\rm{Z}^0 \rightarrow$ e } & N/A &  negl.--1\%\\ \hline
{$\pi^{0}$, $\eta$ weights } & negl. & negl. \\ \hline
{RF }& N/A & 2\%--4\% 
\\ \hline
{Luminosity } & {3\%} & {3\%} 
\\ 
\hline
\hline
{Total systematic}& 13$\%$--5$\%$ & 8$\%$--9$\%$ \\
\hline
  \end{tabular*}
 % }
   \label{tab:SystematicSummaryTPCEMCal}
\end{table}


The systematic uncertainty on the track reconstruction and selection efficiency was obtained by multiple variations of the track selection criteria, namely, the minimum number of space points in the TPC, the number of TPC crossed rows, the number of TPC d$E/$d$x$ clusters and the number of hits in the ITS.

The uncertainty due to an imperfect description in the simulation of the TPC--TOF (and the TPC--ITS) track matching was estimated by calculating the difference between efficiencies of the TPC--TOF (and the TPC--ITS) track matching in data and MC. To obtain the matching efficiency, the abundances of primary and secondary particles in data were estimated via template fits to the track impact-parameter distributions, and the relative abundances in the simulation were weighted to match those in data~\cite{ALICE:2017olh, ALICE-PUBLIC-2017-005}. For the low-$B$ field sample in pp collisions, the uncertainty on the track matching between the ITS and TPC is 2$\%$ at $\pt = 0.2$ \GeVc increasing up to $4\%$ at 4 \GeVc, whereas, the uncertainty on the track matching between the TPC and the TOF detector is $4\%$ at $\pt = 0.2$ \GeVc and $2\%$ at 4 \GeVc. In case of the nominal-$B$ field data set,  the uncertainty is about 2$\%$ for the TPC--TOF track matching and about 3$\%$ for the TPC--ITS track matching in the whole \pt range. In the \pPb analysis, the uncertainty for the TPC--TOF track matching, as well as the TPC--ITS track matching, was found to be  around 2$\%$ in the whole $\pt$ range.

The systematic uncertainty on the SPD hit requirement was obtained by varying the condition on the minimum number of hits and the specific layer of the SPD on which a hit was required for both the TPC--TOF and the TPC--EMCal analyses. For the low-$B$ field sample in pp collisions,  the systematic uncertainty due to the SPD hit requirement was about 25\% at \pt $= 0.2$ \GeVc decreasing to 15\% at $\mbox{0.5 \GeVc}$, whereas for nominal-$B$ field data sets in pp and \pPb collisions, the uncertainty is about 10\% at $\mbox{\pt $= 0.5$ \GeVc}$ decreasing to 5\% at 4 GeV/c and becomes negligible in the highest-\pt interval.


The uncertainty on the procedure of track matching to EMCal clusters was obtained by varying the $\Delta \eta - \Delta \varphi$ selection using \pt-independent thresholds ranging from 0.015 to 0.05 rad in $\eta$ and $\varphi$. The resulting uncertainty was found to be negligible.


The uncertainty on the electron identification originates from imprecisions in the description of the detector response in the MC, as well as from potential biases in the procedure employed to select electron candidates and to estimate the hadron contamination. It was studied by varying the electron identification selection  criteria on $n^{\rm{TPC}}_{\sigma,\rm{e}}$, $E/p$, and $\sigma^2_{\rm{long}}$. The assigned systematic uncertainties are listed as ``Electron identification" in the Tables~\ref{tab:SystematicSummaryTPCTOC},~\ref{tab:SystematicSummaryTPCEMCal}, and~\ref{tab:summarysystSN}.
The assigned systematic uncertainties vary from 5$\%$ to 12$\%$ 
depending on the \pt and the analysis method

 Additionally, the robustness of the fit procedure used to extract the hadron contamination in both  electron identification strategies was checked. In the TPC--TOF analysis, different analytical functions were utilised to parameterise the TPC ${\rm{d}}E/{{\rm{d}}x}$, which had negligible effects on the estimated hadron contamination up to a \pt of 3 \GeVc. In the TPC--EMCal analysis, the scaling region of the hadron $E/p$ distribution was varied. The
resulting uncertainties were found to be negligible at low \pt and of the order of 5\% at high \pt.

The uncertainty on the subtraction of photonic electrons is related to the efficiency of finding the partner electron and was studied by varying the selection of partner tracks, i.e. the number of TPC clusters used for (d$E/$d$x$) calculation and the minimum $p_{\rm{T}}$ requirement, as well as the selection on the invariant mass of $\rm e^{+}e^{-}$ pairs. 

The subtraction of $\rm K_{e3}$ decay electrons in pp collisions for $p_{\rm{T}} < 0.5$ GeV$/c$ can be affected by the uncertainty on the parameterisation of the ratio of $\rm K_{e3}$ to photonic electrons, and was found to result in an uncertainty of 15\% at $p_{\rm{T}}=0.2$~\GeVc and to be negligible at $p_{\rm{T}} \geq  0.5$ GeV$/c$. 

The uncertainty on the contribution of electrons from $\rm{W}^{\pm}$ and $\rm{Z}^0$ boson decays was estimated by
varying the yield of electrons from $\rm{W}^{\pm}$ and $\rm{Z}^0$ boson decays by 25\%. The strategy is imported from the most recent measurements from ALICE~\cite{Acharya:2019hao,ALICE:2019nuy}. The resulting uncertainty was found to be negligible for \pPb collisions and only relevant at very high $p_{\rm{T}}$ for pp collisions, where it amounts to about 8\%.

In the MC simulations, the $\pi^{0}$ and $\eta$ meson \pt distributions were weighted such that their measured \pt spectra are reproduced. The uncertainty from the measurements was propagated to the efficiency of finding the partner electron by parameterising the data along the upper and lower ends of their statistical and systematic uncertainties added in quadrature. The uncertainty was found to be about 3\% at $\mbox{$p_{\rm{T}} = 0.2~\GeVc$}$ in pp collisions and to be negligible at $p_{\rm{T}}$ $ \geq $ 0.5 GeV$/c$ in both pp and \pPb collisions. 

The systematic uncertainty on the trigger RF, as explained in Sec.~\ref{ssec:TriggerRejectionFactor}, was propagated on the $p_{\rm{T}}$-differential cross section of electrons from heavy-flavour hadron decays. The uncertainty was of the order of 4\% at the highest \pt. 

The systematic uncertainty on the luminosity was propagated on the $p_{\rm{T}}$-differential cross section of electrons from heavy-flavour hadron decays. The uncertainty was 2.3\% to 5\% in pp collisions depending on the \pt, as the uncertainty from the rejection factors for triggered samples were taken into consideration, and 3\% in \pPb collisions, where the uncertainty from the rejection factors contributed negligibly to the uncertainty on luminosity. 
 
As the geometrical acceptance and reconstruction efficiencies are essentially independent of \dnchdeta in the measured multiplicity range, these corrections and their corresponding systematic uncertainties largely cancel in the ratio to the multiplicity-integrated yield, thus resulting in a lower systematic uncertainty for self-normalised yields compared to the one for the $p_{\rm{T}}$ spectra. 


The total systematic uncertainties on the $p_{\rm{T}}$ spectra and the self-normalised yields were calculated by summing the different contributions in quadrature, as they are considered to be uncorrelated. 
  

\begin{table}[h!]
\centering
\caption{Systematic uncertainty on self-normalised yield in pp collisions at $\sqrt{s} = 13~\rm TeV$ and \pPb collisions at $\sqrt{s_{\rm NN}} = 8.16~\rm TeV$.  }
	\renewcommand{\arraystretch}{1.7}
% \resizebox{\textwidth}{!}{  
 
  \begin{tabular*}{\textwidth}{@{\extracolsep{\fill}} l|c|ccc|ccc}
    \hline
     \multicolumn{1}{c}{ }& \multicolumn{1}{c|}{ }& 
      \multicolumn{3}{c|}{pp, $\sqrt{\rm s}$ = 13 TeV}&
       \multicolumn{3}{c}{\pPb, $\sqrt{s_{\rm NN}}$ = 8.16 TeV}\\
      
       %\chline{lr}{2-6}
    \hline
 
\multicolumn{1}{c|}{} & \multicolumn{1}{c|}{Multiplicity} &
   \multicolumn{3}{c|}{\pt interval (\GeVc)} & \multicolumn{3}{c}{\pt interval (\GeVc)} \\
   [-0.5ex]
 \multicolumn{1}{c|}{} &   \multicolumn{1}{c|}{intervals} &
   \multicolumn{1}{c}{0.5--6} & \multicolumn{1}{c}{6--12} & \multicolumn{1}{c|}{15--30} & \multicolumn{1}{c}{0.5--6} & \multicolumn{1}{c}{6--8} & \multicolumn{1}{c}{14--26} \\
 \hline\hline
 \multirow{3}{*}{Track selection}& {I} & { negl. }  & {2$\%$} & {2$\%$} & {negl.}& {3$\%$}& {3$\%$} \\
& {III}  & {negl.} & {2$\%$} & {2$\%$} & {negl.}& {3$\%$}& {3$\%$} \\
{}& {V}  & {negl.} & {2$\%$} & {2$\%$} & {negl.}& {4$\%$}& {4$\%$} \\
 \hline
 \multirow{3}{*}{SPD hit requirement} & {I} &
{ 10$\%$}  &{6$\%$} & {14$\%$} & {10$\%$}& {2$\%$}& {10$\%$} \\
& {III}  & {3$\%$} & {6$\%$} & {6$\%$} & {2$\%$}& {2$\%$}& {2$\%$} \\
{ }& {V}  & {4$\%$} & {6$\%$} & {6$\%$} & {2$\%$}& {2$\%$}& {2$\%$} \\
\hline
\multirow{3}{*}{Electron identification}& {I} &
{ 1$\%$}  & {3$\%$} &  {3$\%$} &  {1$\%$}&  {2$\%$}&  {2$\%$} \\
 {}& {III}  &  {1$\%$} &  {3$\%$} &  {3$\%$} &  {1$\%$}&  {2$\%$}&  {2$\%$} \\
 {}&  {V}  &  {1$\%$} &  {3$\%$} &  {3$\%$} &  {1$\%$}&  {4$\%$}&  {4$\%$} \\
 \hline
 \multirow{3}{*}{\makecell{Photonic electron\\subtraction}}& {I} &
{ 1$\%$}  & {1$\%$} &  {2$\%$} &  {1$\%$}&  {1$\%$}&  {1$\%$} \\
&  {III}  &  {1$\%$} &  {2$\%$} &  {2$\%$} &  {1$\%$}&  {1$\%$}&    {1$\%$} \\
& {V}  & {1$\%$} & {2$\%$} & {2$\%$} & {1$\%$}& {1$\%$}& {1$\%$} \\
 \hline\hline
 \multirow{3}{*}{Total systematics}&{I} &
{ 10$\%$}  &{7$\%$} & {15$\%$} & {10$\%$}& {4$\%$}& {11$\%$} \\
& {III}  & {3$\%$} & {7$\%$} & {7$\%$} & {2$\%$}& {4$\%$}&{4$\%$} \\
{ }& {V}  & {4$\%$} & {7$\%$} & {7$\%$} & {2$\%$}&{6$\%$}&{6$\%$} \\
   \hline
  \end{tabular*}
 % }
   \label{tab:summarysystSN}
\end{table}
