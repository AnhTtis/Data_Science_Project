\section{Summary}\label{section:summary}
Heavy-flavour production at midrapidity was studied using electrons from heavy-flavour hadron decays in pp collisions at \sqrts $= 13~\rm \TeV$ and in \pPb collisions at \sqrtsNN $= 8.16~\rm TeV$ with the ALICE detector at the LHC. The \pt-differential production cross section of electrons from heavy-flavour hadron decays in pp collisions was compared with FONLL and GM-VFNS (b $\rightarrow$ B $\rightarrow$ D $\rightarrow$ e, b $\rightarrow$ B $\rightarrow$ e, c  $\rightarrow$ D $\rightarrow$ e) pQCD calculations. The data are observed to lie on the upper edge of the FONLL uncertainties. The GM-VFNS calculation underestimates the cross section at low \pt but describes the data within the uncertainties for \pt $>$ 5 \GeVc. The nuclear modification factor in \pPb collisions, $R_{\rm{pPb}}$, was computed and is consistent with unity within the statistical and systematic uncertainties. The $R_{\rm{pPb}}$ measurement shows no effects that could signal the formation of a hot medium and no significant cold nuclear matter effects within the uncertainties of the data in the measured \pt range. The $R_{\rm{pPb}}$ at \sqrtsNN $= 8.16~\rm TeV$ is consistent with that measured at \sqrtsNN $= 5.02~\rm TeV$. 

The multiplicity-dependent production of electrons from heavy-flavour hadron decays was measured using the self-normalised yield as a function of self-normalised charged-particle pseudorapidity density at midrapidity in pp and \pPb collisions as a function of transverse momentum. A faster than linear increase was observed in both pp and p--Pb collisions. While in \pPb collisions, no \pt dependence is observed within uncertainties, in pp collisions a strong \pt dependence is seen with high-\pt electrons showing a faster increase as a function of the self-normalised multiplicity. The measurement of self-normalised yield of electrons from heavy-flavour hadron decays in pp collisions was compared with PYTHIA 8.2 and EPOS 3 simulations, which describe the data. The measurement in \pPb collisions was compared with the EPOS 2.592 model without hydrodynamics, which underestimates the data. The comparison of self-normalised yields of heavy-flavour and light-flavour particles show a similar stronger than linearly increasing trend in both colliding systems. In pp collisions the stronger than linear increase of heavy-flavour particles is mainly driven by auto-correlation effects which are independent of particle species, whereas in the case of \pPb collisions, it is difficult to draw any conclusion since the multiplicity dependence of heavy-flavour production is also largely affected by the presence of multiple binary nucleon–nucleon interactions. 


