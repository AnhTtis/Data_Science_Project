\documentclass[prd,nofootinbib,twocolumn,superscriptaddress]{revtex4}
%\documentclass[review,12pt]{elsarticle}

\usepackage[colorlinks=true,filecolor=cyan,linkcolor=cyan,urlcolor=cyan,citecolor=cyan]{hyperref}

\usepackage{amsfonts,amssymb,amsmath}
\usepackage{epsfig}
\usepackage{mathrsfs}
\usepackage{amssymb}
\usepackage{graphicx}
\graphicspath{ {./images/} }
\usepackage{amsmath}
\usepackage{xcolor}
\usepackage{color}
%\usepackage[toc,page,title,titletoc,header]{appendix}


\newcommand{\TT}[1]{{\color{red}{#1}}}	
\newcommand{\TTT}[1]{{\color{blue}{#1}}}		
\setlength{\paperheight}{11in}

\begin{document}



\title{ Microlensing and multi-images problem of static spherical symmetric wormhole}
\author{Ke Gao}
\email{2021700389@stu.jsu.edu.cn}
\author{Lei-Hua Liu}
\email{liuleihua8899@hotmail.com}


\affiliation{Department of Physics, College of Physics, Mechanical and Electrical Engineering, Jishou University, Jishou 416000, China}




\begin{abstract}
Recently, the gravitational microlensing effect of wormholes has made great progress.
In this paper, we develop a framework for investigating the weak lensing (including the microlensing) effects of the static spherical symmetric wormhole (WH) in terms of the radial equation of state $\eta=\frac{p_r}{\rho}$ (REoS). As for its application, we calculate its lens equation, magnification, and event rate under this REoS, in which we find the relationship between the n-th order lens equation and the radial state coefficient (RSC), $n=2+\frac{1}{\eta}$. In addition, we obtain an analytical expressions for the magnification and the event rate of individual WH source, in which all these metrics belong to the static spherical symmetric WH metric. By calculating the lensing equation of the static spherical symmetric WH, through these two analytical expressions, we discuss the effects of RSC and WH throat radius on magnification and event rate. Our results show that larger throat radius and RSC correspond will lead to the larger values for magnification and event rate. It is even claimed that the event rate of WH will be larger comparing to WH's case under similar conditions. Our results can provide a basis for finding WHs in observations.

\end{abstract}

\maketitle

\section{Introduction}
\label{introduction}


The research on WHs has a history of over a hundred years. In 1916, Flamm studied the internal structure of Schwarzschild’s solution \cite{Flamm:1916}. In 1936, Einstein and Rosen proposed the concept of a bridge structure, which paved the way for further understanding of WHs \cite{Einstein:1935}. The WH, this term, was first introduced by Misner and Wheeler \cite{Misner:1957}. Ellis proposed the idea of a drain hole structure \cite{Ellis:1973}. Then, Morris proposed the concept of a traversable WH \cite{Morris:1988cz}, in which a type of WH that could be traversed by humans or spaceships without being destroyed. These advancements in the understanding of WHs have contributed greatly to the field of theoretical physics. Thereafter, WHs became the subject of extensive research \cite{Damour:2007ap,Kim:2003zb,Bueno:2017hyj,Visser:1989kh,Sushkov:2005kj,Bronnikov:2002rn,Clement:1995ya,Richarte:2017iit,Ayuso:2020vuu,KordZangeneh:2020ixt,Song:2023jdn,Saleem:2023lul,Eid:2023wrd,Godani:2023paj} by physicists. The study of WHs typically begins with an assumption about their geometric structure and then involves with calculating the corresponding material source required to create such a structure. However, such calculations often violate known energy conditions \cite{Hochberg:1998ii,Hochberg:1998ha,Lobo:2002zf,Lobo:2004rp}, indicating that some exotic matter is required to explain the existence of WHs. The study of WHs can deepen our understanding of General rRelativity to explore the Universe.

In 1936, Einstein published the first article on gravitational lens field \cite{Einstein:1936llh}. After a period of silence, gravitational lens has become a hotspot. Gravitational lensing can be classified into two types: weak gravitational lensing including the microlensing \cite{Bartelmann:1999yn,Kaiser:1991qi} and strong gravitational lensing \cite{Bozza:2002zj,Virbhadra:1999nm}. Weak gravitational lensing is caused by relatively small perturbations in the gravitational field and results in slight distortions of images, while strong gravitational lensing involves more significant deformations due to the presence of massive objects like black holes, galaxies or galaxy clusters. Gravitational microlens is an effective method to explore WHs \cite{Perlick:2004tq,SDSS:2002oin}. In the literature, The microlensing effect of a WH is extensively studied \cite{Gao:2022cds,Liu:2022lfb,Sokoliuk:2022owk,Zatrimaylov:2021ijd,Cheng:2021hoc,Li:2019qyb,Kuniyasu:2018cgv,Raidal:2018eoo,Tsukamoto:2017hva,Sajadi:2016hko,Lukmanova:2016czn,Tsukamoto:2016zdu,Kitamura:2016vad,Kitamura:2012zy,Kitamura:2012wcg,Toki:2011zu,Abe:2010ap,Bogdanov:2008zy,Torres:2001gb,Safonova:2001vz,Torres:1998cu,Torres:1998xd}. An interesting question arises as to how many images a light source will generate at the equatorial plane when passing through the WH spacetime, and what this quantity is related to. There has been some discussions about microlensing imaging in WHs spacetime \cite{Kitamura:2016vad,Kuniyasu:2018cgv,Abe:2010ap,Liu:2022lfb}. This issue will be further discussed in our research.
Previously, we discussed the magnification of WHs and compared them with black holes \cite{Gao:2022cds}. In this research, we will extend the event rate introduced into the industrial chain, attempting to discover new physics from it.

In this paper, we will re-examine the problem of lensing effects in spherically symmetric WH metrics from a new perspective by implementing the REoS. The deflection angle is calculated using the Gauss-Bonnet Theorem (GBT), which is widely used in gravitational lenses \cite{Gibbons:2008rj,Gibbons:2008zi,Werner:2012rc,He:2023hsv,Upadhyay:2023yhk,Javed:2023iih,Javed:2022gtz,Huang:2022soh,He:2022yhp,Gao:2023ltr,Cai:2023ite,Ovgun:2023ego}. We find that n-th order lens equation is related to RSC, $n=2+\frac{1}{\eta}$, and its application will be  discussed in detail in Appendix \ref{number of images}. Although we have not solved the multi-image problem, we estimate the maximum number of images that can be measured. We also calculate the magnification and obtain a general formula. Through this formula, we discuss the magnification in terms of RSC and WH throat radius. Additionally, we analytically process the event rate of a single WH source and obtain a geometric representation. Furthermore, we discuss the effects of RSC and WH throat radius on event rates. Our work contributes to the study of WHs and provides new ideas for future research in this field.


The structure of our article is organized as follows: In section \ref{wormhole}, we use REoS to construct a metric. Section \ref{microlensing}, we use the gravitational lens technology to calculate the deflection angle and lens equation and obtain the n-th order lens equation. Besides, magnification and event rate are also discussed. In Section \ref{conclusion and outlook}, we draw our conclusion and outlook. In section \ref{number of images}, we discuss the image number equation in specific examples. 







\section{Bascis of WH}
\label{wormhole}
In this section, we follow the notation of \cite{Lobo:2005us, Lobo:2005yv, Garattini:2007ff}. By starting with a static spherical symmetry metric:
\begin{equation}
\label{eq1}
ds^2=-e^{2\Phi}dt^2+\frac{dr^2}{1-b(r)/r}+r^2d\Omega^2,
\end{equation}
this metric describes a generic  static and spherical WH metric. The Einstein field equation provides the following relationships:
\begin{equation}
\label{eq2}
p_r^\prime=\frac{2}{r}\big(p_t-p_r\big)-\big(\rho+p_r\big)\Phi^\prime,
\end{equation}
\begin{equation}
\label{eq3}
b^\prime=8\pi G\rho(r)r^2,
\end{equation}
\begin{equation}
\label{eq4}
\Phi^\prime=\frac{b+8\pi Gp_rr^3}{2r^2\big(1-b(r)/r\big)},
\end{equation}
where the prime denotes a derivative with respect to the radial coordinate $r$, $p_r$ represents the radial pressure, $p_t$ indicates the tangential pressure, and $\rho$ is the energy density. REoS is defined as follows,
\begin{equation}
\label{eq5}
p_r=\eta \rho.
\end{equation}
where $\eta$ represents RSC. The flaring-out condition and asymptotic flatness take the neccessary condition:
\begin{equation}
\eta>0 \,\, \text{and} \,\, \eta<-1.
\label{flaring out condition}
\end{equation}
Combining these equations \eqref{eq2}-\eqref{eq5}, we can get
\begin{equation}
\begin{aligned}
b(r) = r_0\bigg(\frac{r_0}{r}\bigg)^\frac{1}{\eta}e^{-(2/\eta)[\Phi(r)-\Phi(r_0)]}\times
\\
\bigg[\frac{2}{\eta} \int_{r_0}^r\big(\frac{r}{r_0}\big)^{(1+\eta)/\eta}\Phi^\prime(r)
e^{(2/\eta)[\Phi(r)-\Phi(r_0)]}dr+1\bigg].
\end{aligned}
\end{equation}
Since our research only considers gravitational microlensing, we adopt weak field approximation in which $\Phi(r)\approx constant$, then,
\begin{equation}
b(r)=r_0\bigg(\frac{r_0}{r}\bigg)^{\frac{1}{\eta}}.
\end{equation}
Substitute the above formula into Eq. \eqref{eq1}, then one can get 
\begin{equation}
\label{eq7}
ds^2=-Adt^2+\frac{dr^2}{1-\big(r_0/r\big)^{1+\frac{1}{\eta}}}+r^2d\Omega^2,
\end{equation}
where $A=e^{2\Phi}$. In the next section, we will use this metric to discuss microlensing effect. If we set $\eta=1$ and absorb $A$ into the time coordinate, metric \eqref{eq7} will become an Ellis-Bronnikov WH.





\begin{figure}
    \centering
    \includegraphics[scale=0.7]{wormhole_image_1.png}
    \caption{a brief sketch of a WH. A WH can be visualized as a tunnel-like structure connecting two separate regions of spacetime. In our case, we consider the lensing effects occurring on one side of the WH, either in spacetime 1 or spacetime 2. }
    \label{fig:my_label}
\end{figure}





\section{Microlensing}
\label{microlensing}
In this section, we will calculate the magnification and the event rate of metric \eqref{eq7}. Under REoS,  we also find an explicit relation between the n-th order lens equation of WH and RSC. 






\subsection{Deflection angle}
In microlensing, The deflection angle of a light will be calculated using the GBT under weak field approximation. 
For a photon $ds^2=0$, it (in equatorial plane) will satisfy the following relation as 
\begin{equation}
\label{eq 8}
dt^2=\frac{dr^2}{A\left(1-\big(r_0/r\big)^{1+\frac{1}{\eta}}\right)}+\frac{r^2}{A}d\phi^2.
\end{equation}
Then we define two auxiliary quantities as $du=\frac{dr}{\sqrt{A\big(1-\big(r_0/r\big)^{1+\frac{1}{\eta}}\big)}}$ and $\xi=\frac{r}{\sqrt{A}}$. Gaussian optical curvature can be expressed as
\begin{equation}
K=\frac{-1}{\xi(u)}[\frac{dr}{du}\frac{d}{dr}\big(\frac{dr}{du}\big)\frac{d\xi}{dr}+\big(\frac{dr}{du}\big)^2\frac{d^2\xi}{dr^2}],
\end{equation}
combine with metric \eqref{eq 8}, one can get
\begin{equation}
\label{eq10}
K=\frac{-\sqrt{A}r_0\big(\frac{r_0}{r}\big)^\frac{1}{\eta}\big(1+\frac{1}{\eta}\big)}{2r^3\sqrt{1-\big(\frac{r_0}{r}\big)^{1+\frac{1}{\eta}}}}.
\end{equation}
The expression for deflection angle is derived from GBT.
We first give the GBT
\begin{equation}
\int\int_D KdS+\int_{\partial D}\kappa dt+\sum\limits_i\alpha_i=2\pi\chi(D).
\end{equation}
One choose the integral domain $D$ in Fig. \ref{fig:1},  $OS$ is the geodesic line, so the line integral on $OS$ is zero. Besides, this
Euler index $\chi$ in domain $D$ is one.
\begin{equation}
\int\int_{D}KdS+\int_{\gamma_P}\kappa dt+\sum_i\alpha_i=2\pi.
\end{equation}
We can set $\gamma$ to vertically intersect the geodesic line $OS$ at point O and point S, which means
\begin{equation}
\sum_i\alpha_i=\frac{\pi}{2}(S)+\frac{\pi}{2}(O)=\pi.
\end{equation}
The sum of external angles here is the sum of two right angles, And then we do an integral transformation
\begin{equation}
\kappa dt=\kappa\frac{dt}{d\phi}d\phi.
\end{equation}
Here $\phi$ is the angular coordinate of the center at $wormhole$. It can be done to set up $\kappa\frac{dt}{d\phi}=1$ on $\gamma$, so there is
\begin{equation}
\int\int_{D}KdS+\int_{\phi_O}^{\phi_S}d\phi+\pi=2\pi.
\end{equation}
Geodesic $OS$ is approximately a straight line, that is, the angle that $OS$ spans is $\pi+\alpha$, and we let the angular coordinate of point O be 0. The result is
\begin{equation}
\int\int_{D}KdS+\int_{0}^{\pi+\alpha}d\phi+\pi=\int\int_{D}KdS+\pi+\alpha+\pi=2\pi.
\end{equation}
The final result is
\begin{equation}
\alpha=-\int\int_{D}KdS.
\end{equation}
That is to say, our deflection angle can be written as
\begin{equation}
\label{eq11}
\alpha=-\int_0^\pi\int_{\frac{b}{\sin\phi}}^\infty K\sqrt{\det  h_{ab}} drd\phi,
\end{equation}
where $b$ is impact parameter and $h_{ab}$ is metric \eqref{eq 8}. Note that our results only apply to small angles. Substitute formula \eqref{eq10} to \eqref{eq11}, one can obtain
\begin{equation}
\label{eq 22}
\alpha=\int_0^\pi\int_{\frac{b}{\sin\phi}}^\infty\frac{r_0\big(\frac{r_0}{r}\big)^\frac{1}{\eta}\big(1+\frac{1}{\eta}\big)}{2\sqrt{A}r^2\big(1-\big(\frac{r_0}{r}\big)^{1+\frac{1}{\eta}}\big)}drd\phi.
\end{equation}
Under the weak field approximation, we work out
\begin{equation}
\label{eq13}
\alpha=\frac{\sqrt{\pi}\big(\frac{r_0}{b}\big)^{1+\frac{1}{\eta}}\eta\Gamma[1+\frac{1}{2\eta}]}{2\sqrt{A}\Gamma[\frac{1}{2}\big(3+\frac{1}{\eta}\big)]}, ~~~\text{if } \frac{1}{\eta}>-2.
\end{equation}
 Being armed with deflection angle \eqref{eq13}, one can investigate its corresponding lens equation. 


\begin{figure}
    \centering
    \includegraphics[scale=0.8]{gauss_bonnet_1.png}
    \caption{Illustration of the GBT integral domain. O is the observer, and S is the source. The region D represents the integral domain of the GBT, and integrating over this domain gives us the total deflection angle experienced by the light.}
    \label{fig:1}
\end{figure}


\subsection{Lensing equation}
The lens equation provides the relationship between the image, source, and deflection angle.
The plane geometry in equatorial plane of lens, Fig. \ref{fig:2}, tells us
\begin{equation}
\label{eq14}
\beta=\theta-\frac{D_{LS}}{D_S}\alpha.
\end{equation}
Substitute Eq. \eqref{eq13} to Eq. \eqref{eq14}, one can obtain that
\begin{equation}
\theta^{2+\frac{1}{\eta}}-\beta\theta^{1+\frac{1}{\eta}}-\frac{D_{LS}}{D_S}\frac{\sqrt{\pi}\big(\frac{r_0}{D_L}\big)^{1+\frac{1}{\eta}}\eta\Gamma[1+\frac{1}{2\eta}]}{2\sqrt{A}\Gamma[\frac{1}{3}\big(3+\frac{1}{\eta}\big)]}=0,
\label{general lens eq}
\end{equation}
here we have used the approximation $b\approx\theta D_L$. According to Eq. \eqref{general lens eq}, one can explicitly obtain the relation between the order of lensing equation (n) and $\eta$, 
\begin{equation}
\label{eq 16}
n=2+\frac{1}{\eta}.
\end{equation}
This is an equation about the number of images (If there are at most n solutions on the inference of the n order of equation). When $\eta\rightarrow 0_+$, then $n\rightarrow\infty$, this means that we can at most get an infinite number of images. On the other hand, when $\eta\rightarrow\pm\infty$, had $n\rightarrow 2$, we are expected to have two images. With $\beta=\theta+\delta$, we solve Eq. \eqref{general lens eq} as

\begin{equation}
\theta=\left(-\frac{1}{\delta}\frac{D_{LS}}{D_S}\frac{\sqrt{\pi}\big(\frac{r_0}{D_L}\big)^{1+\frac{1}{\eta}}\eta\Gamma[1+\frac{1}{2\eta}]}{2\sqrt{A}\Gamma[\frac{1}{3}\big(3+\frac{1}{\eta}\big)]}\right)^{\frac{\eta}{\eta+1}}.
\end{equation}
In some cases, there may be multiple solutions to the lensing equation, which are known as multi-image problems. This occurs because photons have different impact parameters $b$, which are related to the energy $E$ and angular momentum $J$ of photons, with $b=\frac{J}{E}$, and $\delta\approx\frac{D_{LS}}{D_S}\alpha(b)$. From an observational perspective, only the real solution is applicable. However, the second order of the lensing equation could have complex solutions. The situation becomes even more complicated for higher order lensing equations, where finding the exact real solution is not always possible. Therefore, the image number equation serves as a guide for exploring image problems in various WHs (as discussed in Sec. \ref{number of images}).
Alternatively, if we detect the maximum number of images from a given source, we can deduce the value of RSC for the WH from the Eq. \eqref{eq 16}. This method is particularly useful when dealing with quantized WHs since calculating RSC directly would yield a variable value. However, by determining the value of RSC from the maximum number of observed similar images, we can obtain a definite value for it. Furthermore, knowing the value of RSC (for example, $\eta=0.5$ for observing up to four images) can aid in exploring the magnification and event rate in subsequent studies.





\begin{figure}
    \centering
    \includegraphics[scale=0.8]{lens_geometry.png}
    \caption{Showing lens plane geometry. $I$ is the location of images, $S$ is location of source, $\alpha$ is the deflection angle, $W$ is the  WH and $b$ is the impact parameter. $\alpha$ is the deflection angle. All of these angles are much less than unity.}
    \label{fig:2}
\end{figure}


\subsection{Magnification}
Magnification is defined as the inverse of the Jacobian matrix, which describes how the lensing distorts the image of the source. When light passes through a lens, the solid angle element $d\beta^2$ is transformed into the solid angle $d\theta^2$, which ultimately affects the observed solid angle under which the source is viewed. This shift in solid angle results in a magnification or demagnification of the received flux. In axisymmetric space, the total magnification can be calculated using the following formula:

\begin{equation}
\label{eq 27}
\mu_{\rm total}=\sum_i \big|\frac{\beta}{\theta_i}\frac{d\beta}{d\theta_i}\big|^{-1},
\end{equation}
where $\theta_i$ is the angle of $i-th$ image of sources. 
substituting Eq. \eqref{eq14} into Eq. \eqref{eq 27}, which leads to


\begin{widetext}
\begin{equation}
\mu=\left|\frac{\pi  D_L D_{LS} 2^{-\frac{1}{\eta }-2} r_0 \left(\frac{r_0}{b}\right)^{1/\eta } \left(\sqrt{A} b^2 D_S\Gamma \left(2+\frac{1}{\eta }\right)-D_L D_{LS} 2^{1/\eta } \eta  (\eta +1) r_0 \Gamma \left(1+\frac{1}{2 \eta }\right)^2 \left(\frac{r_0}{b}\right)^{1/\eta }\right)}{A b^4 D_S^2 \Gamma \left(\frac{1}{2} \left(3+\frac{1}{\eta }\right)\right)^2}+1\right|^{-1}.
\label{total mag}
\end{equation}

\end{widetext}
Equation \eqref{total mag} provides a general formula for calculating the magnification in WH microlensing. Due to the flaring-out condition and asymptotic flatness condition of WHs, RSC can be divided into two intervals: $(-\infty,-1)$ and $(0,\infty)$. We have chosen specific values of RSC, such as $\eta=0.33$ corresponding to $n=5$, $\eta=0.5$ corresponding to $n=4$, $\eta=1$ corresponding to $n=3$, etc., according to the equation $n=2+\frac{1}{\eta}$, to investigate the effect of different $n$ values on the magnification curve. Our results FIG. \ref{fig: 4} and FIG. \ref{fig: 5} show that the shape of the magnification curve of various n values is similar, with single peaks and no occurrence of multiple peaks. A common feature of these curves is that the left side of the peak on the magnification curve gradually contracts to zero, while the right side of the peak on the magnification curve tends to one. When the magnification is equal to one, the light is not deflected. The value of RSC significantly impacts the position of the peak in the magnification curve. For cases where RSC is greater than zero, the larger the value of RSC, the greater the impact parameter corresponding to the peak in the magnification curve. The same applies for cases where RSC is less than -1. This effect may be due to the influence of RSC on mass, which causes a change in the peak position.
TABLE \ref{table:1} shows that $\eta=1$ corresponds to a mass of 109422~$M_\odot$ (see \ref{event rate 1} for details), while $\eta=2$ corresponds to a mass of $1.09422\times 10^8~M_\odot$. The impact of RSC on mass is significant.
We also investigated the effect of the WH throat radius on magnification by fixing $\eta=1$. In order to maintain small angle approximation, we set a safe range of $0.00001\sim 0.00004$ kpc for $r_0$, since $r_0$ not only affects mass, but also is a geometric quantity on the lens plane. Our results FIG. \ref{fig: 6} show that the larger the WH throat radius, the greater the value of $b$ corresponding to the magnification peak. We also observed a peak of demagnification, which is the case where the magnification is less than one. It appears to the right of the peak magnification, closely adjacent to a peak of demagnification. Compared to FIG. \ref{fig: 4}, we found that the demagnification peak is highly dependent on RSC, but not significantly dependent on the WH throat radius. When RSC is less than one and greater than zero, this demagnification peak becomes apparent.
In the literature \cite{Johnson:2019ljv,Gralla:2019drh}, it has been pointed out that Kerr black holes also exhibit demagnification phenomena of images, and this demagnification is related to the spin of the black hole, which can be used as a method for determining spin. 




\begin{figure}
    \center
    \includegraphics[scale=0.55]{MG1.pdf}
    \caption{The case of $\eta>0$, the magnification given by Eq. \eqref{total mag} varies with the impact parameters $b$ (Unit: kpc), and the legend in the figure indicates the corresponding $\eta$ values for different curves. Here we set the parameters as follows: $D_S=2D_L=2D_{LS}=20$ kpc, $r_0=1\times 10^{-5}$ kpc, and $A=1$. According to $n=2+\frac{1}{\eta}$, $\eta=0.33$ corresponding $n=5$, $\eta=0.5$ corresponding $n=4$, $\eta=1$ corresponding $n=3$, $\eta=5$ corresponding $n\approx 2$.}
    \label{fig: 4}
\end{figure}

\begin{figure}
    \centering
    \includegraphics[scale=0.74]{MG2.pdf}
    \caption{The case of $\eta<-1$, the magnification given by Eq. \eqref{total mag} also varies with the impact parameters $b$, and the legend in the figure indicates the corresponding $\eta$ values for different curves. We set the parameters as follows: $D_S=2D_L=2D_{LS}=20$ kpc, $r_0=1\times 10^{-5}$ kpc, and $A=1$. The curves where $\eta=-5$ and $\eta=-10$ have already overlapped. The case where $\eta=-1$ is actually an extreme value, corresponding to $n=1$. }
    \label{fig: 5}
\end{figure}

\begin{figure}
    \centering
    \includegraphics[scale=0.74]{MG3.pdf}
    \caption{The magnification varies with the radius of the WH throat $r_0$. The various values of $r_0$ correspond to different curves indicated in the legend. We set the parameters under the Planck system of units as $D_S=2D_L=2D_{LS}=20$ kpc, $\eta=1$, $A=1$.}
    \label{fig: 6}
\end{figure}

\subsection{Event rate}
\label{event rate 1}
The microlensing effect is a rare phenomenon in observations. To describe the probability of this event, we need to calculate the optical depth $\tau$. The optical depth represents the probability of observing the microlens effect of a source at a certain location $D_S$, which reflects the number of observable microlensing events per unit time. If we observe N sources, we can calculate the microlensing event rate $\Gamma=\frac{d (N \tau)}{dt}$. There have been numerous studies on numerical processing of event rates. However, we aim to develop an analytical approach to the event rate of a single WH as a lens. The geometric image of this situation is shown in Fig. \ref{fig: 6}. We begin with the WH metric \eqref{eq7}. The effective mass of WHs can be calculated as follows:
\begin{equation}
M=\frac{r_0}{2}+\int_{r_0}^r4\pi\rho(r^\prime)r^{\prime 2}dr^\prime,
\end{equation}
where the energy density in metric \eqref{eq7} is
\begin{equation}
\rho=-\frac{Ar_0(\frac{r_0}{r})^{\frac{1}{\eta}}}{r^3\eta}\frac{c^4}{8\pi G}.
\end{equation}
Therefore,
\begin{equation}
M=\frac{A c^4r_0\left(\frac{r_0}{D_L}\right)^{\frac{1}{\eta} }}{2 G}-\frac{A c^4 r_0}{2 G}+\frac{r_0}{2},
\end{equation}
where the distance between us as observer and the WH is $D_L$. Using the lens equation, we can solve
Einstein angle $\theta_E=\frac{D_{LS}}{D_S}\alpha$:
\begin{equation}
\theta_E=\left(\frac{r_0}{D_L}\right)^{\frac{1+\frac{1}{\eta}}{2+\frac{1}{\eta}}}\left(\frac{\sqrt{\pi}\eta\Gamma[1+\frac{1}{2\eta}]}{2\sqrt{A}\Gamma[\frac{1}{2}(3+\frac{1}{\eta})]}\frac{D_{LS}}{D_S}\right)^{\frac{1}{2+\frac{1}{\eta}}}.
\end{equation}
The Einstein ring is a photon ring observed when the source is at position $\beta=0$, where the light source, lens, and observer are aligned,
and Einstein ring is generally assumed to be the cross-section for microlensing
\begin{equation}
\sigma_{micro}=\pi\theta^2_E.
\end{equation}
This is the solid angle within which a source has to be placed in order to produce a detectable microlensing signal. Due to the relative motion between the source and lens, we need to define a physical quantity to measure.
The Einstein radius crossing time is
\begin{equation}
\label{te}
t_E=\frac{r_E}{v}=\frac{D_S\theta_E}{v}
\end{equation}
where $v$ is the relative velocity of the source and lens, as shown in Fig. \ref{fig: 7}.
We can make it clearer by actually considering that the source is fixed on the background while the lens moves at velocity $v$. This velocity is in the plane direction.
The optical depth $\tau$ to some distance $D_S$ is the probability that a source at that distance gives rise to a detectable microlensing event. 
\begin{equation}
\tau=\frac{1}{\Omega}\int_0^{D_S}\sigma_{micro}dN_L.
\end{equation}
we assume that the number density of lenses varies as a function of the lens distance
as $n(D_L)$. Then, the number of lenses within the solid angle $\Omega$ at distances between $D_L$ to $D_L+dD_L$ is
\begin{equation}
dN_L=\Omega D_L^2n(D_L)dD_L.
\end{equation}
Then, the optical depth is
\begin{equation}
\tau(D_S)=\frac{1}{\Omega}\int_0^{D_S}[\Omega D_L^2n(D_L)](\pi\theta_E^2)dD_L.
\end{equation}
Because we only consider one WH, we have $n(D_L)=\frac{\rho}{M}$. Thus,
\begin{equation}
\tau(D_S)=\int_0^{D_S}D_L^2  \frac{\rho(D_L)}{M}\pi \theta_E^2dD_L.
\end{equation}
The integration interval is $(0, D_S)$, which is approximately equal to $(D_L,r_0)\cup(r_0,D_{LS})$, based on the small angle approximation $\theta(r_E)\gg \theta(r_0)$.  On the other hand, we set parameter $c=G=A=1$, eventually, the integration result is


\begin{equation}
\label{eq. 39}
\left|
\frac{\left(\frac{\sqrt{\pi}\eta\Gamma[1+\frac{1}{2\eta}]}{2\Gamma[\frac{1}{2}(3+\frac{1}{\eta})]}\frac{D_{LS}}{D_S}\right)^{\frac{2}{2+\frac{1}{\eta}}} \left(r_0^{\frac{2(1+\eta)}{1+2\eta}}D_{LS}^{-\frac{2(1+\eta)}{1+2\eta}}-1 \right)   (1+2\eta) } {4\eta(1+\eta)}\right|.
\end{equation}
Because the radial coordinates of the WH are at $(-\infty,\infty)$. We could choose the side that makes the optical depth positive. This is the reason why we take the absolute value. 
We can differentiate the optical depth as
\begin{equation}
d\tau=\frac{1}{\Omega}\int_0^{D_S}n(D_L)\Omega 2r_EvdtdD_L=\int_0^{D_S}2n(D_L)r_E^2\frac{dt}{t_E}dD_L.
\end{equation}
We may observe while monitoring a certain number of sources $N$ for a specific time, then we could represent the event rate as 
\begin{equation}
\label{eq 41}
\Gamma=\frac{d(N\tau)}{dt}=\frac{2N}{\pi}\int_0^{D_S}n(D_L)\frac{\pi r_E^2}{t_E}dD_L=\frac{2N}{\pi t_E}\tau,
\end{equation}
Substituting the previous calculation results Eq. \eqref{te} and Eq. \eqref{eq. 39} into Eq. \eqref{eq 41}, we obtain


\begin{widetext}




\begin{equation}
\label{event rate}
\Gamma=\frac{2\chi \sigma_{micro}}{\pi\frac{D_S}{v}\left(\frac{r_0}{D_L}\right)^{\frac{1+\frac{1}{\eta}}{2+\frac{1}{\eta}}}\left(\frac{\sqrt{\pi}\eta\Gamma[1+\frac{1}{2\eta}]}{2\Gamma[\frac{1}{2}(3+\frac{1}{\eta})]}\frac{D_{LS}}{D_S}\right)^{\frac{1}{2+\frac{1}{\eta}}}}\times
\left|\left(\frac{\sqrt{\pi}\eta\Gamma[1+\frac{1}{2\eta}]}{2\Gamma[\frac{1}{2}(3+\frac{1}{\eta})]}\frac{D_{LS}}{D_S}\right)^{\frac{2}{2+\frac{1}{\eta}}}
\frac{ \left(r_0^{\frac{2(1+\eta)}{1+2\eta}}D_{LS}^{-\frac{2(1+\eta)}{1+2\eta}}-1 \right) (1+2\eta)  } {4\eta(1+\eta)}\right|.
\end{equation}

\begin{table}
\centering
\begin{tabular}{|c c c c c c c c c c|} 
 \hline
 $\eta$ & $\chi$ & $v$ & $r_0$ & Mass & $\theta_E$ & $r_E$ & $t_E$ & $\tau$ & $\Gamma$ \\ 
 - & & $m/s$ & $m$ & $M_\odot$ & rad & m & year & &  year$^{-1}$ \\ 
 \hline
 -10 & $1\times 10^{14}$ & $3\times 10^4$ & $3.24\times 10^{14}$ & $4.35616\times 10^{11}$ & - & - & - & 0.296932 & -\\ 
  -1 & $1\times 10^{14}$ & $3\times 10^4$ & $3.24\times 10^{14}$ & $1.09422\times 10^{17}$ & - & - & -& $\infty$ &  $\infty$\\
0.33 & $1\times 10^{14}$ & $3\times 10^4$ & $3.24\times 10^{14}$ & $\approx 0$ & $9.80012\times 10^{-6}$ & $6.33558\times 10^{15}$ &6696.89 &  0.373769 & $1.07207 $ \\
0.5 & $1\times 10^{14}$ & $3\times 10^4$ & $3.24\times 10^{14}$  & 0.109426 & $0.0000202052$ & $1.30622\times 10^{16}$ & 13807.1 & 0.272166  &  $1.60947 $ \\
 1 & $1\times 10^{14}$ & $3\times 10^4$ & $3.24\times 10^{14}$ & 109422 & $0.0000732296$ & $4.73415\times 10^{16}$ & 50041.2 & 0.201097  &  $ 4.31002$ \\
  2 &$1\times 10^{14}$ & $3\times 10^4$ & $3.24\times 10^{13}$ & $3.46022\times 10^6$ & 0.0000597872 & $3.86512\times 10^{16}$ & 40855.4 & 0.187058 &   3.27320\\
 2 &$1\times 10^{14}$ & $3\times 10^4$ & $3.24\times 10^{14}$ & $1.09422\times 10^8$ & 0.000238017 & $1.53873\times 10^{17}$ & 162648 & 0.187058 &  $ 1.30308\times 10^1$ \\
  2 &$1\times 10^{14}$ & $3\times 10^4$ & $3.24\times 10^{15}$ & $3.46022\times 10^9$ & 0.000947564 & $6.12581\times 10^{17}$ & 647515 & 0.187058 &  $ 5.18767 \times 10^{1}$ \\
 5 & $1\times 10^{14}$ & $3\times 10^4$ & $3.24\times 10^{14}$ & $6.90405\times 10^9$ & 0.000788302 & $5.09621\times 10^{17}$ & 538684 & 0.200010 & $ 4.61458 \times 10^{1}$\\
 10 & $1\times 10^{14}$ & $3\times 10^4$ & $3.24\times 10^{14}$  & $2.74855\times 10^{10}$ & 0.00152692 & $9.87125\times10^{17}$ & 1.04342$\times 10^6$  & 0.214840 & $ 9.60107\times 10^{1}$ \\
  100 & $1\times 10^{14}$ & $3\times 10^4$ & $3.24\times 10^{14}$  & $9.53024\times 10^{10}$ & 0.00675573 & $4.36744\times10^{18}$ & 4.61651$\times 10^6$  & 0.243226 & $ 4.80917\times 10^{2}$ \\
 \hline
\end{tabular}
\caption{The change of event rate $\Gamma$ with $\eta$. $\chi$ is the number of sources observed per unit angular area $\pi \theta^2$, $v$ is the relative velocity between the WH and the source plane, $r_0$ is the throat radius of the WH. In terms of mass, we choose solar mass $M_\odot$ as the unit, $\theta_E$ is the Einstein angle, $r_E$ is the Einstein radius, and $\tau$ is the optical depth. We set the parameters $D_S=21$ kpc, and $A=1$.}
\label{table:1}
\end{table}    

\begin{table}
    \centering
    \begin{tabular}{|c | c c c c c c c|}
    \hline
     $r_0$ (km) & $3.24\times 10^1$ & $3.24\times 10^2$ &  $3.24 \times10^3$ &  $3.24 \times10^4$ &  $3.24 \times10^5$ &  $3.24 \times10^6$ &  $3.24 \times10^7$  \\
      \hline
   $\Gamma$ (year$^{-1}$)  &  $1.86\times 10^{-6}$ & $8.62\times 10^{-6}$ & $4.00\times 10^{-5}$ & $1.86\times 10^{-4}$  & $8.62  \times 10^{-4}$ & $4.00 \times 10^{-3}$ & $1.86 \times 10^{-2}$ \\
    \hline
    \end{tabular}
    \caption{the numerical results by Eq \eqref{event rate} : event rate $\Gamma$ of Ellis-Bronnikov WH corresponding to throat radius $r_0$. $D_S=21$ kpc is assumed. $v=0.0001c$, $\chi=2\times 10^{14}$ and $n(D_L)=\frac{\rho}{M}$ are assumed.}
    \label{tab: 2}
\end{table}

\begin{table}
    \centering
    \begin{tabular}{|c|c c c c c c c|}
    \hline
        $r_0$ (km) & 10 & $10^2$ &  $10^3$ &  $10^4$ &  $10^5$ &  $10^6$ &  $10^7$ \\
    \hline
   $\Gamma$ (year$^{-1}$) &  $1.88 \times 10^{-14}$ & $8.73 \times 10^{-14} $ & $4.05  \times 10^{-13}$ & $1.88 \times 10^{-12} $ & $8.73  \times 10^{-12}$ & $4.05 \times 10^{-11}$ & $1.88 \times 10^{-10}$ \\
   \hline
    \end{tabular}
    \caption{The numerical results by \cite{Abe:2010ap}:  event rate $\Gamma$ of Ellis-Bronnikov WH corresponding to throat radius $r_0$. $D_S=8$ kpc is assumed. $v=5000$ km/s and $n=4.97\times 10^-9~\rm pc^{-3}$ are assumed. The number of  WH sources is one.
}
    \label{tab: 3}
\end{table}

\end{widetext}

\begin{figure}
    \centering
    \includegraphics[scale=0.8]{event_rate_image.png}
    \caption{The illustration of the microlensing rate of WH
moving along this 2D plane. This 2D plane is the source plane, and we assume that the number of sources within the Einstein ring is $\chi\sigma _{micro}$.}
    \label{fig: 7}
\end{figure}

The value of $N$ can be expressed as $\chi \sigma_{micro}$, where $\chi$ is a constant determined by the number of observations, and $\chi \propto D_S^2$. This parameter has a significant impact on the results, and we set $\chi=1\times 10^{14}$ and It's obvious $\Gamma\propto \chi $. We choose to fix some variables and then create a table of event rates because we should comply with weak field conditions and the angles involved should comply with the small angle approximation, which are not easily met simultaneously. We consider a lens in the range of $D_S=2D_L=2D_{LS}=21$ kpc, and the throat of the WH in the range of $3.24\times 10^{13}\sim 3.24\times 10^{15}$ m. Accroding to Eq. \eqref{event rate}, one could see, including $\eta$, $v$, $r_0$ and $D_L$, that these data are sufficient for us to determine the mass and Einstein angle. Furthermore we set $v=3\times 10^1$ km/s based on literature \cite{Zaris:2019soz,Sollima:2009wh,Noyola:2010ab}. Then, All our calculation results are shown in TABLE \ref{table:1}. Our results indicate that the event rate increases with an increase in RSC. This is because RSC has a significant impact on the quality of the WH, which can affect the size of the Einstein angle and, consequently, the observable area. However, the impact of RSC on optical depth is not significant, at least not in a monotonic fashion. Another important point is for cases where RSC is less than $-1$, where the Einstein radius and Einstein angle become complex numbers. Such a WH cannot be found through gravitational microlensing. This indicates that for a WH lens, its lens equation is at least quadratic, based on the relation $n=2+\frac{1}{\eta}$. 



We could use Eq. \eqref{event rate} to calculate the event rate of a single Ellis-Bronnikov WH with different throat radii, as shown in TABLE \ref{tab: 2}. In reference \cite{Abe:2010ap}, the authors numerically processed the event rates of clusters in Ellis-Bronnikov WHs with different throat radii, and their results are shown in TABLE \ref{tab: 3}. Both our data and theirs support the conclusion that the larger the WH throat, the greater the event rate. This situation is not unique to Ellis-Bronnikov WH. We set $\eta=2$ and then adjust the throat radius of the WH to get the same conclusion (see TABLE \ref{table:1}). However, there is a significant difference in magnitude between our results and theirs. This is because we consider the relatively concentrated mass density of individual WHs, while in clusters, the mass density becomes sparse, equivalent to WHs mixing vacuum.  Additionally, We are curious if event rate can be used to distinguish between black holes and WHs. In the study presented in \cite{Kiroglu:2021mej},  the n8-rv0.5-rg8-z0.1 case in the CMC Cluster Catalog models \cite{Kremer:2019iul}, they observe $2.75\times 10^5$ sources under the condition lens mass of $M=2\times 10^5~M_\odot$ and final time $t=12$ Gyr, and obtain an event rate of $<10^{-5}$ year$^{-1}$ for single black holes. In comparison, our study for single Ellis-Bronnikov WH case where $\eta=1$ and $M=1.1\times 10^5$, we observe $4.9\times 10^5$ sources with an Einstein radius crossing time of $5.0\times 10^5$ year, resulting in an event rate of 4.3 year$^{-1}$. When converted to the conditions of the n8-rv0.5-rg8-z0.1 case for mass and Einstein radius crossing time, the event rate of Ellis-Bronnikov WH is estimated to be about $10^{-2}\sim 10^{-3}$ year$^{-1}$. Our estimation shows that under the same mass and Einstein radius crossing time, WHs can have an event rate at least two orders of magnitude higher than black holes. In fact, by knowing the geometry of the lens, the metric of the WH, and the relative velocity between the WH and the source plane, we can calculate the event rate using Eq. \eqref{event rate}. Due to the actual physical conditions, there are other material sources that disturb our WH metric. At a relatively large WH mass density (The influence of other material sources can be ignored, which is not contradictory to the weak field approximation.), our calculated results can be directly used to compare with the observed results.










\section{Conclusion and outlook}
\label{conclusion and outlook}
This paper presents a comprehensive investigation of the microlensing effects and multi-image problem of the static spherical WH described by Eq. \eqref{eq7}. By introducing the so-called REoS parameter $\eta=\frac{p_r}{\rho}$, we reformulate the metric, allowing us to re-examine the microlensing effect of the WH, including its magnification and event rate. 
we employ the GBT to calculate the deflection angle corresponding to the metric under weak field approximation.  The resulting lens equation contains an explicit formula $n=2+\frac{1}{\eta}$ that reflects the order of the lens equation on the equatorial plane. This formula can be used to determine RSC of the WH based on the maximum number of detected images. In addition, we discuss the application of this formula in Appendix \ref{number of images}, which includes the cases of vacuum, Ellis-Bronnikov WH, charge WH, and quantum correction WH. This analysis provides valuable insights into the behavior of various types of WHs and how their properties affect the observed number of images.


In addition to reformulating the lens equation, we also derive a general formula for calculating the magnification in terms of RSC. This formula allows us to analyze how the magnification changes with RSC and the WH throat radius $r_0$. Our analysis reveals that larger values of $\eta$ correspond to larger values of $v$ at the position of the peak magnification. Similarly, larger values of $r_0$ correspond to larger values of the impact parameter $b$ at the position of the peak magnification. These observations may be related to the impact of $\eta$ and $r_0$ on the mass distribution of the WH.


To provide a more complete analysis, we also perform analytical calculations of the event rate for a single WH source. Our results, along with the numerical findings reported in \cite{Abe:2010ap}, indicate that for the Ellis-Bronnikov WH, the event rate increases as the throat radius of the WH increases. Interestingly, this pattern may also hold true for other types of WHs, as suggested by the results presented in TABLE \ref{table:1}. Moreover, our analysis reveals that for a single WH lens, the event rate increases as RSC becomes larger, but the effect of RSC on optical depth is not significant. This is due to the fact that optical depth is a purely geometric quantity and is not sensitive to the total mass, even though RSC has a significant impact on mass.


Our work aims to bridge theoretical models with observational data by examining the microlensing effects and multi-image problem of WHs. By introducing the REoS, we are able to provide a more general analysis of WH lensing.
While our study focuses on the specific case of static spherical WHs, exploring the number of objects, magnification, and event rate in various scenarios is an interesting direction for future research \cite{Cai:2022kbp,Kiroglu:2021mej}. Moreover, we acknowledge that our analysis assumes a constant value of RSC throughout the WH, as we do not consider the situation of $\eta=\eta(r)$ due to technological limitations. This limitation represents one of our key areas for future investigation. Our results can be used to guide future research efforts and contribute to a deeper understanding of the structure and behavior of WH.



 \section*{Acknowledgements}
 We appreciate that Hai-Qing Zhang and Bi-Chu Li give lots of suggestions to improve this manuscript. LH and KG are funded by NSFC grant NO. 12165009 and  Hunan Natural Science Foundation NO. 2023JJ30487. 

\section{Appendix I}
\label{number of images}
In this part, we check the performance of our equations $n=2+\frac{1}{\eta}$ in specific situations which includes the Vacuum case, Ellis-Bronnikov WH, charged WH, and quantum correction WH.

\subsection{Vacuum Case}
In the special case of $\eta=-1$, the metric \eqref{eq7} reduces to a Schwarzschild-like metric:
\begin{equation}
ds^2=-Adt^2+dr^2+r^2d\Omega^2.
\end{equation}
Our calculations show that the energy-momentum tensor vanishes, leading us to refer to this as the vacuum case. Solving the lensing equation in this case yields an expression for the deflection angle:
\begin{equation}
\theta=\frac{\pi D_{LS}}{2D_S\sqrt{A}}+\beta.
\end{equation}
For comparison, we substitute $\eta=-1$ into Eq. \eqref{eq 16} and obtain $n=1$. This result implies that in the vacuum case, there will be at most one image, which is consistent with our physical intuition.

\subsection{Ellis-Bronnikov WH}
We begin our analysis by discussing the number of images using traditional methods and comparing it with formula \eqref{eq 16}.
In the case where the redshift parameter $A=1$ and $\eta=1$, our metric becomes the Ellis-Bronnikov WH:
\begin{equation}
ds^2=-dt^2+\frac{dr^2}{1-\big(\frac{r_0}{r}\big)^{2}}+r^2d\Omega^2.
\end{equation}
The deflection angle for Eq. \eqref{eq 22} ($\eta=1$) is given by
\begin{equation}
\alpha=\frac{\pi}{4}\big(\frac{r_0}{b}\big)^2.
\end{equation}
Substituting this into the lens equation yields a cubic equation:
\begin{equation}
\theta^3-\beta\theta^2-\frac{\pi D_{LS}r_0^2}{4D_S D_L^2}=0.
\end{equation}
Although this is a cubic equation, there are only two physical solutions. Using formula \eqref{eq 16}, we obtain the order of the lens equation:
\begin{equation}
n=2+\frac{1}{1}=3.
\end{equation}
Alternatively, by accurately observing the maximum number of solutions, we can determine the value of RSC. This approach provides a complementary method to formula \eqref{eq 16} for determining the RSC.

\begin{figure}
    \centering
    \includegraphics[scale=0.55]{RN.pdf}
    \caption{The plot shows the relationship between RSC and the radial distance. When $r\to \infty$ in the weak field approximation, it shows that $\eta=1$.}
    \label{fig:3}
\end{figure}

\subsection{Charged WH}
In reference to article \cite{Liu:2022lfb}, it is shown that the charged spherical symmetric WH can produce at most three images, and its metric \cite{Kim:2001ri} can be expressed as
\begin{equation}
ds^2=-\big(1+\frac{Q^2}{r^2}\big)dt^2+\big(1-\frac{r_0^2}{r^2}+\frac{Q^2}{r^2}\big)^{-1}dr^2+r^2d\Omega^2,
\end{equation}
where $\frac{r_0^2}{r}$ represents the mass term and $Q$ is the electric charge. We calculate RSC with $\eta=\frac{p_r}{\rho}$ and obtain the expression:
\begin{equation}
\eta=-\frac{r^4 \left(Q^4+Q^2 \left(r^2-r_0^2\right)+r^2 r_0^2\right)}{\left(Q^2+r^2\right)^2 \left(Q^2-r_0^2\right) \left(Q^2+r^2-r_0^2\right)}.
\end{equation}
Although RSC is not a constant in this case, it is approximately constant within our integral region, as shown in our numerical diagram Fig. \ref{fig:3}.
In the case of a weak field and $r_0\ll r$, we obtain $\eta=1$, which, when substituted into Eq. \eqref{eq 16}, yields $n=3$. This result is consistent with the maximum number of images produced by charged WHs.
We acknowledge that Eq. \eqref{eq 16} was derived under the assumption of a constant redshift parameter, but this is reasonable given the limitation of microlensing. As the observer is quite remote from the source and WH, we have $\underset{r\rightarrow\infty}{\lim}(1+\frac{Q^2}{r^2})\approx 1$. This justifies the assumption of a constant redshift parameter and supports the validity of our analysis.




\subsection{WH with quantum correction}
In some cases, it may not be possible to directly calculate the value of RSC, but its value can be determined through observations of the images produced by the WH.
For example, the metric for a Morris-Thorne type wormhole with quantum corrections \cite{Jusufi:2018kmk} is given by:
\begin{equation}
ds^2=-(1+\frac{\hbar q}{r^2})dt^2+\frac{dr^2}{1-\frac{b_0^2}{r^2}+\frac{\hbar q}{r^2}}+r^2d\Omega^2.
\end{equation}
We can calculate RSC using this metric:
\begin{equation}
\eta=-\frac{r^4(b_0^2(-r^2+q\hbar)-q\hbar(r^2+q\hbar))}{(r^2+q\hbar)^2(-b_0^2+r^2+q\hbar)(b_0^2-q\hbar)}.
\end{equation}
Under the weak field approximation, we can simplify this expression as:
\begin{equation}
\eta=\frac{b_0^2+q\hbar}{b_0^2-q\hbar}.
\end{equation}
However, in the presence of quantum effects ($b_0^2=\mathcal{O}(q\hbar)$), RSC is not a constant and is related to the mass and charge of the WH. In such cases, it may not be possible to directly determine the value of RSC. However, if we observe no more than $n$ images, we can calculate RSC through the equation for the number of images. 

\section*{References}
\begin{thebibliography}{99}


%\cite{Flamm:1916}
\bibitem{Flamm:1916}
Ludwig, Flamm,
%
Beitr¨age zur Einstein schen gravitations theorie. Hirzel, 1916.
%2 itations counted in INSPIRE as of 14 Mar 2023

%\cite{Einstein:1935}
\bibitem{Einstein:1935}
Albert Einstein and Nathan Rosen.
% 
“The particle problem in the general theory of relativity”.
In: Physical Review 48.1 (1935), p. 73.
%

%\cite{Misner:1957}
\bibitem{Misner:1957}
Charles W Misner and John A Wheeler.
% “Classical physics as geometry”. In: Annals of
physics 2.6 (1957), pp. 525–603.
%

%\cite{Ellis:1973}
\bibitem{Ellis:1973}
Homer G Ellis.
% “Ether flow through a drainhole: A particle model in general relativity”. In:
Journal of Mathematical Physics 14.1 (1973),
pp. 104–118.
%

%\cite{Morris:1988cz}
\bibitem{Morris:1988cz}
M.~S.~Morris and K.~S.~Thorne,
%``Wormholes in space-time and their use for interstellar travel: A tool for teaching general relativity,''
Am. J. Phys. \textbf{56} (1988), 395-412
doi:10.1119/1.15620
%1823 citations counted in INSPIRE as of 14 Mar 2023

%\cite{Damour:2007ap}
\bibitem{Damour:2007ap}
T.~Damour and S.~N.~Solodukhin,
%``Wormholes as black hole foils,''
Phys. Rev. D \textbf{76} (2007), 024016
doi:10.1103/PhysRevD.76.024016
[arXiv:0704.2667 [gr-qc]].
%176 citations counted in INSPIRE as of 14 Mar 2023

%\cite{Kim:2003zb}
\bibitem{Kim:2003zb}
W.~T.~Kim, J.~J.~Oh and M.~S.~Yoon,
%``Traversable wormholes construction in (2+1)-dimensions,''
Phys. Rev. D \textbf{70} (2004), 044006
doi:10.1103/PhysRevD.70.044006
[arXiv:gr-qc/0307034 [gr-qc]].
%29 citations counted in INSPIRE as of 14 Mar 2023


%\cite{Bueno:2017hyj}
\bibitem{Bueno:2017hyj}
P.~Bueno, P.~A.~Cano, F.~Goelen, T.~Hertog and B.~Vercnocke,
%``Echoes of Kerr-like wormholes,''
Phys. Rev. D \textbf{97} (2018) no.2, 024040
doi:10.1103/PhysRevD.97.024040
[arXiv:1711.00391 [gr-qc]].
%150 citations counted in INSPIRE as of 15 Mar 2023

%\cite{Visser:1989kh}
\bibitem{Visser:1989kh}
M.~Visser,
%``Traversable wormholes: Some simple examples,''
Phys. Rev. D \textbf{39} (1989), 3182-3184
doi:10.1103/PhysRevD.39.3182
[arXiv:0809.0907 [gr-qc]].
%403 citations counted in INSPIRE as of 15 Mar 2023

%\cite{Sushkov:2005kj}
\bibitem{Sushkov:2005kj}
S.~V.~Sushkov,
%``Wormholes supported by a phantom energy,''
Phys. Rev. D \textbf{71} (2005), 043520
doi:10.1103/PhysRevD.71.043520
[arXiv:gr-qc/0502084 [gr-qc]].
%346 citations counted in INSPIRE as of 15 Mar 2023

%\cite{Bronnikov:2002rn}
\bibitem{Bronnikov:2002rn}
K.~A.~Bronnikov and S.~W.~Kim,
%``Possible wormholes in a brane world,''
Phys. Rev. D \textbf{67} (2003), 064027
doi:10.1103/PhysRevD.67.064027
[arXiv:gr-qc/0212112 [gr-qc]].
%203 citations counted in INSPIRE as of 15 Mar 2023

%\cite{Clement:1995ya}
\bibitem{Clement:1995ya}
G.~Clement,
%``Wormhole cosmic strings,''
Phys. Rev. D \textbf{51} (1995), 6803-6809
doi:10.1103/PhysRevD.51.6803
[arXiv:gr-qc/9502033 [gr-qc]].
%36 citations counted in INSPIRE as of 15 Mar 2023

%\cite{Richarte:2017iit}
\bibitem{Richarte:2017iit}
M.~G.~Richarte, I.~G.~Salako, J.~P.~Morais Gra\c{c}a, H.~Moradpour and A.~\"Ovg\"un,
%``Relativistic Bose-Einstein condensates thin-shell wormholes,''
Phys. Rev. D \textbf{96} (2017) no.8, 084022
doi:10.1103/PhysRevD.96.084022
[arXiv:1710.05886 [gr-qc]].
%36 citations counted in INSPIRE as of 15 Mar 2023

%\cite{Ayuso:2020vuu}
\bibitem{Ayuso:2020vuu}
I.~Ayuso, F.~S.~N.~Lobo and J.~P.~Mimoso,
%``Wormhole geometries induced by action-dependent Lagrangian theories,''
Phys. Rev. D \textbf{103} (2021) no.4, 044018
doi:10.1103/PhysRevD.103.044018
[arXiv:2012.00047 [gr-qc]].
%4 citations counted in INSPIRE as of 15 Mar 2023

%\cite{KordZangeneh:2020ixt}
\bibitem{KordZangeneh:2020ixt}
M.~Kord Zangeneh and F.~S.~N.~Lobo,
%``Dynamic wormhole geometries in hybrid metric-Palatini gravity,''
Eur. Phys. J. C \textbf{81} (2021) no.4, 285
doi:10.1140/epjc/s10052-021-09059-y
[arXiv:2011.01745 [gr-qc]].
%13 citations counted in INSPIRE as of 15 Mar 2023

%\cite{Song:2023jdn}
\bibitem{Song:2023jdn}
S.~Q.~Song and E.~G\"udekli,
%``Traversable wormholes in Rastall teleparallel gravity with non-commutative geometry,''
New Astron. \textbf{100} (2023), 101993
doi:10.1016/j.newast.2022.101993
%0 citations counted in INSPIRE as of 15 Mar 2023

%\cite{Saleem:2023lul}
\bibitem{Saleem:2023lul}
R.~Saleem, M.~I.~Aslam and K.~Rasool,
%``Wormhole solutions in Rastall-like-torsion-trace gravity,''
Chin. J. Phys. \textbf{82} (2023), 1-14
doi:10.1016/j.cjph.2022.12.015
%0 citations counted in INSPIRE as of 15 Mar 2023

%\cite{Eid:2023wrd}
\bibitem{Eid:2023wrd}
A.~Eid and A.~Alkaoud,
%``Dynamics and stability of Hayward -de Sitter thin-shell wormhole,''
New Astron. \textbf{101} (2023), 102021
doi:10.1016/j.newast.2023.102021
%0 citations counted in INSPIRE as of 15 Mar 2023

%\cite{Godani:2023paj}
\bibitem{Godani:2023paj}
N.~Godani,
%``Stability of Heyward wormhole in f(R) gravity,''
New Astron. \textbf{100} (2023), 101994
doi:10.1016/j.newast.2022.101994
%1 citations counted in INSPIRE as of 15 Mar 2023

%\cite{Hochberg:1998ii}
\bibitem{Hochberg:1998ii}
D.~Hochberg and M.~Visser,
%``The Null energy condition in dynamic wormholes,''
Phys. Rev. Lett. \textbf{81} (1998), 746-749
doi:10.1103/PhysRevLett.81.746
[arXiv:gr-qc/9802048 [gr-qc]].
%202 citations counted in INSPIRE as of 15 Mar 2023

%\cite{Hochberg:1998ha}
\bibitem{Hochberg:1998ha}
D.~Hochberg and M.~Visser,
%``Dynamic wormholes, anti-trapped surfaces, and energy conditions,''
Phys. Rev. D \textbf{58} (1998), 044021
doi:10.1103/PhysRevD.58.044021
[arXiv:gr-qc/9802046 [gr-qc]].
%243 citations counted in INSPIRE as of 15 Mar 2023

%\cite{Lobo:2002zf}
\bibitem{Lobo:2002zf}
F.~Lobo and P.~Crawford,
%``Weak energy condition violation and superluminal travel,''
Lect. Notes Phys. \textbf{617} (2003), 277-291
doi:10.1007/3-540-36973-2\_15
[arXiv:gr-qc/0204038 [gr-qc]].
%31 citations counted in INSPIRE as of 15 Mar 2023

%\cite{Lobo:2004rp}
\bibitem{Lobo:2004rp}
F.~S.~N.~Lobo,
%``Energy conditions, traversable wormholes and dust shells,''
Gen. Rel. Grav. \textbf{37} (2005), 2023-2038
doi:10.1007/s10714-005-0177-x
[arXiv:gr-qc/0410087 [gr-qc]].
%66 citations counted in INSPIRE as of 15 Mar 2023

%\cite{Einstein:1936llh}
\bibitem{Einstein:1936llh}
A.~Einstein,
%``Lens-Like Action of a Star by the Deviation of Light in the Gravitational Field,''
Science \textbf{84} (1936), 506-507
doi:10.1126/science.84.2188.506
%398 citations counted in INSPIRE as of 15 Mar 2023


%\cite{Bartelmann:1999yn}
\bibitem{Bartelmann:1999yn}
M.~Bartelmann and P.~Schneider,
%``Weak gravitational lensing,''
Phys. Rept. \textbf{340} (2001), 291-472
doi:10.1016/S0370-1573(00)00082-X
[arXiv:astro-ph/9912508 [astro-ph]].
%1742 citations counted in INSPIRE as of 15 Mar 2023

%\cite{Kaiser:1991qi}
\bibitem{Kaiser:1991qi}
N.~Kaiser,
%``Weak gravitational lensing of distant galaxies,''
Astrophys. J. \textbf{388} (1992), 272
doi:10.1086/171151
%665 citations counted in INSPIRE as of 15 Mar 2023

%\cite{Bozza:2002zj}
\bibitem{Bozza:2002zj}
V.~Bozza,
%``Gravitational lensing in the strong field limit,''
Phys. Rev. D \textbf{66} (2002), 103001
doi:10.1103/PhysRevD.66.103001
[arXiv:gr-qc/0208075 [gr-qc]].
%434 citations counted in INSPIRE as of 15 Mar 2023


%\cite{Virbhadra:1999nm}
\bibitem{Virbhadra:1999nm}
K.~S.~Virbhadra and G.~F.~R.~Ellis,
%``Schwarzschild black hole lensing,''
Phys. Rev. D \textbf{62} (2000), 084003
doi:10.1103/PhysRevD.62.084003
[arXiv:astro-ph/9904193 [astro-ph]].
%701 citations counted in INSPIRE as of 15 Mar 2023

%\cite{Perlick:2004tq}
\bibitem{Perlick:2004tq}
V.~Perlick,
%``Gravitational lensing from a spacetime perspective,''
Living Rev. Rel. \textbf{7} (2004), 9
%249 citations counted in INSPIRE as of 15 Mar 2023

%\cite{SDSS:2002oin}
\bibitem{SDSS:2002oin}
C.~Stoughton \textit{et al.} [SDSS],
%``The Sloan Digital Sky Survey: Early Data Release,''
Astron. J. \textbf{123} (2002), 485-548
doi:10.1086/324741
%1628 citations counted in INSPIRE as of 15 Mar 2023

%\cite{Gao:2022cds}
\bibitem{Gao:2022cds}
K.~Gao, L.~H.~Liu and M.~Zhu,
%``Microlensing effects of wormholes associated to blackhole spacetimes,''
[arXiv:2211.17065 [gr-qc]].
%2 citations counted in INSPIRE as of 15 Mar 2023

%\cite{Liu:2022lfb}
\bibitem{Liu:2022lfb}
L.~H.~Liu, M.~Zhu, W.~Luo, Y.~F.~Cai and Y.~Wang,
%``Microlensing effect of a charged spherically symmetric wormhole,''
Phys. Rev. D \textbf{107} (2023) no.2, 024022
doi:10.1103/PhysRevD.107.024022
[arXiv:2207.05406 [gr-qc]].
%5 citations counted in INSPIRE as of 15 Mar 2023

%\cite{Sokoliuk:2022owk}
\bibitem{Sokoliuk:2022owk}
O.~Sokoliuk, S.~Praharaj, A.~Baransky and P.~K.~Sahoo,
%``Accretion flows around exotic tidal wormholes - I. Ray-tracing,''
Astron. Astrophys. \textbf{665} (2022), A139
doi:10.1051/0004-6361/202244358
[arXiv:2207.07193 [gr-qc]].
%1 citations counted in INSPIRE as of 15 Mar 2023

%\cite{Zatrimaylov:2021ijd}
\bibitem{Zatrimaylov:2021ijd}
K.~Zatrimaylov,
%``Dark Matter, Rotation Curves, and the Morphology of Galaxies,''
[arXiv:2108.13350 [astro-ph.CO]].
%0 citations counted in INSPIRE as of 15 Mar 2023

%\cite{Cheng:2021hoc}
\bibitem{Cheng:2021hoc}
X.~T.~Cheng and Y.~Xie,
%``Probing a black-bounce, traversable wormhole with weak deflection gravitational lensing,''
Phys. Rev. D \textbf{103} (2021) no.6, 064040
doi:10.1103/PhysRevD.103.064040
%31 citations counted in INSPIRE as of 15 Mar 2023

%\cite{Li:2019qyb}
\bibitem{Li:2019qyb}
Z.~Li and J.~Jia,
%``The finite-distance gravitational deflection of massive particles in stationary spacetime: a Jacobi metric approach,''
Eur. Phys. J. C \textbf{80} (2020) no.2, 157
doi:10.1140/epjc/s10052-020-7665-8
[arXiv:1912.05194 [gr-qc]].
%36 citations counted in INSPIRE as of 15 Mar 2023

%\cite{Kuniyasu:2018cgv}
\bibitem{Kuniyasu:2018cgv}
M.~Kuniyasu, K.~Nanri, N.~Sakai, T.~Ohgami, R.~Fukushige and S.~Komura,
%``Can we identify massless braneworld black holes by observations?,''
Phys. Rev. D \textbf{97} (2018) no.10, 104063
doi:10.1103/PhysRevD.97.104063
[arXiv:1806.00231 [gr-qc]].
%5 citations counted in INSPIRE as of 15 Mar 2023

%\cite{Raidal:2018eoo}
\bibitem{Raidal:2018eoo}
M.~Raidal, S.~Solodukhin, V.~Vaskonen and H.~Veerm\"ae,
%``Light Primordial Exotic Compact Objects as All Dark Matter,''
Phys. Rev. D \textbf{97} (2018) no.12, 123520
doi:10.1103/PhysRevD.97.123520
[arXiv:1802.07728 [astro-ph.CO]].
%27 citations counted in INSPIRE as of 15 Mar 2023

%\cite{Tsukamoto:2017hva}
\bibitem{Tsukamoto:2017hva}
N.~Tsukamoto and Y.~Gong,
%``Extended source effect on microlensing light curves by an Ellis wormhole,''
Phys. Rev. D \textbf{97} (2018) no.8, 084051
doi:10.1103/PhysRevD.97.084051
[arXiv:1711.04560 [gr-qc]].
%27 citations counted in INSPIRE as of 15 Mar 2023

%\cite{Sajadi:2016hko}
\bibitem{Sajadi:2016hko}
S.~N.~Sajadi and N.~Riazi,
%``Gravitational lensing by multi-polytropic wormholes,''
Can. J. Phys. \textbf{98} (2020) no.11, 1046-1054
doi:10.1139/cjp-2019-0524
[arXiv:1611.04343 [gr-qc]].
%13 citations counted in INSPIRE as of 15 Mar 2023

%\cite{Lukmanova:2016czn}
\bibitem{Lukmanova:2016czn}
R.~Lukmanova, A.~Kulbakova, R.~Izmailov and A.~A.~Potapov,
%``Gravitational Microlensing by Ellis Wormhole: Second Order Effects,''
Int. J. Theor. Phys. \textbf{55} (2016) no.11, 4723-4730
doi:10.1007/s10773-016-3095-7
%21 citations counted in INSPIRE as of 15 Mar 2023

%\cite{Tsukamoto:2016zdu}
\bibitem{Tsukamoto:2016zdu}
N.~Tsukamoto and T.~Harada,
%``Light curves of light rays passing through a wormhole,''
Phys. Rev. D \textbf{95} (2017) no.2, 024030
doi:10.1103/PhysRevD.95.024030
[arXiv:1607.01120 [gr-qc]].
%64 citations counted in INSPIRE as of 15 Mar 2023

%\cite{Kitamura:2016vad}
\bibitem{Kitamura:2016vad}
T.~Kitamura, Gravitational lensing in an exotic spacetime
%``Gravitational lensing in an exotic spacetime,''
%0 citations counted in INSPIRE as of 15 Mar 2023

%\cite{Kitamura:2012zy}
\bibitem{Kitamura:2012zy}
T.~Kitamura, K.~Nakajima and H.~Asada,
%``Demagnifying gravitational lenses toward hunting a clue of exotic matter and energy,''
Phys. Rev. D \textbf{87} (2013) no.2, 027501
doi:10.1103/PhysRevD.87.027501
[arXiv:1211.0379 [gr-qc]].
%50 citations counted in INSPIRE as of 15 Mar 2023

%\cite{Kitamura:2012wcg}
\bibitem{Kitamura:2012wcg}
T.~Kitamura, Astrometric microlensing by the Ellis Wormhole
%``Astrometric microlensing by the Ellis Wormhole,''
%0 citations counted in INSPIRE as of 15 Mar 2023

%\cite{Toki:2011zu}
\bibitem{Toki:2011zu}
Y.~Toki, T.~Kitamura, H.~Asada and F.~Abe,
%``Astrometric Image Centroid Displacements due to Gravitational Microlensing by the Ellis Wormhole,''
Astrophys. J. \textbf{740} (2011), 121
doi:10.1088/0004-637X/740/2/121
[arXiv:1107.5374 [astro-ph.CO]].
%85 citations counted in INSPIRE as of 15 Mar 2023

%\cite{Abe:2010ap}
\bibitem{Abe:2010ap}
F.~Abe,
%``Gravitational Microlensing by the Ellis Wormhole,''
Astrophys. J. \textbf{725} (2010), 787-793
doi:10.1088/0004-637X/725/1/787
[arXiv:1009.6084 [astro-ph.CO]].
%135 citations counted in INSPIRE as of 15 Mar 2023

%\cite{Bogdanov:2008zy}
\bibitem{Bogdanov:2008zy}
M.~B.~Bogdanov and A.~M.~Cherepashchuk,
%``Search for exotic matter from gravitational microlensing observations of stars,''
Astrophys. Space Sci. \textbf{317} (2008), 181-192
doi:10.1007/s10509-008-9870-z
[arXiv:0807.2774 [astro-ph]].
%14 citations counted in INSPIRE as of 15 Mar 2023

%\cite{Torres:2001gb}
\bibitem{Torres:2001gb}
D.~F.~Torres, E.~F.~Eiroa and G.~E.~Romero,
%``On the possibility of an astronomical detection of chromaticity effects in microlensing by wormhole - like objects,''
Mod. Phys. Lett. A \textbf{16} (2001), 1849-1861
doi:10.1142/S0217732301005126
[arXiv:gr-qc/0109041 [gr-qc]].
%7 citations counted in INSPIRE as of 15 Mar 2023

%\cite{Safonova:2001vz}
\bibitem{Safonova:2001vz}
M.~Safonova, D.~F.~Torres and G.~E.~Romero,
%``Microlensing by natural wormholes: Theory and simulations,''
Phys. Rev. D \textbf{65} (2002), 023001
doi:10.1103/PhysRevD.65.023001
[arXiv:gr-qc/0105070 [gr-qc]].
%103 citations counted in INSPIRE as of 15 Mar 2023

%\cite{Torres:1998cu}
\bibitem{Torres:1998cu}
D.~F.~Torres, G.~E.~Romero and L.~A.~Anchordoqui,
%``Wormholes, gamma-ray bursts and the amount of negative mass in the universe,''
Mod. Phys. Lett. A \textbf{13} (1998), 1575-1582
doi:10.1142/S0217732398001650
[arXiv:gr-qc/9805075 [gr-qc]].
%42 citations counted in INSPIRE as of 15 Mar 2023

%\cite{Torres:1998xd}
\bibitem{Torres:1998xd}
D.~F.~Torres, G.~E.~Romero and L.~A.~Anchordoqui,
%``Might some gamma-ray bursts be an observable signature of natural wormholes?,''
Phys. Rev. D \textbf{58} (1998), 123001
doi:10.1103/PhysRevD.58.123001
[arXiv:astro-ph/9802106 [astro-ph]].
%61 citations counted in INSPIRE as of 15 Mar 2023


%\cite{Gibbons:2008rj}
\bibitem{Gibbons:2008rj}
G.~W.~Gibbons and M.~C.~Werner,
%``Applications of the Gauss-Bonnet theorem to gravitational lensing,''
Class. Quant. Grav. \textbf{25} (2008), 235009
doi:10.1088/0264-9381/25/23/235009
[arXiv:0807.0854 [gr-qc]].
%195 citations counted in INSPIRE as of 15 Mar 2023

%\cite{Gibbons:2008zi}
\bibitem{Gibbons:2008zi}
G.~W.~Gibbons, C.~A.~R.~Herdeiro, C.~M.~Warnick and M.~C.~Werner,
%``Stationary Metrics and Optical Zermelo-Randers-Finsler Geometry,''
Phys. Rev. D \textbf{79} (2009), 044022
doi:10.1103/PhysRevD.79.044022
[arXiv:0811.2877 [gr-qc]].
%104 citations counted in INSPIRE as of 15 Mar 2023

%\cite{Werner:2012rc}
\bibitem{Werner:2012rc}
M.~C.~Werner,
%``Gravitational lensing in the Kerr-Randers optical geometry,''
Gen. Rel. Grav. \textbf{44} (2012), 3047-3057
doi:10.1007/s10714-012-1458-9
[arXiv:1205.3876 [gr-qc]].
%150 citations counted in INSPIRE as of 15 Mar 2023

%\cite{He:2023hsv}
\bibitem{He:2023hsv}
X.~He, T.~Xu, Y.~Yu, A.~Karamat, R.~Babar and R.~Ali,
%``Deflection angle evolution with plasma medium and without plasma medium in a parameterized black hole,''
Annals Phys. \textbf{451} (2023), 169247
doi:10.1016/j.aop.2023.169247
%0 citations counted in INSPIRE as of 15 Mar 2023

%\cite{Upadhyay:2023yhk}
\bibitem{Upadhyay:2023yhk}
S.~Upadhyay, S.~Mandal, Y.~Myrzakulov and K.~Myrzakulov,
%``Weak deflection angle, greybody bound and shadow for charged massive BTZ black hole,''
Annals Phys. \textbf{450} (2023), 169242
doi:10.1016/j.aop.2023.169242
[arXiv:2303.02132 [gr-qc]].
%0 citations counted in INSPIRE as of 15 Mar 2023

%\cite{Javed:2023iih}
\bibitem{Javed:2023iih}
W.~Javed, M.~Atique, R.~C.~Pantig and A.~\"Ovg\"un,
%``Weak Deflection Angle, Hawking Radiation and Greybody Bound of Reissner\textendash{}Nordstr\"om Black Hole Corrected by Bounce Parameter,''
Symmetry \textbf{15} (2023) no.1, 148
doi:10.3390/sym15010148
[arXiv:2301.01855 [gr-qc]].
%1 citations counted in INSPIRE as of 15 Mar 2023

%\cite{Javed:2022gtz}
\bibitem{Javed:2022gtz}
W.~Javed, H.~Irshad, R.~C.~Pantig and A.~\"Ovg\"un,
%``Weak Deflection Angle by Kalb\textendash{}Ramond Traversable Wormhole in Plasma and Dark Matter Mediums,''
Universe \textbf{8} (2022) no.11, 599
doi:10.3390/universe8110599
[arXiv:2211.07009 [gr-qc]].
%4 citations counted in INSPIRE as of 15 Mar 2023

%\cite{Huang:2022soh}
\bibitem{Huang:2022soh}
Y.~Huang and Z.~Cao,
%``Generalized Gibbons-Werner method for deflection angle,''
Phys. Rev. D \textbf{106} (2022) no.10, 104043
doi:10.1103/PhysRevD.106.104043
%1 citations counted in INSPIRE as of 15 Mar 2023

%\cite{He:2022yhp}
\bibitem{He:2022yhp}
J.~He, Q.~Wang, Q.~Hu, L.~Feng and J.~Jia,
%``Deflection in higher dimensional spacetime and asymptotically non-flat spacetimes,''
Class. Quant. Grav. \textbf{40} (2023) no.6, 065006
doi:10.1088/1361-6382/acbade
[arXiv:2210.00938 [gr-qc]].
%0 citations counted in INSPIRE as of 15 Mar 2023

%\cite{Gao:2023ltr}
\bibitem{Gao:2023ltr}
X.~J.~Gao, X.~k.~Yan, Y.~Yin and Y.~P.~Hu,
%``Gravitational lensing by a charged spherically symmetric black hole immersed in thin dark matter,''
[arXiv:2303.00190 [gr-qc]].
%0 citations counted in INSPIRE as of 15 Mar 2023

%\cite{Cai:2023ite}
\bibitem{Cai:2023ite}
T.~Cai, Z.~Wang, H.~Huang and M.~Zhu,
%``Higher order correction to weak-field lensing of Ellis-Bronnikov wormhole,''
[arXiv:2302.13704 [gr-qc]].
%0 citations counted in INSPIRE as of 15 Mar 2023

%\cite{Ovgun:2023ego}
\bibitem{Ovgun:2023ego}
A.~\"Ovg\"un, R.~C.~Pantig and \'A.~Rinc\'on,
%``4D scale-dependent Schwarzschild-AdS/dS black holes: study of shadow and weak deflection angle and greybody bounding,''
Eur. Phys. J. Plus \textbf{138} (2023) no.3, 192
doi:10.1140/epjp/s13360-023-03793-w
[arXiv:2303.01696 [gr-qc]].
%1 citations counted in INSPIRE as of 15 Mar 2023

%\cite{Lobo:2005us}
\bibitem{Lobo:2005us}
F.~S.~N.~Lobo,
%``Phantom energy traversable wormholes,''
Phys. Rev. D \textbf{71} (2005), 084011
doi:10.1103/PhysRevD.71.084011
[arXiv:gr-qc/0502099 [gr-qc]].
%445 citations counted in INSPIRE as of 15 Mar 2023

%\cite{Lobo:2005yv}
\bibitem{Lobo:2005yv}
F.~S.~N.~Lobo,
%``Stability of phantom wormholes,''
Phys. Rev. D \textbf{71} (2005), 124022
doi:10.1103/PhysRevD.71.124022
[arXiv:gr-qc/0506001 [gr-qc]].
%211 citations counted in INSPIRE as of 15 Mar 2023

%\cite{Garattini:2007ff}
\bibitem{Garattini:2007ff}
R.~Garattini and F.~S.~N.~Lobo,
%``Self sustained phantom wormholes in semi-classical gravity,''
Class. Quant. Grav. \textbf{24} (2007), 2401-2413
doi:10.1088/0264-9381/24/9/016
[arXiv:gr-qc/0701020 [gr-qc]].
%80 citations counted in INSPIRE as of 15 Mar 2023


%\cite{Kim:2001ri}
\bibitem{Kim:2001ri}
S.~W.~Kim and H.~Lee,
%``Exact solutions of a charged wormhole,''
Phys. Rev. D \textbf{63}, 064014 (2001)
doi:10.1103/PhysRevD.63.064014
[arXiv:gr-qc/0102077 [gr-qc]].
%119 citations counted in INSPIRE as of 19 Mar 2023

%\cite{Jusufi:2018kmk}
\bibitem{Jusufi:2018kmk}
K.~Jusufi, A.~\"Ovg\"un, A.~Banerjee and \textperiodcentered{}.~I.~Sakall\i{},
%``Gravitational lensing by wormholes supported by electromagnetic, scalar, and quantum effects,''
Eur. Phys. J. Plus \textbf{134} (2019) no.9, 428
doi:10.1140/epjp/i2019-12792-9
[arXiv:1802.07680 [gr-qc]].
%55 citations counted in INSPIRE as of 30 Mar 2023


%\cite{Zaris:2019soz}
\bibitem{Zaris:2019soz}
J.~Zaris, D.~Veske, J.~Samsing, Z.~M\'arka, I.~Bartos and S.~M\'arka,
%``Constraining Black Hole Populations in Globular Clusters using Microlensing: Application to Omega Centauri,''
Astrophys. J. Lett. \textbf{894} (2020) no.1, L9
doi:10.3847/2041-8213/ab89a3
[arXiv:1912.05701 [astro-ph.HE]].
%2 citations counted in INSPIRE as of 03 Jun 2023

%\cite{Sollima:2009wh}
\bibitem{Sollima:2009wh}
A.~Sollima, M.~Bellazzini, R.~L.~Smart, M.~Correnti, E.~Pancino, F.~R.~Ferraro and D.~Romano,
%``The non-peculiar velocity dispersion profile of the stellar system omega Centauri,''
Mon. Not. Roy. Astron. Soc. \textbf{396} (2009), 2183
doi:10.1111/j.1365-2966.2009.14864.x
[arXiv:0904.0571 [astro-ph.SR]].
%24 citations counted in INSPIRE as of 03 Jun 2023

%\cite{Cai:2022kbp}
\bibitem{Cai:2022kbp}
R.~G.~Cai, T.~Chen, S.~J.~Wang and X.~Y.~Yang,
%``Gravitational microlensing by dressed primordial black holes,''
JCAP \textbf{03} (2023), 043
doi:10.1088/1475-7516/2023/03/043
[arXiv:2210.02078 [astro-ph.CO]].
%3 citations counted in INSPIRE as of 01 Apr 2023

%\cite{Kiroglu:2021mej}
\bibitem{Kiroglu:2021mej}
F.~K\i{}ro\u{g}lu, N.~C.~Weatherford, K.~Kremer, C.~S.~Ye, G.~Fragione and F.~A.~Rasio,
%``Gravitational Microlensing Rates in Milky Way Globular Clusters,''
Astrophys. J. \textbf{928} (2022) no.2, 181
doi:10.3847/1538-4357/ac5895
[arXiv:2111.14866 [astro-ph.GA]].
%3 citations counted in INSPIRE as of 01 Apr 2023

%\cite{Johnson:2019ljv}
\bibitem{Johnson:2019ljv}
M.~D.~Johnson, A.~Lupsasca, A.~Strominger, G.~N.~Wong, S.~Hadar, D.~Kapec, R.~Narayan, A.~Chael, C.~F.~Gammie and P.~Galison, \textit{et al.}
%``Universal interferometric signatures of a black hole\textquoteright{}s photon ring,''
Sci. Adv. \textbf{6} (2020) no.12, eaaz1310
doi:10.1126/sciadv.aaz1310
[arXiv:1907.04329 [astro-ph.IM]].
%143 citations counted in INSPIRE as of 03 Jun 2023

%\cite{Gralla:2019drh}
\bibitem{Gralla:2019drh}
S.~E.~Gralla and A.~Lupsasca,
%``Lensing by Kerr Black Holes,''
Phys. Rev. D \textbf{101} (2020) no.4, 044031
doi:10.1103/PhysRevD.101.044031
[arXiv:1910.12873 [gr-qc]].
%99 citations counted in INSPIRE as of 03 Jun 2023

%\cite{Noyola:2010ab}
\bibitem{Noyola:2010ab}
E.~Noyola, K.~Gebhardt, M.~Kissler-Patig, N.~Lutzgendorf, B.~Jalali, P.~T.~de Zeeuw and H.~Baumgardt,
%``VLT Kinematics for omega Centauri: Further Support for a Central Black Hole,''
Astrophys. J. Lett. \textbf{719} (2010), L60
doi:10.1088/2041-8205/719/1/L60
[arXiv:1007.4559 [astro-ph.GA]].
%54 citations counted in INSPIRE as of 03 Jun 2023

%\cite{Kremer:2019iul}
\bibitem{Kremer:2019iul}
K.~Kremer, C.~S.~Ye, N.~Z.~Rui, N.~C.~Weatherford, S.~Chatterjee, G.~Fragione, C.~L.~Rodriguez, M.~Spera and F.~A.~Rasio,
%``Modeling Dense Star Clusters in the Milky Way and Beyond with the CMC Cluster Catalog,''
Astrophys. J. Suppl. \textbf{247} (2020) no.2, 48
doi:10.3847/1538-4365/ab7919
[arXiv:1911.00018 [astro-ph.HE]].
%104 citations counted in INSPIRE as of 05 Jun 2023

\end{thebibliography}
\bibliography{mybibfile}








\end{document}
