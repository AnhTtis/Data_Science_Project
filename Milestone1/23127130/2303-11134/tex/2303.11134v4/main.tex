\documentclass[prd,nofootinbib,twocolumn,superscriptaddress]{revtex4}
%\documentclass[review,12pt]{elsarticle}

\usepackage{hyperref}

\usepackage{amsfonts,amssymb,amsmath}
\usepackage{epsfig}
\usepackage{mathrsfs}
\usepackage{amssymb}
\usepackage{graphicx}
\graphicspath{ {./images/} }
\usepackage{amsmath}
\usepackage{xcolor}
\usepackage{color}
%\usepackage[toc,page,title,titletoc,header]{appendix}

\hypersetup{hidelinks}
\newcommand{\TT}[1]{{\color{red}{#1}}}	
\newcommand{\TTT}[1]{{\color{blue}{#1}}}		
\setlength{\paperheight}{11in}

\begin{document}



\title{ Microlensing and event rate of static spherically symmetric wormhole}
\author{Ke Gao}
\email{2021700389@stu.jsu.edu.cn}
\author{Lei-Hua Liu}
\email{liuleihua8899@hotmail.com}


\affiliation{Department of Physics, College of Physics, Mechanical and Electrical Engineering, Jishou University, Jishou 416000, China}




\begin{abstract}
Since the lensing effects play a vital role in modern cosmology. A new framework is developed for the static spherically symmetrical wormhole (WH) in terms of the radial equation of state (REoS). Following the standard procedure, we calculate the lensing equation, magnification, and event rate according to REoS, where our analysis indicates that the image problem of light source is complicated. As for the event rate, our investigations indicate that the larger values for the throat radius of WH and REoS will lead to larger values of the event rate. Compared with the event rate of blackhole, it is also claimed that the value of WH will be larger, in which their mass and the distance of them (blackhole or WH) between the light source and observer are comparable. Thus, our study could provide a possibility for distinguishing the WH and blackhole under similar circumstances. 


 

\end{abstract}

\maketitle

\section{Introduction}
\label{introduction}

The first research of WH originated from the inside of Schwarzschild's blackhole \cite{Flamm:1916}. Thereafter, Einstein and Rosen explicitly proposed a vacuum solution that connects two remote regimes \cite{Einstein:1935}. Ref. \cite{Misner:1957} firstly introduced the concept of WH. Then, the most simple WH named the Ellis WH was found by \cite{Ellis:1973} whose ADM mass is zero. To traverse the WH, one kind of traversable WH was introduced by  \cite{Morris:1988cz}. Then, the wormhole was extensively studied by \cite{Damour:2007ap,Kim:2003zb,Bueno:2017hyj,Visser:1989kh,Sushkov:2005kj,Bronnikov:2002rn,Clement:1995ya,Richarte:2017iit,Ayuso:2020vuu,KordZangeneh:2020ixt,Song:2023jdn,Saleem:2023lul,Eid:2023wrd,Godani:2023paj}. The way for realizing the WH, the introduction of exotic matter is mandantory whose energy density is negative, which would violate the Null energy condition (NEC) \cite{Hochberg:1998ii,Hochberg:1998ha,Lobo:2002zf,Lobo:2004rp}.

After all the WH is just a hypothetic object predicted by General Relativity. To detect its existence, the lensing effects played a vital role in observations. Once again, Einstein also was the first one to develop the lensing equation, which proposed the well-known concept "Einstein angle" \cite{Einstein:1936llh}. With the development of the technology for observation, gravitational lensing has become a standard method to detect astral objects including WH, dwarf, blackhole $\it e.t.c$. In modern astronomy, lensing mainly contains weak gravitational lensing, strong gravitational lens, and microlensing. Weak lensing was caused by the weak potential of the gravitational source which only slightly distorts the light as passing through some gravitational source \cite{Bartelmann:1999yn,Kaiser:1991qi}. While for the strong lensing effects, the situation is the opposite, the potential of gravitational source, including WH, blackhole, and so on, is so strong that leads to the strong distortion of the light \cite{Bozza:2002zj,Virbhadra:1999nm}.  The microlensing effect is an astronomical phenomenon caused by the weak gravitational lensing effect, which is used to detect objects from planetary mass to stellar mass. In this paper, we will focus on the microlensing effects. Refs. \cite{Perlick:2004tq,SDSS:2002oin} implemented the microlensing effect to explore WH. Thereafter, such extensive research of microlensing effects for WH were investigated by \cite{Gao:2022cds,Liu:2022lfb,Sokoliuk:2022owk,Zatrimaylov:2021ijd,Cheng:2021hoc,Li:2019qyb,Kuniyasu:2018cgv,Raidal:2018eoo,Tsukamoto:2017hva,Sajadi:2016hko,Lukmanova:2016czn,Tsukamoto:2016zdu,Kitamura:2016vad,Kitamura:2012zy,Kitamura:2012wcg,Toki:2011zu,Abe:2010ap,Bogdanov:2008zy,Torres:2001gb,Safonova:2001vz,Torres:1998cu,Torres:1998xd}. Being similar with optics, the gravitational source will bend the light, thus one can observe several images of light source after bending the light. How many images can be formed after bending the light that naturally became a question in the lensing effects, which was discussed in \cite{Kitamura:2016vad,Kuniyasu:2018cgv,Abe:2010ap,Liu:2022lfb}. 


Distinguishing the WH and blackhole is always an essential goal, especially from observation. As an attempt, our previous work tried to utilize magnification for distinguishing them \cite{Gao:2022cds}. However, this method was so difficult since the distance of the target object between the light source and gravitational source is un-fixed, which leads to the magnification always changing. To provide more possibilities, we will implement the REoS to re-formulate the magnification under Gauss-Bonnet Theorem (GBT). The most significant quantity is the deflection angle. There were so many works related to the deflection angle in terms of GBT \cite{Gibbons:2008rj,Gibbons:2008zi,Werner:2012rc,He:2023hsv,Upadhyay:2023yhk,Javed:2023iih,Javed:2022gtz,Huang:2022soh,He:2022yhp,Gao:2023ltr,Cai:2023ite,Ovgun:2023ego}. We will make use of REoS to investigate the event rate of WH, then it will be compared with blackhole when their mass and the distance between the light source and observer are comparable.  


This paper is organized as follows: In Sec. \ref{wormhole}, the REoS will be implemented to rewrite the static spherically symmetrical metric. In Sec. \ref{microlensing}, we will use GBT to calculate the deflection angle and lensing equation in light of REoS, meanwhile, the magnification and event rate will be discussed, which are also explicitly related to REoS. In Sec. \ref{conclusion and outlook}, we will give our conclusions and outlook. 


 






\section{Bascis of WH}
\label{wormhole}
In this section, we follow the notation in Refs. \cite{Lobo:2005us, Lobo:2005yv, Garattini:2007ff}. By starting with a spherically symmetical metric, 
\begin{equation}
\label{eq1}
ds^2=-e^{2\Phi}dt^2+\frac{dr^2}{1-b(r)/r}+r^2d\Omega^2,
\end{equation}
which describes a generic static and spherically symmetrical WH metric. By assuming the application of a perfect fluid, its corresponding Einstein equation can be derived as follows, 
\begin{equation}
\label{eq2}
p_r^\prime=\frac{2}{r}\big(p_t-p_r\big)-\big(\rho+p_r\big)\Phi^\prime,
\end{equation}
\begin{equation}
\label{eq3}
b^\prime=8\pi G\rho(r)r^2,
\end{equation}
\begin{equation}
\label{eq4}
\Phi^\prime=\frac{b+8\pi Gp_rr^3}{2r^2\big(1-b(r)/r\big)},
\end{equation}
where the prime denotes a derivative with respect to the radial coordinate $r$, $p_r$ represents the pressure in the radial component, $p_t$ indicates the tangential pressure and $\rho$ is the energy density. REoS is defined as follows,
\begin{equation}
\label{eq5}
p_r=\eta \rho.
\end{equation}
where $\eta$ represents REoS. The flaring-out condition and asymptotic flatness take the necessary condition:
\begin{equation}
\eta>0 \,\, \text{or} \,\, \eta<-1.
\label{flaring out condition}
\end{equation}
 Moreover, it should be noticed that $\eta > 0$ is possible only if $\rho < 0$.
Combining Eqs. \eqref{eq2}-\eqref{eq5}, one can get
\begin{equation}
\begin{aligned}
b(r) = r_0\bigg(\frac{r_0}{r}\bigg)^\frac{1}{\eta}e^{-(2/\eta)[\Phi(r)-\Phi(r_0)]}\times
\\
\bigg[\frac{2}{\eta} \int_{r_0}^r\big(\frac{r}{r_0}\big)^{(1+\eta)/\eta}\Phi^\prime(r)
e^{(2/\eta)[\Phi(r)-\Phi(r_0)]}dr+1\bigg].
\end{aligned}
\end{equation}
Since we focus on the microlensing effects, it means that the potential of WH is quite weak. For simplicity, one can reasonably assume that $\Phi(r)\approx constant=\Phi(r_0)$. More precisely, the potential varies very slowly from the initial value $\Phi(r_0)$. Therefore, the impact parameter can be simplified into 
\begin{equation}
b(r)=r_0\bigg(\frac{r_0}{r}\bigg)^{\frac{1}{\eta}}.
\label{simplified impact parameter}
\end{equation}
Substitute this simplified impact parameter \eqref{simplified impact parameter} into Eq. \eqref{eq1}, then the metric can be transformed towards, 
\begin{equation}
ds^2=-Adt^2+\frac{dr^2}{1-\big(r_0/r\big)^{1+\frac{1}{\eta}}}+r^2d\Omega^2,
\label{simplified metric}
\end{equation}
where $A=e^{2\Phi}$. One can easily observe that metric \eqref{simplified metric} will become an Ellis-Bronnikov WH if the factor $A$ was absorbed into temporal part and $\eta=1$. Next section, we will implement REoS to investigate the microlensing effect of metric \eqref{simplified metric}. 







\begin{figure}
    \centering
    \includegraphics[scale=0.7]{wormhole_image_1.png}
    \caption{A brief illustration of WH. The WH is connecting two remote regimes of spacetime. In our case, we consider the lensing effects occurring on one side of the WH that is in spacetime 1 or spacetime 2. }
    \label{fig:my_label}
\end{figure}





\section{Microlensing}
\label{microlensing}
In this section, we will calculate the magnification and the event rate of metric \eqref{simplified metric}. First and foremost, the GBT will be implemented for further investigations. 







\subsection{Deflection angle}
\label{deflection angle}
In lensing effects including microlensing, the most essential quantity is the deflection angle, which will be calculated by GBT under the weak field approximation. Before introducing GBT, the optical Gaussian curvature will be computed, where the photon is traveling, thus the metric is the light-like case corresponding to $ds^2=0$. Our calculation can be performed in the equatorial plane due to the spherical symmetry. Then, the metric \eqref{simplified metric} will become 
\begin{equation}
\label{eq 8}
dt^2=\frac{dr^2}{A\left(1-\big(r_0/r\big)^{1+\frac{1}{\eta}}\right)}+\frac{r^2}{A}d\phi^2.
\end{equation}
For convinence, we introduce two auxiliary quantities:  $du=\frac{dr}{\sqrt{A\big(1-\big(r_0/r\big)^{1+\frac{1}{\eta}}\big)}}$ and $\xi=\frac{r}{\sqrt{A}}$. Gaussian optical curvature can be expressed as, 
\begin{equation}
K=\frac{-1}{\xi(u)}[\frac{dr}{du}\frac{d}{dr}\big(\frac{dr}{du}\big)\frac{d\xi}{dr}+\big(\frac{dr}{du}\big)^2\frac{d^2\xi}{dr^2}],
\end{equation}
combine with metric \eqref{eq 8}, one can get
\begin{equation}
\label{eq10}
K=\frac{-\sqrt{A}r_0\big(\frac{r_0}{r}\big)^\frac{1}{\eta}\big(1+\frac{1}{\eta}\big)}{2r^3\sqrt{1-\big(\frac{r_0}{r}\big)^{1+\frac{1}{\eta}}}}.
\end{equation}
Once obtaining the Guassian curvature, we could introduce the GBT whose formula is given by 
\begin{equation}
\int\int_D KdS+\int_{\partial D}\kappa dt+\sum\limits_i\alpha_i=2\pi\chi(D),
\label{GBT0}
\end{equation}
where $D$ is the integral domain denoted in Fig. \ref{fig:1}.  Choosing $OS$ as the geodesic line, thus the integral along $OS$ is zero. Besides, this
Euler index $\chi$ is one n domain $D$. Then, GBT \eqref{GBT0} will become 
\begin{equation}
\int\int_{D}KdS+\int_{\gamma_P}\kappa dt+\sum_i\alpha_i=2\pi.
\end{equation}
\begin{figure}
	\centering
	\includegraphics[scale=0.8]{gauss_bonnet_1.png}
	\caption{Illustration of the GBT integral domain. $O$ is the observer, and $S$ is the light source. The region $D$ represents the integral domain of the GBT, and integrating over this domain gives us the total deflection angle experienced by the light.}
	\label{fig:1}
\end{figure}
One can set $\gamma$ to vertically intersect with geodesic line $OS$ at point $O$ and point $S$, Then the sum of external angles is $\pi$ as showing
\begin{equation}
\sum_i\alpha_i=\frac{\pi}{2}(S)+\frac{\pi}{2}(O)=\pi.
\end{equation}
Then, an integral transformation can be done as follows, 
\begin{equation}
\kappa dt=\kappa\frac{dt}{d\phi}d\phi.
\end{equation}
Here $\phi$ is some kinds of angular coordinate where the gravitational source $W$ is the central point as showing in Fig. \ref{fig:2}.  It can be done to set up $\kappa\frac{dt}{d\phi}=1$ on $\gamma$, therefore we have
\begin{equation}
\int\int_{D}KdS+\int_{\phi_O}^{\phi_S}d\phi+\pi=2\pi.
\end{equation}
Geodesic line $OS$ could approximate to be a straight line. Due to the existence of lensing effects, the range of $\phi$ could span from zero (at point $O$) to $\pi+\alpha$ (at point S), where $\alpha$ is the so-called deflection angle,
\begin{equation}
\int\int_{D}KdS+\int_{0}^{\pi+\alpha}d\phi+\pi=\int\int_{D}KdS+\pi+\alpha+\pi=2\pi.
\end{equation}
Then, the deflection angle is obtained as follows, 
\begin{equation}
\alpha=-\int\int_{D}KdS.
\label{deflection angle1}
\end{equation}
Being armed with previous calculations for Gaussian curvature \eqref{eq10}, the deflection angle can be furtherly determined by 
\begin{equation}
\label{eq11}
\alpha=-\int_0^\pi\int_{\frac{b}{\sin\phi}}^\infty K\sqrt{\det  h_{ab}} drd\phi,
\end{equation}
where $b$ is impact parameter and $h_{ab}$ is the optical metric in terms of $u$ and $\phi$. Substituting Eq. \eqref{eq10} to \eqref{eq11}, one can obtain
\begin{equation}
\label{eq 22}
\alpha=\int_0^\pi\int_{\frac{b}{\sin\phi}}^\infty\frac{r_0\big(\frac{r_0}{r}\big)^\frac{1}{\eta}\big(1+\frac{1}{\eta}\big)}{2\sqrt{A}r^2\big(1-\big(\frac{r_0}{r}\big)^{1+\frac{1}{\eta}}\big)}drd\phi.
\end{equation}
Under the weak field approximation $\frac{r_0}{r}\ll 1$, 
\begin{equation}
\label{eq13}
\alpha=\frac{\sqrt{\pi}\big(\frac{r_0}{b}\big)^{1+\frac{1}{\eta}}\eta\Gamma[1+\frac{1}{2\eta}]}{2\sqrt{A}\Gamma[\frac{1}{2}\big(3+\frac{1}{\eta}\big)]}, ~~~\text{if } \frac{1}{\eta}>-2.
\end{equation}
 Being armed with deflection angle \eqref{eq13}, one can investigate its corresponding lensing equation. As for the special case of $A=1$ and $\eta=1$, our metric becomes the Ellis-Bronnikov WH:
 \begin{equation}
 ds^2=-dt^2+\frac{dr^2}{1-\big(\frac{r_0}{r}\big)^{2}}+r^2d\Omega^2.
 \end{equation}
 The deflection angle for Eq. \eqref{eq 22} ($\eta=1$) is given by
 \begin{equation}
 \alpha=\frac{\pi}{4}\big(\frac{r_0}{b}\big)^2,
 \end{equation}
where it is agreed with Refs. \cite{Gao:2022cds,Nakajima:2012pu,Jusufi:2017gyu} in the first order. 



\subsection{Lensing equation}

According to Fig. \ref{fig:2}, the plane geometry could yield the lensing equation
\begin{equation}
\label{eq14}
\beta=\theta-\frac{D_{LS}}{D_S}\alpha.
\end{equation}
\begin{figure}
	\centering
	\includegraphics[scale=0.8]{lens_geometry.png}
	\caption{Showing lens plane geometry. $I$ is the location of image of light source, $S$ is location of light source, $\alpha$ is the deflection angle, $W$ is the  WH and $b$ is the impact parameter. All of these angles are much less than unity.}
	\label{fig:2}
\end{figure}
Substitute Eq. \eqref{eq13} to Eq. \eqref{eq14}, one can obtain that
\begin{equation}
\theta^{2+\frac{1}{\eta}}-\beta\theta^{1+\frac{1}{\eta}}-\frac{D_{LS}}{D_S}\frac{\sqrt{\pi}\big(\frac{r_0}{D_L}\big)^{1+\frac{1}{\eta}}\eta\Gamma[1+\frac{1}{2\eta}]}{2\sqrt{A}\Gamma[\frac{1}{3}\big(3+\frac{1}{\eta}\big)]}=0,
\label{general lens eq}
\end{equation}
where we have used the approximation $b\approx\theta D_L$. According to Eq. \eqref{general lens eq}, one can explicitly obtain the relation between the order of lensing equation (n) and $\eta$, 
\begin{equation}
\label{eq 16}
n=2+\frac{1}{\eta}.
\end{equation}
This is an equation about the number of images for light source (If there are $n$ various real solutions of the $n-th$ order equation). When $\eta\rightarrow 0_+$, then $n\rightarrow\infty$, this means that we can at most get an infinite numbers of image. On the other hand, when $\eta\rightarrow\pm\infty$, it expcts to be two images as showing in Einstein ring. However, the realistic situation is more complicated. One can actually obtain $n$ solutions of lensing Eq. \eqref{general lens eq}, in which some of them are complex solutions that one cannot observe its corresponding images. We take $\eta=1$ as an illustration since the charged WH \cite{Kim:2001ri} and WH with quantum corrections \cite{Jusufi:2018kmk}, $\it e.t.c,$ are all in this case. Thereafter, lensing equation \eqref{general lens eq} becomes 
\begin{equation}
	\theta^3-\beta\theta^2-M=0,
	\label{lens eq with eta=1}
\end{equation}
where we have set $\frac{D_{LS}}{D_S}=\frac{1}{2}$ for simplicity and $M=\frac{\pi}{16\sqrt{A}\Gamma[4/3]}\frac{r_0^2}{D_L^2}>0$. Lensing equation \eqref{lens eq with eta=1} is the third order equation in terms of $\theta$, in which its overall discriminant $\Delta>0$ in light of appendix I in \cite{Liu:2022lfb}. Thus, there is only a real solution for Eq. \eqref{lens eq with eta=1} corresponding to only one image of the light source. As for the second order of $\theta$, the value of $\eta$ is huge whose value will set to be $10$, then the lensing equation \eqref{general lens eq} will be written by
\begin{equation}
\theta^2-\beta\theta-\frac{5}{2\sqrt{A}}\frac{r_0}{D_L}=0,
\label{second order lensing eq}	
\end{equation}
where we also have set $\frac{D_{LS}}{D_S}=\frac{1}{2}$ and its overall discriminant is also larger than zero, thus it will be of two real solutions corresponding to two images of the light source. When $n=4$ and $\sqrt{\frac{\pi}{A}}\frac{1}{8\Gamma[5/3]}\frac{r_0^3}{D_L^3}=0.08$, the lensing equation \eqref{general lens eq} could have four real solutions corresponding to the four images of light source.  As for the higher order equation of $\theta$ ($n>4$), it is more complicated whose real solutions are difficult to determine. Although we have obtained the explicit relation $n=2+1/\eta$, the number of images of the light source is still not figured out in some sense. 











\subsection{Magnification}

Similar to optics, the images of the light source will be magnified or demagnified due to varying the cross-section of light rays, whose definition is determined by the ratio between distinct solid angles, 
\begin{equation}
\label{eq 27}
\mu_{\rm total}=\sum_i \bigg|\frac{\beta}{\theta_i}\frac{d\beta}{d\theta_i}\bigg|^{-1},
\end{equation}
where $\theta_i$ is the angle of the $i-th$ image of light  source.
Substituting Eq. \eqref{eq14} into Eq. \eqref{eq 27}, which leads to


\begin{widetext}
\begin{equation}
\mu=\left|\frac{\pi  D_L D_{LS} 2^{-\frac{1}{\eta }-2} r_0 \left(\frac{r_0}{b}\right)^{1/\eta } \left(\sqrt{A} b^2 D_S\Gamma \left(2+\frac{1}{\eta }\right)-D_L D_{LS} 2^{1/\eta } \eta  (\eta +1) r_0 \Gamma \left(1+\frac{1}{2 \eta }\right)^2 \left(\frac{r_0}{b}\right)^{1/\eta }\right)}{A b^4 D_S^2 \Gamma \left(\frac{1}{2} \left(3+\frac{1}{\eta }\right)\right)^2}+1\right|^{-1}.
\label{total mag}
\end{equation}
\end{widetext}
Eq. \eqref{total mag} provides a general formula for calculating the magnification of the static spherically symmetric WH. Due to the flaring-out condition \eqref{flaring out condition}, REoS can be divided into two regimes: $(-\infty,-1)$ and $(0,\infty)$. We have chosen specific values of $\eta$ to investigate the magnifcation including in Figs. \ref{fig: 4} and \ref{fig: 5}.
\begin{figure}
	\center
	\includegraphics[scale=0.72]{MG1.pdf}
	\caption{In the case of $\eta>0$, the magnification given by Eq. \eqref{total mag} varies with the impact parameters $b$ (Unit: kpc), and the legend in the figure indicates the corresponding REoS values for different curves. Here we set the parameters as follows: $D_S=2D_L=2D_{LS}=20$ kpc, $r_0=1\times 10^{-10}$ kpc, and $A=1$. According to $n=2+\frac{1}{\eta}$, $n=5$ with $\eta=\frac{1}{3}$, $n=4$ with $\eta=\frac{1}{2}$ and $n=3$ with $\eta=1$, in which the peak is not including $\eta=2,5$ since it is beyond the scope.}
	\label{fig: 4}
\end{figure}

\begin{figure}
	\centering
	\includegraphics[scale=0.85]{MG2.pdf}
	\caption{In the case of $\eta<-1$, the magnification given by Eq. \eqref{total mag} also varies with the impact parameters $b$, and the legend in the figure indicates the corresponding REoS values for different curves. We set the parameters as follows: $D_S=2D_L=2D_{LS}=20$ kpc, $r_0=1\times 10^{-10}$ kpc, and $A=1$. The curves where $\eta=-5$ and $\eta=-10$ have almost overlapped. The case of $\eta=-1$  corresponds to $n=1$ (only one image of light source) whose peak is also not included in this scale.}
	\label{fig: 5}
\end{figure}
 The trend of magnification in Figs. \ref{fig: 4} and \ref{fig: 5} are the same, which there is only one peak. With the enhancement of the value of $\eta$, the position of the corresponding magnification peak will occur in larger values of $b$. Thus, the cases of $\eta=2,5,-1$ are not shown in Figs. \ref{fig: 4} and \ref{fig: 5}, in which one cannot observe any image of light source since the magnification is zero in this scale (the range of $b$ as showing in Figs. \ref{fig: 4} and \ref{fig: 5}). The total varying trend is that: the demagnification will appear at some certain scale, thereafter the magnification will approach the maximal value (the peak of magnification), and finally it will tend to be one that the size of image of light source is the same with the light source itself. The value of $\eta$ could highly impact the position of the peak of magnification since it mainly influences the mass of WH whose details will be thoroughly investigated in Sec. \ref{event rate 1}. A simple analysis could understand this physical picture, the larger mass of WH will more significantly distort the image of light source. 
 
 
\begin{figure}
	\centering
	\includegraphics[scale=0.85]{MG3.pdf}
	\caption{The magnification varies with the radius $r_0$ of
		 throat for WH. The various values of $r_0$ correspond to different curves indicated in the legend. We set the parameters as follows: $D_S=2D_L=2D_{LS}=20$ kpc, $\eta=1$, $A=1$.}
	\label{fig: 6}
\end{figure}
For completeness, we also give a plot of magnification highly related to the radius $r_0$ of WH as shown in Fig. \ref{fig: 6}. The total trend is also the same only containing one peak of magnification. With the larger values of $r_0$, the position of the peak will appear in larger values of $b$ since the $r_0$ explicitly relates to the mass of WH. We need to emphasize that the demagnification of Kerr blackhole will be influenced by its angular momentum \cite{Johnson:2019ljv,Gralla:2019drh}, thus we may extend our method to the rotating blackhole and WH. To sum up, the factor who influences the mass of WH will highly impact the position of the peak of $\mu$. 


 







\subsection{Event rate}
\label{event rate 1}
The microlensing effect is a rare phenomenon in observations. To describe the probability of this event, we need to calculate the optical depth $\tau$. The optical depth represents the probability of observing the microlening event of a source at a certain location $D_S$, which reflects the number of microlensing events per unit of time. If we observe N light sources, we can calculate the microlensing event rate $\Gamma=\frac{d (N \tau)}{dt}$. Here, we aim to develop an analytic formula for calculating the event rate under the metric \eqref{simplified metric}. For better understanding the event rate, we give a Fig. \ref{fig: 7} as an illustration. 
 \begin{figure}
	\centering
	\includegraphics[scale=0.8]{event_rate_image.png}
	\caption{The illustration of the microlensing rate of WH
		moving along this $2D$ plane. This $2D$ plane is the source plane, and we assume that the number of sources within the Einstein ring is $\chi\sigma _{micro}$.}
	\label{fig: 7}
\end{figure}

Our work is beginned by metric \eqref{simplified metric}. First, the effective mass of WH comes via Ref. \cite{Alcubierre:2017pqm} that is defined by  
\begin{equation}
M=\frac{r_0}{2}+\int_{r_0}^r4\pi\rho(r^\prime)r^{\prime 2}dr^\prime,
\end{equation}
where the energy density can be found in light of Einstein equation, 
\begin{equation}
\rho=-\frac{Ar_0(\frac{r_0}{r})^{\frac{1}{\eta}}}{r^3\eta}\frac{c^4}{8\pi G}.
\end{equation}
Therefore, the effective mass can derived by 
\begin{equation}
M=\frac{A c^4r_0\left(\frac{r_0}{D_L}\right)^{\frac{1}{\eta} }}{2 G}-\frac{A c^4 r_0}{2 G}+\frac{r_0}{2},
\end{equation}
where the integration range is from $r_0$ to $D_L$ as showing in Fig. \ref{fig: 7}. Bofore calculating the event rate, what we need is the Einstein angle whose definition is $\theta_E=\frac{D_{LS}}{D_S}\alpha$ in light of \eqref{eq14}. Then we could implement Eq. \eqref{eq13} to plug into the definition of Einstein angle, one can obtain 
\begin{equation}
\theta_E=\left(\frac{r_0}{D_L}\right)^{\frac{1+\frac{1}{\eta}}{2+\frac{1}{\eta}}}\left(\frac{\sqrt{\pi}\eta\Gamma[1+\frac{1}{2\eta}]}{2\sqrt{A}\Gamma[\frac{1}{2}(3+\frac{1}{\eta})]}\frac{D_{LS}}{D_S}\right)^{\frac{1}{2+\frac{1}{\eta}}},
\end{equation}
where it is explicitly related to Einstein ring where light source, gravitational lensing source and obeserver are aligned. Once obtaining the Einstein angle, one could also define  the cross-section for microlensing as follows, 
\begin{equation}
\sigma_{micro}=\pi\theta^2_E,
\label{solid angle}
\end{equation}
where it is the solid angle producing a detectable microlensing signal. Due to the relative motion between the light source and lens, one can also define the Einstein radius crossing time 
\begin{equation}
\label{te}
t_E=\frac{r_E}{v}=\frac{D_S\theta_E}{v},
\end{equation}
where $v$ is the relative velocity between the light source and lensing source as shown in Fig. \ref{fig: 7}. For simplification, the lensing source could be fixed and the light source is travelling at speed $v$ whose direction is also showing in Fig. \ref{fig: 7}. The optical depth gives rise to a detectable probability of microlensing event, which defines as follows,
\begin{equation}
\tau=\frac{1}{\Omega}\int_0^{D_S}\sigma_{micro}dN_L,
\end{equation}
In a simple analysis, the number of lens sources is changing as varying with $D_L$. Therefore, we could further derive that 
\begin{equation}
dN_L=\Omega D_L^2n(D_L)dD_L.
\end{equation}
Then, the optical depth is
\begin{equation}
\tau(D_S)=\frac{1}{\Omega}\int_0^{D_S}[\Omega D_L^2n(D_L)](\pi\theta_E^2)dD_L.
\label{tau1}
\end{equation}
There is only one lensing source, its energy density is a constant which leads to $n(D_L)=\frac{\rho}{m}$. Therefore, we have 
\begin{equation}
\tau(D_S)=\int_0^{D_S}D_L^2  \frac{\rho(D_L)}{M}\pi \theta_E^2dD_L.
\label{tau}
\end{equation}
The integration interval is $(0, D_S)$, which can be divided into $(D_L,r_0)\cup(r_0,D_{LS})$ according to Fig. \ref{fig:2}. In the interval $(0,r_0)$, the integration of \eqref{tau} is zero since we have neglected the inner structure of WH.  Then, we set parameters as $c=G=A=1$ and $D_S=2D_L=2D_{LS}$. Finally, our integration result is
\begin{equation}
\label{eq. 39}
\left|
\frac{\left(\frac{\sqrt{\pi}\eta\Gamma[1+\frac{1}{2\eta}]}{2\Gamma[\frac{1}{2}(3+\frac{1}{\eta})]}\frac{D_{LS}}{D_S}\right)^{\frac{2}{2+\frac{1}{\eta}}} \left(r_0^{\frac{2(1+\eta)}{1+2\eta}}D_{LS}^{-\frac{2(1+\eta)}{1+2\eta}}-1 \right)   (1+2\eta) } {4\eta(1+\eta)}\right|.
\end{equation}
The obsolute value comes via the positivity of probability. 
We can differentiate the optical depth in light of Eq. \eqref{fig:2}, 
\begin{equation}
d\tau=\frac{1}{\Omega}\int_0^{D_S}n(D_L)\Omega 2r_EvdtdD_L=\int_0^{D_S}2n(D_L)r_E^2\frac{dt}{t_E}dD_L.
\end{equation}
We may observe the microlensing event, while we are monitoring a certain number of light sources $N$ (dubbed as constant) within a specific time, its corresponding event rate is defined as
\begin{equation}
\label{eq 41}
\Gamma=\frac{d(N\tau)}{dt}=\frac{2N}{\pi}\int_0^{D_S}n(D_L)\frac{\pi r_E^2}{t_E}dD_L=\frac{2N}{\pi t_E}\tau.
\end{equation}
Substituting the previous calculation results Eq. \eqref{te} and Eq. \eqref{eq. 39} into Eq. \eqref{eq 41}, we obtain


\begin{widetext}




\begin{equation}
\label{event rate}
\Gamma=\frac{2\chi \sigma_{micro}}{\pi\frac{D_S}{v}\left(\frac{r_0}{D_L}\right)^{\frac{1+\frac{1}{\eta}}{2+\frac{1}{\eta}}}\left(\frac{\sqrt{\pi}\eta\Gamma[1+\frac{1}{2\eta}]}{2\Gamma[\frac{1}{2}(3+\frac{1}{\eta})]}\frac{D_{LS}}{D_S}\right)^{\frac{1}{2+\frac{1}{\eta}}}}\times
\left|\left(\frac{\sqrt{\pi}\eta\Gamma[1+\frac{1}{2\eta}]}{2\Gamma[\frac{1}{2}(3+\frac{1}{\eta})]}\frac{D_{LS}}{D_S}\right)^{\frac{2}{2+\frac{1}{\eta}}}
\frac{ \left(r_0^{\frac{2(1+\eta)}{1+2\eta}}D_{LS}^{-\frac{2(1+\eta)}{1+2\eta}}-1 \right) (1+2\eta)  } {4\eta(1+\eta)}\right|.
\end{equation}




\begin{table}
\centering
\begin{tabular}{|c |c| c| c| c| c| c| c| c| c|} 
 \hline
 $\eta$ & $\chi$ & $v$ & $r_0$ & $Mass$ & $\theta_E$ & $r_E$ & $t_E$ & $\tau$ & $\Gamma$ \\ 
 - & & $\rm m/s$ & $\rm m$ & $\rm M_\odot$ & $\rm rad$ & $\rm m$ & $\rm year$ & &  $\rm year^{-1}$ \\ 
 \hline
 \hline
  -10 & $1.00\times 10^{14}$ & $3.00\times 10^4$ & $3.24\times 10^{9}$ & $1.38\times 10^7$ & $\mathbb{C}$ & $\mathbb{C}$ & $\mathbb{C}$ & 0.297 & $\mathbb{C}$\\
 -2 & $1.00\times 10^{14}$ & $3.00\times 10^4$ & $3.24\times 10^{9}$ & $3.46\times 10^{11}$ & $\mathbb{C}$ & $\mathbb{C}$ & $\mathbb{C}$ & 0.477 & $\mathbb{C}$\\ 
  -1.1 & $1.00\times 10^{14}$ & $3.00\times 10^4$ & $3.24\times 10^{9}$ & $1.09\times 10^{16}$ & $\mathbb{C}$ & $\mathbb{C}$ & $\mathbb{C}$ & 1.84 &  $\mathbb{C}$\\
0.33 & $1.00\times 10^{14}$ & $3.00\times 10^4$ & $3.24\times 10^{9}$ & $\approx 0$ & $9.67\times 10^{-10}$ & $6.25\times 10^{11}$ & 0.660 & 0.374 & $1.06\times 10^{-4}$ \\
0.5 & $1.00\times 10^{14}$ & $3.00\times 10^4$ & $3.24\times 10^{9}$  & $\approx 0$ & $3.59\times 10^{-9}$ & $2.32\times 10^{12}$ & 2.46 & 0.272  & $2.86\times 10^{-4}$\\
 1 & $1.00\times 10^{14}$ & $3.00\times 10^4$ & $3.24\times 10^{9}$ & 1.09$\times 10^{-5}$ & $3.40\times 10^{-8}$ & $2.20\times 10^{13}$ & 23.2 & 0.201  & $2.00\times 10^{-3}$ \\
  1.5 & $1.00\times 10^{14}$ & $3.00\times 10^4$ & $3.24\times 10^{9}$ & 5.08$\times 10^{-2}$ & $1.12\times 10^{-7}$ & $7.25\times 10^{13}$ & 76.7 & 0.189  & $6.20\times10^{-3}$ \\
  2 &$1.00\times 10^{14}$ & $3.00\times 10^4$ & $3.24\times 10^{8}$ & 0.109 & $5.98\times 10^{-8}$ & $3.87\times 10^{13}$ & 40.9 & 0.187 &   $3.27\times 10^{-3}$\\
 2 &$1.00\times 10^{14}$ & $3.00\times 10^4$ & $3.24\times 10^{9}$ & 3.46 & $2.38\times 10^{-7}$ & $1.54\times 10^{14}$ & 163 & 0.187 &  $1.30\times 10^{-2}$ \\
  2 &$1.00\times 10^{14}$ & $3.00\times 10^4$ & $3.24\times 10^{10}$ & 109 & $9.48 \times 10^{-7}$ & $6.13\times 10^{14}$ & 648 & 0.187 &  $5.19\times 10^{-2}$ \\
  2.5 &$1.00\times 10^{14}$ & $3.00\times 10^4$ & $3.24\times 10^{9}$ & 43.6 & $4.02\times 10^{-7}$ & $2.60\times 10^{14}$ & 274 & 0.188 &  $2.21\times 10^{-2}$ \\
 3 & $1.00\times 10^{14}$ & $3.00\times 10^4$ & $3.24\times 10^{9}$ & 236 & $5.92\times 10^{-7}$ & $3.83\times 10^{14}$ & 405 & 0.191 & $3.30\times 10^{-2}$ \\
 
 \hline
\end{tabular}
\caption{The change of event rate $\Gamma$ with $\eta$. $\chi$ is the number of sources observed per unit angular area $\pi \theta^2$, $v$ is the relative velocity between the WH and the source plane, $r_0$ is the throat radius of the WH. In terms of mass, we choose solar mass $\rm M_\odot$ as the unit, $\theta_E$ is the Einstein angle, $r_E$ is the Einstein radius, and $\tau$ is the optical depth. We set the parameters $D_S=21~\rm kpc$, and $A=1$.}
\label{table:1}
\end{table}    

\begin{table}
    \centering
    \begin{tabular}{|c || c| c| c| c| c| c| c|}
    \hline
     $r_0$ ($km$) & $3.24\times 10^1$ & $3.24\times 10^2$ &  $3.24 \times10^3$ &  $3.24 \times10^4$ &  $3.24 \times10^5$ &  $3.24 \times10^6$ &  $3.24 \times10^7$  \\
      \hline
   $\Gamma$ ($year^{-1}$)  &  $1.86\times 10^{-6}$ & $8.62\times 10^{-6}$ & $4.00\times 10^{-5}$ & $1.86\times 10^{-4}$  & $8.62  \times 10^{-4}$ & $4.00 \times 10^{-3}$ & $1.86 \times 10^{-2}$ \\
    \hline
    \end{tabular}
    \caption{The numerical results by Eq \eqref{event rate} : event rate $\Gamma$ of Ellis-Bronnikov WH corresponding to throat radius $r_0$. $D_S=21~\rm kpc$ is assumed. $v=0.0001c$, $\chi=2\times 10^{14}$ and $n(D_L)=\frac{\rho}{M}$ are assumed.}
    \label{tab: 2}
\end{table}

\begin{table}
    \centering
    \begin{tabular}{|c||c| c| c| c| c| c| c|}
    \hline
        $r_0$ ($km$) & 10 & $10^2$ &  $10^3$ &  $10^4$ &  $10^5$ &  $10^6$ &  $10^7$ \\
    \hline
   $\Gamma$ ($year^{-1}$) &  $1.88 \times 10^{-14}$ & $8.73 \times 10^{-14} $ & $4.05  \times 10^{-13}$ & $1.88 \times 10^{-12} $ & $8.73  \times 10^{-12}$ & $4.05 \times 10^{-11}$ & $1.88 \times 10^{-10}$ \\
   \hline
    \end{tabular}
    \caption{The numerical results by \cite{Abe:2010ap}: The various event rates $\Gamma$ of Ellis-Bronnikov WH correspond to different throat radius $r_0$. $D_S=8~\rm kpc$ is assumed. $v=5000~ \rm km/s$ and $n=4.97\times 10^{-9}~\rm pc^{-3}$ are assumed.
}
    \label{tab: 3}
\end{table}

\end{widetext}
where we express $N$ as $\chi\sigma_{micro}$, $\chi\propto D_S^2$ is constant determined by observation. $\chi$ has a significant impact on the event rate that leads to $\Gamma\propto \chi$. Comparing with \cite{Zaris:2019soz,Sollima:2009wh,Noyola:2010ab}, we reasonably set $\chi=1\times 10^{14}$, $D_S=2D_L=2D_{LS}=21~\rm kpc$ and $v=3\times10^{4}~\rm m/s$. Being armed with these parameters, one can determine the mass of WH and Einstein's angle. To illustrate how $\eta$ impacts the event rate, all of our calculations are listed in Tab \ref{table:1}. It indicates that the event rate and Einstein angle will be enhanced by increasing the value of $\eta$ but not for optical depth. One could see that $\Gamma$ is of order $10^{-2}$ when there are two images per light source. Another point needs to be noticed is that it is meaningless as $\eta<-1$ since the $\Gamma$ is a complex number. Finally, it could be seen that there are at least two images per light source based on $n=2+\frac{1}{\eta}$. For completeness, we also give a plot to show how the radius $r_0$ of WH impacts $\Gamma$ shown in Fig. \ref{fig: event rate}. It explicitly indicates that the event rate will be enhanced by increasing the value of $r_0$. We also show the results of \cite{Abe:2010ap} in Tab. \ref{tab: 3}, where they computed the event rate of clusters in Ellis-Bronnikov WHs with different throat radii. It indicates that the event rate will be enhanced by improving the value of throat radii which supports our numerical results. Their results are different from ours since they consider the cluster of WHs. It leads to the mass becoming sparse which is different from our assumption, where we only consider one WH as the lens source whose energy density is larger. To sum up, the event rate will be larger as increasing the value of $\eta$ and $r_0$.

Additionally, we are curious if event rate can be used to distinguish between black holes and WHs. In Ref. \cite{Kiroglu:2021mej}, they implement  the CMC Cluster Catalog model to study
 $n8-rv0.5-rg8-z0.1$ case, where the mass of lens is $M=2\times 10^5~\rm M_\odot$ and $t=12 \rm ~Gyr$, then one can obtain the event rate $<10^{-5}~\rm year^{-1}$ for the single blackhole. In comparison, we set $\eta=1$ and $r_0=3.24\times 10^{14}~\rm m$ corresponding to $M=1.1\times 10^5~\rm M_{\odot}$. We could observe $5.0\times 10^5$ light sources with an Einstein radius crossing time of $5.0\times 10^5~\rm year$. When the $M$ and $t$ converted into the $n8-rv0.5-rg8-z0.1$ case, its corresponding event rate is about  $10^{-2}\sim 10^{-3}~\rm year^{-1}$. Our estimation shows that 
the event rate of WH is two orders higher compared with blackhole as the mass and Einstein crossing time are comparable. We should emphasize the importance of Tab. \ref{table:1}, in which the observation can explicitly compare with our numerical results as fixing $t$, mass of WH, and the number of light sources. If it is compatible with our predictions, it could imply that the lensing object is WH and the value is smaller than ours which could be blackhole.




\begin{figure}
    \centering
    \includegraphics[scale=1.0]{event_rate1.jpg}
    \caption{The relationship between the number of microlensing events observed each year and REoS. Each curve in the figure corresponds to a different wormhole throat radius. We set other parameters consistent with Tab \ref{table:1}.}
    \label{fig: event rate}
\end{figure}














\section{Conclusion and outlook}
\label{conclusion and outlook}
This paper presents a comprehensive investigation of the microlensing effects of the static spherically symmetric WH described by Eq. \eqref{simplified metric}. By introducing the so-called REoS parameter $\eta=\frac{p_r}{\rho}$, we reformulate the metric, allowing us to re-examine the microlensing effect metric \eqref{simplified metric}, including its magnification and event rate. We employ the GBT to calculate the deflection angle of metric \eqref{simplified metric} under the weak field approximation.  The resulting lensing equation contains an explicit formula $n=2+\frac{1}{\eta}$ that reflects the order of the lensing equation on the equatorial plane. We take $\eta=1$ ($n=3$) as an illustration, in which we have shown that there is only one real solution of lensing equation \eqref{general lens eq} since the overall discriminant is larger than zero, which means that there is only one image of the light source. As for $n=2$ (we have set $\eta=10$), it shows that there are two real solutions of \eqref{general lens eq}.  Even one could observe four images of the light source as $n=4$, where we have fixed the values of $\frac{r_0}{D_L}$, $A$ and $D_{LS}/D_S$. For a higher-order lensing equation, it is more complicated. In some sense, the image problem of the light source is still not worked out due to the complication of lensing equation \eqref{general lens eq}. 


In order to reformulate the lensing equation, we also derive a general formula for calculating the magnification in terms of REoS. This formula allows us to analyze how the magnification changes with REoS and the WH throat radius $r_0$. Our analysis reveals that the larger values of $\eta$ will lead to the position of magnification's peak located at the larger values of impact parameter $b$ as shown in Figs. \ref{fig: 4} and \ref{fig: 5}. A similar trend is applied for the radius of the throat of WH as shown in Fig. \ref{fig: 6}. To provide a more complete analysis, we also perform the analytical calculations of event rate for a single WH source. We mainly list our numerical results in Tab. \ref{table:1} and Fig. \ref{fig: event rate}, which clearly indicates that the larger values of $\eta$ and $r_0$ will cause the larger values of event rate. As one example of our metric, the investigation of Ref. \cite{Abe:2010ap} listed in Tab. \ref{tab: 3} holds the same results as ours. Especially, our results as shown in Tab. \ref{table:1} could provide an explicit comparison with observations as fixing $t$, mass of WH, and the number of light source, which guides for distinguishing the WH and blackhole via event rate.

Our work is just a preliminary investigation of a single gravitational source. The lensing effects can be applied to more realistic situations, $\it i.e.$ the primordial blackhole plays an important role of dark matter, we could extend our method to this direction \cite{Cai:2022kbp}. Another natural extension is for the microlensing of the clusters of galaxy \cite{Kiroglu:2021mej}. One can utilize the approximated metric to mimic their dynamical background. The difference comes via the energy-momentum tensor. Further, we could also extend our method to the strong lensing regime that may include the topological effects of spacetime, in which the deflection angle and lensing equation are different. To fully address this issue, the calculation technology should be developed. 





 \section*{Acknowledgements}
 We appreciate that Hai-Qing Zhang and Bi-Chu Li give lots of suggestions to improve this manuscript. And we are also grateful to Prof. Wentao Luo that he gives professional guidance on some concepts of microlensing. We are very grateful for the critical reading from Dr. Xin-Fei Li. LH and KG are funded by NSFC grant NO. 12165009 and Hunan Natural Science Foundation NO. 2023JJ30487. 







\section*{References}
\begin{thebibliography}{99}


%\cite{Flamm:1916}
\bibitem{Flamm:1916}
Ludwig, Flamm,
%
Beitr¨age zur Einstein schen gravitations theorie. Hirzel, 1916.
%2 itations counted in INSPIRE as of 14 Mar 2023

%\cite{Einstein:1935}
\bibitem{Einstein:1935}
Albert Einstein and Nathan Rosen.
% 
“The particle problem in the general theory of relativity”.
In: Physical Review 48.1 (1935), p. 73.
%

%\cite{Misner:1957}
\bibitem{Misner:1957}
Charles W Misner and John A Wheeler.
% “Classical physics as geometry”. In: Annals of
physics 2.6 (1957), pp. 525–603.
%

%\cite{Ellis:1973}
\bibitem{Ellis:1973}
Homer G Ellis.
% “Ether flow through a drainhole: A particle model in general relativity”. In:
Journal of Mathematical Physics 14.1 (1973),
pp. 104–118.
%

%\cite{Morris:1988cz}
\bibitem{Morris:1988cz}
M.~S.~Morris and K.~S.~Thorne,
%``Wormholes in space-time and their use for interstellar travel: A tool for teaching general relativity,''
Am. J. Phys. \textbf{56} (1988), 395-412
doi:10.1119/1.15620
%1823 citations counted in INSPIRE as of 14 Mar 2023

%\cite{Damour:2007ap}
\bibitem{Damour:2007ap}
T.~Damour and S.~N.~Solodukhin,
%``Wormholes as black hole foils,''
Phys. Rev. D \textbf{76} (2007), 024016
doi:10.1103/PhysRevD.76.024016
[arXiv:0704.2667 [gr-qc]].
%176 citations counted in INSPIRE as of 14 Mar 2023

%\cite{Kim:2003zb}
\bibitem{Kim:2003zb}
W.~T.~Kim, J.~J.~Oh and M.~S.~Yoon,
%``Traversable wormholes construction in (2+1)-dimensions,''
Phys. Rev. D \textbf{70} (2004), 044006
doi:10.1103/PhysRevD.70.044006
[arXiv:gr-qc/0307034 [gr-qc]].
%29 citations counted in INSPIRE as of 14 Mar 2023


%\cite{Bueno:2017hyj}
\bibitem{Bueno:2017hyj}
P.~Bueno, P.~A.~Cano, F.~Goelen, T.~Hertog and B.~Vercnocke,
%``Echoes of Kerr-like wormholes,''
Phys. Rev. D \textbf{97} (2018) no.2, 024040
doi:10.1103/PhysRevD.97.024040
[arXiv:1711.00391 [gr-qc]].
%150 citations counted in INSPIRE as of 15 Mar 2023

%\cite{Visser:1989kh}
\bibitem{Visser:1989kh}
M.~Visser,
%``Traversable wormholes: Some simple examples,''
Phys. Rev. D \textbf{39} (1989), 3182-3184
doi:10.1103/PhysRevD.39.3182
[arXiv:0809.0907 [gr-qc]].
%403 citations counted in INSPIRE as of 15 Mar 2023

%\cite{Sushkov:2005kj}
\bibitem{Sushkov:2005kj}
S.~V.~Sushkov,
%``Wormholes supported by a phantom energy,''
Phys. Rev. D \textbf{71} (2005), 043520
doi:10.1103/PhysRevD.71.043520
[arXiv:gr-qc/0502084 [gr-qc]].
%346 citations counted in INSPIRE as of 15 Mar 2023

%\cite{Bronnikov:2002rn}
\bibitem{Bronnikov:2002rn}
K.~A.~Bronnikov and S.~W.~Kim,
%``Possible wormholes in a brane world,''
Phys. Rev. D \textbf{67} (2003), 064027
doi:10.1103/PhysRevD.67.064027
[arXiv:gr-qc/0212112 [gr-qc]].
%203 citations counted in INSPIRE as of 15 Mar 2023

%\cite{Clement:1995ya}
\bibitem{Clement:1995ya}
G.~Clement,
%``Wormhole cosmic strings,''
Phys. Rev. D \textbf{51} (1995), 6803-6809
doi:10.1103/PhysRevD.51.6803
[arXiv:gr-qc/9502033 [gr-qc]].
%36 citations counted in INSPIRE as of 15 Mar 2023

%\cite{Richarte:2017iit}
\bibitem{Richarte:2017iit}
M.~G.~Richarte, I.~G.~Salako, J.~P.~Morais Gra\c{c}a, H.~Moradpour and A.~\"Ovg\"un,
%``Relativistic Bose-Einstein condensates thin-shell wormholes,''
Phys. Rev. D \textbf{96} (2017) no.8, 084022
doi:10.1103/PhysRevD.96.084022
[arXiv:1710.05886 [gr-qc]].
%36 citations counted in INSPIRE as of 15 Mar 2023

%\cite{Ayuso:2020vuu}
\bibitem{Ayuso:2020vuu}
I.~Ayuso, F.~S.~N.~Lobo and J.~P.~Mimoso,
%``Wormhole geometries induced by action-dependent Lagrangian theories,''
Phys. Rev. D \textbf{103} (2021) no.4, 044018
doi:10.1103/PhysRevD.103.044018
[arXiv:2012.00047 [gr-qc]].
%4 citations counted in INSPIRE as of 15 Mar 2023

%\cite{KordZangeneh:2020ixt}
\bibitem{KordZangeneh:2020ixt}
M.~Kord Zangeneh and F.~S.~N.~Lobo,
%``Dynamic wormhole geometries in hybrid metric-Palatini gravity,''
Eur. Phys. J. C \textbf{81} (2021) no.4, 285
doi:10.1140/epjc/s10052-021-09059-y
[arXiv:2011.01745 [gr-qc]].
%13 citations counted in INSPIRE as of 15 Mar 2023

%\cite{Song:2023jdn}
\bibitem{Song:2023jdn}
S.~Q.~Song and E.~G\"udekli,
%``Traversable wormholes in Rastall teleparallel gravity with non-commutative geometry,''
New Astron. \textbf{100} (2023), 101993
doi:10.1016/j.newast.2022.101993
%0 citations counted in INSPIRE as of 15 Mar 2023

%\cite{Saleem:2023lul}
\bibitem{Saleem:2023lul}
R.~Saleem, M.~I.~Aslam and K.~Rasool,
%``Wormhole solutions in Rastall-like-torsion-trace gravity,''
Chin. J. Phys. \textbf{82} (2023), 1-14
doi:10.1016/j.cjph.2022.12.015
%0 citations counted in INSPIRE as of 15 Mar 2023

%\cite{Eid:2023wrd}
\bibitem{Eid:2023wrd}
A.~Eid and A.~Alkaoud,
%``Dynamics and stability of Hayward -de Sitter thin-shell wormhole,''
New Astron. \textbf{101} (2023), 102021
doi:10.1016/j.newast.2023.102021
%0 citations counted in INSPIRE as of 15 Mar 2023

%\cite{Godani:2023paj}
\bibitem{Godani:2023paj}
N.~Godani,
%``Stability of Heyward wormhole in f(R) gravity,''
New Astron. \textbf{100} (2023), 101994
doi:10.1016/j.newast.2022.101994
%1 citations counted in INSPIRE as of 15 Mar 2023

%\cite{Hochberg:1998ii}
\bibitem{Hochberg:1998ii}
D.~Hochberg and M.~Visser,
%``The Null energy condition in dynamic wormholes,''
Phys. Rev. Lett. \textbf{81} (1998), 746-749
doi:10.1103/PhysRevLett.81.746
[arXiv:gr-qc/9802048 [gr-qc]].
%202 citations counted in INSPIRE as of 15 Mar 2023

%\cite{Hochberg:1998ha}
\bibitem{Hochberg:1998ha}
D.~Hochberg and M.~Visser,
%``Dynamic wormholes, anti-trapped surfaces, and energy conditions,''
Phys. Rev. D \textbf{58} (1998), 044021
doi:10.1103/PhysRevD.58.044021
[arXiv:gr-qc/9802046 [gr-qc]].
%243 citations counted in INSPIRE as of 15 Mar 2023

%\cite{Lobo:2002zf}
\bibitem{Lobo:2002zf}
F.~Lobo and P.~Crawford,
%``Weak energy condition violation and superluminal travel,''
Lect. Notes Phys. \textbf{617} (2003), 277-291
doi:10.1007/3-540-36973-2\_15
[arXiv:gr-qc/0204038 [gr-qc]].
%31 citations counted in INSPIRE as of 15 Mar 2023

%\cite{Lobo:2004rp}
\bibitem{Lobo:2004rp}
F.~S.~N.~Lobo,
%``Energy conditions, traversable wormholes and dust shells,''
Gen. Rel. Grav. \textbf{37} (2005), 2023-2038
doi:10.1007/s10714-005-0177-x
[arXiv:gr-qc/0410087 [gr-qc]].
%66 citations counted in INSPIRE as of 15 Mar 2023

%\cite{Einstein:1936llh}
\bibitem{Einstein:1936llh}
A.~Einstein,
%``Lens-Like Action of a Star by the Deviation of Light in the Gravitational Field,''
Science \textbf{84} (1936), 506-507
doi:10.1126/science.84.2188.506
%398 citations counted in INSPIRE as of 15 Mar 2023


%\cite{Bartelmann:1999yn}
\bibitem{Bartelmann:1999yn}
M.~Bartelmann and P.~Schneider,
%``Weak gravitational lensing,''
Phys. Rept. \textbf{340} (2001), 291-472
doi:10.1016/S0370-1573(00)00082-X
[arXiv:astro-ph/9912508 [astro-ph]].
%1742 citations counted in INSPIRE as of 15 Mar 2023

%\cite{Kaiser:1991qi}
\bibitem{Kaiser:1991qi}
N.~Kaiser,
%``Weak gravitational lensing of distant galaxies,''
Astrophys. J. \textbf{388} (1992), 272
doi:10.1086/171151
%665 citations counted in INSPIRE as of 15 Mar 2023

%\cite{Bozza:2002zj}
\bibitem{Bozza:2002zj}
V.~Bozza,
%``Gravitational lensing in the strong field limit,''
Phys. Rev. D \textbf{66} (2002), 103001
doi:10.1103/PhysRevD.66.103001
[arXiv:gr-qc/0208075 [gr-qc]].
%434 citations counted in INSPIRE as of 15 Mar 2023


%\cite{Virbhadra:1999nm}
\bibitem{Virbhadra:1999nm}
K.~S.~Virbhadra and G.~F.~R.~Ellis,
%``Schwarzschild black hole lensing,''
Phys. Rev. D \textbf{62} (2000), 084003
doi:10.1103/PhysRevD.62.084003
[arXiv:astro-ph/9904193 [astro-ph]].
%701 citations counted in INSPIRE as of 15 Mar 2023

%\cite{Perlick:2004tq}
\bibitem{Perlick:2004tq}
V.~Perlick,
%``Gravitational lensing from a spacetime perspective,''
Living Rev. Rel. \textbf{7} (2004), 9
%249 citations counted in INSPIRE as of 15 Mar 2023

%\cite{SDSS:2002oin}
\bibitem{SDSS:2002oin}
C.~Stoughton \textit{et al.} [SDSS],
%``The Sloan Digital Sky Survey: Early Data Release,''
Astron. J. \textbf{123} (2002), 485-548
doi:10.1086/324741
%1628 citations counted in INSPIRE as of 15 Mar 2023

%\cite{Gao:2022cds}
\bibitem{Gao:2022cds}
K.~Gao, L.~H.~Liu and M.~Zhu,
%``Microlensing effects of wormholes associated to blackhole spacetimes,''
[arXiv:2211.17065 [gr-qc]].
%2 citations counted in INSPIRE as of 15 Mar 2023

%\cite{Liu:2022lfb}
\bibitem{Liu:2022lfb}
L.~H.~Liu, M.~Zhu, W.~Luo, Y.~F.~Cai and Y.~Wang,
%``Microlensing effect of a charged spherically symmetric wormhole,''
Phys. Rev. D \textbf{107} (2023) no.2, 024022
doi:10.1103/PhysRevD.107.024022
[arXiv:2207.05406 [gr-qc]].
%5 citations counted in INSPIRE as of 15 Mar 2023

%\cite{Sokoliuk:2022owk}
\bibitem{Sokoliuk:2022owk}
O.~Sokoliuk, S.~Praharaj, A.~Baransky and P.~K.~Sahoo,
%``Accretion flows around exotic tidal wormholes - I. Ray-tracing,''
Astron. Astrophys. \textbf{665} (2022), A139
doi:10.1051/0004-6361/202244358
[arXiv:2207.07193 [gr-qc]].
%1 citations counted in INSPIRE as of 15 Mar 2023

%\cite{Zatrimaylov:2021ijd}
\bibitem{Zatrimaylov:2021ijd}
K.~Zatrimaylov,
%``Dark Matter, Rotation Curves, and the Morphology of Galaxies,''
[arXiv:2108.13350 [astro-ph.CO]].
%0 citations counted in INSPIRE as of 15 Mar 2023

%\cite{Cheng:2021hoc}
\bibitem{Cheng:2021hoc}
X.~T.~Cheng and Y.~Xie,
%``Probing a black-bounce, traversable wormhole with weak deflection gravitational lensing,''
Phys. Rev. D \textbf{103} (2021) no.6, 064040
doi:10.1103/PhysRevD.103.064040
%31 citations counted in INSPIRE as of 15 Mar 2023

%\cite{Li:2019qyb}
\bibitem{Li:2019qyb}
Z.~Li and J.~Jia,
%``The finite-distance gravitational deflection of massive particles in stationary spacetime: a Jacobi metric approach,''
Eur. Phys. J. C \textbf{80} (2020) no.2, 157
doi:10.1140/epjc/s10052-020-7665-8
[arXiv:1912.05194 [gr-qc]].
%36 citations counted in INSPIRE as of 15 Mar 2023

%\cite{Kuniyasu:2018cgv}
\bibitem{Kuniyasu:2018cgv}
M.~Kuniyasu, K.~Nanri, N.~Sakai, T.~Ohgami, R.~Fukushige and S.~Komura,
%``Can we identify massless braneworld black holes by observations?,''
Phys. Rev. D \textbf{97} (2018) no.10, 104063
doi:10.1103/PhysRevD.97.104063
[arXiv:1806.00231 [gr-qc]].
%5 citations counted in INSPIRE as of 15 Mar 2023

%\cite{Raidal:2018eoo}
\bibitem{Raidal:2018eoo}
M.~Raidal, S.~Solodukhin, V.~Vaskonen and H.~Veerm\"ae,
%``Light Primordial Exotic Compact Objects as All Dark Matter,''
Phys. Rev. D \textbf{97} (2018) no.12, 123520
doi:10.1103/PhysRevD.97.123520
[arXiv:1802.07728 [astro-ph.CO]].
%27 citations counted in INSPIRE as of 15 Mar 2023

%\cite{Tsukamoto:2017hva}
\bibitem{Tsukamoto:2017hva}
N.~Tsukamoto and Y.~Gong,
%``Extended source effect on microlensing light curves by an Ellis wormhole,''
Phys. Rev. D \textbf{97} (2018) no.8, 084051
doi:10.1103/PhysRevD.97.084051
[arXiv:1711.04560 [gr-qc]].
%27 citations counted in INSPIRE as of 15 Mar 2023

%\cite{Sajadi:2016hko}
\bibitem{Sajadi:2016hko}
S.~N.~Sajadi and N.~Riazi,
%``Gravitational lensing by multi-polytropic wormholes,''
Can. J. Phys. \textbf{98} (2020) no.11, 1046-1054
doi:10.1139/cjp-2019-0524
[arXiv:1611.04343 [gr-qc]].
%13 citations counted in INSPIRE as of 15 Mar 2023

%\cite{Lukmanova:2016czn}
\bibitem{Lukmanova:2016czn}
R.~Lukmanova, A.~Kulbakova, R.~Izmailov and A.~A.~Potapov,
%``Gravitational Microlensing by Ellis Wormhole: Second Order Effects,''
Int. J. Theor. Phys. \textbf{55} (2016) no.11, 4723-4730
doi:10.1007/s10773-016-3095-7
%21 citations counted in INSPIRE as of 15 Mar 2023

%\cite{Tsukamoto:2016zdu}
\bibitem{Tsukamoto:2016zdu}
N.~Tsukamoto and T.~Harada,
%``Light curves of light rays passing through a wormhole,''
Phys. Rev. D \textbf{95} (2017) no.2, 024030
doi:10.1103/PhysRevD.95.024030
[arXiv:1607.01120 [gr-qc]].
%64 citations counted in INSPIRE as of 15 Mar 2023

%\cite{Kitamura:2016vad}
\bibitem{Kitamura:2016vad}
T.~Kitamura, Gravitational lensing in an exotic spacetime
%``Gravitational lensing in an exotic spacetime,''
%0 citations counted in INSPIRE as of 15 Mar 2023

%\cite{Kitamura:2012zy}
\bibitem{Kitamura:2012zy}
T.~Kitamura, K.~Nakajima and H.~Asada,
%``Demagnifying gravitational lenses toward hunting a clue of exotic matter and energy,''
Phys. Rev. D \textbf{87} (2013) no.2, 027501
doi:10.1103/PhysRevD.87.027501
[arXiv:1211.0379 [gr-qc]].
%50 citations counted in INSPIRE as of 15 Mar 2023

%\cite{Kitamura:2012wcg}
\bibitem{Kitamura:2012wcg}
T.~Kitamura, Astrometric microlensing by the Ellis Wormhole
%``Astrometric microlensing by the Ellis Wormhole,''
%0 citations counted in INSPIRE as of 15 Mar 2023

%\cite{Toki:2011zu}
\bibitem{Toki:2011zu}
Y.~Toki, T.~Kitamura, H.~Asada and F.~Abe,
%``Astrometric Image Centroid Displacements due to Gravitational Microlensing by the Ellis Wormhole,''
Astrophys. J. \textbf{740} (2011), 121
doi:10.1088/0004-637X/740/2/121
[arXiv:1107.5374 [astro-ph.CO]].
%85 citations counted in INSPIRE as of 15 Mar 2023

%\cite{Abe:2010ap}
\bibitem{Abe:2010ap}
F.~Abe,
%``Gravitational Microlensing by the Ellis Wormhole,''
Astrophys. J. \textbf{725} (2010), 787-793
doi:10.1088/0004-637X/725/1/787
[arXiv:1009.6084 [astro-ph.CO]].
%135 citations counted in INSPIRE as of 15 Mar 2023

%\cite{Bogdanov:2008zy}
\bibitem{Bogdanov:2008zy}
M.~B.~Bogdanov and A.~M.~Cherepashchuk,
%``Search for exotic matter from gravitational microlensing observations of stars,''
Astrophys. Space Sci. \textbf{317} (2008), 181-192
doi:10.1007/s10509-008-9870-z
[arXiv:0807.2774 [astro-ph]].
%14 citations counted in INSPIRE as of 15 Mar 2023

%\cite{Torres:2001gb}
\bibitem{Torres:2001gb}
D.~F.~Torres, E.~F.~Eiroa and G.~E.~Romero,
%``On the possibility of an astronomical detection of chromaticity effects in microlensing by wormhole - like objects,''
Mod. Phys. Lett. A \textbf{16} (2001), 1849-1861
doi:10.1142/S0217732301005126
[arXiv:gr-qc/0109041 [gr-qc]].
%7 citations counted in INSPIRE as of 15 Mar 2023

%\cite{Safonova:2001vz}
\bibitem{Safonova:2001vz}
M.~Safonova, D.~F.~Torres and G.~E.~Romero,
%``Microlensing by natural wormholes: Theory and simulations,''
Phys. Rev. D \textbf{65} (2002), 023001
doi:10.1103/PhysRevD.65.023001
[arXiv:gr-qc/0105070 [gr-qc]].
%103 citations counted in INSPIRE as of 15 Mar 2023

%\cite{Torres:1998cu}
\bibitem{Torres:1998cu}
D.~F.~Torres, G.~E.~Romero and L.~A.~Anchordoqui,
%``Wormholes, gamma-ray bursts and the amount of negative mass in the universe,''
Mod. Phys. Lett. A \textbf{13} (1998), 1575-1582
doi:10.1142/S0217732398001650
[arXiv:gr-qc/9805075 [gr-qc]].
%42 citations counted in INSPIRE as of 15 Mar 2023

%\cite{Torres:1998xd}
\bibitem{Torres:1998xd}
D.~F.~Torres, G.~E.~Romero and L.~A.~Anchordoqui,
%``Might some gamma-ray bursts be an observable signature of natural wormholes?,''
Phys. Rev. D \textbf{58} (1998), 123001
doi:10.1103/PhysRevD.58.123001
[arXiv:astro-ph/9802106 [astro-ph]].
%61 citations counted in INSPIRE as of 15 Mar 2023


%\cite{Gibbons:2008rj}
\bibitem{Gibbons:2008rj}
G.~W.~Gibbons and M.~C.~Werner,
%``Applications of the Gauss-Bonnet theorem to gravitational lensing,''
Class. Quant. Grav. \textbf{25} (2008), 235009
doi:10.1088/0264-9381/25/23/235009
[arXiv:0807.0854 [gr-qc]].
%195 citations counted in INSPIRE as of 15 Mar 2023

%\cite{Gibbons:2008zi}
\bibitem{Gibbons:2008zi}
G.~W.~Gibbons, C.~A.~R.~Herdeiro, C.~M.~Warnick and M.~C.~Werner,
%``Stationary Metrics and Optical Zermelo-Randers-Finsler Geometry,''
Phys. Rev. D \textbf{79} (2009), 044022
doi:10.1103/PhysRevD.79.044022
[arXiv:0811.2877 [gr-qc]].
%104 citations counted in INSPIRE as of 15 Mar 2023

%\cite{Werner:2012rc}
\bibitem{Werner:2012rc}
M.~C.~Werner,
%``Gravitational lensing in the Kerr-Randers optical geometry,''
Gen. Rel. Grav. \textbf{44} (2012), 3047-3057
doi:10.1007/s10714-012-1458-9
[arXiv:1205.3876 [gr-qc]].
%150 citations counted in INSPIRE as of 15 Mar 2023

%\cite{He:2023hsv}
\bibitem{He:2023hsv}
X.~He, T.~Xu, Y.~Yu, A.~Karamat, R.~Babar and R.~Ali,
%``Deflection angle evolution with plasma medium and without plasma medium in a parameterized black hole,''
Annals Phys. \textbf{451} (2023), 169247
doi:10.1016/j.aop.2023.169247
%0 citations counted in INSPIRE as of 15 Mar 2023

%\cite{Upadhyay:2023yhk}
\bibitem{Upadhyay:2023yhk}
S.~Upadhyay, S.~Mandal, Y.~Myrzakulov and K.~Myrzakulov,
%``Weak deflection angle, greybody bound and shadow for charged massive BTZ black hole,''
Annals Phys. \textbf{450} (2023), 169242
doi:10.1016/j.aop.2023.169242
[arXiv:2303.02132 [gr-qc]].
%0 citations counted in INSPIRE as of 15 Mar 2023

%\cite{Javed:2023iih}
\bibitem{Javed:2023iih}
W.~Javed, M.~Atique, R.~C.~Pantig and A.~\"Ovg\"un,
%``Weak Deflection Angle, Hawking Radiation and Greybody Bound of Reissner\textendash{}Nordstr\"om Black Hole Corrected by Bounce Parameter,''
Symmetry \textbf{15} (2023) no.1, 148
doi:10.3390/sym15010148
[arXiv:2301.01855 [gr-qc]].
%1 citations counted in INSPIRE as of 15 Mar 2023

%\cite{Javed:2022gtz}
\bibitem{Javed:2022gtz}
W.~Javed, H.~Irshad, R.~C.~Pantig and A.~\"Ovg\"un,
%``Weak Deflection Angle by Kalb\textendash{}Ramond Traversable Wormhole in Plasma and Dark Matter Mediums,''
Universe \textbf{8} (2022) no.11, 599
doi:10.3390/universe8110599
[arXiv:2211.07009 [gr-qc]].
%4 citations counted in INSPIRE as of 15 Mar 2023

%\cite{Huang:2022soh}
\bibitem{Huang:2022soh}
Y.~Huang and Z.~Cao,
%``Generalized Gibbons-Werner method for deflection angle,''
Phys. Rev. D \textbf{106} (2022) no.10, 104043
doi:10.1103/PhysRevD.106.104043
%1 citations counted in INSPIRE as of 15 Mar 2023

%\cite{He:2022yhp}
\bibitem{He:2022yhp}
J.~He, Q.~Wang, Q.~Hu, L.~Feng and J.~Jia,
%``Deflection in higher dimensional spacetime and asymptotically non-flat spacetimes,''
Class. Quant. Grav. \textbf{40} (2023) no.6, 065006
doi:10.1088/1361-6382/acbade
[arXiv:2210.00938 [gr-qc]].
%0 citations counted in INSPIRE as of 15 Mar 2023

%\cite{Gao:2023ltr}
\bibitem{Gao:2023ltr}
X.~J.~Gao, X.~k.~Yan, Y.~Yin and Y.~P.~Hu,
%``Gravitational lensing by a charged spherically symmetric black hole immersed in thin dark matter,''
[arXiv:2303.00190 [gr-qc]].
%0 citations counted in INSPIRE as of 15 Mar 2023

%\cite{Cai:2023ite}
\bibitem{Cai:2023ite}
T.~Cai, Z.~Wang, H.~Huang and M.~Zhu,
%``Higher order correction to weak-field lensing of Ellis-Bronnikov wormhole,''
[arXiv:2302.13704 [gr-qc]].
%0 citations counted in INSPIRE as of 15 Mar 2023

%\cite{Ovgun:2023ego}
\bibitem{Ovgun:2023ego}
A.~\"Ovg\"un, R.~C.~Pantig and \'A.~Rinc\'on,
%``4D scale-dependent Schwarzschild-AdS/dS black holes: study of shadow and weak deflection angle and greybody bounding,''
Eur. Phys. J. Plus \textbf{138} (2023) no.3, 192
doi:10.1140/epjp/s13360-023-03793-w
[arXiv:2303.01696 [gr-qc]].
%1 citations counted in INSPIRE as of 15 Mar 2023

%\cite{Lobo:2005us}
\bibitem{Lobo:2005us}
F.~S.~N.~Lobo,
%``Phantom energy traversable wormholes,''
Phys. Rev. D \textbf{71} (2005), 084011
doi:10.1103/PhysRevD.71.084011
[arXiv:gr-qc/0502099 [gr-qc]].
%445 citations counted in INSPIRE as of 15 Mar 2023

%\cite{Lobo:2005yv}
\bibitem{Lobo:2005yv}
F.~S.~N.~Lobo,
%``Stability of phantom wormholes,''
Phys. Rev. D \textbf{71} (2005), 124022
doi:10.1103/PhysRevD.71.124022
[arXiv:gr-qc/0506001 [gr-qc]].
%211 citations counted in INSPIRE as of 15 Mar 2023

%\cite{Garattini:2007ff}
\bibitem{Garattini:2007ff}
R.~Garattini and F.~S.~N.~Lobo,
%``Self sustained phantom wormholes in semi-classical gravity,''
Class. Quant. Grav. \textbf{24} (2007), 2401-2413
doi:10.1088/0264-9381/24/9/016
[arXiv:gr-qc/0701020 [gr-qc]].
%80 citations counted in INSPIRE as of 15 Mar 2023

%\cite{Nakajima:2012pu,Jusufi:2017gyu}
\bibitem{Nakajima:2012pu}
K.~Nakajima and H.~Asada,
%``Deflection angle of light in an Ellis wormhole geometry,''
Phys. Rev. D \textbf{85} (2012), 107501
doi:10.1103/PhysRevD.85.107501
[arXiv:1204.3710 [gr-qc]].
%123 citations counted in INSPIRE as of 04 Oct 2023

%\cite{Jusufi:2017gyu}
\bibitem{Jusufi:2017gyu}
K.~Jusufi,
%``Deflection angle of light by wormholes using the Gauss\textendash{}Bonnet theorem,''
Int. J. Geom. Meth. Mod. Phys. \textbf{14} (2017) no.12, 1750179
doi:10.1142/S0219887817501791
[arXiv:1706.01244 [gr-qc]].
%41 citations counted in INSPIRE as of 04 Oct 2023



%\cite{Kim:2001ri}
\bibitem{Kim:2001ri}
S.~W.~Kim and H.~Lee,
%``Exact solutions of a charged wormhole,''
Phys. Rev. D \textbf{63}, 064014 (2001)
doi:10.1103/PhysRevD.63.064014
[arXiv:gr-qc/0102077 [gr-qc]].
%119 citations counted in INSPIRE as of 19 Mar 2023

%\cite{Jusufi:2018kmk}
\bibitem{Jusufi:2018kmk}
K.~Jusufi, A.~\"Ovg\"un, A.~Banerjee and \textperiodcentered{}.~I.~Sakall\i{},
%``Gravitational lensing by wormholes supported by electromagnetic, scalar, and quantum effects,''
Eur. Phys. J. Plus \textbf{134} (2019) no.9, 428
doi:10.1140/epjp/i2019-12792-9
[arXiv:1802.07680 [gr-qc]].
%55 citations counted in INSPIRE as of 30 Mar 2023


%\cite{Johnson:2019ljv}
\bibitem{Johnson:2019ljv}
M.~D.~Johnson, A.~Lupsasca, A.~Strominger, G.~N.~Wong, S.~Hadar, D.~Kapec, R.~Narayan, A.~Chael, C.~F.~Gammie and P.~Galison, \textit{et al.}
%``Universal interferometric signatures of a black hole\textquoteright{}s photon ring,''
Sci. Adv. \textbf{6} (2020) no.12, eaaz1310
doi:10.1126/sciadv.aaz1310
[arXiv:1907.04329 [astro-ph.IM]].
%143 citations counted in INSPIRE as of 03 Jun 2023

%\cite{Gralla:2019drh}
\bibitem{Gralla:2019drh}
S.~E.~Gralla and A.~Lupsasca,
%``Lensing by Kerr Black Holes,''
Phys. Rev. D \textbf{101} (2020) no.4, 044031
doi:10.1103/PhysRevD.101.044031
[arXiv:1910.12873 [gr-qc]].
%99 citations counted in INSPIRE as of 03 Jun 2023


\bibitem{Alcubierre:2017pqm} 
M.~Alcubierre and F.~S.~N.~Lobo, 
``Wormholes, Warp Drives and Energy Conditions,'' Fundam. Theor. Phys. \textbf{189} (2017), pp.-279 Springer, 2017, doi:10.1007/978-3-319-55182-1 [arXiv:2103.05610 [gr-qc]]. %64 citations counted in INSPIRE as of 29 Aug 2023


%\cite{Zaris:2019soz}
\bibitem{Zaris:2019soz}
J.~Zaris, D.~Veske, J.~Samsing, Z.~M\'arka, I.~Bartos and S.~M\'arka,
%``Constraining Black Hole Populations in Globular Clusters using Microlensing: Application to Omega Centauri,''
Astrophys. J. Lett. \textbf{894} (2020) no.1, L9
doi:10.3847/2041-8213/ab89a3
[arXiv:1912.05701 [astro-ph.HE]].
%2 citations counted in INSPIRE as of 03 Jun 2023

%\cite{Sollima:2009wh}
\bibitem{Sollima:2009wh}
A.~Sollima, M.~Bellazzini, R.~L.~Smart, M.~Correnti, E.~Pancino, F.~R.~Ferraro and D.~Romano,
%``The non-peculiar velocity dispersion profile of the stellar system omega Centauri,''
Mon. Not. Roy. Astron. Soc. \textbf{396} (2009), 2183
doi:10.1111/j.1365-2966.2009.14864.x
[arXiv:0904.0571 [astro-ph.SR]].
%24 citations counted in INSPIRE as of 03 Jun 2023

%\cite{Noyola:2010ab}
\bibitem{Noyola:2010ab}
E.~Noyola, K.~Gebhardt, M.~Kissler-Patig, N.~Lutzgendorf, B.~Jalali, P.~T.~de Zeeuw and H.~Baumgardt,
%``VLT Kinematics for omega Centauri: Further Support for a Central Black Hole,''
Astrophys. J. Lett. \textbf{719} (2010), L60
doi:10.1088/2041-8205/719/1/L60
[arXiv:1007.4559 [astro-ph.GA]].
%54 citations counted in INSPIRE as of 03 Jun 2023

%\cite{Kiroglu:2021mej}
\bibitem{Kiroglu:2021mej}
F.~K\i{}ro\u{g}lu, N.~C.~Weatherford, K.~Kremer, C.~S.~Ye, G.~Fragione and F.~A.~Rasio,
%``Gravitational Microlensing Rates in Milky Way Globular Clusters,''
Astrophys. J. \textbf{928} (2022) no.2, 181
doi:10.3847/1538-4357/ac5895
[arXiv:2111.14866 [astro-ph.GA]].
%3 citations counted in INSPIRE as of 01 Apr 2023

%\cite{Cai:2022kbp}
\bibitem{Cai:2022kbp}
R.~G.~Cai, T.~Chen, S.~J.~Wang and X.~Y.~Yang,
%``Gravitational microlensing by dressed primordial black holes,''
JCAP \textbf{03} (2023), 043
doi:10.1088/1475-7516/2023/03/043
[arXiv:2210.02078 [astro-ph.CO]].
%3 citations counted in INSPIRE as of 01 Apr 2023













\end{thebibliography}









\end{document}
