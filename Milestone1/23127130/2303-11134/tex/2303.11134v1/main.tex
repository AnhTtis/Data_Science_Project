\documentclass[prd,nofootinbib,12pt]{revtex4}
%\documentclass[review,12pt]{elsarticle}

\usepackage{lineno,hyperref}
\hypersetup{hidelinks}
\usepackage{amsfonts,amssymb,amsmath}
\usepackage{epsfig}
\usepackage{mathrsfs}
\usepackage{amssymb}
\usepackage{graphicx}
\graphicspath{ {./images/} }
\usepackage{amsmath}
\usepackage{xcolor}
\usepackage{color}
%\usepackage[toc,page,title,titletoc,header]{appendix}


\newcommand{\TT}[1]{{\color{red}{#1}}}	
\newcommand{\TTT}[1]{{\color{blue}{#1}}}		


\begin{document}



\title{ Microlensing and multi-images problem of static spherical symmetric wormhole}
\author{Ke Gao$^{1}$}
\author{Lei-Hua Liu$^{1}$}
\email{liuleihua8899@hotmail.com}


\affiliation{$^1$Department of Physics, College of Physics, Mechanical and Electrical Engineering, Jishou University, Jishou 416000, China}




\begin{abstract}
In this paper, we develop a framework to re-examine the weak lensing (including the microlensing) effects of the static spherical symmetric wormhole in terms of the radial equation of state $\eta=\frac{p_r}{\rho}$ (REoS). As for its application, we calculate its magnification, and event rate under this REoS, in which we show that the maximal value of magnification of the Ellis-Bronnikov wormhole is only related to the relative position and intrinsic angle, whose the maximal value is around five. For the event rate, our results indicate that one cannot distinguish the Eillis-Bronnikov wormhole and charged wormhole, but its order is much higher than the vacuum case, in which all these metrics belong to the static spherical symmetric wormhole metric. By calculating the lensing equation of the static spherical symmetric wormhole,	we find an explicit formula between the maximal number of images of the wormhole and $\eta$. It shows that this relation is consistent with the classical wormhole, but the case for wormhole with quantum corrections is still mysterious.  Our new method may shed new light on distinguishing the wormhole and blackhole via the event rate.

\end{abstract}

\maketitle

\section{Introduction}
\label{introduction}


Wormhole has been proposed more than one hundred years. In 1916, Flamm studied the internal structure of Schwarzschild’s solution \cite{Flamm:1916}. Einstein and Rosen proposed the concept of a bridge structure, which paved the way for further understanding of wormholes \cite{Einstein:1935}. The wormhole was first introduced by Misner and Wheeler \cite{Misner:1957}. Ellis proposed the idea of a drain hole structure \cite{Ellis:1973}. Then, Morris proposed the concept of a traversable wormhole \cite{Morris:1988cz}, in which a type of wormhole that could be traversed by humans or spaceships without being destroyed. These advancements in the understanding of wormholes have contributed greatly to the field of theoretical physics. Thereafter, wormholes became the subject of extensive research \cite{Damour:2007ap,Kim:2003zb,Bueno:2017hyj,Visser:1989kh,Sushkov:2005kj,Bronnikov:2002rn,Clement:1995ya,Richarte:2017iit,Ayuso:2020vuu,KordZangeneh:2020ixt,Song:2023jdn,Saleem:2023lul,Eid:2023wrd,Godani:2023paj} by physicists. The study of wormholes typically begins with an assumption about their geometric structure and then involves calculating the corresponding material source required to create such a structure. However, such calculations often violate known energy conditions \cite{Hochberg:1998ii,Hochberg:1998ha,Lobo:2002zf,Lobo:2004rp}, indicating that some exotic matter is required to explain the existence of wormholes. The study of wormholes can deepen our understanding of general relativity and explore space.

As for the lensing effects, Einstein published the first article on gravitational lens field \cite{Einstein:1936llh}. After a period of silence, gravitational lens has become a hotspot. Gravitational lensing can be classified into two types: weak gravitational lensing including the microlensing \cite{Bartelmann:1999yn,Kaiser:1991qi} and strong gravitational lensing \cite{Bozza:2002zj,Virbhadra:1999nm}. Weak gravitational lensing is caused by relatively small perturbations in the gravitational field and results in slight distortions of images, while strong gravitational lensing involves more significant deformations due to the presence of massive objects like black holes, galaxies or galaxy clusters. Gravitational microlens is an effective method to explore wormholes \cite{Perlick:2004tq,SDSS:2002oin}. In the literature, The microlensing effect of a wormhole is extensively studied \cite{Gao:2022cds,Liu:2022lfb,Sokoliuk:2022owk,Zatrimaylov:2021ijd,Cheng:2021hoc,Li:2019qyb,Kuniyasu:2018cgv,Raidal:2018eoo,Tsukamoto:2017hva,Sajadi:2016hko,Lukmanova:2016czn,Tsukamoto:2016zdu,Kitamura:2016vad,Kitamura:2012zy,Kitamura:2012wcg,Toki:2011zu,Abe:2010ap,Bogdanov:2008zy,Torres:2001gb,Safonova:2001vz,Torres:1998cu,Torres:1998xd}. An interesting question arises as to how many images a light source will generate at the equatorial plane when passing through the wormhole space-time, and what this quantity is related to. There has been some discussions about microlensing imaging in wormholes spacetime \cite{Kitamura:2016vad,Kuniyasu:2018cgv,Abe:2010ap,Liu:2022lfb}. 

In this paper, we will re-examine this problem from a new perspective by implementing the REoS to depict the lensing effects of the spherically symmetric wormhole metric. Then we calculate the deflection angle of this metric by Gauss-Bonnet Theorem (GBT) \cite{Gibbons:2008rj,Gibbons:2008zi,Werner:2012rc} which is widely used in gravitational lenses to calculate deflection angle \cite{He:2023hsv,Upadhyay:2023yhk,Javed:2023iih,Javed:2022gtz,Huang:2022soh,He:2022yhp,Gao:2023ltr,Cai:2023ite,Ovgun:2023ego}, and we find that the state coefficient $\eta$ determines the maximum value of the image of a light source in the equatorial plane. In the last section, we discuss the applicability of the image number formula $n=2+\frac{1}{\eta}$. We also calculate the magnification, and get a general formula for magnification. We analyze the magnification of Ellis-Bronnikov wormhole as an instance, and detect that it has a maximum value which only depends on the relative position $\frac{D_{LS}}{D_{S}}$ and intrinsic angle $\beta$ in the first-order lens equation case. In addition, we study the event rate of a large scale wormhole. We regard it as an object with continuous mass distribution, and obtain the relationship between the event rate and the radial state coefficient.  Our work may bring new ideas to the study of wormholes. 

The structure of our article is organized as follows: In section \ref{wormhole}, we use REoS to construct a metric. Section \ref{microlensing}, we use the gravitational lens technology to calculate the deflection angle and lens equation and obtain a formula for the number of images. Besides, magnification and event rate are also discussed. In Section \ref{number of images}, we discuss the applicability of the formula in specific examples. In section \ref{conclusion and outlook}, we draw our conclusion and outlook. In appendix \ref{appendix}, we unify all of the parameters in SI unit. 



\section{Wormhole}
\label{wormhole}
In this section, we follow the notation of \cite{Lobo:2005us, Lobo:2005yv, Garattini:2007ff}. By starting with a static spherical symmetry metric:
\begin{equation}
\label{eq1}
ds^2=-e^{2\Phi}dt^2+\frac{dr^2}{1-b(r)/r}+r^2d\Omega^2,
\end{equation}
this metric describes a generic  static and spherical wormhole metric. The Einstein field equation provides the following relationships:
\begin{equation}
\label{eq2}
p_r^\prime=\frac{2}{r}\big(p_t-p_r\big)-\big(\rho+p_r\big)\Phi^\prime,
\end{equation}
\begin{equation}
\label{eq3}
b^\prime=8\pi G\rho(r)r^2,
\end{equation}
\begin{equation}
\label{eq4}
\Phi^\prime=\frac{b+8\pi Gp_rr^3}{2r^2\big(1-b(r)/r\big)},
\end{equation}
where the prime denotes a derivative with respect to the radial coordinate $r$, $p_r$ represents the radial pressure, $p_t$ indicates the tangential pressure, and $\rho$ is the energy density. REoS is defined as follows,
\begin{equation}
\label{eq5}
p_r=\eta \rho.
\end{equation}
where $\eta$ represents radial state coefficeient. The flaring-out condition and asymptotic flatness take the neccessary condition,
\begin{equation}
\eta>0 \,\, \text{and} \,\, \eta<-1.
\label{flaring out condition}
\end{equation}
Combining these equations \eqref{eq2}-\eqref{eq5}, we can get
\begin{equation}
\begin{aligned}
b(r) = r_0\bigg(\frac{r_0}{r}\bigg)^\frac{1}{\eta}e^{-(2/\eta)[\Phi(r)-\Phi(r_0)]}\times
\\
\bigg[\frac{2}{\eta} \int_{r_0}^r\big(\frac{r}{r_0}\big)^{(1+\eta)/\eta}\Phi^\prime(r)
e^{(2/\eta)[\Phi(r)-\Phi(r_0)]}dr+1\bigg].
\end{aligned}
\end{equation}
We choose a domain in which $\Phi(r)\approx constant$, then,
\begin{equation}
b(r)=r_0\bigg(\frac{r_0}{r}\bigg)^{\frac{1}{\eta}}.
\end{equation}
Substitute the above formula into Eq. \eqref{eq1}, then one can get 
\begin{equation}
\label{eq7}
ds^2=-Adt^2+\frac{dr^2}{1-\big(r_0/r\big)^{1+\frac{1}{\eta}}}+r^2d\Omega^2,
\end{equation}
where $A=-e^{2\Phi}$. In the next section, we will use this metric to discuss microlensing effect.





\begin{figure}
    \centering
    \includegraphics[scale=0.9]{wormhole_image_1.png}
    \caption{A sketch of a wormhole. In our case, we consider the lensing effects occur on one side of wormhole (spacetime 1 or spacetime 2). }
    \label{fig:my_label}
\end{figure}





\section{Microlensing}
\label{microlensing}
In this section, we will calculate the magnification and the event rate of metric \eqref{eq7}. Under REoS,  we also find an explicit relation between the maximal number of images of wormhole and $\eta$. 






\subsection{Deflection angle}
In this subsection, we will implement GBT to calculate the deflection angle. 
For a photon $ds^2=0$, it (in equatorial plane) will satisfy the following relation as 
\begin{equation}
\label{eq 8}
dt^2=\frac{dr^2}{A\left(1-\big(r_0/r\big)^{1+\frac{1}{\eta}}\right)}+\frac{r^2}{A}d\phi^2.
\end{equation}
Then we define two auxiliary quantities as $du=\frac{dr}{\sqrt{A\big(1-\big(r_0/r\big)^{1+\frac{1}{\eta}}\big)}}$ and $\xi=\frac{r}{\sqrt{A}}$. Gaussian optical curvature can be expressed as
\begin{equation}
K=\frac{-1}{\xi(u)}[\frac{dr}{du}\frac{d}{dr}\big(\frac{dr}{du}\big)\frac{d\xi}{dr}+\big(\frac{dr}{du}\big)^2\frac{d^2\xi}{dr^2}],
\end{equation}
combine with metric \eqref{eq 8}, one can get
\begin{equation}
\label{eq10}
K=\frac{-\sqrt{A}r_0\big(\frac{r_0}{r}\big)^\frac{1}{\eta}\big(1+\frac{1}{\eta}\big)}{2r^3\sqrt{1-\big(\frac{r_0}{r}\big)^{1+\frac{1}{\eta}}}}.
\end{equation}
Now let's derive the expression of deflection angle.
We first give the Gauss-Bonnet theorem
\begin{equation}
\int\int_D KdS+\int_{\partial D}\kappa dt+\sum\limits_i\alpha_i=2\pi\chi(D).
\end{equation}
We choose the integral domain $D$ in Fig. \ref{fig:1},  $OS$ is the geodesic line, so the line integral on $OS$ is zero. Besides, this
Euler index $\chi$ in domain $D$ is 1.
\begin{equation}
\int\int_{D_2}KdS+\int_{\gamma_P}\kappa dt+\sum_i\alpha_i=2\pi
\end{equation}
We can choose $\gamma$ to vertically intersect the geodesic line $OS$ at point O and point S, which means
\begin{equation}
\sum_i\alpha_i=\frac{\pi}{2}(S)+\frac{\pi}{2}(O)=\pi.
\end{equation}
The sum of external angles here is the sum of two right angles, And then we do an integral transformation
\begin{equation}
\kappa dt=\kappa\frac{dt}{d\phi}d\phi.
\end{equation}
Here $\phi$ is the angular coordinate of the center at $wormhole$. It can be done to set up $\kappa\frac{dt}{d\phi}=1$ on $\gamma$, so there is
\begin{equation}
\int\int_{D_2}KdS+\int_{\phi_O}^{\phi_S}d\phi+\pi=2\pi.
\end{equation}
Geodesic $OS$ is approximately a straight line, that is, the angle that $OS$ spans is $\pi+\alpha$, and we let the angular coordinate of point O be 0. The result is
\begin{equation}
\int\int_{D_2}KdS+\int_{0}^{\pi+\alpha}d\phi+\pi=\int\int_{D_2}KdS+\pi+\alpha+\pi=2\pi.
\end{equation}
The final result is
\begin{equation}
\alpha=-\int\int_{D_2}KdS.
\end{equation}
That is to say, our deflection angle can be written as
\begin{equation}
\label{eq11}
\alpha=-\int_0^\pi\int_{\frac{b}{\sin\phi}}^\infty K\sqrt{\det  h_{ab}} drd\phi,
\end{equation}
where $b$ is impact parameter and $h_{ab}$ is metric \eqref{eq 8}. Note that our results only apply to small angles. Substitute formula \eqref{eq10} to \eqref{eq11}, one can obtain
\begin{equation}
\label{eq 22}
\alpha=\int_0^\pi\int_{\frac{b}{\sin\phi}}^\infty\frac{r_0\big(\frac{r_0}{r}\big)^\frac{1}{\eta}\big(1+\frac{1}{\eta}\big)}{2\sqrt{A}r^2\big(1-\big(\frac{r_0}{r}\big)^{1+\frac{1}{\eta}}\big)}drd\phi.
\end{equation}
Under the weak field approximation, we work out
\begin{equation}
\label{eq13}
\alpha=\frac{\sqrt{\pi}\big(\frac{r_0}{b}\big)^{1+\frac{1}{\eta}}\eta\Gamma[1+\frac{1}{2\eta}]}{2\sqrt{A}\Gamma[\frac{1}{2}\big(3+\frac{1}{\eta}\big)]}, ~~~\text{if } \frac{1}{\eta}>-2.
\end{equation}
This deflection angle is valid for the first order of $r_0/b$. Being armed with deflection angle \eqref{eq13}, one can investigate its corresponding lens equation. 





\subsection{Lensing equation}
Plane geometry in equatorial plane of lens Fig. \ref{fig:2} tells us
\begin{equation}
\label{eq14}
\beta=\theta-\frac{D_{LS}}{D_S}\alpha.
\end{equation}
Substitute Eq. \eqref{eq13} to Eq. \eqref{eq14}, one can obtain that
\begin{equation}
\theta^{2+\frac{1}{\eta}}-\beta\theta^{1+\frac{1}{\eta}}-\frac{D_{LS}}{D_S}\frac{\sqrt{\pi}\big(\frac{r_0}{D_L}\big)^{1+\frac{1}{\eta}}\eta\Gamma[1+\frac{1}{2\eta}]}{2\sqrt{A}\Gamma[\frac{1}{3}\big(3+\frac{1}{\eta}\big)]}=0,
\label{general lens eq}
\end{equation}
here we have used the approximation $b\approx\theta D_L$. According to eq. \eqref{general lens eq}, one can explicitly obtain the relation between the order of lensing equation (n) and $\eta$, 
\begin{equation}
\label{eq 16}
n=2+\frac{1}{\eta}.
\end{equation}
This is an equation about the number of images (If there are at most n solutions on the inference of the n order of equation). When $\eta\rightarrow 0_+$, then $n\rightarrow\infty$, this means that we can at most get an infinite number of images. On the other hand, when $\eta\rightarrow\pm\infty$, had $n\rightarrow 2$, we are expected to have two images. 
 
 From the perspective of observation, only the real solution is applicable. However, the second order of the lensing equation could have complex solutions. The situation of the higher order lensing equation is more complicated. To be more precise, one cannot find the exact real solution of the higher-order lensing equation. Thus, the image number equation will be guiding us to explore the image problems in various wormholes.



\begin{figure}
    \centering
    \includegraphics[scale=1.1]{gauss_bonnet_1.png}
    \caption{The illustration of the Gauss-Bonnet theorem integral domain.}
    \label{fig:1}
\end{figure}


\begin{figure}
    \centering
    \includegraphics[scale=1.1]{lens_geometry.png}
    \caption{Showing lens plane geometry. $I$ is the location of images, $S$ is location of source, $\alpha$ is the deflection angle, $W$ is the  wormhole and $b$ is the impact parameter. $\alpha$ is the deflection angle. All of these angles are much less than unity.}
    \label{fig:2}
\end{figure}








\subsection{Magnification}
Magnification is a result of the distortion caused by lensing. By applying the lens equation, the solid angle element $d\beta^2$ is transformed into the solid angle $d\theta^2$, ultimately affecting the observed solid angle under which the source is viewed. This shift in solid angle results in a magnification or demagnification of the received flux. The total magnification can be calculated as follows:

\begin{equation}
\label{eq 27}
\mu_{\rm total}=\sum_i \big|\frac{\beta}{\theta_i}\frac{d\beta}{d\theta_i}\big|^{-1}.
\end{equation}
substituting eq \eqref{eq14} into eq \eqref{eq 27}, which leads to
\begin{equation}
\mu_{\rm total}=\sum_i\big|\big(2+\frac{1}{\eta}\big)\frac{\beta}{\theta_i}-\big(1+\frac{1}{\eta}\big)\frac{\beta^2}{\theta^2_i}
\big|^{-1},
\label{total mag}
\end{equation}
where $\theta_i$ is the angle of $i-th$ image of wormhole. Eq. \eqref{total mag} is a general formula for calculating the magnification. We calculate the Ellis-Bronnikov wormhole as an example. For simplicity, one can use the $b\approx \theta D_L$ into the total magnification \eqref{total mag} with $\eta=1$. Consequently, one can get 
\begin{equation}
\mu=\big|\frac{3\beta}{\beta+\frac{D_{LS}}{D_S}\frac{\pi}{4}\big(\frac{r_0}{b}\big)^2}-\frac{2\beta^2}{\big(\beta+\frac{D_{LS}}{D_S}\frac{\pi}{4}\big(\frac{r_0}{b}\big)^2\big)^2}\big|^{-1}.
\end{equation}
We plot the magnification of Ellis-Bronnikov wormhol in Fig. \ref{fig:my_label 4}. It clearly shows that the magnification will be increased as enhancing $b$, where $r_0$ is fixed. This means that the more closely the photon trajectory is to the wormhole's throat, the more distorted the light can be, the greater magnification. Then there is an maximum value for this magnification, which depends on the relative position $\frac{D_{LS}}{D_S}$ and the intrinsic angle $\beta$, and the maximum value in our case is $4.92992\approx 5$. 



\begin{figure}
    \centering
    \includegraphics[scale=1.1]{magnification.png}
    \caption{The magnification of Ellis wormhole.  As $b\to r_0$, $\frac{D_{LS}}{D_S}=\frac{1}{2}$ and $\beta=0.03$, it shows that $\mu=5$.}
    \label{fig:my_label 4}
\end{figure}


\subsection{Event rate}
The event rate of Ellis-Bronnikov wormhole has been studied by F.Abe \cite{Abe:2010ap}, which he discussed that wormholes of a throat $10\sim 10^{11}$ km are uniformly distributed in the universe, and he also calculated the corresponding optic depth and event rate. We adopt different assumptions: the throat of wormholes are quite large $r_0=10^{20}$ m and wormholes evenly distributed in the universe. We study a wormhole ($n=1$) which is regarded as an object with continuous mass distribution, and its event rate.
The effective mass of wormholes can be calculated as follows
\begin{equation}
M=\frac{r_0}{2}+\int_{r_0}^r4\pi\rho(r^\prime)r^{\prime 2}dr^\prime,
\end{equation}
where the energy density is
\begin{equation}
\rho=-\frac{Ar_0(\frac{r_0}{r})^{\frac{1}{\eta}}}{r^3\eta}\frac{c^4}{8\pi G}.
\end{equation}
Therefore,
\begin{equation}
M=\frac{Ac^4 D_L\big(\frac{r_0}{D_L}\big)^{1+\frac{1}{\eta}}}{2G}+\frac{r_0G-Ac^4}{2G}.
\end{equation}
There is a famous parameter in the field of gravitational lens---Einstein angle:
\begin{equation}
\theta_E\equiv \sqrt{\frac{4GM}{c^2}\frac{D_{LS}}{D_LD_S}},
\end{equation}
and Einstein ring is generally assumed to be the cross-section for microlensing
\begin{equation}
\sigma_{micro}=\pi\theta^2_E.
\end{equation}
This is the solid angle within which a source has to be placed in order to produce a detectable microlensing signal.
The Einstein radius crossing time is
\begin{equation}
\begin{aligned}
t_E&=\frac{D_L\theta_E}{v}\\
&=\sqrt{\frac{2\big(Ac^4D_L\big(\frac{r_0}{D_L}\big)^{1+\frac{1}{\eta}}+r_0G-Ac^4\big)}{c^2v^2}\frac{D_{LS}D_L}{D_S}},
\end{aligned}
\end{equation}
where $v$ is the speed of the observed object.
The optical depth $\tau$ to some distance $D_S$ is the probability that a source at that distance gives rise to a detectable microlensing event . For
\begin{equation}
\tau(D_S)=\frac{4\pi G}{c^2}D_S^2\int_0^1\rho(x)x(1-x)dx,
\end{equation}
where $x=\frac{D_L}{D_S}, dx=\frac{dD_L}{D_S}$, the integral results is
\begin{equation}
\begin{aligned}
&\tau(D_S)=|\frac{Ac^2r_0\big(\frac{r_0}{D_L}\big)^\frac{1}{\eta}(D_L(1+\eta)-D_S)}{2D_SD_L(1+\eta)}
\\
&-\frac{Ac^2(r_0(1+\eta)-D_S)}{2D_S(1+\eta)}|.
\end{aligned}
\end{equation}
To know the rate of microlensing events
we may observe while monitoring a certain number of sources for a specific time, then we could represent the event rate as 
\begin{equation}
\Gamma=\frac{d(N\tau)}{dt}=\frac{2N}{\pi}\int_0^{D_S}n(D_L)\frac{\pi r_E^2}{t_E}dD_L,
\end{equation}
Assuming the Einstein crossing times to all subject are identical, the result is
\begin{equation}
\Gamma=\frac{2N}{\pi t_E}\tau.
\end{equation}
where
\begin{equation}
\begin{aligned}
&\Gamma=\frac{2N}{\pi\sqrt{\frac{2\big(Ac^4D_L\big(\frac{r_0}{D_L}\big)^{1+\frac{1}{\eta}}+r_0G-Ac^4\big)}{c^2v^2}\frac{D_{LS}D_L}{D_S}}}\times
\\
&|\frac{Ac^2r_0\big(\frac{r_0}{D_L}\big)^\frac{1}{\eta}\big(D_L(1+\eta)-D_S\big)}{2D_SD_L(1+\eta)}-\frac{Ac^2\big(r_0(1+\eta)-D_S\big)}{2D_S(1+\eta)}|.
\end{aligned}
\label{event rate}
\end{equation}

\begin{figure}
    \centering
    \includegraphics[scale=1.2]{event_rate_0.png}
    \caption{Event rate \eqref{event rate} varies with $\eta$ whose range covers from $9$ to $10$. As $\eta<2$, $\Gamma$ will nearly be divergent. }
    \label{fig:my_label 1}
\end{figure}

\begin{figure}
    \centering
    \includegraphics[scale=1.2]{event_rate_1.png}
    \caption{Event rate \eqref{event rate} varies with $\eta$ whose range is from $-10$ to $-1$. $\eta=-1$ corresponds to the vacuum case (will show in \ref{number of images}).}
    \label{fig:my_label 2}
\end{figure}
We draw a graph of event rate and $\eta$ as Fig. \ref{fig:my_label 1} and Fig. \ref{fig:my_label 2}.
We find that the event rate in interval $0<\eta<1$ is divergent because at this interval we have get $n\to \infty$, which is restricted by the flaring-out condition \eqref{flaring out condition}, and every image can be observed microlensing events. 
Take Ellis-Bronnikov wormhole ($\eta=1$) as an instance, we can observe about $9\times 10^4$ microlensing events in one year for $1\times10^6$ sources. If in vacuum ($\eta=-1$), we can observe about 300 times a year. We find that Ellis-Bronnikov wormhole ($\eta=1$ will show in Sec. \ref{number of images}) will make the event rate much higher than vacuum. In addition, if two wormholes have the same $\eta$ value, we cannot distinguish them by event rate. Because the event rate is only affected by $\eta$ in our case.







\section{Number of images}
\label{number of images}
In this part, we check the performance of our equations $n=2+\frac{1}{\eta}$ in specific situations which includes the Vacuum case, Ellis-Bronnikov wormhole, charged wormhole.
\subsection{Vacuum Case}
When $\eta=-1$, the metric \eqref{eq7} becomes Schwarzschild-like metric:
\begin{equation}
ds^2=-Adt^2+dr^2+r^2d\Omega^2.
\end{equation}
Our calculation shows the energy-momentum tensor is vanishing, thus we call it as the vacuum case. We solve the lensing equation and obtain
\begin{equation}
\theta=\frac{\pi D_{LS}}{2D_S\sqrt{A}}+\beta.
\end{equation}
For comparison, we substitute $\eta=-1$ into equation \eqref{eq 16}
\begin{equation}
n=2+\frac{1}{-1}=1.
\end{equation}
In the vacuum case, there will be at most one image, which is consistent with physical intuition.


\subsection{Ellis-Bronnikov wormhole}
We first discuss the number of images with the traditional method and then compare it with formula \eqref{eq 16}.
When redshift parameter $A=1$ and $\eta=1$ our metric turn to Ellis-Bronnikov wormhole
\begin{equation}
ds^2=-dt^2+\frac{dr^2}{1-\big(\frac{r_0}{r}\big)^{2}}+r^2d\Omega^2.
\end{equation}
The deflection angle for eq \eqref{eq 22} ($\eta=1$) is
\begin{equation}
\alpha=\frac{\pi}{4}\big(\frac{r_0}{b}\big)^2.
\end{equation}
So, the lens equation is
\begin{equation}
\theta^3-\beta\theta^2-\frac{\pi D_{LS}r_0^2}{4D_SD_L^2}=0.
\end{equation}
for a general cubic equation
$
ax^3+bx^2+cx^1+d=0,
$
The part discriminant is given
$
A=b^2-3ac \quad B=bc-9ad \quad   C=c^2-3bd.
$
The total discriminant is 
$
\varDelta =B^2-4AC.
$
We need $\varDelta<0$ to have three real solutions.
Using equation \eqref{eq 16},
\begin{equation}
n=2+\frac{1}{1}=3.
\end{equation}
We found that these two are consistent. This is not surprising, because Ellis-Bronnikov wormhole fits our previous metric \eqref{eq7}.



\subsection{Charged wormhole}
Referring to article \cite{Liu:2022lfb}, they show that there will be three images at most for the charged spherical symmetric wormhole and its metric \cite{Kim:2001ri} can also be expressed from 
\begin{equation}
ds^2=-\big(1+\frac{Q^2}{r^2}\big)dt^2+\big(1-\frac{r_0^2}{r^2}+\frac{Q^2}{r^2}\big)^{-1}dr^2+r^2d\Omega^2.
\end{equation}
Where $\frac{r_0^2}{r}$ is mass term.
We calculate the state coefficient with $\eta=\frac{p_r}{\rho}$. the result is
\begin{equation}
\eta=-\frac{r^4 \left(Q^4+Q^2 \left(r^2-r_0^2\right)+r^2 r_0^2\right)}{\left(Q^2+r^2\right)^2 \left(Q^2-r_0^2\right) \left(Q^2+r^2-r_0^2\right)},
\end{equation}
where $\eta$ is not a constant here, but $\eta$ is an approximately constant in our integral region, as shown in our numerical diagram Fig. \ref{fig:3}.
In the condition of the weak field and $r_0\ll r$ results in $\eta=1$, substituting equation \eqref{eq 16}
\begin{equation}
n=2+\frac{1}{1}=3.
\end{equation}
It demonstrates consistency in charged wormholes. We know equation \eqref{eq 16} was derived under the assumption of a constant redshift parameter, but this is reasonable that the limitation of microlensing makes $\underset{r\rightarrow\infty}{\lim}1+\frac{Q^2}{r^2}\approx 1$, which means that the observer is quite remote with source and wormhole. 

\begin{figure}
    \centering
    \includegraphics[scale=1.2]{RN_eta.png}
    \caption{The plot shows the relationship between $\eta$ and the radial distance. When $r\to \infty$ in the weak field approximation, it shows that $\eta=1$.}
    \label{fig:3}
\end{figure}

\section{Conclusion and outlook}
\label{conclusion and outlook}
In this paper, we investigate the microlensing effects and the multi-image problem of static spherical wormhole \eqref{eq7}. By introducing the so-called REoS $\eta=\frac{p_r}{\rho}$, we can reformulate the metric \eqref{eq7}. Consequently, one can re-examine the microlensing effect of \eqref{eq7}, including its magnification, and event rate. In particular, one could obtain an explicit formula $n=2+\frac{1}{\eta}$ that reflects the maximum number of images on the equatorial plane. Our analysis shows that the investigation is consistent with the vacuum case, Ellis-Bronnikov wormhole, and charged wormhole. However, this formula is still unknown if considering a quantum-corrected wormhole whose metric is 
\begin{equation}
ds^2=-(1+\frac{\hbar q}{r^2})dt^2+\frac{dr^2}{1-\frac{b_0^2}{r^2}+\frac{\hbar q}{r^2}}+r^2d\Omega^2,
\label{quantum correction wormhole}
\end{equation}
where $\hbar q$ is from the quantum corrections. In the weak field approximation, one can obtain $\eta=\frac{b_0^2+q\hbar}{b_0^2-q\hbar}$. In the large scale ($b_0^2\gg \hbar q$), it will recover the Ellis-Bronnikov wormhole. In the small scale ($b_0^2\approx  \hbar q$), $\eta$ is variable that is determined by the $b_0^2/r$ (dubbed as mass part) and $q$ (dubbed as charge part). From another aspects, the order of lens equation is also unknown. For us, it is still mysterious for the wormhole with quantum corrections. Thus, we conclude that $n=2+\frac{1}{\eta}$ is valid for metric \eqref{eq7}. 





We also reformulate the magnification in terms of $\eta$, then used it to analyze the magnification of the Ellis-Bronnikov wormhole as an instance.
We find that the maximum magnification of an Ellis-Bronnikov wormhole only depends on the relative position $\frac{D_{LS}}{D_S}$ and intrinsic angle $\beta$, and in our case the maximum is about $5$, in which the photon is travelling near to the throat of wormhole. For the other types of static spherical wormholes, one can naturally extend our methods into multi-images cases which will be left for the future work. 

Instead of considering the hypothesis of a small wormhole ($r_0$ is at $10^3-10^{14}$ m), we consider a large-scale wormhole ($r_0$ is around $10^{20}$ m) that is evenly distributed throughout the universe. We calculated the event rate for a single wormhole situation. The event rates of the Ellis-Bronnikov wormhole and the charged wormhole are equal with the same $\eta$ under the microlensing conditions, and they are many orders of magnitude higher than vacuum. An interesting question is whether black holes and wormholes can be distinguished by event rates. Here, we only list some specific cases of the static spherical symmetric wormhole. For implementation into the blackhole, the difficulty is that we need the shape function of the metric of blackhole. Once finding the shape function as shown in \eqref{eq1}, it is possible to distinguish the wormhole and its associated blackhole \cite{Gao:2022cds}. 

 \section*{Acknowledgements}
 We appreciate that Hai-Qing Zhang and Bi-Chu Li give lots of suggestions to improve this manuscript. LH and KG are funded by NSFC grant NO. 12165009. 

\section{Appendix}
\label{appendix}
We unify all of the parameters in SI unit and only impose their values as follows:
\\
$A=1, \,\,\, c=3\times10^8, \,\,\, D_L=2\times10^{24}, \,\,\, D_{LS}=2\times10^{24},\,\,\,
D_S=4\times10^{24}, \,\,\, G=6.67\times 10^{-11},$
\\
$
N=1\times10^6,\,\,\, r_0=5\times10^{20},\,\,\,
v=3\times10^4,\,\,\,
b=5\times10^{22},\,\,\, \beta =0.03.
$

\section*{References}
\begin{thebibliography}{99}
%\cite{Flamm:1916}
\bibitem{Flamm:1916}
Ludwig, Flamm,
%
Beitr¨age zur Einstein schen gravitations theorie. Hirzel, 1916.
%2 itations counted in INSPIRE as of 14 Mar 2023

%\cite{Einstein:1935}
\bibitem{Einstein:1935}
Albert Einstein and Nathan Rosen.
% 
“The particle problem in the general theory of relativity”.
In: Physical Review 48.1 (1935), p. 73.
%

%\cite{Misner:1957}
\bibitem{Misner:1957}
Charles W Misner and John A Wheeler.
% “Classical physics as geometry”. In: Annals of
physics 2.6 (1957), pp. 525–603.
%

%\cite{Ellis:1973}
\bibitem{Ellis:1973}
Homer G Ellis.
% “Ether flow through a drainhole: A particle model in general relativity”. In:
Journal of Mathematical Physics 14.1 (1973),
pp. 104–118.
%

%\cite{Morris:1988cz}
\bibitem{Morris:1988cz}
M.~S.~Morris and K.~S.~Thorne,
%``Wormholes in space-time and their use for interstellar travel: A tool for teaching general relativity,''
Am. J. Phys. \textbf{56} (1988), 395-412
doi:10.1119/1.15620
%1823 citations counted in INSPIRE as of 14 Mar 2023

%\cite{Damour:2007ap}
\bibitem{Damour:2007ap}
T.~Damour and S.~N.~Solodukhin,
%``Wormholes as black hole foils,''
Phys. Rev. D \textbf{76} (2007), 024016
doi:10.1103/PhysRevD.76.024016
[arXiv:0704.2667 [gr-qc]].
%176 citations counted in INSPIRE as of 14 Mar 2023

%\cite{Kim:2003zb}
\bibitem{Kim:2003zb}
W.~T.~Kim, J.~J.~Oh and M.~S.~Yoon,
%``Traversable wormholes construction in (2+1)-dimensions,''
Phys. Rev. D \textbf{70} (2004), 044006
doi:10.1103/PhysRevD.70.044006
[arXiv:gr-qc/0307034 [gr-qc]].
%29 citations counted in INSPIRE as of 14 Mar 2023


%\cite{Bueno:2017hyj}
\bibitem{Bueno:2017hyj}
P.~Bueno, P.~A.~Cano, F.~Goelen, T.~Hertog and B.~Vercnocke,
%``Echoes of Kerr-like wormholes,''
Phys. Rev. D \textbf{97} (2018) no.2, 024040
doi:10.1103/PhysRevD.97.024040
[arXiv:1711.00391 [gr-qc]].
%150 citations counted in INSPIRE as of 15 Mar 2023

%\cite{Visser:1989kh}
\bibitem{Visser:1989kh}
M.~Visser,
%``Traversable wormholes: Some simple examples,''
Phys. Rev. D \textbf{39} (1989), 3182-3184
doi:10.1103/PhysRevD.39.3182
[arXiv:0809.0907 [gr-qc]].
%403 citations counted in INSPIRE as of 15 Mar 2023

%\cite{Sushkov:2005kj}
\bibitem{Sushkov:2005kj}
S.~V.~Sushkov,
%``Wormholes supported by a phantom energy,''
Phys. Rev. D \textbf{71} (2005), 043520
doi:10.1103/PhysRevD.71.043520
[arXiv:gr-qc/0502084 [gr-qc]].
%346 citations counted in INSPIRE as of 15 Mar 2023

%\cite{Bronnikov:2002rn}
\bibitem{Bronnikov:2002rn}
K.~A.~Bronnikov and S.~W.~Kim,
%``Possible wormholes in a brane world,''
Phys. Rev. D \textbf{67} (2003), 064027
doi:10.1103/PhysRevD.67.064027
[arXiv:gr-qc/0212112 [gr-qc]].
%203 citations counted in INSPIRE as of 15 Mar 2023

%\cite{Clement:1995ya}
\bibitem{Clement:1995ya}
G.~Clement,
%``Wormhole cosmic strings,''
Phys. Rev. D \textbf{51} (1995), 6803-6809
doi:10.1103/PhysRevD.51.6803
[arXiv:gr-qc/9502033 [gr-qc]].
%36 citations counted in INSPIRE as of 15 Mar 2023

%\cite{Richarte:2017iit}
\bibitem{Richarte:2017iit}
M.~G.~Richarte, I.~G.~Salako, J.~P.~Morais Gra\c{c}a, H.~Moradpour and A.~\"Ovg\"un,
%``Relativistic Bose-Einstein condensates thin-shell wormholes,''
Phys. Rev. D \textbf{96} (2017) no.8, 084022
doi:10.1103/PhysRevD.96.084022
[arXiv:1710.05886 [gr-qc]].
%36 citations counted in INSPIRE as of 15 Mar 2023

%\cite{Ayuso:2020vuu}
\bibitem{Ayuso:2020vuu}
I.~Ayuso, F.~S.~N.~Lobo and J.~P.~Mimoso,
%``Wormhole geometries induced by action-dependent Lagrangian theories,''
Phys. Rev. D \textbf{103} (2021) no.4, 044018
doi:10.1103/PhysRevD.103.044018
[arXiv:2012.00047 [gr-qc]].
%4 citations counted in INSPIRE as of 15 Mar 2023

%\cite{KordZangeneh:2020ixt}
\bibitem{KordZangeneh:2020ixt}
M.~Kord Zangeneh and F.~S.~N.~Lobo,
%``Dynamic wormhole geometries in hybrid metric-Palatini gravity,''
Eur. Phys. J. C \textbf{81} (2021) no.4, 285
doi:10.1140/epjc/s10052-021-09059-y
[arXiv:2011.01745 [gr-qc]].
%13 citations counted in INSPIRE as of 15 Mar 2023

%\cite{Song:2023jdn}
\bibitem{Song:2023jdn}
S.~Q.~Song and E.~G\"udekli,
%``Traversable wormholes in Rastall teleparallel gravity with non-commutative geometry,''
New Astron. \textbf{100} (2023), 101993
doi:10.1016/j.newast.2022.101993
%0 citations counted in INSPIRE as of 15 Mar 2023

%\cite{Saleem:2023lul}
\bibitem{Saleem:2023lul}
R.~Saleem, M.~I.~Aslam and K.~Rasool,
%``Wormhole solutions in Rastall-like-torsion-trace gravity,''
Chin. J. Phys. \textbf{82} (2023), 1-14
doi:10.1016/j.cjph.2022.12.015
%0 citations counted in INSPIRE as of 15 Mar 2023

%\cite{Eid:2023wrd}
\bibitem{Eid:2023wrd}
A.~Eid and A.~Alkaoud,
%``Dynamics and stability of Hayward -de Sitter thin-shell wormhole,''
New Astron. \textbf{101} (2023), 102021
doi:10.1016/j.newast.2023.102021
%0 citations counted in INSPIRE as of 15 Mar 2023

%\cite{Godani:2023paj}
\bibitem{Godani:2023paj}
N.~Godani,
%``Stability of Heyward wormhole in f(R) gravity,''
New Astron. \textbf{100} (2023), 101994
doi:10.1016/j.newast.2022.101994
%1 citations counted in INSPIRE as of 15 Mar 2023

%\cite{Hochberg:1998ii}
\bibitem{Hochberg:1998ii}
D.~Hochberg and M.~Visser,
%``The Null energy condition in dynamic wormholes,''
Phys. Rev. Lett. \textbf{81} (1998), 746-749
doi:10.1103/PhysRevLett.81.746
[arXiv:gr-qc/9802048 [gr-qc]].
%202 citations counted in INSPIRE as of 15 Mar 2023

%\cite{Hochberg:1998ha}
\bibitem{Hochberg:1998ha}
D.~Hochberg and M.~Visser,
%``Dynamic wormholes, anti-trapped surfaces, and energy conditions,''
Phys. Rev. D \textbf{58} (1998), 044021
doi:10.1103/PhysRevD.58.044021
[arXiv:gr-qc/9802046 [gr-qc]].
%243 citations counted in INSPIRE as of 15 Mar 2023

%\cite{Lobo:2002zf}
\bibitem{Lobo:2002zf}
F.~Lobo and P.~Crawford,
%``Weak energy condition violation and superluminal travel,''
Lect. Notes Phys. \textbf{617} (2003), 277-291
doi:10.1007/3-540-36973-2\_15
[arXiv:gr-qc/0204038 [gr-qc]].
%31 citations counted in INSPIRE as of 15 Mar 2023

%\cite{Lobo:2004rp}
\bibitem{Lobo:2004rp}
F.~S.~N.~Lobo,
%``Energy conditions, traversable wormholes and dust shells,''
Gen. Rel. Grav. \textbf{37} (2005), 2023-2038
doi:10.1007/s10714-005-0177-x
[arXiv:gr-qc/0410087 [gr-qc]].
%66 citations counted in INSPIRE as of 15 Mar 2023

%\cite{Einstein:1936llh}
\bibitem{Einstein:1936llh}
A.~Einstein,
%``Lens-Like Action of a Star by the Deviation of Light in the Gravitational Field,''
Science \textbf{84} (1936), 506-507
doi:10.1126/science.84.2188.506
%398 citations counted in INSPIRE as of 15 Mar 2023


%\cite{Bartelmann:1999yn}
\bibitem{Bartelmann:1999yn}
M.~Bartelmann and P.~Schneider,
%``Weak gravitational lensing,''
Phys. Rept. \textbf{340} (2001), 291-472
doi:10.1016/S0370-1573(00)00082-X
[arXiv:astro-ph/9912508 [astro-ph]].
%1742 citations counted in INSPIRE as of 15 Mar 2023

%\cite{Kaiser:1991qi}
\bibitem{Kaiser:1991qi}
N.~Kaiser,
%``Weak gravitational lensing of distant galaxies,''
Astrophys. J. \textbf{388} (1992), 272
doi:10.1086/171151
%665 citations counted in INSPIRE as of 15 Mar 2023

%\cite{Bozza:2002zj}
\bibitem{Bozza:2002zj}
V.~Bozza,
%``Gravitational lensing in the strong field limit,''
Phys. Rev. D \textbf{66} (2002), 103001
doi:10.1103/PhysRevD.66.103001
[arXiv:gr-qc/0208075 [gr-qc]].
%434 citations counted in INSPIRE as of 15 Mar 2023


%\cite{Virbhadra:1999nm}
\bibitem{Virbhadra:1999nm}
K.~S.~Virbhadra and G.~F.~R.~Ellis,
%``Schwarzschild black hole lensing,''
Phys. Rev. D \textbf{62} (2000), 084003
doi:10.1103/PhysRevD.62.084003
[arXiv:astro-ph/9904193 [astro-ph]].
%701 citations counted in INSPIRE as of 15 Mar 2023

%\cite{Perlick:2004tq}
\bibitem{Perlick:2004tq}
V.~Perlick,
%``Gravitational lensing from a spacetime perspective,''
Living Rev. Rel. \textbf{7} (2004), 9
%249 citations counted in INSPIRE as of 15 Mar 2023

%\cite{SDSS:2002oin}
\bibitem{SDSS:2002oin}
C.~Stoughton \textit{et al.} [SDSS],
%``The Sloan Digital Sky Survey: Early Data Release,''
Astron. J. \textbf{123} (2002), 485-548
doi:10.1086/324741
%1628 citations counted in INSPIRE as of 15 Mar 2023

%\cite{Gao:2022cds}
\bibitem{Gao:2022cds}
K.~Gao, L.~H.~Liu and M.~Zhu,
%``Microlensing effects of wormholes associated to blackhole spacetimes,''
[arXiv:2211.17065 [gr-qc]].
%2 citations counted in INSPIRE as of 15 Mar 2023

%\cite{Liu:2022lfb}
\bibitem{Liu:2022lfb}
L.~H.~Liu, M.~Zhu, W.~Luo, Y.~F.~Cai and Y.~Wang,
%``Microlensing effect of a charged spherically symmetric wormhole,''
Phys. Rev. D \textbf{107} (2023) no.2, 024022
doi:10.1103/PhysRevD.107.024022
[arXiv:2207.05406 [gr-qc]].
%5 citations counted in INSPIRE as of 15 Mar 2023

%\cite{Sokoliuk:2022owk}
\bibitem{Sokoliuk:2022owk}
O.~Sokoliuk, S.~Praharaj, A.~Baransky and P.~K.~Sahoo,
%``Accretion flows around exotic tidal wormholes - I. Ray-tracing,''
Astron. Astrophys. \textbf{665} (2022), A139
doi:10.1051/0004-6361/202244358
[arXiv:2207.07193 [gr-qc]].
%1 citations counted in INSPIRE as of 15 Mar 2023

%\cite{Zatrimaylov:2021ijd}
\bibitem{Zatrimaylov:2021ijd}
K.~Zatrimaylov,
%``Dark Matter, Rotation Curves, and the Morphology of Galaxies,''
[arXiv:2108.13350 [astro-ph.CO]].
%0 citations counted in INSPIRE as of 15 Mar 2023

%\cite{Cheng:2021hoc}
\bibitem{Cheng:2021hoc}
X.~T.~Cheng and Y.~Xie,
%``Probing a black-bounce, traversable wormhole with weak deflection gravitational lensing,''
Phys. Rev. D \textbf{103} (2021) no.6, 064040
doi:10.1103/PhysRevD.103.064040
%31 citations counted in INSPIRE as of 15 Mar 2023

%\cite{Li:2019qyb}
\bibitem{Li:2019qyb}
Z.~Li and J.~Jia,
%``The finite-distance gravitational deflection of massive particles in stationary spacetime: a Jacobi metric approach,''
Eur. Phys. J. C \textbf{80} (2020) no.2, 157
doi:10.1140/epjc/s10052-020-7665-8
[arXiv:1912.05194 [gr-qc]].
%36 citations counted in INSPIRE as of 15 Mar 2023

%\cite{Kuniyasu:2018cgv}
\bibitem{Kuniyasu:2018cgv}
M.~Kuniyasu, K.~Nanri, N.~Sakai, T.~Ohgami, R.~Fukushige and S.~Komura,
%``Can we identify massless braneworld black holes by observations?,''
Phys. Rev. D \textbf{97} (2018) no.10, 104063
doi:10.1103/PhysRevD.97.104063
[arXiv:1806.00231 [gr-qc]].
%5 citations counted in INSPIRE as of 15 Mar 2023

%\cite{Raidal:2018eoo}
\bibitem{Raidal:2018eoo}
M.~Raidal, S.~Solodukhin, V.~Vaskonen and H.~Veerm\"ae,
%``Light Primordial Exotic Compact Objects as All Dark Matter,''
Phys. Rev. D \textbf{97} (2018) no.12, 123520
doi:10.1103/PhysRevD.97.123520
[arXiv:1802.07728 [astro-ph.CO]].
%27 citations counted in INSPIRE as of 15 Mar 2023

%\cite{Tsukamoto:2017hva}
\bibitem{Tsukamoto:2017hva}
N.~Tsukamoto and Y.~Gong,
%``Extended source effect on microlensing light curves by an Ellis wormhole,''
Phys. Rev. D \textbf{97} (2018) no.8, 084051
doi:10.1103/PhysRevD.97.084051
[arXiv:1711.04560 [gr-qc]].
%27 citations counted in INSPIRE as of 15 Mar 2023

%\cite{Sajadi:2016hko}
\bibitem{Sajadi:2016hko}
S.~N.~Sajadi and N.~Riazi,
%``Gravitational lensing by multi-polytropic wormholes,''
Can. J. Phys. \textbf{98} (2020) no.11, 1046-1054
doi:10.1139/cjp-2019-0524
[arXiv:1611.04343 [gr-qc]].
%13 citations counted in INSPIRE as of 15 Mar 2023

%\cite{Lukmanova:2016czn}
\bibitem{Lukmanova:2016czn}
R.~Lukmanova, A.~Kulbakova, R.~Izmailov and A.~A.~Potapov,
%``Gravitational Microlensing by Ellis Wormhole: Second Order Effects,''
Int. J. Theor. Phys. \textbf{55} (2016) no.11, 4723-4730
doi:10.1007/s10773-016-3095-7
%21 citations counted in INSPIRE as of 15 Mar 2023

%\cite{Tsukamoto:2016zdu}
\bibitem{Tsukamoto:2016zdu}
N.~Tsukamoto and T.~Harada,
%``Light curves of light rays passing through a wormhole,''
Phys. Rev. D \textbf{95} (2017) no.2, 024030
doi:10.1103/PhysRevD.95.024030
[arXiv:1607.01120 [gr-qc]].
%64 citations counted in INSPIRE as of 15 Mar 2023

%\cite{Kitamura:2016vad}
\bibitem{Kitamura:2016vad}
T.~Kitamura, Gravitational lensing in an exotic spacetime
%``Gravitational lensing in an exotic spacetime,''
%0 citations counted in INSPIRE as of 15 Mar 2023

%\cite{Kitamura:2012zy}
\bibitem{Kitamura:2012zy}
T.~Kitamura, K.~Nakajima and H.~Asada,
%``Demagnifying gravitational lenses toward hunting a clue of exotic matter and energy,''
Phys. Rev. D \textbf{87} (2013) no.2, 027501
doi:10.1103/PhysRevD.87.027501
[arXiv:1211.0379 [gr-qc]].
%50 citations counted in INSPIRE as of 15 Mar 2023

%\cite{Kitamura:2012wcg}
\bibitem{Kitamura:2012wcg}
T.~Kitamura, Astrometric microlensing by the Ellis Wormhole
%``Astrometric microlensing by the Ellis Wormhole,''
%0 citations counted in INSPIRE as of 15 Mar 2023

%\cite{Toki:2011zu}
\bibitem{Toki:2011zu}
Y.~Toki, T.~Kitamura, H.~Asada and F.~Abe,
%``Astrometric Image Centroid Displacements due to Gravitational Microlensing by the Ellis Wormhole,''
Astrophys. J. \textbf{740} (2011), 121
doi:10.1088/0004-637X/740/2/121
[arXiv:1107.5374 [astro-ph.CO]].
%85 citations counted in INSPIRE as of 15 Mar 2023

%\cite{Abe:2010ap}
\bibitem{Abe:2010ap}
F.~Abe,
%``Gravitational Microlensing by the Ellis Wormhole,''
Astrophys. J. \textbf{725} (2010), 787-793
doi:10.1088/0004-637X/725/1/787
[arXiv:1009.6084 [astro-ph.CO]].
%135 citations counted in INSPIRE as of 15 Mar 2023

%\cite{Bogdanov:2008zy}
\bibitem{Bogdanov:2008zy}
M.~B.~Bogdanov and A.~M.~Cherepashchuk,
%``Search for exotic matter from gravitational microlensing observations of stars,''
Astrophys. Space Sci. \textbf{317} (2008), 181-192
doi:10.1007/s10509-008-9870-z
[arXiv:0807.2774 [astro-ph]].
%14 citations counted in INSPIRE as of 15 Mar 2023

%\cite{Torres:2001gb}
\bibitem{Torres:2001gb}
D.~F.~Torres, E.~F.~Eiroa and G.~E.~Romero,
%``On the possibility of an astronomical detection of chromaticity effects in microlensing by wormhole - like objects,''
Mod. Phys. Lett. A \textbf{16} (2001), 1849-1861
doi:10.1142/S0217732301005126
[arXiv:gr-qc/0109041 [gr-qc]].
%7 citations counted in INSPIRE as of 15 Mar 2023

%\cite{Safonova:2001vz}
\bibitem{Safonova:2001vz}
M.~Safonova, D.~F.~Torres and G.~E.~Romero,
%``Microlensing by natural wormholes: Theory and simulations,''
Phys. Rev. D \textbf{65} (2002), 023001
doi:10.1103/PhysRevD.65.023001
[arXiv:gr-qc/0105070 [gr-qc]].
%103 citations counted in INSPIRE as of 15 Mar 2023

%\cite{Torres:1998cu}
\bibitem{Torres:1998cu}
D.~F.~Torres, G.~E.~Romero and L.~A.~Anchordoqui,
%``Wormholes, gamma-ray bursts and the amount of negative mass in the universe,''
Mod. Phys. Lett. A \textbf{13} (1998), 1575-1582
doi:10.1142/S0217732398001650
[arXiv:gr-qc/9805075 [gr-qc]].
%42 citations counted in INSPIRE as of 15 Mar 2023

%\cite{Torres:1998xd}
\bibitem{Torres:1998xd}
D.~F.~Torres, G.~E.~Romero and L.~A.~Anchordoqui,
%``Might some gamma-ray bursts be an observable signature of natural wormholes?,''
Phys. Rev. D \textbf{58} (1998), 123001
doi:10.1103/PhysRevD.58.123001
[arXiv:astro-ph/9802106 [astro-ph]].
%61 citations counted in INSPIRE as of 15 Mar 2023


%\cite{Gibbons:2008rj}
\bibitem{Gibbons:2008rj}
G.~W.~Gibbons and M.~C.~Werner,
%``Applications of the Gauss-Bonnet theorem to gravitational lensing,''
Class. Quant. Grav. \textbf{25} (2008), 235009
doi:10.1088/0264-9381/25/23/235009
[arXiv:0807.0854 [gr-qc]].
%195 citations counted in INSPIRE as of 15 Mar 2023

%\cite{Gibbons:2008zi}
\bibitem{Gibbons:2008zi}
G.~W.~Gibbons, C.~A.~R.~Herdeiro, C.~M.~Warnick and M.~C.~Werner,
%``Stationary Metrics and Optical Zermelo-Randers-Finsler Geometry,''
Phys. Rev. D \textbf{79} (2009), 044022
doi:10.1103/PhysRevD.79.044022
[arXiv:0811.2877 [gr-qc]].
%104 citations counted in INSPIRE as of 15 Mar 2023

%\cite{Werner:2012rc}
\bibitem{Werner:2012rc}
M.~C.~Werner,
%``Gravitational lensing in the Kerr-Randers optical geometry,''
Gen. Rel. Grav. \textbf{44} (2012), 3047-3057
doi:10.1007/s10714-012-1458-9
[arXiv:1205.3876 [gr-qc]].
%150 citations counted in INSPIRE as of 15 Mar 2023

%\cite{He:2023hsv}
\bibitem{He:2023hsv}
X.~He, T.~Xu, Y.~Yu, A.~Karamat, R.~Babar and R.~Ali,
%``Deflection angle evolution with plasma medium and without plasma medium in a parameterized black hole,''
Annals Phys. \textbf{451} (2023), 169247
doi:10.1016/j.aop.2023.169247
%0 citations counted in INSPIRE as of 15 Mar 2023

%\cite{Upadhyay:2023yhk}
\bibitem{Upadhyay:2023yhk}
S.~Upadhyay, S.~Mandal, Y.~Myrzakulov and K.~Myrzakulov,
%``Weak deflection angle, greybody bound and shadow for charged massive BTZ black hole,''
Annals Phys. \textbf{450} (2023), 169242
doi:10.1016/j.aop.2023.169242
[arXiv:2303.02132 [gr-qc]].
%0 citations counted in INSPIRE as of 15 Mar 2023

%\cite{Javed:2023iih}
\bibitem{Javed:2023iih}
W.~Javed, M.~Atique, R.~C.~Pantig and A.~\"Ovg\"un,
%``Weak Deflection Angle, Hawking Radiation and Greybody Bound of Reissner\textendash{}Nordstr\"om Black Hole Corrected by Bounce Parameter,''
Symmetry \textbf{15} (2023) no.1, 148
doi:10.3390/sym15010148
[arXiv:2301.01855 [gr-qc]].
%1 citations counted in INSPIRE as of 15 Mar 2023

%\cite{Javed:2022gtz}
\bibitem{Javed:2022gtz}
W.~Javed, H.~Irshad, R.~C.~Pantig and A.~\"Ovg\"un,
%``Weak Deflection Angle by Kalb\textendash{}Ramond Traversable Wormhole in Plasma and Dark Matter Mediums,''
Universe \textbf{8} (2022) no.11, 599
doi:10.3390/universe8110599
[arXiv:2211.07009 [gr-qc]].
%4 citations counted in INSPIRE as of 15 Mar 2023

%\cite{Huang:2022soh}
\bibitem{Huang:2022soh}
Y.~Huang and Z.~Cao,
%``Generalized Gibbons-Werner method for deflection angle,''
Phys. Rev. D \textbf{106} (2022) no.10, 104043
doi:10.1103/PhysRevD.106.104043
%1 citations counted in INSPIRE as of 15 Mar 2023

%\cite{He:2022yhp}
\bibitem{He:2022yhp}
J.~He, Q.~Wang, Q.~Hu, L.~Feng and J.~Jia,
%``Deflection in higher dimensional spacetime and asymptotically non-flat spacetimes,''
Class. Quant. Grav. \textbf{40} (2023) no.6, 065006
doi:10.1088/1361-6382/acbade
[arXiv:2210.00938 [gr-qc]].
%0 citations counted in INSPIRE as of 15 Mar 2023

%\cite{Gao:2023ltr}
\bibitem{Gao:2023ltr}
X.~J.~Gao, X.~k.~Yan, Y.~Yin and Y.~P.~Hu,
%``Gravitational lensing by a charged spherically symmetric black hole immersed in thin dark matter,''
[arXiv:2303.00190 [gr-qc]].
%0 citations counted in INSPIRE as of 15 Mar 2023

%\cite{Cai:2023ite}
\bibitem{Cai:2023ite}
T.~Cai, Z.~Wang, H.~Huang and M.~Zhu,
%``Higher order correction to weak-field lensing of Ellis-Bronnikov wormhole,''
[arXiv:2302.13704 [gr-qc]].
%0 citations counted in INSPIRE as of 15 Mar 2023

%\cite{Ovgun:2023ego}
\bibitem{Ovgun:2023ego}
A.~\"Ovg\"un, R.~C.~Pantig and \'A.~Rinc\'on,
%``4D scale-dependent Schwarzschild-AdS/dS black holes: study of shadow and weak deflection angle and greybody bounding,''
Eur. Phys. J. Plus \textbf{138} (2023) no.3, 192
doi:10.1140/epjp/s13360-023-03793-w
[arXiv:2303.01696 [gr-qc]].
%1 citations counted in INSPIRE as of 15 Mar 2023

%\cite{Lobo:2005us}
\bibitem{Lobo:2005us}
F.~S.~N.~Lobo,
%``Phantom energy traversable wormholes,''
Phys. Rev. D \textbf{71} (2005), 084011
doi:10.1103/PhysRevD.71.084011
[arXiv:gr-qc/0502099 [gr-qc]].
%445 citations counted in INSPIRE as of 15 Mar 2023

%\cite{Lobo:2005yv}
\bibitem{Lobo:2005yv}
F.~S.~N.~Lobo,
%``Stability of phantom wormholes,''
Phys. Rev. D \textbf{71} (2005), 124022
doi:10.1103/PhysRevD.71.124022
[arXiv:gr-qc/0506001 [gr-qc]].
%211 citations counted in INSPIRE as of 15 Mar 2023

%\cite{Garattini:2007ff}
\bibitem{Garattini:2007ff}
R.~Garattini and F.~S.~N.~Lobo,
%``Self sustained phantom wormholes in semi-classical gravity,''
Class. Quant. Grav. \textbf{24} (2007), 2401-2413
doi:10.1088/0264-9381/24/9/016
[arXiv:gr-qc/0701020 [gr-qc]].
%80 citations counted in INSPIRE as of 15 Mar 2023




%\cite{Kim:2001ri}
\bibitem{Kim:2001ri}
S.~W.~Kim and H.~Lee,
%``Exact solutions of a charged wormhole,''
Phys. Rev. D \textbf{63}, 064014 (2001)
doi:10.1103/PhysRevD.63.064014
[arXiv:gr-qc/0102077 [gr-qc]].
%119 citations counted in INSPIRE as of 19 Mar 2023






\end{thebibliography}
\bibliography{mybibfile}








\end{document}
