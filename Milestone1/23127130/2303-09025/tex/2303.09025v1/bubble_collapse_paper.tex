\documentclass[12pt, letter]{article}

% for arXiv submissions
\pdfoutput=1

%Packages
\usepackage{graphicx}
\usepackage{caption}
\usepackage{subcaption}

% For mathematics
\usepackage{amsmath}
\usepackage{amsthm}
\usepackage{amssymb}
\usepackage{esint}
\usepackage[linkcolor=blue,colorlinks=true]{hyperref}

% for title ONLY IF ARXIV TYPE OF PAPER TO BE PUBLISHED
\usepackage[affil-it]{authblk} 

% for upright greek letters
\usepackage{upgreek}


% UNCOMMENT FOR ARXIV TYPE OF PAPER
\title{Power Law Singularity for Cavity Collapse in a Compressible Euler Fluid with Tait-Murnaghan Equation of State.} 
\author{Daniels Krimans and Seth Putterman}
\affil{Department of Physics and Astronomy, \\
University of California Los Angeles, CA 90095}
\date{March 16, 2023}

\begin{document}
\maketitle
Motivated by the high energy focusing found in rapidly collapsing bubbles that is relevant to implosion processes that concentrate energy density, such as sonoluminescence, we consider a calculation of an empty cavity collapse in a compressible Euler fluid. We review and then use the method based on similarity theory [Hunter, J. Fluid Mech. 8(2):241–263 (1960)] that was previously used to compute the power law exponent $n$ for the collapse of an empty cavity in water during the late stage of the collapse. We extend this calculation by considering different fluids surrounding the cavity, all of which are parametrized by the Tait-Murnaghan equation of state through parameter $\gamma$. As a result, we obtain the dependence of $n$ on $\gamma$ for a wide range of $\gamma$, and indeed see that the collapse is sensitive to the equation of state of an outside fluid.

\section{Introduction}
Rayleigh \cite{rayleigh_1917} calculated that the radius $R$ of an empty cavity in an incompressible ideal fluid collapsed to zero at a finite time $t_0$ as:
\begin{equation}
\label{rayleigh_collapse_power_law}
R(t) = A(t_0 - t)^{2/5},
\end{equation}
where $A = (5/2)^{2/5} (E / 2 \pi \rho)^{1/5}$, $\rho$ is the mass density of the fluid and $E = (4 \pi/3) p_0 R_m^3$ is the energy of the fluid, which is the energy to form the initial cavity of radius $R_m$ in an otherwise stationary fluid, where $p_0$ is the ambient externally applied pressure. If there are $N$ atoms inside the cavity the collapse will arrest at a finite radius $R_c$  and at this moment the energy per particle will be $E/N$. A typical experiment \cite{Lfstedt1993TowardAH} reaches $R_m = 50$  $\upmu$m with $N = 1.2 \cdot 10^{10}$. Due to the finite size of the atoms the collapse arrests at $R_c \approx 0.5$ $\upmu$m and at this stagnation point the average energy delivered to each particle is $E/N \approx 25$ eV. As the emission of ultraviolet photons, with an energy of $6$ eV \cite{PhysRevLett.69.1182}, can be observed from the interior of the collapsed bubble, the Rayleigh bubble dynamics is widely regarded as providing a zeroth order picture of sonoluminescence. 
\par
As the moment of the Rayleigh collapse is approached, from \eqref{rayleigh_collapse_power_law} the velocity of the cavity wall approaches infinity as $\dot{R}(t) = (-2A/5) (t_0 - t)^{-3/5}$. Consider again the situation where gas is contained in the cavity. On the one hand it will arrest the collapse prior to reaching zero radius. On the other hand the speed of the gas at the boundary of the cavity $r = R(t)$ can become supersonic relative to the medium in the bubble which, in contrast to the external fluid, is compressible \cite{Lfstedt1993TowardAH}. If the supersonic motion occurs well before arresting of the collapse then an imploding shock wave can form. The shock can focus to the origin, $r = 0$, independent of the presence of matter. In this case $E/N \approx 1000$ eV has been theoretically predicted \cite{PhysRevLett.70.3424}. Realization of this handover in the focusing of energy density would raise prospects for the use of acoustics to achieve thermal fusion \cite{bass_2008}. Ramsey \cite{PhysRevLett.110.154301}, \cite{Ramsey_PhD} has emphasized that a small compressibility of the surrounding fluid might slow down the collapse and affect the attainment of a next stage in energy focusing. In particular, one notes that in 1960 Hunter \cite{hunter_1960} calculated that for water (which is the fluid of choice for most experiments) near the moment of the collapse:
\begin{equation}
R(t) = A_n(t_0 - t)^{n},
\end{equation}
where $n = 0.5552 \approx 5/9$. Hunter used the Tait-Murnaghan \cite{cole_1948}, \cite{Murnaghan1944TheCO} form of the equation of state for the fluid's pressure $p$:
\begin{equation}
p(\rho) = B\left(\left( \frac{\rho}{\rho_0} \right)^{\gamma} - 1 \right),
\end{equation}
where $B = 3000$ atm, $\gamma = 7$, $\rho_0 = 1000$ kg/$\textrm{m}^3$. The compressibility of water slows down the collapse and preliminary analysis indicates that the handover to an imploding shock wave can be suppressed \cite{doi:10.1121/2.0000869}.
\par
Even for ideal hydrodynamics the compressibility of the fluid has a strong influence on the extent to which energy density is concentrated. Motivated by this perspective we present a calculation for $n(\gamma)$ for a wide range of $\gamma$. Key results are displayed in figures \ref{figure_n_water} and \ref{figure_n_large}. These calculations might motivate a search for candidate liquids to achieve greater levels of energy focusing. These calculations also provide an asymptotic limit that can be used to evaluate accuracy of more general numerical solutions, for example, simulations of all Mach number bubble dynamics \cite{FUSTER2018752}. 

\begin{figure}[t]
    \centering
    \begin{minipage}{0.45\textwidth}
        \centering
        \includegraphics[width=0.9\textwidth]{values_close_to_water.pdf} % first figure itself
        \caption{Predicted values of $n$ as a function of $\gamma$ are denoted as dots. Here, values of $\gamma$ are those close to water that has $\gamma = 7$. The solid line is the proposed fit of the form $n = 0.4 + a \gamma^{-b}$ and the dashed line
is the value for the incompressible fluid, $n = 0.4$. }
\label{figure_n_water}
    \end{minipage}\hfill
    \begin{minipage}{0.45\textwidth}
        \centering
        \includegraphics[width=0.9\textwidth]{all_values.pdf} % second figure itself
        \caption{Predicted values of $n$ as a function of $\gamma$ are denoted as dots, and the dashed line is the value for the incompressible fluid, $n = 0.4$. }
\label{figure_n_large}
    \end{minipage}
\end{figure}

\section{Theory}
To perform computations, we use methods described in \cite{hunter_1960}. In this section, we give a summary of such methods. \par

We assume spherically symmetric scenario in which spherical empty cavity has its center placed in the origin of the coordinate system and whose radius is desribed by a time dependent function $R(t)$. Outside of this radius, we assume an
infinite ideal fluid which is described by mass conservation law and Euler's equation, where due to spherical symmetry fluid's only nonzero component of velocity is radial component $u$ and that both radial velocity component and 
mass density $\rho$ are only functions of radial coordinate $r$ and time $t$. We do not consider equation for entropy as we assume that the flow is homentropic. The following equations are considered for $r > R(t)$. 
\begin{equation}
\label{euler_equation}
\begin{gathered}
\rho \left( \frac{\partial u}{\partial t} + u \frac{\partial u}{\partial r} \right) = - \frac{\partial p}{\partial r} 
\end{gathered}
\end{equation}
\begin{equation}
\label{mass_conservation}
\begin{gathered}
\frac{\partial \rho}{\partial t} + u \frac{\partial \rho}{\partial r} + \rho \left( \frac{2 u}{r} + \frac{\partial u}{\partial r }\right) = 0
\end{gathered}
\end{equation}\par
To describe pressure $p$, we assume Tait-Murnaghan equation of state, where $s$ is the specific entropy.
\begin{equation}
\label{equation_of_state}
p(\rho, s) = B(s) \left( \left(\frac{\rho}{\rho_0(s)}\right)^{\gamma} -1 \right)
\end{equation}\par
For simplicity, as the flow is homentropic, from now on we will not explicitly write dependence on specific entropy of $B, \rho_0$, or other entropy dependent variables. \par
We would like to enforce two boundary conditions for both radial velocity $u$ and pressure $p$, where one is at the interface between empty cavity and fluid $r = R(t)$, and the other one is far from the cavity as $r \to \infty$. For radial velocity,
we assume that empty cavity is a free surface and that far away from the cavity fluid is at rest. For pressure, we assume that at the surface of the cavity pressure is zero as the cavity is empty, and far away it approaches some finite
value $p_{\infty}$. Boundary conditions are summarized next, where dot represents derivative with respect to time.
\begin{equation}
\label{velocity_bc}
\begin{gathered}
u(R(t), t) = \dot{R}(t), \qquad \lim \limits_{r \to \infty} u(r,t) = 0
\end{gathered}
\end{equation} 
\begin{equation}
\label{pressure_bc}
\begin{gathered}
p(R(t), t) = 0, \qquad \lim \limits_{r \to \infty} p(r,t) = p_{\infty}
\end{gathered}
\end{equation} \par
Using assumed equation state (\ref{equation_of_state}), it is possible to compute speed of sound squared $c^2$ as a function of $\rho$, and change variables describing fluid from $u, \rho$ to $u, c^2$. This is convenient in order 
to apply similarity theory for the later parts of the collapse when $R(t) \to 0$, as both variables $u$ and $c^2$ can be directly compared to the velocity of cavity's wall $\dot{R}(t)$. 
\begin{equation}
\label{speed_of_sound}
\begin{gathered}
c^2 = \frac{\partial p}{\partial \rho}= \frac{B \gamma \rho^{\gamma-1}}{\rho_0^{\gamma}}, \qquad \rho= \left(\frac{\rho_0^{\gamma} c^2}{B \gamma} \right)^{1/(\gamma-1)}
\end{gathered}
\end{equation}
Using expression for $c^2$ in terms of density $\rho$ as in equation \eqref{speed_of_sound}, the rewritten spherical Euler's equation (\ref{euler_equation}) and mass conservation law (\ref{mass_conservation}) in terms of variables $u, c^2$ look as follows.
\begin{equation}
\label{euler_equation_2}
\begin{gathered}
\frac{\partial u}{\partial t} + u \frac{\partial u}{\partial r}+ \frac{1}{(\gamma-1)}\frac{\partial c^2}{\partial r} = 0 
\end{gathered}
\end{equation}
\begin{equation}
\label{mass_conservation_2}
\begin{gathered}
\frac{\partial c^2}{\partial t} + u \frac{\partial c^2}{\partial r} + c^2(\gamma-1)\left( \frac{2u}{r} + \frac{\partial u}{\partial r} \right) = 0
\end{gathered}
\end{equation}
Using equations (\ref{equation_of_state}), (\ref{speed_of_sound}), it is possible to compute boundary conditions for $c^2$ denoted by constants $c_0^2, c_{\infty}^2$, from the corresponding boundary conditions for pressure as in (\ref{pressure_bc}).
\begin{equation}
\label{speed_of_sound_bc}
\begin{gathered}
c^2(R(t), t) = \frac{B \gamma}{\rho_0} = c_0^2, \qquad \lim \limits_{r \to \infty} c^2(r,t) = c^2_{\infty}
\end{gathered}
\end{equation}\par
Instead of solving system of partial differential equations (\ref{euler_equation_2}), (\ref{mass_conservation_2}) which would mean that we have to supply initial conditions, we consider similarity theory that is motivated by numerical results in \cite{hunter_1960}. 
As we approach the last phase of the collapse of the empty cavity when $R(t) \to 0$, we assume that length scale of the problem is given by $R(t)$ and scale for velocities in the problem is given by $\dot{R}(t)$. So, we seek solutions of the following form, where
we are interested in finding functions $f$ and $g$. The goal is to reduce problem to a system of ordinary differential equations for $f$ and $g$. 
\begin{equation}
\label{similarity_assumption}
\begin{gathered}
\frac{u(r,t)}{\dot{R}(t)} = f \left( \frac{r}{R(t)} \right) \qquad \frac{c^2(r,t)}{\dot{R}^2(t) } = g \left( \frac{r}{R(t)} \right)
\end{gathered}
\end{equation} \par
Also, motivated by the solution in the incompressible case \cite{rayleigh_1917} and numerical results in \cite{hunter_1960}, we also assume the power law form for $R(t) = A_n(t_0 - t)^n$, where $t_0$ is a time at which collapse happens and $n$ is a power law exponent that we would like to compute. Using such power law assumption and similarity approach for $u, c^2$ as in (\ref{similarity_assumption}), we can rewrite equations (\ref{euler_equation_2}), (\ref{mass_conservation_2}) as two coupled ordinary
differential equations for functions $f$ and $g$, where we introduce variable $x = r/R(t)$. For such variable, differential equations have to be solved in the range $x > 1$. \par
\begin{equation}
\label{diff_eq_f}
\begin{gathered}
f'(x) \left( f(x)-x \right) + \left( 1 - \frac{1}{n} \right) f(x) + \frac{g'(x) }{(\gamma-1)}= 0
\end{gathered}
\end{equation}
\begin{equation}
\label{diff_eq_g}
\begin{gathered}
g'(x) \left( f(x)-x \right) + 2 \left( 1 - \frac{1}{n} \right) g(x) + g(x)(\gamma-1) \left( \frac{2f(x)}{x} + f'(x) \right)= 0
\end{gathered}
\end{equation} \par
As these equations are ordinary differential equations, we do not have to worry about what kind of initial conditions to choose for $u, c^2$, as functions $f$ and $g$ can be solved only by the boundary conditions. To compute boundary conditions from those of $u,c^2$ given in equations (\ref{velocity_bc}), (\ref{speed_of_sound_bc}), we use definitions of $f,g$ in terms of $u,c^2$ (\ref{similarity_assumption}). However, for the assumed form of $c^2$ in (\ref{similarity_assumption}), boundary conditions cannot be satisfied. Instead, because $\dot{R}(t) \to -\infty$ as $R(t) \to 0$, in order to get finite speed of sounds at boundaries, we assume that $g$ is zero at the boundaries, if all we are interested in is the late stage of the collapse. 
\begin{equation}
\label{f_bc}
\begin{gathered}
f(1) = 1, \qquad \lim \limits_{x \to \infty} f(x) = 0
\end{gathered}
\end{equation}
\begin{equation}
\label{g_bc}
\begin{gathered}
g(1) = 0, \qquad \lim \limits_{x \to \infty} g(x) = 0
\end{gathered}
\end{equation}\par
The goal now is for each value of $\gamma$ describing the equation of state of the fluid to find the value of $n$ so that differential equations (\ref{diff_eq_f}), (\ref{diff_eq_g}) are satisfied with the appropriate boundary conditions (\ref{f_bc}), (\ref{g_bc}). 

\section{Results}
It is convenient to rewrite differential equations (\ref{diff_eq_f}), (\ref{diff_eq_g}) in the following way so that each equation contains derivative of only one of the functions. To do this, insert the expression for $g'(x)$ from \eqref{diff_eq_g} to (\ref{diff_eq_f}), or insert the expression for $f'(x)$ from \eqref{diff_eq_f} to \eqref{diff_eq_g}. 
\begin{equation}
\label{diff_eq_f_2}
\begin{gathered}
f'(x) \left( (f(x)-x)^2 - g(x) \right) = - \left( 1 - \frac{1}{n}\right) f(x)(f(x)-x) \\
+2 \left( 1 - \frac{1}{n} \right) \frac{g(x)}{(\gamma-1)} + \frac{2 f(x)g(x)}{x}
\end{gathered}
\end{equation}
\begin{equation}
\label{diff_eq_g_2}
\begin{gathered}
g'(x) \left( (f(x)-x)^2 - g(x) \right) = \left( 1 - \frac{1}{n} \right)(\gamma - 1)f(x) g(x) \\
- 2 \left( 1 - \frac{1}{n} \right) g(x)(f(x)-x) + 2 (\gamma - 1) f(x) g(x) \frac{(x - f(x))}{x}
\end{gathered}
\end{equation}
Notice that the differential equations (\ref{diff_eq_f_2}), (\ref{diff_eq_g_2}) are singular at $x = 1$ with the boundary conditions chosen as (\ref{f_bc}), (\ref{g_bc}). This means that at $x = 1$ it is not possible to solve for $f'(1), g'(1)$ which then would be used to 
numerically approximate values of $f,g$ for some $x > 1$. So, instead of solving equations numerically from $x = 1$, we start from $x = 1 + \varepsilon$, where $0 < \varepsilon \ll 1$. To do that, we have to understand what are the new boundary conditions 
at such point. To compute them, we expand both functions in $\varepsilon$ around $x =1$ as given next, where we use boundary conditions at $x = 1$ as in equations (\ref{f_bc}), (\ref{g_bc}). 
\begin{equation}
\begin{gathered}
f(1 + \varepsilon) = f(1) + f'(1) \varepsilon + O(\varepsilon^2) = 1 + f'(1) \varepsilon + O(\varepsilon^2)  \\ f'(1 + \varepsilon) = f'(1) + f''(1) \varepsilon + O(\varepsilon^2)
\end{gathered}
\end{equation}
\begin{equation}
\begin{gathered}
g(1 + \varepsilon) = g(1) + g'(1) \varepsilon + O(\varepsilon^2) =  g'(1) \varepsilon + O(\varepsilon^2) \\ g'(1 + \varepsilon) = g'(1) + g''(1) \varepsilon + O(\varepsilon^2)
\end{gathered}
\end{equation}
Consider differential equations (\ref{diff_eq_f_2}), (\ref{diff_eq_g_2}) up to first order in $\varepsilon$. As equations are singular at $x = 1$, no information is obtained from zeroth order in $\varepsilon$. However, from the first order in $\varepsilon$, it is possible to compute values of $f'(1), g'(1)$ and to approximate boundary conditions as follows.
\begin{equation}
\label{f_bc_2}
\begin{gathered}
f(1 + \varepsilon) = 1 + \frac{\left(3 - 2\left(1 - \frac{1}{n}\right) - 2\gamma \right)}{\gamma} \varepsilon + O(\varepsilon^2) \\
\approx 1 + \frac{\left(3 - 2\left(1 - \frac{1}{n}\right) - 2\gamma \right)}{\gamma} \varepsilon
\end{gathered}
\end{equation}
\begin{equation}
\label{g_bc_2}
\begin{gathered}
g(1 + \varepsilon) =  \left(1 - \frac{1}{n} \right)(1 - \gamma) \varepsilon + O(\varepsilon^2) \approx \left(1 - \frac{1}{n} \right)(1 - \gamma) \varepsilon
\end{gathered}
\end{equation}\par
For a given choice of $\gamma$, we search through values of $n$ and for each guess of $n$ we numerically integrate equations (\ref{diff_eq_f_2}), (\ref{diff_eq_g_2}) starting from $x = 1 + \varepsilon$, where boundary conditions (\ref{f_bc_2}), (\ref{g_bc_2})
are used, until $x = 5$. Maximum value of $x$ is chosen from practical considerations as then it is clear whether boundary conditions as $x \to \infty$ given in (\ref{f_bc}), (\ref{g_bc}) are satisfied or not. If they are, we report this value of $n$ as predicted
value for the power law exponent of the cavity wall's collapse. Value of $\varepsilon$ used in numerical calculations is $\varepsilon = 10^{-3}$. It was checked that if this value is taken to be smaller then results do not change significantly. For example, if $\gamma = 8$, then predicted value for $\varepsilon = 10^{-3}$ is $n = 0.540799$, and predicted value for $\varepsilon = 10^{-5}$ is $n = 0.540800$. As an example of a search of $n$ for $\gamma = 8$, consider results in figure \ref{figure_g_plots} which show how solutions of the function values $g(x)$ look like if $n$ is smaller, equal or greater than the predicted value of $n = 0.540799$. 

\begin{figure}[t]
	\centering
	\begin{subfigure}{.3\textwidth}
		\includegraphics[width=0.95\textwidth]{smaller_than_n.pdf}
		\caption{$n = 0.540699$}
	\end{subfigure}
%%%%%%%%%%%%%%
	\begin{subfigure}{.3\textwidth}
		\includegraphics[width=0.95\textwidth]{correct_n.pdf}
		\caption{$n = 0.540799$}
	\end{subfigure}
%%%%%%%%%%%%%%
	\begin{subfigure}{.3\textwidth}
		\includegraphics[width=0.95\textwidth]{larger_than_n.pdf}
		\caption{$n = 0.540899$}
	\end{subfigure}
	\caption{Numerically obtained solutions for $g(x)$ shown here from $x = 1+ \varepsilon$ until $x = 2$. In (a), the guessed value of $n$ is smaller than the predicted one. In (b), it is equal to the predicted value. In (c), it is larger than the predicted value.}
\label{figure_g_plots}
\end{figure} \par
First, we consider obtained results for $n$ as a function of $\gamma$, where range of $\gamma$ is close to water for which $\gamma = 7$. We predict that the majority of materials have $\gamma$ values close to the one of water, so for this range
we propose the following fit that might be useful for practical applications, $n = 0.4 + a\gamma^{-b}$, where $a = 0.657424$, $b = 0.735226$. For this range of $\gamma$, the results are shown in figure \ref{figure_n_water}. However, the described method can also be used to compute $n$ for large values of $\gamma$, results for which are given in figure \ref{figure_n_large}. We see that as $\gamma \to \infty$, values approach the incompressible limit $n = 0.4$ \cite{rayleigh_1917}.

\section{Conclusion}
The collapse of a bubble has been shown to be sensitive to the equation of the state of an outside fluid. We have solved for the power law exponent of the collapse $n$ as a function of compressibility described by $\gamma$ for a wide range of $\gamma$ values. From the obtained results, we see that in the late stages of collapse, the collapse is faster if $\gamma$ is larger. Therefore, the selection of materials with high $\gamma$ will facilitate the attainment of higher levels of focusing of energy density in a bubble.

\section{Acknowledgments}
This research has been funded by the DoD via HQ0034-20-1-0034 and the AFOSR under FA9550-22-1-0425 (20223965). We thank Steven Ruuth and Seth Pree for many valuable discussions.

% bibliography BibTex for arxiv:
\bibliographystyle{unsrt}
\bibliography{bubble_collapse_paper}
\end{document}

