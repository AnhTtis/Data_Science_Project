%% ****** Start of file apstemplate.tex ****** %
%%
%%
%%   This file is part of the APS files in the REVTeX 4.2 distribution.
%%   Version 4.2a of REVTeX, January, 2015
%%
%%
%%   Copyright (c) 2015 The American Physical Society.
%%
%%   See the REVTeX 4 README file for restrictions and more information.
%%
%
% This is a template for producing manuscripts for use with REVTEX 4.2
% Copy this file to another name and then work on that file.
% That way, you always have this original template file to use.
%
% Group addresses by affiliation; use superscriptaddress for long
% author lists, or if there are many overlapping affiliations.
% For Phys. Rev. appearance, change preprint to twocolumn.
% Choose pra, prb, prc, prd, pre, prl, prstab, prstper, or rmp for journal
%  Add 'draft' option to mark overfull boxes with black boxes
%  Add 'showkeys' option to make keywords appear
\documentclass[aps,prl,reprint,groupedaddress]{revtex4-1}
%\documentclass[aps,prl,preprint,superscriptaddress]{revtex4-2}
%\documentclass[aps,prl,reprint,groupedaddress]{revtex4-2}

\newcommand{\gtsim}{\mbox{{\raisebox{-0.4ex}{$\stackrel{>}{{\scriptstyle\sim}}$}}}}
\newcommand{\ltsim}{\mbox{{\raisebox{-0.4ex}{$\stackrel{<}{{\scriptstyle\sim}}$}}}}

\usepackage{graphicx}
\usepackage{dcolumn}
\usepackage{amsmath}
\usepackage{amssymb}
\usepackage{wasysym}
\usepackage{color}

\usepackage{mathtools}
\DeclarePairedDelimiter{\ceil}{\lceil}{\rceil}
\DeclarePairedDelimiter\floor{\lfloor}{\rfloor}


% You should use BibTeX and apsrev.bst for references
% Choosing a journal automatically selects the correct APS
% BibTeX style file (bst file), so only uncomment the line
% below if necessary.
%\bibliographystyle{apsrev4-2}

\begin{document}

% Use the \preprint command to place your local institutional report
% number in the upper righthand corner of the title page in preprint mode.
% Multiple \preprint commands are allowed.
% Use the 'preprintnumbers' class option to override journal defaults
% to display numbers if necessary
%\preprint{}

%Title of paper
\title{Thermodynamic evidence for electron correlation-driven flattening of the quasiparticle bands in the high-$T_{\rm c}$ cuprates %and implications for pairing
%Excess electronic entropy in the high-$T_{\rm c}$ cuprates and its implications for the flatness of the quasiparticle bands  and quantum criticality
%A universally strong renormalization of the thermodynamic effective mass in the high-$T_{\rm c}$ cuprates
%Quantum oscillations reveal the origin of the phase fluctuations constraining the superconducting transition temperature in the high-$T_{\rm c}$ cuprates
}

% repeat the \author .. \affiliation  etc. as needed
% \email, \thanks, \homepage, \altaffiliation all apply to the current
% author. Explanatory text should go in the []'s, actual e-mail
% address or url should go in the {}'s for \email and \homepage.
% Please use the appropriate macro foreach each type of information

% \affiliation command applies to all authors since the last
% \affiliation command. The \affiliation command should follow the
% other information
% \affiliation can be followed by \email, \homepage, \thanks as well.
\author{N.~Harrison and M.~K.~Chan}
%\email[]{Your e-mail address}
%\homepage[]{Your web page}
%\thanks{}
%\altaffiliation{}
%\affiliation{$^1$National High Magnetic Field Laboratory, Los Alamos National Laboratory, Los Alamos, NM 87545, USA}
\affiliation{Los Alamos National Laboratory, Los Alamos, NM 87545, USA}


%Collaboration name if desired (requires use of superscriptaddress
%option in \documentclass). \noaffiliation is required (may also be
%used with the \author command).
%\collaboration can be followed by \email, \homepage, \thanks as well.
%\collaboration{}
%\noaffiliation

\date{\today}

\begin{abstract}
A flattened electronic band is one of several possible routes for increasing the strength of the pairing interactions in a superconductor. With this in mind, we show here that thermodynamic measurements of the high-$T_{\rm c}$ cuprates reveal an appreciably stronger electron correlation-driven flattening of the quasiparticle bands than has previously been indicated. Specifically, we find that thermodynamic measurements indicate an electronic entropy in excess of that that can be accounted for by the value of the universal Fermi velocity inferred from photoemission experiments. The observed band flattening implies that the Van Hove singularity features prominently in calorimetry measurements, causing it undermine prior arguments for a divergence in the renormalization of the effective mass near a critical doping $p^\ast$ based on calorimetry measurements. The band flattening is also sufficient to drive the cuprates into a strong pairing regime where the maximum transition temperature becomes constrained by phase fluctuations. %
%
%
\end{abstract}

% insert suggested keywords - APS authors don't need to do this
%\keywords{}

%\maketitle must follow title, authors, abstract, and keywords
\maketitle
%
%


%\section{Introduction}
The degree of flatness of the electronic bands in a material has been shown to be a chief determining factor in the realization of correlated insulating states and superconducting states with strong pairing interactions~\cite{cao2018,park2021}. Correlated insulating states are well documented in the undoped parent phases of the high temperature superconducting cuprates~\cite{keimer2015,lee2006}. There are also a number of physical phenomena in hole-doped cuprates that are consistent with strong pairing interactions~\cite{baskaran1987,emery1995,timusk1999,hufner2008,harrison2022,uemura1989,uemura1991},  suggesting therefore a possible role of phase fluctuations in constraining the maximum value of the critical temperature $T_{\rm c}$~\cite{hazra2019,shi2022,quasi2D,ries2015,li2007,guo2020}. In a tight-binding lattice, phase fluctuations are predicted to cause the upper limit for the transition temperature to be given by $k_{\rm B}T_{\rm c}/t\sim$~0.2~\cite{denteneer1993,chen1999,keller2001,toschi2005,paiva2010,pasrija2016,paiva2004}, where $t$ is the in-plane hopping. Given the large value of $t\approx$~360~meV predicted by electronic structure theory in the cuprates~\cite{matteiss1987,massidda1988,singh1992,andersen1995}, a significant band renormalization is required for the upper limit of $T_{\rm c}$ to have been reached~\cite{keimer2015}. 

At present, our understanding of the effective mass renormalization in the cuprates relies heavily on angle-resolved photoemission spectroscopy measurements~~\cite{johnson2001,byczuki2007,lanzara2001}, to the extent that it is photoemission experimental results against which all other experimental results are routinely compared~\cite{padilla2005,sebastian2012,ramshaw2015,legros2022,michon2019,zhong2022,storey2008,doiron2007,sous2022}. Photoemission experiments have reported the nodal Fermi velocity to exhibit a universal value of $v_{\rm F}\approx$~2.7~$\times$~10$^5$~ms$^{-1}$~\cite{yoshida2003,damascelli2003}, which is roughly half the velocity $v_{\rm b}\approx$~5.1~$\times$~10$^5$~ms$^{-1}$ inferred from non-interacting electronic band theory~\cite{matteiss1987,massidda1988,singh1992,andersen1995}, and corresponds to a reduced hopping of $t\approx$~190~meV~\cite{horio2018,michon2019,hopping,drozdov2018,zhong2022}. Upon taking a simple ratio of velocities, ${v_{\rm b}}/{v_{\rm F}}$, one can infer a modest enhancement of the nodal photoemission effective mass $m^\ast$ relative to the band mass $m_{\rm b}$ of ${m^\ast}/{m_{\rm b}}\approx$~1.9. 
%

\begin{figure}
\begin{center}
\includegraphics[width=0.95\linewidth]{entropyschematic022323}
\textsf{\caption{{\bf a}, Schematic electronic density density of states for values of the nodal velocity (or $t$) according to band theory (grey), the universal velocity inferred from angle-resolved photoemission spectroscopy measurements (blue) and that we infer from thermodynamic measurements (red). {\bf b}, Schematic of the corresponding entropy versus temperature curves.
}
\label{bandentropy}}
\end{center}
\vspace{-0.7cm}
\end{figure}

In this paper, we show that there is a significant excess electronic entropy in thermodynamic measurements that is not accounted for by the universal velocity inferred from photoemission measurements~\cite{yoshida2003}. Moreover, rather than being restricted to a singular point~\cite{mathur1998,laughlin2001,sachdev2010,shibauchi2013,coleman2005}, as has been suggested~\cite{michon2019}, the excess entropy occurs over a wide range of hole dopings.  It is most clearly evidenced by the magnitude of the electronic entropy determined from calorimetry measurements extending up to room temperature~\cite{loram2001,loram1994,loram1998}. It is also evidenced by the small values of the orbitally-averaged Fermi velocity $v_{\rm F}^{\rm orb}$ obtained from quantum oscillation measurements made at low temperatures~\cite{rourke2010,ramshaw2015,tan2015,barisic2013,chan2016,kunisada2020,oliviero2022}. Finally, we show that it can account for recent low temperature calorimetry measurements~\cite{horio2018,michon2019} without invoking quantum critical fluctuations.  We find that the excess electronic entropy in all three thermodynamic measurements can be consistently understood by considering a significantly smaller value of the hopping of $t\approx$~107~meV (see illustration in Fig.~\ref{bandentropy}), corresponding to a reduced nodal Fermi velocity of $v_{\rm F}\approx$~1.5~$\times$~10$^5$~ms$^{-1}$. We further show that this reduced hopping increases the prominence of the Van Hove singularity in calorimetry measurements, thereby weakening arguments for a divergence in the effective mass enhancement at the critical doping $p^\ast$~\cite{michon2019}. We show that the reduced hopping is also sufficient to drive the pairing into a regime where phase fluctuations become relevant in constraining $T_{\rm c}$. 

Figures~\ref{entropy}, \ref{orbitalvelocity} and \ref{lowtcalorimetry} show the thermodynamic evidence for flatter quasiparticle bands arising from stronger electronic correlations in the cuprates. To assist in the interpretation of the experimental data, we first clarify the relationships between $t$ and the values of Sommerfeld coefficient $\gamma$, the Fermi velocity $v_{\rm F}$ and the entropy $S$. They can be understood by considering the standard tight binding approximation $\epsilon_{\bf k}=-2t(\cos ak_x+\cos ak_y)+4t^\prime\cos ak_x\cos ak_y-2t^{\prime\prime}(\cos2ak_x+\cos2ak_y)-\mu$ to the electronic dispersion~\cite{drozdov2018,tightbindingparameters}, where $t^\prime$ and $t^{\prime\prime}$ are higher order hopping parameters and $a$ is the in-plane lattice parameter. 
%
The magnitude of the nodal Fermi velocity is given by the gradient~\cite{ashcroft1976} 
\begin{equation}\label{fermivelocity}
v_{\rm F}=\Big|\frac{1}{\hbar}\nabla_{\bf k}\varepsilon_{\bf k}\Big|
\end{equation}
and by substituting $k_x=k_y=k_{\rm F}/\sqrt{2}$ for the nodal direction, where $k_{\rm F}$ here refers to the nodal Fermi wave vector. 
%
The Sommerfeld coefficient at a given temperature is obtained from the integral~\cite{ashcroft1976,harrison2022i} 
\begin{equation}\label{sommerfeld}
\gamma T={N_{\rm A}}\int_{-\infty}^\infty\varepsilon D(\varepsilon){\partial{\rm f}^\prime(\varepsilon-\mu)}{\rm d}\varepsilon
\end{equation}
over the product of the electronic density of states $D(\varepsilon)=\frac{2na^2}{\pi^2}\frac{\partial}{\partial\varepsilon_{\bf k}}\int^{\pi/a}_0|{\rm Re}(k_y)|{\rm d}k_x$ and the derivative in temperature ${\rm f}^\prime(\varepsilon-\mu)$ of the Fermi-Dirac distribution function. Here, $N_{\rm A}$ is Avogadro's number, $n$ is the number of CuO$_2$ planes, while $\mu$ is the chemical potential, which is adjusted to produce a band filling consistent with experiment.
%
Finally, the entropy is given by the integral 
\begin{equation}\label{entropyintegration}
S=\int_0^\infty\gamma{\rm d}T.
\end{equation}
As illustrated in Fig.~\ref{bandentropy}, a smaller $t$ produces a flatter quasiparticle band with a larger electronic density of states, for which the entropy rises towards the saturation value of $R\ln4$ more rapidly with increasing temperature (where $R=N_{\rm A}k_{\rm B}$ is the gas constant), yielding the approximate proportionalities $\gamma\propto{1}/{v_{\rm F}}\propto S\propto t$. 

\begin{figure}
\begin{center}
\includegraphics[width=0.95\linewidth]{entropy031723}
\textsf{\caption{{\bf a}, {\bf b}, {\bf c} and {\bf d}, $S$ per planar Cu versus $T$ for YBa$_2$Cu$_3$O$_{6+x}$ (YBCO) at $p=$~0.195, La$_{2-x}$Sr$_x$CuO$_4$ (LSCO) at $p=$~0.23 Bi$_2$Sr$_2$CaCu$_2$O$_{8-\delta}$ (Bi2212) at $p=$~0.19 and 20~\%~Ca-doped YBa$_2$Cu$_3$O$_{6+x}$ (Ca-YBCO) at $p=$~0.2. After Fig.~\ref{bandentropy}, blue and grey curves correspond to $S$ versus $T$ calculated using the universal velocity (from photoemission) and the band velocity, respectively (to avoid clutter, only the first two digits of $v_{\rm F}$ are indicated). For the red curves, $t$ is adjusted (yielding $v_{\rm F}$ values indicated) to match the experimental data; their average yields $v_{\rm F}=$~(1.5~$\pm$~0.2)~$\times$~10$^5$~ms$^{-1}$.
}
\label{entropy}}
\end{center}
\vspace{-0.7cm}
\end{figure}

\begin{figure}
\begin{center}
\includegraphics[width=0.95\linewidth]{orbitalvelocity022423}
\textsf{\caption{Orbitally-averaged $v_{\rm F}$ ($\circ$ symbols) for cuprates in which quantum oscillations have been observed, including overdoped Tl$_2$Ba$_2$CuO$_6$ (Tl2201)~\cite{rourke2010}, and underdoped YBa$_2$Cu$_3$O$_{6+x}$ (Y123)~\cite{ramshaw2015}, YBa$_2$Cu$_4$O$_8$ (Y124)~\cite{tan2015}, HgBa$_2$CuO$_{4+\delta}$ (Hg1201)~\cite{barisic2013,chan2016}, Ba$_2$Ca$_4$Cu$_5$O$_{10}$(F,O)$_2$ (Ba245)~\cite{kunisada2020} and HgBa$_2$Ca$_2$Cu$_3$O$_{8+\delta}$ (Hg1223)~\cite{oliviero2022}. These are compared against band theory (grey line), the photoemission universal velocity~\cite{yoshida2003} (blue line), and the thermodynamic $v_{\rm F}$ from Fig.~\ref{entropy} (red line). Also shown are $v_{\rm F}$ values determined from high resolution laser photoemission measurements~\cite{vishik2010,plumb2010,wreedhar2020} ($\diamond$ symbols).
}
\label{orbitalvelocity}}
\end{center}
\vspace{-0.7cm}
\end{figure}

\begin{figure}
\begin{center}
\includegraphics[width=0.95\linewidth]{vanhove031723}
\textsf{\caption{{\bf a}, Measured values of $\gamma$ versus $p$ at $T=$~0.5~K for  La$_{2-y-x}$Nd$_y$Sr$_x$CuO$_4$ (Nd-LSCO)~\cite{michon2019} (black $\circ$), La$_{2-y-x}$Eu$_y$Sr$_x$CuO$_4$ (Eu-LSCO) (black $\square$). 
%
These are compared against predictions from band theory (grey curve), the photoemission universal velocity (blue curve) and that corresponding to $v_{\rm F}$ from thermodynamic measurements averaged from Fig.~\ref{entropy} (red curve). To avoid clutter, only the first two digits of $v_{\rm F}$ are indicated. Dotted curves include the effect of $c$-axis hopping ($t_z=0.07t$~\cite{horio2018}). For the red curve, $v_{\rm F}$ (via $t$) is adjusted to account for the smooth variation of $S$ with $p$ observed at 300~K in LSCO~\cite{loram2001}. {\bf b}, Representation of the measured (solid black line) and extrapolated (dotted black line) $\gamma$ versus $\ln T$ curve in Eu-LSCO~\cite{michon2019}. Red, blue and grey lines show $\gamma$ versus $T$ at the Van Hove singularity ($p\approx$~0.23) using the same velocity values as in ({\bf a}).
}
\label{lowtcalorimetry}}
\end{center}
\vspace{-0.7cm}
\end{figure}

We turn first to the entropy as a function of temperature determined from calorimetry measurements extending from low temperature up to room temperature in Fig.~\ref{entropy}~\cite{loram2001,loram1994,loram1998}. The experimental evidence for flatter quasiparticle bands is directly apparent from the fact that the measured entropy $S$ curves (plotted in black) for slightly overdoped YBCO, LSCO, Bi2212 and Ca-YBCO lie significantly above those calculated (blue curves) using Equations~(\ref{fermivelocity}), (\ref{sommerfeld}) and (\ref{entropyintegration}) and using the universal $v_{\rm F}=$~2.7~$\times$~10$^5$~ms$^{-1}$ from photoemission experiments~\cite{yoshida2003}. 

Turning next to the results of magnetic quantum oscillation measurements in Fig.~\ref{orbitalvelocity}, more strongly renormalized quasiparticle bands are suggested by our finding that the orbitally-averaged values of the Fermi velocity $v^{\rm orb}_{\rm F}$ (circle symbols) fall significantly below the universal $v_{\rm F}$ (horizontal blue line) inferred from photoemission measurements~\cite{yoshida2003}. The orbitally-averaged velocity is estimated using $v^{\rm orb}_{\rm F}=\sqrt{2e\hbar F}/m^\ast$, where $k_{\rm F}^{\rm orb}=\sqrt{2eF/\hbar}$ is the orbitally-averaged Fermi radius, $F$ is the quantum oscillation frequency~\cite{shoenberg1984} and $m^\ast$ is the quasiparticle effective mass obtained from the temperature-dependence of the oscillations~\cite{rourke2010,ramshaw2015,tan2015,barisic2013,chan2016,kunisada2020,oliviero2022}. While some degree of reduction of $v_{\rm F}^{\rm orb}$ relative to $v_{\rm F}$ can be attributed to Fermi surface reconstruction and other factors~\cite{otherfactors}, this is not expected to be the case for the large unreconstructed Fermi surface of the overdoped cuprate Tl2212~\cite{rourke2010}. 

Finally, turning to recent low temperature calorimetry measurements in Fig.~\ref{lowtcalorimetry}, we find stronger electronic correlations to be revealed by the fact that the experimental values of $\gamma$ (black symbols) in Nd-LSCO and Eu-LSCO~\cite{michon2019} exceed those (solid blue curve) calculated using Equations~(\ref{fermivelocity}) and (\ref{sommerfeld}) assuming the universal $v_{\rm F}$ inferred from photoemission measurements~\cite{horio2018,michon2019,hopping,drozdov2018}. Note that while the hole doping $p^\ast\approx$~0.24 at which the normal state pseudogap has been reported to close coincides with a Van Hove singularity, a finite interlayer hopping inferred from photoemission measurements has been argued to cause the calculated values of $\gamma$ at the Van Hove singularity to be truncated (blue dotted line)~\cite{horio2018}. Quasiparticle scattering effects have been argued to further erode the Van Hove singularity~\cite{michon2019,horio2018}. 

What we find to be particularly striking from the entropy data in Fig.~\ref{entropy} is that the measured values at room temperature are already between 20\% and 30\% of the fully saturated entropy of $R\ln4$  for a half-filled electronic band. As illustrated in Fig.~\ref{bandentropy}, an electronic entropy that rises more quickly with temperature than expected is a clear indication of a significant part of the spectral weight in the electronic density of states having been shifted to lower energies by correlation effects~\cite{georges1996}. 
%
One way these results can be understood is by considering a Gutzwiller-like renormalization of the quasiparticle dispersion analogous to that considered in $^3$He~\cite{vollhardt1984}. From the significant fraction of $R\ln4$ and the continued presence of the large slopes in $S$ versus $T$ at room temperature in Fig.~\ref{entropy}, we can infer that the strongly renormalized portion of the quasiparticle bands must extend over a range in energy at least of order $\sim$~30~meV on either side of the chemical potential.

We find that all of the thermodynamic data in Figs.~\ref{entropy}, \ref{orbitalvelocity} and \ref{lowtcalorimetry} can be understood by considering a significantly reduced hopping of $t\approx$~107~$\pm$~10~meV, corresponding to a reduced nodal Fermi velocity of $v_{\rm F}=$~(1.5~$\pm$~0.2)~$\times$~10$^5$~ms$^{-1}$. Here, $v_{\rm F}=$~1.5~$\times$~10$^5$~ms$^{-1}$ is the value we obtain by taking an average of the values of $v_{\rm F}$ that are able to produce entropy curves (red curves) that overlap with the experimental data (black curves) for each of the cuprates in Fig.~\ref{entropy}. 

In Fig.~\ref{orbitalvelocity}, we find that the reduced Fermi velocity of $v_{\rm F}=$~1.5~$\times$~10$^5$~ms$^{-1}$ (red horizontal line) is in quantitative agreement with $v^{\rm orb}_{\rm F}$ measured in the overdoped cuprate Tl2212~\cite{rourke2010}. It also provides an upper bound for $v^{\rm orb}_{\rm F}$ values measured in the underdoped regime ($p\lesssim$~0.2), where the Fermi surface is reconstructed.

In Fig.~\ref{lowtcalorimetry}a, we find that $t=$~107~meV is able to produce a peak in $\gamma$ versus $p$ (red curve) that is consistent with the experimentally measured values of $\gamma$ at low temperatures for $p\gtrsim$~0.24. For $p\lesssim$~0.24, the calculated values of $\gamma$ exceed those observed experimentally. However, $p\lesssim$~0.24 is also where the pseudogap opens, and this is also accompanied by long-range stripe ordering~\cite{ma2021,collignon2017}; we make no attempt to include these effects in Equation~(\ref{lowtcalorimetry}). 

Finally, in Fig.~\ref{lowtcalorimetry}b we find that insertion of $t=$~107~meV into Equation~(\ref{sommerfeld}) produces a logarithmically diverging $\gamma$ at $p=$~0.24 whose slope in $\ln T$ matches that measured in Eu-LSCO (and Nd-LSCO) over a broad span in temperature~\cite{michon2019}; this is a simple consequence of the Van Hove singularity. For the other compositions that were measured at $p<$~0.24~\cite{michon2019}, we find Equation~(\ref{sommerfeld}) to yield values of $\gamma$ that are saturated at near constant values  below 10~K, which is consistent with the rapid loss of $\ln T$ behavior once $p\neq$~0.24~\cite{weak}. 

The reduced $t$ implies an increased relevance of the Van Hove singularity in the phase diagram. While quantum criticality has been indicated as an important factor at $p^\ast\approx$~0.23 in Eu-LSCO and Nd-LSCO~\cite{legros2019}, we find that the majority of the increase in $\gamma$ is caused by the increased density of states of the Van Hove singularity (see Fig.~\ref{bandentropy}). Thus, there is little to no additional renormalization of $\gamma$ in low temperature calorimetry measurements (in Fig.~\ref{lowtcalorimetry}) beyond that simply accounted for by using the smaller values of $t\approx$~107~meV estimated from entropy measurements~\cite{loram2001,loram1994,loram1998}. 

We further note that whilst quantum criticality has been advocated as an explanation for the logarithmic divergence in Fig.~\ref{lowtcalorimetry}b, the agreement between Equation~(\ref{sommerfeld}) and the experimental slope in $\gamma$ versus $\ln{T}$ implies that interlayer tunneling (and scattering) cannot be suppressing the Van Hove singularity to the extent that has been suggested~\cite{tunnelingreason,hossain2010,horio2018,markiewicz1997}. A very recent study of LSCO  provides supporting arguments~\cite{zhong2022} for both the interlayer hopping and the effects of scattering having been overestimated in low temperature calorimetry measurements~\cite{michon2019}. The increased density of states of the Van Hove singularity caused by the smaller $t$ must therefore cause it to feature prominently in determinations of the quasiparticle scattering rate within the normal state at $p^\ast$~\cite{pattnaik1992,newns1994,legros2019}.

The reduced $t$ also implies that $m^\ast/m_{\rm b}=$~3.4~$\pm$~0.3. This renormalization lies between that  $\approx$~2.8 of ambient pressure $^3$He~\cite{vollhardt1984} and that $\approx$~4.0 of Sr$_2$RuO$_4$~\cite{bergemann2003} (taking an average over its three Fermi surface sheets), against which the cuprates are often compared~\cite{lee2006,bergemann2003}. A possible clue as to the origin of the difference between thermodynamic measurements and universal velocity inferred from photoemission measurements is provided by recent high resolution laser photoemission experiments, which find evidence for band flattening within $\omega\approx$~10~meV of the Fermi surface~\cite{vishik2010,plumb2010,wreedhar2020} (where $\omega=|\varepsilon-\mu|$). Many of the values of the nodal Fermi velocity obtained from laser photoemission ($\diamond$ symbols in Fig.~\ref{orbitalvelocity}) are found to be in excellent agreement with the value $v_{\rm F}=$~1.5~$\times$~10$^5$~ms$^{-1}$ we obtain from thermodynamic measurements. Laser photoemission and thermodynamic measurements are therefore in agreement on the existence of quasiparticles residing within $\sim$~10~meV of the Fermi surface. However, since a band flattening restricted to $\omega<$~10~meV cannot account for the excess entropy extending to high temperatures in Figs.~\ref{entropy} and~\ref{lowtcalorimetry}, it implies that a significant fraction of the excitations contributing to thermodynamic measurements at $\omega\gtrsim$~10~meV must be incoherent in photoemission measurements.  
%
One possibility is that coherent quasiparticles contributing to thermodynamic measurements constitute novel forms of composite or fractionalized fermion for which the corresponding features in photoemission measurements are weak or heavily broadened~\cite{jain2009,coleman2001,punk2015,moon2011,rice2012}. 

Finally, the reduced $t$ implies that $k_{\rm B}T_{\rm c}/t\approx$~0.1, which means that the pairing is sufficiently strong for $T_{\rm c}$ to be in a regime where phase fluctuations start to become an important consideration~\cite{hazra2019,shi2022,baskaran1987,emery1995}. Because the maximum theoretical ratio $k_{\rm B}T_{\rm c}/t\approx$~0.2~\cite{denteneer1993,chen1999,keller2001,toschi2005,paiva2010,pasrija2016,paiva2004} is based on the superfluid density of the large unreconstructed Fermi surface, a reduction in the superfluid density of the magnitude  reported by Uemura {\it et al.}~\cite{uemura1991,uemura1989} will be more than sufficient to give rise to a Brezinksii-Kosterlitz-Thouless transition or drive the superconductivity into the Bose pairing regime~\cite{shi2022,quasi2D}. Possible causes include the proximity to an insulating state~\cite{rullieralbenque2008} or the reconstruction of the Fermi surface into small pockets~\cite{ramshaw2015,tan2015,barisic2013,chan2016,kunisada2020,oliviero2022}. 


\begin{acknowledgments}
The reduced Fermi velocity analysis was supported by the Department of Energy (DoE) BES project `Science of 100 tesla.' The entropy analysis was performed as part of a LDRD DR project at Los Alamos National Laboratory. The National High Magnetic Field Laboratory is funded by NSF Cooperative Agreements DMR-1157490 and 1164477, the State of Florida and DoE. NH thanks Arkady Shekhter for stimulating discussions. MKC acknowledges support from NSF IR/D program while serving at the National Science Foundation. Any opinion, findings, and conclusions or recommendations expressed in this material are those of the author(s) and do not necessarily reflect the views of the National Science Foundation.

\end{acknowledgments}







\bibliographystyle{naturemag}

\begin{thebibliography}{99}

\bibitem{cao2018} Cao, Y., Fatemi, V., Fang, S., Watanabe, K., Taniguchi, T., Takashi, E. and Jarillo-Herrero, P., Unconventional superconductivity in magic-angle graphene superlattices. {\it Nature} {\bf 556}, 43-50 (2018).

\bibitem{park2021} Park, J.~M., Cao, Y., Watanabe, K., Taniguchi, T., Jarillo-Herrero, P., Tunable strongly coupled superconductivity in magic-angle twisted trilayer graphene. {\it Nature} {\bf 590}, 249-255 (2021).

%

\bibitem{lee2006} Lee, P. A., Nagaosa, N., Wen, X.-G. Doping a Mott insulator: Physics of high-temperature superconductivity. {\it Rev. Mod. Phys.} {\bf 78}, 17-85 (2006).

\bibitem{keimer2015} Keimer, B.,  Kivelson, S. A., Norman, M. R., Uchida, S. Zaanen, J. From quantum matter to high-temperature superconductivity in copper oxides. {\it Nature} {\bf 518}, 179-186 (2015).

%

\bibitem{timusk1999} Timusk, T., Statt, B. The pseudogap in high-temperature superconductors: an experimental survey. {\it Rep. Prog. Phys.} {\bf 62}, 61-122 (1999).

\bibitem{hufner2008} H\"{u}fner, S., Hossain, M.A., Damascelli, A., Sawatsky, G.A., Two gaps make a high-temperature superconductor? {\it Rep. Prog. Phys.} {\bf 71}, 062501 (2008).

\bibitem{harrison2022} Harrison, N., and Chan, M. K., Magic gap ratio for optimally robust fermionic condensation and Its implications for high-$T_{\rm c}$ Superconductivity. {\it Phys. Rev. Lett.} {\bf 129}, 017001 (2022). 

%

\bibitem{baskaran1987} Baskaran, G., Zou, Z. and Anderson, P. W., The resonating valence bond state and high-$T_{\rm c}$ superconductivity -- a mean field theory. {\it Solid State Commun.} {bf 63}, 973 (1987). 

\bibitem{uemura1989} Uemura, Y. J., Luke, G. M., Sternlieb, Brewer, J. H., Carolan, J. F., Hardy, W. N., Kadono, R., Kempton, J. R., Kiefl, R. F., Kreitzman, S. R., Mulhern, P., Riseman, T. M., Williams, D. L., Yang, B. X., Uchida, S., Takagi, H., Gopalakrishnan, J., Sleight, A. W., Subamanian, M. A., Chien, C. L., Cieolak, M. Z., Xiao, G., Lee, V. Y., Statt, N. W., Stronach, C. E., Kossler, W. J., Yu, X. H., Universal correlations between $T_{\rm c}$ and $n_{\rm s}/m^\ast$ (carrier density over effective mass) in high-$T_{\rm c}$ cuprate superconductors. {\it Phys. Rev. Lett.} {\bf 62}, 2317-2320 (1989).

\bibitem{uemura1991} Uemura, Y. J., Le, L. P., Luke, G. M., Sternlieb, B. J., Wu, W. D., Brewer, J. H., Riseman, T. M., Seaman, C. L., Maple, M. B., Ishikawa, M., Hinks, D. G., Jorgensen, J. D., Saito, G. and Yamochi, H. Basic similarities among cuprate, bismuthate, organic, Chevrel-phase, and heavy-fermion superconductors shown by penetration-depth measurements. {\it Phys. Rev. Lett.} {\bf 66}, 2665 (1991). 

\bibitem{emery1995} Emery, V.J., Kivelson, S.A., Importance of phase fluctuations in superconductors with small superfluid density. {\it Nature} {\bf 374}, 434-437 (1995).

\bibitem{hazra2019} Hazra, T., Verma, N. and Randeria, M., Bounds on the superconducting transition temperature: applications to twisted bilayer graphene and cold atoms. {\it Physical Review X} {\bf 9}, 031049 (2019).

\bibitem{shi2022} Shi, T. T., Zhang, W. and de Melo, C. A. R. S., Density-induced BCS-Bose evolution in gated two-dimensional superconductors: The role of the interaction range in the Berezinskii-Kosterlitz-Thouless transition. {\it EPL}, {\bf 139}, 36003 (2022).

\bibitem{quasi2D} The theories are mostly for a two-dimensional lattice for which a Brezinksii-Kosterlitz-Thouless (BKT) transition is expected. A BKT has been reported in the cuprates~\cite{li2007,guo2020}, while experiments on a quasi-two-dimensional system have reported the transition temperature to be only slightly higher than that of an ideal two-dimensional system~\cite{ries2015}.

\bibitem{li2007} Li, Q. Hucker, M., Gu, G. D., Tsvelik, A. M. and Tranquada, J. M., Two-dimensional superconducting fluctuations in stripe-ordered La$_{1.875}$Ba$_{0.125}$CuO$_4$. {\it Phys. Rev. Lett.} {\bf 99}, 067001 (2007). 

\bibitem{guo2020} Guo, J., Zhou, Y.-Z., Huang, C., Cai, S., Sheng, Y. T., Gu, G. D., Yang, C. L., Lin, G. C., Yang, K. 

\bibitem{ries2015} Ries, M. G., Wenz, A. N., Z\"{u}rn, G., Bayha, L., Boettcher, I., Kedar, D., Murthy, P. A., Neidig, M., Lompe, T. and Jochim, S. Observation of pair condensation in the quasi-2D BEC-BCS crossover. {\it Phys. Rev. Lett.} {\bf 114}, 230401 (2015). 

%

%\bibitem{keller1999} Keller, M. and Metzner, W., Thermodynamics of a superconductor with strongly bound Cooper pairs. {\it Phys. Rev. B} {\bf 60}, 3499-3507 (1999}

\bibitem{denteneer1993} Denteneer, P. J. H., An, G. and van Leeuwen, J. M. J., Helicity modulus in the two-dimensional Hubbard model. {\it Phys. Rev. B} {\bf 47}, 6256-7272 (1993). 

\bibitem{chen1999} Chen, Q., Kosztin, I., Jank\'{o}, B. and Levin, K. Superconducting transitions from the pseudogap state: $d$-wave symmetry, lattice, and low-dimensional effects. {\it Phys. Rev. B} {\bf 59}, 7083-7093 (1999).

\bibitem{keller2001} Keller, M., Metzner, W. and Schollw\"{o}ck, U., {\it Phys. Rev. Lett.} {\bf 86}, 4612-4615 (2001). 

\bibitem{paiva2004} Paiva, T., dos Santos, R. R., Scalettar, R. T. and Denteneer, P. J. H., Critical temperature for the two-dimensional attractive Hubbard model. {\it Phys. Rev. B} {\bf 69}, 184501 (2004). 

\bibitem{toschi2005} Toschi, A., Barone, P., Capone, M. and Castellani, C., Pairing and superconductivity from weak to strong coupling in the attractive Hubbard model. {\it New J. Phys.} {\bf 7}, 7 (2005).

\bibitem{paiva2010} Paiva, T., Scalettar, R., Randeria, M. and Trivedi, N., Fermions in 2D optical lattices: temperature and entropy scales for observing antiferromagnetism and superfluidity. {\it Phys. Rev. Lett.} {\bf 104}, 066406 (2010). 

\bibitem{pasrija2016}Pasrija, K., Chakraborty, P. B. and Kumar, S., Effective Hamiltonian based Monte Carlo for the BCS to BEC crossover in the attractive Hubbard model. {\it Phys. Rev. B} {\bf 94} 165150 (2016).

%

\bibitem{matteiss1987} Mattheiss, L. F. Electronic band properties and superconductivity in La$_{2-y}X_y$CuO$_4$. {\it Phys. Rev,. Lett.} {\bf 58}, 1028 (1987). 

\bibitem{massidda1988} Massidda S., Yu, J. and Freeman, A. J. Electronic structure and properties of Bi$_2$Sr$_2$CaCu$_2$O$_8$, the third high-$T_{\rm c}$ superconductor. {\it Physica C} {\bf 152}, 251 (1988). 

\bibitem{singh1992} Singh, D. J. and Pickett, W. E. Electronic characteristics of Tl$_2$Ba$_2$CuO$_6$. Fermi surface, positron wavefunction, electric field gradients and transport parameters. {\it Physica C} {\bf 203}, 193 (1992). 

\bibitem{andersen1995} Andersen, O. K., Liechtenstein, A. I., Jepsen, O. and Paulsen, E. LDA energy bands, low-energy Hamiltonians, $t^\prime$, $t^{\prime\prime}$, $t_\perp(k)$, and $J_\perp$. {\it J. Phys. Chem. Solids} {\bf 56}, 1573 (1995). 

%

\bibitem{lanzara2001}  Lanzara, A., Bogdanov, P. V., Zhou, X. J., Kellar, S. A., Feng, D. L.,  Lu, E. D., Yoshida, T., Eisaki, H., Fujimori, A., Kishio, K.,  Shimoyama, J.-I., Noda, T., Uchida, S., Hussain Z. and Shen Z.-X., Evidence for ubiquitous strong electron–phonon coupling in high-temperature superconductors. {\it Nature} {\bf 412}, 510 (2001). 

\bibitem{johnson2001} Johnson, P. D., Valla, T., Fedorov, A. V., Yusof, Z. , Wells, B. O., Li, Q., Moodenbaugh, A. R., Gu, G. D., Koshizuka, N., Kendziora, C., Jian, S. and Jinks, D. G.., Doping and temperature dependence of the mass enhancement observed in the Cuprate Bi$_2$Sr$_2$CaCu$_2$O$_{8-\delta}$. {\it Phys. Rev. Lett.} {\bf }, (2001).

\bibitem{byczuki2007} Byczuki, K., Kollar, M., Held, K., Yang, Y.-F., Nekrasov, I. A., Pruschke, TH. and Vollhardt, D. Kinks in the dispersion of strongly correlated electrons. {\it Nature Phys.} {\bf 3}, 168 (2007). 

%

\bibitem{padilla2005} Padilla, W. J.,  Lee, Y. S., Dumm, M., Blumberg, G., Ono, S., Segawa, K.,  Komiya, S., Ando Y. and  Basov, D. N., Constant effective mass across the phase diagram of high-$T_{\rm c}$ cuprates. {\it Phys. Rev. B} {\bf 72}, 060511(R) (2005). 

\bibitem{doiron2007} Doiron-Leyraud, N., Proust, C. LeBoeuf, D., Levallois, J., Bonnemaison, B., Liang, R.-X. Hardy, W. N., Bonn, D. A., Taillefer, L. Quantum oscillations and the Fermi surface in an underdoped high-$T_{\rm c}$ superconductor. {\it Nature} {\bf 447}, 565-568 (2007).

\bibitem{storey2008} Storey, J. G., Tallon, J. L., and Williams, G. V. M., Thermodynamic properties of Bi$_2$Sr$_2$CaCu$_2$O$_8$ calculated from the electronic dispersion. {\it Phys. Rev. B} {\bf 77}, 052504 (2008).

\bibitem{sebastian2012} Sebastian, S.E., Harrison, N., Lonzarich, G.G. Towards resolution of the Fermi surface in underdoped high-$T_{\rm c}$ superconductors. {\it Rep. Prog. Phys.} {\bf 75}, 102501 (2012).

\bibitem{ramshaw2015} Ramshaw, B. J., Sebastian, S. E., McDonald, R. D.,  Day, J., Tan, B. S., Zhu, Z., Betts, J.B., Liang, R.-X., Bonn, D.A., Hardy, W.N., Harrison, N., Quasiparticle mass enhancement approaching optimal doping in a high-$T_{\rm c}$ superconductor. {\it Science} {\bf 348}, 317-320 (2015).

\bibitem{michon2019} Michon, B., Girod, C., Badoux, S., Ka\u{c}mar\u{c}\'{i}k, J., Ma, Q., Dragomir, M., Dabkowska, H. A., Gaulin, B.D., Zhou, J.-S., Pyon, S., Takayama, T., Takagi, H., Verret, S., Doiron-Leyraud, N., Marcenat, C., Taillefer, L., Klein, T., Thermodynamic signatures of quantum criticality in cuprate superconductors. {\it Nature} {\bf 567}, 218-222 (2019). 

\bibitem{legros2022} Legros, A., Post, K. W., Chauhan, P., Rickel, D. G., He, X., Xu, X., Shi, X., B\u{o}zovi\'{c}, I., Crooker, S. A. and Armitage, N. P., Evolution of the cyclotron mass with doping in La$_{2-x}$Sr$_x$CuO$_4$. {\it Phys. Rev. B} {\bf 106}, 195110 (2022).

\bibitem{zhong2022} Zhong, Y., Chen, Z., Chen, S.-D., Xu, K.-J., Hashimoto, N., He, Y., Mo, S.-K. and Shen, Z.-X., Differentiated roles of Lifshitz transition on thermodynamics and superconductivity in La$_{2-x}$Sr$_x$CuO$_4$. {\it Proc. Natl. Acad. USA} {\bf 119}, e2204630119 (2022). 

\bibitem{sous2022} Sous, J., He, Y. and Kivelson, S. A., Absence of a BCS-BEC crossover in the cuprate superconductors. arXiv:cond-mat/0610442 (2022).

%

\bibitem{yoshida2003} Zhou, X. J., Yoshida, T., Lanzara, A., Bogdanov, P. V., Kellar, S. A., Shen, K. M., Yang, W. L., Ronning, F., Sasagawa, T., Kakeshita, T., Noda, T., Eisaki, H., Uchida, S., Lin, C. T., Zhou, F., Xiong, J. W., Ti, W. X., Zhao, Z. X., Fujimori, A., Hussain, Z. and Shen, Z.-X., Universal nodal Fermi velocity. {\it Nature} {\bf 423}, 398 (2003). 

\bibitem{damascelli2003} Damascelli, A., Hussain, Z., Shen, Z.-X. Angle-resolved photoemission studies of the cuprate superconductors. {\it Rev. Mod. Phys.} {\bf 75}, 473-541 (2003).


%\bibitem{notableexceptions} A subset of photoemission experiments have reported smaller values of the Fermi velocity at lower energies~\cite{vishik2010,plumb2010}.
%
%\bibitem{plumb2010} Plumb, N. C., Reber, T. J., Koralek, J. D., Sun, Z., Douglas, J. F., Aiura, Y., Oka, K., Eisaki, H. and Dessau, D. S., Low-Energy ($<$10 meV) feature in the nodal electron self-energy and strong temperature dependence of the Fermi velocity in Bi$_2$Sr$_2$CaCu$_2$O$_{8+\delta}$. {\it Phys. Rev. Lett.} {\bf 105}, 046402 (2010).
%
%\bibitem{vishik2010} There is one notable exception; Vishik, I. M., Lee, W. S., Schmitt, F., Moritz, B., Sasagawa, T., Uchida, S., Fujita, K., Ishida, S., Zhang, C., Devereaux, T. P. and Shen Z. X., Doping-dependent nodal Fermi velocity of the high-temperature superconductor Bi$_2$Sr$_2$CaCu$_2$O$_{8-\delta}$ revealed using high-resolution angle-resolved photoemission spectroscopy. {\it Phys. Rev. Lett.} {\bf 104}, 207002 (2010). 

%\bibitem{supplementary} Supplementary Information. Includes Refs. \cite{vishik2010}.

%

\bibitem{horio2018} Horio, M., Hauser, K., Sassa, Y., Mingazheva, Z., Sutter, D., Kramer, K., Cook, A., Nocerino, E.,
Forslund, O. K., Tjernberg, O., Kobayashi, M., Chikina, A., Schr\"{o}ter, N. B. M., Krieger, J. A., 
Schmitt, T., Strocov, V. N., Pyon, S., Takayama, T., Takagi, H., Lipscombe, O. J., Hayden, S. M., Ishikado, M., Eisaki, H., Neupert, T.,  M\o{a}nsson, M., Matt, C. E. and Chang, J. Three-dimensional Fermi surface of overdoped La-based cuprates. {\it Phys. Rev. Lett.} {\bf 121}, 077004 (2018). 

\bibitem{hopping} For tight-binding electronic band, the nodal Fermi velocity at ${\bf k}=(\pi/2a,\pi/2a)$ is given by $v_{\rm F}=2\sqrt{2}at/\hbar$, where $a$ is the in-plane lattice parameter, for which we obtain $t\approx$~164~meV. However, because the Fermi surface does not coincide with ${\bf k}=(\pi/2a,\pi/2a)$~\cite{drozdov2018}, the hopping required to produce the experimental value is 190~meV~\cite{horio2018,michon2019}. 

\bibitem{drozdov2018} Drozdov, I . K., Pletikosi\'{c}, I., Kim, C.-K., Fujita, K., Gu, G. D., Davis, J. C. S., Johnson, P. D., Bo\u{z}ovi\'{c}, I. and Valla, T., Phase diagram of Bi$_2$Sr$_2$CaCu$_2$O$_{8+\delta}$ revisited. {\it Nature Commun.} {\bf 9}, 5210 (2018). 

%

\bibitem{mathur1998} Mathur, N. D., Grosche, F. M., Julian, S. R., Walker,, I. R., Freye, D. M., Haselwimmer, R. K. W. and Lonzarich, G. G., Magnetically mediated superconductivity in heavy fermion compounds. {\it Nature} {\bf 394}, 39-43 (1998). 

\bibitem{laughlin2001} Laughlin, R. B., Lonzarich, G. G., Monthoux, P. and Pines, D. The quantum criticality conundrum. {\it Adv. Phys.} {\bf 50}, 361-365 (2001)

\bibitem{sachdev2010} Sacdev, S. Where is the quantum critical point int he cuprate superconductors? {\it Phys. Status Solidi} {\bf B247}, 537-543 (2010). 

\bibitem{shibauchi2013} Shibauchi, T., Carrington, A. and Matsuda, Y., A quantum critical point lying beneath the superconducting dome in iron pnictides. {\it  Annual Review of Condensed Matter Physics} {\bf 5}, 113-135 (2013). 

\bibitem{coleman2005} Coleman, P. and Schofield, A. J., {\it Nature} {\bf 433}, 226-229 (2005). 

%

\bibitem{loram1994} Loram, J. W., Mirza, K. A., Cooper, J. R. Liang, W. Y. and Wade, J. M., Electronic specific heat of YBa$_2$Cu$_3$0$_{6+x}$ from 1.8 to 300 K. {\it J. Superconductivity} {\bf 7}, 243-249 (1994).

\bibitem{loram1998} Loram, J. W., Mirza, K. A., Cooper, J. R., Tallon, J. L. Specific heat evidence of the normal state pseudogap. {\it J. Phys. Chem. Solids} {\bf 59}, 2091-2094 (1998).

\bibitem{loram2001} Loram, J. W., Luo, J., Cooper, J. R., Liang, W. Y., Tallon, J. L., Evidence on the pseudogap and condensate from the electronic specific heat, {\it J. Physics and Physical Chemistry of Solids} {\bf 62}, 59-64 (2001).

%

\bibitem{rourke2010} Rourke, P. M. C., Bangura, A. F., Benseman, T. M., Matusiak, M., Cooper, J. R., Carrington, A., Hussey, N. E., A detailed de Haas–van Alphen effect study of the overdoped cuprate Tl$_2$Ba$_2$CuO$_{6+\delta}$, {\it New J. Phys.} {\bf 12}, 105009 (2010).

\bibitem{barisic2013} Bari\u{s}i\'{c}, N., Badoux, S., Chan, M. K., Dorow, C., Tabis, W., Vignolle, B., Guichan, Y., Beard, J., Zhao, X., Proust, C. Universal quantum oscillations in the underdoped cuprate superconductors, {\it Nature Phys.} {\bf 9}, 761-764 (2013).

\bibitem{tan2015} Tan, B. S. , Harrison, N., Zhu, Z., Balakirev, F., Ramshaw, B. J., Srivastava, A., Sabok-Sayr, S. A., Dabrowski, B., Lonzarich, G. G. and Sebastian, S. E. Fragile charge order in the nonsuperconducting ground state of the underdoped high-temperature superconductors. {\it Proc. Natl. Acad. USA} {\bf 112} 9568 (2015). 

\bibitem{chan2016} Chan, M. K., Harrison, N., McDonald, R. D., Ramshaw, B. J., Modic, K. A., Bari\u{s}i\'{c}, N. and Greven, M., Single reconstructed Fermi surface pocket in an underdoped single-layer cuprate superconductor. {\it Nature Commun.} {\bf 7}, 12244 (2016). 

\bibitem{kunisada2020} Kunisada, S., Isono, S., Kohama, Y., Sakai, S., Bareille, C., Sakuragi, S., Noguchi, R., Kurokawa, K., Kuroda, K., Ishida, Y., Adachi, S., Sekine, R., Kim, T. K., Cacho, C., Shin, S., Tohyama, T., Tokiwa, K., Kondo, Y., Observation of small Fermi pockets protected by clean CuO$_2$ sheets of a high-$T_{\rm c}$ superconductor. {\it Science} {\bf 369}, 833-838 (2020).

\bibitem{oliviero2022} Oliviero, V., Benhabib, S., Gilmutdinov, I., Vignolle, B., Drigo, L., Massoudzadegan, M.,  Leroux, M., Rikken, G. L. J. A., Forget, A., Colson, D., Vignolles, D. and Proust, C. Magnetotransport signatures of antiferromagnetism coexisting with charge order in the trilayer cuprate HgBa$_2$Ca$_2$Cu$_3$O$_{8+\delta}$. {\it Nature Commun.} {\bf 13}, 1568 (2022).

%

\bibitem{tightbindingparameters} We have used $\epsilon_{\bf k}=-2t(\cos k_x+\cos k_y)+4t^\prime\cos k_x\cos k_y-2t^{\prime\prime}(\cos2k_x+\cos2k_y)-\mu$ for the tight-binding dispersion where we have used $t^\prime/t=$~$-$~0.32, $-$~0.12, $-$~0.30, $-$~0.32 and $-$~0.14, and $t^{\prime\prime}/t^\prime=$~$-$~0.50, $-$~0.50, $-$~0.33, $-$~0.5, and $-$~0.5, for YBCO, LSCO Bi2210, Ca-YBCO, and Nd- and Eu-LSCO, respectively~\cite{matteiss1987,andersen1995,drozdov2018,horio2018}. The electronic density of states is obtained using $D(\varepsilon_{\bf k})=\frac{n}{\pi^2}\frac{\partial}{\partial\varepsilon_{\bf k}}\int^\pi_0k_y(\varepsilon_{\bf k}){\rm d}k_x$.

%

\bibitem{ashcroft1976} Ashcroft, N. W. and Mermin, N. D. {\it Solid state physics} (Saunders College Publishing, Orlando 1976).

%

\bibitem{harrison2022i} Wartenbe, M., Tobash, P. H., Singleton, J., Winter, L. E., Richmond, S. and Harrison, N., Pseudogap in elemental plutonium. {\it Phys. Rev. B} {\bf 105}, L041107 (2022).

%

\bibitem{shoenberg1984} Shoenberg, D. {\it Magnetic Oscillations in Metals} (Cambridge Univ. Press, 1984).

%

\bibitem{otherfactors} Other factors could include a Lifshitz transition, a Van Hove singularity occurring at a different doping in other cuprates, or quantum criticality associated with the phase that reconstructs the Fermi surface. 

%

\bibitem{georges1996} Georges, A., Kotliar, G., Krauth, W. and Rozenberg, M. J., Dynamical mean-field theory of strongly correlated fermion systems and the limit of infinite dimensions. {\it Rev. Mod. Phys.} {\bf 68}, 13-125 (1996).

%

\bibitem{vollhardt1984} Vollhardt, D., Normal $^3$He: an almost localized Fermi liquid. {\it Rev. Mod. Phys.} {\bf 56}, 99 (1984).

%

\bibitem{ma2021} Ma, Q., Rule, K. C., Cronkwright, Z. W., Dragomir, M., Mitchell, G., Smith, E. M., Chi, S., Kolesnikov, A. I., Stone, M. B., Gaulin, B. D., Parallel spin stripes and their coexistence with superconducting ground states at optimal and high doping in La$_{1.6-x}$Nd$_{0.4}$Sr$_x$CuO$_4$. {\it Phys. Rev. Research} {\bf 3}, 023151 (2021).

\bibitem{collignon2017} Collignon, C., Badoux, S., Afshar, S. A. A., Michon, B.,  Lalibert\'{e}, F., Cyr-Choini\`{e}re, O., Zhou, J.-S., Licciardello, S., Wiedmann, S., Doiron-Leyraud, N. and Taillefer, L. Fermi-surface transformation across the pseudogap critical point of the cuprate superconductor La$_{1.6-x}$Nd$_{0.4}$Sr$_x$CuO$_4$. {\it Phys. Rev. B} {\bf 95}, 224517 (2017).

%

\bibitem{weak} We speculate that residual weaker slopes in $\ln T$ at $p=$~0.21 and $p=$~0.22 could be the consequence of slight sample inhomogeneity. In our calculations, we find that for $p=$~0.21 and $p=$~0.22, $\gamma$ versus $T$ exhibits a kink at $T>$~10~K, above which the slope in $\ln T$ is similar to that at $p=$~0.24. However, there is no experimental data in this region. Finally, $p<$~0.24 is where the pseudogap is argued to open, whose effect in reducing $\gamma$ we make no attempt to model.

%

\bibitem{tunnelingreason} This could be a consequence of the assumed form of the interlayer dispersion being poorly representative of that in experiments~\cite{horio2018}, the presence of higher harmonics in the hopping, or a suppression of the interlayer hopping caused by electronic correlations or quantum criticality~\cite{hossain2010}. Other photoemission measurements, for example, have indicated that the Van~Hove singularity is more extended in momentum-space~\cite{markiewicz1997}.

\bibitem{hossain2010} Fournier, D., Levy, G., Pennec, Y., McChesney, J. L., Bostwick, A., Rotenberg, E., Liang, R., Hardy, W. N., Bonn, D. A., Elfimov, I. S. and Damascelli, A., Loss of nodal quasiparticle integrity in underdoped YBa$_2$Cu$_3$O$_{6+x}$. {\it Nature Phys.} {\bf 6}, 905-911 (2010).

\bibitem{markiewicz1997} Markiewicz, R. S., A Survey of the Van Hove scenario for high-$T_{\rm c}$ superconductivity with special emphasis on pseudogaps and striped phases. {\it J. Phys. Cha,. Solids} {\bf 58}, 1179-1310 (1997).

%

\bibitem{legros2019} Legros, A., Benhabib, S., Tabis, W., Lalibert\'{e}, F., Dion, M., Lizaire, M., Vignolle, B., Vignolles, D., Raffy, H., Li, Z. Z., Auban-Senzier, P., Doiron-Leyraud, N., Fournier, P., Colson, D., Taillefer, L. and Proust, C., Universal $T$-linear resistivity and Planckian dissipation in overdoped cuprates. {\it Nature Phys.} {\bf 15}, 142-147 (2019). 

%

\bibitem{pattnaik1992} Pattnaik, P. C., Kane, C. L., Newns, D. M. and Tsuei, C. C., Evidence for the van Hove scenario in high-temperature superconductivity from quasiparticle-lifetime broadening. {\it Phys. Rev. B} {\bf 45}, 5714-5717 (1992). 

\bibitem{newns1994} Newns, D. M., Tsuei, C. C., Huebener, R. P., van Bentum, P. J. M., Pattnaik, P. C. and Chi, C. C. Quasiclassical Transport at a van Hove Singularity in Cuprate Superconductors. {\it Phys. Rev. Lett.} {\bf 73}, 1695-1698 (1994). 

%

\bibitem{bergemann2003} Bergemann, C., Mackenzie, A. P., Julian, S. R., Forsythe, D. and Ohmichi, E., Quasi-two-dimensional Fermi liquid properties of the unconventional superconductor Sr$_2$RuO$_4$. {\it Adv. Phys.} {\bf 52}, 639 (2003).

%

\bibitem{vishik2010} Vishik, I. M., Lee, W. S., Schmitt, F., Moritz, B., Sasagawa, T., Uchida, S., Fujita, K., Ishida, S., Zhang, C., Devereaux, T. P. and Shen, Z. X., Doping-dependent nodal Fermi velocity of the high-yemperature superconductor Bi$_2$Sr$_2$CaCu$_2$O$_{8+\delta}$ revealed using high-resolution angle-resolved
photoemission spectroscopy. {\it Phys. Rev. Lett.} {\bf 104}, 207002 (2010). 

\bibitem{plumb2010} Plumb, N. C., Reber, T. J., Koralek, J. D., Sun, Z., Douglas, J. F., Aiura, Y., Oka, K., Eisaki, H. and Dessau, D. S., Low-energy ($<$10 meV) feature in the nodal electron self-energy and strong temperature
dependence of the Fermi velocity in Bi$_2$Sr$_2$CaCu$_2$O$_{8+\delta}$. {\it Phys. Rev. Lett.} {\bf 105}, 046402 (2010). 

\bibitem{wreedhar2020} Sreedhar, S. A., Rossi, A., Nayak, J., Anderson, Z. W., Tang, Y., Gregory, B., Hashimoto, M., Lu, D.-H., Rotenberg, E., Birgeneau, R. J., Greven, M., Yi, M. and Vishik, I. M., Three interaction energy scales in the single-layer high-$T_{\rm c}$ cuprate HgBa$_2$CuO$_{4+\delta}$. {\it Phys. Rev. B} {\bf 102}, 205109 (2020).

%

\bibitem{jain2009} Jain, J. K., and Anderson, P. W. Beyond the Fermi liquid paradigm: hidden Fermi liquids. {\it Proc. Natl. Acad. Sci. USA} {\bf 106}, 9131-9134 (2009).
%
\bibitem{coleman2001} Coleman, P., P\'{e}pin, C., Si, Q., Ramazashvili, R. How do Fermi liquids get heavy and die? {\it J. Phys.: Condens. Matter} {\bf 13}, R723 (2001).
%
\bibitem{punk2015} Punk, M., Allais, A. and Sachdev, S., Quantum dimer model for the pseudogap metal. {\it Proc. Natl. Acad. USA} {\bf 112}, 9552-9557 (2015).
%
%%
%
\bibitem{moon2011} Moon, E. G., and Sachdev, S., Underdoped cuprates as fractionalized Fermi liquids: Transition to superconductivity. {\it Phys. Rev. B} {\bf 83}, 224508 (2011).
%
\bibitem{rice2012} Rice, T.M., Yang, K.-Y., Zhang, F.C., A phenomenological theory of the anomalous pseudogap phase in underdoped cuprates. {\it Rep. Prog. Phys.} {\bf 75}, 016502 (2012).



%\bibitem{pan2007} Pan, Z.-H., Richard, P., Fedorov, A. V., Kondo, T., Takeuchi, T., Li, S. L., Dai, P., Gu, G. D., Ku, W., Wang, Z. and Ding, H., Universal quasiparticle decoherence in hole- and electron-doped high-$T_{\rm c}$ cuprates. arXiv:cond-mat/0610442 (2006).
%
%\bibitem{chang2007} Chang, J., Pailh\'{e}s, S., Shi, M., M\o{a}nsson, M., Claesson, T., Tjernberg, O., Voigt, J., Perez, V., Patthey, L., Momono, N., Oda, M., Ido, M., Schnyder, A., Mudry, C. and Mesot, J., When low- and high-energy electronic responses meet in cuprate superconductors. {\it Phys. Rev. B} {\bf 75}, 224508 (2007). 
%
%\bibitem{graf2007} Graf, J., Gweon, G.-H., McElroy, K., Zhou, S. Y., Jozwiak, C., Rotenberg, E., Bill, A., Sasagawa, T., Eisaki, H., Uchida, S., Takagi, H., Lee, D.-H. and Lanzara, A. Universal high energy anomaly in the angle-resolved photoemission spectra of high temperature superconductors: possible evidence of spinon and holon branches. {\it Phys. Rev. Lett.} {\bf 98}, 067004 (2007).
%
%\bibitem{graf2007i} Graf, J., Gweon, G.-H. and Lanzara, A., Universal waterfall-like feature in the spectral function
%of high temperature superconductors. {\it Physica C} {\bf 460-462}, 194-107 (2007).
%
%%
%
%\bibitem{zhu2008} Zhu, L., Aji, V., Shekhter, A. and Varma, C. M., Universality of single-particle spectra of cuprate superconductors. {\it Phys. Rev. Lett.} {\bf 100}, 057001 (2008). 

%

%\bibitem{electronphonon} In phonon-mediated BCS superconductors, the effective mass is renormalized by a factor $(1+\lambda_{\rm ph})$~\cite{engelsberg1970}, so that $m^\ast/m_{\rm b}\geq(1+\lambda_{\rm ph})$. Detailed studies of BCS superconductors have found that $\lambda_{\rm ph}$ lies in the range 0.4~$\lesssim\lambda_{\rm ph}\lesssim$~3~\cite{carbotte1990}, implying that 1.4~$\lesssim m^\ast/m_{\rm b}\lesssim$~4. 
%
%\bibitem{engelsberg1970} Engelsberg, S. and Simpson, G., Influence of electron-phonon interactions on the de Haas-van Alphen effect. {\it Phys. Rev. B} {\bf 2}, 1657 (1970).
%
%\bibitem{carbotte1990} Carbotte, J. P., Properties of boson-exchange superconductors. {\it Rev. Mod. Phys.} {\bf 62}, 1027-1157 (1990).

%

%\bibitem{sebastian2010} Sebastian, S. E., Harrison, N., Altarawneh, M. M., Liang, R., Bonn, D. A., Hardy, W. N. and Lonzarich, G. G., Fermi-liquid behavior in an underdoped high-$T_{\rm c}$ superconductor. {\it Phys. Rev. B} {\bf 81}, 140505(R) (2010). 

%


%
%%
%
%
%%
%
%\bibitem{hossain2008} Hossain, M. A., Mottershead, J. D., F., Fournier, D., Bostwick, A., McChesney, J. L., Rotenberg, E., Liang, R., Hardy, W. N., Sawatzky, G. A., Elfimov, I. S., Bonn, D. A. and Damascelli, A., In situ doping control of the surface of high-temperature superconductors. {\it Nature Phys.} {\bf 4}, 527-531 (2008). 
%
%%
%
%\bibitem{vishik2014} Vishik, I. I., Bari\u{s}i\'{c}, N., Chan, M. K., Li, Y., Xia, D. D., Yu, G., Zhao, X., Lee, W. S., Meevasana, W., Devereaux, T. P., Greven, M. and Shen, Z.-X., Angle-resolved photoemission spectroscopy study of HgBa$_2$CuO$_{4+\delta}$. {\it Phys. Rev. B} {\bf 89}, 195141 (2014). 

%

%\bibitem{exception} A notable exception is the file-layer cuprates Ba245~\cite{kunisada2020}, which exhibits long-range antiferromagnetic order. 

%


%

%\bibitem{eliashberg1960} Eliashberg G.M., Interactions between electrons and lattice vibrations in a superconductor. {\it Soviet Phys. JETP} {\bf 11}m 696 (1960).


%


%

\bibitem{rullieralbenque2008} Rullier-Albenque, F., Alloul, H., Balakirev, F., Proust, C., Disorder, metal-insulator crossover and phase diagram in high-$T_{\rm c}$ cuprates. {\it EPL} {\bf 81}, 37008 (2008).




%\bibitem{assumption} This is using the value of $\gamma\approx$~7.2~mJmol$^{-1}$K$^{-2}$ for $t=$~190~meV according to Michon {\it et al}~\cite{michon2019}, from which we estimate $\gamma\approx$~4~mJmol$^{-1}$K$^{-2}$ for the non-interacting electronic structure.

%


%






%

%\bibitem{botelho2006} Botelho, S. S. and S\'{a} de Melo, C. A. R., Vortex-antivortex lattice in ultracold fermionic gases. {\it Phys. Rev. Lett.} {\bf 96}, 040404 (2006). 

%\bibitem{BKT} Larger values of the ratio $k_{\rm B}T_{\rm c}/\varepsilon_{\rm F}$ than $\frac{1}{8}$ suggest either that interlayer coherence length is sufficiently long for the system to be three-dimensional~\cite{hazra2019} or that the electronic band deviates strongly from the parabolic form used to define a BKT transition~\cite{botelho2006}. 

%\bibitem{supplementary} See Supplementary Information.


%
%
%\bibitem{fournier2010} Fournier, D., Levy, G., Pennec, Y., McChesney, J. L., Bostwick, A., Rotenberg, E., Liang, R., Hardy, W. N., Bonn, D. A., Elfimov, I. S. and Damascelli, A., Loss of nodal quasiparticle integrity in underdoped YBa$_2$Cu$_3$O$_{6+x}$. {\it Nature Phys.} {\bf 6}, 905-911 (2010).
%
%
%
%
%
%
%
%
%
%
%
%
%
%%
%
%\bibitem{norman1999} Norman, M. R.,  Ding, H., Fretwell, H., Randeria, M., and Campuzano, J. C., Extraction of the electron self-energy from angle-resolved photoemission data: Application to Bi$_2$Sr$_2$CaCu$_2$O$_{8+x}$. {\it Phys. Rev. B} {\bf 60}, 7585-7590 (1999). 
%
%%
%
%\bibitem{helm2015} Helm, T., Kartsovnik, M. V., Proust, C., Vignolle, B., Putzke, C., Kampert, E., Sheikin, I., Choi, E.-S., Brooks, J. S., Bittner, N., Biberacher, W., Erb, A., Wosnitza, J. and Gross, R., Correlation between Fermi surface transformations and superconductivity in the electron-doped high-$T_{\rm c}$ superconductor Nd$_{2-x}$Ce$_x$CuO$_4$. {\it Phys. Rev. B} {\bf 92}, 094501 (2015).
%
%%
%
%
%
%
%
%
%
%
%
%
%
%\bibitem{sebastian2010i} Sebastian, S. E., Harrison, N., Altarawneh, M. M., Mielke, C. H., Liang, R., Bonn, D. A., Hardy, W. N., Lonzarich, G. G., Metal-insulator quantum critical point beneath the high $T_{\rm c}$ superconducting dome. {\it Proc. Natl. Acad. USA} {\bf 107} 6175 (2010).
%
%
%%
%
%
%\bibitem{yelland2008} Yelland, E. A., Singleton, J., Mielke, C. H., Harrison, N., Balakirev, F. F., Dabrowski, B., Cooper, J. R. Quantum oscillations in the underdoped cuprate YBa$_2$Cu$_4$O$_8$. {\it Phys. Rev. Lett.} {\bf 100}, 047003 (2008).
%
%\bibitem{vignolle2008} Vignolle, B., Carrington, A., Cooper, R. A., French, M. M. J., Mackenzie, A. P., Jaudet, C., Vignolles, D., Proust, C. and Hussey, N. E. Quantum oscillations in an overdoped high-$T_{\rm c}$ superconductor. {\it Nature} {\bf 455}, 952 (2008).  
%
%\bibitem{helm2009} Helm, T., Kartsovnik, M. V., Bartkowiak, M., Bittner, N., Lambacher, M., Erb, A., Wosnitza, J. and Gross, R. Evolution of the Fermi surface of the electron-doped high-temperature superconductor Nd$_{2-x}$Ce$_x$CuO$_4$ revealed by Shubnikov–de Haas oscillations. {\it Phys. Rev. Lett.} {\bf 103}, 157002 (2009).
%
%
%\bibitem{breznay2016} Breznay, N. P., Hayes, I. M., Ramshaw, B, J., McDonald, R. D., Krockenberger, Y., Ikeda, A., Irie, H., Yamamoto, H. and Analytis, J. G. Shubnikov-de Haas quantum oscillations reveal a reconstructed Fermi surface near optimal doping in a thin film of the cuprate superconductor Pr$_{1.86}$Ce$_{0.14}$CuO$_{4\pm\delta}$. {\it Phys. Rev. B} {\bf 94}, 104514 (2016).
%
%\bibitem{higgins2018} Higgins, J. S., Chan, M. K., Sarkar, T., McDonald, R. D., Greene, R. L. and Butch, N. P. {New J. Phys.} {\bf 20}, 043019 (2018).
%
%
%
%%
%
%\bibitem{girod2021} Girod, C., LeBoeuf, D., Demuer, A., Seyfarth, G., Imajo, S., Kindo, K., Kohama, Y., Lizaire, M., Legros, A., Gourgout, A., Takagi, H., Kurosawa, T., Oda, M., Momono, N., Chang, J., Ono, S., Zheng, G.-Q., Marcenat, C., Taillefer, L. and Klein, T. Normal state specific heat in the cuprate superconductors 
%La$_{2-x}$Sr$_x$CuO$_4$ and Bi$_{2+y}$Sr$_{2-x-y}$La$_x$CuO$_{6+\delta}$ near the critical point of the pseudogap phase. {\it Phys. Rev. B} {\bf 103}, 214506 (2021). 
%
%%
%
%\bibitem{wade1994} Wade, J. M., Loram, J. W., Mirza, K. A., Cooper, J. R., Tallon, J. R., Electronic specific heat of Tl$_2$Ba$_2$CuO$_{6+\delta}$ from 2~K to 300~K for 0~$\leq\delta\leq$~0.1. {\it J. Superconductivity} {\bf 7}, 261-264 (1994).
%
%
%%
%
%\bibitem{holedopedcuprates} The hole doped cuprates include La$_{2-x}$Sr$_x$CuO$_4$, Bi$_2$Sr$_2$CaCu$_2$O$_8$ (Bi2212),  Bi$_2$Sr$_2$CuO$_6$ (Bi221), (Ca$_{2-x}$Na$_x$)CuOCl$_2$ (Na-CCOC), and La$_{2-y-x}$Nd$_y$Sr$_x$CuO$_4$ (Nd-LSCO)~(LSCO)~\cite{padilla2005}, La$_{2-y-x}$Eu$_y$Sr$_x$CuO$_4$ (Eu-LSCO)~\cite{michon2019}, YBa$_2$Cu$_3$O$_{6+x}$ (Y123)~\cite{doiron2007}, YBa$_2$Cu$_4$O$_8$ (Y124)~\cite{yelland2008}, Tl$_2$Ba$_2$CuO$_6$ (Tl2201)~\cite{vignolle2008}, HgBa$_2$CuO$_{4+\delta}$ (Hg1201)~\cite{barisic2013}, Ba$_2$Ca$_4$Cu$_5$O$_{10}$(F,O)$_2$ (Ba245)~\cite{kunisada2020} and HgBa$_2$Ca$_2$Cu$_3$O$_{8+\delta}$ (Hg1223)~\cite{oliviero2022}.
%
%%
%
%
%\bibitem{sebastian2014} Sebastian, S. E., Harrison, N., Balakirev, F. F., Altarawneh, M. M., Goddard, P. A., Liang, R., Bonn, D. A., Hardy, W. N. and Lonzarich, G. G., Normal-state nodal electronic structure in underdoped high-$T_{\rm c}$ copper oxides. {\it Nature} {\bf 511},  61 (2014).
%
%%
%
%
%%
%
%
%%
%
%\bibitem{electrondopedcuprates} The electron-doped cuprates are Nd$_{2-x}$Ce$_x$CuO$_4$ (NCCO)~\cite{helm2009}, Pr$_{2-x}$Ce$_x$CuO$_{4\pm\delta}$ (PCCO)~\cite{breznay2016} and La$_{2-x}$Ce$_x$CuO$_{4\pm\delta}$ (LCCO)~\cite{higgins2018}.
%
%%%%%%%%%%%%%%%%%%%%%%%%
%
%\bibitem{norman2010} Norman, M. R. and Millis, A. J., Lifshitz transition in underdoped cuprates. {\it Phys. Rev. B} {\bf 81}, 180513(R) (2010).
%
%\bibitem{leboeuf2011} LeBoeuf, D., Doiron-Leyraud, N., Vignolle, B., Sutherland, M., Ramshaw, B, J., Levallois, J., Lalibert\'{e}, F., Cur-Choini\`{e}re, Chang, J., Jo, Y. J., Balicas, L., Liang, R., Bonn, D. A., Hardy, W. N., Proust, C., Taillefer, L., Lifshitz critical point in the cuprate superconductor YBa$_2$Cu$_3$O$_y$ from high-field Hall effect measurements. {\it Phys. Rev. B} {\bf 83}, 054506 (2011).
%
%%
%
%\bibitem{chan2020} Chan, M. K., McDonald, R. D.,, Ramshaw, B. J., Betts, J. B., Shekhter, A., Bauer, E. D., Harrison, N. Extent of Fermi-surface reconstruction in the high-temperature superconductor HgBa2CuO4$_\delta$. {\it Proc. Nat. Acad. Sci. USA} {\bf 117},  9782-9786 (2020).
%
%%
%
%\bibitem{armitage2003} Armitage, N. P, Lu, D. H., Kim, C., Damascelli, A., Shen, K. M., Ronning, F., Feng, D. L., Bogdanov, P., Zhou, X. J., Yang, W. L., Hussain, Z., Mang, P. K., Kaneko, N., Greven, M., Onose, Y., Taguchi, Y., Tokura,Y. and Shen, Z.-X., Angle-resolved photoemission spectral function analysis of the electron-doped cuprate Nd$_{1.85}$Ce$_{0.15}$CuO$_4$. {\it Phys. Rev. B} {\bf 68}, 064517(2003). 
%
%%
%
%\bibitem{lohneysen2007} L\"{o}neysen, H., Rosch, A., Vojta, M. and W\"{o}lfle, P., Fermi-liquid instabilities at magnetic quantum phase transitions. {\it Rev. Mod. Phys.} {\bf 79}, 1015 (2007).
%
%
%%
%
%\bibitem{badoux2016} Badoux, S., Tabis, W., Lalibert\'{e}, F., Grissonnanche, G., Vignolle, B., Vignolles, D., B\'{e}ard, J., Bonn, D. A., Hardy, W. N., Liang, R., Doiron-Leyraud, N., Taillefer, L. and Proust, C., Change of carrier density at the pseudogap critical point of a cuprate superconductor. {\it Nature} {\bf 531}, 210 (2016).
%
%%
%
%\bibitem{lizaire2021} Lizaire,M., Legros, A., Gourgout, A., Benhabib, S., Badoux, S., Lalibert\'{e}, F., Boulanger, M.-E., Ataei, A., Grissonnanche, G., LeBoeuf, D., Licciardello, S., Wiedmann, S., Ono, S., Raffy, H., Kawasaki, S., Zheng, G.-Q., Doiron-Leyraud, N., Proust, C. and Taillefer, L., Transport signatures of the pseudogap critical point in the cuprate superconductor Bi$_2$Sr$_{2-x}$La$_x$CuO$_{6+\delta}$. {\it Phys. Rev. B} {\bf 104}, 014515 (2021). 
%
%\bibitem{putzke2021} Putzke, C., Benhabib, S., Tabis, W., Ayres, J., Wang, Z., Malone, L., Licciardello, S., Lu, J., Kondo, T., Takeuchi, T., Hussey, N. E., Cooper, J. R. and Carrington, A., Reduced Hall carrier density in the overdoped strange metal regime of cuprate superconductors. {\it Nature Phys} {\bf 17}, 826 (2021). 
%
%%
%
%\bibitem{badoux2016i} Badoux, S., Afshar, S. A. A., Michon, B., Ouellet, A., Fortier, S., LeBoeuf, D., Croft, T. P., Lester, C., Hayden, S. M., Takagi, H., Yamada, K., Graf, D., Doiron-Leyraud, N., Taillefer, L., Critical doping for the onset of Fermi-surface reconstruction by charge-density-wave order in the cuprate superconductor La$_{2-x}$Sr$_x$CuO$_4$. {\it Phys. Rev. B} {\bf 6}, 021004 (2016). 
%
%\bibitem{takagi1989} Takagi, H., Ido, T., Ishibashi, S., Uota, M., Uchida, S., Tokura, Y., Superconductor-to-nonsuperconductor transition in (La$_{1-x}$Sr$_x$)$_2$CuO$_4$ as investigated by transport and magnetic measurements. {\it Phys. Rev. B} {\bf 40}, 2254-2261 (1989).
%
%%
%
%
%%
%
%
%
%%
%
%
%%
%
%\bibitem{hucker2014} H\"{u}cker, M., Christensen, N.B., Holmes, A.T., Blackburn, E., Forgan, E.M., Liang, R., Bonn, D.A., Hardy, W.N., Gutowski, O., Zimmermann, M.v., Hayden, S.M., Chang, J., Competing charge, spin, and superconducting orders in underdoped YBa$_2$Cu$_3$O$_y$.  {\it Phys. Rev. B} {\bf 90}, 054514 (2014). 
%
%\bibitem{blancocanosa2014} Blanco-Canosa, S., Frano, A., Schierle, E., Porras, J., Loew, T., Minola, M., Bluschke, M., Weschke, E., Keimer, B., Le Tacon, M., Resonant x-ray scattering study of charge-density wave correlations in YBa$_2$Cu$_3$O$_{6+x}$. {\it Phys. Rev. B} {\bf 90}, 054513 (2014).
%
%\bibitem{tabis2017} Tabis, W., Yu, B., Bialo, I., Bluschke, M., Kolodziej, T., Kozlowski, A., Blackburn, E., Sen, K., Forgan, E. M., v. Zimmermann, M., Tang, Y., Weschke, E., Vignolle, B., Hepting, M., Gretarsson, H., Sutarto, R., He, F., Le Tacon, M., Bari\u{s}i\'{c}, N., Yu, G. and Greven, M., Synchrotron x-ray scattering study of charge-density-wave order in HgBa$_2$CuO$_{4+\delta}$. {\it Phys. Rev. B} {\bf 96}, 134510 (1970). 
%
%%
%
%
%%
%
%\bibitem{shekhter2017} Shekhter, A., Modic, K. A., McDonald, R. D.,  and Ramshaw, B. J., Thermodynamic constraints on the amplitude of quantum oscillations. {\it Phys. Rev. B} {\bf 95}, 121106 (2017). 
%
%%
%
%
%%
%
%\bibitem{luttinger1960a} Luttinger, J. M., and Ward, J. C., Ground-State Energy of a Many-Fermion System. II. {\it Phys. Rev.} {\bf 118}, 1417-1427 (1960).
%
%\bibitem{luttinger1960b} Luttinger. J. M., Fermi Surface and Some Simple Equilibrium Properties of a System of Interacting Fermions. {\it Phys. Rev.} {\bf 119}, 1153-1163 (1960).
%
%%
%
%
%\bibitem{yamaji2011} Yamaji, Y., and Imada, M., Composite-Fermion Theory for Pseudogap, Fermi Arc, Hole Pocket, and Non-Fermi Liquid of Underdoped Cuprate Superconductors. {\it Phys. Rev. Lett.} {\bf 106}, 016404 (2011). 
%
%%
%\bibitem{ando2002} Ando, Y. and Segawa, K., Magnetoresistance of untwinned YBa$_2$Cu$_3$O$_y$ single crystals in a wide range of doping: anomalous hole-doping dependence of the coherence lLength. {\it Phys. Rev. Lett.} {\bf 88}, 167005 (2002).
%%
%\bibitem{grissonnanche2014} Grissonnanche, G., Cyr-Choini\`{e}re, O., Lalibert\'{e}, F., Ren\'{e} de Cotret, S., Juneau-Fecteau, A., Dufour-Beaus\'{e}jour, S., Delage, M.-\`{E}, LeBoeuf, D., Chang, J., Ramshaw, B. J., Bonn, D. A., Hardy, W. N., Liang, R., Adachi, S., Hussey, N. E., Vignolle, B., Proust, C., Sutherland, M., Kr\"{a}mer, S., Park, J.-H., Graf, D., Doiron-Leyraud, N., Taillefer, L. Direct measurement of the upper critical field in cuprate superconductors. {\it Nature Commun.} {\bf 5}, 3280 (2014).
%%
%%%
%%
%\bibitem{basov1995} Basov, D. N., Liang, R., Bonn, D. A., Hardy, W. N., Dabrowski, B., Quijada, M., Tanner, D. B., Rice, J. P., Ginsberg, D. M. and Timusk, T. In-plane anisotropy of the penetration depth in YBa$_2$Cu$_3$O$_{7-x}$ and YBa$_2$Cu$_4$O$_8$ superconductors. {\it Phys. Rev. Lett.} {\bf 74}, 598 (1995). 
%%
%%%
%%
%\bibitem{franck1993} Franck, J. P., Harker, S. and Brewer, J. H. Copper and oxygen isotope effects in La$_{2-x}$Sr$_x$CuO$_4$. {\it Phys. Rev. Lett.} {\bf 71}, 283 (1993). 
%
%%
%
%
%


%%%%%%%%%%%%%%%

%
%%
%
%\bibitem{ramshaw2011} Ramshaw, B. J., Vignolle, B., Day, J., Liang, R., Hardy, W., Proust, C.
%and  Bonn, D. A., Angle dependence of quantum oscillations in YBa$_2$Cu$_3$O$_{6.59}$ shows free-spin behaviour of quasiparticles. {\it Nature Phys.} {\bf 7}, 234 (2011).
%
%%
%
%\bibitem{ando2002} Ando, Y. and Segawa, K., Magnetoresistance of untwinned YBa$_2$Cu$_3$O$_y$ single crystals in a wide range of doping: anomalous hole-doping dependence of the coherence lLength. {\it Phys. Rev. Lett.} {\bf 88}, 167005 (2002).
%
%\bibitem{grissonnanche2014} Grissonnanche, G., Cyr-Choini\`{e}re, O., Lalibert\'{e}, F., Ren\'{e} de Cotret, S., Juneau-Fecteau, A., Dufour-Beaus\'{e}jour, S., Delage, M.-\`{E}, LeBoeuf, D., Chang, J., Ramshaw, B. J., Bonn, D. A., Hardy, W. N., Liang, R., Adachi, S., Hussey, N. E., Vignolle, B., Proust, C., Sutherland, M., Kr\"{a}mer, S., Park, J.-H., Graf, D., Doiron-Leyraud, N., Taillefer, L. Direct measurement of the upper critical field in cuprate superconductors. {\it Nature Commun.} {\bf 5}, 3280 (2014).
%
%%
%
%\bibitem{basov1995} Basov, D. N., Liang, R., Bonn, D. A., Hardy, W. N., Dabrowski, B., Quijada, M., Tanner, D. B., Rice, J. P., Ginsberg, D. M. and Timusk, T. In-plane anisotropy of the penetration depth in YBa$_2$Cu$_3$O$_{7-x}$ and YBa$_2$Cu$_4$O$_8$ superconductors. {\it Phys. Rev. Lett.} {\bf 74}, 598 (1995). 
%
%%
%
%\bibitem{franck1993} Franck, J. P., Harker, S. and Brewer, J. H. Copper and oxygen isotope effects in La$_{2-x}$Sr$_x$CuO$_4$. {\it Phys. Rev. Lett.} {\bf 71}, 283 (1993). 
%
%
%
%
%
%
%
%%
%
%\bibitem{lee2017} Lee, J. A., Xin, Y., Stolt, I., Halperin, W. P., Reyes, A. P., Kuhns, P. L., and Chan, M. K., Magnetic-field-induced vortex-lattice transition in HgBa$_2$CuO$_{4+\delta}$. {\it Phys. Rev. B} {\bf 95}, 024512 (2017). 
%
%%
%
%\bibitem{friedberg1989} Friedberg, R., Lee, T. D., Gap energy and long-range order in the boson-fermion model of superconductivity. {\it Phys. Rev. B} {\bf 40}, 6745-6762 (1989). 
%
%\bibitem{micnas1990} Micnas, R., Ranninger, J., Robaszkiewicz, S., Superconductivity in narrow-band systems with local nonretarded attractive interactions. {\it Rev. Mod. Phys.} {\bf 62}, 113-171 (1990).
%
%\bibitem{drechsler1992} Drechsler, M., Zwerger, W., Crossover from BCS-superconductivity to Bose-condensation. {\it Ann. Physik} {\bf 1}, 15-23 (1992). 
%
%\bibitem{sademelo1993} S\'{a} de Melo, C. A. R., Randeria, M., Engelbrecht, J. R., Crossover from BCS to Bose superconductivity: transition temperature and time-dependent Ginzburg-Landau theory. {\it Phys. Rev. Lett.} {\bf 71}, 3202-3205 (1993).
%
%\bibitem{harrison2022} Harrison, N. and Chan, M. K., Magic gap ratio for optimally robust fermionic condensation and its implications for high-$T_{\rm c}$ superconductivity. {\it Phys. Rev. Lett.} {\bf 129}, 017001 (2022). 
%
%%
%
%\bibitem{pasrija2016} Pasrija, K., Chakraborty, P. B., Kumar, S., Effective Hamiltonian based Monte Carlo for the BCS to BEC crossover in the attractive Hubbard model. {\it Phys. Rev. B} {\bf 94},165150 (2016). 
%
%%
%

%
%%
%
%\bibitem{botelho2006} Botelho, S. S. and S\'{a} de Melo, C. A. R. Vortex-antivortex lattice in ultracold Fermionic gases. {\it Phys. Rev. Lett.} {\bf 96}, 040404 (2006).
%
%\bibitem{nakagawa2021} Nakagawa, Y., Kasahara, Y., Nomoto, T., Arita, R., Nojima, T., Iwasa, Gate-controlled BCS-BEC crossover in a two-dimensional superconductor. {\it Science} {\bf 372}, 190-195 (2021).
%
%\bibitem{tinkham1996} Tinkham, M. Introduction to Superconductivity (McGraw-Hill, 1996). 


%%%%%%%%%%%%%%%%%%%%%%%%

%
%\bibitem{bogdanov2000} Bogdanov, P.V.,  Lanzara, A., Kellar, S. A., Zhou, X. J., Lu, E. D., Zheng, W. J., Gu, G.,  Shimoyama, J.-I., Kishio, K., Ikeda, H., Yoshizaki, R., Hussain, Z. and Shen, Z. X., Evidence for an energy scale for quasiparticle dispersion in Bi$_2$Sr$_2$CaCu$_2$O$_8$. {\it Phys. Rev. Lett.} {\bf 85}, 2581 (2000).
%%
%
%\bibitem{kaminski2001} Kaminski, A., Randeria, M., Campuzano, J. C., Norman, M. R., Fretwell, H., Mesot, J., Sato, T., Takahashi, T. and Kadowaki, K., Renormalization of Spectral Line Shape and Dispersion below $T_{\rm c}$ in Bi$_2$Sr$_2$CaCu$_2$O$_{8-\delta}$. {\it Phys. Rev. Lett.} {\bf 86}, 1070 (2001).
%
%%
%
%\bibitem{he2019} He, J., Rotundu, C. R., Scheurer, M. S., He, Y., Hashimoto, M., Xu, K.-J., Wang, Y., Huang, E. W., Jia, T., Chen, S., Moritz, B., Lu, D., Lee, Y. S., Devereaux, T. P., Shen, Z.X., Fermi surface reconstruction in electron-doped cuprates without antiferromagnetic long-range order. {\it Proc. natl. Acad. USA} {\bf 116}, 3449 (2019). 
%
%%
%
%\bibitem{harrison2015} Harrison, N., Ramshaw, B. J. and Shekhter, A., Nodal bilayer-splitting controlled by spin-orbit interactions in underdoped high-$T_{\rm c}$ cuprates. {\it Scientific Reports} {\bf 5} 10914 (2015).
%
%\bibitem{harrison2020} Hartstein, M., Hsu, Y.-T., Modic, K. A., Porras, J., Loew, T., Le Tacon, M., Zuo, H., Wang, J., Zhu, Z., Chan, M. K., McDonald, M. K., Lonzarich, G. G., Keimer, B., Sebastian, S. E. and Harrison, N., Hard antinodal gap revealed by quantum oscillations in the pseudogap regime of underdoped high-$T_{\rm c}$ superconductors. {\it Nature Phys.} {\bf 16}, 841 (2020). 
%
%%
%
%
%%
%
%\bibitem{ramshaw2017} Ramshaw, B. J., Harrison, N., Sebastian, S. E., Ghannadzadeh, S., Modic, K. A., Bonn, D. A., Hardy, W. N., Liang, R. and Goddard, P. A., Broken rotational symmetry on the Fermi surface of a high-$T_{\rm c}$ superconductor. {\it npj Quantum Materials} {\bf 2}, 8 (2017).
%
%%
%
%\bibitem{kohn1961} Kohn, W., Cyclotron resonance and de Haas-van Alphen oscillations of an
%Interacting electron gas. {\it Phys.  Rev.} {\bf 123}, 1242 (1961).
%
%
%
%
%\bibitem{macdonald1989} MacDonald, A. H., and Kallin, C., Cyclotron resonance in two dimensions: electron-electron interactions and band nonparabolicity. {\it Phys. Rev. B} {\bf 40}, 5791 (1989).
%
%%
%

%
%


%



%


%

%\bibitem{loram1993} Loram, J. W., Mirza, K. A., Cooper, J. R., Liang, W. Y. Electronic specific heat of YBa$_2$Cu$_3$O$_{6+x}$ from 1.8 to 300~K. {\it Phys. Rev. Lett.} {\bf 71}, 1740-1743 (1993).

%



%

%\bibitem{zaanen2019} Zaanen, J., Planckian dissipation, minimal viscosity and the transport in cuprate strange metals. {\it SciPost Phys.} {\bf 6}, 061 (2019).


%
%
%%







%
%
%\bibitem{shishido2005} Shishido, H., Settai, R., Harima, H. and Onuki, Y., A drastic change of the Fermi surface at a critical pressure in CeRhIn$_5$: dHvA study under pressure. {\it J. Phys. Soc. Japan} {\bf 74}, 1103 (2005). 
%
%
%\bibitem{jiang2014} Jiang, J., Tsirkin, S. S., Shimada, K., wasawa, H., Arita, M., Anzai, H., Namatame, H., Taniguchi, M., Sklyadneva, I. Y., Heid, R., Bohnen, K.-P., Echenique, P. M. and Chulkov, E. V., Many-body interactions and Rashba splitting of the surface state on Cu(110). {\it Phys. Rev. B} {\bf 89}, 085404 (2014).
%
%\bibitem{abrahams2011} Abrahams, E. and Si. Q., Quantum criticality in the iron pnictides and chalcogenides. {\it J. Phys. Condens,: Matter} {\bf 23}, 223201 (2011). 
%
%%
%
%
%%
%
%
%%




%




%


%










%
%
%
%
%
%
%
%
%\bibitem{loktev2001} Loktev,V. M., Quick, R. M. and Sharapov, S. G., {\it Phys. Rep.} {\bf 349} 1 (2001).
%
%%
%
%\bibitem{uemura1993}  Uemura, Y. J., Keren, A., Le, L. P., Luke, G. M., Wu, W. D., Kubo, Y., Manako, T., Shimakawa, Y., Subramanian, M., Cobb, J. L.  and Markert, J. T., Magnetic-field penetration depth in TI$_2$Ba$_2$CuO$_{6+\delta}$ in the overdoped regime. {\it Nature} {\bf 364}, 605 (1993). 
%
%\bibitem{niedermayer19993} Niedermayer, Ch., Bernhard, C., Binninger, U., Gl\"{u}ckler, H., Tallon, J. L., Ansaldo, E. J. and Budnick, J. I. Muon spin rotation study of the correlation between $T_{\rm c}$ and $n_{\rm s}/m^\ast$ in overdoped Tl$_2$Ba$_2$CuO$_{6+\delta}$. {\it Phys. Rev. Lett.} {\bf 71}, 1764 (1993). 
%
%\bibitem{weber1991} Weber, M., Birrer, P., Gygax, F. N., Hitti, B., Lippelt, E., Maletta, H. and Schenck, A. {\it Hyperfine Interactions} {63}, 93 (1991). 
%
%\bibitem{weber1993} Weber, M., Amato, A., Gygax, F. N., Schenck, A., Maletta, H., Duginov, V. N., Grebinnik, V. G., Lazarev, A. B., Olshevsky, V. G., Yu. Pomjakushin, V., Shilov, S. N., Zhukov, V. A., Kirillov, B. F., Pirogov, A. V., Ponomarev, A. N., Storchak, V. G., Kapusta, S. and Bock, J. Magnetic-flux distribution and the magnetic penetration depth in superconducting polycrystalline Bi$_2$Sr$_2$Ca$_{1-x}$Y$_x$Cu$_2$O$_{8+\delta}$ and Bi$_{2-x}$Pb$_x$Sr$_2$CaCu$_2$O$_{8+\delta}$. {\it Phys. Rev. B} {\bf 48}, 13022 (1993). 
%
%
%
%
%
%
%
%
%
%
%
%
%
%
%\bibitem{cao2018} Cao, Y., Fatemi, V., Fang, S., Watanabe, K., Taniguchi, T., Kaxiras, E., Jarillo-Herrero, P. Unconventional superconductivity in magic-angle graphene superlattices. {\it Nature} {\bf 556}, 43 (2018).
%
%%
%
%
%%
%
%
%%
%
%
%\bibitem{leggett1965} Leggett, A. J. Theory of a superfluid Fermi Liquid. I. General formalism and static properties. {\it Phys. Rev.} {\bf 140}, A1869 (1965). 
%
%%
%
%\bibitem{fermienergynote} Using $n_{\rm s}/m^\ast=\varepsilon_{\rm F}/\pi\hbar^2l$ in a layered superconductor.
%
%%
%
%
%%
%
%
%
%%
%
%
%
%
%%
%
%
%
%%
%
%
%
%%
%
%
%%
%
%
%%
%
%
%%
%
%
%
%
%
%%
%
%
%
%%
%
%
%
%
%
%
%
%
%%
%
%\bibitem{mondal2011} Mondal, M., Kamlapure, A., Chand, M., Saraswat, G., Kumar, S., Jesudasan, J., Benfatto, L., Tripathi, V. and Raychaudhuri, P. Phase fluctuations in a strongly disordered $s$-wave NbN superconductor close
% to the metal-insulator transition. {\it Phys. Rev. Lett.} {\bf 106}, 047001 (2011).
%
%
%%
%
%
%\bibitem{nelson1977} Nelson, D. R. and Kosterlitz, J. M. Universal jump in the superfluid density of two-dimensional superfluids. {\it Phys. Rev. Lett.} {\bf 39}, 1201 (1977). 
%
%%
%
%\bibitem{randeria1989} Randeria, M., Duan, J.-M., Shieh, L.-Y., Bound states, Cooper pairing, and Bose condensation in two dimensions, {\it Phys. Rev. Lett.} {\bf 62}, 981-984 (1989). 
%
%%
%
%
%%
%
%
%
%
%
%
%
%
%%
%
%
%%\bibitem{helm2010} Helm, T., Kartsovnik, M., Sheikin, I., Bartkowiak, M., Wolff-Fabris, F., Bittner, N., Biberacher, W., Lambacher, M., Erb, A., Wosnitza, J. and Gross, R. Magnetic breakdown in the electron-doped cuprate superconductor Nd$_{2-x}$Ce$_x$CuO$_4$: the reconstructed Fermi surface survives in the strongly overdoped regime. {\it Phys. Rev. Lett.} {\bf 105}, 247002 (2010).
%
%
%
%%
%
%
%
%
%
%
%
%
%
%
%
%
%
%
%
%
%\bibitem{jochim2003} Jochim, S.,Bartenstein, M., Altmeyer, A ., Hendl, G., Riedl, S., Chin, C.., Denschlag, J.H., Grimm, R., Bose-Einstein condensation of molecules, {\it Science} {\bf 302}, 2101-2103 (2003). 
%
%\bibitem{zwierlein2003} Zwierlein, M.W., Stan, C.A., Schunck, C.H., Raupach, S.M.F., Gupta, S., Hadzibabic, Z., Ketterle, W.,  Observation of Bose-Einstein condensation of molecules, {\it Phys. Rev. Lett.} {\bf 91}, 250401 (2003).
%
%
%
%
%
%
%
%
%%
%
%
%
%
%%
%
%\bibitem{bloch2008} Bloch, I., Dalibard, J., Zwerger, W., Many-body physics with ultracold gases. {\it Rev. Mod. Phys.} {\bf 80}, 885-964 (2008).
%
%\bibitem{giorgini2008} Giorgini, S., Pitaevskii, L. P., Stringari, S., Theory of ultracold atomic Fermi gases. {\it Rev. Mod. Phys.} {\bf 80}, 1215-1274 (2008).
%
%%
%
%\bibitem{ku2012} Ku, M. J. H., Sommer, A. T., Cheuk, L. W., Zwierlein, M. W.,  Revealing the superfluid Lambda transition in the universal thermodynamics of a unitary Fermi gas. {\it Science} {\bf 335}, 563-567 (2012).
%
%\bibitem{regal2004} Regal, C. A., Greiner, M., Jin, D. S., Observation of resonance condensation of fermionic atom pairs. {\it Phys. Rev. Lett.} {\bf 92}, 040403 (2004).
%
%\bibitem{zwielein2004} Zwierlein, M. W., Stan, C. A., Schunck, C. H., Raupach, S. M. F., Kerman, A. J., Ketterle, W., Condensation of pairs of Fermionic atoms near a Feshbach resonance. {\it Phys. Rev. Lett.} {\bf 92}, 120403 (2004). 
%
%%
%
%\bibitem{haussmann2007} Haussmann, R., Rantner, W., Cerrito, S., Zwerger, W., Thermodynamics of the BCS-BEC crossover. {\it Phys. Rev. A} {\bf 75}, 023610 (2007).
%
%%
%
%\bibitem{chen2005} Chen, Q., Stajic, J., Tan, S., Levin, K. BCS–BEC crossover: From high temperature superconductors to ultracold superfluids. {\it Phys. Reports} {\bf 412}, 1-88 (2005).
%
%\bibitem{randeria2014} Randeria, M., Taylor, E., Crossover from Bardeen-Cooper-Schrieffer to Bose-Einstein condensation and the unitary Fermi gas. {\it Annu. Rev. Condens. Matter Phys.} {\bf 5}, 209-232 (2014). 
%
%\bibitem{strinati2018} Strinati, G. C., Pieri, P. P., R\"{o}pke, G., Schuck, P., Urban, M. The BCS–BEC crossover: From ultra-cold Fermi gases to nuclear systems. {\it Phys. Rep.} {\bf 738}, 1-76 (2018).
%
%
%
%\bibitem{kasahara2014} Kasahara, S., Watashige, T., Hanaguri, T., Kohsaka, Y., Yamashita, T., Shimoyama, Y., Mizukami, Y., Endo, R., Ikeda, H., Aoyama, K., Terashima, T., Uji, S., Wolf, T., von Lohneysen, H., Shibauchi, T., Matsuda, Y. Field-induced superconducting phase of FeSe in the BCS-BEC cross-over. {\it Proc. Nat. Acad. Sci. USA} {\bf 111}, 116309-16313 (2014).
%
%\bibitem{rinott2017} Rinott, S., Chashka, K. B., Ribak, A., Rienks, E. D. L., Taleb-Ibrahimi, A., Le Fevre, P., Bertran, F., Randeria, M., Kanigel, A., Tuning across the BCS-BEC crossover in the multiband superconductor Fe$_{1+y}$Se$_x$Te$_{1-x}$: An angle-resolved photoemission study. {\it Sci. Adv.} {\bf 3}, 1602372 (2017).
%
%%%%%%%%%%%%%%%%%%%%%%
%
%
%
%\bibitem{li2010} Li, L., Wang, Y., Komiya,. S., Ono, S., Ando, Y., Gu, G.D., Ong, N.P. Diamagnetism and Cooper pairing above $T_{\rm c}$ in cuprates. {\it Phys. Rev. B} {\bf 81}, 054510 (2010).
%
%\bibitem{dubroka2011} Dubroka, A., R\"{o}ssle, M., Kim, K.W., Malik, V.K., Munzar, D., Basov, D.N., Scafgans, A.A., Moon, S.J., Lin, C.T., Haug, D., Hinkov, V., Keimer, B., Wolf, Th., Storey, J.G., Tallon, J.L., Bernhard, C., Evidence of a precursor superconducting phase at temperatures as high as 180~K in $R$Ba$_2$Cu$_3$O$_{7-\delta}$ ($R=$~Y, Gd, Eu) superconducting crystals from infrared spectroscopy. {\it Phys. Rev. Lett.} {\bf 106}, 047006 (2011).
%
%\bibitem{hu2014} Hu, W., Kaiser, S., Nicoletti, D., Hunt, C.R., Gierz, I., Homann, M.C., Le Tacon, M., Loew, T., Keimer, B., Cavalleri, A., Optically enhanced coherent transport in YBa$_2$Cu$_3$O$_{6.5}$ by ultrafast redistribution of interlayer coupling. {\it Nature Materials} {\bf 13}, 705-711 (2014).
%
%\bibitem{kaiser2014} Kaiser, S., Hunt, C.R., Nicoletti, D., Hu, W., Gierz, I., Liu, H.Y., Le Tacon, M., Loew, T., Haug, D., Keimer, B., Cavalleri, A., Optically induced coherent transport far above $T_c$ in underdoped YBa$_2$Cu$_3$O$_{6+\delta}$. {\it Phys. Rev. B} {\bf 89}, 184516 (2014).
%
%\bibitem{zhou2019} Zhou, P., Chen, L., Liu, Y., Sochnikov, I., Bollinger, A. T., Han, M.-G., Zhu, Y., He, X., Bo\u{z}ovi\'{c}, I., Natelson, D., Electron pairing in the pseudogap state revealed by shot noise in copper oxide junctions. {\it Nature} {\bf 572}, 493-496 (2019).
%
%%
%
%\bibitem{supplemental} See Supplementary Information,
%which discusses data pertaining to $\delta\gamma(T_{\rm c})$ in conventional BCS and Fe- and Ni based superconductors; information about cuprate hole dopings $p$ used in constructing graphs; raw $\gamma$ and $\chi$ data including the locations of $T_\gamma$ and $T_\chi$; Knight shift data on BLSCO; a discussion pertaining to prior modeling of $\delta\gamma(T_{\rm c})$ versus $p$ by Chen {\it et al.}; details concerning cold atomic gas data used; estimates of the gap ratio at the unitary point of a cold atomic gas; a discussion of the thermodynamics of maxima in $\gamma$ and $\chi$; coherence length estimates in various cuprates based on $H_{\rm c2}$; estimates of the Fermi energy in the cuprates; a discussion of the higher values of $p^\ast$ in cuprates with lower $T_{\rm c}$'s; the origin of `Fermi arcs;' the $T$-dependence of the antinodal gap; lifetime effects on $N_0$; problems with a smaller gap option for $\Delta$ in the cuprates;  a hidden maximum in the heat capacity of a cold atomic Fermi gas; reports of peaks in $\delta\gamma(T_{\rm c})$, $\gamma$ and $m^\ast$ associated with quantum criticality, and includes Refs.~\cite{radcliffe1996,liang2006,zhou1996,tallon1995,meingast2009,kawasaki2010,kittel2004,nolthing2009,suzuki1991,rourke2010,leboeuf2007,rullieralbenque2008,fukuzumi1996,nachumi1996,takagi1989,carrington1994,chubukov2007,demello2012,fradkin2015,agterberg2020,ding1996,renner1998,norman2014,park2008,leblanc2009,norman1998,damascelli2003}.
%
%
%
%
%
%
%%supplemetary references
%%%%%%%%%%%%%%%%%%%%%%%
%%%%%%%%%%%%%%%%%%%%%%%
%%%%%%%%%%%%%%%%%%%%%%%
%%%%%%%%%%%%%%%%%%%%%%%
%%%%%%%%%%%%%%%%%%%%%%%
%%%%%%%%%%%%%%%%%%%%%%%
%%%%%%%%%%%%%%%%%%%%%%%
%%%%%%%%%%%%%%%%%%%%%%%
%%%%%%%%%%%%%%%%%%%%%%%
%%%%%%%%%%%%%%%%%%%%%%%
%
%
%%
%
%\bibitem{suzuki1991} Suzuki M., Hikita, M., Resistive transition, magnetoresistance, and anisotropy in La$_{2-x}$Sr$_x$CuO$_4$ single-crystal thin films. {\it Phys. Rev. B} {\bf 44}, 249-261 (1991).
%
%\bibitem{carrington1994} Carrington, A., Mackenzie, A. P., Sinclair, D. C., Cooper, J. R., Field dependence of the resistive transition in Tl$_2$Ba$_2$CuO$_{6+\delta}$. {\it Phys. Rev. B} {\bf 49}, 13243-13246 (1994).
%
%\bibitem{tallon1995} Tallon, J.L., Bernhard, C., Shaked, H., Hitterman, R.L., Jorgensen, J.D., Generic superconducting phase-behavior in high-$T_{\rm c}$ variation with hole concentration in YBa$_2$Cu$_3$O$_{7-\delta}$, {\it Phys. Rev. B} {\bf 51}, 12911-12914 (1995). 
%
%\bibitem{ding1996} Ding, H., Campuzano, J.C., Takahashi, T., Randeria, M., Norman, M.R., Mochikull, T., Kadowaki, K., Giapintzaki, J., Spectroscopic evidence for a pseudogap in the normal state of underdoped high-$T_{\rm c}$ superconductors. {\it Nature} {\bf 382}, 51-54 (1996).
%
%\bibitem{fukuzumi1996} Fukuzumi, Y., Mizuhashi, K., Takenaka, K., Uchida, S., Universal superconductor-insulator transition and $T_{\rm c}$ depression in Zn-substituted high-$T_{\rm c}$ cuprates in the underdoped regime. {\it Phys. Rev. Lett.} {\bf 76}, 684-687 (1996).
%
%\bibitem{nachumi1996} Nachumi, B., Keren, A., Kojima, K., Larkin, M., Luke, G. M., Merrin, J., Tchernysh\"{o}v, O., Uemura, Y. J., Ichikawa, N., Goto, M., Uchida, S., Muon spin relaxation studies of Zn-substitution effects in high-$T_{\rm c}$ cuprate superconductors. {\it Phys. Rev. Lett.} {\bf 77}, 5421-5424 (1996).
%
%
%\bibitem{zhou1996} Zhou, J.-S., Goodenough, J.B., Dabrowski, B., Rogacki, K., 
%Transport properties of a YBa$_2$Cu$_4$O$_8$ crystal under high pressure. {\it Phys. Rev. Lett.} {\bf 77}, 4253-4256 (1996)
%
%\bibitem{norman1998} Norman, M.R., Ding, H., Randeria, M., Campuzano, J.C., Yokoya, T., Takeuchik, T, Takahashi, T., Mochiku, T., Kadowaki, K., Guptasarma, P., Hinks, D.G., Destruction of the Fermi surface in underdoped high-$T_{\rm c}$ superconductors. {\it Nature} {\bf 392}, 157-160 (1998).
%
%\bibitem{renner1998} Renner, Ch., Revaz, B., Genoud, J.-Y., Kadowaki, K., Fischer, \O., Pseudogap precursor of the superconducting gap in under- and overdoped Bi$_2$Sr$_2$CaCu$_2$O$_{8+\delta}$. {\it Phys. Rev. Lett.} {\bf 80}, 149-152 (1998).
%
%
%\bibitem{kittel2004} Kittel, C.,  Introduction to Solid State Physics, eighth ed. (Wiley, New York, 2004).
%
%\bibitem{liang2006} Liang, R., Bonn, D. A., Hardy, W. N. Evaluation of CuO$_2$ plane hole doping in YBa$_2$Cu$_3$O$_{6+x}$ single crystals. {\it Phys. Rev. B} {\bf 73}, 180505 (2006).
%
%\bibitem{chubukov2007} Chubukov, A.V., Norman, M.R., Millis, A.J., Abrahams, E., Gapless pairing and the Fermi arc in the cuprates. {\it Phys. Rev. B} {\bf 76}, 180501 (2007).
%
%\bibitem{leboeuf2007} LeBoeuf, D., Doiron-Leyraud, N., Levallois, J., Daou, R., Bonnemaison, J. B., Hussey, N. E., Balicas, L., Ramshaw, B. J., Liang, R. X., Bonn, D. A., Hardy, W. N., Adachi, S., Proust, C., Taillefer, L. Electron pockets in the Fermi surface of hole-doped high-$T_{\rm c}$ superconductors. {\it Nature} {\bf 450}, 533-536 (2007).
%
%\bibitem{park2008} Park, T., Graf, M. J., Boulaevskii, L., Sarrao, J. L., Thompson, J. D. Electronic duality in strongly correlated matter. {\it Proc. Nat. Acad. Sci. USA} {\bf 105}, 6825-6828 (2008).
%
%
%\bibitem{leblanc2009} LeBlanc, J.P.F., Nicol, E.J., Carbotte, J.P., Specific heat of underdoped cuprates: Resonating valence bond description versus Fermi arcs. {\it Phys. Rev. B} {\bf 80}, 060505 (2009).
%
%\bibitem{meingast2009} Meingast, C., Inaba, A., Heid, R., Pankoke, V., Bohnen, K.-P., Reichardt, W., Wolf, T., Specific-heat of YBa$_2$Cu$_3$O$_x$ up to 400~K: high-resolution adiabatic measurements and {\it Ab-initio} LDA phonon calculations. {\it J. Phys. Soc. Japan} {\it 78}, 074706 (2009).
%
%\bibitem{nolthing2009} Nolthing, W., Ramakanth, A., Quantum Theory of Magnetism (Springer, New York, 2009).
%
%\bibitem{kawasaki2010} Kawasaki, S., Lin, C., Kuhns, P.L., Reyes, A.P., Zheng, G.-Q., Carrier-concentration dependence of the pseudogap ground state of superconducting Bi$_2$Sr$_{2-x}$La$_x$CuO$_{6+\delta}$ revealed by $^{63,65}$Cu-nuclear magnetic resonance in very high magnetic fields. {\it Phys. Rev. Lett.} {\bf 105}, 137002 (2010).
%
%
%\bibitem{demello2012} de~Mello, E.V.L., Disordered-based theory of pseudogap, superconducting gap, and Fermi arc of cuprates. {\it Euro. Phys. Lett.} {\bf 99}, 37003 (2012).
%
%\bibitem{norman2014} Mishra, V., Chatterjee, U., Campuzano, J. C., Norman, M. R., Effect of the pseudogap on the transition temperature in the cuprates and implications for its origin. {\it Nature Phys.} {\bf 10}, 357-360 (2014).
%
%\bibitem{fradkin2015} Fradkin, E., Kivelson, S.A., Tranquada, J.M., Colloquium: Theory of intertwined orders in high temperature superconductors. {\it Rev. Mod. Phys.} {\bf 87}, 457-482 (2015).
%
%\bibitem{agterberg2020} Agterberg, D.F., Davis, J.C.S., Edkins, S.D., Fradkin, E.,Van~Harlingen, D.J., Kivelson, S.A., Lee, P.A., Radzihovsky, L., The physics of pair-density waves: cuprate superconductors and beyond. {\it Annu. Rev. Condens. Matter Phys.} {\bf 11}, 231–270 (2020).
%
%
%%%%%%%%%%%%%%%%%%%%%%%
%%%%%%%%%%%%%%%%%%%%%%%
%%%%%%%%%%%%%%%%%%%%%%%
%%%%%%%%%%%%%%%%%%%%%%%
%%%%%%%%%%%%%%%%%%%%%%%
%%%%%%%%%%%%%%%%%%%%%%%
%%%%%%%%%%%%%%%%%%%%%%%
%%%%%%%%%%%%%%%%%%%%%%%
%%%%%%%%%%%%%%%%%%%%%%%
%%%%%%%%%%%%%%%%%%%%%%%
%%%%%%%%%%%%%%%%%%%%%%%
%
%%
%
%
%
%%
%
%
%\bibitem{prevailingview} The prevailing view~\cite{keimer2015} is that the Fermi surface reconstruction producing the pockets, for instance by a charge density wave~\cite{hucker2014,blancocanosa2014}, cannot by itself account for the pseudogap. 
%
%%
%
%
%
%
%
%
%
%%%%%%%%%%%%%%%%
%
%\bibitem{chen2014} Chen, Q. and Wang, J., Pseudogap phenomena in ultracold atomic Fermi gases. {\it Front. Phys.} {\bf 9} 539-570 (2014).
%
%%
%
%
%%
%
%\bibitem{gaebler2010} Gaebler, J. P., Stewart, J. T., Drake, T. E., Jin, D. S., Perali, A., Pieri, P., Strinati, G. C., Observation of pseudogap behaviour in a strongly interacting Fermi gas. {\it Nat. Phys.}{\bf 6}, 569-573 (2010).
%
%\bibitem{magierski2011} Magierski, P., Wlazłowski, G., Bulgac, A., Onset of a pseudogap regime in ultracold Fermi gases. {\it Phys. Rev. Lett.} {\bf 107}, 145304 (2011).
%
%\bibitem{chin2004} Chin, C., Bartenstein, M., Altmeyer, A., Riedl, S., Jochim, S., Hecker Denschlag, J., Grimm, R., Observation of the pairing gap in a strongly interacting Fermi gas. {\it Science} {\bf 305}, 1128-1130 (2004).%pairing gap observation
%
%\bibitem{perali2011} Perali, A., Palestini, F., Pieri, P., Strinati, G. C., Stewart, J. T., Gaebler, J. P., Drake, T. E., Jin, D. S., Evolution of the normal state of a strongly interacting Fermi gas from a pseudogap phase to a molecular Bose gas. {\it Phys. Rev. Lett.} {\bf 106}, 060402 (2011).
%
%%
%
%\bibitem{tsuchiya2009} Tsuchiya, S., Watanabe, R., Ohashi, Y., Single-particle properties and pseudogap effects in the BCS-BEC crossover regime of an ultracold Fermi gas above $T_{\rm c}$. {\it Phys. Rev. A} {\bf 80}, 033613 (2009).
%
%\bibitem{jensen2020} Jensen, S., Gilbreth, C. N., Alhassid, Y. Pairing correlations across the superfluid phase transition in the unitary Fermi gas. {\it Phys. Rev. Lett.} {\bf 124}, 090604 (2020).
%
%\bibitem{richie2020} Richie-Halford, A., Drut, J. E., Bugac, A., Emergence of a pseudogap in the BCS-BEC crossover. {\it Phys. Rev. Lett.} {\bf 125} 060403 (2020).
%
%%
%
%
%%
%
%\bibitem{chakravarty2001} Chakravarty, S., Laughlin, R.B., Morr, D.K., Nayak, C., Hidden order in the cuprates. {\it Phys. Rev. B} {\bf 63}, 094503 (2001).
%
%
%\bibitem{varma2006} Varma, C.M., Theory of the pseudogap state of the cuprates. {\it Phys. Rev. B} {\bf 73}, 155113 (2006).
%
%
%\bibitem{nie2014} Nie, L., Tarjus, G., Kivelson, S.A., Quenched disorder and vestigial nematicity in the
%pseudogap regime of the cuprates. {\it Proc. Nat. Acad. Sci. USA} {\bf 111}, 7980-7985 (2014).
%
%\bibitem{schmalian1999} Schmalian, J., Pines, D., Stojkovi\'{c}, Microscopic theory of weak pseudogap behavior in the underdoped cuprate superconductors: General theory and quasiparticle properties. {\it Phys. Rev. B} {\bf 60}, 667-686 (1999).
%
%%%%%%%%%%%%%%%%%%%%%
%
%
%\bibitem{inosov2011} Inosov, D., S., Park, J. T., Charnukha, A., Yuan, L., Boris, A. V., Keimer, B., Hinkov, V., Crossover from weak to strong pairing in unconventional superconductors. {\it Phys. Rev. B} {\bf 83}, 214520 (20011).
%
%\bibitem{schirotzek2008} Schirotzek, A., Shin, Y., Schunck, C. H., Ketterle, W., Determination of the superfluid gap in atomic Fermi gases by quasiparticle spectroscopy. {\it Phys. Rev. Lett.} {\bf 101}, 140403 (208). 
%
%%
%
%\bibitem{cuprategap} Since Fig.~\ref{gap}c includes spectroscopic data taken at low temperatures, we assume this to provide an estimate of $\Delta$~\cite{supplemental}. 
%
%%
%
%
%
%%%%%%%%%%%%%%%%%%%
%
%\bibitem{leggett1980} Leggett, A. J. in {\it Modern Trends in the Theory of Condensed Matter} (eds Pekalski, A. \& Przystawa, J.) 13–27 (Proc. XVIth Karpacz Winter School of Theoretical Physics, Springer, Berlin, 1980).
%
%%
%
%\bibitem{zwerger2016} Zwerger, W. in {\it Proceedings of the International School of Physics ``Enrico Fermi'' --- Course 191 ``Quantum matter at ultralow temperatures''} (Inguscio, M., Ketterle, W., Roati, Q. eds.) 63-142 (IOS Press, Amsterdam; SIF Bologna, 2016).
%
%%
%
%
%
%
%
%
%%%%%%%%%%%%%%%%%%
%
%
%
%\bibitem{radcliffe1996} Radcliffe, J.W., Loram, J.W., Wade, J.M., Wltschek, G., Tallon, J.W., Electronic specific heat of overdoped TI$_2$Ba$_2$CuO$_{6+\delta}$ in a magnetic field. {\it J. Low. Temp. Phys.} {\bf 105}, 903-908 (1996).
%
%
%\bibitem{loram1998} Loram, J. W., Mirza, K. A., Cooper, J. R., Tallon, J. L. Specific heat evidence of the normal state pseudogap. {\it J. Phys. Chem. Solids} {\bf 59}, 2091-2094 (1998).
%
%
%\bibitem{mirmelstein1995} Mirmelstein, A., Junod, A., Triscone, G., Wang, K.-Q., Muller, J., Specific heat of Tl$_2$Ba$_2$CuO$_6$ (``2201'') 90~K superconducting ceramics in magnetic fields up to 14~T. {\it Physica C} {\bf 248}, 225-342 (1995).
%
%\bibitem{wen2009} Wen, H.-H., Mu, G., Luo, H., Yang, H., Shan, L., Ren, C., Cheng, P., Yan, J., Fan, L., Specific-heat measurement of a residual superconducting state in the normal state of underdoped Bi$_2$Sr$_{2-x}$La$_x$CuO$_{6+\delta}$ cuprate cuperconductors. {\it Phys. Rev. Lett.} {\bf 103}, 067002 (2009).
%
%
%\bibitem{compositions} For the high $T_{\rm c}$ cuprates LSCO, YBCO, Ca-YBCO, BSCCO, BSLCO, TBCO, Nd-LSCO and HBCO refer to La$_{2-x}$Sr$_x$CuO$_4$~\cite{loram2001}, YBa$_2$Cu$_3$O$_{6+x}$~\cite{loram1993,loram2001} and YBa$_2$Cu$_4$O$_8$~\cite{curro1997}, Y$_{0.8}$Ca$_{0.2}$Ba$_2$Cu$_3$O$_{6+x}$~\cite{loram1998,michon2019}, 
%%
%Bi$_2$Sr$_2$CaCu$_2$O$_{8+\delta}$ doped with 20\% Pb or 15\% Y~\cite{loram2001}, Bi$_2$Sr$_{2-x}$La$_x$CuO$_{6+\delta}$~\cite{wen2009}, % and Pb$_{0.55}$Bi$_{1.5}$Sr$_{1.6}$La$_{0.4}$CuO$_{6+\delta}$\cite{he2011}, 
%%
%Tl$_2$Ba$_2$CuO$_{6+\delta}$~\cite{radcliffe1996,kubo1991} La$_{2-y-x}$Nd$_y$Sr$_x$CuO$_4$~\cite{michon2019}, and HgBa$_2$CuO$_{4+\delta}$~\cite{chan2020}, respectively.
%
%%%%%%%%%%%%%%%%%%%%%%
%%%%%%%%%%%%%%%%%%%%%%%
%%%%%%%%%%%%%%%%%%%%%%%
%
%
%\bibitem{curro1997} Curro, N.J., Imai, T., Slichter, C.P., Dabrowski, B., High-temperature $^{63}$Cu(2) nuclear quadrupole and magnetic resonance measurements of YBa$_2$Cu$_4$O$_8$. {\it Phys. Rev. B} {\bf 56}, 877-885 (1997). 
%
%\bibitem{kubo1991} Kubo, Y., Shimakawa, Y.,  Manako, T., Igarashi, H. Transport and magnetic properties of Tl$_2$Ba$_2$Cu0$_{6+\delta}$ showing a $\delta$-dependent gradual transition from an 85~K superconductor to a nonsuperconducting metal. {\it Phys. Rev. B} {\bf 43}, 7875-7882 (1991).
%
%
%%%%%%%%%%%%%%%%%%%%%%%
%%%%%%%%%%%%%%%%%%%%%%%
%%%%%%%%%%%%%%%%%%%%%%%
%
%\bibitem{curty2003} Curty, P., Beck, H., Thermodynamics and phase diagram of high temperature superconductors. {\it Phys. Rev. Lett.} {\bf 91}, 257002 (2003).
%
%\bibitem{banerjee2011} Banerjee, S., Ramakrishnan, T. V., Dasgupta, C., Phenomenological Ginzburg-Landau-like theory for superconductivity in the cuprates. {\it Phys. Rev. B} {\bf 83}, 024510 (2011).
%
%\bibitem{noat2021} Noat, Y., Mauger, A., Nohara, M., Eisaki, H., Sacks, W., How ‘pairons’ are revealed in the electronic specific heat of cuprates. {\it Solid State Commun.} {\bf 323}, 114109 (2021).
%
%%%%%%%%%%%%%%%%%%%%%%
%
%\bibitem{moshe2019} Moshe, A. G., Farber, E., Deutscher, G., Optical conductivity of granular aluminum films near the Mott metal-to-insulator transition. {\it Phys. Rev. B} {\bf 99}, 224503 (2019).
%
%\bibitem{pisani2018} Pisani, L., Pieri, P., Strinati, G. C., Gap equation with pairing correlations beyond the mean-field approximation and its equivalence to a Hugenholtz-Pines condition for fermion pairs. {\it Phys. Rev. B} {\it Phys. Rev. B} {\bf 98}, 104507 (2018).
%
%%%%%%%%%%%%%%%%%%
%
%
%
%%%%%%%%%%%%%%%%%%
%
%
%\bibitem{polynomial} In the cuprates, we have used the smaller scatter of the locations of the maxima in $\gamma$ extracted from experimental data~\cite{supplemental} (plotted in Fig.~\ref{gap}d) to constrain the functional form of $\Delta$ versus $p$ in Fig.~\ref{gap}c by fitting. 
%Fitting yields $T_\gamma=T_0+T_1p+T_2p^2$, where $T_0=$~478~$\pm$~8~K, $T_1=$~-2970~$\pm$~110~K and $T_2=$~4590~$\pm$~380~K.
%
%%
%
%\bibitem{sutherland2003} Sutherland, M., Hawthorn, D.G., Hill, R.W., Ronning, F., Wakimoto, S., Zhang, H., Proust, C., Boaknin, E., Lupien, C., Taillefer, L., Liang, R., Bonn, D.A., Hardy, W.N., Gagnon, R., Hussey, N.E., Kimura, T., Nohara, M., Takagi, H., Thermal conductivity across the phase diagram of cuprates: Low-energy quasiparticles and doping dependence of the superconducting gap. {\it Phys. Rev. B} {bf 67}, 174520 (2003).
%
%\bibitem{mukhopadhyay2019} Mukhopadhyay, S., Sharma, R., Kim, C.K., Edkins, S.D., Hamidian, M.H., Eisaki, H., Uchida, S.-I., Kim, E.-A., Lawler, M.J., Mackenzie, A.P., Davis, J.C.S., Fujita, K., Evidence for a vestigial nematic state in the cuprate pseudogap phase. {\it Proc. Nat. Acad. Sci. USA} {\bf 116}, 13249-13254 (2019).
%
%%%%%%%%%%%%%%%%%
%
%\bibitem{engelbrecht1998} Engelbrecht, J. R., Nazarenko, A., Randeria, M., Dagotto, E., Pseudogap above $T_{\rm c}$ in a model with $d_{x^2-y^2}$ pairing. {\it Phys. Rev. B} {\bf 57}, 13406-13409 (1998).
%
%\bibitem{wyk2016} van Wyk, P., Tajima, H., Hanai, R., Ohashi, Y., Specific heat and effects of pairing fluctuations in the BCS-BEC-crossover regime of an ultracold Fermi gas. {\it Phys. Rev. A} {\bf 93}, 013621 (2016).
%
%
%\bibitem{enss2012i} Enss, T., Haussmann, R., Quantum mechanical limitations to spin diffusion in the unitary Fermi gas. {\it Phys. Rev. Lett.} {\bf 109}, 195303 (2012).
%
%\bibitem{tajima2014} Tajima, H., Kashimura, T., Hanai, R., Watanabe, R., Ohashi, Y., Uniform spin susceptibility and spin-gap phenomenon in the BCS-BEC-crossover regime of an ultracold Fermi gas. {\it Phys. Rev. A} {\bf 89}, 033617 (2014).
%
%\bibitem{loram1994} Loram, J. W., Mirza, K. A., Wade, J. M., Cooper, J. R., Liang, W. Y., The electronic specific heat of cuprate superconductors. {\it Physica C} {\bf 235-240}, 134-137 (1994).
%
%\bibitem{alloul1989} Alloul, H., Ohno, T., Mendels, P., $^{89}$ Y NMR evidence for a Fermi-liquid behavior in YBa$_2$Cu$_3$O$_{6+x}$. {\it Phys. Rev. Lett.} {\bf 63}, 1700-1703 (1989).
%
%\bibitem{johnston1989} Johnston, D.C., Magnetic susceptibility scaling in La$_{2-x}$Sr$_x$CuO$_{4-\delta}$. {\it Phys. Rev. Lett.} {\bf 62}, 957-960 (1989).
%
%
%\bibitem{nakano1994} Nakano, T., Oda, M., Manabe, C., Momono, N., Miura, Y., Ido, M., Magnetic properties and electronic conduction of superconducting La$_{2-x}$Sr$_x$CuO$_4$. {\it Phys. Rev. B} {\bf 49}, 16000-16008 (1994).
%
%
%\bibitem{vyaselev1994} Vyaselev, O. M., Kolesnikov, N. N., Schegolev, I. F., Transition from strong to weak coupling regime with lowering $T_{\rm c}$ in Tl$_2$Ba$_2$CuO$_{6+x}$. {\it Physica C} {\bf 235-240}, 1613-1614 (1994).
%
%
%\bibitem{crocker2011} Crocker, J., Dioguardi, A. P., apRoberts-Warren, N., Shockley, A. C., Grafe, H.-J., Xu, Z., Wen, J., Gu, G., Curro, N. J., NMR studies of pseudogap and electronic inhomogeneity in Bi$_2$Sr$_2$CaCu$_2$O$_{8+\delta}$. {\it Phys. Rev. B} {\bf 84}, 224502 (20110.
%
%\bibitem{alloul2016} Alloul, H. in {\it Quantum Materials: Experiments and Theory} (Pavarini, E., Koch, E., van den Brink, J., Sawatsky, G. eds.) 13.1-13.30 (Forschungszentrum J\"{u}lich GmbH Institute for Advanced Simulation, 2016). at $<$https://juser.fz-juelich.de/record/819465/files/correl16.pdf$>$
%
%\bibitem{susceptibilitynote} In the cuprates, the $T$-dependences of $\chi_{\rm m}$ and $K$ are dominated by the spin susceptibility $\chi$ at a sufficiently high $T$ compared to $T_{\rm c}$, and in a sufficiently strong magnetic field (in the case of $K$), enabling $\chi_{\rm m}$ and $K$ to be considered as representative of the $T$-dependence of $\chi$. 
%
%
%\bibitem{gapnote} In the unitary and BEC regimes of a Fermi gas, the pseudogap at $T>T_{\rm c}$ consists primarily of a gap of comparable energy to the low $T$ pairing gap that is extensively smeared by $T$-dependent line broadening effects~\cite{strinati2018,chen2014,zwerger2016}. The line broadening is primarily associated with the loss of phase coherence above $T_{\rm c}$. 
%%
%Extensive line broadening leads to a minimum in the density of states instead of a well defined gap, causing the maxima in $\gamma$ and $\chi$ to become less pronounced. 
%
%%%%%%%%%%%%%%%%%
%
%\bibitem{chen2000} Chen, Q. J. {\it Generalization of BCS theory to short coherence length superconductors: A BCS-Bose-Einstein crossover scenario}. Ph.D. thesis, University of Chicago (2000). (freely accessible in the ProQuest Dissertations \& Theses Database online).
%
%\bibitem{chen2001} Chen, Q, Levin, K., Kosztin, I., Superconducting phase coherence in the presence of a pseudogap: Relation to specific heat, tunneling, and vortex core spectroscopies. {\it Phys. Rev. B} {\bf 63}, 184519 (2001). 
%
%%
%
%
%
%%%%
%
%\bibitem{cooper2000} Cooper, J. R., Loram, J. W., The normal state gap and other strange properties of cuprate superconductors. {\it  J. Phys. IV France} {\bf 10}, Pr3-213-224 (2000).
%
%%%%%%%%%%%%%%%%%
%
%\bibitem{bozovic2016} Bo\u{z}ovi\'{c}, I., He, X., Wu, J., Bollinger, A. T., Dependence of the critical temperature in overdoped copper oxides on superfluid density. {\it Nature} {\bf 536}, 309-311 (2016).
%
%%
%
%
%%%%%%%%%%%%%%%%%
%
%\bibitem{cooper2009} Cooper, R. A., Wang, Y., Vignolle, B., Lipscombe, O. J., Hayden, S. M., Tanabe, Y., Adachi, T., Koike, Y., Nohara, M., Takagi, H., Proust, C., Hussey, N. E. Anomalous criticality in the electrical resistivity of La$_{\rm 2-x}$Sr$_{\rm x}$CuO$_4$. {\it Science} {\bf 323}, 603 (2009).
%
%\bibitem{giraldogallo2018} Giraldo-Gallo, P., Galvis, J. A., Stegen, Z., Modic, K. A., Balakirev, F. F., Betts, J. B., Lian, X., Moir, C., Riggs, S. C., Wu, J., Bollinger, A. T., He, X., Bo\u{z}ovi\'{c}, I., Ramshaw, B. J., McDonald, R. D., Boebinger, G. S., Shekhter, A., Scale-invariant magnetoresistancein a cuprate superconductor. {\it Science} {\bf 361}, 479-481 (2018).
%
%\bibitem{legros2019} Legros, A., Benhabib, S., Tabis, W., Lalibert\'{e}, F., Dion, M., Lizaire, M., Vignolle, B., Vignolles, D., Raffy, H., Li, Z. Z., Auban-Senzier, P., Doiron-Leyraud, N., Fournier, P., Colso, D., Taillefer, L., Proust, C., Universal $T$-linear resistivity and Planckian dissipation in overdoped cuprates. {\it Nature Phys.} {\bf 15}, 142-147 (2019).
%
%%
%
%
%%
%
%\bibitem{cao2010} Cao, C., Elliott, E., Joseph, J., Wu, H., Petricka,, J., Schafer, T., Thomas, J. E., Universal quantum viscosity in a unitary Fermi gas, {\it Science} {\bf 331}, 58-61 (2010).
%
%\bibitem{enss2012ii} Enss, T., Quantum critical transport in the unitary Fermi gas. {\it Phys, Rev. A} {\bf 86}, 013616 (2012).
%
%%
%
%\bibitem{pistolesi1996} Pistolesi, F., Strinati, G. C., Evolution from BCS superconductivity to Bose condensation:
%Calculation of the zero-temperature phase coherence length. {\it Phys. Rev. B} {\bf 53}, 15168-15192 (1996).
%
%\bibitem{engelbrecht1997} Engelbrecht, J. R., Randeria, M., S\'{a} de Melo, C. A. R., BCS to Bose crossover: Broken-symmetry state. {\it Phys. Rev. B} {\bf 55}, 15153-15156 (1997). 
%
%%
%
%
%%
%
%


%



\end{thebibliography}




\bibliography{basename of .bib file}

%



\end{document}
%
% ****** End of file apstemplate.tex ******

