\section{Introduction}
\label{sec:intro}
Recovering 3D shapes of real-world scenes is a fundamental problem in computer vision, and multi-view stereo~(MVS) has emerged as a mature geometric method for reconstructing dense scene points. Using 2D images taken from different viewpoints, MVS finds dense correspondences between images based on the photo-consistency assumption, that a scene point's brightness should appear similar across different viewpoints~\cite{multiview2007vogiatzis,multiview2007goesele,vu2011high, multiview2005vogiatzis}. However, MVS struggles with textureless or specular surfaces, as the lack of texture leads to ambiguities in establishing correspondences, and the presence of specular reflections violates the photo-consistency assumption~\cite{furukawa2015multi}.

Photometric stereo (PS) offers an alternative approach for dealing with textureless and specular surfaces~\cite{shi2019}. 
By estimating single-view surface normals using varying lighting conditions~\cite{woodham1980photometric}, PS enables high-fidelity 2.5D surface reconstruction~\cite{nehab2005efficiently}. However, extending PS to a multi-view setup, known as multi-view photometric stereo~(MVPS)~\cite{hernandez2008multiview}, significantly increases image acquisition costs, as it requires multi-view and multi-light images under highly controlled lighting conditions~\cite{li2020multi}. 

To mitigate image acquisition costs, simpler lighting setups such as circularly or symmetrically placed lights have been explored~\cite{alldrin2007toward,Zhou2010,chandraker2012differential,minami2022symmetric}. 
With these lighting setups, estimating the surface normal's azimuth (the angle in the image plane) becomes considerably easier than estimating the zenith (the angle from the camera optical axis)~\cite{alldrin2007toward,chandraker2012differential,minami2022symmetric}. 
The ease of azimuth estimation also appears in polarization imaging~\cite{rahmann2001reconstruction}.
While azimuth can be determined up to a $\pi$-ambiguity using only polarization data, zenith estimation requires more complex steps~\cite{smith2016linear,miyazaki2003polarizationtwoview,stolz2012shape}.

In this paper, we introduce Multi-View Azimuth Stereo~(MVAS), a method that effectively uses calibrated multi-view azimuth maps for shape recovery (\cref{fig.teaser}). 
MVAS is particularly advantageous when working with accurate azimuth acquisition techniques. 
With circular-light photometric stereo~\cite{chandraker2012differential}, MVAS has the potential to be applied to surfaces with arbitrary isotropic materials.
With polarization imaging~\cite{dave2022pandora}, MVAS allows a passive image acquisition as simple as MVS while being more effective for textureless or specular surfaces.

The key insight enabling MVAS is the concept of Tangent Space Consistency (TSC) for multi-view azimuth angles. 
We find that the azimuth can be transformed into a tangent using camera orientation.
Therefore, multi-view azimuth observations of the same surface point should be lifted to the same tangent space (\cref{fig.tsc}). 
TSC helps determine if a 3D point lies on the surface, similar to photo-consistency for finding image correspondences. 
Moreover, TSC can directly determine the surface normal as the vector orthogonal to the tangent space, enabling high-fidelity reconstruction comparable to MVPS methods. 
Notably, TSC is invariant to the $\pi$-ambiguity of the azimuth angle, making MVAS well-suited for polarization imaging.

With TSC, we reconstruct the surface implicitly represented as a neural signed distance function (SDF), by constraining the surface normals (\ie, the gradients of the SDF). Experimental results show that MVAS achieves comparable reconstruction performance to MVPS methods~\cite{kaya2022uncertainty, yang2022psnerf, park2016robust}, even in the absence of zenith information. Further, MVAS outperforms MVS methods~\cite{schoenberger2016mvs} in textureless or specular surfaces using azimuth maps from symmetric-light photometric stereo~\cite{minami2022symmetric} or a snapshot polarization camera~\cite{dave2022pandora}.

In summary, this paper's key contributions are:
\begin{itemize}
	\setlength{\itemsep}{0.2em}
	\setlength{\parskip}{0.2em}
	\item Multi-View Azimuth Stereo (MVAS), which enables accurate shape reconstruction even for textureless and specular surfaces;
	\item Tangent Space Consistency (TSC), which establishes the correspondence between multi-view azimuth observations, thereby facilitating the effective use of azimuth data in 3D reconstruction; and
	\item A comprehensive analysis of TSC, including its necessary conditions, degenerate scenarios, and the application to optimizing neural implicit representations.
\end{itemize}

\begin{figure}
	\centering
	\includegraphics[width=\linewidth]{TSC}
	\caption{Tangent space consistency. The azimuth can be converted to a tangent by camera orientation. The tangents in different views, but projected from the same surface point, should lie in the same tangent space and can directly determine the surface normal. }
	\label{fig.tsc}
\end{figure}