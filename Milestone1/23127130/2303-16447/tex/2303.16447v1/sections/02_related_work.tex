\section{Related Tasks and Concept}
\label{sec.related_work}

This section discusses the relation of MVAS to multi-view photometric stereo (MVPS) and shape-from-polarization (SfP), and compares TSC to photo-consistency.
\vspace{-1em}
\paragraph{MVPS versus MVAS}
MVPS aims for high-fidelity shape and reflectance recovery using images from different angles and under different lighting conditions~\cite{hernandez2008multiview, logothetis2019differential}.
These ``multi-light'' images can be used for estimating and fusing multi-view normal maps~\cite{chang2007multiview,kaya2022uncertainty}, for refining coarse meshes initialized by MVS~\cite{park2016robust}, or for jointly estimating the shape and materials in an inverse-rendering manner~\cite{yang2022psnerf}.

Compared to MVPS, MVAS has the potential to be applied to (1) surfaces of a broader range of materials and/or (2) in uncontrolled scenarios, benefiting from azimuth inputs.
First, azimuth estimation is valid for arbitrary isotropic materials using an uncalibrated circular moving light~\cite{chandraker2012differential}, while MVPS methods require specific surface reflectance modeling (\eg, Lambertian~\cite{chang2007multiview} or the microfacet model~\cite{yang2022psnerf}) or prior learning~\cite{kaya2022uncertainty}.
Second, MVAS allows passive image capture with polarization imaging, while MVPS has to actively illuminate the scene, limiting MVPS's application in highly controlled environments. 

\vspace{-1em}
\paragraph{SfP versus MVAS}
SfP recovers surfaces using polarization imaging~\cite{sonyPolar}. 
For dielectric surfaces, the measured angle of polarization (AoP) aligns with the surface normal's azimuth component, up to a $\pi$ ambiguity.
SfP studies determine surface normals by resolving this $\pi$-ambiguity and estimating the zenith component~\cite{smith2016linear, smith2018height,drbohlav2001unambiguous,fukao2021polarimetric,ding2021polarimetric,kadambi2015polarized,kadambi2017depth,rahmann2001reconstruction,zhu2019depth}.
Some studies use polarization data to refine coarse shapes initialized by multi-view reconstruction methods~\cite{Cui_2017_CVPR,zhao2020polarimetric}, but the geometric relation between multi-view azimuth angles are not considered.

With TSC and MVAS, both the $\pi$-ambiguity and zenith estimation can be bypassed.
Our method relies on TSC, not requiring MVS methods to initialize shapes.

\vspace{-1em}
\paragraph{Photo-consistency versus tangent space consistency}
Photo-consistency is a key assumption in MVS for establishing correspondence between multi-view images. This assumption states that a scene point appears similar across different views and struggles with specular surfaces~\cite{furukawa2009accurate}.

In contrast, TSC is derived from geometric principles and strictly holds for multi-view azimuth angles. Further, TSC can determine the surface normal, providing more information than photo-consistency. However, TSC requires at least three cameras with non-parallel optical axes and can degrade to photo-consistency under certain camera configurations. 
Similar to photo-consistency's challenges with textureless surfaces, TSC might struggle to establish correspondences for planar surfaces.
Details are in \cref{sec.functional}.