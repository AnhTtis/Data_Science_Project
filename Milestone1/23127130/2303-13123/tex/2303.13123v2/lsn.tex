% This is samplepaper.tex, a sample chapter demonstrating the
% LLNCS macro package for Springer Computer Science proceedings;
% Version 2.21 of 2022/01/12
%
\documentclass[runningheads]{llncs}
%
\usepackage[T1]{fontenc}
% T1 fonts will be used to generate the final print and online PDFs,
% so please use T1 fonts in your manuscript whenever possible.
% Other font encondings may result in incorrect characters.
%
\usepackage{tabularray}
%\UseTblrLibrary{booktabs}
\usepackage{booktabs}
\usepackage[export]{adjustbox}
\usepackage{capt-of}
\usepackage{xparse}
\usepackage{comment}
\usepackage{graphicx}
\usepackage{amsmath}
\usepackage{mathtools}
\usepackage{amssymb}
\usepackage{multirow}
\usepackage[dvipsnames]{xcolor}
%\usepackage{unicode-math}
\usepackage{pifont}% http://ctan.org/pkg/pifont
\usepackage{subcaption}
\usepackage{hyperref}       % hyperlinks
\usepackage{url}            % simple URL typesetting

\usepackage{algorithm} %added for pseudocode
\usepackage{algpseudocode} %added for pseudocode

\newcommand{\xmark}{\ding{55}}%
% Redefine \xmark to use a different color, e.g., blue

% Redefine \greencheck to use a different color, e.g., blue
%\newcommand{\greencheck}{\textcolor{BurntOrange}{\ding{51}}}

% Redefine \yellowcheck to use a different color, e.g., red
%\newcommand{\yellowcheck}{\textcolor{ForestGreen}{\ding{51}}}


% Redefine \greencheck for use in both text and math modes
\newcommand{\greencheck}{\ifmmode\text{\textcolor{BurntOrange}{\ding{51}}}\else\textcolor{BurntOrange}{\ding{51}}\fi}

% Redefine \yellowcheck for use in both text and math modes
\newcommand{\yellowcheck}{\ifmmode\text{\textcolor{ForestGreen}{\ding{51}}}\else\textcolor{ForestGreen}{\ding{51}}\fi}

\newcommand\blfootnote[1]{%
  \begingroup
  \renewcommand\thefootnote{}\footnote{#1}%
  \addtocounter{footnote}{-1}%
  \endgroup
}

\newcommand{\littletaller}{\mathchoice{\vphantom{\big|}}{}{}{}}
\newcommand{\restr}[2]{{% we make the whole thing an ordinary symbol
  \left.\kern-\nulldelimiterspace% automatically resize the bar with \right
  #1% the function
  \littletaller% pretend it's a little taller at normal size
  \right|_{#2}% this is the delimiter
  }}
\newcommand*\diff{\mathop{}\!\mathrm{d}}
\newcommand{\MAP}{\textsc{map}}

% Used for displaying a sample figure. If possible, figure files should
% be included in EPS format.
%
% If you use the hyperref package, please uncomment the following two lines
% to display URLs in blue roman font according to Springer's eBook style:
%\usepackage{color}
%\renewcommand\UrlFont{\color{blue}\rmfamily}
%\urlstyle{rm}
%


\begin{document}
%
\title{Laplacian Segmentation Networks Improve Epistemic Uncertainty Quantification}
%
\titlerunning{Laplacian Segmentation Networks}
% If the paper title is too long for the running head, you can set
% an abbreviated paper title here
%
\author{Kilian Zepf\inst{1*}  \and
Selma Wanna\inst{2*} \and
Marco Miani\inst{1} \and
Juston Moore\inst{2} \and
Jes Frellsen\inst{1} \and
Søren Hauberg\inst{1} \and
Frederik Warburg\inst{3} \and
Aasa Feragen\inst{1} }

% index{Zepf, Kilian}
% index{Wanna, Selma}
% index{Miani, Marco}
% index{Moore, Juston}
% index{Frellsen, Jes}
% index{Hauberg, Søren}
% index{Warburg, Frederik}
% index{Feragen, Aasa}

\authorrunning{K. Zepf et al.}
% First names are abbreviated in the running head.
% If there are more than two authors, 'et al.' is used.
%
\institute{Technical University of Denmark, Kongens Lyngby, Denmark \email{\{kmze,mmia,jefr,sohau,afhar\}@dtu.dk} \and
Los Alamos National Laboratory, Los Alamos, USA \email{\{slwanna,jmoore01\}@lanl.gov} \and
Teton.ai, Copenhagen, Denmark \email{frederik@teton.ai}\\
* denotes equal contribution
}
%
\maketitle              % typeset the header of the contribution
%
\begin{abstract}
Image segmentation relies heavily on neural networks which are known to be overconfident, especially when making predictions on out-of-distribution (OOD) images. This is a common scenario in the medical domain due to variations in equipment, acquisition sites, or image corruptions. This work addresses the challenge of OOD detection by proposing Laplacian Segmentation Networks (LSN): methods which jointly model epistemic (model) and aleatoric (data) uncertainty for OOD detection. In doing so, we propose the first Laplace approximation of the weight posterior that scales to large neural networks with skip connections that have high-dimensional outputs. We demonstrate on three datasets that the LSN-modeled parameter distributions, in combination with suitable uncertainty measures, gives superior OOD detection.

\keywords{Uncertainty Quantification  \and Image Segmentation}
\end{abstract}
%
%
%

\section{Introduction}
\section{Introduction}

The ability to reason about plans is critical for performing long-horizon tasks \citep{erol1996hierarchical, sohn2018hierarchical, sharma-etal-2022-skill}, compositional generalization \citep{corona-etal-2021-modular} and generalization to unseen tasks and environments \citep{shridhar2020alfred}.
Consider a simple long-horizon planning scenario where a robot is tasked with preparing a meal and serving it on the table. 
This presents a non-trivial planning problem since the agent needs to understand the sequence of operations required to perform the task and search for the relevant objects in the unfamiliar environment by interacting with various objects. %



Large language models have been recently shown to possess commonsense knowledge about the world such as object affordances and physical dynamics \citep{ouyang2022training,chowdhery2022palm}.
Early approaches considered text based environments and fine-tuned PLMs to predict actions given the history of past observations and actions \citep{jansen-2020-visually,micheli-fleuret-2021-language,yao-etal-2020-keep}.
Recent work has used this ability to reason about plans from text instructions in simulated household environments with simplifying assumptions such as text-only environment observations or feedback \citep{huang2022language,ahn2022can,li2022pre,logeswaran-etal-2022-shot}.


We focus on \emph{visually grounded planning} with PLMs --- the ability to adapt plans based on interaction and visual feedback from the environment.
While PLMs have strong planning commonsense priors, predictions from a PLM may not be directly realizable in the environment since the observation and action spaces are unknown.
This requires \emph{grounding} the PLM in the environment and adapting it to observe visual feedback, which is highly non-trivial.
Some prior works assume the availability of a pre-trained affordance function \citep{ahn2022can} or a success detector \citep{mirchandani2021ella}.
Notably, SayCan \citep{ahn2022can} completely decouples the PLM from observation information by selecting actions that have both high affordability (through a pre-trained affordance model) and high PLM likelihood.
Although this partially addresses the grounding problem, the use of visual feedback for action affordance alone is limited.
Often an agent must choose one of many affordable actions using information from observations.
For example, a driving agent should re-navigate and possibly turn around when encountering a ``road closed'' sign, but both turning around and driving forward are indistinguishable to SayCan because they are both affordable and the PLM is blind to observations.

Another workaround explored in prior work is translating the information in the visual observations to text using a pre-trained captioning system \citep{shridhar2021alfworld,huang2022language}.
However, it can be difficult to faithfully describe an image in words and information is lost in this inherently noisy process, which limits the information available to the planner.



Recent work shows that PLMs can be adapted for various natural language tasks by inserting tunable embeddings or soft prompts at the input of the PLM (also called prompt tuning or prefix tuning)~\citep{li-liang-2021-prefix,lester-etal-2021-power}.
This approach also extends to multi-modal understanding tasks such as image captioning \citep{mokady2021clipcap} and VQA \citep{tsimpoukelli2021multimodal} where images are encoded as soft prompts and finetuned for the target task.
Transformer based architectures have also been successfully applied to offline Reinforcement Learning in recent work \citep{chen2021decision,janner2021offline,li2022pre,reid2022can}.

Taking inspiration from these works, we propose the simple approach of embedding visual observations (`visual prompts') and \textit{directly inserting them as PLM input embeddings}.
The visual encoder and PLM are jointly trained for the target task, an approach we call \textbf{\oursfull}~(\ours).
By teaching the PLM to use observations for planning in an end to end manner, we remove the dependency on external data such as captions and affordability information that was used in prior work.
We show that this simple approach performs better than prior PLM-based planning approaches on two embodied planning benchmarks based on ALFWorld~\citep{shridhar2021alfworld} and Virtualhome~\cite{puig2018virtualhome}.



\section{Background}
\section{Related Work}

%Here we summarize prior work on transfer learning and property inference.

%\shortsection{Transfer Learning}
%%Transfer learning reuses features learned by pre-trained models for new tasks, with the pretext that inherent similarities in the generic features will be useful for the downstream tasks and hence reducing their cost of downstream training. Specifically, the downstream model trainer will use a pre-trained upstream model as the starting point for the downstream training, with inclusion of (or replacement with) the task-specific classification layer/module. The downstream model is then trained by either updating all layers of the model (including ones reused from upstream model) or freezing some earlier layers of the reused parts as the ``feature extractor'' and only updating the rest. The latter approach is more popular as the reused feature extractors can already learn useful feature representations and the training cost is also much lower and affordable for individuals with limited computational resources. We study the vulnerability of the latter transfer learning approach in this paper. 


%\shortsection{Transfer Learning} 
Several works have demonstrated risks associated with transfer learning across a variety of attack goals. Wang et al.~\cite{wang2018great} and Yao et al.~\cite{yao2019latent} consider manipulating the upstream model such that the fine-tuned downstream models contain backdoors, misclassifying test inputs that contain predefined backdoor triggers. These transfer manipulations are tailored to their particular attack goals and cannot be applied for the property inference goal considered in this paper. Zou et al.~\cite{zou2020privacy} study the threat of membership inference attacks on transfer learning, but with normally trained upstream models.  
%\dnote{its clear that the goals are different for these attacks, but how similar are the methods?} \ynote{similarity of the methods? more details about the methods? do not know what is expected here}
%In contrast, we investigate the possibility of boosting the effectiveness of property inference by manipulating the upstream model training. % Schuster et al.~\cite{schuster2020humpty} show that the attacker can modify the corpus on which the word embedding is trained such that the downstream NLP models which use that embedding will behave abnormally.

%\shortsection{Property Inference}
The risk of property inference was introduced by Ateniese et al.~\cite{ateniese2015hacking}, % introduces the threat of inferring properties of the training data from pre-trained models, 
and several subsequent works have developed property inference (also known as distribution inference) attacks~\cite{Wang2022GroupPI, suri2022formalizing, Jurez2022BlackBoxAF, Hartmann2022DistributionIR}.
% Ganju et al.~\cite{ganju2018property} and Suri and Evans~\cite{suri2022formalizing} 
These works study property inference against normally trained models, and they launch attacks using a variety of black-box and white-box attacks. All the white-box attacks use meta-classifiers, which take the permutation-invariant representation~\cite{ganju2018property} of the model parameters as the features. We use the state-of-the-art white-box attack~\cite{suri2022formalizing} in our experiments.
%We will use the state-of-the-art white-box method proposed by Ganju et al.~\cite{ganju2018property} and later extended by suri et al.~\cite{suri2022formalizing} in this paper.
%\dnote{do we use these attacks?} 
Melis et al.~\cite{melis2019exploiting} and Zhang et al.~\cite{zhang2021leakage} focus on property inference in distributed training scenarios. In their settings, the attacker is a participant in the global model training and conducts property inference using meta-classifiers that are trained on model outputs or gradients. Similarly, Suri et al.~\cite{suri2022subject} focus on federated learning settings where the attacker is a participant (or the central server) that utilizes black-box attacks for inferring membership of data from particular subjects. %\dnote{if we use black-box attacks, explain which ones, or how ours are related to previous ones} 
For our experiments, We improve the black-box meta-classifier proposed by Zhang et al.~\cite{zhang2021leakage} using the ``query tuning'' technique in Xu et al.~\cite{xu2019detecting}. 

The closest works to ours are Chase et al.~\cite{saeed} and Chaudhari et al.~\cite{Chaudhari2022SNAPEE}, which both consider a scenario where the attacker can manipulate some of the training data of the model to induce a model that significantly increases property inference risk.
% \dnote{it enables precise property inference attacks?}.
These works assume an adversary with the ability to poison the victim's training data, while the adversary in our scenario has no access to the victim's training data, and therefore, their methods are not applicable.
% \dnote{example how different from ours, and why the methods are not applicable}
%Thus, their methods are not applicable to our transfer learning scenario.
%Their methods rely on inducing certain behavior correlated with the properties to be inferred, and thus are not applicable to our transfer learning scenario. \anote{Still a bit unclear why that is the case.}
%
There are also works similar to ours that leverage ``adversarial initializations'' for attack purposes.
% \cite{grosse2019adversarial, boenisch2021curious, wen2022fishing, fowl2021robbing}.
Grosse et al.~\cite{grosse2019adversarial} focus on scenarios where the attacker can control the parameter initialization of a model, and demonstrate that the attacker can use special initializations to damage the performance of the trained model. %This attack is orthogonal to ours.
Other works \cite{boenisch2021curious, wen2022fishing, fowl2021robbing} show that the malicious central server in a federated learning protocol can reconstruct some training samples via falsifying the global model in some training rounds and then analyzing the submitted gradients. These kinds of attacks do not apply to our transfer-learning scenario since the attacker cannot access the downstream gradients, and can only manipulate the upstream training.

\iffalse %%%%%%%%%%%%%%%%%%%%%%%%%%%%%%%%

In this section, we provide the background and also the summary of prior attacks on transfer learning (Section~\ref{sec:transfer_learning}) and property inference (Section~\ref{sec:property_inference}). Then, we introduce the closely related manipulation attacks against machine learning models to boost different privacy risks in Section~\ref{sec:active_inference_attacks}.

%\anote{Do we really need a dedicated section for this? It's barely 2 paragraphs right now.}

%\dnote{the most closely related work to ours are works that attempt to amplify inference attacks by poisoning models, the two most relevant I know of are \url{https://www.computer.org/csdl/proceedings-article/sp/2022/131600b569/1CIO8nmuota} and \url{https://arxiv.org/abs/2204.00032}, but need to look thoroughly for others. We should definitely be describing this and relating it to our work, probably in the introduction. Most of what is here is Background, but should be clear what this section is for (not muddling background and related work)}

\subsection{Transfer Learning} \label{sec:transfer_learning}
Transfer learning reuses features learned by pre-trained models for new tasks, with the pretext that inherent similarities in generic features can be useful for downstream tasks, thus reducing the cost of downstream training. Specifically, the downstream model trainer uses a pre-trained upstream model as the starting point for downstream training, with the inclusion (or replacement) of task-specific classification layers/modules. The downstream model is then trained by either updating all layers of the model (including ones reused from the upstream model) or freezing some earlier layers of the reused parts as the ``feature extractor'' and only updating the rest. The latter approach is more popular as the reused feature extractors can already learn useful feature representations and the training cost is also much lower and affordable for individuals with limited computational resources. We study the vulnerability of the latter transfer learning approach in this paper. 
%mainly in two ways:  1) all the layers (including ones reused from ) and tune the full model; the other one is to freeze some earlier layers of the model as the feature extractor and only tune the rest later layers. The second update strategy could achieve better efficiency since the frozen layers can already produce meaningful feature representations~\cite{wang2018great,yao2019latent}, and we will study the transfer learning using this strategy. 

Recently, various attacks have been proposed for the transfer learning setting, but with different attack goals from ours. Wang et al.~\cite{wang2018great} generate adversarial examples against black-box student models that transfer knowledge from publicly available teacher models without repeated queries. Yao et al.~\cite{yao2019latent} propose to manipulate the upstream model such that the downstream models derived from the upstream model contain backdoors, which would misclassify test inputs that contain some predefined backdoor triggers. Zou et al.~\cite{zou2020privacy} study the threat of membership inference attacks on transfer learning and the upstream models are trained normally. In contrast, we investigate the possibility of boosting the effectiveness of property inference by manipulating the upstream model training. Schuster et al.~\cite{schuster2020humpty} show that the attacker can modify the corpus on which the word embedding is trained such that the downstream NLP models which use that embedding will behave abnormally.

%This additionally allows model trainers to achieve satisfactory performance with limited training samples, leading to reduced computational costs. The most common approach reuses parameters in the earlier layers of the pre-trained model, either by fixing them as the feature extractor or just using them for initialization, to conduct downstream training.

\subsection{Property Inference} \label{sec:property_inference}

\shortsection{Property Inference Attacks} In property inference attacks, the adversary aims to infer some sensitive properties of some data, given a model trained on it. For example, the adversary may be interested in sensitive properties like the presence of people of a specific race in the dataset~\cite{ateniese2015hacking, melis2019exploiting}), or even be curious about the 
the statistics of the training set (e.g, the ratio of people with a specific gender~\cite{saeed, ganju2018property, suri2022formalizing, zhang2021leakage}).


Ateniese et al.~\cite{ateniese2015hacking} were the first to identify the threat of inferring properties of the training data from pre-trained models. Ganju et al.~\cite{ganju2018property} and Suri and Evans~\cite{suri2022formalizing} 
study property inference against normally trained models, and they launch attacks using white-box meta-classifiers, which utilize the permutation-invariance representation~\cite{ganju2018property} of the model parameters, while other works focus on distributed training~\cite{zhang2021leakage} where the attacker is a participant in the global model training and conducts property inference using meta-classifiers trained on model outputs. Similarly, Suri et al.~\cite{suri2022subject} focus on federated learning, where the attacker is a participant (or the central server) that utilizes black-box attacks for inferring membership of data from particular subjects. Chase et al.~\cite{saeed} propose an active property inference attack for data poisoning scenarios, which we will cover and compare to in Section~\ref{sec:active_inference_attacks}.

%The closest work to ours are by Chase et al.~\cite{saeed} and Tramer et al.~\cite{tramer2022truth}. In their work, the attacker can manipulate some of the training data of the model such that a model trained (from scratch) on the poisoned data has an increased inference risk. However, their methods are not applicable to the transfer learning scenario. 
%In this work, we will focus on the property inference in transfer learning scenarios in which the attacker releases the upstream model and infer sensitive properties of the downstream models tuned from that upstream model.
% 

\shortsection{Defenses}
Defending against property inference attacks is an open problem. There are no studies in the current literature on active adversaries, and only a couple on passive ones. Ma et. al.~\cite{ma2021nosnoop} propose a defense against property inference attacks on data batches in the  collaborative learning setting. However, adversaries in the transfer-learning setting do not have access to batch-wise gradients of the downstream trainer. Chen and Ohrimenko~\cite{chen2022protecting} utilize mechanisms that add carefully-crafted noise to features to provide theoretical guarantees against inference adversaries, but focus on query-based access to the underlying dataset, not a machine learning model trained on it. These existing defenses thus do not apply to our threat model.

%propose a framework that reduces property inference to Boolean functions of individual members, posing the ratio of members satisfying the given function in a dataset as the property. These property inference attacks have since then been proposed as distribution inference attacks~\cite{suri2022formalizing}, presenting such attacks as inferring properties of the distributions used to sample datasets, differentiating them from exact inference attacks like dataset inference~\cite{maini2021dataset}. Nearly all property inference attacks use meta-classifiers to perform inference: training models on versions of datasets with and without the target property, followed by training a meta-classifier on top of these classifiers's model representations. These representations can take several forms: using model weights themselves with permutation-invariance~\cite{ganju2018property}, or model activations or logits for a generated set of query points~\cite{xu2019detecting}. However, the capability of such approaches is limited: the most that these attacks have been shown to work is medium-sized convolutional networks on the CelebA dataset~\cite{suri2022formalizing}.


\subsection{Active Privacy Attacks} \label{sec:active_inference_attacks}
% Perhaps the closely related works to ours as ones that proactively enhance the effectiveness of privacy attacks by manipulating the model training process in certain ways~\cite{saeed, melis2019exploiting, nasr2019comprehensive, tramer2022truth}. 
%shown that the adversary can, by using proactive ways, achieve stronger attacks that infer private information from deep learning systems~\cite{nasr2019comprehensive, melis2019exploiting, tramer2022truth, saeed}. In this section, we introduce the ones that are close to ours.

In the decentralized federated learning training, by submitting specially crafted gradients to the central server, malicious agents can increase membership inference risk~\cite{nasr2019comprehensive} and property inference risks~\cite{melis2019exploiting} of other benign agents' training data. However, these attacks do not apply to transfer learning scenario, as the attacker cannot control model gradients of downstream training. In the centralized setting, researchers propose attacks to poison the victim's training data such that the impacts of attribute inference and membership inference~\cite{tramer2022truth} and property inference~\cite{saeed} attacks are amplified on the poisoned model.
The ability to poison the victim's data is a threat model orthogonal to ours, since we have no access to the victim's downstream data. While there is scope to combine such approaches for stronger attacks (albeit with stronger access assumptions), we choose to focus on the scenario with no read/write access to the victim's data.

\fi %%%%%%%%%%%%%%%%%%%%%%%%%%%%%%%%

\section{The Laplacian Segmentation Networks}
To model the posterior distribution in Eq.~\eqref{pred_dist} we apply Laplace's method which approximates the weight posterior with a Gaussian distribution $q(\theta)$ around a local mode $\theta_{\MAP}$ using the Hessian matrix $\mathbf{H}$ \cite{mackay1992bayesian}
\begin{equation}
   q(\theta)= \mathcal{N}(\theta \vert \theta_{\MAP}, \mathbf{H}^{-1}). 
\end{equation}
Evaluating $\mathbf{H}$ is computationally infeasible because of the quadratic complexity in network parameters and the large output dimensions for segmentation. We improve upon Hessian approximation techniques \cite{daxberger2021laplaceredux,botev2020} by extending recent progress in scaling LA for images \cite{miani_2022_neurips} to segmentation networks with skip connections.
\subsection{Laplace Approximation of the Mean Network}
We can reformulate the integral for the predictive distribution over the binary predictions $y$ in Eq.~\eqref{pred_dist} by integrating over logits $\eta$ to obtain 
\begin{equation}
    \label{pred_dist_with_logits}
 p(y \vert x, D) = \iint p(y \vert \eta) p(\eta \vert x,\theta) p(\theta \vert D) \diff \eta \diff \theta.
\end{equation}
Following \cite{kendall2017} and \cite{monteiro2020}, we model the conditional distribution over logits $p(\eta \vert x,\theta)$ as a normal distribution parametrized by neural networks $\mu$ and $\Sigma$ :
\begin{equation}
    \label{logit_dist}
    \eta \vert x \sim \mathcal{N}(\mu(x, \theta_1),\Sigma(x,\theta_2)),
\end{equation}
and assume pixel-wise independence for the predicted labels given the logits. Thus, we can model $p(y \vert \eta)$ for each pixel $s$ as a Bernoulli distribution parametrized by the sigmoid of the respective logit. 
Since the size of the covariance matrix $\Sigma$ scales quadratically with the number of pixels $S$ in the image, we use the low-rank parameterisation of \cite{monteiro2020}:
\begin{equation}
    \Sigma(x) = D(x) + P(x)^T P(x),
\end{equation}
i.e.~the variance network $\Sigma(x)$ is implemented with two networks $D(x)$ and $P(x)$. 

The vectors $\theta_1 \in \Theta_1 = \mathbb{R}^T$ and $\theta_2 \in \Theta_2 = \mathbb{R}^T$ parameterize the mean and variance networks (c.f.\@ Eq.~\ref{logit_dist}) and share the first $t$ entries, i.e.we define the shared weight vector $\theta_t$ of the network by 
\begin{equation}
    \theta_{t} \coloneqq (\theta_{1_1}, \ldots ,\theta_{1_t} ) = (\theta_{2_1}, \ldots ,\theta_{2_t} ) \in \Theta_t =\mathbb{R}^t.
\end{equation}
Then $\theta \in \Theta = \mathbb{R}^{(t+2\cdot(T-t))}$ contains all model parameters
\begin{equation}
    \theta \coloneqq (\theta_t, \theta_{1_{t+1}}, \ldots, \theta_{1_{T}}, \theta_{2_{t+1}}, \ldots, \theta_{2_{T}}).
\end{equation}
The post-hoc Laplace approximation first finds a mode $\theta_{\MAP}$ by minimizing 
\begin{multline}
    \mathcal{L}(\theta) = - \log \mathbb{E}_{p(\eta \vert x, \theta)} [p(y \vert \eta)] - \log p(\theta) \approx  \\-\text{logsumexp}_{m=1}^M \left(\sum_{s=1}^S \log p(y_s \vert \eta_s^{(m)})\right) + \log(M) ,
\end{multline}
where $M$ logits $\eta$ are sampled from the distribution in Eq.~\eqref{logit_dist} and where the term $\log p(\theta)$ vanishes assuming a flat prior $\nabla_{\theta} p(\theta) = 0$. 
Since current algorithms for fast Hessian computations have no implementation for this loss function, we instead make use of the shared weights in the parameter vectors to estimate the mean and variance of the logit distribution based on the feature maps of a deep deterministic segmentation model. Using only one convolutional layer each for mean and variance estimation, we omit the entries of the variance heads on the parameter vector $\theta_{\MAP}$, i.e.\@ we set 
\begin{equation}
   \theta_{\MAP}^* \coloneqq \restr{\theta_{\MAP}}{(\theta_t, \theta_{1_{t+1}}, ..., \theta_{1_{T}})} \in \Theta_{\textrm{mean}} =\mathbb{R}^{T}.
\end{equation}
We can make use of the fact that the SSN loss function reduces to the binary cross entropy loss under zero variance, which allows us to fall back on the fast Hessian computation frameworks available. The posterior is then found by Laplace's method resulting in a Gaussian approximation in the parameter space $\Theta_{\textrm{mean}}$
\begin{equation}
    q(\theta^{*})= \mathcal{N}\left(\theta^{*} \vert \theta_{\MAP}^{*}, \mathbf{H^{*}}^{-1}\right), 
\end{equation}
with $\mathbf{H^*}$ defined as 
$
    \mathbf{H^*} = - \nabla_{\theta^{*}} \nabla_{\theta^{*}} \log \restr{p(\theta^{*} \vert D)} {\theta^{*} = \theta_{\MAP}^{*}}.
$
During inference we can now sample segmentation networks from the posterior distribution in form of the Laplace approximation. Each sampled segmentation network predicts one logit distribution.  Figure~\ref{fig:model_overview} gives an schematic overview of the proposed Laplacian Segmentation Network (LSN) and derived uncertainty measures. 
\begin{figure*}[t]
\begin{center}
\includegraphics[width=0.8\linewidth]{model_overview_grey.png}
%\vspace*{-\baselineskip}
\end{center}
   \caption{Model overview - uncertainty measures are calculated by approximating expectations by Monte Carlo-sampling mean networks from the Laplace approximation $q(\theta^*)$ and predicting the respective logit distributions $p(\eta \vert x, \theta)$ for $x$. }
\label{fig:model_overview}
%\vspace*{-\baselineskip}
\end{figure*}
\subsection{Fast Hessian Approximations for Segmentation Networks with Skip Connections}\label{sec:fastH}
Computation of second order derivatives for Segmentation Networks is expensive due to the vast amount of parameters and pixels in the output. Standard methods approximate the Hessian with the diagonal of the Generalized Gauss Newton (\textsc{ggn}) matrix \cite{foresee1997ggn,botev2020}. This approximation, besides enforcing positive definiteness, also allows for an efficient backpropagation-like algorithm. The required compute scales linearly in the number of parameters and quadratic in the number pixels. The quadratic dependency is prohibitive already with images of size $64 \times 64$. We therefore make use of the diagonal backpropagation ($\textsc{db}$) proposed by \cite{miani_2022_neurips}, which returns a trace-preserving approximation of the diagonal of the \textsc{ggn}. The complexity of this approximation scales linearly with the number of pixels, allowing the computation of the Hessian also for larger images. The idea is to add a diagonal operator $\mathcal{D}$ in-between each backpropagation step. For each layer $l$
\begin{align}
    [\nabla_{\theta} & \nabla_{\theta} \log p(\theta \vert D)]_l
    \! \overset{\textsc{ggn}}{\approx} \!
    [J_\theta f_\theta (x)^\top \textbf{H}^{(L)} J_\theta f_\theta(x)]_l = \\ \nonumber
    & =
    J_\theta {f^{(l)}}^\top
        \left(
        \prod_{i=l+1}^L J_x {f^{(i)}}^\top
        \textbf{H}^{(L)}
        \prod_{i=L}^{l+1} J_x f^{(i)}
        \right)
    J_\theta f^{(l)} \\ \nonumber
    & \overset{\textsc{db}}{\approx}
    J_\theta {f^{(l)}}^\top
        \mathcal{D}
        \left(
        J_x {f^{(l+1)}}^\top
        \mathcal{D}
            \left(
            \dots
            \right)
        J_x f^{(l+1)}
        \right)
    J_\theta f^{(l)}
\end{align}
where $J_\theta$ denotes the Jacobian and $\textbf{H}^{(L)}$ the Hessian of the binary cross entropy loss with respect to the logits. The Hessian matrix can be expressed in closed form as a diagonal matrix plus an outer product matrix.

Moreover, we extend the \texttt{StochMan} library \cite{software:stochman} with support for skip-connection layers. For a given submodule $f_\theta$, a skip-connection layer $\textsc{sc}_f$ concatenates the function with the identity, such that $\textsc{sc}_f(x) = (f_\theta(x), x)$. The Jacobian is then defined as $J_x \textsc{sc}_f(x) := (J_x f_\theta(x), \mathbb{I}_x) $. We utilize the block structure of the Jacobian matrix and efficiently backpropagate its diagonal only. With a recursive call on the submodule $f$, the backpropagation supports nested skip-connections, i.e.\@ when some submodules of $f$ are skip-connections as well. This unlocks the use of various curvature-based methods for segmentation architectures with skip connection in future research. For a technical description of the used Hessian approximation we refer to the supplementary material.
\section{Experiments}
\section{Experiments}
\label{sec:experiments}

\subsection{Setup}
\textbf{Datasets.} We evaluate RFFR with four challenging datasets specifically designed for deepfake detection. We adopt the high quality (HQ) version of Faceforensics++ (FF)~\cite{ff} for training our deepfake detector. Faceforensics++ includes videos of real faces as well as four subsets of fake faces, each manipulated with a different algorithm, namely Deepfakes (DF), Face2Face (F2F), FaceSwap (FSW) and NeuralTextures (NT). We also utilize the test set of Celeb-DF~\cite{celeb-df} and DFDC~\cite{dfdc} for evaluating the cross-dataset performance of our model. Finally, in addition to real faces of Faceforensics++, we adopt the real face images from ForgeryNet (FN)~\cite{forgerynet} for learning RFFR, which helps improve representation learning with additional data.

\textbf{Implementation Details.} We extract the frames from all video datasets and use RetinaFace~\cite{retinaface} to detect and align the faces. All images are scaled to the size of $224 \times 224$. For our RFFR model, we adopt a base version of Masked Autoencoder (MAE)~\cite{mae} and train it on real faces with a batch size of $128$. Following MAE, we set the learning rate at $7.5 \times 10^{-5}$ and adjust it with a schedule with warmup and cosine decay. By default, we train this model with the real faces from both FF~\cite{ff} and FN~\cite{forgerynet}. 

For training the deepfake detector, we divide each image with $k = 4$ (Refer to Appendix for the motivation of choosing $k$). Each block enters the classifier with a probability of $p = 0.25$, and the residual images are amplified by $\alpha=4$. No data augmentation is applied to the images. We initialize both branches of Vision Transformer with ImageNet-pretrained weights and train them with a learning rate of $2 \times 10^{-5}$. During testing, we iteratively mask and restore all blocks to obtain a full residual image for the detector to process. We evaluate the testing results with AUC (Area Under Curve). 

\subsection{Cross-domain performance evaluation}
In this section, we test the performance of our RFFR-based deepfake detector with cross-manipulation and cross-dataset evaluations. 

\textbf{Cross-manipulation evaluations.} We train our deepfake detector on each subset of Faceforensics++ and test on all four subsets to demonstrate our model's ability to identify different manipulations, including those not seen during training. \emph{We adopt the HQ version of FF for both training and testing, and only use one frame every video for testing.} We compare our results with state-of-the-art image-based methods Multi-Attention~\cite{multiatt}, DCL~\cite{dcl}, RECCE~\cite{recce} and UIA-ViT~\cite{uia}. We ran the public code of RECCE and UIA-ViT to produce results under the same setting.

In~\cref{tab:cross-manipulation}, we show that our method outperforms the state-of-the-art methods under most settings, with a maximum improvement of $10.25\%$ (F2F $\rightarrow$FSW). Meanwhile, our model remains effective under the four intra-domain settings, which are shown in gray. The method tends to slightly underperform when trained on NeuralTextures, likely because its manipulation patterns only exist in certain small regions, and may be neglected during our block sampling. Nevertheless, compared to existing methods, our deepfake detector yields much better overall performances. 

\begin{table}[t]
\setlength\tabcolsep{4.5pt} 
\caption{Cross-manipulation performances in terms of AUC(\%) compared with previous methods. Classifiers are trained on one subset of FF and tested on all four subsets. Intra-domain results are marked in gray. We ran the public code of methods marked with "*" to produce results under identical settings \emph{(HQ for training and single frames for testing).}}
\vspace{-1.5em}
\label{tab:cross-manipulation}
\begin{center}  
\scalebox{0.80}{
\begin{tabular}{c|l|cccc|c}
\toprule
Training &\multirow{2}*{Method} & \multicolumn{4}{c|}{Test data} & \multirow{2}*{Avg} \\
\cmidrule(lr){3-6}
     data  &            ~                   & DF    & F2F   & FSW   & NT    & ~   \\
     
\midrule
\multirow{5}*{DF}
& MultiAtt~\cite{multiatt} & \cellcolor{Gray}99.92 & 75.23 & 40.61 & 71.08 & 71.71                \\ 
& DCL~\cite{dcl}       & \cellcolor{Gray}\textbf{99.98} & \textbf{77.13} & 61.01 & 75.01 & 78.28              \\
& RECCE*~\cite{recce}     & \cellcolor{Gray}99.19 & 74.39 & 57.42 & \textbf{85.04} & 79.01                \\ 
& UIA-ViT*~\cite{uia}  & \cellcolor{Gray}99.39      &   74.44    &   53.89    &   70.92    & 74.66 \\ 
& Ours  & \cellcolor{Gray}99.19 & 76.61 & \textbf{68.96} & 74.83 & \textbf{79.90}            \\ 
       
\midrule
\multirow{5}*{F2F}
        & MultiAtt~\cite{multiatt}       & 86.15 & \cellcolor{Gray}99.13 & 60.14 & 64.59 & 77.50 \\
        & DCL~\cite{dcl}       & 91.91 & \cellcolor{Gray}99.21 & 59.58 & 66.67 & 79.34 \\
       & RECCE*~\cite{recce}       & 88.04 & \cellcolor{Gray}98.93 & 67.35 & 74.16 & 82.12 \\
       & UIA-ViT*~\cite{uia}       & 83.39 & \cellcolor{Gray}98.32 & 68.37 & 67.17 & 79.31 \\
       & Ours                                  & \textbf{93.75} & \cellcolor{Gray}\textbf{99.61} & \textbf{78.62} & \textbf{79.56} & \textbf{87.81} \\

\midrule
\multirow{5}*{FSW}
& MultiAtt~\cite{multiatt} & 64.13 & 66.39 & \cellcolor{Gray}99.67 & 50.10 & 70.07              \\
& DCL~\cite{dcl}           & 74.80 & 69.75 & \cellcolor{Gray}99.90 & 52.60 & 74.26              \\
& RECCE*~\cite{recce}       & 66.66 & 73.66 & \cellcolor{Gray}\textbf{99.76} & \textbf{57.46} & 74.39               \\

& UIA-ViT*~\cite{uia}       &   81.02    &   66.30    & \cellcolor{Gray}99.04      &   49.26    & 73.91 \\ 
& Ours                                           & \textbf{87.46} & \textbf{75.96} & \cellcolor{Gray}99.42 & 55.87 & \textbf{79.68}            \\ 

\midrule
\multirow{5}*{NT}
& MultiAtt~\cite{multiatt} & 87.23 & 75.33 & 48.22 & \cellcolor{Gray}98.66 & 77.36                \\
& DCL~\cite{dcl}      & 91.23 & 79.31 & 52.13 & \cellcolor{Gray}\textbf{98.97} & 80.41                \\
& RECCE*~\cite{recce}    & \textbf{90.20}  & 76.65 & \textbf{58.06} & \cellcolor{Gray}97.17 & \textbf{80.52}                \\
 & UIA-ViT*~\cite{uia}  &    79.37   &   67.98    &   45.94    &\cellcolor{Gray}94.59       & 71.97 \\
 & Ours     & 84.31 & \textbf{81.04} & 54.67 & \cellcolor{Gray}96.19 & 79.05          \\
       
\bottomrule
\end{tabular}}
\vspace{-2em}
\end{center}
\end{table}

\textbf{Cross-dataset evaluations.} We train our model on the Faceforensics++ dataset and evaluate its performance on the test sets of Celeb-DF\cite{celeb-df} and DFDC~\cite{dfdc}. Specifically, following the previous practice in~\cite{lip}, we validate the model on Celeb-DF and use the selected model to test on DFDC.  \emph{We adopt the HQ version of FF for training, and only use one frame every video for testing.} Under the same setting, we ran the public code of RECCE~\cite{recce}, UIA-ViT~\cite{uia} and SBI~\cite{sbi} to produce corresponding results. In Table~\ref{tab:cross-dataset}, we show a competitive performance with existing image-based methods, signaling satisfying adaptability of RFFR to different datasets, especially high quality datasets like Celeb-DF. 
  
SBI~\cite{sbi} is a recent powerful deepfake detection method. By utilizing a hand-crafted blending algorithm to produce diverse fake samples, it achieves highly competitive performances on datasets including Celeb-DF. We show that by training on fake samples generated by SBI, our approach can further improve upon their state-of-the-art result. 

\begin{table}[]
\setlength\tabcolsep{4.5pt} 
\caption{Cross-dataset performances in terms of AUC(\%) compared with previous methods. Classifiers are trained on FF and tested on Celeb-DF and DFDC. We ran the public code of methods marked with "*" to produce results under identical settings \emph{(HQ for training and single frames for testing).}}
\vspace{-1em}
\label{tab:cross-dataset}
\begin{center}  
\scalebox{0.90}{
\begin{tabular}{l|cc}
\toprule
\multirow{2}*{Method} & \multicolumn{2}{c}{Test data}\\
\cmidrule{2-3}
        ~                           &     Celeb-DF         &  DFDC \\
\midrule
      Xception~\cite{xception}  &     65.30       &    -  \\
      Face X-ray~\cite{xray}          &     74.20       &     70.00 \\
      MultiAtt~\cite{multiatt}        &     67.44       &     67.34 \\
      SPSL~\cite{SPSL}                &     76.88        &   -  \\
      SOLA~\cite{sola}                &       76.02         &  -    \\
      SLADD~\cite{sladd}              &    79.70       &  -  \\
      RECCE*~\cite{recce}             &     68.94       &   68.34   \\
      UIA-ViT*~\cite{uia}             &     80.31      &   67.93   \\
      SBI*~\cite{sbi}                       &       86.46     &   66.60     \\
\midrule
 	Ours                                      &   81.97  & \textbf{72.08}  \\
    Ours + SBI~\cite{sbi}                  &  \textbf{88.98}           &    67.84   \\
\bottomrule
\end{tabular}}
\vspace{-2.5em}
\end{center}
\end{table}

\subsection{Ablation Study}
\label{ablation}

In this section, we analyze the effect of our implementations for RFFR learning and deepfake detection. 

\textbf{Effect of the training data for RFFR.} The effectiveness of deepfake detection with RFFR depends on the quality of representation learning, where the real faces plays an important role. In this experiment, we examine the effect of scaling the real face dataset for representation learning. As a baseline, we learn RFFR with only real faces from Faceforensics++ (FF), the same data we use for the downstream classification tasks. Meanwhile, another model is supplemented with real faces from both FF and ForgeryNet (FN), a significantly larger and more diverse dataset. We train deepfake detectors on the F2F subset of FF with residual images produced by these two models. In Table~\ref{tab:data}, we demonstrate that including the extra dataset of ForgeryNet for learning RFFR consistently improves the performances of the deepfake detector in all tests, creating a maximum performance gain of $9.57\%$  in terms of AUC (F2F $\rightarrow$ NT).

We note that learning RFFR with FF already allows our deepfake detector to outperform the state-of-the-arts. Nevertheless, learning with extra data enhances the efficacy of our real face foundation representations, and further improves the downstream task of deepfake detection. Therefore, refining the representation learning of real faces, especially with large-scale datasets, could be a viable path for further improving generalized deepfake detection. 

In addition, we examine the scalability of RECCE under the same setting, considering that RECCE~\cite{recce} also involves learning to reconstruct real samples for deepfake detection. However, their performance gain is less significant than ours. Although the reconstruction branch of RECCE~\cite{recce} is able to highlight forgery cues with residual images, they tend to involve more background noise caused by imperfect reconstructions, as depicted in~\cref{fig:unet_comparison},. This undermines the ability of residual images to expose artifacts for deepfake detection. 

\begin{table}[t]
\setlength\tabcolsep{4.5pt} 
\caption{Deepfake detection performances of RECCE~\cite{recce} and our method with different real face dataset, namely the real faces from Faceforensics++ (FF) alone, and FF combined with ForgeryNet (FF + FN). Classifiers are trained on F2F and tested on four subsets of FF. We present the results in AUC (\%).  }
\vspace{-1.5em}
\label{tab:data}
\begin{center}  
\scalebox{0.90}{
\begin{tabular}{c|c|cccc|c}
\toprule
\multirow{2}*{Method} & Real face  & \multicolumn{4}{c|}{Test data} & \multirow{2}*{Avg} \\
\cmidrule(lr){3-6}
&dataset  &      DF    & F2F   & FSW   & NT    & ~   \\
    \midrule
\multirow{2}*{RECCE~\cite{recce}}&FF           & 88.04          & 98.93          & 67.35          & 74.16          & 82.12          \\
&FN + FF &  90.12       & 99.24       & 69.89    & 79.59     & 84.71		\\
    \midrule
\multirow{2}*{Ours}&FF           & 90.16          & 98.56          & 74.10          & 69.99          & 83.20          \\
&FN + FF & \textbf{93.44}       & \textbf{99.61}        & \textbf{78.62}       & \textbf{79.56}        & \textbf{87.81}		\\
\bottomrule
\end{tabular}}
\vspace{-1em}
\end{center}
\end{table}

\textbf{Effect of masked image modeling for RFFR.} We analyze the effect of using MIM-based residual images for deepfake detection. We train a UNet-based autoencoder (AE) to learn the reconstruction of real faces and obtain residual images. Our MIM-trained inpainting model and the AE are compared on the quality of reconstruction in~\cref{fig:unet_comparison}. Note that despite being trained with real faces, the AE "generalizes" well to fake images, preserving delicate details, including the artifacts caused by manipulations. Such generalization leaves the residual images empty with little information. 

\begin{figure}
\centering
  \includegraphics[width=0.9\columnwidth]{figs/compare_ICCV_Final.pdf}
  \vspace{-1em}
   \caption{Reconstruction results and residual images of the autoencoder (AE), RECCE~\cite{recce} and our inpainting model. AE reconstructs both images perfectly, leaving no information in residual images. RECCE~\cite{recce} suffers from insufficient training. Our model successfully highlights potential artifacts in the residual image of only the fake face, and therefore can best facilitate deepfake detection. }
\vspace{-1em}
\label{fig:unet_comparison}
\end{figure}

Masked image modeling enables our model to learn better real face representations and inpaint fake faces with real textures instead of artifacts. In the downstream task of deepfake detection,  our classifier generalizes significantly better than the AE-based classifier, which performs only marginally better than learning with no residuals (detailed in Appendix). Both the reconstruction results and the downstream performance confirm the validity of our choice to learn RFFR with MIM instead of direct reconstruction. 


\textbf{Effect of classifier backbone.} In Table~\ref{tab:backbone}, we present the deepfake detection results of vanilla Xception~\cite{xception} and Vision Transformer (ViT)~\cite{vit}, both trained with full original images. The models are trained with the F2F subset of FF and tested on all four subsets. While a larger backbone increases a deepfake detector's generalization performance in some cases, it is not the primary factor of our performance improvement. Instead, it is the residual input aided by RFFR that leads the performance gain.

\begin{table}[t]
\setlength\tabcolsep{4.5pt} 
\caption{Comparing ours results with vanilla backbones. We present the results in AUC (\%).  }
\label{tab:backbone}
\vspace{-1.5em}
\begin{center}  
\scalebox{0.90}{
\begin{tabular}{c|c|cccc|c}
\toprule
Training  &  \multirow{2}*{Method}    &   \multicolumn{4}{c|}{Test Data} & \multirow{2}*{Avg} \\
\cmidrule(lr){3-6}
 data  &   ~  &   DF    & F2F   & FSW   & NT    & ~   \\
    \midrule
\multirow{3}*{F2F} & Xception~\cite{xception} & 84.94          & 99.26          & 58.82          & 71.19          & 78.55          \\
                                   & ViT~\cite{vit}      & 84.25          & 97.89          & 65.53          & 65.18          & 78.21          \\
                                   & Ours     & \textbf{93.44} & \textbf{99.61} & \textbf{78.62} & \textbf{79.56} & \textbf{87.81} \\
\bottomrule
\end{tabular}}
\vspace{-1.5em}
\end{center}
\end{table}

\textbf{Effect of classifier design.} We compare different variants of our classifier design. Specifically, we analyze the performance gains brought by the introduction of two branches and the random input mechanism. We test six variants of our classifier by training them with the F2F subset of FF and testing with the FSW subset. The settings of these variants are specified by the input data they accept, as shown in~\cref{tab:classifier}. 

\begin{table}[t]
\caption{Deepfake detection performances with classifiers of different inputs in terms of AUC (\%). We train the classifiers on F2F and test on FSW.}
\label{tab:classifier}
\vspace{-1.5em}
\begin{center}
\begin{tabular}{c|c|c|c|c}
\toprule
\multicolumn{2}{c|}{Original Image} & \multicolumn{2}{c|}{Residual Image} & \multirow{2}*{AUC (\%)} \\
\cline{1-4}
               Full        &             Random           &          Full          &          Random          &   ~\\
 \hline
\checkmark        &                                       &                            &                                   &  65.53\\
% \hline
                              &                                      &   \checkmark    &                                   &  66.30  \\
 %\hline
\checkmark        &                                      &   \checkmark    &                                   &  71.48  \\
 %\hline
                             &       \checkmark          &                             &                                   &  70.76  \\
%\hline
                             &                                       &                             &      \checkmark       &  68.10  \\
 %\hline
                             &        \checkmark         &                             &      \checkmark       &  \textbf{78.62}  \\
\bottomrule
\end{tabular}
\vspace{-2em}
\end{center}
\end{table}

\begin{table*}[t]
\setlength\tabcolsep{4.5pt} 
\caption{Deepfake detection performances of validated and non-validated models. Classifiers are trained on F2F and tested on four subsets of FF. We present the results and the performance gaps in AUC (\%). Second best results are underlined. }
\label{tab:validation}
\vspace{-1em}
\begin{center}  
\scalebox{0.90}{
\begin{tabular}{c|c|llll|l}
\toprule
\multirow{2}*{Method}  & \multirow{2}*{Validated} & \multicolumn{4}{c|}{Test Data} & \multirow{2}*{Avg} \\
\cmidrule(lr){3-6}
~                   &                      ~                   &      DF               & F2F                    & FSW                 & NT                    & ~   \\
    \midrule
\multirow{2}*{Xception\cite{xception}} &   \checkmark    & 84.94                 & 99.26                & 58.82                 & 71.19                & 78.55            \\
~ &                                             -                              & 83.08   (- 1.86) & 99.12   (- 0.14) & 46.63   (- 12.19) & 64.93   (- 6.26)  & 73.44   (- 5.11)  \\
 \hline
 \multirow{2}*{RECCE\cite{recce}} &\checkmark               & 88.04                & 98.93                 & 67.35                & 74.16                & 82.12            \\
 ~&                                                -                  & 74.51   (- 8.57) & 99.22   (+ 0.29)  & 50.17   (- 17.18) & 59.46   (- 14.70)  & 70.84   (- 11.28) \\
 \hline
\multirow{2}*{Ours} &    \checkmark  & \textbf{93.44}            & \textbf{99.61}            & \textbf{78.62}            & \textbf{79.56}            & \textbf{87.81}            \\
 ~&  - & \underline{91.56} (- 1.88) & \underline{99.39}   (- 0.22) & \underline{76.00}   (- 2.62)  & \underline{76.41} (- 3.15) & \underline{85.84}   ( - 1.97)    \\
\hline
\end{tabular}}
\vspace{-2em}
\end{center}
\end{table*}

We treat the vanilla ViT with full original image input as a baseline, which achieves an AUC of $65.53\%$. By switching to accept the full residual images, we obtain a $0.77\%$ performance gain. Combining the two modalities to form a dual-branch classifier further increases our result to $71.48\%$. This demonstrates that the artifacts are better exploited when both the original and the residual images enter the classifier, and are used as references to each other. Therefore, both modalities should be considered for classification. 

In addition, we improve on the test by merely modifying the baseline ViT to accept randomly selected original image blocks. This results in a $5.23\%$ increase in performance. Similarly, changing full residual input to random residual blocks also results in a $1.8\%$ improvement. These observations confirm our hypothesis in \cref{sec:method_deepfake_detection} that models benefit from learning with random inputs, which prevents the model from only focusing on the most prominent features in an image, and forces it to learn from subtle artifacts. 

Finally, bringing in the random input mechanism for the dual-branch classifier completes our full implementation, which maximally exploits the artifacts exposed by RFFR and achieves the best performance of $78.62\%$. 



\subsection{Validation-free Model Selection}
\label{sec:validation-free}

\begin{figure}
\centering
  \includegraphics[width=0.5\textwidth]{figs/validation-free_ICCV_Final.png}
  \vspace{-1.5em}
   \caption{Comparing the validation curves of RFFR-based deepfake detector and previous methods. Detectors are trained on the F2F subset of FF for $15k$ iterations and validated on four different subsets. (a) to (d) correspond to experiments on DF, F2F, FSW and NT.  Results are reported in AUC (\%). All three methods perform well when validated on F2F. However, under cross-manipulation settings, only our method avoids overfitting during training. The curves are smoothed for better visibility.}
\label{fig:validation-free}
\vspace{-1em}
\end{figure}

Models expected to generalize to other domains benefit from target domain validations~\cite{domainbed}. By frequently performing model validation, we can select the model  that best suits the detection of target manipulation, resulting in high performance on the test set. While using such an \textit{oracle} could be acceptable for the early development of cross-domain algorithms~\cite{domainbed}, it is not ideal for applications, as labeled data of unseen manipulation is usually not available. 

In this section, we demonstrate the potential of our deepfake detector to circumvent this practice and therefore avoid the need for extra validation data. As shown in \cref{tab:validation}, we train our classifier on F2F for 15k iterations and directly use the final model for testing. Simultaneously, we employ four validation sets to select the models with the best validation performances on target data. All validated and non-validated models are tested under the same conditions. We report all results on the target test sets in Table~\ref{tab:validation}. The performance gaps between validated and non-validated models are reported along with the test results. Although our non-validated models are not performing as well as those selected with a validation set, we show that our model remains effective on target data, with a maximum performance drop of $3.15\%$ and an average drop of $1.97\%$. However, previous methods~\cite{xception, recce} suffer from significantly larger performance drops when evaluated under the same procedure. 

To take a closer look at how the cross-manipulation performances vary during training, we train the deepfake detectors again with F2F. We test the AUC performances on all target subsets every 50 iterations to produce validation curves in \cref{fig:validation-free}. Our RFFR-based deepfake detector consistently maintains a high performance long after its peaks without serious overfitting. On the contrary, both previous methods compared here overfit quickly after reaching their highest target domain performances. In addition, compared methods exhibit large fluctuations across different evaluations, while our model remains stable. This suggests that with RFFR, our model focuses exclusively on generalizable features which fall outside the distribution of RFFR. Such resistance to overfitting guarantees our model a satisfying performance even when labeled validation sets are not available, which is generally expected in practice. We present more results on validation-free evaluations in Appendix.
\section{Discussion and Conclusion}
\section{Conclusion, Limitations, and Future Work}
\label{sec:future}
We presented \ours, a NeRF editing method conditioned on text and sketch. Using novel loss functions, our framework allows for local editing of neural fields.
\begin{wrapfigure}{r}{0.2\textwidth} 
\vspace{-10pt}
  \begin{center}
    \includegraphics[width=0.2\textwidth]{figs/failures_Ali.jpg}
  \end{center}
    \vspace{-15pt}
 \vspace{1pt}
\end{wrapfigure} 
Similar to previous works \cite{poole2022dreamfusion, lin2022magic3d, metzer2022latent}, our approach utilizes the SDS Loss and may be vulnerable to the well-known "multiface issue" (inset figure) depending on the choice of diffusion model and prompt. Our method supports a single set of prompt and sketch views at a time. A simple workaround is to apply our method multiple times progressively (Fig.~\ref{fig:progressive}). 
Our results rely on the publicly available Stable-Diffusion model \cite{rombach2021highresolution}, which is less amenable to directional text prompts and produces lower quality 3D generated outputs compared to commercial diffusion models used by previous works~\cite{poole2022dreamfusion, lin2022magic3d}. In Fig~\ref{fig:diff_diff} we show that it is possible to get better results by using the Deepfloyd-IF model \cite{deepfloyd}.


Future directions may expand our method to better support for non-opaque materials, or condition on other modalities, possibly through the diffusion model. More research may further extend the usage of sketch scribbles for animation, similar to \cite{dvoro2020monstermash}. 



% \orrc{In addition, the interface of our method may further close the gap with non data-driven methods, through allowing inflated single sketch views or other primitive based sketch interfaces. Mention also we didn't explore half-transparent objects enough

% \orr{
% Limitations: 1. Janus effect / multiface problem (cat with santa hat), 2. sketching multiple disjoint regions at once. 3. mention that quality presented in this work depend on the diffusion model used? (we can't compete with the larger IMAGINE / e-diffi).

% Notes: remember thanking people: Andrey for SGMT code and mention mesh sources. (the cat, the plate, the horse)
% }


\begin{credits}
\subsubsection{\ackname}  
This work was supported by VILLUM FONDEN (grants 15334, 42062), the European Research Council under the European Union's Horizon 2020 research and innovation programme (grant 757360), Novo Nordisk Foundation (NNF20O-C0062606), LANL (LA-UR-24-23937) LDRD grant 20210043DR (U.S. DOE NNSA Contract 89233218CNA000001), and the Pioneer Centre for AI (DNRF grant P1). 

\subsubsection{\discintname}
The authors have no competing interests to declare that are
relevant to the content of this article.

\end{credits}

%
% ---- Bibliography ----
%
% BibTeX users should specify bibliography style 'splncs04'.
% References will then be sorted and formatted in the correct style.
%
% \bibliographystyle{splncs04}
% \bibliography{mybibliography}
%
{
\bibliographystyle{splncs04}
\bibliography{references}
}

\appendix

%
\title{Supplementary to Laplacian Segmentation Networks Improve Epistemic Uncertainty Quantification}
%
\titlerunning{Supplementary Laplacian Segmentation Networks}
% If the paper title is too long for the running head, you can set
% an abbreviated paper title here
%
\author{}
%
\authorrunning{F. Author et al.}
% First names are abbreviated in the running head.
% If there are more than two authors, 'et al.' is used.
%
\institute{}
%
\maketitle              % typeset the header of the contribution
\vspace*{-\baselineskip}
\vspace*{-\baselineskip}
%

%
%
%

\section{Implementation and Training Details}

Training configurations are provided in Table \ref{training_configs}. All models were trained with the Adam optimizer.  The U-net backbone was constructed with feature maps of size $8, 16, 32, 64, 128$. Uncertainty measures were approximated from $50$ samples from the posterior and $20$ samples from the logit distribution. 

\vspace*{-\baselineskip}
\begin{table}
% \scriptsize
\begin{center}
  \caption{Implementation and Training Details.  Dropout models for the ISIC dataset were trained with a $0.0005$ learning rate to improve convergence. }
\label{training_configs}
   \vspace{.5em}
  \centering
  \begin{tabular}{p{30mm}p{20mm}|p{20mm}|p{20mm}}
    \toprule
    
    & \multicolumn{3}{c}{Dataset}\\
    \cmidrule{2-4}
    % \cmidrule{6-9}
    % \cmidrule{10-12}
    
    
    Configuration       & ISIC & Prostate & Brats \\
    \hline
   Epochs & 60 & 150 & 600 \\
   \hline
    Batch Size  & 32 & 10 & 32  \\
    \hline
    Learning Rate   & 0.001* & 0.001    & 0.001  \\
    \bottomrule
  \end{tabular}

   \vspace{.5em}
  \centering
 
\vspace*{-\baselineskip}
\vspace*{-\baselineskip}
\end{center}
\end{table}
\vspace*{-\baselineskip}


\section{Fast Hessian Approximation}

Consider a neural network (\textsc{nn}) $f_\theta:\mathcal{X}\rightarrow\mathcal{Y}$ with $L$ layers. The parameter $\theta = (\theta_1, \dots, \theta_L) \in\Theta$ is the concatenation of the parameters for each layer $i \in \{1,...,L \}$. The \textsc{nn} $f_\theta=
f^{(L)}_{\theta_L}\circ f^{(L-1)}_{\theta_{L-1}} 
\circ\,\dots\,\circ 
f^{(2)}_{\theta_2} \circ f^{(1)}_{\theta_1}$ is a composition of $L$ functions $f^{(L)},f^{(L-1)},\dots,f^{(1)}$, where $f^{(i)}$ is parametrized by $\theta_{i}$. Let $x_0\in\mathcal{X}$ be the input and  $x_i:=f^{(i)}_{\theta_i}(x_{i-1})$ for $i=1,\dots,L$, such that the \textsc{nn} output is $x_L\in\mathcal{Y}$. We define the \emph{diagonal} operator $\mathcal{D}: 
\mathbb{R}^{m\times m}
\rightarrow
\mathbb{R}^{m\times m} $ on quadratic matrices as
\[
[\mathcal{D}(M)]_{ij}
:=
\left\{
\begin{array}{ll}
    M_{ij} & \text{ if } i=j \\
    0 & \text{ if } i\not=j
\end{array}
\right.
\qquad
\forall i,j=1,\dots,m.
\]
The \textbf{Jacobian} $J_\theta f_\theta(x_0)$ of the \textsc{nn} %w.r.t.\@ the parameter 
has a layer block structure, block $i$ is %$\theta_i$ of layer $i$ as
\begin{equation*}\label{eq:jacobian_chain_rule}
J_{\theta_i}f_{\theta}(x_0)
=
J_{\theta_i}
\left(
        f^{(i)}_{\theta_i}
        \circ\dots\circ
        f^{(L)}_{\theta_L}
    \right)
(x_{i-1}) 
=
\left(
    \prod_{j=L}^{i+1} 
    J_{x_{j-1}}f^{(j)}_{\theta_j}(x_{j-1})
\right)
J_{\theta_i}f^{(i)}_{\theta_i}(x_{i-1}).
\end{equation*}

The Laplace approximation requires the Hessian \textbf{H} of the loss w.r.t. the parameters $\nabla^2_{\theta} \mathcal{L}(f_\theta(x_0))
\in\mathbb{R}^{|\theta|\times|\theta|}$. Using the chain rule it holds, that
\begin{equation*}\label{eq:hessian_chain_rule}
\underbrace{\nabla^2_{\theta} \mathcal{L}(f_\theta(x_0))}_{=:H_\theta}
=
\underbrace{
J_{\theta}f_\theta(x_0)^\top
    \cdot 
    \nabla^2_{x_L}\mathcal{L}(x_L) 
    \cdot
J_{\theta}f_\theta(x_0)
}_{=:\textsc{ggn}_\theta}
+
\sum_{o=1}^{|x_L|} 
    [\nabla_{x_L}\mathcal{L}(x_L)]_o \cdot
    \nabla^2_{\theta} [f_\theta(x_0)]_o,
\end{equation*}
where $[v]_o$ refers to the $o$-th component of vector $v$ and $|v|$ to its length. We can write the diagonal block $\textsc{ggnb}^{(i)}_\theta=J_{\theta_i}f_\theta(x_0)^\top \textbf{H}^{\mathcal{L}} J_{\theta_i}f_\theta(x_0)$ of the $i$-th layer as

\begin{equation}\label{eq:ggn_chain_rule}
\begin{aligned}
    \textsc{ggnb}^{(i)}_\theta
    = &
    J_{\theta_i}f_{\theta}(x_0)^\top \cdot \textbf{H}^{\mathcal{L}} \cdot J_{\theta_i}f_{\theta}(x_0)
\end{aligned}
\end{equation}

From this expression, plus the chain rule expansion of the Jacobian, we can build an efficient backpropagation-like algorithm to compute $\textsc{ggnb}_\theta$, it start from $\textbf{H}^{\mathcal{L}}$ and then iterated backward over layers. The same holds for the diagonal approximation, which we refer to as $\textsc{ggnd}_\theta:=\mathcal{D}(\textsc{ggn}_\theta)=\mathcal{D}(\textsc{ggnb}_\theta)$. This approach already scales linearly in the number of parameter $|\theta|$.
On top of that, the diagonal backpropagation approximates the diagonal of the Generalized Gauss-Newton matrix. It is defined, for each layer $i$, by adding a diagonalization operator in between each Jacobian product, marked \textcolor{BrickRed}{red} in Algorithm 1. Without this extra operator the algorithm would return the exact diagonal.

\vspace*{-\baselineskip}
\begin{algorithm}[H]
\caption{Computation of $\textsc{db}_\theta$}\label{alg:exact_backprop_diag}
\begin{algorithmic}
\State $M$ = $\textbf{H}^{\mathcal{L}}$
\For{$j=L,L-1,\dots,1$}
\State $\textsc{db}^{(j)}_\theta$ = $\mathcal{D}\left(J_{\theta_j}f^{(j)}_{\theta_j}(x_{j-1})^\top \cdot M \cdot J_{\theta_j}f^{(j)}_{\theta_j}(x_{j-1})\right)$
\State $M$ = $\textcolor{BrickRed}{\mathcal{D}}\textcolor{BrickRed}{\Big(} J_{x_{j-1}}f^{(j)}_{\theta_j}(x_{j-1})^\top \cdot M \cdot J_{x_{j-1}}f^{(j)}_{\theta_j}(x_{j-1})\textcolor{BrickRed}{\Big)}$
\EndFor
\State $\textsc{db}_\theta$ = $(\textsc{db}^{(1)}_\theta, \dots, \textsc{db}^{(L)}_\theta)$
\State \textbf{return} $\textsc{db}_\theta$
\end{algorithmic}
\end{algorithm}
\vspace*{-\baselineskip}

\textbf{Proposition.} For an autoencoder network, the memory requirement of the Algorithm scale \emph{linearly} both in number of parameter and in number of pixels.
\begin{proof}
    The bottlenecks are the storage of the matrixes $\textsc{db}^{(j)}_\theta
\in\mathbb{R}^{|\theta_j|}$ and $
M
\in\mathbb{R}^{|x_{j-1}|}
$ at each step $j$
\end{proof}

\paragraph{Skip-connections}
For any given submodule $g_\theta$, a skip-connection layer $\textsc{sc}(g)_\theta$ is defined as $x  \longmapsto (g_\theta(x), x)$. The Jacobian with respect to the parameter is the same as the Jacobian of the $g_\theta$ while the Jacobian with respect to the input is $
    J_x \textsc{sc}(g)_\theta(x) 
    =
    \left(\begin{array}{c}
        J_x g_\theta(x) \\ \hline
        \mathbb{I}
    \end{array}\right)
    \in\mathbb{R}^{(O+I)\times I}$. 

\textbf{Proposition.} If $M$ is diagonal, then one step of Alg 1 can be computed as
\begin{equation*}\label{eq:skipconn_recursive_def}
    \mathcal{D}\left( J_x \textsc{SC}(g)_\theta(x)^\top \cdot M \cdot J_x \textsc{SC}(g)_\theta(x) \right)
    =
    \mathcal{D}\left( J_x g_\theta(x)^\top M_{11} J_x g_\theta(x) \right)
    + \mathcal{D} (M_{22}).
\end{equation*}
\begin{proof}
    Let $
    M
    =
    \begin{pmatrix}
    M_{11} & M_{12} \\
    M_{21} & M_{22}
    \end{pmatrix}$ and then
    
\begin{align*}
    J_x \textsc{SC}(g)_\theta(x)^\top \cdot &\,M \cdot J_x \textsc{SC}(g)_\theta(x) 
     =
    \left(\begin{array}{c|c}
        J_x g_\theta(x)^\top &
        \mathbb{I}
    \end{array}\right)
    \begin{pmatrix}
    M_{11} & M_{12} \\
    M_{21} & M_{22}
    \end{pmatrix}
    \left(\begin{array}{c}
        J_x g_\theta(x) \\ \hline
        \mathbb{I}
    \end{array}\right) \\
    & =
    J_x g_\theta(x)^\top M_{11} J_x g_\theta(x)
    + M_{12} J_x g_\theta(x)
    + J_x g_\theta(x)^\top M_{21}
    + M_{22}
\end{align*}
\end{proof}


%
% ---- Bibliography ----
%
% BibTeX users should specify bibliography style 'splncs04'.
% References will then be sorted and formatted in the correct style.
%
% \bibliographystyle{splncs04}
% \bibliography{mybibliography}

\begin{comment}
    

{\small
\bibliographystyle{splncs04}
\bibliography{references}
}

\end{comment}



\end{document}


\typeout{get arXiv to do 4 passes: Label(s) may have changed. Rerun}