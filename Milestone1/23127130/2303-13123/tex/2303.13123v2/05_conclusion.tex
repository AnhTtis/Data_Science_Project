In this paper, we have demonstrated how Laplace approximations can scale to image segmentation tasks, through a trace-preserving diagonal Hessian approximation. Importantly, this scales linearly with the number of image pixels, unlike past work which exhibited a quadratic complexity. We have demonstrated across different datasets that the parameter distributions obtained by Laplace's method, in combination with suitable uncertainty measures, can lead to superior OOD detection performance on image level. 

Our experimental findings support the recent initiative in research for finding better measures for epistemic uncertainty than Mutual Information \cite{schweighofer2023}. Marginalizing over all datasets and aggregations strategies, our findings show that EPKL and Pixel Variance, not Mutual Information, provide the strongest discriminative power for classifying images as either ID or OOD (cf. Fig. \ref{fig:auroc_discussion}, left). Further we find that there is still a strong correlation between aleatoric and epistemic measures across all method combinations, visible by comparable AUROC performance over all datasets(cf. Fig. \ref{fig:auroc_discussion}, right).  

Further research might investigate in more depth how different logit distributions interplay with Laplace approximations in theoretically suggested uncertainty measures that target aleatoric and epistemic uncertainty. The presented method provides an extendable framework for other researchers to build upon. 
