%%% STRATEGY FOR FIRST DRAFT %%%

% You should first prepare an overall system diagram, showing inputs and
% outputs, your new contributions in light blue color fill, and existing
% components in white. Then you pen down the following in sub-section
% form, and in enumerated list. Please do not write big paragraphs. We
% need to quickly extract the main points in the initial review rounds
% to gauge the real value of the contributions.
% ===========================================
% *Introduction and Target Research Problem:* 1-2 points on the field
% and use-cases. 1-2 points on the major research problems in achieving
% those use-cases. 2-3 points on your main target research problem from
% there, and why is this problem important and worthy to investigate.

% *Major State-of-the-Art and Their Limitations:* Discuss the
% state-of-the-art (SOTA) in categorized form, i.e. do not discuss
% individual papers, rather use the following style of discussion: The
% XYZ body of research explores ABC [REF1][REF2][REF3]... however, they
% lack XYZ. Another category is based on XYZ [ref][ref][ref], that
% targets ABC [ref][ref], or DEF [ref][ref][ref], but they all lack XYZ
% and XYZ due to their ABC.

% *Key Scientific Challenges:* So what are the open research challenges
% that SOTA have not addressed yet, and why. Pen-down a list of such
% open challenges that you will target in this paper through your novel
% contributions.

% *Motivational Case Study:* Can you provide a nice case study to
% corroborate your above discussed challenges through quantitative
% analysis, and not only words? This is super-important to make sure the
% problem is real, and significant.

% *Novel Contributions & Key Results: *A list of key scientific
% contributions => it should be aligned to the above-discussed
% scientific challenges. Note: these are not engineering/implementation
% related points, rather solid scientific methods, etc. then discuss 2-3
% points about your implementation depth, how solid it is, and what are
% the key results compared to SOTA, and how much have you improved upon
% them. => for this also put your key results here...
% ===========================================

% *Very Important Note:* After every round of writing, you should read
% your text Super-Critically, as if this is not from you, rather now you
% are a very critical reviewer, and want to actively find weaknesses
% (scientific, implementation related, and writing related). If you
% master this skill, you can proactively avoid such comments, and can
% significantly improve . Do not be over-convinced with your own work!
% Otherwise you write a paper like "what is done" and think it's great,
% while a reader seeks for "what is done, and why is it done, and is it
% really worthwhile and done in a credible way or not".

%%% END STRATEGY FOR FIRST DRAFT %%%


%%% OVERALL CHAIN OF ARGUMENTS / PAPER WRITING METHODOLOGY %%%

% \vm{\textbf{Suggested Chain of Arguments:} If you answer the following questions, this will form a connected chain of argument. Note, for each statement/claim that you provide, you should provide concrete numbers, solid references, and/or your own analysis to support them properly. Every claim related to the contributions should be corroborated by proper in-depth results and technical discussion.
% \begin{itemize}
% \item What is the main area and relevant problem that you are going to target in this paper? Why is this problem relevant and important for the research community?
% \item What are most important state-of-the-art in this area that also aim at the similar problem? Give brief description in categorized form. Detailed related work should be put together in a separate section.
% \item What are the main limitations of the above state-of-the-art techniques? Considering this, formulate the concrete research problem that you will target in this paper, and discuss associated scientific challenges which have to be addressed to meet that goal.
% \item What are your key novel contributions and what makes them novel? Give concrete description that can give a glimpse of the new technique, methodology, etc.
% \item What are the most important results? Give glimpse about the realistic experimental setup, so that results are credible. Also provide a  link, if you are going to  make the tool chain and results open-source.
% \item Paper Organization: Afterwards, provide a concise paper organization stating what would come under different sections.
% \end{itemize}
% }

% \vm{\textbf{Target Problem and Associated Scientific Challenges:}
% Put the concerte research problem targeted in this paper here followed by a list of the associated scientific challenges. These challenges are then directly addressed by different novel contribution stated below.
% \begin{itemize}
% \item TODO
% \item TODO
% \item TODO
% \end{itemize}
% }

% \textbf{Main Contributions:}
% \begin{itemize}
%     \item An ML-based optimization method that improves optimizing compilers by applying standard optimization passes in custom sequences that are adapted for the individual programs.
%     \item Implementation of the method as (1) a custom LLVM optimization pass that uses a pre-trained PyTorch model via libtorch and (2) a training bench in Python that allows training/refining/retargeting the PyTorch model for different platforms and/or application domains.
%     \item Evaluating the proposed method and implementation with the RISC-V BEEBS benchmarks.
% \end{itemize}

% \vm{\textbf{TODO for David:} Make  figure connecting different novel contributions and guiding to their respective sections. Input and Outputs should be clearly mentioned.}

% \vm{\textbf{Paper Organization:}
% The rest of the paper is organized as follows. Section II presents the related work in XYZ and XYZ areas. Section III provides the necessary background knowledge to understand the contributions of this paper. Section IV provides an overview of our methodology followed by details of different contributions. Section V presents the results and discussion, followed by the conclusion in Section VI.}

%%% END OVERALL CHAIN OF ARGUMENTS / PAPER WRITING METHODOLOGY %%%


%%% CHAIN OF ARGUMENTS FIRST DRAFT %%%

% Here is an example flow of discussion that I would suggest you to refine
% your chain-of-argument.
% * 2-3 sentences about DNNs
% * 2-3 sentences about the main reliability problem, with a focus on your key
% problem of fault injection. This forms the basis of your main claims later. =>
% Your technical content is fine. I will give comments directly on your content
% later.
% * Categorized state-of-the-art (SOTA) for addressing the above problem.
% Highlight the best approach(es) that you will use for motivation and final
% result comparison. => Here you start making claims, what is not good in SOTA
% that you will address => Make clear claim in a quantitative way in terms of
% metrics that you will also use for result comparison... These are claims, for
% which you will need motivational case study and main results.
% =======
% * We highlight the targeted research problem and limitations of SOTA with the
% help of the following motivational case study.
% [Section 1.1] Motivational Case Study: Then comes a nice experimental case study
% that shows the key problem, potential for improvement, and any other important
% quantitative analysis that will support the importance of problem, issues with
% SOTA, and lay foundation for your novel concepts. Afterwards, highlighting the
% scientific challenges addressed in this paper.
% For instance, here you can tell how complex are the SOTA framework, their time
% and space complexity, their coverage, their limitations in terms of support of
% fault models, etc.

% [Section 1.2] Scientific Challenges to be Addressed in this Paper: Then list
% down the key scientific challenges that your contributions will address

% [Section 1.3] Our Novel Concept and Contributions: To address the above
% challenges we proposed the XYZ-CoolName framework [Section XYZ] that does XYZ.
% The main concept is to leverage TODO and TODO. Our framework employs the
% following novel techniques:
% 1) A XYZ Algorithm [Section XYZ] that does XYZ to do ABC. It employs TODO for
% doing TODO and TODO. => say scientific name of what kind of algorithm is this...
% polynomial time?? also give space and time complexity of this algorithm in the
% main technical section after the psuedo-code of this algo.
% 2) A XYZ Algorithm [Section XYZ] that does XYZ.
% 3) Full System Implementation in XYZ Design Flow [Section XYZ]: TODO...

% Evaluation & Validation [Section XYZ]: We evaluate our framework and techniques
% for TODO, GPUs. Tell about DNN models, Datasets, other testing features, etc...
% Afterwards, give key results and main message. compare with XYZ and XYZ
% state-of-the-art techniques.

% Open-Source Contributions: TODO...

% Before proceeding to the technical details of our TODO framework, we present the
% necessary background information and related work discussion.

% [Section 2] Background and Related Work.

%%% END OF CHAIN OF ARGUMENTS FIRST DRAFT %%%


%%% IEEE INSTRUCTIONS / TEMPLATE %%%

% The very first letter is a 2 line initial drop letter followed
% by the rest of the first word in caps.
%
% form to use if the first word consists of a single letter:
% \IEEEPARstart{A}{demo} file is ....
%
% form to use if you need the single drop letter followed by
% normal text (unknown if ever used by the IEEE):
% \IEEEPARstart{A}{}demo file is ....
%
% Some journals put the first two words in caps:
% \IEEEPARstart{T}{his demo} file is ....
%
% Here we have the typical use of a "T" for an initial drop letter
% and "HIS" in caps to complete the first word.
%\IEEEPARstart{T}{his} demo file is intended to serve as a ``starter file''
%for IEEE journal papers produced under \LaTeX\ using
%IEEEtran.cls version 1.8b and later.
% You must have at least 2 lines in the paragraph with the drop letter
% (should never be an issue)
%I wish you the best of success.

%\hfill mds

%\hfill August 26, 2015

%\subsection{Subsection Heading Here}
%Subsection text here.

% needed in second column of first page if using \IEEEpubid
%\IEEEpubidadjcol

%\subsubsection{Subsubsection Heading Here}
%Subsubsection text here.


% An example of a floating figure using the graphicx package.
% Note that \label must occur AFTER (or within) \caption.
% For figures, \caption should occur after the \includegraphics.
% Note that IEEEtran v1.7 and later has special internal code that
% is designed to preserve the operation of \label within \caption
% even when the captionsoff option is in effect. However, because
% of issues like this, it may be the safest practice to put all your
% \label just after \caption rather than within \caption{}.
%
% Reminder: the "draftcls" or "draftclsnofoot", not "draft", class
% option should be used if it is desired that the figures are to be
% displayed while in draft mode.
%
%\begin{figure}[!t]
%\centering
%\includegraphics[width=2.5in]{myfigure}
% where an .eps filename suffix will be assumed under latex,
% and a .pdf suffix will be assumed for pdflatex; or what has been
% decalred via \DeclareGraphicsExtensions.
%\caption{Simulation results for the network.}
%\label{fig_sim}
%\end{figure}

% Note that the IEEE typically puts floats only at the top, even when
% this results in a large percentage of a column being occupied by
% floats.


% An example of a double column floating figure using two subfigures.
% (The subfig.sty package must be loaded for this to work.)
% The subfigure \label commands are set within each subfloat command,
% and the \label for the overall figure must come after \caption.
% \hfil is used as a separator to get equal spacing.
% Watch out that the combined width of all the subfigures on a
% line do not exceed the text width or a line break will occur.
%
%\begin{figure*}[!t]
%\centering
%\subfloat[Case I]{\includegraphics[width=2.5in]{box}%
%\label{fig_first_case}}
%\hfil
%\subfloat[Case II]{\includegraphics[width=2.5in]{box}%
%\label{fig_second_case}}
%\caption{Simulation results for the network.}
%\label{fig_sim}
%\end{figure*}
%
% Note that often IEEE papers with subfigures do not employ subfigure
% captions (using the optional argument to \subfloat[]), but instead
% will reference/describe all of them (a), (b), etc., within the main
% captions. Be aware that for subfig.sty to generate the (a), (b), etc.,
% subfigure labels, the optional argument to \subfloat must be present.
% If a subcaption is not desired, just leave its contents blank,
% e.g., \subfloat[].


% An example of a floating table. Note that, for IEEE style tables, the
% \caption command should come BEFORE the table and, given that table
% captions serve much like titles, are usually capitalized except for
% words such as a, an, and, as, at, but, by, for, in, nor, of, on, or,
% the, to and up, which are usually not capitalized unless they are the
% first or last word of the caption. Table text will default to
% \footnotesize as the IEEE normally uses this smaller font for tables.
% The \label must come after \caption as always.
%
%\begin{table}[!t]
%% increase table row spacing, adjust to taste
%\renewcommand{\arraystretch}{1.3}
% if using array.sty, it might be a good idea to tweak the value of
% \extrarowheight as needed to properly center the text within the cells
%\caption{An Example of a Table}
%\label{table_example}
%\centering
%% Some packages, such as MDW tools, offer better commands for making
%% tables than the plain LaTeX2e tabular which is used here.
%\begin{tabular}{|c||c|}
%\hline
%One & Two\\
%\hline
%Three & Four\\
%\hline
%\end{tabular}
%\end{table}

% Note that the IEEE does not put floats in the very first column
% - or typically anywhere on the first page for that matter. Also,
% in-text middle ("here") positioning is typically not used, but it
% is allowed and encouraged for Computer Society conferences (but
% not Computer Society journals). Most IEEE journals/conferences use
% top floats exclusively.
% Note that, LaTeX2e, unlike IEEE journals/conferences, places
% footnotes above bottom floats. This can be corrected via the
% \fnbelowfloat command of the stfloats package.

%%% END IEEE INSTRUCTIONS / TEMPLATE %%%


\glsresetall

% \import{content/paper/sections/}{0_round0}

\import{content/paper/sections/}{1_introduction}

\import{content/paper/sections/}{2_background}

\import{content/paper/sections/}{3_related_work}

\import{content/paper/sections/}{4_methodology}

\import{content/paper/sections/}{5_experimental_setup}

\import{content/paper/sections/}{6_evaluation}

\import{content/paper/sections/}{7_conclusion}

% \import{content/paper/sections/}{a_acknowledgment}

\import{content/paper/sections/}{b_references}
