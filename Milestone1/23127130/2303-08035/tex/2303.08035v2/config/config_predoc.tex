%%% PACKAGE CONFIGS %%%

% setup for captions
\captionsetup{format=plain}
% more complex caption setup to add citation with \setcaptioncitation
% not working well, use \caption*
% \DeclareCaptionFormat{citation}{%
%   \ifx\captioncitation\relax\relax\else
%     \captioncitation\par
%   \fi
%   #1#2#3\par}
% \newcommand*\setcaptioncitation[1]{\def\captioncitation{\textit{Source:}~#1}}
% \let\captioncitation\relax
% \captionsetup{format=citation,justification=centering}

% setup for hyperlinks
% it clashes with \author in titlepage_setup for the presentation
% it should be placed after it, so that the original author
% can be set with beamer
% NOTE: the clash is problematic if the title page setup is run in the
% document, but it works outside of it
\hypersetup{
    colorlinks=true,
    % we want citations and internal links to have no color
    citecolor=.,
    linkcolor=.,
    % only URLs should have colors
    urlcolor={ForestGreen},
}

\sisetup{detect-weight=true,detect-family=true}

% setup for svg
% NOTE: The latest version of Overleaf is not working with SVG files
% the inkscape command is not run at all
% the solution is uploading the files manually
\svgsetup{%
    % only for svg-extract
    % extractformat={pdf},
    % to export all the drawing even if it goes outside the page area
    inkscapearea=drawing,
    % in this way the svgsubdir svg-inkscape will be created inside the path
    % to the svg image
    inkscapepath=svgsubdir,
    inkscapeformat=pdf, % to force usage of PDF
    inkscapelatex=false, % to disable latex rendering of text, produces errors
    inkscapeopt=-T,  % in this way, we pass the option to export the text as path, limiting compatibility issues
}

% paths don't work well with import, it requires deeper analysis

% path for svgs
\svgpath{{images/vector/}}

% main path for graphics
\graphicspath{{images/raster/}{images/vector/exported-pdf/}}

% avoid hypenation
\lefthyphenmin9
\righthyphenmin9

% to set the thousand separators for numbers
% \, means a small space
\npthousandsep{\,}

% to use gls in caption without \protect, must be done for all used versions of gls
\robustify{\glsentrytitlecase}
\robustify{\gls}
\robustify{\glsxtrshort}
\robustify{\glsxtrshortpl}
\robustify{\GlsXtrIfUnusedOrUndefined}
\robustify{\glsunset}

%%% PACKAGE CONFIGS %%%


%%% GLOSSARIES-EXTRA CONFIGS %%%

% we call the resource containing the definitions
% not needed when using bib2gls
%\makeglossaries

% not required for abbreviations
% \setabbreviationstyle[acronym]{long-short}
\loadglsentries{abbreviations/abbreviations}
\glsaddall

% in this way the glossary for abbreviations will show title-cased descriptions
\glssetcategoryattribute{abbreviation}{glossdesc}{title}

% to disable hyperlink for every entry
\setkeys{glslink}{hyper=false}
% to disable all hyperlink for entries for a specific category
% use \glssetcategoryattribute{<category>}{nohyper}{true}

%%% END GLOSSARIES-EXTRA CONFIGS %%%


%%% FONT DIMENSIONS %%%

% custom lengths, used for the spacing around floats and captions
% \setlength{\textfloatsep}{5pt plus 2.0pt minus 3.0pt}
% \setlength{\dbltextfloatsep}{5pt plus 2.0pt minus 3.0pt}
% \setlength{\floatsep}{5pt plus 2.0pt minus 3.0pt}
% \setlength{\dblfloatsep}{5pt plus 2.0pt minus 3.0pt}
% \setlength{\intextsep}{5pt plus 2.0pt minus 3.0pt}
% \setlength{\abovecaptionskip}{2pt plus 1.0pt minus 1.0pt}
% \setlength{\belowcaptionskip}{2pt plus 1.0pt minus 1.0pt}

% custom dimension
% \renewcommand{\floatpagefraction}{.9}
% \setcounter{topnumber}{99}
% \renewcommand{\topfraction}{0.99}
% \def\BibTeX{{\rm B\kern-.05em{\sc i\kern-.025em b}\kern-.08em
%     T\kern-.1667em\lower.7ex\hbox{E}\kern-.125emX}}

%%% END FONT DIMENSIONS %%%


%%% BIBLIOGRAPHY AND ITS FONT SIZES %%%

% bibliography file
\addbibresource{references/bibliography.bib}

% to change font size for bibliography
% NOT NEEDED AS IT IS DONE IN EACH SPECIFIC DOCUMENT TYPE
% ONLY RENEWCOMMAND IS REQUIRED
% when using biblatex
% default is \normalsize
% font list:
% \tiny, \scriptsize, \footnotesize, \small, \normalsize, \large, \Large, \LARGE, \huge and \Huge
% \def\bibliographyfontsize{\fontsize{7.3}{8.3pt}\selectfont}
\renewcommand*{\bibfont}{\bibliographyfontsize}

%%% END BIBLIOGRAPHY AND ITS FONT SIZES %%%


%%% WORK NAME AND INFO COMMANDS %%%
% for emphasizing words in a coherent way
\newcommand{\emphasis}[1]{\emph{#1}}

% this command emphasizes the work name
\newcommand{\emphasizedworkname}{\emphasis{\workname}}

% this command makes the presenting author email into a link
\newcommand{\presentingauthoremailwithlink}{\href{mailto:\presentingauthoremail}{\presentingauthoremail}}

% this command makes the presenting author webpage into a link
\newcommand{\presentingauthorwebpagewithlink}{\href{\presentingauthorwebpage}{\presentingauthorwebpage}}

%%% END WORK NAME AND INFO COMMANDS

%%% MATH COMMANDS

% taken from https://arxiv.org/format/1805.12233
\newcommand{\esm}[1]{\ensuremath{#1}}
\newcommand{\mr}[1]{\esm{\mathrm{#1}}}
\newcommand{\ms}[1]{\esm{\mathsf{#1}}}
\newcommand{\mi}[1]{\esm{\mathit{#1}}}
\newcommand{\mb}[1]{\esm{\mathbf{#1}}}
\newcommand{\mathsc}[1]{{\normalfont \textsc{#1}}}
\newcommand{\msc}[1]{\esm{\mathsc{#1}}}

%%% END MATH COMMANDS


%%% CUSTOM COMMANDS %%%

% for checkmark and xmark, for doing ticks and xs
\newcommand{\customcheckmark}{\ding{51}}
\newcommand{\customxmark}{\ding{55}}
% https://tex.stackexchange.com/a/384538
% in this way we can call \colourmark[green,customcheckmark] to use \greencustomcheckmark
\newcommand*\colourmark[3]{%
  \expandafter\newcommand\csname #3\endcsname{\textcolor{#1}{#2}}}
% \newcommad{\greencheckmark}{{\textcolor{green}\customcheckmark}}
% \newcommad{\redxmark}{{\textcolor{red}\customxmark}}

% line comments for algorithm pseudocode
% use \LComment
% \algnewcommand{\LineComment}[1]{\State \(\triangleright\) #1}

% blue highlight for new parts
\newcommand{\blue}[1]{\textcolor{blue}{#1}}

% green highlight for comments
\newcommand{\green}[1]{\textcolor{green}{#1}}

% red highlight for todos, comments and corrections
\newcommand{\red}[1]{\textcolor{red}{#1}}

% for printing symbols in author list
\makeatletter
\newcommand{\printfnsymbol}[1]{%
    \textsuperscript{\@fnsymbol{#1}}%
}
\makeatother

% to have a proper inline equation
% first argument is label, second is equation
\makeatletter
\newcommand*{\inlineequation}[2][]{%
  \begingroup
    % Put \refstepcounter at the beginning, because
    % package `hyperref' sets the anchor here.
    \refstepcounter{equation}%
    \ifx\\#1\\%
    \else
      \label{#1}%
    \fi
    % prevent line breaks inside equation
    \relpenalty=10000 %
    \binoppenalty=10000 %
    \ensuremath{%
      % \displaystyle % larger fractions, ...
      #2%
    }%
    ~\@eqnnum
  \endgroup
}
\makeatother

% circled is used when referring to circles in figures,
% it shows the number inside a small circle
% this new version produces no errors and it looks identical
\newcommand*{\circled}[2][]{\tikz[baseline=(C.base)]{
    \node[inner sep=0pt] (C) {\vphantom{1g}#2};
    \node[draw, circle, inner sep=0.8pt, yshift=1pt]
        at (C.center) {\vphantom{1g}};}}

% an alternative to the custom \circled,
% one can use the package circledsteps and
% use the following options to change the way they look
% black circle filling and white text
\definecolor{blanchedalmond}{rgb}{1.0, 0.92, 0.8}
\definecolor{champagne}{rgb}{0.97, 0.91, 0.81}
 	\definecolor{beige}{rgb}{0.96, 0.96, 0.86}
 	\definecolor{carnelian}{rgb}{0.7, 0.11, 0.11}
\definecolor{crimson}{rgb}{0.86, 0.08, 0.24}
\definecolor{yellowcirclefill}{RGB}{255, 246, 221}
\definecolor{redcircleborder}{RGB}{192, 0, 0}
\definecolor{inkscapered}{RGB}{255, 0, 0}
\definecolor{inkscapedarkgreen}{RGB}{0, 128, 0}

\newcommand{\BlackCircled}[1]{\Circled[inner color=white,outer color=black,fill color=black]{#1}}
% a better version with bold and small included
\newcommand{\GoodBlackCircled}[1]{\BlackCircled{\textbf{\small{#1}}}}
% yellowish filling and red circle and text
% \newcommand{\RedCircled}[1]{\Circled[inner color=crimson,outer color=crimson,fill color=beige]{#1}}
\newcommand{\RedCircled}[1]{\Circled[inner color=white,outer color=inkscapered,fill color=inkscapered]{#1}}
% a better version with bold and small included
\newcommand{\GoodRedCircled}[1]{\RedCircled{\textbf{\small{#1}}}}
% using inkscapegreen here
\newcommand{\DarkGreenCircled}[1]{\Circled[inner color=white,outer color=inkscapedarkgreen,fill color=inkscapedarkgreen]{#1}}
% a better version with bold and small included
\newcommand{\GoodDarkGreenCircled}[1]{\DarkGreenCircled{\textbf{\small{#1}}}}
% % an alternative to the custom \circled, one can use the package circledsteps and use the following options to change the way they look
% \pgfkeys{/csteps/inner color=white}
% % we use none to disable it
% \pgfkeys{/csteps/outer color=black}
% \pgfkeys{/csteps/fill color=black}

% both titlecase have issues if unused and in a caption/title
% to title case an abbreviation
% \GlsXtrIfUnusedOrUndefined{label}{true}{false} for checking whether a label is set with bib2gls, otherwise use ifglsused
% docs http://tug.ctan.org/macros/latex/contrib/glossaries/glossaries-user.pdf page 175
\newcommand{\titlecaseabbreviation}[1]{\GlsXtrIfUnusedOrUndefined{#1}{\glsentrytitlecase{#1}{long}\space(\glsxtrshort{#1})\glsunset{#1}}{\glsxtrshort{#1}\glsunset{#1}}}
% to title case an abbreviation with plural
% \GlsXtrIfUnusedOrUndefined{label}{true}{false} for checking whether a label is set with bib2gls, otherwise use ifglsused
% docs http://tug.ctan.org/macros/latex/contrib/glossaries/glossaries-user.pdf page 175
\newcommand{\titlecaseabbreviationpl}[1]{\GlsXtrIfUnusedOrUndefined{#1}{\glsentrytitlecase{#1}{longpl}\space(\glsxtrshortpl{#1})\glsunset{#1}}{\glsxtrshortpl{#1}\glsunset{#1}}}


% to have footnotes without number, either using \blankfootnote{text} or \footnote[]{text}
% https://tex.stackexchange.com/questions/170511/footnotes-without-numbering
% the following line saves the current definition of footnote
% so that it can be reloaded after using the custom footnote
\let\svthefootnote\thefootnote
\newcommand\blankfootnote[1]{%
  \let\thefootnote\relax\footnotetext{#1}%
  \let\thefootnote\svthefootnote%
}
\let\svfootnote\footnote
\renewcommand\footnote[2][?]{%
  \if\relax#1\relax%
    \blankfootnote{#2}%
  \else%
    \if?#1\svfootnote{#2}\else\svfootnote[#1]{#2}\fi%
  \fi
}

% this command sets up the page style for the to appear at header
% it requires the text to be put in the header
\newcommand{\setuptoappearheader}[1]{%
    \fancypagestyle{toappearfirstpageheader}{% Page style for first page
        \fancyhf{}% Clear header/footer
        %\renewcommand{\headrulewidth}{0.4pt}% Header rule
        %\renewcommand{\footrulewidth}{0.4pt}% Footer rule
        \chead{#1}  % centered header, there is the same for l/r and foot
        %\fancyfoot[C]{\thepage}% Footer % this sets up the number of the page in the footer
    }
}

% this commands sets the current page style to the one defined with the previous command, by setting the text
\newcommand{\toappearheader}{%
    \thispagestyle{toappearfirstpageheader}
}

%%% END CUSTOM COMMANDS %%%
\let\svfootnote\footnote

\let\svfootnote\footnote
