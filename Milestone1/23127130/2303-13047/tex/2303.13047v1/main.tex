\documentclass{article}


% if you need to pass options to natbib, use, e.g.:
\PassOptionsToPackage{numbers, compress}{natbib}
% before loading neurips_2022

% ready for submission
% \usepackage{neurips_2022}
\usepackage[preprint]{neurips_2022}
% \usepackage[final]{neurips_2022}
% \usepackage[nonatbib]{neurips_2022}

% to compile a preprint version, e.g., for submission to arXiv, add add the
% [preprint] option:
%     \usepackage[preprint]{neurips_2022}


% to compile a camera-ready version, add the [final] option, e.g.:
    % \usepackage[final]{neurips_2022}


% to avoid loading the natbib package, add option nonatbib:
%    \usepackage[nonatbib]{neurips_2022}


\usepackage[utf8]{inputenc} % allow utf-8 input
\usepackage[T1]{fontenc}    % use 8-bit T1 fonts
\usepackage[hidelinks]{hyperref}       % hyperlinks
\usepackage{url}            % simple URL typesetting
\usepackage{booktabs}       % professional-quality tables
\usepackage{amsfonts}       % blackboard math symbols
\usepackage{nicefrac}       % compact symbols for 1/2, etc.
\usepackage{microtype}      % microtypography
\usepackage{xcolor}         % colors

\usepackage{bm}
\usepackage{amsmath}
\usepackage{amssymb}
\usepackage{multirow}
\usepackage{subfigure}
\usepackage{array}
\usepackage{amsthm}
\usepackage{rotating}
\usepackage{pifont}
\usepackage{wrapfig}

% \title{Dynamic Graph Learning with Transformers}
\title{Towards Better Dynamic Graph Learning: \\New Architecture and Unified Library}

% The \author macro works with any number of authors. There are two commands
% used to separate the names and addresses of multiple authors: \And and \AND.
%
% Using \And between authors leaves it to LaTeX to determine where to break the
% lines. Using \AND forces a line break at that point. So, if LaTeX puts 3 of 4
% authors names on the first line, and the last on the second line, try using
% \AND instead of \And before the third author name.


\author{%
  % Anonymous Author(s)
  Le Yu, Leilei Sun, Bowen Du, Weifeng Lv\\
  % \thanks{Use footnote for providing further information
    % about author (webpage, alternative address)---\emph{not} for acknowledging
    % funding agencies.} \\
  School of Computer Science and Engineering\\
  State Key Laboratory of Software Development Environment\\
  Beihang University\\
  \texttt{\{yule,leileisun,dubowen,lwf\}@buaa.edu.cn} \\
  % examples of more authors
  % \And
  % Coauthor \\
  % Affiliation \\
  % Address \\
  % \texttt{email} \\
  % \AND
  % Coauthor \\
  % Affiliation \\
  % Address \\
  % \texttt{email} \\
  % \And
  % Coauthor \\
  % Affiliation \\
  % Address \\
  % \texttt{email} \\
}


\newtheorem{definition}{Definition}
% \newcommand{\citet}[1]{\citeauthor{#1} \shortcite{#1}}
% \newcommand{\citet}[1]{\cite \ref{#1}}
\newcommand{\figref}[1]{Figure \ref{#1}}
\newcommand{\tabref}[1]{Table \ref{#1}}
\newcommand{\secref}[1]{Section \ref{#1}}
\newcommand{\equref}[1]{Equation (\ref{#1})}
\newcommand{\appendixref}[1]{Appendix}
\newcommand{\algoref}[1]{Algorithm \ref{#1}}

\begin{document}


\maketitle


\begin{abstract}
We propose DyGFormer, a new Transformer-based architecture for dynamic graph learning that solely learns from the sequences of nodes' historical first-hop interactions. DyGFormer incorporates two distinct designs: (\romannumeral1) a neighbor co-occurrence encoding scheme that explores the correlations of the source node and destination node based on their sequences; (\romannumeral2) a patching technique that divides each sequence into multiple patches and feeds them to Transformer, allowing the model to effectively and efficiently benefit from longer histories. We also introduce DyGLib, a unified library with standard training pipelines, extensible coding interfaces, and comprehensive evaluating protocols to promote reproducible, scalable, and credible dynamic graph learning research. By performing extensive experiments on thirteen datasets from various domains for transductive/inductive dynamic link prediction and dynamic node classification tasks, we observe that: (\romannumeral1) DyGFormer achieves state-of-the-art performance on most of the datasets, demonstrating the effectiveness of capturing nodes' correlations and long-term temporal dependencies; (\romannumeral2) the results of baselines vary across different datasets and some findings are inconsistent with previous reports, which may be caused by their diverse pipelines and problematic implementations. We hope our work can provide new insights and facilitate the development of the dynamic graph learning field. All the resources including datasets, data loaders, algorithms, and executing scripts are publicly available at \url{https://github.com/yule-BUAA/DyGLib}.


\end{abstract}

\section{Introduction}
\label{section-1}

Dynamic graphs denote entities as nodes and represent their interactions as edges with timestamps \cite{DBLP:journals/jmlr/KazemiGJKSFP20}, which are pervasive in a variety of real-world scenarios such as social networks \cite{DBLP:conf/kdd/KumarZL19,DBLP:conf/wsdm/Song0WCZT19,alvarez2021evolutionary}, user-item interaction systems \cite{DBLP:conf/icdm/LiZWLWY20,DBLP:conf/cikm/FanLZX0Y21,DBLP:conf/www/YuWS0L22,zhang2022dynamic,DBLP:journals/corr/abs-2204-05490}, traffic networks \cite{DBLP:conf/ijcai/YuYZ18,DBLP:conf/ijcai/WuPLJZ19,DBLP:conf/aaai/GuoLFSW19,DBLP:conf/nips/0001YL0020,DBLP:journals/ijon/YuDHSHL21}, and physical systems \cite{DBLP:conf/nips/HuangS020,DBLP:conf/icml/Sanchez-Gonzalez20,DBLP:conf/iclr/PfaffFSB21}. In recent years, representation learning on dynamic graphs has attracted the wide attention of many scholars \cite{DBLP:journals/jmlr/KazemiGJKSFP20,skarding2021foundations,xue2022dynamic}. Existing methods can be roughly divided into two categories: discrete-time methods \cite{DBLP:conf/aaai/ParejaDCMSKKSL20,DBLP:journals/kbs/GoyalCC20,DBLP:conf/wsdm/SankarWGZY20,DBLP:journals/corr/abs-2111-10447,DBLP:conf/kdd/YouDL22} and continuous-time methods \cite{DBLP:conf/kdd/KumarZL19,DBLP:conf/iclr/TrivediFBZ19,DBLP:conf/iclr/XuRKKA20,DBLP:journals/corr/abs-2006-10637,DBLP:conf/sigir/0001GRTY20,DBLP:conf/cikm/ChangLW0FS020,DBLP:journals/corr/abs-2105-07944,DBLP:conf/iclr/WangCLL021,DBLP:conf/sigmod/WangLLXYWWCYSG21,jin2022neural,luo2022neighborhoodaware,cong2023do}. In this paper, we focus on the latter approaches because they can offer better flexibility and performance than the formers and are being increasingly investigated.

% Issues of existing methods for dynamic graph learning
Despite the rapid development of dynamic graph learning methods, they still suffer from two limitations. On the one hand, most of the previous methods independently learn the temporal representations of nodes in an interaction without exploiting their correlations, which are often indicative of future interactions. Moreover, existing methods follow the interaction-level learning paradigm and only work for nodes with fewer interactions. When nodes have longer histories, they rely on sampling strategies to truncate the interactions for feasible calculations of the computationally expensive modules such as graph convolutions \cite{DBLP:conf/iclr/TrivediFBZ19,DBLP:conf/iclr/XuRKKA20,DBLP:journals/corr/abs-2006-10637,DBLP:conf/sigir/0001GRTY20,DBLP:conf/cikm/ChangLW0FS020,DBLP:conf/sigmod/WangLLXYWWCYSG21}, temporal random walks \cite{DBLP:conf/iclr/WangCLL021,jin2022neural} and sequential models \cite{DBLP:journals/corr/abs-2105-07944,cong2023do}. Even if some approaches utilize memory networks \cite{DBLP:journals/corr/WestonCB14,DBLP:conf/nips/SukhbaatarSWF15} to sequentially process interactions with affordable computational costs \cite{DBLP:conf/kdd/KumarZL19,DBLP:conf/iclr/TrivediFBZ19,DBLP:journals/corr/abs-2006-10637,DBLP:conf/sigir/0001GRTY20,DBLP:conf/sigmod/WangLLXYWWCYSG21,luo2022neighborhoodaware}, they are usually troubled with the staleness problem \cite{DBLP:journals/jmlr/KazemiGJKSFP20} or vanishing/exploding gradient issues due to the usage of recurrent neural networks \cite{DBLP:conf/icml/PascanuMB13,DBLP:journals/corr/abs-2105-07944}. Therefore, we conclude that \textit{previous methods lack the ability in capturing either the correlations between nodes or long-term temporal dependencies}. 

On the other hand, the training pipelines of different methods are inconsistent and lead to poor reproducibility.
% Some methods even have problematic implementations (See \secref{section-appendix-issues-existing-methods} for more details.).
Moreover, existing methods are implemented by diverse frameworks (e.g., Pytorch \cite{DBLP:conf/nips/PaszkeGMLBCKLGA19}, Tensorflow \cite{DBLP:conf/osdi/AbadiBCCDDDGIIK16}, DGL \cite{DBLP:journals/corr/abs-1909-01315}, PyG \cite{DBLP:journals/corr/abs-1903-02428}, C++), making it time-consuming and difficult for researchers to quickly understand the algorithms and further dive into the core of dynamic graph learning. Although there exist some libraries for dynamic graph learning \cite{DBLP:journals/corr/abs-1811-10734,DBLP:conf/cikm/RozemberczkiSHP21,zhou2022tgl}, they mainly focus on dynamic network embedding methods \cite{DBLP:journals/corr/abs-1811-10734}, discrete-time graph learning methods \cite{DBLP:conf/cikm/RozemberczkiSHP21}, or engineering techniques for training on large-scale dynamic graphs \cite{zhou2022tgl} (elaborated in \secref{section-2}). Currently, we find that \textit{there are still no standard tools for continuous-time dynamic graph learning}.

% Present work and contributions
In this paper, we aim to address the above drawbacks for better learning on dynamic graphs. We have the following two key technical contributions.
% and our technical contributions are two-fold.
% (\romannumeral1) a new Transformer-based dynamic graph learning architecture to capture correlations between the source node and destination node as well as long-term temporal dependencies; and (\romannumeral2) a unified continuous-time dynamic graph learning library to facilitate reproducible, scalable, and credible dynamic graph learning research.

\textbf{We propose a new Transformer-based dynamic graph learning architecture (DyGFormer)}. DyGFormer is conceptually simple and only needs to learn from the sequences of nodes’ historical first-hop interactions. To be specific, DyGFormer is equipped with a neighbor co-occurrence encoding scheme, which encodes the appearing frequencies of each neighbor in the sequences of the source node and destination node, and can explicitly explore the correlations between two nodes. Instead of learning at the interaction level, DyGFormer splits each source/destination node's sequence into multiple patches and feeds them to Transformer \cite{DBLP:conf/nips/VaswaniSPUJGKP17}. The patching technique allows DyGFormer to capture long-term temporal dependencies since it can not only make DyGFormer effectively benefit from longer histories by preserving the local temporal proximities, but also efficiently reduce the computational complexity to a constant level that is irrelevant to the input sequence length.

% ability to effectively and efficiently break through the bottleneck of previous methods in capturing long-term temporal dependencies via three advantages: \textcolor{red}{local temporal proximities are preserved in each patch; the computational complexity is reduced to a constant level that is irrelevant to the input sequence length; longer histories can be attended by the model.}
% efficiently the computational complexity is reduced to a constant level that is irrelevant to the input sequence length and effectively make the model benefit from longer histories.

\textbf{We present a unified continuous-time dynamic graph learning library (DyGLib)}. DyGLib is an open-source toolkit with standard training pipelines, extensible coding interfaces, and comprehensive evaluating strategies, aiming to foster standard, scalable, and reproducible dynamic graph learning research. DyGLib is implemented by PyTorch and has integrated thirteen datasets from various domains as well as nine continuous-time dynamic graph learning methods (including DyGFormer). DyGLib trains all the methods via the same pipeline to eliminate the influence of different implementations of previous methods. It also adopts a modularized design for good extensibility and allows developers to conveniently incorporate new datasets and algorithms based on specific requirements. Moreover, DyGLib supports the commonly used dynamic link prediction and dynamic node classification tasks with exhaustive evaluating strategies to provide comprehensive comparisons of existing methods. 
% DyGLib significantly decreases the usage difficulty and makes it easier for researchers to go deep into the dynamic graph learning field. 

To evaluate the model performance, we conduct extensive experiments based on DyGLib, including dynamic link prediction under both transductive and inductive settings with three negative sampling strategies as well as the dynamic node classification task. From the results, we find that: (\romannumeral1) DyGFormer outperforms existing methods on most datasets, which demonstrates its advantages in capturing nodes' correlations and long-term temporal dependencies; (\romannumeral2) the baselines show unstable results across different datasets and some observations are inconsistent with those in previous works due to their less rigorous evaluations. 
We also provide an in-depth analysis of the neighbor co-occurrence encoding and patching technique for a better understanding of DyGFormer.

% The rest of this paper is organized as follows:
% \secref{section-2} reviews the related research.
% \secref{section-3} presents the background of the studied problem.
% \secref{section-4} presents the framework and introduces each component step by step.
% \secref{section-5} conducts experiments to evaluate the proposed model.
% Finally, \secref{section-6} concludes the entire paper.


\section{Related Work}
\label{section-2}

\textbf{Dynamic Graph Learning}.
Representation learning on dynamic graphs has been widely studied in recent years \cite{DBLP:journals/jmlr/KazemiGJKSFP20,skarding2021foundations,xue2022dynamic}. 
Discrete-time methods manually divide the dynamic graph into a sequence of snapshots and apply static graph learning methods on each snapshot, which ignore the temporal order of nodes in each snapshot \cite{DBLP:conf/aaai/ParejaDCMSKKSL20,DBLP:journals/kbs/GoyalCC20,DBLP:conf/wsdm/SankarWGZY20,DBLP:journals/corr/abs-2111-10447,DBLP:conf/kdd/YouDL22}. In contrast, continuous-time methods directly learn on the whole dynamic graph with temporal graph neural networks \cite{DBLP:conf/iclr/TrivediFBZ19,DBLP:conf/iclr/XuRKKA20,DBLP:journals/corr/abs-2006-10637,DBLP:conf/sigir/0001GRTY20,DBLP:conf/cikm/ChangLW0FS020,DBLP:conf/sigmod/WangLLXYWWCYSG21}, memory networks \cite{DBLP:conf/kdd/KumarZL19,DBLP:conf/iclr/TrivediFBZ19,DBLP:journals/corr/abs-2006-10637,DBLP:conf/sigir/0001GRTY20,DBLP:conf/sigmod/WangLLXYWWCYSG21,luo2022neighborhoodaware}, temporal random walks \cite{DBLP:conf/iclr/WangCLL021,jin2022neural} or sequential models \cite{DBLP:journals/corr/abs-2105-07944,cong2023do}. Although insightful, most existing dynamic graph learning methods neglect the correlations between two nodes in an interaction. They also fail to handle nodes with longer interactions due to unaffordable computational costs of complex modules or issues in optimizing models (e.g., the vanishing/exploding gradients). In this paper, we propose a new Transformer-based architecture to show the necessity of capturing nodes' correlations and long-term temporal dependencies, which is achieved by two designs: a neighbor co-occurrence encoding scheme and a patching technique.

\textbf{Transformer-based Applications in Various Fields}. 
Transformer \cite{DBLP:conf/nips/VaswaniSPUJGKP17} is an innovative model that employs the self-attention mechanism to handle sequential data, which has been successfully applied in a variety of domains, such as natural language processing \cite{DBLP:conf/naacl/DevlinCLT19,DBLP:journals/corr/abs-1907-11692,DBLP:conf/nips/BrownMRSKDNSSAA20}, computer vision \cite{DBLP:conf/eccv/CarionMSUKZ20,DBLP:conf/iclr/DosovitskiyB0WZ21,DBLP:conf/iccv/LiuL00W0LG21} and time series forecasting \cite{DBLP:conf/nips/LiJXZCWY19,DBLP:conf/aaai/ZhouZPZLXZ21,DBLP:conf/nips/WuXWL21}. The idea of dividing the original data into patches as inputs of the Transformer has been attempted in some studies. ViT \cite{DBLP:conf/iclr/DosovitskiyB0WZ21} splits an image
into multiple patches and feeds the sequence of patches' linear embeddings into a Transformer, which achieves surprisingly good performance on image classification. PatchTST \cite{nie2023a} divides a time series into subseries-level patches and calculates the patches by a channel-independent Transformer for long-term multivariate time series forecasting. In this work, we propose a patching technique to learn on dynamic graphs, which can provide our approach with the ability to handle nodes with longer histories.

\textbf{Graph Learning Library}. Currently, there exist many libraries for static graphs \cite{DBLP:journals/corr/abs-1806-01261,DBLP:journals/corr/abs-1909-01315,DBLP:journals/corr/abs-1903-02428,DBLP:conf/nips/HuFZDRLCL20,DBLP:journals/corr/abs-2103-00959,DBLP:conf/icse/LiXCZL21,DBLP:journals/jmlr/LiuLWXYGYXZLYLF21}, but few for dynamic graph learning \cite{DBLP:journals/corr/abs-1811-10734,DBLP:conf/cikm/RozemberczkiSHP21,zhou2022tgl}. 
DynamicGEM \cite{DBLP:journals/corr/abs-1811-10734} focuses on dynamic graph embedding methods, which just consider the graph topology and cannot leverage node features. PyTorch Geometric Temporal \cite{DBLP:conf/cikm/RozemberczkiSHP21} implements discrete-time algorithms for spatiotemporal signal processing and is mainly applicable for nodes with aligned historical observations. TGL \cite{zhou2022tgl} trains on large-scale dynamic graphs with some engineering tricks. Though TGL has integrated some continuous-time methods, they are somewhat out-of-date, resulting in the lack of state-of-the-art models. Moreover, TGL is implemented by both PyTorch and C++, which needs additional compilation and increases the usage difficulty. In this paper, we present a unified continuous-time dynamic graph learning library with thorough baselines, diverse datasets, extensible implementations, and comprehensive evaluations to facilitate dynamic graph learning research.


\section{Preliminaries}
\label{section-3}

% In this paper, we focus on continuous-time dynamic graph learning methods, which have been demonstrated to be more flexible and effective than discrete-time dynamic graph learning methods \cite{DBLP:conf/kdd/KumarZL19,DBLP:conf/iclr/TrivediFBZ19,DBLP:conf/iclr/XuRKKA20,DBLP:journals/corr/abs-2006-10637,DBLP:conf/sigir/0001GRTY20,DBLP:conf/cikm/ChangLW0FS020,DBLP:journals/corr/abs-2105-07944,DBLP:conf/iclr/WangCLL021,DBLP:conf/sigmod/WangLLXYWWCYSG21,jin2022neural,luo2022neighborhoodaware,cong2023do}. 

\begin{definition}
    \textbf{Dynamic Graph}. We represent a dynamic graph as a sequence of non-decreasing chronological interactions $\mathcal{G}=\left\{\left(u_1,v_1,t_1\right), \left(u_2,v_2,t_2\right), \cdots \right\}$ with $0 \leq t_1 \leq t_2 \leq \cdots$, where $u_i, v_i \in \mathcal{N}$ denote the source node and destination node of the $i$-th link at timestamp $t_i$.
    % \footnote{In this paper, we mainly focus on link addition, which is a widely studied interaction type in previous research. We leave the investigations of other interaction types (e.g., node addition/deletion/feature transformations and link deletion/feature transformations) for future work.}. 
    $\mathcal{N}$ is the set of all the nodes. Each node $u \in \mathcal{N}$ can be associated with node feature $\bm{x}_u \in \mathbb{R}^{d_N}$, and each interaction $\left(u,v,t\right)$ has link feature $\bm{e}_{u,v}^t \in \mathbb{R}^{d_E}$. $d_N$ and $d_E$ denote the dimensions of the node feature and link feature. If the graph is non-attributed, we simply set the node feature and link feature to zero vectors, i.e., $\bm{x}_u=\bm{0}$ and $\bm{e}_{u,v}^t=\bm{0}$. 
\end{definition}

\begin{definition}
    \textbf{Problem Formalization}. Given the source node $u$, destination node $v$, timestamp $t$, and historical interactions before $t$, i.e., $\left\{\left(u^\prime,v^\prime,t^\prime\right) | t^\prime < t \right\}$, representation learning on dynamic graph aims to design a model to learn time-aware representations $\bm{h}_u^t \in \mathbb{R}^d$ and $\bm{h}_v^t \in \mathbb{R}^d$ for $u$ and $v$ with $d$ as the dimension. We validate the effectiveness of the learned representations via two classic tasks in dynamic graph learning: (\romannumeral1) dynamic link prediction, which predicts whether $u$ and $v$ are connected at $t$; (\romannumeral2) dynamic node classification, which infers the state of $u$ or $v$ at $t$.
\end{definition}


% which predicts the probability of an edge connecting two nodes at a specific timestamp
% which classifies the state of a node in an interaction at a specific timestamp

% We denote a dynamic graph as a sequence of events with timestamps $\mathcal{G}=\left(e\left(t_1\right), e_2\left(t_2\right), \cdots\right)$ with $0 \leq t_1 \leq t_2 \leq \cdots$, where event $e\left(t\right)$ can be either node-wise or pairwise:



\section{Methodology}
\label{section-4}

% We promote the development of dynamic graph learning by presenting: (\romannumeral1) a new Transformer-based architecture; and (\romannumeral2) a unified continuous-time dynamic graph learning library.

\subsection{DyGFormer: Transformer-based Architecture for Dynamic Graph Learning}
\begin{figure}[!ht]
    \centering
    \includegraphics[width=1.0\columnwidth]{figures/DyGFormer_framework.jpg}
    % \includegraphics[scale=0.4]{figures/DyGFormer_framework.jpg}
    \caption{Framework of the proposed model.}
    \label{fig:DyGFormer_framework}
\end{figure}
The framework of our model is shown in \figref{fig:DyGFormer_framework}, which employs Transformer \cite{DBLP:conf/nips/VaswaniSPUJGKP17} as the backbone. Given an interaction $\left(u,v,t\right)$, we first extract historical first-hop interactions of source node $u$ and destination node $v$ before timestamp $t$ and obtain two interaction sequences $\mathcal{S}_u^t$ and $\mathcal{S}_v^t$. Next, in addition to computing the encodings of neighbors, links, and time intervals for each sequence, we also encode the frequencies of every neighbor's appearances in both $\mathcal{S}_u^t$ and $\mathcal{S}_v^t$ to exploit the correlations between $u$ and $v$, resulting in four encoding sequences for $u$ and $v$ in total. Then, we divide each encoding sequence into multiple patches and feed all the patches into a Transformer for capturing long-term temporal dependencies. Finally, the outputs of the Transformer are averaged to derive time-aware representations of $u$ and $v$ at timestamp $t$ (i.e., $\bm{h}_u^t$ and $\bm{h}_v^t$), which can be applied in various downstream tasks, such as dynamic link prediction and dynamic node classification.

\textbf{Learning from Historical First-hop Interactions}. Unlike most previous methods that need to retrieve nodes' interactions from multiple hops, we only learn from the sequences of first-hop interactions of nodes, which turns the dynamic graph learning task into a simpler sequence learning problem. Mathematically, given an interaction $\left(u,v,t\right)$, for source node $u$ and destination node $v$, we respectively obtain the sequences that involve historical interactions of $u$ and $v$ before timestamp $t$ (including $u$ and $v$), which are denoted by $\mathcal{S}_u^t=\left\{\left(u,u^\prime,t^\prime\right) | t^\prime < t \right\} \cup \left\{\left(u^\prime,u,t^\prime\right) | t^\prime < t \right\}$ and $\mathcal{S}_v^t=\left\{\left(v,v^\prime,t^\prime\right) | t^\prime < t \right\} \cup \left\{\left(v^\prime,v,t^\prime\right) | t^\prime < t \right\}$.

\textbf{Encoding Neighbors, Links, and Time Intervals}. As illustrated in \secref{section-3}, a dynamic graph is often associated with the features of nodes and links. Therefore, for node $u$, we just need to retrieve the features of involved neighbors and links in sequence $\mathcal{S}_u^t$ based on the given features to represent their encodings, which are denoted by $\bm{X}_{u,N}^t \in \mathbb{R}^{|\mathcal{S}_u^t| \times d_N}$ and $\bm{X}_{u,E}^t \in \mathbb{R}^{|\mathcal{S}_u^t| \times d_E}$. Following \cite{DBLP:conf/iclr/XuRKKA20}, we learn the periodic temporal patterns by encoding the time interval $\Delta t^\prime = t - t^\prime$ via $\sqrt{\frac{1}{d_T}}\left[\cos\left(w_1 \Delta t^\prime \right), \sin\left(w_1 \Delta t^\prime \right), \cdots, \cos\left(w_{d_T} \Delta t^\prime \right), \sin \left(w_{d_T} \Delta t^\prime \right)\right]$, where $w_1,\cdots,w_{d_T}$ are trainable parameters. $d_T$ is the dimension of time interval encoding. Based on the above function, we get the time interval encodings of interactions in $\mathcal{S}_u^t$, that is, $\bm{X}_{u,T}^t \in \mathbb{R}^{|\mathcal{S}_u^t| \times d_T}$. Following the same process, we can also obtain the corresponding encodings for node $v$, which are denoted by $\bm{X}_{v,N}^t \in \mathbb{R}^{|\mathcal{S}_v^t| \times d_N}$, $\bm{X}_{v,E}^t \in \mathbb{R}^{|\mathcal{S}_v^t| \times d_E}$, and $\bm{X}_{v,T}^t \in \mathbb{R}^{|\mathcal{S}_v^t| \times d_T}$.

\textbf{Neighbor Co-occurrence Encoding}. Existing methods separately compute the temporal representations of node $u$ and $v$ without considering their correlations, which motivates us to present a neighbor co-occurrence encoding scheme to address this limitation. The main assumption is that the appearing frequency of a neighbor in a sequence indicates its importance, and the occurrences of a neighbor in the sequences of $u$ and $v$ (i.e., co-occurrence) could reflect the correlations between $u$ and $v$. That is to say, if $u$ and $v$ have more common historical neighbors in their sequences, they are more likely to have interactions in the future.

Formally, given the interaction sequences $\mathcal{S}_u^t$ and $\mathcal{S}_v^t$, we count the occurrences of each neighbor in both $\mathcal{S}_u^t$ and $\mathcal{S}_v^t$, and derive a two-dimensional vector. By packing together the vectors of all the neighbors, we can get the neighbor co-occurrence features for $u$ and $v$, which are represented by $\bm{C}_{u}^t \in \mathbb{R}^{|\mathcal{S}_u^t| \times 2}$ and $\bm{C}_{v}^t \in \mathbb{R}^{|\mathcal{S}_v^t| \times 2}$. For example, suppose the historical neighbors of $u$ and $v$ are $\left\{a, b, a\right\}$ and $\left\{b, b, a, c\right\}$. The appearing frequencies of $a$, $b$, and $c$ in $u$/$v$'s historical interactions are 2/1, 1/2, and 0/1, respectively. Therefore, the neighbor co-occurrence features of $u$ and $v$ can be denoted by $\bm{C}_u^t=\left[\left[2, 1\right], \left[1, 2\right],\left[2, 1\right]\right]^\top$ and $\bm{C}_v^t=\left[\left[1, 2\right], \left[1, 2\right],\left[2, 1\right],\left[0, 1\right]\right]^\top$. Then, we apply a function $f\left(\cdot\right)$ to encode the neighbor co-occurrence features by 
\begin{equation}
\label{equ:node_co_occurrence_encoding}
\begin{split}
   \bm{X}_{*,C}^t = f\left(\bm{C}_*^t\left[:,0\right]\right) + f\left(\bm{C}_*^t\left[:,1\right]\right) \in \mathbb{R}^{|\mathcal{S}_*^t| \times d_C},
\end{split}
\end{equation}
where $*$ could be $u$ or $v$. The input and output dimensions of $f\left(\cdot\right)$ are 1 and $d_C$. In this paper, we implement $f\left(\cdot\right)$ by a two-layer perceptron with ReLU activation \cite{DBLP:conf/icml/NairH10}.
It is important to note that the neighbor co-occurrence encoding scheme is general and can be easily integrated into some dynamic graph learning methods for better results. We will demonstrate its generalizability in \secref{section-5-node_cooccurrence_generalizability}. 

% $g\left(s,\mathcal{S}\right)=|\left\{\right\}|$
% We then leverage a two-layered Multi-Layer Perceptron with ReLU activation function to encode the neighbor co-occurrence features, which is computed by
% \begin{equation}
% \label{equ:node_co_occurrence_encoding}
% \begin{split}
%    \bm{X}_{u,C}^t = MLP(\bm{C}_u^t\left[:,0\right]) + MLP(\bm{C}_u^t\left[:,1\right]),\\
%     \bm{X}_{v,C}^t = MLP(\bm{C}_v^t\left[:,0\right]) + MLP(\bm{C}_v^t\left[:,1\right]),
% \end{split}
% \end{equation}

\textbf{Patching Technique}. Instead of focusing on the interaction level, we divide the encoding sequence into multiple non-overlapping patches to break through the bottleneck of existing methods in capturing long-term temporal dependencies. Let $P$ denote the patch size, and thus each patch is composed of $P$ temporally adjacent interactions with flattened encodings and can preserve local temporal proximities. Take the patching of $\bm{X}_{u,N}^t \in \mathbb{R}^{|\mathcal{S}_u^t| \times d_N}$ as an example. $\bm{X}_{u,N}^t$ will be divided into $l_u^t =\lceil \frac{|\mathcal{S}_u^t|}{P} \rceil$ patches in total (note that we will pad $\bm{X}_{u,N}^t$ if its length $|\mathcal{S}_u^t|$ cannot be divided by $P$), and the patched encoding is represented by $\bm{M}_{u,N}^t \in \mathbb{R}^{l_u^t \times d_N \cdot P}$. Similarly, we can also get the patched encodings $\bm{M}_{u,E}^t \in \mathbb{R}^{l_u^t  \times d_E \cdot P}$, $\bm{M}_{u,T}^t \in \mathbb{R}^{l_u^t \times d_T \cdot P}$, $\bm{M}_{u,C}^t \in \mathbb{R}^{l_u^t  \times d_C \cdot P}$, 
$\bm{M}_{v,N}^t \in \mathbb{R}^{l_v^t  \times d_N \cdot P}$, $\bm{M}_{v,E}^t \in \mathbb{R}^{l_v^t  \times d_E \cdot P}$, $\bm{M}_{v,T}^t \in \mathbb{R}^{l_v^t  \times d_T \cdot P}$, and $\bm{M}_{v,C}^t \in \mathbb{R}^{l_v^t  \times d_C \cdot P}$. Note that when $|\mathcal{S}_u^t|$ becomes longer, we will correspondingly increase $P$, making the number of patches (i.e., $l_u^t$ and $l_v^t$) at a constant level to reduce the computational cost. 
% The patch technique allows DyGFormer to effectively utilize longer histories via preserving local temporal proximities. Additionally, it reduces computational complexity to a constant level that is independent of the input sequence length.
% preserve the temporal proximities in each patch; decrease the model complexity to a constant level that is agnostic to the input sequence length; and enable the model to access longer histories. These characteristics will be validated in \secref{section-5-investigation_patch_technique}.

\textbf{Transformer Encoder}. We first align the patched encodings to the same dimension $d$ with trainable weight $\bm{W}_* \in \mathbb{R}^{d_* \cdot P \times d}$ and $\bm{b}_* \in \mathbb{R}^{d}$ to obtain $\bm{Z}_{u,*}^t \in \mathbb{R}^{l_u^t \times d}$ and $\bm{Z}_{v,*}^t \in \mathbb{R}^{l_v^t \times d}$, where $*$ could be $N$, $E$, $T$ or $C$. To be specific, the alignments are realized by
\begin{equation}
    \label{equ:projection_layer}
    \bm{Z}_{u,*}^t = \bm{M}_{u,*}^t \bm{W}_* + \bm{b}_* \in \mathbb{R}^{l_u^t \times d}, \bm{Z}_{v,*}^t = \bm{M}_{v,*}^t \bm{W}_* + \bm{b}_* \in \mathbb{R}^{l_v^t \times d}.
\end{equation}
Then, we concatenate the aligned encodings of $u$ and $v$, and get $\bm{Z}_{u}^t = \bm{Z}_{u,N}^t \| \bm{Z}_{u,E}^t \| \bm{Z}_{u,T}^t \| \bm{Z}_{u,C}^t \in \mathbb{R}^{l_u^t \times 4d}$ and $\bm{Z}_{v}^t = \bm{Z}_{v,N}^t \| \bm{Z}_{v,E}^t \| \bm{Z}_{v,T}^t \| \bm{Z}_{v,C}^t \in \mathbb{R}^{l_v^t \times 4d}$. 

Next, we employ a Transformer encoder to capture the temporal dependencies, which is built by stacking $L$ Multi-head Self-Attention (MSA) and Feed-Forward Network (FFN) blocks. The residual connection \cite{DBLP:conf/cvpr/HeZRS16} is employed after every block. We follow \cite{DBLP:conf/iclr/DosovitskiyB0WZ21} by using GELU \cite{hendrycks2016gaussian} instead of ReLU \cite{DBLP:conf/icml/NairH10} between the two-layer perception in each FFN block and applying Layer Normalization (LN) \cite{ba2016layer} before each block rather than after.
% , which are slightly different from \cite{DBLP:conf/nips/VaswaniSPUJGKP17}. 
Instead of individually processing $\bm{Z}_{u}^t$ and $\bm{Z}_{v}^t$, our Transformer encoder takes the stacked $\bm{Z}^t=\left[\bm{Z}_{u}^t; \bm{Z}_{v}^t\right] \in \mathbb{R}^{(l_u^t + l_v^t) \times 4d}$ as inputs, aiming to learn the temporal dependencies within and across the sequences of $u$ and $v$. The calculation process is
% \begin{gather*} 
\begin{gather} 
\label{equ:transformer}
   \text{MSA}\left(\bm{Q}, \bm{K}, \bm{V}\right)= \text{Softmax}\left(\frac{\bm{Q} \bm{K}^\top}{\sqrt{d_k}}\right) \bm{V},\\
   \text{FFN}\left(\bm{O}, \bm{W}_1, \bm{b}_1, \bm{W}_2, \bm{b}_2\right)= \text{GELU}\left(\bm{O}\bm{W}_1+\bm{b}_1\right)\bm{W}_2 + \bm{b}_2,\\
   \bm{O}_i^{t,l} = \text{MSA}\left( \text{LN}(\bm{Z}^{t,l-1}) \bm{W}_{Q,i}^l, \text{LN}(\bm{Z}^{t,l-1}) \bm{W}_{K,i}^l, \text{LN}(\bm{Z}^{t,l-1}) \bm{W}_{V,i}^l \right),\\
   % \bm{O}^{t,l} = \text{CONCAT}\left(\bm{O}_1^{t,l},\cdots,\bm{O}_I^{t,l}\right)\bm{W}_O^l + \bm{Z}^{t,l-1},\\
   % \bm{O}^{t,l} = \left[\bm{O}_1^{t,l};\cdots;\bm{O}_I^{t,l}\right]\bm{W}_O^l + \bm{Z}^{t,l-1},\\
   \bm{O}^{t,l} = \left(\bm{O}_1^{t,l} \| \cdots \| \bm{O}_I^{t,l}\right)\bm{W}_O^l + \bm{Z}^{t,l-1},\\
   % \bm{O}^{t,l} = \|_{i = 1}^I \bm{O}_i^{t,l} \bm{W}_O^l + \bm{Z}^{t,l-1},\\
   \bm{Z}^{t,l} = \text{FFN}\left(\text{LN}\left(\bm{O}^{t,l}\right), \bm{W}_1^l, \bm{b}_1^l, \bm{W}_2^l, \bm{b}_2^l\right) + \bm{O}^{t,l}.
\end{gather}
% \end{gather*}
$\bm{W}_{Q,i}^l \in \mathbb{R}^{4d \times d_k}$, $\bm{W}_{K,i}^l \in \mathbb{R}^{4d \times d_k}$, $\bm{W}_{V,i}^l \in \mathbb{R}^{4d \times d_v}$, $\bm{W}_O^l \in \mathbb{R}^{I \cdot d_v \times 4d}$, $\bm{W}_1^l \in \mathbb{R}^{4d \times 16d}$, $\bm{b}_1^l \in \mathbb{R}^{16d}$, $\bm{W}_2^l \in \mathbb{R}^{16d \times 4d}$ and $\bm{b}_2^l \in \mathbb{R}^{4d}$ are trainable parameters at the $l$-th layer. We set $d_k=d_v=4d/I$ with $I$ as the number of attention heads. The input of the first layer is $\bm{Z}^{t,0} = \bm{Z}^t \in \mathbb{R}^{(l_u^t + l_v^t) \times 4d}$, and the output of the $L$-th layer is denoted by $\bm{H}^t=\bm{Z}^{t,L} \in \mathbb{R}^{(l_u^t + l_v^t) \times 4d}$.

\textbf{Time-aware Node Representation}. The time-aware representations of node $u$ and $v$ at timestamp $t$ are derived by averaging their related representations in $\bm{H}^t$ with an output layer,
\begin{equation}
\label{equ:final_temporal_representation}
\begin{split}
   \bm{h}_u^{t} & = \text{MEAN}\left(\bm{H}^t[:l_u^t,:]\right) \bm{W}_{out} + \bm{b}_{out} \in \mathbb{R}^{d_{out}},\\
   \bm{h}_v^{t} & = \text{MEAN}\left(\bm{H}^t[l_u^t:l_u^t + l_v^t,:]\right) \bm{W}_{out} + \bm{b}_{out} \in \mathbb{R}^{d_{out}},\\
\end{split}
\end{equation}
where $\bm{W}_{out} \in \mathbb{R}^{4d \times d_{out}}$ and $\bm{b}_{out} \in \mathbb{R}^{d_{out}}$ are trainable weights with $d_{out}$ as the output dimension.

\subsection{DyGLib: Unified Library for Continuous-Time Dynamic Graph Learning}
\begin{figure}[!ht]
    \centering
    \includegraphics[scale=0.5]{figures/DyGLib_procedure.jpg}
    \caption{DyGLib is equipped with \textcolor[RGB]{0,176,80}{standard training pipelines}, \textcolor[RGB]{255,0,0}{extensible coding interfaces}, and \textcolor[RGB]{46,117,182}{comprehensive evaluating protocols}. DyG denotes the abbreviation of Dynamic Graph.}
    \label{fig:procedure_DyGLib}
\end{figure}
We introduce a unified library with standard training pipelines, extensible coding interfaces, and comprehensive evaluating strategies to facilitate reproducible, scalable, and credible continuous-time dynamic graph learning research. The overall procedure of DyGLib is shown in \figref{fig:procedure_DyGLib}.


\textbf{Standard Training Pipelines}. To eliminate the influence of different training pipelines in previous works, we unify the data format, create a customized data loader and train all the methods with the same model trainers. Our standard training pipelines guarantee reproducible performance and enable users to quickly identify the key components of different models. Researchers only need to focus on designing the model architecture without considering other irrelevant implementation details.

\textbf{Extensible Coding Interfaces}. We provide extensible coding interfaces for the datasets and algorithms, which are all implemented by PyTorch. These scalable designs enable users to incorporate new datasets and popular models based on specific requirements, which can significantly reduce the usage difficulty for beginners and allow experts to conveniently validate new ideas. Currently, DyGLib has integrated thirteen datasets from various domains and nine continuous-time dynamic graph learning methods. Note that we also find some issues in the implementations of previous studies and have fixed them in DyGLib (see details in \secref{section-appendix-issues-existing-methods}).

\textbf{Comprehensive Evaluating Protocols}. DyGLib supports the commonly used downstream tasks for dynamic graph learning, including transductive/inductive dynamic link prediction and dynamic node classification tasks. Previous methods are mainly evaluated on the dynamic link prediction task with the random negative sampling strategy. However, many models can achieve saturation performance under such a strategy, making it hard to distinguish more advanced designs. For more reliable comparisons, we adopt three strategies (i.e., random, historical, and inductive negative sampling strategies) in \cite{poursafaei2022towards} to comprehensively evaluate the model performance on the dynamic link prediction task. The dynamic node classification task is also included for providing additional results.

% $\mathcal{S}_u=\left\{\left(u^\prime, t^\prime\right) | \left(\left(u,u^\prime,t^\prime\right) \land t^\prime < t \right) \lor \left(\left(u^\prime,u,t^\prime\right) \land t^\prime < t \right)  \right\}$ and $\mathcal{S}_v=\left\{\left(v^\prime, t^\prime\right) | \left(\left(v,v^\prime,t^\prime\right) \land t^\prime < t \right) \lor \left(\left(v^\prime,v,t^\prime\right) \land t^\prime < t \right)  \right\}$. 

% $\mathcal{S}_u=\left\{\left(u^\prime, t^\prime\right) | \left(u^\prime,v^\prime,t^\prime\right) \land t^\prime < t \land \left(u^\prime == u or v^\prime == u\right) \right\}$

% $\mathcal{S}_u^t=\left\{\left(u,u^\prime,t^\prime\right) \lor \left(u^\prime,u,t^\prime\right) | t^\prime < t \right\}$ and $\mathcal{S}_v^t=\left\{\left(v,v^\prime,t^\prime\right) \lor \left(v^\prime,v,t^\prime\right) | t^\prime < t \right\}$.

% \begin{equation}
% \label{equ:transformer}
% \begin{split}
%    MSA\left(\bm{Q}, \bm{K}, \bm{V}\right)= Softmax\left(\frac{\bm{Q} \bm{K}^\top}{\sqrt{d_k}}\right) \bm{V},\\
%    FFN\left(\bm{O}\right)= GELU\left(\bm{O}\bm{W}_1+\bm{b}_1\right)\bm{W}_2 + \bm{b}_2,\\
%    \bm{O}_i^{t,l} = MSA\left(LN(\bm{Z}^{t,l-1}) \bm{W}_{Q,i}^l, LN(\bm{Z}^{t,l-1}) \bm{W}_{K,i}^l, LN(\bm{Z}^{t,l-1}) \bm{W}_{V,i}^l \right),\\
%    \bm{O}^{t,l} = Concat\left(\bm{O}_1^{t,l},\cdots,\bm{O}_I^{t,l}\right)\bm{W}_O^l + \bm{Z}^{t,l-1},\\
%    \bm{H}^{t,l} = FFN\left(LN\left(\bm{O}^{t,l}\right)\right) + \bm{O}^{t,l}.
% \end{split}
% \end{equation}

% \begin{equation}
%     MSA\left(\bm{Q}, \bm{K}, \bm{V}\right)= Softmax\left(\frac{\bm{Q} \bm{K}^\top}{\sqrt{d_k}}\right) \bm{V},
% \end{equation}
% \begin{equation}
%     FFN\left(\bm{O}\right)= GELU\left(\bm{O}\bm{W}_1+\bm{b}_1\right)\bm{W}_2 + \bm{b}_2,
% \end{equation}
% \begin{equation}
%     \bm{O}_i^{t,l} = MSA\left(LN(\bm{Z}^{t,l-1}) \bm{W}_{Q,i}^l, LN(\bm{Z}^{t,l-1}) \bm{W}_{K,i}^l, LN(\bm{Z}^{t,l-1}) \bm{W}_{V,i}^l \right),
% \end{equation}
% \begin{equation}
%     \bm{O}^{t,l} = Concat\left(\bm{O}_1^{t,l},\cdots,\bm{O}_I^{t,l}\right)\bm{W}_O^l + \bm{Z}^{t,l-1},
% \end{equation}
% \begin{equation}
%     \bm{H}^{t,l} = FFN\left(LN\left(\bm{O}^{t,l}\right)\right) + \bm{O}^{t,l}.
% \end{equation}


\section{Experiments}
\label{section-5}
In this section, extensive experiments are conducted. We report the results of various models using DyGLib, which can be directly referenced in the follow-up research. We also demonstrate the superiority of DyGFormer over existing methods and give an in-depth analysis of the neighbor co-occurrence encoding scheme and the patching technique.

\subsection{Experimental Settings}
\textbf{Datasets and Baselines}. We experiment with thirteen datasets (Wikipedia, Reddit, MOOC, LastFM, Enron, Social Evo., UCI, Flights, Can. Parl., US Legis., UN Trade, UN Vote, and Contact), which are collected by \cite{poursafaei2022towards} and cover diverse domains. Details of the datasets are shown in \secref{section-appendix-descriptions_datasets}. We compare DyGFormer with eight popular continuous-time dynamic graph learning baselines that are based on graph convolutions, memory networks, random walks, and sequential models, including JODIE \cite{DBLP:conf/kdd/KumarZL19}, DyRep \cite{DBLP:conf/iclr/TrivediFBZ19}, TGAT \cite{DBLP:conf/iclr/XuRKKA20}, TGN \cite{DBLP:journals/corr/abs-2006-10637}, CAWN \cite{DBLP:conf/iclr/WangCLL021}, EdgeBank \cite{poursafaei2022towards}, TCL \cite{DBLP:journals/corr/abs-2105-07944}, and GraphMixer \cite{cong2023do}. We give the descriptions of baselines in \secref{section-appendix-descriptions_baselines}.

\textbf{Evaluation Tasks and Metrics}.
We closely follow \cite{DBLP:conf/iclr/XuRKKA20,DBLP:journals/corr/abs-2006-10637,DBLP:conf/iclr/WangCLL021,poursafaei2022towards} by evaluating the model performance for dynamic link prediction, which predicts the probability of a link occurring between two given nodes at a specific time. This task has two settings: the transductive setting aims to predict future links between nodes that are observed during training, while the inductive setting predicts the future links between unseen nodes. We utilize a multi-layer perceptron to take the concatenated representations of two nodes as inputs and return the probability of a link as the output. Average Precision (AP) and Area Under the Receiver Operating Characteristic Curve (AUC-ROC) are adopted as the evaluation metrics. To provide more comprehensive comparisons, we adopt three negative sampling strategies in \cite{poursafaei2022towards} for evaluating dynamic link prediction, including random (rnd), historical (hist), and inductive (ind) negative sampling strategies, where the latter two strategies are more challenging. Please refer to \cite{poursafaei2022towards} for more details. We also follow \cite{DBLP:conf/iclr/XuRKKA20,DBLP:journals/corr/abs-2006-10637} to conduct the dynamic node classification task, which estimates the state of a node in a given interaction at a specific time. We employ another multi-layer perceptron to map the node representations to the labels. We use AUC-ROC as the evaluation metric due to the label imbalance. For all the tasks, we chronologically split each dataset with the ratio of 70\%, 15\%, and 15\% for training, validation, and testing. 

\textbf{Model Configurations}. For baselines, in addition to following their official settings, we also perform an exhaustive grid search to find the optimal configurations of some critical hyperparameters for more reliable comparisons. As DyGFormer can access longer histories, we vary each node's input sequence length from 32 to 4096 by a factor of 2. To keep the computational complexity at a constant level that is irrelevant to the input length, we correspondingly increase the patch size from 1 to 128. Please see \secref{section-appendix-configurations} for the detailed configurations of different models.

\textbf{Implementation Details}. Adam \cite{DBLP:journals/corr/KingmaB14} is adopted as the optimizer. We train all the models for 100 epochs and use the early stopping strategy with patience of 20. We select the model that achieves the best performance on the validation set for testing. We set the learning rate and batch size to 0.0001 and 200 for all the methods on all the datasets. We run the methods five times with seeds from 0 to 4 and report the average performance to eliminate deviations. The experiments are conducted on an Ubuntu machine equipped with one Intel(R) Xeon(R) Gold 6130 CPU @ 2.10GHz with 16 physical cores. The GPU device is NVIDIA Tesla T4 with 15 GB memory. 

\subsection{Performance Comparisons and Discussions}
Due to space limitations, we report the performance of different methods on the AP metric for transductive dynamic link prediction with three negative sampling strategies in \tabref{tab:average_precision_transductive_dynamic_link_prediction}. The best and second-best results are emphasized by \textbf{bold} and \underline{underlined} fonts. Note that the results are multiplied by 100 for a better display layout. Please refer to \secref{section-appendix-numerical-performance-dynamic-link-prediction} for the results of AP for inductive dynamic link prediction as well as AUC-ROC for transductive and inductive link prediction tasks. Note that EdgeBank can be only evaluated for transductive dynamic link prediction, so its results under the inductive setting are not presented. From \tabref{tab:average_precision_transductive_dynamic_link_prediction} and \secref{section-appendix-numerical-performance-dynamic-link-prediction}, we have two main observations. 

\begin{table}[!htbp]
\centering
\caption{AP for transductive dynamic link prediction with random, history, and inductive negative sampling strategies. NSS is the abbreviation of Negative Sampling Strategies.}
\label{tab:average_precision_transductive_dynamic_link_prediction}
\resizebox{1.01\textwidth}{!}
{
\setlength{\tabcolsep}{1.05mm}
{
\begin{tabular}{c|c|ccccccccc}
\hline
NSS                    & Datasets    & JODIE        & DyRep        & TGAT         & TGN          & CAWN         & EdgeBank     & TCL          & GraphMixer   & DyGFormer    \\ \hline
\multirow{14}{*}{rnd}  & Wikipedia   & 96.50 ± 0.14 & 94.86 ± 0.06 & 96.94 ± 0.06 & 98.45 ± 0.06 & \underline{98.76 ± 0.03} & 90.37 ± 0.00 & 96.47 ± 0.16 & 97.25 ± 0.03 & \textbf{99.03 ± 0.02} \\
                       & Reddit      & 98.31 ± 0.14 & 98.22 ± 0.04 & 98.52 ± 0.02 & 98.63 ± 0.06 & \underline{99.11 ± 0.01} & 94.86 ± 0.00 & 97.53 ± 0.02 & 97.31 ± 0.01 & \textbf{99.22 ± 0.01} \\
                       & MOOC        & 80.23 ± 2.44 & 81.97 ± 0.49 & 85.84 ± 0.15 & \textbf{89.15 ± 1.60} & 80.15 ± 0.25 & 57.97 ± 0.00 & 82.38 ± 0.24 & 82.78 ± 0.15 & \underline{87.52 ± 0.49} \\
                       & LastFM      & 70.85 ± 2.13 & 71.92 ± 2.21 & 73.42 ± 0.21 & 77.07 ± 3.97 & \underline{86.99 ± 0.06} & 79.29 ± 0.00 & 67.27 ± 2.16 & 75.61 ± 0.24 & \textbf{93.00 ± 0.12} \\
                       & Enron       & 84.77 ± 0.30 & 82.38 ± 3.36 & 71.12 ± 0.97 & 86.53 ± 1.11 & \underline{89.56 ± 0.09} & 83.53 ± 0.00 & 79.70 ± 0.71 & 82.25 ± 0.16 & \textbf{92.47 ± 0.12} \\
                       & Social Evo. & 89.89 ± 0.55 & 88.87 ± 0.30 & 93.16 ± 0.17 & \underline{93.57 ± 0.17} & 84.96 ± 0.09 & 74.95 ± 0.00 & 93.13 ± 0.16 & 93.37 ± 0.07 & \textbf{94.73 ± 0.01} \\
                       & UCI         & 89.43 ± 1.09 & 65.14 ± 2.30 & 79.63 ± 0.70 & 92.34 ± 1.04 & \underline{95.18 ± 0.06} & 76.20 ± 0.00 & 89.57 ± 1.63 & 93.25 ± 0.57 & \textbf{95.79 ± 0.17} \\
                       & Flights     & 95.60 ± 1.73 & 95.29 ± 0.72 & 94.03 ± 0.18 & 97.95 ± 0.14 & \underline{98.51 ± 0.01} & 89.35 ± 0.00 & 91.23 ± 0.02 & 90.99 ± 0.05 & \textbf{98.91 ± 0.01} \\
                       & Can. Parl.  & 69.26 ± 0.31 & 66.54 ± 2.76 & 70.73 ± 0.72 & 70.88 ± 2.34 & 69.82 ± 2.34 & 64.55 ± 0.00 & 68.67 ± 2.67 & \underline{77.04 ± 0.46} & \textbf{97.36 ± 0.45} \\
                       & US Legis.   & 75.05 ± 1.52 & \underline{75.34 ± 0.39} & 68.52 ± 3.16 & \textbf{75.99 ± 0.58} & 70.58 ± 0.48 & 58.39 ± 0.00 & 69.59 ± 0.48 & 70.74 ± 1.02 & 71.11 ± 0.59 \\
                       & UN Trade    & 64.94 ± 0.31 & 63.21 ± 0.93 & 61.47 ± 0.18 & 65.03 ± 1.37 & \underline{65.39 ± 0.12} & 60.41 ± 0.00 & 62.21 ± 0.03 & 62.61 ± 0.27 & \textbf{66.46 ± 1.29} \\
                       & UN Vote     & \underline{63.91 ± 0.81} & 62.81 ± 0.80 & 52.21 ± 0.98 & \textbf{65.72 ± 2.17} & 52.84 ± 0.10 & 58.49 ± 0.00 & 51.90 ± 0.30 & 52.11 ± 0.16 & 55.55 ± 0.42 \\
                       & Contact     & 95.31 ± 1.33 & 95.98 ± 0.15 & 96.28 ± 0.09 & \underline{96.89 ± 0.56} & 90.26 ± 0.28 & 92.58 ± 0.00 & 92.44 ± 0.12 & 91.92 ± 0.03 & \textbf{98.29 ± 0.01} \\ \cline{2-11} 
                       & Avg. Rank   & 5.08         & 5.85         & 5.69         & \underline{2.54}         & 4.31         & 7.54         & 6.92         & 5.46         & \textbf{1.62}         \\ \hline
\multirow{14}{*}{hist} & Wikipedia   & 83.01 ± 0.66 & 79.93 ± 0.56 & 87.38 ± 0.22 & 86.86 ± 0.33 & 71.21 ± 1.67 & 73.35 ± 0.00 & \underline{89.05 ± 0.39} & \textbf{90.90 ± 0.10} & 82.23 ± 2.54 \\
                       & Reddit      & 80.03 ± 0.36 & 79.83 ± 0.31 & 79.55 ± 0.20 & \underline{81.22 ± 0.61} & 80.82 ± 0.45 & 73.59 ± 0.00 & 77.14 ± 0.16 & 78.44 ± 0.18 & \textbf{81.57 ± 0.67} \\
                       & MOOC        & 78.94 ± 1.25 & 75.60 ± 1.12 & 82.19 ± 0.62 & \textbf{87.06 ± 1.93} & 74.05 ± 0.95 & 60.71 ± 0.00 & 77.06 ± 0.41 & 77.77 ± 0.92 & \underline{85.85 ± 0.66} \\
                       & LastFM      & 74.35 ± 3.81 & 74.92 ± 2.46 & 71.59 ± 0.24 & \underline{76.87 ± 4.64} & 69.86 ± 0.43 & 73.03 ± 0.00 & 59.30 ± 2.31 & 72.47 ± 0.49 & \textbf{81.57 ± 0.48} \\
                       & Enron       & 69.85 ± 2.70 & 71.19 ± 2.76 & 64.07 ± 1.05 & 73.91 ± 1.76 & 64.73 ± 0.36 & \underline{76.53 ± 0.00} & 70.66 ± 0.39 & \textbf{77.98 ± 0.92} & 75.63 ± 0.73 \\
                       & Social Evo. & 87.44 ± 6.78 & 93.29 ± 0.43 & \underline{95.01 ± 0.44} & 94.45 ± 0.56 & 85.53 ± 0.38 & 80.57 ± 0.00 & 94.74 ± 0.31 & 94.93 ± 0.31 & \textbf{97.38 ± 0.14} \\
                       & UCI         & 75.24 ± 5.80 & 55.10 ± 3.14 & 68.27 ± 1.37 & 80.43 ± 2.12 & 65.30 ± 0.43 & 65.50 ± 0.00 & 80.25 ± 2.74 & \textbf{84.11 ± 1.35} & \underline{82.17 ± 0.82} \\
                       & Flights     & 66.48 ± 2.59 & 67.61 ± 0.99 & \textbf{72.38 ± 0.18} & 66.70 ± 1.64 & 64.72 ± 0.97 & 70.53 ± 0.00 & 70.68 ± 0.24 & \underline{71.47 ± 0.26} & 66.59 ± 0.49 \\
                       & Can. Parl.  & 51.79 ± 0.63 & 63.31 ± 1.23 & 67.13 ± 0.84 & 68.42 ± 3.07 & 66.53 ± 2.77 & 63.84 ± 0.00 & 65.93 ± 3.00 & \underline{74.34 ± 0.87} & \textbf{97.00 ± 0.31} \\
                       & US Legis.   & 51.71 ± 5.76 & \textbf{86.88 ± 2.25} & 62.14 ± 6.60 & 74.00 ± 7.57 & 68.82 ± 8.23 & 63.22 ± 0.00 & 80.53 ± 3.95 & 81.65 ± 1.02 & \underline{85.30 ± 3.88} \\
                       & UN Trade    & 61.39 ± 1.83 & 59.19 ± 1.07 & 55.74 ± 0.91 & 58.44 ± 5.51 & 55.71 ± 0.38 & \textbf{81.32 ± 0.00} & 55.90 ± 1.17 & 57.05 ± 1.22 & \underline{64.41 ± 1.40} \\
                       & UN Vote     & \underline{70.02 ± 0.81} & 69.30 ± 1.12 & 52.96 ± 2.14 & 69.37 ± 3.93 & 51.26 ± 0.04 & \textbf{84.89 ± 0.00} & 52.30 ± 2.35 & 51.20 ± 1.60 & 60.84 ± 1.58 \\
                       & Contact     & 95.31 ± 2.13 & \underline{96.39 ± 0.20} & 96.05 ± 0.52 & 93.05 ± 2.35 & 84.16 ± 0.49 & 88.81 ± 0.00 & 93.86 ± 0.21 & 93.36 ± 0.41 & \textbf{97.57 ± 0.06} \\ \cline{2-11} 
                       & Avg. Rank   & 5.46         & 5.08         & 5.08         & \underline{3.85}         & 7.54         & 5.92         & 5.46         & 4.00         & \textbf{2.62}         \\ \hline
\multirow{14}{*}{ind}  & Wikipedia   & 75.65 ± 0.79 & 70.21 ± 1.58 & \underline{87.00 ± 0.16} & 85.62 ± 0.44 & 74.06 ± 2.62 & 80.63 ± 0.00 & 86.76 ± 0.72 & \textbf{88.59 ± 0.17} & 78.29 ± 5.38 \\
                       & Reddit      & 86.98 ± 0.16 & 86.30 ± 0.26 & 89.59 ± 0.24 & 88.10 ± 0.24 & \textbf{91.67 ± 0.24} & 85.48 ± 0.00 & 87.45 ± 0.29 & 85.26 ± 0.11 & \underline{91.11 ± 0.40} \\
                       & MOOC        & 65.23 ± 2.19 & 61.66 ± 0.95 & 75.95 ± 0.64 & \underline{77.50 ± 2.91} & 73.51 ± 0.94 & 49.43 ± 0.00 & 74.65 ± 0.54 & 74.27 ± 0.92 & \textbf{81.24 ± 0.69} \\
                       & LastFM      & 62.67 ± 4.49 & 64.41 ± 2.70 & 71.13 ± 0.17 & 65.95 ± 5.98 & 67.48 ± 0.77 & \textbf{75.49 ± 0.00} & 58.21 ± 0.89 & 68.12 ± 0.33 & \underline{73.97 ± 0.50} \\
                       & Enron       & 68.96 ± 0.98 & 67.79 ± 1.53 & 63.94 ± 1.36 & 70.89 ± 2.72 & \underline{75.15 ± 0.58} & 73.89 ± 0.00 & 71.29 ± 0.32 & 75.01 ± 0.79 & \textbf{77.41 ± 0.89} \\
                       & Social Evo. & 89.82 ± 4.11 & 93.28 ± 0.48 & 94.84 ± 0.44 & \underline{95.13 ± 0.56} & 88.32 ± 0.27 & 83.69 ± 0.00 & 94.90 ± 0.36 & 94.72 ± 0.33 & \textbf{97.68 ± 0.10} \\
                       & UCI         & 65.99 ± 1.40 & 54.79 ± 1.76 & 68.67 ± 0.84 & 70.94 ± 0.71 & 64.61 ± 0.48 & 57.43 ± 0.00 & \underline{76.01 ± 1.11} & \textbf{80.10 ± 0.51} & 72.25 ± 1.71 \\
                       & Flights     & 69.07 ± 4.02 & 70.57 ± 1.82 & \underline{75.48 ± 0.26} & 71.09 ± 2.72 & 69.18 ± 1.52 & \textbf{81.08 ± 0.00} & 74.62 ± 0.18 & 74.87 ± 0.21 & 70.92 ± 1.78 \\
                       & Can. Parl.  & 48.42 ± 0.66 & 58.61 ± 0.86 & 68.82 ± 1.21 & 65.34 ± 2.87 & 67.75 ± 1.00 & 62.16 ± 0.00 & 65.85 ± 1.75 & \underline{69.48 ± 0.63} & \textbf{95.44 ± 0.57} \\
                       & US Legis.   & 50.27 ± 5.13 & \textbf{83.44 ± 1.16} & 61.91 ± 5.82 & 67.57 ± 6.47 & 65.81 ± 8.52 & 64.74 ± 0.00 & 78.15 ± 3.34 & 79.63 ± 0.84 & \underline{81.25 ± 3.62} \\
                       & UN Trade    & 60.42 ± 1.48 & 60.19 ± 1.24 & 60.61 ± 1.24 & 61.04 ± 6.01 & \underline{62.54 ± 0.67} & \textbf{72.97 ± 0.00} & 61.06 ± 1.74 & 60.15 ± 1.29 & 55.79 ± 1.02 \\
                       & UN Vote     & \textbf{67.79 ± 1.46} & 67.53 ± 1.98 & 52.89 ± 1.61 & \underline{67.63 ± 2.67} & 52.19 ± 0.34 & 66.30 ± 0.00 & 50.62 ± 0.82 & 51.60 ± 0.73 & 51.91 ± 0.84 \\
                       & Contact     & 93.43 ± 1.78 & 94.18 ± 0.10 & \underline{94.35 ± 0.48} & 90.18 ± 3.28 & 89.31 ± 0.27 & 85.20 ± 0.00 & 91.35 ± 0.21 & 90.87 ± 0.35 & \textbf{94.75 ± 0.28} \\ \cline{2-11} 
                       & Avg. Rank   & 6.62         & 6.38         & \underline{4.15}         & 4.38         & 5.46         & 5.62         & 4.69         & 4.46         & \textbf{3.23}         \\ \hline
\end{tabular}
}
}
\end{table}

(\romannumeral1) DyGFormer outperforms existing methods in most cases and achieves an average rank of 2.49/2.69 on AP/AUC-ROC for transductive dynamic link prediction and 2.69/2.56 on AP/AUC-ROC for inductive dynamic link prediction across the three negative sampling strategies on thirteen datasets. We summarize the superiority of DyGFormer in two aspects. Firstly, DyGFormer presents the neighbor co-occurrence encoding scheme to exploit the correlations of the source node and destination node, which are often informative in predicting their future linked probability. Secondly, the patching technique allows DyGFormer to effectively leverage longer histories and capture long-term temporal dependencies. As shown in Table \tabref{tab:num_neighbors_configuration}, the input sequence lengths of DyGFormer are 256+ on half of the datasets, which are significantly longer than those of the baselines, demonstrating that DyGFormer can benefit from longer sequences. 
% From \tabref{tab:num_neighbors_configuration}, we observe the input sequence lengths of DyGFormer are 256+ on half of the datasets, which are much longer than those of baselines, demonstrating DyGFormer can benefit from longer sequences. 
% We will give a deeper analysis of the above two designs in \secref{section-5-node_cooccurrence_generalizability} and \secref{section-5-investigation_patch_technique}. 

\begin{wraptable}{R}{8.0cm}
\vspace{-10pt}
\caption{AUC-ROC for dynamic node classification.}
\label{tab:auc_roc_dynamic_node_classification}
\setlength{\tabcolsep}{1.1mm}
{
\begin{tabular}{c|cc|c}
\hline
Methods    & Wikipedia                & Reddit                   & \begin{tabular}[c]{@{}c@{}}Avg.\\ Rank\end{tabular} \\ \hline
JODIE      & \textbf{0.8899 ± 0.0105} & 0.6037 ± 0.0258          & 4.50                                                \\
DyRep      & 0.8639 ± 0.0098          & 0.6372 ± 0.0132          & 5.00                                                \\
TGAT       & 0.8409 ± 0.0127          & \textbf{0.7004 ± 0.0109} & \underline{4.00}                                                \\
TGN        & 0.8638 ± 0.0234          & 0.6327 ± 0.0090          & 6.00                                                \\
CAWN       & 0.8488 ± 0.0133          & 0.6634 ± 0.0178          & 5.00                                                \\
TCL        & 0.7783 ± 0.0213          & {\underline{0.6887 ± 0.0215}}    & 5.00                                                \\
GraphMixer & 0.8680 ± 0.0079          & 0.6422 ± 0.0332          & \underline{4.00}                                                \\
DyGFormer  & {\underline{0.8744 ± 0.0108}}    & 0.6800 ± 0.0174          & \textbf{2.50}                                                \\ \hline
\end{tabular}
}
\vspace{-10pt}
\end{wraptable}

(\romannumeral2) Some of our findings differ from previous reports. For instance, the performance of some baselines can be significantly improved by properly setting certain hyperparameters. Additionally, some methods would obtain worse results after we fix the problems or make adaptions in their implementations. More explanations can be found in \secref{section-appendix-inconsistent-observations}. These observations highlight the importance of rigorously evaluating different methods by a unified library and verify the necessity of introducing DyGLib to facilitate the development of dynamic graph learning.

In addition, we present the results of various methods for dynamic node classification on Wikipedia and Reddit (the only two datasets with dynamic labels) in \tabref{tab:auc_roc_dynamic_node_classification}. We observe that DyGFormer obtains better performance than most baselines and achieves an impressive average rank of 2.50 among them, which shows the superiority of DyGFormer once again. Due to space limitations, we only report the AP metric for transductive dynamic link prediction with the random sampling strategy in the subsequent sections. Similar trends could also be observed in other situations (e.g., AUC-ROC metric, inductive setting, historical/inductive negative sampling strategy).

\subsection{Ablation Study}\label{section-5-ablation_study}
\begin{figure}[!htbp]
    \centering
    \includegraphics[width=1.0\columnwidth]{figures/ablation_study.jpg}
    \caption{Ablation study of the components in DyGFormer.}
    \label{fig:ablation_study}
\end{figure}

We further validate the effectiveness of some designs in DyGFormer via an ablation study, including the usage of \textbf{N}eighbor \textbf{Co}-occurrence \textbf{E}ncoding (NCoE), the usage of \textbf{T}ime \textbf{E}ncoding (TE), and the \textbf{Mix}ing of the sequence of \textbf{S}ource node and \textbf{D}estination node (MixSD). We respectively remove these modules and denote the remaining parts as w/o NCoE, w/o TE, and w/o MixSD. We also \textbf{Sep}arately encode the \textbf{N}eighbor \textbf{O}ccurrence in the source node's or destination node's sequence and denote this variant as w/ SepNO. We report the performance of different variants on MOOC, Social Evo., UCI, and UN Trade datasets from four domains in \figref{fig:ablation_study}. 
We find that DyGFormer usually performs best when using all the components, and the results would be worse when any component is removed. The neighbor co-occurrence encoding scheme has the most significant impact on the performance as it effectively captures correlations between nodes. Separately encoding neighbor occurrences or encoding the temporal information could also improve performance. Mixing the sequences of the source node and destination node causes relatively minor improvements due to the usage of the neighbor co-occurrence encoding scheme because both of them aim to explore the node correlations.

\subsection{Generalizability of Neighbor Co-occurrence Encoding}\label{section-5-node_cooccurrence_generalizability}
Our neighbor co-occurrence encoding scheme is general and can be incorporated into sequential model-based dynamic graph learning methods, such as TCL and GraphMixer. Therefore, we integrate the neighbor co-occurrence encoding with the inputs of TCL and GraphMixer and show the performance in \tabref{tab:generalizability_neighbor_co_occurrence_encoding}. 
% Therefore, for TCL and GraphMixer, we integrate the neighbor co-occurrence encoding with their inputs via combination and addition, and report better performance of these two implementations.
We could find that TCL and GraphMixer usually yield better results when the neighbor co-occurrence encoding is employed, which indicates the effectiveness and versatility of the proposed encoding scheme, and also highlights the importance of capturing correlations between nodes. Additionally, since both TCL and DyGFormer are built upon the Transformer architecture, they achieve similar performance on datasets that enjoy shorter input sequences (e.g., Wikipedia, Reddit, Social Evo., UCI, Contact), where the patching technique in DyGFormer contributes little. However, when datasets exhibit long-term temporal dependencies, the performance gap between TCL and DyGFormer would become more significant.

\begin{table}[!htbp]
\centering
\caption{AP for different methods when equipped with the neighbor co-occurrence encoding.}
\label{tab:generalizability_neighbor_co_occurrence_encoding}
\resizebox{1.01\textwidth}{!}
{
\setlength{\tabcolsep}{1.05mm}
{
\begin{tabular}{c|ccccccc}
\hline
\multirow{2}{*}{Datasets} & \multicolumn{3}{c|}{TCL}                               & \multicolumn{3}{c|}{GraphMixer}                        & \multirow{2}{*}{DyGFormer} \\ \cline{2-7}
                          & Original & w/ NCoE & \multicolumn{1}{c|}{Improvements} & Original & w/ NCoE & \multicolumn{1}{c|}{Improvements} &                            \\ \hline
Wikipedia                 & 0.9647   & \textbf{0.9909}  & \multicolumn{1}{c|}{2.72\%}       & 0.9725   & 0.9790  & \multicolumn{1}{c|}{0.67\%}       & \underline{0.9903}                     \\
Reddit                    & 0.9753   & \underline{0.9904}  & \multicolumn{1}{c|}{1.55\%}       & 0.9731   & 0.9763  & \multicolumn{1}{c|}{0.33\%}       & \textbf{0.9922}                     \\
MOOC                      & 0.8238   & \underline{0.8692}  & \multicolumn{1}{c|}{5.51\%}       & 0.8278   & 0.8358  & \multicolumn{1}{c|}{0.97\%}       & \textbf{0.8752}                     \\
LastFM                    & 0.6727   & \underline{0.8402}  & \multicolumn{1}{c|}{24.90\%}      & 0.7561   & 0.7648  & \multicolumn{1}{c|}{1.15\%}       & \textbf{0.9300}                     \\
Enron                     & 0.7970   & \underline{0.9018}  & \multicolumn{1}{c|}{13.15\%}      & 0.8225   & 0.8883  & \multicolumn{1}{c|}{8.00\%}       & \textbf{0.9247}                     \\
Social Evo.               & 0.9313   & 0.9406  & \multicolumn{1}{c|}{1.00\%}       & 0.9337   & \underline{0.9437}  & \multicolumn{1}{c|}{1.07\%}       & \textbf{0.9473}                     \\
UCI                       & 0.8957   & \underline{0.9469}  & \multicolumn{1}{c|}{5.72\%}       & 0.9325   & 0.9348  & \multicolumn{1}{c|}{0.25\%}       & \textbf{0.9579}                     \\
Flights                   & 0.9123   & \underline{0.9771}  & \multicolumn{1}{c|}{7.10\%}       & 0.9099   & 0.9690  & \multicolumn{1}{c|}{6.50\%}       & \textbf{0.9891}                     \\
Can. Parl.                & 0.6867   & 0.6934  & \multicolumn{1}{c|}{0.98\%}       & \underline{0.7704}   & 0.7638  & \multicolumn{1}{c|}{-0.86\%}      & \textbf{0.9736}                     \\
US Legis.                 & 0.6959   & 0.6947  & \multicolumn{1}{c|}{-0.17\%}      & \underline{0.7074}   & 0.7026  & \multicolumn{1}{c|}{-0.68\%}      & \textbf{0.7111}                     \\
UN Trade                  & 0.6221   & \underline{0.6346}  & \multicolumn{1}{c|}{2.01\%}       & 0.6261   & 0.6277  & \multicolumn{1}{c|}{0.26\%}       & \textbf{0.6646}                    \\
UN Vote                   & 0.5190   & 0.5152  & \multicolumn{1}{c|}{-0.73\%}      & 0.5211   & \underline{0.5213}  & \multicolumn{1}{c|}{0.04\%}       & \textbf{0.5555}                     \\
Contact                   & 0.9244   & \underline{0.9798}  & \multicolumn{1}{c|}{5.99\%}       & 0.9192   & 0.9794  & \multicolumn{1}{c|}{6.55\%}       & \textbf{0.9829}                     \\ \hline
Avg. Rank                 & 4.62     & \underline{2.62}    & ---                               & 3.85     & 2.85    & ---                                & \textbf{1.08}                       \\ \hline
\end{tabular}
}
}
\end{table}
 

\subsection{Advantages of Patching Technique}\label{section-5-investigation_patch_technique}
We aim to validate the advantages of the patching technique in (\romannumeral1) preserving the local temporal proximities to help DyGFormer effectively benefit from longer histories and (\romannumeral2) efficiently reducing the computational complexity to a constant level that is agnostic to the input sequence length. We conduct experiments on LastFM and Can. Parl. since these two datasets tend to achieve improvements from longer historical records. For baselines, we sample more neighbors or perform more causal anonymous walks to make them access longer histories (starts from 32). The results are depicted in \figref{fig:baselines_varying_input_lengths}, where the x-axis is represented by a logarithmic scale with base 2. We also plot the performance of baselines with the optimal length by unconnected points according to \tabref{tab:num_neighbors_configuration}. Note that some results of baselines are not shown as they raise the out-of-memory error when the lengths are longer. For example, TGAT is only computationally feasible when extending the input length to 32, resulting in two discrete points with length 20 (the optimal length) and 32.
\begin{figure}[!htbp]
    \centering
    \includegraphics[width=0.98\columnwidth]{figures/baselines_varying_input_lengths.jpg}
    \caption{Performance of different methods with varying input lengths.}
    \label{fig:baselines_varying_input_lengths}
\end{figure}

From \figref{fig:baselines_varying_input_lengths}, we could conclude that: (\romannumeral1) most baselines (excluding CAWN) perform worse when the input lengths become longer, indicating they lack the ability in capturing long-term temporal dependencies; (\romannumeral2) the baselines usually encounter expensive computational costs when computing on longer histories. Although memory network-based methods (i.e., DyRep and TGN) can handle longer histories with affordable computational costs, they cannot benefit from longer histories due to the staleness or vanishing/exploding gradient issues; (\romannumeral3) DyGFormer consistently achieves gains from longer sequences, demonstrating the advantages of the patching technique in preserving local temporal proximities and enabling effective access to longer histories. 

We also compare the running time and memory usage during the training process of DyGFormer with and without the patching technique when using input sequences of the same length. The results are shown in \tabref{tab:comparison_training_time_memory_usage_with_without_patching}. We could observe that the patching technique efficiently reduces model training costs in both time and space, and allows DyGFormer to attend to longer histories. As the input sequence length increases, the reductions become more significant. We also find that when equipped with the patching technique, DyGFormer achieves an average improvement of 0.31\% and 0.74\% in performance on LastFM and Can. Parl., compared to DyGFormer without patching. This observation further demonstrates the advantage of the patching technique in leveraging the local temporal proximities for better results.

\begin{table}[!htbp]
\centering
\caption{Comparisons of running time and memory usage of DyGFormer with and without the patching technique, where OOM stands for Out-Of-Memory.}
\label{tab:comparison_training_time_memory_usage_with_without_patching}
% \resizebox{1.01\textwidth}{!}
% {
% \setlength{\tabcolsep}{1.05mm}
% {
\begin{tabular}{c|c|c|cc|c}
\hline
Datasets                    & \begin{tabular}[c]{@{}c@{}}Input \\ Lengths\end{tabular} & Metrics      & \begin{tabular}[c]{@{}c@{}}DyGFormer \\ w/ patching\end{tabular} & \begin{tabular}[c]{@{}c@{}}DyGFormer \\ w/o patching\end{tabular} & \begin{tabular}[c]{@{}c@{}}Reduced \\ Ratios\end{tabular} \\ \hline
\multirow{8}{*}{LastFM}     & \multirow{2}{*}{64}                                      & running time & 13min 58s                                                        & 21min 01s                                                         & 1.50                                                      \\
                            &                                                          & memory usage & 3,945 MB                                                         & 7,953 MB                                                          & 2.02                                                      \\ \cline{2-6} 
                            & \multirow{2}{*}{128}                                     & running time & 16min 54s                                                        & 45min 29s                                                         & 2.69                                                      \\
                            &                                                          & memory usage & 4,677 MB                                                         & 7,585 MB                                                          & 1.62                                                      \\ \cline{2-6} 
                            & \multirow{2}{*}{256}                                     & running time & 24min 41s                                                        & 2h 5min 50s                                                       & 5.10                                                      \\
                            &                                                          & memory usage & 4,635 MB                                                         & 14,583 MB                                                         & 3.15                                                      \\ \cline{2-6} 
                            & \multirow{2}{*}{512}                                     & running time & 37min 04s                                                        & ---                                                               & ---                                                       \\
                            &                                                          & memory usage & 7,547 MB                                                         & OOM                                                               & ---                                                       \\ \hline
\multirow{8}{*}{Can. Parl.} & \multirow{2}{*}{64}                                      & running time & 58s                                                              & 1min 14s                                                          & 1.28                                                      \\
                            &                                                          & memory usage & 2,121 MB                                                         & 3,263 MB                                                          & 1.54                                                      \\ \cline{2-6} 
                            & \multirow{2}{*}{128}                                     & running time & 1min 02s                                                         & 2min 39s                                                          & 2.56                                                      \\
                            &                                                          & memory usage & 2,369 MB                                                         & 6,417 MB                                                          & 2.71                                                      \\ \cline{2-6} 
                            & \multirow{2}{*}{256}                                     & running time & 1min 26s                                                         & 6min 37s                                                          & 4.62                                                      \\
                            &                                                          & memory usage & 2,855 MB                                                         & 14,923 MB                                                         & 5.23                                                      \\ \cline{2-6} 
                            & \multirow{2}{*}{512}                                     & running time & 1min 57s                                                         & ---                                                               & ---                                                       \\
                            &                                                          & memory usage & 4,511 MB                                                         & OOM                                                               & ---                                                       \\ \hline
\end{tabular}
% }
% }
\end{table}

\section{Conclusion}
\label{section-6}
In this paper, we proposed a new Transformer-based architecture (DyGFormer) and a unified library (DyGLib) to foster the development of dynamic graph learning. DyGFormer differs from previous methods in (\romannumeral1) a neighbor co-occurrence encoding scheme to exploit the correlations of nodes in each interaction; and (\romannumeral2) a patching technique to help the model capture long-term temporal dependencies. DyGLib served as a toolkit for reproducible, scalable, and credible continuous-time dynamic graph learning with standard training pipelines, extensible coding interfaces, and comprehensive evaluating protocols. We hope our work can provide new perspectives on designing new dynamic graph learning frameworks and encourage more researchers to dive into this field. In the future, we will continue to enrich DyGLib by incorporating the recently released datasets and state-of-the-art models.


% \bibliographystyle{plain}
% \bibliographystyle{plainnat}
% \bibliographystyle{unsrtnat}
\bibliographystyle{ACM-Reference-Format}
\bibliography{reference.bib}

\appendix
\section{Skew Equations}
We will justify and show the three equations used in Lemma \ref{skew rel} to narrow our search for these skew axial algebras. Although they do not provide much use to understanding how these algebras could be constructed, they do make the proof easier.

Suppose $v$ is an $\mu$-eigenvector of an axis, $x$, where $\mu\neq1$. Then the projection on that axis should be equal to 0; that is, $\lm_x(v)=0$. Coincidentally, nearly all of the eigenvectors in Lemma \ref{eigen a} and \ref{eigen b} satisfy that rule. However we have
\begin{equation*}
 0=\lm_b\left(-\frac{P}{\bt}a+Pb+c\right) = -\frac{P}{\bt}\lmf_1+P+\lmf_2.
\end{equation*}
Whence we get Equation (\ref{proof1}).

\begin{defn}
Let $x$ be a $\mon{\al,\bt}$-axis in $A$, $\lm\in \{1,0, \al, \bt\}$ and $v\in A$. We denote $[v]^x_\lm$ to be the component of $v$ in $ A_\lm(x)$. 
\end{defn}
\begin{lem}
Let $w:=\frac{1}{2}(b-c)$. We have $[a]^a_\bt=0$, $[b]^a_\bt=w$, $[c]^a_\bt=-w$, $[\sg]^a_\bt=0$. Further, $[ab]^a_\bt=\bt w$, $[ac]^a_\bt=-\bt w$, $[bc]^a_\bt=0$, $[a\sg]^a_\bt=0$, $[b\sg]^a_\bt=\dt^fw$, $[c\sg]^a_\bt=-\dt^fw$ and $[\sg^2]^a_\bt=0$.
\end{lem}
\proof
As $a\in A_1(a)$, it has no $\bt$-component in $A_\bt(a)$ and $[a]^a_\bt=0$. As $\sg\in A_{\{1,0,\al\}}(a)$, it has no $\bt$-component in $A_\bt(a)$ and $[\sg]^a_\bt=0$. We can express $b$ in terms of the eigenvectors of $\text{ad}_a$ in Lemma \ref{eigen a}. The reader can check
\[ b= \lm_1 a+ \frac{1}{\al}\left(\ep a+\frac{1}{2}(\al-\bt)(b+c)-\sg\right)+ \frac{1}{\al}\left(\gm a +\frac{1}{2}\bt(b+c)+\sg\right)+\frac{1}{2}(b-c).\]
Thus $[b]_\bt^a=w$. As $c=b^{\tu{a}}$, we get $[c]_\bt^a=-w$.

Let $x, y \in A_{\{0,1,\al\}}(a)$ and notice $x^2, xy\in A_{\{1,0,\al\}}(a)$ and so has no $\bt$-component in $A_\bt(a)$. Therefore $[\sg^2]^a_\bt=[a\sg]^a_\bt=0$. Also
\[ [bc]_\bt^a=P\left([a]_\bt^a+\frac{1}{\bt}[\sg]_\bt^a\right)=0.\]
Note that
\[ [ab]_\bt^a=[\sg]_\bt^a+\bt[a]_\bt^a+\bt[b]_\bt^a=\bt w\]
and 
\[ [b\sg]_\bt^a=(\al-\bt)[\sg]_\bt^a+\bt(\al-\bt)[a]_\bt^a+dt^f[b]_\bt^a=\dt^f w.\]
Applying $\tu{a}$, we get $[ac]_\bt^a$ and $[c\sg]_\bt^a$. \qed



Let $u:= (b -\al)a - \bt b=\sg -(\al-\bt)a$. As $A_\bt(b)=\{0\}$, we have that $u\in A_{\{1,0\}}(b)$. By Lemma \ref{Seress}, the following holds
\[b(au)=(ba)u.\]
Notice
\[ au = a(\sg -(\al-\bt)a)=(\dt -(\al-\bt))a+\frac{1}{2}\bt(\al-\bt)(b+c)+(\al-\bt)\sg\]
and so
\begin{eqnarray*}
[b(au)]_\bt^a &=& (\dt -(\al-\bt))[ab]_\bt^a+\frac{1}{2}\bt(\al-\bt)([b]_\bt^a+[bc]_\bt^a)+(\al-\bt)[b\sg]_\bt^a\\
& =& \left(\bt(\dt -(\al-\bt))+\frac{1}{2}\bt(\al-\bt)+(\al-\bt)\dt^f\right)w
\end{eqnarray*}
We also have 
\begin{eqnarray*}
[(ba)u]_\bt^a&=&[(\sg+\bt a +\bt b)(\sg -(\al-\bt)a)]_\bt^a\\
&=& [\sg^2]_\bt^a -(\al-2\bt)[a\sg]_\bt^a +\bt [b\sg]_\bt^a -\bt(\al-\bt)[a]_\bt^a - \bt(\al-\bt)[ab]_\bt^a\\
&=& (\bt\dt^f -\bt^2(\al-\bt)) w
\end{eqnarray*}
By Lemma \ref{Seress}, we have $0=(ba)u-b(au)$ moreover $0=[(ba)u]_\bt-[b(au)]_\bt$. Looking at the coefficient of $w$, we have
\begin{eqnarray*} 
0&=& (\bt\dt^f-\bt^2(\al-\bt))\\
& -& \left(\bt \dt -\bt(\al-\bt)+\frac{1}{2}\bt(\al-\bt)+(\al-\bt)\dt^f\right)\\
&=&-\bt^2(\al-\bt) -\bt\dt+\frac{1}{2}\bt(\al-\bt)-(\al-2\bt)\dt^f.
\end{eqnarray*}
Rearranging we get Equation (\ref{proof2}).

Let $v:=Pa+\frac{P}{\bt}\sg -\al c=c(b-\al)$. Notice that $v \in A_{\{1,0\}}(b)$. Again by Lemma \ref{Seress}, the following holds
\[b(av)=(ba)v.\]
We have
\begin{eqnarray*}
av &=& Pa +\frac{P}{\bt}\left(\dt a + \frac{1}{2}\bt(\al-\bt)(b+c) +(\al-\bt)\sg\right)\\
& -&\al(\bt a +\bt c +\sg)\\
&=&\left(P +\frac{P}{\bt}\dt -\al\bt\right)a+\left(\frac{1}{2}(\al-\bt)P\right)b\\
&+&\left(\frac{1}{2}(\al-\bt)P-\al\bt\right)c+\left(\frac{P}{\bt}(\al-\bt)-\al\right)\sg.
\end{eqnarray*}
Therefore
\begin{eqnarray*}
[b(av)]_\bt^a &=&\left(P +\frac{P}{\bt}\dt -\al\bt\right)[ab]_\bt^a+\left(\frac{1}{2}(\al-\bt)P\right)[b]_\bt^a\\
&+&\left(\frac{1}{2}(\al-\bt)P-\al\bt\right)[bc]_\bt^a+\left(\frac{P}{\bt}(\al-\bt)-\al\right)[b\sg]_\bt^a.\\
&=&\left(\bt \left(P +\frac{P}{\bt}\dt -\al\bt\right)+\dt^f\left(\frac{P}{\bt}(\al-\bt)-\al\right)\right)w
\end{eqnarray*}
We also have
\begin{eqnarray*}
[(ba)v]_\bt^a&=&\left[\left(\bt a +\bt b +\sg\right)\left(Pa+\frac{P}{\bt}\sg -\al c\right)\right]_\bt^a\\
&=&2P[a\sg]_\bt^a +\frac{P}{\bt}[\sg^2]_\bt^a -\al [c \sg]_\bt^a + \bt P [a]_\bt^a -\al\bt [ac]_\bt^a\\
&+&\bt P [ab]_\bt^a +P[b\sg]_\bt^a -\al\bt [bc]_\bt^a\\
&=&\left(\al \dt^f +\al\bt^2 +\bt^2 P  +P\dt^f\right)w
\end{eqnarray*}
By Lemma \ref{Seress}, $0=[b(av)]^a_\bt-[(ba)v]^a_\bt$ and looking at the coefficient of $w$, we get 
\begin{eqnarray*}
0&=&[b(av)]_\bt-[(ba)v]_\bt\\
&=&\left(\bt P +\dt P -\al\bt^2+\frac{1}{2}(\al-\bt)P+\frac{P}{\bt}(\al-\bt)\dt^f -\al\dt^f\right)\\
&-&\left(\bt^2P +P\dt^f+\al\dt^f +\al\bt^2 \right)\\ 
&=&\left(\frac{P}{\bt}\left[\bt^2 +\bt\dt+\frac{1}{2}\bt(\al-\bt)+(\al-2\bt)\dt^f-\bt^3\right]-2\al(\dt^f+\bt^2)\right).
\end{eqnarray*}
From Equation (\ref{proof2}), we get that
\begin{eqnarray*}
0&=&\frac{P}{\bt}\left[\bt^2 -\bt^2(\al-\bt) +\frac{1}{2}\bt(\al-\bt)-(\al-2\bt)\dt^f\right.\\
&+&\left.\frac{1}{2}\bt(\al-\bt)+(\al-2\bt)\dt^f-\bt^3\right]-2\al(\dt^f+\bt^2)\\
&=&\frac{P}{\bt}\left[\bt^2 -\bt^2(\al-\bt) +\bt(\al-\bt)-\bt^3\right]-2\al(\dt^f+\bt^2)\\
&=&\frac{P}{\bt}\al\bt\left[1-\bt\right]-2\al(\dt^f+\bt^2).
\end{eqnarray*}
Hence we get Equation (\ref{proof3}).

\section*{Acknowledgements}
I would like to thank Professor Sergey Shpectorov for his guidance throughout my PhD studies so far and pushing me to complete this paper. I would also like to thank my family for their continuing support. 

\end{document}