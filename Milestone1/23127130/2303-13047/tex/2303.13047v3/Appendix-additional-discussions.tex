\section{Additional Discussions and Details}
\subsection{Limitations}
One potential limitation of DyGFormer lies in the ignorance of high-order relationships between nodes since it solely learns from the first-hop interactions of nodes. In certain scenarios where nodes' high-order relationships are essential, DyGFormer may be suboptimal compared with baselines that learn the higher-order interactions. However, trivially feeding the multi-hop neighbors of nodes into DyGFormer would incur expensive computational costs. It is promising to design more efficient and effective frameworks to model nodes' high-order relationships for dynamic graph learning.

Another potential limitation is the sensitivity of the neighbor co-occurrence encoding scheme against different negative sampling strategies (discussed in \secref{section-varying-performance-DyGFormer-negative-sampling-strategies}). When the assumption of our neighbor co-occurrence encoding scheme is violated, its performance may drop drastically in some cases. It is an insightful direction to design more robust encoding schemes to tackle this issue.

\subsection{Licenses}
All the used codes and datasets are publicly available and permit usage for research purposes under either MIT License or Apache License 2.0.

\subsection{Overall Procedure of DyGLib}\label{section-appendix-DyGLib-procedure}
\figref{fig:procedure_DyGLib} shows the overall procedure of DyGLib.
\begin{figure}[!ht]
    \centering
    \includegraphics[scale=0.5]{figures/DyGLib_procedure.jpg}
    \caption{DyGLib is equipped with \textcolor[RGB]{0,176,80}{standard training pipelines}, \textcolor[RGB]{255,0,0}{extensible coding interfaces}, and \textcolor[RGB]{9,182,205}{comprehensive evaluating protocols}. DyG denotes the abbreviation of Dynamic Graph.}
    \label{fig:procedure_DyGLib}
\end{figure}
