\section{Depth Noises in SSC} 
\label{sec:noise}
In this section, we first compute the percentage of the two noise types: zero noise and delta noise, taking the standard datasets NYU~\cite{silberman2012indoor} and NYUCAD~\cite{firman2016NYUCAD} as examples.
%

Then, we show the performance gap between the learned models \emph{with} and \emph{without} these noises.
%
The aim is to quantitatively demonstrate that these two kinds of noises ignored by the existing work of SSC are unfortunately making severe negative effects on the learning of SSC models.
%
% -----------------------------

\noindent
\textbf{The Quantity of Depth Noises.}
In Figure~\ref{fig:noise} (b) and Figure~\ref{fig:noise} (a), we show the class-wise quantities of zero noises and delta noises, respectively, on NYU~\cite{silberman2012indoor} (including both training and test sets). Each class is represented by a unique color, and the correspondence between colors and class labels is given in Figure~\ref{fig:noise} (c).
%
% -----------------------------
The way of calculating the percentage of the zero noise on each semantic class of the depth value is expressed as:
\begin{equation} 
\label{eq:zeronoise}
Zero(c) = \frac{\sum \mathbbm{1}_{(Y(d)=c, d \neq 0, d'=0)}}{\sum \mathbbm{1}_{(Y(d)=c, d \neq 0)}},
\end{equation}
% -----------------------------
where $\mathbbm{1}$ is an indicator (\ie, when conditions in the right parentheses are all satisfied, the current value is incremented by one). $d'$ and $d$ denote the noisy and the noise-free depth value, respectively. $Y(d)$ is the class label of the corresponding visible surface whose 3D position is decided by $d$. $c$ denotes a certain semantic class. The zero noise rate of each class is shown as a pie chart in Figure~\ref{fig:noise} (b).
% 
The way of calculating the percentage of delta noise between $c$ (class of the clean visible surface) and $c'$ (class of the noisy visible surface) is formulated as:
% -----------------------------
\begin{equation} 
\label{eq:deltanoise}
Delta(c,c') = \frac{\sum \mathbbm{1}_{(Y(d)=c, (d \cdot d') \neq 0, Y(d') = c', c \neq c')}}{\sum \mathbbm{1}_{(Y(d)=c, (d \cdot d') \neq 0)}}.
\end{equation}
% -----------------------------
% 
It is visualized as a block ($c$ row and $c'$ column) on the confusion matrix in Figure~\ref{fig:noise} (a). 
% 

We can observe from Figure~\ref{fig:noise} (b) that all classes in NYU~\cite{silberman2012indoor} have surprisingly significant zero noise rates, \ie, between $51.8\%$ and $95.5\%$.
This means the model trained on such data might be biased by the incomplete visible surface and resulted in an incomplete prediction.
%
Besides, from Figure~\ref{fig:noise} (a), we can observe that the main semantic confusion (caused by delta noises) is between any semantic class and the special class ``{empty}'', ranging from $29.6\%$ (between ``{empty}'' and ``{TVs}'') to $94.2\%$ (between ``{empty}'' and ``{ceiling}"). Besides, among the non-empty semantic classes of the noisy visible surface, we find that: 1)~several classes confused with the ``{wall}" class; 2)~the class ``{chair}" is easily confused with other non-empty classes. 
% 
The reasons for these phenomena may be that: 1) NYU is an indoor dataset, the visible surfaces of objects are easily shifted to the wall; 2) the chair is not face-structured, and it is difficult to obtain accurate depth values.

% 

\begin{algorithm}[h]
   \caption{GCRL with planning + \highlight{\ALGname}}
   \label{alg:framework}
\begin{algorithmic}
\State {\bfseries Input}: Number of training episodes $M$, horizon $H$
\State Initialize replay buffer $\mathcal{B} \leftarrow \varnothing$.
\State Initialize the parameters of goal-conditioned policy $\pi_{\theta}$.
\State Initialize the parameters of action-value function $Q_{\phi}$.
\For{$m=1, 2, 3, \ldots M$}
\State Reset the environment.
\State Sample a target goal $g$ and an initial state $s_{0}$.
    \For{$t=1, 2, 3, \ldots H$}
    \State Build a graph $\mathcal{H} = (\mathcal{V}, \mathcal{E}, d)$ using $\mathcal{B}$.
    \State Find the shortest subgoal-path $\tau_{g}$ from $s_{t}$ to $g$.
    \State \highlight{Find a desired subgoal $l^{*}$ via Algorithm~\ref{alg:skip}.}
    \State Collect a transition $(s_{t}, a_{t}, r_{t})$ using $\pi_{\theta} (s_{t}, l^{*})$.
    \State Store the transition and the planned path $\tau_{g}$ in $\mathcal{B}$.
    \EndFor
\State Update $Q_{\phi}$ using $\mathcal{L}_{\mathtt{critic}} (\phi)$ of Equation~\ref{eq:ddpg_critic}
\State Update $\pi_{\theta}$ using $\mathcal{L}_{\mathtt{actor}} (\theta) + \highlight{\lambda \mathcal{L}_{\mathtt{\ALGname}} (\theta)} $ of Equation~\ref{eq:total_loss}
\EndFor
\end{algorithmic}
\end{algorithm}

\noindent
\textbf{The Effects of Depth Noises.} 
Figure~\ref{fig:noise} (c) shows the performance gap, which is is obtained from the experimental results on the noise-free NYUCAD dataset minus the experimental results on the noisy NYU dataset.
% 
We choose three top-performing methods, CCPNet~\cite{zhang2019cascaded-ccpnet}, 3D-Sketch~\cite{chen20203d} and FFNet~\cite{wang2022ffnet} as the baseline models. 
Each bar denotes a specific method, \eg, the blue bar for CCPNet~\cite{zhang2019cascaded-ccpnet}, the orange bar for 3D-Sketch~\cite{chen20203d}, and the yellow bar denotes FFNet~\cite{wang2022ffnet}, separately. 
% 
For each method, we present the difference of IoU according to the results reported in its paper, which are respectively trained on the NYUCAD~\cite{firman2016NYUCAD} and NYU~\cite{silberman2012indoor} datasets.
In addition to IoU for each semantic class, SSC mean IoU (SSC-mIoU) and SC-IoU are also used as overall evaluation metrics.
% 
% --------------------------------
As in Figure~\ref{fig:noise} (c), most of the performance gaps on IoU exceed $10\%$, and some of them even exceed $20\%$, such as CCPNet~\cite{zhang2019cascaded-ccpnet} in ``{ceiling}'', and FFNet~\cite{wang2022ffnet} in ``{wall}'', which are significant for SSC.
We can also observe that there is a large performance gap between the experimental results \emph{with} or \emph{without} noise from SSC-mIoU and SC-IoU. The above quantitative analysis validates that the depth noises have indeed caused substantial damage to the experimental results of SSC models.
%
% 

