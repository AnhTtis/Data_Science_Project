%\textcolor{purple}{
%brainstorming titles here:
%\begin{enumerate}
%    \item Automatic Rigging and Animation of Children's Drawings of Humans
%\end{enumerate}
%I'm losing the ``highly varied'' but I think that that can be emphasized in the intro and abstract.   I think that calling them Humans is ok even though some of them are super heros, gods, etc.   I prefer Children's drawings to merely hand drawn as without that flag, people will think of much more professional drawings than what we are working with as that is what most of the prior literature is working with.    What do you think?   I know that comms thought that children's drawings was creepy but Jerome said that it was ok to use that term.}

%\hjs{Title is good. I'm considering augmenting it to `...Young Children's Drawings...' since we're targeting the younger ones, but I'm not sure if it's redundant to add. I've got no problem with having `children' in the title. Because we're basically not using any FB user data and in light of the positive reaction we got from the parents @ fb group, I think we should broadcast the child element. 

%Re: humans, I think we should think about that a bit before deciding. Many of the figures we're targetting are decidedly un-human (mermaids, robots, animals with humane-like proportions). I've been learning towards `humanoid figures' but the phrase is bulky. }{}

%\textcolor{purple}{I'd say that we were ok without the ``young'' as that seems implied by the ``children'' to me and it will cause the title not to fit on one line.    For a similar reason, I'd prefer humans in the title and then we can expand the definition to humanoid later in the abstract, intro etc.   I'm also not sure how many of those non-human examples are going to work.   That mermaid with the sideways tail that we looked at yesterday is going to be tough unless we can just do the arms.}

Children's drawings have a wonderful inventiveness, creativity, and variety to them.
We present a system that automatically animates children's drawings of the human figure, is robust to the variance inherent in these depictions, and is simple and straightforward enough for anyone to use. 
We demonstrate the value and broad appeal of our approach by building and releasing the \AD Demo, a freely available public website that has been used by millions of people around the world.
We present a set of experiments exploring the amount of training data needed for fine-tuning, as well as a perceptual study demonstrating the appeal of a novel \textit{twisted perspective} retargeting technique. 
Finally, we introduce the Amateur Drawings Dataset, a first-of-its-kind \hjs{annotated} dataset, collected via the public demo, containing 180,000 amateur drawings and corresponding \hjs{user-accepted} character bounding box, segmentation mask, and joint location annotations.





%public-facing web 
%We focus on the consequence of all that variety in their drawings of humans 
%as we develop an algorithm to bring them to life through automatic animation.   
%Our goal in this work is to develop a pipeline that can with high probability, and without either adult or child intervention, identify the humans-like figures in a drawing, segment them from the background, rig them by locating the key bones, body parts, and joints, and animate them using motion capture data.   
%We assess the quality of our results by comparing them with crowd-sourced manual annotations and through a series of perceptual user studies.