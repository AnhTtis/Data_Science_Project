% Importance and appeal of children's drawings
Children's depictions of the human figure are highly expressive and varied.
As one of the very first subjects children attempt to draw, the representation begins as an almost unintelligible cloud of scribbles. 
As the child grows, their representation of the human figure becomes more developed and is extended to graphically represent many different types of characters: people, animals, and even personified objects (see Figure 1).

Who among us has not wished, either as a child or as an adult, to see such figures come to life and move around on the page?
Sadly, while it is relatively fast to produce a single drawing, creating the sequence of images necessary for animation is a much more tedious endeavor, requiring discipline, skill, patience, and sometimes complicated software.
As a result, most of these figures remain static upon the page.

% We built a system to animate them.
Inspired by the importance and appeal of the drawn human figure, we design and build a system to automatically animate it given an in-the-wild photograph of a child's drawing. 
Our system is fast, intuitive, and robust to much of the variation present in these types of drawings, making it well-suited to allow our target audience--children--to see their own characters coming to life.
The system is comprised of four stages: figure detection, segmentation masking, pose estimation/rigging, and animation. 
We describe each stage and identify common causes of failure in each. 
For object detection and pose estimation, we make use of existing computer vision models designed to detect human figures and joints in photographs; we fine-tune these models for use with children's drawings.
For segmentation, we present a straightforward, image processing-based method that, for animation purposes, is more useful and accurate than segmentation masks obtained from a fine-tuned object detection model.
During the animation step, we take advantage of the \textit{twisted perspective} commonly seen in children’s drawings to retarget motion capture data onto the character in a novel and appealing way.

% We use existing machine learning models. However, given the wide domain gap it's not clear how much fine-tuning data was needed. So we ran some experiments to find out and report it.
While our system leverages existing models and techniques, most are not directly applicable to the task due to the many differences between photographic images and simple pen and paper representations. 
To this end, we couple the presentation of our system with a set of experiments exploring the relationship between fine-tuning training set size and success rates.
We also include a perceptual study validating viewer preference for incorporating \textit{twisted perspective} into the motion retargeting step.

We validate the desirability and appeal of our system by building and publicly releasing a version of it as the \AD Demo \,\cite{animateddrawings}.
Launched in December 2021, this demo has been used by millions of people around the world to animate their children's drawings.
Inspired by this reception, our second contribution is The Amateur Drawings Dataset: \hjs{180,000 drawings and user-accepted annotations collected, with consent, through the demo. See Section \ref{sec:UI} for a description of how the annotations were generated.}
We believe this dataset will be a resource to researchers from various fields seeking to better understand the space of amateur drawings, evaluate new algorithms in this domain, or develop new drawing-based tools in general.

To summarize, our contributions are as follows:
\begin{enumerate}
    \item 
    We explore the problem of automatic sketch-to-animation for children's drawings of human figures and present a framework that achieves this effect. We also present a set of experiments determining the amount of training data necessary to achieve high levels of success and a perceptual study validating the usefulness of our motion retargeting technique.
    \item To encourage additional research in the domain of amateur drawings, we present a first-of-its-kind dataset of 180,000 user-submitted amateur drawings, along with user-accepted bounding box, segmentation mask, and joint location annotations.
\end{enumerate}

Upon acceptance of this paper, we plan to publicly release the Amateur Drawings Dataset, project code, and fine-tuned model weights.
