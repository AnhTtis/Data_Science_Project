\documentclass[acmtog]{acmart}
\newcommand{\AD}{Animated Drawings }
\newcommand{\acknowledgement[2]}{{#2}{}}

\newcommand{\hjs[2]}{{#2}{}}

\usepackage{censor}

%% NOTE that a single column version may be required for 
%% submission and peer review. This can be done by changing
%% the \doucmentclass[...]{acmart} in this template to 
%% \documentclass[manuscript,screen,review]{acmart}https://www.overleaf.com/project/5ffe206cd7828dd95bf1b7ba
%% 
%% To ensure 100% compatibility, please check the white list of
%% approved LaTeX packages to be used with the Master Article Template at
%% https://www.acm.org/publications/taps/whitelist-of-latex-packages 
%% before creating your document. The white list page provides 
%% information on how to submit additional LaTeX packages for 
%% review and adoption.
%% Fonts used in the template cannot be substituted; margin 
%% adjustments are not allowed.
%%
%% \BibTeX command to typeset BibTeX logo in the docs
\AtBeginDocument{%
  \providecommand\BibTeX{{%
    \normalfont B\kern-0.5em{\scshape i\kern-0.25em b}\kern-0.8em\TeX}}}

%% Rights management information.  This information is sent to you
%% when you complete the rights form.  These commands have SAMPLE
%% values in them; it is your responsibility as an author to replace
%% the commands and values with those provided to you when you
%% complete the rights form.
\setcopyright{acmcopyright}
\copyrightyear{2023}
\acmYear{2023}
% \acmDOI{10.1145/1122445.1122456}
\acmBooktitle{}

%% These commands are for a PROCEEDINGS abstract or paper.
%\acmConference[Woodstock '18]{Woodstock '18: ACM Symposium on Neural
%  Gaze Detection}{June 03--05, 2018}{Woodstock, NY}

%\acmPrice{15.00}
%\acmISBN{978-1-4503-XXXX-X/18/06}


%%
%% Submission ID.
%% Use this when submitting an article to a sponsored event. You'll
%% receive a unique submission ID from the organizers
%% of the event, and this ID should be used as the parameter to this command.
\acmSubmissionID{0115}

%%
%% The majority of ACM publications use numbered citations and
%% references.  The command \citestyle{authoryear} switches to the
%% "author year" style.
%%
%% If you are preparing content for an event
%% sponsored by ACM SIGGRAPH, you must use the "author year" style of
%% citations and references.
%% Uncommenting
%% the next command will enable that style.
\citestyle{acmauthoryear}

%%
%% end of the preamble, start of the body of the document source.


%% Jackie: Added for straikethrough texts using \st{}
\usepackage{soul}


\begin{document}

%%
%% The "title" command has an optional parameter,
%% allowing the author to define a "short title" to be used in page headers.
\title{A Method for Animating Children's Drawings of the Human Figure}

%%
%% The "author" command and its associated commands are used to define
%% the authors and their affiliations.
%% Of note is the shared affiliation of the first two authors, and the
%% "authornote" and "authornotemark" commands
%% used to denote shared contribution to the research.

\author{Harrison Jesse Smith}
\affiliation{%
  \institution{Meta AI Research}
  \country{USA}
}
\email{hjessmith@gmail.com}

\author{Qingyuan Zheng}
\affiliation{%
  \institution{Tencent America}
  \country{USA}
}
\email{qyzzheng@global.tencent.com}
\authornote{Author was affiliated with Meta AI Research while contributing to this work.}

\author{Yifei Li}
\affiliation{%
  \institution{MIT CSAIL}
  \country{USA}
}
\email{liyifei@csail.mit.edu}
\authornotemark[1]

\author{Somya Jain}
\affiliation{%
  \institution{Meta AI Research}
  \country{USA}
}
\email{somyaj@gmail.com}

\author{Jessica K. Hodgins}
\affiliation{%
  \institution{Carnegie Mellon University}
  \country{USA}
}
\email{jkh@cmu.edu}
\authornotemark[1]

%%
%% By default, the full list of authors will be used in the page
%% headers. Often, this list is too long, and will overlap
%% other information printed in the page headers. This command allows
%% the author to define a more concise list
%% of authors' names for this purpose.
\renewcommand{\shortauthors}{Smith et al.}

%%
%% The abstract is a short summary of the work to be presented in the
%% article.
\begin{abstract}
\begin{abstract}
%
Deformable image registration is a fundamental task in medical image analysis and plays a crucial role in a wide range of clinical applications. 
Recently, deep learning-based approaches have been widely studied for deformable medical image registration and achieved promising results. However, existing deep learning image registration techniques do not  theoretically guarantee %diffeomorphic 
topology-preserving transformations. This is a key property to preserve anatomical structures and achieve plausible transformations that can be used in real clinical settings.
%
We propose a novel framework for deformable image registration. Firstly, we introduce a  novel regulariser based on conformal-invariant properties in a nonlinear elasticity setting.
Our regulariser enforces  the deformation field  to be  smooth, invertible and orientation-preserving. %differentiable.  
More importantly, we strictly guarantee topology preservation yielding to a clinical meaningful registration.  Secondly, 
we boost the performance of our regulariser through coordinate MLPs, where one can view the to-be-registered images as continuously differentiable entities. 
We demonstrate, through numerical and visual experiments, that our framework is able to outperform current techniques for image registration.
%
\keywords{Homeomorphic image registration \and Lung CT \and Conformal invariant hyperelastic regularisation.}
%
\end{abstract}
\end{abstract}

%%
%% The code below is generated by the tool at http://dl.acm.org/ccs.cfm.
%% Please copy and paste the code instead of the example below.
%%
\begin{CCSXML}
<ccs2012>
<concept>
<concept_id>10010147.10010371.10010352</concept_id>
<concept_desc>Computing methodologies~Animation</concept_desc>
<concept_significance>500</concept_significance>
</concept>
<concept>
<concept_id>10010147.10010371.10010382</concept_id>
<concept_desc>Computing methodologies~Image manipulation</concept_desc>
<concept_significance>300</concept_significance>
</concept>
</ccs2012>
\end{CCSXML}





\ccsdesc[500]{Computing methodologies~Animation}
\ccsdesc[300]{Computing methodologies~Image manipulation}

%%
%% Keywords. The author(s) should pick words that accurately describe
%% the work being presented. Separate the keywords with commas.
%\keywords{datasets, neural networks, gaze detection, text tagging}

%% A "teaser" image appears between the author and affiliation
%% information and the body of the document, and typically spans the
%% page.

\begin{teaserfigure}
  \includegraphics[width=\textwidth]{images/banner_out.png}
  \caption{\hjs{We present a fast and easy-to-use method for animating the types of abstract and varied human-like figures drawn by children.}}
  \Description{Banner Caption}
  \label{fig:teaser}
\end{teaserfigure}

%%
%% This command processes the author and affiliation and title
%% information and builds the first part of the formatted document.
\maketitle

\section{Introduction}
% Importance and appeal of children's drawings
Children's depictions of the human figure are highly expressive and varied.
As one of the very first subjects children attempt to draw, the representation begins as an almost unintelligible cloud of scribbles. 
As the child grows, their representation of the human figure becomes more developed and is extended to graphically represent many different types of characters: people, animals, and even personified objects (see Figure 1).

Who among us has not wished, either as a child or as an adult, to see such figures come to life and move around on the page?
Sadly, while it is relatively fast to produce a single drawing, creating the sequence of images necessary for animation is a much more tedious endeavor, requiring discipline, skill, patience, and sometimes complicated software.
As a result, most of these figures remain static upon the page.

% We built a system to animate them.
Inspired by the importance and appeal of the drawn human figure, we design and build a system to automatically animate it given an in-the-wild photograph of a child's drawing. 
Our system is fast, intuitive, and robust to much of the variation present in these types of drawings, making it well-suited to allow our target audience--children--to see their own characters coming to life.
The system is comprised of four stages: figure detection, segmentation masking, pose estimation/rigging, and animation. 
We describe each stage and identify common causes of failure in each. 
For object detection and pose estimation, we make use of existing computer vision models designed to detect human figures and joints in photographs; we fine-tune these models for use with children's drawings.
For segmentation, we present a straightforward, image processing-based method that, for animation purposes, is more useful and accurate than segmentation masks obtained from a fine-tuned object detection model.
During the animation step, we take advantage of the \textit{twisted perspective} commonly seen in children’s drawings to retarget motion capture data onto the character in a novel and appealing way.

% We use existing machine learning models. However, given the wide domain gap it's not clear how much fine-tuning data was needed. So we ran some experiments to find out and report it.
While our system leverages existing models and techniques, most are not directly applicable to the task due to the many differences between photographic images and simple pen and paper representations. 
To this end, we couple the presentation of our system with a set of experiments exploring the relationship between fine-tuning training set size and success rates.
We also include a perceptual study validating viewer preference for incorporating \textit{twisted perspective} into the motion retargeting step.

We validate the desirability and appeal of our system by building and publicly releasing a version of it as the \AD Demo \,\cite{animateddrawings}.
Launched in December 2021, this demo has been used by millions of people around the world to animate their children's drawings.
Inspired by this reception, our second contribution is The Amateur Drawings Dataset: \hjs{180,000 drawings and user-accepted annotations collected, with consent, through the demo. See Section \ref{sec:UI} for a description of how the annotations were generated.}
We believe this dataset will be a resource to researchers from various fields seeking to better understand the space of amateur drawings, evaluate new algorithms in this domain, or develop new drawing-based tools in general.

To summarize, our contributions are as follows:
\begin{enumerate}
    \item 
    We explore the problem of automatic sketch-to-animation for children's drawings of human figures and present a framework that achieves this effect. We also present a set of experiments determining the amount of training data necessary to achieve high levels of success and a perceptual study validating the usefulness of our motion retargeting technique.
    \item To encourage additional research in the domain of amateur drawings, we present a first-of-its-kind dataset of 180,000 user-submitted amateur drawings, along with user-accepted bounding box, segmentation mask, and joint location annotations.
\end{enumerate}

Upon acceptance of this paper, we plan to publicly release the Amateur Drawings Dataset, project code, and fine-tuned model weights.


\section{Background}
Our work builds on existing methods from several fields but is, to our knowledge, the first work focused specifically on fully automatic animation of children's drawings of human figures. 
To ground the work, we provide a summary of salient observations from the field of children's art analysis.
In addition, we briefly review related work on 2D image-to-animation and object and pose estimation for abstract images. 


\subsection{Analysis of Children's Drawings}

\hjs{
Children's drawings have long been of interest to the scientific community.
For well over a century, researchers from multiple fields have 
collected\,\cite{IndianaS55:online,kellogg1967rhoda,AWebbasedDatabaseforDrawingsofGods,geist2002they}
and analyzed them, seeking to understand and measure children's thought processes\,\cite{sully2021studies,barnes1892study,clark1897child,buhler2013mental}, 
intellectual development\,\cite{goodenough1926measurement},
and perceptions\,\cite{chambers1983stereotypic,doi:10.1080/01443410500344167}.
}
Particular attention has been given to drawings of human figures, one of the first and most frequently drawn subjects throughout childhood\,\cite{cox2013children}.

As the child develops, the schemas they employ to represent the human form become more complete (see Figure \ref{fig:tadpole-transitional-conventional}).
Even within these schemas, there is significant variation.
In addition to asymmetries and variation in color and proportion, many body parts appear optional to include; a study of drawings by 4-6 year old children showed that, while heads, legs, and eyes are almost universally present, other body parts (including torsos, arms, hands, and feet) were frequently absent\,\cite{cox2013children}.
Inversely, non-human body parts are frequently added in order to represent other subject classes\,\cite{kellogg1969analyzing}. With the addition of large ears, the figure may represent a cat or bear (Figures \ref{fig:maskrcnn_before_after}.m and \ref{fig:maskrcnn_before_after}.g); with the addition of a crown, it can represent a pineapple (Figure \ref{fig:maskrcnn_before_after}.n).
All of these sources of character variation make automatic character animation from drawings a non-trivial task.

\begin{figure}
\includegraphics[width=\linewidth]{images/tadpole-transition-conventional.png}
\caption{
As children learn to draw the human figure, the morphologies of the schemas they employ vary and evolve considerably\,\cite{cox2014drawings}.
Children frequently begin by drawing a \textit{tadpole figure}, a circular head region from which arms and legs extend. 
Some will progress to a \textit{transitional figure}, dropping the arms down so they extend from the legs. 
When a line is drawn between the legs, creating the separate torso region, the \textit{conventional figure} is formed.
Though these are small changes from the perspective of the drawer, they result in significantly different character morphologies when viewed through the lens of character animation.
A successful drawing-to-animation system must be robust to these types of variations.}
\label{fig:tadpole-transitional-conventional}
\end{figure}

Many researchers have focused closely on the unique style of children's drawings.
The psychologist and artist John Willats argues that, in order to understand the style of children's drawings, one must understand that the primary picture primitives employed by children are \textit{regions}, or 2D areas\,\cite{willats2006making}.
A squat volume, such as a head or torso, may be represented by a circular or ellipsoid region, whereas an elongated volume, such as a leg, may be represented by a long, thin region or even a single line.
These regions are not depictions of the object from any particular point of view. 
Rather, they are \textit{3D volumetric object-centered descriptions}\,\cite{marr1982vision},
2D areas with attributes perceptually similar to those of 3D object they are meant to represent.
%The regions begin as circles and lines, but later become modified to better reflect the perceptually impactful aspects of the objects they represent; a region representing a sugar cube or die may be given square corners, and a long region representing an arm may be given a bend to depict the elbow or split at the end to represent fingers (CITE Willats, 2005).

There are two stylistic outcomes of these \textit{object-centered descriptions} that bear mention.
First, the use of foreshortening is very rare in children's drawings \,\cite{piaget1956, willats1992representation}. 
This design choice is understandable; foreshortening a long region, such as a limb, results in a short region which does not adequately reflect the \textit{longness} of the object.
Second, the human figure may appear to have been drawn from many different perspectives, so as to make each part of the character maximally recognizable.
For example, the head and torso may face forward while the legs and feet are pointed to the side.
This technique, often referred to as \textit{twisted perspective}, is frequently seen and well-documented\,\cite{dziurawiec1992twisted}.
Both of these stylistic aspects are used to guide the design decisions of our system when applying human motion capture data onto the character.


\subsection{2D Image to Animation}

Previous researchers have proposed methods to animate drawings or photographs, many of which rely upon additional modes of user input.
Hornung et al. present a method to animate a 2D character in a photograph, given user-annotated joint locations\,\cite{Hornung2007anim2Dpicmotion}.
Pan and Zhang demonstrate a method to animate 2D characters with user-annotated joint locations via a variable-length needle model\,\cite{Pan2011}.
Jain et al. present an integrated approach to generate 3D proxies for animation given joint locations, segmentation masks, and per-part bounding boxes\,\cite{jain:2012}. 
Levi and Gotsman provide a method to create an articulated 3D object from a set of annotated 2D images and an initial 3D skeletal pose\,\cite{ArtiSketch}.
\textit{Live Sketch}\,\cite{su2018livesketch}
tracks control points from a video and applies their motion to user-specified control points upon a character.
Other approaches allow the user to specify character motions through a puppeteer interface, using RGB or RGB-D cameras\,\cite{held20123d,barnes2008video}.
\textit{ToonCap}\,\cite{Fan:2018:TAL} focuses on an inverse problem, capturing poses of a known cartoon character, given a previous image of the character annotated with layers, joints, and handles. 


\textit{Toonsynth}\,\cite{Dvoroznak18-SIG} and \textit{Neural Puppet}\,\cite{poursaeed2020neural} both present methods to synthesize animations of hand-drawn characters given a small set of drawings of the character in specified poses.
Hinz et al. train a network to generate new animation frames of a single character given 8-15 training images with user-specified keypoint annotations\,\cite{hinz2022charactergan}.

\textit{Monster Mash}\,\cite{Dvoroznak20-SA} presents an intuitive framework for sketch-based modeling and animation, and \textit{2.5D Cartoon Models}\,\cite{10.1145/1778765.1778796} presents a novel method of constructing 3D-like characters from a small number of 2D representations. 
Both of these are intuitive and well designed animation tools targeted towards amateur users.


\hjs{
Some animation methods are specifically tailored toward particular forms, such as faces\,\cite{elor2017bringingPortraits}, coloring book characters\,\cite{magnenat2015live}, or characters with human-like proportions. 
One notable work that is focused on the human form is \textit{Photo Wake Up}\,\cite{weng2019photo}. 
The authors show a method for creating a rigged and textured 3D mesh from a single image of a human-like figure.
Similar to us, their end goal is to allow users to seamlessly bring 2D characters to life; their work does an impressive job of accomplishing this.
Our method differs in two significant ways. 
First, while their work is focused on creating a 3D model for a mixed reality use case, 
ours is specifically focused on animating twisted perspective figures while staying within a 2D plane.
Second, children's drawings are much more abstract, incorrectly proportioned, and non human-like than the examples demonstrated in the paper.
We test our method upon the more abstract examples demonstrated in their paper and, with minor segmentation cleaning, they were successfully animated by our method.
}












\hjs{While the approaches listed here are wonderful tools to ease the burden of animation, none were perfectly suited to our use case.
Some require additional user input beyond the drawing itself, making the animation process more complex.
Others require the user to consistently draw the same character in multiple poses, which is beyond the skills of young children.
Others are focused on animating specific forms, precluding their use on children's drawings of the human figure.}


%Siarohin and colleagues propose a method for animating arbitrary classes of subjects,
%but require training videos of class members moving\,\cite{Siarohin_2019_NeurIPS}, making it unsuitable children's drawings.


\subsection{Detection, Segmentation, and Pose Estimation on Non-Photorealistic Images}

\hjs{
Aided by the the existence of large annotated datasets\,\cite{lin2014microsoft,6909866,6682899}, researchers have made considerable progress solving the problems of object detection, segmentation, and pose estimation from photographs. See, for example\,\cite{MaskRCNNhe2017mask,openpose19,guler2018densepose,alphapose,toshev2014deeppose}.
We explain the methods in this area that we leverage in Sections \ref{sec:character_detection} and \ref{sec:joint_detection}.

While traditional methods for detection, segmentation, and pose estimation of non-photorealistic images exist\,\cite{choi2012retrieval,bregler2002turning,davis2006sketching,eitz2012humans}, the lack of easily available datasets has resulted in slower adoption of deep learning models.
Some researchers are addressing this problem by developing methods and releasing datasets focused on the domain of anime characters\,\cite{chen2022bizarre,10.1145/3011549.3011552}, professional sketches\,\cite{brodt2022sketch2pose}, and mouse doodles\,\cite{ha2017neural}.
Other researchers have presented a non-deep learning method for inferring character poses from \textit{gesture drawings}\,\cite{Gesture3D}.
}
Because the Amateur Drawings Dataset is comprised of in-the-wild photographs of drawings created by the general public, we believe it will complement the value of existing datasets and allow for new dimensions of exploration and analysis.


\section{Method}
\section{Testing for anisotropy}\label{sec:TestingAnisotropy}
The specific hypothesis to be tested is whether, above some energy threshold, $E_{\rm th}$, the mean composition of UHECRs coming from directions near to the galactic plane is significantly higher in mass than those arriving further from it. This is to be tested using \xmax{} as a mass sensitive parameter. Typically, \xmax{} based composition analyses leverage the first two moments of \xmax{} distributions binned in energy, to comment on primary mass. This approach, however, does not lend itself well to quantifying the significance of a result testing the above statement. Instead, a test statistic, $TS$, which quantifies the degree of dissimilarity between the \xmax{} distributions in the two regions in a single value is preferred. For this, the returned value from the Anderson-Darling two-sample homogeneity test \cite{andersondarling}, \textit{AD-test}, has been selected as it scales with the dissimilarity of the tested distributions. The AD-test has good sensitivity to the full width of a distribution \cite{scholz1987k}, and has more power than the Kolmorogov-Smirnov test while remaining robust against false positives \cite{engmann2011comparing}.

To use the AD-test and \xmax{} for this purpose, two modifications are required. First, a single $TS$ comparing all events in each region above $E_{\rm th}$ is desired. So, all events with $E\geq E_{\rm th}$ in the on- and off-plane samples separately need to be collected into a common on-plane distribution and a common off-plane distribution. To do this, the natural evolution of \xmax{} with energy needs to be removed so that spectral features in the flux do not influence the result. Therefore, we define an energy-normalized \xmax{} value
\begin{equation}\label{eq:XmaxNorm}
X_{\text{max}}^{'} =  X_{\text{max}} -  \underbrace{\left(649 + 63.1 \, Z_{18} + 1.97 \, Z_{18}^{2}\right)}_\text{EPOS-LHC elongation rate for iron},
\end{equation}
where $Z_{18}=\log_{10}\left(E_\text{rec}/\,\text{EeV}\right)$. The last term in \autoref{eq:XmaxNorm} is the natural energy evolution of mean \xmax{} for iron primaries as predicted by EPOS-LHC~\cite{Pierog:2013ria}\footnote{Choice of hadronic interaction model varies result by $\sim0.02$\,\gcm{}.}. Second, the \xmaxnorm{} distribution of an on-plane sample populated with primaries which are on average heavier than those in the off-plane sample will display a lower mean and a narrower width than that of the off-plane \xmaxnorm{} distribution. Since the null hypothesis is that there is either no composition difference or a heavier off-plane sample, a $TS$ sensitive to the ordering of the \xmaxnorm{} distributions is required\footnote{Modifying the test to also require $\sigma( X_{\text{max}}^\prime)^{\rm on} < \sigma( X_{\text{max}}^\prime)^{\rm off}$ would be more restrictive, but conservatively has not been applied.}. The AD-test is insensitive to ordering, so it is modified to
\begin{equation}
TS =
\begin{cases}
    AD: \langle X_{\text{max}}^\prime \rangle^{\rm on} < \langle X_{\text{max}}^\prime \rangle^{\rm off} \\
    -3\hspace{1mm}: \text{else}
\end{cases},
\end{equation}
where $AD$ is the result of the AD-test comparing the on- and off-plane distributions, and $-3$ is selected as it is well below the minimum of the AD-test.

\vspace{-.1cm}
\myparagraph{Scan for energy and galactic latitude thresholds}
\vspace{-.1cm}
A scan has been used to select the optimal on/off splitting latitude, $b_{\rm split}$, and minimum energy, $E_{\rm th}$, as uncertainties in GMF models and source distributions make other approaches impractical. In this scan, each trial [$E_{\rm th}$, $b_{\rm split}$] pair is used to form on- and off-plane subsets and the $TS$ is extracted. To preserve the statistical strength of the sparse FD data set, a coarse scan of $5^\circ$ steps in $\abs{\,b\,}$ from $20^\circ$ to $35^\circ$ and 0.1\,\lge{} steps in energy from $18.4$ to $19.4$\,\lge{} is used. The scan is performed on the data set from~\cite{Aab:2014kda}, which includes events through Dec 31\textsuperscript{st} 2012. At the time of writing, this \textit{scan data set} represents $54\,\%$ of the analyzed events. The remaining $46\,\%$ of events, the \textit{post-scan data set}, is reserved as blind. 

\begin{figure}[!htb]
    % \vspace{-.4cm}
    \centering
    \includegraphics[width=0.45\textwidth]{Figures/ScanResults.pdf}
    \vspace{-2mm}
    \caption{Parameter scan over 54\% of the data.}\label{fig:PRDScan}
    % \vspace{-7mm}
\end{figure}

Interestingly, as shown in \autoref{fig:PRDScan}, all tested pairs result in $\langle X_{\text{max}}^\prime \rangle^{\rm on} < \langle X_{\text{max}}^\prime \rangle^{\rm off}$. An optimal [$E_{\rm th}$, $b_{\rm split}$] of [$10^{18.7}$\,eV,$30^\circ$] was found with a $TS = 8.4$. The selected [$E_{\rm th}$, $b_{\rm split}$] is applied as a prescription to the post-scan data set, which independently confirms the result with a $TS = 12.6$, for a total $TS=21.0$ for the full data set. 

\vspace{-.1cm}
\myparagraph{Statistical significance}
\vspace{-.1cm}
The chance probability of the observed TS occurring with in an isotropic sky is tested using Monte Carlo methods on randomized skies derived from the real data. To form each randomized sky, the arrival direction is first decoupled from the energy and \xmaxnorm{} values of each event. These are then randomly re-paired to create a new sky which maintains the real \xmax{}, energy, and sky exposure distributions, but has a scrambled arrival direction/composition pairing. The above analysis is then used to extract a $TS$ from each sky which is compared to the result in data. Skies which display more extreme on-/off-plane differences than those observed in data are tallied and used to calculate the probability of an isotropic sky generating the observed $TS$. The results of this procedure are shown in  \autoref{fig:TStoSignificanceConversionNew}.

\begin{figure}[!htb]
    \centering
    \includegraphics[width=.8\columnwidth]{Figures/MCADSig.pdf}
    % \vspace{-3mm}
    \caption{The Monte Carlo determination of the post-scan (red) and all-data (blue) significance with 1 and 10 billion randomized skies, respectively.}\label{fig:TStoSignificanceConversionNew}
    % \vspace{-3mm}
\end{figure}

For the blind, post-scan data set, the prescribed [$E_{\rm th},b_{\rm split}$] pair is used to split each randomized sky into on- and off-plane samples and a $TS$ is extracted. In one billion random skies, only 5865 resulted in a more extreme $TS$ than the 12.6 observed in data. This indicates a chance probability of $5.87\times10^{-6}$ which corresponds to 4.4\,$\sigma$. 

To calculate the significance of the result when the scan- and post-scan data sets are combined, the entire analysis chain, including the scan, is duplicated. In each random sky, 54\,\% of the data is used to scan for the [$E_{\rm th},b_{\rm split}$] pair which results in the most extreme result, fully penalizing for the scan. These values are then used to split all data in the random sky into on- and off-plane subsamples and the $TS$ for the sky is extracted. From 10 billion random skies, only 5964 resulted in a more extreme $TS$ than the 21.0 observed in data. This indicates to a chance probability of $5.96\times10^{-7}$ which corresponds to 4.9\,$\sigma$. The strong penalization of the scanned data is evident as the additional 54\,\% of the data (with \Dxmaxmunorm{} $= 8.5$\,\gcm{}) only resulted in an 11\,\% increase of the significance of the observation. 

\myparagraph{\xmax{} moments and trends}

To illustrate the difference in composition on and off the plane, the first two moments of the \xmax{} distribution in each 0.1\,\lge{} energy bin has been plotted in \autoref{fig:CompositionPlots} for both regions. Above $10^{18.7}$\,eV there is a clear separation in \xmaxmu{} for all energy bins. Most energy bins also display a separation in \xmaxsigma{}. Heavier primaries are expected to, on average, have a shallower \xmax{} and lower shower-to-shower fluctuations. Therefore the correlated difference seen here indicates that, for this data sample, primaries from the on-plane region have a higher mean mass than that of the off-plane region above $10^{18.7}$\,eV.

To evaluate the degree to which fluctuation plays a role in the observed result, the growth of the $TS$ over time has been plotted in \autoref{fig:TimeEvolution}. The time evolution of the signal is consistent with linear growth at a rate of 1.3\,$TS$\,yr$^{-1}$. This behavior is in line with expectations for a real difference in mean mass between the subsamples. The shaded region of \autoref{fig:TimeEvolution} shows preliminary data from 2019. These reconstructions were not subject to a validated reconstruction chain and may change. Still, when added, a 3.7/4.4\,$\sigma$ (post-scan/all data) statistical significance is expected. The best fit rate of growth of 1.3\,$TS$\,yr$^{-1}$ remains unchanged.

\begin{figure*}[!hbt]
\centering
    \begin{minipage}{.63\textwidth}
      \centering
      \captionsetup{width=.9\linewidth}
      \includegraphics[width=.49\textwidth,valign=t]{Figures/Mean-crop.pdf}
      \includegraphics[width=.49\textwidth,valign=t]{Figures/Sigma-crop.pdf}
      \vspace{-1mm}
      \caption{The first (left) and second (right) moments of the \xmax{} distributions from on- and off-plane regions.}
      \label{fig:CompositionPlots} 
    \end{minipage}%
    \hfill
    \begin{minipage}{.35\textwidth}
      \centering
      \vspace{-1mm}
      \includegraphics[width=\textwidth,valign=t]{Figures/TimePredict.pdf}
      \vspace{1.5mm}
      \captionof{figure}{The time evolution of the TS with significance indicated on the right. The shaded region is preliminary data.}
      \label{fig:TimeEvolution}
    \end{minipage}
\end{figure*}

\section{Evaluation and Analysis}
We evaluate our system in three ways.
First, we briefly describe the public reception of the demo.
Second, we present a set of experiments exploring the effect of training data size on the system's success rate.
Third, we perform a user study to validate the appeal and desirability of twisted perspective motion retargeting.

\subsection{Public Reception}
On December 16, 2021, a version of the proposed system was publicly released as the \AD Demo\,\cite{animateddrawings}.
The launch was accompanied by several high-profile social media posts and a blog post; however, all subsequent online promotion
came from users organically sharing the demo within their networks.

Over the next nine months, over 3.2 million unique users visited the site and spent, on average, over five minutes using the demo.
They uploaded 6.7 million images and, on average, generated four animations per image.
Based upon a subset of highly visible social media posts, the demo is especially popular among parents, elementary school teachers, technology enthusiasts, and artists.


\subsection{Effect of Training Sample Size}
\label{sec:effect_of_training_sample_size}

Our system incorporates repurposed computer vision models trained on photographs of real-world objects.
Because the domain of children's drawings is significantly different in appearance, these models must be fine-tuned prior to use.
However, given the abstract and varied nature of the drawings, it is not obvious how many drawings must be collected and annotated for training.
Therefore, we present a set of experiments exploring the relationship between training dataset size and model prediction success.

We report the performance of the models in two ways. 
Because the models employed have pre-established accuracy metrics, we first report the achieved mean average precision (mAP) (AP:5:.95:05) for each model.
However, our goal--animation--is a somewhat distinct downstream use of these predictions, and the mAP may not fully reflect the rate of success.
For example, a predicted bounding box that overlaps ground truth by 90\% would contribute to a very high mAP; 
however, if the prediction excluded a figure's foot or cut off half of its head, the resulting animation would be considered a failure.
Therefore, we also report the percent of predictions that result in successful animations, as determined by visual inspection.

\hjs{We compare the performance of several different fine-tuned versions of our models.
First, we fine-tune using 179,000 images from the Amateur Drawings Dataset; we excluded 1072 images to use for validation (as described below).
However, a portion of the user-accepted annotations are noisy and inaccurate; therefore, we also fine-tune models using 'clean' training datasets of multiple sizes.}
To obtain these, we randomly selected and manually reviewed images and annotations from with the Amateur Drawings Dataset.
Images that had clearly incorrect annotations were rejected.
Common reasons for rejection included: segmentation masks that did not contain the entire figure or included background elements, 
limbs that were fused together, joints that did not lie on the figure. 
In total, we identified 2,500 images with suitable user-accepted annotations to serve as our training and validation data.

We randomly selected 500 images to serve as the validation set across all training runs, while the remaining 2,000 served as the training sample pool.
We created eight different training sets, varying in size from 10 through 2,000. 
For each training set, we randomly selected data samples from the training sample pool of 2,000 until we obtained the appropriately sized set.

We used the model architecture and training parameters specified in Sections \ref{sec:character_detection} (for both detection and segmentation predictions) and \ref{sec:joint_detection}.
Because our goal is to show the effect of training sample size, rather than optimize absolute accuracy, we restrict ourselves to a single model architecture and keep all hyper-parameters constant.

To evaluate the percent of predictions suitable for animation, we used the same training sets as described above, but also included the additional set of all 2,500 images. 
For evaluation, we randomly selected an additional 571 images that were uploaded to the \AD Demo. 
While we reviewed these images to ensure that their contents were suitable for animation, we did not review, nor do we make use of, their user-approved annotations. Instead, model predictions were visually inspected to determine whether they would result in a successful animation.

This evaluation was meant to give an assessment of the models' in-the-wild success rates, and not have it be biased towards simpler drawings that our system could already predict perfectly, or those that took very little work to clean up.
A detection was classified as failure if it did not detect the human figure, detected it multiple times, falsely detected non-human figures in the scene, had a bounding box that cut off a portion of the figure necessary for animation (such as an arm or foot), or had a bounding box extending to include other markings that were not a part of the figure.
A segmentation was classified as failure if it included background elements that were not part of the figure, did not tightly confirm to the bounds of the figure, contained holes in the interior of the figure, was more than one distinct polygon, or connected figure limbs at locations without a joint.
A pose estimation was classified as failure if the nose, shoulders, hips, elbows, knees, wrists, or ankles were not located on or in close proximity to the correct body part.

\subsubsection{Results}
Validation set mAP as a function of fine-tuning training set size is shown in Table \ref{table:mAP}. %and Fig. \ref{fig:sample-size-vs-map}.
\hjs{Using a Linux server with two NVIDIA Quadro GP100 graphics cards, models trained with 179,000 samples converged in 20 hours, whereas the smaller training sets all converged in under 5 hours.}
For comparison, we also show the mAP obtained when using pretrained model weights (essentially, a fine-tuning training set size of zero) and considering the drawn human figures to be instances of the \textit{person} object class.


\begin{table}[]
\resizebox{\columnwidth}{!}{
\begin{tabular}{r|ccc}
\hline
Fine-Tuning &Bounding Box & Segmentation & Pose Estimation\\
Training Set Size & mAP & mAP & mAP \\
\hline
(no fine-tuning) 0 & 0.06 & 0.04 & 0.09 \\
10 & 0.27 & 0.30 & 0.34 \\
100 & 0.51 & 0.51 & 0.76 \\
250 & 0.58 & 0.57 & 0.80 \\
500 & 0.69 & 0.63 & 0.82 \\
1000 & 0.77 & 0.68 & 0.84 \\
1500 & 0.80 & 0.70 & 0.85 \\
2000 & 0.81 & 0.71 & 0.85 \\
\hjs{(noisy) 179,000} & \hjs{0.82} & \hjs{0.49} & \hjs{0.90} \\
\hline
\end{tabular}}
\caption{
Per stage final mean average precision obtained on validation set as a function of fine-tuning training set size.
} 
\label{table:mAP}
\end{table}

%\begin{figure*}
%\includegraphics[width=\linewidth]{images/sample-size-vs-map.png}
%\caption{Training curves showing effect of training dataset size vs. validation set mAP for Mask R-CNN bounding box and segmentation predictions (left and center) and pose estimation model predictions (right).}
%\label{fig:sample-size-vs-map}
%\end{figure*}


The percentage of successful, animation-ready model predictions on the random 571 test images are given in Table \ref{table:animation_success_rate}.
We report the percentage of predictions that were successful in each stage, as well as the percentage of images for which predictions in all three stages were successful.
Because our system uses the image processing-based approach described in Section \ref{sec:character_segmentation}, we also evaluate this technique's performance using the same segmentation success-failure criteria; 42.4\% of segmentation masks obtained this way were successful.
In parentheses in the rightmost column of Table \ref{table:animation_success_rate}, we report the percentage of images for which predictions in all three stages were successful when the image processing-based segmentation algorithm is used instead of a fine-tuned model prediction.


\begin{table*}[]
%\resizebox{\textwidth}{!}{
\begin{tabular}{r|cccccc}
\hline
Fine-Tuning       & Bounding Box  & Segmentation  & Segmentation  & Pose Estimation  & All Stages & All Stages\\
Training Set Size & Success Rate  & Success Rate  & Success Rate  & Success Rate     & Success Rate & Success Rate\\
                  &               & \textcolor{gray}{(Mask R-CNN)}  & \textcolor{gray}{(Image Process)}  &            & \textcolor{gray}{(Mask R-CNN Seg.)} & \textcolor{gray}{(Image Process Seg.)}\\
\hline
(no fine-tuning) 0& 0.4  & 0.0  & |    & 0.6  & 0.0  & 0.0\\
                10 & 27.1 & 0.4  & |    & 2.1  & 0.0  & 0.9\\
               100 & 60.9 & 6.4  & |    & 54.1 & 4.9  & 19.4 \\
               250 & 62.2 & 8.2  & |    & 69.5 & 6.4  & 24.0 \\
               500 & 74.4 & 14.5 & 42.4 & 77.4 & 12.8 & 30.5 \\
              1000 & 83.0 & 19.4 & |    & 83.0 & 17.7 & 34.7 \\
              1500 & 89.8 & 20.3 & |    & 87.4 & 19.1 & 37.7 \\
              2000 & 91.8 & 22.7 & |    & 89.5 & 21.2 & 38.9 \\
              2500 & 92.5 & 24.7 & |    & 90.2 & 23.3 & 39.4 \\
\hjs{(noisy) 179,000} & \hjs{92.5} & \hjs{16.1} & |    & \hjs{94.6} & \hjs{16.1} & \hjs{40.6} \\
\hline
\end{tabular}%}
\caption{
Percentage of model predictions that can successfully be used for animation, as a function of model fine-tuning training set size.
We report the successes per stage for the bounding box, segmentation mask, and pose estimation model predictions.
In the right-most column, we report the percentage of images for which the bounding box, segmentation mask, and pose estimation model predictions were all successful. 
We also report, in parentheses, the percentages of images that were successful when the image processing-based approach, which has a success rate of 42.4\%, was used instead of the segmentation mask model prediction.} 
\label{table:animation_success_rate}
\end{table*}















\subsubsection{Discussion}
As Table \ref{table:mAP} shows, using models weight trained with real-world images results in very low mAP scores for bounding box, segmentation masks, and pose estimation predictions upon children's drawings.
However, fine-tuning results in a large gain in accuracy across all steps.
Continuing to increase the number of fine-tuning training samples results in continuing, yet slowing, improvements in mAP; 
increasing training set size from 1,500 to 2,000 increases performance by a single percentage point for bounding box and segmentation predictions and does not measurably improve pose estimation.

\hjs{
Interestingly, using the dataset of 179,000 images with noisy annotations results in a minor improvement for bounding box predictions, a significant improvement for pose estimation predictions, and negatively impacts segmentation predictions.
Fixing segmentation masks within the \AD Demo is more tedious than fixing bounding box or joint locations.
Therefore, it is likely that more users skipped the segmentation clean up step. 
This resulted in more noisy segmentation masks within the Amateur Drawings dataset, which lowered the performance of the fine-tuned models.
This suggests that, depending upon the complexity of the prediction clean-up tasks offloaded onto the user during data collection, it may or may not be worthwhile to perform additional processing and refinement upon the collected annotations.
}

\hjs{Table \ref{table:animation_success_rate} shows the percentage of model predictions that could successfully be used for animation.
Similarly, without fine-tuning only a very small percentage of bounding box and pose estimation predictions are usable; none of the segmentation predictions are usable.
When fine-tuning with 2,500 `clean' samples, the percentage of usable bounding boxes, segmentation masks, and pose estimations increases to 92.5\%, 24.7\%, and 90.2\%, respectively.
When using the noisy training set of 179,000 images, the pose estimation success rate increased to 94.6\%, while the bounding box success rate was unchanged and the segmentation success rate dropped substantially.
} 
In the supplemental materials, we present many examples of successful and unsuccessful detection and pose predictions from the models trained with 2,500 samples.

Segmentation mask predictions, by contrast, require many more training samples to obtain comparable rates of success;
by a large margin, this step is the most difficult and failure-prone.
With even 2,500 training samples, fewer than one quarter of predictions from Mask R-CNN are suitable for animation without some sort of manual clean-up.
In part, this can be attributed to the presence of many `hollow' or `outline' figures within the dataset, for which the texture of the figure and the texture of background are identical. 
A hollow character's predicted segmentation mask often contains holes within the sparse, non-detailed parts of the figure, and includes connections between non-attached body parts that are drawn close together.
Model predictions also often fail on stick legs and stick arms, which are often missed, especially when other parts of the figure are 2D regions with area.
We present examples of all of these types of failures in the supplemental material.

In comparison, our image processing-based segmentation approach results in a 42.4\% success rate. 
While this approach does a better job of following the outline of the figure, it frequently fails on images with hard shadows introduced during the photographing of the drawing, drawings on lined paper, and figures that are not watertight or have limbs that do not connect.
With the image processing-based segmentation approach, 39.4\% of figures could be fully automatically animated without any manual intervention.
Clearly, further work on robustly segmenting hand drawn figures, or automatically refining the segmentation masks, would be useful in improving the overall success rate.

%This surprisingly high success rate is due, in part, to the fact that many of the images contain only a single humanoid character upon a blank white background, making the detection task quite straightforward.
%Performance increases with additional training samples; with 2500 training samples, the largest training set size tested, suitable bounding boxes were obtained 92.5\% of the time.
%Examples of images that failed can be seen in Figure \ref{ref:det_failures},.

%Examples of failures can be seen in Figure \ref{ref:seg_failures}.
%While these masks can be fixed by a user quite easily, further work on robustly segmenting these characters would be useful in improving the overall success rate.
%For example, both our segmentation approaches fail to appropriately handle characters that are drawn with limbs touching non-connected body parts.
%To be robust to these types of drawings, a parts-based segmentation approach should be used.



%When extracting segmentation masks suitable for animation, many more training samples are needed to obtain good results;
%by a large margin, this is the most difficult and failure-prone step.
%With even 2500 training samples, fewer than one quarter of predictions from Mask R-CNN are suitable for animation without manual clean-up.

%In part, this can be attributed to the presence of many 'hollow' or 'outline' characters within the dataset, for which the texture of the character and the texture of background are identical. This can result in segmentation masks containing holes within sparse, non-detailed parts of the character (Fig. 9, row 1, 1st image), as well as connections between non-attached body parts which are drawn close together (Fig. 9, row 2, 3rd image).
%Mask R-CNN predictions also often fail on stick legs and stick arms, which are often missed, especially when other parts of the character contain volume (Fig. 9, row 2, 6th and 7th image)



%Examples of failures can be seen in Figure \ref{ref:seg_failures}.
%While these masks can be fixed by a user quite easily, further work on robustly segmenting these characters would be useful in improving the overall success rate.
%For example, both our segmentation approaches fail to appropriately handle characters that are drawn with limbs touching non-connected body parts.
%To be robust to these types of drawings, a parts-based segmentation approach should be used.


%When performing pose detection upon the character's bounding boxes, performance increases with the addition of more training samples. 
%While 10 training samples only results in usable predictions 2.1\% of the time, 100 training samples increases this number to 54.1\%. 
%We continue to see performance improve as this is increased to 2500 training samples, with a successful prediction rate of 90.2\%.
%Examples of failures that persisted, even with 2500 training samples, can be seen in Figure \ref{ref:pose_failures}.


%\begin{figure}
%\includegraphics[width=\linewidth]{images/iter_vs_segmap.png}
%\caption{Effect of training dataset size vs. validation set accuracy for Mask R-CNN segmentation %mask predictions.}
%\label{ref:iter_vs_seg}
%\end{figure}

%\begin{figure}
%\includegraphics[width=\linewidth]{images/pose_epoch_vs_AP.png}
%\caption{Effect of training dataset size vs. validation set accuracy for pose detection model %predictions.}
%\label{ref:iter_vs_pose}
%\end{figure}


%\begin{figure}
%\includegraphics[width=\linewidth]{images/highest_bb_acc_vs_size.png}
%\caption{Highest level of validation set bounding box mAP achieved as training dataset size is increased.}
%\label{ref:bb_acc_vs_size}
%\end{figure}

%\begin{figure}
%\includegraphics[width=\linewidth]{images/highest_seg_acc_vs_size.png}
%\caption{Highest level of validation set segmentation mAP achieved as training dataset size is increased.}
%\label{ref:seg_acc_vs_size}
%\end{figure}

%\begin{figure}
%\includegraphics[width=\linewidth]{images/highest_pose_acc_vs_size.png}
%\caption{Highest level of validation set pose mAP achieved as training dataset size is increased.}
%\label{ref:pose_acc_vs_size}
%\end{figure}


%\begin{figure*}
%\includegraphics[width=\linewidth]{images/bad_detections.png}
%\caption{Examples of detection prediction failures.}
%\label{ref:det_failures}
%\end{figure*}

%\begin{figure*}
%\includegraphics[width=\linewidth]{images/bad_segmentations.png}
%\caption{Examples of segmentation predictions failures.}
%\label{ref:seg_failures}
%\end{figure*}

%\begin{figure*}
%\includegraphics[width=\linewidth]{images/pose_failures.png}
%\caption{Examples of pose predictions failures.}
%\label{ref:pose_failures}
%\end{figure*}





\subsection{Twisted Perspective Animation Retargeting}
\label{sec:twisted_perspective_perceptual_study}
We evaluate our use of twisted perspective retargeting through a perceptual user study on Amazon Mechanical Turk with 66 subjects.
Subjects were shown a set of 20 videos: four figures that were successfully detected, segmented, and rigged by our system, each performing five different motions (see top of Table~\ref{perceptual_study_results_table}).
Within each video were two side-by-side animations: one animation had been created with twisted perspective, by projecting the lower body and upper body onto different planes, while the other animation used only a single plane of projection.
The side upon which the twisted perspective condition appeared was randomized. 
Both animations played simultaneously, and viewers were asked to select, in a forced-choice manner, the animation whose character motion was `more appealing.' 
To ensure subjects paid attention, four `filter' questions were embedded in the stimuli, in which worker's were explicitly directed to select either the left or the right animation.

We present the results in Table~\ref{perceptual_study_results_table}.
For each character and each motion type, we report the percentage of viewers who preferred the animation with twisted perspective motion retargeting over a single perspective.
In parentheses we report significance as the result of a binomial test comparing the distribution of responses to random chance.

In 16 of the 20 videos, a significant preference for twisted perspective was observed. 
In the remaining four videos, there was no significant preference for either type.
Taken together, this shows that, for these character and motion combinations, twisted perspective retargeting often results in more preferable animation. 
\hjs{Interestingly, three of the four videos in which users had no significant preference depicted figures performing the 'Wave Hello' motion.
As can been in the supplemental video, there is significantly less bending of the legs in the 'Wave Hello' motion relative to the other motions tested; as a result, twisted perspective retargeting and single perspective retargeting result in more similar character poses.
This observation suggests that twisted perspective retargeting may not be necessary in all situations; rather it is more useful when both the arms and the legs have substantial motion in different planes.
}

\begin{table}[h]
  \centering
  \includegraphics[width=\linewidth]{images/perceptual_study_results.png}
  \caption{The results of our perceptual study on the use of twisted perspective when retargeting motion. For each character and motion type, we show the percentage of viewers who preferred twisted perspective retargeting and the p-value indicating difference from random chance.}
  \Description{Descr.}
  \label{perceptual_study_results_table}
  
\end{table}



\section{Amateur Drawing Dataset}
\documentclass[../main.tex]{subfiles}

\begin{document}

% Add dataset in this file.
% Including dataset collection, clean, selection, dataset format...
\subsection{Collection}
To better demonstrate the capability of \MODEL\ model to efficiently and independently learn from multiple domains, we collect datasets in 40 domains, with a large amount of data in four major domains: Chinese, English, Bilingual (Chinese and English) and code. The remaining domains with smaller portion consists of 26 other monolingual natural languages, 6 programming languages, and textual data from finance, health, law, and poetry domains, respectively. 

For Chinese texts, we collect the WuDaoCorpora 2.0~\cite{Yuan2021WuDaoCorporaAS} which contains 200GB and the CLUECorpus2020~\cite{Xu2020CLUECorpus2020AL} which contains 100GB. For English texts, the Pile dataset~\cite{Gao2021ThePA} which contains 800GB and C4 dataset~\cite{2020Exploring} which contains 750GB were collected. For code, we use the Python code (147GB) which has been used in PanGu-Coder~\cite{Christopoulou2022PanGuCoderPS}, as well as the Java code (161GB) from GHTorrent~\cite{Gousios2013TheGD} , which are then filtered by file size ($<$1MB), average number of characters per line ($<$200), maximum number of characters per line ($<$1000) and their compilablity. Then, these collected English, Chinese and code texts data was sampled and distributed to the four major domains. Finally, we get more than 300B tokens for the four major domains. The detailed statistics of data distribution and data sources in four major domains are presented in Table~\ref{tab:data_dis}.

For the remaining 36 domains, the data for 26 monolingual domains are mainly from CCAligned~\cite{ElKishky2019AMC} and CCMatrix~\cite{Schwenk2019CCMatrixMB}. Similar to the code domain mentioned above, the data for 6 programming language domains are collected through GHTorrent~\cite{Gousios2013TheGD} and filtered in the similar way. Finance domain data is filtered from the WuDaoCorpora 2.0~\cite{Yuan2021WuDaoCorporaAS} using the tags. Health domain data is from Chinese MedDialog Dataset ~\cite{Zeng2020MedDialogLM}. Law domain data is sampled from CAIL2018~\cite{Xiao2018CAIL2018AL}. Poetry domain dataset is from Werneror-Poetery~\footnote{\url{https://github.com/Werneror/Poetry}}. Finally, we sampled more than 25B tokens for the 36 domains.


\begin{table}
\centering
\caption{Data distribution and data sources in four main domains}
\label{tab:data_dis}
\scalebox{0.8}{
\begin{tabular}{cccc} 
\hline
Domain ID & Domain                                                                & Tokens (Billion)                                                             & Data source                                                                                                    \\ 
\hline
0         & \begin{tabular}[c]{@{}c@{}}\textbf{Bilingual}\\(Chinese, English)\end{tabular}      & \begin{tabular}[c]{@{}c@{}}77.51 B\\Chinese (38.75) + English(38.76B)\end{tabular} & CLUECorpus2020 , C4                                                                                            \\ 
\hline
1         & \textbf{\textbf{Chinese}}                                             & 75.47 B                                                                            & WuDaoCorpora 2.0~ ~                                                                                            \\ 
\hline
2         & \textbf{\textbf{English}}                                             & 75.90 B                                                                            & Pile , C4                                                                                                      \\ 
\hline
3         & \begin{tabular}[c]{@{}c@{}}\textbf{Code}\\(Python, Java)\end{tabular} & \begin{tabular}[c]{@{}c@{}}75.24 B\\Python (50.24B) + Java (25B)~~~~~\end{tabular} & \begin{tabular}[c]{@{}c@{}}Python (PanGu-Coder)~\\Java (GHTorrent)\end{tabular}  \\ 
\hline
\end{tabular}
}
\end{table}

\subsection{Format}
For the four major domains, each can be adapted to different downstream tasks. In order to better support domain-specific downstream tasks, this paper uses different data format for different domains. For Chinese and English domains, the <EOT> token which indicates the end of training text is inserted at the end of each training sample.

\begin{figure*}[!ht]
    \centering
    \includegraphics[width=0.8\textwidth]{sections/fig/zhoupingyi/sample_1.pdf}
    \caption{Data format of Chinese and English domains.
    }
    \label{fig:sample_1}
\end{figure*}

For Bilingual domain, the <EN> or <CN> token is inserted into the head of the training sample according to the source of the training sample (either from the Chinese dataset or the English dataset), and the <EOT> token is inserted at the end of each training sample.

\begin{figure*}[!ht]
    \centering
    \includegraphics[width=0.8\textwidth]{sections/fig/zhoupingyi/sample_2.pdf}
    \caption{Data format of Bilingual domain.
    }
    \label{fig:sample_2}
\end{figure*}

For the code domain, the <Python> or <Java> token is inserted into the head of the training sample based on the programming language type of the training sample, and the <EOT> token is inserted at the end of each training sample.

For the remaining 36 domains, the data formats of 26 monolingual domains, finance, health, law, and poetry domains are the same as the Chinese and English domains, and the data format of 6 programming language domains is the same as the code domain.

\begin{figure*}[!ht]
    \centering
    \includegraphics[width=0.8\textwidth]{sections/fig/zhoupingyi/sample_3.pdf}
    \caption{Data format of Code domain.
    }
    \label{fig:sample_3}
\end{figure*}

For a formatted data set $D$, suppose it contains n training samples $D=\left \{ s_{1}, s_{2}, \dots,  s_{n} \right \} $. To make full use of the computing power of the Ascend 910 cluster and accelerate training in the pre-training phase, we concatenate all samples in the data set into a sequence, and then intercept training instances in the concatenated sequence according to the fixed length (1024), as shown in Figure~\ref{fig:format_1}. In the fine-tune phase, for each training sample in the formatted dataset, if the length is less than the fixed length, we pad the sample to the fixed length with a special token <Pad>. If the length is greater than the fixed length, the extra part is truncated. Figure~\ref{fig:format_2} shows the process. Different to 
PanGu-$\alpha$ model, each training sample of \MODEL\ model contains two field: input sequence of token IDs which are training instance and their domain ID. The domain ID indicates which domain the training instance belongs to. The RRE layers of the \MODEL\ model decide which experts the training tokens is routed to by the domain ID. 

\begin{figure*}[!ht]
    \centering
    \includegraphics[width=0.7\textwidth]{sections/fig/zhoupingyi/format_1.pdf}
    \caption{Input format during model pre-training.
    }
    \label{fig:format_1}
\end{figure*}

\begin{figure*}[!ht]
    \centering
    \includegraphics[width=0.7\textwidth]{sections/fig/zhoupingyi/format_2.pdf}
    \caption{Input format during model fine-tuning.
    }
    \label{fig:format_2}
\end{figure*}

\end{document}

%\section{}
%\label{sec:resDir}


\section{Conclusion}
\label{sec:conclusion}
% <>
Since its advent in 1931, Koopman operator theory \cite{koopman:1931} has only recently been actively utilized for solving practical problems, thanks to the introduction of the DMD algorithm in 2008 \cite{schmid:2008}. Since then, a multitude of DMD algorithm variations have risen to prominence and found utility across various fields. A notable feature of our survey paper was reviewing and categorizing the results of over 100 research papers based on both application and algorithm type in smart mobility and vehicle engineering  (see Table~\ref{tab1} and Section~\ref{sec:vehicApp}).  Additionally, this survey paper identified potential research gaps in smart mobility and vehicular engineering applications (Remarks~\ref{remGap1}--\ref{remGap6}). Finally, this review paper discussed theoretical aspects of Koopman operator theory that have been largely neglected by the smart mobility and vehicle engineering community and yet have large potential for contributing to solving open problems in these areas (see Section~\ref{subsec:theorIssue}).

\noindent{\textbf{Future Research Directions.}}	Given the emergence of cyber-threats against connected and autonomous vehicles as well as robotic systems (see, e.g.,~\cite{nekouei2021randomized,mohammadi2022generation}), a future research direction might include utilizing Koopman operator-based algorithms for designing cyber-resilient vehicular and smart mobility applications (see, e.g.,~\cite{taheri2022data} for a related line of research). Another potential research direction is using Koopman operator-based algorithms for predicting the motion of vulnerable road users (VRUs), e.g., pedestrians and cyclists (see, e.g.,~\cite{pool2019context,scholler2020constant}). Finally, rehabilitation robotics and robotic exoskeletons can be the benefactors of the predictive capabilities of Koopman operator-based algorithms for detecting tripping events and/or system  identification in various modes of locomotion (see, e.g.,~\cite{kumar2019extremum,aprigliano2019pre}).



%Fig. 1 depicts the accumulation of such algorithms since 2014, which are particular to vehicle engineering and smart mobility, i.e., the focus of this review. Table 1 summarizes the varieties of relevant algorithms developed in those studies. Furthermore, we have highlighted theoretical issues, whose expansion will have potential applications to the wide research area of smart mobility and vehicle engineering.  

%Although fairly comprehensive, we have found several gaps in this research area. In particular, we could not find any studies related to elevators, robots/vehicles employing crawling, slithering, hopping or peristaltic locomotion, arctic or special-terrain vehicles such as those employing screws or tracks, hovercraft and other amphibious vehicles or subsystems which tolerate flexible environments, classification or guidance systems related to vehicles for drilling or agriculture, or for current-ripple, power-split, battery health monitoring, nuclear propulsion, exoskeletons/prosthetics, personal mobility, motorsports, specialized rovers or similar open problems in emerging areas.  These examples are, of course, not exhaustive.  
%
%The purely data-driven nature of Koopman operators holds the promise of capturing unknown and complex dynamics for reduced-order model generation and system identification, through which the rich machinery of linear control techniques can be utilized. The emergent nature of the smart mobility and vehicular-related applications, where  the Koopman operator  in each particular application needs to be approximated, implies that the development of various Koopman operator approximation  algorithms is expected to grow along with the vehicular problems they aim to solve.  Given the ongoing development of this research area and the many existing open problems in the fields of smart mobility and vehicle engineering, a survey of techniques and open challenges of applying Koopman operator theory to this vibrant area is warranted.  To the best of our knowledge, this survey paper is the \emph{first of its kind} reviewing the applications of Koopman operator theory within a focused research area, namely, smart mobility and vehicle engineering applications. A \emph{notable feature} of our survey paper is reviewing and categorizing the results of over 100 research papers based on both application and algorithm type  (see Tables~\ref{tab1}--~\ref{tab4} and Section~\ref{sec:vehicApp}) that are concerned with the applications of Koopman operator theory to the field of smart mobility and vehicular engineering. Such a \emph{comprehensive and  detailed categorization} will be beneficial to the research practitioners working in the field.  Furthermore, this review paper discusses theoretical aspects of Koopman operator theory that have been largely neglected by the smart mobility and vehicle engineering community and yet have large potential for contributing to solving open problems in these areas. Additionally, our survey paper seeks to \emph{identify gaps} in the smart mobility and vehicle engineering research where new and existing Koopman operator-based methods have the potential to further develop and address unsolved problems  potentially benefiting from the perspectives of nonlinear system identification, control, global linearization, and the predictive powers that Koopman operator theory has to offer (see, e.g., Remarks~\ref{remGap1}--\ref{remGap6}). 


\bibliographystyle{ACM-Reference-Format}
\bibliography{Ref}


\end{document}
\endinput
