As part of the Animated Drawing Demo, users were asked to consent to a data usage agreement, allowing their uploaded image and annotations to be used for research purposes, including release as part of a public dataset.
Consenting was optional, and refusal to do so did not restrict the experience in any way.
\hjs{Images collected prior to April 20th, 2022 were considered for inclusion into the Amateur Drawings Dataset.
By that date, site users had uploaded over 3.5 million images and consented to the data usage agreement for 1.7 million images.}

\subsection{Refinement}
Many of the images uploaded to the site were photographs of actual people, pets, anime characters, brand logos, and other out-of-domain content.
Therefore, submitted images needed to be filtered to ensure they contained amateur drawings.
This refinement was performed in two steps. 
First, a self-supervised clustering approach was used to identify and filter out-of-domain images.
Second, the remaining images were manually reviewed to ensure their suitability.
\subsubsection{Cluster-based Filtering}
\label{sec:cluster_filtering}
A self-supervised approach \cite{chen2020improved} was used to train a ResNet-50 feature extractor specific to the consent images.
The feature extractor took the image contents of the figure bounding box and projected it onto a 2048-dimensional embedding space.
Within this space, k-means was used to cluster the embeddings into 100 separate clusters.
From visual inspection, 68 clusters contained out-of-domain subjects, while the remaining 32 clusters primarily contained images of amateur, hand-drawn characters, suitable for inclusion (see Figure \ref{fig:clustering_examples}). 

\begin{figure}[ht]
  \centering
  \includegraphics[width=\linewidth]{images/Example_Clusters2.png}
  \caption{Example images from the clusters obtained via cluster-based filtering. Certain clusters contained similarly depicted characters, such as stick figures, hollow characters, and solid marker characters (retained clusters 1, 2, and 3, respectively). Other clusters contained out-of-domain images, such as anime faces or anime full-body characters (discarded clusters 4, 5 respectively).}
  \Description{fig:clustering_examples}
  \label{fig:clustering_examples}
  
\end{figure}

\begin{figure}[ht]
  \centering
  \includegraphics[width=\linewidth]{images/distance.png}
  \caption{Two input images and their six nearest neighbors within the learned embedding space. As these examples show, duplicates or near-duplicates are quite close within the space, which is useful for filtering them. Similar but distinct figures are also close together within the embedding space; the top figure is close to others with long hair and dresses, while the bottom figure is close to other heart-shaped tadpole figures.}
  \label{fig:embedding_distance}
\end{figure}

Within those 32 clusters were many near-duplicates, images of the same drawing taken from slightly different angles or under slightly different lighting conditions.
Such near-duplicates are close together in the learned embedding space  (see Figure \ref{fig:embedding_distance}).
We detected near-duplicates by computing the Euclidean distance between each pair of images in the embedding space, and removing one of the images if this distance was less than 0.5, a value empirically selected by the authors.
After filtering out-of-domain clusters and removing duplicates, 471,405 images remained.

\subsubsection{Manual Review}
An agency was contracted to review 283,146 of the remaining images.
Reviewers were instructed to ensure images were free-hand, physical drawings containing at least one full-bodied human figure, did not contain characters that are protected intellectual property (such as Mickey Mouse\textsuperscript{TM}), and contained no personally identifiable information or vulgar content. 
Because the reviewers were primarily English speakers, images that contained non-English words were excluded on the basis they might contain inappropriate content.
After manual review, 178,166 images remained.

\hjs{Of the images that were excluded, 
30\% were not freehand drawings, 
24\% did not contain full-bodied human figures, 
20\% contained personally identifiable information, 
15\% contained protected intellectual property, 
4\% were uploaded and annotated with an incorrect orientation,
4\% had out of domain content, 
and 3\% contained vulgar content.}

\subsection{Release}
We are pleased to provide the retained images, along with their annotations, for use by the research community. 
While the \AD Demo was specifically designed for use with children's drawings, the artists' ages were not recorded. 
We therefore refer to the dataset as the Amateur Drawings Dataset.

While the dataset includes the user-accepted character bounding boxes, segmentation masks, and joint positions, we have not attempted to guarantee the accuracy of these annotations.
From a random sampling of 5,000 dataset images and annotations, we observed that
35\% of bounding box detections were modified, 20\% of masks were modified, and  29\% of joint skeletons were modified.
By visual spot check, we confirmed that, in the vast majority of cases, these modifications improved the quality of the annotations.
