\documentclass[%
 aip,
% jmp,
% bmf,
% sd,
% rsi,
 amsmath,amssymb,
preprint,%
%reprint,%
%author-year,%
%author-numerical,%
% Conference Proceedings
floatfix
]{revtex4-1}

\usepackage{graphicx}% Include figure files
\usepackage{dcolumn}% Align table columns on decimal point
\usepackage{bm}% bold math
%\usepackage[mathlines]{lineno}% Enable numbering of text and display math
%\linenumbers\relax % Commence numbering lines

\usepackage[utf8]{inputenc}
\usepackage[T1]{fontenc}
\usepackage{mathptmx}
\usepackage[font=footnotesize]{caption}
\renewcommand{\figurename}{FIG. S}
\renewcommand{\tablename}{TABLE S}


\begin{document}

\preprint{}



\title[\textbf{Suppl. material} - Strain effects on magnetic compensation and spin reorientation transition of ...]{Supplementary material -  Strain effects on magnetic compensation and spin reorientation transition of Co/Gd synthetic ferrimagnets}
% Force line breaks with \\

\date{\today}% It is always \today, today,

\maketitle


\subsection*{\textbf{S1} - Magnetostatics model for the Spin Reorientation Transition}

In the expression of the free energy, a term describing the effect of intermixing at the multilayer system interfaces is added. The expression is 

 \begin{equation} \tag{S.1}\label{demag_mix}
 \begin{array}{l}
E_{mix}=\frac{1}{2}\mu_0 \int_{0}^{a_0x} M_{Co}^2+ \left( M_{Gd}exp(-q/\lambda_{Gd}) \right)^2 \,dq= \\ \frac{1}{2}\mu_0 a_0 M_{Co}^2 x  + \frac{1}{4}\mu_0 \lambda_{Gd} M_{Gd}^2 \left(1-\exp\left(\frac{-2a_0x}{\lambda_{Gd}}\right)\right).
\end{array}
\end{equation}

  Therefore, the total energy $E_{tot}= -E_K - E_{mix} - K_{ME}+E_{d,Co}+E_{d,Gd}$, including all the terms will be:

 \begin{equation} \tag{S.2} \label{total_mix}
 \begin{array}{l}
E_{tot}=-K_s + \Delta K \left(1-\exp\left(\frac{-2x}{\lambda_{K}}\right)\right) -K_{ME} - \frac{1}{2}\mu_0 a_0 M_{Co}^2 x   \\ -\frac{1}{4}\mu_0 \lambda_{Gd} M_{Gd}^2 \left(1-\exp\left(\frac{-2a_0x}{\lambda_{Gd}}\right)\right)+ \frac{1}{2}\mu_0 M_{Co}^2 y  \\ + \frac{1}{4}\mu_0 M_{Gd}^2 \lambda_{Gd} \left(1-\exp\left(\frac{-2x}{\lambda_{Gd}}\right)\right).
\end{array}
\end{equation}

Here  $x$ and $y$ are, respectively, the Gd and Co thicknesses in the phase diagram of Fig. S\ref{fig_S02}. The value of the parameters used in our model are listed in Table S \ref{tab_material} .


\begin{table}[h!]
    \centering
    \begin{tabular}{||c c c||} 
 \hline
    Parameter & Value  & Description \\ [0.5ex] 
 \hline\hline
 $K_s$ & 1.7 mJ/m$^2$  &	Interfacial anisotropy (from exp.) \\
  \hline
$K_{ME}$ & 0.02 mJ/m$^2$  &	Magnetoelastic anisotropy (from exp.) \\
  \hline

 $M_{Co}$ &  1.3 MA/m &	Cobalt magnetization (from exp.) \\ 
  \hline
  $M_{Gd}$ & 1.4 MA/m & Gadolinium magnetization at Co/Gd interface (from Ref. \cite{kools})  \\ 
 \hline
  $a_{0}$ & 0.13 (-) & Growth parameter of intermixing region (from exp.)  \\ 
 \hline
 $\lambda_K$ & 0.51(15) nm &	Change of PMA energy characteristic decay length (Fit parameter) \\
  \hline

 $\lambda_{Gd}$ &  0.59(22) nm &	Gd magnetization decay characteristic decay length (Fit parameter) \\ 
  \hline
  $\Delta_K$ & 3.96(41)$\times10^{-4}$ J/m$^2$ &	Change of PMA energy (Fit parameter)  \\ 

 \hline
\end{tabular}
\caption{Parameters used in the model for the magnetostatics of uncompensated Co/Gd synthetic ferrimagnets used for the calculations of the SRT. The term $K_{ME}$ is considered zero for when external strain is not applied to the sample. }
\label{tab_material}
\end{table}


The values of $\lambda_K$, $\lambda_{Gd}$ and $\Delta_K$ are instead determined using a numerical fit and are reported in Table S \ref{tab_material}. To fit this equation to the phase diagram obtained experimentally, it is convenient to find the Co-thickness (y) where the anisotropy energy ($E_{tot}$) is equal to zero (spin reorientation transition, SRT). Solving Eq. \ref{total_mix} for y gives:


 \begin{equation} \tag{S.3} \label{total_fit}
 \begin{array}{l}
y_0(x)= \frac{2}{M_{Co}^2 \mu_0}\left(-\left(K_s - \Delta K \left(1-\exp\left(\frac{-x}{\lambda_{K}}\right)\right)\right)\right) - \frac{1}{2}\mu_0 a_0 M_{Co}^2 x \\ -\frac{1}{4}\mu_0 \lambda_{Gd} M_{Gd}^2 \left(1-\exp\left(\frac{-2a_0x}{\lambda_{Gd}}\right)\right) + \frac{1}{4}\mu_0 M_{Gd}^2 \lambda_{Gd} \left(1-\exp\left(\frac{-2x}{\lambda_{Gd}}\right)\right).
\end{array}
\end{equation}

Note that for determining the fit parameters, the measurements were taken without externally applied strain. Accordingly, the magnetoelastic energy term $K_{ME}$ is set to zero in Eq. \ref{total_fit}. The experimental data used for the numerical fit are reported in Fig. S\ref{fig_S02}. The sample used for the numerical fit, explores a wide thickness range (Gd from 0 to 6 nm) in a double wedge fashion to improve accuracy. Consequently, the  dimensions of this sample exceed the 1x1 cm size of the bending device. For this reason, single wedge samples have been deposited for the strain-dependent study. 

 \begin{figure}[h!]
\centering\includegraphics[width=12cm]{figures/S02.PNG}
\caption{\label{fig_S02} Values for the SRT obtained experimentally on a $Ta(4nm)/Pt(4)/$ $Co(t_{Co})/Gd(t_{Gd})/TaN(4)$ sample and used to extract the fitting parameters in Eq. \ref{total_fit}.}
\end{figure}

The feature around $t_{Gd}$=2 nm in Fig. S\ref{fig_S02} is not captured by out toy model, and might be due to the additional intermixing caused during sputtering, not included in Eq. \ref{total_mix}. 

 %\begin{figure}[h!]
%\centering\includegraphics[width=16cm]{figures/S02.PNG}
%\caption{\label{fig_S02} (a) XRD angular scan of the Ni/Fe multilayer sample after sputtering. In the inset:  Fe 110/Ni 111 peak of the multilayer as-deposited, annealed and after irradiation. (b) X-Ray reflectometry (XRR) measurement for a multilayer of $[Ni(2$ $nm)/Fe(2$ $nm)]\times 8$ irradiated with different He$^+$ fluences. The changes in the curves indicate increasing intermixing at the interfaces of our multilayer with increasing ion fluences. (c),  (d) and (e) ToF-SIMS measurements for multilayer as-deposited, after irradiation and thermal annealing respectively. }
%\end{figure}
 

\subsection*{ \textbf{S2} -Application of strain}


To obtain information about the magnetoelastic properties of the material, the substrate was bent mechanically with a 3 point bending sample holder, as shown schematically in Fig. S\ref{fig_S01} (a). A square sample of 1 by 1 cm is vertically constrained on two sides and pushed uniformly from below by a cylinder that has an off-centered rotation axis. The device generates a tensile strain in the plane of the sample up to $0.1$ $\%$ when the cylinder is rotated by 90$^\circ$. The strain is mostly uniaxial and has been measured with a strain gauge on the substrate surface. 

 \begin{figure}[h!]
\centering\includegraphics[width=13cm]{figures/S01.PNG}
\caption{\label{fig_S01} (a) schematic of the three point bending method used to externally strain the sample. The strain is mostly uniaxial along the $x$ direction. (b) hysteresis loops measured before (blue) and during (red) application of in-plane strain for a sample of Pt/Co(1.85 nm)/Ta. The area highlighted in red corresponds to the magnetoelastic energy in the strained system. The magnetic field was applied along the OOP direction ($z$).}
\end{figure}



Magnetic hysteresis loops are recorded before and after the application of the tensile strain and are used to estimate the magnetoelastic anisotropy. As  previously reported\cite{PMA,spinvalve} the magnetic  anisotropy $K_{eff}$ is linked to the energy stored in the magnetization curves. For example the PMA energy is given by the area enclosed between the magnetic loops measured with field along IP and OOP direction.  If then the strain in the film is non-zero, the magneto-elastic coupling contributes in principle to the effective anisotropy. Two hysteresis loops measurements, before and after the application of strain, are sufficient to estimate $K_{ME}$. Indeed the total anisotropy of the system is $K_{eff}=K_{s}$ and $K_{eff}=K_{s}+K_{ME}$ before and after the application of strain, respectively.  The magnetoelastic anisotropy $K_{ME}=-\frac{3}{2}\lambda_s Y \epsilon$ is linked to reversible part of the hysteresis loops (close to the saturation) according to 


 \begin{equation} \tag{S.4} \label{eq_strain_eanis}
K_{ME}=M_s \Delta E=-\frac{3}{2}\lambda_s Y \epsilon
\end{equation}

where $\Delta E$ is the anisotropy energy measured by the difference in area below the strained and unstrained curves, $\epsilon$ is the strain $\lambda_s$ is the magnetostriction and Y is the Young's mudulus of the material. In our case $\epsilon=0.1\%$ and Y=200 GPa. $\Delta E$ corresponds to the reversible part, i.e. the red marked area in Fig. S\ref{fig_S01} (b). The value of magnetoelastic anisotropy was calculated using the value of saturation magnetization ($M_s$) of the stack taken from literature and reported in Table S \ref{tab_material}.



\newpage


%%%%%%%%%% If preparing manually:
\begin{thebibliography}{1}
\newcommand{\enquote}[1]{``#1''}



\bibitem{kools} Kools, T. J., van Gurp, M. C., Koopmans, B., and Lavrijsen, R. (2022). Magnetostatics of room temperature compensated Co/Gd/Co/Gd-based synthetic ferrimagnets. Applied Physics Letters, 121(24), 242405.

\bibitem{PMA}Johnson, M. T., Bloemen, P. J. H., Den Broeder, F. J. A., and De Vries, J. J. (1996). Magnetic anisotropy in metallic multilayers. Reports on Progress in Physics, 59(11), 1409.

\bibitem{spinvalve} Baril, L., Gurney, B., Wilhoit, D.,  Speriosu, V. (1999). Magnetostriction in spin valves. Journal of Applied Physics, 85(8), 5139-5141.






 \end{thebibliography}


\end{document}


