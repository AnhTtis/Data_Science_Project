\documentclass[11pt]{article}
\usepackage{caption}
\usepackage{rotating}
\usepackage{float}
\usepackage{times}
\usepackage{amssymb}
\usepackage{amsmath}
\usepackage{ulem}
\usepackage{graphicx, tabularx,color}
%\usepackage[all]{xy}
\usepackage{url}
\usepackage{cite}
\usepackage{hyperref}

% adjust the margins

\oddsidemargin 0.25in
\textwidth 5.75in
\topmargin -0.5in
\textheight 9.0in
%\columnsep 0.25in

%% Define a macro for inserting postscript images
%% ==============================================
%% This is a macro which nominally takes 3 parameters, 
%% it would be used as follows to insert and encapsulated postscript
%% image at the location where it is used.
%%
%% \EPSFIG{epsfilename}{caption}{label}
%% - epsfilename is the name of the encapsulated postscript file to be
%%               inserted at this location
%% - caption is the text to be shown as the figure caption, it will be
%%           prepended by Figure X.  The number X can be referenced
%%           using the label parameter.
%% - label is a name given to the figure, it can be referenced using the
%%         \ref{label} command.
%% \being{figure}[H] - place graphs here




\def\EPSFIG[#1]#2#3#4{
\begin{figure}
  \begin{center}
    \includegraphics[clip,#1]{#2}
     \end{center}
  \caption{#3}
  \label{#4}
\end{figure}
}

% user defined commands

\newcommand{\avg}[1]{\left \langle #1 \right \rangle }

% set paths

\graphicspath{{fig/}}
\DeclareGraphicsExtensions{.eps}


\DeclareCaptionLabelFormat{myformat}{#1~#2}
\captionsetup{labelformat=myformat}
    
    % Define fields to be displayed by /maketitle
\author{Thomas Dufils$^\dagger$, Lisanne Knijff$^\dagger$, Yunqi Shao$^\dagger$ Chao Zhang$^\dagger$$^*$\\\\
  \small $^\dagger$\it{Department of Chemistry - \AA ngstr\"om Laboratory, Uppsala University, L\"agerhyddsv\"agen 1,}\\
  \small \it{BOX 538, 75121, Uppsala, Sweden} \\\\
  chao.zhang@kemi.uu.se}
\title{PiNNwall: heterogeneous electrode models from integrating machine learning and atomistic simulation}

\date{}
%Supporting Information

\begin{document}
%\twocolumn
\maketitle
%thispagestyle{empty}

\makeatletter 
\renewcommand{\thefigure}{S\@arabic\c@figure} \renewcommand{\thetable}{S\@arabic\c@table}
\renewcommand{\theequation}{S\@arabic\c@equation} 
\renewcommand{\thepage}{S\arabic{page}}

\makeatother

\appendix{} 
\setcounter{figure}{0} 
\setcounter{table}{0}


\centerline{\bf\Large{Supplementary Material}}

\newpage 

\section*{Validation and implementation of base charges predicted from PiNet-dipole}
To validate the charges predicted from the PiNet-dipole model, molecular analogues of the target structures were used for each of the functionalized graphene models. To validate that the charges predicted from PiNet-dipole have a physical basis, comparisons were made to charges computed using several population analysis techniques.

To perform the population analysis, DFT calculations were run using Gaussian09 ~\cite{gaussian}. The B3LYP~\cite{becke1992density,stephens1994ab} functional and the cc-pVDZ basis set \cite{dunning1989gaussian} were used. The population analyses that were performed are: CM5 \cite{marenich2012charge}, Mulliken~\cite{mulliken1955electronic}, Hirshfeld~\cite{hirshfeld1977bonded} and Merz-Singh-Kollman (MSK)~\cite{Singh84, besler1990atomic}. The molecular analogues were deemed fit as a references if the predicted charges corresponded to chemical intuition and the dipole moment was comparable to that calculated using DFT. 

For the graphene sheet doped with nitrogen, a planar form of trimethylamine was used as a reference.

\begin{figure}[H]
\begin{center}
\includegraphics[width=\linewidth]{FIG_S1.png}
\end{center}
\caption{\textbf{Charges of trimethylamine}. The structure of trimethylamine (a). Computed with b) PiNet-dipole, c) CM5, d) Mulliken, e) Hirshfeld, and f) MSK.   \label{trimethylamine_charges}}
\end{figure}


\begin{table}[H]
\begin{center}
\begin{tabular}{lrrr}
\hline
Method & D$_{x}$ (D) & D$_{y}$ (D) & D$_{z}$ (D)\\
\hline 
DFT & -0.340 & -0.286 & -0.010 \\
PiNet-dipole & -0.435 & -0.169 & -0.016 \\
CM5 & -0.126 & -0.123 & -0.013 \\
MSK & -0.320 & -0.280 & -0.009 \\
\hline
\end{tabular}
\caption{\textbf{Dipole moment of trimethylamine}.}
\end{center}
\end{table}

\begin{table}[H]
\begin{center}
\begin{tabular}{lr}
\hline
Element & PiNet charge (e)\\
\hline 
N & -0.19527 \\
C & 0.03033 \\
C & 0.08597 \\
C & 0.07899 \\
\hline
\end{tabular}
\caption{\textbf{Base charges from planar trimethylamine as implemented in the N-doped graphene}.}
\end{center}
\end{table}


The charges of the methyl groups are placed the carbon atoms when charges are used in the real system, to ensure a charge neutral entity. 

For the graphene sheet doped with epoxy groups, ethylene oxide was used as the reference molecule.

\begin{figure}[H]
\begin{center}
\includegraphics[width=\linewidth]{FIG_S2.png}
\end{center}
\caption{\textbf{Charges of ethylene oxide}. The structure of ethylene oxide (a). Computed with b) PiNet-dipole, c) CM5, d) Mulliken, e) Hirshfeld, and f) MSK.   \label{ethylene_oxide_charges}}
\end{figure}

\begin{table}[H]
\begin{center}
\begin{tabular}{lrrr}
\hline
Method & D$_{x}$ (D) & D$_{y}$ (D) & D$_{z}$ (D)\\
\hline 
DFT & 1.806 & 0.000 & -0.141 \\
PiNet-dipole & 1.811 & 0.000 & -0.142 \\
CM5 & 2.052 & 0.000 & -0.161 \\
MSK & 1.837 & 0.000 & -0.144 \\
\hline
\end{tabular}
\caption{\textbf{Dipole moment of ethylene oxide}.}
\end{center}
\end{table}

\begin{table}[H]
\begin{center}
\begin{tabular}{lr}
\hline
Element & PiNet charge (e)\\
\hline 
O & -0.19297 \\
C & 0.09648 \\
C & 0.09648 \\
\hline
\end{tabular}
\caption{\textbf{Based charges from ethylene oxide as implemented in the epoxy-terminated graphene oxide}.}
\end{center}
\end{table}

Here, the charges of the hydrogen atoms are also combined with that of the carbon atoms when the charges are transferred to functionalized graphene. Once again, to ensure charge neutrality and to localize the charges on the graphene sheet. 

For the graphene sheet doped with hydroxyl groups, methanol was used as the molecular analogue.

\begin{figure}[H]
\begin{center}
\includegraphics[width=\linewidth]{FIG_S3.png}
\end{center}
\caption{\textbf{Charges of methanol}. The structure of methanol (a). Computed with b) PiNet-dipole, c) CM5, d) Mulliken, e) Hirshfeld, and f) MSK.   \label{methanol_charges}}
\end{figure}

\begin{table}[H]
\begin{center}
\begin{tabular}{lrrr}
\hline
Method & D$_{x}$ (D) & D$_{y}$ (D) & D$_{z}$ (D)\\
\hline 
DFT & 1.340 & 0.825 & 0.000 \\
PiNet-dipole & 1.327 & 1.067 & 0.000 \\
CM5 & 1.384 & 0.957 & 0.000 \\
MSK & 1.326 & 0.851 & 0.000 \\
\hline
\end{tabular}
\caption{\textbf{Dipole moment of methanol}.}
\end{center}
\end{table}


\begin{table}[H]
\begin{center}
\begin{tabular}{lr}
\hline
Element & PiNet charge (e)\\
\hline 
O & -0.41668 \\
H & 0.32197 \\
C & 0.09471 \\
\hline
\end{tabular}
\caption{\textbf{Base charges from methanol as implemented in the hydroxyl-terminated graphene oxide}.}
\end{center}
\end{table}

The charge on the carbon atom is set so that it includes the charges of the hydrogen atoms as well. In this way, it compensates for the charge on the hydroxyl group, and the whole group is charge neutral.

Finally, for the graphene sheet functionalized with carboxyl groups, a smaller graphene flake with carboxyl groups was used as a reference. This was done because alternative analogues showed large fluctuations in the charges when changing the charge state of the analogue which did not correspond to chemical intuition. While the dipole moment for the carboxyl flake shows discrepancies to that of DFT in the $x$- and $y$-direction, the $z$-direction, which is the most important direction for when it comes to the carboxyl group, agrees within a reasonable error margin. 

\begin{figure}[H]
\begin{center}
\includegraphics[width=\linewidth]{FIG_S4.png}
\end{center}
\caption{\textbf{Charges of the neutral carboxyl flake}. The structure of carboxyl protonated flake (a). Computed with b) PiNet-dipole, c) CM5, d) Mulliken, e) Hirshfeld, and f) MSK.   \label{carboxyl_neutral_charges}}
\end{figure}

\begin{table}[H]
\begin{center}
\begin{tabular}{lr}
\hline
Element & PiNet charge (e)\\
\hline 
O$_{\mathrm{double-bonded \ O}}$ & -0.28245 \\
O$_{\mathrm{OH}}$ & -0.31622 \\
H$_{\mathrm{OH}}$ & 0.32579 \\
C & 0.27288 \\
\hline
\end{tabular}
\caption{\textbf{Charges from the neutral carboxyl flake}.}
\end{center}
\end{table}

\begin{table}[H]
\begin{center}
\begin{tabular}{lrrr}
\hline
Method & D$_{x}$ (D) & D$_{y}$ (D) & D$_{z}$ (D)\\
\hline 
DFT & 0.2953 & 0.0873 & 0.3845 \\
PiNet-dipole & 3.301 & 1.699 & 0.145 \\
CM5 & 0.435 & 0.230 & 0.297 \\
MSK & 0.265 & 0.087 & 0.354 \\
\hline
\end{tabular}
\caption{\textbf{Dipole moment from the neutral carboxyl flake}.}
\end{center}
\end{table}

\begin{figure}[H]
\begin{center}
\includegraphics[width=\linewidth]{FIG_S5.png}
\end{center}
\caption{\textbf{Charges of the  protonated carboxyl flake}. The structure of carboxyl protonated flake (a). Computed with b) PiNet-dipole, c) CM5, d) Mulliken, e) Hirshfeld, and f) MSK.   \label{carboxyl_protonated_charges}}
\end{figure}

\begin{table}[H]
\begin{center}
\begin{tabular}{lr}
\hline
Element & PiNet charge (e)\\
\hline 
O$_{\mathrm{double-bonded \ O}}$ & -0.33339 \\
H$_{\mathrm{double-bonded \ O}}$ & 0.27653 \\
O$_{\mathrm{OH}}$ & -0.36729 \\
H$_{\mathrm{OH}}$ & 0.33747 \\
C & 1.08668 \\
\hline
\end{tabular}
\caption{\textbf{Base charges from the protonated carboxyl flake as implemented in the protonated side of carboxyl-terminated graphene oxide}.}
\end{center}
\end{table}


Since the investigated structures contain neutral, protonated, and deprotonated forms of the carboxyl groups, these are also the structures for which the charges were predicted. Here the charge of the carbon atom is set such that the total charge of the protonated carboxyl group is +1. Once again the charges of the atoms are predicted using PiNet-dipole. Then, the charge of the carbon atom is simply set to ensure that the charge of the protonated carboxyl group sums to +1. This is done to keep the charges localized, and because it is the simplest way to adjust the charge without the need for an arbitrary charge division. It also prevents unphysical modifications to the other charges from being made. This is supported by Figure \ref{carboxyl_protonated_charges}, as this shows that the charge analysis performed with DFT methods the excess charge is also mostly located on the carbon atom. 


\begin{figure}[H]
\begin{center}
\includegraphics[width=\linewidth]{FIG_S6.png}
\end{center}
\caption{\textbf{Charges of the deprotonated carboxyl flake}. The structure of the deprotonated carboxyl flake (a). Computed with b) PiNet-dipole, c) CM5, d) Mulliken, e) Hirshfeld, and f) MSK.   \label{carboxyl_deprotonated_charges}}
\end{figure}

\begin{table}[H]
\begin{center}
\begin{tabular}{lr}
\hline
Element & PiNet charge (e)\\
\hline 
O$_{\mathrm{double-bonded \ O}}$ & -0.28599 \\
O$_{\mathrm{OH}}$ & -0.28723 \\
C & -0.42678 \\
\hline
\end{tabular}
\caption{\textbf{Base charges from the deprotonated carboxyl flake as implemented in the deprotonated side of carboxyl-terminated graphene oxide}.}
\end{center}
\end{table}

Similarly, for the deprotonated cases, the charge of the carbon atom is set such that the total charge of the deprotonated carboxyl group is -1. As can be seen in Figure \ref{carboxyl_deprotonated_charges}, for the DFT charge methods the negative charge is spread across the carobyxl flake, mostly at the edges. As a first approximation, the excess charge is localized on the carbon atom in our implementation, which avoids any size-inconsistent charge divisions.  

\bibliographystyle{acs}
%\bibliography{references.bib}
\begin{thebibliography}{1}

\bibitem{gaussian}
M.~J. Frisch, G.~W. Trucks, H.~B. Schlegel, G.~E. Scuseria, M.~A. Robb, J.~R.
  Cheeseman, G.~Scalmani, V.~Barone, B.~Mennucci, G.~A. Petersson,
  H.~Nakatsuji, M.~Caricato, X.~Li, H.~P. Hratchian, A.~F. Izmaylov, J.~Bloino,
  G.~Zheng, J.~L. Sonnenberg, M.~Hada, M.~Ehara, K.~Toyota, R.~Fukuda,
  J.~Hasegawa, M.~Ishida, T.~Nakajima, Y.~Honda, O.~Kitao, H.~Nakai, T.~Vreven,
  J.~A. {Montgomery Jr.}, J.~E. Peralta, F.~Ogliaro, M.~Bearpark, J.~J. Heyd,
  E.~Brothers, K.~N. Kudin, V.~N. Staroverov, T.~Keith, R.~Kobayashi,
  J.~Normand, K.~Raghavachari, A.~Rendell, J.~C. Burant, S.~S. Iyengar,
  J.~Tomasi, M.~Cossi, N.~Rega, J.~M. Millam, M.~Klene, J.~E. Knox, J.~B.
  Cross, V.~Bakken, C.~Adamo, J.~Jaramillo, R.~Gomperts, R.~E. Stratmann,
  A.~J.~Austin O.~Yazyev, R.~Cammi, C.~Pomelli, J.~W. Ochterski, R.~L. Martin,
  K.~Morokuma, V.~G. Zakrzewski, G.~A. Voth, P.~Salvador, J.~J. Dannenberg,
  S.~Dapprich, A.~D. Daniels, O.~Farkas, J.~B. Foresman, J.~V. Ortiz,
  J.~Cioslowski, and D.~J. Fox,
\newblock ``Gaussian 09, {Revision} {D.01} ({Gaussian}, {Inc.}, {Wallingford}
  {CT})''  (2013).

\bibitem{becke1992density}
Axel~D Becke,
\newblock ``Density-functional thermochemistry. {I.} {The} effect of the
  exchange-only gradient correction'',
\newblock {\em J. Chem. Phys.} {\bf 96}(3), pp. 2155--2160  (1992).

\bibitem{stephens1994ab}
Philip~J Stephens, Frank~J Devlin, Cary~F Chabalowski, and Michael~J Frisch,
\newblock ``Ab initio calculation of vibrational absorption and circular
  dichroism spectra using density functional force fields'',
\newblock {\em J. Phys. Chem.} {\bf 98}(45), pp. 11623--11627  (1994).

\bibitem{dunning1989gaussian}
Thom~H Dunning~Jr.,
\newblock ``Gaussian basis sets for use in correlated molecular calculations.
  {I.} {The} atoms boron through neon and hydrogen'',
\newblock {\em J. Chem. Phys.} {\bf 90}(2), pp. 1007--1023  (1989).

\bibitem{marenich2012charge}
Aleksandr~V Marenich, Steven~V Jerome, Christopher~J Cramer, and Donald~G
  Truhlar,
\newblock ``Charge model 5: {An} extension of {Hirshfeld} population analysis
  for the accurate description of molecular interactions in gaseous and
  condensed phases'',
\newblock {\em J. Chem. Theory Comput.} {\bf 8}(2), pp. 527--541  (2012).

\bibitem{mulliken1955electronic}
Robert~S Mulliken,
\newblock ``Electronic population analysis on {LCAO}--{MO} molecular wave
  functions. {I}'',
\newblock {\em J. Chem. Phys.} {\bf 23}(10), pp. 1833--1840  (1955).

\bibitem{hirshfeld1977bonded}
Fred~L Hirshfeld,
\newblock ``Bonded-atom fragments for describing molecular charge densities'',
\newblock {\em Theor. Chim. Acta} {\bf 44}(2), pp. 129--138  (1977).

\bibitem{Singh84}
U.~Chandra Singh and Peter~A. Kollman,
\newblock ``An approach to computing electrostatic charges for molecules'',
\newblock {\em J. Comput. Chem.} {\bf 5}(2), pp. 129--145  (1984).

\bibitem{besler1990atomic}
Brent~H Besler, Kenneth~M Merz~Jr, and Peter~A Kollman,
\newblock ``Atomic charges derived from semiempirical methods'',
\newblock {\em J. Comput. Chem.} {\bf 11}(4), pp. 431--439  (1990).

\end{thebibliography}
\end{document}