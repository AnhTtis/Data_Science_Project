\PassOptionsToPackage{table}{xcolor}
\documentclass[10pt,twocolumn,letterpaper]{article}

\usepackage{iccv}
\usepackage{times}
\usepackage{epsfig}
\usepackage{graphicx}
\usepackage{amsmath}
\usepackage{amssymb}

\usepackage{booktabs}
% extrea package
\usepackage{multirow}
\usepackage[export]{adjustbox}
\usepackage{threeparttable}
\usepackage{float}
\usepackage{subcaption}
\captionsetup{compatibility=false}
% \usepackage{subfloat}
% \usepackage{}
\usepackage{tabularx}
\usepackage{adjustbox}
\usepackage{color}
% \usepackage[table]{xcolor}
% \usepackage{amsmath, amssymb, multirow, overpic, textpos}
\usepackage{graphicx, amsmath, amssymb, multirow, overpic, textpos}
% table style
\usepackage[table]{xcolor}

\newcommand{\tablestyle}[2]{\setlength{\tabcolsep}{#1}\renewcommand{\arraystretch}{#2}\centering\footnotesize}

\renewcommand{\paragraph}[1]{\vspace{1.25mm}\noindent\textbf{#1}}
\newcommand\blfootnote[1]{\begingroup\renewcommand\thefootnote{}\footnote{#1}\addtocounter{footnote}{-1}\endgroup}

\newlength\savewidth\newcommand\shline{\noalign{\global\savewidth\arrayrulewidth
		\global\arrayrulewidth 1pt}\hline\noalign{\glob\textbf{}al\arrayrulewidth\savewidth}}
\definecolor{baselinecolor}{gray}{.9}
\newcommand{\baseline}[1]{\cellcolor{baselinecolor}{#1}}

% For math command 
\newcommand{\hatd}{\hat{d}}
\newcommand{\haty}{\hat{y}}
\newcommand{\hatp}{\hat{p}}
\newcommand{\hatm}{\hat{m}}
\newcommand{\hatc}{\hat{c}}
\newcommand{\noobject}{\varnothing}
\newcommand{\denc}{d_{\rm enc}}
\newcommand{\ddec}{d_{\rm dec}}
\renewcommand{\Re}{\mathbb{R}}
\newcommand{\hy}{\hat{y}}
\newcommand{\hb}{\hat{b}}
\newcommand{\hp}{\hat{p}}
\newcommand{\ty}{\tilde{y}}
\usepackage[misc]{ifsym} % use \Letter



% % For comments 
% \newcommand{\cavan}[1]{{\color{magenta}(cavan: {#1})}} % cavan's comments
% \newcommand{\lxt}[1]{{\color{red}(lxt: {#1})}} % xiangtai li's comments
% \newcommand{\gl}[1]{{\color{blue}(guangliang: {#1})}} % guangliang's comments
% \newcommand{\jiangmiao}[1]{{\color{blue}(Jiangmiao: {#1})}} %Jiangmiao's comments
% \newcommand{\zww}[1]{{\color{cyan}(zww: {#1})}} %zww's comments
% \newcommand{\haobo}[1]{{\color{violet}(haobo: please double check): {#1})}} % haobo's comments


% For reference link
\definecolor{linkcolor}{RGB}{255,0,0}
\definecolor{urlcolor}{RGB}{255,105,180}
\definecolor{citecolor}{RGB}{66,168,235}
\usepackage[pagebackref,breaklinks,colorlinks]{hyperref}
\hypersetup{colorlinks=true,linkcolor=linkcolor,urlcolor=urlcolor,citecolor=citecolor}

% Support for easy cross-referencing
\usepackage[capitalize]{cleveref}
\crefname{section}{Sec.}{Secs.}
\Crefname{section}{Section}{Sections}
\Crefname{table}{Table}{Tables}
\crefname{table}{Tab.}{Tabs.}





% Include other packages here, before hyperref.

% If you comment hyperref and then uncomment it, you should delete
% egpaper.aux before re-running latex.  (Or just hit 'q' on the first latex
% run, let it finish, and you should be clear).
% \usepackage[pagebackref=true,breaklinks=true,letterpaper=true,colorlinks,bookmarks=false]{hyperref}

\iccvfinalcopy % *** Uncomment this line for the final submission

% \def\iccvPaperID{4188} % *** Enter the ICCV Paper ID here
% \def\httilde{\mbox{\tt\raisebox{-.5ex}{\symbol{126}}}}

% Pages are numbered in submission mode, and unnumbered in camera-ready
\ificcvfinal\pagestyle{empty}\fi

\begin{document}

%%%%%%%%% TITLE
% \title{Tube-Link: A Efficient, and Flexible Framework for \\ Universal Video Segmentation}

\title{Tube-Link: A Flexible Cross Tube Baseline for Universal Video Segmentation}
\author{
Xiangtai Li$^{1}$ \quad
Haobo Yuan$^{2}$ \quad
Wenwei Zhang$^{1,4}$ \quad \\
Guangliang Cheng$^{3}$ \quad
Jiangmiao Pang$^{4}$ \quad
Chen Change Loy$^{1 \textrm{\Letter}}$ 
\\[0.1cm]
\small $ ^1$ S-Lab, Nanyang Technological University \quad 
\small $ ^2$ Wuhan University  \quad  \small $ ^3$ SenseTime Research \quad   \small $ ^4$ Shanghai AI Lab \\
{\tt\small \{xiangtai.li, ccloy\}@ntu.edu.sg} \\
 \url{https://github.com/lxtGH/Tube-Link}
}

\maketitle

% Remove page # from the first page of camera-ready.
\ificcvfinal\thispagestyle{empty}\fi

%The level of delegation to be granted to AI systems will in particular heavily depend on how methodological research replies to questions of robustness. This naturally brings us back to the development of statistical learning techniques that are reliable even in presence of partly contaminated data, due to biases in measurements or the deliberate intention to impair the operation of the automated system. Preference data, observed in the form of (complete) rankings in the simplest situations, are no exception of course and the demand for appropriate concepts and tools is all the more pressing given that technologies fed by or producing this type of data (\textit{e.g.} search engines, recommending systems) are now massively deployed. The lack of vector space structure for the set of rankings (\textit{i.e.} the symmetric group $\mathfrak{S}_n$) and the very complex nature of statistics usually considered in ranking data analysis make the formulation of robustness objectives in this domain extremely challenging. In this paper, we introduce notions of robustness, together with dedicated statistical methods, for \textit{Consensus Ranking}, the flagship problem in ranking data analysis, aiming at summarizing a probability distribution on $\mathfrak{S}_n$ by a \textit{median} ranking. Precisely, we propose specific extensions of the popular concept of \textit{breakdown point}, tailored to consensus ranking, and address the related computational issues. Beyond the theoretical contributions, the relevance of the approach proposed is supported by a detailed experimental study.

As the issue of robustness in AI systems becomes vital, statistical learning techniques that are reliable even in presence of partly contaminated data have to be developed. Preference data, in the form of (complete) rankings in the simplest situations, are no exception and the demand for appropriate concepts and tools is all the more pressing given that technologies fed by or producing this type of data (\textit{e.g.} search engines, recommending systems) are now massively deployed. However, the lack of vector space structure for the set of rankings (\textit{i.e.} the symmetric group $\mathfrak{S}_n$) and the complex nature of statistics considered in ranking data analysis make the formulation of robustness objectives in this domain challenging. In this paper, we introduce notions of robustness, together with dedicated statistical methods, for \textit{Consensus Ranking} the flagship problem in ranking data analysis, aiming at summarizing a probability distribution on $\mathfrak{S}_n$ by a \textit{median} ranking. Precisely, we propose specific extensions of the popular concept of \textit{breakdown point}, tailored to consensus ranking, and address the related computational issues. Beyond the theoretical contributions, the relevance of the approach proposed is supported by an experimental study.

\section{Introduction}
Object detection~\cite{fasterrcnn, ssd, yolo, fcos} is a fundamental computer vision task, aiming to localize and recognize the objects of predefined categories in an image. Owing to the rapid development of deep neural networks (DNN)~\cite{vgg,resnet,densenet,mobilenets,googlenet1,googlenet2,googlenet3}, the detection performance has been significantly improved in the past decade. During the evolution of object detectors, one important trend is to remove the hand-crafted components to achieve end-to-end detection.

One hand-crafted component in object detection is the design of training samples. For decades, anchor boxes have been dominantly used in modern object detectors such as Faster RCNN~\cite{fasterrcnn}, SSD~\cite{ssd} and RetinaNet~\cite{focalloss}. However, the performance of anchor-based detectors is sensitive to the shape and size of anchor boxes. To mitigate this issue, anchor-free~\cite{fcos,foveabox} and query-based~\cite{detr,deformdetr,dynamicdetr,conditionaldetr} detectors have been proposed to replace anchor boxes by anchor points and learnable positional queries, respectively.

Another hand-crafted component is non-maximum suppression (NMS) to remove duplicated predictions. The necessity of NMS comes from the one-to-many (o2m) label assignment~\cite{primesample,ota,atss,paa,gfocal,gfocalv2}, which assigns multiple positive samples to each GT object during the training process. This can result in duplicated predictions in inference and impede the detection performance. Since NMS has hyper-parameters to tune and introduces additional cost, NMS-free end-to-end object detection is highly desired.

\begin{figure}[tbp]
    \centering
    \includegraphics[width=0.4\textwidth]{fig_in_intro.pdf}
    \caption{The positive and negative weights of different anchors (A, B, C and D) in the classification loss during early and later training stages. Each anchor has a positive loss weight $t$ (in orange color) and a negative loss weight $1-t$ (in blue color). In our method,  A is a fully positive anchor, D is a fully negative anchor, and B and C are ambiguous anchors. One can see that for o2o and o2m label assignment schemes, the weights for all anchors are fixed during the training process, while for our o2f scheme, the weights for ambiguous anchors are dynamically adjusted.}
    \label{fig_in_intro}
    \vspace{-3mm}
\end{figure}

With a transformer architecture, DETR~\cite{detr} achieves competitive end-to-end detection performance. Subsequent studies~\cite{poto,onenet} find that the one-to-one (o2o) label assignment in DETR plays a key role for its success. Consequently, the o2o strategy has been introduced in fully convolutional network (FCN) based dense detectors for lightweight end-to-end detection. However, o2o can impede the training efficiency due to the limited number of positive samples. This issue becomes severe in dense detectors, which usually have more than 10k anchors in an image. What’s more, two semantically similar anchors can be adversely defined as positive and negative anchors, respectively. Such a ‘label conflicts’ problem further decreases the discrimination of feature representation. As a result, the performance of end-to-end dense detectors still lags behind the ones with NMS. Recent studies~\cite{dndetr,groupdetr,hybriddetr} on DETR try to overcome this shortcoming of o2o scheme by introducing independent query groups to increase the number of positive samples. The independency between different query groups is ensured by the self-attention computed in the decoder, which is however infeasible for FCN-based detectors.

In this paper, we aim to develop an efficient FCN-based dense detector, which is NMS-free yet end-to-end trainable. We observe that it is inappropriate to set the ambiguous anchors that are semantically similar to the positive sample as fully negative ones in o2o. Instead, they can be used to compute both positive and negative losses during training, without influencing the end-to-end capacity if the loss weights are carefully designed. Based on the above observation, we propose to assign dynamic soft classification labels for those ambiguous anchors. As shown in Fig.~\ref{fig_in_intro}, unlike o2o which sets an ambiguous anchor (anchor B or C) as a fully negative sample, we label each ambiguous anchor as partially positive and partially negative. The degrees of positive and negative labels are adaptively adjusted during training to keep a good balance between ‘representation learning’ and ‘duplicated prediction removal’. In particular, we begin with a large positive degree and a small negative degree in the early training stage so that the network can learn  the feature representation ability more efficiently, while in the later training stage, we gradually increase the negative degrees of ambiguous anchors to supervise the network learning to remove duplicated predictions. 
We name our method as a one-to-few (o2f) label assignment since one object can have a few soft anchors. We instantiate the o2f LA into dense detector FCOS, and our experiments on COCO~\cite{coco} and CrowHuman~\cite{crowdhuman} demonstrate that it achieves on-par or even better performance than the detectors with NMS.

\section{Related Work}
\label{section:related_work}

\import{tables/}{sota_comparison}

There have been many work focusing on fault injection for \titlecaseabbreviationpl{dnn}, focusing on the tools required for fault injection as well as the different sampling models. We show a comparison in Table \ref{table:sota_comparison}.
enpheeph \cite{colucciEnpheephFaultInjection2022a}, TensorFI \cite{liTensorFIConfigurableFault2018} and LLTFI \cite{agarwalLLTFIFrameworkAgnostic2022} focus on providing innovative fault injection frameworks to speed up the overhead of running fault injection compared to normal \titlecaseabbreviation{dnn} execution. However, they do not employ any specific algorithm for sampling, resorting to random uniform sampling. At the same time, they are able to cover the whole search space, as they do not filter the possible faults.
On the other hand, BinFI \cite{chenBinFIEfficientFault2019} and AVFI \cite{jhaMLBasedFaultInjection2019} provide improved fault models, not focusing on how the fault injection is executed as the previous works, but how to choose the faults using external knowledge, in their case human knowledge. However, they still employ random uniform sampling, but as they decrease the extension of the fault search space, they reach higher precision than the previous tools.
\emphasizedworkname{} employs a novel sampling method based on importance sampling, hence it is capable of achieving similar precision while still requiring no human knowledge and without reducing the fault search space.
\section{Method}\label{sec:method}
\begin{figure*}
    \centering
    \includegraphics[width=\linewidth,keepaspectratio]{figures/pipelines/pipeline_figure_full.pdf}
    %\vspace{-0.7cm}
    \caption[]{
    \OURS{} first generates a set of pseudo masks (top) to initiate self-training (bottom) for unsupervised 3D instance segmentation.
    We leverage features from 3D self-supervised pre-training in combination with 2D self-supervised features on an input mesh.
    These multi-modal features are then aggregated on geometric segment primitives, integrating low- and high-level signals for pseudo mask segmentation.
    These initial pseudo masks are then used as supervision for a 3D transformer-based model to produce updated instance masks that are integrated into the supervision of multiple self-training cycles.
    Finally, we obtain clean and dense instance segmentation without using any manual annotations.
    } 
    \label{fig:full_pipeline}
\end{figure*}

\paragraph{Problem definition}
We propose an unsupervised learning-based method for 3D instance segmentation. Formally, we assume a set of training 3D scenes $\{X_i\}_{i=1}^{n_t}$, represented as meshes, where each scene $X_i$ contains an unknown set of $n_i$ objects. We aim to train a model that can predict for a previously unseen input scene $X$, a set of 3D masks representing the different object instances in that scene. 

\paragraph{Method overview}
We employ a two-stage scheme for unsupervised 3D instance segmentation.
First, we group scene points into contiguous regions based on geometric primitives, and aggregate self-supervised features in these regions to generate an initial set of pseudo masks using Normalized Cut.  This grouping technique enables efficient handling of high-dimensional 3D data.
%
We then follow a series of self-training cycles to refine the pseudo annotations.
An overview of our approach is shown in Figure~\ref{fig:full_pipeline}.

\subsection{Geometric oversegmentation}\label{sec:oversegmentation}
Normalized Cut~\cite{shi2000normalized_cut} (NCut) with deep features has been successfully applied in the 2D domain on dense graphs generated from image patches \cite{wang2022tokencut,lis2022attentropy,wang2023cut}. However, adopting this directly to 3D would be computationally infeasible due to the cubic growth with dimensionality.

We thus propose a solution in the form of geometric oversegmentation through graph coarsening. We begin by creating a graph where each node represents a mesh vertex. Then, we aggregate nodes with similar normal direction and color values and cluster them into contiguous mesh segments using the efficient method proposed by \cite{felzenszwalb2004efficient}. This process reduces the graph size by multiple orders of magnitude. 
%
Our geometry aware segments, as opposed to voxels or points, additionally provide regularization to the subsequent stage of feature aggregation for generating pseudo masks, which we will describe in the following section.

\subsection{Initial pseudo mask generation}\label{sec:dataset_gen}

We first predict an initial set of pseudo masks.  Additional masks will be added during the self-training phase described in Section~\ref{sec:self_train}. 
Thus, we favor generating a reliable set of masks at the cost of restricting to a sparse initial set (i.e., missing potential instances rather than generating noisy masks for them).

\paragraph{Feature aggregation}

We aim to employ a strong set of features for pseudo mask generation and thus consider complementary geometric and color signals from RGB-D scan data.
We leverage geometric 3D self-supervised features from a state-of-the-art 3D pre-training approach, Contrastive Scene Contexts (CSC)~\cite{hou2021exploring}.
We additionally consider 2D self-supervised features from DINO~\cite{caron2021emerging_dino}, extracted from the RGB images and projected to 3D using the corresponding camera poses.
Both the 3D and 2D features are aggregated within each of our geometry-aware segments. 

\paragraph{Masked foreground separation}
We apply NCut to our aggregated features to extract foreground regions $M$ as our initial pseudo masks. 
%
Starting with an empty set $M^0=\{\}$, we iteratively compute the adjacency matrix and retrieve the masks. 
%
That is, we start from $N$ geometric segments with their corresponding $D$-dimensional features $\mathcal{F} \in \mathcal{R}^{N\times D}$, and construct the similarity matrix $A = sim(\mathcal{F})$, where $sim$ denotes cosine similarity. 
Additionally, for the multi-modal setup we calculate similarity matrices $A_{2D}$ and $A_{3D}$ independently and take their weighted average to obtain the final scores. 
Empirically, we found this to be more robust than direct feature fusion of the different modalities, due to their different statistical characteristics.
%
\indent We obtain $W_j$ for Equation~\ref{eq:general_eigenval} by thresholding $A$ at $\tau_{cut}$, where $j$ denotes the $j^{th}$ NCut iteration. 
Using $W_j$, we solve for the second eigenvector $v_j$ and threshold it to retrieve the partition $m_j$. 
We keep all separated foregrounds in $M^0$, where for each upcoming iteration, we mask out the row and column vectors from $W_i$, where $m_i \in M^0$ was already accepted as a foreground instance and $i$ being the segment ids. 
This allows greedy separation of instances in order of confidence  in every cut iteration.
Examples of our generated pseudo masks are visualized in Figures \ref{fig:freemask_vs_ncut} and \ref{fig:self_training_refinement}. \\
%
\indent As the adjacency graph is unaware of the mesh connectivity, NCut often results in masks that span  spatially separated scene regions. 
In 3D, we can leverage knowledge of physical distance to constrain masks to be contiguous in the coarsened scene connectivity graph. We thus filter masks that have separated components, keeping only the ones that contain the item with the maximum absolute value in  $v_j$. Separation is performed before saving $m_j$ into $M^0$, thus allowing for repeated separation of every component. 
We iterate until the maximum number of instances $M^0 = \{m_i\}_{i=1}^{N_m}$ are obtained, or there are no segments left in the scene. 

\subsection{Self-Training}\label{sec:self_train}

Our initial pseudo masks can provide a set of proposed instances $M^0$; however, these pseudo masks are quite sparse in the scenes and sometimes over- or under-split nearby instances.
We thus refine the pseudo mask data through an iterative self-training strategy, producing final instance segmentation predictions $M'$ with more dense and complete instance proposals.

We leverage a state-of-the-art 3D transformer-based backbone~\cite{Schult23mask3d} for our self-training from pseudo mask data as supervision. 
Through multiple training cycles we save the proposals of the $t^{th}$ iteration into $M^{t}$, from the self-trained model, and save these masks as an extension to the original pseudo dataset obtaining $M^t \supseteq M^0$. 
From the second training iteration, we can extract the most confident $K$ predictions and sample these new instance proposals as an addition to the pseudo annotations. 
Further, we only accept new instances if the added information value is larger than a minimum threshold, which we measure by simple segment IoU scores. This way, we can effectively densify the originally sparse annotations, but without limiting the quality of the originally clean pseudo masks. 
%
\paragraph{Loss} \label{par:losses}
We adapt DropLoss \cite{wang2023cut} for our self-training cycles, which is robust to sparse data and missing annotations. 
In particular, we use a weighted combination of cross-entropy and Dice \cite{sudre2017generalised_diceloss} losses for bipartite-matching with pseudo annotations.
We then drop losses for backpropagation which do not have at least $\tau_{drop}$ overlap with the annotations from the previous cycle.

\subsection{Implementation Details}\label{sec:implementation}
%
\paragraph{Backbones.} 
We use a Res16UNet34C sparse-voxel UNet implemented in the MinkowskiEngine~\cite{choy20194d} for 3D pre-trained feature extraction as well as for the 3D transformer during self-training. 

\paragraph{Self-training.} We employ the 3D transformer architecture of \cite{Schult23mask3d}, initialized from scratch. 
The first self-training cycle is trained for 600 epochs with a batch size of 8 until convergence, which takes $\approx 3$ days on a single NVIDIA RTX A6000 GPU. 
Further self-training cycles are all initialized from the previous state and finetuned for an additional 50 epochs in $\approx 4$ hours and for a total of 4 training cycles to produce the final set of instance predictions $S$. 
For the Hungarian assignment, we take the original weighted combination of dice and binary cross-entropy losses and only apply the DropLoss condition in the backpropagation phase.
\section{Experiments}
\label{sec:exp}
% logic:
% 1, Experiment Settings 

% 2, Benchmark results 
    % 1, VIP-Seg
    % 2, VSPW
    % 3, KITTI-STEP

% 3, Ablation studies and analysis. 
    % 1, improvements on baseline 
    % 2, design choices of temporal contarstive loss 
    % 3, design choices of label assigin stragety
    % 4, Effect of tube frames choices for CS loss 
    % 5, Effect of large window size / overlap inference. 
    % 6, Comparison with the different tracking choices. 
    % 7, increased GFLops/Parameters analysis. 
    % 8. FPS/Window Cruves.

% 4, visualization results. 
    % 1, comparison with strong baseline. 
    % 2, attention mask arcoss differnt tube. 
    

% Due to the unavailability of the test set, we report the results on the \textit{validation set}. 

% The former mainly focuses on mask proposal level as PQ~\cite{kirillov2019panoptic} with different window sizes, while the latter emphasizes pixel-level segmentation and tracking without any thresholds.
% KITTI-STEP has 21 and 29 sequences for training and testing, respectively. The training sequences are split into a training set (12 sequences) and a validation set (9 sequences).

\subsection{Experimental Settings}
\noindent
\textbf{Dataset.} We conduct experiments on five video datasets: VIPSeg~\cite{miao2022large}, VSPW~\cite{miao2021vspw}, KITTI-STEP~\cite{STEP}, and YouTube-VIS-19/21~\cite{vis_dataset}. We mainly conduct experiments on VIPSeg due to its scene diversity and long-length clips. The training, validation, and test sets of VIPSeg contain 2,806/343/387 videos with 66,767/8,255/9,728 frames, respectively. Although VSPW and VIPSeg share the same video clips, the training details are different since they are different tasks. Please refer to the \textit{supplementary material} for other datasets.


\noindent
\textbf{Evaluation Metrics.} For the VPS task, we adopt two metrics: $VPQ$~\cite{kim2020vps} and $STQ$~\cite{STEP}. The metric $STQ$ contains geometric mean of two items: Segmentation Quality ($SQ$) and Association Quality ($AQ$), where $ STQ = (SQ \times AQ)^{\frac{1}{2}}$. The former evaluates the pixel-level tracking, while the latter evaluates the pixel-level segmentation results in a video clip. For the VSS task, the Mean Intersection over Union (\textit{mIoU}) and mean Video Consistency ($mVC$)~\cite{miao2021vspw} are used for reference. For the VIS task, \textit{mAP} is adopted.


\noindent
\textbf{Implementation Details and Baselines.} We implement our models in PyTorch~\cite{pytorch_paper} with the MMDetection toolbox~\cite{chen2019mmdetection}. We use the distributed training framework with 16 V100 GPUs. Each mini-batch has one image per GPU. Following previous work, we use the image baseline pre-trained on COCO dataset~\cite{coco_dataset}. ResNet~\cite{resnet}, STDC~\cite{STDCNet}, and Swin Transformer~\cite{liu2021swin} are adopted as the backbone networks, which are pre-trained on ImageNet, and the remaining layers adopt the Xavier initialization~\cite{xavier_init}. 
For the detailed settings of other datasets, pretraining, and fine-tuning, please refer to the \textit{supplementary material}. To further verify the effectiveness of our approach, we build a stronger baseline by unifying Video K-Net with Mask2Former, where we replace the image encoder with Mask2Former. We term it Video K-Net+. We denote the extended Mask2Former-VIS for VPS as Mask2Former-VIS+.


%%%%%%%% VIP-SEG %%%%%%%%%%%
\begin{table}[!t]
	\centering
	\caption{\small \textbf{Results on VIPSeg-VPS~\cite{miao2022large} validation dataset.} We report VPQ and STQ for reference. Following Miao~\etal~\cite{miao2022large}, we report VPQ scores at different window sizes (1, 2, 4, 6). We report the results obtained from either an efficient or a strong backbone for comparison.}
	\label{tab:vipseg_results}
  \scalebox{0.65}{
    \begin{tabular}{ r|c|cccccc}
    \toprule[0.15em]
     Method& backbone & $VPQ^{1}$ & $VPQ^{2}$ & $VPQ^{4}$ & $VPQ^{6}$ & VPQ & STQ \\
    \toprule[0.15em]
    VIP-DeepLab~\cite{ViPDeepLab} & ResNet50 & 18.4 & 16.9 & 14.8 & 13.7 & 16.0 & 22.0 \\
    VPSNet~\cite{kim2020vps} & ResNet50 & 19.9 & 18.1 & 15.8 & 14.5 & 17.0 & 20.8 \\
    SiamTrack~\cite{woo2021learning_associate_vps} & ResNet50 & 20.0 & 18.3 & 16.0 & 14.7 & 17.2 & 21.1 \\
    Clip-PanoFCN~\cite{miao2022large} & ResNet50 & 24.3 & 23.5 & 22.4 & 21.6 & 22.9 & 31.5 \\
    Video K-Net~\cite{li2022videoknet} & ResNet50 & 29.5 & 26.5 & 24.5 & 23.7 & 26.1 & 33.1 \\
    Video K-Net+~\cite{cheng2021mask2former,li2022videoknet} & ResNet50 & 32.1 & 30.5 & 28.5 & 26.7 & 29.1 & 36.6  \\
    Video K-Net~\cite{li2022videoknet} & Swin-base & 43.3 & 40.5 & 38.3 & 37.2 & 39.8 & 46.3 \\
    \hline
    Tube-Link & STDCv1 & 32.1 & 31.3 & 30.1 & 29.1 & 30.6 & 32.0 \\
    Tube-Link & STDCv2 & 33.2  & 31.8 & 30.6 & 29.6  &  31.4 & 32.8 \\
    \hline
    Tube-Link & ResNet50 & 41.2 & 39.5  & 38.0 & 37.0 &  39.2 & 39.5 \\
    Tube-Link & Swin-base & 54.5 & 51.4 & 48.6 & 47.1 & 50.4 & 49.4 \\
    % Tube-Link & Swin-large &  \lxt{wait results} \\
    \bottomrule[0.2em]
    \end{tabular}
}
\end{table}


%%%%%% VIS-Youtube %%%%%%%%%
\begin{table}[t]
  \centering
   \caption{\small \textbf{Results on the YouTube-VIS datasets.} We report the mAP metric. \textdagger~adopt COCO video pseudo labels. Axial means using the extra Axial Attention~\cite{axialDeeplab}. Our method does not apply these techniques for simplicity.}
  \label{tab:ytvis}
  \scalebox{0.68}{
  \begin{tabular}{l c | c  | c }
    \toprule[0.2em]
    Method & Backbone  & YTVIS-2019 & YTVIS-2021 \\
    \toprule[0.2em]
VISTR~\cite{VIS_TR} & ResNet50 & 36.2 & -  \\
TubeFormer~\cite{kim2022tubeformer} & ResNet50 + Aixal & 47.5  & 41.2  \\
IFC~\cite{hwang2021video} & ResNet50 & 42.8 & 36.6 \\
SeqFormer~\cite{seqformer} & ResNet50 & 47.4 & 40.5  \\
Mask2Former-VIS~\cite{cheng2021mask2former_vis}& ResNet50 & 46.4 & 40.6 \\
IDOL~\cite{IDOL} & ResNet50 & 46.4 & 43.9\\
IDOL~\cite{IDOL} \textdagger & ResNet50 & 49.5 & -\\
VITA~\cite{heo2022vita} \textdagger & ResNet50 & 49.8 & 45.7  \\
Min-VIS~\cite{huang2022minvis} &ResNet50& 47.4 & 44.2 \\
% GenVIS~\cite{heo2022generalized} & ResNet50 & 51.3 & 46.3 \\
\hline
Tube-Link & ResNet50 & 52.8 & 47.9  \\% & - \\
\hline
SeqFormer~\cite{seqformer} & Swin-large  & 59.3 & 51.8 \\% & - \\
Mask2Former-VIS~\cite{cheng2021mask2former_vis} & Swin-large &  60.4 & 52.6 \\
IDOL~\cite{IDOL}  & Swin-large  & 61.5 & 56.1 \\ %& 42.6\\
IDOL~\cite{IDOL}  & Swin-large \textdagger  & 64.3 & -\\
VITA~\cite{heo2022vita} \textdagger & Swin-large & 63.0 & 57.5 \\ 
Min-VIS~\cite{huang2022minvis} & Swin-large & 61.6 & 55.3 \\
\hline
Tube-Link & Swin-large  & 64.6 & 58.4  \\
    \bottomrule[0.2em]
  \end{tabular}
}
\end{table}



%%%%%% VSPW and VIP-Seg VSS%%%%%%%%%
\begin{table}[t]
  \centering
    \caption{\small \textbf{Results on VSPW-VSS validation set}. $mVC_{c}$ means that a clip with $c$ frames is used.}
    \label{tab:vspw}
  \scalebox{0.68}{
  \begin{tabular}{l c c c c c }
    \toprule[0.2em]
    \textbf{VPSW} & Backbone & mIoU & $mVC_{8}$ &$mVC_{16}$  \\
    \toprule[0.2em]
    DeepLabv3+~\cite{deeplabv3plus} & ResNet101 & 35.7 & 83.5 & 78.4 \\
    TCB(PSPNet)~\cite{miao2021vspw,zhao2017pyramid} & ResNet101 & 37.5 & 86.9 & 82.1  \\
    Video K-Net (Deeplabv3+)~\cite{li2022videoknet,deeplabv3plus} & ResNet101  & 37.9 & 87.0 & 82.1 \\
    Video K-Net (PSPNet)~\cite{li2022videoknet,zhao2017pyramid} & ResNet101  & 38.0 & 87.2  & 82.3 \\
    MRCFA~\cite{sun2022mining} & MiT-B5 & 49.9 & 90.9  &  87.4  \\
    CFFM~\cite{sun2022vss} & MiT-B5 & 49.3 & 90.8 & 87.1 \\
    TubeFormer~\cite{kim2022tubeformer} & Axial-ResNet50x64  &  63.2 &  92.1 & 88.0 \\
    \hline
    Tube-Link & ResNet50 & 42.3 & 86.8 & 83.2 \\
    Tube-Link & Swin-large & 59.7 & 90.3 & 88.4 \\
    \bottomrule[0.2em]
  \end{tabular}
  }

\end{table}


\begin{table}[t]
  \centering
    \caption{\small \textbf{Results on VIP-Seg-VSS validation set}. $mVC_{c}$ means that a clip with $c$ frames is used.}
    \label{tab:vipseg_vss}
  \scalebox{0.68}{
  \begin{tabular}{l c c c c c }
    \toprule[0.2em]
    \textbf{VPSW} & Backbone & mIoU & $mVC_{8}$ &$mVC_{16}$  \\
    \toprule[0.2em]
    Video K-Net (Deeplabv3+)~\cite{li2022videoknet,deeplabv3plus} & ResNet101  & 38.3 & 88.0 & 83.1 \\
    Video K-Net (PSPNet)~\cite{li2022videoknet,zhao2017pyramid} & ResNet101  & 39.0 & 88.2  & 84.2 \\
    Mask2Former~\cite{cheng2021mask2former} &  ResNet50 & 38.4 & 87.5 & 82.5 \\
    Video K-Net+~\cite{cheng2021mask2former,li2022videoknet} &  Swin-base & 57.2 & 90.1 & 87.8  \\
    \hline
    Tube-Link & ResNet50 & 43.4 & 89.2 & 85.4 \\
    Tube-Link & Swin-base & 62.3 & 91.4 & 89.3 \\
    Tube-Link & Swin-large & 64.9 & 92.4 & 89.9 \\
    \bottomrule[0.2em]
  \end{tabular}
  }

\end{table}


\subsection{Benchmark Results}


%%%%%% KITTI-STEP %%%%%%
\begin{table}[t]
  \centering
   \caption{\small \textbf{Results on the KITTI val set.} OF refers to an optical flow network~\cite{teed2020raft}.}
  \label{tab:kitti_step}
  \scalebox{0.68}{
  \begin{tabular}{l c c || c c c c }
    \toprule[0.2em]
    \textbf{KITTI-STEP} & Backbone & OF & STQ & AQ & SQ & VPQ \\
    \toprule[0.2em]
    P + Mask Propagation & ResNet50 & \checkmark & 0.67 & 0.63 & 0.71 & 0.44 \\
    Motion-Deeplab~\cite{STEP}& ResNet50 &  & 0.58 & 0.51 & 0.67 & 0.40  \\
    VPSNet~\cite{kim2020vps}& ResNet50  & \checkmark & 0.56 & 0.52 & 0.61 & {0.43}  \\
    TubeFormer-DeepLab~\cite{kim2022tubeformer} & ResNet-50 + Axial &  & 0.70 & 0.64 &  0.76 & 0.51 \\
    Video K-Net~\cite{li2022videoknet} & ResNet50 &  & 0.71 & 0.70  & 0.71  &  0.46 \\
    Video K-Net~\cite{li2022videoknet} & Swin-base &  & 0.73 & 0.72 & 0.73 & 0.53 \\
    \hline
    Tube-Link & ResNet50 &  & 0.68 & 0.67 & 0.69 & 0.51 \\
    Tube-Link & Swin-base &  & 0.72 & 0.69 & 0.74 & 0.56 \\
    \bottomrule[0.2em]
  \end{tabular}
  }
  \vspace{-4mm}
\end{table}

% \lxt{will be changed by test set Figure Results Further. This figure will be merged into it as subfigure.}
\begin{figure}[t]
  \centering
   \includegraphics[width=0.80\linewidth]{./figs/teaser_trade_off.pdf}
   \caption{\small Tube-Link also achieves the best accuracy and speed trade-off on VIP-Seg dataset. FPS is measured on RTX GPU.}
   \label{fig:curve_trade_off_vipseg}
\end{figure}

\noindent
\textbf{[VPS] Results on VIPSeg.} 
We present the results of our Tube-Link method compared to previous works on the VIPSeg dataset in Tab.~\ref{tab:vipseg_results}. Our approach outperforms Video K-Net\cite{li2022videoknet} (under the same backbone) with 12\%-15\% VPQ and 7\%-10\% STQ improvements, respectively. Notably, our method with Swin-base~\cite{liu2021swin} backbone achieves new state-of-the-art results. 
%
We also evaluate our method using a lightweight backbone~\cite{STDCNet} for more efficient inference on video clips, and it achieves even better results than all previous methods with a larger ResNet50 backbone. 
%
These results demonstrate the effectiveness of our approach in exploiting temporal information.  Benefiting from the joint inference of subclips, our method achieves a much faster inference speed, as shown in Fig.~\ref{fig:curve_trade_off_vipseg}. 



\begin{table*}[h!]
    \footnotesize
	\centering
	\caption{\small \textbf{Ablation studies and comparative analysis on VIPSeg validation set with the ResNet50 backbone.} 
	}
    \subfloat[Ablation Study on Each Component.]{
    \label{tab:ablation_a}
	    \begin{tabularx}{0.43\textwidth}{c c c c c} 
		        				\toprule[0.15em]
    	baseline  & TCL & CTL & $\mathrm{VPQ_{th}}$ & VPQ \\
        \toprule[0.15em]
            Mask2Former-VIS+ (F) & - & - & 29.4 & 32.4 \\
            \hline
            Mask2Former-VIS+ (T) & - & - & 31.0 & 34.5\\
             & \checkmark & - & 34.6  & 36.8  \\  
          \rowcolor{gray!15}  & \checkmark & \checkmark & 35.1 & 37.5 \\  
        \bottomrule[0.1em]
	    \end{tabularx}
    } \hfill
    \subfloat[Design Choices of TCL.]{
    \label{tab:ablation_b}
		\begin{tabularx}{0.28\textwidth}{c c c} 
			\toprule[0.15em]
			Method & VPQ & STQ \\
			\midrule[0.15em]
            Dense Query~\cite{qdtrack} & 30.2  & 30.1  \\
            Sparse Query~\cite{li2022videoknet} & 34.5  & 35.1 \\
            \rowcolor{gray!15} Global Query(Ours) &  37.5  & 36.5 \\
			\bottomrule[0.1em]
		\end{tabularx}
    } \hfill
    \subfloat[Association Target Assign.]{
    \label{tab:ablation_c}
		\begin{tabularx}{0.24\textwidth}{c c c} 
			\toprule[0.15em]
			Method & VPQ & STQ  \\
			\midrule[0.15em]
			All-Masks~\cite{qdtrack} & 30.1 & 29.2 \\
			GT-Mask~\cite{li2022videoknet} & 35.6 & 35.9 \\
			\rowcolor{gray!15} Tube-Mask & 37.5 & 36.5 \\
			\bottomrule[0.1em]
		\end{tabularx}
    } \hfill
    \vspace{2mm}
    \subfloat[Input Sub-clip Size with Tube Window Size of 2 as Input.]{
     \label{tab:ablation_d}
	    \begin{tabularx}{0.30\textwidth}{c c c c} 
		        				\toprule[0.15em]
    		 Clip Size & STQ & VPQ & $\mathrm{VPQ_{th}}$  \\
    		\toprule[0.15em]
    	    T=1 & 34.5 & 35.6 & 30.2 \\
    	    \rowcolor{gray!15} T=2 & 36.5 & 37.5 & 35.1 \\
    	    T=2(ovl) & 35.9 & 37.3 & 35.0 \\
    	    T=3 &  36.4 & 37.0 & 35.3 \\
        	\bottomrule[0.1em]
	    \end{tabularx}
    } \hfill
    \subfloat[Tube-Window for Inference with Input Sub-clip Size 2 for Training.]{
     \label{tab:ablation_e}
	    \begin{tabularx}{0.30\textwidth}{c  c c c} 
		        				\toprule[0.15em]
    		 Window Size & STQ & VPQ  & $\mathrm{VPQ_{th}}$ \\
    		\toprule[0.15em]
    	    W=2 &  36.5 & 37.5 & 35.1 \\
    	    W=4 &  39.2 & 39.0 & 38.2 \\
    	   \rowcolor{gray!15} W=6 &  39.5 & 39.2 & 38.9 \\
    	    W=8 &  38.3 & 38.5 & 37.3 \\
        	\bottomrule[0.1em]
	    \end{tabularx}
    } \hfill
    \subfloat[Tracking Choices with the Default Setting of Tab.(d). ]{
     \label{tab:ablation_f}
	    \begin{tabularx}{0.35\textwidth}{c c c c} 
		        				\toprule[0.15em]
    		 Settings  &  STQ & VPQ & $\mathrm{VPQ_{th}}$ \\
    		 \toprule[0.15em]
    		  Extra Tracker~\cite{wangUnitrack,deepsort}& 33.9 & 36.6 & 34.1 \\
    		  RoI Features~\cite{qdtrack} & 34.5 & 35.9 & 34.5 \\
    		  Query Embedding~\cite{li2022videoknet}  & 33.1  & 36.0  & 33.0 \\
    	     \rowcolor{gray!15} Our Tube embedding & 36.5 & 37.5 & 35.1\\
        	\bottomrule[0.1em]
	    \end{tabularx}
    } \hfill
\end{table*}


\noindent
\textbf{[VIS] Results on YouTube-VIS-2019/2021.} In Tab.~\ref{tab:ytvis}, we compare our method with state-of-the-art VIS methods on the YouTube-VIS 2019 and 2021 datasets. Our method achieves a 3.0\% and 2.2\% mAP gain over VITA~\cite{heo2022vita} when using the ResNet50 backbone. Furthermore, compared with the Mask2Former-VIS baseline~\cite{cheng2021mask2former_vis}, our method achieves 4-5\% mAP gains on the two datasets with different backbones. Our method also outperforms the previous near-online method TubeFormer~\cite{kim2022tubeformer} by 5-6\% in terms of mAP on the two VIS datasets.


\noindent
\textbf{[VSS] Results on VSPW and VIP-Seg.} We further conduct experiments on VSPW dataset~\cite{miao2021vspw} for VSS to demonstrate the generalization of Tube-Link. As shown in Tab.~\ref{tab:vspw}, our method achieves over 4\% mIoU improvement compared to the Mask2Former baseline. Under the same ResNet101 backbone, our method achieves the best results. Using the Swin base backbone, our method achieves about 3.7\% mIoU gains over Video K-Net+ with consistent improvements on $mVC$. Our method with a lightweight backbone achieves comparable results to DeepLabv3+ with ResNet101, but with about four times faster inference speed (shown in Fig.~\ref{fig:curve}). Without using any additional techniques, our method also outperforms recent methods specifically designed for VSS~\cite{sun2022vss,sun2022mining}. In Tab.~\ref{tab:vipseg_vss}, we also compare the video semantic segmentation methods in recent VIPSeg datasets with higher-resolution images. Compared with previous state-of-the-art methods, our approaches also achieve state-of-the-art results.

% Moreover, compared with the previous state-of-the-art Tubeformer~\cite{kim2022tubeformer}, our method achieves a better 1.7\% mIoU.


\noindent
\textbf{[VPS] Results on KITTI STEP.} 
We further validate our method on KITTI STEP~\cite{STEP} and report the results in Tab.~\ref{tab:kitti_step}. Our method achieves 0.51 VPQ with the ResNet50 backbone, setting a new state-of-the-art result \textit{without} using temporal attention or optical flow warping. When using a strong Swin-base~\cite{liu2021swin} backbone, our method still achieves better results than Video K-Net~\cite{li2022videoknet} by 3\% VPQ and comparable results on STQ. It is worth noting that one can further improve the performance of Tube-Link by employing a better tracker design.

\subsection{Ablation Study and Visual Analysis}
\label{sec:ablation}
% 1, improvements on baseline 
% 2, design choices of temporal contarstive loss 
% 3, design choices of label assigin stragety
% 4, Effect of tube frames choices for CS loss 
% 5, Effect of large window size / overlap inference. 
% 6, Comparison with the different tracking choices. 
% 7, increased GFLops/Parameters analysis. 
% 8. FPS/Window Cruves.

% In this section, we present some \textit{\textbf{key}} ablations on component design and analysis using VIPSeg dataset with ResNet50 backbone. 
%The default setting used in our model is indicated in gray.
%More results are provided in the supplementary material. 

% \cavan{image part? or do you mean feature extractor or encoder}
% \cite{li2022videoknet} as the baseline by replacing its encoder with Mask2Former~\cite{cheng2021mask2former}
\noindent
\textbf{Improvements over Strong VPS Baseline.} 
In Tab.~\ref{tab:ablation_a}, we demonstrate the effectiveness of each component proposed in Sec.~\ref{sec:tb_framework}. 
The first row shows the results of the frame matching baseline. After adopting the tube matching, we obtain a gain of 1.6\% $\mathrm{VPQ_{th}}$ and 2.1\% on VPQ, even without any specific tracking design, which results in the same observation as shown in Tab.~\ref{tab:toy_exp}. Thus, we use Mask2Former-VIS+ (T, T=2) as our baseline by default, which achieves a strong starting point of 34.5 VPQ. $\mathrm{VPQ_{th}}$ refers to the VPQ for the thing class. This result shows the effectiveness of the na\"{i}ve framework. The addition of TCL further boosts performance, with a gain of 3.5\% on $\mathrm{VPQ_{th}}$ and 1.7\% on VPQ. Furthermore, adding CTL, which makes the association more consistent, improves $\mathrm{VPQ_{th}}$ by 1.5\%.


\noindent
\textbf{Ablation on Temporal Contrastive Loss.} We also compare our TCL design with previous works that use dense queries~\cite{qdtrack} or sparse queries~\cite{li2022videoknet} for matching. Both settings use only one frame, while our subclip size is two. As shown in Tab.~\ref{tab:ablation_b}, our method achieves the best results since tube matching encodes more temporal information. In particular, we observe 3.0\% VPQ improvements compared to the strong Video K-Net baseline.


\begin{figure}[t!]
	\centering
	\includegraphics[width=1.0\linewidth]{./figs/tube_link_vis_results_1st.pdf}
	\caption{\small Comparison results on VIP-Seg and YuoTube-VIS. Our method achieves consistent segmentation (shown in orange boxes) and better tracking results (shown in red boxes).}
	\label{fig:visulize}
\end{figure}

% \gl{R-50 and R50 should be consistent. The same as R-101 and R101.}
\begin{figure}[t]
  \centering
   \includegraphics[width=1.\linewidth]{./figs/both.pdf}
   \caption{\small Efficiency Analysis of Tube-Link. Left: Segmentation results (mIoU) of VSPW with different subclip sizes. Right: Inference speed (FPS) with different subclip sizes.}
   \label{fig:curve}
\end{figure}



\noindent
\textbf{Ablation on Association Target Assignment.} 
In Table \ref{tab:ablation_c}, we show the results of the ablation study on building association targets. We find that using a tube-level mask achieves the best results. Using the mask from one of the input subclips leads to inferior results. This is because the ground truth masks of a single frame are not aligned with the input global queries, where the global queries are learned from multiple frames using Equation \eqref{equ:sp_attention}.

\noindent
\textbf{Effect of Sub-clip Size for Training.} 
In Tab.~\ref{tab:ablation_d}, we investigate the impact of subclip size on training. Tube-Link becomes an online method when the subclip size is 1. As shown in the table, enlarging the subclip size improves the performance. We also examine overlapping during sampling, denoted as ovl, where two input subclips overlap at one frame. As shown in Tab.~\ref{tab:ablation_d}, enlarging the subclip size to 2 achieves significant improvement. However, we find that either frame overlapping or using a larger subclip size ($T=3$) does not bring extra gains. Adding more frames does not benefit temporal association learning, since most instances are similar within a subclip. Moreover, using more frames is not memory-friendly during training. Thus, the subclip size is set to 2. We can enlarge the size for inference, as shown in Tab.~\ref{tab:ablation_e}.
  
% \cavan{to what value and why? During training?}
% \cavan{you mean we can set the subcip size to 2 during training and expand it during inference? This point is not clearly articulated here.}
%  \cavan{where? In future work?}
% \cavan{not sure why we use `Moreover', it doesn't connect well to the previous sentence.}
% % Hence, we can enlarge the subclip size for more efficient inference and global consistency within each tube. 

\noindent
\textbf{Effect of Sub-clip Size for Inference.} 
\if 0
The global queries for each tube learn to perform temporal association via cross-attention within each subclip. Despite the subclip size is limited during the training due to the memory issues, we can expand it during the inference.
For example, the subclip size is 2 during training and is set to 6 for inference. As shown in Tab.~\ref{tab:ablation_e}, we prove that enlarging subclip size for inference improves the performance by a significant margin for all three metrics: STQ, VPQ and $\mathrm{VPQ_{th}}$. When the size is 8, the performance drops. This is because the global queries cannot handle larger subclips as the offline method. Besides the effectiveness, increasing subclip size can also lead to faster speed for each clip input due to full utilization of GPU memory, as shown in Fig.~\ref{fig:curve}.
\fi
%
During training, the subclip size is limited due to memory constraints, but we can expand it during inference to improve the performance. For instance, we use a subclip size of 2 during training and increase it to 6 during inference. Tab.~\ref{tab:ablation_e} shows that enlarging the subclip size for inference improves the performance considerably for all three metrics: STQ, VPQ, and $\mathrm{VPQ_{th}}$. However, when the subclip size is further increased to 8, the performance drops because the global queries are not designed to handle larger subclips. Increasing the subclip size can also speed up the inference process by utilizing the full GPU memory, as demonstrated in Fig.~\ref{fig:curve}.

% \cavan{`lead to a higher number of frames leads to faster speed'? Rephrase this sentence.}

\noindent
\textbf{Different Tracking Choices.} 
\if 0
In Tab.~\ref{tab:ablation_f}, we compare different tracking approaches that were used in previous studies~\cite{qdtrack,li2022videoknet,deepsort}. The default Tube Embedding works best in our framework. It does not require any association embedding head or the RoI crop operation on the VIPSeg dataset. Our Tube-Link only uses the learned tube-level embedding for the association.
\fi
In Tab.~\ref{tab:ablation_f}, we compare different tracking approaches used in previous studies~\cite{qdtrack,li2022videoknet,deepsort} with our Tube-Link. Our Tube-Link only uses the learned tube-level embedding for the association. We find that the default tube embedding works best in our framework, without requiring any association embedding head or RoI crop operation on the VIPSeg dataset.  

\subsection{Visualization and More Analysis}
\label{sec:vis_analysis}

\noindent
\textbf{GFLops and Parameter Analysis.} Compared with Mask2Former baseline, we only add one $\mathrm{Emb}$ head and one self-attention layer, introducing only 2.2\% GFLops and 1.4\% extra parameters with $720 \times 1280$ input. 

%\cavan{the font size is too small to be visible. You can use a common legend for both plots and place it underneath the plots}



% \subsection{Visualization and Analysis}
% \cavan{Do we really need this section? The `Speed and Accuracy with different Input Subclip Size' can be merged with `Effect of Sub-clip Size For Inference.' in the ablation study. `Visual Improvements on Baseline' can be merged with `Improvements over Strong VPS Baseline'.}

\noindent
\textbf{Speed and Accuracy with Different Input Subclip Size.} 
As shown in Table \ref{tab:ablation_e}, adding more frames improves the VPS results. To further analyze the speed-accuracy trade-off, we present a detailed comparison of different methods on the VSPW dataset in Fig.~\ref{fig:curve}. The left plot shows that enlarging the subclip size also improves the VSS results. The right plot illustrates that increasing the subclip size improves the single-frame baseline by 1.25-1.5\% for various backbones. Both performance and speed reach a plateau when the size increases to 6. The experiment justifies our choice of using an input subclip size of 6 for inference.

\noindent
\textbf{Visual Improvements on Baselines.} In Fig.~\ref{fig:visulize}, we present the visual comparison with several strong baselines (Video K-Net+ and Mask2Fomer-VIS) in VPS and VIS settings. The results are randomly sampled from a long clip. We achieve better results on both segmentation and tracking. More visual examples can be found in the supplementary material. 

 \section{Conclusion}
 In this paper, we have presented a tactile manipulation system that is able to rotate different objects without vision. We showed an end-to-end reinforcement learning framework to learn tactile dexterity over the proposed system. We carried out experiments both in simulation and real to demonstrate its effectiveness. Our work demonstrated that we are able to achieve tactile dexterity as humans in real for the first time. In the future, there are many promising future directions to investigate, such as exploring the use of a more dense contact sensor array and scaling up the system to solve more diverse tasks. We hope that our work can pave the way for more intelligent robot hands.
\appendix
\section{Appendix}


\noindent
\textbf{Overview.} In addition to the main paper, we further list the following details and more experiment results as supplementary to our work.

\begin{enumerate}
\setlength{\leftmargin}{-1em}
\setlength{\parsep}{0ex} 
\setlength{\topsep}{0ex}
\setlength{\itemsep}{0.5ex}  
\setlength{\labelsep}{0.5em} 
\setlength{\itemindent}{-0.5em} 
\setlength{\listparindent}{0em} 
\item More detailed description and comparison of Tube-Link. (Sec.~\ref{sec:more_decription_tube_link})
\item Detailed experiment settings and implementation details for each dataset. (Sec.~\ref{sec:implemntation_details})
\item More ablation studies and experiment results.(Sec.~\ref{sec:more_ablation_and_exp_results}) 
\item More visual results. (Sec.~\ref{sec:vis_results})
\end{enumerate}



\begin{table*}[!t]
   \centering
    \caption{\small Different Setting Comparison with previous VIS and VPS methods.}
   \scalebox{0.65}{
   \setlength{\tabcolsep}{2.5mm}{\begin{tabular}{c | c c   c | c  c | c | c c c c}
      \toprule[0.15em]
        Method &  VSS & VIS & VPS & Online & Nearly Online & Joint Mulitple Frames  & Frame Matching & Tube Matching & Mask Matching & No Association (use Query Index) \\
        \hline 
        \rowcolor{gray!15} CFFM~\cite{sun2022vss} & \checkmark &  &  &  & \checkmark  & \checkmark &  &  &  & \checkmark \\
         \rowcolor{gray!15} MRCFA~\cite{sun2022mining} & \checkmark &  &  &  & \checkmark  & \checkmark &  & & & \checkmark  \\
         \hline
        \rowcolor{orange!15} Cross-VIS~\cite{CrossVIS} & &  \checkmark &  & \checkmark & & & \checkmark & & &\\
        \rowcolor{orange!15} IDOL~\cite{IDOL} & &  \checkmark &  & \checkmark & & & \checkmark & & &  \\
        \rowcolor{orange!15} SeqFormer~\cite{seqformer} & & \checkmark & &  & \checkmark & \checkmark & & & & \checkmark \\
         \rowcolor{orange!15}EfficientVIS~\cite{EfficientVIS}& & \checkmark & & & \checkmark & \checkmark & & & & \checkmark\\   
         \rowcolor{orange!15}VITA~\cite{heo2022vita} & & \checkmark & & & \checkmark & \checkmark & & & & \checkmark \\
        \rowcolor{orange!15} Min-VIS~\cite{huang2022minvis} & & \checkmark & & \checkmark & & & \checkmark & & & \\
         \rowcolor{orange!15} IFC~\cite{hwang2021video} & & \checkmark &   & & \checkmark & \checkmark &  &  \checkmark & & \\
         \rowcolor{orange!15} Gen-VIS~\cite{heo2022generalized} & & \checkmark & & \checkmark & \checkmark & \checkmark & & \checkmark & &  \\
        \hline
        \rowcolor{blue!15} SLOT-VPS~\cite{slot_vps} &  & & \checkmark &  & \checkmark & \checkmark & & & & \checkmark \\
        \rowcolor{blue!15} TubeFormer~\cite{kim2022tubeformer} &  \checkmark & \checkmark & \checkmark &  & \checkmark & \checkmark & & & \checkmark &   \\
        \rowcolor{blue!15} Video K-Net~\cite{li2022videoknet} & \checkmark & \checkmark & \checkmark & \checkmark &  &  & \checkmark & & & \\
        \rowcolor{red!15} Our Tube-Link & \checkmark & \checkmark & \checkmark & \checkmark & \checkmark & \checkmark &  & \checkmark &  & \\
      \bottomrule[0.10em]
   \end{tabular}}}
   \label{tab:more_detailed_comparison}
\end{table*}


\subsection{More Detailed Description of Tube-Link}
\label{sec:more_decription_tube_link}


This section presents the method details, including several baselines and Tube-Link inference for different datasets. 
%
Then, due to the limited pages in the main paper, we compare several closely related works in VIS and VPS in detail. 


\vspace{2mm}
\noindent
\textbf{Video K-Net+ Baseline.} This baseline is based on two previous state-of-the-art methods, including Video K-Net~\cite{li2022videoknet} and Mask2Former~\cite{cheng2021mask2former}. 
%
In particular, Video K-Net is based on K-Net~\cite{zhang2021knet}, an image panoptic segmentation model. We replace K-Net with Mask2Former~\cite{cheng2021mask2former}, and the remaining parts are the same as the Video K-Net. 
%
Since the performance of Mask2Former is better than K-Net on image segmentation datasets, Video K-Net+ serves as the strong online baseline for both VPS and VSS tasks.

\vspace{2mm}
\noindent
\textbf{Detailed Inference Procedure of Panoptic Matching.} During the inference, we perform tube-level panoptic matching according to learned association embeddings only on final panoptic tube masks.
%
In particular, we save the index of each global query from the final panoptic tube results, and then we use these indexed queries via the embedding head $Emb$. 

\vspace{2mm}
\noindent
\textbf{Detailed Inference Procedure of VSS and VIS.} Since VSS does not need tracking, we do not apply the extra tracking embedding during the inference. 
%
Instead, the tube mask logits are obtained directly from the dot product between global queries and spatial-temporal decoder features. 
%
The final segmentation labels are directly obtained via argmax on predicted logits. 
%
For VIS, we follow nearly the same procedure as VPS, except no stuff queries are involved.


\vspace{2mm}
\noindent
\textbf{More Detailed Comparison with Previous Nearly Online Approaches in VIS and VPS.} In addition to the main paper, in Tab.~\ref{tab:more_detailed_comparison}, we present a more detailed comparison with previous works on VIS and VPS. From the table, our method uses tube-wised matching and supports all three video segmentation tasks in one architecture. 

In particular, both SLOT-VPS~\cite{slot_vps} and SeqFormer~\cite{seqformer} also adopt multiple frames design. However, there are no data association processes involved. Moreover, they are designed for VIS and VPS, individually, and our method outperforms the SeqFormer on two VIS datasets, as shown in Tab.~\ref{tab:ytvis_supp}. Furthermore, unlike SLOT-VPS and SeqFormer, our method can handle long video inputs.

Gen-VIS~\cite{heo2022generalized} also adopts the tube-wised design, which combines the nearly online method and online method in one framework. However, it can not support other video segmentation tasks, including VSS and VPS. Moreover, it is  
not verified in more complex scenes, including the driving dataset KITTI-STEP~\cite{STEP} and the recent more challenging dataset VIP-Seg~\cite{miao2022large}. In contrast, our Tube-Link is fully verified by three different video segmentation tasks and five different datasets. In particular, using the same ResNet50 backbone and detector~\cite{cheng2021mask2former}, even without COCO video joint training, our method works better than Gen-VIS~\cite{heo2022generalized}, as shown in Tab.~\ref{tab:ytvis_supp}.

% % online apporaches.
% Compared with previous online approaches, our method considers 

\subsection{Implementation Details}
\label{sec:implemntation_details}

\noindent
\textbf{Detailed Training and Inference on VIP-Seg.} We use the COCO-pretrained model following~\cite{miao2022large}. The entire training process takes eight epochs. We adopt multiscale training where the scale ranges from 1.0 to 2.0 of the original image size, and then we apply a random crop of 720 $\times$ 720 patches. In particular, we perform the augmentation for each frame in the sampled subclips. For the inference, the subclip window size is set to six by default. We pad the remaining frames in the last subclip by repeating the last frame. We drop the padded results for evaluation.

\vspace{2mm}
\noindent
\textbf{Detailed COCO pretraining setting.} 
For COCO~\cite{coco_dataset} panoptic segmentation dataset pretraining, all the models are trained following original Mask2Former settings~\cite{zhang2021knet}. We adopt the multiscale training setting as previous work~\cite{detr} by resizing the input images such that the shortest side is at least 480 and 800 pixels, while the longest size is at most 1333. For data augmentation, we use the default large-scale jittering (LSJ) augmentation with a random scale sampled from the range 0.1 to 2.0 with the crop size of 1024 $\times$ 1024. For ResNet50~\cite{resnet} and Swin-base~\cite{liu2021swin} model, we train the model with 50 epochs following the original settings. For STDC model~\cite{STDCNet}, we train the model for 36 epochs.

\vspace{2mm}
\noindent
\textbf{Detailed Training and Inference on KITTI-STEP dataset.} For KITTI-STEP training, we follow previous Motion-Deeplab~\cite{STEP} and Video K-Net~\cite{li2022videoknet}, we adopt multiscale training where the scale ranges from 1.0 to 2.0 of origin images size. We then apply a random crop of 384 $\times$ 1248 patches. The total training epoch is set to 12. The inference procedure is the same as Cityscapes-VPS dataset. Following the previous works~\cite{STEP,li2022videoknet}, we also use Cityscapes dataset~\cite{cordts2016cityscapes} pretraining before training on STEP. Pretraining on Cityscapes STEP datasets further leads to 3\% VPQ and 2\% STQ improvements. We adopt the same inference pipeline as VIP-Seg, where we set the subclip window size to 2. We \textbf{do not} pre-train our model on the COCO dataset for a fair comparison.

\vspace{2mm}
\noindent
\textbf{Detailed Training and Inference on VSPW dataset.} We adopt nearly the same training pipeline for VSPW as VIP-Seg. The main difference is that we adopt longer training epochs, where we set the training epochs to 12, where we find about 1\% mIoU gain over different baselines. Moreover, we remove the tracking loss since we only focus on segmentation quality.

\vspace{2mm}
\noindent
\textbf{Detailed Training and Inference on Youtube-VIS-2019/2021 datasets.} We follow the same setting as Mask2Former-VIS~\cite{cheng2021mask2former_vis}. We train our models for 6k iterations, with a batch size of 16 for YouTubeVIS-2019 and 8k iterations for YouTubeVIS-2021.
All models are initialized with COCO instance segmentation models of Mask2Former. Different from previous SOTA VIS models~\cite{heo2022vita,seqformer,IDOL}, we only use YouTubeVIS training data, and \textit{do not} use COCO video images for data augmentation. Moreover, we also do not apply clip-wised copy-paste that is used in TubeFormer~\cite{kim2022tubeformer}. The same training procedure is adopted for the OVIS dataset as well.



\begin{table}[t]
  \centering
   \caption{\small \textbf{More Ablation on Tube-Wised Matching in Youtube-VIS dataset.}}
  \label{tab:tube_wisedmethod}
  \scalebox{0.90}{
  \begin{tabular}{l | c c  }
    \toprule[0.2em]
    \textbf{Settings} & Youtube-VIS-2019 & Youtube-VIS-2021 \\
    \toprule[0.2em]
     tube size=1 & 47.8 & 44.2 \\
     tube size=2 & 49.8 &  45.9  \\
     tube size=3 & 51.3 &  46.2  \\
     \rowcolor{gray!15} tube size=4 & 52.8  &  47.9  \\
     tube size=6 &  51.2 & 46.8  \\
    \bottomrule[0.1em]
  \end{tabular}
  }
\end{table}


\begin{table}[t]
  \centering
   \caption{\small \textbf{Ablation on Inference with Overlapped Frames.} We use the STDC-v1 backbone. The subclip window size is 6.}
  \label{tab:over_lap_window_size}
  \scalebox{0.90}{
  \begin{tabular}{l | c c c c }
    \toprule[0.2em]
    \textbf{Settings} & STQ & VPQ & SQ & FPS \\
    \toprule[0.2em]
     \rowcolor{gray!15} No Overlapping &  32.0 & 30.6 & 28.4 &  16.2 \\
     Overlapping=1 & 31.0  &  30.5 & 28.5 & 14.6 \\
     Overlapping=2 & 32.3  &  31.2 & 29.1 & 10.2 \\
     Overlapping=4 &  33.1  & 31.6  & 28.6 & 8.4 \\
    \bottomrule[0.1em]
  \end{tabular}
  }
\end{table}


\begin{table}[t]
  \centering
   \caption{\small \textbf{Ablation on Effect of COCO Pretraining.} We use the STDC-v1 backbone.}
  \label{tab:abl_coco_pretrain}
  \scalebox{0.90}{
  \begin{tabular}{l c | c c c }
    \toprule[0.2em]
    \textbf{Settings} & Method & STQ & VPQ & SQ \\
    \toprule[0.2em]
    w COCO pretrained & Video K-Net+ & 26.1 & 25.8  & 25.2 \\
   \rowcolor{gray!15} w/o COCO pretrained & Video K-Net+ & 12.4 & 12.4  & 18.3 \\
    \hline
    w COCO pretrained & Tube-Link & 32.0 & 30.6 & 28.4 \\
   \rowcolor{gray!15} w/o COCO pretrained & Tube-Link & 21.8 & 16.8 & 20.3 \\

    \bottomrule[0.1em]
  \end{tabular}
  }
\end{table}


\begin{table}[t]
  \centering
   \caption{\small \textbf{Ablation on Training Epochs.} We use the STDC-v1 backbone.}
  \label{tab:ablation_train_epoch}
  \scalebox{0.95}{
  \begin{tabular}{l | c c c }
    \toprule[0.2em]
    \textbf{Settings} & STQ & VPQ & SQ \\
    \toprule[0.2em]
    Epoch=4 & 29.2  & 28.1 & 26.5  \\
   \rowcolor{gray!15} Epoch=8 &  32.0 & 30.6 & 28.4 \\
    Epoch=12 & 31.6 & 30.8  & 29.1 \\
    \bottomrule[0.1em]
  \end{tabular}
  }
\end{table}


\begin{table*}[t]
  \centering
   \caption{\small \textbf{Detailed Results on the Youtube-VIS datasets (2019/2021).} We report the mAP metric. \textdagger~adopts COCO video pseudo labels~\cite{heo2022vita,heo2022generalized,heo2022vita}. Axial means using the extra Axial Attention~\cite{axialDeeplab}. Our method does not apply these techniques for simplicity.}
  \label{tab:ytvis_supp}
  \scalebox{0.95}{
  \begin{tabular}{l c | ccccc | cccccc }
    \toprule[0.2em]
    Method & Backbone  & \multicolumn{5}{c}{YTVIS-2019} & \multicolumn{5}{c}{YTVIS-2021} \\
    - & - &  AP        & AP$_{50}$ & AP$_{75}$ & AR$_1$    & AR$_{10}$ & AP        & AP$_{50}$ & AP$_{75}$ & AR$_1$    & AR$_{10}$ \\
    \toprule[0.2em]
VISTR~\cite{VIS_TR} & ResNet50 & 36.2 & 59.8 & 36.9 & 37.2 & 42.4 & - & - & - & - & - \\
EfficientVIS~\cite{EfficientVIS} & ResNet-50     & 37.9      & 59.7      & 43.0      & 40.3      & 46.6  & 34.0     & 57.5    & 37.3      & 33.8      & 42.5  \\
TubeFormer~\cite{kim2022tubeformer} & ResNet50 + Aixal & 47.5 & 68.7 & 52.1 & 50.2 & 59.0 & 41.2 & 60.4 & 44.7 & 40.4 & 54.0  \\
IFC~\cite{hwang2021video} & ResNet50 & 41.2      & 65.1      & 44.6      & 42.3      & 49.6                                                                   & 35.2      & 55.9      & 37.7      & 32.6      & 42.9 \\
Seqformer~\cite{seqformer} & ResNet50 & 47.4      & 69.8      & 51.8      & 45.5      & 54.8                                                                   & 40.5      & 62.4      & 43.7      & 36.1      & 48.1  \\
Mask2Former-VIS~\cite{cheng2021mask2former_vis}& ResNet50 & 46.4      & 68.0      & 50.0      & -         & -  & 40.6      & 60.9      & 41.8      & -         & -   \\
IDOL~\cite{IDOL} & ResNet50 & 46.4  & - & -   & -         & - & 43.9  & - & -   & -         & - \\
IDOL~\cite{IDOL} \textdagger & ResNet50 & 49.5 & - & -   & -         & -   & - & -   & -         & -   & -  \\
VITA~\cite{heo2022vita} \textdagger & ResNet50 & 49.8      & 72.6     & 54.5      & 49.4      & 61.0  & 45.7      & 67.4      & 49.5      & 40.9     & 53.6  \\
Min-VIS~\cite{huang2022minvis} &ResNet50& 47.4      & 69.0      & 52.1      & 45.7      & 55.7     & 44.2      & 66.0      & 48.1      & 39.2      & 51.7 \\
Cross-VIS~\cite{CrossVIS} & ResNet50 & 36.3      & 56.8      & 38.9      & 35.6      & 40.7   & 34.2      & 54.4      & 37.9      & 30.4      & 38.2 \\
VISOLO~\cite{VISOLO} & ResNet50 &  38.6      & 56.3      & 43.7      & 35.7      & 42.5    & 36.9      & 54.7      & 40.2      & 30.6      & 40.9  \\
GenVIS~\cite{heo2022generalized} & ResNet50 & {51.3}  & 72.0    & {57.8}  & {49.5}  & 60.0 & {46.3}  & 67.0 & {50.2} & 40.6 & 53.2 \\
\hline
Tube-Link & ResNet50 & 52.8 & 75.4 & 56.5 & 49.3 &59.9 & 47.9  & 70.0 & 50.2 & 42.3 & 55.2 \\ 
\hline
SeqFormer~\cite{seqformer} & Swin-large  & 59.3      & 82.1      & 66.4      & 51.7      & 64.4  & 51.8      & 74.6      & 58.2      & 42.8      & 58.1  \\
Mask2Former-VIS~\cite{cheng2021mask2former_vis} & Swin-large &  60.4      & 84.4      & 67.0      & -    & -   & 52.6      & 76.4      & 57.2      & -         & -     \\
IDOL~\cite{IDOL}  & Swin-large  & 61.5 & -   & -         & -   & -  & 56.1 & -   & -   & -   & -  \\ 
IDOL~\cite{IDOL}  & Swin-large \textdagger  & {64.3}      & {87.5}      & {71.0}      & 55.6      & 69.1 & 56.1      & 80.8      & 63.5      &  45.0      & 60.1 \\
VITA~\cite{heo2022vita} \textdagger & Swin-large & 63.0    & {86.9}      & 67.9      & {56.3}    & 68.1    & 57.5      & 80.6   & 61.0      & 47.7   & 62.6 \\ 
Min-VIS~\cite{huang2022minvis} & Swin-large & 61.6      & 83.3      & 68.6      & 54.8      & 66.6    & 55.3      & 76.6      & 62.0      & 45.9      & 60.8\\
\hline
Tube-Link & Swin-large  & 64.6 & 86.6 & 71.3 &  55.9 & 69.1 & 58.4 & 79.4  & 64.3 & 47.5 & 63.6 \\
    \bottomrule[0.2em]
  \end{tabular}
}
\end{table*}

\begin{table}[t]
  \centering
   \caption{\small \textbf{Results on the OVIS datasets.} We report the mAP metric. \textdagger~adopts COCO video pseudo labels. Axial means using the extra Axial Attention~\cite{axialDeeplab}. Our method does not apply these techniques for simplicity.}
  \label{tab:ovis}
  \scalebox{0.95}{
  \begin{tabular}{l c c c c  c}
    \toprule[0.2em]
    Method & AP   & AP$_{50}$ & AP$_{75}$ & AR$_1$   & AR$_{10}$ \\
    \toprule[0.2em]
    CrossVIS~\cite{CrossVIS} & 14.9      & 32.7          & 12.1          & 10.3          & 19.8 \\
    VISOLO~\cite{VISOLO} & 15.3      & 31.0          & 13.8          & 11.1          & 21.7  \\
    TeViT~\cite{TeViT}  & 17.4      & 34.9          & 15.0          & 11.2          & 21.8  \\
    VITA~\cite{heo2022vita} & 19.6      & 41.2          & 17.4          & 11.7          & 26.0 \\
    DeVIS~\cite{DeVIS} & 23.8      & 48.0          & 20.8          & -             & -     \\
    Min-VIS~\cite{huang2022minvis} & 25.0      & 45.5          & 24.0          & 13.9          & 29.7 \\
    IDOL~\cite{IDOL}  & 30.2      & 51.3          & 30.0          & 15.0          & 37.5 \\
    VITA~\cite{heo2022vita} \textdagger&  19.6 & 41.2  & 17.4 & 11.7 & 26.0 \\
\hline
Tube-Link & 29.5 & 51.5 &  30.2 & 15.5 & 34.5 \\
    \bottomrule[0.2em]
  \end{tabular}
}
\end{table}


\begin{table}[!t]
    \caption{More Experiment Results.}
    \begin{subtable}{.50\linewidth}
        \centering
        \footnotesize
        \caption{More results on ViP-Seg dataset.}
        \label{tab:more_result_for_tab1}
        \scalebox{0.50}{
        \setlength{\tabcolsep}{2.5mm}{\begin{tabular}{c c c c} 
            \toprule[0.1em]
            Method & STQ & SQ & AQ \\
            \midrule[0.1em]
            Video K-Net &  33.1 &  35.0 & 29.6 \\
          \rowcolor{gray!15} Video K-Net + tube matching (Ours)  & 34.7 & 36.8 & 30.8 \\
           Video K-Net + tube matching (IFC) & 33.3 & 35.7 & 28.4 \\
            \bottomrule[0.1em]
        \end{tabular}}}
    \end{subtable}%
    \begin{subtable}{.5\linewidth}
        \centering
        \footnotesize
        \caption{KITTI-STEP test set results.}
        \label{tab:test_dev_kitti_step}
        \scalebox{0.80}{
        \setlength{\tabcolsep}{2.5mm}{\begin{tabular}{c c c } 
            \toprule[0.1em]
            Method & Backbone & STQ  \\
            \midrule[0.1em]
            Motion-Deeplab & ResNe50 & 0.52 \\
            Video K-Net & ResNet50 & 0.59 \\
             Video K-Net & ResNet50 & 0.63 \\
             \hline
             Tube-Link & ResNet50 & 0.60 \\
             Tube-Link & Swin-base & 0.65 \\
            \bottomrule[0.1em]
        \end{tabular}}}
    \end{subtable}
\end{table}

\subsection{More Ablations and Experiment Results}
\label{sec:more_ablation_and_exp_results}

In this section, we first present more detailed ablations for Tube-Link. Then, we present more detailed results on several datasets, including VIS datasets~\cite{vis_dataset}, OVIS dataset~\cite{qi2022occluded} and VSPW test set~\cite{miao2021vspw}.




\vspace{2mm}
\noindent
\textbf{More Ablations on Effectiveness of Tube-Wised Matching.}
In Tab.~\ref{tab:tube_wisedmethod}, we present more detailed ablations on tube size in Youtube-VIS. Note that, for simplicity, the input subclip size is the same as the tube size. As we enlarge the tube size, we find a significant improvement in the final performance. After enlarging the size to 4, the performance is the best. Using a tube size of 6, the performance slightly degrades. However, it still performs better than single-frame matching. All the models are trained under the same tube size (default is 2). The findings also verify our motivation for using clip-level matching, which shares similar findings on the VIP-Seg dataset in the main paper.



\vspace{2mm}
\noindent
\textbf{Inference with Overlapped Frames in VIP-Seg.} In Tab.~\ref{tab:over_lap_window_size}, we explore the effect of the overlapping size for nearby windows. As shown in that table, increasing the overlapping size leads to better performance for all three metrics: VPQ, STQ, and SQ. This is because we can use multiple frames twice, which leads to more consistent segmentation results. Moreover, instances in smaller windows are easier to be tracked. However, to save computation costs and increase inference speed, we do not introduce overlapping during inference. All the results in the main paper use non-overlapping inference. 


\vspace{2mm}
\noindent
\textbf{Effect of COCO Pretraining in VIP-Seg.} In Tab.~\ref{tab:abl_coco_pretrain}, we show the effect of COCO pretraining on both Video K-Net+ and our Tube-Link. From the table, we can see that COCO pretraining plays an important role for VIP-Seg datasets, which shares the same conclusion with previous work~\cite{li2022videoknet,miao2022large}. Without COCO pretraining, both Video K-Net+ and Tube-Link drop a lot. However, as shown in the gray area, our method \textit{without} COCO pretraining outperforms the Video K-Net+ baseline by a large margin, where we still achieve over 8\% STQ gain and 14\% VPQ gain. The results suggest the effectiveness of our framework on better usage of temporal information.


\vspace{2mm}
\noindent
\textbf{Effect of Training Epoch on VIP-Seg.} We perform ablation on training epochs as in Tab.~\ref{tab:ablation_train_epoch}. With more training epochs, we do not observe performance gain with the COCO pre-trained model due to the overfitting issues. We use eight training epochs by default for all models. 


\begin{figure*}[t!]
	\centering
	\includegraphics[width=0.85\linewidth]{.//figs/yt19_clip_vis_comparison.pdf}
	\caption{Visual Comparison Results from Tube-Link with ResNet50 backbone. Our method (middle) achieves consistent segmentation and better segmentation/tracking results than the Mask2Former-VIS baseline (top). We also visualize the difference maps (bottom). \textbf{Best viewed by zooming in.}}
	\label{fig:yt_vis_2019_comparison}
\end{figure*}

\begin{figure*}[t!]
	\centering
	\includegraphics[width=0.85\linewidth]{./figs/more_vis_results_vip_seg.pdf}
	\caption{More Visual Results from Tube-Link with ResNet50 backbone. Our method (top) achieves consistent segmentation and better tracking results than the Video K-Net+ baseline (bottom). \textbf{Best viewed by zooming in.}}
	\label{fig:more_vis_vip_seg}
\end{figure*}


\begin{figure*}[t!]
	\centering
	\includegraphics[width=0.85\linewidth]{./figs/tb_sup_kitti_step.pdf}
	\caption{Visual Results from Tube-Link with ResNet50 backbone on the KITTI-STEP dataset. \textbf{Best viewed by zooming in.}}
	\label{fig:more_vis_kitti_step}
\end{figure*}


\begin{figure*}[t!]
	\centering
	\includegraphics[width=0.85\linewidth]{./figs/FailureCases.pdf}
	\caption{Visual Results on Failure Cases of Tube-Link. (a), Remote objects lead to ID switches and inferior segmentation results. (b), Heavy occlusion leads to an ID switch. (c), Segmentation consistency problems caused by camera motion.}
	\label{fig:failure_cases}
\end{figure*}

\begin{table}[!t]
        \centering
        \footnotesize
        \caption{Effect Of Quasi-Dense Tracker (Results on Youtube-VIS-2019 validation set).}
        \label{tab:effect_of_quasi_dense_tracker}
        \scalebox{0.80}{
        \setlength{\tabcolsep}{2.5mm}{\begin{tabular}{c c c c} 
            \toprule[0.1em]
            Method & Naive Tracker & Quani-Dense Tracker &  mAP \\
                  \midrule[0.1em]
            MinVIS  &  \checkmark & -  & 47.4 \\
            MinVIS  &  - & \checkmark  & 48.0 (+0.6) \\
            MiniVIS + tube matching &  \checkmark & - & 48.8 (+1.4) \\
            \hline
            Tube-Link &  \checkmark & - &  52.6 (-0.2) \\
            Tube-Link &  - & \checkmark & 52.8 \\
            \bottomrule[0.1em]
        \end{tabular}}}
\end{table}

\noindent
\textbf{Impact of Quasi-Dense Tracker.}
We adopt the same quasi-dense tracker for all experiments in the main paper, and we can achieve 3.0\% VPQ improvement upon the baseline. In Tab.~\ref{tab:effect_of_quasi_dense_tracker}, we perform an extra experiment by replacing our tracker with a naive tracker used in MinVIS, where we only found 0.2\% mAP drop. This proves the robustness and generalizability of Tube-Link. In contrast, we add the quasi-dense tracker to MinVIS, and we only find 0.6\% mAP improvements. Directly extending a method with tube matching leads to more improvements. The results also indicate that the effect of the tracker is not apparent on the Youtube-VIS dataset, since the instance number is limited and occlusion is not heavy. Thus, we adopt \textit{simple tube-matching} for VIS datasets. 

\vspace{2mm}
\noindent
\textbf{Detailed Results Youtube-VIS.} In Tab.~\ref{tab:ytvis_supp}, we report the detailed results on Youtube-VIS-2019 and Youtube-VIS-2021 datasets. We follow the baseline method, Mask2Former-VIS~\cite{cheng2021mask2former_vis}. 
As shown in that table, our method achieves all the best metrics on both datasets \textit{without COCO video joint training or clip-wised copy-paste}. 


\noindent
\textbf{More Results on Test Set.}. Moreover, we also report our results on the KITTI-STEP test set. As shown in Tab.~\ref{tab:test_dev_kitti_step}, Our method can still achieve better results. 


\vspace{2mm}
\noindent
\textbf{Detailed Results on OVIS.} In Tab.~\ref{tab:ovis}, we also report our model results on OVIS. Again, without bells and whistles, our method achieves comparable results with IDOL. We use the ResNet50 backbone for a fair comparison.









\subsection{Visual Results}
\label{sec:vis_results}

\noindent
\textbf{Visual Comparison on Youtube-VIS-2019 dataset.} In Fig.~\ref{fig:yt_vis_2019_comparison}, we compare our Tube-Link with strong baseline Mask2Former-VIS with the same ResNet50 backbone. Our methods achieve more consistent tracking and segmentation results in two examples.


\noindent
\textbf{More Visual Results on VIP-Seg Dataset.} In Fig.~\ref{fig:more_vis_vip_seg}, we present more visual examples on our Tube-Link. Compared with the Video K-Net+ baseline, our method achieves better segmentation and tracking consistency. 


\noindent
\textbf{Visual Results on KITTI-STEP Dataset.} In Fig.~\ref{fig:more_vis_kitti_step}, we present visual results on the KITTI-STEP dataset, where we achieve consistent segmentation and tracking on the driving scene. 



\noindent
\textbf{Failure Cases Analysis.} In Fig.~\ref{fig:failure_cases}, we show several failure cases on the KITTI-STEP and VIP-Seg datasets using our best models. We observe three error sources: (1). remote and small objects. (2). heavy occlusion. (3). segmentation consistency caused by camera motion. We will handle these issues in future work.



{\small
\bibliographystyle{ieee_fullname}
\bibliography{egbib}
}

\end{document}