\renewcommand{\tabcolsep}{.5pt}{
\def\sym#1{\ifmmode^{#1}\else\(^{#1}\)\fi}
\begin{tabular*}{\hsize}{@{\hskip\tabcolsep\extracolsep\fill}l*{3}{c}}
\toprule
\addlinespace
\multicolumn{1}{c}{\parbox{3cm}{Establishment size}} & \multicolumn{1}{c}{\parbox{2cm}{Annual labor expenditure}} & \multicolumn{1}{c}{\parbox{2cm}{Annual COGS}} & \multicolumn{1}{c}{\parbox{2cm}{Labor-to-wholesale expenditure share}} \\
\addlinespace

\midrule

$<$ 20 employees       &  \$180,593 & \multirow{2}{2cm}{\$860,330} & 0.21 \\

\addlinespace
$<$ 500 employees & \$253,352 & & 0.29 \\

\addlinespace

\bottomrule

\end{tabular*}
\begin{minipage}[h]{\textwidth}
\medskip
\small \emph{Notes:} This table compares labor expenditure and wholesale expenditure for cannabis retail establishments in Washington state for the year 2019. Average annual payroll is calculated using the 2019 SUSB and pertains to establishments in Washington state belonging to "All Other Miscellaneous Store Retailers (except Tobacco Stores), including Marijuana Stores, Medicinal and Recreational establishments", i.e. NAICS 453998. Annual COGS is the average wholesale expenditure for cannabis retailers for the year 2019. This is based on the universe of wholesale transactions made by cannabis retailers in the LCB data (including wholesale sales from processor-only licenses). Data from SUSB and Top Shelf Data.

\end{minipage}

}
