

\prAtEndRestatexxi*

\makeatletter\Hy@SaveLastskip\label{proofsection:prAtEndxxi}\ifdefined\pratend@current@sectionlike@label\immediate\write\@auxout{\string\gdef\string\pratend@section@for@proofxxi{\pratend@current@sectionlike@label}}\fi\Hy@RestoreLastskip\makeatother\begin{proof}[Proof]\phantomsection\label{proof:prAtEndxxi}Notice that \begin {align*} \gamma _1\left (x\right ) &= \frac { e^x }{ e^x } = 1 \\ \gamma _2\left (x\right ) &= \frac { e^x }{ e^{2x} } = e^{-x} \\ \gamma _3\left (x\right ) &= e^x e^x = e^{2x}. \nonumber \end {align*} Therefore, \begin {enumerate}[label=(\roman *)] \item holds trivially. \item holds trivially. \item The Lipschitzness of \(\gamma _2\left (x\right )\) is equivalent to whether \(e^x\) is Lipschitness, which it is not. \item The Lipschitzness of \(\gamma _3\left (x\right )\) is equivalent to whether \(e^{2x}\) is Lipschitness, which it is not. \end {enumerate}\end{proof}

\prAtEndRestatexxii*

\makeatletter\Hy@SaveLastskip\label{proofsection:prAtEndxxii}\ifdefined\pratend@current@sectionlike@label\immediate\write\@auxout{\string\gdef\string\pratend@section@for@proofxxii{\pratend@current@sectionlike@label}}\fi\Hy@RestoreLastskip\makeatother\begin{proof}[Proof]\phantomsection\label{proof:prAtEndxxii}First, the log probability density of the half-normal distribution is given as \begin {align} \log f\left (x; \sigma \right ) = \left (\alpha - 1\right ) \log x - \beta x \nonumber \end {align} where \(\log Z\) is a constant normalizer. The derivatives of the inverse gamma distribution with a bijection \(\psi ^{-1}\) is given as \begin {align*} &\frac {d \log f\left (\psi ^{-1}\left (x\right ); \sigma \right )}{dx} \\ &\;= \frac {d \log f\left (\psi ^{-1}\left (x\right ); \sigma \right )}{d\psi } \frac {d \psi ^{-1}}{dx} \\ &\;= \frac {d \psi ^{-1}}{dx} \left ( \left (\alpha - 1\right ) \frac {1}{\psi ^{-1}\left (x\right )} - \beta \right ) \\ &\;= \left (\alpha - 1 \right ) \frac { {\left ( \psi ^{-1} \right )}^{\prime } \left (x\right ) }{\psi ^{-1}\left (x\right )} + \beta \psi ^{-1}\left (x\right ). \\ &\;= \left (\alpha - 1 \right ) \gamma _1\left (x\right ) + \beta {\left ( \psi ^{-1} \right )}^{\prime } \left (x\right ) \end {align*} Thus, the smoothness is euquivalent to the Lipschitzness of \(\gamma _3\left (x\right )\). \par By the triangle inequlity, the two-way inequality \begin {alignat*}{2} &\abs { \left (\alpha - 1 \right ) \abs { \gamma _1\left (x\right ) - \gamma _1\left (y\right ) } - \beta \abs { {\left ( \psi ^{-1} \right )}^{\prime } \left (x\right ) - {\left ( \psi ^{-1} \right )}^{\prime } \left (y\right ) } } \\ &\quad \leq \abs { \frac {d \log f\left (\psi ^{-1}\left (x\right ); \alpha , \beta \right )}{dx} - \frac {d \log f\left (\psi ^{-1}\left (y\right ); \alpha , \beta \right )}{dy} } \\ &\quad \quad \leq \left ( \alpha - 1 \right ) \abs { \gamma _1\left (x\right ) - \gamma _1\left (y\right ) } + \beta \abs { {\left ( \psi ^{-1} \right )}^{\prime } \left (x\right ) - {\left ( \psi ^{-1} \right )}^{\prime } \left (y\right ) } \end {alignat*} holds. The upper bound proves the implication (\(\Rightarrow \)), while the lower bounds (\(\Leftarrow \)) proves the converse by contraposition.\end{proof}

\prAtEndRestatexxiii*

\makeatletter\Hy@SaveLastskip\label{proofsection:prAtEndxxiii}\ifdefined\pratend@current@sectionlike@label\immediate\write\@auxout{\string\gdef\string\pratend@section@for@proofxxiii{\pratend@current@sectionlike@label}}\fi\Hy@RestoreLastskip\makeatother\begin{proof}[Proof]\phantomsection\label{proof:prAtEndxxiii}First, the log probability density of an inverse gamma distribution is given as \begin {align} \log f\left (x; \alpha , \beta \right ) = \left (-\alpha - 1\right ) \log x - \frac {\beta }{x} + \log Z, \nonumber \end {align} where \(\log Z\) is a constant normalizer. The derivatives of the inverse gamma distribution with a bijection \(\psi ^{-1}\) is given as \begin {align*} &\frac {d \log f\left (\psi ^{-1}\left (x\right ); \alpha , \beta \right )}{dx} \\ &\;= \frac {d \log f\left (\psi ^{-1}\left (x\right ); \alpha , \beta \right )}{d\psi } \frac {d \psi ^{-1}}{dx} \\ &\;= \frac {d \psi ^{-1}}{dx} \left ( \left ( - \alpha - 1 \right ) \frac {1}{\psi ^{-1}\left (x\right )} + \beta \frac {1}{{\psi ^{-1}\left (x\right )}^2} \right ) \\ &\;= \left ( - \alpha - 1 \right ) \frac { {\left ( \psi ^{-1} \right )}^{\prime } \left (x\right ) }{\psi ^{-1}\left (x\right )} + \beta \frac { {\left ( \psi ^{-1} \right )}^{\prime } \left (x\right ) }{{\psi ^{-1}\left (x\right )}^2}. \\ &\;= \left ( - \alpha - 1 \right ) \gamma _1\left (x\right ) + \beta \gamma _2\left (x\right ). \end {align*} \par By the triangle inequlity, the two-way inequality \begin {alignat*}{2} &\abs { \left ( \alpha + 1 \right ) \abs { \gamma _1\left (x\right ) - \gamma _1\left (y\right ) } - \beta \abs { \gamma _2\left (x\right ) - \gamma _2\left (y\right ) } } \\ &\quad \leq \abs { \frac {d \log f\left (\psi ^{-1}\left (x\right ); \alpha , \beta \right )}{dx} - \frac {d \log f\left (\psi ^{-1}\left (y\right ); \alpha , \beta \right )}{dy} } \\ &\quad \quad \leq \left ( \alpha + 1 \right ) \abs { \gamma _1\left (x\right ) - \gamma _1\left (y\right ) } + \beta \abs { \gamma _2\left (x\right ) - \gamma _2\left (y\right ) } \end {alignat*} holds. The upper bound proves the implication (\(\Rightarrow \)), while the lower bounds (\(\Leftarrow \)) proves the converse by contraposition.\end{proof}

\prAtEndRestatexxiv*

\makeatletter\Hy@SaveLastskip\label{proofsection:prAtEndxxiv}\ifdefined\pratend@current@sectionlike@label\immediate\write\@auxout{\string\gdef\string\pratend@section@for@proofxxiv{\pratend@current@sectionlike@label}}\fi\Hy@RestoreLastskip\makeatother\begin{proof}[Proof]\phantomsection\label{proof:prAtEndxxiv}First, the log probability density of the half-normal distribution is given as \begin {align} \log f\left (x; \sigma \right ) = -\frac {x^2}{2\sigma ^2} + \log Z, \nonumber \end {align} where \(\log Z\) is a constant normalizer. The derivatives of the inverse gamma distribution with a bijection \(\psi ^{-1}\) is given as \begin {align*} &\frac {d \log f\left (\psi ^{-1}\left (x\right ); \sigma \right )}{dx} \\ &\;= \frac {d \log f\left (\psi ^{-1}\left (x\right ); \sigma \right )}{d\psi } \frac {d \psi ^{-1}}{dx} \\ &\;= \frac {d \psi ^{-1}}{dx} \left ( -\frac { \psi ^{-1}\left (x\right ) }{\sigma ^2} \right ) \\ &\;= -\frac {1}{\sigma ^2} {\left ( \psi ^{-1} \right )}^{\prime }\left (x\right ) \psi ^{-1}\left (x\right ) \\ &\;= -\frac {1}{\sigma ^2} \gamma _3\left (x\right ) \end {align*} Thus, the smoothness is euquivalent to the Lipschitzness of \(\gamma _3\left (x\right )\).\end{proof}
