

\prAtEndRestateii*

\makeatletter\Hy@SaveLastskip\label{proofsection:prAtEndii}\ifdefined\pratend@current@sectionlike@label\immediate\write\@auxout{\string\gdef\string\pratend@section@for@proofii{\pratend@current@sectionlike@label}}\fi\Hy@RestoreLastskip\makeatother\begin{proof}[Proof]\phantomsection\label{proof:prAtEndii}\begin {alignat}{2} &\norm {\nabla _{\vlambda } f\left ( \vt _{\vlambda }\left (\vu \right ) \right ) }_2^2 \nonumber \\ &\;= {\left ( \frac { \partial \vt _{\vlambda }\left (\vu \right ) }{ \partial \vlambda } \nabla f\left ( \vt _{\vlambda }\left (\vu \right ) \right ) \right )}^{\top } \frac { \partial \vt _{\vlambda }\left (\vu \right ) }{ \partial \vlambda } \nabla f\left ( \vt _{\vlambda }\left (\vu \right ) \right ) \nonumber \\ &\;= {\nabla f^{\top }\left ( \vt _{\vlambda }\left (\vu \right ) \right )} {\left ( \frac { \partial \vt _{\vlambda }\left (\vu \right ) }{ \partial \vlambda } \right )}^{\top } \frac { \partial \vt _{\vlambda }\left (\vu \right ) }{ \partial \vlambda } \nabla f\left ( \vt _{\vlambda }\left (\vu \right ) \right ) \nonumber \\ &\;= {\vg _{f}^{\top }} {\left ( \frac { \partial \vt _{\vlambda }\left (\vu \right ) }{ \partial \vlambda } \right )}^{\top } \frac { \partial \vt _{\vlambda }\left (\vu \right ) }{ \partial \vlambda } \vg _{f}.\label {eq:thm:variational_gradient_norm_identity_eq1} \end {alignat} \par \paragraph {Proof for Full-Rank Cholesky (\cref {def:fullrank})} \par Let \(p\) denote the number of scalar variational parameters such that \(\vlambda = (\lambda _1, \ldots , \lambda _p)\). Then, \begin {alignat*}{2} &{\left ( \frac { \partial \vt _{\vlambda }\left (\vu \right ) }{ \partial \vlambda } \right )}^{\top } \frac { \partial \vt _{\vlambda }\left (\vu \right ) }{ \partial \vlambda } \\ &\;= \sum ^{d}_{i=1} \frac { \partial \vt _{\vlambda }\left (\vu \right ) }{ \partial m_i } {\left ( \frac { \partial \vt _{\vlambda }\left (\vu \right ) }{ \partial m_i } \right )}^{\top } + \sum ^{d}_{i=1} \sum ^{d}_{j=1} \frac { \partial \vt _{\vlambda }\left (\vu \right ) }{ \partial \lambda _{C_{ij}} } {\left ( \frac { \partial \vt _{\vlambda }\left (\vu \right ) }{ \partial \lambda _{C_{ij}} } \right )}^{\top }, \end {alignat*} where \(\lambda _{C_{ij}}\) denote the parameter responsible for the \(ij\)-th entry of \(\mC \), \(C_{ij}\). \par For the derivatives with respect to \(m_i\) and \(C_{ij}\), \citet {domke_provable_2020, domke_provable_2019} show that \begin {alignat}{2} \frac {\partial \vt _{\vlambda }\left (\vu \right ) }{ \partial m_i } &= \ve _i \quad \frac {\partial \vt _{\vlambda }\left (\vu \right ) }{ \partial C_{ij} } &= \ve _i u_j,\label {eq:covariance_derivative} \end {alignat} where \(\ve _i\) is the unit basis of the \(i\)th component. \par Therefore, \begin {alignat}{2} &{\left ( \frac { \partial \vt _{\vlambda }\left (\vu \right ) }{ \partial \vlambda } \right )}^{\top } \frac { \partial \vt _{\vlambda }\left (\vu \right ) }{ \partial \vlambda } \nonumber \\ &\;= \sum ^{d}_{i=1} \ve _i \ve _i^{\top } + \sum ^{d}_{i=1} \sum ^{d}_{j=1} \frac { \partial \vt _{\vlambda }\left (\vu \right ) }{ \partial \lambda _{C_{ij}} } {\left ( \frac { \partial \vt _{\vlambda }\left (\vu \right ) }{ \partial \lambda _{C_{ij}} } \right )}^{\top } \nonumber \\ &\;= \mI + \underbrace { \sum ^{d}_{i=1} \frac { \partial \vt _{\vlambda }\left (\vu \right ) }{ \partial \lambda _{C_{ii}} } {\left ( \frac { \partial \vt _{\vlambda }\left (\vu \right ) }{ \partial \lambda _{C_{ii}} } \right )}^{\top } }_{\text {diagonal of \(\mC \)}} \nonumber \\ &\qquad + \underbrace { \sum ^{d}_{i=1} \sum ^{d}_{j=1, j \neq i} \frac { \partial \vt _{\vlambda }\left (\vu \right ) }{ \partial \lambda _{C_{ij}} } {\left ( \frac { \partial \vt _{\vlambda }\left (\vu \right ) }{ \partial \lambda _{C_{ij}} } \right )}^{\top }}_{\text {off-diagonal of \(\mC \)}}, \label {eq:thm:variational_gradient_norm_identity_eq2} \end {alignat} leaving us with the derivatives of the scale term. \par The gradient with respect to \(\lambda _{C_{ij}}\), however, depends on the parameterization. That is, \begin {alignat}{2} \frac { \partial \vt _{\vlambda }\left (\vu \right ) }{ \partial \lambda _{C_{ij}} } &= \frac { \partial \vt _{\vlambda }\left (\vu \right ) }{ \partial C_{ij} } \frac { \partial C_{ij} }{ \partial \lambda _{C_{ij}} } &= \ve _i u_j \frac { \partial C_{ij} }{ \partial \lambda _{C_{ij}} }. \label {eq:thm:variational_gradient_norm_identity_covderivative} \end {alignat} \par For the diagonal elements, \(\lambda _{C_{ii}} = d_i\). Thus, \begin {align} \frac { \partial C_{ii} }{ \partial d_i } = \frac { \partial \phi _+\left (d_i\right ) }{ \partial d_i } = \phi _+^{\prime }\left (d_i\right ). \label {eq:thm:variational_gradient_norm_identity_diag} \end {align} And for the off-diagonal elements, \(\lambda _{L_{ij}} = L_{ij}\), and \begin {align} \frac { \partial C_{ij} }{ \partial L_{ij} } = 1. \label {eq:thm:variational_gradient_norm_identity_offdiag} \end {align} \par Plugging \cref {eq:thm:variational_gradient_norm_identity_diag,eq:thm:variational_gradient_norm_identity_offdiag,eq:thm:variational_gradient_norm_identity_covderivative} into \cref {eq:thm:variational_gradient_norm_identity_eq2}, \begin {alignat}{2} &{\left ( \frac { \partial \vt _{\vlambda }\left (\vu \right ) }{ \partial \vlambda } \right )}^{\top } \frac { \partial \vt _{\vlambda }\left (\vu \right ) }{ \partial \vlambda } \nonumber \\ &\;= \mI + \underbrace { \sum ^{d}_{i=1} {\left ( u_j \phi _+^{\prime }\left (d_i\right ) \right )}^2 \ve _i \ve _i^{\top } }_{\text {diagonal of \(\mC \)}} + \underbrace { \sum ^{d}_{i=1} \sum ^{d}_{j=1, j \neq i} u_j^2 \, \ve _i \ve _i^{\top } }_{\text {off-diagonal of \(\mC \)}} \nonumber \\ &\;= \mI + \sum ^{d}_{i=1} u_i^2 {\left (\phi _+^{\prime }\left (d_i\right ) \right )}^2 \ve _i \ve _i^{\top } + \sum ^{d}_{i=1} \sum ^{d}_{j=1} u_j^2 \, \ve _i \ve _i^{\top } - \vu \vu ^{\top } \nonumber \\ &\;= \mI + \left (\vu \vu ^{\top }\right ) \Phi \Phi + \norm {\vu }_2^2 \, \mI - \vu \vu ^{\top } \nonumber \\ &\;= \left ( 1 + \norm {\vu }_2^2 \right )\mI + \left (\vu \vu ^{\top }\right ) \left ( \Phi \Phi - \mI \right ), \label {eq:eq:thm:variational_gradient_norm_identity_jacinner} \end {alignat} where \(\Phi \) is a diagonal matrix containing the derivatives of \(\phi _+\) as \begin {align*} \Phi = \mathrm {diag}\left ( \left [ \phi _+^{\prime }\left (d_1\right ), \ldots , \phi _+^{\prime }\left (d_d\right ) \right ] \right ). \end {align*} \par Coming back to \cref {eq:thm:variational_gradient_norm_identity_eq1}, \begin {alignat}{2} &\norm {\nabla _{\vlambda } f\left ( \vt _{\vlambda }\left (\vu \right ) \right )} \nonumber \\ &\;= \vg _f^{\top } {\left ( \frac { \partial \vt _{\vlambda }\left (\vu \right ) }{ \partial \vlambda } \right )}^{\top } \frac { \partial \vt _{\vlambda }\left (\vu \right ) }{ \partial \vlambda } \vg \nonumber \\ &\;= \vg _{f}^{\top } \left \{\, \left (1 + \norm {\vu }_2^2 \right )\mI + \left (\vu \vu ^{\top }\right ) \left ( \Phi ^2 - \mI \right ) \,\right \} \vg _{f} \nonumber \\ &\;= \left (1 + \norm {\vu }_2^2\right ){\lVert \vg _{f} \rVert }_2^2 + \vg _{f}^{\top } \left (\vu \vu ^{\top }\right ) \left ( \Phi ^2 - \mI \right ) \vg _{f} \nonumber \\ &\;= \left (1 + \norm {\vu }_2^2\right ){\lVert \vg _{f} \rVert }_2^2 + \mathrm {tr}\left ( \vg _{f} \vg _{f}^{\top }\, \vu \vu ^{\top } \left ( \Phi ^2 - \mI \right ) \right ). \label {eq:eq:thm:variational_gradient_norm_identity_conclusion} \end {alignat} \par \paragraph {Proof for Mean-field (\cref {def:meanfield})} For the mean-field variational family, the covariance has only diagonal elements. Therefore, \cref {eq:eq:thm:variational_gradient_norm_identity_jacinner} becomes \begin {alignat}{2} {\left ( \frac { \partial \vt _{\vlambda }\left (\vu \right ) }{ \partial \vlambda } \right )}^{\top } \frac { \partial \vt _{\vlambda }\left (\vu \right ) }{ \partial \vlambda } &= \mI + \left (\vu \vu ^{\top }\right ) \Phi ^2, \nonumber \end {alignat} and \cref {eq:eq:thm:variational_gradient_norm_identity_conclusion} becomes \begin {alignat}{2} \norm {\nabla _{\vlambda } f\left ( \vt _{\vlambda }\left (\vu \right ) \right )} &= \vg _{f}^{\top } \left ( \mI + \left (\vu \vu ^{\top }\right ) \Phi ^2 \right ) \vg _{f} \nonumber \\ &= {\lVert \vg _{f} \rVert }_2^2 + \mathrm {tr}\left ( \vg _{f} \vg _{f}^{\top } \, \vu \vu ^{\top } \Phi ^2 \right ). \nonumber \end {alignat}\end{proof}
