
\begin{assumption}\label{assumption:bijector_lipschitz_condition1}
  Let \(\phi_{+}\left(x\right)\) be a bijector, where \(x \mapsto \nicefrac{\phi^{\prime}\left(x\right)}{\phi\left(x\right)} \) is \(L_{+}\)-Lipschitz.
\end{assumption}

\begin{proposition}
  Let \(\phi_{+}\left(x\right) = e^x\). 
  Then, \(\phi_{+}\left(x\right)\) satisfies \cref{assumption:bijector_lipschitz_condition1} with any \(L_{+} = \epsilon > 0\).
\end{proposition}
\begin{proof}
  Since \(\nicefrac{\phi^{\prime}_+\left(x\right)}{\phi_+\left(x\right)} = 1\), the conclusion follows from 
  \( \abs{\nicefrac{\phi^{\prime}_+\left(x\right)}{\phi_+\left(x\right)} - \nicefrac{\phi^{\prime}_+\left(y\right)}{\phi_+\left(y\right)}} = 0.\)
\end{proof}

\begin{proposition}
  Let \(\phi_{+}\left(x\right) = \mathrm{softplus}\left(x\right)\). 
  Then, \(\phi_{+}\left(x\right)\) satisfies \cref{assumption:bijector_lipschitz_condition1} with \(L_{+} = 2\).
\end{proposition}
\begin{proof}
  Since \(\mathrm{softplus}\left(x\right) = \log\left(1 + e^x\right)\), 
  \begin{align*}
    \phi_{+}^{\prime}\left(x\right) = \frac{e^x}{1 + e^x} = \sigma\left(x\right),
  \end{align*}
  where \(\sigma\left(x\right)\) is known as the sigmoid/logistic function with the derivative \(\sigma^{\prime}\left(x\right) = \sigma\left(x\right) \left(1 - \sigma\left(x\right)\right).\)
  Now, by the Mean Value Theorem, 
  \begin{align}
    \abs{ \frac{\phi^{\prime}_{+}\left(x\right)}{\phi\left(x\right)} - \frac{\phi^{\prime}_{+}\left(y\right)}{\phi\left(y\right)} }
    \leq
    \abs{ \frac{d}{dx} \frac{\phi^{\prime}_{+}\left(x\right)}{\phi\left(x\right)} }
    \abs{
      x - y
    }.
  \end{align}
  Now, 
  \begin{align*}
    \abs{ \frac{d}{dx} \frac{\phi^{\prime}_{+}\left(x\right)}{\phi\left(x\right)} }
    &=
    \abs{
      \phi^{\prime\prime}_{+}\left(x\right) \frac{1}{\phi_{+}\left(x\right)}
      -
      {\left( \phi^{\prime}_{+}\left(x\right) \right)}^2 \frac{1}{\phi_{+}^2\left(x\right)}
    }
    \\
    &=
    \frac{1}{\phi_{+}\left(x\right)} \abs{ \phi^{\prime\prime}_{+}\left(x\right) - \frac{{\left( \phi^{\prime}_{+}\left(x\right) \right)}^2}{\phi_{+}\left(x\right)}
    }
    \\
    &=
    \frac{1}{\phi_{+}\left(x\right)} \abs{
      \sigma\left(x\right) \left( 1 - \sigma\left(x\right) \right) - \frac{\sigma^2\left(x\right)}{\phi_{+}\left(x\right)}
    }
    \\
    &=
    \frac{\sigma\left(x\right)}{\phi_{+}\left(x\right)} \abs{
       1 - \sigma\left(x\right) - \frac{\sigma\left(x\right)}{\phi_{+}\left(x\right)}
    }
    \\
    &\leq
    \frac{\sigma\left(x\right)}{\phi_{+}\left(x\right)} \abs{
       1 - \sigma\left(x\right) 
    }
    +
    {\left(
    \frac{\sigma\left(x\right)}{\phi_{+}\left(x\right)} 
    \right)}^2
    \\
    &\leq
    \frac{\sigma\left(x\right)}{\phi_{+}\left(x\right)} 
    +
    {\left(
    \frac{\sigma\left(x\right)}{\phi_{+}\left(x\right)} 
    \right)}^2,
  \end{align*}
  since \(0 \leq \sigma\left(x\right) \leq 1\), \(0 \leq \phi_{+}\left(x\right)\).
  Notice that 
  \begin{align*}
    \frac{\sigma\left(x\right)}{\phi_{+}\left(x\right)} 
    =
    \frac{e^x}{1 + e^x} 
    \frac{1}{ \log\left(1 + e^x\right) } 
  \end{align*}
  is monotonically decreasing.
  Therefore, \(\nicefrac{\sigma\left(x\right)}{\phi_{+}\left(x\right)} \leq \lim_{x \rightarrow - \infty} \nicefrac{\sigma\left(x\right)}{\phi_{+}\left(x\right)} = 1\).

  Thus,
  \begin{align*}
    \abs{ \frac{\phi^{\prime}_{+}\left(x\right)}{\phi\left(x\right)} - \frac{\phi^{\prime}_{+}\left(y\right)}{\phi\left(y\right)} }
    &\leq
    \abs{ \frac{d}{dx} \frac{\phi^{\prime}_{+}\left(x\right)}{\phi\left(x\right)} }
    \abs{
      x - y
    }
    \\
    &\leq
    \abs{
      \frac{\sigma\left(x\right)}{\phi_{+}\left(x\right)} 
      +
      {\left(
        \frac{\sigma\left(x\right)}{\phi_{+}\left(x\right)} 
        \right)}^2
    }
    \abs{
      x - y
    }
    \\
    &\leq
    2 \,
    \abs{
      x - y
    }.
  \end{align*}
\end{proof}

\begin{proposition}
  Let \(q_{\vlambda}\) belong to the location scale family as \cref{def:family}, where \(\mC \in \mathbb{R}^{d \times d}\) is formed through the Cholesky factor parameterization.
  \(\vlambda \mapsto H\left[q_{\vlambda}\right]\) is \(L\)-smooth if and only if the bijector \(\phi_+\) satisfies \cref{assumption:bijector_lipschitz_condition1}, where the smoothness constant is \(L = d L_{+}\).
\end{proposition}
\begin{proof}
  For members of the location scale family with parameters \(\left(\vm, \mC\right)\), the entropy and its gradient are given as
  \begin{align*}
    \mathrm{H}\left[q_{\vlambda}\right] = \log \abs{ \mC } \quad\text{and}\quad \nabla_{\mC} \mathrm{H}\left[q_{\vlambda}\right] = - \mC^{-\top}.
  \end{align*}

  Let \(\mC\) form a Cholesky factor.
  Then, since the determinant of a triangular matrix is the product of its diagonal elements,
  \begin{align*}
    \mathrm{H}\left[q_{\vlambda}\right] = \log \abs{\mC} = \sum_{i=1}^d \log C_{ii} = \sum_{i=1}^d \log \phi_{+}\left( d_{i} \right).
  \end{align*}
  It is apparent that the derivatives of the entropy term are now
  \begin{align*}
    \frac{\partial \mathrm{H}\left[q_{\vlambda}\right]}{ \partial d_i }
    =
    \frac{\partial \log \phi_{+}\left(d_i\right)}{ \partial d_i }
    =
    \frac{\phi^{\prime}_{+}\left(d_i\right)}{\phi_{+}\left(d_i\right)}
    %
    \quad\text{and}\quad
    %
    \frac{\partial \mathrm{H}\left[q_{\vlambda}\right]}{ \partial L_{ij} }
    &=
    0.
  \end{align*}
  Then,
  \begin{align*}
    \norm{ \nabla_{\vlambda} \mathrm{H}\left[q_{\vlambda}\right] - \nabla_{\vlambda^{\prime}} \mathrm{H}\left[q_{\vlambda}\right] }_2^2
    &=
    \sum^{d}_{i=1}
    {\left(
    \frac{\phi^{\prime}_{+}\left(d_i\right)}{\phi_{+}\left(d_i\right)}
    -
    \frac{\phi^{\prime}_{+}\left(d_i^{\prime}\right)}{\phi_{+}\left(d_i^{\prime}\right)}
    \right)}^2.
  \end{align*}

  Thus, \(\vlambda \mapsto \mathrm{H}\left[q_{\vlambda}\right]\) is (\(d L_{+}\))-smooth if and only if \(d_i \mapsto \frac{\phi^{\prime}_{+}\left(d_i\right)}{\phi_{+}\left(d_i\right)}\) is \(L_{+}\)-Lipschitz.
\end{proof}

\begin{theorem}
  Let \(f\left(\vlambda\right)\) be \(L_{f}\)-smooth, \(g\left(\vlambda\right)\) be \(L_{g}\)-smooth.
  Then,
  \begin{enumerate}
    \item \(F_{\text{ELBO}}\left(\vlambda\right)\) is (\(L_f + L_g + \))-smooth
    \item \(\Esub{\rvveta \sim q_{\vlambda}\left(\right)}{ } \)
  \end{enumerate}
\end{theorem}
\begin{proof}
  
\end{proof}

%%% Local Variables:
%%% TeX-master: "master"
%%% End:
