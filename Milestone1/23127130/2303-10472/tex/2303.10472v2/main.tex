%%%%%%%% ICML 2023 EXAMPLE LATEX SUBMISSION FILE %%%%%%%%%%%%%%%%%

\documentclass{article}

% Use the following line for the initial blind version submitted for review:
\usepackage[accepted,nohyperref]{icml2023}

% If accepted, instead use the following line for the camera-ready submission:
% \usepackage[accepted]{icml2023}

% Recommended, but optional, packages for figures and better typesetting:
\usepackage{graphicx}
\usepackage{caption}
\usepackage{subcaption}
\usepackage{booktabs} % for professional tables
\usepackage{nicefrac}
\usepackage{xcolor}
\usepackage{pifont}
\usepackage[inline]{enumitem}


% hack for fixing ``Too many math alphabets used in version normal.''
% error on pdfLaTeX
% cf. https://tex.stackexchange.com/a/243541/91665
\newcommand{\hmmax}{0}
\newcommand{\bmmax}{0}
\usepackage{amssymb}

\usepackage{amsmath,amsthm}
\usepackage{mathtools}
\usepackage{iftex}

% Math and main text font nonsense
\ifxetex
  \usepackage{microtype}
  \usepackage{unicode-math}
  \setmathfont{STIXTwoMath-Regular.otf}
  %\setmainfont[
  %  BoldFont={STIXTwoText-Bold.otf}, 
  %  ItalicFont={STIXTwoText-Italic.otf},
  %  BoldItalicFont={STIXTwoText-BoldItalic.otf}]{STIXTwoText-Regular.otf}
  \setmainfont{Times New Roman}

  %% \setmathfont{TeX Gyre Schola Math}
  %% %%% Scale down!
  %% \setmathfont[range=\mathit/greek,Scale=.8]{TeX Gyre Schola Math}
  %% %%% Circumvent a bug in unicode-math
  %% \setmathfont[range=\int]{TeX Gyre Schola Math}

  %\setmainfont{TeX Gyre Schola}[Scale=0.93] % Text
  % Math: mix of Erewhon and Schola
  %% \setmathfont{Erewhon Math}
  %% \setmathfont{TeX Gyre Schola Math}[Scale=0.93,
  %%    range={up/{latin,Latin,num}, it/{latin,Latin,num},
  %%           bfup/{latin,Latin,num}, bfit/{latin,Latin,num}}]

  \newcommand{\mathmat}[1]{\mathbfit{#1}}
  \newcommand{\mathvec}[1]{\mathbfit{#1}}
  \newcommand{\mathvecgreek}[1]{\mathbfit{#1}}
  \newcommand{\mathmatgreek}[1]{\mathbfit{#1}}
  
  \newcommand{\boldupright}[1]{\symbfup{#1}}

  \newcommand{\mathrv}[1]{\mathsfit{#1}}
  \newcommand{\mathrvvec}[1]{\mathbfsfit{#1}}
  \newcommand{\mathrvgreek}[1]{\mathsfit{#1}}
  \newcommand{\mathrvvecgreek}[1]{\mathbfsfit{#1}}
  \usepackage{fontspec}
\else
  \usepackage[notext,lcgreekalpha]{stix2}
  \usepackage{times}
  \usepackage{textcomp}

  \newcommand{\mathmat}[1]{\mathbfit{#1}}
  \newcommand{\mathvec}[1]{\mathbfit{#1}}
  \newcommand{\mathvecgreek}[1]{\mathbfit{#1}}
  \newcommand{\mathmatgreek}[1]{\mathbfit{#1}}
  
  \newcommand{\boldupright}[1]{\mathbf{#1}}

  \newcommand{\mathrv}[1]{\mathsfit{#1}}
  \newcommand{\mathrvvec}[1]{\mathbfsfit{#1}}
  \newcommand{\mathrvgreek}[1]{\mathsfit{#1}}
  \newcommand{\mathrvvecgreek}[1]{\mathbfsfit{#1}}
\fi

\usepackage[backgroundcolor=blue!20!white,bordercolor=white]{todonotes}
\newcommand\todoerror[2][]{\todo[backgroundcolor=red,textcolor=white,caption={2do},#1]{#2}}

\usepackage{booktabs,threeparttable,arydshln}
\usepackage{multirow}
\usepackage{nicematrix}

\usepackage[algo2e, ruled]{algorithm2e}

\usepackage{pgfplots}
\pgfkeys{/pgfplots/tuftelike/.style={
  semithick,
  tick style={major tick length=3pt,minor tick length=1.5pt,semithick,black},
  xtick align=outside,
  ytick align=outside,
  separate axis lines,
  axis x line*=bottom,
  axis x line shift=3pt,
  axis y line*=left,
  axis y line shift=3pt,
  }
  }
  
\usepackage{proof-at-the-end}

\declaretheoremstyle[
    %spaceabove=0\topsep,
    %spacebelow=0\topsep,
    bodyfont=\normalfont\itshape,
]{theoremsty}

\declaretheorem[name=Theorem,    style=theoremsty]{theorem}
\declaretheorem[name=Theorem,    style=theoremsty]{framedtheorem}
\declaretheorem[name=Proposition,style=theoremsty]{proposition}
\declaretheorem[name=Corollary,  style=theoremsty]{corollary}
\declaretheorem[name=Lemma,      style=theoremsty]{lemma}
\declaretheorem[name=Theorem,    style=theoremsty, numbered=no]{theorem*}
\declaretheorem[name=Proposition,style=theoremsty, numbered=no]{proposition*}
\declaretheorem[name=Corollary,  style=theoremsty, numbered=no]{corollary*}
\declaretheorem[name=Lemma,      style=theoremsty, numbered=no]{lemma*}

\declaretheoremstyle[
    %spaceabove=0\topsep,
    %spacebelow=0\topsep,
    bodyfont=\normalfont,
]{normalsty}
\declaretheorem[name=Remark,    style=normalsty]{remark}
\declaretheorem[name=Definition,style=normalsty]{definition}
\declaretheorem[name=Assumption,style=normalsty]{assumption}
\declaretheorem[name=Remark,    style=normalsty, numbered=no]{remark*}
\declaretheorem[name=Definition,style=normalsty, numbered=no]{definition*}
\declaretheorem[name=Assumption,style=normalsty, numbered=no]{assumption*}

\newenvironment{proofsketch}{%
  \renewcommand{\proofname}{Proof Sketch}\proof%
  \renewcommand{\qedsymbol}{}%
  }{\endproof}

\usepackage[many]{tcolorbox}
%\newtheorem{framedtheorem}{Theorem}[section]
\tcolorboxenvironment{framedtheorem}{
  colback=blue!5!white,
  boxrule=0pt,
  boxsep=1pt,
  left=2pt,right=2pt,top=2pt,bottom=2pt,
  %borderline west={2pt}{0pt}{blue!75!black},
  oversize=2pt,
  sharp corners,
  before skip=\topsep,
  after skip=\topsep,
}

\usepackage{url}
\usepackage[bookmarksopen=true,bookmarks=true,colorlinks=true,backref=page]{hyperref}
% \usepackage{hyperref}
\usepackage[noabbrev, capitalise, nameinlink]{cleveref}

\renewcommand*{\backref}[1]{}
\renewcommand*{\backrefalt}[4]{({\footnotesize%
    \ifcase #1 Not cited.%
          \or page~#2%
          \else pages #2%
    \fi%
    })}
\renewcommand*{\backreftwosep}{\backrefsep}
\renewcommand*{\backreflastsep}{\backrefsep}  

\definecolor{linkcolor}{HTML}{005D5D}
\definecolor{citecolor}{HTML}{00539A}
\hypersetup{colorlinks={true},linkcolor=linkcolor,citecolor=citecolor}

\crefname{assumption}{Assumption}{Assumption}

\newtheorem{subassumption}{Assumption\normalfont}[assumption]
\renewcommand{\thesubassumption}{\theassumption\alph{subassumption}}

%% \newenvironment{assumption*}
%%  {\ifnum\value{subassumption}=0 \stepcounter{assumption}\fi\subassumption}
%%  {\endsubassumption}
%% \newenvironment{assumption+}[1]
%%  {\renewcommand{\thesubassumption}{#1}\subassumption}
%%  {\endsubassumption}

\crefname{framedtheorem}{Theorem}{Theorems}
\crefname{framedproposition}{Proposition}{Propositions}
\crefname{framedlemma}{Lemma}{Lemmas}

\makeatletter
\def\adl@drawiv#1#2#3{%
        \hskip.5\tabcolsep
        \xleaders#3{#2.5\@tempdimb #1{1}#2.5\@tempdimb}%
                #2\z@ plus1fil minus1fil\relax
        \hskip.5\tabcolsep}
\newcommand{\cdashlinelr}[1]{%
  \noalign{\vskip\aboverulesep
           \global\let\@dashdrawstore\adl@draw
           \global\let\adl@draw\adl@drawiv}
  \cdashline{#1}
  \noalign{\global\let\adl@draw\@dashdrawstore
           \vskip\belowrulesep}}
\makeatother

\pgfkeys{/prAtEnd/global custom defaults/.style={
    % proof at the end,
    end,
    restate,
    %debug,
    %% text link={\textit{Proof.} See page~\pageref{proof:prAtEnd\pratendcountercurrent} of the \textit{supplementary material}.},
    %text link={\textit{Proof.} The proof is in the \textit{supplementary material}.
    text proof={Proof},
    %text link section, 
    text link={\textit{Proof.} See the \hyperref[proof:prAtEnd\pratendcountercurrent]{\textit{full proof}} in~\cref{proof:prAtEnd\pratendcountercurrent}.}
  }
}

\def\code#1{\texttt{#1}}
\DeclareMathOperator*{\minimize}{minimize}
\DeclareMathOperator*{\maximize}{maximize}
\DeclareMathOperator*{\argmax}{arg\,max}
\DeclareMathOperator*{\argmin}{arg\,min} 

\newcommand*\xbar[1]{%
  \hbox{%
    \vbox{%
      \hrule height 0.6pt % The actual bar
      \kern0.33ex%         % Distance between bar and symbol
      \hbox{%
        \kern-0.1em%      % Shortening on the left side
        \ensuremath{#1}%
        \kern-0.1em%      % Shortening on the right side
      }%
    }%
  }%
} 

\newcommand{\E}[1]{\mathbb{E}\left[ #1 \right]}
\newcommand{\Esub}[2]{\mathbb{E}_{#1}\left[ #2 \right]}
\newcommand{\V}[1]{\mathbb{V}\left[ #1 \right]}
\newcommand{\Vsub}[2]{\mathbb{V}_{#1}\left[ #2 \right]}
\newcommand{\Cov}[1]{\mathrm{Cov}\left( #1 \right)}
\newcommand{\Covsub}[2]{\mathrm{Cov}_{#1}\left( #2 \right)}
\newcommand{\Corr}[1]{\mathrm{Corr}\left( #1 \right)}

\newcommand{\Df}[2]{D_{f}(#1, #2)}
\newcommand{\DKL}[2]{D_{\mathrm{KL}}(#1, #2)}
\newcommand{\DChi}[2]{D_{\ch^2}(#1, #2)}
\newcommand{\norm}[1]{{\left\lVert #1 \right\rVert}}
\newcommand{\abs}[1]{{\left| #1 \right|}}
\newcommand{\DTV}[2]{{d_{\mathrm{TV}}\left(#1, #2\right)}}

\newcommand{\vX}{\mathrv{X}}
\newcommand{\vY}{\mathrv{Y}}
\newcommand{\vZ}{\mathrv{Z}}

\newcommand{\va}{\mathvec{a}}
\newcommand{\vb}{\mathvec{b}}
\newcommand{\vc}{\mathvec{c}}
\newcommand{\vd}{\mathvec{d}}
\newcommand{\ve}{\mathvec{e}}
\newcommand{\vf}{\mathvec{f}}
\newcommand{\vg}{\mathvec{g}}
\newcommand{\vh}{\mathvec{h}}
\newcommand{\vi}{\mathvec{i}}
\newcommand{\vj}{\mathvec{j}}
\newcommand{\vk}{\mathvec{k}}
\newcommand{\vl}{\mathvec{l}}
\newcommand{\vm}{\mathvec{m}}
\newcommand{\vn}{\mathvec{n}}
\newcommand{\vo}{\mathvec{o}}
\newcommand{\vp}{\mathvec{p}}
\newcommand{\vq}{\mathvec{q}}
\newcommand{\vr}{\mathvec{r}}
\newcommand{\vs}{\mathvec{s}}
\newcommand{\vt}{\mathvec{t}}
\newcommand{\vu}{\mathvec{u}}
\newcommand{\vv}{\mathvec{v}}
\newcommand{\vw}{\mathvec{w}}
\newcommand{\vx}{\mathvec{x}}
\newcommand{\vy}{\mathvec{y}}
\newcommand{\vz}{\mathvec{z}}
\newcommand{\valpha}{\mathvecgreek{\alpha}}
\newcommand{\veta}{\mathvecgreek{\eta}}
\newcommand{\vmu}{\mathvecgreek{\mu}}
\newcommand{\vtheta}{\mathvecgreek{\theta}}
\newcommand{\vlambda}{\mathvecgreek{\lambda}}
\newcommand{\vsigma}{\mathvecgreek{\sigma}}
\newcommand{\vgamma}{\mathvecgreek{\gamma}}
\newcommand{\vxi}{\mathvecgreek{\xi}}
\newcommand{\vzeta}{\mathvecgreek{\zeta}}

\newcommand{\rva}{\mathrv{a}}
\newcommand{\rvb}{\mathrv{b}}
\newcommand{\rvc}{\mathrv{c}}
\newcommand{\rvd}{\mathrv{d}}
\newcommand{\rve}{\mathrv{e}}
\newcommand{\rvf}{\mathrv{f}}
\newcommand{\rvg}{\mathrv{g}}
\newcommand{\rvh}{\mathrv{h}}
\newcommand{\rvi}{\mathrv{i}}
\newcommand{\rvj}{\mathrv{j}}
\newcommand{\rvk}{\mathrv{k}}
\newcommand{\rvl}{\mathrv{l}}
\newcommand{\rvm}{\mathrv{m}}
\newcommand{\rvn}{\mathrv{n}}
\newcommand{\rvo}{\mathrv{o}}
\newcommand{\rvp}{\mathrv{p}}
\newcommand{\rvq}{\mathrv{q}}
\newcommand{\rvr}{\mathrv{r}}
\newcommand{\rvs}{\mathrv{s}}
\newcommand{\rvt}{\mathrv{t}}
\newcommand{\rvu}{\mathrv{u}}
\newcommand{\rvv}{\mathrv{v}}
\newcommand{\rvw}{\mathrv{w}}
\newcommand{\rvx}{\mathrv{x}}
\newcommand{\rvy}{\mathrv{y}}
\newcommand{\rvz}{\mathrv{z}}
\newcommand{\rvA}{\mathrv{A}}
\newcommand{\rvB}{\mathrv{B}}
\newcommand{\rvC}{\mathrv{C}}
\newcommand{\rvD}{\mathrv{D}}
\newcommand{\rvE}{\mathrv{E}}
\newcommand{\rvF}{\mathrv{F}}
\newcommand{\rvG}{\mathrv{G}}
\newcommand{\rvH}{\mathrv{H}}
\newcommand{\rvI}{\mathrv{I}}
\newcommand{\rvJ}{\mathrv{J}}
\newcommand{\rvK}{\mathrv{K}}
\newcommand{\rvL}{\mathrv{L}}
\newcommand{\rvM}{\mathrv{M}}
\newcommand{\rvN}{\mathrv{N}}
\newcommand{\rvO}{\mathrv{O}}
\newcommand{\rvP}{\mathrv{P}}
\newcommand{\rvQ}{\mathrv{Q}}
\newcommand{\rvR}{\mathrv{R}}
\newcommand{\rvS}{\mathrv{S}}
\newcommand{\rvT}{\mathrv{T}}
\newcommand{\rvU}{\mathrv{U}}
\newcommand{\rvV}{\mathrv{V}}
\newcommand{\rvW}{\mathrv{W}}
\newcommand{\rvX}{\mathrv{X}}
\newcommand{\rvY}{\mathrv{Y}}
\newcommand{\rvZ}{\mathrv{Z}}
\newcommand{\rveta}{\mathrvgreek{\eta}}
\newcommand{\rvlambda}{\mathrvgreek{\lambda}}
\newcommand{\rvepsilon}{\mathrvgreek{\epsilon}}

\newcommand{\rvva}{\mathrvvec{a}}
\newcommand{\rvvb}{\mathrvvec{b}}
\newcommand{\rvvc}{\mathrvvec{c}}
\newcommand{\rvvd}{\mathrvvec{d}}
\newcommand{\rvve}{\mathrvvec{e}}
\newcommand{\rvvf}{\mathrvvec{f}}
\newcommand{\rvvg}{\mathrvvec{g}}
\newcommand{\rvvh}{\mathrvvec{h}}
\newcommand{\rvvi}{\mathrvvec{i}}
\newcommand{\rvvj}{\mathrvvec{j}}
\newcommand{\rvvk}{\mathrvvec{k}}
\newcommand{\rvvl}{\mathrvvec{l}}
\newcommand{\rvvm}{\mathrvvec{m}}
\newcommand{\rvvn}{\mathrvvec{n}}
\newcommand{\rvvo}{\mathrvvec{o}}
\newcommand{\rvvp}{\mathrvvec{p}}
\newcommand{\rvvq}{\mathrvvec{q}}
\newcommand{\rvvr}{\mathrvvec{r}}
\newcommand{\rvvs}{\mathrvvec{s}}
\newcommand{\rvvt}{\mathrvvec{t}}
\newcommand{\rvvu}{\mathrvvec{u}}
\newcommand{\rvvv}{\mathrvvec{v}}
\newcommand{\rvvw}{\mathrvvec{w}}
\newcommand{\rvvx}{\mathrvvec{x}}
\newcommand{\rvvy}{\mathrvvec{y}}
\newcommand{\rvvz}{\mathrvvec{z}}
\newcommand{\rvvlambda}{\mathrvvecgreek{\lambda}}
\newcommand{\rvveta}{\mathrvvecgreek{\eta}}
\newcommand{\rvvepsilon}{\mathrvvecgreek{\epsilon}}
\newcommand{\rvvzeta}{\mathrvvecgreek{\zeta}}

\newcommand{\mA}{\mathmat{A}}
\newcommand{\mB}{\mathmat{B}}
\newcommand{\mC}{\mathmat{C}}
\newcommand{\mD}{\mathmat{D}}
\newcommand{\mE}{\mathmat{E}}
\newcommand{\mF}{\mathmat{F}}
\newcommand{\mG}{\mathmat{G}}
\newcommand{\mH}{\mathmat{H}}
\newcommand{\mI}{\mathmat{I}}
\newcommand{\mJ}{\mathmat{J}}
\newcommand{\mK}{\mathmat{K}}
\newcommand{\mL}{\mathmat{L}}
\newcommand{\mM}{\mathmat{M}}
\newcommand{\mN}{\mathmat{N}}
\newcommand{\mO}{\mathmat{O}}
\newcommand{\mP}{\mathmat{P}}
\newcommand{\mQ}{\mathmat{Q}}
\newcommand{\mR}{\mathmat{R}}
\newcommand{\mS}{\mathmat{S}}
\newcommand{\mT}{\mathmat{T}}
\newcommand{\mU}{\mathmat{U}}
\newcommand{\mV}{\mathmat{V}}
\newcommand{\mW}{\mathmat{W}}
\newcommand{\mX}{\mathmat{X}}
\newcommand{\mY}{\mathmat{Y}}
\newcommand{\mZ}{\mathmat{Z}}
\newcommand{\mSigma}{\mathmatgreek{\Sigma}}
\newcommand{\mPhi}{\mathmatgreek{\Phi}}

\newcommand{\inner}[2]{\left\langle #1, #2 \right\rangle}
\newcommand{\ind}[1]{\mathds{1}_{#1}}



\usepackage{etoc}
%\etocmulticolstyle{\noindent\bfseries\footnotesize
%\leaders\hrule height1pt\hfill
%\MakeUppercase{Contents}}
\etocframedstyle[2]{\textbf{\textsc{Table of Contents}}}
\etocsettocdepth{3}

% Attempt to make hyperref and algorithmic work together better:
\newcommand{\theHalgorithm}{\arabic{algorithm}}

%\renewcommand\cftchapafterpnum{\vskip10pt}
%\renewcommand\cftsecafterpnum{\vskip15pt}

\usepackage{tikz}
\usetikzlibrary{patterns}
\usetikzlibrary{spy}
\usetikzlibrary{arrows.meta}
\usetikzlibrary{pgfplots.groupplots}
\usepgfplotslibrary{fillbetween}
\usepgfplotslibrary{external} 

%% \tikzset{external/system call={xelatex -shell-escape -halt-on-error -interaction=batchmode -jobname "\image" "\texsource"}}
%% \tikzexternalize
%% \tikzexternaldisable

\pgfplotsset{
    name nodes near coords/.style={
        every node near coord/.append style={
            name=#1-\coordindex,
            alias=#1-last,
        },
    },
    name nodes near coords/.default=coordnode
}

\definecolor{color1}{HTML}{EE5396}
\definecolor{color2}{HTML}{A56EFF}
\definecolor{color3}{HTML}{4589FF}
\definecolor{color4}{HTML}{1192E8}

% \newcommand{\debugmode}{}

\newcommand{\lemmaproofoption}{%
  %\ifthenelse{\debugmode}{debug}{all end}%
  \ifx\debugmode\undefined
    all end%
  \else
    debug%
  \fi
}%

\newcommand{\keylemmaproofoption}{%
  %\ifthenelse{\debugmode}{debug}{all end}%
  \ifx\debugmode\undefined
    end%
  \else
    debug%
  \fi
}%

\newcommand{\theoremproofoption}{%
  %\ifthenelse{\debugmode}{debug}{all end}%
  \ifx\debugmode\undefined
    %
  \else
    debug%
  \fi
}%

% Todonotes is useful during development; simply uncomment the next line
%    and comment out the line below the next line to turn off comments
%\usepackage[disable,textsize=tiny]{todonotes}


% The \icmltitle you define below is probably too long as a header.
% Therefore, a short form for the running title is supplied here:
\icmltitlerunning{Gradient Variance Bounds for Black-Box Variational Inference}

\begin{document}

\twocolumn[
\icmltitle{Practical and Matching Gradient Variance Bounds for \\ Black-Box Variational Bayesian Inference}

% It is OKAY to include author information, even for blind
% submissions: the style file will automatically remove it for you
% unless you've provided the [accepted] option to the icml2023
% package.

% List of affiliations: The first argument should be a (short)
% identifier you will use later to specify author affiliations
% Academic affiliations should list Department, University, City, Region, Country
% Industry affiliations should list Company, City, Region, Country

% You can specify symbols, otherwise they are numbered in order.
% Ideally, you should not use this facility. Affiliations will be numbered
% in order of appearance and this is the preferred way.

\begin{icmlauthorlist}
\icmlauthor{Kyurae Kim}{upenncis}
\icmlauthor{Kaiwen Wu}{upenncis}
\icmlauthor{Jisu Oh}{ncsustat}
\icmlauthor{Jacob R. Gardner}{upenncis}
\end{icmlauthorlist}

\icmlaffiliation{upenncis}{Department of Computer and Information Sciences, University of Pennsylvania, Philadelphia, Pennsylvania, United States}
\icmlaffiliation{ncsustat}{Department of Statistics, North Carolina State University, Raleigh, North Carolina, United States}

\icmlcorrespondingauthor{Kyurae Kim}{kyrkim@seas.upenn.edu}
\icmlcorrespondingauthor{Kaiwen Wu}{kaiwenwu@seas.upenn.edu}
\icmlcorrespondingauthor{Jacob R. Gardner}{jrgardner@seas.upenn.edu}

% You may provide any keywords that you
% find helpful for describing your paper; these are used to populate
% the "keywords" metadata in the PDF but will not be shown in the document
\icmlkeywords{Bayesian inference, variational inference, stochastic gradient descent, SGD, VI, gradient variance, reparameterization trick}

\vskip 0.3in
]

% this must go after the closing bracket ] following \twocolumn[ ...

% This command actually creates the footnote in the first column
% listing the affiliations and the copyright notice.
% The command takes one argument, which is text to display at the start of the footnote.
% The \icmlEqualContribution command is standard text for equal contribution.
% Remove it (just {}) if you do not need this facility.

%\printAffiliationsAndNotice{}  % leave blank if no need to mention equal contribution
\printAffiliationsAndNotice{} % otherwise use the standard text.

\begin{abstract}
  Understanding the gradient variance of black-box variational inference (BBVI) is a crucial step for establishing its convergence and developing algorithmic improvements.
  However, existing studies have yet to show that the gradient variance of BBVI satisfies the conditions used to study the convergence of stochastic gradient descent (SGD), the workhorse of BBVI.
  In this work, we show that BBVI satisfies a \textit{matching} bound corresponding to the \(ABC\) condition used in the SGD literature when applied to smooth and quadratically-growing log-likelihoods.
  Our results generalize to nonlinear covariance parameterizations widely used in the practice of BBVI.
  Furthermore, we show that the variance of the mean-field parameterization has provably superior dimensional dependence.
\end{abstract}


\section{Introduction}
Variational inference (VI; \citealt{jordan_introduction_1999,blei_variational_2017,zhang_advances_2019}) algorithms are fast and scalable Bayesian inference methods widely applied in fields of statistics and machine learning.
In particular, black-box VI (BBVI; \citealt{ranganath_black_2014,titsias_doubly_2014}) leverages 
stochastic gradient descent (SGD; \citealt{robbins_stochastic_1951,bottou_online_1999}) for inference of non-conjugate probabilistic models.
With the development of bijectors~\citep{kucukelbir_automatic_2017,dillon_tensorflow_2017,fjelde_bijectors_2020}, most of the methodological advances in BBVI have now been abstracted out through various probabilistic programming frameworks~\citep{carpenter_stan_2017,ge_turing_2018,dillon_tensorflow_2017,bingham_pyro_2019,salvatier_probabilistic_2016}.

Despite the advances of BBVI, little is known about its theoretical properties.
Even when restricted to the location-scale family (\cref{def:family}), it is unknown whether BBVI is guaranteed to converge without having to modify the algorithms used in practice, for example, by enforcing bounded domains, bounded support, bounded gradients, and such.
This theoretical insight is necessary since BBVI methods are known to be less robust~\citep{yao_yes_2018,dhaka_robust_2020,welandawe_robust_2022,dhaka_challenges_2021,domke_provable_2020} compared to other inference methods such as Markov chain Monte Carlo.
Although progress has been made to formalize the theory of BBVI with some generality, the gap between our understanding of BBVI and the convergence guarantees of SGD remains open.
For example,~\citet{domke_provable_2019,domke_provable_2020} provided smoothness and gradient variance guarantees. 
Still, these results do not yet yield a full convergence guarantee and do not extend to \textit{nonlinear} covariance parameterizations used in practice.

%Currently, most convergence proofs of SGD require that the gradient variance (or, more precisely, the expected squared norm) is bounded by certain quantities.

In this work, we investigate whether recent progress in relaxing the gradient variance assumptions used in SGD~\citep{tseng_incremental_1998,vaswani_fast_2019,schmidt_fast_2013,bottou_optimization_2018,gower_sgd_2019,gower_stochastic_2021,nguyen_sgd_2018} apply to BBVI. These extensions have led to new insights that the structure of the gradient bounds can have non-trivial interactions with gradient-adaptive SGD algorithms~\citep{zhang_adam_2022}.
For example, when the ``interpolation assumption'' (the gradient noise converges to 0;~\citealt{schmidt_fast_2013,ma_power_2018,vaswani_fast_2019}) does not hold, ADAM~\citep{kingma_adam_2015} provably diverges with certain stepsize combinations~\citep{zhang_adam_2022}.
Until BBVI can be shown to conform to the assumptions used by these recent works, it is unclear how these results relate to BBVI.

While the variance of BBVI gradient estimators has been studied before~\citep{xu_variance_2019,domke_provable_2019,mohamed_monte_2020,fujisawa_multilevel_2021}, the connection with the conditions used in SGD has yet to be established.
As such, we answer the following question:
%
\vspace{-2ex}%
\begin{quote}
  \textit{Does the gradient variance of BBVI conform to the conditions assumed in convergence guarantees of SGD without modifying the implementations used in practice?}
\end{quote}
\vspace{-2ex}%
%
The answer is yes!
Assuming the target log joint distribution is smooth and quadratically growing, we show that the gradient variance of BBVI satisfies the \textit{ABC} condition (\cref{assumption:abc}) used by~\citet{polyak_pseudogradient_1973,khaled_better_2023,gower_stochastic_2021}.
Our analysis extends the previous result of \citet{domke_provable_2019} to covariance parameterizations involving nonlinear functions for conditioning the diagonal (see \cref{section:covariance_parameterization}), as commonly done in practice.
Furthermore, we prove that the gradient variance of the mean-field parameterization \citep{peterson_mean_1987,peterson_explorations_1989,hinton_keeping_1993} results in better dimensional dependence compared to full-rank ones.
%Overall, our results should act as stepping stones towards full convergence guarantees of BBVI.

Our contributions are summarized as follows:
\begin{itemize}
  \vspace{-2ex}
  \setlength\itemsep{-1.5ex}
  \item[\ding{182}] We provide upper bounds on the gradient variance of BBVI that matches the \textit{ABC condition} (\cref{assumption:abc}) used for analyzing SGD.
    \begin{itemize}[leftmargin=1.5em,]
      \item[\ding{228}] \cref{thm:gradient_upper_bound,thm:gradient_upper_bound_kl} do not require any modification of the algorithms used in practice.
      \item[\ding{228}] \cref{thm:gradient_upper_bound_bounded_entropy} achieves better constants under the stronger \textit{bounded entropy} assumption.
    \end{itemize}
    % 
    \item[\ding{183}] Our analysis applies to BBVI parameterizations (\cref{section:covariance_parameterization}) widely used in practice (\cref{table:parameterization_survey}).
    \begin{itemize}[leftmargin=1.5em,]
        \item[\ding{228}] \cref{thm:general_variational_gradient_norm_identity} enables the bounds to cover nonlinear covariance parameterizations.
        \item[\ding{228}] \cref{thm:meanfield_u_identity,remark:meanfield_superiority} shows that the gradient variance of the mean-field parameterization has superior dimensional scaling.
    \end{itemize}
    %
    \item[\ding{184}] We provide a matching lower bound (\cref{thm:gradient_lower_bound}) on the gradient variance, showing that, under the stated assumptions, the ABC condition is the weakest assumption applicable to BBVI.
\end{itemize}
  \vspace{-2ex}

%%% Local Variables:
%%% TeX-master: "main"
%%% End:


\section{Preliminaries}
{%small
\paragraph{Notation}
Random variables are denoted in serif, while their realization is in regular font.
(\textit{i.e}, \(x\) is a realization of \(\rvx\), \(\vx\) is a realization of the vector-valued \(\rvvx\).)
\(\norm{\vx}_2 = \){\footnotesize\(\sqrt{\inner{\vx}{\vx}} = \sqrt{\vx^{\top}\vx}\)} denotes the Euclidean norm, while \(\norm{\mA}_{\mathrm{F}} =\) {\footnotesize\(\sqrt{\mathrm{tr}\left(\mA^{\top} \mA\right)}\)} is the Frobenius norm, where {\footnotesize\(\mathrm{tr}\left(\mA\right) = \sum^{d}_{i=1} A_{ii} \)} is the matrix trace.
}%

\vspace{-1ex}
\subsection{Variational Inference}
\vspace{-.5ex}
Variational inference~\citep{peterson_mean_1987,hinton_keeping_1993} is a family of inference algorithms devised to solve the problem
{%
  \setlength{\belowdisplayskip}{1.ex} \setlength{\belowdisplayshortskip}{1.ex}%
  \setlength{\abovedisplayskip}{1.ex} \setlength{\abovedisplayshortskip}{1.ex}%
\begin{align}
  \minimize_{\vlambda \in \mathbb{R}^p} \; \DKL{q_{\psi,\vlambda}}{\pi}, \label{eq:kl}
\end{align}
}%
where \(q_{\psi,\vlambda}\) is called the ``variational approximation'', while \(\pi\) is a distribution of interest, and \(D_{\text{KL}}\) is the (exclusive) Kullback-Leibler (KL) divergence.

For Bayesian inference, \(\pi\) is the posterior distribution
{%
  \setlength{\belowdisplayskip}{1.ex} \setlength{\belowdisplayshortskip}{1.ex}%
  \setlength{\abovedisplayskip}{1.ex} \setlength{\abovedisplayshortskip}{1.ex}%
\begin{align*}
  \pi\left(\vz\right)
  \propto 
  \ell\left(\vx \mid \vz \right) p\left(\vz\right)
  =
  \ell\left(\vx, \vz \right),
\end{align*}
}%
where \(\ell\left(\vx \mid \vz \right)\) is the likelihood, and \(p\left(\vz\right)\) is the prior.
In practice, one only has access to the likelihood and the prior.
Thus,~\cref{eq:kl} cannot be directly solved.
Instead, we can minimize the negative \textit{evidence lower bound} (ELBO; \citealt{jordan_introduction_1999}) function \(F\left(\vlambda\right)\).

\vspace{-1.5ex}
\paragraph{Evidence Lower Bound}
More formally, we solve
{%
\setlength{\belowdisplayskip}{1.ex} \setlength{\belowdisplayshortskip}{1.ex}
\setlength{\abovedisplayskip}{1.ex} \setlength{\abovedisplayshortskip}{1.ex}
\[
  \minimize_{\vlambda \in \mathbb{R}^p} \; F\left(\vlambda\right),
\]
}%
where \(F\) is defined as
{%
\setlength{\belowdisplayskip}{1.ex} \setlength{\belowdisplayshortskip}{1.ex}
\setlength{\abovedisplayskip}{1.ex} \setlength{\abovedisplayshortskip}{1.ex}
\begin{align}
  F\left(\vlambda\right) 
  &\triangleq
  -\Esub{\rvvz \sim q_{\psi,\vlambda}}{ \log \ell\left(\vx, \rvvz\right) } - \mathrm{H}\left(q_{\psi,\vlambda}\right),
  \label{eq:elbo_H_form}
  \\
  &=
  -\Esub{\rvvz \sim q_{\psi,\vlambda}}{ \log \ell\left(\vx|\rvvz\right) } + \DKL{q_{\psi,\vlambda}}{p},
  \label{eq:elbo_kl_form}
\end{align}
}%
\begin{center}
  \vspace{-1.5ex}
  {\begingroup
    \setlength\tabcolsep{1.5pt} 
  \begin{tabular}{ll}
    \(\rvvz\) & is the latent (random) variable, \\
    \(q_{\psi,\vlambda}\) & is the variational distribution, \\
    \(\psi\) & is a bijector (support transformation), and  \\
    \(\mathrm{H}\)    & is the differential entropy.
    % \(\mJ_{\psi^{-1}}\left(\rvveta\right)\) & is the Jacobian of the inverse of \(\psi\), and \\
    % \(h\) & is a deterministic regularizer term.
  \end{tabular}
  \endgroup}
  \vspace{-1.5ex}
\end{center}

The bijector \(\psi\)~\citep{dillon_tensorflow_2017,fjelde_bijectors_2020,leger_parametrization_2023} is a differentiable bijective map that is used to de-constrain the support of constrained random variables.
For example, when \(z\) is expected to follow a gamma distribution, using \(\eta = \psi\left(z\right)\) with \(\psi\left(z\right) = \log z\) lets us work with \(\eta\), which can be any real number, unlike \(z\).
The use of \(\psi^{-1}\) corresponds to the automatic differentiation VI formulation (ADVI;~\citealt{kucukelbir_automatic_2017}), which is now widespread.

\vspace{-2.5ex}
\subsection{Variational Family}
\vspace{-.5ex}
In this work, we specifically consider the location-scale variational family with a standardized base distribution.

\begin{definition}[\textbf{Reparameterization Function}]\label{def:reparam}
  An affine mapping \(\vt_{\vlambda} : \mathbb{R}^d \rightarrow \mathbb{R}^d\) defined as
{
\setlength{\belowdisplayskip}{1.ex} \setlength{\belowdisplayshortskip}{1.ex}
\setlength{\abovedisplayskip}{1.ex} \setlength{\abovedisplayshortskip}{1.ex}
  \begin{align*}
    &\vt_{\vlambda}\left(\vu\right) \triangleq \mC \vu + \vm
  \end{align*}
}%
  with \(\vlambda\) containing the parameters for forming the location \(\vm \in \mathbb{R}^d\) and scale \(\mC = \mC\left(\vlambda\right) \in \mathbb{R}^{d \times d}\) is called the (location-scale) \textit{reparameterization function}.
\end{definition}

\begin{definition}[\textbf{Location-Scale Family}]\label{def:family}
  Let \(\varphi\) be some \(d\)-dimensional distribution.
  Then, \(q_{\vlambda}\) such that
{%
\setlength{\belowdisplayskip}{1.5ex} \setlength{\belowdisplayshortskip}{1.5ex}%
\setlength{\abovedisplayskip}{1.5ex} \setlength{\abovedisplayshortskip}{1.5ex}%
  \begin{alignat*}{2}
    \rvvzeta \sim q_{\vlambda}  \quad\Leftrightarrow\quad &\rvvzeta \stackrel{d}{=} \vt_{\vlambda}\left(\rvvu\right); \quad \rvvu \sim  \varphi
  \end{alignat*}
  }%
  is said to be a member of the location-scale family indexed by the base distribution \(\varphi\) and parameter \(\vlambda\).
\end{definition}

This family includes commonly used variational families, such as the mean-field Gaussian, full-rank Gaussian, Student-T, and other elliptical distributions.
\begin{remark}[\textbf{Entropy of Location-Scale Distributions}]\label{thm:location_scale_entropy}
  The differential entropy of a location-scale family distribution (\cref{def:family}) is 
{
\setlength{\belowdisplayskip}{1.ex} \setlength{\belowdisplayshortskip}{1.ex}
\setlength{\abovedisplayskip}{1.ex} \setlength{\abovedisplayshortskip}{1.ex}
  \[
    \mathrm{H}\left(q_{\vlambda}\right) = \mathrm{H}\left(\varphi\right) + \log \abs{ \mC }.
  \]
}
\end{remark}

\begin{definition}[\textbf{ADVI Family}; \citealt{kucukelbir_automatic_2017}]\label{def:advi}
  Let \(q_{\vlambda}\) be some \(d\)-dimensional distribution.
  Then, \(q_{\psi,\vlambda}\) such that
{
\setlength{\belowdisplayskip}{1.ex} \setlength{\belowdisplayshortskip}{1.ex}
\setlength{\abovedisplayskip}{1.ex} \setlength{\abovedisplayshortskip}{1.ex}
  \begin{alignat*}{2}
    \rvvz \sim q_{\psi,\vlambda}  \quad\Leftrightarrow\quad &\rvvz \stackrel{d}{=} \psi^{-1}\left(\rvvzeta\right); \quad \rvvzeta \sim q_{\vlambda}
  \end{alignat*}
}%
  is said to be a member of the ADVI family with the base distribution \(q_{\vlambda}\) parameterized with \(\vlambda\).
\end{definition}

We impose assumptions on the base distribution \(\varphi\).
\begin{assumption}[\textbf{Base Distribution}]\label{assumption:symmetric_standard}
  \(\varphi\) is a \(d\)-dimensional distribution such that \(\rvvu \sim \varphi\) and \(\rvvu = \left(\rvu_1, \ldots, \rvu_d \right)\) with indepedently and identically distributed components.
  Furthermore, \(\varphi\) is
  \begin{enumerate*}[label=\textbf{(\roman*)}]
      \item symmetric and standardized such that \(\mathbb{E}\rvu_i = 0\), \(\mathbb{E}\rvu_i^2 = 1\), \(\mathbb{E}\rvu_i^3 = 0\), and 
      \item has finite kurtosis \(\mathbb{E}\rvu_i^4 = \kappa < \infty\).
  \end{enumerate*}
\end{assumption}
These assumptions are already satisfied in practice by, for example, generating \(\rvu_i\) from a univariate normal or Student-T with \(\nu > 4\) degrees of freedom.

\subsection{Reparameterization Trick}
\vspace{-.5ex}
When restricted to location scale families~(\cref{def:family,def:advi}), we can invoke Change-of-Variable, or more commonly known as the ``reparameterization trick,'' such that
{%
\begin{align*}
  \mathbb{E}_{\rvvz \sim q_{\psi,\vlambda}} \log\ell\left(\vx, \rvvz\right)
  &=
  \mathbb{E}_{\rvvzeta \sim q_{\vlambda}} \log\ell\left(\vx, \psi^{-1}\left( \rvvzeta\right)\right)
  \\
  &=
  \mathbb{E}_{\rvvu \sim \varphi} \log\ell\left(\vx, \psi^{-1}\left(\vt_{\vlambda}\left(\rvvu\right)\right)\right)
\end{align*}
}%
through the Law of the Unconcious Statistician.
Differentiating this results in the \textit{reparameterization} or \textit{path} gradient, which often achieves lower variance than alternatives~\citep{xu_variance_2019,mohamed_monte_2020a}.

\vspace{-.5ex}
\paragraph{Objective Function}
For generality, we represent our objective as a regularized infinite sum problem:
\begin{definition}[\textbf{Regularized Infinite Sum}]\label{def:generic_elbo}
\begin{equation*}
   F\left(\vlambda\right)
   =
   \mathbb{E}_{\rvvu \sim \varphi} f\left(\vt_{\vlambda}\left(\rvvu\right)\right)
   +
   h\left(\vlambda\right),
   \label{eq:F}
\end{equation*}
where \((\vlambda, \rvvu) \mapsto f \circ \vt_{\vlambda} : \mathbb{R}^p \times \mathbb{R}^d \rightarrow \mathbb{R}\) is some bivariate stochastic function of \(\vlambda\) and the ``noise source'' \(\rvvu\), while \(h\) is a deterministic regularization term.
\end{definition}
By appropriately defining \(f\) and \(h\), we retrieve the two most common formulations of the ELBO in \cref{eq:elbo_H_form} and \cref{eq:elbo_kl_form} respectively:
\begin{definition}[\textbf{ELBO Entropy-Regularized Form}]\label{def:entropy_form}
\begin{align*}
  f_{\mathrm{H}}\left(\vzeta\right) 
  &= -\log \underbrace{\ell\left( \vx, \psi^{-1}\left( \vzeta \right) \right)}_{\text{Joint Likelihood}} - \log \abs{ \mJ_{\psi^{-1}}\left(\vzeta\right) }  &
  \\
  h_{\mathrm{H}}\left(\vlambda\right) &= - \mathrm{H}\left(q_{\vlambda}\right).
\end{align*}
\end{definition}
%
\begin{definition}[\textbf{ELBO KL-Regularized Form}]\label{def:kl_form}
\begin{align*}
  f_{\text{KL}}\left(\vzeta\right) 
  &= - \log \underbrace{\ell\left(\vx \mid \psi^{-1}\left( \vzeta \right)  \right)}_{\text{Likelihood}} - \log \abs{ \mJ_{\psi^{-1}}\left(\vzeta\right) } &
  \\
  h_{\text{KL}}\left(\vlambda\right) &= \DKL{q_{\vlambda}}{p}.
\end{align*}
\end{definition}%
Here, \(\mJ_{\psi^{-1}}\) is the Jacobian of the bijector.
%
Since \(\DKL{q_{\vlambda}}{p}\) is seldomly available in tractable form, the entropy-regularized form is the most widely used, while the KL regularized is common for Gaussian processes and variational autoencoders.
%However, even when the KL-regularized form is intractable, it still exists in theory, and we will use jump between the two forms while proving some of our upper bounds.

\vspace{-1.0ex}
\paragraph{Gradient Estimator}
We denote the \(M\)-sample estimator of the gradient of \(F\) as
\begin{align}
  \rvvg_{M}\left(\vlambda\right) &\triangleq \frac{1}{M} \, \sum^{M}_{m=1} \rvvg_m\left(\vlambda\right), \;\text{where}\; \label{eq:def_gradient_M_est} \\
  \rvvg_m\left(\vlambda\right)   &\triangleq \nabla_{\vlambda} f\left(\vt_{\vlambda}\left(\rvvu_m\right)\right) + \nabla h(\vlambda); \quad \rvvu_m \sim \varphi. \label{eq:def_gradient_m_est}
\end{align}
We will occasionally drop \(\vlambda\) for clarity.

\vspace{-.5ex}
\subsection{Gradient Variance Assumptions in \linebreak Stochastic Gradient Descent}
\vspace{-.5ex}
\paragraph{Gradient Variance Assumptions in SGD}
For a while, most convergence proofs in SGD have relied on the ``bounded variance'' assumption.
That is, for a gradient estimator \(\rvvg\), \(\mathbb{E} \norm{\rvvg}^2_2 \leq G\) for some finite constant \(G\).
This assumption is problematic because
\begin{enumerate*}
  \item[\ding{182}] these types of global constants result in loose bounds, 
  \item[\ding{183}] and it directly contradicts the strong-convexity assumption~\citep{nguyen_sgd_2018}.
\end{enumerate*}
Thus, retrieving previously known SGD convergence rates under weaker assumptions has been an important research direction~\citep{tseng_incremental_1998,vaswani_fast_2019,schmidt_fast_2013,bottou_optimization_2018,gower_sgd_2019,gower_stochastic_2021,nguyen_sgd_2018}.

\vspace{-.5ex}
\paragraph{ABC Condition}\label{section:abc}
In this work, we focus on the recently rediscovered \textit{expected smoothness}, or \textit{ABC}, condition~\citep{polyak_pseudogradient_1973,gower_stochastic_2021}.
\begin{assumption}[\textbf{Expected Smoothness; \(ABC\)}]\label{assumption:abc}
  \(\rvvg\) is said to satisfy the expected smoothness condition if 
  \begin{align*}
    \mathbb{E}\norm{\rvvg_M\left(\vlambda\right)}_2^2
    \leq
    2 \, A \left( F\left(\vlambda\right) - F^* \right)
    + B \, \norm{ \nabla F\left(\vlambda\right) }_2^2 + C.
  \end{align*}
  for some finite \(A, B, C \geq 0\), where \(F^* = \inf_{\vlambda \in \mathrm{R}^p} F\left(\vlambda\right) \).
\end{assumption}
As shown by \citet{khaled_better_2022}, this condition is not only strictly weaker than many of the previously used assumptions but also \textit{generalizes} them by retrieving known convergence rates when tweaking the constants.
With the \(ABC\) condition, for nonconvex \(L\)-smooth functions, under a fixed stepsize of \(\gamma = \nicefrac{1}{L B}\), SGD converges to a \(\mathcal{O}\left(L C \gamma \right)\) neighborhood in a \(\mathcal{O}\left(\nicefrac{{\left(1 + L \gamma^2 A\right)}^T}{\left(\gamma T\right)}\right)\) rate.
The \textit{ABC} condition has also been used to prove convergence of SGD for quasar convex functions~\citet{gower_sgd_2021}, stochastic heavy-ball/momentum methods~\citet{liu_almost_2022}, and stochastic proximal methods~\citep{li_unified_2022}.
Given the influx of results based on the \textit{ABC} condition, connecting with it would significantly broaden our theoretical understanding of BBVI.

\vspace{-1ex}
\subsection{Covariance Parameterizations}\label{section:covariance_parameterization}
\vspace{-.5ex}

When using the location-scale family (\cref{def:family}), the scale matrix \(\mC\) can be parameterized in different ways.
Any parameterization that results in a positive definite covariance \(\mC\mC^{\top} \in \mathbb{S}_{++}^{d}\) is valid.
We consider multiple parameterizations as the choice can result in different theoretical properties.
A brief survey on the use of different parameterizations is shown in \cref{section:survey}.

\vspace{-2ex}
\paragraph{Linear Parameterization}
The previous results by~\citet{domke_provable_2019} considered the matrix square root parameterization, which is linear with respect to the variational parameters.
{
\setlength{\belowdisplayskip}{1.ex} \setlength{\belowdisplayshortskip}{1.ex}
\setlength{\abovedisplayskip}{1.ex} \setlength{\abovedisplayshortskip}{1.ex}
\begin{definition}[\textbf{Matrix Square Root}]\label{def:squareroot}
  {
    \[
      \mC\left(\vlambda\right) = \mC,
    \]
    where \(\mC \in \mathbb{R}^{d \times d}\) is a matrix, \(\vlambda_{\mC} = \mathrm{vec}\left(\mC\right) \in \mathbb{R}^{d^2}\) such that \(\vlambda = \left(\vm, \vlambda_{\mC}\right)\).
  }
\end{definition}
}%
Note that \(\mC\) is not constrained to be symmetric so this is not a matrix square root in a narrow sense.
Also, this parameterization does not guarantee \(\mC\mC^{\top}\) to be positive definite (only positive semidefinite), which occasionally results in the entropy term \(h_{\mathrm{H}}\) blowing up~\citep{domke_provable_2020}.
\citeauthor{domke_provable_2020} proposed to fix this by using proximal operators.

\vspace{-2.0ex}
\paragraph{Nonlinear Parameterizations}
In practice, optimization is preferably done in unconstrained \(\mathbb{R}^p\), which then positive definiteness can be ensured by explicitly mapping the diagonal elements to positive numbers.
We denote this by the \textit{diagonal conditioner} \(\phi\). (See 
\cref{section:survey} for a brief survey on their use).
The following two parameterizations are commonly used, where \(\mD = \mathrm{diag}\left(\phi\left(\vs\right)\right) \in \mathbb{R}^{d \times d}\) denotes a diagonal matrix such that \(D_{ii} = \phi\left(s_i\right) > 0\).
{
\setlength{\belowdisplayskip}{1.ex} \setlength{\belowdisplayshortskip}{1.ex}
\setlength{\abovedisplayskip}{1.ex} \setlength{\abovedisplayshortskip}{1.ex}
\begin{definition}[\textbf{Mean-Field}]\label{def:meanfield}
  \begin{align*}
    \mC\left(\vlambda, \phi\right) &= \mathrm{diag}\left(\phi\left(\vs\right)\right),
  \end{align*}
  where \(\vs \in \mathbb{R}^{d}\) and \(\vlambda = \left(\vm, \vs\right)\).
\end{definition}

\begin{definition}[\textbf{Cholesky}]\label{def:fullrank}
  \begin{align*}
    \mC\left(\vlambda, \phi\right) &= \mathrm{diag}\left(\phi\left(\vs\right)\right) + \mL,
  \end{align*}
  where \(\vs \in \mathbb{R}^{d}\), \(\mL \in \mathbb{R}^{d \times d}\) is a strictly lower triangular matrix, \(\vlambda_{\mL} = \mathrm{vec}\left(\mL\right) \in \mathbb{R}^{\left(d - 1\right) \, d / 2}\) such that \(\vlambda = \left(\vm, \vs, \vlambda_{\mL}\right)\).
  The special case of \(\phi\left(x\right) = x\) is called the ``linear Cholesky'' parameterization.
\end{definition}
}%
%
\vspace{-2.5ex}
\paragraph{Diagonal conditioner}
For the diagonal conditioner, the softplus function \(\phi\left(x\right) = \mathrm{softplus}(x) = \log(1 + e^x)\)~\citep{dugas_incorporating_2000} or the exponential function \(\phi\left(x\right) = e^x\) is commonly used.
While using these nonlinear functions significantly complicates the analysis, assuming \(\phi\) to be 1-Lipschitz retrieves practical guarantees.

\vspace{0.5ex}
\begin{assumption}[\textbf{Lipschitz Diagonal Conditioner}]\label{assumption:phi_lipschitz}
  The diagonal conditioner \(\phi\) is 1-Lipschitz continuous.
\end{assumption}
\vspace{0.5ex}

\begin{remark}
  The softplus function is 1-Lipschitz continuous.
\end{remark}

%%% Local Variables:
%%% TeX-master: "main"
%%% End:

% \newpage

\section{Main Results}
\vspace{-.5ex}

%
\begin{assumption}\label{assumption:bijector_lipschitz_condition1}
  Let \(\phi_{+}\left(x\right)\) be a bijector, where \(x \mapsto \nicefrac{\phi^{\prime}\left(x\right)}{\phi\left(x\right)} \) is \(L_{+}\)-Lipschitz.
\end{assumption}

\begin{proposition}
  Let \(\phi_{+}\left(x\right) = e^x\). 
  Then, \(\phi_{+}\left(x\right)\) satisfies \cref{assumption:bijector_lipschitz_condition1} with any \(L_{+} = \epsilon > 0\).
\end{proposition}
\begin{proof}
  Since \(\nicefrac{\phi^{\prime}_+\left(x\right)}{\phi_+\left(x\right)} = 1\), the conclusion follows from 
  \( \abs{\nicefrac{\phi^{\prime}_+\left(x\right)}{\phi_+\left(x\right)} - \nicefrac{\phi^{\prime}_+\left(y\right)}{\phi_+\left(y\right)}} = 0.\)
\end{proof}

\begin{proposition}
  Let \(\phi_{+}\left(x\right) = \mathrm{softplus}\left(x\right)\). 
  Then, \(\phi_{+}\left(x\right)\) satisfies \cref{assumption:bijector_lipschitz_condition1} with \(L_{+} = 2\).
\end{proposition}
\begin{proof}
  Since \(\mathrm{softplus}\left(x\right) = \log\left(1 + e^x\right)\), 
  \begin{align*}
    \phi_{+}^{\prime}\left(x\right) = \frac{e^x}{1 + e^x} = \sigma\left(x\right),
  \end{align*}
  where \(\sigma\left(x\right)\) is known as the sigmoid/logistic function with the derivative \(\sigma^{\prime}\left(x\right) = \sigma\left(x\right) \left(1 - \sigma\left(x\right)\right).\)
  Now, by the Mean Value Theorem, 
  \begin{align}
    \abs{ \frac{\phi^{\prime}_{+}\left(x\right)}{\phi\left(x\right)} - \frac{\phi^{\prime}_{+}\left(y\right)}{\phi\left(y\right)} }
    \leq
    \abs{ \frac{d}{dx} \frac{\phi^{\prime}_{+}\left(x\right)}{\phi\left(x\right)} }
    \abs{
      x - y
    }.
  \end{align}
  Now, 
  \begin{align*}
    \abs{ \frac{d}{dx} \frac{\phi^{\prime}_{+}\left(x\right)}{\phi\left(x\right)} }
    &=
    \abs{
      \phi^{\prime\prime}_{+}\left(x\right) \frac{1}{\phi_{+}\left(x\right)}
      -
      {\left( \phi^{\prime}_{+}\left(x\right) \right)}^2 \frac{1}{\phi_{+}^2\left(x\right)}
    }
    \\
    &=
    \frac{1}{\phi_{+}\left(x\right)} \abs{ \phi^{\prime\prime}_{+}\left(x\right) - \frac{{\left( \phi^{\prime}_{+}\left(x\right) \right)}^2}{\phi_{+}\left(x\right)}
    }
    \\
    &=
    \frac{1}{\phi_{+}\left(x\right)} \abs{
      \sigma\left(x\right) \left( 1 - \sigma\left(x\right) \right) - \frac{\sigma^2\left(x\right)}{\phi_{+}\left(x\right)}
    }
    \\
    &=
    \frac{\sigma\left(x\right)}{\phi_{+}\left(x\right)} \abs{
       1 - \sigma\left(x\right) - \frac{\sigma\left(x\right)}{\phi_{+}\left(x\right)}
    }
    \\
    &\leq
    \frac{\sigma\left(x\right)}{\phi_{+}\left(x\right)} \abs{
       1 - \sigma\left(x\right) 
    }
    +
    {\left(
    \frac{\sigma\left(x\right)}{\phi_{+}\left(x\right)} 
    \right)}^2
    \\
    &\leq
    \frac{\sigma\left(x\right)}{\phi_{+}\left(x\right)} 
    +
    {\left(
    \frac{\sigma\left(x\right)}{\phi_{+}\left(x\right)} 
    \right)}^2,
  \end{align*}
  since \(0 \leq \sigma\left(x\right) \leq 1\), \(0 \leq \phi_{+}\left(x\right)\).
  Notice that 
  \begin{align*}
    \frac{\sigma\left(x\right)}{\phi_{+}\left(x\right)} 
    =
    \frac{e^x}{1 + e^x} 
    \frac{1}{ \log\left(1 + e^x\right) } 
  \end{align*}
  is monotonically decreasing.
  Therefore, \(\nicefrac{\sigma\left(x\right)}{\phi_{+}\left(x\right)} \leq \lim_{x \rightarrow - \infty} \nicefrac{\sigma\left(x\right)}{\phi_{+}\left(x\right)} = 1\).

  Thus,
  \begin{align*}
    \abs{ \frac{\phi^{\prime}_{+}\left(x\right)}{\phi\left(x\right)} - \frac{\phi^{\prime}_{+}\left(y\right)}{\phi\left(y\right)} }
    &\leq
    \abs{ \frac{d}{dx} \frac{\phi^{\prime}_{+}\left(x\right)}{\phi\left(x\right)} }
    \abs{
      x - y
    }
    \\
    &\leq
    \abs{
      \frac{\sigma\left(x\right)}{\phi_{+}\left(x\right)} 
      +
      {\left(
        \frac{\sigma\left(x\right)}{\phi_{+}\left(x\right)} 
        \right)}^2
    }
    \abs{
      x - y
    }
    \\
    &\leq
    2 \,
    \abs{
      x - y
    }.
  \end{align*}
\end{proof}

\begin{proposition}
  Let \(q_{\vlambda}\) belong to the location scale family as \cref{def:family}, where \(\mC \in \mathbb{R}^{d \times d}\) is formed through the Cholesky factor parameterization.
  \(\vlambda \mapsto H\left[q_{\vlambda}\right]\) is \(L\)-smooth if and only if the bijector \(\phi_+\) satisfies \cref{assumption:bijector_lipschitz_condition1}, where the smoothness constant is \(L = d L_{+}\).
\end{proposition}
\begin{proof}
  For members of the location scale family with parameters \(\left(\vm, \mC\right)\), the entropy and its gradient are given as
  \begin{align*}
    \mathrm{H}\left[q_{\vlambda}\right] = \log \abs{ \mC } \quad\text{and}\quad \nabla_{\mC} \mathrm{H}\left[q_{\vlambda}\right] = - \mC^{-\top}.
  \end{align*}

  Let \(\mC\) form a Cholesky factor.
  Then, since the determinant of a triangular matrix is the product of its diagonal elements,
  \begin{align*}
    \mathrm{H}\left[q_{\vlambda}\right] = \log \abs{\mC} = \sum_{i=1}^d \log C_{ii} = \sum_{i=1}^d \log \phi_{+}\left( d_{i} \right).
  \end{align*}
  It is apparent that the derivatives of the entropy term are now
  \begin{align*}
    \frac{\partial \mathrm{H}\left[q_{\vlambda}\right]}{ \partial d_i }
    =
    \frac{\partial \log \phi_{+}\left(d_i\right)}{ \partial d_i }
    =
    \frac{\phi^{\prime}_{+}\left(d_i\right)}{\phi_{+}\left(d_i\right)}
    %
    \quad\text{and}\quad
    %
    \frac{\partial \mathrm{H}\left[q_{\vlambda}\right]}{ \partial L_{ij} }
    &=
    0.
  \end{align*}
  Then,
  \begin{align*}
    \norm{ \nabla_{\vlambda} \mathrm{H}\left[q_{\vlambda}\right] - \nabla_{\vlambda^{\prime}} \mathrm{H}\left[q_{\vlambda}\right] }_2^2
    &=
    \sum^{d}_{i=1}
    {\left(
    \frac{\phi^{\prime}_{+}\left(d_i\right)}{\phi_{+}\left(d_i\right)}
    -
    \frac{\phi^{\prime}_{+}\left(d_i^{\prime}\right)}{\phi_{+}\left(d_i^{\prime}\right)}
    \right)}^2.
  \end{align*}

  Thus, \(\vlambda \mapsto \mathrm{H}\left[q_{\vlambda}\right]\) is (\(d L_{+}\))-smooth if and only if \(d_i \mapsto \frac{\phi^{\prime}_{+}\left(d_i\right)}{\phi_{+}\left(d_i\right)}\) is \(L_{+}\)-Lipschitz.
\end{proof}

\begin{theorem}
  Let \(f\left(\vlambda\right)\) be \(L_{f}\)-smooth, \(g\left(\vlambda\right)\) be \(L_{g}\)-smooth.
  Then,
  \begin{enumerate}
    \item \(F_{\text{ELBO}}\left(\vlambda\right)\) is (\(L_f + L_g + \))-smooth
    \item \(\Esub{\rvveta \sim q_{\vlambda}\left(\right)}{ } \)
  \end{enumerate}
\end{theorem}
\begin{proof}
  
\end{proof}

%%% Local Variables:
%%% TeX-master: "master"
%%% End:


%2
\begin{theorem}
  
\end{theorem}
\begin{proof}
  \begin{alignat}{2}
    \norm{ \mH_{\ell} }_{\mathrm{F}}^2
    &=
    \sum_{i} \sum_{j} {\left\{ \mathbb{E} \frac{\partial^2}{\partial \lambda_i \partial \lambda_j}  f\left(\vt_{\vlambda}\left(\rvvu\right)\right) \right\}}^2
    \\
    &=
    \sum_{i} \sum_{j} {\left\{
      \mathbb{E} \frac{\partial}{\partial \lambda_j}
      \inner{
        \frac{\partial \vt_{\vlambda}\left(\rvvu\right)}{\partial \lambda_i}
      }{
        \nabla f\left(\vt_{\vlambda}\left(\rvvu\right)\right)
      }
      \right\}}^2
    \\
    &=
    \sum_{i} \sum_{j} { \left\{
      \mathbb{E} 
      \inner{
        \frac{\partial \vt_{\vlambda}\left(\rvvu\right)}{\partial \lambda_i}
      }{
        \frac{\partial}{\partial \lambda_j} \nabla f\left(\vt_{\vlambda}\left(\rvvu\right)\right)
      }
      +
      \inner{
        \frac{\partial^2 \vt_{\vlambda}\left(\rvvu\right)}{\partial \lambda_i \partial \lambda_j}
      }{
        \nabla f\left(\vt_{\vlambda}\left(\rvvu\right)\right)
      }
      \right\} }^2
  \end{alignat}
\end{proof}

%%% Local Variables:
%%% TeX-master: "master"
%%% End:


% \vspace{-2.5ex}
% \begin{proof}
%     The derivative of the softplus function is the sigmoid function \(\mathrm{sigmoid}\left(x\right) = 1/\left(1 + e^{-x}\right) \leq 1\).
%     By the mean-value theorem, any function that has bounded derivatives \( \phi^{\prime}\left(x\right) < L\) for all \(x \in \mathbb{R}\) is \(L\)-Lipschitz.
% \end{proof}

%\vspace{-2ex}
\subsection{Key Lemmas}
%\vspace{-.5ex}
The main challenge in studying BBVI is that the gradient of the composed function \(\nabla_{\vlambda} f \left( \vt_{\vlambda} \left( \vu \right) \right)\) is different from \(\nabla f\).
For the matrix square root parameterization, \citet[Lemma 1]{domke_provable_2019} established the connection through \cref{thm:variational_gradient_norm_identity}.
We generalize this result to nonlinear parameterizations:


\begin{theoremEnd}[\keylemmaproofoption,category=upperboundkeylemmagradientnormidentity]{lemma}\label{thm:general_variational_gradient_norm_identity}
  Let \(\vt_{\vlambda}: \mathbb{R}^d \rightarrow \mathbb{R}^d\) be a location-scale reparameterization function (\cref{def:reparam}) with some differentiable function \(f : \mathbb{R}^d \rightarrow \mathbb{R} \).
  Then, for \(\vg_{f} \triangleq \nabla f\left( \vt_{\vlambda}\left(\vu\right) \right)\), 
  \begin{enumerate}[label=(\roman*)]
    \vspace{-2ex}
    \setlength\itemsep{-1ex}
    \item Mean-Field
      \vspace{-1ex}
    {\small
    \setlength{\belowdisplayskip}{1ex} \setlength{\belowdisplayshortskip}{1ex}
    \setlength{\abovedisplayskip}{1ex} \setlength{\abovedisplayshortskip}{1ex}
    \begin{alignat*}{2}
      \hspace{-3em}
      \norm{ \nabla_{\vlambda} f\left( \vt_{\vlambda}\left(\vu\right) \right) }_2^2
      = 
      {\lVert \vg_{f} \rVert}_2^2
      +
      \vg_{f}^{\top}
      \mU \mPhi
      \vg_{f},
      \qquad\qquad
    \end{alignat*}
  }

    \item Cholesky
      {\small%
    {
    \setlength{\belowdisplayskip}{1ex} \setlength{\belowdisplayshortskip}{1ex}
    \setlength{\abovedisplayskip}{1ex} \setlength{\abovedisplayshortskip}{1ex}
    \begin{alignat*}{2}
      \norm{ \nabla_{\vlambda} f\left( \vt_{\vlambda}\left(\vu\right) \right) }_2^2
      &=
      {\lVert \vg_{f} \rVert}_2^2 + \vg_{f}^{\top} \mSigma \vg_{f}
      +
      \vg_{f}^{\top}
      \mU
      \left(
      \mPhi
      - 
      \boldupright{I}
      \right)
      \vg_{f},
    \end{alignat*}
      }%
      %\hspace{-2ex}
  }
  \end{enumerate}
  where \(\mU,\mPhi,\mSigma\) are diagonal matrices, which the diagonals are defined as 
  {
    \setlength{\belowdisplayskip}{.5ex} \setlength{\belowdisplayshortskip}{.5ex}
    \setlength{\abovedisplayskip}{.5ex} \setlength{\abovedisplayshortskip}{.5ex}
    \[
    U_{ii} = u_i^2,\quad
    \Phi_{ii} = {\phi^{\prime}\left(s_i\right)}^2,\quad
    \Sigma_{ii} = {\textstyle\sum^{i}_{j=1}} u_j^2,
    \]
  }
  %\vspace{-2ex}
  and \(\phi\) is a diagonal conditioner for the scale matrix.
\end{theoremEnd}
\vspace{-2ex}
\begin{proofEnd}
  The proof starts by applying the Chain Rule and then computing the quadratic norm of the gradient as
  \begin{alignat}{2}
    &\norm{\nabla_{\vlambda} f\left( \vt_{\vlambda}\left(\vu\right) \right) }_2^2
    \nonumber
    \\
    &\;= 
    {\left(
      \frac{
        \partial \vt_{\vlambda}\left(\vu\right)
      }{
        \partial \vlambda
      }
      \nabla f\left( \vt_{\vlambda}\left(\vu\right) \right)
    \right)}^{\top}
    \frac{
      \partial \vt_{\vlambda}\left(\vu\right)
    }{
      \partial \vlambda
    }
    \nabla f\left( \vt_{\vlambda}\left(\vu\right) \right)
    \nonumber
    \\
    &\;=
    {\nabla f^{\top}\left( \vt_{\vlambda}\left(\vu\right) \right)}
    {\left(
      \frac{
        \partial \vt_{\vlambda}\left(\vu\right)
      }{
        \partial \vlambda
      }
    \right)}^{\top}
    \frac{
      \partial \vt_{\vlambda}\left(\vu\right)
    }{
      \partial \vlambda
    }
    \nabla f\left( \vt_{\vlambda}\left(\vu\right) \right)
    \nonumber
    \\
    &\;=
    {\vg_{f}^{\top}}
    {\left(
      \frac{
        \partial \vt_{\vlambda}\left(\vu\right)
      }{
        \partial \vlambda
      }
    \right)}^{\top}
    \frac{
      \partial \vt_{\vlambda}\left(\vu\right)
    }{
      \partial \vlambda
    }
    \vg_{f}.\label{eq:variational_gradient_norm_identity_eq1}
  \end{alignat}
  Naturally, the derivative of the reparameterization function will depend on the specific parameterization used.

  \paragraph{Proof for Cholesky}

  Let \(p\) denote the number of scalar variational parameters such that \(\vlambda = (\lambda_1, \ldots, \lambda_p)\).
  Then,
  \begin{alignat*}{2}
    &{\left(
      \frac{
        \partial \vt_{\vlambda}\left(\vu\right)
      }{
        \partial \vlambda
      }
    \right)}^{\top}
    \frac{
      \partial \vt_{\vlambda}\left(\vu\right)
    }{
      \partial \vlambda
    }
    \\
    &\;=
    \sum^{d}_{i=1} 
    \frac{
      \partial \vt_{\vlambda}\left(\vu\right)
    }{
      \partial m_i
    }
    {\left(
    \frac{
      \partial \vt_{\vlambda}\left(\vu\right)
    }{
      \partial m_i
    }
    \right)}^{\top}
    +
    \sum^{d}_{i=1} 
    \sum^{d}_{j \leq i} 
    \frac{
      \partial \vt_{\vlambda}\left(\vu\right)
    }{
      \partial \lambda_{C_{ij}}
    }
    {\left(
    \frac{
      \partial \vt_{\vlambda}\left(\vu\right)
    }{
      \partial \lambda_{C_{ij}}
    }
    \right)}^{\top},
  \end{alignat*}
  where \(\lambda_{C_{ij}}\) denote the parameter responsible for the \(ij\)-th entry of \(\mC\), \(C_{ij}\).
  Notice that, unlike for the matrix square root parameterization~\citep{domke_provable_2019}, the sum for \(C_{ij}\) is only over the lower triangular section.

  For the derivatives with respect to \(m_i\) and \(C_{ij}\), \citet{domke_provable_2020, domke_provable_2019} show that
  \begin{alignat}{2}
    \frac{\partial \vt_{\vlambda}\left(\vu\right) }{ \partial m_i }   &= \boldupright{e}_i \quad
    \frac{\partial \vt_{\vlambda}\left(\vu\right) }{ \partial C_{ij} } &= \boldupright{e}_i u_j,\label{eq:covariance_derivative}
  \end{alignat}
  where \(\boldupright{e}_i\) is the unit basis of the \(i\)th component.

  Therefore,
  \begin{alignat}{2}
    &{\left(
      \frac{
        \partial \vt_{\vlambda}\left(\vu\right)
      }{
        \partial \vlambda
      }
    \right)}^{\top}
    \frac{
      \partial \vt_{\vlambda}\left(\vu\right)
    }{
      \partial \vlambda
    }
    \nonumber
    \\
    &\;=
    \sum^{d}_{i=1} 
    \boldupright{e}_i
    \boldupright{e}_i^{\top}
    +
    \sum^{d}_{i=1} 
    \sum_{j \leq i} 
    \frac{
      \partial \vt_{\vlambda}\left(\vu\right)
    }{
      \partial \lambda_{C_{ij}}
    }
    {\left(
    \frac{
      \partial \vt_{\vlambda}\left(\vu\right)
    }{
      \partial \lambda_{C_{ij}}
    }
    \right)}^{\top}
    \nonumber
    \\
    &\;=
    \boldupright{I}
    +
    \underbrace{
    \sum^{d}_{i=1} 
    \frac{
      \partial \vt_{\vlambda}\left(\vu\right)
    }{
      \partial \lambda_{C_{ii}}
    }
    {\left(
    \frac{
      \partial \vt_{\vlambda}\left(\vu\right)
    }{
      \partial \lambda_{C_{ii}}
    }
    \right)}^{\top}
    }_{\text{diagonal of \(\mC\)}}
    \nonumber
    \\
    &\qquad+
    \underbrace{
    \sum^{d}_{i=1} 
    \sum_{j < i} 
    \frac{
      \partial \vt_{\vlambda}\left(\vu\right)
    }{
      \partial \lambda_{C_{ij}}
    }
    {\left(
    \frac{
      \partial \vt_{\vlambda}\left(\vu\right)
    }{
      \partial \lambda_{C_{ij}}
    }
    \right)}^{\top}}_{\text{off-diagonal of \(\mC\)}},
    \label{eq:thm:variational_gradient_norm_identity_eq2}
  \end{alignat}
  leaving us with the derivatives of the scale term.

  The gradient with respect to \(\lambda_{C_{ij}}\), however, depends on the parameterization.
  That is,
  \begin{alignat}{2}
    \frac{
      \partial \vt_{\vlambda}\left(\vu\right)
    }{
      \partial \lambda_{C_{ij}}
    }
    &=
    \frac{
      \partial \vt_{\vlambda}\left(\vu\right)
    }{
      \partial C_{ij}
    }
    \frac{
      \partial C_{ij}
    }{
      \partial \lambda_{C_{ij}}
    }
    &=
    \boldupright{e}_i u_j
    \frac{
      \partial C_{ij}
    }{
      \partial \lambda_{C_{ij}}
    }. \label{eq:thm:variational_gradient_norm_identity_covderivative}
  \end{alignat}

  For the diagonal elements, \(\lambda_{C_{ii}} = s_i\).
  Thus,
  \begin{align}
    \frac{
      \partial C_{ii}
    }{
      \partial s_i
    }
    =
    \frac{
      \partial \phi\left(s_i\right)
    }{
      \partial s_i
    }
    =
    \phi^{\prime}\left(s_i\right). \label{eq:thm:variational_gradient_norm_identity_diag}
  \end{align}
  And for the off-diagonal elements, \(\lambda_{L_{ij}} = L_{ij}\), and
  \begin{align}
    \frac{
      \partial C_{ij}
    }{
      \partial L_{ij}
    }
    =
    1. \label{eq:thm:variational_gradient_norm_identity_offdiag}
  \end{align}

  Plugging \cref{eq:thm:variational_gradient_norm_identity_diag,eq:thm:variational_gradient_norm_identity_offdiag,eq:thm:variational_gradient_norm_identity_covderivative} into \cref{eq:thm:variational_gradient_norm_identity_eq2},
  \begin{alignat}{2}
    &{\left(
      \frac{
        \partial \vt_{\vlambda}\left(\vu\right)
      }{
        \partial \vlambda
      }
    \right)}^{\top}
    \frac{
      \partial \vt_{\vlambda}\left(\vu\right)
    }{
      \partial \vlambda
    }
    \nonumber
    \\
    &\;=
    \boldupright{I}
    +
    \underbrace{
    \sum^{d}_{i=1} 
    {\left( u_i \phi^{\prime}\left(s_i\right) \right)}^2
    \boldupright{e}_i \boldupright{e}_i^{\top}
    }_{\text{diagonal of \(\mC\)}}
    +
    \underbrace{
    \sum^{d}_{i=1} 
    \sum_{j=1, j < i} 
    u_j^2 \, \boldupright{e}_i \boldupright{e}_i^{\top}
    }_{\text{off-diagonal of \(\mC\)}}
    \nonumber
    \\
    &\;=
    \boldupright{I}
    +
    \underbrace{
    \sum^{d}_{i=1} 
    u_i^2 {\left(\phi^{\prime}\left(s_i\right) \right)}^2
    \boldupright{e}_i \boldupright{e}_i^{\top}
    }_{\text{diagonal of \(\mC\)}}
    +
    \underbrace{
    \sum^{d}_{i=1} 
    \sum_{j \leq i} 
    u_j^2 \, \boldupright{e}_i \boldupright{e}_i^{\top}
    -
    \sum^{d}_{i=1} 
    u_i^2 \, \boldupright{e}_i \boldupright{e}_i^{\top}
    }_{\text{off-diagonal of \(\mC\)}}
    \nonumber
    \\
    &\;=
    \boldupright{I}
    +
    \underbrace{
      \mU \, \mPhi
    }_{\text{diagonal of \(\mC\)}}
    +
    \underbrace{
    \mSigma
    -
    \mU
    }_{\text{off-diagonal of \(\mC\)}}
    \nonumber
    \\
    &\;=
    \left( \boldupright{I} + \mSigma \right)
    +
    \mU \left( \mPhi - \boldupright{I} \right), \label{eq:variational_gradient_norm_identity_jacinner}
  \end{alignat}
  where \(\mU,\mPhi,\mSigma\) are diagonal matrices defined as
  \begin{alignat*}{2}
    &
    \mPhi
    &&=
    \mathrm{diag}\left(
    \left[ {\phi^{\prime}\left(s_1\right)}^2, \ldots , {\phi^{\prime}\left(s_d\right)}^2 \right]
    \right)
    \\
    &\mU
    &&=
    \mathrm{diag}\left(
    \left[ u_1^2, \ldots, u_d^2 \right]
    \right)
    \\
    &\mSigma
    &&=
    \mathrm{diag}\left(
    \left[ u_1^2, u_1^2 + u_2^2,\, \ldots\,, {\textstyle\sum^{d}_{i=1} u_i^2} \right]
    \right).
  \end{alignat*}
  The major difference with the proof of \citet[Lemma 8]{domke_provable_2019} for the matrix square root case is that we only sum the \(u_j^2 \boldupright{e}_i \boldupright{e}_i^{\top}\) terms over the \textit{lower diagonal elements}.
  This is the variance reduction effect we get from using the Cholesky parameterization.

  Coming back to \cref{eq:variational_gradient_norm_identity_eq1}, 
  \begin{alignat}{2}
    &\norm{\nabla_{\vlambda} f\left( \vt_{\vlambda}\left(\vu\right) \right)}_2^2
    \nonumber
    \\
    \;&=
      \vg_f^{\top}
      {\left(
        \frac{
          \partial \vt_{\vlambda}\left(\vu\right)
        }{
          \partial \vlambda
        }
        \right)}^{\top}
      \frac{
        \partial \vt_{\vlambda}\left(\vu\right)
      }{
        \partial \vlambda
      }
      \vg
    \nonumber
    \\
    &=
      \vg_{f}^{\top}
      \Big(
        \left( \boldupright{I} + \mSigma \right)
        +
        \mU \left( \mPhi - \boldupright{I} \right)
      \Big)
      \vg_{f}
    \nonumber
    \\
    &=
    {\lVert \vg_{f} \rVert}_2^2
    +
    \vg_{f}^{\top} \mSigma \vg_{f}
    +
    \vg_{f}^{\top}
    \mU \left( \mPhi - \boldupright{I} \right)
    \vg_{f}.
    \label{eq:variational_gradient_norm_identity_conclusion}
  \end{alignat}

  \paragraph{Proof for Mean-field}
  For the mean-field variational family, the covariance has only diagonal elements.
  Therefore, \cref{eq:variational_gradient_norm_identity_jacinner} becomes
  \begin{alignat}{2}
    {\left(
      \frac{
        \partial \vt_{\vlambda}\left(\vu\right)
      }{
        \partial \vlambda
      }
    \right)}^{\top}
    \frac{
      \partial \vt_{\vlambda}\left(\vu\right)
    }{
      \partial \vlambda
    }
    &=
    \boldupright{I} + \mU \mPhi,
    \nonumber
  \end{alignat}
  and \cref{eq:variational_gradient_norm_identity_conclusion} becomes
  \begin{alignat}{2}
    \norm{\nabla_{\vlambda} f\left( \vt_{\vlambda}\left(\vu\right) \right)}_2^2
    =
      \vg_{f}^{\top}
      \left(
        \boldupright{I}
        +
        \mU \mPhi
      \right)
      \vg_{f}
    =
    {\lVert \vg_{f} \rVert}_2^2
    +
    \vg_{f}^{\top}
    \mU \mPhi
    \vg_{f}.
    \nonumber
  \end{alignat}
\end{proofEnd}

Note that the relationships in this lemma are all equalities, which can be bounded with known quantities, as done in the next lemma.
We note here that if any of our analyses were to be improved, this shall by done by obtaining tighter bounds on the equalities in \cref{thm:general_variational_gradient_norm_identity}.

\begin{theoremEnd}[\keylemmaproofoption,category=upperboundkeylemmagradientnormbound]{lemma}\label{thm:general_variational_gradient_norm_bound}
Let \(\vt_{\vlambda}: \mathbb{R}^d \rightarrow \mathbb{R}^d\) be a location-scale reparameterization function (\cref{def:reparam}), \(f : \mathbb{R}^d \rightarrow \mathbb{R} \) be a differentiable function, and let \(\phi\) satisfy \cref{assumption:phi_lipschitz}.
  \vspace{-5ex}
  \begin{enumerate}[label=(\roman*)]
    \setlength\itemsep{-1ex}
    \item Mean-Field
    {%
    \setlength{\belowdisplayskip}{1ex} \setlength{\belowdisplayshortskip}{1ex}%
    \setlength{\abovedisplayskip}{1ex} \setlength{\abovedisplayshortskip}{1ex}%
      \[
        \norm{\nabla_{\vlambda} f\left( \vt_{\vlambda}\left(\vu\right) \right)}_2^2
        \leq
        \left(1 + \norm{ \mU }_{\mathrm{F}} \right)
        {\lVert \nabla f\left( \vt_{\vlambda}\left(\vu\right) \right) \rVert}_2^2,
      \]
      where \(\mU\) is a diagonal matrix such that \(U_{ii} = u_i^2\).
    }%
    \item Cholesky
    {%
    \setlength{\belowdisplayskip}{1ex} \setlength{\belowdisplayshortskip}{1ex}%
    \setlength{\abovedisplayskip}{1ex} \setlength{\abovedisplayshortskip}{1ex}%
      \[
        \norm{\nabla_{\vlambda} f\left( \vt_{\vlambda}\left(\vu\right) \right)}_2^2
        \leq
        \left(1 + \norm{ \vu }_2^2 \right)
        {\lVert \nabla f\left( \vt_{\vlambda}\left(\vu\right) \right) \rVert}_2^2,
      \]
      }%
      where the equality holds for the matrix square root parameterization.
  \end{enumerate}
\end{theoremEnd}
\vspace{-2ex}
\begin{proofEnd}
  The proof continues from the result of \cref{thm:general_variational_gradient_norm_identity}.

  \paragraph{Proof for Cholesky}
  \cref{thm:general_variational_gradient_norm_identity} shows that
  \begin{alignat*}{2}
    \norm{ \nabla_{\vlambda} f\left( \vt_{\vlambda}\left(\vu\right) \right) }_2^2
    = 
    {\lVert \vg_{f} \rVert}_2^2
    +
    \vg_{f}^{\top} \mSigma \vg_{f}
    +
    \vg_{f}^{\top}
    \mU \left( \mPhi - \boldupright{I} \right)
    \vg_{f},
  \end{alignat*}
  where \(\vg_f = \nabla f\left(\vt_{\vlambda}\left(\vu\right)\right) \).

  By the 1-Lipschitz assumption, the entries of the diagonal matrix \(\Phi\) satisfy
  \begin{align*}
    \Phi_{ii} = {\phi^{\prime}\left(d_i\right)}^2 \leq 1,
  \end{align*}
  which means
  \begin{alignat*}{2}
    \mPhi \preceq \boldupright{I}
    \;\Rightarrow\;
    \mU \left( \mPhi - \boldupright{I} \right)
    \preceq
    0
    \;\Rightarrow\;
    {\vg_{f}}^{\top} \mU \left( \mPhi - \boldupright{I} \right) \vg_{f} \leq 0.
  \end{alignat*}
  Therefore, for the full-rank Cholesky parameterization and a 1-Lipschitz conditioner \(\phi\),
  \begin{alignat*}{2}
    &\norm{\nabla_{\vlambda} f\left( \vt_{\vlambda}\left(\vu\right) \right)}_2^2
    \\
    &\;=
    {\lVert \nabla f\left( \vt_{\vlambda}\left(\vu\right) \right) \rVert}_2^2
    +
    {\vg_{f}}^{\top}
    \mSigma
    \vg_{f}
    +
    {\vg_{f}}^{\top} \mU \left( \mPhi - \boldupright{I} \right) \vg_{f}
    \\
    &\;\leq
    {\lVert \nabla f\left( \vt_{\vlambda}\left(\vu\right) \right) \rVert}_2^2
    +
    {\vg_{f}}^{\top}
    \mSigma
    \vg_{f}
    \\
    &\;\leq
    {\lVert \nabla f\left( \vt_{\vlambda}\left(\vu\right) \right) \rVert}_2^2
    +
    \norm{ \mSigma }_{2,2} {\lVert \nabla f\left( \vt_{\vlambda}\left(\vu\right) \right) \rVert}_2^2
    \\
    &\;=
    {\lVert \nabla f\left( \vt_{\vlambda}\left(\vu\right) \right) \rVert}_2^2
    +
    \left( \sum^{d}_{i=1} u_{i}^2 \right)  {\lVert \nabla f\left( \vt_{\vlambda}\left(\vu\right) \right) \rVert}_2^2
    \\
    &\;=
    \left(1 + \norm{\vu}^2_2\right) {\lVert \nabla f\left( \vt_{\vlambda}\left(\vu\right) \right) \rVert}_2^2,
  \end{alignat*}
  where \(\norm{\mU}_{2,2}\) is the \(L_2\) operator norm of \(\mU\).
  This upper bound coincides with that of the matrix square root parameteration.
  Thus, unforunately, this bound fails to acknowledge the lower variance of the Cholesky parameterization, coinciding with that of the matrix square root parameterization.

  \paragraph{Proof for Mean-field (\cref{def:meanfield})}
  For the mean-field parameterization,~\cref{thm:general_variational_gradient_norm_identity} shows that
  \begin{alignat*}{2}
    \norm{ \nabla_{\vlambda} f\left( \vt_{\vlambda}\left(\vu\right) \right) }_2^2
    =
    {\lVert \vg_{f} \rVert}_2^2
    +
    \vg_{f}^{\top}
    \mU \mPhi
    \vg_{f}.
  \end{alignat*}

  For the second term,  
  \begin{alignat*}{2}
    \vg_{f}^{\top}
    \mU \mPhi
    \vg_{f}
    \leq
    {\lVert \mU \rVert}_{2,2} {\lVert \mPhi \rVert}_{2,2}
    {\lVert \vg_{f} \rVert}^2_2.
  \end{alignat*}
  By the \(1\)-Lipschitzness of \(\phi\),
  \[
    {\lVert \mPhi \rVert}_{2,2}
    = \sigma_{\mathrm{max}}\left( \mPhi \right)
    = \max_{i = 1, \ldots, d} {\phi^{\prime}\left( s_i \right)}^2
    \leq 1.
  \]
  Then,
  \begin{alignat}{2}
    \vg_{f}^{\top}
    \left( \mU \mPhi \right)
    \vg_{f}
    &\leq
    {\lVert \mU \rVert}_{2,2} \,
    {\lVert \vg_{f} \rVert}^2_2 \label{eq:variational_gradient_norm_identity_mf_eq1}
    \\
    &\leq
    {\lVert \mU \rVert}_{\mathrm{F}} \,
    {\lVert \vg_{f} \rVert}^2_2, \label{eq:variational_gradient_norm_identity_mf_eq2}
  \end{alignat}
  which gives the result.
  Here, unlike the bounds on \(\mPhi\), the bounds in \cref{eq:variational_gradient_norm_identity_mf_eq1,eq:variational_gradient_norm_identity_mf_eq2} are quite loose, and become looser as the dimensionality increases.

%%   We conclude as
%%   \begin{alignat*}{2}
%%     \norm{ \nabla_{\vlambda} f\left( \vt_{\vlambda}\left(\vu\right) \right) }_2^2
%%     \leq
%%     \norm{ \nabla f\left(\vt_{\vlambda}\left(\vu\right)\right) }_2^2
%%     + \frac{1}{2}{\lVert \vg_{f} \rVert}_2^2 + \frac{1}{2}\norm{\vu}_2^2
%%     =
%%     \frac{3}{2} \norm{ \nabla f\left(\vt_{\vlambda}\left(\vu\right)\right) }_2^2
%%     +
%%     \frac{1}{2} \norm{\vu}_2^2
%%   \end{alignat*}
\end{proofEnd}


%%% Local Variables:
%%% TeX-master: "main"
%%% End:


\cref{thm:general_variational_gradient_norm_identity} act as the interface between the properties of the parameterization and the likelihood \(f\).

\begin{remark}[\textbf{Variance Reduction Through \(\phi\)}]
  A \textit{nonlinear} Cholesky parameterization with a 1-Lipschitz \(\phi\) achieves lower or equal variance compared to the matrix square root and \textit{linear} Cholesky, where the equality is achieved with the matrix square root parameterization.
\end{remark}

\vspace{-1.5ex}%
\paragraph{Dimension Dependence of Mean-Field}
The superior dimensional dependence of the mean-field parameterization is given by the following lemma:


\begin{theoremEnd}[\keylemmaproofoption,category=upperboundkeylemmameanfield]{lemma}\label{thm:meanfield_u_identity}
  Let the assumptions of~\cref{thm:general_variational_gradient_norm_bound} hold and \(\rvvu \sim \varphi\) satisfy \cref{assumption:symmetric_standard}.
  Then, for the mean-field parameterization,
    {
    \setlength{\belowdisplayskip}{1.5ex} \setlength{\belowdisplayshortskip}{1.5ex}
    \setlength{\abovedisplayskip}{1.5ex} \setlength{\abovedisplayshortskip}{1.5ex}
  \begin{alignat*}{2}
    &\mathbb{E}\norm{\vt_{\vlambda}\left(\rvvu\right) - \vz}_2^2 \left(1 + \norm{\mathbfsfit{U}}_{\mathrm{F}} \right)
    \\
    &\quad\leq
    \left(\sqrt{d \kappa} + \kappa\sqrt{d} + 1\right) \, \norm{ \vm - \vz }_2^2
    +
    \left(2 \kappa \sqrt{d} + 1\right)
    \norm{\mC}_{\mathrm{F}}^2.
  \end{alignat*}
  }
\end{theoremEnd}
\vspace{-1ex}
\begin{proofEnd}
  The key idea is to prove a similar result as \cref{thm:reparam_u_identity}, but with better constants to reflect that the mean-field parameterization has a lower variance. 

  First,
  \begin{align}
    &\mathbb{E}\norm{\vt_{\vlambda}\left(\rvvu\right) - \vz}_2^2 \, \left( 1 + \norm{\mathbfsfit{U}}_{\mathrm{F}} \right) \nonumber \\
    &\;=
    \mathbb{E}\norm{\vt_{\vlambda}\left(\rvvu\right) - \vz}_2^2 
    + \mathbb{E} \norm{\mathbfsfit{U}}_{\mathrm{F}} \, \norm{\vt_{\vlambda}\left(\rvvu\right) - \vz}_2^2, 
    \nonumber
\shortintertext{applying \cref{thm:reparam_quadratic},}
    &\;=
    \norm{ \vm - \vz }_2^2 + \norm{\mC}_{\mathrm{F}}
    +
    \mathbb{E} \norm{\mathbfsfit{U}}_{\mathrm{F}} \, \norm{\vt_{\vlambda}\left(\rvvu\right) - \vz}_2^2.
    \label{thm:meanfield_eq0}
  \end{align}
  
  The last term decomposes as 
  \begin{alignat}{2}
    \mathbb{E} \norm{\mathbfsfit{U}}_{\mathrm{F}} \norm{\vt_{\vlambda}\left(\rvvu\right) - \vz}_2^2
    \nonumber
    &=
    \underbrace{
      \mathbb{E} \norm{\mathbfsfit{U}}_{\mathrm{F}} \, \rvvu^{\top} \mC^{\top} \mC \rvvu 
    }_{\text{Term \ding{182}}}
    \nonumber
    \\
    &\quad+
    2\, 
    \underbrace{
      \mathbb{E} \norm{\mathbfsfit{U}}_{\mathrm{F}} \, \rvvu^{\top} \mC^{\top} \left( \vm - \vz \right)
    }_{\text{Term \ding{183}}}
    \nonumber
    \\
    &\quad+
    \underbrace{\mathbb{E} \norm{\mathbfsfit{U}}_{\mathrm{F}}}_{\text{Term \ding{184}}} \, \norm{ \vm - \vz }_2^2.
    \label{thm:meanfield_eq1}
  \end{alignat}
  We will now focus on the stochastic terms \ding{182}-\ding{184} one by one.
  
  First, for Term \ding{182}, notice that the mean-field parameterization implies that \(\mC = \mathrm{diag}\left( c_1, \ldots, c_d \right)\).
  Thus, 
  \begin{alignat}{2}
    \mathbb{E} \norm{\mathbfsfit{U}}_{\mathrm{F}} \,  \rvvu^{\top} \mC^{\top} \mC \rvvu 
    &=
    \mathbb{E} \left( \sqrt{ \sum_{i=1}^d \rvu_i^4 } \right) \left( \sum_{i=1}^d c_i^2 \, u_i^2 \right)
    \nonumber
    \\
    &=
    \sum_{i=1}^d c_i^2 \, \mathbb{E} \left( \sqrt{ \sum_{j=1}^d \rvu_j^4 } \right) \rvu_i^2,
    \nonumber
\shortintertext{applying Cauchy-Schwarz inequality for expectations,}
    &\leq
    \sum_{i=1}^d c_i^2 \sqrt{ \left( \mathbb{E} \sum_{j=1}^d \rvu_j^4  \right) \left( \mathbb{E}  \rvu_i^4 \right) }
    \nonumber
\shortintertext{and given \cref{assumption:symmetric_standard},}
    &\;=
    \sum_{i=1}^d c_i^2 \, \sqrt{ d \kappa^2 }
    \nonumber
    \\
    &\;=
    \kappa \sqrt{d} \, \norm{ \mC }_{\mathrm{F}}^2.
    \label{thm:meanfield_eq5}
  \end{alignat}
  
  Term \ding{183} can be bounded as 
  \begin{alignat}{2}
    &\mathbb{E} 
      \norm{\mathbfsfit{U}}_{\mathrm{F}} \, \rvvu^{\top} 
    \mC^{\top} \left( \vm - \vz \right) 
    \nonumber
\shortintertext{using the Cauchy-Schwarz inequality for vectors as}
    &\;\leq
    %
    \mathbb{E} \norm{\mathbfsfit{U}}_{\mathrm{F}} \, \norm{\mC \rvvu}_2 \norm{ \vm - \vz }_2,
    \nonumber
\shortintertext{again, applying the inequality for expectations,}
    &\;=
    \sqrt{
    \mathbb{E}
    \norm{\mathbfsfit{U}}_{\mathrm{F}}^2 \,
    \mathbb{E}
    \norm{\mC \rvvu}_2^2 
    } \,
    \norm{ \vm - \vz }_2
    \nonumber
    \\
    &\;=
    \sqrt{
    \mathbb{E}
    \left(
    \sum^d_{i=1} \rvu_i^4  
    \right) \,
    \mathrm{tr}\left(\mC^{\top} \mC \, \mathbb{E} \rvvu \rvvu^{\top} \right)
    } \,
    \norm{ \vm - \vz }_2,
    \nonumber
\shortintertext{from \cref{assumption:symmetric_standard},}
    &\;=
    \sqrt{
    d \kappa \,
    \mathrm{tr}\left(\mC^{\top} \mC \right)
    } \,
    \norm{ \vm - \vz }_2
    \nonumber
    \\
    &\;=
    \sqrt{d \kappa} \,
    \norm{\mC}_{\mathrm{F}} \,
    \norm{ \vm - \vz }_2
     \nonumber
    \\
    &\;=
    \sqrt{d \kappa} \,
    \sqrt{ \norm{\mC}_{\mathrm{F}}^2 \,
           \norm{ \vm - \vz }_2^2 },
    \nonumber
\shortintertext{and by the arithmetic mean-geometric mean inequality,}
    &\;=
    \frac{\sqrt{d \kappa}}{2}
    \left(
      \norm{\mC}_{\mathrm{F}}^2
      +
      \norm{ \vm - \vz }_2^2
    \right).
    \label{thm:meanfield_eq3}
  \end{alignat}

  Finally, Term \ding{184} follows as
  \begin{alignat}{2}
    \mathbb{E} \norm{\mathbfsfit{U}}_{\mathrm{F}}
    &=
    \mathbb{E} \sqrt{ \sum_{i=1}^d \rvu_i^4  },
    \nonumber
\shortintertext{using Jensen's inequality,}
    &\leq
    \sqrt{ \mathbb{E} \sum_{i=1}^d \rvu_i^4 }
    \nonumber
    \\
    &=
    \sqrt{ d \kappa }.
    \label{thm:meanfield_eq2}
  \end{alignat}

  Combining all the results, \cref{thm:meanfield_eq0} becomes
  \begin{alignat*}{2}
    &\mathbb{E}\norm{\vt_{\vlambda}\left(\rvvu\right) - \vz}_2^2 \, \left( 1 + \norm{\mathbfsfit{U}}_{\mathrm{F}} \right)
    \\
    &\;\leq
    \norm{ \vm - \vz }_2^2 + \norm{\mC}_{\mathrm{F}}^2
    \\ 
    &\;\quad+
    \mathbb{E} \norm{\mathbfsfit{U}}_{\mathrm{F}} \, \rvvu^{\top} \mC^{\top} \mC \rvvu
    \\ 
    &\;\quad+
    2\,\mathbb{E} \norm{\mathbfsfit{U}}_{\mathrm{F}} \, \rvvu^{\top} \mC^{\top} \norm{ \vm - \vz }_2^2
    \\ 
    &\;\quad+
    \mathbb{E} \norm{\mathbfsfit{U}}_{\mathrm{F}} \, \norm{ \vm - \vz }_2^2
\shortintertext{and applying \cref{thm:meanfield_eq2,thm:meanfield_eq3,thm:meanfield_eq5},}
    &\;\leq
    \norm{ \vm - \vz }_2^2 + \norm{\mC}_{\mathrm{F}}^2
    \\ 
    &\;\quad+
    \kappa\sqrt{d} \norm{\mC}_{\mathrm{F}}
    \\ 
    &\;\quad+
    \kappa \sqrt{d} \left(
     \norm{\mC}_{\mathrm{F}}^2 + \norm{\vm - \vz}_2^2
    \right)
    \\ 
    &\;\quad+
    \sqrt{d \kappa} \, \norm{ \vm - \vz }_2^2
    \\ 
    &\;=
    \left(\sqrt{d \kappa} + \kappa\sqrt{d} + 1\right) \, \norm{ \vm - \vz }_2^2
    +
    \left(2 \kappa \sqrt{d} + 1\right)
    \norm{\mC}_{\mathrm{F}}^2.
  \end{alignat*}

\end{proofEnd}

%%% Local Variables:
%%% TeX-master: "main"
%%% End:


\begin{remark}[\textbf{Superior Variance of Mean-Field}]\label{remark:meanfield_superiority}
  The mean-field parameterization has {\small\(\mathcal{O}\left(\sqrt{d}\right)\)} dimensional dependence compared to the \(\mathcal{O}\left(d\right)\) dimensional dependence of the full-rank parameterizations in \cref{thm:reparam_u_identity}.
\end{remark}



\begin{theoremEnd}[category=common]{lemma}[\citealt{domke_provable_2019}, Lemma 9]
\label{thm:u_identities}
  Let \(\rvvu = \left(\rvu_1, \rvu_2, \ldots, \rvu_d\right)\) be a \(d\)-dimensional vector-valued random variable with zero-mean independently and identically distributed components.
  Then,
  \begin{alignat*}{2}
    &\mathbb{E}\rvvu \rvvu^{\top} &&= \left( \mathbb{E} \rvu_i^2 \right) \boldupright{I}
    \\
    &\mathbb{E}\norm{\rvvu}_2^2 &&= d \, \mathbb{E} \rvu_i^2
    \\
    &\mathbb{E} \rvvu \left( 1 + \norm{\rvvu}_2^2 \right) &&= \left( \mathbb{E} \rvu_i^3 \right) \mathbf{1}
    \\
    &\mathbb{E} \rvvu \rvvu^{\top} \rvvu \rvvu^{\top} &&= \left( \left(d - 1\right) \, {\left( \mathbb{E} \rvu_i^2 \right)}^2 + \mathbb{E}\rvu_i^4 \right) \boldupright{I}.
  \end{alignat*}
\end{theoremEnd}

\begin{theoremEnd}[category=common]{lemma}[\citealt{domke_provable_2019}, Lemma 1]
\label{thm:variational_gradient_norm_identity}
  Let \(\vt_{\vlambda}: \mathbb{R}^d \rightarrow \mathbb{R}^d\) be a location-scale reparameterization function (\cref{def:reparam}).
  Also, let \(f : \mathbb{R}^d \mapsto \mathbb{R} \) be some differentiable function.
  Then,
  \begin{alignat*}{2}
    \norm{\nabla_{\vlambda} f\left( \vt_{\vlambda}\left(\vu\right) \right) }_2^2
    = 
    \norm{\nabla f\left( \vt_{\vlambda}\left(\vu\right) \right) }_2^2 \left(1 + \norm{\vu}_2^2\right).
  \end{alignat*}
\end{theoremEnd}

\begin{theoremEnd}[category=common]{lemma}[\citealt{domke_provable_2019}, Lemma 1]
\label{thm:reparam_u_identity}
  Let \(\vt_{\vlambda}: \mathbb{R}^d \rightarrow \mathbb{R}^d\) be a location-scale reparameterizaiton function (\cref{def:reparam}).
  Also, let \(\vz \in \mathbb{R}^d\) be some vector and \(\rvvu \sim \varphi\) satisfy~\cref{assumption:symmetric_standard}.
  Then,
  \begin{alignat*}{2}
    \mathbb{E} \norm{\vt_{\vlambda}\left(\rvvu\right) - \vz}_2^2 \left(1 + \norm{\rvvu}_2^2\right)
    =
    \left(d+1\right) \norm{\vm - \vz}_2^2 + \left(d + \kappa\right) \norm{\mC}^2_{\mathrm{F}}.
  \end{alignat*}
\end{theoremEnd}

%%% Local Variables:
%%% TeX-master: "main"
%%% End:


Lastly, the following lemma is the basic building block for all of our upper bounds:


\begin{theoremEnd}[\keylemmaproofoption,category=upperboundkeylemmavariancegeneral]{lemma}\label{thm:gradient_variance_general_upper_bound}
  Let \(\vg_{M}\) be the \(M\)-sample gradient estimator of \(F\) (\cref{def:generic_elbo}) for some function \(f,h\) and let \(\rvvu\) be some random variable.
  Then, 
  {%
  \setlength{\belowdisplayskip}{1ex} \setlength{\belowdisplayshortskip}{1ex}%
  \setlength{\abovedisplayskip}{1ex} \setlength{\abovedisplayshortskip}{1ex}%
  \begin{align*}
    \mathbb{E} \norm{\vg_M }^2_2
    \leq
    \frac{1}{M} \mathbb{E}{ \norm{
      \nabla_{\vlambda} f\left(\vt_{\vlambda}\left(\rvvu\right)\right)
      }_2^2
    }
    + \norm{ \nabla F\left(\vlambda\right) }^2_2.
  \end{align*}
  }%
\end{theoremEnd}
\vspace{-1ex}
%% \begin{proofsketch}
%%   We use the affine property of the variance,%
%%   {%
%%   \setlength{\belowdisplayskip}{1ex} \setlength{\belowdisplayshortskip}{1ex}%
%%   \setlength{\abovedisplayskip}{1ex} \setlength{\abovedisplayshortskip}{1ex}%
%%   \begin{align*}
%%     \mathrm{tr}\,\V{\textstyle \frac{1}{M} \sum_{m=1}^M \vg_m }
%%     =
%%     \frac{1}{M} \mathrm{tr}\,\V{
%%       \nabla_{\vlambda} f\left(\vt_{\vlambda}\left(\rvvu\right)\right)
%%     }.
%%   \end{align*}
%%   }%
%%   This erases the effects of the deterministic elements of \(\vg\) (\textit{e.g.}, gradient of the regularization term), and simplifies the sample average.
%%   Also, this is the point where our proof can be further extended to data subsampling.
%% \end{proofsketch}
%\vspace{-2ex}
\begin{proofEnd}
  From the definition of variance,
  \begin{alignat}{2}
    &\mathbb{E} \norm{\vg_M }^2_2
    \nonumber
    \\
    &\;=
    \mathrm{tr}\,\V{ \vg_M } + \norm{\mathbb{E} \vg_M }^2_2,
    \nonumber
\shortintertext{following the definition in \cref{eq:def_gradient_M_est},}
    &\;=
    \mathrm{tr}\,\V{ \frac{1}{M} \sum_{m=1}^M \vg_m } + \norm{ \nabla F\left(\vlambda\right) }^2_2,
    \nonumber
\shortintertext{and then the definition in \cref{eq:def_gradient_m_est},}
    &\;=
    \mathrm{tr}\,\V{
      \frac{1}{M} \sum_{m=1}^M \nabla_{\vlambda} f\left(\vt_{\vlambda}\left(\rvvu_m\right)\right) + \nabla h\left(\vlambda\right)
    }
    + \norm{ \nabla F\left(\vlambda\right) }^2_2,
    \nonumber
\shortintertext{by the linearity of variance,}
    &\;=
    \frac{1}{M} \mathrm{tr}\,\V{
      \nabla_{\vlambda} f\left(\vt_{\vlambda}\left(\rvvu\right)\right)
    }
    + \norm{ \nabla F\left(\vlambda\right) }^2_2
    \nonumber
    \\
    &\;=
    \frac{1}{M} \left(
    \mathbb{E}{ \norm{\nabla_{\vlambda} f\left(\vt_{\vlambda}\left(\rvvu\right)\right)}_2^2 }
    -
    \norm{ \mathbb{E}{ \nabla_{\vlambda} f\left(\vt_{\vlambda}\left(\rvvu\right)\right)} }_2^2
    \right)
    \nonumber
    \\
    &\qquad+ \norm{ \nabla F\left(\vlambda\right) }^2_2
    \label{eq:thm_gradient_variance_general_definition}
    \\
    &\;\leq
    \frac{1}{M} \mathbb{E}{ \norm{
      \nabla_{\vlambda} f\left(\vt_{\vlambda}\left(\rvvu\right)\right)
      }_2^2
    }
    + \norm{ \nabla F\left(\vlambda\right) }^2_2.
    \nonumber
  \end{alignat}
  % The regularization term is neglected since it is a deterministic function.
  % This means the bound holds for both the entropy form and the KL form.
  % The last inequality introduces some looseness when \(\norm{\mathbb{E}{ \nabla f\left(\vt_{\vlambda}\left(\rvvu\right)\right) }}_2^2\) is large.
  % This means that the variance bound is loose initially but will become tighter as we progress.
\end{proofEnd}

%%% Local Variables:
%%% TeX-master: "main"
%%% End:



\begin{theoremEnd}[\lemmaproofoption, category=upperboundlemma]{lemma}\label{thm:reparam_quadratic}
  Let \(\vt_{\vlambda}: \mathbb{R}^d \rightarrow \mathbb{R}^d\) be a location-scale reparameterizaiton function (\cref{def:reparam}).
  Also, let \(\vz \in \mathbb{R}^d\) be some vector, and let \(\rvvu \sim \varphi\) satisfy~\cref{assumption:symmetric_standard}.
  Then,
  \begin{alignat*}{2}
    \mathbb{E}\norm{\vt_{\vlambda}\left(\rvvu\right) - \vz}_2^2
    &=
    \norm{\vm - \vz}^2_2 + \norm{\mC}_{\mathrm{F}}^2.
  \end{alignat*}
\end{theoremEnd}
\begin{proofEnd}
  \begin{alignat}{2}
    \mathbb{E}\norm{ \vt_{\vlambda}\left(\rvvu\right) - \vz }_2^2
    &=
    \mathbb{E}\norm{ \mC \rvvu + \vm - \vz }_2^2
    \nonumber
    \\
    &=
    \mathbb{E} \rvvu^{\top} \mC^{\top} \mC \rvvu + 2\,\mathbb{E} \rvvu^{\top} \mC^{\top} \vm - 2\,\mathbb{E} \rvvu^{\top} \mC^{\top} \vz
    \nonumber
    \\
    &\;\quad + \vm^{\top} \vm - 2\,\vm^{\top} \vz + \vz^{\top} \vz.
    \label{eq:reparam_quadratic_eq1}
  \end{alignat}
  The first three terms follow as
  \begin{alignat*}{2}
    &\mathbb{E} \rvvu^{\top} \mC^{\top} \mC \rvvu + 2\,\mathbb{E} \rvvu^{\top} \mC^{\top} \vm - 2\,\mathbb{E} \rvvu^{\top} \mC^{\top} \vz
    \\
    &\;=
    \mathbb{E} \mathrm{tr}\left(\rvvu^{\top} \mC^{\top} \mC \rvvu\right) + 2\,\mathbb{E} \rvvu^{\top} \mC^{\top} \vm -  2\,\mathbb{E} \rvvu^{\top} \mC^{\top} \vz
    \\
    &\;=
    \mathrm{tr}\left( \mC^{\top} \mC \mathbb{E} \rvvu \rvvu^{\top}\right) + 2\,\mathbb{E} \rvvu^{\top} \mC^{\top} \vm -  2\,\mathbb{E} \rvvu^{\top} \mC^{\top} \vz,
\shortintertext{applying \cref{thm:u_identities},}
    &\;=
    \mathrm{tr}\left( \mC^{\top} \mC \right)
    \\
    &\;=
    \norm{ \mC }_{\mathrm{F}}^2.
  \end{alignat*}
  Applying this to \cref{eq:reparam_quadratic_eq1},
  \begin{alignat*}{2}
    \mathbb{E}\norm{\vt_{\vlambda}\left(\rvvu\right) - \vz}_2^2
    &=
    \vm^{\top} \vm - 2\,\vm^{\top} \vz + \vz^{\top} \vz + \norm{\mC}_{\mathrm{F}}^2
    \\
    &=
    \norm{\vm - \vz}^2_2 + \norm{\mC}_{\mathrm{F}}^2.
  \end{alignat*}
\end{proofEnd}

% \begin{theoremEnd}[\lemmaproofoption, category=upperboundlemma]{lemma}\label{thm:tspace_distance}
%   Let \(\vt_{\vlambda}: \mathbb{R}^d \rightarrow \mathbb{R}^d\) be defined as in \cref{def:reparam} with parameters \(\vlambda = \left(\vm, \mC\right)\) such that \(\vm \in \mathbb{R}^d\) and \(\mC \in \mathbb{R}^{d \times d}\).
%   Also, let \(\rvvu \sim \varphi\) be some vector-valued random variable, where \(\varphi\) is defined as in \cref{assumption:symmetric_standard}.
%   Then,
%   \begin{alignat*}{2}
%     \mathbb{E}\norm{\vt_{\vlambda}\left(\rvvu\right) - \vt_{\vlambda^{\prime}}\left(\rvvu\right)}_2^2
%     &=
%     \norm{\vm - \vm^{\prime}}^2_2 + \norm{\mC - \mC^{\prime}}_{\mathrm{F}}^2.
%   \end{alignat*}
% \end{theoremEnd}
% \begin{proofEnd}
%   \begin{alignat}{2}
%     \mathbb{E}\norm{ \vt_{\vlambda}\left(\rvvu\right) - \vt_{\vlambda^{\prime}}\left(\rvvu\right) }_2^2
%     &=
%     \mathbb{E}\norm{ \left(\mC\rvvu + \vm\right) - \left( \mC^{\prime}\rvvu + \vm^{\prime} \right)  }_2^2
%     \nonumber
%     \\
%     &=
%     \mathbb{E}\norm{ \left(\mC - \mC^{\prime}\right)\rvvu + \left(\vm - \vm^{\prime}\right) }_2^2
%     \nonumber
%     \\
%     &=
%     \mathbb{E}\rvvu^{\top} {\left(\mC - \mC^{\prime}\right)}^{\top} \left(\mC - \mC^{\prime}\right)\rvvu
%     \nonumber
%     \\
%     &\;\quad+ 2 \mathbb{E} \rvvu^{\top} {\left(\mC - \mC^{\prime}\right)}^{\top} {\left(\vm - \vm^{\prime}\right)}
%     \nonumber
%     \\
%     &\;\quad+ {\left(\vm - \vm\right)}^{\top} \left(\vm - \vm^{\prime}\right),
%     \nonumber
% \shortintertext{invoking the trace trick,}
%     &=
%     \mathbb{E} \mathrm{tr}\left( \rvvu^{\top} {\left(\mC - \mC^{\prime}\right)}^{\top} \left(\mC - \mC^{\prime}\right)\rvvu \right)
%     \nonumber
%     \\
%     &\;\quad+ 2 \mathbb{E} \rvvu^{\top} {\left(\mC - \mC^{\prime}\right)}^{\top} {\left(\vm - \vm^{\prime}\right)}
%     \nonumber
%     \\
%     &\;\quad+ {\left(\vm - \vm^{\prime}\right)}^{\top} \left(\vm - \vm^{\prime}\right),
%     \nonumber
% \shortintertext{using the cyclic property of the trace,}
%     &=
%     \mathrm{tr}\left( {\left(\mC - \mC^{\prime}\right)}^{\top} \left(\mC - \mC^{\prime}\right) \mathbb{E} \rvvu \rvvu^{\top}  \right)
%     \nonumber
%     \\
%     &\;\quad+ 2 \mathbb{E} \rvvu^{\top} {\left(\mC - \mC^{\prime}\right)}^{\top} {\left(\vm - \vm\right)}
%     \nonumber
%     \\
%     &\;\quad+ {\left(\vm - \vm^{\prime}\right)}^{\top} \left(\vm - \vm^{\prime}\right),
%     \nonumber
% \shortintertext{and applying \cref{thm:u_identities},}
%     &=
%     \mathrm{tr}\left( {\left(\mC - \mC^{\prime}\right)}^{\top} \left(\mC - \mC^{\prime}\right) \right)
%     \nonumber
%     \\
%     &\;\quad+ {\left(\vm - \vm^{\prime}\right)}^{\top} \left(\vm - \vm^{\prime}\right)
%     \nonumber
%     \\
%     &=
%     \norm{ \mC - \mC^{\prime} }_{\mathrm{F}}^2
%     + \norm{ \vm - \vm^{\prime} }_2^2.
%     \nonumber
%   \end{alignat}
% \end{proofEnd}

%% \begin{theoremEnd}[\lemmaproofoption, category=upperboundlemma]{lemma}\label{thm:tspace_lambdaspace}
%%   If \(\phi : \mathbb{R} \mapsto \mathbb{R}_+\) is a \(1\)-Lipschitz continuous strictly positive function, 
%%   \begin{align*}
%%     \norm{ \vm - \vm^{\prime} }_{2}^2 + \norm{ \mC - \mC^{\prime} }_{\mathrm{F}}^2
%%     \leq
%%     \norm{ \vlambda - \vlambda^{\prime} }_2^2
%%   \end{align*}
%%   holds for both the mean-field (\cref{def:meanfield}) and Cholesky parameterizations (\cref{def:fullrank}),
%% \end{theoremEnd}
%% \begin{proofEnd}
%%   The first term satisfies \(\norm{ \vm - \vm^{\prime} }_{2}^2 = \norm{\vlambda_{\vm} - \vlambda_{\vm^{\prime}}}_2^2\)  by definition.
%%   Therefore, we only need to prove the inequality for the scale term.

%%   First, for the Cholesky (\cref{def:fullrank}),
%%   \begin{align*}
%%     \norm{ \mC - \mC^{\prime} }_{\mathrm{F}}^2
%%     &=
%%     \underbrace{
%%       \norm{ \phi\left(\vs\right) - \phi\left(\vs^{\prime}\right) }_{\mathrm{F}}^2
%%     }_{\text{diagonal of \(\mC,\mC^{\prime}\)}}
%%     +
%%     \underbrace{
%%       \norm{ \mL - \mL^{\prime} }_{\mathrm{F}}^2
%%     }_{\text{off-diagonal of \(\mC,\mC^{\prime}\)}}
%%     \\
%%     &=
%%     \sum^{d}_{i=1} \abs{ \phi\left(s_i\right) - \phi\left(s_i^{\prime}\right) }^2
%%     +
%%     \norm{ \mL - \mL^{\prime} }_{\mathrm{F}}^2,
%% \shortintertext{and by applying 1-Lipschitzness of \(\phi\),}
%%     &\leq
%%     \sum^{d}_{i=1} \abs{ s_i - s_i^{\prime} }^2
%%     +
%%     \norm{ \mL - \mL^{\prime} }_{\mathrm{F}}^2
%%     \\
%%     &=
%%     \norm{ \vs - \vs^{\prime} }_2^2
%%     +
%%     \norm{ \mL - \mL^{\prime} }_{\mathrm{F}}^2.
%%   \end{align*}
%%   Finally,
%%   \begin{align*}
%%     &\norm{ \vm - \vm^{\prime} }_{2}^2 + \norm{ \mC - \mC^{\prime} }_{\mathrm{F}}^2
%%     \\
%%     &\;\leq
%%     \norm{ \vm - \vm^{\prime} }_{2}^2
%%     +
%%     \norm{ \vs - \vs^{\prime} }_2^2
%%     +
%%     \norm{ \mL - \mL^{\prime} }_{\mathrm{F}}^2.
%%     \\
%%     &\;=
%%     \norm{ \vlambda - \vlambda^{\prime} }_2^2.
%%   \end{align*}
%%   The same proof holds for the mean-field case (\cref{def:meanfield}) by neglecting the off-diagonal (\(\mL\)) terms.
%% \end{proofEnd}

%% \begin{theoremEnd}[\keylemmaproofoption, category=upperboundkeylemma]{lemma}\label{thm:reparam_quadratic}
%%   \begin{alignat*}{2}
%%     \mathbb{E}\norm{\nabla f\left(\vt_{\vlambda}\left(\rvvu\right)\right)}_2^2 \left(1 + \norm{\rvvu}_2^2 \right)
%%     \leq
%%     \frac{2\,L^2 }{\mu} \left(d + \kappa\right) \left(  \mathbb{E}f\left(\vt_{\vlambda}\left(\rvvu\right)\right) - f^* \right)
%%   \end{alignat*}
%% \end{theoremEnd}
%% \begin{proofEnd}
%%   \begin{alignat}{2}
%%     &\mathbb{E}\norm{\nabla f\left(\vt_{\vlambda}\left(\rvvu\right)\right)}_2^2 \left(1 + \norm{\rvvu}_2^2 \right)
%%     \nonumber
%%     \\
%%     &\;=
%%     L^2\left(\left(d + 1\right) \norm{\vm - \bar{\vz}}_2^2 + \left(d + \kappa\right) \norm{\mC}_{\mathrm{F}}^2 \right)
%%     \nonumber
%% \shortintertext{since the kurtosis satisfies \(\kappa \geq 1\),}
%%     &\;\leq
%%     L^2\left(d + \kappa\right) \left( \norm{\vm - \bar{\vz}}_2^2 + \norm{\mC}_{\mathrm{F}}^2 \right),
%%     \nonumber
%% \shortintertext{by \cref{thm:reparam_quadratic},}
%%     &\;=
%%     L^2 \left(d + \kappa\right) \mathbb{E} \norm{\vt_{\vlambda}\left(\rvvu\right) - \bar{\vz} }_2^2,
%%     \nonumber
%% \shortintertext{assuming \(f\) satisfies \cref{assumption:quadratic_growth},}
%%     &\;\leq
%%     \frac{2\,L^2 }{\mu} \left(d + \kappa\right) \left(  \mathbb{E}f\left(\vt_{\vlambda}\left(\rvvu\right)\right) - f^* \right).
%%     \nonumber
%%   \end{alignat}
%% \end{proofEnd}

%%% Local Variables:
%%% TeX-master: "main"
%%% End:


%^\vspace{-1ex}
\subsection{Upper Bounds}\label{section:upper_bound}
\vspace{-.5ex}

We restrict our analysis to the class of log-likelihoods that satisfy the following conditions: 
\vspace{.5ex}
\begin{definition}[\textbf{\(L\)-smoothness}]\label{def:L_smoothness}
  A function \(f : \mathbb{R}^d \rightarrow \mathbb{R}\) is \(L\)-smooth if it satisfies the following for all \(\vzeta, \vzeta^{\prime} \in \mathbb{R}^d\):
  {\small%
  \setlength{\belowdisplayskip}{1.ex} \setlength{\belowdisplayshortskip}{1.ex}%
  \setlength{\abovedisplayskip}{1.ex} \setlength{\abovedisplayshortskip}{1.ex}%
  \begin{align*}
    {\lVert \nabla f\left(\vzeta\right) - \nabla f\left(\vzeta^{\prime}\right) \rVert}_2
    \leq
    L \, {\lVert \vzeta - \vzeta^{\prime} \rVert}_2.
  \end{align*}
  }%
\end{definition}
%

\vspace{.5ex}
\begin{definition}[\textbf{Quadratic Functional Growth}]\label{def:quadratic_growth}
  A function \(f : \mathbb{R}^d \rightarrow \mathbb{R}\) is \(\mu\)-quadratically growing if
  {\small%
  \setlength{\belowdisplayskip}{1.ex} \setlength{\belowdisplayshortskip}{1.ex}%
  \setlength{\abovedisplayskip}{1.ex} \setlength{\abovedisplayshortskip}{1.ex}%
  \begin{align*}
   \frac{\mu}{2} {\lVert \vzeta - \bar{\vzeta} \rVert}_2^2
    \leq
    f\left(\vzeta\right) - f^*
  \end{align*}
  }%
  for all \(\vzeta \in \mathbb{R}^d\), where \(\bar{\vzeta} \in \mathbb{R}^d\) is an arbitrary stationary point of \(f\) and \(f^* = \inf_{\vzeta \in \mathbb{R}^d} f\left(\vzeta\right)\).
\end{definition}
\vspace{-1ex}
%
For instance, \(\mu\)-strongly (quasar) convex functions~\citep{hinder_nearoptimal_2020,jin_convergence_2020} satisfy \cref{def:quadratic_growth}, but our analysis does \textit{not} require (quasar) convexity.

Both assumptions are commonly used in SGD.
For studying the gradient variance of BBVI, assuming both smoothness and quadratic growth is weaker than the assumptions of \citet{xu_variance_2019} but stronger than those of~\citet{domke_provable_2019}, who assumed only smoothness.
The additional assumption on growth is necessary to extend his results to establish the \textit{ABC} condition.

For the variational family, we assume the followings:
\vspace{.5ex}
\begin{assumption}\label{assumption:q}
\(q_{\psi,\vlambda}\) is a member of the ADVI family (\cref{def:advi}), where the underlying  \(q_{\vlambda}\) is a member of the location-scale family (\cref{def:family}) with its base distribution \(\varphi\) satisfying \cref{assumption:symmetric_standard}.
\end{assumption}

%% \todo[inline]{
%%   Obtaining a similar upper bound with the condition below is still a major goal.
%%   But we'll probably be able to publish this even if we don't succeed in H\"older-generalizing.
%% }
%% \begin{assumption*}{\textbf{(H\"older-smoothness)}}\label{assumption:holder_smoothness}
%%   A function \(f\) is \((L, \alpha)\)-H\"older-smooth if it satisfies
%%   \begin{align*}
%%     \norm{ \nabla f\left(\vz\right) - \nabla f\left(\vz^{\prime}\right) }_2
%%     \leq
%%     L \, \norm{\vz - \vz^{\prime}}_2^{\alpha}.
%%   \end{align*}
%% \end{assumption*}

%% \begin{assumption*}{\textbf{(H\"older-growth)}}\label{assumption:holder_growth}
%%   A function \(f\) satisfies \((L, \alpha)\)-H\"older-growth if it satisfies
%%   \begin{align*}
%%     \norm{ \vz - \vz^* }_2^{2 \alpha}
%%     \leq
%%     f\left(\vz\right),
%%   \end{align*}
%%   or
%%   \begin{align*}
%%     \norm{ \vz - \vz^* }_2^{1 + \alpha}
%%     \leq
%%     f\left(\vz\right),
%%   \end{align*}
%%   where \(\vz^*\) is a stationary point of \(f\) such that \(\nabla f\left(\vz^*\right) = \mathbf{0}\).
%% \end{assumption*}

\paragraph{Entropy-Regularized Form}
First, we provide the upper bound for the ELBO in entropy-regularized form.
This result does \textit{not} require any modifications to vanilla SGD.

\vspace{.5ex}

\begin{theoremEnd}[\theoremproofoption,category=upperboundtheorem]{theorem}\label{thm:gradient_upper_bound}
  Let \(\rvvg_{M}\) be an \(M\)-sample estimate of the gradient of the ELBO in entropy regularized form (\cref{def:entropy_form}).
  Also, assume that \cref{assumption:q,assumption:phi_lipschitz} hold,
%
  \begin{itemize}[leftmargin=3em]
    \vspace{-1.5ex}
    \setlength\itemsep{0ex}
    \item \(f_{\mathrm{H}}\) is \(L_{\mathrm{H}}\)-smooth, and
    \item \(f_{\mathrm{KL}}\) is \(\mu_{\mathrm{KL}}\)-quadratically growing.
    \vspace{-1.5ex}
  \end{itemize}
  %
  Then, 
  {\small%
  \setlength{\belowdisplayskip}{1ex} \setlength{\belowdisplayshortskip}{1ex}%
  \setlength{\abovedisplayskip}{1ex} \setlength{\abovedisplayshortskip}{1ex}%
  \begin{align*}
    \hspace{-1.5em}
    \mathbb{E}\norm{\rvvg_{M}}_2^2
    &\leq
    \frac{4 L^2_{\mathrm{H}}}{\mu_{\mathrm{KL}} M} C\left(d, \kappa\right) \left( F\left(\vlambda\right) - F^* \right)
    + \norm{ \nabla F\left(\vlambda\right) }_2^2
    \\
    &\quad+ \frac{2 L^2_{\mathrm{H}}}{M} C\left(d, \kappa\right) {\lVert \bar{\vzeta}_{\mathrm{KL}} - \bar{\vzeta}_{\mathrm{H}} \rVert}_2^2
    \\
    &\quad+ \frac{4 L^2_{\mathrm{H}}}{\mu_{\mathrm{KL}} M} C\left(d, \kappa\right) \left( F^* - f_{\mathrm{KL}}^* \right),
  \end{align*}
  }%
  where
  {\small%
  \setlength{\belowdisplayskip}{1ex} \setlength{\belowdisplayshortskip}{1ex}%
  \setlength{\abovedisplayskip}{1ex} \setlength{\abovedisplayshortskip}{1ex}%
  \begin{alignat*}{2}
    C\left(d, \kappa\right) &= 2 \kappa \sqrt{d} + 1 &&\;\text{for mean-field,} \\
    C\left(d, \kappa\right) &= d + \kappa          &&\;\text{for the Cholesky and matrix square root,}
  \end{alignat*}
  }
  \(\bar{\zeta}_{\mathrm{KL}}\), \(\bar{\zeta}_{\mathrm{H}}\) are the stationary points of \(f_{\mathrm{KL}}\), \(f_{\mathrm{H}}\), respectively,
  \(F^* = \inf_{\vlambda \in \mathbb{R}^p} F\left(\vlambda\right)\), and \(f_{\mathrm{KL}}^* = \inf_{\vzeta \in \mathbb{R}^d} f\left(\zeta\right)\).
\end{theoremEnd}
\vspace{-1ex}
\begin{proofsketch}
  From \cref{thm:gradient_variance_general_upper_bound}, we can see that the key quantity of upper bounding the gradient variance is to analyze \(\mathbb{E} \norm{ \nabla_{\vlambda} f_{\mathrm{H}} \left( \vt_{\vlambda}\left(\rvvu\right) \right) } \).
  The bird's eye view of the proof is as follows:
  \begin{enumerate}
  \vspace{-1ex}
    \setlength\itemsep{-.5ex}
    \item[\ding{182}] The relationship between \( \norm{ \nabla_{\vlambda} f_{\mathrm{H}} \left( \vt_{\vlambda}\left(\rvvu\right) \right) }_2^2 \) and \( \norm{ \nabla f_{\mathrm{H}} \left( \vt_{\vlambda}\left(\rvvu\right) \right) }_2^2 \) is established through \cref{thm:general_variational_gradient_norm_bound}.
    \item[\ding{183}] Then, the \(L_{\mathrm{H}}\)-smoothness of \(f_{\mathrm{H}}\) relates \( \norm{ \nabla f_{\mathrm{H}} \left( \vt_{\vlambda}\left(\rvvu\right) \right) }_2^2 \) with \( {\lVert \vt_{\vlambda}\left(\rvvu\right) - \bar{\vzeta}_{\mathrm{H}} \rVert}_2^2\), the average squared distance from \(f_{\mathrm{H}}\)'s stationary point.
    \item[\ding{184}] The average squared distance enables the simplification of stochastic terms through \cref{thm:meanfield_u_identity,thm:reparam_u_identity}. This step also introduces dimension dependence.
  \vspace{-1ex}
  \end{enumerate}
  From here, we are now left with the \(\mathbb{E} {\lVert \vt_{\vlambda}\left(\rvvu\right) - \bar{\vzeta}_{\mathrm{H}} \rVert}_2^2\) term.
  One might be tempted to assume the quadratic growth assumption on \(f_{\mathrm{H}}\) and proceed as
  {%
  \setlength{\belowdisplayskip}{1.ex} \setlength{\belowdisplayshortskip}{1.ex}%
  \setlength{\abovedisplayskip}{1.ex} \setlength{\abovedisplayshortskip}{1.ex}%
  \begin{align*}
    \mathbb{E} {\Vert \vt_{\vlambda}\left(\rvvu\right) - \bar{\vzeta}_{\mathrm{H}} \rVert}_2^2
    \leq \frac{2}{\mu} \left( f_{\mathrm{H}}\left(\vt_{\vlambda}\left(\rvvu\right)\right) - f^*_{\mathrm{H}}\right).
  \end{align*}
  }%
  However, for the entropy-regularized form, this soon runs into a dead end since in
  {%
  \setlength{\belowdisplayskip}{1ex} \setlength{\belowdisplayshortskip}{1ex}%
  \setlength{\abovedisplayskip}{1ex} \setlength{\abovedisplayshortskip}{1ex}%
  \begin{align*}
    \mathbb{E} f_{\mathrm{H}}\left(\vt_{\vlambda}\left(\rvvu\right)\right) - f^*_{\mathrm{H}}
    &= F\left(\vlambda\right) - h\left(\vlambda\right) - f^* \\
    &= \left( F\left(\vlambda\right) - F^* \right) + \left(F^* - f^*\right) - h_{\mathrm{H}}\left(\vlambda\right),
  \end{align*}
  }%
  the negative entropy term \(h_{\mathrm{H}}\) is not bounded unless we rely on assumptions that need modifications to the BBVI algorithms. (\textit{e.g.}, bounded support, bounded domain).
  Fortunately, the following inequality cleverly side-steps this problem:
  {%
  \setlength{\belowdisplayskip}{1.5ex} \setlength{\belowdisplayshortskip}{1.5ex}%
  \setlength{\abovedisplayskip}{1.5ex} \setlength{\abovedisplayshortskip}{1.5ex}%
  \begin{align}
    \hspace{-1.0em}
    \mathbb{E} {\Vert \vt_{\vlambda}\left(\rvvu\right) - \bar{\vzeta}_{\mathrm{H}} \rVert}_2^2
    &\leq
    2\,\mathbb{E} {\lVert \vt_{\vlambda}\left(\rvvu\right) - \bar{\vzeta}_{\text{KL}} \rVert}_2^2
    +
    2 \, {\lVert \bar{\vzeta}_{\text{KL}} - \bar{\vzeta}_{\mathrm{H}} \rVert}_2^2,
    \label{eq:thm_upper_bound_parallel}
  \end{align}
  }%
  albeit at the cost of some looseness.
  By converting the entropy-regularized form into the KL-regularized form, the regularizer term becomes \(h_{\mathrm{KL}} = \DKL{q_{\vlambda}}{p} \geq 0\), which is bounded below by definition, unlike the entropic-regularizer \(h_{\mathrm{H}}\). 
  The proof completes by 
  \begin{enumerate}
  \vspace{-1ex}
    \setlength\itemsep{-.5ex}
    \item[\ding{185}] applying the quadratic growth assumption to relate the parameter distance with the function suboptimality gap, and 
    \item[\ding{186}] upper bounding the KL regularizer term.
  \end{enumerate}
  \vspace{-4ex}
\end{proofsketch}
\vspace{-2ex}
\begin{proofEnd}
  The proof uses the \(L_{\mathrm{H}}\)-smoothness of \(f_{\mathrm{H}}\) such that 
  \begin{align}
    \mathbb{E} \norm{
    \nabla f_{\mathrm{H}}\left(\vt_{\vlambda}\left(\rvvu\right)\right)
    }_2^2
    &=
    \mathbb{E} {\lVert
    \nabla f_{\mathrm{H}}\left(\vt_{\vlambda}\left(\rvvu\right)\right)
    -
    \nabla f_{\mathrm{H}}\left(\bar{\vzeta}_{\mathrm{H}}\right)
    \rVert}_2^2
    \nonumber
    \\
    &\leq
    L^2_{\mathrm{H}}
    \mathbb{E} {\lVert
    \vt_{\vlambda}\left(\rvvu\right)
    -
    \bar{\vzeta}_{\mathrm{H}}
    \rVert}_2^2,
    \label{eq:thm_upper_bound_smoothness}
  \end{align}
  where \(\bar{\vzeta}_{\mathrm{H}}\) is a stationary point of \(f_{\mathrm{H}}\) such that \(\nabla f_{\mathrm{H}}\left(\bar{\vzeta}_{\mathrm{H}}\right) = \mathbf{0}\).
  These steps have been previously used by \citet[Theorem 3]{domke_provable_2019} to prove the special case for the matrix square root parameterization.

  For the mean-field parameterization, we start from \cref{thm:general_variational_gradient_norm_bound} and apply~\cref{eq:thm_upper_bound_smoothness} as
  \begin{align}
    &\mathbb{E} \norm{
      \nabla_{\vlambda} f_{\mathrm{H}}\left(\vt_{\vlambda}\left(\rvvu\right)\right)
    }_2^2 
    \nonumber
    \\
    &\;\leq
    \mathbb{E} \norm{
      \nabla f_{\mathrm{H}}\left(\vt_{\vlambda}\left(\rvvu\right)\right)
    }_2^2 \left(1 + \norm{\mathbfsfit{U}}_{\mathrm{F}}\right)
    \nonumber
    \\
    &\;\leq
    L^2_{\mathrm{H}} \,
    \mathbb{E} {\lVert
    \vt_{\vlambda}\left(\rvvu\right)
    +
    \bar{\vzeta}_{\mathrm{H}}
    \rVert}_2^2 \left(1 + \norm{\mathbfsfit{U}}_{\mathrm{F}}\right),
    \nonumber
\shortintertext{applying \cref{thm:meanfield_u_identity},}
    &\;\leq
    L^2_{\mathrm{H}}
    \left( \kappa \sqrt{d} + \sqrt{\kappa d} + 1 \right) {\lVert \vm - \bar{\vzeta}_{\mathrm{H}} \rVert}_2^2
    \nonumber
    \\
    &\;\;\quad+
    L^2_{\mathrm{H}} \left(2 \kappa \sqrt{d} + 1\right) \norm{\mC}_{\mathrm{F}}^2,
    \nonumber
\shortintertext{and since the kurtosis satisfies \(\kappa \geq 1\) and thus \(\kappa \geq \sqrt{\kappa}\),}
    &\;\leq
    L^2_{\mathrm{H}}
    \left( 2 \kappa \sqrt{d} + 1 \right) \left( {\lVert \vm - \bar{\vzeta}_{\mathrm{H}} \rVert}_2^2
    + \norm{\mC}_{\mathrm{F}}^2 \right).
    \label{eq:thm1_meanfield}
  \end{align}

  Similarly, for the full-rank parameterizations, we start from \cref{thm:general_variational_gradient_norm_bound} and apply~\cref{eq:thm_upper_bound_smoothness} as
  \begin{alignat}{2}
    &\mathbb{E} \norm{
      \nabla_{\vlambda} f_{\mathrm{H}}\left(\vt_{\vlambda}\left(\rvvu\right)\right)
    }_2^2 
    \\
    &\;\leq\mathbb{E}\norm{\nabla f_{\mathrm{H}}\left(\vt_{\vlambda}\left(\rvvu\right)\right)}_2^2 \left(1 + \norm{\rvvu}_2^2 \right),
    \nonumber
    \\
    &\;\leq
    L^2_{\mathrm{H}}\,\mathbb{E}{\lVert \vt_{\vlambda}\left(\rvvu\right) - \bar{\vzeta}_{\mathrm{H}} \rVert}_2^2 \left(1 + \norm{\rvvu}_2^2 \right),
    \nonumber
\shortintertext{applying \cref{thm:reparam_u_identity},}
    &\;=
    L^2_{\mathrm{H}}\left(\left(d + 1\right) {\lVert \vm - \bar{\vzeta}_{\mathrm{H}} \rVert}_2^2 + \left(d + \kappa\right) \norm{\mC}_{\mathrm{F}}^2 \right),
    \nonumber
\shortintertext{and since the kurtosis satisfies \(\kappa \geq 1\),}
    &\;\leq
    L^2_{\mathrm{H}}\left(d + \kappa\right) \left( {\lVert \vm - \bar{\vzeta}_{\mathrm{H}} \rVert}_2^2 + \norm{\mC}_{\mathrm{F}}^2 \right).
    \label{eq:thm1_fullrank}
  \end{alignat}
  
  Both \cref{eq:thm1_meanfield,eq:thm1_fullrank} can now be denoted as
  \begin{alignat}{2}
    \mathbb{E}\norm{\nabla_{\vlambda} f_{\mathrm{H}}\left(\vt_{\vlambda}\left(\rvvu\right)\right)}_2^2 
    &\leq
    L^2_{\mathrm{H}}\, C\left(d, \kappa\right)  \left( {\lVert \vm - \bar{\vzeta}_{\mathrm{H}} \rVert}_2^2 + \norm{\mC}_{\mathrm{F}}^2 \right),
    \nonumber
\shortintertext{where by \cref{thm:reparam_quadratic},}
    &=
    L^2_{\mathrm{H}}\, C\left(d, \kappa\right) \mathbb{E} {\Vert \vt_{\vlambda}\left(\rvvu\right) - \bar{\vzeta}_{\mathrm{H}} \rVert}_2^2,
    \label{eq:thm1_parameter_suboptimality_unified}
  \end{alignat}
  and the constants are \(C\left(d, \kappa\right) = \kappa \sqrt{d} + 1\) for mean-field and \(C\left(d, \kappa\right) = d + \kappa\) for the full-rank parameterizations.

  As mentioned in the sketch, it is necessary to convert the entropy-regularized form into the KL-regularized form through the following inequality:
  {%
  \setlength{\belowdisplayskip}{1ex} \setlength{\belowdisplayshortskip}{1ex}%
  \setlength{\abovedisplayskip}{1ex} \setlength{\abovedisplayshortskip}{1ex}%
  \begin{alignat*}{2}
    \hspace{-0.5em}
    \mathbb{E} {\Vert \vt_{\vlambda}\left(\rvvu\right) - \bar{\vzeta}_{\mathrm{H}} \rVert}_2^2
    &\leq
    2\,\mathbb{E} {\lVert \vt_{\vlambda}\left(\rvvu\right) - \bar{\vzeta}_{\text{KL}} \rVert}_2^2
    +
    2 \, {\lVert \bar{\vzeta}_{\text{KL}} - \bar{\vzeta}_{\mathrm{H}} \rVert}_2^2.
  \end{alignat*}
  }%
  where \(\bar{\vzeta}_{\mathrm{KL}} = \Pi_{f_{\mathrm{KL}}}\left(\bar{\vzeta}_{\mathrm{H}}\right)\) is a projection of \(\bar{\vzeta}_{\mathrm{H}}\) to the set of minimizers of \(f_{\mathrm{KL}}\).
  Note that the KL-regularized form does not need to be tractable; only its existence suffices.
  We can now apply the quadratic growth assumption as
  {%
  \setlength{\belowdisplayskip}{1ex} \setlength{\belowdisplayshortskip}{1ex}%
  \setlength{\abovedisplayskip}{1ex} \setlength{\abovedisplayshortskip}{1ex}%
  \begin{alignat}{2}
    \hspace{-1em}
    \mathbb{E} {\lVert \vt_{\vlambda}\left(\rvvu\right) - \bar{\vzeta}_{\text{KL}} \rVert}_2^2
    \nonumber
    &\leq
    \frac{2}{\mu_{\mathrm{KL}}} \left( \mathbb{E} f_{\text{KL}}\left(\vt_{\vlambda}\left(\rvvu\right)\right) - f^*_{\text{KL}} \right)
    \nonumber
    \\
    &=
    \frac{2}{\mu_{\mathrm{KL}}} 
    \left( F\left(\vlambda\right) - h_{\mathrm{KL}}\left(\vlambda\right) - f^*_{\text{KL}} \right),
    \nonumber
\shortintertext{and since \(-h_{\mathrm{KL}}\left(\vlambda\right) = -\DKL{q_{\vlambda}}{p} \leq 0\) by definition,}
    &\leq
    \frac{2}{\mu_{\mathrm{KL}}} 
    \left( F\left(\vlambda\right) - f^*_{\text{KL}} \right)
    \label{eq:thm_upper_bound_kl_upper_bound}
    \\
    &=
    \frac{2}{\mu_{\mathrm{KL}}} \left(
    \left( F\left(\vlambda\right) - F^* \right) 
    + \left( F^* - f^*_{\text{KL}} \right)
    \right).
    \label{eq:thm_upper_bound_f_quadratic}
  \end{alignat}
  }%
%
  Combining \cref{eq:thm1_parameter_suboptimality_unified} with \cref{eq:thm_upper_bound_parallel},
  {%\small
  \begin{align*}
    &\mathbb{E}\norm{\nabla_{\vlambda} f_{\mathrm{H}}\left(\vt_{\vlambda}\left(\rvvu\right)\right)}_2^2 
    \\
    &\;\leq
    2 \, L^2_{\mathrm{H}}\, C\left(d, \kappa\right) \mathbb{E} {\Vert \vt_{\vlambda}\left(\rvvu\right) - \bar{\vzeta}_{\mathrm{H}} \rVert}_2^2
    + 2 \, L^2_{\mathrm{H}}\, C\left(d, \kappa\right) {\lVert \bar{\vzeta}_{\text{KL}} - \bar{\vzeta}_{\mathrm{H}} \rVert}_2^2,
\shortintertext{and applying \cref{eq:thm_upper_bound_f_quadratic},}
    &\;\leq
    \frac{4 \, L^2_{\mathrm{H}}}{\mu_{\mathrm{KL}}} C\left(d, \kappa\right) \left(
    \left( F\left(\vlambda\right) - F^* \right) 
    +
    \left( F^* - f^*_{\text{KL}} \right)
    \right)
    \\
    &\;\quad+
    2 \, L^2_{\mathrm{H}}\, C\left(d, \kappa\right) {\lVert \bar{\vzeta}_{\text{KL}} - \bar{\vzeta}_{\mathrm{H}} \rVert}_2^2
  \end{align*}
  }%
  Plugging this into \cref{thm:gradient_variance_general_upper_bound} yields the result.
\end{proofEnd}

%%% Local Variables:
%%% TeX-master: "main"
%%% End:


\vspace{1ex}
\begin{remark}
  If the bijector \(\psi\) is an identity function, \(\vzeta_{\mathrm{KL}}\) and \(\vzeta_{\mathrm{H}}\) are the maximum likelihood (ML) and maximum a-posteriori (MAP) estimates, respectively.
  Thus, with enough datapoints, the term \( {\lVert \bar{\vzeta}_{\mathrm{KL}} - \bar{\vzeta}_{\mathrm{H}} \rVert}_2^2 \) will be negligible since the ML and MAP estimates will be close.
\end{remark}

\vspace{1ex}
\begin{remark}
  Let \(\kappa_{\mathrm{cond.}} = \nicefrac{L_{\mathrm{H}}}{\mu_{\mathrm{KL}}}\) be the \textit{condition number} of the problem.
  For the full-rank parameterizations and smooth quadratic functions, the variance is bounded as \(\mathcal{O}\left( L_{\mathrm{H}} \kappa_{\mathrm{cond.}} \left(d + \kappa\right) / M \right)\).
  The variance depends linearly on 
  \begin{enumerate}
    \vspace{-1ex}
    \setlength\itemsep{-1ex}
    \item[\ding{182}] the scaling of the problem \(L_{\mathrm{H}}\), 
    \item[\ding{183}] the conditioning of the problem \(\kappa_{\mathrm{cond.}}\),
    \item[\ding{184}] the dimensionality of the problem \(d\), and
    \item[\ding{185}] the tail properties of the variational family \(\kappa\),
    \vspace{-1ex}
  \end{enumerate}
  where the number of Monte Carlo samples \(M\) linearly reduces the variance.
\end{remark}

\vspace{-1.ex}
\paragraph{KL-Regularized Form}
We now prove an equivalent result for the KL-regularized form.
Here, we do not have to rely on~\cref{eq:thm_upper_bound_parallel} since we already start from \(f_{\mathrm{KL}}\), which results in better constants.


\begin{theoremEnd}[\theoremproofoption,category=upperboundtheoremklform]{theorem}\label{thm:gradient_upper_bound_kl}
  Let \(\rvvg_{M}\) be an \(M\)-sample estimator of the gradient of the ELBO in KL-regularized form (\cref{def:kl_form}). 
  Also, assume that
%
  \begin{itemize}[leftmargin=3em]
    \vspace{-1.5ex}
    \setlength\itemsep{0ex}
    \item \(f_{\mathrm{KL}}\) is \(L_{\mathrm{KL}}\)-smooth,
    \item \(f_{\mathrm{KL}}\) is \(\mu_{\mathrm{KL}}\)-quadratically growing,
    \vspace{-1.5ex}
  \end{itemize}
  %
  and~\cref{assumption:q,assumption:phi_lipschitz} hold.
  Then, the gradient variance is bounded above as
  {%
  \setlength{\belowdisplayskip}{1ex} \setlength{\belowdisplayshortskip}{1ex}%
  \setlength{\abovedisplayskip}{1ex} \setlength{\abovedisplayshortskip}{1ex}%
  \begin{align*}
    \mathbb{E}\norm{\rvvg_{M}}_2^2
    &\leq
    \frac{2 L^2_{\mathrm{KL}}}{\mu_{\mathrm{KL}} M} C\left(d, \kappa\right) \left( F\left(\vlambda\right) - F^* \right)
    + \norm{ \nabla F\left(\vlambda\right) }_2^2
    \qquad\qquad
    \\
    &\quad+ \frac{2 L^2_{\mathrm{KL}}}{\mu_{\mathrm{KL}} M} C\left(d, \kappa\right) \left( F^* - f_{\mathrm{KL}}^*\right),
  \end{align*}
  }%
  where
  {%
  \setlength{\belowdisplayskip}{1ex} \setlength{\belowdisplayshortskip}{1ex}%
  \setlength{\abovedisplayskip}{1ex} \setlength{\abovedisplayshortskip}{1ex}%
  \begin{alignat*}{2}
    C\left(d, \kappa\right) &= 2 \kappa \sqrt{d} + 1 &&\;\text{for mean-field,} \\
    C\left(d, \kappa\right) &= d + \kappa          &&\;\text{for the Cholesky and matrix square root,}
  \end{alignat*}
  }%
  \(F^* = \inf_{\vlambda \in \mathbb{R}^p} F\left(\vlambda\right)\), and \(f_{\mathrm{KL}}^* = \inf_{\vzeta \in \mathbb{R}^d} f\left(\zeta\right)\).
\end{theoremEnd}
\vspace{-1ex}
\begin{proofEnd}
  This proof uses the smoothness of \(f_{\mathrm{KL}}\) instead of \(f_{\mathrm{H}}\).
  That is,
  \begin{align}
    \mathbb{E} \norm{
      \nabla f_{\mathrm{KL}}\left(\vt_{\vlambda}\left(\rvvu\right)\right)
    }_2^2 
    &=
    \mathbb{E} {\lVert
      \nabla f_{\mathrm{KL}}\left(\vt_{\vlambda}\left(\rvvu\right)\right)
      -
      \nabla f_{\mathrm{KL}}\left( \bar{\vzeta}_{\mathrm{KL}} \right)
    \rVert}_2^2 
    \nonumber
\shortintertext{applying \cref{eq:thm_upper_bound_smoothness},}
    &\leq
    L^2_{\mathrm{KL}} \,
    \mathbb{E} {\lVert
    \vt_{\vlambda}\left(\rvvu\right)
    -
    \bar{\vzeta}_{\mathrm{KL}}
    \rVert}_2^2, \label{eq:thm3_kl_smoothness}
  \end{align}
  where \(\bar{\vzeta}_{\mathrm{KL}}\) is a stationary point of \(f_{\mathrm{KL}}\).

  Substituting \cref{eq:thm3_kl_smoothness} in \cref{eq:thm1_parameter_suboptimality_unified},
  \begin{alignat}{2}
    &\mathbb{E}\norm{\nabla_{\vlambda} f_{\mathrm{KL}}\left(\vt_{\vlambda}\left(\rvvu\right)\right)}_2^2 
    \nonumber
    \\
    &\;\leq
    L^2_{\mathrm{KL}} C\left(d, \kappa\right) \mathbb{E} {\Vert \vt_{\vlambda}\left(\rvvu\right) - \bar{\vzeta}_{\mathrm{KL}} \rVert}_2^2,
    \nonumber
\shortintertext{and by applying \cref{eq:thm_upper_bound_f_quadratic} for \(f_{\mathrm{KL}}\),}
    &\;=
    \frac{2 L^2_{\mathrm{KL}} }{\mu_{\mathrm{KL}}} C\left(d, \kappa\right) \left( F\left(\vlambda\right) - F^*  \right) 
    + \frac{2 L^2_{\mathrm{KL}} }{\mu_{\mathrm{KL}}} C\left(d, \kappa\right)
    \left(
      F^* - f^*_{\text{KL}}
    \right).
    \nonumber
  \end{alignat}
  Plugging this to \cref{thm:gradient_variance_general_upper_bound} proves the result.
\end{proofEnd}

%%% Local Variables:
%%% TeX-master: "main"
%%% End:


\vspace{-.5ex}
\subsection{Upper Bound Under Bounded Entropy}
\vspace{-.5ex}
The bound in \cref{thm:gradient_upper_bound} is loose due to the use of~\cref{eq:thm_upper_bound_parallel} (\(\times 2\) loose) and \cref{eq:thm_upper_bound_kl_upper_bound}.
An alternative bound can be obtained by assuming the following:
%
\begin{assumption}[\textbf{Bounded Entropy}]\label{assumption:bounded_entropy}
  The regularization term is bounded below as
  \( h_{\mathrm{H}}\left(\vlambda\right) \geq h_{\mathrm{H}}^* \).
\end{assumption}
%.
For the entropy-regularized form, this corresponds to the entropy being bounded above by some constant since \(h\left(\vlambda\right) = - \mathrm{H}\left(q_{\vlambda}\right)\).
When using the nonlinear parameterizations (\cref{def:meanfield,def:fullrank}), this assumption can be practically enforced by bounding the output of \(\phi\) by some large \(S\).
%
\vspace{.5ex}
\begin{proposition}
    Let the diagonal conditioner \(\phi\) be bounded as \(\phi\left(x\right) \leq S\).
    Then, for any \(d\)-dimensional distribution \(q_{\vlambda}\) in the location-scale family with the mean-field (\cref{def:meanfield}) or Cholesky (\cref{def:fullrank}) parameterizations,
  {%
  \setlength{\belowdisplayskip}{1.ex} \setlength{\belowdisplayshortskip}{1.ex}%
  \setlength{\abovedisplayskip}{1.ex} \setlength{\abovedisplayshortskip}{1.ex}%
    \[ h_{\mathrm{H}}\left(\vlambda\right) = -\mathrm{H}\left(q_{\vlambda}\right) \geq -\mathrm{H}\left(\varphi\right) - d \log S. \]
  }%
\end{proposition}
\vspace{-2ex}
\begin{proof}
    From \cref{thm:location_scale_entropy}, \(\mathrm{H}\left(q_{\vlambda}\right) = \mathrm{H}\left(\varphi\right) + \log \abs{\mC} \).
    Since \(\mC\) under \cref{def:meanfield,def:fullrank} is a diagonal or triangular matrix, the log absolute determinant is the log sum of the diagonals.
    The conclusion follows from the fact that the diagonals \(C_{ii} = \phi\left(s_i\right)\) are bounded by \(S\).
\end{proof}
%
This is essentially a weaker version of the bounded domain assumption, though only the diagonal elements of \(\mC\), \(s_1, \ldots, s_d\), are bounded.
While this assumption results in an admittedly less realistic algorithm, it enables a tighter bound for the entropy-regularized form ELBO.


\begin{theoremEnd}[\theoremproofoption,category=upperboundtheoremboundedentropy]{theorem}\label{thm:gradient_upper_bound_bounded_entropy}
  Let \(\rvvg_{M}\) be an \(M\)-sample estimator of the gradient of the ELBO in entropy-regularized form (\cref{def:entropy_form}). 
  Also, assume that
%
  \begin{itemize}[leftmargin=3em]
    \vspace{-1.5ex}
    \setlength\itemsep{0ex}
    \item \(f_{\mathrm{H}}\) is \(L_{\mathrm{H}}\)-smooth,
    \item \(f_{\mathrm{H}}\) is \(\mu_{\mathrm{H}}\)-quadratically growing,
    \item \(h_{\mathrm{H}}\) is bounded as \(h_{\mathrm{H}}\left(\vlambda\right) > h_{\mathrm{H}}^*\) (\cref{assumption:bounded_entropy}),
      \vspace{-1.5ex}
  \end{itemize}
  %
  and~\cref{assumption:q,assumption:phi_lipschitz} hold.
  Then, the gradient variance of \(\vg_{M}\) is bounded above as
  {%
  \setlength{\belowdisplayskip}{1.ex} \setlength{\belowdisplayshortskip}{1.ex}%
  \setlength{\abovedisplayskip}{1.ex} \setlength{\abovedisplayshortskip}{1.ex}%
  \begin{align*}
    \mathbb{E}\norm{\rvvg_{M}}_2^2
    &\leq
    \frac{2 L^2_{\mathrm{H}}}{\mu_{\mathrm{H}} M} C\left(d, \kappa\right) \left( F\left(\vlambda\right) - F^* \right)
    + \norm{ \nabla F\left(\vlambda\right) }_2^2
    \qquad\qquad
    \\
    &\quad+ \frac{2 L^2_{\mathrm{H}}}{\mu_{\mathrm{H}} M} C\left(d, \kappa\right) \left( F^* - f_{\mathrm{H}}^* - h^*_{\mathrm{H}} \right),
  \end{align*}
  }
  where
  {%
  \setlength{\belowdisplayskip}{1.ex} \setlength{\belowdisplayshortskip}{1.ex}%
  \setlength{\abovedisplayskip}{1.ex} \setlength{\abovedisplayshortskip}{1.ex}%
  \begin{alignat*}{2}
    C\left(d, \kappa\right) &= 2 \kappa \sqrt{d} + 1 &&\;\text{for mean-field,} \\
    C\left(d, \kappa\right) &= d + \kappa          &&\;\text{for the Cholesky parameterization,}
  \end{alignat*}
  }
  \(F^* = \inf_{\vlambda \in \mathbb{R}^p} F\left(\vlambda\right)\), and \(f_{\mathrm{H}}^* = \inf_{\vzeta \in \mathbb{R}^d} f\left(\zeta\right)\).
\end{theoremEnd}
\vspace{-2ex}
\begin{proofsketch}
   Instead of using \cref{eq:thm_upper_bound_parallel}, we apply the quadratic assumption directly to \(f_{\text{H}}\).
   The remaining entropic-regularizer term can now be bounded through the bounded entropy assumption.
\end{proofsketch}
\vspace{-2ex}
\begin{proofEnd}
  The proof is similar to that of \cref{thm:gradient_upper_bound}.
  As mentioned in the \textit{proof sketch}, we use the fact that the entropic regularizer is bounded such that
  \[ -h_{\mathrm{H}}\left(\vlambda\right) < -h_{\mathrm{H}}^*. \]
  By applying the quadratic growth assumption directly to \(f_{\mathrm{H}}\), 
  \begin{alignat}{2}
    \mathbb{E} {\lVert \vt_{\vlambda}\left(\rvvu\right) - \bar{\vzeta}_{\text{H}} \rVert}_2^2
    &\leq
    \frac{2}{\mu_{\mathrm{H}}} \left( \mathbb{E} f_{\text{H}}\left(\vt_{\vlambda}\left(\rvvu\right)\right) - f^*_{\text{H}} \right)
    \nonumber
    \\
    &=
    \frac{2}{\mu_{\mathrm{H}}} 
    \left( F\left(\vlambda\right) - h_{\mathrm{H}}\left(\vlambda\right) - f^*_{\text{H}} \right),
    \nonumber
\shortintertext{and by \cref{assumption:bounded_entropy},}
    &\leq
    \frac{2}{\mu_{\mathrm{H}}} 
    \left( F\left(\vlambda\right) - F^* \right) 
    + \frac{2}{\mu_{\mathrm{H}}} \left( F^* - f^*_{\text{H}} - h_{\mathrm{H}}^*\right).
    \label{eq:thm_upper_bound_f_H_quadratic}
  \end{alignat}
  
  The proof resumes from \cref{eq:thm1_parameter_suboptimality_unified} as
  {%
  \setlength{\belowdisplayskip}{1.ex} \setlength{\belowdisplayshortskip}{1.ex}%
  \setlength{\abovedisplayskip}{1.ex} \setlength{\abovedisplayshortskip}{1.ex}%
  \begin{alignat}{2}
    &\mathbb{E}\norm{\nabla_{\vlambda} f_{\mathrm{H}}\left(\vt_{\vlambda}\left(\rvvu\right)\right)}_2^2
    \nonumber
    \\
    &\;\leq
    L^2_{\mathrm{H}} C\left(d, \kappa\right) \mathbb{E} {\Vert \vt_{\vlambda}\left(\rvvu\right) - \bar{\vzeta}_{\mathrm{H}} \rVert}_2^2,
    \nonumber
\shortintertext{and by applying \cref{eq:thm_upper_bound_f_H_quadratic},}
    &\;=
    \frac{2 L^2_{\mathrm{H}} }{\mu_{\mathrm{H}}} C\left(d, \kappa\right) \left( F\left(\vlambda\right) - F^*  \right) 
    + \frac{2 L^2_{\mathrm{H}} }{\mu_{\mathrm{H}}} C\left(d, \kappa\right)
    \left(
      F^* - f^*_{\mathrm{H}} - h_{\mathrm{H}}^*
    \right).
    \nonumber
  \end{alignat}
  }%
  Plugging this to \cref{thm:gradient_variance_general_upper_bound} proves the result.
\end{proofEnd}

%%% Local Variables:
%%% TeX-master: "main"
%%% End:


\begin{figure*}[t]
  \vspace{-2ex}
  \centering
  
%\tikzexternalenable
{\hypersetup{linkbordercolor=black,linkcolor=black}
\begin{tikzpicture}
  \begin{groupplot}
    [
      group style={
        group size=4 by 1,
        % columns=2,
        % rows=2,
        %xlabels at=edge bottom,
        %ylabels at=edge left,
        horizontal sep=0.07\textwidth
      },
    ]
    
% Tikz examples & demos
% https://holatex.app/examples.html?package=tikz

  \nextgroupplot[
      legend style={
        legend image post style = {scale=0.5},
        legend columns          = -1,
        column sep              = 1em,
        %at                     ={(0.5,1.5)},
        anchor                  = north,
        legend cell align       = left,
        line width              = 0.8pt,
        draw                    = none,
        legend to name          = grouplegend
      },
      %
      tuftelike, 
      %
      xlabel style={yshift=-0.8ex},
      %
      axis line style = thick,
      every tick/.style={black,thick},
      %axis lines = left,
      %grids=both,
      ymode  = log,
      xmin   = 1,
      xmax   = 1000,
      xtick  = {1, 200, 400, 600, 800, 1000},
      ytick  = {1e+4, 1e+5, 1e+6, 1e+7, 1e+8},
      %
      ymin   = 1e+4,
      ymax   = 1e+8,
      %
      xlabel = {Iteration},
      %ylabel = {\(\mathbb{E}\norm{\rvvg}^2_2\)},
      tick label style={font=\small},  
      height = 4cm,
      width  = 4.7cm,
      axis on top,
    ]
    \addplot[name path=gvar, color1, mark=none, thick]
      table [x=t, y=gvar] {data/simulation/quadratic_softpluschol_generalbound.csv}
      %node[above=3pt,pos=0.8] {\scriptsize\(\mu_{\mathrm{KL}}, L_{\mathrm{H}}\) bound}
      coordinate [pos=0.95] (gvarcoord);
    \addlegendentry{\scriptsize Gradient Variance\;\; \( \mathbb{E}\norm{ \rvvg }_2^2 \)}

    \addplot[name path=ABC, color2, mark=none, thick]
      table [x=t, y=ABC] {data/simulation/quadratic_softpluschol_generalbound.csv}
      coordinate [pos=0.05] (ABCcoord1) coordinate [pos=0.95] (ABCcoord2);
    \addlegendentry{\scriptsize Upper Bound\;\; \(2 A \left(F\left(\vlambda\right) - F^*\right) + B \norm{\nabla F}_2^2 + C\)}

    %\addplot[color4, draw=none, mark=none, thick] {x};
    %\addlegendentry{\scriptsize\(2 A \left(\mathbb{E}_{q_{\vlambda}} f_{\mathrm{KL}} - f^*_{\mathrm{KL}} \right) + B \norm{\nabla F}_2^2 + C_{\Delta \zeta} {\lVert \bar{\zeta}_{\mathrm{KL}} - \bar{\zeta}_{\mathrm{H}} \rVert}^2_2 \)}

    \addplot[name path=opt, color3, mark=none, thick] %
      table [x=t, y=opt] {data/simulation/quadratic_softpluschol_generalbound.csv} %
      %node[above=3pt,pos=0.8] {\scriptsize\(\DKL{q_{\vlambda}}{p}\)}
      coordinate [pos=0.95] (optcoord);
    %\addlegendentry{\scriptsize\(2 A \left(\mathbb{E}_{q_{\vlambda}} f_{\mathrm{H}} - f^*_{\mathrm{H}} \right) + B \norm{\nabla F}_2^2\)}

    \addplot[name path=axis, domain=0:990, fill=none, no markers, draw=none] {1e+4}
      coordinate [pos=0.05] (axiscoord);

    %\addplot[name path=ABC, color2, mark=none]
    %  table [x=t, y=ABC] {data/simulation/quadratic_softpluschol_generalbound.csv}
    %  ;

    \addplot[name path=C, fill=none, gray, mark=none]
      table [x=t, y=C] {data/simulation/quadratic_softpluschol_generalbound.csv}
      coordinate [pos=0.05] (Ccoord);

    \addplot[thick, color=color3, fill=color3, fill opacity=0.2] fill between[of=ABC and opt];
    \addplot[thick, color=color2, fill=color2, fill opacity=0.2] fill between[of=ABC and   C];

    \addplot[thick, color=color1, fill=color1, fill opacity=0.2] fill between[of=opt and gvar];
    \addplot[thick, color=gray,   fill=gray,   fill opacity=0.2] fill between[of=C   and axis];

    \draw[thick,color=color3,{Latex[scale=0.5]}-{Latex[scale=0.5]}] (ABCcoord2) -- (optcoord)
      node[midway,xshift=-18pt] {\scriptsize\(\DKL{q_{\vlambda}}{p}\)};

    %\draw[thick,color=color1,latex-latex] (gvarcoord) -- (optcoord)
    %  %node[midway,xshift=-32pt] {\scriptsize\(\mu_{\mathrm{KL}}, L_{\mathrm{H}}, \Delta \vz^*\) bound}
    %  coordinate [pos=0.5] (muL_coord);

    \draw[thick,color=gray,latex-] (Ccoord) -- (axiscoord)
      node[midway,xshift=5pt] {\scriptsize\(C\) };

%%     \node[%
%%       pin={%
%%         [%
%%           pin distance=0.3cm,%
%%           pin edge={color1},%
%%           text=color1%
%%         ]265:{\scriptsize\(\mu_{\mathrm{KL}}, L_{\mathrm{H}}, \Delta \vz^*, \phi \) bound}},%
%%       inner sep=0pt,%
%%     ] at (muL_coord) {};

    %\draw[thick,color=color2,latex-latex] (ABCcoord1) -- (Ccoord)
    %  node[midway,xshift=25pt] {\scriptsize\(2 A \left( F\left(\lambda\right) - F^* \right)\) };

    \input{figures/group_quadratic_softpluschol_boundedentropy}
    \input{figures/group_quadratic_softplusmf_generalbound}
    
% Tikz examples & demos
% https://holatex.app/examples.html?package=tikz

  \nextgroupplot[
      %% legend style={
      %%   legend image post style={scale=0.5},
      %%   at={(0.5,1.5)},
      %%   anchor=north,
      %%   legend cell align=left,
      %%   line width=0.8pt,
      %%   %draw=none % Unterdrücke Box
      %% },
      %
      tuftelike, 
      axis line style = thick,
      every tick/.style={black,thick},
      %axis lines = left,
      %grids=both,
      ymode  = log,
      xmin   = 1,
      xmax   = 1000,
      xtick  = {1, 200, 400, 600, 800, 1000},
      ytick  = {1e+3, 1e+4, 1e+5, 1e+6, 1e+7, 1e+8},
      %
      xlabel style={yshift=-0.8ex},
      %
      ymin   = 1e+3,
      ymax   = 1e+8,
      %
      xlabel = {Iteration},
      %ylabel = {\(\mathbb{E}\norm{\rvvg}^2_2\)},
      tick label style={font=\small},  
      height = 4cm,
      width  = 4.7cm,
      axis on top,
    ]
    \addplot[name path=gvar, color1, mark=none, thick]
      table [x=t, y=gvar] {data/simulation/quadratic_softplusmf_boundedentropybound.csv}
      %node[above=3pt,pos=0.8] {\scriptsize\(\mu_{\mathrm{KL}}, L_{\mathrm{H}}\) bound}
      coordinate [pos=0.95] (gvarcoord);
    %\addlegendentry{\scriptsize\( \mathbb{E}\norm{ \rvvg }_2^2 \)}

    \addplot[name path=opt, color3, mark=none, thick] %
      table [x=t, y=opt] {data/simulation/quadratic_softplusmf_boundedentropybound.csv} %
      %node[above=3pt,pos=0.8] {\scriptsize\(\DKL{q_{\vlambda}}{p}\)}
      coordinate [pos=0.95] (optcoord);
    %\addlegendentry{\scriptsize\(2 A \left(\mathbb{E}_{q_{\vlambda}} f_{\mathrm{H}} - f^*_{\mathrm{H}}\right)\)}

    \addplot[name path=ABC, color2, mark=none, thick]
      table [x=t, y=ABC] {data/simulation/quadratic_softplusmf_boundedentropybound.csv}
      coordinate [pos=0.05] (ABCcoord1) coordinate [pos=0.95] (ABCcoord2);
    %\addlegendentry{\scriptsize\(2 A \left(F\left(\vlambda\right) - F^*\right) + B \norm{\nabla F}_2^2 + C\)}

    \addplot[name path=axis, domain=0:990, fill=none, no markers, draw=none] {1e+3}
      coordinate [pos=0.05] (axiscoord);

    %\addplot[name path=ABC, color2, mark=none]
    %  table [x=t, y=ABC] {data/simulation/quadratic_softpluschol_boundedentropybound.csv}
    %  ;

    \addplot[name path=C, fill=none, gray, mark=none]
      table [x=t, y=C] {data/simulation/quadratic_softplusmf_boundedentropybound.csv}
      coordinate [pos=0.05] (Ccoord);

    \addplot[thick, color=color3, fill=color3, fill opacity=0.2] fill between[of=ABC and opt];
    \addplot[thick, color=color2, fill=color2, fill opacity=0.2] fill between[of=ABC and   C];

    \addplot[thick, color=color1, fill=color1, fill opacity=0.2] fill between[of=opt and gvar];
    \addplot[thick, color=gray,   fill=gray,   fill opacity=0.2] fill between[of=C   and axis];

    \draw[thick,color=color3,{Latex[scale=0.5]}-{Latex[scale=0.5]}] (ABCcoord2) -- (optcoord)
      node[midway,xshift=-18pt] {\scriptsize\(h\left(\vlambda\right) - h^*\)}
      coordinate [pos=0.5] (dklarrow_coord);

    %\draw[thick,color=color1,{Latex[scale=0.5]}-{Latex[scale=0.5]}] (gvarcoord) -- (optcoord)
      %node[midway,xshift=-25pt] {\scriptsize\(\mu_{\mathrm{KL}}, L_{\mathrm{H}}\) bound}
    %  coordinate [pos=0.5] (muL_coord);

    \draw[thick,color=gray,latex-] (Ccoord) -- (axiscoord)
      node[midway,xshift=5pt] {\scriptsize\(C\) };

    %% \node[%
    %%   pin={%
    %%     [%
    %%       pin distance=1cm,%
    %%       pin edge={color3},%
    %%       text=color3%
    %%     ]100:{\scriptsize\( h\left(\vlambda\right) - h^* \)}%
    %%   },%
    %%   inner sep=0pt,
    %% ] at (dklarrow_coord) {};

    %% \node[%
    %%   pin={%
    %%     [%
    %%       pin distance=1cm,%
    %%       pin edge={color1},%
    %%       text=color1%
    %%     ]130:{\scriptsize\(\mu_{\mathrm{H}}, L_{\mathrm{H}}\) bound}},%
    %%   inner sep=0pt,%
    %% ] at (muL_coord) {};


    %\draw[thick,color=color2,latex-latex] (ABCcoord1) -- (Ccoord)
    %  node[midway,xshift=25pt] {\scriptsize\(2 A \left( F\left(\lambda\right) - F^* \right)\) };

    %\input{figures/group_quadratic_softplusmf_generalbound}
  \end{groupplot}
  
  \coordinate (c1r1southwest) at ($(group c1r1) + (-1.7cm, -2.5cm)$); 
  \coordinate (c1r1southeast) at ($(group c1r1) + (1.7cm,  -2.5cm)$); 

  \coordinate (c2r1southwest) at ($(group c2r1) + (-1.7cm, -2.5cm)$);
  \coordinate (c2r1southeast) at ($(group c2r1) + (1.7cm,  -2.5cm)$);

  \coordinate (c3r1southwest) at ($(group c3r1) + (-1.7cm, -2.5cm)$); 
  \coordinate (c3r1southeast) at ($(group c3r1) + ( 1.7cm, -2.5cm)$); 

  \coordinate (c4r1southwest) at ($(group c4r1) + (-1.7cm, -2.5cm)$);
  \coordinate (c4r1southeast) at ($(group c4r1) + ( 1.7cm, -2.5cm)$);
  
  \draw[thick,color=black] (c1r1southwest) -- (c2r1southeast) node[midway,below] {\small Cholesky\;\; \(\phi\left(x\right) = \mathrm{softplus}\left(x\right)\)};

  \draw[thick,color=black] (c3r1southwest) -- (c4r1southeast) node[midway,below] {\small Mean-Field\;\; \(\phi\left(x\right) = \mathrm{softplus}\left(x\right)\)};

  \draw[thick,color=black] ($(c1r1southwest) + (0,-0.7cm)$) -- ($(c1r1southeast) + (0,-0.7cm)$) node[midway,below] {\small\cref{thm:gradient_upper_bound}};
  \draw[thick,color=black] ($(c2r1southwest) + (0,-0.7cm)$) -- ($(c2r1southeast) + (0,-0.7cm)$) node[midway,below] {\small\cref{thm:gradient_upper_bound_bounded_entropy}};
  \draw[thick,color=black] ($(c3r1southwest) + (0,-0.7cm)$) -- ($(c3r1southeast) + (0,-0.7cm)$) node[midway,below] {\small\cref{thm:gradient_upper_bound}};
  \draw[thick,color=black] ($(c4r1southwest) + (0,-0.7cm)$) -- ($(c4r1southeast) + (0,-0.7cm)$) node[midway,below] {\small\cref{thm:gradient_upper_bound_bounded_entropy}};


  \node at ($(group c2r1) + (2.25cm,1.75cm)$) {\ref*{grouplegend}}; 
\end{tikzpicture}
}
%\tikzexternaldisable

  \vspace{-4ex}
  \caption{
    \textbf{Evaluation of the bounds for a perfectly conditioned quadratic target function.}
    The \textcolor{color3}{blue regions} are the loosenesses resulting from either using (\cref{thm:gradient_upper_bound}) or not using (\cref{thm:gradient_upper_bound_bounded_entropy}) the bounded entropy assumption (\cref{assumption:bounded_entropy}), while the \textcolor{color1}{red regions} are the remaining ``technical loosesnesses.''
    The gradient variance was estimated from \(10^3\) samples.
  }\label{fig:quadratic}
  \vspace{-2ex}
\end{figure*}


%% \subsection{Upper Bound of the STL Estimator}

%% Unlike the previously considered estimators, the \textit{sticking the landing} (STL; \citealt{roeder_sticking_2017}) estimator uses Monte Carlo estimates of the entropy term.
%% %
%% \begin{definition}[\textbf{STL Estimator}]
%%   \begin{align*}
%%     f\left(\vzeta\right)
%%     &= 
%%     - \nabla_{\vlambda} \log\ell\left(\rvvx, \rvvz = \psi^{-1}\left( \vzeta \right) \right)
%%     - \log \abs{ \mJ_{\phi^{-1}}\left(\vzeta\right) } \\ &\qquad+ 
%%     \underbrace{\nabla_{\vlambda} \log q_{\vgamma}\left(\vzeta\right),}_{\text{Monte Carlo entropy estimate}}
%%   \end{align*}
%%   where \(\vgamma = \vlambda\) and \(h\left(\vlambda\right) = 0\).
%% \end{definition}
%% %
%% This estimator contains more stochastic elements than the entropy-form and KL-form estimators.
%% Even though the entropy term now \textit{adds} noise, it has shown in practice to result in lower variance and, therefore, faster, stable convergence.
%% This is because the entropy term acts as a control variate \citep{geffner_using_2018}, reducing variance as \(q_{\vlambda}\) becomes closer to \(\pi\).
%% We also provide an upper bound for the STL estimator.

%% 
  
\begin{theoremEnd}[\keylemmaproofoption,category=upperboundlemma]{lemma}\label{eq:lemma_stl_entropy_term_variance}
  Let \(V\) denote the expected squared norm of the derivative of \(\log \varphi\) such that \(
  V =
  \norm{
    \nabla \log \varphi\left( \vu \right)
  }_2^2.
  \)
  Then,
  \begin{align*}
    \mathbb{E} {\lVert
      \nabla \log q_{\gamma}\left(\vt_{\vlambda}\left(\vu\right)\right)
    \rVert}_2^2
    \leq
    2 \, V \left( \mathrm{H}\left(\varphi\right) - \mathrm{H}\left(q_{\vlambda}\right) \right).
  \end{align*}
\end{theoremEnd}
\begin{proofEnd}
  \begin{alignat}{2}
    &\mathbb{E} {\lVert
      \nabla \log q_{\vgamma}\left(\vt_{\vlambda}\left(\vu\right)\right)
    \rVert}_2^2 \, \Big\lvert_{\vgamma = \vlambda}
    \nonumber
    \\
    &\;=
    \mathbb{E} \norm{
      \frac{\partial}{\partial \vz}\left( \log \varphi\left( \mC^{-1}\left( \vz - \vm \right) \right) - \log \abs{\mC} \right)
    }_2^2 \, \Big\lvert_{\vz = \vt_{\vlambda}\left(\vu\right)}
    \nonumber
    \\
    &\;=
    \mathbb{E} {\lVert
      \mC^{-\top} \nabla \log \varphi\left( \mC^{-1}\left( \vt_{\vlambda}\left(\vu\right) - \vm \right) \right)
    \rVert}_2^2
    \nonumber
    \\
    &\;=
    \mathbb{E} {\lVert
      \mC^{-\top} \nabla \log \varphi\left( \vu \right)
    \rVert}_2^2
    \nonumber
    \\
    &\;\leq
    {\lVert
      \mC^{-\top}
    \rVert}^2_{2}
    \,
    \mathbb{E}
    {\lVert
       \nabla \log \varphi\left( \vu \right)
    \rVert}_2^2
    \nonumber
    \\
    &\;\leq
    {\lVert
      \mC^{-\top}
    \rVert}^2_{\mathrm{F}}
    \,
    \mathbb{E}
    {\lVert
       \nabla \log \varphi\left( \vu \right)
    \rVert}_2^2
    \nonumber
    \\
    &\;=
    V\,
    {\lVert
      \mC^{-1}
    \rVert}_{\mathrm{F}}^2
    \label{eq:lemma_stl_entropy_term_variance_eq1}
  \end{alignat}

  Now, for location-scale families, the entropy of \(q_{\vlambda}\) follows as
  \begin{alignat*}{2}
    \mathrm{H}\left(q_{\vlambda}\right) 
    &=
    \mathrm{H}\left(\varphi\right) 
    +
    \log \abs{\mC}
    \\
    &=
    \mathrm{H}\left(\varphi\right) 
    -
    \frac{1}{2}
    \log \abs{\mC^{-\top} \mC^{-1}}
    \\
    &=
    \mathrm{H}\left(\varphi\right) 
    -
    \frac{1}{2}
    \sum_{i=1}^d \log \sigma_i\left(\mC^{-\top} \mC^{-1}\right),
\shortintertext{and applying the inequality \(-x \leq -\log x\),}
    &\geq
    \mathrm{H}\left(\varphi\right) 
    -
    \frac{1}{2}
    \sum_{i=1}^d \sigma_i\left(\mC^{-\top} \mC^{-1}\right)
    \\
    &=
    \mathrm{H}\left(\varphi\right) 
    -
    \frac{1}{2}
    \mathrm{tr}\left(\mC^{-\top} \mC^{-1}\right)
    \\
    &=
    \mathrm{H}\left(\varphi\right) 
    -
    \frac{1}{2}
    {\lVert \mC^{-1} \rVert}_{\mathrm{F}}^2.
  \end{alignat*}

  Utilizing this in \cref{eq:lemma_stl_entropy_term_variance_eq1},
  \begin{alignat*}{2}
    \mathbb{E} {\lVert
      \nabla \log q_{\vgamma}\left(\vt_{\vlambda}\left(\vu\right)\right)
    \rVert}_2^2 \, \Big\lvert_{\vgamma = \vlambda}
    &\leq
    V\,
    {\lVert
      \mC^{-1}
    \rVert}_{\mathrm{F}}^2
    \\
    &\leq
    2 \, V \left( \mathrm{H}\left(\varphi\right) - \mathrm{H}\left(q_{\vlambda}\right) \right).
  \end{alignat*}

  \todo[inline]{
    The following bound based on the trace might be tighter.
  \begin{alignat*}{2}
    &\mathbb{E} \norm{
      \mC^{-\top} \nabla \log \varphi\left( \vu \right)
    }_2^2
    \\
    &\;=
    \mathbb{E} \,
    \nabla \log \varphi^{\top}\left( \vu \right)
    \mC^{-1} \mC^{-\top}
    \nabla \log \varphi\left( \vu \right)
    \\
    &\;=
    \mathbb{E} \mathrm{tr}\left(
      \nabla \log \varphi^{\top}\left( \vu \right)
      \mC^{-1} \mC^{-\top}
      \nabla \log \varphi\left( \vu \right)
    \right)
    \\
    &\;=
    \mathrm{tr}\left(
    \mathbb{E} 
      \nabla \log \varphi\left( \vu \right)
      \nabla \log \varphi^{\top}\left( \vu \right)
      \mC^{-1} \mC^{-\top}
    \right).
  \end{alignat*}
  
  Notice that 
  \begin{align*}
    \mathbb{E} 
      \nabla \log \varphi\left( \vu \right)
      \nabla \log \varphi^{\top}\left( \vu \right)
  \end{align*}
  is in fact similar to a Fisher information matrix.
  (Here, the derivative is with respect to the input instead of the parameters.
  So this is technically not a Fisher information matrix.)
  We can use the information identity to see that
  \begin{align*}
    \mathbb{E} 
    \nabla \log \varphi\left( \vu \right)
    \nabla \log \varphi^{\top}\left( \vu \right)
    =
    \mH_{\log \varphi}.
  \end{align*}
  But since each component of \(\varphi\) is \(i.i.d.\), the Hessian matrix is a diagonal matrix.

  \begin{alignat*}{2}
    &\mathbb{E} {\lVert
      \nabla \log q_{\vgamma}\left(\vt_{\vlambda}\left(\vu\right)\right)
    \rVert}_2^2 \, \Big\lvert_{\vgamma = \vlambda}
    \\
    &\;=
    \mathrm{tr}\left(
    \mathbb{E} 
      \nabla \log \varphi\left( \vu \right)
      \nabla \log \varphi^{\top}\left( \vu \right)
      \mC^{-1} \mC^{-\top}
    \right)
    \\
    &\;=
    \mathrm{tr}\left(
      \mH_{\log \varphi}
      \mC^{-1} \mC^{-\top}
    \right)
  \end{alignat*}
  If we assume some properties about \(\mH_{\log \varphi}\) then maybe?
  }
\end{proofEnd}

\begin{theoremEnd}[\theoremproofoption,category=upperboundtheoremstl]{theorem}
  Let \(f\) be a function satisfying \cref{assumption:L_smoothness,assumption:quadratic_growth}.
  Then, the variance of the \(M\)-sample gradient estimator is bounded below as
  \begin{align*}
    \mathbb{E}\norm{\rvvg\left(\vlambda\right)}_2^2
    \leq
    2 A \left( F\left(\vlambda\right) - F^* \right)
    + B \norm{ \nabla F\left(\vlambda\right) }_2^2
    + C,
  \end{align*}
  where \(F^* = \inf_{\vlambda \in \Lambda} F\left(\vlambda\right)\), for the finite constants
  \[
  A = \frac{L^2 \, \left(d + \kappa\right)}{M},\;
  B = \frac{M-1}{M},\;
  \;\text{and}\;
  C = \frac{L^2 \, \left(d + \kappa\right)}{K} F^*.
  \]
\end{theoremEnd}
\begin{proofsketch}
  We apply the Parallelogram Law 
  \[
  \norm{\va - \vb}_2^2 = 2 \, \norm{\va}^2_2 + 2\,\norm{\vb}^2_2 - \norm{\va + \vb}_2^2 \leq 2\,\norm{\va}^2_2 + 2\,\norm{\vb}^2_2
  \] as
  \begin{alignat*}{2}
    &\mathbb{E}{\lVert
      \nabla f\left(\vt_{\vlambda}\left(\vu\right)\right)
      - \nabla \log q_{\vgamma}\left(\vt_{\vlambda}\left(\vu\right)\right)
    \rVert}_2^2
    \\
    &\,=
    2 \, \mathbb{E} \norm{
      \nabla f\left(\vt_{\vlambda}\left(\vu\right)\right)
    }
    +
    2 \, \mathbb{E} {\lVert
      \nabla \log q_{\vlambda}\left(\vt_{\vlambda}\left(\vu\right)\right)
    \rVert}_2^2 
    \\
    &\quad-
    \underbrace{
    \mathbb{E}{\lVert
      \nabla f\left(\vt_{\vlambda}\left(\vu\right)\right) 
      +
      \nabla \log q_{\vgamma}\left(\vt_{\vlambda}\left(\vu\right)\right)
    \rVert}^2_2
    }_{\text{control variate effect}}
    \\
    &\,\leq
    2 \, \mathbb{E} \norm{
      \nabla f\left(\vt_{\vlambda}\left(\vu\right)\right)
    }
    +
    2 \, \mathbb{E} {\lVert
      \nabla \log q_{\vlambda}\left(\vt_{\vlambda}\left(\vu\right)\right)
    \rVert}_2^2.
  \end{alignat*}
  Note, that this step makes the bound looser than the ``regularized'' form estimator, which opposes the previous observation that the STL estimator achieves superior variance reduction in practice~\citep{roeder_sticking_2017, geffner_using_2018, agrawal_advances_2020}.

  This mismatch, however, is to be expected.
  The bounds we are establishing are ``worst case'' bounds, and the worst case performance of the STL estimator is known to be worse than the regularized form estimator.
  The advantage of the STL estimator is only realized when \(q_{\vlambda}\) and \(f\) becomes strongly correlated, acting as a control variate.
  In this case, the control variate effect appears as \(\norm{ \nabla_{\vlambda} f + \nabla_{\vlambda} \log q_{\vgamma} }\) being large, tightening the bound.
\end{proofsketch}
\begin{proofEnd}
  We first start from the definition of variance,
  \begin{alignat*}{2}
    &\mathbb{E} \norm{\vg_M }^2_2
    \\
    &\;=
    \mathrm{tr}\,\V{ \vg_M } + \norm{\mathbb{E} \vg}^2_2,
\shortintertext{following the definition in \cref{eq:def_gradient_M_est},}
    &\;=
    \mathrm{tr}\,\V{ \frac{1}{M} \sum_{m=1}^M \vg_m } + \norm{ \nabla F\left(\vlambda\right) }^2_2
\shortintertext{and then the definition in \cref{eq:def_gradient_m_est},}
    &\;=
    \mathrm{tr}\,\V{
      \frac{1}{M} \sum_{m=1}^M \frac{\partial }{\partial \vlambda} \Big( f\left(\vt_{\vlambda}\left(\rvvu_m\right)\right) - \log q_{\vgamma}\left(\vt_{\vlambda}\left(\rvvu_m\right) \right) \Big)
    } \Bigg\lvert_{\vgamma = \vlambda}
    \\
    &\;\quad+ \norm{ \nabla F\left(\vlambda\right) }^2_2,
\shortintertext{by the linearity of variance,}
    &\;=
    \frac{1}{M} \mathrm{tr}\,\V{
      \frac{\partial }{\partial \vlambda} \Big( f\left(\vt_{\vlambda}\left(\rvvu_m\right)\right) - \log q_{\vgamma}\left(\vt_{\vlambda}\left(\rvvu_m\right) \right) \Big)
    } \, \Big\lvert_{\vgamma = \vlambda}
    \\
    &\;\quad+ \norm{ \nabla F\left(\vlambda\right) }^2_2
    \\
    &\;=
    \frac{1}{M} \mathbb{E}\norm{
      \frac{\partial }{\partial \vlambda} \Big( f\left(\vt_{\vlambda}\left(\rvvu_m\right)\right) - \log q_{\vgamma}\left(\vt_{\vlambda}\left(\rvvu_m\right) \right) \Big)
    }_2^2 \, \Big\lvert_{\vgamma = \vlambda}
    \\
    &\;\quad+ \norm{ \nabla F\left(\vlambda\right) }^2_2,
\shortintertext{applying \cref{thm:general_variational_gradient_norm_identity},}
    &\;\leq
    \frac{1}{M} \mathbb{E}{\lVert
      \nabla f\left(\vt_{\vlambda}\left(\rvvu\right)\right)
      - \nabla \log q_{\vlambda}\left(\vt_{\vlambda}\left(\rvvu_m\right)\right)
    \rVert}_2^2 \left(1 + \norm{\rvvu}_2^2\right)
    \\
    &\;\quad+ \norm{ \nabla F\left(\vlambda\right) }^2_2
  \end{alignat*}

  We know apply the inequality \(\norm{\va - \vb}_2^2 \leq 2\,\norm{\va}^2_2 + 2\,\norm{\vb}^2_2\) as
  \begin{alignat*}{2}
    &\mathbb{E}{\lVert
      \nabla f\left(\vt_{\vlambda}\left(\vu\right)\right)
      - \nabla \log q_{\vgamma}\left(\vt_{\vlambda}\left(\vu\right)\right)
    \rVert}_2^2
    \\
    &\,=
    2 \, \mathbb{E} \norm{
      \nabla f\left(\vt_{\vlambda}\left(\vu\right)\right)
    }
    +
    2 \, \mathbb{E} {\lVert
      \nabla \log q_{\vlambda}\left(\vt_{\vlambda}\left(\vu\right)\right)
    \rVert}_2^2 
    \\
    &\quad-
    \underbrace{
    \mathbb{E}{\lVert
      \nabla f\left(\vt_{\vlambda}\left(\vu\right)\right) 
      +
      \nabla \log q_{\vgamma}\left(\vt_{\vlambda}\left(\vu\right)\right)
    \rVert}^2_2
    }_{\text{control variate effect}}
    \\
    &\,\leq
    2 \, \mathbb{E} \norm{
      \nabla f\left(\vt_{\vlambda}\left(\vu\right)\right)
    }
    +
    2 \, \mathbb{E} {\lVert
      \nabla \log q_{\vlambda}\left(\vt_{\vlambda}\left(\vu\right)\right)
    \rVert}_2^2.
  \end{alignat*}

  Due to \cref{eq:lemma_stl_entropy_term_variance}, 
  \begin{alignat*}{2}
    &\mathbb{E} {\lVert
      \nabla \log q_{\vgamma}\left(\vt_{\vlambda}\left(\vu\right)\right)
    \rVert}_2^2 \, \Big\lvert_{\vgamma = \vlambda}
    \\
    &\;=
    2 \, V \left( \mathrm{H}\left(\varphi\right) - \mathrm{H}\left(q_{\vlambda}\right) \right)
    \\
    &\;=
    2 \, V \left( \mathbb{E}f\left(\vt_{\vlambda}\left(\rvvu\right)\right) - \mathrm{H}\left(q_{\vlambda}\right) - \mathbb{E}f\left(\vt_{\vlambda}\left(\rvvu\right)\right) + \mathrm{H}\left(\varphi\right) \right)
    \\
    &\;=
    2 \, V \left( F\left(\vlambda\right)  - \mathbb{E}f\left(\vt_{\vlambda}\left(\rvvu\right)\right) + \mathrm{H}\left(\varphi\right) \right)
    \\
    &\;\leq
    2 \, V \left( F\left(\vlambda\right) - f^* + \mathrm{H}\left(\varphi\right) \right).
  \end{alignat*}

  \begin{alignat*}{2}
    &\mathbb{E} \norm{\vg_M }^2_2
    \\
    &\;\leq
    \frac{1}{M} \mathbb{E}{\lVert
      \nabla f\left(\vt_{\vlambda}\left(\rvvu\right)\right)
      - \nabla \log q_{\vlambda}\left(\vt_{\vlambda}\left(\rvvu_m\right)\right)
    \rVert}_2^2 \left(1 + \norm{\rvvu}_2^2\right)
    \\
    &\;\quad+ \norm{ \nabla F\left(\vlambda\right) }^2_2
    \\
    &\;\leq
    \frac{1}{M} \left(
    2 \, \mathbb{E} \norm{
      \nabla f\left(\vt_{\vlambda}\left(\vu\right)\right)
    }_2^2
    +
    2 \, \mathbb{E} {\lVert
      \nabla \log q_{\vlambda}\left(\vt_{\vlambda}\left(\vu\right)\right)
    \rVert}_2^2
    \right)
    \\
    &\;\quad+ \norm{ \nabla F\left(\vlambda\right) }^2_2
    \\
    &\;\leq
    \frac{1}{M} \Bigg(
    \frac{2\,L^2 }{\mu} \left(d + \kappa\right) \left(  F\left(\vlambda\right) - h\left(\vlambda\right) - f^* \right)
    \\
    &\quad\quad+
    2 \, V \left( F\left(\vlambda\right) - f^* + \mathrm{H}\left(\varphi\right) \right)
    \Bigg)
    \\
    &\;\quad+ \norm{ \nabla F\left(\vlambda\right) }^2_2
  \end{alignat*}
\end{proofEnd}

%%% Local Variables:
%%% TeX-master: "main"
%%% End:


%% Naturally, since gradient variance bounds are worst-case bounds, this estimator has \textit{worse} variance guarantee.


\subsection{Matching Lower Bound}
Finally, we present a matching lower bound on the gradient variance of BBVI.
Our lower bound holds broadly for smooth and strongly convex problem instances that are well-conditioned and high-dimensional.

\vspace{.5ex}
%!TEX root=main.tex

% \begin{theoremEnd}[\lemmaproofoption,category=lowerboundlemma]{lemma}
%   Let \(\vt_{\vlambda}: \mathbb{R}^d \mapsto \mathbb{R}^d\) be defined as in \cref{def:reparam} with parameters \(\vlambda = \left(\vm, \mC\right)\) such that \(\vm \in \mathbb{R}^d\) and \(\mC \in \mathbb{R}^{d \times d}\).
%   Also, let \(\rvvu \sim \varphi\), where \(\varphi\) is defined as in \cref{assumption:symmetric_standard}.
%   Then,
%   \begin{alignat*}{2}
%     \mathbb{E}\vt_{\vlambda}\left(\rvvu\right) \left(1 + \norm{\rvvu}_2^2\right)
%     = \left(d + 1\right) \vm.
%   \end{alignat*}
% \end{theoremEnd}
% \begin{proofEnd}
%   Since \(\rvu_i\) follows a symmetric and standardized distribution, \(\mathbb{E}\rvu_i^3 = 0, \mathbb{E} \rvu_i^2 = 1\).
%   Then,
%   \begin{alignat*}{2}
%     \mathbb{E}\vt_{\vlambda}\left(\rvvu\right) \left(1 + \norm{\rvvu}_2^2\right)
%     &=
%     \mathbb{E} \left(\mC \rvvu + \vm \right) \left(1 + \norm{\rvvu}_2^2\right)
%     \\
%     &=
%     \mC \, \mathbb{E} \rvvu \left(1 + \norm{\rvvu}_2^2\right) + \vm \left(1 + \mathbb{E} \norm{\rvvu}_2^2\right),
% \shortintertext{applying \cref{thm:u_identities},}
%     &=
%     \left( \mathbb{E}\rvu_i^3 \right) \mC \mathbf{1} + \vm \left(1 + d \, \mathbb{E} \rvu_i^2\right),
% \shortintertext{and by~\cref{assumption:symmetric_standard},}
%     &=
%     \left(d + 1\right) \vm.
%   \end{alignat*}
% \end{proofEnd}

% \begin{theoremEnd}[\lemmaproofoption,category=lowerboundlemma]{lemma}\label{thm:general_quad_reparam}
%   Let \(\vt_{\vlambda}: \mathbb{R}^d \mapsto \mathbb{R}^d\) be defined as in \cref{def:reparam} with parameters \(\vlambda = \left(\vm, \mC\right)\) such that \(\vm \in \mathbb{R}^d\) and \(\mC \in \mathbb{R}^{d \times d}\).
%   Also, let \(\mSigma \in \mathbb{R}^{d \times d}\) be some matrix, \(\vmu \in \mathbb{R}^d\) be some vector, and \(\rvvu \sim \varphi\) be a vector-valued random variable, where \(\varphi\) is defined as in \cref{assumption:symmetric_standard}.
%   Then,
%   \begin{align*}
%     &\mathbb{E} \, {\left( \vt_{\vlambda}\left(\rvvu\right) - \vmu \right)}^{\top} \mSigma \left(\vt_{\vlambda}\left(\rvvu\right) - \vmu\right)
%     \\
%     &\;= {\left(\vm - \vmu\right)}^{\top} \mSigma \left(\vm - \vmu\right)
%     + \mathrm{tr}\left({\mSigma}^{-1} \mC \mC^{\top}\right).
%   \end{align*}
% \end{theoremEnd}
% \begin{proofEnd}
%   \begin{alignat}{2}
%     &\mathbb{E} {\left( \vt_{\vlambda}\left(\rvvu\right) - \vmu \right)}^{\top} \mSigma \, \left(\vt_{\vlambda}\left(\rvvu\right) - \vmu \right)
%     \nonumber
%     \\
%     &\;=
%     \mathbb{E}\, {\vt_{\vlambda}\left(\rvvu\right)}^{\top} \mSigma \, {\vt_{\vlambda}\left(\rvvu\right)}
%     -2\, \vmu^{\top} {\mSigma} \, \mathbb{E}\vt_{\vlambda}\left(\rvvu\right)
%     + \vmu^{\top} {\mSigma} \, \vmu,
%     \nonumber
% \shortintertext{where \(\mathbb{E}\vt_{\vlambda}\left(\rvvu\right) = \mC \mathbb{E}{\rvvu} + \vm\), and by \cref{assumption:symmetric_standard},}
%     &\;=
%     \mathbb{E}\, {\vt_{\vlambda}\left(\rvvu\right)}^{\top} \mSigma \, {\vt_{\vlambda}\left(\rvvu\right)}
%     -2\, \vmu^{\top} {\mSigma} \, \vm
%     + \vmu^{\top} {\mSigma} \, \vmu.\label{eq:thm:general_quad_reparam_eq1}
%   \end{alignat}
%   Furthermore,
%   \begin{alignat*}{2}
%     &\mathbb{E} {\vt_{\vlambda}\left(\rvvu\right)}^{\top} \mSigma  \, {\vt_{\vlambda}\left(\rvvu\right)}
%     \\
%     &\;=
%     \mathbb{E}{\left(\mC \rvvu + \vm\right)}^{\top}  \mSigma \left(\mC \rvvu + \vm\right)
%     \\
%     &\;=
%     \mathbb{E} \rvvu^{\top} \mC^{\top} \mSigma \, \mC \rvvu
%     + \vm^{\top} \mSigma \, \mC \, \mathbb{E}  \rvvu 
%     + \mathbb{E} \rvvu^{\top} \mC^{\top} \mSigma \vm^{\top} 
%     + \vm^{\top} \mSigma \, \vm,
% \shortintertext{by \cref{assumption:symmetric_standard},}
%     &\;=
%     \mathbb{E} \rvvu^{\top} \mC^{\top} {\mSigma} \, \mC \rvvu
%     + \vm^{\top} \mSigma \, \vm,
% \shortintertext{invoking the trace trick,}
%     &\;=
%     \mathbb{E} \mathrm{tr}\left( \rvvu^{\top} \mC^{\top} {\mSigma}\, \mC \rvvu \right)
%     + \vm^{\top} \mSigma \, \vm,
% \shortintertext{pusing the expectation into the trace,}
%     &\;=
%     \mathrm{tr}\left(\mathbb{E} \rvvu  \rvvu^{\top} \mC^{\top} {\mSigma}\, \mC \right)
%     + \vm^{\top} \mSigma \, \vm,
% \shortintertext{invoking \cref{thm:u_identities},}
%     &\;=
%     \mathrm{tr}\left(\mC^{\top} {\mSigma}\, \mC \right)
%     + \vm^{\top} \mSigma \, \vm
%     \\
%     &\;=
%     \mathrm{tr}\left({\mSigma} \, \mC \mC^{\top}\right)
%     + \vm^{\top} \mSigma \, \vm.
%   \end{alignat*}
%   Applying this to \cref{eq:thm:general_quad_reparam_eq1},
%   \begin{alignat*}{2}
%     &\mathbb{E} {\left( \vt_{\vlambda}\left(\rvvu\right) - \vmu \right)}^{\top} {\mSigma}^{-1} \left(\vt_{\vlambda}\left(\rvvu\right) - \vmu \right)
%     \\
%     &\;=
%     \mathrm{tr}\left({\mSigma}^{-1} \mC \mC^{\top}\right)
%     + \vm^{\top} \mSigma^{-1} \vm
%     -2\, \vmu^{\top} {\mSigma}^{-1} \vm
%     + \vmu^{\top} {\mSigma}^{-1} \vmu
%     \\
%     &\;=
%     {\left(\vm - \vmu\right)}^{\top} {\mSigma}^{-1} \left(\vm - \vmu\right)
%     + \mathrm{tr}\left({\mSigma}^{-1} \mC \mC^{\top}\right).
%   \end{alignat*}
% \end{proofEnd}

% \begin{theoremEnd}[\lemmaproofoption,category=lowerboundlemma]{corollary}\label{thm:q_quad_reparam}
%   Let \(\vt_{\vlambda}: \mathbb{R}^d \mapsto \mathbb{R}^d\) be defined as in \cref{def:reparam} with parameters \(\vlambda = \left(\vm, \mC\right)\) such that \(\vm \in \mathbb{R}^d\) and \(\mC \in \mathbb{R}^{d \times d}\).
%   Also, let \(\rvvu \sim \varphi\) be a vector-valued random variable, where \(\varphi\) is defined as in \cref{assumption:symmetric_standard}.
%   Then,
%   \begin{align*}
%     \mathbb{E} \, {\left( \vt_{\vlambda}\left(\rvvu\right) - \vm \right)}^{\top} {\left(\mC \mC^{\top}\right)}^{-1} \left(\vt_{\vlambda}\left(\rvvu\right) - \vm\right) = d.
%   \end{align*}
% \end{theoremEnd}
% \begin{proofEnd}
%   From, \cref{thm:general_quad_reparam} we have
%   \begin{alignat*}{2}
%     &\mathbb{E} \, {\left( \vt_{\vlambda}\left(\rvvu\right) - \vm \right)}^{\top} {\left(\mC \mC^{\top}\right)}^{-1} \left(\vt_{\vlambda}\left(\rvvu\right) - \vm\right)
%     \\
%     &\;= {\left(\vm - \vm\right)}^{\top} {\left(\mC \mC^{\top}\right)}^{-1} \left(\vm - \vm\right)
%     + \mathrm{tr}\left({\left( \mC \mC^{\top} \right)}^{-1} \mC \mC^{\top}\right)
%     \\
%     &\;= \mathrm{tr}\left(\mI\right)
%     \\
%     &\;= d.
%   \end{alignat*}
% \end{proofEnd}

% \begin{theoremEnd}[\lemmaproofoption,category=lowerboundlemma]{lemma}\label{thm:general_grad_norm_reparam}
%   Let \(\vt_{\vlambda}: \mathbb{R}^d \mapsto \mathbb{R}^d\) be defined as in \cref{def:reparam} with parameters \(\vlambda = \left(\vm, \mC\right)\) such that \(\vm \in \mathbb{R}^d\) and \(\mC \in \mathbb{R}^{d \times d}\).
%   Also, let \(\mSigma_1, \mSigma_2 \in \mathbb{R}^{d \times d}\) be some matrices, \(\vmu_1, \vmu_2 \in \mathbb{R}^{d}\) be some vectors, and \(\rvvu = \left(\rvu_1, \ldots, \rvu_d \right)\) be a vector-valued random variable sampled as \(\rvvu \sim \varphi\), where \(\varphi\) is defined as in \cref{assumption:symmetric_standard} with kurtosis \(\kappa = \mathbb{E}\rvu_i^4\).
%   Then,
%   \begin{alignat*}{2}
%     &\mathbb{E} {\left(\vt_{\vlambda}\left(\rvvu\right) - \vmu_1\right)}^{\top} \mSigma_1 \mSigma_2 \left(\vt_{\vlambda}\left(\rvvu\right) - \vmu_2\right) \left( 1 + \norm{\rvvu}_2^2 \right)
%     \\
%     &\;=
%     \left(d + 1\right) \vmu_1^{\top} \mSigma_1 \mSigma_2 \vmu_2
%     +
%     \left(d + \kappa\right) \mathrm{tr}\left(\mC \mC^{\top} \mSigma_1 \mSigma_2 \right).
%   \end{alignat*}
% \end{theoremEnd}
% \begin{proofEnd}
%   \begin{alignat}{2}
%     &\mathbb{E} {\left(\vt_{\vlambda}\left(\rvvu\right) - \vmu_1\right)}^{\top} \mSigma_1 \mSigma_2 \left(\vt_{\vlambda}\left(\rvvu\right) - \vmu_2\right) \left( 1 + \norm{\rvvu}_2^2 \right)
%     \nonumber
%     \\
%     &\;=
%     \mathbb{E} {\vt_{\vlambda}\left(\rvvu\right)}^{\top} \mSigma_1 \mSigma_2 \, \vt_{\vlambda}\left(\rvvu\right) \left( 1 + \norm{\rvvu}_2^2 \right)
%     \nonumber
%     \\
%     &\qquad- \vmu_1^{\top} \mSigma_1 \mSigma_2 \, \mathbb{E} \vt_{\vlambda}\left(\rvvu\right) \left( 1 + \norm{\rvvu}_2^2 \right)
%     \nonumber
%     \\
%     &\qquad- \mathbb{E} {\left(\vt_{\vlambda}\left(\rvvu\right) \left( 1 + \norm{\rvvu}_2^2 \right)\right)}^{\top} \mSigma_1 \mSigma_2 \vmu_1  
%     \nonumber
%     \\
%     &\qquad+ \vmu_1^{\top} \mSigma_1 \mSigma_2 \vmu_2 \, \mathbb{E}\left( 1 + \norm{\rvvu}_2^2 \right),
%     \nonumber
% \shortintertext{invoking \cref{thm:u_identities},}
%     &\;=
%     \mathbb{E} {\vt_{\vlambda}\left(\rvvu\right)}^{\top} \mSigma_1 \mSigma_2 \, \vt_{\vlambda}\left(\rvvu\right) \left( 1 + \norm{\rvvu}_2^2 \right)
%     \nonumber
%     \\
%     &\qquad- \mathbb{E}\rvu_i^3 \, \vmu_1^{\top} \mSigma_1 \mSigma_2 \mathbf{1}
%     \nonumber
%     \\
%     &\qquad- \mathbb{E}\rvu_i^3 \, \mathbf{1}^{\top} \mSigma_1 \mSigma_2 \vmu_1  
%     \nonumber
%     \\
%     &\qquad+ \left( 1 + d\,\mathbb{E}\rvu_i^2 \right) \vmu_1^{\top} \mSigma_1 \mSigma_2 \vmu_2,
%     \nonumber
% \shortintertext{and due to \cref{assumption:symmetric_standard},}
%     &\;=
%     \mathbb{E} {\vt_{\vlambda}\left(\rvvu\right)}^{\top} \mSigma_1 \mSigma_2 \, \vt_{\vlambda}\left(\rvvu\right) \left( 1 + \norm{\rvvu}_2^2 \right)
%     \label{eq:general_grad_norm reparam_eq1}
%     \\
%     &\qquad+ \left( 1 + d \right) \vmu_1^{\top} \mSigma_1 \mSigma_2 \vmu_2.\label{eq:general_grad_norm reparam_eq2}
%   \end{alignat}
%   Now for \cref{eq:general_grad_norm reparam_eq1},
%   \begin{alignat}{2}
%     &\mathbb{E} {\vt_{\vlambda}\left(\rvvu\right)}^{\top} \mSigma_1 \mSigma_2 \, \vt_{\vlambda}\left(\rvvu\right) \left( 1 + \norm{\rvvu}_2^2 \right)
%     \nonumber
%     \\
%     &\;=
%     \mathbb{E} {\left(\mC \rvvu + \vm \right)}^{\top} \mSigma_1 \mSigma_2 \left(\mC \rvvu + \vm \right) \left( 1 + \norm{\rvvu}_2^2 \right)
%     \nonumber
%     \\
%     &\;=
%     \mathbb{E} \mathrm{tr}\left( \rvvu^{\top} \mC^{\top} \mSigma_1 \mSigma_2 \mC \rvvu \right)
%     \nonumber
%     \\
%     &\qquad+
%     \mathbb{E} \mathrm{tr}\left( \rvvu^{\top} \mC^{\top} \mSigma_1 \mSigma_2 \mC \, \rvvu \rvvu^{\top} \rvvu  \right)
%     \label{eq:thm:general_grad_norm_reparam_eq1}
%     \\
%     &\qquad+
%     \vm^{\top} \mSigma_1 \mSigma_2 \mC \, \mathbb{E} \rvvu \left( 1 + \norm{\rvvu}_2^2 \right)
%     \nonumber
%     \\
%     &\qquad+
%     \mathbb{E} \rvvu^{\top} \left( 1 + \norm{\rvvu}_2^2 \right) \mC^{\top} \mSigma_1 \mSigma_2 \vm
%     \nonumber
%     \\
%     &\qquad+
%     \vm^{\top} \mSigma_1 \mSigma_2 \vm \, \mathbb{E} \rvvu \left( 1 + \norm{\rvvu}_2^2 \right),
%     \nonumber
% %
% \shortintertext{using the cyclic property of the trace on \cref{eq:thm:general_grad_norm_reparam_eq1},}
% %
%     &\;=
%     \mathrm{tr}\left( \mathbb{E}  \rvvu \rvvu^{\top} \mC^{\top} \mSigma_1 \mSigma_2 \mC  \right)
%     \nonumber
%     \\
%     &\qquad+
%     \mathrm{tr}\left( \mC^{\top} \mSigma_1 \mSigma_2 \mC \, \mathbb{E} \rvvu \rvvu^{\top} \rvvu \rvvu^{\top} \right)
%     \nonumber
%     \\
%     &\qquad+
%     \vm^{\top} \mSigma_1 \mSigma_2 \mC \, \mathbb{E} \rvvu \left( 1 + \norm{\rvvu}_2^2 \right)
%     \nonumber
%     \\
%     &\qquad+
%      \mathbb{E} \rvvu^{\top} \left( 1 + \norm{\rvvu}_2^2 \right) \mC^{\top} \mSigma_1 \mSigma_2  \, \vm
%     \nonumber
%     \\
%     &\qquad+
%     \vm^{\top} \mSigma_1 \mSigma_2 \vm \, \mathbb{E} \rvvu \left( 1 + \norm{\rvvu}_2^2 \right)
%     \nonumber
% %
% \shortintertext{invoking \cref{thm:u_identities},}
% %
%     &\;=
%     \mathrm{tr}\left( \left(\mathbb{E}\rvu_i^2\right) \mI \mC^{\top} \mSigma_1 \mSigma_2 \mC  \right)
%     \nonumber
%     \\
%     &\qquad+
%     \mathrm{tr}\left( \mC^{\top} \mSigma_1 \mSigma_2 \mC \left(\left(d-1\right) {\left(\mathbb{E}\rvu_i^2\right)}^2 + \mathbb{E}\rvu_i^4 \right) \mI \right)
%     \nonumber
%     \\
%     &\qquad+
%     \vm^{\top} \mSigma_1 \mSigma_2 \mC \, \left(\mathbb{E} \rvu_i^3\right) \mathbf{1} 
%     \nonumber
%     \\
%     &\qquad+
%     \left(\mathbb{E} \rvu_i^3\right) \mathbf{1}^{\top} \mC^{\top} \mSigma_1 \mSigma_2  \vm
%     \nonumber
%     \\
%     &\qquad+
%     \vm^{\top} \mSigma_1 \mSigma_2 \vm \, \left( \mathbb{E} \rvu_i^3 \right) \mathbf{1},
%     \nonumber
% %
% \shortintertext{and due to \cref{assumption:symmetric_standard},}
% %
%     &\;=
%     \mathrm{tr}\left( \mC^{\top} \mSigma_1 \mSigma_2 \mC  \right)
%     +
%     \left(d - 1 + \kappa\right) \mathrm{tr}\left( \mC^{\top} \mSigma_1 \mSigma_2 \mC\right)
%     \nonumber
%     \\
%     &\;=
%     \left(d + \kappa\right) \mathrm{tr}\left(\mC^{\top} \mSigma_1 \mSigma_2 \mC \right)
%     \nonumber
%     \\
%     &\;=
%     \left(d + \kappa\right) \mathrm{tr}\left(\mC  \mC^{\top} \mSigma_1 \mSigma_2 \right).
%     \nonumber
%   \end{alignat}
%   Finally, applying this to \cref{eq:general_grad_norm reparam_eq2},
%   \begin{alignat*}{2}
%     &\mathbb{E} \left(\vt_{\vlambda}\left(\rvvu\right) - \vmu_1\right) \mSigma_1 \mSigma_2 \left(\vt_{\vlambda}\left(\rvvu\right) - \vmu_2\right) \left( 1 + \norm{\rvvu}_2^2 \right)
%     \nonumber
%     \\
%     &\;=
%     \left(d + 1\right) \vmu_1^{\top} \mSigma_1 \mSigma_2 \vmu_2
%     +
%     \left(d + \kappa\right) \mathrm{tr}\left(\mC \mC^{\top} \mSigma_1 \mSigma_2 \right).
%   \end{alignat*}
% \end{proofEnd}

% \begin{theoremEnd}[\lemmaproofoption,category=lowerboundlemma]{corollary}\label{thm:q_grad_norm_reparam}
%   Let \(\vt_{\vlambda}: \mathbb{R}^d \mapsto \mathbb{R}^d\) be defined as in \cref{def:reparam} with parameters \(\vlambda = \left(\vm, \mC\right)\) such that \(\vm \in \mathbb{R}^d\) and \(\mC \in \mathbb{R}^{d \times d}\).
%   Also, let \(\rvvu = \left(\rvu_1, \ldots, \rvu_d \right)\) be a vector-valued random variable sampled as \(\rvvu \sim \varphi\), where \(\varphi\) is defined as in \cref{assumption:symmetric_standard} with kurtosis \(\kappa = \mathbb{E}\rvu_i^4\).
%   Then,
%   \begin{alignat*}{2}
%     &\mathbb{E} \left(\vt_{\vlambda}\left(\rvvu\right) - \vm\right) {\left( \mC \mC^{\top}\right)}^{-\top} {\left( \mC \mC^{\top}\right)}^{-1} \left(\vt_{\vlambda}\left(\rvvu\right) - \vm\right) \left( 1 + \norm{\rvvu}_2^2 \right)
%     \\
%     &\;=
%     \left(d + 1\right) \vm^{\top} {\left( \mC \mC^{\top}\right)}^{-\top} {\left( \mC \mC^{\top}\right)}^{-1} \vm
%     +
%     \left(d + \kappa\right) \mathrm{tr}\left( {\left( \mC \mC^{\top} \right)}^{-1} \right).
%   \end{alignat*}
% \end{theoremEnd}
% \begin{proofEnd}
%   From \cref{thm:general_grad_norm_reparam}, we have
%   \begin{alignat}{2}
%     &\mathbb{E} {\left(\vt_{\vlambda}\left(\rvvu\right) - \vm\right)}^{\top} {\left( \mC \mC^{\top}\right)}^{-\top} {\left( \mC \mC^{\top}\right)}^{-1} \left(\vt_{\vlambda}\left(\rvvu\right) - \vm\right) \left( 1 + \norm{\rvvu}_2^2 \right)
%     \nonumber
%     \\
%     &\;=
%     \left(d + 1\right) \vm^{\top} {\left( \mC \mC^{\top}\right)}^{-\top} {\left( \mC \mC^{\top}\right)}^{-1} \vm
%     \nonumber
%     \\
%     &\quad+
%     \left(d + \kappa\right) \mathrm{tr}\left(\mC \mC^{\top} {\left( \mC \mC^{\top}\right)}^{-\top} {\left( \mC \mC^{\top}\right)}^{-1}\right).\label{thm:q_grad_norm_reparam_eq1}
%   \end{alignat}
%   Since, by the basic properties of the trace,
%   \begin{alignat*}{2}
%     \mathrm{tr}\left(\mC \mC^{\top} {\left( \mC \mC^{\top}\right)}^{-\top} {\left( \mC \mC^{\top}\right)}^{-1}\right)
%     &=
%     \mathrm{tr}\left( {\left( \mC \mC^{\top}\right)}^{-1} \mC \mC^{\top} {\left( \mC \mC^{\top}\right)}^{-\top} \right)
%     \\
%     &=
%     \mathrm{tr}\left( \mI \, {\left( \mC \mC^{\top}\right)}^{-\top} \right)
%     \\
%     &=
%     \mathrm{tr}\left( {\left( \mC \mC^{\top}\right)}^{-\top} \right)
%     \\
%     &=
%     \mathrm{tr}\left( {\left( \mC \mC^{\top} \right)}^{-1} \right),
%   \end{alignat*}
%   \cref{thm:q_grad_norm_reparam_eq1} becomes
%   \begin{alignat*}{2}
%     &\mathbb{E} {\left(\vt_{\vlambda}\left(\rvvu\right) - \vm\right)}^{\top} {\left( \mC \mC^{\top}\right)}^{-\top} {\left( \mC \mC^{\top}\right)}^{-1} \left(\vt_{\vlambda}\left(\rvvu\right) - \vm\right) \left( 1 + \norm{\rvvu}_2^2 \right)
%     \\
%     &\;=
%     \left(d + 1\right) \vm^{\top} {\left( \mC \mC^{\top}\right)}^{-\top} {\left( \mC \mC^{\top}\right)}^{-1} \vm
%     +
%     \left(d + \kappa\right) \mathrm{tr}\left( {\left( \mC \mC^{\top} \right)}^{-1} \right).
%   \end{alignat*}
% \end{proofEnd}


% \begin{assumption}[\textbf{Polyak-Łojasiewicz Condition; PL}]\label{assumption:pl}
%   If the gradient of a function \(f : \mathbb{R}^d \rightarrow \mathbb{R}\) satisfies
%   \begin{align*}
%     \frac{1}{2\mu}\norm{ \nabla f\left(\vz\right) }_2^2 \geq  f\left(\vz\right) - f^*
%   \end{align*}
%   for some constant $\mu > 0$, where \(f^* = \inf_{\vz \in \mathbb{R}^d} f\left(\vz\right)\), then $f$ is said to satisfy the Polyak-Łojasiewicz condition.
% \end{assumption}


\begin{theoremEnd}[\theoremproofoption,category=lowerboundtheorem]{theorem}\label{thm:gradient_lower_bound}
Let \(\rvvg_{M}\) be an \(M\)-sample estimator of the gradient of the ELBO in either the entropy- or KL-regularized form.
Also, let ~\cref{assumption:q} hold where the matrix square root parameterization is used.
Then, for all \(L\)-smooth and \(\mu\)-strongly convex functions \(f\) such that $\nicefrac{L}{\mu} < \sqrt{d + 1}$, the variance of \(\rvvg_{M}\) is bounded below by some strictly positive constant as
\begin{align*}
  \mathbb{E}\norm{\rvvg_M}_2^2
  &\geq
  \frac{2\mu^2 \left(d + 1\right) - 2 L^2}{ML} \left( F\left(\vlambda\right) - F^* \right) 
  + \norm{ \nabla F\left(\vlambda\right) }_2^2 \\
  &\quad+ \frac{2 \mu^2 \left(d + 1\right) - 2 L^2}{ML} \left(\mathbb{E}{f\left(\vt_{\vlambda^*}\left(\vu\right)\right)} - f^*\right),  
\end{align*}
as long as $\vlambda$ is in a local neighborhood around the unique global optimum $\vlambda^* = \argmin_{\vlambda \in \mathbb{R}^p} F\left(\vlambda\right)$, where \(F^* = F\left(\vlambda^*\right)\) and \(f^* = \argmin_{\vzeta \in \mathbb{R}^d} f\left(\vzeta\right)\).
\end{theoremEnd}
\vspace{-3ex}
\begin{proofsketch}
  We use the fact that, with the matrix square root parameterization, if \(f\) is \(L\)-smooth, $\mathbb{E} f\left(\vt_{\vlambda}\left(\rvvu\right)\right)$ is also $L$-smooth~\citep{domke_provable_2020}.
  From this, the parameter suboptimality can be related to the function suboptimality as
  {%
  \setlength{\belowdisplayskip}{1ex} \setlength{\belowdisplayshortskip}{1ex}%
  \setlength{\abovedisplayskip}{1ex} \setlength{\abovedisplayshortskip}{1ex}%
  \begin{align*}
    {\lVert \vlambda - \bar{\vlambda} \rVert}^2_2
    \geq
    \left({2}/{L}\right)
    \left(
    \mathbb{E}{f\left(\vt_{\vlambda}\left(\rvvu\right)\right)} 
    - f^*
    \right),
  \end{align*}
  }%
  where \(\bar{\vlambda} = \left(\bar{\vzeta}, \boldupright{O}\right)\).
  For the entropy term, we circumvent the need to directly bound its value by restricting our interest to the neighborhood of the minimizer \(\vlambda^*\), where the contribution of \(h\left(\vlambda^*\right) - h\left(\vlambda\right)\) will be marginal enough for the lower bound to hold.
\end{proofsketch}

\vspace{-2ex}
\begin{proofEnd}
% Note that
% \begin{align*}
% \mathbb{E}\norm{ \vg_M }_2^2
%  = \frac{1}{M} \left(
% \mathbb{E}{ \norm{\nabla_{\vlambda} f\left(\vt_{\vlambda}\left(\rvvu\right)\right)}_2^2 }
% -
% \norm{\mathbb{E}{ \nabla_{\vlambda} f\left(\vt_{\vlambda}\left(\rvvu\right)\right)} }_2^2
% \right)
% + \norm{ \nabla F\left(\vlambda\right) }^2_2.
% \end{align*}
When using the matrix square root parameterization,~\citet{domke_provable_2020} have shown that if $f$ is $L$-smooth, $\mathbb{E} f\left(\vt_{\vlambda}\left(\rvvu\right)\right)$ is also $L$-smooth.
Therefore, we have
\begin{align} 
    \norm{\mathbb{E}{ \nabla_{\vlambda} f\left(\vt_{\vlambda}\left(\rvvu\right)\right)} }_2^2 \leq 2 L \left(\mathbb{E} f\left(\vt_{\vlambda}\left(\rvvu\right)\right) - f^*\right).
    \label{eq:thm_lower_bound_eq1}
\end{align}

Furthermore, let $\bar{\vzeta}$ be the minimizer of $f$, namely $f^* = f\left(\bar{\vzeta}\right)$.
From \cref{thm:variational_gradient_norm_identity}, we have
\begin{align*}
    \mathbb{E}{\norm{\nabla_{\vlambda} f\left(\vt_{\vlambda}\left(\rvvu\right)\right)}_2^2} 
    & = \mathbb{E}{\norm{\nabla f\left(\vt_{\vlambda}\left(\rvvu\right)\right)}_2^2 \left(1 + \norm{\rvvu}_2^2\right)},
\shortintertext{by the \(\mu\)-strong convexity of \(f\),}
    & \geq 2 \mu \, \mathbb{E}{\left(f\left(\vt_{\vlambda}\left(\rvvu\right)\right) - f^*\right) \left(1 + \norm{\rvvu}_2^2\right)} \\
    & \geq \mu^2 \, \mathbb{E}{{\lVert\vt_{\vlambda}\left(\rvvu\right) - \bar{\vzeta}\rVert}^2_2 \left(1 + \norm{\rvvu}_2^2\right)},
\shortintertext{applying \Cref{thm:reparam_u_identity},}
    & = \mu^2 \, \left(\left(d + 1\right) {\lVert \vm - \bar{\vzeta} \rVert}^2_2 + \left(d + \kappa\right) \norm{\mC}_{\mathrm{F}}^2\right),
\shortintertext{and by the property of the kurtosis that \(\kappa \geq 1\),}
    & \geq \mu^2 \, \left(d + 1\right) {\lVert \vlambda - \bar{\vlambda} \rVert}^2_2,
\end{align*}
where $\bar{\vlambda} = \left(\bar{\vzeta}, \boldupright{O}\right)$.

Observe that $\bar{\vlambda}$ is the minimizer of $\mathbb{E}{f\left(\vt_{\vlambda}\left(\rvvu\right)\right)}$ 
 such that 
\[
   \mathbb{E}{f\left(\vt_{\bar{\vlambda}}\left(\rvvu\right)\right)} = f\left(\bar{\vzeta}\right) = f^* \leq \mathbb{E}{f\left(\vt_{\vlambda}\left(\rvvu\right)\right)}
\]
for any $\vlambda$.
Furthermore, from the $L$-smoothness of $\mathbb{E}{f\left(\vt_{\vlambda}\left(\rvvu\right)\right)}$, we have
\begin{align*}
    &\mu^2 \left(d + 1\right) {\lVert \vlambda - \bar{\vlambda} \rVert}^2_2 \\
    &\quad\geq \frac{2 \mu^2 \left(d + 1\right)}{L} \left(\mathbb{E}{f\left(\vt_{\vlambda}\left(\rvvu\right)\right)} - \mathbb{E}{f\left(\vt_{\bar{\vlambda}}\left(\rvvu\right)\right)}\right).
\end{align*}
Thus, we have
\begin{align}
    \mathbb{E}{\norm{\nabla_{\vlambda} f\left(\vt_{\vlambda}\left(\rvvu\right)\right)}_2^2} 
    &\geq 
    \frac{2 \mu^2 \left(d + 1\right)}{L} \left(\mathbb{E}{f\left(\vt_{\vlambda}\left(\rvvu\right)\right)} - f^*\right).
    \label{eq:thm_lower_bound_eq2}
\end{align}

Now, from \cref{eq:thm_gradient_variance_general_definition},
\begin{align*}
\mathbb{E}\norm{ \vg_M }_2^2
  &= \frac{1}{M} \left(
        \mathbb{E}{ \norm{\nabla_{\vlambda} f\left(\vt_{\vlambda}\left(\rvvu\right)\right)}_2^2 }
        -
        \norm{\mathbb{E}{ \nabla_{\vlambda} f\left(\vt_{\vlambda}\left(\rvvu\right)\right)} }_2^2
    \right) \\
    &\qquad+ \norm{ \nabla F\left(\vlambda\right) }^2_2,
\shortintertext{applying \cref{eq:thm_lower_bound_eq1},}
  &\geq
  \frac{1}{M} 
  \left(
    \mathbb{E}{ \norm{\nabla_{\vlambda} f\left(\vt_{\vlambda}\left(\rvvu\right)\right)}_2^2 }
    -
    2 L^2 \left(\mathbb{E}{f\left(\vt_{\vlambda}\left(\rvvu\right)\right)} - f^*\right) 
  \right) \\
  &\qquad+ \norm{ \nabla F\left(\vlambda\right) }^2_2 
\shortintertext{applying \cref{eq:thm_lower_bound_eq2},}
  &\geq
  \frac{2 \mu^2 \left(d + 1\right) - 2L^2}{ML} 
  \left(\mathbb{E}{f\left(\vt_{\vlambda}\left(\rvvu\right)\right)} - f^*\right) \\
  &\qquad + \norm{ \nabla F\left(\vlambda\right) }^2_2 
  \\
  &\geq 
  \frac{2 \mu^2 \left(d + 1\right) - 2L^2}{ML} \left(F\left(\vlambda\right) - h\left(\vlambda\right) - f^*\right) \\
  &\qquad+ \norm{ \nabla F\left(\vlambda\right) }^2_2 
  \\
  &= \frac{2 \mu^2 \left(d + 1\right) - 2L^2}{ML} \left(F\left(\vlambda\right) - F^*\right) + \norm{ \nabla F\left(\vlambda\right) }^2_2  \\
  &\qquad+ \frac{2 \mu^2 \left(d + 1\right) - 2 L^2}{ML} \left(F^* - f^* - h\left(\vlambda\right) \right).
\end{align*}
The last term
\begin{align*}
    \frac{2 \mu^2 \left(d + 1\right) - 2 L^2}{ML} \left(F^* - f^* - h\left(\vlambda\right) \right)
\end{align*}
can be shown to be positive if $\vlambda$ is sufficiently close to the optimum.
Let $\vlambda^* = \argmin_{\vlambda} F\left(\vlambda\right)$ be the minimizer of $F$.
Then, we have
\begin{align*}
    F^* - f^* - h\left(\vlambda\right)  
    &= 
    \mathbb{E}{f\left(\vt_{\vlambda^*}\left(\vu\right)\right)} + h\left(\vlambda^*\right) - f^* - h\left(\vlambda\right) 
    \\
    &=
    \left(\mathbb{E}{f\left(\vt_{\vlambda^*}\left(\vu\right)\right)} - f^*\right) + \left(h\left(\vlambda^*\right) - h\left(\vlambda\right)\right),
\end{align*}
where the first term is strictly positive and the second term goes to zero as $\vlambda \to \vlambda^*$.
\end{proofEnd}

%%% Local Variables:
%%% TeX-master: "main"
%%% End:


\begin{remark}[\textbf{Matching Dimensional Dependence}]
  For well-conditioned problems such that \(\nicefrac{L}{\mu} < \sqrt{d+1}\), a lower bound of the same dimensional dependence with our upper bounds holds near the optimum. 
\end{remark}

\begin{remark}[\textbf{Unimprovability of the ABC Condition}]
  The lower bound suggests that the \(ABC\) gradient variance condition is unimprovable within the class of smooth, quadratically growing functions.
\end{remark}

%%% Local Variables:
%%% TeX-master: "main"
%%% End:

%
\subsection{Non-Smoothness of the ADVI Family}

For the bijection function \(\psi^{-1}: \mathbb{R} \mapsto \mathbb{R}_+\), we provide the following assumptions to verify establish the smoothness of some univariate constrained support distributions.
\begin{assumption}\label{assumption:smooth_bijector}
  \(\psi^{-1}\) is \(L_{\psi}\)-smooth.
\end{assumption}

\begin{assumption}\label{assumption:gamma1}
  \(
  \gamma_1\left(x\right) = \nicefrac{ {\left( \psi^{-1} \right)}^{\prime} \left(x\right) }{ \psi^{-1}\left(x\right) } 
  \)
  is \(L_1\)-Lipschitz.
\end{assumption}

\begin{assumption}\label{assumption:gamma2}
  \(
  \gamma_2\left(x\right) = \nicefrac{ {\left( \psi^{-1} \right)}^{\prime}\left(x\right) }{ {\left(\psi^{-1}\left(x\right)\right)}^2 } 
  \)
  is \(L_2\)-Lipschitz.
\end{assumption}

\begin{assumption}\label{assumption:gamma3}
  \(
  \gamma_3\left(x\right) = {\left(\psi^{-1} \right)}^{\prime}\left(x\right) \, \psi^{-1}\left(x\right) 
  \)
  is \(L_3\)-Lipschitz.
\end{assumption}

\begin{theoremEnd}[category=smoothness]{proposition}
  Let \(\psi^{-1} : \mathbb{R} \mapsto \mathbb{R}_+ \) be the exponential bijector \(\psi^{-1}\left(x\right) = e^x\). 
  Then, \(\psi^{-1}\) 
  \begin{enumerate}[label=(\roman*)]
    \item does not satisfy \cref{assumption:smooth_bijector},
    \item satisfies \cref{assumption:gamma1} for any \(L_1 > 0\),
    \item does not satisfy \cref{assumption:gamma2},
    \item does not satisfy \cref{assumption:gamma3}.
  \end{enumerate}
\end{theoremEnd}
\begin{proofEnd}
  Notice that
  \begin{align*}
    \gamma_1\left(x\right) &= \frac{ e^x }{ e^x } = 1 \\
    \gamma_2\left(x\right) &= \frac{ e^x }{ e^{2x} } = e^{-x} \\
    \gamma_3\left(x\right) &= e^x e^x = e^{2x}.
    \nonumber
  \end{align*}
  Therefore, 
  \begin{enumerate}[label=(\roman*)]
    \item holds trivially.
    \item holds trivially.
    \item The Lipschitzness of \(\gamma_2\left(x\right)\) is equivalent to whether \(e^x\) is Lipschitness, which it is not.
    \item The Lipschitzness of \(\gamma_3\left(x\right)\) is equivalent to whether \(e^{2x}\) is Lipschitness, which it is not.
  \end{enumerate}
\end{proofEnd}


\begin{theoremEnd}[category=smoothness]{proposition}[\textbf{Smoothness of Gamma}]
  Let \(f\left(x; \alpha, \beta\right)\) be the probability density function of the Gamma distribution with parameters \(\alpha, \beta\).
  Also, let \(\psi^{-1} : \mathbb{R} \mapsto \mathbb{R}_+\) be a bijection function.
  \(f \circ \psi^{-1}\) is \(L\)-smooth if and only if \cref{assumption:gamma1,assumption:smooth_bijector} hold, where the the smoothnes constant is \(L = \left(\alpha - 1\right) L_1 + \beta L_{\psi} \).
\end{theoremEnd}
\begin{proofEnd}
  First, the log probability density of the half-normal distribution is given as
  \begin{align}
    \log f\left(x; \sigma\right)
    =
    \left(\alpha - 1\right) \log x - \beta x
    \nonumber
  \end{align}
  where \(\log Z\) is a constant normalizer.
  The derivatives of the inverse gamma distribution with a bijection \(\psi^{-1}\) is given as
  \begin{align*}
    &\frac{d \log f\left(\psi^{-1}\left(x\right); \sigma\right)}{dx}
    \\
    &\;=
    \frac{d \log f\left(\psi^{-1}\left(x\right); \sigma\right)}{d\psi}
    \frac{d \psi^{-1}}{dx}
    \\
    &\;=
    \frac{d \psi^{-1}}{dx}
    \left(
    \left(\alpha - 1\right) \frac{1}{\psi^{-1}\left(x\right)} - \beta 
    \right)
    \\
    &\;=
    \left(\alpha - 1 \right) \frac{ {\left( \psi^{-1} \right)}^{\prime} \left(x\right) }{\psi^{-1}\left(x\right)} + \beta \psi^{-1}\left(x\right).
    \\
    &\;=
    \left(\alpha - 1 \right) \gamma_1\left(x\right)
    +
    \beta {\left( \psi^{-1} \right)}^{\prime} \left(x\right)
  \end{align*}
  Thus, the smoothness is euquivalent to the Lipschitzness of \(\gamma_3\left(x\right)\).

  By the triangle inequlity, the two-way inequality
  \begin{alignat*}{2}
    &\abs{
      \left(\alpha - 1 \right)
    \abs{
      \gamma_1\left(x\right) - \gamma_1\left(y\right)
    }
    -
    \beta
    \abs{
      {\left( \psi^{-1} \right)}^{\prime} \left(x\right)
      -
      {\left( \psi^{-1} \right)}^{\prime} \left(y\right)
    }
    }
    \\
    &\quad\leq
    \abs{
    \frac{d \log f\left(\psi^{-1}\left(x\right); \alpha, \beta\right)}{dx}
    -
    \frac{d \log f\left(\psi^{-1}\left(y\right); \alpha, \beta\right)}{dy}
    }
    \\
    &\quad\quad\leq
    \left( \alpha - 1 \right)
    \abs{
      \gamma_1\left(x\right) - \gamma_1\left(y\right)
    }
    +
    \beta 
    \abs{
      {\left( \psi^{-1} \right)}^{\prime} \left(x\right)
      -
      {\left( \psi^{-1} \right)}^{\prime} \left(y\right)
    }
  \end{alignat*}
  holds.
  The upper bound proves the implication (\(\Rightarrow\)), while the lower bounds (\(\Leftarrow\)) proves the converse by contraposition.
\end{proofEnd}

%%% Local Variables:
%%% TeX-master: "main"
%%% End:



\begin{theoremEnd}[category=smoothness]{proposition}[\textbf{Smoothness of Inverse Gamma}]\label{thm:inverse_gamma_smoothness}
  Let \(f\left(x; \alpha, \beta\right)\) be the probability density function of the inverse gamma distribution with parameters \(\alpha, \beta\).
  Also, let \(\psi^{-1} : \mathbb{R} \mapsto \mathbb{R}_+\) be a bijection function.
  \(f \circ \psi^{-1}\) is \(L\)-smooth if and only if \cref{assumption:gamma1,assumption:gamma2} hold, where the the smoothnes constant is \(L = L_1 \left(\alpha + 1\right) + L_2 \beta\).
\end{theoremEnd}
\begin{proofEnd}
  First, the log probability density of an inverse gamma distribution is given as
  \begin{align}
    \log f\left(x; \alpha, \beta\right)
    =
    \left(-\alpha - 1\right) \log x - \frac{\beta}{x}
    + 
    \log Z,
    \nonumber
  \end{align}
  where \(\log Z\) is a constant normalizer.
  The derivatives of the inverse gamma distribution with a bijection \(\psi^{-1}\) is given as
  \begin{align*}
    &\frac{d \log f\left(\psi^{-1}\left(x\right); \alpha, \beta\right)}{dx}
    \\
    &\;=
    \frac{d \log f\left(\psi^{-1}\left(x\right); \alpha, \beta\right)}{d\psi}
    \frac{d \psi^{-1}}{dx}
    \\
    &\;=
    \frac{d \psi^{-1}}{dx}
    \left(
    \left( - \alpha - 1 \right) \frac{1}{\psi^{-1}\left(x\right)} + \beta \frac{1}{{\psi^{-1}\left(x\right)}^2}
    \right)
    \\
    &\;=
    \left( - \alpha - 1 \right) \frac{ {\left( \psi^{-1} \right)}^{\prime} \left(x\right) }{\psi^{-1}\left(x\right)} + \beta \frac{ {\left( \psi^{-1} \right)}^{\prime} \left(x\right) }{{\psi^{-1}\left(x\right)}^2}.
    \\
    &\;=
    \left( - \alpha - 1 \right) \gamma_1\left(x\right) + \beta \gamma_2\left(x\right).
  \end{align*}

  By the triangle inequlity, the two-way inequality
  \begin{alignat*}{2}
    &\abs{
      \left(  \alpha + 1 \right)
    \abs{
      \gamma_1\left(x\right) - \gamma_1\left(y\right)
    }
    -
    \beta
    \abs{
      \gamma_2\left(x\right) - \gamma_2\left(y\right)
    }
    }
    \\
    &\quad\leq
    \abs{
    \frac{d \log f\left(\psi^{-1}\left(x\right); \alpha, \beta\right)}{dx}
    -
    \frac{d \log f\left(\psi^{-1}\left(y\right); \alpha, \beta\right)}{dy}
    }
    \\
    &\quad\quad\leq
    \left( \alpha + 1 \right)
    \abs{
      \gamma_1\left(x\right) - \gamma_1\left(y\right)
    }
    +
    \beta 
    \abs{
      \gamma_2\left(x\right) - \gamma_2\left(y\right)
    }
  \end{alignat*}
  holds.
  The upper bound proves the implication (\(\Rightarrow\)), while the lower bounds (\(\Leftarrow\)) proves the converse by contraposition.
\end{proofEnd}

%%% Local Variables:
%%% TeX-master: "master"
%%% End:



\begin{theoremEnd}[category=smoothness]{proposition}[\textbf{Smoothness of Half-Normal}]
  Let \(f\left(x; \sigma\right)\) be the probability density function of the half-normal distribution with parameter \(\sigma\).
  Also, let \(\psi^{-1} : \mathbb{R} \mapsto \mathbb{R}_+\) be a bijection function.
  \(f \circ \psi^{-1}\) is \(L\)-smooth if and only if \cref{assumption:gamma3} hold, where the the smoothnes constant is \(L = L_3 / \sigma^{2}\).
\end{theoremEnd}
\begin{proofEnd}
  First, the log probability density of the half-normal distribution is given as
  \begin{align}
    \log f\left(x; \sigma\right)
    =
    -\frac{x^2}{2\sigma^2}
    + 
    \log Z,
    \nonumber
  \end{align}
  where \(\log Z\) is a constant normalizer.
  The derivatives of the inverse gamma distribution with a bijection \(\psi^{-1}\) is given as
  \begin{align*}
    &\frac{d \log f\left(\psi^{-1}\left(x\right); \sigma\right)}{dx}
    \\
    &\;=
    \frac{d \log f\left(\psi^{-1}\left(x\right); \sigma\right)}{d\psi}
    \frac{d \psi^{-1}}{dx}
    \\
    &\;=
    \frac{d \psi^{-1}}{dx}
    \left(
    -\frac{ \psi^{-1}\left(x\right) }{\sigma^2}
    \right)
    \\
    &\;=
    -\frac{1}{\sigma^2} {\left( \psi^{-1} \right)}^{\prime}\left(x\right) \psi^{-1}\left(x\right)
    \\
    &\;=
    -\frac{1}{\sigma^2} \gamma_3\left(x\right)
  \end{align*}
  Thus, the smoothness is euquivalent to the Lipschitzness of \(\gamma_3\left(x\right)\).
\end{proofEnd}

%%% Local Variables:
%%% TeX-master: "main"
%%% End:


%\input{thm_beta_smoothness}

%%% Local Variables:
%%% TeX-master: "main"
%%% End:


\vspace{-1ex}
\section{Simulations}
\vspace{-.5ex}
We now evaluate our bounds and the insights gathered during the analysis through simulations.
We implemented a bare-bones implementation of BBVI in Julia~\citep{bezanson_julia_2017} with plain SGD.
The stepsize were manually tuned so that all problems converge at similar speeds.
For all problems, we use a unit Gaussian base distribution such that \(\varphi\left(u\right) = \mathcal{N}\left(u; 0, 1\right)\) resulting in a kurtosis of \(\kappa = 3\) and use \(M = 10\) Monte Carlo samples.

%% \begin{figure}[t]
%%   \centering
%%   \input{figures/fig_quadratic_softpluschol_generalbound.tex}
%%   \caption{
%%   }
%% \end{figure}

%% \begin{figure}[t]
%%   \centering
%%   \input{figures/fig_quadratic_softpluschol_boundedentropybound.tex}
%%   \caption{
%%   }
%% \end{figure}

%% \begin{figure}[t]
%%   \centering
%%   \input{figures/fig_quadratic_linearchol_generalbound.tex}
%%   \caption{
%%   }
%% \end{figure}

%% \begin{figure}[t]
%%   \centering
%%   \input{figures/fig_quadratic_linearchol_boundedentropybound.tex}
%%   \caption{
%%   }
%% \end{figure}

%% \begin{figure}[t]
%%   \centering
%%   \input{figures/fig_quadratic_squareroot_generalbound.tex}
%%   \caption{
%%   }
%% \end{figure}

\vspace{-.5ex}
\subsection{Synthetic Problem}\label{section:quadratic}
To test the \textit{ideal} tightness of the bounds, we consider quadratics achieving the tightest bound for the constants \(L_{\mathrm{H}}, L_{\mathrm{KL}}, \mu_{\mathrm{H}}, \mu_{\mathrm{KL}}\) given as
{%
\setlength{\belowdisplayskip}{1.ex} \setlength{\belowdisplayshortskip}{1.ex}%
\setlength{\abovedisplayskip}{1.ex} \setlength{\abovedisplayshortskip}{1.ex}%
\[
  \log \ell\left(\vx \mid \vz \right) = -\frac{N}{\sigma^2} \norm{ \vz - \vz^* }_2^2;\quad
  \log p\left(\vz \right)             = -\frac{1}{\lambda}  \norm{ \vz  }_2^2,
\]
}%
where \(N\) simulates the effect of the number of datapoints.
We set the constants as \(\sigma = 0.3\), \(\lambda = 8.0\), and \(N = 100\), the mode \(\vz^*\) is randomly sampled from a Gaussian, and the dimension of the problem is \(d = 20\).
For the bounded entropy case, we set \(S = 2.0\) (the true standard deviation is in the order of 1e-3).

\vspace{-.5ex}
\paragraph{Quality of Upper Bound}
The results for the Cholesky and mean-field parameterizations with a softplus bijector are shown in~\cref{fig:quadratic}.
For the Cholesky parameterization, the bulk of the looseness comes from the treatment of the regularization term (\textcolor{color3}{blue region}).
The remaining ``technical looseness'' (\textcolor{color1}{red region}) is relatively tight and can be shown to be tighter when using linear parameterizations (\(\phi\left(x\right) = x\)) and the square root parameterization, which is the tightest.
However, for the mean-field parameterization, despite the superior constants~(\cref{remark:meanfield_superiority}), there is still room for improvement.
Additional results for other parameterizations can be found in~\cref{section:additional_quadratic}.


\begin{figure}[t]
  \vspace{-1.5ex}
  \hspace{-1.5em}
  \subfloat{
    
\begin{tikzpicture}
\begin{axis}[
      legend style={
        legend image post style = {scale=0.5},
        at                      = {(0.5,1.5)},
        anchor                  = north,
        legend cell align       = left,
        line width              = 0.8pt,
        draw = none,
      },
      %
      tuftelike, 
      %
      xlabel style={yshift=-0.8ex},
      %
      axis line style = thick,
      every tick/.style={black,thick},
      %axis lines = left,
      %grids=both,
      ymode  = log,
      xmin   = 1,
      xmax   = 4000,
      xtick  = {1, 2000, 4000},
      ytick  = {1e+4, 1e+6, 1e+8, 1e+10},
      %
      ymin   = 1e+4,
      ymax   = 1e+10,
      %
      xlabel = {Iteration},
      %ylabel = {\(\mathbb{E}\norm{\rvvg}^2_2\)},
      tick label style={font=\small},  
      height = 4cm,
      width  = 4.3cm,
      axis on top,
    ]
    \addplot[name path=gvar, color1, mark=none, thick]
      table [x=t, y=gvar] {data/simulation/airfoil_softpluschol_generalbound.csv}
      %node[above=3pt,pos=0.8] {\scriptsize\(\mu_{\mathrm{KL}}, L_{\mathrm{H}}\) bound}
      coordinate [pos=0.95] (gvarcoord);
    \addlegendentry{\scriptsize Gradient Variance\;\; \( \mathbb{E}\norm{ \rvvg }_2^2 \)}

    \addplot[name path=ABC, color2, mark=none, thick]
      table [x=t, y=ABC] {data/simulation/airfoil_softpluschol_generalbound.csv}
      coordinate [pos=0.05] (ABCcoord1) coordinate [pos=0.95] (ABCcoord2);
    \addlegendentry{\scriptsize Upper Bound}

    %\addplot[color4, draw=none, mark=none, thick] {x};
    %\addlegendentry{\scriptsize\(2 A \left(\mathbb{E}_{q_{\vlambda}} f_{\mathrm{KL}} - f^*_{\mathrm{KL}} \right) + B \norm{\nabla F}_2^2 + C_{\Delta \zeta} {\lVert \bar{\zeta}_{\mathrm{KL}} - \bar{\zeta}_{\mathrm{H}} \rVert}^2_2 \)}

    \addplot[name path=opt, color3, mark=none, thick] %
      table [x=t, y=opt] {data/simulation/airfoil_softpluschol_generalbound.csv} %
      %node[above=3pt,pos=0.8] {\scriptsize\(\DKL{q_{\vlambda}}{p}\)}
      coordinate [pos=0.95] (optcoord);
    %\addlegendentry{\scriptsize\(2 A \left(\mathbb{E}_{q_{\vlambda}} f_{\mathrm{H}} - f^*_{\mathrm{H}} \right) + B \norm{\nabla F}_2^2\)}

    \addplot[name path=axis, domain=0:3990, fill=none, no markers, draw=none] {1e+4}
      coordinate [pos=0.05] (axiscoord);

    %\addplot[name path=ABC, color2, mark=none]
    %  table [x=t, y=ABC] {data/simulation/quadratic_softpluschol_generalbound.csv}
    %  ;

    \addplot[name path=C, fill=none, gray, mark=none]
      table [x=t, y=C] {data/simulation/airfoil_softpluschol_generalbound.csv}
      coordinate [pos=0.05] (Ccoord);

    \addplot[thick, color=color3, fill=color3, fill opacity=0.2] fill between[of=ABC and opt];
    \addplot[thick, color=color2, fill=color2, fill opacity=0.2] fill between[of=ABC and   C];

    \addplot[thick, color=color1, fill=color1, fill opacity=0.2] fill between[of=opt and gvar];
    \addplot[thick, color=gray,   fill=gray,   fill opacity=0.2] fill between[of=C   and axis];

    \draw[thick,color=color3,draw=none] (ABCcoord2) -- (optcoord)
      node[midway,xshift=-18pt] {}
      coordinate [pos=0.5] (DKLcoord);

    %\draw[thick,color=color1,latex-latex] (gvarcoord) -- (optcoord)
    %  %node[midway,xshift=-32pt] {\scriptsize\(\mu_{\mathrm{KL}}, L_{\mathrm{H}}, \Delta \vz^*\) bound}
    %  coordinate [pos=0.5] (muL_coord);

    \draw[thick,color=gray,latex-] (Ccoord) -- (axiscoord)
      node[pos=0.8,xshift=5pt] {\scriptsize\(C\) };

    \node[%
      pin={%
        [%
          pin distance=0.3cm,%
          pin edge={color3},%
          text=color3%
        ]100:{\scriptsize\(\DKL{q_{\vlambda}}{p}\)}},%
      inner sep=0pt,%
    ] at (DKLcoord) {};
\end{axis}
\end{tikzpicture}
\label{fig:airfoil_bound}
  }
  \subfloat{
    \hspace{-2em}
    
\tikzset{subcaptionstyle/.style={
    text width=2in,yshift=-3mm, align=center,anchor=north
}}

\begin{tikzpicture}[%
    /pgfplots/set layers,
    spy using outlines={circle, magnification=3, connect spies}%
  ]
  \begin{axis}[
      legend style={
        legend image post style={scale=0.5},
        at={(0.5,1.6)},
        anchor=north,
        legend cell align=left,
        line width=0.8pt,
        draw=none % Unterdrücke Box
      },
      tuftelike, 
      axis line style = thick,
      every tick/.style={black,thick},
      %axis lines = left,
      %grids=both,
      ymode  = log,
      xmin   = 1,
      xmax   = 1000,
      xtick  = {1, 500, 1000},
      %
      xlabel style={yshift=-0.8ex},
      %
      ymin   = 1e+4,
      ymax   = 1e+7,
      %
      xlabel = {Iteration},
      tick label style={font=\small},  
      ylabel = {\(\mathbb{E}\norm{\rvvg}^2_2\)},
      height = 4cm,
      width  = 4.3cm,
      axis on top,
    ]
    \addplot[color1, thick, mark=none] table[x=t, y=squareroot, mark=none] {data/replay_airfoil.csv};
    \addlegendentry{\scriptsize{Matrix square root}};

    \addplot[color2, thick, mark=none] table[x=t, y=linearcholesky, mark=none] {data/replay_airfoil.csv};
    \addlegendentry{\scriptsize{Cholesky \(\phi(x) = x\)}};

    %\addplot[color3, thick,  mark=none] table[x=t, y=expcholesky, mark=none] {data/replay_airfoil.csv};
    %\addlegendentry{\scriptsize{Cholesky \(\phi(x) = \exp\left(x\right)\)}};

    \addplot[color4, thick, mark=none] table[x=t, y=softpluscholesky, mark=none] {data/replay_airfoil.csv};
    \addlegendentry{\scriptsize{Cholesky \(\phi(x) = \mathrm{softplus}\left(x\right)\)}};

    % Square Root
    \addplot [name path=squarerootl, fill=none, draw=none, forget plot] table [x=t, y=squarerootl] {data/replay_airfoil.csv} \closedcycle;
    \addplot [name path=squarerootu, fill=none, draw=none, forget plot] table [x=t, y=squarerootu] {data/replay_airfoil.csv} \closedcycle;
    \addplot[color1!30] fill between[of=squarerootu and squarerootl];

    % Cholesky Linear
    \addplot [name path=linearcholeskyl, fill=none, draw=none, forget plot] table [x=t, y=linearcholeskyl] {data/replay_airfoil.csv} \closedcycle;
    \addplot [name path=linearcholeskyu, fill=none, draw=none, forget plot] table [x=t, y=linearcholeskyu] {data/replay_airfoil.csv} \closedcycle;
    \addplot[color2!40] fill between[of=linearcholeskyu and linearcholeskyl];

    % Cholesky Exp
%%     \addplot [name path=expcholeskyl, fill=none, draw=none, forget plot] table [x=t, y=expcholeskyl] {data/replay_airfoil.csv} \closedcycle;
%%     \addplot [name path=expcholeskyu, fill=none, draw=none, forget plot] table [x=t, y=expcholeskyu] {data/replay_airfoil.csv} \closedcycle;
%%     \addplot[color3!40] fill between[of=expcholeskyu and expcholeskyl];

    % Cholesky Softplus
    \addplot [name path=softpluscholeskyl, fill=none, draw=none, forget plot] table [x=t, y=softpluscholeskyl] {data/replay_airfoil.csv} \closedcycle;
    \addplot [name path=softpluscholeskyu, fill=none, draw=none, forget plot] table [x=t, y=softpluscholeskyu] {data/replay_airfoil.csv} \closedcycle;
    \addplot[color4!40] fill between[of=softpluscholeskyu and softpluscholeskyl];

    \coordinate (spypoint)     at (axis cs:280,1e+6);
    \coordinate (magnifyglass) at (axis cs:800,1.5e+6);
    \begin{scope}
      \spy [black, size=1.5cm] on (spypoint) in node[fill=white] at (magnifyglass);
    \end{scope}
  \end{axis}
\end{tikzpicture}
\label{fig:airfoil_parameterizations}
  }
  \vspace{-1.0ex}
  \caption{
    \textbf{
      Linear regression on the \textsc{Airfoil} dataset.
      (\textsf{left}) Evaluation of the upper bound (\cref{thm:gradient_upper_bound}).
      (\textsf{right}) Comparison of the variance of different parameterizations resulting in the same \(\vm\), \(\mC\).
    }
  }
  \vspace{-3.0ex}
\end{figure}

\vspace{-.5ex}
\subsection{Real Dataset}\label{section:linearreg}
\vspace{-.5ex}
\paragraph{Model}
We now evaluate the theoretical results with real datasets.
Given a regression dataset \((\mX, \vy)\), we use the linear Gaussian model 
{%
\setlength{\belowdisplayskip}{1.ex} \setlength{\belowdisplayshortskip}{1.ex}%
\setlength{\abovedisplayskip}{1.ex} \setlength{\abovedisplayshortskip}{1.ex}%
\[
  y   \sim \mathcal{N}\left(\mX \vw, \sigma^2\right);\quad
  \vw \sim \mathcal{N}\left(\mathbf{0}, \lambda \boldupright{I}\right),
\]
}%
where \(\lambda\) and \(\sigma\) are hyperparameters.
The smoothness and quadratic growth constants for this model are given as the max- and minimum eigenvalues of \(\sigma^{-2} \mX^{\top} \mX + \lambda^{-1} \boldupright{I}\) (for \(f_{\text{H}}\)) and \(\sigma^{-2} \mX^{\top} \mX\) (for \(f_{\text{KL}}\)).
\(f_{\text{KL}}^*\) and \(f_{\text{H}}^*\) are given as the mode of the likelihood and the posterior, while \(F^*\) is the negative marginal log-likelihood.

\vspace{-.5ex}
\paragraph{Quality of Upper Bound}
\cref{fig:airfoil_bound} shows the result on the \textsc{Airfoil} dataset~\citep{Dua:2019}.
The constants are \(L_{\mathrm{H}} = 3.520 \times 10^4, \mu_{\mathrm{KL}}=2.909 \times 10^3\).
Due to poor conditioning, the bound is much looser compared to the quadratic case.
We note that generalizing our bounds to utilize matrix smoothness and matrix-quadratic growth as done by \citep{domke_provable_2019} would tighten the bounds.
But the theoretical gains would be marginal.
Detailed information about the datasets and additional results for other parameterizations can be found in~\cref{section:additional_linearreg}.

\vspace{-.5ex}
\paragraph{Comparison of Parameterizations}
\cref{fig:airfoil_parameterizations} compares the gradient variance resulting from the different parameterizations.
For a fair comparison, the gradient is estimated on the \(\vlambda\) that results in the same \(\vm, \mC\) for all three parameterizations.
%The nonlinear parameterizations result in the lowest variance.
%Furthermore, the matrix square root parameterization results in the highest variance.
This shows the gradual increase in variance by
\begin{enumerate*}[label=\textbf{(\roman*)}]
  \item not using a nonlinear conditioner (linear Cholesky)
  \item and increasing the number of variational parameters (matrix square root).
\end{enumerate*}

%%% Local Variables:
%%% TeX-master: "main"
%%% End:


\section{Related Works}
\vspace{-.5ex}
\paragraph{Controlling Gradient Variance}
The main algorithmic challenge in BBVI is to control the gradient noise~\cite{ranganath_black_2014}.
This has led to various methods for reducing the variance of VI gradient estimators using control variates~\citep{ranganath_black_2014,miller_reducing_2017,geffner_using_2018}, ensembling of estimators~\citep{geffner_rule_2020}, modifying the differentiation procedure~\citep{roeder_sticking_2017}, quasi-Monte Carlo~\citep{buchholz_quasimonte_2018, liu_quasimonte_2021}, and multilevel Monte Carlo~\citep{fujisawa_multilevel_2021}.
Cultivating a deeper understanding of the properties of gradient variance could further extend this list.

\vspace{-2ex}
\paragraph{Convergence Guarantees}
Obtaining full convergence guarantees has been an important task for understanding BBVI algorithms.
However, most guarantees so far have relied on strong assumptions such as that the log-likelihood is Lipschitz~\citep{cherief-abdellatif_generalization_2019,alquier_nonexponentially_2021}, that the gradient variance is bounded by constant~\citep{liu_quasimonte_2021,buchholz_quasimonte_2018,domke_provable_2020,hoffman_blackbox_2020}, and that the support of \(q_{\vlambda}\) is bounded~\citep{fujisawa_multilevel_2021}.
Our result shows that similar results can be obtained under relaxed assumptions.
Meanwhile,~\citet{bhatia_statistical_2022} have recently proven a full complexity guarantee for a variant of BBVI.
%\textcolor{blue}{
But similarly to \citet{hoffman_blackbox_2020}, they only optimize the scale matrix \(\mC\), and the specifics of the algorithm diverge from the usual BBVI implementations as it uses the stochastic power iterations instead of SGD.
%}

\vspace{-1ex}
\paragraph{Gradient Variance Guarantees}
Studying the actual gradient variance properties of BBVI has only started to make progress recently.
~\citet{fan_fast_2015} first provided bounds by assuming the log-likelihood to be Lipschitz.
Under more general conditions,~\citet{domke_provable_2019} provided tight bounds for smooth log-likelihoods, which our work builds upon.
\citeauthor{domke_provable_2019}'s result can also be seen as a direct generalization of the results of~\citet{xu_variance_2019}, which are restricted to quadratic log-likelihoods and the mean-field family.
Lastly,~\citet{mohamed_monte_2020} provides a conceptual evaluation of gradient estimators used in BBVI.

%%% Local Variables:
%%% TeX-master: "main"
%%% End:


\vspace{-.5ex}
\section{Discussions}
\vspace{-.5ex}
In this work, we have proven upper bounds on the gradient variance of BBVI with the location-scale family for smooth, quadratically-growing log-likelihoods.
Specifically, we have provided bounds for both the ELBO in entropy-regularized and KL-regularized forms.
Our guarantees work without a single modification to the algorithms used in practice, although stronger assumptions establish a tighter bound for the entropy-regularized form ELBO.

\vspace{-1ex}
\paragraph{Limitations}
Our results have the following limitations:
\begin{enumerate*}
    \item[\ding{182}] Our results only apply to smooth and quadratically-growing log likelihoods and
    \item[\ding{183}] the location-scale ADVI family. Also, 
    \item[\ding{184}] our bounds cannot distinguish the variance of the Cholesky and matrix square root parameterizations, 
    \item[\ding{185}] and empirically, the bounds for the mean-field parameterization appear loose. Furthermore, 
    \item[\ding{186}] our results only work with 1-Lipschitz diagonal conditioners such as the softplus function.
\end{enumerate*}
In practice, non-Lipschitz conditioners such as the exponential functions are widely used.
While obtaining similar bounds with such conditioners would be challenging, constructing a theoretical framework that extends to such would be an important future research direction.
 
%% \paragraph{Signal-to-Noise Ratio}
%% To empirically quantify variance in VI, the gradient signal-to-noise ratio metric
%% \begin{align*}
%%   \mathrm{SNR}\left(\rvvg\right) =
%%   \frac{ \norm{\mathbb{E}\rvvg}^2_2 }{ \mathbb{E}\norm{\rvvg}^2_2 }
%%   =
%%   \frac{ \norm{ \nabla F }^2_2 }{ \mathrm{tr} \mathbb{V}\rvvg + \norm{ \nabla F }^2_2  } \leq 1,
%% \end{align*}
%% where the last equality holds since \(\rvvg\) is unbiased, has recently seen use~\citep{pmlr-v139-geffner21a, rainforth_tighter_2018, fujisawa_multilevel_2021}.
%% This metric can be though as quantifying the relative magnitude of the gradient noise with respect to the true gradient.
%% Under the ABC condition, as discussed in~\cref{section:abc}, each of the \(A\) and \(C\) term contribute to the convergence speed (\(A\)) and radius fo the stationary region (\(C\)), which will be unique to different estimators.
%% However, when using the SNR ratio for comparing different estimatosr, it is not possible to understand the effect of each term.
%% Thus, if one would go with the the ABC condition, analyzing the contribution of each term separately would be more informative.

%Under the ESG condition, the SNR is provides an estimate of the lower bound of \(B\).
%However, the ESG condition only holds for the interpolation regime where \(\mathbb{E}\norm{\rvvg}_2^2\).
%Thus, under the ABC condition, it is unsure how the SNR relates to the convergence aspects of SGD.

%% \begin{proposition}
%%   Let \(F\) be \(L\)-smooth and convex.
%%   Also, let \(\mathrm{SNR}\left(\rvvg\left(\vlambda\right)\right) > 0\) be the gradient SNR at \(\vlambda\).
%%   Then, 
%%   \[
%%      \mathbb{E}\norm{\rvvg\left(\vlambda\right)}_2^2
%%      \leq 
%%      L \left(\frac{1}{\mathrm{SNR}\left(\rvvg\left(\vlambda\right)\right)} - 1\right)
%%      \left(F\left(\vlambda\right) - F^*\right)
%%      +
%%      \norm{\mathbb{E} \rvvg }_2^2,
%%   \]
%%   where \(F^* = F\left(\vlambda^*\right)\) for the global minimum \(\vlambda^*\).
%% \end{proposition}
%% \begin{proof}
%%   From smoothness and convexity, it follows that
%%   \begin{align}
%%      \norm{ \nabla F\left(\vlambda\right) }_2^2 
%%        \leq
%%        L \inner{\nabla F\left(\vlambda\right)}{\vlambda - \bar{\vlambda}}
%%        \leq
%%       L \left( F\left(\vlambda\right) - F^* \right),\label{eq:snr_bound_eq1}
%%   \end{align}
%%   where the first inequality uses the co-coercivity of Lipschitz gradients, the last bound is the definition of convexity.
%%   From the definition of the SNR,
%%   \begin{align*}
%%     \mathbb{E}\norm{\rvvg\left(\vlambda\right)}_2^2 
%%     &=
%%     \frac{1}{\mathrm{SNR}\left(\rvvg\left(\vlambda\right)\right)} 
%%     \norm{ \nabla F\left(\vlambda\right)  }_2^2
%%     \\
%%     &=
%%     \left(
%%     \frac{1}{\mathrm{SNR}\left(\rvvg\left(\vlambda\right)\right)} 
%%     - 1
%%     \right)
%%     \norm{ \nabla F\left(\vlambda\right)  }_2^2
%%     +
%%     \norm{ \nabla F\left(\vlambda\right)  }_2^2.
%%   \end{align*}
%%   Plugging \cref{eq:snr_bound_eq1} to the first \( \norm{ \nabla F\left(\vlambda\right)  }_2^2 \) yields the result.
%% \end{proof}

%% \begin{remark}
%%   Let an unbiased estimator of \(\nabla F\), \(\rvvg\), satisfy \cref{assumption:abc} with \(B=1\).
%%   Then, the gradient SNR is bounded as
%%   \[
%%     \frac{ \norm{\nabla F\left(\vlambda\right)}_2^2 }{
%%       2 A \left( F\left(\vlambda\right) - F^* \right)
%%       +
%%       \norm{ \nabla F\left(\vlambda\right) }_2^2
%%       +
%%       C
%%     }
%%     \leq 
%%     \mathrm{SNR}\left(\rvvg\left(\vlambda\right)\right)
%%   \]
%%   where \(F^* = \inf_{\vlambda \in \mathbb{R}^p}\).
%% \end{remark}

%% \(A\) and \(C\) are determined by the unique properties of the estimator.
%% Thus, comparing the SNR of different estimators is an indirect way to compare the constants.

\vspace{-1ex}
\paragraph{Interpolation Property}
Our lower bound shows that some problems will not have the interpolation property.
For these types of problems,~\citet{zhang_adam_2022} have shown that ADAM~\citep{kingma_adam_2015} diverges with a broad range of stepsizes.
This strictly differs from overparameterized deep neural networks where the interpolation property is reasonable.
It is thus possible that some of the current optimization practices may not be optimal for BBVI, pointing towards new research directions.

%%% Local Variables:
%%% TeX-master: "main"
%%% End:


%
\begin{theorem}
  %%Let \(f\left(\vz\right) = \left(\vz - \vm\right) \mSiga^{-1} \left(\vz - \vm\right) \)
  For some functions \(\mathcal{L}\) defined as \(\mathcal{L} = \mathbb{E} f\left(\vt_{\vw}\left(\rvvu\right)\right) \), where \(\vt\) is the reparameterization estimator, the variance of their \(M\)-sample gradient estimator is bounded below as
  \begin{align}
    \mathbb{E}\norm{\vg}_2^2
    \geq
    2 A \left( \mathcal{L}\left(\vw\right) - \mathcal{L}\left(\vw^*\right) \right)
    + B \norm{ \nabla \mathcal{L} }_2^2
  \end{align}
  for some positive finite constant \(A\) and \(B\).
\end{theorem}
\begin{proof}
  \begin{alignat*}{2}
    \mathcal{L}\left(w\right)
    &= \mathbb{E} f\left(\vt_{\vw}\left(\rvvu\right)\right)
    \\
    &=
    \frac{1}{2} {\left(\vm - \vmu\right)}^{\top} \mSigma^{-1} \left(\vm - \vmu\right) + \frac{1}{2} \mathrm{tr}\left(\mSigma \mC \mC^{\top} \right)
  \end{alignat*}
  Clearly, this function is minimized with \(\vm = \vmu^*\) and \(\mC^* \mC^{* \top} = \mSigma \) such that \(\vw^* = \left(\vm^*, \mSigma^*\right)\) and \(\mathcal{L}\left(\vw^*\right) = 0\).
  Therefore, the suboptimality gap is simply
  \begin{alignat*}{2}
    \mathcal{L}\left(\vw\right) - \mathcal{L}\left(\vw^*\right)
    &=
    \mathcal{L}\left(\vw\right)
    \\
    &=
    \frac{1}{2} {\left(\vm - \vmu\right)}^{\top} \mSigma^{-1} \left(\vm - \vmu\right) + \frac{1}{2} \mathrm{tr}\left(\mSigma^{-1} \mC \mC^{\top} \right).
  \end{alignat*}
  The second moment of the \(M\)-sample gradient can be decomposed as 
  \begin{alignat*}{2}
    \mathbb{E} \norm{\vg}_2^2
    &=
    \frac{1}{M} \mathbb{E} \norm{ \nabla_{\vw} f\left( \vt_{\vw}\left(\rvvu\right) \right)  }_2^2
    + \frac{M-1}{M} \norm{ \nabla \mathcal{L} }_2^2
    \\
    &=
    \frac{1}{M} \mathbb{E} \norm{ \nabla f\left( \vt_{\vw}\left(\rvvu\right) \right)  }_2^2 \left( 1 + \norm{\rvvu}_2^2 \right)
    + \frac{M-1}{M} \norm{ \nabla \mathcal{L} }_2^2.
  \end{alignat*}
  Since the gradient of \(f\) is \(\nabla f = \mSigma^{-1} \left( \vt_{\vw}\left(\rvvu\right) - \vmu \right)\), 
  \begin{alignat*}{2}
    &\mathbb{E}\norm{ \nabla f\left( \vt_{\vw}\left(\rvvu\right) \right) }_2^2 \left( 1 + \norm{\rvvu}_2^2 \right)
    \\
    &\;=
    \mathbb{E}
    \norm{
      \mSigma^{-1} \left( \vt_{\vw}\left(\rvvu\right) - \vmu \right)
    }_2^2 \left( 1 + \norm{\rvvu}_2^2 \right)
    \\
    &\;=
    \mathbb{E}
    {\vt_{\vw}\left(\rvvu\right)}^{\top} \mSigma^{-1} \vt_{\vw}\left(\rvvu\right)
    \left( 1 + \norm{\rvvu}_2^2 \right)
    \\
    &\quad -
    2 
    {\vmu}^{\top} \mSigma^{-1} \mathbb{E} \vt_{\vw}\left(\rvvu\right)
    \left( 1 + \norm{\rvvu}_2^2 \right)
    \\
    &\quad +
    \vmu^{\top} \mSigma^{-1} \vmu \,
    \mathbb{E} \left( 1 + \norm{\rvvu}_2^2 \right)
  \end{alignat*}

  \begin{alignat*}{2}
    &\mathbb{E}{\vt_{\vw}\left(\rvvu\right)}^{\top} \mSigma^{-1} \vt_{\vw}\left(\rvvu\right)
    \left( 1 + \norm{\rvvu}_2^2 \right)
    \\
    &\;=
    \mathbb{E}{\left( \mC + \vm \right)}^{\top} \mSigma^{-1} {\left( \mC + \vm \right)}
    \left( 1 + \norm{\rvvu}_2^2 \right)
    \\
    &\;=
    \mathbb{E}\rvvu^{\top} \mC^{\top} \mSigma^{-1} \mC \rvvu  \left( 1 + \norm{\rvvu}_2^2 \right)
    \\
    &\quad+ \mathbb{E}2 \, \vm^{\top} \mSigma^{-1} \rvvu \left( 1 + \norm{\rvvu}_2^2 \right)
    \\
    &\quad+ \mathbb{E}\vm^{\top} \mSigma^{-1} \vm \left( 1 + \norm{\rvvu}_2^2 \right)
  \end{alignat*}

  \begin{alignat*}{2}
    \mathbb{E} \vt_{\vw}\left(\rvvu\right)
    \left( 1 + \norm{\rvvu}_2^2 \right)
    &=
    \mathbb{E}\left( \mC \rvvu + \vm \right) \left(1 + \norm{\rvvu}^2_2\right)
    \\
    &=
    \left( \mC \mathbb{E} \rvvu \left(1 + \norm{\rvvu}^2_2\right) \right) + \vm \mathbb{E} \left(1 + \norm{\rvvu}^2_2\right) 
    \\
    &=
    \left( \mC \, \mathbf{1} \mathbb{E}\rvu^3 \right) + \left(d + 1\right) \vm  
    \\
    &=
    \left(d + 1\right) \vm  
  \end{alignat*}

  \begin{alignat*}{2}
    &
    \mathbb{E} \mathrm{tr}\left( \rvvu^{\top} \mC^{\top} \mSigma^{-1} \mC \rvvu \left( 1 + \norm{\rvvu}_2^2 \right) \right)  
    \\
    &\;=
    \mathbb{E} \mathrm{tr}\left( \rvvu^{\top} \mC^{\top} \mSigma^{-1} \mC \rvvu \left( 1 + \norm{\rvvu}_2^2 \right) \right)  
    \\
    &\;=
    \mathbb{E} \mathrm{tr}\left( \mC^{\top} \mSigma^{-1} \mC \rvvu \rvvu^{\top}  \right)  
    +  \mathbb{E} \mathrm{tr}\left( \mC^{\top} \mSigma^{-1} \mC \rvvu \rvvu^{\top} \rvvu \rvvu^{\top}  \right)
    \\
    &\;=
    \mathbb{E} \mathrm{tr}\left( \mC^{\top} \mSigma^{-1} \mC \rvvu \rvvu^{\top} \right)  
    +  \mathbb{E} \mathrm{tr}\left( \mC^{\top} \mSigma^{-1} \mC \rvvu \rvvu^{\top} \rvvu \rvvu^{\top}  \right)
    \\
    &\;=
    \mathrm{tr}\left( \mC^{\top} \mSigma^{-1} \mC \mathbb{E} \rvvu \rvvu^{\top} \right)  
    + \mathrm{tr}\left( \mC^{\top} \mSigma^{-1} \mC \, \mathbb{E} \rvvu \rvvu^{\top} \rvvu \rvvu^{\top}  \right)
    \\
    &\;=
    \mathrm{tr}\left( \mC^{\top} \mSigma^{-1} \mC \mI \right)  
    + \mathrm{tr}\left( \mC^{\top} \mSigma^{-1} \mC \left(\left(d-1\right) {\left(\mathbb{E} \rvu^2\right)}^2 + \mathbb{E}\rvu^4 \right)  \right)
    \\
    &\;=
    \mathrm{tr}\left( \mC^{\top} \mSigma^{-1} \mC\right)  
    + \left(\left(d-1\right) {\left(\mathbb{E} \rvu^2\right)}^2 + \mathbb{E}\rvu^4 \right) \mathrm{tr}\left( \mC^{\top} \mSigma^{-1} \mC   \right)
    \\
    &\;=
    \mathrm{tr}\left( \mC^{\top} \mSigma^{-1} \mC\right) 
    + \left(\left(d-1\right) + \kappa \right) \mathrm{tr}\left( \mC^{\top} \mSigma^{-1} \mC   \right)
    \\
    &\;=
    \left(d + \kappa \right) \mathrm{tr}\left( \mSigma^{-1} \mC \mC^{\top} \right)
  \end{alignat*}

  \begin{alignat*}{2}
    \mathbb{E}\vm^{\top} \mSigma^{-1} \rvvu \left( 1 + \norm{\rvvu}_2^2 \right)
    =
    \vm^{\top} \mSigma^{-1} \mathbf{1} \, \mathbb{E} \rvu^3 
    = 0
  \end{alignat*}

  Combining all this,
  \begin{alignat*}{2}
    &\mathbb{E}
    {\vt_{\vw}\left(\rvvu\right)}^{\top} \mSigma^{-1} \vt_{\vw}\left(\rvvu\right)
    \left( 1 + \norm{\rvvu}_2^2 \right)
    \\
    &\;=
    \mathbb{E}\rvvu^{\top} \mC^{\top} \mSigma^{-1} \mC \rvvu  \left( 1 + \norm{\rvvu}_2^2 \right)
    \\
    &\quad+ \mathbb{E}2 \, \vm^{\top} \mSigma^{-1} \rvvu \left( 1 + \norm{\rvvu}_2^2 \right)
    \\
    &\quad+ \mathbb{E}\vm^{\top} \mSigma^{-1} \vm \left( 1 + \norm{\rvvu}_2^2 \right)
    \\
    &\;=
    \left(d + \kappa \right) \mathrm{tr}\left( \mSigma^{-1} \mC \mC^{\top} \right)
    +
    \left( d + 1 \right) \vm^{\top} \mSigma^{-1} \vm.
  \end{alignat*}

  \begin{alignat*}{2}
    \mathbb{E}\vm^{\top} \mSigma^{-1} \vm \left( 1 + \norm{\rvvu}_2^2 \right)
    = \left( d + 1 \right) \vm^{\top} \mSigma^{-1} \vm
  \end{alignat*}

  \begin{alignat*}{2}
    &\mathbb{E}\norm{ \nabla f\left( \vt_{\vw}\left(\rvvu\right) \right) }_2^2 \left( 1 + \norm{\rvvu}_2^2 \right)
    \\
    &\;=
    \mathbb{E}
    {\vt_{\vw}\left(\rvvu\right)}^{\top} \mSigma^{-1} \vt_{\vw}\left(\rvvu\right)
    \left( 1 + \norm{\rvvu}_2^2 \right)
    \\
    &\quad -
    2 
    {\vmu}^{\top} \mSigma^{-1} \mathbb{E} \vt_{\vw}\left(\rvvu\right)
    \left( 1 + \norm{\rvvu}_2^2 \right)
    \\
    &\quad +
    \vmu^{\top} \mSigma^{-1} \vmu \,
    \mathbb{E} \left( 1 + \norm{\rvvu}_2^2 \right)
    \\
    &\;=
    \left(d + \kappa \right) \mathrm{tr}\left( \mSigma^{-1} \mC \mC^{\top} \right)
    +
    \left( d + 1 \right) \vm^{\top} \mSigma^{-1} \vm
    \\
    &\quad -
    \left(d + 1\right) 2 
    {\vmu}^{\top} \mSigma^{-1}  \vm 
    \\
    &\quad +
    \left(d + 1\right) \vmu^{\top} \mSigma^{-1} \vmu \,
    \\
    &\;\geq
    \left(d + 1 \right) \mathrm{tr}\left( \mSigma^{-1} \mC \mC^{\top} \right)
    +
    \left( d + 1 \right) \vm^{\top} \mSigma^{-1} \vm
    \\
    &\quad -
    \left(d + 1\right) 2 
    {\vmu}^{\top} \mSigma^{-1}  \vm 
    \\
    &\quad +
    \left(d + 1\right) \vmu^{\top} \mSigma^{-1} \vmu \,
    \\
    &\;=
    \left(d + 1 \right)
    \left(
    \mathrm{tr}\left( \mSigma^{-1} \mC \mC^{\top} \right)
    +
    \vm^{\top} \mSigma^{-1} \vm
    -
    2 {\vmu}^{\top} \mSigma^{-1}  \vm 
    +
    \vmu^{\top} \mSigma^{-1} \vmu
    \right)
    \\
    &\;=
    \left(d + 1 \right)
    \left(
    \mathrm{tr}\left( \mSigma^{-1} \mC \mC^{\top} \right)
    +
    {\left(\vm - \vmu\right)}^{\top} \mSigma^{-1} \left(\vm - \vmu\right)
    \right)
    \\
    &\;=
    2 \left( d + 1 \right) \left( \mathcal{L}\left(\vw\right) - \mathcal{L}\left(\vw^*\right) \right).
  \end{alignat*}

  \begin{alignat*}{2}
    \mathbb{E}\norm{\vg}_2^2
    &=
    \frac{1}{M} \mathbb{E} \norm{ \nabla f\left( \vt_{\vw}\left(\rvvu\right) \right)  }_2^2 \left( 1 + \norm{\rvvu}_2^2 \right)
    + \frac{M-1}{M} \norm{ \nabla \mathcal{L} }_2^2
    \\
    &\;\geq
    2 \left( d + 1 \right) \left( \mathcal{L}\left(\vw\right) - \mathcal{L}\left(\vw^*\right) \right)
    + \frac{M-1}{M} \norm{ \nabla \mathcal{L} }_2^2,
  \end{alignat*}
  where the statement follows from \(A = d + 1\) and \(B = \left(M-1\right)/M\).

\end{proof}

%%% Local Variables:
%%% TeX-master: "master"
%%% End:


\section*{Acknowledgements}
This work was supported by NSF award IIS-2145644.

\clearpage
\bibliography{references}
\bibliographystyle{icml2023}

%%%%%%%%%%%%%%%%%%%%%%%%%%%%%%%%%%%%%%%%%%%%%%%%%%%%%%%%%%%%%%%%%%%%%%%%%%%%%%%
%%%%%%%%%%%%%%%%%%%%%%%%%%%%%%%%%%%%%%%%%%%%%%%%%%%%%%%%%%%%%%%%%%%%%%%%%%%%%%%
% APPENDIX
%%%%%%%%%%%%%%%%%%%%%%%%%%%%%%%%%%%%%%%%%%%%%%%%%%%%%%%%%%%%%%%%%%%%%%%%%%%%%%%
%%%%%%%%%%%%%%%%%%%%%%%%%%%%%%%%%%%%%%%%%%%%%%%%%%%%%%%%%%%%%%%%%%%%%%%%%%%%%%%
\clearpage
\appendix
\onecolumn

%\doparttoc % Tell to minitoc to generate a toc for the parts
%\faketableofcontents % Run a fake tableofcontents command for the partocs
%\part{} % Start the document part
%\parttoc % Insert the document TOC

%\addcontentsline{toc}{section}{Appendix} % Add the appendix text to the document TOC
%\part{Appendix} % Start the appendix part
%\parttoc % Insert the appendix TOC
%\tableofcontents

\clearpage

{\hypersetup{linkbordercolor=black,linkcolor=black}
\tableofcontents
}


{\hypersetup{linkbordercolor=black,linkcolor=black}
\section{Detailed Comparison Against \citealt*{domke_provable_2019}}
}%
Under the assumption that \(f_{\mathrm{H}}\) is \(L_{\mathrm{H}}\)-smooth and the linear full-rank Cholesky parameterization, under our notation, \citep{domke_provable_2019} prove the following bound:
%
\begin{equation*}
  \mathbb{E} \norm{\rvvg_{1}}_2^2
  \leq 
  L_{\mathrm{H}}^2 \left( (d+1) {\lVert \vm - \bar{\vzeta}_{\mathrm{H}} \rVert}_2^2 + (d+\kappa) \norm{\mC}_{\mathrm{F}}^2 \right).
  \label{eq:domke_bound}
\end{equation*}
We extend \citeauthor{domke_provable_2019}'s analysis in three original directions.

\paragraph{\ding{182} Generalization to Nonlinear Parameterizations}
First, we generalize the bounds to support nonlinear parameterizations.
In particular, \cref{thm:general_variational_gradient_norm_identity} and \cref{thm:general_variational_gradient_norm_bound} generalize Lemma 1 of \citet{domke_provable_2019} to 1-Lipschitz nonlinear conditioners.
From here, the analysis become identical to \citeauthor{domke_provable_2019}'s setup, until we reach our original analysis we discuss in Item \ding{184}.

\paragraph{\ding{183} Tighter Bound for the Mean-Field Parameterization}
Second, for the mean-field parameterization, we prove a tighter bound such that 
\[
  \mathbb{E} \norm{\rvvg_{1}}_2^2
  \leq 
  L_{\mathrm{H}}^2 \left( ( \sqrt{d \kappa} + \kappa \sqrt{d} + 1 ) {\lVert \vm - \bar{\vzeta}_{\mathrm{H}} \rVert}_2^2 + (2 \kappa \sqrt{d} + 1) \norm{\mC}_{\mathrm{F}}^2 \right),
\]
as a direct consequence of \cref{thm:u_identities}.

\paragraph{\ding{184} Connecting with the ABC Condition}
Furthermore, we extend the bounds above and establish the ABC condition (\cref{assumption:abc}) for the ELBO, through the quadratic function growth condition (\cref{def:quadratic_growth}).
Specifically, in our proof of \cref{thm:gradient_upper_bound}, the derivation past \cref{eq:thm1_fullrank} is original.

\clearpage

\section{Additional Simulation Results}

\vspace{-1ex}
\subsection{Synthetic Problem}\label{section:additional_quadratic}
We provide additional results for the simulations with quadratics in~\cref{section:quadratic}.

\begin{figure}[H]
  \vspace{-2ex}
  \centering
  \input{figures/fig_quadratic_appendix.tex}
  \vspace{-2ex}
  \caption{
    \textbf{Evaluation of the bounds for a perfectly conditioned quadratic target.}
    The \textcolor{color3}{blue regions} are the loosenesses resulting from either using (\cref{thm:gradient_upper_bound}) or not using (\cref{thm:gradient_upper_bound_bounded_entropy}) the bounded entropy assumption (\cref{assumption:bounded_entropy}), while the \textcolor{color1}{red regions} are the remaining ``technical loosesnesses.''
    The gradient variance was estimated from \(10^3\) samples.
  }\label{fig:quadratic_add}
\end{figure}

\vspace{-1.5ex}
\subsection{Real Datasets}\label{section:additional_linearreg}
We provide detailed information and additional results for the linear regression problem in~\cref{section:linearreg}.
The constants for the linear regression datasets are shown in \cref{table:datasets}, while additonal results for the nonlinear Cholesky (\cref{fig:linearreg_softpluschol}), linear Cholesky (\cref{fig:linearreg_linearchol}), nonlinear mean-field (\cref{fig:linearreg_softplusmf}), and matrix square root (\cref{fig:linearreg_squareroot}) parameterizations are displayed.


{\hypersetup{linkbordercolor=black,linkcolor=black}
\begin{table*}[ht]
  \vspace{-2ex}
\caption{Properties of the Linear Regression Datasets}\label{table:datasets}
\begin{center}
  {\small
\begin{threeparttable}
\begin{tabular}{lrrrrrrrrr}
    \toprule
    \multicolumn{1}{c}{\multirow{2}{*}{\textbf{Dataset}}}
    & \multicolumn{1}{c}{\multirow{2}{*}{\(d\)}}
    & \multicolumn{1}{c}{\multirow{2}{*}{\(N\)}}
    & \multicolumn{1}{c}{\multirow{2}{*}{\(L_{\mathrm{H}}\)}}
    & \multicolumn{1}{c}{\multirow{2}{*}{\(\mu_{\mathrm{KL}}\)}}
    & \multicolumn{1}{c}{\multirow{2}{*}{\(\kappa_{\mathrm{cond.}}\)}}
    & \multicolumn{1}{c}{\multirow{2}{*}{\({\lVert \bar{\vzeta}_{\mathrm{KL}} - \bar{\vzeta}_{\mathrm{H}} \rVert}_2^2\)}}
    & \multicolumn{2}{c}{\textbf{Constants for \cref{thm:gradient_upper_bound}}} &  \\ \cmidrule{8-9}
    & & & & & & & \multicolumn{1}{c}{\(A\)} & \multicolumn{1}{c}{\(C\)}
    \\ \midrule
    \textsc{Fertility} & 9 & 100 & \(1.840 \times 10^3\) & \(5.017 \times 10^2\) &  4 & \(5.167 \times 10^{-9}\) & \(1.620 \times 10^4\) & \(1.313 \times 10^6\) \\
    \textsc{Pendulum}  & 9 & 630 & \(1.525 \times 10^4\) & \(1.897 \times 10^3\) &  8 & \(1.243 \times 10^{-10}\) & \(2.942 \times 10^5\) & \(2.858 \times 10^7\) \\
    \textsc{Airfoil}   & 5 & 1,503 & \(3.520 \times 10^4\) & \(2.909 \times 10^3\) & 12 & \(2.937 \times 10^{-10}\) & \(6.815 \times 10^5\) & \(3.936 \times 10^7\) \\
    \textsc{Wine}      & 11 & 1,599 & \(5.526 \times 10^4\) & \(1.786 \times 10^3\) & 31 & \(6.628 \times 10^{-9}\)  & \(4.787 \times 10^6\) & \(6.054 \times 10^8\) \\
    \bottomrule
\end{tabular}
\begin{tablenotes}
\item[*] \(N\) is the number of datapoints in the dataset, \(\kappa_{\mathrm{cond.}} = L_{\mathrm{H}} / \mu_{\mathrm{KL}}\) is the condition number.
\end{tablenotes}
\end{threeparttable}
  }%
\end{center}
%\vspace{-4ex}
\end{table*}
}

%%% Local Variables:
%%% TeX-master: "main"
%%% End:


\begin{figure}[H]
  \vspace{-3ex}
  \centering
  
%\tikzexternalenable
{\hypersetup{linkbordercolor=black,linkcolor=black}
\begin{tikzpicture}
  \begin{groupplot}
    [
      group style={
        group size=3 by 1,
        % columns=2,
        % rows=2,
        %xlabels at=edge bottom,
        %ylabels at=edge left,
        horizontal sep=0.1\textwidth
      },
    ]
    
  \nextgroupplot[
      legend style={
        legend image post style = {scale=0.5},
        legend columns          = -1,
        column sep              = 1em,
        %at                     ={(0.5,1.5)},
        anchor                  = north,
        legend cell align       = left,
        line width              = 0.8pt,
        draw                    = none,
        legend to name          = grouplegend
      },
      %
      tuftelike, 
      axis line style = thick,
      every tick/.style={black,thick},
      %axis lines = left,
      %grids=both,
      ymode  = log,
      xmin   = 1,
      xmax   = 4000,
      xtick  = {1, 1000, 2000, 3000, 4000},
      ytick  = {1e+2, 1e+4, 1e+6, 1e+8},
      %
      ymin   = 1e+2,
      ymax   = 1e+8,
      %
      xlabel = {Iteration},
      %ylabel = {\(\mathbb{E}\norm{\rvvg}^2_2\)},
      height = 4.5cm,
      width  = 5cm,
      axis on top,
    ]
    \addplot[name path=gvar, color1, mark=none, thick]
      table [x=t, y=gvar] {data/fertility_softpluschol_generalbound.csv}
      %node[above=3pt,pos=0.8] {\scriptsize\(\mu_{\mathrm{KL}}, L_{\mathrm{H}}\) bound}
      coordinate [pos=0.95] (gvarcoord);
    \addlegendentry{\scriptsize\( \mathbb{E}\norm{ \rvvg }_2^2 \)}

    \addplot[name path=ABC, color2, mark=none, thick]
      table [x=t, y=ABC] {data/fertility_softpluschol_generalbound.csv}
      coordinate [pos=0.05] (ABCcoord1) coordinate [pos=0.95] (ABCcoord2);
    \addlegendentry{\scriptsize\(2 A \left(F\left(\vlambda\right) - F^*\right) + B \norm{\nabla F}_2^2 + C\)}

    \addplot[name path=opt, color3, mark=none, thick] %
      table [x=t, y=opt] {data/fertility_softpluschol_generalbound.csv} %
      %node[above=3pt,pos=0.8] {\scriptsize\(\DKL{q_{\vlambda}}{p}\)}
      coordinate [pos=0.95] (optcoord);
    %\addlegendentry{\scriptsize\(2 A \left(\mathbb{E}_{q_{\vlambda}} f_{\text{KL}} - f^*_{\text{KL}} \right)\)}

    \addplot[name path=axis, domain=0:3990, fill=none, no markers, draw=none] {1e+2}
      coordinate [pos=0.05] (axiscoord);

    \addplot[name path=C, fill=none, gray, mark=none]
      table [x=t, y=C] {data/fertility_softpluschol_generalbound.csv}
      coordinate [pos=0.05] (Ccoord);

    \addplot[thick, color=color3, fill=color3, fill opacity=0.2] fill between[of=ABC and opt];
    \addplot[thick, color=color2, fill=color2, fill opacity=0.2] fill between[of=ABC and   C];

    \addplot[thick, color=color1, fill=color1, fill opacity=0.2] fill between[of=opt and gvar];
    \addplot[thick, color=gray,   fill=gray,   fill opacity=0.2] fill between[of=C   and axis];

    \draw[thick,color=color3,latex-latex] (ABCcoord2) -- (optcoord)
      node[midway,xshift=-20pt] {\scriptsize\(\DKL{q_{\vlambda}}{p}\)};

    \draw[thick,color=gray,latex-] (Ccoord) -- (axiscoord)
      node[pos=0.8,xshift=5pt] {\scriptsize\(C\) };

    \input{figures/group_pendulum_softpluschol_generalbound}
    \input{figures/group_wine_softpluschol_generalbound}
  \end{groupplot}
  
  \draw[thick,color=black] ($(group c1r1) + (0cm,-2.7cm)$) node {\small \textsc{Fertility} };
  \draw[thick,color=black] ($(group c2r1) + (0cm,-2.7cm)$) node {\small \textsc{Pendulum} };
  \draw[thick,color=black] ($(group c3r1) + (0cm,-2.7cm)$) node {\small \textsc{Wine} };

  \node at ($(group c2r1) + (0cm,2cm)$) {\ref*{grouplegend}}; 
\end{tikzpicture}
}
%\tikzexternaldisable

  \vspace{-2ex}
  \caption{
    \textbf{Evaluation of \cref{thm:gradient_upper_bound} with the nonlinear Cholesky (\(\phi\left(x\right) = \mathrm{softplus}\left(x\right)\)) parameterization on linear regression datasets.}
    The gradient variance was estimated from \(4 \times 10^3\) samples.
  }\label{fig:linearreg_softpluschol}
\end{figure}

\begin{figure}[H]
  \centering
  \input{figures/fig_linearreg_linearchol_generalbound.tex}
  \vspace{-4ex}
  \caption{
    \textbf{Evaluation of \cref{thm:gradient_upper_bound} with the linear Cholesky (\(\phi\left(x\right) = x\)) parameterization on linear regression datasets.}
    The gradient variance was estimated from \(4 \times 10^3\) samples.
  }\label{fig:linearreg_linearchol}
\end{figure}

\begin{figure}[H]
  \centering
  \input{figures/fig_linearreg_softplusmf_generalbound.tex}
  \vspace{-4ex}
  \caption{
    \textbf{Evaluation of \cref{thm:gradient_upper_bound} with the nonlinear mean-field (\(\phi\left(x\right) = \mathrm{softplus}\left(x\right)\)) parameterization on linear regression datasets.}
    The gradient variance was estimated from \(4 \times 10^3\) samples.
  }\label{fig:linearreg_softplusmf}
\end{figure}

\begin{figure}[H]
  \centering
  
%\tikzexternalenable
{\hypersetup{linkbordercolor=black,linkcolor=black}
\begin{tikzpicture}
  \begin{groupplot}
    [
      group style={
        group size=4 by 1,
        % columns=2,
        % rows=2,
        %xlabels at=edge bottom,
        %ylabels at=edge left,
        horizontal sep=0.07\textwidth
      },
    ]
    \input{figures/group_fertility_squareroot_generalbound}
    
  \nextgroupplot[
      %% legend style={
      %%   legend image post style = {scale=0.5},
      %%   legend columns          = -1,
      %%   column sep              = 1em,
      %%   %at                     ={(0.5,1.5)},
      %%   anchor                  = north,
      %%   legend cell align       = left,
      %%   line width              = 0.8pt,
      %%   draw                    = none,
      %%   legend to name          = grouplegend
      %% },
      %
      tuftelike, 
      axis line style = thick,
      every tick/.style={black,thick},
      %axis lines = left,
      %grids=both,
      ymode  = log,
      xmin   = 1,
      %xmax   = 500,
      xtick  = {1, 1000, 2000},
      ytick  = {1e+3, 1e+5, 1e+7, 1e+9, 1e+11},
      %
      ymin   = 1e+3,
      ymax   = 1e+11,
      %
      xlabel = {Iteration},
      %ylabel = {\(\mathbb{E}\norm{\rvvg}^2_2\)},
      height = 5cm,
      width  = 4.7cm,
      axis on top,
    ]
    \addplot[name path=gvar, color1, mark=none, thick]
      table [x=t, y=gvar] {data/simulation/pendulum_squareroot_generalbound.csv}
      %node[above=3pt,pos=0.8] {\scriptsize\(\mu_{\mathrm{KL}}, L_{\mathrm{H}}\) bound}
      coordinate [pos=0.95] (gvarcoord);
    %% \addlegendentry{\scriptsize\( \mathbb{E}\norm{ \rvvg }_2^2 \)}

    \addplot[name path=ABC, color2, mark=none, thick]
      table [x=t, y=ABC] {data/simulation/pendulum_squareroot_generalbound.csv}
      coordinate [pos=0.05] (ABCcoord1) coordinate [pos=0.95] (ABCcoord2);
    %% \addlegendentry{\scriptsize\(2 A \left(F\left(\vlambda\right) - F^*\right) + B \norm{\nabla F}_2^2 + C\)}

    \addplot[name path=opt, color3, mark=none, thick] %
      table [x=t, y=opt] {data/simulation/pendulum_squareroot_generalbound.csv} %
      %node[above=3pt,pos=0.8] {\scriptsize\(\DKL{q_{\vlambda}}{p}\)}
      coordinate [pos=0.95] (optcoord);
    %\addlegendentry{\scriptsize\(2 A \left(\mathbb{E}_{q_{\vlambda}} f_{\text{KL}} - f^*_{\text{KL}} \right)\)}

    \addplot[name path=axis, domain=0:1990, fill=none, no markers, draw=none] {1e+3}
      coordinate [pos=0.05] (axiscoord);

    \addplot[name path=C, fill=none, gray, mark=none]
      table [x=t, y=C] {data/simulation/pendulum_squareroot_generalbound.csv}
      coordinate [pos=0.05] (Ccoord);

    \addplot[thick, color=color3, fill=color3, fill opacity=0.2] fill between[of=ABC and opt];
    \addplot[thick, color=color2, fill=color2, fill opacity=0.2] fill between[of=ABC and   C];

    \addplot[thick, color=color1, fill=color1, fill opacity=0.2] fill between[of=opt and gvar];
    \addplot[thick, color=gray,   fill=gray,   fill opacity=0.2] fill between[of=C   and axis];

    \draw[thick,color=color3,draw=none,latex-latex] (ABCcoord2) -- (optcoord)
      node[midway,xshift=-20pt] {}
      coordinate [pos=0.5] (DKLcoord);

    \draw[thick,color=gray,latex-] (Ccoord) -- (axiscoord)
      node[pos=0.8,xshift=5pt] {\scriptsize\(C\) };

    \node[%
      pin={%
        [%
          pin distance=0.3cm,%
          pin edge={color3},%
          text=color3%
        ]100:{\scriptsize\(\DKL{q_{\vlambda}}{p}\)}},%
      inner sep=0pt,%
    ] at (DKLcoord) {};

    \input{figures/group_airfoil_squareroot_generalbound}
    \input{figures/group_wine_squareroot_generalbound}
  \end{groupplot}
  
  \draw[thick,color=black] ($(group c1r1) + (0cm,-3cm)$) node {\small \textsc{Fertility} };
  \draw[thick,color=black] ($(group c2r1) + (0cm,-3cm)$) node {\small \textsc{Pendulum} };
  \draw[thick,color=black] ($(group c3r1) + (0cm,-3cm)$) node {\small \textsc{Airfoil} };
  \draw[thick,color=black] ($(group c4r1) + (0cm,-3cm)$) node {\small \textsc{Wine} };

  \node at ($(group c2r1) + (2.35cm,2.2cm)$) {\ref*{grouplegend}}; 
\end{tikzpicture}
}
%\tikzexternaldisable

  \vspace{-4ex}
  \caption{
    \textbf{Evaluation of \cref{thm:gradient_upper_bound} matrix square root parameterization on linear regression datasets.}
    The gradient variance was estimated from \(4 \times 10^3\) samples.
  }\label{fig:linearreg_squareroot}
\end{figure}


%%% Local Variables:
%%% TeX-master: "main"
%%% End:


\clearpage
\twocolumn

%
\section{Assumptions on the Gradient Variance}

\begin{definition}[\textbf{Expected Strong Growth; ESG}]
  \(\rvvg\) is said to satisfy the expected strong growth condition if 
  \begin{align*}
    \mathbb{E} \norm{\rvvg\left(\vlambda\right)}^2_2 \leq B \, \norm{ \nabla F\left(\vlambda\right) }^2
  \end{align*}
  for some finite \(B > 0\).
\end{definition}
This is a weaker version of the maximal strong growth condition~\citep{tseng_incremental_1998, schmidt_fast_2013}, where the bound is assumed to hold uniformly instead of in expectation.
Under ESG, for \(L\)-smooth non-convex objectives, \citet{vaswani_fast_2019} prove that SGD with a fixed stepsize of \(\gamma = \nicefrac{1}{B L}\) converges as \(\mathcal{O}\left(2 B L / T\right)\).

Arguably, the most widely recognized gradient condition is the relaxed strong growth by \citet{bottou_optimization_2018}.
\begin{definition}[\textbf{Relaxed Strong Growth; RSG}]
  \(\rvvg\) is said to satisfy the expected strong growth condition if 
  \begin{align*}
    \mathbb{E} \norm{\rvvg\left(\vlambda\right)}^2_2 \leq B \, \norm{ \nabla F\left(\vlambda\right) }^2_2 + C
  \end{align*}
  for some finite \(B, C > 0\).
\end{definition}
\citeauthor{bottou_optimization_2018} prove that, for \(L\)-smooth non-convex objectives and some additional regularity conditions, SGD with a fixed stepsize in \(\mathcal{O}\left(\nicefrac{1}{L B}\right)\) converges to a \(\mathcal{O}\left(\alpha L C\right)\) neighborhood in a rate of \(\mathcal{O}\left(\nicefrac{1}{T \alpha}\right) \).

\begin{definition}[\textbf{Convex Expected Smoothness; C-ES}]
  \(\rvvg\) is said to satisfy the expected strong growth condition if 
  \begin{align*}
    \mathbb{E} \norm{\rvvg\left(\vlambda\right) - \rvvg\left(\vlambda^*\right)}^2_2 \leq 2 A \left( F\left(\vlambda\right) - F\left(\vlambda^*\right) \right)
  \end{align*}
  for some finite \(A > 0\), where \(\vlambda^*\) is the global minimizer.
\end{definition}
This condition was first used by~\cite{gower_stochastic_2021} for studying the JacSketch algorithm  for strongly convex problems.

%%% Local Variables:
%%% TeX-master: "main"
%%% End:

\clearpage
\section{Proofs}
\subsection{External Lemmas}\label{section:external_lemmas}


\begin{theoremEnd}[category=common]{lemma}[\citealt{domke_provable_2019}, Lemma 9]
\label{thm:u_identities}
  Let \(\rvvu = \left(\rvu_1, \rvu_2, \ldots, \rvu_d\right)\) be a \(d\)-dimensional vector-valued random variable with zero-mean independently and identically distributed components.
  Then,
  \begin{alignat*}{2}
    &\mathbb{E}\rvvu \rvvu^{\top} &&= \left( \mathbb{E} \rvu_i^2 \right) \boldupright{I}
    \\
    &\mathbb{E}\norm{\rvvu}_2^2 &&= d \, \mathbb{E} \rvu_i^2
    \\
    &\mathbb{E} \rvvu \left( 1 + \norm{\rvvu}_2^2 \right) &&= \left( \mathbb{E} \rvu_i^3 \right) \mathbf{1}
    \\
    &\mathbb{E} \rvvu \rvvu^{\top} \rvvu \rvvu^{\top} &&= \left( \left(d - 1\right) \, {\left( \mathbb{E} \rvu_i^2 \right)}^2 + \mathbb{E}\rvu_i^4 \right) \boldupright{I}.
  \end{alignat*}
\end{theoremEnd}

\begin{theoremEnd}[category=common]{lemma}[\citealt{domke_provable_2019}, Lemma 1]
\label{thm:variational_gradient_norm_identity}
  Let \(\vt_{\vlambda}: \mathbb{R}^d \rightarrow \mathbb{R}^d\) be a location-scale reparameterization function (\cref{def:reparam}).
  Also, let \(f : \mathbb{R}^d \mapsto \mathbb{R} \) be some differentiable function.
  Then,
  \begin{alignat*}{2}
    \norm{\nabla_{\vlambda} f\left( \vt_{\vlambda}\left(\vu\right) \right) }_2^2
    = 
    \norm{\nabla f\left( \vt_{\vlambda}\left(\vu\right) \right) }_2^2 \left(1 + \norm{\vu}_2^2\right).
  \end{alignat*}
\end{theoremEnd}

\begin{theoremEnd}[category=common]{lemma}[\citealt{domke_provable_2019}, Lemma 1]
\label{thm:reparam_u_identity}
  Let \(\vt_{\vlambda}: \mathbb{R}^d \rightarrow \mathbb{R}^d\) be a location-scale reparameterizaiton function (\cref{def:reparam}).
  Also, let \(\vz \in \mathbb{R}^d\) be some vector and \(\rvvu \sim \varphi\) satisfy~\cref{assumption:symmetric_standard}.
  Then,
  \begin{alignat*}{2}
    \mathbb{E} \norm{\vt_{\vlambda}\left(\rvvu\right) - \vz}_2^2 \left(1 + \norm{\rvvu}_2^2\right)
    =
    \left(d+1\right) \norm{\vm - \vz}_2^2 + \left(d + \kappa\right) \norm{\mC}^2_{\mathrm{F}}.
  \end{alignat*}
\end{theoremEnd}

%%% Local Variables:
%%% TeX-master: "main"
%%% End:

% \printProofs[common]
\printProofs[upperboundlemma]

\clearpage
\subsection{Proof of Key Lemmas}
\subsubsection{Proof of \cref*{thm:general_variational_gradient_norm_identity}}
\printProofs[upperboundkeylemmagradientnormidentity]
\newpage
\subsubsection{Proof of \cref*{thm:general_variational_gradient_norm_bound}}
\printProofs[upperboundkeylemmagradientnormbound]
\newpage
\subsubsection{Proof of \cref*{thm:meanfield_u_identity}}
\printProofs[upperboundkeylemmameanfield]
\newpage
\subsubsection{Proof of \cref*{thm:gradient_variance_general_upper_bound}}
\printProofs[upperboundkeylemmavariancegeneral]

\clearpage
\subsection{Proof of Theorems}
\subsubsection{Proof of \cref*{thm:gradient_upper_bound}}
\printProofs[upperboundtheorem]
\newpage
\subsubsection{Proof of \cref*{thm:gradient_upper_bound_kl}}
\printProofs[upperboundtheoremklform]
\newpage
\subsubsection{Proof of \cref*{thm:gradient_upper_bound_bounded_entropy}}
\printProofs[upperboundtheoremboundedentropy]
%\newpage
%\printProofs[upperboundtheoremstl]

\newpage
\subsubsection{Proof of \cref*{thm:gradient_lower_bound}}
\printProofs[lowerboundtheorem]

%\clearpage
%\section{Smoothness Proofs}
%\printProofs[smoothness]

%\clearpage
%\section{Preservation of Strong Convexity}
%\printProofs[convexity]

%% \onecolumn
%% \section{You \emph{can} have an appendix here.}

%% You can have as much text here as you want. The main body must be at most $8$ pages long.
%% For the final version, one more page can be added.
%% If you want, you can use an appendix like this one, even using the one-column format.
%%%%%%%%%%%%%%%%%%%%%%%%%%%%%%%%%%%%%%%%%%%%%%%%%%%%%%%%%%%%%%%%%%%%%%%%%%%%%%%
%%%%%%%%%%%%%%%%%%%%%%%%%%%%%%%%%%%%%%%%%%%%%%%%%%%%%%%%%%%%%%%%%%%%%%%%%%%%%%%


\end{document}


% This document was modified from the file originally made available by
% Pat Langley and Andrea Danyluk for ICML-2K. This version was created
% by Iain Murray in 2018, and modified by Alexandre Bouchard in
% 2019 and 2021 and by Csaba Szepesvari, Gang Niu and Sivan Sabato in 2022. 
% Previous contributors include Dan Roy, Lise Getoor and Tobias
% Scheffer, which was slightly modified from the 2010 version by
% Thorsten Joachims & Johannes Fuernkranz, slightly modified from the
% 2009 version by Kiri Wagstaff and Sam Roweis's 2008 version, which is
% slightly modified from Prasad Tadepalli's 2007 version which is a
% lightly changed version of the previous year's version by Andrew
% Moore, which was in turn edited from those of Kristian Kersting and
% Codrina Lauth. Alex Smola contributed to the algorithmic style files.
