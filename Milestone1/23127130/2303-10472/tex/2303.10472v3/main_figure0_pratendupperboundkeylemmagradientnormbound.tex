

\prAtEndRestateiii*

\makeatletter\Hy@SaveLastskip\label{proofsection:prAtEndiii}\ifdefined\pratend@current@sectionlike@label\immediate\write\@auxout{\string\gdef\string\pratend@section@for@proofiii{\pratend@current@sectionlike@label}}\fi\Hy@RestoreLastskip\makeatother\begin{proof}[Proof]\phantomsection\label{proof:prAtEndiii}The proof continues from the result of \cref {thm:general_variational_gradient_norm_identity}. \par \paragraph {Proof for Cholesky} \cref {thm:general_variational_gradient_norm_identity} shows that \begin {alignat*}{2} \norm { \nabla _{\vlambda } f\left ( \vt _{\vlambda }\left (\vu \right ) \right ) }_2^2 = {\lVert \vg _{f} \rVert }_2^2 + \vg _{f}^{\top } \mY \vg _{f} + \vg _{f}^{\top } \mU \left ( \mPhi - \mI \right ) \vg _{f}. \end {alignat*} \par By the 1-Lipschitz assumption, the entries of the diagonal matrix \(\Phi \) satisfy \begin {align*} \Phi _{ii} = {\phi ^{\prime }\left (d_i\right )}^2 \leq 1, \end {align*} which means \begin {alignat*}{2} \mPhi \preceq \mI \;\Rightarrow \; \mU \left ( \mPhi - \mI \right ) \preceq 0 \;\Rightarrow \; {\vg _{f}}^{\top } \mU \left ( \mPhi - \mI \right ) \vg _{f} \leq 0. \end {alignat*} Therefore, for the full rank Cholesky parameterization and a 1-Lipschitz positive map \(\phi \), \begin {alignat*}{2} &\norm {\nabla _{\vlambda } f\left ( \vt _{\vlambda }\left (\vu \right ) \right )}_2^2 \\ &\;= {\lVert \nabla f\left ( \vt _{\vlambda }\left (\vu \right ) \right ) \rVert }_2^2 + {\vg _{f}}^{\top } \mY \vg _{f} + {\vg _{f}}^{\top } \mU \left ( \mPhi - \mI \right ) \vg _{f} \\ &\;\leq {\lVert \nabla f\left ( \vt _{\vlambda }\left (\vu \right ) \right ) \rVert }_2^2 + {\vg _{f}}^{\top } \mY \vg _{f} \\ &\;\leq {\lVert \nabla f\left ( \vt _{\vlambda }\left (\vu \right ) \right ) \rVert }_2^2 + \norm { \mY }_{2,2} {\lVert \nabla f\left ( \vt _{\vlambda }\left (\vu \right ) \right ) \rVert }_2^2 \\ &\;= {\lVert \nabla f\left ( \vt _{\vlambda }\left (\vu \right ) \right ) \rVert }_2^2 + \left ( \sum ^{d}_{i=1} u_{i}^2 \right ) {\lVert \nabla f\left ( \vt _{\vlambda }\left (\vu \right ) \right ) \rVert }_2^2 \\ &\;= \left (1 + \norm {\vu }^2_2\right ) {\lVert \nabla f\left ( \vt _{\vlambda }\left (\vu \right ) \right ) \rVert }_2^2, \end {alignat*} where \(\norm {\mU }_{2,2}\) is the \(L_2\) operator norm of \(\mU \). This upper bound coincides with that of the matrix square root parameteration. Thus, unforunately, this bound fails to acknowledge the lower variance of the Cholesky, coinciding with that of the matrix square root parameterization. \par \paragraph {Proof for Mean-field (\cref {def:meanfield})} For the mean-field parameterization,~\cref {thm:general_variational_gradient_norm_identity} shows that \begin {alignat*}{2} \norm { \nabla _{\vlambda } f\left ( \vt _{\vlambda }\left (\vu \right ) \right ) }_2^2 = {\lVert \vg _{f} \rVert }_2^2 + \mathrm {tr}\left ( \vg _{f} \vg _{f}^{\top } \mU \mPhi \right ), \end {alignat*} \par For the second term, \begin {alignat*}{2} \mathrm {tr}\left ( \vg _{f} \vg _{f}^{\top } \mU \mPhi \right ) &= \vg _{f}^{\top } \left ( \mU \mPhi \right ) \vg _{f} \\ &\leq {\lVert \mU \rVert }_{2,2} {\lVert \mPhi \rVert }_{2,2} {\lVert \vg _{f} \rVert }^2_2. \end {alignat*} By the \(1\)-Lipschitzness of \(\phi \), \[ {\lVert \mPhi \rVert }_{2,2} = \sigma _{\mathrm {max}}\left ( \mPhi \right ) = \max _{i = 1, \ldots , d} {\phi ^{\prime }\left ( s_i \right )}^2 \leq 1. \] Then, \begin {alignat}{2} \vg _{f}^{\top } \left ( \mU \mPhi \right ) \vg _{f} &\leq {\lVert \mU \rVert }_{2,2} \, {\lVert \vg _{f} \rVert }^2_2 \label {eq:variational_gradient_norm_identity_mf_eq1} \\ &\leq {\lVert \mU \rVert }_{\mathrm {F}} \, {\lVert \vg _{f} \rVert }^2_2, \label {eq:variational_gradient_norm_identity_mf_eq2} \end {alignat} which gives the result. Here, unlike the bounds on \(\mPhi \), the bounds in \cref {eq:variational_gradient_norm_identity_mf_eq1,eq:variational_gradient_norm_identity_mf_eq2} are quite loose, and become looser as the dimensionality increases. \par \end{proof}
