
\vspace{-1ex}
\section{Simulations}
\vspace{-.5ex}
We now evaluate our bounds and the insights gathered during the analysis through simulations.
We implemented a bare-bones implementation of BBVI in Julia~\citep{bezanson_julia_2017} with plain SGD.
The stepsize were manually tuned so that all problems converge at similar speeds.
For all problems, we use a unit Gaussian base distribution such that \(\varphi\left(u\right) = \mathcal{N}\left(u; 0, 1\right)\) resulting in a kurtosis of \(\kappa = 3\) and use \(M = 10\) Monte Carlo samples.

%% \begin{figure}[t]
%%   \centering
%%   \input{figures/fig_quadratic_softpluschol_generalbound.tex}
%%   \caption{
%%   }
%% \end{figure}

%% \begin{figure}[t]
%%   \centering
%%   \input{figures/fig_quadratic_softpluschol_boundedentropybound.tex}
%%   \caption{
%%   }
%% \end{figure}

%% \begin{figure}[t]
%%   \centering
%%   \input{figures/fig_quadratic_linearchol_generalbound.tex}
%%   \caption{
%%   }
%% \end{figure}

%% \begin{figure}[t]
%%   \centering
%%   \input{figures/fig_quadratic_linearchol_boundedentropybound.tex}
%%   \caption{
%%   }
%% \end{figure}

%% \begin{figure}[t]
%%   \centering
%%   \input{figures/fig_quadratic_squareroot_generalbound.tex}
%%   \caption{
%%   }
%% \end{figure}

\vspace{-.5ex}
\subsection{Synthetic Problem}\label{section:quadratic}
To test the \textit{ideal} tightness of the bounds, we consider quadratics achieving the tightest bound for the constants \(L_{\mathrm{H}}, L_{\mathrm{KL}}, \mu_{\mathrm{H}}, \mu_{\mathrm{KL}}\) given as
{%
\setlength{\belowdisplayskip}{1.ex} \setlength{\belowdisplayshortskip}{1.ex}%
\setlength{\abovedisplayskip}{1.ex} \setlength{\abovedisplayshortskip}{1.ex}%
\[
  \log \ell\left(\vx \mid \vz \right) = -\frac{N}{\sigma^2} \norm{ \vz - \vz^* }_2^2;\quad
  \log p\left(\vz \right)             = -\frac{1}{\lambda}  \norm{ \vz  }_2^2,
\]
}%
where \(N\) simulates the effect of the number of datapoints.
We set the constants as \(\sigma = 0.3\), \(\lambda = 8.0\), and \(N = 100\), the mode \(\vz^*\) is randomly sampled from a Gaussian, and the dimension of the problem is \(d = 20\).
For the bounded entropy case, we set \(S = 2.0\) (the true standard deviation is in the order of 1e-3).

\vspace{-.5ex}
\paragraph{Quality of Upper Bound}
The results for the Cholesky and mean-field parameterizations with a softplus bijector are shown in~\cref{fig:quadratic}.
For the Cholesky parameterization, the bulk of the looseness comes from the treatment of the regularization term (\textcolor{color3}{blue region}).
The remaining ``technical looseness'' (\textcolor{color1}{red region}) is relatively tight and can be shown to be tighter when using linear parameterizations (\(\phi\left(x\right) = x\)) and the square root parameterization, which is the tightest.
However, for the mean-field parameterization, despite the superior constants~(\cref{remark:meanfield_superiority}), there is still room for improvement.
Additional results for other parameterizations can be found in~\cref{section:additional_quadratic}.


\begin{figure}[t]
  \vspace{-1.5ex}
  \hspace{-1.5em}
  \subfloat{
    
\begin{tikzpicture}
\begin{axis}[
      legend style={
        legend image post style = {scale=0.5},
        at                      = {(0.5,1.5)},
        anchor                  = north,
        legend cell align       = left,
        line width              = 0.8pt,
        draw = none,
      },
      %
      tuftelike, 
      %
      xlabel style={yshift=-0.8ex},
      %
      axis line style = thick,
      every tick/.style={black,thick},
      %axis lines = left,
      %grids=both,
      ymode  = log,
      xmin   = 1,
      xmax   = 4000,
      xtick  = {1, 2000, 4000},
      ytick  = {1e+4, 1e+6, 1e+8, 1e+10},
      %
      ymin   = 1e+4,
      ymax   = 1e+10,
      %
      xlabel = {Iteration},
      %ylabel = {\(\mathbb{E}\norm{\rvvg}^2_2\)},
      tick label style={font=\small},  
      height = 4cm,
      width  = 4.3cm,
      axis on top,
    ]
    \addplot[name path=gvar, color1, mark=none, thick]
      table [x=t, y=gvar] {data/simulation/airfoil_softpluschol_generalbound.csv}
      %node[above=3pt,pos=0.8] {\scriptsize\(\mu_{\mathrm{KL}}, L_{\mathrm{H}}\) bound}
      coordinate [pos=0.95] (gvarcoord);
    \addlegendentry{\scriptsize Gradient Variance\;\; \( \mathbb{E}\norm{ \rvvg }_2^2 \)}

    \addplot[name path=ABC, color2, mark=none, thick]
      table [x=t, y=ABC] {data/simulation/airfoil_softpluschol_generalbound.csv}
      coordinate [pos=0.05] (ABCcoord1) coordinate [pos=0.95] (ABCcoord2);
    \addlegendentry{\scriptsize Upper Bound}

    %\addplot[color4, draw=none, mark=none, thick] {x};
    %\addlegendentry{\scriptsize\(2 A \left(\mathbb{E}_{q_{\vlambda}} f_{\mathrm{KL}} - f^*_{\mathrm{KL}} \right) + B \norm{\nabla F}_2^2 + C_{\Delta \zeta} {\lVert \bar{\zeta}_{\mathrm{KL}} - \bar{\zeta}_{\mathrm{H}} \rVert}^2_2 \)}

    \addplot[name path=opt, color3, mark=none, thick] %
      table [x=t, y=opt] {data/simulation/airfoil_softpluschol_generalbound.csv} %
      %node[above=3pt,pos=0.8] {\scriptsize\(\DKL{q_{\vlambda}}{p}\)}
      coordinate [pos=0.95] (optcoord);
    %\addlegendentry{\scriptsize\(2 A \left(\mathbb{E}_{q_{\vlambda}} f_{\mathrm{H}} - f^*_{\mathrm{H}} \right) + B \norm{\nabla F}_2^2\)}

    \addplot[name path=axis, domain=0:3990, fill=none, no markers, draw=none] {1e+4}
      coordinate [pos=0.05] (axiscoord);

    %\addplot[name path=ABC, color2, mark=none]
    %  table [x=t, y=ABC] {data/simulation/quadratic_softpluschol_generalbound.csv}
    %  ;

    \addplot[name path=C, fill=none, gray, mark=none]
      table [x=t, y=C] {data/simulation/airfoil_softpluschol_generalbound.csv}
      coordinate [pos=0.05] (Ccoord);

    \addplot[thick, color=color3, fill=color3, fill opacity=0.2] fill between[of=ABC and opt];
    \addplot[thick, color=color2, fill=color2, fill opacity=0.2] fill between[of=ABC and   C];

    \addplot[thick, color=color1, fill=color1, fill opacity=0.2] fill between[of=opt and gvar];
    \addplot[thick, color=gray,   fill=gray,   fill opacity=0.2] fill between[of=C   and axis];

    \draw[thick,color=color3,draw=none] (ABCcoord2) -- (optcoord)
      node[midway,xshift=-18pt] {}
      coordinate [pos=0.5] (DKLcoord);

    %\draw[thick,color=color1,latex-latex] (gvarcoord) -- (optcoord)
    %  %node[midway,xshift=-32pt] {\scriptsize\(\mu_{\mathrm{KL}}, L_{\mathrm{H}}, \Delta \vz^*\) bound}
    %  coordinate [pos=0.5] (muL_coord);

    \draw[thick,color=gray,latex-] (Ccoord) -- (axiscoord)
      node[pos=0.8,xshift=5pt] {\scriptsize\(C\) };

    \node[%
      pin={%
        [%
          pin distance=0.3cm,%
          pin edge={color3},%
          text=color3%
        ]100:{\scriptsize\(\DKL{q_{\vlambda}}{p}\)}},%
      inner sep=0pt,%
    ] at (DKLcoord) {};
\end{axis}
\end{tikzpicture}
\label{fig:airfoil_bound}
  }
  \subfloat{
    \hspace{-2em}
    
\tikzset{subcaptionstyle/.style={
    text width=2in,yshift=-3mm, align=center,anchor=north
}}

\begin{tikzpicture}[%
    /pgfplots/set layers,
    spy using outlines={circle, magnification=3, connect spies}%
  ]
  \begin{axis}[
      legend style={
        legend image post style={scale=0.5},
        at={(0.5,1.6)},
        anchor=north,
        legend cell align=left,
        line width=0.8pt,
        draw=none % Unterdrücke Box
      },
      tuftelike, 
      axis line style = thick,
      every tick/.style={black,thick},
      %axis lines = left,
      %grids=both,
      ymode  = log,
      xmin   = 1,
      xmax   = 1000,
      xtick  = {1, 500, 1000},
      %
      xlabel style={yshift=-0.8ex},
      %
      ymin   = 1e+4,
      ymax   = 1e+7,
      %
      xlabel = {Iteration},
      tick label style={font=\small},  
      ylabel = {\(\mathbb{E}\norm{\rvvg}^2_2\)},
      height = 4cm,
      width  = 4.3cm,
      axis on top,
    ]
    \addplot[color1, thick, mark=none] table[x=t, y=squareroot, mark=none] {data/replay_airfoil.csv};
    \addlegendentry{\scriptsize{Matrix square root}};

    \addplot[color2, thick, mark=none] table[x=t, y=linearcholesky, mark=none] {data/replay_airfoil.csv};
    \addlegendentry{\scriptsize{Cholesky \(\phi(x) = x\)}};

    %\addplot[color3, thick,  mark=none] table[x=t, y=expcholesky, mark=none] {data/replay_airfoil.csv};
    %\addlegendentry{\scriptsize{Cholesky \(\phi(x) = \exp\left(x\right)\)}};

    \addplot[color4, thick, mark=none] table[x=t, y=softpluscholesky, mark=none] {data/replay_airfoil.csv};
    \addlegendentry{\scriptsize{Cholesky \(\phi(x) = \mathrm{softplus}\left(x\right)\)}};

    % Square Root
    \addplot [name path=squarerootl, fill=none, draw=none, forget plot] table [x=t, y=squarerootl] {data/replay_airfoil.csv} \closedcycle;
    \addplot [name path=squarerootu, fill=none, draw=none, forget plot] table [x=t, y=squarerootu] {data/replay_airfoil.csv} \closedcycle;
    \addplot[color1!30] fill between[of=squarerootu and squarerootl];

    % Cholesky Linear
    \addplot [name path=linearcholeskyl, fill=none, draw=none, forget plot] table [x=t, y=linearcholeskyl] {data/replay_airfoil.csv} \closedcycle;
    \addplot [name path=linearcholeskyu, fill=none, draw=none, forget plot] table [x=t, y=linearcholeskyu] {data/replay_airfoil.csv} \closedcycle;
    \addplot[color2!40] fill between[of=linearcholeskyu and linearcholeskyl];

    % Cholesky Exp
%%     \addplot [name path=expcholeskyl, fill=none, draw=none, forget plot] table [x=t, y=expcholeskyl] {data/replay_airfoil.csv} \closedcycle;
%%     \addplot [name path=expcholeskyu, fill=none, draw=none, forget plot] table [x=t, y=expcholeskyu] {data/replay_airfoil.csv} \closedcycle;
%%     \addplot[color3!40] fill between[of=expcholeskyu and expcholeskyl];

    % Cholesky Softplus
    \addplot [name path=softpluscholeskyl, fill=none, draw=none, forget plot] table [x=t, y=softpluscholeskyl] {data/replay_airfoil.csv} \closedcycle;
    \addplot [name path=softpluscholeskyu, fill=none, draw=none, forget plot] table [x=t, y=softpluscholeskyu] {data/replay_airfoil.csv} \closedcycle;
    \addplot[color4!40] fill between[of=softpluscholeskyu and softpluscholeskyl];

    \coordinate (spypoint)     at (axis cs:280,1e+6);
    \coordinate (magnifyglass) at (axis cs:800,1.5e+6);
    \begin{scope}
      \spy [black, size=1.5cm] on (spypoint) in node[fill=white] at (magnifyglass);
    \end{scope}
  \end{axis}
\end{tikzpicture}
\label{fig:airfoil_parameterizations}
  }
  \vspace{-1.0ex}
  \caption{
    \textbf{
      Linear regression on the \textsc{Airfoil} dataset.
      (\textsf{left}) Evaluation of the upper bound (\cref{thm:gradient_upper_bound}).
      (\textsf{right}) Comparison of the variance of different parameterizations resulting in the same \(\vm\), \(\mC\).
    }
  }
  \vspace{-3.0ex}
\end{figure}

\vspace{-.5ex}
\subsection{Real Dataset}\label{section:linearreg}
\vspace{-.5ex}
\paragraph{Model}
We now evaluate the theoretical results with real datasets.
Given a regression dataset \((\mX, \vy)\), we use the linear Gaussian model 
{%
\setlength{\belowdisplayskip}{1.ex} \setlength{\belowdisplayshortskip}{1.ex}%
\setlength{\abovedisplayskip}{1.ex} \setlength{\abovedisplayshortskip}{1.ex}%
\[
  y   \sim \mathcal{N}\left(\mX \vw, \sigma^2\right);\quad
  \vw \sim \mathcal{N}\left(\mathbf{0}, \lambda \boldupright{I}\right),
\]
}%
where \(\lambda\) and \(\sigma\) are hyperparameters.
The smoothness and quadratic growth constants for this model are given as the max- and minimum eigenvalues of \(\sigma^{-2} \mX^{\top} \mX + \lambda^{-1} \boldupright{I}\) (for \(f_{\text{H}}\)) and \(\sigma^{-2} \mX^{\top} \mX\) (for \(f_{\text{KL}}\)).
\(f_{\text{KL}}^*\) and \(f_{\text{H}}^*\) are given as the mode of the likelihood and the posterior, while \(F^*\) is the negative marginal log-likelihood.

\vspace{-.5ex}
\paragraph{Quality of Upper Bound}
\cref{fig:airfoil_bound} shows the result on the \textsc{Airfoil} dataset~\citep{Dua:2019}.
The constants are \(L_{\mathrm{H}} = 3.520 \times 10^4, \mu_{\mathrm{KL}}=2.909 \times 10^3\).
Due to poor conditioning, the bound is much looser compared to the quadratic case.
We note that generalizing our bounds to utilize matrix smoothness and matrix-quadratic growth as done by \citep{domke_provable_2019} would tighten the bounds.
But the theoretical gains would be marginal.
Detailed information about the datasets and additional results for other parameterizations can be found in~\cref{section:additional_linearreg}.

\vspace{-.5ex}
\paragraph{Comparison of Parameterizations}
\cref{fig:airfoil_parameterizations} compares the gradient variance resulting from the different parameterizations.
For a fair comparison, the gradient is estimated on the \(\vlambda\) that results in the same \(\vm, \mC\) for all three parameterizations.
%The nonlinear parameterizations result in the lowest variance.
%Furthermore, the matrix square root parameterization results in the highest variance.
This shows the gradual increase in variance by
\begin{enumerate*}[label=\textbf{(\roman*)}]
  \item not using a nonlinear conditioner (linear Cholesky)
  \item and increasing the number of variational parameters (matrix square root).
\end{enumerate*}

%%% Local Variables:
%%% TeX-master: "main"
%%% End:
