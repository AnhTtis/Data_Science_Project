
\begin{theoremEnd}[category=smoothness]{proposition}[\textbf{Smoothness of Half-Normal}]
  Let \(f\left(x; \sigma\right)\) be the probability density function of the half-normal distribution with parameter \(\sigma\).
  Also, let \(\psi^{-1} : \mathbb{R} \mapsto \mathbb{R}_+\) be a bijection function.
  \(f \circ \psi^{-1}\) is \(L\)-smooth if and only if \cref{assumption:gamma3} hold, where the the smoothnes constant is \(L = L_3 / \sigma^{2}\).
\end{theoremEnd}
\begin{proofEnd}
  First, the log probability density of the half-normal distribution is given as
  \begin{align}
    \log f\left(x; \sigma\right)
    =
    -\frac{x^2}{2\sigma^2}
    + 
    \log Z,
    \nonumber
  \end{align}
  where \(\log Z\) is a constant normalizer.
  The derivatives of the inverse gamma distribution with a bijection \(\psi^{-1}\) is given as
  \begin{align*}
    &\frac{d \log f\left(\psi^{-1}\left(x\right); \sigma\right)}{dx}
    \\
    &\;=
    \frac{d \log f\left(\psi^{-1}\left(x\right); \sigma\right)}{d\psi}
    \frac{d \psi^{-1}}{dx}
    \\
    &\;=
    \frac{d \psi^{-1}}{dx}
    \left(
    -\frac{ \psi^{-1}\left(x\right) }{\sigma^2}
    \right)
    \\
    &\;=
    -\frac{1}{\sigma^2} {\left( \psi^{-1} \right)}^{\prime}\left(x\right) \psi^{-1}\left(x\right)
    \\
    &\;=
    -\frac{1}{\sigma^2} \gamma_3\left(x\right)
  \end{align*}
  Thus, the smoothness is euquivalent to the Lipschitzness of \(\gamma_3\left(x\right)\).
\end{proofEnd}

%%% Local Variables:
%%% TeX-master: "main"
%%% End:
