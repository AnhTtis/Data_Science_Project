

\prAtEndRestatexx*

\makeatletter\Hy@SaveLastskip\label{proofsection:prAtEndxx}\ifdefined\pratend@current@sectionlike@label\immediate\write\@auxout{\string\gdef\string\pratend@section@for@proofxx{\pratend@current@sectionlike@label}}\fi\Hy@RestoreLastskip\makeatother\begin{proof}[Proof]\phantomsection\label{proof:prAtEndxx}Recall that \begin {align} F\left (\vlambda \right ) = \underbrace {\mathbb {E} f\left (\vt _{\vlambda }\left (\rvvu \right )\right )}_{\text {likelihood term}} + \underbrace {h\left (\vlambda \right )}_{\text {regularization term}}. \end {align} We first establish the strong convexity of the likelihood term and then the convexity of the regularization term. \par \paragraph {Strong Convexity of likelihood term} For the matrix square root parameterization (\cref {def:squareroot}), if \(f\) is \(\mu \)-strongly convex, then the likelihood term is also \(\mu \)-strongly convex \citep {domke_provable_2020}. This means that, \begin {align} & \mathbb {E} f\left (\vt _{\vlambda }\left (\rvvu \right )\right ) - \mathbb {E} f\left (\vt _{\vlambda ^{\prime }}\left (\rvvu \right )\right ) \nonumber \\ &\;= \inner {\nabla _{\vlambda ^{\prime }} \mathbb {E} f\left ( \vt _{\vlambda ^{\prime }}\left (\rvvu \right ) \right ) }{ \vlambda - \vlambda ^{\prime } } + \frac {\mu }{2} \mathbb {E}\norm {\vt _{\vlambda }\left (\rvvu \right ) - \vt _{\vlambda ^{\prime }}\left (\rvvu \right )}_2^2 \nonumber \\ &= \inner { \mathbb {E} \frac { \partial \vt _{\vlambda ^{\prime }}\left (\rvvu \right ) }{ \partial \vlambda ^{\prime } } \nabla f \left ( \vt _{\vlambda ^{\prime }}\left (\rvvu \right ) \right ) }{ \vlambda - \vlambda ^{\prime } } + \frac {\mu }{2} \mathbb {E}\norm {\vt _{\vlambda }\left (\rvvu \right ) - \vt _{\vlambda ^{\prime }}\left (\rvvu \right )}_2^2. \label {eq:thm_convexity_inequality} \end {align} Focusing on the inner product term, \begin {align*} &\inner { \mathbb {E} \frac { \partial \vt _{\vlambda ^{\prime }}\left (\rvvu \right ) }{ \partial \vlambda ^{\prime } } \nabla f \left ( \vt _{\vlambda ^{\prime }}\left (\rvvu \right ) \right ) }{ \vlambda - \vlambda ^{\prime } } \\ &\;= \inner { \mathbb {E} \frac { \partial \vt _{\vlambda ^{\prime }}\left (\rvvu \right ) }{ \partial \vm ^{\prime } } \nabla f \left ( \vt _{\vlambda ^{\prime }}\left (\rvvu \right ) \right ) }{ \vm - \vm ^{\prime } } \\ &\;\quad + \sum ^d_{j=1} \sum ^d_{i=1} \left ( \mathbb {E} \frac { \partial \vt _{\vlambda ^{\prime }}\left (\rvvu \right ) }{ \partial C_{ij}^{\prime } } \nabla f \left ( \vt _{\vlambda ^{\prime }}\left (\rvvu \right ) \right ) \right ) \left (C_{ij} - C_{ij}^{\prime }\right ) \\ &\;= \inner { \mathbb {E} \frac { \partial \vt _{\vlambda }\left (\rvvu \right ) }{ \partial \vm ^{\prime } } \nabla f \left ( \vt _{\vlambda ^{\prime }}\left (\rvvu \right ) \right ) }{ \vm - \vm ^{\prime } } \\ &\;\quad + \underbrace { \sum ^d_{j=1} \sum ^d_{i=1,\,i \neq j} \left ( \mathbb {E} \frac { \partial \vt _{\vlambda ^{\prime }}\left (\rvvu \right ) }{ \partial C_{ij}^{\prime } } \nabla f \left ( \vt _{\vlambda ^{\prime }}\left (\rvvu \right ) \right ) \right ) \left (C_{ij} - C_{ij}^{\prime }\right ) }_{\text {off-diagonal of \(\mC \)}} \\ &\;\quad + \underbrace { \sum ^d_{i=1} \left ( \mathbb {E} \frac { \partial \vt _{\vlambda ^{\prime }}\left (\rvvu \right ) }{ \partial C_{ii}^{\prime } } \nabla f \left ( \vt _{\vlambda ^{\prime }}\left (\rvvu \right ) \right ) \right ) \left (C_{ii} - C_{ii}^{\prime }\right ) }_{\text {diagonal of \(\mC \)}}. \end {align*} Furthermore, the diagonal terms can be shown to satisfy \begin {align} &\sum ^d_{i=1} \left ( \mathbb {E} \frac { \partial \vt _{\vlambda ^{\prime }}\left (\rvvu \right ) }{ \partial C_{ii}^{\prime } } \nabla f \left ( \vt _{\vlambda ^{\prime }}\left (\rvvu \right ) \right ) \right ) \left (C_{ii} - C_{ii}^{\prime }\right ) \nonumber \\ &\;= \sum ^d_{i=1} \mathbb {E} u_i \ve _i^{\top } \nabla f \left ( \vt _{\vlambda ^{\prime }}\left (\rvvu \right ) \right ) \left (C_{ii} - C_{ii}^{\prime }\right ).\label {eq:thm_convexity_diagonal_squareroot} \end {align} For the parameterizations in \cref {def:meanfield} and \cref {def:fullrank}, the derivatives only differ for the diagonal of \(\mC \). Specifically, since the diagonals are \(C_{\mathrm {FR},ii} = \phi \left (s_i\right )\), the inner product term of the diagonal entries of the full rank parameterization (\cref {def:fullrank}) is now \begin {align*} &\sum ^d_{i=1} \left ( \mathbb {E} \frac { \partial \vt _{\vlambda ^{\prime }}\left (\rvvu \right ) }{ \partial C_{\mathrm {FR},ii}^{\prime } } \nabla f \left ( \vt _{\vlambda ^{\prime }}\left (\rvvu \right ) \right ) \right ) \left (C_{\mathrm {FR},ii} - C_{\mathrm {FR},ii}^{\prime }\right ) \\ &\;= \sum ^d_{i=1} \mathbb {E} \phi ^{\prime }\left (s_i^{\prime }\right ) u_i \ve _i^{\top } \nabla f \left ( \vt _{\vlambda ^{\prime }}\left (\rvvu \right ) \right ) \left (C_{\mathrm {FR},ii} - C_{\mathrm {FR},ii}^{\prime }\right ) \\ &\;\leq \sum ^d_{i=1} \mathbb {E} u_i \ve _i^{\top } \nabla f \left ( \vt _{\vlambda ^{\prime }}\left (\rvvu \right ) \right ) \left (C_{\mathrm {FR},ii} - C_{\mathrm {FR},ii}^{\prime }\right ), \end {align*} which is equal to~\cref {eq:thm_convexity_diagonal_squareroot}. Therefore, the bound in \cref {eq:thm_convexity_inequality} is preserved, where the strong convexity term \par \paragraph {Convexity of regularization term} For the full-rank Cholesky parameterization \cref {def:fullrank}, the regularization term is \begin {align*} h\left (\vlambda \right ) &= - \mathrm {H}\left (q_{\vlambda }\right ) \\ &= - \mathrm {H}\left (\varphi \right ) - \log \abs { \mC } \\ &= - \mathrm {H}\left (\varphi \right ) - \sum _{i=1}^{d} \log \phi \left (s_i\right ). \end {align*} Showing that \(-\log \phi \) is convex shows that \(h\) is convex since the sum of convex functions is convex. Now, since \begin {align*} -\frac {d^2 \log \phi }{ds^2} = \frac { {\phi ^{\prime }\left (s\right )}^{2} - \phi ^{\prime \prime }\left (x\right ) }{ {\phi \left (s\right )}^{2} } \end {align*} and \(\phi \left (s\right ) > 0\) for all \(s\), any \(\phi \) satisfying results in \(h\) being convex.\end{proof}
