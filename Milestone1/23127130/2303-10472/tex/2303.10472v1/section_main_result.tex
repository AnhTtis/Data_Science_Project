
\section{Main Results}
\vspace{-.5ex}

%
\begin{assumption}\label{assumption:bijector_lipschitz_condition1}
  Let \(\phi_{+}\left(x\right)\) be a bijector, where \(x \mapsto \nicefrac{\phi^{\prime}\left(x\right)}{\phi\left(x\right)} \) is \(L_{+}\)-Lipschitz.
\end{assumption}

\begin{proposition}
  Let \(\phi_{+}\left(x\right) = e^x\). 
  Then, \(\phi_{+}\left(x\right)\) satisfies \cref{assumption:bijector_lipschitz_condition1} with any \(L_{+} = \epsilon > 0\).
\end{proposition}
\begin{proof}
  Since \(\nicefrac{\phi^{\prime}_+\left(x\right)}{\phi_+\left(x\right)} = 1\), the conclusion follows from 
  \( \abs{\nicefrac{\phi^{\prime}_+\left(x\right)}{\phi_+\left(x\right)} - \nicefrac{\phi^{\prime}_+\left(y\right)}{\phi_+\left(y\right)}} = 0.\)
\end{proof}

\begin{proposition}
  Let \(\phi_{+}\left(x\right) = \mathrm{softplus}\left(x\right)\). 
  Then, \(\phi_{+}\left(x\right)\) satisfies \cref{assumption:bijector_lipschitz_condition1} with \(L_{+} = 2\).
\end{proposition}
\begin{proof}
  Since \(\mathrm{softplus}\left(x\right) = \log\left(1 + e^x\right)\), 
  \begin{align*}
    \phi_{+}^{\prime}\left(x\right) = \frac{e^x}{1 + e^x} = \sigma\left(x\right),
  \end{align*}
  where \(\sigma\left(x\right)\) is known as the sigmoid/logistic function with the derivative \(\sigma^{\prime}\left(x\right) = \sigma\left(x\right) \left(1 - \sigma\left(x\right)\right).\)
  Now, by the Mean Value Theorem, 
  \begin{align}
    \abs{ \frac{\phi^{\prime}_{+}\left(x\right)}{\phi\left(x\right)} - \frac{\phi^{\prime}_{+}\left(y\right)}{\phi\left(y\right)} }
    \leq
    \abs{ \frac{d}{dx} \frac{\phi^{\prime}_{+}\left(x\right)}{\phi\left(x\right)} }
    \abs{
      x - y
    }.
  \end{align}
  Now, 
  \begin{align*}
    \abs{ \frac{d}{dx} \frac{\phi^{\prime}_{+}\left(x\right)}{\phi\left(x\right)} }
    &=
    \abs{
      \phi^{\prime\prime}_{+}\left(x\right) \frac{1}{\phi_{+}\left(x\right)}
      -
      {\left( \phi^{\prime}_{+}\left(x\right) \right)}^2 \frac{1}{\phi_{+}^2\left(x\right)}
    }
    \\
    &=
    \frac{1}{\phi_{+}\left(x\right)} \abs{ \phi^{\prime\prime}_{+}\left(x\right) - \frac{{\left( \phi^{\prime}_{+}\left(x\right) \right)}^2}{\phi_{+}\left(x\right)}
    }
    \\
    &=
    \frac{1}{\phi_{+}\left(x\right)} \abs{
      \sigma\left(x\right) \left( 1 - \sigma\left(x\right) \right) - \frac{\sigma^2\left(x\right)}{\phi_{+}\left(x\right)}
    }
    \\
    &=
    \frac{\sigma\left(x\right)}{\phi_{+}\left(x\right)} \abs{
       1 - \sigma\left(x\right) - \frac{\sigma\left(x\right)}{\phi_{+}\left(x\right)}
    }
    \\
    &\leq
    \frac{\sigma\left(x\right)}{\phi_{+}\left(x\right)} \abs{
       1 - \sigma\left(x\right) 
    }
    +
    {\left(
    \frac{\sigma\left(x\right)}{\phi_{+}\left(x\right)} 
    \right)}^2
    \\
    &\leq
    \frac{\sigma\left(x\right)}{\phi_{+}\left(x\right)} 
    +
    {\left(
    \frac{\sigma\left(x\right)}{\phi_{+}\left(x\right)} 
    \right)}^2,
  \end{align*}
  since \(0 \leq \sigma\left(x\right) \leq 1\), \(0 \leq \phi_{+}\left(x\right)\).
  Notice that 
  \begin{align*}
    \frac{\sigma\left(x\right)}{\phi_{+}\left(x\right)} 
    =
    \frac{e^x}{1 + e^x} 
    \frac{1}{ \log\left(1 + e^x\right) } 
  \end{align*}
  is monotonically decreasing.
  Therefore, \(\nicefrac{\sigma\left(x\right)}{\phi_{+}\left(x\right)} \leq \lim_{x \rightarrow - \infty} \nicefrac{\sigma\left(x\right)}{\phi_{+}\left(x\right)} = 1\).

  Thus,
  \begin{align*}
    \abs{ \frac{\phi^{\prime}_{+}\left(x\right)}{\phi\left(x\right)} - \frac{\phi^{\prime}_{+}\left(y\right)}{\phi\left(y\right)} }
    &\leq
    \abs{ \frac{d}{dx} \frac{\phi^{\prime}_{+}\left(x\right)}{\phi\left(x\right)} }
    \abs{
      x - y
    }
    \\
    &\leq
    \abs{
      \frac{\sigma\left(x\right)}{\phi_{+}\left(x\right)} 
      +
      {\left(
        \frac{\sigma\left(x\right)}{\phi_{+}\left(x\right)} 
        \right)}^2
    }
    \abs{
      x - y
    }
    \\
    &\leq
    2 \,
    \abs{
      x - y
    }.
  \end{align*}
\end{proof}

\begin{proposition}
  Let \(q_{\vlambda}\) belong to the location scale family as \cref{def:family}, where \(\mC \in \mathbb{R}^{d \times d}\) is formed through the Cholesky factor parameterization.
  \(\vlambda \mapsto H\left[q_{\vlambda}\right]\) is \(L\)-smooth if and only if the bijector \(\phi_+\) satisfies \cref{assumption:bijector_lipschitz_condition1}, where the smoothness constant is \(L = d L_{+}\).
\end{proposition}
\begin{proof}
  For members of the location scale family with parameters \(\left(\vm, \mC\right)\), the entropy and its gradient are given as
  \begin{align*}
    \mathrm{H}\left[q_{\vlambda}\right] = \log \abs{ \mC } \quad\text{and}\quad \nabla_{\mC} \mathrm{H}\left[q_{\vlambda}\right] = - \mC^{-\top}.
  \end{align*}

  Let \(\mC\) form a Cholesky factor.
  Then, since the determinant of a triangular matrix is the product of its diagonal elements,
  \begin{align*}
    \mathrm{H}\left[q_{\vlambda}\right] = \log \abs{\mC} = \sum_{i=1}^d \log C_{ii} = \sum_{i=1}^d \log \phi_{+}\left( d_{i} \right).
  \end{align*}
  It is apparent that the derivatives of the entropy term are now
  \begin{align*}
    \frac{\partial \mathrm{H}\left[q_{\vlambda}\right]}{ \partial d_i }
    =
    \frac{\partial \log \phi_{+}\left(d_i\right)}{ \partial d_i }
    =
    \frac{\phi^{\prime}_{+}\left(d_i\right)}{\phi_{+}\left(d_i\right)}
    %
    \quad\text{and}\quad
    %
    \frac{\partial \mathrm{H}\left[q_{\vlambda}\right]}{ \partial L_{ij} }
    &=
    0.
  \end{align*}
  Then,
  \begin{align*}
    \norm{ \nabla_{\vlambda} \mathrm{H}\left[q_{\vlambda}\right] - \nabla_{\vlambda^{\prime}} \mathrm{H}\left[q_{\vlambda}\right] }_2^2
    &=
    \sum^{d}_{i=1}
    {\left(
    \frac{\phi^{\prime}_{+}\left(d_i\right)}{\phi_{+}\left(d_i\right)}
    -
    \frac{\phi^{\prime}_{+}\left(d_i^{\prime}\right)}{\phi_{+}\left(d_i^{\prime}\right)}
    \right)}^2.
  \end{align*}

  Thus, \(\vlambda \mapsto \mathrm{H}\left[q_{\vlambda}\right]\) is (\(d L_{+}\))-smooth if and only if \(d_i \mapsto \frac{\phi^{\prime}_{+}\left(d_i\right)}{\phi_{+}\left(d_i\right)}\) is \(L_{+}\)-Lipschitz.
\end{proof}

\begin{theorem}
  Let \(f\left(\vlambda\right)\) be \(L_{f}\)-smooth, \(g\left(\vlambda\right)\) be \(L_{g}\)-smooth.
  Then,
  \begin{enumerate}
    \item \(F_{\text{ELBO}}\left(\vlambda\right)\) is (\(L_f + L_g + \))-smooth
    \item \(\Esub{\rvveta \sim q_{\vlambda}\left(\right)}{ } \)
  \end{enumerate}
\end{theorem}
\begin{proof}
  
\end{proof}

%%% Local Variables:
%%% TeX-master: "master"
%%% End:


%2
\begin{theorem}
  
\end{theorem}
\begin{proof}
  \begin{alignat}{2}
    \norm{ \mH_{\ell} }_{\mathrm{F}}^2
    &=
    \sum_{i} \sum_{j} {\left\{ \mathbb{E} \frac{\partial^2}{\partial \lambda_i \partial \lambda_j}  f\left(\vt_{\vlambda}\left(\rvvu\right)\right) \right\}}^2
    \\
    &=
    \sum_{i} \sum_{j} {\left\{
      \mathbb{E} \frac{\partial}{\partial \lambda_j}
      \inner{
        \frac{\partial \vt_{\vlambda}\left(\rvvu\right)}{\partial \lambda_i}
      }{
        \nabla f\left(\vt_{\vlambda}\left(\rvvu\right)\right)
      }
      \right\}}^2
    \\
    &=
    \sum_{i} \sum_{j} { \left\{
      \mathbb{E} 
      \inner{
        \frac{\partial \vt_{\vlambda}\left(\rvvu\right)}{\partial \lambda_i}
      }{
        \frac{\partial}{\partial \lambda_j} \nabla f\left(\vt_{\vlambda}\left(\rvvu\right)\right)
      }
      +
      \inner{
        \frac{\partial^2 \vt_{\vlambda}\left(\rvvu\right)}{\partial \lambda_i \partial \lambda_j}
      }{
        \nabla f\left(\vt_{\vlambda}\left(\rvvu\right)\right)
      }
      \right\} }^2
  \end{alignat}
\end{proof}

%%% Local Variables:
%%% TeX-master: "master"
%%% End:


% \vspace{-2.5ex}
% \begin{proof}
%     The derivative of the softplus function is the sigmoid function \(\mathrm{sigmoid}\left(x\right) = 1/\left(1 + e^{-x}\right) \leq 1\).
%     By the mean-value theorem, any function that has bounded derivatives \( \phi^{\prime}\left(x\right) < L\) for all \(x \in \mathbb{R}\) is \(L\)-Lipschitz.
% \end{proof}

%\vspace{-2ex}
\subsection{Key Lemmas}
%\vspace{-.5ex}
The main challenge in studying BBVI is that the gradient of the composed function \(\nabla_{\vlambda} f \left( \vt_{\vlambda} \left( \vu \right) \right)\) is different from \(\nabla f\).
For the matrix square root parameterization, \citet[Lemma 1]{domke_provable_2019} established the connection through \cref{thm:variational_gradient_norm_identity}.
We generalize this result to nonlinear parameterizations:


\begin{theoremEnd}[\keylemmaproofoption,category=upperboundkeylemmagradientnormidentity]{lemma}\label{thm:general_variational_gradient_norm_identity}
  Let \(\vt_{\vlambda}: \mathbb{R}^d \rightarrow \mathbb{R}^d\) be a location-scale reparameterization function (\cref{def:reparam}) with some differentiable function \(f : \mathbb{R}^d \rightarrow \mathbb{R} \).
  Then, for \(\vg_{f} \triangleq \nabla f\left( \vt_{\vlambda}\left(\vu\right) \right)\), 
  \begin{enumerate}[label=(\roman*)]
    \vspace{-2ex}
    \setlength\itemsep{-1ex}
    \item Mean-Field
      \vspace{-1ex}
    {\small
    \setlength{\belowdisplayskip}{1ex} \setlength{\belowdisplayshortskip}{1ex}
    \setlength{\abovedisplayskip}{1ex} \setlength{\abovedisplayshortskip}{1ex}
    \begin{alignat*}{2}
      \hspace{-3em}
      \norm{ \nabla_{\vlambda} f\left( \vt_{\vlambda}\left(\vu\right) \right) }_2^2
      = 
      {\lVert \vg_{f} \rVert}_2^2
      +
      \vg_{f}^{\top}
      \mU \mPhi
      \vg_{f},
      \qquad\qquad
    \end{alignat*}
  }

    \item Cholesky
      {\small%
    {
    \setlength{\belowdisplayskip}{1ex} \setlength{\belowdisplayshortskip}{1ex}
    \setlength{\abovedisplayskip}{1ex} \setlength{\abovedisplayshortskip}{1ex}
    \begin{alignat*}{2}
      \norm{ \nabla_{\vlambda} f\left( \vt_{\vlambda}\left(\vu\right) \right) }_2^2
      &=
      {\lVert \vg_{f} \rVert}_2^2 + \vg_{f}^{\top} \mSigma \vg_{f}
      +
      \vg_{f}^{\top}
      \mU
      \left(
      \mPhi
      - 
      \boldupright{I}
      \right)
      \vg_{f},
    \end{alignat*}
      }%
      %\hspace{-2ex}
  }
  \end{enumerate}
  where \(\mU,\mPhi,\mSigma\) are diagonal matrices, which the diagonals are defined as 
  {
    \setlength{\belowdisplayskip}{.5ex} \setlength{\belowdisplayshortskip}{.5ex}
    \setlength{\abovedisplayskip}{.5ex} \setlength{\abovedisplayshortskip}{.5ex}
    \[
    U_{ii} = u_i^2,\quad
    \Phi_{ii} = {\phi^{\prime}\left(s_i\right)}^2,\quad
    \Sigma_{ii} = {\textstyle\sum^{i}_{j=1}} u_j^2,
    \]
  }
  %\vspace{-2ex}
  and \(\phi\) is a diagonal conditioner for the scale matrix.
\end{theoremEnd}
\vspace{-2ex}
\begin{proofEnd}
  The proof starts by applying the Chain Rule and then computing the quadratic norm of the gradient as
  \begin{alignat}{2}
    &\norm{\nabla_{\vlambda} f\left( \vt_{\vlambda}\left(\vu\right) \right) }_2^2
    \nonumber
    \\
    &\;= 
    {\left(
      \frac{
        \partial \vt_{\vlambda}\left(\vu\right)
      }{
        \partial \vlambda
      }
      \nabla f\left( \vt_{\vlambda}\left(\vu\right) \right)
    \right)}^{\top}
    \frac{
      \partial \vt_{\vlambda}\left(\vu\right)
    }{
      \partial \vlambda
    }
    \nabla f\left( \vt_{\vlambda}\left(\vu\right) \right)
    \nonumber
    \\
    &\;=
    {\nabla f^{\top}\left( \vt_{\vlambda}\left(\vu\right) \right)}
    {\left(
      \frac{
        \partial \vt_{\vlambda}\left(\vu\right)
      }{
        \partial \vlambda
      }
    \right)}^{\top}
    \frac{
      \partial \vt_{\vlambda}\left(\vu\right)
    }{
      \partial \vlambda
    }
    \nabla f\left( \vt_{\vlambda}\left(\vu\right) \right)
    \nonumber
    \\
    &\;=
    {\vg_{f}^{\top}}
    {\left(
      \frac{
        \partial \vt_{\vlambda}\left(\vu\right)
      }{
        \partial \vlambda
      }
    \right)}^{\top}
    \frac{
      \partial \vt_{\vlambda}\left(\vu\right)
    }{
      \partial \vlambda
    }
    \vg_{f}.\label{eq:variational_gradient_norm_identity_eq1}
  \end{alignat}
  Naturally, the derivative of the reparameterization function will depend on the specific parameterization used.

  \paragraph{Proof for Cholesky}

  Let \(p\) denote the number of scalar variational parameters such that \(\vlambda = (\lambda_1, \ldots, \lambda_p)\).
  Then,
  \begin{alignat*}{2}
    &{\left(
      \frac{
        \partial \vt_{\vlambda}\left(\vu\right)
      }{
        \partial \vlambda
      }
    \right)}^{\top}
    \frac{
      \partial \vt_{\vlambda}\left(\vu\right)
    }{
      \partial \vlambda
    }
    \\
    &\;=
    \sum^{d}_{i=1} 
    \frac{
      \partial \vt_{\vlambda}\left(\vu\right)
    }{
      \partial m_i
    }
    {\left(
    \frac{
      \partial \vt_{\vlambda}\left(\vu\right)
    }{
      \partial m_i
    }
    \right)}^{\top}
    +
    \sum^{d}_{i=1} 
    \sum^{d}_{j \leq i} 
    \frac{
      \partial \vt_{\vlambda}\left(\vu\right)
    }{
      \partial \lambda_{C_{ij}}
    }
    {\left(
    \frac{
      \partial \vt_{\vlambda}\left(\vu\right)
    }{
      \partial \lambda_{C_{ij}}
    }
    \right)}^{\top},
  \end{alignat*}
  where \(\lambda_{C_{ij}}\) denote the parameter responsible for the \(ij\)-th entry of \(\mC\), \(C_{ij}\).
  Notice that, unlike for the matrix square root parameterization~\citep{domke_provable_2019}, the sum for \(C_{ij}\) is only over the lower triangular section.

  For the derivatives with respect to \(m_i\) and \(C_{ij}\), \citet{domke_provable_2020, domke_provable_2019} show that
  \begin{alignat}{2}
    \frac{\partial \vt_{\vlambda}\left(\vu\right) }{ \partial m_i }   &= \boldupright{e}_i \quad
    \frac{\partial \vt_{\vlambda}\left(\vu\right) }{ \partial C_{ij} } &= \boldupright{e}_i u_j,\label{eq:covariance_derivative}
  \end{alignat}
  where \(\boldupright{e}_i\) is the unit basis of the \(i\)th component.

  Therefore,
  \begin{alignat}{2}
    &{\left(
      \frac{
        \partial \vt_{\vlambda}\left(\vu\right)
      }{
        \partial \vlambda
      }
    \right)}^{\top}
    \frac{
      \partial \vt_{\vlambda}\left(\vu\right)
    }{
      \partial \vlambda
    }
    \nonumber
    \\
    &\;=
    \sum^{d}_{i=1} 
    \boldupright{e}_i
    \boldupright{e}_i^{\top}
    +
    \sum^{d}_{i=1} 
    \sum_{j \leq i} 
    \frac{
      \partial \vt_{\vlambda}\left(\vu\right)
    }{
      \partial \lambda_{C_{ij}}
    }
    {\left(
    \frac{
      \partial \vt_{\vlambda}\left(\vu\right)
    }{
      \partial \lambda_{C_{ij}}
    }
    \right)}^{\top}
    \nonumber
    \\
    &\;=
    \boldupright{I}
    +
    \underbrace{
    \sum^{d}_{i=1} 
    \frac{
      \partial \vt_{\vlambda}\left(\vu\right)
    }{
      \partial \lambda_{C_{ii}}
    }
    {\left(
    \frac{
      \partial \vt_{\vlambda}\left(\vu\right)
    }{
      \partial \lambda_{C_{ii}}
    }
    \right)}^{\top}
    }_{\text{diagonal of \(\mC\)}}
    \nonumber
    \\
    &\qquad+
    \underbrace{
    \sum^{d}_{i=1} 
    \sum_{j < i} 
    \frac{
      \partial \vt_{\vlambda}\left(\vu\right)
    }{
      \partial \lambda_{C_{ij}}
    }
    {\left(
    \frac{
      \partial \vt_{\vlambda}\left(\vu\right)
    }{
      \partial \lambda_{C_{ij}}
    }
    \right)}^{\top}}_{\text{off-diagonal of \(\mC\)}},
    \label{eq:thm:variational_gradient_norm_identity_eq2}
  \end{alignat}
  leaving us with the derivatives of the scale term.

  The gradient with respect to \(\lambda_{C_{ij}}\), however, depends on the parameterization.
  That is,
  \begin{alignat}{2}
    \frac{
      \partial \vt_{\vlambda}\left(\vu\right)
    }{
      \partial \lambda_{C_{ij}}
    }
    &=
    \frac{
      \partial \vt_{\vlambda}\left(\vu\right)
    }{
      \partial C_{ij}
    }
    \frac{
      \partial C_{ij}
    }{
      \partial \lambda_{C_{ij}}
    }
    &=
    \boldupright{e}_i u_j
    \frac{
      \partial C_{ij}
    }{
      \partial \lambda_{C_{ij}}
    }. \label{eq:thm:variational_gradient_norm_identity_covderivative}
  \end{alignat}

  For the diagonal elements, \(\lambda_{C_{ii}} = s_i\).
  Thus,
  \begin{align}
    \frac{
      \partial C_{ii}
    }{
      \partial s_i
    }
    =
    \frac{
      \partial \phi\left(s_i\right)
    }{
      \partial s_i
    }
    =
    \phi^{\prime}\left(s_i\right). \label{eq:thm:variational_gradient_norm_identity_diag}
  \end{align}
  And for the off-diagonal elements, \(\lambda_{L_{ij}} = L_{ij}\), and
  \begin{align}
    \frac{
      \partial C_{ij}
    }{
      \partial L_{ij}
    }
    =
    1. \label{eq:thm:variational_gradient_norm_identity_offdiag}
  \end{align}

  Plugging \cref{eq:thm:variational_gradient_norm_identity_diag,eq:thm:variational_gradient_norm_identity_offdiag,eq:thm:variational_gradient_norm_identity_covderivative} into \cref{eq:thm:variational_gradient_norm_identity_eq2},
  \begin{alignat}{2}
    &{\left(
      \frac{
        \partial \vt_{\vlambda}\left(\vu\right)
      }{
        \partial \vlambda
      }
    \right)}^{\top}
    \frac{
      \partial \vt_{\vlambda}\left(\vu\right)
    }{
      \partial \vlambda
    }
    \nonumber
    \\
    &\;=
    \boldupright{I}
    +
    \underbrace{
    \sum^{d}_{i=1} 
    {\left( u_i \phi^{\prime}\left(s_i\right) \right)}^2
    \boldupright{e}_i \boldupright{e}_i^{\top}
    }_{\text{diagonal of \(\mC\)}}
    +
    \underbrace{
    \sum^{d}_{i=1} 
    \sum_{j=1, j < i} 
    u_j^2 \, \boldupright{e}_i \boldupright{e}_i^{\top}
    }_{\text{off-diagonal of \(\mC\)}}
    \nonumber
    \\
    &\;=
    \boldupright{I}
    +
    \underbrace{
    \sum^{d}_{i=1} 
    u_i^2 {\left(\phi^{\prime}\left(s_i\right) \right)}^2
    \boldupright{e}_i \boldupright{e}_i^{\top}
    }_{\text{diagonal of \(\mC\)}}
    +
    \underbrace{
    \sum^{d}_{i=1} 
    \sum_{j \leq i} 
    u_j^2 \, \boldupright{e}_i \boldupright{e}_i^{\top}
    -
    \sum^{d}_{i=1} 
    u_i^2 \, \boldupright{e}_i \boldupright{e}_i^{\top}
    }_{\text{off-diagonal of \(\mC\)}}
    \nonumber
    \\
    &\;=
    \boldupright{I}
    +
    \underbrace{
      \mU \, \mPhi
    }_{\text{diagonal of \(\mC\)}}
    +
    \underbrace{
    \mSigma
    -
    \mU
    }_{\text{off-diagonal of \(\mC\)}}
    \nonumber
    \\
    &\;=
    \left( \boldupright{I} + \mSigma \right)
    +
    \mU \left( \mPhi - \boldupright{I} \right), \label{eq:variational_gradient_norm_identity_jacinner}
  \end{alignat}
  where \(\mU,\mPhi,\mSigma\) are diagonal matrices defined as
  \begin{alignat*}{2}
    &
    \mPhi
    &&=
    \mathrm{diag}\left(
    \left[ {\phi^{\prime}\left(s_1\right)}^2, \ldots , {\phi^{\prime}\left(s_d\right)}^2 \right]
    \right)
    \\
    &\mU
    &&=
    \mathrm{diag}\left(
    \left[ u_1^2, \ldots, u_d^2 \right]
    \right)
    \\
    &\mSigma
    &&=
    \mathrm{diag}\left(
    \left[ u_1^2, u_1^2 + u_2^2,\, \ldots\,, {\textstyle\sum^{d}_{i=1} u_i^2} \right]
    \right).
  \end{alignat*}
  The major difference with the proof of \citet[Lemma 8]{domke_provable_2019} for the matrix square root case is that we only sum the \(u_j^2 \boldupright{e}_i \boldupright{e}_i^{\top}\) terms over the \textit{lower diagonal elements}.
  This is the variance reduction effect we get from using the Cholesky parameterization.

  Coming back to \cref{eq:variational_gradient_norm_identity_eq1}, 
  \begin{alignat}{2}
    &\norm{\nabla_{\vlambda} f\left( \vt_{\vlambda}\left(\vu\right) \right)}_2^2
    \nonumber
    \\
    \;&=
      \vg_f^{\top}
      {\left(
        \frac{
          \partial \vt_{\vlambda}\left(\vu\right)
        }{
          \partial \vlambda
        }
        \right)}^{\top}
      \frac{
        \partial \vt_{\vlambda}\left(\vu\right)
      }{
        \partial \vlambda
      }
      \vg
    \nonumber
    \\
    &=
      \vg_{f}^{\top}
      \Big(
        \left( \boldupright{I} + \mSigma \right)
        +
        \mU \left( \mPhi - \boldupright{I} \right)
      \Big)
      \vg_{f}
    \nonumber
    \\
    &=
    {\lVert \vg_{f} \rVert}_2^2
    +
    \vg_{f}^{\top} \mSigma \vg_{f}
    +
    \vg_{f}^{\top}
    \mU \left( \mPhi - \boldupright{I} \right)
    \vg_{f}.
    \label{eq:variational_gradient_norm_identity_conclusion}
  \end{alignat}

  \paragraph{Proof for Mean-field}
  For the mean-field variational family, the covariance has only diagonal elements.
  Therefore, \cref{eq:variational_gradient_norm_identity_jacinner} becomes
  \begin{alignat}{2}
    {\left(
      \frac{
        \partial \vt_{\vlambda}\left(\vu\right)
      }{
        \partial \vlambda
      }
    \right)}^{\top}
    \frac{
      \partial \vt_{\vlambda}\left(\vu\right)
    }{
      \partial \vlambda
    }
    &=
    \boldupright{I} + \mU \mPhi,
    \nonumber
  \end{alignat}
  and \cref{eq:variational_gradient_norm_identity_conclusion} becomes
  \begin{alignat}{2}
    \norm{\nabla_{\vlambda} f\left( \vt_{\vlambda}\left(\vu\right) \right)}_2^2
    =
      \vg_{f}^{\top}
      \left(
        \boldupright{I}
        +
        \mU \mPhi
      \right)
      \vg_{f}
    =
    {\lVert \vg_{f} \rVert}_2^2
    +
    \vg_{f}^{\top}
    \mU \mPhi
    \vg_{f}.
    \nonumber
  \end{alignat}
\end{proofEnd}

Note that the relationships in this lemma are all equalities, which can be bounded with known quantities, as done in the next lemma.
We note here that if any of our analyses were to be improved, this shall by done by obtaining tighter bounds on the equalities in \cref{thm:general_variational_gradient_norm_identity}.

\begin{theoremEnd}[\keylemmaproofoption,category=upperboundkeylemmagradientnormbound]{lemma}\label{thm:general_variational_gradient_norm_bound}
Let \(\vt_{\vlambda}: \mathbb{R}^d \rightarrow \mathbb{R}^d\) be a location-scale reparameterization function (\cref{def:reparam}), \(f : \mathbb{R}^d \rightarrow \mathbb{R} \) be a differentiable function, and let \(\phi\) satisfy \cref{assumption:phi_lipschitz}.
  \vspace{-5ex}
  \begin{enumerate}[label=(\roman*)]
    \setlength\itemsep{-1ex}
    \item Mean-Field
    {%
    \setlength{\belowdisplayskip}{1ex} \setlength{\belowdisplayshortskip}{1ex}%
    \setlength{\abovedisplayskip}{1ex} \setlength{\abovedisplayshortskip}{1ex}%
      \[
        \norm{\nabla_{\vlambda} f\left( \vt_{\vlambda}\left(\vu\right) \right)}_2^2
        \leq
        \left(1 + \norm{ \mU }_{\mathrm{F}} \right)
        {\lVert \nabla f\left( \vt_{\vlambda}\left(\vu\right) \right) \rVert}_2^2,
      \]
      where \(\mU\) is a diagonal matrix such that \(U_{ii} = u_i^2\).
    }%
    \item Cholesky
    {%
    \setlength{\belowdisplayskip}{1ex} \setlength{\belowdisplayshortskip}{1ex}%
    \setlength{\abovedisplayskip}{1ex} \setlength{\abovedisplayshortskip}{1ex}%
      \[
        \norm{\nabla_{\vlambda} f\left( \vt_{\vlambda}\left(\vu\right) \right)}_2^2
        \leq
        \left(1 + \norm{ \vu }_2^2 \right)
        {\lVert \nabla f\left( \vt_{\vlambda}\left(\vu\right) \right) \rVert}_2^2,
      \]
      }%
      where the equality holds for the matrix square root parameterization.
  \end{enumerate}
\end{theoremEnd}
\vspace{-2ex}
\begin{proofEnd}
  The proof continues from the result of \cref{thm:general_variational_gradient_norm_identity}.

  \paragraph{Proof for Cholesky}
  \cref{thm:general_variational_gradient_norm_identity} shows that
  \begin{alignat*}{2}
    \norm{ \nabla_{\vlambda} f\left( \vt_{\vlambda}\left(\vu\right) \right) }_2^2
    = 
    {\lVert \vg_{f} \rVert}_2^2
    +
    \vg_{f}^{\top} \mSigma \vg_{f}
    +
    \vg_{f}^{\top}
    \mU \left( \mPhi - \boldupright{I} \right)
    \vg_{f},
  \end{alignat*}
  where \(\vg_f = \nabla f\left(\vt_{\vlambda}\left(\vu\right)\right) \).

  By the 1-Lipschitz assumption, the entries of the diagonal matrix \(\Phi\) satisfy
  \begin{align*}
    \Phi_{ii} = {\phi^{\prime}\left(d_i\right)}^2 \leq 1,
  \end{align*}
  which means
  \begin{alignat*}{2}
    \mPhi \preceq \boldupright{I}
    \;\Rightarrow\;
    \mU \left( \mPhi - \boldupright{I} \right)
    \preceq
    0
    \;\Rightarrow\;
    {\vg_{f}}^{\top} \mU \left( \mPhi - \boldupright{I} \right) \vg_{f} \leq 0.
  \end{alignat*}
  Therefore, for the full-rank Cholesky parameterization and a 1-Lipschitz conditioner \(\phi\),
  \begin{alignat*}{2}
    &\norm{\nabla_{\vlambda} f\left( \vt_{\vlambda}\left(\vu\right) \right)}_2^2
    \\
    &\;=
    {\lVert \nabla f\left( \vt_{\vlambda}\left(\vu\right) \right) \rVert}_2^2
    +
    {\vg_{f}}^{\top}
    \mSigma
    \vg_{f}
    +
    {\vg_{f}}^{\top} \mU \left( \mPhi - \boldupright{I} \right) \vg_{f}
    \\
    &\;\leq
    {\lVert \nabla f\left( \vt_{\vlambda}\left(\vu\right) \right) \rVert}_2^2
    +
    {\vg_{f}}^{\top}
    \mSigma
    \vg_{f}
    \\
    &\;\leq
    {\lVert \nabla f\left( \vt_{\vlambda}\left(\vu\right) \right) \rVert}_2^2
    +
    \norm{ \mSigma }_{2,2} {\lVert \nabla f\left( \vt_{\vlambda}\left(\vu\right) \right) \rVert}_2^2
    \\
    &\;=
    {\lVert \nabla f\left( \vt_{\vlambda}\left(\vu\right) \right) \rVert}_2^2
    +
    \left( \sum^{d}_{i=1} u_{i}^2 \right)  {\lVert \nabla f\left( \vt_{\vlambda}\left(\vu\right) \right) \rVert}_2^2
    \\
    &\;=
    \left(1 + \norm{\vu}^2_2\right) {\lVert \nabla f\left( \vt_{\vlambda}\left(\vu\right) \right) \rVert}_2^2,
  \end{alignat*}
  where \(\norm{\mU}_{2,2}\) is the \(L_2\) operator norm of \(\mU\).
  This upper bound coincides with that of the matrix square root parameteration.
  Thus, unforunately, this bound fails to acknowledge the lower variance of the Cholesky parameterization, coinciding with that of the matrix square root parameterization.

  \paragraph{Proof for Mean-field (\cref{def:meanfield})}
  For the mean-field parameterization,~\cref{thm:general_variational_gradient_norm_identity} shows that
  \begin{alignat*}{2}
    \norm{ \nabla_{\vlambda} f\left( \vt_{\vlambda}\left(\vu\right) \right) }_2^2
    =
    {\lVert \vg_{f} \rVert}_2^2
    +
    \vg_{f}^{\top}
    \mU \mPhi
    \vg_{f}.
  \end{alignat*}

  For the second term,  
  \begin{alignat*}{2}
    \vg_{f}^{\top}
    \mU \mPhi
    \vg_{f}
    \leq
    {\lVert \mU \rVert}_{2,2} {\lVert \mPhi \rVert}_{2,2}
    {\lVert \vg_{f} \rVert}^2_2.
  \end{alignat*}
  By the \(1\)-Lipschitzness of \(\phi\),
  \[
    {\lVert \mPhi \rVert}_{2,2}
    = \sigma_{\mathrm{max}}\left( \mPhi \right)
    = \max_{i = 1, \ldots, d} {\phi^{\prime}\left( s_i \right)}^2
    \leq 1.
  \]
  Then,
  \begin{alignat}{2}
    \vg_{f}^{\top}
    \left( \mU \mPhi \right)
    \vg_{f}
    &\leq
    {\lVert \mU \rVert}_{2,2} \,
    {\lVert \vg_{f} \rVert}^2_2 \label{eq:variational_gradient_norm_identity_mf_eq1}
    \\
    &\leq
    {\lVert \mU \rVert}_{\mathrm{F}} \,
    {\lVert \vg_{f} \rVert}^2_2, \label{eq:variational_gradient_norm_identity_mf_eq2}
  \end{alignat}
  which gives the result.
  Here, unlike the bounds on \(\mPhi\), the bounds in \cref{eq:variational_gradient_norm_identity_mf_eq1,eq:variational_gradient_norm_identity_mf_eq2} are quite loose, and become looser as the dimensionality increases.

%%   We conclude as
%%   \begin{alignat*}{2}
%%     \norm{ \nabla_{\vlambda} f\left( \vt_{\vlambda}\left(\vu\right) \right) }_2^2
%%     \leq
%%     \norm{ \nabla f\left(\vt_{\vlambda}\left(\vu\right)\right) }_2^2
%%     + \frac{1}{2}{\lVert \vg_{f} \rVert}_2^2 + \frac{1}{2}\norm{\vu}_2^2
%%     =
%%     \frac{3}{2} \norm{ \nabla f\left(\vt_{\vlambda}\left(\vu\right)\right) }_2^2
%%     +
%%     \frac{1}{2} \norm{\vu}_2^2
%%   \end{alignat*}
\end{proofEnd}


%%% Local Variables:
%%% TeX-master: "main"
%%% End:


\cref{thm:general_variational_gradient_norm_identity} act as the interface between the properties of the parameterization and the likelihood \(f\).

\begin{remark}[\textbf{Variance Reduction Through \(\phi\)}]
  A \textit{nonlinear} Cholesky parameterization with a 1-Lipschitz \(\phi\) achieves lower or equal variance compared to the matrix square root and \textit{linear} Cholesky, where the equality is achieved with the matrix square root parameterization.
\end{remark}

\vspace{-1.5ex}%
\paragraph{Dimension Dependence of Mean-Field}
The superior dimensional dependence of the mean-field parameterization is given by the following lemma:


\begin{theoremEnd}[\keylemmaproofoption,category=upperboundkeylemmameanfield]{lemma}\label{thm:meanfield_u_identity}
  Let the assumptions of~\cref{thm:general_variational_gradient_norm_bound} hold and \(\rvvu \sim \varphi\) satisfy \cref{assumption:symmetric_standard}.
  Then, for the mean-field parameterization,
    {
    \setlength{\belowdisplayskip}{1.5ex} \setlength{\belowdisplayshortskip}{1.5ex}
    \setlength{\abovedisplayskip}{1.5ex} \setlength{\abovedisplayshortskip}{1.5ex}
  \begin{alignat*}{2}
    &\mathbb{E}\norm{\vt_{\vlambda}\left(\rvvu\right) - \vz}_2^2 \left(1 + \norm{\mathbfsfit{U}}_{\mathrm{F}} \right)
    \\
    &\quad\leq
    \left(\sqrt{d \kappa} + \kappa\sqrt{d} + 1\right) \, \norm{ \vm - \vz }_2^2
    +
    \left(2 \kappa \sqrt{d} + 1\right)
    \norm{\mC}_{\mathrm{F}}^2.
  \end{alignat*}
  }
\end{theoremEnd}
\vspace{-1ex}
\begin{proofEnd}
  The key idea is to prove a similar result as \cref{thm:reparam_u_identity}, but with better constants to reflect that the mean-field parameterization has a lower variance. 

  First,
  \begin{align}
    &\mathbb{E}\norm{\vt_{\vlambda}\left(\rvvu\right) - \vz}_2^2 \, \left( 1 + \norm{\mathbfsfit{U}}_{\mathrm{F}} \right) \nonumber \\
    &\;=
    \mathbb{E}\norm{\vt_{\vlambda}\left(\rvvu\right) - \vz}_2^2 
    + \mathbb{E} \norm{\mathbfsfit{U}}_{\mathrm{F}} \, \norm{\vt_{\vlambda}\left(\rvvu\right) - \vz}_2^2, 
    \nonumber
\shortintertext{applying \cref{thm:reparam_quadratic},}
    &\;=
    \norm{ \vm - \vz }_2^2 + \norm{\mC}_{\mathrm{F}}
    +
    \mathbb{E} \norm{\mathbfsfit{U}}_{\mathrm{F}} \, \norm{\vt_{\vlambda}\left(\rvvu\right) - \vz}_2^2.
    \label{thm:meanfield_eq0}
  \end{align}
  
  The last term decomposes as 
  \begin{alignat}{2}
    \mathbb{E} \norm{\mathbfsfit{U}}_{\mathrm{F}} \norm{\vt_{\vlambda}\left(\rvvu\right) - \vz}_2^2
    \nonumber
    &=
    \underbrace{
      \mathbb{E} \norm{\mathbfsfit{U}}_{\mathrm{F}} \, \rvvu^{\top} \mC^{\top} \mC \rvvu 
    }_{\text{Term \ding{182}}}
    \nonumber
    \\
    &\quad+
    2\, 
    \underbrace{
      \mathbb{E} \norm{\mathbfsfit{U}}_{\mathrm{F}} \, \rvvu^{\top} \mC^{\top} \left( \vm - \vz \right)
    }_{\text{Term \ding{183}}}
    \nonumber
    \\
    &\quad+
    \underbrace{\mathbb{E} \norm{\mathbfsfit{U}}_{\mathrm{F}}}_{\text{Term \ding{184}}} \, \norm{ \vm - \vz }_2^2.
    \label{thm:meanfield_eq1}
  \end{alignat}
  We will now focus on the stochastic terms \ding{182}-\ding{184} one by one.
  
  First, for Term \ding{182}, notice that the mean-field parameterization implies that \(\mC = \mathrm{diag}\left( c_1, \ldots, c_d \right)\).
  Thus, 
  \begin{alignat}{2}
    \mathbb{E} \norm{\mathbfsfit{U}}_{\mathrm{F}} \,  \rvvu^{\top} \mC^{\top} \mC \rvvu 
    &=
    \mathbb{E} \left( \sqrt{ \sum_{i=1}^d \rvu_i^4 } \right) \left( \sum_{i=1}^d c_i^2 \, u_i^2 \right)
    \nonumber
    \\
    &=
    \sum_{i=1}^d c_i^2 \, \mathbb{E} \left( \sqrt{ \sum_{j=1}^d \rvu_j^4 } \right) \rvu_i^2,
    \nonumber
\shortintertext{applying Cauchy-Schwarz inequality for expectations,}
    &\leq
    \sum_{i=1}^d c_i^2 \sqrt{ \left( \mathbb{E} \sum_{j=1}^d \rvu_j^4  \right) \left( \mathbb{E}  \rvu_i^4 \right) }
    \nonumber
\shortintertext{and given \cref{assumption:symmetric_standard},}
    &\;=
    \sum_{i=1}^d c_i^2 \, \sqrt{ d \kappa^2 }
    \nonumber
    \\
    &\;=
    \kappa \sqrt{d} \, \norm{ \mC }_{\mathrm{F}}^2.
    \label{thm:meanfield_eq5}
  \end{alignat}
  
  Term \ding{183} can be bounded as 
  \begin{alignat}{2}
    &\mathbb{E} 
      \norm{\mathbfsfit{U}}_{\mathrm{F}} \, \rvvu^{\top} 
    \mC^{\top} \left( \vm - \vz \right) 
    \nonumber
\shortintertext{using the Cauchy-Schwarz inequality for vectors as}
    &\;\leq
    %
    \mathbb{E} \norm{\mathbfsfit{U}}_{\mathrm{F}} \, \norm{\mC \rvvu}_2 \norm{ \vm - \vz }_2,
    \nonumber
\shortintertext{again, applying the inequality for expectations,}
    &\;=
    \sqrt{
    \mathbb{E}
    \norm{\mathbfsfit{U}}_{\mathrm{F}}^2 \,
    \mathbb{E}
    \norm{\mC \rvvu}_2^2 
    } \,
    \norm{ \vm - \vz }_2
    \nonumber
    \\
    &\;=
    \sqrt{
    \mathbb{E}
    \left(
    \sum^d_{i=1} \rvu_i^4  
    \right) \,
    \mathrm{tr}\left(\mC^{\top} \mC \, \mathbb{E} \rvvu \rvvu^{\top} \right)
    } \,
    \norm{ \vm - \vz }_2,
    \nonumber
\shortintertext{from \cref{assumption:symmetric_standard},}
    &\;=
    \sqrt{
    d \kappa \,
    \mathrm{tr}\left(\mC^{\top} \mC \right)
    } \,
    \norm{ \vm - \vz }_2
    \nonumber
    \\
    &\;=
    \sqrt{d \kappa} \,
    \norm{\mC}_{\mathrm{F}} \,
    \norm{ \vm - \vz }_2
     \nonumber
    \\
    &\;=
    \sqrt{d \kappa} \,
    \sqrt{ \norm{\mC}_{\mathrm{F}}^2 \,
           \norm{ \vm - \vz }_2^2 },
    \nonumber
\shortintertext{and by the arithmetic mean-geometric mean inequality,}
    &\;=
    \frac{\sqrt{d \kappa}}{2}
    \left(
      \norm{\mC}_{\mathrm{F}}^2
      +
      \norm{ \vm - \vz }_2^2
    \right).
    \label{thm:meanfield_eq3}
  \end{alignat}

  Finally, Term \ding{184} follows as
  \begin{alignat}{2}
    \mathbb{E} \norm{\mathbfsfit{U}}_{\mathrm{F}}
    &=
    \mathbb{E} \sqrt{ \sum_{i=1}^d \rvu_i^4  },
    \nonumber
\shortintertext{using Jensen's inequality,}
    &\leq
    \sqrt{ \mathbb{E} \sum_{i=1}^d \rvu_i^4 }
    \nonumber
    \\
    &=
    \sqrt{ d \kappa }.
    \label{thm:meanfield_eq2}
  \end{alignat}

  Combining all the results, \cref{thm:meanfield_eq0} becomes
  \begin{alignat*}{2}
    &\mathbb{E}\norm{\vt_{\vlambda}\left(\rvvu\right) - \vz}_2^2 \, \left( 1 + \norm{\mathbfsfit{U}}_{\mathrm{F}} \right)
    \\
    &\;\leq
    \norm{ \vm - \vz }_2^2 + \norm{\mC}_{\mathrm{F}}^2
    \\ 
    &\;\quad+
    \mathbb{E} \norm{\mathbfsfit{U}}_{\mathrm{F}} \, \rvvu^{\top} \mC^{\top} \mC \rvvu
    \\ 
    &\;\quad+
    2\,\mathbb{E} \norm{\mathbfsfit{U}}_{\mathrm{F}} \, \rvvu^{\top} \mC^{\top} \norm{ \vm - \vz }_2^2
    \\ 
    &\;\quad+
    \mathbb{E} \norm{\mathbfsfit{U}}_{\mathrm{F}} \, \norm{ \vm - \vz }_2^2
\shortintertext{and applying \cref{thm:meanfield_eq2,thm:meanfield_eq3,thm:meanfield_eq5},}
    &\;\leq
    \norm{ \vm - \vz }_2^2 + \norm{\mC}_{\mathrm{F}}^2
    \\ 
    &\;\quad+
    \kappa\sqrt{d} \norm{\mC}_{\mathrm{F}}
    \\ 
    &\;\quad+
    \kappa \sqrt{d} \left(
     \norm{\mC}_{\mathrm{F}}^2 + \norm{\vm - \vz}_2^2
    \right)
    \\ 
    &\;\quad+
    \sqrt{d \kappa} \, \norm{ \vm - \vz }_2^2
    \\ 
    &\;=
    \left(\sqrt{d \kappa} + \kappa\sqrt{d} + 1\right) \, \norm{ \vm - \vz }_2^2
    +
    \left(2 \kappa \sqrt{d} + 1\right)
    \norm{\mC}_{\mathrm{F}}^2.
  \end{alignat*}

\end{proofEnd}

%%% Local Variables:
%%% TeX-master: "main"
%%% End:


\begin{remark}[\textbf{Superior Variance of Mean-Field}]\label{remark:meanfield_superiority}
  The mean-field parameterization has {\small\(\mathcal{O}\left(\sqrt{d}\right)\)} dimensional dependence compared to the \(\mathcal{O}\left(d\right)\) dimensional dependence of the full-rank parameterizations in \cref{thm:reparam_u_identity}.
\end{remark}



\begin{theoremEnd}[category=common]{lemma}[\citealt{domke_provable_2019}, Lemma 9]
\label{thm:u_identities}
  Let \(\rvvu = \left(\rvu_1, \rvu_2, \ldots, \rvu_d\right)\) be a \(d\)-dimensional vector-valued random variable with zero-mean independently and identically distributed components.
  Then,
  \begin{alignat*}{2}
    &\mathbb{E}\rvvu \rvvu^{\top} &&= \left( \mathbb{E} \rvu_i^2 \right) \boldupright{I}
    \\
    &\mathbb{E}\norm{\rvvu}_2^2 &&= d \, \mathbb{E} \rvu_i^2
    \\
    &\mathbb{E} \rvvu \left( 1 + \norm{\rvvu}_2^2 \right) &&= \left( \mathbb{E} \rvu_i^3 \right) \mathbf{1}
    \\
    &\mathbb{E} \rvvu \rvvu^{\top} \rvvu \rvvu^{\top} &&= \left( \left(d - 1\right) \, {\left( \mathbb{E} \rvu_i^2 \right)}^2 + \mathbb{E}\rvu_i^4 \right) \boldupright{I}.
  \end{alignat*}
\end{theoremEnd}

\begin{theoremEnd}[category=common]{lemma}[\citealt{domke_provable_2019}, Lemma 1]
\label{thm:variational_gradient_norm_identity}
  Let \(\vt_{\vlambda}: \mathbb{R}^d \rightarrow \mathbb{R}^d\) be a location-scale reparameterization function (\cref{def:reparam}).
  Also, let \(f : \mathbb{R}^d \mapsto \mathbb{R} \) be some differentiable function.
  Then,
  \begin{alignat*}{2}
    \norm{\nabla_{\vlambda} f\left( \vt_{\vlambda}\left(\vu\right) \right) }_2^2
    = 
    \norm{\nabla f\left( \vt_{\vlambda}\left(\vu\right) \right) }_2^2 \left(1 + \norm{\vu}_2^2\right).
  \end{alignat*}
\end{theoremEnd}

\begin{theoremEnd}[category=common]{lemma}[\citealt{domke_provable_2019}, Lemma 1]
\label{thm:reparam_u_identity}
  Let \(\vt_{\vlambda}: \mathbb{R}^d \rightarrow \mathbb{R}^d\) be a location-scale reparameterizaiton function (\cref{def:reparam}).
  Also, let \(\vz \in \mathbb{R}^d\) be some vector and \(\rvvu \sim \varphi\) satisfy~\cref{assumption:symmetric_standard}.
  Then,
  \begin{alignat*}{2}
    \mathbb{E} \norm{\vt_{\vlambda}\left(\rvvu\right) - \vz}_2^2 \left(1 + \norm{\rvvu}_2^2\right)
    =
    \left(d+1\right) \norm{\vm - \vz}_2^2 + \left(d + \kappa\right) \norm{\mC}^2_{\mathrm{F}}.
  \end{alignat*}
\end{theoremEnd}

%%% Local Variables:
%%% TeX-master: "main"
%%% End:


Lastly, the following lemma is the basic building block for all of our upper bounds:


\begin{theoremEnd}[\keylemmaproofoption,category=upperboundkeylemmavariancegeneral]{lemma}\label{thm:gradient_variance_general_upper_bound}
  Let \(\vg_{M}\) be the \(M\)-sample gradient estimator of \(F\) (\cref{def:generic_elbo}) for some function \(f,h\) and let \(\rvvu\) be some random variable.
  Then, 
  {%
  \setlength{\belowdisplayskip}{1ex} \setlength{\belowdisplayshortskip}{1ex}%
  \setlength{\abovedisplayskip}{1ex} \setlength{\abovedisplayshortskip}{1ex}%
  \begin{align*}
    \mathbb{E} \norm{\vg_M }^2_2
    \leq
    \frac{1}{M} \mathbb{E}{ \norm{
      \nabla_{\vlambda} f\left(\vt_{\vlambda}\left(\rvvu\right)\right)
      }_2^2
    }
    + \norm{ \nabla F\left(\vlambda\right) }^2_2.
  \end{align*}
  }%
\end{theoremEnd}
\vspace{-1ex}
%% \begin{proofsketch}
%%   We use the affine property of the variance,%
%%   {%
%%   \setlength{\belowdisplayskip}{1ex} \setlength{\belowdisplayshortskip}{1ex}%
%%   \setlength{\abovedisplayskip}{1ex} \setlength{\abovedisplayshortskip}{1ex}%
%%   \begin{align*}
%%     \mathrm{tr}\,\V{\textstyle \frac{1}{M} \sum_{m=1}^M \vg_m }
%%     =
%%     \frac{1}{M} \mathrm{tr}\,\V{
%%       \nabla_{\vlambda} f\left(\vt_{\vlambda}\left(\rvvu\right)\right)
%%     }.
%%   \end{align*}
%%   }%
%%   This erases the effects of the deterministic elements of \(\vg\) (\textit{e.g.}, gradient of the regularization term), and simplifies the sample average.
%%   Also, this is the point where our proof can be further extended to data subsampling.
%% \end{proofsketch}
%\vspace{-2ex}
\begin{proofEnd}
  From the definition of variance,
  \begin{alignat}{2}
    &\mathbb{E} \norm{\vg_M }^2_2
    \nonumber
    \\
    &\;=
    \mathrm{tr}\,\V{ \vg_M } + \norm{\mathbb{E} \vg_M }^2_2,
    \nonumber
\shortintertext{following the definition in \cref{eq:def_gradient_M_est},}
    &\;=
    \mathrm{tr}\,\V{ \frac{1}{M} \sum_{m=1}^M \vg_m } + \norm{ \nabla F\left(\vlambda\right) }^2_2,
    \nonumber
\shortintertext{and then the definition in \cref{eq:def_gradient_m_est},}
    &\;=
    \mathrm{tr}\,\V{
      \frac{1}{M} \sum_{m=1}^M \nabla_{\vlambda} f\left(\vt_{\vlambda}\left(\rvvu_m\right)\right) + \nabla h\left(\vlambda\right)
    }
    + \norm{ \nabla F\left(\vlambda\right) }^2_2,
    \nonumber
\shortintertext{by the linearity of variance,}
    &\;=
    \frac{1}{M} \mathrm{tr}\,\V{
      \nabla_{\vlambda} f\left(\vt_{\vlambda}\left(\rvvu\right)\right)
    }
    + \norm{ \nabla F\left(\vlambda\right) }^2_2
    \nonumber
    \\
    &\;=
    \frac{1}{M} \left(
    \mathbb{E}{ \norm{\nabla_{\vlambda} f\left(\vt_{\vlambda}\left(\rvvu\right)\right)}_2^2 }
    -
    \norm{ \mathbb{E}{ \nabla_{\vlambda} f\left(\vt_{\vlambda}\left(\rvvu\right)\right)} }_2^2
    \right)
    \nonumber
    \\
    &\qquad+ \norm{ \nabla F\left(\vlambda\right) }^2_2
    \label{eq:thm_gradient_variance_general_definition}
    \\
    &\;\leq
    \frac{1}{M} \mathbb{E}{ \norm{
      \nabla_{\vlambda} f\left(\vt_{\vlambda}\left(\rvvu\right)\right)
      }_2^2
    }
    + \norm{ \nabla F\left(\vlambda\right) }^2_2.
    \nonumber
  \end{alignat}
  % The regularization term is neglected since it is a deterministic function.
  % This means the bound holds for both the entropy form and the KL form.
  % The last inequality introduces some looseness when \(\norm{\mathbb{E}{ \nabla f\left(\vt_{\vlambda}\left(\rvvu\right)\right) }}_2^2\) is large.
  % This means that the variance bound is loose initially but will become tighter as we progress.
\end{proofEnd}

%%% Local Variables:
%%% TeX-master: "main"
%%% End:



\begin{theoremEnd}[\lemmaproofoption, category=upperboundlemma]{lemma}\label{thm:reparam_quadratic}
  Let \(\vt_{\vlambda}: \mathbb{R}^d \rightarrow \mathbb{R}^d\) be a location-scale reparameterizaiton function (\cref{def:reparam}).
  Also, let \(\vz \in \mathbb{R}^d\) be some vector, and let \(\rvvu \sim \varphi\) satisfy~\cref{assumption:symmetric_standard}.
  Then,
  \begin{alignat*}{2}
    \mathbb{E}\norm{\vt_{\vlambda}\left(\rvvu\right) - \vz}_2^2
    &=
    \norm{\vm - \vz}^2_2 + \norm{\mC}_{\mathrm{F}}^2.
  \end{alignat*}
\end{theoremEnd}
\begin{proofEnd}
  \begin{alignat}{2}
    \mathbb{E}\norm{ \vt_{\vlambda}\left(\rvvu\right) - \vz }_2^2
    &=
    \mathbb{E}\norm{ \mC \rvvu + \vm - \vz }_2^2
    \nonumber
    \\
    &=
    \mathbb{E} \rvvu^{\top} \mC^{\top} \mC \rvvu + 2\,\mathbb{E} \rvvu^{\top} \mC^{\top} \vm - 2\,\mathbb{E} \rvvu^{\top} \mC^{\top} \vz
    \nonumber
    \\
    &\;\quad + \vm^{\top} \vm - 2\,\vm^{\top} \vz + \vz^{\top} \vz.
    \label{eq:reparam_quadratic_eq1}
  \end{alignat}
  The first three terms follow as
  \begin{alignat*}{2}
    &\mathbb{E} \rvvu^{\top} \mC^{\top} \mC \rvvu + 2\,\mathbb{E} \rvvu^{\top} \mC^{\top} \vm - 2\,\mathbb{E} \rvvu^{\top} \mC^{\top} \vz
    \\
    &\;=
    \mathbb{E} \mathrm{tr}\left(\rvvu^{\top} \mC^{\top} \mC \rvvu\right) + 2\,\mathbb{E} \rvvu^{\top} \mC^{\top} \vm -  2\,\mathbb{E} \rvvu^{\top} \mC^{\top} \vz
    \\
    &\;=
    \mathrm{tr}\left( \mC^{\top} \mC \mathbb{E} \rvvu \rvvu^{\top}\right) + 2\,\mathbb{E} \rvvu^{\top} \mC^{\top} \vm -  2\,\mathbb{E} \rvvu^{\top} \mC^{\top} \vz,
\shortintertext{applying \cref{thm:u_identities},}
    &\;=
    \mathrm{tr}\left( \mC^{\top} \mC \right)
    \\
    &\;=
    \norm{ \mC }_{\mathrm{F}}^2.
  \end{alignat*}
  Applying this to \cref{eq:reparam_quadratic_eq1},
  \begin{alignat*}{2}
    \mathbb{E}\norm{\vt_{\vlambda}\left(\rvvu\right) - \vz}_2^2
    &=
    \vm^{\top} \vm - 2\,\vm^{\top} \vz + \vz^{\top} \vz + \norm{\mC}_{\mathrm{F}}^2
    \\
    &=
    \norm{\vm - \vz}^2_2 + \norm{\mC}_{\mathrm{F}}^2.
  \end{alignat*}
\end{proofEnd}

% \begin{theoremEnd}[\lemmaproofoption, category=upperboundlemma]{lemma}\label{thm:tspace_distance}
%   Let \(\vt_{\vlambda}: \mathbb{R}^d \rightarrow \mathbb{R}^d\) be defined as in \cref{def:reparam} with parameters \(\vlambda = \left(\vm, \mC\right)\) such that \(\vm \in \mathbb{R}^d\) and \(\mC \in \mathbb{R}^{d \times d}\).
%   Also, let \(\rvvu \sim \varphi\) be some vector-valued random variable, where \(\varphi\) is defined as in \cref{assumption:symmetric_standard}.
%   Then,
%   \begin{alignat*}{2}
%     \mathbb{E}\norm{\vt_{\vlambda}\left(\rvvu\right) - \vt_{\vlambda^{\prime}}\left(\rvvu\right)}_2^2
%     &=
%     \norm{\vm - \vm^{\prime}}^2_2 + \norm{\mC - \mC^{\prime}}_{\mathrm{F}}^2.
%   \end{alignat*}
% \end{theoremEnd}
% \begin{proofEnd}
%   \begin{alignat}{2}
%     \mathbb{E}\norm{ \vt_{\vlambda}\left(\rvvu\right) - \vt_{\vlambda^{\prime}}\left(\rvvu\right) }_2^2
%     &=
%     \mathbb{E}\norm{ \left(\mC\rvvu + \vm\right) - \left( \mC^{\prime}\rvvu + \vm^{\prime} \right)  }_2^2
%     \nonumber
%     \\
%     &=
%     \mathbb{E}\norm{ \left(\mC - \mC^{\prime}\right)\rvvu + \left(\vm - \vm^{\prime}\right) }_2^2
%     \nonumber
%     \\
%     &=
%     \mathbb{E}\rvvu^{\top} {\left(\mC - \mC^{\prime}\right)}^{\top} \left(\mC - \mC^{\prime}\right)\rvvu
%     \nonumber
%     \\
%     &\;\quad+ 2 \mathbb{E} \rvvu^{\top} {\left(\mC - \mC^{\prime}\right)}^{\top} {\left(\vm - \vm^{\prime}\right)}
%     \nonumber
%     \\
%     &\;\quad+ {\left(\vm - \vm\right)}^{\top} \left(\vm - \vm^{\prime}\right),
%     \nonumber
% \shortintertext{invoking the trace trick,}
%     &=
%     \mathbb{E} \mathrm{tr}\left( \rvvu^{\top} {\left(\mC - \mC^{\prime}\right)}^{\top} \left(\mC - \mC^{\prime}\right)\rvvu \right)
%     \nonumber
%     \\
%     &\;\quad+ 2 \mathbb{E} \rvvu^{\top} {\left(\mC - \mC^{\prime}\right)}^{\top} {\left(\vm - \vm^{\prime}\right)}
%     \nonumber
%     \\
%     &\;\quad+ {\left(\vm - \vm^{\prime}\right)}^{\top} \left(\vm - \vm^{\prime}\right),
%     \nonumber
% \shortintertext{using the cyclic property of the trace,}
%     &=
%     \mathrm{tr}\left( {\left(\mC - \mC^{\prime}\right)}^{\top} \left(\mC - \mC^{\prime}\right) \mathbb{E} \rvvu \rvvu^{\top}  \right)
%     \nonumber
%     \\
%     &\;\quad+ 2 \mathbb{E} \rvvu^{\top} {\left(\mC - \mC^{\prime}\right)}^{\top} {\left(\vm - \vm\right)}
%     \nonumber
%     \\
%     &\;\quad+ {\left(\vm - \vm^{\prime}\right)}^{\top} \left(\vm - \vm^{\prime}\right),
%     \nonumber
% \shortintertext{and applying \cref{thm:u_identities},}
%     &=
%     \mathrm{tr}\left( {\left(\mC - \mC^{\prime}\right)}^{\top} \left(\mC - \mC^{\prime}\right) \right)
%     \nonumber
%     \\
%     &\;\quad+ {\left(\vm - \vm^{\prime}\right)}^{\top} \left(\vm - \vm^{\prime}\right)
%     \nonumber
%     \\
%     &=
%     \norm{ \mC - \mC^{\prime} }_{\mathrm{F}}^2
%     + \norm{ \vm - \vm^{\prime} }_2^2.
%     \nonumber
%   \end{alignat}
% \end{proofEnd}

%% \begin{theoremEnd}[\lemmaproofoption, category=upperboundlemma]{lemma}\label{thm:tspace_lambdaspace}
%%   If \(\phi : \mathbb{R} \mapsto \mathbb{R}_+\) is a \(1\)-Lipschitz continuous strictly positive function, 
%%   \begin{align*}
%%     \norm{ \vm - \vm^{\prime} }_{2}^2 + \norm{ \mC - \mC^{\prime} }_{\mathrm{F}}^2
%%     \leq
%%     \norm{ \vlambda - \vlambda^{\prime} }_2^2
%%   \end{align*}
%%   holds for both the mean-field (\cref{def:meanfield}) and Cholesky parameterizations (\cref{def:fullrank}),
%% \end{theoremEnd}
%% \begin{proofEnd}
%%   The first term satisfies \(\norm{ \vm - \vm^{\prime} }_{2}^2 = \norm{\vlambda_{\vm} - \vlambda_{\vm^{\prime}}}_2^2\)  by definition.
%%   Therefore, we only need to prove the inequality for the scale term.

%%   First, for the Cholesky (\cref{def:fullrank}),
%%   \begin{align*}
%%     \norm{ \mC - \mC^{\prime} }_{\mathrm{F}}^2
%%     &=
%%     \underbrace{
%%       \norm{ \phi\left(\vs\right) - \phi\left(\vs^{\prime}\right) }_{\mathrm{F}}^2
%%     }_{\text{diagonal of \(\mC,\mC^{\prime}\)}}
%%     +
%%     \underbrace{
%%       \norm{ \mL - \mL^{\prime} }_{\mathrm{F}}^2
%%     }_{\text{off-diagonal of \(\mC,\mC^{\prime}\)}}
%%     \\
%%     &=
%%     \sum^{d}_{i=1} \abs{ \phi\left(s_i\right) - \phi\left(s_i^{\prime}\right) }^2
%%     +
%%     \norm{ \mL - \mL^{\prime} }_{\mathrm{F}}^2,
%% \shortintertext{and by applying 1-Lipschitzness of \(\phi\),}
%%     &\leq
%%     \sum^{d}_{i=1} \abs{ s_i - s_i^{\prime} }^2
%%     +
%%     \norm{ \mL - \mL^{\prime} }_{\mathrm{F}}^2
%%     \\
%%     &=
%%     \norm{ \vs - \vs^{\prime} }_2^2
%%     +
%%     \norm{ \mL - \mL^{\prime} }_{\mathrm{F}}^2.
%%   \end{align*}
%%   Finally,
%%   \begin{align*}
%%     &\norm{ \vm - \vm^{\prime} }_{2}^2 + \norm{ \mC - \mC^{\prime} }_{\mathrm{F}}^2
%%     \\
%%     &\;\leq
%%     \norm{ \vm - \vm^{\prime} }_{2}^2
%%     +
%%     \norm{ \vs - \vs^{\prime} }_2^2
%%     +
%%     \norm{ \mL - \mL^{\prime} }_{\mathrm{F}}^2.
%%     \\
%%     &\;=
%%     \norm{ \vlambda - \vlambda^{\prime} }_2^2.
%%   \end{align*}
%%   The same proof holds for the mean-field case (\cref{def:meanfield}) by neglecting the off-diagonal (\(\mL\)) terms.
%% \end{proofEnd}

%% \begin{theoremEnd}[\keylemmaproofoption, category=upperboundkeylemma]{lemma}\label{thm:reparam_quadratic}
%%   \begin{alignat*}{2}
%%     \mathbb{E}\norm{\nabla f\left(\vt_{\vlambda}\left(\rvvu\right)\right)}_2^2 \left(1 + \norm{\rvvu}_2^2 \right)
%%     \leq
%%     \frac{2\,L^2 }{\mu} \left(d + \kappa\right) \left(  \mathbb{E}f\left(\vt_{\vlambda}\left(\rvvu\right)\right) - f^* \right)
%%   \end{alignat*}
%% \end{theoremEnd}
%% \begin{proofEnd}
%%   \begin{alignat}{2}
%%     &\mathbb{E}\norm{\nabla f\left(\vt_{\vlambda}\left(\rvvu\right)\right)}_2^2 \left(1 + \norm{\rvvu}_2^2 \right)
%%     \nonumber
%%     \\
%%     &\;=
%%     L^2\left(\left(d + 1\right) \norm{\vm - \bar{\vz}}_2^2 + \left(d + \kappa\right) \norm{\mC}_{\mathrm{F}}^2 \right)
%%     \nonumber
%% \shortintertext{since the kurtosis satisfies \(\kappa \geq 1\),}
%%     &\;\leq
%%     L^2\left(d + \kappa\right) \left( \norm{\vm - \bar{\vz}}_2^2 + \norm{\mC}_{\mathrm{F}}^2 \right),
%%     \nonumber
%% \shortintertext{by \cref{thm:reparam_quadratic},}
%%     &\;=
%%     L^2 \left(d + \kappa\right) \mathbb{E} \norm{\vt_{\vlambda}\left(\rvvu\right) - \bar{\vz} }_2^2,
%%     \nonumber
%% \shortintertext{assuming \(f\) satisfies \cref{assumption:quadratic_growth},}
%%     &\;\leq
%%     \frac{2\,L^2 }{\mu} \left(d + \kappa\right) \left(  \mathbb{E}f\left(\vt_{\vlambda}\left(\rvvu\right)\right) - f^* \right).
%%     \nonumber
%%   \end{alignat}
%% \end{proofEnd}

%%% Local Variables:
%%% TeX-master: "main"
%%% End:


%^\vspace{-1ex}
\subsection{Upper Bounds}\label{section:upper_bound}
\vspace{-.5ex}

We restrict our analysis to the class of log-likelihoods that satisfy the following conditions: 
\vspace{.5ex}
\begin{definition}[\textbf{\(L\)-smoothness}]\label{def:L_smoothness}
  A function \(f : \mathbb{R}^d \rightarrow \mathbb{R}\) is \(L\)-smooth if it satisfies the following for all \(\vzeta, \vzeta^{\prime} \in \mathbb{R}^d\):
  {\small%
  \setlength{\belowdisplayskip}{1.ex} \setlength{\belowdisplayshortskip}{1.ex}%
  \setlength{\abovedisplayskip}{1.ex} \setlength{\abovedisplayshortskip}{1.ex}%
  \begin{align*}
    {\lVert \nabla f\left(\vzeta\right) - \nabla f\left(\vzeta^{\prime}\right) \rVert}_2
    \leq
    L \, {\lVert \vzeta - \vzeta^{\prime} \rVert}_2.
  \end{align*}
  }%
\end{definition}
%

\vspace{.5ex}
\begin{definition}[\textbf{Quadratic Functional Growth}]\label{def:quadratic_growth}
  A function \(f : \mathbb{R}^d \rightarrow \mathbb{R}\) is \(\mu\)-quadratically growing if
  {\small%
  \setlength{\belowdisplayskip}{1.ex} \setlength{\belowdisplayshortskip}{1.ex}%
  \setlength{\abovedisplayskip}{1.ex} \setlength{\abovedisplayshortskip}{1.ex}%
  \begin{align*}
   \frac{\mu}{2} {\lVert \vzeta - \bar{\vzeta} \rVert}_2^2
    \leq
    f\left(\vzeta\right) - f^*
  \end{align*}
  }%
  for all \(\vzeta \in \mathbb{R}^d\), where \(\bar{\vzeta} \in \mathbb{R}^d\) is an arbitrary stationary point of \(f\) and \(f^* = \inf_{\vzeta \in \mathbb{R}^d} f\left(\vzeta\right)\).
\end{definition}
\vspace{-1ex}
%
For instance, \(\mu\)-strongly (quasar) convex functions~\citep{hinder_nearoptimal_2020,jin_convergence_2020} satisfy \cref{def:quadratic_growth}, but our analysis does \textit{not} require (quasar) convexity.

Both assumptions are commonly used in SGD.
For studying the gradient variance of BBVI, assuming both smoothness and quadratic growth is weaker than the assumptions of \citet{xu_variance_2019} but stronger than those of~\citet{domke_provable_2019}, who assumed only smoothness.
The additional assumption on growth is necessary to extend his results to establish the \textit{ABC} condition.

For the variational family, we assume the followings:
\vspace{.5ex}
\begin{assumption}\label{assumption:q}
\(q_{\psi,\vlambda}\) is a member of the ADVI family (\cref{def:advi}), where the underlying  \(q_{\vlambda}\) is a member of the location-scale family (\cref{def:family}) with its base distribution \(\varphi\) satisfying \cref{assumption:symmetric_standard}.
\end{assumption}

%% \todo[inline]{
%%   Obtaining a similar upper bound with the condition below is still a major goal.
%%   But we'll probably be able to publish this even if we don't succeed in H\"older-generalizing.
%% }
%% \begin{assumption*}{\textbf{(H\"older-smoothness)}}\label{assumption:holder_smoothness}
%%   A function \(f\) is \((L, \alpha)\)-H\"older-smooth if it satisfies
%%   \begin{align*}
%%     \norm{ \nabla f\left(\vz\right) - \nabla f\left(\vz^{\prime}\right) }_2
%%     \leq
%%     L \, \norm{\vz - \vz^{\prime}}_2^{\alpha}.
%%   \end{align*}
%% \end{assumption*}

%% \begin{assumption*}{\textbf{(H\"older-growth)}}\label{assumption:holder_growth}
%%   A function \(f\) satisfies \((L, \alpha)\)-H\"older-growth if it satisfies
%%   \begin{align*}
%%     \norm{ \vz - \vz^* }_2^{2 \alpha}
%%     \leq
%%     f\left(\vz\right),
%%   \end{align*}
%%   or
%%   \begin{align*}
%%     \norm{ \vz - \vz^* }_2^{1 + \alpha}
%%     \leq
%%     f\left(\vz\right),
%%   \end{align*}
%%   where \(\vz^*\) is a stationary point of \(f\) such that \(\nabla f\left(\vz^*\right) = \mathbf{0}\).
%% \end{assumption*}

\paragraph{Entropy-Regularized Form}
First, we provide the upper bound for the ELBO in entropy-regularized form.
This result does \textit{not} require any modifications to vanilla SGD.

\vspace{.5ex}

\begin{theoremEnd}[\theoremproofoption,category=upperboundtheorem]{theorem}\label{thm:gradient_upper_bound}
  Let \(\rvvg_{M}\) be an \(M\)-sample estimate of the gradient of the ELBO in entropy regularized form (\cref{def:entropy_form}).
  Also, assume that \cref{assumption:q,assumption:phi_lipschitz} hold,
%
  \begin{itemize}[leftmargin=3em]
    \vspace{-1.5ex}
    \setlength\itemsep{0ex}
    \item \(f_{\mathrm{H}}\) is \(L_{\mathrm{H}}\)-smooth, and
    \item \(f_{\mathrm{KL}}\) is \(\mu_{\mathrm{KL}}\)-quadratically growing.
    \vspace{-1.5ex}
  \end{itemize}
  %
  Then, 
  {\small%
  \setlength{\belowdisplayskip}{1ex} \setlength{\belowdisplayshortskip}{1ex}%
  \setlength{\abovedisplayskip}{1ex} \setlength{\abovedisplayshortskip}{1ex}%
  \begin{align*}
    \hspace{-1.5em}
    \mathbb{E}\norm{\rvvg_{M}}_2^2
    &\leq
    \frac{4 L^2_{\mathrm{H}}}{\mu_{\mathrm{KL}} M} C\left(d, \kappa\right) \left( F\left(\vlambda\right) - F^* \right)
    + \norm{ \nabla F\left(\vlambda\right) }_2^2
    \\
    &\quad+ \frac{2 L^2_{\mathrm{H}}}{M} C\left(d, \kappa\right) {\lVert \bar{\vzeta}_{\mathrm{KL}} - \bar{\vzeta}_{\mathrm{H}} \rVert}_2^2
    \\
    &\quad+ \frac{4 L^2_{\mathrm{H}}}{\mu_{\mathrm{KL}} M} C\left(d, \kappa\right) \left( F^* - f_{\mathrm{KL}}^* \right),
  \end{align*}
  }%
  where
  {\small%
  \setlength{\belowdisplayskip}{1ex} \setlength{\belowdisplayshortskip}{1ex}%
  \setlength{\abovedisplayskip}{1ex} \setlength{\abovedisplayshortskip}{1ex}%
  \begin{alignat*}{2}
    C\left(d, \kappa\right) &= 2 \kappa \sqrt{d} + 1 &&\;\text{for mean-field,} \\
    C\left(d, \kappa\right) &= d + \kappa          &&\;\text{for the Cholesky and matrix square root,}
  \end{alignat*}
  }
  \(\bar{\zeta}_{\mathrm{KL}}\), \(\bar{\zeta}_{\mathrm{H}}\) are the stationary points of \(f_{\mathrm{KL}}\), \(f_{\mathrm{H}}\), respectively,
  \(F^* = \inf_{\vlambda \in \mathbb{R}^p} F\left(\vlambda\right)\), and \(f_{\mathrm{KL}}^* = \inf_{\vzeta \in \mathbb{R}^d} f\left(\zeta\right)\).
\end{theoremEnd}
\vspace{-1ex}
\begin{proofsketch}
  From \cref{thm:gradient_variance_general_upper_bound}, we can see that the key quantity of upper bounding the gradient variance is to analyze \(\mathbb{E} \norm{ \nabla_{\vlambda} f_{\mathrm{H}} \left( \vt_{\vlambda}\left(\rvvu\right) \right) } \).
  The bird's eye view of the proof is as follows:
  \begin{enumerate}
  \vspace{-1ex}
    \setlength\itemsep{-.5ex}
    \item[\ding{182}] The relationship between \( \norm{ \nabla_{\vlambda} f_{\mathrm{H}} \left( \vt_{\vlambda}\left(\rvvu\right) \right) }_2^2 \) and \( \norm{ \nabla f_{\mathrm{H}} \left( \vt_{\vlambda}\left(\rvvu\right) \right) }_2^2 \) is established through \cref{thm:general_variational_gradient_norm_bound}.
    \item[\ding{183}] Then, the \(L_{\mathrm{H}}\)-smoothness of \(f_{\mathrm{H}}\) relates \( \norm{ \nabla f_{\mathrm{H}} \left( \vt_{\vlambda}\left(\rvvu\right) \right) }_2^2 \) with \( {\lVert \vt_{\vlambda}\left(\rvvu\right) - \bar{\vzeta}_{\mathrm{H}} \rVert}_2^2\), the average squared distance from \(f_{\mathrm{H}}\)'s stationary point.
    \item[\ding{184}] The average squared distance enables the simplification of stochastic terms through \cref{thm:meanfield_u_identity,thm:reparam_u_identity}. This step also introduces dimension dependence.
  \vspace{-1ex}
  \end{enumerate}
  From here, we are now left with the \(\mathbb{E} {\lVert \vt_{\vlambda}\left(\rvvu\right) - \bar{\vzeta}_{\mathrm{H}} \rVert}_2^2\) term.
  One might be tempted to assume the quadratic growth assumption on \(f_{\mathrm{H}}\) and proceed as
  {%
  \setlength{\belowdisplayskip}{1.ex} \setlength{\belowdisplayshortskip}{1.ex}%
  \setlength{\abovedisplayskip}{1.ex} \setlength{\abovedisplayshortskip}{1.ex}%
  \begin{align*}
    \mathbb{E} {\Vert \vt_{\vlambda}\left(\rvvu\right) - \bar{\vzeta}_{\mathrm{H}} \rVert}_2^2
    \leq \frac{2}{\mu} \left( f_{\mathrm{H}}\left(\vt_{\vlambda}\left(\rvvu\right)\right) - f^*_{\mathrm{H}}\right).
  \end{align*}
  }%
  However, for the entropy-regularized form, this soon runs into a dead end since in
  {%
  \setlength{\belowdisplayskip}{1ex} \setlength{\belowdisplayshortskip}{1ex}%
  \setlength{\abovedisplayskip}{1ex} \setlength{\abovedisplayshortskip}{1ex}%
  \begin{align*}
    \mathbb{E} f_{\mathrm{H}}\left(\vt_{\vlambda}\left(\rvvu\right)\right) - f^*_{\mathrm{H}}
    &= F\left(\vlambda\right) - h\left(\vlambda\right) - f^* \\
    &= \left( F\left(\vlambda\right) - F^* \right) + \left(F^* - f^*\right) - h_{\mathrm{H}}\left(\vlambda\right),
  \end{align*}
  }%
  the negative entropy term \(h_{\mathrm{H}}\) is not bounded unless we rely on assumptions that need modifications to the BBVI algorithms. (\textit{e.g.}, bounded support, bounded domain).
  Fortunately, the following inequality cleverly side-steps this problem:
  {%
  \setlength{\belowdisplayskip}{1.5ex} \setlength{\belowdisplayshortskip}{1.5ex}%
  \setlength{\abovedisplayskip}{1.5ex} \setlength{\abovedisplayshortskip}{1.5ex}%
  \begin{align}
    \hspace{-1.0em}
    \mathbb{E} {\Vert \vt_{\vlambda}\left(\rvvu\right) - \bar{\vzeta}_{\mathrm{H}} \rVert}_2^2
    &\leq
    2\,\mathbb{E} {\lVert \vt_{\vlambda}\left(\rvvu\right) - \bar{\vzeta}_{\text{KL}} \rVert}_2^2
    +
    2 \, {\lVert \bar{\vzeta}_{\text{KL}} - \bar{\vzeta}_{\mathrm{H}} \rVert}_2^2,
    \label{eq:thm_upper_bound_parallel}
  \end{align}
  }%
  albeit at the cost of some looseness.
  By converting the entropy-regularized form into the KL-regularized form, the regularizer term becomes \(h_{\mathrm{KL}} = \DKL{q_{\vlambda}}{p} \geq 0\), which is bounded below by definition, unlike the entropic-regularizer \(h_{\mathrm{H}}\). 
  The proof completes by 
  \begin{enumerate}
  \vspace{-1ex}
    \setlength\itemsep{-.5ex}
    \item[\ding{185}] applying the quadratic growth assumption to relate the parameter distance with the function suboptimality gap, and 
    \item[\ding{186}] upper bounding the KL regularizer term.
  \end{enumerate}
  \vspace{-4ex}
\end{proofsketch}
\vspace{-2ex}
\begin{proofEnd}
  The proof uses the \(L_{\mathrm{H}}\)-smoothness of \(f_{\mathrm{H}}\) such that 
  \begin{align}
    \mathbb{E} \norm{
    \nabla f_{\mathrm{H}}\left(\vt_{\vlambda}\left(\rvvu\right)\right)
    }_2^2
    &=
    \mathbb{E} {\lVert
    \nabla f_{\mathrm{H}}\left(\vt_{\vlambda}\left(\rvvu\right)\right)
    -
    \nabla f_{\mathrm{H}}\left(\bar{\vzeta}_{\mathrm{H}}\right)
    \rVert}_2^2
    \nonumber
    \\
    &\leq
    L^2_{\mathrm{H}}
    \mathbb{E} {\lVert
    \vt_{\vlambda}\left(\rvvu\right)
    -
    \bar{\vzeta}_{\mathrm{H}}
    \rVert}_2^2,
    \label{eq:thm_upper_bound_smoothness}
  \end{align}
  where \(\bar{\vzeta}_{\mathrm{H}}\) is a stationary point of \(f_{\mathrm{H}}\) such that \(\nabla f_{\mathrm{H}}\left(\bar{\vzeta}_{\mathrm{H}}\right) = \mathbf{0}\).
  These steps have been previously used by \citet[Theorem 3]{domke_provable_2019} to prove the special case for the matrix square root parameterization.

  For the mean-field parameterization, we start from \cref{thm:general_variational_gradient_norm_bound} and apply~\cref{eq:thm_upper_bound_smoothness} as
  \begin{align}
    &\mathbb{E} \norm{
      \nabla_{\vlambda} f_{\mathrm{H}}\left(\vt_{\vlambda}\left(\rvvu\right)\right)
    }_2^2 
    \nonumber
    \\
    &\;\leq
    \mathbb{E} \norm{
      \nabla f_{\mathrm{H}}\left(\vt_{\vlambda}\left(\rvvu\right)\right)
    }_2^2 \left(1 + \norm{\mathbfsfit{U}}_{\mathrm{F}}\right)
    \nonumber
    \\
    &\;\leq
    L^2_{\mathrm{H}} \,
    \mathbb{E} {\lVert
    \vt_{\vlambda}\left(\rvvu\right)
    +
    \bar{\vzeta}_{\mathrm{H}}
    \rVert}_2^2 \left(1 + \norm{\mathbfsfit{U}}_{\mathrm{F}}\right),
    \nonumber
\shortintertext{applying \cref{thm:meanfield_u_identity},}
    &\;\leq
    L^2_{\mathrm{H}}
    \left( \kappa \sqrt{d} + \sqrt{\kappa d} + 1 \right) {\lVert \vm - \bar{\vzeta}_{\mathrm{H}} \rVert}_2^2
    \nonumber
    \\
    &\;\;\quad+
    L^2_{\mathrm{H}} \left(2 \kappa \sqrt{d} + 1\right) \norm{\mC}_{\mathrm{F}}^2,
    \nonumber
\shortintertext{and since the kurtosis satisfies \(\kappa \geq 1\) and thus \(\kappa \geq \sqrt{\kappa}\),}
    &\;\leq
    L^2_{\mathrm{H}}
    \left( 2 \kappa \sqrt{d} + 1 \right) \left( {\lVert \vm - \bar{\vzeta}_{\mathrm{H}} \rVert}_2^2
    + \norm{\mC}_{\mathrm{F}}^2 \right).
    \label{eq:thm1_meanfield}
  \end{align}

  Similarly, for the full-rank parameterizations, we start from \cref{thm:general_variational_gradient_norm_bound} and apply~\cref{eq:thm_upper_bound_smoothness} as
  \begin{alignat}{2}
    &\mathbb{E} \norm{
      \nabla_{\vlambda} f_{\mathrm{H}}\left(\vt_{\vlambda}\left(\rvvu\right)\right)
    }_2^2 
    \\
    &\;\leq\mathbb{E}\norm{\nabla f_{\mathrm{H}}\left(\vt_{\vlambda}\left(\rvvu\right)\right)}_2^2 \left(1 + \norm{\rvvu}_2^2 \right),
    \nonumber
    \\
    &\;\leq
    L^2_{\mathrm{H}}\,\mathbb{E}{\lVert \vt_{\vlambda}\left(\rvvu\right) - \bar{\vzeta}_{\mathrm{H}} \rVert}_2^2 \left(1 + \norm{\rvvu}_2^2 \right),
    \nonumber
\shortintertext{applying \cref{thm:reparam_u_identity},}
    &\;=
    L^2_{\mathrm{H}}\left(\left(d + 1\right) {\lVert \vm - \bar{\vzeta}_{\mathrm{H}} \rVert}_2^2 + \left(d + \kappa\right) \norm{\mC}_{\mathrm{F}}^2 \right),
    \nonumber
\shortintertext{and since the kurtosis satisfies \(\kappa \geq 1\),}
    &\;\leq
    L^2_{\mathrm{H}}\left(d + \kappa\right) \left( {\lVert \vm - \bar{\vzeta}_{\mathrm{H}} \rVert}_2^2 + \norm{\mC}_{\mathrm{F}}^2 \right).
    \label{eq:thm1_fullrank}
  \end{alignat}
  
  Both \cref{eq:thm1_meanfield,eq:thm1_fullrank} can now be denoted as
  \begin{alignat}{2}
    \mathbb{E}\norm{\nabla_{\vlambda} f_{\mathrm{H}}\left(\vt_{\vlambda}\left(\rvvu\right)\right)}_2^2 
    &\leq
    L^2_{\mathrm{H}}\, C\left(d, \kappa\right)  \left( {\lVert \vm - \bar{\vzeta}_{\mathrm{H}} \rVert}_2^2 + \norm{\mC}_{\mathrm{F}}^2 \right),
    \nonumber
\shortintertext{where by \cref{thm:reparam_quadratic},}
    &=
    L^2_{\mathrm{H}}\, C\left(d, \kappa\right) \mathbb{E} {\Vert \vt_{\vlambda}\left(\rvvu\right) - \bar{\vzeta}_{\mathrm{H}} \rVert}_2^2,
    \label{eq:thm1_parameter_suboptimality_unified}
  \end{alignat}
  and the constants are \(C\left(d, \kappa\right) = \kappa \sqrt{d} + 1\) for mean-field and \(C\left(d, \kappa\right) = d + \kappa\) for the full-rank parameterizations.

  As mentioned in the sketch, it is necessary to convert the entropy-regularized form into the KL-regularized form through the following inequality:
  {%
  \setlength{\belowdisplayskip}{1ex} \setlength{\belowdisplayshortskip}{1ex}%
  \setlength{\abovedisplayskip}{1ex} \setlength{\abovedisplayshortskip}{1ex}%
  \begin{alignat*}{2}
    \hspace{-0.5em}
    \mathbb{E} {\Vert \vt_{\vlambda}\left(\rvvu\right) - \bar{\vzeta}_{\mathrm{H}} \rVert}_2^2
    &\leq
    2\,\mathbb{E} {\lVert \vt_{\vlambda}\left(\rvvu\right) - \bar{\vzeta}_{\text{KL}} \rVert}_2^2
    +
    2 \, {\lVert \bar{\vzeta}_{\text{KL}} - \bar{\vzeta}_{\mathrm{H}} \rVert}_2^2.
  \end{alignat*}
  }%
  where \(\bar{\vzeta}_{\mathrm{KL}} = \Pi_{f_{\mathrm{KL}}}\left(\bar{\vzeta}_{\mathrm{H}}\right)\) is a projection of \(\bar{\vzeta}_{\mathrm{H}}\) to the set of minimizers of \(f_{\mathrm{KL}}\).
  Note that the KL-regularized form does not need to be tractable; only its existence suffices.
  We can now apply the quadratic growth assumption as
  {%
  \setlength{\belowdisplayskip}{1ex} \setlength{\belowdisplayshortskip}{1ex}%
  \setlength{\abovedisplayskip}{1ex} \setlength{\abovedisplayshortskip}{1ex}%
  \begin{alignat}{2}
    \hspace{-1em}
    \mathbb{E} {\lVert \vt_{\vlambda}\left(\rvvu\right) - \bar{\vzeta}_{\text{KL}} \rVert}_2^2
    \nonumber
    &\leq
    \frac{2}{\mu_{\mathrm{KL}}} \left( \mathbb{E} f_{\text{KL}}\left(\vt_{\vlambda}\left(\rvvu\right)\right) - f^*_{\text{KL}} \right)
    \nonumber
    \\
    &=
    \frac{2}{\mu_{\mathrm{KL}}} 
    \left( F\left(\vlambda\right) - h_{\mathrm{KL}}\left(\vlambda\right) - f^*_{\text{KL}} \right),
    \nonumber
\shortintertext{and since \(-h_{\mathrm{KL}}\left(\vlambda\right) = -\DKL{q_{\vlambda}}{p} \leq 0\) by definition,}
    &\leq
    \frac{2}{\mu_{\mathrm{KL}}} 
    \left( F\left(\vlambda\right) - f^*_{\text{KL}} \right)
    \label{eq:thm_upper_bound_kl_upper_bound}
    \\
    &=
    \frac{2}{\mu_{\mathrm{KL}}} \left(
    \left( F\left(\vlambda\right) - F^* \right) 
    + \left( F^* - f^*_{\text{KL}} \right)
    \right).
    \label{eq:thm_upper_bound_f_quadratic}
  \end{alignat}
  }%
%
  Combining \cref{eq:thm1_parameter_suboptimality_unified} with \cref{eq:thm_upper_bound_parallel},
  {%\small
  \begin{align*}
    &\mathbb{E}\norm{\nabla_{\vlambda} f_{\mathrm{H}}\left(\vt_{\vlambda}\left(\rvvu\right)\right)}_2^2 
    \\
    &\;\leq
    2 \, L^2_{\mathrm{H}}\, C\left(d, \kappa\right) \mathbb{E} {\Vert \vt_{\vlambda}\left(\rvvu\right) - \bar{\vzeta}_{\mathrm{H}} \rVert}_2^2
    + 2 \, L^2_{\mathrm{H}}\, C\left(d, \kappa\right) {\lVert \bar{\vzeta}_{\text{KL}} - \bar{\vzeta}_{\mathrm{H}} \rVert}_2^2,
\shortintertext{and applying \cref{eq:thm_upper_bound_f_quadratic},}
    &\;\leq
    \frac{4 \, L^2_{\mathrm{H}}}{\mu_{\mathrm{KL}}} C\left(d, \kappa\right) \left(
    \left( F\left(\vlambda\right) - F^* \right) 
    +
    \left( F^* - f^*_{\text{KL}} \right)
    \right)
    \\
    &\;\quad+
    2 \, L^2_{\mathrm{H}}\, C\left(d, \kappa\right) {\lVert \bar{\vzeta}_{\text{KL}} - \bar{\vzeta}_{\mathrm{H}} \rVert}_2^2
  \end{align*}
  }%
  Plugging this into \cref{thm:gradient_variance_general_upper_bound} yields the result.
\end{proofEnd}

%%% Local Variables:
%%% TeX-master: "main"
%%% End:


\vspace{1ex}
\begin{remark}
  If the bijector \(\psi\) is an identity function, \(\vzeta_{\mathrm{KL}}\) and \(\vzeta_{\mathrm{H}}\) are the maximum likelihood (ML) and maximum a-posteriori (MAP) estimates, respectively.
  Thus, with enough datapoints, the term \( {\lVert \bar{\vzeta}_{\mathrm{KL}} - \bar{\vzeta}_{\mathrm{H}} \rVert}_2^2 \) will be negligible since the ML and MAP estimates will be close.
\end{remark}

\vspace{1ex}
\begin{remark}
  Let \(\kappa_{\mathrm{cond.}} = \nicefrac{L_{\mathrm{H}}}{\mu_{\mathrm{KL}}}\) be the \textit{condition number} of the problem.
  For the full-rank parameterizations and smooth quadratic functions, the variance is bounded as \(\mathcal{O}\left( L_{\mathrm{H}} \kappa_{\mathrm{cond.}} \left(d + \kappa\right) / M \right)\).
  The variance depends linearly on 
  \begin{enumerate}
    \vspace{-1ex}
    \setlength\itemsep{-1ex}
    \item[\ding{182}] the scaling of the problem \(L_{\mathrm{H}}\), 
    \item[\ding{183}] the conditioning of the problem \(\kappa_{\mathrm{cond.}}\),
    \item[\ding{184}] the dimensionality of the problem \(d\), and
    \item[\ding{185}] the tail properties of the variational family \(\kappa\),
    \vspace{-1ex}
  \end{enumerate}
  where the number of Monte Carlo samples \(M\) linearly reduces the variance.
\end{remark}

\vspace{-1.ex}
\paragraph{KL-Regularized Form}
We now prove an equivalent result for the KL-regularized form.
Here, we do not have to rely on~\cref{eq:thm_upper_bound_parallel} since we already start from \(f_{\mathrm{KL}}\), which results in better constants.


\begin{theoremEnd}[\theoremproofoption,category=upperboundtheoremklform]{theorem}\label{thm:gradient_upper_bound_kl}
  Let \(\rvvg_{M}\) be an \(M\)-sample estimator of the gradient of the ELBO in KL-regularized form (\cref{def:kl_form}). 
  Also, assume that
%
  \begin{itemize}[leftmargin=3em]
    \vspace{-1.5ex}
    \setlength\itemsep{0ex}
    \item \(f_{\mathrm{KL}}\) is \(L_{\mathrm{KL}}\)-smooth,
    \item \(f_{\mathrm{KL}}\) is \(\mu_{\mathrm{KL}}\)-quadratically growing,
    \vspace{-1.5ex}
  \end{itemize}
  %
  and~\cref{assumption:q,assumption:phi_lipschitz} hold.
  Then, the gradient variance is bounded above as
  {%
  \setlength{\belowdisplayskip}{1ex} \setlength{\belowdisplayshortskip}{1ex}%
  \setlength{\abovedisplayskip}{1ex} \setlength{\abovedisplayshortskip}{1ex}%
  \begin{align*}
    \mathbb{E}\norm{\rvvg_{M}}_2^2
    &\leq
    \frac{2 L^2_{\mathrm{KL}}}{\mu_{\mathrm{KL}} M} C\left(d, \kappa\right) \left( F\left(\vlambda\right) - F^* \right)
    + \norm{ \nabla F\left(\vlambda\right) }_2^2
    \qquad\qquad
    \\
    &\quad+ \frac{2 L^2_{\mathrm{KL}}}{\mu_{\mathrm{KL}} M} C\left(d, \kappa\right) \left( F^* - f_{\mathrm{KL}}^*\right),
  \end{align*}
  }%
  where
  {%
  \setlength{\belowdisplayskip}{1ex} \setlength{\belowdisplayshortskip}{1ex}%
  \setlength{\abovedisplayskip}{1ex} \setlength{\abovedisplayshortskip}{1ex}%
  \begin{alignat*}{2}
    C\left(d, \kappa\right) &= 2 \kappa \sqrt{d} + 1 &&\;\text{for mean-field,} \\
    C\left(d, \kappa\right) &= d + \kappa          &&\;\text{for the Cholesky and matrix square root,}
  \end{alignat*}
  }%
  \(F^* = \inf_{\vlambda \in \mathbb{R}^p} F\left(\vlambda\right)\), and \(f_{\mathrm{KL}}^* = \inf_{\vzeta \in \mathbb{R}^d} f\left(\zeta\right)\).
\end{theoremEnd}
\vspace{-1ex}
\begin{proofEnd}
  This proof uses the smoothness of \(f_{\mathrm{KL}}\) instead of \(f_{\mathrm{H}}\).
  That is,
  \begin{align}
    \mathbb{E} \norm{
      \nabla f_{\mathrm{KL}}\left(\vt_{\vlambda}\left(\rvvu\right)\right)
    }_2^2 
    &=
    \mathbb{E} {\lVert
      \nabla f_{\mathrm{KL}}\left(\vt_{\vlambda}\left(\rvvu\right)\right)
      -
      \nabla f_{\mathrm{KL}}\left( \bar{\vzeta}_{\mathrm{KL}} \right)
    \rVert}_2^2 
    \nonumber
\shortintertext{applying \cref{eq:thm_upper_bound_smoothness},}
    &\leq
    L^2_{\mathrm{KL}} \,
    \mathbb{E} {\lVert
    \vt_{\vlambda}\left(\rvvu\right)
    -
    \bar{\vzeta}_{\mathrm{KL}}
    \rVert}_2^2, \label{eq:thm3_kl_smoothness}
  \end{align}
  where \(\bar{\vzeta}_{\mathrm{KL}}\) is a stationary point of \(f_{\mathrm{KL}}\).

  Substituting \cref{eq:thm3_kl_smoothness} in \cref{eq:thm1_parameter_suboptimality_unified},
  \begin{alignat}{2}
    &\mathbb{E}\norm{\nabla_{\vlambda} f_{\mathrm{KL}}\left(\vt_{\vlambda}\left(\rvvu\right)\right)}_2^2 
    \nonumber
    \\
    &\;\leq
    L^2_{\mathrm{KL}} C\left(d, \kappa\right) \mathbb{E} {\Vert \vt_{\vlambda}\left(\rvvu\right) - \bar{\vzeta}_{\mathrm{KL}} \rVert}_2^2,
    \nonumber
\shortintertext{and by applying \cref{eq:thm_upper_bound_f_quadratic} for \(f_{\mathrm{KL}}\),}
    &\;=
    \frac{2 L^2_{\mathrm{KL}} }{\mu_{\mathrm{KL}}} C\left(d, \kappa\right) \left( F\left(\vlambda\right) - F^*  \right) 
    + \frac{2 L^2_{\mathrm{KL}} }{\mu_{\mathrm{KL}}} C\left(d, \kappa\right)
    \left(
      F^* - f^*_{\text{KL}}
    \right).
    \nonumber
  \end{alignat}
  Plugging this to \cref{thm:gradient_variance_general_upper_bound} proves the result.
\end{proofEnd}

%%% Local Variables:
%%% TeX-master: "main"
%%% End:


\vspace{-.5ex}
\subsection{Upper Bound Under Bounded Entropy}
\vspace{-.5ex}
The bound in \cref{thm:gradient_upper_bound} is loose due to the use of~\cref{eq:thm_upper_bound_parallel} (\(\times 2\) loose) and \cref{eq:thm_upper_bound_kl_upper_bound}.
An alternative bound can be obtained by assuming the following:
%
\begin{assumption}[\textbf{Bounded Entropy}]\label{assumption:bounded_entropy}
  The regularization term is bounded below as
  \( h_{\mathrm{H}}\left(\vlambda\right) \geq h_{\mathrm{H}}^* \).
\end{assumption}
%.
For the entropy-regularized form, this corresponds to the entropy being bounded above by some constant since \(h\left(\vlambda\right) = - \mathrm{H}\left(q_{\vlambda}\right)\).
When using the nonlinear parameterizations (\cref{def:meanfield,def:fullrank}), this assumption can be practically enforced by bounding the output of \(\phi\) by some large \(S\).
%
\vspace{.5ex}
\begin{proposition}
    Let the diagonal conditioner \(\phi\) be bounded as \(\phi\left(x\right) \leq S\).
    Then, for any \(d\)-dimensional distribution \(q_{\vlambda}\) in the location-scale family with the mean-field (\cref{def:meanfield}) or Cholesky (\cref{def:fullrank}) parameterizations,
  {%
  \setlength{\belowdisplayskip}{1.ex} \setlength{\belowdisplayshortskip}{1.ex}%
  \setlength{\abovedisplayskip}{1.ex} \setlength{\abovedisplayshortskip}{1.ex}%
    \[ h_{\mathrm{H}}\left(\vlambda\right) = -\mathrm{H}\left(q_{\vlambda}\right) \geq -\mathrm{H}\left(\varphi\right) - d \log S. \]
  }%
\end{proposition}
\vspace{-2ex}
\begin{proof}
    From \cref{thm:location_scale_entropy}, \(\mathrm{H}\left(q_{\vlambda}\right) = \mathrm{H}\left(\varphi\right) + \log \abs{\mC} \).
    Since \(\mC\) under \cref{def:meanfield,def:fullrank} is a diagonal or triangular matrix, the log absolute determinant is the log sum of the diagonals.
    The conclusion follows from the fact that the diagonals \(C_{ii} = \phi\left(s_i\right)\) are bounded by \(S\).
\end{proof}
%
This is essentially a weaker version of the bounded domain assumption, though only the diagonal elements of \(\mC\), \(s_1, \ldots, s_d\), are bounded.
While this assumption results in an admittedly less realistic algorithm, it enables a tighter bound for the entropy-regularized form ELBO.


\begin{theoremEnd}[\theoremproofoption,category=upperboundtheoremboundedentropy]{theorem}\label{thm:gradient_upper_bound_bounded_entropy}
  Let \(\rvvg_{M}\) be an \(M\)-sample estimator of the gradient of the ELBO in entropy-regularized form (\cref{def:entropy_form}). 
  Also, assume that
%
  \begin{itemize}[leftmargin=3em]
    \vspace{-1.5ex}
    \setlength\itemsep{0ex}
    \item \(f_{\mathrm{H}}\) is \(L_{\mathrm{H}}\)-smooth,
    \item \(f_{\mathrm{H}}\) is \(\mu_{\mathrm{H}}\)-quadratically growing,
    \item \(h_{\mathrm{H}}\) is bounded as \(h_{\mathrm{H}}\left(\vlambda\right) > h_{\mathrm{H}}^*\) (\cref{assumption:bounded_entropy}),
      \vspace{-1.5ex}
  \end{itemize}
  %
  and~\cref{assumption:q,assumption:phi_lipschitz} hold.
  Then, the gradient variance of \(\vg_{M}\) is bounded above as
  {%
  \setlength{\belowdisplayskip}{1.ex} \setlength{\belowdisplayshortskip}{1.ex}%
  \setlength{\abovedisplayskip}{1.ex} \setlength{\abovedisplayshortskip}{1.ex}%
  \begin{align*}
    \mathbb{E}\norm{\rvvg_{M}}_2^2
    &\leq
    \frac{2 L^2_{\mathrm{H}}}{\mu_{\mathrm{H}} M} C\left(d, \kappa\right) \left( F\left(\vlambda\right) - F^* \right)
    + \norm{ \nabla F\left(\vlambda\right) }_2^2
    \qquad\qquad
    \\
    &\quad+ \frac{2 L^2_{\mathrm{H}}}{\mu_{\mathrm{H}} M} C\left(d, \kappa\right) \left( F^* - f_{\mathrm{H}}^* - h^*_{\mathrm{H}} \right),
  \end{align*}
  }
  where
  {%
  \setlength{\belowdisplayskip}{1.ex} \setlength{\belowdisplayshortskip}{1.ex}%
  \setlength{\abovedisplayskip}{1.ex} \setlength{\abovedisplayshortskip}{1.ex}%
  \begin{alignat*}{2}
    C\left(d, \kappa\right) &= 2 \kappa \sqrt{d} + 1 &&\;\text{for mean-field,} \\
    C\left(d, \kappa\right) &= d + \kappa          &&\;\text{for the Cholesky parameterization,}
  \end{alignat*}
  }
  \(F^* = \inf_{\vlambda \in \mathbb{R}^p} F\left(\vlambda\right)\), and \(f_{\mathrm{H}}^* = \inf_{\vzeta \in \mathbb{R}^d} f\left(\zeta\right)\).
\end{theoremEnd}
\vspace{-2ex}
\begin{proofsketch}
   Instead of using \cref{eq:thm_upper_bound_parallel}, we apply the quadratic assumption directly to \(f_{\text{H}}\).
   The remaining entropic-regularizer term can now be bounded through the bounded entropy assumption.
\end{proofsketch}
\vspace{-2ex}
\begin{proofEnd}
  The proof is similar to that of \cref{thm:gradient_upper_bound}.
  As mentioned in the \textit{proof sketch}, we use the fact that the entropic regularizer is bounded such that
  \[ -h_{\mathrm{H}}\left(\vlambda\right) < -h_{\mathrm{H}}^*. \]
  By applying the quadratic growth assumption directly to \(f_{\mathrm{H}}\), 
  \begin{alignat}{2}
    \mathbb{E} {\lVert \vt_{\vlambda}\left(\rvvu\right) - \bar{\vzeta}_{\text{H}} \rVert}_2^2
    &\leq
    \frac{2}{\mu_{\mathrm{H}}} \left( \mathbb{E} f_{\text{H}}\left(\vt_{\vlambda}\left(\rvvu\right)\right) - f^*_{\text{H}} \right)
    \nonumber
    \\
    &=
    \frac{2}{\mu_{\mathrm{H}}} 
    \left( F\left(\vlambda\right) - h_{\mathrm{H}}\left(\vlambda\right) - f^*_{\text{H}} \right),
    \nonumber
\shortintertext{and by \cref{assumption:bounded_entropy},}
    &\leq
    \frac{2}{\mu_{\mathrm{H}}} 
    \left( F\left(\vlambda\right) - F^* \right) 
    + \frac{2}{\mu_{\mathrm{H}}} \left( F^* - f^*_{\text{H}} - h_{\mathrm{H}}^*\right).
    \label{eq:thm_upper_bound_f_H_quadratic}
  \end{alignat}
  
  The proof resumes from \cref{eq:thm1_parameter_suboptimality_unified} as
  {%
  \setlength{\belowdisplayskip}{1.ex} \setlength{\belowdisplayshortskip}{1.ex}%
  \setlength{\abovedisplayskip}{1.ex} \setlength{\abovedisplayshortskip}{1.ex}%
  \begin{alignat}{2}
    &\mathbb{E}\norm{\nabla_{\vlambda} f_{\mathrm{H}}\left(\vt_{\vlambda}\left(\rvvu\right)\right)}_2^2
    \nonumber
    \\
    &\;\leq
    L^2_{\mathrm{H}} C\left(d, \kappa\right) \mathbb{E} {\Vert \vt_{\vlambda}\left(\rvvu\right) - \bar{\vzeta}_{\mathrm{H}} \rVert}_2^2,
    \nonumber
\shortintertext{and by applying \cref{eq:thm_upper_bound_f_H_quadratic},}
    &\;=
    \frac{2 L^2_{\mathrm{H}} }{\mu_{\mathrm{H}}} C\left(d, \kappa\right) \left( F\left(\vlambda\right) - F^*  \right) 
    + \frac{2 L^2_{\mathrm{H}} }{\mu_{\mathrm{H}}} C\left(d, \kappa\right)
    \left(
      F^* - f^*_{\mathrm{H}} - h_{\mathrm{H}}^*
    \right).
    \nonumber
  \end{alignat}
  }%
  Plugging this to \cref{thm:gradient_variance_general_upper_bound} proves the result.
\end{proofEnd}

%%% Local Variables:
%%% TeX-master: "main"
%%% End:


\begin{figure*}[t]
  \vspace{-2ex}
  \centering
  
%\tikzexternalenable
{\hypersetup{linkbordercolor=black,linkcolor=black}
\begin{tikzpicture}
  \begin{groupplot}
    [
      group style={
        group size=4 by 1,
        % columns=2,
        % rows=2,
        %xlabels at=edge bottom,
        %ylabels at=edge left,
        horizontal sep=0.07\textwidth
      },
    ]
    
% Tikz examples & demos
% https://holatex.app/examples.html?package=tikz

  \nextgroupplot[
      legend style={
        legend image post style = {scale=0.5},
        legend columns          = -1,
        column sep              = 1em,
        %at                     ={(0.5,1.5)},
        anchor                  = north,
        legend cell align       = left,
        line width              = 0.8pt,
        draw                    = none,
        legend to name          = grouplegend
      },
      %
      tuftelike, 
      %
      xlabel style={yshift=-0.8ex},
      %
      axis line style = thick,
      every tick/.style={black,thick},
      %axis lines = left,
      %grids=both,
      ymode  = log,
      xmin   = 1,
      xmax   = 1000,
      xtick  = {1, 200, 400, 600, 800, 1000},
      ytick  = {1e+4, 1e+5, 1e+6, 1e+7, 1e+8},
      %
      ymin   = 1e+4,
      ymax   = 1e+8,
      %
      xlabel = {Iteration},
      %ylabel = {\(\mathbb{E}\norm{\rvvg}^2_2\)},
      tick label style={font=\small},  
      height = 4cm,
      width  = 4.7cm,
      axis on top,
    ]
    \addplot[name path=gvar, color1, mark=none, thick]
      table [x=t, y=gvar] {data/simulation/quadratic_softpluschol_generalbound.csv}
      %node[above=3pt,pos=0.8] {\scriptsize\(\mu_{\mathrm{KL}}, L_{\mathrm{H}}\) bound}
      coordinate [pos=0.95] (gvarcoord);
    \addlegendentry{\scriptsize Gradient Variance\;\; \( \mathbb{E}\norm{ \rvvg }_2^2 \)}

    \addplot[name path=ABC, color2, mark=none, thick]
      table [x=t, y=ABC] {data/simulation/quadratic_softpluschol_generalbound.csv}
      coordinate [pos=0.05] (ABCcoord1) coordinate [pos=0.95] (ABCcoord2);
    \addlegendentry{\scriptsize Upper Bound\;\; \(2 A \left(F\left(\vlambda\right) - F^*\right) + B \norm{\nabla F}_2^2 + C\)}

    %\addplot[color4, draw=none, mark=none, thick] {x};
    %\addlegendentry{\scriptsize\(2 A \left(\mathbb{E}_{q_{\vlambda}} f_{\mathrm{KL}} - f^*_{\mathrm{KL}} \right) + B \norm{\nabla F}_2^2 + C_{\Delta \zeta} {\lVert \bar{\zeta}_{\mathrm{KL}} - \bar{\zeta}_{\mathrm{H}} \rVert}^2_2 \)}

    \addplot[name path=opt, color3, mark=none, thick] %
      table [x=t, y=opt] {data/simulation/quadratic_softpluschol_generalbound.csv} %
      %node[above=3pt,pos=0.8] {\scriptsize\(\DKL{q_{\vlambda}}{p}\)}
      coordinate [pos=0.95] (optcoord);
    %\addlegendentry{\scriptsize\(2 A \left(\mathbb{E}_{q_{\vlambda}} f_{\mathrm{H}} - f^*_{\mathrm{H}} \right) + B \norm{\nabla F}_2^2\)}

    \addplot[name path=axis, domain=0:990, fill=none, no markers, draw=none] {1e+4}
      coordinate [pos=0.05] (axiscoord);

    %\addplot[name path=ABC, color2, mark=none]
    %  table [x=t, y=ABC] {data/simulation/quadratic_softpluschol_generalbound.csv}
    %  ;

    \addplot[name path=C, fill=none, gray, mark=none]
      table [x=t, y=C] {data/simulation/quadratic_softpluschol_generalbound.csv}
      coordinate [pos=0.05] (Ccoord);

    \addplot[thick, color=color3, fill=color3, fill opacity=0.2] fill between[of=ABC and opt];
    \addplot[thick, color=color2, fill=color2, fill opacity=0.2] fill between[of=ABC and   C];

    \addplot[thick, color=color1, fill=color1, fill opacity=0.2] fill between[of=opt and gvar];
    \addplot[thick, color=gray,   fill=gray,   fill opacity=0.2] fill between[of=C   and axis];

    \draw[thick,color=color3,{Latex[scale=0.5]}-{Latex[scale=0.5]}] (ABCcoord2) -- (optcoord)
      node[midway,xshift=-18pt] {\scriptsize\(\DKL{q_{\vlambda}}{p}\)};

    %\draw[thick,color=color1,latex-latex] (gvarcoord) -- (optcoord)
    %  %node[midway,xshift=-32pt] {\scriptsize\(\mu_{\mathrm{KL}}, L_{\mathrm{H}}, \Delta \vz^*\) bound}
    %  coordinate [pos=0.5] (muL_coord);

    \draw[thick,color=gray,latex-] (Ccoord) -- (axiscoord)
      node[midway,xshift=5pt] {\scriptsize\(C\) };

%%     \node[%
%%       pin={%
%%         [%
%%           pin distance=0.3cm,%
%%           pin edge={color1},%
%%           text=color1%
%%         ]265:{\scriptsize\(\mu_{\mathrm{KL}}, L_{\mathrm{H}}, \Delta \vz^*, \phi \) bound}},%
%%       inner sep=0pt,%
%%     ] at (muL_coord) {};

    %\draw[thick,color=color2,latex-latex] (ABCcoord1) -- (Ccoord)
    %  node[midway,xshift=25pt] {\scriptsize\(2 A \left( F\left(\lambda\right) - F^* \right)\) };

    \input{figures/group_quadratic_softpluschol_boundedentropy}
    \input{figures/group_quadratic_softplusmf_generalbound}
    
% Tikz examples & demos
% https://holatex.app/examples.html?package=tikz

  \nextgroupplot[
      %% legend style={
      %%   legend image post style={scale=0.5},
      %%   at={(0.5,1.5)},
      %%   anchor=north,
      %%   legend cell align=left,
      %%   line width=0.8pt,
      %%   %draw=none % Unterdrücke Box
      %% },
      %
      tuftelike, 
      axis line style = thick,
      every tick/.style={black,thick},
      %axis lines = left,
      %grids=both,
      ymode  = log,
      xmin   = 1,
      xmax   = 1000,
      xtick  = {1, 200, 400, 600, 800, 1000},
      ytick  = {1e+3, 1e+4, 1e+5, 1e+6, 1e+7, 1e+8},
      %
      xlabel style={yshift=-0.8ex},
      %
      ymin   = 1e+3,
      ymax   = 1e+8,
      %
      xlabel = {Iteration},
      %ylabel = {\(\mathbb{E}\norm{\rvvg}^2_2\)},
      tick label style={font=\small},  
      height = 4cm,
      width  = 4.7cm,
      axis on top,
    ]
    \addplot[name path=gvar, color1, mark=none, thick]
      table [x=t, y=gvar] {data/simulation/quadratic_softplusmf_boundedentropybound.csv}
      %node[above=3pt,pos=0.8] {\scriptsize\(\mu_{\mathrm{KL}}, L_{\mathrm{H}}\) bound}
      coordinate [pos=0.95] (gvarcoord);
    %\addlegendentry{\scriptsize\( \mathbb{E}\norm{ \rvvg }_2^2 \)}

    \addplot[name path=opt, color3, mark=none, thick] %
      table [x=t, y=opt] {data/simulation/quadratic_softplusmf_boundedentropybound.csv} %
      %node[above=3pt,pos=0.8] {\scriptsize\(\DKL{q_{\vlambda}}{p}\)}
      coordinate [pos=0.95] (optcoord);
    %\addlegendentry{\scriptsize\(2 A \left(\mathbb{E}_{q_{\vlambda}} f_{\mathrm{H}} - f^*_{\mathrm{H}}\right)\)}

    \addplot[name path=ABC, color2, mark=none, thick]
      table [x=t, y=ABC] {data/simulation/quadratic_softplusmf_boundedentropybound.csv}
      coordinate [pos=0.05] (ABCcoord1) coordinate [pos=0.95] (ABCcoord2);
    %\addlegendentry{\scriptsize\(2 A \left(F\left(\vlambda\right) - F^*\right) + B \norm{\nabla F}_2^2 + C\)}

    \addplot[name path=axis, domain=0:990, fill=none, no markers, draw=none] {1e+3}
      coordinate [pos=0.05] (axiscoord);

    %\addplot[name path=ABC, color2, mark=none]
    %  table [x=t, y=ABC] {data/simulation/quadratic_softpluschol_boundedentropybound.csv}
    %  ;

    \addplot[name path=C, fill=none, gray, mark=none]
      table [x=t, y=C] {data/simulation/quadratic_softplusmf_boundedentropybound.csv}
      coordinate [pos=0.05] (Ccoord);

    \addplot[thick, color=color3, fill=color3, fill opacity=0.2] fill between[of=ABC and opt];
    \addplot[thick, color=color2, fill=color2, fill opacity=0.2] fill between[of=ABC and   C];

    \addplot[thick, color=color1, fill=color1, fill opacity=0.2] fill between[of=opt and gvar];
    \addplot[thick, color=gray,   fill=gray,   fill opacity=0.2] fill between[of=C   and axis];

    \draw[thick,color=color3,{Latex[scale=0.5]}-{Latex[scale=0.5]}] (ABCcoord2) -- (optcoord)
      node[midway,xshift=-18pt] {\scriptsize\(h\left(\vlambda\right) - h^*\)}
      coordinate [pos=0.5] (dklarrow_coord);

    %\draw[thick,color=color1,{Latex[scale=0.5]}-{Latex[scale=0.5]}] (gvarcoord) -- (optcoord)
      %node[midway,xshift=-25pt] {\scriptsize\(\mu_{\mathrm{KL}}, L_{\mathrm{H}}\) bound}
    %  coordinate [pos=0.5] (muL_coord);

    \draw[thick,color=gray,latex-] (Ccoord) -- (axiscoord)
      node[midway,xshift=5pt] {\scriptsize\(C\) };

    %% \node[%
    %%   pin={%
    %%     [%
    %%       pin distance=1cm,%
    %%       pin edge={color3},%
    %%       text=color3%
    %%     ]100:{\scriptsize\( h\left(\vlambda\right) - h^* \)}%
    %%   },%
    %%   inner sep=0pt,
    %% ] at (dklarrow_coord) {};

    %% \node[%
    %%   pin={%
    %%     [%
    %%       pin distance=1cm,%
    %%       pin edge={color1},%
    %%       text=color1%
    %%     ]130:{\scriptsize\(\mu_{\mathrm{H}}, L_{\mathrm{H}}\) bound}},%
    %%   inner sep=0pt,%
    %% ] at (muL_coord) {};


    %\draw[thick,color=color2,latex-latex] (ABCcoord1) -- (Ccoord)
    %  node[midway,xshift=25pt] {\scriptsize\(2 A \left( F\left(\lambda\right) - F^* \right)\) };

    %\input{figures/group_quadratic_softplusmf_generalbound}
  \end{groupplot}
  
  \coordinate (c1r1southwest) at ($(group c1r1) + (-1.7cm, -2.5cm)$); 
  \coordinate (c1r1southeast) at ($(group c1r1) + (1.7cm,  -2.5cm)$); 

  \coordinate (c2r1southwest) at ($(group c2r1) + (-1.7cm, -2.5cm)$);
  \coordinate (c2r1southeast) at ($(group c2r1) + (1.7cm,  -2.5cm)$);

  \coordinate (c3r1southwest) at ($(group c3r1) + (-1.7cm, -2.5cm)$); 
  \coordinate (c3r1southeast) at ($(group c3r1) + ( 1.7cm, -2.5cm)$); 

  \coordinate (c4r1southwest) at ($(group c4r1) + (-1.7cm, -2.5cm)$);
  \coordinate (c4r1southeast) at ($(group c4r1) + ( 1.7cm, -2.5cm)$);
  
  \draw[thick,color=black] (c1r1southwest) -- (c2r1southeast) node[midway,below] {\small Cholesky\;\; \(\phi\left(x\right) = \mathrm{softplus}\left(x\right)\)};

  \draw[thick,color=black] (c3r1southwest) -- (c4r1southeast) node[midway,below] {\small Mean-Field\;\; \(\phi\left(x\right) = \mathrm{softplus}\left(x\right)\)};

  \draw[thick,color=black] ($(c1r1southwest) + (0,-0.7cm)$) -- ($(c1r1southeast) + (0,-0.7cm)$) node[midway,below] {\small\cref{thm:gradient_upper_bound}};
  \draw[thick,color=black] ($(c2r1southwest) + (0,-0.7cm)$) -- ($(c2r1southeast) + (0,-0.7cm)$) node[midway,below] {\small\cref{thm:gradient_upper_bound_bounded_entropy}};
  \draw[thick,color=black] ($(c3r1southwest) + (0,-0.7cm)$) -- ($(c3r1southeast) + (0,-0.7cm)$) node[midway,below] {\small\cref{thm:gradient_upper_bound}};
  \draw[thick,color=black] ($(c4r1southwest) + (0,-0.7cm)$) -- ($(c4r1southeast) + (0,-0.7cm)$) node[midway,below] {\small\cref{thm:gradient_upper_bound_bounded_entropy}};


  \node at ($(group c2r1) + (2.25cm,1.75cm)$) {\ref*{grouplegend}}; 
\end{tikzpicture}
}
%\tikzexternaldisable

  \vspace{-4ex}
  \caption{
    \textbf{Evaluation of the bounds for a perfectly conditioned quadratic target function.}
    The \textcolor{color3}{blue regions} are the loosenesses resulting from either using (\cref{thm:gradient_upper_bound}) or not using (\cref{thm:gradient_upper_bound_bounded_entropy}) the bounded entropy assumption (\cref{assumption:bounded_entropy}), while the \textcolor{color1}{red regions} are the remaining ``technical loosesnesses.''
    The gradient variance was estimated from \(10^3\) samples.
  }\label{fig:quadratic}
  \vspace{-2ex}
\end{figure*}


%% \subsection{Upper Bound of the STL Estimator}

%% Unlike the previously considered estimators, the \textit{sticking the landing} (STL; \citealt{roeder_sticking_2017}) estimator uses Monte Carlo estimates of the entropy term.
%% %
%% \begin{definition}[\textbf{STL Estimator}]
%%   \begin{align*}
%%     f\left(\vzeta\right)
%%     &= 
%%     - \nabla_{\vlambda} \log\ell\left(\rvvx, \rvvz = \psi^{-1}\left( \vzeta \right) \right)
%%     - \log \abs{ \mJ_{\phi^{-1}}\left(\vzeta\right) } \\ &\qquad+ 
%%     \underbrace{\nabla_{\vlambda} \log q_{\vgamma}\left(\vzeta\right),}_{\text{Monte Carlo entropy estimate}}
%%   \end{align*}
%%   where \(\vgamma = \vlambda\) and \(h\left(\vlambda\right) = 0\).
%% \end{definition}
%% %
%% This estimator contains more stochastic elements than the entropy-form and KL-form estimators.
%% Even though the entropy term now \textit{adds} noise, it has shown in practice to result in lower variance and, therefore, faster, stable convergence.
%% This is because the entropy term acts as a control variate \citep{geffner_using_2018}, reducing variance as \(q_{\vlambda}\) becomes closer to \(\pi\).
%% We also provide an upper bound for the STL estimator.

%% 
  
\begin{theoremEnd}[\keylemmaproofoption,category=upperboundlemma]{lemma}\label{eq:lemma_stl_entropy_term_variance}
  Let \(V\) denote the expected squared norm of the derivative of \(\log \varphi\) such that \(
  V =
  \norm{
    \nabla \log \varphi\left( \vu \right)
  }_2^2.
  \)
  Then,
  \begin{align*}
    \mathbb{E} {\lVert
      \nabla \log q_{\gamma}\left(\vt_{\vlambda}\left(\vu\right)\right)
    \rVert}_2^2
    \leq
    2 \, V \left( \mathrm{H}\left(\varphi\right) - \mathrm{H}\left(q_{\vlambda}\right) \right).
  \end{align*}
\end{theoremEnd}
\begin{proofEnd}
  \begin{alignat}{2}
    &\mathbb{E} {\lVert
      \nabla \log q_{\vgamma}\left(\vt_{\vlambda}\left(\vu\right)\right)
    \rVert}_2^2 \, \Big\lvert_{\vgamma = \vlambda}
    \nonumber
    \\
    &\;=
    \mathbb{E} \norm{
      \frac{\partial}{\partial \vz}\left( \log \varphi\left( \mC^{-1}\left( \vz - \vm \right) \right) - \log \abs{\mC} \right)
    }_2^2 \, \Big\lvert_{\vz = \vt_{\vlambda}\left(\vu\right)}
    \nonumber
    \\
    &\;=
    \mathbb{E} {\lVert
      \mC^{-\top} \nabla \log \varphi\left( \mC^{-1}\left( \vt_{\vlambda}\left(\vu\right) - \vm \right) \right)
    \rVert}_2^2
    \nonumber
    \\
    &\;=
    \mathbb{E} {\lVert
      \mC^{-\top} \nabla \log \varphi\left( \vu \right)
    \rVert}_2^2
    \nonumber
    \\
    &\;\leq
    {\lVert
      \mC^{-\top}
    \rVert}^2_{2}
    \,
    \mathbb{E}
    {\lVert
       \nabla \log \varphi\left( \vu \right)
    \rVert}_2^2
    \nonumber
    \\
    &\;\leq
    {\lVert
      \mC^{-\top}
    \rVert}^2_{\mathrm{F}}
    \,
    \mathbb{E}
    {\lVert
       \nabla \log \varphi\left( \vu \right)
    \rVert}_2^2
    \nonumber
    \\
    &\;=
    V\,
    {\lVert
      \mC^{-1}
    \rVert}_{\mathrm{F}}^2
    \label{eq:lemma_stl_entropy_term_variance_eq1}
  \end{alignat}

  Now, for location-scale families, the entropy of \(q_{\vlambda}\) follows as
  \begin{alignat*}{2}
    \mathrm{H}\left(q_{\vlambda}\right) 
    &=
    \mathrm{H}\left(\varphi\right) 
    +
    \log \abs{\mC}
    \\
    &=
    \mathrm{H}\left(\varphi\right) 
    -
    \frac{1}{2}
    \log \abs{\mC^{-\top} \mC^{-1}}
    \\
    &=
    \mathrm{H}\left(\varphi\right) 
    -
    \frac{1}{2}
    \sum_{i=1}^d \log \sigma_i\left(\mC^{-\top} \mC^{-1}\right),
\shortintertext{and applying the inequality \(-x \leq -\log x\),}
    &\geq
    \mathrm{H}\left(\varphi\right) 
    -
    \frac{1}{2}
    \sum_{i=1}^d \sigma_i\left(\mC^{-\top} \mC^{-1}\right)
    \\
    &=
    \mathrm{H}\left(\varphi\right) 
    -
    \frac{1}{2}
    \mathrm{tr}\left(\mC^{-\top} \mC^{-1}\right)
    \\
    &=
    \mathrm{H}\left(\varphi\right) 
    -
    \frac{1}{2}
    {\lVert \mC^{-1} \rVert}_{\mathrm{F}}^2.
  \end{alignat*}

  Utilizing this in \cref{eq:lemma_stl_entropy_term_variance_eq1},
  \begin{alignat*}{2}
    \mathbb{E} {\lVert
      \nabla \log q_{\vgamma}\left(\vt_{\vlambda}\left(\vu\right)\right)
    \rVert}_2^2 \, \Big\lvert_{\vgamma = \vlambda}
    &\leq
    V\,
    {\lVert
      \mC^{-1}
    \rVert}_{\mathrm{F}}^2
    \\
    &\leq
    2 \, V \left( \mathrm{H}\left(\varphi\right) - \mathrm{H}\left(q_{\vlambda}\right) \right).
  \end{alignat*}

  \todo[inline]{
    The following bound based on the trace might be tighter.
  \begin{alignat*}{2}
    &\mathbb{E} \norm{
      \mC^{-\top} \nabla \log \varphi\left( \vu \right)
    }_2^2
    \\
    &\;=
    \mathbb{E} \,
    \nabla \log \varphi^{\top}\left( \vu \right)
    \mC^{-1} \mC^{-\top}
    \nabla \log \varphi\left( \vu \right)
    \\
    &\;=
    \mathbb{E} \mathrm{tr}\left(
      \nabla \log \varphi^{\top}\left( \vu \right)
      \mC^{-1} \mC^{-\top}
      \nabla \log \varphi\left( \vu \right)
    \right)
    \\
    &\;=
    \mathrm{tr}\left(
    \mathbb{E} 
      \nabla \log \varphi\left( \vu \right)
      \nabla \log \varphi^{\top}\left( \vu \right)
      \mC^{-1} \mC^{-\top}
    \right).
  \end{alignat*}
  
  Notice that 
  \begin{align*}
    \mathbb{E} 
      \nabla \log \varphi\left( \vu \right)
      \nabla \log \varphi^{\top}\left( \vu \right)
  \end{align*}
  is in fact similar to a Fisher information matrix.
  (Here, the derivative is with respect to the input instead of the parameters.
  So this is technically not a Fisher information matrix.)
  We can use the information identity to see that
  \begin{align*}
    \mathbb{E} 
    \nabla \log \varphi\left( \vu \right)
    \nabla \log \varphi^{\top}\left( \vu \right)
    =
    \mH_{\log \varphi}.
  \end{align*}
  But since each component of \(\varphi\) is \(i.i.d.\), the Hessian matrix is a diagonal matrix.

  \begin{alignat*}{2}
    &\mathbb{E} {\lVert
      \nabla \log q_{\vgamma}\left(\vt_{\vlambda}\left(\vu\right)\right)
    \rVert}_2^2 \, \Big\lvert_{\vgamma = \vlambda}
    \\
    &\;=
    \mathrm{tr}\left(
    \mathbb{E} 
      \nabla \log \varphi\left( \vu \right)
      \nabla \log \varphi^{\top}\left( \vu \right)
      \mC^{-1} \mC^{-\top}
    \right)
    \\
    &\;=
    \mathrm{tr}\left(
      \mH_{\log \varphi}
      \mC^{-1} \mC^{-\top}
    \right)
  \end{alignat*}
  If we assume some properties about \(\mH_{\log \varphi}\) then maybe?
  }
\end{proofEnd}

\begin{theoremEnd}[\theoremproofoption,category=upperboundtheoremstl]{theorem}
  Let \(f\) be a function satisfying \cref{assumption:L_smoothness,assumption:quadratic_growth}.
  Then, the variance of the \(M\)-sample gradient estimator is bounded below as
  \begin{align*}
    \mathbb{E}\norm{\rvvg\left(\vlambda\right)}_2^2
    \leq
    2 A \left( F\left(\vlambda\right) - F^* \right)
    + B \norm{ \nabla F\left(\vlambda\right) }_2^2
    + C,
  \end{align*}
  where \(F^* = \inf_{\vlambda \in \Lambda} F\left(\vlambda\right)\), for the finite constants
  \[
  A = \frac{L^2 \, \left(d + \kappa\right)}{M},\;
  B = \frac{M-1}{M},\;
  \;\text{and}\;
  C = \frac{L^2 \, \left(d + \kappa\right)}{K} F^*.
  \]
\end{theoremEnd}
\begin{proofsketch}
  We apply the Parallelogram Law 
  \[
  \norm{\va - \vb}_2^2 = 2 \, \norm{\va}^2_2 + 2\,\norm{\vb}^2_2 - \norm{\va + \vb}_2^2 \leq 2\,\norm{\va}^2_2 + 2\,\norm{\vb}^2_2
  \] as
  \begin{alignat*}{2}
    &\mathbb{E}{\lVert
      \nabla f\left(\vt_{\vlambda}\left(\vu\right)\right)
      - \nabla \log q_{\vgamma}\left(\vt_{\vlambda}\left(\vu\right)\right)
    \rVert}_2^2
    \\
    &\,=
    2 \, \mathbb{E} \norm{
      \nabla f\left(\vt_{\vlambda}\left(\vu\right)\right)
    }
    +
    2 \, \mathbb{E} {\lVert
      \nabla \log q_{\vlambda}\left(\vt_{\vlambda}\left(\vu\right)\right)
    \rVert}_2^2 
    \\
    &\quad-
    \underbrace{
    \mathbb{E}{\lVert
      \nabla f\left(\vt_{\vlambda}\left(\vu\right)\right) 
      +
      \nabla \log q_{\vgamma}\left(\vt_{\vlambda}\left(\vu\right)\right)
    \rVert}^2_2
    }_{\text{control variate effect}}
    \\
    &\,\leq
    2 \, \mathbb{E} \norm{
      \nabla f\left(\vt_{\vlambda}\left(\vu\right)\right)
    }
    +
    2 \, \mathbb{E} {\lVert
      \nabla \log q_{\vlambda}\left(\vt_{\vlambda}\left(\vu\right)\right)
    \rVert}_2^2.
  \end{alignat*}
  Note, that this step makes the bound looser than the ``regularized'' form estimator, which opposes the previous observation that the STL estimator achieves superior variance reduction in practice~\citep{roeder_sticking_2017, geffner_using_2018, agrawal_advances_2020}.

  This mismatch, however, is to be expected.
  The bounds we are establishing are ``worst case'' bounds, and the worst case performance of the STL estimator is known to be worse than the regularized form estimator.
  The advantage of the STL estimator is only realized when \(q_{\vlambda}\) and \(f\) becomes strongly correlated, acting as a control variate.
  In this case, the control variate effect appears as \(\norm{ \nabla_{\vlambda} f + \nabla_{\vlambda} \log q_{\vgamma} }\) being large, tightening the bound.
\end{proofsketch}
\begin{proofEnd}
  We first start from the definition of variance,
  \begin{alignat*}{2}
    &\mathbb{E} \norm{\vg_M }^2_2
    \\
    &\;=
    \mathrm{tr}\,\V{ \vg_M } + \norm{\mathbb{E} \vg}^2_2,
\shortintertext{following the definition in \cref{eq:def_gradient_M_est},}
    &\;=
    \mathrm{tr}\,\V{ \frac{1}{M} \sum_{m=1}^M \vg_m } + \norm{ \nabla F\left(\vlambda\right) }^2_2
\shortintertext{and then the definition in \cref{eq:def_gradient_m_est},}
    &\;=
    \mathrm{tr}\,\V{
      \frac{1}{M} \sum_{m=1}^M \frac{\partial }{\partial \vlambda} \Big( f\left(\vt_{\vlambda}\left(\rvvu_m\right)\right) - \log q_{\vgamma}\left(\vt_{\vlambda}\left(\rvvu_m\right) \right) \Big)
    } \Bigg\lvert_{\vgamma = \vlambda}
    \\
    &\;\quad+ \norm{ \nabla F\left(\vlambda\right) }^2_2,
\shortintertext{by the linearity of variance,}
    &\;=
    \frac{1}{M} \mathrm{tr}\,\V{
      \frac{\partial }{\partial \vlambda} \Big( f\left(\vt_{\vlambda}\left(\rvvu_m\right)\right) - \log q_{\vgamma}\left(\vt_{\vlambda}\left(\rvvu_m\right) \right) \Big)
    } \, \Big\lvert_{\vgamma = \vlambda}
    \\
    &\;\quad+ \norm{ \nabla F\left(\vlambda\right) }^2_2
    \\
    &\;=
    \frac{1}{M} \mathbb{E}\norm{
      \frac{\partial }{\partial \vlambda} \Big( f\left(\vt_{\vlambda}\left(\rvvu_m\right)\right) - \log q_{\vgamma}\left(\vt_{\vlambda}\left(\rvvu_m\right) \right) \Big)
    }_2^2 \, \Big\lvert_{\vgamma = \vlambda}
    \\
    &\;\quad+ \norm{ \nabla F\left(\vlambda\right) }^2_2,
\shortintertext{applying \cref{thm:general_variational_gradient_norm_identity},}
    &\;\leq
    \frac{1}{M} \mathbb{E}{\lVert
      \nabla f\left(\vt_{\vlambda}\left(\rvvu\right)\right)
      - \nabla \log q_{\vlambda}\left(\vt_{\vlambda}\left(\rvvu_m\right)\right)
    \rVert}_2^2 \left(1 + \norm{\rvvu}_2^2\right)
    \\
    &\;\quad+ \norm{ \nabla F\left(\vlambda\right) }^2_2
  \end{alignat*}

  We know apply the inequality \(\norm{\va - \vb}_2^2 \leq 2\,\norm{\va}^2_2 + 2\,\norm{\vb}^2_2\) as
  \begin{alignat*}{2}
    &\mathbb{E}{\lVert
      \nabla f\left(\vt_{\vlambda}\left(\vu\right)\right)
      - \nabla \log q_{\vgamma}\left(\vt_{\vlambda}\left(\vu\right)\right)
    \rVert}_2^2
    \\
    &\,=
    2 \, \mathbb{E} \norm{
      \nabla f\left(\vt_{\vlambda}\left(\vu\right)\right)
    }
    +
    2 \, \mathbb{E} {\lVert
      \nabla \log q_{\vlambda}\left(\vt_{\vlambda}\left(\vu\right)\right)
    \rVert}_2^2 
    \\
    &\quad-
    \underbrace{
    \mathbb{E}{\lVert
      \nabla f\left(\vt_{\vlambda}\left(\vu\right)\right) 
      +
      \nabla \log q_{\vgamma}\left(\vt_{\vlambda}\left(\vu\right)\right)
    \rVert}^2_2
    }_{\text{control variate effect}}
    \\
    &\,\leq
    2 \, \mathbb{E} \norm{
      \nabla f\left(\vt_{\vlambda}\left(\vu\right)\right)
    }
    +
    2 \, \mathbb{E} {\lVert
      \nabla \log q_{\vlambda}\left(\vt_{\vlambda}\left(\vu\right)\right)
    \rVert}_2^2.
  \end{alignat*}

  Due to \cref{eq:lemma_stl_entropy_term_variance}, 
  \begin{alignat*}{2}
    &\mathbb{E} {\lVert
      \nabla \log q_{\vgamma}\left(\vt_{\vlambda}\left(\vu\right)\right)
    \rVert}_2^2 \, \Big\lvert_{\vgamma = \vlambda}
    \\
    &\;=
    2 \, V \left( \mathrm{H}\left(\varphi\right) - \mathrm{H}\left(q_{\vlambda}\right) \right)
    \\
    &\;=
    2 \, V \left( \mathbb{E}f\left(\vt_{\vlambda}\left(\rvvu\right)\right) - \mathrm{H}\left(q_{\vlambda}\right) - \mathbb{E}f\left(\vt_{\vlambda}\left(\rvvu\right)\right) + \mathrm{H}\left(\varphi\right) \right)
    \\
    &\;=
    2 \, V \left( F\left(\vlambda\right)  - \mathbb{E}f\left(\vt_{\vlambda}\left(\rvvu\right)\right) + \mathrm{H}\left(\varphi\right) \right)
    \\
    &\;\leq
    2 \, V \left( F\left(\vlambda\right) - f^* + \mathrm{H}\left(\varphi\right) \right).
  \end{alignat*}

  \begin{alignat*}{2}
    &\mathbb{E} \norm{\vg_M }^2_2
    \\
    &\;\leq
    \frac{1}{M} \mathbb{E}{\lVert
      \nabla f\left(\vt_{\vlambda}\left(\rvvu\right)\right)
      - \nabla \log q_{\vlambda}\left(\vt_{\vlambda}\left(\rvvu_m\right)\right)
    \rVert}_2^2 \left(1 + \norm{\rvvu}_2^2\right)
    \\
    &\;\quad+ \norm{ \nabla F\left(\vlambda\right) }^2_2
    \\
    &\;\leq
    \frac{1}{M} \left(
    2 \, \mathbb{E} \norm{
      \nabla f\left(\vt_{\vlambda}\left(\vu\right)\right)
    }_2^2
    +
    2 \, \mathbb{E} {\lVert
      \nabla \log q_{\vlambda}\left(\vt_{\vlambda}\left(\vu\right)\right)
    \rVert}_2^2
    \right)
    \\
    &\;\quad+ \norm{ \nabla F\left(\vlambda\right) }^2_2
    \\
    &\;\leq
    \frac{1}{M} \Bigg(
    \frac{2\,L^2 }{\mu} \left(d + \kappa\right) \left(  F\left(\vlambda\right) - h\left(\vlambda\right) - f^* \right)
    \\
    &\quad\quad+
    2 \, V \left( F\left(\vlambda\right) - f^* + \mathrm{H}\left(\varphi\right) \right)
    \Bigg)
    \\
    &\;\quad+ \norm{ \nabla F\left(\vlambda\right) }^2_2
  \end{alignat*}
\end{proofEnd}

%%% Local Variables:
%%% TeX-master: "main"
%%% End:


%% Naturally, since gradient variance bounds are worst-case bounds, this estimator has \textit{worse} variance guarantee.


\subsection{Matching Lower Bound}
Finally, we present a matching lower bound on the gradient variance of BBVI.
Our lower bound holds broadly for smooth and strongly convex problem instances that are well-conditioned and high-dimensional.

\vspace{.5ex}
%!TEX root=main.tex

% \begin{theoremEnd}[\lemmaproofoption,category=lowerboundlemma]{lemma}
%   Let \(\vt_{\vlambda}: \mathbb{R}^d \mapsto \mathbb{R}^d\) be defined as in \cref{def:reparam} with parameters \(\vlambda = \left(\vm, \mC\right)\) such that \(\vm \in \mathbb{R}^d\) and \(\mC \in \mathbb{R}^{d \times d}\).
%   Also, let \(\rvvu \sim \varphi\), where \(\varphi\) is defined as in \cref{assumption:symmetric_standard}.
%   Then,
%   \begin{alignat*}{2}
%     \mathbb{E}\vt_{\vlambda}\left(\rvvu\right) \left(1 + \norm{\rvvu}_2^2\right)
%     = \left(d + 1\right) \vm.
%   \end{alignat*}
% \end{theoremEnd}
% \begin{proofEnd}
%   Since \(\rvu_i\) follows a symmetric and standardized distribution, \(\mathbb{E}\rvu_i^3 = 0, \mathbb{E} \rvu_i^2 = 1\).
%   Then,
%   \begin{alignat*}{2}
%     \mathbb{E}\vt_{\vlambda}\left(\rvvu\right) \left(1 + \norm{\rvvu}_2^2\right)
%     &=
%     \mathbb{E} \left(\mC \rvvu + \vm \right) \left(1 + \norm{\rvvu}_2^2\right)
%     \\
%     &=
%     \mC \, \mathbb{E} \rvvu \left(1 + \norm{\rvvu}_2^2\right) + \vm \left(1 + \mathbb{E} \norm{\rvvu}_2^2\right),
% \shortintertext{applying \cref{thm:u_identities},}
%     &=
%     \left( \mathbb{E}\rvu_i^3 \right) \mC \mathbf{1} + \vm \left(1 + d \, \mathbb{E} \rvu_i^2\right),
% \shortintertext{and by~\cref{assumption:symmetric_standard},}
%     &=
%     \left(d + 1\right) \vm.
%   \end{alignat*}
% \end{proofEnd}

% \begin{theoremEnd}[\lemmaproofoption,category=lowerboundlemma]{lemma}\label{thm:general_quad_reparam}
%   Let \(\vt_{\vlambda}: \mathbb{R}^d \mapsto \mathbb{R}^d\) be defined as in \cref{def:reparam} with parameters \(\vlambda = \left(\vm, \mC\right)\) such that \(\vm \in \mathbb{R}^d\) and \(\mC \in \mathbb{R}^{d \times d}\).
%   Also, let \(\mSigma \in \mathbb{R}^{d \times d}\) be some matrix, \(\vmu \in \mathbb{R}^d\) be some vector, and \(\rvvu \sim \varphi\) be a vector-valued random variable, where \(\varphi\) is defined as in \cref{assumption:symmetric_standard}.
%   Then,
%   \begin{align*}
%     &\mathbb{E} \, {\left( \vt_{\vlambda}\left(\rvvu\right) - \vmu \right)}^{\top} \mSigma \left(\vt_{\vlambda}\left(\rvvu\right) - \vmu\right)
%     \\
%     &\;= {\left(\vm - \vmu\right)}^{\top} \mSigma \left(\vm - \vmu\right)
%     + \mathrm{tr}\left({\mSigma}^{-1} \mC \mC^{\top}\right).
%   \end{align*}
% \end{theoremEnd}
% \begin{proofEnd}
%   \begin{alignat}{2}
%     &\mathbb{E} {\left( \vt_{\vlambda}\left(\rvvu\right) - \vmu \right)}^{\top} \mSigma \, \left(\vt_{\vlambda}\left(\rvvu\right) - \vmu \right)
%     \nonumber
%     \\
%     &\;=
%     \mathbb{E}\, {\vt_{\vlambda}\left(\rvvu\right)}^{\top} \mSigma \, {\vt_{\vlambda}\left(\rvvu\right)}
%     -2\, \vmu^{\top} {\mSigma} \, \mathbb{E}\vt_{\vlambda}\left(\rvvu\right)
%     + \vmu^{\top} {\mSigma} \, \vmu,
%     \nonumber
% \shortintertext{where \(\mathbb{E}\vt_{\vlambda}\left(\rvvu\right) = \mC \mathbb{E}{\rvvu} + \vm\), and by \cref{assumption:symmetric_standard},}
%     &\;=
%     \mathbb{E}\, {\vt_{\vlambda}\left(\rvvu\right)}^{\top} \mSigma \, {\vt_{\vlambda}\left(\rvvu\right)}
%     -2\, \vmu^{\top} {\mSigma} \, \vm
%     + \vmu^{\top} {\mSigma} \, \vmu.\label{eq:thm:general_quad_reparam_eq1}
%   \end{alignat}
%   Furthermore,
%   \begin{alignat*}{2}
%     &\mathbb{E} {\vt_{\vlambda}\left(\rvvu\right)}^{\top} \mSigma  \, {\vt_{\vlambda}\left(\rvvu\right)}
%     \\
%     &\;=
%     \mathbb{E}{\left(\mC \rvvu + \vm\right)}^{\top}  \mSigma \left(\mC \rvvu + \vm\right)
%     \\
%     &\;=
%     \mathbb{E} \rvvu^{\top} \mC^{\top} \mSigma \, \mC \rvvu
%     + \vm^{\top} \mSigma \, \mC \, \mathbb{E}  \rvvu 
%     + \mathbb{E} \rvvu^{\top} \mC^{\top} \mSigma \vm^{\top} 
%     + \vm^{\top} \mSigma \, \vm,
% \shortintertext{by \cref{assumption:symmetric_standard},}
%     &\;=
%     \mathbb{E} \rvvu^{\top} \mC^{\top} {\mSigma} \, \mC \rvvu
%     + \vm^{\top} \mSigma \, \vm,
% \shortintertext{invoking the trace trick,}
%     &\;=
%     \mathbb{E} \mathrm{tr}\left( \rvvu^{\top} \mC^{\top} {\mSigma}\, \mC \rvvu \right)
%     + \vm^{\top} \mSigma \, \vm,
% \shortintertext{pusing the expectation into the trace,}
%     &\;=
%     \mathrm{tr}\left(\mathbb{E} \rvvu  \rvvu^{\top} \mC^{\top} {\mSigma}\, \mC \right)
%     + \vm^{\top} \mSigma \, \vm,
% \shortintertext{invoking \cref{thm:u_identities},}
%     &\;=
%     \mathrm{tr}\left(\mC^{\top} {\mSigma}\, \mC \right)
%     + \vm^{\top} \mSigma \, \vm
%     \\
%     &\;=
%     \mathrm{tr}\left({\mSigma} \, \mC \mC^{\top}\right)
%     + \vm^{\top} \mSigma \, \vm.
%   \end{alignat*}
%   Applying this to \cref{eq:thm:general_quad_reparam_eq1},
%   \begin{alignat*}{2}
%     &\mathbb{E} {\left( \vt_{\vlambda}\left(\rvvu\right) - \vmu \right)}^{\top} {\mSigma}^{-1} \left(\vt_{\vlambda}\left(\rvvu\right) - \vmu \right)
%     \\
%     &\;=
%     \mathrm{tr}\left({\mSigma}^{-1} \mC \mC^{\top}\right)
%     + \vm^{\top} \mSigma^{-1} \vm
%     -2\, \vmu^{\top} {\mSigma}^{-1} \vm
%     + \vmu^{\top} {\mSigma}^{-1} \vmu
%     \\
%     &\;=
%     {\left(\vm - \vmu\right)}^{\top} {\mSigma}^{-1} \left(\vm - \vmu\right)
%     + \mathrm{tr}\left({\mSigma}^{-1} \mC \mC^{\top}\right).
%   \end{alignat*}
% \end{proofEnd}

% \begin{theoremEnd}[\lemmaproofoption,category=lowerboundlemma]{corollary}\label{thm:q_quad_reparam}
%   Let \(\vt_{\vlambda}: \mathbb{R}^d \mapsto \mathbb{R}^d\) be defined as in \cref{def:reparam} with parameters \(\vlambda = \left(\vm, \mC\right)\) such that \(\vm \in \mathbb{R}^d\) and \(\mC \in \mathbb{R}^{d \times d}\).
%   Also, let \(\rvvu \sim \varphi\) be a vector-valued random variable, where \(\varphi\) is defined as in \cref{assumption:symmetric_standard}.
%   Then,
%   \begin{align*}
%     \mathbb{E} \, {\left( \vt_{\vlambda}\left(\rvvu\right) - \vm \right)}^{\top} {\left(\mC \mC^{\top}\right)}^{-1} \left(\vt_{\vlambda}\left(\rvvu\right) - \vm\right) = d.
%   \end{align*}
% \end{theoremEnd}
% \begin{proofEnd}
%   From, \cref{thm:general_quad_reparam} we have
%   \begin{alignat*}{2}
%     &\mathbb{E} \, {\left( \vt_{\vlambda}\left(\rvvu\right) - \vm \right)}^{\top} {\left(\mC \mC^{\top}\right)}^{-1} \left(\vt_{\vlambda}\left(\rvvu\right) - \vm\right)
%     \\
%     &\;= {\left(\vm - \vm\right)}^{\top} {\left(\mC \mC^{\top}\right)}^{-1} \left(\vm - \vm\right)
%     + \mathrm{tr}\left({\left( \mC \mC^{\top} \right)}^{-1} \mC \mC^{\top}\right)
%     \\
%     &\;= \mathrm{tr}\left(\mI\right)
%     \\
%     &\;= d.
%   \end{alignat*}
% \end{proofEnd}

% \begin{theoremEnd}[\lemmaproofoption,category=lowerboundlemma]{lemma}\label{thm:general_grad_norm_reparam}
%   Let \(\vt_{\vlambda}: \mathbb{R}^d \mapsto \mathbb{R}^d\) be defined as in \cref{def:reparam} with parameters \(\vlambda = \left(\vm, \mC\right)\) such that \(\vm \in \mathbb{R}^d\) and \(\mC \in \mathbb{R}^{d \times d}\).
%   Also, let \(\mSigma_1, \mSigma_2 \in \mathbb{R}^{d \times d}\) be some matrices, \(\vmu_1, \vmu_2 \in \mathbb{R}^{d}\) be some vectors, and \(\rvvu = \left(\rvu_1, \ldots, \rvu_d \right)\) be a vector-valued random variable sampled as \(\rvvu \sim \varphi\), where \(\varphi\) is defined as in \cref{assumption:symmetric_standard} with kurtosis \(\kappa = \mathbb{E}\rvu_i^4\).
%   Then,
%   \begin{alignat*}{2}
%     &\mathbb{E} {\left(\vt_{\vlambda}\left(\rvvu\right) - \vmu_1\right)}^{\top} \mSigma_1 \mSigma_2 \left(\vt_{\vlambda}\left(\rvvu\right) - \vmu_2\right) \left( 1 + \norm{\rvvu}_2^2 \right)
%     \\
%     &\;=
%     \left(d + 1\right) \vmu_1^{\top} \mSigma_1 \mSigma_2 \vmu_2
%     +
%     \left(d + \kappa\right) \mathrm{tr}\left(\mC \mC^{\top} \mSigma_1 \mSigma_2 \right).
%   \end{alignat*}
% \end{theoremEnd}
% \begin{proofEnd}
%   \begin{alignat}{2}
%     &\mathbb{E} {\left(\vt_{\vlambda}\left(\rvvu\right) - \vmu_1\right)}^{\top} \mSigma_1 \mSigma_2 \left(\vt_{\vlambda}\left(\rvvu\right) - \vmu_2\right) \left( 1 + \norm{\rvvu}_2^2 \right)
%     \nonumber
%     \\
%     &\;=
%     \mathbb{E} {\vt_{\vlambda}\left(\rvvu\right)}^{\top} \mSigma_1 \mSigma_2 \, \vt_{\vlambda}\left(\rvvu\right) \left( 1 + \norm{\rvvu}_2^2 \right)
%     \nonumber
%     \\
%     &\qquad- \vmu_1^{\top} \mSigma_1 \mSigma_2 \, \mathbb{E} \vt_{\vlambda}\left(\rvvu\right) \left( 1 + \norm{\rvvu}_2^2 \right)
%     \nonumber
%     \\
%     &\qquad- \mathbb{E} {\left(\vt_{\vlambda}\left(\rvvu\right) \left( 1 + \norm{\rvvu}_2^2 \right)\right)}^{\top} \mSigma_1 \mSigma_2 \vmu_1  
%     \nonumber
%     \\
%     &\qquad+ \vmu_1^{\top} \mSigma_1 \mSigma_2 \vmu_2 \, \mathbb{E}\left( 1 + \norm{\rvvu}_2^2 \right),
%     \nonumber
% \shortintertext{invoking \cref{thm:u_identities},}
%     &\;=
%     \mathbb{E} {\vt_{\vlambda}\left(\rvvu\right)}^{\top} \mSigma_1 \mSigma_2 \, \vt_{\vlambda}\left(\rvvu\right) \left( 1 + \norm{\rvvu}_2^2 \right)
%     \nonumber
%     \\
%     &\qquad- \mathbb{E}\rvu_i^3 \, \vmu_1^{\top} \mSigma_1 \mSigma_2 \mathbf{1}
%     \nonumber
%     \\
%     &\qquad- \mathbb{E}\rvu_i^3 \, \mathbf{1}^{\top} \mSigma_1 \mSigma_2 \vmu_1  
%     \nonumber
%     \\
%     &\qquad+ \left( 1 + d\,\mathbb{E}\rvu_i^2 \right) \vmu_1^{\top} \mSigma_1 \mSigma_2 \vmu_2,
%     \nonumber
% \shortintertext{and due to \cref{assumption:symmetric_standard},}
%     &\;=
%     \mathbb{E} {\vt_{\vlambda}\left(\rvvu\right)}^{\top} \mSigma_1 \mSigma_2 \, \vt_{\vlambda}\left(\rvvu\right) \left( 1 + \norm{\rvvu}_2^2 \right)
%     \label{eq:general_grad_norm reparam_eq1}
%     \\
%     &\qquad+ \left( 1 + d \right) \vmu_1^{\top} \mSigma_1 \mSigma_2 \vmu_2.\label{eq:general_grad_norm reparam_eq2}
%   \end{alignat}
%   Now for \cref{eq:general_grad_norm reparam_eq1},
%   \begin{alignat}{2}
%     &\mathbb{E} {\vt_{\vlambda}\left(\rvvu\right)}^{\top} \mSigma_1 \mSigma_2 \, \vt_{\vlambda}\left(\rvvu\right) \left( 1 + \norm{\rvvu}_2^2 \right)
%     \nonumber
%     \\
%     &\;=
%     \mathbb{E} {\left(\mC \rvvu + \vm \right)}^{\top} \mSigma_1 \mSigma_2 \left(\mC \rvvu + \vm \right) \left( 1 + \norm{\rvvu}_2^2 \right)
%     \nonumber
%     \\
%     &\;=
%     \mathbb{E} \mathrm{tr}\left( \rvvu^{\top} \mC^{\top} \mSigma_1 \mSigma_2 \mC \rvvu \right)
%     \nonumber
%     \\
%     &\qquad+
%     \mathbb{E} \mathrm{tr}\left( \rvvu^{\top} \mC^{\top} \mSigma_1 \mSigma_2 \mC \, \rvvu \rvvu^{\top} \rvvu  \right)
%     \label{eq:thm:general_grad_norm_reparam_eq1}
%     \\
%     &\qquad+
%     \vm^{\top} \mSigma_1 \mSigma_2 \mC \, \mathbb{E} \rvvu \left( 1 + \norm{\rvvu}_2^2 \right)
%     \nonumber
%     \\
%     &\qquad+
%     \mathbb{E} \rvvu^{\top} \left( 1 + \norm{\rvvu}_2^2 \right) \mC^{\top} \mSigma_1 \mSigma_2 \vm
%     \nonumber
%     \\
%     &\qquad+
%     \vm^{\top} \mSigma_1 \mSigma_2 \vm \, \mathbb{E} \rvvu \left( 1 + \norm{\rvvu}_2^2 \right),
%     \nonumber
% %
% \shortintertext{using the cyclic property of the trace on \cref{eq:thm:general_grad_norm_reparam_eq1},}
% %
%     &\;=
%     \mathrm{tr}\left( \mathbb{E}  \rvvu \rvvu^{\top} \mC^{\top} \mSigma_1 \mSigma_2 \mC  \right)
%     \nonumber
%     \\
%     &\qquad+
%     \mathrm{tr}\left( \mC^{\top} \mSigma_1 \mSigma_2 \mC \, \mathbb{E} \rvvu \rvvu^{\top} \rvvu \rvvu^{\top} \right)
%     \nonumber
%     \\
%     &\qquad+
%     \vm^{\top} \mSigma_1 \mSigma_2 \mC \, \mathbb{E} \rvvu \left( 1 + \norm{\rvvu}_2^2 \right)
%     \nonumber
%     \\
%     &\qquad+
%      \mathbb{E} \rvvu^{\top} \left( 1 + \norm{\rvvu}_2^2 \right) \mC^{\top} \mSigma_1 \mSigma_2  \, \vm
%     \nonumber
%     \\
%     &\qquad+
%     \vm^{\top} \mSigma_1 \mSigma_2 \vm \, \mathbb{E} \rvvu \left( 1 + \norm{\rvvu}_2^2 \right)
%     \nonumber
% %
% \shortintertext{invoking \cref{thm:u_identities},}
% %
%     &\;=
%     \mathrm{tr}\left( \left(\mathbb{E}\rvu_i^2\right) \mI \mC^{\top} \mSigma_1 \mSigma_2 \mC  \right)
%     \nonumber
%     \\
%     &\qquad+
%     \mathrm{tr}\left( \mC^{\top} \mSigma_1 \mSigma_2 \mC \left(\left(d-1\right) {\left(\mathbb{E}\rvu_i^2\right)}^2 + \mathbb{E}\rvu_i^4 \right) \mI \right)
%     \nonumber
%     \\
%     &\qquad+
%     \vm^{\top} \mSigma_1 \mSigma_2 \mC \, \left(\mathbb{E} \rvu_i^3\right) \mathbf{1} 
%     \nonumber
%     \\
%     &\qquad+
%     \left(\mathbb{E} \rvu_i^3\right) \mathbf{1}^{\top} \mC^{\top} \mSigma_1 \mSigma_2  \vm
%     \nonumber
%     \\
%     &\qquad+
%     \vm^{\top} \mSigma_1 \mSigma_2 \vm \, \left( \mathbb{E} \rvu_i^3 \right) \mathbf{1},
%     \nonumber
% %
% \shortintertext{and due to \cref{assumption:symmetric_standard},}
% %
%     &\;=
%     \mathrm{tr}\left( \mC^{\top} \mSigma_1 \mSigma_2 \mC  \right)
%     +
%     \left(d - 1 + \kappa\right) \mathrm{tr}\left( \mC^{\top} \mSigma_1 \mSigma_2 \mC\right)
%     \nonumber
%     \\
%     &\;=
%     \left(d + \kappa\right) \mathrm{tr}\left(\mC^{\top} \mSigma_1 \mSigma_2 \mC \right)
%     \nonumber
%     \\
%     &\;=
%     \left(d + \kappa\right) \mathrm{tr}\left(\mC  \mC^{\top} \mSigma_1 \mSigma_2 \right).
%     \nonumber
%   \end{alignat}
%   Finally, applying this to \cref{eq:general_grad_norm reparam_eq2},
%   \begin{alignat*}{2}
%     &\mathbb{E} \left(\vt_{\vlambda}\left(\rvvu\right) - \vmu_1\right) \mSigma_1 \mSigma_2 \left(\vt_{\vlambda}\left(\rvvu\right) - \vmu_2\right) \left( 1 + \norm{\rvvu}_2^2 \right)
%     \nonumber
%     \\
%     &\;=
%     \left(d + 1\right) \vmu_1^{\top} \mSigma_1 \mSigma_2 \vmu_2
%     +
%     \left(d + \kappa\right) \mathrm{tr}\left(\mC \mC^{\top} \mSigma_1 \mSigma_2 \right).
%   \end{alignat*}
% \end{proofEnd}

% \begin{theoremEnd}[\lemmaproofoption,category=lowerboundlemma]{corollary}\label{thm:q_grad_norm_reparam}
%   Let \(\vt_{\vlambda}: \mathbb{R}^d \mapsto \mathbb{R}^d\) be defined as in \cref{def:reparam} with parameters \(\vlambda = \left(\vm, \mC\right)\) such that \(\vm \in \mathbb{R}^d\) and \(\mC \in \mathbb{R}^{d \times d}\).
%   Also, let \(\rvvu = \left(\rvu_1, \ldots, \rvu_d \right)\) be a vector-valued random variable sampled as \(\rvvu \sim \varphi\), where \(\varphi\) is defined as in \cref{assumption:symmetric_standard} with kurtosis \(\kappa = \mathbb{E}\rvu_i^4\).
%   Then,
%   \begin{alignat*}{2}
%     &\mathbb{E} \left(\vt_{\vlambda}\left(\rvvu\right) - \vm\right) {\left( \mC \mC^{\top}\right)}^{-\top} {\left( \mC \mC^{\top}\right)}^{-1} \left(\vt_{\vlambda}\left(\rvvu\right) - \vm\right) \left( 1 + \norm{\rvvu}_2^2 \right)
%     \\
%     &\;=
%     \left(d + 1\right) \vm^{\top} {\left( \mC \mC^{\top}\right)}^{-\top} {\left( \mC \mC^{\top}\right)}^{-1} \vm
%     +
%     \left(d + \kappa\right) \mathrm{tr}\left( {\left( \mC \mC^{\top} \right)}^{-1} \right).
%   \end{alignat*}
% \end{theoremEnd}
% \begin{proofEnd}
%   From \cref{thm:general_grad_norm_reparam}, we have
%   \begin{alignat}{2}
%     &\mathbb{E} {\left(\vt_{\vlambda}\left(\rvvu\right) - \vm\right)}^{\top} {\left( \mC \mC^{\top}\right)}^{-\top} {\left( \mC \mC^{\top}\right)}^{-1} \left(\vt_{\vlambda}\left(\rvvu\right) - \vm\right) \left( 1 + \norm{\rvvu}_2^2 \right)
%     \nonumber
%     \\
%     &\;=
%     \left(d + 1\right) \vm^{\top} {\left( \mC \mC^{\top}\right)}^{-\top} {\left( \mC \mC^{\top}\right)}^{-1} \vm
%     \nonumber
%     \\
%     &\quad+
%     \left(d + \kappa\right) \mathrm{tr}\left(\mC \mC^{\top} {\left( \mC \mC^{\top}\right)}^{-\top} {\left( \mC \mC^{\top}\right)}^{-1}\right).\label{thm:q_grad_norm_reparam_eq1}
%   \end{alignat}
%   Since, by the basic properties of the trace,
%   \begin{alignat*}{2}
%     \mathrm{tr}\left(\mC \mC^{\top} {\left( \mC \mC^{\top}\right)}^{-\top} {\left( \mC \mC^{\top}\right)}^{-1}\right)
%     &=
%     \mathrm{tr}\left( {\left( \mC \mC^{\top}\right)}^{-1} \mC \mC^{\top} {\left( \mC \mC^{\top}\right)}^{-\top} \right)
%     \\
%     &=
%     \mathrm{tr}\left( \mI \, {\left( \mC \mC^{\top}\right)}^{-\top} \right)
%     \\
%     &=
%     \mathrm{tr}\left( {\left( \mC \mC^{\top}\right)}^{-\top} \right)
%     \\
%     &=
%     \mathrm{tr}\left( {\left( \mC \mC^{\top} \right)}^{-1} \right),
%   \end{alignat*}
%   \cref{thm:q_grad_norm_reparam_eq1} becomes
%   \begin{alignat*}{2}
%     &\mathbb{E} {\left(\vt_{\vlambda}\left(\rvvu\right) - \vm\right)}^{\top} {\left( \mC \mC^{\top}\right)}^{-\top} {\left( \mC \mC^{\top}\right)}^{-1} \left(\vt_{\vlambda}\left(\rvvu\right) - \vm\right) \left( 1 + \norm{\rvvu}_2^2 \right)
%     \\
%     &\;=
%     \left(d + 1\right) \vm^{\top} {\left( \mC \mC^{\top}\right)}^{-\top} {\left( \mC \mC^{\top}\right)}^{-1} \vm
%     +
%     \left(d + \kappa\right) \mathrm{tr}\left( {\left( \mC \mC^{\top} \right)}^{-1} \right).
%   \end{alignat*}
% \end{proofEnd}


% \begin{assumption}[\textbf{Polyak-Łojasiewicz Condition; PL}]\label{assumption:pl}
%   If the gradient of a function \(f : \mathbb{R}^d \rightarrow \mathbb{R}\) satisfies
%   \begin{align*}
%     \frac{1}{2\mu}\norm{ \nabla f\left(\vz\right) }_2^2 \geq  f\left(\vz\right) - f^*
%   \end{align*}
%   for some constant $\mu > 0$, where \(f^* = \inf_{\vz \in \mathbb{R}^d} f\left(\vz\right)\), then $f$ is said to satisfy the Polyak-Łojasiewicz condition.
% \end{assumption}


\begin{theoremEnd}[\theoremproofoption,category=lowerboundtheorem]{theorem}\label{thm:gradient_lower_bound}
Let \(\rvvg_{M}\) be an \(M\)-sample estimator of the gradient of the ELBO in either the entropy- or KL-regularized form.
Also, let ~\cref{assumption:q} hold where the matrix square root parameterization is used.
Then, for all \(L\)-smooth and \(\mu\)-strongly convex functions \(f\) such that $\nicefrac{L}{\mu} < \sqrt{d + 1}$, the variance of \(\rvvg_{M}\) is bounded below by some strictly positive constant as
\begin{align*}
  \mathbb{E}\norm{\rvvg_M}_2^2
  &\geq
  \frac{2\mu^2 \left(d + 1\right) - 2 L^2}{ML} \left( F\left(\vlambda\right) - F^* \right) 
  + \norm{ \nabla F\left(\vlambda\right) }_2^2 \\
  &\quad+ \frac{2 \mu^2 \left(d + 1\right) - 2 L^2}{ML} \left(\mathbb{E}{f\left(\vt_{\vlambda^*}\left(\vu\right)\right)} - f^*\right),  
\end{align*}
as long as $\vlambda$ is in a local neighborhood around the unique global optimum $\vlambda^* = \argmin_{\vlambda \in \mathbb{R}^p} F\left(\vlambda\right)$, where \(F^* = F\left(\vlambda^*\right)\) and \(f^* = \argmin_{\vzeta \in \mathbb{R}^d} f\left(\vzeta\right)\).
\end{theoremEnd}
\vspace{-3ex}
\begin{proofsketch}
  We use the fact that, with the matrix square root parameterization, if \(f\) is \(L\)-smooth, $\mathbb{E} f\left(\vt_{\vlambda}\left(\rvvu\right)\right)$ is also $L$-smooth~\citep{domke_provable_2020}.
  From this, the parameter suboptimality can be related to the function suboptimality as
  {%
  \setlength{\belowdisplayskip}{1ex} \setlength{\belowdisplayshortskip}{1ex}%
  \setlength{\abovedisplayskip}{1ex} \setlength{\abovedisplayshortskip}{1ex}%
  \begin{align*}
    {\lVert \vlambda - \bar{\vlambda} \rVert}^2_2
    \geq
    \left({2}/{L}\right)
    \left(
    \mathbb{E}{f\left(\vt_{\vlambda}\left(\rvvu\right)\right)} 
    - f^*
    \right),
  \end{align*}
  }%
  where \(\bar{\vlambda} = \left(\bar{\vzeta}, \boldupright{O}\right)\).
  For the entropy term, we circumvent the need to directly bound its value by restricting our interest to the neighborhood of the minimizer \(\vlambda^*\), where the contribution of \(h\left(\vlambda^*\right) - h\left(\vlambda\right)\) will be marginal enough for the lower bound to hold.
\end{proofsketch}

\vspace{-2ex}
\begin{proofEnd}
% Note that
% \begin{align*}
% \mathbb{E}\norm{ \vg_M }_2^2
%  = \frac{1}{M} \left(
% \mathbb{E}{ \norm{\nabla_{\vlambda} f\left(\vt_{\vlambda}\left(\rvvu\right)\right)}_2^2 }
% -
% \norm{\mathbb{E}{ \nabla_{\vlambda} f\left(\vt_{\vlambda}\left(\rvvu\right)\right)} }_2^2
% \right)
% + \norm{ \nabla F\left(\vlambda\right) }^2_2.
% \end{align*}
When using the matrix square root parameterization,~\citet{domke_provable_2020} have shown that if $f$ is $L$-smooth, $\mathbb{E} f\left(\vt_{\vlambda}\left(\rvvu\right)\right)$ is also $L$-smooth.
Therefore, we have
\begin{align} 
    \norm{\mathbb{E}{ \nabla_{\vlambda} f\left(\vt_{\vlambda}\left(\rvvu\right)\right)} }_2^2 \leq 2 L \left(\mathbb{E} f\left(\vt_{\vlambda}\left(\rvvu\right)\right) - f^*\right).
    \label{eq:thm_lower_bound_eq1}
\end{align}

Furthermore, let $\bar{\vzeta}$ be the minimizer of $f$, namely $f^* = f\left(\bar{\vzeta}\right)$.
From \cref{thm:variational_gradient_norm_identity}, we have
\begin{align*}
    \mathbb{E}{\norm{\nabla_{\vlambda} f\left(\vt_{\vlambda}\left(\rvvu\right)\right)}_2^2} 
    & = \mathbb{E}{\norm{\nabla f\left(\vt_{\vlambda}\left(\rvvu\right)\right)}_2^2 \left(1 + \norm{\rvvu}_2^2\right)},
\shortintertext{by the \(\mu\)-strong convexity of \(f\),}
    & \geq 2 \mu \, \mathbb{E}{\left(f\left(\vt_{\vlambda}\left(\rvvu\right)\right) - f^*\right) \left(1 + \norm{\rvvu}_2^2\right)} \\
    & \geq \mu^2 \, \mathbb{E}{{\lVert\vt_{\vlambda}\left(\rvvu\right) - \bar{\vzeta}\rVert}^2_2 \left(1 + \norm{\rvvu}_2^2\right)},
\shortintertext{applying \Cref{thm:reparam_u_identity},}
    & = \mu^2 \, \left(\left(d + 1\right) {\lVert \vm - \bar{\vzeta} \rVert}^2_2 + \left(d + \kappa\right) \norm{\mC}_{\mathrm{F}}^2\right),
\shortintertext{and by the property of the kurtosis that \(\kappa \geq 1\),}
    & \geq \mu^2 \, \left(d + 1\right) {\lVert \vlambda - \bar{\vlambda} \rVert}^2_2,
\end{align*}
where $\bar{\vlambda} = \left(\bar{\vzeta}, \boldupright{O}\right)$.

Observe that $\bar{\vlambda}$ is the minimizer of $\mathbb{E}{f\left(\vt_{\vlambda}\left(\rvvu\right)\right)}$ 
 such that 
\[
   \mathbb{E}{f\left(\vt_{\bar{\vlambda}}\left(\rvvu\right)\right)} = f\left(\bar{\vzeta}\right) = f^* \leq \mathbb{E}{f\left(\vt_{\vlambda}\left(\rvvu\right)\right)}
\]
for any $\vlambda$.
Furthermore, from the $L$-smoothness of $\mathbb{E}{f\left(\vt_{\vlambda}\left(\rvvu\right)\right)}$, we have
\begin{align*}
    &\mu^2 \left(d + 1\right) {\lVert \vlambda - \bar{\vlambda} \rVert}^2_2 \\
    &\quad\geq \frac{2 \mu^2 \left(d + 1\right)}{L} \left(\mathbb{E}{f\left(\vt_{\vlambda}\left(\rvvu\right)\right)} - \mathbb{E}{f\left(\vt_{\bar{\vlambda}}\left(\rvvu\right)\right)}\right).
\end{align*}
Thus, we have
\begin{align}
    \mathbb{E}{\norm{\nabla_{\vlambda} f\left(\vt_{\vlambda}\left(\rvvu\right)\right)}_2^2} 
    &\geq 
    \frac{2 \mu^2 \left(d + 1\right)}{L} \left(\mathbb{E}{f\left(\vt_{\vlambda}\left(\rvvu\right)\right)} - f^*\right).
    \label{eq:thm_lower_bound_eq2}
\end{align}

Now, from \cref{eq:thm_gradient_variance_general_definition},
\begin{align*}
\mathbb{E}\norm{ \vg_M }_2^2
  &= \frac{1}{M} \left(
        \mathbb{E}{ \norm{\nabla_{\vlambda} f\left(\vt_{\vlambda}\left(\rvvu\right)\right)}_2^2 }
        -
        \norm{\mathbb{E}{ \nabla_{\vlambda} f\left(\vt_{\vlambda}\left(\rvvu\right)\right)} }_2^2
    \right) \\
    &\qquad+ \norm{ \nabla F\left(\vlambda\right) }^2_2,
\shortintertext{applying \cref{eq:thm_lower_bound_eq1},}
  &\geq
  \frac{1}{M} 
  \left(
    \mathbb{E}{ \norm{\nabla_{\vlambda} f\left(\vt_{\vlambda}\left(\rvvu\right)\right)}_2^2 }
    -
    2 L^2 \left(\mathbb{E}{f\left(\vt_{\vlambda}\left(\rvvu\right)\right)} - f^*\right) 
  \right) \\
  &\qquad+ \norm{ \nabla F\left(\vlambda\right) }^2_2 
\shortintertext{applying \cref{eq:thm_lower_bound_eq2},}
  &\geq
  \frac{2 \mu^2 \left(d + 1\right) - 2L^2}{ML} 
  \left(\mathbb{E}{f\left(\vt_{\vlambda}\left(\rvvu\right)\right)} - f^*\right) \\
  &\qquad + \norm{ \nabla F\left(\vlambda\right) }^2_2 
  \\
  &\geq 
  \frac{2 \mu^2 \left(d + 1\right) - 2L^2}{ML} \left(F\left(\vlambda\right) - h\left(\vlambda\right) - f^*\right) \\
  &\qquad+ \norm{ \nabla F\left(\vlambda\right) }^2_2 
  \\
  &= \frac{2 \mu^2 \left(d + 1\right) - 2L^2}{ML} \left(F\left(\vlambda\right) - F^*\right) + \norm{ \nabla F\left(\vlambda\right) }^2_2  \\
  &\qquad+ \frac{2 \mu^2 \left(d + 1\right) - 2 L^2}{ML} \left(F^* - f^* - h\left(\vlambda\right) \right).
\end{align*}
The last term
\begin{align*}
    \frac{2 \mu^2 \left(d + 1\right) - 2 L^2}{ML} \left(F^* - f^* - h\left(\vlambda\right) \right)
\end{align*}
can be shown to be positive if $\vlambda$ is sufficiently close to the optimum.
Let $\vlambda^* = \argmin_{\vlambda} F\left(\vlambda\right)$ be the minimizer of $F$.
Then, we have
\begin{align*}
    F^* - f^* - h\left(\vlambda\right)  
    &= 
    \mathbb{E}{f\left(\vt_{\vlambda^*}\left(\vu\right)\right)} + h\left(\vlambda^*\right) - f^* - h\left(\vlambda\right) 
    \\
    &=
    \left(\mathbb{E}{f\left(\vt_{\vlambda^*}\left(\vu\right)\right)} - f^*\right) + \left(h\left(\vlambda^*\right) - h\left(\vlambda\right)\right),
\end{align*}
where the first term is strictly positive and the second term goes to zero as $\vlambda \to \vlambda^*$.
\end{proofEnd}

%%% Local Variables:
%%% TeX-master: "main"
%%% End:


\begin{remark}[\textbf{Matching Dimensional Dependence}]
  For well-conditioned problems such that \(\nicefrac{L}{\mu} < \sqrt{d+1}\), a lower bound of the same dimensional dependence with our upper bounds holds near the optimum. 
\end{remark}

\begin{remark}[\textbf{Unimprovability of the ABC Condition}]
  The lower bound suggests that the \(ABC\) gradient variance condition is unimprovable within the class of smooth, quadratically growing functions.
\end{remark}

%%% Local Variables:
%%% TeX-master: "main"
%%% End:
