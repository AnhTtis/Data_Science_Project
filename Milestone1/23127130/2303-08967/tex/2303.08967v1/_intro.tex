\section{Introduction}
\label{sec:intro}

With growing popularity of microphone arrays in devices such as wearables, speech enhancement takes advantage of multi-channel processing by exploiting the signal spatial characteristics. Multi-channel speech enhancement has applications in hearing aids, augmented reality, teleconferencing and robot audition \cite{Doclo2010,Lollmann2017,HaebUmbach2019} and typically consists of a \ac{MISO} block optionally followed by a single-channel post-processing. The focus in the work is on the \ac{MISO} block for a single target, however, the problem can be extended to multi-target by repeating a method for different targets.

The existing \ac{MISO} methods can be generally grouped into analytical or \ac{ML} approaches. The \ac{ML} methods \cite{Liu2018,Erdogan2016} have shown promising results but may not generalise to unseen conditions. For wearable arrays, this is potentially problematic due to the additional dimensions of complexity and variation caused by rapid movements (translation and rotation) of the array.

For analytical approaches, beamforming has been widely used for decades due to its computational simplicity and robustness. Signal-independent beamformers, such as superdirective \cite{Bitzer2001a} or delay-and-sum, provide fast and robust computation with pre-calculated weights whereas adaptive beamformers, such as \ac{MVDR} \cite{Trees2002,Capon1969}, can potentially provide better results but at the cost of more computation and the risk of signal distortion \cite{Cox1973a,Ehrenberg2010} due to errors in the array steering vector or target \ac{DOA}. A particular challenge for adaptive beamforming in the context of wearable arrays is that the short-term stationarity assumption \cite{Gannot2017} may be violated during head rotations, especially for anisotropic noise fields.

In this work, we consider the `cocktail party' \cite{Cherry1953} scenario where the subject wearing the array and the single target are surrounded by temporally dynamic ambient noise such as babble noise and with possible presence of nearby interference(s). The target \ac{DOA} with respect to the rotated array and the array's \acp{ATF} with realistic accuracy are assumed to be either known a \textit{priori} or else can be estimated \cite{Zhang2019a,Gannot2017,Schwartz2016b}.

The remainder of the paper is structured as follows: Section~\ref{sec:baselines} reviews the technical background for the signal-independent and adaptive beamformers used as baseline. Section~\ref{sec:proposed} introduces the proposed method and its novel blocks. In Section~\ref{sec:eval}, two versions of the proposed method are compared with the baseline using real-recording `cocktail party' scenario with head-worn array. Finally, conclusions are given in Section~\ref{sec:conclusion}.