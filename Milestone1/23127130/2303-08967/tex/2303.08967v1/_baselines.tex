\section{Baseline Beamformers}
\label{sec:baselines}


Let $\beamin\TF=[\beaminnum_{1}\TF\ldots\beaminnum_{\nMic}\TF]^{T}\in\mathbb{C}^{\nMic\times1}$ denote the vector of the observed signals $\beaminnum_{\mic}\TF$ at time frame index $\t$, frequency index $\f$, microphone index $\mic$ for a total of $
\nMic$ microphones. The beamformer output is
\begin{equation}
    \beamoutnum\TF = \weight^{H}\TF\beamin\TF,\label{eq:beamforming}
\end{equation}
where $\weight\in\mathbb{C}^{\nMic\times1}$ is beamformer weights and $(.)^H$ is the Hermitian transpose. For notational simplicity, the $\TF$ dependency will be omitted for the remaining of the paper unless stated otherwise.

Based on the \ac{MVDR} beamformer, the weights can be derived as \cite{Capon1969}
\begin{equation}
    \weight = (\ATF^{H}\NCM^{-1}\ATF)^{-1}\NCM^{-1}\ATF,\label{eq:mvdr}
\end{equation}
where $\ATF=\steervec(\doa_{s})\in\mathbb{C}^{\nMic\times1}$ is the steering vector $\steervec$ for the target \ac{DOA} $\doa_{s}$, $\NCM\in\mathbb{C}^{\nMic\times\nMic}$ is the \ac{NCM} and $(.)^{-1}$ denotes the inversion operator.

\subsection{Iso-MVDR (Superdirective)}
\label{ssec:Iso}

The Isotropic-\ac{MVDR}, referred to as Iso, assumes a stationary spherically isotropic \ac{NCM} as $\NCM$ in \eqref{eq:mvdr}. This is equivalent to assuming uncorrelated plane waves with equal power arriving from all directions. The spherically isotropic diffuse covariance matrix can be obtained as
\begin{equation}
	\NCM_{\gamma} = \int_{\doa} \steervec(\doa) {\steervec}^{H}(\doa) d\doa, 
\label{eq:theoretical_iso}
\end{equation}
where $\int_{\doa} d\doa= \int_{0}^{2\pi} \int_{0}^{\pi} \sin(\inc) d\inc d\az$ denotes integration along azimuth $\az\in[0,2\pi)$ and inclination $\inc\in[0,\pi]$.

Assuming the \ac{ATF} of the array is available for a discrete set of directions $\mathcal{I}$ from a grid of uniform distribution across azimuth and inclination, then \eqref{eq:theoretical_iso} is approximated by quadrature-weighting the grid of discrete points as
\begin{equation}
   \NCM_{\text{Iso}} =
\sum_{i \in \mathcal{I}}w_{i}\steervec(\doa_{i}) {\steervec}^{H}(\doa_{i}),
\label{eq:iso}
\end{equation}
where $w_{i}$ is the quadrature weight for each sample point given by \cite{Driscoll1994}
\begin{equation}
   w_{i} = \frac{2\sin \inc_i}{N_\az N_\inc}
\sum_{m=0}^{0.5N_\inc-1} \frac{\sin\left(\left( 2m+1\right) \inc_i \right)}{2m+1},
\label{eq:quadrature_weights}
\end{equation}
in which $\inc_i$ is the inclination of sample point $i$ and  the number of sample points in azimuth and inclination are $N_\az$ and $N_\inc$ respectively. Note that the quadrature weighting is done to preserve the uniform power isotropy by compensating for higher density of points closer to the poles in a uniform grid spatial sampling scheme. The use of other spatial sampling schemes may require no or different weighting. Using $\NCM=\NCM_{\text{Iso}}$ in \eqref{eq:mvdr} and substituting it in \eqref{eq:beamforming}, the output of this beamformer is denoted as $\beamoutnum_{\text{Iso}}$.


\subsection{MPDR}
\label{ssec:MPDR}

The \ac{MPDR} assumes \ac{SCM} as the $\NCM$ in \eqref{eq:mvdr} given as 
\begin{equation}
    \NCM_{\text{SCM}} =\expectOp\{\beamin\beamin^{H}\},\label{eq:SCM}
\end{equation}
where $\expectOp\{.\}$ is the expectation operator. An estimate of the \ac{SCM} is obtained by applying an \ac{EMA} to the instantaneous covariance matrix 
\begin{align}
    \NCM_{\text{MPDR}}\TF=\alpha\NCM_{\text{MPDR}}(\t-1,\f)\label{eq:mpdr} \\ +    (1-\alpha)\beamin\TF\beamin^{H}\TF,\nonumber
\end{align}
where $\alpha=e^{-\Delta t / T}$ is the smoothing factor (between $0$ and $1$), $\Delta t$ is the time step between frames and $T$ is the time constant.