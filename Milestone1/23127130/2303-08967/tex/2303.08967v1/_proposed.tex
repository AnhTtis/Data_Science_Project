\section{Proposed Method}
\label{sec:proposed}

\begin{figure}[t]
\centering
\includegraphics[width=.5\textwidth]{images/block.png}
\caption{The proposed system block diagram.}
\label{fig:system}
\end{figure}

As illustrated in Fig.~\ref{fig:system}, the proposed method consists of two-stage multi-channel processing. In the first stage two types of beamforming are performed at every \ac{TF} bin. In the second stage, the spectrum output of both beamformers at each time frame are combined to form a two-channel data on which \acs{PCA} \cite{Jolliffe2002} is performed to output the final enhanced signal. In addition to Iso-\ac{MVDR}, described in Section \ref{ssec:Iso}, the system makes use of two novel processing blocks described as follows.

\subsection{Hybrid-MVDR}
\label{ssec:Hybrid}

This block, referred to as Hybrid, performs multiple \ac{MVDR} beamforming with the same steering vector for a known target direction but various \acp{NCM} taken from a dictionary of pre-defined noise sound field models. The dictionary denoted as $P=\{\weight_{m}(\f,\Omega)\}$ contains the pre-calculated beamformer weights for each sample noise field model $m$, frequency index $\f$ and steering direction $\Omega$ where $1\leq m\leq\nModel$ for a total number of  $\nModel$ models. The output with the minimum power among the models is selected and denoted as 
\begin{align}
    & \beamoutnum_{\text{Hyb}}= (\weight_{j})^{H}\beamin,\\
    \mathrm{with\quad} & j = \argmin_{m}{\{\lVert (\weight_{m})^{H}\beamin \rVert^{2}\}}.\label{eq:minimisation}
\end{align}


The dictionary can contain beamformer weights based on a variety of noise field models such as isotropic, anisotropic, \acp{PW}, spatially uncorrelated (diagonal covariance), and potentially more complex ones such as combination of basic ones or previously measured \acp{NCM}. The size and the models used in the dictionary are described in Section~\ref{ssec:pool}.

As will be shown in Section~\ref{sec:eval}, although Hybrid beamformer results in stronger acoustic noise reduction, the output contains `musical noise' due to rapid switching of beam pattern caused by potential selection of highly different models for neighbouring time frames and frequencies. To suppress this musical noise, which is assumed to be uncorrelated, or only partially correlated, with the acoustic noise, the next proposed block extracts that component of Hybrid-\ac{MVDR} output which is correlated with the Iso-\ac{MVDR} output.

\subsection{Spectral PCA Denoising}
\label{ssec:PCA}

In each time frame, let $\beamout(\t)=[\beamoutnum(\t,1)\ldots\beamoutnum(\t,\nFreq)]^{T}$ denote the spectrum output for a beamformer with a total of $\nFreq$ frequency bands. The associated outputs from Hybrid and Iso beamformers (identified by subscript) are joint to form a two-channel complex-value array of data
\begin{equation}
    \outarray(\t) = [\beamout_{\text{Hyb}}(\t), \beamout_{\text{Iso}}(\t)].\label{eq:array_constuct}
\end{equation}
The $2\times2$ inter-channel covariance matrix of $\outarray(\t)$ is then
\begin{equation}
    \NCM(\t) =\expectOp\{\outarray^{H}(\t)\outarray(\t)\},\label{eq:Rz_theoretical}
\end{equation}
which can be approximated, using \ac{EMA}, as
\begin{equation}
    \NCM_{\text{Z}}(\t) =\alpha\NCM_{\text{Z}}(\t-1)+(1-\alpha)(\outarray^{H}(\t)\outarray(\t)).\label{eq:Rz}
\end{equation}
Using \ac{EVD}, $\NCM_{\text{Z}}$ is decomposed as
\begin{equation}
    \NCM_{\text{Z}}(\t) = \eigvec(\t)\eigval(\t)\eigvec^{-1}(\t),
\end{equation}
where $\eigvec\in\mathbb{C}^{2\times2}$ and $\eigval$ are respectively the eigenvectors and diagonal matrix of eigenvalues. Assuming the columns of $\eigval$ are sorted in descending order of eigenvalues, the first column of $\mathbf{U}(\t)=[\mathbf{U}_{S}(\t),\mathbf{U}_{N}(\t)]$ is considered as signal eigenvector denoted as $\mathbf{U}_{S}(\t)\in\mathbb{C}^{2\times1}$. 

The \ac{SS} of the $\outarray(\t)$ is reconstructed as 
\begin{equation}
    \outarray_{\text{SS}}(\t) = \outarray(\t)\eigvec_{S}(\t)\eigvec_{S}^{H}(\t),\label{eq:SS} 
\end{equation}
where the first column of $\outarray_{\text{SS}}(\t)=[\beamout_{\text{SS-Hyb}}(\t),\beamout_{\text{SS-Iso}}(\t)]$ is considered as the final spectrum output of the system and denoted as $\beamout_{\text{SS-Hyb}}$.

\subsection{Dictionary composition}
\label{ssec:pool}

Two variations of dictionary $P$, in Hybrid are considered using available \acp{ATF}. In the first version, denoted as SS-Hyb, the \ac{NCM} models consist of the identity matrix, spherically isotropic noise and five unimodal anisotropic distributions across horizon, as shown in Fig.~\ref{fig:isotropy}, horizontally rotated for every $\SI{6}{\deg}$ azimuth spacing as for the peak position and were quadrature weighted along the inclination to form the 3D sound field. For unimodal anisotropic models, the power was a linear function of azimuth with power dynamic ranges of $\{8,16,24,32,40\} \SI{}{\dB}$. The second version, denoted as SS-HybX, extends the number of models in the dictionary by additionally including individual \ac{PW} models for all available \ac{ATF} directions.

\begin{figure}[t]
    \centering
    \includegraphics[scale=.7]{images/anisotropy.png}
    \caption{Isotropic and five unimodal anisotropic models for the horizontal isotropy of the noise field.}
    \label{fig:isotropy}
\end{figure}