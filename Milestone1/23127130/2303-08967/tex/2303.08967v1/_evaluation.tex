\section{Evaluations}
\label{sec:eval}

\begin{figure*}[t]
    \centering
    \includegraphics[width=\textwidth]{images/ICASSP_results.png}
    \caption{The absolute (top) and relative (bottom) STOI \cite{Taal2011}, fwSegSNR \cite{Hu2008b}, PESQ \cite{Rix2001}, SDR \cite{vincentPerformanceMeasurementBlind2006a}, SIR \cite{vincentPerformanceMeasurementBlind2006a} and SAR \cite{vincentPerformanceMeasurementBlind2006a}.}
    \label{fig:results}
\end{figure*}

In this section, the two implementations of the proposed method are compared with the Iso-\ac{MVDR} and adaptive \ac{MPDR} baseline beamformers as well as the `passthrough' signal at the reference microphone. 

\subsection{Dataset and Array}
\label{ssec:dataset}
In the context of augmented hearing and augmented reality, the EasyCom dataset \cite{Donley2021a} was used for evaluation. It contains recordings of `cocktail-party' scenarios where a subject with 6-channel head-worn array (four microphones fixed to a pair of glasses and two positioned in the ears) was sat down with multiple talkers at a table surrounded by ten loudspeakers playing restaurant-like ambient noises. The target \ac{DOA} over time is provided via head-tracking metadata for each talker. For this evaluation, the dataset is split into $\SI{6}{\s}$ chunks such that the target onset occurs after $\SI{2}{\s}$ based on the voice activity metadata in \cite{Donley2021a}. For the results visualization, the chunks were categorized according to the number of active sources per chunk denoted as `nSources' = $[1, 2, 3]$. Close-talking headset microphones for each talker, also included in \cite{Donley2021a}, were time- and level-aligned to the array reference microphone for use in intrusive metrics.

To avoid spatial aliasing due to the array geometry, data was down-sampled to $\SI{10}{\kHz}$ sample rate. The \ac{STFT} used $\SI{16}{\ms}$ time-window and $\SI{8}{\ms}$ step. The smoothing factor $\alpha$ in \eqref{eq:mpdr} and \eqref{eq:Rz} was chosen empirically with $T=\SI{50}{\ms}$ and $T=\SI{80}{\ms}$, respectively, for
\ac{MPDR} and SS-Hybrid. The condition number of the $\NCM$ for PWs in the dictionary of SS-HybX was limited to maximum of $100$ via \acs{NCM} diagonal loading to avoid ill-condition covariance matrices. Our investigation showed no necessity of condition number limiting for the other \ac{NCM} models, at least for this dataset.

\subsection{Results and Discussions}
\label{ssec:dataset}

\begin{figure}[t]
    \centering
    \includegraphics[scale=.32]{images/stft.png}
    \caption{Spectrograms of the methods for a trial (nSources=$1$).}
    \label{fig:stft}
\end{figure}

Some audio examples of the results and animated visualization of the beam patterns are available at \cite{Demo}. Figure~\ref{fig:results} shows the absolute (top row) and relative (bottom row) performance according to various intrusive metrics. The black dot indicates the mean while a star indicates no significant difference from the Iso-MVDR (dashed line) according to paired t-test at $p=5\%$ level. 

It can be seen that SS-Hyb outperforms the best baseline (Iso) by an average of $0.01$ STOI, $\SI{3}{\dB}$ fwSegSNR, $\SI{1.5}{\dB}$ SDR, $\SI{2}{\dB}$ SIR, $\SI{1.8}{\dB}$ SAR while sharing the best PESQ with Iso. On the other hand, SS-HybX performs better than SS-Hyb in noise suppression with additional mean of $\SI{2}{\dB}$ fwSegSNR, $\SI{1}{\dB}$ SDR, SIR and SAR particularly in the presence of interferer talker(s) (nSource$>1$), due to utilization of \ac{PW} models, while sharing the same STOI and PESQ with Iso. Although \ac{MPDR} provides the highest noise suppression, substantial target distortion caused by proneness to imperfection in target \acp{ATF} and \ac{DOA} leads to poor STOI, PESQ, SIR and SAR scores.

Figure~\ref{fig:stft} shows the spectrograms of Passthrough, Iso, Hybrid and SS-Hybrid for a representative trial. Although Hybrid provides more noise suppression than Iso, it contains noticeable musical noise, which is suppressed in SS-Hybrid via \ac{PCA}. 