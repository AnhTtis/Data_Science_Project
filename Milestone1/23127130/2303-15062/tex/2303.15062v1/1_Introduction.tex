\section{Introduction}
\label{sec:intro}

Recently proposed instance segmentation methods~\cite{(MRCNN)he2017mask,(solov2)wang2020solov2,(mask_transfiner)ke2022mask,(SOLQ)dong2021solq,(queryinst)fang2021instances,(HTC)chen2019hybrid,(condinst)tian2020conditional,(blendmask)chen2020blendmask,(centermask)lee2020centermask,(yolact)bolya2019yolact} have achieved remarkable performance owing to the availability of abundant of segmentation labels for training.
However, compared to other label types ($e.g.,$ bounding box or point), segmentation labels necessitate delicate pixel-level annotations, demanding much more monetary cost and human effort. Consequently, weakly-supervised instance segmentation (WSIS) and semi-supervised instance segmentation (SSIS) approaches have gained attention to reduce annotation costs. WSIS approaches alternatively utilize inexpensive weak labels such as image-level labels~\cite{(BESTIE)kim2022beyond,(irn)ahn2019weakly,(prm)zhou2018weakly}, point labels~\cite{(BESTIE)kim2022beyond,cheng2022pointly,(wisenet)laradji2020proposal} or bounding box labels~\cite{(BBAM)lee2021bbam, (boxinst)tian2021boxinst,(tightness)hsu2019weakly}.
Besides, SSIS approaches~\cite{(NB)wang2022noisy,(shapeprop)zhou2020learning} employ a small amount of pixel-level (fully) labeled data and a massive amount of unlabeled data.
Although they have shown potential in budget-efficient instance segmentation, there still exists a large performance gap between theirs and the results of fully-supervised learning methods.

\begin{figure}[t]
    \centering
    \includegraphics[width=0.95\linewidth]{figures/motivation.pdf}
    \caption{
        \textbf{Proposals and instance masks}. The absence of a proposal leads to the missing mask, even though the mask could be generated if given the correct proposal (zebra). Also, noise proposal often leads to noisy masks. Our motivation stems from the bottleneck in the proposal branch, and this paper shows economic point labels can be leveraged to resolve it.
    }
    \label{fig:motivation}
    \vspace{-2mm}
\end{figure}


Specifically, SSIS approaches often adopt the following training pipeline: (1) train a base network with fully labeled data, (2) generate pseudo instance masks for unlabeled images using the base network, and (3) train a target network using both full and pseudo labels.
The major challenge of SSIS approaches comes from the trade-off between the number of missing ($i.e.,$ false-negative) and noise ($i.e.,$ false-positive) samples in the pseudo labels.
Namely, some strategies for reducing false-negatives, which is equivalent to increasing true-positives, often end up increasing false-positives accordingly; an abundance of false-negatives or false-positives in pseudo labels impedes stable convergence of the target network.
However, optimally reducing false-negatives/positives while increasing true-positives is quite challenging and remains a significant challenge for SSIS.

To address this challenge, we first revisit the fundamental behavior of the instance segmentation framework.
Most existing instance segmentation methods adopt a two-step inference process as shown in Figure \ref{fig:motivation}: (1) generate instance proposals where an instance is represented as a box~\cite{(MRCNN)he2017mask, (HTC)chen2019hybrid, (centermask)lee2020centermask,(msrcnn)huang2019mask} or point~\cite{(solov2)wang2020solov2, (condinst)tian2020conditional,(solo)wang2020solo,(centermask)wang2020centermask} in proposal branch, and (2) produce instance masks for each instance proposal in mask branch.
As shown in Figure \ref{fig:motivation}, if the network fails to obtain an instance proposal ($i.e.,$ false-negative proposal), it cannot produce the corresponding instance mask.
Although the network could represent the instance mask in the mask branch, the absence of the proposal becomes the bottleneck for producing the instance mask.
From the behavior of the network, we suppose that addressing the bottleneck in the proposals is a shortcut to the success of the SSIS.

Motivated by the above observations, we rethink the potential of using point labels as weak supervision. 
The point label contains only a one-pixel categorical instance cue but is budget-friendly as it is as easy as providing image-level labels by human annotators~\cite{(what_point)bearman2016s}.
We note that the point label can be leveraged as an effective source to (i) resolve the performance bottleneck of the instance segmentation network and (ii) optimally balance the trade-off between false-negative and false-positive proposals.
Thus, we formulate a new practical training scheme, \textbf{Weakly Semi-Supervised Instance Segmentation (WSSIS) with point labels}.
In the WSSIS task, we utilize a small amount of fully labeled data and a massive amount of point labeled data for budget-efficient and high-performance instance segmentation.

Under the WSSIS setting, we filter out the proposals to keep only true-positive proposals using the point labels.
Then, given the true-positive proposals, we exploit the mask representation of the network learned by fully labeled data to produce high-quality pseudo instance masks.
For properly leveraging point labels, we consider the characteristics of the feature pyramid network (FPN)~\cite{(fpn)lin2017feature}, which consists of multi-level feature maps for multi-scale instance recognition.
Each pyramid level is trained to recognize instances of particular sizes, and extracting instance masks from unfit levels often causes inaccurate predictions, as shown in Figure \ref{fig:pyramid}.
However, since point labels do not have instance size information, we handle this using an effective strategy named Adaptive Pyramid-Level Selection.
We estimate which level is the best fit based on the reliability of the network ($i.e.,$ confidence score) and then adaptively produce an instance mask at the selected level.

Meanwhile, on an extremely limited amount of fully labeled data, the network often fails to sufficiently represent the instance mask in the mask branch, resulting in rough and noisy mask outputs.
In other words, the true-positive proposal does not always lead to a true-positive instance mask in this case.
To cope with this limitation, we propose a MaskRefineNet to refine the rough instance mask.
The MaskRefineNet takes three input sources, $i.e.,$ image, rough mask, and point;
the image provides visual information about the target instance, the rough mask is used as the prior knowledge to be refined, and the point information explicitly guides the target instance.
Using the richer instructive input sources, MaskRefineNet can be stably trained even with a limited amount of fully labeled data.

To demonstrate the effectiveness of our method, we conduct extensive experiments on the COCO~\cite{(coco)lin2014microsoft} and BDD100K~\cite{(bdd100k)yu2020bdd100k} datasets.
When training with half of the fully labeled images and the rest of the point labeled images on the COCO dataset ($i.e.,$ 50\% COCO), we achieve a competitive performance with the fully-supervised performance (38.8\% vs. 39.7\%).
In addition, when using a small amount of fully labeled data, $e.g.,$ 5\% of COCO data, the proposed method shows much superior performance than the state-of-the-art SSIS method~\cite{(NB)wang2022noisy} (33.7\% vs. 24.9\%).

\begin{figure*}[t]
    \centering
    \begin{subfigure}[b]{0.31\textwidth}
        \centering
        \includegraphics[width=\textwidth]{figures/result_thresh_01.png}
        \caption{Confidence Threshold=0.1}
        \label{fig:quality_of_pseudo_label_semi_th01}
    \end{subfigure}
    \begin{subfigure}[b]{0.31\textwidth}
        \centering
        \includegraphics[width=\textwidth]{figures/result_thresh_05.png}
        \caption{Confidence Threshold=0.5}
        \label{fig:quality_of_pseudo_label_semi_th05}
    \end{subfigure}
    \begin{subfigure}[b]{0.31\textwidth}
        \centering
        \includegraphics[width=\textwidth]{figures/result_point.png}
        \caption{With Point Labels}
        \label{fig:quality_of_pseudo_label_weak_semi}
    \end{subfigure}
    \caption{
        \textbf{The qualitative results of pseudo instance masks}. (a) and (b): the quality of pseudo masks is largely affected by the confidence score of the proposal due to the trade-off between false-negative and false-positive instance proposals. (c): our point-driven method can filter the proposals to keep only true-positive proposals, resulting in clearer quality of pseudo instance masks.
    }
    \label{fig:quality_of_pseudo_label}
\end{figure*}

In summary, the contributions of our paper are
\begin{itemize}
    \item We show that point labels can be leveraged as effective weak supervisions for budget-efficient and high-performance instance segmentation. Further, based on this observation, we establish a new training protocol named Weakly Semi-Supervised Instance Segmentation (WSSIS) with point labels.
    \item To further boost the quality of the pseudo instance masks when the amount of fully labeled data is extremely limited, we propose the MaskRefineNet, which refines noisy parts of the rough instance masks.
    \item Extensive experimental results show that the proposed method can achieve competitive performance to those of the fully-supervised models while significantly outperforming the semi-supervised methods.
\end{itemize}