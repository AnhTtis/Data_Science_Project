\begin{table}[t]
  \centering
  \small
  \begin{tabular}{c|ccc|cc}
    \toprule
    \multirow{2}{*}{Label Types} & \multicolumn{3}{c|}{COCO \textit{train5K}} & \multicolumn{2}{c}{COCO \textit{val}} \\
    \cline{2-6}
     & $AP$↑ & $AP_{50}$↑ & $AR_{100}$↑ & $AP$↑ & $AP_{50}$↑  \\
    %\midrule \midrule
    \midrule
    $\mathcal{U}$ ($\tau$=0.1)    & 6.0 & 11.3  & 33.8 & 20.8 & 33.5 \\
    $\mathcal{U}$ ($\tau$=0.3)    & 13.1 & 22.9 & 23.4 & 25.9 & 41.5 \\
    $\mathcal{U}$ ($\tau$=0.5)    & 12.2 & 19.3 & 15.6 & 24.3 & 38.2 \\
    $\mathcal{I}$ ($\tau$=0.3)    & 19.5 & 33.1 & 24.1 & 29.5 & 48.9 \\
    \midrule
    $\mathcal{P}$ & 28.6 & 56.7 & 42.6 &  32.2 & 52.3 \\
    $\mathcal{P}^{\dagger}$ & 39.1 & 65.3 & 52.0 & 35.5 & 56.0 \\
    \bottomrule
  \end{tabular}
  \caption{
    \textbf{Impact of using point labels}. We have the notations: $\mathcal{U}$ (unlabeled data), $\mathcal{I}$ (image-level label), $\mathcal{P}$ (point label), and $\mathcal{F}$ (full label).
    We use COCO \textit{train5K} to measure the quality of pseudo labels and COCO \textit{val} to evaluate the baseline network trained with the pseudo labels.
    $\tau$ is a confidence threshold in the proposal branch. $\dagger$ means applying our point-guided MaskRefineNet.
  }
  \label{tab:ablation_threshold}
\end{table}
