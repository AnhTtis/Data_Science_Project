\section{Related Work}

\subsection{Instance Segmentation}
Mask R-CNN~\cite{(MRCNN)he2017mask} is the most widely used method for instance segmentation.
They represent an instance as a bounding box and produce the instance mask after pooling each box region.
These box-based approaches have many variants, such as \cite{(HTC)chen2019hybrid, (centermask)lee2020centermask,(msrcnn)huang2019mask,(detectors)qiao2021detectors} and have shown state-of-the-art results.
Meanwhile, there is a different type of approach, named point-based approaches~\cite{(solov2)wang2020solov2, (condinst)tian2020conditional,(centermask)wang2020centermask,(solo)wang2020solo}.
They represent an instance as a point and generate the instance mask using the point-encoded mask representation.
For example, SOLOv2~\cite{(solov2)wang2020solov2} extracts point-encoded kernel parameters and generates instance masks with a dynamic convolution scheme.
We note that the inference pipeline of these two approaches is the same as shown in Figure~\ref{fig:motivation}; they generate proposals in the form of either bounding boxes or points and then produce an instance mask for each proposal.
Here, the proposal is indispensable for producing the instance mask.


\subsection{Budget-Efficient Instance Segmentation}
Instance segmentation requires a huge amount of instance-level segmentation labels.
However, the annotation cost of segmentation labels is much higher than other labels. 
According to seminar works~\cite{(what_point)bearman2016s, (budget_aware)bellver2019budget}, the annotations time is measured on VOC dataset~\cite{(voc)everingham2010pascal} as follows: image-level (20.0 \textit{s/img}), point (23.3 \textit{s/img}), bounding box (38.1 \textit{s/img}), full mask (239.7 \textit{s/img}).
To reduce the annotation cost, weakly-supervised instance segmentation (WSIS) and semi-supervised instance segmentation (SSIS) have been actively studied.
The WSIS methods exploit the activation maps generated by self-attention of the network trained with only cost-efficient labels such as image-level~\cite{(irn)ahn2019weakly,(BESTIE)kim2022beyond,(prm)zhou2018weakly}, point~\cite{(BESTIE)kim2022beyond, cheng2022pointly,(wisenet)laradji2020proposal}, and bounding box~\cite{(BBAM)lee2021bbam, (boxinst)tian2021boxinst,(tightness)hsu2019weakly} labels.
Meanwhile, the SSIS methods~\cite{(NB)wang2022noisy,(shapeprop)zhou2020learning} use a small amount of fully labeled data and an abundant amount of unlabeled data.
Utilizing the knowledge of the segmentations learned with the fully labeled data, they generate pseudo instance masks for the unlabeled data.
Although they can reduce the annotation cost, their performance is still far behind those of the fully-supervised models.


\subsection{Weakly Semi-Supervised Object Detection}
There exist some previous attempts to tackle the weakly semi-supervised object detection problem using point labels (WSSOD)~\cite{(point_detr)chen2021points, (group_rcnn)zhang2022group}.
Namely, they use a few box-labeled data and a lot of point-labeled data.
Leveraging the point labels, they show improved detection performances compared to the semi-supervised setting.
Object detection and instance segmentation tasks share a similar goal: both are object-level recognition tasks.
However, we point out that the motivation for leveraging point labels is different. We focus on the fundamental drawback of the instance segmentation network to handle the trade-off between false-negative and false-positive proposals.
In contrast, PointDETR~\cite{(point_detr)chen2021points} leverages the point labels as input queries for single-level feature map inference of DETR~\cite{(detr)carion2020end} architecture, and Group R-CNN~\cite{(group_rcnn)zhang2022group} employs the point labels to filter and augment proposals with improved positive sample assignments.
In addition, we propose the MaskRefineNet for high-fidelity mask refinement to handle the distinct challenge of instance segmentation, which is a pixel-level recognition task.