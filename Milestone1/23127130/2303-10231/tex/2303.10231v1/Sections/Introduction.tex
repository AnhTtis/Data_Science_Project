\section{Introduction}

% \blue{Motivate a mathematical notion of robustness: }
Achieving stable bipedal locomotion is a challenging control task---especially when locomoting on rough terrain. 
% especially when considering uncertain terrain.
One approach with demonstrated success is to generate nominal walking behaviors, encoded by periodic orbits, and then use either feedback controllers or online planning to drive the system to these nominal behaviors \cite{grizzle2014models}. A benefit of this approach is that the stability of the nominal gait can then be analyzed using the method of Poincar\'e sections for systems with impulse effects \cite{morris2005restricted, morris2009hybrid}, i.e., one need only check the eigenvalues of the Poincar\'e map. 
Yet this notion of stability is inherently local and does not give insight into the stability in the presence of disturbances such as those experienced with varying terrain height.  

% away from the orbit.  Thus, existing tools for establishing stability to not characterize robustness: the ability to remain stable in the presence of disturbances such as those experienced with varying terrain height.  

% \cite{morris2005restricted, morris2009hybrid}. 

% \blue{Existing notions of stability do not capture robustness: }
% However, this existing notion of locomotive stability, along with others, focuses on the stability of a periodic orbit rather than its ability to remain stable in the presence of disturbances. In other words, the existing tools for analyzing stability do not characterize the robustness of the orbit to perturbations in the environment such as changes to the terrain height. 

There have been approaches that have aimed to analyze the robustness of bipedal walking.  
Examples include the gait sensitivity norm \cite{hobbelen2007disturbance}, and the transverse linearization \cite{manchester2011stable}.  Yet these tools do not provide theoretical certificates of robustness. Similarly, existing work has synthesized bipedal walking gaits that are maximally robust to known environmental disturbances \cite{dai2012optimizing,park2012finite,hamed2016exponentially,tucker2022robust}.  While these have worked well in practice, again there is a lack of theoretic tools to formally asses their robustness, i.e., characterizing the domain on which behaviors are stable.  As a step in this direction, input-to-state stability (ISS)  \cite{sontag2008input} has been effectively leveraged in the context of robotic walking and running for uncertain dynamics \cite{kolathaya2018input, ma2017bipedal}.  But framing the robustness of walking gaits to uncertain environments remains an open problem. 

% Note that previous work has applied ISS in the context of locomotion \cite{kolathaya2018input, ma2017bipedal}, but for uncertain dynamics rather than uncertain environments. 

\begin{figure}[tb]
    \centering
    \includegraphics[width=1\linewidth]{Figures/uncertain_guards5.pdf}
    \vspace{-6mm}
    \caption{A depiction of (left) the configuration coordinates for a seven-link walker and (right) the uncertain guard condition. }
    % The nominal level-ground guard conditions is shown on the left along with the pinned configuration coordinates of the seven-link walker model with the uncertain guard condition is illustrated on the right.}
    \label{fig: uncertainguard}
    \vspace{-3mm}
\end{figure}

% Uneven terrain necessarily transforms periodic walking into a non-periodic behavior. Consequently, stability analysis tools designed for periodic motions, such as the proof of stability via Poincar\'e map analysis, are no longer valid. 

This paper formulates a notion of robust walking that quantifies the gap between stability and robustness mathematically. Explicitly, the goal of our work is to define what it means for a periodic orbit to be certifiably robust to uncertain terrain as illustrated in Fig. \ref{fig: uncertainguard}. Towards this, we define the $\delta$-robustness of periodic orbits in hybrid systems by leveraging input-to-state stability of the Poincar\'e return map. 
% Motivated by previous work on ISS in the context of locomotion, we extend the application of ISS to consider disturbances to the environment.
% input-to-state stability is a theoretical tool which identifies sufficient conditions for stability in the presence of bounded disturbances . \cite{veer2016local}
Specifically, $\delta$-robustness is defined as the maximum disturbance in the guard condition (commonly selected to be the ground height) that can be accommodated while remaining stable to a neighborhood. The main result of our paper is the formulation of robust Lyapunov functions that certify the robustness of periodic orbits to disturbances in the environment. 
The leads to an algorithm for certifying the $\delta$-robustness of walking gaits, as  
demonstrated in simulation with a seven-link bipedal robot walking on uneven terrain. 

% Notably, Dai et. al \cite{dai2012optimizing} demonstrated a method for designing periodic trajectories that maximize robustness to terrain uncertainty by drawing possible ground profiles from a distribution and simultaneously minimizing the expected infinite-horizon cost-to-go for each of the drawn samples. Additionally, \cite{hamed2016exponentially} introduced the notion of uncertain guard conditions in order to numerically compute sensitivity matrices of periodic orbits and systematically tune parameters of a continuous-time controller for optimal robustness to uncertain terrain. Importantly, these previous results demonstrated that robust reference trajectories are capable of reducing the control effort required for bipedal robots to remain stable \cite{dai2012optimizing} and would minimize the need for reactive controllers \cite{park2012finite}.


% The goal in formulating this notion of robustness is to find a single scalar constant, $\delta \geq 0 $, that characterizes the robustness of a periodic orbit $\O$ in the context of uncertain guard height.


% and phase to state stability (PSS) \cite{ma2017bipedal}.  





% One method of offline gait generation is the Hybrid Zero Dynamics method which synthesizes provably stable periodic walking behaviors encoded by periodic orbits \cite{grizzle2014models}. Previous work has demonstrated that through careful design of feedback stabilizing controllers, legged systems can be stabilized around these nominal behaviors, even in the presence of disturbances \cite{ames2014rapidly}.


% Three common approaches towards achieving stable locomotion on bipedal robots include generating walking behaviors offline and enforcing these behaviors through feedback control, online planning using either full-order models or simplified models, and policies obtained through machine learning. Regardless of the utilized approach, the existence of nominal periodic orbits that are inherently robust to ground height disturbances would reduce the required control effort \cite{dai2012optimizing} and would minimize the need for reactive controllers \cite{park2012finite}.
% While these methods each have advantages and disadvantages, one can argue that generating robust nominal gaits is advantageous for all three scenarios since these gaits are often relied on as references .
% generating nominal walking gaits that are naturally robust to uncertain terrain makes it easier to design higher level planning tasks. In other words, 
% instead of relying on immediate disturbance rejection

% \begin{figure}
%     \centering
%     \includegraphics[width=\linewidth]{example-image-a}
%     \caption{Intro Figure Illustrating Robust Walking}
%     \label{fig: introfigure}
% \end{figure}



