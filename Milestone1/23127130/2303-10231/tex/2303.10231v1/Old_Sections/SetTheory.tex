\newpage
\section{A Set Theoretic View of Walking}
To relax the requirement on periodicity for hybrid systems with impulse events, we instead will take a set theoretic perspective.

Consider a compact set of pre-impact states in the guard, $\D \subset S$, with each element of $\D$ being defined as $\D_i := x^-_i \in S$. Note that this set definition assumes a constant and fixed impact surface $S$ (we will relax this assumption in section \ref{sec: uncertainguard}). The corresponding post-impact states is denoted as $\Delta(\D)$. Using the continuity assumption on the impact map, we know since $\D$ is compact, $\Delta(\D)$ is also compact. Using this, we can define the set containing the continuous-time solutions to $\Delta(\D)$ as,
$$
\C := \{\varphi_t(\Delta(x^-_i)) \in \X\} \mid 0 \leq t \leq T_I(x^-_i), \forall x_i^- \in \D\}
$$

Taking the Poincar\'e map to again be the guard, and applying the Poincar\'e map \eqref{eq: poincare} to every point in our set, we can denote the set-based Poincar\'e map as $P(\D) := \{P(x^-_i) \mid \forall x^-_i \in \D\}$. Using this set-based Poincar\'e maps, we can determine stability of the discrete-time system \eqref{eq: discrete-system} through the following theorem.

%%%%%%%%%%%%%%%%%%%%%%%%%%%%%%%%%%%%%%%%%%%%%%%%%%%%%%%%%%%%%%%%%%%%%%%%%%%%%%%%
% \begin{theorem}
%     Let $P$ be the Poincar\'e map $P: S \to S$ applied to a hybrid system $\H\C$. The discrete-time system $$x^-_{k+1} = P(x^-_k)$$ is forward invariant w.r.t $\D$ for any $x_0 \in \D$ if and only if $P(\D) \subset \D$.
% \end{theorem}

% Thus, 
% \begin{definition}
%     A hybrid control system $\H\C$ is defined as \textit{hybrid set invariant} if $P(\D) \subset \D$.
% \end{definition}


%%%%%%%%%%%%%%%%%%%%%%%%%%%%%%%%%%%%%%%%%%%%%%%%%%%%%%%%%%%%%%%%%%%%%%%%%%%%%%%%

\begin{definition}
    A hybrid system $\H\C$ is defined as \textit{Hybrid Set Invariant} with respect to the set $\D \subset S$ if $P(\D) \subset \D$.
\end{definition}

\begin{theorem}
    A hybrid system $\H\C$ is Hybrid Set Invariant ($P(\D) \subset \D)$ if and only if the set of flows $\C := \{\varphi_t(\Delta(x^-_i) \mid 0 \leq t \leq T_I(x^-_i), \forall x^-_i \in \D\}$ is forward invariant and $\Delta(\C \cap S) \subset \C$.
\end{theorem}
\begin{proof}
    $(\Rightarrow)$ If $P(\D) \subset \D$, then by definition $\varphi_{T_I(x^-_i)}(\Delta(x^-_i)) \subset \D$. Thus, the set of all flows $\varphi_t(\Delta(x^-_i))$ must be forward invariant. Additionally, the impact map applied to initial conditions must be contained within this set of flows by construction of $\C$, a.k.a. $\varphi_0(\Delta(x^-_i)) = \Delta(x^-_i) \subset \C$ for all $x^-_i \in \D$. Thus, writing this in terms of the set, $\Delta(\D) \subset \C$. \\
    $(\Leftarrow)$ If $\C$ is forward invariant, then there must exist some set $\D := \{\C \cap S\}$ such that $\forall x_0 \in \C$, $\exists T \in \R^+$ s.t. $\varphi_T(x_0) \in \D$. Then, since $\Delta(\D) \subset \C$, the solution to the flow $\varphi_T(\Delta(\D)) \subset \D$. Thus, by definition of the Poincar\'e map, $P(\D) \subset \D$. 
\end{proof}

% Note that this definition is similar to the definition of \textit{hybrid invariance} as given in \cite{morris2009hybrid}, with the exception that \cite{morris2009hybrid} was in reference to a submanifold $\Z \subset \X$. Thus, the condition for hybrid invariance of $\Z$ in \cite{morris2009hybrid} was the condition $P(S \cap \Z) \subset S \cap \Z$.
%%%%%%%%%%%%%%%%%%%%%%%%%%%%%%%%%%%%%%%%%%%%%%%%%%%%%%%%%%%%%%%%%%%%%%%%%%%%%%%%
\begin{comment}
\blue{Definition 1 of \cite{morris2009hybrid} defined the following.}
\begin{definition}
    For an autonomous system with impulse effects, a submanifold $\Z \subset \X$ is \textit{forward invariant} if for each point $x \in \Z$, $f(x) \in T_x\Z$ where $T_x\Z$ is the tangent space of the manifold $\Z$ at point $x$. A submanifold $\Z$ is \textit{impact invariant} in an autonomous system with impulse effects, or in a control system with impulse effects, if for each point $x$ in $\S \cap \Z$, $\Delta(x) \in \Z$. A submanifold $\Z$ is \textit{hybrid invariant} if it is both forward invariant and impact invariant.
\end{definition}
Also from \cite{morris2009hybrid}, hybrid invariance of $\C$ is reflected in the Poincar\'e map as: $$P(\S \cap \C) \subset \S \cap \C$$. 
\end{comment}

Equipped with this set-based perspective on hybrid flows, we can determine the stability of the solution based on the stability of the set $\D$. Specifically, we propose the following theorem:
\begin{theorem}
    Let $\H\C$ denote a hybrid system with a certain guard condition $S$. If there exists a set $\D \subset S$ such that $P(\D) \subset \D$, then for any initial condition $x^-_0 \in \Delta(\D)$, the discrete-time system is stable as a set..?. \blue{The discrete-time system $z(k+1) = P(z(k))$ is stable if and only if there exists a set $\D$ such that $P(\D) \subset \D$.}
\end{theorem}
\begin{proof}
    \red{Not sure how to formally prove this since it seems almost too intuitive... aka if $P(\D) \subset \D$, then by definition it is forward set invariant.}
\end{proof}

\begin{comment}
\red{We could also come up with stable, asymptotically stable, and exponentially stable definitions for set stability... called something like ``Hybrid Set Stable'', ``Hybrid Set Asymptotically Stable'', and ``Hybrid Set Exponentially Stable''. Corresponding to the conditions $P(\D) \subseteq \D$, $P(\D) \subset \D$, and $P^i(\D) \subset \text{exp. shrinking set}$}
\end{comment}

Note that classic definitions of stability via periodicity still hold for the set-based perspective on stability.
\begin{corollary} 
If there exists a fixed point, defined by $P(x^*) = x^*$, then this still satisfies the conditions for set-based walking: $P(x^*) \subset x^*$. Therefore, stable periodic walking is sufficient for stable set-based walking. (Periodic walking is a subset of set-based walking).
\end{corollary}
\red{An interesting motivating example may be the compass gait biped, with convergence to a periodic orbit but nonperiodic behaviors in the beginning}

\begin{comment}
\subsection{Demonstration for Rabbit}
The domain of attraction for periodic orbits in the full dimensional model are difficult to estimate. However, Westervelt et al. have shown that domains of attraction for restricted Poincar\'e maps can be computed analytically. For example, in Section IV of \cite{westervelt2003hybrid}, the domain of attraction of the fixed point of the restricted Poincar\'e map, $\rho: S \cap \Z_{\alpha} \to S \cap \Z_{\alpha}$, was computed analytically for the RABBIT 5-link walker. 
\end{comment}

\subsection{Connections with Input to State Stability}
It is also interesting to note that the presentation of set-based walking is similar to input-to-state stable walking. Notably, the system $\dot{x} = f(x) + g(x)u$, is input to state stable (ISS) if there exists $\beta \in \KL$ and $\iota \in \K_{\infty}$ such that:
\begin{align}
    |\varphi_t(x_0)| \leq \beta(|x_0|,t) + \iota(\|u\|_{\infty}), ~ \forall x_0,u, ~ \forall t \geq 0.
\end{align}


Mainly, the interesting observation is the f ollowing lemma:

\begin{lemma}
    Given a hybrid periodic orbit that is Input to State Stable (ISS), the resulting system is also set-based stable.
\end{lemma}
\begin{proof}
    \red{From ISS we know that $\C$ is contained in a closed and bounded tube and we know that $P(\D) \subset \D$ for $\D := (\O \oplus \B_{\epsilon}(\O)) \cap S$.
    Note: Define ISS for Poincar\'e map (as discrete time system), then give conditions on CLFs to be ISS-CLFs.}
\end{proof}