\newpage
\section{Connections with Walking}
The set-based definitions of walking can be specialized for the specific application of legged locomotion. Inspired from Definition 1 of \cite{ames2015first}, we can define the set of hybrid flows $\{\C,\D\}$ as a \textit{walking gait} using the following definition. Notably, this definition of walking includes but is not limited to periodic orbits. 
\begin{definition}
A collection of hybrid flows $\{\C, \D\}$ is defined as a \textit{walking gait} if there exist constants $T_{I}^{\text{min}} > 0$ and $p_{com}^{\text{min}} > 0$ such that for every element $x^-_i \in \D$:
% with the corresponding hybrid flow $\C_i \in \C$, 
\begin{align}
    T_I(x^-_i) > T_{I}^{\text{min}}, \tag{Non-Zeno Behavior} \\
    \dot{p}_{\com}^x(P(x^-_i)) > 0, \tag{Progress} \\
    \min_{t \in \leq T_I(x^-_i)} p_{\com}^y(\varphi_t(\Delta(x^-_i))) > p_{\com}^{\min}, \tag{Upright} \\
    % P(\D^i) = \D_{i+1}. \tag{Continuity}
\end{align}
\label{def: walkinggait}
\end{definition}

While Definition \ref{def: walkinggait} provides us with a set-based perspective of walking, we do not yet have constructive tools for how to enforce such set-based walking gaits. To this end, we propose the following theorem,

\begin{theorem}
    Assume there exists a set $\D$ such that $P(\D) \subset \D$, then the conditions listed in Def. \ref{def: walkinggait} are satisfied for all $x_0 \in \D$.
\end{theorem}
\begin{proof}
    \red{Prove this by construction with barrier functions.}
\end{proof}

\begin{corollary}
    When enforced, the following three barrier functions result in set-based stable walking.
    \begin{align}
        h_1 &:= \tau_{i+1}-\tau_{i}-\tau_{\text{min}} \tag{Barrier for Dwell Time} \\
        h_2 &:= \dot{p}_{\text{com}}^x(x_i(\tau_{i+1})) \tag{Barrier for Progress} \\
        h_3 &:= p_{\com}^y(x_i(t)) - p_{\com}^{\min} \tag{Barrier for Upright}
    \end{align}
\end{corollary}

\begin{comment}
\subsection{Enforcing Set-Based Stability}
Although we now have set-based definitions of discrete-time stability for systems with impulse effects, we do not yet have constructive tools for enforcing this stability, or even checking whether a system has a set $\D$ such that $P(\D) \subset \D$. 

One method towards enforcing set-based stability is via  barrier functions.

One method of checking whether a system is set-based stable is via sampling.
\end{comment}