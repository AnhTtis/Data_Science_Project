\section{A Set Theoretic Perspective on Walking: $\delta$-Robustness}
\label{sec: uncertainguard}

This section provides the key formulation of robustness considered throughout this paper---that of $\delta$-robustness.  The core concept behind this definition is stability in and of itself is not a sufficiently rich concept to capture robustness, since it is purely local.  Thus, we define a set-theoretic notion of robustness leveraging the Poincar\'e map. 


\begin{figure}[tb]
    \centering
    \includegraphics[width=\linewidth]{Figures/uncertain_guards3.pdf}
    \caption{Illustration of Uncertain Guard Condition for Five-Link Walker. The nominal level-ground guard conditions is shown on the left along with the configuration coordinates of the five-link walker model used in this work. In comparison, the proposed guard condition for uncertain terrain is illustrated on the right.}
    \label{fig: uncertainguard}
    \vspace{-2mm}
\end{figure}


\begin{figure}[tb]
    \centering
    \includegraphics[width=\linewidth]{Figures/motivation_figure.pdf}
    \caption{Demonstration of a periodic orbit with $\max |\lambda(D P(x^*))| < 1$ (illustrated on the left) not being robust to variations in the guard condition (for this example, the ground height was increased by 1cm), while a periodic orbit with a larger $|\lambda|$ (illustrated on the right) is comparatively more robust.}
    \label{fig: motivation}
\end{figure}

\newsec{Motivation}
Practically, the stability of periodic orbits can be analyzed by evaluating the eigenvalues of the Poincar\'e return map linearized around the fixed point. Specifically, if the magnitude of the eigenvalues of $DP(x^*) = \frac{\partial P}{\partial x}(x^*)$ is less than one (i.e. $\max |\lambda (DP(x^*))| < 1$), then the fixed point is stable \cite{perko2013differential,morris2005restricted,wendel2010rank}. 
% Note that in this process, the eigenvalues corresponding to states known to be nonperiodic, such as the horizontal position of the floating base frame. 
Since it is often difficult to compute the Poincar\'e map analytically, it is commonly numerically approximated.  By applying small perturbations to each state and forward simulating one step to obtain $P(x^*+\delta )$ which is then used to construct each row of the Jacobian successively.  This implicitly implies that the Poincar\'e map is robust to $\delta$ perturbations---as long as $\delta$ is sufficiently small.   Yet, this small amount of robustness inherent in stable gaits is often confounded with the eigenvalues themselves.  That is, it is sometimes assumed that the magnitude of the eignevalues of the Poincar\'e map say something deeper about the broader robustness of the periodic orbit to perturbations.  This is not the case, as the following example illustrates. 


    
% \begin{remark}
%     Orbits $\O$ with lower $\lambda_{\max}$ are regarded as being \textit{more stable} than orbits with higher $\lambda_{\max}$ since they are more contractive around the fixed point.
% \end{remark}

% However, since the Poincar\'e map is evaluated at the fixed point, this notion of stability is only valid locally. This means that while this tool is useful for checking the stability of periodic orbits, it is no longer insightful for systems with uncertain guard events or disturbances that push the system away from the fixed point. 

% \subsection{Simulation Example}
% \blue{Take two gaits, with one gait having a smaller eigenvalues than the other. Show that in the presence of uncertain terrain, the "more stable gait" fails, while the other one doesn't.}

\begin{example}
Consider a 5-link bipedal robot shown in Figure \ref{fig: uncertainguard}.  To illustrate how the eigenvalues associated with the linearization fail to tell the whole story, we will consider the robustness of these gaits to differing ground height conditions.  
As illustrated in Figure \ref{fig: motivation}, the classic Poincar\'e analysis does not accurately reflect the robustness of periodic orbits to local disturbances in the guard condition. That is, the gait with the smaller maximum eigenvalue (magnitude) is more fragile to changing ground heights. 
%In this work, we will instead propose tools for discussing the stability of walking (described by either periodic and nonperiodic orbits) from a set-theoretic perspective. This will then allow us to introduce tools for analyzing the stability of walking away from the fixed point.
\end{example}


\newsec{Set-Based Robustness}
% Next, we will extend the notion of set-stable walking to include uncertain guard conditions. 
We now formulate a notion of robustness of the extended Poincar\'e map to uncertain guard conditions. 
Specifically, we consider the general guard (as defined in \cite{hamed2016exponentially}) for some $d \in \R$,
\begin{align}
    S_d &= \{x\in \mathcal{X} \mid h(x) = d, ~\dot h(x) < 0\},
    \label{eq: heightguard}
\end{align}
with $d \in \mathbb{D}$ and $\mathbb{D} := [d^-,d^+] \subset \R$ for some $d^- < 0 < d^+$. Using this general guard definition, the previous guard \eqref{eq: zeroguard} is now denoted as $S_0$.  Under the assumption that $S_d \subset D$ for all $d \in \mathbb{D}$, we have a corresponding hybrid system: 
\begin{numcases}{\mathcal{H}_d  = }
\dot{x} = f_{\rm cl}(x) := f(x) + g(x) k(x) & $x \in D \setminus S_d$, \label{eq: continuousd}
\\
x^+ = \Delta(x^-) & $x^- \in S_d$, \label{eq: discreted}
\end{numcases}

% It is important to note that the typical assumptions made for systems considering $\S_0$ still apply for systems considering $\S_d$ with $d \not= 0$. Mainly,
% \begin{enumerate}
%     \item The continuous-time dynamical system \eqref{eq: continuous} is continuous on $\X$ for any guard condition $S_{d}$
%     \item The solution to \eqref{eq: continuous} from any initial condition $x_0 \in \Delta(S_d)$ is unique and depends continuously on $x_0$ for any guard condition $S_{d}$
%     \item The function $h: \X \to \R$ is differentiable such that for every $s \in S_d$, $\frac{\partial h}{\partial s}(x) \not= 0$ for any $d \in \mathbb{D}$
% \end{enumerate}

Consider the extended time to impact function $T_e : B_{\rho}(x^*) \to \R$ defined implicitly in \eqref{eqn:extendedtimetoimpact}.  This function can be further extended (as a partial function) to account for varying guards: $T_e : B_{\rho}(x^*) \times \mathbb{D} \partialto \R$:
\begin{align}
    T_e(x_0,d) := \inf\{ t \geq 0 \mid \varphi_t(x_0) \in S_d \}.
\end{align}
Importantly, this is a partial function because (by the implicit function theorem) it is only well-defined for $d = 0$ and by continuity sufficiently small $d^-$ and $d^+$.  
Using this extended time-to-impact function, we can define the extended Poincar\'e map as a partial function: $P_d: B_{\rho}(x^*) \partialto S_d$:
\begin{align}
    P_{d}(x^-) := \varphi_{T_e(\Delta(x^-),d)}(\Delta(x^-)).
\end{align}
We now frame robust walking as the forward invariance of this map, for all $d$. 
%Note that the key feature of this definition $\delta$, wherein the goal is to find periodic gaits with maximum $\delta$. 

% The extended time-to-impact function $T_e: \bar{\X} \times \mathbb{D} \to \R^+$ is then defined similarly to the time-to-impact function, with the exception that it is specific to the guard $S_d$:
% \begin{align}
%     T_e(x_0,d) := \inf\{ t \geq 0 \mid \varphi_t(x_0) \in S_d \}.
% \end{align}
% Note here that $T_e$ is well-defined for the neighborhood $\bar{\X}$ satisfying:
% $$\bar{\X} := \{x \in \X \mid 0 < T_e(x,d) < \infty, \dot{h}(x) < 0, \forall d \in \mathbb{D}\}.$$
% Using this extended time-to-impact function, we can define the extended Poincar\'e map, $P_{d}: \bar{S} \to S_d$:
% \begin{align}
%     P_{d}(x^-) := \varphi_{T_e(\Delta(x^-),d)}(\Delta(x^-)),
% \end{align}
% that is well-defined for the open set $\bar{S} := \Delta^{-1}(\bar{X})$.



% Note that this extended Poincar\'e map is well-defined and continuous in a neighborhood around $x_{k-1}$ (\red{can prove in Appendix by extending Lemma 3 in \cite{grizzle2001asymptotically}}). In other words, the Poincar\'e map is defined as $P_{d_k}: \widetilde{S}_{d_{k-1}} \to S_{d_k}$ with $\widetilde{S}_{d_{k-1}} := \Delta^-(\widetilde{\X})$ for $\widetilde{X} := \{x \in \X \mid 0 < T_e(x,d_k) < \infty \}$. 

% Also note that the extended Poincar\'e map for some interval range $d \in [d^-,d^+] \in \R$ can also be interpreted as a set-valued Poincar\'e map:
% \begin{align}
% \P_{[d^-,d^+]}(x):= \{ P_d(x) \mid \forall d \in [d^-,d^+]\}
% \label{eq: setvaluedP}
% \end{align}

% We are now prepared for the following definition:
\begin{definition}  
\label{def: deltarobust}
Let $\O$ be a periodic orbit with fixed point $P_0(x^*) = x^* \in \O \cap S_0$ corresponding to the guard condition $\S_0$ as in \eqref{eq: heightguard}. The orbit $\O$ is \textbf{$\delta$-robust} for a given $\delta > 0$ if:
\begin{align*}
    P_d(\B_{\delta}(x^*)) \subseteq \B_{\delta}(x^*), ~\forall d \in [d^-_{\delta},d^+_{\delta}],
\end{align*}
with the bounds on $d$, $[d^-_{\delta},d^+_{\delta}]$, implicitly determined by: 
$$
d^-_{\delta} = \inf_{x \in B_{\delta}(x^*)} h(x), \qquad 
d^+_{\delta} = \sup_{x \in B_{\delta}(x^*)} h(x).
$$
% Moreover, $\O$ is \textbf{$\delta$-robustly exponentially stable} if there exists an $\epsilon > 0$ such that:
% \begin{align}
%   \forall ~ x \in B_{\delta + \epsilon}(x^*) \backslash B_{\delta}(x^*)  &  \quad \implies \quad  \nonumber\\
%  & \| P^i_d(x) \|_{B_{\delta}(x^*)} \leq M\alpha^i \|x \|_{B_{\delta}(x^*)}, \nonumber
% \end{align}
% % \begin{align}
% %   \forall ~ x  \:\: \mathrm{s.t.}  \:\:  &  0 < \|x \|_{B_{\delta}(x^*)} < \epsilon \quad \implies \quad  \nonumber\\
% %  & \| P^i(x) \|_{B_{\delta}(x^*)} \leq M\alpha^i \|x \|_{B_{\delta}(x^*)}, \nonumber
% % \end{align}
% for $M > 0$ and $\alpha \in (0,1)$. 
% \begin{align*}
%     d^-_{\delta} := \inf\{ h(x) \mid x \in \B_{\delta}(x^*)\}, \\
%     d^+_{\delta} := \sup \{h(x) \mid x \in \B_{\delta}(x^*)\}.
% \end{align*}
\end{definition}


\begin{remark}
Implicit in this definition is that $P_d$ is a well-defined function, not a partial function, on the entire ball $B_{\delta}(x^*)$.  This is clearly not a guaranteed, or even likely, for a given $\delta > 0$.  Hence the key role of $\delta$ in the definition.  As will be seen, the goal will be to determine the largest such $\delta$ for a given periodic orbit.  More generally, this also leads to a synthesis goal: \emph{find periodic orbits that have the largest $\delta$-robustness.}  
\end{remark}


\newsec{Dynamical System Representation}
It is important to note that we can view $\delta$-robustness from the perspective of forward set invariance of $B_{\delta}(x^*)$.  
%
Consider the following discrete-time dynamical system, dependent on a specific guard height $d = h(x)$ (viewed as a control input):
\begin{align}
    \label{eq: discretetime}
    x_{k+1} = P_{d_k}(x_k), \qquad d_k \in [d^-,d^+] \subset \R
\end{align}
for some sequence of $d_k$, $k \in \N_{\geq 0}$, determining the guard height specific to step $k \in \mathbb{Z}^+$ such that $x_{k+1} \in S_{d_k}$.


\begin{lemma}
    \label{lemma:forwadinvariance}
    Consider the discrete-time system \eqref{eq: discretetime}. If $B_{\delta}(x^*)$ is forward invariant: 
    \begin{eqnarray}
    \label{eqn:forwardsetinvariance}
    x_0 \in B_{\delta}(x^*) \quad \implies  \quad 
    x_{k+1} = P_{d_k}(x_k) \in B_{\delta}(x^*) 
    \end{eqnarray}
    for all $x \in \N_{\geq 0}$ and for every sequence $\{d_k\}_{k \in \N_{\geq 0}}$ with $d_k \in [d^-_{\delta},d^+_{\delta}]$, then $\O$ is $\delta$-robust. 
\end{lemma}

\begin{proof}
Assume that $\O$ is not $\delta$-robust.  Then there exists an $x_0 \in B_{\delta}(x^*)$ and a $d_0 \in [d^-_{\delta},d^+_{\delta}]$ such that $P_{d_0}(x_0) \notin B_{\delta}(x^*)$.  This contradicts the implication in \eqref{eqn:forwardsetinvariance}. 
\end{proof}


% \begin{lemma}
%     \label{lemma: recursiveset}
%     Consider the discrete-time system \eqref{eq: discretetime}. If $\forall x_0 \in \B_{\delta}(x^*)$, $P_{d_k}(x_0) \in \B_{\delta}(x^*)$, then $x_{k+1} \in \B_{\delta}(x^*)$, $\forall k \in \mathbb{Z}^+$.
% \end{lemma}

% \begin{proof}
%     The lemma can be proven via recursion. Let $x_1 := P_{d_k}(x_0)$. Applying the set-valued Poincar\'e return map to $x_1$ yields $x_2 := P_{d_k}(x_1)$. Per the assumption, $x_1 \in \B_{\delta}(x^*)$. Therefore, given that $\forall x \in \B_{\delta}(x^*)$, $P_{d_k}(x) \in \B_{\delta}(x^*)$, then since $x_1 \in \B_{\delta}(x^*)$, $x_2 \in \B_{\delta}(x^*)$. Applying this $k \in \mathbb{Z}^+$ times will eventually yield that $x_{k+1} \in \B_{\delta}(x^*)$.
% \end{proof}


\newsec{Connections with Stability}
We note that there are natural connections between $\delta$-robustness and stability.  As the following result shows, every exponentially stable periodic orbit is $\delta$-robust.  The key point is: one does not know the $\delta$ associated with this orbit, hence does not have a quantification of the robustness.  This is the key differentiator between stability and robustness: the ability to quantify the size of perturbations that are allowed.  

\begin{proposition} 
\label{prop:stabiltytorobust}
Let $\O$ be a periodic orbit with a fixed point $x^* \in \S_0$, i.e., $P_0(x^*) = x^*$. If $x^*$ is an exponentially stable equilibrium point of $x_{k+1} = P_0(x_k)$, then there exists a $\delta > 0$, s.t. $\O$ is $\delta$-robust. 
\end{proposition}


\begin{proof}
Following the conditions on exponential stability as defined in Theorem. \ref{def: expstability}, since $\O$ has an exponentially stable equilibrium point, then $\exists \delta' > 0$,  $i \in \mathbb{N}_{\geq 0}$, such that for all $x \in \B_{\delta'}(x^*)$,such that $P_0^i(x) \in \B_{\delta'}(x^*)$.

Similarly to the proof of Theorem 1 in \cite{grizzle2001asymptotically}, there must exist some non-zero interval $t \in [T_e(x,d^+),T_e(x,d^-)]$ with $d^+ > 0$ and $d^- < 0$ during which $\dot{h}(\varphi_t(x)) < 0$. 

Furthermore, we can select $d^+$ and $d^-$ sufficiently small, such that $\{\varphi_{T_e(x,d^+)}(x),\varphi_{T_e(x,d^+)}(x)\} \in \B_{\delta'}(x^*)$. Therefore, $\{P_{d^-}(x),P_{d^+}(x)\} \in \B_{\delta}(x^*)$. 

Next, since $\varphi_t(x_0)$ is unique and depends continuously on $x_0$, and since $P_d(x) = P_0(x)$ for $d = 0$ lies in the interior of the segment $\varphi_t(x)$ for $t \in [T_e(x,d^+),T_e(x,d^-)]$, then again we can choose $d^+$ and $d^-$ sufficiently small such that
$$ \|P_d(x) - P_0\| \leq \max \{\|P_{d^-}(x) - P_0(x)\|,\|P_{d^+}(x) - P_0(x)\| \}.$$

Finally, let $\delta > 0$ be defined as the ball such that:
\begin{align*}
    d^- = \inf\{h(x) \mid \forall x \in \B_{\delta}(x^*)\}, \\
    d^+ = \sup\{h(x) \mid \forall x \in \B_{\delta}(x^*)\}.
\end{align*}
Then for any $d \in [d^-,d^+]$, $P_d(x) \in \B_{\delta}(x^*)$, $\forall x \in \B_{\delta}(x^*)$.
\red{STILL IN PROGRESS...}
\end{proof}

% \begin{proof}
% Following the conditions on exponential stability as defined in Theorem. \ref{def: expstability}, since $\O$ has an exponentially stable equilibrium point, then $\exists \delta' > 0$,  $i \in \mathbb{N}_{\geq 0}$, such that for all $x \in \B_{\delta'}(x^*)$,such that $P_0^i(x) \in \B_{\delta'}(x^*)$.

% Similarly to the proof of Theorem 1 in \cite{grizzle2001asymptotically}, there must exist some non-zero interval $t \in [T_e(x,d^+),T_e(x,d^-)]$ with $d^+ > 0$ and $d^- < 0$ during which $\dot{h}(\varphi_t(x)) < 0$. 

% Furthermore, we can select $d^+$ and $d^-$ sufficiently small, such that $\{\varphi_{T_e(x,d^+)}(x),\varphi_{T_e(x,d^+)}(x)\} \in \B_{\delta'}(x^*)$. Therefore, $\{P_{d^-}(x),P_{d^+}(x)\} \in \B_{\delta}(x^*)$. 

% Next, since $\varphi_t(x_0)$ is unique and depends continuously on $x_0$, and since $P_d(x) = P_0(x)$ for $d = 0$ lies in the interior of the segment $\varphi_t(x)$ for $t \in [T_e(x,d^+),T_e(x,d^-)]$, then again we can choose $d^+$ and $d^-$ sufficiently small such that
% $$ \|P_d(x) - P_0\| \leq \max \{\|P_{d^-}(x) - P_0(x)\|,\|P_{d^+}(x) - P_0(x)\| \}.$$

% Finally, let $\delta > 0$ be defined as the ball such that:
% \begin{align*}
%     d^- = \inf\{h(x) \mid \forall x \in \B_{\delta}(x^*)\}, \\
%     d^+ = \sup\{h(x) \mid \forall x \in \B_{\delta}(x^*)\}.
% \end{align*}
% Then for any $d \in [d^-,d^+]$, $P_d(x) \in \B_{\delta}(x^*)$, $\forall x \in \B_{\delta}(x^*)$.
% \red{STILL IN PROGRESS...}
% \end{proof}

\newsec{Connections with Input-to-state Stability}
Note that beyond simply finding the $\delta$-robustness of a gait, it may be desirable to ensure exponential stability to $B_{\delta}(x^*)$.  This leads to the following definition: 

\begin{definition}
The periodic orbit $\O$ is \textbf{$\delta$-robustly exponentially stable} if it is $\delta$-robust and there exists an open and connected set $W$ with $B_{\delta}(x^*) \subset W $ such that:
\begin{align}
  \forall ~ x_0 \in W   &  \quad \implies \quad  \nonumber\\
 & \| x_k \|_{B_{\delta}(x^*)} \leq M\alpha^k \|x_0 \|_{B_{\delta}(x^*)},  \quad \forall k \in \N_{\geq 0} \nonumber
\end{align}
% \begin{align}
%   \forall ~ x \in B_{\delta + \epsilon}(x^*) \backslash B_{\delta}(x^*)  &  \quad \implies \quad  \nonumber\\
%  & \| P^i_d(x) \|_{B_{\delta}(x^*)} \leq M\alpha^i \|x \|_{B_{\delta}(x^*)}, \nonumber
% \end{align}
for $M > 0$ and $\alpha \in (0,1)$. 
\end{definition}


There are obvious connections with input-to-state stability \cite{jiang2001input}.  In our setting, the discrete time system $x_{k+1} = P_{d_k}(x_k)$ (with $d_k$ viewed as an input) is (exponentially) input to state stable if: 
\begin{eqnarray}
\label{eqn:expiss}
\| x_k \| \leq M\alpha^k \| x_0\|  + \gamma( \max\{-d^-,d^+\} )
\end{eqnarray}
for $k \in \N_{\geq 0}$, $M > 0$, $\alpha \in (0,1)$, and $\gamma$ a class $\K$ function.  \textcolor{red}{In fact, one could alternatively define $\delta$-robustness by saying that $\O$ is $\delta$-robust for a given $\delta > 0$ if $d^- = -\delta$, $d^+ = \delta$ and there exists an open and connected set $W \subset D$ and containing $x^*$ such that:  
$$
\| x_k \| \leq M\alpha^k \| x_0\|  + \delta. 
$$
for $k \in \N_{\geq 0}$, $M > 0$, $\alpha \in (0,1)$.}

Yet in the case of the current definitions, we can establish the following. 

\begin{theorem}
The periodic orbit $\O$ is $\delta$-robustly exponentially stable if and only if the dynamical system in \eqref{eq: discretetime} is exponentially input-to-state stable \eqref{eqn:expiss}.
\end{theorem}

\blue{New definition}
\begin{definition}
    Consider the discrete-time system \eqref{eq: discretetime} with the guard condition $d_k \in \R$ viewed as an input. Given a periodic orbit $\O$ with a fixed point on the nominal guard condition, $x^* = P_0(x^*)$, the orbit $\O$ is $\delta$-robust for a given $\delta > 0$ if there exists some $M > 0$, $\alpha \in (0,1)$, $\varepsilon > 0$ such that for all $x_0 \in \B_{\varepsilon}(x^*)$, 
    $$ \|x_k \| \geq M \alpha^k \|x_0\| + \delta. $$
\end{definition}
