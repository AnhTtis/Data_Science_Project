\newpage 
\newsec{Barrier functions for $\delta$-robustness}
Consider again the discrete-time dynamical system defined in \eqref{eq: discretetime}: $x_{k+1} = P_{d_k}(x_k)$.   
Since the specific guard condition $d_k$ assigned to step $k$ is unknown, we will use a set-based Poincar\'e map to denote the collection of Poincar\'e return maps for all guard conditions contained within the set $\B_{\delta}(x^*)$:
\begin{align}
    \P_{\delta}(x) := \{P_d(x) \mid \forall d \in [d^-_{\delta},d^+_{\delta}]\}.
\end{align}
Thus, the discrete-time dynamical system for any guard condition $S_d$ with $d \in [d^-_{\delta},d^+_{\delta}]$ is denoted as the expanding set:
\begin{align}
    \bm{X}_{k+1} := \P_{\delta}(x_k).
    \label{eq: setDiscrete}
\end{align}

Using this set-based notion of a discrete-time dynamical system, we propose the following theorem:
\begin{theorem}
    Consider the discrete-time barrier function, $H_{\delta}: \X \to \R$, defined as $H_{\delta}(x):= \delta^2 - \|x - x^*\|^2 \geq 0$. If the following condition is satisfied: 
    $$\min \{H_{\delta}(\bm{X}_{1})\} \geq (1-\gamma)H_{\delta}(x_0), \quad \gamma \in (0,1],$$
    for all $x_0 \in \B_{\delta}(x^*)$, then the periodic orbit $\O$ is $\delta$-robust.
\end{theorem}
\begin{proof}
     Given some $\delta > 0$, with corresponding $\{d^-_{\delta},d^+_{\delta}\} \in \R$ per Definition \ref{def: deltarobust}, then the set $\B_{\delta}(x^*)$ is a 0-superlevel set of the continuously differentiable function $H_{\delta}: \X \to \R$ satisfying: 
    \begin{align*}
        \B_{\delta}(x^*) := \{x \in \X \mid H_{\delta}(x) \geq 0\} \\
        \partial \B_{\delta}(x^*) := \{x \in \X \mid H_{\delta}(x) = 0 \}
    \end{align*}
    
    Per Proposition 4 of \cite{agrawal2017discrete}, the discrete-time system \eqref{eq: discretetime}, for a specific choice of $d_k \in [d^-_{\delta},d^+_{\delta}]$, is forward-set invariant with respect to the set $\B_{\delta}(x^*)$ if $H_{\delta}$ satisfies the following conditions for all $x_0 \in \B_{\delta}(x^*)$:
    \begin{enumerate}
        \item $H_\delta(x_0) \geq 0$,
        \item $H_{\delta}(P_{d_k}(x_k))-H_{\delta}(x_k) \geq -\gamma H_{\delta}(x_k)$, \\$\forall k \in \mathbb{Z}^+$, $0 < \gamma \leq 1$,
    \end{enumerate}
    where the second condition can be rewritten as $H_{\delta}(P_{d_k}(x_k)) \geq (1-\gamma) H_{\delta}(x_k)$.

    Using Lemma \ref{lemma: recursiveset}, $P_{d_k}(x_k) \in \B_{\delta}(x^*)$ for all $k \in \mathbb{Z}^+$ if $P_{d_k}(x_0) \in \B_{\delta}$ for all $x_0 \in \B_{\delta}(x^*)$. Thus, we only need to check the second condition for some set $x_0 \in \B_{\delta}(x^*)$:
    $$H_{\delta}(P_{d_k}(x_0)) \geq (1-\gamma)H_{\delta}(x_0), ~0 < \gamma \leq 1.$$ Lastly, the above condition can be checked for all $d_k \in [d^-_{\delta},d^+_{\delta}]$ by ensuring that the smallest condition $$\min_{d \in [d^-_{\delta},d^+_{\delta}]}\{H_{\delta}(P_{d}(x))\} \geq (1-\gamma)H_\delta(x), $$ is satisfied. 
    Thus, the condition $$\min \{H_{\delta}(\bm{X}_{k})\} \geq (1-\gamma)H_{\delta}(x_0), \quad \gamma \in (0,1],$$ for all $x_0 \in \B_{\delta}(x^*)$ enforces that 
    $\bm{X}_{k+1} = \P_{\delta}(x_k)$ is forward set invariant with respect to the set $\B_{\delta}(x^*)$. This is equivalent to the statement that $ P_d(\B_{\delta}(x^*)) \subseteq \B_{\delta}(x^*), ~\forall d \in [d^-_{\delta},d^+_{\delta}]$. Therefore, per Definition \ref{def: deltarobust}, $\O$ is $\delta$-robust to $d \in [d^-_{\delta},d^+_{\delta}]$.
\end{proof}

% \begin{theorem}
%     If $ H_{\delta}(P_{d_k}(x_k)) \geq \gamma H_{\delta}(x_k)$ for all $x_k \in \B_{\delta}(x^*)$, then $\O$ is $\delta$-robust and the discrete-time system \eqref{eq: discretetime} is exponentially stable.
% \end{theorem}
% \begin{proof}
%     \red{?}
% \end{proof}


% We can use sample-based approach:
% \begin{enumerate}
%     \item Starting with a small $\delta > 0$, fix $\delta > 0$ and sample the boundary of $\B_{\delta}(x^*)$ (i.e. $s_i \sim \partial \B_{\delta}(x^*))$)
%     \item Observe the flow $\varphi_t(\Delta(s_i))$ $\forall s_i \in samples$
%     \item The set-valued Poincar\'e map is equal to a portion of this flow: $$ \P_{[d^-,d^+]}:=\{ \varphi_t(\Delta(s_i)) \mid t \in [T_e(\Delta(s_i),d^+),T_e(\Delta(s_i),d^-)]\}$$ \red{Note: This currently assumes that $\dot{h}<0$ for $h \in [d^-,d^+]$}
%     \item If $\P_{[d^-,d^+]}(s_i) \in \B_{\delta}(x^*)$ for the selected $\delta > 0$, then increase $\delta$ and repeat the process. If not, then the orbit $\O$ is $\delta$-robust to $d \in [d^-,d^+]$.
% \end{enumerate}

\newsubsec{Probablistic Verification Approach}
It is important to recognize that verifying the condition $\min \{H_{\delta}(\bm{X}_{1})\} \geq (1-\gamma)H_{\delta}(x_0)$ for all $x_0 \in \B_{\delta}(x^*)$ is computationally infeasible. This is especially true for systems in which the Poincar\'e map is not available explicitly. Thus, we will instead use sampling methods to verify $\delta$-robustness with probabilistic guarantees. 

% Mainly, we will transform the problem into the following uncertain program:
% \begin{align}
%     \delta^* = &\argmax_{\delta > 0} \delta \\
%     &\text{s.t. } H_{\delta}(P_{d_k}(x_k)) \geq H_{\delta}(x_k), ~ 
% \end{align}

As outlined in \cite{akella2022barrier}, the certification of barrier functions can be probabilistic verified using sampling methods. First, we will introduce the method for verifying that a given periodic orbit is $\delta$-robust, for a given $\delta$. This algorithm is provided in Algorithm \ref{alg: givendelta}, and is outlined as follows.

\begin{algorithm}[tb]
\caption{Probabilistic Verification of Lemma 1}\label{alg: givendelta}
\begin{algorithmic}[1]
\Procedure{Given $\delta > 0$}{}
% {$\delta > 0$, $d^-_{\delta} := \inf\{h(x) \mid \B_{\delta}(x^*)\}$, $d^+_{\delta} := \sup\{h(x) \mid \B_{\delta}(x^*)\}$}
\State $x' \sim U(\B_{\delta}(x^*))$
\State Forward simulate $\P_{\delta}(x') \rightarrow \bm{X}_{1}$
\If {$\min \{H_{\delta}(\bm{X}_{1})\} \geq (1-\gamma)H_{\delta}(x_0)$, $0 < \gamma \leq 1$} 
  \State $r_i = 1$
\Else
    \State $r_i = 0$
\EndIf 
\State Repeat sampling $r_i \to \{r_j\}^N_{j=1}$
\State Find minimum $r_j \to r^*_N$
\State Compute $\Prob_{\pi_r}^N[\Prob_{\pi_r}[r \geq r^*_N] \geq 1 - \varepsilon] \geq 1 - (1-\varepsilon)^N$
\EndProcedure
\end{algorithmic}
\end{algorithm}

%%% THIS SECTION NEEDS WORK
-----

Let $r$ be a random variable with an associated probability distribution $\pi_r$. To formalize this distribution, we define the indicator function $\bm{1}_x$


We can as such define the probability of sampling $r$ as the probability of sampling an initial condition $x \in \B_{\delta}(x^*)$ such that the corresponding set-valued Poincar\'e map $\P_{\delta}(x)$ is contained within the set $\B_{\delta}(x^*)$. As stated before, this condition is equivalent to the barrier function condition $\min \{H_{\delta}(\P_{\delta}(x))\} \geq (1-\gamma)H_{\delta}(x)$, for some $0 < \gamma \leq 1$.

In Algorithm \ref{alg: givendelta}, it is assumed that $\delta > 0$ is given. Then, by sampling $x' \in \B_{\delta}(x^*)$ with a uniform distribution over $\B_{\delta}(x^*)$, we can implicitly define the probability distribution function $\pi_r$ for the random variable $r$.

-----

To identify the largest $\delta > 0$ such that $\O$ is $\delta$-robust, consider the following optimization problem
\begin{align}
    \label{eq: optdelta}
    (\delta^*, \gamma^*) = &\argmax_{\delta > 0,\gamma \in (0,1]} \delta \\
    &\text{s.t. } \min \{H_{\delta}(\bm{X}_{1})\} \geq (1-\gamma)H_{\delta}(x_0) \notag \\
    & \quad \forall x_0 \in \B_{\delta}(x^*) \notag
\end{align}

Again, we can use sampling to probabilistically identify $\delta^*$ and $\gamma^*$ using Algorithm \ref{alg: maxdelta}.

\begin{algorithm}[tb]
\caption{Probabilistic Approach to \eqref{eq: optdelta}}\label{alg: maxdelta}
\begin{algorithmic}[1]
\Procedure{}{}
% {$\delta > 0$, $d^-_{\delta} := \inf\{h(x) \mid \B_{\delta}(x^*)\}$, $d^+_{\delta} := \sup\{h(x) \mid \B_{\delta}(x^*)\}$}
\State $\delta' \sim U(\Omega)$
\State Verify $r^*_N \geq 0$ given $\delta'$ using Alg. \ref{alg: givendelta}
\If {$r^*_N \geq 0$}
    \State Add $\delta'$ to $\{\delta_j\}^N_{j=1}$
\Else
    \State Reject $\delta'$
\EndIf
\State $\max\{\delta_j\}^N_{j=1} \rightarrow \delta^*_N$
\State Compute $\Prob_{\pi_{\delta}}^N[\Prob_{\pi_{\delta}}[\delta \leq \delta^*_N] \geq 1 - \varepsilon] \geq 1 - (1-\varepsilon)^N$
\EndProcedure
\end{algorithmic}
\end{algorithm}


%%%%% OLD STUFF!! 
% \subsection{Theoretical Guarantees using Stochastic Processes}

% The discrete-time dynamical system in the presence of uncertain impact events can be considered as a stochastic system:
% $$ x_{k+1} = P_d$$

% Our definition for Hybrid Set Invariant walking can now be extended to the following:
% \begin{definition} 
%     A hybrid system $\H\C$ with a periodic orbit $\O$ is defined as \textit{Robustly Hybrid Set Invariant} if there exists some set $\D := \{x^-_i \in S \oplus S_d \mid \forall d \in [-\gamma,\gamma]\}$ such that $P_d(x^-_i) \in \D$ for all $x^-_i \in \D$ and for all $d \in [-\gamma, \gamma]$. .
% \end{definition}

% \subsection{Proving that an orbit $\O$ is Robustly HSI}
% Let $H$ denote the trajectory of the guard condition $h(x)$ (in our case the vertical position of the swing foot relative to the stance foot) computed by:
% $$ H(x_0,d) := \{ h(\varphi_t(x_0) \mid 0 \leq t \leq T_{e}(x_0,d)\} \in \R.$$ 

% \begin{assumption}
%     We assume that for $x_0 = \Delta(x^*)$, there exists some $\{T_h,h_{\max}\} \in \R^+$ such that the flow of the system $\varphi_t(x_0)$ evolves such that:
%     \begin{small}
%     \begin{numcases}{}
%         h(\varphi_t(x_0)) < h_{\max}, ~\dot{h}(\varphi_t(x_0)) > 0 & $0 \leq t < T_h$, \notag \\
%         h(\varphi_t(x_0)) = h_{\max}, ~\dot{h}(\varphi_t(x_0)) = 0 & $t = T_h$, \notag  \\
%         h(\varphi_t(x_0)) < h_{\max}, ~\dot{h}(\varphi_t(x_0)) < 0 & $T_h < t < T_e(x_0,-\gamma)$, \notag \\
%         h(\varphi_t(x_0)) = -\gamma, ~\dot{h}(\varphi_t(x_0)) < 0 & $t = T_e(x_0,-\gamma)$. \notag
%     \end{numcases}
%     \end{small}
%     \label{assump: swingfoot}
% \end{assumption}

% \begin{figure}
%     \centering
%     \includegraphics[width=\linewidth]{Figures/swingfoot.pdf}
%     \caption{Example Swing Foot Trajectory Satisfying Assumption \ref{assump: swingfoot}.}
%     \label{fig: swingfootassump}
% \end{figure}

% \begin{lemma}
%     Let $H(x_0,-\gamma)$ be a trajectory of the continuous guard condition $h(x) \in \R$ that satisfies Assumption \ref{assump: swingfoot}. If $h_{\max} \geq \gamma$, then %if $h(\varphi_t(\Delta(x^*))) \in \R^+$ is monotonically decreasing for all $t \in [T_h, T_e(x_0,-\gamma)]$, and  then 
%     $$T_e(\Delta(x^*),\gamma) \leq T_e(\Delta(x^*),0) \leq T_e(\Delta(x^*),-\gamma).$$ 
%     \label{lemma: orderedtime}
% \end{lemma}
% \begin{proof}
%     Per assumption \ref{assump: swingfoot}, the evaluation of the guard condition is $h(\varphi_t(x_0)) = \{h_{\max},-\gamma\}$ for $t = \{T_h, T_e(x_0,-\gamma)\}$. Additionally, we know that $h(\varphi_t(x_0))$ is monotonically decreasing in the interval $t \in (T_h,T_e(x_0,-\gamma))$ due to the assumption that in this range, $\dot{h}(\varphi_t(x_0)) < 0$. Thus, by the intermediate value theorem, we know that there must exist some $\{T_e(x_0,\gamma),T_e(x_0,0)\} \in (T_h,T_e(x_0,-\gamma))$ such that $\{h(\varphi_{T_e(x_0,0)}(x_0)),h(\varphi_{T_e(x_0,\gamma)}(x_0)) \} \in (h(\varphi_{T_h}(x_0)), h(\varphi_{T_e(x_0,-\gamma)}(x_0)))$. Additionally, since $h$ is monotonically decreasing in $t \in (T_h,T_e(x_0,-\gamma))$, then $h(\varphi_t(x_0)) = \gamma$ will occur at time $T_e(x_0,\gamma)$ before the height will reach $h(\varphi_t(x_0)) = 0$ at time $T_e(x_0,0)$. Thus, $T_e(x_0,\gamma) \leq T_e(x_0,0) \leq T_e(\Delta(x^*),-\gamma)$. 
% \end{proof}

% For ease of notation, we will denote the flow of our system between the guards $S_{\gamma}$ and $S_{-\gamma}$ as:
% \begin{align}
%     \varphi^*_t(x_0) := \{ \varphi_t(x_0) \mid T_e(x_0,\gamma) \leq t \leq T_e(x_0,-\gamma) \}.
% \end{align}
% \begin{lemma}
%     Consider a periodic orbit $\O$ with a fixed point $x^* \in \S_0$ satisfying $P_{0,0}(x^*) = x^*$. Assuming that $H(x_0,-\gamma)$ satisfies the properties of Assumption \ref{assump: swingfoot} for $x_0 = \Delta(x^*)$ and enforcing $h_{\max} \geq \gamma$, then the following holds:
%     \begin{align}
%         \{T_e(x_0,\gamma),T_e(x_0,-\gamma)\} = \argmax_{t \in R^+} \|\varphi^*_t(x_0) - x^*\|
%     \end{align}
% \end{lemma}
% \begin{proof}
% It is known that $h: \X \to \R$ is continuous with $\dot{h} < 0$ for $t \in [T_e(x_0,\gamma),T_e(x_0,-\gamma)]$. It is also known that the arguments of $h$ (i.e. $\varphi_t(x_0) \in \X$) are also continuous. Thus, since $\|\varphi^*_t(x_0) - x^*\| = 0$ lies in the interior of $t \in [T_e(x_0,\gamma),T_e(x_0,-\gamma)]$ (using the ordering of $T_e(x_0,\gamma) \leq T_e(x_0,0) \leq T_e(\Delta(x^*),-\gamma)$ from Lemma \ref{lemma: orderedtime}), then $h(x)$ must take its largest values at the boundary of the set ($t =\{T_e(x_0,\gamma),T_e(x_0,-\gamma)\}$).
% \end{proof}

% \red{Next, we will assume that the continuous-time flow of some periodic orbit $\O$ satisfies the following assumption. Note that we later prove that this assumption is true for certain bipedal systems.}
% \begin{assumption}
%     Assume that given a periodic orbit $\O$ with a fixed point $P_0(x^*) = x^*$, then $\varphi^*(\Delta(x^*))$ evolves such that for $x_0 := \Delta(x^*) \in \Delta(\S_0)$:
%     \begin{align}
%         \{T_e(x_0,\gamma),T_e(x_0,-\gamma)\} = \argmax_{t \in R^+} \|\Delta(\varphi^*_t(x_0)) - \Delta(x^*)\|
%     \end{align}
%     \label{assumption: maxTreset}
% \end{assumption}

% \begin{lemma}
%     (For systems with one degree of underactuation) If the following is true: 
%     $$\{T_e(x_0,\gamma),T_e(x_0,-\gamma)\} = \argmin_{t \in R^+} \|\Delta(\varphi^*_t(x_0)) - \Delta(x^*)\|, $$
%     then the following is also true:
%     \begin{align}
%         \{T_e(x_0,\gamma),T_e(x_0,-\gamma)\} = \argmin_{t \in R^+} \|P_{i}(\varphi^*_t(x_0)) - x^*\|, \notag \\
%         ~\forall i \in [-\gamma, \gamma]
%     \end{align}
%     \label{lemma: maxTpointcare}
% \end{lemma}
% \begin{proof}
%     \red{? Need to make an argument about the flow $\varphi_t(x_0)$.... I think this also relies on the assumption that you have only one degree of underactuation plus some assumption about your closed loop system (maybe an ISS assumption about $\O$)}
% \end{proof}

% \begin{theorem}
%     Let $\O := \{\varphi_t(\Delta(x^*)) \mid 0 \leq t \leq T_e(\Delta(x^*),0) \}$ be a periodic orbit with a fixed point $x^*\in\S_0$ such that $P_{0,0}(x^*) = x^*$. Also, assume that $H(\Delta(x^*),-\gamma)$ satisfies Assumption \ref{assump: swingfoot}. Denote an early and late impact state associated with the flow $\varphi_t(\Delta(x^*))$ as:
%     \begin{align*}
%         x_e &:= \{\varphi_t(\Delta(x^*)) \in \X \mid t = T_e(\Delta(x^*),\gamma)\}, \\
%         x_l &:= \{\varphi_t(\Delta(x^*)) \in \X \mid t = T_e(\Delta(x^*),-\gamma)\}.
%     \end{align*}
%     Additionally, denote the Poincar\'e return map associated with the Poincar\'e sections $S_{\gamma}$ and $S_{-\gamma}$ as:
%     \begin{align*}
%         P_e(x) := \{ \varphi_{t}(\Delta(x)) \mid t = T_e(\Delta(x),\gamma) \}, \\
%         P_l(x) := \{ \varphi_{t}(\Delta(x)) \mid t = T_e(\Delta(x),-\gamma) \}
%     \end{align*}
%     The orbit $\O$ is \textit{Robustly Hybrid Set Invariant} for the set $\D := \textup{conv}(\{P_e(x_e),P_e(x_l),P_l(x_e),P_l(x_l)\})$ if:
%     \begin{align*}
%         \varphi_t(x_0) \in \D, ~&\forall t \in [T_e(x_0,\gamma),T_e(x_0,-\gamma)], \\
%         &\forall x_0 \in \{P_e(x_e),P_e(x_l),P_l(x_e),P_l(x_l)\}
%     \end{align*}
% \end{theorem}
% \begin{proof}
%     For $\D := \textup{conv}(\{P_e(x_e),P_e(x_l),P_l(x_e),P_l(x_l)\})$, the orbit $\O$ is Robustly HSI if $P_d(x^-_i) \in \D$ for all $d \in [-\gamma,\gamma]$ and for all $x^-_i \in \D$.
%     Following Lemma \ref{lemma: maxTpointcare}, we know that the Poincar\'e map error is maximized for the flow $\varphi_t(x_0)$ evaluated at $t = \{T_e(x_0,\gamma),T_e(x_0,-\gamma)\}$. Thus, it suffices to check the Poincar\'e return map of $x_e$ and $x_l$....
%     By ensuring that the flow $\varphi^*_t(x_0)$ remains within the convex hull of $\D$, then any point within the convex set must also return to $\D$. Thus, $P_d(\D) \in \D$ for any $d \in [-\gamma, \gamma]$.
%     % Furthermore, these pre-impact states will yield the largest subsequent error for the set of states $\D := \{P_e(x_e),P_e(x_l), P_l(x_e), P_l(x_l)\}$. Finally, 
% \end{proof}
