\section{An ISS Perspective on Walking: $\delta$-Robustness}
\label{sec: uncertainguard}

This section provides the key formulation of robustness considered throughout this paper---that of $\delta$-robustness.  The core concept behind this definition is stability in and of itself is not a sufficiently rich concept to capture robustness, since it is purely local.  Thus, we define a notion of robustness leveraging the extended Poincar\'e map \new{(which extends the Poincar\'e map to consider general guard conditions)} and input-to-state stability, wherein the inputs are the disturbances associated with uncertain guard conditions. 


\newsec{Motivation}
Practically, the stability of periodic orbits can be analyzed by evaluating the eigenvalues of the Poincar\'e return map linearized around the fixed point. Specifically, if the magnitude of the eigenvalues of $DP(x^*) = \frac{\partial P}{\partial x}(x^*)$ is less than one (i.e. $\max |\lambda (DP(x^*))| < 1$), then the fixed point is stable \cite{morris2005restricted,perko2013differential}. 
% Note that in this process, the eigenvalues corresponding to states known to be nonperiodic, such as the horizontal position of the floating base frame. 
%
% Since it is often difficult to compute the Poincar\'e map analytically, it is commonly numerically approximated. Each row of the Jacobian is successively computed by applying small perturbations to each state and forward simulating one step to obtain $P(x^*+\delta )$.  
% This implicitly implies that the Poincar\'e map is robust to $\delta$ perturbations---as long as $\delta$ is sufficiently small. 
\new{While this property implies that the Poincar\'e map is robust to sufficiently small perturbations, it is often incorrectly assumed that the magnitude of the eigenvalues say something deeper about the broader robustness of the periodic orbit to perturbations.}
% Yet, this small amount of robustness inherent in stable gaits is often confounded with the eigenvalues themselves.  That is, it is sometimes assumed that the magnitude of the eigenvalues of the Poincar\'e map say something deeper about the broader robustness of the periodic orbit to perturbations.  
This is not the case, as the following example illustrates. 



% However, since the Poincar\'e map is evaluated at the fixed point, this notion of stability is only valid locally. This means that while this tool is useful for checking the stability of periodic orbits, it is no longer insightful for systems with uncertain guard events or disturbances that push the system away from the fixed point. 



\begin{example}
Consider a seven-link bipedal robot as shown in Figure \ref{fig: uncertainguard}.  To illustrate how the eigenvalues associated with the linearization fail to tell the whole story, we will consider the robustness of two gaits to differing ground height conditions.  
As illustrated in Figure \ref{fig: motivation}, the classic Poincar\'e analysis does not accurately reflect the robustness of periodic orbits to local disturbances in the guard condition. That is, the gait with the smaller maximum eigenvalue (magnitude) is more fragile to changing ground heights. 
\end{example}



\begin{figure}[tb]
    \centering
    % \includegraphics[width=\linewidth]{Figures/Example1.pdf}
    % \caption{This figure illustrates the flow of a periodic orbit for uncertain guard conditions $S_{d_k}$ with $d_k \sim U(-\delta,\delta)$ (in this example, $\delta = 1.5$cm) being uniformly sampled for $k = 500$ steps. The results demonstrate that a periodic orbit with $\max |\lambda(D P_0(x^*))| < 1$ (illustrated on the left) is not robust to variations in the guard condition (the periodic orbit diverged after only 7 cycles), while a periodic orbit with a larger $|\lambda|$ (illustrated on the right) is comparatively more robust. The periodic orbits for the nominal guard condition $S_0$ are illustrated in blue.\red{Need to mention that it is plotted for the zero dynamics..?}}
    % \label{fig: motivation}
    \includegraphics[width=\linewidth]{Figures/phase_portraits2.pdf}
    \vspace{-0.6cm}
    \caption{The phase portraits at the top of the figure illustrate the walking for uncertain guard conditions $S_{d_k}$ with $d_k \sim U(-\delta,\delta)$ (in this example, $\delta = 1.5$cm) for $k = 500$ steps. Visualizations of the walking gaits for three step conditions are provided at the bottom. The results demonstrate that a periodic orbit with $\max |\lambda(D P_0(x^*))| < 1$ (on the left) is not robust to variations in the guard condition (the orbit diverged after only 13 steps), while a periodic orbit with a larger $|\lambda|$ (on the right) is comparatively more robust. This motivates the need for an ISS perspective.}
    \label{fig: motivation}
    \vspace{-0.4cm}
\end{figure}

\newsec{Uncertain Guard Conditions}
\new{To formulate a notion of robustness, uncertain guard conditions are considered---this, for example, captures uncertain ground height for walking robots. Specifically, as done in \cite{hamed2016exponentially}, the Poincar\'e map can be extended to explicitly consider changes to the guard condition (i.e., $h(x) = d$). First, define a general guard as:}
\begin{align}
    S_d &= \{x\in \mathcal{X} \mid h(x) = d, ~\dot h(x) < 0\},
    \label{eq: heightguard}
\end{align}
with $d \in \mathbb{D}$ and $\mathbb{D} := [d^-,d^+] \subset \R$ for some $d^- < 0 < d^+$. Using this general guard definition, the previous guard \eqref{eq: zeroguard} is now denoted as $S_0$.  Under the assumption that $S_d \subset D$ for all $d \in \mathbb{D}$, we have a corresponding hybrid system: 
\begin{numcases}{\mathcal{H}_d  = }
\dot{x} = f_{\rm cl}(x) := f(x) + g(x) k(x) & $x \in D \setminus S_d$, \label{eq: continuousd}
\\
x^+ = \Delta(x^-) & $x^- \in S_d$, \label{eq: discreted}
\end{numcases}
% Specifically, we consider the general guard (as defined in \cite{hamed2016exponentially}):% for some $d \in \R$,

% in which $d \in [-d_{\max},d_{\max}]$ denotes a closed neighborhood of the origin for some $d_{\max} \in \R_{>0}$. 

\new{Next, we must modify the time-to-impact function to be defined on a neighborhood of the fixed point $x^*$. In particular, the time-to-impact function exists as a result of the implicit function theorem \cite{lee2010introduction} applied to the implicit function (of time) $ h(\varphi_t(\Delta(x)))$ which therefore satisfies: 
$h(\varphi_T(\Delta(x^*))) = 0$, and $\dot{h}(\varphi_T(\Delta(x^*))) < 0$,
for $x^* \in \O \cap S$. Thus, there exists an explicit function $T_e: B_{\rho}(x^*) \subset D \to \R$, for some $\rho > 0$\footnote{We assume throughout the paper that for all $\rho > 0$ of interest, the domain $D$ of the continuous dynamics is appropriately chosen so that $B_{\rho}(x^*) \subset D$.}, termed the \emph{extended time-to-impact function} satisfying:
\begin{eqnarray}
\label{eqn:extendedtimetoimpact}
h(\varphi_{T_e(x)}(\Delta(x))) = 0, \qquad \forall ~ x \in B_{\rho}(x^*).
\end{eqnarray}
It follows that $T_I$ in \eqref{eqn:timetoimpact} is just $T_I = T_e|_{S}$, wherein the Poincar\'e map is given by considering only $x \in B_{\rho}(x^*) \cap S$.}  
% Consider the extended time-to-impact function $T_e : B_{\rho}(x^*) \to \R$ defined implicitly in \eqref{eqn:extendedtimetoimpact}.  
This function can be further extended (as a partial function) to account for varying guards: $T_e : B_{\rho}(x^*) \times \mathbb{D} \partialto \R$:
\begin{align}
    T_e(x_0,d) := \inf\{ t \geq 0 \mid \varphi_t(\Delta(x_0)) \in S_d \}.
\end{align}
Importantly, this is a partial function because (by the implicit function theorem) it is only well-defined for $d = 0$ and by continuity sufficiently small $d^-$ and $d^+$.  
Using this extended time-to-impact function, we can define the \textit{extended Poincar\'e map} as a partial function: $P_d: B_{\rho}(x^*) \partialto S_d$:
\begin{align}
\label{eqn:Pd}
    P_{d}(x^-) := \varphi_{T_e(x^-,d)}(\Delta(x^-)).
\end{align}
This allows us to frame walking with uncertain guards as a discrete-time control system.

% Yet, this restriction is not necessary---which leads to the notion of the extending the Poincar\'e map so its domain of definition is $B_{\rho}(x^*)$ (see \cite{perko2013differential}).

% In particular, define the \emph{extended Poincar\'e map}: 
% \begin{eqnarray}
% \label{eqn:extendedpoincare}
% P_0 : B_{\rho}(x^*) \subset D & \to & S \\
% x & \mapsto & \varphi_{T_e(x)}(\Delta(x)).  \nonumber
% \end{eqnarray}


% The importance of extending the domain of definition of the Poincar\'e map will be seen in the  context of set theoretic notions of robust walking. 

% To formulate a notion of robustness of the extended Poincar\'e map, uncertain guard conditions are considered---this, for example, captures uncertain ground height for walking robots. 



% It is important to note that the typical assumptions made for systems considering $\S_0$ still apply for systems considering $\S_d$ with $d \not= 0$. Mainly,
% \begin{enumerate}
%     \item The continuous-time dynamical system \eqref{eq: continuous} is continuous on $\X$ for any guard condition $S_{d}$
%     \item The solution to \eqref{eq: continuous} from any initial condition $x_0 \in \Delta(S_d)$ is unique and depends continuously on $x_0$ for any guard condition $S_{d}$
%     \item The function $h: \X \to \R$ is differentiable such that for every $s \in S_d$, $\frac{\partial h}{\partial s}(x) \not= 0$ for any $d \in \mathbb{D}$
% \end{enumerate}


%, but first we note the following:





%\newsec{Dynamical System Representation}
\newsec{Connections with Input-to-State Stability}
It is important to note that we can view \eqref{eqn:Pd} as a dynamical system evolving with an ``input'' given by the guard height: $d = h(x)$. 
%
In particular, this leads to the discrete-time dynamical system: 
\begin{align}
    \label{eqn:discretetimeP}
    x_{k+1} = \P(x_k,d_k) := P_{d_k}(x_k), 
\end{align}
for some sequence of $d_k \in [d^-,d^+] \subset \R$, $k \in \N_{\geq 0}$, determining the guard height specific to step $k \in \N_{\geq 0}$ such that $x_{k+1} \in S_{d_k}$.  The result is a partial function:
$$
\P : B_{\rho}(x^*) \times  [d^-,d^+] \partialto S_{[d^-,d^+]} : = \bigcup_{d \in [d^-,d^+]} S_d,
$$
wherein we assume that $B_{\rho}(x^*) \subset S_{[d^-,d^+]}$ (or a smaller $\rho$ is chosen so that this holds). The partial function nature of $\P$ implies that solutions may not exist for all time, i.e., the solution $x_k$ might leave the ball $B_{\rho}(x^*)$ on which $\P$ is well-defined.

% \begin{remark}
% The partial function nature of $\P$ implies that solutions may not exist for all time (the solution $x_k$ might leave the ball $B_{\rho}(x^*)$ on which $\P$ is well-defined.   In all definitions where we consider $x_k$ with $k \in \N_{\geq 0}$ we implicitly assume that $\P$ is well defined for this solution, i.e., that this solution is forward complete.  
% \end{remark}


Given the discrete-time system \eqref{eqn:discretetimeP}, and the fact that we view the input $d$ as a disturbance, there are obvious connections with input-to-state stability \cite{jiang2001input}.  In our setting, the discrete-time system $x_{k+1} = \P(x_k,d_k)$ (with $d_k$ viewed as an input) is \emph{input-to-state stable (ISS)} if: 
\begin{eqnarray}
\label{eqn:expiss}
\| x_k - x^* \| \leq   \beta(\| x_0 - x^*\|,k)  + \gamma( \| d \|_{\infty}  )
\end{eqnarray}
for $k \in \N_{\geq 0}$, $\beta$ a class $\KL$ function, and $\gamma$ a class $\K$ function.  Note that here $\| d \|_{\infty}  = \max\{-d^-,d^+\}  $ since $d : \N_{\geq 0} \to [d^-,d^+]$ is scalar valued and takes values in an interval.  Also note that, in the context of locomotion, we are especially interested in exponential stability.  To certify exponential ISS, the class $\KL$ function becomes: $\beta(r,k) = M\alpha^k r$ for $M > 0$ and $\alpha \in (0,1)$.  The end result is the exponential ISS (E-ISS) condition: 
\begin{eqnarray}
\label{eqn:expiss}
\| x_k - x^* \| \leq M\alpha^k \| x_0 - x^*\|  + \gamma( \max\{-d^-,d^+\} )
\end{eqnarray}
This allows us to formulate a notion of robustness. 

\newsec{$\bm{\delta}$-Robustness}  We now have the necessary components to present the key concept of this paper: $\delta$-robustness.  The goal in formulating this notion of robustness is to find a single scalar constant, $\delta \geq 0 $, that characterizes the robustness of a periodic orbit $\O$ in the context of uncertain guard height.  In this context, we wish to leverage \eqref{eqn:expiss}---yet the class $\K$ function $\gamma$ gives a degree of freedom that is undesirable in designing a metric for robustness.  This observation leads to: 


% \begin{definition}
% \textcolor{red}{The periodic orbit $\O$ is \textbf{$\delta$-robust} for a given $\delta > 0$ if for the discrete time dynamical system in \eqref{eq: discrete-system} with $d^- = -\delta$ and $d^+ = \delta$: 
% \begin{eqnarray}
% \label{eqn:Pdelta}
% x_{k+1} = \P(x_k,d_k), \qquad  d_k \in [-\delta,\delta],
% \end{eqnarray}
% there exists an $\rho > 0$ such that 
% $B_{\rho}(x^*) \subset S_{[-\delta,\delta]}$ is forward invariant and for $x_0 \in B_{\rho}(x^*)$: 
% \begin{eqnarray}
% \label{eqn:deltarobustness}
% \| x_k - x^* \| \leq M\alpha^k \| x_0 - x^* \|  + \gamma \delta, \qquad \forall k \in \N_{\geq 0}, 
% \end{eqnarray}
% for some $\gamma > 0$, $M > 0$, and $\alpha \in (0,1)$.}

% The periodic orbit is \textbf{robust} if it is $\delta$-robust for some $\delta > 0$, and the largest scalar value $\overline{\delta}$ such that $\O$ is $\overline{\delta}$-robust is the \textbf{robustness} of $\O$. 
% % The scalar value $\delta$ is the \textbf{robustness} of the periodic orbit $\O$. 
% \end{definition}

% \begin{definition}
% \textcolor{red}{The periodic orbit $\O$ is \textbf{$\delta$-robust} for a given $\delta > 0$ if for the discrete time dynamical system in \eqref{eq: discrete-system} with $d^- = -\delta$ and $d^+ = \delta$: 
% \begin{eqnarray}
% \label{eqn:Pdelta}
% x_{k+1} = \P(x_k,d_k), \qquad  d_k \in [-\delta,\delta],
% \end{eqnarray}
% there exists a forward invariant set $W$ containing $x^*$ such that 
% $W \subset S_{[-\delta,\delta]}$, and for $x_0 \in W$: 
% \begin{eqnarray}
% \label{eqn:deltarobustness}
% \| x_k - x^* \| \leq M\alpha^k \| x_0 - x^* \|  + \gamma \delta, \qquad \forall k \in \N_{\geq 0}, 
% \end{eqnarray}
% for some $\gamma > 0$, $M > 0$, and $\alpha \in (0,1)$.}

% The periodic orbit is \textbf{robust} if it is $\delta$-robust for some $\delta > 0$, and the largest scalar value $\overline{\delta}$ such that $\O$ is $\overline{\delta}$-robust is the \textbf{robustness} of $\O$. 
% % The scalar value $\delta$ is the \textbf{robustness} of the periodic orbit $\O$. 
% \end{definition}

\begin{definition}
The periodic orbit $\O$ is \textbf{$\bm{\delta}$-robust} for a given $\delta > 0$ if for the discrete-time dynamical system in \eqref{eqn:discretetimeP} with $d^- = -\delta$ and $d^+ = \delta$, that is: 
\begin{align}
 \P : & B_{\rho}(x^*) \times [-\delta,\delta] \to S_{[-\delta,\delta]} \nonumber\\
\label{eqn:Pdelta}
& x_{k+1} = \P(x_k,d_k), \qquad   d_k \in [-\delta,\delta],
\end{align}
there exists a forward invariant set $W \subset B_{\rho}(x^*)$ and for all $x_0 \in W$: 
\begin{eqnarray}
\label{eqn:deltarobustness}
\| x_k - x^* \| \leq M\alpha^k \| x_0 - x^* \|  + \gamma \delta, \qquad \forall k \in \N_{\geq 0}, 
\end{eqnarray}
for some $\gamma > 0$, $M > 0$, and $\alpha \in (0,1)$.
%
The periodic orbit is \textbf{robust} if it is $\delta$-robust for some $\delta > 0$, and the largest scalar $\overline{\delta}$ such that $\O$ is $\overline{\delta}$-robust is the \textbf{robustness} of $\O$. 
% The scalar value $\delta$ is the \textbf{robustness} of the periodic orbit $\O$. 
\end{definition}

This seemingly simple definition encodes a surprising amount of information.  First, the forward invariance of $W \subset B_{\rho}(x^*)$ implies that $\P : B_{\rho}(x^*) \times [-\delta,\delta] \to S_{[-\delta,\delta]}$ is a function (rather than a partial function) when restricted to the set $W$. Additionally, the actual $\delta$-robustness condition \eqref{eqn:deltarobustness} is an ISS condition, albeit slightly stronger to remove the dependence on the class $\K$ function and replace this with the constant $\gamma$. Even so, the connections with ISS are important since the associated machinery can be leveraged. 


To provide an example of how ISS can inform our thinking on $\delta$-robustness, consider the case when $\O$ is exponentially stable, i.e., $x_{k+1} = \P(x_k,0)$ has an exponentially stable fixed point: $x^* = P_0(x^*)$, i.e., the 0-input system is exponentially stable.  There are no guarantees that $\O$ is thus $\delta$-robust (see \cite{jiang2001input} where a counter example shows that \new{given arbitrarily bounded disturbances, then local asymptotic stability is not enough to guarantee ISS}).  That is, stability does not imply robustness.  
%Conversely, robustness does imply stability since if $\O$ is $\delta$-robust for a given $\delta$, it is $\delta$-robust for all smaller $\delta$ (if $\delta' < \delta$, take $\gamma' = \gamma \frac{\delta'}{\delta}$ wherein $\O$ is $\delta'$-robust).  Therefore, $\delta = 0$ yields exponential stability as in Theorem \ref{thm:conttoPexpstability}. 


\begin{example}
Returning to the example of the seven-link walker, we can heuristically calculate the $\delta$-robustness associated with the two gaits. 
% the more robust gait, while showing that in fact that fragile gait is not robust.  
Specifically, Fig. \ref{fig: exampleissbound} illustrates the ISS-perspective of $\delta$-robustness for the orbits first illustrated in Fig. \ref{fig: motivation}. As shown, the orbit that was robust in Fig. \ref{fig: motivation} satisfies the condition that $W \subset B_{\rho}(x^*)$ is forward invariant ($\delta = 1.5$cm in this example), and $\|x_k - x^*\|$ remains bounded for $\gamma = 36.8$. Comparatively, the orbit that was not robust in Fig. \ref{fig: motivation} experienced a pre-impact state that was outside of $B_{\rho}(x^*)$ and therefore $W$ was not forward invariant.
\end{example}

\begin{figure}[tb]
    \centering
    \includegraphics[width=\linewidth]{Figures/Example2FF.pdf}
    \vspace{-0.6cm}
    \caption{
    % This figure illustrates the definition of $\delta$-robustness. 
    On the left, the non-robust periodic orbit (as illustrated on the left of Fig. \ref{fig: motivation}) does not satisfy the conditions for $\delta$-robustness for $\delta = 0.015$ (specifically, there does not exist a forward invariant set $W$). In comparison, the robust orbit (as illustrated on the right of Fig. \ref{fig: motivation}) satisfies the definition of $\delta$-robustness with $\gamma = 36.8$ and $\delta = 0.015$m.}
    \label{fig: exampleissbound}
    \vspace{-0.4cm}
\end{figure}

