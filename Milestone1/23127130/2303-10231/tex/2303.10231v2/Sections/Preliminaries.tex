\section{Preliminaries}
\label{sec: preliminaries}
Walking naturally lends itself to be modeled as a hybrid system because of the presence of both continuous dynamics (during the swing phase) and discrete dynamics (at swing foot impacts) \cite{westervelt2018feedback}. Additionally, the dynamics of walking can be separated into those that can be controlled using actuation, and those that are uncontrollable -- termed the zero dynamics. 
% Previous work has developed CLFs to certify stability of these hybrid systems in the presence of input uncertainty (called input to state stability or ISS) 
% and phase uncertainty (called phase to state stability or PSS). 
% In this section, we will briefly present hybrid systems, CLFs and ISS.

\newsec{Hybrid Systems}
Consider a hybrid control system with states $x \in \X \subset \R^n$ and a control input $u \in \U \subset \R^m$. Given a continuously differentiable function\new{\footnote{\new{Note that $h$ must be selected such that it does not lie within the null space of the actuation matrix, i.e., $L_gh(x)\not=0$}}} $h:\mathcal{X}\to\R$, 
% with Lie derivative $L_gh(x)\not=0$
let $D \subset \X$ denote the admissible domain on which the continuous-time dynamics evolve and $S \subset D$ denote the \textit{guard} (also commonly called the \textit{switching surface}), defined as:
\begin{align}
    D &= \{x\in\mathcal{X} \mid h(x) \ge 0\}, \\
    S &= \{x\in \mathcal{X} \mid h(x) = 0,~\dot h(x) < 0\}.
    \label{eq: zeroguard}
\end{align}
For states $x^-\in S$, a discrete impact map $\Delta:S \to D$, termed the \textit{reset map} is applied. Thus, the complete hybrid system can be modeled as:
\begin{numcases}{\mathcal{H}\mathcal{C}  = }
\dot{x} = f(x) + g(x) u & $x \in D \setminus S$, \label{eq: continuouscontrol}
\\
x^+ = \Delta(x^-) & $x^- \in S$, \label{eq: discretecontrol}
\end{numcases}
where \eqref{eq: continuouscontrol} and \eqref{eq: discretecontrol} denote the continuous-time and discrete-time dynamics respectively.  It is assumed (as is typical) that all quantities in $\mathcal{H}\mathcal{C}$ are locally Lipshitz continuous, e.g., the impact map $\Delta$ is locally Lipschitz.
This follows from the assumption of perfectly plastic impacts \cite{glocker1992dynamical}. 
    Importantly, note that for impact maps based on rigid-body contacts \cite{hurmuzlu1994rigid}, the impact map does not depend on the ground height. 

% with Lipschitz constant $L_{\Delta} > 0$.  


% \begin{remark}
% \label{assump: impactmap}
% Note that the typical reset map utilized for impact events assuming a perfectly plastic impact (see Equation (27) of \cite{glocker1992dynamical}) is continuously differentiable. 
%     Also, note that for impact maps based on rigid-body contacts \cite{hurmuzlu1994rigid}, the impact map does not depend explicitly on the ground height. 
% \end{remark}


Given a locally Lipschitz feedback controller $u = k(x)$, the result of applying this to the hybrid control system results in a hybrid system: 
\begin{numcases}{\mathcal{H}  = }
\dot{x} = f_{\rm cl}(x) := f(x) + g(x) k(x) & $x \in D \setminus S$, \label{eq: continuous}
\\
x^+ = \Delta(x^-) & $x^- \in S$, \label{eq: discrete}
\end{numcases}
The local Lipschitz continuity of the continuous dynamics \eqref{eq: continuous} implies that solutions exist and are unique locally.  
We will use the flow notation for these solutions, $\varphi_t(x_0)$, which is the solution to the continuous dynamics at time $t\in \R_{\geq 0}$ with initial condition $x_0 \in D$. 
Under the assumption of non-Zenoness, the flow of the hybrid system is given by: 
$$
\varphi_t(x_0) = \varphi_{t-\tau_k}(x_k^+) , \qquad t \in [\tau_k,\tau_{k+1})
$$
where $\tau_k$ are the ``impact'' times and $x_k^+$ the post-impact states, determined by the consistency conditions:
\begin{eqnarray}
\label{eqn:consistency}
x_k^+ = \Delta(x^-_k), \qquad x_k^- = \varphi_{\tau_k-\tau_{k-1}}(x^+_{k-1}) \in S,
\end{eqnarray}
for $k \geq 1$, with $\tau_0 = 0$ and $x_0 \in D$ the initial condition.  When $x_0 \in S$ one trivially takes $x^-_1 = x_0$ and $\tau_1 = \tau_0$.  



\newsec{Periodicity of Hybrid Systems}
The flow $\varphi_t(x_0)$ of \eqref{eq: continuous} is periodic with period $T \in \R_{\geq 0}$ if there exists a point $x^* \in S$ satisfying $\varphi_T(\Delta(x^*)) = x^*$. The periodic orbit associated with this periodic flow is denoted:
\begin{align}
    \O := \{\varphi_t(\Delta(x^*)) \in D \mid 0 \leq t \leq T_I(x^*) = T\},
\end{align}
with $T_I: \widetilde{S} \to \R$ being the time-to-impact function:
\begin{align}
\label{eqn:timetoimpact}
    T_I(x) = \inf\{t \geq 0 \mid \varphi_t(\Delta(x)) \in S\}.
\end{align}
As proven in Lemma 3 of \cite{grizzle2001asymptotically}, the time-to-impact function is continuous at points $x \in \widetilde{S}$ satisfying the conditions $\widetilde{S} := \{ x \in S \mid 0 < T_I(x) < \infty\}$. Thus, $T_I$ is well-defined for $\widetilde{S}$.  The periodic orbit, $\O$, is exponentially stable if it is exponentially stable as a set: for $x_0 \in D$:
$$
\| \varphi_t(x_0) \|_{\O} \leq 
M e^{-\alpha t} \| x_0 \|_{\O}
$$
where $\| x \|_{\O}= \inf_{y \in \O} \| x - y \|$ is the set distance. 

% A fixed point $x^*$ is exponentially stable if:
% \begin{align}
%     \| \varphi_t(x_0) - x^* \|_2 \leq M \exp^{-\lambda t} \| x_0 - x^* \|_2,
% \end{align}
% for some $M,\lambda > 0$. 

The exponential stability of this periodic orbit $\O$ can be analyzed via the Poincar\'e map.  In particular, $S$ is a Poincar\'e section (and well-defined as such due to the assumption that $\dot{h}(x) < 0$), and associated with this Poincar\'e section is the Poincar\'e map $P: \widetilde{S} \to S$ defined as:
\begin{align}
    P(x^-) := \varphi_{T_I(x^-)}\left(\Delta(x^-)\right).
    \label{eq: poincare}
\end{align}
% Since $\Delta$ is continuous, the Poincar\'e map is well-defined in $\widetilde{S}:= \Delta^{-1}(\tilde{\X})$, which is an open subset of $S$.
The Poincar\'e map describes the evolution of the hybrid system as a discrete-time system:
\begin{align}
    x^-_{k+1} = P(x^-_k), ~k=0,1,\dots,
    \label{eq: discrete-system}
\end{align}
wherein $x_k^-$ is just given as in \eqref{eqn:consistency}. 
In \cite{morris2005restricted} (see also \cite{nersesov2002generalization}, Theorem 2.1), it was proven that a periodic orbit $\O$ is exponentially stable if and only if $x^* \in \O \cap S$ is an exponentially stable fixed point of the discrete-time system \eqref{eq: discrete-system}. This is summarized in the following: 


% In other words, we can directly relate the stability of the periodic orbit $\O$ with the stability of the Poincar\'e map $P$. This is summarized by the following definition:


\begin{theorem}[\cite{morris2005restricted}]
\label{thm:conttoPexpstability}
A periodic orbit $\O$ is exponentially stable if and only if for the corresponding fixed point $P(x^*) = x^* \in S$, there exist $M > 0$, $\alpha \in (0,1)$, and some $\delta > 0$ such that:
%such that $\forall x$ satisfying $\|x - x^*\| \leq \delta$, for the fixed point $P(x^*) = x^*$, then
\begin{align}
  \forall ~ x \in B_{\delta}(x^*) \cap \widetilde{S} &\quad \implies \quad  \nonumber\\
 & \| P^i(x) - P(x^*)\| \leq M\alpha^i \|x - x^*\|, \nonumber
\end{align}
with $P^i(x)$ denoting the Poincar\'e map applied $i \in \N_{\geq 0}  = \{0,1,\dots,n,\dots\}$ times.
\label{def: expstability}
\end{theorem}

% \newsec{The Extended Poincar\'e Map} \new{As done in \cite{hamed2016exponentially}, the Poincar\'e map can be extended to explicitly consider changes to the guard condition (i.e., $h(x) = d$, in which $d \in [-d_{\max},d_{\max}]$ denotes a closed neighborhood of the origin for some $d_{\max} \in \R_{>0}$. This extension requires modifying the time-to-impact function to be defined on a neighborhood of the fixed point $x^*$.}
% % such that it is defined on a neighborhood of the fixed point $x^*$.  
% In particular, the time-to-impact function exists as a result of the implicit function theorem \cite{lee2010introduction} applied to the implicit function (of time) $ h(\varphi_t(\Delta(x)))$ which therefore satisfies: 
% $h(\varphi_T(\Delta(x^*))) = 0$, and $\dot{h}(\varphi_T(\Delta(x^*))) < 0$,
% for $x^* \in \O \cap S$. Thus, there exists an explicit function $T_e: B_{\rho}(x^*) \subset D \to \R$, for some $\rho > 0$\footnote{We assume throughout the paper that for all $\rho > 0$ of interest, the domain $D$ of the continuous dynamics is appropriately chosen so that $B_{\rho}(x^*) \subset D$.}, termed the \emph{extended time-to-impact function} satisfying:
% \begin{eqnarray}
% \label{eqn:extendedtimetoimpact}
% h(\varphi_{T_e(x)}(\Delta(x))) = 0, \qquad \forall ~ x \in B_{\rho}(x^*).
% \end{eqnarray}
% It follows that $T_I$ in \eqref{eqn:timetoimpact} is just $T_I = T_e|_{S}$, wherein the Poincar\'e map is given by considering only $x \in B_{\rho}(x^*) \cap S$.  Yet, this restriction is not necessary---which leads to the notion of the extending the Poincar\'e map so its domain of definition is $B_{\rho}(x^*)$ (see \cite{perko2013differential}).  In particular, define the \emph{extended Poincar\'e map}: 
% \begin{eqnarray}
% \label{eqn:extendedpoincare}
% P_0 : B_{\rho}(x^*) \subset D & \to & S \\
% x & \mapsto & \varphi_{T_e(x)}(\Delta(x)).  \nonumber
% \end{eqnarray}
% The importance of extending the domain of definition of the Poincar\'e map will be seen in the  context of set theoretic notions of robust walking. 


\begin{comment}
Specifically, The proof for hybrid periodic orbits is provided in \cite{morris2009hybrid}. 
This theorem was proved using sets. We will rewrite this proof using our notation here:
\begin{proof}
    To show necessity, let $\varepsilon > 0$ and note that the set $S_{x^*} = \{x \in \D \mid x = \varphi_{T_I(x^*)}(x^*) = x^-_l, l = 1,\dots,N\}$ contains $N$ points, where $x^* := x^-_1$. Furthermore, let $x^{*+} = \lim_{\tau \to 0} \varphi_{\tau}(x^*)$ and let $\hat{\varepsilon} > 0$. It follows from joint continuity of solutions of $\HC$ that there exists $\hat{\delta} = \hat{\delta}(x^*,\hat{\varepsilon})$ such that if $\|x_0' - x^{*+}\| + |t-t'| < \hat{\delta}$, then $\|\varphi_{t}(x_0') - \varphi_{t'}(x^{*+}\| < \hat{\varepsilon}$, where $t \in (0,\tau_1(x^'_0)]$ and $t' \in (0,\tau_1(p^{*+})]$. Next, it .....
\end{proof}
\end{comment}

% Specifically, the Poincar\'e map is (locally) exponentially stable at the fixed point $x^*$ if and only if the hybrid periodic orbit $\O$ is exponentially stable. 
% \red{The stability of the periodic orbit $\O_{\Z}$ in the hybrid zero dynamics is determined by the restricted Poincar\'e map}

