\section{Lyapunov Conditions for $\delta$-robustness}
\label{sec:main}

In this section we present the main theoretic result: Lyapunov conditions for the $\delta$-robustness of periodic orbits.  These conditions, and constructions, follow naturally from the ISS perspective employed in defining $\delta$-robustness.  But care is needed given the complexity of the Poincar\'{e} map.  Importantly, these conditions will lead to an approach for the verification of $\delta$-robustness, as presented in the next section. 


\begin{definition}
Consider the discrete-time  dynamical system in \eqref{eqn:Pdelta}.  A function $V : B_{\rho}(x^*) \to \R_{\geq 0}$, for $B_{\rho}(x^*)$ as in \eqref{eqn:extendedtimetoimpact}, is a \textbf{robust Lyapunov function} if:  
\begin{align}
\label{eqn:lyap1}
       k_1 & \| x - x^* \|^c   \leq  V(x)   \leq k_2 \|   x - x^* \|^c  \\
\label{eqn:lyapimplication}
   \| x  - x^* \|  \geq  \chi d & \quad \implies \quad \\
  \Delta V(x,d) &  :=  V(\P(x,d)) - V(x)  \leq - k_3 \| x - x^* \|^c \nonumber
\end{align}
for $\chi, k_1,k_2,k_3, c > 0$ and all $x \in B_{\rho}(x^*)$.  
\end{definition}

\begin{remark}
Note that \eqref{eqn:lyapimplication} can be equivalently restated as: 
\begin{eqnarray}
   V(\P(x,d)) - V(x)    \leq - k_4 \|  x - x^*  \|^c  + \frac{1}{2} \sigma |d|^c,
  \label{eqn:lyap2}
   \end{eqnarray}
 where $\sigma > 0$.  In particular, the corresponding quantities are related via: $k_3 = \frac{1}{2} k_4$ and $\chi = k_4^{-\frac{1}{c}} \sigma^{\frac{1}{c}}$. 
\end{remark}




\newsec{Main result}  We can now state the main result of the paper.  To do so, recall that a 
\emph{Lyapunov sublevel set} is given by: 
\begin{eqnarray}
\Omega_{r} = \{ x \in \R^n ~ | ~
V(x) \leq r\}. 
\end{eqnarray}
This will be essential in establishing: 


\begin{theorem}
\label{thm:main}
Consider the discrete-time dynamical system $x_{k+1} = \P(x_k,d_k)$ in \eqref{eqn:Pdelta} with associated periodic orbit $\O$.  If there exists a robust Lyapunov function, $V : B_{\rho}(x^*) \to \R_{\geq 0}$, and:
\begin{eqnarray}
\label{eqn:deltabound}
\delta < \delta_{\max} := \left(\frac{k_1}{\chi^c k_2}\right)^{\frac{1}{c}} \rho,
\end{eqnarray}
then the periodic orbit $\O$ is $\delta$-robust with: 
\begin{align}
 & \qquad   W =   \Omega_{r(\delta)},  \quad \mathrm{for} \quad r(\delta) := k_2 (\chi \delta) ^c  \\
\gamma = &\left( \frac{k_2}{k_1} \right)^{\frac{1}{c}} \chi,   \quad 
M= \left(\frac{k_2}{k_1}\right)^{\frac{1}{c}}, \quad 
\alpha = \left( 1 - \frac{k_3}{k_2}\right)^{\frac{1}{c}}. \nonumber
\end{align}
\end{theorem}


This theorem is, overall, a variation on Lemma 3.5 in \cite{jiang2001input}.  The proof here follows a similar overall arc, although there are key differences made necessary by the fact that $\P$ is only a partial function. 
This motivates the first Lemma. 

\begin{lemma}
The function $\P: B_{\rho}(x^*) \times [-\delta,\delta] \to S_{[-\delta,\delta]}$ given in \eqref{eqn:Pdelta} is well-defined for all $x \in  B_{\rho}(x^*) $, i.e., for all $x \in B_{\rho}(x^*) $, $\P(x,d)$ exists and satisfies $\P(x,d) \in S_{[-\delta,\delta]}$.
\end{lemma}

\begin{proof}
By the construction of the extended Poincar\'e map, $P_0$ is well-defined on $B_{\rho}(x^*)$, i.e., for all $x \in B_{\rho}(x^*)$ it follows that $\P(x,0) \in S_0$, i.e., $h(\varphi_{T_e(x,0)}(\Delta(x))) = 0$.  Therefore:
$$
h(\varphi_{t}(\Delta(x))) = \int_{T_e(x,0)}^t 
\dot{h}(\varphi_{\tau}(\Delta(x))) d \tau.
$$
But $\dot{h}(x) < 0$ for all $x \in  S_{[-\delta,\delta]}$ by definition.  Therefore, on the closed set defined by $- \delta \leq h(x) \leq \delta$, $\dot{h}$ takes a minimum and maximum value: $\underline{h} < \overline{h} < 0$.  This implies that: 
$$
\underline{h} (t - T_e(x,0))  \leq h(\varphi_{t}(\Delta(x))) \leq \overline{h}  (t - T_e(x,0)).
$$
Thus, there exists a $t$ (possibly negative) such that $h(\varphi_{t}(\Delta(x))) = d$. This $t = T_e(x,d)$. 
\end{proof}

Since $\P$ is well-defined, we can now find a set such that $x_{k+1} = \P(x_k,d_k)$ is defined for all $k$, i.e., a forward invariant set contained in $B_{\rho}(x^*)$, using Lyapunov sublevel sets. 


\begin{lemma}
\label{lem:levelset}
If $\delta < \delta_{\max}$, with $\delta_{\max}$ in \eqref{eqn:deltabound}, 
then for $r(\delta) := k_2 (\chi \delta) ^c$ it follows that:
$$
B_{\chi \delta}(x^*) \subset \Omega_{r(\delta)} \subset B_{\rho}(x^*).
$$
Moreover, the set $\Omega_{r(\delta)}$ is forward invariant. 
\end{lemma}

% Let $r(\rho) > 0$ such that $r(\rho) < k_1 \rho^c$ which yields:
% $$
% V(x) \leq r(\rho) \quad \Rightarrow \quad 
% k_1 \| x - x^* \|^c \leq V(x)  \leq r(\rho) <  k_1 \rho^c 
% % \quad \implies \quad 
% % \| x - x^* \| < \rho
% % \quad \Rightarrow \quad 
% % \Omega_{r(\rho)} \subset B_{\rho}(x^*)
% $$
% And therefore: $\Omega_{r(\rho)} \subset B_{\rho}(x^*)$.   This leads to: 


% \begin{lemma}
% \label{lem:levelset}
% If 
% \begin{eqnarray}
% \label{eqn:deltabound}
% \delta < \delta_{\max} := \left(\frac{k_1}{\chi k_2}\right)^c \rho
% \end{eqnarray}
% then for $r(\delta) := k_2 (\chi \delta) ^c$ it follows that:
% $$
% B_{\chi \delta}(x^*) \subset \Omega_{r(\delta)} \subset B_{\rho}(x^*)
% $$
% Moreover, the set $\Omega_{r(\delta)}$ is forward invariant. 
% \end{lemma}

% This lemma gives an upper bound on the robustness of a given periodic orbit $\O$ based upon the domain of definition of $\P$; namely, $\delta_{\max}$.  

\begin{proof}
For $x \in B_{\chi \delta}(x^*)$: 
$$
\| x - x^* \| < \chi \delta ~  \Rightarrow  ~ 
V(x) \leq k_2 \| x - x^* \|^c < k_2 (\chi \delta)^c = r(\delta)
$$
and therefore $B_{\chi \delta}(x^*) \subset \Omega_{r(\delta)}$.  Now if $r(\delta) < k_1 \rho^c $ (which is equivalent to the condition \eqref{eqn:deltabound}) it follows that:
$$
V(x) \leq r(\delta) \quad \Rightarrow \quad 
k_1 \| x - x^* \|^c \leq V(x)  \leq r(\delta) <  k_1 \rho^c 
% \quad \implies \quad 
% \| x - x^* \| < \rho
% \quad \Rightarrow \quad 
% \Omega_{r(\rho)} \subset B_{\rho}(x^*)
$$
And therefore: $\Omega_{r(\delta)} \subset B_{\rho}(x^*)$.   
% But the condition that $r(\delta) < k_1 \rho^c $ is equivalent to \eqref{eqn:deltabound}. 
%
% \blue{
% \begin{align*}
%     r(\delta) := k_2(\chi \delta)^c &< k_1 \rho^c \\
%     \delta^c &< \frac{k_1}{\chi^{c}k_2}\rho^c \\
%     \delta &< \left(\frac{k_1}{\chi^c k_2}\right)^{\frac{1}{c}} \rho 
% \end{align*}
% }
%
Finally, since for $\delta < \delta_{\max}$ we have $B_{\chi \delta}(x^*) \subset \Omega_{r(\delta)}$, it follows that on the boundary of $\Omega_{r(\delta)}$, namely $\partial \Omega_{r(\delta)}$, condition \eqref{eqn:lyapimplication} is active and therefore: $\Delta V(x,d) < 0$.  The forward invariance of $\Omega_{r(\delta)}$ follows. %: $x_k \in V_{r(\delta)}$ for all $k \in \N_{\geq 0}$.  
\end{proof}

Lemma \ref{lem:levelset} gives an upper bound on the $\delta$-robustness of a given periodic orbit $\O$, namely $\delta_{\max}$, based upon the domain of definition of $\P$.  It also establishes the forward invariance of  $\Omega_{r(\delta)}$.  
Leveraging this, we can prove the main result. 

\begin{proof}[Proof of Theorem \ref{thm:main}]
Let $x_0 \in \Omega_{r(\delta)}$, wherein the forward invariance of $\Omega_{r(\delta)}$ (Lemma \ref{lem:levelset}) implies $x_k \in \Omega_{r(\delta)} \subset B_{\rho}(x^*)$ for all $k \in \N_{\geq 0}$.  Thus both $\P$ and $V$ are well-defined.
%, and the conditions \eqref{eqn:lyap1} and \eqref{eqn:lyapimplication} hold. 
We consider two cases: $x_0 \notin B_{\chi \delta}(x^*)$ and $x_0 \in B_{\chi \delta}(x^*)$. 


\vspace{0.1cm}
\underline{$\| x_0 - x^* \| \geq \chi \delta$:}  In this case the implication \eqref{eqn:lyapimplication} is active: 
$$
\Delta V \leq - \frac{k_3}{k_2} V \quad \implies \quad
V(x_k) \leq \left( 1 - \frac{k_3}{k_2}\right)^k V(x_0)
$$
where the implication follows from applying the inequality on the right recursively (see also the comparison lemma \cite{jiang2002converse}).  Therefore, using the inequalities in \eqref{eqn:lyap1} we have: 
\begin{eqnarray}
\label{eqn:Malpha}
\| x_k - x^* \| \leq  \underbrace{\left( \frac{k_2}{k_1} \right)^{\frac{1}{c}}}_{M} \underbrace{\left( 1 - \frac{k_3}{k_2}\right)^{\frac{k}{c}}}_{\alpha^k} \| x_0 - x^* \|.
\end{eqnarray}
Finally, note that $k_3/k_2 < 1$ as otherwise $V(x_k)$ would be negative for $k = 1$ which is impossible.  Therefore, $\alpha < 1$. 
%This also implies that $\exists K$ such that $x_K \in B_{\chi \delta}(x^*)$. 

\vspace{0.1cm}
\underline{$\| x_0 - x^* \| < \chi \delta$:}  While the implication in \eqref{eqn:lyapimplication} no longer holds, we still have $x_k \in \Omega_{r(\delta)}$.  As a result: 
\begin{align}
k_1 \| x_k - x^* \|^c \leq  V(k_k) \leq  r(\delta) =  & k_2 (\chi \delta) ^c 
  \nonumber\\
  \label{eqn:gammaeqn}
 \quad \implies \quad \| x_k - x^* \| \leq &  \underbrace{\left( \frac{k_2}{k_1} \right)^{\frac{1}{c}} \chi}_{\gamma} \delta 
\end{align}
Therefore, for $M$, $\alpha$ in \eqref{eqn:Malpha} and $\gamma$ in \eqref{eqn:gammaeqn} we have:
\begin{eqnarray}
\| x_k - x^* \| & \leq  & \max \{ M \alpha^k \| x_0 - x^* \|, \gamma \delta \} \nonumber\\
& \leq &  M \alpha^k \| x_0 - x^* \| + \gamma \delta \nonumber
\end{eqnarray}
as desired, i.e., $\delta$-robustness is established with $W =  \Omega_{r(\delta)}$ the required forward invariant set.  
% We first note that $V_{r(\delta)}$ is forward invariant.  In particular, for $\delta < \delta_{\max}$ we have $B_{\chi \delta}(x^*) \subset V_{r(\delta)}$.  Thus, on the boundary $\Delta V(x,d) < 0$ and forward invariance follows. 
\end{proof}

% we assume the reader is familiar with this proof, although we still attempt to make the proof self contained to note the key differences---which stem from the fact that $\P$ is only a partial function. 



% \begin{lemma}
%     \label{lemma:forwadinvariance}
%     Consider the discrete-time system \eqref{eq: discretetime}. If $B_{\delta}(x^*)$ is forward invariant: 
%     \begin{eqnarray}
%     \label{eqn:forwardsetinvariance}
%     x_0 \in B_{\delta}(x^*) \quad \implies  \quad 
%     x_{k+1} = P_{d_k}(x_k) \in B_{\delta}(x^*) 
%     \end{eqnarray}
%     for all $x \in \N_{\geq 0}$ and for every sequence $\{d_k\}_{k \in \N_{\geq 0}}$ with $d_k \in [d^-_{\delta},d^+_{\delta}]$, then $\O$ is $\delta$-robust. 
% \end{lemma}

% \begin{proof}
% Assume that $\O$ is not $\delta$-robust.  Then there exists an $x_0 \in B_{\delta}(x^*)$ and a $d_0 \in [d^-_{\delta},d^+_{\delta}]$ such that $P_{d_0}(x_0) \notin B_{\delta}(x^*)$.  This contradicts the implication in \eqref{eqn:forwardsetinvariance}. 
% \end{proof}



% \newsec{Connections with Input-to-state Stability}
% Note that beyond simply finding the $\delta$-robustness of a gait, it may be desirable to ensure exponential stability to $B_{\delta}(x^*)$.  This leads to the following definition: 

% \begin{definition}
% The periodic orbit $\O$ is \textbf{$\delta$-robustly exponentially stable} if it is $\delta$-robust and there exists an open and connected set $W$ with $B_{\delta}(x^*) \subset W $ such that:
% \begin{align}
%   \forall ~ x_0 \in W   &  \quad \implies \quad  \nonumber\\
%  & \| x_k \|_{B_{\delta}(x^*)} \leq M\alpha^k \|x_0 \|_{B_{\delta}(x^*)},  \quad \forall k \in \N_{\geq 0} \nonumber
% \end{align}
% % \begin{align}
% %   \forall ~ x \in B_{\delta + \epsilon}(x^*) \backslash B_{\delta}(x^*)  &  \quad \implies \quad  \nonumber\\
% %  & \| P^i_d(x) \|_{B_{\delta}(x^*)} \leq M\alpha^i \|x \|_{B_{\delta}(x^*)}, \nonumber
% % \end{align}
% for $M > 0$ and $\alpha \in (0,1)$. 
% \end{definition}




% \begin{theorem}
% The periodic orbit $\O$ is $\delta$-robustly exponentially stable if and only if the dynamical system in \eqref{eq: discretetime} is exponentially input-to-state stable \eqref{eqn:expiss}.
% \end{theorem}