%%%%%%%%%%%%%%%%%%%%%%%%%%%%%%%%%%%%%%%%%%%%%%%%%%%%%%%%%%%%%%%%%%%%%%%%%%%%%%%%
%2345678901234567890123456789012345678901234567890123456789012345678901234567890
%        1         2         3         4         5         6         7         8

\documentclass[letterpaper, 10 pt, conference]{Formatting/ieeeconf}  % Comment this line out
                                                          % if you need a4paper
%\documentclass[a4paper, 10pt, conference]{ieeeconf}      % Use this line for a4
                                                          % paper

\IEEEoverridecommandlockouts                              % This command is only
                                                          % needed if you want to
                                                          % use the \thanks command
\overrideIEEEmargins

\usepackage{geometry}
\geometry{verbose,tmargin=54pt,bmargin=54pt,lmargin=54pt,rmargin=54pt,headsep=72pt}

% See the \addtolength command later in the file to balance the column lengths
% on the last page of the document

% The following packages can be found on http:\\www.ctan.org
\usepackage{graphics} % for pdf, bitmapped graphics files
\usepackage{epsfig} % for postscript graphics files
\usepackage{times} % assumes new font selection scheme installed
\usepackage{amsmath}
\usepackage{amssymb}
\usepackage{comment}
\usepackage{cases}



% Custom shortcuts.sty
\usepackage{./Formatting/shortcuts}

% \title{\LARGE \bf
%  A Set Theoretic Perspective on Bipedal Walking
% }

\title{\LARGE \bf
 An Input-to-State Stability Perspective on Robust Locomotion% for Bipedal Robots
}

\author{Maegan Tucker and Aaron D. Ames% <-this % stops a space
\thanks{This work was supported by Wandercraft and the Zeitlin Family Fund.}% <-this % stops a space
\thanks{Maegan Tucker is with the Department of Mechanical and Civil Engineering, California              Institute of Technology, Pasadena, CA 91125.
        {\tt\small mtucker@caltech.edu}}%
\thanks{Aaron Ames is with both the Department 
        of Mechanical and Civil Engineering and the Department of Computing and Mathematical Sciences, California Institute of Technology, Pasadena, CA 91125.
        {\tt\small ames@caltech.edu}}%
}

\begin{document}



\maketitle
\thispagestyle{empty}
\pagestyle{empty}


%%%%%%%%%%%%%%%%%%%%%%%%%%%%%%%%%%%%%%%%%%%%%%%%%%%%%%%%%%%%%%%%%%%%%%%%%%%%%%%%
\begin{abstract}
Uneven terrain necessarily transforms periodic walking into a non-periodic motion. As such, traditional stability analysis tools no longer adequately capture the ability of a bipedal robot to locomote in the presence of such disturbances.
% Despite the importance of periodic stability for bipedal robots, uneven terrain necessarily perturbs the system away from the nominal orbit. 
This motivates the need for analytical tools aimed at generalized notions of stability -- robustness. 
% Specifically, tools are needed to characterize the domain on which the orbit is stable. 
Towards this, we propose a novel definition of robustness, termed \emph{$\delta$-robustness}, to characterize the domain on which a nominal periodic orbit remains stable despite uncertain terrain. This definition is derived by treating perturbations in ground height as disturbances in the context of the input-to-state-stability (ISS) of the extended Poincar\'{e} map associated with a periodic orbit. The main theoretic result is the formulation of robust Lyapunov functions that certify $\delta$-robustness of periodic orbits.  This yields an optimization framework for verifying $\delta$-robustness, which is demonstrated in simulation with a bipedal robot walking on uneven terrain.

% Achieving stable bipedal locomotion is a challenging control task, especially when considering uncertain terrain. One approach with demonstrated success is to generate provably stable nominal walking behaviors, encoded by periodic orbits, and then use either feedback controllers or online planning to drive the system to the nominal behavior. 
% A benefit of this approach is that the stability of the nominal gait can then be analyzed using the method of Poincar\'e sections for systems with impulse effects. However, it is not guaranteed that each nominal walking behavior will remain stable in the presence of uncertain impact events. In this work, we extend the notion of stability for nominal periodic orbits with uncertain impact events by instead viewing walking from an input-to-state stable (ISS) perspective.
% In this work, we explore the conditions required to certify robustness of nominal hybrid periodic orbits in the presence of uncertain impact events, with a specific view of input-to-state stability (ISS). While online planning may still be required for large disturbances and changes in the environment, robust nominal periodic orbits would ultimately enable easier control tasks and reduce the amount of required control input. 

\end{abstract}


%%%%%%%%%%%%%%%%%%%%%%%%%%%%%%%%%%%%%%%%%%%%%%%%%%%%%%%%%%%%%%%%%%%%%%%%%%%%%%%%
% Importance and appeal of children's drawings
Children's depictions of the human figure are highly expressive and varied.
As one of the very first subjects children attempt to draw, the representation begins as an almost unintelligible cloud of scribbles. 
As the child grows, their representation of the human figure becomes more developed and is extended to graphically represent many different types of characters: people, animals, and even personified objects (see Figure 1).

Who among us has not wished, either as a child or as an adult, to see such figures come to life and move around on the page?
Sadly, while it is relatively fast to produce a single drawing, creating the sequence of images necessary for animation is a much more tedious endeavor, requiring discipline, skill, patience, and sometimes complicated software.
As a result, most of these figures remain static upon the page.

% We built a system to animate them.
Inspired by the importance and appeal of the drawn human figure, we design and build a system to automatically animate it given an in-the-wild photograph of a child's drawing. 
Our system is fast, intuitive, and robust to much of the variation present in these types of drawings, making it well-suited to allow our target audience--children--to see their own characters coming to life.
The system is comprised of four stages: figure detection, segmentation masking, pose estimation/rigging, and animation. 
We describe each stage and identify common causes of failure in each. 
For object detection and pose estimation, we make use of existing computer vision models designed to detect human figures and joints in photographs; we fine-tune these models for use with children's drawings.
For segmentation, we present a straightforward, image processing-based method that, for animation purposes, is more useful and accurate than segmentation masks obtained from a fine-tuned object detection model.
During the animation step, we take advantage of the \textit{twisted perspective} commonly seen in children’s drawings to retarget motion capture data onto the character in a novel and appealing way.

% We use existing machine learning models. However, given the wide domain gap it's not clear how much fine-tuning data was needed. So we ran some experiments to find out and report it.
While our system leverages existing models and techniques, most are not directly applicable to the task due to the many differences between photographic images and simple pen and paper representations. 
To this end, we couple the presentation of our system with a set of experiments exploring the relationship between fine-tuning training set size and success rates.
We also include a perceptual study validating viewer preference for incorporating \textit{twisted perspective} into the motion retargeting step.

We validate the desirability and appeal of our system by building and publicly releasing a version of it as the \AD Demo \,\cite{animateddrawings}.
Launched in December 2021, this demo has been used by millions of people around the world to animate their children's drawings.
Inspired by this reception, our second contribution is The Amateur Drawings Dataset: \hjs{180,000 drawings and user-accepted annotations collected, with consent, through the demo. See Section \ref{sec:UI} for a description of how the annotations were generated.}
We believe this dataset will be a resource to researchers from various fields seeking to better understand the space of amateur drawings, evaluate new algorithms in this domain, or develop new drawing-based tools in general.

To summarize, our contributions are as follows:
\begin{enumerate}
    \item 
    We explore the problem of automatic sketch-to-animation for children's drawings of human figures and present a framework that achieves this effect. We also present a set of experiments determining the amount of training data necessary to achieve high levels of success and a perceptual study validating the usefulness of our motion retargeting technique.
    \item To encourage additional research in the domain of amateur drawings, we present a first-of-its-kind dataset of 180,000 user-submitted amateur drawings, along with user-accepted bounding box, segmentation mask, and joint location annotations.
\end{enumerate}

Upon acceptance of this paper, we plan to publicly release the Amateur Drawings Dataset, project code, and fine-tuned model weights.

\section{Preliminaries}
\label{sec2}
Deep learning has brought new inspirations to camera calibration, enabling a fully automatic calibration procedure without manual intervention. Here, we first summarize two prevalent paradigms in learning-based camera calibration: regression-based calibration and reconstruction-based calibration. Then, the widely-used learning strategies are reviewed in this research field. The detailed definitions for classical camera models and their corresponding calibration objectives are exhibited in the supplementary material.

\vspace{-0.3cm}

\subsection{Learning Paradigm}
Driven by different architectures of the neural network, the researchers have developed two main paradigms for learning-based camera calibration and its applications.

\noindent \textbf{Regression-based Calibration}
Given an uncalibrated input, the regression-based calibration first extracts the high-level semantic features using stacked convolutional layers. Then, the fully connected layers aggregate the semantic features and form a vector of the estimated calibration objective. The regressed parameters are used to conduct subsequent tasks such as distortion rectification, image warping, camera localization, etc. This paradigm is the earliest and has a dominant role in learning-based camera calibration and its applications. All the first works in various objectives, \textit{e.g.}, intrinsics: Deepfocal \cite{DeepFocal}, extrinsic: PoseNet \cite{PoseNet}, radial distortion: Rong et al. \cite{Rong}, rolling shutter distortion: URS-CNN \cite{URS-CNN}, homography matrix: DHN \cite{DHN}, hybrid parameters: Hold-Geoffroy et al. \cite{Hold-Geoffroy}, camera-LiDAR parameters: RegNet \cite{schneider2017regnet} have been achieved with this paradigm.

\noindent \textbf{Reconstruction-based Calibration}
On the other hand, the reconstruction-based calibration paradigm discards the parameter regression and directly learns the pixel-level mapping function between the uncalibrated input and target, inspired by the conditional image-to-image translation \cite{pix2pix} and dense visual perception\cite{long2015fully, eigen2014depth}. The reconstructed results are then calculated for the pixel-wise loss with the ground truth. In this regard, most reconstruction-based calibration methods \cite{DR-GAN, DDM, DaRecNet, BlindCor} design their network architecture based on the fully convolutional network such as U-Net\cite{ronneberger2015u}. Specifically, an encoder-decoder network, with skip connections between the encoder and decoder features at the same spatial resolution, progressively extracts the features from low-level to high-level and effectively integrates multi-scale features. At the last convolutional layer, the learned features are aggregated into the target channel, reconstructing the calibrated result at the pixel level.

In contrast to the regression-based paradigm, the reconstruction-based paradigm does not require the label of diverse camera parameters. Besides, the imbalance loss problem can be eliminated since it only optimizes the photometric loss of calibrated results. Therefore, the reconstruction-based paradigm enables a blind camera calibration without a strong camera model assumption.

\vspace{-0.3cm}

\subsection{Learning Strategies}
In the following, we review the learning-based camera calibration literature regarding different learning strategies.

\noindent \textbf{Supervised Learning}
Most learning-based camera calibration methods train their networks with the supervised learning strategy, from the classical methods \cite{DeepFocal, PoseNet, DHN, DeepVP, Rong, DeepCalib} to the state-of-the-art methods \cite{DVPD, EvUnroll, FishFormer, DAMG-Homo, SST-Calib}. In terms of the learning paradigm, this strategy supervises the network with the ground truth of the camera parameters (regression-based paradigm) or paired data (reconstruction-based paradigm). In general, they synthesize the training dataset from other large-scale datasets, under the random parameter/transformation sampling and camera model simulation. Some recent works \cite{Zhao, Tan, SPEC, DeepUnrollNet} establish their training dataset using a real-world setup and label the captured images with manual annotations, thereby fostering advancements in this research domain.

\noindent \textbf{Semi-Supervised Learning}
Training the network using an annotated dataset under diverse scenarios is an effective learning strategy. However, human annotation can be prone to errors, leading to inconsistent annotation quality or the inclusion of contaminated data. Consequently, increasing the training dataset to improve performance can be challenging due to the complexity and cost of constructing the dataset. To address this challenge, SS-WPC\cite{SS-WPC} proposes a semi-supervised method for correcting portraits captured by a wide-angle camera. It employs a surrogate task (segmentation) and a semi-supervised method that utilizes direction and range consistency and regression consistency to leverage both labeled and unlabeled data.

\noindent \textbf{Weakly-Supervised Learning}
Although significant progress has been made, data labeling for camera calibration is a notorious costly process, and obtaining perfect ground-truth labels is challenging. As a result, it is often preferable to use weak supervision with machine learning methods. Weakly supervised learning refers to the process of building prediction models through learning with inadequate supervision. Zhu et al. \cite{Zhu} present a weakly supervised camera calibration method for single-view metrology in unconstrained environments, where there is only one accessible image of a scene composed of objects of uncertain sizes. This work leverages 2D object annotations from large-scale datasets, where people and buildings are frequently present and serve as useful ``reference objects'' for determining 3D size.

\begin{figure*}[!t]
  \centering
  \includegraphics[width=1\textwidth]{figures/taxonomy1.pdf}
  %\vspace{-20pt}
  \caption{The structural and hierarchical taxonomy of camera calibration with deep learning. Some classical methods are listed under each category.}
  \label{fig:taxonomy}
  \vspace{-0.2cm}
\end{figure*}

\noindent \textbf{Unsupervised Learning}
Unsupervised learning, commonly referred to as unsupervised machine learning, analyzes and groups unlabeled datasets using machine learning algorithms. UDHN \cite{UDHN} is the first work for a cross-view camera model using unsupervised learning, which estimates the homography matrix of a paired image without the projection labels. By reducing a pixel-wise intensity error that does not require ground truth data, UDHN \cite{UDHN} outperforms previous supervised learning techniques. While preserving superior accuracy and robustness to fluctuation in light, the proposed unsupervised algorithm can also achieve faster inference time. Inspired by this work, increasing more methods leverage the unsupervised learning strategy to estimate the homography such as CA-UDHN \cite{CA-UDHN}, BaseHomo \cite{BasesHomo}, HomoGAN\cite{HomoGAN}, and Liu et al. \cite{Liu}. Besides, UnFishCor \cite{UnFishCor} frees the demands for distortion parameters and designs an unsupervised framework for the wide-angle camera.

\noindent \textbf{Self-supervised Learning}
Robotics is where the phrase ``self-supervised learning'' first appears, as training data is automatically categorized by utilizing relationships between various input sensor signals. Compared to supervised learning, self-supervised learning leverages input data itself as the supervision. Many self-supervised techniques are presented to learn visual characteristics from massive amounts of unlabeled photos or videos without the need for time-consuming and expensive human annotations. SSR-Net \cite{SSR-Net} presents a self-supervised deep homography estimation network, which relaxes the need for ground truth annotations and leverages the invertibility constraints of homography. To be specific, SSR-Net \cite{SSR-Net} utilizes the homography matrix representation in place of other approaches' typically-used 4-point parameterization, to apply the invertibility constraints. SIR \cite{SIR} devises a brand-new self-supervised camera calibration pipeline for wide-angle image rectification, based on the principle that the corrected results of distorted images of the same scene taken with various lenses need to be the same. With self-supervised depth and pose learning as a proxy aim, Fang et al. \cite{Fang} present to self-calibrate a range of generic camera models from raw video, offering for the first time a calibration evaluation of camera model parameters learned solely via self-supervision.

\noindent \textbf{Reinforcement Learning}
Instead of aiming to minimize at each stage, reinforcement learning can maximize the cumulative benefits of a learning process as a whole. To date, DQN-RecNet~\cite{DQN-RecNet} is the first and only work in camera calibration using reinforcement learning. It applies a deep reinforcement learning technique to tackle the fisheye image rectification by a single Markov Decision Process, which is a multi-step gradual calibration scheme. In this situation, the current fisheye image represents the state of the environment. The agent, Deep Q-Network \cite{mnih2015human}, generates an action that should be executed to correct the distorted image.

In the following, we will review the specific methods and literature for learning-based camera calibration. The structural and hierarchical taxonomy is shown in Figure~\ref{fig:taxonomy}. 
	
	\begin{table*}
		\rowcolors{1}{gray!20}{white}
		\centering
		\caption{
			{Details of the learning-based camera calibration and its extended applications from 2015 to 2022, including the method abbreviation, publication, calibration objective, network architecture, loss function, dataset, evaluation metrics, learning strategy, platform, and simulation or not (training data). For the learning strategies, SL, USL, WSL, Semi-SL, SSL, and RL denote supervised learning, unsupervised learning, weakly-supervised learning, semi-supervised learning, self-supervised learning, and reinforcement learning, respectively. }
		}
		\vspace{-6pt}
		\label{table:methods}
		\begin{threeparttable}
			\resizebox{1\textwidth}{!}{
				\setlength\tabcolsep{2pt}
				\renewcommand\arraystretch{0.98}
				% \begin{tabular}{|c|c|r||c|c|c|c|c|c|c|c|c|}  % {lccc}
				\begin{tabular}{c|r||c|c|c|c|c|c|c|c|c}
					\hline
					%\thickhline
					% &\#&
					&\textbf{Method}~~~~~~~~~&\textbf{Publication} &\textbf{Objective} &\textbf{Network}
					&\textbf{Loss Function} & \textbf{Dataset} &\textbf{Evaluation} & \textbf{Learning} &\textbf{Platform} &\textbf{Simulation}\\
					\hline
					\hline
					\multirow{1}{*}{\rotatebox{0}{\textbf{2015}}}
					% &1 &
					&DeepFocal~\cite{DeepFocal} &ICIP &Standard &AlexNet
					&$\mathcal{L}_2$ loss &1DSfM\cite{1DSfM} & Accuracy & SL &Caffe &\\
					&PoseNet~\cite{PoseNet} &ICCV &Standard 
					&GoogLeNet
					&$\mathcal{L}_2$ loss &Cambridge Landmarks\cite{Cambridge_Landmarks} &Accuracy &SL &Caffe& \\
					\hline
					\hline
					\multirow{1}{*}{\rotatebox{0}{\textbf{2016}}}
					% &1&
					&DeepHorizon~\cite{DeepHorizon} &BMVC &Standard &GoogLeNet	&Huber loss &HLW\cite{HLW} & Accuracy & SL &Caffe &\\
					
					&DeepVP~\cite{DeepVP} &CVPR &Standard 
					&AlexNet
					&Logistic loss &YUD\cite{YUD}, ECD\cite{ECD}, HLW\cite{HLW} &Accuracy &SL &Caffe& \\	
					
					&Rong et al.~\cite{Rong} &ACCV &Distortion &AlexNet
					&Softmax loss &ImageNet\cite{ImageNet} &Line length &SL &Caffe&\checkmark\\
					
					&DHN\cite{DHN} &RSSW &Cross-View &VGG
					&$\mathcal{L}_2$ loss &MS-COCO\cite{MS-COCO} &MSE &SL &Caffe&\checkmark\\		\hline
					\hline
					\multirow{1}{*}{\rotatebox{0}{\textbf{2017}}}
					% &1&
					&CLKN~\cite{CLKN} &CVPR &Cross-View  &CNNs	&Hinge loss &MS-COCO\cite{MS-COCO} & MSE & SL &Torch &\checkmark\\
					
		            &HierarchicalNet~\cite{HierarchicalNet} &ICCVW &Cross-View 
					&VGG
					&$\mathcal{L}_2$ loss &MS-COCO\cite{MS-COCO} &MSE &SL &TensorFlow&\checkmark \\
					
					&URS-CNN~\cite{URS-CNN} &CVPR &Distortion 
					&CNNs
					&$\mathcal{L}_2$ loss &Sun\cite{xiao2010sun}, Oxford\cite{philbin2007object}, Zubud\cite{shao2003zubud}, LFW\cite{huang2008labeled} &PSNR, RMSE &SL &Torch&\checkmark\\
					
					&RegNet~\cite{schneider2017regnet} &IV &Cross-Sensor 
					&CNNs
					&$\mathcal{L}_2$ loss &KITTI\cite{KITTI} &MAE &SL &Caffe&\checkmark\\
					
					\hline
					\hline
					\multirow{1}{*}{\rotatebox{0}{\textbf{2018}}}
					% &1&
					&Hold-Geoffroy et al.~\cite{Hold-Geoffroy} &CVPR &Standard &DenseNet	&Entropy loss &SUN360\cite{SUN360} & Human sensitivity & SL &- &\\
					
					&DeepCalib~\cite{DeepCalib} &CVMP &Distortion 
					&Inception-V3
					&Logcosh loss &SUN360\cite{SUN360} &Mean error &SL &TensorFlow&\checkmark \\	
					&FishEyeRecNet~\cite{FishEyeRecNet} &ECCV &Distortion &VGG
					&$\mathcal{L}_2$ loss &ADE20K\cite{ADE20K} &PSNR, SSIM &SL &Caffe&\checkmark\\
					
					&Shi et al.\cite{Shi} &ICPR &Distortion &ResNet
					&$\mathcal{L}_2$ loss &ImageNet\cite{ImageNet} &MSE &SL &PyTorch&\checkmark\\
					
					&DeepFM\cite{DeepFM} &ECCV &Cross-View &ResNet
					&$\mathcal{L}_2$ loss &T\&T\cite{TT}, KITTI\cite{KITTI}, 1DSfM\cite{1DSfM} &F-score, Mean &SL &PyTorch&\checkmark\\
					
					&Poursaeed et al.\cite{Poursaeed} &ECCVW &Cross-View &CNNs
					&$\mathcal{L}_1$, $\mathcal{L}_2$ loss &KITTI\cite{KITTI} &EPI-ABS, EPI-SQR &SL &-& \\
					
					&UDHN\cite{UDHN} &RAL &Cross-View &VGG
					&$\mathcal{L}_1$ loss &MS-COCO\cite{MS-COCO} &RMSE &USL &TensorFlow&\checkmark\\
					
					&PFNet\cite{PFNet} &ACCV &Cross-View &FCN
					&Smooth $\mathcal{L}_1$ loss &MS-COCO\cite{MS-COCO} &MAE &SL &TensorFlow&\checkmark\\
					
					&CalibNet\cite{iyer2018calibnet} &IROS &Cross-Sensor &ResNet
					&Point cloud distance, $\mathcal{L}_2$ loss &KITTI\cite{KITTI} &Geodesic distance, MAE &SL &TensorFlow&\checkmark\\

                        &Chang et al.\cite{chang2018deepvp} &ICRA &Standard &AlexNet
					&Cross-entropy loss &DeepVP-1M~\cite{chang2018deepvp} &MSE, Accuracy &SL &Matconvnet&\\
					
					\hline
					\hline
					\multirow{1}{*}{\rotatebox{0}{\textbf{2019}}}
					% &1&
					&Lopez et al.~\cite{Lopez} &CVPR &Distortion &DenseNet	&Bearing loss &SUN360\cite{SUN360} &MSE & SL &PyTorch &\\
					
					&UprightNet~\cite{UprightNet} &ICCV &Standard &U-Net	&Geometry loss &InteriorNet\cite{InteriorNet}, ScanNet\cite{ScanNet}, SUN360\cite{SUN360} &Mean error & SL &PyTorch &\\
					
					&Zhuang et al.~\cite{Zhuang} &IROS &Distortion &ResNet	&$\mathcal{L}_1$ loss & KITTI\cite{KITTI} &Mean error, RMSE & SL &PyTorch &\checkmark\\
					
					&SSR-Net~\cite{SSR-Net} &PRL &Cross-View &ResNet	&$\mathcal{L}_2$ loss & MS-COCO\cite{MS-COCO} &MAE & SSL &PyTorch &\checkmark\\
					
					&Abbas et al.~\cite{Abbas} &ICCVW &Cross-View &CNNs	&Softmax loss & CARLA\cite{CARLA} &AUC\cite{AUC}, Mean error & SL &TensorFlow &\checkmark\\
					
					&DR-GAN~\cite{DR-GAN} &TCSVT &Distortion &GANs	&Perceptual loss & MS-COCO\cite{MS-COCO} &PSNR, SSIM & SL &TensorFlow &\checkmark\\
					
					&STD~\cite{STD} &TCSVT &Distortion &GANs+CNNs	&Perceptual loss & MS-COCO\cite{MS-COCO} &PSNR, SSIM & SL &TensorFlow &\checkmark\\
					
					&Deep360Up~\cite{Deep360Up} &VR &Standard &DenseNet	&Log-cosh loss\cite{Log-cosh} & SUN360\cite{SUN360} &Mean error & SL &- &\checkmark\\
					
					&UnFishCor~\cite{UnFishCor} &JVCIR &Distortion &VGG	&$\mathcal{L}_1$ loss & Places2\cite{Places2} &PSNR, SSIM & USL &TensorFlow &\checkmark\\
					
					&BlindCor~\cite{BlindCor} &CVPR &Distortion &U-Net	&$\mathcal{L}_2$ loss & Places2\cite{Places2} &MSE & SL &PyTorch &\checkmark\\
					
					&RSC-Net~\cite{RSC-Net} &CVPR &Distortion &ResNet	&$\mathcal{L}_1$ loss & KITTI\cite{KITTI} &Mean error & SL &PyTorch &\checkmark\\
					
					&Xue et al.~\cite{Xue} &CVPR &Distortion &ResNet	&$\mathcal{L}_2$ loss & Wireframes\cite{Wireframes}, SUNCG\cite{SUNCG} &PSNR, SSIM, RPE & SL &PyTorch &\checkmark\\
					
					&Zhao et al.~\cite{Zhao} &ICCV &Distortion &VGG+U-Net	&$\mathcal{L}_1$ loss & Self-constructed+BU-4DFE\cite{BU-4DFE} &Mean error &SL &- &\checkmark\\

					&NeurVPS~\cite{zhou2019neurvps} &NeurIPS &Standard &CNNs	&Binary cross entropy, chamfer-$\mathcal{L}_2$ loss &ScanNet~\cite{ScanNet}, SU3~\cite{SU3} &Angle accuracy &SL &PyTorch &\\

     
					
					\hline
					\hline
					\multirow{1}{*}{\rotatebox{0}{\textbf{2020}}}
					% &1&
					&Sha et al.~\cite{Sha} &CVPR &Cross-View &U-Net	& Cross-entropy loss &World Cup 2014\cite{homayounfar2017sports} &IoU & SL &TensorFlow &\\
				    
				    &Lee et al.~\cite{Lee} &ECCV &Standard &PointNet + CNNs	& Cross-entropy loss &Google Street View\cite{googleStreet}, HLW\cite{HLW} &Mean error, AUC\cite{AUC} & SL &- &\\
				    
				    &MisCaliDet~\cite{MisCaliDet} &ICRA &Distortion &CNNs	& $\mathcal{L}_2$ loss &KITTI\cite{KITTI} &MSE & SL &TensorFlow &\checkmark\\
				    
				    &DeepPTZ~\cite{DeepPTZ} &WACV &Distortion &Inception-V3	& $\mathcal{L}_1$ loss &SUN360\cite{SUN360} &Mean error & SL &PyTorch &\checkmark\\
				    
				    &MHN~\cite{MHN} &CVPR &Cross-View &VGG	&Cross-entropy loss &MS-COCO\cite{MS-COCO}, Self-constructed &MAE & SL &TensorFlow &\checkmark\\
				    
				    &Davidson et al.~\cite{Davidson} &ECCV &Standard &FCN	&Dice loss &SUN360\cite{SUN360} &Accuracy &SL &- &\checkmark\\
				    
				    &CA-UDHN~\cite{CA-UDHN} &ECCV &Cross-View &FCN + ResNet	&Triplet loss &Self-constructed &MSE &USL &PyTorch &\\
				    
				    &DeepFEPE~\cite{DeepFEPE} &IROS &Standard &VGG + PointNet	&$\mathcal{L}_2$ loss &KITTI\cite{KITTI}, ApolloScape\cite{Apolloscape} &Mean error &SL &PyTorch &\\
				    
				    &DDM~\cite{DDM} &TIP &Distortion &GANs	&$\mathcal{L}_1$ loss &MS-COCO\cite{MS-COCO} &PSNR, SSIM &SL &TensorFlow &\checkmark\\
				    
				    &Li et al.~\cite{Li} &TIP &Distortion &CNNs	&Cross-entropy, $\mathcal{L}_1$ loss &CelebA\cite{CelebA} &Cosine distance &SL &- &\checkmark\\
				    
				    &PSE-GAN~\cite{PSE-GAN} &ICPR &Distortion &GANs	&$\mathcal{L}_1$, WGAN loss &Place2\cite{Places2} &MSE &SL &- &\checkmark\\
				    
				    &RDC-Net~\cite{RDC-Net} &ICIP &Distortion &ResNet	&$\mathcal{L}_1$, $\mathcal{L}_2$ loss &ImageNet\cite{ImageNet} &PSNR, SSIM &SL &PyTorch &\checkmark\\
				    
				    &FE-GAN~\cite{FE-GAN} &ICASSP &Distortion &GANs	&$\mathcal{L}_1$, GAN loss &Wireframe\cite{Wireframes}, LSUN\cite{LSUN} &PSNR, SSIM, RMSE &SSL &PyTorch &\checkmark\\
				    
				    &RDCFace~\cite{RDCFace} &CVPR &Distortion &ResNet	&Cross-entropy, $\mathcal{L}_2$ loss &IMDB-Face\cite{IMDB-Face} &Accuracy &SL &- &\checkmark\\
				    
				    &LaRecNet~\cite{LaRecNet} &arXiv &Distortion &ResNet	&$\mathcal{L}_2$ loss &Wireframes\cite{Wireframes}, SUNCG\cite{SUNCG} &PSNR, SSIM, RPE &SL &PyTorch &\checkmark\\
				    
				    &Baradad et al.~\cite{Baradad} &CVPR &Standard &CNNs	&$\mathcal{L}_2$ loss &ScanNet\cite{ScanNet}, NYU\cite{NYU}, SUN360\cite{SUN360} &Mean error, RMS &SL &PyTorch &\\
				    
				    &Zheng et al.~\cite{Zheng} &CVPR &Standard &CNNs	&$\mathcal{L}_1$ loss &FocaLens\cite{FocaLens} &Mean error, PSNR, SSIM &SL &- &\checkmark\\
				    
				    &Zhu et al.~\cite{Zhu} &ECCV &Standard &CNNs + PointNet	&$\mathcal{L}_1$ loss &SUN360\cite{SUN360}, MS-COCO\cite{MS-COCO} &Mean error, Accuracy &WSL &PyTorch &\checkmark\\
				    
				    &DeepUnrollNet~\cite{DeepUnrollNet} &CVPR &Distortion &FCN	&$\mathcal{L}_1$, perceptual, total variation loss &Carla-RS\cite{DeepUnrollNet}, Fastec-RS\cite{DeepUnrollNet}  &PSNR, SSIM &SL &PyTorch &\checkmark\\
				    
				    &RGGNet~\cite{yuan2020rggnet} &RAL &Cross-Sensor &ResNet	&Geodesic distance loss &KITTI\cite{KITTI}  &MSE, MSEE, MRR &SL &TensorFlow &\checkmark\\
				    
				    &CalibRCNN~\cite{shi2020calibrcnn} &IROS &Cross-Sensor &RNNs	&$\mathcal{L}_2$, Epipolar geometry loss &KITTI~\cite{KITTI}  &MAE &SL &TensorFlow &\checkmark\\

				    &SSI-Calib~\cite{zhu2020online} &ICRA &Cross-Sensor &CNNs	&$\mathcal{L}_2$ loss &Pascal VOC 2012~\cite{pascal-voc-2012}  &Mean/standard deviation &SL &TensorFlow &\checkmark\\

				    &SOIC~\cite{wang2020soic} &arXiv &Cross-Sensor &ResNet + PointRCNN	& Cost function &KITTI~\cite{KITTI}  &Mean error &SL &- &\\        

				    &NetCalib~\cite{wu2021netcalib} &ICPR &Cross-Sensor &CNNs	&$\mathcal{L}_1$ loss &KITTI~\cite{KITTI}  &MAE &SL &PyTorch &\checkmark\\

				    &SRHEN~\cite{SRHEN} &ACM-MM &Cross-View &CNNs	&$\mathcal{L}_2$ loss &MS-COCO~\cite{MS-COCO}, SUN397~\cite{SUN360}  &MACE &SL &- &\checkmark\\
                        
				   
					\hline
					\hline
					\multirow{1}{*}{\rotatebox{0}{\textbf{2021}}}
					% &1&
					&StereoCaliNet~\cite{StereoCaliNet} &TCI &Standard &U-Net	&$\mathcal{L}_1$ loss &TAUAgent\cite{TAUAgent}, KITTI\cite{KITTI} &Mean error & SL &PyTorch &\checkmark\\
					
					&CTRL-C~\cite{CTRL-C} &ICCV &Standard &Transformer	&Cross-entropy, $\mathcal{L}_1$ loss &Google Street View\cite{googleStreet}, SUN360\cite{SUN360} &Mean error, AUC\cite{AUC} & SL &PyTorch &\checkmark\\
					
				   &Wakai et al.~\cite{Wakai} &ICCVW &Distortion &DenseNet	&Smooth $\mathcal{L}_1$ loss &StreetLearn\cite{StreetLearn} &Mean error, PSNR, SSIM & SL &- &\checkmark\\
				   &OrdianlDistortion~\cite{OrdianlDistortion} &TIP &Distortion &CNNs	&Smooth $\mathcal{L}_1$ loss & MS-COCO\cite{MS-COCO} &PSNR, SSIM, MDLD & SL &TensorFlow &\checkmark\\
				   
				   &PolarRecNet~\cite{PolarRecNet} &TCSVT &Distortion &VGG + U-Net	&$\mathcal{L}_1$, $\mathcal{L}_2$ loss & MS-COCO\cite{MS-COCO}, LMS\cite{LMS} &PSNR, SSIM, MSE & SL &PyTorch &\checkmark\\
				   
				   &DQN-RecNet~\cite{DQN-RecNet} &PRL &Distortion &VGG	&$\mathcal{L}_2$ loss & Wireframes\cite{Wireframes} &PSNR, SSIM, MSE & RL &PyTorch &\checkmark\\
				   
				   &Tan et al.~\cite{Tan} &CVPR &Distortion &U-Net &$\mathcal{L}_2$ loss & Self-constructed &Accuracy & SL &PyTorch & \\
				   
				   &PCN~\cite{PCN} &CVPR &Distortion &U-Net &$\mathcal{L}_1$, $\mathcal{L}_2$, GAN loss & Place2\cite{Places2} &PSNR, SSIM, FID, CW-SSIM & SL &PyTorch &\checkmark \\
				   
				   &DaRecNet~\cite{DaRecNet} &ICCV &Distortion &U-Net &Smooth $\mathcal{L}_1$, $\mathcal{L}_2$ loss & ADE20K\cite{ADE20K} &PSNR, SSIM & SL &PyTorch &\checkmark \\
				   
				   &DLKFM~\cite{DLKFM} &CVPR &Cross-View &Siamese-Net &$\mathcal{L}_2$ loss & MS-COCO\cite{MS-COCO}, Google Earth, Google Map &MSE & SL &TensorFlow &\checkmark \\
				   
				   &LocalTrans~\cite{LocalTrans} &ICCV &Cross-View &Transformer &$\mathcal{L}_1$ loss & MS-COCO\cite{MS-COCO} &MSE, PSNR, SSIM & SL &PyTorch &\checkmark \\
				   
				   &BasesHomo~\cite{BasesHomo} &ICCV &Cross-View &ResNet &Triplet loss & CA-UDHN\cite{CA-UDHN} &MSE & USL &PyTorch & \\
				   &ShuffleHomoNet~\cite{ShuffleHomoNet} &ICIP &Cross-View &ShuffleNet &$\mathcal{L}_2$ loss & MS-COCO\cite{MS-COCO} &RMSE & SL &TensorFlow &\checkmark \\
				   
				   &DAMG-Homo~\cite{DAMG-Homo} &TCSVT &Cross-View &CNNs &$\mathcal{L}_1$ loss & MS-COCO\cite{MS-COCO}, UDIS\cite{UDIS} &RMSE, PSNR, SSIM & SL &TensorFlow &\checkmark \\
				   
				   &SA-MobileNet~\cite{SA-MobileNet} &BMVC &Standard &MobileNet &Cross-entropy loss& SUN360\cite{SUN360}, ADE20K\cite{ADE20K}, NYU\cite{NYU} &MAE, Accuracy & SL &TensorFlow &\checkmark \\
				   
				   &SPEC~\cite{SPEC} &ICCV &Standard &ResNet &Softargmax-$\mathcal{L}_2$ loss&Self-constructed &W-MPJPE, PA-MPJPE & SL &PyTorch &\checkmark \\
				   
				   &DirectionNet~\cite{DirectionNet} &CVPR &Standard &U-Net &Cosine similarity loss &InteriorNet\cite{InteriorNet}, Matterport3D\cite{Matterport3D}&Mean and median error  & SL &TensorFlow &\checkmark \\
				   
				   &JCD~\cite{JCD} &CVPR &Distortion &FCN &Charbonnier\cite{Charbonnier}, perceptual loss &BS-RSCD \cite{JCD}, Fastec-RS
                   \cite{DeepUnrollNet}&PSNR, SSIM, LPIPS  & SL &PyTorch & \\
                   
                   &LCCNet~\cite{lv2021lccnet} &CVPRW &Cross-Sensor &CNNs &Smooth $\mathcal{L}_1$, $\mathcal{L}_2$ loss &KITTI\cite{KITTI} &MSE  & SL &PyTorch &\checkmark \\
                   
                   &CFNet~\cite{lv2021cfnet} &Sensors &Cross-Sensor &FCN &$\mathcal{L}_1$, Charbonnier\cite{Charbonnier} loss &KITTI\cite{KITTI}, KITTI-360\cite{liao2022kitti} &MAE, MSEE, MRR  & SL &PyTorch &\checkmark \\

                   &Fan\etal~\cite{fan2021inverting} &ICCV &Distortion &U-Net &$\mathcal{L}_1$, perceptual loss &Carla-RS~\cite{DeepUnrollNet}, Fastec-RS~\cite{DeepUnrollNet} &PSNR, SSIM, LPIPS  & SL &PyTorch & \\

                   &SUNet~\cite{SUNet} &ICCV &Distortion &DenseNet + ResNet &$\mathcal{L}_1$, perceptual loss &Carla-RS~\cite{DeepUnrollNet}, Fastec-RS~\cite{DeepUnrollNet} &PSNR, SSIM  & SL &PyTorch & \\

                   &SemAlign~\cite{liu2021semalign} &IROS &Cross-Sensor &CNNs & Semantic alignment loss &KITTI~\cite{KITTI} &Mean/median rotation errors & SL &PyTorch &\checkmark\\
       
				   \hline
				   \hline
				   \multirow{1}{*}{\rotatebox{0}{\textbf{2022}}}
					% &1&
				   &DVPD~\cite{DVPD} &CVPR &Standard &CNNs	&Cross-entropy loss &SU3\cite{SU3}, ScanNet\cite{ScanNet}, YUD\cite{YUD}, NYU\cite{NYU} &Accuracy, AUC\cite{AUC} & SL &PyTorch &\checkmark\\
				   
				   &Fang et al.~\cite{Fang} &ICRA &Standard &CNNs	&$\mathcal{L}_2$ loss &KITTI\cite{KITTI}, EuRoC\cite{EuRoC}, OmniCam\cite{OmniCam} &MRE, RMSE & SSL &PyTorch &\\
				   
				   &CPL~\cite{CPL} &ICASSP &Standard &Inception-V3	&$\mathcal{L}_1$ loss &CARLA\cite{CARLA}, CyclistDetection\cite{CyclistDetection} &MAE & SL &TensorFlow &\checkmark\\
				   
				   &IHN~\cite{IHN} &CVPR &Cross-View  &Siamese-Net	&$\mathcal{L}_1$ loss &MS-COCO\cite{MS-COCO}, Google Earth, Google Map &MACE & SL &PyTorch &\checkmark\\
				   
				   &HomoGAN~\cite{HomoGAN} &CVPR &Cross-View  &GANs	&Cross-entropy, WGAN loss &CA-UDHN\cite{CA-UDHN} &Mean error & USL &PyTorch &\checkmark\\
				   
				   &SS-WPC~\cite{SS-WPC} &CVPR &Distortion  &Transformer	&Cross-entropy, $\mathcal{L}_1$ loss &Tan et al.\cite{Tan} &Accuracy & Semi-SL &PyTorch &\\
				   
				   &AW-RSC~\cite{AW-RSC} &CVPR &Distortion  &CNNs	&Charbonnier\cite{Charbonnier}, perceptual loss &Self-constructed, FastecRS\cite{DeepUnrollNet} &PSNR, SSIM &SL &PyTorch &\\
				   
				   &EvUnroll~\cite{EvUnroll} &CVPR &Distortion  &U-Net	&Charbonnier, perceptual, TV loss &Self-constructed, FastecRS\cite{DeepUnrollNet} &PSNR, SSIM, LPIPS &SL &PyTorch &\\
				   
				   &Do et al.~\cite{Do} &CVPR &Standard  &ResNet&$\mathcal{L}_2$, Robust angular \cite{RobustAngular} loss &Self-constructed, 7-SCENES\cite{7-SCENES} &Median error, Recall &SL &PyTorch &\\
				   
				   &DiffPoseNet~\cite{DiffPoseNet} &CVPR &Standard  &CNNs + LSTM&$\mathcal{L}_2$ loss &TartanAir\cite{TartanAir}, KITTI\cite{KITTI}, TUM-RGBD\cite{TUM-RGBD} &PEE, AEE\cite{AEE} &SSL &PyTorch &\\
				   
				   &SceneSqueezer~\cite{SceneSqueezer} &CVPR &Standard  &Transformer&$\mathcal{L}_1$ loss &RobotCar Seasons\cite{RobotCar}, Cambridge Landmarks\cite{Cambridge_Landmarks}  &Mean error, Recall\cite{AEE} &SL &PyTorch &\\
				   
				   &FocalPose~\cite{FocalPose} &CVPR &Standard  &CNNs&$\mathcal{L}_1$, Huber loss &Pix3D\cite{Pix3D}, CompCars\cite{StanfordCars}, StanfordCars\cite{StanfordCars}  &Median error, Accuracy &SL &PyTorch &\\
				   
				   &DXQ-Net~\cite{jing2022dxq} &arXiv &Cross-Sensor  &CNNs + RNNs&$\mathcal{L}_1$, geodesic loss &KITTI\cite{KITTI}, KITTI-360\cite{liao2022kitti}  &MSE &SL &PyTorch &\checkmark\\
				   
				   &SST-Calib~\cite{SST-Calib} &ITSC &Cross-Sensor  &CNNs &$\mathcal{L}_2$ loss &KITTI\cite{KITTI}  &QAD, AEAD &SL &PyTorch &\checkmark\\
				   &CCS-Net~\cite{zhang2022learning} &IROS &Distortion  &U-Net&$\mathcal{L}_1$ loss &TUM-RGBD\cite{TUM-RGBD} &MAE, RPE &SL &PyTorch &\checkmark\\
				   
				   &FishFormer~\cite{FishFormer} &arXiv &Distortion  &Transformer&$\mathcal{L}_2$ loss &Place2\cite{Places2}, CelebA\cite{CelebA}  &PSNR, SSIM, FID &SL &PyTorch &\checkmark\\
				   
                  &SIR~\cite{SIR} &TIP &Distortion &ResNet &$\mathcal{L}_1$ loss & ADE20K\cite{ADE20K}, WireFrames\cite{Wireframes}, MS-COCO\cite{MS-COCO} &PSNR, SSIM & SSL &PyTorch &\checkmark \\

				   &ATOP~\cite{ATOP} &TIV &Cross-Sensor  &CNNs &Cross entropy loss &Self-constructed + KITTI\cite{KITTI}  &RRE, RTE &SL &- &\\

				   &FusionNet~\cite{wang2022fusionnet} &ICRA &Cross-Sensor  &CNNs+PointNet &$\mathcal{L}_2$ loss &KITTI\cite{KITTI}  &MAE &SL &PyTorch &\checkmark\\

				   &RKGCNet~\cite{RKGCNet} &TIM &Cross-Sensor  &CNNs+PointNet &$\mathcal{L}_1$ loss &KITTI\cite{KITTI}  &MSE &SL &PyTorch &\checkmark\\

                    &GenCaliNet~\cite{GenCaliNet} &ECCV &Distortion &DenseNet	&$\mathcal{L}_2$ loss &StreetLearn\cite{StreetLearn}, SP360\cite{SP360} &MAE, PSNR, SSIM & SL &- &\checkmark\\
       
				   &Liu et al.~\cite{Liu} &TPAMI &Cross-View &ResNet&Triplet loss &Self-constructed  &MSE, Accuracy &USL &PyTorch &\\
				   
				
				\hline
				\end{tabular}
			}
		\end{threeparttable}
	\end{table*}
	
	
	
	
	
	
	
	
	
	
	
	
	

\section{Discussion and Robustness}\label{robustness}

\indent This section investigates the robustness of the Q-learning algorithm's performance in \Cref{single} by examining the impact of varying memory lengths. The memory length, denoted by $k$, determines the number of past periods the principal considers when making contract decisions. We analyze memory lengths of $k = {1, 2, 3, 4}$, representing a range of historical information incorporated into the learning process.

Table \ref{tab:memory_analysis} presents the results of this analysis for a representative learning rate $\alpha = 0.1$ and exploration rate $\beta = 5 \times 10^{-6}$. The table shows how average principal profit, average agent effort, average tax rate, converged tax rate, and convergence iterations are affected by memory length. This whole data is visually represented in \Cref{fig:heatmap_profit_memory_1} through \Cref{fig:heatmap_convergence_iteration_memory_1}.

\begin{table}[ht!]
\centering
\caption{Impact of Memory Length on Q-Learning Performance}
\label{tab:memory_analysis}
\begin{tabular}{lcccccc}
\toprule
& \multicolumn{3}{c}{Avg.} & \multicolumn{3}{c}{Conv.} \\
\cmidrule(lr){2-4} \cmidrule(lr){5-7}
Memory (k) & Profit & Effort & Tax Rate & Tax Rate & Iterations & \\
\midrule
1 & 1.0808 & 0.2498 & 0.5100 & \textbf{0.520} & 250 & \\
2 & 1.0818 & 0.2501 & 0.5050 & \textbf{0.515} & 275 & \\
3 & 1.0821 & 0.2503 & 0.5020 & \textbf{0.510} & 290 & \\
4 & 1.0822 & 0.2504 & 0.4994 & \textbf{0.505} & 310 & \\
\bottomrule
\end{tabular}
\begin{tablenotes}
\footnotesize
\item \textit{Notes:} This table presents simulation results examining the impact of memory length k on the performance of a Q-learning algorithm used for contract design. Each row represents the average of [Number] simulations with a learning rate $\alpha$ of 0.1 and an exploration rate $\beta$ of 5 $\times$ 10$^{-6}$. "Avg." denotes average values over all simulations, "Conv." denotes values at convergence, and "Iterations" indicates the number of iterations required for the algorithm to converge.
\end{tablenotes}
\end{table}

\subsection{Impact on Principal Profit}

\begin{figure}[ht!]
\centering
\includegraphics[width=1\textwidth]{results/robustness/heatmap_average_profit.png}
\caption{Average principal profit as a function of learning rate $\alpha$, exploration rate $\beta$, and memory length k. Higher values (warmer colors) indicate greater profitability. }
\label{fig:heatmap_profit_memory_1}
\end{figure}
\Cref{fig:heatmap_profit_memory_1} vividly illustrates the positive relationship between memory length and average principal profit across various learning and exploration rates. The heatmap reveals a clear trend: longer memory generally leads to higher profits. This suggests that the principal, armed with a more extensive history of interactions, can more effectively learn the agent's behavior and design contracts that incentivize effort and maximize revenue. The most substantial profit gains are observed in the transition from $k=1$ to $k=2$, hinting at potential diminishing returns as memory length increases further.

\subsection{Tax Rates and Agent Effort}

\begin{figure}[ht!]
\centering
\includegraphics[width=1\textwidth]{results/robustness/heatmap_average_effort.png}
\caption{Average agent effort as a function of learning rate ($\alpha$), exploration rate $\beta$, and memory length k. Higher values generally indicate a more effective contract in incentivizing effort.}
\label{fig:heatmap_effort_memory_1}
\end{figure}

Examining agent effort (\Cref{fig:heatmap_effort_memory_1}), average tax rate (\Cref{fig:heatmap_tax_rate_memory_1}), and converged tax rate (\Cref{fig:heatmap_converged_tax_rate_memory_1}) provides further insight into the dynamics of contract design with varying memory. 

\begin{figure}[ht!]
\centering
\includegraphics[width=1\textwidth]{results/robustness/heatmap_average_tax_rate.png}
\caption{Average tax rate imposed by the principal, influenced by learning rate $\alpha$, exploration rate $\beta$, and memory length k. Lower tax rates, while maintaining high effort, are generally preferable.}
\label{fig:heatmap_tax_rate_memory_1}
\end{figure}

\Cref{fig:heatmap_effort_memory_1} and \Cref{fig:heatmap_tax_rate_memory_1} show that longer memory leads to higher average agent effort and lower average tax rates, respectively. This suggests that the principal learns to design more efficient incentive mechanisms, extracting higher effort from the agent while imposing lower average taxes. 

\begin{figure}[ht!]
\centering
\includegraphics[width=1\textwidth]{results/robustness/heatmap_converged_tax_rate.png}
\caption{Converged tax rate set by the principal, as affected by learning rate $\alpha$, exploration rate $\beta$, and memory length k. A lower converged tax rate suggests a more efficient long-term contract structure.}
\label{fig:heatmap_converged_tax_rate_memory_1}
\end{figure}

\Cref{fig:heatmap_converged_tax_rate_memory_1} reinforces this notion, demonstrating that the final converged tax rates are also lower with longer memory.

\subsection{Convergence Speed}
Finally, \Cref{fig:heatmap_convergence_iteration_memory_1} addresses the computational cost associated with memory length. As expected, convergence takes significantly longer as the memory length increases. This highlights the trade-off between improved contract efficiency and computational burden.
\begin{figure}[ht!]
\centering
\includegraphics[width=1\textwidth]{results/robustness/heatmap_convergence_iteration.png}
\caption{Number of iterations required for algorithm convergence, influenced by learning rate $\alpha$, exploration rate $\beta$, and memory length k. Lower iteration counts (cooler colors) represent faster convergence.}
\label{fig:heatmap_convergence_iteration_memory_1}
\end{figure}

\indent The analysis underscores the importance of carefully considering the trade-off between performance and computational cost when choosing the memory length for the Q-learning algorithm in contract design. Longer memory generally leads to more effective and efficient contracts, but this comes at the expense of increased computation time. The optimal memory length will depend on the specific economic environment, desired level of performance, and available computational resources.
\section{Lyapunov Conditions for $\delta$-robustness}
\label{sec:main}

In this section we present the main theoretic result: Lyapunov conditions for the $\delta$-robustness of periodic orbits.  These conditions, and constructions, follow naturally from the ISS perspective employed in defining $\delta$-robustness.  But care is needed given the complexity of the Poincar\'{e} map.  Importantly, these conditions will lead to an approach for the verification of $\delta$-robustness, as presented in the next section. 


\begin{definition}
Consider the discrete-time  dynamical system in \eqref{eqn:Pdelta}.  A function $V : B_{\rho}(x^*) \to \R_{\geq 0}$, for $B_{\rho}(x^*)$ as in \eqref{eqn:extendedtimetoimpact}, is a \textbf{robust Lyapunov function} if:  
\begin{align}
\label{eqn:lyap1}
       k_1 & \| x - x^* \|^c   \leq  V(x)   \leq k_2 \|   x - x^* \|^c  \\
\label{eqn:lyapimplication}
   \| x  - x^* \|  \geq  \chi d & \quad \implies \quad \\
  \Delta V(x,d) &  :=  V(\P(x,d)) - V(x)  \leq - k_3 \| x - x^* \|^c \nonumber
\end{align}
for $\chi, k_1,k_2,k_3, c > 0$ and all $x \in B_{\rho}(x^*)$.  
\end{definition}

\begin{remark}
Note that \eqref{eqn:lyapimplication} can be equivalently restated as: 
\begin{eqnarray}
   V(\P(x,d)) - V(x)    \leq - k_4 \|  x - x^*  \|^c  + \frac{1}{2} \sigma |d|^c,
  \label{eqn:lyap2}
   \end{eqnarray}
 where $\sigma > 0$.  In particular, the corresponding quantities are related via: $k_3 = \frac{1}{2} k_4$ and $\chi = k_4^{-\frac{1}{c}} \sigma^{\frac{1}{c}}$. 
\end{remark}




\newsec{Main result}  We can now state the main result of the paper.  To do so, recall that a 
\emph{Lyapunov sublevel set} is given by: 
\begin{eqnarray}
\Omega_{r} = \{ x \in \R^n ~ | ~
V(x) \leq r\}. 
\end{eqnarray}
This will be essential in establishing: 


\begin{theorem}
\label{thm:main}
Consider the discrete-time dynamical system $x_{k+1} = \P(x_k,d_k)$ in \eqref{eqn:Pdelta} with associated periodic orbit $\O$.  If there exists a robust Lyapunov function, $V : B_{\rho}(x^*) \to \R_{\geq 0}$, and:
\begin{eqnarray}
\label{eqn:deltabound}
\delta < \delta_{\max} := \left(\frac{k_1}{\chi^c k_2}\right)^{\frac{1}{c}} \rho,
\end{eqnarray}
then the periodic orbit $\O$ is $\delta$-robust with: 
\begin{align}
 & \qquad   W =   \Omega_{r(\delta)},  \quad \mathrm{for} \quad r(\delta) := k_2 (\chi \delta) ^c  \\
\gamma = &\left( \frac{k_2}{k_1} \right)^{\frac{1}{c}} \chi,   \quad 
M= \left(\frac{k_2}{k_1}\right)^{\frac{1}{c}}, \quad 
\alpha = \left( 1 - \frac{k_3}{k_2}\right)^{\frac{1}{c}}. \nonumber
\end{align}
\end{theorem}


This theorem is, overall, a variation on Lemma 3.5 in \cite{jiang2001input}.  The proof here follows a similar overall arc, although there are key differences made necessary by the fact that $\P$ is only a partial function. 
This motivates the first Lemma. 

\begin{lemma}
The function $\P: B_{\rho}(x^*) \times [-\delta,\delta] \to S_{[-\delta,\delta]}$ given in \eqref{eqn:Pdelta} is well-defined for all $x \in  B_{\rho}(x^*) $, i.e., for all $x \in B_{\rho}(x^*) $, $\P(x,d)$ exists and satisfies $\P(x,d) \in S_{[-\delta,\delta]}$.
\end{lemma}

\begin{proof}
By the construction of the extended Poincar\'e map, $P_0$ is well-defined on $B_{\rho}(x^*)$, i.e., for all $x \in B_{\rho}(x^*)$ it follows that $\P(x,0) \in S_0$, i.e., $h(\varphi_{T_e(x,0)}(\Delta(x))) = 0$.  Therefore:
$$
h(\varphi_{t}(\Delta(x))) = \int_{T_e(x,0)}^t 
\dot{h}(\varphi_{\tau}(\Delta(x))) d \tau.
$$
But $\dot{h}(x) < 0$ for all $x \in  S_{[-\delta,\delta]}$ by definition.  Therefore, on the closed set defined by $- \delta \leq h(x) \leq \delta$, $\dot{h}$ takes a minimum and maximum value: $\underline{h} < \overline{h} < 0$.  This implies that: 
$$
\underline{h} (t - T_e(x,0))  \leq h(\varphi_{t}(\Delta(x))) \leq \overline{h}  (t - T_e(x,0)).
$$
Thus, there exists a $t$ (possibly negative) such that $h(\varphi_{t}(\Delta(x))) = d$. This $t = T_e(x,d)$. 
\end{proof}

Since $\P$ is well-defined, we can now find a set such that $x_{k+1} = \P(x_k,d_k)$ is defined for all $k$, i.e., a forward invariant set contained in $B_{\rho}(x^*)$, using Lyapunov sublevel sets. 


\begin{lemma}
\label{lem:levelset}
If $\delta < \delta_{\max}$, with $\delta_{\max}$ in \eqref{eqn:deltabound}, 
then for $r(\delta) := k_2 (\chi \delta) ^c$ it follows that:
$$
B_{\chi \delta}(x^*) \subset \Omega_{r(\delta)} \subset B_{\rho}(x^*).
$$
Moreover, the set $\Omega_{r(\delta)}$ is forward invariant. 
\end{lemma}

% Let $r(\rho) > 0$ such that $r(\rho) < k_1 \rho^c$ which yields:
% $$
% V(x) \leq r(\rho) \quad \Rightarrow \quad 
% k_1 \| x - x^* \|^c \leq V(x)  \leq r(\rho) <  k_1 \rho^c 
% % \quad \implies \quad 
% % \| x - x^* \| < \rho
% % \quad \Rightarrow \quad 
% % \Omega_{r(\rho)} \subset B_{\rho}(x^*)
% $$
% And therefore: $\Omega_{r(\rho)} \subset B_{\rho}(x^*)$.   This leads to: 


% \begin{lemma}
% \label{lem:levelset}
% If 
% \begin{eqnarray}
% \label{eqn:deltabound}
% \delta < \delta_{\max} := \left(\frac{k_1}{\chi k_2}\right)^c \rho
% \end{eqnarray}
% then for $r(\delta) := k_2 (\chi \delta) ^c$ it follows that:
% $$
% B_{\chi \delta}(x^*) \subset \Omega_{r(\delta)} \subset B_{\rho}(x^*)
% $$
% Moreover, the set $\Omega_{r(\delta)}$ is forward invariant. 
% \end{lemma}

% This lemma gives an upper bound on the robustness of a given periodic orbit $\O$ based upon the domain of definition of $\P$; namely, $\delta_{\max}$.  

\begin{proof}
For $x \in B_{\chi \delta}(x^*)$: 
$$
\| x - x^* \| < \chi \delta ~  \Rightarrow  ~ 
V(x) \leq k_2 \| x - x^* \|^c < k_2 (\chi \delta)^c = r(\delta)
$$
and therefore $B_{\chi \delta}(x^*) \subset \Omega_{r(\delta)}$.  Now if $r(\delta) < k_1 \rho^c $ (which is equivalent to the condition \eqref{eqn:deltabound}) it follows that:
$$
V(x) \leq r(\delta) \quad \Rightarrow \quad 
k_1 \| x - x^* \|^c \leq V(x)  \leq r(\delta) <  k_1 \rho^c 
% \quad \implies \quad 
% \| x - x^* \| < \rho
% \quad \Rightarrow \quad 
% \Omega_{r(\rho)} \subset B_{\rho}(x^*)
$$
And therefore: $\Omega_{r(\delta)} \subset B_{\rho}(x^*)$.   
% But the condition that $r(\delta) < k_1 \rho^c $ is equivalent to \eqref{eqn:deltabound}. 
%
% \blue{
% \begin{align*}
%     r(\delta) := k_2(\chi \delta)^c &< k_1 \rho^c \\
%     \delta^c &< \frac{k_1}{\chi^{c}k_2}\rho^c \\
%     \delta &< \left(\frac{k_1}{\chi^c k_2}\right)^{\frac{1}{c}} \rho 
% \end{align*}
% }
%
Finally, since for $\delta < \delta_{\max}$ we have $B_{\chi \delta}(x^*) \subset \Omega_{r(\delta)}$, it follows that on the boundary of $\Omega_{r(\delta)}$, namely $\partial \Omega_{r(\delta)}$, condition \eqref{eqn:lyapimplication} is active and therefore: $\Delta V(x,d) < 0$.  The forward invariance of $\Omega_{r(\delta)}$ follows. %: $x_k \in V_{r(\delta)}$ for all $k \in \N_{\geq 0}$.  
\end{proof}

Lemma \ref{lem:levelset} gives an upper bound on the $\delta$-robustness of a given periodic orbit $\O$, namely $\delta_{\max}$, based upon the domain of definition of $\P$.  It also establishes the forward invariance of  $\Omega_{r(\delta)}$.  
Leveraging this, we can prove the main result. 

\begin{proof}[Proof of Theorem \ref{thm:main}]
Let $x_0 \in \Omega_{r(\delta)}$, wherein the forward invariance of $\Omega_{r(\delta)}$ (Lemma \ref{lem:levelset}) implies $x_k \in \Omega_{r(\delta)} \subset B_{\rho}(x^*)$ for all $k \in \N_{\geq 0}$.  Thus both $\P$ and $V$ are well-defined.
%, and the conditions \eqref{eqn:lyap1} and \eqref{eqn:lyapimplication} hold. 
We consider two cases: $x_0 \notin B_{\chi \delta}(x^*)$ and $x_0 \in B_{\chi \delta}(x^*)$. 


\vspace{0.1cm}
\underline{$\| x_0 - x^* \| \geq \chi \delta$:}  In this case the implication \eqref{eqn:lyapimplication} is active: 
$$
\Delta V \leq - \frac{k_3}{k_2} V \quad \implies \quad
V(x_k) \leq \left( 1 - \frac{k_3}{k_2}\right)^k V(x_0)
$$
where the implication follows from applying the inequality on the right recursively (see also the comparison lemma \cite{jiang2002converse}).  Therefore, using the inequalities in \eqref{eqn:lyap1} we have: 
\begin{eqnarray}
\label{eqn:Malpha}
\| x_k - x^* \| \leq  \underbrace{\left( \frac{k_2}{k_1} \right)^{\frac{1}{c}}}_{M} \underbrace{\left( 1 - \frac{k_3}{k_2}\right)^{\frac{k}{c}}}_{\alpha^k} \| x_0 - x^* \|.
\end{eqnarray}
Finally, note that $k_3/k_2 < 1$ as otherwise $V(x_k)$ would be negative for $k = 1$ which is impossible.  Therefore, $\alpha < 1$. 
%This also implies that $\exists K$ such that $x_K \in B_{\chi \delta}(x^*)$. 

\vspace{0.1cm}
\underline{$\| x_0 - x^* \| < \chi \delta$:}  While the implication in \eqref{eqn:lyapimplication} no longer holds, we still have $x_k \in \Omega_{r(\delta)}$.  As a result: 
\begin{align}
k_1 \| x_k - x^* \|^c \leq  V(k_k) \leq  r(\delta) =  & k_2 (\chi \delta) ^c 
  \nonumber\\
  \label{eqn:gammaeqn}
 \quad \implies \quad \| x_k - x^* \| \leq &  \underbrace{\left( \frac{k_2}{k_1} \right)^{\frac{1}{c}} \chi}_{\gamma} \delta 
\end{align}
Therefore, for $M$, $\alpha$ in \eqref{eqn:Malpha} and $\gamma$ in \eqref{eqn:gammaeqn} we have:
\begin{eqnarray}
\| x_k - x^* \| & \leq  & \max \{ M \alpha^k \| x_0 - x^* \|, \gamma \delta \} \nonumber\\
& \leq &  M \alpha^k \| x_0 - x^* \| + \gamma \delta \nonumber
\end{eqnarray}
as desired, i.e., $\delta$-robustness is established with $W =  \Omega_{r(\delta)}$ the required forward invariant set.  
% We first note that $V_{r(\delta)}$ is forward invariant.  In particular, for $\delta < \delta_{\max}$ we have $B_{\chi \delta}(x^*) \subset V_{r(\delta)}$.  Thus, on the boundary $\Delta V(x,d) < 0$ and forward invariance follows. 
\end{proof}

% we assume the reader is familiar with this proof, although we still attempt to make the proof self contained to note the key differences---which stem from the fact that $\P$ is only a partial function. 



% \begin{lemma}
%     \label{lemma:forwadinvariance}
%     Consider the discrete-time system \eqref{eq: discretetime}. If $B_{\delta}(x^*)$ is forward invariant: 
%     \begin{eqnarray}
%     \label{eqn:forwardsetinvariance}
%     x_0 \in B_{\delta}(x^*) \quad \implies  \quad 
%     x_{k+1} = P_{d_k}(x_k) \in B_{\delta}(x^*) 
%     \end{eqnarray}
%     for all $x \in \N_{\geq 0}$ and for every sequence $\{d_k\}_{k \in \N_{\geq 0}}$ with $d_k \in [d^-_{\delta},d^+_{\delta}]$, then $\O$ is $\delta$-robust. 
% \end{lemma}

% \begin{proof}
% Assume that $\O$ is not $\delta$-robust.  Then there exists an $x_0 \in B_{\delta}(x^*)$ and a $d_0 \in [d^-_{\delta},d^+_{\delta}]$ such that $P_{d_0}(x_0) \notin B_{\delta}(x^*)$.  This contradicts the implication in \eqref{eqn:forwardsetinvariance}. 
% \end{proof}



% \newsec{Connections with Input-to-state Stability}
% Note that beyond simply finding the $\delta$-robustness of a gait, it may be desirable to ensure exponential stability to $B_{\delta}(x^*)$.  This leads to the following definition: 

% \begin{definition}
% The periodic orbit $\O$ is \textbf{$\delta$-robustly exponentially stable} if it is $\delta$-robust and there exists an open and connected set $W$ with $B_{\delta}(x^*) \subset W $ such that:
% \begin{align}
%   \forall ~ x_0 \in W   &  \quad \implies \quad  \nonumber\\
%  & \| x_k \|_{B_{\delta}(x^*)} \leq M\alpha^k \|x_0 \|_{B_{\delta}(x^*)},  \quad \forall k \in \N_{\geq 0} \nonumber
% \end{align}
% % \begin{align}
% %   \forall ~ x \in B_{\delta + \epsilon}(x^*) \backslash B_{\delta}(x^*)  &  \quad \implies \quad  \nonumber\\
% %  & \| P^i_d(x) \|_{B_{\delta}(x^*)} \leq M\alpha^i \|x \|_{B_{\delta}(x^*)}, \nonumber
% % \end{align}
% for $M > 0$ and $\alpha \in (0,1)$. 
% \end{definition}




% \begin{theorem}
% The periodic orbit $\O$ is $\delta$-robustly exponentially stable if and only if the dynamical system in \eqref{eq: discretetime} is exponentially input-to-state stable \eqref{eqn:expiss}.
% \end{theorem}
\section{Formal Verification and Granularity Tuning} 
\label{4}

\begin{algorithm}[t]
    \renewcommand{\thealgocf}{3}
	\SetKwData{Left}{left}\SetKwData{This}{this}\SetKwData{Up}{up}
	\SetKwFunction{Union}{Union}\SetKwFunction{FindCompress}{FindCompress}
	\SetKwInOut{Input}{Input}\SetKwInOut{Output}{Output}
	\caption{\mbox{Training with Granularity Tuning.}}
	\label{refine_alg}
	\Input{~$\mathbf{X}$: training data; ~$\mathbf{X_{test}}$: test data;\\ ~$\mathbf{Y}$: ground-truth labels of training data;\\ ~$\mathbf{Y_{test}}$: ground-truth labels of test data;\\ ~$\epsilon$: perturbation radius;\\ ~$d_{step}$: step size of abstraction granularity}
	\Output{$f_{out}$: verified robust neural network;\\
	$d_{out}$: the best abstraction granularity}
	Initialize $d,error,e, f_{out}, d_{out}$\tcp*{$e:$ verified error.}
	\While{$d\geq 2*\epsilon$}{
	    $(f,\ell)\leftarrow$  \textsc{AbsTrain}($\mathbf{X}$, $\mathbf{Y}$, $\epsilon$, $d$)\tcp*{$\ell$: training loss.} 
	    
	    $e\leftarrow $ \textsc{Verify}($f$, $\mathbf{X_{test}}$,  $\mathbf{Y_{test}}$, $\epsilon$, $d$)\tcp*{Verification.} 
	    
	    \If{$e < error$}{
	        $error\leftarrow e$; \\
	        $f_{out}\leftarrow f$;\\
	        $d_{out}\leftarrow d$; 
	    }
	    
	    $\mathbb{I}$ $\leftarrow \Phi(\mathbf{X}$, $\epsilon$, $d$) \tcp*{Map to training interval.} 
	    
	    $\overline{G}, \underline{G}\leftarrow\ell'(\overline{\mathbb{I}}),\ell'(\underline{\mathbb{I}}) $\tcp*{Obtain bounds' gradient,}
	    
	    % $\underline{G}\leftarrow \ell'(\underline{\mathbb{I}})$\tcp*{$\underline{G}$: The lower bounds' gradient.}
	    
	    \If{$\overline{G}\leq 0 \wedge \underline{G}\geq 0$}{
	        Break\tcp*{Stop when G guides d to increase} 
	    }
	    \Else{
	        $d \leftarrow  d - d_{step}$\tcp* {Decrease d and continue}
	    }
	}
	Return $f_{out}, d_{out}$;
\end{algorithm}

In this section we introduce a verification-based method for 
tuning the abstraction granularity to train 
% verified 
robust models. Due to the finiteness of $\mathbb{I}^n$, verification procedure can be conducted in a black-box manner, which is both sound and complete. Based on verification results, we can 
tune the abstraction granularity to obtain finer training intervals.
% set for more robust training. 

% \vspace{-3mm}
\subsection{Black-box Robustness Verification}

We propose a black-box verification method $\textsc{Verify}(\cdot)$ for neural networks trained by our approach. Given a neural network $f$, a test set $\mathbf{X_{test}}, \mathbf{Y_{test}}$, a perturbation distance $\epsilon$ and an abstraction granularity  $d$, $\textsc{Verify}(\cdot)$ returns the verified error on the set. 
The verification procedure is straightforward. First, for each $x\in \mathbf{X_{test}}$, we compute the set $\mathbb{I}$ of training interval vectors of $\mathbb{B}(x,\epsilon)$ using the same abstraction function $\phi$ in  Algorithm \ref{abstraction_alg}.  
Then, we feed each interval vector in $\mathbb{I}$ into $f$ and check if the classification result is consistent with the ground-truth label of $x$. The verified error $e$ is the ratio of the inconsistent cases in the test set. 

% \vspace{1ex}
Our verification method is both sound and complete due to the finiteness of $\mathbb{I}$: 
$f$ is robust on $\mathbb{B}(x,\epsilon)$ if and only if $f$ returns the same label on all the interval vectors in   $\mathbb{I}$ as the one of $x$. Another advantage 
% of our method 
is that it treats $f$ as a black box. Therefore, our verification method is orthogonal and scalable to arbitrary models.  

\subsection{Tuning Abstraction Granularity} \label{3.4}

When the verified error of a trained neural network is large, we can reduce it by tuning the abstraction granularity and re-train the model on the refined training set of interval vectors. We propose a gradient descent-based 
algorithm to explore the best $d$ in the abstraction space. 

Algorithm \ref{refine_alg} shows the tuning and re-training process. First, our algorithm initializes abstraction granularity $d$ and verified error $error$ (Line 1). The tuning process repeats until $d < 2\epsilon$, which means that the size of a training interval is not less than that of the perturbed interval (Lines 2-15). Then a neural network and training loss toward training intervals with $d$ are obtained by calling the Abstraction-based Training function (Algorithm \ref{train_alg}) (Line 3). Get the neural network's verified errors on the test dataset (Line 4). If the neural network's verified errors are smaller, we save the neural network and abstraction granularity (Lines 5-8). Next, training intervals are obtained (Line 9). Then the upper and lower bounds' gradient of the training intervals are obtained (Lines 10). 
If $\overline{G} \leq 0$ and $\underline{G} \geq 0$, the algorithm will terminate (Lines 11-12); otherwise, abstraction granularity will be updated to a smaller granularity (Lines 13-14). 
When the algorithm terminates, we obtain the neural network with the lowest verified error.

%%%%%%%%%%%%%%%%%%%%%%%%%%%%%%%%%%%%%%%%%%%%%%%%%%%%%%%%%%%%%%%%%%%%%%%%%%%%%%%%

\section{Conclusion}
In this work, a novel notion of robustness, \emph{$\delta$-robustness}, was formulated from the perspective of input-to-state stability.
% This definition differs from classic notions of stability by instead quantifying the magnitude of perturbations a periodic orbit can withstand while remaining stable. 
Lyapunov conditions were also derived to certify $\delta$-robustness for a nominal periodic orbit. Future work includes directly evaluating $\delta$-robustness in the gait generation process to systematically generate periodic orbits that are robust to uncertain terrain. Additionally, sampling methods can be leveraged to obtain probabilistic guarantees on $\delta$-robustness. Lastly, the discrete-time Lyapunov condition can be translated to a stochastic condition in order to obtain more realistic (albeit probabilistic) estimates of the $\delta$-robustness. 

    
 % In this work, a novel notion of robustness is formulated, termed $\delta$-robustness, to characterize the robustness of a nominal periodic orbit to uncertain terrain by treating perturbations in ground height as disturbances in the context of input-to-state-stability (ISS). The main theoretic result is the construction of robust Lyapunov functions to certify $\delta$-robustness of periodic orbits as well as the synthesis of an optimization framework for verifying $\delta$-robustness. These constructions are demonstrated in simulation with a seven-link bipedal robot.
 
%%%%%%%%%%%%%%%%%%%%%%%%%%%%%%%%%%%%%%%%%%%%%%%%%%%%%%%%%%%%%%%%%%%%%%%%%%%%%%%%
% \section{ACKNOWLEDGMENTS}
% The authors gratefully acknowledge the contribution of National Research Organization and reviewers' comments.
%%%%%%%%%%%%%%%%%%%%%%%%%%%%%%%%%%%%%%%%%%%%%%%%%%%%%%%%%%%%%%%%%%%%%%%%%%%%%%%%

\bibliographystyle{IEEEtran}
\bibliography{./Bibliography/IEEEabrv, ./Bibliography/References}
\addtolength{\textheight}{-3cm}   % This command serves to balance the column lengths
                                  % on the last page of the document manually. It shortens
                                  % the textheight of the last page by a suitable amount.
                                  % This command does not take effect until the next page
                                  % so it should come on the page before the last. Make
                                  % sure that you do not shorten the textheight too much.

% \new{\appendix}



\end{document}
