\newpage
\section{Motivation for Set-Theoretic Perspective}
% Because of the ability to analyze the stability of periodic orbits, to date researchers have primarily described walking as a periodic orbits (or a family of periodic orbits). However, these tools breakdown when uncertain terrain and perturbations require the walking to be no longer be periodic. We will present an example to motivate this.

\begin{figure}[tb]
    \centering
    \includegraphics[width=\linewidth]{Figures/motivation_figure.pdf}
    \caption{Demonstration of a periodic orbit with $\lambda_{\max}(\frac{\partial P}{\partial x} (x^*)) > 0$ (illustrated on the left) not being robust to variations in the guard condition (for this example, the ground height was increased by 1cm), while a periodic orbit with a larger $\lambda_{\max}$ (illustrated on the right) is comparatively more robust. This motivates the need for a set-theoretic perspective towards robustness of periodic orbits to uncertain guard conditions.}
    \label{fig: motivation}
\end{figure}

Practically, the stability of periodic orbits can be analyzed by evaluating the eigenvalues of the Poincar\'e return map linearized around the fixed point. Specifically, if the magnitude of the eigenvalues of $\frac{\partial P}{\partial x}(x^*)$ is less than one (i.e. $\lambda_{\max} \left (\frac{\partial P}{\partial x}(x^*)\right) < 1$), then the fixed point is stable \cite{perko2013differential,morris2005restricted,wendel2010rank}. 
% Note that in this process, the eigenvalues corresponding to states known to be nonperiodic, such as the horizontal position of the floating base frame. 
Since it is often difficult to compute the Poincar\'e map analytically, it is commonly numerically approximated. %by applying small perturbations to each state and forward simulating one step to obtain $P(x^*+\delta )$ which is then used to construct each row of the Jacobian successively.

% \begin{lemma}
%     Let $\O$ be a periodic orbit with fixed point $x^* \in S$. If $\lambda_{\max}(J_P) < 1$, then the $\O$ is contracting and stable.
% \end{lemma}
    
\begin{remark}
    Orbits $\O$ with lower $\lambda_{\max}$ are regarded as being \textit{more stable} than orbits with higher $\lambda_{\max}$ since they are more contractive around the fixed point.
\end{remark}

However, since the Poincar\'e map is evaluated at the fixed point, this notion of stability is only valid locally. This means that while this tool is useful for checking the stability of periodic orbits, it is no longer insightful for systems with uncertain guard events or disturbances that push the system away from the fixed point. 

\subsection{Simulation Example}
\blue{Take two gaits, with one gait having a smaller eigenvalues than the other. Show that in the presence of uncertain terrain, the "more stable gait" fails, while the other one doesn't.}

As illustrated in Figure \ref{fig: motivation}, the classic Poincar\'e analysis does not accurately reflect the robustness of periodic orbits to local disturbances in the guard condition. In this work, we will instead propose tools for discussing the stability of walking (described by either periodic and nonperiodic orbits) from a set-theoretic perspective. This will then allow us to introduce tools for analyzing the stability of walking away from the fixed point.

