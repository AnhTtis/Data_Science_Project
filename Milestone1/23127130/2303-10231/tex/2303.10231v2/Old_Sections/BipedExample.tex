\newpage
\section{Synthesizing a $\delta$-Robust Periodic Orbit for a Bipedal Robot}

\begin{figure}[tb]
    \centering
    \includegraphics[width=\linewidth]{Figures/uncertain_guards3.pdf}
    \caption{\textbf{Illustration of Uncertain Guard Condition for Five-Link Walker.} The nominal level-ground guard conditions is shown on the left along with the configuration coordinates of the five-link walker model used in this work. In comparison, the proposed guard condition for uncertain terrain of height $\gamma$ at impact is illustrated on the right.}
    \label{fig: uncertainguard}
    \vspace{-2mm}
\end{figure}


\red{Through the construction of B\'ezier polynomials, we could ensure that the control points between $\{\varphi_{T_e(x_k,d^-)}(x_k), \varphi_{T_e(x_k,d^+)}(x_k) \}$ lie inside of $\B_{\delta}(x^*)$}

As described in \cite{grizzle2014models}, a bipedal robot is classically modeled as a tree structure composed of rigid links by constructing a floating-base Lagrangian model and adding holonomic constraints to enforce the ground contact forces. Towards this, we will denote the floating-base reference frame of the bipedal robot as $R_b := [p,\phi]^{\top}$ with $p \in \R^3$ denoting the Cartesian position of the body-fixed reference and $\phi \in SO(3)$ denoting the orientation with respect to the world frame. The generalized configuration coordinates of the robot are then denoted as:
$$ q := (p,\phi,q_b) \in \Q = R^3 \times SO(3) \times D_b,$$
with $q_b \in \R^n$ being the body coordinates, representing the positions of $n \in \Z^+$ actuated joints. The state of the system is then denoted by $x := [q,\dot{q}]^{\top} \in \Q \times T\Q = R^{2n}$.

Next, the continuous dynamics of the system can be obtained via the Euler-Lange equation and holonomic constraints:
\begin{align}
    D(q)\ddot{q} + H(q,\dot{q}) = Bu + J_h^{\top}(q)F,
    \label{eq: Euler-Lagrange}
\end{align} 
where $D: \R^n \to \R^{n \times n}$ is the inertia matrix, $H(q,\dot{q}) := C(q,\dot{q})\dot{q} + G(q)$ is the vector containing the Coriolis and Gravity terms, $B: \Q \to \R^n$ is the actuation matrix, $u \in \R^m$ is the control input, $J_h(q) := \frac{\partial \nu(q)}{\partial q} \in \R^{J \times n}$ is the Jacobian of the holonomic constraints with $\nu(q) \in \R^h$ being the $h \in \Z^+$ holonomic constraints (stance foot position for robot with point-foot contacts; stance foot position and orientation for robot with flat-foot contacts). Lastly, $F \in \R^h$ is the contact wrenches which can be computed by solving the second order derivative of the holonomic constraints,
\begin{align}
    J_h(q)\ddot{q} + \dot{J}_h(q,\dot{q})\ddot{q}) = 0,    
\end{align}
and \eqref{eq: Euler-Lagrange} simultaneously. Next, we derive an impact map, $\Delta(q,\dot{q}) := [\Delta_q(q), \Delta_{\dot{q}}(q,\dot{q})]^{\top}$ based on the rigid impact model from \cite{hurmuzlu1994rigid} and assuming that the impacting foot neither slips nor rotates. Explicitly, the impact map is derived using the following system of equations (as in (27) of \cite{glocker1992dynamical}):
\begin{align}
    \begin{bmatrix}D(q) & -J_h^{\top}(q) \\ J_h(q) & 0_{h \times h}\end{bmatrix}\begin{bmatrix}\dot{q}^+ \\ F\end{bmatrix} = \begin{bmatrix} D \dot{q}^- \\ 0_{h \times n} \end{bmatrix}
\end{align}
Solving for the post-impact velocity, the resulting impact map is:
\begin{align}
    x^+ := (q^+,\dot{q}^+)^{\top} = \Delta(q^-,\dot{q}^-) = \begin{bmatrix} \Delta_q(q^-) \\ \Delta_{\dot{q}}(q^-,\dot{q}^-)   \end{bmatrix} \\
    ~= \begin{bmatrix}
        R q^- \\ (R - R(D^{-1}J_h^{\top}(J_hD^{-1}J_h^{\top})^{-1}J_h))\dot{q}^-
    \end{bmatrix},
\end{align}
with the arguments of $D(Rq^-)$ and $J_h(Rq^-)$ suppressed for ease of notation, and with $R \in \R^{n \times n}$ denoting the relabeling matrix (if no relabeling is required then $R = I_{n \times n}$).

One method of synthesizing periodic orbits $\O$ for bipedal robots is using the method of \textit{(Partial) Hybrid Zero Dynamics} \cite{ames2014human, westervelt2003hybrid}. This methods synthesizes output-level trajectories (often referred to as \textit{gaits}) which describe the desired behavior of the actuated coordinates of the robot (termed virtual constraints) such that when these outputs are enforced, the full nonlinear dynamics are rendered stable even in the presence of underactuation and impact events. Specifically, the virtual constraints, $y: \Q \to \R^m$, are defined as:
$$ y(q,\alpha) = y^a(q) - y^d(\tau(q),\alpha),$$
with $y^a(q) \in \R^m$ being the actual measured outputs and $y^d(\tau(q),\alpha) \in \R^m$ being the desired output parameterized by a monotonically increasing phasing variable $\tau: \Q \to \R$. Note that we choose specifically to parameterize $y^d$ using $N^{\text{th}}$ order B\'ezier polynomials, parameterized by the B\'ezier coefficients $\alpha \in \R^{m \times N}$. Using this paramterization, the virtual constraints can be shaped to satisfy an impact-invariance condition and stable closed loop dynamics via the trajectory optimization problem (as in \cite{tucker2021preference}):
\begin{ruledtable}
\vspace{-2mm}
{\textbf{\normalsize Trajectory Optimization:}}
\par\vspace{-4mm}{\small 
\begin{align*} \label{eq:opt}
   \{\alpha^*,X^*\} = \argmin_{\alpha,X} &~  \Phi(X) \\
    \text{s.t.}\quad 
    & \dot{x} = f_{cl}(x) \tag{Closed-loop Dynamics} \\
    & \Delta(\mathcal{S} \cap \mathcal{Z}_\alpha) \subset \mathcal{Z}_\alpha \tag{HZD Condition} \\
    % & y+K_py+K_s\dot y = 0 \tag{Dynamics Condition} \\
    & X_{\text{min}}  \preceq X \preceq X_{\text{max}} \tag{Decision Variables} \\
    & c_{\text{min}}  \preceq c(X) \preceq c_{\text{max}} \tag{Physical Constraints} 
    \label{eq: essential} 
    % &\text{(C1)~}\text{hybrid dynamics} ~\eqref{eq:EOM} \notag\\
    % &\text{(C2)~}%\text{HZD condition} \notag\\ 
    %     y^- = \dot y^- = y^+ = \dot y^+ = 0 \notag\\ 
    % &\text{(C2)~} \red{y+K_p y+ K_d\dot y= 0} \notag \\
    % &\text{(C3)~} % \text{physical feasibility}
    %     \mathcal{F}(q, \dot{q}) \preceq 0 \quad \forall t \notag\\ 
    % &\text{(C4)~} \text{essential constraints}  \notag 
\end{align*}}\vspace{-6mm}\par
\end{ruledtable}
\noindent where $X = (x_0,...,x_C,T)$ is the collection of all decision variables with $x_i$ the state at the $i^{th}$ discretization and $T$ the duration, $\Phi(X)$ is the cost function, and $c(X)$ is the set of physical constraints on the optimization problem. These physical constraints are included in every gait generation framework to encode the physical laws of real-word, such as the friction cone condition, workspace limit, and motor capacity \cite{reher2020algorithmic}.

 
%%%  SWITCH IN TOPIC...?
% Under Assumption 2, there is a unique mapping between $h(q)$ and $\tau(q)$ for states such that $\dot{h} < 0$. We will denote this mapping as $\Omega: \R \to \R$, with $$\Omega(d) := \{\tau(q) \mid h(q) = d, \dot{h}(q) < 0\}.$$

% \begin{theorem}
%     A periodic orbit $\O$ corresponding to the virtual constraints $y(q,\alpha^*)$ is Robust HSI to uncertain ground heights $d \in [-\gamma,\gamma]$ if $y^d(\Omega(\gamma),\alpha^*)$ and $y^d(\Omega(-\gamma),\alpha^*)$ ... 
% \end{theorem}


% \begin{lemma}
%     Given a biped with either point-foot contacts, or with flat-foot contacts and a swing-foot controller enforcing that the foot remains flat throughout the swing phase, then Assumption \ref{assumption: maxTreset} is true if $\forall q \in \varphi^*_t(x_0)$:
%     \begin{align}
%         \frac{\partial^2 J_h(q)}{\partial q^2} > 0, \\
%         \frac{\partial^2 D(q)}{\partial q^2} > 0
%     \end{align}
% \end{lemma}
% \begin{proof}
    
% \end{proof}

%%%  HOW TO SYNTHESIZE GAIT!
Let $\gamma > 0$ denote the largest ground height disturbance. Also, assume that the periodic orbit $\O$ satisfies the following assumption:

\begin{assumption}
    We assume that for $x_0 = \Delta(x^*)$, there exists some $\{T_h,h_{\max}\} \in \R^+$ such that the flow of the system $\varphi_t(x_0)$ evolves such that:
    \begin{small}
    \begin{numcases}{}
        h(\varphi_t(x_0)) < h_{\max}, ~\dot{h}(\varphi_t(x_0)) > 0 & $0 \leq t < T_h$, \notag \\
        h(\varphi_t(x_0)) = h_{\max}, ~\dot{h}(\varphi_t(x_0)) = 0 & $t = T_h$, \notag  \\
        h(\varphi_t(x_0)) < h_{\max}, ~\dot{h}(\varphi_t(x_0)) < 0 & $T_h < t < T_e(x_0,-\gamma)$, \notag \\
        h(\varphi_t(x_0)) = -\gamma, ~\dot{h}(\varphi_t(x_0)) < 0 & $t = T_e(x_0,-\gamma)$. \notag
    \end{numcases}
    \end{small}
    \label{assump: swingfoot}
\end{assumption}

\begin{figure}
    \centering
    \includegraphics[width=\linewidth]{Figures/swingfoot.pdf}
    \caption{Example Swing Foot Trajectory Satisfying Assumption \ref{assump: swingfoot}.}
    \label{fig: swingfootassump}
\end{figure}

Under this assumption, we can show that $\O$ is guaranteed to intersect the guard conditions $S_{\gamma},S_{-\gamma}$ while respecting ordered time-to-impacts: $T_e(\Delta(x^*),\gamma) \leq T_e(\Delta(x^*),0) \leq T_e(\Delta(x^*),-\gamma)$.

\begin{lemma}
    Let $H(x_0,-\gamma)$ be a trajectory of the continuous guard condition $h(x) \in \R$ that satisfies Assumption \ref{assump: swingfoot}. If $h_{\max} \geq \gamma$, then %if $h(\varphi_t(\Delta(x^*))) \in \R^+$ is monotonically decreasing for all $t \in [T_h, T_e(x_0,-\gamma)]$, and  then 
    $$T_e(\Delta(x^*),\gamma) \leq T_e(\Delta(x^*),0) \leq T_e(\Delta(x^*),-\gamma).$$ 
    \label{lemma: orderedtime}
\end{lemma}
\begin{proof}
    Per assumption \ref{assump: swingfoot}, the evaluation of the guard condition is $h(\varphi_t(x_0)) = \{h_{\max},-\gamma\}$ for $t = \{T_h, T_e(x_0,-\gamma)\}$. Additionally, we know that $h(\varphi_t(x_0))$ is monotonically decreasing in the interval $t \in (T_h,T_e(x_0,-\gamma))$ due to the assumption that in this range, $\dot{h}(\varphi_t(x_0)) < 0$. Thus, by the intermediate value theorem, we know that there must exist some $\{T_e(x_0,\gamma),T_e(x_0,0)\} \in (T_h,T_e(x_0,-\gamma))$ such that $\{h(\varphi_{T_e(x_0,0)}(x_0)),h(\varphi_{T_e(x_0,\gamma)}(x_0)) \} \in (h(\varphi_{T_h}(x_0)), h(\varphi_{T_e(x_0,-\gamma)}(x_0)))$. Additionally, since $h$ is monotonically decreasing in $t \in (T_h,T_e(x_0,-\gamma))$, then $h(\varphi_t(x_0)) = \gamma$ will occur at time $T_e(x_0,\gamma)$ before the height will reach $h(\varphi_t(x_0)) = 0$ at time $T_e(x_0,0)$. Thus, $T_e(x_0,\gamma) \leq T_e(x_0,0) \leq T_e(\Delta(x^*),-\gamma)$. 
\end{proof}

The disturbance to the pre-impact state caused by this guard uncertainty can be represented as:
\begin{align}
    P_{h_{\max}}(x) = P_0(x) + 
\end{align}