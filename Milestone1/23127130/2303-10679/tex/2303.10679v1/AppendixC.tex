\section{Appendix C - Preparation and positioning of Nanodiamonds using Pick-and-Place}
In preparation of transferring SiV hosting NDs to the bullseye antenna, a dispersion of SiV containing NDs was drop-casted onto a fused silica coverslip.
The NDs were fabricated using high-pressure high temperature (HPHT) treatment of the catalyst metals-free hydrocarbon growth system. The NDs were
treated with \ch{HNO3 + HClO4 + H2SO4} and HF to remove sp2 carbon and were
washed and dried afterwards. FIB milled marker structures on the fused silica coverslip serve as a position reference to later locate specific SiV-containing NDs
with an AFM. To find SiV-hosting NDs the sample is examined using a homebuilt confocal microscope, orchestrated based on the open-source software qudi \cite{Binder2017Qudi:Processing}. A 2D galvo scanning mirror, a 4f-system and a 1.35 NA oil objective form the basis of this optical setup. Narrowband filtering (740 ± 13 nm)
around the ZPL of the SiV increases the signal to noise ratio when exciting offresonantly with 532 nm, as the dominant emission originates from SiV centers
within this filter window. After suitable NDs have been located, AFM imaging
of the area of interest is carried out. Triangulation of the ND position is done
with the help of the FIB markers, as well as positions of other fluorescing NDs
in the confocal image. In order to pick and place \cite{Schell2011ADevices} the ND of interest, a platin-coated AFM cantilever is approached with a constant force until the ND attaches to the tip. The pick-up is indicated by either the disappearing ND in a subsequent non-contact AFM scan or by image artifacts such as ’double-tip’ features. Those features also hint towards a successful placement strategy of the ND in the following step. With the ND attached to the cantilever tip, the sample is exchanged for the target sample, here the bullseye antenna. A careful non-contact imaging reveals the target position on the structure. For the antenna structure considered herein the central hole in the structure eases the placement of the ND due to its topology. A contrast in height is useful to detach the ND from
the cantilever. After successful ND placement (compare Fig. \ref{fig:C}b), the sample is again examined in the confocal microscope, with the objective being exchanged to 0.55 NA for the sake of a longer working distance and to leverage the collection efficiency \cite{Waltrich2021High-purityAntenna}. The verification of a successful SiV hosting ND placement is seen in Fig. \ref{fig:C}c.

\label{AppendixC}