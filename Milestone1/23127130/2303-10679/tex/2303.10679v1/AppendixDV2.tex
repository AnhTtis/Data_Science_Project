\section{Appendix D - Estimation of in-fiber collection efficiency}

To estimate the in-fiber collection efficiency, we calculated the fraction of collected photons from all emitted photons using back excitation. First, using the back focal plane image(\ref{fig:E}.d), the relative collection efficiency into NA = 0.22 (which is the NA of our MM fiber) was calculated to set the upper limit of the achievable coupling using this device. 
\begin{equation}
   \frac{\eta_{NA = 0.22}}{\eta_{NA = 0.9}} = \frac{ \int_{0}^{\theta_{NA = 0.22}}d\theta sin(\theta) \int_{0}^{2\pi} I(\theta,\phi)d\phi}{\int_{0}^{\theta_{NA = 0.9}}d\theta sin(\theta) \int_{0}^{2\pi} I(\theta,\phi) d\phi}
\end{equation}
Where $\eta$ is the collection efficiency and $I(\theta,\phi)$ is the intensity in the back focal planee image. we find
\begin{equation}
   \eta_{NA = 0.22} = 0.32   \qquad  \eta_{NA = 0.9} = 0.91
\end{equation}
and thus,
\begin{equation}
    \frac{\eta_{NA = 0.22}}{\eta_{NA = 0.9}} = 0.3516 
\end{equation}
Hence, $35.19 \% $ of the light collected by our 0.9 NA objective is emitted at angles smaller than $\theta_{NA = 0.22}$. Next, to estimate the experimental coupling efficiency, we compared the counts obtained on a Pixis spectrometer camera for the whole emission spectral band using a collection with a 0.22 NA fiber and a 0.9 NA objective.
\begin{equation}
    \frac{I_{Fiber\: Coupled}}{I_{0.9 NA\: objective}} = \frac{\int_{\lambda = 550 nm}^{\lambda = 700 nm} I_{Fiber\: Coupled}(\lambda) d\lambda \cdot \gamma_{Fiber}^{-1}}{\int_{\lambda = 550 nm}^{\lambda = 700 nm} I_{0.9 NA\: objective}(\lambda) d\lambda \cdot \gamma_{Objective}^{-1}}
    \label{equation:fiberc}
\end{equation}
where the integral represents the area under the spectrum and $\gamma$ stands for the total measured loss in each collection setup (free space and fiber collection setups). The losses in the objective system are divided into two distinct types:
\begin{equation}
    \gamma_{objective} = \gamma_{system \: loses} \cdot \gamma_{spectrometer \: slit}
\end{equation}
where the losses in the system,$\gamma_{system \: loses}$, are losses caused by the optical system itself (partial transmission of mirrors and filters) and losses caused by the spectrometer's entrance slit,$\gamma_{spectrometer \: slit}$. Since the entrance slit is slightly smaller than the entire emission spatial beam, only a portion of the light that reaches the rectangular slit is transmitted. To calculate the fraction of transmitted light, we use the following expression:
\begin{equation}
    \gamma_{spectrometer \: slit} = \frac{\Gamma w}{\pi \cdot \Gamma^2}
    \label{equation:slit}
\end{equation}
Where $\Gamma$ is the FWHM of the Gaussian emission profile on the spectrometer camera along the slit axis, and $w$ is the slit width. To calculate the FWHM of the power distribution, we sum the intensities along the camera and not along the grating. As seen in the figure below \ref{fig:pixel} the FWHM is 44 pixels.
\begin{figure} [H] % opens the figure environment. the '[H]' forces the image to be Here
    \centering % puts the image in the horizontal center of the page
    \includegraphics[width = 0.7\textwidth]{Images/objective.jpg} 
    \caption{FWHM of the intensity distribution on the spectrometer camera}% this prints the caption below the figure
    \label{fig:pixel} % this internally labels the figure for future referencing.
\end{figure}
Since the pixel size is 82 microns, the FWHM is $2132 \unit{\mu m}$. By using the known width of the spectrometer slit which is $500 \unit{\mu m}$, we can substitute it into \ref{equation:slit}:
\begin{equation}
    \gamma_{spectrometer \: slit} = \frac{2132 \unit{\mu m} \cdot 500 \unit{\mu m}}{\pi \cdot (2132 \unit{\mu m})^2} \approx 0.074
\end{equation}

To determine the loss coefficient for the objective system, we used a He:Ne laser. By measuring the power before the objective and after all the mirrors and filters, we found that the measured power was $6.65 \unit{\mu W}$ and $17 \unit{\mu W}$, respectively. Therefore the loss coefficient for the objective system is approximately $\gamma_{system : loses} \approx 0.3911$. 

The losses in the fiber coupling system were calculated using the same He:Ne laser. The intensity of the laser was measured before the entrance to the fiber. After the filters on the other side of the fiber, the intensity before the entrance was $17 \unit{\mu W}$ and $14.6 \unit{\mu W}$ at the end of the system (after all of the mirrors and filters).
Hence, the loss of the fiber system was  $\gamma_{fiber \: loss} \approx 0.859$.
Here again, we must find the loss due to the spectrometer slit.
\begin{equation}
 \gamma_{Fiber} = \gamma_{fiber \: system} \cdot \gamma_{fiber \: spectrometer\:slit} 
\end{equation}
Again, we will find the FWHM of the intensity distribution on the camera. 
As it can be seen, the FWHM is 43 pixels\ref{fig:pixel2}; multiplying that by the size of the pixels, we get 1118 microns.
 
\begin{figure} [H] % opens the figure environment. the '[H]' forces the image to be Here
    \centering % puts the image in the horizontal center of the page
    \includegraphics[width = 0.7\textwidth]{Images/fiber.jpg} 
    \caption{fiber measurement: FWHM of the intensity distribution on the spectrometer camera}% this prints the caption below the figure
    \label{fig:pixel2} % this internally labels the figure for future referencing.
\end{figure}

\begin{equation}
        \gamma_{spectrometer \: slit} = \frac{\Gamma w}{\pi \cdot \Gamma^2}
        = \frac{500 \cdot 1118}{\pi \cdot (1118)^2} = 0.1423
    \label{equation:slit2}
\end{equation}
and hence,
\begin{equation}
    \gamma_{Fiber} = 0.0854 \cdot 0.859 \approx 0.1222
\end{equation}
Now substituting all the values into the previous equation \ref{equation:fiberc}. 
\begin{equation}
    \frac{I_{Fiber\: Coupled}}{I_{0.9 NA\: objective}} \approx \frac{\int_{\lambda = 550 nm}^{\lambda = 700 nm} I_{Fiber\: Coupled}(\lambda) d\lambda \cdot 8.177}{\int_{\lambda = 550 nm}^{\lambda = 700 nm} I_{0.9 NA\: objective}(\lambda) d\lambda \cdot 34.25} = \frac{98300 \cdot 8.177}{237500 \cdot 34.25}  \approx 0.0988
    \label{equation:fiberc2}
\end{equation}
And hence the effective coupling efficiency is  $9.88 \pm 0.48 \%$.
\label{AppendixA}