% fabrication and hole drilling
\section{Appendix A - Design and fabrication of the nano-antenna devices}
%
We used the template stripping method in order to fabricate and achieve high quality bullseye antennas, as reported previously \cite{Abudayyeh2021OvercomingSource,Abudayyeh2021SingleNanoantennas}.
With this method, after the stripping we will be left with a device that consists of a sapphire slide on top of which are two transparent layers -
SU8 3010 photo-resist with a thickness of $\sim$ 10 microns, and on top a layer of gold with a 
thickness of 250 nm.
We used a Focused Ion Beam (FIB) machine to drill a hole in the center of the antenna. 
The FIB allowed us to drill a hole in a controlled manner so that it was possible to detect when the gold layer was removed and immediately stop. In this way we could remove the gold layer and be left with only the two bottom layers which are transparent.
% For a hole with a diameter of 400 nm we used a 30 Kv and 7.7 pA ion beam. 
For a hole with a diameter of 400 nm we used a  30 kV ion beam with 7.7 pA current. 

In order to determine the desired size of the hole, we used a Lumerical simulation. 
As mentioned in the main text, the hole size must be large enough to allow transmission of the laser light through the hole and excite
 the dipole, and small enough to maintain the CE by reducing the transmission of the light
 emitted from the dipole source. 
For this purpose, we performed a simulation of the laser transition and the CE as a function of the hole diameter, and thus we could choose a size that would match both requirements, as shown in the in Fig. \ref{fig:A}d in the main text.
The simulation was based on the model described in  \cite{Abudayyeh2017} with the addition of the hole with different diameter values.
Before placing the emitter on the antenna the sample was covered with a dielectric layer of aluminum oxide using ALD deposition which allows us to determine with nanometric precision the thickness of the dielectric layer.
We measured the thickness of the aluminum oxide layer in the region of the hole and the hole size ($d$) using AFM as illustrated in the inset of Fig. \ref{fig:A}a. 
The optimized values of our samples, which are shown in Fig. \ref{fig:A}a are $d$ =400 (400)nm, $D$=1260 (1430)nm, $\Lambda$=560 (635)nm, $h$= 215 (243)nm for the CQD (SiV) respectively. 




% MOVE THIS PARA TO SI (write a full SI section on fabrication and hole drilling)--> As for the depth of the hole, the drilling can be controlled, so that only the gold layer is removed. Underneath the gold is a sapphire layer with SU-8 photoresist on it which are both transparent. These 2 transparent layers, together with the hole size, affect the shape of the dielectric layer that is added after digging the hole, and therefore determines the height of the dipole source.



%In the next step, which is performed before adding a dielectric layer, we drilled a hole in the center of the nanoantenna using Focused Ion Beam (FIB). 



%Note that the diameter accuracy is important because the hole size determines the threshold of wavelengths that can pass through the hole (see \cite{Bethe1944TheoryHoles} and more\cite{MorrisJ.Ehrlich2004StudiesField, Guha2005DescriptionTheory}). % add more details?? word A18
%As for the depth of the hole, the drilling can be controlled, so that only the gold layer is removed. Underneath the gold is a sapphire layer with SU-8 photoresist on it which are both transparent.
%These 2 transparent layers, together with the hole size, affect the shape of the dielectric layer that is added after digging the hole, and therefore determines the height of the dipole source (as illustrated in the inset of Fig. \ref{fig:A}a).
%%In other words, the size of the hole is required to meet several conditions - it must be large enough to allow transmission of the laser light through the hole and excite the dipole, and small enough to maintain the CE by reducing the transmission of the light emitted from the dipole source and enable coupling to the spatial modes of the antenna.
%The transmission of light through small circular holes is % discussed in the paper of (Bethe) and %https://aip.scitation.org/doi/10.1063/1.1721989 and discussed in detail in (https://opg.optica.org/oe/fulltext.cfm?uri=oe-13-5-1424&id=82896 ) using the Hertz vector formalism. 
%o determine the optimal hole size for our device, we performed calculations based on FDTD simulations.
%The appropriate range for the hole size should- on the one hand, allow enough light to transmit through the hole %and to excite the dipole, and on the other hand, to %maintain a high collection efficiency of the emission. 


%hole diameter(d)=400 nm,
%central cavity diameter (D)=1260 nm and 1430 nm, 
%groove period ($\Lambda$)=560 nm and 635 nm,
%dielectric layer thickness (h)= 215 nm and 243 nm for emission at a wavelength of 650 nm for the QD and 737 nm for the SiV respectively.
%Note that the thickness of the layer is taken into account in the calculation of the diameter of the hole mentioned in the previous paragraph, since it affects the height of the dipole.


