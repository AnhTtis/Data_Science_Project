\section{Appendix B - Characterization of gQDs and Positioning by Dip-pen Nanolithography}\label{AppendixB}

Prior to integration into bullseye antennas, gQDs were characterized in the solution phase for absorption and photoluminescence (PL) properties (Figure \ref{fig:appendix b}a, b). Steady-state emission spectra, lifetime, and quantum yield (QY) measurements were made using an Edinburgh Instruments FLS1000 fluorescence spectrometer equipped with 450 W xenon lamp (steady-state emission spectra and QY measurements) and AGILE supercontinuum laser (lifetime measurements) excitation sources. Absorption measurements were made with a Cary 5000 UV-Vis-NIR spectrometer. PL QY was determined using a standard Edinburgh Instruments integrating sphere. QY measurements on hexane dispersions of CdSe/CdS gQDs in quartz cuvettes were made in triplicate, and absorbance spectra were acquired on all samples to confirm that the absorbance value at the excitation wavelength was $<$ 0.1 to limit self-absorption effects. The sample was excited at a wavelength of 490 nm with a monochromator grating of 6.00 nm, and data were collected with an emission monochromator grating of 0.1 nm. The resulting QYs obtained over three measurements afforded a value of 77 ± 2\%. PL lifetime measurements were made at an excitation wavelength of 400 nm with a pulse repetition rate of 1 MHz, and the detection wavelength was set to 650 nm (center of the gQD emission signal). Excitation and emission monochromator gratings were set to 3.00 nm and 1.00 nm, respectively. The resulting PL decay curve is shown in Fig. \ref{fig:appendix b}b.

The technique used to place individual gQDs into the holes located at the center of the bullseye structures is known as dip-pen nanolithography (DPN). It allowed us to transfer gQDs that are suspended  in a high-boiling solvent (dichlorobenzene) and wicked onto the tip of an atomic force microscopy (AFM) cantilever from the AFM tip to the target substrate. The general method was described previously in Ref. \cite{Abudayyeh2021SingleNanoantennas}, and modified slightly here as follows. The AFM tips were placed in the gQD suspension for $\sim$30 s, and excess ink was removed prior to writing by conducting a series of quick scans on a region of the substrate well away from the bullseye structures until ink could no longer be seen coming off of the tip in the DPN’s 10x optical microscope. The employed tip-surface contact times for writing into the hole containing bullseyes was 0.20 - 0.25 s. While it is certain that the writing tip successfully targeted the bullseye center in each placement attempt, it was not possible to determine whether the tip made contact within the hole or at the top surface or hole edge. Nevertheless, in each of four attempts gQDs were successfully delivered into a hole, with 3 receiving single - gQDs and 1 a small cluster of gQDs. By stating this, we do not mean to imply that DPN can deliver small numbers of nanocrystals with 100\% certainty, as the attempted depositions are too few from which to draw such a conclusion. In our previous work studying hole-free bullseyes, we demonstrated a 25\% success rate for depositing either a single gQD or a small cluster,\cite{Abudayyeh2021SingleNanoantennas} and it remains to be determined whether the hole itself plays a role in directing fluid movement on the surface of the device, essentially assisting in the placement process.

After placement into the antennas, we immediately characterized the deposited gQDs using a combination of single-emitter PL spectroscopy, PL lifetime evaluation, and $g^{(2)}(0)$ determination (Fig. \ref{fig:appendix b} c-h). This allowed us to confirm the success of our deposition, including whether single or multiple gQDs were deposited. Specifically. a picosecond pulsed laser (Picoquant LDH-D-C-405) with a wavelength of 405 nm and pulse width of 56 ps was used for excitation. The laser was reflected through a dichroic beamsplitter (Semrock Di02-R405) and then focused onto the sample to a diffraction limited spot size using a 50×, 0.7 NA Olympus objective microscope (LCPLFLN50xLCD), which was used to both excite the sample and collect the PL. Collected photons after passing through the same dichroic beamsplitter and a 590 nm long-pass filter either go to a spectrometer + CCD (Acton SP2300i, pylon100) or a Hanbury Brown-Twiss setup consisting of a 50:50 beamsplitter and two single photon avalanche photodiodes (Excelitas SPCM-AQRH-14). TRPL was analyzed using a TCSPC module (Picoquant Hydraharp 400). Excitation power before the objective is about 500 nW. 

\begin{figure}
    \centering
    \includegraphics[width = 0.8\textwidth]{Images/AppendixB.jpg}
    \caption{(a) Absorption and emission spectra and (b) PL decay lifetime obtained from gQDs suspended in hexanes. (c) PL spectrum for a single gQD embedded in a bullseye antenna. (d) PL decay lifetime for the single gQD in (c). (e) second-order fluorescence intensity correlation ($g^{(2)}$) for the single gQD in (c). (f) and (g) Correlation data for two other single-gQD/antenna couples. (h) Correlation data for a gQD cluster/antenna couple. In (e)-(h) a time gating of 50 ns was employed to remove contributions from faster biexciton emission (compared to slower exciton emission). Correlation data were used after deposition by dip-pen nanolithography to assess whether single or multiple gQDs were placed. Since $g^{(2)}(0)$ is still large after time-gating for the gQD-antenna device in (h), multiple gQDs are understood to be present. }
    \label{fig:appendix b}
\end{figure}




