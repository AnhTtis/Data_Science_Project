% ****** Start of file apssamp.tex ******
%
%   This file is part of the APS files in the REVTeX 4.2 distribution.
%   Version 4.2a of REVTeX, December 2014
%
%   Copyright (c) 2014 The American Physical Society.
%
%   See the REVTeX 4 README file for restrictions and more information.
%
% TeX'ing this file requires that you have AMS-LaTeX 2.0 installed
% as well as the rest of the prerequisites for REVTeX 4.2
%
% See the REVTeX 4 README file
% It also requires running BibTeX. The commands are as follows:
%
%  1)  latex apssamp.tex
%  2)  bibtex apssamp
%  3)  latex apssamp.tex
%  4)  latex apssamp.tex
%
\documentclass[%
 reprint,
superscriptaddress,
%groupedaddress,
%unsortedaddress,
%runinaddress,
%frontmatterverbose, 
%preprint,
%preprintnumbers,
%nofootinbib,
%nobibnotes,
%bibnotes,
 amsmath,amssymb,
 aps,
%pra,
%prb,
%rmp,
%prstab,
%prstper,
%floatfix,
]{revtex4-2}

\usepackage{graphicx}% Include figure files
\usepackage{dcolumn}% Align table columns on decimal point
\usepackage{bm}% bold math
\usepackage{float}
%\usepackage{hyperref}% add hypertext capabilities
%\usepackage[mathlines]{lineno}% Enable numbering of text and display math
%\linenumbers\relax % Commence numbering lines

%\usepackage[showframe,%Uncomment any one of the following lines to test 
%%scale=0.7, marginratio={1:1, 2:3}, ignoreall,% default settings
%%text={7in,10in},centering,
%%margin=1.5in,
%%total={6.5in,8.75in}, top=1.2in, left=0.9in, includefoot,
%%height=10in,a5paper,hmargin={3cm,0.8in},
%]{geometry}

\begin{document}

%\preprint{APS/123-QED}

\title{Magnetic tuning of the tunnel coupling in an optical active quantum dot molecule\\
Supplemental Material}% Force line breaks with \\
%\thanks{A footnote to the article title}%

\author{Frederik Bopp}
\altaffiliation{These authors contributed equally to this work}
%\email{frederik.bopp@wsi.tum.de}

\affiliation{%
 Walter Schottky Institut, School of Natural Sciences, and MCQST, Technische Universit\"at M\"unchen, Am Coulombwall 4, 85748 Garching, Germany
}%


\author{Charlotte Cullip}
\altaffiliation{These authors contributed equally to this work}
\affiliation{%
 Walter Schottky Institut, School of Natural Sciences, and MCQST, Technische Universit\"at M\"unchen, Am Coulombwall 4, 85748 Garching, Germany
}%

\author{Christopher Thalacker}
\affiliation{%
 Walter Schottky Institut, School of Natural Sciences, and MCQST, Technische Universit\"at M\"unchen, Am Coulombwall 4, 85748 Garching, Germany
}%

\author{Michelle Lienhart}
\affiliation{%
 Walter Schottky Institut, School of Natural Sciences, and MCQST, Technische Universit\"at M\"unchen, Am Coulombwall 4, 85748 Garching, Germany
}%

\author{Johannes Schall}
\affiliation{%
 Institut f\"ur Festk\"orperphysik, Technische Universit\"at Berlin, Hardenbergstraße 36, 10623 Berlin, Germany
}%

\author{Nikolai Bart}
\affiliation{%
 Lehrstuhl f\"ur Angewandte Festk\"orperphysik, Ruhr-Universit\"at Bochum, Universit\"atsstraße 150, 44801 Bochum, Germany
}%

\author{Friedrich Sbresny}
\affiliation{%
 Walter Schottky Institut, School of Computation, Information and Technology, and MCQST, Technische Universit\"at M\"unchen, Am Coulombwall 4, 85748 Garching, Germany
}%

\author{Katarina Boos}
\affiliation{%
 Walter Schottky Institut, School of Computation, Information and Technology, and MCQST, Technische Universit\"at M\"unchen, Am Coulombwall 4, 85748 Garching, Germany
}%

\author{Sven Rodt}
\affiliation{%
 Institut f\"ur Festk\"orperphysik, Technische Universit\"at Berlin, Hardenbergstraße 36, 10623 Berlin, Germany
}%

\author{Dirk Reuter}
\affiliation{%
Paderborn University, Department of Physics, Warburger Straße 100, 33098 Paderborn, Germany
}%

\author{Arne Ludwig}
\affiliation{%
 Lehrstuhl f\"ur Angewandte Festk\"orperphysik, Ruhr-Universit\"at Bochum, Universit\"atsstraße 150, 44801 Bochum, Germany
}%

\author{Andreas D. Wieck}
\affiliation{%
 Lehrstuhl f\"ur Angewandte Festk\"orperphysik, Ruhr-Universit\"at Bochum, Universit\"atsstraße 150, 44801 Bochum, Germany
}%

\author{Stephan Reitzenstein}
\affiliation{%
 Institut f\"ur Festk\"orperphysik, Technische Universit\"at Berlin, Hardenbergstraße 36, 10623 Berlin, Germany
}%

\author{Filippo Troiani}
\affiliation{%
 Centro S3, CNR-Istituto Nanoscienze, Via Campi 213/a, 41125 Modena, Italy
}%

\author{Guido Goldoni}
\affiliation{%
Centro S3, CNR-Istituto Nanoscienze, Via Campi 213/a, 41125 Modena, Italy
}%
\affiliation{%
Dipartimento di Scienze Fisiche, Informatiche e Matematiche, Universit\`a di Modena e Reggio Emilia, Via Campi 213/a, 41125 Modena, Italy
}%

\author{Elisa Molinari}
\affiliation{%
Centro S3, CNR-Istituto Nanoscienze, Via Campi 213/a, 41125 Modena, Italy
}%
\affiliation{%
Dipartimento di Scienze Fisiche, Informatiche e Matematiche, Universit\`a di Modena e Reggio Emilia, Via Campi 213/a, 41125 Modena, Italy
}%

\author{Kai M\"uller}
\affiliation{%
 Walter Schottky Institut, School of Computation, Information and Technology, and MCQST, Technische Universit\"at M\"unchen, Am Coulombwall 4, 85748 Garching, Germany
}%

\author{Jonathan J. Finley}%
 \email{finley@wsi.tum.de}
\affiliation{%
 Walter Schottky Institut, School of Natural Sciences, and MCQST, Technische Universit\"at M\"unchen, Am Coulombwall 4, 85748 Garching, Germany
}%


\date{\today}% It is always \today, today,
             %  but any date may be explicitly specified
%\keywords{Suggested keywords}%Use showkeys class option if keyword

\maketitle
\onecolumngrid
\renewcommand{\figurename}{Fig.}
\renewcommand{\thefigure}{S\arabic{figure}}

%%%%%%%%%%%%%%%%%%%%%%%%%%%%%%%%%%%%%%%%%%%%%%%%%%%%%%%%%%%%%%
\section{Sample and Setup}
\label{sec:Sample}

\begin{figure}
\includegraphics[scale=0.5]{SSample.pdf}
\caption{\label{fig:SSample} Schematic of the sample design (left) and the simulated electric field distribution (right). A QDM is located at the interface of the two illustrations. The inset shows a scanning electron microscope image of the QDM-circular Bragg grating device.}
\end{figure}

The investigated QDM was grown by solid-source molecular beam epitaxy. It consisted of two vertically stacked InAs QDs, embedded in a GaAs matrix. The height of the top (bottom) QD was fixed to 2.9\,nm (2.7\,nm) via the In-flush technique during growth\,\cite{Wasilewski1999}. This height configuration facilitates electric field-induced tunnel coupling of orbital states in the conduction band.
A wetting layer to wetting layer separation of 10\,nm and an Al$_x$Ga$_{(x-1)}$As barrier ($x=0.33$) with a thickness of 2.5\,nm placed between the dots determines the coupling strength at 0\,T. A 50\,nm thick Al$_x$Ga$_{(x-1)}$As tunnel barrier ($x=0.33$) was grown 5 nm below the QDM to prolong electron tunneling times. The molecule was embedded into a p-i-n diode, with the doped regions used as contacts to gate the sample. The diode contacts are placed more than 150\,nm from the molecule to prevent uncontrolled charge tunneling into the QDM. Furthermore, a distributed Bragg reflector was grown below the diode and a circular Bragg grating is positioned deterministically via in-situ electron beam lithography above an individual QDM to improve photon in- and outcoupling efficiencies\,\cite{Schall2021}. Figure \ref{fig:SSample} shows a schematic of the sample design (left) and simulated electric field distribution (right) calculated using JCMSuite\,\cite{pomplun2007}. The inset depicts a scanning electron microscope image of the circular Bragg grating.
All measurements are performed at 10 K inside an Oxford Instruments magnet. A tunable diode laser is used to excite the QDM. 




%%%%%%%%%%%%%%%%%%%%%%%%%%%%%%%%%%%%%%%%%%%%%%%%%%%%%%%%%%%%%
\section{Two-state model}
\label{sec: $X^0$ Hamiltonian}

The neutral exciton in our system can be described by the following Hamiltonian:
\begin{align}
    H_X^0 = E_0 \cdot \hat{I} + 
    \begin{bmatrix}
        \Gamma - edF & t \\
        t & 0
    \end{bmatrix} + \hat{H}_{QCSE},
\end{align}
where $E_0$ is the zero-field energy of the direct exciton, $\hat{I}$ is the identity matrix, $\Gamma$ is the energy needed to move an electron from the upper to the lower dot, and $t$ is the tunnel coupling. $edF$ accounts for the Stark shift for inter-dot excitons, with $e$ the electron charge, $d$ the separation between electron and hole, and $F$ the electric field.
The final term $\hat{H}_{QCSE}$ accounts for the energy shift from the quantum-confined Stark-effect, and is given by
\begin{align}
    \Delta E_{QCSE} = \vec{p} \cdot \vec{F} + \beta F^2,
\end{align}
where $\vec{p}$ is the dipole moment and $\beta$ is the polarizability.
Solving the full Hamiltonian gives the following eigenvalues:
\begin{align}
    E_{\pm} = \frac{1}{2}(2E_0-edF+\Gamma + 2\Delta E_{QCSE}\\ 
    \pm \sqrt{(edF-\Gamma)^2 + (2t)^2}).
\end{align}
In the main manuscript, these energies for the symmetric (dotted pink) and anti-symmetric (dotted green) eigenstates are shown in Figure 1 (b).
The splitting $\Delta$E between these eigenenergies is then given as 
\begin{align}
\label{equ:DEanalytical}
    \Delta E = \sqrt{(edF-\Gamma)^2 + (2t)^2},
\end{align}
where $2t$ is the minimum splitting between the two eigenstates, i.e. at the avoided crossing. This equation is used for fitting the data points in Figure 3 (a) of the main manuscript. 




%%%%%%%%%%%%%%%%%%%%%%%%%%%%%%%%%%%%%%%%%%%%%%%%%%%%%%%%%%%
\section{Single particle states in 3D confinement potential}
\label{sec:3dconfinment}
To solve Equation 2 of the main manuscript, we discretize the confinement potential on a real-space grid of $N=N_1\times N_2\times N_3$ points, where $N=64\times64\times128$. The simulation space has a size of $700\times700\times220$\,nm$^3$. Each point is identified by the vector $\textbf{r}_i=\sum_{k=1}^3\left(\lambda_i^k-N_k/2\right)\Delta_k\textbf{e}_k$, with $\lambda_i^k=1,...,N_k$ and $\textbf{e}_{1,2,3}=\textbf{x},\textbf{y},\textbf{z}$. By rephrasing the resulting finite-difference equation, we obtain a discrete eigenvalue problem:

\begin{equation}
\begin{split}
\label{equ:dep}
\sum_{j=1}^{N}\biggl\{\frac{1}{2m_\chi^*}\biggl[-\hbar^2\left(\nabla^2\right)_{ij}+\frac{2i\hbar q_\chi}{c}\left(\textbf{A}\cdot\nabla\right)_{ij}
+\frac{q_\chi^2}{c^2}\left(\textbf{A}^2\right)_{ij}\biggr]+V^\chi_{ij}\biggr\}\phi^\chi_{\alpha,j}=\epsilon_\alpha\phi^\chi_{\alpha,i}\ \text{,}
\end{split}
\end{equation}
where $\phi^\chi_{\alpha,i}=\phi^\chi_\alpha(\textbf{r}_i)$ is the eigenstate of the Hamiltonian, $\chi=e,h$ for electrons and holes, and $q_h=-q_e=|e|$. In the real-space basis, the vector- and confining-potential operators are diagonal, thus $V_{ij}^\chi=\delta_{i,j}V_\chi(\textbf{r}_i)$ and $(\textbf{A}^2)_{ij}=\delta_{i,j}[\textbf{A}(\textbf{r}_i)]^2$. When applying the differential operator on the wave function $\phi^\chi_{\alpha,i}$, we obtain: 
\begin{equation}
\begin{split}
\label{equ:dep}
\sum_{j=1}^{N} & \left(\nabla^2\right)_{ij}\phi_{\alpha,j}=\sum_{k=1}^{3}\frac{\phi_\alpha\left(\textbf{r}_i+\Delta_k\textbf{e}_k\right)-2\phi_\alpha\left(\textbf{r}_i\right)+\phi_\alpha\left(\textbf{r}_i-\Delta_k\textbf{e}_k\right)}{\Delta_k^2}
\end{split}
\end{equation}
and
\begin{equation}
\begin{aligned}
\label{equ:dep}
\sum_{j=1}^{N} & \left(\textbf{A}\cdot\nabla\right)_{ij}\phi_{\alpha,j} = \sum_{k=1}^{3}A_{k,i}\frac{\phi_\alpha\left(\textbf{r}_i+\Delta_k\textbf{e}_k\right)-\phi_\alpha\left(\textbf{r}_i-\Delta_k\textbf{e}_k\right)}{2\Delta_k}\ \text{,}
\end{aligned}
\end{equation}where $A_{k,i}=A_k(\textbf{r}_i)$.

The Hamiltonian is calculated for a 3D-potential, which consists of an asymmetric double square well along the \textbf{z} direction and parabolic potentials along the \textbf{x} and \textbf{y} direction. The square wells have a width of $d_1=2.7$\,nm and $d_2=2.9$\,nm, which corresponds to the height of the bottom and top QD, respectively. The potential depth $V_0=690$\,meV matches the conduction band offset between GaAs and InAs. An electric field equivalent to $3.63$\,meV/nm is included into the model to facilitate the hybridisation of the two lowest electron eigenstates. The model is calibrated by setting the effective electron mass $m_e^*$, as the only free parameter, such that the energy gap between the two lowest eigenstates matches the experimental results. We obtain an effective mass of $m^*=0.0495\,m_{e,0}$ ($m_{e,0}$ is the free-electron mass), which is between the recommended effective masses of GaAs ($m^*_{\text{GaAs}}=0.067$) and InAs ($m^*_{\text{InAs}}=0.023$)\,\cite{Bouarissa1999}. A magnetic field is applied along the in-plane direction \textbf{x}.

\begin{figure}
\includegraphics{STheo.pdf}
\caption{\label{fig:Stheo} (a): Absolute value of the wave function along the growth axes of the lowest (pink) and second lowest (green) electron eigenstate at resonance, and of the lowest hole eigenstate (b). (c) and (d): Two dimensional wave function along \textbf{x} and \textbf{z} for the lowest electron and hole eigenstate, respectively. (e) and (f) magnetic field dependent energy of the two lowest electron and lowest hole eigenstate, respectively. (g): Energy difference of the two lowest electron eigenstates for varying bias (black). The data points are fitted using equation 5 (blue) (h) Coulomb energy between the lowest (second lowest) electron and lowest hole eigenstate in pink (green).}
\end{figure}

Figure \ref{fig:Stheo}\,(a) and (b) show the absolute value of the electron and hole wave functions along the growth direction \textbf{z} at $0$\,T. The green (pink) data set represents the lowest (second lowest) electron eigenstate $\phi_1^e$ ($\phi_2^e$). Hereafter we refer to them as e1 and e2, respectively. Panels\,(c) and (d) show surface plots of the wave function for the lowest electron and hole eigenstate, respectively. The magnetic field dependent single particle energies of e1, e2 and the lowest hole eigenstate are shown in panels (e) and (f), respectively. Bias sweeps are performed to ensure that the simulations are performed at the coupling condition. Figure \ref{fig:Stheo}\,(g) shows the energy difference between e1 and e2 as a function of the bias at $0$\,T. The energy difference is fitted by equation \ref{equ:DEanalytical} to obtain the resonance bias and the minimum energy splitting.

In the experiment, we analyze the energy splitting $\Delta$E of the neutral exciton, instead of a single particle state. To account for the Coulomb interaction, we calculate the direct matrix involving e1, e2 and the lowest energy hole state h1. 
The direct and attractive Coulomb matrix elements are given by:
\begin{equation}
\begin{split}
\label{equ:dep}
V_{\alpha\alpha\beta\beta}^{eh}=\iint
\frac{[\phi_\alpha^e(\textbf{r})]^*[\phi_\beta^{h}(\textbf{r'})]^*\phi_\beta^{h}(\textbf{r'})\phi_\alpha^e(\textbf{r})}
{\kappa_r|\textbf{r}-\textbf{r'}|} 
\,d\textbf{r}\,d\textbf{r'}\ \text{,}
\end{split}
\end{equation}
where $\kappa_r$ is the static dielectric constant of the semiconductor medium. The matrix elements are numerically calculated from the expression
\begin{equation}
\begin{split}
\label{equ:dep}
V_{\alpha\beta\gamma\delta}^{\chi\chi'}=
\pm\frac{e^2}{\kappa}\int\mathcal{F}^{-1}\left[\frac{1}{k^2}\tilde{\Phi}_{\alpha\beta}^{\chi}(\textbf{k})\right]\Phi_{\gamma\delta}^{\chi'}(\textbf{r})
\,d\textbf{r}\ \text{,}
\end{split}
\end{equation}
where $\Phi_{\alpha\beta}^{\chi}(\textbf{r})=[\phi_\alpha^\chi(\textbf{r})]^*\phi_\beta^\chi(\textbf{r})$, and $\tilde{\Phi}_{\alpha\beta}^{\chi}(\textbf{k})=\mathcal{F}[\phi_{\alpha\beta}^\chi(\textbf{r})]$ is its Fourier transform.

The exciton energy is then computed by including the Coulomb interaction in first-order perturbation theory:
\begin{equation}
    E_{X_{0,l}} = \epsilon^e_l + \epsilon^h_1 + V_{ll11}^{eh}\ ,
\end{equation}
where $l=1,2$ specifies the involved electron state and the corresponding exciton. 
The energy splitting $\Delta$E plotted in Figure 2\,(b) incorporates the single particle energy difference $\Delta$E$_{SP}$ as well as the difference in Coulomb energy $\Delta\text{E}_{C}$, and is given by: 
\begin{equation}
    \Delta E^{Th} = E_{X_{0,2}} - E_{X_{0,1}} = \epsilon^e_2 - \epsilon^e_1 + V_{2211}^{eh} - V_{1111}^{eh} = \Delta\text{E}_{SP}+\Delta\text{E}_{C}.
\end{equation}

Figure \ref{fig:Stheo}\,(h) shows the Coulomb energy between e1 (e2) and lowest energy hole state in pink (green). 
%%%%%%%%%%%%%%%%%%%%%%%%%%%%%%%%%%%%%%%%%%%%%%%%%%%%%%%%%%%%%%
\bibliography{Supplemental.bbl}% Produces the bibliography via BibTeX.


\end{document}
%
% ****** End of file apssamp.tex ******
