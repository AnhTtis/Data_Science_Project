\ifdefined\isArXiv
\else
\documentclass[10pt,twocolumn,letterpaper]{article}

\usepackage{iccv}
\usepackage{times}
\usepackage{epsfig}
\usepackage{graphicx}
\usepackage{amsmath}
\usepackage{amssymb}

%
\usepackage{subcaption}
\usepackage{tabularx}
\usepackage[usenames,dvipsnames]{xcolor}
\usepackage{booktabs}  %
 %
\usepackage{fancyvrb}
\usepackage{fvextra}
%
\makeatletter
\@namedef{ver@everyshi.sty}{}
\makeatother
\usepackage{tikz}
\usetikzlibrary{shapes}

\usepackage[numbers,sort]{natbib}

%
%

\usepackage{xspace}
\def\eg{\emph{e.g.}\@\xspace}
\def\ie{\emph{i.e.}\@\xspace}
\def\etal{\emph{et al.}\@\xspace}
\def\cf{\emph{c.f.}\@\xspace}
\def\etc{\emph{etc}}

\renewcommand{\paragraph}[1]{\smallskip\noindent{\bf{#1}}}
%

\def\APall{mAP\textsubscript{all}\@\xspace}
\def\APrare{mAP\textsubscript{rare}\@\xspace}
\def\APcomm{mAP\textsubscript{comm}\@\xspace}
\def\APfreq{mAP\textsubscript{freq}\@\xspace}
\def\LVISminusrare{LVIS\textsubscript{-R}\@\xspace}

\def\zerorec{0Ways\@\xspace}
\def\onerec{1Ways\@\xspace}
\def\tworec{2Ways\@\xspace}
\def\threerec{3Ways\@\xspace}

\newcommand{\isArXiv}[2]{#1}
%



%
%
%
\usepackage[pagebackref=true,breaklinks=true,letterpaper=true,colorlinks,bookmarks=false]{hyperref}

%

\def\iccvPaperID{1802}
\def\httilde{\mbox{\tt\raisebox{-.5ex}{\symbol{126}}}}

%
\ificcvfinal\pagestyle{empty}\fi

\begin{document}

%
\def\fulltitle{Three ways to improve feature alignment for open vocabulary detection}
\title{\fulltitle \\ -- Supplementary material --}
\hypersetup{pdfauthor={\iccvPaperID},pdftitle={\iccvPaperID: \fulltitle}}


\maketitle
%
\ificcvfinal\thispagestyle{empty}\fi

\tableofcontents
\fi

\newcommand{\tabArhParams}{
\def\y{\checkmark}
\def\n{~}
\begin{table*}[t]
\centering
\begin{tabular}{ccccrr} \toprule
&&&& \multicolumn{2}{c}{Batch size} \\
\cmidrule{5-6}
Backbone & Self-training & Init lr & Device & \LVISminusrare & CC12M \\
\midrule
NFNet-F0 & \n & $3 \times 10^{-4}$ & 16 TPUv3 chips & 128 & / \\
NFNet-F0 & \y & $3 \times 10^{-4}$ & 16 TPUv3 chips & 128 & 1024 \\
NFNet-F6 & \n & $1 \times 10^{-4}$ & 16 TPUv3 chips & 64 & / \\
NFNet-F6 & \y & $1 \times 10^{-4}$ & 32 TPUv4 chips & 32 & 256 \\
\bottomrule
\end{tabular}
\caption{{\bf Architecture-specific optimization parameters.}
}
\label{tab:archparams}
\end{table*}
}


\newcommand{\tabLvisAll}{
\begin{table*}[t]
\definecolor{cheat}{HTML}{999999}
\centering
\begin{tabular}{ccccccc} \toprule
& & & \multicolumn{2}{c}{LVIS} & \multicolumn{2}{c}{Transfer} \\
\cmidrule(l){4-5} \cmidrule(l){6-7}
Method & Train dataset & Backbone & \APall & \APrare & COCO & O365 \\
\midrule
\tworec & LVIS & NFNet-F0 & \textcolor{cheat}{\meanstd{34.4}{0.11}} & \textcolor{cheat}{\meanstd{23.7}{1.41}} & \meanstd{40.8}{0.11} & \meanstd{14.7}{0.07} \\
\tworec & \LVISminusrare & NFNet-F0 & \meanstd{33.8}{0.15} & \meanstd{20.9}{0.34} & \meanstd{40.6}{0.14} & \meanstd{14.6}{0.04} \\
\threerec & \LVISminusrare & NFNet-F0 & \meanstd{35.7}{0.20} & \meanstd{25.6}{1.12} & \meanstd{41.5}{0.16} & \meanstd{16.4}{0.15} \\
\tworec & LVIS & NFNet-F6 & \textcolor{cheat}{\meanstd{45.1}{0.18}} & \textcolor{cheat}{\meanstd{37.0}{0.97}} & \meanstd{46.5}{0.04} & \meanstd{20.8}{0.05} \\
\tworec & \LVISminusrare & NNetF-F6 & \meanstd{43.5}{0.12} & \meanstd{27.6}{0.80} & \meanstd{46.5}{0.26} & \meanstd{20.3}{0.05} \\
\threerec & \LVISminusrare & NFNet-F6 & \meanstd{44.6}{0.31} & \meanstd{30.1}{1.83} & \best{\meanstd{46.9}{0.05}} & \best{\meanstd{22.8}{0.14}} \\
\bottomrule
\end{tabular}
\caption{{\bf Training on all of LVIS vs self-training.}
Training \emph{\tworec} on full LVIS predictably performs better on LVIS than when
training on \LVISminusrare (\ie LVIS without annotations for rare classes).
However, it loses to the self-trained detector (\emph{\threerec}) when transferring to COCO and Objects365.
}
\label{tab:lvisall}
\end{table*}
}

%
%
%
%


\newcommand{\figOrig}{
\begin{figure*}[t]
\definecolor{arrow}{HTML}{6aa84f}
\definecolor{gate}{HTML}{ff9900}
%
\definecolor{back}{HTML}{cfe2f3}
\includegraphics[width=\linewidth]{figures/arch_orig.pdf}
\caption{{\bf Standard single-stage detector architecture.}
\textcolor{CornflowerBlue}{Light blue} and \textcolor{Apricot}{light yellow} parallelograms
represent the backbone and FPN feature maps, respectively;
circles with ↑2 are non-trainable up--sampling,
squares are trainable modules (\eg convolutions),
convolution blocks show the kernel size
and potential striding (`s2': stride 2).
The alignment preserving architecture is shown in \isArXiv{Figure~\ref{fig:gating}.}{Figure 3 of the main paper.}
}
\label{fig:orig}
\end{figure*}
}


\newcommand{\zs}[1]{\textcolor{RedOrange}{#1}}
\newcommand{\ms}[1]{\textcolor{Gray}{#1}}

\newcommand{\figQualThreeVsTwo}{
\begin{figure*}[t]
\def\imW{0.495\linewidth}
\centering%
\begin{tabular}{c@{~~}c}
\includegraphics[width=\imW]{figures/qual/f6_2/P1040771.jpeg}
&
\includegraphics[width=\imW]{figures/qual/f6_3/P1040771.jpeg}
\\
\includegraphics[width=\imW]{figures/qual/f6_2/Parisian_gargoyle_Unsplash.jpg}
&
\includegraphics[width=\imW]{figures/qual/f6_3/Parisian_gargoyle_Unsplash.jpg}
\\
\includegraphics[width=\imW]{figures/qual/f6_2/1280px-Macaroons_at_Smiths.jpg}
&
\includegraphics[width=\imW]{figures/qual/f6_3/1280px-Macaroons_at_Smiths.jpg}
\\
(a) \tworec & (b) \threerec
\end{tabular}
\caption{{\bf \emph{\tworec} vs \emph{\threerec} (NFNet-F6).}
Detections with a score larger than 0.3 are shown.
The queries for the three examples are:
(top) \zs{Eiffel tower}, \zs{French flag}, \zs{cannon}, \ms{poster};
(middle) \zs{Eiffel tower}, \zs{gargoyle};
(bottom) \ms{cake}, \zs{macaron}.
No human-annotated boxes were seen during training for
\zs{all queries} apart from \ms{poster} and \ms{cake}.
\emph{\tworec} has zero-shot detection capability,
but \emph{\threerec} generally outperforms it.
}
\label{fig:qual:threevstwo}
\end{figure*}
}


\newcommand{\figQualThreeParis}{
\begin{figure*}[t]
\def\imHu{4.7cm}
\def\imHd{4.79cm}
\includegraphics[height=\imHu]{figures/qual/f6_3/1280px-Wolves_chasing_a_wapiti,_Yellowstone_River_2.jpg}
\includegraphics[height=\imHu]{figures/qual/f6_3/Jumping_over_the_moon.jpg}
\includegraphics[height=\imHu]{figures/qual/f6_3/1280px-Round_hay_bales_and_a_hot_air_balloon_somewhere_in_Luxembourg} \\
\includegraphics[height=\imHd]{figures/qual/f6_3/1280px-Men_Ice_hockey.jpg}
\includegraphics[height=\imHd]{figures/qual/f6_3/Venezia-gondola_on_canal_grande.jpeg}
\includegraphics[height=\imHd]{figures/qual/f6_3/1280px-POOL_HALL_-_NARA_-_543975.jpg}
\caption{{\bf \emph{\threerec} (NFNet-F6) zero-shot detection examples.}
Detections with a score larger than 0.3 are shown.
The queries for the six examples are:
(top-left) \zs{wolf}, \zs{wapiti}, \ms{duck};
(top-middle) \zs{rocket}, \zs{moon};
(top-right) \zs{hot-air balloon}, \zs{haystack};
(bottom-left) \ms{helmet}, \zs{puck}, \zs{hockey stick};
(bottom-middle) \zs{gondola}, \ms{pole};
(bottom-right) \zs{pool table}, \ms{ball}, \ms{deer}, \ms{wallclock}.
Networks were trained with human-annotated boxes for
\ms{duck}, \ms{helmet}, \ms{pole}, \ms{ball}, \ms{deer} and \ms{wallclock},
while no human-annotated boxes were seen during training for
\zs{wolf}, \zs{wapiti}, \zs{rocket}, \zs{moon}, \zs{hot-air balloon}, \zs{haystack}, \zs{puck}, \zs{hockey stick},
\zs{gondola}, and \zs{pool table}.
Note that \ms{dog} (a frequent training category) rightfully has a
lower confidence than \zs{wolf} for the top-left image.
}
\label{fig:qual:threeparis}
\end{figure*}
}


\newcommand{\figQualLvis}{
\begin{figure*}[t]
\vspace{-0.5cm}
\centering
\includegraphics[width=0.95\linewidth]{figures/qual/lvis_6rows}
\caption{{\bf Random sample: LVIS validation set (\emph{\threerec}, NFNet-F6).}
Up to 20 detections with a score larger than 0.5 are shown.
Best viewed digitally.
}
\label{fig:qual:lvis}
\end{figure*}
}


\newcommand{\figQualCoco}{
\begin{figure*}[t]
\vspace{-0.5cm}
\centering
\includegraphics[width=0.95\linewidth]{figures/qual/coco_6rows}
\caption{{\bf Random sample: COCO validation set (\emph{\threerec}, NFNet-F6).}
The detector is trained on \LVISminusrare and transferred
to COCO without any additional training.
Up to 20 detections with a score larger than 0.5 are shown.
Best viewed digitally.
}
\label{fig:qual:coco}
\end{figure*}
}


\newcommand{\figQualObj}{
\begin{figure*}[t]
\vspace{-0.5cm}
\centering
\includegraphics[width=0.95\linewidth]{figures/qual/objects365v1_6rows}
\caption{{\bf Random sample: Objects365-v1 validation set (\emph{\threerec}, NFNet-F6).}
The detector is trained on \LVISminusrare and transferred
to Objects365 without any additional training.
Up to 20 detections with a score larger than 0.5 are shown.
Best viewed digitally.
}
\label{fig:qual:o365}
\end{figure*}
}



%
\section*{Appendix overview}

Appendix~\ref{sec:app:imp} contains full details of the detector architecture
and training procedure,
while
Appendix~\ref{sec:app:init} provides more information on the
alignment preserving architecture.
In Appendix~\ref{sec:app:selft} we give further details on the
implementation of self-training, reimplementation of Detic, and
relation to other methods.
Finally, qualitative detection examples are shown in Appendix~\ref{sec:app:qual}.


%
\section{Implementation details}
\label{sec:app:imp}

\paragraph{Architecture.}
For closed vocabulary detectors, the final fully connected layer
also trains a per-class bias, and initializing it according to the focal loss
scheme~\cite{lin2017focal} stabilizes the beginning of training.
For the open vocabulary case it is not possible to have this per-class bias term,
so we use a learnt shared bias, \ie a single scalar added to the scalar product
between the classification features and the text embeddings,
initialized in the same way, \ie
$\text{bias}_\text{init} = -\log\left(\left(1 - p\right) / p\right)$, where $p=0.01$.
In initial experiments we observed this indeed aided stability.

The original T-Head detector head architecture~\cite{feng2021tood} uses
the ``task-aligned predictor'' (TAP) only on the regression and classification branches
and not on the quality branch. For consistency we simply use TAP
on all three branches.


\paragraph{Input and annotation processing.}
Training and evaluation is done with $800 \times 1024$ images for all datasets
(\LVISminusrare, COCO, Objects365) apart from the pseudo-labelled images
from CC12M where we found that $400 \times 512$ is sufficient.
Training uses large scale jittering~\cite{ghiasi2021simple}
for \LVISminusrare and small scale jittering for CC12M.
Following~\cite{zhou2021probabilistic,minderer2022simple},
50 pseudo-negatives are used for each training sample.

\paragraph{Total loss} is the sum of three equally weighted standard losses:
binary focal loss~\cite{lin2017focal} with default FCOS parameters ($\gamma=2$, $\alpha=0.25$),
gIoU loss\cite{rezatofighi2019generalized},
and the IoU prediction loss~\cite{wu2020iou}.

\tabArhParams

\paragraph{Optimization} is done with AdamW~\cite{loshchilov2019decoupled} and $10^{-4}$ weight decay,
and,
as is standard for NFNets,
adaptive gradient clipping~\cite{brock2021highperformance}
with the clipping threshold of 0.04 is used.
Depending on the backbone architecture and whether self-training is performed,
we adjust the training hardware, batch size and initial (post-warmup) learning rate
-- Table~\ref{tab:archparams} shows the parameters for all configurations.
All networks are trained for 32 epochs
and the learning rate schedule consists of linear warm-up for 0.25 epochs,
followed by 10 fold reductions at 67\% and 89\% of the training.
The same schedule is used for self-training.

\isArXiv{The implementation uses JAX~\cite{google2018jax} and the DeepMind JAX ecosystem~\cite{deepmind2020jax}.}{}


%
\section{Alignment preserving architecture}
\label{sec:app:init}

\figOrig

\isArXiv{Figure~\ref{fig:gating}}{Figure~3 of the main paper} shows our alignment preserving architecture
with shortcuts and trainable gates.
For reference, Figure~\ref{fig:orig} shows the classic
design~\cite{lin2017feature,tian2019fcos} of a single-stage detector.

\paragraph{Trainable gating.}
Recall that the gates perform the following operation:
$o = x (1-\tan\alpha) + y \tan{\alpha}$,
where $x$ and $y$ are the inputs and $\alpha$ is a learnt parameter initialized to $0$.
We briefly experimented with the Flamingo~\cite{alayrac2022flamingo} formulation:
$o = x + y \tan{\alpha}$, and found it performs similarly.

\paragraph{Matching dimensionalities.}
For clarity of presentation,
we glanced over one detail that we elaborate on here --
gates and shortcuts assume the dimensionalities of all input and output
features are the same,
while the default design of the feature pyramid network (FPN)
and detection heads includes changes in the channel dimension.
We do not attempt to fix this by changing the default settings for the FPN and detector head
dimensionalities and making all channels the same size,
as that would increase the parameter count, be less memory and computation efficient,
and be harder to compare fairly with existing approaches.

Specifically, the issue is the initial $1 \times 1$ convolution in the FPN
applied on the final backbone features, as this reduces channels to 256
and thus loses information.
All following processing maintains 256-D features until the very end
where there is an increase back to the original dimensionality
to produce classification features compatible with text embeddings
(786-D and 1376-D for NFNet-F0 and NFNet-F6, respectively).
We address this pragmatically -- the dimensionality reduction convolution is initialized
such that distances and scalar products are preserved as much as possible,
though use of an orthogonal projection.
Similarly, the dimensionality increase at the end of the network
is initialized to the transpose of this projection.
We find that the details are not crucial -- in fact the axis aligned
orthogonal matrix (\ie a rectangular ``cropped'' identity matrix)
is used for simplicity, and even without this initialization
the training still works well as it only needs to learn one
projection to bring features back into alignment.

%


%
\section{Self-training}
\label{sec:app:selft}

\paragraph{Implementation details.}
Self-training simultaneously processes a batch from \LVISminusrare
with human-annotations and a batch from CC12M with pseudo-annotations.
All the three losses (classification, bounding box regression, quality prediction)
are used for the pseudo-labelled data as well as this performs slightly better
than only using the classification loss.
For Detic$^\dagger$ and \emph{Image bbox} baselines we found that,
consistent with the paper~\cite{zhou2022detecting},
using only the classification loss performs best, and report only these results.
The losses for the two human- and pseudo-labelled data are simply summed up,
different relative weighting was not found to be beneficial for any of
the methods or baselines.


\paragraph{Relation to other methods.}
We elaborate on the comparison to other methods from \isArXiv{Section~\ref{sec:selft}}{Section~3.3 of the main paper}.
Recall that~\cite{redmon2017yolo9000,sohn2020simple,zoph2020rethinking,ramanathan2020dlwl}
are closed vocabulary detectors and and there is no need or way to use
batch-negatives.
GLIPv2~\cite{zhang2022glipv2} does not consider the zero-shot scenario
and uses noun phrases instead of entire captions like we do.
Furthermore, their deep fusion architecture restricts the use
of batch-negatives to the pre-fusion features making its effect
indirect, while our loss operates on the very final detector head
classification features which are directly used for detecting the object.

In contrast to our approach and GLIPv2,
Detic~\cite{zhou2022detecting} does not make use of the image caption
to produce the pseudo-detection, but instead
picks the largest bounding box proposal instead. The method does use
batch-negatives in an additional loss,
although this is done on the global image-level rather
than the bounding box-level.

Finally, the Detic paper~\cite{zhou2022detecting} also implements
baselines where the closed vocabulary pseudo-labelling methods mentioned above~\cite{redmon2017yolo9000,sohn2020simple,zoph2020rethinking,ramanathan2020dlwl}
are adapted to the open vocabulary setting.
For these baselines, the paper uses the image caption to produce pseudo-detections,
but does not use batch-negatives.
We showed in \isArXiv{Section~\ref{sec:selft:res}}{Section~3.3.1 of the main paper} that batch-negatives are crucial for
good performance.


\paragraph{Comparisons with Detic~\cite{zhou2022detecting}.}
Detic~\cite{zhou2022detecting} focuses on the zero-shot detection task
but the main results forgo the open vocabulary claim as
the knowledge of the names of the \LVISminusrare test classes
is explicitly used during training,
\eg the captions are filtered for LVIS class names and this noisy label is
assigned to the image.
We focus on zero-shot open vocabulary detection and do not use test class names
during training, so we compare only to the open vocabulary version of Detic
(\ie according to Detic terminology, no ``image label'' supervision, only ``caption'' supervision).

Furthermore, for a fair comparison as we use different backbones, detector architectures,
pretraining and self-training datasets, \etc,
we also reimplement the open vocabulary Detic, referred to as Detic$^\dagger$.
Our \emph{\threerec} approach and Detic$^\dagger$ are identical everywhere
apart from that pseudo-labelling is performed differently, where
\emph{\threerec} uses the most confident detection produced by \emph{\tworec} given the caption,
Detic$^\dagger$ uses the largest bounding box proposal from \emph{\tworec}.
It should be noted that our FCOS-based detector architecture is single-stage
and thus technically does not have bounding box proposals,
but we found that we can simulate producing proposals by filtering out the dense boxes
through thresholding the \emph{quality score};
we sweep a range of thresholds and pick the best performing one (0.75).
Our Detic$^\dagger$ reimplementation surpasses the original Detic
(\APall improved from 30.4\% to 34.8\%, and \APrare from 17.4\% to 23.4\%),
but our \emph{\threerec} approach beats this further (\APall of 35.7\% and \APrare of 25.6\%).


\tabLvisAll

\paragraph{Comparison with full LVIS training.}
In all other experiments detectors are trained on \LVISminusrare,
a version of the LVIS dataset where annotations for rare classes are removed
in order to test zero-shot detection.
Here we investigate the value of those annotations by training a
\emph{\tworec} detector on the entire LVIS (\ie without removing annotations for rare classes).
As expected,
\emph{\tworec} trained on full LVIS beats \emph{\tworec} trained on \LVISminusrare
when evaluated on LVIS itself, due its superior performance on the rare classes (Table~\ref{tab:lvisall}).
However, the effects of the dataset change are marginal
when transferring to COCO and Objects365,
and our self-trained \emph{\threerec} detector is still decidedly the best.
Therefore, pseudo-labelled data can be more valuable than ground truth annotations.
Furthermore, when using the NFNet-F0 backbone, \emph{\threerec} even beats the `cheating'
\emph{\tworec} on LVIS itself, but this result does not hold for the larger and better
NFNet-F6 backbone.


%
\section{Qualitative results}
\label{sec:app:qual}

\figQualThreeVsTwo
\figQualThreeParis
\figQualLvis
\figQualCoco
\figQualObj

\paragraph{Zero-shot detection.}
Figure~\ref{fig:qual:threevstwo} compares the zero-shot performance of
the \emph{\tworec} and \emph{\threerec} networks -- \emph{\tworec} is capable
of detecting some classes not seen during training, but \emph{\threerec}
is better.
Figure~\ref{fig:qual:threeparis} further showcases
\emph{\threerec}'s zero-shot capabilities.


\paragraph{Random samples from benchmark datasets.}
Figure~\ref{fig:qual:lvis} shows detections on a sample of LVIS validation
images.
Transfer to COCO and Objects365 is shown in Figure~\ref{fig:qual:coco}
and~\ref{fig:qual:o365}, respectively.
Kuo \etal~\cite{kuo2023fvlm} estimate the overlap between the training
set classes of \LVISminusrare and the classes of
COCO and Objects365 to be 91\% and 63\%, respectively.
Therefore, Figure~\ref{fig:qual:o365} also illustrates
impressive zero-shot capabilities of our method.


\paragraph{Failure cases.}
Instead of the typical failure cases such as
difficulty in detecting objects that are highly occluded, small, in unusual poses,
\etc,
here we focus on failure cases specific to zero-shot open vocabulary detection.
While exhibiting impressive zero-shot performance, we observe that
the network does not detect some common objects such as
houses, buildings, windows, bricks, trees, leaves, rocks, eyes, \etc.
We speculate that this is because images containing these objects are often
seen during training but never annotated with a bounding box,
thus the network learns they are not-an-object and does not assign
a high enough quality score.
There is also the philosophical question of what is an object,
for example, object parts such as eyes generally do not get detected
by our networks.
However, the system does detect wheels, as they are annotated in
the \LVISminusrare training set.
All of the above indicates that that there is room for improvement
in the zero-shot capabilities of the objectness component of detectors
(`quality score' here, region proposal networks in two-stage detectors),
or that this concept should be removed completely
as being somewhat incompatible with zero-shot open vocabulary detection.

\paragraph{Image attribution.}
Images used in \isArXiv{%
Figure~\ref{fig:teaser},~\ref{fig:qual:threevstwo}
}{%
Figure~\ref{fig:qual:threevstwo}}
and~\ref{fig:qual:threeparis}
were downloaded from Wikimedia%
\isArXiv{ or are personal photos of an author (Relja Arandjelović)},
and all are free to use and modify -- we made no modifications apart
from resolution change and the overlaying of the detections.
The Wikimedia originals can be found at the following links:

\begin{Verbatim}[fontsize=\tiny, breaklines=true, breakafter=\_, breakaftersymbolpre=none]
https://commons.wikimedia.org/wiki/File:Parisian_gargoyle_(Unsplash).jpg
https://commons.wikimedia.org/wiki/File:Venezia-gondola_on_canal_grande.JPG
https://commons.wikimedia.org/wiki/File:Macaroons_at_Smiths.jpg
https://commons.wikimedia.org/wiki/File:Round_hay_bales_and_a_hot_air_balloon_somewhere_in_Luxembourg.jpg
https://commons.wikimedia.org/wiki/File:POOL_HALL_-_NARA_-_543975.jpg
https://commons.wikimedia.org/wiki/File:Jumping_over_the_moon.jpg
https://commons.wikimedia.org/wiki/File:Wolves_chasing_a_wapiti,_Yellowstone_River_(2).jpg
https://commons.wikimedia.org/wiki/File:2020-01-11_Men's_Ice_hockey_3x3_Preliminary_round_Team_Blue_vs._Team_Orange_(2020_Winter_Youth_Olympics)_by_Sandro_Halank–057.jpg
\end{Verbatim}


%

\isArXiv{}{
{\small
\bibliographystyle{ieee_fullname}
\bibliography{bib/shortstrings,bib/more,bib/vgg_other,bib/vgg_local}
}

\end{document}
}
