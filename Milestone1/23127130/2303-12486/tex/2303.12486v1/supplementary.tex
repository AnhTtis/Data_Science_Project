% Use only LaTeX2e, calling the article.cls class and 12-point type.

\documentclass[12pt]{article}

% Users of the {thebibliography} environment or BibTeX should use the
% scicite.sty package, downloadable from *Science* at
% www.sciencemag.org/about/authors/prep/TeX_help/ .
% This package should properly format in-text
% reference calls and reference-list numbers.

\usepackage{scicite}
\usepackage{graphicx}% Include figure files
% Use times if you have the font installed; otherwise, comment out the
% following line.
\usepackage{amsfonts}
\usepackage{times}
\usepackage{amsmath}
\usepackage{xcolor}
\usepackage{comment}
% The preamble here sets up a lot of new/revised commands and
% environments.  It's annoying, but please do *not* try to strip these
% out into a separate .sty file (which could lead to the loss of some
% information when we convert the file to other formats).  Instead, keep
% them in the preamble of your main LaTeX source file.

% self-imported package
\usepackage{multicol}

% The following parameters seem to provide a reasonable page setup.

\topmargin 0.0cm
\oddsidemargin 0.2cm
\textwidth 16cm 
\textheight 21cm
\footskip 1.0cm

%The next command sets up an environment for the abstract to your paper.
\newenvironment{sciabstract}{%
\begin{quote} \bf}
{\end{quote}}

% If your reference list includes text notes as well as references,
% include the following line; otherwise, comment it out.

\renewcommand\refname{References and Notes}
\renewcommand{\thefigure}{S\arabic{figure}}
\newcommand{\TODO}[1]{\textcolor{red}{[#1]}}


% The following lines set up an environment for the last note in the
% reference list, which commonly includes acknowledgments of funding,
% help, etc.  It's intended for users of BibTeX or the {thebibliography}
% environment.  Users who are hand-coding their references at the end
% using a list environment such as {enumerate} can simply add another
% item at the end, and it will be numbered automatically.

\newcounter{lastnote}
\newenvironment{scilastnote}{%
\setcounter{lastnote}{\value{enumiv}}%
\addtocounter{lastnote}{+1}%
\begin{list}%
{\arabic{lastnote}.}
{\setlength{\leftmargin}{.22in}}
{\setlength{\labelsep}{.5em}}}
{\end{list}}

% Include your paper's title here

\title{Supplementary material} 
% Place the author information here.  Please hand-code the contact
% information and notecalls; do *not* use \footnote commands.  Let the
% author contact information appear immediately below the author names
% as shown.  We would also prefer that you don't change the type-size
% settings shown here.
\begin{comment}
\author
{Hongyi Xiao,$^{1,5,6\dagger}$ Ge Zhang,$^{1,2\dagger}$, Entao Yang,$^{3\dagger}$ Robert Ivancic,$^{4\dagger}$ Sean Ridout,$^{1}$ \\ Robert Riggleman,$^{3}$ Douglas Durian,$^{1}$ Andrea Liu$^{1\ast}$\\
\\
\normalsize{$^{1}$Department of Physics, University of Pennsylvania, Philadelphia, PA, USA}\\
\normalsize{$^{2}$City University of Hong Kong, Hong Kong, China}\\
\normalsize{$^{3}$Department of Chemical and Biomolecular Engineering, University of Pennsylvania,}
\\ \normalsize{Philadelphia, PA, USA}\\
\normalsize{$^{4}$Materials Science and Engineering Division, National Institute of Standards and Technology, }
\\ \normalsize{Gaithersburg, MD, USA}\\
\normalsize{$^{5}$Institute for Multiscale Simulation, Friedrich-Alexander-Universit\"{a}t Erlangen-N\"{u}rnberg, }
\\ \normalsize{Erlangen, Germany}
\\ \normalsize{$^{6}$Department of Mechanical Engineering, University of Michigan, Ann Arbor, MI, USA}\\
%\normalsize{209 S 33rd St, Philadelphia, PA 19104, USA}\\
%\normalsize{$^{2}$Another Unknown Address, Palookaville, ST 99999, USA}\\
\\
\normalsize{$^\dagger$Equal contribution.}
\\
\normalsize{$^\ast$To whom correspondence should be addressed; E-mail:  ajliu@physics.upenn.edu.}
}
\end{comment}

% Include the date command, but leave its argument blank.

\date{}

%%%%%%%%%%%%%%%%% END OF PREAMBLE %%%%%%%%%%%%%%%%



\begin{document} 

% Double-space the manuscript.

\baselineskip24pt

% Make the title.

\maketitle 


% In setting up this template for *Science* papers, we've used both
% the \section* command and the \paragraph* command for topical
% divisions.  Which you use will of course depend on the type of paper
% you're writing.  Review Articles tend to have displayed headings, for
% which \section* is more appropriate; Research Articles, when they have
% formal topical divisions at all, tend to signal them with bold text
% that runs into the paragraph, for which \paragraph* is the right
% choice.  Either way, use the asterisk (*) modifier, as shown, to
% suppress numbering.

\section{Particle systems}
\subsection{System I: Soft repulsive disks}

%\cite{ozawa2020role}
System I is a computer-simulated 2D polydisperse soft-disk system detailed in Ref.~(47). We received configurations equilibrated at several different initial temperatures using the swap Monte Carlo algorithm from the authors of Ref.~(47). Note that the configurations prepared initially at $T=0.025$, used for System IB, may not have been completely equilibrated before the quench to $T=0$, but this does not affect interpretation of the results.


%~\cite{zhang2021interplay}
After potential-energy minimization to zero temperature, $T=0$, these configurations are sheared quasistatically by repeatedly applying a small simple shear strain of $\delta \epsilon = 10^{-5}$ and minimizing the total potential energy, until the total strain reaches $\epsilon_{\text{end}}=0.1$. Energy minimization is carried out using the steepest descent algorithm with a very conservative step size to simulate overdamped dynamics, as described in the supplementary material of Ref.~(36). During the minimization, we study intermediate configurations using the same protocol as Ref.~(36). These intermediate configurations allow us to investigate the interplay of relevant quantities during an avalanche.

\subsection{System II: Granular rafts}
%~\cite{xiao2020strain}

System II is an experimental granular raft made of a disordered monolayer of polydisperse granular particles floating at an air-oil interface~(9). Mineral oil was used with a density of $\rho_{\text{oil}}=$870 $\pm$ 10\,kg/m$^3$, and the particles are made of Styrofoam with a density of 15\,kg/m$^3$. Systems IIA and IIB correspond to rafts with different particle size distributions, with average particle sizes of 1.0\,$\pm$0.1\,mm for IIA and 3.3\,$\pm$\,0.3\,mm for IIB. Capillary attractions exist between nearby particles, with a characteristic interaction range set by the capillary length of the oil, $l_c=\sqrt{\gamma_{\text{oil}}/\rho_{\text{oil}} g}=1.8$\,mm, where $\gamma_{\text{oil}}=27.4$\,dyn\,cm$^{-1}$, and $g=$9.8\, m/s$^2$. 
%~\cite{xiao2020strain}
For each experiment, particles were assembled into a disordered monolayer (pillar) with a rectangular shape~(9). Quasi-static tensile tests with a strain rate on the order of $10^{-5}$\,s$^{-1}$ were then conducted and all particle positions and the global tensile force were tracked throughout the experiments. For extracting microscopic information and constructing StEP models, results from 50 experiments~(9) with $80\sigma\times40\sigma$ sized pillars for each particle size were used. For comparing with StEP model results, results from 12 experiments of larger pillars of size $120\sigma\times60\sigma$ for each particle size were used for better statistics.

\subsection{System III: Polymer nanopillars}

% IVANCIC
%~\cite{Ivancic2019}
%
Using LAMMPS, we simulate bead-spring polymer nanopillars with $N = 5$ monomers per chain as detailed in Ref.~(39). We bond the monomers using a stiff harmonic potential $U_{i,j}^b = \frac{k}{2} \left( r_{i,j} - \sigma \right)^2$ where $r_{i,j}$ is the distance between monomers $i$ and $j$ and $k = 2000 \epsilon/\sigma$. We take the nonbonded interactions as a modified Lennard-Jones potential $U_{i,j}^{nb} = 4 \epsilon \left( \left( \frac{\sigma'}{r_{i,j}-\Delta} \right)^{12} - \left( \frac{\sigma'}{r_{i,j}-\Delta} \right)^{6} \right)$ where $\sigma' = (1-3/2^{13/6}) \sigma$ and $\Delta = 3\sigma/4$. This modification increases the curvature of the potential while maintaining the minima at $r_{\text{min}} = 2^{1/6} \sigma$, causing more brittle failure~(8). For all of our simulations, we use a timestep of $0.000663652$ to compensate for this additional curvature. We thermalize our simulations within a cylindrical, harmonic confining wall at $T=0.5$ that we fix to ensure the density of the monomers within the wall is $\rho = 0.3$. We then cool our simulation at a rate of $5 \times 10^{-4}$ past the glass transition temperature of $T_{g} \approx 0.38$ to $T=0.30$ (Case A) and $T=0.05$ (Case B) causing the density to increase to $\rho \approx 1.0$ at the lower temperature. We then deform the nanopillars at a true strain rate of $\dot{\epsilon} = 10^{-4}$. We repeat this procedure for $50$ replicas. We output monomer positions every $10000$ timesteps. 

\section{Softness training}
\subsection{Softness training using an SVM}

\begin{figure}[t]
\centering
\includegraphics[width=0.45\linewidth]{figure_P_soft_given_D2min.pdf}
\caption{Probability that a particle of a given $D^2_{\text{min}}$ is soft for the granular raft experiments (System II) and polymer nanopillar simulations (System III). Here we scale $D^2_{\text{min}}$ by the particle radii $\sigma$ in each case.}
\label{fig:P_soft_given_D2min}
\end{figure}

%As suggested in Ref.~\citenum{rocks2021learning},
% IVANCIC
%Refs.~\citenum{cubuk2015identifying} and \citenum{Schoenholz2016}
%~\cite{falk1998dynamics}
We train the softness field for Systems II and III similarly to Refs.~(27) and~(28). For System III we analyze the time-averaged monomer positions during these trajectories, averaging every $5$ timesteps in the $500$ timesteps before each frame. We first extract a set of rearranging and non-rearranging particles from the early stage of the deformation (pre-shear band formation). To differentiate between these groups, we calculate a $D^2_{\text{min}}$ field~(19) in which we take the strain between frames to be commensurate with the amount of strain for a particle to complete a rearrangement ($\epsilon_G \approx 2.0\times10^{-3}$ for System II and $\epsilon_G \approx 6.6\times10^{-4}$ for System III) and take the cutoff radius ($R_{c} = 1.75 \sigma$) to include the first shell of neighbors. We consider this field's local maxima and minima to be the rearranging and non-rearranging particles. We next encode the local structure around these training examples as a high-dimensional vector ($\vec{G} \in \mathbb{R}^N$) 
%of functions similar to the Behler-Perrinello \cite{behler2007generalized} functions, which we describe in detail elsewhere\cite{yang2020effect}. 
which we describe in detail elsewhere~(28). Because large longitudinal strains cause significant changes in local density at later strains, we also consider these structural features scaled by deviations of the local packing density from the average density $\left(\rho - \langle \rho \rangle \right)/\langle \rho \rangle$. We append this scaled vector to the original vector to fully describe the local structure of our training examples, $\vec{G}^\frown \left(\left(\rho - \langle \rho \rangle \right)/\langle \rho \rangle \vec{G} \right)$. We use these appended structural descriptors in $\mathbb{R}^{2N}$ as the input for a linear support vector machine (SVM) classification to calculate the hyperplane that best separates rearranging particles from non-rearranging particles. We compute softness as the signed distance from the hyperplane to the data point in the appended high-dimensional structural descriptor space. We exclude particles on the exterior of the pillars from training and testing for simplicity. We use the same softness field for both particle sizes in the granular raft experiment (System II) by rescaling by the average particle radius. To validate this model, we consider the probability of a particle being soft at a given $D^2_{\text{min}}$ $P(S>0|D^2_{\text{min}})$ for a set of pillars independent from the pillars that were use to construct the test set, Fig.~\ref{fig:P_soft_given_D2min}. This function has a strong monotonically increasing dependence on $D^2_{\text{min}}$ that plateaus near $1$ suggesting a strong correlation between being soft and obtaining a large dynamic event. 

%The machine-learned softness shows high correlation with the rearranging events observed in early stage deformation. For the granular experiments, 12 runs of 1\,mm particle pillars were used for training with XXX example structures in total, and the hyperplane is tested against 50 runs for each particle sizes. High training accuracy is achieved, which is demonstrated in Fig. (MAKE A FIGURE). For the oligomer simulations, ........

\subsection{Softness training using a neural network}

The standard definition of softness used for Systems II and III correlates only very weakly with rearrangements for System IB.  For System I we therefore use a neural network instead of a support vector machine to define softness.

%~\cite{zhang2022structuro}
We train a 17-layer residual convolutional neural network to predict the local yield stress of each particle. The architecture is similar to the ``wide residual network''~(48), except for the following differences. First, the input is changed to 128-by-128 grayscale (single channel) images, cropped from an image of the entire configuration, which has a resolution of 2048-by-2048 for configurations with $N=10000$ particles and 4096-by-4096 for configurations with $N=64000$ particles. One example of such an image is presented in Fig.~\ref{fig:neuralNetworkExampleInput}. Second, we used four groups of convolutions instead of three, since our input image is larger. Each group contains two ResNet blocks, as in~(48). Third, we used the ``basic'' version of ResNet blocks in Fig.~1 of Ref.~(48). While Ref.~(48) found that wider versions are better for image classification tasks, we find that they provide little improvement for our tasks. Fourth, We removed the last global-average pooling layer and softmax layer of the original neural network, because they are only suitable for image classification tasks. These layers are replaced with a fully-connected layer with a single neuron outputting a predicted local yield stress. Our loss function is the squared difference between the predicted and the actual local yield stress. We calculate the actual local yield stress using the procedure detailed in Ref.~(37). Last, we impose an L2 regularizer with regularization parameter $0.2$ on all weights and biases of the neural network. We also augment the training data by randomly flipping it in both the horizontal and the vertical directions.

The neural network was trained on the configurations after equilibration at three different temperatures but before performing quasistatic shear. Both the training dataset and the test dataset are derived from five independent configurations with  $N=10000$ particles equilibrated at $T_a=0.025$, one configuration with $N=64000$ and $T_a=0.1$, and one configuration with $N=64000$ and $T_a=0.2$. For each configuration, we calculate the local yield stress of every particle using the protocol detailed in Ref.~(37), and asked the neural network to predict it. The neural network was trained in 40 epochs with batch size 16. We used Adam minimizer for training with a learning rate that decays exponentially from $3\times10^{-4}$ to $3\times10^{-6}$. After training, the coefficient of determination on the test set is $R^2=0.5809$, i.e., 58.09\% of the variance in the test data is captured by the prediction. Let the predicted local yield stress for a particle be $Y_p$, we then define softness as $S=\langle Y_p \rangle -Y_p$, where $\langle Y_p \rangle=12.09$ is the average of $Y_p$ at $T_a=0.2$.

\begin{figure}[t]
\centering
\includegraphics[width=0.35\linewidth, trim={0.92cm 0.65cm 0 0},clip]{neuralNetworkExampleInput.png}
\caption{An example input of the neural network for System I.}
\label{fig:neuralNetworkExampleInput}
\end{figure}

\section{Constructing StEP models from simulations and experiments}

\subsection{Elastic kernel functions in 2D and 3D}

%\cite{budrikis2013avalanche}~\cite{nicolas2013mesoscopic}

The elastic kernel function is the strain change in other sites caused by a unit strain release at the origin. Our model in 2D explicitly includes two strain components: $\epsilon_{xy}$ and $\epsilon_{xx-yy}=(\epsilon_{xx}-\epsilon_{yy})/2$, and therefore uses four elastic kernels: a $\epsilon_{xy}$ to $\epsilon_{xy}$ kernel, a $\epsilon_{xx-yy}$ to $\epsilon_{xx-yy}$ kernel, and two cross-term kernels. Each kernel was derived using the ``Fourier discretized'' method~(40). Specifically, we start from analytical Fourier-space kernel functions, discretize on a lattice with the same size as the real-space system, and then perform an inverse discrete Fourier transform to obtain the kernel in the real space. The reason for this process is that digitization in the Fourier space is equivalent to applying periodic boundary conditions in the real space.

The analytical expression for the Fourier-space kernel in 2D is obtained from Ref.~(49). It is
\begin{equation}
\begin{pmatrix}
\epsilon_{xx-yy}(\mathbf q)\\
\epsilon_{xy}(\mathbf q)
\end{pmatrix}
=\frac{1}{|\mathbf q|^4}
\begin{pmatrix}
-(q_x^2-q_y^2)^2 & -2q_xq_y(q_x^2-q_y^2)\\
-2q_xq_y(q_x^2-q_y^2) & -4q_x^2q_y^2
\end{pmatrix}
\begin{pmatrix}
\epsilon_{p, xx-yy}\\
\epsilon_{p, xy}
\end{pmatrix}
\end{equation}
where $\epsilon_{xx-yy}(\mathbf q)$ and $\epsilon_{xy}(\mathbf q)$ are the elastic strain changes, $\mathbf q=(q_x, q_y)$ is the Fourier-space coordinates, and $\epsilon_{p, xx-yy}$ and $\epsilon_{p, xy}$ are the strain releases at the origin.

In 3D, we have five strain components: $\epsilon_{xx}$, $\epsilon_{xy}$, $\epsilon_{xz}$, $\epsilon_{yy}$, and $\epsilon_{yz}$. Thus, there are $25$ elastic kernels in total. Similar as 2D cases, each kernel can be derived using the ``Fourier discretized'' method and the analytical expression is provided below:

\begin{equation}
\begin{pmatrix}
\epsilon_{xx}(\mathbf q)\\
\epsilon_{xy}(\mathbf q)\\
\epsilon_{xz}(\mathbf q)\\
\epsilon_{yy}(\mathbf q)\\
\epsilon_{yz}(\mathbf q)
\end{pmatrix}
=\frac{1}{|\mathbf q|^4}
\begin{pmatrix}
G_{11} & G_{12} & G_{13} & G_{14} & G_{15} \\
G_{21} & G_{22} & G_{23} & G_{24} & G_{25} \\
G_{31} & G_{32} & G_{33} & G_{34} & G_{35} \\
G_{41} & G_{42} & G_{43} & G_{44} & G_{45} \\
G_{51} & G_{52} & G_{53} & G_{54} & G_{55} 
\end{pmatrix}
\begin{pmatrix}
\epsilon_{p, xx}\\
\epsilon_{p, xy}\\
\epsilon_{p, xz}\\
\epsilon_{p, yy}\\
\epsilon_{p, yz}
\end{pmatrix}
\end{equation}

\begin{multicols}{2}
\begin{equation}
\begin{pmatrix}
G_{11}\\
G_{21}\\
G_{31}\\
G_{41}\\
G_{51}\\
\end{pmatrix}
=
\begin{pmatrix}
2q_x^2(q^2-q_x^2+q_z^2) - q^4 \\ 
q_xq_y(q^2 - 2q_x^2+2q_z^2) \\
2q_xq_z(-q_x^2+q_z^2) \\
2q_y^2(-q_x^2 + q_z^2) \\
q_yq_z(-q^2-2q_x^2+2q_z^2)
\end{pmatrix}
\end{equation}

\begin{equation}
\begin{pmatrix}
G_{12}\\
G_{22}\\
G_{32}\\
G_{42}\\
G_{52}\\
\end{pmatrix}
=
\begin{pmatrix}
2q_xq_y(q^2-2q_x^2) \\ 
-4q_x^2q_y^2 - q^2q_z^2 \\
q_yq_z(q^2 - 4q_x^2) \\
2q_xq_y(q^2 - 2q_y^2) \\
q_xq_z(q^2-4q_y^2)
\end{pmatrix}
\end{equation}
\end{multicols}

\begin{multicols}{2}
\begin{equation}
\begin{pmatrix}
G_{13}\\
G_{23}\\
G_{33}\\
G_{43}\\
G_{53}\\
\end{pmatrix}
=
\begin{pmatrix}
2 q_x q_z(q^2 - 2q_x^2) \\ 
q_y q_z(q^2 - 4q_x^2) \\
-4 q_x^2 q_z^2 - q^2 q_y^2 \\
-4 q_x q_y^2 q_z \\
q_x q_y(q^2 - 4 q_z^2)
\end{pmatrix}
\end{equation}

\begin{equation}
\begin{pmatrix}
G_{14}\\
G_{24}\\
G_{34}\\
G_{44}\\
G_{54}\\
\end{pmatrix}
=
\begin{pmatrix}
2 q_x^2 (-q_y^2 + q_z^2) \\ 
q_xq_y(q^2 - 2q_y^2 + 2q_z^2) \\
q_xq_z(-q^2 - 2q_y^2 + 2q_z^2) \\
2 q_y^2 (q^2 - q_y^2 + q_z^2) - q^4 \\
2 q_yq_z (-q_y^2 + q_z^2)
\end{pmatrix}
\end{equation}

\end{multicols}

\begin{equation}
\begin{pmatrix}
G_{15}\\
G_{25}\\
G_{35}\\
G_{45}\\
G_{55}\\
\end{pmatrix}
=
\begin{pmatrix}
-4 q_x^2 q_y q_z \\ 
q_x q_z (q^2 - 4 q_y^2) \\
q_x q_y (q^2 - 4 q_z^2) \\
2 q_y q_z (q^2 - 2 q_y^2) \\
-4 q_y^2 q_z^2 - q^2 q_x^2
\end{pmatrix}
\end{equation}
where $\epsilon_{xx}(\mathbf q)$, $\epsilon_{xy}(\mathbf q)$, $\epsilon_{xz}(\mathbf q)$, $\epsilon_{yy}(\mathbf q)$, and $\epsilon_{yz}(\mathbf q)$ are the elastic strain changes, $\mathbf q=(q_x, q_y, q_z)$ is the Fourier-space coordinates, and $\epsilon_{p, xx}$, $\epsilon_{p, xy}$, $\epsilon_{p, xz}$, $\epsilon_{p, yy}$, and $\epsilon_{p, yz}$ are the strain releases at the origin.

\subsection{Obtaining yield strain distributions for Systems I-III}

\begin{figure}[t]
\centering
%\includegraphics[width=0.32\linewidth]{{meanYpVsY.pdf}}
%\includegraphics[width=0.32\linewidth]{{multipleHistogramGaussian_200.pdf}}
%\includegraphics[width=0.32\linewidth]{{multipleHistogramGaussian_025.pdf}}
\includegraphics[width=1.0\linewidth]{figure_yieldstrain_softDisk.pdf}
\caption{Measurement of local yield stress for System I. (A) Mean actual local yield stress $\langle Y \rangle$ versus predicted local yield stress $Y_p$ for System I. Black dotted line represents $Y_p=Y$, i.e., perfect predictions from the neural network. (B) The distribution of the actual local yield stress $Y$ for particles with predicted local stress $Y_p$ in the range (red) $10<Y_p<10.5$, (yellow) $13<Y_p<13.3$, (green) $16<Y_p<16.5$, and (blue) $18<Y_p<19$, respectively, for the 2D soft-disk systems equilibrated at $T_a=0.2$ (System IA). Dots are histograms, and lines are their Gaussian fits. (C) Same as (B), except for $T_a=0.025$ (System IB).}
\label{fig:localyieldstrain2}
\end{figure}

\begin{figure}[t]
\centering
\includegraphics[width=1.0\linewidth]{figure_yieldstrain_granular.pdf}
\caption{Measurement of local yield strain for System II, granular rafts. (A) Mean yield strain vs. softness for 1.0\,mm rafts (System IIA, red) and 3.3\,mm rafts (System IIB, blue). The dotted lines are corresponding linear fittings, $\langle\epsilon_Y\rangle = 0.0134 - 0.000611 S$, for IIA, and, $\langle\epsilon_Y\rangle = 0.0118 - 0.00245 S$, for IIB. (B) and (C) distribution of yield strain for specific softness for the 1\,mm (System IIA) and 3\,mm (System IIB) rafts, respectively. The dotted curves are Weibull distribution with the mean $\langle\epsilon_Y\rangle$ and the shape parameter $k=1.9$ for System IIA rafts and $k=2.1$ for System IIB.}
\label{fig:localyieldstrain}
\end{figure}

%~\cite{zhang2022structuro}~\cite{barbot2018local}
The local yield strain is measured in two different ways. For the soft disk simulations of System I, a ``frozen matrix method'' was used~(37). As detailed in Ref.~(37), the method is similar to Ref.~(23) except that we do not project the yield strains in different directions to the global-shear direction. For the experimental granular rafts (System II) and the polymer nanopillars (System III), a 'rewinding' method was used. For each rearrangement, its yield strain $\epsilon_Y$ is measured as the deviatoric elastic strain accumulated between $t_0$ and $t_{re}$, where $t_0$ is the time corresponding to the beginning of the deformation or the end of the previous rearrangement at the same location, and $t_{re}$ is the beginning of the current rearrangement. 

For System I, 2D soft disks, the softness $S=\langle Y_p \rangle -Y_p$ is defined as a function of the neural network's predicted local yield stress $Y_p$. Therefore, it is more straightforward to compare $Y$ and $Y_p$. In general, $Y_p$ and $Y$ are close but not exactly the same, since the neural network is not perfectly accurate. In Fig.~\ref{fig:localyieldstrain2}B and C we plot the distribution of $Y$ for particles with $Y_p$ residing in different bins. We find that the distribution shapes are closer to Gaussian rather than Weibull functions. In Fig.~\ref{fig:localyieldstrain2}A we show that the mean of $Y$ is always very close to $Y_p$. We also found that the standard deviation of the distributions does not vary significantly across bins. Based on these observations, in StEP simulations we let the distribution of $Y$ be a Gaussian with mean equal to $Y_p$, and standard deviation equal to 3.1 for $T_a=0.2$ (System IA) and 2.2 for $T_a=0.025$ (System IB). 
The local yield {\it strain} $\epsilon_Y$ can then be calculated by dividing $Y$ by the average shear modulus, $G=89$.

Figure~\ref{fig:localyieldstrain} shows the relation between the local yield strain and softness for System II, the experimental granular rafts. The measured local yield strain is binned according to the local softness before rearrangement and the binned-averaged value, $\langle \epsilon_Y\rangle$, is plotted against $S$ in Fig.~\ref{fig:localyieldstrain}A for granular materials. Examples of measured distributions for specific softness ranges are shown in Fig.~\ref{fig:localyieldstrain}B and C. Results for polymer nanopillars, System III, are shown in Fig.~\ref{fig:localyieldstrain3}.

\begin{figure}[t]
\centering
%\includegraphics[width=0.3\linewidth]{polymer figures/Figure S4/Sum_ave_ey.pdf}
%\includegraphics[width=0.3\linewidth]{polymer figures/Figure S4/T005_weibull.pdf}
%\includegraphics[width=0.35\linewidth]{polymer figures/Figure S4/T030_weibull.pdf}
\includegraphics[width=1.0\linewidth]{figure_yieldstrain_polymer.pdf}
\caption{Measurement of local yield strain in System III, polymer nanopillars. (A) Mean yield strain vs. softness for and $T=0.30$ pillars (IIIA, red) and $T=0.05$ pillars (IIIB, blue). The dash lines are corresponding linear fittings, $\langle\epsilon_Y\rangle = \max (0.0484 - 0.102S,  0.0382)$, for IIIA, and,  $\langle\epsilon_Y\rangle = 0.0472 - 0.0045S$, for IIIB. (B) and (C) are Distribution of yield strain for specific softnesses for (B) IIIA and (C) IIIB. The dashed curves are Weibull distributions with the mean $\langle\epsilon_Y\rangle = \max (0.0484 - 0.102S,  0.0382)$ and the shape parameter $k=2.02$ for IIIA, and $\langle\epsilon_Y\rangle = 0.0472 - 0.0045S$ and $k=1.84$ for IIIB. The color gradient represents the softness gradient.}
\label{fig:localyieldstrain3}
\end{figure}


\subsection{Near-field structural contribution to softness kernel}

\begin{figure}[t]
\centering
\includegraphics[width=0.8\linewidth]{figure_restore_softDisk.pdf}
\caption{Near-field structural contribution to the change of the predicted local yield stress, $\Delta Y_p$, due to rearrangements for 2D soft repulsive disks (System I), after subtracting the strain contribution $\Delta Y_{p, strain}$. (A) restoring effect for $T_a=0.2$ (System IA), for neighbors of rearrangers within distance $r<1.34$ (blue), $2.25<r<3.78$ (green), $6.35<r<10.68$ (orange), and $17.96<r<30.20$ (red); (B) restoring effect for $T_a=0.025$ (System IB), colors have the same meanings as (A); (C) restoring coefficient for systems IA (red) and IB (blue); and (D) variance of softness change, colors have the same meanings as (C). In (C), we found that two common equations, $\eta(r)=0.0779r^{-2.5}$ and $\eta(r)c=0.253r^{-2.5}$, can roughly fit $\eta(r)$ and $\eta c(r)$ for both temperatures, and the fit is therefore indicated by a black dotted line.}
\label{fig:restore_softDisk}
\end{figure}


\begin{figure}[t]
\centering
\includegraphics[width=0.8\linewidth]{figure_restore_granular.pdf}
\caption{Near-field contribution to softness change due to rearrangements in granular raft experiments (System II). (A) restoring effect for 1.0\,mm rafts (System IIA). $\Delta S_{\text{inf}}$ is the mean softness change in the background.  (B) restoring effect for 3.3\,mm rafts (System IIB). The dotted lines are linear fits. (C) restoring coefficient vs. distance to rearrangers. Inset: average softness change vs. distance. The curves are fits for System IIA, $\eta(r)=0.10e^{-r/1.40}$ and $c=0.08$, and for System IIB, $\eta(r)=0.09e^{-r/0.65}$ and $c=1.20$. (D) variance of softness change vs. distance to rearrangers.}
\label{fig:restore_granular}
\end{figure}

\begin{figure}[t]
\centering
\includegraphics[width=0.8\linewidth]{figure_restore_polymer.pdf}

\caption{Near-field structural contribution to softness change due to rearrangements in polymer nanopillars (System III). (A) restoring effect for $T=0.30$ (System IIIA). (B) restoring effect for $T=0.05$ (System IIIB).  (C) Restoring coefficient. The curves are fits for System IIIA, $\eta(r)=0.206e^{-r/1.02} + 0.059$, and for System IIIB, $\eta(r)=0.162e^{-r/1.45} + 0.011$. Notes that the $\eta$ decay length, $\xi_\eta$, scales with the $D_{min}^2$ correlation length, $\xi$, where $\xi_\eta / \xi \approx 1.33$ for both temperature. (D) Variance of softness change. Dot points are the variance measured in MD simulation and dash lines are the variance estimated from $\eta$.}
\label{fig:restore_oligomer}
\end{figure}

%~\cite{zhang2022structuro}
To measure the near-field softness kernel due to structural change induced by rearrangements, $\Delta S_{\text{struct}}=\eta(r)(\langle S \rangle  - S + c) + \delta(r)$, the change of the softness field $\Delta S(r) - \Delta S_\infty$ is linearly fitted as a function of the softness $S$ before the rearrangement~(37). This is evaluated at different distances to the rearranger and the results for the three respective systems are shown in Fig.~\ref{fig:restore_softDisk}A and B,~\ref{fig:restore_granular}A and B, and ~\ref{fig:restore_oligomer}A and B. For all three system, a negative slope $-\eta(r)$ is seen and $\eta$ decreases with the distance to the rearranger. For System I, we found a power-law decay fits $\eta(r)$ better, shown in Figure ~\ref{fig:restore_softDisk}C.
For Systems II and III, $\eta(r)$ is fitted to an exponential relation as shown in Fig.~\ref{fig:restore_granular}c and ~\ref{fig:restore_oligomer}C. In addition, a non-zero intercept $\eta(r)c$ exists for the granular experiments and the soft disk simulations, which is plotted in the inset of Fig.~\ref{fig:restore_softDisk}C and~\ref{fig:restore_granular}C.

The variance of the noise term, $\Gamma(r)$, follows a previously derived detailed balance-like relation, $\Gamma(r)=\eta(r)[2-\eta(r)]\sigma^2$, which describes the measured data well, see Fig.~\ref{fig:restore_softDisk}D, ~\ref{fig:restore_granular}D, and~\ref{fig:restore_oligomer}D.




\subsection{Far-field elastic contribution to softness kernel}
%~\cite{zhang2022structuro}

To calculate the parameters in the elastic contribution to the softness kernel, $\Delta S_{\text{elastic}}=\zeta\tilde{\epsilon}_{\text{vol}} + \kappa \Delta\left | \tilde \epsilon \right |^2$, a method based on global mean softness change was used~(37). 
%{\bf GZ: should $\zeta$ depend on $r$ and $\theta$? Mine does not.} 
Ensemble-averaged global mean softness, $\bar{S}(\epsilon_G)$, was calculated at different global strain. An assumption is made that at the very beginning of the deformation, all particles experience similar elastic strain and the plastic contribution of the few individual rearrangers is negligible. Under this assumption, we calculate $\zeta$ and $\kappa$ by fitting $\bar{S}(\epsilon_G)$ with $\zeta\epsilon_{\text{vol,G}} + \kappa \Delta\left | \epsilon_{\text{dev,G}} \right |^2$. Here, $\epsilon_{\text{vol,G}}=(1-\nu)\epsilon_G$ is the global volumetric strain and $\epsilon_{\text{dev,G}}=\frac{1+\nu}{2}\epsilon_G$ is the global deviatoric strain, and $\nu$ is the Poisson's ratio. For the 1.0\,mm granular raft (System IIA), $\nu=0.5$, $\zeta=2.37$, $\kappa=139$. For the 3.3\,mm granular raft (System IIB), $\nu=0.5$, $\zeta=1.74$, $\kappa=339$. 
Similar trend has been found in the polymer nanopillars (System III). For $T=0.30$ (System IIIA), $\zeta=9.17$, $\kappa=13.50$. For $T=0.05$ (System IIIB), $\zeta=16.60$, $\kappa=11.05$. 
%{\bf GZ: Do we have a sign error here or is $\kappa$ actually negative? EY: Fixed by only using the initial elastic regime for fitting. Results become even better.}

For 2D repulsive disks (System I), since we applied a simple shear strain rather than a tensile strain, we could only find $\kappa$ from this fit. We found $\kappa=411$ fits both temperatures well. To extract $\zeta$, we measured locally-fit volumetric and deviatoric strains of particles during energy-minimization simulations. We performed fit $\Delta S_{\text{elastic}}=\zeta\tilde{\epsilon}_{\text{vol, local}} + \kappa \Delta\left | \tilde \epsilon_{\text{local}} \right |^2$, and found $\zeta=226$ for $T_a=0.025$ and $\zeta=244$ for $T_a=0.2$.


\subsection{Triggering probability for rearrangements}

The rearrangements observed in the experiments and simulations typically involve a few particles. In the StEP simulation, each block has the length corresponding to the particle diameter. Thus, a plastic event in the StEP simulation should contain several blocks that rearrange at the same time. This is realized by letting the center rearranging block ($|\epsilon|>\epsilon_Y$) triggering nearby blocks to rearrange, with a probability $C(r)$. To approximate this probability, we use the spatial $D^2_{min}$ correlation at early stages of the deformation, $C(r)=C_{d2min}(r)$, and fit an exponential relation to the data, as shown in Fig.~\ref{fig:d2mincorrelation}. Note that for polymer nanopillars (System III), we do not force the correlation fitting to be continuous for the small range ($r < 1.0$), due to the existence of bonds. 

For the thermal system, we also need to scale the term accounting rearrangement size when using the $D^2_{min}$ correlation length.
This is because $D^2_{min}$ measures the non-affine displacement and scales with temperature. 
In Figure \ref{fig:thermal_scaled_d2min}, we plotted the measured $P_R$ as a function of distance to the rearranging particle for different softness.
We can see that putting the correlation length directly overestimates $C(r)$, especially near the rearranging particle. 
Thus, we scale it to particles' rearranging probability, $P_R$, at the average softness, $P_R(\langle S \rangle)$, which gives us a scaling factor of $0.15$. The same scaling factor is then used for the low T as well.
We found that our results are qualitatively insensitive to the choice of this scaling factor, if the overall particle rearranging ratio in the StEP model is on the same order of magnitude as the corresponding MD simulation.

\begin{figure}[t]
\centering
\includegraphics[width=1.0\linewidth]{d2minCorrelation_figure.pdf}

%\includegraphics[width=0.32\linewidth]{{figure_d2corr.pdf}}
%\includegraphics[width=0.32\linewidth]{{polymer figures/D2min_corr.pdf}}
%\includegraphics[width=0.32\linewidth]{{d2minCorrelation_misaki.pdf}}
\caption{Correlation of $D_{min}^2$ vs. distance to rearranger $r$. (A) Soft repulsive disks (System I), with $T_a=0.2$ (System IA) data fitted to $C_{d2min}(r)=1.0e^{-r/1.88}$ (red) and $T_a=0.025$ (System IB) data fitted to $C_{d2min}(r)=1.0e^{-r/1.63}$ (blue). (B) is same as (A), except for the granular rafts (System II), with 1.0\,mm data fitted to $C_{d2min}(r)=1.0e^{-r/1.13}$ (red), and 3.0\,mm data fitted to $C_{d2min}(r)=1.0e^{-r/1.12}$ (blue). (C) is same (A), except for the polymer nanopillars (System III), with $T=0.3$ (System IIIA) data fitted to $C_{d2min}(r)=1.40e^{-r/0.75}$ (red), and $T=0.05$ (System IIIB) data fitted to $C_{d2min}(r)=0.87e^{-r/1.11}$ (blue).}

%(a) Granular rafts and the dotted curves are fits for the 1.0\,mm rafts (red), $C_{d2min}(r)=1.0e^{-r/1.13}$, and for the 3.3\,mm rafts (blue), $C_{d2min}(r)=1.0e^{-r/1.12}$. (b) Same as (a), except for the 3D polymer nanopillars, with $T=0.05$ data fitted to $C_{d2min}(r)=0.87e^{-r/1.11}$ (blue) and $T=0.3$ data fitted to $C_{d2min}(r)=1.40e^{-r/0.75}$ (red). (c) Same as (a), except for the 2D repulsive disks system, with $T_a=0.025$ data fitted to $C_{d2min}(r)=1.0e^{-r/1.63}$ (blue) and $T_a=0.2$ data fitted to $C_{d2min}(r)=1.0e^{-r/1.88}$ (red).
\label{fig:d2mincorrelation}
\end{figure}

\begin{figure}[t]
\centering
\includegraphics[width=0.6\linewidth]{updated_scale_factor.pdf}
\caption{Scaling $P_R$ at $T=0.30$ to make the rearranging probability in MD simulation and StEP model on the same order of magnitude. The color gradient represents the softness gradients from -3.0 (blue) to 3.0 (red). The dash lines are the calculated $P_R-P_{R, bulk}$ before and after scaling. We choose a scale factor of 0.15 and the same scale factor is then applied to pillars at $T=0.05$.}
\label{fig:thermal_scaled_d2min}
\end{figure}

\begin{figure}[t]
\centering
\includegraphics[width=0.85\linewidth]{T_barrier_polymer.pdf}
\caption{\label{fig:thermal_eyring} Free energy barrier (blue), yield strain (orange), and their ratio (green) as a function of softness: (A) System IIIA with $T=0.30$; (B) System IIIB with $T=0.05$. At both temperature, $\Delta F (S) / \epsilon_y^2$ decreases with softness.}
\end{figure}

\subsection{Rearrangements triggered by thermal fluctuation}

The polymer nanopillars (System III) are different from the other two systems because they are at non-zero temperature.
To account for the thermal fluctuations that also induce rearrangements, a temperature-dependent rearranging mechanism is introduced for the polymer nanopillars. As explained before, a block starts rearranging if its elastic deviatoric strain, $\left | \tilde{\epsilon} \right |$, is larger than its local yield strain, $\epsilon_Y$. 
In the thermal system, if this rearranging criteria is not met, the site can also rearrange with a probability of $C_{th}$:
%
\begin{equation}
    C_{th} (S, \left | \tilde{\epsilon} \right |) = \exp(\Delta F (S) - V_0 G \left | \tilde{\epsilon} \right |^2) 
    \label{eq:thermal_re}  
\end{equation}
%

%\cite{Schoenholz2016}
where $\Delta F$ is the free energy barrier for rearrangements in quiescent system, $V_0$ is the activation volume, $G$ is the local elastic modulus. 
The local elastic modulus is assumed to be the same for all the particles and equal to the  elastic modulus of the pillars, which are 29.13 in System IIIA and 74.58 in System IIIB. 
As shown by previous softness study~(28), $\Delta F (S)$ has the expression:
%
\begin{equation}
    \Delta F (S) = (\epsilon_0 - \epsilon_1 \cdot S) + T \cdot(- (\Sigma_0 - \Sigma_1 \cdot S)).
    \label{eq:thermal_F}  
\end{equation}
%
Here, $\epsilon_i$ and $\Sigma_i$ represent the enthalpic and entropic barrier for rearrangements, which are independent of both the temperature and the softness. 

Considering we have two rearranging mechanisms in the StEP model, we should expect that their predicted rearranging probabilities are continuous at $\left | \tilde{\epsilon} \right | = \epsilon_y$ for every particle. This gives out another constraint for the thermal yielding criteria:
%
\begin{equation}
    V_0 G = \Delta F (S) / \epsilon_Y^2.
    \label{eq:thermal_constiant}  
\end{equation}
%
We find that $\Delta F (S) / \epsilon_Y^2$ decreases with softness (see Fig.~\ref{fig:thermal_eyring}) at both temperatures. 
This trend meets our expectations because recent work shows that the activation volume, $V_0$, should decrease with softness~(50).
%

% Your references go at the end of the main text, and before the
% figures.  For this document we've used BibTeX, the .bib file
% scibib.bib, and the .bst file Science.bst.  The package scicite.sty
% was included to format the reference numbers according to *Science*
% style.


%\section{Summary of StEP parameters}
%In table~\ref{table1}, we list all the StEP input parameters investigated in Fig.~6 in the main text, and highlight the key parameters that determines ductility for each system.

% \begin{table}
% %\begin{center}
% \caption{Summary of StEP model parameters measured from the corresponding particle systems for soft disks (SD), granular rafts (GR), and the polymer nanopillars (NP). Key parameters that dominate the ductility in each system are highlighted in red.}
% \label{table1}
% \resizebox{\textwidth}{!}{
% \begin{tabular}{ |c|| c| c| c |c |c| c|}
% \hline
%   Parameters & SD (low $T_a$) & SD (high $T_a$)  & GR (1.0 mm) & GR(3.0 mm) & NP (low $T, T=0.05$) & NP (high $T, T=0.30$)\\
%   \hline
%  $\langle S \rangle$ & \color{red}{-2.73} & \color{red}{0.00}  & -0.09 & -0.04 & 0.085 & 0.575\\
%  $STD(S)$ & \color{red}{2.28} & \color{red}{3.18}  & 0.78 & 0.70 & 0.72 & 0.70\\
%  $k\,or\,STD$ & $\,STD=2.20$ & $\,STD=3.08$  & $k=1.9$ & $k=2.1$ & $k=1.84$ & $k=2.02$\\
%  $\langle\epsilon_Y\rangle(S)$ & $(12.09-S)/89$ &  $(12.09-S)/89$ & \color{red}{$0.013-0.0006S$} & \color{red}{$0.012 -0.00245S$} & $0.0484 - 0.0045S$ & $\max (0.0484 - 0.102S,  0.0382)$\\
%  $\eta(r)$ & $0.78r^{-2.5}$ & $0.78r^{-2.5}$  & \color{red}{$0.1e^{-r/1.4}$} & \color{red}{$0.09e^{-r/0.65}$} & $0.162e^{-r/1.45} + 0.011$ & $0.206e^{-r/1.02} + 0.059$\\
%  $c$ & -0.325 & -0.325  & \color{red}{0.08} & \color{red}{1.20} & 0 & 0\\
%  $\kappa$ & 411 & 411  & 139 & 339 & 11.05 & 13.50\\
%  $\zeta$ & 226 & 244  & 2.37 & 1.74 & 16.60 & 9.17\\
%  $C_{d2min}$ & $1.0e^{-r/1.63}$ & $1.0e^{-r/1.88}$  & $1.0e^{-r/1.13}$ & $1.0e^{-r/1.12}$ & \color{red}{$0.87e^{-r/1.11}$} & \color{red}{$1.40e^{-r/0.75}$}\\
%  \hline
% \end{tabular}
% }
% %\caption{Summary of StEP model parameters measured from the corresponding particle systems for soft disks (SD), granular rafts (GR), and the polymer nanopillars (NP).}
% %\label{table1}
% %\end{center}
% \end{table}



%\bibliography{scibib}

%\bibliographystyle{Science}

% Following is a new environment, {scilastnote}, that's defined in the
% preamble and that allows authors to add a reference at the end of the
% list that's not signaled in the text; such references are used in
% *Science* for acknowledgments of funding, help, etc.

% \begin{scilastnote}
% \item We've included in the template file \texttt{scifile.tex} a new
% environment, \texttt{\{scilastnote\}}, that generates a numbered final
% citation without a corresponding signal in the text.  This environment
% can be used to generate a final numbered reference containing
% acknowledgments, sources of funding, and the like, per {\it Science\/}
% style.  Along those lines, we'd like to thank readers of this document
% for their attention, and invite them to address any questions to
% Stewart Wills, at swills@aaas.org.
% \end{scilastnote}




% For your review copy (i.e., the file you initially send in for
% evaluation), you can use the {figure} environment and the
% \includegraphics command to stream your figures into the text, placing
% all figures at the end.  For the final, revised manuscript for
% acceptance and production, however, PostScript or other graphics
% should not be streamed into your compliled file.  Instead, set
% captions as simple paragraphs (with a \noindent tag), setting them
% off from the rest of the text with a \clearpage as shown  below, and
% submit figures as separate files according to the Art Department's
% instructions.


% \clearpage

% \noindent {\bf Fig. 1.} Please do not use figure environments to set
% up your figures in the final (post-peer-review) draft, do not include graphics in your
% source code, and do not cite figures in the text using \LaTeX\
% \verb+\ref+ commands.  Instead, simply refer to the figure numbers in
% the text per {\it Science\/} style, and include the list of captions at
% the end of the document, coded as ordinary paragraphs as shown in the
% \texttt{scifile.tex} template file.  Your actual figure files should
% be submitted separately.



\end{document}




















