% Use only LaTeX2e, calling the article.cls class and 12-point type.

\documentclass[12pt]{article}

% Users of the {thebibliography} environment or BibTeX should use the
% scihhhcite.sty package, downloadable from *Science* at
% www.sciencemag.org/about/authors/prep/TeX_help/ .
% This package should properly format in-text
% reference calls and reference-list numbers.

\usepackage{scicite}
\usepackage{graphicx}% Include figure files
% Use times if you have the font installed; otherwise, comment out the
% following line.
%\usepackage{amsmath}
\usepackage{times}
\usepackage{makecell}
\usepackage{comment}

% These packages aren't part of the template:
\usepackage{amsmath}
\usepackage{cases}
\usepackage{ulem}
\usepackage{color}
\usepackage{xcolor,colortbl}
%\usepackage{german}
\definecolor{r}{rgb}{1,0,0}   
\definecolor{g}{rgb}{0,1,0}   
\definecolor{b}{rgb}{0,0,1}
\newcommand{\TODO}[1]{\textcolor{red}{[#1]}}

\definecolor{variables}{rgb}{1.0,1.0,1.0}
\definecolor{softness}{rgb}{0.55,0.75,0.05}
\definecolor{kappa}{rgb}{0,0.67,0.86}
\definecolor{zeta}{rgb}{0.57,0.44,0.90}
\definecolor{itac}{rgb}{0.95,0.35,0.14}
\definecolor{cd2min}{rgb}{0.98,0.69,0.23}
\definecolor{highlight}{rgb}{0.95,0.95,0.95}


% The preamble here sets up a lot of new/revised commands and
% environments.  It's annoying, but please do *not* try to strip these
% out into a separate .sty file (which could lead to the loss of some
% information when we convert the file to other formats).  Instead, keep
% them in the preamble of your main LaTeX source file.
% The following parameters seem to provide a reasonable page setup.

\topmargin 0.0cm
\oddsidemargin 0.2cm
\textwidth 16cm 
\textheight 21cm
\footskip 1.0cm

%The next command sets up an environment for the abstract to your paper.
\newenvironment{onesentence}{%
\begin{quote} }
{\end{quote}}


\newenvironment{sciabstract}{%
\begin{quote} \bf}
{\end{quote}}

% If your reference list includes text notes as well as references,
% include the following line; otherwise, comment it out.

\renewcommand\refname{References and Notes}



% The following lines set up an environment for the last note in the
% reference list, which commonly includes acknowledgments of funding,
% help, etc.  It's intended for users of BibTeX or the {thebibliography}
% environment.  Users who are hand-coding their references at the end
% using a list environment such as {enumerate} can simply add another
% item at the end, and it will be numbered automatically.

\newcounter{lastnote}

\newenvironment{scilastnote}{%
\setcounter{lastnote}{\value{enumiv}}%
\addtocounter{lastnote}{+1}%
\begin{list}%
{\arabic{lastnote}.}
{\setlength{\leftmargin}{.22in}}
{\setlength{\labelsep}{.5em}}}
{\end{list}}

% Include your paper's title here

%\title{Predicting the brittle-to-ductile transition of amorphous solids via machine learning-informed Structuro-elastoplasticity} 

\title{Machine learning-informed structuro-elastoplasticity predicts ductility of disordered solids} 

% Place the author information here.  Please hand-code the contact
% information and notecalls; do *not* use \footnote commands.  Let the
% author contact information appear immediately below the author names
% as shown.  We would also prefer that you don't change the type-size
% settings shown here.

\author
{
Hongyi Xiao,$^{1,2,3\dagger}$ Ge Zhang,$^{1,4\dagger}$, Entao Yang,$^{5\dagger}$ Robert J. S.  Ivancic,$^{6\dagger}$\\ Sean A. Ridout,$^{1}$ Robert Riggleman,$^{5}$ Douglas J. Durian,$^{1}$ and Andrea J. Liu$^{1\ast}$\\
\\
\normalsize{$^{1}$Department of Physics and Astronomy, University of Pennsylvania, Philadelphia, PA, USA}\\
\normalsize{$^{2}$Institute for Multiscale Simulation, Friedrich-Alexander-Universit\"{a}t Erlangen-N\"{u}rnberg, } 
\\ \normalsize{Erlangen, Germany}
\\ \normalsize{$^{3}$Department of Mechanical Engineering, University of Michigan, Ann Arbor, MI, USA}\\
\normalsize{$^{4}$Department of Physics, City University of Hong Kong, Hong Kong, China}\\
\normalsize{$^{5}$Department of Chemical and Biomolecular Engineering, University of Pennsylvania,}
\\ \normalsize{Philadelphia, PA, USA}\\
\normalsize{$^{6}$Materials Science and Engineering Division, National Institute of Standards and Technology, }
\\ \normalsize{Gaithersburg, MD, USA}\\
%\normalsize{209 S 33rd St, Philadelphia, PA 19104, USA}\\
%\normalsize{$^{2}$Another Unknown Address, Palookaville, ST 99999, USA}\\
\normalsize{$^\dagger$Equal contribution.}
\\
\normalsize{$^\ast$To whom correspondence should be addressed; E-mail:  ajliu@physics.upenn.edu.}
}

% Include the date command, but leave its argument blank.

\date{}

%%%%%%%%%%%%%%%%% END OF PREAMBLE %%%%%%%%%%%%%%%%



\begin{document} 
% Double-space the manuscript.

\baselineskip24pt

% Make the title.

%TC:ignore

\maketitle 

%\begin{onesentence}

%\textbf{One sentence summary}: Theory using a quantity machine-learned from simulation and experiment reveals microscopics controlling ductility in disordered solids.

%\end{onesentence}


% Place your abstract within the special {sciabstract} environment.
\begin{sciabstract}
All solids yield under sufficiently high mechanical loads. Below yield, the mechanical responses of all disordered solids are nearly alike, but above yield every different disordered solid responds in its own way. Brittle systems can shatter without warning, like ordinary window glass, or exhibit strain localization prior to fracture, like metallic or polymeric glasses. Ductile systems, e.g. foams like shaving cream or emulsions like mayonnaise, can flow indefinitely with no strain localization. 
While there are empirical strategies for tuning the degree of strain localization, there is no framework that explains their effectiveness or limitations. We show that Structuro-Elastoplastic (StEP) models provide microscopic understanding of how strain localization depends on the interplay of structure, plasticity and elasticity. 


%We construct StEP models for three model systems, a computer model of an atomic glass, an experimental granular system and a computer model of a polymer glass. We tune each system using a different strategy to exhibit two different levels of strain localization and uncover the microscopic factors responsible. In all cases we reduce the descriptions down to a few microscopic features that control strain localization: the initial structure, near-field and far-field changes of structure due to particle rearrangements and the range of spatial correlations of structure. %reproduces the shear band formation process in brittle deformation and dispersed local plastic rearrangement in ductile deformation for three different systems exhibiting such a transition. The model incorporates structural information by a machine-learned quantity, softness, and is constructed by measuring the interplay between elasticity, softness, and plastic rearrangement. Semi-quantitative agreements of the stress-strain curves and softness statistics emerge, without fitting.
%The StEP model also reveals the underlying mechanisms for the change in ductility, which are initial structure, interactions, and rearrangement sizes, respectively for the three systems.
%\TODO{(DJD) Anything useful/exciting to say about any of the mechanism? - The abstract has been updated}
\end{sciabstract}

% In setting up this template for *Science* papers, we've used both
% the \section* command and the \paragraph* command for topical
% divisions.  Which you use will of course depend on the type of paper
% you're writing.  Review Articles tend to have displayed headings, for
% which \section* is more appropriate; Research Articles, when they have
% formal topical divisions at all, tend to signal them with bold text
% that runs into the paragraph, for which \paragraph* is the right
% choice.  Either way, use the asterisk (*) modifier, as shown, to
% suppress numbering.

%TC:endignore

Disordered solids, such as metallic, molecular, polymeric, nanoparticle or colloidal glasses or granular packings, exhibit plastic behaviors distinct from those of crystalline solids~\cite{falk2011deformation, Greer2013, bonn2017yield, tanguy2021elasto}. %\TODO{DJD: I added RobR's Greer et al. paper; are there other such non-EP reviews we should cite here regarding general phenomenology?}. 
Below yield, these behaviors are surprisingly universal with a consistent value of yield strain~\cite{cubuk2017structure}. Beyond yield, however, some disordered solids, such as foams and certain polymer glasses, are ductile with little or no strain localization, while others, like molecular and metallic glasses, typically exhibit sharp strain localization in shear bands as precursors to brittle failure. 

Several empirical strategies have been discovered for tuning strain localization. Reducing the range of inter-particle attractions~\cite{falk1999molecular, dauchot2011athermal, lin2019distinguishing, xiao2020strain, karmakar2011effect}, equilibrating better~\cite{ozawa2018random}, annealing at slower cooling rates~\cite{shavit2014strain} or cooling while loading~\cite{matsushige1976pressure,lin2019distinguishing} all enhance strain localization. Friction~\cite{karimi2019plastic}, composite constituents~\cite{wang2019stretchable}, particle shape~\cite{zhang2013using} and degree of crystallinity~\cite{dauchot2011athermal} also influence ductility. However, we do not understand why or how these factors influence strain localization. 

%electrochemical properties~\cite{khosrownejad2016model},

At the microscopic scale, plasticity in solids is accomplished by rearrangements in which constituent particles change neighbors. %Plasticity theories typically consider how strain from rearrangements and loading triggers additional rearrangements.  
While other approaches exist~\cite{sollich1997rheology,barlow2020ductile,falk1998dynamics,hinkle2017coarse}, elastoplasticity (EP) models~\cite{nicolas2018deformation} update and record yield strain and strain at each lattice site, corresponding to a coarse-grained region. Such models typically assume an underlying distribution of local yield strains that controls the degree of strain localization~\cite{popovic2018elastoplastic} and is put in by hand~\cite{nicolas2018deformation,tanguy2021elasto}.  

% vandembroucq2011mechanical

Here we follow a different path. %by including local structure into a modified EP model that is informed by particle-level simulations and experiments. 
A particle's local yield strain--and hence its probability to rearrange--depends on its local structural environment~\cite{barbot2018local,liu2021elastoplastic,castellanos2021insights,castellanos4015207history}. We extract a structural predictor of local yield strain, called softness, $S$~\cite{cubuk2015identifying,Schoenholz2016}, using neural networks or support-vector machines, demonstrating that our framework can support any local structural measure that predicts rearrangements or local yield stress (e.g.~\cite{richard2020predicting,bapst2020unveiling,paret2020assessing, boattini2021averaging,fan2021predicting,font2022predicting,jung2022predicting}). Following Zhang \textit{et al.}~\cite{zhang2021interplay,zhang2022structuro}, we unravel the interplay between strain, rearrangements and softness, to incorporate softness into structuro-elasto-plasticity (StEP) models. In contrast to EP models, here the local yield strain distribution \emph{emerges} as a collective property.


%Softness is a weighted sum of the local pair and three-body correlation functions~\cite{Schoenholz2016,zhang2021interplay}. Particles with higher softness tend to have lower yield strains and are therefore more susceptible to rearrangement. For very stable solids, we find that softness does not predict rearrangements well, so we use machine learning to predict the yield strain directly, obtaining a structural predictor $Y$ of the local yield strain. Because the local yield strain determines the propensity of a particle to rearrange, $Y$ also serves as a structural predictor of rearrangements.

%In StEP models, rearrangements trigger other rearrangements either by pushing a particle beyond its local strain (determined by softness $S$), or by altering the local yield strain by altering $S$. The latter occurs in two different ways. First, because a rearrangement changes the particle positions, it modifies $S$ for nearby particles through a \textit{near-field} contribution to the change of softness, $\Delta S(r)$, where $r$ is the distance from the original rearrangement. Second, a rearrangement can be treated as an Eshelby inclusion that exerts a \textit{far-field} strain that decays as a power law in $r$~\cite{maloney2006amorphous, picard2004elastic,albaret2016mapping}.  This long-range effect distorts the local structural environments of even quite distant particles, and is referred to as ``elastic facilitation''~\cite{cubuk2017structure}. 

We develop StEP models for three vastly different systems that each can each be tuned in different ways to exhibit different degrees of strain localization. We demonstrate that the models can be used to gain microscopic insight, opening the door to a quantitative, particle-level approach to engineer advanced structure-property relations in disordered solids. 

\section*{Systems studied}
We examine three systems that differ in dimensionality, temperature, loading condition, and/or interaction potential \cite{sm}. System~I is made of 2D simulated polydisperse circular disks with $1/r_d^{12}$ pairwise repulsions, where $r_d$ is the separation, equilibrated using Monte Carlo swap methods~\cite{ozawa2018random}. Realizations are initially equilibrated at a high temperature $T_a=0.2$ (System~IA) or at a very low temperature of $T_a=0.025$ (System~IB); IA is ductile while IB exhibits strong strain localization and is brittle~\cite{ozawa2018random}. System~II is an experimental granular raft of polydisperse spheres floating at an air-oil interface~\cite{xiao2020strain}. The gravitational capillary length controls the attractive interaction range; in System~IIA the particle diameter $\sigma=1.0 \pm 0.1$~mm is smaller than the capillary length while in IIB the particle diameter $\sigma=3.3 \pm 0.3$~mm exceeds the capillary length. IIB exhibits more strain localization than IIA under tensile strain. Finally, System~III is a three-dimensional polymer nanopillar with chains of coarse-grained particles. The system is simulated at two different temperatures below the glass transition temperature $T_g=0.38$~\cite{lin2019distinguishing, Ivancic2019}, $T=0.3$ (System~IIIA) and $T=0.05$ (IIIB); IIIB exhibits more strain localization under tensile strain than IIIA. For details of all three systems see~\cite{sm}. 

\section*{Softness and yield strain}
For Systems II and III, softness is a weighted sum of local pair and three-body correlation functions obtained using a support vector machine~\cite{Schoenholz2016,zhang2021interplay}. Particles with higher softness tend to have lower yield strains and are therefore more susceptible to rearrangement (Fig.~S1). For details of the definition of softness see~\cite{sm}. The local yield strain $\epsilon_Y$ in poorly annealed jammed packings depends on softness~\cite{zhang2022structuro}, following a Weibull distribution:
%
\begin{eqnarray}
P(\epsilon_Y,S)=\frac{k}{\lambda}\left(\frac{\epsilon_Y}{\lambda}\right)^{k-1}\exp\left[-(\epsilon_Y/\lambda)^{k}\right]
\label{eq:weibull}
\end{eqnarray}
%
where $k(S)$ and $\lambda(S)$ characterize the distribution at each softness $S$ (Figs. S4, S5). For System~III, we allow rearrangements to be triggered also by thermal fluctuations~\cite{sm}.  

For System~IB, the standard definition of softness is not nearly as predictive of rearrangements as local yield stress~\cite{richard2020predicting}. We therefore use ResNet to obtain a structural prediction of local yield stress, which we call $Y_p$, see ~\cite{sm} for details (Figs. S2, S3). Softness for this model is simply
\begin{equation}
S=Y_0 -Y_p, \label{eq:SYdef}
\end{equation}
where $Y_0=12.09$ is the mean value of $Y_p$ for the system initially equilibrated at the higher temperature (System~IA), $T_a=0.2$, before quenching to $T=0$. Because local yield strain is a predictor of rearrangements, $S$ serves as a structural predictor of rearrangements~\cite{sm}. The initial value of the mean, $\langle S\rangle$, and the standard deviation, $\sigma_S$, of the softness distribution for each system is shown in Fig.~\ref{fig:flowChart}.
%This Weibull distribution can also be characterized by the shape parameter $k$ and the mean local yield strain, $\langle \epsilon_Y \rangle=\lambda\, \Gamma(1+1/k)$, where $\Gamma(x)$ is the gamma function.

\section*{Structuro-elastoplastic (StEP) models}
We extend the recently-developed 2D and athermal StEP framework~\cite{zhang2022structuro} to the lattice model depicted in Fig.~\ref{fig:flowChart}. We neglect variations of local elastic constants so stress is proportional to elastic strain at the block and system level. For System~I, the softness of site $i$ directly determines the neural-network-predicted local yield stress with $Y_p=Y_0-S_i$ from Eq.~\ref{eq:SYdef} with a distribution determined by that of the actual local yield strain $\epsilon_{Y,i}$~\cite{sm}. For Systems II and III, we assign a yield strain ranking to each site to determine the yield strain from the $S$-dependent distribution of Eq.~\ref{eq:weibull}. This ranking is reassigned after each plastic event at the site~\cite{zhang2022structuro}. Each block's softness is randomly initialized according to the distribution measured in Systems I-IIIAB.

In addition to softness, each block stores a local deviatoric strain tensor, $\tilde \epsilon$.  Each system is driven by a global strain, $\epsilon_G$, uniformly added to all blocks at each time step in the form of simple shear (System~I) or tensile strain (Systems II and III). This uniform strain deforms local structural environments and therefore changes softness. For all of our systems, the change of softness with each strain step is given by $\Delta S_{\text{load}}=\kappa \Delta |\tilde \epsilon|^2$~\cite{zhang2022structuro}, where $\Delta |\tilde \epsilon|^2$ is the increment of $\tilde \epsilon_{\alpha \beta}\tilde \epsilon_{\beta \alpha}$ in each strain step. 

Block $i$ rearranges (undergoes a plastic event) if its strain $|\tilde{\epsilon}|>\epsilon_Y$, the local yield strain. Because StEP models include not only coupling between plasticity and elasticity, but also with structure, a rearrangement at site $i$ affects other sites $j$ via not only an elastic but also a softness kernel. The elastic kernel is described in the supplementary material~\cite{sm}; it is similar to the one commonly used for the $xy$-strain~\cite{budrikis2013avalanche}, but includes more strain components. For block $i$ itself, the elastic strain is converted to plastic strain $\tilde{\epsilon}_p$.
%\TODO{(YET: what is this $\tilde \epsilon_{ij}$? GZ: Isn't that just the ith row, jth column element of the strain matrix?)}



%
\begin{figure*}[tb!]
%\centering
\begin{minipage}{0.34\textwidth}
\includegraphics[width=\textwidth]{flowchart_table.pdf}
\end{minipage}
\begin{minipage}{0.66\textwidth}
\resizebox{\textwidth}{!}{
\renewcommand{\arraystretch}{1.45}
\begin{tabular}{ |c|| c| c| c |c |c| c|}
\hline
\hline
\rowcolor{variables}
  Parameters & System~IA & System~IB  & System~IIA & System~IIB & System~IIIA & System~IIIB\\
  \hline
 \rowcolor{softness}
 $\langle S \rangle$ & \color{highlight}{0.00} & \color{highlight}{-2.73}  & -0.09 & -0.04 & 0.575 & 0.085\\ \hline
  \rowcolor{softness}
 $\sigma_S$ & \color{highlight}{3.18} & \color{highlight}{2.28}  & 0.78 & 0.70 & 0.70 & 0.72\\ \hline
  \rowcolor{softness}
 $\langle\epsilon_Y\rangle(S)$ & $(12.09-S)/89$ &  $(12.09-S)/89$ & \color{highlight}{$0.013-0.0006S$} & \color{highlight}{$0.012 -0.00245S$} & \makecell{$\max (0.0484$ \\ $ - 0.102S, 0.0382)$} & $0.0484 - 0.0045S$\\ \hline
  \rowcolor{softness}
 $\epsilon_Y$ distribution &\colorbox{softness}{ \makecell{ Gaussian with \\ $STD=3.08$}} &\colorbox{softness}{  \makecell{ Gaussian with \\ $STD=2.20$}}  &\colorbox{softness}{  \makecell{ Weibull with \\ $k=1.9$}} &\colorbox{softness}{  \makecell{ Weibull with \\ $k=2.1$}} &\colorbox{softness}{  \makecell{ Weibull with \\ $k=2.02$}} &\colorbox{softness}{\makecell{ Weibull with \\ $k=1.84$}}\\ \hline
 \rowcolor{itac}
 $\eta(r)$ & $0.78r^{-2.5}$ & $0.78r^{-2.5}$  & \color{highlight}{$0.1e^{-r/1.4}$} &  \color{highlight}{$0.09e^{-r/0.65}$} & \colorbox{itac}{ \makecell{$0.206e^{-r/1.02} $ \\ $+ 0.059$} }& \colorbox{itac}{ \makecell{ $0.162e^{-r/1.45} $ \\ $ + 0.011$}}\\ \hline
 \rowcolor{itac}
 $c$ & -0.325 & -0.325  & \color{highlight}{0.08} & \color{highlight}{1.20} & 0 & 0\\ \hline
  \rowcolor{zeta}
 $\zeta$ & 244 & 226  & 2.37 & 1.74 & 9.17 & 16.60\\ \hline
 \rowcolor{kappa}
 $\kappa$ & 411 & 411  & 139 & 339 & 13.50 & 11.05\\ \hline
 \rowcolor{cd2min}
 $C_{d2min}$ & $1.0e^{-r/1.88}$ & $1.0e^{-r/1.63}$  & $1.0e^{-r/1.13}$ & $1.0e^{-r/1.12}$ & \color{highlight}{$1.40e^{-r/0.75}$} & \color{highlight}{$0.87e^{-r/1.11}$}\\ \hline
 \hline
\end{tabular}
}
\end{minipage}
\caption{
Schematic and parameters of the StEP model. A strain release (plastic rearrangement event) at a given block changes the softness of nearby blocks, and elastically propagates a long-ranged deviatoric strain field. Softness determines the yield strain for each block. A new rearrangement is triggered if the deviatoric strain exceeds the yield strain. In thermal systems, rearrangements can also be triggered by thermal fluctuations, see supplementary materials for details (Fig.~S11). Structural/elastic/plastic/thermal components of the model are in green/gray/orange/brown, respectively.  Each arrow represents an independently-determined equation, and corresponding parameters shown in the table with matching colors. Key parameters for triggering the ductile-to-brittle transition are highlighted in white and are different for each system.
}
\label{fig:flowChart}
\end{figure*}

%\begin{figure*}[tb!]
%\centering
%\includegraphics[width=\textwidth]{flowchart.pdf}
%\caption{
%Schematic of the StEP model. A strain release (plastic rearrangement event) at a given block changes the softness of nearby blocks, and elastically propagates a long-ranged deviatoric strain field. Softness determines the yield strain for each block. A new rearrangement is triggered if the deviatoric strain exceeds the yield strain. In thermal systems, rearrangements can also be triggered by thermal fluctuations, see supplementary materials for details (Fig.~S11). Structural/elastic/plastic/thermal components of the model are in green/blue/orange/purple, respectively.  Each arrow represents an independently-determined equation.
%}
%\label{fig:flowChart}
%\end{figure*}


%
The softness kernel consists of two main pieces. The first contribution, $\Delta S_{\text{n}}$, is a near-field effect from the change of local structure near a rearrangement, which alters softness of nearby particles directly. For all three systems, we find that this contribution tends to restore $S$ to a value close to the local angular average softness, consistent with Ref.~\cite{zhang2021interplay}. The far-field term, $\Delta S_{\text{f}}$, arises from treating a rearrangement as an Eshelby inclusion that exerts a \textit{far-field} strain that decays as a power law in $r$~\cite{maloney2006amorphous, picard2004elastic,albaret2016mapping}.  This ``elastic facilitation" distorts local structural environments via the volumetric strain, $\tilde\epsilon_\mathrm{vol}$ ~\cite{zhang2021interplay,zhang2022structuro}. Fluctuations around this average behavior are approximated with a Gaussian noise term, $\delta(r)$. Altogether, the softness kernel at distance $r$ from a rearrangement is
\begin{equation}
    \Delta S(r, S)= \underbrace{\eta(r) (\langle S \rangle +c - S) + c^\prime }_{\Delta S_{\text{n}}}+ \underbrace{\zeta\tilde{\epsilon}_{\text{vol}} }_{\Delta S_{\text{f}}}+ \delta(r),
    \label{eq:ds_overall}  
\end{equation}
where the parameters $\eta$, $c$ and $\zeta$ are measured from the corresponding simulations or experiments for each of our systems~\cite{sm}. Here $c^\prime$ is a lattice artifact defined as the average of $-c\langle \eta(r) \rangle$ over all sites; it vanishes in the continuum limit. These parameters for all three systems are extracted from particle simulations or experiments (Figs.~S6-S8~\cite{sm}).
 

In standard EP models, the lattice size is the rearrangement size. However, the softness kernel in Eq.~\ref{eq:ds_overall} uses $r$ in units of particle size. To avoid rescaling the kernel, we allow
rearrangements in the StEP model to span several blocks~\cite{zhang2022structuro}. The rearrangement size is characterized by the decay length, $\xi$, of the correlation function of particle non-affine displacement~\cite{falk1998dynamics}, $C_{d2min}$, measured for Systems I-III~\cite{sm}, so we allow  blocks at distance $r$ from the rearranging block to release their elastic strain with a probability, $C(r) \propto \exp(-r/\xi)$ (See Figs.~S9-S10~\cite{sm}).

In summary, there are no manually-adjustable parameters in the StEP model. The parameters that appear in the softness kernel, the mean and variance of the softness distribution and the rearrangement size, are all extracted directly from the particle simulations or experiment and are listed in Fig.~\ref{fig:flowChart} for Systems I-III A-B.


%The oligomer pillars are different from the other two systems because they have non-zero temperatures. 
%Thus, we need to account for the thermal fluctuations, which can also induce rearrangements.
%This is implemented by introducing another rearranging mechanism. 
%As explained before, a block starts rearranging if its elastic deviatoric strain, $\left | \tilde{\epsilon} \right |$, is larger than its local yield strain, $\epsilon_y$. 
%In this thermal system, if this rearranging criteria is not met, the site can also rearrange with a probability of $\nu$:
%
%\begin{equation}
%    \nu(S, \left | \tilde{\epsilon} \right |) = \exp(\Delta F (S) - V_0 G \left | \tilde{\epsilon} \right |^2) 
%    \label{eq:thermal_re}  
%\end{equation}
%
%where $\Delta F$ is the free energy barrier for rearrangements in quiescent system, $V_0$ is the activation volume, $G$ is the local elastic modulus. 
%The local elastic modulus is assumed to be the same for all the particles and equal to the  elastic modulus of the pillars. 
%As shown by previous softness study\cite{Schoenholz2016}, $\Delta F (S)$ has the expression:
%
%\begin{equation}
%    \Delta F (S) = (\epsilon_0 - \epsilon_1 \cdot S) + T \cdot(- (\Sigma_0 - \Sigma_1 \cdot S)).
%    \label{eq:thermal_F}  
%\end{equation}
%
%Here, $\epsilon_i$ and $\Sigma_i$ represent the enthalpic and entropic barrier for rearrangements, which are independent of both the temperature and the softness. 

%Considering we have two rearranging mechanisms in the StEP model, we should expect that their predicted rearranging probabilities are continuous at $\left | \tilde{\epsilon} \right | = \epsilon_y$ for every particle. This gives out another constraint for the thermal yielding criteria:
%
%\begin{equation}
%    V_0 G = \Delta F (S) / \epsilon_y^2.
%    \label{eq:thermal_constiant}  
%\end{equation}
%
%Our results suggest that $\Delta F (S) / \epsilon_y^2$ decreases with softness (see supplementary materials) at both temperatures. 
%This trend meets our expectation because recent work shows that the activation volume, $V_0$, generally decreases with softness. \cite{Yang_2022}

\section*{StEP model predictions}

%The StEP model was applied to three distinct disordered particle systems that exhibit a ductile-to-brittle transition. The first system is 2D simulated athermal and circular disks with a soft interaction potential that is sheared quasi-statically~\cite{richard2020predicting}. Preparations that are initially equilibrated at lower temperatures are more brittle. The second system is an experimental granular raft with a monolayer of polydisperse spheres floating at an air-oil interface~\cite{xiao2020strain}. The interaction range of attraction is controlled by the gravitational capillary length; pillars made of particles larger than this range are more brittle. The third system is a three-dimensional polymer nanopillar that is \TODO{\sout{thermal} equilibrated (?)} at different temperatures below the system glass transition temperature $T_g$ \cite{lin2019distinguishing, Ivancic2019}. The system is more brittle at lower temperatures. \TODO{(DJD) It might flow better to describe the systems first, then the model, then its predictions}

%\begin{figure*}
%	\centering
%	\includegraphics[width=140mm]{figure-disks.png}
%	\caption{Comparison of spatial plasticity field for particle simulations of System~I and StEP models. System~I consists of $N$ polydisperse disks at density $\rho=1$ with $1/r_d^{12}$ pairwise repulsions, where $r_d$ is the separation~\cite{ozawa2018random}. System~IA (first row) has $N=10000$ disks initially equilibrated at $T_a=0.2$, while System~IB (third row) has $N=64000$ disks initially equilibrated at $T_{a}=0.025$, before rapid quenching to $T=0$. The second and fourth rows show results for the StEP models associated with Systems IA and IB, respectively. Each column shows the plasticity fields at a different global shear strain $\epsilon_G$.}
	
%	\label{fig:disks}
%\end{figure*}

\begin{figure*}
	\centering
	\includegraphics[width=0.95\textwidth]{figure-ductile-brittle.png}
	\caption{Comparison of spatial plasticity field for particle results and StEP models.  Each column shows the plasticity fields at a different global shear strain $\epsilon_G$. For each system, the first row shows the particle simulation/experiment result, and the second row shows the StEP results. System~I consists of $N$ polydisperse disks at density $\rho=1$ with $1/r_d^{12}$ pairwise repulsion, where $r_d$ is the separation~\cite{ozawa2018random}. System~IA has $N=10000$ disks initially equilibrated at $T_a=0.2$, while System~IB has $N=64000$ disks initially equilibrated at $T_{a}=0.025$, before rapid quenching to $T=0$. System~II consists of $120\sigma\times60\sigma$ granular rafts of polydisperse Styrofoam spheres of mean diameter $\sigma=1.0 \pm 0.1$\,mm (System~IIA) and $\sigma=3.3 \pm 0.3$\,mm (System~IIB), respectively. The spheres float on an air-oil interface and the rafts are subjected to quasi-static extension as indicated. The corresponding StEP models are computed on a 120$\times$120 grid. System~III consists of polymer nanopillars made of $\approx 4 \times 10^4$ bead-spring polymers, with $5$ monomers per chain at temperature $T=0.30$ for System~IIIA and $T=0.05$ for System~IIIB. Each pillar is a cylinder with a length of $100$ bead diameters with periodic boundary conditions, and radius of $25$ bead diameters. The corresponding StEP models are computed on a 21$\times$21$\times$21 grid in units of bead diameter. }
	\label{fig:systems}
\end{figure*}

The softness kernel, Eq.~\ref{eq:ds_overall}, successfully captures the difference in ductility in all three model systems. We demonstrate this qualitatively in Fig.~\ref{fig:systems} and quantitatively in Fig.~\ref{fig:stress-softness}, which shows the stress-strain curve as well as the mean and the standard deviation of the softness field for all systems. In the particle systems, plasticity is quantified by the non-affine displacement $D^2_{min}$~\cite{falk1998dynamics}, in units of $\sigma^2$, during a short small applied strain interval $\Delta\epsilon_G$, while in the StEP models, it is the plastic strain $|\tilde{\epsilon}_p|$ during the same strain interval. 
The spatial distribution of the accumulated plasticity is shown at different applied global strains $\epsilon_G$ in Fig.~\ref{fig:systems}. For each system, the difference in strain localization in cases A and B is captured well by the StEP models.

For poorly soft repulsive disks quenched from $T_a=0.2$ to $T=0$, (System~IA),  there is no strain localization and the stress-strain curve shows a smooth yielding process with no stress drops in both StEP-model and particle simulations (Fig.~\ref{fig:stress-softness}A). Spatially correlated rearrangements appear at larger $\epsilon_G$, but there is no system-spanning shear band. For the well-annealed~\cite{ozawa2018random,popovic2018elastoplastic,barlow2020ductile} case quenched from $T_a=0.025$ to $T=0$ (IB), the StEP model captures the sharp shear band that emerges at higher strain, along with accompanying the sharp stress drop (Fig.~\ref{fig:stress-softness}A). Features in the stress-strain curves and the softness statistics (Fig.~\ref{fig:stress-softness}A and D) are captured reasonably well by the StEP models, although the StEP models yield at a lower strain than the particle simulations in the brittle case and the stress drop is less pronounced.

For the experimental granular raft pillars (System~II) in the elastic regime with $\epsilon_G=0.1\%$, small plastic events are distributed throughout the system; this can be seen in both the StEP model and the experiment for Systems~IIA and IIB. As $\epsilon_G$ increases to 0.8\%, transient shear bands at 45$^\circ$ to the principal extension direction are apparent in both the StEP and experimental results. At $\epsilon_G=2.5\%$ , system-spanning shear bands appear in both pillars but with a different morphology. For System~IIA, where the particle interaction range exceeds the particle size, the shear bands are composed of rather sparse rearrangements and are transient in the StEP model; in the experiment the shear bands are locked in the same location due to necking of the pillar. For System~IIB, where the interaction range is smaller than the particle size, the shear band is sharper with much more concentrated plastic events than for IIA, indicating greater strain localization and the location of the shear band is fixed in both the StEP model and experiments. The stress-strain curves, and the evolution of the mean and standard deviation of $S$ with strain are captured remarkably well by the StEP models (Fig.~\ref{fig:stress-softness}B and E). The one exception is the standard deviation of softness after the long-lasting shear band forms in the 3.3~mm particle pillar (System~IIB). This is expected since $S$ is not trained for large fractures in the particle packing.

%\begin{figure*}
%	\centering
%	\includegraphics[width=140mm]{figure-granular.png}
%	\caption{Comparison of the spatial plasticity field for experiments on System~II and StEP models. System~II consists of $120\sigma\times60\sigma$ granular rafts of polydisperse Styrofoam spheres of mean diameter $\sigma=1.0 \pm 0.1$\,mm (System~IIA, first row) and $\sigma=3.3 \pm 0.3$\,mm (System~IIB, third row), respectively. The spheres float on an air-oil interface and the rafts are subjected to quasi-static extension as indicated.  The corresponding StEP models (second and fourth rows) are computed on a 120$\times$120 grid. Each column shows plasticity fields at a different value of global tensile strain $\epsilon_G$.}
%	\label{fig:granular}
%\end{figure*}


%\begin{figure*}
%	\centering
%	\includegraphics[width=140mm]{figure-polymer.png}
%	\caption{Comparison of the spatial plasticity field for simulations of System~III and StEP models. System~III consists of polymer nanopillars made of $\approx 4 \times 10^4$ bead-spring polymers, with $5$ monomers per chain at temperature $T=0.30$ for System~IIIA (first row) and $T=0.05$ for System~IIIB (third row). Each pillar is a cylinder with a length of $100$ bead diameters with periodic boundary conditions, and radius of $25$ bead diameters. The corresponding StEP models are computed on a 21$\times$21$\times$21 grid in units of bead diameter. The second and fourth rows show the plasticity fields for StEP models corresponding to IIIA and IIIB, respectively. Each column shows the plasticity fields at a different global tensile strain $\epsilon_G$.}
%	\label{fig:polymers}
%\end{figure*}

%For the polymer nanopillars of System~III, we compare plasticity in the StEP model and particle simulations in Fig.~\ref{fig:systems}. 
For the thermal and 3D polymer nanopillar systems (Fig.~\ref{fig:systems}), 
%Each column represents a global strain value, $\epsilon_G$. We started from $\epsilon_G=4\%$ and used a lag strain of $0.5\%$ to calculate the accumulated plasticity. 
%In other words, for results of $\epsilon_G=4\%$, we used the trajectory from $\epsilon_G=3.5\%$ to $4.0\%$ for the calculation.
%In the MD simulation, we used $D^2_{min}$ as the measurement of plasticity, which is a well-established method for quantifying strain localization. \cite{falk1998dynamics, Ivancic2019}
%While in the StEP model, we use the accumulated plastic strain on each block, which equals to the elastic devaitoric strain released by the rearrangements.
the StEP model still captures the transition from ductile to brittle behaviors. At large $\epsilon_G$, isolated rearrangements are seen for the higher temperature system at $T=0.3$ (IIIA), while strain localization and shear banding occur for the lower temperature case at $T=0.05$ (IIIB) for both the StEP model and simulations. Quantitative comparisons (Fig.~\ref{fig:stress-softness}C and F) show that stress reaches a higher value in the StEP model, but differences between IIIA and IIIB are captured well. 


%\begin{figure*}
%	\centering
%	\includegraphics[width=160mm]{Strain-stress-softness}
%	\caption{Quantitative comparison between particle simulation/experiment results and StEP models. Cases (I-III)A are denoted by red curves and cases (I-III)B by blue curves. Darker shades correspond to the simulations/experiments while lighter shades correspond to the StEP model. First row:  stress-strain curves; stress is calculated as the product of elastic strain modulus summed over all sites and global modulus measured in the corresponding system. Second row: average softness vs.~strain; insets show standard deviation of the softness distribution vs.~strain.}
%	\label{fig:stress-softness}
%\end{figure*}

\begin{figure*}
	\centering
	\includegraphics[width=160mm]{Strain-stress-softness-params}
	\caption{Quantitative comparison between particle simulation/experiment results and StEP models. Cases (I-III)A are denoted by red curves and cases (I-III)B by blue curves. Darker shades correspond to the simulations/experiments while lighter shades correspond to the StEP model. First row:  stress-strain curves; stress is calculated as the product of elastic strain modulus summed over all sites and global modulus measured in the corresponding system. Second row: average softness vs.~strain; insets show standard deviation of the softness distribution vs.~strain. In G-I, the stress-strain curve for the StEP models for Case B is shown as a thick solid black line. For each microscopic factor characterizing the StEP models, the corresponding parameters in Fig.~\ref{fig:flowChart} were varied from their values for Case B to those for Case A. 
	%Only one or two affect the stress-strain curves appreciably, allowing us to extract the microscopic physics that controls strain localization, leading to differences between Cases A and B.
	}
	\label{fig:stress-softness}
\end{figure*}

Systems~I-III differ in the size of the constituent particles (and hence importance of temperature), inter-particle interactions, system dimension, importance of friction, and preparation history. Yet in all 6 cases, the StEP models describe the simulations or experiments quantitatively, indicating that they capture the salient microscopic differences in the interplay between elasticity, plasticity and the disordered structure.


\section*{Elucidating how microscopic mechanisms control ductility}

Although preparation history, interaction range and temperature are known to affect strain localization, it is not understood \emph{why} these factors are important. Unlike EP models, StEP models allow us to gain insight by transmuting these factors into the softness kernel and the softness distribution. In particular, the softness kernel is characterized by near-field and far-field contributions in Eq.~\ref{eq:ds_overall}.  By exploring how these two terms, the elastic kernel, the softness distribution and rearrangement size vary between the A and B variations within each system, as well as among our three model systems, we can gain insight into the underlying microscopic mechanisms that control ductility. 

For System~I, we see from the table in Fig.~\ref{fig:flowChart} that Eq.~\ref{eq:ds_overall} is identical for IA and IB. The only striking difference is in the initial mean softness, $\langle S \rangle$. The initial standard deviation of the softness distribution, the relation between $S$ and the yield strain $\epsilon_Y$, and the size of rearrangements are also slightly different. To determine the significance of each of these factors, we start with System~IB (bold black curve in Fig.~\ref{fig:stress-softness}G) and systematically vary each of these parameters (colored curves) one at a time from its value in IB (brittle) to its value in IA (ductile) and assess the effect on the stress-strain curve to see which parameter has the strongest influence. Fig.~\ref{fig:stress-softness}G shows that the only factor that qualitatively affects the stress-strain curve is the initial distribution of $S$. Switching the initial $S$ distribution to its value in the ductile case of System~IA removes the large stress drop in the stress-strain curve, rendering the system ductile. This implies that the only significant difference between Systems IA and IB is the initial softness. This makes sense; the only difference between Systems IA and IB is in the preparation history, which affects the initial value of $S$. Brittle systems obtained by annealing at lower temperature are more stable~\cite{ozawa2018random,berthier2019zero,singh2020brittle}, leading to lower $S$ and hence higher values of the local yield strain from Eq.~\ref{eq:SYdef}. System~I shows that extracting StEP models from particle-level data can lead to correct identification of the microscopic factors controlling strain localization.


%\begin{figure*}
%	\centering
%	\includegraphics[width=85mm]{Vary_params}
%	\caption{Stress-strain curves for StEP models corresponding to System~I (A), II (B) and III (C). In each panel, the stress-strain curve for the StEP models for Case B is shown as a thick solid black line. For each microscopic factor characterizing the StEP models (e.g. range of near-field softness change), we vary the corresponding parameters in Table~\ref{table:params} from their values for Case B to those for Case A. Only one or two affect the stress-strain curves appreciably, allowing us to extract the microscopic physics that controls strain localization, leading to differences between Cases A and B.}
%	\label{vary-params}
%\end{figure*}

%\textbf{For the soft disks, the determining factor is the initial softness distribution, which directly captures the effect of preparation history. The more brittle systems annealed at lower temperatures are conventionally believed to be more stable; this is quantified in the StEP framework as having a lower average softness. Switching the initial softness distribution to that of the poorly annealed system removes the sharp stress drop and make the system ductile, as shown in Fig.~\ref{vary-params}. HX: should this entire paragraph be deleted? GZ: I agree.}

%For the granular pillars, changing two of the parameters affects the stress-strain curve. First, reducing the dependence of the yield strain (see supplementary for comparison) on softness flattens the stress-strain curve. Second, the near-field softness changes caused by rearrangement, \textit{i.e.}, the first term in Eq.~(\ref{eq:ds_overall}), is different. The ductile case have a larger decay length of $\eta(r)$ and a smaller value of $c$. In other words, rearrangements in the brittle system with shorter particle interaction range tend to bring more localized and higher softness increase to its surrounding, which, coupled with higher sensitivity of the yield strain to softness, leads to the brittle behavior.

For the granular pillars, System~II, the values of nearly all of the parameters differ from System~IIA to IIB. However, by changing them individually (Fig.~\ref{fig:stress-softness}H) from their values for System~IIB (bold black curve) to those for System~IIA, we can see that only two sets of parameters make a significant difference to the stress-strain curve. First, one can flatten the curve by reducing the dependence of the yield strain on softness, characterized by the Weibull exponent $k$ and mode $\lambda(S)$. Second, the near-field softness change caused by rearrangement, \textit{i.e.}, the first term in Eq.~(\ref{eq:ds_overall}), is important. This near-field effect is characterized by the range $\eta (r)$ and offset $c$. System~IIA has a larger decay length of $\eta (r)$ and a smaller value of $c$. In other words, rearrangements in System~IIB, which has a shorter particle interaction range, alter local yield strain more strongly over a shorter distance.  The range of the near-field softness change is comparable to the interaction range. 

\newcommand{\dtwomincorr}[0]{\ensuremath{\xi}}
The values of most of the parameters for the polymer nanopillar StEP models differ from System~IIIA to IIIB (table in Fig.~\ref{fig:flowChart}), but we change them one by one from their values for System~IIIB to their values for System~IIIA in Fig.~\ref{fig:stress-softness}I. None of them affect the stress-strain curve appreciably except rearrangement size, $\dtwomincorr$. The rearrangements in the low temperature system, System~IIIB, have a larger size $\dtwomincorr$, allowing for more facilitation.
%This could also be a result of the thermal fluctuation shielding the mechanical triggering signals of the center rearranging particle. 


\section*{Discussion}

For three systems, we have shown that structuro-elasto-plasticity models provide physical insight into the microscopic mechanisms that govern their ductility.  The systems were designed to be very different, and we find that the microscopic mechanisms underlying the degree of strain localization are also different. For simulated glasses with different preparation histories, we find that it is the initial distribution of softness, and hence of local yield stress, that controls ductility. For experimental granular pillars with different interaction ranges, it is the sensitivity of local yield strain on local structure and the near-field change of softness due to rearrangements that change, with a larger change of local yield stress occurring over a shorter range in systems with higher strain localization. 
%Finally, for simulated polymer glass nanopillars, we find that static spatial correlations are somewhat longer-ranged in cooler systems, leading to larger rearrangements and greater strain localization.
Finally, for the simulated polymer glass nanopillars, we find that the rearrangement size is larger in cooler systems, allowing for more facilitation and leading to greater strain localization. These insights would not be possible if we had not been able to connect local structure with rearrangement propensity. The introduction of the softness field is critical.

%\cite{harrington2019machine}
Our results raise the possibility that StEP models could be used to design materials with desired ductility by optimizing over multiple controllable factors. This would be a substantial improvement over empirical approaches. This hope does not seem unrealistic; softness has been proved to be highly predictive of rearrangements in a wide range of disordered solids~\cite{cubuk2017structure, yang2022understanding}. Moreover, the StEP model framework is highly adaptable, accommodating any structural predictor of rearrangements, as we have shown by introducing a structural predictor of local yield stress obtained using image classification methods instead of the more standard predictor of rearrangements based on a weighted average of two- and three-point structure functions.

It is straightforward to extend StEP models to include dynamics by using time-dependent softness and elastic kernels~\cite{liu2021elastoplastic}, in order to capture rheology. Like EP models, StEP models can also readily be extended to include local elastic moduli and other effects~\cite{nicolas2018deformation}.

%TC:ignore

\section*{Acknowledgements}

We thank Misaki Ozawa for providing the initial soft-disk configurations used in this study. \textbf{Funding}: This work was supported by the National Science Foundation through grant MRSEC/DMR-1720530 (D.J.D., R.A.R., H.X., E.Y., G.Z.), and the Simons Foundation via the ``Cracking the glass problem" collaboration \#45945 (S.A.R., A.J.L.) and Investigator Award \#327939 (A.J.L.). The Extreme Science and Engineering Discovery Environment (XSEDE), supported by National Science Foundation grant number ACI-1548562. This work used the XSEDE Stampede2 at the Texas Advanced Computing Center through allocation TG-DMR150034 (E.Y., R.J.S.I, R.A.R.). \textbf{Author contributions}: G.Z. performed simulations for System~I, H.X. performed experiments for System~II, R.J.S.I. performed simulations for System~III, G.Z., H.X., E.Y., R.J.S.I., S.A.R. performed analysis to data and developed the StEP model predictions. G.Z., H.X., R.J.S.I, E.Y. drafted the original manuscript, all participated in reviewing and editing, A.J.L, D.J.D, R.A.R. were responsible for conceptualization, funding acquisition, and supervision. \textbf{Competing interests}: the authors have no competing interests. \textbf{Materials availability}: Materials are available upon request to ajliu@physics.upenn.edu.

\nocite{ozawa2020role}
%\nocite{rocks2021learning}
%\nocite{behler2007generalized}
%\nocite{yang2020effect}
\nocite{zagoruyko2016wide}
\nocite{nicolas2013mesoscopic}
\nocite{Yang_2022}

\bibliography{scibib}
\bibliographystyle{Science}
%TC:endignore

%\break
\begin{comment}
\begin{table}[hbt!]
\caption{Summary of StEP model parameters measured from the corresponding systems. Key parameters that dominate the ductility in each system are highlighted in red.}
\vspace{+5pt}
\label{table:params}
\resizebox{\textwidth}{!}{
\begin{tabular}{ |c|| c| c| c |c |c| c|}
\hline
  Parameters & System~IA & System~IB  & System~IIA & System~IIB & System~IIIA & System~IIIB\\
  \hline
 $\langle S \rangle$ & \color{red}{0.00} & \color{red}{-2.73}  & -0.09 & -0.04 & 0.575 & 0.085\\ \hline
 $STD(S)$ & \color{red}{3.18} & \color{red}{2.28}  & 0.78 & 0.70 & 0.70 & 0.72\\ \hline
 $\langle\epsilon_Y\rangle(S)$ & $(12.09-S)/89$ &  $(12.09-S)/89$ & \color{red}{$0.013-0.0006S$} & \color{red}{$0.012 -0.00245S$} & $\max (0.0484 - 0.102S,  0.0382)$ & $0.0484 - 0.0045S$\\ \hline
 $\epsilon_Y$ distribution & \makecell{ Gaussian with \\ $STD=3.08$} & \makecell{ Gaussian with \\ $STD=2.20$}  & \makecell{ Weibull with \\ $k=1.9$} & \makecell{ Weibull with \\ $k=2.1$} & \makecell{ Weibull with \\ $k=2.02$} & \makecell{ Weibull with \\ $k=1.84$}\\ \hline
 $\eta(r)$ & $0.78r^{-2.5}$ & $0.78r^{-2.5}$  & \color{red}{$0.1e^{-r/1.4}$} & \color{red}{$0.09e^{-r/0.65}$} & $0.206e^{-r/1.02} + 0.059$ & $0.162e^{-r/1.45} + 0.011$\\ \hline
 $c$ & -0.325 & -0.325  & \color{red}{0.08} & \color{red}{1.20} & 0 & 0\\ \hline
 $\kappa$ & 411 & 411  & 139 & 339 & 13.50 & 11.05\\ \hline
 $\zeta$ & 244 & 226  & 2.37 & 1.74 & 9.17 & 16.60\\ \hline
 $C_{d2min}$ & $1.0e^{-r/1.88}$ & $1.0e^{-r/1.63}$  & $1.0e^{-r/1.13}$ & $1.0e^{-r/1.12}$ & \color{red}{$1.40e^{-r/0.75}$} & \color{red}{$0.87e^{-r/1.11}$}\\ \hline
 \hline
\end{tabular}
}
\end{table}
\end{comment}

% Following is a new environment, {scilastnote}, that's defined in the
% preamble and that allows authors to add a reference at the end of the
% list that's not signaled in the text; such references are used in
% *Science* for acknowledgments of funding, help, etc.

%\begin{scilastnote}

% \item We've included in the template file \texttt{scifile.tex} a new
% environment, \texttt{\{scilastnote\}}, that generates a numbered final
% citation without a corresponding signal in the text.  This environment
% can be used to generate a final numbered reference containing
% acknowledgments, sources of funding, and the like, per {\it Science\/}
% style.  Along those lines, we'd like to thank readers of this document
% for their attention, and invite them to address any questions to
% Stewart Wills, at swills@aaas.org.
%\end{scilastnote}




% For your review copy (i.e., the file you initially send in for
% evaluation), you can use the {figure} environment and the
% \includegraphics command to stream your figures into the text, placing
% all figures at the end.  For the final, revised manuscript for
% acceptance and production, however, PostScript or other graphics
% should not be streamed into your compliled file.  Instead, set
% captions as simple paragraphs (with a \noindent tag), setting them
% off from the rest of the text with a \clearpage as shown  below, and
% submit figures as separate files according to the Art Department's
% instructions.


% \clearpage

% \noindent {\bf Fig. 1.} Please do not use figure environments to set
% up your figures in the final (post-peer-review) draft, do not include graphics in your
% source code, and do not cite figures in the text using \LaTeX\
% \verb+\ref+ commands.  Instead, simply refer to the figure numbers in
% the text per {\it Science\/} style, and include the list of captions at
% the end of the document, coded as ordinary paragraphs as shown in the
% \texttt{scifile.tex} template file.  Your actual figure files should
% be submitted separately.



\end{document}




















