%!TEX root = graphsc.tex

\section{Related Work}
\label{sec:related}

\ptitle{Point cloud correspondence.}
%
Extracting accurate correspondences between point clouds plays a crucial role in computer vision and graphics tasks.
Detection-based methods first extract geometrically-discriminative keypoints and their descriptors, either with hand-crafted~\cite{johnson1999using,rusu2008aligning,rusu2009fast,tombari2010unique} or learning-based~\cite{zeng20173dmatch,deng2018ppfnet,deng2018ppf,gojcic2019perfect,choy2019fully,bai2020d3feat,huang2021predator,ao2021spinnet} descriptors, which are then matched as correspondences.
However, it is difficult to detect repeatable keypoints between point clouds, especially in low-overlap cases, such that detection-based methods still suffer from low inlier ratio.
Recently, detection-free methods~\cite{yu2021cofinet,qin2022geometric,li2022lepard} bypass keypoint detection by considering all possible point pairs in a coarse-to-fine matching pipeline, which significantly improves matching and registration accuracy.
There are also methods dedicated to non-rigid matching by explicitly modeling shape deformation~\cite{groueix20183d,trappolini2021shape,saleh2022bending} or leveraging functional maps~\cite{ovsjanikov2012functional,litany2017deep,donati2020deep}.
And scene flow estimation methods~\cite{liu2019flownet3d,wu2020pointpwc,puy2020flot,teed2021raft} predict the frame-to-frame motion of points in the scene.
Although great progress has been made, existing methods are still prone to outliers, which significantly harms the registration performance.


\ptitle{Non-rigid registration.}
%
To describe the non-rigid deformation, the warping function can be formulated into different representations, \eg, dense displacement field~\cite{li2022lepard}, dense affine transformation field~\cite{li2022non}, and embedded deformation graph~\cite{sumner2007embedded}.
Dense displacement field computes a 3D motion vector for each point in the scene, which is the most direct way to represent deformation.
Dense affine transformation field computes an affine transformation for each point, which can better model complex deformation.
Neural Deformation Pyramid~\cite{li2022non} establishes a hierarchical dense affine transformation field with multiple MLPs for coarse-to-fine non-rigid registration.
And embedded deformation graph~\cite{sumner2007embedded} parameterizes the deformation with a set of graph nodes connected with undirected edges, where each node is associated with an affine transformation.
This can be efficiently solved by the Non-rigid Iterative Closest Point (N-ICP) algorithm~\cite{li2008global}.
NNRT~\cite{bozic2020neural} proposes a differentiable N-ICP solver for end-to-end training, and~\cite{bozic2021neural} learns a deformation graph in a data-driven manner.
There are still other warping function formulations, and we refer the readers to~\cite{deng2022survey} for more details.

\ptitle{Outlier rejection for point cloud registration.}
%
Pruning outliers in rigid registration has been broadly studied. The most popular methods are RANSAC~\cite{fischler1981random} and its variants~\cite{chum2003locally,barath2018graph,barath2020magsac++}, which solve for the rigid transformation in a hypothesize-and-verify manner.
However, they suffer from slow convergence and could degenerate under high outlier ratio.
Other methods~\cite{leordeanu2005spectral,chen2022sc2} leverage spatial consistency to suppress outliers. Recent learning-based methods~\cite{choy2020deep,pais20203dregnet,bai2021pointdsc} filter outliers with a neural network.
PointDSC~\cite{bai2021pointdsc} designs a spatial consistency non-local module to prune outliers and attains promising rigid registration performance.
Nevertheless, due to complex deformations, similar success has yet to be achieved in non-rigid registration.
A closely related work to ours is~\cite{huang2008non}, which extends the traditional spectral matching~\cite{leordeanu2005spectral} technique to geodesic space. However, it computes pairwise geodesic distances between correspondences, which is time-consuming. And geodesic distance could be erroneous and unstable due to occlusion.
In this work, we fill this gap with \ours{} for efficient and accurate non-rigid outlier pruning.
