\section{Design of Robot Experiments}
We now describe our experiments for testing the local planners on a physical robot. We use a ClearPath Robotics Jackal robot~\cite{JackalDatasheet} equipped with LiDAR, set to a scanning frequency of 5Hz. 
The experiments cover a range of scenarios that an autonomous robot would encounter in the physical world.
%and is required to traverse successfully.
A failed traversal translates into a robot's collision with a static obstacle (e.g. wall), or a dynamic obstacle (e.g. pedestrian).
Moreover, if the planner fails to complete the global plan, then the robot fails that scenario. In addition to simply measuring success/failure, we are also interested in the nature of the trajectory produced by each approach. Is it smooth? How does the robot react to an obstacle? 

\textbf{Test cases.} The experiments were conducted in a facility that includes an open room and a maze component with tight corners and narrow doorways shown in Fig.\ref{fig:test_cases}. We refer to the maze shown in the first two images of Fig.\ref{fig:test_cases} as the \emph{UNIX maze room} (named after letters that make up obstacles in four separate rooms). We describe four test cases:

\begin{figure}[!t]
	\centering
	\subfigure{\label{fig:test1}
		\includegraphics[width=0.15\textwidth]{test1_zoom.jpg}}\hspace{-1mm}
	\subfigure{\label{fig:test2}
		\includegraphics[width=0.15\textwidth]{test2_zoom.jpg}}\hspace{-1mm}	
	\subfigure{\label{fig:test3}
		\includegraphics[width=0.15\textwidth]{test3_zoom.jpg}}\hspace{-1mm}	
	\vspace{-2mm}
	\caption{Robot experiment test cases.}\label{fig:test_cases}
\end{figure}

\noindent $\bullet$ \textbf{(C1) Room I to room N through doorway:} shown in Fig.\ref{fig:test_cases} (left). Here the robot's task is to travel through a narrow doorway while making a 180-degree turn. In this case, all the obstacles are fixed and included in the global map, and so the only job of the local planner is to follow the global plan (which will be a collision-free path from start to goal) as closely as possible. However, in order to make the turn smoothly the planner must maintain a small turn radius (the ratio between linear and angular velocity).\\
$\bullet$ \textbf{(C2) Room I to room X with ``unexpected'' static obstacle:} shown in Fig.\ref{fig:test_cases} (mid). In this experiment, the robot goal is selected before the obstacle is in place. After the goal is selected and the global plan is computed, a static obstacle (a cardboard cutout of a person) is placed in the robot's global plan. 
%The points on the global plan that are covered by the obstacle are not eligible to be waypoints.
As the robot nears the obstacle, the next eligible waypoint will be beyond the obstacle and the local planner will need to navigate round the obstacle.\\
$\bullet$ \textbf{(C3) Avoiding a walking pedestrian on a straight path:} shown in Fig.\ref{fig:test_cases} (right). Here the robot must traverse a straight path while a pedestrian is walking towards the robot. This case tests the local planner's ability to detect and navigate around a moving object. For this experiment it would always be possible to generate an ``unavoidable collision'' by having the pedestrian walk quickly at high speed into the robot. To avoid this we ask the pedestrian to stop when they are right in front of the robot. The desired behavior is then for the robot to back up or turn round the pedestrian. The undesired behavior is to keep on moving forward into the pedestrian.\\ 
\noindent $\bullet$ \textbf{(C4) Pedestrian crossing the robot path:} We extend the previous test case (C3) by asking the pedestrian to perpendicularly cross the robot's global path. The desired behavior is for the robot to wait and then continue after the pedestrian has crossed. 

\noindent \textbf{Local planners.} 
We test with the DrQ variant of SAC since it had the best training performance of all the RL algorithms in Section~\ref{sec:training}. We log the trajectories for the resulting {\em SACPlanner} and compare against the Dynamic Window Approach (DWA), as well as the Shortest Path (SP) planner discussed in Section~\ref{sec:previous} that always tries to get to the next waypoint using a shortest path in the Occupancy Grid. 
%DWA planner picks a candidate trajectory from the set of possible paths and abides to the picked trajectory's velocity vector. DWA planner persistently scans for the next best trajectory as the robot moves along the global path. 
%The shortest path planner uses the waypoint generator to cut the global path into several local paths. As the name implies, the goal is to reach the next waypoint by traveling the shortest distance between the current point and the next waypoint.
\iffalse
We test all planners on a Jackal robot fitted with a LiDAR scanner for sampling the current state.
As mentioned in the introduction, we upload the planners' modules to a central cloud computing node that contains ROS's navigation and local planning modules (along with the utility modules).
The planner communicates wirelessly with the robot's motors in order to execute the action at each timestep. 
For all experiments and planners, we set a timestep's duration to be $0.2$ seconds. In addition, we set the LiDAR's scanning frequency to be $5$ Hz and the local planner's inflation layer cost scaling factor and inflation radius to be $0.5$. 
For each of the planners' results, we show the global paths for $2$ runs given the same starting and goal points in the three experiments. For the static obstacle and walking pedestrian's cases, we note that the global plan is determined by the waypoint generator \emph{before} the pedestrian is detected by the local planner.
\fi 
