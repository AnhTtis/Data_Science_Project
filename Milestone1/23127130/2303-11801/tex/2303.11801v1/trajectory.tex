\subsection{Trajectory Analysis}\label{sec:trajectory}
In order to understand more deeply the difference in behavior of DWA and SACPlanner,  Fig.\ref{fig:ped_ex} depicts a single run from test case (C3). 
%We analyze the trajectories from log data to understand how the velocity action commands translate into the actual movements by the test case (C3) pictured in Fig.\ref{fig:test3} run by the DWA and SAC agents as in Fig.\ref{fig:ped_ex}. 
Fig.\ref{fig:ped_traj} shows the trajectory, Fig.\ref{fig:ped_linv} plots the linear velocity, Fig.\ref{fig:ped_mdist} shows the distance to the nearest `front obstacle' (within $\pm\frac{\pi}{4}$rad range from the current yaw), and Fig.\ref{fig:ped_angv} plots the angular velocity. 

The key feature of these plots is that when the pedestrian is close, DWA slows down and turns a little, whereas SACPlanner goes into reverse (note the blue color in Fig.\ref{fig:ped_linv}\&(c)) and turns a lot so as to go around the pedestrian. This ``reactiveness'' to obstacles also manifests in more turning even when the robot can go in a straight line. 

%Fig.  each agent's $(x,y)$ location (in meters; $m$) of trajectory between the start and goal positions are displayed with surrounding obstacles such as walls/poles in the global map. The nearest location of the pedestrian to the robot is spotted by a gray blob in the local costmap like a stain which is detected among his all steps approaching toward the robot from the front until the agent stops or backs-off. A colored trajectory indicates the transitional velocity action values to show the forward movements along the path with either a full speed (in red), slowing down (yellow-green) or backwards (blue) as in Fig.\ref{fig:ped_linv}. We truncated a few first and last samples since the departure or parking behavior is not of interest to this analysis. The DWA agent detected him early enough and slowed down for a few seconds, but decided to move anyway with a full speed then ended up a collision to the pedestrian. The SAC agent, on the other hand, detoured after backing off before a collision (which indicates a `correctional behavior') when it first saw him. Fig.\ref{fig:ped_mdist} shows the distance of each agent to the nearest `front' (within $\pm\frac{\pi}{4}$rad range from current face-yaw) obstacle over time (in seconds; $s$) with the same color scheme as the transitional velocity action values. Since the wall is at right-hand side in diagonal within this front-angle range from the view of a forward facing robot, the distance value showed approximately $2m$ when each agent traversed in the straight forward direction. The area indicated by each gray block is the duration until each agent eventually gets back to traverse the original path after encountering the pedestrian (which is calibrated by detecting a blob stain in the $8m\times8m$ local costmap). The pedestrian had to step aside after the collision with the DWA agent to clear the rest of pathway, its total travel time is shorter than that of the SAC agent. The vertical lower `risk' bound of $0.4m$ indicates that the nearest front obstacle is close enough to be reached in $1s$ if robot is now moving with a full forward speed considering the inertia. The corresponding rotational velocity action values over the traverse time are in Fig.\ref{fig:ped_angv} with the red for the full speed left turn versus blue for the right, also pointed by the direction of each arrowhead. Fig.\ref{fig:ped_mdist} and \ref{fig:ped_angv} together show that the DWA agent failed to detour or back-off even with a full right turn around at $6$-$8s$, but then started moving forward therefore hitting the pedestrian. At around $9s$, the distance to the nearest obstacle of the DWA agent suddenly jumped up to $1.5m$ as if it lost the sight of him in the local costmap which was a mistake. The rotational velocity action commands of the SAC agent jitters more causing the jerkiness when moving forward with a full transitional speed compared to that of the DWA agent being relatively stable pattern as in Fig.\ref{fig:ped_angv}, but this proved the advantage of agile reactions by the SAC agent upon the appearance of an unexpected interruption along the planned path.


