\section{Experimental Results}
The robot trajectories for each of (C1)-(C4) are shown in Fig.\ref{fig:case_comp}. We denote the start and goal along with the collision points. For (C1)\&(C2) we swap the direction of travel for half the runs. The color of the trajectory represents linear velocity. We also show the Occupancy Grid values in gray (taken from the map and the LiDAR). For (C3)\&(C4) with a dynamic obstacle the gray shading captures all the positions of the obstacle over time. The 3 local planners have qualitatively different behavior which we now describe in detail for each case. 
%Considering the aforementioned characteristics of the movements in trajectory by each agent, we summarize here the performance of all $10$ robot runs driven by each agent using the DWA, Shortest Path (SP) and trained by the SAC algorithms for each test case (C1)-(C3) as in Fig.\ref{fig:case_comp} with the same color scheme. For (C1)\&(C2), we have switched the position between departure and goal for half times to investigate differences in left or right sharp turns in a narrow pathway.
\begin{figure*}[ht]
	\centering
	\subfigure{\label{fig:case1_comp}
		\includegraphics[width=0.24\textwidth]{case1_comp.png}}\hspace{-1mm}
	\subfigure{\label{fig:case2_comp}
		\includegraphics[width=0.24\textwidth]{case2_comp.png}}\hspace{-1mm}	
	\subfigure{\label{fig:case3_comp}
		\includegraphics[width=0.24\textwidth]{case3_comp.png}}\hspace{-1mm}	
	\subfigure{\label{fig:case4_comp}
		\includegraphics[width=0.24\textwidth]{case4_comp.png}}	
	\vspace{-2mm}
	\caption{Trajectory comparison between DWA, Shortest Path (SP) vs. SAC agent for each test case.}\label{fig:case_comp}	
\end{figure*}
\begin{figure*}[!th]
	\centering
	\subfigure{\label{fig:ped_traj}
	\includegraphics[width=0.22\textwidth]{ped_traj.png}}	\hspace{-3mm}
	\subfigure{\label{fig:ped_linv}
	\includegraphics[width=0.22\textwidth]{ped_linv.png}}	\hspace{-3mm}
	\subfigure{\label{fig:ped_mdist}
	\includegraphics[width=0.22\textwidth]{ped_mdist.png}}	\hspace{-3mm}
	\subfigure{\label{fig:ped_angv}
	\includegraphics[width=0.22\textwidth]{ped_angv.png}}	
\vspace{-3mm}
	\caption{Trajectory comparison between DWA and SACPlanner based on logs from the test case (C3).}\label{fig:ped_ex}
\end{figure*}

\noindent $\bullet$ \textbf{(C1):} DWA (which generates circular arcs) has the smoothest trajectory through the door. However, when starting at the top it miscalculated the best turning radius and aborted next to the `N' obstacle each time.
%collided to the wall in room N after a sharp left turn through the door due to the too fast speed for all $5$ times when it departed at upper $y$, and could not finish the whole journey by getting an abort control from the ROS. But it showed smooth trajectories in all the rest $5$ runs in the reverse traverse by getting to the goal successfully. 
The SP planner never collided with an obstacle and (not surprisingly since it was running shortest paths on a grid) it traveled in a series of straight lines (whose endpoints are denoted with green dots).  
%in all $10$ times, but it showed nearly `discrete' behavior selecting $3$ different types of actions in a sequence  `go-stop-turn' (like a machinery movement) as indicated by the straight lines between each two consecutive green dots in all trajectories.
SACPlanner was also successful in all cases. However, it had to ``back off'' multiple times  (denoted by the blue parts of the trajectory) before aligning correctly with the doorway. 
%got to the goal in all $10$ runs while almost equally balancing the both sides of the space between left and right through the narrow pathway, and it backed off (blue) at the door then go straight as in the vertical lines (like a cautious human movement).
   
\noindent $\bullet$ \textbf{(C2):} In this case a static obstacle appears on the global plan. Although DWA tried to deviate from the global plan, it did not do so enough, and therefore collided with the obstacle every time. 
%The DWA agent got too close to the walls and stepped on the bottom edge (the foot) of the cardboard obstacle in most of runs regardless of departure locations and even pushed it to be fell one occasion. In the meanwhile, when 
The SP planner was succesful when starting from the bottom. When starting from the top, the shortest path around the obstacle alternated between ``going left'' and ``going right''. This indecision led to some collisions. 
%agent departed from the upper $y$ position, the waypoint around the obstacle was pingponging between left and right due to the optimal route exploration since both are the shortest path around it, which caused the agent to push the cardboard while jiggling between both sides then even the robot rode over on top of the fallen obstacle in several times. 
SACPlanner often backed off multiple times when confronted with the obstacle. However, it eventually made it round the obstacle every time. %did back-and-forth multiple times which took more time to detour around the cardboard obstacle, but it all successfully got to the goal without any collision with overall smooth trajectories. 

\noindent $\bullet$ \textbf{(C3):} Both DWA and SP were unable to deal with the fact that the pedestrian obstacle was approaching and hence the ``correct'' trajectory kept changing. Even though the pedestrian stopped right in front of the robot, both DWA and SP kept going and caused a collision. SACPlanner went backwards when the pedestrian got close and then directed the robot to take a wide berth in the available open space. %agent detected the approaching obstacle earlier than the others and stopped for a few seconds, but eventually decided to force itself going forward and collided to the pedestrian in all $10$ runs. The SP agent reacted later or never than the other agents and collided either to the wall or the pedestrian every time. These are shown as the lower positions of collisions by the SP agent in terms of $y$-axis in the plot. The SAC agent successfully avoided the coming pedestrian by backing off then detoured widely enough by utilizing the open space and got to the goal by following the original path in the end for all $10$ runs. 

   

%In addition to these qualitative diagnosis of each agent's trajectories, 
\noindent \textbf{Quantitative metrics:} In Table \ref{tab:exp_results} we show the mean travel time ($s$), mean travel distance ($m$), mean speed ($m/s$), and collision rate for the 3 local planners on (C1)-(C3) across all runs. For DWA on (C1) we only consider the non-aborted runs. For (C2)-(C3) we remove the obstacle after each collision and so the robot will still reach the goal. We note that the ``backing off'' behavior of SACPlanner leads to greater distances/times than DWA and SP, but this how it is able to achieve a much lower collision rate. 

%(P4) distance to the nearest obstacle ($m$); and (P5) the standard deviation of `(P4)', (P6) count of collision, (P7) on average when the collision happened ($s$), and (P8) the mean yaw deviation (rad). Here (P3)$=$(P2)/(P1), and (P6) is the opposite of success rate since we removed obstacle after collision to let the agent finish its path except $5$ DWA (C1) with ROS abort. For (C1), the SP and SAC agents were the two best ones by balancing the space in the maze doorways with the good overall speed. For unexpected obstacle avoidance in (C2)\&(C3), the SAC agent was the best as explained before.   

\begin{table}[h!]
	\caption{Summary statistics of trajectories from test cases.}
	\vspace{-3mm}
	\label{tab:exp_results}
	\centering{\tiny%\scriptsize
	\begin{tabular}%{@{\hspace{1pt}}c@{\hspace{1pt}}|r|r|r|r|r|r|r|r|r}
	{c|r|r|r|r|r|r|r|r|r}
	\hline
& \multicolumn{3}{c|}{(C1)} & \multicolumn{3}{c|}{(C2)} & \multicolumn{3}{c}{(C3)} \\\hline
&	\multicolumn{1}{c|}{DWA}	&	\multicolumn{1}{c|}{SP}	&	\multicolumn{1}{c|}{SAC}	&	\multicolumn{1}{c|}{DWA}	&	\multicolumn{1}{c|}{SP}	&	\multicolumn{1}{c|}{SAC}		&	\multicolumn{1}{c|}{DWA}	&	\multicolumn{1}{c|}{SP}	&	\multicolumn{1}{c}{SAC}		\\\hline
Time	&	21.80	&	30.10	&	37.20	&	30.70	&	20.90	&	28.50	&	27.50	&	22.30	&	33.10	\\\hline
Distance	&	7.13	&	8.93	&	10.70	&	5.47	&	6.26	&	8.57	&	8.77	&	8.01	&	10.80	\\\hline
Speed	&	0.33	&	0.30	&	0.29	&	0.18	&	0.30	&	0.30	&	0.32	&	0.36	&	0.33	\\\hline
%mean obs dist	&	0.08	&	{\bf 0.22}	&	{\bf 0.22}	&	0.16	&	0.21	&	{\bf 0.30}	&	0.28	&	0.24	&	{\bf 0.31}	\\\hline
%stddev obs dist	&	0.04	&	0.03	&	0.05	&	0.07	&	0.04	&	0.06	&	0.25	&	0.12	&	0.08	\\\hline
Collision	&	0.5	&	{\bf 0}	&	{\bf 0}	&	1.0	&	0.3	&	{\bf 0}	&	1.0	&	0.9	&	{\bf 0}	\\\hline
%collision time	&	15.90	&		&		&	9.46	&	6.40	&		&	7.88	&	8.17	&		\\\hline
%yaw dev	&	0.26	&	0.36	&	0.33	&	0.20	&	0.34	&	0.31	&	0.17	&	0.22	&	0.33	\\\hline
	\end{tabular}}
\end{table}

\noindent $\bullet$ \textbf{(C4)} When the pedestrian switches to walking across the robot's path rather than walking towards it, the results are similar to (C3). 
%setting the pedestrian to perpendicularly cut cross the robot's global path in a wider open space to investigate each agent's reaction. Here the global costmap does not contain any static obstacle, but the chairs and tables around this area appears in the local costmap at the left side of robot's traverse direction. We did $3$ runs each with all the agents in this case, and the trajectories are also displayed in Fig.\ref{fig:case4_comp} where the gray blobs smudged as horizontally reflecting the dynamic movement of the crossing pedestrian. 
Both DWA and SP are not reactive enough and collide every time. However, SACPlanner backs off when the pedestrian is close, and then resumes traveling towards the goal after the pedestrian has passed through.   
%agent straightly backed off only until the margin allows from the side tables and chairs then continued its original path only after the pedestrian had already passed through. 

%
%We show the experiments' results in this section on the three test cases.
%For the room I to room N case, we show in Figure~\ref{fig:log_full_maze_door} the trajectory paths for $2$ runs for the RL planner and the baselines.
%%and the trajectories of DWA in Figure~\ref{fig:dwa_room_i_n}.
%It can be observed that SAC planner is able to successfully complete the path without hitting other obstacles. The baselines however fail in at least one of the runs to complete the path.
%%both planners are able to maneuver through the tight corridor and into the doorway without hitting the barriers. 
%%In one of the experiment runs however, DWA aborts the global plan and fails to reach the destination after passing through the doorway.
%In the DWA failed run, DWA approaches the static barrier after passing through the doorway. DWA's failure may be due to its inability to find a suitable velocity trajectory close to the barrier.
%In one of the runs, we observed the shortest path planner to hit the barrier when passing through the doorway.
%The RL planner however completes all of its test runs without collisions.
%
%For the room I to room X case, we provide the results in Figure~\ref{fig:log_full_maze_obstacle} for the local planners used.
%The RL planner is able to complete all the test runs without hitting the obstacle nor the walls. 
%In both cases, the RL planner would act defensively once it encountered an obstacle in front of the robot. The RL planner would back away from the obstacle and give itself ample space to maneuver around the closed path. We believe this defensive behavior is key in the RL planner's successful test runs.
%When facing the obstacle in one of the experiment runs, DWA aborted the global plan and was unable to navigate around the obstacle.
%For the shortest path planner, the robot collided with the moving pedestrian in all runs.
%
%For the final case with a moving pedestrian, we show the trajectories for all planners in Figure~\ref{fig:log_full_pedestrian}.
%With DWA, the robot collided with the pedestrian walking towards the robot in all $2$ runs, while the RL planner exhibited the same defensive posture we observed in the previous two test cases. The RL planner was able to navigate successfully without hitting the pedestrian in all test runs. 
%Both the DWA and shortest path planners collided with the pedestrian in all test runs.
%The shortest path planner's failure might be due to its commitment to the current goal and disregard to any object appearing in the current plan. We believe the DWA planner failed since it searches for velocity trajectories around the approaching obstacle. As the pedestrian approaches the robot, the selected trajectory becomes unsuitable to the changing environment.
%%The RL planner has a success rate of $\%$, showing clearly that it outperforms the DWA planner.
%
