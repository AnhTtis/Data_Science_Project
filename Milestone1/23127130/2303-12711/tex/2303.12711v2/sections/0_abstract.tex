


\markboth{Abstract}{ABSTRACT}
\null\vspace{2cm}
\begin{abstract}
Barrett's Esophagus (BE) is the only known precursor to Esophageal adenocarcinoma (EAC), an increasingly more common subtype of esophageal cancer that has a poor prognosis once diagnosed. Because of this, the diagnosis of BE is crucial in the prevention and treatment of esophageal cancer. 
% BE is characterized by a change of the cells lining the esophagus, which can be classified according to four different progressions stages. However, the process of determining these progression stages by pathologists is expensive, as well as highly subjective. 
Recently, an increase in supervised machine learning supporting the diagnosis of BE has been seen, but these methods are limited by a high interobserver variability present in histopathological training data. Unsupervised representation learning through Variational Autoencoders (VAEs) presents itself as a promising alternative, as these models learn to map input data to a lower-dimensional manifold containing only useful features. Such a manifold space can therefore characterize the progression of BE, leading to a meaningful representation that has the potential to improve downstream tasks and give insight into the disease progression as well. 
However, because the VAE's latent space is assumed to be Euclidean, relationships between points are distorted, which greatly hinders the model's ability to model disease progression. Geometric VAEs solve this issue by providing additional geometric structure to the latent space through the use of alternative manifold topologies. 
Hence, this work studies two such models: RHVAE, which assumes a Riemannian manifold and \svae, which assumes a hyperspherical manifold. We show that \svae obtains better reconstruction losses and representation classification accuracies, and higher quality generated images and interpolations than the vanilla VAE in lower-dimensional settings. We moreover show that these results can be further improved by disentangling rotation information from the latent space through an extension to a group-based architecture. Furthermore, we take initial steps towards \sae, a novel autoencoder model that can be used to generate qualitative images without the need for a variational framework, while still retaining benefits of autoencoders such as improved stability and reconstruction quality.  

\end{abstract}
