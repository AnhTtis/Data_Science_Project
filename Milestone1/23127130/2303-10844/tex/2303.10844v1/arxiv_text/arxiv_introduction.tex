
\section{Introduction} 

Blockchains were designed to enable decentralized digital currencies, but evolved into technology for avoiding censorship, managing decentralized organizations and providing data transparency. The global blockchain market was valuated at USD 5.92 billion in 2021 and is predicted to grow as blockchain applications continue to bring value to a wide range of industries~\cite{grandviewblockchain}.

While the interest in distributed ledger technology has increased, long-standing blockchain issues such as low throughput, high latency and trade-offs between security, scalability and decentralization~\cite{vukolic2015quest, croman2016scaling, gervais2016security} are still prevalent. For instance, Bitcoin and Ethereum achieve a transaction throughput of roughly 7 and 15 transactions per second (tps) respectively, and are often compared to traditional payment systems such as Visa, which claims to be capable of processing 24,000 tps\footnote{https://usa.visa.com/run-your-business/small-business-tools/retail.html}, to highlight the large gap between their performance. 

Several works have evaluated and identified issues affecting the performance of both public and permissioned blockchains~\cite{aldweesh2019opbench,baliga2018fabric,baliga2018quorum,dabbagh2020performance,dinh2017blockbench,hyperledgercaliper,wang2019performance,fan2020performance,kuzlu2019performance,saingre2020bctmark,xu2021latency,wang2021xbcbench,thakkar2018performance,sedlmeir2021dlps, gramolidiablo2023}. For instance, Hyperledger Fabric v0.6.0 is unable to scale beyond 16 nodes because of implementation issues in its consensus protocol~\cite{dinh2017blockbench}. 

Similarly, certain cryptography operations and REST API calls have led to transaction validation bottlenecks in Hyperledger Fabric v1.0~\cite{thakkar2018performance}. Beyond evaluating the system's performance, the authors propose optimizations that lead to a 16x increase in throughput. Such examples highlight the importance of in-depth performance evaluations.

In modern blockchain applications, which increasingly require  exchanging data between different blockchains (a.k.a. cross-chain communication)~\cite{forbes2022interoperability, cointelegraph2022interoperability,abebe2019enabling, gordon2018blockchain}, bottlenecks may have a severe impact as they degrade the performance and increase waiting times for all blockchains participating in an operation. Cross-chain communication requires specific protocols to connect and reliably deliver data between blockchains, however, those protocols introduce additional points of failure and new bottlenecks, more so given that many are still under active development~\cite{belchior2020survey, johnson2019sidechains}.

Various protocols have been proposed to enable cross-chain communication~\cite{herlihy2018atomic,kiayias2019proof,gazi2019proof, ethrelay, karantias2019proof, kwon2019cosmos, wood2016polkadot, nick2020liquid, interledgerprotocol, btcrelay, zamyatin2019xclaim, back2014enabling, abebe2020verifiable}, each with different trade-offs between security, performance and deployment complexity~\cite{chervinski2022characterizing}. Relay protocols such as BTC Relay~\cite{btcrelay} and ETH Relay~\cite{ethrelay} employ smart contracts to verify transactions added to external blockchains, but work only one-way and are costly to maintain. Sidechains~\cite{back2014enabling, kiayias2019proof, gazi2019proof, nick2020liquid} enable two blockchains to send assets back and forth via a two-way peg mechanism, but must be designed for a specific pair of blockchains. Blockchain ecosystems, such as Cosmos~\cite{kwon2019cosmos} and Polkadot~\cite{wood2016polkadot}, host independent blockchain applications and provide messaging protocols to facilitate cross-chain communication, but impose constraints on blockchains to make them compatible with the ecosystem.

Cross-chain operations differ from those executed in isolated blockchains in the following ways: they must undergo consensus to be recorded in all participating blockchains; they require a reliable communication channel through which data can be sent from one blockchain to another; \revision{they might be rejected by a participating blockchain, therefore they require atomicity in order to maintain a consistent state across ledgers~\cite{han2019optionality, hardjono2021blockchain, belchior2022hermes}}. Hence, cross-chain performance analysis requires a different approach from the ones used in isolated blockchains and those proposed by existing evaluation frameworks.

Among proposals to enable cross-chain communication, Cosmos (3 billion USD market cap) and Polkadot (6.2 billion USD market cap)\footnote{https://coinmarketcap.com/} have attracted significant attention in the past few years. Cosmos is a network of blockchains and decentralized applications built using its own SDK. Blockchains deployed in the Cosmos network are interconnected and can use the Inter-Blockchain Communication (IBC) protocol to exchange information. Binance Chain, Osmosis, Secret Network and Ethermint\footnote{https://v1.cosmos.network/ecosystem/apps} are examples of chains deployed in the Cosmos ecosystem. Similarly, Polkadot is a network of interconnected blockchains, called parachains, developed with the Substrate framework. Parachains can communicate through the XCMP (Cross-Chain Message Passing) protocol, built on top of the XCM (Cross-consensus Message) format. The Moonbeam network is currently deployed in the Polkadot ecosystem and a Chainlink implementation is currently under development\footnote{https://parachains.info/}. Despite their popularity, there is no performance evaluation of the cross-chain communication protocols of those networks. The lack of performance analysis for IBC and XCMP may be hindered by their active development stage, which leads to breaking changes, outdated documentation and scarcity of information regarding setup and configuration.

In this paper, we present the first performance analysis of the Inter-Blockchain Communication protocol paired with the Hermes Relayer, both of which are currently used to enable cross-chain communication in the Cosmos Network. IBC is an open source protocol designed to facilitate communication between heterogeneous blockchains. It defines a standard for constructing cross-chain messages and requires only the implementation of a minimal set of functions in the communicating blockchains. This gives IBC the potential to be used not only between blockchains deployed in the Cosmos ecosystem, but any blockchains that meet the requirements. \revision{Unlike XCMP, which is currently under active development and being replaced by a temporary protocol~\cite{polkadotlearnxcmp}, working implementations of IBC are already being used to connect over 50 homogeneous blockchains including the Cosmos Hub, Osmosis and the Juno Network\footnote{https://hub.mintscan.io/chains/ibc-network}. Additionally, Polymer Labs and Hyperledger YUI Labs are working toward supporting IBC on existing blockchain platforms such as Ethereum, Hyperledger Fabric, Hyperledger Besu, Corda, Solana, Polygon and Fantom~\cite{yuilabs, polymerlabsibc}.}

Evaluating the performance of cross-chain communication protocols such as IBC poses additional challenges compared to isolated blockchains: 
\begin{enumerate}[label=(\roman*)]
\item The protocol is composed of multiple steps ({\tt transfer}, {\tt receive}, {\tt acknowledge} and {\tt timeout}) that must be recorded in the two communicating blockchains. 
\item Blockchains are unable to communicate directly and require additional agents and/or protocols to deliver messages from one blockchain to another. 
\item Operations may fail after having steps recorded in the blockchain. For instance, when a message is not delivered before timing out, intermediary steps recorded in transactions remain in the blockchain even though the cross-chain operation was not completed. 
\end{enumerate}

We emphasize that (i) increases the complexity of the experimental setup as multiple systems need to be deployed and connected to each other. Similarly, (ii) and (iii) increase analysis complexity by requiring the state of operations to be tracked across all of the systems involved in cross-chain communication.

In this paper we make four major contributions:

\begin{itemize}
    \item We analyze the performance of cross-chain communications between Cosmos Gaia blockchains using the Inter-blockchain Communication protocol and the Hermes Relayer\footnote{https://github.com/informalsystems/ibc-rs}. Throughout our comprehensive throughput, latency and relayer scalability analysis we identify several surprising results and issues impairing performance. First, we show that using two relayers to relay for a single cross-chain channel decreases throughput by 33\% compared to using a single relayer. Second, we show that the relayer application processes cross-chain transfers in batches, leading to high transfer completion latency, e.g., 455 seconds for 5,000 transfers. Third, we show that the main cross-chain communication bottleneck, which lies in the blockchain's RPC implementation, causes 69\% of the time required to process cross-chain transfers to be spent on querying data from the blockchains.
    
    \item  \revision{We propose a novel framework for evaluating the performance of cross-chain communication protocols. As a first instantiation of our framework, we implement and make available an open source tool~\cite{code-data} to measure the performance of cross-chain communication between Cosmos blockchains using the IBC protocol}. Our tool reduces the effort required to deploy and connect two Cosmos blockchains using the Hermes relayer and provides seven configurable parameters that can be used to evaluate different blockchain configurations. Additionally, our tool generates execution reports to assist in performance evaluations for different setup configurations.
    
    \item We discuss five challenges faced during the deployment of the Cosmos Gaia blockchains and the Hermes Relayer. In Section~\ref{sec:challenges} we discuss how those challenges can impact cross-chain performance and increase the difficulty of using such kind of systems.

    \item We provide a 158GB dataset of execution logs for assisting future research~\cite{code-data}.

\end{itemize}

