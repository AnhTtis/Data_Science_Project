\section{Related Work}

Most previous works on empirical blockchain performance analysis have essentially targeted the execution of workloads in isolated blockchains. Those workloads are usually designed to stress test specific blockchain functionalities such as the consensus protocol, smart contract language or a decentralized application. Hyperledger Fabric has been extensively analyzed using micro-benchmarks~\cite{baliga2018fabric}, custom workloads~\cite{thakkar2018performance, nakaike2020hyperledger}, variable transaction submission rates~\cite{kuzlu2019performance} and blockchain performance benchmarking tools~\cite{wang2021xbcbench, dinh2017blockbench, saingre2020bctmark}. Following, Ethereum has also been the target of several studies analyzing the platform's throughput and latency~\cite{dinh2017blockbench, wang2021xbcbench, sedlmeir2021dlps}. Both Hyperledger Fabric and Ethereum have also undergone fault tolerance and on-chain smart contract performance analysis~\cite{saingre2020bctmark, aldweesh2019opbench, dinh2017blockbench}. 

Interoperability in the domain of computer systems is an established topic in the literature~\cite{viho2001towards, leal2019interoperability}, however, works that discuss interoperability, performance and conformity of cross-chain communication solutions only recently began to appear ~\cite{belchior2023hephaestus,belchior2022interoperabilitysolution,koens2019assessing}. 
For instance, a framework has been proposed to systematically assess the interoperability degree of cross-chain communication solutions, including cross-chain performance as a guideline~\cite{belchior2022interoperabilitysolution}.
Similarly, another work proposes a set of properties to evaluate interoperability solutions for distributed ledgers, however, the only performance metric considered is scalability~\cite{koens2019assessing}. 
Lastly, Hephaestus~\cite{belchior2023hephaestus} generates models to identify malicious behavior in cross-chain communication. 
However, unlike ours, none of those previous works provides a concrete framework to guide the process of empirically evaluating the performance of cross-chain communication solutions.
In addition, we provide the first comprehensive evaluation of cross-chain communication using the IBC protocol.




