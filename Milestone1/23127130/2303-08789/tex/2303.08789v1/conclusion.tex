\section{Conclusion and Future Work}

We have introduced \appr, a sensorimotor model architecture  geared towards pretraining on robotic manipulation-relevant data that is realistically available in quantity. When training on human-collected demonstrations, \appr's transformer-based components get an efficiency boost thanks to using relative positional encoding.

While we believe that \appr\ has great potential as a model architecture for general robotic manipulation, our current implementation of it has some limitations:
\begin{itemize}[leftmargin=*]
    \item In our experiments all the training data was collected on the same embodiment/robot on which the trained model was ultimately deployed. Although visuomotor data $\vmt$ can be collected on the target embodiment, a vast amount of realistically available multi-task video demonstration data $\mtvd$ was generated by other robots or even people. This can cause a mismatch between the target robot's capabilities and the demonstrated way of completing a task. Planning hierarchically first in the skill space as, e.g., in \citet{lynch2019play}, and then in the observation embedding space can address this issue.
    \item We have demonstrated \appr's ability to train its own embedding space from scratch using the inverse dynamics loss $\LL_{EX}$ (\Cref{eq:invd_loss}), but this can expensive for large amounts of data. A more practical approach is to use a large pretrained encoder such as DINO and possibly have $\LL_{EX}$ finetune it.
    \item So far, we have trained \appr\ on simulated data, which is less varied than real-world demonstrations. Pretraining \appr\ on a large or even internet-scale data corpus will be a more conclusive test of \appr's capabilities.
    \item With the rise of powerful LLMs such as \citet{instructrl2022}, language is becoming a far more potent task specification modality than images. Switching \appr\ to language for task spcification can lead \appr's planner to generalize much better across tasks.
\end{itemize}
We see \appr\ as flexible enough to allow addressing these issues without significantly changing the architecture and look forward to doing so in the future.
