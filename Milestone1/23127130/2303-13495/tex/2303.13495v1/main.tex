\documentclass[10pt,twocolumn,letterpaper]{article}

\usepackage{iccv}
\usepackage{times}
\usepackage{epsfig}
\usepackage{graphicx}
\usepackage{amsmath}
\usepackage{amssymb}

% Include other packages here, before hyperref.

% If you comment hyperref and then uncomment it, you should delete
% egpaper.aux before re-running latex.  (Or just hit 'q' on the first latex
% run, let it finish, and you should be clear).
\usepackage[pagebackref=true,breaklinks=true,letterpaper=true,colorlinks,bookmarks=false]{hyperref}


%%%%%%%%% Customized Packages and Commands %%%%%%%%%

% Repeating Terms in this Paper
\newcommand{\RI}{\textit{Relation Inversion}}
\newcommand{\R}{$\langle$R$\rangle$ }

% Additional Packages
\usepackage{caption}

% for Xhline
\usepackage{booktabs, siunitx}
\usepackage{multirow, booktabs}
\usepackage{siunitx}
\usepackage{makecell} 

% DDPM formula notations
\usepackage{mathtools}
\newcommand{\vardbtilde}[1]{\tilde{\raisebox{0pt}[0.85\height]{$\tilde{#1}$}}}
\newcommand{\defeq}{\coloneqq}
\newcommand{\grad}{\nabla}
\newcommand{\E}{\mathbb{E}}
\newcommand{\Var}{\mathrm{Var}}
\newcommand{\Cov}{\mathrm{Cov}}
\newcommand{\Ea}[1]{\E\left[#1\right]}
\newcommand{\Eb}[2]{\E_{#1}\!\left[#2\right]}
\newcommand{\Vara}[1]{\Var\left[#1\right]}
\newcommand{\Varb}[2]{\Var_{#1}\left[#2\right]}
\newcommand{\kl}[2]{D_{\mathrm{KL}}\!\left(#1 ~ \| ~ #2\right)}
\newcommand{\pdata}{{p_\mathrm{data}}}
\newcommand{\bA}{\mathbf{A}}
\newcommand{\bI}{\mathbf{I}}
\newcommand{\bJ}{\mathbf{J}}
\newcommand{\bH}{\mathbf{H}}
\newcommand{\bL}{\mathbf{L}}
\newcommand{\bM}{\mathbf{M}}
\newcommand{\bQ}{\mathbf{Q}}
\newcommand{\bR}{\mathbf{R}}
\newcommand{\bzero}{\mathbf{0}}
\newcommand{\bone}{\mathbf{1}}
\newcommand{\bb}{\mathbf{b}}
\newcommand{\bu}{\mathbf{u}}
\newcommand{\bv}{\mathbf{v}}
\newcommand{\bw}{\mathbf{w}}
\newcommand{\bx}{\mathbf{x}}
\newcommand{\by}{\mathbf{y}}
\newcommand{\bz}{\mathbf{z}}
\newcommand{\bxh}{\hat{\mathbf{x}}}
\newcommand{\btheta}{{\boldsymbol{\theta}}}
\newcommand{\bphi}{{\boldsymbol{\phi}}}
\newcommand{\bepsilon}{{\boldsymbol{\epsilon}}}
\newcommand{\bmu}{{\boldsymbol{\mu}}}
\newcommand{\bnu}{{\boldsymbol{\nu}}}
\newcommand{\bSigma}{{\boldsymbol{\Sigma}}}
\newtheorem{lemma}{Lemma}
\newtheorem{theorem}{Theorem}

% argmin argmax
\usepackage{amsmath}
\DeclareMathOperator*{\argmax}{arg\,max}
\DeclareMathOperator*{\argmin}{arg\,min}

\usepackage{marvosym, ifsym}

\usepackage{multicol}

%%%%%%%%% End of Customized Packages and Comments



\iccvfinalcopy % *** Uncomment this line for the final submission

\def\httilde{\mbox{\tt\raisebox{-.5ex}{\symbol{126}}}}

% Pages are numbered in submission mode, and unnumbered in camera-ready
\ificcvfinal\pagestyle{empty}\fi



\begin{document}

%%%%%%%%% TITLE
\title{ReVersion: Diffusion-Based Relation Inversion from Images}

%%%%%%%%% Author

\author{Ziqi Huang$^{*}$
\quad
Tianxing Wu$^{*}$
\quad
Yuming Jiang
\quad
Kelvin C.K. Chan
\quad
Ziwei Liu\textsuperscript{\Letter}\\
S-Lab, Nanyang Technological University
\quad
\\
{\tt\small \{ziqi002, twu012, yuming002, chan0899, ziwei.liu\}@ntu.edu.sg}\\
}


%%%%%%%%%%%%%%%%%%%%%% Teaser 
\twocolumn[{%
            \renewcommand\twocolumn[1][]{#1}%
            \vspace{-1em}
            \maketitle
            \vspace{-1em}
            \begin{center}
                \vspace{-20pt}
                \centering
                \includegraphics[width=1.0\textwidth]{figures/fig_teaser.pdf}
                \vspace{-20pt}
                \captionof{figure} {We propose a new task, \textbf{Relation Inversion}: Given a few exemplar images, where a relation co-exists in every image, we aim to find a relation prompt \R  to capture this interaction, and apply the relation to new entities to synthesize new scenes. The above images are generated by our \textbf{ReVersion} framework.}
                \label{teaser}
            \end{center}%
        }]

%%%%%%%%% Paper


\begin{abstract} 
Diffusion models gain increasing popularity for their generative capabilities. Recently, there have been surging needs to generate customized images by inverting diffusion models from exemplar images. However, existing inversion methods mainly focus on capturing object \textbf{appearances}. How to invert object \textbf{relations}, another important pillar in the visual world, remains unexplored.
%
In this work, we propose \textbf{ReVersion} for the \textbf{\RI{}} task, which aims to learn a specific relation (represented as ``relation prompt'') from exemplar images. Specifically, we learn a relation prompt from a frozen pre-trained text-to-image diffusion model. The learned relation prompt can then be applied to generate relation-specific images with new objects, backgrounds, and styles. 
%
%footnote
\makeatletter{\renewcommand*{\@makefnmark}{}
\footnotetext{$^*$ indicates equal contribution. \href{https://ziqihuangg.github.io/projects/reversion.html}{Project page} and \href{https://github.com/ziqihuangg/ReVersion}{code} are available.}\makeatother}
%

Our key insight is the \textbf{``preposition prior''} - real-world relation prompts can be sparsely activated upon a set of basis prepositional words. Specifically, we propose a novel relation-steering contrastive learning scheme to impose two critical properties of the relation prompt: \textbf{1)} The relation prompt should capture the interaction between objects, enforced by the preposition prior. \textbf{2)} The relation prompt should be disentangled away from object appearances. 
%
We further devise relation-focal importance sampling to emphasize high-level interactions over low-level appearances (\eg, texture, color).
% 
To comprehensively evaluate this new task, we contribute \textbf{ReVersion Benchmark}, which provides various exemplar images with diverse relations. Extensive experiments validate the superiority of our approach over existing methods across a wide range of visual relations.
\end{abstract}

\section{Introduction}


Recently, Text-to-image (T2I) diffusion models~\cite{rombach2022ldm, ramesh2022dalle2, saharia2022imagen} have shown promising results and enabled subsequent explorations of various generative tasks. 
%
There have been several attempts~\cite{gal2022textualinversion, ruiz2022dreambooth, kumari2022customdiffusion} to \textbf{\textit{invert}} a pre-trained text-to-image model, obtaining a text embedding representation to capture the object in the reference images. While existing methods have made substantial progress in capturing object appearances, such exploration for relations is rare. 
%
Capturing object relation is intrinsically a harder task as it requires the understanding of interactions between objects as well as the composition of an image, and existing inversion methods are unable to handle the task due to entity leakage from the reference images. Yet, this is an important direction that worths our attention.


In this paper, we study the \textbf{\textit{Relation Inversion}} task, whose objective is to learn a relation that co-exists in the given exemplar images. Specifically, with objects in each exemplar image following a specific relation, we aim to obtain a relation prompt in the text embedding space of the pre-trained text-to-image diffusion model. By composing the relation prompt with user-devised text prompts, users are able to synthesize images using the corresponding relation, with customized objects, styles, backgounds, \etc.


To better represent high-level relation concepts with the learnable prompt, we introduce a simple yet effective \textbf{\textit{preposition prior}}. The preposition prior is based on a premise and two observations in the text embedding space. Specifically, we find that 1) prepositions are highly related to relations, 2) prepositions and words of other Parts-of-Speech are individually clustered in the text embedding space, and 3) complex real-world relations can be expressed with a basic set of prepositions. Our experiments show that this language-based prior can be effectively used as a high-level guidance for the relation prompt optimization.


Based on our preposition prior, we propose the \textbf{ReVersion} framework to tackle the Relation Inversion problem. Notably, we design a novel \textit{relation-steering contrastive learning scheme} to steer the relation prompt towards a relation-dense region in the text embedding space. A set of basis prepositions are used as positive samples to pull the embedding into the sparsely activated region, while words of other Parts-of-Speech (\eg, nouns, adjectives) in text descriptions are regarded as negatives so that the semantics related to object appearances are disentangled away.
%
To encourage attention on object interactions, we devise a \textit{relation-focal importance sampling} strategy. It constrains the optimization process so that high-level interactions rather than low-level details are emphasized, effectively leading to better relation inversion results. 


As the first attempt in this direction, we further contribute the \textbf{ReVersion Benchmark}, which provides various exemplar images with diverse relations.
%
The benchmark serves as an evaluation tool for future research in the Relation Inversion task. Results on a variety of relations demonstrate the power of preposition prior and our ReVersion framework.

Our contributions are summarized as follows:
%
\vspace{-0.4em}
\begin{itemize}
    \setlength\itemsep{0em}
    \item We study a new problem, \textbf{\textit{Relation Inversion}}, which requires learning a relation prompt for a relation that co-exists in several exemplar images. While existing T2I inversion methods mainly focus on capturing appearances, we take the initiative to explore relation, an under-explored yet important pillar in the visual world.
    %
    \item We propose the \textbf{\textit{ReVersion framework}}, where the \textit{relation-steering contrastive learning scheme} steers the relation prompt using our \textbf{\textit{``preposition prior''}}, and effectively disentangles the learned relation away from object appearances. \textit{Relation-focal importance sampling} further emphasizes high-level relations over low-level details.
    %
    \item We contribute the \textit{\textbf{ReVersion Benchmark}}, which serves as a diagnostic and benchmarking tool for the new task of \RI{}. 
\end{itemize}

\section{Related Work}
\noindent\textbf{Diffusion Models.}
%
Diffusion models~\cite{ho2020ddpm, sohl2015deep, song2020score, rombach2022ldm, gu2022vqdiffusion, song2020ddim} have become a mainstream approach for image synthesis~\cite{dhariwal2021beatgan, esser2021imagebart, meng2021sdedit} apart from GANs~\cite{goodfellow2014gan}, 
%
and have shown success in various domains such as video generation~\cite{harvey2022fdm,villegas2022phenaki,singer2022makeavideo,ho2022videoDM}, image restoration~\cite{saharia2022sr3, ho2022cascaded}, and many more~\cite{baranchuk2021label,graikos2022diffusion, amit2021segdiff, austin2021structured}.
%
In the diffusion-based approach, models are trained using score-matching objectives~\cite{hyvarinen2005estimation, vincent2011connection} at various noise levels, and sampling is done via iterative denoising. 
%
Text-to-Image (T2I) diffusion models~\cite{ramesh2022dalle2, rombach2022ldm, esser2021imagebart, gu2022vqdiffusion, jiang2022text2human, nichol2021glide, saharia2022imagen} demonstrated impressive results in converting a user-provided text description into images. 
%
Motivated by their success, we build our framework on a state-of-the-art T2I diffusion model, Stable Diffusion~\cite{rombach2022ldm}.



% Framework Figure
\begin{figure*}[t]
  \centering
   \includegraphics[width=0.99\textwidth]{figures/fig_framework.pdf}
    \vspace{-5pt}
   \caption{\textbf{ReVersion Framework}. Given exemplar images and their entities' coarse descriptions, our ReVersion framework optimizes the relation prompt \R to capture the relation that co-exists in all the exemplar images. During optmization, the \textit{relation-focal importance sampling} strategy encourages \R to focus on high-level relations, and the \textit{relation-steering contrastive learning} scheme induces the relation prompt \R towards our \textit{preposition prior} and away from entities or appearances. Upon optimization, \R can be used as a word in new sentences to make novel entities interact via the relation in exemplar images. 
   }
   \label{fig:framework}
\end{figure*}



\noindent\textbf{Relation Modeling.}
%
Relation modeling has been explored in discriminative tasks such as scene graph generation~\cite{xu2017scene,vg17ijcv,shang2017video,ji2020action, yang2022psg, yang2023pvsg} and visual relationship detection~\cite{lu2016visual, yu2017visual, zhuang2017towards}.
These works aim to detect visual relations between objects in given images and classify them into a predefined, closed-set of relations.
However, the finite relation category set intrinsically limits the diversity of captured relations.
In contrast, Relation Inversion regards relation modeling as a generative task, aiming to capture arbitrary, open-world relations from exemplar images and apply the resulting relation for content creation.



\noindent\textbf{Diffusion-Based Inversion.} Given a pre-trained T2I diffusion model, inversion~\cite{gal2022textualinversion, ruiz2022dreambooth, kumari2022customdiffusion, kawar2022imagic} aims to find a text embedding vector to express the concepts in the given exemplar images. 
%
For example, given several images of a particular \textit{``cat statue''}, 
Textual Inversion~\cite{gal2022textualinversion} learns a new word to describe the appearance of this item - finding a vector in LDM~\cite{rombach2022ldm}'s text embedding space, so that the new word can be composed into new sentences to achieve personalized creation.
Rather than inverting the appearance information (\eg, color, texture), our proposed \textbf{\textit{Relation Inversion}} task extracts high-level object relations from exemplar images, which is a harder problem as it requires comprehending image compositions and object relationships.


\section{The Relation Inversion Task}
%
Relation Inversion aims to extract the common relation \R from several exemplar images.
%
Let $\mathcal{I}\,{=}\, \{I_{1}, I_{2}, ... I_{n}\}$ be a set of exemplar images, and $E_{i,A}$ and $E_{i,B}$ be two dominant entities in image $I_i$. In Relation Inversion, we assume that the entities in each exemplar image interacts with each other through a common relation $R$. A set of coarse descriptions ${C}~{=}\, \{c_{1}, c_{2}, ... c_{n}\}$ is associated to the exemplar images, where \mbox{``${c}_i\,{=}\,E_{i, A}$ \R 
$E_{i, B}$''} denotes the caption corresponding to image $I_i$. Our objective is to optimize the relation prompt \R such that the co-existing relation can be accurately represented by the optimized prompt. 


An immediate application of Relation Inversion is relation-specific text-to-image synthesis. Once the prompt is acquired, one can generate images with novel objects interacting with each other following the specified relation. More generally, this task reveals a new direction of inferring relations from a set of exemplar images. This could potentially inspire future research in representation learning, few-shot learning, visual relation detection, scene graph generation, and many more.



\section{The ReVersion Framework}

\subsection{Preliminaries}

%-----------------------------------------------
\noindent \textbf{Stable Diffusion.}
%
Diffusion models are a class of generative models that gradually denoise the Gaussian prior $\bx_T$ to the data $\bx_0$ (\eg, a natural image). 
%
The commonly used training objective $L_\mathrm{DM}$~\cite{ho2020ddpm} is:
%
\begin{gather}
    L_\mathrm{DM}(\theta) \defeq \Eb{t, \bx_0, \bepsilon}{ \left\| \bepsilon - \bepsilon_\theta(\bx_t, t) \right\|^2}, \label{eq:training_objective_simple}
\end{gather}
%
where $\bx_t$ is an noisy image constructed by adding noise $\bepsilon \sim \mathcal{N}(\bzero, \bI)$ to the natural image $\bx_0$, and the network $\bepsilon_\theta(\cdot)$ is trained to predict the added noise.
%
To sample data $\bx_0$ from a trained diffusion model $\bepsilon_\theta(\cdot)$, we iteratively denoise $\bx_t$ from $t = T$ to $t = 0$ using the predicted noise $\bepsilon_\theta(\bx_t, t)$ at each timestep $t$.



Latent Diffusion Model (LDM)~\cite{rombach2022ldm}, the predecessor of Stable Diffusion, mainly introduced two changes to the vanilla diffusion model~\cite{ho2020ddpm}. First, instead of directly modeling the natural image distribution, LDM models images' projections in autoencoder's compressed latent space. Second, LDM enables text-to-image generation by feeding encoded text input to the UNet~\cite{ronneberger2015unet} $\bepsilon_\theta(\cdot)$. The LDM loss is:
%
\begin{gather}
    L_\mathrm{LDM}(\theta) \defeq \Eb{t, \bx_0, \bepsilon}{ { \left\| \bepsilon - \bepsilon_\theta(\bx_t, t, \tau_\theta(c)) \right\|^2}}, \label{eq:ldm_loss}
\end{gather}
% 
where $\bx$ is the autoencoder latents for images, and $\tau_\theta(\cdot)$ is a BERT text encoder~\cite{devlin2018bert} that encodes the text descriptions $c$.

Stable Diffusion extends LDM by training on the larger LAION dataset~\cite{schuhmann2022laion}, and changing the trainable BERT text encoder to the pre-trained CLIP~\cite{radford2021clip} text encoder.

%-----------------------------------------------
\noindent \textbf{Inversion on Text-to-Image Diffusion Models.} 
%
Existing inversion methods focus on appearance inversion. Given several images that all contain a specific entity, they~\cite{gal2022textualinversion, ruiz2022dreambooth, kumari2022customdiffusion} find a text embedding V* for the pre-trained T2I model. The obtained V* can then be used to generate this entity in different scenarios.

In this work, we aim to capture object relations instead. Given several exemplar images which share a common relation $R$, we aim to find a relation prompt \R to capture this relation, such that \mbox{``$E_{A}$ \R 
$E_{B}$''} can be used to generate an image where \textit{$E_{A}$} and \textit{$E_{B}$} interact via relation \R.



\subsection{Preposition Prior}
\label{sec: prior}


Appearance inversion focuses on inverting low-level features of a specific entity, thus the commonly used pixel-level reconstruction loss is sufficient to learn a prompt that captures the shared information in exemplar images. In contrast, \textit{relation} is a high-level visual concept. A pixel-wise loss alone cannot accurately extract the target relation. Some linguistic priors need to be introduced to represent relations.


In this section, we present the \textbf{\textit{``preposition prior''}}, a language-based prior that steers the relation prompt towards a relation-dense region in the text embedding space. This prior is motivated by a well-acknowledged premise and two interesting observations on natural language.



% insight Single Column
\begin{figure}[t]
  \centering
   \includegraphics[width=0.80\linewidth]{figures/fig_tsne.pdf}
   \vspace{-5pt}
   \caption{
   \textbf{Part-of-Speech (POS) Clustering}. We use t-SNE~\cite{van2008visualizing} to visualize word distribution in CLIP's input embedding space, where \R is optimized in our ReVersion framework. We observe that words of the same Part-of-Speech (POS) are closely clustered together, while words of different POS are generally at a distance from each other.
   }
   \label{fig:observation_a}
\end{figure}
%

\noindent\textit{\textbf{Premise: Prepositions describe relations.}} In natural language, prepositions are words that express the relation between elements in a sentence~\cite{huddleston_pullum_2002}. This language prior naturally leads us to use prepositional words to regularize our relation prompt.


\noindent\textit{\textbf{Observation \uppercase\expandafter{\romannumeral1}: POS clustering.}} As shown in Figure~\ref{fig:observation_a}, in the text embedding space of language models, embeddings are generally clustered according to their Part-of-Speech (POS) labels. This observation together with the \textit{Premise} inspire us to steer our relation prompt \R towards the preposition subspace (\ie,~the red region in Figure~\ref{fig:observation_a}).


\noindent\textit{\textbf{Observation \uppercase\expandafter{\romannumeral2}: Sparse activation.}} As shown in Figure~\ref{fig:observation_b}, feature similarity between the a real-world relation and the prepositional words are sparsely distributed, and the activated prepositions are usually related to this relation's semantic meaning. For example, for the relation ``swinging'', the sparsely activated prepositions are ``underneath'', ``down'', ``beneath'', ``aboard', \etc., which together collaboratively describe the ``swinging'' interaction. This pattern suggests that only a subset of prepositions should be activated during optimization, leading to our noise-robust design in Section~\ref{sec: contrastive}.



% insight Single Column
\begin{figure}[t]
  \centering
   \includegraphics[width=0.98\linewidth]{figures/fig_heatmap.pdf}
   \vspace{-6pt}
   \caption{\textbf{Sparse Activation}. We visualize the cosine similarities between real-world relations and basis prepositional words, and observe that relation is generally sparsely activated \textit{w.r.t.} the basis prepositions. Note that each row of similarity scores are sparsely distributed, with few peak values in red.}
   \label{fig:observation_b}
   \vspace{-4pt}
\end{figure}



Based on the aforementioned analysis, we hypothesize that a common \textit{visual relation} can be generally expressed as a set of basis prepositions, with only a small subset of highly semantically-related prepositions activated. Motivated by this, we design a \textit{relation-steering contrastive learning scheme} to steer the relation prompt \R into a relation-dense region in the text embedding space.



\subsection{Relation-Steering Contrastive Learning} 
\label{sec: contrastive}

Recall that our goal is to acquire a relation prompt \R that accurately captures the co-existing relation in the exemplar images. 
A basic objective is to reconstruct the exemplar images using \R:
%
\begin{gather}
    \langle{R}\rangle = \argmin_{\langle{r}\rangle} \Eb{t, \bx_0, \bepsilon}{ { \left\| \bepsilon - \bepsilon_\theta(\bx_t, t, \tau_\theta(c)) \right\|^2}}, \label{eq:ti_loss}
\end{gather}
where $\bepsilon \sim \mathcal{N}(\bzero, \bI)$, \R is the optimized text embedding, and $\bepsilon_\theta(\cdot)$ is a pre-trained text-to-image diffusion model whose weights are frozen throughout optimization. $\langle{r}\rangle$ is the relation prompt being optimized, and is fed into the pre-trained T2I model as part of the text description $c$. 

However, as discussed in Section \ref{sec: prior} , this pixel-level reconstruction loss mainly focus on low-level reconstruction rather than visual relations. Consequently, directly applying this loss could result in appearance leakage and hence unsatisfactory relation inversion.


Motivated by our \textit{Premise} and \textit{Observation \uppercase\expandafter{\romannumeral1}}, we adopt the preposition prior as an important guidance to steer the relation prompt towards the relation-dense text embedding subspace. Specifically, we can use the prepositions as positive samples and other POS' words (\ie, nouns, adjectives) as negative samples to construct a contrastive loss. Following InfoNCE~\cite{oord2018representation}, this preliminary contrastive loss is derived as:
%
\begin{gather}
    L_\mathrm{pre} = -log\frac{e^{R^{\top}\cdot P_{i} / \gamma}}{e^{R^{\top}\cdot P_{i} / \gamma}  + \sum_{k=1}^{K}e^{R^{\top}\cdot N_i^k / \gamma}},
\label{eq:pre-loss}
\end{gather}
%
where $R$ is the relation embedding, and $\gamma$ is the temperature parameter. $P_i$ (\ie,~positive sample) is a randomly sampled preposition embedding at the $i$-th optimization iteration, and $N_i=\{N_i^1, ..., N_i^K\}$ (\ie,~negative samples) is a set of randomly sampled embeddings from other POS. All embeddings are normalized to unit length.


Since the relation prompt should also be disentangled away from object appearance, we further propose to select the object descriptions of exemplar images as the improved negative set. In this way, our choice of negatives serves two purposes: 1) provides POS guidance away from non-prepositional clusters, and 2) prevents appearance leakage by including exemplar object descriptions in the negative set.


In addition, \textit{Observation \uppercase\expandafter{\romannumeral2}} (sparse activation) implies that only a small set of prepositions should be considered as true positives.
%
Therefore, we need a contrastive loss that is tolerant about noises in the positive set (\ie, not all prepositions should be activated). Inspired by \cite{miech2020end}, we revise Equation~\ref{eq:pre-loss} to a noise-robust contrastive loss as our final Steering Loss:
%
\begin{gather}
    L_\mathrm{steer} = -log\frac{\sum_{l=1}^{L}{e^{R^{\top}\cdot P_{i}^l / \gamma}}}{\sum_{l=1}^{L}{e^{R^{\top}\cdot P_{i}^l / \gamma}  + \sum_{m=1}^{M}e^{R^{\top}\cdot N_i^m / \gamma}}},
    \label{eq:l_steer}
\end{gather}
%
where $P_i=\{P_i^1, ..., P_i^L\}$ refers to positive samples randomly drawn from a set of basis prepositions (more details provided in Supplementary File), and $N_i=\{N_i^1, ..., N_i^M\}$ refers to the improved negative samples.



\subsection{Relation-Focal Importance Sampling}
%
In the sampling process of diffusion models, high-level semantics usually appear first, and fine details emerge at later stages~\cite{wang2023diffusion, huang2023collaborative}. As our objective is to capture the relation (a high-level concept) in exemplar images, it is undesirable to emphasize on low-level details during optimization. Therefore, we conduct an importance sampling strategy to encourage the learning of high-level relations. Specifically, unlike previous reconstruction objectives, which samples the timestep $t$ from a uniform distribution, we skew the sampling distribution so that a higher probability is assigned to a larger $t$. The Denoising Loss for relation-focal importance sampling becomes:
%
\begin{gather}
\begin{split}
    L_\mathrm{denoise} &= \Eb{t\sim f, \bx_0, \bepsilon}{ { \left\| \bepsilon - \bepsilon_\theta(\bx_t, t, \tau_\theta(c)) \right\|^2}}, \qquad \\
    f(t)        &= \frac{1}{T}(1 - \alpha \cos{\frac{\pi t}{T}}),
    \label{eq:importance_sampling}
\end{split}
\end{gather}
%
where $f(t)$ is the importance sampling function, which characterizes the probability density function to sample $t$ from. The skewness of $f(t)$ increases with $\alpha\,{\in}\,(0, 1]$. We set $\alpha=0.5$ throughout our experiments.
%
The overall optimization objective of the ReVersion framework is written as:
\begin{gather}
    \langle{R}\rangle = \argmin_{\langle{r}\rangle} (\lambda_\mathrm{steer}L_\mathrm{steer} + \lambda_\mathrm{denoise}L_\mathrm{denoise}), \label{eq:overall_loss}
\end{gather}
%
where $\lambda_\mathrm{steer}$ and $\lambda_\mathrm{denoise}$ are the weighting factors. 

\section{The ReVersion Benchmark}



% Result Figure
\begin{figure*}[t]
  \centering
   \includegraphics[width=0.99\linewidth]{figures/fig_results.pdf}
   \caption{\textbf{Qualitative Results}. Our ReVersion framework successfully captures the relation that co-exists in the exemplar images, and applies the extracted relation prompt \R to compose novel entities. 
   }
   \label{fig:qualitative_results}
\end{figure*}



To facilitate fair comparison for Relation Inversion, we present the \textbf{ReVersion Benchmark}. It consists of diverse \textit{relations} and \textit{entities}, along with a set of well-defined text \textit{descriptions}. This benchmark can be used for conducting qualitative and quantitative evaluations.


\noindent\textbf{Relations and Entities.} We define ten representative object relations with different abstraction levels, ranging from basic spatial relations (\eg,~\textit{``on top of''}), entity interactions (\eg, \textit{``shakes hands with''}), to abstract concepts (\eg,~\textit{``is carved by''}). A wide range of entities, such as animals, human, household items, are involved to further increase the diversity of the benchmark.

\noindent\textbf{Exemplar Images and Text Descriptions.} For each relation, we collect four to ten exemplar images containing different entities. We further annotate several text templates for each exemplar image to describe them with different levels of details\footnote{For example, a photo of a cat sitting on a box could be annotated as \textbf{1)} \textit{``cat \R box"}, \textbf{2)} \textit{``an orange cat \R a black box"} and \textbf{3)} \textit{``an orange cat \R a black box, with trees in the background"}.}. These training templates can be used for the optimization of the relation prompt.

\noindent\textbf{Benchmark Scenarios.} To validate the robustness of the relation inversion methods, we design 100 inference templates composing of different object entities for each of the ten relations. This provides a total of 1,000 inference templates for performance evaluation.
\section{Experiments}



% Baseline Comparison Figure
\begin{figure*}[t]
  \centering
   \includegraphics[width=0.99\linewidth]{figures/fig_baseline_comparison.pdf}
   \caption{\textbf{Qualitative Comparisons with Existing Methods}. Our method significantly surpasses both baselines in terms of relation accuracy and entity accuracy. 
   }
   \label{fig:baseline_comparison}
\end{figure*}


We present qualitative and quantitative results in this section, and more experiments and analysis are in the Supplementary File.
We adopt Stable Diffusion~\cite{rombach2022ldm} for all experiments since it achieves a good balance between quality and speed. We generate images at $512\times512$ resolution.


\subsection{Comparison Methods}

\noindent \textbf{Text-to-Image Generation using Stable Diffusion~\cite{rombach2022ldm}}. We use the original Stable Diffusion 1.5 as the text-to-image generation baseline. Since there is no ground-truth textual description for the relation in each set of exemplar images, we use natural language that can best describe the relation to replace the \R token.
%
For example, in Figure~\ref{fig:baseline_comparison} (a), the co-existing relation in the reference images can be roughly described as \textit{``is painted on"}. Thus we use it to replace the \R token in the inference template \textit{``Spiderman \R building''}, resulting in a sentence \textit{``Spiderman is painted on building''}, which is then used as the text prompt for generation.

\noindent \textbf{Textual Inversion~\cite{gal2022textualinversion}}.
For fair comparison with our method developed on Stable Diffusion 1.5, we use the \textit{diffuser}~\cite{diffusers} implementation of Textual Inversion~\cite{gal2022textualinversion} on Stable Diffusion 1.5. Based on the default hyper-parameter settings, we tuned the learning rate and batch size for its optimal performance on our Relation Inversion task. We use Textual Inversion's LDM objective to optimize \R for 3000 iterations, and generate images using the obtained \R.

\subsection{Qualitative Comparisons}
%
In Figure~\ref{fig:qualitative_results}, we provide the generation results using \R inverted by ReVersion. We observe that our framework is capable of 1) synthesizing the entities in the inference template and 2) ensuring that entities follow the relation co-existing in the exemplar images.
%
We then compare our method with 1) text-to-image generation via Stable Diffusion~\cite{rombach2022ldm} and 2) Textual Inversion~\cite{gal2022textualinversion} in Figure~\ref{fig:baseline_comparison}.
%
For example, in the first row, although the text-to-image baseline successfully generates both entities (Spiderman and building), it fails to \textit{paint} Spiderman on the building as the exemplar images do. Text-to-image generation severely relies on the bias between two entities: Spiderman usually \textit{climbs}/\textit{jumps} on the buildings, instead of being \textit{painted} onto the buildings. Using exemplar images and our ReVersion framework alleviates this problem. In Textual Inversion, entities in the exemplar images like canvas are leaked to \R, such that the generated image shows a Spiderman on the canvas even when the word  \textit{``canvas"} is not in the inference prompt.

\subsection{Quantitative Comparisons}

We conduct a user study with 37 human evaluators to assess the performance of our ReVersion framework on the Relation Inversion task. We sampled 20 groups of images. Each group has three images generated by different methods. For each group, apart from generated images, the following information is presented: 1) exemplar images of a particular relation 2) text description of the exemplar images. We then ask the evaluators to vote for the best generated image respect to the following metrics.



\begin{table}[t]
  \centering
  \caption{\textbf{Quantitative Results.} Percentage of votes where users favor our results vs. comparison methods. Our method outperforms the baselines under both metrics.}
  \vspace{-0.5em}
    \small 
    \begin{tabular}{l|c|c}
    \Xhline{1pt}
    \textbf{Method} & \textbf{Relation} & \textbf{Entity} \\ \Xhline{1pt}
    % \midrule
    Text-to-Image Generation & 7.86\% & 15.49\%  \\ 
    Textual Inversion & 8.94\% &  10.05\% \\ 
    \textbf{Ours} & \textbf{83.20\%} & \textbf{74.46\%} \\
    \Xhline{1pt}
  \end{tabular}
  \label{tab:quantitative}
\end{table}



\noindent \textbf{Entity Accuracy}.
Given an inference template in the form of \textit{``Entity A \R Entity B''}, we ask evaluators to determine whether \textit{Entity A} and \textit{Entity B} are both authentically generated in each image.


\noindent \textbf{Relation Accuracy}.
Human evaluators are asked to evaluate whether the relations of the two entities in the generated image are consistent with the relation co-existing in the exemplar images.
%
As shown in Table \ref{tab:quantitative}, our method clearly obtains better results under the two quality metrics.


\begin{table}[t]
  \centering
  \caption{\textbf{Ablation Study}. Suppressing steering or importance sampling introduces performance drops, which shows the necessity of both relation-steering and importance sampling.
  }
  \vspace{-0.5em}
    \small % 
    \begin{tabular}{l|c|c}
    \Xhline{1pt}
    \textbf{Method} & \textbf{Relation} & \textbf{Entity} \\ \Xhline{1pt}
    % \midrule
    w/o Steering & 11.20\% & 10.90\% \\ 
    w/o Importance Sampling & 11.20\% & 13.62\% \\ 
    \textbf{Ours} & \textbf{77.60\%} & \textbf{75.48\%} \\
    \Xhline{1pt}
  \end{tabular}
  \label{tab:ablation}
\end{table}



% Ablation Comparison Figure
\begin{figure*}[t]
  \centering
  \vspace{-10pt}
   \includegraphics[width=0.99\linewidth]{figures/fig_ablation_comparison.pdf}
   \vspace{-5pt}
   \caption{\textbf{Qualitative Comparisons with Ablation Variants}. Without relation-steering, \R suffers from appearance leak (\eg, white puppy in (a), gray background in (b)) and inaccurate relation capture (\eg, dog not being on top of plate in (b)). Without importance sampling, \R focuses on lower-level visual details (\eg, rattan around puppy in (a)) and misses high-level relations.
   }
   \vspace{-5pt}
   \label{fig:ablation_comparison}
\end{figure*}

\subsection{Ablation Study}

From Table~\ref{tab:ablation}, we observe that removing steering or importance sampling results in deterioration in both relation accuracy and entity accuracy. This corroborates our observations that 1) relation-steering effectively guides \R towards the relation-dense ``preposition prior'' and disentangles \R away from exemplar entities, and 2) importance sampling emphasizes high-level relations over low-level details, aiding \R to be relation-focal. We further show qualitatively the necessity of both modules in Figure~\ref{fig:ablation_comparison}.


\noindent \textbf{Effectiveness of Relation-Steering.}
%
In ``w/o Steering'', we remove the Steering Loss $L_\mathrm{steer}$ in the optimization process. As shown in Figure~\ref{fig:ablation_comparison}~(a), the appearance of the white puppy in the lower-left exemplar image is leaked into \R, resulting in similar puppies in the generated images. In Figure~\ref{fig:ablation_comparison}~(b), many appearance elements are leaked into \R, such as the gray background, the black cube, and the husky dog. The dog and the plate also do not follow the relation of \textit{``being on top of''} as shown in exemplar images. Consequently, the images generated via \R do not present the correct relation and introduced unwanted leaked imageries.


\noindent \textbf{Effectiveness of Importance Sampling.}
%
We replace our relation-focal importance sampling with uniform sampling, and observe that \R pays too much attention to low-level details rather than high-level relations. For instance, in Figure~\ref{fig:ablation_comparison}~(a) ``w/o Importance Sampling'', the basket rattan wraps around the puppy's head in the same way as the exemplar image, instead of containing the puppy inside.


\section{Conclusion}

In this work, we take the first step forward and propose the \textbf{\textit{Relation Inversion}} task, which aims to learn a relation prompt to capture the relation that co-exists in multiple exemplar images. Motivated by the \textbf{\textit{preposition prior}}, our \textit{relation-steering contrastive learning} scheme effectively guides the relation prompt towards relation-dense regions in the text embedding space. We also contribute the \textbf{\textit{ReVersion Benchmark}} for performance evaluation. Our proposed \textbf{\textit{Relation Inversion}} task would be a good inspiration for future works in various domains such as generative model inversion, representation learning, few-shot learning, visual relation detection, and scene graph generation.

\noindent \textbf{Limitations}.
Our performance is dependent on the generative capabilities of Stable Diffusion. It might produce sub-optimal synthesis results for entities that Stable Diffusion struggles at, such as human body and human face. 


\noindent\textbf{Potential Negative Societal Impacts}.
The entity relational composition capabilities of \textit{ReVersion} could be applied maliciously on real human figures.


%%%%%%%%% References

{\small
\bibliographystyle{ieee_fullname}
\bibliography{main}
}

%%%%%%%%% Supp

\onecolumn
\appendix
\section*{Supplementary}
\renewcommand\thesection{\Alph{section}}
\renewcommand\thefigure{A\arabic{figure}}
\renewcommand\thetable{A\arabic{table}}
\setlength{\parskip}{3pt}

\noindent
% Our supplementary materials contains: 

% \begin{enumerate}
%     \setlength\itemsep{0em}
%     \item This \textbf{\textit{supplementary file}}.
%     %
%     \item A \textbf{\textit{demo video}}: please see the attached \texttt{demo\_video.mp4} file.
%     %
% \end{enumerate}



In this \textbf{\textit{supplementary file}}, we provide more experimental details in Section~\ref{sec:experiment}, and elaborate on the ReVersion Benchmark details in Section~\ref{sec:benchmark_details}. We then provide further explanations on basis prepositions in Section~\ref{sec:prepositions}. We also discuss our limitations in Section~\ref{sec:limitations}, and the potential societal impacts of our work in Section~\ref{sec:impact}.  At the end of the supplementary file, we show various qualitative results of ReVersion in Section~\ref{sec:qualitative}.





\section{More Experimental Comparisons}
\label{sec:experiment}


In this section, we provide more experimental comparison results and analysis. For each method in comparison, we use the 1,000 inference templates in the ReVersion benchmark, and generate 10 images using each template.


\begin{table}[b]
  \centering
  \caption{\textbf{Additional Quantitative Results (Baselines).}}
  \vspace{-0.5em}
    \small 
    \begin{tabular}{l|c|c}
    \Xhline{1pt}
    \textbf{Method} & \textbf{Relation Accuracy Score (\%) $\uparrow$} & \textbf{Entity Accuracy Score (\%) $\uparrow$} \\ \Xhline{1pt}
    Text-to-Image Generation~\cite{rombach2022ldm} &  35.16\% &  \textbf{28.96\%}  \\ 
    Textual Inversion~\cite{gal2022textualinversion} &  37.85\% &   26.79\% \\ 
    \textbf{Ours} & \textbf{38.17\%} & \underline{28.20}\% \\
    \Xhline{1pt}
  \end{tabular}
  \label{tab:baseline}
\end{table}


\begin{table}[b]
  \centering
  \caption{\textbf{Additional Quantitative Results (Ablation Study).}}
  \vspace{-0.5em}
    \small
    \begin{tabular}{l|c|c}
    \Xhline{1pt}
    \textbf{Method} & \textbf{Relation Accuracy Score (\%) $\uparrow$} & \textbf{Entity Accuracy Score (\%) $\uparrow$} \\ \Xhline{1pt}
    Ours w/o Steering &  37.48\% &  27.66\%  \\ 
    Ours w/o Importance Sampling &  34.64\% &   27.90\% \\ 
    \textbf{Ours} & \textbf{38.17\%} & \textbf{28.20\%} \\
    \Xhline{1pt}
  \end{tabular}
  \label{tab:ablation}
\end{table}


\subsection{Relation Accuracy Score}


We devise objective evaluation metric to measure the quality and accuracy of the inverted relation. To do this, we train relation classifiers that categorize the ten relations in our ReVersion benchmark. We then use these classifiers to determine whether the entities in the generated images follow the specified relation. We employ PSGFormer~\cite{yang2022psg}, a pre-trained scene-graph generation network, to extract the relation feature vectors from a given image. The feature vectors are averaged-pooled and fed into linear SVMs for classification.

We calculate the \textit{Relation Accuracy Score} as the percentage of generated images that follow the relation class in the exemplar images. Table~\ref{tab:baseline} shows that our method outperforms text-to-image generation~\cite{rombach2022ldm} and Textual Inversion~\cite{gal2022textualinversion}. Additionally, Table~\ref{tab:ablation} reveals that removing the steering scheme or importance sampling scheme results in a performance drop in relation accuracy.

\subsection{Entity Accuracy Score}

To evaluate whether the generated image contains the entities specified by the text prompt, we compute the CLIP score~\cite{radford2021clip} between a revised text prompt and the generated image, which we refer to as the \textit{Entity Accuracy Score}.

CLIP~\cite{radford2021clip} is a vision-language model that has been trained on large-scale datasets. It uses an image encoder and a text encoder to project images and text into a common feature space. The CLIP score is calculated as the cosine similarity between the normalized image and text embeddings. A higher score usually indicates greater consistency between the output image and the text prompt.
%
In our approach, we calculate the CLIP score between the generated image and the revised text prompt ``$E_{A}$\textit{,} $E_{B}$'', which only includes the entity information. 

In Table~\ref{tab:baseline}, we observe that our method  outperforms Textual Inversion in terms of entity accuracy. This is because the \R learned by Textual Inversion might contain leaked entity information, which might distract the model from generating the desired ``$E_{A}$'' and ``$E_{B}$''. Our steering loss effectively prevents entity information from leaking into \R, allowing for accurate entity synthesis. Furthermore, our approach achieves comparable entity accuracy score with text-to-image generation using Stable Diffusion~\cite{rombach2022ldm}, and significantly surpasses it in terms of relation accuracy. Table~\ref{tab:ablation} shows that removing the steering or importance sampling scheme results in a drop in entity accuracy.


% Figure: DBooth
\begin{figure*}[h]
  \centering
   \includegraphics[width=0.99\linewidth]{figures/fig_supp_dbooth.pdf}
   \caption{\textbf{Qualitative Comparison with DreamBooth~\cite{ruiz2022dreambooth}.} In (a), the Spiderman generated by DreamBooth is mostly climbing on the building rather than being a painting. In (b), DreamBooth fails to capture the \textit{``sits back to back with''} relation. In (c), while DreamBooth successfully captures the relation, the appearance of the basket from exemplar images are severely leaked into the generated images via \R.
   }
   \label{fig_supp_dbooth}
\end{figure*}




%%%%%%%%%%%%%%%%%%%%%%%%%%%%%%%%%%%%%%%%%%%%%%%%%%%%%%%%%%%%%%
\subsection{Comparison with Fine-Tuning Based Method}
%
We further compare our method with a fine-tuning based method, DreamBooth~\cite{ruiz2022dreambooth}. In order to adapt DreamBooth to our relation inversion task, we follow the original implementation to design a text prompt ``A photo of \R relation" containing the unique identifier ``\R'' to fine-tune the model. The class-specific prior preservation loss is also added with a text prompt ``A photo of relation" to avoid overfitting and language drift. As shown in Figure~\ref{fig_supp_dbooth}, directly using DreamBooth on our task could result in poor object relation and appearance leakage. For example, in Figure~\ref{fig_supp_dbooth}~(a) the Spiderman is mostly climbing on the building rather than being a painting, while in Figure~\ref{fig_supp_dbooth}~(c) the appearance of the basket in exemplar images is severely leaked into the DreamBooth generated images.




\section{ReVersion Benchmark Details}
\label{sec:benchmark_details}

In this section, we provide the details of our ReVersion Benchmark. The full benchmark will be publicly available.



\begin{figure*}[h!t]
  \centering
   \includegraphics[width=0.99\linewidth]{figures_supp/fig_supp_benchmark.pdf}
   \caption{\textbf{Benchmark Sample}. We present \textit{exemplar images} and \textit{text descriptions} that illustrate the relation where ``$E_{A}$ \textit{sits back to back with} $E_{B}$''. The \textit{exemplar images} feature both human figures and animals to demonstrate the invariant \textit{``back to back''} relationship in various scenarios. The \textit{text descriptions} are provided at several levels, ranging from simple class name mentions to detailed descriptions of the entities and their surroundings. During optimization, the \R{} in each description will be replaced with the learnable relation prompt.
   }
   \vspace{10pt}
   \label{fig_supp_benchmark}
\end{figure*}




\subsection{Relations}
To benchmark the Relation Inversion task, we define ten diverse and representative object relations as follows:
% \vspace{-0.4em}

\begin{itemize}
    \setlength\itemsep{0em}
    \item $E_{A}$ \textbf{\textit{is painted on (the surface of)}} $E_{B}$
    \item $E_{A}$ \textbf{\textit{is carved by / is made of the material of}} $E_{B}$
    \item $E_{A}$ \textbf{\textit{shakes hands with}} $E_{B}$
    \item $E_{A}$ \textbf{\textit{hugs}} $E_{B}$
    \item $E_{A}$ \textbf{\textit{sits back to back with}} $E_{B}$
    \item $E_{A}$ \textbf{\textit{is contained inside}} $E_{B}$
    \item $E_{A}$ \textbf{\textit{on / is on top of}} $E_{B}$
    \item $E_{A}$ \textbf{\textit{is hanging from}} $E_{B}$
    \item $E_{A}$ \textbf{\textit{is wrapped in}} $E_{B}$
    \item $E_{A}$ \textbf{\textit{rides (on)}} $E_{B}$
\end{itemize}
%
%
where $E_{A}$ and $E_{B}$ are the two entities that follow the specified relation. It is worth mentioning that the relations can be best described by the exemplar images, and the text descriptions provided above are simply approximated summarizations of the true relations.

\subsection{Exemplar Images}
A wide range of entities, such as animals, human, household items, are involved to further increase the diversity of the benchmark. In Figure~\ref{fig_supp_benchmark}, we show the \textit{exemplar images} and \textit{text descriptions} for the relation ``$E_{A}$ \textit{sits back to back with} $E_{B}$''. The exemplar images contain both human figures and animals to emphasize the invariant \textit{``back to back''} relation in different scenarios. 


\subsection{Text Descriptions}
As shown in Figure~\ref{fig_supp_benchmark}, the \textit{text descriptions} for each image contains several levels, from short sentences which only mention the class names, to complex and comprehensive sentences that describe each entity and the scene backgrounds. The \R{} in each description will be replaced by the learnable relation prompt during optimization.




\subsection{Inference Templates}
To evaluate the performance of relation inversion methods, we devise 100 inference templates for each relation. The inference templates contains diverse entity combinations to test the robustness and generalizability of the inverted relation \R. To quantitatively evaluate relation inversion performance, we use each inference template to synthesize 10 images, resulting in a total of 1,000 synthesized images for each inverted \R.

Below, we show the 100 inference templates for the relation ``$E_{A}$ \textit{sits back to back with} $E_{B}$'':
%
\vspace{-0.4em}
\begin{itemize}
    \setlength\itemsep{0em}
    \item \textit{man \R{} man, man \R{} woman, man \R{} child, man \R{} cat, man \R{} rabbit, man \R{} monkey, man \R{} dog, man \R{} hamster, man \R{} kangaroo, man \R{} panda, }
    \item \textit{woman \R{} man, woman \R{} woman, woman \R{} child, woman \R{} cat, woman \R{} rabbit, woman \R{} monkey, woman \R{} dog, woman \R{} hamster, woman \R{} kangaroo, woman \R{} panda, } 
    \item \textit{child \R{} man, child \R{} woman, child \R{} child, child \R{} cat, child \R{} rabbit, child \R{} monkey, child \R{} dog, child \R{} hamster, child \R{} kangaroo, child \R{} panda,  }
    \item \textit{cat \R{} man, cat \R{} woman, cat \R{} child, cat \R{} cat, cat \R{} rabbit, cat \R{} monkey, cat \R{} dog, cat \R{} hamster, cat \R{} kangaroo, cat \R{} panda,  }
    \item \textit{rabbit \R{} man, rabbit \R{} woman, rabbit \R{} child, rabbit \R{} cat, rabbit \R{} rabbit, rabbit \R{} monkey, rabbit \R{} dog, rabbit \R{} hamster, rabbit \R{} kangaroo, rabbit \R{} panda,  }
    \item \textit{monkey \R{} man, monkey \R{} woman, monkey \R{} child, monkey \R{} cat, monkey \R{} rabbit, monkey \R{} monkey, monkey \R{} dog, monkey \R{} hamster, monkey \R{} kangaroo, monkey \R{} panda,  }
    \item \textit{dog \R{} man, dog \R{} woman, dog \R{} child, dog \R{} cat, dog \R{} rabbit, dog \R{} monkey, dog \R{} dog, dog \R{} hamster, dog \R{} kangaroo, dog \R{} panda,  }
    \item \textit{hamster \R{} man, hamster \R{} woman, hamster \R{} child, hamster \R{} cat, hamster \R{} rabbit, hamster \R{} monkey, hamster \R{} dog, hamster \R{} hamster, hamster \R{} kangaroo, hamster \R{} panda,}  
    \item \textit{kangaroo \R{} man, kangaroo \R{} woman, kangaroo \R{} child, kangaroo \R{} cat, kangaroo \R{} rabbit, kangaroo \R{} monkey, kangaroo \R{} dog, kangaroo \R{} hamster, kangaroo \R{} kangaroo, kangaroo \R{} panda,  }
    \item \textit{panda \R{} man, panda \R{} woman, panda \R{} child, panda \R{} cat, panda \R{} rabbit, panda \R{} monkey, panda \R{} dog, panda \R{} hamster, panda \R{} kangaroo, panda \R{} panda}
\end{itemize}



\section{Further Explanations on Basis Prepositions}
\label{sec:prepositions}


As stated in the manuscript, we devise a set of basis prepositions to steer the learning process of the relation prompt. Specifically, we collect a comprehensive list of $\sim$100 prepositions from~\cite{stevenson2010oxford}, and drop the prepositions that describes non-visual-relations (\ie, temporal relations, causal relations, \textit{etc}.), while keep the ones that are related to visual relations. For example, the prepositional word \textit{``until"} is discarded as a temporal preposition, while words like \textit{``above"}, \textit{``beneath"}, \textit{``toward"} will be kept as plausible basis prepositions.

The basis preposition set contains a total of 56 words, listed as follows:
%
\begin{multicols}{4}[\setlength\itemsep{-2em}]
\begin{itemize}
    \item \textit{aboard}
    \item \textit{about}
    \item \textit{above}
    \item \textit{across}
    \item \textit{after}
    \item \textit{against}
    \item \textit{along}
    \item \textit{alongside}
    \item \textit{amid}
    \item \textit{amidst}
    \item \textit{among}
    \item \textit{amongst}
    \item \textit{anti}
    \item \textit{around}
    \item \textit{astride}
    \item \textit{at}
    \item \textit{atop}
    \item \textit{before}
    \item \textit{behind}
    \item \textit{below}
    \item \textit{beneath}
    \item \textit{beside}
    \item \textit{between}
    \item \textit{beyond}
    \item \textit{by}
    \item \textit{down}
    \item \textit{following}
    \item \textit{from}
    \item \textit{in}
    \item \textit{including}
    \item \textit{inside}
    \item \textit{into}
    \item \textit{near}
    \item \textit{of}
    \item \textit{off}
    \item \textit{on}
    \item \textit{onto}
    \item \textit{opposite}
    \item \textit{out}
    \item \textit{outside}
    \item \textit{over}
    \item \textit{past}
    \item \textit{regarding}
    \item \textit{round}
    \item \textit{through}
    \item \textit{throughout}
    \item \textit{to}
    \item \textit{toward}
    \item \textit{towards}
    \item \textit{under}
    \item \textit{underneath}
    \item \textit{up}
    \item \textit{upon}
    \item \textit{versus}
    \item \textit{with}
    \item \textit{within}
\end{itemize}
\end{multicols}


\section{Implementation Details}
\label{sec:implementation_details}


All experiments are conducted on $512{\times}512$ image resolution. 
To ensure that the numerical values $\lambda_\mathrm{denoise}L_\mathrm{denoise}$ and $\lambda_\mathrm{steer}L_\mathrm{steer}$ are in comparable order of magnitude, we set $\lambda_\mathrm{denoise}=1.0$ and  $\lambda_\mathrm{steer}=0.01$. The temperature parameter $\gamma$ in the steering loss $L_\mathrm{steer}$ is set as 0.07, following~\cite{he2020momentum}. During the optimization process, we first initialize our relation prompt \R using the word \textit{``and''}, then optimize the prompt using the AdamW~\cite{loshchilov2017adamw} optimizer for 3,000 steps, with learning rate $2.5{\times}10^{-4}$ and batch size 2. In each iteration, 8 positive samples are randomly selected from the basis preposition set. During the inference process, we use classifier-free guidance for all experiments including the baselines and ablation variants, with a constant guidance weight 7.5.



\section{Potential Societal Impacts}
\label{sec:impact}


Although \textit{ReVersion} can generate diverse entity combinations through inverted relations, this capability can also be exploited to synthesize real human figures interacting in ways they never did. As a result, we strongly advise users to only use \textit{ReVersion} for proper recreational purposes.

The rapid advancement of generative models has unlocked new levels of creativity but has also introduced various societal concerns. First, it is easier to create false imagery or manipulate data maliciously, leading to the spread of misinformation. Second, data used to train these models might be revealed during the sampling process without explicit consent from the data owner~\cite{tinsley2021face}. Third, generative models can suffer from the biases present in the training data~\cite{esser2020note}. We used the pre-trained Stable Diffusion~\cite{rombach2022ldm} for \textit{ReVersion}, which has been shown to suffer from data bias in certain scenarios. For example, when prompted with the phrase \textit{``a professor''}, Stable Diffusion tends to generate human figures that are white-passing and male-passing. We hope that more research will be conducted to address the risks and biases associated with generative models, and we advise everyone to use these models with discretion.

\section{More Qualitative Results}
\label{sec:qualitative}

We show various qualitative results in Figure~\ref{fig_supp_styles}-\ref{fig_supp_hanging_from_shake_hands}, which are located at the end of this Supplementary File.


\subsection{ReVersion with Diverse Styles and Backgrounds}
As shown in Figure~\ref{fig_supp_styles}, we apply the \R inverted by ReVersion in scenarios with diverse backgrounds and styles, and show that \R robustly adapt these environments with impressive results.


% Figure: Styles
\begin{figure*}[t]
  \centering
   \includegraphics[width=0.99\linewidth]{figures/fig_supp_styles.pdf}
   \caption{\textbf{ReVersion for Diverse Styles and Backgrounds}. The \R inverted by ReVersion can be applied robustly to related entities in scenes with diverse backgrounds or styles.
   }
   \label{fig_supp_styles}
\end{figure*}


\subsection{ReVersion with Arbitrary Entity Combinations}
In Figure~\ref{fig_supp_painted_on} and \ref{fig_supp_carved_by}, we show that the \R inverted by ReVersion can be applied to robustly relate arbitrary entity combinations. For example, in Figure~\ref{fig_supp_painted_on}, for the \R  extracted from the exemplar images where one entity is \textit{``painted on''} the other entity, we enumerate over all combinations among \textit{``\{cat / flower / guitar / hamburger / Michael Jackson / Spiderman\} \R \{building / canvas / paper / vase / wall\}''}, and observe that \R successfully links these entities together via exactly the same relation in the exemplar images. 

% Figure: Grid - Painted On
\begin{figure*}[t]
  \centering
   \includegraphics[width=0.65\linewidth]{figures/fig_supp_painted_on.pdf}
   \caption{\textbf{Arbitrary Entity Combinations}. The \R inverted by ReVersion can be robustly applied to arbitrary entity combinations. For example, for the \R  extracted from the exemplar images where one entity is \textit{``painted on''} the other entity, we enumerate over all combinations among \textit{``\{cat / flower / guitar / hamburger / Michael Jackson / Spiderman\} \R \{building / canvas / paper / vase / wall\}''}, and observe that \R successfully links these entities together via exactly the same relation in the exemplar images.
   }
   \label{fig_supp_painted_on}
\end{figure*}

% Figure: Grid - Carved By
\begin{figure*}[t]
  \centering
   \includegraphics[width=0.99\linewidth]{figures/fig_supp_carved_by.pdf}
   \caption{\textbf{Arbitrary Entity Combinations}. The \R inverted by ReVersion can be applied to arbitrary entity combinations. For example, for the \R  extracted from the exemplar images where one entity is \textit{``is made of the material of / is carved by''} the other entity, we enumerate over all combinations among \textit{``\{cat / swan / horse / lion / rose / rabbit\} \R \{apple / carrot / clay / glass / jade / marble / metal / wood\}''}, and observe that \R successfully links these entities together via exactly the same relation in the exemplar images.
   }
   \label{fig_supp_carved_by}
\end{figure*}


% Figure: Random - Ride On & Hug
\begin{figure*}[t]
  \centering
   \includegraphics[width=0.99\linewidth]{figures/fig_supp_ride_on_hug.pdf}
   \caption{\textbf{More Qualitative Results}.
   }
   \label{fig_supp_ride_on_hug}
\end{figure*}


% Figure: Random - Inside A & Back2Back
\begin{figure*}[t]
  \centering
   \includegraphics[width=0.99\linewidth]{figures/fig_supp_inside_a_back2back.pdf}
   \caption{\textbf{More Qualitative Results}.
   }
   \label{fig_supp_inside_a_back2back}
\end{figure*}

% Figure: Random - Inside B
\begin{figure*}[t]
  \centering
   \includegraphics[width=0.99\linewidth]{figures/fig_supp_inside_b.pdf}
   \caption{\textbf{More Qualitative Results}.
   }
   \label{fig_supp_inside_b}
\end{figure*}


% Figure: Random - Hanging From & Shake Hands
\begin{figure*}[t]
  \centering
   \includegraphics[width=0.99\linewidth]{figures/fig_supp_hanging_from_shake_hands.pdf}
   \caption{\textbf{More Qualitative Results}.
   }
   \label{fig_supp_hanging_from_shake_hands}
\end{figure*}




\subsection{Additional Qualitative Results}
We show additional qualitative results of ReVersion in Figure~\ref{fig_supp_ride_on_hug}, \ref{fig_supp_inside_a_back2back}, \ref{fig_supp_inside_b}, and \ref{fig_supp_hanging_from_shake_hands}.



\end{document}