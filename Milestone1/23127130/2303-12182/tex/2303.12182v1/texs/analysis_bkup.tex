%%%%%%%%%%%%%%%%%%%%
% Backup

%%%%%%%%%%%%%%%%%%%
\section{Analysis}
We now demonstrate that under mild conditions on the state-based score function $\stbsf$, the origin of the closed-loop system obtained when control laws~\eqref{eq:omega-ctrl-law} and~\eqref{eq:v-ctrl-law} are applied to system ~\eqref{eq:omega-ctrl-law-ss} and~\eqref{eq:v-ctrl-law-ss} is [AS?].
This closed loop is 
\begin{align}
     \dot\theta &= \gamma \stbsf,\label{eq:thetacloop}\\
     \dot d &= \vctrl \sin\theta,\label{eq:dcloop}
\end{align}
where $\stbsf = \Ftd$, a StBSF.

These conditions are simply that $F = 0$ at the origin and $\stbsf$ be monotonic in each of its arguments, that is, 
\begin{align}
    \stbsf(\vec 0) &= 0,\label{eq:conditionsonstbsf1}\\
    \frac{\partial{\stbsf}}{\partial \theta} &< 0\ \forall (\theta,d) \in \state, \text{ and} \label{eq:conditionsonstbsf2}\\
    \frac{\partial{\stbsf}}{\partial d} &< 0\ \forall (\theta,d) \in \state.\label{eq:conditionsonstbsf3}
\end{align}
These conditions imply the following properties.


\begin{lemma}
There exists a continuous function $h \colon \R \to \R$ and an interval $\thetainterval = [-\theta^*,\theta^*]$ such that $h(0)=0$ and $\forall \theta \in \thetainterval$ the following hold:
\begin{enumerate}
    \item $h'(\theta) < 0$
    \item $\stbsf(\theta,h(\theta)) = 0$
    \item $h(\theta) \leq 0$
\end{enumerate}
\end{lemma}
\begin{proof}
    [Implicit Function Theorem + Conditions]
\end{proof}

The next result shows that 
\def\cinvset{S_c} % positively invariant set given by level set of F
\begin{lemma}
    Consider the set $\cinvset = \{\vx \in  \state \colon \left| \stbsf(\vx) \right| \leq c \}$, where $\stbsf$ satisfies conditions~\eqref{eq:conditionsonstbsf1}-\eqref{eq:conditionsonstbsf3}. For any $c^*>0$, there exists $\alpha, \beta, \gamma$ such that $\cinvset$ is positively invariant for any $c \geq c^*$ under dynamics~\eqref{eq:thetacloop}-\eqref{eq:dcloop}. 
\end{lemma}

\begin{theorem}
Let $\stbsf$ satisfy conditions~\eqref{eq:conditionsonstbsf1}-\eqref{eq:conditionsonstbsf3}. Let $\alpha$, $\beta$, and $\gamma$ satisfy \dots. %Then, the origin of~\eqref{eq:thetacloop}-\eqref{eq:dcloop} is asymptotically stable.
\end{theorem}
\begin{proof}
    Consider the Lyapunov-like function $V(\vx) = \frac{1}{2} F(\vx)^2$. Then, the derivative of $V(\vx)$ along solutions of~\eqref{eq:thetacloop}-\eqref{eq:dcloop} is 
    \begin{align}
        \dot V(\vx) &= \frac{\partial V}{\partial \theta} \dot \theta +\frac{\partial V}{\partial d} \dot d   \\
        & = F(\vx) \frac{\partial F}{\partial \theta} \gamma F(\vx) +  F(\vx) \frac{\partial F}{\partial d} \vctrl \sin\theta \\
        & =   -  \left| \frac{\partial F}{\partial \theta}  \right| \gamma F(\vx)^2 +   \frac{\partial F}{\partial d} F(\vx) \vctrl \sin\theta \label{eq:v1dot}
    \end{align}
    % Consider the sets \begin{align}
    %     S_1 = \{(\theta,d) \in \state : F(\vx) \theta > 0,  | \theta | < \pi \} \\
    %     S_2 = \{(\theta,d) \in \state : F(\vx) \theta < 0,  | \theta | < \pi \}
    % \end{align}
    % Then, $$ (\theta,d) \in S_1 \implies  \dot V(\theta,d) < 0$$ 
    Equation~\eqref{eq:v1dot} implies that a sufficient condition for $\dot V(\vx)$ to decrease along solutions is
    \begin{equation}
        \left| \frac{\partial F}{\partial \theta}  \right| \gamma  |F| >      \left|  \frac{\partial F}{\partial d}\right|\vctrl \left| \sin\theta  \right| \label{eq:v1dotnegcondition}   
    \end{equation}
    
All terms on the right hand side of inequality~\eqref{eq:v1dotnegcondition} are bounded, but the term on the left hand side is unbounded due to $|F|$. Thus, there exists some constant $\levelset$ such that $$F(\theta,d) > \levelset \implies V(\theta,d) <0$$
\end{proof}

\subsection{Preliminaries}
\begin{figure}[tb]
    \centering
    \begin{tikzpicture}[align = center,>=stealth',scale=2]
\tikzstyle{myedge}=[shorten >=1pt,auto,semithick]
\tikzstyle{axes}=[->,opacity = 1.0]
\coordinate (A) at (0,0);
\draw[fill=blue!20] (A) +(-1,-1) rectangle +(1,1);
\def\cxyline{(0.8,-1) (0.2,-0.4) (-0,0) (-0.4,0.1) (-0.7,0.5)  (-0.9,0.7) (-0.95,1)}
\draw [thick, name = cxyline,fill = green!20] plot [smooth] coordinates {\cxyline } -- (1,1) -- (1,-1) -- cycle;
\draw[very thick, red] plot [smooth] coordinates {\cxyline } (A)++(-45:0.25) node[above right,rotate=-45]{$F(d,\theta)=0$};
\draw[->,thick] (A) +(-1,0) -- +(1,0) node[right] {$\theta$};
\draw[->,thick] (A) +(0,-1) -- +(0,1) node[above] {$d$};
\end{tikzpicture}
    \caption{A depiction of the line $F(d,\theta) = 0$}
    \label{fig:my_label}
\end{figure}


\begin{align}
    \label{eq:ls-dbound} \invf{h}(\tlb) &\le \dlb < 0 < \dub \le \invf{h}(\dub) \\
    \label{eq:ls-tbound} -\pi &< \tlb < 0 < \tub < \pi \\
    \label{eq:ls-qone} \Qone &\triangleq \left\{ (\theta,d) : \theta\in(0,\tub), d\in[0,\dub) \right\} \\
    \label{eq:ls-qtwo} \Qtwo &\triangleq \left\{ (\theta,d) : \theta\in(\tlb, 0], d\in(0,\dub) \right\} \\
    \label{eq:ls-qthr} \Qthr &\triangleq \left\{ (\theta,d) : \theta\in(\tlb, 0), d\in(\dlb,0] \right\} \\
    \label{eq:ls-qfor} \Qfor &\triangleq \left\{ (\theta,d) : \theta\in[0,\tub), d\in(\dlb,0) \right\} \\
    \label{eq:ls-qone-line} \Qoneic &\triangleq \left\{ (\theta,d) : (\theta,d)\in\Qone, d\le -\frac{\dub}{\tub}\theta + \dub \right\} \\
    \label{eq:ls-qthr-line} \Qthric &\triangleq \left\{ (\theta,d) : (\theta,d)\in\Qthr, d\le -\frac{\dlb}{\tlb}\theta + \dlb \right\} \\
    \label{eq:ls-Xo} \Xo &\triangleq \Qoneic \cup \Qtwo \cup \Qthric \cup \Qfor \\
    \label{eq:ls-X} X &\triangleq \Xo \cup \Qone \cup \Qthr \\
    \label{eq:traj-with-ic} \vx(t;\vxz) &= \vxz + \int_0^t \dot\vx(\tau) d\tau
\end{align}


\subsection{LS Stability}
\begin{fact}
\label{fact:ls-stability}
    Let $\vxz\in\Xo$.
    Then, $\exists 0 < \alpha, \beta, \gamma < \infty$ s.t. the control laws~\eqref{eq:omega-ctrl-law} and~\eqref{eq:v-ctrl-law} ensure that, $\forall t>0$, $\vx(t)\in X$.\todo{BC: Change conditions to $\alpha \ge 0$}
\end{fact}


\begin{proof}
\emph{Quadrant 1}:
    Let $\vxz \in \Qoneic$.
    Define $L_1 \triangleq \Qoneic\cap\left\{ d = -\frac{\dub}{\tub}\theta + \dub  \right\}$ and $L_3 \triangleq \Qthric\cap\left\{ d = -\frac{\dlb}{\tlb}\theta - \dlb  \right\}$ similar to the proof of Fact~\ref{fact:d-axis-crossing}.
    Select
    \[ \uF = \min_{\vx\in(L_1\cup L_3)} |F(\vx)|, \]
    and let $\alpha,\beta,\gamma > 0$ such that
    \[ \frac{\beta'}{\gamma} < \uF \cdot \min\left\{ \frac{\dub}{\tub}, \frac{\dlb}{\tlb} \right\}. \]
    Fact~\ref{fact:d-axis-crossing} shows that with the choice of hyper-parameters listed previously, $\forall \vxz\in\Qoneic$, $\exists t < \infty$ s.t. $\vx(t;\vxz)\in\Qtwo$, where $\forall t'<t$, $\vx(t';\vxz)\in\Qone$.

    \emph{Quadrant 2}:
    Starting on the $d$ axis, note that $\forall \vxz\in\{(0,d) : d > 0 \}$, $\dot \theta < 0$ and $\dot d = 0$.
    Thus, we must enter $\Qtwo\setminus\left\{ (0,d) : d > 0 \right\}$.
    %Once in $\Qtk$ the proof is rather simple.
    Once in $\Qtwo\setminus\left\{ (0,d) : d > 0 \right\}$ the proof is rather simple.
    $\forall\vxz\in\Qtwo\setminus\left\{ (0,d) : d > 0 \right\}$, $\dot d < 0$, and on either side of the line $F(\theta,d) = 0$, $\dot\theta$ pushes us back towards this line.
    This ensures our trajectory $\vx(t;\vxz)$ remains bounded in $\Qtwo$ or exits across the $\theta$ axis into quadrant 3 along the line 
    $L_2 \triangleq \left\{(\theta,0) : \theta\in(\tlb,0) \right\}$.
    If we do not ever cross the axis, then we either remain in $\Qtwo$ for all time, or we are driven to the equilibrium.
    If we cross the $\theta$ axis within $L_2$, given our selection of $\alpha,\beta,\gamma$, the same argument given above for quadrants 1 and 2 holds for quadrants 3 and 4.
    While there is no guarantee of the trajectory heading towards $\vxe$, this does imply that $\forall t, \vx(t;\vxz)\in X$.
    In essence, if we cross the $\theta$ axis along $L_2$, we will be guaranteed to at least re-enter $X_0$ if it does not go to the equilibrium.
    This concludes the proof.
\end{proof}

\begin{remark}
Fact~\ref{fact:ls-stability} can be rewritten to show that $\exists 0<\alpha,\beta,\gamma<\infty$ s.t. control laws~\eqref{eq:omega-ctrl-law} and~\eqref{eq:v-ctrl-law} make the mobile robot a Lyapunov stable system using a slightly more restrictive initial condition set.
Define $X = \left\{ (\theta,d) | \theta\in(\tlb,\tub), d\in(\dlb,\dub) \right\}\setminus\{ (0,0) \}$.
Then, Fact~\ref{fact:ls-stability} implies that $\forall \epsilon>0$, $\exists \delta>0$ s.t. 
\[ || \vxz - \vxe ||_1 < \delta \implies || \vx(t;\vxz) - \vxe ||_\infty < \epsilon \]
$\forall t \ge 0$.\todo{BC: This isn't quite right, as I need to make $X$ a perfect square. This should be a good starting point for what may be interesting to show though.}
\end{remark}

\begin{fact}
    \label{fact:d-axis-crossing}
    Let $\vxz\in\Qoneic$.
    Then, $\exists 0<\alpha,\beta,\gamma<\infty$ such that the following two conditions hold:
    \begin{enumerate}
        \item \label{fact:d-axis-crossing-cond1} $\exists t<\infty$ s.t. $\vx(t;\vxz) = [0,d(t)]^\intercal$, where $d(t)<\dub$, and
        \item \label{fact:d-axis-crossing-cond2} $\forall t'<t$, $\vx(t';\vxz) \in \Qone$.
    \end{enumerate}
\end{fact}


\begin{proof}
We begin by showing the hyper-parameter selection needed to guarantee condition~\ref{fact:d-axis-crossing-cond1} holds.
Fact~\ref{fact:finite-time} shows that $\forall \vxz\in\Qoneic$, $\exists t<\infty$ such that $\theta(t) = 0$.
Let $t$ be this finite choice, and thus we know $\vx(t;\vxz) = [0, d(t)]^\intercal$.
All that remains to show is that our hyper-parameter selection will guarantee $d(t) < \dub$.

To begin, we denote the line $L_1 \triangleq \Qoneic\cap\left\{ d = -\frac{\dub}{\tub}\theta + \dub  \right\}$.
Note that all initial conditions along $L_1$ takes the form
\[ (\theta_0,d_0) = \left(\theta_0, -\frac{\dub}{\tub}\theta_0 + \dub\right). \]
In essence, what we need to show is that for the given $t$ corresponding to $\theta(t) = 0$,
\[ d(t) = d_0 + \int_0^t \dot d(\tau) d\tau < \dub. \]
First, let $-\uF = \max_{\vx\in L_1} F(\vx)$, which corresponds to the smallest negative value $F$ can take along or above this line.
%Next, we note that $\forall \alpha>0$, $F < \uF < 0$, the following bound holds
Next, fix $\alpha > 0$.
The selection of both $\alpha$ and $-\uF$ allow $\dot d$ to be bounded as
\[ \dot d(t) \le \beta e^{-\alpha(-\uF)^2}\sin(\theta(t)). \]
Since $\alpha$ and $\uF$ are fixed values, we can denote this as a constant $\beta' = \beta e^{-\alpha\uF^2}$.

For each trajectory $\vx(t;\vxz)$, there are two possible outcomes:
\begin{enumerate}
    \item $\vx(t;\vxz)$ remains below the line $L_1$, or
    \item $\vx(t;\vxz)$ lies on or above the line at some point in time.
\end{enumerate}
If the first condition is satisfied, we clearly must cross below the point $d(t) = \dub$, which concludes the proof.
Thus, we will assume the opposite holds.

Since $\vx(t;\vxz)$ is on or above the line, $F(\vx) \le \uF$.
The function $F$ can also be reparameterized as a function of the initial conditions and time as $F(\vx(t)) = F(t;\vxz)$.
Lastly, note that $\forall \vx\in\Qone$, $F(\vx(t)) < 0$.
These facts allow the bound to be reparameterized in the following way:
\begin{align*}
    d(t) &= d_0 + \int_0^t \beta e^{-\alpha F^2}\sin\theta(\tau)d\tau \\
    &\le d_0 + \int_0^t \beta'\sin\theta(\tau)d\tau \\
    &= d_0 + \int_0^t \beta'\sin\theta(\tau)\frac{\gamma F(\tau;\vxz)}{\gamma F(\tau;\vxz)}d\tau \\
    &\le d_0 + \int_0^t \frac{1}{\gamma(-\uF)} \beta'\sin\theta(\tau) \gamma F(\tau;\vxz) d\tau \\
    &= d_0 + \frac{\beta'}{\gamma(-\uF)} \int_0^t \sin\theta(\tau)\gamma F(\tau;\vxz)d\tau
\end{align*}
Note that $\forall t$, $\sin\theta(t) > 0$ and $F(t;\vxz) < 0$, implying that the integral term must be less than zero.
Thus, to avoid later sign confusion, the bound is simplified as
\[ d(t) \le d_0 + \frac{\beta'}{\gamma\uF} \int_0^t \sin\theta(\tau)\gamma F(\tau;\vxz)d\tau, \]
where $\uF > 0$.

We will now make the following substitutions:
\begin{itemize}
    \item $u = \theta$
    \item $du = \dot \theta d\tau = \gamma F(\tau;\vxz)d\tau$
    \item $\theta(0) = \theta_0$
    \item $\theta(T) = 0$.
\end{itemize}
The bound is now simplified to
\begin{align*}
    d(t) &\le d_0 + \frac{\beta'}{\gamma\uF} \int_{\theta_0}^0 \sin u\,du \\
    &= d_0 + \frac{\beta'}{\gamma\uF} (1 - \cos\theta_0).
\end{align*}

Now the final part is to show $\exists \beta,\gamma$ s.t. $\forall d_0\in\Qoneic$, $d(t;d_0) < \dub$.
\begin{align*}
    d(t) &\le d_0 + \frac{\beta'}{\gamma\uF} (1-\cos\theta_0)\\
    &= -\frac{\dub}{\tub}\theta_0 + \dub + \frac{\beta'}{\gamma\uF} (1-\cos\theta_0).
\end{align*}
Note that $\forall \theta_0\in\Qone$, $1-\cos\theta_0 \le \theta_0$.
This turns the final bound into
\begin{align}
    \label{eq:d-final-bound}
    d(t) &\le -\frac{\dub}{\tub}\theta_0 + \dub + \frac{\beta'}{\gamma\uF}\theta_0 \\ \nonumber
    &= \left( \frac{\beta'}{\gamma\uF} - \frac{\dub}{\tub} \right)\theta_0 + \dub.
\end{align}
Equation~\eqref{eq:d-final-bound} must remain less than $\dub\ \forall\theta_0\in\Qone$, implying
\[ \frac{\beta'}{\gamma} < \frac{\dub\uF}{\tub}. \]
Since $\tub, \dub$, and $\uF$ are all finite and positive, this concludes the proof of the first condition.

The second condition can be proven in the following way.
%To begin, note that $(\XL\cap\Qop)\subset\Qok$.
To begin, note that $\Qoneic\subset\Qone$.
We select $\frac{\beta'}{\gamma} < \frac{\dub\uF}{\tub}$, which implies that we must cross the $\theta$ axis below the point $[0,\dub]^\intercal$.
Since $\dot\theta < 0$, we cannot leave $\Qone$ along the $\theta$ direction unless we cross the $\theta$ axis.
Likewise, since $d(t) < \dub$ and $\dot d > 0$, this implies that $\forall t' < t, d(t') < d(t) < \dub$.
Thus, given our choice of $\alpha$, $\beta$, and $\gamma$, $\vx(t',\vxz)\in\Qone$, $0\le t' < t$.
\end{proof}

% \begin{remark}
% The bound
% \[ \frac{\beta'}{\gamma} < \frac{\oF}{k} \]
% has physically insightful meaning between both $\oF$ and $k$.
% For a fixed $\gamma$, as $\oF\rightarrow 0$, $\beta\rightarrow 0$.
% This implies that as our smallest $F$ value in a bounded region decreases, we need to decrease the forward velocity gain $\beta$ to compensate for the slower turning speed $\dot\theta = \gamma F$.
% Similarly, for a fixed $\gamma$, as $k\rightarrow\infty$, $\beta\rightarrow\infty$.
% This implies that as our bound on the desired $d(t)$ value grows, we can increase our forward velocity gain $\beta$ while still maintaining our ultimate bound on $d(t)$ when crossing the $\theta$ axis.\todo{BC: Add remarks about selection of $\alpha$ in here too.}
% \end{remark}


\begin{fact}
\label{fact:finite-time}
Let $0 < \alpha, \beta, \gamma < \infty$.
Then, $\forall \vxz \in \Qoneic$, $\exists t<\infty$ s.t. $\vx (t;\vxz) = [ 0, d(t) ]^\intercal$.
\end{fact}

\begin{proof}
Let $0 < \alpha, \beta, \gamma < \infty$.
For $[\theta_0, d_0]^\intercal = \vxz\in\Qoneic \setminus \{ (\theta, 0) : \theta\in(0,\tub) \}$, conditions~\eqref{eq:delfd} and~\eqref{eq:delft} in conjunction with control laws~\eqref{eq:v-control-law} and~\eqref{eq:w-control-law} imply that, $\forall t$ s.t. $\vx(t;\vxz)\in\Qoneic$:
\[ 0 < F(0,d_0) < F(\theta(t), d(t)). \]
Let $\underline{F} = F(0,d_0)$.
Thus, for any of these initial conditions, it satisfies
\begin{align*}
    \theta(t) &= \theta_0 + \int_0^t \dot \theta d\tau \\
    &\le \theta_0 + \int_0^t \gamma\underline{F} d\tau \\
    &= \theta_0 + \gamma\underline{F} t.
\end{align*}
Let $t = -\frac{\theta_0}{\gamma\underline{F}}<\infty$, and note that $\theta(t) \le 0$ and that $t$ is finite.

For the elements along the line $\vxz\in \{ (\theta,0) : \theta\in(0,\tub) \}$, $\dot d > 0$.
Thus, all points along this line must enter into the set $\Qoneic \setminus \{ (\theta,0) \}$, concluding the proof.


% Note that $\forall \vxz$, $\dot d > 0$.
% Thus, we must enter $\Qonenot$.
% In $\Qop$, note that $\dot d > 0$ and $\dot\theta < 0$.
% Let $[\theta_1, d_1]^\intercal = \vx_1\in\Qop$, and we let $-\oF = F(0,d_1)$, where $\oF > 0$.
% %Conditions~\eqref{eq:delfx} and~\eqref{eq:delft} imply that $\forall \vx_1\in\Qok, \dot \theta \le -\gamma\oF$.
% Conditions~\eqref{eq:delfx} and~\eqref{eq:delft} imply that for our given choice of $\vx_1$ and $\oF$, if $\vx(t;\vx_1)\in\Qone$, $\dot \theta \le -\gamma\oF$.
% This implies the following:
% \begin{align*}
%     \theta(t) &= \theta_1 + \int_0^t \dot \theta d\tau \\
%     &\le \theta_1 + \int_0^t -\gamma\oF d\tau \\
%     &= \theta_1 - \gamma\oF t.
% \end{align*}
% Let $t = \frac{\theta_1}{\gamma\oF}<\infty$, and note that $\theta(t) \le 0$.
% Since $t$ is finite, this concludes the proof.
\end{proof}






% \section{OLD LS Analysis}
% Preliminaries:
% \begin{align}
%     \label{eq:ddot} \dot d &= \beta e^{-\alpha F^2}\sin\theta \\
%     \label{eq:tdot} \dot \theta &= \gamma F \\
%     \label{eq:tbound} \tbound &\in (0,\pi) \\
%     \label{eq:dbound} \dbound &\in(0,h^{-1}(\pi)) \\
%     \label{eq:xlplus} \XLp &\triangleq \{ (\theta, d): d \le -\frac{\dbound}{\tbound}\theta + \dbound \} \\
%     \label{eq:xlminus} \Xlm &\triangleq \{ (\theta, d): d \le -\frac{\dbound}{\tbound}\theta - \dbound \} \\
%     \label{eq:qone} \Qone &\triangleq \{ (\theta, d) : \theta\in (0,\tbound), d\in(0,\dbound) \} \\
%     \label{eq:qtwo} \Qtwo &\triangleq \{ (\theta, d) : \theta\in (-\tbound,0), d\in(0,\dbound) \} \\
%     \label{eq:qthr} \Qthr &\triangleq \{ (\theta, d) : \theta\in (-\tbound,0), d\in(-\dbound,0) \} \\
%     \label{eq:qfor} \Qfor &\triangleq \{ (\theta, d) : \theta\in (0,\tbound), d\in(-\dbound,0) \} \\
%     \label{eq:levelsetcont} \Qonenot &\triangleq \Qone\cap\XLp \\
%     \label{eq:levelsetcontneg} \Qthrnot &\triangleq \Qthr\cap\XLm  \\
%     \label{eq:Forigin} F(\vec 0) &= 0 \\
%     \label{eq:delfd} \frac{\partial F}{\partial d} &< 0\qquad\forall -\dbound \le d \le \dbound \\
%     \label{eq:delft} \frac{\partial F}{\partial \theta} &< 0\qquad\forall -\tbound\le\theta\le\tbound\\
%     \label{eq:traj} \vec x(t;\vec x_0) &= \vec x_0 + \int_0^t \dot{\vec x}(\tau)d\tau \\
% \end{align}
% \clonelabel{eq:ls-prelims-start}{eq:ddot}
% \clonelabel{eq:ls-prelims-end}{eq:traj}
% \clonelabel{eq:F-conds-begin}{eq:Forigin}
% \clonelabel{eq:F-conds-end}{eq:delft}


% \begin{fact}
%     Let $\tbound\in(0,\pi)$ and $\dbound\in(0,\invf{h}(\tbound))$.
%     Let $\vxz\in\left( \Qonenot\cup\Qtwo\cup\Qthrnot\cup\Qfor \right)$.
%     Then, $\exists 0 < \alpha, \beta, \gamma < \infty$ s.t. the control laws~\eqref{eq:ddot} and~\eqref{eq:tdot} ensure that, $\forall t>0$, $\vx(t)\in\left( 
%  \Qone\cup\Qtwo\cup\Qthr\cup\Qfor \right)$.
% \end{fact}


% \begin{proof}
% \emph{Quadrant 1}:
%     Let $\vxz \in X_0$, let $-\oF = \max_{(\theta,x)\in\XL\cap\Qop} F(\theta,x)$ and let $\alpha,\beta,\gamma > 0$, $\frac{\beta}{\gamma} > \frac{\oF}{\theta_0^\star k}$.
%     Since $\dot d > 0$, we must enter $\XLo$.
%     Let $\vx_1\in\XLo$.
%     With our selection $\alpha, \beta$, and $\gamma$, facts~\ref{fact:d-axis-crossing} and~\ref{fact:finite-time} show that $\vx_1$ must cross the $d$ axis below the point $[0,\frac{\pi}{k}]^\intercal$ in finite time while staying within the bounded set $\Qok$.

%     \emph{Quadrant 2}:
%     Starting on the $d$ axis, note that $\forall \vxz\in\{(0,d) : d > 0 \}$, $\dot \theta < 0$.
%     Thus, we must enter $\Qtk$.
%     Once in $\Qtk$ the proof is rather simple.
%     $\forall\vxz\in\Qtk$, $\dot d < 0$, and on either side of $g(\theta,x) = 0$, $\dot\theta$ pushes us towards this line.
%     This ensures our trajectory $\vx(t;\vxz)$ remains bounded in $\Qtk$ or exits across the $\theta$ axis into quadrant 3 along the set $-X_0 \triangleq \{(\theta,0) : -\pi < \theta < 0 \}$.
%     If we cross the $\theta$ axis within $-X_0$, since $F$ is symmetric, everything above applies similarly, although the signs are reversed.
%     In essence, if we cross the $\theta$ axis along $-X_0$, we will be guaranteed to at least re-enter $X_0$ if it does not go to the equilibrium.
%     If we do not ever cross the axis, then we either remain in $\Qtk$ for all time, or we are driven to the equilibrium.

%     We summarize the proof in the following way:
%     If we begin in the set $\XLo \cup \Qtk \cup \{ (0,d) : 0 < d < \frac{\pi}{k} \}$ (or it's counterpart flipped across the $\theta$ axis), we will remain bounded in the set $\{ (\theta, d) : \theta\in(-\pi,\pi), d\in(-\frac{\pi}{k},\frac{\pi}{k}) \}$.
% \end{proof}


% \begin{fact}
%     \label{fact:d-axis-crossing}
%     Let $\vxz\in\XL^+\cap\Qone$.
%     Then, $\exists 0<\alpha,\beta,\gamma<\infty$ such that the following two conditions hold:
%     \begin{enumerate}
%         \item \label{fact:d-axis-crossing-cond1} $\exists t<\infty$ s.t. $\vx(t;\vxz) = [0,d(t)]^\intercal$, where $d(t)<\frac{\pi}{k}$, and
%         \item \label{fact:d-axis-crossing-cond2} $\forall t'<t$, $\vx(t';\vxz) \in \Qok$.
%     \end{enumerate}
% \end{fact}

% \begin{proof}
% We begin by showing the hyper-parameter selection needed to guarantee condition~\ref{fact:d-axis-crossing-cond1} holds.
% Fact~\ref{fact:finite-time} shows that $\forall \vxz\in X_0\cup\Qone$, $\exists t<\infty$ such that $\theta(t) = 0$.
% Let $t$ be this finite choice, and thus we know $\vx(t;\vxz) = [0, d(t)]^\intercal$.
% All that remains to show is that our hyper-parameter selection will guarantee $d(t) < \frac{\pi}{k}$.

% To begin, we note that all initial conditions along the line $\XL\cap\Qop$ takes the form
% \[ (\theta_0,d_0) = \left(\theta_0, \frac{\pi - \theta_0}{k}\right). \]
% In essence, what we need to show is that for the given $T$ corresponding to $\theta(T) = 0$,
% \[ d(T) = d_0 + \int_0^T \dot d(\tau) d\tau < \frac{\pi}{k}. \]
% First, let $-\oF = \max_{(\theta,d)\in\XL\cap\Qone} F(\theta,x)$, which corresponds to the smallest negative value $F$ can take along or above this line.
% Next, we note that $\forall \alpha>0$, $F < -\oF$, the following bound holds
% \[ \dot d(t) \le \beta e^{-\alpha\oF^2}\sin(\theta(t)). \]
% Since $\alpha$ and $\oF$ are fixed values, we can denote this as a constant $\beta' = \beta e^{-\alpha\oF^2}$.

% For each $\vx(t;\vxz)$, there are two possible conditions:
% \begin{enumerate}
%     \item $\vx(t;\vxz)$ remains below the line $\XL\cap\Qone$, and
%     \item $\vx(t;\vxz)$ goes above the line at some point in time.
% \end{enumerate}
% If the first condition is satisfied, we clearly must cross below the point $d(t) = \frac{\pi}{k}$, which concludes the proof.
% Thus, we will assume the opposite holds.

% Since $\vx(t;\vxz)$ is above the line, we know everywhere above this line, $-F(\vx) < -\oF$.
% We can use this fact to reparameterize our bound in the following way:
% \begin{align}
%     d(T) &= d_0 + \int_0^T \beta e^{-\alpha F^2}\sin\theta(\tau)d\tau \\
%     &\le d_0 + \int_0^T \beta'\sin\theta(\tau)d\tau \\
%     &= d_0 + \int_0^T \beta'\sin\theta(\tau)\frac{\gamma F(\tau)}{\gamma F(\tau)}d\tau \\
%     &\le d_0 + \int_0^T \frac{1}{\gamma\oF} \beta'\sin\theta(\tau) \gamma F(\tau) d\tau \\
%     &= d_0 + \frac{\beta'}{\gamma\oF} \int_0^T \sin\theta(\tau)\gamma F(\tau)d\tau
% \end{align}
% We will now make the following substitutions:
% \begin{itemize}
%     \item $u = \theta$
%     \item $du = \dot \theta d\tau = \gamma F(\tau)d\tau$
%     \item $\theta(0) = \theta_0$
%     \item $\theta(T) = 0$.
% \end{itemize}
% The bound is now simplified to
% \begin{align}
%     d(t) &\le d_0 + \frac{\beta'}{\gamma\oF} \int_{\theta_0}^0 \sin u\,du \\
%     &= d_0 + \frac{\beta'}{\gamma\oF} (1 - \cos\theta_0).
% \end{align}

% Now the final part is to show $\exists \beta,\gamma$ s.t. $\forall d_0\in\XL\cap\Qop$, $d(t;d_0) < \frac{\pi}{k}$.
% \begin{align}
%     d(t) &\le d_0 + \frac{\beta'}{\gamma\oF} (1-\cos\theta_0)\\
%     &= \frac{\pi - \theta_0}{k} + \frac{\beta'}{\gamma\oF} (1-\cos\theta_0).
% \end{align}
% Note that $\forall \theta_0\in X_0$, $1-\cos\theta_0 \le \theta_0$.
% This turns the final bound into
% \begin{align}
%     d(t) &\le \frac{\pi - \theta_0}{k} + \frac{\beta'}{\gamma\oF} \theta_0 < \frac{\pi}{k}\\
%     &\implies \frac{\beta'}{\gamma} < \frac{\oF}{k},
% \end{align}
% where $\oF$ and $k$ are finite.
% Thus,
% \[ \frac{\beta'}{\gamma} < \frac{\oF}{k} \implies \forall \vxz\in\XL\cap\Qop, d(t;d_0) < \frac{\pi}{k}. \]

% We have now proven the first condition in the fact.
% The second condition can be proven in the following way.
% To begin, note that $(\XL\cap\Qop)\subset\Qok$.
% We select $\frac{\beta'}{\gamma} < \frac{\oF}{k}$, which implies that we must cross the $\theta$ axis below the point $[0,\frac{\pi}{k}]^\intercal$.
% Since $\dot\theta < 0$, we cannot leave $\Qok$ along the $\theta$ direction unless we cross the $\theta$ axis.
% Likewise, since $d(t) < \frac{\pi}{k}$ and $\dot d > 0$, this implies that $\forall t' < t, d(t') < d(t) < \frac{\pi}{k}$.
% Thus, given our choice of $\alpha$, $\beta$, and $\gamma$, $\vx(t',\vxz)\in\Qok$, $0\le t' < t$.
% \end{proof}

% \begin{remark}
% The bound
% \[ \frac{\beta'}{\gamma} < \frac{\oF}{k} \]
% has physically insightful meaning between both $\oF$ and $k$.
% For a fixed $\gamma$, as $\oF\rightarrow 0$, $\beta\rightarrow 0$.
% This implies that as our smallest $F$ value in a bounded region decreases, we need to decrease the forward velocity gain $\beta$ to compensate for the slower turning speed $\dot\theta = \gamma F$.
% Similarly, for a fixed $\gamma$, as $k\rightarrow\infty$, $\beta\rightarrow\infty$.
% This implies that as our bound on the desired $d(t)$ value grows, we can increase our forward velocity gain $\beta$ while still maintaining our ultimate bound on $d(t)$ when crossing the $\theta$ axis.\todo{BC: Add remarks about selection of $\alpha$ in here too.}
% \end{remark}


% \begin{fact}
% \label{fact:finite-time}
% Let $0 < \alpha, \beta, \gamma < \infty$.
% Then, $\forall \vxz \in X_0\cup\Qonenot$, $\exists t<\infty$ s.t. $\vx (t;\vxz) = [ 0, d(t) ]^\intercal$.
% \end{fact}

% \begin{proof}
% Let $0 < \alpha, \beta, \gamma < \infty$, and $\vxz \in X_0$.
% Note that $\forall \vxz$, $\dot d > 0$.
% Thus, we must enter $\Qonenot$.
% In $\Qop$, note that $\dot d > 0$ and $\dot\theta < 0$.
% Let $[\theta_1, d_1]^\intercal = \vx_1\in\Qop$, and we let $-\oF = F(0,d_1)$, where $\oF > 0$.
% %Conditions~\eqref{eq:delfx} and~\eqref{eq:delft} imply that $\forall \vx_1\in\Qok, \dot \theta \le -\gamma\oF$.
% Conditions~\eqref{eq:delfx} and~\eqref{eq:delft} imply that for our given choice of $\vx_1$ and $\oF$, if $\vx(t;\vx_1)\in\Qone$, $\dot \theta \le -\gamma\oF$.
% This implies the following:
% \begin{align*}
%     \theta(t) &= \theta_1 + \int_0^t \dot \theta d\tau \\
%     &\le \theta_1 + \int_0^t -\gamma\oF d\tau \\
%     &= \theta_1 - \gamma\oF t.
% \end{align*}
% Let $t = \frac{\theta_1}{\gamma\oF}<\infty$, and note that $\theta(t) \le 0$.
% Since $t$ is finite, this concludes the proof.
% \end{proof}



\subsection{LAS Analysis}
Preliminaries (include preliminaries~\eqref{eq:ls-prelims-start}---~\eqref{eq:ls-prelims-end}):
\begin{align}
    \label{eq:F0} F_0 &\triangleq \{ (\theta,d) : F(\theta,d) = 0 \} \\
    %\label{eq:Fh} F_h(\theta) &\triangleq F(\theta, h(\theta))\text{ s.t. } (\theta,h(\theta))\in F_0\\
    \label{eq:FL} F_L(\theta) &\triangleq F\left(\theta, -L\theta \right)
\end{align}



\begin{fact}
    There exists $\alpha, \beta, \gamma > 0$ such that, $\forall \vxz\in X_0$, $\lim_{t\rightarrow\infty}\vx(t;\vxz) = 0$.
\end{fact}

\begin{proof}
    Let $L$ satisfy Fact~\ref{fact:L-bound}.
    Let $\alpha > 0$ and $\frac{\beta}{\gamma}$ satisfy facts~\ref{fact:d-axis-crossing} and~\ref{fact:L-vf-bound} according to the $L$ chosen previously.

    We begin this proof by incorporating previous results from the proof of Fact~\ref{fact:ls-stability}.
    This fact showed that, $\forall \vxz\in\Qone$, we will enter $\Qtwo$ below the point $\dub$ and pass through the line $F_0$.
\end{proof}



\begin{fact}
\label{fact:L-vf-bound}
%Let $-\theta\in(-\tbound, 0)$, and define $L$ based on Fact~\ref{fact:L-bound}.
Define $L$ based on Fact~\ref{fact:L-bound}.
Denote the line $L$ makes in quadrant two as $L_2\triangleq\Qtwo\cap\{ (\theta,d) : d = -L\theta \}$
Then, $\exists \alpha,\beta,\gamma > 0$ such that, $\forall (\theta,d)\in L_2$, the following inequality holds:
\[ \tan{\left(-L\right)} \le \tan{\left( \frac{\dot d}{\dot\theta} \right)}. \]
\end{fact}

\begin{proof}
We begin the proof as
    \begin{align*}
        \tan{\left( -L \right)} &\le \tan{\left( \frac{\dot d}{\dot\theta} \right)} \\
        \implies -L &\le \frac{\dot d}{\dot\theta} \\
        \implies -\dot\theta L &\le \dot d \\
        \implies -\gamma F_L(\theta) L &\le \beta e^{-\alpha F_L^2(\theta)}\sin(-\theta)\\
        \implies \gamma F_L(\theta) L &\ge \beta\sin(\theta) \ge \beta e^{-\alpha F_L^2(\theta)}\sin(\theta)\\
        \implies \frac{\gamma}{\beta} &\ge \frac{1}{L} \frac{\sin(\theta)}{F_L(\theta)} \\
        \implies \frac{\beta}{\gamma} &\le \frac{L F_L(\theta)}{\sin(\theta)}.
    \end{align*}
    The last inequality holds since our choice of $L$ guarantees that $F_L(\theta)$ remains positive along that line.
    The last thing to show is that the fraction $\frac{F_L(\theta)}{\sin\theta}$ remains bounded as $\theta\rightarrow 0$. \todo{Find the minimum of $\frac{F_L(\theta)}{\sin\theta}$ over $\theta$ in an interval and show that it is strictly positive}
    For this we will use L'Hopital's rule.
    \begin{align*}
    \lim_{\theta\rightarrow 0} \frac{F_L(\theta)}{\sin\theta} &= \frac{0}{0}\\
    \implies \lim_{\theta\rightarrow 0} \frac{F_L(\theta)}{\sin\theta} &= \lim_{\theta\rightarrow 0} \frac{F_L'(\theta)}{\cos\theta}\\
    \implies \lim_{\theta\rightarrow 0} \frac{F_L(\theta)}{\sin\theta} &= F_L'(0).
    \end{align*}
    Note that, by the choice of $L$ and conditions ... and ..., $0 < F'_L(\theta) < \infty$. \todo{have you determined these conditions?} 

    Finally, select 
    \[ \frac{\gamma}{\beta} = L\cdot\inf_{\theta\in L2}\left( \frac{F_L(\theta)}{\sin\theta} \right). \]
    Since the hyper-parameter choice is positive and finite, it concludes the proof.
\end{proof}


\begin{fact}
\label{fact:L-bound}
    %Let $\theta\in(\tlb, 0)$.
    %Then, $\exists L>0$ such that $\forall d\in F_0\cap\Qtwo$, $d \ge -L\theta$.
    $\exists L>0$ such that the following two conditions are met:
    \begin{align}
        &\forall d\in F_0\cap\Qtwo, d \ge -L\theta, \text{ and},\\
        &\forall d\in F_0\cap\Qfor, d \le -L\theta.
    \end{align}
\end{fact}

\begin{proof}
    We begin the proof by showing that the first condition holds.
    If so, the second condition can be shown to hold by an argument of symmetry.
    Note that this symmetry does not imply that the maximum $L$ is the same for both conditions, but rather, if $L'$ satisfies the first and $L''$ satisfies the second, then $L=\min\{ L', L''\}$ satisfies both.
    Thus we aim to prove the first condition true, and the second follows by symmetry.
    
    Conditions $\frac{\partial{F}}{\partial{\theta}} < 0$ and $\frac{\partial{F}}{\partial{\theta}} < 0$ in conjunction with the Implicit Function Theorem imply that the graph of $F_0$ can be represented by a function $h(\theta) = d$, where $h$ has the following properties:
    \begin{itemize}
        \item $h$ exists along the entire domain of $F_0$,
        \item $h$ is strictly monotonically decreasing,
        \item $h(0) = 0$, and
        \item $\forall\theta,\, h'(\theta) < 0$.
    \end{itemize}
    With these properties, a trivial bound of $F_0$ is $L=0$, as the monotonicity of $h$ implies that $h(\theta) \ge 0\ \forall\theta\in(\tlb,0)$.
    However, we would like to show that there is a non-zero lower bound on this function.
    Let $L > 0$.
    Then
    \begin{align*}
        h(\theta) &\ge -L\theta \\
        \implies \frac{h(\theta)}{\theta} &\le -L \ (\text{due to the fact that } \theta < 0) \\
        \implies \frac{h(\theta)}{-\theta} &\ge L.
    \end{align*}
    Note that since $\theta<0$, $\frac{h(\theta)}{-\theta} > 0$.

    To ensure that $L$ bounds all values, two conditions are required.
    These conditions are
    \begin{align}
        \lim_{\theta\rightarrow0}\frac{h(\theta)}{-\theta} &\ne 0,\text{ and}\\
        \min_{\theta\in\Qtwo} \frac{h(\theta)}{-\theta} &> 0
    \end{align}
    If both of these conditions are true, then letting $L$ be the minimum of these two conditions will ensure the bound is valid, implying $L > 0$ exists.
    
    To show the first condition holds, we begin by evaluating the limit.
    This limit evaluates to
    \begin{equation*}
        \lim_{\theta\rightarrow0}\frac{h(\theta)}{-\theta} = \frac{0}{0},
    \end{equation*}
    implying that L'Hopital's rule may be used to evaluate this limit.
    This provides
    \begin{equation*}
        \lim_{\theta\rightarrow0}\frac{h(\theta)}{-\theta} = \lim_{\theta\rightarrow0}\frac{h'(\theta)}{1}.
    \end{equation*}
    Since $h'(\theta) < 0$,
    \begin{equation*}
        \lim_{\theta\rightarrow0}\frac{h(\theta)}{-\theta} \ne 0.
    \end{equation*}

    The second condition comes trivially with the first condition in combination with the strictly monotonically decreasing nature of $h$.
\end{proof}

\section{Lyapunov stability}
We would like to show that there exists a choice of parameters $\alpha,\beta$, and $\gamma$ such that the origin is Lyapunov stable.
We first begin by proving the following supporting lemma.
\begin{lemma}
\label{lem:boundedd}
    Let $\delta > 0$.
    Let $\vxz\in\icsetone\triangleq\{ \vx\in X\cap\Qone \colon d = \delta, \theta \le \delta \}$.
    Then, $\exists \alpha,\beta,\gamma$, $\epsilon(\alpha,\beta,\gamma)>0$ and $0<T<\infty$ s.t. $\epsilon$ may be made arbitrarily small and the following conditions are satisfied:
    \begin{align*}
        \theta(T) &= 0,\text{ and }\\
        d(T) &< \delta + \epsilon.
    \end{align*}
\end{lemma}

\begin{proof}
By lemma~\ref{lem:coneinfinitetime}, $\exists T>0$ s.t. $\forall\vxz\in\icsetone$, $\theta(T)=0$.
We also note that equations~\eqref{eq:omega-ctrl-law}, \eqref{eq:v-ctrl-law}, \eqref{eq:tdot}, \eqref{eq:ddot} and StBSF conditions~\eqref{eq:stbsfcondzero}-\eqref{eq:stbsfcondpdd} imply that $\forall t < T$, $0 > F(0,\delta) > F(\vx(t;\vxz))$.
We will denote this as $\stbsfov = F(0,\delta)$.

Analysis of the dynamics shows
    \begin{align}
        \nonumber d(T) = d_0 + \int_0^T\dot d dt &< d_0 + \int_0^T\beta dt\\
         \label{eq:dTbound} &= \delta + \beta T.
    \end{align}
    \begin{align}
        \nonumber \theta(T) &= \theta_0 + \int_0^T\dot\theta dt \le \delta + \int_0^T\dot\theta dt\\
        \nonumber &\le \delta + \int_0^T\gamma\stbsfov dt = \delta + \gamma\stbsfov T \\
        &\implies T \le \frac{-\delta\stbsfov}{\gamma}. \label{eq:Tbound}
    \end{align}

Equations~\eqref{eq:dTbound} and~\eqref{eq:Tbound} can be combined to show
\[ d(T) < \delta + \beta T \le \delta + \beta \frac{-\delta\stbsfov}{\gamma}. \]
Letting $\beta = \epsilon\frac{\gamma}{-\delta\stbsfov} \implies d(T) < \delta + \epsilon$.
Likewise, letting $\epsilon(\alpha,\beta,\gamma) = \frac{\beta}{\gamma}(-\delta\stbsfov)$ completes the proof, as the term $-\delta\stbsfov$ is a fixed, positive term, and the parameters $\beta$ and $\gamma$ may be selected to make $\epsilon$ arbitrarily small.
\end{proof}

%\newcommand{\ballinf}[2]{B_{#1}^\infty\left({#2}\right)}
\newcommand{\ballinf}[2]{B_{#1}^\infty({#2})}
\newcommand{\bdelta}{\ballinf{\delta}{\vec 0}}
\newcommand{\beps}{\ballinf{\epsilon}{\vec 0}}
\newcommand{\inftynorm}[1]{||{#1}||_\infty}
We now show that, given certain parameter selection to the control law, the origin of the system is Lyapunov stable.
We first begin by defining an infinity norm ``ball'' (which will be a square in the state space).
\begin{definition}
An $\alpha$-infinity norm ball, centered at a point $\vx$, is the set
\[ \ballinf{\alpha}{\vx} \triangleq \{ \vx'\in X \colon \inftynorm{\vx-\vx'}<\alpha \}. \]
\end{definition}

We now present the lemma for Lyapunov stability.
\begin{lemma}[Lyapunov Stability]
\label{lem:lyap-stab}
$\forall \epsilon > 0$, $\exists \delta(\epsilon)>0$ s.t. $\forall\vxz\in\bdelta$ and $\forall t>0$, $\vx(t;\vxz)\in\beps$.
\end{lemma}

\begin{proof}
    We will show that in each quadrant $\quadrant_i$, $\forall\vxz\in\bdelta\cap \quadrant_i$, $\vx(t;\vxz)\in\beps\cap \quadrant_i$.
    Since $\cup_i \left(\quadrant_i\cap\beps\right) = \beps$, this will imply $\vx(t;\vxz)\in\beps\ \forall t>0$.
    We also argue along the lines of symmetry, that any trajectory occurring in $\Qone$ will have a similar (but sign flipped) trajectory occurring in $\Qthr$, and the same for $\Qtwo$ and $\Qfor$.
    Thus, we will limit our analysis to quadrants $\Qone$ and $\Qtwo$, and the rest follows by symmetry.\newline
    \underline{$\vxz\in\Qone\cap\bdelta$}:  Lemma~\ref{lem:coneinfinitetime} showed that $\forall \vxz\in\Qone$, $\exists T<\infty$ s.t. $\vx(T;\vxz) = [0,d(T)]^\intercal$ for some $d(T)>0$.
    We just need to show that for our choice of $\alpha, \beta, \gamma, \text{ and } \epsilon$, $d(T)<\epsilon$.
    Note that $\forall\vxz\in\Qone\cap\bdelta$, one of two outcomes can occur:
    \begin{enumerate}
        \item $\theta(T) = 0 \implies d(T) < \delta$, $\implies d(T) < \epsilon$, and
        \item $\exists T'<T$ s.t. $\theta(T') > 0$ and $d(T') = \delta$.
    \end{enumerate}
    For the first case, it is already the case that $\vx(t;\vxz)\in\Qone\cap\beps\ \forall t<T$.
    Thus we limit ourselves to the second case.

    Analysis similar to Lemma~\ref{lem:boundedd} shows that we can bound our state value for $d$ as
    \[ d(T) < d(T') + \int_{T'}^T \beta dt = d(T') + \beta (T-T') = \delta + \beta(T-T'). \]
    The same lemma also showed that we may select an arbitrarily small $\epsilon'(\alpha,\beta,\gamma) > 0$ s.t.
    \[ d(T) < \delta + \epsilon'. \]
    Selecting parameters $\beta$ and $\gamma$ s.t. $\epsilon' < \epsilon - \delta$ implies $d(T) < \epsilon$, further implying that $\forall t<T, \vx(t;\vxz)\in\beps\cap\Qone$.\newline
    \underline{$\vxz\in\Qtwo\cap\bdelta$}: First, note that $\forall\vxz\in\Qtwo\cap\bdelta\setminus\{0\}, \dot d < 0$.
    Likewise, note that $\forall d>h(\theta)$, $\dot\theta < 0$, and $\forall d<h(\theta)$, $\dot\theta > 0$.
    These facts imply that any point entering into $\Qtwo$ must do so along the positive $d$-axis, and any point leaving $\Qtwo$ (other than approaching the origin) must do so along the negative $\theta$-axis.
    These facts also imply that for any point crossing into $\Qtwo$ along the point $[0,d_0]^\intercal$, the furthest $\vx(t;\vxz)$ could cross into $\Qthr$ along the $\theta$-axis would occur at $[\invf{h}(d_0),0]^\intercal$.
    The coordinates $[0,d_0]^\intercal$ and $[\invf{h}(d_0),0]^\intercal$ form a box in $\Qtwo$ containing any valid solution $\vx(t;\vxz)$.
    Thus, selecting $\delta=\max\{ d_0, \invf{h}(d_0) \}$ ensures that $\forall t>0$ s.t. $\forall\vx(t;\vxz)\in\Qtwo$, $\vx(t;\vxz)\in\bdelta\cap\Qtwo\subset\beps\cap\Qtwo$.

    As a final note, the bound $\delta$ for $\Qtwo$ relies on the crossing point into $\Qtwo$ from $\Qone$.
    However, to ensure $\delta$ is uniform for all quadrants, we need to ensure that the crossing point $d_0$ can be made arbitarily small.
    Note that by the analysis for quadrant $\Qone$, we can choose any $\delta$ bound and parameters $\beta$ and $\gamma$ such that we remain bounded by any $\epsilon > 0$.
    Thus, we only need to ensure that for $\delta_1$ chosen in $\Qone$ and the $\delta_2$ chosen in $\Qtwo$, $\delta$ satisfies $\delta = \min\{ \delta_1, \delta_2 \}$.
\end{proof}

\begin{remark}
    Lyapunov stability analysis typically involves the use of balls defined on the $L$-2 norm.
    However, due the equivalency of norms, our definition provided above still satisfies the typical definition of Lyapunov stability.
\end{remark}


\section{Line Following Robots}
One application of the proposed continuous control law is the explanation of stability found in line following robots.
A line following robot is a mobile robot that is equipped with two sensors equally spaced on both sides.
These sensors emit a signal depending on whether some characteristic of a line is detected, such as an infrared sensor detecting a black strip along the ground.
Depending upon which sensor picks up the line, a different signal is emitted, such as a $1$ and $-1$ for either sensor, and a $0$ if neither sensor is picking anything up.

Let $\vec v = [v_r, v_l]^\intercal$ be the commanded right and left wheel velocities, respectively.
The typical control scheme for a line following robot is
\begin{equation*}
    \vec v(F)=
    \begin{cases}
        [1,-1]^\intercal & \text{if } F = -1\\
        [1, 1]^\intercal & \text{if } F = 0\\
        [-1,1]^\intercal & \text{if } F = 1
    \end{cases}
\end{equation*}
which corresponds to the following two cases in our state space
\begin{equation*}
    \dot d(F)=
    \begin{cases}
    0 & \text{if } F = 1,-1\\
    2\sin\theta & \text{if } F = 0
    \end{cases}
\end{equation*}
and
\begin{equation*}
    \dot\theta(F)=
    \begin{cases}
        F & \text{if } F = 1,-1\\
        0 & \text{if } F = 0.
    \end{cases}
\end{equation*}