\newcommand{\state}{X}
\newcommand{\thetainterval}{I_\theta}
\newcommand{\levelset}{c}
\section{Analysis}
We now demonstrate that when the state-based score function $\stbsf$ satisfies conditions~\eqref{eq:stbsfcondzero}-\eqref{eq:stbsfcondpdd}, we can find values for parameters $\alpha$, $\beta$ and $\gamma$ such that all trajectories of the closed-loop system 
\begin{align}
     \dot\theta &= \gamma \stbsf,\label{eq:thetacloop}\\
     \dot d &= \vctrl \sin\theta,\label{eq:dcloop}
\end{align}
reach a bounded neighborhood of the origin. 


Consider the set $F_s^0$ given by
$$\stbsf_0 = \{ \vx \in \state \colon \stbsf(\vx) = 0\}.$$
We can express the implicit curve $\stbsf_s^0$ using an explicit function $h_s \colon \R \to \R$ so that $d = h_s(\theta)$ for $\vx \in \stbsf_s^0$. %
\begin{lemma}
\label{lem:explicitrepofF}
Let the StBSF $\stbsf(\vx)$ satisfy conditions~\eqref{eq:stbsfcondzero}-\eqref{eq:stbsfcondpdd}. %
Then, there exists a continuous function $h \colon \R \to \R$ and an interval such that $h(0)=0$ and $\forall \theta$ the following hold:
\begin{enumerate}
    \item $\vx \in F_0 \iff \stbsf(\theta,h(\theta))=0$
    \item $-\infty < h'(\theta) < 0$
    \item $\theta \neq 0 \implies \theta h(\theta) < 0$
\end{enumerate}
\end{lemma}
\begin{proof}
    Since $\pd{F}{d} < 0$ for all $\vx \in \state$, the implicit function theorem guarantees existence of a unique function $h(\theta)$ such that $\stbsf(\theta,h(\theta))=0$. %
    Since $F(0,0) = 0$, we must have $h(0)=0$. 
    The partial derivative of this function is exactly $$h'(\theta) = -\frac{\pd{F}{\theta}}{\pd{F}{d}}.$$ 
    The function $h'(\theta)$ is strictly negative and bounded due to~\eqref{eq:stbsfcondpdtheta} and~\eqref{eq:stbsfcondpdd}. 
    Since $h(0) = 0$ and $h'(\theta) < 0$, it is easy to see that $\theta >0 \implies h(\theta)<0$, and  $\theta <0 \implies h(\theta)>0$ . Therefore, $\theta \neq 0 \implies \theta h(\theta) < 0$. 
\end{proof}