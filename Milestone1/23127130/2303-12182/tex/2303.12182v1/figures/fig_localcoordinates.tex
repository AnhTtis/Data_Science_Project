%%%%%%%%%%% Differential drive
\newcommand{\ddtikz}[3]{
\begin{scope}[xshift=#1cm,yshift=#2cm,rotate=#3,scale = 1.3]
\draw[thick,rounded corners=0.04cm] (-0.3,-0.3) rectangle (0.3,0.3);
\draw[thick,rounded corners=0.04cm,pattern=vertical lines] (-0.5,0.35) rectangle (0.5,0.5);
\draw[thick,rounded corners=0.04cm,pattern=vertical lines] (-0.5,-0.5) rectangle (0.5,-0.35);
\draw[thick] (0.15,0) +(0.25,-0.15) -- +(0.15,-0.1) -- +(0.15,0.1) -- +(0.25,0.15) -- cycle;
\draw[fill = blue,thick] (0,0) circle (0.05) node[left]{P};
% \draw[fill = red,thick] (0.35,0) circle (0.05);
% \draw[fill = green!70!black,thick] (0,0.425) circle (0.05);
% \draw[fill = orange,thick] (0,-0.425) circle (0.05);

\end{scope}
}

\begin{figure}[tb]
\centering
\begin{tikzpicture}[scale=1.0,>=stealth']
\coordinate (orig) at (0,0);
\def\xaxissize{1.5}
\def\yaxissize{1.5}
\def\robotrotate{60}
\def\axisangle{-65}
\ddtikz{0}{0}{60}
% forward velocity:
\coordinate (fvelbase) at ($(orig)+(10+\robotrotate+\axisangle:0.85*\xaxissize)$);
\draw[->,color=blue,line width = 0.5mm] (fvelbase) -- ($(fvelbase)+(0+\robotrotate:1.0*\xaxissize)$) node[right,blue,text width = 2cm] {forward \\ velocity $v$};
\draw[->,dashed] (fvelbase) -- ($(fvelbase)+(0+\robotrotate:0.75*\xaxissize)$) arc (0+\robotrotate:\robotrotate+\axisangle+90:0.75*\xaxissize);
\node[->,dashed] at ($(fvelbase)+(0+\robotrotate+15:0.9*\xaxissize)$) {$\localangle$}; % angle
\path[dashed,draw] (fvelbase) -- +(\robotrotate+\axisangle+90:1.0*\xaxissize);
% angular velocity:
\draw[->,orange,line width = 0.5mm] ($(orig)+(0,0) + (60+\robotrotate:1.0*\xaxissize)$) arc (60+\robotrotate:190+\robotrotate:1.0*\xaxissize) node[right,orange,text width = 2cm] {angular \\ velocity $\omega$};
% frenet-serret frame origin
\coordinate (fsorig) at ($(orig)+(0,0)+ (\robotrotate+\axisangle:2.5*\xaxissize)$);
\draw[dashed] (orig) -- (fsorig);
%%%%%%%%%%%%%% Quadrotor
% \coordinate (quad) at (0,0);
% \def\quadrot{15}
% \def\quadbladesize{0.25*\xaxissize}
% \foreach \x in {0,90,180,270}
% {\draw[thick] (quad) -- ($(quad) + (\quadrot+\x:0.25)$);
% \draw[thick]  ($(quad) + (\quadrot+\x:0.25+\quadbladesize)$) circle (\quadbladesize);}
% \draw[fill = red] (quad) ++(\quadrot+45:0.2) -- +(\quadrot+45+30:0.3) -- +(\quadrot+45-30:0.3) -- cycle;
%%%%%%%%%%
\path[fill] (fsorig) -- ++(\robotrotate+\axisangle:1.5) coordinate (pathcentre) circle (0.05);
\draw (pathcentre) edge[->] +(\robotrotate+180+\axisangle-30:1.5) +(\robotrotate+180+\axisangle-45:0.75) node[above right, text width = 1.8cm]{path radius $1/\rho$};
\draw[dashed] (pathcentre) +(\robotrotate+180+\axisangle-55:1.5) arc (\robotrotate+180+\axisangle-55:\robotrotate+180+\axisangle+55:1.5);
\path[thick,draw]  (fsorig) to [out =  \robotrotate+\axisangle+90, in = 220] ++(65:1.0*\xaxissize) to [out =  40, in = -40] ++(90:0.2*\xaxissize) node[above] {path};
\path[thick,draw] (fsorig)  to [out = \robotrotate+\axisangle-90,in = 80] ++(-90:0.5*\xaxissize) to [out = -100,in = 30] ++(-145:0.2*\xaxissize);
\draw[very thick,->,red] (fsorig) -- ++(\robotrotate+\axisangle+180:0.95*\xaxissize);
\draw[very thick,->,red] (fsorig) -- ++(\robotrotate+\axisangle+90:0.95*\xaxissize );
\draw[very thick,|<->|,purple] ($(fsorig)+(\robotrotate+\axisangle-90:0.35)$) -- ($(orig)+(\robotrotate+\axisangle-90:0.35)$);
\node[purple] at ($0.5*(fsorig)+0.5*(orig)+(\robotrotate+\axisangle-90:0.6)$) {$\deviation$};
\end{tikzpicture}
\caption{\small A wheeled mobile robot with forward speed $v$, and angular velocity $\omega$. The curved black line represents a local segment of the path, with instantaneous path curvature $\rho$, that the WMR must follow. The local Frenet-Serret frame (red) attached to the path is also shown. The WMR's state consists of the offset $\deviation$ and angle $\localangle$ with respect to the path.}
\label{fig:ddwmr}
\end{figure} 