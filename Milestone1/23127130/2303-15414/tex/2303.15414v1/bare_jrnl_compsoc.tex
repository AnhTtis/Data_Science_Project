
%% bare_jrnl_compsoc.tex
%% V1.4b
%% 2015/08/26
%% by Michael Shell
%% See:
%% http://www.michaelshell.org/
%% for current contact information.
%%
%% This is a skeleton file demonstrating the use of IEEEtran.cls
%% (requires IEEEtran.cls version 1.8b or later) with an IEEE
%% Computer Society journal paper.
%%
%% Support sites:
%% http://www.michaelshell.org/tex/ieeetran/
%% http://www.ctan.org/pkg/ieeetran
%% and
%% http://www.ieee.org/

%%*************************************************************************
%% Legal Notice:
%% This code is offered as-is without any warranty either expressed or
%% implied; without even the implied warranty of MERCHANTABILITY or
%% FITNESS FOR A PARTICULAR PURPOSE! 
%% User assumes all risk.
%% In no event shall the IEEE or any contributor to this code be liable for
%% any damages or losses, including, but not limited to, incidental,
%% consequential, or any other damages, resulting from the use or misuse
%% of any information contained here.
%%
%% All comments are the opinions of their respective authors and are not
%% necessarily endorsed by the IEEE.
%%
%% This work is distributed under the LaTeX Project Public License (LPPL)
%% ( http://www.latex-project.org/ ) version 1.3, and may be freely used,
%% distributed and modified. A copy of the LPPL, version 1.3, is included
%% in the base LaTeX documentation of all distributions of LaTeX released
%% 2003/12/01 or later.
%% Retain all contribution notices and credits.
%% ** Modified files should be clearly indicated as such, including  **
%% ** renaming them and changing author support contact information. **
%%*************************************************************************


% *** Authors should verify (and, if needed, correct) their LaTeX system  ***
% *** with the testflow diagnostic prior to trusting their LaTeX platform ***
% *** with production work. The IEEE's font choices and paper sizes can   ***
% *** trigger bugs that do not appear when using other class files.       ***                          ***
% The testflow support page is at:
% http://www.michaelshell.org/tex/testflow/


\documentclass[10pt,journal,compsoc]{IEEEtran}
%
% If IEEEtran.cls has not been installed into the LaTeX system files,
% manually specify the path to it like:
% \documentclass[10pt,journal,compsoc]{../sty/IEEEtran}





% Some very useful LaTeX packages include:
% (uncomment the ones you want to load)


% *** MISC UTILITY PACKAGES ***
%
%\usepackage{ifpdf}
% Heiko Oberdiek's ifpdf.sty is very useful if you need conditional
% compilation based on whether the output is pdf or dvi.
% usage:
% \ifpdf
%   % pdf code
% \else
%   % dvi code
% \fi
% The latest version of ifpdf.sty can be obtained from:
% http://www.ctan.org/pkg/ifpdf
% Also, note that IEEEtran.cls V1.7 and later provides a builtin
% \ifCLASSINFOpdf conditional that works the same way.
% When switching from latex to pdflatex and vice-versa, the compiler may
% have to be run twice to clear warning/error messages.



% \usepackage[pagebackref=true,breaklinks=true,colorlinks,bookmarks=false]{hyperref}
\usepackage[breaklinks=true,bookmarks=false]{hyperref}
\usepackage{comment}
\usepackage{epsfig}
\usepackage{graphicx}
\usepackage{amsmath}
\usepackage{amssymb}
\usepackage{verbatim}
\usepackage{bm}
\usepackage{color}
\usepackage{subfig} 
\usepackage{caption}
\usepackage{booktabs}
\usepackage{subfig}
\usepackage{multirow}
\usepackage{cases}
% *** CITATION PACKAGES ***
%
\ifCLASSOPTIONcompsoc
  % IEEE Computer Society needs nocompress option
  % requires cite.sty v4.0 or later (November 2003)
  \usepackage[nocompress]{cite}
\else
  % normal IEEE
  \usepackage{cite}
\fi
% cite.sty was written by Donald Arseneau
% V1.6 and later of IEEEtran pre-defines the format of the cite.sty package
% \cite{} output to follow that of the IEEE. Loading the cite package will
% result in citation numbers being automatically sorted and properly
% "compressed/ranged". e.g., [1], [9], [2], [7], [5], [6] without using
% cite.sty will become [1], [2], [5]--[7], [9] using cite.sty. cite.sty's
% \cite will automatically add leading space, if needed. Use cite.sty's
% noadjust option (cite.sty V3.8 and later) if you want to turn this off
% such as if a citation ever needs to be enclosed in parenthesis.
% cite.sty is already installed on most LaTeX systems. Be sure and use
% version 5.0 (2009-03-20) and later if using hyperref.sty.
% The latest version can be obtained at:
% http://www.ctan.org/pkg/cite
% The documentation is contained in the cite.sty file itself.
%
% Note that some packages require special options to format as the Computer
% Society requires. In particular, Computer Society  papers do not use
% compressed citation ranges as is done in typical IEEE papers
% (e.g., [1]-[4]). Instead, they list every citation separately in order
% (e.g., [1], [2], [3], [4]). To get the latter we need to load the cite
% package with the nocompress option which is supported by cite.sty v4.0
% and later. Note also the use of a CLASSOPTION conditional provided by
% IEEEtran.cls V1.7 and later.





% *** GRAPHICS RELATED PACKAGES ***
%
\ifCLASSINFOpdf
  % \usepackage[pdftex]{graphicx}
  % declare the path(s) where your graphic files are
  % \graphicspath{{../pdf/}{../jpeg/}}
  % and their extensions so you won't have to specify these with
  % every instance of \includegraphics
  % \DeclareGraphicsExtensions{.pdf,.jpeg,.png}
\else
  % or other class option (dvipsone, dvipdf, if not using dvips). graphicx
  % will default to the driver specified in the system graphics.cfg if no
  % driver is specified.
  % \usepackage[dvips]{graphicx}
  % declare the path(s) where your graphic files are
  % \graphicspath{{../eps/}}
  % and their extensions so you won't have to specify these with
  % every instance of \includegraphics
  % \DeclareGraphicsExtensions{.eps}
\fi
% graphicx was written by David Carlisle and Sebastian Rahtz. It is
% required if you want graphics, photos, etc. graphicx.sty is already
% installed on most LaTeX systems. The latest version and documentation
% can be obtained at: 
% http://www.ctan.org/pkg/graphicx
% Another good source of documentation is "Using Imported Graphics in
% LaTeX2e" by Keith Reckdahl which can be found at:
% http://www.ctan.org/pkg/epslatex
%
% latex, and pdflatex in dvi mode, support graphics in encapsulated
% postscript (.eps) format. pdflatex in pdf mode supports graphics
% in .pdf, .jpeg, .png and .mps (metapost) formats. Users should ensure
% that all non-photo figures use a vector format (.eps, .pdf, .mps) and
% not a bitmapped formats (.jpeg, .png). The IEEE frowns on bitmapped formats
% which can result in "jaggedy"/blurry rendering of lines and letters as
% well as large increases in file sizes.
%
% You can find documentation about the pdfTeX application at:
% http://www.tug.org/applications/pdftex






% *** MATH PACKAGES ***
%
%\usepackage{amsmath}
% A popular package from the American Mathematical Society that provides
% many useful and powerful commands for dealing with mathematics.
%
% Note that the amsmath package sets \interdisplaylinepenalty to 10000
% thus preventing page breaks from occurring within multiline equations. Use:
%\interdisplaylinepenalty=2500
% after loading amsmath to restore such page breaks as IEEEtran.cls normally
% does. amsmath.sty is already installed on most LaTeX systems. The latest
% version and documentation can be obtained at:
% http://www.ctan.org/pkg/amsmath





% *** SPECIALIZED LIST PACKAGES ***
%
\usepackage{algorithmic}
% algorithmic.sty was written by Peter Williams and Rogerio Brito.
% This package provides an algorithmic environment fo describing algorithms.
% You can use the algorithmic environment in-text or within a figure
% environment to provide for a floating algorithm. Do NOT use the algorithm
% floating environment provided by algorithm.sty (by the same authors) or
% algorithm2e.sty (by Christophe Fiorio) as the IEEE does not use dedicated
% algorithm float types and packages that provide these will not provide
% correct IEEE style captions. The latest version and documentation of
% algorithmic.sty can be obtained at:
% http://www.ctan.org/pkg/algorithms
% Also of interest may be the (relatively newer and more customizable)
% algorithmicx.sty package by Szasz Janos:
% http://www.ctan.org/pkg/algorithmicx




% *** ALIGNMENT PACKAGES ***
%
%\usepackage{array}
% Frank Mittelbach's and David Carlisle's array.sty patches and improves
% the standard LaTeX2e array and tabular environments to provide better
% appearance and additional user controls. As the default LaTeX2e table
% generation code is lacking to the point of almost being broken with
% respect to the quality of the end results, all users are strongly
% advised to use an enhanced (at the very least that provided by array.sty)
% set of table tools. array.sty is already installed on most systems. The
% latest version and documentation can be obtained at:
% http://www.ctan.org/pkg/array


% IEEEtran contains the IEEEeqnarray family of commands that can be used to
% generate multiline equations as well as matrices, tables, etc., of high
% quality.




% *** SUBFIGURE PACKAGES ***
%\ifCLASSOPTIONcompsoc
%  \usepackage[caption=false,font=footnotesize,labelfont=sf,textfont=sf]{subfig}
%\else
%  \usepackage[caption=false,font=footnotesize]{subfig}
%\fi
% subfig.sty, written by Steven Douglas Cochran, is the modern replacement
% for subfigure.sty, the latter of which is no longer maintained and is
% incompatible with some LaTeX packages including fixltx2e. However,
% subfig.sty requires and automatically loads Axel Sommerfeldt's caption.sty
% which will override IEEEtran.cls' handling of captions and this will result
% in non-IEEE style figure/table captions. To prevent this problem, be sure
% and invoke subfig.sty's "caption=false" package option (available since
% subfig.sty version 1.3, 2005/06/28) as this is will preserve IEEEtran.cls
% handling of captions.
% Note that the Computer Society format requires a sans serif font rather
% than the serif font used in traditional IEEE formatting and thus the need
% to invoke different subfig.sty package options depending on whether
% compsoc mode has been enabled.
%
% The latest version and documentation of subfig.sty can be obtained at:
% http://www.ctan.org/pkg/subfig




% *** FLOAT PACKAGES ***
%
%\usepackage{fixltx2e}
% fixltx2e, the successor to the earlier fix2col.sty, was written by
% Frank Mittelbach and David Carlisle. This package corrects a few problems
% in the LaTeX2e kernel, the most notable of which is that in current
% LaTeX2e releases, the ordering of single and double column floats is not
% guaranteed to be preserved. Thus, an unpatched LaTeX2e can allow a
% single column figure to be placed prior to an earlier double column
% figure.
% Be aware that LaTeX2e kernels dated 2015 and later have fixltx2e.sty's
% corrections already built into the system in which case a warning will
% be issued if an attempt is made to load fixltx2e.sty as it is no longer
% needed.
% The latest version and documentation can be found at:
% http://www.ctan.org/pkg/fixltx2e


%\usepackage{stfloats}
% stfloats.sty was written by Sigitas Tolusis. This package gives LaTeX2e
% the ability to do double column floats at the bottom of the page as well
% as the top. (e.g., "\begin{figure*}[!b]" is not normally possible in
% LaTeX2e). It also provides a command:
%\fnbelowfloat
% to enable the placement of footnotes below bottom floats (the standard
% LaTeX2e kernel puts them above bottom floats). This is an invasive package
% which rewrites many portions of the LaTeX2e float routines. It may not work
% with other packages that modify the LaTeX2e float routines. The latest
% version and documentation can be obtained at:
% http://www.ctan.org/pkg/stfloats
% Do not use the stfloats baselinefloat ability as the IEEE does not allow
% \baselineskip to stretch. Authors submitting work to the IEEE should note
% that the IEEE rarely uses double column equations and that authors should try
% to avoid such use. Do not be tempted to use the cuted.sty or midfloat.sty
% packages (also by Sigitas Tolusis) as the IEEE does not format its papers in
% such ways.
% Do not attempt to use stfloats with fixltx2e as they are incompatible.
% Instead, use Morten Hogholm'a dblfloatfix which combines the features
% of both fixltx2e and stfloats:
%
% \usepackage{dblfloatfix}
% The latest version can be found at:
% http://www.ctan.org/pkg/dblfloatfix




% \ifCLASSOPTIONcaptionsoff
%  \usepackage[nomarkers]{endfloat}
% \let\MYoriglatexcaption\caption
% \renewcommand{\caption}[2][\relax]{\MYoriglatexcaption[#2]{#2}}
% \fi
% endfloat.sty was written by James Darrell McCauley, Jeff Goldberg and 
% Axel Sommerfeldt. This package may be useful when used in conjunction with 
% IEEEtran.cls'  captionsoff option. Some IEEE journals/societies require that
% submissions have lists of figures/tables at the end of the paper and that
% figures/tables without any captions are placed on a page by themselves at
% the end of the document. If needed, the draftcls IEEEtran class option or
% \CLASSINPUTbaselinestretch interface can be used to increase the line
% spacing as well. Be sure and use the nomarkers option of endfloat to
% prevent endfloat from "marking" where the figures would have been placed
% in the text. The two hack lines of code above are a slight modification of
% that suggested by in the endfloat docs (section 8.4.1) to ensure that
% the full captions always appear in the list of figures/tables - even if
% the user used the short optional argument of \caption[]{}.
% IEEE papers do not typically make use of \caption[]'s optional argument,
% so this should not be an issue. A similar trick can be used to disable
% captions of packages such as subfig.sty that lack options to turn off
% the subcaptions:
% For subfig.sty:
% \let\MYorigsubfloat\subfloat
% \renewcommand{\subfloat}[2][\relax]{\MYorigsubfloat[]{#2}}
% However, the above trick will not work if both optional arguments of
% the \subfloat command are used. Furthermore, there needs to be a
% description of each subfigure *somewhere* and endfloat does not add
% subfigure captions to its list of figures. Thus, the best approach is to
% avoid the use of subfigure captions (many IEEE journals avoid them anyway)
% and instead reference/explain all the subfigures within the main caption.
% The latest version of endfloat.sty and its documentation can obtained at:
% http://www.ctan.org/pkg/endfloat
%
% The IEEEtran \ifCLASSOPTIONcaptionsoff conditional can also be used
% later in the document, say, to conditionally put the References on a 
% page by themselves.
\usepackage{float}
\usepackage{subfig}

% \usepackage{subfigure} 

% *** PDF, URL AND HYPERLINK PACKAGES ***
%
\usepackage{url}
% url.sty was written by Donald Arseneau. It provides better support for
% handling and breaking URLs. url.sty is already installed on most LaTeX
% systems. The latest version and documentation can be obtained at:
% http://www.ctan.org/pkg/url
% Basically, \url{my_url_here}.

\usepackage[ruled,vlined,linesnumbered,commentsnumbered]{algorithm2e}
% \usepackage{amsmath}
% \usepackage{amssymb}
% \usepackage{verbatim}

% *** Do not adjust lengths that control margins, column widths, etc. ***
% *** Do not use packages that alter fonts (such as pslatex).         ***
% There should be no need to do such things with IEEEtran.cls V1.6 and later.
% (Unless specifically asked to do so by the journal or conference you plan
% to submit to, of course. )
\def\comp{\ensuremath\mathop{\scalebox{.6}{$\circ$}}}
\newcommand\gm{{GMTracker}}
\newcommand{\MCG}{\mathcal{G}}
\newcommand{\MCV}{\mathcal{V}}
\newcommand{\MCE}{\mathcal{E}}
\newcommand{\MPI}{\mathbf{\Pi}}
\newcommand{\MA}{\mathbf{A}}
\newcommand{\MK}{\mathbf{K}}
\newcommand{\MBR}{\mathbb{R}}
\newcommand{\MB}{\mathbf{B}}
\newcommand{\Mh}{\mathbf{h}}
\newcommand{\Mm}{\mathbf{m}}
\newcommand{\MCT}{\mathcal{T}}
\newcommand{\MCD}{\mathcal{D}}
\newcommand{\MCA}{\mathcal{A}}
\newcommand{\MCF}{\mathcal{F}}
\DeclareMathOperator*{\minimize}{minimize}
\DeclareMathOperator*{\diag}{diag}
\DeclareMathOperator*{\subjectto}{subject\;to}
\newcommand{\norm}[1]{\| #1 \|}
\newcommand{\dd}{\mathsf{d}}
% correct bad hyphenation here
\hyphenation{op-tical net-works semi-conduc-tor}
% \newtheorem{theorem}{Theorem}[section]
\newtheorem{proposition}{Proposition}[section]
% \newtheorem{proposition}{Corollary}[section]
\newtheorem{lemma}[proposition]{Lemma}
\newtheorem{proof}{Proof}[section]
\newtheorem{corollary}[proposition]{Corollary}
\newcommand{\0}{\phantom{0}}
\def\degree{${}^{\circ}$}
\renewcommand{\b}[1]{\textbf{#1}}
\begin{document}
%
% paper title
% Titles are generally capitalized except for words such as a, an, and, as,
% at, but, by, for, in, nor, of, on, or, the, to and up, which are usually
% not capitalized unless they are the first or last word of the title.
% Linebreaks \\ can be used within to get better formatting as desired.
% Do not put math or special symbols in the title.
\title{Learnable Graph Matching: A Practical Paradigm for Data Association}
%
%
% author names and IEEE memberships
% note positions of commas and nonbreaking spaces ( ~ ) LaTeX will not break
% a structure at a ~ so this keeps an author's name from being broken across
% two lines.
% use \thanks{} to gain access to the first footnote area
% a separate \thanks must be used for each paragraph as LaTeX2e's \thanks
% was not built to handle multiple paragraphs
%
%
%\IEEEcompsocitemizethanks is a special \thanks that produces the bulleted
% lists the Computer Society journals use for "first footnote" author
% affiliations. Use \IEEEcompsocthanksitem which works much like \item
% for each affiliation group. When not in compsoc mode,
% \IEEEcompsocitemizethanks becomes like \thanks and
% \IEEEcompsocthanksitem becomes a line break with idention. This
% facilitates dual compilation, although admittedly the differences in the
% desired content of \author between the different types of papers makes a
% one-size-fits-all approach a daunting prospect. For instance, compsoc 
% journal papers have the author affiliations above the "Manuscript
% received ..."  text while in non-compsoc journals this is reversed. Sigh.

% \author{Michael~Shell,~\IEEEmembership{Member,~IEEE,}
%         John~Doe,~\IEEEmembership{Fellow,~OSA,}
%         and~Jane~Doe,~\IEEEmembership{Life~Fellow,~IEEE}% <-this % stops a space
% \IEEEcompsocitemizethanks{\IEEEcompsocthanksitem M. Shell was with the Department
% of Electrical and Computer Engineering, Georgia Institute of Technology, Atlanta,
% GA, 30332.\protect\\
% % note need leading \protect in front of \\ to get a newline within \thanks as
% % \\ is fragile and will error, could use \hfil\break instead.
% E-mail: see http://www.michaelshell.org/contact.html
% \IEEEcompsocthanksitem J. Doe and J. Doe are with Anonymous University.}% <-this % stops an unwanted space
% \thanks{Manuscript received April 19, 2005; revised August 26, 2015.}}
\author{Jiawei~He,
        Zehao~Huang,
        Naiyan~Wang,
        and~Zhaoxiang~Zhang
        \IEEEcompsocitemizethanks{\IEEEcompsocthanksitem{
          Jiawei~He and Zhaoxiang~Zhang are with Center for Research on Intelligent Perception and Computing (CRIPAC), National Laboratory of Pattern Recognition (NLPR), Institute of Automation, Chinese Academy of Sciences (CASIA), Beijing 100190, China, and also with School of Artificial Intelligence, University of Chinese Academy of Sciences, Beijing 100049, China.  E-mail: \{hejiawei2019, zhaoxiang.zhang\}@ia.ac.cn.}
          \IEEEcompsocthanksitem{Zhaoxiang~Zhang is also with Centre for Artificial Intelligence and Robotics, Hong Kong Institute of Science and Innovation, Chinese Academy of Sciences (HKISI\_CAS), Hong Kong, China.}
          \IEEEcompsocthanksitem{
          Zehao~Huang and Naiyan~Wang are with Tusimple, Beijing 100020,
          China. E-mail: \{zehaohuang18, winsty\}@gmail.com.}}
          }
% note the % following the last \IEEEmembership and also \thanks - 
% these prevent an unwanted space from occurring between the last author name
% and the end of the author line. i.e., if you had this:
% 
% \author{....lastname \thanks{...} \thanks{...} }
%                     ^------------^------------^----Do not want these spaces!
%
% a space would be appended to the last name and could cause every name on that
% line to be shifted left slightly. This is one of those "LaTeX things". For
% instance, "\textbf{A} \textbf{B}" will typeset as "A B" not "AB". To get
% "AB" then you have to do: "\textbf{A}\textbf{B}"
% \thanks is no different in this regard, so shield the last } of each \thanks
% that ends a line with a % and do not let a space in before the next \thanks.
% Spaces after \IEEEmembership other than the last one are OK (and needed) as
% you are supposed to have spaces between the names. For what it is worth,
% this is a minor point as most people would not even notice if the said evil
% space somehow managed to creep in.



% The paper headers
\markboth{Journal of \LaTeX\ Class Files,~Vol.~14, No.~8, August~2015}%
{Shell \MakeLowercase{\textit{et al.}}: Bare Demo of IEEEtran.cls for Computer Society Journals}
% The only time the second header will appear is for the odd numbered pages
% after the title page when using the twoside option.
% 
% *** Note that you probably will NOT want to include the author's ***
% *** name in the headers of peer review papers.                   ***
% You can use \ifCLASSOPTIONpeerreview for conditional compilation here if
% you desire.



% The publisher's ID mark at the bottom of the page is less important with
% Computer Society journal papers as those publications place the marks
% outside of the main text columns and, therefore, unlike regular IEEE
% journals, the available text space is not reduced by their presence.
% If you want to put a publisher's ID mark on the page you can do it like
% this:
%\IEEEpubid{0000--0000/00\$00.00~\copyright~2015 IEEE}
% or like this to get the Computer Society new two part style.
%\IEEEpubid{\makebox[\columnwidth]{\hfill 0000--0000/00/\$00.00~\copyright~2015 IEEE}%
%\hspace{\columnsep}\makebox[\columnwidth]{Published by the IEEE Computer Society\hfill}}
% Remember, if you use this you must call \IEEEpubidadjcol in the second
% column for its text to clear the IEEEpubid mark (Computer Society jorunal
% papers don't need this extra clearance.)



% use for special paper notices
%\IEEEspecialpapernotice{(Invited Paper)}



% for Computer Society papers, we must declare the abstract and index terms
% PRIOR to the title within the \IEEEtitleabstractindextext IEEEtran
% command as these need to go into the title area created by \maketitle.
% As a general rule, do not put math, special symbols or citations
% in the abstract or keywords.
\IEEEtitleabstractindextext{%
\begin{abstract}
  Data association is at the core of many computer vision tasks, e.g., multiple object tracking, image matching, and point cloud registration. Existing methods usually solve the data association problem by network flow optimization, bipartite matching, or end-to-end learning directly. Despite their popularity, we find some defects of the current solutions: they mostly ignore the intra-view context information; besides, they either train deep association models in an end-to-end way and hardly utilize the advantage of optimization-based assignment methods, or only use an off-the-shelf neural network to extract features. In this paper, we propose a general learnable graph matching method to address these issues. Especially, we model the intra-view relationships as an undirected graph. Then data association turns into a general graph matching problem between graphs. Furthermore, to make optimization end-to-end differentiable, we relax the original graph matching problem into continuous quadratic programming and then incorporate training into a deep graph neural network with KKT conditions and implicit function theorem. In MOT task, our method achieves state-of-the-art performance on several MOT datasets. For image matching, our method outperforms state-of-the-art methods with half training data and iterations on a popular indoor dataset, ScanNet. Code will be available at \url{https://github.com/jiaweihe1996/GMTracker}.
\end{abstract}

% Note that keywords are not normally used for peerreview papers.
\begin{IEEEkeywords}
Graph matching, data association, multiple object tracking, image matching.
\end{IEEEkeywords}}


% make the title area
\maketitle


% To allow for easy dual compilation without having to reenter the
% abstract/keywords data, the \IEEEtitleabstractindextext text will
% not be used in maketitle, but will appear (i.e., to be "transported")
% here as \IEEEdisplaynontitleabstractindextext when the compsoc 
% or transmag modes are not selected <OR> if conference mode is selected 
% - because all conference papers position the abstract like regular
% papers do.
\IEEEdisplaynontitleabstractindextext
% \IEEEdisplaynontitleabstractindextext has no effect when using
% compsoc or transmag under a non-conference mode.



% For peer review papers, you can put extra information on the cover
% page as needed:
% \ifCLASSOPTIONpeerreview
% \begin{center} \bfseries EDICS Category: 3-BBND \end{center}
% \fi
%
% For peerreview papers, this IEEEtran command inserts a page break and
% creates the second title. It will be ignored for other modes.
\IEEEpeerreviewmaketitle



% \IEEEraisesectionheading{\section{Introduction}\label{sec:introduction}}
% Computer Society journal (but not conference!) papers do something unusual
% with the very first section heading (almost always called "Introduction").
% They place it ABOVE the main text! IEEEtran.cls does not automatically do
% this for you, but you can achieve this effect with the provided
% \IEEEraisesectionheading{} command. Note the need to keep any \label that
% is to refer to the section immediately after \section in the above as
% \IEEEraisesectionheading puts \section within a raised box.




% The very first letter is a 2 line initial drop letter followed
% by the rest of the first word in caps (small caps for compsoc).
% 
% form to use if the first word consists of a single letter:
% \IEEEPARstart{A}{demo} file is ....
% 
% form to use if you need the single drop letter followed by
% normal text (unknown if ever used by the IEEE):
% \IEEEPARstart{A}{}demo file is ....
% 
% Some journals put the first two words in caps:
% \IEEEPARstart{T}{his demo} file is ....
% 
% Here we have the typical use of a "T" for an initial drop letter
% and "HIS" in caps to complete the first word.

% You must have at least 2 lines in the paragraph with the drop letter
% (should never be an issue)
\section{Introduction}
\label{sec:introduction}
% \begin{itemize}
%     % Diffusion of FL
%     \item {\st{Diffusion of FL}}
%     % Security threats to FL
%     \item {\st{Security threats to FL with particular focus on model poisoning}}
%     % Limitations of existing countermeasures
%     \item {\st{Current countermeasures (e.g., KRUM) and their limitations}}
%     % Proposed method and its advantages
%     \item {\st{Intuitive description of the proposed method and its difference (i.e., advantages) w.r.t. state of the art}}
%     % Main contributions
%     \item {\st{Summary of the main contributions of this work}}
%     % Paper's structure and organization
%     \item {\st{Paper's structure and organization}}
% \end{itemize}

% Diffusion of FL
Recently, {\em federated learning} (FL) has emerged as the leading paradigm for training distributed, large-scale, and privacy-preserving machine learning (ML) systems~\cite{mcmahan2017googleai,mcmahan2017aistats}. 
The core idea of FL is to allow multiple edge clients to collaboratively train a shared, global model without disclosing their local private training data.
%Specifically, an FL system consists of a central server and many edge clients; 
A typical FL round involves the following steps: {\em(i)} the server randomly picks some clients and sends them the current, global model; {\em(ii)} each selected client locally trains its model with its own private data; then, it sends the resulting local model to the server;\footnote{Whenever we refer to global/local model, we mean global/local model {\em parameters}.} {\em(iii)} the server updates the global model by computing an \emph{aggregation function}, usually the average (FedAvg), on the local models received from clients.
% \begin{enumerate}
%     \item[{\em(i)}] the server sends the current, global model to the clients and appoints some of them for training;
%     \item[{\em(ii)}] each selected client locally trains its copy of the global model with its own private data; then, it sends the resulting local model back to the server;\footnote{Whenever we refer to global/local model, we mean global/local model {\em parameters}.}
%     \item[{\em(iii)}] the server updates the global model by computing an \emph{aggregation function} on the local models received from clients (by default, the average, also referred to as FedAvg~\cite{mcmahan2017aistats}).
% \end{enumerate}
This process goes on until the global model converges. %(e.g., after a certain number of rounds or other similar stopping criteria).
%\\
% The advantages of FL over the traditional, centralized learning paradigm are undoubtedly clear in terms of flexibility/scalability (clients can join/disconnect from the FL network dynamically), network communications (only model weights\footnote{We will use \textit{parameters} and \textit{weights} interchangeably.} are exchanged between clients and server), and privacy (each client's private training data is kept local at the client's end and not uploaded to the server).
\\
% Security threats to FL
%However, the growing adoption of FL also raises security concerns~\cite{costa2022covert}, particularly about its confidentiality, integrity, and availability.
Although its advantages over standard ML, FL also raises security concerns~\cite{costa2022covert}. %, particularly about its confidentiality, integrity, and availability~\cite{costa2022covert}.
% OLD, LONG VERSION
% Indeed, some work deals with privacy leakage that may expose the local data of some clients~\cite{melis2019sp}. 
% A large body of work, instead, investigates attacks that usually aim to detriment the predictive accuracy of the learned global model. For instance, \emph{data poisoning} attacks achieve this goal by letting an adversary pollute the training set of some corrupt FL clients with maliciously crafted examples~\cite{jagielski2018sp}.
% Similarly, in \emph{model poisoning} the attacker attempts to tweak the global model weights~\cite{bhagoji2019pmlr} by directly perturbing the local model's weights of some infected FL clients before these are sent to the central server for aggregation, usually via so-called Byzantine attacks. 
% It turns out that Byzantine model poisoning attacks severely impact standard FedAvg; therefore, more robust aggregation functions must be designed to make FL systems secure.
Here, we focus on \emph{untargeted model poisoning} attacks~\cite{bhagoji2019pmlr}, where an adversary attempts to tweak the global model weights %\footnote{We will use the terms \textit{parameters} and \textit{weights} interchangeably.} 
by directly perturbing the local model's parameters of some infected clients before these are sent to the central server for aggregation.
In doing so, the adversary aims to jeopardize the global model \textit{indiscriminately} at inference time.
Such model poisoning attacks severely impact standard FedAvg; therefore, more robust aggregation functions must be designed to secure FL systems.
\\
% In this paper, we focus on designing a novel robust aggregation scheme at the server's end to contrast the effect of Byzantine model poisoning attacks.
%
% Current countermeasures and their limitations
%Several countermeasures have been proposed in the literature to combat model poisoning attacks on FL systems.
% Some methods use simple statistics more robust than plain average to smooth the impact of malicious updates (e.g., Trimmed Mean and FedMedian~\cite{yin2018icml}). 
% Other defenses implement outlier detection techniques to discard malicious updates from the aggregation performed at the server's end. Those are either based on heuristics (e.g., Krum/Multi-Krum~\cite{blanchard2017nips} and Bulyan~\cite{mhamdi2018pmlr}) or data-driven approaches (e.g., K-means clustering~\cite{shen2016acm} or DnC via spectral analysis~\cite{shejwalkar2021ndss}). 
% Finally, some strategies rely on a centralized ``source of trust'' to spot potential malicious updates (e.g., FLTrust~\cite{cao2020fltrust}).
% Several countermeasures have been proposed in the literature to combat model poisoning attacks on FL systems, i.e., to discard possible malicious local updates from the aggregation performed at the server's end. 
% These techniques range from simple statistics more robust than plain average (e.g., Trimmed Mean and FedMedian~\cite{yin2018icml}) to outlier detection heuristics (e.g., Krum/Multi-Krum~\cite{blanchard2017nips} and Bulyan~\cite{mhamdi2018pmlr}) or data-driven approaches (e.g., spectral analysis via K-means clustering~\cite{shen2016acm} or spectral analysis), or methods based on ``source of trust'' (e.g., FLTrust~\cite{cao2020fltrust}).
% OLD, LONG VERSION
%Several countermeasures have been proposed in the literature to combat Byzantine model poisoning attacks on FL systems.
% Descriptive statistics
% For example, Trimmed Mean and FedMedian aggregate local model updates using more robust statistics than standard average~\cite{yin2018icml}.
%
% % Heuristics for outlier detection
% Many existing Byzantine-resilient strategies implement some outlier detection heuristics to discard the model updates sent by potentially malicious clients from the input of the aggregation function.
% One of the most popular heuristics is Krum~\cite{blanchard2017nips}.
% This strategy tries to mitigate the impact of Byzantine attacks by selecting as a global model the local model with the smallest sum of Euclidean distances to {\em all} the other local models.
% Although powerful, Krum requires the server to know (or, at least, estimate) the number of malicious FL clients upfront, which is generally impossible in a realistic attack scenario. %
% Moreover, Krum may become ineffective for complex, high-dimensional model parameter spaces due to the curse of dimensionality.
% Bulyan~\cite{mhamdi2018pmlr} tries to overcome this issue by combining Krum with a variant of Trimmed Mean.
% % Data-driven outlier detection
% Other strategies use data-driven outlier detection techniques -- e.g., via K-means clustering~\cite{shen2016acm} -- to spot potential malicious local model updates. 
% %For instance, Shen et al. propose to cluster local model updates with K-means and thus identify outliers.
%
% % Other techniques
% As far as the server is concerned, any local model received can be from a potential malicious client. 
% FLTrust~\cite{cao2020fltrust} assumes the server acts as a client, i.e., trains a local model on an additional {\em trustworthy} dataset at the server's end and compares it against all the local models from other clients. 
% This way, the server can rely on some ``source of trust'' when discarding potentially malicious clients.
%\\
% Limitations of existing Byzantine-resilient strategies
Unfortunately, existing defense mechanisms either rely on simple heuristics (e.g., Trimmed Mean and FedMedian by~\cite{yin2018icml}) or need strong and unrealistic assumptions to work effectively (e.g., foreknowledge or estimation of the number of malicious clients in the FL system, as for Krum/Multi-Krum~\cite{blanchard2017nips} and Bulyan~\cite{mhamdi2018pmlr}, which, however, cannot exceed a fixed threshold).
Furthermore, outlier detection methods using K-means clustering~\cite{shen2016acm} or spectral analysis like DnC~\cite{shejwalkar2021ndss} do not directly consider the temporal evolution of local model updates received.
Finally, strategies like FLTrust~\cite{cao2020fltrust} require the server to collect its own dataset and act as a proper client, thereby altering the standard FL protocol.
\\
% OLD, LONG VERSION
% Overall, existing Byzantine-resilient strategies are either simple heuristics (e.g., FedMedian) or, if they are more complex, they rely on strong and unrealistic assumptions to work effectively (e.g., knowing the number of malicious clients in the FL system in advance, as for Krum and alike).
% Furthermore, data-driven outlier detection methods do not consider the temporary evolution of local model updates received (e.g., K-means clustering). 
% Finally, strategies like FLTrust requires the server to collect its own dataset and act as a proper client, thereby altering the standard FL protocol.
%
% Description of the proposed method
This work introduces a novel pre-aggregation \textit{filter} robust to untargeted model poisoning attacks. Notably, this filter $(i)$ operates without requiring prior knowledge or constraints on the number of malicious clients and $(ii)$ inherently integrates temporal dependencies. 
The FL server can employ this filter as a preprocessing step before applying \textit{any} aggregation function, be it standard like FedAvg or robust like Krum or Bulyan.
Specifically, we formulate the problem of identifying corrupted updates as a multidimensional (i.e., matrix-valued) time series anomaly detection task. 
The key idea is that legitimate local updates, resulting from well-calibrated iterative procedures like stochastic gradient descent (SGD) with an appropriate learning rate, show \textit{higher predictability} compared to malicious updates. This hypothesis stems from the fact that the sequence of gradients (thus, model parameters) observed during legitimate training exhibit regular patterns, as validated in Section~\ref{subsec:intuition}. %until convergence. 
%This regularity may be more pronounced for smooth convex loss functions, but it can still be captured within an appropriate time window, even for more complex and convoluted loss surfaces. 
%We provide evidence of this claim in Appendix~B, where we show that the average mutual information (i.e., ``predictability''), calculated over pairs of legitimate model updates sent at different FL rounds, is significantly higher than the corresponding computation for a malicious client.
\\
Inspired by the matrix autoregressive (MAR) framework for multidimensional time series forecasting~\cite{chen2021je}, we propose the FLANDERS ({\em \textbf{F}ederated \textbf{L}earning meets \textbf{AN}omaly \textbf{DE}tection for a \textbf{R}obust and \textbf{S}ecure}) filter.
The main advantages of FLANDERS over existing strategies like FLDetector~\cite{zhao2020multivariate} are its resilience to large-scale attacks, where $50\%$ or more FL participants are hostile, and the capability of working under realistic non-iid scenarios.
We attribute such a capability to two key factors: $(i)$ FLANDERS works without knowing a priori the ratio of corrupted clients, and $(ii)$ it embodies temporal dependencies between intra- and inter-client updates, quickly recognizing local model drifts caused by evil players. Below, we summarize our main contributions:

\begin{itemize}
\item[{\em(i)}]
We provide empirical evidence that the sequence of models sent by legitimate clients is more predictable than those of malicious participants performing untargeted model poisoning attacks.
\\
\item[{\em(ii)}] 
We introduce FLANDERS, the first pre-aggregation filter for FL robust to untargeted model poisoning based on multidimensional time series anomaly detection.
\\
\item[{\em(iii)}] 
We integrate FLANDERS into Flower,\footnote{\scriptsize{\url{https://flower.dev/}}} a popular FL simulation framework for reproducibility.
\\
\item[{\em(iv)}] 
We show that FLANDERS improves the robustness of the existing aggregation methods under multiple settings: different datasets, client's data distribution (non-iid), models, and attack scenarios.
\\
\item[{\em(v)}] 
We publicly release all the implementation code of FLANDERS along with our experiments.\footnote{\scriptsize{\url{https://anonymous.4open.science/r/flanders_exp-7EEB}}}
\end{itemize}

% Paper's structure and organization
The remainder of the paper is structured as follows. %some related work and the current state-of-the-art solutions to security issues that FL entails. 
Section~\ref{sec:background} covers background and preliminaries. 
In Section~\ref{sec:related}, we discuss related work.
Section~\ref{sec:problem} and Section~\ref{sec:method} describe the problem formulation and the method proposed. % to tackle it. 
Section~\ref{sec:experiments} gathers experimental results. %, and Section~\ref{sec:limitations} discusses some limitations of this work.
Finally, we conclude in Section~\ref{sec:conclusion}.
 %discusses the limitations of this work and draws future research directions.
%reports conclusions and draws perspectives for future research directions.

%%%%%%% OLD %%%%%%%
%to overcome the resilience of Byzantine failures in distributed Stochastic Gradient Descent computations. 
% The strength of Krum is its time complexity, which is linear in the gradient dimension. 
% However, the robustness of the approach is guaranteed for gradient-based learning applications only when the majority of the clients are not compromised. 
% Besides, the aggregation mechanism of Krum, as well as that of similar methods, is robust from a coarse-grained perspective and does not provide solutions to errors and perturbations that may occur at inference time.
%A related approach to~\cite{blanchard2017nips} is the work of Su et al.~\cite{su2016dc}. Here, the authors propose an iterated approximate agreement to tackle a multi-layer scenario attacked by Byzantine agents. 
%However, the method works efficiently on the sole discrete context and it is inapplicable to continuous state environments.
%\gabri{Maybe, we should just talk about the main limitations of existing countermeasures without digging into their details (or, we can just mention Krum as this is the most popular one). I will move the description of all these methods to the Related Work section.}
\section{Related Work}
\label{sec:relatedwork}

%%%%%%%%%%%%%%%%%%%%%%%%%% Outline %%%%%%%%%%%%%%%%%%%%%%%%%%%%%%%%%%%%%
%(1) Evasion Attacks
%(1.1) Surveys on evasion attacks and their relation to data properties - Michael
%(1.2) Individual papers that study non-data related reasons behind evasion attacks - Michael
%(1.3) Techniques related to evasion attacks and defenses (new) - Gabby
%(2) Non-Evasion Attacks (new), and - ???
%(3) Effects of training data on standard generalization - done 
%
%
%
%(1) Evasion Attacks
%(1.1) A number of surveys review literature on evasion attacks. - Michael
%Most of them do not focus specifically on properties of data but also discuss attack and defense mechanisms, non-data-related reasons for adversarial vulnarability, and  more. ~\jr{cite 4}.
%Yet, they these surveys mention data and its relation to evasion attacks. Specifically \jr{what they say about data.}
%The most close to ours is concurrent work by XXX + concrete facts that we have and they don't.
%
%(1.2) individual papers that study non-data related reasons behind evasion attacks, - Michael
%Literature identifies multiple reasons for adversarial vulnerability, in particular, for evasion attacks. 
%These include data-related properties extensively discussed in this survey, as well as reasons related to the models 		   themselves, computations resources, and feature representations. We discuss these below. 
%
%\jr{the rest is from the paper (non-data related reasons for adversarial vulnerability), with sections potentially renamed.}
%
%{\bf Model.}
%
%{\bf Computational Resources.}
%
%{\bf Robustness of Features.}
%
%(1.3) Techniques Related to Evasion Attacks and Defenses (new) - Gabby
%A number of works focus on techniques for generating evasion attacks, countermeasures against these attacks, 
%and defining the notion of the attack itself.   
%
%{\bf Attacks and Defense.}
%Here are the 5 remaining surveys + 1 additional paper for the reviewer.
%
%{\bf Adversarial Examples.}
%2 surveys lines 13 and 14 + 1 additional paper for the reviewer.
%
%(2) Non-Evasion Attacks (new) 
%Need to say that there are other type of attacks, define them, cite surveys (Bo's survey, maybe something else). 
%Only one work explicitly focus on effects of data. 
%
%
%(3) Effects of training data on standard generalization (done)

%%%%%%%%%%%%%%%%%%%%%%%%% Outline %%%%%%%%%%%%%%%%%%%%%%%%%%%%%%%%%%%%%


\revreplace{
We divide related work into three categories:
(1) surveys on adversarial robustness and its relation to data properties,
(2) surveys that discuss the influence of data properties on standard generalization, and
(3) individual papers that study non-data-related reasons for adversarial vulnerability.\\
}
{
This survey investigates properties of training data in the context of model robustness under evasion attacks. 
We start the discussion of related work by reviewing other surveys that focus on evasion attacks and 
include some discussion about data (Section~\ref{sec:relatedwork-surveys-data}).  
We then discuss non-data related reasons behind evasion attacks (Section~\ref{sec:relatedwork-not-data}),
as well as techniques related to evasion attacks and defenses (Section~\ref{sec:relatedwork-attacks}). 
Finally, we discuss data-related concerns for non-evasion attacks (Section~\ref{sec:relatedwork-poisoning}) and
the effects of training data on standard generalization (Section~\ref{sec:relatedwork-standard}).
}

%\vspace{-0.1in}
\subsection{Surveys on Evasion Attacks that Discuss Data}
\label{sec:relatedwork-surveys-data}
Numerous existing surveys 
\revreplace{focus on attack and defense techniques for adversarial robustness. 
%~\cite{Biggio:Roli:PR:2018,
%Rosenberg:Shabtai:Elovici:Rokach:CSUR:2021,
%Li:Li:Ye:Xu:CSUR:2021,
%Maiorca:Biggio:Giorgio:CSUR:2019,
%Demetrio:Coull:Biggio:Lagorio:Armando:Roli:ACMTPS:2021,
%Liu:Tantithamthavorn:Li:Liu:CSUR:2022,
%Liu:Nogueria:Fernandes:Kantarci:IEEECST:2022,
%Akhtar:Mian:IEEEAccess:2018,
%Akhtar:Mian:Kardan:Shah:IEEEAccess:2021,
%Serban:Poll:Visser:CSUR:2020,
%Machado:Silva:Goldschmidt:CSUR:2021,
%Zhang:Sheng:Alhazmi:Li:ACMTIST:2020}.
Only a few of these works mention the relationship between adversarial robustness and properties of the underlying data.} 
{review the literature on evasion attacks.
Most of these works do not focus specifically on properties of data but discuss attack and defense mechanisms, non-data-related reasons for adversarial vulnerability, 
and the different threat models. 
Only a few of these works mention data-related reasons for the existence of adversarial examples~\cite{Serban:Poll:Visser:CSUR:2020, Machado:Silva:Goldschmidt:CSUR:2021, Akhtar:Mian:Kardan:Shah:IEEEAccess:2021, Akhtar:Mian:IEEEAccess:2018}.
}
Specifically, Serban et al.~\cite{Serban:Poll:Visser:CSUR:2020} observe that adversarial vulnerability can be caused by an insufficient training sample size %~\cite{Schmidt:Santurkar:Tsipras:Talwar:Madry:NeurIPS:2018}
and high data dimensionality. %~\cite{Gilmer:Metz:Faghri:Schoenholz:Raghu:Wattenberg:Goodfellow:ICLR:2018}.
Similarly, Machado et al.~\cite{Machado:Silva:Goldschmidt:CSUR:2021} mention that the lack of sufficient training data, high dimensionality, 
and high concentration contribute to adversarial vulnerability.
\revadd{
Akhtar et al.~\cite{Akhtar:Mian:IEEEAccess:2018, Akhtar:Mian:Kardan:Shah:IEEEAccess:2021} also mention high dimensionality, along with other non-data-related reasons, 
as a source of adversarial examples.}

\revadd{A concurrent work by Han et al.~\cite{Han:Lin:Shen:Wang:Guan:CSUR:2023} (published at the end of April 2023) 
studies the origins of adversarial vulnerability in deep learning w.r.t. the model, data, and other perspectives.
The authors mention high dimensionality, distributions with high concentration, a small number of output classes, data imbalance, and the perceptual difference in image frequencies as potential sources of adversarial examples.
However, as (a) the focus of that survey is not on data-related properties in particular, 
(b) its paper search was conducted in 2021, and 
(c) it focuses on deep learning models only, 
our work was able to identify more than 50 additional relevant papers which focus on other types of models, 
e.g., non-parametric and linear classifiers, 
and/or discuss additional types of data-related properties, 
such as, types of distribution, class density, separation, and label quality.}
\revreplace{Yet, none of these surveys explicitly collect and analyze work that focuses on the effects of data properties
on adversarial robustness.}
{In summary, by explicitly focusing on the effects of data properties on evasion attacks in our survey, 
we are able to provide a more complete and detailed discussion on this topic, not covered in prior surveys.}

\vspace{-0.05in}
\subsection{Non-data-related Reasons Behind Evasion Attacks}
\label{sec:relatedwork-not-data}

%\vspace{-0.1in}
%\subsection{Non-data Related Reasons for Adversarial Vulnerability}

There has been a variety of hypotheses regarding the reasons behind adversarial vulnerability of ML systems, particularly for evasion attacks.
%\revreplace{
%In addition to the data used for training,  adversarial robustness could also depend on the choice of the model architecture,
%the training procedure, and the interplay between data and the learning algorithm, i.e., correspondence between the complexity of a model to that of the data.
%This section summarizes the key hypotheses regarding these aspects.
%%The hypotheses reviewed in this section are complementary to the potential influence from the data.
%}
These include data-related properties extensively discussed in this survey, as well as reasons related to the models themselves, 
computational resources, and feature learning procedures. We discuss these below.

%\jr{there is a lot of undefined terminology and jargon in this section.}

\vspace{0.02in}
\noindent
\textbf{Model.}
When Szegedy et al.~\cite{Szegedy:Zaremba:Sutskever:Bruna:Erhan:Goodfellow:Fergus:ICLR:2014} first discovered adversarial examples for visual models, they suspected that the high non-linearity of DNNs resulted in low probability `pockets' of adversarial examples in the learned representation manifold.
They hypothesize that while these pockets can be found through attack algorithms, the samples residing in these pockets have different distributions compared to normal samples and are thus subsequently harder to find when randomly sampling from the input space.
Instead, Goodfellow et al.~\cite{Goodfellow:Shlens:Szegedy:ICLR:2015} hypothesize that
the linearity from activation functions, like ReLU and sigmoid found in high-dimensional neural networks, induce vulnerability towards adversarial perturbations.
To support their claim, they present the attack method FGSM that exploits the linearity of the target classifier.
Fawzi et al.~\cite{Fawzi:Fawzi:Frossard:ICMLWorkshop:2015} also argue against the hypothesis of high non-linearity as the cause for adversarial examples.
They show that all classifiers are susceptible to adversarial attacks and claim that it is the low flexibility of the classifier compared to the complexity of the classification task that results in vulnerability.
The lack of consensus on the primary causes of model vulnerability invites more studies on this topic.

Singla et al.~\cite{Singla:Ge:Basri:Jacobs:NeurIPS:2021} show that enforcing invariance to circular shifts (e.g., rotation) in neural networks induces decision boundaries with a smaller margin than normal, fully connected networks,
which, in turn, reduces the adversarial robustness of the model.
Moosavi{-}Dezfooli et al.~\cite{Moosavi-Dezfooli:Fawzi:Fawzi:Frossard:Soatto:ICLR:2018} introduce universal,
input-agnostic perturbations to mislead the classifier and hypothesize that the vulnerability of a multi-class classifier to such perturbations is related to the shape of its decision boundaries, e.g.,
linear classifiers with decision boundaries that are parallel to each other and
nonlinear classifier with decision boundaries that are curved in a similar way
tend to be less robust as
perturbations in one direction can change the prediction label for a different class.

Tanay and Griffin~\cite{Tanay:Griffin:ArXiv:2016} conjecture that the decision boundary learned by the classifier being too close to (or `tilted towards') the data manifold instead of being perpendicular to it,
results in small perturbations being sufficient to move samples across the decision boundary for misclassification.
%data manifold refers to the underlying structure that the data exhibit

\vspace{0.02in}
\noindent
\textbf{Computational Resources.}
Bubeck et al.~\cite{Bubeck:Lee:Price:Razenshteyn:ICML:2019} use computational hardness theory to show that the time complexity for learning a robust model is exponential to the size of input data and thus is computationally intractable.
Hence, they attribute adversarial vulnerability to computational limitations of current learning algorithms.
Degwekar et al.~\cite{Degwekar:Nakkiran:Vaikuntanathan:COLT:2019} further extend this work and also show the impossibility of efficiently training robust classifiers.

%\subsubsection{Ineffective Learning Perspective}
\vspace{0.02in}
\noindent
\textbf{Feature Learning.}
Ilyas et al.~\cite{Ilyas:Santurkar:Tsipras:Engstrom:Tran:Madry:NeurIPS:2019} show that adversarial vulnerability can be a consequence of a model exploiting well-generalizing but non-robust features,
i.e., features that are spurious and sometimes incomprehensible to humans;
when constraining the model to use robust features, the adversarial robustness increases together with the
interpretability of the learned features.
However, Tsipras et al.~\cite{Tsipras:Santurkar:Engstrom:Turner:Madry:ICLR:2019} note that, as the features for achieving high accuracy may be different from the ones for achieving high robustness, robustness may be at odds with standard accuracy.
%
%\jr{why is it called Ineffective learning when it is about features.}\gx{I put it under ineffective learning as in this case, the model learns/decides the features for generalization, and when given the correct objective, the model in fact, can learn more robust features, so I think the underlying reason is objective we gave for the model didn't guide the model to learn the right features}
%
Instead of seeing adversarial vulnerability as a product of classifiers being overly sensitive to changes in spurious features, Jacobsen et al.~\cite{Jacobsen:Behrmann:Zemel:Bethge:ICLR:2019} hypothesize that classifiers can rather be
overly insensitive to relevant semantic information, e.g., images with drastically different content can share similar latent representations.
The authors introduce a new type of adversarial examples that exploit such insensitivity, where the content of images is altered without changing the resulting prediction label.
%As both insensitivity to semantic content and sensitivity to spurious changes can simultaneously exist in models,
%more investigation into how to define proper objectives for models to effectively distinguish the relevant information is needed.

While all these works propose possible reasons for adversarial vulnerabilities, they are orthogonal to our survey, which focuses particularly on the influence of training data.

\vspace{-0.05in}
\revadd{
\subsection{Evasion Attacks and Defenses}
\label{sec:relatedwork-attacks}
A number of works focus on techniques for generating evasion attacks, countermeasures against these attacks, 
and defining the notion of the attack itself.

%\jr{need to include~\cite{Biggio:Roli:PR:2018,
%Rosenberg:Shabtai:Elovici:Rokach:CSUR:2021,
%Li:Li:Ye:Xu:CSUR:2021,
%Maiorca:Biggio:Giorgio:CSUR:2019,
%Demetrio:Coull:Biggio:Lagorio:Armando:Roli:ACMTPS:2021,
%Liu:Tantithamthavorn:Li:Liu:CSUR:2022,
%Liu:Nogueria:Fernandes:Kantarci:IEEECST:2022,
%Zhang:Sheng:Alhazmi:Li:ACMTIST:2020} x and one more survey.}
%\js{\cite{Biggio:Roli:PR:2018, Rosenberg:Shabtai:Elovici:Rokach:CSUR:2021} moved to Adversarial Examples.
%\cite{Rosenberg:Shabtai:Elovici:Rokach:CSUR:2021,
%Li:Li:Ye:Xu:CSUR:2021,
%Maiorca:Biggio:Giorgio:CSUR:2019, Liu:Tantithamthavorn:Li:Liu:CSUR:2022,
%Liu:Nogueria:Fernandes:Kantarci:IEEECST:2022,
%Zhang:Sheng:Alhazmi:Li:ACMTIST:2020, Demetrio:Coull:Biggio:Lagorio:Armando:Roli:ACMTPS:2021} in Attacks and Defense. \cite{Sun:Dou:Yang:Zhang:Wang:Philip:He:Li:TKDE:2022} was the "one more survey" and is also in Attacks and Defenses.}

\vspace{0.02in}
\noindent
{\bf Attacks and Defense.}
Several works~\cite{Liu:Tantithamthavorn:Li:Liu:CSUR:2022,Liu:Nogueria:Fernandes:Kantarci:IEEECST:2022,Sun:Dou:Yang:Zhang:Wang:Philip:He:Li:TKDE:2022, Demetrio:Coull:Biggio:Lagorio:Armando:Roli:ACMTPS:2021} survey adversarial attacks and defenses, observing that most work focuses on computer vision and NLP domains. 
Zhang et al.~\cite{Zhang:Sheng:Alhazmi:Li:ACMTIST:2020}, 
Rosenberg et al.~\cite{Rosenberg:Shabtai:Elovici:Rokach:CSUR:2021},
Li et al.~\cite{Li:Li:Ye:Xu:CSUR:2021}, and 
Maiorca et al.~\cite{Maiorca:Biggio:Giorgio:CSUR:2019}, 
survey attacks and defenses in the NLP domain, cybersecurity domain for networks, Android malware, and PDF malware, respectively. 
These works identify a similar trend of new attacks constantly bypassing defenses, which gives rise to new defenses being proposed, only to be broken again (a.k.a. the `cat and mouse race' or the `arms race'). 
They also observe that research in this field studies attacks / defenses at a feature-level, which restricts 
the practicality of the developed techniques by the feasibility of perturbing the corresponding features in real life. 

%practical attacks are quite difficult and require some basic knowledge about the model or training data such as the feature set or model architecture. 
%Zhang et al.~\cite{Zhang:Sheng:Alhazmi:Li:ACMTIST:2020}, who study adversarial attacks and defenses in the NLP domain,  
%also find that there are obstacles to generating attacks in real-time. 
%For instance, methods that iteratively use gradients to create adversarial examples can be time-consuming, while one-time approaches may fail to produce potent adversarial examples.
%Several works~\cite{Liu:Tantithamthavorn:Li:Liu:CSUR:2022,Liu:Nogueria:Fernandes:Kantarci:IEEECST:2022,Sun:Dou:Yang:Zhang:Wang:Philip:He:Li:TKDE:2022, Demetrio:Coull:Biggio:Lagorio:Armando:Roli:ACMTPS:2021} 
%discuss how most new attacks and defenses are explored in computer vision and NLP, prior to other fields.


%our survey finds the state of the art w.r.t. data properties
%our survey finds that dimensionality is bad ...
%
%%%Here are the 5 remaining surveys + 1 additional paper for the reviewer.
%Numerous surveys have explored the landscape of adversarial evasion attacks and defenses. 
%For instance, Akhtar et al.~\cite{Akhtar:Mian:IEEEAccess:2018, Akhtar:Mian:Kardan:Shah:IEEEAccess:2021} survey the literature on adversarial robustness of deep learning models from Computer Vision field.
%They review popular attacks on visual models, and provided a categorization of existing defense techniques based on the components it modify in the visual model system \gx{Check}.
%
%Rosenberg et al.~\cite{Rosenberg:Shabtai:Elovici:Rokach:ACMComputingSurvey:2021}, Li et al. ~\cite{Li:Li:Ye:Xu:ACMComputingSurvey:2021} and Demetrio et al.~\cite{Demetrio:Coull:Biggio:Lagorio:Armando:Roli:ACMTPS:2021} review the literature on evasion attacks for cyber-security fields. 
%Li et al. proposed a partial order scheme to compare key attacks and defenses techniques for malware detection in Windows, Android, and PDF domains. 
%
%Zhang et al.~\cite{Zhang:Sheng:Alhazmi:Li:ACMTIST:2020} review the literature on adversarial attacks on deep-learning models for textual classification.
%They pointed out the intrinsic differences between Computer Vision and Natural Language Processing fields that pose challenges to directly apply attacks proposed for Visual models to NLP models and identified the strategies proposed that overcomes the barriers.
%The challenges they identified for creating realistic attacks in NLP fields are from a domain characteristics perspective (e.g., definition of imperceptible perturbations, measurement of the semantic changes),  we differ from them by trying to understand the adversarial robustness of machine learning from the characteristics of underlying data. 
%
%Attack and Defenses for wireless and Mobile systems~\cite{Liu:Nogueria:Fernandes:Kantarci:IEEECST:2022}
%
%

More recent research, not included in the surveys above, has also started investigating the 
susceptibility of newer models to adversarial evasion attacks. 
For example, several studies~\cite{Wang:Pan:Hu:Duan:Pan:IJSWIS:2022,Yin:Lin:Sun:Wei:Chen:TIFS:2023, 
Shi:Han:Tan:Kuang:NeurIPS:2022, Wang:Xie:Microsoft:ChatGPT:ArXiv:2023} proposed attack techniques against contemporary models, 
such as Graph Neural Networks, Generative Pre-training Transformers (GPT), and Vision Transformers. 
These studies showed that adversarial examples persist even for the newer models, some of which are 
trained with large volumes of data. 
As all these works focus on attack and defense mechanisms rather than 
the effects of data on adversarial robustness, our work extends and complements this research.
}

\revadd{
\vspace{0.02in}
\noindent
{\bf Adversarial Examples.}
%2 surveys lines 13 and 14 + 1 additional paper for the reviewer.
Adversarial examples are inputs constructed by perturbing a correctly classified sample in a way that makes the change imperceptible to a human. % but causes the model to misclassify the sample.
However, as `imperceptible to a human' is hard to define, existing research on adversarial examples approximates imperceptibility with a small perturbation measured through $L_p$ norms.
A line of research~\cite{Gilmer:Adams:Goodfellow:Anderson:Dahl:ArXiv:2018,Sharif:Bauer:Reiter:CVPRW:2018,Fezza:Bakhti:Hamidouche:Deforges:QoMEX:2019, Mezher:Deng:Karam:EUVIP:2022} 
investigates the validity of this assumption. 
This work shows that perturbations generated by $L_p$ norms do not entirely align with human perceptions, 
i.e., some changes with a small $L_p$ norm can be apparent to humans. 
In addition, adversarial examples with the minimum $L_p$ perturbation may be less effective and transferable than 
higher perturbation~\cite{Biggio:Roli:PR:2018,Rosenberg:Shabtai:Elovici:Rokach:CSUR:2021}. 
Hence, a number of approaches explore metrics for imperceptibility 
in computer vision and NLP domains~\cite{Fezza:Bakhti:Hamidouche:Deforges:QoMEX:2019,Mezher:Deng:Karam:EUVIP:2022, Zhang:Sheng:Alhazmi:Li:ACMTIST:2020}. 
Yet another issue with $L_p$ norms is that they cannot be used reliably in domains other than images. 
For example, in the case of software/malware, simply generating adversarial examples with $L_p$ norms 
may result in feature representations that are not possible in 
the problem space~\cite{Rosenberg:Shabtai:Elovici:Rokach:CSUR:2021,Pierazzi:Pendlebury:Cortellazz:Cavallaro:2020}. 

While all these works focus on the properties of adversarial examples, 
they are orthogonal to the topic of our survey, as we rather focus on how properties of the training data 
affect the success of adversarial examples.
}

%Gilmer et al.~\cite{Gilmer:Adams:Goodfellow:Anderson:Dahl:ArXiv:2018} argue that, while constraining the perturbations by sufficiently small $L_p$ norms can generate indistinguishable samples for most inputs, the actual imperceptibility of the changes depends on the input sample. 
%Several individual studies~\cite{Sharif:Bauer:Reiter:CVPRW:2018,Fezza:Bakhti:Hamidouche:Deforges:QoMEX:2019, Mezher:Deng:Karam:EUVIP:2022} find faults with using $L_p$ norms to generate adversarial examples. They show that the changes measured by $L_p$ norm, does not entirely align with human perceptions, i.e., some changes with a small $L_p$ norm appear apparent to humans. 
%In some domains adversarial examples do not need to be imperceptible but rather semantically preserving. 
%For example, in the case of Android malware~\cite{Rosenberg:Shabtai:Elovici:Rokach:CSUR:2021}, adversarial examples are small perturbations which fool a model while preserving the semantics of the sample, 
%i.e., a malware stays malicious even after the perturbation. 
%This highlights another problem with $L_p$ norm based adversarial examples as Dong et al.~\cite{Dong:Liu:Shang:NeurIPS:2022} show that the semantics of a sample change during adversarial training. 
%Hence, there is a need for metrics to measure the size of perturbations that is imperceptible or semantically preserving.
%Fezza et al.~\cite{Fezza:Bakhti:Hamidouche:Deforges:QoMEX:2019} and Mezher et al.~\cite{Mezher:Deng:Karam:EUVIP:2022} propose to use objective metrics for image quality to approximate the imperceptibility in the computer vision domain.
%Zhang et al.~\cite{Zhang:Sheng:Alhazmi:Li:ACMTIST:2020}, focusing on providing such a metric for Natural Language Processing.
%Vadillo et al.~\cite{Vadillo:Santana:CS:2022} also highlight conducted subject studies to evaluate the noticeability of audio adversarial examples.

%Even in computer vision, adversarial examples are not always imperceptible. For example, Machado et al.~\cite{Machado:Silva:Goldschmidt:CSUR:2021} find that visible perturbations such as adversarial patch~\cite{Brown:Mane:Roy:Abadi:Gilmer:ArXiv:2017}, and graffiti on stop signs~\cite{Eykholt:Evtimov:Fernandes:Li:Rahmati:Xiao:Prakash:Kohno:Song:CVPR:2018} are also considered adversarial examples in research.

%The aforementioned research examines the work on defining and creating adversarial examples, demonstrating the insufficiency of using conventional $L_p$ norms to evaluate the imperceptibility and semantics between clean and adversarial examples. 

\vspace{-0.1in}
\revadd{
\subsection{Non-Evasion Attacks}
\label{sec:relatedwork-poisoning}
Similar to evasion attacks, data poisoning and backdoor attacks aim to compromise model accuracy. 
However, they achieve it by tampering the training data to create deceptive model decision boundaries. 
%Data poisoning attacks involve modifying the training data to create deceptive decision boundaries, either to manipulate the prediction outcomes of a specific input or the entire model.
%Meanwhile, Backdoor attacks are a form of poisoning attacks where the attacker inject tempered training data with triggers 
% and then activates the attack by showing the trigger pattern at inference time.
In addition, backdoor attacks also require perturbing the test instance to result in a misclassification. 
This is achieved by introducing manipulated training data with triggers that can be activated during the testing phase.

Goldblum et al.~\cite{Goldblum:Tsipras:Xie:Chen:Schwarzchild:song:Madry:Li:Goldstein:TPAMI:2022} and Cinà et al.~\cite{Cina:Grosse:Demontis:Sebastiano:Zellinger:Moser:Oprea:Biggio:Pelillo:Roli:CSUR:2023} 
review recent literature on attack methodologies and countermeasures for both poisoning and backdoor attacks.
Both of these surveys found that existing research made overly-optimistic assumptions when designing / validating attack techniques, e.g., assuming the knowledge of a large portion of training data. 
They advocate for researchers to test proposed methods in more realistic situations to better assess the potential threats. 
Furthermore, they encourage exploration of the relationship between poisoning attacks and evasion attacks. 
This could lead to the creation of attacks that produce less noticeable poisoning examples, 
or defensive strategies that can safeguard models against both backdoor and evasion attacks.
%Their survey catalogs and systematizes the threats in the dataset creation process, and discuss the open problems that benefits the understanding of dataset security. 

In addition to undermining model accuracy, 
adversarial attacks also aim at breaching the privacy and confidentiality of training data. 
In particular, membership inference attacks~\cite{Shokri:Stronati:Song:Shmatikov:SP:2017} attempt to determine whether a specific data point was part of the training set used to train the model.
Hu et al.~\cite{Hu:Salcic:Sun:Dobbie:Yu:Zhang:CSUR:2022} present a comprehensive survey of existing research efforts on membership inference attacks. 
They find that, similar to evasion attacks, the membership inference attack success rate decreases as 
%the training data better represents the whole data distribution, i.e., 
the number of training samples increases.
%and model stealing attacks~\cite{Oliynyk:Mayer:Rauber:CSUR:2023} are designed to breach the privacy of training data and machine learning models. 
However, all these attacks are orthogonal to our survey, as we focus on adversarial evasion attacks.

%Li et al. ~\cite{Li:Jiang:Li:Xia:TNNLS:2022} 
%provide the first survey that focuses on backdoor attacks and identified common scenarios in which backdoor attack happen in real life. 
%Furthermore, they proposed a systematic taxonomy for backdoor attacks and defenses for researchers and practitioners to identify the characteristics and limitations of each method. 

%Wang et al.~\cite{Wang:Ma:Wang:Hu:Qin:Ren:CSUR:2022} and Tian et al.~\cite{Tian:Cui:Liang:Yu:CSUR:2022} argue federated learning~\cite{McMahan:Moore:Ramage:Hampson:Arcas:AISTATS:2017} 
%creates new venue for poisoning attack, and survey recent literature on poisoning attacks for both standard and federated learning scenarios. 
%They present a unified framework to categorize both data poisoning and model poisoning attacks, and compared the defense techniques proposed for each of the learning framework, analyzed their advantages and disadvantages.
}

\vspace{-0.1in}
\subsection{Effects of Training Data on Standard Generalization}
\label{sec:relatedwork-standard}
A number of surveys investigate the influence of data properties on standard
rather than robust generalization.
One of the earliest is probably the work of Raudys and Jain~\cite{Raudys:Jain:TPAMI:1991},
who review studies related to the influence of sample size on binary classifiers, showing that
a limited sample size usually leads to sub-optimal generalization.
%With the development of deep learning and the ever-increasing need for larger training datasets,
%a variety of data augmentation techniques have been proposed.
Bansal et al.~\cite{Bansal:Sharma:Kathuria:CSUR:2021} and
Bayer et al.~\cite{Bayer:Kaufhold:Reuter:CSUR:2022} also survey papers addressing the data scarcity problem,
focusing in particular on the recent advancements in data augmentation techniques in the fields of computer vision, security, and text classification.
Their results show that augmentation techniques %exist for various application domain and
can help improve a model's generalization by reducing the problem of model overfitting.
%They evaluate the effectiveness of such techniques in improving the accuracy of machine learning models.

%Limited sample size is also one of the culprit behind poor robust generalization~\cite{Schmidt:Santurkar:Tsipras:Talwar:Madry:NeurIPS:2018}, we collected a number of researches characterize the sample complexity for robust generalization or propose data augmentation techniques to fill in the sample complexity gap.

Label noise is another aspect of data that influences both standard and robust generalization.
Most works on this topic find that the presence of noisy labels increases the need for a greater number of training samples and may result in unnecessarily complex decision boundaries~\cite{Frenay:Verleysen:TNNLS:2014,Song:Kim:Park:Shin:Lee:TNNLS:2022}.
For example, Fr\'{e}nay and Verleysen~\cite{Frenay:Verleysen:TNNLS:2014} show
that overfitting to label noise greatly degrades a model's standard generalization;
the same effect has been observed in the case of robust generalization~\cite{Sanyal:Dokania:Kanade:Torr:ICLR:2021}.
Song et al.~\cite{Song:Kim:Park:Shin:Lee:TNNLS:2022} survey the impact of label noise in deep learning, arguing
that the presence of noisy labels is a more serious concern for deep models as they contain a larger number of parameters which makes them prone to overfitting to the noise in training data.
%They also point out the connection between adversarial poisoning attacks and noisy labels as
%the countermeasures for both share the goal of learning noise-resilient representations.
They mention that adversarial defense techniques, e.g., adversarial training, are effective against label noise~\cite{Zhu:Zhang:Han:Liu:Niu:Yang:Kankanhalli:Sugiyama:ArXiv:2021, Fatras:Damodaran:Lobry:Flamary:Tuia:Courty:TPAMI:2022}
but do not discuss how label noise influences a deep learning model's robustness under attacks.

Lorena et al.~\cite{Lorena:Garcia:Lehmann:Souto:Ho:CSUR:2020} identify a collection of 26 quantitative metrics that measure data complexity with respect to
(1) ambiguity of classes, i.e., whether the classes can be clearly distinguished with the given features,
(2) sparsity and dimensionality of data, 
%i.e., whether enough information are provided to learn confident decision boundaries, and
(3) complexity of boundary separating the classes, i.e., whether more intricate functions are required to describe the decision boundaries.
The authors also discuss how these metrics help estimate the difficulty of performing classification on a given dataset.
Similar to our survey, the authors show that high dimensionality and small separation between classes hinder standard generalization.
However, the relationship of some of the metrics reviewed by these authors, e.g.,
%faction of borderline points (i.e., a measure for the complexity of the required decision boundary) and
%the fraction of hyperspheres covering data (i.e.,
the number of non-intersecting spheres needed to enclose all data points of a class,
to robust generalization is not studied, according to our survey.

%Moreover, the effect of XXX on standard generalization needs future investigation as well (that is if we found something they do not have).

%Knowing the characteristics of a dataset according to these perspectives can assist researchers and practitioners to select optimal learning algorithms~\cite{Ho:Basu:TPAMI:2002}.

He and Garcia~\cite{He:Garcia:TKDE:2009} focus on the imbalance learning problem. %~--
%the disproportion in the number of samples belonging to each class in a given dataset.
The authors found that most standard algorithms %are designed with the assumption of a balanced class distribution.
%These algorithms
fail to reliably represent the characteristics of the imbalanced data and result in unfavorable performance across classes.
Furthermore, L\'{o}pez et al.~\cite{Lopez:Fernandez:Garcia:Palade:Herrera:InfSci:2013} discuss six intrinsic data characteristics that potentially complicate learning from imbalanced data:
low density, sample overlap between classes, noisy data, borderline instances,
dataset shift between training and testing distributions, and
small disjuncts, i.e., disperse small clusters of samples from a single class.
Their analysis concludes that while all these ``unfavorable'' data characteristics further complicate the data imbalance
issues, data overlap between classes is probably one of the most harmful.
To follow up on this point, Santos et al.~\cite{Santos:Henriques:Pedro:Japkowicz:Fernandez:Soares:Wilk:Santos:AIR:2022}
focus on the joint effect of data imbalance and class overlap on model generalization.
The negative impact of data imbalance, low separation, and noisy data on robust generalization was also discussed in our survey.
Yet, the compounding effect of these factors, as well as the effect of other properties,
on robust generalization needs future investigation.

Recently, Yang et al.~\cite{Yang:Jiang:Song:Guo:IJCV:2022} summarized relevant studies focusing on
long-tailed distributions in the field of Computer Vision.
% and categorize the main methods for alleviating the issues caused by long-tailed distribution.
%They present quantitative metrics for measuring data imbalance and .
This survey also includes work on the influence of long-tail distributions on a model's adversarial robustness~\cite{Wu:Liu:Huang:Wang:Lin:CVPR:2021}, which is covered in our survey.
%which is included in our survey,
The authors advocate for more research on adapting long-tailed-based approaches for standard generalization to improve robust generalization.

Finally, Moreno-Torres et al.~\cite{MorenoTorres:Raeder:Rodrigues:Chawla:Herrera:PR:2012} present a unifying framework to categorize existing definitions of dataset shift~-- the case where the joint distribution of inputs and outputs differs between training and testing data.
While ML models are normally trained under the premise that testing data has a similar distribution to the training data,
in reality, the observed data distribution may be different from the historical data that the model is trained on.
Such difference can substantially compromise the quality of model predictions.
The authors analyze the possible causes for dataset shift, e.g., malicious software that evolves over time, and
review the techniques dealing with dataset shift.
They characterize adversarial attacks as one form of dataset shift, where adversaries adaptively
change test instances to create a distribution that differs from training data.
%All works discussed in our survey assumed similar distribution on training and testing data, treating adversarial attacks as the only dataset shift in the problem setup.
%However, in real applications, the underlying data distribution itself can be non-stationary, and the characterize the influence of the dataset shift between training and testing data on the adversarial robustness is yet to be investigated.

\revadd{Overall, despite the similarities with our work, literature discussed in this section focuses on standard generalization while our survey discusses 
the effect of data on robust generalization.}

%More works use the connection between adversarial attacks and distributional shift to analyze the effect of adversaries on generalization performance~\cite{Tu:Zhang:Tao:NeurIPS:2019}.
%However, we do not discuss them in detail, as they focus more on models instead of data.
%\jr{How is that relevant to data properties section?} \gx{This can be removed, as it an individual work we filtered}

\vspace{-0.1in}
\subsection{Summary}
\revadd{
Our survey is the first to explicitly focus on properties of training data in the context of model robustness under evasion attacks.
Numerous other surveys on evasion attacks discuss attack and defense mechanisms, non-data-related reasons for adversarial vulnerability, and the different threat models. 
We identified only five surveys that considered data-related reasons for evasion attacks. 
However, as these surveys are older and do not focus on data in particular, our work provides a more extensive
and comprehensive view on this topic. 
By including more than 50 papers not covered in prior work, we were able to 
identify additional relevant properties, practical suggestions, and future research directions in this area. 

Additional work studies non-data-related reasons for evasion attacks, as well as non-evasion attacks, 
such as poisoning and backdoor. 
Yet another body of literature examines how data properties affect standard generalization. These works show that 
some of the properties discussed in our survey, such as 
the number of samples, dimensionality, and label quality, also affect clean accuracy. 
There are also additional data properties that are covered exclusively by these or by our work. 
Studying the interplay between data properties for clean and robust accuracy is an interesting research direction, 
which could be facilitated by our work. 
However, all these current works are orthogonal and complementary to ours.
}

%\ad{
%The related work of our survey can be categorized into four key topics: 
%The first topic examines data for other adversarial attacks, this include the research that investigates the link between the data characteristics and model's resilience against poisoning attacks as well as the studies that explore data poisoning and backdoor attacks and their countermeasures. \jr{same issues as before: this is meta-summary, we need a concrete summary.}
%These studies complement our survey as they highlight the threats directly aimed at data, thus emphasizing the importance of secure data collection. 
%The second topic focuses on the relationship between various properties of training data and model's standard generalization ability. 
%This body of work suggests that data traits such as number of samples, dimensionality, label quality also influence model's ability to generalize in standard classification. \jr{this looks more concrete!}
%
%The third strand of research concerns adversarial evasion attacks. 
%The work in this area encompasses the research frontier in evasion attacks and the countermeasures. 
%Due to the large volume of work in this area, there are numerous surveys that gives more detail on the advancement. 
%\jr{meta-summary again}
%In addition to attacks and defenses, one relevant line of work investigates the alignment of the conventional similarity metrics used for adversarial examples and human perception, showing the need for supplementary metrics. \jr{why important?}
%These studies \jr{which "these studies"?} collectively present an extensive overview of other types of work conducted on adversarial robustness.
%The last category of work proposes alternative explanations for model vulnerability to adversarial examples.
%These studies presented hypothesis showing the characteristics of machine learning models, e.g., nonlinearity, invariance to rotational shift etc, induces susceptibility to attacks, as well as limited computational resources and non-robust feature representations. \jr{all text based on previous related work looks somewhat concrete; the new additions should be at least at the same level, or better.}
%These studies supplement our work, offering a broader perspective of potential factors affecting model's robust generalization ability. }
%


\section{Graph Matching Formulation for Data Association}
\label{sec:relax}
In this section, we will formulate data association problem as a graph matching problem. Instead of solving the original Quadratic Assignment Problem (QAP), we relax the graph matching formulation as a convex quadratic programming (QP) and extend the formulation from the edge weights to the edge features. The relaxation facilitates the differentiable and joint learning of feature representation and combinatorial optimization.
\begin{comment}
\textcolor{red}{In this section, we will describe the formulation of graph matching and how to relax it as a convex quadratic programming (QP) problem. Then we will show how to extend the formulation from edge weights to edge features.}
\end{comment}
%In general, the \textbf{intuition} of our derivation below is that we relax the basic formulation of graph matching to a convex quadratic programming (QP), and it can become a QP layer in our learnable pipeline. To expand the formulation from edge weights to edge features, we finally have a formulation Eq.\ref{finalQP}.
\subsection{Basic Graph Matching Formulation for Data Association}
% Given the detection graph $\MCG_D$ and the tracklet graph $\MCG_T$, the graph matching problem is to maximize the similarities between the matched vertices and corresponding edges connected by these vertices. In the following derivation, we use the general notation $\mathcal{G}_1$ and $\mathcal{G}_2$ to obtain a general graph matching formulation.
We define the aim of data association is to match the vertices in graph $\mathcal{G}_1$ and $\mathcal{G}_2$ constructd in view 1 and view 2 respectively. So, it can be seen as a graph matching problem, which is to maximize the similarities between the matched vertices and corresponding edges connected by these vertices.
As defined in \cite{LawlerMS63}, the graph matching problem is a Quadratic Assignment Problem (QAP) . A practical mathematical form is named \emph{Koopmans-Beckmann's} QAP \cite{kbqap}:
\begin{equation}
\label{equ:KBQAP}
\begin{aligned}
& \underset{\MPI}{\text{maximize}}
&& \mathcal{J}(\MPI)=\text{tr}(\MA_1\MPI\MA_2\MPI^\top)+\text{tr}(\MB^\top\MPI),  \\
& \text {s.t.}
&& \MPI\mathbf{1}_{n}= \mathbf{1}_{n}, \MPI^\top\mathbf{1}_{n}= \mathbf{1}_{n},
\end{aligned}
\end{equation}
where $\MPI\in\{0,1\}^{n\times{n}}$ is a permutation matrix that denotes the matching between the vertices of two graphs, $\MA_1\in\MBR^{n\times n}$, $\MA_2\in\MBR^{n\times n}$ are the weighted adjacency matrices of graph $\mathcal{G}_1$ and $\mathcal{G}_2$ respectively, and $\MB\in\MBR^{n\times n}$ is the vertex affinity matrix between $\mathcal{G}_1$ and $\mathcal{G}_2$. $\mathbf{1}_{n}$ denotes an n-dimensional vector with all values to be 1.

\subsection{Reformulation and Convex Relaxation}
\label{sec:qp}
For \emph{Koopmans-Beckmann's} QAP, as $\MPI$ is a permutation matrix, i.e., $\MPI^\top\MPI=\MPI\MPI^\top=\mathbf{I}$. Following~\cite{facgm}, Eq.~\ref{equ:KBQAP} can be rewritten as
\begin{equation}
\begin{aligned}
\mathbf{\Pi}^*
&=\underset{\mathbf{\Pi}}{\arg\min} \ \frac{1}{2}||\mathbf{A_1}\mathbf{\Pi}-\mathbf{\Pi}\mathbf{A_2}||_F^2-\text{tr}(\mathbf{B}^\top\mathbf{\Pi}).
\end{aligned}
\label{K-B}
\end{equation}
%\begin{comment}
%    &=  \underset{\mathbf{\Pi}\in\mathcal{P}}{\arg\max} \ \text{tr}(\mathbf{A_1}\mathbf{\Pi}\mathbf{A_2}\mathbf{\Pi}^\top)+\text{tr}(\mathbf{K}^\top\mathbf{\Pi}) \\
%&=  \underset{\mathbf{\Pi}\in\mathcal{P}}{\arg\max} \ \text{tr}(\mathbf{A_1}\mathbf{\Pi}\mathbf{A_2}\mathbf{\Pi}^\top)  - \frac{1}{2} \text{tr}(\mathbf{A_1}\mathbf{A_1}\mathbf{\Pi}\mathbf{\Pi}^\top) \\
%&\ \ \ \ -\frac{1}{2}\text{tr}(\mathbf{A_2}\mathbf{A_2}\mathbf{\Pi}^\top\mathbf{\Pi})+\text{tr}(\mathbf{K}^\top\mathbf{\Pi}) \\
%&= \underset{\mathbf{\Pi}\in\mathcal{P}}{\arg\max} \ - \frac{1}{2} ||\mathbf{A_1}\mathbf{\Pi}-\mathbf{\Pi}\mathbf{A_2}||_F^2+\text{tr}(\mathbf{K}^\top\mathbf{\Pi}) \\
%\end{comment}
This formulation is more intuitive than that in Eq.~\ref{equ:KBQAP}. For two vertices $i, i' \in \MCG_1$  
and their corresponding vertices $j, j' \in \MCG_2$, the first term in Eq.~\ref{K-B} denotes the difference of the weight of edge $(i, i')$ and $(j, j')$, and the second term denotes the vertex affinities between $i$ and $j$. Then the goal of the optimization is to maximize the vertex affinities between all matched vertices, and minimize the difference of edge weights between all matched edges.
 
It can be proven that the convex hull of the permutation matrix lies in the space of the doubly-stochastic matrix. So, as shown in \cite{aflalo2015convex}, the QAP (Eq.~\ref{K-B}) can be relaxed to its tightest convex relaxation by only constraining the permutation matrix $\mathbf{\Pi}$ to be a double stochastic matrix $\mathbf{X}$, formed as the following QP problem:
\begin{equation}
\mathbf{X}^*=\underset{\mathbf{X}\in\mathcal{D}}{\arg\min} \ \frac{1}{2}||\mathbf{A_1}\mathbf{X}-\mathbf{X}\mathbf{A_2}||_F^2-\text{tr}(\mathbf{B}^\top\mathbf{X}),
\label{QP}
\end{equation}
where $\mathcal{D}=\{\mathbf{X}:\mathbf{X}\mathbf{1}_n= \mathbf{1}_n, \mathbf{X}^\top\mathbf{1}_n=\mathbf{1}_n,\mathbf{X}\geq\mathbf{0}\}$.
%By the property of doubly stochastic matrix, The optimal solution $\mathbf{X}^*$ lies on the vertices of the constraint set $\mathcal{D}$. Thus it is equivalent to the optimal solution $\MPI^*$ in the original problem in Eq.~\ref{K-B}.
%\begin{comment}
%{\color{red}
%Then, to formulate the QP(Eq. \ref{QP}) as a standard QP formulation, we should first vectorize the matrix $\mathbf{X}$, i.e. $\mathbf{x}=\text{vec}(\mathbf{X})$. So, the relaxed QP can be formed as 
%\begin{equation}
%\begin{aligned}
%\mathbf{x}^*
%&=\underset{\mathbf{x}\in\mathcal{D}^{'}}{\arg\min} \ ||\mathbf{B_1}\mathbf{x}-\mathbf{B_2}\mathbf{x}||_2^2-\mathbf{k}_p^\top\mathbf{x}  \\
%&=\underset{\mathbf{x}\in\mathcal{D}^{'}}{\arg\min} \ (\mathbf{B_1}\mathbf{x}-\mathbf{B_2}\mathbf{x})^\top(\mathbf{B_1}\mathbf{x}-\mathbf{B_2}\mathbf{x})-\mathbf{k}_p^\top\mathbf{x} \\
%&= \underset{\mathbf{x}\in\mathcal{D}^{'}}{\arg\min} \ \mathbf{x}^\top(\mathbf{B_1}^\top\mathbf{B_1}-\mathbf{B_2}^\top\mathbf{B_1}-\mathbf{B_1}^\top\mathbf{B_2} \\
%&\ \ \ \ +\mathbf{B_2}^\top\mathbf{B_2})\mathbf{x}-\mathbf{k}_p^\top\mathbf{x} \\
%&=\underset{\mathbf{x}\in\mathcal{D}^{'}}{\arg\min} \ \frac{1}{2}\mathbf{x}^\top\mathbf{Q}\mathbf{x}-\mathbf{k}_p^\top\mathbf{x}
%\label{vecQP}
%\end{aligned}
%\end{equation}
%where, $ \mathcal{D}^{'}=\{\mathbf{x}:\mathbf{A}\mathbf{x}=\mathbf{1},\mathbf{x}\geq\mathbf{0},\mathbf{A}=\left[\begin{array}{ccc}
%\mathbf{1}_{n}^\top\otimes{\mathbf{I}_{n}} \\
%\mathbf{I}_{n}\otimes{\mathbf{1}_{n}^\top}
%\end{array} 
%\right ]\}$, $\mathbf{B_1}=\mathbf{I}\otimes\mathbf{A_1}$ and $\mathbf{B_2}=\mathbf{A_2^\top}\otimes\mathbf{I}$, $\mathbf{k}_p=\text{vec}(\mathbf{K_p})$.
%}
%\end{comment}
\section{Graph Matching for MOT}
In this section, we introduce the problem definition of MOT and our graph matching formulation for the data association in MOT task.
\subsection{Detection and Tracklet Graphs Construction}
\label{sec:construct}
As an online tracker, we track objects frame by frame. 
In frame $t$, we define $\MCD^t=\{D_1^t, D_2^t,\cdots, D_{n_d}^t\}$ as the set of detections in current frame and $\MCT^t=\{T_1^t, T_2^t, \cdots, T_{n_t}^t\}$ as the set of tracklets obtained from past frames. $n_d$ and $n_t$ denote the number of detected objects and tracklet candidates. A detection is represented by a triple $D_p^t=(\mathbf{I}_p^t, \mathbf{g}_p^t, t)$, where $\mathbf{I}_p^t$ contains the image pixels in the detected area, $\mathbf{g}_p^t=(x_p^t,y_p^t,w_p^t,h_p^t)$ is a geometric vector including the central location and size of the detection bounding box. Each tracklet contains a series of detected objects with the same tracklet id. With a bit abuse of notations, the generation of $T_{id}^t$ can be represented as $T_{id}^{t} \gets T_{id}^{t-1} \cup \{D^{t-1}_{(id)}\}$, which means we add $D^{t-1}_{(id)}$ to the tracklet $T_{id}^{t-1}$.

Then we define the detection graph in frame $t$ as $\MCG_D^t=(\MCV_D^t, \MCE_D^t)$ and the tracklet graph up to the frame $t$ as $\MCG_T^t=(\MCV_T^t, \MCE_T^t)$. Each vertex $i \in \MCV_D^t$ and vertex $j\in \MCV_T^t$ represents the detection $D_i^t$ and the tracklet $T_j^t$, respectively. The $e_u=(i,i')$ is the edge in $\MCE_D^t$ and $e_v=(j,j')$ is the edge in $\MCE_T^t$. Both of these two graphs are complete graphs. Then the data association in frame $t$ can be formulated as a graph matching problem between $\MCG_D^t$ and $\MCG_T^t$. For simplicity, we will ignore $t$ in the following sections.
\subsection{From Edge Weights to Edge Features}
In the general formulation of graph matching, the element $a_{i,i'}$ in the weighted adjacency matrix $\MA \in\MBR^{n\times n}$ is a scalar denoting the weight on the edge $(i, i')$. To facilitate the application in our MOT problem, we expand the relaxed QP formulation by using an \emph{$l_2$-normalized} edge feature $\mathbf{h}_{i,i'} \in \MBR^d$ instead of the scalar-formed edge weight $a_{i,i'}$ in $\mathbf{A}$. We build a weighted adjacency tensor $\mathbf{H} \in \MBR^{d \times n \times n}$ where $\mathbf{H}^{\cdot, i,i'}$ = $\mathbf{h}_{i, i'}$, i.e., we consider the each dimension of $\mathbf{h}_{i,i'}$ as the element $a_{i,i'}$ in $\mathbf{A}$ and concatenate them along channel dimension. The $\mathbf{H_D}$ and $\mathbf{H_T}$ are the weighted adjacency tensors for $\MCG_D$ and $\MCG_T$, respectively. Then the optimization objective in Eq.~\ref{K-B} can be further expanded to consider the $l_2$ distance between two corresponding \emph{n-d} edge features other than the scalar differences:
\begin{equation}
\begin{aligned}
\mathbf{\Pi}^*
&=\underset{\mathbf{\Pi}}{\arg\min} \ \sum_{c=1}^{d}\frac{1}{2}||\mathbf{H}_D^c\mathbf{\Pi}-\mathbf{\Pi}\mathbf{H}_T^c||_F^2-\text{tr}(\mathbf{B}^\top\mathbf{\Pi}) \\
&=\underset{\mathbf{\Pi}}{\arg\min} \ \sum_{i=1}^{n}\sum_{i'=1}^{n}\sum_{j=1}^{n}\sum_{j'=1}^{n}  \frac{1}{2}||\mathbf{h}_{ii'}\pi_{ij}-\mathbf{h}_{jj'}\pi_{i'j'}||_2^2 \\
&\ \ \ \ -\text{tr}(\mathbf{B}^\top\mathbf{\Pi}) \\
&=\underset{\mathbf{\Pi}}{\arg\min} \ \sum_{i=1}^{n}\sum_{i'=1}^{n}\sum_{j=1}^{n}\sum_{j'=1}^{n}  \frac{1}{2}(\pi_{ij}^2-2\pi_{ij}\pi_{i'j'}\mathbf{h}_{ii'}^\top\mathbf{h}_{jj'} \\
&\ \ \ \ +\pi_{i'j'}^2)-\text{tr}(\mathbf{B}^\top\mathbf{\Pi}),
\end{aligned}
\label{before_edge}
\end{equation}
where $n$ is the number of vertices in graph $\mathcal{G}_D$ and $\mathcal{G}_T$, the subscript $i$ and $i'$ are the vertices in graph $\mathcal{G}_D$ and $j$ and $j'$ are in graph $\mathcal{G}_T$. We reformulate Eq.~\ref{before_edge} as: 
\begin{equation}
\bm{\pi}^*=\underset{\bm{\pi}}{\arg\min} \ 
\bm{\pi}^\top((n-1)^2\mathbf{I}-\mathbf{M})\bm{\pi}-\mathbf{b}^\top\bm{\pi},
\label{edge}
\end{equation}
where $\bm{\pi}=\text{vec}(\MPI)$, $\mathbf{b}=\text{vec}(\MB)$ and $\mathbf{M}\in \MBR^{n^2\times n^2}$ is the symmetric quadratic affinity matrix between all the possible edges in two graphs.

Following the relaxation in Section \ref{sec:qp}, the formulation Eq.~\ref{edge} using edge features can be relaxed to a QP:\\
\begin{equation}
\begin{aligned}
\mathbf{x}^*
&=\underset{\mathbf{x}\in\mathcal{D^{'}}}{\arg\min} \ 
\mathbf{x}^\top((n-1)^2\mathbf{I}-\mathbf{M})\mathbf{x}-\mathbf{b}^\top\mathbf{x}, \\
% &=\underset{\mathbf{x}\in\mathcal{D^{'}}}{\arg\min} \ 
% \frac{1}{2}\mathbf{x}^\top\mathbf{Q}\mathbf{x}+\mathbf{q}^\top\mathbf{x} 
\end{aligned}
\label{finalQP}
\end{equation}
where $ \mathcal{D}^{'}=\{\mathbf{x}:\mathbf{R}\mathbf{x}=\mathbf{1},\mathbf{U}\mathbf{x}\leq \mathbf{1},\mathbf{x}\geq\mathbf{0},\mathbf{R}=\mathbf{1}_{n_2}^\top\otimes{\mathbf{I}_{n_1}},\mathbf{U}=\mathbf{I}_{n_2}^\top\otimes{\mathbf{1}_{n_1}}\}$, $\otimes$ denotes Kronecker product.
\begin{figure}
          \centering
           \includegraphics[width=\linewidth]{figures/me2m.pdf}
             \caption{An example of the derivation from edge affinity matrix $\mathbf{M_e}$ to quadratic affinity matrix $\mathbf{M}$.}
             \label{fig:me2m}
  \end{figure}    

 In the implementation, we first compute the cosine similarity between the edges in $\mathcal{G}_D$ and $\mathcal{G}_T$ to construct the matrix $\mathbf{M_e}\in\MBR^{|\mathcal{E}_D|\times |\mathcal{E}_T|}$. 
 The element of the matrix $\mathbf{M_e}$ is the cosine similarity between edge features $\mathbf{h}_{i,i'}$ and $\mathbf{h}_{j,j'}$ in two graphs:
\begin{align}
  \mathbf{M}_e^{u,v}=\mathbf{h}_{i,i'}^\top\mathbf{h}_{j,j'},
  \label{eq:e2e}
\end{align}
where $e_u=(i,{i'})$ is the edge in $\mathcal{G}_D$ and $e_v=(j,{j'})$ is the edge in $\mathcal{G}_T$. 
 
 And following \cite{zanfir2018deep}, we map each element of matrix $\mathbf{M_e}$ to the \emph{symmetric} quadratic affinity matrix $\mathbf{M}$:
\begin{align}
  \mathbf{M}=(\mathbf{S_D}\otimes\mathbf{S_T})\text{diag}(\text{vec}(\mathbf{M_e}))(\mathbf{T_D}\otimes\mathbf{T_T})^\top,
  \label{eq:me2m}
\end{align}
where $\text{diag}(\cdot)$ means constructing a diagonal matrix by the given vector, $\mathbf{S_D} \in \{0,1\}^{|\MCV_D| \times |\MCE_D|}$ and $\mathbf{S_T} \in \{0,1\}^{|\MCV_T| \times |\MCE_T|}$, whose elements are an indicator function:
\begin{equation}
  \mathbb{I}_s(i,u):= \begin{cases}
    1 & \text{if $i$ is the start vertex of edge $e_u$}, \\
    0 & \text{if $i$ is not the start vertex of edge $e_u$},
  \end{cases}
\end{equation}
$\mathbf{T_D} \in \{0,1\}^{|\MCV_D| \times |\MCE_D|}$ and $\mathbf{T_T} \in \{0,1\}^{|\MCV_T| \times |\MCE_T|}$, whose elements are another indicator function:
\begin{equation}
  \mathbb{I}_t(i',u):= \begin{cases}
    1 & \text{if $i'$ is the end vertex of edge $e_u$}, \\
    0 & \text{if $i'$ is not the end vertex of edge $e_u$}.
  \end{cases}
\end{equation}
An example of the derivation from $\mathbf{M_e}$ to $\mathbf{M}$ is illustrated in Fig.~\ref{fig:me2m}. %In brief, each element in $\mathbf{M_e}$ is mapped to the matrix $\mathbf{M}$ to make the edge affinity matrix a symmetric matrix indexed by the vertex pairs.

Besides, each element in the vertex affinity matrix $\mathbf{B}$ is the cosine similarities between feature $\mathbf{h}_i$ on vertex $i \in \MCV_D$ and feature $\mathbf{h}_j$ on vertex $j\in \MCV_T$:
\begin{align}
  \mathbf{B}_{{i,j}}=\mathbf{h}_{i}^\top\mathbf{h}_{j}
  \label{eq:n2n}
\end{align}
\par
\begin{figure*}[ht]
          \centering
           \includegraphics[width=\linewidth]{figures/pipeline.pdf}
             \caption{Overview of our GMTracker method. We first extract features from detections and construct the detection graph using these features. The tracklet graph construction step is similar to the detection graph, but we average the features in a tracklet. Then the cross-graph GCN is adopted to enhance the features. The weight $w_{i,j}$ is from the feature similarity and geometric information. The core of our method is the differentiable graph matching layer built as a QP layer from the formulation in Eq. \ref{finalQP}. The $\mathbf{M}_e$ and $\mathbf{B}$ in the graph matching layer denote the edge affinity matrix from Eq. \ref{eq:e2e} and the vertex affinity matrix from Eq. \ref{eq:n2n} respectively.}
             \label{pipeline}
  \end{figure*}  
\section{Method}
\label{sec:method}

% \ml{``Inconsistent'' to ``large variation''}

% In this section, we propose our methods based on the observations in Section \ref{sec:motivation}.
In this section, we propose two techniques to further enhance the strong baseline to capture the variation of activation distributions better.
We first introduce spatial re-scaling to adapt the network to pixel-to-pixel variation.
We then propose channel-wise shifting and re-scaling to better capture the channel-to-channel variation.
Meanwhile, as both of the two methods are image-dependent, the image-to-image variation can be captured naturally.
By combining the two methods with our strong baseline, we build our enhanced BNN for SR, named EBSR.

% Because the activation distributions among pixels, channels and images have large variations \red{**are highly inconsistent} in SR networks, we introduce spatial re-scaling to adapt to pixel-wise variations and channel shift and re-scaling to adapt to channel-wise variations. And both of them are image-dependent to adapt to image-wise variations, which means during inference our network re-scales and shifts the distributions of activations flexibly for different input images. Based on these methods, we build an enhanced binary neural network for image super-resolution (EBSR).

% According to [3], the difference of activation magnitudes indicates different scaling factors are needed for each pixel.

\subsection{Spatial Re-scaling}
% It is better to use different scaling factors for different pixels to reduce the quantization error and retain more detailed information for image super-resolution. 

% \ml{In the main method, we do not need to introduce the previous works but can focus on introducing our own method. Channel rescaling in Real-to-binary Net is not relevant in this context.}

% Re-scaling the output of binary convolutions was proposed at the birth of BNN in XNOR-Net \cite{rastegari2016xnor} to reduce quantization error and improve accuracy for image classification tasks.
% It is computed as below:
% \begin{equation}
% \mathcal{A} * \mathcal{W} \approx(\operatorname{sign}(\mathcal{A}) \circledast \operatorname{sign}(\mathcal{W})) \odot \mathcal{K} \alpha
% \label{eq:xnor-net rescale}
% \end{equation}
% where $\circledast$ denotes the binary convolution and $\odot$ denotes the element-wise multiplication.
% $\mathcal{A}$, $\mathcal{W}$, $\alpha$, and $\mathcal{K}$ denote the activation, weight, weight scaling factor, and activation scaling factor, respectively.
%  Later in XNOR-Net++ \cite{bulat2019xnor}, Bulat et al. fuse the activation and weight scaling factors into a single one that is learned end-to-end based on gradients and this improves the classification accuracy on ImageNet dataset.

% % It is computed as Eq.~\ref{eq:xnor-net rescale}, where $\circledast$ denotes 
% %  the binary convolution and $\odot$ denotes the element-wise multiplication. The binary convolution of $\mathcal{A}$ and $\mathcal{W}$ is rescaled by the weight scaling factor $\alpha$ and the activation scaling factor $\mathcal{K}$, both of which are calculated analytically.


% \zc{Similarly, you should explain the meaning of A, W and the operators $\circledast$ in the formula}
% Then in Real-to-binary Net \cite{martinez2020training}, Martinez et al. used a data-driven channel re-scaling module that takes the pre-convolution activations as input to predict the activation scaling factor. Unlike that in XNOR-Net++ \cite{bulat2019xnor}, these scaling factors are not fixed during inference but rather inferred from data. By doing this, they further improved the classification accuracy on ImageNet over XNOR-Net++. 
As is shown in Figure \ref{fig:pixel}, activation distributions have large pixel-to-pixel variation in SR networks
and the difference of activation magnitudes indicates different scaling factors are preferred for different pixels.
Inspired by \cite{martinez2020training}, we propose spatial re-scaling to better adapt the network to the spatial variation
of activation distributions in SR networks.
% fit the various pixel-wise distributions in SR networks.
We take the real-valued activations $A$ before convolution as input and predict pixel-wise scaling factors $S(A)$, which re-scale the binary convolution output. Spatial re-scaling process can be formulated as follows:
\begin{equation}
A * W \approx(\operatorname{sign}(A) \circledast \operatorname{sign}(W)) \odot \alpha \odot S(A)
\label{eq:spatial rescale}
\end{equation}
where $\circledast$ denotes 
the binary convolution and $\odot$ denotes the element-wise multiplication. $A$, $W$, $\alpha$, and $S\left(A\right)$ denote real-valued activations, weights, the scaling factor of weights, and the spatial-wise scaling factor of activations respectively. $S\left(A\right) \in \mathbb{R}^{1\times H\times W}$ can be calculated with a convolution and a sigmoid function.
% as $\sigma\left( CONV\left(A\right)\right)$. 
As shown in Figure \ref{fig:method}(a), real-valued activations first go through a convolution layer,
which has an input channel of $C$ and an output channel of 1, 
and then pass through a sigmoid function to produce the scaling factors $S(A)$ along the spatial dimension.
During inference, the scaling factor will change dynamically according to different input feature maps.
By re-scaling binary convolution output using $S(A)$, we can reduce the quantization error and the original pixel-wise information in FP activation
will be preserved much better.
Spatial re-scaling leads to a large PSNR improvement of 0.24 dB (from 30.30 dB to 31.54 dB) on Set5 and 0.22 dB (from 25.09 dB to 25.31 dB)
on Urban100 compared with our strong baseline. 

\subsection{Channel-wise Shifting and Re-scaling}

\begin{table}[!tb]
\centering
\caption{Comparison between whether to fuse channel-wise shifting and re-scaling or not based on our baseline with spatial re-scaling. }
\label{tab:fusing}

\scalebox{0.65}{
\begin{tabular}{c|cc|cc|cc}
\hline
\multirow{2}{*}{Method}     & \multirow{2}{*}{OPs} & \multirow{2}{*}{Params} & \multicolumn{2}{c|}{Set5} & \multicolumn{2}{c}{Urban100} \\ \cline{4-7} 
                            &                      &                         & PSNR        & SSIM        & PSNR          & SSIM         \\ \hline
Baseline + spatial re-scale & 2.16G                & 0.05M                   & 31.54       & 0.883       & 25.31         & 0.759        \\
+ channel-wise shift and re-scale             & 2.34G                & 0.09M                   & 31.61       & 0.885       & 25.35         & 0.761        \\
+ w/ fusing                   & 2.27G                & 0.08M                   & \textbf{31.64}       & \textbf{0.885}       & \textbf{25.36}         & \textbf{0.761}        \\ \hline
\end{tabular}
}
\end{table}

In SR networks, activation distributions exhibit larger channel-to-channel variation (Figure \ref{fig:chl}).
Both the mean and magnitude of the activation distributions vary significantly across channels.
% Thus we use channel-wise shifting and re-scaling to adapt to various channel-wise distributions. 
\cite{martinez2020training} has proposed the data-driven channel re-scaling, 
but our method differs from them in further introducing data-driven thresholds to handle the channel-wise variation of both mean and magnitude.
Since the blocks to generate the scaling factors and thresholds are very similar, we further propose to fuse them into one module.
% and fusing channel-wise shifting and re-scaling into one module.
We evaluate the effect of fusing the two blocks in Table \ref{tab:fusing}.
With channel-wise shifting and re-scaling fused, our models have fewer operations and parameters overhead and slightly higher performance.

For the specific process, we take the real-valued activations as input and predict different thresholds and scaling factors for each channel. They are also image dependent, e.g., $\beta_{i}$ in Eq.\ref{eq:act_binarize} is no longer fixed during inference but generated according to different input feature maps. Channel-wise shifting and re-scaling can be formulated as follows:
\begin{equation}
A * W \approx(\operatorname{sign}(A-C_s(A)) \circledast \operatorname{sign}(W)) \odot \alpha \odot C_r(A)
\label{eq:channel-wise_shift_and_rescale}
\end{equation}
where $\circledast$ denotes 
the binary convolution and $\odot$ denotes the element-wise multiplication. $C_s(A), C_r(A) \in \mathbb{R}^{C\times1\times1}$ denote the channel-wise threshold and scaling factor, respectively. 
We show the block diagram in Figure \ref{fig:method}(b).
The real-valued input feature map is first squeezed to a ${C\times1\times1}$ vector by a global average pooling (GAP) layer.
The subsequent fully connected layers and ReLU learn the channel-wise information and output a ${2C\times1\times1}$ vector.
Then the ${2C\times1\times1}$ vector is split into two ${C\times1\times1}$ vectors.
We use the first $C$ channels as the channel-wise bias and pass the last $C$ channels through a sigmoid layer 
as the channel-wise scaling factor, which are used to shift the real-valued activations and re-scale the binary convolution output, respectively. 


% \ml{We can mention previously, channel-wise re-scale has been proposed. We propose to fuse them. Add the comparison between fuse v.s. no fuse.}

\begin{figure}[!tbp]%
  \centering
    \includegraphics[width=0.4\textwidth]{fig/methods.png}
  
% \subfloat[channel-wise shifting\&re-scale]{
%     \label{subfig:channel-wise shifting and re-scale}
%     \includegraphics[width=0.2\textwidth]{fig/chl shift and rescale.png}
%   }

  \caption{Block diagram for spatial re-scaling, and channel-wise shifting and re-scaling.} 
  % Input A is the real-valued activation tensor and C, H, and W denote its dimension. GAP stands for global average pooling. The reduction ratio r is set to 16 for a better trade-off between the performance and the number of operations and parameters.}
  \label{fig:method}
\end{figure}


\subsection{Network Structure}

Combining the spatial re-scaling and the channel-wise shifting and re-scaling methods, we construct the enhanced convolution layer (E-Conv).
Then we build our EBSR model based on E-Conv.
In Figure \ref{fig:E-conv}, we compare the binary convolution layer used in the baseline network and our proposed E-Conv.
We use spatial and channel-wise scaling factors to re-scale the binary convolution output,
and use channel-wise shifting to learn appropriate thresholds for each channel before binarization.
The scaling factors and threshold used in E-Conv are learnable and depend on the real-valued input activations.
In this way, our proposed EBSR can adapt to pixel-to-pixel, channel-to-channel, and image-to-image variations
to reduce the large binarization error and preserve more details.
% In this way, our proposed E-Conv reduces the large quantization error caused by binarization and keeps the original information of input feature maps to a large extent.


\begin{figure}[!tb]%
  \centering

    \includegraphics[width=0.5\textwidth]{fig/E-conv.png}

  \caption{Comparison of (a) the binary convolution layer with a skip connection used in our baseline network and (b) the proposed E-Conv.}
  \label{fig:E-conv}
\end{figure}


Figure \ref{fig:network} shows the basic block based on the E-Conv and our EBSR composed of the basic blocks. Following existing works, the convolution layers in the head and tail modules are not binarized. We choose the lightweight EDSR which has 16 basic blocks and 64 channels, and EDSR which has 32 basic blocks and 256 channels as our backbones, which correspond to EBSR-light and EBSR, respectively.

\begin{figure}[!tb]%
  \centering
  {
    \includegraphics[width=0.35\textwidth]{fig/network.png}
  }
  
  \caption{The structure of our proposed EBSR.  Convolution layers in purple are real-valued vanilla 3x3 convolutions.}
  \label{fig:network}
\end{figure}
\subsection{Gradients of the Graph Matching Layer}
The gradients of the graph matching layer we need for backward can be derived from the KKT conditions with the help of the implicit function theorem. Here, we show the details of deriving the gradients of a standard QP optimization. 
\par
For a quadratic programming (QP), the standard formulation is as
\begin{equation}
    \begin{aligned}
    & \minimize_x
    & & \frac{1}{2}x^\top Q(\theta) x + q(\theta)^\top x \\
    & \subjectto
    && G(\theta)x \leq h(\theta) \\
    &&& A(\theta)x=b(\theta).
    \end{aligned}
    \end{equation}
So the Lagrangian is given by 
\begin{equation}
    L(x,\nu,\lambda)=\frac{1}{2}x^\top Qx+\lambda^\top (Gx-h)+q^\top x+\nu^\top (Ax-b),
\end{equation}
where, $\nu$ and $\lambda$ are the dual variables.\\
The $(x^*,\lambda^*,\nu^*)$ are the optimal solution if and only if they satisfy the KKT conditions:
\begin{equation}
    \begin{split}
    \nabla_x L(x^*,\lambda^*,\nu^*) &= 0 \\
    Qx^* +q+A^\top \nu^*+G^\top \lambda^* &= 0 \\
    Ax^*-b &= 0 \\
    \diag (\lambda^*)(Gx^*-h) &= 0 \\
    Gx^* -h &\leq 0 \\
    \lambda^*&\geq 0.
    \end{split}
    \end{equation}
We define the function
\begin{equation}
    g(x,\lambda,\nu,\theta) = \begin{bmatrix}
    \nabla_x L({x},\lambda,\nu,\theta) \\
    \diag(\lambda)\lambda^\top (G(\theta)x-h(\theta)) \\
    A(\theta)x-b(\theta)
    \end{bmatrix},
    \end{equation}
and the optimal solution $x^*, \lambda^*, \nu^*$ satisfy the equation $g(x^*, \lambda^*, \nu^*,\theta)=0$. \\
According to the implicit function theorem, as proven in \cite{barratt2018differentiability}, the gradients where the primal variable $x$ and the dual variables $\nu$ and $\lambda$ are the optimal solution, can be formulated as 
    \begin{equation}
        J_\theta x^* = -J_x g(x^*,\lambda^*,\nu^*,\theta)^{-1} J_\theta g(x^*,\lambda^*,\nu^*,\theta),
    \label{eq:jacobian}
    \end{equation}
where, $J_x g(x^*,\lambda^*,\nu^*,\theta)$ and $J_\theta g(x^*,\lambda^*,\nu^*,\theta)$ are the Jacobian matrices. Each element of them is the partial derivative of function $g$ with respect to variable $x$ and $\theta$, respectively.
\subsection{Inference Details}
\label{sec:tarcker}
%After training, the proposed pipeline can be straightly adopted as an online tracker. Graph construction, feature encoding and graph matching procedures are exactly the same as the training stage.
Due to the continuous relaxation, the output of the QP layer may not be binary. To get a valid assignment, we use the greedy rounding strategy to generate the final permutation matrix from the predicted matching score map, i.e., we match the detection with the tracklet with the maximum score. 
After matching, like DeepSORT \cite{wojke2017simple}, we need to handle the born and death of tracklets. We keep the matching between detection and tracklet only if it satisfies all the following constraints: 1) The appearance similarity between detection and tracklet is above the threshold $\sigma$. 2) The detection is not far away from the tracklet. We set a threshold $\kappa$ as the Mahalanobis distance between the predicted distribution of the tracklet bounding box by the motion model and the detection bounding box in pixel coordinates, called the motion gate. 3) The detection bounding box overlaps with the position of tracklet predicted by the motion model. The constraints above can be written as
    \begin{equation}
        \left\{
    \begin{aligned}
    &\mathbf{B}_{i,j}>\sigma,\\ 
    &\mathtt{KF}_{i,j}>\kappa,\\ 
    &\mathtt{iou}_{i,j}>0,
\end{aligned}
        \right.
    \label{eq:cons}
\end{equation}
where, $(i.j) \in (\mathcal{V}_D,\mathcal{V}_T)$.

Here, besides the Kalman Filter adopted to estimate the geometric information in Section \ref{sec:gcn}, we apply an Enhanced Correlation Coefficient (ECC) \cite{ecc} in our motion model additionally to compensate the camera motion. Besides, we apply the IoU association between the filtered detections and the unmatched tracklets by the Hungarian algorithm to compensate for some incorrect filtering.
% \footnote{Confused, since one of the criteria is no overlap with any tracklet. How to do this association? Hungarian again?}
 Then the remaining detections are considered as a new tracklet. We delete a tracklet if it has not been updated since $\delta$ frames ago, called \emph{max age}. 
 \begin{figure}[h] 
    \centering
    \includegraphics[width=0.7\linewidth]{figures/edgematch.pdf}
    \caption{An illustration of edge matching. Here, for matched pair $(c_1^D,c_1^T)$ in $\bm{\pi}_c$, we find best matching between edge $e_{1,i'}$ and $e_{1,j'}$, drawn in the same color.}
    \vspace{-10pt}
    \label{fig:edgematch}
\end{figure}
\subsection{GST: A Practical Algorithm for Quadratic Assignment}
However, due to the process of solving the quadratic programming, the inference speed is relatively slow compared with other mainstream MOT algorithms. To speed up, we design the gated search tree (GST) algorithm. Utilizing constraints Eq.~\ref{eq:cons}, the feasible region is limited much smaller than the original quadratic programming, which greatly accelerates the process of solving the quadratic assignment problem.\par
The GST algorithm (Alg.~\ref{alg:gst}) contains three main steps. Firstly, we construct a bipartite graph $\mathcal{G}=(\mathcal{V}_D,\mathcal{V}_T,\mathcal{E}_{DT})$, in which the edges are only between tracklets and detections meeting the constraints, i.e., $\mathcal{E}_{DT}=\{(i,j)|i\in \mathcal{V}_D,j\in \mathcal{V}_T, \mathbf{B}_{i,j}>\sigma, \mathtt{KF}_{i,j}>\kappa,
 \mathtt{iou}_{i,j}>0\}$. Secondly, we use depth-first search to find all the connected components $\{\mathcal{G}_c\}=\{\mathcal{G}_c=(\mathcal{V}^D_c,\mathcal{V}^T_c,\mathcal{E}^{DT}_c)|c=(1,2,\cdots,k),|\mathcal{V}^D_1|+|\mathcal{V}^T_1|\geq|\mathcal{V}^D_2|+|\mathcal{V}^T_2|\geq\cdots\geq|\mathcal{V}^D_k|+|\mathcal{V}^T_k|\}$.
 Last, we calculate the matching cost $\mathcal{L}(\bm{\pi}_c)$ for all matching candidates in each independent connected component $\mathcal{G}_c$ parallelly.\par
The matching cost follows the objective function of the quadratic programming Eq.~\ref{finalQP}. However, as a searching algorithm, the convex relaxation in the objective function shows no advantage. So, the matching cost can be denoted back to the objective function of the original QAP, as
\begin{equation}
    \mathcal{L}(\bm{\pi})=-\bm{\pi}^\top\mathbf{M}\bm{\pi}-\mathbf{b}^\top\bm{\pi}.
    \label{costpi}
\end{equation}
To calculate the matching cost parallelly and reduce the computation in each independent connected component, we partition the quadratic affinity matrix $\mathbf{M}$ and vertex affinity matrix $\mathbf{B}$. We denote $\bm{\pi}^\top=[\bm{\pi}_c^\top,\bm{\pi}_m^\top]$, where $\bm{\pi}_c$ is in current connected component $\mathcal{G}_c$ and $\bm{\pi}_m$ is between the complement of the vertex sets, i.e., $\mathcal{G}_m=(\mathcal{V}_D\backslash \mathcal{V}^D_c,\mathcal{V}_T\backslash \mathcal{V}^T_c,\mathcal{E}^{DT}_m)$. Then, the matching cost is
\begin{equation}
\begin{split}
    \mathcal{L}(\bm{\pi}_c)=&-\bm{\pi}^\top\mathbf{M}\bm{\pi}-\mathbf{b}^\top\bm{\pi}\\
    =&-[\bm{\pi}_c^\top, {\bm{\pi}_m^{*\top}}]     
    \left[
    \begin{array}{cc}
        \mathbf{M}_c &\mathbf{M}_r\\
        \mathbf{M}_l &\mathbf{M}_m
    \end{array}
    \right]
    \begin{bmatrix}
        \bm{\pi}_c\\
        \bm{\pi}_m^*
    \end{bmatrix}\\
    &-[\mathbf{b}_c^\top, \mathbf{b}_m^\top]\begin{bmatrix}
        \bm{\pi}_c\\
        \bm{\pi}_m^*
    \end{bmatrix},\\
    =&-[\bm{\pi}_c^\top, {\bm{\pi}_m^{*\top}}]     
    \left[
    \begin{array}{c c}
        \mathbf{M}_c &\mathbf{M}_r\\
        \mathbf{M}_l &\mathbf{0}
    \end{array}
    \right]
    \begin{bmatrix}
        \bm{\pi}_c\\
        \bm{\pi}_m^*
    \end{bmatrix}-\mathbf{b}_c^\top\bm{\pi}_c,\\
    =&-\bm{\pi}_c^\top\mathbf{M}_c\bm{\pi}_c-2\bm{\pi}_m^{*\top}\mathbf{M}_l\bm{\pi}_c-\mathbf{b}_c^\top\bm{\pi}_c,\\
    =&-\bm{\pi}_c^\top\mathbf{M}_c\bm{\pi}_c+2\mathcal{L}_e(\bm{\pi}_c,{\bm{\pi}_m^{*}})-\mathbf{b}_c^\top\bm{\pi}_c,
    \label{costpic}
\end{split}
\end{equation}
where $\mathbf{M}_c\in \MBR^{|\mathcal{V}^D_c|\times|\mathcal{V}^T_c|}, \mathbf{b}_c\in\MBR^{|\mathcal{V}^D_c||\mathcal{V}^T_c|}$. Here, the matching cost contains the pairwise cost $\mathcal{L}_e(\bm{\pi}_c,{\bm{\pi}_m^{*}})$ that depends on the optimal solution $\bm{\pi}_m^{*}$, not available from connected component $\mathcal{G}_c$. To make it independent of the other components, we only consider the optimal solution $\widetilde{\bm{\pi}}_m^*$ given $\bm{\pi}_c$ instead the global optimal solution $\bm{\pi}_m^*$, i.e.,
% \begin{equation}
%     \begin{split}
% \mathcal{L}_e(\bm{\pi}_c,{\bm{\pi}_m^{*}})\approx&\mathcal{L}_e({\bm{\pi}_m^*|\bm{\pi}_c})\\
% =&\mathcal{L}_e({\bm{\pi}_m^{e*}|\bm{\pi}_c})\\
% =&-\max_{\pi_m} \bm{\pi}_m^{\top}\mathbf{M}_l\bm{\pi}_c\\
% =&-\max_{\pi_m}\sum_{i=1}^{\min\{n_d^c,n_t^c\}}\bm{\pi}_m^{\top}
% \end{split}
% \end{equation}
\begin{equation}
    \begin{split}
\mathcal{L}_e(\bm{\pi}_c,{\bm{\pi}_m^{*}})\approx&\mathcal{L}_e({\widetilde{\bm{\pi}}_m^*|\bm{\pi}_c})=-\max_{\pi_m} \bm{\pi}_m^{\top}\mathbf{M}_l\bm{\pi}_c.
\end{split}
\end{equation}
Intuitively, it is to find the best matching between the edges in the detection graph and tracklet graph with the start vertices fixed to an existing set of matches $\bm{\pi}_c$. As shown in Fig.~\ref{fig:edgematch}, for each matched pair $(c_i^D,c_j^T)$ in $\bm{\pi}_c$, we adopt bipartite matching between edge set $\{e_{i,i'}\}$ and $\{e_{j,j'}\}$ that start from $c_i^D$ and $c_j^T$ respectively.

% where $\bm{\pi}=\text{vec}(\MPI)$, $\mathbf{b}=\text{vec}(\MB)$ and $\mathbf{M}\in \MBR^{n_dn_t\times n_dn_t}$ is the symmetric quadratic affinity matrix between all the possible edges in two graphs.\par

% Besides, the relaxed continuous quadratic programming Eq.~\ref{finalQP}
% .
% , the optimal solution of the QP (Eq.~\ref{finalQP}) does not equal to the original QAP (Eq.~\ref{equ:KBQAP}). As a searching algorithm, the relaxation shows no advantage. So, we define the objective function back to the original QAP as
% \begin{equation}
%     \bm{\pi}^*=\underset{\bm{\pi}}{\arg\max} \ 
%     \bm{\pi}^\top\mathbf{M}\bm{\pi}+\mathbf{b}^\top\bm{\pi},
%     \label{edge2}
%     \end{equation}
% where $\bm{\pi}=\text{vec}(\MPI)$, $\mathbf{b}=\text{vec}(\MB)$ and $\mathbf{M}\in \MBR^{n^2\times n^2}$ is the symmetric quadratic affinity matrix between all the possible edges in two graphs.\par






\begin{algorithm}[ht!]
\caption{Gated Search Tree (GST)}
    \label{alg:gst}
  
    \KwIn{$\mathbf{M}, \mathbf{B},\mathtt{iou}, \mathtt{KF},\sigma,\kappa$}
    \KwOut{$\mathbf{\Pi}$}
    {\small //\ Construct\ graph}\\

    \For{$(i.j) \in \mathtt{range}(n_d,n_t)$}{
    $\mathbf{A}_{i.j}\leftarrow\mathbb{I}\{{\mathtt{iou}_{i,j}>0 \land \mathtt{KF}_{i,j}>\kappa \land \mathbf{B}_{i,j}>\sigma}\}$}

    % $\mathcal{G}(\mathbf{A}) \leftarrow \mathbf{A}_{i.j}$\\
    {\small //\ Find Independent Connected Components (Alg.~\ref{alg:FCS})}\\
    $\{\mathcal{G}_k\}\leftarrow\mathtt{FICC(\mathcal{G}(\mathbf{A}))}$\\
    {\small //\ Find Best Matching}\\
    \For{$\mathcal{G}_c \in \{\mathcal{G}_k\}$}{
    $\mathbf{M}_c, \mathbf{b}_c = \mathbf{M}[\{c\},\{c\}], \mathtt{vec}(\mathbf{B}[\{c\},\{c\}])$\\
    $\mathcal{L}(\bm{\pi}_c)=-\bm{\pi}_c^\top\mathbf{M}_c\bm{\pi}_c+2\mathcal{L}_e({\widetilde{\bm{\pi}}_m^*|\bm{\pi}_c})-\mathbf{b}_c^\top\bm{\pi}_c,$\\
    $\bm{\pi}_c^*\leftarrow\arg\min_{\bm{\pi}_c}\mathcal{L}(\bm{\pi}_c)$
    }
    $\mathbf{\Pi}\leftarrow \bigcup_{c=1}^{k} \bm{\pi}_c^*$\\
    \Return{$\mathbf{\Pi}$}
    

\end{algorithm}
\begin{algorithm}[]
    \caption{Find Independent Connected Components (FICC)}
        \label{alg:FCS}
        % \SetAlgoLined
        \SetKwProg{Fn}{def}{:}{end}
        \KwIn{$\mathcal{G}=(\mathcal{V}_D,\mathcal{V}_T,\mathcal{E}_{DT})$}
        \KwOut{$\{\mathcal{G}_c\}$}
        $c\leftarrow 0$; $\{\mathcal{G}_c\}\leftarrow\varnothing$\\
        \For{$v_p\in \mathcal{V}_D\cup\mathcal{V}_T$}{
            $\mathtt{visited}[v_p]\leftarrow\mathtt{False}$
        }

        \For{$v_p \in \mathcal{V}_D\cup\mathcal{V}_T$}{
            \If{$\neg\mathtt{visited}[v_p]$}{
                $\mathcal{E}_c^{DT},\mathcal{V}_c^D,\mathcal{V}_c^T \leftarrow \varnothing$\\
                \If {$v_p\in \mathcal{V}_D$}{
                $\mathcal{V}_c^D \leftarrow \mathcal{V}_c^D \cup\{v_p\}$}
                \ElseIf{$v_p\in \mathcal{V}_T$}{$\mathcal{V}_c^T \leftarrow \mathcal{V}_c^T \cup\{v_p\}$}
                $\mathtt{visit}(v_p)$\\
                $c \leftarrow c+1$}
        }
        \Fn{$\mathtt{visit}(v_p)$}{
            $\mathtt{visited}[v_p]\leftarrow\mathtt{True}$\\
            \For{$e_{p,q}\in \mathcal{E}_{DT}$}{
                $\mathcal{E}_c^{DT} \leftarrow \mathcal{E}_c^{DT}\cup\{e_{p,q}\}$\\
                \If{$\neg\mathtt{visited}[v_q]$}{
                    $\mathtt{visit}(v_q)$\\
                    \If {$v_q\in \mathcal{V}_D$}{
                $\mathcal{V}_c^D \leftarrow \mathcal{V}_c^D \cup\{v_q\}$}
                \ElseIf{$v_i\in \mathcal{V}_T$}{$\mathcal{V}_c^T \leftarrow \mathcal{V}_c^T \cup\{v_q\}$}
                }
            }


        }
        \Return{$\{\mathbf{G_c}\}$}
        
\end{algorithm}

% \begin{algorithm}[ht!]
% \caption{Maximum Weighted Independent Set (MWIS)}
%     \label{alg:MWIS}
  
%     \KwIn{$\mathbf{G}$}
%     \KwOut{$\mathbf{\Pi}$}
 
%     \Return{$\mathbf{\Pi}$}
    

% \end{algorithm}
Then, we discuss the time cost of the original QP solver and the GST algorithm:
\begin{proposition}[Original complexity]
    The quadratic programming Eq.~\ref{finalQP} can be solved in $O(n_d^3n_t^3)$ arithmetic operations.
\end{proposition}
% \begin{proof}
%     The quadratic programming is proved~\cite{ye1989extension} that the optimal solution can be derived in $O(n^3)$ arithmetic operations, where $n$ is the lenth of variable vector. In Eq.~\ref{finalQP}, $\mathbf{x} \in \MBR^{n_dn_t}$, so the Eq.~\ref{finalQP} can be solved in $O(n_d^3n_t^3)$ arithmetic operations. 
% \end{proof}
% \begin{proposition}[Complexity of GST]
%     The time complexity of GST algorithm is 
%     \begin{equation}
%     O(2^{k_m}n_mk_dk_t),
%     \end{equation}
% where $n_m=\min\{n_d,n_t\}, k_d=|U_k|, k_t=|V_k|,k_m=\min\{k_d,k_t\}$.
% \end{proposition}
% \begin{proof}
% \textcolor{red}{this is the proof.}
% \end{proof}
\begin{proposition}
    The running time of Algorithm~\ref{alg:gst} in parallel mode is
    \begin{equation}
        T = c\cdot n_m|\mathcal{V}^T_1||\mathcal{V}^D_1|+\epsilon,
\end{equation}
where $c$ is a constant factor, $\epsilon$ represents low-order terms of $n$ and communication overhead between threads, $n_m=\max\{n_d,n_t\}$.
\end{proposition}
% \begin{proof}
%     \textcolor{red}{this is the proof.}
%     \end{proof}
\begin{figure*}[!h] 
    \centering
    \vspace{-10pt}
    \includegraphics[width=\linewidth]{figures/gmatcher.pdf}
    \vspace{-10pt}
    \caption{Pipeline of our image matching network, GMatcher. The backbone is an FPN-like module. The edge and vertex features are from the stride-8 and stride-2 feature maps respectively. Edge and vertex AGNN are operated independently. The learnable graph matching layer replaces the Sinkhorn layer in SuperGlue.}
    \vspace{-10pt}
    \label{fig:gmatcher}
\end{figure*}
\section{Learnable Graph Matching for Image Matching Task}
Besides the MOT task, our learnable graph matching method can be easily adapted to other data association tasks with slight modifications. In this section, we take the image matching task as an example. We formulate the image matching task as a graph matching problem between the keypoints in two images and utilize our learnable graph matching algorithm to build an end-to-end keypoint-based neural network.
\subsection{Problem Formulation}
Given keypoints $\mathcal{P}=\{\mathbf{p}_1,\mathbf{p}_2,\cdots,\mathbf{p}_m\}$, $\mathcal{P}'=\{\mathbf{p}'_1,\mathbf{p}'_2,\cdots,\mathbf{p}'_n\}$ on the image $I$ and $I'$ of the same scene respectively, where $\mathbf{p}_i=(x_i,y_i)$ is the keypoint position in image coordinates, the image matching task is to find the best matching between $\mathcal{P}$ and $\mathcal{P}'$ and thus estimate the relative camera pose $\mathbf{T}\in SE(3)$. 
The keypoints are often on the corners, textured areas, or the boundary of the objects, where the local features are relatively robust and less affected by illumination and viewing angle.
They can be derived from traditional methods, like SIFT~\cite{lowe2004distinctive}, or deep learning-based methods, like SuperPoint~\cite{detone2018superpoint}. And the descriptor $\mathbf{d}_i \in \mathbb{R}^c$ is a $c$-dimentional local discriminative feature, corresponding to the keypoint $\mathbf{p}_i$. In this paper, we extract the features in an end-to-end way, with only the positions of the keypoints from the off-the-shelf neural network. \par
In our end-to-end graph matching neural network, called \emph{GMatcher}, we take two images and the keypoints on each image as the input, solving the graph matching problem (Eq.~\ref{finalQP}) from $\mathcal{P}$ and $\mathcal{P}'$, and we finally obtain the assignment matrix $\mathbf{\Pi} \in \mathbb{R}^{m\times n}$ to represent the matching between two keypoint sets. 
\subsection{End-to-end Graph Matching Network for Image Matching}
Our method is mainly based on SuperGlue~\cite{sarlin2020superglue}, which utilizes the attentional graph neural network (AGNN) module to aggregate long-range dependencies. However, compared with graph matching, although SuperGlue stacks self- and cross- attention modules many times to fuse the intra- and inter-image information, it does not explicitly define the intra-image graph and consider the edge similarities between images in keypoint matching. \par
To better utilize the high-order information, i.e., edge in the intra-image graph, we use FPN-like~\cite{lin2017feature} backbone to extract multiscale features. The stride-2 feature map and stride-8 feature map are used to extract vertex feature $\mathbf{d}_i^v$ and edge feature $\mathbf{d}_{i,j}^e$ respectively. Note that the edge feature $\mathbf{d}_{i,j}^e$ is the concatenation of the endpoints' feature $\mathbf{d}_i^e$ and $\mathbf{d}_j^e$ on stride-8 feature map. The feature $\mathbf{d}_i$ on each keypoint $\mathbf{p}_i$ is extracted from the feature map that is restored to the original image resolution using bilinear interpolation.\par
Like the position embedding in the transformer, we use MLP to encode the position information into the vertex feature $\mathbf{d}_i^v$ and edge feature $\mathbf{d}_{i,j}^e$, i.e.,
\begin{equation}
    \begin{aligned}
    \label{eq:keypoint-encoder}
    \mathbf{f}_i^v &= \mathbf{d}_i^v + \text{MLP}_{\text{pos}}(\mathbf{p}_i),\\
    \mathbf{f}_{i,j}^e &= \mathbf{d}_{i,j}^e + [\text{MLP}_{\text{pos}}(\mathbf{p}_i),\text{MLP}_{\text{pos}}(\mathbf{p}_j)],
    \end{aligned}
\end{equation}
where, $[\cdot,\cdot]$ denotes concatenation.\par
Then, similar to the AGNN module in SuperGlue, we conduct self-attentional and cross-attentional message passing for $l$ times in vertex features and edge features separately. The detailed design can be referred to SuperGlue~\cite{sarlin2020superglue}. We use the output vertex features and edge features of AGNN to construct the final graphs $\mathcal{G}_1=(\mathcal{V}=\{\mathbf{d}_i^{v(l)}\},\mathcal{E}=\{\mathbf{d}_{i,j}^{e(l)}\}))$ and $\mathcal{G}_2=(\mathcal{V}=\{\mathbf{d}_{i'}^{v(l)}\},\mathcal{E}=\{\mathbf{d}_{i',j'}^{e(l)}\}))$ for two images and matching with our differentiable graph matching layer mentioned in Sec.~\ref{sec:diffgm}.
   
We present in section~\ref{ssec:faces} an application of PnP-HVAE on face images, using a pretrained state-of-the-art hierarchical VAE. 
Next, we study the application of our framework to natural images. To that end, we introduce  in section~\ref{ssec:patchVDVAE}  a patch hierachical VAE architecture, that is able to model natural images of different resolutions. In section~\ref{ssec:app_nat}, we provide deblurring, super-resolution and inpainting experiments to demonstrate the relevance of the proposed method.

Additional results are presented in Appendix~\ref{app:add}. All experiments can be reproduced using the code available at \url{https://github.com/jprost76/PnP-HVAE}.



\subsection{Face Image restoration (FFHQ)}\label{ssec:faces}
We first demonstrate the effectiveness of PnP-HVAE on highly structured data, by performing face image restoration.
Latent variable generative models can accurately model structured images such as face images \cite{karras2019style,vahdat2020nvae,child2021very,kingma2018glow}, and then be used to produce high quality restoration of such data. 
In our experiments, we use the VDVAE model of~\cite{child2021very}, pre-trained on the FFHQ dataset~\cite{karras2019style}, as our hierarchical VAE prior.
VDVAE has $L=66$ latent variable groups in its hierarchy and generates images at resolution $256\times256$.

We compare PnP-HVAE with the intermediate layer optimization algorithm (ILO)~\cite{daras2021intermediate} that is based on a different class of generative models than HVAE. ILO is a GAN inversion method which optimizes the image latent code along with the intermediate layer representation of a StyleGAN to generate an image consistent with a degraded observation.
We use the official implementation of ILO, along with a StyleGAN2 model~\cite{karras2020analyzing, stylegan2pytorch}, that was trained for 550k iterations on images of resolution $256\times256$ from FFHQ.  
As VDVAE and StyleGAN models are not trained on the same train-test split of FFHQ, we chose to evaluate the methods on a subset of 100 images from the CelebA dataset~\cite{liu2018large}. 
For super-resolution, the degradation model corresponds to the application of a gaussian low-pass filter followed by a $\times 4$ sub-sampling, and the addition of a gaussian white noise with $\sigma=3$.
For the deblurring, we considered motion blur and  gaussian kernels, both with a noise level $\sigma=8$. %

We provide quantitative comparisons in table~\ref{table:comp_ILO}, along with a visual comparison of the results in figure~\ref{fig:face_restoration}.
PnP-HVAE has the best  PSNR and SSIM results for all the considered restoration tasks, while ILO provides better results  for the perceptual distance.
By jointly optimizing the image and its latent variable, PnP-HVAE provides  results that are both realistic and consistent with the degraded observation.
On the other hand,  ILO  only optimizes on an extended latent space. This method generates  sharp and realistic images with better LPIPS scores,   
but the results lack  of consistency with respect to the observation, which explains the overall lower PSNR performance. 






\subsection{PatchVDVAE: a HVAE for natural images}\label{ssec:patchVDVAE}
Available generative models in the literature operate on images of  fixed resolutions and
are either restrained to datasets of limited diversity, or even to registered face images~\cite{kingma2018glow,child2021very, vahdat2020nvae, karras2019style}, or requiring additional class information~\cite{brock2018large, dhariwal2021diffusion, song2020score, luhman2022optimizing}.
Fitting an unconditional model on natural images appears to be a more difficult task, as their resolution can change, and their content is highly diverse.
The complexity of the problem can be reduced by learning a prior model on patches of reduced dimension. 
For image restoration problems, the patch model can be reused on images of higher dimensions~\cite{zoran2011learning,prost2021learning,altekruger2022patchnr}. When the model is a full CNN, the prior on the set of the  patches can  be computed efficiently by applying the network on the full image~\cite{prost2021learning}.

We thus introduce  patchVDVAE, a fully convolutional hierarchical VAE.
Contrary to existing HVAE models whose resolution is constrained by the constant tensor at the input of the top-down block, patchVDVAE can generate images of different resolutions by controlling the dimension of the input latent. 
This amounts to defining a prior on patches whose dimension corresponds to the receptive field of the VAE. A similar model is used for image denoising in~\cite{prakash2021interpretable}.

 
For PatchVDVAE architecture, we use the same bottom-up and top-down blocks as VDVAE~\cite{child2021very}, and replace the constant trainable input in the first top-down block by a latent variable, to make the model fully convolutional (details on the  architecture are given in Appendix~\ref{app:details}). 
The training dataset is composed of $128\times 128$ patches extracted from a combination of DIV2K~\cite{agustsson2017ntire} and Flickr2K~\cite{Lim_2017_CVPR_workshops} datasets.
We perform data augmentation by extracting  patches at $3$ resolutions: HR-images and $\times 2$ and $\times 4$ downscaled images. 
The model is trained for $7.10^5$ iterations with a batch size of $64$. Following the recommendation of~\cite{hazami2022efficient}, we use Adamax optimizer with an exponential moving average and gradient smoothing of the variance.
We set the decoder model to be a gaussian with diagonal covariance, as in~\cite{luhman2022optimizing}.
PatchVDVAE is fully convolutional and can generate images of dimension that are multiples of $64$ as illustrated by
figure~\ref{fig:vdvae}.

\newlength{\patchwidth}
\setlength{\patchwidth}{0.135\columnwidth}
\begin{figure}[!ht]
    \centering
    \begin{subfigure}[t]{.34\columnwidth}\hspace{0.1cm}
        \setlength{\tabcolsep}{0.02pt}
\renewcommand{\arraystretch}{0}
        \begin{tabular}{*{2}{p{1.03\patchwidth}}}
            \includegraphics[width=\patchwidth]{figures_arxiv/patchVDVAE/samples/generated/64x64/setup-5-image-0018.png} &
            \includegraphics[width=\patchwidth]{figures_arxiv/patchVDVAE/samples/generated/64x64/setup-5-image-0016.png} \\
            \includegraphics[width=\patchwidth]{figures_arxiv/patchVDVAE/samples/generated/64x64/setup-5-image-0008.png} &
            \includegraphics[width=\patchwidth]{figures_arxiv/patchVDVAE/samples/generated/64x64/setup-5-image-0019.png}   
        \end{tabular}
    \end{subfigure}\hspace{-0.15cm}
    \begin{subfigure}[t]{.64\columnwidth}
\begin{tabular}{cc}\vspace{-0.1cm}
\includegraphics[width=2\patchwidth]{figures_arxiv/patchVDVAE/samples/generated/256x256/setup-2-image-0009.png}&
        \includegraphics[width=2\patchwidth]{figures_arxiv/patchVDVAE/samples/generated/256x256/setup-2-image-0002.png}\end{tabular}

    \end{subfigure}
    \caption{\label{fig:vdvae} Left: $64\times64$ patches samples from our patchVDVAE model trained on patches from natural images.
    Right: PatchVDVAE is fully convolutional and it can generate images of higher resolution (here: $128\times128$).\vspace{-0.2cm}}
\end{figure}

\subsection{Natural images restoration}\label{ssec:app_nat}
We  evaluate PnP-HVAE on natural image restoration.
For each task, we report the average value of the PSNR, the SSIM, and the LPIPS metrics on $20$ images from the test set of the BSD dataset~\cite{MartinFTM01}.\\


\noindent
{\bf Image deblurring.}
In the experiments, we consider $2$ gaussian kernels and $2$ motion blur kernels from~\cite{levin2009understanding}, with $3$ different noise levels 
$\sigma \in \{2.55, 7.65, 12.75\}$.
As a baseline we consider  EPLL~\cite{zoran2011learning}, which learns a prior on image patches with a gaussian mixture model.
We also compare PnP-HVAE  with PnP-MMO and GS-PnP, $2$ competing convergent Plug-and-Play methods based on CNN denoisers.
PnP-MMO~\cite{pesquet2021learning} restricts the denoiser to be contraction in order to guarantee the convergence of the PnP forward-backard algorithm. GS-PnP~\cite{hurault2022gradient} considers a gradient step denoiser and reaches state-of-the-art performances of non converging methods~\cite{zhang2021plug}.
We set the temperature $\tau$  in our method as $0.95$, $0.8$ and $0.6$ for noise levels $2.55$, $7.65$ and $12.75$ respectively, and we let it run for a maximum of $50$ iterations. 
For the three compared methods we use the official implementations and pre-trained models provided by the respective authors. 
Details on the choice of hyperparameters for the concurrent methods are provided in the Appendix~\ref{app:details}
Figure~\ref{fig:deblurring_bsd} illustrates that our method provides correct deblurring results. 

According to table~\ref{tab:deb}, the performance of PnP-HVAE is between those of EPLL and GS-PnP and it outperforms PnP-MMO for large noise levels.\\

\begin{table}
\begin{center}\footnotesize
    \begin{tabular}{>{\centering}m{.3cm}*{5}{c}}
    $\sigma$ &Method & PSNR$\uparrow$ & SSIM$\uparrow$ & LPIPS$\downarrow$  \\ 
    \hline
    \multirow{4}{*}{\vcell{$2.55$}}
    & PnP-HVAE & $27.75$ & $0.79$ & $0.31$\\
    & GS-PNP \cite{hurault2022gradient} & $\mathbf{29.59}$ & $\mathbf{0.84}$ & $\mathbf{0.22}$\\
    & EPLL \cite{zoran2011learning} & $26.49$ & $0.71$ & $0.36$\\ 
    & PnP-MMO \cite{pesquet2021learning} & $\underbar{29.50}$ & $\underbar{0.83}$ & $\underbar{0.20}$ \\ \hline
    \multirow{4}{*}{\vcell{$7.65$}}
    & PnP-HVAE & $\underbar{26.36}$ & $\underbar{0.72}$ & $\underbar{0.40}$\\
    & GS-PNP \cite{hurault2022gradient} & $\mathbf{27.33}$ & $\mathbf{0.77}$ & $\mathbf{0.31}$\\
    & EPLL \cite{zoran2011learning} & $24.04$ & $0.66$ & $0.45$ \\ 
    & PnP-MMO \cite{pesquet2021learning} & $25.34$ & $0.69$ & $0.34$\\
    \hline
    \multirow{4}{*}{\vcell{$12.75$}}
    & PnP-HVAE & $\underbar{25.12}$ & $\mathbf{0.73}$ & $\underbar{0.47}$\\
    & GS-PNP \cite{hurault2022gradient} & $\mathbf{26.32}$ & $\mathbf{0.73}$ & $\mathbf{0.37}$\\
    & EPLL \cite{zoran2011learning} & $23.28$ & $0.61$ & $0.51$ \\ 
    & PnP-MMO \cite{pesquet2021learning} & $22.42$ & $0.53$& $0.54$ \\
    \hline
    &\vspace*{-.3cm}\\
            \multicolumn{2}{c}{Blur and motion kernels}& \multicolumn{3}{c}{
        \includegraphics*[scale=1]{figures_arxiv/kernels/4.png}\;\includegraphics*[scale=1]{figures_arxiv/kernels/7.png}\;\includegraphics*[scale=1]{figures_arxiv/kernels/9.png}\;\includegraphics*[scale=1]{figures_arxiv/kernels/11.png}} 
    \end{tabular}
        \caption{\label{tab:deb}Comparison  of PnP-HVAE  and other restoration methods on deblurring. Results are averaged on $4$ kernels.\vspace{-0.2cm}}% on image deblurring.}
    \end{center}
\end{table}

\begin{figure}
    
    \begin{subfigure}[h]{\linewidth}
        \centering
        \includegraphics*[width=\columnwidth]{figures_arxiv/deb_s255_k7.pdf}\vspace{-0.1cm}
        \caption{Gaussian blur, $\sigma=2.55$}
    \end{subfigure}
    \begin{subfigure}[h]{\linewidth}
        \centering
        \includegraphics*[width=\columnwidth]{figures_arxiv/deb_s765_k11.pdf}\vspace{-0.1cm}
        \caption{Motion blur, $\sigma=7.65$}
    \end{subfigure}\vspace*{-0.1cm}
    \caption{\label{fig:deblurring_bsd} Natural image deblurring\vspace{-0.1cm}}
\end{figure}

\noindent {\bf Effect of the temperature.}
PnP-HVAE gives control on the temperature of the prior over the latent space.
In figure~\ref{fig:temp_effect}, we illustrate that reducing the temperature increases the strength of the regularization prior. In this example the tuning $\tau=0.7$ produces the best performance.\\
\begin{figure}[!ht]
   
    \includegraphics[width=\columnwidth]{figures_arxiv/demo_temp.pdf}\vspace{-0.15cm}
    \caption{ \label{fig:temp_effect} Effect of the temperature in PnP-VAE on a deblurring problem, with $\sigma=7.65$.\vspace{-0.15cm}}
\end{figure}


\noindent
{\bf Image inpainting.}
Next we consider the task of noisy image inpainting. 
We compose a test-set of 10 images from the validation set of BSD~\cite{MartinFTM01} and we create masks
  by occluding diverse objects of small size in the images. 
A gaussian white noise with $\sigma=3$ is added to the images.
As a comparaison, we still consider GS-PnP and EPLL.
For PnP-HVAE, the temperature is set to $\tau=0.6$, and the algorithm is run for a maximum of $200$ iterations, unless the residual $||\x_{k+1}-\x_k||$ is on a plateau.
We provide on Table~\ref{tab:inpainting_bsd} the distortion metrics with the ground truth, as well as a visual
\begin{table}



\begin{center}
    \begin{tabular}{cccc}
        & PSNR$\uparrow$ & SSIM$\uparrow$ &LPIPS$\downarrow$ \\\hline
        PnP-HVAE  & $\mathbf{29.54}$ & $\mathbf{0.93}$ & $\mathbf{0.06}$\\
        GS-PNP & $28.52$ & $\mathbf{0.93}$ & $0.09$\\
        EPLL & $\underline{29.16}$ & $\mathbf{0.93}$ & $\mathbf{0.06}$\\
    \end{tabular}
    \caption{\label{tab:inpainting_bsd}Quantitative evaluation for inpainting on BSD.}
    \end{center}
\end{table}
comparison on figure~\ref{fig:inpainting_bsd}. 
With its hierarchical structure,  PnP-HVAE outperforms the compared methods. \vspace{0.05cm}



\begin{figure}[!h]
    \includegraphics[width=\columnwidth]{figures_arxiv/demo_inp_bsd2.pdf}\vspace{-0.1cm}
    \caption{\label{fig:inpainting_bsd}Natural image inpainting\vspace{-0.3cm}}
\end{figure}












% needed in second column of first page if using \IEEEpubid
%\IEEEpubidadjcol

\bibliographystyle{IEEEtran}
\bibliography{egbib}
% argument is your BibTeX string definitions and bibliography database(s)
%\bibliography{IEEEabrv,../bib/paper}
%
% <OR> manually copy in the resultant .bbl file
% set second argument of \begin to the number of references
% (used to reserve space for the reference number labels box)
% \begin{thebibliography}{1}

% \bibitem{IEEEhowto:kopka}
% H.~Kopka and P.~W. Daly, \emph{A Guide to \LaTeX}, 3rd~ed.\hskip 1em plus
%   0.5em minus 0.4em\relax Harlow, England: Addison-Wesley, 1999.

% \end{thebibliography}

% biography section
% 
% If you have an EPS/PDF photo (graphicx package needed) extra braces are
% needed around the contents of the optional argument to biography to prevent
% the LaTeX parser from getting confused when it sees the complicated
% \includegraphics command within an optional argument. (You could create
% your own custom macro containing the \includegraphics command to make things
% simpler here.)
%\begin{IEEEbiography}[{\includegraphics[width=1in,height=1.25in,clip,keepaspectratio]{mshell}}]{Michael Shell}
% or if you just want to reserve a space for a photo:

% \begin{IEEEbiography}{Michael Shell}
% Biography text here.
% \end{IEEEbiography}

% % if you will not have a photo at all:
% \begin{IEEEbiographynophoto}{John Doe}
% Biography text here.
% \end{IEEEbiographynophoto}

% % insert where needed to balance the two columns on the last page with
% % biographies
% %\newpage

% \begin{IEEEbiographynophoto}{Jane Doe}
% Biography text here.
% \end{IEEEbiographynophoto}

% You can push biographies down or up by placing
% a \vfill before or after them. The appropriate
% use of \vfill depends on what kind of text is
% on the last page and whether or not the columns
% are being equalized.

%\vfill

% Can be used to pull up biographies so that the bottom of the last one
% is flush with the other column.
%\enlargethispage{-5in}


\begin{IEEEbiography}[{\includegraphics[width=1in,height=1.25in,clip,keepaspectratio]{photos/jiawei.jpg}}]{Jiawei He}
  is a PhD student in BRAVE group of Center for Research on Intelligent Perception and Computing (CRIPAC), the National Laboratory of Pattern Recognition (NLPR), Institute of Automation, Chinese Academy of Sciences, Beijing, China, supervised by Prof. Zhaoxiang Zhang. Before this, he got his BS degree in automation from Xi'an Jiaotong University, China, in 2019. His research interests are in Computer Vision, Deep Learning, Learning-based Combinatorial Optimization, including video analysis, graph matching, multiple object tracking, 3D perception, etc.
\end{IEEEbiography}
\begin{IEEEbiography}[{\includegraphics[width=1in,height=1.25in,clip,keepaspectratio]{photos/zehao.png}}]{Zehao Huang}
  received the BS degree in automatic control from Beihang University, Beijing,
  China, in 2015. He is currently an algorithm engineer at TuSimple. His research interests include computer vision and image processing.
\end{IEEEbiography}
\begin{IEEEbiography}[{\includegraphics[width=1in,height=1.25in,clip,keepaspectratio]{photos/winsty.jpg}}]{Naiyan Wang}
  is currently the chief scientist of TuSimple. he leads the algorithm research group
  in the Beijing branch. Before this, he got his PhD degree from CSE department, HongKong University of Science and Technology in 2015. His supervisor is Prof. Dit-Yan Yeung. He got his BS degree from Zhejiang University, 2011 under the
  supervision of Prof. Zhihua Zhang. His research interest focuses on applying statistical computational model to real problems in computer vision
  and data mining. Currently, He mainly works on the vision based perception and localization part of autonomous driving. Especially He integrates and improves the cutting-edge technologies in academia, and makes them work properly in the autonomous truck.
\end{IEEEbiography}
\begin{IEEEbiography}[{\includegraphics[width=1in,height=1.25in,clip,keepaspectratio]{photos/zzx.png}}]{Zhaoxiang Zhang}
received the bachelor's degree in circuits and systems
  from the University of Science and Technology
  of China (USTC) in 2004 and the Ph.D. degree
  from the National Laboratory of Pattern Recognition
  (NLPR), Institute of Automation, Chinese Academy of Sciences (CASIA), Beijing, China, in 2009.
  In October 2009, he joined the School of Computer
  Science and Engineering, Beihang University, and
  worked as an Assistant Professor from 2009 to 2011,
  an Associate Professor from 2012 to 2015, and
  the Vice-Director of the Department of Computer Application Technology
  from 2014 to 2015. In July 2015, he returned to the CASIA, to join as a
  Professor, where he is currently a Professor with the Center for Research on
  Intelligent Perception and Computing. He has published more than 200 papers
  in reputable conferences and journals. His major research interests include
  pattern recognition, computer vision, machine learning, and bio-inspired visual
  computing. He has won the best paper awards in several conferences and
  championships in international competitions. He has served as the Area Chair
  and a Senior PC for many international conferences, such as CVPR, ICCV,
  AAAI, and IJCAI. He has served or is serving as an Associate Editor for IEEE
  TRANSACTIONS ON CIRCUITS AND SYSTEMS FOR VIDEO TECHNOLOGY,
  Pattern Recognition, and Neurocomputing.
\end{IEEEbiography}
 

% that's all folks
\end{document}


