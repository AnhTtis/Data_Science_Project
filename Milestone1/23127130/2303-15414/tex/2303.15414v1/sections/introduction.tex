% \vspace{-10pt}
\IEEEraisesectionheading{\section{Introduction}\label{sec:introduction}}
% \begin{comment}
    \begin{figure*}[t]
        \centering
        \subfloat[Bipartite Matching~\cite{kuhn1955hungarian}]{
            \includegraphics[width=0.4\linewidth]{figures/fig1-1.pdf}}
            \quad
            \subfloat[Graph Matching]{
            \includegraphics[width=0.4\linewidth]{figures/fig1-2.pdf}}
        \caption{An illustration of intra-graph relationship used in our graph matching formulation. We utilize the second-order edge to model the pairwise relationship, which is more robust in challenging scenes, such as heavy occluded in MOT task. For example, in view 2, the entity with ID 1 can not be associated with the entity correctly in view 1. However, with graph matching, the pairwise relationship helps data association.}
        \label{fig1}
    \end{figure*}
% \end{comment}
\IEEEPARstart{D}{ata} association is at the core of many computer vision tasks, for example, instances with the same identity are associated between different frames in \emph{Multiple Object Tracking}, 2D or 3D keypoints are associated between different views for \emph{Image Matching} and \emph{Point Cloud Registration}. These tasks can be described as detecting entities (objects/keypoints/points) in different views, then establishing the correspondences between entities in these views. In this paradigm, the latter process is called \emph{Data Association} and becomes the important part of these tasks. The traditional methods define data association task as a bipartite matching problem, ignoring the context information in each view, i.e., the pairwise relationship between entities. 
% However, utilizing the context information in the view is necessary for data association and one appropriate formulation is to construct a graph in each view, where traditional individual information can be placed on the vertex of the graph and the context information can be considered as the edge feature. Thus, graph matching becomes a universal formulation for data association.\par
In this paper, we argue that the relationship between the entities within the same view is also crucial for some challenging cases in data association. 
% For example, we can match an occluded object to the correct tracklet solely by the past relationships with neighborhood objects. 
% Fig.~\ref{fig1} just shows such an example.
Interestingly, these pairwise relationships within the same view can be represented as edges in a general graph.
To this end, the popular bipartite matching across views can be updated to general graph matching between them. 
To further integrate this novel assignment formulation with powerful feature learning, we first relax the original formulation of graph matching \cite{LawlerMS63, kbqap} to a quadratic programming, and then derive a differentiable QP layer based on the KKT conditions and the implicit function theorem for the graph matching problem, inspired by the OptNet \cite{amos2017optnet}. Finally, the assignment problem can be learned in synergy with the features. In Fig.~\ref{fig1}, we show how pairwise relationship is used in our learnable graph matching method.\par %Not surprisingly, our \gm\ achieves state-of-the-art performance on the publicly available MOT datasets. 

To reveal the effectiveness and universality of our learnable graph matching method, we apply our method to some important and popular compuer vision tasks, \emph{Multiple Object Tracking} (MOT) and \emph{Image Matching}.
MOT is a fundamental computer vision task that aims at associating the same object across successive frames in a video clip. A robust and accurate MOT algorithm is indispensable in broad applications, such as autonomous driving and video surveillance.
The \textit{tracking-by-detection} is currently the dominant paradigm in MOT. This paradigm consists of two steps: (1) obtaining the bounding boxes of objects by detection frame by frame; (2) generating trajectories by associating the same objects between frames. With the rapid development of deep learning based object detectors, the first step is largely solved by the powerful detectors such as~\cite{lin2017feature, lin2017focal}. 
As for the second one, recent MOT work focuses on improving the performance of data association mainly from the following two aspects: (1) formulating the association problem as a combinatorial graph partitioning problem and solve it by advanced optimization techniques \cite{berclaz2011multiple, bewley2016simple, wang2017learning, braso2020learning, xu2020train, hornakova2020lifted}; (2) improving the appearance models by the power of deep learning\cite{kuo2011does, yang2012online, leal2016learning, wojke2017simple}. Although very recently, some work~\cite{zhu2018online, sun2019deep, braso2020learning, xu2020train} trying to unify feature learning and data association into an end-to-end trained neural network, these two directions are almost isolated so that these recent attempts hardly utilize the progress from the combinatorial graph partitioning.

\begin{comment}
\textcolor{red}{In the graph view of MOT, each vertex represents a detection bounding box or a tracklet, while the edges are constructed between the vertices across different frames to represent the similarities between them. Then the association problem can be formulated as a min-cost flow problem \cite{zhang2008global}. %The cost assigned to an edge represents the distance between objects, which can be defined as the similarity of appearance features of two objects \cite{wojke2017simple} or the intersection-over-union (IOU) between two bounding-boxes \cite{bewley2016simple}. 
The most popular used method is to construct a bipartite graph between two frames and adopt Hungarian algorithm \cite{kuhn1955hungarian} to solve it. This online strategy is widely used in practice because of its simplicity \cite{bewley2016simple, wojke2017simple}. 
Nevertheless, these methods are not robust to occlusions due to the lack of history trajectories and long term memory. 
This problem is traditionally solved by constructing graph from multiple frames, and then deriving the best association based on the optimal solution of this min-cost flow problem~\cite{zhang2008global}.}
\end{comment}
In this paper, we propose a learnable graph matching based online tracker, called GMTracker. We construct the tracklet graph and the detection graph, and put the detections and tracklets on the vertices respectively. The edge features on each graph represent the pairwise relationship between two connected vertices. According to the general design of the learnable graph matching module mentioned above, the learnable vertex and edge features on the two graphs are the input of the differentiable GM layer. The supervision acts on these features constrained by graph matching solutions. So the learning process converges to a more robust and reasonable solution than the traditional approach.\par
%Recently, MPNTrack \cite{braso2020learning} proposed to partition the graph into trajectories by supervised learning directly. %Message Passing Network (MPN) \cite{gilmer2017neural} is adopted to aggregate features across the graph. 
%With the help of supervised learning, the proposed method yields significant improvement on MOT benchmarks \cite{leal2015motchallenge, milan2016mot16}. %However, it is an offline method and does not utilize the relationship between trajectories since there are not connections among nodes in the same frame in the whole graph.
% All existing work focuses on finding the best matching across frames, but ignoring the context within the frame.
However, when the quantity of the objects increases, the graph matching method is not efficient enough. The time cost will become unbearable. So, we speed up the graph matching solver by introducing the gate mechanism, called Gated Search Tree (GST) for MOT. The feasible region is limited much smaller than the original quadratic programming formulation, so that the running time of solving the graph matching problem can be substantially reduced.\par
% these informations between nodes in the same frame are important for data association.  Based on this motivation, we propose a novel online tracker named \gm. In our method, we construct two graphs, tracklet graph which stores the information of history trajectories and detection graph which is constructed by objects on current frame. Then a message passing network is adopted cross these two graph to enhance the features of nodes and edges. Finally, we do graph matching between these two graphs to find the assignment of detections to trajectories. To make the whole framework end-to-end learnable, we relax the original graph matching's QAP formulation \cite{LawlerMS63, LoiolaEJOR07} to a quadratic programming, and build the graph matching layer as a QP layer \cite{amos2017optnet}. 
%The whole framework contains three main parts: 
%1) tracklet graph and detection graph construction; 
%2) feature embedding and message passing between two graphs; 
%3) graph matching and rounding. 
In Image Matching task, recently, learning-based methods have been popular. SuperGlue~\cite{sarlin2020superglue} is one of the representative work. Taking the keypoints from traditional methods, e.g., SIFT~\cite{lowe2004distinctive}, or learning-based methods, e.g., SuperPoint~\cite{detone2018superpoint} as the input, SuperGlue proposes transformer-based network to aggregate the keypoint features and matching the corresponding keypoints by a Sinkhorn layer. However, Sinkhorn is only a kind of learnable bipartite matching method, and no intra-frame relationship has been utilized explicitly. Based on SuperGlue~\cite{sarlin2020superglue}, we construct the graph in each frame and replace the Sinkhorn matching module with our differentiable graph matching network, called GMatcher. On the widely used indoor image matching dataset ScanNet~\cite{dai2017scannet}, the result fully embodies the characteristics of fast convergence and good performance of our learnable graph matching method.\par
In summary, our work has the following contributions:
\begin{itemize}
\item Instead of only focusing on data association across views, we emphasize the importance of intra-view relationships. Particularly, we propose to represent the relationships as a general graph, and formulate the data association problem as general graph matching.
\item To solve this challenging assignment problem, and further incorporate it with deep feature learning, we derive a differentiable quadratic programming layer based on the continuous relaxation of the problem, and utilize implicit function theorem and KKT conditions to derive the gradient w.r.t the input features during back-propagation.
\item We design the Gated Search Tree (GST) algorithm, greatly accelerating the process of solving quadratic assignment problem in data association. Utilizing the new GST algorithm, the association stage is about 21$\times$ faster than the original Quadratic Programming solver. 
\item In MOT task, we evaluate our proposed {\gm} on the large scale open benchmark. Our method could remarkably advance the state-of-the-art performance in terms of association metrics.
\item In image matching task, compared with SOTA method SuperGlue, we use about half training data and training iterations and obtain the gain of about 1 camera pose estimation AUC.
\end{itemize}