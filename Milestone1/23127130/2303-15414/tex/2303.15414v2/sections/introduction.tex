\IEEEraisesectionheading{\section{Introduction}\label{sec:introduction}}
    \begin{figure*}[t]
        \centering
        \subfloat[Bipartite Matching~\cite{kuhn1955hungarian}]{
            \includegraphics[width=0.46\linewidth]{figures/fig1-1.pdf}}
            \quad
            \subfloat[Graph Matching]{
            \includegraphics[width=0.46\linewidth]{figures/fig1-2.pdf}}
            \par
            \subfloat[Multiple Object Tracking]{
            \includegraphics[width=0.31\linewidth]{figures/gmtraker_pami_fig1-2.pdf}}
            \quad
            \subfloat[Image Matching]{
            \includegraphics[width=0.31\linewidth]{figures/gmtraker_pami_fig1-1.pdf}}
            \quad
            \subfloat[Point Cloud Registration]{
            \includegraphics[width=0.31\linewidth]{figures/gmtraker_pami_fig1-3.pdf}}
        \caption{An illustration of intra-graph relationship used in our graph matching formulation. We utilize the second-order edge to model the pairwise relationship, which is more robust in challenging scenes, such as heavy occluded in MOT task. For example, in view 2, the entity with ID 1 can not be associated with the entity correctly in view 1. However, with graph matching, the pairwise relationship helps data association.}
        \label{fig1}
    \end{figure*}
\IEEEPARstart{D}{ata} association is at the core of many computer vision tasks, for example, instances with the same identity are associated between different frames in \emph{Multiple Object Tracking}, 2D or 3D keypoints are associated between different views for \emph{Image Matching} and \emph{Point Cloud Registration}. These tasks can be described as detecting entities (objects/keypoints/points) in different views, then establishing the correspondences between entities in these views. In this paradigm, the latter process is called \emph{Data Association} and becomes an important part of these tasks. The traditional methods define the data association task as a bipartite matching problem, ignoring the context information in each view, i.e., the pairwise relationship between entities. 
In this paper, we argue that the relationship between the entities within the same view is also crucial for some challenging cases in data association. 
Interestingly, these pairwise relationships within the same view can be represented as edges in a general graph.
To this end, the popular bipartite matching across views can be updated to general graph matching between them. 
To further integrate this novel assignment formulation with powerful feature learning, we first relax the original formulation of graph matching \cite{LawlerMS63, kbqap} to quadratic programming and then derive a differentiable QP layer based on the KKT conditions and the implicit function theorem for the graph matching problem, inspired by the OptNet \cite{amos2017optnet}. Finally, the assignment problem can be learned in synergy with the features. In Fig.~\ref{fig1}, we show how a pairwise relationship is used in our learnable graph matching method.\par 

To reveal the effectiveness and universality of our learnable graph matching method, we apply our method to some important and popular computer vision tasks, \emph{Multiple Object Tracking} (MOT), \emph{Image Matching}, and \emph{Point Cloud Registration}.
MOT is a fundamental computer vision task that aims at associating the same object across successive frames in a video clip. A robust and accurate MOT algorithm is indispensable in broad applications, such as autonomous driving and video surveillance.
The \textit{tracking-by-detection} is currently the dominant paradigm in MOT. This paradigm consists of two steps: (1) obtaining the bounding boxes of objects by detection frame by frame; (2) generating trajectories by associating the same objects between frames. With the rapid development of deep learning-based object detectors, the first step is largely solved by powerful detectors such as~\cite{lin2017feature, lin2017focal}. 
As for the second one, recent MOT work focuses on improving the performance of data association mainly from the following two aspects: (1) formulating the association problem as a combinatorial graph partitioning problem and solving it by advanced optimization techniques \cite{berclaz2011multiple, bewley2016simple, wang2017learning, braso2020learning, xu2020train, hornakova2020lifted}; (2) improving the appearance models by the power of deep learning\cite{kuo2011does, yang2012online, leal2016learning, wojke2017simple}. Although very recently, some work~\cite{zhu2018online, sun2019deep, braso2020learning, xu2020train} trying to unify feature learning and data association into an end-to-end trained neural network, these two directions are almost isolated so that these recent attempts hardly utilize the progress from the combinatorial graph partitioning.\par
In this paper, we propose a learnable graph matching-based online tracker, called GMTracker. We construct the tracklet graph and the detection graph, and put the detections and tracklets on the vertices respectively. The edge features on each graph represent the pairwise relationship between two connected vertices. According to the general design of the learnable graph matching module mentioned above, the learnable vertex and edge features on the two graphs are the input of the differentiable GM layer. The supervision acts on these features constrained by graph matching solutions. So the learning process converges to a more robust and reasonable solution than the traditional approach.\par
However, when the quantity of the objects increases, the graph matching method is not efficient enough. The time cost will become unbearable. So, we speed up the graph matching solver by introducing the gate mechanism, called Gated Search Tree (GST) for MOT. The feasible region is limited and much smaller than the original quadratic programming formulation, so the running time of solving the graph matching problem can be substantially reduced.\par
In the Image Matching task, recently, learning-based methods have been popular. SuperGlue~\cite{sarlin2020superglue} is one of the representative works. Taking the keypoints from traditional methods, e.g., SIFT~\cite{lowe2004distinctive}, or learning-based methods, e.g., SuperPoint~\cite{detone2018superpoint} as the input, SuperGlue proposes transformer-based network to aggregate the keypoint features and matching the corresponding keypoints by a Sinkhorn layer. However, Sinkhorn is only a kind of learnable bipartite matching method, and no intra-frame relationship has been utilized explicitly. Based on SuperGlue~\cite{sarlin2020superglue}, we construct the graph in each frame and replace the Sinkhorn matching module with our differentiable graph matching network, called GMatcher. On the widely used indoor image matching dataset ScanNet~\cite{dai2017scannet}, the result fully embodies the characteristics of fast convergence and good performance of our learnable graph matching method.\par
In summary, our work has the following contributions:
\begin{itemize}
\item Instead of only focusing on data association across views, we emphasize the importance of intra-view relationships. Particularly, we propose to represent the relationships as a general graph, and formulate the data association problem as general graph matching.
\item To solve this challenging assignment problem, and further incorporate it with deep feature learning, we derive a differentiable quadratic programming layer based on the continuous relaxation of the problem, and utilize implicit function theorem and KKT conditions to derive the gradient w.r.t the input features during back-propagation.
\item We design the Gated Search Tree (GST) algorithm, greatly accelerating the process of solving the quadratic assignment problem in data association. Utilizing the new GST algorithm, the association stage is about 21$\times$ faster than the original Quadratic Programming solver. 
\item In the MOT task, we evaluate our proposed {\gm} on the large-scale open benchmark. Our method could remarkably advance state-of-the-art performance in terms of association metrics.
\item In the image matching and point cloud registration tasks, compared with newly reported methods, we achieve SOTA/SOTA comparable performance.
\end{itemize}