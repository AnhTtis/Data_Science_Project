\section{Proof of Proposition \ref{proof_UCestimate}}
\label{proof_estimate}
{\noindent\bf Proof}: Noticing that the permutations that map $\boldsymbol{\gamma}$ to each row of $\boldsymbol{\theta}^{Y|X}$ are arbitrary, there is no loss of generality in assuming $\gamma_1 \geq \gamma_2 \geq \cdots \geq \gamma_{|\mathcal{Y}|}$, \textit{i.e.}, $\boldsymbol{\gamma}\in \mathcal{K}_{|\mathcal{Y}|}$, the so-called \textit{monotone cone} \citep{Best1990}. 
Furthermore, it is more convenient to formulate the problem w.r.t. the inverse permutations, denoted as $\tau_1,\ldots, \tau_{|\mathcal{X}|} \in \mathbb{S}_{|\mathcal{Y}|}$. The problem can thus be written as
\begin{align}
    \hat{\boldsymbol \gamma}, \hat{\tau}_1, \ldots, \hat{\tau}_{|\mathcal{X}|} \;  = \hspace{-0.5cm} \underset{ \begin{array}{c} \boldsymbol \gamma \in (\Delta _{|\mathcal{Y}|-1} \cap \mathcal{K}_{|\mathcal{Y}|}) \\ 
    \tau_1,...,\tau_{|\mathcal{X}|} \in \mathbb{S}_{|\mathcal{Y}|}\end{array} }{\text{arg max}} \hspace{-0.5cm} \mathcal{L}(\boldsymbol \gamma, \tau_1,...,\tau_{|\mathcal{X}|} ), \label{thirdscenarioc}
\end{align}
where 
\begin{align}
    \mathcal{L}(\boldsymbol \gamma, \tau_1,...,\tau_{|\mathcal{X}|} ) = \sum_{x\in \mathcal{X}} \sum_{y\in \mathcal{Y}}  N_{x,\tau_x (y)}\log \gamma_y. \label{thirdscenario_bc}
\end{align}
 
The assumption $\gamma_1 \geq \gamma_2 \geq \cdots \geq \gamma_{|\mathcal{Y}|}$ makes the  maximization w.r.t. $\tau_1,...,\tau_{|\mathcal{X}|}$ independent of the particular values of $\boldsymbol{\gamma}$ as well as separable into a collection of $|\mathcal{X}|$ independent maximizations, 
\begin{equation}
    \hat{\tau}_x = \underset{ \tau_x \in  \mathbb{S}_{|\mathcal{Y}|}}{\text{arg max}} \sum_{y\in \mathcal{Y}}  N_{x,\tau_x (y)}\log \gamma_y,\label{thirdscenario2}
\end{equation}
for $x\in\mathcal{X}$. Solving \eqref{thirdscenario2} is a simple application of the \textit{rearrangement inequality}\footnote{Given any two non-decreasing sequences of $n$ real numbers, $x_1 \leq \ldots \leq x_n$ and $y_1 \leq \ldots \leq y_n$,  
\[
\forall\sigma \in \mathbb{S}_n, \;\; \sum\limits_{i=1}^n x_{\sigma_{(i)}} y_i \leq \sum\limits_{i=1}^n x_i y_i.
\]} \citep{rearrangementinequality}. Since $\log \gamma_1 \geq \log \gamma_2 \geq \cdots \geq \log \gamma_{|\mathcal{Y}|}$, the solution $\hat{\tau}_x$ is any permutation that also sorts $\{N_{x,1},..., N_{x,|\mathcal{Y}|}\}$ into non-increasing order:
\begin{equation}
\hat{\tau}_x\;\; \mbox{is such that}\;\; N_{x, \hat{\tau}_x (1)} \geq \cdots \geq N_{x, \hat{\tau}_x (|\mathcal{Y}|)}.\label{sortedNx}
\end{equation}
If all the elements of $\{N_{x,1},..., N_{x,|\mathcal{Y}|}\}$ are different, the optimal permutation is unique; otherwise, there are several optimal permutations, all achieving the same maximum. The cost of finding $\hat{\tau}_1, \ldots, \hat{\tau}_{|\mathcal{X}|}$ is $O(|\mathcal{X}|\, |\mathcal{Y}|\, \log |\mathcal{Y}|)$, since it requires $|\mathcal{X}|$ sorting operations, each with $|\mathcal{Y}|$ elements. 

Plugging $\hat{\tau}_1, \ldots, \hat{\tau}_{|\mathcal{X}|}$ back into \eqref{thirdscenario}--\eqref{thirdscenario_b} and swapping the summation order, yields 
\begin{align}
    \hat{\boldsymbol{\gamma}} \;\;  = \!\! \underset{\boldsymbol \gamma \in (\Delta _{|\mathcal{Y}|-1} \cap \mathcal{K}_{|\mathcal{Y}|}) }{\text{arg max}} \sum_{y\in \mathcal{Y}}  \log \gamma_y \sum_{x\in \mathcal{X}}  N_{x,\hat\tau_x (y)}. \label{thirdscenario3}
\end{align}
This problem is the same as \eqref{eq:UDC_known}, with $\hat\tau_x$ in the place of $\tau_x$ and with the additional constraint $\boldsymbol{\gamma} \in \mathcal{K}_{|\mathcal{Y}|}$. Temporarily ignoring this constraint leads to (see \eqref{approximategamma}), 
\begin{equation}
    {\displaystyle \hat{\gamma}_y = \frac{1}{N}\sum\limits_{x\in \mathcal{X}} N_{x, \hat\tau_x (y)}}, \;\; \; \mbox{for $y \in \mathcal{Y}$}.
\label{approximategamma2}
\end{equation}
The fact that $N_{x, \hat{\tau}_x (1)} \geq N_{x, \hat{\tau}_x (2)} \geq \cdots \geq N_{x, \hat{\tau}_x (|\mathcal{Y}|)}$, for any $x\in\mathcal{X}$, implies that $\hat{\boldsymbol{\gamma}} \in \mathcal{K}_{|\mathcal{Y}|}$, without having to include this constraint.  Consequently, problem \eqref{thirdscenario}--\eqref{thirdscenario_b}  has a global solution given by \eqref{approximategamma2} and \eqref{sortedNx}.
\hfill $\blacksquare$