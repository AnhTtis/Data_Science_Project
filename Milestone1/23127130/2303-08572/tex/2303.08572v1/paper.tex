%\documentclass[anon,12pt]{clear2023}% Anonymized submission
\documentclass[final,12pt]{clear2023} % Include author names
%\usepackage[nonatbib]{neurips_2022}

% The following packages will be automatically loaded:
% amsmath, amssymb, natbib, graphicx, url, algorithm2e

\title[Distinguishing Cause from Effect on Categorical Data]{Distinguishing Cause from Effect on Categorical Data: \\
The Uniform Channel Model}

\usepackage{times}

\usepackage{graphicx} % Enhanced LaTeX Graphics
\usepackage{siunitx}
\usepackage{comment}
\usepackage[american]{babel}
\usepackage{enumitem}
%\usepackage{wasysym}

\usepackage[utf8]{inputenc} % allow utf-8 input
\usepackage[T1]{fontenc}    % use 8-bit T1 fonts
\usepackage{hyperref}       % hyperlinks
\usepackage{url}            % simple URL typesetting
\usepackage{booktabs}       % professional-quality tables
\usepackage{amsfonts}       % blackboard math symbols
\usepackage{nicefrac}       % compact symbols for 1/2, etc.
\usepackage{microtype}      % microtypography
\usepackage{xcolor}         % colors

\usepackage{array}
%\usepackage{amsmath}
%\usepackage{amsfonts}
%\usepackage{amsthm}
%\usepackage{amssymb}
%\usepackage{fancybox,framed}
\usepackage{enumitem}

%\usepackage{natbib} % has a nice set of citation styles and commands%5  
%\bibliographystyle{abbrvnat}
%    \renewcommand{\bibsection}{\subsubsection*{References}}
\usepackage{mathtools} % amsmath with fixes and additions
% \usepackage{siunitx} % for proper typesetting of numbers and units
%\usepackage{booktabs} % commands to create good-looking tables
\usepackage{tikz} % nice language for creating drawings and diagrams


%\newtheorem{theorem}{Theoremm}
%\newtheorem{theoremm}[theorem]{Theorem}
%\newtheorem{corollary}{Corollary}[theorem]
%\newtheorem{lemma}[theorem]{Lemma}
%\newtheorem{definition}[theorem]{Definition}
%\newtheorem{proposition}[theorem]{Proposition}
%\newtheorem{example}[theorem]{Example}
%\newtheorem{remark}{Remark}

% Use \Name{Author Name} to specify the name.
% If the surname contains spaces, enclose the surname
% in braces, e.g. \Name{John {Smith Jones}} similarly
% if the name has a "von" part, e.g \Name{Jane {de Winter}}.
% If the first letter in the forenames is a diacritic
% enclose the diacritic in braces, e.g. \Name{{\'E}louise Smith}

% Two authors with the same address
\clearauthor{\Name{Mário A. T. Figueiredo,} \Email{mario.figueiredo@tecnico.ulisboa.pt} \\
  \Name{Catarina Oliveira} \Email{catarina.a.oliveira@tecnico.ulisboa.pt}\\
  \addr Instituto de Telecomunicações and LUMLIS (Lisbon ELLIS Unit), \\
  Instituto Superior Técnico, Universidade de Lisboa, Portugal}

%Three or more authors with the same address:
% \author[1]{\href{mailto:<jj@example.edu>?Subject=Your UAI 2022 paper}{Jane~J.~von~O'L\'opez}{}}
% \clearauthor{\Name{Catarina A.  Oliveira} \Email{catarina.a.oliveira@tecnico.ulisboa.pt}\\
% \Name{Alexandra M. Carvalho} \Email{alexandra.carvalho@tecnico.ulisboa.pt}\\
%  \Name{M\'{a}rio A. T. Figueiredo} \Email{mario.figueiredo@tecnico.ulisboa.pt}\\
%  \addr Instituto de Telecomunicações, Instituto Superior Técnico, Lisboa Portugal\\
%  Lisbon (ELLIS) Unit for Learning and Intelligent Systems (LUMLIS)}

% Authors with different addresses:
%\clearauthor{%
% \Name{Author Name1} \Email{abc@sample.com}\\
% \addr Address 1
% \AND
% \Name{Author Name2} \Email{xyz@sample.com}\\
% \addr Address 2%
%}

%\author{%
%  M\'{a}rio A. T. Figueiredo\thanks{Use footnote for providing further information
%    about author (webpage, alternative address)---\emph{not} %for acknowledging
%    funding agencies.} \\
%  Instituto de Telecomunicações, Instituto Superior Técnico, Lisboa Portugal \\
%  \texttt{mario.figueiredo@tecnico.ulisboa.pt} \\
  % examples of more authors
  % \And
  % Coauthor \\
  % Affiliation \\
  % Address \\
  % \texttt{email} \\
  % \AND
  % Coauthor \\
  % Affiliation \\
  % Address \\
  % \texttt{email} \\
  % \And
  % Coauthor \\
  % Affiliation \\
  % Address \\
  % \texttt{email} \\
  % \And
  % Coauthor \\
  % Affiliation \\
  % Address \\
  % \texttt{email} \\
%}


%\usepackage[ruled,vlined]{algorithm2e}


%\newtheorem{theorem}{Theorem}
%\newtheorem{theorem}[theorem]{Theorem}
%\newtheorem{corollary}{Corollary}[theorem]
%\newtheorem{lemma}[theorem]{Lemma}
%\newtheorem{definition}[theorem]{Definition}
%\newtheorem{proposition}[theorem]{Proposition}
%\newtheorem{example}[theorem]{Example}
%\newtheorem{remark}{Remark}

%%
%% The amsthm package provides extended theorem environments
%% http://www.ctan.org/tex-archive/help/Catalogue/entries/amsthm.html
%\usepackage{float}
\usepackage[normalem]{ulem} 
%\usepackage[hang,small,bf,tight]{subfigure}
%\usepackage[ruled,vlined,norelsize]{algorithm2e}

\begin{document}

\maketitle

\begin{abstract}%
Distinguishing cause from effect using observations of a pair of random variables is a core problem in causal discovery. Most approaches proposed for this task, namely \textit{additive noise models} (ANM), are only adequate for quantitative data. We propose a criterion to address the cause-effect problem with categorical variables (living in sets with no meaningful order), inspired by seeing a conditional \textit{probability mass function} (pmf) as a discrete memoryless channel. We select as the most likely causal direction the one which the conditional pmf is closer to a \textit{uniform channel} (UC). The rationale is that, in a UC, as in an ANM, the conditional entropy (of the effect given the cause) is independent of the cause distribution, in agreement with the principle of \textit{independence of cause and mechanism}. Our approach, which we call the \textit{uniform channel model} (UCM), thus extends the ANM rationale to categorical variables. To assess how \textit{close} a conditional pmf (estimated from data) is to a UC, we use statistical testing, supported by a closed-form estimate of a UC channel. On the theoretical front, we prove identifiability of the UCM and show its equivalence with a structural causal model with a low-cardinality exogenous variable. Finally, the proposed method compares favorably with recent state-of-the-art alternatives in experiments on synthetic, benchmark, and real data.

\end{abstract}

%\begin{keywords}%
%Causal discovery, categorical data, cause-effect pairs, communication channels, additive noise models; independence of cause and mechanism.
%\end{keywords}


\section{Introduction}

The increasing complexity of source code poses a key challenge to the reliability of large-scale software systems. Software bugs in these systems can lead to safety issues~\cite{bug_safety} for users around the world as well as cause non-negligible financial losses~\cite{bug_loss}. As such, developers have to spend a large amount of time and effort on bug fixing. Consequently, \aprfull (\apr), designed to automatically generate patches to fix software bugs, has attracted wide attention from both academia and industry~\cite{long2016prophet, legoues2012genprog, long2015spr, lou2020can, tufano2018empstudy}. 


To achieve \apr, one popular approach is known as Generate-and-Validate (G\&V)~\cite{qi2015gv, ghanbari2019prapr, lou2020can, le2016hdrepair, legoues2012genprog, wen2018capgen, hua2018sketchfix, martinez2016astor, koyuncu2020fixminder, liu2019tbar, liu2019avatar}, which is typically based on the following pipeline: First, fault localization techniques~\cite{wong2016fl, abreu2007ochiai, zhang2013injecting, papadakis2015metallaxis, li2019deepfl, li2017transforming} are applied to determine the suspicious locations in programs where bugs are likely to exist. Then, the buggy locations are used by the \apr tools to generate a list of patches that replace buggy lines with correct lines. Afterward, each patch is validated against the original test suite to identify any \emph{plausible patches} (i.e., passing all tests in the test suite). Finally, to determine the \emph{correct patches}, developers examine the list of plausible patches to see if any of them can correctly fix the bug. 

Traditional \apr tools can mainly be categorized into heuristic-based~\cite{legoues2012genprog, le2016hdrepair, wen2018capgen}, constraint-based~\cite{mechtaev2016angelix, le2017s3, demacro2014nopol, long2015spr} and \template~\cite{ghanbari2019prapr, hua2018sketchfix, martinez2016astor, liu2019tbar, liu2019avatar}. Among these traditional tools, \template \apr tools~\cite{ghanbari2019prapr, liu2019tbar, benton2020effectiveness} have been able to achieve state-of-the-art results. \Template \apr tools typically leverage pre-defined templates (e.g., adding a nullness check) for bug fixing. However, since these fix templates are typically handcrafted, the number and types of bugs they are able to fix can be limited. 



To address the limitations of traditional \apr, researchers have proposed various \learning \apr tools~\cite{li2020dlfix, chen2018sequencer, jiang2021cure, lutellier2020coconut, zhu2021recoder, ye2022rewardrepair} based on the \nmtfull (\nmt) architecture~\cite{sutskever2014mt} where the input is the buggy code snippets and the goal is to translate the buggy code snippets into a fixed version. To accomplish this, \learning \apr tools require supervised training datasets with pairs of both buggy and fixed code snippets in order to learn how to perform this translation step. These training data are usually obtained by mining historical bug fixes using heuristics/keywords~\cite{dallmeier2007benchmark}, which can be imprecise for identifying bug-fixing commits; even the actual bug-fixing commits can include irrelevant code changes, leading to further pollution in the dataset~\cite{xia2022alpharepair}.
% 
Moreover, it can be hard for such \apr tools to generalize and fix bug types unseen during training. 



To better leverage recent advances in \plmfull{s} (\plm{s}), researchers~\cite{xia2022alpharepair, xia2023repairstudy, kolak2022patch, prenner2021codexws} have directly applied \plm{s} to generate patches without bug-fixing datasets. These \llm-based \apr tools work by either directly generating a complete code function~\cite{prenner2021codexws, xia2023repairstudy} or predict/infill the correct code snippet given its surrounding context~\cite{xia2022alpharepair, xia2023repairstudy}. By directly using \llm{s} that are pre-trained on billions of open-source code snippets, \llm-based \apr tools can achieve state-of-the-art performance on many repair datasets~\cite{xia2022alpharepair}. 


% 
%
%

Traditional \apr tools have long used the insight of the \emph{plastic surgery hypothesis}~\cite{barr2014plastic} where it states that the code ingredients to fix a bug already exist within the same project. Traditional \apr tools have manually designed pattern-~\cite{ghanbari2019prapr, saha2017elixir} or heuristic-based~\cite{jiang2018simfix, legoues2012genprog} approaches to finding and using such relevant code ingredients to generate fixes for bugs. However, the plastic surgery hypothesis has been largely ignored in \llm-based \apr. In fact, \llm provides a unique opportunity to fully automate the plastic surgery hypothesis idea via fine-tuning (learning project-specific information via model updates from the buggy project) and prompting (directly providing relevant code ingredients to the model), and make it directly applicable to different languages (since the \llm{s} are typically multi-lingual).%
Moreover, despite the intensive manual efforts involved, traditional \apr tools still cannot fully leverage project-specific information due to large search space for leveraging/composing existing code ingredients. In contrast, the project-specific information can effectively leveraged by \llm{s} due to their power in code understanding/vectorization, e.g., even partial/imprecise information may still guide \llm{s} in correct patch generation!
 To this end, we ask the question: \emph{How useful is the plastic surgery hypothesis in the era of \plm{s}}?








\mypara{Our Work.} To answer the question, we present \ourtech{\xspace} -- a \llm-based approach that automatically utilizes the plastic surgery hypothesis by systematically combining multiple fine-tuning and prompting strategies for \apr. \ourtech fine-tunes \plm{s} using two novel domain-specific training strategies: \textbf{\epfinetune} -- we fine-tune using the original buggy project by aggressively masking out a high percentage of tokens, which allows \plm to learn project-specific code tokens and programming styles; and \textbf{\rofinetune} -- which only masks out a single continuous code sequence per training sample, allowing the model to get used to the final \csapr task of predicting a single continuous code sequence. Furthermore, we directly leverage the ability for \plm{s} to understand natural language instructions and introduce a novel prompting strategy, \textbf{\idprompting}, which uses information retrieval and static analysis to obtain a list of relevant identifiers for the buggy lines. While such relevant identifiers are critical for fixing some difficult bugs, they may not be seen by the \llm during inference due to limited context window size. Through the use of prompting, we directly tell the model to use these extracted identifiers (relevant code ingredients) to generate the correct code. Finally, to perform repair, we combine all four model variants (including the base model, both fine-tuned models and the base model with prompting) for the final repair.





While our insight of leveraging the plastic surgery hypothesis for \llm-based \apr is generalizable across different types of \plm{s}, to implement \ourtech, we choose a recent \plm{\xspace}, \ctfive~\cite{wang2021codet5}, which is pre-trained on millions of open-source code snippets. \ctfive is an encoder-decoder model trained using \mspfull (\msp) objective where a percentage of tokens are masked out and each continuous masked token sequence is referred to as a masked span. Also, although we only extract relevant identifiers from the current buggy project (since this paper focuses on the plastic surgery hypothesis), our work can be easily extended to obtain other code information (such as relevant statements or functions) from other sources, such as  the massive pre-training corpora~\cite{husain2020codesearchnet} or historical bug-fixing datasets~\cite{jiang2019infer}, which can provide more coding knowledge for \llm{s}. Besides, although we mainly focus on using traditional string comparison algorithms for information retrieval in this paper, these techniques can be easily replaced by other frequency-based retrieval~\cite{robertson2009probabilistic} and neural search (or embedding-based search)~\cite{reimers2019sentence}.
  In summary, this paper makes the following contributions:


%


\begin{itemize}[noitemsep, leftmargin=*, topsep=0pt]
    \item \textbf{Dimension.} This paper is the first to revisit the important plastic surgery hypothesis in the era of \llm{s}. It opens up a new dimension for \llm-based \apr to incorporate previously neglected information from the buggy project itself to boost \apr performance. Furthermore, it demonstrates the promising future of retrieval-based prompting for modern \llm-based \apr.
    \item \textbf{Implementation.} We implement \ourtech based on the recent \ctfive model. We augment the model using two novel fine-tuning strategies: \epfinetune and \rofinetune, along with a novel prompting strategy based on information retrieval and static analysis: \idprompting. We combine the patches generated by all four models together and perform patch ranking to speed up \apr.% 
    \item \textbf{Evaluation Study.} We conduct an extensive evaluation against state-of-the-art \apr tools. On the widely studied \dfj 1.2 and 2.0 datasets~\cite{just2014dfj}, \ourtech is able to achieve the new state-of-the-art results of 89 and 44 correct bug fixes (15 and 8 more than best baseline) respectively.  Furthermore, we perform a broad ablation study to justify our design. \ourtech demonstrates for the first time that the plastic surgery hypothesis can substantially boost \llm-based \apr and advance state-of-the-art \apr, while being fully automated and general. Moreover, even partial/imprecise code ingredients may still effectively guide \llm{s} for \apr!
\end{itemize}


%%\input{02_backg}
\section{Method}
\label{sec:method}

% \ml{``Inconsistent'' to ``large variation''}

% In this section, we propose our methods based on the observations in Section \ref{sec:motivation}.
In this section, we propose two techniques to further enhance the strong baseline to capture the variation of activation distributions better.
We first introduce spatial re-scaling to adapt the network to pixel-to-pixel variation.
We then propose channel-wise shifting and re-scaling to better capture the channel-to-channel variation.
Meanwhile, as both of the two methods are image-dependent, the image-to-image variation can be captured naturally.
By combining the two methods with our strong baseline, we build our enhanced BNN for SR, named EBSR.

% Because the activation distributions among pixels, channels and images have large variations \red{**are highly inconsistent} in SR networks, we introduce spatial re-scaling to adapt to pixel-wise variations and channel shift and re-scaling to adapt to channel-wise variations. And both of them are image-dependent to adapt to image-wise variations, which means during inference our network re-scales and shifts the distributions of activations flexibly for different input images. Based on these methods, we build an enhanced binary neural network for image super-resolution (EBSR).

% According to [3], the difference of activation magnitudes indicates different scaling factors are needed for each pixel.

\subsection{Spatial Re-scaling}
% It is better to use different scaling factors for different pixels to reduce the quantization error and retain more detailed information for image super-resolution. 

% \ml{In the main method, we do not need to introduce the previous works but can focus on introducing our own method. Channel rescaling in Real-to-binary Net is not relevant in this context.}

% Re-scaling the output of binary convolutions was proposed at the birth of BNN in XNOR-Net \cite{rastegari2016xnor} to reduce quantization error and improve accuracy for image classification tasks.
% It is computed as below:
% \begin{equation}
% \mathcal{A} * \mathcal{W} \approx(\operatorname{sign}(\mathcal{A}) \circledast \operatorname{sign}(\mathcal{W})) \odot \mathcal{K} \alpha
% \label{eq:xnor-net rescale}
% \end{equation}
% where $\circledast$ denotes the binary convolution and $\odot$ denotes the element-wise multiplication.
% $\mathcal{A}$, $\mathcal{W}$, $\alpha$, and $\mathcal{K}$ denote the activation, weight, weight scaling factor, and activation scaling factor, respectively.
%  Later in XNOR-Net++ \cite{bulat2019xnor}, Bulat et al. fuse the activation and weight scaling factors into a single one that is learned end-to-end based on gradients and this improves the classification accuracy on ImageNet dataset.

% % It is computed as Eq.~\ref{eq:xnor-net rescale}, where $\circledast$ denotes 
% %  the binary convolution and $\odot$ denotes the element-wise multiplication. The binary convolution of $\mathcal{A}$ and $\mathcal{W}$ is rescaled by the weight scaling factor $\alpha$ and the activation scaling factor $\mathcal{K}$, both of which are calculated analytically.


% \zc{Similarly, you should explain the meaning of A, W and the operators $\circledast$ in the formula}
% Then in Real-to-binary Net \cite{martinez2020training}, Martinez et al. used a data-driven channel re-scaling module that takes the pre-convolution activations as input to predict the activation scaling factor. Unlike that in XNOR-Net++ \cite{bulat2019xnor}, these scaling factors are not fixed during inference but rather inferred from data. By doing this, they further improved the classification accuracy on ImageNet over XNOR-Net++. 
As is shown in Figure \ref{fig:pixel}, activation distributions have large pixel-to-pixel variation in SR networks
and the difference of activation magnitudes indicates different scaling factors are preferred for different pixels.
Inspired by \cite{martinez2020training}, we propose spatial re-scaling to better adapt the network to the spatial variation
of activation distributions in SR networks.
% fit the various pixel-wise distributions in SR networks.
We take the real-valued activations $A$ before convolution as input and predict pixel-wise scaling factors $S(A)$, which re-scale the binary convolution output. Spatial re-scaling process can be formulated as follows:
\begin{equation}
A * W \approx(\operatorname{sign}(A) \circledast \operatorname{sign}(W)) \odot \alpha \odot S(A)
\label{eq:spatial rescale}
\end{equation}
where $\circledast$ denotes 
the binary convolution and $\odot$ denotes the element-wise multiplication. $A$, $W$, $\alpha$, and $S\left(A\right)$ denote real-valued activations, weights, the scaling factor of weights, and the spatial-wise scaling factor of activations respectively. $S\left(A\right) \in \mathbb{R}^{1\times H\times W}$ can be calculated with a convolution and a sigmoid function.
% as $\sigma\left( CONV\left(A\right)\right)$. 
As shown in Figure \ref{fig:method}(a), real-valued activations first go through a convolution layer,
which has an input channel of $C$ and an output channel of 1, 
and then pass through a sigmoid function to produce the scaling factors $S(A)$ along the spatial dimension.
During inference, the scaling factor will change dynamically according to different input feature maps.
By re-scaling binary convolution output using $S(A)$, we can reduce the quantization error and the original pixel-wise information in FP activation
will be preserved much better.
Spatial re-scaling leads to a large PSNR improvement of 0.24 dB (from 30.30 dB to 31.54 dB) on Set5 and 0.22 dB (from 25.09 dB to 25.31 dB)
on Urban100 compared with our strong baseline. 

\subsection{Channel-wise Shifting and Re-scaling}

\begin{table}[!tb]
\centering
\caption{Comparison between whether to fuse channel-wise shifting and re-scaling or not based on our baseline with spatial re-scaling. }
\label{tab:fusing}

\scalebox{0.65}{
\begin{tabular}{c|cc|cc|cc}
\hline
\multirow{2}{*}{Method}     & \multirow{2}{*}{OPs} & \multirow{2}{*}{Params} & \multicolumn{2}{c|}{Set5} & \multicolumn{2}{c}{Urban100} \\ \cline{4-7} 
                            &                      &                         & PSNR        & SSIM        & PSNR          & SSIM         \\ \hline
Baseline + spatial re-scale & 2.16G                & 0.05M                   & 31.54       & 0.883       & 25.31         & 0.759        \\
+ channel-wise shift and re-scale             & 2.34G                & 0.09M                   & 31.61       & 0.885       & 25.35         & 0.761        \\
+ w/ fusing                   & 2.27G                & 0.08M                   & \textbf{31.64}       & \textbf{0.885}       & \textbf{25.36}         & \textbf{0.761}        \\ \hline
\end{tabular}
}
\end{table}

In SR networks, activation distributions exhibit larger channel-to-channel variation (Figure \ref{fig:chl}).
Both the mean and magnitude of the activation distributions vary significantly across channels.
% Thus we use channel-wise shifting and re-scaling to adapt to various channel-wise distributions. 
\cite{martinez2020training} has proposed the data-driven channel re-scaling, 
but our method differs from them in further introducing data-driven thresholds to handle the channel-wise variation of both mean and magnitude.
Since the blocks to generate the scaling factors and thresholds are very similar, we further propose to fuse them into one module.
% and fusing channel-wise shifting and re-scaling into one module.
We evaluate the effect of fusing the two blocks in Table \ref{tab:fusing}.
With channel-wise shifting and re-scaling fused, our models have fewer operations and parameters overhead and slightly higher performance.

For the specific process, we take the real-valued activations as input and predict different thresholds and scaling factors for each channel. They are also image dependent, e.g., $\beta_{i}$ in Eq.\ref{eq:act_binarize} is no longer fixed during inference but generated according to different input feature maps. Channel-wise shifting and re-scaling can be formulated as follows:
\begin{equation}
A * W \approx(\operatorname{sign}(A-C_s(A)) \circledast \operatorname{sign}(W)) \odot \alpha \odot C_r(A)
\label{eq:channel-wise_shift_and_rescale}
\end{equation}
where $\circledast$ denotes 
the binary convolution and $\odot$ denotes the element-wise multiplication. $C_s(A), C_r(A) \in \mathbb{R}^{C\times1\times1}$ denote the channel-wise threshold and scaling factor, respectively. 
We show the block diagram in Figure \ref{fig:method}(b).
The real-valued input feature map is first squeezed to a ${C\times1\times1}$ vector by a global average pooling (GAP) layer.
The subsequent fully connected layers and ReLU learn the channel-wise information and output a ${2C\times1\times1}$ vector.
Then the ${2C\times1\times1}$ vector is split into two ${C\times1\times1}$ vectors.
We use the first $C$ channels as the channel-wise bias and pass the last $C$ channels through a sigmoid layer 
as the channel-wise scaling factor, which are used to shift the real-valued activations and re-scale the binary convolution output, respectively. 


% \ml{We can mention previously, channel-wise re-scale has been proposed. We propose to fuse them. Add the comparison between fuse v.s. no fuse.}

\begin{figure}[!tbp]%
  \centering
    \includegraphics[width=0.4\textwidth]{fig/methods.png}
  
% \subfloat[channel-wise shifting\&re-scale]{
%     \label{subfig:channel-wise shifting and re-scale}
%     \includegraphics[width=0.2\textwidth]{fig/chl shift and rescale.png}
%   }

  \caption{Block diagram for spatial re-scaling, and channel-wise shifting and re-scaling.} 
  % Input A is the real-valued activation tensor and C, H, and W denote its dimension. GAP stands for global average pooling. The reduction ratio r is set to 16 for a better trade-off between the performance and the number of operations and parameters.}
  \label{fig:method}
\end{figure}


\subsection{Network Structure}

Combining the spatial re-scaling and the channel-wise shifting and re-scaling methods, we construct the enhanced convolution layer (E-Conv).
Then we build our EBSR model based on E-Conv.
In Figure \ref{fig:E-conv}, we compare the binary convolution layer used in the baseline network and our proposed E-Conv.
We use spatial and channel-wise scaling factors to re-scale the binary convolution output,
and use channel-wise shifting to learn appropriate thresholds for each channel before binarization.
The scaling factors and threshold used in E-Conv are learnable and depend on the real-valued input activations.
In this way, our proposed EBSR can adapt to pixel-to-pixel, channel-to-channel, and image-to-image variations
to reduce the large binarization error and preserve more details.
% In this way, our proposed E-Conv reduces the large quantization error caused by binarization and keeps the original information of input feature maps to a large extent.


\begin{figure}[!tb]%
  \centering

    \includegraphics[width=0.5\textwidth]{fig/E-conv.png}

  \caption{Comparison of (a) the binary convolution layer with a skip connection used in our baseline network and (b) the proposed E-Conv.}
  \label{fig:E-conv}
\end{figure}


Figure \ref{fig:network} shows the basic block based on the E-Conv and our EBSR composed of the basic blocks. Following existing works, the convolution layers in the head and tail modules are not binarized. We choose the lightweight EDSR which has 16 basic blocks and 64 channels, and EDSR which has 32 basic blocks and 256 channels as our backbones, which correspond to EBSR-light and EBSR, respectively.

\begin{figure}[!tb]%
  \centering
  {
    \includegraphics[width=0.35\textwidth]{fig/network.png}
  }
  
  \caption{The structure of our proposed EBSR.  Convolution layers in purple are real-valued vanilla 3x3 convolutions.}
  \label{fig:network}
\end{figure}
\section{Channel Estimation from Data}
\label{sec:channel}
This section addresses the problem of estimating channel parameters $\boldsymbol \theta^{X\rightarrow Y}$ from $N$ independent and identically distributed samples of $(X,Y): (x_1,y_1),..., (x_N, y_N)$. This will play a key role in translating the causal inference criterion proposed in Section \ref{sec:UCModel} to the realistic scenario where there is only access to a finite amount of data, rather than perfect knowledge of the joint pmf $p_{X,Y}$.

Before estimating the channel parameters (conditional pmf), notice that estimating the marginal pmf $\boldsymbol \theta^{X}$ is trivial: with $N_x$ denoting the number of samples $(x_i, y_i)$ with $x_i = x$, the \textit{maximum likelihood} (ML) estimate of $\boldsymbol \theta^X$ is given by 
\begin{equation}
    \hat{\boldsymbol \theta}^{X} = \; \arg \max_{{\boldsymbol \theta} \in \Delta _{|\mathcal{X}|-1}} \sum_{x\in\mathcal{X}} N_x \log  \theta^X_{x} = \Bigl( \frac{N_1}{N},...,\frac{N_{|\mathcal{X}|}}{N}\Bigr).
    \label{eq:ML_thetax}
\end{equation}

For estimating $\boldsymbol \theta^{X\rightarrow Y}$, we consider the following 4 scenarios:  \textbf{(1)} arbitrary channel; \textbf{(2)} UCM, with known permutations; \textbf{(3)} UCM,  with \textit{unknown} permutations; \textbf{(4)} CUCM, with \textit{unknown} cyclic permutations. Scenarios 1 and 2 are trivial and considered only as they provide the building blocks to address scenarios 3 and 4.

\subsection{Scenario 1: Arbitrary Channel}
\label{subsec: firstscenario}
Let $N_{x,y}$ be the number of samples $(x_i, y_i)$ such that $x_i = x$ and $y_i=y$. In the absence of constraints other than each row of $\boldsymbol \theta^{X\rightarrow Y}$ must be a valid pmf, the ML estimate is 

\begin{equation}
    \hat{\boldsymbol{\theta}}^{X\rightarrow Y}  =  \underset{\boldsymbol{\theta} \in (\Delta_{|\mathcal{Y}|-1})^{|\mathcal{X}|}}{\text{arg max}} \sum_{x\in \mathcal{X}} \sum_{y\in \mathcal{Y}}  N_{x,y}\log \theta_{x,y}.
    \label{objectivefunction}
\end{equation}
Since both the objective function and the constraints in \eqref{objectivefunction} are separable across $x= 1,..., |\mathcal{X}|$, the problem is also separable into a collection of problems, each yielding the classical ML estimates
\begin{equation}
    \hat{\theta}^{X\rightarrow Y}_{x,y} = N_{x,y}/N_x, \hspace{0.3cm} \mbox{for $(x,y)\in \mathcal{X}\times \mathcal{Y}$}.
\label{estimate_thetaxy}
\end{equation}

\subsection{Scenario 2: UCM with Known Permutations}
\label{2ndscenario}
In a UC, each row $\boldsymbol \theta_x^{X\rightarrow Y}$ is as given in \eqref{eq_gama_perm}. If the permutations $\sigma_1, \ldots, \sigma_{|\mathcal{X}|}$ are known, the ML estimate of $\boldsymbol \gamma$ is given by 
\begin{equation}
    \hat{\boldsymbol{\gamma}} = \;  \underset{\boldsymbol{\gamma} \in \Delta _{|\mathcal{Y}|-1} }{\text{arg max}} \sum_{x \in \mathcal{X}} \sum_{y\in \mathcal{Y}}  N_{x,y}\log \gamma_{\sigma_x (y)}.
    \label{objectivefunction_symmetric}
\end{equation}
This problem is not separable, as all the rows of the channel matrix share the same probability values, although with different permutations. Swapping the summation order  and using the inverse permutations $\tau_x = \sigma_x^{-1}$ to do a change of variable in the sum over $\mathcal{Y}$, problem \eqref{objectivefunction_symmetric}  can be rewritten as
\begin{equation}
    \hat{\boldsymbol{\gamma}} = \; \underset{\boldsymbol{\gamma} \in \Delta _{|\mathcal{Y}|-1} }{\text{arg max}}  \sum\limits_{z\in \mathcal{Y}} \log \gamma_z \sum_{x\in \mathcal{X}} N_{x,\tau_{x}(z)}.
    \label{eq:UDC_known}
\end{equation}
Problem \eqref{eq:UDC_known} has the same form as \eqref{eq:ML_thetax}, the solution being simply
\begin{equation}
    {\displaystyle \hat{\gamma}_y =  \frac{1}{N}\sum\limits_{x\in \mathcal{X}} N_{x, \tau_x (y)}}, \;\; \; \mbox{for $y \in \mathcal{Y}$}.
\label{approximategamma}
\end{equation}
Notice that $N_{x, \tau_x (y)}$ is the number of samples $(x_i,y_i)$ such that $x_i=x$ and $y_i = \tau_x (y)$.

\subsection{Scenario 3: UC with Unknown Permutations}
\label{subsec: thirdscenario}
In this case, the log-likelihood is maximized, not only w.r.t. $\boldsymbol\gamma$, but also the permutations. Although, at first sight, this may look like a very hard problem, as there are $(|\mathcal{Y}|!)^{|\mathcal{X}|}$ combinations of permutations, we show next that it can be solved very efficiently. The optimization problem in hand (formulated w.r.t. the inverse permutations, denoted as $\tau_1,\ldots, \tau_{|\mathcal{X}|} \in \mathbb{S}_{|\mathcal{Y}|}$) is
\begin{align}
    \hat{\boldsymbol \gamma}, \hat{\tau}_1, \ldots, \hat{\tau}_{|\mathcal{X}|} \;  = \hspace{-0.5cm} \underset{ \begin{array}{c} \boldsymbol \gamma \in \Delta _{|\mathcal{Y}|-1} \\ 
    \tau_1,...,\tau_{|\mathcal{X}|} \in \mathbb{S}_{|\mathcal{Y}|}\end{array} }{\text{arg max}} \hspace{-0.5cm} \mathcal{L}(\boldsymbol \gamma, \tau_1,...,\tau_{|\mathcal{X}|} ), \label{thirdscenario}
\end{align}
where 
\begin{align}
    \mathcal{L}(\boldsymbol \gamma, \tau_1,...,\tau_{|\mathcal{X}|} ) = \sum_{x\in \mathcal{X}} \sum_{y\in \mathcal{Y}}  N_{x,\tau_x (y)}\log \gamma_y. \label{thirdscenario_b}
\end{align}
The following proposition (proved in Appendix \ref{proof_estimate}) provides the solution to this problem.

\begin{proposition}\label{proof_UCestimate}
A globally optimal solution to the problem specified in \eqref{thirdscenario}--\eqref{thirdscenario_b} is given as follows. For $x\in \mathcal{X}$, $\hat{\tau}_x$ is any permutation that sorts $\{N_{x,1},..., N_{x,|\mathcal{Y}|}\}$ into non-increasing order,
\begin{equation}
\hat{\tau}_x\;\; \mbox{is such that}\;\; N_{x, \hat{\tau}_x (1)} \geq \cdots \geq N_{x, \hat{\tau}_x (|\mathcal{Y}|)},\label{sortedNx_a}
\end{equation}
and, for  $y \in \mathcal{Y}$,
\begin{equation}
    {\displaystyle \hat{\gamma}_y = \frac{1}{N}\sum\limits_{x\in \mathcal{X}} N_{x, \hat\tau_x (y)}}.
\label{approximategamma2_a}
\end{equation}
\end{proposition}

The solution in \eqref{sortedNx_a}--\eqref{approximategamma2_a} is a global, but not unique, optimum; in fact, any pmf $\hat{\boldsymbol{\xi}}$ that is a permutation of $\hat{\boldsymbol{\gamma}}$, \textit{i.e.}, $\hat{\gamma}_y = \hat{\xi}_{\rho(y)}$, where $\rho \in \mathbb{S}_{|\mathcal{Y}|}$, yields 
\[
\hat{\theta}_{x,y}^{X\rightarrow Y} = \hat{\gamma}_{\hat{\sigma}_x(y)} = \hat{\xi}_{\rho(\hat{\sigma}_x(y))}.
\]
That is, $\boldsymbol\gamma$ is identifiable only up to a permutation, since any permutation of $\hat{\boldsymbol{\gamma}}$, combined with the inverse of that permutation composed with each $\sigma_x$, yields the same conditional pmf estimate $\hat{\boldsymbol{\theta}}^{X\rightarrow Y}\!\!\!$, thus the same maximum value of the log-likelihood. Finally, notice that the cost of computing this solution scales as $O(|\mathcal{X}|\, |\mathcal{Y}|\, \log |\mathcal{Y}|)$, due to the number $|\mathcal{X}|$ of sorting operations, each of size $|\mathcal{Y}|$.

\subsection{Scenario 4: CUC with Unknown Permutations}
\label{subsec:fourth}
The difference between this and the previous case is that the permutations are now cyclic. Thus, the corresponding optimization problem is identical to \eqref{thirdscenario}--\eqref{thirdscenario_b}, but with the constraint $\tau_1,...,\tau_{|\mathcal{X}|} \in \mathbb{S}_{|\mathcal{Y}|}$ replaced with $\tau_1,...,\tau_{|\mathcal{X}|} \in \mathbb{C}_{|\mathcal{Y}|}$, where $\mathbb{C}_{|\mathcal{Y}|}$ is the set of cyclic permutations of $\{1,...,|\mathcal{Y}|\}$. 

Although the cardinality of $\mathbb{C}_{|\mathcal{Y}|}$ is $|\mathcal{Y}|$, much smaller than that of $\mathbb{S}_{|\mathcal{Y}|}$, which is $|\mathcal{Y}|!$, this problem is harder than  \eqref{thirdscenario}--\eqref{thirdscenario_b}. Whereas the cost of the exact solution of \eqref{thirdscenario}--\eqref{thirdscenario_b} scales with $O(|\mathcal{X}|\, |\mathcal{Y}|\, \log |\mathcal{Y}|)$, exactly solving this problem by exhaustive search costs $O(|\mathcal{Y}|^{|\mathcal{X}|})$. It happens that this problem is a variant of a class of problems known as \textit{multireference alignment}, which is known to be NP-hard \citep{Bandeira}. Exact solutions are thus out of the question for large problems. Here, we propose an alternating maximization approach with two steps:
\begin{itemize}[leftmargin=0.8cm,itemsep=0cm]
    \item Given the current permutation estimates $\hat{\tau}_1, ..., \hat{\tau}_{|\mathcal{X}|}$, update $\hat{\boldsymbol{\gamma}}$ according to \eqref{approximategamma}, with $\tau_x = \hat{\tau}_x$. 
    \item Given the current $\hat{\boldsymbol \gamma}$, maximize w.r.t. the permutations, which is separable across $\tau_1,...,\tau_{|\mathcal{X}|}$:
    \begin{equation}
    \hat{\tau}_x = \underset{\tau \in \mathbb{C}_{|\mathcal{Y}|}}{\text{arg max}} \sum\limits_{y=1}^{|\mathcal{Y}|} N_{x,\tau(y)} \log \hat{\gamma}_y, \;\;\mbox{for $x=1,...,|{\mathcal X}|$.}
    \label{eq: cyclic}
\end{equation}
    This maximization is carried out exactly by considering all the $|\mathcal{Y}|$ cyclic permutations. 
    \end{itemize}
The costs of both steps of this algorithm scale as $O (|\mathcal{X}|\, |\mathcal{Y}|)$. Convergence can be proved via the same approach that is used to prove convergence of the $K$-means algorithm \citep{kmeans}, since both algorithms share a common structure: alternate between exact maximization with respect to real quantities (cluster centers, in K-means, $\boldsymbol{\gamma}$ in our algorithms) and an exact combinatorial optimization (the point-to-cluster assignments in K-means, the cyclic permutations in the proposed algorithms). 

\begin{comment}
\begin{algorithm}
\caption{Algorithm to estimate a UC with unknown permutations}
\label{alg:symmetric}
\KwIn{$|\mathcal{X}|\times |\mathcal{Y}|$ count matrix $\mathbf{N}= [N_{x,y}]$ }
\KwOut{$\hat{\gamma}$ and $\hat{\tau}_x$, for $x= 1,\ldots, |\mathcal{X}|$}
 Find the permutation $\rho_x $ that sorts the entries of the $x$-th row of $\mathbf{N}$ in increasing order, for $x= 2,\ldots, |\mathcal{X}|$\;
Set $\hat{\tau}_1 = (1,2,\ldots, |\mathcal{Y}|)$

 Set initial parameter estimate $\hat{\gamma}$
 
 \Repeat{objective function stops changing}{
 \textbf{Update the permutations:} 
 
 Find the permutation $\eta$ that sorts the components of $\hat{\gamma}$ in increasing order. 
 
 Set $\hat{\tau}_x = \rho_x \circ \eta^{-1}$, for $x= 2,\ldots, |\mathcal{X}|$\;
 
 \textbf{Update the parameters:} Update $\hat{\gamma}$ according to: 
 \begin{equation*}
     \hat{\gamma}_y = \frac{1}{N} \sum_{x\in \mathcal{X}} N_{x, \hat{\tau}_x (y)}, \quad \text{for} \quad y = 1,\ldots,|\mathcal{Y}|.
 \end{equation*}
 
 }
\end{algorithm}
\end{comment}


\begin{comment}
\begin{algorithm}
\small
\SetAlgoLined
\KwIn{$|\mathcal{X}|\times |\mathcal{Y}|$ count matrix $\mathbf{N}= [N_{x,y}]$ }
\KwOut{$\hat{\gamma}$ and $\hat{\tau}_x$, for $x= 1,\ldots, |\mathcal{X}|$}
Set $\hat{\tau}_1 = (1,2,\ldots, |\mathcal{Y}|)$\;
 Set initial parameter estimate $\hat{\gamma}$\;
 \Repeat{objective function stops changing}{
 \textbf{Update the permutations:} 
 
 Find $\hat{\tau}_x $ according to \eqref{eq: cyclic}, for $x=2,\ldots,|\mathcal{X}|$\;
 \textbf{Update $\hat{\gamma}$:} Update $\hat{\gamma}$ according to: 
 \begin{align*}
      \hat{\gamma}_y = \frac{1}{N} \sum_{x\in \mathcal{X}} N_{x, \hat{\tau}_x (y)}, \quad \text{for} \quad y = 1,\ldots,|\mathcal{Y}|\;    
 \end{align*}
 }
 \caption{Algorithm to estimate a CUDC with unknown cyclic permutations}
 \label{alg:cyclic_symmetric}
\end{algorithm}

\end{comment}

\begin{comment}

\subsection{Algorithm Convergence}\label{sec:convergence}
We study the convergence of the algorithms described in Subsection \ref{sec:estimating} by exploiting their structural similarity to the K-means clustering algorithm \citep{kmeans}: alternate between maximization w.r.t. real quantities (cluster centers, in K-means, $\boldsymbol{\gamma}$ in our algorithms) and a combinatorial optimization (the point-to-cluster assignments in K-means, the permutations in our algorithms). 

Before stating and proving the result, we formally present the algorithms in a common framework. Denote the collection of all the unknown permutations as $\boldsymbol \tau = (\tau_2, ..., \tau_{|\mathcal{X}|})$ (recall that $\tau_1 = (1, ..., |\mathcal{Y}|)$) and the objective function in \eqref{thirdscenario}  as $\mathcal{L}(\boldsymbol \gamma, \boldsymbol \tau)$.  The problems of estimating a UC or a CUC can both be written as
\begin{equation}
         \underset{\boldsymbol \gamma, \boldsymbol \tau}{\text{max}} \quad \mathcal{L}(\boldsymbol \gamma, \boldsymbol \tau), \hspace{1cm}\; \text{subject to} \quad  \boldsymbol\tau \in \mathbb{D}, \quad \boldsymbol\gamma \in \Delta_{|\mathcal{Y}| - 1},
    \label{problemP}
\end{equation}
where $\mathbb{D}= (\mathbb{S}_{|\mathcal{Y}|})^{|\mathcal{X}| -1}$, for a UC, or  $\mathbb{D}=(\mathbb{C}_{|\mathcal{Y}|})^{|\mathcal{X}| -1}$, for a CUC. Notice that if $Y$ is binary, there is no difference between using a UC or a CUC, since $\mathbb{S}_2 = \mathbb{C}_2$. With this notation, the algorithms can both be written as Algorithm 1, which also describes more in detail the stopping criterion.

\begin{algorithm}
{\footnotesize
\SetAlgoLined
\caption{Algorithm to Estimate a UC or a CUC}
\label{alg:common}
\KwResult{Estimates $\hat{\boldsymbol \tau}$ and $\hat{\boldsymbol \gamma}$}
Set $t=0$, ${\tt stop} = 0$, and an initial parameter estimate $\hat{\gamma}^{(0)}$\;

Find the initial permutation: $
\hat{\boldsymbol \tau}^{(0)} = \underset{\boldsymbol \tau \in \mathbb{D}}{\text{arg max}} \quad \mathcal{L}(\hat{\boldsymbol \gamma}^{(0)}, \boldsymbol \tau)$\;

 \Repeat{\texttt{stop} = 1}{
Update the parameter estimate: 
\vspace{-0.2cm}\begin{equation}
\hat{\boldsymbol \gamma}^{(t+1)} =\underset{\boldsymbol \gamma \in \Delta_{|\mathcal{Y}|-1}}{\text{arg max}} \quad \mathcal{L}(\boldsymbol \gamma, \hat{\boldsymbol \tau}^{(t)}). \label{b:gamma} 
\end{equation}

\eIf{$\mathcal{L}(\hat{\boldsymbol \gamma}^{(t+1)}, \hat{\boldsymbol \tau}^{(t)}) = \mathcal{L}(\hat{\boldsymbol \gamma}^{(t)}, \hat{\boldsymbol \tau}^{(t)}) $}
{Set $\hat{\boldsymbol \tau} = \hat{\boldsymbol \tau}^{(t)}$ and $\hat{\boldsymbol \gamma} = \hat{\boldsymbol \gamma}^{(t)}$

Set \texttt{stop} = 1
}{Update the permutations:

\vspace{-0.2cm}\begin{equation}
\hat{\boldsymbol \tau}^{(t+1)} = \underset{\boldsymbol \tau \in \mathbb{D}}{\text{arg max}} \quad \mathcal{L}(\hat{\boldsymbol \gamma}^{(t+1)}, \boldsymbol \tau) \label{a:tau}
\end{equation}

\If{$\mathcal{L}(\hat{\boldsymbol \gamma}^{(t+1)}, \hat{\boldsymbol \tau}^{(t+1)}) = \mathcal{L}(\hat{\boldsymbol \gamma}^{(t+1)}, \hat{\boldsymbol \tau}^{(t)}) $}{Set $\hat{\boldsymbol \tau} = \hat{\boldsymbol \tau}^{(t)}$ and $\hat{\boldsymbol \gamma} = \hat{\boldsymbol \gamma}^{(t+1)}$\; 
Set \texttt{stop} = 1}
}}}
\end{algorithm}



\begin{theorem}
Algorithm 1 converges in a finite number of iterations to a POS \textit{(partial optimal solution)} \citep{kmeans},
which is defined as a point $(\boldsymbol \gamma^*, \boldsymbol \tau^*)$ satisfying
\[
  \mathcal{L}(\boldsymbol \gamma^*, \boldsymbol \tau^*) \geq   \mathcal{L}(\boldsymbol \gamma, \boldsymbol \tau^*), \;\; \forall\boldsymbol\gamma \in \Delta_{|\mathcal{Y}| - 1} \hspace{0.5cm} \mbox{and} \hspace{0.5cm}
             \mathcal{L}(\boldsymbol \gamma^*, \boldsymbol \tau^*) \geq   \mathcal{L}(\boldsymbol \gamma^*, \boldsymbol \tau), \;\; \forall\boldsymbol\tau \in \mathbb{D}. 
\]
\end{theorem}

\begin{proof} Since the two steps of the algorithm are solved exactly, it is clear that
\begin{equation}
\mathcal{L}(\hat{\boldsymbol \gamma}^{(t+1)}, \hat{\boldsymbol \tau}^{(t+1)}) \geq \mathcal{L}(\hat{\boldsymbol \gamma}^{(t+1)}, \hat{\boldsymbol \tau}^{(t)}) \geq \mathcal{L}(\hat{\boldsymbol \gamma}^{(t)}, \hat{\boldsymbol \tau}^{(t)}).
\end{equation}
If any of these two inequalities is not strict,  the algorithm stops.
%Moreover, since $\gamma_y \leq 1$, for $y = 1,\ldots, |\mathcal{Y}|$, then $\log \gamma_y \leq 0$. Thus, we have that
%\begin{equation}
%  \mathcal{L}(\boldsymbol \gamma, \boldsymbol \tau) = \sum_{x\in \mathcal{X}}  \sum_{y\in \mathcal{Y}}  N_{x, \tau_x (y)} %\log \gamma_y \leq 0.
%\end{equation}
%Therefore, the sequence $\mathcal{L}(\hat{\boldsymbol \gamma}^{(t)}, \hat{\boldsymbol \tau}^{(t)}), t=1,2,\ldots$ is %monotonically increasing and bounded above. Thus, according to the monotone convergence theorem, the sequence is convergent. 
Note that $(\boldsymbol\gamma^*, \boldsymbol\tau^*)$ is a POS of \eqref{problemP} if and only if $\boldsymbol\gamma^*$ solves \eqref{b:gamma} for $\hat{\boldsymbol\tau}^{(t)} = \boldsymbol\tau^*$ and $\boldsymbol\tau^*$ solves \eqref{a:tau} with $\hat{\boldsymbol\gamma}^{(t+1)} = \boldsymbol\gamma^*$. The set POS thus coincides with that of fixed point of the algorithm. To show convergence to a POS in a finite number of iterations it suffices to show that any point of $\mathbb{D}$ (which is finite) is visited at most once before the algorithm stops. Assume that this was not true: that $\hat{\boldsymbol \tau}^{(t)} = \hat{\boldsymbol \tau}^{(t')}$, for some $t \neq t'$. Updating $\hat{\boldsymbol \gamma}$ according to \eqref{b:gamma} yields $\hat{\boldsymbol \gamma}^{(t+1)}$ and $\hat{\boldsymbol \gamma}^{(t'+1)}$, which satisfy
\begin{equation}
\mathcal{L}(\hat{\boldsymbol \gamma}^{(t+1)}, \hat{\boldsymbol \tau}^{(t)})  =  \mathcal{L}(\hat{\boldsymbol \gamma}^{(t+1)}, \hat{\boldsymbol \tau}^{(t')}) = \mathcal{L}(\hat{\boldsymbol \gamma}^{(t'+1)}, \hat{\boldsymbol \tau}^{(t')}).
\label{sub:wrong}
\end{equation}
where the first equality results from assuming that $\hat{\boldsymbol \tau}^{(t)} = \hat{\boldsymbol \tau}^{(t')}$ and the second one from the fact that $\hat{\boldsymbol \gamma}^{(t+1)}$ and $\hat{\boldsymbol \gamma}^{(t'+1)}$ both result from solving \eqref{b:gamma} (which has a unique solution) with the same $\hat{\boldsymbol \tau}^{(t)} = \hat{\boldsymbol \tau}^{(t')}$. However, as seen above, unless the algorithm stops, we have strict monotonicity at each step, thus \eqref{sub:wrong} cannot be true, contradicting that we can have $\hat{\boldsymbol \tau}^{(t)} = \hat{\boldsymbol \tau}^{(t')}$, for some $t \neq t'$. Finally, $\mathbb{D}$ is  finite, thus the algorithm reaches a POS and stops a finite number of steps.
\end{proof}

\vspace{-0.6cm}
\end{comment}

\section{Applying the UCM Principle from Data}
\label{sec:criterion_data}
Applying the proposed causal inference principle amounts to performing  hypothesis testing concerning the UC nature of the conditional pmf estimates $\hat{\boldsymbol\theta}^{X\rightarrow Y}$ and $\hat{\boldsymbol\theta}^{Y\rightarrow X}$. This is closely related to classical tests for two-way contingency tables \citep{Agresti, Read}. Given a table of counts $N_{x,y}$, let the \textit{null hypothesis} $H_0$ be that these counts can be explained by a UCM in the $X\rightarrow Y$ direction. To test this hypothesis, consider the corresponding maximum log-likelihood (noting that $p_{X,Y}(x,y) = \mathbb{P}[X=x,Y=y] = \theta_x^{X} \, \gamma_{\sigma_x(y)}$, for a UCM),
\begin{equation}
\mathcal{L}_{H_0} = \sum\limits_{x \in \mathcal{X}}\sum\limits_{y \in \mathcal{Y}} N_{x,y} \log\bigl( \hat{\theta}_x^{X} \, \hat{\gamma}_{\hat{\sigma}_x(y)}\bigr) = \sum\limits_{x \in \mathcal{X}} N_x \log  \hat{\theta}_x^{X} + \sum\limits_{x \in \mathcal{X}}\sum\limits_{y \in \mathcal{Y}} N_{x,y} \log\bigl( \hat{\gamma}_{\hat{\sigma}_x(y)}\bigr) ,
\end{equation}
where $\hat{\boldsymbol\theta}^X$, $\hat{\sigma_1},...,\hat{\sigma}_{|\mathcal{X}|}$, and $\hat{\boldsymbol\gamma}$ are the ML estimates obtained as shown in Section \ref{sec:channel}.

The alternative hypothesis is that the channel is arbitrary, with maximum log-likelihood 
\begin{equation}
\mathcal{L}_{\bar{H}_0} = \sum\limits_{x \in \mathcal{X}} N_x \log  \hat{\theta}_x^{X} + \sum\limits_{x \in \mathcal{X}}\sum\limits_{y \in \mathcal{Y}} N_{x,y} \log\bigl( N_{x,y}/N_x\bigr),
\end{equation}
since the ML estimates of the conditional pmf parameters are as given in \eqref{estimate_thetaxy}, and the ML estimate of the marginal $\boldsymbol\theta^X$  is the same, regardless of the channel being uniform or not. These models are nested: a UCM is a particular case of the set of all valid channels, thus it is always true that $\mathcal{L}_{H_0} \leq \mathcal{L}_{\bar{H}_0}$.

The \textit{likelihood-ratio statistic} (LRS), denoted $G^2$, is then given by 
\begin{equation}
G_{X\rightarrow Y}^2 = 2 (\mathcal{L}_{\bar{H}_0} - \mathcal{L}_{H_0}) =  2 \sum\limits_{x \in \mathcal{X}}\sum\limits_{y \in \mathcal{Y}} N_{x,y} \log\Bigl( \frac{N_{x,y}}{\hat{\gamma}_{\hat{\sigma}_x(y)} N_x}\Bigr);
\end{equation}
notice that $\hat{\gamma}_{\hat{\sigma}_x(y)} N_x$ is the expected value of $N_{x,y}$ under the null hypothesis. This is the LRS in the $X\rightarrow Y$ direction, which we indicate with the subscript $X\rightarrow Y$. The LRS in the reverse direction, denoted $G_{Y\rightarrow X}^2$, is computed in the same way, after swapping the roles of $X$ and $Y$.

%Alternatively, we may use the so-called Pearson statistic (usually denoted as $X^2$, but here denoted as $P^2$ since we are using $X$ to denote one of the random variables), 
%\begin{equation}
%P^2 = \sum\limits_{x \in \mathcal{X}}\sum\limits_{y \in \mathcal{Y}} \frac{(N_{x,y} - \hat{\gamma}_{\hat{\sigma}_x(y))} N_x)^2}{\hat{\gamma}_{\hat{\sigma}_x(y)} N_x}.
%\end{equation}

It is well known that $G^2$ is asymptotically $\chi^2$-distributed with df $=(|\mathcal{X}|-1)(|\mathcal{Y}|-1)$ \textit{degrees of freedom}, yielding the ${\tt p}$-value 
\[
{\tt p} = \mathbb{P}[ \chi^2_{\mbox{df}} \geq G^2] = 1-\mathbb{P}[ \chi^2_{\mbox{df}} < G^2],
\]
where $\mathbb{P}[ \chi^2_{\mbox{df}} < G^2]$ is the cumulative distribution function of a $\chi_{\mbox{df}}^2$ distribution. If ${\tt p}$ is less than some significance level (\textit{i.e.}, the test statistic $G^2$ is too large), the null hypothesis is rejected.

Let ${\tt p}^{X\rightarrow Y}$ and ${\tt p}^{Y\rightarrow X}$ be the ${\tt p}-$values of the LRS for testing the uniformity of the channels in both directions, and let $\alpha$ be a significance level for the test \citep{Agresti, Read}, \textit{i.e.}, the null hypothesis is rejected if the ${\tt p}-$value is less than $\alpha$. Having a statistical test of whether an estimated conditional pmf corresponds to a UCM, we adopt a procedure similar to the one proposed by \citet{anm2011}.
    \begin{itemize}
        \item If ${\tt p}^{X\rightarrow Y} \geq \alpha$ and  ${\tt p}^{Y\rightarrow X} < \alpha$, declare $X\rightarrow Y$.
    \item If ${\tt p}^{X\rightarrow Y} < \alpha$ and  ${\tt p}^{Y\rightarrow X} \geq \alpha$, declare $Y\rightarrow X$.
     \item If ${\tt p}^{X\rightarrow Y} < \alpha$ and  ${\tt p}^{Y\rightarrow X} < \alpha$, declare "undecided: wrong model".
     \item If ${\tt p}^{X\rightarrow Y} \geq \alpha$ and  ${\tt p}^{Y\rightarrow X} \geq \alpha$, declare "undecided: both directions possible".
     \end{itemize}

The fourth case (\textit{i.e.}, the hypotheses that the channel is uniform in both directions cannot be rejected) is asymptotically improbable, unless $X$ and $Y$ are independent, due to the identifiability guarantee. Alternatively, to force the method to make a decision between the two causal directions, one may simply decide for $X\rightarrow Y$, if ${\tt p}^{X\rightarrow Y} > {\tt p}^{Y\rightarrow X}$, and for $Y\rightarrow X$, otherwise. 

%The method is summarized in Algorithm \ref{alg:causalinferencemethod}.

%\begin{algorithm}
%\small
%\SetAlgoLined
%\KwIn{$|\mathcal{X}|\times |\mathcal{Y}|$ count matrix $\mathbf{N}= [N_{x,y}]$}
%\KwOut{Causal direction between $X$ and $Y$}
%\begin{enumerate}
%    \item Construct the conditional probability matrix $\boldsymbol \theta^{Y|X}$. Estimate the associated uniformly dispersive channel  using Algorithm \eqref{alg:symmetric} (or \eqref{alg:cyclic_symmetric} if $Y$ is cyclic), by setting $\hat{\gamma}$ to the row-vector $\boldsymbol \theta^{Y|X}_1$.
%    \item  Construct the conditional probability matrix $\boldsymbol \theta^{X|Y}$. Estimate the associated uniformly dispersive channel  using Algorithm \eqref{alg:symmetric} (or \eqref{alg:cyclic_symmetric} if $X$ is cyclic), by setting $\hat{\gamma}$ to the row-vector $\boldsymbol \theta^{X|Y}_1$.
%    \item Calculate $D^{X\rightarrow Y}$ and $D^{Y\rightarrow X}$.
%   \item Decide the causal direction:\\
%    If $D^{X \rightarrow Y}  < D^{Y \rightarrow X}$, output $X \rightarrow Y$;\\[0.1cm]
%    If $D^{X \rightarrow Y}  > D^{Y \rightarrow X}$, output $Y \rightarrow X$;\\
%    Else, output "Inconclusive".
%    \end{enumerate}
% \caption{Causal Inference Method via Least Difference in Log-likelihood (\acs{LDL})}
% \label{alg:causalinferencemethod}
%\end{algorithm}


\section{Results}
\label{sec:resul}
We compare the proposed approach, on synthetic, benchmark, and real data, with two state-of-the-art methods for categorical variables, for which code is publicly available: DC \citep{dc} ({\small \url{eda.mmci.uni-saarland.de/prj/cisc/}}, with $\epsilon = 0$) and HCR \citep{hcr} (a Python version of the $R$ code available at {\small \url{cran.r-project.org/web/packages/HCR/index.html}}). The code for all the experiments will be made available upon acceptance of the manuscript. 

% https://github.com/catsoliveira/LDLforCausalInference

In the ML estimates underlying our approach (namely \eqref{estimate_thetaxy} and \eqref{approximategamma}), to avoid the problem of zero or vanishing probabilities, we use a small amount ($10^{-3}$) of additive (a.k.a. Dirichlet) smoothing. 

%\subsection{Synthetic Data}

\begin{comment}
\subsubsection*{Experiment 1: Estimating Uniform Channels}
We first investigate the performance of Algorithms \ref{alg:symmetric} and \ref{alg:cyclic_symmetric} on synthetic datasets. We simulate data by generating it from a given UC or CUC. Afterwards, we assess the ability of the method to estimate it. We use the Kullback–Leibler divergence (KLD)   \cite{code} to compare the true UC and the estimated one. We use the standard symmetrized KLD, averaged over the rows of $\boldsymbol \theta^{Y|X}$, weighted by the (marginal) probabilities of the rows: 
\begin{equation*}
\small
    \text{KL} \!= \! \frac{1}{2}\sum\limits_{x=1}^{|\mathcal{X}|} 
     \left( \sum\limits_{y=1}^{|\mathcal{Y}|} \theta_{x,y}^{Y|X} \log \frac{\theta_{x,y}^{Y|X}}{\hat{\theta}_{x,y}^{Y|X}} + \sum\limits_{y=1}^{|\mathcal{Y}|} \hat{\theta}_{x,y}^{Y|X} \log \frac{\hat{\theta}_{x,y}^{Y|X}}{\theta_{x,y}^{Y|X}}  \right) \theta_{x}^X.
\end{equation*}

We consider synthetic data for an increasing number of sampled observations $N \in \{50$ , $100$, $200$, $300$, $400$, $500$, $1000$, $1500$, $2000$, $2500$, $3000\}$ and increasingly larger supports $\mathcal{X}$ and $\mathcal{Y}$. We choose the support sizes to be $(|\mathcal{X}|, |\mathcal{Y}|) \in \{(2,2)$, $(3,3)$, $(4,4)$, $(5,5)$, $(2,3)$, $(2,4)$, $(2,5)$, $(3,2)$, $(4,2)$, $(5,2)\}$. For each support setting and each number of samples $N$, we generate $50$ independent datasets. Consequently, the performance results obtained for each $N$, are the average across them.

\begin{figure*}[ht]
\centering
\includegraphics[scale=0.45]{kl.png}
\qquad 
\includegraphics[scale=0.45]{kl_cyclic.png}
\caption{average KL divergence  for the numbers of samples previously mentioned and $|\mathcal{X}| = |\mathcal{Y}| = 2, 3, 4, 5$. Left: UDC. Right: CUDC.}
\label{fig:performance_measure_cyclic}
\end{figure*}

Figure \ref{fig:performance_measure_cyclic}  shows that the KLD between the true UDCs or CUDCs and the estimated ones  increases with the size of the support of the variables, in a manner compatible with the increase of the number of parameters being estimated, while, as expected, it decrease approximately with the inverse of the number of samples. We can thus conclude that Algorithms \ref{alg:symmetric} and \ref{alg:cyclic_symmetric} accurately estimate the underlying UDCs or CUDCs. For support settings in $\{(2,3), (2,4), (2,5), (3,2), (4,2), (5,2)\}$, a similar behaviour is found, thus we omit the corresponding plots. 
%However, one thing worth noticing is that, in general, a larger difference between the variables' support sizes induces a higher KL divergence between the original pmf and the estimated one, although these values only differ in decimal places.

\end{comment}

\subsection{Identifying the UCM Direction}
The first set of experiments is a sanity check, assessing the ability of the proposed criterion to identify the UCM direction, using synthetic data, with different sample sizes $N$ 
%$N \in \{50$ , $100$, $200$, $300$, $400$, $500$, $1000$, $1500$, $2000$, $2500$, $3000\}$, 
and different sizes of the support sets, $|\mathcal{X}|$, and $|\mathcal{Y}|$.
%$(|\mathcal{X}|, |\mathcal{Y}|) \in \{(2,2)$, $(3,3)$, $(4,4)$, $(5,5)$, $(2,3)$, $(2,4)$, $(2,5)$, $(3,2)$, $(4,2)$, $(5,2)\}$.
For each pair $(|\mathcal{X}|, |\mathcal{Y}|)$ and each $N$, we generate $100$ independent datasets using randomly generated UCMs in the $X\rightarrow Y$ direction and the results reported for each $N$ are the corresponding averages. The decision rule is simply to choose $X\rightarrow Y$, if ${\tt p}^{X\rightarrow Y} \geq {\tt p}^{Y\rightarrow X}$ (equivalently, $G_{X\rightarrow Y}^2 \leq G_{Y\rightarrow X}^2$), and $Y\rightarrow X$ (which is wrong), otherwise. The results in Fig. \ref{fig:prob} show that the accuracy achieves high values, close to 100\%, for $N> 500\sim 1000$, without a clear effect of the sizes of the support sets or difference between the non-cyclic and the cyclic cases.  


\begin{figure*}[ht]
\center
\includegraphics[scale=0.49]{accuracy_difsup.png}
\qquad 
\includegraphics[scale=0.49]{accuracy_difsup_cyclic.png}
\vspace{-0.1cm}
\caption{Accuracy in selecting the UCM direction, using the proposed criterion, for different sample and support sizes. Left plot: general UCMs; right plot: cyclic UCMs (CUCM).}
\label{fig:prob}
\end{figure*}

\subsection{Benchmark Data}
We use the 112 pairs in the \textit{cause-effect pairs} benchmark set \citep{causeeffectwebsite} where both variables are categorical and that have as ground truth that either $X \rightarrow Y$ or $Y \rightarrow X$. We set $\alpha = 0.05$ and compare UCM with the two methods mentioned above: DC \citep{dc} and HCR \citep{hcr}. Decisions of ``undecided'' are counted as wrong. The average accuracies of UCM, DC, and HCR reported in Table \ref{tab:results_bench} shows that UCM outperforms both HCR and DC on this dataset. Notice that a random decision would yield accuracy equal to 1/3.


\begin{table*}[hbt]
\centering
     \caption{Average accuracy results on the 112 pairs of the benchmark dataset.}
\vspace{0.5cm}
     \label{tab:results_bench}
     \makebox[\linewidth]{
     \begin{tabular}{| c | c |  c |}
     \hline
     UCM & DC & HCR \\
     \hline\hline
     0.61 & 0.41 & 0. 47\\
     \hline
\end{tabular} }
\end{table*}

\begin{comment}

As is common practice, we plot the accuracy against the fraction of decisions that the methods are forced to make. The pairs are sorted in decreasing order of absolute score difference $|D^{X \rightarrow Y} - D^{Y \rightarrow X}|$ (which quantifies the confidence in the decision), and the accuracy is computed over the top $k\%$ pairs. If $D^{X \rightarrow Y} = D^{Y \rightarrow X}$, the decision is considered wrong. For the other methods, we follow a similar procedure. 

\begin{figure}[hbt]
    \centering
    \includegraphics[scale=0.5]{benchmark2.png}
    \vspace{-0.2cm}
    \caption{Accuracy vs decision rate for categorical pairs in the \textit{Cause-effect pairs} benchmark dataset.}
    \label{fig:dec_rate}
\end{figure}

Fig. \ref{fig:dec_rate} shows that all methods correctly infer the causal direction for the pair for which each method has the highest confidence. At rate 100\% (all pairs), LDL has an accuracy of roughly $59\%$, whereas DC and HCR achieve global accuracy of $41\%$ and $47\%$, respectively. Moreover, LDL outperforms both competing methods for all decision rate above 60\%. Also, it achieves acceptable results even when the decision rate is 1. In conclusion, it can be claimed that, on this dataset, LDL outperforms the competing alternatives.

\end{comment}

\subsection{Real Data}\label{sec:real_data}
Finally, we evaluate the UCM method (again with $\alpha = 0.05$) on real data from the \textit{UCI Machine Learning Repository} \citep{uciwebsite}. We use pairs of variables from the following datasets: Adult, Pittsburgh Bridges, Accute Inflamation, Temperature, and Horse Colic. The datasets and selected pairs, as well as the criteria used to decide what is the ground truth causal direction, are described in Appendix \ref{app_data}. We include only pairs for which a test of independence, at significance level 0.05 \citep{Agresti}, rejects the null hypothesis of independence. Furthermore, we include only pairs for which at least one of the three tested methods chooses one of the causal directions. 
Table \ref{tab:results} shows that the UCM and HCR approaches found the ``correct" causal direction in 5 out of 9 pairs, and DC in 4 pairs. UCM returned only correct decisions or abstained from deciding. This small number of experiments does now allow for reaching any strong conclusions but suggests that UCM performs on par, arguably somewhat better, with DC and HCR.  


\begin{table*}[hbt]
\centering
\normalsize
{
     \caption{Results on real data. Wrong decisions are shown in red; UWM stands for "undecided: wrong model". Month is a cyclic variable, thus a CUC was used in the $Y\rightarrow X$ direction.}
\vspace{-0.2cm}
     \label{tab:results}
     \makebox[\linewidth]{
     \begin{tabular}{ c | c  c  c  c c}
     Dataset & $X$ & $Y$ & UCM & DC & HCR\\[0.1cm]
     \hline
     Adult & Occupation & Income & UWM & $X
\rightarrow Y$ &$X \rightarrow Y$ \\
     Adult & Work Class & Income & UWM & $X
\rightarrow Y$   &$X \rightarrow Y$    \\
      Acute Inflammation & Inflam. of urinary bladder & Lumbar pain & $Y
\rightarrow X$ & Inconcl. &  Inconcl.\\
      Acute Inflammation & Inflam. of urinary bladder & Nausea & $Y
\rightarrow X$ & Inconcl. &  Inconcl.\\
 Acute Inflammation & Inflam. of urinary bladder & Burning urethra & $Y
\rightarrow X$ & Inconcl. &  Inconcl.\\
     Pittsburgh Bridges & Material & Lanes & $X \rightarrow Y$ &
\color{red} $Y \rightarrow X$ & $X \rightarrow Y$ \\
     Pittsburgh Bridges & Purpose & Type & UWM &
\color{red} $Y \rightarrow X$ & \color{black} $X \rightarrow Y$ \\
     Temperature & Month & Temperature & $X \rightarrow Y$ &
$X \rightarrow Y$ & \color{red} $Y \rightarrow X$ \\
     Horse Colic & Abdomen Status & Surgical Lesion & UWM &
$X \rightarrow Y$ & $X \rightarrow Y$ \\
\hline 
     \end{tabular} }}
\end{table*}


\section{Conclusion}
We propose \modelname for real-time instance segmentation. Built on a query-based segmentation framework~\cite{cheng2021mask2former} and three designed efficient components, \ie, instance activation-guided queries, dual-path update strategy, and ground truth mask-guided learning, \modelname achieves excellent performance on the popular COCO dataset while maintaining a fast inference speed. Extensive experiments demonstrate the effectiveness of core ideas and the superiority of \modelname over previous state-of-the-art real-time counterparts. We hope this work can serve as a new baseline for real-time instance segmentation and promote the development of query-based image segmentation algorithms. 

\noindent\textbf{Limitations.} (1) Like general query-based models~\cite{detr, cheng2021mask2former,li2021panopticsegformer}, \modelname is not good at small targets. Even though using stronger pixel decoders or larger feature maps improves it, it introduces heavier computational burdens, and the result is still unsatisfactory. We look forward to an essential solution to handle this problem. (2) although GT mask-guided learning improves the performance of masked attention, it increases training costs. We hope a more elegant method can be proposed to replace it.



% Acknowledgments---Will not appear in anonymized version
%\acks{We thank a bunch of people and funding agency.}

%\bibliographystyle{abbrv}
%\bibliographystyle{abbrvnat}

\clearpage
\bibliography{references}

\clearpage
\appendix
\section{Proof of Theorem \ref{identifiability}}
\label{app_proof}
{\noindent\bf Proof}: Let us denote $\boldsymbol{\theta}^X = \boldsymbol{\beta}  \in \Delta_{|\mathcal{X}|-1}$ and recall that under the UC assumption, matrix $\boldsymbol \theta^{X\rightarrow Y}$ has the form
\[
\boldsymbol \theta^{X\rightarrow Y} = \begin{bmatrix}
\gamma_{\sigma_1(1)} & \gamma_{\sigma_1(2)} & \cdots & \gamma_{\sigma_1(|\mathcal{Y}|)}\\
\gamma_{\sigma_2(1)} & \gamma_{\sigma_2(2)} & \cdots & \gamma_{\sigma_2(|\mathcal{Y}|)} \\
\vdots & \vdots & \ddots & \vdots \\
\gamma_{\sigma_{|\mathcal{X}|}(1)} & \gamma_{\sigma_{|\mathcal{X}|}(2)} & \cdots & \gamma_{\sigma_{|\mathcal{X}|}(|\mathcal{Y}|)}
\end{bmatrix},
\]
where $\sigma_1,...,\sigma_{|\mathcal{X}|}\in \mathbb{S}_{|\mathcal{Y}|}$ are permutations and 
$\boldsymbol{\gamma} = (\gamma_1, ...., \gamma_{|\mathcal{Y}|} )\in \Delta_{|\mathcal{Y}|-1}$. The assumption that the rows of this matrix are not all equal to each other precludes the two following condition from holding: $\sigma_1 = \sigma_2 = \cdots = \sigma_{|\mathcal{X}|}$ and  $\boldsymbol{\gamma} = (1,1,...,1)/|\mathcal{Y}|$.
%Without loss of generality, we can assume that $\sigma_1 = \iota$, the identity permutation, thus $\boldsymbol{\gamma}$ is the first row of the matrix. 

Using Bayes' law, it is trivial to obtain the reverse channel, the elements of which are given by
\begin{equation}
{\theta}_{y,x}^{Y\rightarrow X} = p_{X|Y}(x|y) = \frac{p_{Y|X}(y|x)\, p_X(x)}{p_{Y}(y)} = \frac{ \gamma_{\sigma_x(y)} \, \beta_x }{ \sum_{x'\in\mathcal{X}} \gamma_{\sigma_{x'}(y)} \, \beta_{x'}  } =  \frac{a_{y,x}}{A_y} , \label{eq:reverse_theta}
\end{equation}
where $a_{y,x} = \beta_x \gamma_{\sigma_x(y)} $ and $A_y = p_Y(y) \neq 0$ (by assumption).
As in the binary example, using variables $Y$ and $X$ to index rows and columns, respectively, $\boldsymbol{\theta}^{Y\rightarrow X}$ is a row-stochastic matrix:
\[
\boldsymbol{\theta}^{Y\rightarrow X} = \begin{bmatrix} 
a_{1,1}/A_1  & \cdots & a_{1,|\mathcal{X}|}/A_1\\
\vdots & \ddots & \vdots \\
a_{|\mathcal{Y}|,1}/A_{|\mathcal{Y}|} & \cdots & a_{|\mathcal{Y}|,|\mathcal{X}|}/A_{|\mathcal{Y}|}
\end{bmatrix}.
\]
For $\boldsymbol{\theta}^{Y\rightarrow X}$ to correspond to a UC, its rows must be permutations of each other, which is equivalent to all being permutations of one of them, say the first, without loss of generality. We exclude the case where these permutations are all equal to identity, since that would correspond to all rows of $\boldsymbol{\theta}^{Y\rightarrow X}$ being equal to each other, \textit{i.e.}, $X \perp\!\!\!\perp Y$,  which is excluded in the conditions of the theorem. The condition that all the rows are permutations of the first one can be written formally as
\begin{equation}
\exists (\rho_2, ..., \rho_{|\mathcal{Y}|}) \in \mathbb{L}   : \forall y\in \mathcal{Y}\setminus \{1\}, \,\forall x\in\mathcal{X}, \; a_{1,x} / A_1 = a_{y,\rho_y(x)} / A_y , \label{eq:condition}
\end{equation}
where $\mathbb{L} = (\mathbb{S}_{|\mathcal{X}|})^{|\mathcal{Y}|-1} \setminus \mathbb{I}$, with $\mathbb{I} = \left\{\rho_2, ..., \rho_{|\mathcal{Y}|}: \rho_2 = ... =  \rho_{|\mathcal{Y}|} = \iota \right\}$, and $\iota$ is the identity permutation. In words, $\mathbb{L}$ is the set of all $(|\mathcal{Y}|-1)$-tuples of permutations of $|\mathcal{X}|$ elements, except for the one in which all permutations are identity. 

The equality $a_{1,x} / A_1 = a_{y,\rho_y(x)} / A_y$ is equivalent to $(a_{1,x} \, A_y  - a_{y,\rho_y(x)} \, A_1 )^2 = 0$,
thus the following equivalence holds:
\[
\bigl(\forall y\in \mathcal{Y}\setminus \{1\}, \,\forall x\in\mathcal{X}, \; a_{1,x} / A_1 = a_{y,\rho_y(x)} / A_y\bigr) \;\;\; \Leftrightarrow  \;\;\; Q_{\boldsymbol{\rho}}(\boldsymbol{\theta})  = 0,
\]
with 
\begin{equation}
Q_{\boldsymbol{\rho}}(\boldsymbol{\theta}) = \sum_{y\in \mathcal{Y}\setminus \{1\}} \sum_{x\in\mathcal{X}} (a_{1,x} \, A_y  - a_{y,\rho_y(x)} \, A_1 )^2,\label{eq_Qrho}
\end{equation}
where we have written the model parameters compactly as $\boldsymbol{\theta} = (\boldsymbol{\beta} ,\boldsymbol{\gamma}) \in \Delta_{|\mathcal{X}|-1} \times \Delta_{|\mathcal{Y}|-1} \subset \mathbb{R}^{|\mathcal{X}|} \times \mathbb{R}^{|\mathcal{Y}|}$, and denoted $\boldsymbol{\rho} = ( \rho_2, ..., \rho_{|\mathcal{Y}|} ) \in \mathbb{L}$. A key observation is that $Q_{\boldsymbol{\rho}}(\boldsymbol{\theta})$ is a polynomial in the elements of $\boldsymbol{\theta}$, since the $a_{y,x}$ and the $A_y$ are themselves polynomials (either products of two elements or sums of products of pairs of elements) as is clear in \eqref{eq:reverse_theta}.  

Finally, the existential quantifier  in  \eqref{eq:condition} can be re-written using a product, \textit{i.e.},
\[
\bigl( \exists \boldsymbol{\rho} \in \mathbb{L} :\;  Q_{\boldsymbol{\rho}}(\boldsymbol{\theta}) = 0 \bigr) \;\;\;\Leftrightarrow \;\;  R(\boldsymbol{\theta}) = 0, \hspace{0.7cm} \mbox{where}\;\; R(\boldsymbol{\theta}) \; =\!\!  \prod_{\boldsymbol{\rho} \in \mathbb{L} } Q_{\boldsymbol{\rho}}(\boldsymbol{\theta}).
\]
Since $R(\boldsymbol{\theta})$ a product of polynomials, it is itself a polynomial. Consequently, we have shown that the UC condition in \eqref{eq:condition} corresponds to having $\boldsymbol{\theta}$ as a root of a polynomial. 

The rest of the proof relies on a classical result about polynomials \citep{Federer1969}: let $S:\mathbb{R}^n\rightarrow \mathbb{R}$ be a polynomial that is not identically zero; then, the set $S^{-1}(0) = \{{\bf u}\in\mathbb{R}^n: \, S({\bf u})=0\}$ has zero Labesgue measure in $\mathbb{R}^n$. All that is left to show then is that $R(\boldsymbol{\theta})$ is not identically zero. For this purpose, we can ignore the valid parameter space $\Delta_{|\mathcal{X}|-1} \times \Delta_{|\mathcal{Y}|-1}$, because if $R^{-1}(0)$ has zero Lebesgue measure in $\mathbb{R}^{|\mathcal{X}|} \times \mathbb{R}^{|\mathcal{Y}|}$, so does the intersection $R^{-1}(0) \cap (\Delta_{|\mathcal{X}|-1} \times \Delta_{|\mathcal{Y}|-1})$. We can also ignore the condition  $\boldsymbol{\gamma} \neq (1,...,1)/|\mathcal{Y}|$, since this is a single point, thus a set of zero measure. 

A sufficient and necessary condition for $R(\boldsymbol{\theta})$ not to be identically zero is that none of its factors $Q_{\boldsymbol{\rho}}(\boldsymbol{\theta})$ is identically zero\footnote{Recall that a product of two polynomials with real coefficients is identically zero only if at least one of the factors is identically zero. This is a classical result from abstract algebra, which in the language thereof is stated as follows: the ring of all polynomials in $n$ variables with real coefficients is an \textit{integral domain} or \textit{entire ring}, that is, it does not have divisors of zero \citep{Lang}. The result generalizes trivially, by induction, to products of more than two polynomials.}. To show that no $Q_{\boldsymbol{\rho}}(\boldsymbol{\theta})$ is  identically zero, let us write it explicitly, using the definitions of $a_{y,x}$ and $A_y$ in \eqref{eq:reverse_theta}:
\begin{equation}
Q_{\boldsymbol{\rho}}(\boldsymbol{\theta}) = \sum_{y\in \mathcal{Y}\setminus \{1\}} \sum_{x\in\mathcal{X}} \Bigl( \beta_x \gamma_{\sigma_x(1)} \sum_{x'\in \mathcal{X}} \beta_{x'} \gamma_{\sigma_{x'}(y)} - \beta_{\rho_{y}(x)} \gamma_{\sigma_{\rho_y(x)}(y)}\sum_{x'\in \mathcal{X}} \beta_{x'} \gamma_{\sigma_{x'}(1)} \Bigr)^2.\label{eq_Qrho2}
\end{equation}
Since $Q_{\boldsymbol{\rho}}(\boldsymbol{\theta})$ is a sum of non-negative terms, to show that it is not identically zero, it suffices to show that one of the terms in the sum is strictly positive for some choice of $\boldsymbol{\theta}$. The condition $\boldsymbol{\rho} = (\rho_2,...,\rho_{|\mathcal{Y}|}) \in \mathbb{L}$ means that at least one of the permutations  $\rho_2,...,\rho_{|\mathcal{Y}|}$ is not the identity, which implies that there is at least one pair $(x,y)$ such that $\rho_y (x) \neq x$. Let $y$ and $x$ be one such pair. Choosing $\boldsymbol{\gamma} = (1,...,1)$ and $\boldsymbol{\beta} \in \Delta_{|\mathcal{X}|-1}$ such that all components are different from each other ($i\neq j\, \Rightarrow \beta_i \neq \beta_j$), we have (noticing that $\sum_{x'\in\mathcal{X}}\beta_{x'} = 1$)
\begin{align}
Q_{\boldsymbol{\rho}}(\boldsymbol{\theta}) & = \Bigl( \beta_x  \sum_{x'\in \mathcal{X}} \beta_{x'} - \beta_{\rho_{y}(x)} \sum_{x'\in \mathcal{X}} \beta_{x'} \Bigr)^2 +  
\sum_{y' \neq y } \sum_{x'\neq x } (\cdots)^2 \label{eq_Qrho3}\\
& = \bigl( \beta_x - \beta_{\rho_{y}(x)} \bigr)^2 +\; \mbox{non-negative terms} > 0.
\end{align}
In conclusion, since none of the $Q_{\boldsymbol{\rho}}(\boldsymbol{\theta})$ polynomials is identically zero, $R(\boldsymbol{\theta})$ is also not identically zero, consequently its zero set has zero Lebesgue measure. 
\hfill $\blacksquare$

\section{Proof of Proposition \ref{proof_UCestimate}}
\label{proof_estimate}
{\noindent\bf Proof}: Noticing that the permutations that map $\boldsymbol{\gamma}$ to each row of $\boldsymbol{\theta}^{Y|X}$ are arbitrary, there is no loss of generality in assuming $\gamma_1 \geq \gamma_2 \geq \cdots \geq \gamma_{|\mathcal{Y}|}$, \textit{i.e.}, $\boldsymbol{\gamma}\in \mathcal{K}_{|\mathcal{Y}|}$, the so-called \textit{monotone cone} \citep{Best1990}. 
Furthermore, it is more convenient to formulate the problem w.r.t. the inverse permutations, denoted as $\tau_1,\ldots, \tau_{|\mathcal{X}|} \in \mathbb{S}_{|\mathcal{Y}|}$. The problem can thus be written as
\begin{align}
    \hat{\boldsymbol \gamma}, \hat{\tau}_1, \ldots, \hat{\tau}_{|\mathcal{X}|} \;  = \hspace{-0.5cm} \underset{ \begin{array}{c} \boldsymbol \gamma \in (\Delta _{|\mathcal{Y}|-1} \cap \mathcal{K}_{|\mathcal{Y}|}) \\ 
    \tau_1,...,\tau_{|\mathcal{X}|} \in \mathbb{S}_{|\mathcal{Y}|}\end{array} }{\text{arg max}} \hspace{-0.5cm} \mathcal{L}(\boldsymbol \gamma, \tau_1,...,\tau_{|\mathcal{X}|} ), \label{thirdscenarioc}
\end{align}
where 
\begin{align}
    \mathcal{L}(\boldsymbol \gamma, \tau_1,...,\tau_{|\mathcal{X}|} ) = \sum_{x\in \mathcal{X}} \sum_{y\in \mathcal{Y}}  N_{x,\tau_x (y)}\log \gamma_y. \label{thirdscenario_bc}
\end{align}
 
The assumption $\gamma_1 \geq \gamma_2 \geq \cdots \geq \gamma_{|\mathcal{Y}|}$ makes the  maximization w.r.t. $\tau_1,...,\tau_{|\mathcal{X}|}$ independent of the particular values of $\boldsymbol{\gamma}$ as well as separable into a collection of $|\mathcal{X}|$ independent maximizations, 
\begin{equation}
    \hat{\tau}_x = \underset{ \tau_x \in  \mathbb{S}_{|\mathcal{Y}|}}{\text{arg max}} \sum_{y\in \mathcal{Y}}  N_{x,\tau_x (y)}\log \gamma_y,\label{thirdscenario2}
\end{equation}
for $x\in\mathcal{X}$. Solving \eqref{thirdscenario2} is a simple application of the \textit{rearrangement inequality}\footnote{Given any two non-decreasing sequences of $n$ real numbers, $x_1 \leq \ldots \leq x_n$ and $y_1 \leq \ldots \leq y_n$,  
\[
\forall\sigma \in \mathbb{S}_n, \;\; \sum\limits_{i=1}^n x_{\sigma_{(i)}} y_i \leq \sum\limits_{i=1}^n x_i y_i.
\]} \citep{rearrangementinequality}. Since $\log \gamma_1 \geq \log \gamma_2 \geq \cdots \geq \log \gamma_{|\mathcal{Y}|}$, the solution $\hat{\tau}_x$ is any permutation that also sorts $\{N_{x,1},..., N_{x,|\mathcal{Y}|}\}$ into non-increasing order:
\begin{equation}
\hat{\tau}_x\;\; \mbox{is such that}\;\; N_{x, \hat{\tau}_x (1)} \geq \cdots \geq N_{x, \hat{\tau}_x (|\mathcal{Y}|)}.\label{sortedNx}
\end{equation}
If all the elements of $\{N_{x,1},..., N_{x,|\mathcal{Y}|}\}$ are different, the optimal permutation is unique; otherwise, there are several optimal permutations, all achieving the same maximum. The cost of finding $\hat{\tau}_1, \ldots, \hat{\tau}_{|\mathcal{X}|}$ is $O(|\mathcal{X}|\, |\mathcal{Y}|\, \log |\mathcal{Y}|)$, since it requires $|\mathcal{X}|$ sorting operations, each with $|\mathcal{Y}|$ elements. 

Plugging $\hat{\tau}_1, \ldots, \hat{\tau}_{|\mathcal{X}|}$ back into \eqref{thirdscenario}--\eqref{thirdscenario_b} and swapping the summation order, yields 
\begin{align}
    \hat{\boldsymbol{\gamma}} \;\;  = \!\! \underset{\boldsymbol \gamma \in (\Delta _{|\mathcal{Y}|-1} \cap \mathcal{K}_{|\mathcal{Y}|}) }{\text{arg max}} \sum_{y\in \mathcal{Y}}  \log \gamma_y \sum_{x\in \mathcal{X}}  N_{x,\hat\tau_x (y)}. \label{thirdscenario3}
\end{align}
This problem is the same as \eqref{eq:UDC_known}, with $\hat\tau_x$ in the place of $\tau_x$ and with the additional constraint $\boldsymbol{\gamma} \in \mathcal{K}_{|\mathcal{Y}|}$. Temporarily ignoring this constraint leads to (see \eqref{approximategamma}), 
\begin{equation}
    {\displaystyle \hat{\gamma}_y = \frac{1}{N}\sum\limits_{x\in \mathcal{X}} N_{x, \hat\tau_x (y)}}, \;\; \; \mbox{for $y \in \mathcal{Y}$}.
\label{approximategamma2}
\end{equation}
The fact that $N_{x, \hat{\tau}_x (1)} \geq N_{x, \hat{\tau}_x (2)} \geq \cdots \geq N_{x, \hat{\tau}_x (|\mathcal{Y}|)}$, for any $x\in\mathcal{X}$, implies that $\hat{\boldsymbol{\gamma}} \in \mathcal{K}_{|\mathcal{Y}|}$, without having to include this constraint.  Consequently, problem \eqref{thirdscenario}--\eqref{thirdscenario_b}  has a global solution given by \eqref{approximategamma2} and \eqref{sortedNx}.
\hfill $\blacksquare$

\section{Detailed Description of the Datasets Used in Section \ref{sec:real_data}}
\label{app_data}
\textbf{Adult} - This dataset consists of 48832 records from the census database of the US in 1994. We consider the following pairs: \textit{(occupation, income)} and \textit{(work class,income)}. The variable \textit{occupation} takes values in the set $\{$\textit{admin, armed-force, blue-collar, white-collar, service, sales, professional}, \textit{other-occupation}$\}$. The variable \textit{work class} takes categories in $\{$\textit{private, self-employed, public servant, unemployed}$\}$. Finally, \textit{income} is a binary variable taking value in $\{ >50,\, \leq 50\}$. Following \citet{paper_stochastic}, we assume that the ground truth is \textit{occupation} $\rightarrow$ \textit{income} and \textit{work class} $\rightarrow$ \textit{income}.

\textbf{Pittsburgh Bridges} - This dataset contains records about 108 bridges and some of their characteristics. We consider the  pairs (\textit{purpose}, \textit{type}) and (\textit{material},\textit{lanes}). The variable \textit{purpose} takes values in $\{$\textit{Walk, Aqueduct, RR, Highway}$\}$; variable \textit{type} takes values in $\{$\textit{Wood, Suspen, Simple-T, Arch, Cantilev, CONT-T}$\}$; variable \textit{material} takes values in $\{$\textit{Steel, Iron, Wood}$\}$; variable \textit{lanes} takes values in $ \{1,2,4,6\}$. Following \citet{hcr}, the ground truths assumed are \textit{material} $\rightarrow$ \textit{lanes} and \textit{purpose} $\rightarrow$ \textit{type}. 

\textbf{Acute Inflammations} - This dataset consists of 120 patients and whether each patient is experiencing a specific symptom, the temperature, and whether he/she suffers from acute inflammations of the urinary bladder and/or acute nephritis. We consider the binary variables \textit{ocurrence of nausea} $(Y_1)$, \textit{lumbar pain} $(Y_2)$, and \textit{burning of urethra} $(Y_3)$ (naturally, taking value 1 if the patient has that symptom, and 0 otherwise). Moreover, $X$ represent the diagnosis \textit{inflammation of urinary bladder}. Following \citet{anm2011}, the goal is to model the diagnosis process, thus we expect $Y_j \rightarrow X$, for $j = 1,...,3$. Notice that the variables $X_i$ only correspond to the diagnosis, not necessarily the truth, otherwise they would be considered the cause, rather than the effect.  

The $\chi^2$ test did not reject the null hypothesis of independence (at a significance level of 5\%) for the following pairs of variables in this dataset: (\textit{Inflammation of urinary bladder}, \textit{Occurrence of nausea}); (\textit{Inflammation of urinary bladder}, \textit{Burning of urethra}); (\textit{Nephritis}, \textit{Micturition pains}). Although, intuitively, we would expected a causal relation between those, the statistical evidence is not strong enough in favour of their mutual dependency (according to the $\chi^2$ test), therefore, these pairs were discarded from the experiments. 

\textbf{Temperature} - This dataset consists of 9162 daily values of temperature measured in Furtwangen (Germany), with the variable \textit{day of the year} taking integer values from 1 to 365 (or 366 for leap years) and \textit{temperature} in $^{\circ} C$. Here, we aggregate days associated with each month and take \textit{month} $\rightarrow$ \textit{temperature} as the ground truth, as \citet{anm2011}. Notice that \textit{month} assumes a cyclic structure.

\textbf{Horse Colic} - This dataset contains 368 medical records of horses. We study the causal relationship between the variable \textit{abdomen status}, which takes the values in $\{$\textit{Normal, Other, Firm feces in the large intestine, Distended small intestine, Distended large intestine}$\}$, and the binary variable \textit{surgical lesion}, indicating whether the lesion was surgical or not. As \citet{horsedata}, we regard \textit{abdomen status} $\rightarrow$ \textit{surgical lesion} as ground truth.

%This is a boring technical proof.

%\section{My Proof of Theorem 2}

%This is a complete version of a proof sketched in the main text.

\end{document}
