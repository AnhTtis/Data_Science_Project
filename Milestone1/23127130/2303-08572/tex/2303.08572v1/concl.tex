\vspace{-0.3cm}
\section{Conclusions}
\label{sec:concl}
We introduced the \textit{uniform channel model} (UCM) to address the cause-effect problem with categorical variables. The proposed approach is based on viewing conditional distributions as communication channels. The UCM can be seen as an ANM-type instantiation of the principle of \textit{independence of cause and mechanism}, preserving a key feature of ANM for quantitative data: the conditional entropy (uncertainty) of the effect given the cause is independent of the cause. The core results of this paper are a proof of the identifiability of the UCM and a proof of its equivalence to a structural causal model with an exogenous variable of fixed cardinality.

To instantiate the approach on finite data, we used classical statistical tests to decide in which of the two directions (if any) the conditional distribution is close enough to correspond to a UCM. The experimental results confirmed the adequacy of the proposed method. By comparing our method with two other recent methods (DC, by \citet{dc}, and HCR, by \citet{hcr}), we found that UCM outperforms those other methods on benchmark datasets and performs on par with those methods on real data.  

As future work, we will aim to extend the proposed method to handle more than two variables. For example, closeness to a uniform channel of the conditional distribution of each variable given its parents can be used in a score-based method. Another direction of research will look at  cases where one variable is categorical and the other is continuous.  In fact, we envision a generalization that subsumes both ANM and the proposed UCM as follows. In the correct causal direction, the conditional distribution of the effect should be equivariant under some transformation group (the element of which is selected by the cause) that is relevant to its domain: shifts, in the ANM case, permutations, for categorical variables, and cyclic permutations for variables with cyclic structure. If this equivariance holds in the causal direction but not in the reverse one, we have identifiability.

\section*{Acknowledgments}
We thank Afonso Bandeira, André Gomes, Francisco Andrade, João Xavier, and José Mourão, for fruitful conversations and important (some of them instrumental) suggestions. We also thank the anonymous reviewers for some important suggestions. 
