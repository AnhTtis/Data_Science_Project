
Discrete-valued data arise in diverse applied scientific areas, ranging from finance to molecular biology and epidemiology. 
%A common problem arising in different scientific disciplines is variable selection in discrete-valued data. 
For example, as discussed in~\cite{wu2017diversity}, in molecular biology, non-coding genes are emerging as potential key regulators of the expression of protein-coding genes. Yet, among numerous non-coding genes, only a few are likely to be involved for explaining the expression of the coding genes. Consequently, variable selection will help to identify the relevant non-coding genes by obtaining sparse estimators, meaning that most of them are zero. A  popular approach in statistics for performing variable selection is the Lasso proposed by \cite{tibshirani:1996}. %This leads to more accurate and interpretable models. Meanwhile, Lasso benefits from a low computational load, whereas classical model selection methods are computationally expensive. 
Besides, \cite{zhao:2006} showed that Lasso has theoretical guarantees, under some mild conditions. More particularly, Lasso is model selection consistent, meaning that Lasso chooses the true model. However, the consistency results are established in a Gaussian setting, which may not hold for discrete-valued data. 

In this work we consider the following sparse Poisson model. Let $Y_1, \ldots, Y_n$ be independent random variables such that for all $i$, 
\begin{equation}
Y_i\sim\texttt{Poisson}(\lambda^{\star}_i) \quad \text{with} \quad \lambda^{\star}_i = \exp(x_i\pmb{\beta}^{\star}), 
\label{eq_model}
\end{equation}
where $x_i$ is the $i$th row of a $n \times p$ design matrix $\mathbf{X}$ and $\pmb{\beta}^{\star}$ is a sparse vector of regression coefficients in $\mathbb{R}^p$. The non-null coefficients correspond to the predictors that are relevant to explain the response. In the following we will consider an estimation method inspired by \cite{friedman_hastie_tibshirani:2010} and provide a novel sign consistency result.

The paper is organised as follows. Firstly, in Section 2 we present the statistical approach for estimating $\pmb{\beta}^{\star}$. Next, in Section 3 we establish its sign-consistency. The detailed proof is available in the Appendix.
