%% 
%% Copyright 2007-2020 Elsevier Ltd
%% 
%% This file is part of the 'Elsarticle Bundle'.
%% ---------------------------------------------
%% 
%% It may be distributed under the conditions of the LaTeX Project Public
%% License, either version 1.2 of this license or (at your option) any
%% later version.  The latest version of this license is in
%%    http://www.latex-project.org/lppl.txt
%% and version 1.2 or later is part of all distributions of LaTeX
%% version 1999/12/01 or later.
%% 
%% The list of all files belonging to the 'Elsarticle Bundle' is
%% given in the file `manifest.txt'.
%% 

%% Template article for Elsevier's document class `elsarticle'
%% with numbered style bibliographic references
%% SP 2008/03/01
%%
%% 
%%
%% $Id: elsarticle-template-num.tex 190 2020-11-23 11:12:32Z rishi $
%%
%%
%\documentclass[preprint,12pt]{elsarticle}

%% Use the option review to obtain double line spacing
%% \documentclass[authoryear,preprint,review,12pt]{elsarticle}

%% Use the options 1p,twocolumn; 3p; 3p,twocolumn; 5p; or 5p,twocolumn
%% for a journal layout:
 \documentclass[final,1p,times]{elsarticle}
%% \documentclass[final,1p,times,twocolumn]{elsarticle}
%% \documentclass[final,3p,times]{elsarticle}
%% \documentclass[final,3p,times,twocolumn]{elsarticle}
%% \documentclass[final,5p,times]{elsarticle}
%% \documentclass[final,5p,times,twocolumn]{elsarticle}

%% For including figures, graphicx.sty has been loaded in
%% elsarticle.cls. If you prefer to use the old commands
%% please give \usepackage{epsfig}

%% The amssymb package provides various useful mathematical symbols
\usepackage{amsmath,amsthm,amssymb,enumitem,hyperref}
\usepackage{amsfonts}
%\usepackage{mathtext}
%\usepackage{mathrsfs}
\usepackage{textcomp}
\usepackage{xcolor}
\usepackage[utf8]{inputenc}
%% The amsthm package provides extended theorem environments
%% \usepackage{amsthm}

%% The lineno packages adds line numbers. Start line numbering with
%% \begin{linenumbers}, end it with \end{linenumbers}. Or switch it on
%% for the whole article with \linenumbers.
%% \usepackage{lineno}

\newtheorem{theorem}{Theorem}[section]
\newtheorem{corollary}{Corollary}[theorem]
\newtheorem{lemma}[theorem]{Lemma}
\newtheorem{prop}[theorem]{Proposition}

\newcommand{\bracenom}{\genfrac{\lbrace}{\rbrace}{0pt}{}}

%\journal{Nuclear Physics B}

\usepackage{ulem}
\newcommand{\LS}[1]{\textcolor{blue}{#1}}

\begin{document}

\begin{frontmatter}

%% Title, authors and addresses

%% use the tnoteref command within \title for footnotes;
%% use the tnotetext command for theassociated footnote;
%% use the fnref command within \author or \address for footnotes;
%% use the fntext command for theassociated footnote;
%% use the corref command within \author for corresponding author footnotes;
%% use the cortext command for theassociated footnote;
%% use the ead command for the email address,
%% and the form \ead[url] for the home page:
%% \title{Title\tnoteref{label1}}
%% \tnotetext[label1]{}
%% \author{Name\corref{cor1}\fnref{label2}}
%% \ead{email address}
%% \ead[url]{home page}
%% \fntext[label2]{}
%% \cortext[cor1]{}
%% \affiliation{organization={},
%%             addressline={},
%%             city={},
%%             postcode={},
%%             state={},
%%             country={}}
%% \fntext[label3]{}

\title{Sign-consistent estimation in a sparse Poisson model}

%% use optional labels to link authors explicitly to addresses:
%% \author[label1,label2]{}
%% \affiliation[label1]{organization={},
%%             addressline={},
%%             city={},
%%             postcode={},
%%             state={},
%%             country={}}
%%
%% \affiliation[label2]{organization={},
%%             addressline={},
%%             city={},
%%             postcode={},
%%             state={},
%%             country={}}

\author{M. Gomtsyan, C. L\'evy-Leduc, S. Ouadah, L. Sansonnet}

\affiliation{organization={Université Paris-Saclay, AgroParisTech, INRAE, UMR MIA Paris-Saclay},%Department and Organization
            addressline={22 place de l'agronomie}, 
            city={Palaiseau},
            postcode={91120}, 
            country={France}}

\begin{abstract}
In this work, we consider an estimation method in sparse Poisson models inspired by \cite{friedman_hastie_tibshirani:2010} and provide novel sign consistency results under mild conditions.  
\end{abstract}

%%Graphical abstract
%\begin{graphicalabstract}
%\includegraphics{grabs}
%\end{graphicalabstract}

%%Research highlights
%\begin{highlights}
%\item Research highlight 1
%\item Research highlight 2
%\end{highlights}

\begin{keyword}
%% keywords here, in the form: keyword \sep keyword
Lasso \sep sign-consistency \sep variable selection \sep Poisson model

%% PACS codes here, in the form: \PACS code \sep code

%% MSC codes here, in the form: \MSC code \sep code
%% or \MSC[2008] code \sep code (2000 is the default)

\end{keyword}

\end{frontmatter}

%% \linenumbers

%% main text
\section{Introduction}
\section{Introduction}

The increasing complexity of source code poses a key challenge to the reliability of large-scale software systems. Software bugs in these systems can lead to safety issues~\cite{bug_safety} for users around the world as well as cause non-negligible financial losses~\cite{bug_loss}. As such, developers have to spend a large amount of time and effort on bug fixing. Consequently, \aprfull (\apr), designed to automatically generate patches to fix software bugs, has attracted wide attention from both academia and industry~\cite{long2016prophet, legoues2012genprog, long2015spr, lou2020can, tufano2018empstudy}. 


To achieve \apr, one popular approach is known as Generate-and-Validate (G\&V)~\cite{qi2015gv, ghanbari2019prapr, lou2020can, le2016hdrepair, legoues2012genprog, wen2018capgen, hua2018sketchfix, martinez2016astor, koyuncu2020fixminder, liu2019tbar, liu2019avatar}, which is typically based on the following pipeline: First, fault localization techniques~\cite{wong2016fl, abreu2007ochiai, zhang2013injecting, papadakis2015metallaxis, li2019deepfl, li2017transforming} are applied to determine the suspicious locations in programs where bugs are likely to exist. Then, the buggy locations are used by the \apr tools to generate a list of patches that replace buggy lines with correct lines. Afterward, each patch is validated against the original test suite to identify any \emph{plausible patches} (i.e., passing all tests in the test suite). Finally, to determine the \emph{correct patches}, developers examine the list of plausible patches to see if any of them can correctly fix the bug. 

Traditional \apr tools can mainly be categorized into heuristic-based~\cite{legoues2012genprog, le2016hdrepair, wen2018capgen}, constraint-based~\cite{mechtaev2016angelix, le2017s3, demacro2014nopol, long2015spr} and \template~\cite{ghanbari2019prapr, hua2018sketchfix, martinez2016astor, liu2019tbar, liu2019avatar}. Among these traditional tools, \template \apr tools~\cite{ghanbari2019prapr, liu2019tbar, benton2020effectiveness} have been able to achieve state-of-the-art results. \Template \apr tools typically leverage pre-defined templates (e.g., adding a nullness check) for bug fixing. However, since these fix templates are typically handcrafted, the number and types of bugs they are able to fix can be limited. 



To address the limitations of traditional \apr, researchers have proposed various \learning \apr tools~\cite{li2020dlfix, chen2018sequencer, jiang2021cure, lutellier2020coconut, zhu2021recoder, ye2022rewardrepair} based on the \nmtfull (\nmt) architecture~\cite{sutskever2014mt} where the input is the buggy code snippets and the goal is to translate the buggy code snippets into a fixed version. To accomplish this, \learning \apr tools require supervised training datasets with pairs of both buggy and fixed code snippets in order to learn how to perform this translation step. These training data are usually obtained by mining historical bug fixes using heuristics/keywords~\cite{dallmeier2007benchmark}, which can be imprecise for identifying bug-fixing commits; even the actual bug-fixing commits can include irrelevant code changes, leading to further pollution in the dataset~\cite{xia2022alpharepair}.
% 
Moreover, it can be hard for such \apr tools to generalize and fix bug types unseen during training. 



To better leverage recent advances in \plmfull{s} (\plm{s}), researchers~\cite{xia2022alpharepair, xia2023repairstudy, kolak2022patch, prenner2021codexws} have directly applied \plm{s} to generate patches without bug-fixing datasets. These \llm-based \apr tools work by either directly generating a complete code function~\cite{prenner2021codexws, xia2023repairstudy} or predict/infill the correct code snippet given its surrounding context~\cite{xia2022alpharepair, xia2023repairstudy}. By directly using \llm{s} that are pre-trained on billions of open-source code snippets, \llm-based \apr tools can achieve state-of-the-art performance on many repair datasets~\cite{xia2022alpharepair}. 


% 
%
%

Traditional \apr tools have long used the insight of the \emph{plastic surgery hypothesis}~\cite{barr2014plastic} where it states that the code ingredients to fix a bug already exist within the same project. Traditional \apr tools have manually designed pattern-~\cite{ghanbari2019prapr, saha2017elixir} or heuristic-based~\cite{jiang2018simfix, legoues2012genprog} approaches to finding and using such relevant code ingredients to generate fixes for bugs. However, the plastic surgery hypothesis has been largely ignored in \llm-based \apr. In fact, \llm provides a unique opportunity to fully automate the plastic surgery hypothesis idea via fine-tuning (learning project-specific information via model updates from the buggy project) and prompting (directly providing relevant code ingredients to the model), and make it directly applicable to different languages (since the \llm{s} are typically multi-lingual).%
Moreover, despite the intensive manual efforts involved, traditional \apr tools still cannot fully leverage project-specific information due to large search space for leveraging/composing existing code ingredients. In contrast, the project-specific information can effectively leveraged by \llm{s} due to their power in code understanding/vectorization, e.g., even partial/imprecise information may still guide \llm{s} in correct patch generation!
 To this end, we ask the question: \emph{How useful is the plastic surgery hypothesis in the era of \plm{s}}?








\mypara{Our Work.} To answer the question, we present \ourtech{\xspace} -- a \llm-based approach that automatically utilizes the plastic surgery hypothesis by systematically combining multiple fine-tuning and prompting strategies for \apr. \ourtech fine-tunes \plm{s} using two novel domain-specific training strategies: \textbf{\epfinetune} -- we fine-tune using the original buggy project by aggressively masking out a high percentage of tokens, which allows \plm to learn project-specific code tokens and programming styles; and \textbf{\rofinetune} -- which only masks out a single continuous code sequence per training sample, allowing the model to get used to the final \csapr task of predicting a single continuous code sequence. Furthermore, we directly leverage the ability for \plm{s} to understand natural language instructions and introduce a novel prompting strategy, \textbf{\idprompting}, which uses information retrieval and static analysis to obtain a list of relevant identifiers for the buggy lines. While such relevant identifiers are critical for fixing some difficult bugs, they may not be seen by the \llm during inference due to limited context window size. Through the use of prompting, we directly tell the model to use these extracted identifiers (relevant code ingredients) to generate the correct code. Finally, to perform repair, we combine all four model variants (including the base model, both fine-tuned models and the base model with prompting) for the final repair.





While our insight of leveraging the plastic surgery hypothesis for \llm-based \apr is generalizable across different types of \plm{s}, to implement \ourtech, we choose a recent \plm{\xspace}, \ctfive~\cite{wang2021codet5}, which is pre-trained on millions of open-source code snippets. \ctfive is an encoder-decoder model trained using \mspfull (\msp) objective where a percentage of tokens are masked out and each continuous masked token sequence is referred to as a masked span. Also, although we only extract relevant identifiers from the current buggy project (since this paper focuses on the plastic surgery hypothesis), our work can be easily extended to obtain other code information (such as relevant statements or functions) from other sources, such as  the massive pre-training corpora~\cite{husain2020codesearchnet} or historical bug-fixing datasets~\cite{jiang2019infer}, which can provide more coding knowledge for \llm{s}. Besides, although we mainly focus on using traditional string comparison algorithms for information retrieval in this paper, these techniques can be easily replaced by other frequency-based retrieval~\cite{robertson2009probabilistic} and neural search (or embedding-based search)~\cite{reimers2019sentence}.
  In summary, this paper makes the following contributions:


%


\begin{itemize}[noitemsep, leftmargin=*, topsep=0pt]
    \item \textbf{Dimension.} This paper is the first to revisit the important plastic surgery hypothesis in the era of \llm{s}. It opens up a new dimension for \llm-based \apr to incorporate previously neglected information from the buggy project itself to boost \apr performance. Furthermore, it demonstrates the promising future of retrieval-based prompting for modern \llm-based \apr.
    \item \textbf{Implementation.} We implement \ourtech based on the recent \ctfive model. We augment the model using two novel fine-tuning strategies: \epfinetune and \rofinetune, along with a novel prompting strategy based on information retrieval and static analysis: \idprompting. We combine the patches generated by all four models together and perform patch ranking to speed up \apr.% 
    \item \textbf{Evaluation Study.} We conduct an extensive evaluation against state-of-the-art \apr tools. On the widely studied \dfj 1.2 and 2.0 datasets~\cite{just2014dfj}, \ourtech is able to achieve the new state-of-the-art results of 89 and 44 correct bug fixes (15 and 8 more than best baseline) respectively.  Furthermore, we perform a broad ablation study to justify our design. \ourtech demonstrates for the first time that the plastic surgery hypothesis can substantially boost \llm-based \apr and advance state-of-the-art \apr, while being fully automated and general. Moreover, even partial/imprecise code ingredients may still effectively guide \llm{s} for \apr!
\end{itemize}



\section{Statistical approach}

% Let $Y_1, \ldots, Y_n$ be independent random variables such that for all $i$, $Y_i\sim\texttt{Poisson}(\lambda^{\star}_i)$ with $\lambda^{\star}_i = \exp(x_i\pmb{\beta}^{\star})$, where $x_i$ is the $i$th row of a $n \times p$ design matrix $\mathbf{X}$ and $\pmb{\beta}^{\star}$ is the vector of regression coefficients in $\mathbb{R}^p$. The model is assumed to be \textit{sparse}, namely many $\beta_k^{\star}$ are null. They correspond to the
% predictors that are irrelevant to explain the response.
To estimate $\pmb{\beta}^{\star}$ defined in Model (\ref{eq_model}), we shall use the approximation proposed by \cite{friedman_hastie_tibshirani:2010} which consists in maximizing with respect to $\pmb{\beta}$ the second order Taylor approximation of the log-likehood $l$ at the current estimate $\pmb{\Tilde{\beta}}$, namely:
\begin{equation*}
  l(\pmb{\Tilde{\beta}}) +  \sum_{i=1}^n \sum_{k=1}^p{x_i}_k (Y_i - \Tilde{\lambda}_i) (\beta_k - \Tilde{\beta}_k)
  -\frac{1}{2}\sum_{i=1}^n  \sum_{1\leq k,\ell\leq p}  \Tilde{\lambda}_i  {x_i}_k (\beta_k - \Tilde{\beta}_k) {x_i}_\ell (\beta_\ell - \Tilde{\beta}_\ell),
    %\label{eq2}
\end{equation*}   
where $\Tilde{\lambda}_i = \exp(x_i  \pmb{\Tilde{\beta}})$, %  It is thus equivalent to maximize the following expression with respect to $\pmb{\beta}$:
% \begin{equation*}
%   \sum_{i=1}^n(Y_i - \Tilde{\lambda}_i) x_i(\pmb{\beta} - \pmb{\Tilde{\beta}})-\frac{1}{2}\sum_{i=1}^n \Tilde{\lambda}_i
%   \left({x_i}(\pmb{\beta} - \pmb{\Tilde{\beta}}) \right)^2
% \end{equation*}
which means maximizing
% $$
% \sum_{i=1}^n \Tilde{\lambda}_i \left({x_i}(\pmb{\beta} - \pmb{\Tilde{\beta}}) \right)^2 - 2\sum_{i=1}^n(Y_i - \Tilde{\lambda}_i) x_i(\pmb{\beta}- \pmb{\Tilde{\beta}})
% $$
% which can be rewritten as
% $$
% \sum_{i=1}^n  \left(\sqrt{\Tilde{\lambda}_i}\left({x_i}(\pmb{\beta} - \pmb{\Tilde{\beta}}\right) \right)^2
% - 2\sum_{i=1}^n\frac{(Y_i - \Tilde{\lambda}_i)}{\sqrt{\Tilde{\lambda}_i}} \left(\sqrt{\Tilde{\lambda}_i}x_i(\pmb{\beta}- \pmb{\Tilde{\beta}})\right)
% =\sum_{i=1}^n \left(\sqrt{\Tilde{\lambda}_i}\left({x_i}(\pmb{\beta} - \pmb{\Tilde{\beta}})\right) -\frac{(Y_i - \Tilde{\lambda}_i)}{\sqrt{\Tilde{\lambda}_i}}\right)^2
% $$
% \LS{Rq: ce n'est pas égal, je propose de réécrire les deux équations comme suit :}
\begin{equation*}
  \sum_{i=1}^n(Y_i - \Tilde{\lambda}_i) x_i(\pmb{\beta} - \pmb{\Tilde{\beta}})-\frac{1}{2}\sum_{i=1}^n \Tilde{\lambda}_i
  \left({x_i}(\pmb{\beta} - \pmb{\Tilde{\beta}}) \right)^2
=  \sum_{i=1}^n\frac{(Y_i - \Tilde{\lambda}_i)}{\sqrt{\Tilde{\lambda}_i}} \left(\sqrt{\Tilde{\lambda}_i}x_i(\pmb{\beta}- \pmb{\Tilde{\beta}})\right) - \frac{1}{2} \sum_{i=1}^n  \left(\sqrt{\Tilde{\lambda}_i}{x_i}\left(\pmb{\beta} - \pmb{\Tilde{\beta}}\right) \right)^2.
\end{equation*}
This boils down to minimizing
$$
\sum_{i=1}^n \left(\sqrt{\Tilde{\lambda}_i}{x_i}\left(\pmb{\beta} - \pmb{\Tilde{\beta}}\right) -\frac{Y_i - \Tilde{\lambda}_i}{\sqrt{\Tilde{\lambda}_i}}\right)^2.
$$
Minimizing this criterion can be viewed as
the minimization with respect to $\pmb{\beta}$ of the following least-squares criterion:
$\|\mathcal{Y} - \mathcal{X} \pmb{\beta} \|_2^2$ where $\|u\|_2^2=\sum_{i=1}^n u_i^2$ for a vector $u=(u_1,\dots,u_n)$ in $\mathbb{R}^n$,
\begin{equation}\label{eq:Y_X}
   \mathcal{Y}=\mathcal{X}\pmb{\Tilde{\beta}}+\boldsymbol{\Tilde{\Lambda}}^{-1/2}(\textbf{Y}-\boldsymbol{\tilde{\lambda}})
   \quad \textrm{ and }\quad
   \mathcal{X}=\boldsymbol{\Tilde{\Lambda}}^{1/2}\textbf{X},
\end{equation}
 $\boldsymbol{\Tilde{\Lambda}}$ denoting the diagonal matrix having the
 $\Tilde{\lambda}_i$'s as diagonal elements, $\boldsymbol{\tilde{\lambda}}$ being a column vector having the $\Tilde{\lambda}_i$'s as components
 and $\textbf{Y}$ denoting a column vector having the $Y_i$'s as components.
   
Thus, in order to obtain a sparse estimation of $\pmb{\beta}^\star$, we will focus on finding $\pmb{\hat{\beta}}(\alpha)$ defined for $\alpha>0$ by:
\begin{equation}\label{eq:def_beta_hat}
\pmb{\hat{\beta}}(\alpha) = \underset{\pmb{\beta}\in\mathbb{R}^p}{\arg\min} \Bigg\{ \|\mathcal{Y} - \mathcal{X} \pmb{\beta}\|_2^2 + \alpha \|\pmb{\beta}\|_1 \Bigg\},
\end{equation}
where $\|v\|_1=\sum_{k=1}^p|v_k|$ for a vector $v=(v_1,\dots,v_p)$ in $\mathbb{R}^p$.

We shall establish the sign consistency of $\pmb{\hat{\beta}}$ in Theorem \eqref{theo1} of the following section. 


\section{Sign consistency}
%Notations, assumptions.
%
Let 
\begin{equation}\label{eq:def_C_W}
\textbf{C} = \frac{\mathcal{X}^T \mathcal{X}}{n} \quad \textrm{and}\quad 
    \textbf{W} = \frac{\mathcal{X}^T \pmb{\Tilde{\varepsilon}}}{n} ,
\end{equation}
where $A^T$ denotes the transpose of the matrix $A$,
\begin{equation}
\boldsymbol{\Tilde{\varepsilon}} = (\Tilde{\varepsilon}_1, \dots, \Tilde{\varepsilon}_n)^T \quad \text{with} \quad \Tilde{\varepsilon}_k = \frac{Y_k - \Tilde{\lambda}_k}{\sqrt{\Tilde{\lambda}_k}}  \quad \text{for all } \quad 1 \leq k \leq n.
\label{eq_eps}
\end{equation}
Without loss of generality, suppose that $\pmb{\beta}^{\star} = (\beta^{\star}_1, \dots \beta^{\star}_q, \beta^{\star}_{q+1}, \dots \beta^{\star}_p)^T$, where $\beta^{\star}_j \neq 0$ when $1 \leq j \leq q$ and $\beta^{\star}_j =0$ when $q+1 \leq j \leq p$ and denote
\begin{equation}\label{eq:beta_star}
    \pmb{\beta}_1^{\star} = (\beta^{\star}_1, \dots, \beta^{\star}_q)^T \quad \text{and} \quad  \pmb{\beta}_2^{\star} = (\beta^{\star}_{q+1}, \dots, \beta^{\star}_p)^T.
  \end{equation}
Then,
\begin{equation}\label{eq:def_C_W_1_2}
\textbf{C} = 
\begin{pmatrix}
\mathcal{X}_1^T \mathcal{X}_1 / n & \mathcal{X}_1^T \mathcal{X}_2 / n \\
\mathcal{X}_2^T \mathcal{X}_1 / n & \mathcal{X}_2^T \mathcal{X}_2 / n \\
\end{pmatrix} = 
\begin{pmatrix}
C_{11} & C_{12} \\
C_{21} & C_{22} \\
\end{pmatrix}
\quad 
\textrm{ and }
\quad
\textbf{W} = 
\begin{pmatrix}
W_1 \\
W_2 \\
\end{pmatrix}.
\end{equation}
% \paragraph{Statement of the theorem}
\begin{theorem}\label{theo1}
 Assume that $Y_1, \dots, Y_n$ are independent random variables such that for all $i$, $Y_i\sim\texttt{Poisson} (\lambda^{\star}_i)$ with $\lambda^{\star}_i = \exp(x_i\pmb{\beta}^{\star})$, where
$x_i$ is the $i$th row of a design matrix $\mathbf{X}$ and $\pmb{\beta}^{\star}$ is defined in (\ref{eq:beta_star}).
  Assume also that there exist positive constants $M_1$, $M_2$, $M_3$, $M_4$, $M_5$, $M_6$, $M_7$ and $c_1$  such that $0 < c_1 \leq 1$, that
  $p$ does not depend on $n$ and that the following assumptions hold. 
\begin{enumerate}[label=(T\arabic*)]
\item \label{th_1} For all $k$ in $\{1,\dots,n\}$, $\|x_k \|_2 \leq M_1$ and for all $\ell$ in $\{1,\dots,p\}$, $\|x^{(\ell)}\|_2\leq M_7$, where
  $x^{(\ell)}$ denotes the $\ell$th column of $\mathbf{X}$.
  %\item \textcolor{red}{$\frac{1}{n}  \mathcal{X}^T\mathcal{X} \leq M_1$, $\forall i$, where $\mathcal{X}_i$ designates the $i$th column of $\mathcal{X}$}
\item \label{th_5} $\pmb{\Tilde{\beta}}$ is a preliminary estimator of $\pmb{\beta}^\star$ such that
    $|\beta^\star_i - \Tilde{\beta}_i| =O_P(1/n)$, as $n$ tends to infinity, for all $i=1, \dots, p$.
\item \label{th_2} With a probability tending to 1 as $n$ tends to infinity, $\lambda_{\min}(C_{11}) \geq M_2$, where $\lambda_{\min}(A)$ denotes the smallest eigenvalue of the matrix $A$ and $C_{11}$ is defined in (\ref{eq:def_C_W_1_2}).
  \item \label{th_3} With a probability tending to 1 as $n$ tends to infinity, $\lambda_{\max}(C_{12}) \leq M_3$, $\lambda_{\max}(C_{21}) \leq M_4$, and $\lambda_{\max}(C_{22}) \leq M_5$, where $\lambda_{\max}(A)$
    denotes the largest eigenvalue of the matrix $A$.
 % \item $p << n^{\frac{c_1}{2}}$
  %\item $q = q_n = O\left(n^{\frac{c_3}{2}}\right)$
  \item \label{th_4} $n^{\frac{1-c_1}{2}} \min_{1 \leq i \leq q } |\beta_i^{\star}| \geq M_6$.
 % \item \label{th_5} $|\beta^\star_i| \leq M_7$, for all $i=1, \dots, p$
\end{enumerate}
Let us also suppose that the following condition called strong irrepresentable condition holds: there exists  $\tau > 2/3$ such that
\begin{equation}\label{eq:IC}
\left| C_{21} C_{11}^{-1}\text{sign}(\pmb{\beta}^\star_1) \right| \leq 1- \tau,
\end{equation}
with a probability tending to 1 when $n$ tends to infinity, where the inequality has to be understood component by component. 
Then, for all $\alpha=\alpha_n$ such that
$
\alpha_n=O\left(n^\frac{c_2+1}{2} \right), \quad \text{where} \quad 0 < c_2 < c_1 \leq 1$, $\pmb{\hat{\beta}}$ defined in (\ref{eq:def_beta_hat}) satisfies
\begin{equation*}
\mathbb{P} \left( \text{sign}(\pmb{\hat{\beta}}) = \text{sign}(\pmb{\beta}^\star)\right) \rightarrow 1, \quad \text{when} \quad n \rightarrow \infty.
\end{equation*}
\end{theorem}

The proof of Theorem \ref{theo1} relies on the following proposition.

% \paragraph{Proposition for definition of $A_n$ and $B_n$}

\begin{prop}\label{prop1}
  Under the assumptions of Theorem \ref{theo1}, let
\begin{equation}\label{eq:R1_2}
R_1 = (C(\pmb{\beta}^{\star} - \pmb{\Tilde{\beta}}))_1 \quad \text{and} \quad R_2 = (C(\pmb{\beta}^{\star} - \pmb{\Tilde{\beta}}))_2.
\end{equation}
Then,
\begin{equation*}
    \mathbb{P}\Big(\text{sign}(\pmb{\hat{\beta}})=\text{sign}(\pmb{\beta}^{\star})\Big) \geq \mathbb{P}\Big(A_n \cap B_n \Big),
\end{equation*}
where
\begin{equation*}
    A_n = \Big\{ |C_{11}^{-1} W_1| < |\pmb{\beta}_1^{\star}| - \frac{\alpha}{2n} |C_{11}^{-1} \text{sign}(\pmb{\beta}_1^{\star}) | - |C_{11}^{-1} R_1| \Big\}
\end{equation*}
and
\begin{equation*}
    B_n = \left\{ | C_{21} C_{11}^{-1} W_1 - W_2 | \leq \frac{\alpha}{2n} \left( 1 - | C_{21}C_{11}^{-1} \text{sign}(\pmb{\beta}_1^{\star}) | \right) - | C_{21}C_{11}^{-1} R_1 - R_2 |  \right\}.
\end{equation*}
\end{prop}

%\LS{Rq: Pas besoin des hypothèses du théorème ?}


The proof of Proposition \ref{prop1} is in \ref{sec:proof_prop1}.

Proving Theorem \ref{theo1} consists in showing that $\mathbb{P}(A_n^c)$ and $\mathbb{P}(B_n^c)$ go to zero as $n$ tends to infinity
where $S^c$ denotes the complementary set of the set $S$. The proof is given in \ref{sec:proof_theo}.


\appendix 
\section{Proofs}
\subsection{Proof of Proposition \ref{prop1}}\label{sec:proof_prop1}
%Proof of KKT, pages 6-7
%The proof that probability of $A_n^c$ and $B_n^c$ goes to 0 relies on the following Bernstein inequality

First observe that we get from (\ref{eq:Y_X}) and (\ref{eq_eps}) that
\begin{equation}\label{eq:def_Y}
\mathcal{Y}=\mathcal{X}\pmb{\Tilde{\beta}}+\boldsymbol{\Tilde{\varepsilon}}.
\end{equation}
Let us denote $\pmb{\hat{\beta}}$ the estimator $\pmb{\hat{\beta}}(\alpha)$ defined in (\ref{eq:def_beta_hat}), it satisfies the following Karush–Kuhn–Tucker conditions described in \cite[Section 4.2.2]{giraud:2021}:
\begin{center}
\begin{tabular}{cl}
$\displaystyle \big( \mathcal{X}^T(\mathcal{Y} - \mathcal{X} \pmb{\hat{\beta}}) \big)_i = \frac{\alpha}{2} \text{sign}(\hat{\beta}_i)$, & if $\hat{\beta}_i \neq 0$, \\
$\displaystyle \big| \big( \mathcal{X}^T (\mathcal{Y} - \mathcal{X} \pmb{\hat{\beta}}) \big)_i \big| \leq \frac{\alpha}{2}$, & if $\hat{\beta}_i = 0$,
\end{tabular}
\end{center}
which can be rewritten as follows by using (\ref{eq:def_C_W}) and (\ref{eq:def_Y})
\begin{align}
&  \big(\textbf{C}(\pmb{\hat{\beta}} -  \pmb{\beta}^{\star})-\textbf{W}+\textbf{C}(\pmb{\beta}^{\star} - \pmb{\Tilde{\beta}})\big)_i = - \frac{\alpha}{2n} \text{sign}(\hat{\beta}_i), \quad \text{if } \hat{\beta}_i \neq 0,\label{KKT1}\\
&  \big| \big(\textbf{C}(\pmb{\hat{\beta}} -  \pmb{\beta}^{\star})-\textbf{W}+\textbf{C}(\pmb{\beta}^{\star} - \pmb{\Tilde{\beta}})\big)_i\big| \leq \frac{\alpha}{2n}, \quad \text{if } \hat{\beta}_i = 0.\label{KKT2}
\end{align}

If $A_n$ holds then
\begin{equation}\label{eq:AN_consequence}
        -  |\pmb{\beta}_1^{\star}| < C_{11}^{-1} W_1 - \frac{\alpha}{2n} C_{11}^{-1} \text{sign}(\pmb{\beta}_1^{\star}) - C_{11}^{-1} R_1 < |\pmb{\beta}_1^{\star}|.
\end{equation}
Let $\pmb{\check{\beta}}=(\pmb{\check{\beta}}_1^T,\textbf{0}^T)^T$ where
\begin{equation}\label{eq:check_beta}
\pmb{\check{\beta}}_1 = \pmb{\beta}_1^{\star}+C_{11}^{-1} W_1 -\frac{\alpha}{2 n} C_{11}^{-1} \text{sign}(\pmb{\beta}_1^{\star}) - C_{11}^{-1} R_1.
\end{equation}
By (\ref{eq:AN_consequence}), $|\pmb{\check{\beta}}_1-\pmb{\beta}_1^{\star}|<|\pmb{\beta}_1^{\star}|$ which implies that
$\text{sign}(\pmb{\check{\beta}}_1)= \text{sign}(\pmb{\beta}_1^{\star})$. Hence, $\pmb{\check{\beta}}$ satisfies (\ref{KKT1}).

If $B_n$ holds then
\begin{equation*}
  \Big| C_{21} \Big( C_{11}^{-1} W_1 - C_{11}^{-1} R_1 -\frac{\alpha}{2 n} C_{11}^{-1} \text{sign}(\pmb{\beta}_1^{\star}) \Big) - W_2 + R_2 \Big|
  \leq \frac{\alpha}{2 n},
\end{equation*}
which by (\ref{eq:check_beta}) corresponds to (\ref{KKT2}) for $\pmb{\check{\beta}}$ and concludes the proof of Proposition \ref{prop1}.


\subsection{Proof of Theorem \ref{theo1}}\label{sec:proof_theo}

Let us first prove that $\mathbb{P}(A_n^c)$ tends to zero as $n$ tends to infinity.
By denoting %\textcolor{purple}{ne pas noter $\xi $ et $b$}
\begin{equation}
  \xi = (\xi_1, \dots, \xi_q)^T = C_{11}^{-1} W_1  \quad \textrm{ and }\quad
  b = (b_1, \dots, b_q)^T = C_{11}^{-1} \text{sign}(\pmb{\beta}_1^{\star}),
  \label{def:xi}
\end{equation}
we get that
\begin{align*}
  \mathbb{P}(A_n^c) &= \mathbb{P} \left( |C_{11}^{-1}W_1| + |C_{11}^{-1} R_1|+ \frac{\alpha}{2n} \left| C_{11}^{-1} \text{sign}(\pmb{\beta}_1^{\star})\right|
                     \geq |\pmb{\beta}_1^{\star}|    \right) \\
                     &   \leq \sum_{j=1}^q \mathbb{P}\left(|\xi_j| + |(C_{11}^{-1} R_1)_j|+\frac{\alpha}{2n} \big| b_j \big|\geq |\beta_j^{\star}|\right)\\
  &  \leq \sum_{j=1}^q \left\{\mathbb{P}\left(|\xi_j|\geq \frac{|\beta_j^{\star}|}{3}\right)+\mathbb{P}\left(|(C_{11}^{-1} R_1)_j|\geq \frac{|\beta_j^{\star}|}{3}\right)
    +\mathbb{P}\left(\frac{\alpha}{2n} \big| b_j \big|\geq \frac{|\beta_j^{\star}|}{3}\right)\right\}.
  \end{align*}
 By the Cauchy-Schwarz inequality, we get that for all $j$ in $\{1,\dots,q\}$
\begin{equation*}
  |b_j| \leq \sum_{j=1}^{q} |b_j| \leq\sqrt{q} \|b\|_2=\sqrt{q} \| C_{11}^{-1} \text{sign}(\pmb{\beta}_1^{\star}) \|_2 \leq q \| C_{11}^{-1}\|_2
  = q \lambda_{\text{max}}(C_{11}^{-1}).
\end{equation*}
By using \ref{th_4}, \ref{th_2}, $\alpha=O(n^{(c_2+1)/2})$ and $0<c_2<c_1\leq 1$, we obtain that
\begin{equation}\label{eq:An_c_bound}
    \mathbb{P}(A_n^c) \leq \sum_{j=1}^{q} \mathbb{P} \Bigg( |\xi_j| \geq \frac{M_6 n^{\frac{c_1-1}{2} }}{3}\Bigg) + \sum_{j=1}^{q} \mathbb{P} \Bigg(  |(C_{11}^{-1} R_1)_j| \geq \frac{M_6 n^{\frac{c_1-1}{2}}}{3}\Bigg)+o(1).
  \end{equation}

Let us first prove that the second term in the r.h.s of (\ref{eq:An_c_bound}) tends to 0 as $n$ tends to infinity.
  Observing that $R_1$ defined in (\ref{eq:R1_2}) satisfies: $R_1= C_{11}\left( \pmb{\beta}^\star_1 - \pmb{\Tilde{\beta}}_1  \right) + C_{12}\left( \pmb{\beta}^\star_2 - \pmb{\Tilde{\beta}}_2  \right)$, we get by the Cauchy-Schwarz inequality that
 \begin{equation*}
\left | (C_{11}^{-1} R_1)_j \right | \leq \sqrt{q} \left \| C_{11}^{-1} R_1\right \|_2 \leq \sqrt{q} \left \| \pmb{\beta}^\star_1 - \pmb{\Tilde{\beta}}_1  \right\|_2 + \sqrt{q} \left \| C_{11}^{-1} C_{12} \left(\pmb{\beta}^\star_2 - \pmb{\Tilde{\beta}}_2  \right) \right \|_2.
\end{equation*}
By \ref{th_5}, \ref{th_2}, \ref{th_3}, $\left | (C_{11}^{-1} R_1)_j \right | =O_P(n^{-1})$, for all $j$, where the $O_P$ does not depend on $j$,
which proves that the second term in the r.h.s of (\ref{eq:An_c_bound}) tends to 0 as $n$ tends to infinity. 

Let us now prove that the first term in the r.h.s of (\ref{eq:An_c_bound}) tends to 0 as $n$ tends to infinity. By \eqref{def:xi}, \eqref{eq:def_C_W} and \eqref{eq_eps}, denoting $\boldsymbol{\lambda^{\star}}$ the column vector of the $\lambda_i^\star$'s, $\xi$ can be rewritten as follows
\begin{align}\label{eq:xi}
\xi_j & = (C_{11}^{-1} W_1)_j =  \left(C_{11}^{-1} \frac{\mathcal{X}_1^T \pmb{\Tilde{\varepsilon}}}{n}\right)_j  = \left(\frac1n C_{11}^{-1} \mathcal{X}_1^T \boldsymbol{\Tilde{\Lambda}}^{-1/2}(\textbf{Y}-\boldsymbol{\tilde{\lambda}})\right)_j\nonumber\\
 & = \left(\frac1n C_{11}^{-1} \mathcal{X}_1^T \boldsymbol{\Tilde{\Lambda}}^{-1/2}(\textbf{Y}-\boldsymbol{\lambda^{\star}})\right)_j
  +\left(\frac1n C_{11}^{-1} \mathcal{X}_1^T \boldsymbol{\Tilde{\Lambda}}^{-1/2}(\boldsymbol{\lambda^{\star}}-\boldsymbol{\tilde{\lambda}})\right)_j.
\end{align}
For all $j$ in $\{1,\dots,q\}$, all $k$ in $\{1,\dots,n\}$, by (2.3.8) of \cite{golub:96}, we have that
\begin{align}
&\frac1n\left(C_{11}^{-1} \mathcal{X}_1^T \boldsymbol{\Tilde{\Lambda}}^{-1/2}\right)_{jk}  %= \left( C_{11}^{-1} \mathcal{X}_1^T \right)_{jk} \tilde{\lambda}_k^{-1/2} 
\leq \frac{1}{\sqrt{n}}\left|\left(C_{11}^{-1} \frac{\mathcal{X}_1^T}{\sqrt{n}} \right)_{jk}\right| \sup_{k\in\{1,\ldots,n\}}\left(\tilde{\lambda}_k^{-1/2}\right)
\leq \frac{1}{\sqrt{n}}\left\| C_{11}^{-1} \frac{\mathcal{X}_1^T}{\sqrt{n}} \right\|_2 \sup_{k\in\{1,\ldots,n\}}\left(\tilde{\lambda}_k^{-1/2}\right)\nonumber\\ 
&= \frac{1}{\sqrt{n}}\;\rho\left(\left( C_{11}^{-1} \frac{\mathcal{X}_1^T}{\sqrt{n}}\right)^T\left(C_{11}^{-1} \frac{\mathcal{X}_1^T}{\sqrt{n}}\right)\right)^{1/2} \sup_{k\in\{1,\ldots,n\}}\left(\tilde{\lambda}_k^{-1/2}\right), \label{eq:Gjk_9}
\end{align}
where $\rho(A)$ is the spectral radius of the matrix $A$. Note that, by Theorem 1.3.22 of \cite{horn_johnson:2013},
\begin{equation}\label{eq:rho}
  \rho\left(\left(C_{11}^{-1} \frac{\mathcal{X}_1^T}{\sqrt{n}}\right)^T\left(C_{11}^{-1} \frac{\mathcal{X}_1^T}{\sqrt{n}}\right)\right) = \rho\left(\left(C_{11}^{-1}
      \frac{\mathcal{X}_1^T}{\sqrt{n}}\right)\left(C_{11}^{-1} \frac{\mathcal{X}_1^T}{\sqrt{n}}\right)^T\right)
  = \rho\left(C_{11}^{-1} \frac{\mathcal{X}_1^T \mathcal{X}_1}{n} C_{11}^{-1}\right)= \rho\left(C_{11}^{-1} \right).
\end{equation}
From \eqref{th_1} and \eqref{th_5}, we get that 

 %\textcolor{purple}{attention ici et ailleurs un mélange de gras et non gras pour les mêmes objets}
$$\tilde{\lambda}_k^{-1/2}=\frac{1}{\sqrt{\exp(x_k\pmb{\tilde{\beta}})}}=\frac{1}{\sqrt{\exp(x_k\pmb{\beta^\star})\exp(x_k(\pmb{\tilde{\beta}}-\pmb{\beta^\star}))}}={\lambda_k^\star}^{-1/2}\left(1+O_P(n^{-1})\right),$$ where ${\lambda_k^\star}^{-1/2}\leq\exp(-x_k\boldsymbol{\beta^\star}/2)\leq\exp(\|x_k\|_2
\|\boldsymbol{\beta^\star}\|_2/2)\leq\exp(M_1\|\boldsymbol{\beta^\star}\|_2/2)$, by the Cauchy-Schwarz inequality and \ref{th_1}.   Hence,
\begin{equation}\label{eq:suplambda}
\sup_{k\in\{1,\ldots,n\}}\left(\tilde{\lambda}_k^{-1/2}\right)=O_P(1), \textrm{ as } n\to\infty.
\end{equation}
By (\ref{eq:Gjk_9}), (\ref{eq:rho}) and (\ref{eq:suplambda})
$$
\frac1n\left(C_{11}^{-1} \mathcal{X}_1^T \boldsymbol{\Tilde{\Lambda}}^{-1/2}\right)_{jk}=O_P(n^{-1/2}).
$$
Moreover, by \ref{th_1} and \ref{th_5}
\begin{multline}\label{eq:lambdak-lambdakstar}
  \lambda_k^\star-\Tilde{\lambda}_k=\exp(x_k\pmb{\beta^\star})-\exp(x_k\pmb{\tilde{\beta}})=\exp(x_k\pmb{\beta^\star})\left(1-\exp(x_k(\pmb{\tilde{\beta}}-
    \pmb{\beta^\star}))\right)\\
  =-\exp(x_k\boldsymbol{\beta}^\star)\sum_{\ell\geq 1} \frac{(x_k(\boldsymbol{\Tilde{\beta}}-\boldsymbol{\beta^\star}))^\ell}{\ell !}
=O_P(1/n).
\end{multline}
Thus, the second term in the r.h.s of (\ref{eq:xi}) is $O_P(1/\sqrt{n})$. 
Note that
\begin{align}\label{eq:2eme_terme}
&\left(\frac1n C_{11}^{-1} \mathcal{X}_1^T \boldsymbol{\Tilde{\Lambda}}^{-1/2}(\textbf{Y}-\boldsymbol{\lambda^{\star}})\right)_j
=\left(\frac1n C_{11}^{-1} X_1^T (\textbf{Y}-\boldsymbol{\lambda^{\star}})\right)_j\nonumber\\
&=\left(\frac1n (C_{11}^{\star})^{-1} X_1^T (\textbf{Y}-\boldsymbol{\lambda^{\star}})\right)_j+
\left(\frac1n (C_{11}^{-1} -(C_{11}^{\star})^{-1}) X_1^T (\textbf{Y}-\boldsymbol{\lambda^{\star}})\right)_j,
\end{align}
where $C^\star={\mathcal{X}^\star}^T\mathcal{X}^\star/n$ with $\mathcal{X}^\star=(\Lambda^\star)^{1/2}\mathbf{X}$, $\Lambda^\star$ being a diagonal matrix having as diagonal entries the $\lambda_k^\star$'s.
Hence, the second term in the r.h.s of (\ref{eq:2eme_terme}) is bounded by
$$
\|C_{11}^{-1} -(C_{11}^{\star})^{-1}\|_2 \; \|X_1^T\|_2 \; \left(\frac1n\sum_{k=1}^n |Y_k-\lambda_k^\star|\right)=\|C_{11}^{-1} -(C_{11}^{\star})^{-1}\|_2 \; O_P(1),
$$
by Markov's inequality and \ref{th_1}. Since, $C_{11}^{\star}=C_{11}+(C_{11}^{\star}-C_{11})$ , we have
$$(C_{11}^{\star})^{-1}=\left(\textrm{Id}+C_{11}^{-1}(C_{11}^{\star}-C_{11})\right)^{-1}C_{11}^{-1}$$ and thus by Corollary 5.6.16 of \citep{horn_johnson:2013}, we get that
\begin{equation}\label{eq:norme2_bound_C11-1}
\|C_{11}^{-1} -(C_{11}^{\star})^{-1}\|_2=O_P(\|C_{11}^{\star}-C_{11}\|_2)=O_P(1/n^2)
\end{equation}
since by Equation (2.3.8) of \cite{golub:96}, the Cauchy-Schwarz inequality and \eqref{eq:lambdak-lambdakstar}
\begin{equation}\label{eq:norme2_bound}
\|C_{11}^{\star}-C_{11}\|_2\leq \frac{q}{n} \max_{1\leq \ell\leq q}\|x^{(\ell)}\|_2^2\sup_{1\leq k\leq n}|\lambda_k^\star-\tilde{\lambda}_k| =O_P(1/n^2).
\end{equation}
Thus, the second term in the r.h.s of (\ref{eq:2eme_terme}) is $O_P(1/n^2)$. To address the first term in the r.h.s of (\ref{eq:2eme_terme}), we shall use
the following result. 

 \begin{theorem}
 \label{bernstein}
(Bernstein's Inequality, \cite[Corollary 2.11]{boucheron_lugosi_massart:2013}) Let $X_1, \dots, X_n$ be independent real random variables. Suppose that there exist $\nu >0$ and $c>0$ such that $\sum_{k=1}^n \mathbb{E}\big[X_k^2\big] \leq \nu$, and 
\begin{equation*}
\sum_{k=1}^n \mathbb{E}\big[ |X_k|^{\ell} \big] \leq \frac{\ell!}{2} \nu c^{\ell-2}
\end{equation*}
for all integers $\ell \geq 3$. Then, for all $t>0$,
\begin{equation*}
\mathbb{P} \Bigg( \left|\sum_{k=1}^n \big( X_k - \mathbb{E} \big[ X_k \big]\big)\right| \geq t \Bigg) \leq 2\exp \Bigg[ - \frac{t^2}{2(\nu +ct)} \Bigg].
\end{equation*}
\end{theorem}

Denoting $G=n^{-1} (C_{11}^{\star})^{-1} X_1^T(\Lambda^\star)^{1/2}$, we get that
$$
\left(\frac1n (C_{11}^{\star})^{-1} X_1^T (\textbf{Y}-\boldsymbol{\lambda^{\star}})\right)_j=\sum_{k=1}^n G_{jk}\frac{(Y_k-\lambda_k^\star)}{\sqrt{\lambda_k^\star}}.
$$
Let us now apply the Bernstein's inequality to $X_k=G_{jk} Y_k/\sqrt{\lambda_k^\star}$ then
\begin{align*}
\sum_{k=1}^n\mathbb{E}[X_k^2]=\sum_{k=1}^n G_{jk}^2(1+\lambda_k^\star)& \leq n\sup_{1\leq k\leq n}(1+\lambda_k^\star)\;\|G\|_2^2 \\
&\leq n\sup_{1\leq k\leq n}(1+\lambda_k^\star)\;\rho(GG^T)\leq\sup_{1\leq k\leq n}(1+\lambda_k^\star)\;\rho((C_{11}^{\star})^{-1}).
\end{align*}
By Weyl's inequalities \cite[Corollary 4.3.15]{horn_johnson:2013} and \ref{th_2}, we get that with a probability tending to 1,
$$
\lambda_{min}(C_{11}^\star)\geq\lambda_{min}(C_{11})+\lambda_{min}(C_{11}^\star-C_{11})\geq\lambda_{min}(C_{11})-\rho(C_{11}^\star-C_{11})
\geq M_2-\|C_{11}^\star-C_{11}\|_2,
$$
Denoting $\bar{\lambda}=\max(\sup_{1\leq k\leq n}\lambda_k^\star,1)$ and $M_2'$ the positive constant such that $\rho((C_{11}^{\star})^{-1})\leq 1/M_2'$ for large enough $n$,
which exists by (\ref{eq:norme2_bound}), we get by \ref{th_2} that
$
\nu=2\bar{\lambda}/M_2'.
$
Observe that
\begin{equation}\label{eq:E_Xk_l}
  \sum_{k=1}^n \mathbb{E}\big[ |X_k|^{\ell} \big]=\sum_{k=1}^n \mathbb{E} \left[ \left| \frac{G_{jk}}{\sqrt{\lambda_k^{\star}}}Y_k \right|^{\ell} \right]
  = \sum_{k=1}^n \left( \frac{|G_{jk}|}{\sqrt{\lambda_k^{\star}}}\right)^{\ell} \sum_{i=1}^{\ell} {\lambda_k^{\star} }^{\ell} \bracenom{\ell}{i} ,
 \end{equation}
 where $\bracenom{\ell}{i}$ denotes the Stirling number of the second kind and the last equality is due to the definition of the $\ell$-th moment of a
 Poisson random variable. Then we have, for all $\ell \geq 3$, by Equation (2.3.8) of \cite{golub:96},
 \begin{align}\label{eq:Gjk}
   &  \sum_{k=1}^n \Bigg( \frac{|G_{jk}|}{\sqrt{\lambda_k^{\star}}}\Bigg)^{\ell} \sum_{i=1}^{\ell} {\lambda_k^{\star} }^{\ell} \bracenom{\ell}{i}  \leq \sum_{k=1}^n \frac{1}{n^{\ell/2}} \rho((C_{11}^\star)^{-1})^{\ell/2} \bar{\lambda}^{\ell/2} \sum_{i=1}^{\ell}  \bracenom{\ell}{i} \leq n^{1-\ell/2} M_2'^{-\ell/2} \bar{\lambda}^{\ell/2} \ell!
   \nonumber\\
 &= n^{1-\ell/2} M_2'^{1-\ell/2} \bar{\lambda}^{\ell/2-1}  \frac{\ell!}{2}\nu
   \leq\frac{\ell!}{2}\nu \left(\frac{\sqrt{\bar{\lambda}}}{\sqrt{n M_2'}}\right)^{\ell-2}=\frac{\ell!}{2} \nu c^{\ell-2},
 \end{align}
 with $c=\sqrt{\bar{\lambda}/(n M_2')}$ since $\sum_{i=1}^{\ell} \bracenom{\ell}{i} \leq \ell!$.

 Hence, for $t=n^{(c_1-1)/2}$,
 $$
 \frac{t^2}{2(\nu +ct)}=\frac{n^{c_1-1}}{2(\nu+n^{-1/2}\sqrt{\bar{\lambda}/M_2'}n^{(c_1-1)/2})}=O(n^{c_1/2}),
 $$
which gives the expected result.
 
 
%%%%%%%%%%%%%%%%%
%% Proof B_n^c %% 
%%%%%%%%%%%%%%%%%
 
Let us then prove that $\mathbb{P}(B_n^c)$ tends to zero as $n$ tends to infinity. 
By denoting
\begin{equation}
  \zeta = (\zeta_1, \ldots, \zeta_{p-q})^T = C_{21}C_{11}^{-1} W_1  -W_2 %= C_{21}\xi -W_2
  \quad \textrm{ and }\quad
  d = (d_1, \ldots, d_{p-q})^T =  C_{21}C_{11}^{-1} \text{sign}(\pmb{\beta}_1^{\star}),%=C_{21}b,
  \label{def:zeta}
\end{equation}
we get that 
\begin{align}\label{eq:Bn_c_bound}
  \mathbb{P}(B_n^c) &= \mathbb{P} \left( |C_{21}C_{11}^{-1} W_1  -W_2| + |C_{21}C_{11}^{-1} R_1-R_2|+ \frac{\alpha}{2n} \left|  C_{21}C_{11}^{-1} \text{sign}(\pmb{\beta}_1^{\star})\right|
                     >  \frac{\alpha}{2n}   \right) \nonumber\\
  & \leq \sum_{j=1}^{p-q} \mathbb{P}\left(|\zeta_j| + |(C_{21}C_{11}^{-1} R_1-R_2)_j|+\frac{\alpha}{2n} \big| d_j \big|\geq \frac{\alpha}{2n}  \right)\nonumber\\
  &  \leq \sum_{j=1}^{p-q} \left\{\mathbb{P}\left(|\zeta_j|\geq  \frac{\alpha}{6n}\right)+\mathbb{P}\left(|(C_{21}C_{11}^{-1} R_1-R_2)_j|\geq \frac{\alpha}{6n}\right)
    +\mathbb{P}\left(\big| d_j \big|\geq  \frac{1}{3}\right)\right\}.
  \end{align}
 %\textcolor{purple}{Est-ce qu'il ne faudrait pas définir $\check{\zeta}=(0^T,\zeta^T)$ pour éviter les problèmes de numérotation des sommes et pour toujours avoir des vecteurs de taille $p$ ?}
  By the strong irrepresentable condition \eqref{eq:IC}, we get that %for all $j$ in $\{1,\ldots,p-q\}$,
  $\sum_{j=1}^{p-q}\mathbb{P}\left(\big| d_j \big|\geq  \frac{1}{3}\right)=o(1)$.
Let us now prove that the second term in the r.h.s of (\ref{eq:Bn_c_bound}) tends to 0 as $n$ tends to infinity.
  Observing that $R_1$ defined in (\ref{eq:R1_2}) satisfies: $R_1= C_{11}\left( \pmb{\beta}^\star_1 - \pmb{\Tilde{\beta}}_1  \right) + C_{12}\left( \pmb{\beta}^\star_2 - \pmb{\Tilde{\beta}}_2  \right)$, we get by the Cauchy-Schwarz inequality that
 \begin{align*}
\left | (C_{21}C_{11}^{-1} R_1)_j \right | &\leq \sqrt{p-q} \left \| C_{21}C_{11}^{-1} R_1\right \|_2 \\
&\leq \sqrt{p-q} \left \|C_{21}( \pmb{\beta}^\star_1 - \pmb{\Tilde{\beta}}_1 ) \right\|_2 + \sqrt{p-q} \left \| C_{21}C_{11}^{-1} C_{12} \left(\pmb{\beta}^\star_2 - \pmb{\Tilde{\beta}}_2  \right) \right \|_2.
\end{align*}
By \ref{th_5}, \ref{th_2} and \ref{th_3}, $\left | (C_{21}C_{11}^{-1} R_1)_j \right | =O_P(n^{-1})$ for all $j$. Using similar arguments  we get $\left | (R_2)_j \right | =O_P(n^{-1})$.  Then since $\alpha=O\left(n^{(c_2+1)/2}\right)$, 
we get that the second term in the r.h.s of (\ref{eq:Bn_c_bound}) tends to 0 as $n$ tends to infinity. 

Let us now prove that the first term in the r.h.s of (\ref{eq:Bn_c_bound}) tends to 0 as $n$ tends to infinity. By \eqref{def:zeta}, \eqref{eq:def_C_W}, \eqref{eq_eps} and (\ref{eq:Y_X}), $\zeta$ can be rewritten as follows for all $j\in\{1,\ldots,p-q\}$
\begin{equation}\label{eq:zeta}
\zeta_j = (C_{21}C_{11}^{-1} W_1-W_2)_j =  \left(C_{21}\xi- \frac{\mathcal{X}_2^T}{n} \Tilde{\varepsilon}\right)_j=\left(C_{21}\xi-\frac{X_2^T}{n} (\textbf{Y}-\boldsymbol{\lambda^{\star}})\right)_j,
\end{equation}
where we recall that $\boldsymbol{\lambda^{\star}}$ denotes the column vector of the $\lambda_i^\star$'s and $\xi = C_{11}^{-1} W_1$.
%We showed previously that $ \mathbb{P} \Bigg( |\xi_j| \geq \frac{M_6 n^{\frac{c_1-1}{2} }}{3}\Bigg) $ tends to 0 as $n$ tends to infinity which implies that $ \mathbb{P} \Bigg( |(C_{21}\xi)_j| \geq\frac{\alpha}{12n}\Bigg) =  \mathbb{P} \Bigg( |(C_{21}\xi)_j| \geq\frac{n^{\frac{c_2-1}{2}}}{12}\Bigg)$ tends to 0 as $n$ tends to infinity when $c_2>c_1$. inverser les rôles ?
%\textcolor{blue}{attention, c'est $n^{\frac{c_2-1}{2}}$. du coup, cela ne marche pas ... ce que l'on doit montrer ici avec $B_n^c$ :  $ \mathbb{P} \Bigg( |(C_{21}\xi)_j| \geq\frac{\alpha}{12n}\Bigg) =  \mathbb{P} \Bigg( |(C_{21}\xi)_j| \geq\frac{n^{\frac{c_2-1}{2}}}{12}\Bigg)$ tends to 0 as $n$ tends to infinity (c'est pas un égal, attention !) which implies  $ \mathbb{P} \Bigg( |\xi_j| \geq \frac{M_6 n^{\frac{c_1-1}{2} }}{3}\Bigg) $ tends to 0 as $n$ tends to infinity (que l'on trouve dans $A_n^c$).}

Let us consider the first term in \eqref{eq:zeta} and prove that $\mathbb{P} \Bigg( |(C_{21}\xi)_j| \geq\frac{\alpha}{12n}\Bigg)$ tends to $0$ as $n$ tends to infinity. Let us note that the term $\xi_j$ is handled previously in the proof concerning $A_n^c$. Therefore we use the same arguments and adapt it to the term 
\begin{equation}\label{eq:c21xi}
(C_{21}\xi)_j= \left(\frac1n C_{21}C_{11}^{-1} \mathcal{X}_1^T \boldsymbol{\Tilde{\Lambda}}^{-1/2}(\textbf{Y}-\boldsymbol{\lambda^{\star}})\right)_j
  +\left(\frac1n C_{21}C_{11}^{-1} \mathcal{X}_1^T \boldsymbol{\Tilde{\Lambda}}^{-1/2}(\boldsymbol{\lambda^{\star}}-\boldsymbol{\tilde{\lambda}})\right)_j.
  \end{equation}
The second term of \eqref{eq:c21xi} is still $O_P(1/\sqrt{n})$ since in \eqref{eq:rho}, $\rho(C_{21}C_{11}^{-1}C_{21}^T)=\rho(C_{21}C_{11}^{-1}C_{12})$ is bounded by using \ref{th_2} and \ref{th_3}. The first term of \eqref{eq:c21xi} can be decomposed in the same way as \eqref{eq:2eme_terme}:
\begin{align}\label{eq:2eme_terme:c12xi}
&\left(\frac1n C_{21}C_{11}^{-1} \mathcal{X}_1^T \boldsymbol{\Tilde{\Lambda}}^{-1/2}(\textbf{Y}-\boldsymbol{\lambda^{\star}})\right)_j\nonumber\\
&=\left(\frac1n C_{21}^\star(C_{11}^{\star})^{-1} X_1^T (\textbf{Y}-\boldsymbol{\lambda^{\star}})\right)_j+
\left(\frac1n (C_{21}C_{11}^{-1} -C_{21}^\star(C_{11}^{\star})^{-1}) X_1^T (\textbf{Y}-\boldsymbol{\lambda^{\star}})\right)_j.
\end{align}
The second term of \eqref{eq:2eme_terme:c12xi} is bounded in the same manner as the second term of \eqref{eq:2eme_terme} was handled. 
By observing that
  \begin{align*}
C_{21}C_{11}^{-1} -C_{21}^\star(C_{11}^{\star})^{-1}&=(C_{21}-C_{21}^\star)C_{11}^{-1} + C_{21}^\star(C_{11}^{-1} -(C_{11}^\star)^{-1} )\\
&=(C_{21}-C_{21}^\star)C_{11}^{-1} + (C_{21}^\star-C_{21})(C_{11}^{-1} -(C_{11}^\star)^{-1} )+C_{21}(C_{11}^{-1} -(C_{11}^\star)^{-1} ),
\end{align*}
and by using (\ref{eq:norme2_bound_C11-1}), \ref{th_1}, \ref{th_2}, \ref{th_3} and the fact that
\begin{equation}\label{eq:C21}
  \|C_{21}^\star-C_{21}\|_2=O_P(1/n^2),
\end{equation}
where we used the same arguments as in (\ref{eq:norme2_bound}) we get that
  $\|C_{21}C_{11}^{-1} -C_{21}^\star(C_{11}^{\star})^{-1}\|_2=O_P(1/n^2)$. Thus, the second term of\eqref{eq:2eme_terme:c12xi} is $O_P(1/n^2)$.
To handle the first term of \eqref{eq:2eme_terme:c12xi}, let us apply the Bernstein's inequality to $X_k=\widetilde{G}_{jk} Y_k/\sqrt{\lambda_k^\star}$ with $\widetilde{G}=n^{-1}C_{21}^\star(C_{11}^{\star})^{-1} X_1^T(\Lambda^\star)^{1/2}$. Then
$$
\sum_{k=1}^n\mathbb{E}[X_k^2]=\sum_{k=1}^n \widetilde{G}_{jk}^2(1+\lambda_k^\star)\leq n\sup_{1\leq k\leq n}(1+\lambda_k^\star)\;\rho(\widetilde{G}\widetilde{G}^T)\leq\sup_{1\leq k\leq n}(1+\lambda_k^\star)\;\rho(C_{21}^{\star}(C_{11}^\star)^{-1}(C_{21}^{\star})^T).
$$
By Weyl's inequalities \cite[Corollary 4.3.15]{horn_johnson:2013}
  \begin{align*}
    & \lambda_{max}(C_{21}^{\star}(C_{11}^\star)^{-1}(C_{21}^{\star})^T)\leq
      \lambda_{max}((C_{21}^{\star}-C_{21})((C_{11}^{\star})^{-1}-C_{11}^{-1})((C_{21}^{\star})^T-C_{21}^T))\\
&  +\lambda_{max}((C_{21}^{\star}-C_{21})((C_{11}^{\star})^{-1}-C_{11}^{-1})C_{21}^T)+\lambda_{max}((C_{21}^{\star}-C_{21})C_{11}^{-1}((C_{21}^{\star})^T-C_{21}^T))\\
&   +\lambda_{max}((C_{21}^{\star}-C_{21})C_{11}^{-1}C_{21}^T)+\lambda_{max}(C_{21}((C_{11}^{\star})^{-1}-C_{11}^{-1})((C_{21}^{\star})^T-C_{21}^T))\\
&  +\lambda_{max}(C_{21}((C_{11}^{\star})^{-1}-C_{11}^{-1})C_{21}^T)+\lambda_{max}(C_{21}C_{11}^{-1}((C_{21}^{\star})^T-C_{21}^T))
  +\lambda_{max}(C_{21}C_{11}^{-1}C_{21}^T).
  \end{align*}

By (\ref{eq:norme2_bound_C11-1}), (\ref{eq:C21}), \ref{th_2} and \ref{th_3}, we get that for a large enough $n$,
  $\rho(C_{21}^{\star}(C_{11}^\star)^{-1}(C_{21}^{\star})^T)\leq M_3'$, where $M_3'$ is a positive constant. Hence, $\nu=(2\bar{\lambda})/M_3'$. Using the same
  bounds as in (\ref{eq:E_Xk_l}) and (\ref{eq:Gjk}), we get that $c=\sqrt{\bar{\lambda}/(nM'_3)}$. Thus, with $t=n^{(c_2-1)/2}$ in the Bernstein's inequality
  $$
  \frac{t^2}{2(\nu +ct)}=\frac{n^{c_2-1}}{2(\nu+n^{-1/2}\sqrt{\bar{\lambda}/M_3'}n^{(c_2-1)/2})}=O(n^{c_2/2}).
$$
Consequently, we conclude that for $\alpha=O\left(n^{(c_2+1)/2}\right)$, $\mathbb{P} \Bigg( |(C_{21}\xi)_j| \geq\frac{\alpha}{12n}\Bigg)$ tends to $0$ as $n$ tends to infinity.

Finally, let us consider the second term in \eqref{eq:zeta} and prove that $\mathbb{P} \Bigg( \left|\left(\frac{X_2^T}{n} (\textbf{Y}-\boldsymbol{\lambda^{\star}})\right)_j\right| \geq\frac{\alpha}{12n}\Bigg)$ tends to $0$ as $n$ tends to infinity.
Denoting $H=n^{-1} X_2^T(\Lambda^\star)^{1/2}$, we observe that
$$
\left(\frac1n X_2^T (\textbf{Y}-\boldsymbol{\lambda^{\star}})\right)_j=\sum_{k=1}^n H_{jk}\frac{(Y_k-\lambda_k^\star)}{\sqrt{\lambda_k^\star}}
$$
and we apply the Bernstein's inequality to $X_k=H_{jk} Y_k/\sqrt{\lambda_k^\star}$. Then we have
\begin{align*}
\sum_{k=1}^n\mathbb{E}[X_k^2]=\sum_{k=1}^n H_{jk}^2(1+\lambda_k^\star)& \leq n\sup_{1\leq k\leq n}(1+\lambda_k^\star)\;\|H\|_2^2 \\
&\leq n\sup_{1\leq k\leq n}(1+\lambda_k^\star)\;\rho(HH^T)\leq\sup_{1\leq k\leq n}(1+\lambda_k^\star)\;\rho(C_{22}^{\star}).
\end{align*}
By Weyl's inequalities \cite[Corollary 4.3.15]{horn_johnson:2013}
$$
\lambda_{max}(C_{22}^{\star})\leq\lambda_{max}(C_{22})+\lambda_{max}(C_{22}^{\star}-C_{22}).
$$
By using the same arguments as in (\ref{eq:norme2_bound}) and \ref{th_3}, we get that
$\rho(C_{22}^{\star})\leq M_4'$ for large enough $n$ and by denoting $\bar{\lambda}=\max(1,\sup_{1\leq k\leq n}\lambda_k^\star)$, we get  that
$\nu=2\bar{\lambda}M_4'$. Then, using exactly the same argument as before we get
$$ \sum_{k=1}^n \mathbb{E}\big[ |X_k|^{\ell} \big]\leq \frac{\ell!}{2} \nu c^{\ell-2},$$ 
with $c=\sqrt{\bar{\lambda}/(n M_4')}$. Hence, for $t=n^{(c_2-1)/2}$ in the Bernstein's inequality,
  we obtain the expected result, which concludes the proof.
 
% \subsection{$A_n^c$}
% \subsubsection{First half of the probability}
% Until page 14.
% beta are close therefore lambda are close. don't include the proof?
% \subsubsection{Second half of the probability}
% \subsection{$B_n^c$}
% \subsubsection{First half of the probability}
% \subsubsection{Second half of the probability}

%% The Appendices part is started with the command \appendix;
%% appendix sections are then done as normal sections
%% \appendix

%% \section{}
%% \label{}

%% If you have bibdatabase file and want bibtex to generate the
%% bibitems, please use
%%
\bibliographystyle{elsarticle-num} 
\bibliography{biblio}

%% else use the following coding to input the bibitems directly in the
%% TeX file.

% \begin{thebibliography}{00}

% %% \bibitem{label}
% %% Text of bibliographic item

% \bibitem{}

% \end{thebibliography}


\end{document}
\endinput
%%
%% End of file `elsarticle-template-num.tex'.
