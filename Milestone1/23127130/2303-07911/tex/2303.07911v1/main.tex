
\documentclass[aps,superscriptaddress,twocolumn,floatfix]{revtex4-2}


\usepackage{amsmath}
\usepackage{amssymb}
\usepackage[utf8]{inputenc}
\usepackage{graphicx}% Include figure files
\usepackage{dcolumn}% Align table columns on decimal point
\usepackage{bm}% bold math
\usepackage[mathlines]{lineno}% Enable numbering of text and display math
%\linenumbers\relax % Commence numbering lines
\usepackage{newtxtext,newtxmath}
\usepackage{anyfontsize}

\usepackage{subfigure}
\usepackage[colorlinks = true, linkcolor = blue, urlcolor=blue, citecolor = red, anchorcolor = green,]{hyperref}

\newtheorem{theorem}{Theorem}
\newtheorem{acknowledgement}[theorem]{Acknowledgement}
\newtheorem{algorithm}[theorem]{Algorithm}
\newtheorem{axiom}[theorem]{Axiom}
%\newtheorem{case}[theorem]{Case}
\newtheorem{claim}[theorem]{Claim}
\newtheorem{conclusion}[theorem]{Conclusion}
\newtheorem{condition}[theorem]{Condition}
\newtheorem{conjecture}[theorem]{Conjecture}
\newtheorem{corollary}[theorem]{Corollary}
\newtheorem{criterion}[theorem]{Criterion}
\newtheorem{definition}[theorem]{Definition}
\newtheorem{example}[theorem]{Example}
\newtheorem{exercise}[theorem]{Exercise}
\newtheorem{lemma}[theorem]{Lemma}
\newtheorem{notation}[theorem]{Notation}
\newtheorem{problem}[theorem]{Problem}
\newtheorem{proposition}[theorem]{Proposition}
\newtheorem{remark}[theorem]{Remark}
\newtheorem{solution}[theorem]{Solution}
\newtheorem{summary}[theorem]{Summary}
\newcommand{\qipeb}{QIP$_{\operatorname{EB}}(2)$}
\newenvironment{proof}[1][Proof]{\noindent\textbf{#1.} }{\ \rule{0.5em}{0.5em}}


\allowdisplaybreaks

\pdfstringdefDisableCommands{\def\eqref#1{(\ref{#1})}}


\begin{document}
\preprint{APS/123-QED}


\title{Quantum Steering Algorithm for Estimating Fidelity of Separability}

\author{Aby Philip}
\author{Soorya Rethinasamy}
\affiliation{School of Applied and Engineering Physics,
Cornell University, Ithaca, New York 14850, USA}

\author{Vincent Russo}
\affiliation{Unitary Fund}


\author{Mark M. Wilde}
\affiliation{School of Electrical and Computer Engineering,
Cornell University, Ithaca, New York 14850, USA}
\affiliation{School of Applied and Engineering Physics,
Cornell University, Ithaca, New York 14850, USA}

\date{\today}

\begin{abstract}
    Quantifying entanglement is an important task by which the resourcefulness of a state can be measured. Here we develop a quantum algorithm that tests for and quantifies the separability of a general bipartite state, by making use of the quantum steering effect. Our first separability test consists of a distributed quantum computation involving two parties: a computationally limited client, who prepares a purification of the state of interest, and a computationally unbounded server, who tries to steer the reduced systems to a probabilistic ensemble of pure product states. To design a practical algorithm, we replace the role of the server by a combination of parameterized unitary circuits and classical optimization techniques to perform the necessary computation. The result is a variational quantum steering algorithm (VQSA), which is our second separability test that is better suited for the capabilities of quantum computers available today. This VQSA has an additional interpretation as a distributed variational quantum algorithm (VQA) that can be executed over a quantum network, in which each node is equipped with classical and quantum computers capable of executing VQA. We then simulate our VQSA on noisy quantum simulators and find favorable convergence properties on the examples tested. We also develop semidefinite programs, executable on classical computers, that benchmark the results obtained from our VQSA. Our findings here thus provide a meaningful connection between steering, entanglement, quantum algorithms, and quantum computational complexity theory. They also demonstrate the value of a parameterized mid-circuit measurement in a VQSA and represent a first-of-its-kind application for a distributed VQA. Finally, the whole framework  generalizes to the case of multipartite states and entanglement. 
\end{abstract}


\maketitle



    \textit{Introduction}.---Entanglement is a unique feature of quantum mechanics, initially brought to light by Einstein, Podolsky, and Rosen   \cite{Einstein1935}. Many years later, the modern definition of entanglement was given \cite{W89}, which we recall now.
    A bipartite quantum state $\sigma_{AB}$ of two spatially separated systems $A$ and $B$ is separable (unentangled) if it can be written as a probabilistic mixture of product states \cite{W89}:
    \begin{equation}
    \label{eqn:sep-state-informal}
    \sigma_{AB}=\sum_{x \in \mathcal{X}} p(x)\, \psi^{x}_{A} \otimes \phi^{x}_{B}    ,
    \end{equation}
     where $\{p(x)\}_{x\in \mathcal{X}}$ is a probability distribution and $\psi^{x}_{A}$ and $\phi^{x}_{B} $ are pure states. The idea here is that the correlations between~$A$ and $B$ can be fully attributed to a classical, inaccessible random variable with probability distribution $\{p(x)\}_{x\in \mathcal{X}}$.
      
     The definition above is straightforward to write down, but it is a different matter to formulate an algorithm to decide if a general state is separable; in fact, it has been proven to be computationally difficult in a variety of frameworks \cite{G03, G10, HMW13, HMW14, GHMW15}. Intuitively, deciding the answer requires performing a search over all possible probabilistic decompositions of the state, and there are too many possibilities to consider. Regardless, determining whether a general state $\rho_{AB}$ is separable or entangled, known as the separability problem, is a fundamental problem of interest relevant to various fields of physics, including condensed matter \cite{RevModPhys.80.517,cramer2011measuring, laflorencie2016quantum},  quantum gravity \cite{Takayanagi_2012,bose2017spin,marletto2017gravitationally,Qi2018,Swingle18}, quantum optics \cite{RevModPhys.92.035005}, and quantum key distribution \cite{Ekert91,PhysRevLett.113.140501}. In quantum information science, entanglement is the core resource in several basic quantum information processing tasks \cite{Ekert91,bennett1992communication,bennett1993teleporting}, making the separability problem essential in this field as well.

    Part of the challenge in using entangled states for various tasks is that they are hard to produce and maintain faithfully on any physical platform. The utility of entangled states drops off dramatically the further they are from being perfectly or maximally entangled. Therefore, assessing the quality of entangled states produced becomes an important task, thus motivating the problem of quantifying entanglement \cite{BDSW96,VPRK97,VP98,Horodecki2009}, in addition to deciding whether entanglement is present. 
    
    
    To check whether a state is entangled and to quantify its entanglement content experimentally, a rudimentary approach employs state tomography to reconstruct the density matrix and check whether the matrix represents a state that is entangled \cite{Home2006, Steffen2006}. However, the computational complexity of this method scales exponentially with the number of qubits, thus prohibiting its use on larger states of interest. With the rapid development of quantum computers of increasing size, it is already infeasible to perform tomography in order to estimate the density matrix describing the state of these computers. It is even further daunting to address the separability problem using various well-known one-sided entanglement tests  \cite{Peres1996, Horodecki1996,W89a, Doherty2004}. This leaves us to seek out alternative methods for addressing the separability problem, and one forward-thinking direction is to employ a quantum computer to do so \cite{HMW13, HMW14, GHMW15,LRW21}. 

    An approach to addressing the separability problem, which we employ here, involves the quantum steering effect \cite{Schrodinger1935, Schroedinger1935}. The idea of steering is that, if two distant systems are entangled, distinct probabilistic ensembles of states can be prepared on one system by performing distinct measurements on the other system. To describe this phenomenon more precisely, we can employ some elementary notions from quantum mechanics. Let $\psi_{CD}$ be a pure state of two distant quantum systems~$C$ and~$D$, and let $\rho_C = \operatorname{Tr}_D[\psi_{CD}]$ be the reduced state of the system~$C$. Then by performing a measurement on the system $D$, it is possible to realize a probabilistic ensemble $\{(p(z),\psi^z_C)\}_z$ of pure states on the system $C$ that satisfies $\rho_C = \sum_z p(z)\psi^z_C$. Moreover, to each such possible probabilistic decomposition of $\rho_C$, there exists a measurement acting on~$D$ that can realize this decomposition.  In more recent years, steering has been a topic of interest on its own, with applications to quantum key distribution \cite{CS17,UCNG20}, quantum optics \cite{PhysRevLett.128.200401,PhysRevA.106.042414}, and the foundations of quantum mechanics \cite{wittmann2012loophole,PhysRevA.91.012112}.

    
    
    As suggested above, we can make a non-trivial link between the separability problem and steering, which offers a quantum mechanical method for approaching the former. To see it, recall that
    a purification of the separable state $\sigma_{AB}$ in \eqref{eqn:sep-state-informal} 
      is a pure state $\varphi_{RAB} $ that satisfies $\operatorname{Tr}_R[\varphi_{RAB}] = \sigma_{AB}$, and consider that one such  choice of the state vector $|\varphi\rangle_{RAB}$ in this case is as follows:
    \begin{equation}
        |\varphi\rangle_{RAB}=\sum_{x} \sqrt{p(x)}\,|x\rangle_R\otimes |\psi^{x}\rangle_{A} \otimes |\phi^{x}\rangle_{B},
        \label{eq:purif-sep-state}
    \end{equation}
    where $\{ |x\rangle_R\}_x$ is an orthonormal basis. Purifications are not unique, but all other purifications of $\sigma_{AB}$ are related to the one in \eqref{eq:purif-sep-state} by the action of a unitary operation on the reference system $R$ \cite{NC00}.
    By inspecting~\eqref{eq:purif-sep-state}, we see that the systems $A$ and $B$ can be steered into the probabilistic ensemble $\{(p(x),\psi^{x}_{A} \otimes \phi^{x}_{B})\}_x$ of product states by performing the projective measurement $\{|x\rangle\!\langle x|_R\}_x$ on the reference system $R$ of $\varphi_{RAB}$. This leads to an idea for testing separability in the general case. If a purification of a general state $\rho_{AB}$ is available and the state $\rho_{AB}$ is indeed separable, then one can a) try to find the unitary that realizes the purification in \eqref{eq:purif-sep-state} and b) perform the measurement $\{|x\rangle\!\langle x|_R\}_x$ on the reference system~$R$. After receiving the outcome $x$,  one can then finally test whether the reduced state is a product state.
    
    As we will see in much more detail later, the basic idea outlined above is at the heart of our method to test whether a state is separable. Even better, we will see that this approach leads to a quantum algorithm and complexity-theoretic statements for quantifying the amount of entanglement in the state. As such, our findings here provide a meaningful connection between steering, entanglement, quantum algorithms, and quantum computational complexity theory, which hitherto has not been observed. 
    
    In this paper, we expound on the idea sketched above to develop various separability tests that make use of the quantum steering effect. One separability test   consists of a distributed quantum computation involving two parties: a computationally unbounded server, called a prover and which can in principle perform any quantum computation imaginable, and a computationally limited client, called a verifier, which can perform time-efficient quantum computations (see Figure~\ref{fig:Max_Sep_Fidelity_QIP_EB}). We also employ concepts from quantum computational complexity theory \cite{watrous2009complexity,VW15} in order to understand how difficult this test is to perform. Another separability test results from a modification of the previous one, in an attempt to design a practical algorithm: we replace the prover with a combination of parameterized unitary circuits and classical optimization techniques to perform the necessary computation.  This results in a variational quantum steering algorithm (VQSA) that approximates the first separability test, while suiting the capabilities of quantum computers available today (see Figure~\ref{fig:Max_Sep_Fidelity_VQA}). The concept of quantum steering is again at the heart of our VQSA, just like the test for separability it approximates. Interestingly, the acceptance probability of both tests is related to an entanglement measure called fidelity of separability \cite{VPRK97,VP98}.
    
    

    Next we report the results of simulations of the VQSA on publicly available quantum simulators and find that they show favorable convergence properties on the examples tested. In light of the limited scale and error tolerance of near-term quantum computers, we develop semidefinite programs (SDP) to approximate the fidelity of separability using positive-partial-transpose (PPT) conditions \cite{Peres1996, Horodecki2009} and $k$-extendibility \cite{W89a,Doherty2004} to benchmark the results obtained from our VQSA. Keeping in mind that variational quantum algorithms (VQAs), in general, are prone to encountering barren plateaus \cite{McClean2018}, we also explore how we can mitigate this issue for our algorithms, by making use of the ideas presented in \cite{Cerezo2021a}.

    

    Our approach introduced here is distinct from recent work on quantum algorithms for estimating entanglement.
    For example, VQAs have been used to address this problem by estimating the Hilbert--Schmidt distance \cite{Consiglio2022}, by creating a zero-sum game using parameterized unitary circuits \cite{Yin2022}, and by employing symmetric extendibility tests \cite{LRW21}. 
    The work of \cite{AM22} is the closest related to ours, but the basic test used there requires two copies of the state of interest and controlled swap operations, while our VQSA does not require either.
    In contrast, in our paper, we introduce a paradigm for VQAs involving parameterized mid-circuit measurements, which is the core of our method for estimating entanglement, and we suspect that this approach will be useful in future work for a wide variety of VQAs. Furthermore, as we show in Theorems~\ref{theorem:qip-eb_msf} and~\ref{theorem:global_cost_func}, the acceptance probabilities of our algorithms, in the ideal case, are directly related to a bona fide entanglement measure, the fidelity of separability.
     
    

    \textit{Test for separability of pure states}.---As an initial ``warmup'' step, let us formulate a simple test for the separability of pure states. From \eqref{eqn:sep-state-informal}, we can see that a pure bipartite state $\varphi_{AB}$ is separable  if it can be written in product form, as
    \begin{equation}
     \label{eq:pure-sep-state}
     \varphi_{AB}=\psi_{A} \otimes \phi_{B},
    \end{equation}
    where $\psi_{A}$ and $ \phi_{B}$ are pure states.
    The test that we develop below is important because it will reappear as part of the test for separability in the general case, along with quantum steering. Additionally, our approach is slightly different from the standard approach for testing entanglement of pure states, which employs two copies of the state in a swap test \cite{Brennen03,HM10,GHMW15}. Instead, our approach requires only a single copy of the state.
    
    
    

    Our pure-state separability test consists of a distributed quantum computation  involving a prover and a verifier (see Figure~\ref{fig:swaptest_pure}). 
    The computation starts with the  verifier preparing the pure state $\psi_{AB}$. The prover sends the verifier the pure state $\phi_{A^\prime}$ in register $A^\prime$. (We note that the prover can send a mixed state; however, the maximum acceptance probability of the test is achieved by a pure state. Hence, without loss of generality, it suffices for the prover to send a pure state.) 
    The verifier then performs the standard swap test \cite{BBD+97,BCWW01} on $A$ and $A^\prime$ and accepts if the measurement outcome is zero. In the standard model of quantum computational complexity \cite{watrous2009complexity,VW15}, the prover attempts to get the verifier to accept the swap test with as high a probability as possible. Thus, in this scenario, the prover selects  $\phi_{A^\prime}$ to maximize the overlap between the reduced stated $\psi_{A} \coloneqq \operatorname{Tr}_B[\psi_{AB}]$ and $\phi_{A^\prime}$. The maximum acceptance probability is then equal to
    \begin{align}
    & \max_{\phi}\operatorname{Tr}[(\Pi_{A^{\prime}A}^{\operatorname{sym}}\otimes I_B)(\phi_{A^\prime} \otimes \psi_{AB})]\notag \\
    & =
    \frac{1}{2}\left(1+\max_{\phi}\operatorname{Tr}[F_{A'A}(\phi_{A'} \otimes \psi_A)]\right) \\
    & =
    \frac{1}{2}\left(1+\max_{\phi}\operatorname{Tr}[\phi_{A}  \psi_A]\right) 
     = \frac{1}{2}\left(1+\left\|\psi_{A}\right\|_{\infty}\right),    
    \label{eq:initial-swap-test}
    \end{align}
    where $F_{A'A}$ is the unitary swap operator acting on systems $A'$ and $A$, the projector $\Pi_{A^{\prime}A}^{\operatorname{sym}} \coloneqq  \frac{1}{2} \left( I_{A'A} + F_{A'A}\right)$ projects onto the symmetric subspace of $A^{\prime}$ and $A$, and $\left\|\psi_{A}\right\|_{\infty}$ is the spectral norm of the reduced state $\psi_{A}$ (equal to its largest eigenvalue). Since $\left\|\psi_{A}\right\|_{\infty} = 1$ if and only if $\psi_{A}$ is a pure state and this occurs if and only if $\psi_{AB}$ is a product state, it follows that the maximal acceptance probability is equal to one if and only if~$\psi_{AB}$ is a product state.
    \begin{figure}
        \includegraphics[width=0.7\columnwidth]{figures/swaptest_pure_new.pdf}
        \caption{Pure-state separability test: The verifier is in possession of the pure state $\psi_{AB}$ of interest. The prover (indicated by the dotted box) sends the verifier a pure state $\phi_{A^\prime}$, who then performs the standard swap test on systems $A'$ and $A$. 
        As mentioned in \eqref{eq:initial-swap-test}, the acceptance probability is equal to $ \frac{1}{2}(1+\left\|\psi_{A}\right\|_{\infty})$.} 
        \label{fig:swaptest_pure}
    \end{figure} 
    
    \textit{Test for separability of mixed states}.---At this point, we generalize the above distributed quantum computation to a mixed-state separability test, by making use of the quantum steering effect. 
    Recall that a bipartite state is separable or unentangled if it can be written in the form of \eqref{eqn:sep-state-informal}, where  $\left\vert\mathcal{X}\right\vert\leq\text{rank}(\sigma_{AB})^{2}$ \cite{W89,watrous_2018}. 
    The computation (depicted in Figure~\ref{fig:Max_Sep_Fidelity_QIP_EB}) begins with the verifier preparing a purification $\psi_{RAB}$\ of $\rho_{AB}$. The verifier sends the system $R$ to a quantum prover, whom, in our model, we restrict to performing entanglement-breaking channels. The prover thus performs an entanglement-breaking channel on the reference system $R$ and sends a system $A^{\prime}$ to the verifier. An entanglement-breaking channel $\mathcal{E}_{R\rightarrow A^{\prime}}$ can always be written as a measure-and-prepare channel \cite{HSR03}, as follows:
    \begin{equation}\label{eqn:ent_breaking}
        \mathcal{E}_{R\rightarrow A^{\prime}}(\cdot)=\sum_{x}\operatorname{Tr}[\mu^{x}_{R}(\cdot)]\phi^{x}_{A^{\prime}},
    \end{equation}
    where $\{\mu^{x}_{R}\}_{x}$ is a rank-one positive operator-valued measure (POVM) and $\{\phi^{x}_{A^{\prime}}\}_{x}$ is a set of pure states. (Due to the above measure-and-prepare decomposition of an entanglement-breaking channel, we can alternatively think of the prover as being split into two provers, a first who is allowed to perform a general quantum operation, followed by the communication of classical data to a second prover, who then is allowed to perform a general operation before communicating quantum data to the verifier. However, we proceed with the single-prover terminology in what follows.) By performing the measurement portion of the entanglement-breaking channel, the prover has in essence steered the verifier's systems $A$ and $B$ to a certain probabilistic ensemble of pure states. After steering the verifier's system, the prover then sends system $A^\prime$ to the verifier, by using the preparation portion of the entanglement-breaking channel. The verifier finally performs a swap test on system $A$ and $A^\prime$ and accepts if and only if the measurement outcome of the swap test is zero. As we indicated previously, the standard model in quantum computational complexity theory \cite{watrous2009complexity,VW15} is that the prover is always trying to get the verifier to accept the computation: in this scenario, the prover steers the verifier's systems $A$ and $B$ to an ensemble that has maximum overlap with a product-state ensemble and then sends an appropriate state to pass the swap test with the highest probability possible.
    
    The maximum acceptance probability of the distributed quantum computation detailed above is equal to
    \begin{equation}
    \label{eqn:qip-eb_accept_prob}
        \max_{\mathcal{E}\in\operatorname{EB}_{R\to A^\prime}}\operatorname{Tr}[(\Pi_{A^{\prime}A}^{\operatorname{sym}}\otimes I_{RB})\mathcal{E}_{R\rightarrow A^{\prime}}(\psi_{RAB})],
    \end{equation}
    where $\Pi_{A^{\prime}A}^{\operatorname{sym}}$ is the projector onto the symmetric subspace of the $A^{\prime}$ and $A$ systems, and  $\operatorname{EB}_{R\to A^\prime}$ denotes the set of all entanglement-breaking channels with input system $R$ and output system $A^\prime$.  We find in Theorem~\ref{theorem:qip-eb_msf} below that the maximum acceptance probability in~\eqref{eqn:qip-eb_accept_prob} can be expressed as a simple function of the fidelity of separability of $\rho_{AB}$, the latter defined as \cite{VPRK97,VP98}
    \begin{equation}
    \label{eq:max-sep-fid-def}
        F_{s}(\rho_{AB}) \coloneqq \max_{\sigma_{AB}\in \operatorname{SEP}(A:B)} F(\rho_{AB},\sigma_{AB}),
    \end{equation}
    where $\operatorname{SEP}(A\!:\!B)$ is the set of separable states shared between Alice and Bob and $F(\rho,\sigma) \coloneqq \left\|\sqrt{\rho}\sqrt{\sigma}\right\|_1^2$ is the fidelity of the states $\rho$ and $\sigma$ \cite{Uhlmann1976}. The fidelity of separability is also known as the maximum separable fidelity \cite{HMW13,HMW14,GHMW15}. With this definition, we state the first key theoretical result of our paper: 
    

    \begin{figure}
        \includegraphics[width=\columnwidth]{figures/fig_QIP_EB_new.pdf}
        \caption{Test for separability of mixed states. 
        The verifier uses a unitary circuit $U^\rho$ to produce the state $\psi_{RAB}$, which is a purification of $\rho_{AB}$. The prover (indicated by the dotted box) then applies an entanglement-breaking channel $\mathcal{E}_{R\rightarrow A^{\prime}}$ on $R$ by measuring the rank-one POVM $\{\mu^{x}_{R}\}_{x}$ and then, depending on the outcome $x$, prepares a state from the set $\{\phi^{x}_{A^{\prime}}\}_{x}$. The final state is sent to the verifier, who performs a swap test. Theorem~\ref{theorem:qip-eb_msf} states that the maximum acceptance probability of this interactive proof is equal to $\frac{1}{2}(1 + F_{s}(\rho_{AB}))$, i.e., a simple function of the fidelity of separability.}
        \label{fig:Max_Sep_Fidelity_QIP_EB}
    \end{figure}
    
    \begin{theorem}
    \label{theorem:qip-eb_msf}
        For a pure state $\psi_{RAB}$, the following holds%
        \begin{equation}\label{eqn:qip-eb_msf}
            \max_{\mathcal{E}\in\operatorname{EB}_{R\to A^\prime}}\operatorname{Tr}[(\Pi_{A^{\prime}A}^{\operatorname{sym}}\otimes I_{RB})\mathcal{E}_{R\rightarrow A^{\prime}}(\psi_{RAB})]=\frac{1 + F_{s}(\rho_{AB})}{2},
        \end{equation}
        where $F_s(\rho_{AB})$ is the fidelity of separability of the state $\rho_{AB} = \operatorname{Tr}_{R}[\psi_{RAB}]$.
    \end{theorem}
    
    See Appendix~\ref{appendix:proof_swap-test-eb-channel} for our proof, which relies on two important facts. The first is that the fidelity of separability can be written in terms of a convex roof as follows \cite[Theorem~1]{Streltsov2010}:
        \begin{equation}
        \label{eq:convex-decomp-max-sep-fid}
            F_{s}(\rho_{AB})=  \max_{\substack{\{(p(x),\psi_{AB}^{x})\}_{x},\\\rho_{AB}=\sum_{x}p(x)\psi_{AB}^{x}}}\sum_{x}p(x)F_{s}(\psi_{AB}^{x}),
        \end{equation}
        where $\{p(x)\}_x$ is a probability distribution and each $\psi_{AB}^{x}$ is a pure state.
        See also \cite[Lemma~1]{Regula2018}.
        The second fact is that, for a pure bipartite state, $F_{s}(\psi_{AB})$ can be rewritten as \cite[Section~6.2]{Streltsov2010}
        \begin{equation}
            F_{s}(\psi_{AB})= \left\Vert \psi_{A}\right\Vert _{\infty}.
            \label{eq:max-sep-fid-inf-norm}%
        \end{equation}      
For completeness, we provide proofs of \eqref{eq:convex-decomp-max-sep-fid} and \eqref{eq:max-sep-fid-inf-norm} in Appendices~\ref{appendix:proof_alt_streltsov} and \ref{appendix:proof_max-sep-fid-inf-norm}, respectively. It follows from \eqref{eqn:sep-state-informal} and \eqref{eq:max-sep-fid-inf-norm}  that $\sum_{x}p(x)\left\Vert \psi_{A}^{x}\right\Vert _{\infty} = 1$ for a separable state, which is the maximum possible value of $F_{s}(\rho_{AB})$. Hence the distributed quantum computation in Figure~\ref{fig:Max_Sep_Fidelity_QIP_EB} tests and quantifies the separability of a state, by estimating its fidelity of separability. However, we note that there are complexity-theoretic subtleties associated with the parallel repetition of this algorithm, which we discuss in detail in Appendix~\ref{app:par-rep}. Finally, note that the computation in Figure~\ref{fig:Max_Sep_Fidelity_QIP_EB} can be reduced to that in Figure~\ref{fig:swaptest_pure} if the purifying system $R$ is trivial, implying that the verifier only prepares a pure state on systems $A$ and $B$ in this case. 
        
    \textit{VQSA to test and quantify separability}.---In this section, we provide a practical method to estimate the fidelity of separability. In the two tests for separability detailed prior, we note that the swap test at the end of both computations essentially leads to  a measure of overlap between the state of the verifier's system and the state provided to the verifier by the prover. Another important point to recall is that, in the real world, there is no computationally unbounded quantum prover available to us to provide the ideal states required for the aforementioned tests. 

    Taking both these points into consideration, we modify the computation scenario in Figure~\ref{fig:Max_Sep_Fidelity_QIP_EB} to 1)  measure the necessary overlaps directly and 2) make use of quantum variational techniques \cite{Cerezo2021} (parameterized unitary circuits and classical optimization of parameters) to approximate the actions of a computationally unbounded prover. The resulting procedure also tests and quantifies the separability of a given state by estimating its fidelity of separability. This procedure is a different sort of quantum variational technique, which we call a variational quantum steering algorithm (VQSA). As can be seen in Figure~\ref{fig:Max_Sep_Fidelity_VQA}, quantum steering is at the core of the VQSA via the use of a parameterized mid-circuit measurement. 

    \begin{figure}
        \includegraphics[width=\columnwidth]{figures/fig_VQA_new.pdf}
        \caption{Quantum part of the VQSA to estimate the fidelity of separability $F_s(\rho_{AB})$. 
        The unitary circuit $U^\rho$ prepares the state $\psi_{RAB}$, which is a purification of $\rho_{AB}$. The parameterized circuit $W_R(\Theta)$ acts on $R$ to evolve $\psi_{RAB}$ to another purification of $\rho_{AB}$. The measurement that follows, labeled ``steering measurement,'' steers the systems $AB$ to be in a pure state $\psi_{AB}^{x}$, if the measurement outcome~$x$ occurs. Conditioned on the outcome~$x$, the final parameterized circuit $U^{x}_{A}(\Theta^x)$ and the subsequent measurement accepts with probability~$\left\|\psi_{A}^{x}\right\|_{\infty}$.} 
        \label{fig:Max_Sep_Fidelity_VQA}
    \end{figure}
    
    
    
    Our VQSA can be understood by noting that  \eqref{eq:convex-decomp-max-sep-fid} has two maximization problems: first is the maximization to compute $F_{s}(\psi_{AB}^{x})$ and the other is the maximization over all pure-state decompositions of $\rho_{AB}$, such that $\sum_{x}p(x)\psi_{AB}^{x}=\rho_{AB}$. 
    
    As before, let $\rho_{AB}$ denote the state for which we want to estimate the fidelity of separability, and let $\psi_{RAB}$ be a purification of it, which results from the action of the unitary operator $U^\rho$ on the all-zeros pure state $|0\rangle\!\langle 0|$. 
    Once we have $\psi_{RAB}$, we can attempt to access all possible pure-state decompositions $\{(p(x),\psi_{AB}^{x})\}_x$ of $\rho_{AB}$ by acting on system $R$ with unitary operations. To do so, we use our first parameterized unitary $W_R(\Theta)$.  To ensure that we have a sufficient number of measurement outcomes (to cover the possible case when $|\mathcal{X}| = \text{rank}(\rho_{AB})^2$), we can prepare some ancilla qubits in the all-zeros state, for a system $R'$, and act with $W$ on $R$ and $R'$. However, without loss of generality, these extra qubits can simply be lumped together as part of an overall reference system, relabeled as $R$. 
    
    After the action of $W_R(\Theta)$, the reference system is measured in the standard basis, and based on the outcome $x$, the post-measurement state of the system $AB$ is a pure state $\psi^x_{AB}$. We then estimate the maximum eigenvalue of the reduced state~$\psi^x_{A}$: this can be accomplished by performing a parameterized unitary $U^{x}_{A}(\Theta^x)$, based on the outcome $x$, on the reduced state~$\psi_A^x$, measuring all qubits of $A$ in the computational basis, and accepting if the all-zeros outcome occurs.
    
    Using a hybrid quantum--classical optimization loop, we can maximize the acceptance probability to estimate the value of the fidelity of separability. The quantum part of this VQSA is summarized in Figure~\ref{fig:Max_Sep_Fidelity_VQA}.
    
    
    \begin{theorem}
    \label{theorem:global_cost_func}
        If the parameterized unitary circuits involved in the quantum part of the VQSA, summarized in Figure~\ref{fig:Max_Sep_Fidelity_VQA}, can express all possible unitary operators of their respective systems, then the maximum acceptance probability of the quantum circuit is equal to $F_s(\rho_{AB})$.
    \end{theorem} 
    
    See Appendix~\ref{appendix:step_by_step} for a detailed proof.
    
    Here we note that our algorithm is not unitary, due to the mid-circuit measurement 
    on system $R$ and the consequent conditional operation that is applied on system $A$. This is an important distinction from most VQAs, which do not make use of a parameterized mid-circuit measurement. Additionally, our VQSA can be generalized to measure the fidelity of separability of multipartite states using the appropriate definition of separability in the multipartite scenario and the multipartite version of \eqref{eq:max-sep-fid-def}, as given in \cite{Streltsov2010}. See Appendix~\ref{appendix:mulitpartite} for details. 

    We can also think of our VQSA as a distributed variational quantum algorithm for measuring entanglement in a bipartite state. See \cite{arxiv.2208.00450} for a previous instance of this concept. Indeed, our algorithm can be executed over a quantum network in which each node in the network has quantum and classical computers capable of performing VQA. The initial part of the algorithm distributes $R$ to Rob, $A$ to Alice, and $B$ to Bob, who are all in distant locations. Then Rob performs the parameterized measurement, sends the outcome over a classical channel to Alice, who then performs another parameterized measurement. Then they can repeat this process to assess the quality of the entanglement between Alice and Bob. This interpretation is even more interesting in terms of quantum networks for the multipartite case, in which the classical data gets broadcast from Rob to all the other nodes, with the exception of the last one. See Figure~\ref{fig:Max_Sep_Fidelity_VQSA_Multi} in Appendix~\ref{appendix:mulitpartite} for this multipartite algorithm.

    \textit{Computational complexity-theoretic results}.---
    In this section, we characterize the computational complexity of estimating the fidelity of separability $F_s(\rho_{AB})$. The complexity-theoretic approach gives us a way to classify the separability problem based on its difficulty and complexity. Analyses of this form can be effectively done within the framework of quantum computational complexity theory~\cite{watrous2009complexity,VW15}.
    
    In the paradigm of quantum complexity theory, a complexity class is a set of problems that require similar resources to solve. If a complexity class $A$ is contained within another class $B$, then some problems in $B$ require more computational resources to solve than problems in $A$. To effectively characterize the difficulty of a class of problems, we pick a problem that is representative of the class, or complete for the class. A problem $h$ is said to be complete for a complexity class $A$ if $h$ is contained in the class and the ability to solve problem $h$ can be extended efficiently to solve every other problem in $A$. 

    In this work, we define \qipeb\ to be the complexity class containing problems that can be solved using a prover restricted to using only entanglement-breaking channels, which processes a quantum message received from the verifier and sends back a quantum message to the verifier. Thus, computing the fidelity of separability of a given state then falls within \qipeb, as can be seen from Figure~\ref{fig:Max_Sep_Fidelity_QIP_EB}. To fully characterize this novel complexity class, we provide a complete problem for it. We establish that, given quantum circuits to generate a channel $\mathcal{N}_{A\rightarrow B}$ and a state $\rho_{B}$, estimating the following quantity is complete for \qipeb:%
        \begin{multline}
           \max_{\substack{\{  (p(x),\psi^{x})\}  _{x},\left\{  \varphi^{x}\right\}_{x},\\
           \rho_{B} = \sum_{x}p(x)\psi_{B}^{x}
           }}   \sum_{x}p(x)F(\psi_{B}^{x},\mathcal{N}_{A\rightarrow B}(\varphi_{A}^{x}))  .
        \end{multline}
    See Appendix~\ref{appendix:qipeb} for details, as well as an interpretation of this problem. 

    By placing the problem of estimating the fidelity of separability in the class \qipeb, we establish results that link quantum steering and the separability problem to quantum computational complexity theory. Furthermore, we show that the complexity class \qipeb\ is contained in QIP \cite{watrous2003pspace,kitaev2000parallelization}, and contains QAM \cite{marriott2004quantum} and QSZK \cite{watrous2006zero}. It also follows, as a direct generalization of the hardness results from \cite{HMW13,HMW14}, that the problem of estimating the fidelity of separability is hard for QSZK and NP. All of the aforementioned complexity classes are considered to be, in the worst case, out of reach of the capabilities of efficient quantum computers. See details in Appendix~\ref{appendix:complexity_placements}. However, following the approach of \cite{210808406}, we can try to solve some instances of problems in these classes using parameterized circuits and VQAs.

    \textit{Benchmarking via semidefinite programs}.---In light of the limited scale and error tolerance of near-term quantum computers, we develop semidefinite programs (SDPs) to benchmark the results from our VQSA because the ideal outcomes can be estimated classically for small numbers of qubits.  
    
    First, let us recall that fidelity between two quantum states has an SDP formulation \cite{Wat13}. Since there is no semidefinite constraint that directly corresponds to optimizing over the set of separable states \cite{Fawzi2021}, we can approximate the fidelity of separability of a state by maximizing its fidelity with positive partial transpose (PPT) \cite{Peres1996,Horodecki1996} states and $k$-extendible states \cite{W89a,Doherty2004}. Further noting that the PPT and $k$-extendibility constraints are positive semidefinite constraints, we obtain our first benchmark $\widetilde{F}_s^1(\rho_{AB})$, defined in Appendix~\ref{appendix:ppt-k-state-sdp}.
    
    The second benchmark can be obtained using \eqref{eqn:qip-eb_accept_prob}. Just like PPT and $k$-extendible states were used to approximate separable states for the first benchmark, we use PPT channels \cite{Rai99,Rai01} and $k$-extendible channels \cite{PBHS13,Kaur2018,KDWW21,BBFS18} to approximate entanglement-breaking channels, leading to our second benchmark $\widetilde{F}_s^2(\rho_{AB})$. We show that $\widetilde{F}_s^2(\rho_{AB})$ is an SDP and approximates the fidelity of separability in the following fashion: 
        \begin{equation}
        \label{eqn:swap-test-ppt-k-channel-sdp}
         F_{s}(\rho_{AB}) \leq \widetilde{F}_s^2(\rho_{AB}) \leq 
            F_{s}(\rho_{AB})+\frac{4  \left\vert A\right\vert^{3} \left\vert B\right\vert}{k}.
        \end{equation}
    See Appendix~\ref{appendix:proof_swap-test-ppt-k-channel-sdp} for a  proof.
    
    

    \textit{Simulations}.---We now present a few simulations of our VQSA, which demonstrate that it can obtain an estimate of the fidelity of separability. We have selected two known examples of states to indicate the working of the VQSA and the benchmarks we have developed. The first example is  a random product state, shown in Figure~\ref{fig:Max_Sep_Fidelity_Product}. The fidelity of separability of a product state is equal to one, and the figure shows that our VQSA converges to the correct value. We also evaluate our benchmarks for different levels of the $k$-extendibility hierarchy. We repeat these exercises for a (3/4,1/4) probabilistic mixture of two Bell states. See Figure~\ref{fig:Max_Sep_Fidelity_Bell} for the results. The fidelity of separability of this state is equal to 3/4 and the benchmarks and VQSA converge to this value. The jitters in the value of fidelity between iterations of the VQSA can be attributed to the shot noise in estimating the acceptance probability, using the Qiskit Aer simulator and the fact that we use Qiskit's Simultaneous Perturbation Stochastic Approximation (SPSA) optimizer to do the classical optimization. We provide more examples in Appendix~\ref{appendix:simulations}.
    
    \begin{figure}
        \centering
        \subfigure[Fidelity of separability calculated for a random product state using our VQSA (blue line) converging to the true value of 1.]{\includegraphics[width=\columnwidth]{figures/plot_random_sep_state.pdf}\label{fig:Max_Sep_Fidelity_Product}}
    
        \subfigure[Fidelity of separability calculated for a biased Bell state using our VQSA (blue line) converging to the true value of 0.75.]{\includegraphics[width=\columnwidth]{figures/plot_biased_bell_state.pdf}\label{fig:Max_Sep_Fidelity_Bell}}
        \caption{Fidelity of separability estimated using the VQSA and benchmarked by $\widetilde{F}_s^1$ and $\widetilde{F}_s^2$.}
    \end{figure}
    
    An important issue with variational quantum techniques, such as VQAs, is the emergence of barren plateaus or vanishing gradients as the number of qubits involved increases \cite{McClean2018}. However, recent results have shown that this problem can be mitigated by switching from a global reward function to a local reward function \cite{Cerezo2021a}. In our case, a global reward function is one where we measure all the qubits that constitute system~$A$, as done in the approach discussed in  Theorem~\ref{theorem:global_cost_func}. An example of a local reward function is where one selects a qubit in system $A$ at random to measure in the computational basis and record the outcome, accepting if the outcome is equal to zero. Our proposed local reward function can be used to obtain upper and lower bounds on our initial global reward function, following the approach of \cite[Appendix C]{Khatri2019quantumassisted} and discussed for completeness in Appendix~\ref{appendix:local_cost}. We provide simulations of the local reward function in Appendix~\ref{appendix:simulations}, indicating that the local reward function can also be used to obtain an estimate on the fidelity of separability of a given state.
    

    \textit{Conclusion}.---In this paper, we detailed a distributed quantum computation to test the separability of a quantum state. The test, at its core, makes use of quantum steering. Through this test, we demonstrated a link between quantum steering and the separability problem. The acceptance probability of this distributed quantum computation is directly related to the fidelity of separability. Using the structure of the test, we also showed computational complexity-theoretic results and established a link between quantum steering, quantum algorithms, and quantum computational complexity. By replacing the prover with a parameterized circuit, we modified this distributed quantum computation to develop our VQSA, which is a novel kind of variational quantum algorithm that uses quantum steering to address the problem of estimating the fidelity of separability. This algorithm allows for the direct estimation of the fidelity of separability, without the need for state tomography and  subsequent approximate tests on separability. Finally, we simulated our VQSA  using the noisy Qiskit Aer simulator and showed favorable convergence trends, which were compared against two classical SDP benchmarks. 

    VQSAs can be used to tackle other problems that involve quantum steering, like maximization of pure-state decompositions of quantum states. This technique may also be useful for estimating other entanglement measures that involve an optimization over the set of separable states. By applying the insights of \cite[Appendix~A]{Streltsov2010} and our approach here, it is clear that VQSAs will also be useful for estimating maximal fidelities associated with other resource theories, such as the resource theory of coherence~\cite{BCP14}. More broadly, we suspect that the paradigm of parameterized mid-circuit measurements and distributed variational quantum algorithms will be useful in addressing other computational problems of interest in quantum information science and physics.

    

    \begin{acknowledgments}
        We are especially grateful to Gus Gutoski, for providing the main idea of the quantum interactive proof detailed in Figure~\ref{fig:Max_Sep_Fidelity_QIP_EB}, back in September 2013. We also thank Paul Alsing, Zoe Holmes, Wilfred Salmon for insightful discussions, and Ludovico Lami, Bartosz Regula, and Alexander Streltsov for pointing us to \cite{Regula2018}. AP, SR, and MMW acknowledge support from the National Science Foundation under Grant No.~1907615.
    \end{acknowledgments}
    

    
\bibliographystyle{unsrt}
\bibliography{Ref}

\appendix

\section{Proof of Theorem~\ref{theorem:qip-eb_msf}}

\label{appendix:proof_swap-test-eb-channel}

    In this appendix, we prove Theorem~\ref{theorem:qip-eb_msf}, showing that the acceptance probability of the first test of separability for mixed states is equal to $\frac{1}{2}\left(1+ F_{s}(\rho_{AB})\right)$.
\bigskip 

    
    
    \begin{proof}[Proof of Theorem~\ref{theorem:qip-eb_msf}]
    Recall that an entanglement-breaking channel can be rewritten as%
    \begin{equation}
        \mathcal{E}_{R\rightarrow A^{\prime}}(\cdot)=\sum_{x}\operatorname{Tr}[\mu^{x}_{R}(\cdot)]\phi^{x}_{A^{\prime}},
    \end{equation}
    where $\{\mu^{x}_{R}\}_{x}$ is a rank-one POVM and $\{\phi^{x}_{A^{\prime}}\}_{x}$ is a set of pure states. Then we find, for fixed $\mathcal{E}_{R\rightarrow A^{\prime}}$, that%
    \begin{align}
        & \operatorname{Tr}[\Pi_{A^{\prime}A}^{\operatorname{sym}}\mathcal{E}_{R\rightarrow A^{\prime}}(\psi_{RAB})] \notag \\
        &  =\frac{1}{2}\operatorname{Tr}[(I_{A^{\prime}A}+F_{A^{\prime}A})\mathcal{E}_{R\rightarrow A^{\prime}}(\psi_{RAB})]\\
        &=\frac{1}{2}\left(  1+\operatorname{Tr}[F_{A^{\prime}A}\mathcal{E}_{R\rightarrow A^{\prime}}(\psi_{RAB})]\right).
    \end{align}
    So let us work with the expression $\operatorname{Tr}[F_{A^{\prime}A}\mathcal{E}_{R\rightarrow A^{\prime}}(\psi_{RAB})]$. Consider that%
    \begin{align}
        &  \operatorname{Tr}[F_{A^{\prime}A}\mathcal{E}_{R\rightarrow A^{\prime}}(\psi_{RAB})]\nonumber\\
        &  =\operatorname{Tr}\left[  F_{A^{\prime}A}\sum_{x}\operatorname{Tr}_{R}[\mu^{x}_{R}\psi_{RAB}]\otimes\phi^{x}_{A^{\prime}}\right]  \\
        &  =\operatorname{Tr}\left[  F_{A^{\prime}A}\sum_{x}p(x)\psi_{AB}^{x}\otimes\phi^{x}_{A^{\prime}}\right]  \\
        &  =\operatorname{Tr}\left[  F_{A^{\prime}A}\sum_{x}p(x)\psi_{A}^{x}\otimes\phi^{x}_{A^{\prime}}\right]  \\
        &  =\sum_{x}p(x)\langle\phi^{x}|_{A}\psi_{A}^{x}|\phi^{x}\rangle_{A},
    \end{align}
    where
    \begin{align}
        p(x) &  \coloneqq \operatorname{Tr}[\mu^{x}_{R}\psi_{RAB}],\\
        \psi_{AB}^{x} &  \coloneqq \frac{1}{p(x)}\operatorname{Tr}_{R}[\mu^{x}_{R}\psi_{RAB}].
    \end{align}
    Thus, the acceptance probability for a fixed entanglement-breaking channel is given by%
    \begin{equation}
        \operatorname{Tr}[\Pi_{A^{\prime}A}^{\operatorname{sym}}\mathcal{E}_{R\rightarrow A^{\prime}}(\psi_{RAB})]=\frac{1}{2}\left(  1+\sum_{x}p(x)\langle\phi^{x}|_{A}\psi_{A}^{x}|\phi^{x}\rangle_{A}\right)  .
    \end{equation}
    After optimizing over every element of $\operatorname{EB}_{R\to A^\prime}$, which denotes the set of all entanglement-breaking channels with input system $R$ and output system $A^\prime$, and realizing that optimizing over measurements in $\mathcal{E}_{R\rightarrow A^{\prime}}$ induces a pure-state decomposition of $\rho_{AB}$ and optimizing over preparation channels in $\mathcal{E}_{R\rightarrow A^{\prime}}$ gives the spectral norm of $\psi _{A}^{x}$, we find the claimed formula for the acceptance probability, when combined with the development in Appendices~\ref{appendix:proof_alt_streltsov} and \ref{appendix:proof_max-sep-fid-inf-norm}:%
    \begin{equation}
            \max_{\mathcal{E}\in\operatorname{EB}_{R\to A^\prime}}\operatorname{Tr}[(\Pi_{A^{\prime}A}^{\operatorname{sym}}\otimes I_{RB})\mathcal{E}_{R\rightarrow A^{\prime}}(\psi_{RAB})]=\frac{1 + F_{s}(\rho_{AB})}{2}.
        \end{equation}
    This concludes the proof.
    \end{proof}

\section{Alternative proof of Equation~\eqref{eq:convex-decomp-max-sep-fid}}
    
    \label{appendix:proof_alt_streltsov}

    In this appendix, we provide an alternative proof for Theorem 1 in \cite{Streltsov2010}. This proof relies on Uhlmann's theorem \cite{Uhlmann1976}, the triangle inequality, and the Cauchy--Schwarz inequality. See also \cite[Lemma~1]{Regula2018}.
    
    \begin{theorem}[\cite{Streltsov2010}]\label{theorem:Streltsov}
        The following formula holds 
        \begin{multline}
            F_{s}(\rho_{AB})= \\ \max_{\left\{(p(x),\psi_{AB}^{x})\right\}  _{x}}\left\{\sum_{x}p(x)F_{s}(\psi_{AB}^{x}):\rho_{AB}=\sum_{x}p(x)\psi_{AB}^{x}\right\},
        \end{multline}
        where $\{(p(x),\psi_{AB}^{x})\}  _{x}$ satisfies $\sum_{x}p(x)\psi_{AB}^{x}=\rho_{AB}$, all $\psi_{AB}^{x}$ are pure, and
        \begin{equation}
        F_{s}(\psi_{AB}) = \max_{|\phi\rangle_{A} , |\varphi\rangle_{B}} |\langle\psi|_{AB} |\phi\rangle_{A}\otimes|\varphi\rangle_{B}|^2. 
        \end{equation}
        
    \end{theorem}
    
    \begin{proof}
    Since the definition in \eqref{eq:max-sep-fid-def}\ requires an optimization over all separable states, we take $\left\vert \mathcal{X}\right\vert =\left(\left\vert A\right\vert \left\vert B\right\vert\right)^{2}$. The separable state in \eqref{eqn:sep-state-informal}\ is purified by%
    \begin{equation}
        |\psi^{\sigma}\rangle_{RAB}=\sum_{x\in\mathcal{X}}\sqrt{p(x)}|x\rangle_{R}|\psi^{x}\rangle_{A}|\phi^{x}\rangle_{B}.\label{eq:sep-purify}%
    \end{equation}
    Now consider a generic purification $|\psi^{\rho}\rangle_{R^{\prime}AB}$\ of $\rho_{AB}$. Recall that the dimension of the purifying system $R^{\prime}$ satisfies rank$(\rho_{AB})\leq\left\vert R^{\prime}\right\vert $ and so we can simply set $\left\vert R^{\prime}\right\vert =\left\vert A\right\vert\left\vert B\right\vert $. Taking $R^{\prime\prime}$ to be a system of dimension $\left\vert A\right\vert \left\vert B\right\vert $, we then have that%
    \begin{equation}
        |\psi^{\rho}\rangle_{R^{\prime}AB}|0\rangle_{R^{\prime\prime}}%
    \end{equation}
    purifies $\rho_{AB}$. Applying Uhlmann's theorem \cite{Uhlmann1976}, the maximum separable root fidelity can be written as%
    \begin{multline}
        \max_{\sigma_{AB}\in\operatorname{SEP}(A:B)}\sqrt{F}(\rho_{AB},\sigma_{AB})\\
        =\max_{\substack{U,\\\left\{\left(  p(x),\psi^{x}{}_{A},\phi_{B}^{x}\right)  \right\}_{x}}}\left\vert \left(  \sum_{x^{\prime}}\sqrt{p(x^{\prime})}\langle x^{\prime}|_{R}\langle\psi^{x^{\prime}}|_{A}\langle\phi^{x^{\prime}} |_{B}\right)\right. \times\\ 
        \left.\left(  U_{R^{\prime}R^{\prime\prime}\rightarrow R}\otimes I_{AB}\right)  |\psi^{\rho}\rangle_{R^{\prime}AB}|0\rangle_{R^{\prime\prime}}\right\vert ,
    \end{multline}
    where the maximization is over every unitary $U_{R^{\prime}R^{\prime\prime}\rightarrow R}$. Expanding $U_{R^{\prime}R^{\prime\prime}\rightarrow R}|\psi^{\rho}\rangle_{R^{\prime}AB}|0\rangle_{R^{\prime\prime}}$ in terms of the standard basis $|x\rangle$ as%
    \begin{equation}
        U_{R^{\prime}R^{\prime\prime}\rightarrow R}|\psi^{\rho}\rangle_{R^{\prime}AB}|0\rangle_{R^{\prime\prime}}=\sum_{x\in\mathcal{X}}\sqrt{q(x)}|x\rangle_{R}|\varphi^{x}\rangle_{AB},
    \end{equation}
    we note that $U$ followed by a measurement in the standard basis induces a convex decomposition of $\rho_{AB}$ in terms of the ensemble $\{(q(x),\varphi_{AB}^{x})\}_{x}$. We can write the root fidelity as 
    \begin{align}
        &\max_{\sigma_{AB}\in\operatorname{SEP}(A:B)}\sqrt{F}(\rho_{AB},\sigma_{AB})\nonumber\\
        &=\max_{\substack{\left\{  \left(  p(x),\psi^{x}{}_{A},\phi_{B}^{x}\right)\right\}  _{x},\\\{(q(x),\varphi_{AB}^{x})\}_{x}}}\nonumber \\
        & \qquad \left\vert \sum_{x,x^{\prime}}\sqrt{q(x)p(x^{\prime})}\langle x|x^{\prime}\rangle_{R}\langle\varphi^{x}|_{AB}|\psi^{x^{\prime}}\rangle_{A}|\phi^{x^{\prime}}\rangle_{B}\right\vert \\
        &=\max_{\substack{\left\{  \left(  p(x),\psi^{x}{}_{A},\phi_{B}^{x}\right)\right\}  _{x},\\\{(q(x),\varphi_{AB}^{x})\}_{x}}}\left\vert \sum_{x}\sqrt{q(x)p(x)}\langle\varphi^{x}|_{AB}|\psi^{x}\rangle_{A}|\phi^{x}\rangle_{B}\right\vert \\
        &=\max_{\substack{\left\{  \left(  p(x),\psi^{x}{}_{A},\phi_{B}^{x}\right)\right\}  _{x},\\\{(q(x),\varphi_{AB}^{x})\}_{x}}}\left\vert \sum_{x}\sqrt{q(x)p(x)}\langle\varphi^{x}|_{AB}|\psi^{x}\rangle_{A}|\phi^{x}\rangle_{B}\right\vert .
    \end{align}
    Next, for fixed $\left\{  \left(  p(x),\psi^{x}{}_{A},\phi_{B}^{x}\right)\right\}  _{x}$ and $\{(q(x),\varphi_{AB}^{x})\}_{x}$, we bound the objective function in the optimization above as follows:%
    \begin{align}
        &\left\vert \sum_{x}\sqrt{q(x)p(x)}\langle\varphi^{x}|_{AB}|\psi^{x}\rangle_{A}|\phi^{x}\rangle_{B}\right\vert \nonumber\\
    &\leq\sum_{x}\sqrt{p(x)q(x)}\left\vert \langle\varphi^{x}|_{AB}|\psi
    ^{x}\rangle_{A}|\phi^{x}\rangle_{B}\right\vert \\
    &\leq\sqrt{\sum_{x}p(x)}\sqrt{\sum_{x}q(x)\left\vert \langle\varphi
    ^{x}|_{AB}|\psi^{x}\rangle_{A}|\phi^{x}\rangle_{B}\right\vert ^{2}}\\
    &=\sqrt{\sum_{x}q(x)\left\vert \langle\varphi^{x}|_{AB}|\psi^{x}\rangle
    _{A}|\phi^{x}\rangle_{B}\right\vert ^{2}}.
    \end{align}
    The first inequality follows from the triangle inequality and the second from an application of Cauchy--Schwarz.\ We see that equality is achieved in the second inequality by choosing%
    \begin{equation}
        p(x)=\frac{q(x)\left\vert \langle\varphi^{x}|_{AB}|\psi^{x}\rangle_{A}|\phi^{x}\rangle_{B}\right\vert ^{2}}{\sum_{x}q(x)\left\vert \langle\varphi^{x}|_{AB}|\psi^{x}\rangle_{A}|\phi^{x}\rangle_{B}\right\vert ^{2}}.
    \end{equation}
    We can achieve equality in the first inequality by tuning a global phase for the state $|\psi^{x}\rangle_{A}$, which amounts to a relative phase in \eqref{eq:sep-purify}. Putting everything together, we conclude that%
    \begin{align}
        &\max_{\sigma_{AB}\in\operatorname{SEP}}F(\rho_{AB},\sigma_{AB})=\notag\\
        &\max_{\{(q(x),\varphi_{AB}^{x})\}_{x}}\sum_{x}q(x)\max_{\left(  |\psi^{x}\rangle_{A}|\phi^{x}\rangle_{B}\right)  _{x},}\left\vert \langle\varphi^{x}|_{AB}|\psi^{x}\rangle_{A}|\phi^{x}\rangle_{B}\right\vert ^{2},
    \end{align}
    which is equivalent to the desired equality in \eqref{eq:convex-decomp-max-sep-fid}.
    \end{proof}



    
\section{Proof of Equation~\eqref{eq:max-sep-fid-inf-norm}}

\label{appendix:proof_max-sep-fid-inf-norm}
    
    In this appendix, we show that the fidelity of separability of a bipartite state can be written in terms of the spectral norm, which was also observed in \cite[Section~6.2]{Streltsov2010}.
    \begin{proposition}
        \label{prop:max-sep-fid-inf-norm}For a bipartite state, the following equality holds%
        \begin{multline}
            F_{s}(\rho_{AB})= \\ \max_{\left\{(p(x),\psi_{AB}^{x})\right\}  _{x}}\left\{\sum_{x}p(x)\left\Vert \psi_{A}^{x}\right\Vert _{\infty}:\rho_{AB}=\sum_{x}p(x)\psi_{AB}^{x}\right\}.
        \end{multline}
    \end{proposition}
    
    \begin{proof}
    Consider that the following holds for a pure bipartite state $\psi_{AB}$:
    \begin{align}
        F_{s}(\psi_{AB}) &  =\max_{|\phi\rangle_{A},|\varphi\rangle_{B}}\left\vert\langle\psi|_{AB}|\phi\rangle_{A}\otimes|\varphi\rangle_{B}\right\vert ^{2}\\
        &=\max_{|\phi\rangle_{A},|\varphi\rangle_{B}}\left\vert \langle\phi|_{A}\otimes\langle\varphi|_{B}|\psi\rangle_{AB}\right\vert ^{2}\\
        &=\max_{|\phi\rangle_{A}}\left\Vert \langle\phi|_{A}\otimes I_{B}|\psi\rangle_{AB}\right\Vert _{2}^{2}\\
        &=\max_{|\phi\rangle_{A}}\operatorname{Tr}[(|\phi\rangle\!\langle\phi|_{A}\otimes I_{B})\psi_{AB}]\\
        &=\max_{|\phi\rangle_{A}}\operatorname{Tr}[|\phi\rangle\!\langle\phi|_{A}\psi_{A}]\\
        &=\left\Vert \psi_{A}\right\Vert _{\infty}.
    \end{align}
    The first two equalities follow from the definition and a rewriting. The third equality follows from the variational characterization of the Euclidean norm of a vector. The fourth equality follows because%
    \begin{align}
        & \left\Vert \langle\phi|_{A}\otimes I_{B}|\psi\rangle_{AB}\right\Vert _{2}^{2} \notag \\
        & =\left(\langle\psi|_{AB}|\phi\rangle_{A}\otimes I_{B}\right)\left(\langle\phi|_{A}\otimes I_{B}|\psi\rangle_{AB}\right)  \\
        &=\langle\psi|_{AB}|\phi\rangle\!\langle\phi|_{A}\otimes I_{B}|\psi\rangle_{AB}\\
        & =\operatorname{Tr}[(|\phi\rangle\!\langle\phi|_{A}\otimes I_{B})\psi_{AB}].
    \end{align}
    The next follows from taking a partial trace, and the final equality from the variational characterization of the spectral norm. So this implies the desired equality, after applying \eqref{eq:convex-decomp-max-sep-fid}.
    \end{proof}
    

\section{Proof of Theorem~\ref{theorem:global_cost_func}}\label{appendix:step_by_step}

    
    In this appendix, we show that the acceptance probability of our VQSA is indeed equal to $F_s(\rho_{AB})$ if the parameterized unitary circuits can express all possible unitary operators of their respective systems. For this, let us track the state of the VQSA at the points indicated in Figure~\ref{fig:Max_Sep_Fidelity_costfun_proof}. 

    
    \begin{itemize}
        \item At Step $(1)$, the unitary $U^\rho$ prepares the pure state $\psi_{RAB}$. This is a specific initial purification of $\rho_{AB}$.
        
        \item At Step $(2)$, we apply the parameterized unitary circuit $W_R(\Theta)$ to $\psi_{RAB}$. Expanding $W_R(\Theta)|\psi^{\rho}\rangle_{RAB}$ in terms of the standard basis $\{|x\rangle\}_x$ leads to
            \begin{equation}
                W_R(\Theta)|\psi\rangle_{RAB}=\sum_{x\in\mathcal{X}}\sqrt{q(x)}|x\rangle_{R}|\varphi^{x}\rangle_{AB}.
            \end{equation}
            
        \item At Step $(3)$, the measurement outcome $x$ occurs with probability $q(x)$, and the state vector of registers $A$ and $B$ becomes  $|\varphi^{x}\rangle_{AB}$.
        
        \item At Step $(4)$, depending on the measurement outcome $x$, we apply the parameterized unitary circuit $U^x_A(\Theta^x)$ to register $A$. The state vector is now $U^x_A(\Theta^x)|\varphi^{x}\rangle_{AB}$.
        
        
        
        \item At Step $(5)$, we trace over $B$ and measure $A$ in the standard basis. We accept when we get the all-zeros outcome. The acceptance probability is then equal to
        \begin{multline}
            \sum_{x\in\mathcal{X}}q(x)\, \langle0|\,U^x_A(\Theta^x)\varphi^{x}_{A}\left(U^x_A\right)^\dagger|0\rangle\\
            =\sum_{x\in\mathcal{X}}q(x)\,\langle\phi^x|_A\varphi^{x}_{A}|\phi^x\rangle_A,
        \end{multline}
        where we have defined $|\phi^x\rangle_A \coloneqq \left(U^x_A\right)^\dagger|0\rangle$.
        
        \item Maximizing the acceptance probability corresponds to maximization over the parameters of $W_R(\Theta)$ and~$U^x_A(\Theta^x)$. 
        
        \item Maximization over the parameters of $W_R$ is a maximization over all possible pure-state decompositions of~$\rho_{AB}$. 
        
        \item Maximization over the parameters of $U^x_A(\Theta^x)$ is a maximization of $\langle\phi^x|\varphi^{x}_{A}|\phi^x\rangle$ which yields the value of~$\left\Vert \varphi_{A}^{x}\right\Vert _{\infty}$.
        
        \item The maximum acceptance probability is equal to
        \begin{multline}
             \max_{\left\{(p(x),\psi_{AB}^{x})\right\}  _{x}}\left\{\sum_{x}p(x)\left\Vert \varphi_{A}^{x}\right\Vert _{\infty}:\rho_{AB}=\sum_{x}p(x)\psi_{AB}^{x}\right\}
        \end{multline}
        which is equal to $F_s(\rho_{AB})$, by Proposition~\ref{prop:max-sep-fid-inf-norm}.
    \end{itemize} 
    
    This proves that, if the parameterized unitary circuits can express all possible unitary operators of their respective systems, then the maximum acceptance probability is equal to to $F_s(\rho_{AB})$. However, we  note that any ansatz employed for the parameterized unitary circuits has a limited expressibility. Hence, the maximum acceptance probability obtained via the VQSA will also be closer to the true value of $F_s(\rho_{AB})$ if we use a more expressive ansatz. 

    \begin{figure}
        \includegraphics[width=\columnwidth]{figures/fig_VQA_proof_new.pdf}
        \caption{VQSA to estimate the fidelity of separability $F_s(\rho_{AB})$. 
        The unitary circuit $U^\rho$ produces the state $\psi_{RAB}$, which is a purification of $\rho_{AB}$. The parameterized circuit $W_R(\Theta)$ acts on $R$ to evolve $\psi_{RAB}$ to another pure-state decomposition of $\rho_{AB}$. The measurement that follows steers the system $AB$ to be in a pure state $\psi_{AB}^{x}$, if the measurement outcome $x$ occurs. Conditioned on the outcome $x$, the final parameterized circuit $U^{x}_{A}(\Theta^x)$ and the subsequent measurement estimates~$\left\|\psi_{A}^{x}\right\|_{\infty}$.}
        \label{fig:Max_Sep_Fidelity_costfun_proof}
    \end{figure}


\section{Multipartite Scenarios}

\label{appendix:mulitpartite}

    In this appendix, we discuss the multipartite generalization of the tests of separability of mixed states.

    \begin{definition}
        A state $\rho_{A_1\cdots A_M}\in \mathcal{D}(\mathcal{H}_{A_1\cdots A_M})=\mathcal{D}(\mathcal{H}_{A_1}\otimes\cdots\otimes\mathcal{H}_{A_M})$ is  separable if 
        \begin{equation}
            \rho_{A_1\cdots A_M}=\sum_{x\in\chi}p(x)\psi^{x,1}_{A_1}\otimes\cdots\otimes\psi^{x,M}_{A_M}
        \end{equation}
        where $\psi_{A_i}^{x,i}$ is a pure state for every $x\in\mathcal{X}$ and $i\in\{1,\ldots,M\}$.
    \end{definition}
    Let $M\text{-SEP}$ denote the set of all $\rho_{A_1\cdots A_M}\in \mathcal{D}(\mathcal{H}_{A_1\cdots A_M})$ such that $\rho_{A_1\cdots A_M}$ is separable. The following theorem is important for the rest of this analysis.
    
    \begin{theorem}[\cite{Streltsov2010}]\label{theorem:Multi-Streltsov}
        The following formula holds 
        \begin{multline}\label{eq:convex-decomp-multi-max-sep-fid}
            \max_{\sigma_{A_1\cdots A_M}\in M-\operatorname{SEP}}F(\rho_{A_1\cdots A_M},\sigma_{A_1\cdots A_M})\\
            =\max_{\{(q(x),\varphi_{A_1\cdots A_M}^{x})\}_{x}}\sum_{x}q(x) F_s(\varphi^{x}_{A_1\cdots A_M}),
        \end{multline}
        where the optimization is over every pure-state decomposition $\{(q(x),\varphi_{A_1\cdots A_M}^{x})\}_{x}$   of $\rho_{A_1\cdots A_M}$ (similar to those in Theorem~\ref{theorem:Streltsov}) and 
        \begin{multline}\label{eqn:multi_fid_sep}
            F_{s}(\varphi_{A_1\cdots A_M}^{x}) = \\
            \max_{\left\{|\phi^{x,i}\rangle_{A_i}\right\}_{i=1}^{M}}\left\vert\langle\varphi^{x}|_{A_1\cdots A_M}|\phi^{x,1}\rangle_{A_1}\otimes\cdots\otimes |\phi^{x,M}\rangle_{A_M}\right\vert^2.
        \end{multline}
    \end{theorem}
    
    For the multipartite case of the distributed quantum computation, the verifier prepares a purification $\psi^{\rho}_{R^{\prime}A_1\cdots A_M}$\ of $\rho_{A_1\cdots A_M}$. The prover applies a multipartite entanglement-breaking channel on $R$ which can be written as: 
    \begin{equation}
    \label{eqn:ent-break-mult}
        \mathcal{E}_{R\rightarrow A^{\prime}_1\cdots A^{\prime}_{M-1}}(\cdot)=\sum_{x}\operatorname{Tr}[\mu^{x}_{R}(\cdot)]\left(\phi^{x,1}_{A^{\prime}_1}\otimes\cdots\otimes\phi^{x,M-1}_{A^{\prime}_{M-1}}\right),
    \end{equation}
     where $\{\mu^{x}_{R}\}_{x}$ is a rank-one POVM and $\{\phi^{x,i}_{A^{\prime}_i}\}_{x,i}$ is a set of pure states. The prover sends systems $(A^{M-1})^{\prime}\equiv  A^{\prime}_1\cdots A^{\prime}_{M-1}$ to the verifier. Now, the verifier performs a collective swap test of these systems with $A_1\cdots A_M$. The acceptance probability of this distributed quantum computation is given by%
    \begin{equation}
        \max_{\mathcal{E}\in\operatorname{EB}_{M-1}}\operatorname{Tr}[\Pi_{(A^{M-1})^{\prime}A^{M-1}}^{\operatorname{sym}}\mathcal{E}_{R\rightarrow (A^{M-1})^{\prime}}(\psi_{RA^{M-1}})],
    \end{equation}
    where $\mathcal{E}\in\operatorname{EB}_{M-1}$ denotes the set of entanglement-breaking channels defined in \eqref{eqn:ent-break-mult}.
    This leads to the following theorem:
    \begin{theorem}\label{theorem:qip-eb_msf_multi}
        For a pure state $\psi_{RA^{M}} \equiv \psi_{RA_1\cdots A_{M}}$, the following equality holds:%
        \begin{multline}
            \label{eq:mult-part-fid-sep-test-acc-prob}
            \max_{\mathcal{E}\in\operatorname{EB}_{M-1}} \operatorname{Tr}[ \Pi_{(A^{M-1})^{\prime}(A^{M-1})}^{\operatorname{sym}} \mathcal{E}_{R\rightarrow A^{\prime}_1\cdots A^{\prime}_{M-1}}(\psi_{RA^{M}})]
            \\
            =\frac{1}{2}\left(1 +\max_{\sigma_{A_1\cdots A_M}\in M-\operatorname{SEP}}F(\rho_{A_1\cdots A_M},\sigma_{A_1\cdots A_M})\right).
        \end{multline}
    \end{theorem}
    
    \begin{figure}
        \includegraphics[width=\columnwidth]{figures/fig_QIP_EB_new_multi2.pdf}
        \caption{Test for separability of multipartite mixed states. The verifier uses a unitary circuit $U^\rho$ to produce the state $\psi_{RA_1 A_2 A_3 A_4}$, which is a purification of $\rho_{A_1 A_2 A_3 A_4}$. The prover (indicated by the dotted box) then applies an entanglement-breaking channel $\mathcal{E}_{R\rightarrow A^{\prime}_1 A^{\prime}_2 A^{\prime}_3}$ on $R$ by measuring the rank-one POVM $\{\mu^{x}_{R}\}_{x}$ and then, depending on the outcome $x$, prepares a state from the set $\{\phi^{x,1}_{A^{\prime}_1}\otimes\phi^{x,2}_{A^{\prime}_2}\otimes\phi^{x,3}_{A^{\prime}_3}\}_{x}$. The final state is sent to the verifier, who performs a collective swap test. Theorem~\ref{theorem:qip-eb_msf_multi} states that the maximum acceptance probability of this interactive proof is equal to $\frac{1}{2}(1 + F_{s}(\rho_{A_1 A_2 A_3 A_4}))$, i.e., a simple function of the fidelity of separability.}
        \label{fig:Max_Sep_Fidelity_QIP_EB_Multi}
    \end{figure}
    
    \begin{proof}
        The circuit diagram is given in Figure~\ref{fig:Max_Sep_Fidelity_QIP_EB_Multi}. The verifier prepares a purification $\psi^{\rho}_{R^{\prime}A_1\cdots A_M}$\ of $\rho_{A_1\cdots A_M}$. The prover applies a multipartite entanglement-breaking channel on $R$ which can be written as: 
    \begin{equation}
        \mathcal{E}_{R\rightarrow A^{\prime}_1\cdots A^{\prime}_{M-1}}(\cdot)=\sum_{x}\operatorname{Tr}[\mu^{x}_{R}(\cdot)]\left(\phi^{x,1}_{A^{\prime}_1}\otimes\cdots\otimes\phi^{x,M-1}_{A^{\prime}_{M-1}}\right),
    \end{equation}
     where $\{\mu^{x}_{R}\}_{x}$ is a rank-one POVM and $\{\phi^{x,i}_{A^{\prime}_i}\}_{x,i}$ is a set of pure states. The prover sends the systems $(A^{M-1})^{\prime}= A^{\prime}_1\cdots A^{\prime}_{M-1}$ to the verifier. Now, the verifier performs a collective swap test on $A_1\cdots A_M$, as depicted at the final part of the circuit diagram in Figure~\ref{fig:Max_Sep_Fidelity_QIP_EB_Multi}. The acceptance probability of this interactive proof system is thus given by%
    \begin{equation}\label{eqn:qipeb-mult}
        \max_{\mathcal{E}\in\text{EB}}\operatorname{Tr}[\Pi_{(A_1\cdots A_{M-1})^{\prime}A_1\cdots A_{M-1}}^{\text{sym}}\mathcal{E}_{R\rightarrow (A_1\cdots A_{M-1})^{\prime}}(\psi_{RA_1\cdots A_{M}})],
    \end{equation}
    where%
    \begin{multline}
    \Pi^{\text{sym}}_{(A_1\cdots A_{M-1})^{\prime}A_1\cdots A_{M-1}}:=\\ \frac{I_{(A_1\cdots A_{M-1})^{\prime}A_1\cdots A_{M-1}}+F_{(A_1\cdots A_{M-1})^{\prime}A_1\cdots A_{M-1}}}{2}
    \end{multline}
    is the projector onto the symmetric subspace of $A^{\prime}$ and $A$ and $F_{(A_1\cdots A_{M-1})^{\prime}A_1\cdots A_{M-1}}$ is a tensor product of individual swaps $F_{A'_i A_i}$, for $i\in\{1,\ldots,M\}$. Then we find, for fixed $\mathcal{E}_{R\rightarrow A^{\prime}_1\cdots A^{\prime}_{M-1}}$, that
    \begin{align}
        &\operatorname{Tr}[\Pi_{(A_1\cdots A_{M-1})^{\prime}(A_1\cdots A_{M-1})}^{\text{sym}}\mathcal{E}_{R\rightarrow A^{\prime}_1\cdots A^{\prime}_{M-1}}(\psi_{RA_1\cdots A_{M}})]\notag \\
        &=\frac{1}{2}\operatorname{Tr}[(I_{(A_1\cdots A_{M-1})^{\prime}(A_1\cdots A_{M-1})}+F_{(A_1\cdots A_{M-1})^{\prime}(A_1\cdots A_{M-1})})\notag\\
        &\qquad\qquad\qquad\mathcal{E}_{R\rightarrow A^{\prime}_1\cdots A^{\prime}_{M-1}}(\psi_{RA_1\cdots A_{M}})]\\
        &=\frac{1}{2}+\frac{1}{2}\operatorname{Tr}[(F_{(A_1\cdots A_{M-1})^{\prime}(A_1\cdots A_{M-1})})\mathcal{E}_{R\rightarrow A^{\prime}_1\cdots A^{\prime}_{M-1}}(\psi_{RA^{M}})]\\
        &=\frac{1}{2}+\frac{1}{2}\operatorname{Tr}[(F_{(A_1\cdots A_{M-1})^{\prime}(A_1\cdots A_{M-1})})\sum_{x}\operatorname{Tr}[\mu^{x}_{R}(\psi_{RA^{M}})]\notag\\ 
        &\qquad\qquad\qquad\phi^{x,1}_{A^{\prime}_1}\otimes\cdots\otimes\phi^{x,M-1}_{A^{\prime}_{M-1}}],\\
        &=\frac{1}{2}+\frac{1}{2}\operatorname{Tr}[(F_{(A_1\cdots A_{M-1})^{\prime}(A_1\cdots A_{M-1})})\sum_{x}p(x)(\psi^{x}_{A_1\cdots A_{M}})\notag\\ 
        &\qquad\qquad\qquad\phi^{x,1}_{A^{\prime}_1}\otimes\cdots\otimes\phi^{x,M-1}_{A^{\prime}_{M-1}}],\\
        &=\frac{1}{2}+\frac{1}{2}\sum_{x}p(x)\operatorname{Tr}\left[\left(\phi^{x,1}_{A_1}\otimes\cdots\otimes \phi^{x,M-1}_{A_{M-1}}\right)\psi^{x}_{A_1\cdots A_{M-1}}\right]\label{eqn:mult-fid-sep-intermediate}
    \end{align}
    where
    \begin{align}
        p(x) &  \coloneqq \operatorname{Tr}[\mu^{x}_{R}(\psi_{RA^{M}})],\\
        \psi^{x}_{A_1\cdots A_{M}} &  \coloneqq \frac{1}{p(x)}\operatorname{Tr}_R[\mu^{x}_{R}(\psi_{RA^{M}})].
    \end{align}
    For a given $x$, let us try to simplify $F_s(\varphi_{A_1\cdots A_M})$ as defined in~\eqref{eqn:multi_fid_sep}, 
   \begin{multline}
        F_s(\varphi_{A_1\cdots A_M})=\\\max_{\left\{|\phi^{i}\rangle_{A_i}\right\}_{i=1}^M}\left\vert\langle\varphi|_{A_1\cdots A_M}|\phi^{1}\rangle_{A_1}\otimes\cdots\otimes |\phi^{M}\rangle_{A_M}\right\vert^2
        \end{multline}
    \begin{align}  
        &=\max_{\left\{|\phi^{i}\rangle_{A_i}\right\}_{i=1}^M}\left\vert\langle\phi^{1}|_{A_1}\otimes\cdots\otimes \langle\phi^{M}|_{A_M}|\varphi\rangle_{A_1\cdots A_M}\right\vert^2\\
        &=\max_{\left\{|\phi^{i}\rangle_{A_i}\right\}_{i=1}^{M-1}}\left\Vert \langle\phi^{1}|_{A_1}\otimes\cdots\otimes\langle\phi^{M-1}|_{A_{M-1}}\otimes I_{A_M}|\varphi\rangle_{A^M}\right\Vert _{2}^{2}\\
        &=\max_{\left\{|\phi^{i}\rangle_{A_i}\right\}_{i=1}^{M-1}}\operatorname{Tr}[(|\phi^{1}\rangle\langle\phi^{1}|_{A_1}\otimes\cdots\notag\\
        &\qquad\qquad\qquad\otimes|\phi^{M-1}\rangle\langle\phi^{M-1}|_{A_{M-1}}\otimes I_{A_M})\varphi_{A_1\cdots A_M}]\\
        &=\max_{\left\{|\phi^{i}\rangle_{A_i}\right\}_{i=1}^{M-1}}\operatorname{Tr}[(|\phi^{1}\rangle\langle\phi^{1}|_{A_1}\otimes\cdots\notag\\
        &\qquad\qquad\qquad\qquad\otimes|\phi^{M-1}\rangle\langle\phi^{M-1}|_{A_{M-1}})\varphi_{A_1\cdots A_{M-1}}]\label{eqn:mult-fid-sep-pure}
    \end{align}
    The first two equalities are from the definition and a rewriting. The third equality follows from the variational characterization of the Euclidean norm of a vector. Noting the form in  \eqref{eqn:mult-fid-sep-pure} and applying the maximization over entanglement breaking channels of the form described in \eqref{eqn:ent-break-mult} to \eqref{eqn:mult-fid-sep-intermediate}, we arrive at the desired claim in \eqref{eq:mult-part-fid-sep-test-acc-prob}.
    \end{proof}
    


    \begin{figure}
        \includegraphics[width=\columnwidth]{figures/fig_VQA_new_multi3.pdf}
        \caption{VQSA to estimate the multipartite fidelity of separability $F_s(\rho_{A_1A_2A_3A_4})$. 
        The unitary circuit $U^\rho$ prepares the state $\psi_{RA_1A_2A_3A_4}$, which is a purification of $\rho_{A_1A_2A_3A_4}$. The parameterized circuit $W_R(\Theta)$ acts on $R$ to evolve the state to another purification of $\rho_{A_1A_2A_3A_4}$. The measurement that follows, labeled ``steering measurement,'' steers the remaining systems to be in a state $\psi_{A_1A_2A_3A_4}^{x}$, if the measurement outcome $x$ occurs. Conditioned on the outcome $x$, the final parameterized circuits $U^{x,1}_{A_1}(\Theta^x_1)$, $U^{x,2}_{A_2}(\Theta^x_2)$, and $U^{x,3}_{A_3}(\Theta^x_3)$ are applied and the subsequent measurement estimates~$\operatorname{Tr}\left[\left(\phi^{x,1}_{A_1}\otimes\phi^{x,2}_{A_2}\otimes \phi^{x,3}_{A_{3}}\right)\psi^{x}_{A_1 A_2 A_{3}}\right]$.} 
        \label{fig:Max_Sep_Fidelity_VQSA_Multi}
    \end{figure}
    
    \bigskip 
    We can then use the generalized test of separability of mixed states to develop a VQSA for the multipartite case. See Figure~\ref{fig:Max_Sep_Fidelity_VQSA_Multi}. This involves replacing the collective swap test in Figure~\ref{fig:Max_Sep_Fidelity_QIP_EB_Multi} with an overlap measurement, similar to how we got Figure~\ref{fig:Max_Sep_Fidelity_VQA} from Figure~\ref{fig:Max_Sep_Fidelity_QIP_EB}.


    \section{Complexity Class \texorpdfstring{\qipeb}{Lg}} \label{appendix:qipeb}

    In this appendix, we establish a complete problem for \qipeb, and then we interpret this problem in Remark~\ref{rem:interp-QIP-EB2}.
    
    \begin{theorem}
        Given a circuit to generate a channel $\mathcal{N}_{G\rightarrow S}$ and a state $\rho_{S}$, estimating the following quantity is a complete problem for $\operatorname{QIP}_{\operatorname{EB}}(2)$:%
        \begin{multline}
            \max_{\substack{\{  (p(x),\psi^{x})\}  _{x},\\
            \left\{  \varphi^{x}\right\}_{x},\\
            \sum_{x}p(x)\psi_{S}^{x}=\rho_{S}
            }}   \sum_{x}p(x)F(\psi_{S}^{x},\mathcal{N}_{G\rightarrow S}(\varphi_{G}^{x}))  ,
            \label{eq:accept-prob-qip_eb-2}
        \end{multline}
        where the optimization is over every pure-state decomposition of $\rho_{S}$, as $\sum_{x}p(x)\psi_{S}^{x}=\rho_{S}$, and $\left\{  \varphi^{x}\right\}_{x}$ is a set of pure states.
    \end{theorem}

    \begin{proof}
    Consider an interactive proof system in \qipeb\ that begins with the verifier preparing a bipartite pure state $\psi_{RS}$, followed by the system $R$ being sent to the prover, who subsequently performs an entanglement-breaking channel. The verifier then performs a unitary $V_{R^{\prime}S\rightarrow DG}$ and projects onto the $|1\rangle\!\langle 1|$ state of the decision qubit. Indeed, the acceptance probability is given by%
    \begin{equation}
        \max_{\mathcal{E}\in\operatorname{EB}}\operatorname{Tr}[(|1\rangle\!\langle1|_{D}\otimes I_{G})\mathcal{V}_{R^{\prime}S\rightarrow DG}(\mathcal{E}_{R\rightarrow R^{\prime}}(\psi_{RS}))],
    \end{equation}
    where $\mathcal{V}_{R^{\prime}S\rightarrow DG}$ is the unitary channel corresponding to the unitary operator $V_{R^{\prime}S\rightarrow DG}$.
    By previous reasoning, we find that%
    \begin{equation}
        \mathcal{E}_{R\rightarrow R^{\prime}}(\psi_{RS})=\sum_{x}p(x)\phi_{R^{\prime}}^{x}\otimes\psi_{S}^{x},
    \end{equation}
    so that the acceptance probability is equal to
    \begin{multline}
          \max_{\substack{\{  (p(x),\psi^{x})\}  _{x},\\
          \left\{  \phi^{x}\right\}_{x},\\
          \sum_{x}p(x)\psi_{S}^{x}=\psi_{S}}}
            \operatorname{Tr}\left[  (|1\rangle\!\langle1|_{D}\otimes I_{G})\mathcal{V}\left(  \sum_{x}p(x)\phi_{R^{\prime}}^{x}\otimes\psi_{S}^{x}\right)  \right]    \\
          =\max_{\substack{\{  (p(x),\psi^{x})\}  _{x},\\
          \left\{  \phi^{x}\right\}_{x},\\
          \sum_{x}p(x)\psi_{S}^{x}=\psi_{S}}}
            \sum_{x}p(x)\operatorname{Tr}\left[  (|1\rangle\!\langle 1|_{D}\otimes I_{G})\mathcal{V}
            \left(\phi_{R^{\prime}}^{x}\otimes\psi_{S}^{x}\right)  \right] ,
    \end{multline}
    where we have used the shorthand $\mathcal{V} \equiv \mathcal{V}_{R^{\prime}S\rightarrow DG}$.
    Consider that%
    \begin{align}
        &  \operatorname{Tr}\left[  (|1\rangle\!\langle1|_{D}\otimes I_{G})\mathcal{V} \left(  \phi_{R^{\prime}}^{x}\otimes\psi_{S}^{x}\right)  \right] \nonumber   \\
        &  =\left\Vert \langle1|_{D}\otimes I_{G}) V|\phi^{x}\rangle_{R^{\prime}}\otimes|\psi^{x}\rangle_{S}\right\Vert _{2}^{2}\\
        &  =\max_{|\varphi^{x}\rangle_{G}}\left\vert \langle1|_{D}\otimes\langle\varphi^{x}|_{G}) V |\phi^{x}\rangle_{R^{\prime}}\otimes|\psi^{x}\rangle_{S}\right\vert ^{2}\\
        &  =\max_{|\varphi^{x}\rangle_{G}}\operatorname{Tr}\left[  V^{\dag}(|1\rangle\!\langle1|_{D}\otimes|\varphi^{x}\rangle\!\langle\varphi^{x}|_{G})\right.\times \notag \\
        &\left. \qquad \qquad \qquad V\phi^{x}_{R^{\prime}}\otimes|\psi^{x}\rangle\!\langle\psi^{x}|_{S}\right]  \\
        &  =\max_{|\varphi^{x}\rangle_{G}}\operatorname{Tr}\left[  \mathcal{W}_{G\rightarrow R^{\prime}S}(|\varphi^{x}\rangle\!\langle\varphi^{x}|_{G})\phi^{x}_{R^{\prime}}\otimes|\psi^{x}\rangle\!\langle\psi^{x}|_{S}\right],
    \end{align}
    where the isometric channel $\mathcal{W}_{G \to R'S}$ is defined as 
    \begin{equation}
        \mathcal{W}_{G\rightarrow R^{\prime}S}(\cdot)\coloneqq (V_{R^{\prime}S\rightarrow DG})^{\dag}(|1\rangle\!\langle1|_{D}\otimes(\cdot)_{G})V_{R^{\prime}S\rightarrow DG}.
    \end{equation}
    Then the acceptance probability is given by%
    \begin{equation}
    \max_{\substack{\{  (p(x),\psi^{x})\}  _{x},\\\left\{\phi^{x}\right\}  _{x},\left\{  \varphi^{x}\right\}  _{x}}}
    \left\{\begin{array}
            [c]{c}%
            \sum_{x}p(x)\operatorname{Tr}\left[\mathcal{W}_{G\rightarrow R^{\prime}S}(|\varphi^{x}\rangle\!\langle\varphi^{x}|_{G})\times \right.\\
            \left.\phi^{x}_{R^{\prime}}\otimes|\psi^{x}\rangle\!\langle\psi^{x}|_{S}\right]  :\\
            \sum_{x}p(x)\psi_{S}^{x}=\psi_{S}%
        \end{array}\right\}  .
    \end{equation}
    Since the optimization over $\phi^{x}_{R^{\prime}}$ is arbitrary, we can also write%
    \begin{align}
        &  \max_{|\phi^{x}\rangle_{R^{\prime}}}\operatorname{Tr}\left[  \mathcal{W}_{G\rightarrow R^{\prime}S}(|\varphi^{x}\rangle\!\langle\varphi^{x}|_{G})\phi^{x}_{R^{\prime}}\otimes|\psi^{x}\rangle\!\langle\psi^{x}|_{S}\right]  \notag \\
        &  =\max_{|\phi^{x}\rangle_{R^{\prime}}}\left\vert \langle\phi^{x}|_{R^{\prime}}\otimes\langle\psi^{x}|_{S}W_{G\rightarrow R^{\prime}S}|\varphi^{x}\rangle_{G}\right\vert ^{2} \notag\\
        &  =\left\Vert I_{R^{\prime}}\otimes\langle\psi^{x}|_{S}W_{G\rightarrow R^{\prime}S}|\varphi^{x}\rangle_{G}\right\Vert _{2}^{2} \notag\\
        &  =\left(  \langle\varphi^{x}|_{G}\left(  W_{G\rightarrow R^{\prime}S}\right)  ^{\dag}I_{R^{\prime}}\otimes|\psi^{x}\rangle_{S}\right)  \times \notag \\
        &\qquad  \left(I_{R^{\prime}}\otimes\langle\psi^{x}|_{S} W_{G\rightarrow R^{\prime}S}|\varphi^{x}\rangle_{G}\right)  \notag\\
        &  =\langle\varphi^{x}|_{G}\left(W_{G\rightarrow R^{\prime}S}\right)^{\dag}\left(  I_{R^{\prime}}\otimes|\psi^{x}\rangle\!\langle\psi^{x}|_{S}\right)  W_{G\rightarrow R^{\prime}S}|\varphi^{x}\rangle_{G}\notag\\
        &  =\operatorname{Tr}\left[  \left(  I_{R^{\prime}}\otimes|\psi^{x}\rangle\!\langle\psi^{x}|_{S}\right)  W_{G\rightarrow R^{\prime}S}|\varphi^{x}\rangle\!\langle\varphi^{x}|_{G}\left(  W_{G\rightarrow R^{\prime}S}\right)^{\dag}\right]  \notag\\
        &  =\operatorname{Tr}[|\psi^{x}\rangle\!\langle\psi^{x}|_{S}\mathcal{N}_{G\rightarrow S}(|\varphi^{x}\rangle\!\langle\varphi^{x}|_{G})],
    \end{align}
    where we define the channel $\mathcal{N}_{G\rightarrow S}$ as
    \begin{equation}
        \mathcal{N}_{G\rightarrow S}(\cdot)\coloneqq \operatorname{Tr}_{R^{\prime}}[(V_{R^{\prime}S\rightarrow DG})^{\dag}(|1\rangle\!\langle1|_{D}\otimes(\cdot)_{G})V_{R^{\prime}S\rightarrow DG}].
    \end{equation}
    Then we find that the acceptance probability is given by
    \begin{multline}
          \max_{\substack{\{  (p(x),\psi^{x})\}  _{x},\\
          \left\{  \varphi^{x}\right\}_{x},\\
          \sum_{x}p(x)\psi_{S}^{x}=\psi_{S}}}  \sum_{x}p(x)\operatorname{Tr}[|\psi^{x}\rangle\!\langle\psi^{x}|_{S}\mathcal{N}_{G\rightarrow S}(|\varphi^{x}\rangle\!\langle\varphi^{x}|_{G})]  \\
         =\max_{\substack{\{  (p(x),\psi^{x})\}  _{x},\left\{  \varphi^{x}\right\}_{x}\\
         \sum_{x}p(x)\psi_{S}^{x}=\psi_{S}
         }}  \sum_{x}p(x)F(\psi_{S}^{x},\mathcal{N}_{G\rightarrow S}(\varphi_{G}^{x}))  .
    \end{multline}
    This concludes the proof.
    \end{proof}

    \begin{remark}
    \label{rem:interp-QIP-EB2}
    The quantity in \eqref{eq:accept-prob-qip_eb-2} can be interpreted as follows: Given a channel $\mathcal{N}$ and a source $\rho$, calculate the largest average ensemble fidelity attainable in reproducing the source at the output of the channel. This means it is necessary to find the ensemble decomposition $\{  (p(x),\psi^{x})\}  _{x}$ of $\rho$ as well as a set $\left\{  \varphi^{x}\right\}_{x}$ of encoding states 
    that lead to the largest ensemble fidelity (and this is what is left to the prover). This criterion is similar to one that is used in Schumacher data compression \cite{PhysRevA.51.2738}, but this seems more similar to the setting of the source-channel separation theorem \cite{6287590}, in which the goal is to transmit an information source over a quantum channel. The channel $\mathcal{N}$ here could consist of a fixed encoding $\mathcal{E}$, noisy channel $\mathcal{M}$, and fixed decoding $\mathcal{D}$, (i.e., $\mathcal{N} = \mathcal{D} \circ \mathcal{M} \circ \mathcal{E}$) and then the goal is to test how well a given fixed scheme $(\mathcal{E},\mathcal{D})$ can communicate a source $\rho$ over a channel $\mathcal{M}$, according to the ensemble fidelity criterion.
    \end{remark}

    \section{Placement of \texorpdfstring{\qipeb}{Lg}}
    
    \label{appendix:complexity_placements}

    In this appendix, we show the containment of \qipeb\ as follows:
    \begin{equation}
    \text{QAM} \subseteq \text{\qipeb} \subseteq \text{QIP}.    
    \end{equation}
    See Figure~\ref{fig:placement} for a detailed diagram.
    
    \begin{figure}
        \includegraphics[width=\columnwidth]{figures/placement_diagram.pdf}
        \caption{Placement of \qipeb\ relative to other known complexity classes. The complexity classes are organized such that if a class is connected to a class above it, the complexity class placed lower is a subset of the class above. For example, \qipeb\ is subset of QIP and \qipeb\ is a superset of both QMA and QAM.}
        \label{fig:placement}
    \end{figure}

    \subsection{QAM \texorpdfstring{$\subseteq$}{Lg} \texorpdfstring{\qipeb}{Lg}}

    First, recall that QAM consists of the verifier selecting a classical letter $x$ uniformly at random, sending the choice to the prover, who then sends back a pure state $\psi_x$ to the verifier, who finally performs an efficient measurement to decide whether to accept the computation \cite{marriott2004quantum}. Note that QAM contains QMA~\cite{marriott2004quantum}.
    
    To see the containment QAM $\subseteq$ \qipeb, consider that the verifier's first circuit in \qipeb\ can consist of preparing a random classical bitstring in a system $R$. The verifier sends system $R$ to the prover. Then the prover's action just amounts to preparing some state that gets sent back to the verifier. The rest of the protocol then simulates a QAM protocol.

    

    \subsection{\texorpdfstring{\qipeb}{Lg} \texorpdfstring{$\subseteq$}{Lg} QIP}

    \label{app:qipeb-2-in-qip}
    
    To see the containment \qipeb\ $\subseteq$ QIP, we need to simulate \qipeb in QIP. If we can simulate \qipeb\ in QIP(4), then we know from the general result of \cite{kitaev2000parallelization} that QIP(4) = QIP(3) = QIP. 
    
    The first step begins with the verifier acting with the first circuit from the original \qipeb\ protocol. Then the verifier sends some of the output qubits to the QIP prover.

    Now, we need to simulate a general entanglement-breaking channel from \qipeb\ with, say, $p(n)$ input qubits and $q(n)$ output qubits, where $p(n)$ and $q(n)$ are polynomials in the complexity parameter $n$. What we need is a handle on the size of the classical register in between. In general, this could be arbitrarily large. However, the Choi state of this channel is separable, and the bound $\left\vert\mathcal{X}\right\vert\leq\text{rank}(\sigma_{AB})^{2}$ \cite{watrous_2018} stated in the main text implies that the number of terms in the decomposition need not be any larger than $2^{2 p(n) q(n)}$. This means that the classical register for the channel need not be any larger than $2 p(n)  q(n)$ qubits.

    So the rest of the QIP simulation consists of the prover applying a general unitary, sending back $2 p(n) q(n)$ qubits, the verifier decohering / measuring these qubits by attaching to each an ancilla in the all-zeros state $|0\rangle\!\langle 0|$, applying CNOTs, and keeping the second qubits in each pair locally. Then the verifier sends the decohered qubits to the prover, who acts with a general unitary, and sends back some of the outputs to the verifier. The verifier then acts with the final unitary from the original \qipeb\ protocol and measures the decision qubit.

    This concludes the proof, as all entanglement-breaking channels can be simulated in this way, and the number of classical bits needed in the middle, by the above argument, need not be larger than twice  the product of the number of input and output qubits, which is still polynomial in the input length. 

    \subsection{Error reduction for \texorpdfstring{\qipeb}{Lg} via parallel repetition}

    \label{app:par-rep}

    An important question that arises when defining a complexity class is the issue of repetition, since the verifier cannot make a decision from just one trial, but rather needs to accumulate evidence before doing so \cite{watrous2009complexity,VW15}. For complexity classes like BPP (efficient probabilistic classical computing) or BQP (efficient quantum computing), this issue is a simple matter of repeating the algorithm a sufficient number of times and taking a majority vote of the outcomes. By an application of the Chernoff bound, it follows that a polynomial number of repetitions implies an exponential decrease in the error probability in making the correct decision.

    This issue becomes more subtle when interacting with a quantum prover, because the prover has the ability to correlate his actions across multiple trials \cite{watrous2009complexity,VW15}. Thus, when defining the complexity class \qipeb\ and performing a \qipeb\ algorithm multiple times in parallel, this issue arises because the prover can act with a joint entanglement-breaking channel across all of the systems received from the verifier and then return a state that is entangled across all systems for the final measurements of the verifier. Thus, it is necessary to analyze a method for reducing errors in this case.

    Fortunately, for the case of analyzing this problem with \qipeb, we can make direct use of a prior analysis \cite[Section~3.2]{JUW09} for the parallel repetition and error reduction for the class QIP(2), essentially verbatim. We note that other approaches for error reduction with QIP(2) are available~\cite{molina2012parallel,hornby2018concentration}, but we do not make use of them here.

    To summarize the idea from \cite[Section~3.2]{JUW09}, starting from a fixed \qipeb\ protocol, which consists of an initial verifier circuit $V_1$, a final verifier circuit $V_2$ with a measurement on a single decision qubit, and predetermined bounds $a$ and $b$ on the completeness and soundness probabilities, we define  a new \qipeb\ parallelized protocol, which consists of an initial verifier circuit given by $V_1^{\otimes st}$ and a final verifier circuit given by $V_2^{\otimes st}$, where $s$ and $t$ are positive integers that are no more than polynomial in the problem size. The verifier then receives $st$ measurement outcomes (bits) after measuring all of the decision qubits of $V_2^{\otimes st}$ and has to decide whether to accept or reject based on them. The verifier groups the measurement outcomes into $s$ groups of $t$ measurement outcomes each. For the $i$th group, let $z_i$ be equal to one if the number of times the protocol accepted in the $i$th group, divided by $t$, is greater than a threshold equal to the average $(a+b)/2$, and let it be equal to zero otherwise. The parallelized verifier then accepts if and only if every $z_i$ is equal to one.
    
    By the same analysis given in \cite[Section~3.2]{JUW09}, if we are in a yes instance in which the verifier should accept, then the prover can act with a tensor product $\mathcal{E}^{\otimes n}$ of the entanglement-breaking channel $\mathcal{E}$ that leads to the highest acceptance probability in an individual trial. By doing so, it then follows from the analysis in \cite[Section~3.2]{JUW09} that the acceptance probability of the parallelized protocol in this case is exponentially close to one.

    Now consider the case that we are in a no instance in which the verifier should not accept. By the containment \qipeb~$\subseteq$~QIP discussed in Section~\ref{app:qipeb-2-in-qip}, it follows that there is a QIP simulation of the \qipeb\ protocol, with no change in the completeness and soundness probabilities. Thus, following the same reasoning from \cite[Section~3.2]{JUW09}, the $t$ parallel repetitions of the protocol in the $i$th group (call this the $i$th protocol) can be precisely simulated by a QIP, and the general result from \cite{kitaev2000parallelization} on parallel repetition of QIP applies here. Thus, the acceptance probability of the overall parallelized protocol is no greater than the product of the acceptance probabilities for the individual protocols, and the parameters $s$ and $t$ can be chosen in such a way that the acceptance probability of the overall protocol decays exponentially fast to zero.

    Thus, for the parallelized protocol with the classical postprocessing discussed above and with $st$ no more than a polynomial in the problem size, the acceptance probability increases to one exponentially fast in the case of a yes instance, and it decays to zero exponentially fast in the case of a no instance, as desired.





\section{First Benchmarking SDP \texorpdfstring{$\widetilde{F}_s^1$}{Lg}}

\label{appendix:ppt-k-state-sdp}

     In this appendix, we detail the derivation of our first benchmarking SDP $\widetilde{F}_s^1$, based on the SDP for fidelity \cite{Wat13}. Let $\rho_{AB}$ and $\sigma_{AB}$ be bipartite states. The SDP for the root fidelity $\sqrt{F}(\rho_{AB},\sigma_{AB})$, which makes use of Uhlmann's theorem~\cite{Uhlmann1976}, is as follows:
    \begin{equation}
    \label{eqn:fidelity_sdps}
        \sqrt{F}(\rho_{AB},\sigma_{AB}) = \max_{\substack{X_{AB} \in\mathcal{L}(\mathcal{H}_{AB})}}
        \left\{\begin{array}
                [c]{c}%
                \operatorname{Re}[\operatorname{Tr}[X_{AB}]]:\\%
                \begin{bmatrix}
                \rho_{AB} & X_{AB}\\
                X_{AB}^{\dag} & \sigma_{AB}%
                \end{bmatrix}
                \geq0 
        \end{array}\right\},
    \end{equation}
    where $\mathcal{L}(\mathcal{H}_{AB})$ is the set of all linear operators acting on the Hilbert space $\mathcal{H}_{AB}$.

    Note that there is no semidefinite constraint that directly corresponds to optimizing over the set of separable states \cite{Fawzi2021}. Instead, we can constrain $\sigma_{AB}$ to have a positive partial transpose (PPT) \cite{Peres1996,Horodecki1996} and be $k$-extendible \cite{W89a,Doherty2004}, since all separable states satisfy these constraints. Let $\widetilde{F}_s^1(\rho_{AB})$ denote the resulting quantity, the square root of which is defined as follows:
    \begin{multline}
        \sqrt{\widetilde{F}_s^1}(\rho_{AB}) \coloneqq \\ 
        \max_{\substack{X_{AB} \in\mathcal{L}(\mathcal{H}_{AB}),\\\sigma_{AB^{k}}\geq0}}
        \left\{\begin{array}
                [c]{c}
                \operatorname{Re}[\operatorname{Tr}[X_{AB}]]:\\%
                \begin{bmatrix}
                \rho_{AB} & X_{AB}\\
                X_{AB}^{\dag} & \sigma_{AB_{1}}%
                \end{bmatrix}
                \geq0,\\
                \operatorname{Tr}[\sigma_{AB^{k}}]=1,\\
                \sigma_{AB^{k}}=\mathcal{P}_{B^{k}}(\sigma_{AB^{k}}),\\ 
                T_{B_{1\cdots j}}(\sigma_{AB_{1\cdots j}})\geq 0 \quad \forall j\leq k
        \end{array}\right\}
    \end{multline}
    where $T_R$ denotes the partial transpose map acting on system $R$ and $\mathcal{P}_{B^{k}}$ denotes the channel that performs a uniformly random permutation of systems $B_1$ through $B_k$. Due to the containment discussed above, we note that
    \begin{equation}
        F_s(\rho_{AB}) \leq \widetilde{F}_s^1(\rho_{AB}).
    \end{equation}
    An upper bound on $\widetilde{F}_s^1(\rho_{AB})$ in terms of $F_s(\rho_{AB})$ can be obtained by applying \cite[Theorem~II.7']{christandl2007one} and the Fuchs-van-de-Graaf inequalities.

\section{Second benchmarking SDP \texorpdfstring{$\widetilde{F}_s^2$}{Lg} and proof of Equation~\eqref{eqn:swap-test-ppt-k-channel-sdp}}

\label{appendix:proof_swap-test-ppt-k-channel-sdp}

    In this appendix, we detail the derivation of our second benchmark SDP $\widetilde{F}_s^2$, which is an SDP that approximates \eqref{eqn:qip-eb_accept_prob}. Consider a version of the distributed quantum computation that led to \eqref{eqn:qip-eb_accept_prob} where, instead of restricting the prover to only entanglement-breaking channels, we just insist that the prover sends back $k$ systems labeled as $A_{1}\cdots A_{k}$. Then the verifier randomly selects one of the $k$ systems and performs a swap test on the $A$ system of the state $\psi_{RAB}$. This random selection is conducted so that the prover output is effectively reduced to that of an approximate entanglement-breaking channel. Note that the resulting interactive proof is in QIP(2). More specifically, the acceptance probability of this interactive proof system is given by%
    \begin{equation}\label{eqn:part_prep_channel}
       \max_{\mathcal{P}_{R\rightarrow A_{1}^{\prime}\cdots A_{k}^{\prime}}}\operatorname{Tr}[\Pi_{A^{\prime}A}^{\operatorname{sym}}\overline{\mathcal{P}}_{R\rightarrow A^{\prime}}(\psi_{RAB})],
    \end{equation}
    where
    \begin{equation}\label{eqn:whole_prep_channel}
        \overline{\mathcal{P}}_{R\rightarrow A^{\prime}}\coloneqq \frac{1}{k}\sum_{i=1}^{k}\operatorname{Tr}_{A_{1}^{k\prime}\backslash A_{i}}\circ\mathcal{P}_{R\rightarrow A_{1}^{\prime}\cdots A_{k}^{\prime}},
    \end{equation}
    and $\mathcal{P}$ is a preparation channel. Observing that $\overline{\mathcal{P}}_{R\rightarrow A^{\prime}}$ is a $k$-extendible channel \cite{PBHS13,Kaur2018,KDWW21,BBFS18}, it follows that%
    \begin{multline}\label{eqn:part_prep_chan_k-ext}
        \max_{\mathcal{P}_{R\rightarrow A_{1}^{\prime}\cdots A_{k}^{\prime}}}\operatorname{Tr}[\Pi_{A^{\prime}A}^{\operatorname{sym}}\overline{\mathcal{P}}_{R\rightarrow A^{\prime}}(\psi_{RAB})]\\=\max_{\mathcal{E}_{R\rightarrow A^{\prime}}^{k}\in\text{EXT}_{k}}\operatorname{Tr}[\Pi_{A^{\prime}A}^{\operatorname{sym}}\mathcal{E}_{R\rightarrow A^{\prime}}^{k}(\psi_{RAB})]
        \\
        \eqqcolon \frac{1}{2} (1+\widetilde{F}_s^2(\rho_{AB})),
    \end{multline}
    where EXT$_{k}$ denotes the set of $k$-extendible channels. These are defined by $\mathcal{E}_{R\rightarrow A^{\prime}}^{k}(\rho_{SR})\in$ EXT$_{k}(S\!:\!A^{\prime})$ for every input state $\rho_{SR}$, where EXT$_{k}(S\!:\!A^{\prime})$ denotes the set of $k$-extendible states. Hence, we estimate \eqref{eqn:qip-eb_accept_prob} using $\widetilde{F}_s^2(\rho_{AB})$ and it is given by the following SDP: 
    \begin{multline}
    \frac{1}{2}(1+\widetilde{F}_s^2(\rho_{AB})) =
    \\
    \max_{\Gamma^{\mathcal{E}^{k}}_{RA^{\prime k}}\geq 0}
    \left\{\begin{array}
            [c]{c}%
            \operatorname{Tr}[\Pi_{A^{\prime}A}^{\operatorname{sym}} \operatorname{Tr}_{R}[T_R(\psi_{RAB})\Gamma^{\mathcal{E}^{k}}_{RA^{\prime}_1}]]:\\%
            \operatorname{Tr}_{A^{\prime k}}[\Gamma^{\mathcal{E}^{k}}_{RA^{\prime k}}]=I_R,\\
            \Gamma^{\mathcal{E}^{k}}_{RA^{\prime k}}=\mathcal{P}_{A^{\prime k}}(\Gamma^{\mathcal{E}^{k}}_{RA^{\prime k}}),\\ 
            T_{A^{\prime}_{1\cdots j}}(\Gamma^{\mathcal{E}^{k}_{RA^{\prime k}}}) \geq 0 \quad \forall j\leq k
    \end{array}\right\}
    \end{multline}
    where $\Gamma^{\mathcal{E}^{k}}_{RA^{\prime}}$ is Choi operator of $\mathcal{E}^{k}$, the map $T_R$ is the partial transpose map acting on system $R$, and $\mathcal{P}_{A^{\prime k}}$ is the channel that randomly permutes the systems $A^{\prime k}$.
    
    The following theorem indicates how $\widetilde{F}_s^2$ approximates $F_{s}(\rho_{AB})$.

    \begin{proposition}
    \label{prop:swap-test-ppt-k-channel-sdp}
    The following bound holds for a bipartite state~$\rho_{AB}$:
        \begin{equation}
            F_{s}(\rho_{AB}) \leq \widetilde{F}_s^2(\rho_{AB}) \leq 
            F_{s}(\rho_{AB})+\frac{4  \left\vert A\right\vert^{3} \left\vert B\right\vert}{k}.
        \end{equation}
    \end{proposition}
    \begin{proof}
    Since every entanglement-breaking channel is $k$-extendible, we trivially find that%
    \begin{align}
        & \frac{1+F_{s}(\rho_{AB})}{2} \notag \\
        & =\max_{\mathcal{E}\in\operatorname{EB}}\operatorname{Tr}[(\Pi_{A^{\prime}A}^{\operatorname{sym}}\otimes I_{RB})\mathcal{E}_{R\rightarrow A^{\prime}}(\psi_{RAB})]\\
        & \leq\max_{\mathcal{E}_{R\rightarrow A^{\prime}}^{k}\in\text{EXT}_{k}}\operatorname{Tr}[(\Pi_{A^{\prime}A}^{\operatorname{sym}}\otimes I_{RB})\mathcal{E}_{R\rightarrow A^{\prime}}^{k}(\psi_{RAB})]
        \\
        & = \frac{1+\widetilde{F}_s^2(\rho_{AB})}{2}.
    \end{align}
    Consider the following bound for a $k$-extendible state $\omega_{AB}^{k}$ \cite[Theorem~II.7']{christandl2007one}:%
    \begin{equation}
        \min_{\sigma_{AB}\in\operatorname{SEP}(A:B)}\frac{1}{2}\left\Vert\omega_{AB}^{k}-\sigma_{AB}\right\Vert _{1}\leq\frac{2\left\vert B\right\vert ^{2}}{k}.
    \end{equation}
    We can use it and the result of \cite[Lemma~7]{Wallman_2014} to conclude that
    \begin{equation}
        \min_{\mathcal{E}\in\operatorname{EB}}\frac{1}{2}\left\Vert \mathcal{E}^{k}-\mathcal{E}\right\Vert _{\diamond}\leq\frac{2\left\vert R\right\vert\left\vert A^{\prime}\right\vert ^{2}}{k}.
    \end{equation}
    Then consider, for every fixed choice of $\mathcal{E}_{R\rightarrow A^{\prime}}^{k}$, there exists an entanglement-breaking channel $\mathcal{E}$ satisfying%
    \begin{equation}
        \frac{1}{2}\left\Vert \mathcal{E}^{k}-\mathcal{E}\right\Vert _{\diamond}\leq\frac{2\left\vert R\right\vert \left\vert A^{\prime}\right\vert ^{2}}{k}.
    \end{equation}
    Then we find that%
    \begin{align}
        & \operatorname{Tr}[(\Pi_{A^{\prime}A}^{\operatorname{sym}}\otimes I_{RB})\mathcal{E}_{R\rightarrow A^{\prime}}^{k}(\psi_{RAB})]\notag\\
        &\leq\operatorname{Tr}[(\Pi_{A^{\prime}A}^{\operatorname{sym}}\otimes I_{RB})\mathcal{E}_{R\rightarrow A^{\prime}}(\psi_{RAB})]+\frac{2\left\vert R\right\vert \left\vert A^{\prime}\right\vert ^{2}}{k}\\
        &\leq\max_{\mathcal{E}\in\operatorname{EB}}\operatorname{Tr}[(\Pi_{A^{\prime}A}^{\operatorname{sym}}\otimes I_{RB})\mathcal{E}_{R\rightarrow A^{\prime}}(\psi_{RAB})]+\frac{2\left\vert R\right\vert \left\vert A^{\prime}\right\vert ^{2}}{k}\\
        &=\frac{1+F_{s}(\rho_{AB})}{2}+\frac{2\left\vert R\right\vert \left\vert A^{\prime}\right\vert ^{2}}{k}.
    \end{align}
    Since the inequality holds for every $\mathcal{E}_{R\rightarrow A^{\prime}}^{k}\in$ EXT$_{k}$, it follows that%
    \begin{multline}
        \max_{\mathcal{E}_{R\rightarrow A^{\prime}}^{k}\in\text{EXT}_{k}}\operatorname{Tr}[(\Pi_{A^{\prime}A}^{\operatorname{sym}}\otimes I_{RB})\mathcal{E}_{R\rightarrow A^{\prime}}^{k}(\psi_{RAB})]\\
        \leq\frac{1+F_{s}(\rho_{AB})}{2}+\frac{2\left\vert R\right\vert \left\vert A^{\prime}\right\vert ^{2}}{k}.
    \end{multline}
    This concludes the proof, after recalling that $|R| \leq |A| |B|$, observing that $|A|=|A'|$, and performing some simple algebra.
    \end{proof}

    \begin{remark}
        Although the correction term in the upper bound in Proposition~\ref{prop:swap-test-ppt-k-channel-sdp} decreases with increasing $k$, it is clear that, in order for it to become arbitrarily small,  $k$ needs to be larger than $|A|^3|B|$, which is exponential in the number of qubits for the state $\rho_{AB}$. Thus, this approach does not lead to an efficient method for placing the fidelity of separability estimation problem in QIP(2).
    \end{remark}
    
    \section{Local Reward Function}
    
    \label{appendix:local_cost}

    In this appendix, we develop a local reward function, as an alternative to the global reward function considered in the main text, i.e., the acceptance probability in  Theorem~\ref{theorem:global_cost_func}. 
    The acceptance probability in Theorem~\ref{theorem:global_cost_func} can be thought as global reward function because it corresponds to the probability of measuring zero in every register. As indicated in \cite{Cerezo2021a}, it is helpful to employ a local reward function in order to mitigate the barren plateau problem \cite{McClean2018}, which plagues all variational quantum algorithms. 
    
    Let us define the local and global reward functions. To begin, let $Z_i$ be the event of measuring zero in the $i$th register. We then set the local reward function to be the probability of measuring zero in a register chosen uniformly at random; that is, it is given by the following:
    \begin{equation}
        L \equiv \frac{1}{n} \sum_i \Pr\!\left(Z_i\right).
    \end{equation}
    The event of measuring all zeros is given by $\bigcap_i Z_i$, and the probability that this event occurs is $G \equiv \Pr\!\left(\bigcap_i Z_i\right)$, which is what we used in the main text as the global reward function.
    
    Now, we are interested in determining inequalities that relate the global and local reward functions, and the following analysis  employs the same ideas used in \cite[Appendix C]{Khatri2019quantumassisted}. Using DeMorgan's laws, we find that
    \begin{equation}
        \Pr\!\left(\bigcap_i Z_i\right) = \Pr\!\left(\left(\bigcup_i Z_i^c\right)^c\right) = 1- \Pr\!\left(\bigcup_i Z_i^c\right).
    \end{equation}
    We can then use the union bound to conclude that
    \begin{equation}
        \Pr\!\left(\bigcap_i Z_i\right) = 1- \Pr\!\left(\bigcup_i (Z_i)^c\right) \geq 1- \sum_i \Pr\!\left((Z_i)^c\right).
    \end{equation}
    Finally, consider that
    \begin{align}
    \label{eqn:lower_bnd_local_cost}
        G & = \Pr\!\left(\bigcap_i Z_i\right) \\
        & \geq 1- \sum_i \Pr\!\left(Z_i^c\right) \\
        & = \sum_i \Pr\!\left(Z_i\right) - (n-1)\\
        & = n L - (n-1)\\
        & = n (L-1) + 1.
    \end{align}
    
    
    
    We can also derive an upper bound on the global reward function in terms of the local reward function. Recall the following inequality, which holds for every set $\{A_1,A_2,\ldots,A_n\}$ of events:
    \begin{equation}
        \Pr\!\left(\bigcup_i A_i\right) \geq \frac{1}{n} \sum_i \Pr\!\left(A_i\right).
    \end{equation}
    Setting $A_i=Z_i^c$, we get
    \begin{equation}
        \Pr\!\left(\bigcup_i Z_i^c\right) \geq \frac{1}{n} \sum_i \Pr\!\left(Z_i^c\right).
    \end{equation}
    Using DeMorgan's laws, we obtain the desired upper bound as follows:
    \begin{align}
        \Pr\!\left(\bigcap_i Z_i\right) & \leq 1-\frac{1}{n} \sum_i \left(1-\Pr\!\left(Z_i\right)\right)\\
        &= \frac{1}{n} \sum_i \Pr\!\left(Z_i\right).
    \end{align}

    In summary, we have established the following bounds:
    \begin{equation}
          n (L-1) + 1 \leq  G \leq L,
    \end{equation}
    so that $G = 1 $ if and only if $L=1$. Since we always have $G \in [ 0,1]$, the lower bound is only nontrivial if $L$ is sufficiently large, i.e., if $L \geq 1 - \frac{1}{n}$.

    

    \section{More Simulations}
    \label{appendix:simulations}

\begin{figure}
        \centering
        \subfigure[Fidelity of separability calculated for a three-qubit GHZ state using our VQSA (green line) converging to the true value of 0.5.]{\includegraphics[width=\columnwidth]{figures/plot_ghz_state.pdf}\label{fig:Max_Sep_Fidelity_GHZ}}
        \subfigure[Fidelity of separability calculated for a three-qubit W state using our VQSA (green line) converging to the true value of 4/9.]{\includegraphics[width=\columnwidth]{figures/plot_w_state.pdf}\label{fig:Max_Sep_Fidelity_W}}
        \caption{Fidelity of separability estimated for multipartite states using the VQSA and benchmarked known values of fidelity of separability.}
    \end{figure}

    \begin{figure}
        \centering
        \subfigure[Fidelity of separability calculated for a random product state using the local reward function of the VQSA and benchmarked by $\widetilde{F}_s^1$.]{\includegraphics[width=\columnwidth]{figures/plot_random_prod_state_3_4_2_2.pdf}\label{fig:local_cost_function1}}
        \subfigure[Fidelity of separability calculated for a random entangled state using the local reward function of the VQSA and benchmarked by $\widetilde{F}_s^1$.]{\includegraphics[width=\columnwidth]{figures/plot_random_state_2_4_2_2.pdf}\label{fig:local_cost_function2}}
        \caption{Fidelity of separability estimated using the local reward function of the VQSA and benchmarked by $\widetilde{F}_s^1$.} 
    \end{figure}
    
    Here we report the results of simulations of the multipartite extension to our VQSA, for three-qubit GHZ and W states. These simulations are based on \eqref{eqn:multi_fid_sep} since GHZ and W states are pure states. From \cite{Chen2010}, we also know the values of the fidelity of separability for tripartite  GHZ and W states, and these simulations can be seen in
    Figures~\ref{fig:Max_Sep_Fidelity_GHZ} and \ref{fig:Max_Sep_Fidelity_W}, respectively.
    
    
    In Figure~\ref{fig:local_cost_function1}, we report simulation results after generating a random bipartite product state, with each partition containing two qubits, and calculating the fidelity of separability using both the local reward function of the VQSA and the benchmark $\widetilde{F}^1_s$, the latter discussed in Appendix~\ref{appendix:ppt-k-state-sdp}. In Figure~\ref{fig:local_cost_function2}, we do the same for a random bipartite state with each partition containing two qubits and four qubits in the reference system. All random states are generated using the hardware efficient ansatz (HEA) \cite{KMTTBCG17}. The local reward function of the VQSA requires more classical processing (like picking a qubit at random to measure) and seems to require more iterations to reach the right value. However, these downsides are outweighed by the fact that it is less susceptible to the emergence of barren plateaus. 
    

    \section{Software}
    \label{appendix:software}
    All of our Python source files are available with the arXiv posting of this paper. We perform all simulations using the noisy Qiskit Aer simulator. The Picos Python package~\cite{sagnol2012picos} was used to invoke the CVXOPT solver~\cite{vandenberghe2010cvxopt} for solving the SDPs and the toqito Python package~\cite{russo2021toqito} was used for carrying out specific operations on the matrices representing quantum systems.  

    

\end{document}
