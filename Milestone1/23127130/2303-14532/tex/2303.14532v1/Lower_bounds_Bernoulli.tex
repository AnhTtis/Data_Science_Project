
\documentclass[eqnumis, pdf]{yjb}
%%%%%%%%%%%%%%%%%%%%%%%%%%%%%%%%%%%%%%%%%%%%%%%%%%%%%%%%%%%%%%%%%%%%%%%%%%%%%%%%%%%%%%%%%%%%%%%%%%%%%%%%%%%%%%%%%%%%%%%%%%%%%%%%%%%%%%%%%%%%%%%%%%%%%%%%%%%%%%%%%%%%%%%%%%%%%%%%%%%%%%%%%%%%%%%%%%%%%%%%%%%%%
\usepackage{amsfonts}
\usepackage{amsmath}
\usepackage{footnote}
\usepackage{multicol}
\usepackage{booktabs}
\usepackage[flushleft]{threeparttable}
\usepackage[font=small,labelfont=bf]{caption}
\usepackage{float}
\usepackage{hyperref}
\DeclareMathOperator{\sech}{sech}
\hypersetup{
colorlinks,
citecolor=magenta,
linkcolor=blue,
urlcolor=magenta}
\usepackage{mathrsfs,upref}



\begin{document}
%please do not change this part%%%%%

%%%%%%%%%%%%%%%%%%%%%%%%%%%%%%
%\giris
%{ } %yazarlar
%{ } %başlık
%{ } %ilk sayfa no
%author, title, first page number
%%%%%%%%%%%%%%%%%%%%%%%%%%%%%%
%Corresponding Author Email:
% 
%%%%%%%%%%%%%%%%%%%%%%%%%%%%%%

\markboth{Yogesh J. Bagul}{Stringent lower bounds for non-zero Bernoulli numbers}

\title{Stringent lower bounds for non-zero Bernoulli numbers}

\author{Yogesh J. Bagul}

\address{Department of Mathematics\\ K. K. M. College, Manwath\\ Dist: Parbhani (M. S.) - 431505, India
	}
\email{yjbagul@gmail.com}

\maketitle
\begin{abstract}
The author presents new several sharper and sharper lower bounds for the non-zero Bernoulli numbers using Euler's formula for the Riemann zeta function. 
\end{abstract}% the abstract
\subjclass{11B68, 11M06, 26D99}  % AMS subject classifications
\keywords{Bernoulli number, Riemann zeta function, Stirling formula, lower bound.}        % Keywords

%please do not change this part%%%%%%%%%%%receive, accept
%%%%%%%%%%%%%%%%%%%%%%%%%%%%%%%%%%%

%Accepted Date: 
%Received Date: 
%DOI: 
\vspace{10pt}

\section{Introduction}\label{sec1}
The classical Bernoulli numbers frequently occur in mathematical analysis and other related branches of mathematics and science. These fascinating numbers which are a sequence of rational numbers were discovered by the Swiss mathematician Jacob Bernoulli. They are denoted by $ B_k $ and may be defined by (\cite[p. 804]{abramowitz}, \cite[p. 3]{temme})
$$ \frac{x}{e^x - 1} = \sum_{k=0}^\infty B_k \frac{x^k}{k!},  \, \, \vert x \vert < 2\pi. $$
$ B_1 = -1/2$ and all the odd-indexed Bernoulli numbers $B_{2k+1} , \, \,  k \in \mathbb{N}$ are zero. On the other hand, all the even-indexed Bernoulli numbers $B_{2k}$ are non-zero and they alternate in signs, i.e., for $ k= 1, 2, 3, 4, 5, 6, 7, 8, \cdots$, the numbers $ B_{2k} $ are respectively given by $1/6, -1/30, 1/42, -1/30, 5/66, -691/2730, 7/6, -3617/510, \cdots.$ Hence, $$ \vert B_{2k} \vert = (-1)^{k+1} B_{2k}, \, \, k = 1, 2, 3, \cdots. $$
The bounds for even-indexed Bernoulli numbers play a vital role in the theory of inequalities and thus bounding these Bernoulli numbers has been the topic of interest.

The double inequality 
\begin{align}\label{eqn1.1}
\frac{2 \cdot(2k)!}{(2 \pi)^{2k}} < \vert B_{2k} \vert < \frac{2 \cdot(2k)!}{(2 \pi)^{2k}} \cdot \frac{2^{2k-1}}{(2^{2k-1}-1)} = \frac{(2k)!}{\pi^{2k}(2^{2k-1}-1)}; \, \, k = 1, 2, \cdots
\end{align}
was appeared in \cite[p. 805; 23.1.15]{abramowitz}. In 1989, D. J. Leeming \cite[p. 128]{leeming} established that 
\begin{align}\label{eqn1.2}
4 \sqrt{\pi k} \left( \frac{k}{\pi e} \right)^{2k} < \vert B_{2k} \vert < 5 \sqrt{\pi k}\left( \frac{k}{\pi e} \right)^{2k}; \, \, k \geq 2.
\end{align}
As stated in \cite{daniello} the left inequality of (\ref{eqn1.2}) was already established by A. Laforgia \cite[p. 2]{laforgia} in 1980. Using Fourier representation, C. D'Aniello \cite{daniello} obtains 
\begin{align}\label{eqn1.3}
B_{2k} = \frac{(-1)^{k-1} \cdot 2 \cdot (2k)!}{(2\pi)^{2k}} \sum_{n=1}^\infty \frac{1}{n^{2k}}
\end{align}
and hence
\begin{align}\label{eqn1.4}
\frac{2 \cdot(2k)!}{(2 \pi)^{2k}} \left( 1 + \frac{1}{2^{2k}} \right) < \vert B_{2k} \vert ; \, \, k = 1, 2, \cdots.
\end{align}
Further, using the Stirling formula \cite{abramowitz}, the inequality 
\begin{align}\label{eqn1.5}
4 \sqrt{\pi k} \left( \frac{k}{\pi e} \right)^{2k} \left( 1 + \frac{1}{2^{2k}} \right) < \vert B_{2k} \vert ; \, \, k = 1, 2, \cdots
\end{align}
was also obtained in \cite{daniello}. In the same paper \cite{daniello}, D'Aniello used a relation 
\begin{align}\label{eqn1.6}
\vert B_{2k} \vert = \frac{2 \cdot (2k)!}{\pi^{2k}} \frac{1}{(2^{2k}-1)}\sum_{n=0}^\infty \frac{1}{(2n+1)^{2k}} ; \, \, k = 1, 2, \cdots
\end{align}
to establish 
\begin{align}\label{eqn1.7}
\frac{2 \cdot (2k)!}{\pi^{2k}} \frac{1}{(2^{2k}-1)} < \vert B_{2k} \vert ; \, \, k = 1, 2, \cdots.
\end{align}
Among all the lower bounds of $ \vert B_{2k} \vert $ listed above, the lower bound in (\ref{eqn1.7}) is stringent. Thus, by combining (\ref{eqn1.1}) and (\ref{eqn1.7}) we have 
\begin{align}\label{eqn1.8}
\frac{2 \cdot(2k)!}{\pi^{2k}(2^{2k}-1)} < \vert B_{2k} \vert <  \frac{(2k)!}{\pi^{2k}(2^{2k-1}-1)}; \, \, k = 1, 2, \cdots.
\end{align}
H. Alzer \cite{alzer} showed that the constants $ \alpha = 0 $ and $ \beta = 2 + \frac{\ln(1-6/\pi^2)}{\ln 2} $ such that 
\begin{align}\label{eqn1.9}
\frac{2 \cdot (2k)!}{(2\pi)^{2k}} \frac{1}{1-2^{\alpha - 2k}} \leq \vert B_{2k} \vert \leq \frac{2 \cdot (2k)!}{(2\pi)^{2k}} \frac{1}{1-2^{\beta - 2k}}; \, \, k \in \mathbb{N}
\end{align}
are the best possible. The upper bound in (\ref{eqn1.9}) is sharper than that in (\ref{eqn1.8}). Another upper bound for $ \vert B_{2k} \vert $ was given by H.-F. Ge \cite{ge} as follows:
\begin{align}\label{eqn1.10}
\vert B_{2k} \vert \leq \frac{2(2^{2n}-1)}{2^{2n}} \zeta(2n) \frac{(2k)!}{\pi^{2k}(2^{2k}-1)}; \, \,  k \geq n \, \, \text{and} \, \, n \in \mathbb{N}.
\end{align}

 The upper bound in (\ref{eqn1.8}) has been nicely sharpened, viz. in (\ref{eqn1.9}) and (\ref{eqn1.10}). However, to the best of the author's knowledge and belief, the lower bound in (\ref{eqn1.8}) is not presented in its refined form. The aim of this article is to present refined lower bounds for $ \vert B_{2k} \vert. $ In fact, from (\ref{eqn1.3}) and (\ref{eqn1.6}) we immediately have the following proposition.
\begin{proposition}\label{prop1.1}
\begin{itemize}
\item[(i)] For $ m, k \in \mathbb{N}, $ we have
\begin{align}\label{eqn1.11}
\frac{ 2 \cdot (2k)!}{(2\pi)^{2k}} \sum_{n=1}^m \frac{1}{n^{2k}} < \vert B_{2k} \vert.
\end{align} 
\item[(ii)] For $ m = 0, 1, 2, \cdots $ and $ k \in \mathbb{N} $ we have 
\begin{align}\label{eqn1.12}
\frac{2 \cdot (2k)!}{\pi^{2k}} \frac{1}{(2^{2k}-1)}\sum_{n=0}^m \frac{1}{(2n+1)^{2k}} <  \vert B_{2k} \vert.
\end{align}
\end{itemize} 
\end{proposition}  
It is not difficult to prove that the lower bound in (\ref{eqn1.12}) is finer than that in (\ref{eqn1.11}). Putting $ m = 2 $ and $ m = 0 $ in (\ref{eqn1.11}) and (\ref{eqn1.12}) respectively we get the inequalities (\ref{eqn1.4}) and (\ref{eqn1.7}). To obtain sharper lower bounds, one should increase the value of $ m $ in (\ref{eqn1.11}) and (\ref{eqn1.12}). For example, by putting $ m=3 $ and $ m = 1$ in (\ref{eqn1.11}) and (\ref{eqn1.12}) respectively we get
\begin{align}\label{eqn1.13}
\frac{ 2 \cdot (2k)!}{(2\pi)^{2k}} \left(1 + \frac{1}{2^{2k}} + \frac{1}{3^{2k}}\right) < \vert B_{2k} \vert ; \, \, k = 1, 2, \cdots
\end{align}
and 
\begin{align}\label{eqn1.14}
\frac{2 \cdot (2k)!}{\pi^{2k}} \frac{1}{(2^{2k}-1)} \left( 1 + \frac{1}{3^{2k}} \right) <  \vert B_{2k} \vert ; \, \, k = 1, 2, \cdots.
\end{align}

Irrespective of this, we will present more stringent and convincing bounds for $ \vert B_{2k} \vert $ in the next section.

\section{Main results}\label{sec2}

The following are the main results of the paper.
\begin{proposition}\label{prop2.1}
Let $ \mathcal{P} = \left\lbrace p_1 = 3, p_2 = 5, p_3 = 7, p_4 = 11, p_5 = 13, p_6 = 17, \cdots \right\rbrace $ be the set of all odd prime numbers. Then 
\begin{align}\label{eqn2.1}
\frac{2 \cdot (2k)!}{\pi^{2k} (2^{2k}-1)} \prod_{n=1}^m \left(\frac{p_n^{2k}}{p_n^{2k}-1}\right) < \vert B_{2k} \vert , 
\end{align}
where $m, k \in \mathbb{N} $ and $p_n \in \mathcal{P}$  
\end{proposition}
\begin{proof}
Using Euler's fabulous product formula \cite{gautschi} for the zeta function, we have 
$$ \zeta(2k) = \left(\frac{2^{2k}}{2^{2k - 1}}\right) \prod_{n \in \mathbb{N}} \left(\frac{p_n^{2k}}{p_n^{2k}-1}\right), $$ where $ k \in \mathbb{N}, \, \, p_n \in \mathcal{P}. $ This implies 
$$ \zeta(2k) > \left(\frac{2^{2k}}{2^{2k - 1}}\right) \prod_{n=1}^m \left(\frac{p_n^{2k}}{p_n^{2k}-1}\right). $$
It is given in \cite[p. 5, (1.14)]{temme} that 
$$ \vert B_{2k} \vert = \frac{2 \cdot (2k)!}{(2\pi)^{2k}} \zeta(2k), \, \, k \in \mathbb{N}. $$ Hence
$$ \frac{\pi^{2k}(2^{2k} - 1)}{2 \cdot (2k)!} \vert B_{2k} \vert > \prod_{n=1}^m \left(\frac{p_n^{2k}}{p_n^{2k}-1}\right), $$ which gives the desired result.
\end{proof}
In particular, for $ m = 1, 2, $ Proposition \ref{prop2.1} yields respectively the following inequalities:
\begin{align}\label{eqn2.2}
\frac{2 \cdot (2k)!}{\pi^{2k} (2^{2k} - 1)} \frac{3^{2k}}{(3^{2k}-1)} < \vert B_{2k} \vert , \, \, k \in \mathbb{N}
\end{align}
and
\begin{align}\label{eqn2.3}
\frac{2 \cdot (2k)!}{\pi^{2k} (2^{2k} - 1)} \frac{3^{2k}}{(3^{2k}-1)} \frac{5^{2k}}{(5^{2k}-1)} < \vert B_{2k} \vert , \, \, k \in \mathbb{N}.
\end{align}
It is easy to check that the inequalities (\ref{eqn2.2}) and (\ref{eqn2.3}) are refinements of the inequality (\ref{eqn1.14}). In Proposition \ref{prop2.1}, we also observe that the larger the value of $ m $ is, the sharper is the corresponding inequality.

\begin{remark} The statement of Proposition \ref{eqn2.1} can be reformulated as follows: 

Let $ \mathcal{P}_0 = \left\lbrace p_0 = 2, p_1 = 3, p_2 = 5, p_3 = 7, p_4 = 11, p_5 = 13, p_6 = 17, \cdots \right\rbrace $ be the set of all prime numbers and $ m \in \left\lbrace 0, 1, 2, \cdots \right\rbrace. $ Then 
\begin{align}\label{eqn2.4}
\frac{2 \cdot (2k)!}{(2\pi)^{2k}} \prod_{n=0}^m \left(\frac{p_n^{2k}}{p_n^{2k}-1}\right) < \vert B_{2k} \vert , 
\end{align}
where $k \in \mathbb{N} $ and $p_n \in \mathcal{P}_0.$  
\end{remark}
Here, (\ref{eqn2.4}) clearly generalizes the inequality (\ref{eqn1.7}).

The next Proposition shows that the lower bounds in Proposition \ref{prop2.1} are sharper than those in Proposition \ref{prop1.1}.
\begin{proposition}\label{prop2.2}
If $ \mathcal{P} = \left\lbrace p_1 = 3, p_2 = 5, p_3 = 7, p_4 = 11, p_5 = 13, p_6 = 17, \cdots \right\rbrace $ is the set of all odd prime numbers, then 
$$ \prod_{n=1}^m \left(\frac{p_n^{2k}}{p_n^{2k}-1}\right) > \sum_{n=0}^m \frac{1}{(2n+1)^{2k}}, $$
where $k \in \mathbb{N} $ and $p_n \in \mathcal{P}$.  
\end{proposition}
\begin{proof}
For $k \in \mathbb{N} $ and $p_n \in \mathcal{P}$, we write
$$ \prod_{n=1}^m \left(\frac{p_n^{2k}}{p_n^{2k}-1}\right) = \prod_{n=1}^m \left(\frac{1}{1-\frac{1}{p_n^{2k}}}\right), $$ i.e.,
$$\prod_{n=1}^m \left(\frac{p_n^{2k}}{p_n^{2k}-1}\right) = \left(\frac{1}{1-\frac{1}{3^{2k}}}\right) \cdot \left(\frac{1}{1-\frac{1}{5^{2k}}}\right) \cdot \left(\frac{1}{1-\frac{1}{7^{2k}}}\right) \cdot \left(\frac{1}{1-\frac{1}{11^{2k}}}\right) \cdots \left(\frac{1}{1-\frac{1}{p_m^{2k}}}\right).$$ Making use of $$ \frac{1}{1-x} = 1 + x + x^2 + x^3 + \cdots, \, \, \vert x \vert < 1, $$ we get
\begin{multline}\nonumber
\prod_{n=1}^m \left(\frac{p_n^{2k}}{p_n^{2k}-1}\right) = \left( 1 + \frac{1}{3^{2k}} + \frac{1}{9^{2k}} + \frac{1}{3^{6k}} + \cdots \right) \times \left( 1 + \frac{1}{5^{2k}} + \frac{1}{5^{4k}} + \frac{1}{5^{6k}} + \cdots \right) \times  \\ 
\left( 1 + \frac{1}{7^{2k}} + \frac{1}{7^{4k}} + \frac{1}{7^{6k}} + \cdots \right) \times \left( 1 + \frac{1}{11^{2k}} + \frac{1}{11^{4k}} + \frac{1}{11^{6k}} + \cdots \right) \times \\
\cdots \left( 1 + \frac{1}{p_m^{2k}} + \frac{1}{p_m^{4k}} + \frac{1}{p_m^{6k}} + \cdots \right).\\
\end{multline}
Thus 
\begin{align*}
\prod_{n=1}^m \left(\frac{p_n^{2k}}{p_n^{2k}-1}\right) = &\left(1 + \frac{1}{3^{2k}} + \frac{1}{5^{2k}} + \frac{1}{7^{2k}} + \frac{1}{9^{2k}} \cdots + \frac{1}{(2m+1)^{2k}} + \cdots + \frac{1}{p_m^{2k}}\right) \\
&+ \text{sum of numbers greater than zero} \\
&> 1 + \frac{1}{3^{2k}} + \frac{1}{5^{2k}} + \frac{1}{7^{2k}} + \frac{1}{9^{2k}} \cdots + \frac{1}{(2m+1)^{2k}} = \sum_{n=0}^m \frac{1}{(2n+1)^{2k}}.
\end{align*}
\end{proof}
Let $L_{2k} $ be the lower bound of $ \vert B_{2k} \vert $ given in (\ref{eqn2.1}). For $ m = 1, $ we list the approximate numerical values $ \vert B_{2k} \vert - L_{2k}$ in the below table.

{
\begin{table}[H]
\centering
\caption{ Numerical values of $ \vert B_{2k} \vert - L_{2k}$ for $ m = 1.$ }
\vspace{.5cm}
{\renewcommand{\arraystretch}{2}
{\small
\begin{tabular}{c|c|c|c|c|c|c|c|}
\cline{2-6}
& \multicolumn{5}{|c|}{$ m = 1 $}  \\ \hline
\multicolumn{1}{|c|}{$k$} &$ 1 $ &$ 2  $ &$ 3$ &$4 $ & $5$  \\\hline
\multicolumn{1}{|c|}{$\vert B_{2k} \vert - L_{2k}$} & $ 1.46... \times 10^{-2}$  &$ 7.15... \times 10^{-5} $  &$  1.74... \times 10^{-6} $  &$  9.13... \times 10^{-8}  $ & $ 8.02... \times 10^{-9}$   \\ \hline
\cline{2-6}
& \multicolumn{4}{|c|}{$ m = 1$}    \\ \hline
\multicolumn{1}{|c|}{$k$} &$6$ &$7$ &$ 8$ &$9 $ & $10$  \\
\hline
\multicolumn{1}{|c|}{$\vert B_{2k} \vert - L_{2k}$} & $1.05... \times 10^{-9}$  &$ 1.92... \times 10^{-10} $  &$ 4.66... \times 10^{-11} $  &$ 1.43... \times 10^{-11} $ & $5.11... \times 10^{-12}$   \\
\hline
\end{tabular}
}}
\label{tab1}
\end{table}
}

From these values, we conclude that the lower bound in (\ref{eqn2.1}) is sufficiently sharp even for $ m=1.$ One should get sharper and sharper lower bounds for larger values of $m.$

\begin{corollary}\label{cor1}
Suppose that $ a $ is any real number and $ k \in \mathbb{N}. $ Then 
\begin{align}\label{eqn2.5}
\frac{2 \cdot (2k)!}{\pi^{2k} (2^{2k} - 1)} \frac{a^{2k}}{(a^{2k}-1)} < \vert B_{2k} \vert
\end{align}
if $ a \geq 3.$
\end{corollary}
\begin{proof}
For $ a \geq 3, $ it is obvious that
$$ \frac{a^{2k}}{(a^{2k}-1)} \leq \frac{3^{2k}}{(3^{2k}-1)}. $$ Then by (\ref{eqn2.2}), we get the inequality (\ref{eqn2.5}).
\end{proof}
\begin{remark}
Applying the same argument as in the proof of Corollary \ref{cor1}, for any real number $ a $ and $ m, k \in \mathbb{N}, $ we have 
\begin{align}
\frac{2 \cdot (2k)!}{\pi^{2k}(2^{2k}-1)}\frac{3^{2k}}{(3^{2k}-1)}\frac{5^{2k}}{(5^{2k}-1)} \cdots \frac{p_{m-1}^{2k}}{(p_{m-1}^{2k}-1)} \frac{a^{2k}}{(a^{2k}-1)} < \vert B_{2k} \vert
\end{align}
if $ a \geq p_m, $ where $ p_i \in \mathcal{P}.$
\end{remark}

\begin{thebibliography}{99}      
        
\bibitem{abramowitz}
         M. Abramowitz and I. A. Stegun (Editors),
         {\em Handbook of Mathematical Functions with Formulas, Graphs and Mathematical Tables}, National Bureau of Standards, Applied Mathematics Series {\bf 55}, 9th printing, Washington, DC, 1970.
         
\bibitem{alzer}
        H. Alzer,
        {\em Sharp bounds for the Bernoulli numbers,}  Arch. Math., {\bf 74} (2000), No. 3, pp. 207--211. \url{https://doi.org/10.1007/s000130050432}           
         
\bibitem{daniello}
        C. D'Aniello,
        {\em On some inequalities for the Bernoulli numbers,} Rend. Circ. Mat. Palermo (2), {\bf 43} (1994), No. 3, pp. 329--332. \url{https://doi.org/10.1007/BF02844246}        

\bibitem{gautschi} 
        W. Gautschi,
        {\em Leonhard Euler: His Life, the Man, and His Works,} SIAM Review, {\bf 50} (2008), No. 1, pp. 3--33. \url{http://www.jstor.org/stable/20454060}  

\bibitem{ge}
        H.-F. Ge,
        {\em New sharp bounds for the Bernoulli numbers and refinement of Becker-Stark inequalities,} J. Appl. Math., {\bf 2012} (2012), Art. ID 137507, 7 pages. \url{https://doi.org/10.1155/2012/137507}               
        
\bibitem{laforgia}
       A. Laforgia, 
       {\em Inequalities for Bernoulli and Euler numbers,} Boll. Unione Mat. Ital., V. Ser., A {\bf 17} (1980), No. 1, 98--101.        
        
\bibitem{leeming}
       D. J. Leeming,
       {\em The real zeros of the Bernoulli polynomials,} J. Approx. Theory, {\bf 58} (1989), No. 2, pp. 124--150. \url{https://doi.org/10.1016/0021-9045(89)90016-6}       
        
\bibitem{qi}
         F. Qi,
         {\em A double inequality for the ratio of two non-zero neighbouring Bernoulli numbers,} J. Comput. Appl. Math., {\bf 351} (2019), No. 1, pp. 1--5. \url{https://doi.org/10.1016/j.cam.2018.10.049} 

\bibitem{shuang}
       Y. Shuang, B.-N. Guo, and F. Qi,
       {\em Logarithmic convexity and increasing property of the Bernoulli numbers and their ratios,} RACSAM, {\bf 115} (2021), 12 pages. \url{https://doi.org/10.1007/s13398-021-01071-x}  

\bibitem{temme}
        N. M. Temme,
        {\em Special Functions: An Introduction to Classical Functions of Mathematical Physics,} Wiley 1996.
        
         
\bibitem{zhu}
        L. Zhu,
        {\em New bounds for the ratio of two adjacent even-indexed Bernoulli numbers,} RACSAM, {\bf 114} (2020), 13 pages. \url{https://doi.org/10.1007/s13398-020-00814-6}               
        
      
                
        
  
\end{thebibliography}
\end{document}

