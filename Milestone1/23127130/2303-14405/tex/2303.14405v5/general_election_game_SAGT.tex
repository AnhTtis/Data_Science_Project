\documentclass[runningheads]{llncs}
\usepackage{fullpage}
\usepackage{booktabs} % For formal tables
%\usepackage{pifont}
\usepackage{bbding}
%\usepackage{array}
\usepackage{amssymb}
%\usepackage{amssymb}
%\usepackage{fullpage}
\usepackage{amsmath,bm}
%\usepackage{bbm}
%\usepackage{amsthm}
%\usepackage{latexsym}
\usepackage{graphicx}
%\usepackage[ruled]{algorithm2e} % For algorithms
%\usepackage{flushend}
\usepackage{color}
\usepackage[numbers,sort&compress]{natbib}
\DeclareMathOperator*{\argmax}{arg\,max}
\DeclareMathOperator*{\argmin}{arg\,min}
\usepackage{algorithm}
\usepackage{algorithmic}
\renewcommand{\algorithmiccomment}[1]{/* #1 *\!/}
\renewcommand{\algorithmicrequire}{\textbf{Input:}}
\renewcommand{\algorithmicensure}{\textbf{Output:}}
\usepackage{enumitem}
\usepackage{comment}
%\newtheorem{Conj}{Conjecture}
\newcommand\genatop[2]{\genfrac{}{}{0pt}{}{#1\hfill}{#2\hfill}}
\renewcommand{\bibsection}{\section*{References}}
\renewcommand{\thealgorithm}{}

\begin{document}
%%%%%%%%%%%%%%%%%%%%%%%%%%%%%%%%%%%%%%%%%%%%%%%%%%%%%%%%%%%%%%%%%%%


\title{On the Efficiency of An Election Game of Two or More Parties: How Bad Can It Be?\thanks{This work is supported by the Taiwan Ministry of Science and Technology under grant no. NSTC 110-2222-E-032-002-MY2, NSTC 111-2410-H-A49-022-MY2 and NSTC 112-2221-E-032-018-MY3. A previous version appeared at the 6th Games, Agents, and Incentives Workshop (GAIW-24).}} %We also thank Roman Akchurin for part of the implementation of the game instance generation.}}

\iffalse
\title{On the Efficiency of An Election Game of Two or More Parties: How Bad Can It Be? \thanks{This work is supported by the Taiwan Ministry of Science and Technology under grant no. NSTC 110-2222-E-032-002-MY2, NSTC 111-2410-H-A49-022-MY2 and NSTC 112-2221-E-032-018-MY3.}}
\fi

\titlerunning{How Bad Can the Election Game with Two or More Parties Be?}

\author{Chuang-Chieh Lin\inst{1} \and Chi-Jen Lu\inst{2}\and Po-An Chen\inst{3}\thanks{Corresponding author.}}

\authorrunning{C.-C. Lin, C.-J. Lu, and P.-A. Chen}

\institute{Department of Computer Science and Information Engineering, Tamkang University\\No.~151, Yingzhuan Rd., Tamsui Dist., New Taipei City 25137, Taiwan \\\email{josephcclin@gms.tku.edu.tw}\and 
Institute of Information Science, Academia Sinica\\128 Academia Road, Section 2, Nankang, Taipei 11529, Taiwan
\\\email{cjlu@iis.sinica.edu.tw}\and
Institute of Information Management, National Yang-Ming Chiao-Tung University\\1001 University Rd, Hsinchu City 300, Taiwan \\\email{poanchen@nctu.edu.tw}}

%%%%%%%%%%%%%%%%%%%%%%%%%%%%%%%%%%%%%%%%%%%%%%%%%%%%%%%%%%%%%%%%%%%
\maketitle
\begin{abstract}
%We extend the previous work on two-party election competition [Lin, Lu \& Chen~2021] to the setting of three or more parties. 
An election campaign among two or more parties can be viewed as a game of two or more players, each of which has its own candidates as the pure strategies. People, as voters, comprise supporters for each party, and a candidate brings utility for the supporters of each party. Each party nominates exactly one of its candidates to compete against the other party's. \emph{A candidate is assumed to win the election with greater or equal odds if it brings more utility for all the people.} The payoff of each player is the expected utility that its supporters get. The game is \emph{egoistic} if every candidate benefits its party's supporters more than any candidate from a competing party does. 
%%
In this paper, we first prove that it is {\sf NP}-complete to determine whether an election game in a succinct representation, which is called the \emph{general form}, has a pure-strategy Nash equilibrium even if it is egoistic. Next, 
we %propose two sufficient conditions for an egoistic election game to have a pure-strategy Nash equilibrium. Based on these conditions, we 
propose a fixed-parameter tractable algorithm to compute a pure-strategy Nash equilibrium of an egoistic election game and show that a na\"{i}ve constant time algorithm leads to a $(1+e)$-approximate pure-strategy Nash equilibrium when the winning probability is computed by a softmax function. Finally, perhaps surprisingly, we show that the price of anarchy for egoistic election games is upper bounded by the number of parties. Our results suggest that an election becomes unpredictable in terms of stability and efficiency when more than two parties are involved, and, to some extent, also provides supporting arguments for why the two-party system is prevalent in democratic countries. 
%moreover, the price-of-anarchy bound deteriorates with an increasing number of participating parties. This provides one of supporting arguments why the two-party system is prevalent in democratic countries. 
\keywords{Election game, Nash equilibrium, Price of anarchy, Egoism, Monotonicity}
\end{abstract}


%%%%%%%%%%%%%%%%%%%%%%%%%%%%%%%%%%%%%%%%%%%%%%%%%%%%%%%%%%%%%%%%%%%%%
\section{Introduction}
\label{sec:intro}
%%%%%%%%%%%%%%%%%%%%%%%%%%%%%%%%%%%%%%%%%%%%%%%%%%%%%%%%%%%%%%%%%%%%%


Modern democracy runs on political parties and their competitions in elections. %prevail in democratic countries. 
In an election campaign, political parties compete with each other and exhaust their efforts and resources for %to attract
voters' ballots, which can be regarded as the aggregation of voters' beliefs or preferences. They nominate %compete by nominating 
their ``best" candidates in elections and try to appeal mainly to their supporters. At the first sight, it seems to be difficult to foresee the result of an election especially when voters are strategic and different voting procedures and allocation rules could lead to totally different results. 
Duverger's law suggests \emph{plurality voting} is in favor of the two-party system~\cite{Duv54}.
Also, Dellis~\cite{Del2013} explained why a two-party system emerges under plurality voting and other voting procedures permitting truncated ballots which allow a voter to cast zero score for a single candidate. These motivated the investigation of the efficiency of a two-party system in~\cite{LLC2021}, in which the above mentioned issues in a micro scale were bypassed, and a macro-model was considered instead by formalizing the political competition between two parties as a non-cooperative game, which is called the \emph{two-party election game}. 

In the two-party election game, each party is regarded as a player who treats its candidates as pure strategies and its payoff is the expected utility for its supporters. The expectation comes from the uncertainty of winning or losing the election. The odds of winning an election for a candidate nominated by a party over another candidate nominated by the competing party are assumed to be related to two factors---the total benefits that it brings to the whole society, including the supporters and non-supporters, and those that its competitor brings to the whole society. Usually, a candidate nominated by a party responds and caters more to the needs and inclination of its party's supporters and less to those of the supporters of the other party. Naturally, a candidate then brings different utility to the supporters and to the non-supporters. It can be expected that a candidate never wins with lower probability if it brings more total utility to the whole society. We call such an election game with this property a \emph{monotone} two-party election game. 
\iffalse
We focused on pure strategies in the election game for the reason that, compared with \emph{mixed strategies} that are represented as a probability distribution over a subset of available actions, pure strategies are arguably more realistic for players to play in practical world. %From another point of view, the real-valued utility summarizes infinitely many possibilities how a candidate can benefit the voters. 
\fi


Suppose that the winning probability of a candidate against the opponent is calculated using either a linear function or the softmax function. %, where the former is linear in the difference between the total utility brought by the two competing candidates and the latter is the ratio of one exponential normalized total utility to the sum of both. 
Based upon the game setting as above, in~\cite{LLC2021} it has been proved that a pure-strategy Nash equilibrium (PSNE) always exists in the monotone two-party election game under a mild condition---\emph{egoism}, which states that any candidate benefits its party's supporters more than any candidate from the competing party does. The existence of equilibria can bring positive implications. For decades, Nash equilibrium has been known as a kind of solution concept which provides a more predictable outcome of a non-cooperative game modeling behaviors of strategic players. Although a Nash equilibrium always exists in a finite game when mixed strategies are considered~\cite{nash_1950,nash_1951}, PSNE is not guaranteed to exist~\cite{OR94}. 
%Using price of anarchy~\cite{KP09} as the inefficiency measure of an equilibrium, 
Lin \textit{et al}.~\cite{LLC2021} showed the monotone egoistic two-party election game has a constantly bounded price of anarchy. This shows in some sense that a game between two parties in candidate nomination for an election benefits the people with a social welfare at most constantly far from the optimum. The existence guarantee of pure-strategy Nash equilibria and bounded price of anarchy for the monotone egoistic two-party election game suggest that a two-party system is ``good" from these perspectives. On the other hand, however, when the egoistic property is not satisfied, the game might not have any PSNE, and the price of anarchy can be unbounded, hence, the game can be extremely inefficient. However, it is still unclear how good or bad the monotone election game of more than two parties is with and without egoism.


%=====================================================================
\subsection{Our Contributions}
\label{subsec:contribution}
%=====================================================================

In this work, we generalize Lin \textit{et al}.'s work~\cite{LLC2021} to deal with $m\geq 2$ parties. %Roughly speaking, we would like to know \emph{how good or bad a system of two or more parties can be}. 
Throughout this paper, we consider the monotone egoistic election game and omit ``monotone" when the context is clear. 
%Note that, without the egoism guarantee, the two-party election game may have no PSNE and it can be extremely inefficient in terms of the unbounded price of anarchy~\cite{LLC2021}. On the other hand, the egoistic two-party election game always has a PSNE when the winning probability of a candidate is calculated using the softmax function, and its price of anarchy is constantly bounded~\cite{LLC2021}.  
%As we have shown in~\cite{LLC2021} that, without the egoism guarantee, the two-party election game may have no PSNE and it can be extremely inefficient in terms of the unbounded price of anarchy, in this work we focus on the egoistic election game for two or more parties. 
%Briefly, we call such a generalized game the \emph{egoistic election game}, and 
For the egoistic election game of two or more parties, we aim at investigating the following questions:
\begin{enumerate}
    \item Does the egoistic election game using the softmax function to calculate the winning probability of a candidate against its opponents always have a PSNE, even for three or more parties? Does the function for computing such a winning probability matter? 
    \item What is the computational complexity of computing a PSNE of the egoistic election game when two or more parties are involved in general? 
    \item What is the price of anarchy of the egoistic election game of two or more parties? Is it still constantly bounded?
    \item Is there any incentive for the parties to form a coalition? 
\end{enumerate}
    
\paragraph{Our answers to the above questions are summarized as follows.} 

\begin{enumerate}
    \item We give examples to confirm that a PSNE does not always exist in the egoistic election game of three or more parties even using the softmax function to calculate the winning probability of a candidate. Furthermore, there exist instances in which no PSNE exists even if it satisfies a stronger notion of egoism. Nevertheless, we propose two sufficient conditions for the egoistic election game to have a PSNE. 
    \item We prove that to compute a PSNE of the egoistic election game is {\sf NP}-hard in the general form representation. Based on the two sufficient conditions, we identify two %natural 
    /parameters %, the \emph{nominating depth} and \emph{number of irresolute parties}, 
    of the egoistic election game, and propose a fixed-parameter tractable algorithm to find a PSNE of the game if it exists. Namely, a PSNE of the egoistic election game can be found in time polynomial in the number of parties and number of candidates in each party if the two proposed parameters are as small as constants. Moreover, we show that there exists a na\"{i}ve algorithm that can compute a $(1+e)$-approximate PSNE in constant time when the winning probability of a party is determined by a softmax function.
    \item Perhaps surprisingly, we prove that the price of anarchy of the egoistic election game is upper bounded by the number of competing parties~$m$. 
    %when the odds of winning the election for a party is calculated by any \emph{monotone} function, which includes the hardmax function, the natural function and the softmax function, etc. 
    This upper bound is arguably tight, as we show that it is tight for the game using the hardmax function to compute the winning probability of a candidate. This, to a certain degree, suggests that the social welfare deteriorates when there are more competing parties participating in the election campaign. Our work provides an explanation alternative to Duverger's law that why the two-party system is prevalent in democratic countries. As shown in~\cite{LLC2021} and this work as well, the price of anarchy of the egoistic election game is lower bounded by~2 for the two-party election game using either the hardmax function or the softmax function to compute the winning probability of a candidate, our results in this study imply that the price of anarchy bound for the egoistic two-party election game is tight. 
    \item We also remark that coalition of party players with the strongly egoism guarantee, in which a composite-like candidate for a coalition is considered, makes the game collapse on the non-cooperative egoistic election game setting. This observation reveals that if the election game is cooperative and strongly egoistic, it will be efficient in the sense of decreasing price of anarchy. We also show that any party player gains no less payoff by joining a coalition. 
\end{enumerate}

 \subsubsection*{Discussion of our work.} %\textcolor{red}{(Feel free to move this paragraph back to the related work if it is better.)}
%==============================================================
Most of the literature on the voting theory and the game of elections focuses on the voters' behavior on a \emph{micro}-level and outcomes by various election rules, which proceed with either one or multiple rounds and aim to bring one or multiple winners. Voters can be strategic and have different preferences for the candidates. Their behaviors can be even affected by the other voters (e.g., see~\cite{AFS19,BFM18}) and also dependent on the social choice rule, such as the design of ballots, rounds of selection, and the allocation rule, etc. In this work, we bypass the above involved factors in an election and focus on a \emph{macro-level} analysis instead. By introducing uncertainty in the competition between candidates participating in the competition, the payoff as the expected utility for the supporters of a party can also be regarded as the sum of fractional social welfare one party's supporters can get from all the competing candidates. 
Moreover, the monotonicity of the game is arguably natural in the sense that a party can attract more voters when it nominates a candidate who benefits the voters more. 
%Moreover, we consider the winning probability computed by a monotone function, which is arguably natural in the sense that a party can attract more voters when it nominates a candidate who benefits the voters more. 
%Through several monotone winning probability functions as illustrating examples for the nonexistence of PSNE, eventually 
By the monotonicity of the game, we show that the price of anarchy is upper bounded by the number of parties. It deserves to be noted that this upper bound holds for \emph{any} function computing the winning probability by which the game is monotone. Our work provides an alternative perspective and a simpler evidence on the inefficiency of multiple parties competition, which complements previous relevant work. 

\subsubsection*{Organization of this paper.} We briefly survey related work in Sect.~\ref{subsec:related_work}. Preliminaries are given in Sect.~\ref{sec:preliminaries}. In Sect.~\ref{sec:hardness_no_PSNE_examples}, we investigate the egoistic election game through examples to show that the hardmax function for winning probability computation leads the game to always have a PSNE, while there exist instances of three parties in which there is no PSNE when the %natural function or the 
softmax function is adopted. Then, we show that it is {\sf NP}-complete to determine if the egoistic election game has a PSNE. In Sect.~\ref{sec:algos}, %\ref{sec:two_sufficient_conditions_PSNE}, 
%we propose two sufficient conditions for the egoistic election game to have a PSNE. Then, 
we first propose a fixed-parameter tractable algorithm to compute a PSNE of the game in the general form representation when the game is parameterized by two natural parameters. Second, we provide a na\"{i}ve algorithm which can compute a $(1+e)$-approximate PSNE in constant time. In Sect.~\ref{sec:PoA}, we show the upper bounds on the price of anarchy of the egoistic election game. Coalition of party players with the strongly egoism guarantee is also discussed therein. 
Concluding remarks will be discussed in Sect.~\ref{sec:future}.


%=====================================================================
\subsection{Related work}
\label{subsec:related_work}
%=====================================================================



\subsubsection*{On Duverger's law.} Duverger's law that suggests plurality voting in favor of the two-party system~\cite{Duv54}, and this can be explained either by the strategic 
behavior of the voters~\cite{Del2013,Fed92,Fey97,MW93,Pal89} or that of the 
candidates~\cite{Cal05,CW07,Pal84,Web92}. The latter  considers models where two selected candidates face  the third candidate as a potential threat. Dellis~\cite{Del2013} explained (with mild assumptions on voters'
preferences) why a two-party system emerges under plurality voting and other voting procedures permitting truncated ballots. Nevertheless, why the two candidates are selected is not discussed.


\subsubsection*{On spatial theory of voting.} Most of the works on equilibria of a political competition are mainly based on {\em Spatial Theory of Voting}~\cite{Hot29,Dow57,LRL2007,Pal84,Web92}, 
which can be traced back to~\cite{Hot29}. 
In such settings, there are two parties and voters with single-peaked preferences over a unidimensional metric space. Each party 
chooses a kind of ``policy" that is as close as possible to voters' preferences. When the policy space is unidimensional, the Spatial Theory of Voting states that the parties' strategies would be determined by the median voter's preference. However, pure-strategy Nash equilibria may not exist for policies over a multi-dimensional space~\cite{Dug16}. 


\subsubsection*{On the Hotelling-Downs model.} 
Hotelling-Downs model~\cite{Hot29} originally considers the problem that two strategical ice cream vendors along a stretch of beach try to attract as many customers as possible by placing themselves. This framework can be extended to the setting that two parties nominate their candidates on a political spectrum. The model has variations involving \emph{multiple} agents with \emph{restricted options}, and 
this is in line with the competition of multiple parties with 
a few nominees as possible candidates. For the variation of Hotelling-Downs model as such, Harrenstein et al.~\cite{HLST21} show that computing a Nash equilibrium is 
{\sf NP}-complete in general but can be done in linear time when there are only two competing parties. Sabato et al.~\cite{SORR17} consider \emph{real candidacy games}, in which competing agents select intervals on the real line and then the outcome of the competition follows a given social choice rule. They establish conditions for existence of a Nash equilibrium, yet the computation complexity is not discussed. For such Hotelling-Downs-like models, players only care about winning, while in our work, party players focus on the expected utility of their supporters. 


\subsubsection*{Other work modelling an election as a game.} Ding and Lin~\cite{DL2014} considered open list proportional representation as the election mechanism which has been used in Europe elections for parliament seats. Each voter is given a set of lists of candidates to vote and exactly one list will be cast. The mechanism proceeds in two rounds to compute the winners. They formulate the election of exactly two parties as a two player zero-sum game and show that the game always has a PSNE while it is {\sf NP}-hard to compute it. %The hardness comes from deciding the best way to for the lists for each party such that the outcome as the number of seats a party can win is maximized. 
The setting in Laslier's work~\cite{L00a,L00b} is close to our work. Two parties are considered as two players each of which provides a finite set of alternatives for the voters. Yet, a party's strategy is viewed as a ``mixed one" instead, that is, a non-spatial alternative is identified by a fraction of the voters. With standard analysis the mixed-strategy Nash equilibrium is guaranteed to exist. Compared with our work, the winning probabilities, the expected utility of each party as well as the price of anarchy are not considered in~\cite{L00a,L00b}. 


\iffalse
\paragraph{On distortion of voting rules.} Procaccia and Rosenschein~\cite{PR2006} introduced the notion ``distortion" which resembles 
the price of anarchy, while the latter is used in games of strategic players. 
The distortion measures the inefficiency when a social choice rule (e.g., voting) is applied. Generally, voters with cardinal preferences~\cite{CP2011,PR2006}) or metric preferences~\cite{ABP2015,AP2016,CDK2017}) are considered herein. As an embedding on a voter' ballot is allowed, Caragiannis and Procaccia~\cite{CP2011} 
discussed the distortion of social choice when each voter's ballot receives an embedding, which maps 
the preference to the output ballot. Cheng et al.~\cite{CDK2017} focused 
on the distribution of voters as well as the candidates of parties and they justified that the expected distortion is small when the candidates are drawn from the same distribution as the voters.
\fi


\iffalse
\subsubsection*{On dynamics of a competition.} A political competition can be considered as a simultaneous one-shot game, similar to what we consider in this work, though it can also be viewed as a dynamic process in multiple rounds. In~\cite{KK09}, each time a player competes by investing some of its budget or resource in a component battle to collect a value if it wins. Players fight in multiple battles, and their budgets are consumed over time. In a two-player zero-sum version of such games, a strategic player needs to make adequate sequential actions to win the contest against dynamic competition over time~\cite{CCH2018}.
\fi


%==============================================================


%%%%%%%%%%%%%%%%%%%%%%%%%%%%%%%%%%%%%%%%%%%%%%%%%%%%%%%%%%%%%%%%%%%%%%
\section{Preliminaries}
\label{sec:preliminaries}
%%%%%%%%%%%%%%%%%%%%%%%%%%%%%%%%%%%%%%%%%%%%%%%%%%%%%%%%%%%%%%%%%%%%%%

For an integer $k>0$, let $[k]$ denote the set $\{1,2,\ldots,k\}$. We assume that the society consists of voters, and each voter is a supporter of one of the $m\geq 2$ parties $\mathcal{P}_1,\mathcal{P}_2,\ldots,\mathcal{P}_m$. These $m$ parties compete in an election campaign. Each party $\mathcal{P}_i$, having $n_i\geq 2$ candidates $x_{i,1}, x_{i,2}, \ldots, x_{i,n_i}$ for each $i\in [m]$, has to designate one candidate to participate in the election. Let $n = \max_{i\in [m]} n_i$ denote the maximum number of candidates in a party. Let $u_j(x_{i,s})$ denote the \emph{utility} that party $\mathcal{P}_j$'s supporters can get when candidate $x_{i,s}$ is elected, for $i,j\in [m], s\in[n_{i}]$. For each $i\in [m]$ and $s\in [n_i]$, we denote by $u(x_{i,s}) := \sum_{j\in [m]}u_j(x_{i,s})$ as the \emph{social utility} candidate $x_{i,s}$ can bring to all the voters. Assume that the social utility is nonnegative and bounded, specifically, we assume $u(x_{i,s})\in [0, \beta]$ for some real $\beta\geq 1$, for each $i\in [m], s\in [n_i]$.
Assume that candidates in each party are 
sorted according to the utility for its party's supporters. 
Namely, we assume that $u_i(x_{i,1})\geq u_i(x_{i,2})\geq \ldots \geq u_i(x_{i,n_i})$ for each $i\in [m]$. 
%W.l.o.g., we assume that $u_1(x_{i,1})\geq u_2(x_{2,1})\geq \ldots \geq u_m(x_{m,1})$ to break the symmetry. 


The election competition is viewed as a game of $m$ players such that each party corresponds to a player, called a \emph{party player}. With a slight abuse of notation, $\mathcal{P}_i$ also denotes the party player with respect to party $\mathcal{P}_i$. Each party player $\mathcal{P}_i$, $i\in [m]$, has $n_i$ \emph{pure strategies}, each of which is a candidate selected to participate in the election. We consider an assumption as a desired \emph{monotone property} that \emph{a party wins the election with higher or equal odds if it selects a candidate with a higher social utility}. %We call it the \emph{monotone property}. 
Moreover, the odds of winning then depend on the social utility brought by the candidates. Suppose that $\mathcal{P}_i$ designates candidates $x_{i,s_i}$ for $i\in [m]$ and let $\mathbf{s} = (x_{1,s_1},x_{2,s_2},\ldots,x_{m,s_m})$ (or simply $(s_1,s_2,\ldots,s_m)$ when it is clear from the context) be the \emph{profile} of the designated candidates of all the party players. Let $p_{i,\mathbf{s}}$ denote the probability of party player~$\mathcal{P}_i$ winning the election with respect to profile~$\mathbf{s}$. 
As defined and discussed in~\cite{LLC2021}, we can concretely formulate $p_{i,\mathbf{s}}$ and preserve the monotone property as follows.%\footnote{In~\cite{LLC2021}, the linear model based on the dueling bandit setting~\cite{AJK2014} is applied for the election game of exactly two party players. As it is designed for pairwise comparison, we do not consider it for general $m\geq 2$ in this work.}. 
%Based upon them, we can formally define the payoff for each party player in an election game.  
\begin{itemize}
\iffalse
\item Linear link model~\cite{AJK2014}: \[p_{i,\mathbf{s}} := \frac{1 +(u(A_i)-u(B_j))/b}{2}.\] 
    \begin{itemize}
        \item This is inspired by the exploration method used in the multi-armed bandit problem~\cite{Kul00} and the 
            probabilistic comparison used in the dueling bandits problem~\cite{AJK2014,YBKJ2012}. The winning odds is then regarded as 
            a \emph{linear} function of the \emph{difference} between the social utility brought by candidates $A_i$ and $B_j$.
    \end{itemize}
    \vspace{6pt}
\fi
\item The hardmax function: 
\[
p_{i,\mathbf{s}} = \left\{\begin{array}{ll}
1 & \mbox{ if } i = \min\{\argmax_{j\in [m]} u(x_{j,s_j})\}\\
0 & \mbox{ otherwise.}
\end{array}\right.
\]
    \begin{itemize}
        \item The hardmax function simply allocates all probability mass to the candidate of maximum social utility. If the size of
        $\argmax_{j\in [m]} u(x_{j,s_j})$ is larger than~1, then all the probability mass is allocated to the one with minimum index in~$\argmax_{j\in [m]} u(x_{j,s_j})$. 
    \end{itemize}
\item The softmax function~\cite{Kul00,SB98}: 
\vspace{-8pt}
	\[p_{i,\mathbf{s}} := \frac{e^{u(x_{i,s_i})/\beta}}{\sum_{j\in [m]} e^{u(x_{j,s_j})/\beta}}.\]
    \begin{itemize}
	    \item The softmax function %is extensively used in machine learning to normalize the output into a probability distribution. It 
        is formulated as the ratio of one exponential normalized social utility to the sum of both. Clearly, it is \emph{nonlinear} in the social utility $u(x_{i,s_i})$  and produces a probability strictly in $(0, 1)$.
	\end{itemize}
 \vspace{5pt}
\iffalse
\item The natural function~\cite{Bra54,YBKJ2012,DWH2020}: \[p_{i,\mathbf{s}} := \frac{u(x_{i,s_i})}{\sum_{j\in [m]} u(x_{j,s_j})}.\] 
    \begin{itemize}
    \item We treat the probability $p_{i,\mathbf{s}}$ as ratio of the social utility brought by a candidate to the sum of the social utility brought by all candidates.\footnote{We assume that $\sum_{j\in [m]} u(x_{j,s_j})>0$ for the natural function.} This function is \emph{linear} in the social utility $u(x_{i,s_i})$ and produces a probability in~$[0, 1]$
    \end{itemize}
\fi
\end{itemize}


Note that the monotone property guarantees that $p_{i,(s_i',\mathbf{s}_{-i})}\geq p_{i,\mathbf{s}}$ for $(s'_i,\mathbf{s}_{-i}) = (s_1,s_2,\ldots,s'_i,\ldots,s_m)$ whenever $u(x_{i,s_i'})\geq u(x_{i,s_i})$ %(i.e., party player~$i$ unilaterally deviates its strategy to $x_{i,s_i'}$ only when the latter brings larger social utility). 
We call the functions calculating the winning probability of a candidate against the others \emph{WP functions}. The WP functions by which the egoistic election game is monotone are called \emph{monotone WP functions}. The hardmax function is monotone since raising the social utility never makes a party player lose. By Lemma~\ref{lem:fractional_monotone}, we know that the softmax %and natural functions are 
function is also a monotone WP function. Throughout this paper, we consider monotone WP functions unless otherwise specified. 


\begin{lemma}\label{lem:fractional_monotone}
Let $r, s > 0$ be two positive real numbers such that $r<s$. Then, for any $d>0$, $r/s < (r+d)/(s+d)$. 
\end{lemma}
\iffalse
\begin{proof}
Let $r, s > 0$ be two positive real numbers such that $r<s$. Then $r/s - (r+d)/(s+d) = d(r-s)/(s^2+sd)$. Clearly, $r/s - (r+d)/(s+d) < 0$ if $r < s$. 
%The lemma is then proved. 
\qed
\end{proof}
\fi

The payoff of party player $\mathcal{P}_i$ given the profile $\mathbf{s}$ is denoted by $r_{i,\mathbf{s}}$, which is the \emph{expected utility} that party~$\mathcal{P}_i$'s supporters obtain in~$\mathbf{s}$. 
Namely, 
%\begin{equation*}
$r_{i}(\mathbf{s}) = \sum_{j\in [m]} p_{j,\mathbf{s}}u_i(x_{j,s_j})$,
%\vspace{-3pt}
%\end{equation*} 
which can be computed in $O(m)$ time for each~$i$. 
We define the {\em social welfare} of the profile $\mathbf{s}$ as $SW(\mathbf{s}) = \sum_{i\in [m]} r_{i}(\mathbf{s})$. More explicitly, we have $SW(\mathbf{s}) = \sum_{i\in [m]}\sum_{j\in [m]} p_{j,\mathbf{s}}u_i(x_{j,s_j}) = \sum_{j\in[m]}p_{j,\mathbf{s}} \sum_{i\in [m]} u_i(x_{j,s_j}) = \sum_{j\in[m]}p_{j,\mathbf{s}} u(x_{j,s_j})$. We say that 
a profile $\mathbf{s}$ is a {\em pure-strategy Nash equilibrium} (PSNE) if $r_{i}(s'_{i},\mathbf{s}_{-i})\leq r_{i}(\mathbf{s})$ for any~$s'_i$, where $\mathbf{s}_{-i}$ denotes the profile without party player $\mathcal{P}_i$'s strategy. That is, in $\mathbf{s}$, none of the party players has the incentive to deviate from its current strategy. The {\em (pure) price of anarchy} (PoA) of the game~$\mathcal{G}$ is defined as 
\vspace{-7pt}
\[
\mbox{PoA}(\mathcal{G})=\frac{SW(\mathbf{s}^*)}{SW(\hat{\mathbf{s}})} = 
\frac{\sum_{j\in [m]} r_{j}(\mathbf{s}^*)}{\sum_{j\in [m]} r_{j}(\hat{\mathbf{s}})},
\]
where $\mathbf{s}^* = \argmax_{\mathbf{s}\in\prod_{i\in [m]}[n_i]} SW(\mathbf{s})$ is the {\em optimal profile}, 
which has the best (i.e., highest) social welfare among all possible profiles, and $$\hat{\mathbf{s}} = \argmin_{\mathbf{s}\in\prod_{i\in [m]} [n_i], \,\mathbf{s}\mbox{\tiny \;is a PSNE}} SW(\mathbf{s})$$
%$\hat{\mathbf{s}} = \argmin\limits_{\genatop{\mathbf{s}\in\prod_{i\in [m]} [n_i]}{\mathbf{s}\mbox{\tiny \;is a PSNE for }\mathcal{G}}} SW(\mathbf{s})$ 
is the PSNE with the worst (i.e., lowest) social welfare\footnote{For $\hat{\mathbf{s}}$ 
being the PSNE with the best (i.e., highest) social welfare, the \emph{(pure) price of stability} (PoS) of~$\mathcal{G}$ is defined as~$SW(\mathbf{s}^*)/SW(\hat{\mathbf{s}})$.}. 
Note that an egoistic election game may not necessarily have a PSNE so an upper bound on the PoA is defined over only games with PSNE, i.e., $\max_{\mathcal{G}\mbox{\scriptsize\, has a PSNE}}\mbox{PoA}(\mathcal{G})$, accordingly. 
\iffalse
In an election, probabilistically (instead of deterministically) nominating a candidate or imaging repeated nominations of candidates is almost infeasible in the reality. Thus, adopting PSNE as the equilibrium concept \textcolor{black}{best reflects the situation of an election.} Nonetheless, it is theoretically natural for one to consider mixed Nash equilibria or other more general notions of equilibria as solution concepts where the existence of equilibria is always guaranteed. It may take other analysis frameworks such as~\cite{KM2015,R2009} to study their corresponding price-of-anarchy bounds.
\fi
\paragraph{Remark.} We regard a party, which consists of its supporters, as a collective concept. A candidate is nominated by the supporters of the party to compete in an election campaign. Supporters of a party can benefit not only from the candidate nominated for them but also from that from the competing party. Since the winning candidate serves for the whole society, that is, for \emph{all} the voters, 
%not only for its supporters, 
we consider the payoff of the party as the expected utility the supporters can get so that we can formulate the payoff whenever the party wins or loses well. The following properties will be used throughout this paper. 

\begin{definition}\label{defn:egoistic}
We call the election game {\em egoistic} if $u_i(x_{i,s_i})> u_i(x_{j,s_j})$ for all $i\in [m], s_i\in\!~[n_i], s_j\in\!~[n_j]$.
\end{definition}
\subsubsection*{Remark.} This property guarantees that \emph{any candidate benefits its supporters more than any candidate from any other competing party}. Such a property is natural and reasonable. Indeed, as a party is a collective concept which consists of its supporters, a candidate of the party is expected to be more favorable than that of a competing party for the supporters. However, it is not always the case in real world. For example, a smaller party that has fewer supporters and less resource is difficult to compete with a larger party, so a candidate of the larger party is possibly more beneficial to the supporters of the smaller party than that of a smaller party. 
Below we define a stronger version of egoism.

\begin{definition}\label{defn:strong_egoistic}
%We say that 
The election game is {\em strongly egoistic} if $u_i(x_{i,s_i})> \break \sum_{j\in [m]\setminus\{i\}} \max_{s_j\in [n_j]}u_i(x_{j,s_j})$ for all $i\in [m], s_i\in [n_i]$.
\end{definition}
\paragraph{Remark.} The strongly egoistic property states that any candidate benefits its supporters more than all the other candidates nominated from the competing parties jointly do. 
%sum of the most beneficial candidates from all the other competing parties.

\begin{definition}%[Strategy Domination]
\label{defn:strategy_domination}
Given a profile $\mathbf{s}$, for $i\in [m]$, %we say that 
Strategy $x_{i,s_i}$ {\em weakly surpasses} $x_{i,s_i'}$ if $s_i < s_i'$ (i.e., $u_i(x_{i,s_i})\geq u_i(x_{i,s_i'})$) and 
$u(x_{i,s_i})\geq u(x_{i,s_i'})$. %We say that 
Strategy $x_{i,s_i}$ {\em surpasses} $x_{i,s_i'}$ if $x_{i,s_i}$ weakly surpasses $x_{i,s_i'}$ and either $p_{i,\mathbf{s}} > p_{i,(s_i',\mathbf{s}_{-i})}$ or $u_i(x_{i,s})>u_i(x_{i,s'})$. 
\end{definition}
By Definition~\ref{defn:strategy_domination}, a candidate surpasses another one in the same party if it brings more (resp., no less) utility to the supporters and has no lower (resp., higher) winning probability. 
\iffalse
\paragraph{Remark.}  
By the monotone property, we have $p_{i,(s_i,\mathbf{s}_{-i})}\geq p_{i,(s_i',\mathbf{s}_{-i})}$ if and only if $u(x_{i,s_i})\geq u(x_{i,s_{i'}})$.
\fi


%%%%%%%%%%%%%%%%%%%%%%%%%%%%%%%%%%%%%%%%%%%%%%%%%%%%%%%%%%%%%%%%%%%%%%
\section{Hardness and Counterexamples}
\label{sec:hardness_no_PSNE_examples}
%%%%%%%%%%%%%%%%%%%%%%%%%%%%%%%%%%%%%%%%%%%%%%%%%%%%%%%%%%%%%%%%%%%%%%


We first note that when the hardmax function is adopted as the monotone WP function, the egoistic election game always has a PSNE. With ties broken arbitrarily, the maximum of a finite set of numbers always exists and hence the candidate $x_{i^*,s_{i^*}}$, where $i^* := \argmax_{i\in [m]} u(x_{i,s_i})$ with respect to the profile $\mathbf{s}$, wins with probability~1 by the definition of hardmax. Therefore, consider the candidate $(i^*,s_{i^*}^*) := \argmax_{i\in [m], s_i\in [n_i]} u(x_{i,s_i})$ with the maximum social utility and let $I = \{s\in [n_{i^*}]: u(x_{i^*,s})\geq u(x_{j,s_j})\mbox{ for all }j\in [m]\setminus \{i^*\}, s_j\in [n_j]\}$ collect all the candidates in party $i^*$ which bring more social utility than any one in the other parties. Clearly, $I$ is not empty since $s_{i^*}^*\in I$. Next, choose the minimum index  $s_{i^*}^{**}$ in~$I$, which brings the maximum utility for party $i^*$'s supporters, then we have that any profile $\mathbf{s}$ with party player $i^*$ choosing $s_{i^*}^{**}$ is a PSNE. 
Indeed, any other party player $\mathcal{P}_j$ for $j\neq i^*$ has no incentive to deviate its strategy because the payoff can never be better off. As for party player $i^*$, by the egoistic property we know that it gets less utility from the other party than that from its own candidates whenever it loses, so choosing $s_{i^*}^{**}$ guarantees his maximum possible reward. 
However, such a PSNE in this case is not necessarily the optimal profile. For example, consider the instance in Table~\ref{tab:AlwaysPSNE_HM}. The social welfare of the PSNE is~$50$ which is only half of the optimum.
%, though that it of the optimal profile is~$100$, which is twice higher.


%\textcolor{orange}{
In~\cite{LLC2021}, it has been shown that the egoistic two-party election game always has a PSNE if a linear function or the softmax function is adopted as the monotone WP function. %\footnote{Though it is not the case for the natural function.}. 
One might be curious about whether the egoistic property is sufficient for such an election game of \emph{three or more parties} to possess a PSNE. Unfortunately, %through program simulations 
we find game instances as counterexamples, which imply that the egoistic election game of three or more parties does not always have a PSNE (see %Table~\ref{tab:NoPSNE_BT} and 
Table~\ref{tab:NoPSNE_softmax}). %\footnote{Even for a strongly egoistic election game, such a counterexample still exists (see %Table~\ref{tab:NoPSNE_BT_strong} and Table~\ref{tab:NoPSNE_softmax_strong} in 
%Appendix~B).}.  
Indeed, party player $\mathcal{P}_3$ has a dominant strategy $x_{3,1}$ since $r_{3,(i,j,1)}>r_{3,(i,j,2)}$ for any $i,j\in \{1,2\}$. Then, the game instance resembles a two-party election game instance yet there is a deviation-cycle, which shows that a PSNE does not exist (see Fig.~\ref{fig:deviation_example}). 
Note that the two-party election game using softmax function as the monotone WP function has been guaranteed to have a PSNE~\cite{LLC2021}. Unlike the two-party election game case, the increase of winning probability mass due to an unilateral strategy deviation of~$\mathcal{P}_1$ contributes to the decrease of winning probability mass of both~$\mathcal{P}_2$ and $\mathcal{P}_3$. Hence, the analysis in~\cite{LLC2021} does not apply herein.%}
%This suggests that the election game of more than two parties is more unpredictable. 


\begin{table}[ht]
\begin{center}
\begin{tabular}[c]{ l l l | l l l | l l l }
	%\centering
	{\footnotesize $u_1(x_{1,i})$} & {\footnotesize $u_2(x_{1,i})$} & {\footnotesize $u_3(x_{1,i})$} & 
    {\footnotesize $u_1(x_{2,i})$} & {\footnotesize $u_2(x_{2,i})$} & {\footnotesize $u_3(x_{2,i})$} & 
    {\footnotesize $u_1(x_{3,i})$} & {\footnotesize $u_2(x_{3,i})$} & {\footnotesize $u_3(x_{3,i})$}
        \\
	\hline
	50  &  0   &  0  &  15 &  31 &  0 &  10 &   10 &  24\\
        49 &  29 &  22&  16 &  30 &  0 &  10 &  10 &  23 \\
	\hline
\end{tabular}
\vspace{7pt}\\
\begin{tabular}[c]{ l l l | l l l}
	\centering
	%\multicolumn{2}{l}{payoff matrix} \vspace{7pt}\\
	%\hline
	$r_{1,(1,1,1)}$& $r_{2,(1,1,1)}$ & $r_{3,(1,1,1)}$
     &   $r_{1,(1,1,2)}$& $r_{2,(1,1,2)}$& $r_{3,(1,1,2)}$\\
	\hline
	$r_{1,(1,2,1)}$& $r_{2,(1,2,1)}$& $r_{3,(1,2,1)}$
     &  $r_{1,(1,2,2)}$& $r_{2,(1,2,2)}$& $r_{3,(1,2,2)}$\\
	%\hline
\end{tabular}
$=$
\begin{tabular}[c]{ l l l | l l l }
	%\multicolumn{2}{c}{} \vspace{7pt}\\
	\centering
	% & \\
	%\hline
	50&  0 &  0 & 50  &  0 &   0\\
	\hline
	50 &  0 &  0  & 50 &  0 &  0\\
	%\hline
\end{tabular}
\vspace{7pt}\\
\begin{tabular}[c]{ l l l | l l l}
	\centering
	%\multicolumn{2}{l}{payoff matrix} \vspace{7pt}\\
	%\hline
	$r_{1,(2,1,1)}$& $r_{2,(2,1,1)}$& $r_{3,(2,1,1)}$
     & $r_{1,(2,1,2)}$& $r_{2,(2,1,2)}$&  $r_{3,(2,1,2)}$\\
	\hline
	$r_{1,(2,2,1)}$& $r_{2,(2,2,1)}$& $r_{3,(2,2,1)}$
     & $r_{1,(2,2,2)}$& $r_{2,(2,2,2)}$& $r_{3,(2,2,2)}$\\
	%\hline
\end{tabular}
$=$
\begin{tabular}[c]{ l l l | l l l }
	%\multicolumn{2}{c}{} \vspace{7pt}\\
	\centering
	% & \\
	%\hline
	49&  29&  22 &  49&  29&  22\\
	\hline
	49&  29&  22& 49&  29&  22\\
	%\hline
\end{tabular}
\vspace{10pt}
\caption{An egoistic election game instance of three parties that always has a PSNE ($\beta=100, n_i=2$ for $i\in\{1,2,3\}$). The hardmax function is used as the monotone WP function. For example, here $u_{x_{1,2}} = 49+29+22 = 100$, $u_{x_{2,2}} = 16+30+0 = 46$, and $u_{x_{3,2}} = 10+10+23 = 43$, so for profile $(2,2,2)$ the winner is $x_{1,2}$ and the rewards of the three party players are $49$, $29$ and $22$ respectively, which result in social welfare $49+29+22 = 100$. By checking all the four profiles we know the optimal profile has social welfare~$100$. It is easy to see that profiles $(1,1,1)$, $(1,1,2)$, $(1,2,1)$ and $(1,2,2)$ are all PSNE.}%, however, each of them has the social welfare 50, which is only half of that of the optimal profile.}
\label{tab:AlwaysPSNE_HM}
\vspace{-20pt}
\end{center}
\end{table}


\begin{table}[ht]
\begin{center}
\begin{tabular}[c]{ l l l | l l l | l l l }
	%\centering
	{\footnotesize $u_1(x_{1,i})$}& {\footnotesize $u_2(x_{1,i})$}& {\footnotesize $u_3(x_{1,i})$} & 
        {\footnotesize $u_1(x_{2,i})$}& {\footnotesize $u_2(x_{2,i})$}& {\footnotesize $u_3(x_{2,i})$} & 
        {\footnotesize $u_1(x_{3,i})$}& {\footnotesize $u_2(x_{3,i})$}& {\footnotesize $u_3(x_{3,i})$}
        \\
	\hline
	29  &  4   &  21  &  23  &  59  &  7  &  8  &  32  &  54 \\
        27  &  43  &  3 &  3  &  57  &  38  &  20  &  13  &  53 \\
	\hline
\end{tabular}
\vspace{7pt}\\
\begin{tabular}[c]{ l l l | l l l}
	\centering
	%\multicolumn{2}{l}{payoff matrix} \vspace{7pt}\\
	%\hline
	$r_{1,(1,1,1)}$ & $r_{2,(1,1,1)}$ & $r_{3,(1,1,1)}$
     & $r_{1,(1,1,2)}$ & $r_{2,(1,1,2)}$ & $r_{3,(1,1,2)}$\\
	\hline
	$r_{1,(1,2,1)}$ & $r_{2,(1,2,1)}$ & $r_{3,(1,2,1)}$ 
     & $r_{1,(1,2,2)}$ & $r_{2,(1,2,2)}$ & $r_{3,(1,2,2)}$\\
	%\hline
\end{tabular}
$\approx$
\begin{tabular}[c]{ l l l | l l l }
	%\multicolumn{2}{c}{} \vspace{7pt}\\
	\centering
	% & \\
	%\hline
	18.81 \; &  34.64  \;  &  28.51  \; & \, 23.49  \; &  27.82 \; &   27.38\\
	\hline
	11.27 \; &  34.67  \; &  39.70  \; & \, 15.57  \; &  28.09 \; &  38.93\\
	%\hline
\end{tabular}
\vspace{7pt}\\
\begin{tabular}[c]{ l l l | l l l}
	\centering
	%\multicolumn{2}{l}{payoff matrix} \vspace{7pt}\\
	%\hline
	$r_{1,(2,1,1)}$ & $r_{2,(2,1,1)}$ & $r_{3,(2,1,1)}$ 
     & $r_{1,(2,1,2)}$ & $r_{2,(2,1,2)}$ &  $r_{3,(2,1,2)}$\\
	\hline
	$r_{1,(2,2,1)}$ & $r_{2,(2,2,1)}$ & $r_{3,(2,2,1)}$ 
     & $r_{1,(2,2,2)}$ & $r_{2,(2,2,2)}$ & $r_{3,(2,2,2)}$\\
	%\hline
\end{tabular}
$\approx$
\begin{tabular}[c]{ l l l | l l l }
	%\multicolumn{2}{c}{} \vspace{7pt}\\
	\centering
	% & \\
	%\hline
	18.74 \; &  44.53 \; &  22.84 \;  & \, 23.18  \; &   38.35 \; &  21.61\\
	\hline
	11.58 \; &  44.25 \; &  33.66 \; & \, 15.67 \; &  38.27 \; &  32.77\\
	%\hline
\end{tabular}
\vspace{12pt}
\caption{An egoistic election game instance of three parties that has no PSNE. The softmax function to compute winning probabilities is adopted ($\beta=100, n_i=2$ for $i\in\{1,2,3\}$). Here, for example, $u_{x_{1,1}} = 29+4+21=54$, $u_{x_{2,1}} = 23+59+7=89$, $u_{x_{3,1}}=8+32+54=94$. Then we have $p_{1,(1,1,1)} = e^{54/100}/(e^{54/100}+e^{89/100}+e^{94/100})\approx 0.2557$, $p_{2,(1,1,1)} = e^{89/100}/(e^{54/100}+e^{89/100}+e^{94/100})\approx 0.3629$, $p_{3,(1,1,1)}\approx 0.3814$. So $r_{1,1,1} = 29\cdot p_{1,(1,1,1)} + 23\cdot p_{2,(1,1,1)} + 8\cdot p_{3,(1,1,1)}\approx 18.81$. All the eight profiles are not PSNE. For example, consider profile $(1,1,1)$. Party player $\mathcal{P}_2$ has the incentive to change its strategy to~$x_{2,2}$ from~$x_{2,1}$ because $r_{2,(1,2,1)} = 34.67 > 34.64 = r_{2,(1,1,1)}$. This example is egoistic because $u_1(x_{1,2}) = 27 > \max\{u_1(x_{2,1}), u_1(x_{3,2})\} = 23$, $u_2(x_{2,2}) = 57 > \max\{u_2(x_{1,2}), u_2(x_{3,1})\} = 43$ and $u_3(x_{3,2}) = 53 > \max\{u_3(x_{1,1}), u_3(x_{2,2})\} = 38$.}
\label{tab:NoPSNE_softmax}
\vspace{-18pt}
\end{center}
\end{table}


\begin{figure}[ht]
    \centering
    \includegraphics[scale=0.40]{deviation_illustration.eps}
    \caption{Strategy deviations of the game instance in Table~\ref{tab:NoPSNE_softmax}. The arrows indicate the unilateral strategy deviations of~$\mathcal{P}_1$ and $\mathcal{P}_2$.}
    \label{fig:deviation_example}
\end{figure}


As an egoistic election game (of three or more parties) does not always possess a PSNE, it is reasonable to investigate the hardness of deciding whether an egoistic election game of $m\geq 2$ party players has a PSNE. As pointed out by~\`{A}lvarez et al.~\cite{AGS2011}, the computational complexity for such a decision problem of equilibrium existence %-finding problem 
can be dependent of %much different with respect to 
the degrees of succinctness of the input representation. It also worth noting that in Sect.~4 of~\cite{GGS05}, Gottlob et al. indicate that to determine if a game in the standard normal form\footnote{A game is in the standard normal form if the payoffs of players can be explicitly represented by a single table or matrix.} has a PSNE can be solved in logarithmic space, and hence in polynomial time although such a representation of a game instance is very space consuming. Instead, we consider the \emph{general form} representation~\cite{AGS2011}, which succinctly represents a game instance in a tuple of players, their action sets and a deterministic algorithm, %Turing machine, 
such that the payoffs are not required to be given explicitly. Note that to compute a PSNE of a game under the general form representation is {\sf NP}-hard~\cite{AGS2011}. We shall prove that to determine if the election game has a PSNE, for general $m\geq 2$ parties, is {\sf NP}-complete even with the egoistic property. 
Specifically, the input of the game can be represented by a tuple $(X_1,X_2,\ldots,X_m,f_{\mathcal{G}})$, such that for each $i\in [m]$, $X_i$ consists of a vector of~$m$ real values each of which is the utility for the corresponding supporters of each party, and $f_{\mathcal{G}}$ is the function that computes the payoff for each party player. Clearly, the input size is not tabular of size $\prod_{i\in m}n_i = \Omega(2^m)$, but is $O(nm^2)$ instead. %Indeed, as we can still generate a payoff matrix of the election game to have its standard normal form, however, not all of the entries are uncorrelated so such a game representation results in redundancy. 
The proof of the {\sf NP}-completeness of determining PSNE existence in an egoistic election game is built upon a reduction from the Satisfiability problem, which resembles the arguments in~\cite{AGS2011} but is revised elaborately to deal with the egoistic property and monotonicity, and is deferred to Appendix~\ref{subsec:NPC_proof}. 



\begin{theorem}\label{thm:NPC}
To determine if an egoistic election game of $m\geq 2$ parties has a PSNE is {\sf NP}-complete.
\end{theorem}
\iffalse
\begin{proof}
To check whether a profile is a PSNE requires at most $\sum_{i=1}^m n_i = O(mn)$ time by examining all unilateral deviations. Hence, the problem of determining if the egoistic election game has a PSNE is in \textsf{NP}. In the following, we prove the \textsf{NP}-hardness by a reduction from the satisfiability problem (SAT). Consider an arbitrary instance of SAT with a formula $F$ which contains a set of~$m$ variables $X = \{v_1,v_2,\ldots,v_m\}$. Let $a = (a_1,a_2,\ldots,a_m)$ be a truth assignment of~$X$ such that $a_i$ is the truth assignment of variable~$v_i$ for $i\in [m]$. Hence, $F$ is satisfiable if there exists an assignment $a$ of~$X$ such that $F(a) = \mbox{\tt True}$. 
Set the value of $\beta$ to be~$200$.
For $i\in \{1,2,\ldots, m-2\}$, we construct $m$ parties each of which has two candidates $x_{i,1}$ and $x_{i,2}$ such that $u_i(x_{i,1}) = u_i(x_{i,2}) = \epsilon$ and $u_j(x_{i,1}) = u_j(x_{i,2}) = 0$ for $j\neq i$ and $0<\epsilon<1$. Then, let \[
\begin{array}{cc}
u_{m-1}(x_{m-1,1}) = 83, & u_{m}(x_{m-1,1}) = 1\\
u_{m-1}(x_{m-1,2}) = 80, & u_{m}(x_{m-1,1}) = 19\\
u_{m-1}(x_{m,1}) = 3, & u_{m}(x_{m,1}) = 24\\
u_{m-1}(x_{m,2}) = 9, & u_{m}(x_{m,2}) = 22
\end{array}
\]
and $u_j(x_{m-1,1}) = u_j(x_{m-1,2}) = u_j(x_{m,1}) = u_j(x_{m,2}) = 0$ for $1\leq j\leq m-2$. We define the WP function as 
\[
p_{i,\mathbf{s}} = 
\left\{\begin{array}{ll}
\frac{(u(x_{i,\mathbf{s}_i})/{\beta})^{1-f(a)/10}}{\sum_{j\in [m], u(x_{j,\mathbf{s}_j})>1} (u(x_{j,\mathbf{s}_j})/\beta)^{1-f(a)/10}} & \mbox{ if } u(x_{i,\mathbf{s}_i}) > 1,\\
0 & \mbox{ otherwise,}
\end{array}\right.
\]
where $f(a) = 1$ if $F(a) = \mbox{\tt True}$ and $f(a) = 0$ otherwise. Here $a_i = \mbox{\tt True}$ and $a_i = \mbox{\tt False}$ correspond to the candidates $x_{i,1}$ and $x_{i,2}$, respectively. 
Clearly, for $1\leq i\leq m-2$ we have $r_{i,\mathbf{s}} = 0$ for any profile $\mathbf{s}$. Let $\mathbf{s}'$ denote the profile without the $(m-1)$th and the $m$th party. 
Then, if $\mathbf{s}$ corresponds to the assignment $a$ of $X$ such that $F(a) = {\tt True}$, then
\begin{center}
\begin{tabular}[c]{ l l | l l }
	\centering
	%\multicolumn{2}{l}{payoff matrix} \vspace{7pt}\\
	%\hline
	{\footnotesize $r_{m-1,(\mathbf{s}',s_{\scriptsize m-1}=1,s_m=1)}$}& {\footnotesize $r_{m,(\mathbf{s}',s_{\scriptsize m-1}=1,s_m=1)}$}  
     & {\footnotesize $r_{m-1,(\mathbf{s}',s_{\scriptsize m-1}=1,s_m=2)}$}& {\footnotesize $r_{m,(\mathbf{s}',s_{\scriptsize m-1}=1, s_m=2)}$} \\
	\hline
	{\footnotesize $r_{m-1,(\mathbf{s}',s_{\scriptsize m-1}=2, s_m=1)}$}& {\footnotesize $r_{m,(\mathbf{s}',s_{\scriptsize m-1}=2, s_m=1)}$} 
     & {\footnotesize $r_{m-1,(\mathbf{s}',s_{\scriptsize m-1}=2,s_m=2)}$}& {\footnotesize $r_{m,(\mathbf{s}',s_{\scriptsize m-1}=2,s_m=2)}$} 
	%\hline
\end{tabular}
\end{center}

\begin{center}
$\approx$
\begin{tabular}[c]{ l l | l l }
	%\multicolumn{2}{c}{} \vspace{7pt}\\
	\centering
	% & \\
	%\hline
	61.82 \; &  7.09 \;  & \, 61.57  \; &  7.08 \\
	\hline
	61.75 \; &  20.18 \; & \, 61.53 \; & 19.78 
	%\hline
\end{tabular}
\end{center}
where there exists a PSNE for any $\mathbf{s}'$ and $s_{m-1}=1, s_m = 1$. And, if $\mathbf{s}$ corresponds to the assignment $a$ of $X$ such that $F(a) = {\tt False}$, then
\begin{center}
\begin{tabular}[c]{ l l | l l }
	\centering
	%\multicolumn{2}{l}{payoff matrix} \vspace{7pt}\\
	%\hline
	{\footnotesize $r_{m-1,(\mathbf{s}',s_{m-1}=1,s_m=1)}$}& {\footnotesize $r_{m,(\mathbf{s}',s_{m-1}=1,s_m=1)}$} 
     &  {\footnotesize $r_{m-1,(\mathbf{s}',s_{m-1}=1,s_m=2)}$}& {\footnotesize $r_{m,(\mathbf{s}',s_{m-1}=1,s_m=2)}$} \\
	\hline
	{\footnotesize $r_{m-1,(\mathbf{s}',s_{m-1}=2,s_m=1)}$}& {\footnotesize $r_{m,(\mathbf{s}',s_{m-1}=2,s_m=1)}$} 
     &  {\footnotesize $r_{m-1,(\mathbf{s}',s_{m-1}=2,s_m=2)}$}& {\footnotesize $r_{m,(\mathbf{s}',s_{m-1}=2,s_m=2)}$} 
	%\hline
\end{tabular}
\end{center}
\begin{center}
$\approx$
\begin{tabular}[c]{ l l | l l }
	%\multicolumn{2}{c}{} \vspace{7pt}\\
	\centering
	% & \\
	%\hline
	63.54 \; &  6.59 \;  & \, 63.05  \; &  6.66\\
	\hline
	63.50 \; &  20.07 \; & \, 63.07 \; & 19.72 
	%\hline
\end{tabular}
\end{center}
where there is no PSNE. Next, we show that the game is monotone. For $F(a) = {\tt True}$, 
\begin{center}
\begin{tabular}[c]{ l l | l l }
	\centering
	%\multicolumn{2}{l}{payoff matrix} \vspace{7pt}\\
	%\hline
	{\footnotesize $p_{m-1,(\mathbf{s}',s_{\scriptsize m-1}=1,s_m=1)}$}& {\footnotesize $r_{m,(\mathbf{s}',s_{\scriptsize m-1}=1,s_m=1)}$} 
     &  {\footnotesize $p_{m-1,(\mathbf{s}',s_{\scriptsize m-1}=1,s_m=2)}$}& {\footnotesize $r_{m,(\mathbf{s}',s_{\scriptsize m-1}=1,s_m=2)}$} \\
	\hline
	{\footnotesize $p_{m-1,(\mathbf{s}',s_{\scriptsize m-1}=2,s_m=1)}$}& {\footnotesize $r_{m,(\mathbf{s}',s_{\scriptsize m-1}=2,s_m=1)}$} 
     &  {\footnotesize $p_{m-1,(\mathbf{s}',s_{\scriptsize m-1}=2,s_m=2)}$}& {\footnotesize $r_{m,(\mathbf{s}',s_{\scriptsize m-1}=2,s_m=2)}$} 
	%\hline
\end{tabular}
\end{center}
\begin{center}
$\approx$
\begin{tabular}[c]{ l l | l l }
	%\multicolumn{2}{c}{} \vspace{7pt}\\
	\centering
	% & \\
	%\hline
	0.7353 \; &  0.2647 \;  & \, 0.7104 \; &  0.2896 \\
	\hline
	0.7630 \; &  0.2370 \; & \, 0.7398 \; & 0.2602 
	%\hline
\end{tabular}
\end{center}
and for $F(a) = {\tt False}$, 
\begin{center}
\begin{tabular}[c]{ l l | l l }
	\centering
	%\multicolumn{2}{l}{payoff matrix} \vspace{7pt}\\
	%\hline
	{\footnotesize $p_{m-1,(\mathbf{s}',s_{\scriptsize m-1}=1,s_m=1)}$}& {\footnotesize $r_{m,(\mathbf{s}',s_{\scriptsize m-1}=1,s_m=1)}$} 
     &  {\footnotesize $p_{m-1,(\mathbf{s}',s_{\scriptsize m-1}=1,s_m=2)}$}& {\footnotesize $r_{m,(\mathbf{s}',s_{\scriptsize m-1}=1,s_m=2)}$} \\
	\hline
	{\footnotesize $p_{m-1,(\mathbf{s}',s_{\scriptsize m-1}=2,s_m=1)}$}& {\footnotesize $r_{m,(\mathbf{s}',s_{\scriptsize m-1}=2,s_m=1)}$} 
     &  {\footnotesize $p_{m-1,(\mathbf{s}',s_{\scriptsize m-1}=2,s_m=2)}$}& {\footnotesize $r_{m,(\mathbf{s}',s_{\scriptsize m-1}=2,s_m=2)}$} 
	%\hline
\end{tabular}
\end{center}
\begin{center}
$\approx$
\begin{tabular}[c]{ l l | l l }
	%\multicolumn{2}{c}{} \vspace{7pt}\\
	\centering
	% & \\
	%\hline
	0.7568 \; &  0.2432 \;  & \, 0.7304 \; &  0.2696 \\
	\hline
	0.7857 \; &  0.2143 \; & \, 0.7615 \; & 0.2385 
	%\hline
\end{tabular}.
\end{center}
From the construction of the game, we have $u(x_{m-1,1}) = 83+1=84$, $u(x_{m-1,2}) = 80+19=99$, $u(x_{m,1}) = 3+24=27$, and $u(x_{m,2}) = 9+22=31$, which means that for either the $(m-1)$th or the $m$th party, their second candidate brings more social utility to all the voters than the first does. Consider the case that $F(a) = {\tt True}$. When the $(m-1)$th party switches the candidate $x_{m-1,1}$ to~$x_{m-1,2}$, the winning probability of the party either increases from~0.7353 to~0.7630 (if the corresponding assignment of the resulting profile remains {\tt True}) or~0.7857 (if the corresponding assignment of the resulting profile becomes {\tt False}). 
Consider the case that $F(a) = {\tt False}$.
When the $(m-1)$th party switches the candidate $x_{m-1,1}$ to~$x_{m-1,2}$, the winning probability of the party either increases from~0.7568 to~0.7857 (if the corresponding assignment of the resulting profile remains {\tt False}) or~0.7630 (if the corresponding assignment of the resulting profile becomes {\tt True}). The rest cases can be similarly checked. Hence, we conclude that the constructed game is monotone. 


The construction of the game clearly takes polynomial time and the game is monotone. Therefore, the theorem is proved.
\qed
\end{proof}
\fi

%%%%%%%%%%%%%%%%%%%%%%%%%%%%%%%%%%%%%%%%%%%%%%%%%%%%%%%%%%%%%%%%%%%%%%
%\section{Two Sufficient Conditions for PSNE Existence}% in the Egoistic Election Game}
%\label{sec:two_sufficient_conditions_PSNE}
%%%%%%%%%%%%%%%%%%%%%%%%%%%%%%%%%%%%%%%%%%%%%%%%%%%%%%%%%%%%%%%%%%%%%%

%%%%%%%%%%%%%%%%%%%%%%%%%%%%%%%%%%%%%%%%%%%%%%%%%%%%%%%%%%%%%%%%%%%%%%
\section{Algorithmic Results}
\label{sec:algos}
%%%%%%%%%%%%%%%%%%%%%%%%%%%%%%%%%%%%%%%%%%%%%%%%%%%%%%%%%%%%%%%%%%%%%%

%%%%%%%%%%%%%%%%%%%%%%%%%%%%%%%%%%%%%%%%%%%%%%%%%%%%%%%%%%%%%%%%%%%%%%
\subsection{A Fixed-Parameter Algorithm}% for Finding a PSNE of the Egoistic Election Game}
\label{sec:fpt-algo}
%%%%%%%%%%%%%%%%%%%%%%%%%%%%%%%%%%%%%%%%%%%%%%%%%%%%%%%%%%%%%%%%%%%%%%


Theorem~\ref{thm:dominatingNE} provides two sufficient conditions for the egoistic election game to have a pure-strategy Nash equilibrium when a monotone WP function is adopted. Roughly speaking, if every party player has a dominant strategy as nominating their first candidate, which weakly surpasses all the other ones in their own candidate sets, then clearly the profile is a PSNE. Moreover, if $m-1$ of the party players have such dominant strategies, the remaining party player can simply take a best response with respect to the profile except itself. 

\begin{theorem}\label{thm:dominatingNE}
\begin{enumerate}[label=(\alph*)]
\item If for all $i\in [m]$, strategy $x_{i,1}$ weakly surpasses (surpasses, resp.) each $x_{i,j}$ for $j\in [n_i]\setminus\{1\}$ of party player~$\mathcal{P}_i$, then $(x_{1,1}, x_{2,1}, \ldots, x_{m,1})$ is a PSNE (the unique PSNE, resp.) of the egoistic election game. 
\item If there exists $\mathcal{I}\subset [m]$, $|\mathcal{I}| = m-1$, such that for all $i\in \mathcal{I}$,  
$x_{i,1}$ weakly surpasses each $x_{i,j}$ for $j\in [n_i]\setminus\{1\}$ of party player $\mathcal{P}_i$, then $((x_{i,1})_{i\in\mathcal{I}}, x_{j,s_j^{\#}})$ is a PSNE for $j\notin\mathcal{I}$ and $s_j^{\#} = \arg\max_{\ell\in [n_j]} r_{j}((x_{i,1})_{i\in\mathcal{I}}, x_{j,\ell})$. 
\end{enumerate}
\end{theorem}
\iffalse
\begin{proof}
For the first case, let us consider an arbitrary $i\in [m]$ and an arbitrary $t\in [n_i]\setminus\{1\}$. Denote by $\mathbf{s}$ the profile $(x_{1,1}, x_{2,1}, \ldots, x_{m,1})$. 
Since strategy $x_{i,1}$ weakly surpasses $x_{i,t}$, we have $p_{i,\mathbf{s}}\geq p_{i,(t,\mathbf{s}_{-i})}$ and then $\sum_{j\in [m]\setminus\{i\}} p_{j,\mathbf{s}}\leq \sum_{j\in [m]\setminus\{i\}} p_{j,(t,\mathbf{s}_{-i})}$. Hence, 
\begin{eqnarray*}
& &r_{i,\mathbf{s}} - r_{i,(t, \mathbf{s}_{-i})}\\ 
&=& 
\sum_{j\in [m]} p_{j,\mathbf{s}} u_i(x_{j,1}) - \left(\sum_{j\in [m]\setminus\{i\}} p_{j,(t, \mathbf{s}_{-i})} u_i(x_{j,1}) + p_{i,(t, \mathbf{s}_{-i})} u_i(x_{i,t}) \right)\\
&=& \sum_{j\in [m]\setminus\{i\}} u_i(x_{j,t})(p_{j,\mathbf{s}}-p_{j,(t,\mathbf{s}_{-i})}) + (p_{i,\mathbf{s}} u_i(x_{i,1}) - p_{i,(t,\mathbf{s}_{-i})} u_i(x_{i,t}))\\
&\geq& \sum_{j\in [m]\setminus\{i\}} u_i(x_{i,t})(p_{j,\mathbf{s}}-p_{j,(t,\mathbf{s}_{-i})}) +  u_i(x_{i,t}) (p_{i,\mathbf{s}} - p_{i,(t,\mathbf{s}_{-i})})\\
&=& u_i(x_{i,t})\left( \sum_{j\in [m]\setminus\{i\}} p_{j,\mathbf{s}} - p_{j,(t,\mathbf{s}_{-i})} + (p_{i,\mathbf{s}} - p_{i,(t,\mathbf{s}_{-i})})\right)\\
&=& u_i(x_{i,t})\left(\sum_{j\in [m]} p_{j,\mathbf{s}} - \sum_{j\in [m]} p_{j,(t,\mathbf{s}_{-i})} \right) = 0, 
\end{eqnarray*}
where the inequality follows from the egoistic property and the assumption that strategy~$x_{i,1}$ weakly surpasses~$x_{i,t}$, and last equality holds since $\sum_{j\in [m]} p_{j,\mathbf{s}} = \sum_{j\in [m]} p_{j,(t,\mathbf{s}_{-i})} = 1$ by the law of total probability. Thus, party player $i$ has no incentive to deviate from its current strategy. Note that the uniqueness of the PSNE comes when the inequality becomes ``greater-than". This happens as strategy~$x_{i,1}$ surpasses~$x_{i,t}$. 

For the second case, by the same arguments for the first case, we know that for $i\in \mathcal{I}$, party player $i$ has no incentive to deviate from its current strategy. Let $j\in [m]\setminus\mathcal{I}$ be the only one party player not in~$\mathcal{I}$. As $s_j^{\#}$ is the best response when the other strategies of parties in $\mathcal{I}$ are fixed, party player $j$ has no incentive to deviate from its current strategy. Therefore, the theorem is proved. 
\qed
\end{proof}
\fi
For example, consider $m=2$ (i.e., only two parties exist in the society). Namely, if strategy $x_{1,1}$ of party $\mathcal{P}_1$ (weakly) surpasses each $x_{1,t}$ for $2\leq t\leq n_1$ and strategy $x_{2,1}$ of party $\mathcal{P}_2$ (weakly) surpasses each $x_{2,t'}$ for $2\leq t'\leq n_2$, then $(x_{1,1},x_{2,1})$ is a (weakly) dominant-strategy solution.



%%%%%%%%%%%%%%%%%%%%%%%%%%%%%%%%%%%%%%%%%%%%%%%%%%%%%%%%%%%%%%%%%%%%%%
%\section{A Fixed-Parameter Algorithm}% for Finding a PSNE of the Egoistic Election Game}
%\label{sec:fpt-algo}
%%%%%%%%%%%%%%%%%%%%%%%%%%%%%%%%%%%%%%%%%%%%%%%%%%%%%%%%%%%%%%%%%%%%%%



Below we give a useful lemma which states that a profile with a player's strategy surpassed by its another one is never a PSNE. With a slight abuse of notation, we abbreviate the profile $(x_{1,s_1}, x_{2,s_2},\ldots,\allowbreak x_{m,s_m})$ by $\mathbf{s} = (s_1,s_2,\ldots,s_m)$. 
 
\begin{lemma}\label{lem:dominated_Not_PSNE}
If $s_i$ is surpassed by some $s'_i\in [n_i]\setminus\{s_i\}$, then $(s_i,(\tilde{s}_j)_{j\in [m]\setminus\{i\}})$ is not a PSNE for any %profile 
$(\tilde{s}_j)_{j\in [m]\setminus\{i\}}$.% except $s_i$. 
\end{lemma}
\iffalse
\begin{proof}
Let $\mathbf{s}_{-i} = (\tilde{s}_j)_{j\in [m]\setminus\{i\}}$ be any profile except party player $\mathcal{P}_i$'s strategy. 
If $s_i$ is surpassed by $s'_i$, then we have $s'_i < s_i$, and either:
\begin{itemize}
\item [(a)] $u(x_{i,s'_i}) > u(x_{i,s_i})$ and $u_i(x_{i,s'_i})\geq u_i(x_{i,s_i})$, or
\item [(b)] $u(x_{i,s'_i})\geq u(x_{i,s_i})$ and $u_i(x_{i,s'_i})>u_i(x_{i,s_i})$. 
\end{itemize}
We prove case (a) as follows. Case (b) can be similarly proved. 

We have $u_i(x_{i,s'_i})\geq u_i(x_{i,s_i})$ from the assumption in (a). Also, by the monotone property of the WP functions, we have $p_{i,(s'_i, \mathbf{s}_{-i})} \geq p_{i,(s_i,\mathbf{s}_{-i})}$ and $\sum_{j\in [m]\setminus\{i\}} p_{j, (s'_i, \mathbf{s}_{-i})} \leq \sum_{j\in [m]\setminus\{i\}} p_{j, (s_i,\mathbf{s}_{-i})}$ since $u(x_{i,s'_i}) > u(x_{i,s_i})$. Thus,
\begin{eqnarray*}
& & r_{i}(s_i,\mathbf{s}_{-i}) - r_{i}(s'_i,\mathbf{s}_{-i})\\ 
&=& \left(p_{i,(s_i,\mathbf{s}_{-i})}u_i(x_{i,s_i}) + \sum_{j\in [m]\setminus\{i\}} p_{j,(s_i,\mathbf{s}_{-i})}u_i(x_{j,\tilde{s}_j})\right) \\ 
& & - \left(p_{i,(s'_i,\mathbf{s}_{-i})}u_i(x_{i,s'_i})+\sum_{j\in [m]\setminus\{i\}} p_{j,(s'_i,\mathbf{s}_{-i})}u_i(x_{j,\tilde{s}_j})\right)\\
&=& \left(\sum_{j\in [m]\setminus\{i\}} (p_{j,(s_i,\mathbf{s}_{-i})} - p_{j,(s'_i,\mathbf{s}_{-i})}) u_i(x_{j,\tilde{s}_j})\right) \\
& & + (p_{i,(s_i,\mathbf{s}_{-i})}u_i(x_{i,s_i}) - (p_{i,(s'_i,\mathbf{s}_{-i})}u_i(x_{i,s'_i}))\\
&<& u_i(x_{i,s_i})\left(\sum_{j\in [m]\setminus\{i\}} (p_{j,(s_i,\mathbf{s}_{-i})} - p_{j,(s'_i,\mathbf{s}_{-i})})\right)\\ 
& & + (p_{i,(s_i,\mathbf{s}_{-i})}u_i(x_{i,s_i}) - (p_{i,(s'_i,\mathbf{s}_{-i})}u_i(x_{i,s'_i}))\\
&=& u_i(x_{i,s_i})(p_{i,(s'_i,\mathbf{s}_{-i})} - p_{i,(s_i,\mathbf{s}_{-i})}) + (p_{i,(s_i,\mathbf{s}_{-i})}u_i(x_{i,s_i})\\
& & - p_{i,(s'_i,\mathbf{s}_{-i})}u_i(x_{i,s'_i}))\\
&\leq& (p_{i,(s'_i,\mathbf{s}_{-i})} - p_{i,(s_i,\mathbf{s}_{-i})})(u_i(x_{i,s_i})-u_i(x_{i,s_i})) = 0,
\end{eqnarray*}
where the first inequality follows from the egoistic property. 
Thus, $(s_i,\mathbf{s}_{-i})$ is not a PSNE.
\qed
\end{proof}
\fi

%We have learned two sufficient conditions in Theorem~\ref{thm:dominatingNE} for the egoistic election game to have a PSNE. 
Inspired by Theorem~\ref{thm:dominatingNE} and Lemma~\ref{lem:dominated_Not_PSNE}, we are able to devise an efficient algorithm to find out a PSNE of the egoistic election game whenever it exists, with respect to two parameters: number of \emph{irresolute parties} and \emph{nominating depth}, which are introduced as follows. As we have assumed, candidates in each party are sorted according to the utility for its supporters. That is, $u_i(x_{i,1})\geq u_i(x_{i,2})\geq \ldots \geq u_i(x_{i,n_i})$ for each $i\in [m]$. Let $d_i$ be the index of the candidate that surpasses all candidates $x_{i,d_i+1},\ldots,x_{i,n_i}$ or is set to~$n_i$ if $x_{i,n_i}$ is not surpassed by any candidates of~$\mathcal{P}_i$. Formally, let $\mbox{maxProb}_i := \argmax_{s\in [n_i]}u(x_{i,s})$, then $d_i = \max\{ \arg\max_{s\in \mbox{\scriptsize maxProb}_i} u_i(x_{i,s})\}$. 
From Theorem~\ref{thm:dominatingNE} we know that $x_{i,d_i}$ is a dominant strategy for the ``sub-game instance" in which $\mathcal{P}_i$'s strategy set is reduced to~$\{x_{i,s}\}_{s\in\{d_i,d_i+1,\ldots,n_i\}}$. Thus, $\mathcal{P}_i$ suffices to consider candidates in $\{x_{i,s}\}_{s\in [d_i]}$ as its strategies. 
We then collect the set $\mathcal{D} = \{i\mid i\in [m], d_i = 1\}$. 
For each $i\in\mathcal{D}$, party player $\mathcal{P}_i$ must choose the first candidate $x_{i,1}$. 
Thus, we can reduce the game instance to $(\tilde{X}_i)_{i\in [m]\setminus\mathcal{D}}$, where $\tilde{X}_i = \{x_{i,s}\}_{s\in [d_i]}$. We call  $d:=\max_{i\in [m]} d_i$ the \emph{nominating depth} of the election game. We call a party $\mathcal{P}_i$ \emph{irresolute} if $d_i>1$ and denote by~$k$ the number of irresolute parties. 
Then, we propose Algorithm {\sf FPT-ELECTION-PSNE} to compute a PSNE of the egoistic election game. The complexity of the algorithm is $O(nm^2 + kd^{k+1}m)$. Thus, to compute a PSNE for such a game is \emph{fixed-parameter tractable} with respect to the parameters~$d$ and~$k$. Theorem~\ref{thm:PSNE_compute_tractable} concludes the results (see Appendix~\ref{subsec:FPT_complexity_proof} for the proof). 


\begin{algorithm}\label{alg:fpt_algo}
\caption{$\textbf{\sf FPT-ELECTION-PSNE}$}
\begin{algorithmic}[0]
\REQUIRE an election game instance $\mathcal{G} = (X_1,X_2,\ldots,X_m,f_{\mathcal{G}})$. 
\end{algorithmic}
\begin{algorithmic}[1]
\STATE For each $i$, compute $\mbox{maxProb}_i = \argmax_{s\in [n_i]}u(x_{i,s})$. 
\STATE For each $i$, compute $d_i = \max\{ \arg\max_{s\in \mbox{\scriptsize maxProb}_i} u_i(x_{i,s})\}$.  
%\STATE Compute $d_i = \arg\max_{s\in [n_i]} u_i(x_{i,s})$ for each~$i$. \COMMENT{Preprocessing}
\STATE Collect $\mathcal{D} = \{i\mid i\in [m], d_i = 1\}$ and assign $x_{i,1}$ to party player $\mathcal{P}_i$ for $i\in\mathcal{D}$, 
\STATE Reduce the game instance to $(\tilde{X}_i)_{i\in [m]\setminus\mathcal{D}}$, where $\tilde{X}_i = \{x_{i,s}\}_{s\in [d_i]}$, $i\in [m]\setminus\mathcal{D}$. 
%\STATE Calculate $\sum_{i\in \mathcal{D}}u(x_{i,1})$.\COMMENT{For later use of calculating payoffs}
\FOR{each entry $\mathbf{s}\in \prod_{i\in \mathcal{D}}\{x_{i,1}\}\times\prod_{j\in [m]\setminus\mathcal{D}} \tilde{X}_j$}
\STATE Compute the payoff $r_i(\mathbf{s})$ for each $i$.
\STATE \COMMENT{Then, check if any unilateral deviation is possible}
\IF{$\mathbf{s}$ corresponds to a PSNE}
\STATE \COMMENT{i.e., for each $j\in [m]\setminus\mathcal{D}$, check if $r_j(\mathbf{s})\geq r_j(s'_j,\mathbf{s}_{-j})$ for all $s'_j\in \tilde{X}_j$}
\STATE return $\mathbf{s}$
\ENDIF
\ENDFOR
\STATE Output ``NO"
\end{algorithmic}
\end{algorithm}


\begin{theorem}\label{thm:PSNE_compute_tractable}
Given an election game instance $\mathcal{G}$ of $m\geq 2$ parties each of which has at most $n$ candidates. Suppose that $\mathcal{G}$ has at most~$k$ irresolute parties and the nominating depth of~$\mathcal{G}$ is bounded by~$d$, then to compute a PSNE of~$\mathcal{G}$ takes $O(nm^2 + kd^{k+1}m)$ time if it exists.     
\end{theorem}
\iffalse
\begin{proof}
It costs $O(nm^2)$ time to compute $d$ and $d_i$ for all~$i$, and the set $\mathcal{D}$ can be then obtained. Since playing strategy~1 (i.e., nominating the first candidate) for parties $i\in\mathcal{D}$ is the dominant strategy, we only need to consider party players in~$[m]\setminus\mathcal{D}$. Since for each $i\in [m]\setminus\mathcal{D}$, each strategy in~$\{x_{i,d_i+1},\ldots,x_{i,n_i}\}$ is surpassed by $x_{i,d_i}$ (considering the subgame with respect to~$(x_{i,d_i+1},\ldots,x_{i,n_i})_{i\in [m]}$), by Lemma~\ref{lem:dominated_Not_PSNE} we know that we only need to consider strategies $\{x_{i,1},\ldots,x_{i,d_i}\}$ for such party player~$i$. Thus, the number of entries of the implicit payoff matrix
that we need to check whether it is a PSNE is $\prod_{i\in [m]\setminus\mathcal{D}} d_i = O(d^k)$, where $k = |[m]\setminus \mathcal{D}|$. 
For each of these entries, it takes $O(m)$ time to compute the winning probabilities as well as the payoffs of the party players in~$[m]\setminus\mathcal{S}$. 
In addition, checking whether each of such entries can be done in at most~$k(d-1)$ steps. Therefore, the theorem is proved.
\qed
\end{proof}
\fi


\subsubsection*{Remark.} As candidates in each party are sorted by the utility for the supporters of their own party, the nominating depth $d$ equals the maximum possible number of candidates that we suffice to consider\footnote{In fact, we can skip some candidates in~$[d_i]$ for each $\mathcal{P}_i$. See Appendix~\ref{subsec:strategy_refined} for more discussions.}. On the other hand, a party is irresolute if it does not have a candidate as a dominant strategy, and hence a ``searching" for the state involving irresolute parties is required. If the number $k$ of irresolute parties is small, it leads to limited exponential explosion. In practical, a party usually has very limited number of potential candidates to choose for an election campaign, so we believe the two parameters are natural and small in practice. Hence, the fixed-parameter algorithm is efficient in practical.  


%=====================================================================
\subsection{A Na\"{i}ve Algorithm Obtaining a $(1+e)$-Approximate PSNE}
\label{subsec:approx}
%=====================================================================

Below we show that there exists a na\"{i}ve constant time algorithm to compute a $(1+e)$-approximate PSNE of the egoistic election game for which the softmax WP is applied. 

\begin{theorem}\label{thm:approx}
If the election game is egoistic and consider the softmax function as the WP function, then each party player $\mathcal{P}_i$, $i\in [m]$, choosing the first candidate $x_{i,1}$ constitutes a $(1+e)$-approximate PSNE. %Moreover, it becomes a $e$-approximate PSNE when $m$ approaches to~$\infty$.
\end{theorem}
\begin{proof}
(Sketch; refer to the appendix for more detail) Let $\mathbf{s}$ denote the profile that each party player $\mathcal{P}_{\ell}$ chooses their first candidate $x_{\ell,1}$ for $\ell\in [m]$. Consider an arbitrary party player $\mathcal{P}_i$ and its arbitrary candidate, say $x_{i, r}$, other than $x_{i,1}$. For each $\ell\in [m]$, let $p_{\ell}$ denote the winning probability of $\mathcal{P}_{\ell}$ and let $p_{\ell}'$ denote the winning probability of~$\mathcal{P}_{\ell}$ when $\mathcal{P}_i$ unilaterally deviates from~$x_{i,1}$ to~$x_{i,r}$. Note that we can focus on the case that $p_i'> p_i$ since otherwise $x_{i,r}$ is weakly surpassed by~$x_{i,1}$ and then $\mathcal{P}_i$ has no incentive to deviate from~$x_{i,1}$ to~$x_{i,r}$. Compute the gain of $\mathcal{P}_i$ from such a strategy deviation, we have 
\begin{eqnarray*}
& & \left(p_i' u_i(x_{i,r})+\sum_{j\in [m]\setminus\{i\}} p_j' u_i(x_{j,1})\right) - \sum_{\ell\in [m]} p_{\ell} u_i(x_{\ell, 1}) \leq \sum_{\ell\in [m]} (p_{\ell}'-p_{\ell}) u_i(x_{\ell,1}),
\end{eqnarray*}
where the inequality follows since $u_i(x_{i,1})\geq u_i(x_{i,r})$ for $r\neq 1$.
The ratio of the payoff improvement appears to be upper bounded by~
\begin{eqnarray}
\frac{\sum\limits_{\ell\in [m]} (p_{\ell}'-p_{\ell}) u_i(x_{\ell,1})}{\sum\limits_{\ell\in [m]} p_{\ell} u_i(x_{\ell,1})} 
%&=& \frac{\sum\limits_{\ell\in [m]} p_{\ell}' u_i(x_{\ell,1})}{\sum\limits_{\ell\in [m]} p_{\ell} u_i(x_{\ell,1})} - 1\nonumber
%= \frac{p_i' u_i(x_{i,1})+\sum\limits_{j\in[m]\setminus\{i\}} p_j' u_i(x_{j,1})}{p_i u_i(x_{i,1}) + \sum\limits_{j\in[m]\setminus\{i\}} p_j u_i(x_{j,1})} - 1\nonumber\\
&\leq& \frac{p_i' u_i(x_{i,1})+\sum\limits_{j\in[m]\setminus\{i\}} p_j u_i(x_{j,1})}{p_i u_i(x_{i,1}) + \sum\limits_{j\in[m]\setminus\{i\}} p_j u_i(x_{j,1})} - 1\nonumber
\leq \frac{p_i'}{p_i} - 1\nonumber%\label{eqn:approx}, 
\end{eqnarray}
%where the first inequality holds since $p_j'\leq p_j$.
%In Equation~\ref{eqn:approx}, b
By the definition of the softmax function, $p_i'/p_i$ is upper bounded by~$e+1$. Hence the approximation ratio is upper bounded by~$1+e$.
%$$\frac{e^1/(e^1+C)}{e^0/(e^0+C)} = \frac{e(1+C)}{e+C}\leq e+1,$$
%where $C$ denotes $\sum_{j\in [m]\setminus\{i\}}e^{u(x_{j,1})/\beta}\leq e(m-1)$.
%Let $f(x):= e(1+x)/(e+x)$. 
%Taking the first and second derivatives of~$f$, we have $f'(x) = e(e-1)/(e+x)^2 > 0$ and $f''(x) = -2e(e-1)(e+x)/(e+x)^4 < 0$ for all $x>0$. Substituting $x = e(m-1)$ we have 
%\begin{eqnarray*}
%   f(e(m-1)) &\leq& \frac{e(1+e(m-1))}{e+e(m-1)} \leq e+1. 
%\end{eqnarray*}
%Hence, the approximation ratio is upper bounded by~$1+e$ for each party player choosing the first candidate. % in the egoistic election game using the softmax function as the WP function.
\qed    
\end{proof}



%%%%%%%%%%%%%%%%%%%%%%%%%%%%%%%%%%%%%%%%%%%%%%%%%%%%%%%%%%%%%%%%%%%%%%
\section{Price of Anarchy}% for Egoistic Election Games}
\label{sec:PoA}
%%%%%%%%%%%%%%%%%%%%%%%%%%%%%%%%%%%%%%%%%%%%%%%%%%%%%%%%%%%%%%%%%%%%%%


Below, Proposition~\ref{pro:cases} relates a PSNE to an optimal profile on social welfare. Note that a PSNE may be a profile that is suboptimal in the social welfare in the election game.

%\textcolor{red}{(In Section 2, strategy profile is denoted as $(x_{i,s_i})_{i\in[m]}$ or $(s_i)_{i\in[m]}$ interchangeably?)}

\begin{proposition}\label{pro:cases}
Let $\mathbf{s} = (s_i)_{i\in [m]}$ be a PSNE and $\mathbf{s}^* = (s_i^*)_{i\in [m]}$ be the optimal profile. Then, $\sum_{i\in [m]}u(s_i)\geq \max_{i\in [m]} u(s_i^*)$.
\end{proposition}
\begin{proof}
We assume that $\mathbf{s}\neq \mathbf{s}^*$ since otherwise the proposition trivially holds. By Lemma~\ref{lem:dominated_Not_PSNE}, we know that for each $i\in [m]$, strategy $s_i$ is not surpassed by~$s_i^*$ since $\mathbf{s}$ is a PSNE. Therefore, it suffices to consider the following cases.
\begin{enumerate}[label=(\alph*)]
    \item If $s_i\leq s_i^*$ for all $i\in [m]$, then for each $j\in [m]$, 
    $\sum_{i\in [m]} u(s_i)\geq \sum_{i\in [m]}u_i(s_i)\geq \sum_{i\in [m]} u_i(s_j^*) = u(s_j^*)$, where the second inequality follows from the egoistic property. Hence, we have $\sum_{i\in [m]}u(s_i)\geq \max_{i\in [m]} u(s_i^*)$.  
    \item If $u(s_i^*)\leq u(s_i)$ for all $i\in [m]$, then obviously we have 
	$\sum_{i\in [m]}u(s_i)\geq \sum_{i\in [m]}u(s_i^*)\geq \max_{i\in [m]}u(s_i^*)$. 
    \item Suppose that there exists a subset $W\subset [m]$ such that $s_i\leq s_i^*$ for all $i\in W$ and $u(s_j^*)\leq u(s_j)$ for each $j\in \overline{W}:=[m]\setminus W$. For $j\in W$, we have $\sum_{i\in [m]} u(s_i)\geq \sum_{i\in [m]} u_i(s_i)\geq \sum_{i\in [m]}u_i(s_j^*) = u(s_j^*)$. For $j\in\overline{W}$, we have $\sum_{i\in [m]} u(s_i) = \sum_{i\in [m]\setminus\{j\}} u(s_i) + u(s_j)\geq u(s_j^*)$. Hence, $\sum_{i\in [m]} u(s_i)\geq \max_{i\in [m]}u(s_i^*)$.
\end{enumerate}
\qed
\end{proof}


\iffalse
%\vspace{-12pt}
%=====================================================================
\subsection{Adopting the Natural Function}
\label{subsec:PoA_naive}
%=====================================================================

In Sect.~\ref{sec:hardness_no_PSNE_examples}, we have seen counterexamples that a PSNE does not always exist in the egoistic election game given the natural function as the WP function. Nevertheless, we can still ask how good or bad its PoA can be once a PSNE of the game instance exists. In the following, we show that its PoA is upper bounded by~$m$ when the natural function is adopted as the WP function. The proof is Appendix~\ref{subsec:PoA_BT}. 

\begin{theorem}\label{thm:PoA_BT}
Suppose that the natural function is adopted as the monotone WP function. Then the egoistic election game has the PoA upper bounded by the number of parties~$m$. 
\end{theorem}
\iffalse
\begin{proof}
Let $\mathbf{s} = (s_i)_{i\in [m]}$ be a PSNE and $\mathbf{s}^* = (s_i^*)_{i\in [m]}$ be the optimal profile.
Note that $SW(\mathbf{s}^*) = \sum_{i\in [m]}p_{i^*,\mathbf{s}^*} u(s_i^*)\leq \max_{i\in [m]} u(s_i^*)$. 
By the Cauchy-Schwarz inequality, we derive that  
$\sum_{i\in [m]} u(s_i)^2 \geq (\sum_{i\in [m]} u(s_i))^2/m$. 
Then, we have
\begin{eqnarray*}
SW(\mathbf{s})&=& \sum_{i\in [m]}\frac{u(s_i)}{\sum_{j\in [m]} u(s_j)}\cdot u(s_i) = \frac{\sum_{i\in [m]} u(s_i)^2}{\sum_{j\in [m]} u(s_j)}\\ 
&\geq& \frac{1}{m}\cdot \sum_{i\in [m]}u(s_i).
\end{eqnarray*}

Together, by Proposition~\ref{pro:cases} that $\sum_{i\in [m]}u(s_i)\geq \max_{i\in [m]} u(s_i^*)$, we obtain that $SW(\mathbf{s})\geq SW(\mathbf{s}^*)/m$, Thus, the PoA is at most~$m$. 
\qed
\end{proof}
\fi

\fi


%=====================================================================
\subsection{Adopting the Monotone Function in General}
\label{subsec:PoA_SM}
%=====================================================================

%We have seen analysis that the egoistic election game given the natural function as the monotone WP function has PoA upper bounded by~$m$. In fact, 
By carefully lower bounding the social welfare of a profile, we can show that the bound holds for the game adopting any monotone WP function. 
%Note that 
For any strategy profile $\mathbf{s}$, we have
\vspace{-0pt}
\begin{align}
& SW(\mathbf{s}) = \sum_{i\in [m]} p_{i,\mathbf{s}}\cdot u(s_i)\leq \max_{i\in [m]} u(s_i) \label{eq:UB_SW_softmax}\\
& SW(\mathbf{s}) = \sum_{i\in [m]} p_{i,\mathbf{s}}\cdot u(s_i) \geq \frac{1}{m}\cdot \sum_{i\in [m]} u(s_i) \label{eq:LB_SW_softmax}
\end{align} 
which hold for any monotone WP function. Inequality~(\ref{eq:LB_SW_softmax}) can be justified as follows. 
W.l.o.g., let us assume that $u(s_1)\geq u(s_2)\geq \ldots \geq u(s_m)$ (by relabeling after they are sorted). As the winning probability function is monotone, it is clear that $p_{1,\mathbf{s}}\geq p_{2,\mathbf{s}}\geq\ldots \geq p_{m,\mathbf{s}}$. Let $k\in [m]$ be the index such that $p_{k,\mathbf{s}}\geq 1/m$ and $p_{k+1,\mathbf{s}}< 1/m$ and $k = m$ if $p_{i,\mathbf{s}} = 1/m$ for all $i\in [m]$. Note that such an index $k$ must exist otherwise $\sum_{i\in [m]} p_{i,\mathbf{s}} < 1$ which contradicts the law of total probability. It is clear that Inequality~(\ref{eq:LB_SW_softmax}) holds when $k = m$. For the case $k<m$, we have 
\vspace{-1pt}
\begin{eqnarray*}
& & SW(\mathbf{s}) = \sum_{i\in [m]} p_{i,\mathbf{s}}\cdot u(s_i)\\
&=& \sum_{i=1}^k \left(p_{i,\mathbf{s}} - \frac{1}{m}\right) u(s_i) + \frac{1}{m}\sum_{i=1}^m u(s_i)
 + \sum_{i=k+1}^m \left(p_{i,\mathbf{s}} - \frac{1}{m}\right) u(s_i)\\
&\geq& \sum_{i=1}^k \left(p_{i,\mathbf{s}} - \frac{1}{m}\right) u(s_k) + \frac{1}{m}\sum_{i=1}^m u(s_i)
 + \sum_{i=k+1}^m \left(p_{i,\mathbf{s}} - \frac{1}{m}\right) u(s_k)\\
&=&\left(\sum_{i=1}^m p_{i,\mathbf{s}} - 1\right)u(s_k) + \frac{1}{m}\sum_{i=1}^m u(s_i) = \frac{1}{m}\sum_{i=1}^m u(s_i).
\end{eqnarray*} 
\vspace{-0pt}
Hence the inequality~(\ref{eq:LB_SW_softmax}) is valid. 
%
Now, we are ready for Theorem~\ref{thm:PoA_SF} and its proof. 

\begin{theorem}\label{thm:PoA_SF}
The PoA of the egoistic election game using any monotone WP function is upper bounded by~$m$. 
\end{theorem}
\begin{proof}
Let $\mathbf{s} = (s_i)_{i\in [m]}$ be a PSNE and $\mathbf{s}^* = (s_i^*)_{i\in [m]}$ be the optimal profile. 
Let $\ell = \argmax_{i\in [m]} u(s_i^*)$ be the index of the party players with the maximum social utility with respect to~$\mathbf{s}^*$. Recall that $m\cdot SW(\mathbf{s})\geq \sum_{i\in [m]}u(s_i)$. 
Our goal is to prove Inequality~(\ref{eq:thm4c}):  
\begin{equation}\label{eq:thm4c}
\sum_{i\in [m]}u(s_i)\geq u(s_{\ell}^*).
\end{equation}
By Lemma~\ref{lem:dominated_Not_PSNE} we know that each $i\in [m]$, strategy $s_i$ is not surpassed by $s_i^*$ since $\mathbf{s}$ is a PSNE. Therefore, it suffices to consider the following cases.  
\begin{enumerate}[label=(\alph*)]
    \item For all $i\in [m]$, $s_i\leq s_i^*$. In this case, Inequality~(\ref{eq:thm4c}) holds since
   \begin{enumerate}[label=(\roman*)]
    \item $u_{\ell}(s_{\ell})\geq u_{\ell}(s_{\ell}^*)$,
    \item $u_{j}(s_{j})\geq u_{j}(s_{\ell}^*)$ for each $j\in[m]$. 
    \end{enumerate}
    \vspace{5pt}
    \item For all $i\in [m]$, $u(s_i^*)\leq u(s_i)$. In this case, we have $\sum_{i\in [m]} u(s_i) - u(s_{\ell}^*)\geq \sum_{i\in[m]\setminus\{\ell\}} u(s_i)\geq 0$. Hence, Inequality~(\ref{eq:thm4c}) holds.
    
    \item Suppose that there exists a subset $W\subset [m]$ such that $s_i\leq s_i^*$ for all $i\in W$ and $u(s_j^*)\leq u(s_j)$ for each $j\in \overline{W} := [m]\setminus W$. 
    
    \begin{enumerate}[label=(\roman*)]
    \item Assume that $\ell\in W$. 
    By the arguments similar to (a), Inequality~(\ref{eq:thm4c}) follows from~$u_{\ell}(s_{\ell})\geq u_{\ell}(s_{\ell}^*)$ (by the assumption that $\ell\in W$ and the egoistic property), and $u_{j}(s_{j})\geq u_{j}(s_{\ell}^*)$ for each $j\in[m]\setminus\{\ell\}$ (by the egoistic property).
     \vspace{2pt}
    \item Assume that $\ell\in \overline{W}$. Similar to (b), we have 
        \vspace{-0pt}
        \begin{equation*}
    	\sum_{i\in [m]} u(s_i) - u(s_{\ell}^*)\geq \sum_{i\in \overline{W}} u(s_i) - u(s_{\ell}^*)\geq \sum_{i\in\overline{W}\setminus\{\ell\}}  u(s_i)\geq 0.
    	\end{equation*}
        \iffalse
        \item Assume that $\ell\in W$ and $q\in \overline{W}$. Since $u(s_q)\geq u(s_{q}^*)$, to derive Inequality~(\ref{eq:thm4c}), we need to show that
        \begin{equation}\label{eq:thm4c2}
        \frac{e}{e+m-1}\cdot u(s_q)+\sum_{i\in [m]\setminus\{q\}} u(s_i)\geq \left(\frac{e}{e+m-1}\cdot u(s_{\ell}^*)\right).
        \end{equation}
        $\ell\in W$ implies that $u_{\ell}(s_{\ell})\geq u_{\ell}(s_{\ell}^*)$. By the egoistic property, we have $u_{j}(s_{j})\geq u_{j}(s_{\ell}^*)$ for each $j\in[m]\setminus\{\ell\}$. Therefore, Inequality~(\ref{eq:thm4c2}) holds.
        \fi
    \iffalse
    \item As case (iii), the case that $\ell\in\overline{W}$ and $q\in W$ can be proved similarly. 
    \fi
        \end{enumerate}
\end{enumerate}
Together with Inequality~(\ref{eq:LB_SW_softmax}) and (\ref{eq:UB_SW_softmax}), we derive that $SW(\mathbf{s})\geq SW(\mathbf{s}^*)/m$. 
Therefore, we conclude that the PoA is at most~$m$. 
\qed
\end{proof}

\iffalse
%Denote by $\ell' := \argmax_{i\in [m]\setminus W} u_i(\mathbf{s}^*)$ the index of the party player in $\overline{W}$ with the maximum social utility with respect to~$\mathbf{s}$. 
\fi

\paragraph{Remark.} In~\cite{LLC2021}, the lower bound on the PoA of the egoistic two-party election game using the softmax WP function is~2. 
Here we further point out that the PoA upper bound is tight when the hardmax WP function is applied. Consider the game instance shown in Table~\ref{tab:AlwaysPSNE_HM_2party}. We can see that the PoA of the game approaches~$m$. Through these examples we know that our PoA bound is tight for $m=2$ using either the hardmax or the softmax function as the WP function, and is also tight for general $m\geq 2$ when the hardmax WP function is applied. 

\paragraph{Remark.} In Table~\ref{tab:AlwaysPSNE_HM_2party}, the instance serves as a lower bound approaching~$m$ on the PoS as well since all PSNE share the same social welfare value.

\begin{table}[ht]
\begin{center}
\begin{tabular}[c]{ c c c c |  c c c c |  c  }
	%\centering
	{\footnotesize $u_1(x_{1,i})$}& {\footnotesize $u_2(x_{1,i})$}& \!$\cdots$\! & {\footnotesize $u_m(x_{1,i})$}\! & 
        {\footnotesize $u_1(x_{2,i})$}& {\footnotesize $u_2(x_{2,i})$}& \!$\cdots$ & {\footnotesize $u_m(x_{2,i})$}\! & \!$\cdots$
        \\
	\hline
	$\frac{\beta}{m}+3\epsilon$ & 0 & $\cdots$  & 0 & 
        0 & $\frac{\beta}{m}+2\epsilon$ & $\cdots$  & 0 & \; $\cdots$\\
    $\frac{\beta}{m}$ & $\frac{\beta}{m}$ & $\cdots$  & $\frac{\beta}{m}$ & 
        0 & $\frac{\beta}{m}+\epsilon$ & $\cdots$ & 0 & \; $\cdots$\\
	\hline
\end{tabular}
%\vspace{10pt}
\begin{tabular}[c]{ c c c c }
{\footnotesize $u_1(x_{m,i})$}& {\footnotesize $u_2(x_{m,i})$}& $\cdots$ & {\footnotesize $u_m(x_{m,i})$}\\
\hline
0 & 0 & $\cdots$  & $\frac{\beta}{m}+2\epsilon$\\
0 & 0 & $\cdots$  & $\frac{\beta}{m}+\epsilon$\\
\hline
\end{tabular}
\vspace{10pt}
\caption{An egoistic election game instance of $m\geq 2$ parties that always has a PSNE ($\beta> 0, n_i=2$ for $i\in\{1,2\}$, $0<\epsilon\ll 1$). The hardmax function is used as the monotone WP function. Profile $(1,1,\ldots, 1)$ is a PSNE, and it has the social welfare~$\beta/m+3\epsilon$, which approaches $1/m$ of that of the optimal profile, $\beta$, when $\epsilon$ is close to~0.}
\label{tab:AlwaysPSNE_HM_2party}
\vspace{-20pt}
\end{center}
\end{table}


%=====================================================================
\subsection{Coalition with Strong Egoism Guarantee}
\label{subsec:PoA_coalition}
%=====================================================================


In the rest of this section, we consider a scenario that party players can unite as coalitions and utility can be shared between the members in a coalition, and furthermore, we assume that the election game is strongly egoistic. We call such a game \emph{the strongly egoistic cooperative election game} (SE-CE game). We will show that the set of SE-CE games collapses on the set of original non-cooperative egoistic election games. 


For the SE-CE game, suppose that we have $m'$ coalitions $\mathcal{C}_1,\mathcal{C}_2,\allowbreak\ldots,\mathcal{C}_{m'}$ of the $m$ party players, such that $m'\leq m$, each coalition $\mathcal{C}_i$ is composed of~$n'_i$ party players and $\mathcal{C}_i\cap\mathcal{C}_j=\emptyset$ for each $i,j\in[m'], i\neq j$. That is, for each $i\in [m']$, $\mathcal{C}_i = \{\mathcal{P}_{i_1}, \mathcal{P}_{i_2}, \ldots,\mathcal{P}_{i_{h_i}}\}$ is composed of~$h_i$ \emph{member parties}. Let $X_{i_j}$ denote the set of candidates of party $\mathcal{P}_{i_j}$ (i.e., $X_{i_j} = \{x_{i_j,1}, x_{i_j, 2},\ldots, x_{i_j, n_{i_j}}\}$). 
For a coalition $\mathcal{C}_i$ of parties, we regard $\mathcal{C}_i$ as a player and consider the \emph{coalition candidate} $z_i$ as a pure strategy of~$\mathcal{C}_i$, where $z_i\in X_{i_1}\times X_{i_2}\times\cdots \times X_{i_{h_i}}$ is a composition of~$h_i$ candidates each of which comes from the corresponding member party of~$\mathcal{C}_i$. Clearly, there are $\prod_{j\in [h_i]}n_{i_j} = O(n^{h_i})$ pure strategies of coalition~$\mathcal{C}_i$.


%To ease the notation, denote by $z_{i,1}, z_{i,2}, \ldots, z_{i,K_i}$ the coalition candidates in~$\mathcal{C}_i$. 
For each $j\in [m']$, denote by~$\tilde{u}_j(z_i)$ the utility of a coalition candidate $z_i$ of $\mathcal{C}_i = \{\mathcal{P}_{i_1}, \mathcal{P}_{i_2}, \ldots,\mathcal{P}_{i_{h_i}}\}$ for coalition~$\mathcal{C}_j$, which is the sum of utility brought by~$z_i$ for the supporters of parties in coalition~$\mathcal{C}_j$. Namely, for any coalition candidate~$z_i$ of~$\mathcal{C}_i$, we define $\tilde{u}_j(z_i) = \sum_{\mathcal{P}_{j_{\ell}}\in \mathcal{C}_j}\sum_{t\in [h_i]} u_{j_{\ell}}(z_i(t))$, where $z_i(t)$ corresponds to the candidate in~$X_{i_t}$. The social utility of~$z_i$ is then $\sum_{t\in [h_i]} u(z_i(t))$. 
By the assumption that the SE-CE game is strongly egoistic, we argue that it collapses on the egoistic election game when each coalition is viewed as a party player in the election game. Indeed, consider any coalition $\mathcal{C}_j\neq \mathcal{C}_i$, for any coalition candidate $z_i = (x_{i_1,s_{i_1}},x_{i_2,s_{i_2}},\ldots,x_{i_{h_i},s_{i_{h_i}}})$  of~$\mathcal{C}_i$ and any coalition candidate $z_j = (x_{j_1,s_{j_1}},x_{j_2,s_{j_2}},\ldots,x_{j_{h_j},s_{j_{h_j}}})$ of~$\mathcal{C}_j$, 
we obtain that 
\begin{eqnarray*}
\tilde{u}_i(z_i) &=& \sum_{\mathcal{P}_{i_{\ell}}\in \mathcal{C}_i}\sum_{t\in [h_i]} u_{i_{\ell}}(z_i(t)) = \sum_{t\in [h_i]}\sum_{\mathcal{P}_{i_{\ell}}\in \mathcal{C}_i} u_{i_{\ell}}(x_{i_t,s_{i_t}})\\
&=& \sum_{t\in [h_i]} \left(u_{i_t}(x_{i_t,s_{i_t}})+\sum_{\mathcal{P}_{i_{\ell}}\in \mathcal{C}_i,\ell\neq t} u_{i_{\ell}}(x_{i_t,s_{i_t}})\right)\\
&>& \sum_{t\in [h_i]} \left(\sum_{\mathcal{P}_{j_{\ell'}}\in \mathcal{C}_j} u_{i_t}(x_{j_{\ell'},s_{j_{\ell'}}}) +\!\sum_{\mathcal{P}_{i_{\ell}}\in \mathcal{C}_i,\ell\neq t} u_{i_{\ell}}(x_{i_t,s_{i_t}})\right)\\
&\geq & \sum_{t\in [h_i]} \sum_{\mathcal{P}_{j_{\ell'}}\in \mathcal{C}_j} u_{i_t}(x_{j_{\ell'},s_{j_{\ell'}}}) 
= \sum_{\mathcal{P}_{i_{\ell}}\in \mathcal{C}_i} \sum_{t\in [h_j]} u_{i_{\ell}}(z_j(t)) = \tilde{u}_i(z_j),
\end{eqnarray*}
where the first inequality follows from the strongly egoistic property and the last two equalities follow by the definition of the social utility of coalition. 


\begin{proposition}\label{prop:SE-CEG_is_egoistic}
The SE-CE game with $m'$ coalitions $\mathcal{C}_1,\mathcal{C}_2,\allowbreak\ldots,\allowbreak\mathcal{C}_{m'}$ for $m'<m$ is the egoistic election game in which $\mathcal{C}_i$ is a party player with utility $\tilde{u}_j(z_{i})$ for each coalition candidate $z_{i}$ of~$\mathcal{C}_i$ and each $i,j\in [m']$. 
\end{proposition}


An implication of Proposition~\ref{prop:SE-CEG_is_egoistic} is that the strongly egoistic election game becomes more efficient when it is ``more  cooperative" in terms of more coalitions of party players. This results in an egoistic election game with less party players and hence a less PoA bound is expected. 



%---------------------------------------------------------------------
\subsubsection*{Incentive of Coalition}
%---------------------------------------------------------------------

%\textcolor{brown}{
Assume that there are $m'$ coalitions $\mathcal{C}_1, \mathcal{C}_2, \allowbreak \ldots, \mathcal{C}_{m'}$ of the~$m$ party players such that $\mathcal{C}_j = \{ \mathcal{P}_{j_1}, \mathcal{P}_{j_2}, \ldots,\mathcal{P}_{j_{h_j}}\}$ for $j\in [m']$. 
Consider a party $\mathcal{P}_{i_{\ell}}\in \mathcal{C}_i$. Naturally, we can consider the payoff of~$\mathcal{P}_{i_{\ell}}$ under the setting of SE-CE game as $\tilde{r}_{i_{\ell}} = \sum_{j\in [m']} p_j(\sum_{t\in [h_j]} u_{i_{\ell}}(z_j(t))) = p_i u_{i_{\ell}}(z_i(\ell)) + \sum_{t\in [h_i]\setminus\{\ell\}} p_i\cdot\allowbreak u_{i_{\ell}}(z_i(t)) + \sum_{j\in [m']\setminus\{i\}} p_j(\sum_{t\in [h_j]} u_{i_{\ell}}(z_j(t)))$, where $p_k$ denotes the winning probability of coalition~$\mathcal{C}_k$, $z_k(t)$ corresponds to the candidate in~$X_{k_t}$ and $z_k = (x_{k_1,s_{k_1}},x_{k_2,s_{k_2}},\ldots,x_{k_{h_k},s_{k_{h_k}}})$ is a coalition candidate of~$\mathcal{C}_k$ for $k\in [m']$. Consider the state in which  $\mathcal{P}_{i_{\ell}}$ is a party player not joining any coalition (i.e., there are $m'+1$ coalitions~$\mathcal{P}_{i_{\ell}}, \mathcal{C}_1, \mathcal{C}_2, \allowbreak \ldots, \mathcal{C}_{m'}$ such that $\mathcal{P}_{i_{\ell}}\notin \mathcal{C}_j$ for $j\in [m']$) and let $\tilde{r}'_{i_{\ell}}$ be the payoff of $\mathcal{P}_{i_{\ell}}$ in this state. %(We may want to review the definition of $\tilde{r}_{i_{\ell}}$ as well?) 
We have $\tilde{r}'_{i_{\ell}} = \hat{p}_i u_{i_{\ell}}(z_i(\ell)) + \sum_{t\in [h_i]\setminus\{\ell\}} p'_i u_{i_{\ell}}(z_i(t))) + \sum_{j\in [m']\setminus\{i\}} p'_j(\sum_{t\in [h_j]} u_{i_{\ell}}(z_j(t)))$, %\break
where $\hat{p}_i$ denotes the winning probability of $\mathcal{P}_{i_{\ell}}$ and $p'_j$ denotes the winning probability of coalition~$\mathcal{C}_j$ for $j\in [m']$. %The strong egoism of the game implies that 
%$\tilde{r}'_{i_{\ell}}\leq (1-\sum_{j\in [m']\setminus\{i\}}p'_j) u_{i_{\ell}}(z_i(\ell)) + \sum_{j\in[m']\setminus\{i\}} p'_j (\sum_{t\in [h_j]} u_{i_{\ell}}(z_j(t)))$. 
Then, we derive that \begin{eqnarray*}
\tilde{r}'_{i_{\ell}} - \tilde{r}_{i_{\ell}} &=& 
\sum_{j\in [m']\setminus\{i\}} (p'_j-p_j)\left(\sum_{t\in [h_j]} u_{i_{\ell}}(z_j(t))\right) + \\
& & (p'_i - p_i)\!\!\sum_{t\in [h_i]\setminus\{\ell\}}u_{i_{\ell}}(z_i(t)) + (\hat{p}-p_i)u_{i_{\ell}}(z_i(\ell))\\
&\leq& \sum_{j\in [m']\setminus\{i\}} (p'_j-p_j)\left(\left(\sum_{t\in [h_j]} u_{i_{\ell}}(z_j(t))\right) - u_{i_{\ell}}(z_i(\ell))\right) \leq 0,     
\end{eqnarray*}
where the first inequality follows from $p'_i\leq p_i$ (since the size of coalition $\mathcal{C}_i$ gets smaller) and $\hat{p}-p_i = \sum_{j\in [m']\setminus\{i\}}(p_j-p'_j) - p'_i\leq \sum_{j\in [m']\setminus\{i\}}(p_j-p'_j)$, and the second inequality follows from the strong egoism. Intuitively, coalition increases the winning probability of a ``party player", which benefits its supporters more than the aggregation of the candidates from the other party players. The observation above (i.e., $\tilde{r}'_{i_{\ell}} - \tilde{r}_{i_{\ell}} \leq 0$) reveals the possible incentive of a party player to form a coalition with the other parties.%}   


%%%%%%%%%%%%%%%%%%%%%%%%%%%%%%%%%%%%%%%%%%%%%%%%%%%%%%%%%%%%%%%%%%%%%%
\section{Conclusions and Future Work}%Concluding Remarks}
\label{sec:future}
%%%%%%%%%%%%%%%%%%%%%%%%%%%%%%%%%%%%%%%%%%%%%%%%%%%%%%%%%%%%%%%%%%%%%%


From the perspectives of PSNE existence and the PoA, unlike the two-party case, we have learned that the election game is ``bad" when more than two parties are involved in the sense that the PSNE is no longer guaranteed to exist and the PoA can be in proportional to the number of competing parties. 
Our work provides an alternative explanation why the two-party system is prevalent in democratic countries. 

%We prove that to determine if the election game has a PSNE is {\sf NP}-complete, even for the egoistic election game. Nevertheless, two parameters of the election game, that is, the number of irresolute parties and the nominating depth, are extracted in this work and utilized to devise an efficient parameterized algorithm for computing a PSNE of the egoistic election game. In addition, when we consider mixed-strategy Nash equilibrium of the egoistic election game, the support of each party player can also be identified by slightly modifying Algorithm {\sf FPT-ELECTION-PSNE}. We expect these two parameters to be of independent interest.  


We have shown that the PoA of the egoistic election game is upper bounded by the number of parties~$m$, and this also improves the previous bound in~\cite{LLC2021} using the softmax WP function for the two-party case. 
%Our PoA bound is tight for the egoistic two-party election game using either the hardmax or the softmax function as the WP function, and is also tight for general $m\geq 2$ when the hardmax function is applied as the WP function. 
%It will be interesting to know if the PoA bound is tight for all monotone WP functions and for general $m\geq 2$. 
Nevertheless, PoA provides worst-case measure of inefficiency of a game, so it is also interesting to know if an election game of more party players is still efficient in average over all game instances. 
 

%Coalition between the parties and thus strong equilibria can also be further considered. 
%In Sect.~\ref{subsec:PoA_coalition} 
Under the strongly egoism assumption, we regard a coalition as a unit which receives the whole payoff. Naturally, one might argue that not every member party player in a coalition is ``equally happy". We plan to consider individual strategic behaviors of each party player including the choice of parties for the coalition, and moreover, the payoff of each individual party player will be considered separately. Computation of such a coalition deserves further investigation. %Whether the PoA is getting higher or lower when coalition is allowed deserves further investigation.   

%%%%%%%%%%%%%%%%%%%%%%%%%%%%%%%%%%%%%%%%%%%%%%%%%%%%%%%%%%%%%%%%%%%%%%%%

%%% The acknowledgments section is defined using the "acks" environment
%%% (rather than an unnumbered section). The use of this environment 
%%% ensures the proper identification of the section in the article 
%%% metadata as well as the consistent spelling of the heading.

%%%%%%%%%%%%%%%%%%%%%%%%%%%%%%%%%%%%%%%%%%%%%%%%%%%%%%%%%%%%%%%%%%%%%%%%

%%% The next two lines define, first, the bibliography style to be 
%%% applied, and, second, the bibliography file to be used.


\bibliographystyle{splncs04} 
\bibliography{election_game_2024}

%%%%%%%%%%%%%%%%%%%%%%%%%%%%%%%%%%%%%%%%%%%%%%%%%%%%%%%%%%%%%%%%%%%%%%%%

\newpage
\appendix

%%%%%%%%%%%%%%%%%%%%%%%%%%%%%%%%%%%%%%%%%%%%%%%%%%%%%%%%%%%%%
\section{Omitted Definitions \& Proofs}
\label{sec:appendix_A}
%%%%%%%%%%%%%%%%%%%%%%%%%%%%%%%%%%%%%%%%%%%%%%%%%%%%%%%%%%

%==========================================================
\subsection{Proof of Lemma~\ref{lem:fractional_monotone}}
%==========================================================

\begin{proof}
Let $r, s > 0$ be two positive real numbers such that $r<s$. Then $r/s - (r+d)/(s+d) = d(r-s)/(s^2+sd)$. Clearly, $r/s - (r+d)/(s+d) < 0$ if $r < s$. 
%The lemma is then proved. 
\qed
\end{proof}


%==========================================================
\subsection{Proof of Theorem~\ref{thm:NPC}}
\label{subsec:NPC_proof}
%==========================================================


\begin{proof}
To check whether a profile is a PSNE requires at most $\sum_{i=1}^m n_i = O(mn)$ time by examining all unilateral deviations. Hence, the problem of determining if the egoistic election game has a PSNE is in \textsf{NP}. In the following, we prove the \textsf{NP}-hardness by a reduction from the satisfiability problem (SAT). Consider an arbitrary instance of SAT with a formula $F$ which contains a set of~$m$ variables $X = \{v_1,v_2,\ldots,v_m\}$. Let $a = (a_1,a_2,\ldots,a_m)$ be a truth assignment of~$X$ such that $a_i$ is the truth assignment of variable~$v_i$ for $i\in [m]$. Hence, $F$ is satisfiable if there exists an assignment $a$ of~$X$ such that $F(a) = \mbox{\tt True}$. 
Set the value of $\beta$ to be~$200$.
For $i\in \{1,2,\ldots, m-2\}$, we construct $m$ parties each of which has two candidates $x_{i,1}$ and $x_{i,2}$ such that $u_i(x_{i,1}) = u_i(x_{i,2}) = \epsilon$ and $u_j(x_{i,1}) = u_j(x_{i,2}) = 0$ for $j\neq i$ and $0<\epsilon<1$. Then, let \[
\begin{array}{cc}
u_{m-1}(x_{m-1,1}) = 83, & u_{m}(x_{m-1,1}) = 1\\
u_{m-1}(x_{m-1,2}) = 80, & u_{m}(x_{m-1,1}) = 19\\
u_{m-1}(x_{m,1}) = 3, & u_{m}(x_{m,1}) = 24\\
u_{m-1}(x_{m,2}) = 9, & u_{m}(x_{m,2}) = 22
\end{array}
\]
and $u_j(x_{m-1,1}) = u_j(x_{m-1,2}) = u_j(x_{m,1}) = u_j(x_{m,2}) = 0$ for $1\leq j\leq m-2$. We define the WP function as 
\[
p_{i,\mathbf{s}} = 
\left\{\begin{array}{ll}
\frac{(u(x_{i,\mathbf{s}_i})/{\beta})^{1-f(a)/10}}{\sum_{j\in [m], u(x_{j,\mathbf{s}_j})>1} (u(x_{j,\mathbf{s}_j})/\beta)^{1-f(a)/10}} & \mbox{ if } u(x_{i,\mathbf{s}_i}) > 1,\\
0 & \mbox{ otherwise,}
\end{array}\right.
\]
where $f(a) = 1$ if $F(a) = \mbox{\tt True}$ and $f(a) = 0$ otherwise. Here $a_i = \mbox{\tt True}$ and $a_i = \mbox{\tt False}$ correspond to the candidates $x_{i,1}$ and $x_{i,2}$, respectively. 
Clearly, for $1\leq i\leq m-2$ we have $r_{i,\mathbf{s}} = 0$ for any profile $\mathbf{s}$. Let $\mathbf{s}'$ denote the profile without the $(m-1)$th and the $m$th party. 
Then, if $\mathbf{s}$ corresponds to the assignment $a$ of $X$ such that $F(a) = {\tt True}$, then
\begin{center}
\begin{tabular}[c]{ l l | l l }
	\centering
	%\multicolumn{2}{l}{payoff matrix} \vspace{7pt}\\
	%\hline
	{\footnotesize $r_{m-1,(\mathbf{s}',s_{\scriptsize m-1}=1,s_m=1)}$} & {\footnotesize $r_{m,(\mathbf{s}',s_{\scriptsize m-1}=1,s_m=1)}$}  
     & {\footnotesize $r_{m-1,(\mathbf{s}',s_{\scriptsize m-1}=1,s_m=2)}$} & {\footnotesize $r_{m,(\mathbf{s}',s_{\scriptsize m-1}=1, s_m=2)}$} \\
	\hline
	{\footnotesize $r_{m-1,(\mathbf{s}',s_{\scriptsize m-1}=2, s_m=1)}$} & {\footnotesize $r_{m,(\mathbf{s}',s_{\scriptsize m-1}=2, s_m=1)}$} 
     & {\footnotesize $r_{m-1,(\mathbf{s}',s_{\scriptsize m-1}=2,s_m=2)}$} & {\footnotesize $r_{m,(\mathbf{s}',s_{\scriptsize m-1}=2,s_m=2)}$} 
	%\hline
\end{tabular}
\end{center}

\begin{center}
$\approx$
\begin{tabular}[c]{ l l | l l }
	%\multicolumn{2}{c}{} \vspace{7pt}\\
	\centering
	% & \\
	%\hline
	61.82 \; &  7.09 \;  & \, 61.57  \; &  7.08 \\
	\hline
	61.75 \; &  20.18 \; & \, 61.53 \; & 19.78 
	%\hline
\end{tabular}
\end{center}
where there exists a PSNE for any $\mathbf{s}'$ and $s_{m-1}=1, s_m = 1$. And, if $\mathbf{s}$ corresponds to the assignment $a$ of $X$ such that $F(a) = {\tt False}$, then
\begin{center}
\begin{tabular}[c]{ l l | l l }
	\centering
	%\multicolumn{2}{l}{payoff matrix} \vspace{7pt}\\
	%\hline
	{\footnotesize $r_{m-1,(\mathbf{s}',s_{m-1}=1,s_m=1)}$} & {\footnotesize $r_{m,(\mathbf{s}',s_{m-1}=1,s_m=1)}$} 
     &  {\footnotesize $r_{m-1,(\mathbf{s}',s_{m-1}=1,s_m=2)}$} & {\footnotesize $r_{m,(\mathbf{s}',s_{m-1}=1,s_m=2)}$} \\
	\hline
	{\footnotesize $r_{m-1,(\mathbf{s}',s_{m-1}=2,s_m=1)}$} & {\footnotesize $r_{m,(\mathbf{s}',s_{m-1}=2,s_m=1)}$}
     & {\footnotesize $r_{m-1,(\mathbf{s}',s_{m-1}=2,s_m=2)}$} & {\footnotesize $r_{m,(\mathbf{s}',s_{m-1}=2,s_m=2)}$} 
	%\hline
\end{tabular}
\end{center}
\begin{center}
$\approx$
\begin{tabular}[c]{ l l | l l }
	%\multicolumn{2}{c}{} \vspace{7pt}\\
	\centering
	% & \\
	%\hline
	63.54 \; &  6.59 \;  & \, 63.05  \; &  6.66\\
	\hline
	63.50 \; &  20.07 \; & \, 63.07 \; & 19.72 
	%\hline
\end{tabular}
\end{center}
where there is no PSNE. Next, we show that the game is monotone. For $F(a) = {\tt True}$, 
\begin{center}
\begin{tabular}[c]{ l l | l l }
	\centering
	%\multicolumn{2}{l}{payoff matrix} \vspace{7pt}\\
	%\hline
	{\footnotesize $p_{m-1,(\mathbf{s}',s_{\scriptsize m-1}=1,s_m=1)}$} & {\footnotesize $r_{m,(\mathbf{s}',s_{\scriptsize m-1}=1,s_m=1)}$} 
     &  {\footnotesize $p_{m-1,(\mathbf{s}',s_{\scriptsize m-1}=1,s_m=2)}$} & {\footnotesize $r_{m,(\mathbf{s}',s_{\scriptsize m-1}=1,s_m=2)}$} \\
	\hline
	{\footnotesize $p_{m-1,(\mathbf{s}',s_{\scriptsize m-1}=2,s_m=1)}$} & {\footnotesize $r_{m,(\mathbf{s}',s_{\scriptsize m-1}=2,s_m=1)}$} 
     & {\footnotesize $p_{m-1,(\mathbf{s}',s_{\scriptsize m-1}=2,s_m=2)}$} & {\footnotesize $r_{m,(\mathbf{s}',s_{\scriptsize m-1}=2,s_m=2)}$} 
	%\hline
\end{tabular}
\end{center}
\begin{center}
$\approx$
\begin{tabular}[c]{ l l | l l }
	%\multicolumn{2}{c}{} \vspace{7pt}\\
	\centering
	% & \\
	%\hline
	0.7353 \; &  0.2647 \;  & \, 0.7104 \; &  0.2896 \\
	\hline
	0.7630 \; &  0.2370 \; & \, 0.7398 \; & 0.2602 
	%\hline
\end{tabular}
\end{center}
and for $F(a) = {\tt False}$, 
\begin{center}
\begin{tabular}[c]{ l l | l l }
	\centering
	%\multicolumn{2}{l}{payoff matrix} \vspace{7pt}\\
	%\hline
	{\footnotesize $p_{m-1,(\mathbf{s}',s_{\scriptsize m-1}=1,s_m=1)}$} & {\footnotesize $r_{m,(\mathbf{s}',s_{\scriptsize m-1}=1,s_m=1)}$} 
     & {\footnotesize $p_{m-1,(\mathbf{s}',s_{\scriptsize m-1}=1,s_m=2)}$} & {\footnotesize $r_{m,(\mathbf{s}',s_{\scriptsize m-1}=1,s_m=2)}$} \\
	\hline
	{\footnotesize $p_{m-1,(\mathbf{s}',s_{\scriptsize m-1}=2,s_m=1)}$} & {\footnotesize $r_{m,(\mathbf{s}',s_{\scriptsize m-1}=2,s_m=1)}$} 
     & {\footnotesize $p_{m-1,(\mathbf{s}',s_{\scriptsize m-1}=2,s_m=2)}$} & {\footnotesize $r_{m,(\mathbf{s}',s_{\scriptsize m-1}=2,s_m=2)}$} 
	%\hline
\end{tabular}
\end{center}
\begin{center}
$\approx$
\begin{tabular}[c]{ l l | l l }
	%\multicolumn{2}{c}{} \vspace{7pt}\\
	\centering
	% & \\
	%\hline
	0.7568 \; &  0.2432 \;  & \, 0.7304 \; &  0.2696 \\
	\hline
	0.7857 \; &  0.2143 \; & \, 0.7615 \; & 0.2385 
	%\hline
\end{tabular}.
\end{center}
From the construction of the game, we have $u(x_{m-1,1}) = 83+1=84$, $u(x_{m-1,2}) = 80+19=99$, $u(x_{m,1}) = 3+24=27$, and $u(x_{m,2}) = 9+22=31$, which means that for either the $(m-1)$th or the $m$th party, their second candidate brings more social utility to all the voters than the first does. Consider the case that $F(a) = {\tt True}$. When the $(m-1)$th party switches the candidate $x_{m-1,1}$ to~$x_{m-1,2}$, the winning probability of the party either increases from~0.7353 to~0.7630 (if the corresponding assignment of the resulting profile remains {\tt True}) or~0.7857 (if the corresponding assignment of the resulting profile becomes {\tt False}). 
Consider the case that $F(a) = {\tt False}$.
When the $(m-1)$th party switches the candidate $x_{m-1,1}$ to~$x_{m-1,2}$, the winning probability of the party either increases from~0.7568 to~0.7857 (if the corresponding assignment of the resulting profile remains {\tt False}) or~0.7630 (if the corresponding assignment of the resulting profile becomes {\tt True}). The rest cases can be similarly checked. Hence, we conclude that the constructed game is monotone. 


The construction of the game clearly takes polynomial time and the game is monotone. Therefore, the theorem is proved.
\qed
\end{proof}
%==========================================================
\subsection{Parameterized Problems and Algorithms}
%==========================================================


In the following we provide the formal definitions of a parameterized problem and a fixed-parameter tractability to make our discussions self-contained. %Definition~\ref{defn:parameterized_problems}. ~\ref{defn:fpt}. 


\begin{definition}[\cite{DF13,FG06,Nie06}]
\label{defn:parameterized_problems}
A parameterized problem is a language $\mathcal{L}\subseteq \Sigma^* \times \Sigma^*$, where $\Sigma$ is a finite alphabet. The second component is called the parameter(s) of the problem.
\end{definition}


\begin{definition}[\cite{DF13,FG06,Nie06}]
\label{defn:fpt}
A parameterized problem $L$ is fixed-parameter tractable (FPT) if, for all $(x, k)\in \mathcal{L}$, whether $(x,k)\in \mathcal{L}$ can be determined in $f(k)\cdot n^{O(1)}$ time, where
$f$ is a computable function that depends only on~$k$.
\end{definition}



%==========================================================
\subsection{Proof of Theorem~\ref{thm:dominatingNE}}
\label{subsec:proof_dominatingNE}
%==========================================================


\begin{proof}
For the first case, let us consider an arbitrary $i\in [m]$ and an arbitrary $t\in [n_i]\setminus\{1\}$. Denote by $\mathbf{s}$ the profile $(x_{1,1}, x_{2,1}, \ldots, x_{m,1})$. 
Since strategy $x_{i,1}$ weakly surpasses $x_{i,t}$, we have $p_{i,\mathbf{s}}\geq p_{i,(t,\mathbf{s}_{-i})}$ and then $\sum_{j\in [m]\setminus\{i\}} p_{j,\mathbf{s}}\leq \sum_{j\in [m]\setminus\{i\}} p_{j,(t,\mathbf{s}_{-i})}$. Hence, 
\begin{eqnarray*}
& &r_{i,\mathbf{s}} - r_{i,(t, \mathbf{s}_{-i})} 
= 
\sum_{j\in [m]} p_{j,\mathbf{s}} u_i(x_{j,1})\\ 
& &- \left(\sum_{j\in [m]\setminus\{i\}} p_{j,(t, \mathbf{s}_{-i})} u_i(x_{j,1}) + p_{i,(t, \mathbf{s}_{-i})} u_i(x_{i,t}) \right)\\
&=& \sum_{j\in [m]\setminus\{i\}} u_i(x_{j,t})(p_{j,\mathbf{s}}-p_{j,(t,\mathbf{s}_{-i})})\\
& & + (p_{i,\mathbf{s}} u_i(x_{i,1}) - p_{i,(t,\mathbf{s}_{-i})} u_i(x_{i,t}))\\
&\geq& \sum_{j\in [m]\setminus\{i\}} u_i(x_{i,t})(p_{j,\mathbf{s}}-p_{j,(t,\mathbf{s}_{-i})}) +  u_i(x_{i,t}) (p_{i,\mathbf{s}} - p_{i,(t,\mathbf{s}_{-i})})\\
&=& u_i(x_{i,t})\left( \sum_{j\in [m]\setminus\{i\}} p_{j,\mathbf{s}} - p_{j,(t,\mathbf{s}_{-i})} + (p_{i,\mathbf{s}} - p_{i,(t,\mathbf{s}_{-i})})\right)\\
&=& u_i(x_{i,t})\left(\sum_{j\in [m]} p_{j,\mathbf{s}} - \sum_{j\in [m]} p_{j,(t,\mathbf{s}_{-i})} \right)\\
&=& 0, 
\end{eqnarray*}
where the inequality follows from the egoistic property and the assumption that strategy~$x_{i,1}$ weakly surpasses~$x_{i,t}$, and last equality holds since $\sum_{j\in [m]} p_{j,\mathbf{s}} = \sum_{j\in [m]} p_{j,(t,\mathbf{s}_{-i})} = 1$ by the law of total probability. Thus, party player $i$ has no incentive to deviate from its current strategy. Note that the uniqueness of the PSNE comes when the inequality becomes ``greater-than". This happens as strategy~$x_{i,1}$ surpasses~$x_{i,t}$. 

For the second case, by the same arguments for the first case, we know that for $i\in \mathcal{I}$, party player $i$ has no incentive to deviate from its current strategy. Let $j\in [m]\setminus\mathcal{I}$ be the only one party player not in~$\mathcal{I}$. As $s_j^{\#}$ is the best response when the other strategies of parties in $\mathcal{I}$ are fixed, party player $j$ has no incentive to deviate from its current strategy. Therefore, the theorem is proved. 
\qed
\end{proof}


%==========================================================
\subsection{Proof of Lemma~\ref{lem:dominated_Not_PSNE}}
%==========================================================


\begin{proof}
Let $\mathbf{s}_{-i} = (\tilde{s}_j)_{j\in [m]\setminus\{i\}}$ be any profile except party player $\mathcal{P}_i$'s strategy. 
If $s_i$ is surpassed by $s'_i$, then we have $s'_i < s_i$, and either:
\begin{itemize}
\item [(a)] $u(x_{i,s'_i}) > u(x_{i,s_i})$ and $u_i(x_{i,s'_i})\geq u_i(x_{i,s_i})$, or
\item [(b)] $u(x_{i,s'_i})\geq u(x_{i,s_i})$ and $u_i(x_{i,s'_i})>u_i(x_{i,s_i})$. 
\end{itemize}
We prove case (a) as follows. Case (b) can be similarly proved. 

We have $u_i(x_{i,s'_i})\geq u_i(x_{i,s_i})$ from the assumption in (a). Also, by the monotone property of the WP functions, we have $p_{i,(s'_i, \mathbf{s}_{-i})} \geq p_{i,(s_i,\mathbf{s}_{-i})}$ and $\sum_{j\in [m]\setminus\{i\}} p_{j, (s'_i, \mathbf{s}_{-i})} \leq \sum_{j\in [m]\setminus\{i\}} p_{j, (s_i,\mathbf{s}_{-i})}$ since $u(x_{i,s'_i}) > u(x_{i,s_i})$. Thus,
\begin{eqnarray*}
& & r_{i}(s_i,\mathbf{s}_{-i}) - r_{i}(s'_i,\mathbf{s}_{-i})\\ 
&=& \left(p_{i,(s_i,\mathbf{s}_{-i})}u_i(x_{i,s_i}) + \sum_{j\in [m]\setminus\{i\}} p_{j,(s_i,\mathbf{s}_{-i})}u_i(x_{j,\tilde{s}_j})\right) \\ 
& & - \left(p_{i,(s'_i,\mathbf{s}_{-i})}u_i(x_{i,s'_i})+\sum_{j\in [m]\setminus\{i\}} p_{j,(s'_i,\mathbf{s}_{-i})}u_i(x_{j,\tilde{s}_j})\right)\\
&=& \left(\sum_{j\in [m]\setminus\{i\}} (p_{j,(s_i,\mathbf{s}_{-i})} - p_{j,(s'_i,\mathbf{s}_{-i})}) u_i(x_{j,\tilde{s}_j})\right) \\
& & + (p_{i,(s_i,\mathbf{s}_{-i})}u_i(x_{i,s_i}) - (p_{i,(s'_i,\mathbf{s}_{-i})}u_i(x_{i,s'_i}))\\
&<& u_i(x_{i,s_i})\left(\sum_{j\in [m]\setminus\{i\}} (p_{j,(s_i,\mathbf{s}_{-i})} - p_{j,(s'_i,\mathbf{s}_{-i})})\right)\\ 
& & + (p_{i,(s_i,\mathbf{s}_{-i})}u_i(x_{i,s_i}) - (p_{i,(s'_i,\mathbf{s}_{-i})}u_i(x_{i,s'_i}))\\
&=& u_i(x_{i,s_i})(p_{i,(s'_i,\mathbf{s}_{-i})} - p_{i,(s_i,\mathbf{s}_{-i})}) + (p_{i,(s_i,\mathbf{s}_{-i})}u_i(x_{i,s_i})\\
& & - p_{i,(s'_i,\mathbf{s}_{-i})}u_i(x_{i,s'_i}))\\
&\leq& (p_{i,(s'_i,\mathbf{s}_{-i})} - p_{i,(s_i,\mathbf{s}_{-i})})(u_i(x_{i,s_i})-u_i(x_{i,s_i})) = 0,
\end{eqnarray*}
where the first inequality follows from the egoistic property. 
Thus, $(s_i,\mathbf{s}_{-i})$ is not a PSNE.
\qed
\end{proof}


%==========================================================


%==========================================================
\subsection{Shrinking the Nomination Depth for Each Party Player}
\label{subsec:strategy_refined}
%==========================================================

First, we denote by $L_i^{0}$ the set $[n_i]$ for party player $\mathcal{P}_i$. 
Recall that $x_{i,d_i}$ is a candidate which surpasses the candidates in $\{x_{i, d_i+1},\ldots,x_{i, n_i}\}$ if $d_i\neq n_i$ otherwise $x_{i, d_i} = x_{i,n_i}$. Let us denote $x_{i,d_i}$ by~$z_{i,1}$ to facilitate our discussion. We can apply the same approach to identify the candidate $z_{i,2} = x_{i,d'_i}$ if there exists $d'\in [1, d_i-1)$ such that $x_{i,d'_i}$ surpasses candidates in~$\{x_{i,d_i'+1},\ldots, x_{i,d_i-1}\}$ and set $z_{i,2} = x_{i,d_i-1}$ otherwise (see Figure~\ref{fig:strategy_reduce}). 
By repeatedly calculating $z_{i,1}, z_{i,2},\ldots,z_{i,\tilde{d}_i}$, in which $\tilde{d}_i$ is the number of proceeded repetitions and $z_{i,\tilde{d}_i} = x_{i,1}$. Clearly,  $\tilde{d}_i$ is upper bounded by~$d_i$. As it suffices to consider candidates that are not dominated by any other one for each party to seek a PSNE, we can consider $\{z_{i,1},z_{i,2},\ldots,z_{i, \tilde{d}_i}\}$ as the the strategy set of $\mathcal{P}_i$. 

\begin{figure}[ht]
    \centering
    \includegraphics[scale=0.32]{FPTstrategySetConstruction.eps}
    \caption{Reducing the Strategy Set for $\mathcal{P}_i$.}
    \label{fig:strategy_reduce}
\end{figure}

The nominating depth $d = \max_{i\in [m]}\{d_i\}$ can be substituted by $\tilde{d} := \max_{i\in [m]} \tilde{d}_i$. Clearly, $\tilde{d}\leq d$. Thus, using the refined strategy set $\{\hat{s}_1,\hat{s}_2,\ldots,\hat{s}_{\tilde{d}_i}\}$ for each party player $\mathcal{P}_i$ results in possibly more efficient FPT algorithm (with a possibly smaller exponential base~$\tilde{d}$), though the time complexity in the worst case still coincides. 


%==========================================================
\subsection{Proof of Theorem~\ref{thm:PSNE_compute_tractable}}
\label{subsec:FPT_complexity_proof}
%==========================================================



\begin{proof}
It costs $O(nm^2)$ time to compute $d$ and $d_i$ for all~$i$, and the set $\mathcal{D}$ can be then obtained. Since playing strategy~1 (i.e., nominating the first candidate) for parties $i\in\mathcal{D}$ is the dominant strategy, we only need to consider party players in~$[m]\setminus\mathcal{D}$. Since for each $i\in [m]\setminus\mathcal{D}$, each strategy in~$\{x_{i,d_i+1},\ldots,x_{i,n_i}\}$ is surpassed by $x_{i,d_i}$ (considering the subgame with respect to~$(x_{i,d_i+1},\ldots,x_{i,n_i})_{i\in [m]}$), by Lemma~\ref{lem:dominated_Not_PSNE} we know that we only need to consider strategies $\{x_{i,1},\ldots,x_{i,d_i}\}$ for such party player~$i$. Thus, the number of entries of the implicit payoff matrix
that we need to check whether it is a PSNE is $\prod_{i\in [m]\setminus\mathcal{D}} d_i = O(d^k)$, where $k = |[m]\setminus \mathcal{D}|$. 
For each of these entries, it takes $O(m)$ time to compute the winning probabilities as well as the payoffs of the party players in~$[m]\setminus\mathcal{S}$. 
In addition, checking whether each of such entries can be done in at most~$k(d-1)$ steps. Therefore, the theorem is proved.
\qed
\end{proof}


%==========================================================
\subsection{Proof of Theorem~\ref{thm:approx}}
\label{subsec:approx_proof}
%==========================================================


\begin{proof}
Let $\mathbf{s} = (x_{1,1}, x_{2,1},\ldots,x_{m, 1})$ denote the profile that each party player $\mathcal{P}_{\ell}$ chooses their first candidate $x_{\ell,1}$ for $\ell\in [m]$. Consider an arbitrary party player $\mathcal{P}_i$ and its arbitrary candidate, say $x_{i, r}$, other than $x_{i,1}$. For each $\ell\in [m]$, let $p_{\ell}$ denote the winning probability of $\mathcal{P}_{\ell}$ and let $p_{\ell}'$ denote the winning probability of~$\mathcal{P}_{\ell}$ when $\mathcal{P}_i$ unilaterally deviates from~$x_{i,1}$ to~$x_{i,r}$. Note that we can focus on the case that $p_i'> p_i$ since otherwise $x_{i,r}$ is weakly surpassed by~$x_{i,1}$ and then $\mathcal{P}_i$ has no incentive to deviate from~$x_{i,1}$ to~$x_{i,r}$. Compute the gain of $\mathcal{P}_i$ from such a strategy deviation, we have 
\begin{eqnarray*}
& & \left(p_i' u_i(x_{i,r})+\sum_{j\in [m]\setminus\{i\}} p_j' u_i(x_{j,1})\right) - \sum_{\ell\in [m]} p_{\ell} u_i(x_{\ell, 1}) \\
&=& (p_i' u_i(x_{i,r}) - p_i u_i(x_{i,1})) + \sum_{j\in [m]\setminus\{i\}} (p_j'-p_j) u_i(x_{j,1}) \leq \sum_{\ell\in [m]} (p_{\ell}'-p_{\ell}) u_i(x_{\ell,1}),
\end{eqnarray*}
where the inequality follows since $u_i(x_{i,1})\geq u_i(x_{i,r})$ for $r\neq 1$.
The ratio of the payoff improvement appears to be upper bounded by~
\begin{eqnarray}
\frac{\sum\limits_{\ell\in [m]} (p_{\ell}'-p_{\ell}) u_i(x_{\ell,1})}{\sum\limits_{\ell\in [m]} p_{\ell} u_i(x_{\ell,1})} 
&=& \frac{\sum\limits_{\ell\in [m]} p_{\ell}' u_i(x_{\ell,1})}{\sum\limits_{\ell\in [m]} p_{\ell} u_i(x_{\ell,1})} - 1\nonumber
= \frac{p_i' u_i(x_{i,1})+\sum\limits_{j\in[m]\setminus\{i\}} p_j' u_i(x_{j,1})}{p_i u_i(x_{i,1}) + \sum\limits_{j\in[m]\setminus\{i\}} p_j u_i(x_{j,1})} - 1\nonumber\\
&\leq& \frac{p_i' u_i(x_{i,1})+\sum\limits_{j\in[m]\setminus\{i\}} p_j u_i(x_{j,1})}{p_i u_i(x_{i,1}) + \sum\limits_{j\in[m]\setminus\{i\}} p_j u_i(x_{j,1})} - 1\nonumber
\leq \frac{p_i'}{p_i} - 1\nonumber%\label{eqn:approx}, 
\end{eqnarray}
where the first inequality holds since $p_j'\leq p_j$.
%In Equation~\ref{eqn:approx}, b
By the definition of the softmax function, we obtain that $p_i'/p_i$ is upper bounded by~$$\frac{e^1/(e^1+C)}{e^0/(e^0+C)} = \frac{e(1+C)}{e+C},$$
where $C$ denotes $\sum_{j\in [m]\setminus\{i\}}e^{u(x_{j,1})/\beta}\leq e(m-1)$.
Let $f(x):= e(1+x)/(e+x)$. Taking the first and second derivatives of~$f$, we have $f'(x) = e(e-1)/(e+x)^2 > 0$ and $f''(x) = -2e(e-1)(e+x)/(e+x)^4 < 0$ for all $x>0$. Substituting $x = e(m-1)$ we have 
\begin{eqnarray*}
   f(e(m-1)) &\leq& \frac{e(1+e(m-1))}{e+e(m-1)} \leq e+1. 
\end{eqnarray*}
Hence, the approximation ratio is upper bounded by~$1+e$ for each party player choosing the first candidate in the egoistic election game using the softmax function as the WP function.
\qed    
\end{proof}



\iffalse
%==========================================================
\subsection{Proof of Theorem~\ref{thm:PoA_BT}}
\label{subsec:PoA_BT}
%==========================================================

\begin{proof}
Let $\mathbf{s} = (s_i)_{i\in [m]}$ be a PSNE and $\mathbf{s}^* = (s_i^*)_{i\in [m]}$ be the optimal profile, where $s_i^*:= x_{i, s_i^*}$ when the context is clear.
Note that $SW(\mathbf{s}^*) = \sum_{i\in [m]} p_{i,\mathbf{s}^*} u(s_i^*)\leq \max_{i\in [m]} u(s_i^*)$. 
By the Cauchy-Schwarz inequality, we derive that  
$\sum_{i\in [m]} u(s_i)^2 \geq (\sum_{i\in [m]} u(s_i))^2/m$. 
Then, we have
\begin{eqnarray*}
SW(\mathbf{s})&=& \sum_{i\in [m]}\frac{u(s_i)}{\sum_{j\in [m]} u(s_j)}\cdot u(s_i) = \frac{\sum_{i\in [m]} u(s_i)^2}{\sum_{j\in [m]} u(s_j)}\\ 
&\geq& \frac{1}{m}\cdot \sum_{i\in [m]}u(s_i).
\end{eqnarray*}

Together, by Proposition~\ref{pro:cases} that $\sum_{i\in [m]}u(s_i)\geq \max_{i\in [m]} u(s_i^*)$, we obtain that $SW(\mathbf{s})\geq SW(\mathbf{s}^*)/m$, Thus, the PoA is at most~$m$. 
\qed
\end{proof}
\fi


\iffalse
%\newpage
%%%%%%%%%%%%%%%%%%%%%%%%%%%%%%%%%%%%%%%%%%%%%%%%%%%%%%%%%%%%%
\section{Omitted Tables}
\label{sec:appendix_B}
%%%%%%%%%%%%%%%%%%%%%%%%%%%%%%%%%%%%%%%%%%%%%%%%%%%%%%%%%%%%%



\begin{table}[ht]
\begin{center}
\begin{tabular}[c]{ l l l | l l l | l l l }
	%\centering
	{\footnotesize $u_1(x_{1,i})$}& {\footnotesize $u_2(x_{1,i})$}& {\footnotesize $u_3(x_{1,i})$} & 
        {\footnotesize $u_1(x_{2,i})$}& {\footnotesize $u_2(x_{2,i})$}& {\footnotesize $u_3(x_{2,i})$} & 
        {\footnotesize $u_1(x_{3,i})$}& {\footnotesize $u_2(x_{3,i})$}& {\footnotesize $u_3(x_{3,i})$}
        \\
	\hline
	50  &  9   &  3  &  22  &  42  &  15  &  36  &   1  &  44 \\
        44  &  22  &  30 &  19  &  40  &  32  &  33  &  18  &  42 \\
	\hline
\end{tabular}
\vspace{7pt}\\
\begin{tabular}[c]{ l l l | l l l}
	\centering
	%\multicolumn{2}{l}{payoff matrix} \vspace{7pt}\\
	%\hline
	$r_{1,(1,1,1)}$ & $r_{2,(1,1,1)}$ & $r_{3,(1,1,1)}$  
     &  $r_{1,(1,1,2)}$ & $r_{2,(1,1,2)}$ & $r_{3,(1,1,2)}$\\
	\hline
	$r_{1,(1,2,1)}$ & $r_{2,(1,2,1)}$ & $r_{3,(1,2,1)}$ 
     &  $r_{1,(1,2,2)}$ & $r_{2,(1,2,2)}$ & $r_{3,(1,2,2)}$\\
	%\hline
\end{tabular}
$\approx$
\begin{tabular}[c]{ l l l | l l l }
	%\multicolumn{2}{c}{} \vspace{7pt}\\
	\centering
	% & \\
	%\hline
	34.93 \; &  17.82  \;  &  22.23  \; & \, 33.79  \; &  23.72 \; &   22.55\\
	\hline
	33.10 \; &  18.29  \; &  28.47  \; & \, 32.11  \; &  23.87 \; &  28.471\\
	%\hline
\end{tabular}
\vspace{7pt}\\
\begin{tabular}[c]{ l l l | l l l}
	\centering
	%\multicolumn{2}{l}{payoff matrix} \vspace{7pt}\\
	%\hline
	$r_{1,(2,1,1)}$ & $r_{2,(2,1,1)}$ & $r_{3,(2,1,1)}$  
     & $r_{1,(2,1,2)}$ & $r_{2,(2,1,2)}$ & $r_{3,(2,1,2)}$\\
	\hline
	$r_{1,(2,2,1)}$ & $r_{2,(2,2,1)}$ & $r_{3,(2,2,1)}$  
     & $r_{1,(2,2,2)}$ & $r_{2,(2,2,2)}$ & $r_{3,(2,2,2)}$\\
	%\hline
\end{tabular}
$\approx$
\begin{tabular}[c]{ l l l | l l l }
	%\multicolumn{2}{c}{} \vspace{7pt}\\
	\centering
	% & \\
	%\hline
	34.68 \; &  21.53 \; &  29.80 \;  & \, 33.70  \; &   26.51 \; &   29.74\\
	\hline
	33.09 \; &  21.76 \; &  34.91 \; & \, 32.22 \; &  26.52 \; &  34.64\\
	%\hline
\end{tabular}
\vspace{10pt}
\caption{An egoistic election game instance of three parties that has no PSNE. The natural function to compute winning probabilities is adopted ($\beta=100, n_i=2$ for $i\in\{1,2,3\}$). Here, for example, $u_{x_{1,1}} = 50+9+3=62$, $u_{x_{2,1}} = 22+42+15=79$, $u_{x_{3,1}}=36+1+44=81$. Then we have $p_{1,(1,1,1)} = 62/(62+79+81)\approx 0.2793$, $p_{2,(1,1,1)} = 79/(62+79+81)\approx 0.3559$, $p_{3,(1,1,1)}\approx 0.3649$. So $r_{1,1,1} = 50\cdot p_{1,(1,1,1)} + 22\cdot p_{2,(1,1,1)} + 36\cdot p_{3,(1,1,1)}\approx 34.93$. All the eight profiles are not PSNE. For example, consider profile $(1,1,1)$. Party player $\mathcal{P}_2$ has the incentive to change its strategy to~$x_{2,2}$ from~$x_{2,1}$ because $r_{2,(1,2,1)} = 18.29 > 17.82 = r_{2,(1,1,1)}$. This example is clearly egoistic because $u_1(x_{1,2}) = 44 > \max\{u_1(x_{2,1}), u_1(x_{3,1})\} = 36$, $u_2(x_{2,2}) = 40 > \max\{u_2(x_{1,2}), u_2(x_{3,2})\} = 22$ and $u_3(x_{3,2}) = 42 > \max\{u_3(x_{1,2}), u_3(x_{2,2})\} = 32$.}
\label{tab:NoPSNE_BT}
\vspace{-12pt}
\end{center}
\end{table}


\begin{table}[ht]
\begin{center}
\begin{tabular}[c]{ l l l | l l l | l l l }
	%\centering
	{\footnotesize $u_1(x_{1,i})$}& {\footnotesize $u_2(x_{1,i})$}& {\footnotesize $u_3(x_{1,i})$} & 
    {\footnotesize $u_1(x_{2,i})$}& {\footnotesize $u_2(x_{2,i})$}& {\footnotesize $u_3(x_{2,i})$} & 
    {\footnotesize $u_1(x_{3,i})$}& {\footnotesize $u_2(x_{3,i})$}& {\footnotesize $u_3(x_{3,i})$}
        \\
	\hline
	67  &  10   &  9  &  11  &  45  &  9  &   2  &  25  &  53 \\
        66  &  9   &  11  &   1  &  43  & 27  &  41  &   6  &  49 \\
	\hline
\end{tabular}
\vspace{3pt}\\
\begin{tabular}[c]{ l l l | l l l}
	\centering
	%\multicolumn{2}{l}{payoff matrix} \vspace{7pt}\\
	%\hline
	$r_{1,(1,1,1)}$ & $r_{2,(1,1,1)}$ & $r_{3,(1,1,1)}$ 
     & $r_{1,(1,1,2)}$ & $r_{2,(1,1,2)}$ & $r_{3,(1,1,2)}$\\
	\hline
	$r_{1,(1,2,1)}$ & $r_{2,(1,2,1)}$ & $r_{3,(1,2,1)}$ 
     & $r_{1,(1,2,2)}$ & $r_{2,(1,2,2)}$ & $r_{3,(1,2,2)}$\\
	%\hline
\end{tabular}
$\approx$
\begin{tabular}[c]{ l l l | l l l }
	%\multicolumn{2}{c}{} \vspace{7pt}\\
	\centering
	% & \\
	%\hline
	28.73 \; &  25.04  \;  &  24.24  \; & \, 42.16  \; &  17.66 \; &   24.55\\
	\hline
	25.29 \; &  24.95  \; &  29.24  \; & \, 38.61  \; &  17.74 \; &  29.23\\
	%\hline
\end{tabular}
\vspace{3pt}\\
\begin{tabular}[c]{ l l l | l l l}
	\centering
	%\multicolumn{2}{l}{payoff matrix} \vspace{7pt}\\
	%\hline
	$r_{1,(2,1,1)}$ & $r_{2,(2,1,1)}$ & $r_{3,(2,1,1)}$ 
     & $r_{1,(2,1,2)}$ & $r_{2,(2,1,2)}$ &  $r_{3,(2,1,2)}$\\
	\hline
	$r_{1,(2,2,1)}$ & $r_{2,(2,2,1)}$ & $r_{3,(2,2,1)}$  
     & $r_{1,(2,2,2)}$ & $r_{2,(2,2,2)}$ & $r_{3,(2,2,2)}$\\
	%\hline
\end{tabular}
$\approx$
\begin{tabular}[c]{ l l l | l l l }
	%\multicolumn{2}{c}{} \vspace{7pt}\\
	\centering
	% & \\
	%\hline
	28.36 \; &  24.67 \; &  24.98 \;  & \, 41.81  \; &   17.31 \; &   25.24\\
	\hline
	24.92 \; &  24.59 \; &  29.97 \; & \, 38.27 \; &  17.40 \; &  29.91\\
	%\hline
\end{tabular}
\vspace{10pt}
\caption{A strongly egoistic election game instance of three parties that has no PSNE. The natural winning probability function is adopted ($\beta=100, n_i=2$ for $i\in\{1,2,3\}$). Similar to Table~\ref{tab:NoPSNE_BT} we can compute the rewards of each party player in each profile. This example is strongly egoistic since $u_1(x_{1,2}) = 66 > u_1(x_{2,1}) + u_1(x_{3,2}) = 52$, $u_2(x_{2,2}) = 43 > u_2(x_{1,1}) + u_2(x_{3,1}) = 35$ and $u_3(x_{3,2}) = 49 > u_3(x_{1,2}) + u_3(x_{2,2}) = 38$.}
\label{tab:NoPSNE_BT_strong}
\end{center}
\vspace{-16pt}
\end{table}
\fi


\iffalse
\begin{table}[ht]
\begin{center}
\begin{tabular}[c]{ l l l | l l l | l l l }
	%\centering
	{\footnotesize $u_1(x_{1,i})$}& {\footnotesize $u_2(x_{1,i})$}& {\footnotesize $u_3(x_{1,i})$} & 
     {\footnotesize $u_1(x_{2,i})$}& {\footnotesize $u_2(x_{2,i})$} & {\footnotesize $u_3(x_{2,i})$} & 
     {\footnotesize $u_1(x_{3,i})$}& {\footnotesize $u_2(x_{3,i})$} & {\footnotesize $u_3(x_{3,i})$}
        \\
	\hline
	33  &  42  &   0 &   0  &  93  &   4  &  20  &   6  &  54 \\
        30  &  30  &  29 &   3  &  89  &   1  &   0  &  44  &  50 \\
	\hline
\end{tabular}
\vspace{3pt}\\
\begin{tabular}[c]{ l l l | l l l}
	\centering
	%\multicolumn{2}{l}{payoff matrix} \vspace{7pt}\\
	%\hline
	$r_{1,(1,1,1)}$ & $r_{2,(1,1,1)}$ & $r_{3,(1,1,1)}$ 
     & $r_{1,(1,1,2)}$ & $r_{2,(1,1,2)}$ & $r_{3,(1,1,2)}$\\
	\hline
	$r_{1,(1,2,1)}$ & $r_{2,(1,2,1)}$ & $r_{3,(1,2,1)}$ 
     & $r_{1,(1,2,2)}$ & $r_{2,(1,2,2)}$ & $r_{3,(1,2,2)}$\\
	%\hline
\end{tabular}
$\approx$
\begin{tabular}[c]{ l l l | l l l }
	%\multicolumn{2}{c}{} \vspace{7pt}\\
	\centering
	% & \\
	%\hline
	16.38 \; &  49.80   \;  &  18.73   \; & \, 9.55  \; &  61.09 \; &   18.94\\
	\hline
	17.74 \; &  47.67   \; &  17.84   \; & \, 10.74  \; &  59.23 \; &  18.10\\
	%\hline
\end{tabular}
\vspace{3pt}\\
\begin{tabular}[c]{ l l l | l l l}
	\centering
	%\multicolumn{2}{l}{payoff matrix} \vspace{7pt}\\
	%\hline
	$r_{1,(2,1,1)}$ & $r_{2,(2,1,1)}$ & $r_{3,(2,1,1)}$ 
     & $r_{1,(2,1,2)}$ & $r_{2,(2,1,2)}$ & $r_{3,(2,1,2)}$\\
	\hline
	$r_{1,(2,2,1)}$ & $r_{2,(2,2,1)}$ & $r_{3,(2,2,1)}$ 
     & $r_{1,(2,2,2)}$ & $r_{2,(2,2,2)}$ & $r_{3,(2,2,2)}$\\
	%\hline
\end{tabular}
$\approx$
\begin{tabular}[c]{ l l l | l l l }
	%\multicolumn{2}{c}{} \vspace{7pt}\\
	\centering
	% & \\
	%\hline
	16.11   \; &  45.45   \; &  27.59  \;  & \, 9.57  \; &   56.47 \; &   27.40\\
	\hline
	17.40   \; &  43.36  \; &  26.87  \; & \, 10.71 \; &  54.62 \; &  26.71\\
	%\hline
\end{tabular}
\vspace{10pt}
\caption{A strongly egoistic election game instance of three parties that has no PSNE. The softmax winning probability function is adopted ($\beta=100, n_i=2$ for $i\in\{1,2,3\}$). Similar to Table~\ref{tab:NoPSNE_softmax}, we can derive the rewards of each party player in each profile. This example is strongly egoistic since $u_1(x_{1,2}) = 30 > u_1(x_{2,2}) + u_1(x_{3,1}) = 23$, $u_2(x_{2,2}) = 89 > u_2(x_{1,1}) + u_2(x_{3,2}) = 86$ and $u_3(x_{3,2}) = 50 > u_3(x_{1,2}) + u_3(x_{2,1}) = 33$.}
\label{tab:NoPSNE_softmax_strong}
\end{center}
\vspace{-16pt}
\end{table}
\fi

\end{document}

%%%%%%%%%%%%%%%%%%%%%%%%%%%%%%%%%%%%%%%%%%%%%%%%%%%%%%%%%%%%%%%%%%%%%%%%

