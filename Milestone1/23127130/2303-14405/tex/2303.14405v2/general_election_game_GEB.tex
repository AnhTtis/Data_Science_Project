\documentclass[runningheads]{llncs}
\usepackage{fullpage}
\usepackage{booktabs} % For formal tables
%\usepackage{pifont}
\usepackage{bbding}
%\usepackage{array}
\usepackage{amssymb}
%\usepackage{amssymb}
%\usepackage{fullpage}
\usepackage{amsmath,bm}
%\usepackage{bbm}
%\usepackage{amsthm}
%\usepackage{latexsym}
\usepackage{graphicx}
%\usepackage[ruled]{algorithm2e} % For algorithms
%\usepackage{flushend}
\usepackage{color}
\DeclareMathOperator*{\argmax}{arg\,max}
\DeclareMathOperator*{\argmin}{arg\,min}
\usepackage{algorithm}
\usepackage{algorithmic}
\renewcommand{\algorithmiccomment}[1]{/* #1 *\!/}
\renewcommand{\algorithmicrequire}{\textbf{Input:}}
\renewcommand{\algorithmicensure}{\textbf{Output:}}
\usepackage{enumitem}
\usepackage{comment}
%\newtheorem{Conj}{Conjecture}
\newcommand\genatop[2]{\genfrac{}{}{0pt}{}{#1\hfill}{#2\hfill}}


\begin{document}

%%%%%%%%%%%%%%%%%%%%%%%%%%%%%%%%%%%%%%%%%%%%%%%%%%%%%%%%%%%%%%%%%%%

\title{On the Efficiency of An Election Game of Two or More Parties: How Bad Can It Be? \thanks{This work is  supported by the Taiwan Ministry of Science and Technology under grant no. NSTC 110-2222-E-032-002-MY2 and NSTC 111-2410-H-A49-022-MY2.}}

\titlerunning{How Bad Can the Election Game with Two or More Parties Be?}

\author{Chuang-Chieh Lin\inst{1} \and Chi-Jen Lu\inst{2}\and Po-An Chen\inst{3}\thanks{Corresponding author}}

\authorrunning{C.-C. Lin, C.-J. Lu, and P.-A. Chen}

\institute{Department of Computer Science and Information Engineering, Tamkang University\\No.~151, Yingzhuan Rd., Tamsui Dist., New Taipei City 25137, Taiwan \\\email{josephcclin@gms.tku.edu.tw}\and 
Institute of Information Science, Academia Sinica\\128 Academia Road, Section 2, Nankang, Taipei 11529, Taiwan
\\\email{cjlu@iis.sinica.edu.tw}\and
Institute of Information Management, National Yang-Ming Chiao-Tung University\\1001 University Rd, Hsinchu City 300, Taiwan \\\email{poanchen@nctu.edu.tw}}

%%%%%%%%%%%%%%%%%%%%%%%%%%%%%%%%%%%%%%%%%%%%%%%%%%%%%%%%%%%%%%%%%%%

\maketitle
\begin{abstract}
We extend our previous work on two-party election competition [Lin, Lu \& Chen~2021] to the setting of three or more parties. An election campaign among two or more parties is viewed as a game of two or more players. Each of them has its own candidates as the pure strategies to play. People, as voters, comprise supporters for each party, and a candidate brings utility for the the supporters of each party. Each player nominates exactly one of its candidates to compete against the other party's. \emph{A candidate is assumed to win the election with higher odds if it brings more utility for all the people.} The payoff of each player is the expected utility its supporters get. The game is \emph{egoistic} if every candidate benefits her party's supporters more than any candidate from the competing party does. 
%Three functions satisfying the monotone property are considered for computing the winning probability of a candidate against the others. 
In this work, we first argue that the election game always has a pure Nash equilibrium when the winner is chosen by the hardmax function, while there exist game instances in the three-party election game such that no pure Nash equilibrium exists even the game is egoistic. Next, 
we propose two sufficient conditions for the egoistic election game to have a pure Nash equilibrium. Based on these conditions, we propose a fixed-parameter tractable algorithm to compute a pure Nash equilibrium of the egoistic election game. Finally, perhaps surprisingly, we show that the price of anarchy of the egoistic election game is upper bounded by the number of parties. Our findings suggest that the election becomes unpredictable when more than two parties are involved and, moreover, the social welfare deteriorates with the number of participating parties in terms of possibly increasing price of anarchy. This work alternatively explains why the two-party system is prevalent in democratic countries. 
\keywords{Election game; Nash equilibrium; Price of anarchy; Egoism; Monotone function}
\end{abstract}


%%%%%%%%%%%%%%%%%%%%%%%%%%%%%%%%%%%%%%%%%%%%%%%%%%%%%%%%%%%%%%%%%%%%%
\section{Introduction}
\label{sec:intro}
%%%%%%%%%%%%%%%%%%%%%%%%%%%%%%%%%%%%%%%%%%%%%%%%%%%%%%%%%%%%%%%%%%%%%


Modern democracy runs on political parties and elections prevail in democratic countries. In an election campaign, political parties compete with each other and exhaust their effort to attract voters' ballots, which can be regarded as the aggregation of voters' belief or preferences. They compete by nominating their ``best" candidates in elections and try to appeal mainly to their supporters. At first sight, it seems to be difficult to foresee the result of an election especially when voters are strategic and different voting procedures and allocation rules could lead to totally different results. 
Duverger's law suggests plurality voting in favor of the two-party system~\cite{Duv54}.
Also, Dellis~\cite{Del2013} explained why a two-party system emerges under plurality voting and other voting procedures permitting truncated ballots. These motivated the investigation of the efficiency of a two-party system in our previous work~\cite{LLC2021}, in which we bypassed the above mentioned issues in a micro scale, and considered a macro-model instead by formalizing the political competition between two parties as a non-cooperative two-player game. We call it the \emph{two-party election game}. 


In the two-party election game, each party is regarded as a player who treats its candidates as strategies and her payoff is the expected utility for her supporters. The randomness in the expectation comes from the uncertainty of winning or losing in the game. The odds of winning an election for a candidate nominated by a party over another candidate nominated by the competing party are assumed to be related to two factors---the total benefits that she brings to the whole society, including the supporters and non-supporters, and those that her competitor brings. Usually, a candidate nominated by a party responds and caters more to the needs and inclination of her party's supporters and less to those of the supporters of the other party. Naturally, a candidate then brings different utility to the supporters and to the non-supporters. We expect a candidate to win with higher probability if she brings more total utility to the whole society. 
We focused on pure strategies in the election game for the reason that, compared with \emph{mixed strategies} that are represented as a probability distribution over a subset of available actions, pure strategies are arguably more realistic in practical world. From another point of view, the real-valued utility summarizes infinitely many possibilities how a candidate can benefit the voters. 


Suppose that the winning probability of a candidate against the opponent is calculated using a linear function and the softmax function, where the former is linear in the difference between the total utility brought by the two competing candidates and the latter is the ratio of one exponential normalized total utility to the sum of both. 
Based upon the game setting as above, we proved in our previous work~\cite{LLC2021} that a pure Nash equilibrium (PNE) always exists in the two-party election game under a mild condition---\emph{egoism}, which states that any candidate benefits her party's supporters more than any candidate from the competing party does. The existence of equilibria can bring positive implications. For decades, the Nash equilibrium has been known as a kind of solution concept which provides a more predictable outcome of a non-cooperative game modeling behaviors of strategic players. Although a Nash equilibrium, an equilibrium concept when mixed strategies are considered, always exists in a finite game~\cite{nash_1950,nash_1951}, this is not the case for the PNE~\cite{OR94}. 
We applied the price of anarchy~\cite{KP09} as the inefficiency measure of an equilibrium and proved that the price of anarchy is constantly bounded. This shows in some sense that game between two parties in candidate nomination for an election benefits the people with a social welfare at most constantly far from the optimum. Existence guarantee of pure Nash equilibria and bounded price of anarchy for the egoistic two-party election game suggest that a two-party system is ``good" from these perspectives. On the other hand, however, when the egoistic property is not satisfied, the game might not have any PNE and the price of anarchy can be unbounded, hence, the game can be extremely inefficient. 


%=====================================================================
\subsection{Our Contributions}
\label{subsec:contribution}
%=====================================================================

In this work, we generalize our previous work in~\cite{LLC2021} to deal with $m\geq 2$ parties. 
We would like to know \emph{how good or bad a system of more than two parties can be}. As we have shown that, without the egoism guarantee, the two-party election game may have no PNE and it can be extremely inefficient in terms of the unbounded price of anarchy, in this work we focus on the egoistic election game for two or more parties. Briefly, we call such a generalized game the \emph{egoistic election game}, and aim at investigating the answers to following questions:
\begin{enumerate}
    \item Does the egoistic election game using the softmax function to calculate the winning probability of a candidate against her opponents always have a PNE, even for three or more parties? Does the function for computing such a winning probability matter? 
    \item What is the computational complexity of computing a PNE of the egoistic election game when two ore more parties are involved in general? 
    \item What is the price of anarchy of the egoistic election game of two or more parties? Is it still constantly bounded?
\end{enumerate}
    
\paragraph{Our results with respect to the above questions are summarized in the following:} 

\begin{enumerate}
    \item We give examples to confirm that a PNE does NOT always exist in the egoistic election game of three or more parties even using the softmax function to calculate the winning probability of a candidate. Furthermore, there exist instances in which no PNE exists even it satisfies a stronger notion of egoism. Moreover, we propose two sufficient conditions for the egoistic election game to have a PNE. 
    \item We conjecture that to compute a PNE of the egoistic election game is {\sf NP}-hard in the general form representation. Based on two sufficient conditions, we identify two natural parameters, the \emph{nominating depth} and \emph{number of chaotic parties}, of the egoistic election game, and propose a fixed-parameter tractable algorithm to compute a  PNE of the game if it exists. Namely, a PNE of the egoistic election game can be computed in time polynomial in the number of parties and number of candidates in each party if the two proposed parameters are as small as constants. 
    \item Perhaps surprisingly, we prove that the price of anarchy of the egoistic election game is upper bounded by the number of competing parties~$m$ when the odds of winning the election for a party is calculated by any \emph{monotone} function, which includes the hardmax function, the natural function and the softmax function, etc. This, to a certain degree, suggests that the social welfare deteriorates when there are more competing parties participating in the election campaign. Our work provides an explanation alternative to Duverger's law that why the two-party system is prevalent in democratic countries. As shown in~\cite{LLC2021} and this work as well, the price of anarchy of the egoistic election game is lower bounded by~2 for the two-party election game using either the hardmax function or the softmax function to compute the winning probability of a candidate, our results in this study imply that the price of anarchy bound for the egoistic two-party election game is tight. 
    \item We also remark that coalition of party players with the strongly egoism guarantee, in which a composite-like candidate for a coalition is considered, makes the game collapse on the non-cooperative egoistic election game setting. This observation reveals that if the election game is cooperative and strongly egoistic, it will be efficient in the sense that the price of anarchy decreases. 
\end{enumerate}


\paragraph{Organization of this paper.} We briefly survey related work on political competition in Sect.~\ref{subsec:related_work}. Preliminaries are given in Sect.~\ref{sec:preliminaries}. In Sect.~\ref{sec:hardness_no_PNE_examples}, we investigate the egoistic election game through examples to show that the hardmax function for winning probability computation leads the game to always have a PNE, while there exist instances of three parties in which there is no PNE when the natural function or the softmax function is adopted. In Sect.~\ref{sec:two_sufficient_conditions_PNE}, we propose two sufficient conditions for the egoistic election game to have a PNE. Then, we propose a fixed-parameter tractable algorithm to compute a PNE of the egoistic election game in the general form representation. In Sect.~\ref{sec:PoA}, we show the upper bounds on the price of anarchy of the egoistic election game. Coalition of party players with the strongly egoism guarantee is also discussed therein. 
Concluding remarks and future work will be given and discussed in Sect.~\ref{sec:future}.


%=====================================================================
\subsection{Related work}
\label{subsec:related_work}
%=====================================================================
Duverger's law that suggests plurality voting in favor of the two-party system~\cite{Duv54}.
There have been studies that formalize Duverger's law. It can be explained either by the strategic 
behavior of the voters~\cite{Del2013,Fed92,Fey97,MW93,Pal89} or that of the 
candidates~\cite{Cal05,CW07,Pal84,Web92}.

Most of the works on equilibria of a political competition are mainly based on {\em Spatial Theory of Voting}~\cite{Hot29,Dow57,LRL2007,Pal84,Web92}.
%which can be traced back to~\cite{Hot29}. 
In such settings, there are two parties and voters with single-peaked preferences over a unidimensional metric space. Each party 
chooses a kind of ``policy" that is as close as possible to voters' preferences. When the policy space is unidimensional, the Spatial Theory of Voting states that the parties' strategies would be determined by the median voter's preference. However, pure Nash equilibria may not exist for policies over a multi-dimensional space~\cite{Dug16}. 

Procaccia and Rosenschein~\cite{PR2006} introduced the notion ``distortion" which resembles 
the price of anarchy, while the latter is used in games of strategic players. 
The distortion measures the inefficiency when a social choice rule (e.g., voting) is applied. Generally, voters with cardinal preferences~\cite{CP2011,PR2006}) or metric preferences~\cite{ABP2015,AP2016,CDK2017}) are considered herein. As an embedding on a voter' ballot is allowed, Caragiannis and Procaccia~\cite{CP2011} 
discussed the distortion of social choice when each voter's ballot receives an embedding, which maps 
the preference to the output ballot. Cheng et al.~\cite{CDK2017} focused 
on the distribution of voters as well as the candidates of parties and they justified that the expected distortion is small when the candidates are drawn from the same distribution as the voters.

A political competition can be considered as a simultaneous one-shot game, similar to what we consider in this work, though it can also be viewed as a dynamic process in multiple rounds. In~\cite{KK09}, each time a player competes by investing some of her budget or resource in a component battle to collect a value if she wins. Players fight in multiple battles, and their budgets are consumed over time. \textcolor{black}{In a two-player zero-sum version of such games,} a strategic player needs to make adequate sequential actions to win the contest against dynamic competition over time~\cite{CCH2018}.


\begin{comment}
\begin{quote}
\textcolor{blue}{Hardness of computing a mixed strategy Nash equilibrium and that of a pure Nash equilibrium (?)}
\end{quote}
\end{comment}


%%%%%%%%%%%%%%%%%%%%%%%%%%%%%%%%%%%%%%%%%%%%%%%%%%%%%%%%%%%%%%%%%%%%%%
\section{Preliminaries}
\label{sec:preliminaries}
%%%%%%%%%%%%%%%%%%%%%%%%%%%%%%%%%%%%%%%%%%%%%%%%%%%%%%%%%%%%%%%%%%%%%%

For an integer $k>0$, let $[k]$ denote the set $\{1,2,\ldots,k\}$. We assume that the society consists of voters, and each voter is a supporter of one of the $m\geq 2$ parties $\mathcal{P}_1,\mathcal{P}_2,\ldots,\mathcal{P}_m$. These $m$ parties compete in an election campaign. Each party $\mathcal{P}_i$, having $n_i\geq 2$ candidates $x_{i,1}, x_{i,2}, \ldots, x_{i,n_i}$ for each $i\in [m]$, has to designate one candidate to participate in the election campaign. Let $n = \max_{i\in [m]} n_i$ denote the maximum number of candidates in a party. 
Let $u_j(x_{i,s})$ denote the \emph{utility} that party $\mathcal{P}_j$'s supporters can get when candidate $x_{i,s}$ is elected, for $i,j\in [m], s\in[n_{i}]$. For each $i\in [m]$ and $s\in [n_i]$, we denote by $u(x_{i,s}) := \sum_{j\in [m]}u_j(x_{i,s})$ as the \emph{social utility} candidate $x_{i,s}$ can bring to all the voters. Assume that the social utility is nonnegative and bounded, specifically, we assume $u(x_{i,s})\in [0, \beta]$ for some real $\beta\geq 1$, for each $i\in [m], s\in [n_i]$.
Assume that candidates in each party are 
sorted according to the utility for its party's supporters. 
Namely, we assume that $u_i(x_{i,1})\geq u_i(x_{i,2})\geq \ldots \geq u_i(x_{i,n_i})$ for each $i\in [m]$. 
W.l.o.g., we assume that $u_1(x_{i,1})\geq u_2(x_{2,1})\geq \ldots \geq u_m(x_{m,1})$ to break the symmetry. 

The election campaign is viewed as a game of $m$ players such that each party corresponds to player, we call it a \emph{party player}. With a slight abuse of notation, $\mathcal{P}_i$ also denotes the party player with respect to party $\mathcal{P}_i$. Each party player $\mathcal{P}_i$, $i\in [m]$, has $n_i$ \emph{pure strategies}, each of which is a candidate selected to participate in the election campaign. We consider \textcolor{black}{an assumption} as a desired property that \emph{a party wins the election with higher odds if it selects a candidate with higher social utility}. We call it the \emph{monotone property}. Moreover, the odds of winning then depend on the social utility brought by the candidates. Suppose that $\mathcal{P}_i$ designates candidates $x_{i,s_i}$ for $i\in [m]$ and let $\mathbf{s} = (x_{1,s_1},x_{2,s_2},\ldots,x_{m,s_m})$ (or simply $(s_1,s_2,\ldots,s_m)$ when it is clear from the context) be the \emph{profile} of the designated candidates of all the party players.
As defined and discussed in~\cite{LLC2021}, we formulate the winning odds, $p_{i,\mathbf{s}}$, which stands for the probability of~$\mathcal{P}_i$ winning the election campaign in a way to preserve the monotone property, as follows\footnote{In~\cite{LLC2021}, the linear model based on the dueling bandit setting~\cite{AJK2014} is applied for the election game of exactly two party players. As it is designed for pairwise comparison, we do not consider it for general $m\geq 2$ in this work.}. 
Note that the monotone property guarantees that $p_{i,\mathbf{s}'}\geq p_{i,\mathbf{s}}$ for $\mathbf{s}' = (s_1,s_2,\ldots,s_i',\ldots,s_m)$ and $u(x_{i,s_i'})\geq u(x_{i,s_i})$ (i.e., party player $i$ unilaterally deviates its strategy to $x_{i,s_i'}$ which brings larger social utility). 
%Based upon them, we can formally define the payoff for each party player in an election game.  
\begin{itemize}
\iffalse
\item Linear link model~\cite{AJK2014}: \[p_{i,\mathbf{s}} := \frac{1 +(u(A_i)-u(B_j))/b}{2}.\] 
    \begin{itemize}
        \item This is inspired by the exploration method used in the multi-armed bandit problem~\cite{Kul00} and the 
            probabilistic comparison used in the dueling bandits problem~\cite{AJK2014,YBKJ2012}. The winning odds is then regarded as 
            a \emph{linear} function of the \emph{difference} between the social utility brought by candidates $A_i$ and $B_j$.
    \end{itemize}
    \vspace{6pt}
\fi
\item The hardmax function:
\[
p_{i,\mathbf{s}} = \left\{\begin{array}{ll}
1 & \mbox{ if } i = \argmax_{j\in [m]} u(x_{j,s_j})\\
0 & \mbox{ otherwise.}
\end{array}\right.
\]
    \begin{itemize}
        \item The hardmax function simply allocates all probability mass to the candidate of maximum social utility with ties broken arbitrarily.
    \end{itemize}
\item The softmax function~\cite{Kul00,SB98}: 
	\[p_{i,\mathbf{s}} := \frac{e^{u(x_{i,s})/\beta}}{\sum_{j\in [m]} e^{u(x_{j,s_j})/\beta}}.\]
    \begin{itemize}
	    \item The softmax function is extensively used in machine learning to normalize the output into a probability distribution. It is formulated as the ratio of one exponential normalized social utility to the sum of both. 
	\end{itemize}
 \vspace{5pt}
\item The natural function~\cite{Bra54,YBKJ2012,DWH2020}: \[p_{i,\mathbf{s}} := \frac{u(x_{i,s_i})}{\sum_{j\in [m]} u(x_{j,s_j})}.\] 
    \begin{itemize}
    \item We treat the probability $p_{i,\mathbf{s}}$ as ratio of the social utility brought by a candidate to the sum of the social utility brought by all candidates.\footnote{We assume that $\sum_{j\in [m]} u(x_{j,s_j})>0$ for the natural function.}  
    \end{itemize}
\end{itemize}



We call the functions calculating the winning probability of a candidate against the others \emph{WP functions}. The WP functions fulfilling the monotone property are called \emph{monotone WP functions}. The hardmax function is monotone since raising the social utility never makes a party player lose. In addition, by Lemma~\ref{lem:fractional_monotone} we know that the the softmax and natural functions are monotone WP functions. 


\begin{lemma}\label{lem:fractional_monotone}
Let $r, s > 0$ be two positive real numbers such that $r<s$. Then, for any $d>0$, $r/s < (r+d)/(s+d)$. 
\end{lemma}
\begin{proof}
Let $r, s > 0$ be two positive real numbers such that $r<s$. Then $r/s - (r+d)/(s+d) = d(r-s)/(s^2+sd)$. Clearly, $r/s - (r+d)/(s+d) < 0$ if $r < s$. 
%The lemma is then proved. 
\qed
\end{proof}


The payoff of party player $\mathcal{P}_i$ given the profile $\mathbf{s}$ is denoted by $r_{i,\mathbf{s}}$, which is the \emph{expected utility} that party~$\mathcal{P}_i$'s supporters obtain in~$\mathbf{s}$. 
Namely, 
\begin{equation*}
r_{i}(\mathbf{s}) = \sum_{j\in [m]} p_{j,\mathbf{s}}u_i(x_{j,s_j}),
\vspace{-3pt}
\end{equation*} 
which can be computed in $O(m)$ time for each~$i$. 
We define the {\em social welfare} of the profile $\mathbf{s}$ as $SW(\mathbf{s}) = \sum_{j\in [m]} r_{j}(\mathbf{s})$. We say that 
a profile $\mathbf{s}$ is a {\em pure Nash equilibrium} (PNE) if $r_{i}(s'_{i},\mathbf{s}_{-i})\leq r_{i}(\mathbf{s})$ for any $s'_i\neq s_i$, where $\mathbf{s}_{-i}$ denotes the profile without party player $\mathcal{P}_i$'s strategy. That is, in $\mathbf{s}$, none of the party players has the incentive to deviate from its current strategy. The {\em (pure) price of anarchy} (PoA) of the game~$\mathcal{G}$ is defined as 
\[
\mbox{PoA}(\mathcal{G})=\frac{SW(\mathbf{s}^*)}{SW(\hat{\mathbf{s}})} = 
\frac{\sum_{j\in [m]} r_{j}(\mathbf{s}^*)}{\sum_{j\in [m]} r_{j}(\hat{\mathbf{s}})},
\]
where $\mathbf{s}^* = \argmax_{\mathbf{s}\in\prod_{i\in [m]}[n_i]} SW(\mathbf{s})$ is the {\em optimal profile}, 
which has the best (i.e., highest) social welfare among all possible profiles, and $\hat{\mathbf{s}} = \argmin\limits_{\genatop{\mathbf{s}\in\prod_{i\in [m]} [n_i]}{\mathbf{s}\mbox{\tiny \;is a PNE for }\mathcal{G}}} SW(\mathbf{s})$ is the PNE with the worst (i.e., lowest) social welfare. \textcolor{black}{Note that an egoistic election game may not necessarily have a PNE so an upper bound on the PoA is defined over only games with PNE, i.e., $\max_{\mathcal{G}\mbox{\scriptsize\, has a PNE}}\mbox{PoA}(\mathcal{G})$, accordingly.} 
\iffalse
In an election, probabilistically (instead of deterministically) nominating a candidate or imaging repeated nominations of candidates is almost infeasible in the reality. Thus, adopting PNE as the equilibrium concept \textcolor{black}{best reflects the situation of an election.} Nonetheless, it is theoretically natural for one to consider mixed Nash equilibria or other more general notions of equilibria as solution concepts where the existence of equilibria is always guaranteed. It may take other analysis frameworks such as \cite{KM2015,R2009} to study their corresponding price-of-anarchy bounds.
\fi

We will use the following properties throughout this paper. 

\begin{definition}\label{defn:egoistic}
We say that the election game is {\em egoistic} if $u_i(x_{i,s_i})> u_i(x_{j,s_j})$ for all $i\in [m], s_i\in [n_i], s_j\in [n_j]$.
\end{definition}
This guarantees that \emph{any candidate benefits its supporters more than those from the other competing parties}. Such a property is natural and reasonable in the real world. Below we consider its strong version.

\begin{definition}\label{defn:strong_egoistic}
We say that the election game is {\em strongly egoistic} if $u_i(x_{i,s_i})> \sum_{j\in [m]\setminus\{i\}} \max_{s_j\in [n_j]}u_i(x_{j,s_j})$ for all $i\in [m], s_i\in [n_i]$.
\end{definition}
This property states that any candidate benefits its supporters more than the sum of those from all the other competing parties. 

\begin{definition}%[Strategy Domination]
\label{defn:strategy_domination}
For each $i\in [m]$, we say that strategy $x_{i,s}$ {\em weakly surpasses} $x_{i,s'}$ if $s < s'$ and 
$u(x_{i,s})\geq u(x_{i,s'})$. We say that strategy $x_{i,s}$ {\em surpasses} $x_{i,s'}$ if $x_{i,s}$ weakly surpasses $x_{i,s'}$ and either $u(x_{i,s}) > u(x_{i,s'})$ or $u_i(x_{i,s})>u_i(x_{i,s'})$. 
\end{definition}
Roughly speaking, a candidate surpasses the other one in the same party if she brings more utility to the supporters and has higher winning probability. 
\begin{remark} 
By the monotone property, we have $p_{i,(s_i,\mathbf{s}_{-i})}\geq p_{i,(s_i',\mathbf{s}_{-i})}$ if and only if $u(x_{i,s_i})\geq u(x_{i,s_{i'}})$.
\end{remark}
\iffalse
This means that given party~$B$'s choice of \textcolor{black}{candidate~$j$}, if the total utility when party~$A$'s candidate~$i$ is elected are greater than or equal to those when party~$A$'s candidate~$i'$ is elected, then the chance of candidate~$i$ winning over candidate~$j$ is greater than or equal to that of candidate~$i'$ winning over candidate~$j$, and vice versa.
\fi


%%%%%%%%%%%%%%%%%%%%%%%%%%%%%%%%%%%%%%%%%%%%%%%%%%%%%%%%%%%%%%%%%%%%%%
\section{Hardness and Counterexamples}
\label{sec:hardness_no_PNE_examples}
%%%%%%%%%%%%%%%%%%%%%%%%%%%%%%%%%%%%%%%%%%%%%%%%%%%%%%%%%%%%%%%%%%%%%%


We first note that when the hardmax function is adopted as the monotone WP function, the election game (even though it is not egoistic) always has a PNE. Indeed, with ties broken arbitrarily, the maximum of a finite set of numbers always exists and hence the candidate $x_{i^*,s_{i^*}}$, where $i^* := \argmax_{i\in [m]} u(x_{i,s_i})$ with respect to the profile $\mathbf{s}$, wins with probability~1. Any other party player $\mathcal{P}_j$ for $j\neq i^*$ has no incentive to deviate her strategy because the payoff can never be better off. However, such a PNE in this case is not necessarily the optimal profile. For example, consider the instance in Tab.~\ref{tab:AlwaysPNE_HM}. The social welfare of the PNE is~$50$, though that of the optimal profile is~$100$, which is twice higher.


In our previous work~\cite{LLC2021}, we have shown that the \emph{egoistic} two-party election game always has a PNE if a linear function or the softmax function is adopted as the monotone WP function\footnote{Though it is not the case for the natural function.}. Naturally, one might be curious about whether the egoistic property is sufficient for such an election game of \emph{three or more parties} to possess a PNE. Unfortunately, through program simulations we can find game instances as counterexamples, which imply that the egoistic election game of three or more parties does not always have a PNE (see Table~\ref{tab:NoPNE_BT} and \ref{tab:NoPNE_softmax}). Even for a strongly egoistic election game, such a counterexample still exists (see Table~\ref{tab:NoPNE_BT_strong} and~\ref{tab:NoPNE_softmax_strong}). This somehow provides a possible hint that the result of an election campaign between more than two parties may be more unpredictable than that between only two parties. 


\begin{table}[ht]
\begin{center}
\begin{tabular}[c]{ l l l | l l l | l l l }
	%\centering
	$u_1(x_{1,i})$ & $u_2(x_{1,i})$ & $u_3(x_{1,i})$ & 
        $u_1(x_{2,i})$ & $u_2(x_{2,i})$ & $u_3(x_{2,i})$ & 
        $u_1(x_{3,i})$ & $u_2(x_{3,i})$ & $u_3(x_{3,i})$
        \\
	\hline
	50  &  0   &  0  &  15  &  31  &  0  &  10  &   10  &  24 \\
        49  &  29  &  22 &  16  &  30  &  0  &  10  &  10  &  23 \\
	\hline
\end{tabular}
\vspace{7pt}\\
\begin{tabular}[c]{ l l l | l l l}
	\centering
	%\multicolumn{2}{l}{payoff matrix} \vspace{7pt}\\
	%\hline
	$r_{1,(1,1,1)}$ \; & $r_{2,(1,1,1)}$ \; & $r_{3,(1,1,1)}$ \; 
     &  \, $r_{1,(1,1,2)}$ \; & $r_{2,(1,1,2)}$ \; & $r_{3,(1,1,2)}$\\
	\hline
	$r_{1,(1,2,1)}$ \; & $r_{2,(1,2,1)}$ \; & $r_{3,(1,2,1)}$ \; 
     &  \, $r_{1,(1,2,2)}$ \; & $r_{2,(1,2,2)}$ \; & $r_{3,(1,2,2)}$\\
	%\hline
\end{tabular}
$\approx$\\
\begin{tabular}[c]{ l l l | l l l }
	%\multicolumn{2}{c}{} \vspace{7pt}\\
	\centering
	% & \\
	%\hline
	50 \, &  0  \,  &  0  \, & \, 50  \, &  0 \, &   0\\
	\hline
	50 \, &  0  \, &  0  \, & \, 50  \, &  0 \, &  0\\
	%\hline
\end{tabular}
\vspace{7pt}\\
\begin{tabular}[c]{ l l l | l l l}
	\centering
	%\multicolumn{2}{l}{payoff matrix} \vspace{7pt}\\
	%\hline
	$r_{1,(2,1,1)}$ \, & $r_{2,(2,1,1)}$ \, & $r_{3,(2,1,1)}$ \, 
     & \, $r_{1,(2,1,2)}$ \, & $r_{2,(2,1,2)}$ \, &  $r_{3,(2,1,2)}$\\
	\hline
	$r_{1,(2,2,1)}$ \, & $r_{2,(2,2,1)}$ \, & $r_{3,(2,2,1)}$ \, 
     & \, $r_{1,(2,2,2)}$ \, & $r_{2,(2,2,2)}$ \, & $r_{3,(2,2,2)}$\\
	%\hline
\end{tabular}
$\approx$
\begin{tabular}[c]{ l l l | l l l }
	%\multicolumn{2}{c}{} \vspace{7pt}\\
	\centering
	% & \\
	%\hline
	49 \, &  29 \, &  22 \,  & \, 49  \, &   29 \, &  22\\
	\hline
	49 \, &  29 \, &  22 \, & \, 49 \, &  29 \, &  22\\
	%\hline
\end{tabular}
\vspace{10pt}
\caption{An egoistic election game instance of three parties that always has a PNE ($\beta=100, n_i=2$ for $i\in\{1,2,3\}$). The hardmax function is used as the monotone WP function. Though profile $(1,1,1)$ is a PNE, it has the social welfare 50, which is only half of that of the optimal profile, which is~$100=\beta$.}
\label{tab:AlwaysPNE_HM}
\vspace{-20pt}
\end{center}
\end{table}



\begin{table}[ht]
\begin{center}
\begin{tabular}[c]{ l l l | l l l | l l l }
	%\centering
	$u_1(x_{1,i})$ & $u_2(x_{1,i})$ & $u_3(x_{1,i})$ & 
        $u_1(x_{2,i})$ & $u_2(x_{2,i})$ & $u_3(x_{2,i})$ & 
        $u_1(x_{3,i})$ & $u_2(x_{3,i})$ & $u_3(x_{3,i})$
        \\
	\hline
	50  &  9   &  3  &  22  &  42  &  15  &  36  &   1  &  44 \\
        44  &  22  &  30 &  19  &  40  &  32  &  33  &  18  &  42 \\
	\hline
\end{tabular}
\vspace{7pt}\\
\begin{tabular}[c]{ l l l | l l l}
	\centering
	%\multicolumn{2}{l}{payoff matrix} \vspace{7pt}\\
	%\hline
	$r_{1,(1,1,1)}$ \; & $r_{2,(1,1,1)}$ \; & $r_{3,(1,1,1)}$ \; 
     &  \, $r_{1,(1,1,2)}$ \; & $r_{2,(1,1,2)}$ \; & $r_{3,(1,1,2)}$\\
	\hline
	$r_{1,(1,2,1)}$ \; & $r_{2,(1,2,1)}$ \; & $r_{3,(1,2,1)}$ \; 
     &  \, $r_{1,(1,2,2)}$ \; & $r_{2,(1,2,2)}$ \; & $r_{3,(1,2,2)}$\\
	%\hline
\end{tabular}
$\approx$
\begin{tabular}[c]{ l l l | l l l }
	%\multicolumn{2}{c}{} \vspace{7pt}\\
	\centering
	% & \\
	%\hline
	34.93 \; &  17.82  \;  &  22.23  \; & \, 33.79  \; &  23.72 \; &   22.55\\
	\hline
	33.10 \; &  18.29  \; &  28.47  \; & \, 32.11  \; &  23.87 \; &  28.471\\
	%\hline
\end{tabular}
\vspace{7pt}\\
\begin{tabular}[c]{ l l l | l l l}
	\centering
	%\multicolumn{2}{l}{payoff matrix} \vspace{7pt}\\
	%\hline
	$r_{1,(2,1,1)}$ \; & $r_{2,(2,1,1)}$ \; & $r_{3,(2,1,1)}$ \; 
     & \, $r_{1,(2,1,2)}$ \; & $r_{2,(2,1,2)}$ \; &  $r_{3,(2,1,2)}$\\
	\hline
	$r_{1,(2,2,1)}$ \; & $r_{2,(2,2,1)}$ \; & $r_{3,(2,2,1)}$ \; 
     & \, $r_{1,(2,2,2)}$ \; & $r_{2,(2,2,2)}$ \; & $r_{3,(2,2,2)}$\\
	%\hline
\end{tabular}
$\approx$
\begin{tabular}[c]{ l l l | l l l }
	%\multicolumn{2}{c}{} \vspace{7pt}\\
	\centering
	% & \\
	%\hline
	34.68 \; &  21.53 \; &  29.80 \;  & \, 33.70  \; &   26.51 \; &   29.74\\
	\hline
	33.09 \; &  21.76 \; &  34.91 \; & \, 32.22 \; &  26.52 \; &  34.64\\
	%\hline
\end{tabular}
\vspace{10pt}
\caption{An egoistic election game instance of three parties that has no PNE. The natural function to compute winning probabilities is adopted ($\beta=100, n_i=2$ for $i\in\{1,2,3\}$). Here, for example, $u_{x_{1,1}} = 50+9+3=62$, $u_{x_{2,1}} = 22+42+15=79$, $u_{x_{3,1}}=36+1+44=81$. Then we have $p_{1,(1,1,1)} = 62/(62+79+81)\approx 0.2793$, $p_{2,(1,1,1)} = 79/(62+79+81)\approx 0.3559$, $p_{3,(1,1,1)} = 0.3649$. So $r_{1,1,1} = 50\cdot p_{1,(1,1,1)} + 22\cdot p_{2,(1,1,1)} + 36\cdot p_{3,(1,1,1)}\approx 34.93$. All the eight profiles are not PNE. For example, consider profile $(1,1,1)$. Party player $\mathcal{P}_2$ has the incentive to change her strategy to~$x_{2,2}$ from~$x_{2,1}$ because $r_{2,(1,2,1)} = 18.29 > 17.82 = r_{2,(1,1,1)}$. This example is clearly egoistic because $u_1(x_{1,2}) = 44 > \max\{u_1(x_{2,1}), u_1(x_{3,1})\} = 36$, $u_2(x_{2,2}) = 40 > \max\{u_2(x_{1,2}), u_2(x_{3,2})\} = 22$ and $u_3(x_{3,2}) = 42 > \max\{u_3(x_{1,2}), u_3(x_{2,2})\} = 32$.}
\label{tab:NoPNE_BT}
\vspace{-20pt}
\end{center}
\end{table}



\begin{table}[ht]
\begin{center}
\begin{tabular}[c]{ l l l | l l l | l l l }
	%\centering
	$u_1(x_{1,i})$ & $u_2(x_{1,i})$ & $u_3(x_{1,i})$ & 
        $u_1(x_{2,i})$ & $u_2(x_{2,i})$ & $u_3(x_{2,i})$ & 
        $u_1(x_{3,i})$ & $u_2(x_{3,i})$ & $u_3(x_{3,i})$
        \\
	\hline
	29  &  4   &  21  &  23  &  59  &  7  &  8  &  32  &  54 \\
        27  &  43  &  3 &  3  &  57  &  38  &  20  &  13  &  53 \\
	\hline
\end{tabular}
\vspace{7pt}\\
\begin{tabular}[c]{ l l l | l l l}
	\centering
	%\multicolumn{2}{l}{payoff matrix} \vspace{7pt}\\
	%\hline
	$r_{1,(1,1,1)}$ \; & $r_{2,(1,1,1)}$ \; & $r_{3,(1,1,1)}$ \; 
     & \, $r_{1,(1,1,2)}$ \; & $r_{2,(1,1,2)}$ \; & $r_{3,(1,1,2)}$\\
	\hline
	$r_{1,(1,2,1)}$ \; & $r_{2,(1,2,1)}$ \; & $r_{3,(1,2,1)}$ \; 
     & \, $r_{1,(1,2,2)}$ \; & $r_{2,(1,2,2)}$ \; & $r_{3,(1,2,2)}$\\
	%\hline
\end{tabular}
$\approx$
\begin{tabular}[c]{ l l l | l l l }
	%\multicolumn{2}{c}{} \vspace{7pt}\\
	\centering
	% & \\
	%\hline
	18.81 \; &  34.64  \;  &  28.51  \; & \, 23.49  \; &  27.82 \; &   27.38\\
	\hline
	11.27 \; &  34.67  \; &  39.70  \; & \, 15.57  \; &  28.09 \; &  38.93\\
	%\hline
\end{tabular}
\vspace{7pt}\\
\begin{tabular}[c]{ l l l | l l l}
	\centering
	%\multicolumn{2}{l}{payoff matrix} \vspace{7pt}\\
	%\hline
	$r_{1,(2,1,1)}$ \; & $r_{2,(2,1,1)}$ \; & $r_{3,(2,1,1)}$ \; 
     & \, $r_{1,(2,1,2)}$ \; & $r_{2,(2,1,2)}$ \; &  $r_{3,(2,1,2)}$\\
	\hline
	$r_{1,(2,2,1)}$ \; & $r_{2,(2,2,1)}$ \; & $r_{3,(2,2,1)}$ \; 
     & \, $r_{1,(2,2,2)}$ \; & $r_{2,(2,2,2)}$ \; & $r_{3,(2,2,2)}$\\
	%\hline
\end{tabular}
$\approx$
\begin{tabular}[c]{ l l l | l l l }
	%\multicolumn{2}{c}{} \vspace{7pt}\\
	\centering
	% & \\
	%\hline
	18.74 \; &  44.53 \; &  22.84 \;  & \, 23.18  \; &   38.35 \; &  21.61\\
	\hline
	11.58 \; &  44.25 \; &  33.66 \; & \, 15.67 \; &  38.27 \; &  32.77\\
	%\hline
\end{tabular}
\vspace{6pt}
\caption{An egoistic election game instance of three parties that has no PNE. The softmax function to compute winning probabilities is adopted ($\beta=100, n_i=2$ for $i\in\{1,2,3\}$).}
\label{tab:NoPNE_softmax}
\vspace{-32pt}
\end{center}
\end{table}


As an egoistic election game does not always possess a PNE, it is natural to investigate the hardness of deciding whether an egoistic election game of $m\geq 2$ party players has a PNE. As pointed out by~\`{A}lvarez et al.~\cite{AGS2011}, the computational complexity for such an equilibrium-finding problem can be much different with respect to the degrees of succinctness of the input representation. It also deserves to note that in Sect.~4 of~\cite{GGS05}, Gottlob et al. indicate that to determine if a game in the standard normal form\footnote{A game is in the standard normal form if the payoffs of players can be explicitly represented by a single table or matrix.} has a PNE can be solved in logarithmic space, and hence in polynomial time, though such a representation of a game instance is very space consuming. Instead, we consider the \emph{general form} representation~\cite{AGS2011}, which succinctly represents a game instance in a tuple of players, their action sets and a deterministic Turing machine, such that the payoffs are not required to be given explicitly. Note that to compute a PNE of a game under the general form representation is {\sf NP}-hard~\cite{AGS2011}. We conjecture that to determine if the election game has a PNE, for general $m\geq 2$ parties, is {\sf NP}-complete, even it has the egoistic property. 
Specifically, the input of the game can be represented by a tuple $(X_1,X_2,\ldots,X_m,f_{\mathcal{G}})$, such that for each $i\in [m]$, $X_i$ consists of a vector of~$m$ real values each of which corresponds to the utility for the corresponding supporters of each party, and $f_{\mathcal{G}}$ is the function that computes the payoff for each party player. Clearly, the input size is not a tabular of size $\prod_{i\in m}n_i = \Omega(2^m)$, but is $O(nm^2)$ instead. Indeed, as we can still generate a payoff matrix of the election game to have its standard normal form, however, not all of the entries are uncorrelated so such a game representation results in redundancy. 


\begin{table}
\begin{center}
\begin{tabular}[c]{ l l l | l l l | l l l }
	%\centering
	$u_1(x_{1,i})$ & $u_2(x_{1,i})$ & $u_3(x_{1,i})$ & 
        $u_1(x_{2,i})$ & $u_2(x_{2,i})$ & $u_3(x_{2,i})$ & 
        $u_1(x_{3,i})$ & $u_2(x_{3,i})$ & $u_3(x_{3,i})$
        \\
	\hline
	67  &  10   &  9  &  11  &  45  &  9  &   2  &  25  &  53 \\
        66  &  9   &  11  &   1  &  43  & 27  &  41  &   6  &  49 \\
	\hline
\end{tabular}
\vspace{3pt}\\
\begin{tabular}[c]{ l l l | l l l}
	\centering
	%\multicolumn{2}{l}{payoff matrix} \vspace{7pt}\\
	%\hline
	$r_{1,(1,1,1)}$ \; & $r_{2,(1,1,1)}$ \; & $r_{3,(1,1,1)}$ \; 
     &  \, $r_{1,(1,1,2)}$ \; & $r_{2,(1,1,2)}$ \; & $r_{3,(1,1,2)}$\\
	\hline
	$r_{1,(1,2,1)}$ \; & $r_{2,(1,2,1)}$ \; & $r_{3,(1,2,1)}$ \; 
     &  \, $r_{1,(1,2,2)}$ \; & $r_{2,(1,2,2)}$ \; & $r_{3,(1,2,2)}$\\
	%\hline
\end{tabular}
$\approx$
\begin{tabular}[c]{ l l l | l l l }
	%\multicolumn{2}{c}{} \vspace{7pt}\\
	\centering
	% & \\
	%\hline
	28.73 \; &  25.04  \;  &  24.24  \; & \, 42.16  \; &  17.66 \; &   24.55\\
	\hline
	25.29 \; &  24.95  \; &  29.24  \; & \, 38.61  \; &  17.74 \; &  29.23\\
	%\hline
\end{tabular}
\vspace{3pt}\\
\begin{tabular}[c]{ l l l | l l l}
	\centering
	%\multicolumn{2}{l}{payoff matrix} \vspace{7pt}\\
	%\hline
	$r_{1,(2,1,1)}$ \; & $r_{2,(2,1,1)}$ \; & $r_{3,(2,1,1)}$ \; 
     & \, $r_{1,(2,1,2)}$ \; & $r_{2,(2,1,2)}$ \; &  $r_{3,(2,1,2)}$\\
	\hline
	$r_{1,(2,2,1)}$ \; & $r_{2,(2,2,1)}$ \; & $r_{3,(2,2,1)}$ \; 
     & \, $r_{1,(2,2,2)}$ \; & $r_{2,(2,2,2)}$ \; & $r_{3,(2,2,2)}$\\
	%\hline
\end{tabular}
$\approx$
\begin{tabular}[c]{ l l l | l l l }
	%\multicolumn{2}{c}{} \vspace{7pt}\\
	\centering
	% & \\
	%\hline
	28.36 \; &  24.67 \; &  24.98 \;  & \, 41.81  \; &   17.31 \; &   25.24\\
	\hline
	24.92 \; &  24.59 \; &  29.97 \; & \, 38.27 \; &  17.40 \; &  29.91\\
	%\hline
\end{tabular}
\vspace{3pt}
\caption{A strongly egoistic election game instance of three parties that has no PNE. The natural winning probability function is adopted ($\beta=100, n_i=2$ for $i\in\{1,2,3\}$).}
\label{tab:NoPNE_BT_strong}
\end{center}
\end{table}


\begin{table}
\begin{center}
\begin{tabular}[c]{ l l l | l l l | l l l }
	%\centering
	$u_1(x_{1,i})$ & $u_2(x_{1,i})$ & $u_3(x_{1,i})$ & 
        $u_1(x_{2,i})$ & $u_2(x_{2,i})$ & $u_3(x_{2,i})$ & 
        $u_1(x_{3,i})$ & $u_2(x_{3,i})$ & $u_3(x_{3,i})$
        \\
	\hline
	33  &  42  &   0 &   0  &  93  &   4  &  20  &   6  &  54 \\
        30  &  30  &  29 &   3  &  89  &   1  &   0  &  44  &  50 \\
	\hline
\end{tabular}
\vspace{3pt}\\
\begin{tabular}[c]{ l l l | l l l}
	\centering
	%\multicolumn{2}{l}{payoff matrix} \vspace{7pt}\\
	%\hline
	$r_{1,(1,1,1)}$ \; & $r_{2,(1,1,1)}$ \; & $r_{3,(1,1,1)}$ \; 
     &  \, $r_{1,(1,1,2)}$ \; & $r_{2,(1,1,2)}$ \; & $r_{3,(1,1,2)}$\\
	\hline
	$r_{1,(1,2,1)}$ \; & $r_{2,(1,2,1)}$ \; & $r_{3,(1,2,1)}$ \; 
     &  \, $r_{1,(1,2,2)}$ \; & $r_{2,(1,2,2)}$ \; & $r_{3,(1,2,2)}$\\
	%\hline
\end{tabular}
$\approx$
\begin{tabular}[c]{ l l l | l l l }
	%\multicolumn{2}{c}{} \vspace{7pt}\\
	\centering
	% & \\
	%\hline
	16.38 \; &  49.80   \;  &  18.73   \; & \, 9.55  \; &  61.09 \; &   18.94\\
	\hline
	17.74 \; &  47.67   \; &  17.84   \; & \, 10.74  \; &  59.23 \; &  18.10\\
	%\hline
\end{tabular}
\vspace{3pt}\\
\begin{tabular}[c]{ l l l | l l l}
	\centering
	%\multicolumn{2}{l}{payoff matrix} \vspace{7pt}\\
	%\hline
	$r_{1,(2,1,1)}$ \; & $r_{2,(2,1,1)}$ \; & $r_{3,(2,1,1)}$ \; 
     & \, $r_{1,(2,1,2)}$ \; & $r_{2,(2,1,2)}$ \; &  $r_{3,(2,1,2)}$\\
	\hline
	$r_{1,(2,2,1)}$ \; & $r_{2,(2,2,1)}$ \; & $r_{3,(2,2,1)}$ \; 
     & \, $r_{1,(2,2,2)}$ \; & $r_{2,(2,2,2)}$ \; & $r_{3,(2,2,2)}$\\
	%\hline
\end{tabular}
$\approx$
\begin{tabular}[c]{ l l l | l l l }
	%\multicolumn{2}{c}{} \vspace{7pt}\\
	\centering
	% & \\
	%\hline
	16.11   \; &  45.45   \; &  27.59  \;  & \, 9.57  \; &   56.47 \; &   27.40\\
	\hline
	17.40   \; &  43.36  \; &  26.87  \; & \, 10.71 \; &  54.62 \; &  26.71\\
	%\hline
\end{tabular}
\vspace{3pt}
\caption{A strongly egoistic election game instance of three parties that has no PNE. The softmax winning probability function is adopted ($\beta=100, n_i=2$ for $i\in\{1,2,3\}$).}
\label{tab:NoPNE_softmax_strong}
\end{center}
\end{table}



%%%%%%%%%%%%%%%%%%%%%%%%%%%%%%%%%%%%%%%%%%%%%%%%%%%%%%%%%%%%%%%%%%%%%%
\section{Two Sufficient Conditions for the Existence of PNE in the Egoistic Election Game}
\label{sec:two_sufficient_conditions_PNE}
%%%%%%%%%%%%%%%%%%%%%%%%%%%%%%%%%%%%%%%%%%%%%%%%%%%%%%%%%%%%%%%%%%%%%%


The following theorem provides two sufficient conditions for the egoistic election game to have a pure Nash equilibrium.

\begin{theorem}\label{thm:dominatingNE}
\begin{enumerate}[label=(\alph*)]
\item If for all $i\in [m]$, strategy $x_{i,1}$ weakly surpasses (surpasses, resp.) each $x_{i,j}$ for $j\in [n_i]\setminus\{1\}$ of party player~$\mathcal{P}_i$, then $(x_{1,1}, x_{2,1}, \ldots, x_{m,1})$ is a PNE (the unique PNE, resp.) of the egoistic election game. 
\item If there exists $\mathcal{I}\subset [m]$, $|\mathcal{I}| = m-1$, such that for all $i\in \mathcal{I}$,  
$x_{i,1}$ weakly surpasses each $x_{i,j}$ for $j\in [n_i]\setminus\{1\}$ of party player $\mathcal{P}_i$, then $((x_{i,1})_{i\in\mathcal{I}}, x_{j,s_j^{\#}})$ is a PNE for $j\notin\mathcal{I}$ and $s_j^{\#} = \arg\max_{\ell\in [n_j]} r_{j}((x_{i,1})_{i\in\mathcal{I}}, x_{j,\ell})$. 
\end{enumerate}
\end{theorem}
%\iffalse
\begin{proof}
For the first case, let us consider an arbitrary $i\in [m]$ and an arbitrary $t\in [n_i]\setminus\{1\}$. Denote by $\mathbf{s}$ the profile $(x_{1,1}, x_{2,1}, \ldots, x_{m,1})$. 
Since strategy $x_{i,1}$ weakly surpasses $x_{i,t}$, we have $p_{i,\mathbf{s}}\geq p_{i,(t,\mathbf{s}_{-i})}$ and then $\sum_{j\in [m]\setminus\{i\}} p_{j,\mathbf{s}}\leq \sum_{j\in [m]\setminus\{i\}} p_{j,(t,\mathbf{s}_{-i})}$. Hence, 
\begin{eqnarray*}
r_{i,\mathbf{s}} - r_{i,(t, \mathbf{s}_{-i})} &=& 
\sum_{j\in [m]} p_{j,\mathbf{s}} u_i(x_{j,1}) - \\
& & \left(\sum_{j\in [m]\setminus\{i\}} p_{j,(t, \mathbf{s}_{-i})} u_i(x_{j,1}) + p_{i,(t, \mathbf{s}_{-i})} u_i(x_{i,t}) \right)\\
&=& \sum_{j\in [m]\setminus\{i\}} u_i(x_{j,t})(p_{j,\mathbf{s}}-p_{j,(t,\mathbf{s}_{-i})}) + (p_{i,\mathbf{s}} u_i(x_{i,1}) - p_{i,(t,\mathbf{s}_{-i})} u_i(x_{i,t}))\\
&\geq& \sum_{j\in [m]\setminus\{i\}} u_i(x_{i,t})(p_{j,\mathbf{s}}-p_{j,(t,\mathbf{s}_{-i})}) +  u_i(x_{i,t}) (p_{i,\mathbf{s}} - p_{i,(t,\mathbf{s}_{-i})})\\
&=& u_i(x_{i,t})\left( \sum_{j\in [m]\setminus\{i\}} p_{j,\mathbf{s}} - p_{j,(t,\mathbf{s}_{-i})} + (p_{i,\mathbf{s}} - p_{i,(t,\mathbf{s}_{-i})})\right)\\
&=& u_i(x_{i,t})\left(\sum_{j\in [m]} p_{j,\mathbf{s}} - \sum_{j\in [m]} p_{j,(t,\mathbf{s}_{-i})} \right) = 0, 
\end{eqnarray*}
where the inequality follows from the egoistic property and the assumption that strategy~$x_{i,1}$ weakly surpasses~$x_{i,t}$, and last equality holds since $\sum_{j\in [m]} p_{j,\mathbf{s}} = \sum_{j\in [m]} p_{j,(t,\mathbf{s}_{-i})} = 1$ by the law of total probability. Thus, party player $i$ has no incentive to deviate from its current strategy. Note that the uniqueness of the PNE comes when the inequality becomes ``greater-than". This happens as strategy~$x_{i,1}$ surpasses~$x_{i,t}$. 

For the second case, by the same arguments for the first case, we know that for $i\in \mathcal{I}$, party player $i$ has no incentive to deviate from its current strategy. Let $j\in [m]\setminus\mathcal{I}$ be the only one party player not in~$\mathcal{I}$. As $s_j^{\#}$ is the best response when the other strategies of parties in $\mathcal{I}$ are fixed, party player $j$ has no incentive to deviate from its current strategy. Therefore, the theorem is proved. 
\qed
\end{proof}
For example, consider $m=2$ (i.e., only two parties exist in the society). Namely, if strategy $x_{1,1}$ of party $\mathcal{P}_1$ (weakly) surpasses each $x_{1,t}$ for $2\leq t\leq n_1$ and strategy $x_{2,1}$ of party $\mathcal{P}_2$ (weakly) surpasses each $x_{2,t'}$ for $2\leq t'\leq n_2$, then $(x_{1,1},x_{2,1})$ is a (weakly) dominant-strategy solution.

\iffalse
\begin{lemma}\label{lem:fractionals}
Let $r, s > 0$ be two positive real numbers. Then, for \textcolor{black}{any $d>0$}, $r/s > (r+d)/(s+d)$ if $r > s$ and $r/s < (r+d)/(s+d)$ if $r < s$. 
\end{lemma}
\begin{proof}
Let $r, s > 0$ be two positive real numbers. Then $r/s - (r+d)/(s+d) = d(r-s)/(s^2+sd)$. So $r/s - (r+d)/(s+d) > 0$ if $r > s$ and $r/s - (r+d)/(s+d) < 0$ if $r < s$. 
%The lemma is then proved. 
\qed
\end{proof}
\fi


\iffalse
To make our discussion self-contained, consider the problem setting as follows. We have $k$ party players $\mathcal{P}_1,\mathcal{P}_2,\ldots,\mathcal{P}_k$ and each party player $\mathcal{P}_i$ has $n_i$ candidates as her pure strategies. We are also provided the probability model to compute each party player's winning probability in the election as well as her expected utility as her payoff. Hence, we have a payoff table that consists of $n_1\cdot n_2\cdot \cdots \cdot n_k$ entries as the input, and each entry corresponds to 
a strategy profile of the game. Clearly, it suffices to scan all the entries and for each entry it takes polynomial time check if it is a pure Nash equilibrium. Therefore, to decide whether an election game has a pure Nash equilibrium is polynomially tractable. 
\fi


%%%%%%%%%%%%%%%%%%%%%%%%%%%%%%%%%%%%%%%%%%%%%%%%%%%%%%%%%%%%%%%%%%%%%%
\section{A Fixed-Parameter Algorithm for Finding a Pure Nash Equilibrium of the Egoistic Election Game}
\label{sec:fpt-algo}
%%%%%%%%%%%%%%%%%%%%%%%%%%%%%%%%%%%%%%%%%%%%%%%%%%%%%%%%%%%%%%%%%%%%%%



In the following we provide the formal definitions of a parameterized problem and a fixed-parameter tractability to make our discussions self-contained. %Definition~\ref{defn:parameterized_problems}. ~\ref{defn:fpt}. 


\begin{definition}[\cite{DF13,FG06,Nie06}]
\label{defn:parameterized_problems}
A parameterized problem is a language $\mathcal{L}\subseteq \Sigma^* \times \Sigma^*$, where $\Sigma$ is a finite alphabet. The second component is called the parameter(s) of the problem.
\end{definition}


\begin{definition}[\cite{DF13,FG06,Nie06}]
\label{defn:fpt}
A parameterized problem $L$ is fixed-parameter tractable (FPT) if, for all $(x, k)\in \mathcal{L}$, whether $(x,k)\in \mathcal{L}$ can be determined in $f(k)\cdot n^{O(1)}$ time, where
$f$ is a computable function that depends only on~$k$.
\end{definition}


Below we give a useful lemma which states that a profile with a player's strategy surpassed by other one of it is never a PNE. With a slight abuse of notation, we abbreviate the profile $(x_{1,s_1}, x_{2,s_2},\ldots,x_{m,s_m})$ by $\mathbf{s} = (s_1,s_2,\ldots,s_m)$. 
 
\begin{lemma}\label{lem:dominated_Not_PNE}
If $s_i$ is surpassed by some $s'_i\in [n_i]\setminus\{s_i\}$, then $(s_i,(\tilde{s}_j)_{j\in [m]\setminus\{i\}})$ is not a PNE for any profile $(\tilde{s}_j)_{j\in [m]\setminus\{i\}}$ except $s_i$. 
\end{lemma}
\begin{proof}
Let $\mathbf{s}_{-i} = (\tilde{s}_j)_{j\in [m]\setminus\{i\}}$ be any profile except party player $\mathcal{P}_i$'s strategy. 
If $s_i$ is surpassed by $s'_i$, then we have $s'_i < s_i$, and either:
\begin{itemize}
\item [(a)] $u(x_{i,s'_i}) > u(x_{i,s_i})$ and $u_i(x_{i,s'_i})\geq u_i(x_{i,s_i})$, or
\item [(b)] $u(x_{i,s'_i})\geq u(x_{i,s_i})$ and $u_i(x_{i,s'_i})>u_i(x_{i,s_i})$. 
\end{itemize}
We prove case (a) as follows. Case (b) can be similarly proved. 

We have $u_i(x_{i,s'_i})\geq u_i(x_{i,s_i})$ from the assumption in (a). Also, by the monotone property of the WP functions, we have $p_{i,(s'_i, \mathbf{s}_{-i})} > p_{i,(s_i,\mathbf{s}_{-i})}$ and $\sum_{j\in [m]\setminus\{i\}} p_{j, (s'_i, \mathbf{s}_{-i})}< \sum_{j\in [m]\setminus\{i\}} p_{j, (s_i,\mathbf{s}_{-i})}$ since $u(x_{i,s'_i}) > u(x_{i,s_i})$. Thus,
\begin{eqnarray*}
& & r_{i}(s_i,\mathbf{s}_{-i}) - r_{i}(s'_i,\mathbf{s}_{-i}) = \left(p_{i,(s_i,\mathbf{s}_{-i})}u_i(x_{i,s_i}) + \sum_{j\in [m]\setminus\{i\}} p_{j,(s_i,\mathbf{s}_{-i})}u_i(x_{j,\tilde{s}_j})\right) \\ 
& & - \left(p_{i,(s'_i,\mathbf{s}_{-i})}u_i(x_{i,s'_i})+\sum_{j\in [m]\setminus\{i\}} p_{j,(s'_i,\mathbf{s}_{-i})}u_i(x_{j,\tilde{s}_j})\right)\\
&=& \left(\sum_{j\in [m]\setminus\{i\}} (p_{j,(s_i,\mathbf{s}_{-i})} - p_{j,(s'_i,\mathbf{s}_{-i})}) u_i(x_{j,\tilde{s}_j})\right) + (p_{i,(s_i,\mathbf{s}_{-i})}u_i(x_{i,s_i}) - (p_{i,(s'_i,\mathbf{s}_{-i})}u_i(x_{i,s'_i}))\\
&<& u_i(x_{i,s_i})\left(\sum_{j\in [m]\setminus\{i\}} (p_{j,(s_i,\mathbf{s}_{-i})} - p_{j,(s'_i,\mathbf{s}_{-i})})\right)  + (p_{i,(s_i,\mathbf{s}_{-i})}u_i(x_{i,s_i}) - (p_{i,(s'_i,\mathbf{s}_{-i})}u_i(x_{i,s'_i}))\\
&=& u_i(x_{i,s_i})(p_{i,(s'_i,\mathbf{s}_{-i})} - p_{i,(s_i,\mathbf{s}_{-i})}) + (p_{i,(s_i,\mathbf{s}_{-i})}u_i(x_{i,s_i}) - p_{i,(s'_i,\mathbf{s}_{-i})}u_i(x_{i,s'_i}))\\
&\leq& (p_{i,(s'_i,\mathbf{s}_{-i})} - p_{i,(s_i,\mathbf{s}_{-i})})(u_i(x_{i,s_i})-u_i(x_{i,s_i})) = 0,
\end{eqnarray*}
where the first inequality follows from the egoistic property. 
Thus, $(s_i,\mathbf{s}_{-i})$ is not a PNE.
\qed
\end{proof}


%We have learned two sufficient conditions for the egoistic election game to have a pure Nash equilibrium. 
Inspired by Theorem~\ref{thm:dominatingNE} and Lemma~\ref{lem:dominated_Not_PNE}, we are able to devise an efficient algorithm to find out a PNE of the egoistic election game whenever it exists, with respect to two parameters: number of \emph{chaotic parties} and \emph{nominating depth}, which are introduced as follows. As we have assumed, candidates in each party are sorted according to the utility for its supporters. That is, $u_i(x_{i,1})\geq u_i(x_{i,2})\geq \ldots \geq u_i(x_{i,n_i})$ for each $i\in [m]$. Let $d_i$ be the index of the candidate that surpasses all candidates $x_{i,d_i+1},\ldots,x_{i,n_i}$ or is set to~$n_i$ if $x_{i,n_i}$ is not surpassed by any candidates of~$\mathcal{P}_i$. Formally, let $\mbox{maxProb}_i := \argmax_{s\in [n_i]}u(x_{i,s})$, then $d_i = \max\{ \arg\max_{s\in \mbox{\scriptsize maxProb}_i} u_i(x_{i,s})\}$. From Theorem~\ref{thm:dominatingNE} we know that $x_{i,d_i}$ is a dominant strategy for the ``sub-game instance" in which $\mathcal{P}_i$'s strategy set is reduced to~$\{x_{i,s}\}_{s\in\{d_i,d_i+1,\ldots,n_i\}}$. 
As for each $i\in\mathcal{D}$, party player $\mathcal{P}_i$ must choose the first candidate $x_{i,1}$. 
Thus, we can reduce the game instance to $(\tilde{X}_i)_{i\in [m]\setminus\mathcal{D}}$, where $\tilde{X}_i = \{x_{i,s}\}_{s\in [d_i]}$. We call  $d:=\max_{i\in [m]} d_i$ the \emph{nominating depth} of the election game. We call a party $\mathcal{P}_i$ \emph{chaotic} if $d_i>1$ and denote by~$k$ the number of chaotic parties. 


We propose Algorithm {\sf FPT-ELECTION-PNE} to compute a PNE of the egoistic election game. The complexity of the algorithm is $O(nm^2 + kd^{k+1}m)$. Thus, to compute a PNE for such a game is \emph{fixed-parameter tractable} with respect to the two parameters~$d$ and~$k$. The results are proved and concluded in  Theorem~\ref{thm:PNE_compute_tractable}. 


\begin{algorithm}\label{alg:fpt_algo}
\caption{$\textbf{\sf FPT-ELECTION-PNE}$}
\begin{algorithmic}[0]
\REQUIRE an election game instance $\mathcal{G} = (X_1,X_2,\ldots,X_m,f_{\mathcal{G}})$. 
\end{algorithmic}
\begin{algorithmic}[1]
\STATE For each $i$, compute $\mbox{maxProb}_i = \argmax_{s\in [n_i]}u(x_{i,s})$. 
\STATE For each $i$, compute $d_i = \max\{ \arg\max_{s\in \mbox{\scriptsize maxProb}_i} u_i(x_{i,s})\}$.  
%\STATE Compute $d_i = \arg\max_{s\in [n_i]} u_i(x_{i,s})$ for each~$i$. \COMMENT{Preprocessing}
\STATE Collect $\mathcal{D} = \{i\mid i\in [m], d_i = 1\}$ and assign $x_{i,1}$ to party player $\mathcal{P}_i$ for $i\in\mathcal{D}$, 
\STATE Reduce the game instance to $(\tilde{X}_i)_{i\in [m]\setminus\mathcal{D}}$, where $\tilde{X}_i = \{x_{i,s}\}_{s\in [d_i]}$, $i\in [m]\setminus\mathcal{D}$. 
%\STATE Calculate $\sum_{i\in \mathcal{D}}u(x_{i,1})$.\COMMENT{For later use of calculating payoffs}
\FOR{each entry $\mathbf{s}\in \prod_{i\in \mathcal{D}}\{x_{i,1}\}\times\prod_{j\in [m]\setminus\mathcal{D}} \tilde{X}_j$}
\STATE Compute the payoff $r_i(\mathbf{s})$ for each $i$.
\IF{$\mathbf{s}$ corresponds to a PNE\COMMENT{check if any unilateral deviation is possible}}
\STATE \COMMENT{for each $j\in [m]\setminus\mathcal{D}$, check if $r_j(\mathbf{s})\geq r_j(s'_j,\mathbf{s}_{-j})$ for all $s'_j\in \tilde{X}_j$}
\STATE return $\mathbf{s}$
\ENDIF
\ENDFOR
\STATE Output ``NO"
\end{algorithmic}
\end{algorithm}


\begin{theorem}\label{thm:PNE_compute_tractable}
Given an election game instance $\mathcal{G}$ of $m\geq 2$ parties each of which has at most $n$ candidates. Suppose that $\mathcal{G}$ has at most~$k$ chaotic parties and the nominating depth of~$\mathcal{G}$ is bounded by~$d$, then to compute a PNE of~$\mathcal{G}$ can be done in $O(nm^2 + kd^{k+1}m)$ time if it exists.     
\end{theorem}
\begin{proof}
It costs $O(nm^2)$ time to compute $d$ and $d_i$ for all~$i$, and the set $\mathcal{D}$ can be then obtained. Since playing strategy~1 (i.e., nominating the first candidate) for parties $i\in\mathcal{D}$ is the dominant strategy, we only need to consider party players in~$[m]\setminus\mathcal{D}$. Since for each $i\in [m]\setminus\mathcal{D}$, each strategy in~$\{x_{i,d_i+1},\ldots,x_{i,n_i}\}$ is surpassed by $x_{i,d_i}$ (considering the subgame with respect to~$(x_{i,d_i+1},\ldots,x_{i,n_i})_{i\in [m]}$), by Lemma~\ref{lem:dominated_Not_PNE} we know that we only need to consider strategies $\{x_{i,1},\ldots,x_{i,d_i}\}$ for such party player~$i$. Thus, the number of entries of the implicit payoff matrix
that we need to check whether it is a PNE is $\prod_{i\in [m]\setminus\mathcal{D}} d_i = O(d^k)$, where $k = |[m]\setminus \mathcal{D}|$. 
For each of these entries, it takes $O(m)$ time to compute the winning probabilities as well as the payoffs of the party players in~$[m]\setminus\mathcal{S}$. 
In addition, checking whether each of such entries can be done in at most~$k(d-1)$ steps. Therefore, the theorem is proved.
\qed
\end{proof}


\iffalse
We first show that the two party in the Bradley-Terry model, 
a PNE does not always exist no matter whether egoism exists or not.
Throughout the rest of Section~\ref{sec:m2n2_NE}, the existence of PNE in \emph{egoistic games} is studied.
We then start with the case of two candidates per party, and with the help of Lemma~\ref{lem:dominatingNE} prove existence of PNE in both the linear link model and the softmax model, also using critical lemmas dealing with the complemented condition in Lemma~\ref{lem:dominatingNE}. We eventually reuse these critical lemmas to show existence of PNE for the case of more than two candidates per party.


We claim that the two party election game in the Bradley-Terry model, 
a PNE does not always exist. 
As the upper instance illustrated in Table~\ref{tab:NoPNE_BT}, $m = n = 2$, $u_A(A_1) = 91$, 
$u_B(A_1) = 0$, $u_A(A_2) = 90$, $u_B(A_2) = 8$, $u_B(B_1) = 11$, $u_A(B_1) = 1$, $u_B(B_2) = 10$,
$u_A(B_2) = 20$. We have $p_{1,1} = 91/(91+12)\approx 0.88$, 
$p_{1,2} = 91/(91+30)\approx 0.76$, $p_{2,1} = 98/(98+12)\approx 0.89$, and 
$p_{2,2} = 98/(98+30)\approx 0.77$. 
%Intuitively, this example says that in the Bradley-Terry Model a pure Nash equilibrium may not exist when each party faces a choice between one candidate who benefits almost only the supporters of his or her party and the other candidate who almost matches his or her comrade in terms of the benefits to their own supporters but also benefits the supporters of the other party in some amounts. 
Hence, we obtain the first matrix in Table~\ref{tab:NoPNE_BT}. Furthermore, as the example is egoistic, it also implies that the game in the Bradley-Terry model may not have a PNE even with egoism guarantee. The second instance in Table~\ref{tab:NoPNE_BT} gives a non-egoistic example of no PNE in the Bradley-Terry model. 

\begin{table}[ht]
\begin{center}
\begin{tabular}[c]{ l l | l l }
	%\centering
	\multicolumn{4}{ c }{}\\
	$A$ & \multicolumn{1}{c}{}& $B$ & \\
	\hline
	$u_A(A_i)$ & $u_B(A_i)$ & $u_B(B_j)$ & $u_A(B_j)$\\
	\hline
	91  &  0  &  11  &  1\\
	90  &  8  &  10  &  20\\
	\hline
\end{tabular} \;\;\;\;
\begin{tabular}[c]{ l l | l l }
	%\centering
	\multicolumn{4}{ c }{}\\
	$A$ & \multicolumn{1}{c}{}& $B$ & \\
	\hline
	$u_A(A_i)$ & $u_B(A_i)$ & $u_B(B_j)$ & $u_A(B_j)$\\
	\hline
	44  &  10  &  37  &  17\\
	39  &  55  &  10  &  5\\
	\hline
\end{tabular}
\vspace{7pt}\\
\begin{tabular}[c]{ l | l | l}
	\centering
	%\multicolumn{2}{l}{payoff matrix} \vspace{7pt}\\
	%\hline
	&$B_1$  &  $B_2$\\
	\hline
	$A_1$&$a_{1,1}$, $b_{1,1}$  &  $a_{1,2}$, $b_{1,2}$\\
	\hline
	$A_2$&$a_{2,1}$, $b_{2,1}$  &  $a_{2,2}$, $b_{2,2}$\\
	%\hline
\end{tabular}
$\approx$
\begin{tabular}[c]{  l | l | l }
	%\multicolumn{2}{c}{} \vspace{7pt}\\
	\centering
	% & \\
	%\hline
	&$B_1$  &  $B_2$\\
	\hline
	$A_1$&80.51, 1.28  &  73.84, 2.17\\
	\hline
	$A_2$&80.29, 8.32  &  74.02, 8.23\\
	%\hline
\end{tabular}
,\;\;
\begin{tabular}[c]{  l | l | l }
	%\multicolumn{2}{c}{} \vspace{7pt}\\
	\centering
	% & \\
	%\hline
	&$B_1$  &  $B_2$\\
	\hline
	$A_1$&30.50, 23.50  &  35.52, 10.00\\
	\hline
	$A_2$&30.97, 48.43  &  34.32, 48.81\\
	%\hline
\end{tabular}
\vspace{6pt}
\caption{Two examples of No PNE in the Bradley-Terry model ($m = n = 2, b=100$). Left one is egoistic while the right one is not.}
\label{tab:NoPNE_BT}
\end{center}
\end{table}
\fi



\iffalse
%=====================================================================

\subsection{Two Candidates per Party}

There are exactly two scenarios for the two-party election game with $m = n = 2$ to have no pure Nash equilibrium, as illustrated in Fig.~\ref{fig:2by2_noPNE}. The arrows show the deviations. 
For example, $(D_1)$ stands for $A$'s unilateral deviation from strategy~1 to~2 given $B$ staying at 
strategy~1 and $(D_2)$ stands for $B$'s unilateral deviation from strategy~1 to~2 given $A$ staying at 
strategy~2, while $(D_3)$ stands for $A$'s unilateral deviation from strategy~2 to strategy~1 given $B$ 
staying at strategy~2 and $(D_4)$ stands for $B$'s unilateral deviation from strategy~2 to strategy~1 given $A$ 
staying at strategy~1. 

That is, if the game has no PNE, then at state $(1,1)$, either $A$ or $B$ deviates from strategy~1 to~2. For the former case, since no PNE exists, $B$ wants to deviate unilaterally from 
strategy~1 to~2 at state $(2,1)$, then the game reaches state $(1,2)$, in which $A$ wants to deviate 
unilaterally from strategy~2 to~1. Finally, the entries in the payoff matrix as well as the deviation 
arrows form a cycle as the left scenario of Fig.~\ref{fig:2by2_noPNE} shows. Likewise, the latter 
case corresponds to the right scenario of the Fig.~\ref{fig:2by2_noPNE}. 


\begin{figure}[ht]
	\begin{center}
		\includegraphics[scale=0.35]{eps/deviation_NoPNE.eps}
	\end{center}
	\caption{The two scenarios of having no PNE for $m = n = 2$.}
	\label{fig:2by2_noPNE}
\end{figure}


Let $\Delta(D_i)$ (resp., $\Delta(D'_i)$) denote the gain of payoff by the unilateral deviation $D_i$ 
(resp., $D'_i$) for $i\in \{1,2,3,4\}$. Then, we have 
\begin{eqnarray*}
\Delta(D_1) = -\Delta(D'_1) &=& a_{2,1}-a_{1,1}\\
&=&p_{2,1}u_A(A_2)+(1-p_{2,1})u_A(B_1)\\
&&-(p_{1,1}u_A(A_1)+(1-p_{1,1})u_A(B_1))\\
&=& -p_{1,1}(u_A(A_1)-u_A(A_2))\\ 
&&+ (p_{2,1}-p_{1,1})(u_A(A_2)-u_A(B_1)).\\
\Delta(D_2) = -\Delta(D'_2) &=& b_{2,2}-b_{2,1}\\
&=&(1-p_{2,2})u_B(B_2)+p_{2,2}u_B(A_2)\\ 
&&- ((1-p_{2,1})u_B(B_1)+p_{2,1}u_B(A_2))\\
&=& -(1-p_{2,1})(u_B(B_1)-u_B(B_2))\\
&&+(p_{2,1}-p_{2,2})(u_B(B_2) - u_B(A_2)).
\end{eqnarray*}
Similarly, we derive 
\begin{eqnarray*}
\Delta(D_3) = -\Delta(D'_3) &=& a_{1,2}-a_{2,2}\\ &=&p_{1,2}u_A(A_1)+(1-p_{1,2})u_A(B_2)\\
&&-(p_{2,2}u_A(A_2)+(1-p_{2,2})u_A(B_2))\\
&=& p_{1,2}(u_A(A_1)-u_A(A_2))\\ 
&&+ (p_{1,2}-p_{2,2})(u_A(A_2)-u_A(B_2)).\\
\Delta(D_4) = -\Delta(D'_4) &=& b_{1,1}-b_{1,2}\\ &=&(1-p_{1,1})u_B(B_1)+p_{1,1}u_B(A_1)\\ 
&&- ((1-p_{1,2})u_B(B_2)+p_{1,2}u_B(A_1))\\
&=& (1-p_{1,1})(u_B(B_1)-u_B(B_2))\\
&&+(p_{1,2}-p_{1,1})(u_B(B_2) - u_B(A_1)).
\end{eqnarray*}


In order to check if the two-party election game with $m=n=2$ always has a PNE, 
Lemma~\ref{lem:dominatingNE} suggests us to focus on the case that $u(A_2) > u(A_1)$ and 
$u(B_2) > u(B_1)$ since when either $u(A_1) \geq u(A_2)$ or $u(B_1) \geq u(B_2)$ a PNE always exists. 

\fi


\iffalse
%=====================================================================
\subsubsection{Linear Link Model}
\label{subsec:PNE_LL}
%=====================================================================


In this subsection, we study the linear link model. We show that the two-party election game always has a PNE in this model. 
As previously discussed in this section, we assume that $u(A_2) > u(A_1)$ and 
$u(B_2) > u(B_1)$. The following lemma tells us at least one of the deviations $D_2$ 
and~$D_4$ fails to happen, and analogously, at least one of deviations $D'_1$ and $D'_3$ fails to happen. 


\begin{lemma}\label{lem:devConf_LL}
Consider the two-party election game in the linear link model. Suppose that $u(A_2) > u(A_1)$, then $\Delta(D_4) < 0$ (resp., $\Delta(D_2) < 0$) if $\Delta(D_2) > 0$ (resp., $\Delta(D_4) > 0$). Likewise, suppose that $u(B_2) > u(B_1)$, then $\Delta(D'_3) < 0$ (resp., $\Delta(D'_1) < 0$) if $\Delta(D'_1) > 0$ (resp., $\Delta(D'_3) > 0$). 

\end{lemma}
\begin{proof}
	Suppose we have $u(A_2) > u(A_1)$. To ease the notation, let 
	$\hat{p} = 1-p_{2,1}$, $p' = 1-p_{1,1}$, and $\delta = p_{2,1} - p_{2,2} = p_{1,1} - p_{1,2} = (u(B_2)-u(B_1))/2b$. 
	
	To prove the lemma by contradiction, we assume and focus that $\Delta(D_2)> 0$ yet $\Delta(D_4)\geq 0$, as the other case that $\Delta(D_4)> 0$ yet $\Delta(D_2)\geq 0$ can be proved in the same way. First, from the definitions of $D_4$ and $D_2$, we have  
	\[
	\left\{
	\begin{array}{ccc}
	%\begin{eqnarray*}
	p'(u_B(B_1)-u_B(B_2)) &\geq& \delta\cdot (u_B(B_2)-u_B(A_1)),\\
	\vspace{-5pt}\\
	\hat{p}(u_B(B_1)-u_B(B_2)) &<& \delta\cdot (u_B(B_2)-u_B(A_2)).
	%\end{eqnarray*} 
	\end{array}
	\right.
	\]
	We claim that $\hat{p}, p'>0$. Recall that $p_{1,1} \geq p_{2,1}$ which implies that $\hat{p}\geq p'$, then $p'= 0$ must be true if one of $\hat{p}$ and $p'$ is~$0$. Substituting $p'=0$ in the first inequality above we have $0\geq \delta\cdot (u_B(B_2)-u_B(A_1))$, which contradicts that  $u_B(B_2) > u_B(A_1)$ by egoism assumption. Dividing the inequalities by $p'$ and $\hat{p}$ respectively, we have 
	\[
	\left\{
	\begin{array}{ccc}
	%\begin{eqnarray*}
	u_B(B_1)-u_B(B_2) &\geq& \delta\cdot (u_B(B_2)-u_B(A_1))/p',\\
	\vspace{-5pt}\\
	u_B(B_1)-u_B(B_2) &<& \delta\cdot (u_B(B_2)-u_B(A_2))/\hat{p}.
	%\end{eqnarray*} 
	\end{array}
	\right.
	\]
	Note that $u_B(B_2)> u_B(A_2)$ and $u_B(B_2)> u_B(A_1)$ due to egoism, so $\delta\leq 0$ will make the second inequality fail to hold. Hence we proceed with $\delta > 0$ and compare the right-hand sides of the above inequalities: 
	\begin{eqnarray} \nonumber
	&&\frac{(u_B(B_2)-u_B(A_1))/p'}{(u_B(B_2)-u_B(A_2))/\hat{p}}\\ \nonumber
	&=& 
	\frac{u_B(B_2)-u_B(A_1)}{u_B(B_2)-u_B(A_2)}\cdot \frac{1+(u(B_1)-u(A_2))/b}{1+(u(B_1)-u(A_1))/b}\\
	&=& \frac{u_B(B_2)-u_B(A_1)}{u_B(B_2)-u_B(A_2)}\cdot \frac{b+(u(B_1)-u(A_2))}{b+(u(B_1)-u(A_1))}. \label{eq:1}
	\end{eqnarray}
Note that $u_B(A_2) > u_B(A_1)$ since $u(A_2) > u(A_1)$ and $u_A(A_1)\geq u_A(A_2)$. Then we have 
	\begin{eqnarray*}
	&&\frac{u_B(B_2)-u_B(A_1)}{u_B(B_2)-u_B(A_2)}\\ 
	&\geq & 
	\frac{u_B(B_2)-u_B(A_1)+(u(B_1)-u_B(B_2))}{u_B(B_2)-u_B(A_2)+(u(B_1)-u_B(B_2))}\quad \mbox{(by Lemma~\ref{lem:fractionals})}\\
	&= & \frac{u(B_1)-u_B(A_1)}{u(B_1)-u_B(A_2)}\\
	&\geq& \frac{u(B_1)-u_B(A_1)+(b-u_A(A_1))}{u(B_1)-u_B(A_2)+(b-u_A(A_1))}\quad \mbox{(by Lemma~\ref{lem:fractionals})} \\
	&\geq & \frac{u(B_1)-u_B(A_1)+(b-u_A(A_1))}{u(B_1)-u_B(A_2)+(b-u_A(A_2))}\quad \mbox{(since $u_A(A_1)\geq u_A(A_2)$)}\\
	& =& \frac{u(B_1)-u(A_1)+b}{u(B_1)-u(A_2)+b},  \\
	\end{eqnarray*} 
Finally, together with Equation~(\ref{eq:1}), we derive that 
\begin{eqnarray*}
&&\frac{(u_B(B_2)-u_B(A_1))/p'}{(u_B(B_2)-u_B(A_2))/\hat{p}}\\
&\geq& 
\frac{u(B_1)-u(A_1)+b}{u(B_1)-u(A_2)+b}\cdot  
\frac{b+(u(B_1)-u(A_2))}{b+(u(B_1)-u(A_1))}= 1,
\end{eqnarray*}
which implies that $u_B(B_1)-u_B(B_2) > u_B(B_1)-u_B(B_2)$, hence a contradiction occurs. For the second part of the lemma that $u(B_2) > u(B_1)$, the proof can be similarly derived. 
\qed	
\end{proof}

Theorem~\ref{thm:PNE_LL} holds by Lemma~\ref{lem:dominatingNE} and~\ref{lem:devConf_LL}.

\begin{theorem}\label{thm:PNE_LL}
In the linear link model with $m = n = 2$, the two-party election game always has a PNE.  
\end{theorem}

\fi

\iffalse
%=====================================================================
\subsubsection{Softmax Model}
\label{subsec:PNE_SM}
%=====================================================================


In this subsection, we show that the two-party election game always has a PNE of in the softmax model. 

\begin{lemma}\label{lem:devConf_SM}
Consider the two-party election game in the softmax model. Suppose that $u(A_2)> u(A_1)$, then $\Delta(D_4) < 0$ (resp., $\Delta(D_2) < 0$) if $\Delta(D_2)>0$ (resp., $\Delta(D_4) > 0$). Likewise, suppose that $u(B_2)> u(B_1)$, then $\Delta(D'_3) < 0$ (resp., $\Delta(D'_1) < 0$) if $\Delta(D'_1) > 0$ (resp., $\Delta(D'_3) > 0$).  

\end{lemma}
\begin{proof}
    Suppose that $u(B_2)> u(B_1)$. To ease the notation, let 
	\begin{eqnarray*}
	q &=& p_{1,1} = \frac{e^{u(A_1)/b}}{e^{u(A_1)/b} + e^{u(B_1)/b}},\; 
	 q' = p_{2,1} = \frac{e^{u(A_2)/b}}{e^{u(A_2)/b} + e^{u(B_1)/b}},   \;\\ 
	 \delta &=& q' - q.\\
	\hat{q} &=& p_{1,2} = \frac{e^{u(A_1)/b}}{e^{u(A_1)/b} + e^{u(B_2)/b}},\; 
	 \hat{q}'= p_{2,2} = \frac{e^{u(A_2)/b}}{e^{u(A_2)/b} + e^{u(B_2)/b}}, 
	 \;\\ 
	 \delta' &=& \hat{q}' - \hat{q}.
	\end{eqnarray*} 
	To prove the lemma by contradiction, similarly to the argument in the proof of Lemma~\ref{lem:devConf_LL}, we focus on the case that $\Delta(D'_3) > 0$ yet $\Delta(D'_1)\geq 0$. The proof for the other case that $\Delta(D'_1)> 0$ yet $\Delta(D'_3)\geq 0$ is basically the same.  
	
	By definition, $q,q',\hat{q},\hat{q'} > 0$. 
	
	First, as $q,\hat{q}> 0$, $\Delta(D'_3) > 0$ and  $\Delta(D'_1)\geq 0$ implies that   
	\begin{eqnarray*}
		u_A(A_1)-u_A(A_2) &\geq& \delta\cdot \frac{u_A(A_2)-u_A(B_1)}{q},\mbox{ and }\\
		u_A(A_1)-u_A(A_2) &<& \delta'\cdot \frac{u_A(A_2)-u_A(B_2)}{\hat{q}}.
	\end{eqnarray*}
	By the same argument exploited in the proof of Lemma~\ref{lem:devConf_LL}, it follows that $\delta' > 0$ otherwise the second inequality fails. This implies $u(A_2) > u(A_1)$ so that $\delta>0$. Then we have 
		\[
		\left\{
		\begin{array}{llll}
		&&u_A(A_1)-u_A(A_2)\\ 
		&\geq& 
		\frac{e^{u(B_1)/b}(e^{u(A_2)/b} - e^{u(A_1)/b})}{e^{(u(A_1)+u(A_2))/b} + e^{(u(A_1)+u(B_1))/b}}\cdot (u_A(A_2)-u_A(B_1)) 
		& \triangleq \notag{(*)} \\
		\vspace{-7pt}\\
		&&u_A(A_1)-u_A(A_2)\\ 
		&<& \frac{e^{u(B_2)/b}(e^{u(A_2)/b} - e^{u(A_1)/b})}{e^{(u(A_1)+u(A_2))/b}+e^{(u(A_1)+u(B_2))/b}}\cdot (u_A(A_2)-u_A(B_2)) 
		& \triangleq \notag{(**)} 
		\end{array} 
		\right.
		\]
	
	Then, we compare the right-hand sides of the above two 
	inequalities as below.  
	\begin{eqnarray*}
		\frac{(*)}{(**)} &=& \frac{e^{u(B_1)/b}}{e^{u(B_2)/b}}\cdot 
		\frac{e^{u(A_2)/b}+e^{u(B_2)/b}}{e^{u(A_2)/b}+e^{u(B_1)/b}}\cdot 
		\frac{u_A(A_2)-u_A(B_1)}{u_A(A_2)-u_A(B_2)} \\
		&\geq& e^{(u(B_1)-u(B_2))/b}\cdot \frac{u_A(A_2)-u_A(B_1)}{u_A(A_2)-u_A(B_2)}. 
	\end{eqnarray*}
	Let $c = u_A(A_2) - u_A(B_2)$ and $\rho = u(B_2)-u(B_1) >0$. Then, 
	$u_A(A_2) - u_A(B_1) = c+u_A(B_2)-u_A(B_1)\geq c+(u(B_2)-u(B_1)) = c+\rho$. 
	Note that $e^{(u(B_1)-u(B_2))/b}\geq 
	1 + (u(B_1)-u(B_2))/b = 1 - \rho/b$. 	
	Also, $(u_A(A_2)-u_A(B_1))/(u_A(A_2)-u_A(B_2))\geq (c+\rho)/c = 1+\rho/c$. 
	Therefore, 
	\begin{eqnarray*}
		\frac{(*)}{(**)} &\geq & 
		\left(1-\frac{\rho}{b}\right)\cdot \left(1+\frac{\rho}{c}\right)  \\
		& = & 1 + \frac{\rho}{c} - \frac{\rho}{b} -\frac{\rho^2}{bc}\\
		& = & 1 + \rho\cdot \frac{b-c - \rho}{bc} \\
		&\geq& 1.  
	\end{eqnarray*} 
	where the last inequality follows because 
	$b-c-\rho = b - u_A(A_2) + u_A(B_2) - u_B(B_2) - u_A(B_2) + u_B(B_1) + u_A(B_1) 
	= (b-u_A(A_2)) + (u_B(B_1)-u_B(B_2))+u_A(B_1)\geq 0$. 
	Finally, we obtain a contradiction. The other part of the lemma that $u(B_2)> u(B_1)$ can be similarly proved.  
\qed	
\end{proof}


By Lemma~\ref{lem:dominatingNE} and~\ref{lem:devConf_SM}, we obtain Theorem~\ref{thm:PNE_SM} 
as follows. 


\begin{theorem}\label{thm:PNE_SM}
In the softmax model with $m = n = 2$, the two-party election game always has a PNE. 
\end{theorem}

\fi


\iffalse
%%%%%%%%%%%%%%%%%%%%%%%%%%%%%%%%%%%%%%%%%%%%%%%%%%%%%%%%%%%%%%%%%%%%%%
\subsection{More than Two Candidates per Party}
\label{sec:general_NE}
%%%%%%%%%%%%%%%%%%%%%%%%%%%%%%%%%%%%%%%%%%%%%%%%%%%%%%%%%%%%%%%%%%%%%%


In this section, we show existence of PNE for a generalized case in which party $A$ has $m\geq 2$ 
candidates and party $B$ has $n\geq 2$ candidates. Since the game in the Bradley-Terry model may 
not have a PNE, we focus on the linear link model and the softmax model. 


The two-party election game then has $mn$ possible states (i.e., $\{(i,j)\}_{i\in [m],j\in [n]}$).
The states are bijectively mapped to the entries of the payoff matrix. Regard each entry of the 
payoff matrix as a node and a unilateral deviation with positive gain of payoff as an arc, we 
obtain a directed graph and we call it the {\em state graph}. A {\em best-response walk} is a walk 
on the state graph such that each arc of the walk is a best-response unilateral deviation. Since 
the number of states is finite, any best-response walk on the state graph must contain a loop if the 
game has no PNE. We then derive Theorem~\ref{thm:generel_NE} based on this observation.   


\begin{figure}[ht]
	\begin{center}
		\includegraphics[scale=0.40]{eps/generalNE.eps}
	\end{center}
\caption{Illustration for the proof of Theorem~\ref{thm:generel_NE}.}
\label{fig:general_NE}
\end{figure}


\begin{theorem}\label{thm:generel_NE}
The two-party election game with $m\geq 2$ and $n\geq 2$ always has a PNE both in the linear 
link model and the softmax model.
\end{theorem}
\begin{proof}
Assume, for contradiction, that the game has no PNE. Then, a best-response walk starting 
from any node contains a loop. Let $L = (i,j)\rightarrow (i',j)\rightarrow (i',j'')\rightarrow\cdots 
\rightarrow (i,j')\rightarrow (i,j)$ be such a loop, 
in which $i$ is the smallest indexed party $A$ candidate among all nodes (states) of~$L$ 
(See Fig.~\ref{fig:general_NE} for the illustration). 
Let $\alpha_1$ and $\alpha_2$ denote the unilateral deviations $(i,j'')\rightarrow (i,j)$ and 
$(i',j)\rightarrow (i',j'')$ respectively. Since $\alpha_2$ is on~$L$, we have $\Delta(\alpha_2)>0$.  
Since $(i,j')\rightarrow (i,j)$ is the best response of $B$ when $A$ selects $i$, 
it must be that $b_{i,j} \geq b_{i,j''}$. Thus, if $A$ selects $i$ and $B$ selects $j''$, 
then we must have $\Delta(\alpha_1)\geq 0$. 

Moreover, since $u(A_i)\geq u(A_{i'})$ implies $a_{i,j}\geq a_{i',j}$, which is impossible 
because $(i,j)\rightarrow(i',j)$ is on $L$, we have $u(A_{i'}) > u(A_i)$. 
Then by Lemma~\ref{lem:devConf_LL} and~\ref{lem:devConf_SM} (with $(i,j)$, $(i',j)$, $(i',j'')$, 
and $(i,j'')$ corresponding to $(1,1)$, $(2,1)$, $(2,2)$, and $(1,2)$ in the case for two candidates, 
respectively (i.e., $i=1$, $j=1$, $i'=2$, and $j'=2$ in Lemma~\ref{lem:devConf_LL} and Lemma~\ref{lem:devConf_SM}), we know that $\Delta(\alpha_2)>0$ implies that $\Delta(\alpha_1) < 0$, 
hence a contradiction is derived.

\qed
\end{proof}

\fi


%%%%%%%%%%%%%%%%%%%%%%%%%%%%%%%%%%%%%%%%%%%%%%%%%%%%%%%%%%%%%%%%%%%%%%
\section{Price of Anarchy for Egoistic Election Games}
\label{sec:PoA}
%%%%%%%%%%%%%%%%%%%%%%%%%%%%%%%%%%%%%%%%%%%%%%%%%%%%%%%%%%%%%%%%%%%%%%


Below, Proposition~\ref{pro:cases} relates a PNE to an optimal profile on social welfare. Note that a PNE may be a profile that is suboptimal in the social welfare in the election game.

\begin{proposition}\label{pro:cases}
Let $\mathbf{s} = (s_i)_{i\in [m]}$ be a PNE and $\mathbf{s}^* = (s_i^*)_{i\in [m]}$ be the optimal profile. Then, $\sum_{i\in [m]}u(s_i)\geq \max_{i\in [m]} u(s_i^*)$.
\end{proposition}
\begin{proof}
We assume that $\mathbf{s}\neq \mathbf{s}^*$ since otherwise the proposition trivially holds. By Lemma~\ref{lem:dominated_Not_PNE}, we know that for each $i\in [m]$, strategy $s_i$ is not surpassed by~$s_i^*$ since $\mathbf{s}$ is a PNE. Therefore, it suffices to consider the following cases.
\begin{enumerate}[label=(\alph*)]
    \item If $s_i\leq s_i^*$ for all $i\in [m]$, then for each $j\in [m]$, 
    $\sum_{i\in [m]} u(s_i)\geq \sum_{i\in [m]}u_i(s_i)\geq \sum_{i\in [m]} u_i(s_j^*) = u(s_j^*)$, where the second inequality follows from the egoistic property. Hence, we have $\sum_{i\in [m]}u(s_i)\geq \max_{i\in [m]} u(s_i^*)$.  
    \item If $u(s_i^*)\leq u(s_i)$ for all $i\in [m]$, then obviously we have 
	$\sum_{i\in [m]}u(s_i)\geq \sum_{i\in [m]}u(s_i^*)\geq \max_{i\in [m]}u(s_i^*)$. 
    \item Suppose that there exists a subset $W\subset [m]$ such that $s_i\leq s_i^*$ for all $i\in W$ and $u(s_j^*)\leq u(s_j)$ for each $j\in \overline{W}:=[m]\setminus W$. For $j\in W$, we have $\sum_{i\in [m]} u(s_i)\geq \sum_{i\in [m]} u_i(s_i)\geq \sum_{i\in [m]}u_i(s_j^*) = u(s_j^*)$. For $j\in\overline{W}$, we have $\sum_{i\in [m]} u(s_i) = \sum_{i\in [m]\setminus\{j\}} u(s_i) + u(s_j)\geq u(s_j^*)$. Hence, $\sum_{i\in [m]} u(s_i)\geq \max_{i\in [m]}u(s_i^*)$.
\end{enumerate}
\qed
\end{proof}


\vspace{-12pt}
%=====================================================================
\subsection{Adopting the Natural Function}
\label{subsec:PoA_naive}
%=====================================================================

In Sect.~\ref{sec:hardness_no_PNE_examples}, we have seen counterexamples that a PNE does not always exist in the egoistic election game given the natural function as the WP function. Nevertheless, we can still ask how good or bad its PoA can be once a PNE of the game instance exists. In the following, we show that its PoA is bounded by~$m$ when the natural function is adopted as the WP function.  

\begin{theorem}\label{thm:PoA_BT}
Suppose that the natural function is adopted as the monotone WP function. Then the egoistic election game has the PoA bounded by the number of parties~$m$. 
\end{theorem}
\begin{proof}
Let $\mathbf{s} = (s_i)_{i\in [m]}$ be a PNE and $\mathbf{s}^* = (s_i^*)_{i\in [m]}$ be the optimal profile.
Note that $SW(\mathbf{s}^*) = \sum_{i\in [m]}p_{i^*,\mathbf{s}^*} u(s_i^*)\leq \max_{i\in [m]} u(s_i^*)$. 
By the Cauchy-Schwarz inequality, we derive that  
$\sum_{i\in [m]} u(s_i)^2 \geq (\sum_{i\in [m]} u(s_i))^2/m$. 
Then, we have
\begin{eqnarray*}
SW(\mathbf{s})&=& \sum_{i\in [m]}\frac{u(s_i)}{\sum_{j\in [m]} u(s_j)}\cdot u(s_i) = \frac{\sum_{i\in [m]} u(s_i)^2}{\sum_{j\in [m]} u(s_j)} 
\geq \frac{1}{m}\cdot \sum_{i\in [m]}u(s_i).
\end{eqnarray*}

Together, by Proposition~\ref{pro:cases} that $\sum_{i\in [m]}u(s_i)\geq \max_{i\in [m]} u(s_i^*)$, we finally have that $SW(\mathbf{s})\geq SW(\mathbf{s}^*)/m$, Thus, the PoA is bounded by~$m$. 
\qed
\end{proof}


%=====================================================================
\subsection{Adopting the Monotone Function in General}
\label{subsec:PoA_SM}
%=====================================================================

We have seen analysis that the egoistic election game given the natural function as the monotone WP function has PoA bounded by~$m$. In fact, by carefully lower bounding the social welfare of a profile, we can show that the bound holds for the game adopting any monotone WP function. 

 
Note that for any strategy profile $\mathbf{s}$, we have
\vspace{-5pt}
\begin{eqnarray}\label{eq:UB_SW_softmax}
SW(\mathbf{s}) &=& \sum_{i\in [m]} p_{i,\mathbf{s}}\cdot u(s_i)\leq \max_{i\in [m]} u(s_i),
\end{eqnarray}
and 
\begin{eqnarray}\label{eq:LB_SW_softmax}
SW(\mathbf{s}) &=& \sum_{i\in [m]} p_{i,\mathbf{s}}\cdot u(s_i) \geq \frac{1}{m}\cdot \sum_{i\in [m]} u(s_i)
\end{eqnarray} 
which hold for any monotone WP function. Inequality~(\ref{eq:LB_SW_softmax}) can be justified as follows. 
W.l.o.g., let us assume that $u(s_1)\geq u(s_2)\geq \ldots \geq u(s_m)$ (by relabeling after they are sorted). As the winning probability function is monotone, it is clear that $p_{1,\mathbf{s}}\geq p_{2,\mathbf{s}}\geq\ldots \geq p_{m,\mathbf{s}}$. Let $k\in [m]$ be the index such that $p_{k,\mathbf{s}}\geq 1/m$ and $p_{k+1,\mathbf{s}}< 1/m$ and $k = m$ if $p_{i,\mathbf{s}} = 1/m$ for all $i\in [m]$. Note that such an index $k$ must exist otherwise $\sum_{i\in [m]} p_{i,\mathbf{s}} < 1$ which contradicts the low of total probability. It is clear that Inequality~(\ref{eq:LB_SW_softmax}) holds when $k = m$. For the case $k<m$, we have 
\begin{eqnarray*}
SW(\mathbf{s}) = \sum_{i\in [m]} p_{i,\mathbf{s}}\cdot u(s_i)
&=& \sum_{i=1}^k \left(p_{i,\mathbf{s}} - \frac{1}{m}\right) u(s_i) + \frac{1}{m}\sum_{i=1}^k u(s_i)\\
&& + \sum_{i=k+1}^m \left(p_{i,\mathbf{s}} - \frac{1}{m}\right) u(s_i) + \frac{1}{m}\sum_{i=k+1}^m u(s_i)\\
&\geq& \sum_{i=1}^k \left(p_{i,\mathbf{s}} - \frac{1}{m}\right) u(s_k) + \frac{1}{m}\sum_{i=1}^k u(s_i)\\
&& + \sum_{i=k+1}^m \left(p_{i,\mathbf{s}} - \frac{1}{m}\right) u(s_k) + \frac{1}{m}\sum_{i=k+1}^m u(s_i)\\
&=&\left(\sum_{i=1}^m p_{i,\mathbf{s}} - 1\right)u(s_k) + \frac{1}{m}\sum_{i=1}^m u(s_i)\\
& = & \frac{1}{m}\sum_{i=1}^m u(s_i).
\end{eqnarray*} 
\vspace{-0pt}
Hence the inequality~(\ref{eq:LB_SW_softmax}) is valid. 

Now, we are ready for Theorem~\ref{thm:PoA_SF} and its proof. 

\begin{theorem}\label{thm:PoA_SF}
The PoA of the egoistic election game using any monotone WP function is upper bounded by~$m$. 
\end{theorem}
\begin{proof}
Let $\mathbf{s} = (s_i)_{i\in [m]}$ be a PNE and $\mathbf{s}^* = (s_i^*)_{i\in [m]}$ be the optimal profile. 
Let $\ell = \argmax_{i\in [m]} u(s_i^*)$ be the index of the party players with the maximum social utility with respect to~$\mathbf{s}^*$. Recall that $m\cdot SW(\mathbf{s})\geq \sum_{i\in [m]}u(s_i)$. 
Our goal is to prove Inequality~(\ref{eq:thm4c}):  
\begin{equation}\label{eq:thm4c}
\sum_{i\in [m]}u(s_i)\geq u(s_{\ell}^*).
\end{equation}
By Lemma~\ref{lem:dominated_Not_PNE} we know that each $i\in [m]$, strategy $s_i$ is not surpassed by $s_i^*$ since $\mathbf{s}$ is a PNE. Therefore, it suffices to consider the following cases.  
\begin{enumerate}[label=(\alph*)]
    \item For all $i\in [m]$, $s_i\leq s_i^*$. In this case, Inequality~(\ref{eq:thm4c}) holds since
   \begin{enumerate}[label=(\roman*)]
    \item $u_{\ell}(s_{\ell})\geq u_{\ell}(s_{\ell}^*)$,
    \item $u_{j}(s_{j})\geq u_{j}(s_{\ell}^*)$ for each $j\in[m]$. 
    \end{enumerate}
    \vspace{5pt}
    \item For all $i\in [m]$, $u(s_i^*)\leq u(s_i)$. In this case, we have $\sum_{i\in [m]} u(s_i) - u(s_{\ell}^*)\geq \sum_{i\in[m]\setminus\{\ell\}} u(s_i)\geq 0$. Hence, Inequality~(\ref{eq:thm4c}) holds.
    
    \item Suppose that there exists a subset $W\subset [m]$ such that $s_i\leq s_i^*$ for all $i\in W$ and $u(s_j^*)\leq u(s_j)$ for each $j\in \overline{W} := [m]\setminus W$. 
    
    \begin{enumerate}[label=(\roman*)]
    \item Assume that $\ell\in W$. 
    By the arguments similar to (a), Inequality~(\ref{eq:thm4c}) follows from~$u_{\ell}(s_{\ell})\geq u_{\ell}(s_{\ell}^*)$ (by the assumption that $\ell\in W$ and the egoistic property), and $u_{j}(s_{j})\geq u_{j}(s_{\ell}^*)$ for each $j\in[m]\setminus\{\ell\}$ (by the egoistic property).
     \vspace{2pt}
    \item Assume that $\ell\in \overline{W}$. By the arguments similar to (b), we have 
        \vspace{-0pt}
        \begin{equation*}
    	\sum_{i\in [m]} u(s_i) - u(s_{\ell}^*)\geq \sum_{i\in \overline{W}} u(s_i) - u(s_{\ell}^*)\geq \sum_{i\in\overline{W}\setminus\{\ell\}}  u(s_i)\geq 0.
    	\end{equation*}
        \iffalse
        \item Assume that $\ell\in W$ and $q\in \overline{W}$. Since $u(s_q)\geq u(s_{q}^*)$, to derive Inequality~(\ref{eq:thm4c}), we need to show that
        \begin{equation}\label{eq:thm4c2}
        \frac{e}{e+m-1}\cdot u(s_q)+\sum_{i\in [m]\setminus\{q\}} u(s_i)\geq \left(\frac{e}{e+m-1}\cdot u(s_{\ell}^*)\right).
        \end{equation}
        $\ell\in W$ implies that $u_{\ell}(s_{\ell})\geq u_{\ell}(s_{\ell}^*)$. By the egoistic property, we have $u_{j}(s_{j})\geq u_{j}(s_{\ell}^*)$ for each $j\in[m]\setminus\{\ell\}$. Therefore, Inequality~(\ref{eq:thm4c2}) holds.
        \fi
    \iffalse
    \item As case (iii), the case that $\ell\in\overline{W}$ and $q\in W$ can be proved similarly. 
    \fi
        \end{enumerate}
\end{enumerate}
Together with Inequality~(\ref{eq:LB_SW_softmax}) and (\ref{eq:UB_SW_softmax}), we derive that $SW(\mathbf{s})\geq SW(\mathbf{s}^*)/m$. 
Therefore, we conclude that the PoA is bounded by~$m$. 
\qed
\end{proof}

\iffalse
%Denote by $\ell' := \argmax_{i\in [m]\setminus W} u_i(\mathbf{s}^*)$ the index of the party player in $\overline{W}$ with the maximum social utility with respect to~$\mathbf{s}$. 
\fi

\paragraph{Remark.} In~\cite{LLC2021}, the lower bound on the PoA of the egoistic two-party election game using the softmax WP function is~2. 
In addition, the game instance in Tab.~\ref{tab:AlwaysPNE_HM_2party} shows that the PoA of the game using the hardmax WP function is also~2. Through these examples we know that our PoA bound is tight for $m=2$ using either the hardmax or the softmax function as the WP function. 


\begin{table}[ht]
\begin{center}
\begin{tabular}[c]{ l l  | l l  }
	%\centering
	$u_1(x_{1,i})$ & $u_2(x_{1,i})$  & 
        $u_1(x_{2,i})$ & $u_2(x_{2,i})$  
        \\
	\hline
	50             &  $3\epsilon$    &  0  &  $50+2\epsilon$    \\
        $50-\epsilon$  &  $50+\epsilon$ &  0  &  $50+2\epsilon$    \\
	\hline
\end{tabular}
\vspace{7pt}\\
\begin{tabular}[c]{ l l | l l}
	\centering
	%\multicolumn{2}{l}{payoff matrix} \vspace{7pt}\\
	%\hline
	$r_{1,(1,1)}$ \; & $r_{2,(1,1)}$ \; 
     &  \, $r_{1,(1,2)}$ \; & $r_{2,(1,2)}$ \;\\
	\hline
	$r_{1,(2,1)}$ \; & $r_{2,(2,1)}$ \;  
     &  \, $r_{1,(2,2)}$ \; & $r_{2,(2,2)}$ \; \\
	%\hline
\end{tabular}
$\approx$
\begin{tabular}[c]{ l l | l l }
	%\multicolumn{2}{c}{} \vspace{7pt}\\
	\centering
	% & \\
	%\hline
	50 \, &  $3\epsilon$  \,  & \, 50  \, &  $3\epsilon$ \, \\
	\hline
	$50-\epsilon$ \, &  $50+\epsilon$  \,  & \, $50-\epsilon$  \, &  $50+\epsilon$ \, \\
	%\hline
\end{tabular}
\vspace{6pt}
\caption{An egoistic election game instance of two parties that always has a PNE ($\beta=100, n_i=2$ for $i\in\{1,2\}$, $0<\epsilon\ll 1$). The hardmax function is used as the monotone WP function. Profile $(1,1)$ is a PNE, and it has the social welfare~$50+3\epsilon$, which approaches half of that of the optimal profile when $\epsilon$ is closed to~0.}
\label{tab:AlwaysPNE_HM_2party}
\vspace{-20pt}
\end{center}
\end{table}


%\iffalse
%=====================================================================
\subsection{Coalition with Strongly Egoism Guarantee}
\label{subsec:PoA_coalition}
%=====================================================================


In the rest of this section, we consider a scenario that party players can unite as coalitions and utility can share between the members in a coalition, and furthermore, we assume that the election game is strongly egoistic. We call such a game \emph{the strongly egoistic cooperative election game} (SE-CE game). We will show that the SE-CE game collapses on the original non-cooperative egoistic election game. 


For the SE-CE game, suppose that we have $m'$ coalitions $\mathcal{C}_1,\mathcal{C}_2,\ldots,\mathcal{C}_{m'}$ of the $m$ party players, such that $m'\leq m$, and each coalition $\mathcal{C}_i$ is composed of~$n'_i$ party players. That is, for each $i\in [m']$, $\mathcal{C}_i = \{\mathcal{P}_{i_1}, \mathcal{P}_{i_2}, \ldots,\mathcal{P}_{i_{h_i}}\}$ is composed of $h_i$ \emph{member parties}. Let $X_{i_j}$ denote the set of candidates of party $\mathcal{P}_{i_j}$ (i.e., $X_{i_j} = \{x_{i_j,1}, x_{i_j, 2},\ldots, x_{i_j, n_{i_j}}\}$). 
For a coalition $\mathcal{C}_i$ of parties, we regard $\mathcal{C}_i$ as a player and consider the \emph{coalition candidate} $z_i$ as a pure strategy of~$\mathcal{C}_i$, where $z_i\in X_{i_1}\times X_{i_2}\times\cdots \times X_{i_{h_i}}$ is a composition of $h_i$ candidates each of which comes from the corresponding member parties of~$\mathcal{C}_i$. Clearly, there are $\prod_{j\in [h_i]}n_{i_j} = O(n^{h_i})$ pure strategies of coalition~$\mathcal{C}_i$.


%To ease the notation, denote by $z_{i,1}, z_{i,2}, \ldots, z_{i,K_i}$ the coalition candidates in~$\mathcal{C}_i$. 
For each $j\in [m']$, denote by~$\tilde{u}_j(z_i)$ the utility of a coalition candidate $z_i$ of $\mathcal{C}_i = \{\mathcal{P}_{i_1}, \mathcal{P}_{i_2}, \ldots,\mathcal{P}_{i_{h_i}}\}$ for coalition~$\mathcal{C}_j$, which is the sum of utility brought by~$z$ for the supporters of parties in coalition~$\mathcal{C}_j$. Namely, for any coalition candidate~$z_i$ of~$\mathcal{C}_i$, we define $\tilde{u}_j(z_i) = \sum_{\mathcal{P}_{j_{\ell}}\in \mathcal{C}_j}\sum_{t\in [h_i]} u_{j_{\ell}}(z_i(t))$, where $z_i(t)$ corresponds to the candidate in~$X_{i_t}$. The social utility of~$z_i$ is then $\sum_{t\in [h_i]} u(z_i(t))$. 
By the assumption that the SE-CE game is strongly egoistic, we argue that it collapses on the egoistic election game when each coalition is viewed as a party player in the election game. Indeed, consider any coalition $\mathcal{C}_j\neq \mathcal{C}_i$, for any coalition candidate $z_i = (x_{i_1,s_{i_1}},x_{i_2,s_{i_2}},\ldots,x_{i_{h_i},s_{i_{h_i}}})$  of~$\mathcal{C}_i$ and any coalition candidate $z_j = (x_{j_1,s_{j_1}},x_{j_2,s_{j_2}},\ldots,x_{j_{h_j},s_{j_{h_j}}})$ of~$\mathcal{C}_j$, 
we obtain that 
\begin{eqnarray*}
\tilde{u}_i(z_i) &=& \sum_{\mathcal{P}_{i_{\ell}}\in \mathcal{C}_i}\sum_{t\in [h_i]} u_{i_{\ell}}(z_i(t)) \\
&=& \sum_{t\in [h_i]}\sum_{\mathcal{P}_{i_{\ell}}\in \mathcal{C}_i} u_{i_{\ell}}(x_{i_t,s_{i_t}})\\
&=& \sum_{t\in [h_i]} \left(u_{i_t}(x_{i_t,s_{i_t}})+\sum_{\mathcal{P}_{i_{\ell}}\in \mathcal{C}_i,\ell\neq t} u_{i_{\ell}}(x_{i_t,s_{i_t}})\right)\\
&>& \sum_{t\in [h_i]} \left(\sum_{\mathcal{P}_{j_{\ell'}}\in \mathcal{C}_j} u_{i_t}(x_{j_{\ell'},s_{j_{\ell'}}}) +\sum_{\mathcal{P}_{i_{\ell}}\in \mathcal{C}_i,\ell\neq t} u_{i_{\ell}}(x_{i_t,s_{i_t}})\right)\\
&\geq& \tilde{u}_i(z_j),
\end{eqnarray*}
where the inequality follows from the strongly egoistic property. 


\begin{proposition}\label{prop:SE-CEG_is_egoistic}
The SE-CE game with $m'$ coalitions $\mathcal{C}_1,\mathcal{C}_2,\ldots,\mathcal{C}_{m'}$ for $m'<m$ is the egoistic election game in which $\mathcal{C}_i$ is a party player with utility $\tilde{u}_j(z_{i})$ for each coalition candidate $z_{i}$ of~$\mathcal{C}_i$ and each $i,j\in [m']$. 
\end{proposition}
%\fi

An implication of Proposition~\ref{prop:SE-CEG_is_egoistic} is that the strongly egoistic election game becomes more efficient when it is ``more  cooperative" in terms of more coalitions of party players. This results in an egoistic election game with less party players and hence a less PoA bound is expected. 

%%%%%%%%%%%%%%%%%%%%%%%%%%%%%%%%%%%%%%%%%%%%%%%%%%%%%%%%%%%%%%%%%%%%%%
\section{Concluding Remarks and Future Work}
\label{sec:future}
%%%%%%%%%%%%%%%%%%%%%%%%%%%%%%%%%%%%%%%%%%%%%%%%%%%%%%%%%%%%%%%%%%%%%%


From the perspectives of existence of PNE and the PoA, unlike the two-party case, we have learned that the election game is ``bad" when more than two parties are involved in the sense that the PNE is no longer guaranteed to exist and the PoA can be in proportional to the number of competing parties. 
Our work provides an alternative explanation why the two-party system is prevalent in democratic countries. 

We conjecture that to determine if the election game has a PNE is {\sf NP}-complete, even for the egoistic election game. Nevertheless, two parameters of the election game, that is, the number of chaotic parties and the nominating depth, are extracted in this work and utilized to devise an efficient parameterized algorithm for computing a PNE of the egoistic election game. In addition, when we consider mixed-strategy Nash equilibrium of the egoistic election game, the support of each party player can also be identified by slightly modifying Algorithm {\sf FPT-ELECTION-PNE}. We expect these two parameters to be of independent interest.  


We have shown that the PoA of the egoistic election game is upper bounded by the number of parties~$m$, and this also improves our previous bound  in~\cite{LLC2021} using the softmax WP function for the two-party case. Our PoA bound is tight for the egoistic two-party election game using either the hardmax or the softmax function as the WP function. It will be interesting to know if the PoA bound is tight for all monotone WP functions and for general $m\geq 2$. 
 

\iffalse
Coalition between the parties and thus strong equilibria can also be investigated. Naturally, one might argue that not every member party player in a coalition is ``equally happy". We plan to consider individual strategic behaviors of each party player including the choice of parties for the coalition, and moreover, the payoff of each individual party player in a coalition can be considered separately. It is unclear whether the PoA gets higher or lower in such a cooperative game setting.  
\fi  


Coalition between the parties and thus strong equilibria can also be further considered. In Sect.~\ref{subsec:PoA_coalition} under the strongly egoism assumption, we regard a coalition as a unit which receives the whole payoff. Naturally, one might argue that not every member party player in a coalition is ``equally happy". We plan to consider individual strategic behaviors of each party player including the choice of parties for the coalition, and moreover, the payoff of each individual party player will be considered separately. Whether the PoA is getting higher or lower when coalition is allowed deserves further investigation.   


\bibliographystyle{splncs04}
\bibliography{election_game}

\newpage
\appendix

\iffalse
%%%%%%%%%%%%%%%%%%%%%%%%%%%%%%%%%%%%%%%%%%%%%%%%%%%%%%%%%%%%%
\section{Omitted Definitions \& Proofs}
%%%%%%%%%%%%%%%%%%%%%%%%%%%%%%%%%%%%%%%%%%%%%%%%%%%%%%%%%%

\begin{definition}[\cite{DF13,FG06,Nie06}]
\label{defn:parameterized_problems}
A parameterized problem is a language $\mathcal{L}\subseteq \Sigma^* \times \Sigma^*$, where $\Sigma$ is a finite alphabet. The second component is called the parameter(s) of the problem.
\end{definition}


\begin{definition}[\cite{DF13,FG06,Nie06}]
\label{defn:fpt}
A parameterized problem $L$ is fixed-parameter tractable (FPT) if, for all $(x, k)\in \mathcal{L}$, whether $(x,k)\in \mathcal{L}$ can be determined in $f(k)\cdot n^{O(1)}$ time, where
$f$ is a computable function that depends only on~$k$.
\end{definition}

\fi

\iffalse
%=============================================================
\subsection*{Proof of Theorem~\ref{thm:dominatingNE}}
%=============================================================

For the first case, let us consider an arbitrary $i\in [m]$ and an arbitrary $t\in [n_i]\setminus\{1\}$. Denote by $\mathbf{s} = (x_{1,1}, x_{2,1}, \ldots, x_{m,1})$. 
Since strategy $x_{i,1}$ weakly surpasses $x_{i,t}$, we have $p_{i,\mathbf{s}}\geq p_{i,(t,\mathbf{s}_{-i})}$ and then $\sum_{j\in [m]\setminus\{i\}} p_{j,\mathbf{s}}\leq \sum_{j\in [m]\setminus\{i\}} p_{j,(t,\mathbf{s}_{-i})}$. Hence, 
\begin{eqnarray*}
r_{i,\mathbf{s}} - r_{i,(t, \mathbf{s}_{-i})} &=& 
\sum_{j\in [m]} p_{j,\mathbf{s}} u_i(x_{j,1}) - \\
& & \left(\sum_{j\in [m]\setminus\{i\}} p_{j,(t, \mathbf{s}_{-i})} u_i(x_{j,1}) + p_{i,(t, \mathbf{s}_{-i})} u_i(x_{i,t}) \right)\\
&=& \sum_{j\in [m]\setminus\{i\}} u_i(x_{j,t})(p_{j,\mathbf{s}}-p_{j,(t,\mathbf{s}_{-i})}) + (p_{i,\mathbf{s}} u_i(x_{i,1}) - p_{i,(t,\mathbf{s}_{-i})} u_i(x_{i,t}))\\
&\geq& \sum_{j\in [m]\setminus\{i\}} u_i(x_{i,t})(p_{j,\mathbf{s}}-p_{j,(t,\mathbf{s}_{-i})}) +  u_i(x_{i,t}) (p_{i,\mathbf{s}} - p_{i,(t,\mathbf{s}_{-i})})\\
&=& u_i(x_{i,t})\left( \sum_{j\in [m]\setminus\{i\}} p_{j,\mathbf{s}} - p_{j,(t,\mathbf{s}_{-i})} + (p_{i,\mathbf{s}} - p_{i,(t,\mathbf{s}_{-i})})\right)\\
&=& u_i(x_{i,t})\left(\sum_{j\in [m]} p_{j,\mathbf{s}} - \sum_{j\in [m]} p_{j,(t,\mathbf{s}_{-i})} \right) = 0, 
\end{eqnarray*}
where the inequality follows from the egoistic property and the assumption that strategy~$x_{i,1}$ weakly surpasses~$x_{i,t}$. and last equality holds since $\sum_{j\in [m]} p_{j,\mathbf{s}} = \sum_{j\in [m]} p_{j,(t,\mathbf{s}_{-i})} = 1$ (i.e., the law of total probability). Thus, party player $i$ has no incentive to deviate from her current strategy. Note that the uniqueness of the PNE comes when the inequality becomes ``greater-than". This happens as strategy~$x_{i,1}$ surpasses~$x_{i,t}$. 

For the second case, from the same argument as the first case, we know that for $i\in \mathcal{I}$, party player $i$ has no incentive to deviate from her current strategy. Let $j\in [m]\setminus\mathcal{I}$ (i.e., the only one party player not in~$\mathcal{I}$). As $s_j^{\#}$ is the best response when the other strategies of parties in $\mathcal{I}$ are fixed, party player $j$ has no incentive to deviate from her current strategy. Therefore, the theorem is proved. 
\qed

\fi

\iffalse
%=============================================================
\subsection*{Proof of Lemma~\ref{lem:dominated_Not_PNE}}
%=============================================================

Let $\mathbf{s}_{-i} = (\tilde{s}_j)_{j\in [m]\setminus\{i\}}$ be any profile except party player $\mathcal{P}_i$'s strategy. 
If $s_i$ is surpassed by $s'_i$, then we have $s'_i < s_i$, and either:
\begin{itemize}
\item [(a)] $u(x_{i,s'_i}) > u(x_{i,s_i})$ and $u_i(x_{i,s'_i})\geq u_i(x_{i,s_i})$, or
\item [(b)] $u(x_{i,s'_i})\geq u(x_{i,s_i})$ and $u_i(x_{i,s'_i})>u_i(x_{i,s_i})$. 
\end{itemize}
We prove case (a) as follows. Case (b) can be similarly proved. 

From the assumption in (a), we have that $u_i(x_{i,s'_i})\geq u_i(x_{i,s_i})$. Also, by the monotone property of the natural function and the softmax function, we have $p_{i,(s'_i, \mathbf{s}_{-i})} > p_{i,(s_i,\mathbf{s}_{-i})}$ and $\sum_{j\in [m]\setminus\{i\}} p_{j, (s'_i, \mathbf{s}_{-i})}< \sum_{j\in [m]\setminus\{i\}} p_{j, (s_i,\mathbf{s}_{-i})}$ since $u(x_{i,s'_i}) > u(x_{i,s_i})$. Thus,
\begin{eqnarray*}
& & r_{i}(s_i,\mathbf{s}_{-i}) - r_{i}(s'_i,\mathbf{s}_{-i}) = \left(p_{i,(s_i,\mathbf{s}_{-i})}u_i(x_{i,s_i}) + \sum_{j\in [m]\setminus\{i\}} p_{j,(s_i,\mathbf{s}_{-i})}u_i(x_{j,\tilde{s}_j})\right) \\ 
& & - \left(p_{i,(s'_i,\mathbf{s}_{-i})}u_i(x_{i,s'_i})+\sum_{j\in [m]\setminus\{i\}} p_{j,(s'_i,\mathbf{s}_{-i})}u_i(x_{j,\tilde{s}_j})\right)\\
&=& \left(\sum_{j\in [m]\setminus\{i\}} (p_{j,(s_i,\mathbf{s}_{-i})} - p_{j,(s'_i,\mathbf{s}_{-i})}) u_i(x_{j,\tilde{s}_j})\right) + (p_{i,(s_i,\mathbf{s}_{-i})}u_i(x_{i,s_i}) - (p_{i,(s'_i,\mathbf{s}_{-i})}u_i(x_{i,s'_i}))\\
&<& u_i(x_{i,s_i})\left(\sum_{j\in [m]\setminus\{i\}} (p_{j,(s_i,\mathbf{s}_{-i})} - p_{j,(s'_i,\mathbf{s}_{-i})})\right)  + (p_{i,(s_i,\mathbf{s}_{-i})}u_i(x_{i,s_i}) - (p_{i,(s'_i,\mathbf{s}_{-i})}u_i(x_{i,s'_i}))\\
&=& u_i(x_{i,s_i})(p_{i,(s'_i,\mathbf{s}_{-i})} - p_{i,(s_i,\mathbf{s}_{-i})}) + (p_{i,(s_i,\mathbf{s}_{-i})}u_i(x_{i,s_i}) - p_{i,(s'_i,\mathbf{s}_{-i})}u_i(x_{i,s'_i}))\\
&\leq& (p_{i,(s'_i,\mathbf{s}_{-i})} - p_{i,(s_i,\mathbf{s}_{-i})})(u_i(x_{i,s_i})-u_i(x_{i,s_i})) = 0,
\end{eqnarray*}
where the first inequality follows from the egoistic property. 
Thus, $(s_i,\mathbf{s}_{-i})$ is not a PNE.
\qed

\fi

\iffalse
%=============================================================
\subsection*{Proof of Theorem~\ref{thm:PNE_compute_tractable}}
%=============================================================

It costs $O(nm^2)$ time to compute $d$ and $d_i$ for all~$i$, and the set $\mathcal{D}$ can be then obtained. Since playing strategy~1 (i.e., nominating the first candidate) for parties $i\in\mathcal{D}$ is the dominant strategy, we only need to consider party players in~$[m]\setminus\mathcal{D}$. Since for each $i\in [m]\setminus\mathcal{D}$, each strategy in~$\{x_{i,d_i+1},\ldots,x_{i,n_i}\}$ is surpassed by $x_{i,d_i}$ (considering the subgame with respect to~$(x_{i,d_i+1},\ldots,x_{i,n_i})_{i\in [m]}$), by Lemma~\ref{lem:dominated_Not_PNE} we know that we only need to consider strategies $\{x_{i,1},\ldots,x_{i,d_i}\}$ for such party player~$i$. Thus, the number of entries of the implicit payoff matrix
that we need to check whether it is a PNE is $\prod_{i\in [m]\setminus\mathcal{D}} d_i = O(d^k)$, where $k = |[m]\setminus \mathcal{D}|$. 
For each of these entries, it takes $O(m)$ time to compute the winning probabilities as well as the payoffs of the party players in~$[m]\setminus\mathcal{S}$. 
In addition, checking whether each of such entries can be done in at most~$k(d-1)$ steps. Therefore, the theorem is proved.
\qed

\fi

\iffalse
%=============================================================
\subsection*{Proof of Inequality~(\ref{eq:LB_SW_softmax})}
%=============================================================

W.l.o.g., let us assume that $u(s_1)\geq u(s_2)\geq \ldots \geq u(s_m)$ (by relabeling after they are sorted). As the winning probability function is monotone, it is clear that $p_{1,\mathbf{s}}\geq p_{2,\mathbf{s}}\geq\ldots \geq p_{m,\mathbf{s}}$. Let $k\in [m]$ be the index such that $p_{k,\mathbf{s}}\geq 1/m$ and $p_{k+1,\mathbf{s}}< 1/m$ and $k = m$ if $p_{i,\mathbf{s}} = 1/m$ for all $i\in [m]$. Note that such an index $k$ must exist otherwise $\sum_{i\in [m]} p_{i,\mathbf{s}} < 1$ which contradicts the low of total probability. It is clear that Inequality~(\ref{eq:LB_SW_softmax}) holds when $k = m$ Hence, for $k<m$ we have 
\begin{eqnarray*}
SW(\mathbf{s}) = \sum_{i\in [m]} p_{i,\mathbf{s}}\cdot u(s_i)
&=& \sum_{i=1}^k \left(p_{i,\mathbf{s}} - \frac{1}{m}\right) u(s_i) + \frac{1}{m}\sum_{i=1}^k u(s_i)\\
&& + \sum_{i=k+1}^m \left(p_{i,\mathbf{s}} - \frac{1}{m}\right) u(s_i) + \frac{1}{m}\sum_{i=k+1}^m u(s_i)\\
&\geq& \sum_{i=1}^k \left(p_{i,\mathbf{s}} - \frac{1}{m}\right) u(s_k) + \frac{1}{m}\sum_{i=1}^k u(s_i)\\
&& + \sum_{i=k+1}^m \left(p_{i,\mathbf{s}} - \frac{1}{m}\right) u(s_k) + \frac{1}{m}\sum_{i=k+1}^m u(s_i)\\
&=&\left(\sum_{i=1}^m p_{i,\mathbf{s}} - 1\right)u(s_k) + \frac{1}{m}\sum_{i=1}^m u(s_i)\\
& = & \frac{1}{m}\sum_{i=1}^m u(s_i).
\end{eqnarray*} 
\vspace{-0pt}
Hence the inequality~(\ref{eq:LB_SW_softmax}) is justified. 

\fi


\iffalse
%=============================================================
\subsection*{Detailed Proof of Theorem~\ref{thm:PoA_SF}}
%=============================================================

Let $\mathbf{s} = (s_i)_{i\in [m]}$ be a PNE and $\mathbf{s}^* = (s_i^*)_{i\in [m]}$ be the optimal profile. 
Let $\ell = \argmax_{i\in [m]} u(s_i^*)$ and $q = \argmax_{i\in [m]\setminus\{\ell\}} u(s_i^*)$ be the indices of the party players with the maximum and second-maximum social utility with respect to~$\mathbf{s}^*$ respectively. Recall that $m\cdot SW(\mathbf{s})\geq \sum_{i\in [m]}u(s_i)$. 
Our goal is to prove Inequality~(\ref{eq:thm4c}), that is,  
\[
\sum_{i\in [m]}u(s_i)\geq \left(\frac{e}{e+m-1}\cdot u(s_{\ell}^*) + \frac{m-1}{e+m-1}\cdot u(s_q^*)\right).
\]
By Lemma~\ref{lem:dominated_Not_PNE} we know that each $i\in [m]$, strategy $s_i$ is not surpassed by $s_i^*$ since $\mathbf{s}$ is a PNE. Therefore, it suffices to consider the following cases.  
\begin{enumerate}[label=(\alph*)]
    \item For all $i\in [m]$, $s_i\leq s_i^*$,  
    \vspace{-3pt}
     \begin{equation*}
	\sum_{i\in [m]}u(s_i) \geq \left(\frac{e}{e+m-1}\cdot u(s_{\ell}^*) + \frac{m-1}{e+m-1}\cdot u(s_q^*)\right)
	\end{equation*}
    since $u_{\ell}(s_{\ell})\geq (e/(e+m-1))u_{\ell}(s_{\ell}^*) + ((m-1)/(e+m-1))u_{\ell}(s_q^*)$ and $u_{j}(s_{j})\geq (e/(e+m-1))u_{j}(s_{\ell}^*) + ((m-1)/(e+m-1))u_{j}(s_q^*)$ for each $j\in[m]$. 

    \item For all $i\in [m]$, $u(s_i^*)\leq u(s_i)$. Obviously, 
    \vspace{-0pt}
	\begin{eqnarray*}
	&&\sum_{i\in [m]} u(s_i)- \left(\frac{e}{e+m-1}\cdot u(s_{\ell}^*) + \frac{m-1}{e+m-1}\cdot u(s_q^*)\right) \\
	&>& \frac{m-1}{e+m-1}\cdot u(s_{\ell})+ \frac{e}{e+m-1}\cdot u(s_{q}) \geq 0.
	\end{eqnarray*}
 
    \item Suppose that there exists a subset $W\subset [m]$ such that $s_i\leq s_i^*$ for all $i\in W$ and $u(s_j^*)\leq u(s_j)$ for each $j\in \overline{W} := [m]\setminus W$. 
    
    \begin{enumerate}[label=(\roman*)]
    \item Assume that $\ell,q\in W$. 
    By the arguments similar to (a), we have  
    \vspace{-0pt}
	\begin{equation*}
	\sum_{i\in [m]}u(s_i)\geq \left(\frac{e}{e+m-1}\cdot u(s_{\ell}^*) + \frac{m-1}{e+m-1}\cdot u(s_q^*)\right).
	\end{equation*}
     The inequality follows from~$u_{\ell}(s_{\ell})\geq (e/(e+m-1))u_{\ell}(s_{\ell}^*) + ((m-1)/(e+m-1))u_{\ell}(s_q^*)$, $u_{q}(s_{q})\geq (e/(e+m-1))u_{q}(s_{\ell}^*) + ((m-1)/(e+m-1))u_{q}(s_q^*)$ (by the assumption that $\ell, q\in W$ and the egoistic property), and $u_{j}(s_{j})\geq (e/(e+m-1))u_{j}(s_{\ell}^*) + ((m-1)/(e+m-1))u_{j}(s_q^*)$ for each $j\in[m]\setminus\{\ell,q\}$ (by the egoistic property).
     \vspace{2pt}
    \item Assume that $\ell,q\in \overline{W}$. By the arguments similar to (b), we have 
        \vspace{-0pt}
        \begin{equation*}
    	\sum_{i\in [m]} u(s_i)\geq \sum_{i\in \overline{W}}u(s_i)\geq \left(\frac{e}{e+m-1}\cdot u(s_{\ell}^*) + \frac{m-1}{e+m-1}\cdot u(s_q^*)\right).
    	\end{equation*}
        \item Assume that $\ell\in W$ and $q\in \overline{W}$. Since $u(s_q)\geq u(s_{q}^*)$, to derive Inequality~(\ref{eq:thm4c}), we need to show that
        \begin{equation}\label{eq:thm4c2}
        \frac{e}{e+m-1}\cdot u(s_q)+\sum_{i\in [m]\setminus\{q\}} u(s_i)\geq \left(\frac{e}{e+m-1}\cdot u(s_{\ell}^*)\right).
        \end{equation}
        $\ell\in W$ implies that $u_{\ell}(s_{\ell})\geq u_{\ell}(s_{\ell}^*)$. By the egoistic property, we have $u_{j}(s_{j})\geq u_{j}(s_{\ell}^*)$ for each $j\in[m]\setminus\{\ell\}$. Therefore, Inequality~(\ref{eq:thm4c2}) holds.
    \item As case (iii), the case that $\ell\in\overline{W}$ and $q\in W$ can be proved similarly. 
        \end{enumerate}
\end{enumerate}
Together with Inequality~(\ref{eq:LB_SW_softmax}) and (\ref{eq:UB_SW_softmax}), we derive that $SW(\mathbf{s})\geq SW(\mathbf{s}^*)/m$. 
Therefore, we conclude that the PoA is bounded by~$m$. 
\qed

\fi

\iffalse
\newpage
%%%%%%%%%%%%%%%%%%%%%%%%%%%%%%%%%%%%%%%%%%%%%%%%%%%%%%%%%%%%%
\section{Omitted Tables in Sect.~\ref{sec:hardness_no_PNE_examples}}
\label{sec:appendix_B}
%%%%%%%%%%%%%%%%%%%%%%%%%%%%%%%%%%%%%%%%%%%%%%%%%%%%%%%%%%%%%


\begin{table}
\begin{center}
\begin{tabular}[c]{ l l l | l l l | l l l }
	%\centering
	$u_1(x_{1,i})$ & $u_2(x_{1,i})$ & $u_3(x_{1,i})$ & 
        $u_1(x_{2,i})$ & $u_2(x_{2,i})$ & $u_3(x_{2,i})$ & 
        $u_1(x_{3,i})$ & $u_2(x_{3,i})$ & $u_3(x_{3,i})$
        \\
	\hline
	67  &  10   &  9  &  11  &  45  &  9  &   2  &  25  &  53 \\
        66  &  9   &  11  &   1  &  43  & 27  &  41  &   6  &  49 \\
	\hline
\end{tabular}
\vspace{3pt}\\
\begin{tabular}[c]{ l l l | l l l}
	\centering
	%\multicolumn{2}{l}{payoff matrix} \vspace{7pt}\\
	%\hline
	$r_{1,(1,1,1)}$ \; & $r_{2,(1,1,1)}$ \; & $r_{3,(1,1,1)}$ \; 
     &  \, $r_{1,(1,1,2)}$ \; & $r_{2,(1,1,2)}$ \; & $r_{3,(1,1,2)}$\\
	\hline
	$r_{1,(1,2,1)}$ \; & $r_{2,(1,2,1)}$ \; & $r_{3,(1,2,1)}$ \; 
     &  \, $r_{1,(1,2,2)}$ \; & $r_{2,(1,2,2)}$ \; & $r_{3,(1,2,2)}$\\
	%\hline
\end{tabular}
$\approx$
\begin{tabular}[c]{ l l l | l l l }
	%\multicolumn{2}{c}{} \vspace{7pt}\\
	\centering
	% & \\
	%\hline
	28.73 \; &  25.04  \;  &  24.24  \; & \, 42.16  \; &  17.66 \; &   24.55\\
	\hline
	25.29 \; &  24.95  \; &  29.24  \; & \, 38.61  \; &  17.74 \; &  29.23\\
	%\hline
\end{tabular}
\vspace{3pt}\\
\begin{tabular}[c]{ l l l | l l l}
	\centering
	%\multicolumn{2}{l}{payoff matrix} \vspace{7pt}\\
	%\hline
	$r_{1,(2,1,1)}$ \; & $r_{2,(2,1,1)}$ \; & $r_{3,(2,1,1)}$ \; 
     & \, $r_{1,(2,1,2)}$ \; & $r_{2,(2,1,2)}$ \; &  $r_{3,(2,1,2)}$\\
	\hline
	$r_{1,(2,2,1)}$ \; & $r_{2,(2,2,1)}$ \; & $r_{3,(2,2,1)}$ \; 
     & \, $r_{1,(2,2,2)}$ \; & $r_{2,(2,2,2)}$ \; & $r_{3,(2,2,2)}$\\
	%\hline
\end{tabular}
$\approx$
\begin{tabular}[c]{ l l l | l l l }
	%\multicolumn{2}{c}{} \vspace{7pt}\\
	\centering
	% & \\
	%\hline
	28.36 \; &  24.67 \; &  24.98 \;  & \, 41.81  \; &   17.31 \; &   25.24\\
	\hline
	24.92 \; &  24.59 \; &  29.97 \; & \, 38.27 \; &  17.40 \; &  29.91\\
	%\hline
\end{tabular}
\vspace{3pt}
\caption{A strongly egoistic election game instance of three parties that has no PNE. The natural winning probability function is adopted ($\beta=100, n_i=2$ for $i\in\{1,2,3\}$).}
\label{tab:NoPNE_BT_strong}
\end{center}
\end{table}


\begin{table}
\begin{center}
\begin{tabular}[c]{ l l l | l l l | l l l }
	%\centering
	$u_1(x_{1,i})$ & $u_2(x_{1,i})$ & $u_3(x_{1,i})$ & 
        $u_1(x_{2,i})$ & $u_2(x_{2,i})$ & $u_3(x_{2,i})$ & 
        $u_1(x_{3,i})$ & $u_2(x_{3,i})$ & $u_3(x_{3,i})$
        \\
	\hline
	33  &  42  &   0 &   0  &  93  &   4  &  20  &   6  &  54 \\
        30  &  30  &  29 &   3  &  89  &   1  &   0  &  44  &  50 \\
	\hline
\end{tabular}
\vspace{3pt}\\
\begin{tabular}[c]{ l l l | l l l}
	\centering
	%\multicolumn{2}{l}{payoff matrix} \vspace{7pt}\\
	%\hline
	$r_{1,(1,1,1)}$ \; & $r_{2,(1,1,1)}$ \; & $r_{3,(1,1,1)}$ \; 
     &  \, $r_{1,(1,1,2)}$ \; & $r_{2,(1,1,2)}$ \; & $r_{3,(1,1,2)}$\\
	\hline
	$r_{1,(1,2,1)}$ \; & $r_{2,(1,2,1)}$ \; & $r_{3,(1,2,1)}$ \; 
     &  \, $r_{1,(1,2,2)}$ \; & $r_{2,(1,2,2)}$ \; & $r_{3,(1,2,2)}$\\
	%\hline
\end{tabular}
$\approx$
\begin{tabular}[c]{ l l l | l l l }
	%\multicolumn{2}{c}{} \vspace{7pt}\\
	\centering
	% & \\
	%\hline
	16.38 \; &  49.80   \;  &  18.73   \; & \, 9.55  \; &  61.09 \; &   18.94\\
	\hline
	17.74 \; &  47.67   \; &  17.84   \; & \, 10.74  \; &  59.23 \; &  18.10\\
	%\hline
\end{tabular}
\vspace{3pt}\\
\begin{tabular}[c]{ l l l | l l l}
	\centering
	%\multicolumn{2}{l}{payoff matrix} \vspace{7pt}\\
	%\hline
	$r_{1,(2,1,1)}$ \; & $r_{2,(2,1,1)}$ \; & $r_{3,(2,1,1)}$ \; 
     & \, $r_{1,(2,1,2)}$ \; & $r_{2,(2,1,2)}$ \; &  $r_{3,(2,1,2)}$\\
	\hline
	$r_{1,(2,2,1)}$ \; & $r_{2,(2,2,1)}$ \; & $r_{3,(2,2,1)}$ \; 
     & \, $r_{1,(2,2,2)}$ \; & $r_{2,(2,2,2)}$ \; & $r_{3,(2,2,2)}$\\
	%\hline
\end{tabular}
$\approx$
\begin{tabular}[c]{ l l l | l l l }
	%\multicolumn{2}{c}{} \vspace{7pt}\\
	\centering
	% & \\
	%\hline
	16.11   \; &  45.45   \; &  27.59  \;  & \, 9.57  \; &   56.47 \; &   27.40\\
	\hline
	17.40   \; &  43.36  \; &  26.87  \; & \, 10.71 \; &  54.62 \; &  26.71\\
	%\hline
\end{tabular}
\vspace{3pt}
\caption{A strongly egoistic election game instance of three parties that has no PNE. The softmax winning probability function is adopted ($\beta=100, n_i=2$ for $i\in\{1,2,3\}$).}
\label{tab:NoPNE_softmax_strong}
\end{center}
\end{table}

\fi

\end{document}

