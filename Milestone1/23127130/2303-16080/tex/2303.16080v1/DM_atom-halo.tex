%
%%%%%%%%%%%%%%%%%%%%%%%%%%%%%%%%%%%%%%%%%%%%%%%%%%%%%%%%%%%%%%%%%%%%%%%%
%    INSTITUTE OF PHYSICS PUBLISHING                                   %
%                                                                      %
%   `Preparing an article for publication in an Institute of Physics   %
%    Publishing journal using LaTeX'                                   %
%                                                                      %
%    LaTeX source code `ioplau2e.tex' used to generate `author         %
%    guidelines', the documentation explaining and demonstrating use   %
%    of the Institute of Physics Publishing LaTeX preprint files       %
%    `iopart.cls, iopart12.clo and iopart10.clo'.                      %
%                                                                      %
%    `ioplau2e.tex' itself uses LaTeX with `iopart.cls'                %
%                                                                      %
%%%%%%%%%%%%%%%%%%%%%%%%%%%%%%%%%%
%
%
% First we have a character check
%
% ! exclamation mark    " double quote  
% # hash                ` opening quote (grave)
% & ampersand           ' closing quote (acute)
% $ dollar              % percent       
% ( open parenthesis    ) close paren.  
% - hyphen              = equals sign
% | vertical bar        ~ tilde         
% @ at sign             _ underscore
% { open curly brace    } close curly   
% [ open square         ] close square bracket
% + plus sign           ; semi-colon    
% * asterisk            : colon
% < open angle bracket  > close angle   
% , comma               . full stop
% ? question mark       / forward slash 
% \ backslash           ^ circumflex
%
% ABCDEFGHIJKLMNOPQRSTUVWXYZ 
% abcdefghijklmnopqrstuvwxyz 
% 1234567890
%
%%%%%%%%%%%%%%%%%%%%%%%%%%%%%%%%%%%%%%%%%%%%%%%%%%%%%%%%%%%%%%%%%%%
%
%
\documentclass[12pt]{iopart}
%\newcommand{\gguide}{{\it Standard and non-standard Lagrangians}}
%Uncomment next line if AMS fonts required
\usepackage{amsmath}
\usepackage{iopams}
%

\begin{document}

\maketitle

\title{Atomic Model of Dark Matter Halo and Its Quantum Structure}


\author{Z. E. Musielak}
\address{Department of Physics, The University of Texas at 
Arlington, Arlington, TX 76019, USA}
%\ead{zmusielak@uta.edu}
%

\begin{abstract}
A quantum theory of dark matter particles in a spherical halo 
is developed by using the new asymmetric equation, which 
appeared in [2021, Int. J. Mod. Phys. A, 36, 2150042].  The 
theory predicts that each dark matter halo has its core and 
envelope, which have very distinct physical properties.  The 
core is free of any quantum structure and its dark matter 
particles are in random motion and frequently collide with 
each other.  However, the envelope has a global quantum 
structure that contains quantized orbits populated by the 
particles.  The predicted quantum structure of the halo 
resembles an atom, hence, it is here named the atomic 
model of dark matter halo.  Applications of the theory 
to a dark matter halo with a given density profile are 
described, and predictions of the theory are discussed. 
\end{abstract}

%Uncomment for PACS numbers title message
%\pacs{45.20.-d, 45.30.+s, 05.45.-a}


\section{Introduction} 

Many theories of dark matter (DM) have been proposed [1-20].
One popular theory predicts the existence of weakly interacting 
massive particles (WIMPs) of dark matter (DM). However, so far, 
all attempts to discover WIMPs experimentally have failed (e.g., 
[21-24]), thus, the nature and origin of DM still remains unknown.  
According to the Planck 2018 mission [25], DM constitutes 26.8\% 
of the total mass-energy density of the Universe, which is almost 
5.5 times more than the amount of ordinary matter (OM). 

As recent results shown [19], the existence of the Schr\"odinger 
equation (SE) of nonrelativistic quantum mechanics [26] is 
consistent with the irreducible representations (irreps) of the 
extended Galilean group [27-31].  The irreps also allow for 
the existence of a new asymmetric equation (NAE) that is 
complementary to the SE.  Both the SE [32,33] and NAE 
[20] have been used to develop models of DM halos and 
their particles.  The models based on the SE failed as they
required DM particles of different masses for different halos
[34,35].  The model based on the NAE was too simple (no 
halo's effects on the particles and assumed constant density) 
[20] to describe properly the behavior of DM particles in the 
halo. 

The idea of a DM model using the NAE is justified by the 
fact that the SE describes well OM but fails to represent DM, 
and that the NAE may not be applicable to OM, at least at 
its microscopic level [36], but it may describe DM because 
both the SE and NAE originate from the irreps of the 
extended Galilean group, and they are complementary.  
Thus, in this paper a new quantum theory of a DM halo 
is developed based exclusively on the NAE, and it is 
demonstrated that predictions of this theory have far 
reaching physical implications.

The theory predicts a quantum structure of the halo that 
resembles an atom, hence, it is called the {\it atomic 
model of dark matter halo}.  According to this theory, 
each halo has a core and an envelope that surrounds it.  
The core is free of the quantum structure that is only 
present in the envelope.  The behavior of dark matter 
particles in the core and envelope is very different.  In 
the core particles move randomly and collide, while in 
the envelope they are confined to their orbits, which 
are quantized.  The developed theory is applied to a 
DM halo with a given density profile; predictions of 
the theory are discussed.

This paper is organized as follows: nonrelativistic asymmetric 
equations are derived in Section 2; previous theories of DM 
and their validity are discussed in Section 3; a quantum theory 
of DM particles confined to a spherical DM halo is formulated in 
Section 4;  physical implications of the theory and its predictions 
are discussed in Section 5; and conclusions are given in Section 6.   
%
 
\section{Nonrelativistic asymmetric equations}

\subsection{Group theory derivation of asymmetric equations}
 
The structure of the extended Galilean group is $\mathcal {G}_e 
= [O(3) \otimes_s B(3)] \otimes_s [T(3+1) \otimes U(1)]$, 
where $O(3)$ and $B(3)$ are subgroups of rotations and boosts, 
$T(3+1)$ is an invariant subgroup of combined translations in 
space and time, and $U(1)$ is a one-parameter unitary subgroup;
the irreps of $\mathcal {G}_e$ are well-known [27-31].  

A scalar function $\phi (t, \mathbf {x})$ transforms as one of 
the irreps of $\mathcal {G}_e$ if, and only if, the following 
eigenvalue equations $i \partial_t \phi = \omega \phi$ and $- i 
\nabla \phi = \mathbf {k} \phi$ are satisfied, where $\partial_t =
\partial /\partial t$, and $\omega$ and $\mathbf {k}$ are labels 
of the irreps. The eigenvalue equations can be used to derive 
the following two second-order asymmetric equations [19,20]
%
\begin{equation}
\left [ i {{\partial} \over {\partial t}} + \left ( \frac{\omega}{k^2} 
\right ) \nabla^{2} \right ] \phi (t, \mathbf {x}) = 0\ ,
\label{eq1}
\end{equation}  
%  
and 
%
\begin{equation}
\left [ {{\partial^2} \over {\partial t^2}} -  i \left ( \frac{\omega}{k} 
\right )^2 {\mathbf k} \cdot \nabla \right ] \phi (t, \mathbf {x}) = 0\ .
\label{eq2}
\end{equation}  
%    
where the ratios of the labels of the irreps in both equations are 
constant as required by the eigenvalue equations.  Comparison 
of these equations shows that their mathematical structure is 
complementary, but their wavefunctions must be different.  

The ratios of eigenvalues in the above equations can be expressed 
in terms of the universal constants, so the equations can be used 
to formulate quantum theories of OM and DM [19,20].  However, 
if the ratios are expressed in terms of the wave characteristic speed 
and frequency, then the equations become the wave equations that 
describe propagation of classical waves in uniform or nonuniform 
media [36]; classical waves are not discussed in this paper.
%

\subsection{The Schr\"odinger equation}

By using the de Broglie relationship [26], the ratio of eigenvalues 
in Eq. (\ref{eq1}) can be evaluated as $\omega / k^2 = \hbar / 
2m$, which turns the equation into the Schr\"odinger equation 
(SE). Denoting the wavefunction by $\phi_S$, the SE can be 
written in its standard form [26] as
%
\begin{equation}
\left [ i {{\partial} \over {\partial t}} +  \frac{\hbar}{2 m}
\nabla^{2} \right ] \phi_S (t, \mathbf {x}) = 0\ ,
\label{eq3}
\end{equation}  
%  
where $m$ is the mass of a particle.

The eigenvalue equations can also be used to define operators
of energy, $\hat E$, and momentum, $\mathbf {\hat P}$. This
is achieved by multiplying the eigenvalue equations by $\hbar$, 
and defining $\hat E = i \hbar \partial_t$ and $\mathbf {\hat P} 
= - i \hbar \nabla$.  Then, the eigenvalue equations for these 
operators are: $\hat E \phi_S = E \phi_S$ and $\mathbf {\hat P} 
\phi_S = {\mathbf p} \phi_S$, with the eigenvalues $E = \hbar 
\omega$ and $\mathbf {p} = \hbar \mathbf k$.  Using these 
results, the SE can be written in its operator's form as
%
\begin{equation}
\hat E \phi_S = \frac{1}{2m} \left ( \mathbf {\hat P} \cdot 
\mathbf {\hat P} \right ) \phi_S\ , 
\label{eq4}
\end{equation}  
%    
which gives the following relationship between the eigenvalues 
%
\begin{equation}
E = \frac{\mathbf {p} \cdot \mathbf {p}}{2 m} = 
\frac{p^2}{2 m} =  E_{k} \equiv E_{SE}\ .  
\label{eq5}
\end{equation}  
%      
This shows that the SE is based on the nonrelativistic kinetic 
energy, $E_{k} = E_{SE}$, which is a well-known result 
(e.g., [26]). 
%

\subsection{The new asymmetric equation}

The de Broglie relationship [26] may also be used to determine the 
ratio of eigenfunctions $\omega^2 / k^2 = \varepsilon_o / 2m$, 
where $\varepsilon_o = \hbar \omega_o$ is a fixed quanta of 
energy, and $\omega_o$ is a fixed frequency [19,20].  Then, 
Eq. (\ref{eq2}) becomes
%
\begin{equation}
\left [ {{\partial^2} \over {\partial t^2}} -  i \frac{\varepsilon_o}
{2 m}{\mathbf k} \cdot \nabla \right ] \phi_A (t, \mathbf {x}) = 0\ ,
\label{eq6}
\end{equation}  
%    
where $\phi_A (t, \mathbf {x})$ is the wavefunction of this new
asymmetric equation (NAE).  

To determine the expression for energy that underlies the NAE,
the eigenvalue equations for the operators of energy, $\hat E$, 
and momentum, $\mathbf {\hat P}$ must be determined. Since
the wavefunctions of the NAE and SE are different, the eigenvalue
equations for $\phi_A$ will be different than those found for 
$\phi_S$ in Section 2.2.  The main reason is that the operators 
$\hat E$ and $\mathbf {\hat P}$ acting on different eigenfunctions 
give different eigenvalues. 

Thus, for the energy, $\hat E = i \hbar \partial_t$, and momentum,
$\mathbf {\hat P} = - i \hbar \nabla$, operators, the eigenvalue 
equations are: $\hat E \phi_A = \sqrt{\varepsilon_o E}\ \phi_A$ 
and $(\hbar \mathbf {k} \cdot \mathbf {\hat P}) \phi_A =(\mathbf 
{p} \cdot \mathbf {\hat P}) \phi_A = p^2 \phi_A$, with the 
eigenvalues being $\varepsilon_o = \hbar \omega_o$, $E = 
\hbar \omega$ and $\mathbf {p} = \hbar \mathbf {k}$.  Then, 
Eq. (\ref{eq6}) can be written in the following form
%
\begin{equation}
\left ( \frac{1}{\varepsilon_o} \right ) \hat E^2 \phi_A = 
\frac{1}{2m} \left ( \mathbf {p} \cdot \mathbf {\hat P} 
\right ) \phi_A\ , 
\label{eq7}
\end{equation}  
%    
and the relationship between the eigenvalues becomes
%
\begin{equation}
E = \frac{p^2}{2 m} = E_{k} \equiv E_{NAE}\ , 
\label{eq8}
\end{equation}  
%    
which shows that $E_{NAE} = E_{SE} = E_{k}$ and, 
as expected, both the NAE and SE are based on the 
nonrelativistic kinetic energy.
%

The presented results demonstrate that the main difference 
between the SE and NAE is that the former allows for the 
quanta of energy $\hbar \omega$ to be of any frequency; 
however, the latter is valid only when the quanta of energy 
$\varepsilon_o$ is fixed at one frequency $\omega_o$.  
Moreover, the evolution of the wavefunction $\phi_A (t, 
\mathbf {x})$ described by the NAE depends on direction
as shown by the presence of the term ${\mathbf k} \cdot 
\nabla \phi$, whose values are different in different directions 
with respect to ${\mathbf k}$; the SE does not show such 
a directional dependence. 
%

\section{Previous theories of dark matter and their validity}

The fact that the SE describes the quantum structure of OM
has been known for almost 100 years (e.g., [26]). There 
were also attempts to use the SE to formulate quantum 
theories of DM. The basic idea was proposed by Sin [32], 
who postulated the existence of extremely light bosonic 
DM particles with masses of the order of $10^{-24}$ eV, 
which allows for solving the SE on the galactic scale 
because of the very long Compton wavelength of such 
particles. The gravitational potential added to the SE was 
calculated by solving the Poisson equation with the DM 
density as the forcing term.  The work was followed by 
Hu et al. [33], whose DM particles had masses of the 
order of $10^{-22}$ eV.  However, more detailed 
studies [34,35] revealed that these theories require 
different masses of DM particles in different galactic 
halos, which is difficult to justify from a physical point 
of view. 
    
Since the SE equation describes the quantum structure 
of OM, and since the developed quantum theories of 
DM based on the SE failed, it was suggested that the 
NAE, being complementary to SE, may represent the 
quantum structure of DM [19,20].  If this is correct, 
then DM particles may only exchange the quanta of 
energy $\varepsilon_o = \hbar \omega_o$, whose 
frequency $\omega_o$ is fixed for DM particles.  
This means that while OM emits or absorbs radiation 
at a broad range of frequencies, DM's emission or 
absorption is restricted to only one specific frequency 
$\omega_o$ that is characteristic for DM particles. 

An attempt to formulate a nonrelativistic quantum 
theory of cold DM based on the NAE was done in 
[20].  In this theory, a pair of DM particles interact
only gravitationally and the particles are represented
by a scalar wavefunction, whose evolution in time 
and space is described by the NAE. The obtained 
solutions for the wavefunction show that the particles
may exchange the quanta of energy $\varepsilon_o$ 
between themselves or with other particles in the 
halo.  Moreover, the velocities of the particles in 
the pair may change due collissions with other 
DM particles.  

The main disadvantage of the theory developed in 
[20] is that all gravitational effects of the halo on 
the pair of DM particles were neglected, and that 
the density in the halo was assumed to be constant, 
which is not supported by the known halo models 
(e.g., [37-40]).  In the following, the theory is 
generalized to account for the effects of the halo
including non-constant density.  
%  

\section{Quntum theory of dark matter halo}

\subsection{Quantum structure of the halo}

Consider a spherical halo of DM particles with radius $R_{h}$ 
and total mass $M_{h}$.  Let a DM particle of mass $m$ 
be located at point $P$, whose distance from the center 
of the halo is $R$ (spherical coordinate).  The force acting 
on the particle is $F_g (R) = G M(R)\ m / R^2$, and this 
force is balanced by the centrifugal force $F_c (R) = m R\ 
\Omega_c^2 (R)$, which gives the circular orbital frequency 
of the particle 
%
\begin{equation}
\Omega_c^2 (R) = \frac{G M( R)}{R^3}\ .
\label{eq9}
\end{equation}  
%   
where $M (R) = 4 \pi \int_0^R \rho (\tilde R) \tilde {R}^2 
d \tilde {R}$
%   
with $\rho (R)$ being the density of DM inside the halo,
which must be specified.  

The governing equation of the nonrelativistic theory of DM 
is obtained by modifying the NAE given by Eq. (\ref{eq6}) 
to include $\Omega_{c}^2 (R)$, and writing it in the form
%
\begin{equation}
\left [ {{\partial^2} \over {\partial t^2}} -  i \frac{\varepsilon_o}
{2m} \vert \mathbf k \cdot \mathbf {\hat R} \vert \frac {\partial}
{\partial R} + \Omega^2_c (R) \right ] \phi_A (t, R) = 0\ .
\label{eq10}
\end{equation}  
%   
After separating the variables, $\phi_A (t, R) = \chi (t) \eta (R)$, 
the equation for $\eta (R)$ is 
%
\begin{equation}
\frac{d \eta}{\eta} = i \frac{2m}{\varepsilon_o \vert \mathbf k 
\cdot \mathbf {\hat R} \vert} \left [ \mu^2 - \Omega^2_c (R) 
\right ] dR\ ,
\label{eq11}
\end{equation}  
%   
where $\mathbf {\hat R} = \mathbf {R} / R$, and $\mu^2$ is 
the separation constant to be determined.  
%

Then, the real part of the solution of Eq. (\ref{eq11}) is   
%
\begin{equation}
\eta (r) = \eta_o \cos \left ( \frac{2 G m^2 [ \mu^2 R - I_c (R)]} 
{\varepsilon_o^2 \vert \mathbf {\hat k} \cdot \mathbf {\hat R} 
\vert } \right )\ ,
\label{eq12}
\end{equation}  
%  
where $I_c (R) = \int_0^R \Omega_c (\tilde R) d \tilde {R}$, and 
$\mathbf {k} = k \mathbf {\hat k}$, with $k = 1 / \lambda_o = 
G m^2 / \varepsilon_o$ = const [20]. 

The maxima of the cos function are when its argument is $\pm 2 n \pi$,
with $n = 0$, 1, 2, 3, ... .  After using $\varepsilon_o = \hbar
\omega_o$, the condition for the maxima can be written as 
%
\begin{equation}
\mu^2 - \frac{1}{R} I_c (R)  = \pm n \pi \omega_o^2 \vert \mathbf 
{\hat k} \cdot \mathbf {\hat R} \vert \frac{\hbar^2}{G m^3 R}\ .
\label{eq13}
\end{equation}  
%  

To determine the separation constant $\mu^2$, the boundary 
condition at $R = R_h$ is used to evaluate the integral $I_c (R)$,
which gives
%
\begin{equation}
I_c (R_h) = \int_0^{R_h} \Omega_c (\tilde R) d \tilde {R}
= C_{\rho} R_h \Omega_h^2\ ,
\label{eq14}
\end{equation}  
%   
where the dimensionless constant $C_{\rho}$ depends on the 
halo's density profile (see Section 4.2); the orbital frequency 
$\Omega_h^2$ at the edge of the halo is  
%
\begin{equation}
\Omega_h^2 = \Omega_c^2 (R_h) = \frac{G M(R_h)}{R_h^3} 
= \frac{G M_h}{R_h^3}\ .
\label{eq15}
\end{equation}  
%   
It must be noted that the orbital velocity $v_h^2$ corresponding 
to $\Omega_h^2$ is two times smaller that the escape velocity 
$v_{esc}^2$ at the edge of the halo. 

Then, the separation constant $\mu^2$ is evaluated from
%
\begin{equation}
\mu^2 = C_{\rho} \Omega_h^2 \pm n \pi \omega_o^2 \vert 
\mathbf {\hat k} \cdot \mathbf {\hat R_h} \vert \frac{\hbar^2}
{G m^3 C_{\rho} R_h}\ .
\label{eq16}
\end{equation}  
%  
Let $\mu^2 = C_{\rho} \Omega_n^2$ and $\kappa_h = 
\hbar^2 / (G m^3 C_{\rho} R_h)$, which is the dimensional 
constant that connects the universal constants $G$ and $\hbar$
and the mass $m$ of DM particles to the halo radius $R_h$ 
and the density gradient constant $C_{\rho}$.  Moreover, 
the term $\vert \mathbf {\hat k} \cdot \mathbf {\hat R_h}
\vert = 1$ because the unit vector $\mathbf {\hat R_h}$ is 
not restricted and thus it can always be aligned with the unit 
vector $\mathbf {\hat k}$. 

Then, the quantized orbital frequencies $\Omega_n^2$ are 
given by the following relationship
%
\begin{equation}
\Omega_n^2 = \Omega_h^2 \pm n \pi \kappa_h \omega_o^2\ ,
\label{eq17}
\end{equation}  
%  
where $+$ and $-$ is for the inside and outside of the halo, 
respectively.  The lowest frequency orbit inside the halo is 
at its edge, where $\Omega_0^2 = \omega_h^2$, and 
all remaining orbital frequencies inside the halo are higher
than $\Omega_n^2$, which corresponds to the '+' sign in
Eq. (\ref{eq17}); however, only the internal orbits in the 
halo will be considered.

Multiplying Eq. (\ref{eq17}) by $\hbar^2$, the spectrum of 
quantum energies corresponding to the orbits $\Omega_n^2$ 
inside the halo is obtained 
%
\begin{equation}
E_n^2 = E_h^2 + n \pi \kappa_h \varepsilon_o^2\ .
\label{eq18}
\end{equation}  
%  
The orbits with the quantized orbital frequencies $\Omega_n^2$ 
and their corresponding quantized energies $E_n^2$ are located 
at $R_n$, which are given by the following condition
%
\begin{equation}
\frac{1}{C_{\rho} R_n} I_c (R_n) + n \pi \kappa_h \omega_o^2 
\left ( \frac{R_h}{R_n} - 1 \right ) = \Omega_h^2\ ,
\label{eq19}
\end{equation}  
%  
since $R_h \geq R_n$ for all $n$, the term $(R_h / R_n - 1) \geq 0$.

The described quantum structure of the halo and its energy 
spectrum resemble an atom with its available energy levels, 
hence, the {\it atomic model of DM halo}.  In the following, 
the developed theory is applied to a halo with $\rho (R)$ 
specified, and the physical picture of the dark matter 
emerging from this atomic model is discussed.
%

\subsection{Application of the theory and its predictions}

Let a halo of radius $R_{h}$ and total mass $M_{h}$ have 
its density profile given by 
%
\begin{equation}
\rho (R) = \rho_o \left ( 1 - \frac{R}{R_h} \right )\ ,
\label{eq20}
\end{equation}  
%   
where $\rho_o = 3 M_h / \pi R_h^3$ is the central density of 
the halo; this density profile is much simpler than the so-called 
Einasto profiles (e.g., [37-40]); nevertheless, the model illustrates 
well all important aspects of the developed quantum theory of 
DM halo. 
%

The mass distribution $M(R)$ resulting from this density profile 
gives 
%
\begin{equation}
\Omega_c^2 (R) = 4 \Omega_h^2 \left [ 1 - \frac{3}{4} 
\left ( \frac{R}{R_h} \right ) \right ]\ .
\label{eq21}
\end{equation}  
%   
In the limits of $R = R_h$ and $R = 0$, $\Omega_c^2 (R_h) 
= \Omega_h^2$ and $\Omega_c^2 (0) = 4 \Omega_h^2$,
to represent the range of orbital frequencies allowed in this 
model.

Using Eq. (\ref{eq21}), the integral given by Eq. (\ref{eq14}) 
can be evaluated, and the result is  $I_c (R_h) = (5/2) R_h  
\Omega_h^2$, which shows that $C_{\rho} = 5 / 2.$  Then, 
the resulting spectrum of quantized orbits and energies is given 
by Eqs (\ref{eq17}) and (\ref{eq18}), respectively, with the 
dimensionless constant being $\kappa_h = \hbar^2 / (5 G 
m^3 R_h / 2)$.

To determine the radius $R_n$ of each orbit with $\Omega_n^2$,
the condition given by Eq. (\ref{eq19}) can be used to obtain the 
following equation
%
\begin{equation}
z_n^2 + (\beta_n -1) z_n - \beta_n = 0\ .
\label{eq22}
\end{equation}  
%   
where $z_n = R_n / R_h$ and $\beta_n = (5/3) n \pi \kappa_h 
(\omega_o / \Omega_h)^2$.  The two solutions are: $z_n^{(1)} 
= 1$ and $z_n^{(2)} = - \beta_n$. Being negative, the second 
solution implies that $R_n$ is measured from the edge of the 
halo. To measure $R_n$ from the center, the solution must be 
written as 
%
\begin{equation}
R_n = (1 - \beta_n) R_h  = \left [ 1 - \frac{5}{3} n \pi \kappa_h 
\left ( \frac{\omega_o}{\Omega_h} \right )^2 \right ] R_h\ ,
\label{eq23}
\end{equation}  
%   
with the requirement that $\beta_n > 1$.  If $\beta_n = 1$, then 
%
\begin{equation}
n = \frac{3}{5 \pi \kappa_h} \left ( \frac{\Omega_h}{\omega_o} 
\right )^2\ ,
\label{eq24}
\end{equation}  
% 
where $n$ is the largest integer that corresponds to the first quantized 
orbit above the core $\rho = \rho_o$ at $R_n = R = 0$, or the $nth$ 
quantized orbit from the edge of the halo at $R_0 = R = R_h$.  To 
determine $n$, prior knowledge of $\kappa_h$ and $\omega_o^2$ 
are required;  both are unknown as neither the value of the mass $m$ 
of DM particles nor their characteristic frequency $\omega_o^2$ are 
currently established.  Nevertheless, some estimates can be made.
%

Assuming that the mass of the halo is $M_h \sim 10^{10} M_{sun}
\approx 10^{40}$ $kg$, and its radius is $R_h \sim 10^{20}$ $m$,
then, $\Omega_h^2 \sim 10^{-30}$ $s^{-1}$, which corresponds 
to the orbital velocity at $R = R_h$ to be about $100$ $km s^{-1}$ 
[41,42].  In addition, the dimensionless constant $\kappa_h$ can also 
be calculated by specifying the mass $m$ of the DM particles. Taking 
$m\approx 10 m_p \approx 10^{-26}$ $kg$, where $m_p$ is the 
proton mass, the resulting $\kappa_h \sim 1$; however, $\kappa_h 
\sim 10^{-3}$ for the Higgs boson mass.  Thus, in this model, 
assuming that $\Omega_h / \omega_o \approx 10$, then, there 
are $20$ quantized orbits for $\kappa_h \approx 1$, and $2 
\cdot 10^4$ quantized orbits for $\kappa_h \approx 10^{-3}$. 
The size of the core and the number of quantized orbits may 
significantly change when more realistic density profiles for 
DM halos (e.g., [37-40]) are considered, and the values of 
$m$ and $\omega_o$ for DM are established.
%

\section{Physical implications and predictions}

The developed quantum theory based on a new asymmetric 
equation is significantly different than the QM based on the 
Schr\"odinger equation.  The main difference is that the 
theory can be applied to a spherical halo of DM particles, 
and it predicts its quantum structure that contains a core, 
which is free of quantum orbits, and a surrounding envelope 
that contains many quantized orbits.  This shows that the 
halo's quantum structure resembles that known in atoms.

There are two physical constants $\varepsilon_o$ and $\kappa_h$ 
in the theory that play essential roles in establishing the quantum 
structure of the DM halo on its global scale.  One of these constants 
is the quanta of energy $\varepsilon_o = \hbar \omega_o$, which 
is fixed for DM; however, the value of its characteristic frequency 
$\omega_o$ is unknown.  DM particles may absorb and emit 
$\varepsilon_o$, and they may also interact gravitationally with 
each other by exchanging virtual quanta of energy; for this reason, 
$\varepsilon_o$ is called a {\it dark graviton} [20].  Since the theory 
requires that $\omega_o$ = const, DM particles can only emit or 
absorb radiation with this one frequency.  

The other constant $\kappa_h = \hbar^2 / (G m^3 C_{\rho} R_h)$ 
is dimensional, and an interesting combination of the universal 
constants $\hbar$ and $G$, the mass $m$ of a DM particle, the 
density profile constant $C_{\rho}$, and the radius of the DM halo 
$R_h$. The value of $\kappa_h$ directly affects the orbit and energy 
quantization rules given by Eqs (\ref{eq17}) and (\ref{eq18}), which
can also be used to determine the location $R_n$ and number $n$ 
of quantized orbits given by Eqs (\ref{eq19}), (\ref{eq23}) and 
(\ref{eq24}).  Since the quantized orbital frequency $\Omega_n^2$ 
(or energy $E_n^2$) increases with decreasing $R$, and since 
$\varepsilon_o$ is fixed for DM particles and $\kappa_h$ is fixed 
for a given DM halo with its density profile, the orbits may become 
dense enough to be considered as a continuum ($n \rightarrow 
\infty$).  This implies that there is a core at the center of the halo
where there are no quantized orbits but instead the DM particles 
are free to move on any orbit.

Thus, in the core of the halo, DM particles move randomly and 
undergo frequent collisions.  There is a narrow region in the 
immediate vicinity of the core in which many orbits may exist
close to each other, and form a transition between the core 
and envelope.  On the other hand, the quantized orbits exist 
only in the layers above the core, where DM particles are 
confined to their orbits, which means that there are no 
collisions between DM particles on different orbits. The 
main reason for this lack of collisions is that the differences 
between the orbits ($\Delta \Omega_n = \Omega_{n+1} - 
\Omega_{n}$) are not exact multiples of $\varepsilon_o$, 
because they also depend on $\kappa_h$. As a result, DM 
particles cannot move from one orbit to another by emitting 
or absorbing $\varepsilon_o$.  Instead, the particles are 
confined to their orbits with no other quantum restrictions,
since they are spinless and have no charge.   

The situation may be different near the core where the 
density of orbits is high and DM particles on such orbits 
may directly interact with the randomly moving and 
colliding particles of the core; some exchange of DM 
particles may take place, and an improved version of 
the theory presented in [20] may account for this 
process.  Moreover, the core may be filled with free 
dark gravitons [20], which may contribute to the 
gravitational wave background [43], and make the 
central part of DM halos be different from other 
regions.

The predicted quantum structure of the orbits of 
DM particles in the envelope of the halo is valid 
only for elementary particles with small (comparable 
to $m_p$) masses.  For classical particles with masses 
of many orders of magnitude higher than $m_p$, 
the parameter $\kappa_h$ becomes very small and 
the quantum effects are negligible.  This seems to 
be consistent with the fact that DM cannot form 
gravitationally bounded objects larger than pairs
of DM particles [20].

Finally, the theory cannot be applied to macroscopic objects 
of OM, such as planets or stars, because masses of these 
objects would make the parameter $\kappa_h$ practically 
equal to zero, which means that no quantized orbits can 
exist for these objects.  In addition, the theory cannot be 
applied to gaseous nebulae, which are known to emit 
radiation in the whole range of the electromagnetic 
spectrum, instead of radiation of one fixed ($\omega_o$) 
frequency as required by this theory.  Moreover, the 
electromagnetic and plasma effects in nebulae would 
supersede all gravitational effects described in this 
paper for DM halos. 
%


\section{Conclusions}

A quantum theory of DM particles in a spherical halo is 
developed based on a new asymmetric equation [19,20], 
which is complementary to the Schr\"odinger equation.
The two physical parameters that determine the theory 
and its predictions are the quanta of energy $\varepsilon_o 
= \hbar \omega_o$, with $\omega_o$ being a fixed 
frequency, and a new dimensional parameter $\kappa_h 
= \hbar^2 / (G m^3 C_{\rho} R_h)$, which combines the 
universal constants $\hbar$ and $G$, the mass $m$ of DM 
particles, the density profile constant $C_{\rho}$, and the 
radius of the DM halo $R_h$. 

The theory predicts that the halo contains a core that 
is surrounded by an envelope.  The core is filled with 
free and randomly moving DM particles that collide 
with each other.  The envelope contains the quanitized 
orbits on the halo's global scale with the lowest frequency 
(energy) orbit located at the edge of the halo.  The orbital 
frequency (energy) inreases towards the core and the 
quantized orbits become more densely populated to 
form a continuum inside the core.  The described 
quantum structure of the halo resembles an atom, 
hence, the name the atomic model of DM halos.

To determine the distribution of quantized orbits in 
the envelope, the density profile in the halo as well 
as its radius and total mass must be specified.  This
means that the distribution of quantized orbits is 
consistent with the global physical parameters of 
the halo, and that the orbits are populated by DM 
particles, so that the density profile is accounted for.
Since DM particles that populate the orbits are spinless
and have no charge, there are other quantum limits 
on the number of particles on each orbit.

DM particles are allowed to emit or absorb the quanta
of energy, $\varepsilon_o$, called dark graviton.  
However, since differences between the quantized 
orbits are not exact multiples of $\varepsilon_o$, as 
they also depend on $\kappa_h$, DM particles are 
permanently confined to their orbits.  An exception 
could be a narrow region in the immediate vicinity of 
the core, where many orbits may exit.  DM particles 
may undergo quantum jumps between these orbits 
and the core by absorbing or emitting $\varepsilon_o$,
which requires that the differences between some 
orbits and the core are exactly equal to the quanta 
of energy $\varepsilon_o$.  The existence of the
dark gravitons, or a sea of these gravitons in the 
core, may cause its gravitational wave background 
to be different than that generated by the envelope 
of the halo.\\
%  

\bigskip\noindent
{\bf Acknowledgment:}  The author thanks Dora Musielak 
for valuable comments on the earlier version of this manuscript.  
This work was partially supported by Alexander von Humboldt 
Foundation.
%

\bigskip\noindent
{\bf References} 
\bigskip\noindent


\begin{thebibliography}{}

\bibitem{1} M.J. Rees, Dark Matter - Introduction, Astro-Physics 361 (2003) 2427 
\bibitem{2} K. Freeman, and G. McNamara, In Search of Dark Matter, Springer, 
                  Praxis, Chichester, 2006
\bibitem{3} L. Papantonopoulos, L. (Editor), The Invisible Universe: Dark Matter and
                    Dark Energy, Lecture Notes in Physics 720, Springer, Berlin – Heidelberg, 2007
\bibitem{4} R.H. Sanders, The Dark Matter Problem: A Historical Perspective, Cambridge 
                    Uni. Press, Cambridge, 2010
\bibitem{5} J.A. Frieman, M.B. Turner, and D. Huterer, Ann. Rev. Astron. Astrophys. 
                  46 (2008) 385
\bibitem{6} G. Bartone, and D. Hooper, Rev. Mod. Phys. 90 (2018) 045002
\bibitem{7} E. Oks, New Astron. Rev. 93 (2021) 101632
\bibitem{8} L. Hui, Ann. Rev. Astron. Astrophys. 59 (2021) 247
\bibitem{9} S.-J. Sin, Phys. Rev. D. 50 (1994) 3650
\bibitem{10} J.M. Overduin, and P.S. Wesson, Phys. Rep. 283 (2004) 337
\bibitem{11} R. Barbier et al., Phys. Rep. 420 (2005) 1
\bibitem{12} K. Sugita, Y. Okamoto, M. Sekine, Int. J. Theor. Phys. 47 (2008) 2875
\bibitem{13} N. Arkani-Hamed, D.P. Finkbeiner, T.R. Slatyer, and N. Weiner, Phys. 
                    Rev. D. 79 (2009) 015014
\bibitem{14} E. Komatsu et al., ApJS 192 (2011) 18
\bibitem{15} T.M. Undagoita, and L. Rauch, arXiv:1509.08767v1 [physics.ins-det] 26 Sep 2015 
\bibitem{16} S. Giagu, Front. Phys. 7 (2019) 75
\bibitem{17} Y.J. Ko, and H. K. Park, arXiv:2105.11109v3 [hep-ph] 4 June 2021
\bibitem{18} T.B. Watson, and Z.E. Musielak, Int. J. Mod. Phys. A, 35 (2020) 2050189 (10pp)
\bibitem{19} Z.E. Musielak, Int. J. Mod. Phys. A, 36 (2021) 2150042 (12pp)
\bibitem{20} Z.E. Musielak, Int. J. Mod. Phys. A, 37 (2022) 2250137 (10pp)
\bibitem{21} M. Ackermann et al., Phys. Rev. Let. 107 (2011) 241302
\bibitem{22} A. Ibarra, D. Tran and C. Weniger, Int. J. Mod. Phys., 28 (2013)
                   1330040 (48pp)
\bibitem{23} T.M. Undagoita and L. Rauch,  arXiv:1509.08767v1 [physics.ins-det]
                   26 Sep 2015	
\bibitem{24} Y. Hochberg, Y.F. Kahn, R.K. Leane, et al., Nature Rev. Phys. 4 (2022), 637
\bibitem{25} N. Aghanim, et al., Astron. Astrophys. 641 (2020) A6 (67 pages)
\bibitem{26} E. Merzbacher, Quantum Mechanics, Wiley \& Sons, Inc., New York, 1998
\bibitem{27} E.P. Wigner, Ann. Math. 40 (1939) 149 
\bibitem{28} V. Bargmann, Ann. Math. 59 (1954) 1
\bibitem{29} J.-M. Levy-Leblond, Comm. Math. Phys. 6 (1967) 286
\bibitem{30} J.-M. Levy-Leblond, J. Math. Phys. 12 (1969) 64
\bibitem{31} Y.S. Kim and M.E. Noz, Theory and Applications of the Poincar\'e Group, 
                    Reidel, Dordrecht, 1986
\bibitem{32} S.-J. Sin, Phys. Rev. D 50 (1994) 365
\bibitem{33} W. Hu, R. Barkana, and A. Gruzinov, Phys. Rev. Lett. 85 (2000) 1158
\bibitem{34} Spivey, S.C., Musielak, Z.E. and Fry, J.L., MNRAS 428 (2013) 712 
\bibitem{35} Spivey, S.C., Musielak, Z.E. and Fry, J.L., MNRAS 448 (2015) 1574 
\bibitem{36} Z.E. Musielak, Adv. Math. Phys., in press (2023)
\bibitem{37} Einasto, J., \& Haud, U. 1989, Astron. Astrophys. 223 (1989) 89 
\bibitem{38} Navarro, J.F., C.S. Frenk, C.S., \& White, S.D., Astrophys. J. 
                   462 (1996) 563 
\bibitem{39} Merritt, D., et al., Astron. J. 132 (2006) 2685
\bibitem{40} Navarro, J.F., et al., MNRAS 402 (2010) 21
\bibitem{41} A.N. Bushev, MNRAS 417 (2011) L83
\bibitem{42} K. Garrett and G. Duda, Adv. Astron. (2011) Article ID 968283 (22 pages)
\bibitem{43} Romano, J.D. and Cornish, N.J., Living Rev. Relativ. 20 (2017) (1): 2
\end{thebibliography}
\end{document}
   
