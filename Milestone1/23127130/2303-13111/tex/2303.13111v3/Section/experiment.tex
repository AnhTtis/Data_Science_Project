\section{Experiments}
\subsection{Datasets and Evaluation Metrics}
We conduct experiments on two publicly available datasets: COVID-19-20~\cite{roth2022rapid} and Synapse~\cite{landman2015miccai}. 
COVID-19-20 is comprised of 249 unenhanced chest CT scans, with 199 samples designated for training and 50 samples for testing. All samples are positive for SARS-CoV-2 RT-PCR.
Synapse consists of 30 cases of CT scans, with 14, 4, and 12 cases designated for training, validation, and testing, respectively~\cite{cao2021swinunet}. 
For COVID-19-20, we use the official evaluation metrics from the challenge~\cite{roth2022rapid}, including Dice coefficient (Dice), Intersection over Union (IoU), Surface Dice coefficient (SD), Normalized Volume Difference (NVD), and Hausdorff Distance (HD). 
For Synapse, we follow \cite{chen2021transunet} to adopt Dice and HD as evaluation metrics.
The computational cost is measured with an input size of $192\times192\times48$ and a batch size of 1, in terms of FLOPS (G), Parameters (M), Peak Memory (G) and Throughput (samples/s). For details, please refer to \cite{hou2022vip}. 

\subsection{Implementation Details}
PHNet is implemented using PyTorch and MONAI~\cite{cardoso2022monai} framework and trained on an NVIDIA RTX 3090 GPU. 
For COVID-19-20, all images are interpolated into the voxel spacing of $ 0.74 \times 0.74 \times 5.00 \textup{mm}^3$. 
Three sub-volumes of $ 224 \times 224 \times 28$ are sampled from each scan. 
We train PHNet for a total of 250 epochs on Synapse and 450 epochs on COVID-19-20. For all experiments we adopt the AdamW optimizer~\cite{loshchilov2017decoupled} with an initial learning rate $lr = 10^{-3} \times \frac{\text{batch\_size}}{1024}$, as suggested by~\cite{hou2022vip}. 
The objective function is the summation of Dice loss and cross-entropy loss.
Except for the above, we follow baseline from~\cite{roth2022rapid} and~\cite{isensee2021nnunet} for the COVID-19-20 and Synapse datasets, respectively.


\begin{table}[thbp]
    \caption{Result comparisons with SOTAs on Synapse~\cite{landman2015miccai}.} 
    \centering
    \resizebox{\textwidth}{!}{
    \begin{tabular}{l|l l|l l l l l l l l}
    \hline
    {Method} & {Dice $\uparrow$} & {HD $\downarrow$} & {Aorta } & {Gallb.} & {Kid(L)} & {Kid(R)} & {Liver} & {Panc.} & {Spleen} & {Stom.} \\
    
    \hline
    UNeXt \cite{valanarasu2022unext} & 67.07 & 40.47 & 76.43 & 51.64 & 74.54 & 67.94 & 91.11 &34.95 & 79.20 & 60.70 \\
    TransUnet \cite{chen2021transunet} & 77.48 & 31.69 & 87.23 & 63.13 & 81.87 & 77.02 & 94.08 & 55.86 & 85.08 & 75.62 \\
    SwinUnet \cite{cao2021swinunet} & 79.13 & 21.55 & 85.47 & 66.53 & 83.28 & 79.61 & 94.29 & 56.58 & 90.66 & 76.60 \\
    UNETR \cite{hatamizadeh2022unetr} & 79.57 & 23.87 & 89.99 & 60.56 & 85.66 & \underline{84.80} & 94.46 & 59.25 & 87.81 & 73.99 \\
    CoTr \cite{xie2021cotr} & 78.08 & 27.38 & 85.87 & 61.38 & 84.83 & 79.36 & 94.28 & 57.65 & 87.74 & 73.55 \\
    nnUNet \cite{isensee2021nnunet} & 83.98 & \textbf{12.56} & \textbf{91.73} & 66.22 & \underline{87.3}0 & 84.41 & 96.15 & 76.04 & 94.81 & 75.20 \\
    \hline
    \rowcolor{light_cyan}
    PHNet* & \underline{84.94} & 18.29 & \underline{91.34} & \underline{68.04} & \textbf{87.68} & \textbf{85.22} & \underline{96.17} & \underline{76.19} & \underline{95.45} & \underline{79.42} \\
    \rowcolor{light_cyan}
    PHNet & \textbf{85.54} & \underline{14.62} & 89.83 & \textbf{74.26} & 87.05 & 84.62 & \textbf{96.29} & \textbf{77.19} & \textbf{95.50} & \textbf{79.55} \\
    \hline        
    \end{tabular}}
    \label{Synapse_sota}
\end{table}

\begin{table*}[t]
    \centering
    \normalsize
    \caption{Result comparisons with the state-of-the-art methods and top-12 solutions on COVID-19-20~\cite{roth2022rapid}.}
    \renewcommand\arraystretch{1.2}
    \setlength{\tabcolsep}{13pt}{
    \begin{tabular}{c | r|ccccc cc}
    % \midrule\midrule
    \toprule
    \multicolumn{2}{c|}{\textbf{Methods}} & \textbf{Dice} $\uparrow$ & \textbf{IoU} $\uparrow$ & \textbf{SD} $\uparrow$ & \textbf{NVD} $\downarrow$ & \textbf{HD} $\downarrow$ & \textbf{Method} & \textbf{Dice} $\uparrow$ \\
    % \midrule
    \cmidrule(lr){1-7} \cmidrule(lr){8-9}
    \multirow{2}{*}{\rotatebox{90}{CNN}}& UNet~\cite{falk2019unet} & 69.09 & 55.27 & 62.83 & 40.79 & 134.76 & Rank 1 & 77.09 \\
    & nnUNet \cite{isensee2021nnunet} & 72.51 & 59.40 & 69.04 & 27.87 & 123.65 & Rank 2 & 76.87  \\ 

    % \midrule
    \cmidrule(lr){1-7} 
    & TransUNet~\cite{chen2021transunet}  & 18.64 & 11.70 & 11.93 & 65.06 & 279.03 & Rank 3 & 76.87  \\
    
    & CoTr~\cite{xie2021cotr}             & 59.63 & 45.65 & 52.97 & 44.57 & 172.78 & Rank 4 & 76.78 \\
    
    & UNETR \cite{hatamizadeh2022unetr} & 57.18 & 43.10 & 51.20 & 44.64 & 174.40 &  Rank 5 & 76.77  \\
    
    & SwinUNETR \cite{tang2022swinunetr} & 63.65 & 50.13 & 58.19 & 37.20 & 141.42 & Rank 6 & 76.64  \\
    
    \multirow{-5}{*}{\rotatebox{90}{{Transformer}}} & SwinUNet~\cite{cao2021swinunet}     & 32.82 & 22.08 & 20.71 & 67.62 & 221.46 & Rank 7 & 76.47  \\
    % \midrule
    \cmidrule(lr){1-7} 
    
    \multirow{4}{*}{\rotatebox{90}{MLP}} &
    Shift~\cite{Lian_2021_ASMLP} & 72.18 & 59.32 & 68.58 & 25.27 & 132.01 & Rank 8 & 76.45 \\ &
    Wave~\cite{tang2022wave} & \underline{74.08} & \underline{60.92} & \underline{70.76} & \underline{22.12} & \underline{120.59} & Rank 9 & 76.28 \\ &   UNeXt~\cite{valanarasu2022unext}    & 21.90 & 14.64 & 10.79 & 81.52 & 328.83 & Rank 10 & 76.27  \\     
    & CycleMLP~\cite{chen2022cyclemlp}    & 27.53 & 17.91 & 15.55 &  79.67 & 228.07  & Rank 11 & 76.27  \\

    % \midrule
    \cmidrule(lr){1-7} 
    \multicolumn{2}{r|}{SAMed~\cite{zhang2023customized}}    & 15.43 & 09.52 & 7.66 & 96.36 & 2265.73  & Rank 12 & 76.16 \\
    % \midrule
    \cmidrule(lr){1-7} \cmidrule(lr){8-9}
    
    \multicolumn{2}{r|}{PHNet (Ours)} & \textbf{76.34} & \textbf{63.36} & \textbf{72.10} & \textbf{20.60} & \textbf{108.16} & Ours & \textbf{77.18}  \\
    % \midrule \midrule
    \bottomrule
    \end{tabular}}
    \label{COVID-19-20_sota}
\end{table*}
\section{Visualization On Demand} %Visualization Elements
\label{sec:visrisk}
Based on environment data and trajectory evaluation, we now present ways of communicating the situation and risks on a visual display to achieve an ADAS.
In this context, we employ a renderer that visualizes all the information in a joint Cartesian coordinate system (see section \ref{subsec:sim}). 
Once driving risks are detected, design elements are overlayed on the display with section \ref{subsec:active} and section \ref{subsec:warning}. 

\subsection{Simulator Environment}
\label{subsec:sim}
Nodes of the R-LDM have a range of potential attributes, such as the 3D position or geometrical shape of objects. 
% For instance, the road centerline is a polyline with bounderies to the left and right. Crosswalks have a defined width and buildings a polygonal outline description. 
In the renderer, we always visualize static and quasi-static data that lie in the field of view from the ego vehicle. 
For this, a local 3D model is generated by converting geographic points with (lat, lon, alt) into Cartesian coordinates of (x, y, z). 
% and project the positonal relations from a view perspective with a transformation matrix. 
Fig. \ref{fig:3Dsimulator} depicts an exemplary map section having several intersections in bird's-eye view.
% with several intersections, stop lines and crosswalks. 
On the top right, the first person view of a vehicle approaching a crosswalk is shown. 

The dynamic data is then added to this static view. A zoomed-in excerpt from the map is given at the bottom of Fig. \ref{fig:3Dsimulator} that includes a recorded GNSS trace (red).
We project the trace onto the connected lane center, which is pictured in green. 
% Because we project the ego position on the closest lane segment, on the bottom right the measured trace is changed in red and the aligned trace is marked in green.
Consequently, the virtual horizon and its possible paths are retrieved as described in section \ref{subsec:ldm}. 
We can lastly update and move the excerpt with the current position from the GNSS to obtain a live simulation.

\subsection{Proactive Support}
\label{subsec:active}
Communication of spatial as well as spatio-temporal relations is crucial for risk-averse driver support. 
% This has the reason that humans can estimate the time better than positions (especially for risks). 
% Velocity contains implicitly the time as well. 
Further sources of information are cause, likelihood and severity of a potential risks.  
% if a collision happens. 
The next step for RNS is the choice of suitable design elements. 
In this process, we suppose that we know where the ego vehicle is driving (i.e., the ego path) from its navigation route. 
Yet, for surrounding vehicles, all paths are considered.

\subsubsection{Hazard Route Element}
The so-called hazard route in Fig. \ref{fig:charts} is a concept that consists of a scale portraying distances to an upcoming risk element.
Furthermore, the geometrical area or length of risks is considered.
Risk is thus measured with respect to the ego path, ranging from the current position  $\Delta l \hspace{-0.03cm}=\hspace{-0.03cm} \unit[0]{m}$ to the end of the path $\Delta l_{h}$.
Here, the length $\Delta l_{h}$ can be chosen according to own preferences. 

At an upcoming intersection, risk is defined by the section of the path that lies within the junction.
Since risk corresponds to exposition time, we encode the path part from the intersection $I_z$ with a color, ranging from green for short intersections to red for long ones. 
%allgemein risiko entlang des pfades zu intersection zone
%share of junction segment to navigation route + 
%one case with large intersection far and one case with small intersection close
Fig. \ref{fig:charts}~a) gives two examples of the hazard route.
The left bar shows a large intersection (e.g. multi-lane four-way stop) in vicinity and the right bar has a small and consecutive medium junction. 
% In the case of collision risk, the intersection zone $I_z$ can be used.
% Depending on the value of $I_z$ (low, medium and large), the area is marked from green, to yellow until red for conveying the criticality. 
This emphasizes that we may include more than one intersection in our warnings.

\begin{figure}[t]
  \centering
  \includegraphics[width=0.95\linewidth]{./img/simulator.png}
  \caption{Rendered road network from two perspectives with the ego position being projected on the navigation route. \vspace{0.45cm}}
  \label{fig:3Dsimulator}
\end{figure}

\begin{figure}[t]
  \centering
  \resizebox{\linewidth}{!}{
  \import{img/}{velocity_scale_new.pdf_tex}}  
  \caption{Chart elements for proactive support. Hazard route (left) and velocity scale (right).} %\vspace{-0.3cm}}
  \label{fig:charts} 
\end{figure} 

\subsubsection{Velocity Scale Element}
The velocity scale, Fig. \ref{fig:charts}~b), is a second chart element which qualifies the difference between the current velocity of the vehicle $v_0$ and the target velocity $v_{\text{tar}}$ from the trajectory evaluation of section \ref{subsec:trajeval}. 
The scale shows possible velocity values, from standstill $v\hspace{-0.05cm}=\hspace{-0.05cm}\unit[0]{m/s}$ to a maximal velocity $v_{\text{max}}$. Depending on the difference $|v_0 \hspace{0.05cm} - \hspace{0.05cm} v_{\text{tar}}|$, the situation is rated as safe with $v_0 \hspace{-0.042cm} \approx \hspace{-0.042cm} v_{\text{tar}}$ (green, left), as dangerous with e.g. $v_0 \hspace{-0.05cm} < \hspace{-0.05cm} v_{\text{tar}}$ (yellow, middle) to critical with $v_0 \hspace{-0.07cm} \ll \hspace{-0.07cm} v_{\text{tar}}$ (red, right). The same cases hold true for the opposite circumstances, i.e., $v_0 \hspace{-0.032cm} > \hspace{-0.032cm} v_{\text{tar}}$. 
This velocity scale can be employed for curve or regulatory risks. 
Moreover, we may set an enforced speed limit as the target velocity $v_{\text{tar}}$ for proactive behavior, once there is no risk ahead. 
%\noindent -Warning vs behavior support \\
%-Ghost vehicle as in game \\

\subsection{Short-Term Warning Elements}
\label{subsec:warning}
In order to emphasize the criticality of the situation, we propose to add further intuitive warning elements as e.g. pop-up signs and lane colorings. 
The following elements augment the proactive elements.

\subsubsection{Pop-up Signs}
Explicit symbols indicate the risk cause accompanied with the event time for collisions ($s_E$), distances to the risk spot for turns (i.e., right curve with $d_r$ and left curve with $d_l$) or stopping distance for crosswalks ($d_c$). In Fig. \ref{fig:popups}~a), the pop-up signs are pictured. 
% Besides the velocity difference, the risk type is an indication for the severity of the situation.
%Examples for collision risk are car-to-car crash., curve risk can be  as a single-car accident and regulatory risks will be a car-to-object collision. 
We want to stress that this is just a selection and more risk causes can be added. 
The purpose is also to clarify the reason for the warning and give more human-understandable information.

\subsubsection{Colored Events}
Finally, we highlight lane parts or positions according to the corresponding risks.  
% the determined color rating from the hazard route and velocity scale and relate the risks to the simulator environment. 
In the instance of curve and regulatory risk, the lane is colored from the ego position up to the point of maximal risk. 
For collision risk, we mark the point of the closest encounter as a red cube.
An illustration for regulatory risk induced from a stop line is depicted in Fig. \ref{fig:popups}~b). Again, the color is defined by the deviation $|v_0-v_{\text{tar}}|$. It also shows the therein considered navigation route with length $\Delta l_h$ and another unlikely path. 

It should be noted that the visualization of warnings only occurs if the risks are actually present. 
%\textcolor{red}{improve language, repeat intersection zone and navigation route}
%eingrauen unlikely paths and navigation path and describe in text, maybe delete Iz -> put line from unlikely path to green arrow
Altogether, the RNS provides a variety of tools to analyze and circumvent critical situations in intersection scenarios, while not overloading the driver's awareness.

\begin{figure}[t]
  \centering
  \resizebox{\linewidth}{!}{
  \import{img/}{colored_lane_new.pdf_tex}}  
  \vspace{-0.53cm}
  \caption{Short-term warning elements. Selected pop-up warnings (left) and colored lane (right).}
  \label{fig:popups} 
\end{figure} 



\subsection{Comparisons with State-of-the-Arts} 
\subsubsection{Comparison methods.} 
We compare the proposed PHNet with CNN-based, Transformer-based, MLP-based methods, and foundation model. 
We list the details below. 
\begin{itemize}
\item\textit{CNN-based methods}, including V-Net~\cite{milletari2016v}, UNet~\cite{ronneberger2015u}, nnUNet~\cite{isensee2021nnunet}, and attention-UNet~\cite{schlemper2019attention}. 
These architectures feature a hierarchical contracting path for context aggregation and a symmetric expanding path for resolution recovery and precise localization.
\item\textit{Transformer-based methods}, including ViT~\cite{dosovitskiy2020vit}, UNETR~\cite{hatamizadeh2022unetr}, SwinUNETR~\cite{tang2022swinunetr}, TransUNet~\cite{chen2021transunet}, SwinUNet~\cite{cao2021swinunet}, CoTr~\cite{xie2021cotr}, and CTO-Net~\cite{lin2023rethinking}. 
These architectures can be classified into three categories: 1) classical Transformer (\ie, ViT), 2) encoder-decoder framework with pure Transformer blocks (\ie, SwinUNet), and 3) hybrid architectures with CNN and Transformer (\ie, UNETR, SwinUNETR, TransUNet, CoTr, CTO-Net).
\item\textit{MLP-based methods}, including UNext~\cite{valanarasu2022unext}, CycleMLP~\cite{chen2022cyclemlp}, MLP-Mixer (Mixer)~\cite{tolstikhin2021mixer}, ShiftMLP (Shift)~\cite{Lian_2021_ASMLP}, and WaveMLP (Wave)~\cite{tang2022wave}.
We only replace the MLPP module in PHNet with these alternatives for a fair comparison.
\item\textit{SAMed}~\cite{zhang2023customized} is built upon SAM~\cite{kirillov2023sam} which is a foundation model for semantic segmentation.
SAMed retains the same architecture as SAM, which features a ViT-based encoder, a prompt module, and a mask decoder. 
SAMed applies the LoRA~\cite{hu2021lora} finetuning strategy on target tasks.
\end{itemize}

\subsubsection{Quantitative comparisons.} 
We begin by evaluating the performance of our method in multi-organ segmentation on Synapse~\cite{landman2015miccai} in Table~\ref{Synapse_sota}. 
Result shows that our method achieves the highest average Dice score of 85.62$\%$ and lowest HD of 11.75, outperforming the SOTA methods. 
Specifically, our method achieves 1.64$\%$ and 8.77$\%$ Dice improvements over nnUNet~\cite{isensee2021nnunet} and UNet~\cite{falk2019unet}, whose backbones are built upon 3D CNNs. 
Compared to top-performing competitors of Transformer-based CTO-Net~\cite{lin2023rethinking}, UNETR~\cite{hatamizadeh2022unetr}, MLP-based UNext~\cite{valanarasu2022unext}, Wave~\cite{tang2022wave}, and SAMed~\cite{zhang2023customized}, our method also achieves better performance with 4.52$\%$, 6.05$\%$, 18.55$\%$, 0.53$\%$, 3.74$\%$ Dice gains, respectively. 
These results conform to our argument that PHNet obtains satisfying segmentation performance through effective local-to-global modeling.

We further evaluate the performance on COVID-19-20~\cite{roth2022rapid} and the official evaluation result is presented in Table~\ref{COVID-19-20_sota}. 
Compared to CNN-based methods, our method attains the highest scores in all metrics. This suggests that CNN-based approaches have limitations in long-distance context fusion.
Compared to Transformer-based methods, PHNet also achieves better performance, suggesting that our proposed hybrid structure design facilitates more adept feature learning. 
Compared to MLP-based methods, PHNet outperforms by a remarkable margin. 
This is partly because existing MLP-based methods are designed to segment on 2D slices which lose through-plane features, resulting in severe performance deterioration in challenging volumetric image segmentation tasks. 
The large performance gap between SAMed~\cite{zhang2023customized} and PHNet indicates the constrained generalization capacity of foundation models within specific segmentation tasks.
Additionally, following~\cite{roth2022rapid}, we perform five-fold cross-validation and model ensemble using our proposed method.
The result demonstrates that our method achieves the highest dice score of 77.18\%, surpassing the performance of the top-12 solutions in this challenge\footnote{\url{https://covid-segmentation.grand-challenge.org/evaluation/challenge/leaderboard}}.

Additionally, our method is subjected to further evaluation on two other public datasets, namely the Liver Tumor Segmentation dataset (LiTS)~\cite{bilic2023liver} and the Medical Segmentation Decathlon (MSD) brain tumor segmentation (BraTS)~\cite{antonelli2022medical} dataset.  
The details are presented in the supplementary material.
Results show that our method outperforms other state-of-the-art methods on both datasets, demonstrating the credibility and wide-ranging applicability of our proposed method.


\subsubsection{Qualitative comparisons.}
Visual comparisons on Synapse~\cite{landman2015miccai} are shown in Figure~\ref{fig:visual}. 
PHNet achieves the best visual segmentation results compared to nnUNet~\cite{isensee2021nnunet}, TransUNet~\cite{chen2021transunet}, UNext~\cite{valanarasu2022unext}, and Mixer~\cite{tolstikhin2021mixer}.
As shown in the first row, the comparison indicates that our method achieves a better segmentation where the under-segmentation of pancreas is observed in nnUNet, TransUNet, and UNext. 
Moreover, as shown in the second row, PHNet can accurately delineate boundaries between liver and stomach, while the over-segmentation of liver is observed in both UNeXt and Mixer. 
Similarly, PHNet delineates a more precise boundary between pancreas and stomach as shown in the last row. 
The primary factor could be PHNet helps extract accurate contours by effective local-to-global modeling and aggregation of inter-axis and intra-axis token features. 
Overall, PHNet attains better segmentation results and mitigates the issues associated with under- and over-segmentation contours.


\subsection{Ablation Study} 
We conduct ablation studies on Synapse to validate the effectiveness and efficiency of each component in our method. 
We use the same decoder architecture for all variations.

\begin{table}[t]
\centering
\caption{Result comparisons with different effects of each component on Synapse~\cite{landman2015miccai}. ``{Baseline}'' denotes the vanilla MLP network. ``{IP-}''denotes the proposed axial decomposition operation in IP-MLP module. ``{TP-}'' denotes the proposed operation of decomposition of in-plane feature and through-plane feature in TP-MLP. ``{AA-}'' denotes our proposed AA-MLP module.}
\renewcommand\arraystretch{1.2}
\setlength{\tabcolsep}{6pt}{
\begin{tabular}{c c c c|c c|c c}
     \toprule
     \textbf{Baseline} &\textbf{IP-} & \textbf{TP-} & \textbf{AA-} & \textbf{Dice} $\uparrow$ & \textbf{HD} $\downarrow$ & \textbf{FLOPs} & \textbf{Thro.} \\
     \midrule
     \cmark &       &       &       & 82.80 & 21.12 & 1249 & 1.04  \\
     \cmark &\cmark &       &       & 84.29 & 19.05 & 938 & 1.73  \\
     \cmark &      &\cmark & & 83.39 & 19.66 & 971 & 1.62
     \\
     \cmark &\cmark &\cmark &       & 84.94 & 18.29 & 947 & 1.75  \\
     \cmark &\cmark &\cmark &\cmark & \textbf{85.62} & \textbf{11.75} & 953 & 1.73 \\
     \bottomrule
     \end{tabular}}
    \label{tab:ablation_Synapse}
    \vspace{-4mm}
 \end{table}
\begin{figure}[t]
\centering
\includegraphics[width=0.48\textwidth]{Figure/Fig/fig_visual_synapse_abl.pdf}
\caption{Segmentation visualizations of ablation variations on Synapse~\cite{landman2015miccai}, including (a) original image, (b) ground truth, predictions of (c) {baseline} + {IP} + {TP} + {AA}, (d)  {baseline} +  {IP} +  {TP}, (e)  {baseline} +  {IP}, and (f)  {baseline}. Regions of evident improvements are enlarged to show better details. Better viewed with zooming in.}
\label{fig:visual_abl_Synapse}
\end{figure}
\subsubsection{Effectiveness of each components.}
To verify the effectiveness of core components in our approach, we increase each essential component gradually based on the vanilla MLP network (abbreviated as ``baseline''),
as shown in Table~\ref{tab:ablation_Synapse}. 
Compared with the baseline, the integration of axial decomposition yields a 1.49\% enhancement in Dice and a noteworthy improvement in efficiency. 
This is due to positional information encoding and complexity reduction.
There is an additional improvement of 0.65\% Dice through decomposition into in-plane and through-plane features, which is attributed to the amplified learning capacity of the module as it contains distinct learning parameters for in-plane and through-plane features~\cite{bertasius2021space}. Furthermore, AA-MLP contributes to 0.68\% performance gains with slightly increasing computational cost, and the whole framework achieves an 85.62\% Dice score. 
This is because intra-axis token communication enables better learning representation from larger receptive field, as shown in Figure~\ref{fig:visual_abl_Synapse}.
A similar trend can be observed in the evaluation of HD metric.
These results underscore the effectiveness of each component in our approach.

\begin{figure}[thbp]
\centering
\includegraphics[width=0.45\textwidth]{Figure/Fig/fig_ablation.pdf}
\caption{Impact of (a) different segment length $L$; (b) different number of MLP layers $K$ on Synapse dataset.}
\label{fig:ablation}
\end{figure}
\subsubsection{Impact of segment length.}
In Figure~\ref{fig:ablation}(a), we investigate the impact of different segment lengths $L$ in PHNet. 
Expressly, the segment length is set to various ratios of the width ($W$), \ie, $1$, $\frac{1}{2}$, $\frac{1}{3}$, and $\frac{1}{4}$, respectively. 
With a larger segment length, long-range dependencies can be more effectively captured in deep layers. 
Conversely, smaller segment length implies fewer adjacent tokens are grouped, emphasizing more on local information.
The best performance is achieved when $L=\frac{1}{2}W$, which provides a good balance between local information and long-range dependencies. 

\subsubsection{Impact of MLP layers.}
In Figure~\ref{fig:ablation}(b), we study the influence of a different number of MLP layers $K$ in our PHNet.
Results show that the best performance is achieved when the number of MLP layers is $2$. 
This observation may vary across datasets.
Our method consistently outperforms the baseline method across different configurations of MLP layers, affirming the robustness and stability of our method.

