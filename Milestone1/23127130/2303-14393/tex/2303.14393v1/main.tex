\documentclass{amsart}
\usepackage[utf8]{inputenc}
\usepackage{amsmath}
\usepackage{amsfonts}
\usepackage{biblatex}
\usepackage{amssymb}

\usepackage{textcomp}
\usepackage{gensymb}
\usepackage{pst-node}
\usepackage{auto-pst-pdf}
\usepackage{tikz-cd}
\usepackage{tikz,tkz-euclide}
\usepackage{siunitx}
\usepackage{pgfplots}
\pgfplotsset{compat=1.17}
\usepackage{float} 
\addbibresource{bib.bib}
\setlength{\parindent}{0em}
\bibliography{bib}


\usepackage{amsthm}

\theoremstyle{plain}
\newtheorem{prop}{Proposition}[section]
\newtheorem{crl}{Corollary}[section]
\newtheorem{cor}{Corollary}[section]
\newtheorem{thm}{Theorem}[section]
\newtheorem{lem}{Lemma}[section]

\theoremstyle{definition}
\newtheorem{eg}{Example}[section]
\newtheorem{rem}{Remark}[section]
\newtheorem{defo}{Definition}[section]
\newtheorem{defn}{Definition}[section]

\newtheorem{prob}{Problem Set}
\newtheorem{notes}{Notes}


% Matrix trace.
 \DeclareMathOperator{\Tr}{Tr}
% Set of integers.
\newcommand{\Z}{\mathbb Z}
% A field F.
\newcommand{\F}{\mathbb F}
% a congruent to b in modulo c.
\newcommand{\ModCmp}[3]{#1 \equiv #2 \;(\bmod\; #3)}

% SL3F7.
\newcommand{\SL}{\mathrm{SL}_3(\F_7)}
% PSL3F7.
\newcommand{\PSL}{\mathrm{PSL}_3(\F_7)}
% GF3F7
\newcommand{\GL}{\mathrm{GL}_3(\F_7)}
% GL3F_(7^3).
\newcommand{\GLLarge}{\mathrm{GL}_3(\F_{7^3})}
% SL2F7.
\newcommand{\SLTwo}{\mathrm{SL}_2(\F_7)}
% General SLnF7.
\newcommand{\GenSL}{\mathrm{SL}_n(\F_7)}
%GL2F7
\newcommand{\GLTwo}{\mathrm{GL}_2(\F_7)}
% General GLnF7.
\newcommand{\GenGL}{\mathrm{GL}_n(\F_7)}

% 3 x 3 matrices.
\newcommand{\MatrThree}[9]{
    \begin{pmatrix}
        #1&#2&#3\\
        #4&#5&#6\\
        #7&#8&#9\\
    \end{pmatrix}
}

% 3 x 1 vectors represented as a matrix.
\newcommand{\vecthree}[3]{
    \begin{pmatrix}
        #1\\
        #2\\
        #3\\
    \end{pmatrix}
}

% 2 x 1 vectors represented as a matrix.
\newcommand{\vectwo}[2]{
    \begin{pmatrix}
        #1\\
        #2\\
    \end{pmatrix}
}

% 2 x 2 matrices.
\newcommand{\MatrTwo}[4]{
    \begin{pmatrix}
        #1&#2\\
        #3&#4\\
    \end{pmatrix}
}

% LCM function.
\DeclareMathOperator{\lcm}{lcm}
% Matrix trace.
\DeclareMathOperator{\tr}{tr}

% mbf for the sets
\newcommand{\mbf}[1]{\mathbf{#1}}

% Bold vector notation.
\newcommand{\vv}[1]{\boldsymbol{#1}}


\title{Eigenvalue-less matrices from $\SL$ to $\PSL$}
\author[J.L.C,G.C,Y.J, L.L, J.L, W.L, M.S, H.W]{Juan Lucas Callo, George Chen, Yasiru Jayasooriya*,  Leo Li, Jingni Liao, William Liu, Michael Sun, Haibing Wang}
% way it works is \author[Shorter author list]{Longer author list}
% You can edit it however you want
% Why did I get my name removed? – Jing. Re: we thought u were out but would be good if u could contribute
% P.S. I found an extra source and added it to the acknowledgements section.
\date{October 2021}

\begin{document}

\begin{abstract}
We analyse the set of matrices in $\SL$ without eigenvalues explicitly, extracting nice bijections between the 18 equally sized conjugacy classes contained within. In doing so, we discover a set of $18$ commuting matrices for which every conjugacy class is represented and tells us how to decide when collections of commuting matrices are simultaneously conjugate. The main innovation is the proofs are accessible to undergraduates and do not rely on computer calculations. It is also rare that the details of special cases are written down.
\end{abstract}



\maketitle
\section*{Introduction}

 The \emph{special linear group} $\SL$ consists of $3 \times 3$ matrices
 $$
 \MatrThree{a}{b}{c}{d}{e}{f}{g}{h}{i}.
 $$
with entries $a,b,c,d,,,,i\in\Z/7\Z$, and determinant 1, under matrix multiplication.

The size of the group is
$$2^5\times3^3\times 7^3\times 19.$$
Consider the subgroup $Z$ of $\SL$ consisting of the following matrices.

$$
\MatrThree{1}{0}{0}{0}{1}{0}{0}{0}{1},
\MatrThree{2}{0}{0}{0}{2}{0}{0}{0}{2},
\MatrThree{4}{0}{0}{0}{4}{0}{0}{0}{4}.
$$

The matrices in this $Z$ commute with all matrices in $\SL$ and therefore form a normal subgroup. The quotient group $\SL / Z$ is known as the projective special linear group or $\PSL$ and is a finite simple group of order
$$2^5\times3^2\times 7^3\times 19.$$

One reason why one may be interested in these groups is that these are small examples for which the Inverse Galois Problem is unknown. Or one might simply be interested to explore how linear algebra works over a finite field with opportunities to also apply one's understanding of the theory of finite groups.

The main goal of our project was to investigate the subset of matrices that do not have any eigenvectors. Some questions that motivated our exploration of these matrices were:

\begin{itemize}
    \item How many matrices do not have eigenvectors in $\SL$?
    \item What orders do these matrices have?
    \item How many conjugacy classes are there?
    \item Can we find nice representatives for each conjugacy class?
    \item How are the conjugacy classes related to each other?
    \item Is it possible to find a set of commuting matrices without eigenvectors where each represents a different conjugacy class?
    \item When are collections of commuting matrices without eigenvectors simultaneously conjugate?
    \item What maximal subgroups cover this set?
\end{itemize}


\begin{thm}\label{ConjClassOverview} 
In $\SL$:

\begin{itemize}
    \item There are exactly $18$ conjugacy classes in $\SL$ where none of the matrices in each conjugacy class have an eigenvector with entries in $\F_7$.
    \item Each of these conjugacy classes has size $2^5 \times 3^2 \times 7^3$.
    \item These $18$ conjugacy classes are the sets $\mbf{[0, 1]}$, $\mbf{[0, 2]}$, $\mbf{[0, 4]}$, $\mbf{[1,0]}$, $\mbf{[1, 3]}$, $\mbf{[1, 6]}$, $ \mbf{[2, 0]}$, $\mbf{[2, 5]}$, $\mbf{[2, 6]}$, $\mbf{[3, 1]}$, $\mbf{[3, 4]}$, $\mbf{[4, 0]}$, $\mbf{[4, 3]}$, $\mbf{[4, 6]}$, $\mbf{[5, 1]}$, $\mbf{[5, 2]}$, $\mbf{[6, 2]}$ and $\mbf{[6, 4]}$.
    \item All the elements in each of the conjugacy classes $\mbf{[0, 2]}$, $\mbf{[1, 3]}$, $\mbf{[2, 0]}$, $\mbf{[3, 1]}$, $\mbf{[3, 4]}$ and $\mbf{[4, 3]}$ have order $19$.
    \item All the elements in each of the conjugacy classes $\mbf{[0, 1]}$, $\mbf{[0, 4]}$, $\mbf{[1, 0]}$, $\mbf{[1, 6]}$, $\mbf{[2, 5]}$, $\mbf{[2, 6]}$, $\mbf{[4, 0]}$, $\mbf{[4, 6]}$, $\mbf{[5, 1]}$, $\mbf{[5, 2]}$, $\mbf{[6, 2]}$ and $\mbf{[6, 4]}$ have order $57$.
\end{itemize}
\end{thm}



Undoubtedly much is known about this group either as a special case of many general theorems or as the result of countless computer calculations whose results are stored in online databases. One of our goals is to look at this single group explicitly on its own and prove all of our results by hand and conceptually to produce a more accessible introduction into the topic for younger students. We have also included an extra section where explanatory notes and omitted details are given to help young students, which takes advantage of the relaxing page restrictions due to electronic publications.

An unexpected outcome of this project was to make observations about collections of commuting matrices and their simultaneous conjugacy.

\begin{thm}
There exists $18$ commuting matrices with each from a different conjugacy class $\mbf{[i, j]}$.
\end{thm}

%\begin{eg} Let $M=$ (someone type the M and all its powers)
%$$M,  M^2$$

%\end{eg}

\begin{thm}
Every collection of commuting matrices excluding $2I,4I$ either consists of matrices all with eigenvectors or they consist of matrices all without eigenvectors and all powers of some $M$.
Two such collections are simultaneously conjugate if and only if there exists $M_1$ for the first collection to be powers of $M_1$ and $M_2$ of the same order for the second collection in which all the corresponding powers agree modulo their common order.
\end{thm}

%Working through the details of such a paper can also replace many boring and passive undergraduate courses.
%We were aware of the approaches of extending the field to $\F_{7^3}$ or descending from known results about $\GL$ but did not have time pursue them as we had already found many interesting items approaching the problems directly.

\section{Irreducible characteristic polynomials over $\F_7$}

We compute the characteristic polynomials of matrices without eigenvectors. These will necessarily be irreducible in $\F_7$.
Let
$$
M := \MatrThree{a}{b}{c}{d}{e}{f}{g}{h}{i} \in \SL
$$
and consider its characteristic polynomial
$$\det(\lambda I-M) = \lambda^3-(a+e+i)\lambda^2+(ae+ie+ai-cg-bd-fh)\lambda+\det(-M).$$
With $\tr(M)=a+e+i$ and $\det(-M) = -\det(M) = -1$, we need the equation
$$\lambda^3-\tr(M)\lambda^2+(ae+ie+ai-cg-bd-fh)\lambda-1=0$$
to not hold for $\lambda=0,1,2,\dots,6$.
That is,
\begin{align*} 
\lambda &= 1: \hspace{20px} -(a+e+i)+(ae+ei+ai-cg-bd-fh) &\neq 0. \\
\lambda &= 2: \hspace{20px} 3(a+e+i)+2(ae+ei+ai-cg-bd-fh) &\neq 0. \\
\lambda &= 3: \hspace{20px} -2(a+e+i)+3(ae+ei+ai-cg-bd-fh)-2 &\neq 0. \\
\lambda &= 4: \hspace{20px} -2(a+e+i)-3(ae+ei+ai-cg-bd-fh) &\neq 0. \\
\lambda &= 5: \hspace{20px} 3(a+e+i)-2(ae+ei+ai-cg-bd-fh)-2 &\neq 0. \\
\lambda &= 6: \hspace{20px} -(a+e+i)-(ae+ei+ai-cg-bd-fh)-2 &\neq 0. \\
\end{align*}


We consider all the possible values of $\tr(M)$ and the coefficient of $\lambda$ to get an exhaustive list of characteristic polynomials: of which the trace $0,\pm1$ cases are as follows. \\ 

First, we let $\mbf{[i,-]}$ denote the set of trace $i$ matrices without eigenvectors in $\SL$. Also, we denote $I$ as the $3 \times 3$ identity matrix in $\SL$.


\subsection{Trace 0 matrices with no eigenvectors in $\SL$}
The characteristic polynomials with no roots in $\F_7$ are
$$\lambda^3+\lambda-1.$$
$$\lambda^3+2\lambda-1.$$
$$\lambda^3+4\lambda-1.$$
Using our notation, for ${k=1, 2, 4}$, we denote the set of matrices with the characteristic polynomial ${\lambda^3+k \lambda -1}$ by $\mbf{[0,k]}$. \\

Note that if we multiply a matrix in $\mbf{[0, -]}$ by $2I$, its trace remains zero and the coefficient of $\lambda$ in the characteristic polynomial will be scaled by a factor of $2^2=4$. This is equivalent to $\lambda$ being replaced by $4 \lambda = \frac{\lambda}{2}$ in the characteristic polynomial (using arithmetic in $\F_7$). Hence, if we select a representative element $M$ of $\mbf{[0, 4]}$, then we will have 

\begin{align*}
    2M &\in \mbf{[0,2]}.\\
    4M = 2 \cdot 2M &\in \mbf{[0,1]}.
\end{align*}

We choose a representative matrix in $\mbf{[0, 4]}$ (for lack of a better criteria) with the maximum number of zeroes possible, 
$$
M = \MatrThree{0}{1}{3}{0}{0}{1}{1}{0}{0}.
$$

This matrix can be checked to be in $\mbf{[0, 4]}$. Note that there are no such representative matrices with exactly $6$ zero entries as these would have exactly one non-zero entry in each column and hence would have an eigenvector. Also, there are no representative matrices with $7$ or more zero entries as these would have a row of all zero entries so would have determinant zero. $M$ has order 57 and $M^{19}=4$. This was confirmed by computer and by hand. This also implies that all matrices in $\mbf{[0, -]}$ have an order of either $19$ or $57=3\times 19$, the 3 resulting from the order of the scaling matrix $2I$. In the next section, one such calculation by hand is given explicitly. \\

% this would force some row to consist entirely of zeroes, but this would force the determinant to be $0$ and thus the matrix not in $\SL$, a contradiction.\\ thisis wrong
The matrix $M$ along with $2M$ and $4M$ can be used to represent 3 conjugacy classes of traceless matrices without any eigenvectors. We will endeavour to prove that these are the only three conjugacy classes.  \\ 

The characteristic equations prove that these are not conjugate. Another way of seeing it is to compare eigenvalues $M$ and $2M$ considered as matrices over a larger field of size $7^3$.



%above matrix belongs to some conjugacy class, and is chosen to be the representative of this conjugacy class. We proceed to show that multiplication of the entries in this matrix by 2 yields matrices that belong to different conjugacy classes, thus, giving us three distinct conjugacy classes that live inside the trace 0 set of matrices. to be continued -yee.


%It suffices to prove/calculate the total size of these sets and we can conclude that everything is conjugate to one of $M,2M,4M$.

%\\The trace zero class can be represented by a matrix with 5 zeros and its scalar multiples by 2. Obviously there are no matrices with 6 zeros.

\subsection{Trace 1 matrices without eigenvectors}
One characteristic polynomial with no roots in $\F_7$ will be
$$\lambda^3-\lambda^2-1$$
Using similar notation as earlier, denote the set of matrices associated with this characteristic polynomial as $$\mbf{[1, 0]}$$
(trace is $1$, coefficient of $\lambda$ is $0$). A representative of this set is
$$
M = \MatrThree{0}{1}{0}{0}{1}{-1}{-1}{0}{0}.
$$

This has order $57$, with $M^{19}=4I$. \\

We also have the characteristic polynomial 

$$\lambda^3-\lambda^2+3\lambda-1$$

with no roots in $\SL$. The corresponding set of matrices will be denoted as 
$$\mbf{[1, 3]}.$$
A representative of this set is
$$
M = \MatrThree{0}{1}{4}{0}{1}{5}{1}{0}{0}.
$$

This has order $19$, which was confirmed by hand. \\



Finally, we have the characteristic polynomial 

$$\lambda^3-\lambda^2+5\lambda-1$$

with no roots in $\SL$. The corresponding set of matrices will be denoted as 
$$\mbf{[1, 5]}.$$
A representative of this set is 
$$
M = \MatrThree{0}{3}{2}{0}{1}{1}{1}{0}{0}.
$$
This has order $57$, with $M^{19}=2I$.

%Similar stabiliser calculations confirm the classes have the right size. These were first done by computer first. We will prove it shortly.

\subsection{Trace -1 matrices without eigenvectors}

One characteristic polynomial with no roots in $\SL$ will be

$$\lambda^3+\lambda^2+4\lambda-1$$

The corresponding set of matrices will be 
$$\mbf{[-1, 4]}.$$
A representative of this set is

$$
M = \MatrThree{0}{2}{3}{0}{1}{-1}{1}{0}{0}.
$$

This has order $57$. \\

The second characteristic polynomial with no roots in $\SL$ will be 

$$\lambda^3+\lambda^2+2\lambda-1$$

The corresponding set of matrices will be $\mbf{[-1, 2]}$. A representative of this set is

$$
M = \MatrThree{0}{3}{2}{0}{-1}{-1}{-1}{0}{0}.
$$
The order of $M$ is, which was confirmed by hand and by computer (GC). Also, $M^{19}=2I$, which means that the order is $19$ in $\PSL$.

\subsection{Bijections with other traces}

We get bijections

$$\mbf{[i,-]}\to \mbf{[2i,-]}$$
$$M\mapsto 2M$$

Since multiplying a matrix $M \in \mbf{[i, j]}$ by $2$ causes the the coefficient of $\lambda$ in the corresponding characteristic polynomial to multiply by $2^2 = 4 = \frac{1}{2}$ (in $\F_7$), then the above bijection can be restricted to  

$$\mbf{[i,j]}\to\mbf{[2i,4j]}.$$
This corresponds to 
$$\lambda\mapsto \lambda/2$$
for the characteristic polynomials in $\lambda$. \\

Hence, the sets of matrices $\mbf{[1, -]}, \mbf{[2, -]}, \mbf{[4, -]}$ are very closely related and we have a similar result for the sets of matrices $\mbf{[-1, -]}, \mbf{[-2, -]}, \mbf{[-4, -]}$. \\

Since the matrices in $\mbf{[0, -]}$ can be partitioned into $\mbf{[0,1]}, \mbf{[0, 2]}$ and $\mbf{[0, 4]}$, each non-empty and with distinct corresponding characteristic polynomials, then the three representatives for each of the three subsets belongs to a distinct conjugacy class so there are at least three conjugacy classes that partition $\mbf{[0, -]}$.

This gives at least 18 conjugacy classes.

We will prove in the next section that these 18 conjugacy classes are all of them.

There are some other interesting bijections, which we explore more completely later:
\begin{eg}
$$\mbf{[i, j]}\mapsto \mbf{[j, i]}$$
$$\lambda\mapsto \frac{1}{\lambda}$$
\end{eg}

%Bijections between the trace $1$, trace $2$ and trace $4$ sets can be obtained by scaling by $2$ (multiplying by $2I$) since $2^3=1$. Furthermore, bijections between the trace $-1=6$, trace $5$ and trace $3$ sets can also be obtained by multiplying each entry in each matrix by $2$. Multiplication by $2$ of matrices in the trace $0$ set, however, map back into the same set, and so these matrices can be divided into 3 distinct classes. 



%We will see that these are the conjugacy classes corresponding to the 3 possible characteristic equations. Similarly there will be two conjugacy classes in each of the trace -1 case and three conjugacy classes in the trace 1 case which each class corresponding to one of the possible characteristic equations and having size

%$$2^5\times 3^2\times 7^3.$$

%We proceed to prove that all the conjugacy classes have the same size and that they are all accounted for. Scaling if necessary, we list only the trace 0,1,-1 cases without loss of generality and consider the other traces only as needed.


%Similar stabiliser calculations confirm the classes have the right size. These were first done by computer first. 

\section{Orders of matrices in $\mbf{[i, j]}$}

\begin{prop}All matrices in $\SL$ without eigenvectors have order 19 or 57
\end{prop}
\begin{proof}
We summarise the results given in later sections that prove this claim. Let $M\in \mbf{[i, j]}$ have characteristic polynomial $P\in \F_7[x]$. It can be shown that $P$ has three distinct roots in the extended field $\F_{7^3}$ (none of which are in $\F_7$) of the form $\alpha$, $\alpha^7$ and $\alpha^{49}$ so $M$ is diagonalisable over $\F_{7^3}$ with the three roots being the diagonal entries of the corresponding diagonal matrix (the eigenvalues). The determinant of this diagonal matrix must be $\alpha^{57}=1$. Hence, the order of $\alpha$ (and also $M$) is a factor of $57$. Since $\alpha \not\in \F_7$, then the order cannot be $1$ or $3$, so must be $19$ or $57$. 
\end{proof}

Knowing all have order 19 or 57 is sufficient to establish the conjugacy classes in the next section but we would not know which had order 19 and which had order 57 (though we would know that for every one of order 19 there are 2 of order 57).

There are many ways that the orders can be obtained.

To find out the order for each set, it suffices to know it for one matrix in it. Here is one such matrix:

Table of first 19 powers of a matrix with characteristic polynomial $\lambda^{3}+2\lambda -1$.
\ \\
\begin{tabular}{c|c|c|c}\label{powertable}
power&matrix&trace&class\\\hline
1&$
\MatrThree{0}{2}{-1}{0}{0}{2}{2}{0}{0}
$&0&$\mbf{[0,2]}$\\\hline
2&$
\MatrThree{-2}{0}{-3}{-3}{0}{0}{0}{-3}{-2}
$&3&$\mbf{[3,4]}$\\\hline
3&$
\MatrThree{1}{3}{2}{0}{1}{3}{3}{0}{1}
$&3&$\mbf{[3,4]}$\\\hline
4&$
\MatrThree{-3}{2}{-2}{-1}{0}{2}{2}{-1}{-3}
$&1&$\mbf{[1,3]}$\\\hline
5&$
\MatrThree{3}{1}{0}{-3}{-2}{1}{1}{-3}{3}
$&4&$\mbf{[4,3]}$\\\hline
6&$
\MatrThree{0}{-1}{-1}{2}{1}{-1}{-1}{2}{0}
$&1&$\mbf{[1,3]}$\\\hline
7&$
\MatrThree{-2}{0}{-2}{-2}{-3}{0}{0}{-2}{-2}
$&0&$\mbf{[0,2]}$\\\hline
8&$
\MatrThree{3}{3}{2}{0}{3}{3}{3}{0}{3}
$&2&$\mbf{[2,0]}$\\\hline
9&$
\MatrThree{-3}{-1}{3}{-1}{0}{-1}{-1}{-1}{-3}
$&1&$\mbf{[1,3]}$\\\hline
10&$
\MatrThree{-1}{1}{1}{-2}{-2}{1}{1}{-2}{1}
$&3&$\mbf{[3,1]}$\\\hline\end{tabular}
\quad
\begin{tabular}{c|c|c|c}
power&matrix&trace&class\\\hline
11&$
\MatrThree{2}{-2}{3}{2}{3}{-2}{-2}{2}{2}
$&0&$\mbf{[0,2]}$\\\hline
12&$
\MatrThree{-1}{-3}{1}{3}{-3}{-3}{-3}{3}{-1}
$&2&$\mbf{[2,0]}$\\\hline
13&$
\MatrThree{2}{-2}{2}{1}{-1}{-2}{-2}{1}{2}
$&3&$\mbf{[3,1]}$\\\hline
14&$
\MatrThree{-3}{-3}{1}{3}{2}{-3}{-3}{3}{-3}
$&3&$\mbf{[3,4]}$\\\hline
15&$
\MatrThree{2}{1}{-3}{1}{-1}{1}{1}{1}{2}
$&3&$\mbf{[3,1]}$\\\hline
16&$
\MatrThree{1}{-3}{0}{2}{2}{-3}{-3}{2}{1}
$&4&$\mbf{[4,3]}$\\\hline
17&$
\MatrThree{0}{2}{0}{1}{-3}{2}{2}{1}{0}
$&4&$\mbf{[4,3]}$\\\hline
18&$
\MatrThree{0}{0}{-3}{-3}{2}{0}{0}{-3}{0}
$&2&$\mbf{[2,0]}$\\\hline
19&$
\MatrThree{1}{0}{0}{0}{1}{0}{0}{0}{1}
$&3&$\mbf{[3,3]}$\\\hline
20&$
\MatrThree{0}{2}{-1}{0}{0}{2}{2}{0}{0}
$&0&$\mbf{[0,2]}$\\\hline
\end{tabular}

The table of powers of a matrix can also be used to get the orders of all the other sets because a representative from each set is present.


Another way is to show any $\lambda$ satisfying the characteristic equation will also satisfy $\lambda^{19}=1$ as follows:  


\begin{prop}\label{02nt}
All matrices in $\mbf{[0, 2]}$ have order 19. 
\end{prop}
\begin{proof}  
We apply some results given later in later sections. Since all matrices $M$ in $\mbf{[0,2]}$ have no eigenvalues in $\F_7$, then by Lemma \ref{EigenVectorWhenExtended}, $M$ has three distinct eigenvalues in $\F_{7^3}$ (none of which are in $\F_7$) and each eigenvalue is a root of the equation $\lambda^3 + 2\lambda - 1 = 0$. \\

By Lemma $\ref{OrderNIffExtSame}$ it suffices to prove that any root $\lambda$ of the above equation has order $19$. We will prove that $\lambda^{21} = \lambda^2$, which is equivalent to $\lambda^{19} = 1$. Below we will use the property that $(a+b)^7 = a^7+b^7$ for all $a, \ b \in \F_{7^3}$. \\

First, we note from the cubic equation that $\lambda^3 = 1 - 2\lambda$. Now we compute $\lambda^{21}$ below.

. We show $\lambda^{21}=\lambda^2$:
$$\begin{aligned}\lambda^{21}&=(1-2\lambda)^7)\\
                            &= 1-2^7\lambda^7\\
                            &= 1 - 2^7 \lambda (\lambda^3)^2\\
                            &=1-2\lambda(1-2\lambda)^2\\
                        &=1-2\lambda(1-4\lambda+4\lambda^2)\\
                        &=1-2\lambda+\lambda^2-\lambda^3\\
                        &=1-2\lambda+\lambda^2-(1-2\lambda)\\
                        &=\lambda^2.
                            \end{aligned}$$
                            
Hence, the order of all three eigenvalues is a factor of $19$. Since $19$ is prime and none of the eigenvalues are equal to $1$ (since the eigenvalues are not in $\F_7$), then the eigenvalues have order $19$.
\end{proof}

This above calculation with the $\lambda$ relation can account for all the matrices in the set at once.\footnote{Other such calculations are included in the Explanatory notes Section.}
%and multiply them by $2I$ and $4I$ to get the order 57 sets.

%one of the $\lambda$ calculations that yields order 19 along with 5 powers to get all six sets with order 19
We can also use a combination of these strategies, as is demonstrated in the final proof.

\begin{thm}\label{order} All matrices in $\SL$ without eigenvectors have order 19 or 57. Moreover,    
\begin{itemize}
   \item All the elements in each of the sets $\mbf{[0, 2]}$, $\mbf{[1, 3]}$, $\mbf{[2, 0]}$, $\mbf{[3, 1]}$, $\mbf{[3, 4]}$ and $\mbf{[4, 3]}$ have order $19$.
    \item All the elements in each of the sets $\mbf{[0, 1]}$, $\mbf{[0, 4]}$, $\mbf{[1, 0]}$, $\mbf{[1, 5]}$, $\mbf{[2, 5]}$, $\mbf{[2, 6]}$, $\mbf{[4, 0]}$, $\mbf{[4, 6]}$, $\mbf{[5, 1]}$, $\mbf{[5, 2]}$, $\mbf{[6, 2]}$ and $\mbf{[6, 4]}$ have order $57$.
\end{itemize}
\end{thm}
\begin{proof}
We summarise the proof given in Lemma \ref{CharacteriseAllOrder19And57Matrices}. \\

Consider the matrix 

$$
M = \MatrThree{0}{2}{-1}{0}{0}{2}{2}{0}{0}.
$$

We can use Table \ref{powertable} to select powers of $M$ that are in each of the sets $\mbf{[0, 2]}$, $\mbf{[1, 3]}$, $\mbf{[2, 0]}$, $\mbf{[3, 1]}$, $\mbf{[3, 4]}$ and $\mbf{[4, 3]}$, and these matrices will also have order $19$. \\

Scaling these matrices by $2$ or $4$ and using Lemma \ref{OrderOfDoubleAndQuadrupleIs57} gives us matrices in each of the other $12$ sets having order $57$. The theorem then follows by Lemma \ref{OneOrderNThenAll}.


% For $\mbf{[0, 2]}$ we use Proposition \ref{02nt}, We can now use the $2,4,5,8,10$-th powers of a matrix in this set to get the other 5 sets with order 19 (see Table \ref{powertable}). Scaling these will give the 12 sets with order 57.
%
\end{proof}



\section{Conjugacy classes}

%\newline
%Size of entire group: 5630688
%\newline
%$19\times7^{3}\times3^{3}\times2^{5}$
%\newline
%Number of matrices det 1 trace 0 no eigenvectors: 296352
%\newline
%$7^{3}\times3^{3}\times2^{5}$
%\newline
%Num matrices det 1 trace 1 no eigenvectors: 296352
%\newline
%$7^{3}\times3^{3}\times2^{5}$
%\newline
%Num matrices det 1 trace 2 no eigenvectors: 296352
%\newline
%$7^{3}\times3^{3}\times2^{5}$
%\newline
%Num matrices det 1 trace 3 no eigenvectors: 197568
%\newline
%$7^{3}\times3^{2}\times2^{6}$
%\newline
%Num matrices det 1 trace 4 no eigenvectors: 296352
%\newline
%$7^{3}\times3^{3}\times2^{5}$
%\newline
%Num matrices det 1 trace 5 no eigenvectors: 197568
%\newline
%$7^{3}\times3^{2}\times2^{6}$
%\newline
%Num matrices det 1 trace 6 no eigenvectors: 197568
%\newline
%$7^{3}\times3^{2}\times2^{6}$\\
\begin{center}
    \begin{tabular}{|c|c|c|}
    \hline
        Trace & matrices in $\SL$ without eigenvectors& classes\\
    \hline
    \hline
         0 & $296352 = 3(7^3 \times 3^2 \times 2^5)$&$\mbf{[0, 1]},\mbf{[0, 2]},\mbf{[0, 4]}$\\
    \hline
        1 & $296352 = 3(7^3 \times 3^2 \times 2^5)$&$\mbf{[1, 0]},\mbf{[1, 3]}, \mbf{[1, 5]}$\\
    \hline
        2 & $296352 = 3(7^3 \times 3^2 \times 2^5)$&$\mbf{[2, 0]},\mbf{[2,5]},\mbf{[2,6]}$\\
    \hline
        3 & $197568 = 2(7^3 \times 3^2 \times 2^5)$&$\mbf{[3,4]},\mbf{[3,1]}$\\
    \hline
        4 & $296352 = 3(7^3 \times 3^2 \times 2^5)$&$\mbf{[4,0]},\mbf{[4,6]},\mbf{[4,3]}$\\
    \hline
        5 & $197568 = 2(7^3 \times 3^2 \times 2^5)$&$\mbf{[5,1]},\mbf{[5,2]}$\\
    \hline 
        6 & $197568 = 2(7^3 \times 3^2 \times 2^5)$&$\mbf{[6,2]},\mbf{[6,4]}$\\
    \hline
    \end{tabular}
\end{center}
This was first done by computer and similar results appear online. The text files generated by computer show all the matrices which satisfy the conditions. 
\newline
We proceed to explain and prove these results independent of computational power.


\begin{lem} Let
$$
M = \MatrThree{0}{1}{3}{0}{0}{1}{1}{0}{0}.
$$

The only matrices which commute with $M$ are the $57$ powers of $M$, consequently the size of its conjugacy class is 
$$2^5\times 3^2\times 7^3.$$
\end{lem}
\begin{proof}
The stabiliser of the matrix $M$ can be used to calculate the size of the conjugacy class. Obviously, the stabilisers of $2M$ and $4M$ are identical. For $SM=MS$ we get

$$S=\begin{pmatrix} 
a&b&3b+d\\
d&a-3d&b\\
b&d&a\\
\end{pmatrix}
$$
The determinant 1 condition is
$$a^3+b^3+d^3-3a^2d-3ab^2+2db^2-d^2b-3bad=1$$

One way to see this equation has $57$ solutions is by case bashing (we can reduce the number of cases from $7^3$ to $a=0,1,-1$ since we can scale by $2$ a similar calculation is summarised in the Explanatory notes Section). These must then be the $57$ powers of $M$.

A more indirect approach is as follows. Since these form a subgroup, the number must be divisible by $57$ and also a divisor of $2^5\times3^3\times7^3\times19$. It cannot have a factor of $7$ as $7\times57>7^3$. For $3$, any order 3 matrix not $2I,4I$ must have a one dimensional eigenspace preserved by $M$, a contradiction, while if a matrix has order $9$, its cube has order 3 but cannot equal $I$ since it was order 9, it cannot be $2I$ or $4I$ since $2$ and $4$ are not cubes modulo 7. Of course a matrix with order 27 would cube to a matrix with order 9. Now it suffices to show that no order 2 matrix $S$ commutes with $M$. 


%We also show that no matrix S that commutes with M has order 2. Without loss of generality, let the trace of $S$ be $3$. Then we have $$a=d+1.$$ To have the top left cell of $S\times S$ be $1$, we have the following solutions for $(b,d)$ are $(0,0), (0,5), (1,4), (1,6), (2,2), (2,6), (6,2)$ and $(6,5)$. For the bottom left cell to be $0$, we have solutions for $(b,d)$ are $(0,0), (2,5), (4,2), (4,4), (5,1)$ and $(5,3)$. We see that the only solution in both sets is $b=d=0$. However, this gives the identity matrix, which has order $1$. Therefore there are no matrices that commute with $M$ and have order $2$. \\

Any such matrix must have all eigenvalues $1$ and a two dimensional 1-eigenspace, as otherwise $M$ will have an eigenvector. We show that no such order two matrix exists. Up to conjugacy we can assume
$S$ is of the form
$$
\MatrThree{1}{0}{x}{0}{1}{y}{0}{0}{1}
$$

Consider the last column of $S^2$. The first row entry is $2x$, and the second row entry is $2y$. For $S$ to have order $2$, it follows that $x$ and $y$ are both 0, and so the matrix is the identity matrix, which has order $1$. Therefore there are no matrices that commute with $M$ and have order $2$.

Hence by the Orbit-Stabiliser theorem, the size of the conjugacy classes for $M$, $2M$ and $4M$ are all
$$2^5\times 3^2\times 7^3.$$
\end{proof}

Recall Sylow's theorems from group theory:

\begin{defn}
    \textbf{Sylow $p$-subgroups: } For a prime number $p$, a Sylow $p$-subgroup of a group $G$ is a maximal $p$-subgroup of $G$ (its order is a power of $p$) and is not a proper subgroup of any other $p$-subgroup of $G$.  
\end{defn}
\begin{thm}\label{SylowTheorems}
    \textbf{Sylow's Theorems:} 
    
    \begin{enumerate}
        \item For every prime factor $p$ with multiplicity $n$ of the order of a finite group $G$, there exists a Sylow-$p$ subgroup of $G$ of order $p^n$.
        \item Given a finite group $G$ and a prime number $p$, all Sylow $p$-subgroups of $G$ are conjugate to each other. That is, if $H$ and $K$ are Sylow $p$-subgroups of $G$, then there exists an element $g \in G$ with $g^{-1}Hg=K$.
        \item Let $p$ be a prime factor with multiplicity $n$ of the order of a finite group $G$, so that the order of $G$ can be written as $p^n m$, where $n > 0$ and $p$ does not divide $m$. Let $n_p$ be the number of Sylow $p$-subgroups of $G$. Then the following hold:
        \begin{itemize}
            \item $n_p$ divides $m$, which is the index of the Sylow $p$-subgroup in $G$.
            \item $\ModCmp{n_p}{1}{p}$.
            \item $n_p = |G:N_G(P)|,$ where $P$ is any Sylow $p$-subgroup of $G$ and $N_G$ denotes the normalizer. 
        \end{itemize}
    \end{enumerate}
\end{thm}

\begin{lem}\label{PossibleSubgroupSize}
 If $n_{19}$ is 1 modulo 19 and $n_{19}\mid 2^53^37^3$, then 
 $$n_{19} \in \{1, \  2^5 \cdot 3, \ 7^3, \ 2^4 \cdot 3^2 \cdot 7, \ 2^3 \cdot 3^3 \cdot 7^2, \ 2^5 \cdot 3 \cdot 7^3       \}.$$
\end{lem} 

A further argument will narrow this down to just $2^5 \cdot 3 \cdot 7^3$ in the proof of Theorem \ref{ExactSizeOfConjClass}. This will mean that the normaliser has size $3^2\cdot 19$.
%\begin{proof}
%The order of $\SL$ can be written as $19m$ where $m = 2^5 \times 3^3 \times 7^3$. By Sylow's third theorem with the prime $p = 19$, we obtain the following conditions on $|\mathbb{K}|$.

%\begin{itemize}
 %   \item $|\mathbb{K}| \equiv 1 \; (\bmod\; 19)$
 %   \item $|\mathbb{K}|$ is a factor of $m$.
%\end{itemize}
%By testing the out all factors of $m$ that are $1 \; (\bmod\; 19)$, we obtain the possible values of $|\mathbb{K}|$ listed above.
%\end{proof}


\begin{lem}\label{samestab}
The size of all 18 conjugacy classes above is $2^5 \times3^2\times 7^3$
\end{lem}
\begin{proof}
By the orbit stabiliser theorem it suffices to show the stabilisers of any representative of each conjugacy class have the same size. Take a matrix $M$ of order $19$, the subgroup it generates is a Sylow 19-subgroup and conjugate to all other order 19 subgroups by Sylow's theorem. Therefore for any other conjugacy class, there exists $j$ such that $M^j$ is in that class (or a multiple of something in that class by 2 or 4). Therefore anything that stabilises $M$ will stabilise $M^j$, that is $XMX^{-1}=M$ implies
$XM^jX^{-1}=M^j$. Conversely since 19 is prime, $M^j$ is also a generator and hence $M$ is also a power of $M^j$. Clearly scaling by $2$ does not affect the size of the stabiliser as $2$ commutes with everything.
\end{proof}

\begin{thm}\label{ExactSizeOfConjClass}There are exactly 18 conjugacy classes of matrices without eigenvectors in $\SL$, each have the same size 
$$2^5\times3^2\times7^3$$
which gives a total of
$$2^6\times 3^4\times 7^3.$$
These $18$ conjugacy classes are the sets $\mbf{[0, 1]}$, $\mbf{[0, 2]}$, $\mbf{[0, 4]}$, $\mbf{[1, 0]}$, $\mbf{[1, 3]}$, $\mbf{[1, 5]}$, $\mbf{[2, 0]}$, $\mbf{[2, 5]}$, $\mbf{[2, 6]}$, $\mbf{[3, 1]}$, $\mbf{[3, 4]}$, $\mbf{[4, 0]}$, $\mbf{[4, 3]}$, $\mbf{[4, 6]}$, $\mbf{[5, 1]}$, $\mbf{[5, 2]}$, $\mbf{[6, 2]}$ and $\mbf{[6, 4]}$.
\end{thm}
\begin{proof}
By Lemma \ref{samestab} it suffices to show that we cannot exceed $2^5\times 3^2\times7^3\times18$ matrices without eigenvectors which have orders $19$ or $57$. By Sylow's theorems, there are 5 possible values for the number of order $19$ subgroups (Corollary \ref{PossibleSubgroupSize}), the greatest of which is $2^5\times 3\times 7^3$, these all consist of $18$ distinct matrices without eigenvectors and gives $2^5\times 3\times 7^3\times 18$ such matrices of order 19. Scaling these matrices by 2 and 4 we get the required $2^5\times 3^2\times 7^3\times 18$ matrices, while less would be insufficient.

Now by Proposition \ref{02nt} and similar calculations we see that every matrix without eigenvectors have order 19 or 57.
\end{proof}

\section{Relationships between classes}





Let us analyse this Table \ref{powertable} to see what it tells us.
One thing we notice is that all of the conjugacy classes with order 19 are represented here (this would be an alternate proof that they have order 19 as well). The powers needed to obtain each conjugacy class does not depend on the matrix used only on the conjugacy class. The powers needed for any other class can be deduced from those in one table. The order 19 class is determined by the trace except for when the trace is 3.

%Using these powers, we are able to get a set of commuting matrices each representing a different conjugacy class. This gives a simultaneous six fold conjugacy class of the same size as a single conjugacy class. Multiplying representatives by 2 and 4 will give many more combinations of classes being represented.

\subsection{Bijections between conjugacy classes}
%When two sets are equal in size $n$, one can ask if there is a nice bijection to realise this equality. Of course, there are $n\!$ bijections, many of which are not what we are after. When these sets are conjugacy classes, any time we pick a representative from each class, say $A$ and $B$ we get a bijection
%$$gAg^{-1}\mapsto gBg^{-1}$$
%if the stabilisers of A and B are the same

We summarise the bijections resulting from the different powers.



For three generators modulo 19, we get upon repeated application, the bijections cycling between
$$\mbf{[3,4]}\to\mbf{[1,3]}\to\mbf{[2,0]}\to\mbf{[4,3]}\to\mbf{[3,1]}\to\mbf{[0,2]}$$
$$M\mapsto M^2$$
$$M\mapsto M^3$$
$$M\mapsto M^{14}$$
The inverses of these generators, 
$$M\mapsto M^{10}$$
$$M\mapsto M^{13}$$
$$M\mapsto M^{15}$$
traverse these sets in the opposite direction.

Since $7=2^6$, we get for all valid $i,j$, bijections
$$\mbf{[i,j]}\to \mbf{[i,j]}$$
$$M\mapsto M^7$$
and its inverse
$$M\mapsto M^{11}=M^{7^2}.$$


9,4,6 have two three cycles in forward direction, their inverses 17, 5, 16 in the reverse direction
$$\mbf{[3,4]}\to\mbf{[2,0]}\to\mbf{[3,1]}$$
$$\mbf{[1,3]}\to\mbf{[4,3]}\to\mbf{[0,2]}$$

8 have three pairs with inverse 12, as well as 18 which is its own inverse
$$\mbf{[i,j]}\to\mbf{[j,i]}$$

The 18 recovers the bijection in Example \ref{}.

One way in which these bijections are nice is that they are order preserving and commute with the conjugation action (isomorphism of $G$-sets).


\subsection{Simultaneous conjugacy}
Let us make some observations here about the problem of simultaneous conjugacy for commuting matrices.

Is it possible to get a set of commuting matrices where each represents a different conjugacy class?

When are collections of commuting matrices simultaneously conjugate?

If we take an $n$-tuple of commuting elements in $\SL$ $(A_1,\dots , A_n)$ where $A_k\in \mbf{[i_k,j_k]}$ for some $i_k,j_k$ and there is another commuting $n$-tuple $\SL$ $(B_1,\dots , B_n)$ where $B_k\in \mbf{[i_k,j_k]}$, they might not be simultaneously conjugate because the powers might not match. However if the powers match. then they are simultaneously conjugate.


\begin{thm}
Every collection of commuting matrices excluding $2I,4I$ either consists of matrices all with eigenvectors or they consist of matrices all without eigenvectors and all powers of some $M$.
Two such collections are simultaneously conjugate if and only if there exists $M_1$ for the first collection to be powers of $M_1$ and $M_2$ of the same order for the second collection in which all the corresponding powers agree modulo their common order.
\end{thm}
\begin{proof}
If there is a matrix without any eigenvectors in the collection then by Lemma \ref{samestab}, it can only commute with the $57$ powers of some $M$ without eigenvectors. So among the commuting matrices there is a pair $A,B$ for which $B=A^j$ and this relationship is invariant under conjugacy and therefore must hold for any pair of matrices simultaneously conjugate to $A,B$.
\end{proof}

%do the other many bijections
%relationships between bijections

\section{Understanding the subgroup structure}
A maximal subgroup of $\SL$ consists of all block upper triangular matrices of the form
$$
\MatrThree{a}{b}{c}{0}{e}{f}{0}{h}{i}
$$
with determinant 1 and entries in $\F_7$. The number of these is
$$(7^2-1)(7^2-7)7^2=2^5\times 3^2\times 7^3$$

Clearly, for any block upper triangular matrix, the vector 
$$
e_1 := \vecthree{1}{0}{0}
$$
is an eigenvector with eigenvalue $a$.


Any matrix with an eigenvector is conjugate to such a block upper triangular matrix and thus belongs to a subgroup conjugate to the block upper triangular matrices. 
%insert a diagram at some point.

These maximal subgroups intersect  non trivially either in pairs or 3 at a time but not more. These are those with at least 2 or 3 dimensions worth of eigenspaces are are conjugate to 

$$
\MatrThree{a}{0}{c}{0}{e}{f}{0}{0}{i}
$$

or 
$$
\MatrThree{a}{0}{0}{0}{e}{0}{0}{0}{i}
$$

are diagonalisable.

To see what other sorts of subgroups there are, we first search for matrices in $\SL$ that do not have any eigenvectors. We see there are many order 19 and 57 subgroups which contain them and don't intersect any of the other maximal subgroups. So there is another maximal subgroup not of the kind above. However if one extends to $\F_{7^3}$ it is contained in one of the kind above but with entries in a bigger field.

\begin{lem}
$\SL$ does not have any matrices of order $27$.
\end{lem}
\begin{proof}
Let $x\in\SL$ and Consider the extension of $\F$ so that $x$ is diagonalisable. The orders of the eigenvalues must be a factor of $7-1$, $7^2-1$ and $7^3-1$ but these have at most a factor of $3^2$ not $3^3$.
\end{proof}
We definitely know there are elements of order $3$.
The previous lemma also suggests there might be elements of order $3^2$ though this is not the case.

\begin{lem}
There are no elements of order 9 in $\SL$.
\end{lem}
\begin{proof}
One approach is to show that no eigenvalues have order 9 in the extended field of size $7^2$. It is not cubed because order 9 matrices have an eigenvector.
\end{proof}
% (Yasiru does this agree with the examples generated from your program?)


So far we know from Sylow's theorem that the normaliser $N_{\SL}(P)$ of a subgroup ($P$ of size 19) has size $3^2\times19$, which must contain $P$ together with the center of $\SL$ ($3\times 19$ elements). This means that the size of the normaliser in an subgroup $H$ containing $P$ will have size $19$, $3\times 19$ or $3^2\times19$ depending on whether it is divisible by 3 or 9. 

Questions: The size of normaliser is divisible by $9$, are there any elements of order 9 present?
What is the group of size 9 by 19 up to isomorphism? can we exhibit an isomorphism?
(I know what it should be up to isomorphism)

The other possible values for the number of Sylow $19$-subgroups % ($2^5\times3$, $7^3$ and $2^4 \times3^2 \times7$) 
from Lemma \ref{PossibleSubgroupSize} then give us the possible sizes of subgroups that contain $P$, once mutliplied by the size of the normaliser (orbit stabiliser theorem, with $N_{\SL}(P)$ the stabiliser). That is, possible maximal subgroup sizes are %$2^5\times3\times19$,$2^5\times3^2\times19$,$2^5\times3^3\times19$, $7^3\times 19$, $7^3\times 3\times 19$, $7^3\times 3^2\times 19$ and $2^4 \times3^2 \times7\times 19$,$2^4 \times3^3 \times7\times 19$,$2^4 \times3^4 \times7\times 19$. We eliminate the last one because it does not divide the size of $\SL$. We now have

$2^5\times3\times19$,$2^5\times3^2\times19$,$2^5\times3^3\times19$, $7^3\times 19$, $7^3\times 3\times 19$, $7^3\times 3^2\times 19$ and $2^4 \times3^2 \times7\times 19$,$2^4 \times3^3 \times7\times 19$

\begin{tabular}{c|c|c|c|c|c|c|c}
Lemma\ref{PossibleSubgroupSize}  &$1$&$2^5\times3$&$7^3$& $ 2^4 \cdot 3^2 \cdot 7$&$ 2^3 \cdot 3^3 \cdot 7^2$& $ 2^5 \cdot 3 \cdot 7^3$& \\\hline
$\times 19$&$19$&$2^5\times3\times19$&$7^3\times19$& $ 2^4 \cdot 3^2 \cdot 7\times19$&$ 2^3 \cdot 3^3 \cdot 7^2\times19$& $ 2^5 \cdot 3 \cdot 7^3\times19$&\\
times 3 times 19&$3\times19$&$2^5\times3^2\times19$&$3\times7^3\times19$& $ 2^4 \cdot 3^3 \cdot 7\times19$&$ 2^3 \cdot 3^4 \cdot 7^2\times19$& $ 2^5 \cdot 3^2 \cdot 7^3\times19$&\\
$\times 3^2 \times 19$ &$3^2\times19$&$2^5\times3^3\times19$&$3^2\times7^3\times19$& $ 2^4 \cdot 3^4 \cdot 7\times19$&$ 2^3 \cdot 3^5 \cdot 7^2\times19$& $ 2^5 \cdot 3^3 \cdot 7^3\times19$&\\\hline
\end{tabular}
Removing some obvious inconsistencies gives 


\begin{tabular}{c|c|c|c|c|c|c|c}
%Lemma\ref{PossibleSubgroupSize}  &$1$&$2^5\times3$&$7^3$& $ 2^4 \cdot 3^2 \cdot 7$&$ 2^3 \cdot 3^3 \cdot 7^2$& $ 2^5 \cdot 3 \cdot 7^3$& status\\
$\times 19$&$19$&$2^5\times3\times19$&$7^3\times19$& $ 2^4 \cdot 3^2 \cdot 7\times19$&$ 2^3 \cdot 3^3 \cdot 7^2\times19$& $ 2^5 \cdot 3 \cdot 7^3\times19$&\\
$\times 3 \times 19$&$3\times19$&$2^5\times3^2\times19$&$3\times7^3\times19$& $ 2^4 \cdot 3^3 \cdot 7\times19$&$3^4$ too big& $ 2^5 \cdot 3^2 \cdot 7^3\times19$&\\
$\times 3^2 \times 19$ &$3^2\times19$&$2^5\times3^3\times19$&$3^2\times7^3\times19$& $3^4$ too big&$3^5$ too big & whole group&\\
\end{tabular}

{\bf mini questions:} is there a subgroup of index 3? Or index 9?




\begin{lem} If $H$ is a subgroup containing $P$, then the size of $H$ is one of (the entries of the table above, look there first)
$$19,  2^5\cdot 3\cdot 19, 7^3\cdot 19, 2^4\cdot 3^2\cdot 7\cdot 19, 2^3\cdot 3^3\cdot 7^2\cdot 19,2^5\cdot 3\cdot7^3\cdot19,$$
if stabiliser is 19,

$$3\cdot 19,2^5\times3^2\times19,7^3\times 3\times 19, 2^4 \times3^3 \times7\times 19,$$
if stabiliser is $3\times19$

[next one needs to be multiplied by 9]
$$19,  2^5\cdot 3\cdot 19, 7^3\cdot 19, 2^4\cdot 3^2\cdot 7\cdot 19, 2^3\cdot 3^3\cdot 7^2\cdot 19,2^5\cdot 3\cdot7^3\cdot19,$$

\end{lem}
\begin{proof}
The size of $H$ divides the size $\SL$. The number of Sylow-19 subgroups of $H$ is 1 modulo 19 and divides $\SL$, by Lemma \ref{PossibleSubgroupSize} must be one of 
 $$ \{1, \  2^5 \cdot 3, \ 7^3, \ 2^4 \cdot 3^2 \cdot 7, \ 2^3 \cdot 3^3 \cdot 7^2, \ 2^5 \cdot 3 \cdot 7^3       \}.$$
To get the size of $H$ from this we use the orbit stabiliser theorem for the conjugation action on the Sylow-19 subgroups of $H$. The stabiliser is the normaliser of $P$ in $H$, which has size $19$ or $57$. There is a single orbit by Sylow's theorems.
\end{proof}


\section{Consequences for $\PSL$}

The 18 conjugacy classes in $\SL$ collapse to 6 conjugacy classes once the scaling by 2 and 4 are identified.

We denote these classes using square notation on the class with order 19

$$\mbf{[3,4]}, \mbf{[1,3]},\mbf{[2,0]},\mbf{[4,3]},\mbf{[3,1]},\mbf{[0,2]}$$



The remaining 6  classes are still related by the same bijections before (state them again)

\begin{thm}
$$\mbf{[3,4]}\to\mbf{[1,3]}\to\mbf{[2,0]}\to\mbf{[4,3]}\to\mbf{[3,1]}\to\mbf{[0,2]}$$
$$M\mapsto M^2$$
$$M\mapsto M^3$$
$$M\mapsto M^{14}$$
The inverses of these generators, 
$$M\mapsto M^{10}$$
$$M\mapsto M^{13}$$
$$M\mapsto M^{15}$$
traverse these sets in the opposite direction.

Since $7=2^6$, we get for all valid $i,j$, bijections
$$\mbf{[i,j]}\to \mbf{[i,j]}$$
$$M\mapsto M^7$$
and its inverse
$$M\mapsto M^{11}=M^{7^2}.$$


9,4,6 have two three cycles in forward direction, their inverses 17, 5, 16 in the reverse direction
$$\mbf{[3,4]}\to\mbf{[2,0]}\to\mbf{[3,1]}$$
$$\mbf{[1,3]}\to\mbf{[4,3]}\to\mbf{[0,2]}$$

8 have three pairs with inverse 12, as well as 18 which is its own inverse
$$\mbf{[i,j]}\to\mbf{[j,i]}$$
\end{thm}

\newpage
\section{Expository and explanatory notes}
We include in this section omitted details, further explanations and supplementary arguments in parallel with the main article.

\subsection{Group Sizes}
\begin{lem}\label{first_col_SL_anything}
Let $\vv{v} \neq \vv{0}$ be a vector in $\F_7^n$ where $n \geq 2$ is a positive integer. Then there exists a matrix $M \in \GenSL$ with its first column being $\vv{v}$.
\end{lem}
\begin{proof}
We can extend $\vv{v}$ to the basis $\{ \vv{v_1}, \vv{v_2}, \ldots, \vv{v_n} \}$ of $\F_7^n$ where $\vv{v} = \vv{v_1}$. Then the matrix $A$ with columns $\vv{v_1}, \vv{v_2}, \ldots, \vv{v_n}$ going left to right is invertible and suppose that $\det(A) = k \neq 0$. Thus, if we consider the matrix $M$ which is formed by changing the second column from $\vv{v_2}$ to $\frac{\vv{v_2}}{k}$, then 
$$
\det(M) = \frac{1}{k} \cdot \det(A) = \frac{1}{k} \cdot k = 1.
$$
Hence, we have found the required $M$.
\end{proof}

\begin{lem}\label{GL2F7_size}
The size of $\GLTwo$ is $(7^2-1)(7^2-7)$.
\end{lem}
\begin{proof}
To obtain a matrix $M \in \GLTwo$, the first column $\vv{v_1}$ must be a non-zero vector in $\F_7^2$, giving us $7^2-1$ options. The second column $\vv{v_2}$ must be linearly independent to $\vv{v_1}$. There are $7$ vectors linearly dependent to $\vv{v_1}$, (which are of the form $c \vv{v_1}$ for all $0 \leq c \leq 6$). Therefore, there are $7^2-7$ options for $\vv{v_2}$. Multiplying these numbers of options gives us ${(7^2-1)(7^2-7)}$ invertible matrices $M$, and so this is the size of $\GLTwo$.
\end{proof}

\begin{lem}\label{GL3F7_size}
The size of $\GL$ is $(7^3-1)(7^3-7)(7^3 - 7^2)$.
\end{lem}
\begin{proof}
We use a similar approach as in the proof of Lemma \ref{GL2F7_size}. To obtain a matrix $M \in \GL$, the first column $\vv{v_1}$ must be a non-zero vector in $\F_7^3$, giving us $7^3-1$ options. The second column $\vv{v_2}$ must be linearly independent to $\vv{v_1}$. Since there are $7$ vectors linearly dependent to $\vv{v_1}$, (which are of the form $c \vv{v_1}$ for $c=0, 1 \ldots, 6$), then there are $7^3-7$ options for $\vv{v_2}$. \\

The third column $\vv{v_3}$ must be linearly independent to $\vv{v_1}$ and $\vv{v_2}$. Since $\vv{v_1}$ and $\vv{v_2}$ are themselves independent, then there are $7^2$ vectors linearly dependent to $\vv{v_1}$ and $\vv{v_2}$, (which are of the form $c \vv{v_1} + d \vv{v_2}$ for all $0 \leq c, d \leq 6$). Thus, there are $7^3-7^2$ options for $\vv{v_3}$. Multiplying these numbers of options gives us ${(7^3-1)(7^3-7)(7^3-7^2)}$ invertible matrices $M$, and so this is the size of $\GL$.
\end{proof}

\begin{lem}\label{SL_size_from_GL}
For all positive integers $n \geq 2$ we have $|\GenSL| = \frac{1}{6} \cdot |\GenGL|$.
\end{lem}
\begin{proof}
We can partition $\GenGL$ into equivalence classes each of size $6$ where ${M \sim N}$ if and only if the first column $\vv{v}$ of $M$ is of the form $c \vv{w}$ for some $1 \leq c \leq 6$ where $\vv{w}$ is the first column of $N$. \\

Consider an equivalence class $A$. Note that by definition of our equivalence classes, the determinants of the matrices in $A$ will be of the form $k \times \det(M)$ for all $1 \leq k \leq 6$ and where $M$ is one representative of $A$. Since $M \in \GenGL$, then $\det(M) \neq 0$. Hence, exactly one matrix in $A$ will have determinant one (corresponding to $k = \det(M)^{-1}$). \\

If we apply this to all equivalence classes, we obtain that one sixth of the elements of $|\GenGL|$ are in $|\GenSL|$, so the result follows.
\end{proof}

\begin{lem}\label{SL2_size}
The size of $\SLTwo$ is $(7^2-1) \cdot 7$.
\end{lem}
\begin{proof}
We present two proofs. The first proof is an immediate application of Lemmas \ref{GL2F7_size} and \ref{SL_size_from_GL} where we have 
$$
|\SLTwo| = \frac{1}{6} \cdot |\GLTwo| = \frac{1}{6} \cdot (7^2-1)(7^2-7) = (7^2-1) \cdot 7.
$$
For the second proof, we will apply the orbit-stabiliser theorem. Let $A$ be the set of all non-zero vectors $\vv{v} \in \F_7^2$. Then ${|B| = 7^2-1}$. Let 
$$
\vv{u} = \vectwo{1}{0} \in B.
$$
Note that $B$ is a G-set under the group action of left multiplication by any ${M \in \SLTwo}$. For any ${\vv{w} \in B}$, by Lemma \ref{first_col_SL_anything}, we can choose a matrix ${M \in \SLTwo}$ with its first column equal to $\vv{w}$, so we will have ${A \vv{u} = \vv{w}}$. \\

Thus, the size of the orbit of $\vv{u}$ under the group action is ${|B| = 7^2-1}$. Let $G_{\vv{u}}$ be the set of matrices in $\SLTwo$ which stabilise $\vv{u}$, By the orbit stabiliser theorem we must have ${|\SLTwo| = |G_{\vv{u}}||B|}$. \\

Since for all $A \in \SLTwo$ we have $A \vv{u} = \vv{w}$ where $\vv{w}$ is the first column of $A$, then the matrices in $G_{\vv{u}}$ must be of the form
$$
M = \MatrTwo{1}{j}{0}{k} \in \SLTwo.
$$
Since we have
$$
\det(M) = 1 \cdot k - 0 \cdot j = k,
$$
then $M \in \SLTwo$ if and only if $k = 1$. Then we have $7$ possible choices for $j$, so ${|G_{\vv{u}}| = 7}$. This gives us 
$$
|\SLTwo| = |G_{\vv{u}}||B| = 7 \cdot (7^2 - 1).
$$
\end{proof}

\begin{prop} The size of $\SL$ is
$$\frac{1}{6} \cdot (7^3-1)(7^3-7)(7^3-7^2) = 2^5\times3^3\times 7^3\times 19.$$

\end{prop}
\begin{proof}
Like before, we present two proofs. The first proof is an immediate application of Lemmas \ref{GL3F7_size} and \ref{SL_size_from_GL} where we have
$$
|\SL| = \frac{1}{6} \times |\GL| = \frac{1}{6} \cdot (7^3-1)(7^3-7) (7^3-7^2).
$$
For the second proof, again we will apply the orbit-stabiliser theorem. Let $B$ be the set of all non-zero vectors $\vv{v} \in \F_7^3$. Then ${|B| = 7^3-1}$. Let
$$
\vv{u} = \vecthree{1}{0}{0} \in B.
$$
Note that $B$ is a G-set under the group action of left multiplication by any ${M \in \SL}$. For any ${\vv{w} \in B}$, by Lemma \ref{first_col_SL_anything}, we can choose a matrix ${M \in \SL}$ with its first column equal to $\vv{w}$, so we will have ${A \vv{u} = \vv{w}}$. \\

Thus, the size of the orbit of $\vv{u}$ under the group action is ${|B| = 7^3-1}$. Let $G_{\vv{u}}$ be the set of matrices in $\SL$ which stabilise $\vv{u}$, By the orbit stabiliser theorem we must have ${|\SL| = |G_{\vv{u}}||B|}$. \\

Since for all $A \in \SL$ we have $A \vv{u} = \vv{w}$ where $\vv{w}$ is the first column of $A$, then the matrices in $G_{\vv{u}}$ must be of the form
$$
M = \MatrThree{1}{b}{c}{0}{e}{f}{0}{h}{i}.
$$
Since we have
$$
\det(M) = ei-fh,
$$
then $M \in \SL$ if and only if $ei-fh = 1$, which is equivalent to the sub-matrix
$$
N = \MatrTwo{e}{f}{h}{i}
$$
being in $\SLTwo$. By Lemma \ref{SL2_size}, we have ${|\SLTwo| = (7^2-1) \cdot 7}$ possible choices for ${e, f, h, i}$ and a further $7^2$ choices for $b, c$. Therefore, we have 
$$
|G_{\vv{u}}| = 7^3 \cdot (7^2 - 1).
$$
This finally gives us
$$
|\SL| = |G_{\vv{u}}||B| = 7^3 \cdot (7^2-1)(7^3-1) = 7^2 \cdot (7^3-7)(7^3-1) = \frac{1}{6} \cdot (7^3-1)(7^3-7) (7^3-7^2).
$$
\end{proof}

\subsection{Irreducible characteristic polynomials}

Consider the matrix 
$$
M = \MatrThree{a}{b}{c}{d}{e}{f}{g}{h}{i} \in \SL.
$$
To find the eigenvalues of $M$, we solve the equation $\mathrm{det}(M - \lambda I) = 0$. Since
$$
M = \MatrThree{a-\lambda}{b}{c}{d}{e - \lambda}{f}{g}{h}{i-\lambda}
$$
then the determinant equation becomes 
$$(a-\lambda)(e-\lambda)(i-\lambda) + bfg + cdh - c(e-\lambda)g - fh(a-\lambda) - bd(i-\lambda) = 0.$$ 
This expands as 
$$ aei+bfg + cdh-ceg-afh-bdi + (-1)\lambda^3 + (a+e+i)\lambda^2 + (-ae-ei-ai+cg+fh+bd)\lambda = 0.$$
Also, since
$$
aei + bfg + cdh - ceg - afh - bdi = \det(M) = 1
$$
then the equation rearranges to the characteristic polynomial equation for $M$ of
$$
\lambda^3 - (a+e+i)\lambda^2 + (ae+ei+ai-cg-bd-fh)\lambda -1 = 0.
$$
We can alternatively use the fact ${\tr(M) = a+e+i}$ to write the equation as
$$
\lambda^3 - \tr(M) \lambda^2 + (ae+ei+ai-cg-bd-fh)\lambda -1 = 0.
$$

In order for $M$ to have no eigenvalues, this equation cannot hold for any integer $1 \leq \lambda \leq 6$ (we already know that $\lambda$ cannot be zero because $M$ is invertible). By substituting each value of $\lambda$ into the equation, this gives us the six equations in the introduction that must not hold in order for $M$ to have no eigenvalues.

\subsection{Simple Properties of the Group}
\begin{lem}\label{DoublingAndQuadruplingStillInSL}
Suppose that $M \in \SL$. Then $2M$ and $4M$ are also in $\SL$.
\end{lem}
\begin{proof}
Since $M \in \SL$, then $\det(M) = 1$. We then have
$$
\det(2M) = 2^3 \cdot \det(M) = 8 \cdot 1 = 1
$$
using $\F_7$ arithmetic. Similarly, we have
$$
\det(4M) = 4^3 \cdot \det(M) = 64 \cdot 1 = 1.
$$
Hence, $2M$ and $4M$ are in $\SL$.
\end{proof}

\begin{lem}\label{OrderOfDoubleAndQuadrupleIs57}
Suppose that $M \in \SL$ and $M$ has order $19$. Then $2M$ and $4M$ have order $57$.
\end{lem}
\begin{proof}
We have $M^{19} = I$. Since $M$ has order $19$, then using $\F_7$ arithmetic we have
$$(2M)^{57} = 2^{57} \cdot (M^{19})^3 = (2^3)^{19} \cdot I = I.$$ 
Thus, the order of $2M$ must be a factor of $57$. Note that 
$$(2M)^{19} = 2^{19} \cdot M^{19} = (2^3)^6 \cdot 2 \cdot I = 2I \neq I.$$
Also, 
$$(2M)^3 = 2^3 \cdot M^3 = M^3 \neq I.$$ 
because $M$ has order $19$. Hence, $2M$ has order $57$. We perform similar computations for the matrix $4M$. We have
$$(4M)^{57} = 2^{2 \cdot 57} \cdot (M^{19})^3 = (2^3)^{38} \cdot I = I.$$ 
This means that the order of $4M$ must be a factor of $57$. Note that 
$$(4M)^{19} = 2^{38} \cdot M^{19} = (2^3)^{12} \cdot 4 \cdot I = 4I \neq I.$$ 
Also, 
$$(4M)^3 = (2^3)^2 \cdot M^3 = M^3 \neq I$$ 
because $M$ has order $19$. Hence, $4M$ has order $57$. 
\end{proof}

\begin{lem}\label{2MAnd4MNoEigenvaluesIfMDoesNotHave}
Suppose that $M \in \SL$ and $M$ has no eigenvalues in $\F_7$. Then $2M$ and $4M$ have no eigenvalues in $\F_7$.
\end{lem}
\begin{proof}
Let $t = 2$ or $4$. Suppose for the sake of contradiction that $tM \vv{v} = \lambda \vv{v}$ for some eigenvector $\vv{v} \neq \vv{0}$ in $\F_7^3$ and some eigenvalue $\lambda \in \F_7$. Then $M \vv{v} = \frac{\lambda}{t} \vv{v}$ so $M$ has eigenvalue $\frac{\lambda}{t} \in \F_7$, which is a contradiction.
\end{proof}

\subsection{Preliminary Results on sets of the form [i, j]}
\begin{lem}\label{M2MBijection}
Suppose that $A = \mbf{[i, j]} \subseteq \SL$ and no matrices in $A$ have any eigenvalues in $\F_7$. Then the set $B = \mbf{[2i, 4j]} \subseteq \SL$ has the same size as $A$ and no matrices in $B$ have any eigenvalues in $\F_7$. Also, there is a bijection from $A$ to $B$ defined by $M \mapsto 2M$.
\end{lem}
\begin{proof}
Let 
$$C = \{ 2M \mid M \in A \} \subseteq \SL. $$
Consider any $M \in A$ where
$$
M = \MatrThree{a}{b}{c}{d}{e}{f}{g}{h}{i}.
$$ 
Let $p(\lambda)$ and $q(\lambda)$ be the characteristic polynomials of $M$ and $2M$ respectively. Since the coefficient of $\lambda^2$ in $p$ is $-\tr(M)$, then similarly the coefficient of $\lambda^2$ in $q$ is $-\tr(2M) = -2 \cdot \tr(M)$. The coefficient of $\lambda$ in $p$ is $ae+ie+ai-cg-bd-fh$. Note that in $2M$, each variable in the previous expression doubles so the coefficient of $\lambda$ in $q$ is four times that of $p$. Thus, $2M \in B$, so $C \subseteq B$. We also get a bijective mapping from $A$ to $C$ defined by $M \mapsto 2M$. \\

Let 
$$
D = \{ 4M \mid M \in B \} \subseteq \SL.
$$
Consider any $M \in B$. Define $p(\lambda)$ as earlier and let $r(\lambda)$ be the characteristic polynomial of $4M$. Using similar reasoning as before, the coefficient of $\lambda^2$ in $r$ is four times (or half times in $\F_7$) that of $p$ and the coefficient of $\lambda$ in $r$ is $4^2 = 2 = \frac{1}{4}$ times that of $p$. Hence, $4M \in A$ and so $D \subseteq A$. We also get a bijective mapping from from $B$ to $D$ defined by $M \mapsto 4M$. \\

We have $|A| = |C|$, $|B| = |D|$, $C \subseteq B$ and $D \subseteq A$. Thus, $|C| \leq |B|$ and $|D| \leq |A|$, which forces $|A|=|B|=|C|=|D|$ and so there is a bijection from $A$ to $B$ defined by $M \mapsto 2M$. Using this bijection and  Lemma \ref{2MAnd4MNoEigenvaluesIfMDoesNotHave}, we find that none of the matrices in $B$ have any eigenvalues in $\F_7$.

\end{proof}

\begin{lem}\label{18PairsRepresentAllNonEigenMatrices}
The following $18$ sets represent all the matrices in $\SL$ with no eigenvalues in $\F_7$: $\mbf{[0, 1]}$, $\mbf{[0, 2]}$, $\mbf{[0, 4]}$, $\mbf{[1, 0]}$, $\mbf{[1, 3]}$, $\mbf{[1, 5]}$, $\mbf{[2, 0]}$, $\mbf{[2, 5]}$, $\mbf{[2, 6]}$, $\mbf{[3, 1]}$, $\mbf{[3, 4]}$, $\mbf{[4, 0]}$, $\mbf{[4, 3]}$, $\mbf{[4, 6]}$, $\mbf{[5, 1]}$, $\mbf{[5, 2]}$, $\mbf{[6, 2]}$ and $\mbf{[6, 4]}$.
\end{lem}
\begin{proof}
Recall that $\mbf{[i, j]}$ is defined as the set of all matrices in $\SL$ with the characteristic polynomial $\lambda^3-i\lambda^2+j\lambda-1$ where $i, j \in \F_7$. We found earlier that all the trace $0$ matrices in $\SL$ with no eigenvalues in $\F_7$ (where $i = 0$) are the sets $\mbf{[0,1]}$, $\mbf{[0, 2]}$ and $\mbf{[0, 4]}$. Also, we showed earlier that all such trace $1$ matrices (where $i = 1$) are the sets $\mbf{[1,0]}$, $\mbf{[1,3]}$ and $\mbf{[1, 5]}$. Furthermore, all such trace $-1$ matrices (where $i = -1$) were found to be the sets $\mbf{[-1, 2]}$ and $\mbf{[-1, 4]}$. \\

Note that $\lambda$ is a root of the polynomial $\lambda^3 - i\lambda^2 + j\lambda -1$ over $\F_7$ if and only if $2\lambda$ is a root of the polynomial $\mu^3-2i\mu^2 + 4j\mu -1$ because 
$$(2\lambda)^3-(2i)(2\lambda)^2+(4j)(2\lambda)-1=\lambda^3-i\lambda^2+j\lambda-1$$ 
using arithmetic in $\F_7$. Therefore, the set $\mbf{[i, j]}$ consists only of matrices with no eigenvalues in $\F_7$ if and only if the same is true for $\mbf{[2i, 4j]}$ and for $\mbf{[\frac{i}{2}, \frac{j}{4}]}$ = $\mbf{[4i, 2j]}$. \\

Hence, a set of the form $\mbf{[2, j]}$ having matrices all without eigenvalues in $\F_7$ is equivalent to the same being true for $\mbf{[1, 2j]}$. This forces $j = 0, \ 5$ or $6$, giving us the sets $\mbf{[2,0]}$, $\mbf{[2,5]}$ and $\mbf{[2,6]}$. For sets of the form $\mbf{[3,j]}$ we can equivalently look at $\mbf{[6,4j]}$, which forces $j=1$ or $4$, giving us the sets $\mbf{[3,1]}$ and $\mbf{[3,4]}$. For sets of the form $\mbf{[4,j]}$ we can equivalently look at $\mbf{[1,4j]}$, which forces $j=0, 3$ or $6$, giving us the sets $\mbf{[4,0]}$, $\mbf{[4,3]}$ and $\mbf{[4,6]}$. For sets of the form $\mbf{[5,j]}$ we can equivalently look at $\mbf{[-1, 2j]}$, which forces $j=1$ or $2$, giving us the sets $\mbf{[5,2]}$ and $\mbf{[5,4]}$. \\

Thus, these $18$ sets consist of all the matrices in $\SL$ without eigenvalues in $\F_7$.
\end{proof}


%\textbf{VERY IMPORTANT RECENTLY DISCOVERED LEMMA December 20 -------- all pairs (trace, lambda) must have the same order 19 or 57 by diagonalising in the larger field, use characteristic polynomial for lambda and prove that $\lambda^{19}$ = 1 or $\lambda^{57} = 1$}

%\textbf{Amazing discovery! The bijection lambda -> -1 / lambda could be a bijection between the conjugacy classes???? Because they swap the trace and lambda coefficient values???}


\subsection{Eigenvalues}

\begin{lem}\label{noRepeatRoot}
Let $p$ be a monic cubic polynomial with coefficients in $\mathbb{F}_7$ which is irreducible over $\mathbb{F}_7$. Then $p$ has no repeated roots in $\mathbb{F}_{7^3}$.
\end{lem}
\begin{proof}
Suppose that there exists such a $p$ which has a repeated root $\alpha \in \mathbb{F}_{7^3}$. Since $p$ is irreducible, then it must be the minimal polynomial of $\alpha$. The derivative $p'$ will also have the root $\alpha$. But $\deg(p') < \deg(p)$, which contradicts the minimality of $\deg(p)$.
\end{proof}

\begin{lem}\label{EigenVectorWhenExtended}
Let $M \in \SL$ such that it has no eigenvalues in $\mathbb{F}_7$. Then $M$ must have three distinct eigenvalues in $\mathbb{F}_{7^3}$ which are of the form $\lambda$, \ $\lambda^7$ and $\lambda^{49}$ for some $\lambda \in \F_{7^3}$. 
\end{lem}
\begin{proof}
The characteristic polynomial $p(\lambda)$ of $M$ is monic and cubic in $\lambda$ with no roots in $\F_7$. Then $p$ must have no linear factors with coefficients in $\F_7$ so $p$ is irreducible over $\F_7$. By Lemma \ref{noRepeatRoot}, $p$ has no repeated roots in $\F_{7^3}$. Also, $p$ must split into linear factors over $\F_{7^3}$, so $p$ must have three distinct roots (which are eigenvalues) in $\F_{7^3}$. Since $p$ is monic and irreducible over $\F_7$, its three distinct roots are of the form $\lambda$, $\lambda^7$ and $\lambda^{7^2}=\lambda^{49}$ for some $\lambda \in \F_{7^3}$.
\end{proof}

\begin{lem}\label{ElementLargerFieldOrderFixedCoprime}
Consider a $\lambda \in \F_{7^3}$ with order $n \geq 1$. If $k$ is a positive integer such that $\gcd(k, n)=1$, then $\lambda^k$ also has order $n$.
\end{lem}
\begin{proof}
Let $t$ be the order of $\lambda^k$. Since 
$$(\lambda^k)^n=\lambda^{kn}=(\lambda^n)^k=1^k=1,$$ 
then $t$ divides $n$. Also, since
$$1=(\lambda^k)^t=\lambda^{kt}$$ 
then $kt$ is a multiple of $n$. Since $\gcd(k, n)=1$, then $t$ is a multiple of $n$, so we must have $t=n$.
\end{proof}

\begin{lem}\label{CommutingMatricesEigenspacesPreserved}
Suppose that $A$ and $B$ are commuting matrices in $\SL$ where $A$ has an eigenvalue $\lambda \in \F_7$ and corresponding eigenspace $E_\lambda$ with dimension one. Then $B$ will have an eigenvalue $\mu \in \F_7$ and corresponding eigenspace $E_\mu$ which satisfies $E_\lambda \subseteq E_\mu$.
\end{lem}
\begin{proof}
Suppose that $\vv{v} \neq \vv{0}$ is in $E_\lambda$. Then $A \vv{v}=\lambda \vv{v}$ and so 
$$A(B \vv{v})=(AB) \vv{v}=(BA) \vv{v}=B(A \vv{v})=B(\lambda \vv{v})=\lambda (B \vv{v}).$$ 
Since $B$ is invertible, then $B \vv{v} \neq \mbf{0}$. From above, we obtain $B \vv{v} \in E_\lambda$, so $B \vv{v} = \mu \vv{v}$ for some $\mu \in \F_7$ (since $E_\lambda$ has dimension one). Hence $\mu$ is an eigenvalue of $B$ and $\vv{v} \in E_\mu$. Since $E_\lambda$ has dimension one, then this gives us that $E_\lambda \subseteq E_\mu$.
\end{proof}

\begin{lem}\label{OrderNIffExtSame}
Consider an $M \in \SL$ with no eigenvalues in $\F_7$. Using Lemma \ref{EigenVectorWhenExtended}, consider the three distinct eigenvalues of $M$ in $\F_{7^3}$. Then $M$ has order $n$ if and only if all three of these eigenvalues have order $n$. Also, the order of $M$ must be a factor of $57$.
\end{lem}
\begin{proof}
Let $\lambda_1, \ \lambda_2$ and $\lambda_3$ be the three distinct eigenvalues of $M$ in $\F_{7^3}$. Then, $M$ can be diagonalised in the form $M = XDX^{-1}$ for some $X \in \GLLarge$ and where 
$$
D = \MatrThree{\lambda_1}{0}{0}{0}{\lambda_2}{0}{0}{0}{\lambda_3} \in \GLLarge.
$$

Note that for all positive integers $k$, $M^k = XD^k X^{-1}$ where we have

$$D^k=\begin{pmatrix} 
{\lambda_1}^k&0&0\\
0&{\lambda_2}^k&0\\
0&0&{\lambda_3}^k\\
\end{pmatrix}
$$
% Need to check this part of the proof again, make sure that all three eigenvalues have the same order.

Now, $M^k = I$ is equivalent to $XD^k X^{-1} = I$, or $D^k = X^{-1} I X = I$. Also, $D^k=I$ is equivalent to the orders of all three eigenvalues being a factor of $k$. Let the orders of $\lambda_1$, $\lambda_2$ and $\lambda_3$ be $n_1$, $n_2$ and $n_3$ respectively. Then the smallest value of $k$ such that $D^k=I$ will be $\lcm(n_1, n_2, n_3)$, and this will be the order of $M$.

Next, note that

$$\det(D)=\det(X) \cdot \det(D) \cdot \frac{1}{\det(X)} = \det(XDX^{-1})=\det(M)=1.$$ 

By Lemma \ref{EigenVectorWhenExtended}, we obtain that $(\lambda_1, \lambda_2, \lambda_3)$ is a permutation of $(\lambda, \lambda^7, \lambda^{49})$ for some $\lambda \in \F_{7^3}$. Since $\det(D)=1$, then 
$$\lambda \cdot \lambda^7 \cdot \lambda^{49} = \lambda^{57} = 1.$$
Thus, we also have, $(\lambda^7)^{57} = (\lambda^{49})^{57} = 1$. Hence, $n_1$, $n_2$ and $n_3$ are factors of $57$ so the order of $M$ which equals $\lcm(n_1, n_2, n_3)$ will also be a factor of $57$. This proves the second statement. \\

To prove the first statement, it remains to show that ${n_1=n_2=n_3}$ (since then we would have ${\lcm(n_1, n_2, n_3)=n_1=n_2=n_3}$). Since $\lambda$ has an order $m$ that is a factor of $57$, then we have ${\gcd(m, 7) = \gcd(m, 49) = 1}$. From Lemma \ref{ElementLargerFieldOrderFixedCoprime}, the orders of $\lambda^7$ and $\lambda^{49}$ must also be $m$, so we are done.
\end{proof}

\subsection{Matrix Powers and Orders}
\begin{lem}\label{orderPowerCoprimeFixed}
Consider a $M \in \SL$ with order $n \geq 1$. If $k$ is a positive integer such that $\gcd(k,n)=1$ then $M^k$ also has order $n$.
\end{lem}
\begin{proof}
Let $t$ be the order of $M^k$. Since $M^{kn}=(M^k)^n=I^n=I$, then $t$ divides $n$. Also $I=(M^k)^t=M^{kt}$, so $kt$ is a multiple of $n$. Since $\gcd(k,n)=1$, then $t$ is a multiple of $n$, so we must have $t=n$. 
\end{proof}

\begin{lem}\label{EigenvalueScale}
Consider a $M \in \SL$ with eigenvector $\vv{v} \neq \vv{0}$ in $\F_{7^3}^3$ and corresponding eigenvalue $\lambda \in \F_{7^3}$. Then for any positive integer $k$, the matrix $M^k$ has eigenvector $\vv{v}$ and corresponding eigenvalue $\lambda^k$.
\end{lem}
\begin{proof}
We will use induction.  Suppose that $M^t \vv{v} = \lambda^t \vv{v}$ for some positive integer $t$ (this is true for the base case of $t = 1$). Then 
$$M^{t+1} \vv{v} = M(M^t \vv{v}) = M(\lambda^t \vv{v}) = \lambda^t (M \vv{v}) = \lambda^{t+1} \vv{v}.$$ 
Hence by induction, $M^k \vv{v} = \lambda^k \vv{v}$ for all positive integers $k$.
\end{proof}

\begin{lem}\label{EigenCoprime}
Consider a $M \in \SL$ with order $n \geq 2$ and no eigenvalues in $\F_7$. Then for all positive integers $k$ where $1 \leq k \leq n-1$ and $\gcd(k, n) = 1$, the matrix $M^k$ has no eigenvalues in $\F_7$. 
\end{lem}
\begin{proof}
Suppose that $M^k$ has an eigenvector $\vv{v} \neq \vv{0}$ in $\F_7^3$ and corresponding eigenvalue $\lambda \in \F_7$, Then, $M^k v = \lambda v$. Since $\gcd (k, n) = 1$ then there exists a positive integer $t$ such that $\ModCmp{kt}{1}{n}$. Thus, $kt = sn+1$ for some non-negative integer  $s$. By Lemma \ref{EigenvalueScale}, the matrix 
$$(M^k)^t = M^{sn+1} = (M^n)^s \cdot M = M$$ 
has $\vv{v}$ as an eigenvector, contradiction.
\end{proof}

\begin{lem}\label{PrimeNoEigen}
Consider a $M \in \SL$ with order $p$ where $p$ is prime and where $M$ has no eigenvalues in $\F_7$. Then all the matrices of the form $M^k$ for $1 \leq k \leq p-1$ have no eigenvalues in $\F_7$.

\end{lem}
\begin{proof}
Since $p$ is prime then $\gcd(p, k) = 1$ for $1 \leq k \leq p-1$ so the result follows from Lemma \ref{EigenCoprime}.
\end{proof}

\begin{lem}\label{PowerMatrixSetSameSizeCoprimeOrder}
Let $n \geq 2$ be a fixed positive integer and $A$ be a set of matrices in $\SL$, all of which have order $n$. For a fixed positive integer $k$ where $2 \leq k \leq n$ and $\gcd(k, n) = 1$, let 
$$
B = \{ M^k \mid M \in A \}.
$$
Then $|A| = |B|$.
\end{lem}
\begin{proof}
Since $\gcd(k, n) = 1$, there exists a positive integer $t$ such that ${\ModCmp{kt}{1}{n}}$. Thus, $kt = sn+1$ for some non-negative integer $s$. Since the mapping from $A$ to $B$ defined by $M \mapsto M^k$ is surjective by definition, then $|A| \geq |B|$. Consider an arbitrary $N = M^k \in B$ for some $M \in A$. Note that 
$$N^t = (M^k)^t = M^{sn+1} = (M^n)^s \cdot M = M \in A.$$ 
Thus, the mapping from $B$ to $A$ defined by $M \mapsto M^t$ is surjective, so $|A| \leq |B|$. Hence, $|A| = |B|$. 
\end{proof}

\begin{lem}\label{DifferentSubgroupsDisjoint}
Let $G$ and $H$ be different subgroups of $\SL$ with order $19$. Then, $G \cap H = I$.
\end{lem}
\begin{proof}
Since $G$ and $H$ both contain the identity matrix $I$, then $I \in G \cap H$. Assume for the sake of contradiction that for some non-identity matrix $A \in \SL$, we have $A \in G \cap H$. Since the order $19$ of $G$ and $H$ is a prime number, then both subgroups are cyclic and $A$ is a generator of both $G$ and $H$ (where $A$ has order $19$). Hence, we must have 
$$
G = H = \{ A^k \mid 0 \leq k \leq 18 \}
$$
which contradicts $G$ and $H$ being distinct.
\end{proof}

\begin{lem}\label{AllOrder19SubgroupsAreSylow}
The Sylow $19$-subgroups of $\SL$ are exactly all subgroups of $\SL$ with order $19$.
\end{lem}
\begin{proof}
Suppose that $G$ is a subgroup in $\SL$ with order $19$. Since $19$ is prime, then $G$ is a $19$-subgroup of $\SL$. Suppose that $G$ is a subgroup of some $19$-subgroup $H \subseteq \SL$. By Lagrange's Theorem, $|H|$ is a factor of $|\SL|$. \\

Since $|\SL|$ only contains one factor of $19$, then we must have ${|H| = 19}$. This implies that $G=H$. Hence, $G$ is a maximal $19$-subgroup of $\SL$ so it is a Sylow $19$-subgroup of $\SL$. This proves that all subgroups of $\SL$ with order $19$ are Sylow $19$-subgroups. \\

Since $|\SL|$ only contains one factor of $19$ then there are no Sylow $19$-subgroups of order $19^k$ for some $k \geq 2$, so this completes the proof.
\end{proof}

\begin{lem}\label{NoOrder3NonEigen}
No matrices in $\SL$ with no eigenvalues in $\F_7$ have order $1$ or $3$.
\end{lem}
\begin{proof}
The order one case is ruled out because $\lambda=1$ is an eigenvalue of the identity matrix. Suppose that some $M \in \SL$ with no eigenvalues in $\F_7$ has order $3$. Then by Lemma \ref{OrderNIffExtSame}, $M$ will have three distinct eigenvalues in $\F_{7^3}$, all with order $3$. The eigenvalues will be the roots of the equation 
$$
\lambda^3-1=0.
$$
However, the three roots are $\lambda = 1, 2$ and $4$, which is a contradiction since all the roots are all in $\F_7$.
\end{proof}

\subsection{Conjugacy Classes Preliminary Results}

\begin{lem}\label{ConjClassTrace}
Let $K$ be a conjugacy class in $\SL$. Then all matrices in $K$ have the same trace.
\end{lem}
\begin{proof}
Let $A, \ B \in K$. Then there exists a $X \in \SL$ such that $B = XAX^{-1}$. Since $\tr(NM) = \tr(MN)$ for all matrices $N,M \in \SL$, then 
$$\tr(B) = \tr((XAX^{-1}) = \tr((XA)X^{-1})= \tr(X^{-1}(XA)) = \tr((X^{-1}X)A)= \tr(A).$$
\end{proof}

\begin{lem}\label{SameConjOrder}
Let $K$ be a conjugacy class in $\SL$. Then all the matrices in $K$ have the same order.
\end{lem}
\begin{proof}
Let $M$ be a fixed matrix in $K$. Since $\SL$ is finite, then $M$ has a finite order. Consider any $A \in K$. It suffices to prove that $A, M$ have the same order. We can write $A=XMX^{-1}$ for some $X \in \SL$. For all positive integers $k$, $A^k = XM^k X^{-1}$. Then $A^k = I$ if and only if $I = XM^k X^{-1}$. This is equivalent to
$$
M^k = X^{-1} I X = I.
$$
Hence, $A$ and $M$ have the same order.
\end{proof}

\begin{lem}\label{SameConjEigen}
Let $K$ is a conjugacy class in $\SL$. Then all matrices in $K$ have the same set of eigenvalues in $\mathbb{F}_{7^3}$.
\end{lem}
\begin{proof}
Consider two matrices $A, B \in K$. Suppose that $\vv{v} \neq \vv{0}$ is an eigenvector of $A$ in $\F_{7^3}^3$ with corresponding eigenvalue $\lambda \in \F_{7^3}$. Then we have $M \vv{v} = \lambda \vv{v}$. We can write $A = XBX^{-1}$ for some $X \in \SL$. Thus, $(XBX^{-1}) \vv{v} = \lambda \vv{v}$, so ${B(X^{-1} \vv{v}) = \lambda (X^{-1} \vv{v})}$. Since $X^{-1}$ is invertible, then $X^{-1} \vv{v} \neq \vv{0}$, so $B$ also has the eigenvalue $\lambda$ (with corresponding eigenvector $X^{-1} \vv{v}$). Since $A, B$ are arbitrary elements of $K$, then this proves the result.
\end{proof}

\begin{lem}\label{NoEigenConjClass}
If a matrix $M \in \SL$ has no eigenvalues in $\F_7$, then none of the matrices in its conjugacy class have any eigenvalues in $\F_7$.
\end{lem}
\begin{proof}
This follows immediately from Lemma \ref{SameConjEigen}.
\end{proof}

\begin{lem}\label{SameConjCharacPoly}
Let $K$ be a conjugacy class in $\SL$ where none of the matrices in $K$ have eigenvalues in $\F_7$. Then all matrices $M \in K$ have the same characteristic polynomial.
\end{lem}
\begin{proof}
From Lemmas \ref{EigenVectorWhenExtended} and \ref{SameConjEigen}, all matrices $M \in K$ must have the characteristic polynomial of $p(\lambda) = (\lambda - \lambda_1)(\lambda - \lambda_2)(\lambda - \lambda_3)$ where $\lambda_1, \ \lambda_2, \ \lambda_3$ are the three distinct eigenvalues in $\F_{7^3}$ shared by all elements of $K$.
\end{proof}

\begin{lem}\label{ConjClassSizesAreSameDoubleAndQuadruple}
Let $K$ be a conjugacy class in $\SL$. Let 
\begin{align*}
    L &= \{ 2M \mid M \in K \} \subseteq \SL.\\
    T &= \{ 4M \mid M \in K \} \subseteq \SL.
\end{align*}
Then $L$ and $T$ are conjugacy classes in $\SL$ where $|K| = |L| = |T|$.
\end{lem}
\begin{proof}
The mapping $M \mapsto 2M$ defines a bijection from $K$ to $L$ so $|K| = |L|$. \\

Let $2A$ and $2B$ be arbitrary elements of $L$ where $A, B \in K$. Then, there exists a $C \in \SL$ such that $B = CAC^{-1}$. Thus, $2B = C(2A)C^{-1}$ so $2A$ and $2B$ are conjugates. Suppose that for a fixed $2A \in L$ (where $A \in K$) there exists a $D \in \SL$ such that $2A$ is a conjugate of $D$ but $D \notin L$. Then, there exists a $C \in \SL$ such that $D = C(2A)C^{-1}$. Thus, $4D = \frac{1}{2}D = CAC^{-1}$. Therefore, $4D$ and $A$ are conjugates so $4D \in K$. Hence, $8D = D \in L$, which is a contradiction. Hence, all conjugates of $2A$ must be in $L$, so $L$ is a conjugacy class. \\

Now all the elements in $T$ are of the form $2M$ where $M \in L$. The previous argument can be applied again to prove that $T$ is a conjugacy class in $\SL$ where $|L|=|T|$.
\end{proof}
\begin{lem}\label{PowerSameConjClass}
Let $M, N$ be matrices in $\SL$ that are conjugates. Then $M^k$ and $N^k$ are conjugates for all positive integers $k$.
\end{lem}
\begin{proof}
Since $M, N$ are conjugates then there exists an $X \in \SL$ such that $N = XMX^{-1}$. Then for all positive integers $k$, we have $N^k= XM^k X^{-1}$, so $M^k$ and $N^k$ are conjugates.
\end{proof}

\begin{lem}\label{OrbitStab}
Consider an $M \in \SL$. Let $G_M$ be the set of matrices in $\SL$ that commute with $M$. Then the size of the conjugacy class that contains $M$ is $|\SL|/|G_M|$.
\end{lem}
\begin{proof}
    Let $K$ be the conjugacy class that contains $M$. $K$ is a G-set under the group action of $X \cdot N = XNX^{-1}$ for all $X \in \SL$ and $N \in K$. By definition of $K$, the size of the orbit of $M$ under the group action is $|K|$. The matrices $X \in \SL $ that stabilise $M$ under the group action satisfy $XMX^{-1} = M$, which is equivalent to $XM = MX$ ($X$ commuting with $M$). The number of such matrices $X$ is $|G_M|$. By the orbit stabiliser theorem, $|\SL| = |G_M||K|$, so the result follows.
\end{proof}

\begin{lem}\label{PowerConjClassSameSize}
Let $K$ be a conjugacy class in $\SL$ where every $M \in K$ has order $n \geq 2$. For a fixed positive integer $k$ where $2 \leq k \leq n$ and $\gcd(k, n) = 1$, let 
$$
L = \{ M^k \mid M \in K \}.
$$
Then $L$ is a conjugacy class in $\SL$ where $|L| = |K|$.
\end{lem}
\begin{proof}
From Lemma \ref{PowerMatrixSetSameSizeCoprimeOrder}, $|K| = |L|$. From Lemma \ref{PowerSameConjClass}, any two elements of $|L|$ must be conjugates. Consider a fixed $A \in K$. Then $A^k \in L$. Suppose that there exists a $B \in \SL$ such that $B \notin L$ and $A^k$ and $B$ are conjugates. Then, there exists a $C \in \SL$ such that $B = CA^k C^{-1} = (CAC^{-1})^k$. Since $A \in K$, then $CAC^{-1} \in K$, so $B \in L$, which is a contradiction. Hence, $L$ contains all the conjugates of $A^k$, so $L$ is a conjugacy class.
\end{proof}

\subsection{Computations of Matrix Orders}

In the following proofs, we will use the property that
$$
(a+b)^7 = a^7+b^7
$$
for all $a, b \in \F_{7^3}$.

\begin{prop}[J.L.C]\label{43nt}
All matrices in $\mbf{[4, 3]}$ have order 19.
\end{prop}
\begin{proof} 
Since all matrices $M \in \mbf{[4, 3]}$ have no eigenvalues in $\F_7$, then by Lemma \ref{EigenVectorWhenExtended}, $M$ has three distinct eigenvalues in $\F_{7^3}$ and each eigenvalue is a root of the equation 
$$\lambda^3 - 4\lambda^2 + 3\lambda - 1 = 0.$$
By Lemma $\ref{OrderNIffExtSame}$ it suffices to prove that any root $\lambda$ of the above equation has order $19$. We will prove that $\lambda^{21} = \lambda^2$, which is equivalent to $\lambda^{19} = 1$. \\

We obtain that for all roots $\lambda$ of the cubic equation above, 
$$\lambda^3 = 4\lambda^2 - 3\lambda +1 = 4\lambda^2 + 4\lambda +1 = (2\lambda + 1)^2$$
using arithmetic in $\F_7^3$. Now we compute $\lambda^{21}$ below.


$$\begin{aligned}\lambda^{21}&=(\lambda^3)^7\\
                             &=(2\lambda+1)^{14}\\
                             &=((2\lambda+1)^7)^2\\
                             &=(2^7\lambda^7+1)^2\\
                             &=(2\lambda^7+1)^2\\
                             &=(1+2\lambda(\lambda^3)^2)^2\\
                             &=(1+2\lambda(2\lambda+1)^4)^2\\
                             &=(1+2\lambda(1+8\lambda+24\lambda^2+32\lambda^3+16\lambda^4))^2\\
                             &=(1+2\lambda(1+\lambda+3\lambda^2+4\lambda^3+2\lambda^4))^2\\
                             &=(1+2\lambda(1+\lambda+3\lambda^2+4(4\lambda^2-3\lambda+1)+2\lambda^4))^2\\
                             &=(1+2\lambda(5+3\lambda+5\lambda^2+2\lambda^4))^2\\
                             &=(1+\lambda(3-\lambda+3\lambda^2+4\lambda^4))^2\\
                             &=(1+\lambda(3-\lambda+3\lambda^2+4\lambda(4\lambda^2-3\lambda+1)))^2\\
                             &=(1+\lambda(3+3\lambda+5\lambda^2+2\lambda^3))^2\\
                             &=(1+\lambda(3+3\lambda+5\lambda^2+2(4\lambda^2-3\lambda+1)))^2\\
                             &=(1+\lambda(5+4\lambda+6\lambda^2))^2\\
                             &=(1+5\lambda+4\lambda^2-\lambda^3)^2\\
                             &=(1+5\lambda+4\lambda^2-(4\lambda^2-3\lambda+1))^2 \\
                             &=\lambda^2.\\
    \end{aligned}$$
    Hence, the order of all three eigenvalues is a factor of $19$. Since $19$ is prime and none of the eigenvalues are equal to $1$, then the three eigenvalues have order $19$.
\end{proof}

\begin{prop}[W.L]\label{31nt}
All matrices in $\mbf{[3, 1]}$ have order 19.
\end{prop}
\begin{proof}
Since all matrices $M \in \mbf{[3, 1]}$ have no eigenvalues in $\F_7$, then by Lemma \ref{EigenVectorWhenExtended}, $M$ has three distinct eigenvalues in $\F_{7^3}$ and each eigenvalue is a root of the equation 
$$\lambda^3 - 3\lambda^2 + \lambda - 1 = 0.$$

By Lemma $\ref{OrderNIffExtSame}$ it suffices to prove that any root $\lambda$ of the above equation has order $19$. Below we will compute powers of $\lambda$ to prove that $\lambda^{19} = 1$, using the relation 
$$\lambda^3 = 3\lambda^2-\lambda+1$$
which holds for all three roots $\lambda$ of the cubic equation. Below is the computation of $\lambda^4$.
 
$$\begin{aligned}\lambda^{4}&=3\lambda^{3}-\lambda^{2}+\lambda\\
&= 3(3\lambda^2-\lambda+1)-\lambda^2+\lambda \\
&=\lambda^{2}+5\lambda+3.
\end{aligned}$$
Similarly, we have:
$$
\begin{aligned}
    \lambda^{5} &=\lambda^{2}+2\lambda+1. \\
    \lambda^{6} &=5\lambda^{2}+1. \\
    % \lambda^{7} &=\lambda^{2}+3\lambda+5 \\
    % \lambda^{8} &=6\lambda^{2}+4\lambda+1 \\
    % \textbf{TO BE CONTINUED}
    % \lambda^{9} &=\lambda^{2}+2\lambda+6 \\
    % \lambda^{10} &=5\lambda^{2}+5\lambda+1 \\
    % \lambda^{11} &= 6\lambda^{2}+3\lambda+5 \\
    \lambda^{12} &= (\lambda^6)^2. \\
                 &= (5\lambda^{2}+1)^2\\
                 &=6\lambda+6. \\
    % \lambda^{13} &= 6\lambda^{2}+6\lambda \\
    % \lambda^{14} &=3\lambda^{2}+\lambda+6 \\
    % \lambda^{15} &=3\lambda^{2}+3\lambda+3 \\
    % \lambda^{16} &=5\lambda^{2}+3 \\
    % \lambda^{17} &=\lambda^{2}+5\lambda+5 \\
    \lambda^{18} &= \lambda^{12} \cdot \lambda^{6}.\\
                 &= (6\lambda + 6)(5\lambda^{2} + 1).\\
                 &= \lambda^{2}+4\lambda+1.\\
    \lambda^{19} &=1.\\
\end{aligned}
$$


Hence, the order of all three eigenvalues is a factor of $19$. Since $19$ is prime and none of the eigenvalues are equal to $1$, then the three eigenvalues have order $19$.
\end{proof}


\begin{prop}[L.L]\label{13nt}
All matrices in $\mbf{[1, 3]}$ have order 19.
\end{prop}

\begin{proof} 
Since all matrices $M \in \mbf{[1, 3]}$ have no eigenvalues in $\F_7$, then by Lemma \ref{EigenVectorWhenExtended}, $M$ has three distinct eigenvalues in $\F_{7^3}$ and each eigenvalue is a root of the equation 
$$\lambda^3 - \lambda^2 + 3\lambda - 1 = 0.$$

By Lemma \ref{OrderNIffExtSame} it suffices to prove that any root $\lambda$ of the above equation has order $19$. We will prove that $\lambda^{21} = \lambda^2$, which is equivalent to $\lambda^{19} = 1$. We will use the relation
$$\lambda^3 = \lambda^2 - 3\lambda +1.$$
Now we compute $\lambda^{21}$ below.
$$\begin{aligned}\lambda^{21}&=(\lambda^2-3\lambda+1)^7\\
&=\lambda^{14}-3^7\lambda^7+1\\
&=\lambda^{14}-3\lambda^7+1.\\
\end{aligned}$$
We also have
$$\begin{aligned}\lambda^6&=(\lambda^2-3\lambda+1)^2\\
&=\lambda^4+9\lambda^2+1-6\lambda^3+2\lambda^2-6\lambda \\
&= \lambda(\lambda^2-3\lambda+1) + 9\lambda^2+1-6\lambda^3+2\lambda^2-6\lambda \\
&= 2\lambda^3+\lambda^2+2\lambda+1 \\
&= 2(\lambda^2-3\lambda+1)+\lambda^2+2\lambda+1 \\
&=3(\lambda^2+\lambda+1).\\
\end{aligned}$$
This gives us
$$\begin{aligned}
\lambda^{14}&=\lambda^2(\lambda^6)^2\\
&=\lambda^2 \big(3(\lambda^2+\lambda+1) \big)^2\\
&=2\lambda^2(\lambda^4+\lambda^2+1+2\lambda^3+2\lambda^2+2\lambda)\\
&=2\lambda^6+4\lambda^5+6\lambda^4+4\lambda^3+2\lambda^2 \\
&=2(3)(\lambda^2+\lambda+1) + 4\lambda^2(\lambda^2-3\lambda+1)+6\lambda(\lambda^2-3\lambda+1)+4(\lambda^2-3\lambda+1)+2\lambda^2 \\
&=4\lambda^4-6\lambda^3-2\lambda^2+10\\
&=4\lambda(\lambda^2-3\lambda+1)-6(\lambda^2-3\lambda+1)-2\lambda^2+10\\
&=4\lambda^3-20\lambda^2+22\lambda+4\\
&=4(\lambda^2-3\lambda+1)-20\lambda^2+22\lambda+4\\
&=5\lambda^2+3\lambda+1\\
\end{aligned}$$
We now compute $-3\lambda^7$.
$$\begin{aligned}
-3\lambda^7&=-3\lambda(\lambda^6)\\
&=-3\lambda(3(\lambda^2+\lambda+1))\\
&=5\lambda^3+5\lambda^2+5\lambda\\
&=5(\lambda^2-3\lambda+1)+5\lambda^2+5\lambda\\
&=3\lambda^2-3\lambda-2\\
\end{aligned}$$
Combining the above equalities gives us
$$\begin{aligned}
\lambda^{21}&=\lambda^{14}-3\lambda^7+1\\
&=(5\lambda^2+3\lambda+1)+(3\lambda^2-3\lambda-2)+1\\
&=\lambda^2\\
\end{aligned}$$
Hence, the order of all three eigenvalues is a factor of $19$. Since $19$ is prime and none of the eigenvalues are equal to $1$, then the three eigenvalues have order $19$.
\end{proof}

\subsection{Further Results on sets of the form [i, j]}
\begin{lem}\label{OneOrderNThenAll}
Consider a set $\mbf{[i, j]} \subseteq \SL$ where no matrices in the set have any eigenvalues in $\F_7$. Then all the matrices in $\mbf{[i,j]}$ have the same order.
\end{lem}
\begin{proof}
Consider an arbitrary $M \in \mbf{[i, j]}$ with order $n$. Since $M$ has no eigenvalues in $\F_7$ then by Lemma \ref{OrderNIffExtSame}, the three distinct eigenvalues $\lambda_1, \lambda_2, \lambda_3$ of $M$ in $\F_{7^3}$ have order $n$. Consider any $N \in \mbf{[i, j]}$. Since $M$ and $N$ have the same characteristic polynomial, then $N$ also has the eigenvalues $\lambda_1, \lambda_2, \lambda_3$. By Lemma \ref{OrderNIffExtSame} again, $N$ also has order $n$.
\end{proof}

\begin{lem}\label{PowerConjClassSizeOrder19}
Consider a non-empty set ${A = \mbf{[i, j]} \subseteq \SL}$ where no matrices in the set have any eigenvalues in $\F_7$ and all matrices in the set have order $19$. For a fixed positive integer $k$ where $2 \leq k \leq 18$, let 
$$B = \{ M^k \mid M \in A \} \subseteq \SL.$$
Then $B$ is of the form $\mbf{[v,w]}$ for some $v$ and $w$, and none of the matrices in $B$ have any eigenvalues in $\F_7$. Also $|A| = |B|$ and all the matrices in $B$ have order $19$.
\end{lem}
\begin{proof}
Since $\gcd(k, 19) = 1$ and $k \neq 1$, then there exists a positive integer $t$ such that ${2 \leq t \leq 18}$ and ${\ModCmp{kt}{1}{19}}$. Thus, $kt = 19s+1$ for some non-negative integer $s$. \\

Consider any $M \in A$. Since $M$ has order $19$, then by Lemma \ref{orderPowerCoprimeFixed}, $M^k$ also has order $19$. By Lemma \ref{EigenCoprime}, none of the matrices in $B$ have any eigenvalues in $\F_7$. Since $M$ has no eigenvalues in $\F_7$, then by Lemma \ref{OrderNIffExtSame}, the characteristic polynomial for $M$ must be ${P(\lambda) = (\lambda - \lambda_1)(\lambda - \lambda_2)(\lambda - \lambda_3)}$ for distinct eigenvalues ${\lambda_1, \lambda_2, \lambda_3 \in \F_{7^3}}$ where the eigenvalues all have order $19$.\\

Note that 
$$({\lambda_1}^k)^t = {\lambda_1}^{19s+1} = \lambda_1.$$ 
Similarly, $({\lambda_2}^k)^t = \lambda_2$ and $({\lambda_3}^k)^t = \lambda_3$. Thus, since $\lambda_1, \ \lambda_2$ and $\lambda_3$ are distinct, then $\lambda_1^k, \ \lambda_2^k$ and $\lambda_3^k$ are distinct. Then by Lemma \ref{EigenvalueScale}, $\lambda_1^k, \ \lambda_2^k$ and $\lambda_3^k$ are the three distinct eigenvalues of $M^k$ where these eigenvalues are in $\F_{7^3}$ but not in $\F_7$ (because $M^k$ has no eigenvalues in $\F_7$). \\

Hence, the characteristic polynomial of $M^k$ must be $Q(\lambda) = (\lambda - \lambda_1^k)(\lambda - \lambda_2^k)(\lambda - \lambda_3^k)$. Note that all matrices in $A$ will have the same characteristic polynomial $P$ and since $Q$ is uniquely determined by $P$, then all matrices in $B$ will also have the same characteristic polynomial $Q$, which has no roots in $\F_7$. Therefore, $B \subseteq C$ where $C = \mbf{[v, w]}$ for some $v$ and $w$ and also no matrices in $C$ have eigenvalues in $\F_7$. \\

By Lemma \ref{PowerMatrixSetSameSizeCoprimeOrder}, $|A| = |B|$. Since all matrices in $B \subseteq C$ have order $19$, then by Lemma \ref{OneOrderNThenAll}, all matrices in $C$ have order $19$. Let
$$
D = \{ M^t \mid M \in C \} \subseteq \SL.
$$
Since an arbitrary element of $B$ is of the form $M^k$ for some $M \in A$ and since $B \subseteq C$, then $(M^k)^t \in D$. We can compute 
$$(M^k)^t = M^{19s+1} = M^{19s} \cdot M = M.$$ 
Thus, $M \in D$, so $A \subseteq D$. \\

Using a similar argument as earlier, we also obtain that $D \subseteq \mbf{[x, y]}$ for some $x$ and $y$ and $|C| = |D|$. Then, we have 
$$
\mbf{[i, j]} = A \subseteq D \subseteq \mbf{[x, y]}.
$$
This forces us to have ${\mbf{[i, j]} = \mbf{[x, y]}}$ and so we must have $A = D$, which gives us ${|C| = |D| = |A| = |B|}$. Since $B \subseteq C$, then we must have $B = C$. Hence, $B = \mbf{[v, w]}$.
\end{proof} 

\begin{lem}\label{AllPairsHaveTheSameSize}
The sets $\mbf{[0, 1]}$, $\mbf{[0, 2]}$, $\mbf{[0, 4]}$, $\mbf{[1, 0]}$, $\mbf{[1, 3]}$, $\mbf{[1, 5]}$, $\mbf{[2, 0]}$, $\mbf{[2, 5]}$, $\mbf{[2, 6]}$, $\mbf{[3, 1]}$, $\mbf{[3, 4]}$, $\mbf{[4, 0]}$, $\mbf{[4, 3]}$, $\mbf{[4, 6]}$, $\mbf{[5, 1]}$, $\mbf{[5, 2]}$, $\mbf{[6, 2]}$, $\mbf{[6, 4]}$ all have the same size.
\end{lem}
\begin{proof}
We will first apply Lemma \ref{M2MBijection} repeatedly, as shown in the following equalities. 
\begin{align*} 
|\mbf{[0,1]}| &= |\mbf{[0, 4]}| = |\mbf{[0,2]}|.\\
|\mbf{[1,0]}| &= |\mbf{[2,0]}| = |\mbf{[4,0]}|.\\
|\mbf{[1,3]}| &= |\mbf{[2,5]}| = |\mbf{[4,6]}|.\\
|\mbf{[1,5]}| &= |\mbf{[2,6]}| = |\mbf{[4,3]}|.\\
|\mbf{[3,1]}| &= |\mbf{[6,4]}| = |\mbf{[5,2]}|.\\
|\mbf{[3,4]}| &= |\mbf{[6,2]}| = |\mbf{[5,1]}|.\\
\end{align*}
Let 
$$
M = \MatrThree{0}{2}{-1}{0}{0}{2}{2}{0}{0}.
$$
We proved in the table in Section 2 that $M$ has order $19$. Also from the table we have the following:

\begin{align*}
M &\in  \mbf{[0,2]}.\\
M^2 &\in \mbf{[0,2]}.\\
M^4 &\in \mbf{[0,2]}.\\
M^5 &\in \mbf{[0,2]}.\\
M^8 &\in \mbf{[0,2]}.\\
M^9 &\in \mbf{[3,1]}.\\
\end{align*}
By Lemma \ref{OneOrderNThenAll} and \ref{18PairsRepresentAllNonEigenMatrices}, all matrices in $\mbf{[0,2]}$ must have order $19$ and none have any eigenvalues in $\F_7$. Then using Lemma \ref{PowerConjClassSizeOrder19} we get that 
$$
|\mbf{[0,2]}| = |\mbf{[3, 4]}| = |\mbf{[1,3]}| = |\mbf{[4, 3]}| = |\mbf{[2,0]}| = |\mbf{[3,1]}|.
$$
Combining this with the earlier equalities gives us that all $18$ sets have the same size.
\end{proof}

\subsection{Further Results on Conjugacy Classes}

\begin{lem}\label{PowerOfMInConjClass}
Consider an $M \in \SL$ with order $19$. Consider a collection of conjugacy classes ${K_1, K_2, \ldots, K_m}$ in $\SL$ where all the elements in these conjugacy classes have order $19$. Suppose that $M \in K_1$. Then for each integer $n$ where ${1 \leq n \leq m}$, there exists a positive integer $c_n$ where ${1 \leq c_n \leq 18}$ such that $M^{c_n} \in K_n$. 
\end{lem}
\begin{proof}
Consider some value of $n$ where ${1 \leq n \leq m}$. Let $P$ be the cyclic group of order $19$ generated by $M$. Choose an arbitrary $N \in K_n$ (which has order $19$ by definition). Let $Q$ be the cyclic group of order $19$ generated by $N$. By Lemma \ref{AllOrder19SubgroupsAreSylow}, $P$ and $Q$ are Sylow $19$-subgroups of $\SL$. By Sylow's second theorem, $P$ and $Q$ are conjugate to each other, so there exists an $A \in \SL$ such that $A^{-1}QA = P$. Hence, there exists a positive integer $t$ where $1 \leq t \leq 18$ such that $A^{-1}NA = M^t$. Since $K_m$ is a conjugacy class, then 
$$
M^t = A^{-1}NA \in K_n.
$$
\end{proof}

% \begin{lem}\label{}

\begin{lem}\label{ConjClassWithinAPair}
Let $K$ be a conjugacy class in $\SL$ where none of the matrices in $K$ have eigenvalues in $\F_7$. Then, $K \subseteq \mbf{[i, j]}$ for some values of $i$ and $j$.
\end{lem}
\begin{proof}
This immediately follows from Lemma \ref{SameConjCharacPoly}.
\end{proof}

\begin{lem}\label{DistinctConjClassesSameSizeFoundInThreePairs}
There exist distinct conjugacy classes in $\SL$ contained in each of $\mbf{[0, 1]}$, $\mbf{[0, 2]}$ and $\mbf{[0, 4]}$, all of which have size $2^5 \times 3^2 \times 7^3$.
\end{lem}
\begin{proof}
Let
\begin{align*}
    M_1 &= \MatrThree{0}{1}{3}{0}{0}{1}{1}{0}{0}.\\
    M_2 &= 2M_1.\\
    M_3 &= 4M_1.\\
\end{align*}
We saw in Section $1$ that $M_1 \in \mbf{[0, 4]}$. By Lemma \ref{M2MBijection} we get that $M_2 \in \mbf{[0, 2]}$. By applying Lemma \ref{M2MBijection} on $\mbf{[0, 2]}$ we obtain $M_3 \in \mbf{[0, 1]}$. \\

Let $K_1, K_2, K_3$ be the conjugacy classes in $\SL$ which contain $M_1, M_2, M_3$ respectively. By Lemma \ref{ConjClassWithinAPair}, we have 
\begin{align*}
K_1 &\subseteq \mbf{[0, 4]}.\\
K_2 &\subseteq \mbf{[0, 2]}.\\
K_3 &\subseteq \mbf{[0, 1]}.\\
\end{align*}
In Section 3 we showed that ${|K_2| = 2^5 \times 3^2 \times 7^3}$. By Lemma  \ref{ConjClassSizesAreSameDoubleAndQuadruple}, $K_1$, $K_2$ and $K_3$ all have the same size of ${2^5 \times 3^2 \times 7^3}$.
\end{proof}

\begin{lem}\label{FiveMoreConjClassesFound}
There exist distinct conjugacy classes in $\SL$ contained within each of $\mbf{[1, 3]}$, $\mbf{[2, 0]}$, $\mbf{[3, 1]}$, $\mbf{[3, 4]}$, $\mbf{[4, 3]}$, all of which have size $2^5 \times 3^2 \times 7^3$.
\end{lem}
\begin{proof}
For $i=1, 2, 3$ define $M_i$ and $K_i$ as in Lemma \ref{DistinctConjClassesSameSizeFoundInThreePairs}. Note that the matrix 
$$M_2=2M_1= \MatrThree{0}{2}{-1}{0}{0}{2}{2}{0}{0}.$$
was shown to have order $19$ from the table in Section $2$. Let
\begin{align*}
    M_4 &= {M_2}^6.\\
    M_5 &= {M_2}^8.\\
    M_6 &= {M_2}^{15}.\\
    M_7 &= {M_2}^2.\\
    M_8 &= {M_2}^5.\\
\end{align*}

 From that same table, we have 
 \begin{align*}
     M_4 &\in \mbf{[1, 3]}.\\
     M_5 &\in \mbf{[2,0]}.\\
     M_6 &\in \mbf{[3, 1]}.\\
     M_7 &\in \mbf{[3, 4]}.\\
     M_8 &\in \mbf{[4, 3]}.\\
 \end{align*}
 For $t = 4, 5, \ldots, 8$, let $K_t$ be the conjugacy class in $\SL$ that contains $M_t$. By Lemma \ref{ConjClassWithinAPair}, we have
 \begin{align*}
     K_4 &\subseteq \mbf{[1, 3]}.\\
     K_5 &\subseteq \mbf{[2, 0]}.\\
     K_6 &\subseteq \mbf{[3, 1]}.\\
     K_7 &\subseteq \mbf{[3, 4]}.\\
     K_8 &\subseteq \mbf{[4, 3]}.\\
 \end{align*}
From applying Lemma \ref{PowerConjClassSameSize} to $M_2$ (which has order $19$), we obtain that $K_4, K_5, \ldots, K_8$ have the same size as $K_2$, which we know to be ${2^5 \times 3^2 \times 7^3}$ from Lemma \ref{DistinctConjClassesSameSizeFoundInThreePairs}.
\end{proof}


\begin{lem}\label{RemainingConjClassSizesFound}
There exist distinct conjugacy classes in $\SL$ contained within each of $\mbf{[2, 5]}$, $\mbf{[4, 6]}$, $\mbf{[4, 0]}$, $\mbf{[1, 0]}$, $\mbf{[6, 4]}$, $\mbf{[5, 2]}$, $\mbf{[6, 2]}$, $\mbf{[5, 1]}$, $\mbf{[1, 5]}$, $\mbf{[2, 6]}$, all of which have size $2^5 \times 3^2 \times 7^3$.
\end{lem}
\begin{proof}
For $i=1, 2, \ldots, 8$, define $M_i$ and $K_i$ as in Lemma \ref{FiveMoreConjClassesFound}. Then, we will define
\begin{align*}
M_9 &= 2M_4.\\
M_{10} &= 4M_4.\\
M_{11} &= 2M_5.\\
M_{12} &= 4M_5.\\
M_{13} &= 2M_6.\\
M_{14} &= 4M_6.\\
M_{15} &= 2M_7.\\
M_{16} &= 4M_7.\\
M_{17} &= 2M_8.\\
M_{18} &= 4M_8.\\
\end{align*}
For $9 \leq i \leq 18$, let $K_i$ be the conjugacy class in $\SL$ that contains $M_i$. From applying Lemma \ref{M2MBijection} several times, we have 
\begin{align*}
M_9 &\in \mbf{[2, 5]}.\\
M_{10} &\in \mbf{[4, 6]}.\\
M_{11} &\in \mbf{[4, 0]}.\\
M_{12} &\in \mbf{[1, 0]}.\\
M_{13} &\in \mbf{[6, 4]}.\\
M_{14} &\in \mbf{[5, 2]}.\\
M_{15} &\in \mbf{[6, 2]}.\\
M_{16} &\in \mbf{[5, 1]}.\\
M_{17} &\in \mbf{[1, 5]}.\\
M_{18} &\in \mbf{[2, 6]}.\\
\end{align*}
By Lemma \ref{ConjClassWithinAPair},  we have 
\begin{align*}
K_9 &\subseteq \mbf{[2, 5]}.\\
K_{10} &\subseteq \mbf{[4, 6]}.\\
K_{11} &\subseteq \mbf{[4, 0]}.\\
K_{12} &\subseteq \mbf{[1, 0]}.\\
K_{13} &\subseteq \mbf{[6, 4]}.\\
K_{14} &\subseteq \mbf{[5, 2]}.\\
K_{15} &\subseteq \mbf{[6, 2]}.\\
K_{16} &\subseteq \mbf{[5, 1]}.\\
K_{17} &\subseteq \mbf{[1, 5]}.\\
K_{18} &\subseteq \mbf{[2, 6]}.\\
\end{align*}
It was proven in Lemma \ref{FiveMoreConjClassesFound} that $K_4, K_5, \ldots, K_8$ each have size ${2^5 \times 3^2 \times 7^3}$. From applying Lemma \ref{ConjClassSizesAreSameDoubleAndQuadruple} to $M_4, M_5, \ldots, M_8$, we obtain that $K_9, K_{10}, \ldots, K_{18}$ also have size ${2^5 \times 3^2 \times 7^3}$.
\end{proof}
\begin{lem}\label{All18ConjClassSizesFound}
There exist at least $18$ distinct conjugacy classes in $\SL$ of size $2^5 \times 3^2 \times 7^3$ where each is contained in one of $\mbf{[0, 1]}$, $\mbf{[0, 2]}$, $\mbf{[0, 4]}$, $\mbf{[1, 0]}$, $\mbf{[1, 3]}$, $\mbf{[1, 5]}$, $\mbf{[2, 0]}$, $\mbf{[2, 5]}$, $\mbf{[2, 6]}$, $\mbf{[3, 1]}$, $\mbf{[3, 4]}$, $\mbf{[4, 0]}$, $\mbf{[4, 3]}$, $\mbf{[4, 6]}$, $\mbf{[5, 1]}$, $\mbf{[5, 2]}$, $\mbf{[6, 2]}$ and $\mbf{[6, 4]}$.
\end{lem}

\begin{proof}
This follows from Lemmas \ref{DistinctConjClassesSameSizeFoundInThreePairs}, \ref{FiveMoreConjClassesFound} and \ref{RemainingConjClassSizesFound}.
\end{proof}

\begin{lem}\label{CharacteriseAllOrder19And57Matrices}
All the matrices in $\mbf{[0,2]}$, $\mbf{[1,3]}$, $\mbf{[2,0]}$,  $\mbf{[3,1]}$, $\mbf{[3,4]}$ and $\mbf{[4,3]}$ have order $19$. Also, all the matrices in $\mbf{[0,4]}$, $\mbf{[0,1]}$, $\mbf{[2,5]}$, $\mbf{[4,6]}$, $\mbf{[4,0]}$, $\mbf{[1,0]}$, $\mbf{[6,4]}$, $\mbf{[5,2]}$, $\mbf{[6,2]}$, $\mbf{[5,1]}$, $\mbf{[1,5]}$, $\mbf{[2,6]}$ have order $57$.
\end{lem}
\begin{proof}
For $i=1, 2, 3, \ldots, 18$ define $M_i$ and $K_i$ as in Lemma \ref{RemainingConjClassSizesFound}. Recall that $M_2 \in \mbf{[0, 2]}$ has order $19$. For $j= 4, 5, \ldots, 8$ recall that $M_j$ was defined to be of the form $M_2^k$ for certain values of $k$ where $2 \leq k \leq 18$. Hence, each such $M_j$ has order $19$ because the values of $k$ are coprime to $19$. \\

Next, recall that for $t=9, 10, \ldots, 18$ the matrix $M_t$ was defined to be of the form $2M_j$ or $4M_j$ for certain values of $j$ where $4 \leq j \leq 8$. Also, we have
$$M_1 = \frac{M_2}{2}=4M_2$$
and $M_3=2M_2$. By Lemma \ref{OrderOfDoubleAndQuadrupleIs57} we obtain that the matrices $M_1, M_3$ and $M_t$ for $t=9, 10, \ldots, 18$ all have order $57$. Thus, we have found representative matrices with order $19$ in each of $\mbf{[0,2]}$, $\mbf{[1,3]}$, $\mbf{[2,0]}$,  $\mbf{[3,1]}$, $\mbf{[3,4]}$ and $\mbf{[4,3]}$ and representative matrices with order $57$ in each of $\mbf{[0,4]}$, $\mbf{[0,1]}$, $\mbf{[2,5]}$, $\mbf{[4,6]}$, $\mbf{[4,0]}$, $\mbf{[1,0]}$, $\mbf{[6,4]}$, $\mbf{[5,2]}$, $\mbf{[6,2]}$, $\mbf{[5,1]}$, $\mbf{[1,5]}$, $\mbf{[2,6]}$. \\

The result then follows from Lemma \ref{OneOrderNThenAll}.

% Note that from earlier we have $M_2 \in \mbf{[0, 2]}$ has order $19$. Then we get that $M_4=M_2^6 \in \mbf{[1,3]}$, $M_5=M_2^8 \in \mbf{[2,0]}$, $M_6=M_2^{15} \in \mbf{[3,1]}$, $M_7 = M_2^2 \in \mbf{[3,4]}$ and $M_8=M_2^5 \in \mbf{[4,3]}$ have order $19$ because they are non-identity elements of the cyclic group of order $19$ generated by $M_2$. Then by \ref{OrderOfDoubleAndQuadrupleIs57} we get that $M_1=\frac{M_2}{2}=4M_2 \in \mbf{[0,4]}$, $M_3=2M_2 \in \mbf{[0,1]}$, $M_9=2M_4 \in \mbf{[2,5]}$, $M_{10}=4M_4 \in \mbf{[4, 6]}$, $M_{11} = 2M_5 \in \mbf{[4,0]}$, $M_{12}=4M_5 \in \mbf{[2,0]}$, $M_{13}=2M_6 \in \mbf{[6,4]}$, $M_{14}=4M_6 \in \mbf{[5,2]}$, $M_{15} = 2M_7 \in \mbf{[6,2]}$, $M_{16}=4M_7 \in \mbf{[5,1]}$, $M_{17} = 2M_8 \in \mbf{[1,5]}$ and $M_{18} = 4M_8 \in \mbf{[2,6]}$ all have order $57$. \\

\end{proof} 


\begin{lem}\label{AllMatricesNonEigenOrder19Or57}
All matrices in $\SL$ without any eigenvalues in $\F_7$ have order $19$ or $57$.
\end{lem}
\begin{proof}
There are two ways to arrive at this result. One way is to apply Lemmas \ref{18PairsRepresentAllNonEigenMatrices} and \ref{CharacteriseAllOrder19And57Matrices} together. The second way is to apply Lemmas \ref{OrderNIffExtSame} and \ref{NoOrder3NonEigen} together (since the positive integers that divide $57$ are $1, 3, 19, 57$).
\end{proof}

\begin{lem}\label{AllOrder19MatricesNoEigen}
All matrices in $\SL$ with order $19$ have no eigenvalues in $\F_7$.
\end{lem}
\begin{proof}
Define $M_2$ and $K_2$ as in Lemma \ref{DistinctConjClassesSameSizeFoundInThreePairs}. We know from earlier that $M_2$ has no eigenvalues in $\F_7$ and has order $19$. Also, $K_2$ is the conjugacy class that contains $M_2$. Let $N$ be an arbitrary matrix in $\SL$ with order $19$ and let $K$ be the conjugacy class that contains it.\\

By Lemma \ref{PowerOfMInConjClass}, there exists a positive integer $c$ where $1 \leq c \leq 18$ such that $M_2^c \in K$. Since $19$ is prime, then by Lemma \ref{PrimeNoEigen}, $M_2^c$ has no eigenvalues in $\F_7$. Then by Lemma \ref{NoEigenConjClass}, all matrices in $K$ have no eigenvalues in $\F_7$, so $N$ has no eigenvalues in $\F_7$. 
\end{proof}

\begin{lem}\label{AllOrder57MatricesNoEigen}
All matrices in $\SL$ with order $57$ have no eigenvalues in $\F_7$.
\end{lem}
\begin{proof}
Suppose the contrary, that a matrix $M \in \SL$ has order $57$ and an eigenvalue $\lambda \in \F_7$. Then $M^3$ has order $19$ and also has eigenvalue $\lambda^3 \in \F_7$ by Lemma \ref{EigenvalueScale}. This contradicts Lemma \ref{AllOrder19MatricesNoEigen}.
\end{proof}

\begin{lem}\label{NoOtherOrder19Or57Matrices}
The matrices given in Lemma \ref{CharacteriseAllOrder19And57Matrices} describe all the matrices in $\SL$ that have order $19$ or $57$.
\end{lem}
\begin{proof}
This follows from Lemmas \ref{AllOrder19MatricesNoEigen} and \ref{AllOrder57MatricesNoEigen} because the matrices given in Lemma \ref{CharacteriseAllOrder19And57Matrices} cover all possible matrices in $\SL$ without any eigenvalues in $\F_7$.
\end{proof}

\begin{lem}\label{Matrices18And54TimesSubgroups}
Let $A$ be the set of all matrices in $\SL$ with order $19$. Let $B$ be the set of matrices in $\SL$ with order $19$ or $57$. Using the notation given earlier in the stating of Sylow's theorems, let $n_{19}$ be the number of Sylow $19$-subgroups of $\SL$ (which we know to be precisely the subgroups of order $19$ from Lemma \ref{AllOrder19SubgroupsAreSylow}). Then ${|A| = 18n_{19}}$ and ${|B| = 54n_{19}}$.

\end{lem}

\begin{proof}
% \textbf{IMPORTANT THING TO CHECK: Make sure there are no matrices with eigenvalues and with order 19. Maybe look at Lemma 7.36}
All the Sylow $19$-subgroups are cyclic since $19$ is prime. The non-identity elements of all of these subgroups have order $19$ and so are in $A$. Also, the subgroups generated by each element of $A$ are cyclic subgroups of order $19$ and thus are all Sylow $19$-subgroups. Hence, using Lemma \ref{DifferentSubgroupsDisjoint}, we obtain that $A$ can be partitioned into subsets of size $18$ where each subset contains the $18$ non-identity elements of each Sylow $19$-subgroup. Hence, ${|A| = 18|n_{19}|}$. \\

Let $m = |\mbf{[0,2]}|$. By Lemmas \ref{18PairsRepresentAllNonEigenMatrices}, \ref{AllPairsHaveTheSameSize}, \ref{CharacteriseAllOrder19And57Matrices} and \ref{NoOtherOrder19Or57Matrices}, it follows that ${|A| = 6m}$ and ${|B| = 18m}$. Hence,
$$|B| = 3|A| = 54n_{19}.$$ 

\end{proof}

\begin{lem}\label{PossibleSubgroupSize}
 Define $n_{19}$ as in Lemma \ref{Matrices18And54TimesSubgroups}. Then,
 $$n_{19} \in \{1, \  2^5 \cdot 3, \ 7^3, \ 2^4 \cdot 3^2 \cdot 7, \ 2^3 \cdot 3^3 \cdot 7^2, \ 2^5 \cdot 3 \cdot 7^3       \}.$$
\end{lem} 
\begin{proof}
The order of $\SL$ can be written as $19m$ where $m = 2^5 \times 3^3 \times 7^3$. By Sylow's third theorem with the prime $p = 19$, we obtain the following conditions on $n_{19}$.

\begin{itemize}
    \item $n_{19} \equiv 1 \; (\bmod\; 19)$
    \item $n_{19}$ is a factor of $m$.
\end{itemize}
By testing the out all factors of $m$ that are $1 \; (\bmod\; 19)$, we obtain the possible values of $n_{19}$ listed above.
\end{proof}

\begin{lem}\label{The18PairsThemselvesAreConjClasses}
There exists exactly $18$ distinct conjugacy classes in $\SL$ where none of the matrices in the conjugacy classes have eigenvalues in $\F_7$. Each conjugacy class has size $2^5 \times 3^2 \times 7^3$. These are the sets $\mbf{[0, 1]}$, $\mbf{[0, 2]}$, $\mbf{[0, 4]}$, $\mbf{[1, 0]}$, $\mbf{[1, 3]}$, $\mbf{[1, 5]}$, $\mbf{[2, 0]}$, $\mbf{[2, 5]}$, $\mbf{[2, 6]}$, $\mbf{[3, 1]}$, $\mbf{[3, 4]}$, $\mbf{[4, 0]}$, $\mbf{[4, 3]}$, $\mbf{[4, 6]}$, $\mbf{[5, 1]}$, $\mbf{[5, 2]}$, $\mbf{[6, 2]}$ and $\mbf{[6, 4]}$.
\end{lem}
\begin{proof}
Define $M_1, \ M_2, \ldots, \ M_{18}$ and $K_1, \ K_2, \ldots, \ K_{18}$ as in Lemmas \ref{DistinctConjClassesSameSizeFoundInThreePairs}, \ref{FiveMoreConjClassesFound} and \ref{RemainingConjClassSizesFound}. Let $m$ be the total number of matrices in $\SL$ without any eigenvalues in $\F_7$. Also define $A$, $B$ and $n_{19}$ as in Lemma \ref{Matrices18And54TimesSubgroups}. \\

From Lemma \ref{NoOtherOrder19Or57Matrices}, we obtain that $|B| = m$. In  Lemma \ref{All18ConjClassSizesFound} we proved that there exists a conjugacy class contained within each of the $18$ sets listed above, where each of the conjugacy classes has size ${2^5 \times 3^2 \times 7^3}$. \\

It follows that 
$$|B| = m \geq 2^5 \times 3^2 \times 7^3 \times 18 = 2^6 \times 3^4 \times 7^3.$$ 
From Lemma \ref{Matrices18And54TimesSubgroups}, we have $|B| = 54n_{19}$ so 
$$n_{19} \geq \frac{2^6 \times 3^4 \times 7^3}{54} = 2^5 \times 3 \times 7^3.$$ 
Then from Lemma \ref{PossibleSubgroupSize}, we must have $n_{19}=2^5 \times 3 \times 7^3$. Thus, equality must occur in the previous inequalities so we must have $m = |B| = 2^6 \times 3^4 \times 7^3$. This also means that the $18$ sets listed above themselves must be conjugacy classes. \\

Since these $18$ conjugacy classes partition the set of matrices in $\SL$ with no eigenvalues in $\F_7$, then these form all the conjugacy classes in $\SL$ where none of the matrices in the conjugacy classes have eigenvalues in $\F_7$.
\end{proof}

\begin{lem}\label{TotalNumberOfNonEigenMatricesFound}
There are exactly $2^6 \times 3^4 \times 7^3$ matrices in $\SL$ without any eigenvalues in $\F_7$.
\end{lem}
\begin{proof}
This result follows from the proof of Lemma \ref{The18PairsThemselvesAreConjClasses}.
\end{proof}

% \subsection{Eigenspaces}

% \begin{lem}\label{eigenspace-dim1-not-commute}
% Let $M \in \SL$ have no eigenvalues in $\F_7$. Let $A \in \SL$ have an eigenvalue $\lambda \in \F_7$ where the corresponding eigenspace $E$ as dimension one. Then $MS \neq SM$.
% \end{lem}
% Suppose that $MS = SM$. Let $v \in E$ where $v \neq 0$. Then $Sv =\lambda v$. Also $S(Mv)=(SM)v=(MS)v=M(Sv)=M(\lambda v)=\lambda (Mv)$ so $Mv \in E$. Since $E$ has dimension one then $Mv = \mu v$ for some $\mu \in \F_7$, which contradicts $M$ having no eigenvalues in $\F_7$.
% \begin{proof}

% \end{proof}


% \subsection{Matrices with small orders}

% \begin{lem}\label{kab,0kc,00k-order2or3}
% If $M \in \SL$ is of the form 
% $$M=\begin{pmatrix}
% k&a&b\\
% 0&k&c\\
% 0&0&k\\
% \end{pmatrix}.
% $$
% where $k \in \{1, 2, 4\}$ and $M \neq kI$ then $M$ cannot have order $2$ nor $3$.
% \end{lem}
% \begin{proof}
% We have
% $$M^2=\begin{pmatrix}
% k^2&2ak&ac+2bk\\
% 0&k^2&2ck\\
% 0&0&k^2\\
% \end{pmatrix}.
% $$
% If $M^2=I$ then we must have $2ak=2ck=0$ so $a=c=0$. Also $ac+2bk=0$ but since $a=c=0$ then $b=0$, which contradicts $M \neq kI$. \\

% Next we have

% $$M^3=\begin{pmatrix}
% k^3&3ak^2&3ack+3bk^2\\
% 0&k^3&3ck^2\\
% 0&0&k^3\\
% \end{pmatrix}.
% $$
% If $M^3=I$ then we must have $3ak^2=3ck^2=0$ so $a=c=0$. Also $3ack+3bk^2=0$ but since $a=c=0$, then $b=0$, which contradicts $M \neq I$.
% \end{proof}

% \begin{lem}\label{100,0-1x,001-order2}
% If $M \in \SL$ is of the form 
% $$M=\begin{pmatrix}
% 1&0&0\\
% 0&-1&x\\
% 0&0&-1\\
% \end{pmatrix}.
% $$
% then $M$ cannot have order $2$.
% \end{lem}
% \begin{proof}
% We can compute 
% $$M^2=\begin{pmatrix}
% 1&0&0\\
% 0&1&-2x\\
% 0&0&1\\
% \end{pmatrix}.
% $$
% Since $M \neq I$ then $x \neq 0$, so $-2x \neq 0$ and thus $M$ cannot have order $2$. 
% \end{proof}

% \begin{lem}\label{1ab,0-1c,00-1-order2}
% If $M \in \SL$ is of the form 
% $$
% M=\begin{pmatrix}
% 1&a&b\\
% 0&-1&c\\
% 0&0&-1\\
% \end{pmatrix}.
% $$
% and $M$ is not then $M$ has order $2$ if and only if $c=0$.
% \end{lem}
% \begin{proof}
% We have
% $$M^2=\begin{pmatrix}
% 1&0&ac\\
% 0&1&-2c\\
% 0&0&1\\
% \end{pmatrix}.
% $$
% If $M^2=I$ then we must have $-2c=0$ so $c=0$. Conversely, if $c=0$ then since $ac=0$ and $-2c=0$ then $M^2=I$.
% \end{proof}

% \begin{lem}\label{-1ab,01c,00-1-order2}
% If $M \in \SL$ is of the form 
% $$
% M=\begin{pmatrix}
% -1&a&b\\
% 0&1&c\\
% 0&0&-1\\
% \end{pmatrix}.
% $$
% then $M$ has order $2$ if and only if $ac=2b$.
% \end{lem}
% \begin{proof}
% We can compute 
% $$M^2=\begin{pmatrix}
% 1&0&ac-2b\\
% 0&1&0\\
% 0&0&1\\
% \end{pmatrix}.
% $$
% Immediately we see that $M^2=I$ if and only if $ac=2b$.
% \end{proof}
% \textbf{LOOK AT THIS VIDEO TO COMPLETE THE REST https://www.youtube.com/watch?v=IndruwcYmSM}


%\begin{thm}\label{ConjClassOverview}
%The following statements are true:

%\begin{itemize}
  %  \item There are exactly $18$ conjugacy classes in $\SL$ where none of the matrices in each conjugacy class has an eigenvector.
   % \item Each of these conjugacy classes has size $2^5 \times 3^2 \times 7^3$.
    %\item These $18$ conjugacy classes are the sets $\mbf{[0, 1]}$, $\mbf{[0, 2]}$, $\mbf{[0, 4]}$, $\mbf{[1, 0]}$, $\mbf{[1, 3]}$, $\mbf{[1, 6]}$, $\mbf{[2, 0]}$, $\mbf{[2, 5]}$, $\mbf{[2, 6]}$, $\mbf{[3, 1]}$, $\mbf{[3, 4]}$, $\mbf{[4, 0]}$, $\mbf{[4, 3]}$, $\mbf{[4, 6]}$, $\mbf{[5, 1]}$, $\mbf{[5, 2]}$, $\mbf{[6, 2]}$ and $\mbf{[6, 4]}$.
  %  \item All the elements in each of the conjugacy classes $\mbf{[0, 2]}$, $\mbf{[1, 3]}$, $\mbf{[2, 0]}$, $\mbf{[3, 1]}$, $\mbf{[3, 4]}$ and $\mbf{[4, 3]}$ have order $19$.
   % \item All the elements in each of the conjugacy classes $\mbf{[0, 1]}$, $\mbf{[0, 4]}$, $\mbf{[1, 0]}$, $\mbf{[1, 6]}$, $\mbf{[2, 5]}$, $\mbf{[2, 6]}$, $\mbf{[4, 0]}$, $\mbf{[4, 6]}$, $\mbf{[5, 1]}$, $\mbf{[5, 2]}$, $\mbf{[6, 2]}$ and $\mbf{[6, 4]}$ have order $57$.
%\end{itemize}
%\end{thm}
%\begin{proof}
%\textbf{NOTE: SOME OF THIS COULD BE PROVEN USING MICHAEL'S IDEA USING SYLOW'S THEOREMS THEOREM \ref{ExactSizeOfConjClass} WITHOUT ASSUMING ANY COMPUTER CALCULATIONS}
%Note that the total number of matrices in $\SL$ with no eigenvectors is at least the number of matrices in the $18$ conjugacy classes found in \ref{distinctConjClass}, \ref{ConjClassSize} and \ref{ConjClassSize2}. This number is equal to $2^5 \times 3^2 \times 7^3 \times 18 = 2^6 \times 3^4 \times 7^3$. However, the number $2^6 \times 3^4 \times 7^3$, by assumption, was a lower bound to the total number of matrices in $\SL$ without eigenvectors. Hence, we must have equality, meaning that there are exactly $18$ conjugacy classes in $\SL$ where none of the matrices in each conjugacy class have eigenvectors. \\

%From the proofs of \ref{distinctConjClass}, \ref{ConjClassSize} and \ref{ConjClassSize2}, recall that $H_1 \subseteq \mbf{[0, 4]}$, $H_2 \subseteq \mbf{[0, 2]}$, $H_3 \subseteq \mbf{[0, 1]}$, $H_4 \subseteq \mbf{[1, 3]}$, $H_5 \subseteq \mbf{[2, 0]}$, $H_6 \subseteq \mbf{[3, 1]}$, $H_7 \subseteq \mbf{[3, 4]}$, $H_8 \subseteq \mbf{[4, 3]}$, $H_9 \subseteq \mbf{[2, 5]}$, $H_{10} \subseteq \mbf{[4, 6]}$, $H_{11} \subseteq \mbf{[4, 0]}$, $H_{12} \subseteq \mbf{[1, 0]}$, $H_{13} \subseteq \mbf{[6, 4]}$, $H_{14} \subseteq \mbf{[5, 2]}$, $H_{15} \subseteq \mbf{[6, 2]}$, $H_{16} \subseteq \mbf{[5, 1]}$, $H_{17} \subseteq \mbf{[1, 5]}$ and $H_{18} \subseteq \mbf{[2, 6]}$. However but note that we have exactly $18$ conjugacy classes must partition the set of all matrices in $\SL$ without eigenvectors. \\

%It must follow that $H_1, \ H_2, \ldots , \ H_{18}$ are all the $18$ conjugacy classes and we must have the equalities $H_1 = \mbf{[0, 4]}$, $H_2 = \mbf{[0, 2]}$, $H_3 = \mbf{[0, 1]}$, $H_4 = \mbf{[1, 3]}$, $H_5 = \mbf{[2, 0]}$, $H_6 = \mbf{[3, 1]}$, $H_7 = \mbf{[3, 4]}$, $H_8 = \mbf{[4, 3]}$, $H_9 = \mbf{[2, 5]}$, $H_{10} = \mbf{[4, 6]}$, $H_{11} = \mbf{[4, 0]}$, $H_{12} = \mbf{[1, 0]}$, $H_{13} = \mbf{[6, 4]}$, $H_{14} = \mbf{[5, 2]}$, $H_{15} = \mbf{[6, 2]}$, $H_{16} = \mbf{[5, 1]}$, $H_{17} = \mbf{[1, 5]}$ and $H_{18} = \mbf{[2, 6]}$. So far, this proves the first three statements in the theorem. \\

%Note that from earlier, we found that $M_2$ has order $19$. Since $M_3 = 2M_2$ and $M_1 = \frac{1}{2}M_2 = 4M_2$, it follows from \ref{OrderOfPower} that $M_1$ and $M_3$ have order $57$. Since $M_4$, \ $M_5$, \ $M_6$, \ $M_7$ and $M_8$ are in the cyclic subgroup of order $19$ generated by $M_2$, then each of these matrices also has order $19$. Then, it follows again from \ref{OrderOfPower} that $M_9, \ M_{10}, \ M_{11}, \ldots , \ M_{18}$ have order $57$. \\

%By \ref{EigenCoprime}, it follows that conjugacy classes $H_2, \ H_4, \ H_5, \ H_6, \ H_7$ and $H_8$ all consists only of matrices of order $19$ and the conjugacy classes $H_1, \ H_3, \ H_9, \ H_{10}, \ H_{11}, \ldots , \ H_{18}$ consist only of matrices of order $57$. This completes the proof of the fourth and fifth statements.
%\end{proof}
\newpage
\subsection{Relationships between classes}

Here is a table of the 20 powers of a matrix with trace 1
\ \\

\begin{tabular}{c|c|c|c}
power&matrix&trace&class\\\hline
1&$\begin{pmatrix} 
0&1&-3\\
0&1&-2\\
1&0&0\\
\end{pmatrix}$&1&$\mbf{[1,3]}$\\\hline
2&$\begin{pmatrix} 
-3&1&-2\\
-2&1&-2\\
0&1&-3\\
\end{pmatrix}$ &2&$\mbf{[2,0]}$\\\hline
3& $\begin{pmatrix} 
-2&-2&0\\
-2&-1&-3\\
-3&1&-2\\
\end{pmatrix}$&2&$\mbf{[2,0]}$\\\hline
4&$\begin{pmatrix} 
0&3&3\\
-3&-3&1\\
-2&-2&0\\
\end{pmatrix}$&4&$\mbf{[4,3]}$\\\hline
5&$\begin{pmatrix} 
3&3&1\\
1&1&1\\
0&3&3\\
\end{pmatrix}$&0&$\mbf{[0,2]}$\\\hline
6& $\begin{pmatrix} 
1&-1&-1\\
1&2&2\\
3&3&1\\
\end{pmatrix}$&4&$\mbf{[4,3]}$\\\hline
7&$\begin{pmatrix} 
-1&0&-1\\
2&3&0\\
1&-1&-1\\
\end{pmatrix}$ &1&$\mbf{[1,3]}$\\\hline
8& $\begin{pmatrix} 
-1&-1&3\\
0&-2&2\\
1&0&0\\
\end{pmatrix}$&3&$\mbf{[3,1]}$\\\hline
9& $\begin{pmatrix} 
3&-2&-2\\
2&-2&-3\\
-1&-1&3\\
\end{pmatrix}$&4&$\mbf{[4,3]}$\\\hline
10& $\begin{pmatrix} 
-2&1&2\\
-3&0&-2\\
3&-2&-2\\
\end{pmatrix}$&3&$\mbf{[3,4]}$\\\hline
\end{tabular}
\quad
\begin{tabular}{c|c|c|c}
power&matrix&trace&class\\\hline
11&$\begin{pmatrix} 
2&-1&-3\\
-2&-3&2\\
-2&1&2\\
\end{pmatrix}$ &1&$\mbf{[1,3]}$\\\hline
12&$\begin{pmatrix} 
-3&1&3\\
2&2&-2\\
2&-1&-3\\
\end{pmatrix}$ &3&$\mbf{[3,1]}$\\\hline
13&$\begin{pmatrix} 
3&-2&0\\
-2&-3&-3\\
-3&1&3\\
\end{pmatrix}$ &3&$\mbf{[3,4]}$\\\hline
14&$\begin{pmatrix} 
0&1&2\\
-3&2&-2\\
3&-2&0\\
\end{pmatrix}$ &2&$\mbf{[2,0]}$\\\hline
15&$\begin{pmatrix} 
2&1&-2\\
-2&-1&-2\\
0&1&2\\
\end{pmatrix}$ &3&$\mbf{[3,4]}$\\\hline
16&$\begin{pmatrix} 
-2&3&-1\\
-2&-3&1\\
2&1&-2\\
\end{pmatrix}$ &0&$\mbf{[0,2]}$\\\hline
17&$\begin{pmatrix} 
-1&1&0\\
1&2&-2\\
-2&3&-1\\
\end{pmatrix}$ &0&$\mbf{[0,2]}$\\\hline
18&$\begin{pmatrix} 
0&0&1\\
-2&3&0\\
-1&1&0\\
\end{pmatrix}$ &3&$\mbf{[3,1]}$\\\hline 
19&$\begin{pmatrix} 
1&0&0\\
0&1&0\\
0&0&1\\
\end{pmatrix}$&3&$\mbf{[3,3]}$\\\hline
20&$\begin{pmatrix} 
0&1&-3\\
0&1&-2\\
1&0&0\\
\end{pmatrix}$&1&$\mbf{[1,3]}$\\\hline
\end{tabular} \\


We have bijections
$$\mbf{[1,3]}\to\mbf{[0,2]}$$
$$M\mapsto M^5$$
with inverse $$M\mapsto M^{4}.$$
Also $$M\mapsto M^{16}$$
with inverse $$M\mapsto M^{6}.$$
Also $$M\mapsto M^{17}$$
with inverse $$M\mapsto M^{9}.$$

This can be explained as the three bijections from (1,3) to itself composed with a single bijection to another class. The three bijections are identity, 7th power and $7^2=11$ th power.





\subsection{Subgroups section}
\begin{lem}\label{MaximalSubgroup}
Let $H$ be a proper subgroup of $G = \SL$. Let $A = \{A_1, A_2, \ldots A_n\}$ be a generating set for $G$. Suppose that for any $M \in G/H$, there exists matrices $B_1, B_2, \ldots B_n, C_1, C_2, \ldots C_n \in H$ such that $B_m M C_m = A_m$ for $m = 1, 2, \ldots n$. Then $H$ is a maximal subgroup of $G$.
\end{lem}
\begin{proof}
Suppose the contrary, where there exists a proper subgroup $K$ of $G$ such that $H$ is a proper subgroup of $K$. Then, consider a matrix $M \in K/H$. Since $K$ is a proper subgroup of $G$, then there exists a generator $X$ in $A$ that is not in $K$. However, there exists matrices $B$ and $C$ in $H$ such that $BMC = X$. Since $B, M$ and $C$ are in $K$, then $X$ is in $K$, contradiction. Thus, $H$ is a maximal subgroup of $G$. 
\end{proof}

\begin{thm}The subgroup $H$ of matrices of the form
$$\begin{pmatrix}
a&b&c\\
0&e&f\\
0&h&i\\
\end{pmatrix}$$
is maximal in $\SL$. 
\end{thm}
\begin{proof}

The generators of $\SL$ are as follows:

$$X = \begin{pmatrix}
1&0&1\\
0&-1&-1\\
0&1&0\\
\end{pmatrix}$$

$$Y = \begin{pmatrix}
0&1&0\\
0&0&1\\
1&0&0\\
\end{pmatrix}$$

$$Z = \begin{pmatrix}
0&1&0\\
1&0&0\\
-1&-1&-1\\
\end{pmatrix}$$\\

To prove that $H$ is a maximal subgroup, by \ref{EigenVectorWhenExtended}, it suffices to prove that given any arbitrary matrix $A \in \SL$ that is not in $H$, the generators $X, Y, Z$ for $\SL$ can be obtained by multiplying $A$ (on the left and the right) only by elements of $H$. Since $X$ itself is an element of $H$, we only need to prove that generators $Y$ and $Z$ are obtainable. \\

\subsubsection{Generators Proof for $Y$}
Let 

$$
A = \begin{pmatrix}
a&b&c\\
d&e&f\\
g&h&i\\
\end{pmatrix}
$$\\

Since $A$ is not in $H$, then either $d$ or $g$ must be non-zero. In the case of $g = 0$, then $d$ must be non-zero. We can multiply a matrix on the left to add row $2$ of $A$ to row $3$ to obtain the matrix $B$ as shown below.

$$B = \begin{pmatrix}
1&0&0\\
0&1&0\\
0&1&1\\
\end{pmatrix}
\begin{pmatrix}
a&b&c\\
d&e&f\\
0&h&i\\
\end{pmatrix} = 
\begin{pmatrix}
a&b&c\\
d&e&f\\
d&e+h&f+i\\
\end{pmatrix}$$\\

Otherwise, if $d = 0$ and $g \neq 0$, then we can multiply a matrix on the left to add row $3$ of $A$ to row $2$ to obtain the matrix $B$ as shown below.

$$ B = \begin{pmatrix}
1&0&0\\
0&1&1\\
0&0&1\\
\end{pmatrix}
\begin{pmatrix}
a&b&c\\
0&e&f\\
g&h&i\\
\end{pmatrix} = 
\begin{pmatrix}
a&b&c\\
g&e+h&f+i\\
g&h&i\\
\end{pmatrix}$$\\

Finally, if both $d$ and $g$ in $A$ were non-zero, then we can set $B =A$. In all cases we can obtain a matrix $B$ of the form 

$$ B = \begin{pmatrix}
a&b&c\\
d&e&f\\
g&h&i\\
\end{pmatrix}$$\\

where $d$ and $g$ are non-zero. Now, we can multiply a matrix on the left of $B$ to add multiples of row $3$ to row $2$ and row $1$. We can multiply by another matrix with determinant $1$ to scale row $3$. This procedure can yield a matrix $C$ with its top-left and left-middle entries being zero and its bottom-left entry being $1$, as shown below.

$$ C = 
\begin{pmatrix}
1&0&0\\
0&\frac{1}{l}&0\\
0&0&l\\
\end{pmatrix}
\begin{pmatrix}
1&0&j\\
0&1&k\\
0&0&1\\
\end{pmatrix}
\begin{pmatrix}
a&b&c\\
d&e&f\\
g&h&i\\
\end{pmatrix} = 
\begin{pmatrix}
0&m&n\\
0&p&q\\
1&s&t\\
\end{pmatrix}$$\\

Write $C$ in the form 

$$ C = \begin{pmatrix}
0&b&c\\
0&e&f\\
1&h&i\\
\end{pmatrix}$$\\

Now we can multiply a matrix on the right of $C$ to add multiples of column $1$ to columns $2$ and $3$ to yield matrix $D$ with its bottom-middle and bottom-right entries zero.

$$ D = \begin{pmatrix}
0&b&c\\
0&e&f\\
1&h&i\\
\end{pmatrix}
\begin{pmatrix}
1&-h&-i\\
0&1&0\\
0&0&1\\
\end{pmatrix} = 
\begin{pmatrix}
0&b&c\\
0&e&f\\
1&0&0\\
\end{pmatrix}$$\\

Note that $D \in \SL$ because we performed multiplications only with matrices with determinant $1$. Hence, $\mathrm{det}(D) = bf-ce=1$. If $b = 0$ or $f = 0$, then both $e$ and $c$ would have to be non-zero to ensure the determinant one condition. In this case, we can multiply a matrix on the right to add some multiple of column $2$ to column $3$ to obtain a matrix $E$ with its middle-right entry being non-zero.

$$ E = \begin{pmatrix}
0&b&c\\
0&e&f\\
1&0&0\\
\end{pmatrix}
\begin{pmatrix}
1&0&0\\
0&1&l\\
0&0&1\\
\end{pmatrix} = 
\begin{pmatrix}
0&b&c+bl\\
0&e&f+el\\
1&0&0\\
\end{pmatrix}$$\\

We can re-write $E$ in the form \\

$$\begin{pmatrix}
0&b&e\\
0&c&f\\
1&0&0\\
\end{pmatrix}$$\\

where $f \neq 0$ and $\mathrm{det}(E) = bf - ce = 1$. If $b \neq 0$, then let $F = E$. Otherwise, if $b = 0$ then we again have $c$ and $e$ are non-zero. Here, we can multiply a matrix on the left to add some multiple of row $2$ to row $1$ to obtain a matrix $F$ with its top-middle entry being non-zero.

$$ F = \begin{pmatrix}
1&k&0\\
0&1&0\\
0&0&1\\
\end{pmatrix}
\begin{pmatrix}
0&b&c\\
0&e&f\\
1&0&0\\
\end{pmatrix} = 
\begin{pmatrix}
0&b+ek&c+fk\\
0&e&f\\
1&0&0\\
\end{pmatrix}$$\\

We can re-write $F$ in the form \\

$$\begin{pmatrix}
0&b&e\\
0&c&f\\
1&0&0\\
\end{pmatrix}$$\\

where both $b$ and $f$ are non-zero. Then, the following right-multiplication can give us the matrix $G$.
$$
G = 
\begin{pmatrix}
0&b&c\\
0&e&f\\
1&0&0\\
\end{pmatrix}
\begin{pmatrix}
1&0&0\\
0&1&\frac{-c}{b}\\
0&\frac{-e}{f}&1\\
\end{pmatrix}=
\begin{pmatrix}
0&b-\frac{ce}{f}&0\\
0&0&f - \frac{ec}{b}\\
1&0&0\\
\end{pmatrix}$$\\

As the determinant one condition is preserved, then $b-\frac{ce}{f}$ and 
$f - \frac{ec}{b}$ are non-zero. We can re-write $G$ in the form \\

$$
G = 
\begin{pmatrix}
0&b&0\\
0&0&f\\
1&0&0\\
\end{pmatrix} = 
\begin{pmatrix}
0&b&0\\
0&0&\frac{1}{b}\\
1&0&0\\
\end{pmatrix}
$$\\

where above we use the fact that $b$ and $f$ are non-zero and $\mathrm{det}(G) = bf = 1$. A final right-multiplication below yields the generator $Y$.


$$\begin{pmatrix}
0&b&0\\
0&0&\frac{1}{b}\\
1&0&0\\
\end{pmatrix}
\begin{pmatrix}
1&0&0\\
0&\frac{1}{b}&0\\
0&0&b\\
\end{pmatrix}=
\begin{pmatrix}
0&1&0\\
0&0&1\\
1&0&0\\
\end{pmatrix}
= Y
$$\\

\subsubsection{Generators Proof for $Z$}

As in the previous proof, write $A$ as 

$$A=\begin{pmatrix}
a&b&c\\
d&e&f\\
g&h&i\\
\end{pmatrix}$$
\\

Again, note that at least one of $d$ and $g$ are non-zero. this allows us to left-multiply by a matrix to yield matrix $B$ with its top-left entry being zero, as shown below.

$$
B = 
\begin{pmatrix}
1&j&k\\
0&1&0\\
0&0&1\\
\end{pmatrix}
\begin{pmatrix}
a&b&c\\
d&e&f\\
g&h&i\\
\end{pmatrix}=
\begin{pmatrix}
0&l&m\\
d&e&f\\
g&h&i\\
\end{pmatrix}$$\\

If $d \neq 0$, then let $C = B$. Otherwise, then $g$ is non-zero, so left-multiply by a matrix which adds a multiple of row $3$ to row $2$ to yield matrix $C$ with its left-middle entry non-zero.

$$
C = 
\begin{pmatrix}
1&0&0\\
0&1&j\\
0&0&1\\
\end{pmatrix}
\begin{pmatrix}
a&b&c\\
d&e&f\\
g&h&i\\
\end{pmatrix}=
\begin{pmatrix}
0&l&m\\
d+jg&e+jh&f+ji\\
g&h&i\\
\end{pmatrix}$$\\

We can re-write $C$ in the form \\

$$\begin{pmatrix}
0&b&c\\
d&e&f\\
g&h&i\\
\end{pmatrix}$$\\
 
where $d$ is non-zero. Now, right-multiply by a matrix that turns the second and third entries in the middle column to zero using the first column. Let the new matrix be $D$.

$$
D = 
\begin{pmatrix}
0&b&c\\
d&e&f\\
g&h&i\\
\end{pmatrix}
\begin{pmatrix}
1&j&k\\
0&1&0\\
0&0&1\\
\end{pmatrix}=
\begin{pmatrix}
0&b&c\\
d&0&0\\
g&n&p\\
\end{pmatrix}$$\\

Since $\mathrm{det}(D) = bdp - cdn = 1$, then either $b$ or $c$ must be non-zero. If $b \neq 0$, then let $E = D$. Otherwise, we have $b = 0$ and $c \neq 0$. In this case, we can right-multiply by a matrix that adds a multiple of column $3$ to column $2$ to obtain a matrix $E$ with its top-middle entry non-zero, as shown below. \\

$$
E = 
\begin{pmatrix}
0&b&c\\
d&0&0\\
g&n&p\\
\end{pmatrix}
\begin{pmatrix}
1&0&0\\
0&1&0\\
0&j&1\\
\end{pmatrix}=
\begin{pmatrix}
0&b+cj&c\\
d&0&0\\
g&n+pj&p\\
\end{pmatrix}$$\\

We can re-write $E$ in the form \\

$$
E = 
\begin{pmatrix}
0&b&c\\
d&0&0\\
g&h&i\\
\end{pmatrix}$$\\

where $b$ is non-zero. Now we can right-multiply by a matrix that adds a multiple of column $2$ to column $3$ to obtain the matrix $F$ with its top-right entry being zero.

$$
F = 
\begin{pmatrix}
0&b&c\\
d&0&0\\
g&h&i\\
\end{pmatrix}
\begin{pmatrix}
1&0&0\\
0&1&\frac{-c}{b}\\
0&0&1\\
\end{pmatrix}=
\begin{pmatrix}
0&b&0\\
d&0&0\\
g&h&i - \frac{hc}{b}\\
\end{pmatrix}$$\\

We can re-write $F$ in the form \\

$$
F = 
\begin{pmatrix}
0&b&0\\
d&0&0\\
g&h&i\\
\end{pmatrix}$$\\

We have $\mathrm{det}(F) = -bdi = 1$, so $b$, $d$ and $i$ are non-zero. Now, right-multiply by a matrix that adds a multiple of column $3$ to columns $2$ and left-multiply by a matrix that adds a multiple of row $2$ to row $3$ to obtain the matrix $G$ with the three entries in its bottom row all equal to $i$.

$$
G = 
\begin{pmatrix}
1&0&0\\
0&1&0\\
0&j&1\\
\end{pmatrix}
\begin{pmatrix}
0&b&0\\
d&0&0\\
g&h&i\\
\end{pmatrix}
\begin{pmatrix}
1&0&0\\
0&1&0\\
0&k&1\\
\end{pmatrix}
=
\begin{pmatrix}
0&b&0\\
d&0&0\\
i&i&i\\
\end{pmatrix}
$$\\

Now left-multiply by a matrix to obtain matrix $J$ with its bottom row entries all $-1$ and left-middle entry equal to $1$.

$$
J = 
\begin{pmatrix}
-id&0&0\\
0&\frac{1}{d}&0\\
0&0&\frac{-1}{i}\\
\end{pmatrix}
\begin{pmatrix}
0&b&0\\
d&0&0\\
i&i&i\\
\end{pmatrix}=
\begin{pmatrix}
0&-bid&0\\
1&0&0\\
-1&-1&-1\\
\end{pmatrix}$$\\

Now $\mathrm{det}(J) = -(-bid)(1)(-1) = -bid = 1$, so we have 

$$
J = 
\begin{pmatrix}
0&1&0\\
1&0&0\\
-1&-1&-1\\
\end{pmatrix}
= Z
$$\\

Hence, the generator $Z$ can be attained. This completes the proof that $H$ is a maximal subgroup of $\SL$.

\end{proof} 
%Invariants under conjugation: trace, characteristic equation, determinant, eigenvalues

%First, note that the trace of a matrix is invariant under the order of matrix multiplication, that is, $\text{Tr}(AB) = \text{Tr}(BA)$. It follows that the trace of a matrix is invariant under conjugation. Indeed, $\text{Tr}(X^{-1}MX) = \text{Tr}(XX^{-1}M) = \text{Tr}(M)$. Therefore the set of trace 0 matrices with no eigenvectors is closed under conjugation, and can be split into conjugacy classes.\\
%We suspect that the conjugacy classes are of equal size, $2^5\times 3^2\times 7^3$, however, we only verify this by using the orbit-stabiliser theorem.  [What does this sentence even mean] 
Indeed, consider the group $\SL$ acting on the trace 0 matrices with no eigenvectors by conjugation. The orbit of any given trace 0 matrix is its conjugacy class. By the orbit-stabiliser theorem, the size of this conjugacy class multiplied by the size of the stabiliser of this matrix is the size of the group, $\SL$, which we have already determined. Thus, counting the number of matrices in the stabiliser (which happens to be the centraliser as $X^{-1}MX = M \iff MX = XM$) of a chosen conjugacy class representative is equivalent to counting the number of elements in the conjugacy class. %(this has not been done)






\subsection{Number of solutions to determinant 1 and trace 1 matrices stabiliser }
The number of solutions to the following equation in modulo $7$ must be computed. 
   
    $a^3+b^3+d^3 - 3a^2 d - 3ab^2 + 2db^2 - d^2 b - 3bad = 1$ \\

The equation can be written as a cubic in the variable $a$. \\

    $a^3 - 3da^2 + (-3b^2 - 3bd)a + (b^3 + d^3 + 2db^2 - d^2 b - 1)= 0$ \\


The table below shows the polynomial in $a$ for all $49$ possible pairs of $b, d$ and the corresponding possible values of $a$.
\quad
\begin{center}
    \begin{tabular}{|c|c|c|c|}
    \hline
        \textbf{Value of $b$} & \textbf{Value of $d$} & \textbf{Cubic in $a$} & \textbf{Solutions to $a$} \\
    \hline
         0 & 0 & $a^3+6$ & 1, 2, 4 \\
    \hline
         0 & 1 & $a^3+4a^2$ & 0, 3 \\
    \hline
         0 & 2 & $a^3+a^2$ & 0, 6 \\
    \hline
         0 & 3 & $a^3+5a^2+5$ & 3 \\
    \hline
         0 & 4 & $a^3+2a^2$ & 0, 5 \\
    \hline
         0 & 5 & $a^3+6a^2+5$ & 5 \\
    \hline
         0 & 6 & $a^3+3a^2+5$ & 6 \\
    \hline
         1 & 0 & $a^3+4a$ & 0 \\
    \hline
         1 & 1 & $a^3+4a^2+a+2$ & 2 \\
    \hline
         1 & 2 & $a^3+a^2+5a+1$ & \\
    \hline
         1 & 3 & $a^3+5a^2+2a+3$ & 2 \\
    \hline
         1 & 4 & $a^3+2a^2+6$ & 0, 2, 3 \\
    \hline
         1 & 5 & $a^3+6a^2+3a+5$ & 6 \\
    \hline
         1 & 6 & $a^3+3a^2+3$ & 1, 5 \\
     \hline
         2 & 0 & $a^3+2a$ & 0 \\
    \hline
         2 & 1 & $a^3+4a^2+3a$ & 0, 4, 6 \\
    \hline
         2 & 2 & $a^3+a^2+4a+2$ & 4 \\
    \hline
         2 & 3 & $a^3+5a^2+5a+5$ & 5 \\
    \hline
         2 & 4 & $a^3+2a^2+6a+1$ &  \\
    \hline
         2 & 5 & $a^3+6a^2+3$ & 2, 3 \\
    \hline
         2 & 6 & $a^3 + 3a^2+a+3$ & 4 \\
     \hline
         3 & 0 & $a^3+a+5$ & 1, 3 \\
    \hline
         3 & 1 & $a^3+4a^2+6a$ & 0 \\
    \hline
         3 & 2 & $a^3+a^2+4a+2$ & 4 \\
    \hline
         3 & 3 & $a^3+5a^2+2a+3$ & 2 \\
    \hline
         3 & 4 & $a^3+2a^2+2$ & 4 \\
    \hline
         3 & 5 & $a^3+6a^2+5a+5$ &  \\
    \hline
         3 & 6 & $a^3+3a^2+3a+4$ & \\
     \hline
         4 & 0 & $a^3+a$ & 0 \\
    \hline
         4 & 1 & $a^3+4a^2+3a+1$ &  \\
    \hline
         4 & 2 & $a^3+a^2+5a$ & 0, 1, 5 \\
    \hline
         4 & 3 & $a^3+5a^2+3$ & 4, 6 \\
    \hline
         4 & 4 & $a^3+2a^2+2a+2$ & 1 \\
    \hline
         4 & 5 & $a^3+6a^2+4a+3$ & 1 \\
    \hline
         4 & 6 & $a^3+3a^2+6a+5$ & 3 \\
     \hline
         5 & 0 & $a^3+2a+5$ & 4, 5 \\
    \hline
         5 & 1 & $a^3+4a^2+a+2$ & 2 \\
    \hline
         5 & 2 & $a^3+a^2+2$ & 2 \\
    \hline
         5 & 3 & $a^3+5a^2+6a+4$ &  \\
    \hline
         5 & 4 & $a^3+2a^2+5a$ & 0 \\
    \hline
         5 & 5 & $a^3+6a^2+4a+3$ & 1 \\
    \hline
         5 & 6 & $a^3+3a^2+3a+5$ & \\
    \hline
         6 & 0 & $a^3+4a+5$ & 2, 6 \\
    \hline
         6 & 1 & $a^3+4a^2+2$ & 1 \\
    \hline
         6 & 2 & $a^3+a^2+3a$ & 0 \\
    \hline
         6 & 3 & $a^3+5a^2+6a+5$ & \\
    \hline
         6 & 4 & $a^3+2a^2+2a+2$ & 1 \\
    \hline
         6 & 5 & $a^3+6a^2+5a+4$ & \\
    \hline
         6 & 6 & $a^3+3a^2+a+3$ & 4 \\
    \hline
    \end{tabular}
\end{center}

Altogether, we get $57$ possible triplets $(a, b, c)$ and this is the number of matrices that commute with the given matrix which had trace $0$ and determinant $1$.

%\subsection{More things Yasiru's 2nd Corner}

\section{Background notes}
We put here more background knowledge of a general nature, much of which would be encountered in undergraduate or graduate courses. We still try to keep this parallel to the main article.

\subsubsection{Groups}

The action of group $G$ on $X$ is transitive if $X$ is non-empty and for each pair $x, y$ in $X$ there exists a $g$ in $G$ such that $g \cdot x = y$. \\

Consider a group $G$ acting on a set $X$. The orbit of an element $x$ in $X$ is the set of elements in $X$ to which $x$ can be moved by the elements of $G$. The orbit of $x$ is denoted by $G \cdot x$: 

\[ G \cdot x = \{ g \cdot x: g \in G \}. \]

The set of orbits in $X$ under action of $G$ form a partition of $X$. \\

Given $g$ in $G$ and $x$ in $X$ with $g \cdot x = x$, it is said that "$x$ is a fixed point of $g$" or that "$g$ fixes $x$". For every $x$ in $X$, the stabiliser subgroup of $G$ with respect to $x$ is the set of all elements in $G$ that fix $x$: 

\[ G_x = \{ g \in G : g \cdot x = x \}. \]

This is a subgroup of $G$ (not necessarily normal). \\

There is a bijection between the set $G / G_x$  of cosets for the stabiliser subgroup and the orbit $G \cdot x$, which is the orbit stabiliser theorem. If $G$ is finite then we also have 

\[ | G \cdot x | = |G| / |G_x| \]

\subsubsection{Matrices}

A square matrix $A$ is called diagonalisable if there exists an invertible matrix $P$ and a diagonal matrix $D$ such that 
$P^{-1}AP = D$, or equivalently $A = PDP^{-1}$. The column vectors of $P$ form a basis consisting of eigenvectors of $A$ and the diagonal entries of $D$ are the corresponding eigenvalues ot $T$ with respect to this eigenvector basis. One can raise a diagonal matrix $D$ to a power by simply raising the diagonal entries to that power and the determinant of a diagonal matrix is simply the product of all diagonal entries, such computations generalise easily to $A = PDP^{-1}$. \\

%A square matrix that is not diagonalisable is called defective. It can happen that a matrix $A$ with real entries is defective over the real numbers so the factorisation is impossible where $P$ and $D$ have only real entries, but it is possible for it to be diagonalisable over the complex numbers. \\ 

An $n \times n$ matrix $A$ over a field $F$ is diagonalisable if and only if there exists a basis of $F$ consisting of eigenvectors of $A$. If such a basis has been found, one can form the matrix $P$ having these basis vectors as columns, and $P^{-1}AP$ will be a diagonal matrix whose diagonal entries are the eigenvalues of $A$. \\

An $n \times n$ matrix $A$ is diagonalisable over the field $F$ if it has $n$ distinct eigenvalues in $F$ (i.e. if its characteristic polynomial has $n$ distinct roots in $F$), but the converse is false. E.g. if the eigenvalues are 
$1, 2, 2$ (double root of $2$) then it could be possible that the eigenspace of $A$ associated with eigenvalue $2$ has dimension $2$ (solutions to $Av=\lambda v$ when $\lambda$ is set to $2$). \\

Let $A$ be a matrix over $F$. If it is diagonalisable, then so is any power of it because $A = PDP^{-1}$ implies that 
$A^n = PD^{n}P^{-1}$ where $D^n$ is still a diagonal matrix. \\

%\subsection{Jordan Matrix}
%In the mathematical discpline of matri theory, a Jordan matrix over a rind $R$ whose identities are the zero and one, is a matrix composed of zeroes everywhere except for the diagonal, which is filled with a fixed element $\lambda \in R$, and for the superdiagonal, which is composed of ones. \\

%Any block diagonal matrix whose blocks are Jordan blocks is called a Jordan matrix. It consists of $r$ diagonal blocks, where the first is the Jordan block corresponding to eigenvalue $\lambda_1$, the second for $\lambda_2$ and so on. \\

%\includegraphics[]{yasiru_jayasooriya_files/jordan blocks example.PNG}


%\subsubsection{Jordan normal form}
%In linear algebra, a Jordan normal form, also known as Jordan canonical form, is an upper triangular matrix of a particular form called a Jordan matrix representing a linear operator on a finite-dimensional vector space with respect to some basis. Such a matrix has each non-zero off-diagonal entry equal to 1, immediately above the main diagonal (on the superdiagonal) and with identical diagonal entries to the left and below them. \\

%Let $V$ be a vector space over a field $K$. Then a basis with respect to which the matrix has the required form exists iff all eigenvalues of the matrix lie in $K$, or equivalently if the characteristic polynomial of the operator (linear transformation) splits into linear factors over $K$. The condition is always satisfied if $K$ is algebraically closed (e.g. complex numbers). The diagonal entries of the normal forms are the eigenvalues of the operator, and the number of times each eigenvalue occurs is called the algebraic multiplicity of the eigenvalue. \\

%If the operator is given by a square matrix $M$ then its Jordan normal form is also called the Jordan normal form of $M$. Any square matrix has a Jordan normal form if the field of coefficients is extended to one containing all the eigenvalues of the matrix. In spite of its name, the normal form for a given $M$ is not entirely unique as it is a block diagonal matrix formed of Jordan blocks, the order of which is not fixed. It is conventional to group blocks for the same eigenvalue together but not ordering is imposed among the eigenvalues, nor among the blocks of a given eigenvalue, although the latter could for instance be ordered by weakly decreasing size.


%\includegraphics[]{yasiru_jayasooriya_files/375px-Jordan_blocks.svg.png}

%Not all matrices are diagonalisable, matrices that are not diagonalisable are called defective matrices. Consider the following matrix.

%\includegraphics[]{yasiru_jayasooriya_files/defective_matrix.PNG}

%Including multiplicity, the eigvenvalues are $1, 2, 4, 4$. The dimension of the eigenspace corresponding to eigvenvalue $4$ is $1$ and not $2$ so $A$ is not diagonalisable. However there is an invertible matrix $P$ such that $J = P^{-1}AP$ where 



%The matrix $J$ is almost diagonal. This is the Jordan normal form of $A$. \\

%So we try to find an invertible matrix $P$ such that $P^{-1}AP=J$ is such that the only non-zero entries of $J$ are on the diagonal and the superdiagonal. $J$ is called the Jordan normal form of $A$. 

%\subsubsection{Traces}
%In linear algebra the trace of a square matrix $A$ denoted $tr(A)$. It is defined to be the sum of the elements on the main diagonal from upper left to lower right of $A$. \\ 

%The trace of a matrix is the sum of its complex eigenvalues (counterd with multiplicities), it is invariant with respect to a change of basis. \\

%Sum of eigenvalues -> resemblance to Vieta's formulas? Characteristic equation for eigenvalues.

%\includegraphics[scale = 0.6]{yasiru_jayasooriya_files/matrix_trace_1.PNG}
%\includegraphics[scale = 0.5]{yasiru_jayasooriya_files/matrix_trace_2.PNG}
%\includegraphics[scale = 0.5]{yasiru_jayasooriya_files/matrix_trace_3.PNG}
%\includegraphics[scale = 0.7]{yasiru_jayasooriya_files/matrix_trace_4.PNG}

%Incredible fact - Trace is invariant under conjugation! \\

%\includegraphics[scale = 0.3]{yasiru_jayasooriya_files/trace invariant conjugation.PNG}

\subsubsection{Conjugacy Classes}
In group theory, two elements $a, b$ of a group are conjugate if there is an element $g$ in the group such that $b=g^{-1}ag$. This is an equivalence relation whose equivalence classes are called conjugacy classes. \\

Members of the same conjugacy class cannot be distinguished using only group structure and therefor share many properties. \\

The \textbf{class number} of $G$ is the number of distinct conjugacy classes. All elements belonging to the same conjugacy class have the \textbf{same order}. \\

\begin{itemize}
    \item The identity element is always the only element in its class.
    \item If two elements $a, b$ belong to the same conjugacy class then they have the same order. More generally, every statement about $a$ can be translated into a statement about $b=gag^{-1}$ because the map $x \rightarrow gxg^{-1}$ is an automorphism of $G$.
    \item If $a$ and $b$ are conjugate, so are their powers $a^k$ and $b^k$. Thus, taking $k$th powers gives a map on conjugacy classes and one may conside which conjugacy classes are in its preimage. 
    \item Any element $a \in G$ lies in the centre $Z(G)$ of $G$ if and only if its conjugacy class has only one element, $a$ itself. Recall that the centre is the set of elements that commute with every element of $G$.
    \item More generally, if $C_G(a)$ is the centraliser of $a \in G$ (the subgroup consisting of all elements $g$ such that $ga=ag$), then the index $[G: C_G(a)]$ is equal to the number of elements in the conjugacy class of $a$ by the orbit-stabiliser theorem. Index refers to number of left / right cosets (same as size of group divided by size of centraliser).
    \begin{itemize}
        \item This is because we can make the G-Set $X$ be a conjugacy class of $G$. Then for any $g$ in the entire group, defined $g \cdot x = g^{-1}xg$ where $x$ is in $X$.
        \item The group action is transitive as $X$ is by definition an entire conjugacy class so there is a single orbit of any $x \in X$ whose size is the size of conjugacy class.
        \item The stabiliser subgroup of some fixed $x \in X$ is all $g \in G$ such that $x = g^{-1}xg$ or $gx=xg$ (so all $g$ that commute with $x$).
        \item Since orbit size equals conjugacy class size then by orbit-stabiliser theorem, size of conjugacy class equals size of group divided by size of stabiliser subgroup.
        \item This can be applied to $\PSL$ conjugacy classes.
    \end{itemize}
\end{itemize}

%\subsubsection{Fields}

%In this section let $F$ denote an arbitrary field and $a, b$ be arbitrary elements of $F$. The non-zero elements of $F$ form an abelian group under multiplication, called the multiplicative group. \\

%Every finite subgroup of the multiplicative group of a field is cyclic. \\ 

%Define the product $n \cdot a$ by a positive integer $n$ to be the $n$-fold sum \newline $a + a + \ldots + a$. If there is no positive integer such that $n \cdot 1 = 0$, then $F$ is said to have characteristic $0$ (e.g. the rationals). If there is a smallest positive integer $n$ satisfying the equation then it can be shown that this value must be a prime number $p$, and this is called the characteristic of the field. \\ 

%If $F$ has characteristic $p$ then $p \times a = 0$ for all $a$ in $F$ so this gives us 

%\[ (a+b)^p = a^p + b^p, \]

%since all binomial coefficients are divisible by $p$. Therefore the Frobenius map 

%\[ \text{Fr: } F \rightarrow F, x \longmapsto x^p \]

%is compatible with the addition in $F$ and the multiplication so is a field automorphism. \\

%For every prime number $p$ and every positive integer $k$ there exist fields of order $p^k$, which are all isomorphic. A finite field of order $q$ exists iff $q$ is a prime power $p^k$. A field of order $p^k$ always has characteristic zero. \\

%In a finite field of order $q$, the polynomial $X^q - X$ has all $q$ elements of the field as roots. The multiplicative group of the field is cyclic, so all non-zero elements can be expressed as powers of a single element called the primitive element of the field (in general there are several primitive elements for the given field). \\

%Let $F$ be a finite field. For any element $x$ in $F$ and any integer $n$, denote by $n \cdot x$ the sum of $n$ copies of $x$. Since the field has characteristic $p$. We get that the field is a GF($p$)-vector space. \\

%We have the equality 

%\[ X^p - X = \prod _{a\in {\rm {GF}}(p)}(X-a) \]

%for polynomials in GF($p$). More generally, every element in GF($p^n$) satisfies the polynomial equation $x^{p^n} - x = 0$. \\

%\subsubsection{Field Extensions}

%A field extension is a pair of fields $E \subseteq F$ such that the operations of $E$ are those of $F$ restricted to $E$. The characteristic of a subfield is the same as the characteristic of the larger field. \\

%If $K$ is a subfield of $L$ then $L$ is an extension field or simply extension of $K$, and this pair of fields is a field extension. Such as field extension is denote $ L / K$.

%If $L$ is an extension of $F$, which is in turn an extension of $K$, then $F$ is said to be an intermediate field (or intermediate extension or subextension) or $L / K$.

%Given a field extension $L / K$, the larger field $L$ is a $K-$vector space. The dimension of this vector space is called the degree of the extension and is denote $[L:k]$. \\

%The degree of an extension if $1$ iff the two fields are equal. A finite extension is an extension that has finite degree. \\

%Given two extensions $L / K$ and $M / L$, the extension $M / L$ is finite if and only if both $L / K$ and $M / L$ are finite. In this case, we have 

%\[ [M:K] = [M:L] \cdot [L:K] \]

%Given extension $L / K$ and a subset $S$ of $L$, there is a smallest subfield of $L$ that contains $K$ and $S$. This is the intersection of all subfields of $L$ that contain $K$ and $S$, denoted as $K(S)$. One says that $K(S)$ is the field generated by $S$ over $K$, and that $S$ is a generating set of $K(S)$ over $K$. When $S = \{x_1, \ldots x_n \}$ is finite, one writes $K(x_1, \ldots, x_n)$. One says that $K(S)$ is finitely generated over $K$. If $S$ consists of a single element $s$ then the extension $K(s) / K$ is called a simple extensions and $s$ is called a primitive element of the extension. \\

%In characteristic zero fields, every finite extension is a simple extension. This is the primitive element theorem, which is not true for non-zero characteristic fields. \\

%If a simple extension $K(s) / K$ is not finite, then the field $K(s)$ is isomorphic to the field of rational fractions (same as rational functions) in $s$ over $K$.

%\subsubsection{Field Extension Examples}

%The field 


  %  \begin{align*}
     %   \mathbb{Q}(\sqrt{2}, \sqrt{3}) &= \{ a + b\sqrt{3} | a, b, \in \mathbb{Q}(\sqrt{2})\} \\
      %  &= \{ a + b\sqrt{2} + c \sqrt{3} + d \sqrt{6} | a, b, c, d \in \mathbb{Q} \},
  %  \end{align*}


%is an extension field of both $\mathbb{Q}(\sqrt{2})$ and $\mathbb{Q}$ of degree $2$ and $4$ respectively. It is also a simple extension because one can show that 


  %  \begin{align*}
    %    \mathbb{Q}(\sqrt{2}, \sqrt{3}) &= \mathbb{Q}(\sqrt{2}+\sqrt{3})\\
     %   &= \{ a + b(\sqrt{2} + \sqrt{3}) + c (\sqrt{2} + \sqrt{3})^2 + d (\sqrt{2} + \sqrt{3})^3 | a, b, c, d \in \mathbb{Q} \},
  %  \end{align*}

%It is common to construct an extension field of a given field $F$ as a quotient ring of the polynomial ring $K[X]$ in order to "create" a root for a given polynomial $f(x)$. Suppose for instance that $K$ does not contain any element $x$ with $x^2 = -1$. Then the polynomial $X^2 + 1$ is irreducible in $K[X]$, consequently the ideal generated by this polynomial is maximal, and $L = K[X] / {X^2+1}$ is an extension field of $K$ which does contain an element whose square is $-1$ (namely the residue class of $X$). \\ 

%By iterating the the above construction, one can construct a splitting field of any polynomial from $K[X]$. This is an extension field $L$ of $K$ in which the given polynomial splits into a product of linear factors.  \\ 

%\subsubsection{Algebraic extension}

%An element $x$ is a field extension $L / K$ is algebraic over $K$ if it is a root of a non-zero polynomial with coefficients in $K$. \\ 

%For example $\sqrt{2}$ is algebraic over the rationals because it is a root of $x^2 - 2$. If al element of $x$ of $L$ is algebraic over $K$ then the monic polynomial of lowest degree that has $x$ as a root is called the minimal polynomial of $x$. This minimal polynomial is irreducible over $K$. 

%An element $s$ of $L$ is algebraic over $K$ if and only if the simple extension $K(s) / K$ is a finite extension. In this case the degree of the extension equals the degree of the minimal polynomial and a basis of the $K$-vector space $K(s)$ consists of $1, s, s^2, \ldots , s^{d-1}$ where $d$ is the degree of the minimal polynomial. \\

%The set of elements of $L$ that are algebraic over $K$ form a subextension, which is called the algebraic closure of $K$ in $L$. This results from the preceding characterisation: if $s$ and $t$ are algebraic then the the extensions $K(s) / K$ and $K(s, t) / K(s)$ are finite. Thus, $K(s, t) / K$ is also finite as well as the subextensions $K(s \pm t) / K$, 
%$K(st) / K$ and $K(1/s) / K$. It follows that $s \pm t, st $ and $1/s$ are all algebraic. \\

%An algebraic extension $L / K$ is an extension such that every element of $L$ is algebraic over $K$. Equivalently, an algebraic extension is an extension that is generate by algebraic elements. For example $\mathbb{Q} (\sqrt{2}, \sqrt{3})$ is an algebraic extension of $\mathbb{Q}$, because $\sqrt{2}$ and $\sqrt{3}$ are algebraic over $\mathbb{Q}$. \\

%A simple extension is algebraic if and only if it is finite. This implies that an extension is algebraic if and only if it is the union of its finite subextensions, and that every finite extension is algebraic. 

%\subsubsection{Normal extension}

%An algebraic extension $L / K$ is normal if every irreducible polynomial in $K[x]$ that has a root in $L$ completely factors into linear factors of $L$. Every algebraic extension $F / K$ admits a normal closure $L$, which is an extension field of $F$ such that $L / K$ is normal and which is minimal with this property. \\ 

%These are one of the conditions for algebraic extensions to be a Galois extension. \\

%Let $\overline{F}$ denote the algebraic closure of field $F$ (the algebraic extension of $F$ that is algebraically closed). \\

%Suppose that $L / K$ is an algebraic extension such that $L \subseteq \overline{K}$. Then two equivalent conditions for normal extension are

%\begin{itemize}
 %   \item $L$ is the splitting field of a family of polynomials in $K[X]$.
 %   \item Every irreducible polynomial of $K[X]$ which has a root in $L$ splits into linear factors in $L$.
  %  \item The minimal polynomial over $K$ of every element in $L$ splits in $L$.
  %  \item The group of automorphisms $Aut(L/K)$ of $L$ which fixes elements of $K$, acts transitively on the set of homomorphisms $L \rightarrow \overline{K}$.
%\end{itemize}

%For example, $\mathbb{Q}(\sqrt{2})$ is a normal extension of $\mathbb{Q}$, since it is a splitting field of $x^2-2$. On the other hand, $\mathbb{Q}(\sqrt[3]{2})$ is not a normal extension of $\mathbb{Q}$ since the irreducible polynomial $x^3-2$ has one root in it (namely $\sqrt[3]{2}$, but not all of them (it does not have the non-real cubic roots of $2$). \\

%Recall that the field $\overline{\mathbb{Q}}$ of algebraic numbers contains $\mathbb{Q}(\sqrt[3]{2})$. Since 

%\[   
%\mathbb{Q}(\sqrt[3]{2}) = \{ a + b\sqrt[3]{2} + 
%  c\sqrt[3]{4} \in \overline{\mathbb{Q}} | a, b, c \in \mathbb{Q}\}
%\]

%For a prime $p$, the extension $\mathbb{Q}(\sqrt[p]{2}, \omega)$ is normal of degree $p(p-1)$. It is a splitting field of $x^p-2$, where $w$ is any $p$th root of unity.



%\subsubsection{Separable extension} 

%An algebraic extension $L/K$ is called separable if the minimal polynomial of every element $L$ over $K$ is separable, i.e. has no repeated roots in an algebraic closure over $K$. \\

%A consequence of the primitive root theorem states that every finite separable extension has a primitive element (i.e. is simple). \\

%Given any field extension $L/K$, we can consider its automorphism group $Aut(L/K)$ consisting of all field automorphisms from $L$ to $L$ where all elements $x$ in $K$ are fixed. When the field extension is Galois, this automorphism group is called the Galois group of the extension.

%\subsubsection{The Primitive Element Theorem}

%Looking at finite degree field extensions that can be generated by a single element. Such a generating element is called a primitive element of the field extension, and the extension is called a simple extension in this case. A finite extension is simple if and only if there are only finitely many intermediate fields. \\

%An element is a primitive element of a field extension if every element can be written as a rational function of the element with coefficients in the smaller field. \\

%For example if you take the field extension of $\sqrt{2}$ and $\sqrt{3}$ adjoined to the rationals then this extension field has degree $4$ and can be generated by $\alpha = \sqrt{2} + \sqrt{3}.$ \\

%Every separable field extension of finite degree is simple. 

 
%Any finite field extension of a finite field is separable and simple. That is, if $E$ is a finite field and $F$ is a subfield of $E$ then $E$ is obtained from $F$ by adjoining a single element whose minimal polynomial is separable. \\

%A splitting field of a polynomial with coefficients in a field is the smallest field extension of that field over which the polynomial splits or decomposes into linear factors. \\

%A splitting field of a polynomial $p(x)$ over a field $K$ is a field extension $L$ over which $p$ factors into linear factors 

%\[ p(x) = c\prod _{i=1}^{\deg(p)}(X - a_{i})\]

%where $c \in K$ and for each $i$ we have $ (X - a_i) \in L[X]$ with $a_i$ not necessarily distinct such that the roots 
%$a_i$ generate $L$ over $K$. The extension $L$ is then an extension of minimal degree over $K$ in which $p$ splits. It can be shown that splitting fields exist and are unique up to isomorphism. The amount of freedom in that isomorphism is known as the Galois group of $p$ (if we assume it is separable). \\

%An extension $L$ which is a splitting field for a set of polynomials $p(x)$ over $K$ is called a normal extension of $K$.

%\subsubsection{Fundamental Theorem of Galois Theory}

%A Galois extension is an algebraic field extension $E/F$ that is normal and separable or equivalently, $E/G$ is algebraic, and the field fixed by the automorphism group is precisely the base field $F$. \\ 

%An important theorem of Emil Artin states that for finite extension $E/F$, each of the following statements is equivalent to the statement that $E/F$ is Galois:

%\begin{itemize}
 %   \item $E/F$ is a normal and separable extension.
  %  \item $E$ is a splitting field of a separable polynomial with coefficients in $F$.
   % \item $|Aut(E/F)| = [E:F]$, that is the number of automorphisms equals the degree of the extension. 
%    \item Every irreducible polynomial in $F[X]$ with at least one root in $E$ splits over $E$ and is separable. \item $|Aut(E/F)| \geq [E:F]$, that is the number of automorphisms is at least the degree of the extension.
 %   \item $F$ is the fixed field of a subgroup of $Aut(E)$.
  %  \item $F$ is a fixed field of $Aut(E/F)$.
   % \item There is a one-to-one correspondence between subfields of $E/F$ and subgroups of $Aut(E/F)$.
%\end{itemize}

%There are two basic ways to construct examples of Galois extensions. 

%\begin{itemize}
   % \item Take any field $E$, any subgroup of $Aut(E)$ and let $F$ be a fixed field.
    %\item Take any field $F$, any separable polynomial in $F[X]$ and let $E$ be its splitting field
%\end{itemize}

%For example, adjoining to the raiontals the square root of $2$ gives a Galois extension, white adjoining the cube root of $2$ gives a non-Galois extension. Both these extensions are separable because they have characteristic zero. The first of them is the splitting field of $x^2-2$, the second has normal closure that induces the complex roots of unity and so is not a splitting field. In fact, it has no automorphism other than the identity because it is contained in the reals and $x^3-2$ only has one real root. \\

%The fundamental theory of Galois theory asserts that given a field extension $E/F$ that is finite and Galois, there is a one-to-one correspondence between its intermediate fields and subgroups of its Galois group. \\

%For finite extensions, the correspondence can be described explicitly as follows. 

%\begin{itemize}
 %   \item For any subgroup $H$ of $Gal(E/F)$, the corresponding fixed field, denoted $E^H$, is the set of those elements of $E$ which are fixed by every automorphism in $H$.
  %  \item For any intermediate field $K$ of $E/F$, the corresponding subgroup is $Aut(E/K)$, that is, the set of those automorphisms in $Gal(E/F)$ which fix every element of $K$.
%\end{itemize}

%The fundamental theory says that this correspondence is one-to-one iff $E/F$ is a Galois extension. For example, the topmost field $E$ corresponds to the trivial subgroup of $Gal(E/F)$ and the base field $F$ corresponds to the whole group $Gal(E/F)$. \\

%The notation $Gal(E/F)$ is only used for Galois extensions. If $E/F$ is not Galois then the "correspondence" gives only an injective but not surjective map from the subgroups of $Aut(E/F)$ to the subfields of $E/F$, and a surjective (not injective) map in the reverse direction. In particular, if $E/F$ is not Galois then $F$ is not the fixed field of any subgroup of $Aut(E/F)$. \\

%Properties of the correspondence are below: 

%\begin{itemize}
   % \item It is inclusion-reversing. The inclusion of subgroups $H_1 \subseteq H_2$ holds iff the inclusion of fields $E^{H_1} \supseteq E^{H_2}$ holds.
  %  \item Degrees of extensions are related to orders of groups, in a manner consistent with the inclusion-reversing property. Specifically, if $H$ is a subgroup of $Gal(E/F)$, then $|H| = [E:E^H]$ and $|Gal(E/F)|/|H| = [E^H:F]$.
 %   \item The field $E^H$ is a normal extension of $F$ (or equivalent a Galois extension, since any subextension of a separable extension is separable) iff $H$ is a normal subgroup of $Gal(E/F)$. 
  %  \item See the two examples here: \url{https://en.wikipedia.org/wiki/Fundamental_theorem_of_Galois_theory} 
%\end{itemize}

%\subsubsection{Solvable Groups}
%A solvable group can be constructed from abelian groups using extensions. \\

%A group is called solvable if there exist subgroups $1 = G_0 < G_1 < \ldots < G_k = G$ such that 
%$G_{j-1}$ is normal in $G_j$ and that $G_j / G_{j-1}$ is an abelian group for $j = 1, 2 \ldots k$.\\

%For finite groups, an equivalent definition is that a solvable group is a group with a composition series all of whose factors are cyclic groups of prime order. This is equivalent because a finite group has finite composition length, and every simple abelian group is cyclic of prime order. \\

%A polynomial equation is solvable in radicals if and only if its corresponding Galois group is solvable (theorem only holds in characteristic zero). This means associated to a polynomial 
%$f \in F[x]$ there is a tower of field extensions 

%\[
% F = F_0 \subseteq F_1 \subseteq F_2 \subseteq \ldots \subseteq F_m = K
%\]

%such that 

%\begin{itemize}
 %   \item $F_i = F_{i-1}[\alpha_i]$ where ${a_i}^{m_i} \in F_{i-1}$ so $\alpha_i$ is a solution to the equation $x^{m_i} - a$ where $a \in F_{i-1}$ 
  %  \item $F_m$ contains a splitting field for $f(x)$.
 %   \item See the example over here: $a = \sqrt[5]{\sqrt{2}+\sqrt{3}}$ 
%\end{itemize}

%\subsection{PSL Group Specifics}

%IMPORTANT RESOURCE: groups subwiki

%\newline
%Looking at the structure of special linear group of degree three over a finite field. Take $q$ as the number of elements in the field and $p$ as the underlying prime number so $q$ is a power of $p$.

%\begin{itemize}
 %   \item \textbf{Number of conjugacy classes:} If $q$ is not $1$ modulo $3$ then there are $q^2+q$ of them.
  %  \item If $q$ is $1$ modulo $4$ there are $q^2+q+8$ of them.
   % \item The order of the group is $q^3(q^3-1)(q^2-1)$ which can be factorised more.
   % \item Exponent (UNKNOWN???)
   % \item The exponent of a group is the least common multiple of the orders of all elements of the group. If there is no least common multiple then the exponent is taken as infinity / zero.
%\end{itemize}

%\subsubsection{Number of conjugacy classes}

%The number of conjugacy classes in special linear group of fixed degree over finite field is PORC (polynomial on residue class) function of field size. \\

%Let $n$ be a natural number. Then there exists a PORC function $f$ of degree $n-1$ such that for any prime power $q$, the number of conjugacy classes in the special linear group $SL(n, q)$ (the special linear group of degree $n$ over the finite field of size $q$ ) is $f(q)$.

%A PORC function is a polynomial on residue classes - it looks like differet polynomial functions on different congruence classes modulo a particular number. In this case, we only need to consider congruence modulo $n$ to define this function. For a field size of $q$, the polynomial depends only on the value of $gcd(n, q-1)$. \\

%The degree of all the polynomials are $n-1$. The polynomials are always monic, no common factors to all polynomials. \\

%IS IT TRUE THAT THE COEFFICIENTS OF POLYNOMIAL ARE INTEGERS? - UNKNOWN?????? Are coefficients always positive? - UNKNOWNNNNNN???? \\

%In this case the value of $n$ is $3$ so look at $gcd(3, q-1)$. \\

%When $gcd(3, q-1) = 1$, $q$ can be a power of $3$ or is $2$ mod $3$. The number of conjugacy classes is $q^2+q$. In this case there is an isomorphism between linear groups when degree power map is bijective so $SL(3, q) \equiv PGL(3, q) \equiv PSL(3, q)$.

%\subsubsection{General strategy and summary}
%Element structure of a special linear group over finite field, conjugacy class of elements with semisimple generalised Jordan block does not split in special linear group over finite field, splitting criterion for conjugacy classes in special linear group of prime degree over finite field. \\

%In our case we have $q$ is $1$ mod $3$.

%\subsubsection{Summary of field size 1 mod 3}

%\begin{itemize}
 %   \item One conjugacy class is matrices diagonalisable over $F_q$ with equal diagonal entries hence a scalar - eigenvalues are $a$ (triple multiplicity) where $a^3 =1$. Characteristic polynomial is $(x-a)^3$ for $a$ eigenvalue. \\
    
  %  Conjugacy class size is $1$ and there are $3$ such conjugacy classes.
   % \item Diagonalisable over $F_q$ with one eigenvalue of multiplicity two, other eigenvalue multiplicity $1$. Eigenvalues are $a$ (double multiplicity), $1/{a^2}$ (single multiplicity). \\
    
    %Conjugacy class size is $q^2(q^2+q+1)$ and number of classes is $q-4$.
    
    %\item Diagonalisable over $F_q$ with distinct diagonal entries - distinct three eigenvalues where $abc=1$. \\
    
    %Class size is $q^3(q+1)(q^2+q+1)$ and there are $(q^2-5q+10)/6$ classes.
    
    %\item Diagonalisable over $F_{q^3}$ but not $F_q$ - distinct Galois conjugate triples in $q^3$ size field of norm $1$. If one of the elements is $a$ then the other two are $a^q$ and $a^{q^2}$ (by the Frobenius lemmas I believe), so we must have 
    %$a^{q^2+q+1} = 1$. \\
    
    %Characteristic polynomial - irreducible degree three polynomial over $F_q$. \\
    
    %Size of class is $q^3(q-1)^2(q+1)$ and number of classes is $(q-1)(q-2)/3$.
    
   % \textbf{TO BE CONTINUED}
%\end{itemize}

%\subsubsection{Brauer Website }

%Linear group $L_3(7)$. \\

%Order is $1876896 = 2^5 \times 3^2 \times 7^3 \times 19$. \\

Standard geneartors of $L_3(7)$ are $a, b$ where $a$ has order $2$, $b$ has order $3$, $ab$ has order $19$ and $ababb$ has order $6$. \\

%\includegraphics[]{yasiru_jayasooriya_files/L3 7 conj classes.PNG}


\subsubsection{Calculating entire group size} 
Using the orbit stabiliser theorem. Consider the vector 

$v = \begin{pmatrix}
1\\
0\\
0\\
\end{pmatrix}
$.

Let the set $X$ be all $7^3 - 1$ column non-zero vectors. Let $M$ be a matrix in the entire group. Let $w$ be an element of $X$. Note that $Mv = w$ whenever $M$ has its first column the same as $w$ (we can always find such an $M$ with determinant one).

This means that the orbit of $v$ is the entire set $X$. 

The stabiliser of $v$ is all matrices $M$ of the form

$$M = \begin{pmatrix}
1&a&b\\
0&c&d\\
0&e&f\\
\end{pmatrix}$$

with determinant $cf-de = 1$. So to find stabiliser size find the number of $2 \times 2$ matrices (bottom-right) which have determinant $1$. 

Can do similar computation (inductively e.g. with orbit stabiliser theorem) to get that the number of $2 \times 2$ matrices with determinant $1$ is $7\times (7^2-1)$. 

We can freely choose $a, b$ in $M$ so now we have $7^2\times 7 \times (7^2 -1)$ ways for stabiliser size. Do orbit size times stabiliser size to get $(7^3-1)(7^3)(7^2-1)$ ways



%\subsection{Looking at specific case for trace 0}
%Consider the matrix 

%$$M = \begin{pmatrix}
%0&b&c\\
%d&0&f\\
%g&h&0\\
%\end{pmatrix}
%$$

%\subsection{Traces things November 4 2021}

%Looking at when traces are zero but entries are non-zero? 

%Can do some cases etc. 

%Look at number of matrices where trace = something, determinant = 1, no eigenvalues. 

%The maximal subgroup is certainly the following (with determinant 1).

%\includegraphics[]{yasiru_jayasooriya_files/maximal subgroup.PNG}

%This sends the vector $<1, 0, 0>$ to multiples of it  $<a, 0, 0>$. \\

%This is the group of matrices that stabilise the line spanned by $<1, 0, 0>$. \\

%If a matrix stabilises something then it belongs to a certain subgroup.



%\includegraphics[]{yasiru_jayasooriya_files/trace a e i all zero part 3.jpeg}
%\includegraphics[]{yasiru_jayasooriya_files/trace a e i all zero part 4.jpeg}

%\includegraphics[scale = 0.2]{yasiru_jayasooriya_files/nov4 2021 lesson part2.jpeg}
%\includegraphics[scale = 0.2]{yasiru_jayasooriya_files/nov4 2021 lesson part3.jpeg}

%which has trace zero. 






%\subsection{MORE THINGS TO RESEARCH FOR MY CORNER}

Q%uotient subspaces in linear algebra eigenvectors etc.

%Upper triangular matrix look at how it acts on subspaces.

%\subsection{Recordings of videos for reference}
%\begin{itemize}
 %   \item November 4 2021 - \url{https://www.youtube.com/watch?v=UUDLBPlxwpc&} -> eigenvector problem $\SL$ man versus machine, traces stuff, recap of equations for no eigenvalues with the lambdas.
  %  \begin{itemize}
   %     \item Look at matrices with a lot of ones, no eigenvectors.
    %    \item E.g. trace = 0 when diagonals are all zero -> a = e = i = 0.
     %   \item "Traces are not a bad way to classify things because traces are an invariant"
      %  \item whiteboard link is \url{https://r8.whiteboardfox.com/81832352-5898-2923}
       % \item Basic property of traces.
    %\end{itemize}
%\end{itemize}

\subsection{Remaining items copied}



%The stabiliser calculations carried out by hand


Maximal subgroups 

Matrices with exactly one eigenvalue
- what are the conjugacy classes of matrices whose eigenvalues are all 1.


Matrices with two distinct eigenvalues
-which conjugacy classes of matrices have eigenvalues two -1s and one 1.

generators a order 2 b order 3

%What are the orders of the matrices without eigenvalues



\section*{Acknowledgements} We are very grateful for the encouragement and support of the parents of our talented young students many of whom are still in high school.
We would also like to acknowledge the many helpful online resources such as wikipedia (esp groupprops subwiki), ATLAS, open access initiative, Mathoverflow, mathstackexchange, overleaf, discord, youtube


\begin{enumerate}
    \item \url{http://brauer.maths.qmul.ac.uk/Atlas/v3/lin/L37/}
    \item INVERSE GALOIS PROBLEM FOR SMALL SIMPLE GROUPS DAVID ZYWINA
    \item \url{https://bit.ly/3uhb7J1}
    \item \url{https://groupprops.subwiki.org/w/index.php?title=Element-structure-special-linear-group-of-degree-three-over-a-finite-field&mobile-action=toggleviewmobile}

\end{enumerate}
%Gavin adding 'a new definition to the meaning of attending a class' - Dr Michael Sun, 2021

\begin{thebibliography}{10}
%is anyone else doing anything?
   %{\sc D.~Altschuler, M.~Bauer, C.~Itzykson}, 
   %{\em The branching rules of conformal embeddings}, 
   %Comm.Math.Phys {\bf 132} (1990), no. 2, 349-364.
%	\bibitem{M1873}  
  %{\sc E.~Mathieu cite the sl3 thing}, 
 % {\em Sur la fonction cinq fois transitive de 24 quantités}, 
 % Journal de mathématiques pures et appliquées 2e série, tome 18 (1873), p. 25-46.
 \bibitem{SL3gen}
  {\sc M.~Conder, E.~Robertson, P.~Williams}
  {\em Presentations for 3-Dimensional Special Linear Groups Over Integer Rings}
  Proceedings of the American Mathematical Society {\bf 115} (1992), No.1, 19-20.
 
 %  Annals of Mathematics {\bf 162} (2005), 581--642. 
   %\bibitem{Fr}  
  % {\sc I.B.~Frenkel}, 
  % {\em Representaions of affine Lie algebras, Hecke modular forms and Korteweg de Vries type equations},
  % Lecture Notes in Mathematics {\bf 933} (1981), 71-109.     
   %\bibitem{FFRS}  
   %{\sc J.~Frohlich, J.~Fuchs, I.~Runkel, C.~Schweigert}, 
   %{\em Correspondences of ribbon categories}, 
   %Adv. Math. {\bf 199} (2006), No. 1, 192--329. 
   %\bibitem{Fuchs} 
   %{\sc J.~Fuchs}, 
   %{\em The connections between Wess-Zumino-Witten models and free field theories}, Nucl. Phys. B (Proc. Suppl.) {\bf 6} (1989) 157--159.
   %{\sc J.E.~Hasegawa} , 
   %{\em Spin Module Versions of Weyl's Reciprocity Theorem for Classical Kac-Moody Lie Algebras}, Publ. RIMS Kyoto Univ. {\bf 25} (1989) 741--828.
   %\bibitem{Jónsson}  


 \end{thebibliography}

{\sc Dr Michael Sun's School of Maths, NSW, Australia}


Corresponding Author: Yasiru Jayasooriya

\email{yasiruj2000@gmail.com}


\end{document}
 
Annales Academiae Scientarum Fennicae
Annals of Mathematics II: Series
Applied Mathematics E-Notes
Archivum Mathematicum - Available only in PostScript format

Bulletin, Classes des Sciences Mathematiques et Naturelles, Sciences
Boletin Asociacio Matematica Vanezolana - English
Collectanea Mathematica - University of Barcelona

Documenta Mathematica - Also available in German
European Journal of Combinatorics
Expiremental Mathematics
Electronic Journal of Linear Algebra
Electronic Journal of Combinatorics

Electronic Journal of Undergraduate Mathematics @ Furman University
Electronic Research Announcements of the American Mathematical Society
Electronic Transactions on Numerical Analysis
Indiana University Mathematics Journal
Integers: Electronic Journal of Combinatorial Number Theory
Journal of Formalized Mathematics


Journal of Lie Theory

Le Journal de Maths des Eleve
Lobachevskii Journal of Mathematics
Matematicki Vesnik- Russian but in English
New York Journal of Mathematics
Pacific Journal of Mathematics
Southwest Journal of Pure and Applied Mathematics - Published twice a year
Scientiae Mathematicae - Japanese Association of Mathematical Science
Rose-Hulman Institute of Technology Undergraduate Math Journal
The Visual Math Journal - Complex shapes and objects and their relation to math
The Asian Journal of Mathematics
The Morehead Electronic Journal of Applicable Mathematics

Ulam Quarterly Journal