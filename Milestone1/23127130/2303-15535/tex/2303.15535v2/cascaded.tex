\documentclass[letterpaper, 10 pt, conference]{ieeeconf}

\IEEEoverridecommandlockouts                              

\title{A Compositional Approach to Certifying the Almost \\ Global Asymptotic Stability of Cascade Systems}
\author{Jake Welde, Matthew D. Kvalheim, and Vijay Kumar
\thanks{
	J. Welde and V. Kumar are with the GRASP Laboratory at the University of Pennsylvania, while M. D. Kvalheim is with the Department of Mathematics at the University of Michigan. emails: \texttt{\{jwelde,kumar\}@seas.upenn.edu}, \texttt{kvalheim@umich.edu}.  
	We gratefully acknowledge the support of Qualcomm Research, NSF Grant CCR-2112665, and the NSF Graduate Research Fellowship Program.
}
}


\date{\today}
\let\proof\relax 
\let\endproof\relax 

\usepackage{amsthm}
\usepackage{amsmath}
\usepackage{tikz}
\usetikzlibrary{shapes,arrows}

\usepackage{amssymb}
\usepackage{thmtools}
\usepackage{graphicx}
\usepackage{mathtools}
\let\labelindent\relax
\usepackage[shortlabels]{enumitem}

\newcommand{\Lim}[1]{\raisebox{0.5ex}{\scalebox{0.8}{$\displaystyle \lim_{#1}\;$}}}
\declaretheoremstyle[notefont=\normalfont\itshape,bodyfont=\normalfont]{normaltext}
\declaretheorem[name=System,style=normaltext]{system}

\newtheorem{theorem}{Theorem}
\newtheorem{conjecture}{Conjecture}
\newtheorem{corollary}{Corollary}
\declaretheorem[name=Definition,style=normaltext]{definition}
\declaretheorem[name=Remark,style=normaltext]{remark}

\newtheorem{innercorollary}{Corollary}
\newenvironment{restate_corollary}[1]
{\renewcommand\theinnercorollary{#1}\innercorollary}
{\endinnercorollary}


\newcommand{\R}{\mathbb{R}}

\declaretheoremstyle[notefont=\normalfont\itshape,bodyfont=\normalfont\itshape,headfont=\normalfont\sc]{smallcapsname}
\declaretheorem[name=Step,style=smallcapsname]{step}
\newtheorem{proposition}{Proposition}

\usepackage{cite}

\usepackage{balance}


\renewcommand\qedsymbol{$\blacksquare$}

\begin{document}

\makeatletter
\@addtoreset{step}{theorem}
\makeatother

\maketitle

\begin{abstract}
In this work, we give sufficient conditions for the almost global asymptotic stability of a cascade in which the inner loop and the \textit{unforced} outer loop are each almost globally asymptotically stable. Our qualitative approach relies on the absence of chain recurrence for non-equilibrium points of the unforced outer loop, the hyperbolicity of equilibria, and the precompactness of forward trajectories. We show that the required structure of the chain recurrent set can be readily verified, and describe two important classes of systems with this property. 
We also show that the precompactness requirement can be verified by growth rate conditions on the interconnection term coupling the subsystems. Our results stand in contrast to prior works that require either \textit{global} asymptotic stability of the subsystems (impossible for smooth systems evolving on general manifolds), time scale separation between the subsystems, or strong disturbance robustness properties of the outer loop. The approach has clear applications in stability certification of cascaded controllers for systems evolving on manifolds.%
\end{abstract}









\section{Introduction}

In this work, we are interested in the asymptotic stability of cascade systems in the form
\begin{subequations}
	\begin{align}
		\dot{x} &= f(x,y), \label{outer_general} \\ 
		\dot{y} &= g(y), \label{inner_general}
	\end{align}
\end{subequations}
depicted graphically in in Fig. 1 as  system $\Sigma$. Throughout, we assume that $x$ and $y$ evolve on $X$ and $Y$, which are smooth, connected Riemannian manifolds without boundary (see Remark \ref{choice_of_setting} for an explanation of the choice of setting). We call \eqref{inner_general} the ``inner loop'' and \eqref{outer_general} the ``outer loop''. 



Cascades appear in many interesting and important physical systems. For example, many underactuated mechanical systems can be rendered as a cascade after a feedback transformation \cite{Olfati2000}, and cascade structures appear often in robotic systems, either intrinsically \cite{Murray1997} or after control design \cite{Lee2010}. A long research tradition has studied the implications of cascaded structure to simplify analysis and aid in control \cite{Sepulchre2012,Jankovic1996,Kokotovic2001}. This compositional approach is motivated by the observation that control design for a subsystem is typically easier, due to e.g. lower dimensionality, lower relative degree, or full actuation. The primary difficulty that arises is to ensure that the full system combining these subsystems achieves the desired behavior, since only \textit{local} (as opposed to \textit{global}) asymptotic stability is preserved under cascades for general nonlinear systems 
\cite{Sepulchre2012}.
\begin{figure}
	\tikzstyle{block} = [draw, rectangle, 
	minimum height=3em, minimum width=6em]
	\tikzstyle{sum} = [draw, circle, node distance=1cm]
	\tikzstyle{input} = [coordinate]
	\tikzstyle{output} = [coordinate]
	\tikzstyle{pinstyle} = [pin edge={to-,thin,black}]
	
	\centering
	
	\textbf{Cascade System} 
	
	\
	
	\begin{tikzpicture}[auto, node distance=2cm,>=latex']
		
		\node [block, minimum width={26pt},minimum height={20pt}] (inner) {$g(\hspace{1pt}\cdot\hspace{1pt})$};
		\node [draw, ellipse, right of=inner, node distance=1.4cm, inner sep=2pt] (inner_integral) {$\int$};
		\node [above of=inner, node distance=.65cm, inner sep=0pt, outer sep=0pt] (inner_top) {};
		\node [right of=inner_integral, node distance=.5cm, ellipse, fill=black, minimum height=2.5pt, minimum width=2.5pt, inner sep=0pt] (inner_end) {};
		
		\node [left of=inner, node distance=1.25cm] (system1) {\large $\Sigma:$};
		
		\node [block, right of=inner_integral, minimum width={34pt},minimum height={20pt}] (outer) {$f(\hspace{1pt}\cdot\hspace{1pt},\hspace{-.5pt}\cdot\hspace{1pt})$};
		\node [draw, ellipse, right of=outer, node distance=1.4cm, inner sep=2pt] (outer_integral) {$\int$};
		\node [above of=outer, node distance=.65cm, inner sep=0pt, outer sep=0pt] (outer_top) {};
		\node [right of=outer_integral, node distance=.4cm, outer sep=0pt, inner sep=0pt] (outer_end) {};
		
		
		
		\draw [->] (inner_integral) -- (inner_end) |-  (inner_top) node[right, pos=0.2] {$y$} -- (inner);	
		\draw [->] (inner_end) -- (outer);	
		\draw [->] (inner) -- node[name=u] {$\dot{y}$} (inner_integral);
		
		\draw [->] (outer_integral) -- (outer_end) |-  (outer_top) node[right, pos=0.25] {$x$} -- (outer);	
		\draw [->] (outer) -- node[name=u] {$\dot{x}$} (outer_integral);
		
	\end{tikzpicture}
	

	\vspace{4pt}	
	\textbf{Subsystem Decomposition} 
	
	\vspace{6pt}
	\begin{tikzpicture}[auto, node distance=2cm,>=latex']
		
		\node [block, minimum width={26pt},minimum height={20pt}] (inner) {$g(\hspace{1pt}\cdot\hspace{1pt})$};
		\node [draw, ellipse, right of=inner, node distance=1.4cm, inner sep=2pt] (inner_integral) {$\int$};
		\node [above of=inner, node distance=.65cm, inner sep=0pt, outer sep=0pt] (inner_top) {};
		\node [right of=inner_integral, node distance=.5cm, outer sep=0pt, inner sep=0pt] (inner_end) {};
		
		\node [left of=inner, node distance=1.25cm] (system1) {\large $\Sigma_y:$};
		
		\draw [->] (inner_integral) -- (inner_end) |-  (inner_top) node[right, pos=0.2] {$y$} -- (inner);	
		\draw [->] (inner) -- node[name=u] {$\dot{y}$} (inner_integral);
		
	\end{tikzpicture}
	
	\vspace{6pt}
	\begin{tikzpicture}[auto, node distance=2cm,>=latex']
		
		\node [block, minimum width={34pt},minimum height={20pt}] (inner) {$f(\hspace{1pt}\cdot\hspace{1pt},\hspace{-.5pt}\cdot\hspace{1pt})$};
		\node [draw, ellipse, left of=inner, node distance=1.3cm, inner sep=0pt, minimum height=16pt, minimum width=16pt] (eql) {\scriptsize $0_Y$};		
		\node [draw, ellipse, right of=inner, node distance=1.4cm, inner sep=2pt] (inner_integral) {$\int$};
		\node [above of=inner, node distance=.65cm, inner sep=0pt, outer sep=0pt] (inner_top) {};
		\node [right of=inner_integral, node distance=.5cm, outer sep=0pt, inner sep=0pt] (inner_end) {};
		
		\node [left of=eql, node distance=1.25cm] (system1) {\large $\Sigma_x:$};
		
		\draw [->] (inner_integral) -- (inner_end) |-  (inner_top) node[right, pos=0.2] {$x$} -- (inner);	
		\draw [->] (inner) -- node[name=u] {$\dot{x}$} (inner_integral);
		\draw [->] (eql) -- (inner);	
	\end{tikzpicture}
	
	\vspace{0pt}
	\begin{tikzpicture}[auto, node distance=2cm,>=latex']
		
		\node [block, minimum width={34pt},minimum height={20pt}] (inner) {$f(\hspace{1pt}\cdot\hspace{1pt},\hspace{-.5pt}\cdot\hspace{1pt})$};
		
		\node [block, minimum width={34pt},minimum height={20pt}, below of=inner, node distance=1.25cm] (zero) {$f(\hspace{1pt}\cdot\hspace{1pt},\hspace{-.5pt}\cdot\hspace{1pt})$};
		\node [left of=zero, node distance=2.75cm, inner sep=1pt, minimum height=20pt, minimum width=20pt] (input) {$y$};		
		
		\node [draw, ellipse, left of=inner, node distance=1.3cm, inner sep=0pt, minimum height=16pt, minimum width=16pt] (eql) {\scriptsize $0_Y$};		
		\node [sum, right of=inner, node distance=1.4cm, inner sep=4pt] (inner_integral) {};
		\node [above of=input, node distance=1.9cm,  inner sep=1pt, minimum height=20pt, minimum width=20pt] (inner_top) {$x$};
		\node [right of=inner_top, node distance=.8cm, ellipse, fill=black, minimum height=2.5pt, minimum width=2.5pt, inner sep=0pt] (inner_button) {};
		\node [below of=inner_button, node distance=1.26cm, inner sep=0pt, outer sep=0pt] (input_corner) {};
		
		\node [right of=inner_integral, node distance=.5cm, outer sep=0pt, inner sep=0pt] (inner_end) {};
		
		\node [left of=eql, node distance=2.25cm] (system1) {\large $\Sigma_h:$};
		
		\draw [->] (inner_top) -| (inner);	
		\draw [->] (inner) -- node[name=u, pos=.8] {\scriptsize $-$} (inner_integral);
		\draw [->] (inner_integral) -- node[pos=1, right] {$h(x,y)$} +(1,0);
		\draw [->] (eql) -- (inner);	
		\draw [->] (input)  -- (zero);	
		\draw [->] (zero) -| node[name=u, pos=.95, right] {\scriptsize $+$}  (inner_integral);	
		\draw [->] (inner_button) -- (input_corner) -| (zero);	
	\end{tikzpicture}
	
	
	
	\caption{We 
		give sufficient conditions for the almost global asymptotic stability of 
		a cascade system $\Sigma$
		in terms of qualitative properties of the closed loop systems $\Sigma_y$ (the ``inner loop'') and $\Sigma_x$ (the unforced ``outer loop'') as well as growth rate conditions on 
		the stateless I/O system
		 $\Sigma_h$ (the ``interconnection term''). Above, $0_Y$ is the stable equilibrium of $\Sigma_y$.
}
	\label{block_diagrams}
	\vspace{-16pt}
\end{figure}




\subsection{Prior Work on Cascade Stability}




Approaches to stability certification for nonlinear cascades 
have exploited a wide range of structural features. Singular perturbation techniques \cite{Saksena1984,Sreenath2013} assume a time scale separation between the ``fast'' inner loop and the ``slow'' outer loop, and show that a system's behavior tends toward that of a ``reduced'' system as the ratio between convergence rates tends to zero.
However, this approach necessitates rapid inner loop convergence (which can be especially challenging to achieve for a control system with realistic input limitations).
Other approaches have relied on the robustness of the outer loop to disturbances, leveraging the property of \textit{input to state stability}, which roughly requires the asymptotic response of the system under a disturbance input to be bounded 
by the size of the input (and therefore also implies global asymptotic stability of the system in the absence of disturbances) \cite{Sontag1990}. A classic result then establishes the global asymptotic stability of a cascade for which the outer loop is input to state stable and the inner loop is globally asymptotically stable \cite{Sontag1989}.
Methods which avoid robustness and time scale assumptions have relied on the local exponential stability of the inner loop as well as growth restrictions on the ``interconnection term'' to certify global asymptotic stability of cascades with globally asymptotically stable subsystems \cite{Sepulchre2012}.

These global results have important applications, but their utility in geometric control is rather limited, since only manifolds diffeomorphic to $\mathbb{R}^n$ admit smooth globally stable vector fields \cite{Wilson1967}, which the state space of e.g. free-flying robotic systems is not (see \cite{Bhat2000} for a discussion).
In the smooth non-Euclidean setting, the most one can hope for in either the subsystems or the cascade is \textit{almost} global asymptotic stability. 
This fact has motivated an almost global notion of input to state stability \cite{Angeli2004}, in which an asymptotic gain condition holds for all but a measure zero set of initial conditions; a cascade is then guaranteed to be almost globally asymptotically stable if its outer loop is almost globally input to state stable and its inner loop is almost globally asymptotically stable. While almost global input to state stability can be challenging to verify, this can be achieved under certain conditions on the exponential instability of other equilibria as well as the ``ultimate boundedness'' of trajectories of the system under arbitrary disturbances \cite{Angeli2010}. 

However, not all almost globally asymptotically stable cascades have an outer loop enjoying this property; indeed, almost global input to state stability seems to be an inherently stricter property than necessary, since it characterizes the response of the system to arbitrary disturbances, while for our purposes, the outer loop is almost always subjected to a converging disturbance. Yet, the lack of a comprehensive understanding of such systems has required bespoke stability certificates for almost globally asymptotically stable cascaded controllers in practice, inhibiting generalization; for example, a Lyapunov function for the full system may be handcrafted via human intuition even though the cascaded structure originally inspired the control design \cite{Lee2010}.















\subsection{Motivating Example}

\begin{figure}[t]
	\centering
	
	\begin{tikzpicture}[outer sep=0, inner sep=0]
		\node (image) at (0,0) { \includegraphics[width=.33\columnwidth,trim={15cm 6cm 13.5cm 4.5cm},clip] {trajectories/%
				trajectory_1.618_3.4072_1.5977_3.1428.png%
		}};
		\node  at (0.037,-.75) [ellipse, fill=purple, minimum height=2pt, minimum width=2pt, inner sep=0pt] (inner_end) {};
	\end{tikzpicture}%
	\hfill
	\begin{tikzpicture}[outer sep=0, inner sep=0]
		\node (image) at (0,0) { \includegraphics[width=.33\columnwidth,trim={15cm 6cm 13.5cm 4.5cm},clip] {trajectories/%
trajectory_5.5617_4.1329_5.0026_-4.0129.png%
		}};
		\node  at (0.037,-.75) [ellipse, fill=purple, minimum height=2pt, minimum width=2pt, inner sep=0pt] (inner_end) {};
	\end{tikzpicture}%
	\hfill
	\begin{tikzpicture}[outer sep=0, inner sep=0]
		\node (image) at (0,0) { \includegraphics[width=.33\columnwidth,trim={15cm 6cm 13.5cm 4.5cm},clip] {trajectories/%
trajectory_1.4482_3.4431_1.2237_-2.7408.png%
		}};
		\node  at (0.037,-.75) [ellipse, fill=purple, minimum height=2pt, minimum width=2pt, inner sep=0pt] (inner_end) {};
	\end{tikzpicture}%
		\vspace{-6pt}	
	\caption{A sampling of initial conditions and resulting trajectories of the motivating example system \eqref{outer_example}-\eqref{inner_example}, projected down to $\mathbb{T}^2$ from the full state space ${T\mathbb{T}^2 = T\mathbb{S}^1 \times T\mathbb{S}^1}$, where the ``small'' axis of the torus corresponds to $\theta$ and the ``large'' axis corresponds to $\phi$. All sampled trajectories converge to ${(0,0,0,0) \in T\mathbb{T}^2}$, marked in red. Despite the highly energetic and topologically complex behavior of the trajectories, our results certify the almost global asymptotic stability of the system, without the need to construct an explicit Lyapunov function for the full cascade.
 	}	
	\label{trajectories}
	\vspace{-16pt}
\end{figure}

To motivate the question at hand, we present a simple representative example system. Consider a cascade of the form \eqref{outer_general}-\eqref{inner_general} evolving on the tangent bundle of $\mathbb{T}^2$, where 
${x = (\theta,\dot{\theta}) \in T\mathbb{S}^1}$,
${y = (\phi,\dot{\phi}) \in T\mathbb{S}^1}$, and we make the notationally convenient identification ${T\mathbb{S}^1 \cong \mathbb{S}^1 \times \mathbb{R}}$.
The dynamics are given by
\begin{subequations}
	\begin{align}
	\ddot{\theta} &= -(\sin \theta + \dot{\theta})\cos 2 \phi, \label{outer_example}
	\\
	\ddot{\phi} &= -(\sin \phi + \dot{\phi}), \label{inner_example}
\end{align}
\end{subequations}
and a sampling of system trajectories is shown in Fig. \ref{trajectories}.
Using the LaSalle function ${V : (\phi,\dot{\phi}) \mapsto 1 - \cos \phi + \tfrac{1}{2}\dot{\phi}^2}$,
it can be shown that  that ${(\phi,\dot{\phi})=(0,0)}$ is almost globally asymptotically stable for the inner loop \eqref{inner_example}. By the same reasoning, ${(\theta,\dot{\theta})=(0,0)}$ is also almost globally asymptotically stable for the dynamics given by restricting the outer loop \eqref{outer_example} to the stable equilibrium of the inner loop.

In fact, it turns out that the entire cascade \eqref{outer_example}-\eqref{inner_example} is almost globally asymptotically stable, but the system does not satisfy the hypotheses of any of the previously discussed results. In particular, the subsystems are not globally asymptotically stable, nor is there a time scale separation between the loops. Furthermore, viewing $(\phi,\dot{\phi})$ as a disturbance to \eqref{outer_example}, it can be seen that the outer loop is \textit{not} almost globally input to state stable \cite[Def. 2.1]{Angeli2004}, since the response to the bounded disturbance $(\phi,\dot{\phi})=(\pi/2,0)$ grows unbounded from almost all initial conditions.
 Nonetheless, the results of this paper will guarantee the almost global asymptotic stability of a class of systems that includes \eqref{outer_example}-\eqref{inner_example}.






 

In what follows, we certify the almost global asymptotic stability of cascade systems in the form of $\Sigma$ in Fig. 1, using qualitative, dynamical properties of the other subsystems shown. 
In Sec. II, we present the main result, which pertains to cascades in which  $\Sigma_x$ is almost globally asymptotically stable with all chain recurrent points being hyperbolic equilibria,  
and $\Sigma_y$ is almost globally asymptotically stable and locally exponentially stable, and almost all forward trajectories of $\Sigma$ are precompact. We also discuss two broad and important classes of systems enjoying the stated chain recurrence criteria.
In Sec. III, we show that the precompactness criteria (analogous the forward boundedness of trajectories for a system in $\mathbb{R}^n$) can be verified using a growth rate inequality on the interconnection term coupling the subsystems. In Sec. IV, we return to the motivating example, before discussing the results and concluding the paper in Secs. V and VI.
























\section{Almost Global Asymptotic Stability}


We first give a brief review of relevant concepts from dynamical systems theory, adopting
the definitions of \cite{Mischaikow1995}, whose perspective on the behavior of asymptotically autonomous semiflows is a central ingredient of our approach.



\begin{definition}
	A \textit{nonautonomous semiflow} on a smooth Riemannian manifold $(M,\kappa)$ is a continuous map
	\begin{equation}
		{\Phi : \{(t,s) : 0 \leq s \leq t < \infty \} \times M \rightarrow M}
	\end{equation}
	such that ${\Phi(s,s,x) = x}$ and ${\Phi\big(t,s,\Phi(s,r,x)\big) = \Phi(t,r,x)}$ for all ${ t \geq s \geq r > 0}$. A semiflow is called \textit{autonomous} when additionally,
	${\Phi(t+r,s+r,x) = \Phi(t,s,x)}$ for all $r > 0$.
\end{definition}
In the previous, the parameters $s$ and $t$ can be thought of as respective ``start'' and ``end'' times. Hereafter, we will use the shorthands
${	{\Phi^{(t,s)} : M  \rightarrow M}, \, x \mapsto \Phi(t,s,x)}$
for nonautonomous semiflows and 
${	{\Phi^{t} : M  \rightarrow M}, \, x \mapsto \Phi(t,0,x)}$
for autonomous semiflows. 

\begin{definition}
	The \textit{equilibrium set} of an autonomous semiflow $\Phi$ is the set ${\mathcal{E}(\Phi) = \left\{x : \Phi^t(x) = x \ \forall \ t \geq 0\right\}}$. 	
\end{definition}

\begin{definition}
	For an autonomous semiflow $\Phi$ on $(M,\kappa)$ and constants ${\varepsilon, T > 0}$,
	an \textit{$(\varepsilon,T)$-chain} is a pair of finite sequences $(x_0, x_1,\ldots,x_n)$  and $(t_1,t_2,\ldots,t_n)$ satisfying
	\begin{equation}
		{\mathrm{dist}\big(\Phi^{t_i}(x_{i-1}),x_i\big) < \varepsilon}\textrm{ and }{t_i > T}, \ i = 1, 2, \ldots, n,
	\end{equation}
	where ${\mathrm{dist} : M \times M \rightarrow \mathbb{R}}$ is the distance function induced by $\kappa$. A \textit{closed $(\varepsilon,T)$-chain at $x$} has $x = x_0 = x_n$.
\end{definition}
\begin{definition}
	The \textit{chain recurrent set} of an autonomous semiflow $\Phi$ is the set $\mathcal{R}(\Phi)$ consisting of all points at which there exists a closed $(\varepsilon,T)$-chain for all $\varepsilon, T > 0$.
\end{definition}

\begin{remark}
	\label{choice_of_setting}
	We define chain recurrence using $(\varepsilon,T)$-chains with respect to a distance function and some $\varepsilon > 0$ (versus using positive functions for $\varepsilon$  \cite{Hurley1992} or $(\mathcal{U},T)$-chains \cite{Conley1978}) because we rely on the results of \cite{Mischaikow1995}, in which the same choice is made. The choice of a  distance function induced by a Riemannian metric is appropriate for our purposes, since hyperbolic equilibria are locally exponentially stable with respect to any such distance function.
\end{remark}


\begin{definition}
	A nonautonomous semiflow $\Phi$ is \textit{asymptotically autonomous with limit semiflow $\Theta$} if for any sequences 
	${t_j \rightarrow t}$,  ${s_j \rightarrow \infty}$, and ${x_j \rightarrow x}$, 
	\begin{align}		
		\Phi^{(t_j + s_j,s_j)}_{y_0} (x_j)  \to \Theta^t(x) \textrm{ as } j \rightarrow \infty,
	\end{align}
	where $\Theta^t$ is an autonomous semiflow.
\end{definition}

\begin{definition}
	An equilibrium is \textit{almost globally asymptotically stable} if it is (locally) asymptotically stable and its basin of attraction is full measure and residual, i.e. its complement is measure zero and meager (the countable union of nowhere dense sets).
\end{definition}






\subsection{Main Result}



\begin{theorem}[Almost Global Asymptotic Stability of Cascade]
	\label{almost_global_stability}
	Consider the cascade on $X \times Y$ given by
	\begin{subequations}
		\begin{align}
			\dot{x} &= f(x,y), \label{outer_theorem} \\
			\dot{y} &= g(y). \label{inner_theorem} 
		\end{align}
	\end{subequations}
	Suppose the following conditions hold:
\setlist[enumerate,1]{leftmargin=.5cm}
\begin{enumerate}[{1.}] %
		\item 
		${0_Y \in Y}$ is a hyperbolic almost globally asymptotically stable equilibrium of \eqref{inner_theorem}, with basin of attraction $\mathcal{B}_Y$.
		\item 
		${0_X \in X}$ is an almost globally asymptotically stable equilibrium of the dynamics 
		\begin{equation}
			{\dot{x} = f(x,0_Y)},
			\label{unforced}
		\end{equation}
		and all chain recurrent points of \eqref{unforced} are hyperbolic equilibria.
		\item 
		For any ${x_0 \in X}$ and ${y_0 \in \mathcal{B}_Y}$, the forward trajectory of \eqref{outer_theorem}-\eqref{inner_theorem} starting at $(x_0,y_0)$ is precompact.
	\end{enumerate}
	Then, $(0_X,0_Y)$ is almost globally asymptotically stable and locally exponentially stable for the cascade  \eqref{outer_theorem}-\eqref{inner_theorem}.
\end{theorem}



\begin{proof}
	
	Since ${X \times \mathcal{B}_Y}$ is invariant for \eqref{outer_theorem}-\eqref{inner_theorem} and all forward trajectories beginning in ${X \times \mathcal{B}_Y}$ have compact closure, the cascade induces an autonomous semiflow
	\begin{equation}
		\Psi^t : X \times \mathcal{B}_Y \rightarrow X \times \mathcal{B}_Y.
	\end{equation}
	Similarly, \eqref{unforced} and \eqref{inner_theorem} induce the autonomous semiflows 
	\begin{subequations}
			\begin{align}
			\Theta^t : X \rightarrow X, \ & x_0 \mapsto \mathrm{pr}_1 \circ \Psi^t(x_0,0_Y), \label{limit_semiflow} \\
			\Upsilon^t : \mathcal{B}_Y \rightarrow \mathcal{B}_Y, \ & y_0 \mapsto \mathrm{pr}_2 \circ \Psi^t(0_X,y_0), \label{inner_semiflow}
		\end{align}
	\end{subequations}	
	where $\mathrm{pr}_1$ and $\mathrm{pr}_2$ are the natural projections onto $X$ and $Y$, and
	we have carefully chosen the domains of the semiflows.
	We observe that for each initial condition ${y_0 \in \mathcal{B}_Y}$,
	\eqref{outer_theorem} may be interpreted as time-varying dynamics on $X$ given by 
	\begin{equation}
		\dot{x} = 
		 f\big(x,\Upsilon^t(y_0)\big).
	\end{equation}
	In this manner, each initial condition ${y_0 \in \mathcal{B}_Y}$ induces a corresponding \textit{nonautonomous} semiflow on $X$ given by
	\begin{equation}
		\Phi^{(t,s)}_{y_0} : X \rightarrow X, \ x_0 \mapsto \textrm{pr}_1 \circ \Psi^{t-s} \big(x_0,\Upsilon^s(y_0)\big),
		\label{nonautonomous_semiflow}
	\end{equation}
	such that we may also conclude
	\begin{equation}
		\Psi^t(x_0,y_0) = \big(\Phi^{(t,0)}_{y_0}(x_0), \Upsilon^t(y_0) \big).
		\label{flow_decomposition}
	\end{equation}
With these facts in mind, we prove the claim in five steps.

		
	
	
		
		\begin{step}
			${\mathcal{E}(\Psi)
				 = 
				\mathcal{R}(\Theta) \times \{0_Y\}}$,
			and all points in this set are hyperbolic equilibria, of which only ${(0_X,0_Y)}$ is stable.
		\end{step}
		\begin{proof}
		\renewcommand{\qedsymbol}{$\blacktriangledown$}
			All equilibria  ${(x,y) \in {X \times \mathcal{B}_Y}}$ must have ${y = 0_Y}$ by the definition of $\mathcal{B}_Y$ as a basin of attraction, and therefore it must also hold that ${f(x,0_Y) = 0}$ i.e. $x$ must be an equilibrium of \eqref{unforced}. The equality then follows from the assumption
			that ${\mathcal{R}(\Theta) \subseteq \mathcal{E}(\Theta)}$, since equilibria are always chain recurrent i.e. ${ \mathcal{E}(\Theta) \subseteq  \mathcal{R}(\Theta)}$.
			Denoting 
			the vector field on ${X \times Y}$ describing the full cascade \eqref{outer_theorem}-\eqref{inner_theorem} by ${F : (x,y) \mapsto \big(f(x,y),g(y)\big)}$, we may express its linearization at any equilibrium ${(x,0_Y) \in X \times \mathcal{B}_Y}$ 
			as
			\begin{equation}
				\left.d F \right|_{(x,0_Y)} = \begin{bmatrix}
					\left. \partial_x f \,  \right|_{(x,0_Y)} &
					\left. \partial_y f \,  \right|_{(x,0_Y)} \\
					0 &
					\left.\partial_y g \, \right|_{\, 0_Y\hphantom{x,()}} \\
				\end{bmatrix}{.}
			\end{equation}
			Since the eigenvalues of a triangular block matrix are simply the eigenvalues of the blocks on the diagonal, the 
			claim of hyperbolicity follows directly from
			the	hyperbolicity of $0_Y$ for \eqref{inner_theorem} and the hyperbolicity of all equilibria of \eqref{unforced}. Furthermore, an almost globally asymptotically stable system has exactly one stable equilibrium, so at ${x = 0_X}$ all eigenvalues of the top left block have negative real part, but at least one eigenvalue has positive real part at all other equilibria of \eqref{unforced}. Therefore $(0_X,0_Y)$ is locally exponentially stable, while all other equilibria in $X \times \mathcal{B}_Y$ are unstable.
		\end{proof}
		\noindent To complete the proof, it will therefore suffice to show that the stable equilibrium $(0_X,0_Y)$ is almost globally attractive.
		
		
		

	\begin{step}
	For any ${y_0 \in \mathcal{B}_Y}$, the nonautonomous semiflow $\Phi
	_{y_0}$ is asymptotically autonomous with limit semiflow $\Theta$. 
\end{step}
\begin{proof}
		\renewcommand{\qedsymbol}{$\blacktriangledown$}
For any sequences ${t_j \rightarrow t}$,  ${s_j \rightarrow \infty}$, and ${x_j \rightarrow x}$, 
	\begin{align}
		&\begin{aligned}
					\lim_{j \rightarrow \infty} &
			\Phi^{(t_j + s_j,s_j)}_{y_0} (x_j) \\
			&= \lim_{j \rightarrow \infty} \textrm{pr}_1 \circ \Psi^{t_j} \big(x_j,\Upsilon^{s_j}(y_0)\big)
			\label{defn_nonautonomous}
		\end{aligned}
		\\
		&		\phantom{\lim_{j \rightarrow \infty}}
		= \textrm{pr}_1 \circ \Psi^{\lim\limits_{j \rightarrow \infty}  t_j} \left(\lim\limits_{j \rightarrow \infty} x_j,\lim\limits_{j \rightarrow \infty} \Upsilon^{s_j}(y_0)\right) \label{continuity_limit}\\
		&		\phantom{\lim_{j \rightarrow \infty}}
		= 
		\label{solved_limits} 
		\textrm{pr}_1 \circ \Psi^{t} (x,0_Y)
		=
		 %
		\Theta^{t} (x),
	\end{align}
where \eqref{defn_nonautonomous} follows immediately from \eqref{nonautonomous_semiflow}, \eqref{continuity_limit} is obtained by the continuity of $\mathrm{pr}_1$ and $\Psi$, and \eqref{solved_limits} relies on the attractivity of $0_Y$ for $\eqref{inner_theorem}$.
Thus for any ${y_0 \in \mathcal{B}_Y}$, by definition 
$\Phi_{y_0}$ is asymptotically autonomous with limit semiflow $\Theta$.
\end{proof}
		
		\begin{step}
			Every trajectory of $\Psi$ 
			converges to a hyperbolic equilibrium.
			
		\end{step}
		\begin{proof}
		\renewcommand{\qedsymbol}{$\blacktriangledown$}
			Every precompact forward trajectory of an asymptotically autonomous semiflow converges to the chain recurrent set of its limit semiflow \cite{Mischaikow1995}.  
			Thus, Step 2 implies that for each ${y_0 \in \mathcal{B}_Y}$, every trajectory of $\Phi_{y_0}$ converges to $\mathcal{R}(\Theta)$, and asymptotic stability ensures that every trajectory of $\Upsilon$ 
			converges to $0_Y$. Thus, in view of \eqref{flow_decomposition} it is clear that every trajectory of $\Psi$ 
			converges to ${\mathcal{R}(\Theta) \times \{0_Y\}}$, and all points in this set are hyperbolic equilibria by Step 1. Since hyperbolic equilibria are isolated, 
			 continuity implies that every trajectory
			 converges to a particular hyperbolic equilibrium.
		\end{proof}
		\begin{step}
	Almost no trajectories of \eqref{outer_theorem}-\eqref{inner_theorem} converge to an unstable equilibrium.
		\end{step}
		\begin{proof}
		\renewcommand{\qedsymbol}{$\blacktriangledown$}
			By the global stable manifold theorem, all points converging to an unstable hyperbolic equilibrium lie in the union of countably many embedded submanifolds of positive codimension, which is thus a meager set of measure zero. Moreover, all unstable equilibria in ${X \times \mathcal{B}_Y}$ are hyperbolic by Step 1, and there can be only countably many of these equilibria due to the isolation of hyperbolic equilibria and the second countability of the topology of ${X \times B_Y}$. Therefore, the set of all points in ${X \times \mathcal{B}_Y}$ converging to an unstable equilibrium is a countable union of meager sets of measure zero and is thus also meager and measure zero in $X \times Y$. 
		\end{proof}
		\begin{step}
			Almost every trajectory of \eqref{outer_theorem}-\eqref{inner_theorem} converges to the stable equilibrium ${(0_X,0_Y)}$.
		\end{step}
		\begin{proof}
			Since $\mathcal{B}_Y$ is full measure and residual in $N$ by assumption, ${X \times \mathcal{B}_Y}$ is full measure and residual in $X \times Y$. By Step 3, every initial condition in this set converges to a hyperbolic equilibrium, and by Step 4, the subset converging to an unstable equilibrium is meager and measure zero in $X \times Y$. Since the difference of a residual set of full measure by a meager set of measure zero is residual and full measure, the remainder constitutes a residual set of full measure in $X \times Y$ for which all initial conditions converge to the unique stable equilibrium ${(0_X,0_Y)}$, completing the proof.
		\end{proof}
		\renewcommand{\qedsymbol}{}
		\vspace{-20pt}
	\end{proof}

\begin{remark}
	We emphasize that the main potential pitfall of the unforced outer loop being only \textit{almost} globally asymptotically stable is the possibility that the transient may ``funnel'' a non-negligible (i.e. non-meager, positive measure) set to a point $(x,0_Y)$, where $x$ is an unstable equilibrium of \eqref{unforced}. However, such behavior is precluded by the hyperbolicity of all unstable equilibria of \eqref{unforced}.
	This assumption can be relaxed to the requirement that all unstable equilibria of \eqref{unforced} are isolated and have at least one eigenvalue with positive real part, similar to \cite{Angeli2010}. Then, the argument proceeds similarly, but relies on the center-stable manifold theorem instead of the stable manifold theorem.  Similarly, the hyperbolicity assumption on $0_X$ can be relaxed at the cost of local exponential stability. We present the more succinct but less general result for clarity and brevity. 
	\end{remark}


\subsection{Gradient-Like Systems}


Our main result, Theorem \ref{almost_global_stability}, characterizes the stability of a class of cascades in which the chain recurrent points of the unforced outer loop are all equilibria, a somewhat abstract property. Systems with with this property are often called ``gradient-like''. In Appendix A, we present Theorem \ref{th:R-subset-E}, localizing the chain recurrent set of a dynamical system to a subset of state space, provided there exists a function with certain technical properties which is decreasing along trajectories outside that subset. For convenience, here we restate a corollary of that result which is particularly relevant to our present interests, and direct the reader to Appendix A for the full proof. 




\begin{restate_corollary}{2}
				\label{thm:equilibria}
		If $\mathcal{E}(\Phi)$ consists of isolated points and there is a proper\footnote{A function ${V : M \rightarrow \mathbb{R}}$ is \textit{proper} if it has compact sublevel sets, which morally generalizes the notion of ``radially unbounded'' functions on $\mathbb{R}^n$.}, continuous function ${V:M\to \R}$ 
		that is
		decreasing\footnote{A function ${f : \mathbb{R} \rightarrow \mathbb{R}}$ is \textit{decreasing} if ${f(t_2) < f(t_1)}$ whenever ${t_1 < t_2 }$. Note that this does \textit{not} imply $\dot{f}(t) < 0$ for all $t$, e.g. $t \mapsto -t^3$.} along nonequilibrium trajectories,
		then ${\mathcal{R}(\Phi) = \mathcal{E}(\Phi)}$.
\end{restate_corollary}

\begin{remark}
	From Corollary \ref{thm:equilibria}, it is clear that Theorem 1 also holds if its second condition is replaced by the assumption that for the system \eqref{unforced}, all equilibria are hyperbolic and there exists a Lyapunov function around $0_X$ which is decreasing along all non-equilibrium trajectories.\footnote{Some authors \cite{Benaim1995} call this a \textit{strict} Lyapunov function, but the control community tends to reserve this term for Lyapunov functions with strictly negative derivative along non-constant trajectories \cite{Santibanez1997}, a stronger condition.}
\end{remark}


As a matter of fact, certain systems already prominent in the geometric control literature are gradient-like.

\begin{remark}
	\label{summarize_chain_recurrence_section}
	Two important classes of systems verifying Theorem \ref{thm:equilibria} (and the second condition of Theorem \ref{almost_global_stability}) are as follows. 
	It can be shown that for a Riemannian manifold ${(Q,\kappa)}$,
	a strict Rayleigh dissipation $\nu$, and a proper Morse function ${V : Q \rightarrow \left[0,\infty\right)}$ 
	with a unique minimum at 
	${0_Q \in Q}$,
	both the gradient dynamics on $Q$ given by
				\begin{equation}
		\dot{q} = - \mathrm{grad}_\kappa\, V(q) 
		\label{gradient_descent}
	\end{equation}
	and Euler-Lagrange dynamics on $TQ$ given by\footnote{The maps ${\kappa^\flat, \nu^\flat : TQ \rightarrow T^*Q}$ and ${\kappa^\sharp, \nu^\sharp : T^*Q \rightarrow TQ}$ are the \textit{musical isomorphisms} with respect to the Riemannian metrics $\kappa$ and $\nu$ \cite{BulloAndLewis2004}.}
	\begin{equation}
		\overset{\kappa}{\nabla}_{\dot{q}} \dot{q} = - \mathrm{grad}_\kappa\, V(q) - \kappa^\sharp \circ \nu^\flat (\dot{q}) 
		\label{euler_lagrange}
	\end{equation}
	are almost globally asymptotically stable and locally exponentially stable around ${0_Q \in Q}$ and ${0_{TQ} = (0_Q,0) \in TQ}$ respectively, and moreover all chain recurrent points of both systems are hyperbolic equilibria. An early, influential analysis of the previous stability properties is \cite{Koditschek1989}, while a detailed modern treatment can be found in \cite[Chap. 6]{BulloAndLewis2004}. The chain recurrence claim is immediate by Theorem \ref{thm:equilibria} and the fact that the ``potential'' $V$ and the ``total energy''
	\begin{equation}
		\label{total_energy}
		W : (q,\dot{q}) \mapsto V(q) + \frac{1}{2}\kappa(\dot{q},\dot{q}),
	\end{equation}
	are respectively decreasing along all non-equilibrium trajectories of \eqref{gradient_descent} (by construction) and \eqref{euler_lagrange} \cite[Prop. 4.66]{BulloAndLewis2004}. Proofs of these facts can be found in Appendix B.
\end{remark}
	



















\section{Precompactness of Forward Trajectories}

To apply Theorem \ref{almost_global_stability} to examples, we require practical conditions certifying precompactness, since many manifolds of interest in geometric control are noncompact, such as the tangent bundle of any manifold. 
In $\mathbb{R}^n$ or any Riemannian manifold given a complete Riemannian metric, a subset is compact if and only if it is closed and bounded, so in those settings precompactness amounts to boundedness. 
Informally, such an assumption prevents the finite time escape of any trajectory before the inner loop has the chance to converge. 
In this section, we give a growth rate criterion suited to our geometric setting, characterizing the ``{interconnection term}'' (Fig. 1, system $\Sigma_h$).
The result bears similarities to \cite[Thm. 4.7]{Sepulchre2012} certifying boundedness in systems in $\mathbb{R}^n$.








\begin{theorem}[Precompact Forward Trajectories of Cascade]
	\label{boundedness}
	Consider the cascade on $X \times Y$ given by
	\begin{subequations}
		\begin{align}
			\dot{x} &= f(x,y), \label{outer_bound} \\
			\dot{y} &= g(y). \label{inner_bound} 
		\end{align}
	\end{subequations}
	Let ${0_Y \in Y}$ be a stable hyperbolic equilibrium of \eqref{inner_bound} with basin of attraction $\mathcal{B}_Y$,
	and define the {interconnection term}
	\begin{align}
		h : X \times Y \to TX, \, (x,y) \mapsto f(x,y) - f(x,0_Y).
	\end{align}
	Suppose that the following conditions hold:
	\setlist[enumerate,1]{leftmargin=.5cm}
	\begin{enumerate}[{1.}] %
		\item 	The proper, differentiable function 
		\begin{equation}
			W : X \rightarrow \mathbb{R}_{\geq 0}
		\end{equation}
		is a (non-strict) Lyapunov function for the system
		\begin{equation}
			\dot{x} = f(x,0_Y). \label{unforced_bound}
		\end{equation}
		\item 
		There exists some ${c \in \mathbb{R}}$ and continuous functions
		\begin{align}
			\alpha, \, \beta : \mathcal{B}_Y \rightarrow \mathbb{R}_{\geq 0}	 	
		\end{align} 
		which are vanishing and 
		differentiable at $0_Y$ and 
		\begin{equation}
			\mathcal{L}_{h(x,y)} W 
			\leq 
			\alpha(y) \, 
			W(x) + \beta(y),
			\label{growth_restriction}
		\end{equation}
		for all $(x,y)$ such that $W(x) \geq c$ and $y \in \mathcal{B}_Y$.
	\end{enumerate}
	Then, the forward trajectory of \eqref{outer_bound}-\eqref{inner_bound} through any initial condition ${(x_0,y_0) \in X \times \mathcal{B}_Y}$ is precompact.
\end{theorem}













\begin{proof}
	
	Since $W$ is a proper Lyapunov function for \eqref{unforced_bound}, the forward trajectory through any initial condition of the form ${(x_0,0_Y)}$ is precompact, so it suffices to consider initial conditions $(x_0,y_0)$ with ${y_0\neq 0_Y}$. Fix ${(x_0,y_0)\in X\times B_Y}$ with ${y_0\neq 0_Y}$ and let $\big(x(t),y(t)\big)$ denote its forward trajectory. We prove the claim in two steps.
	
	\begin{step}
		\label{bound_alpha_beta}
		There exist positive constants $A$, $B$, and $\omega$ such that
		$\alpha\big(y(t)\big)\leq A e^{-\omega t}$ and  $\beta\big(y(t)\big)\leq Be^{-\omega t}$ for all $t\geq 0$.
	\end{step}
	\begin{proof}
		\renewcommand{\qedsymbol}{$\blacktriangledown$}
		Let ${d(t)\coloneqq \text{dist}(y(t),0_Y) > 0}$.
		Since a stable hyperbolic equilibrium is locally exponentially stable with respect to the distance associated to any continuous Riemannian metric, there exist $C_0, \omega > 0$ such that, for all $t\geq 0$,
		\begin{equation}\label{eq:d-bound}
			d(t)\leq C_0e^{-\omega t}.
		\end{equation}
		Next, since 
		$\alpha$ and $\beta$ are vanishing and differentiable at $0_Y$, a local coordinate calculation (using uniform equivalence of continuous Riemannian metrics over compact sets) shows
		\begin{equation}
			\limsup_{t\to\infty}\frac{\alpha\big(y(t)\big)}{d(t)} < \infty, \quad \limsup_{t\to\infty}\frac{\beta\big(y(t)\big)}{d(t)} < \infty.
		\end{equation}
		Since the above quotients are also continuous functions of $t$, it follows that they are bounded.
		Hence there exist $C_1, C_2 > 0$ such that $\alpha(y(t))/d(t)< C_1$ and $\beta(y(t))/d(t) < C_2$ for all $t\geq 0$.
		When combined with \eqref{eq:d-bound}, we obtain the desired bounds,
		where $A:= C_0 C_1$ and $B:= C_0 C_2$.
	\end{proof}
	
	\begin{step}
		$W\big(x(t)\big)$ is bounded for all $t > 0$.
	\end{step}
	\begin{proof}
		\renewcommand{\qedsymbol}{$\blacktriangledown$}
		
		Since $W$ is a Lyapunov function for \eqref{unforced_bound}, we have
		\begin{align}
			\label{lyapunov_bound}
			\dot{W} 
			= \mathcal{L}_{f(x,0_Y) + h(x,y)}W 
			\leq \mathcal{L}_{h(x,y)}W.
		\end{align}
		Consider any $t_2 \geq t_1 \geq 0$ such that $W(x([t_1,t_2]))\subseteq [c,\infty)$.
		Then for all $t\in [t_1,t_2]$, \eqref{growth_restriction}, \eqref{lyapunov_bound}, and the conclusion of the previous step imply that
		\begin{equation}
			\tfrac{d}{dt}W\big(x(t)\big)\leq Ae^{-\omega t} \, W\big(x(t)\big) + B e^{-\omega t}.
		\end{equation}
		Thus, by the comparison principle \cite[p.~102, Lem.~3.4]{khalil1996nonlinear},
		\begin{align}
			&
			\begin{aligned}
				W\big(x(t_2)\big)&\leq e^{\int_{t_1}^{t_2}Ae^{-\omega t}\, dt}W\big(x(t_1)\big) \ + \\
				& \qquad \quad \quad  \int_{t_1}^{t_2}e^{\int_{t}^{t_2}Ae^{-\omega s}\, ds}Be^{-\omega t}\, dt
			\end{aligned}
			\\
			&\phantom{W\big(x(t_2)\big)}\leq e^\frac{A}{\omega} \Big(W\big(x(t_1)\big)+ \tfrac{B}{\omega}\Big),
		\end{align}
		where the second inequality holds because
			${\int_a^b e^{-\omega t}\, dt \leq \frac{1}{\omega}}$ for any $b \geq a \geq 0$.
		This implies that for all $t\geq 0$,
		$$W(x(t))\leq C:= \big(\max\big\{c,W\big(x(t_1)\big)\big\}+\tfrac{B}{\omega}\big)e^\frac{A}{\omega}. \qedhere $$
	\end{proof}
	\noindent 
	Hence, by the properness of $W$ and attractiveness of $0_Y$,
the forward trajectory through $(x_0,y_0)$ is 
precompact.
\end{proof}




















		
		
		
		
		
		
		
		
		
		
		
		
		
		
		
		
		
		
		















\section{Application of the Results}






We now revisit the motivating example \eqref{outer_example}-\eqref{inner_example} evolving on 
${T\mathbb{T}^2 = T\mathbb{S}^1 \times T\mathbb{S}^1}$,
using our results to show that the full cascade is in fact almost globally asymptotically stable. The system can be given explicitly in the form \eqref{outer_general}-\eqref{inner_general} by
\begin{subequations}
\begin{align}
	\label{outer_example_state_space}
	\frac{d}{dt}{\begin{bmatrix}
			\theta \\ \dot{\theta}
	\end{bmatrix}} &= {\begin{bmatrix}
			\dot{\theta} \\ -(\sin \theta + \dot{\theta}) \cos 2\phi
	\end{bmatrix}}, \\
	\label{inner_example_state_space}
	\frac{d}{dt}{\begin{bmatrix}
			\phi \\ \dot{\phi}
	\end{bmatrix}} &= {\begin{bmatrix}
			\dot{\phi} \\ -(\sin \phi + \dot{\phi})
	\end{bmatrix}}.
\end{align}
\end{subequations}
It is easily verified that \eqref{inner_example_state_space} takes the form of the Euler-Lagrange dynamics \eqref{euler_lagrange} 
for the kinetic energy metric,  Rayleigh dissipation, and Morse potential function
\begin{gather}
	\kappa = \nu = d \phi \otimes d\phi, \quad  
	V : \mathbb{S}^1 \rightarrow \mathbb{R}, \, \phi \mapsto 1-\cos \phi.
\end{gather}
Thus by Remark \ref{summarize_chain_recurrence_section}, \eqref{inner_example_state_space} is almost globally asymptotically stable and locally exponentially stable with respect to ${y = (\phi,\dot{\phi}) = (0,0)}$, and moreover its chain recurrent set consists solely of hyperbolic equilibria. Clearly, the same is true for
${x = (\theta,\dot{\theta}) = (0,0)}$ with respect to \eqref{outer_example_state_space} restricted to ${y = (0,0)}$. 
Hence by Theorem \ref{almost_global_stability}, it  suffices to show precompactness of forward trajectories, which
we accomplish using Theorem \ref{boundedness} and the total energy function \eqref{total_energy} given by
\begin{equation}
	W : (\theta,\dot{\theta}) \mapsto 1-\cos \theta +  \tfrac{1}{2} \dot{\theta}^2.
\end{equation}
The interconnection term is given by
\begin{align}
	h(x,y) = \begin{bmatrix}
		0 \\ \big(1-\cos 2\phi\big)\big(\sin \theta + \dot{\theta}\big) 
	\end{bmatrix},
\end{align}
and the directional derivative of $W$ along $h$ is 
\begin{equation}
	\mathcal{L}_{h(x,y)}W = 
	  {\big(1-\cos 2\phi\big)}
	  {\big(\sin \theta + \dot{\theta}\big) \dot{\theta}}.
\label{computed_lie_derivative}
\end{equation}
We propose the 
functions
\begin{equation}
	\alpha : 
 (\phi,\dot{\phi}) \mapsto 4 \, (1 - \cos 2\phi),
\quad
	\beta : 
(\phi,\dot{\phi}) \mapsto 0,
\end{equation}
which are differentiable and vanish at $(0,0)$. We compute
\begin{align}
		\alpha(y)	W (x)  + \beta(y)  
		&= 4(1 - \cos 2 \phi) (1-\cos(\theta) +  \tfrac{\dot{\theta}^2}{2})
	\\
	&\geq \big(1 - \cos 2 \phi\big) \, \big(2\dot{\theta}^2\big). \label{simplified_inequality}
\end{align}
Since ${\sin \theta \leq 1}$ and ${\cos \theta \geq -1}$, we have 
\begin{equation}
	2\dot{\theta}^2 \geq \dot{\theta} \sin \theta + \dot{\theta}^2 \textrm{ for all } W(
	x %
	) \geq 4.
	\label{last_inequality}
\end{equation}
Thus in view of \eqref{computed_lie_derivative} and \eqref{simplified_inequality}-\eqref{last_inequality}, we have shown
\begin{equation}
		\mathcal{L}_{h(x,y)}W \leq 
\alpha(y)	\,  W(x) + \beta(y)
		\textrm{ for } W(x)  \geq 4,
\end{equation}
and it follows by Theorem \ref{boundedness} that all forward trajectories of \eqref{outer_example_state_space}-\eqref{inner_example_state_space} with ${y = (\phi,\dot{\phi})}$ starting in the basin of attraction of \eqref{inner_example_state_space} are precompact. Thus by Theorem \ref{almost_global_stability}, the system
is almost globally asymptotically stable and locally exponentially stable with respect to ${(0,0,0,0) \in T\mathbb{T}^2}$.


\section{Discussion}



The disturbance robustness of systems with some similar properties was considered in \cite{Angeli2010}, and the connection to systems whose only chain recurrent points are equilibria was explored in Sec. IV therein. However, those results (when combined with \cite{Angeli2004}) can only certify the stability of a cascade if the outer loop is almost globally input to state stable. Indeed, \cite[Prop. 1]{Angeli2010} certifies robustness to small disturbances, while general robustness is achieved only under the additional assumption of ``ultimate boundedness'', a strong property which is absent from systems such as our motivating example. 


The condition that only equilibria are chain recurrent may seem restrictive, but in the stabilization of a desired state, such a property in the closed loop dynamics is quite desirable; indeed, it would run contrary to the goal of rapid convergence for states other than equilibria to exhibit chain recurrence. An obvious question asks whether our results could be extended to certify chains of more than two cascaded subsystems; any such extension would necessitate a characterization of the chain recurrent set of the full cascade and likely require additional hyperbolicity assumptions.

An interesting implication of our results is that for cascades with both subsystems being almost globally asymptotically stable and having a chain recurrent set of only hyperbolic equilibria, the only obstacle to almost global asymptotic stability of the full cascade is the precompactness of forward trajectories, a conclusion which can be compared with \cite[Prop. 4.1]{Sepulchre2012} for globally asymptotically stable systems. Theorem \ref{boundedness} casts this condition as the requirement that the transient from any initial condition injects only a finite amount of ``generalized energy'' into the outer loop. Since the former properties are enjoyed by cascades of mechanical systems with suitable dissipation and potential, we see promising directions for the constructive synthesis of cascaded geometric controllers with almost global asymptotic stability for robotic systems possessing a geometric flat output (such as quadrotors and aerial manipulators) \cite{Welde2023}, which enjoy a cascade-like structure where the evolution of the system in the shape space is uniquely determined by the evolution in the symmetry group. Indeed, for a constant reference, the error dynamics of the geometric quadrotor controller proposed in \cite{Lee2010} take the form \eqref{outer_general}-\eqref{inner_general}, with the subsystems being dissipative mechanical systems.


\section{Conclusion}

In this work, we present sufficient conditions for the almost global asymptotic stability of a cascade in which the subsystems are only \textit{almost} globally asymptotically stable, a natural setting for geometric control design. The main result relies on the assumption that the only chain recurrent points of the unforced outer loop are hyperbolic equilibria. The required precompactness condition is analogous to the assumption of forward boundedness used in related results for globally asymptotically stable systems in Euclidean space, and can similarly be verified via a growth rate inequality. The qualitative nature of the stability criteria facilitates the verification of control designs for cascades whose subsystems are governed by arbitrarily complex equations, so long as the subsystems enjoy certain fundamental dynamical properties.








\bibliographystyle{IEEEtran}
\bibliography{IEEEabrv,refs}





\appendix

\section*{A. Chain Recurrence and Decreasing Functions}

In this appendix, we localize the chain recurrent set $\mathcal{R}(\Phi)$ of a continuous semiflow ${\Phi^t : M \rightarrow M}$ to a particular subset of the state space, with the aid of a function which is decreasing along all trajectories outside this subset. 
While similar results appear to be known \cite{Benaim1995}, we do not know of a reference providing these facts in our exact setting (e.g. for semiflows on possibly noncompact manifolds). 



\begin{theorem}\label{th:R-subset-E}
	Assume there exists a proper continuous function ${V\colon M\to \R}$ and a subset ${S\subseteq M}$ such that $V(S)$ is nowhere dense in $\R$ and $V$ is decreasing on trajectories outside of $S$.
	Then ${\mathcal{R}(\Phi)\subseteq S}$.
\end{theorem}
\begin{proof}
	Fix any $x\not \in S$.
	Since $V(S)$ is nowhere dense and ${x\not  \in S}$, there exist $b>a>0$ such that 
	\begin{equation}
		\begin{gathered}
			[a,b]\subseteq V(\Phi^{[0,\infty)}(x)) \text{ and }
			\{a\leq V \leq b\}\cap S = \varnothing.
		\end{gathered}
	\end{equation}
	Since $V$ is decreasing along trajectories outside of $S$ and since $V$-sublevel sets are compact, this implies  the existence of $T>0$ such that $V(\Phi^T(x)) \leq a$ and $\Phi^{[T,\infty)}(\{V\leq b\})\subseteq \{V\leq a\}$.
	Compactness of $\{V\leq a\}$ also implies the existence of $\varepsilon > 0$ such that the distance between any point in $\{V\leq a\}$ and any point in $\{V \geq b\}$ is at least $2\varepsilon$. 
	By construction there does not exist an $(\varepsilon, T)$-chain from $x$ to $x$, so $x$ is not chain recurrent. 
	This completes the proof.
\end{proof}

\begin{corollary}\label{co:E-isolated-components}
	Assume there exists a proper continuous function ${V\colon M\to \R}$ and a subset ${S\subseteq M}$ such that $V$ is constant on each connected component of $S$, each connected component of $S$ is isolated, and  $V$ is decreasing on trajectories outside of $S$.
	Then ${\mathcal{R}(\Phi)\subseteq S}$.
\end{corollary}
\begin{proof}
	For each ${t\in \R}$, the set ${S_t\coloneqq S\cap \{V\leq t\}}$ is compact since $V$ is proper, so each $S_t$ has finitely many components since components of $S$ (hence also $S_t$) are isolated.
	This implies that ${V(S_{t+1}\setminus S_t)\subseteq (t,t+1]}$ is finite for each $t$.
	Thus, ${V(S)=\bigcup_{n=1}^{\infty}V(S_{n+1}\setminus S_n)}$ is not dense in any nonempty open subset of $\R$, so $V(S)$ is nowhere dense. 
	The desired result now follows from Theorem~\ref{th:R-subset-E}.
\end{proof}

The following fact is related to the analysis in \cite[Cor. 2.4]{Benaim1995}, however the setting of that work differs from our own, since it considers stochastic processes evolving on $\mathbb{R}^n$.

\begin{corollary}\label{co:equilibria}
		If $\mathcal{E}(\Phi)$ consists of isolated points and there is a proper, continuous function ${V:M\to \R}$ 
that is
decreasing along nonequilibrium trajectories,
then ${\mathcal{R}(\Phi) = \mathcal{E}(\Phi)}$.
\end{corollary}
\begin{proof}
	Since $\mathcal{E}(\Phi) \subseteq \mathcal{R}(\Phi)$, it suffices to show that $\mathcal{R}(\Phi)\subseteq \mathcal{E}(\Phi)$.
	Since each connected component of ${S = \mathcal{E}(\Phi)}$ is a singleton, $V$ is automatically constant on each component of $S$, so the desired result follows from Corollary~\ref{co:E-isolated-components}.
\end{proof}

\begin{remark}
	If the distance function on $M$ is replaced with a new one inducing the same topology, the conclusions of Theorem~\ref{th:R-subset-E} and Corollary~\ref{co:E-isolated-components} imply that the new chain recurrent set is still contained in $S$, and the conclusion of Corollary~\ref{co:equilibria} implies that the new chain recurrent set coincides with the old, i.e. $\mathcal{E}(\Phi)$. Moreover,  $\mathcal{R}(\Phi)$ is independent of the choice of Riemannian metric, since all smooth Riemannian metrics induce the same topology.
\end{remark}



\section*{B. Two Classes of Gradient-Like Systems}



		In this appendix, we consider two important classes of systems of particular relevance to geometric control design, which are already widely known to be almost globally asymptotically stable. With the aid of Corollary \ref{co:equilibria}, we verify the lesser-known fact that such systems have a chain recurrent set consisting solely of hyperbolic equilibria.
		The following facts are not particularly novel (see e.g. the discussion in \cite[Sec. IV]{Angeli2010} and \cite{Benaim1995}), but they are relevant to our larger interests, so we present them for completeness.
		
		
		\subsection{Gradient Systems}
		
		
		
		
		
		
		
		
		
		We first consider dynamical systems induced by descending the gradient of a \textit{Morse function} (i.e. a function whose critical points are all nondegenerate \cite{BulloAndLewis2004}) with a unique minimum. We note that Morse functions with unique minima, including ``perfect'' ones with the minimum possible number of critical points, are well-known for those manifolds typically encountered in geometric control \cite{Maithripala2006}. 
		
		
		
		\begin{proposition}[Gradient System]
			\label{first_order_dynamics}
			For a Riemannian manifold ${(Q,\kappa)}$ and a proper Morse function ${V : Q \rightarrow \left[0,\infty\right)}$ 
			with a unique minimum at 
			${0_Q \in Q}$, the dynamical system 
			\begin{equation}
				\dot{q} = - \mathrm{grad}_\kappa\, V(q) 
				\label{gradient_descent}
			\end{equation}
			is almost globally asymptotically stable and locally exponentially stable with respect to $0_Q$, and all chain recurrent points of \eqref{gradient_descent} are hyperbolic equilibria.
		\end{proposition}
		
		\begin{proof}
			The almost global asymptotic stability of \eqref{gradient_descent} with respect to $0_Q$ is proved in \cite[Proposition 2.1]{Koditschek1989} for compact $Q$. However, the extension to the noncompact case is immediate since the sublevel sets of $V$ are compact and forward invariant, since by direct computation, 
			\begin{align}
				\dot{V} 
				&= dV(-\mathrm{grad}_\kappa V)
				= -\kappa\big(\mathrm{grad}_\kappa V, \mathrm{grad}_\kappa V \big) \leq 0.
			\end{align}
			Since the equilibria of \eqref{gradient_descent} are simply the critical points of $V$, the nondegeneracy of the critical points of Morse functions ensures hyperbolicity and therefore the local exponential stability of $0_Q$.
			Finally, since $V$ is decreasing on non-equilibrium trajectories and hyperbolic equilibria are isolated, Corollary \ref{co:equilibria} implies that the chain recurrent set of \eqref{gradient_descent} is exactly the set of equilibria.
		\end{proof}
		
		
		
		
		\subsection{Dissipative Mechanical Systems}
		
		We now turn our attention to the important class of dissipative mechanical systems arising from kinetic energy, potential energy, and damping. Such systems have been studied at length, since the introduction of artificial dissipation and potential shaping via feedback can result in closed loop dynamics of this form with desirable limit behavior. 
		We direct the reader to the seminal work \cite{Koditschek1989} which studies the global stability properties of such systems, as well as the more recent reference \cite[Chap. 6]{BulloAndLewis2004} which provides a comprehensive and detailed overview.
		 Closed loop dynamics of this form have enabled trajectory tracking on arbitrary Lie groups \cite{Maithripala2006} and have also featured in the inner and outer loops of cascaded geometric controllers for underactuated robotic systems \cite{Lee2010,Sreenath2013}.
		
		
		
		
		
		
		
		
		
		\begin{proposition}[Dissipative Mechanical System]
			\label{second_order_dynamics}
			For a Riemannian manifold ${(Q,\kappa)}$,
			a strict Rayleigh dissipation $\nu$, and a proper Morse function ${V : Q \rightarrow \left[0,\infty\right)}$ 
			with a unique minimum at 
			${0_Q \in Q}$,
			the Euler-Lagrange dynamical system
			\begin{equation}
				\overset{\kappa}{\nabla}_{\dot{q}} \dot{q} = - \mathrm{grad}_\kappa\, V(q) - \kappa^\sharp \circ \nu^\flat (\dot{q}) 
				\label{appendix:euler_lagrange}
			\end{equation}
			is almost globally asymptotically stable and locally exponentially stable with respect to ${0_{TQ} = (0_Q,0) \in TQ}$, and all chain recurrent points of \eqref{appendix:euler_lagrange} are hyperbolic equilibria.
		\end{proposition}
		
		\begin{proof}
			It is clear that the equilibrium set of \eqref{appendix:euler_lagrange} is precisely the image of the critical points of $V$ in the zero section of $TQ$, and moreover these equilibria can be verified to be hyperbolic since $\nu$ is a strict linear dissipation and the critical points of a Morse function are nondegenerate. Moreover, only $0_{TQ}$ is (locally exponentially) stable, while all other equilibria are unstable, since $0_Q$ is the unique minimum of $V$.
			Considering the total energy function given by
			\begin{equation}
				\label{total_energy}
				W : (q,\dot{q}) \mapsto V(q) + \frac{1}{2}\kappa(\dot{q},\dot{q}),
			\end{equation}
			we compute
			\begin{align}
				\dot{W} 
				&= dV(q) \dot{q}  + \kappa(	\overset{\kappa}{\nabla}_{\dot{q}} \dot{q},\dot{q}) \\
				&= dV(q) \dot{q}  + \kappa(	- \mathrm{grad}_\kappa\, V(q) - \kappa^\sharp \circ \nu^\flat (\dot{q})   \, , \, \dot{q} ) \\
				&= dV(q) \dot{q}  - dV(q) \dot{q}  - \nu(\dot{q},\dot{q}) 
				= -\nu(\dot{q},\dot{q}) \leq 0.
			\end{align}	
			For any trajectory $t\mapsto q(t)$ of the Euler-Lagrange dynamics, \eqref{appendix:euler_lagrange} and strictness of $\nu$ imply that $\nu(\dot{q}(t),\dot{q}(t))>0$ for almost all $t$ if and only if the trajectory is nonequilibrium, so $W$ is decreasing along nonequilibrium trajectories. Thus, by Corollary \ref{co:equilibria}, the chain recurrent set of \eqref{appendix:euler_lagrange} is exactly the set of equilibria. Becuase $W$ is proper (since $V$ is proper and $\nu$ is positive definite) and nonincreasing along trajectories, all forward trajectories are precompact and therefore converge to the chain recurrent set \cite{Mischaikow1995}. Since hyperbolic equilibria are isolated, all trajectories converge to some equilibrium. Application of the global stable manifold theorem shows that almost no trajectories converge to an unstable hyperbolic equilibrium, so the unique stable equilibrium $0_{TQ}$ is almost globally asymptotically stable.
		\end{proof}
		
		\begin{remark}
			A primary contribution of \cite{Koditschek1989} is the observation that the global limit behavior of a dissipative mechanical system is essentially determined by the global limit behavior of the associated gradient system, which is often called the ``lifting property'' of dissipative mechanical systems \cite{BulloAndLewis2004}. Here, we have shown that a similar lifting property holds for these systems in regards to the chain recurrent set.
		\end{remark}






\end{document}
