%\documentclass{article}
\documentclass[letterpaper, 10 pt, conference]{ieeeconf}

\IEEEoverridecommandlockouts                              

%\title{Almost Global Asymptotic Stability of Cascades}
\title{A Compositional Approach to Certifying the Almost \\ Global Asymptotic Stability of Cascade Systems}
\author{Jake Welde, Matthew D. Kvalheim, and Vijay Kumar
\thanks{
	J. Welde and V. Kumar are with the GRASP Laboratory at the University of Pennsylvania, while M. D. Kvalheim is with the Department of Mathematics at the University of Michigan. emails: \texttt{\{jwelde,kumar\}@seas.upenn.edu}, \texttt{kvalheim@umich.edu}.  
	We gratefully acknowledge the support of Qualcomm Research, NSF Grant CCR-2112665, and the NSF Graduate Research Fellowship Program.
}
}


\date{\today}
\let\proof\relax 
\let\endproof\relax 
%\usepackage{cases}

\usepackage{amsthm}
\usepackage{amsmath}
\usepackage{tikz}
\usetikzlibrary{shapes,arrows}

\usepackage{amssymb}
\usepackage{thmtools}
\usepackage{graphicx}
\usepackage{mathtools}
\let\labelindent\relax
\usepackage[shortlabels]{enumitem}

\newcommand{\Lim}[1]{\raisebox{0.5ex}{\scalebox{0.8}{$\displaystyle \lim_{#1}\;$}}}
\declaretheoremstyle[notefont=\normalfont\itshape,bodyfont=\normalfont]{normaltext}
\declaretheorem[name=System,style=normaltext]{system}

\newtheorem{theorem}{Theorem}
\newtheorem{conjecture}{Conjecture}
\newtheorem{corollary}{Corollary}
\declaretheorem[name=Definition,style=normaltext]{definition}
\declaretheorem[name=Remark,style=normaltext]{remark}
%    headfont=\normalfont\bfseries, 

% for restating in text
\newtheorem{innercorollary}{Corollary}
\newenvironment{restate_corollary}[1]
{\renewcommand\theinnercorollary{#1}\innercorollary}
{\endinnercorollary}


\newcommand{\R}{\mathbb{R}}

\declaretheoremstyle[notefont=\normalfont\itshape,bodyfont=\normalfont\itshape,headfont=\normalfont\sc]{smallcapsname}
%\declaretheoremstyle[notefont=\normalfont\itshape,bodyfont=\normalfont\itshape,headfont=\normalfont\scshape]{smallcapsname}
\declaretheorem[name=Step,style=smallcapsname]{step}
\newtheorem{proposition}{Proposition}
%\newtheorem{corollary}{Corollary}
%\renewcommand{\qedsymbol}{$\blacksquare$}
%\newtheorem{lemma}{Lemma}

\usepackage{cite}

\usepackage{balance}


\renewcommand\qedsymbol{$\blacksquare$}

\begin{document}

\makeatletter
\@addtoreset{step}{theorem}
\makeatother

\maketitle

\begin{abstract}
In this work, we give sufficient conditions for the almost global asymptotic stability of a cascade in which the inner loop and the \textit{unforced} outer loop are each almost globally asymptotically stable. Our qualitative approach relies on the absence of chain recurrence for non-equilibrium points of the unforced outer loop, the hyperbolicity of equilibria, and the precompactness of forward trajectories. We show that the required structure of the chain recurrent set can be readily verified, and describe two important classes of systems with this property. 
We also show that the precompactness requirement can be verified by growth rate conditions on the interconnection term coupling the subsystems. Our results stand in contrast to prior works that require either \textit{global} asymptotic stability of the subsystems (impossible for smooth systems evolving on general manifolds), time scale separation between the subsystems, or strong disturbance robustness properties of the outer loop. The approach has clear applications in stability certification of cascaded controllers for systems evolving on manifolds.%
%, an important approach in the control of underactuated robotic systems.
\end{abstract}

%Outline of the Paper
%\begin{enumerate}
%	\item Introduction
%	\begin{enumerate}
%		\item Background on cascade stability
%		\begin{enumerate}
%			\item non-manifold results
%			\item manifold results
%			\item practical applications in robotics, interest in navigation-function based approach, etc.
%		\end{enumerate}
%		\item motivating example system on $T \mathbb{T}^2$
%		\item Description of the results
%	\end{enumerate}
%	\item Background: Dynamical Systems Theory
%	\begin{enumerate}
%		\item flows
%		\item chain recurrence
%		\item Asymptotically autonomous semiflows
%	\end{enumerate}
%	\item Stability of Cascades
%	\begin{enumerate}
%		\item main theorem
%	\end{enumerate}
%	\item Boundedness of Trajectories
%	\begin{enumerate}
%		\item main theorem
%	\end{enumerate}
%	\item Application to Dissipative Flows on Riemannian Manifolds
%	\begin{enumerate}
%		\item Structure of the chain recurrent set
%		\item Growth rate of projective term?
%		\item condition (ii) of Theorem 4.7 from constructive book holds
%	\end{enumerate}
%	\item Conclusion
%\end{enumerate}

%Savvas Loizou example, assumed compactness and affineness?

%In this document, we are interested in almost-global stability properties for cascades of dynamical systems on manifolds.






\section{Introduction}

In this work, we are interested in the asymptotic stability of cascade systems in the form
%\begin{subequations}
%	\begin{numcases}{\Sigma : }
%				\dot{x} = f(x,y), \label{outer_general} \\ 
%		\dot{y} = g(y), \label{inner_general}
%	\end{numcases}
%\end{subequations}
\begin{subequations}
	\begin{align}
		\dot{x} &= f(x,y), \label{outer_general} \\ 
		\dot{y} &= g(y), \label{inner_general}
	\end{align}
\end{subequations}
depicted graphically in in Fig. 1 as  system $\Sigma$. Throughout, we assume that $x$ and $y$ evolve on $X$ and $Y$, which are smooth, connected Riemannian manifolds without boundary (see Remark \ref{choice_of_setting} for an explanation of the choice of setting). We call \eqref{inner_general} the ``inner loop'' and \eqref{outer_general} the ``outer loop''. 

\iffalse
%Cascades appear in many interesting and important physical systems, occurring both naturally and artificially. For example, a mechanical system with symmetry and zero momentum or Chaplygin nonholonomic constraints admits a cascaded description, dividing the state space between the \textit{shape} variables (describing the internal joints) and the \textit{group} variables (capturing the symmetry directions) \cite{Murray1997}. Many underactuated mechanical systems can be rendered as a cascade after a feedback transformation \cite{Olfati2000}, and mechanical systems admitting a ``configuration flat output'' \cite{Rathinam1998} of degree 4 can be transformed into a cascade which partitions the phase space between internal and external variables. 
\fi 

Cascades appear in many interesting and important physical systems. For example, many underactuated mechanical systems can be rendered as a cascade after a feedback transformation \cite{Olfati2000}, and cascade structures appear often in robotic systems, either intrinsically \cite{Murray1997} or after control design \cite{Lee2010}. A long research tradition has studied the implications of cascaded structure to simplify analysis and aid in control \cite{Sepulchre2012,Jankovic1996,Kokotovic2001}. This compositional approach is motivated by the observation that control design for a subsystem is typically easier, due to e.g. lower dimensionality, lower relative degree, or full actuation. The primary difficulty that arises is to ensure that the full system combining these subsystems achieves the desired behavior, since only \textit{local} (as opposed to \textit{global}) asymptotic stability is preserved under cascades for general nonlinear systems 
\cite{Sepulchre2012}.
%\cite{Sontag1989}. 
\begin{figure}
	\tikzstyle{block} = [draw, rectangle, 
	minimum height=3em, minimum width=6em]
	\tikzstyle{sum} = [draw, circle, node distance=1cm]
	\tikzstyle{input} = [coordinate]
	\tikzstyle{output} = [coordinate]
	\tikzstyle{pinstyle} = [pin edge={to-,thin,black}]
	
	\centering
	
	\textbf{Cascade System} 
	
	\
	
	\begin{tikzpicture}[auto, node distance=2cm,>=latex']
		
		\node [block, minimum width={26pt},minimum height={20pt}] (inner) {$g(\hspace{1pt}\cdot\hspace{1pt})$};
		\node [draw, ellipse, right of=inner, node distance=1.4cm, inner sep=2pt] (inner_integral) {$\int$};
		\node [above of=inner, node distance=.65cm, inner sep=0pt, outer sep=0pt] (inner_top) {};
		\node [right of=inner_integral, node distance=.5cm, ellipse, fill=black, minimum height=2.5pt, minimum width=2.5pt, inner sep=0pt] (inner_end) {};
		
		\node [left of=inner, node distance=1.25cm] (system1) {\large $\Sigma:$};
		
		\node [block, right of=inner_integral, minimum width={34pt},minimum height={20pt}] (outer) {$f(\hspace{1pt}\cdot\hspace{1pt},\hspace{-.5pt}\cdot\hspace{1pt})$};
		\node [draw, ellipse, right of=outer, node distance=1.4cm, inner sep=2pt] (outer_integral) {$\int$};
		\node [above of=outer, node distance=.65cm, inner sep=0pt, outer sep=0pt] (outer_top) {};
		\node [right of=outer_integral, node distance=.4cm, outer sep=0pt, inner sep=0pt] (outer_end) {};
		
		
		
		\draw [->] (inner_integral) -- (inner_end) |-  (inner_top) node[right, pos=0.2] {$y$} -- (inner);	
		\draw [->] (inner_end) -- (outer);	
		\draw [->] (inner) -- node[name=u] {$\dot{y}$} (inner_integral);
		
		\draw [->] (outer_integral) -- (outer_end) |-  (outer_top) node[right, pos=0.25] {$x$} -- (outer);	
		\draw [->] (outer) -- node[name=u] {$\dot{x}$} (outer_integral);
		
	\end{tikzpicture}
	
%	\

	\vspace{4pt}	
	\textbf{Subsystem Decomposition} 
	
	\vspace{6pt}
	\begin{tikzpicture}[auto, node distance=2cm,>=latex']
		
		\node [block, minimum width={26pt},minimum height={20pt}] (inner) {$g(\hspace{1pt}\cdot\hspace{1pt})$};
		\node [draw, ellipse, right of=inner, node distance=1.4cm, inner sep=2pt] (inner_integral) {$\int$};
		\node [above of=inner, node distance=.65cm, inner sep=0pt, outer sep=0pt] (inner_top) {};
		\node [right of=inner_integral, node distance=.5cm, outer sep=0pt, inner sep=0pt] (inner_end) {};
		
		\node [left of=inner, node distance=1.25cm] (system1) {\large $\Sigma_y:$};
		
		\draw [->] (inner_integral) -- (inner_end) |-  (inner_top) node[right, pos=0.2] {$y$} -- (inner);	
		\draw [->] (inner) -- node[name=u] {$\dot{y}$} (inner_integral);
		
	\end{tikzpicture}
	
	\vspace{6pt}
	\begin{tikzpicture}[auto, node distance=2cm,>=latex']
		
		\node [block, minimum width={34pt},minimum height={20pt}] (inner) {$f(\hspace{1pt}\cdot\hspace{1pt},\hspace{-.5pt}\cdot\hspace{1pt})$};
		%	\node [block, right of=inner_integral, 
		\node [draw, ellipse, left of=inner, node distance=1.3cm, inner sep=0pt, minimum height=16pt, minimum width=16pt] (eql) {\scriptsize $0_Y$};		
		\node [draw, ellipse, right of=inner, node distance=1.4cm, inner sep=2pt] (inner_integral) {$\int$};
		\node [above of=inner, node distance=.65cm, inner sep=0pt, outer sep=0pt] (inner_top) {};
		\node [right of=inner_integral, node distance=.5cm, outer sep=0pt, inner sep=0pt] (inner_end) {};
		
		\node [left of=eql, node distance=1.25cm] (system1) {\large $\Sigma_x:$};
		
		\draw [->] (inner_integral) -- (inner_end) |-  (inner_top) node[right, pos=0.2] {$x$} -- (inner);	
		\draw [->] (inner) -- node[name=u] {$\dot{x}$} (inner_integral);
		\draw [->] (eql) -- (inner);	
	\end{tikzpicture}
	
	\vspace{0pt}
	\begin{tikzpicture}[auto, node distance=2cm,>=latex']
		
		\node [block, minimum width={34pt},minimum height={20pt}] (inner) {$f(\hspace{1pt}\cdot\hspace{1pt},\hspace{-.5pt}\cdot\hspace{1pt})$};
		
		\node [block, minimum width={34pt},minimum height={20pt}, below of=inner, node distance=1.25cm] (zero) {$f(\hspace{1pt}\cdot\hspace{1pt},\hspace{-.5pt}\cdot\hspace{1pt})$};
		\node [left of=zero, node distance=2.75cm, inner sep=1pt, minimum height=20pt, minimum width=20pt] (input) {$y$};		
		
		%	\node [block, right of=inner_integral, 
		\node [draw, ellipse, left of=inner, node distance=1.3cm, inner sep=0pt, minimum height=16pt, minimum width=16pt] (eql) {\scriptsize $0_Y$};		
		\node [sum, right of=inner, node distance=1.4cm, inner sep=4pt] (inner_integral) {};
		\node [above of=input, node distance=1.9cm,  inner sep=1pt, minimum height=20pt, minimum width=20pt] (inner_top) {$x$};
		\node [right of=inner_top, node distance=.8cm, ellipse, fill=black, minimum height=2.5pt, minimum width=2.5pt, inner sep=0pt] (inner_button) {};
		\node [below of=inner_button, node distance=1.26cm, inner sep=0pt, outer sep=0pt] (input_corner) {};
		
		\node [right of=inner_integral, node distance=.5cm, outer sep=0pt, inner sep=0pt] (inner_end) {};
		
		\node [left of=eql, node distance=2.25cm] (system1) {\large $\Sigma_h:$};
		
		\draw [->] (inner_top) -| (inner);	
		\draw [->] (inner) -- node[name=u, pos=.8] {\scriptsize $-$} (inner_integral);
		\draw [->] (inner_integral) -- node[pos=1, right] {$h(x,y)$} +(1,0);
		\draw [->] (eql) -- (inner);	
		\draw [->] (input)  -- (zero);	
		\draw [->] (zero) -| node[name=u, pos=.95, right] {\scriptsize $+$}  (inner_integral);	
		\draw [->] (inner_button) -- (input_corner) -| (zero);	
	\end{tikzpicture}
	
	
	
	\caption{We 
		%	investigate the almost global asymptotic stability of systems in the cascade form of $\Sigma_1$ above. We
		give sufficient conditions for the almost global asymptotic stability of 
		a cascade system $\Sigma$
%		with precompact forward trajectories, 
		in terms of qualitative properties of the closed loop systems $\Sigma_y$ (the ``inner loop'') and $\Sigma_x$ (the unforced ``outer loop'') as well as growth rate conditions on 
		the stateless I/O system
		 $\Sigma_h$ (the ``interconnection term''). Above, $0_Y$ is the stable equilibrium of $\Sigma_y$.
%		 In particular, if $\Sigma_y$ is almost globally asymptotically stable and locally exponentially stable with respect to $0_Y$, while $\Sigma_x$ is almost globally asymptotically stable and its only chain recurrent points are hyperbolic equilibria, then $\Sigma$ is almost globally asymptotically stable.
%		We show that the precompactness of the forward trajectories of $\Sigma$ can be verified by growth rate conditions on the stateless I/O system $\Sigma_h$ (the ``interconnection term'').
}
	\label{block_diagrams}
	\vspace{-16pt}
\end{figure}




\subsection{Prior Work on Cascade Stability}

\iffalse
\begin{figure*}[t]
	\centering
	
%\iffalse
\begin{tikzpicture}[outer sep=0, inner sep=0]
\node (image) at (0,0) { \includegraphics[width=.2\textwidth,trim={15cm 6cm 13.5cm 4.5cm},clip] {trajectories/%
trajectory_1.4327_-0.6457_5.5979_1.7819.png%
}};
\node  at (0,-.965) [ellipse, fill=blue, minimum height=2pt, minimum width=2pt, inner sep=0pt] (inner_end) {};
\end{tikzpicture}%
\hfill
\begin{tikzpicture}[outer sep=0, inner sep=0]
	\node (image) at (0,0) { \includegraphics[width=.2\textwidth,trim={15cm 6cm 13.5cm 4.5cm},clip] {trajectories/%
trajectory_6.156_-2.1518_3.7383_4.6216.png%
	}};
	\node  at (0,-.965) [ellipse, fill=blue, minimum height=2pt, minimum width=2pt, inner sep=0pt] (inner_end) {};
\end{tikzpicture}%
\hfill
\begin{tikzpicture}[outer sep=0, inner sep=0]
	\node (image) at (0,0) { \includegraphics[width=.2\textwidth,trim={15cm 6cm 13.5cm 4.5cm},clip] {trajectories/%
trajectory_1.4789_-0.5198_3.5774_-4.386.png%
	}};
	\node  at (0,-.965) [ellipse, fill=blue, minimum height=2pt, minimum width=2pt, inner sep=0pt] (inner_end) {};
\end{tikzpicture}%
\hfill
\begin{tikzpicture}[outer sep=0, inner sep=0]
	\node (image) at (0,0) { \includegraphics[width=.2\textwidth,trim={15cm 6cm 13.5cm 4.5cm},clip] {trajectories/%
trajectory_6.1014_3.4637_3.1793_-2.2112.png%
	}};
	\node  at (0,-.965) [ellipse, fill=blue, minimum height=2pt, minimum width=2pt, inner sep=0pt] (inner_end) {};
\end{tikzpicture}%
\hfill
\begin{tikzpicture}[outer sep=0, inner sep=0]
	\node (image) at (0,0) { \includegraphics[width=.2\textwidth,trim={15cm 6cm 13.5cm 4.5cm},clip] {trajectories/%
trajectory_1.4444_0.7605_5.0933_-0.9616.png%
	}};
	\node  at (0,-.965) [ellipse, fill=blue, minimum height=2pt, minimum width=2pt, inner sep=0pt] (inner_end) {};
\end{tikzpicture}%

	\caption{A sampling of initial conditions and resulting trajectories of the motivating example system \eqref{outer_example}-\eqref{inner_example}, projected down to $\mathbb{T}^2$ from the full state space $T\mathbb{T}^2$, where the ``small'' axis of the torus corresponds to $\theta$ and the ``large'' axis corresponds to $\phi$. All sampled trajectories converge to the point ${(0,0,0,0) \in T\mathbb{T}^2}$, marked in blue. Despite the highly energetic and topologically complex behavior of the trajectories, our results certify the almost global asymptotic stability of the system, without the need to construct an explicit Lyapunov function for the full cascade. \textbf{To Do: fix colors and placement}
	}	
	\label{trajectories}
	\vspace{-10pt}
\end{figure*}
\fi



%Techniques such as \textit{backstepping} \cite{Kokotovic2001} have exploited explicit knowledge of a Lyapunov function for the outer loop in order to {design} inner loop dynamics enabling the construction of an explicit Lyapunov function  for the full cascade. 
Approaches to stability certification for nonlinear cascades 
%given \textit{a priori} 
have exploited a wide range of structural features. Singular perturbation techniques \cite{Saksena1984,Sreenath2013} assume a time scale separation between the ``fast'' inner loop and the ``slow'' outer loop, and show that a system's behavior tends toward that of a ``reduced'' system as the ratio between convergence rates tends to zero.
%Such methods have certified the stability of controllers for inverted pendulum systems \cite{Young2006}, quadrotors with suspended payloads \cite{Sreenath2013}, and even full scale aircraft \cite{Reiner1996}, 
However, this approach necessitates rapid inner loop convergence (which can be especially challenging to achieve for a control system with realistic input limitations).
%, and the central assumption prevents application to systems without clear time scale separation. 
%Furthermore, although formal stability proofs may rely on singular perturbation arguments, the physical input limits of real systems can sometimes prevent practical implementations from achieving sufficiently rapid convergence for the inner loop, calling the usefulness of the formal certificate into question. 
Other approaches have relied on the robustness of the outer loop to disturbances, leveraging the property of \textit{input to state stability}, which roughly requires the asymptotic response of the system under a disturbance input to be bounded 
%by a class $\mathcal{K}$ function of the norm of 
by the size of the input (and therefore also implies global asymptotic stability of the system in the absence of disturbances) \cite{Sontag1990}. A classic result then establishes the global asymptotic stability of a cascade for which the outer loop is input to state stable and the inner loop is globally asymptotically stable \cite{Sontag1989}.
Methods which avoid robustness and time scale assumptions have relied on the local exponential stability of the inner loop as well as growth restrictions on the ``interconnection term'' to certify global asymptotic stability of cascades with globally asymptotically stable subsystems \cite{Sepulchre2012}.

These global results have important applications, but their utility in geometric control is rather limited, since only manifolds diffeomorphic to $\mathbb{R}^n$ admit smooth globally stable vector fields \cite{Wilson1967}, which the state space of e.g. free-flying robotic systems is not (see \cite{Bhat2000} for a discussion).
%A well-known example of this phenomenon is the ``hairy ball theorem'' which prohibits the same on $\mathbb{S}^{2}$ \cite{Milnor1978}. 
In the smooth non-Euclidean setting, the most one can hope for in either the subsystems or the cascade is \textit{almost} global asymptotic stability. 
This fact has motivated an almost global notion of input to state stability \cite{Angeli2004}, in which an asymptotic gain condition holds for all but a measure zero set of initial conditions; a cascade is then guaranteed to be almost globally asymptotically stable if its outer loop is almost globally input to state stable and its inner loop is almost globally asymptotically stable. While almost global input to state stability can be challenging to verify, this can be achieved under certain conditions on the exponential instability of other equilibria as well as the ``ultimate boundedness'' of trajectories of the system under arbitrary disturbances \cite{Angeli2010}. 

However, not all almost globally asymptotically stable cascades have an outer loop enjoying this property; indeed, almost global input to state stability seems to be an inherently stricter property than necessary, since it characterizes the response of the system to arbitrary disturbances, while for our purposes, the outer loop is almost always subjected to a converging disturbance. Yet, the lack of a comprehensive understanding of such systems has required bespoke stability certificates for almost globally asymptotically stable cascaded controllers in practice, inhibiting generalization; for example, a Lyapunov function for the full system may be handcrafted via human intuition even though the cascaded structure originally inspired the control design \cite{Lee2010}.

%and this result has been employed to certify the stability of robot controllers cascaded with human inputs \cite{Loizou2007}.

%Problem: all these assume GAS to get GAS. We want aGAS!

%have been able to certify global asymptotic stability of cascades with globally asymptotically stable subsystems by relying on the local exponential stability of 

%. While effective, such approaches cannot certify global or almost global stability of systems in which the outer loop is only \textit{almost} globally stable. Since globally stable smooth vector fields do not exist on arbitrary manifolds \cite{Milnor1978}, this limits the usefulness of such results in geometric control. 
%Classical methods of backstepping also apply only to control affine systems.


%The same obstacles to application in geometric control have motivated the developement of an almost global notion of ISS \cite{Angeli2004}, in which only almost all initial conditions must converge under any given disturbance. While almost ISS is not known to be preserved under cascade, it is know that a cascade with an almost ISS outer loop and an almost GAS inner loop is almost GAS \cite{Angeli2004}. In \cite{Angeli2010}, more easily verified conditions for ISS were given, relying on exponential instability of other equilibria and an ``ultimate boundedness condition''. However, as we will show, there exist cascades with almost globally asymptotically stable subsystems in which the outer loop is not almost ISS, in which case the results of \cite{Angeli2004} are of no use. Indeed, it is clear that the property of almost ISS for the inner loop is quite a restrictive assumption, since it characterizes the response of the system with respect to arbitrary disturbances, while the disturbances induced by the transient evolution of the cascade tend zero, perhaps exponentially. The related concept of integral ISS has been proposed for systems which are GAS in the absence of disturbances, but no analagous concept has been proposed for systems which are only almost GAS. (\textbf{Is this true?})

%This cascaded structure has been studied extensively in the control systems literature, owing to its appearance in many natural and manmade phenomena. Mechanical systems often exhibit a cascaded structure, either due to natural decoupling of the dynamics or after a transformation by feedback into a chained form. Often, control design for complex underactuated systems yields closed loop dynamics with a cascaded structure. 





%	Consider the following dynamical system defined on 
%	$\mathbb{T}^2 = \mathbb{S}^1 \times \mathbb{S}^1$.
%	\begin{align}
%			\dot{x} &= -\sin x \cos y \\
%			\dot{y} &= -\sin y
%		\end{align}
%is this ultimately bounded in the Angeli sense? I think so, therefore is it almost input to state stable??? Yes, since bounded disturbance yields a bounded error?
%Consider a system on $SE(2) \times \mathbb{S}^1$ since this is noncompact and also not globally stabilizable by smooth feedback.

%To do: discuss shortcomings for geometric control. 
%
%Sometimes even when a cascaded structure inspires the control design, this structural feature is not directly employed in the proof of stability \cite{Lee2010}. 


\subsection{Motivating Example}

\begin{figure}[t]
	\centering
	
	%	%\iffalse
	%	\begin{tikzpicture}[outer sep=0, inner sep=0]
		%		\node (image) at (0,0) { \includegraphics[width=.2\textwidth,trim={15cm 6cm 13.5cm 4.5cm},clip] {trajectories/%
				%				trajectory_1.4327_-0.6457_5.5979_1.7819.png%
				%		}};
		%		\node  at (0,-.965) [ellipse, fill=blue, minimum height=2pt, minimum width=2pt, inner sep=0pt] (inner_end) {};
		%	\end{tikzpicture}%
	%	\hfill
	\begin{tikzpicture}[outer sep=0, inner sep=0]
		\node (image) at (0,0) { \includegraphics[width=.33\columnwidth,trim={15cm 6cm 13.5cm 4.5cm},clip] {trajectories/%
%				trajectory_1.618_3.4072_1.5977_3.1428.png%
%				trajectory_0.2901_-4.0287_5.1739_1.9483.png%
%				trajectory_6.156_-2.1518_3.7383_4.6216.png%
%trajectory_1.5077_-0.8273_0.312_4.0272.png% 
				trajectory_1.618_3.4072_1.5977_3.1428.png%
		}};
		\node  at (0.037,-.75) [ellipse, fill=purple, minimum height=2pt, minimum width=2pt, inner sep=0pt] (inner_end) {};
	\end{tikzpicture}%
	\hfill
	%	\hspace{-.1\columnwidth}
	\begin{tikzpicture}[outer sep=0, inner sep=0]
		\node (image) at (0,0) { \includegraphics[width=.33\columnwidth,trim={15cm 6cm 13.5cm 4.5cm},clip] {trajectories/%
trajectory_5.5617_4.1329_5.0026_-4.0129.png%
%				trajectory_1.4789_-0.5198_3.5774_-4.386.png%
		}};
		\node  at (0.037,-.75) [ellipse, fill=purple, minimum height=2pt, minimum width=2pt, inner sep=0pt] (inner_end) {};
	\end{tikzpicture}%
	\hfill
	%	\hspace{-.1\columnwidth}
	\begin{tikzpicture}[outer sep=0, inner sep=0]
		\node (image) at (0,0) { \includegraphics[width=.33\columnwidth,trim={15cm 6cm 13.5cm 4.5cm},clip] {trajectories/%
%				trajectory_6.1014_3.4637_3.1793_-2.2112.png%
trajectory_1.4482_3.4431_1.2237_-2.7408.png%
		}};
		\node  at (0.037,-.75) [ellipse, fill=purple, minimum height=2pt, minimum width=2pt, inner sep=0pt] (inner_end) {};
	\end{tikzpicture}%
	%	\hfill
	%	\begin{tikzpicture}[outer sep=0, inner sep=0]
		%		\node (image) at (0,0) { \includegraphics[width=.2\textwidth,trim={15cm 6cm 13.5cm 4.5cm},clip] {trajectories/%
				%				trajectory_1.4444_0.7605_5.0933_-0.9616.png%
				%		}};
		%		\node  at (0,-.965) [ellipse, fill=blue, minimum height=2pt, minimum width=2pt, inner sep=0pt] (inner_end) {};
		%	\end{tikzpicture}%
		\vspace{-6pt}	
	\caption{A sampling of initial conditions and resulting trajectories of the motivating example system \eqref{outer_example}-\eqref{inner_example}, projected down to $\mathbb{T}^2$ from the full state space ${T\mathbb{T}^2 = T\mathbb{S}^1 \times T\mathbb{S}^1}$, where the ``small'' axis of the torus corresponds to $\theta$ and the ``large'' axis corresponds to $\phi$. All sampled trajectories converge to ${(0,0,0,0) \in T\mathbb{T}^2}$, marked in red. Despite the highly energetic and topologically complex behavior of the trajectories, our results certify the almost global asymptotic stability of the system, without the need to construct an explicit Lyapunov function for the full cascade.
 	}	
	\label{trajectories}
	\vspace{-16pt}
\end{figure}

To motivate the question at hand, we present a simple representative example system. Consider a cascade of the form \eqref{outer_general}-\eqref{inner_general} evolving on the tangent bundle of $\mathbb{T}^2$, where 
%$(x,y) \in T \mathbb{T}^2 = T\mathbb{S}^1 \times T\mathbb{S}^1$, where 
${x = (\theta,\dot{\theta}) \in T\mathbb{S}^1}$,
${y = (\phi,\dot{\phi}) \in T\mathbb{S}^1}$, and we make the notationally convenient identification ${T\mathbb{S}^1 \cong \mathbb{S}^1 \times \mathbb{R}}$.
% and $y = (\phi,\dot{\phi}) \in \mathbb{S} \times \mathbb{R} \cong T\mathbb{S}^1$, 
The dynamics are given by
\begin{subequations}
	\begin{align}
	\ddot{\theta} &= -(\sin \theta + \dot{\theta})\cos 2 \phi, \label{outer_example}
	\\
	\ddot{\phi} &= -(\sin \phi + \dot{\phi}), \label{inner_example}
%	\ddot{\phi} &= -(\sin \phi + \dot{\phi}) \label{inner}
\end{align}
\end{subequations}
%This simple example system represents the 
and a sampling of system trajectories is shown in Fig. \ref{trajectories}.
Using the LaSalle function ${V : (\phi,\dot{\phi}) \mapsto 1 - \cos \phi + \tfrac{1}{2}\dot{\phi}^2}$,
it can be shown that  that ${(\phi,\dot{\phi})=(0,0)}$ is almost globally asymptotically stable for the inner loop \eqref{inner_example}. By the same reasoning, ${(\theta,\dot{\theta})=(0,0)}$ is also almost globally asymptotically stable for the dynamics given by restricting the outer loop \eqref{outer_example} to the stable equilibrium of the inner loop.
%, $(\phi,\dot{\phi})=(0,0)$. 

In fact, it turns out that the entire cascade \eqref{outer_example}-\eqref{inner_example} is almost globally asymptotically stable, but the system does not satisfy the hypotheses of any of the previously discussed results. In particular, the subsystems are not globally asymptotically stable, nor is there a time scale separation between the loops. Furthermore, viewing $(\phi,\dot{\phi})$ as a disturbance to \eqref{outer_example}, it can be seen that the outer loop is \textit{not} almost globally input to state stable \cite[Def. 2.1]{Angeli2004}, since the response to the bounded disturbance $(\phi,\dot{\phi})=(\pi/2,0)$ grows unbounded from almost all initial conditions.
%, hence the outer loop lacks robustness to disturbances.
 Nonetheless, the results of this paper will guarantee the almost global asymptotic stability of a class of systems that includes \eqref{outer_example}-\eqref{inner_example}.


%%It is tempting to presume that the almost global asymptotic stability of the overall system is a consequence of the same property for the ``inner'' system \eqref{inner_example} and the ``outer'' system \eqref{outer_example} when restricted to the equilibrium of the inner system. 
%%Such a conclusion may be reasonable when $f(x,y) - f(x,0_Y)$ is small compared to $f(x,0_Y)$ (where $0_Y$ is the stable equilibrium for $y$), or when $y$ converges very rapidly compared to $x$. However, neither of these arguments hold for this example, and yet the cascade is almost-globally asymptotic stable.
%%
%%To Do: show it is \textit{not} input to state stable.
%%
%%TO DO: actual literature review.
%
%
%A natural question arises: for a cascaded system on the product manifold $X \times Y$ given by 
%%\begin{subequations}
%%	\begin{align}
%%		\dot{x} &= f(x,y) \label{outer_general} \\ 
%%		\dot{y} &= g(y) \label{inner_general}
%%	\end{align}
%%\end{subequations}
%under what conditions can the almost global asymptotic stability of the full cascade by deduced from properties of the subsystems? Unfortunately, such a property is not an immediate consequence of the almost global asymptotic stability of the subsystems given by \eqref{inner_general} and the limiting dynamics 
%\begin{equation}
%	\dot{x} = f(x,0_Y)
%	\label{unforced_general}
%\end{equation}
%where $0_Y \in Y$ is the stable equilibrium of \eqref{inner_general}. When the outer loop is \textit{globally} stable and $M = \mathbb{R}^n$, such a conclusion can be drawn under suitable conditions on the growth rate of the so-called interconnection term $f(x,y) - f(x,0_Y)$. Of course, since many manifolds do not admit globally stable smooth vector fields, such a result does not address many circumstances of interest. On the other hand, such a cascaded result can be obtained in the case that the inner subsystem \eqref{inner_general} has the strong property of being ``almost input to state stable'', which is a sort of bounded input bounded response property which requires that for any reasonable disturbance input $y(t)$, all but a measure zero set of initial conditions $x$ yield a trajectory whose error from the stable equilibrium of \eqref{unforced_general} is in a precise sense proportional to the input disturbance.



%is whether this can be deduced solely from the almost global stability of the subsystems. In general, such a conclusion would be false. However, under certain additional conditions, it is possible to draw such a conclusion when the unforced outer subsystem is \textit{globally} stable, or when the outer subsystem satisfies an appropriate almost-global notion of input to state stability. 

 
% We believe that using dynamical systems theory, we can obtain such a result predicated on the structural properties of the outer loop dynamics, namely the structure of the chain recurrent set, as well as a growth rate restriction on the so-called ``interconnection'' term. Such cascaded structures have been achieved for cascades of \textit{globally} asymptotically stable systems before, and for \textit{almost} globally asymptotically stable systems when the inner loop possesses strong properties, such as an almost-global generalization \cite{Angeli2004} of the celebrated Input to State Stability property, which means roughly that a bounded (e.g. constant, finite) disturbance must yield an asymptotically bounded response from the system. However, even our simple example lacks these properties, and is representative of the kinds of cascaded systems we expect to see from flat systems.
%
%
%
%need to discuss related work \cite{Angeli2010}
%\subsection{Motivating Example}
%
%Consider the dynamical system defined on $\mathbb{T}^2 = \mathbb{S}^1 \times \mathbb{S}^1$ given by
%\begin{subequations}
%	\begin{align}
%		\dot{x} &= -\sin x  - \sin y
%		\label{x_torus}
%		\\
%		\dot{y} &= - \sin y
%		\label{y_torus}
%	\end{align}
%\end{subequations}
%where the states $x$ and $y$ are modeled as angles in $\left[-\pi,\pi\right]$ (with $-\pi$ identified with $\pi$, i.e. the ``flat torus''). It can be shown that this system is almost-globally asymptotically stable to the equilibrium $(0,0)$, which can be informally ascertained from the phase portrait plotted in Fig. 1. 
%%to determine attractiveness and considering the linearization of the equilibrium $(0,0)$ to show Lyapunov stability. Indeed, all initial conditions except for the measure-zero set
%%\begin{equation}
%%	\left\{
%%		(x,y)  : y = \pi 
%%	\right\} \cup \left\{
%%	(\pi,0)
%%	\right\} \subseteq  \mathbb{T}^2
%%\end{equation}
%Indeed, almost all initial conditions converge to $(0,0)$, and the linearization at this point has eigenvalues $\{-1,-1\}$ so the system is Lyapunov (and moreover, locally exponentially) stable.
%
%We may remark that the $y$ dynamics are independent of the $x$ state, i.e. the system is an upper triangular cascade:
%\begin{subequations}
%	\begin{align}
%	\dot{x} &= f(x,y) \label{outer} \\
%	\dot{y} &= g(y) \label{inner}
%\end{align}
%\end{subequations}
%Furthermore, ${y = 0}$ is almost-globally asymptotically stable for ${\dot{y} = g(y)}$, the only initial condition not converging to this point being ${y = \pi}$. Moreover, the dynamics ${\dot{x} = f(x,0) = -\sin x}$ also have an almost globally asymptotically stable equilibrium at $x = 0$. 
%
%It is tempting to presume that the stability of the overall system is a consequence of the stability of the ``inner'' system \eqref{inner} combined with that of the ``outer'' system \eqref{outer} when restricted to the equilibrium of the inner system. 
%It may be reasonable to draw such a conclusion when $f(x,y) - f(x,0_Y)$ is small compared to $f(x,0_Y)$, or when $y$ converges very rapidly compared to $x$; however, neither of these arguments hold for this example, and yet the cascade is almost-globally asymptotic stable.
% 
%In this document, our interest is in determining some structural conditions under which stability properties of the cascade can be deduced from the properties of the decoupled subsystems $\dot{x} = f(x,0_Y)$ and $\dot{y} = g(y)$, as well as the so-called \textit{interconnection} term $f(x,y)-f(x,0)$ and topological properties of the state spaces of the subsystems. Such tools will facilitate compositionality in the design of nonlinear controllers for systems with cascaded structure.
%

%\subsection{Our Approach}
%In view of the above discussion and the motivating example, we raise the following question: 
%\begin{quote}
%	\centering
%	\textit{``What qualitative properties of the closed loop subsystems are sufficient to conclude almost global asymptotic stability of the full cascade?''}
%\end{quote}
%\begin{quote}
%\end{quote}
%Instead of relying on disturbance robustness properties of the outer loop or time scale separation, 
In what follows, we certify the almost global asymptotic stability of cascade systems in the form of $\Sigma$ in Fig. 1, using qualitative, dynamical properties of the other subsystems shown. 
%We first give a brief review in Sec. II of some essential concepts of dynamical systems theory which will be employed in our approach. 
In Sec. II, we present the main result, which pertains to cascades in which  $\Sigma_x$ is almost globally asymptotically stable with all chain recurrent points being hyperbolic equilibria,  
and $\Sigma_y$ is almost globally asymptotically stable and locally exponentially stable, and almost all forward trajectories of $\Sigma$ are precompact. We also discuss two broad and important classes of systems enjoying the stated chain recurrence criteria.
%, and almost all forward trajectories of $\Sigma$ are precompact.
%In fact, the desired structure for the chain recurrent set can easily be verified using a function decreasing outside the equilibria set, and two important classes of systems in geometric control enjoy this property. 
In Sec. III, we show that the precompactness criteria (analogous the forward boundedness of trajectories for a system in $\mathbb{R}^n$) can be verified using a growth rate inequality on the interconnection term coupling the subsystems. In Sec. IV, we return to the motivating example, before discussing the results and concluding the paper in Secs. V and VI.



%inner loop and the restriction of the outer loop to the stable equilibrium of the inner loop. 



%The remainder paper is structured as follows. In Sec. II, we give a brief review of some essential concepts of dynamical systems theory. In Sec. III, we present the main result, certifying the stability of a cascaded system under certain conditions on the subsystems. In Sec. IV, we give practical conditions for verifying the boundedness assumptions required for the main result. In Sec. V, we show that cascades of dynamical systems whose subsystems are constructed via very general and popular techniques meet the assumptions of our main result and our boundedness condition, suggesting the broad applicability of the results in certifying the stability of practically important systems. Finally, we conclude the paper in Sec. VI.

%
%			\item 		instead of I/O, rely on qualitative dynamical properties of closed loop systems: inner, and outer converged, subject to precompactness assumption
%\item check precompactness using an I/O growth rate condition on an algebraic i.e. stateless system
%\item suit for almost global stability, so can be applied to control of systems on manifolds, such as robots.


%\clearpage
%
%Introduction Outline
%\begin{enumerate}
%	\item Scene setting
%	\begin{enumerate}
%		\item define cascades
%		\item motivate their importance - theoretical, plus practical
%		\item appearance in mechanical systems and control approaches
%		\item configuration flat systems are cascaded.
%	\end{enumerate}
%	\item cascades in control design
%	\begin{enumerate}
%		\item role of compositionality
%	\end{enumerate}
%	\item Approaches to certify of stability
%	\begin{enumerate}
%		\item stability of linear systems preserved, not true for nonlinear - only local stability preserved.
%		\item singular perturbation models
%		\begin{enumerate}
%			\item fundamentally assume time scale separation
%			\item employed in slung load controller proof
%			\item assume inner loop can be sufficiently fast, which may not be possible
%		\end{enumerate}
%		\item Growth rate and lyapunov conditions
%		\begin{enumerate}
%			\item global stability can be preserved under certain conditions, such as growth rate conditions - but requires global stability
%			\item backstepping approaches in control - require control affine usually, explicit knowledge of lyapunov function
%		\end{enumerate}
%		\item approaches using I/O robustness of inner system
%		\begin{enumerate}
%			\item ISS preserved under cascades
%			\item generalization of ISS to almost global setting, cascade result
%			\item hard to prove almost ISS.  
%			\item ultimate boundedness not always hold, such as stability robustness paper
%			\item multistability relaxes Lyapunov stability requirements?
%		\end{enumerate}
%		\item Ours: 
%		\begin{enumerate}
%			\item 		instead of I/O, rely on qualitative dynamical properties of closed loop systems: inner, and outer converged, subject to precompactness assumption
%			\item check precompactness using an I/O growth rate condition on an algebraic i.e. stateless system
%			\item suit for almost global stability, so can be applied to control of systems on manifolds, such as robots.
%		\end{enumerate}
%	\end{enumerate}
%	\item Outline of paper
%\end{enumerate}
%

%\section{Background on Dynamical Systems Theory}
%To do: write this section! once the rough draft of the rest is complete so we know what's needed and how much space
%
%\textbf{Possibly: move all this stuff into the relevant sections to minimize ``zig-zagging'' while reading, since dynamical systems / CRS / almost global stuff can be moved to Sec. III, proper functions can be in Sec. IV, and Morse functions etc can be Sec. V}
%
%\begin{definition}
%	An \textit{equilibrium} of a vector field ${f : M \rightarrow TM}$ is any point $0_M \in M$ satisfying ${f(0_M) = 0}$. An equilibrium 
%	is \textit{hyperbolic} if all eigenvalues of its linearization $A : T_{0_M}M \rightarrow T_{0_M}M$ have nonzero real part. 
%\end{definition}


%\begin{definition}
%	A system is \textit{gradient-like} if its chain recurrent set consists of only isolated equilibria.
%\end{definition}

%\begin{definition}
%	A system on $M$ is \textit{nice with equilibrium $0_X \in M$} if its chain recurrent set consists of only hyperbolic equilibria, of which some equilibrium $0_X$ is almost globally asymptotically stable.
%\end{definition}

%Clearly, since the set of hyperbolic equilibria is always totally disconnected, nice systems are gradient-like.

%It may seem difficult to verify the niceness of a given system, much less construct a nice system. 

%In the following proposition, we give an example of generating gradient-like systems in the following straightforward manner. In fact, the following is a common technique for designing geometric feedback controllers for mechanical systems evolving on manifolds.





%define boundedness of trajectories

%Riemannian norm:
%$$
%\big|\big| \cdot \big|\big|_M : TM \rightarrow \mathbb{R}, \,  v_m \mapsto \sqrt{\langle\langle v_m, v_m \rangle\rangle}.
%$$
%norm of differential of map between Riemannian manifolds:
%\begin{gather*}
%	\big|
%	\cdot
%	\big| : \mathcal{L}(T_mM,T_nN) \rightarrow \mathbb{R}, \\
%	 dW_m \mapsto 
%	\inf \Big\{
%	c \geq 0 :
%	\big|\big| 
%	dW_m(v_m) 
%	\big|\big|_N 
%	\leq \\ 
%	c \big|\big|v_m\big|\big|_M 
%	\ \forall \ v_m \in T_m M
%	\Big\}
%\end{gather*}
%norm of the differential of a real-valued function on a Riemannian manifold:
%\begin{gather*}
%	\big|
%	\cdot
%	\big| : \mathcal{L}(T_mM,\mathbb{R}) \rightarrow \mathbb{R}, \\
%	dW_m \mapsto 
%	\inf \big\{
%	c \geq 0 :
%	\big| 
%	dW_m(v_m) 
%	\big| 
%	\leq  \\ c \big|\big|v_m\big|\big|_M 
%	\ \forall \ v_m \in T_m M
%	\big\}
%\end{gather*}
%\begin{gather*}
%%	\big|
%%	\cdot
%%	\big| : \mathcal{L}(T_mM,\mathbb{R}) \rightarrow \mathbb{R}, \\
%%	dW_m \mapsto 
%\big|	dW_m \big| =  
%	\sup_{v_m 
%%\in T_mM - \{0_X\}
%\neq 0_X	
%}
%	\frac{dW_m(v_m)}{\big|\big|v_m\big|\big|_M }
%\end{gather*}
%
%%For any point $0_X \in M$, we will write
%%$$
%%\left| \ \cdot \ \right|_{0_X} : M \rightarrow \mathbb{R}, \ x \mapsto \mathrm{d}(x,0_X)
%%$$
%%where $\mathrm{d} : M \times M \rightarrow \mathbb{R}$ is the distance metric induced by a complete Riemannian metric. Sometimes we will drop the reference point $0_X$ when it is clear from context.

%\clearpage 
%\section{Stability of Cascades on Compact Manifolds}
%
%%\subsection{Cascades of Two Systems}
%
%%We have the following theorem.
%
%%\begin{theorem}
%%	\label{simple_cascade}
%%	For compact manifolds without boundary $M$ and $N$, consider the cascade on $X \times Y$ given by
%%	\begin{subequations}
%%		\begin{align}
%%			\dot{x} &= f(x,y) \label{outer_theorem} \\
%%			\dot{y} &= g(y) \label{inner_theorem} 
%%		\end{align}
%%	\end{subequations}
%%	where ${f(0_X,0_Y) = 0}$ and ${g(0_Y) = 0}$ for some points ${0_X \in M}$ and ${0_Y \in N}$.
%%%	 and all trajectories are bounded.
%%	Suppose that:
%%	\begin{enumerate}
%%		\item The point ${0_X \in M}$ is an almost-globally asymptotically stable equilibrium of the dynamics 
%%		\begin{equation}
%%			{\dot{x} = f(x,0_Y)}
%%			\label{unforced}
%%		\end{equation}
%%		and moreover the chain recurrent set of \eqref{unforced} consists solely of hyperbolic equilibria.
%%		\item The point ${0_Y \in N}$ is a hyperbolic, almost-globally asymptotically stable equilibrium of \eqref{inner_theorem}.
%%	\end{enumerate}
%%	Then, $(0_X,0_Y)$ is an almost globally asymptotically stable and locally exponentially stable equilibrium of \eqref{outer_theorem}-\eqref{inner_theorem}.
%%\end{theorem}
%
%\begin{theorem}
%	\label{simple_cascade}
%	For compact manifolds without boundary $M$ and $N$, consider the cascade on $X \times Y$ given by
%	\begin{subequations}
%		\begin{align}
%			\dot{x} &= f(x,y) \label{outer_theorem} \\
%			\dot{y} &= g(y) \label{inner_theorem} 
%		\end{align}
%	\end{subequations}
%%	where ${f(0_X,0_Y) = 0}$ and ${g(0_Y) = 0}$ for some points ${0_X \in M}$ and ${0_Y \in N}$.
%	%	 and all trajectories are bounded.
%	and suppose the following conditions hold:
%	\begin{enumerate}
%		\item The point ${0_Y \in N}$ is an almost-globally asymptotically stable equilibrium of \eqref{inner_theorem}.
%		\item The point ${0_X \in M}$ is an almost-globally asymptotically stable equilibrium of the dynamics 
%		\begin{equation}
%			{\dot{x} = f(x,0_Y)}.
%			\label{unforced}
%		\end{equation}
%%		and moreover the chain recurrent set of \eqref{unforced} consists solely of hyperbolic equilibria.
%%		and moreover the chain recurrent set of \eqref{inner_theorem} consists solely of hyperbolic equilibria.
%		\item The chain recurrent sets of \eqref{inner_theorem} and \eqref{unforced} consist solely of hyperbolic equilibria.
%	\end{enumerate}
%	Then, $(0_X,0_Y)$ is an almost globally asymptotically stable and locally exponentially stable equilibrium of \eqref{outer_theorem}-\eqref{inner_theorem}.
%\end{theorem}
%
%%\begin{theorem}[Cascade of Two Nice Systems]
%%	For compact manifolds without boundary $M$ and $N$, consider the cascade system on $X \times Y$ given by
%%\begin{subequations}
%%	\begin{align}
%%		\dot{x} &= f(x) + h(x,y) \label{outer_theorem} \\
%%		\dot{y} &= g(y) \label{inner_theorem} 
%%	\end{align}
%%\end{subequations}	and suppose that 
%%${\dot{x} = f(x)}$ is nice with equilibrium ${0_X \in M}$, while
%%${\dot{y} = g(y)}$ is nice with equilibrium ${0_Y \in N}$. Moreover, suppose that 
%%\begin{equation}
%%	{g(y) = 0 \implies h(x,y) = 0}.
%%	\label{vanishing_at_inner_eql}
%%\end{equation}
%%i.e. the interconnection term vanishes at every equilibrium of the inner subsystem. %$h(x,y) = 0$ for \textrm{all} $y \in \{y \in N : g(y) = 0\}$. 
%%Then, the cascade system \eqref{outer_theorem}-\eqref{inner_theorem} is nice with equilibrium $(0_X,0_Y)$.
%%\label{simple_cascade}
%%\end{theorem}
%
%
%%\begin{conjecture}
%%	Suppose that all trajectories of the cascade 
%%	\begin{align}
%%		\dot{x} &= f(x,y) \label{outer_theorem} \\
%%		\dot{y} &= g(y) \label{inner_theorem} 
%%	\end{align}
%%	are bounded. Furthermore, suppose that ${0_Y \in Y}$ and ${0_X \in X}$ are almost-globally asymptotically stable equilibria of the systems ${\dot{y} = g(y)}$ and ${\dot{x} = f(x,0_Y)}$ respectively, and that these two systems are gradient-like. Then, the cascade \eqref{outer_theorem}-\eqref{inner_theorem} is gradient-like, and $(0_X,0_Y)$  is an almost-globally asymptotically stable equilibrium.
%%	%	 with respective regions of attraction $\mathcal{A}_X \subseteq X$ and $\mathcal{A}_Y \subseteq Y$.  
%%	%	 then there exists a measure-zero set $U \subseteq X \times Y$ such that all solutions of the cascade \eqref{outer}-\eqref{inner} starting in $(X \times Y) \setminus U$ are either unbounded or converge to $(0_X,0_Y)$.
%%\end{conjecture}
%%
%%\begin{theorem}
%%	Suppose that all trajectories of the cascade 
%%	\begin{align}
%%		\dot{x} &= f(x,y) \label{outer_theorem} \\
%%		\dot{y} &= g(y) \label{inner_theorem} 
%%	\end{align}
%%	are bounded. Suppose that ${0_Y \in Y}$ and ${0_X \in X}$ are almost-globally asymptotically stable equilibria of the decoupled systems ${\dot{y} = g(y)}$ and ${\dot{x} = f(x,0_Y)}$ respectively, and moreover that the respective chain recurrent sets of the decoupled systems consist solely of hyperbolic equilibria.
%%%	and suppose all respective equilibria of these systems are hyperbolic. Finally, suppose that every chain recurrent point of $\dot{x} = f(x,0_Y)$ is an isolated equilibrium. 
%%	Then, $(0_X,0_Y)$ 
%%	is an almost-globally asymptotically stable equilibrium of the cascade \eqref{outer_theorem}-\eqref{inner_theorem}. 
%%	
%%%	WTS: Furthermore, the chain recurrent set of the cascaded system consists solely of hyperbolic equilibria.
%%%	 with respective regions of attraction $\mathcal{A}_X \subseteq X$ and $\mathcal{A}_Y \subseteq Y$.  
%%%	 then there exists a measure-zero set $U \subseteq X \times Y$ such that all solutions of the cascade \eqref{outer}-\eqref{inner} starting in $(X \times Y) \setminus U$ are either unbounded or converge to $(0_X,0_Y)$.
%%\end{theorem}
%\begin{proof}
%%	Denoting by ${X : (x,y) \mapsto \big(f(x,y),g(y)\big)}$
%%	%	X \times Y \rightarrow TX \times TY
%%	the combined vector field on $X \times Y$ describing the full cascade, we may express the linearization of the cascade at the equilibrium ${(0_X,0_Y)}$ in block matrix form as
%%%	\begin{equation}
%%%		\left.d X \right|_{(0_X,0_Y)} = \begin{bmatrix}
%%%			\left. d_x f \,  \right|_{x^\star} 
%%%			%	 		+ \left. d_x h \,  \right|_{(x^\star,y^\star)} 
%%%			&
%%%			\left. d_y h \,  \right|_{(x^\star,y^\star)} \\
%%%			0 &
%%%			\left.d_y g \, \right|_{y^\star\hphantom{x^\star,()}} \\
%%%		\end{bmatrix}.
%%%	\end{equation}
%%	 \begin{equation}
%%			\left.d X \right|_{(0_X,0_Y)} = \begin{bmatrix}
%%					\left. \partial_x f \,  \right|_{(0_X,0_Y)} &
%%					\left. \partial_y f \,  \right|_{(0_X,0_Y)} \\
%%					0 &
%%					\left.\partial_y g \, \right|_{\, 0_Y\hphantom{0_X,()}} \\
%%				\end{bmatrix}{.}
%%		\end{equation}
%%%		 Moreover, since $f(0_X,0_Y) = 0$ we have
%%		In view of the fact that the eigenvalues of an upper triangular block matrix depend only on the eigenvalues diagonal blocks, the hyperbolicity and stability assumptions on the inner loop dynamics \eqref{inner_theorem} and limiting dynamics \eqref{unforced} immediately imply that $(0_X,0_Y)$ is hyperbolic and therefore locally exponentially stable. Hence, to complete the proof it will suffice to show that $(0_X,0_Y)$ is almost globally attractive. 
%		
%Since $M$ and $N$ are compact and without boundary,
%the dynamics \eqref{outer_theorem}-\eqref{inner_theorem}, \eqref{unforced}, and \eqref{inner_theorem} each induce a well-defined flow, denoted respectively by ${\Phi^t : X \times Y \rightarrow X \times Y}$, ${\phi^t : M \rightarrow M}$, and ${\varphi^t : N \rightarrow N}$.
%% Note that we have deliberately chosen a restricted domain for $\Phi^t$ for the purposes of our argument.
%Let ${\mathcal{B}_Y \subseteq N}$ denote the basin of attraction of $0_Y$ with respect to \eqref{inner_theorem}. 
% We now prove the claim in five steps.
%%		almost global attractiveness of ${(0_X,0_Y)}$ 
%%		 almost all trajectories converge to ${(0_X,0_Y)}$.
%
%%		We prove this claim in three steps.
%		
%%	It is immediately clear that the set of equilibria of the cascade is given precisely by
%%	\begin{equation}
%%		\left\{
%%			(x^\star,y^\star) : f(x^\star) = 0 \textrm{ and }g(y^\star) = 0
%%		\right\}.
%%	\end{equation}
%%	Denoting by ${X : (x,y) \mapsto \big(f(x) + h(x,y),g(y)\big)}$
%%%	X \times Y \rightarrow TX \times TY
%%	 the combined vector field describing the full cascade and 
%%	 relying on \eqref{vanishing_at_inner_eql}, we may express the linearization of the cascade at any equilibrium $(x^\star,y^\star)$ in the form 
%%	 \begin{equation}
%%	 		\left.d X \right|_{(x^\star,y^\star)} = \begin{bmatrix}
%%	 		\left. d_x f \,  \right|_{x^\star} 
%%%	 		+ \left. d_x h \,  \right|_{(x^\star,y^\star)} 
%%	 		&
%%	 		\left. d_y h \,  \right|_{(x^\star,y^\star)} \\
%%	 		0 &
%%	 		\left.d_y g \, \right|_{y^\star\hphantom{x^\star,()}} \\
%%	 	\end{bmatrix}.
%%	 \end{equation}
%%% Moreover, since $h(x,0_Y) = 0$ we have
%%% \begin{equation}
%%%	\left.d X \right|_{(0_X,0_Y)} = \begin{bmatrix}
%%%		\left. d_x f \,  \right|_{0_X} &
%%%		\left. d_y h \,  \right|_{(0_X,0_Y)} \\
%%%		0 &
%%%		\left.d_y g \, \right|_{0_Y\hphantom{0_X,()}} \\
%%%	\end{bmatrix}.
%%%\end{equation}
%%	In view of the fact that the eigenvalues of an upper triangular block matrix depend only on the eigenvalues diagonal blocks, we may conclude from the hyperbolicity and stability assumptions that all equilibria of the cascade are hyperbolic, and moreover $(0_X,0_Y)$ is locally asymptotically stable. 
%%		
%%	
%%	Hence, for almost-global asymptotic stability of $(0_X,0_Y)$, it will suffice to also show that $(0_X,0_Y)$ is almost-globally attractive, i.e. almost all initial conditions converge to $(0_X,0_Y)$. We prove this claim in three steps.
%%	
%
%\begin{step}
%${\mathcal{E}(\Phi^t) \cap (X \times \mathcal{B}_Y) = 
%\mathcal{R}(\phi^t) \times \{0_Y\}}$. 
%%All 
%%%equilibria of \eqref{outer_theorem}-\eqref{inner_theorem} in $X \times \mathcal{B}_Y$ 
%%equilibria of $\Phi^t$ contained in ${X \times \mathcal{B}_Y}$
%%are hyperbolic. 
%Moreover, this set consists solely of hyperbolic equilibria, of which ${(0_X,0_Y)}$ is stable while all others are unstable.
%%For all ${x \in M}$ such that ${f(x,0_Y) = 0}$, 	the point ${(x,0_Y)}$ is a hyperbolic equilibrium of \eqref{outer_theorem}-\eqref{inner_theorem}.
%\end{step}
%\begin{proof}
%		\renewcommand{\qedsymbol}{\rotatebox[origin=c]{180}{$\triangle$}}
%		All equilibria  $(x,y) \in {X \times \mathcal{B}_Y}$ must have ${y = 0_Y}$ by the definition of $\mathcal{B}_Y$ as a basin of attraction, and therefore it must also hold that ${f(x,0_Y) = 0}$ i.e. $x$ must be an equilibrium of \eqref{unforced}. Since all equilibria are chain recurrent, the equality follows from the assumptions on the chain recurrent set of \eqref{unforced}.
%		Denoting by ${X : (x,y) \mapsto \big(f(x,y),g(y)\big)}$
%	%	X \times Y \rightarrow TX \times TY
%	the combined vector field on $X \times Y$ describing the full cascade, we may express the linearization of the cascade at any equilibrium ${(x,0_Y) \in X \times \mathcal{B}_Y}$ in block matrix form as
%%		\begin{equation}
%%				\left.d X \right|_{(x^\star,y^\star)} = \begin{bmatrix}
%%						\left. d_x f \,  \right|_{(x^\star,y^\star)}
%%						&
%%						\left. d_y f \,  \right|_{(x^\star,y^\star)} \\
%%						0 &
%%						\left.d_y g \, \right|_{y^\star\hphantom{x^\star,()}} \\
%%					\end{bmatrix}.
%%			\end{equation}
%	\begin{equation}
%		\left.d X \right|_{(x,0_Y)} = \begin{bmatrix}
%			\left. \partial_x f \,  \right|_{(x,0_Y)} &
%			\left. \partial_y f \,  \right|_{(x,0_Y)} \\
%			0 &
%			\left.\partial_y g \, \right|_{\, 0_Y\hphantom{x,()}} \\
%		\end{bmatrix}{.}
%	\end{equation}
%	Since the eigenvalues of a triangular block matrix are simply the eigenvalues of the diagonal blocks, the 
%%	hyperbolicity of $(x,0_Y)$
%	claim of hyperbolicity follows from
%the	hyperbolicity of $0_Y$ for \eqref{inner_theorem} and the structure of the chain recurrent set of \eqref{unforced}, since all equilibria are chain recurrent. Furthermore, the top left block has all negative eigenvalues at ${x = 0_X}$ but at least one positive eigenvalue at all other equilibria of \eqref{unforced}, since an almost globally asymptotically stable system has exactly one stable equilibrium. Therefore $(0_X,0_Y)$ is (locally exponentially) stable while all other equilibria in $X \times \mathcal{B}_Y$ are unstable.
%%immediately imply that is hyperbolic.
%%	 and therefore locally exponentially stable. Hence, to complete the proof it will suffice to show that $(0_X,0_Y)$ is almost globally attractive. 
%\end{proof}
%\noindent Hence, to complete the proof it will suffice to show that the stable equilibrium $(0_X,0_Y)$ is almost globally attractive.
%%\begin{step}
%%$(0_X,0_Y)$ is locally exponentially stable.
%%\end{step}
%%\begin{proof}
%%	\renewcommand{\qedsymbol}{\rotatebox[origin=c]{180}{$\triangle$}}
%%Also, $(0_X,0_Y)$ is hyperbolic by Step 1, and moreover it is locally exponentially stable due to the stability assumptions on \eqref{inner_theorem} and \eqref{unforced}.
%%\end{proof}
%\begin{step}
%	$\mathcal{R}(\Phi^t) \cap (X \times \mathcal{B}_Y) = \mathcal{R}(\phi^t) \times \left\{0_Y\right\}$.
%%	$\mathcal{R}(\Phi^t) = \mathcal{R}(\phi^t) \times \left\{0_Y\right\} = \mathcal{E}(\Phi^t)$.
%%	The chain recurrent set 
%%	All chain recurrent points of $\Phi^t$ are of the form $(x,0_Y)$.
%%	For all ${(x,y) \in \mathcal{R}(\Phi^t)}$, we have ${y = 0_Y}$. 
%%	consists of all $\omega$-limit points of  $\Phi^t$.
%\end{step}
%\begin{proof}
%		\renewcommand{\qedsymbol}{\rotatebox[origin=c]{180}{$\triangle$}}
%%		Every $\omega$-limit point is chain recurrent, so it suffices to show that any point not contained in an $\omega$-limit set is not chain recurrent.
%		We first show that all points ${(x,y) \in X \times \mathcal{B}_Y}$ with $y \neq 0_Y$ are not chain recurrent with respect to $\Phi^t$. It is clear that ${\mathcal{R}(\varphi^t) \cap \mathcal{B}_Y = \{0_Y\}}$, thus
%		\begin{equation}
%			\varphi^t\big(
%				y
%			\big) = 
%		\end{equation}
%		
%		Can we argue that $y \neq 0_Y$ is not chain recurrent for \eqref{inner_theorem} only since it is in the basin of attraction of $0_Y$, and therefore using any product metric the triangle inequality determines that it is not chain recurrent for the full cascade.
%		
%		By some converse Lyapunov theorem, there exists a Lyapunov function ${V : \mathcal{B}_Y \rightarrow \mathbb{R}}$ for \eqref{inner_theorem} such that 
%		\begin{align}
%			V = 0 & \iff y = 0_Y
%			\\
%			\dot{V} < 0 & \iff y \in \mathcal{B}_Y \setminus \left\{0_Y\right\}.
%		\end{align} 
%		Considering any initial point ${(x_0,y_0) \in M \times (\mathcal{B}_Y \setminus \left\{0_Y\right\})}$,
%%		Defining ${W : (x,y) \mapsto V(y)}$,
%		we can show (TODO: show!) via continuity and compactness that for sufficiently large $T$ and sufficiently small $\epsilon$, 
%		\begin{multline}
%			\Phi^T\Big(
%			M \times V^{-1}\big(\left[0,V(y_0)\right]\big)
%%			\big\{(x,y) : V(y) \leq V(y_0)\big\}
%			\Big) \subset
%			\\ \left\{(x,y) : d_{M\times N}\big((x_0,y_0),(x,y)) > \epsilon \right\}
%		\end{multline}
%		and therefore ${(x,y)}$ is not chain recurrent if ${y \neq 0_Y}$.
%%		, while on the contrary all equilibria  On the contrary, all equilibria are chain recurrent
%%		Thus, the chain recurrent set and the $\omega$-limit set of the flow on ${X \times \mathcal{B}_Y}$  are identical.
%%		Since all trajectories are bounded, each initial condition converges to its $\omega$-limit set
%%\end{proof}
%%
%%
%%\begin{step}
%%	%	The chain recurrent set 
%%	%	All chain recurrent points of $\Phi^t$ are of the form $(x,0_Y)$.
%%	For all ${(x,y) \in \mathcal{R}(\Phi^t)}$, we have ${f(x,0_Y) = 0}$. 
%%	%	consists of all $\omega$-limit points of  $\Phi^t$.
%%\end{step}
%%\begin{proof}
%%	\renewcommand{\qedsymbol}{\rotatebox[origin=c]{180}{$\triangle$}}
%	Furthermore, since $X \times Y$ is compact, by the restriction property all chain recurrent points are chain recurrent via {$(\epsilon,T)$-chains} contained in $\mathcal{R}(\Phi^t)$, that is
%	\begin{align}
%				\mathcal{R}(\Phi^t) 
%		&= \mathcal{R}\big(\Phi^t|_{\mathcal{R}(\Phi^t) } \big).
%	\end{align}
%	Having  just shown ${\mathcal{R}(\Phi^t) \subseteq M \times \{0_Y\}}$, we therefore obtain
%	\begin{align}
%		\mathcal{R}(\Phi^t) 
%%		&= \mathcal{R}(\Phi^t|_{\mathcal{R}(\Phi^t) } ) \\
%		&= \mathcal{R}(\Phi^t|_{M \times \{0_Y\}} ) \\
%		&= \mathcal{R}(\phi^t) \times \{0_Y\} 
%	\end{align}
%	where the last equality follows because \eqref{outer_theorem} is independent of $y$ on the invariant set ${M \times \{0_Y\}}$. Finally, all equilibria are chain recurrent, and by assumption all points in $\mathcal{R}(\phi^t)$ are equilibria for \eqref{unforced}, yielding the final equality to be shown.
%%	 and $M \times 0_Y$ is an equilibrium of \eqref{inner_theorem}, while 
%\end{proof}
%%\begin{step}
%%%	The chain recurrent set of the flow on $X \times \mathcal{B}_Y$ is contained in the set $\left\{(x,y) \in X \times \mathcal{B}_Y : y = 0_Y\right\}$.
%%	All $\omega$-limit points of $\Phi^t$ take the form $(x,0_Y)$.
%%\end{step}
%%\begin{proof}
%%	\renewcommand{\qedsymbol}{\rotatebox[origin=c]{180}{$\triangle$}}
%%	It is clear that the $\omega$-limit set of \eqref{inner_theorem} over $\mathcal{B}_Y$ is exactly $\left\{0_Y\right\}$, and since \eqref{inner_theorem} is independent of the evolution of the other subsystem, the $\omega$-limit set of the cascade over $X \times \mathcal{B}_Y$ must also have ${y = 0_Y}$, so the claim follows from the correspondence in the previous step.
%%\end{proof}
%
%
%	\begin{step}
%		Every trajectory beginning in ${X \times \mathcal{B}_Y}$ converges to a hyperbolic equilibrium.
%%		satisfying ${f(x,0_Y) = 0}$.
%%		 point in $\mathcal{R}(\Phi^t)$.
%		
%%		the set $$E = \big\{(x,0_Y) \in X \times Y : f(x,0_Y) = 0\big\}$$
%%		 \subseteq X \times \mathcal{B}_Y . 
%%		i.e. for almost all initial conditions, $y$ converges to the equilibrium $0_Y$ and $x$ converges to some equilibrium of the system $\dot{x} = f(x,0_X)$.
%	\end{step}
%	\begin{proof}
%		\renewcommand{\qedsymbol}{\rotatebox[origin=c]{180}{$\triangle$}}
%%			Any $x \in \mathcal{R}(\phi^t)$ is a hyperbolic equilibrium of \eqref{unforced} by assumption. Thus by Step 1 and the same eigenvalue argument as above, all points $(x,y) \in \mathcal{R}(\Phi^t)$ are hyperbolic equilibria of \eqref{outer_theorem}-\eqref{inner_theorem}
%%			By Step 2, 
%			Since $X \times Y$ is compact, each trajectory converges to its $\omega$-limit set, and all $\omega$-limit points are chain recurrent. Moreover, ${X \times \mathcal{B}_Y}$ is invariant by construction, thus all trajectories beginning in ${X \times \mathcal{B}_Y}$ converge to the set ${\mathcal{R}(\Phi^t) \cap (X \times \mathcal{B}_Y)}$. By Step 2, this set is exactly ${\mathcal{R}(\phi^t) \times \{0_Y\}}$, which by Step 1 consists solely of hyperbolic (and therefore isolated) equilibria.
%			Thus, a simple continuity argument shows that each trajectory beginning in ${M\times \mathcal{B}_Y}$ converges to a particular hyperbolic equilibrium.
%%%			$(x,y) \in \mathcal{R}(\Phi^t)$, 
%%%			
%%%			 is hyperbolic and therefore isolated for all $\mathcal{R}(\Phi^t)$ 
%%%						
%%%			Moreover, $y$ converges to $0_Y$ from every initial condition in $\mathcal{B}_Y$ by definition, and therefore the chain recurrent set is entirely contained in the set $M \times \left\{0_Y\right\}$. 
%%%			Furthermore, by the restriction property of the chain recurrent set, each point in the chain recurrent set is chain recurrent via an $(\epsilon,T)$-chain consisting of only points in the chain recurrent set itself.
%%%			
%%%			we have 
%%%			$$
%%%			\mathcal{R}(X |_{X \times \mathcal{B}_Y}) =
%%%			\mathcal{R}(X |_{\mathcal{R}(X |_{X \times \mathcal{B}_Y})})
%%%			$$
%%%			
%%%			
%%%			Thus, denoting by $\mathcal{R}(X)$ the chain recurrent set of the flow on $X \times \mathcal{B}_Y$, every initial condition converges to the set 
%%%			$$
%%%			\big\{(x,y) \in \mathcal{R}(X) : y = 0_Y \big\}.
%%%			$$
%%
%%%			We now argue that $\mathcal{R(X)|_{\mathcal{X \times \mathcal{B}_Y}}} = \mathcal{\omega(X \times \mathcal{B}_Y})$.
%%%			Furthermore, 
%%%			
%%%			
%%%			Furthermore, the restriction property states that the chain recurrent set is internally chain recurrent, i.e. all points in $\mathcal{R}(X)$ the chain recurrent set of the flow is the same when restricted to its chain recurrent set. 
%%%Moreover, almost every trajectory converges to
%%%			$$
%%%\big\{(x,y) \in \mathcal{R}(X |_{\mathcal{R}(X)}) : y = 0_Y \big\},
%%%$$
%%%and since $\mathcal{R}(X)$ consists of hyperbolic (therefore isolated) equilibria, 
%%%
%%%			and by Conley's argument (citation/explanation?), this is identical to the set 
%%%Therefore,
%%%
%%%\begin{align}
%%%	\mathcal{R}(X) 
%%%	&= \mathcal{R}\big(X(\cdot,0_Y) \big) \\
%%%	&= \mathcal{R}\big(f(\cdot,0_Y) \big) \times \left\{0_Y\right\}
%%%\end{align}
%%%and by assumption, $\mathcal{R}\big(f(\cdot,0_Y)\big)$ consists solely of hyperbolic (hence isolated) equilibria. Therefore, a continuity argument implies that almost every initial condition converges to some particular point in the set $E$.
%	\end{proof}
%	\begin{step}
%		The set of initial conditions in ${X \times \mathcal{B}_Y}$ 
%		converging to an unstable equilibrium is measure zero in $X \times Y$.
%	\end{step}
%	\begin{proof}
%	\renewcommand{\qedsymbol}{\rotatebox[origin=c]{180}{$\triangle$}}
%%	All trajectories starting in ${X \times \mathcal{B}_Y}$ converge to a hyperbolic equilibrium by Step 2, 
%	
%	By definition, all points converging to a hyperbolic equilibrium lie on its global stable manifold. By the stable manifold theorem, the global stable manifold of an unstable hyperbolic equilibrium is a countable union of embedded submanifolds of measure zero, and therefore is also measure zero. Furthermore, by Step 1, all unstable equilibria in $X \times \mathcal{B}_Y$ are hyperbolic, and there can be only countably many unstable equilibria due to the isolation of the equilibria and the second countability of the topology of $M \times {N}$. Therefore, the set of all points in $X \times \mathcal{B}_Y$ converging to an unstable hyperbolic equilibrium is the union of countably many measure zero sets and thus is also measure zero.
%%	Need to clarify that all the equilibria we're talking about here are hyperbolic. 
%%	Thus, its complement is full measure, and therefore almost every point does \textit{not} converge to an unstable equilibrium. 
%	\end{proof}
%	\begin{step}
%	Almost every initial condition in ${X \times Y}$ converges to the stable equilibrium ${(0_X,0_Y)}$.
%\end{step}
%\begin{proof}
%%	\renewcommand{\qedsymbol}{\rotatebox[origin=c]{180}{$\triangle$}}
%	By Step 3, $X \times \mathcal{B}_Y$ can be partitioned into two set $U$ and $S$, respectively containing those initial conditions converging to a stable or unstable equilibrium. Hence, $X \times Y$ is partitioned by the three sets $U$, $S$, and ${C = (M \times \mathcal{N})'}$. 	
%	Since $\mathcal{B}_Y$ is full measure in $N$ by assumption,
%%	 the complement of $X \times \mathcal{B}_Y$
%	 $C$
%	  is  measure zero in ${X \times Y}$. Since $U$ and $C$ are measure zero, and measure zero sets are closed under countable (hence finite) unions, it follows that $S$ is the complement of a measure zero set and thus full measure. Hence it suffices that all initial conditions in $S$ converge to a stable equilibrium, since ${(0_X,0_Y)}$ is the unique stable equilibrium in $S$ by Step 1.
%%	
%%	 Furthermore, it is clear that 
%%	
%%	, and furthermore $S$ is full measure in $X \times Y$, since .
%%	
%%	
%%	
%%	all trajectories beginning in $X \times \mathcal{B}_Y$ converge to a hyperbolic equilibrium, and by Step 3, only the trajectories through a measure zero set of initial conditions converge to an unstable hyperbolic equilibrium. Since all equilibria are either stable hyperbolic or unstable hyperbolic, 
%%	
%%	
%%	. Therefore, all trajectories in the full measure set $X \times \mathcal{B}_Y$ not converging to an unstable equilibrium converge to a stable equilibrium. Since subtracting a measure zero set from a set of full measure yields a set of full measure, we have shown that almost all trajectories in $X \times Y$ converge to a stable equilibrium.
%%%	Since equilibria may be either stable or unstable,
%%%	
%%%	 and only a measure zero set 
%%%	. Moreover, the set of all initial conditions converging to 
%%%	
%%%	Since the 
%%%
%%%	Since measure zero sets are closed under countable (and hence finite) unions, it follows that the intersection of the complement of $X \times \mathcal{B}_Y$ with the set of all trajectories beginning in $X \times \mathcal{B}_Y$ but converging to an unstable equilibrium is measure zero. Furthermore, the previous steps showed that all trajectories 
%%%	
%%%	from Steps 1 and 2 that almost all initial conditions converge to a stable equilibrium.
%%%	
%%	 This proves the claim, since the limiting dynamics \eqref{unforced} have only one stable equilibrium and by the same eigenvalue argument as above, the full cascade can have only one stable equilibrium, namely ${(0_X,0_Y)}$.
%%%	 
%%%	 an almost-globally asymptotically stable system has a unique stable equilibrium. 
%\end{proof}
%\renewcommand{\qedsymbol}{}
%\end{proof}

\section{Almost Global Asymptotic Stability}

\iffalse
In this section, we present the main result, a sufficient condition for the almost global\footnote{	A subset ${S \subseteq M}$ is called \textit{almost global} if it is full measure and residual, and \textit{almost empty} if it is measure zero and meager.} asymptotic stability of a cascade (system $\Sigma$ in Fig. 1) for which the inner loop (system $\Sigma_y$ in Fig. 1) and the unforced outer loop (system $\Sigma_x$ in Fig. 1) are almost globally asymptotically stable.
\fi
% A central ingredient of the approach is the behavior of asymptotically autonomous semiflows described in \cite{Mischaikow1995}, whose definitions we adopt.
%

%\subsection{Autonomous and Nonautonomous Semiflows}
We first give a brief review of relevant concepts from dynamical systems theory, adopting
the definitions of \cite{Mischaikow1995}, whose perspective on the behavior of asymptotically autonomous semiflows is a central ingredient of our approach.
% A  is the behavior of asymptotically autonomous semiflows described in  whose definitions we adopt.



\begin{definition}
	A \textit{nonautonomous semiflow} on a smooth Riemannian manifold $(M,\kappa)$ is a continuous map
	\begin{equation}
		{\Phi : \{(t,s) : 0 \leq s \leq t < \infty \} \times M \rightarrow M}
	\end{equation}
	such that ${\Phi(s,s,x) = x}$ and ${\Phi\big(t,s,\Phi(s,r,x)\big) = \Phi(t,r,x)}$ for all ${ t \geq s \geq r > 0}$. A semiflow is called \textit{autonomous} when additionally,
	${\Phi(t+r,s+r,x) = \Phi(t,s,x)}$ for all $r > 0$.
\end{definition}
In the previous, the parameters $s$ and $t$ can be thought of as respective ``start'' and ``end'' times. Hereafter, we will use the shorthands
%\begin{align}
${	{\Phi^{(t,s)} : M  \rightarrow M}, \, x \mapsto \Phi(t,s,x)}$
%\end{align}
for nonautonomous semiflows and 
%\begin{align}
${	{\Phi^{t} : M  \rightarrow M}, \, x \mapsto \Phi(t,0,x)}$
%\end{align}
for autonomous semiflows. 

\begin{definition}
	The \textit{equilibrium set} of an autonomous semiflow $\Phi$ is the set ${\mathcal{E}(\Phi) = \left\{x : \Phi^t(x) = x \ \forall \ t \geq 0\right\}}$. 	
\end{definition}

\begin{definition}
	For an autonomous semiflow $\Phi$ on $(M,\kappa)$ and constants ${\varepsilon, T > 0}$,
	an \textit{$(\varepsilon,T)$-chain} is a pair of finite sequences $(x_0, x_1,\ldots,x_n)$  and $(t_1,t_2,\ldots,t_n)$ satisfying
	\begin{equation}
		{\mathrm{dist}\big(\Phi^{t_i}(x_{i-1}),x_i\big) < \varepsilon}\textrm{ and }{t_i > T}, \ i = 1, 2, \ldots, n,
	\end{equation}
	where ${\mathrm{dist} : M \times M \rightarrow \mathbb{R}}$ is the distance function induced by $\kappa$. A \textit{closed $(\varepsilon,T)$-chain at $x$} has $x = x_0 = x_n$.
\end{definition}
\begin{definition}
	The \textit{chain recurrent set} of an autonomous semiflow $\Phi$ is the set $\mathcal{R}(\Phi)$ consisting of all points at which there exists a closed $(\varepsilon,T)$-chain for all $\varepsilon, T > 0$.
\end{definition}

\begin{remark}
	\label{choice_of_setting}
	We define chain recurrence using $(\varepsilon,T)$-chains with respect to a distance function and some $\varepsilon > 0$ (versus using positive functions for $\varepsilon$  \cite{Hurley1992} or $(\mathcal{U},T)$-chains \cite{Conley1978}) because we rely on the results of \cite{Mischaikow1995}, in which the same choice is made. The choice of a  distance function induced by a Riemannian metric is appropriate for our purposes, since hyperbolic equilibria are locally exponentially stable with respect to any such distance function.
\end{remark}


\begin{definition}
	A nonautonomous semiflow $\Phi$ is \textit{asymptotically autonomous with limit semiflow $\Theta$} if for any sequences 
	${t_j \rightarrow t}$,  ${s_j \rightarrow \infty}$, and ${x_j \rightarrow x}$, 
	\begin{align}		
		\Phi^{(t_j + s_j,s_j)}_{y_0} (x_j)  \to \Theta^t(x) \textrm{ as } j \rightarrow \infty,
%		\lim_{j \rightarrow \infty}
%\Phi^{(t_j + s_j,s_j)}_{y_0} (x_j)  = \Theta^t(x),
	\end{align}
	where $\Theta^t$ is an autonomous semiflow.
\end{definition}

\begin{definition}
	An equilibrium is \textit{almost globally asymptotically stable} if it is Lyapunov stable and its basin of attraction is full measure and residual, i.e. its complement is measure zero and meager (a countable union of nowhere dense sets).
%	 , i.e. the complement of a meager set of measure zero 
%		set which is 
\end{definition}


%	In this paper, the chain recurrent set is defined with respect to $(\varepsilon, T)$-chains using the given metric, rather than $(\mathcal{U},T)$-chains as in Conley's monograph \cite{Conley1978}, and where $\varepsilon$ is a constant .
%If the metric on $M$ is replaced with a new metric inducing the same topology, the conclusions of Theorem~\ref{th:R-subset-E} and Corollary~\ref{co:E-isolated-components} imply that the new chain recurrent set is still contained in $S$, and the conclusion of Corollary~\ref{co:equilibria} implies that the new chain recurrent set coincides with the old, i.e. $\mathcal{E}(\Phi)$.

%forward trajectory from $x_0$ is the set
%$\left\{\Phi^t(x_0) : t \in \mathbb{R}_{\geq 0}\right\}$

%introduce chain recurrent set

%\subsection{Almost Global Sets}
%
%define almost global in terms full measure and open dense (?), and almost empty as measure zero and countable union of nowhere dense (?)
%
%``almost every'' means ``for points belonging to an almost global set''. ``almost no'' means ``for points belonging to an almost empty set.

\subsection{Main Result}

%, subject to conditions on the hyperbolicity of equilibria, the 
%, and the structure of the chain recurrent set of the unforced outer loop.

\iffalse
The following theorem relies on two premises which may be less familiar to some readers. The first is an assumption on the structure of the chain recurrent set, which may seem quite abstract. In Sec. III, we give a practical sufficient condition for verifying that the chain recurrent set has the required form, and verify this property for two important classes of systems. Secondly, we assume that all forward trajectories starting in a certain set have compact closure. Since any subset of Euclidean space is compact if and only if it is closed and bounded, in the Euclidean setting this amounts to boundedness of forward trajectories. Informally, such an assumption prevents the finite time escape of any trajectory before the inner loop has the chance to converge. We give a practical sufficient condition to certify this property in Sec. IV.
\fi
%and the structure of the chain recurrent set, . We state the theorem in these qualitative, abstract terms, but note that Secs. III and IV explore these properties in detail and give practical sufficient conditions to verify each of these vital properties.

\begin{theorem}[Almost Global Asymptotic Stability of Cascade]
	\label{almost_global_stability}
%	For some manifolds and metric spaces $X$ and $Y$, 
	Consider the cascade on $X \times Y$ given by
	\begin{subequations}
		\begin{align}
			\dot{x} &= f(x,y), \label{outer_theorem} \\
			\dot{y} &= g(y). \label{inner_theorem} 
		\end{align}
	\end{subequations}
	%	where ${f(0_X,0_Y) = 0}$ and ${g(0_Y) = 0}$ for some points ${0_X \in M}$ and ${0_Y \in N}$.
	%	 and all trajectories are bounded.
	Suppose the following conditions hold:
%	\begin{enumerate}
\setlist[enumerate,1]{leftmargin=.5cm}
\begin{enumerate}[{1.}] %[{$C_1$.}]
		\item 
%		The point 
		${0_Y \in Y}$ is a hyperbolic almost globally asymptotically stable equilibrium of \eqref{inner_theorem}, with basin of attraction $\mathcal{B}_Y$.
		\item 
%		The point 
		${0_X \in X}$ is an almost globally asymptotically stable equilibrium of the dynamics 
		\begin{equation}
			{\dot{x} = f(x,0_Y)},
			\label{unforced}
		\end{equation}
		and all chain recurrent points of \eqref{unforced} are hyperbolic equilibria.
%		and the chain recurrent set of \eqref{unforced} consists solely of hyperbolic equilibria.
		\item 
		For any ${x_0 \in X}$ and ${y_0 \in \mathcal{B}_Y}$, the forward trajectory of \eqref{outer_theorem}-\eqref{inner_theorem} starting at $(x_0,y_0)$ is precompact.
%		All trajectories beginning 
%		in the set $X \times \mathcal{B}_Y$ are bounded in forward time, where ${\mathcal{B}_Y \subseteq N}$ is the basin of attraction of $0_Y$.
%		, where ${\mathcal{B}_Y \subseteq N}$ denotes the basin of attraction of $0_Y$ with respect to \eqref{inner_theorem}. 
	\end{enumerate}
	Then, $(0_X,0_Y)$ is almost globally asymptotically stable and locally exponentially stable for the cascade  \eqref{outer_theorem}-\eqref{inner_theorem}.
\end{theorem}

%\begin{theorem}[Cascade of Two Nice Systems]
%	For compact manifolds without boundary $M$ and $N$, consider the cascade system on $X \times Y$ given by
%\begin{subequations}
%	\begin{align}
	%		\dot{x} &= f(x) + h(x,y) \label{outer_theorem} \\
	%		\dot{y} &= g(y) \label{inner_theorem} 
	%	\end{align}
%\end{subequations}	and suppose that 
%${\dot{x} = f(x)}$ is nice with equilibrium ${0_X \in M}$, while
%${\dot{y} = g(y)}$ is nice with equilibrium ${0_Y \in N}$. Moreover, suppose that 
%\begin{equation}
%	{g(y) = 0 \implies h(x,y) = 0}.
%	\label{vanishing_at_inner_eql}
%\end{equation}
%i.e. the interconnection term vanishes at every equilibrium of the inner subsystem. %$h(x,y) = 0$ for \textrm{all} $y \in \{y \in N : g(y) = 0\}$. 
%Then, the cascade system \eqref{outer_theorem}-\eqref{inner_theorem} is nice with equilibrium $(0_X,0_Y)$.
%\label{simple_cascade}
%\end{theorem}


%\begin{conjecture}
%	Suppose that all trajectories of the cascade 
%	\begin{align}
	%		\dot{x} &= f(x,y) \label{outer_theorem} \\
	%		\dot{y} &= g(y) \label{inner_theorem} 
	%	\end{align}
%	are bounded. Furthermore, suppose that ${0_Y \in Y}$ and ${0_X \in X}$ are almost-globally asymptotically stable equilibria of the systems ${\dot{y} = g(y)}$ and ${\dot{x} = f(x,0_Y)}$ respectively, and that these two systems are gradient-like. Then, the cascade \eqref{outer_theorem}-\eqref{inner_theorem} is gradient-like, and $(0_X,0_Y)$  is an almost-globally asymptotically stable equilibrium.
%	%	 with respective regions of attraction $\mathcal{A}_X \subseteq X$ and $\mathcal{A}_Y \subseteq Y$.  
%	%	 then there exists a measure-zero set $U \subseteq X \times Y$ such that all solutions of the cascade \eqref{outer}-\eqref{inner} starting in $(X \times Y) \setminus U$ are either unbounded or converge to $(0_X,0_Y)$.
%\end{conjecture}
%
%\begin{theorem}
%	Suppose that all trajectories of the cascade 
%	\begin{align}
	%		\dot{x} &= f(x,y) \label{outer_theorem} \\
	%		\dot{y} &= g(y) \label{inner_theorem} 
	%	\end{align}
%	are bounded. Suppose that ${0_Y \in Y}$ and ${0_X \in X}$ are almost-globally asymptotically stable equilibria of the decoupled systems ${\dot{y} = g(y)}$ and ${\dot{x} = f(x,0_Y)}$ respectively, and moreover that the respective chain recurrent sets of the decoupled systems consist solely of hyperbolic equilibria.
%%	and suppose all respective equilibria of these systems are hyperbolic. Finally, suppose that every chain recurrent point of $\dot{x} = f(x,0_Y)$ is an isolated equilibrium. 
%	Then, $(0_X,0_Y)$ 
%	is an almost-globally asymptotically stable equilibrium of the cascade \eqref{outer_theorem}-\eqref{inner_theorem}. 
%	
%%	WTS: Furthermore, the chain recurrent set of the cascaded system consists solely of hyperbolic equilibria.
%%	 with respective regions of attraction $\mathcal{A}_X \subseteq X$ and $\mathcal{A}_Y \subseteq Y$.  
%%	 then there exists a measure-zero set $U \subseteq X \times Y$ such that all solutions of the cascade \eqref{outer}-\eqref{inner} starting in $(X \times Y) \setminus U$ are either unbounded or converge to $(0_X,0_Y)$.
%\end{theorem}
\begin{proof}
	%	Denoting by ${X : (x,y) \mapsto \big(f(x,y),g(y)\big)}$
	%	%	X \times Y \rightarrow TX \times TY
	%	the combined vector field on $X \times Y$ describing the full cascade, we may express the linearization of the cascade at the equilibrium ${(0_X,0_Y)}$ in block matrix form as
	%%	\begin{equation}
		%%		\left.d X \right|_{(0_X,0_Y)} = \begin{bmatrix}
			%%			\left. d_x f \,  \right|_{x^\star} 
			%%			%	 		+ \left. d_x h \,  \right|_{(x^\star,y^\star)} 
			%%			&
			%%			\left. d_y h \,  \right|_{(x^\star,y^\star)} \\
			%%			0 &
			%%			\left.d_y g \, \right|_{y^\star\hphantom{x^\star,()}} \\
			%%		\end{bmatrix}.
		%%	\end{equation}
	%	 \begin{equation}
		%			\left.d X \right|_{(0_X,0_Y)} = \begin{bmatrix}
			%					\left. \partial_x f \,  \right|_{(0_X,0_Y)} &
			%					\left. \partial_y f \,  \right|_{(0_X,0_Y)} \\
			%					0 &
			%					\left.\partial_y g \, \right|_{\, 0_Y\hphantom{0_X,()}} \\
			%				\end{bmatrix}{.}
		%		\end{equation}
	%%		 Moreover, since $f(0_X,0_Y) = 0$ we have
	%		In view of the fact that the eigenvalues of an upper triangular block matrix depend only on the eigenvalues diagonal blocks, the hyperbolicity and stability assumptions on the inner loop dynamics \eqref{inner_theorem} and limiting dynamics \eqref{unforced} immediately imply that $(0_X,0_Y)$ is hyperbolic and therefore locally exponentially stable. Hence, to complete the proof it will suffice to show that $(0_X,0_Y)$ is almost globally attractive. 
	
	Since ${X \times \mathcal{B}_Y}$ is invariant for \eqref{outer_theorem}-\eqref{inner_theorem} and all forward trajectories beginning in ${X \times \mathcal{B}_Y}$ have compact closure, the cascade induces an autonomous semiflow
	\begin{equation}
		\Psi^t : X \times \mathcal{B}_Y \rightarrow X \times \mathcal{B}_Y.
	\end{equation}
	Similarly, \eqref{unforced} and \eqref{inner_theorem} induce the autonomous semiflows 
	\begin{subequations}
			\begin{align}
			\Theta^t : X \rightarrow X, \ & x_0 \mapsto \mathrm{pr}_1 \circ \Psi^t(x_0,0_Y), \label{limit_semiflow} \\
			\Upsilon^t : \mathcal{B}_Y \rightarrow \mathcal{B}_Y, \ & y_0 \mapsto \mathrm{pr}_2 \circ \Psi^t(0_X,y_0), \label{inner_semiflow}
		\end{align}
	\end{subequations}	
%	${\Phi^t : X \times Y \rightarrow X \times Y}$, ${\phi^t : M \rightarrow M}$, and ${\theta^t : N \rightarrow N}$ respectively.
%	, where clearly ${\textrm{pr}_2 \circ \Phi^t = \theta^t}$. 
%	, and moreover \eqref{inner_theorem} induces the autonomous semiflow $\varphi^t : N \rightarrow N, y \mapsto \textrm{pr}_1 \circ \Phi^t(x,y)$. 
%	
%	
%	the dynamics \eqref{outer_theorem}-\eqref{inner_theorem}, \eqref{unforced}, and \eqref{inner_theorem} each induce a well-defined flow, denoted respectively by ${\Phi^t : X \times Y \rightarrow X \times Y}$, ${\phi^t : M \rightarrow M}$, and ${\varphi^t : N \rightarrow N}$.
	% Note that we have deliberately chosen a restricted domain for $\Phi^t$ for the purposes of our argument.
	where $\mathrm{pr}_1$ and $\mathrm{pr}_2$ are the natural projections onto $X$ and $Y$, and
	we have carefully chosen the domains of the semiflows.
%	for notational convenience.
	We observe that for each initial condition ${y_0 \in \mathcal{B}_Y}$,
%	, the solution of \eqref{inner_theorem} satisfying $y(0) = y_0$ is given by ${y(t) = \Upsilon^t(y_0)}$, for each point $y_0$ 
	\eqref{outer_theorem} may be interpreted as time-varying dynamics on $X$ given by 
	\begin{equation}
		\dot{x} = 
%		f_{y_0}(t,x) \overset{\textrm{def}}{=}
		 f\big(x,\Upsilon^t(y_0)\big).
	\end{equation}
	In this manner, each initial condition ${y_0 \in \mathcal{B}_Y}$ induces a corresponding \textit{nonautonomous} semiflow on $X$ given by
	\begin{equation}
		\Phi^{(t,s)}_{y_0} : X \rightarrow X, \ x_0 \mapsto \textrm{pr}_1 \circ \Psi^{t-s} \big(x_0,\Upsilon^s(y_0)\big),
		\label{nonautonomous_semiflow}
	\end{equation}
	such that we may also conclude
	\begin{equation}
		\Psi^t(x_0,y_0) = \big(\Phi^{(t,0)}_{y_0}(x_0), \Upsilon^t(y_0) \big).
		\label{flow_decomposition}
	\end{equation}
With these facts in mind, we prove the claim in five steps.

		
	%		almost global attractiveness of ${(0_X,0_Y)}$ 
	%		 almost all trajectories converge to ${(0_X,0_Y)}$.
	
	%		We prove this claim in three steps.
	
	%	It is immediately clear that the set of equilibria of the cascade is given precisely by
	%	\begin{equation}
		%		\left\{
		%			(x^\star,y^\star) : f(x^\star) = 0 \textrm{ and }g(y^\star) = 0
		%		\right\}.
		%	\end{equation}
	%	Denoting by ${X : (x,y) \mapsto \big(f(x) + h(x,y),g(y)\big)}$
	%%	X \times Y \rightarrow TX \times TY
	%	 the combined vector field describing the full cascade and 
	%	 relying on \eqref{vanishing_at_inner_eql}, we may express the linearization of the cascade at any equilibrium $(x^\star,y^\star)$ in the form 
	%	 \begin{equation}
		%	 		\left.d X \right|_{(x^\star,y^\star)} = \begin{bmatrix}
			%	 		\left. d_x f \,  \right|_{x^\star} 
			%%	 		+ \left. d_x h \,  \right|_{(x^\star,y^\star)} 
			%	 		&
			%	 		\left. d_y h \,  \right|_{(x^\star,y^\star)} \\
			%	 		0 &
			%	 		\left.d_y g \, \right|_{y^\star\hphantom{x^\star,()}} \\
			%	 	\end{bmatrix}.
		%	 \end{equation}
	%% Moreover, since $h(x,0_Y) = 0$ we have
	%% \begin{equation}
		%%	\left.d X \right|_{(0_X,0_Y)} = \begin{bmatrix}
			%%		\left. d_x f \,  \right|_{0_X} &
			%%		\left. d_y h \,  \right|_{(0_X,0_Y)} \\
			%%		0 &
			%%		\left.d_y g \, \right|_{0_Y\hphantom{0_X,()}} \\
			%%	\end{bmatrix}.
		%%\end{equation}
		%	In view of the fact that the eigenvalues of an upper triangular block matrix depend only on the eigenvalues diagonal blocks, we may conclude from the hyperbolicity and stability assumptions that all equilibria of the cascade are hyperbolic, and moreover $(0_X,0_Y)$ is locally asymptotically stable. 
		%		
		%	
		%	Hence, for almost-global asymptotic stability of $(0_X,0_Y)$, it will suffice to also show that $(0_X,0_Y)$ is almost-globally attractive, i.e. almost all initial conditions converge to $(0_X,0_Y)$. We prove this claim in three steps.
		%	
		
		\begin{step}
			${\mathcal{E}(\Psi)
%				 \cap (X \times \mathcal{B}_Y) 
				 = 
				\mathcal{R}(\Theta) \times \{0_Y\}}$,
			%All 
			%%equilibria of \eqref{outer_theorem}-\eqref{inner_theorem} in $X \times \mathcal{B}_Y$ 
			%equilibria of $\Phi^t$ contained in ${X \times \mathcal{B}_Y}$
			%are hyperbolic. 
%			Moreover, 
			and all points in this set are hyperbolic equilibria, of which only ${(0_X,0_Y)}$ is stable.
			%For all ${x \in M}$ such that ${f(x,0_Y) = 0}$, 	the point ${(x,0_Y)}$ is a hyperbolic equilibrium of \eqref{outer_theorem}-\eqref{inner_theorem}.
		\end{step}
		\begin{proof}
		\renewcommand{\qedsymbol}{$\blacktriangledown$}
			All equilibria  ${(x,y) \in {X \times \mathcal{B}_Y}}$ must have ${y = 0_Y}$ by the definition of $\mathcal{B}_Y$ as a basin of attraction, and therefore it must also hold that ${f(x,0_Y) = 0}$ i.e. $x$ must be an equilibrium of \eqref{unforced}. The equality then follows from the assumption
%			s on the chain recurrent set of \eqref{unforced}, i.e.
			that ${\mathcal{R}(\Theta) \subseteq \mathcal{E}(\Theta)}$, since equilibria are always chain recurrent i.e. ${ \mathcal{E}(\Theta) \subseteq  \mathcal{R}(\Theta)}$.
			Denoting 
			%	X \times Y \rightarrow TX \times TY
			the vector field on ${X \times Y}$ describing the full cascade \eqref{outer_theorem}-\eqref{inner_theorem} by ${F : (x,y) \mapsto \big(f(x,y),g(y)\big)}$, we may express its linearization at any equilibrium ${(x,0_Y) \in X \times \mathcal{B}_Y}$ 
%			in block matrix form 
			as
			%		\begin{equation}
				%				\left.d X \right|_{(x^\star,y^\star)} = \begin{bmatrix}
					%						\left. d_x f \,  \right|_{(x^\star,y^\star)}
					%						&
					%						\left. d_y f \,  \right|_{(x^\star,y^\star)} \\
					%						0 &
					%						\left.d_y g \, \right|_{y^\star\hphantom{x^\star,()}} \\
					%					\end{bmatrix}.
				%			\end{equation}
			\begin{equation}
				\left.d F \right|_{(x,0_Y)} = \begin{bmatrix}
					\left. \partial_x f \,  \right|_{(x,0_Y)} &
					\left. \partial_y f \,  \right|_{(x,0_Y)} \\
					0 &
					\left.\partial_y g \, \right|_{\, 0_Y\hphantom{x,()}} \\
				\end{bmatrix}{.}
			\end{equation}
			Since the eigenvalues of a triangular block matrix are simply the eigenvalues of the blocks on the diagonal, the 
			%	hyperbolicity of $(x,0_Y)$
			claim of hyperbolicity follows directly from
			the	hyperbolicity of $0_Y$ for \eqref{inner_theorem} and the hyperbolicity of all equilibria of \eqref{unforced}. Furthermore, an almost globally asymptotically stable system has exactly one stable equilibrium, so at ${x = 0_X}$ all eigenvalues of the top left block have negative real part, but at least one eigenvalue has positive real part at all other equilibria of \eqref{unforced}. Therefore $(0_X,0_Y)$ is locally exponentially stable, while all other equilibria in $X \times \mathcal{B}_Y$ are unstable.
			%immediately imply that is hyperbolic.
			%	 and therefore locally exponentially stable. Hence, to complete the proof it will suffice to show that $(0_X,0_Y)$ is almost globally attractive. 
		\end{proof}
		\noindent To complete the proof, it will therefore suffice to show that the stable equilibrium $(0_X,0_Y)$ is almost globally attractive.
%		, which we will accomplish from the viewpoint of asymptotically autonomous semiflows.
		
%		describing the trajectories of \eqref{outer_theorem} such that 
%		 arising from the time varying vector field induced by considering the solution $y(t)$ of \eqref{inner_theorem} satisfying $y(0) = y_0$. 
		
		
		%\begin{step}
		%$(0_X,0_Y)$ is locally exponentially stable.
		%\end{step}
		%\begin{proof}
		%	\renewcommand{\qedsymbol}{\rotatebox[origin=c]{180}{$\triangle$}}
		%Also, $(0_X,0_Y)$ is hyperbolic by Step 1, and moreover it is locally exponentially stable due to the stability assumptions on \eqref{inner_theorem} and \eqref{unforced}.
		%\end{proof}
%		\begin{step}
%			$\mathcal{R}(\Phi^t) \cap (X \times \mathcal{B}_Y) = \mathcal{R}(\phi^t) \times \left\{0_Y\right\}$.
%			%	$\mathcal{R}(\Phi^t) = \mathcal{R}(\phi^t) \times \left\{0_Y\right\} = \mathcal{E}(\Phi^t)$.
%			%	The chain recurrent set 
%			%	All chain recurrent points of $\Phi^t$ are of the form $(x,0_Y)$.
%			%	For all ${(x,y) \in \mathcal{R}(\Phi^t)}$, we have ${y = 0_Y}$. 
%			%	consists of all $\omega$-limit points of  $\Phi^t$.
%		\end{step}
%		\begin{proof}
%			\renewcommand{\qedsymbol}{\rotatebox[origin=c]{180}{$\triangle$}}
%			%		Every $\omega$-limit point is chain recurrent, so it suffices to show that any point not contained in an $\omega$-limit set is not chain recurrent.
%			We first show that all points ${(x,y) \in X \times \mathcal{B}_Y}$ with $y \neq 0_Y$ are not chain recurrent with respect to $\Phi^t$. It is clear that ${\mathcal{R}(\varphi^t) \cap \mathcal{B}_Y = \{0_Y\}}$, thus
%			\begin{equation}
%				\varphi^t\big(
%				y
%				\big) = 
%			\end{equation}
%			
%			Can we argue that $y \neq 0_Y$ is not chain recurrent for \eqref{inner_theorem} only since it is in the basin of attraction of $0_Y$, and therefore using any product metric the triangle inequality determines that it is not chain recurrent for the full cascade.
%			
%			By some converse Lyapunov theorem, there exists a Lyapunov function ${V : \mathcal{B}_Y \rightarrow \mathbb{R}}$ for \eqref{inner_theorem} such that 
%			\begin{align}
%				V = 0 & \iff y = 0_Y
%				\\
%				\dot{V} < 0 & \iff y \in \mathcal{B}_Y \setminus \left\{0_Y\right\}.
%			\end{align} 
%			Considering any initial point ${(x_0,y_0) \in M \times (\mathcal{B}_Y \setminus \left\{0_Y\right\})}$,
%			%		Defining ${W : (x,y) \mapsto V(y)}$,
%			we can show (TODO: show!) via continuity and compactness that for sufficiently large $T$ and sufficiently small $\epsilon$, 
%			\begin{multline}
%				\Phi^T\Big(
%				M \times V^{-1}\big(\left[0,V(y_0)\right]\big)
%				%			\big\{(x,y) : V(y) \leq V(y_0)\big\}
%				\Big) \subset
%				\\ \left\{(x,y) : d_{M\times N}\big((x_0,y_0),(x,y)) > \epsilon \right\}
%			\end{multline}
%			and therefore ${(x,y)}$ is not chain recurrent if ${y \neq 0_Y}$.
%			%		, while on the contrary all equilibria  On the contrary, all equilibria are chain recurrent
%			%		Thus, the chain recurrent set and the $\omega$-limit set of the flow on ${X \times \mathcal{B}_Y}$  are identical.
%			%		Since all trajectories are bounded, each initial condition converges to its $\omega$-limit set
%			%\end{proof}
%			%
%			%
%			%\begin{step}
%			%	%	The chain recurrent set 
%			%	%	All chain recurrent points of $\Phi^t$ are of the form $(x,0_Y)$.
%			%	For all ${(x,y) \in \mathcal{R}(\Phi^t)}$, we have ${f(x,0_Y) = 0}$. 
%			%	%	consists of all $\omega$-limit points of  $\Phi^t$.
%			%\end{step}
%			%\begin{proof}
%			%	\renewcommand{\qedsymbol}{\rotatebox[origin=c]{180}{$\triangle$}}
%			Furthermore, since $X \times Y$ is compact, by the restriction property all chain recurrent points are chain recurrent via {$(\epsilon,T)$-chains} contained in $\mathcal{R}(\Phi^t)$, that is
%			\begin{align}
%				\mathcal{R}(\Phi^t) 
%				&= \mathcal{R}\big(\Phi^t|_{\mathcal{R}(\Phi^t) } \big).
%			\end{align}
%			Having  just shown ${\mathcal{R}(\Phi^t) \subseteq M \times \{0_Y\}}$, we therefore obtain
%			\begin{align}
%				\mathcal{R}(\Phi^t) 
%				%		&= \mathcal{R}(\Phi^t|_{\mathcal{R}(\Phi^t) } ) \\
%				&= \mathcal{R}(\Phi^t|_{M \times \{0_Y\}} ) \\
%				&= \mathcal{R}(\phi^t) \times \{0_Y\} 
%			\end{align}
%			where the last equality follows because \eqref{outer_theorem} is independent of $y$ on the invariant set ${M \times \{0_Y\}}$. Finally, all equilibria are chain recurrent, and by assumption all points in $\mathcal{R}(\phi^t)$ are equilibria for \eqref{unforced}, yielding the final equality to be shown.
%			%	 and $M \times 0_Y$ is an equilibrium of \eqref{inner_theorem}, while 
%		\end{proof}
%		%\begin{step}
%		%%	The chain recurrent set of the flow on $X \times \mathcal{B}_Y$ is contained in the set $\left\{(x,y) \in X \times \mathcal{B}_Y : y = 0_Y\right\}$.
%		%	All $\omega$-limit points of $\Phi^t$ take the form $(x,0_Y)$.
%		%\end{step}
%		%\begin{proof}
%		%	\renewcommand{\qedsymbol}{\rotatebox[origin=c]{180}{$\triangle$}}
%		%	It is clear that the $\omega$-limit set of \eqref{inner_theorem} over $\mathcal{B}_Y$ is exactly $\left\{0_Y\right\}$, and since \eqref{inner_theorem} is independent of the evolution of the other subsystem, the $\omega$-limit set of the cascade over $X \times \mathcal{B}_Y$ must also have ${y = 0_Y}$, so the claim follows from the correspondence in the previous step.
%		%\end{proof}
%		

	\begin{step}
	For any ${y_0 \in \mathcal{B}_Y}$, the nonautonomous semiflow $\Phi
%	^{(t,s)}
	_{y_0}$ is asymptotically autonomous with limit semiflow $\Theta$. 
\end{step}
\begin{proof}
		\renewcommand{\qedsymbol}{$\blacktriangledown$}
%	From the asymptotic stability of $\eqref{inner_theorem}$, we also have that for all ${y_0 \in \mathcal{B}_Y}$,
%	\begin{equation}
%		\lim_{j\rightarrow \infty} \Upsilon^{s_j}(y_0) = 0_Y 
%		\label{attractivity_sequence}
%	\end{equation}
%for any sequence ${s_j \rightarrow \infty}$.  
%We observe that
% in view of \eqref{nonautonomous_semiflow}, 
For any sequences ${t_j \rightarrow t}$,  ${s_j \rightarrow \infty}$, and ${x_j \rightarrow x}$, 
	\begin{align}
		&\begin{aligned}
					\lim_{j \rightarrow \infty} &
			\Phi^{(t_j + s_j,s_j)}_{y_0} (x_j) \\
			&= \lim_{j \rightarrow \infty} \textrm{pr}_1 \circ \Psi^{t_j} \big(x_j,\Upsilon^{s_j}(y_0)\big)
			\label{defn_nonautonomous}
		\end{aligned}
		\\
		&		\phantom{\lim_{j \rightarrow \infty}}
		= \textrm{pr}_1 \circ \Psi^{\lim\limits_{j \rightarrow \infty}  t_j} \left(\lim\limits_{j \rightarrow \infty} x_j,\lim\limits_{j \rightarrow \infty} \Upsilon^{s_j}(y_0)\right) \label{continuity_limit}\\
		&		\phantom{\lim_{j \rightarrow \infty}}
		= 
		\label{solved_limits} 
		\textrm{pr}_1 \circ \Psi^{t} (x,0_Y)
		=
		 %		\\&=
%		\Phi^{(t_j,s_j)}_{0_Y}  (x_j) = 
		\Theta^{t} (x),
%		\label{limit_solution}
	\end{align}
where \eqref{defn_nonautonomous} follows immediately from \eqref{nonautonomous_semiflow}, \eqref{continuity_limit} is obtained by the continuity of $\mathrm{pr}_1$ and $\Psi$, and \eqref{solved_limits} relies on the attractivity of $0_Y$ for $\eqref{inner_theorem}$.
%, and \eqref{limit_solution} follows from \eqref{limit_semiflow}.
Thus for any ${y_0 \in \mathcal{B}_Y}$, by definition 
$\Phi_{y_0}$ is asymptotically autonomous with limit semiflow $\Theta$.
%	In view of \eqref{limit_semiflow} and \eqref{nonautonomous_semiflow} and the fact that ${0_Y \in \mathrm{Fix}(\Upsilon)}$, 
%it is clear that ${\Phi^{(t,s)}_{0_Y} = \Theta^{t-s}}$. 
%	\textbf{How to make this rigorous?}
%	, since $0_Y$ is an equilibrium of \eqref{inner_theorem} and therefore ${\Upsilon^s(y_0) = y_0}$ for all $s$. 
%	\textbf{To Do:} show that the appropriate notion of convergence holds, i.e. $$\Phi^{(t,s)}_{y} \rightarrow \Phi^{(t,s)}_{0_Y} \textrm{ as } y \rightarrow 0_Y.$$ \textbf{How can we do this?} Mischaikow claims that on $\mathbb{R}^n$, a nonautonomous semiflow is asymptotically autonomous if the vector field converges to a time invariant one. Can we extend this
%	to the non-Euclidean setting?
\end{proof}
		
		\begin{step}
			Every trajectory of $\Psi$ 
%			beginning in ${X \times \mathcal{B}_Y}$ 
			converges to a hyperbolic equilibrium.
			%		satisfying ${f(x,0_Y) = 0}$.
			%		 point in $\mathcal{R}(\Phi^t)$.
			
			%		the set $$E = \big\{(x,0_Y) \in X \times Y : f(x,0_Y) = 0\big\}$$
			%		 \subseteq X \times \mathcal{B}_Y . 
			%		i.e. for almost all initial conditions, $y$ converges to the equilibrium $0_Y$ and $x$ converges to some equilibrium of the system $\dot{x} = f(x,0_X)$.
		\end{step}
		\begin{proof}
		\renewcommand{\qedsymbol}{$\blacktriangledown$}
			%			Any $x \in \mathcal{R}(\phi^t)$ is a hyperbolic equilibrium of \eqref{unforced} by assumption. Thus by Step 1 and the same eigenvalue argument as above, all points $(x,y) \in \mathcal{R}(\Phi^t)$ are hyperbolic equilibria of \eqref{outer_theorem}-\eqref{inner_theorem}
			%			By Step 2, 
			Every precompact forward trajectory of an asymptotically autonomous semiflow converges to the chain recurrent set of its limit semiflow \cite{Mischaikow1995}.  
			Thus, Step 2 implies that for each ${y_0 \in \mathcal{B}_Y}$, every trajectory of $\Phi_{y_0}$ converges to $\mathcal{R}(\Theta)$, and asymptotic stability ensures that every trajectory of $\Upsilon$ 
%			beginning in $\mathcal{B}_Y$ 
			converges to $0_Y$. Thus, in view of \eqref{flow_decomposition} it is clear that every trajectory of $\Psi$ 
%			starting in ${X \times \mathcal{B}_Y}$ 
			converges to ${\mathcal{R}(\Theta) \times \{0_Y\}}$, and all points in this set are hyperbolic equilibria by Step 1. Since hyperbolic equilibria are isolated, 
%			a continuity argument implies convergence to a particular hyperbolic equilibrium.
%			
%			Furthermore, ${\mathcal{R}(\phi^t)}$ consists solely of points $x$ such that $(x,0_Y)$ is a hyperbolic equilibrium. 
%			
%			Since $X \times Y$ is compact, each trajectory converges to its $\omega$-limit set, and all $\omega$-limit points are chain recurrent. Moreover, ${X \times \mathcal{B}_Y}$ is invariant by construction, thus all trajectories beginning in ${X \times \mathcal{B}_Y}$ converge to the set ${\mathcal{R}(\Phi^t) \cap (X \times \mathcal{B}_Y)}$. By Step 2, this set is exactly ${\mathcal{R}(\phi^t) \times \{0_Y\}}$, which by Step 1 consists solely of hyperbolic (and therefore isolated) equilibria.
%
%			Thus,
			 continuity implies that every trajectory
%			 beginning in ${M\times \mathcal{B}_Y}$ 
			 converges to a particular hyperbolic equilibrium.
			%%			$(x,y) \in \mathcal{R}(\Phi^t)$, 
			%%			
			%%			 is hyperbolic and therefore isolated for all $\mathcal{R}(\Phi^t)$ 
			%%						
			%%			Moreover, $y$ converges to $0_Y$ from every initial condition in $\mathcal{B}_Y$ by definition, and therefore the chain recurrent set is entirely contained in the set $M \times \left\{0_Y\right\}$. 
			%%			Furthermore, by the restriction property of the chain recurrent set, each point in the chain recurrent set is chain recurrent via an $(\epsilon,T)$-chain consisting of only points in the chain recurrent set itself.
			%%			
			%%			we have 
			%%			$$
			%%			\mathcal{R}(X |_{X \times \mathcal{B}_Y}) =
			%%			\mathcal{R}(X |_{\mathcal{R}(X |_{X \times \mathcal{B}_Y})})
			%%			$$
			%%			
			%%			
			%%			Thus, denoting by $\mathcal{R}(X)$ the chain recurrent set of the flow on $X \times \mathcal{B}_Y$, every initial condition converges to the set 
			%%			$$
			%%			\big\{(x,y) \in \mathcal{R}(X) : y = 0_Y \big\}.
			%%			$$
			%
			%%			We now argue that $\mathcal{R(X)|_{\mathcal{X \times \mathcal{B}_Y}}} = \mathcal{\omega(X \times \mathcal{B}_Y})$.
			%%			Furthermore, 
			%%			
			%%			
			%%			Furthermore, the restriction property states that the chain recurrent set is internally chain recurrent, i.e. all points in $\mathcal{R}(X)$ the chain recurrent set of the flow is the same when restricted to its chain recurrent set. 
			%%Moreover, almost every trajectory converges to
			%%			$$
			%%\big\{(x,y) \in \mathcal{R}(X |_{\mathcal{R}(X)}) : y = 0_Y \big\},
			%%$$
			%%and since $\mathcal{R}(X)$ consists of hyperbolic (therefore isolated) equilibria, 
			%%
			%%			and by Conley's argument (citation/explanation?), this is identical to the set 
			%%Therefore,
			%%
			%%\begin{align}
			%%	\mathcal{R}(X) 
			%%	&= \mathcal{R}\big(X(\cdot,0_Y) \big) \\
			%%	&= \mathcal{R}\big(f(\cdot,0_Y) \big) \times \left\{0_Y\right\}
			%%\end{align}
			%%and by assumption, $\mathcal{R}\big(f(\cdot,0_Y)\big)$ consists solely of hyperbolic (hence isolated) equilibria. Therefore, a continuity argument implies that almost every initial condition converges to some particular point in the set $E$.
		\end{proof}
		\begin{step}
	Almost no trajectories of \eqref{outer_theorem}-\eqref{inner_theorem} converge to an unstable equilibrium.
		\end{step}
		\begin{proof}
		\renewcommand{\qedsymbol}{$\blacktriangledown$}
			%	All trajectories starting in ${X \times \mathcal{B}_Y}$ converge to a hyperbolic equilibrium by Step 2, 
%			By definition, all points converging to a hyperbolic equilibrium lie on its global stable manifold, and by the stable manifold theorem, the global stable manifold of an unstable hyperbolic equilibrium is 
			By the global stable manifold theorem, all points converging to an unstable hyperbolic equilibrium lie in the union of countably many embedded submanifolds of positive codimension, which is thus a meager set of measure zero. Moreover, all unstable equilibria in ${X \times \mathcal{B}_Y}$ are hyperbolic by Step 1, and there can be only countably many of these equilibria due to the isolation of hyperbolic equilibria and the second countability of the topology of ${X \times B_Y}$. Therefore, the set of all points in ${X \times \mathcal{B}_Y}$ converging to an unstable equilibrium is a countable union of meager sets of measure zero and is thus also meager and measure zero in $X \times Y$. 
			%	Need to clarify that all the equilibria we're talking about here are hyperbolic. 
			%	Thus, its complement is full measure, and therefore almost every point does \textit{not} converge to an unstable equilibrium. 
		\end{proof}
		\begin{step}
			Almost every trajectory of \eqref{outer_theorem}-\eqref{inner_theorem} converges to the stable equilibrium ${(0_X,0_Y)}$.
		\end{step}
		\begin{proof}
			%	\renewcommand{\qedsymbol}{\rotatebox[origin=c]{180}{$\triangle$}}
			Since $\mathcal{B}_Y$ is full measure and residual in $N$ by assumption, ${X \times \mathcal{B}_Y}$ is full measure and residual in $X \times Y$. By Step 3, every initial condition in this set converges to a hyperbolic equilibrium, and by Step 4, the subset converging to an unstable equilibrium is meager and measure zero in $X \times Y$. Since the difference of a residual set of full measure by a meager set of measure zero is residual and full measure, the remainder constitutes a residual set of full measure in $X \times Y$ for which all initial conditions converge to the unique stable equilibrium ${(0_X,0_Y)}$, completing the proof.
%			  by Step 1, this completes the proof.
%			Clearly, ${X \times Y}$ can be partitioned into the sets ${(X \times \mathcal{B}_Y)^C}$, $U$, and $S$, where $U$ and $S$ respectively contain those initial conditions in ${X \times \mathcal{B}_Y}$ converging to a stable or unstable equilibrium. Hence, $X \times Y$ is partitioned by the three sets: $U$, $S$, and $C$, namely the complement of ${M \times \mathcal{N}}$ in ${X \times Y}$. 	
%			Since $\mathcal{B}_Y$ is full measure in $N$ by assumption,
%			%	 the complement of $X \times \mathcal{B}_Y$
%			$C$
%			is  measure zero in ${X \times Y}$. Since $U$ and $C$ are measure zero, and measure zero sets are closed under countable (hence finite) unions, it follows that $S$ is the complement of a measure zero set and thus full measure. Since ${(0_X,0_Y)}$ is the unique stable equilibrium in $S$ by Step 1, it suffices that all initial conditions in $S$ converge to a stable equilibrium. 
			%	
			%	 Furthermore, it is clear that 
			%	
			%	, and furthermore $S$ is full measure in $X \times Y$, since .
			%	
			%	
			%	
			%	all trajectories beginning in $X \times \mathcal{B}_Y$ converge to a hyperbolic equilibrium, and by Step 3, only the trajectories through a measure zero set of initial conditions converge to an unstable hyperbolic equilibrium. Since all equilibria are either stable hyperbolic or unstable hyperbolic, 
			%	
			%	
			%	. Therefore, all trajectories in the full measure set $X \times \mathcal{B}_Y$ not converging to an unstable equilibrium converge to a stable equilibrium. Since subtracting a measure zero set from a set of full measure yields a set of full measure, we have shown that almost all trajectories in $X \times Y$ converge to a stable equilibrium.
			%%	Since equilibria may be either stable or unstable,
			%%	
			%%	 and only a measure zero set 
			%%	. Moreover, the set of all initial conditions converging to 
			%%	
			%%	Since the 
			%%
			%%	Since measure zero sets are closed under countable (and hence finite) unions, it follows that the intersection of the complement of $X \times \mathcal{B}_Y$ with the set of all trajectories beginning in $X \times \mathcal{B}_Y$ but converging to an unstable equilibrium is measure zero. Furthermore, the previous steps showed that all trajectories 
			%%	
			%%	from Steps 1 and 2 that almost all initial conditions converge to a stable equilibrium.
			%%	
			%	 This proves the claim, since the limiting dynamics \eqref{unforced} have only one stable equilibrium and by the same eigenvalue argument as above, the full cascade can have only one stable equilibrium, namely ${(0_X,0_Y)}$.
			%%	 
			%%	 an almost-globally asymptotically stable system has a unique stable equilibrium. 
		\end{proof}
		\renewcommand{\qedsymbol}{}
		\vspace{-20pt}
	\end{proof}

\begin{remark}
	We emphasize that the main potential pitfall of the unforced outer loop being only \textit{almost} globally asymptotically stable is the possibility that the transient may ``funnel'' a non-negligible (i.e. non-meager, positive measure) set to a point $(x,0_Y)$, where $x$ is an unstable equilibrium of \eqref{unforced}. However, such behavior is precluded by the hyperbolicity of all unstable equilibria of \eqref{unforced}.
%\end{remark}
%We expect that this requirement could be relaxed to require the presence of at least one positive eigenvalue while retaining the property of almost global asymptotic stability for the cascade, at the cost of foregoing local exponential stability. However, as we will see in Sec. V, in the case of control design, it is tractable to design vector fields on most manifolds of interest in applications which verify these assumptions.
%\begin{remark}
	This assumption can be relaxed to the requirement that all unstable equilibria of \eqref{unforced} are isolated and have at least one eigenvalue with positive real part, similar to \cite{Angeli2010}. Then, the argument proceeds similarly, but relies on the center-stable manifold theorem instead of the stable manifold theorem.  Similarly, the hyperbolicity assumption on $0_X$ can be relaxed at the cost of local exponential stability. We present the more succinct but less general result for clarity and brevity. 
	\iffalse
	Subsequently, we will describe some broad classes of systems verifying all of the required properties of the subsystems including hyperbolicity, and hence such requirements need not be particularly onerous in  control design.
	\fi
%	\textbf{Add comment about Morse-Smale vector fields?}
\end{remark}

%\textbf{To do: add remarks here covering the cut section}

\subsection{Gradient-Like Systems}

%In the previous result, the chain recurrent set condition may seem rather abstract; however, the following discussion shows that such a property can be readily verified. In Appendix A, it is shown that the chain recurrent set can be localized to a subset using a function with certain technical properties which decreases along all trajectories outside that subset. For conven

Our main result, Theorem \ref{almost_global_stability}, characterizes the stability of a class of cascades in which the chain recurrent points of the unforced outer loop are all equilibria, a somewhat abstract property. Systems with with this property are often called ``gradient-like''. In Appendix A, we present Theorem \ref{th:R-subset-E}, localizing the chain recurrent set of a dynamical system to a subset of state space, provided there exists a function with certain technical properties which is decreasing along trajectories outside that subset. For convenience, here we restate a corollary of that result which is particularly relevant to our present interests, and direct the reader to Appendix A for the full proof. 
%This fact is related to the analysis in \cite[Cor. 2.4]{Benaim1995}, however the setting of that work differs from our own, since it considers stochastic processes evolving on $\mathbb{R}^n$.

%We are unable to include the proof in the initial submission due to space constraints, but note that very similar facts are discussed in \cite[Cor. 2.4]{Benaim1995} and \cite[Sec. IV]{Angeli2010}.
%, and systems with the desired property are often called ``gradient-like''.



%\begin{theorem}
\begin{restate_corollary}{2}
				\label{thm:equilibria}
		If $\mathcal{E}(\Phi)$ consists of isolated points and there is a proper\footnote{A function ${V : M \rightarrow \mathbb{R}}$ is \textit{proper} if it has compact sublevel sets, which morally generalizes the notion of ``radially unbounded'' functions on $\mathbb{R}^n$.}, continuous function ${V:M\to \R}$ 
		%		such that $V$ is 
		that is
		decreasing\footnote{A function ${f : \mathbb{R} \rightarrow \mathbb{R}}$ is \textit{decreasing} if ${f(t_2) < f(t_1)}$ whenever ${t_1 < t_2 }$. Note that this does \textit{not} imply $\dot{f}(t) < 0$ for all $t$, e.g. $t \mapsto -t^3$.} along nonequilibrium trajectories,
		then ${\mathcal{R}(\Phi) = \mathcal{E}(\Phi)}$.
\end{restate_corollary}
%		\end{theorem}

\begin{remark}
	From Corollary \ref{thm:equilibria}, it is clear that Theorem 1 also holds if its second condition is replaced by the assumption that for the system \eqref{unforced}, all equilibria are hyperbolic and there exists a Lyapunov function around $0_X$ which is decreasing along all non-equilibrium trajectories.\footnote{Some authors \cite{Benaim1995} call this a \textit{strict} Lyapunov function, but the control community tends to reserve this term for Lyapunov functions with strictly negative derivative along non-constant trajectories \cite{Santibanez1997}, a stronger condition.}
\end{remark}

%It happens that two classes of systems which are particularly important in geometric control are gradient-like. 

As a matter of fact, certain systems already prominent in the geometric control literature are gradient-like.

\begin{remark}
	\label{summarize_chain_recurrence_section}
	Two important classes of systems verifying Theorem \ref{thm:equilibria} (and the second condition of Theorem \ref{almost_global_stability}) are as follows. 
	It can be shown that for a Riemannian manifold ${(Q,\kappa)}$,
	a strict Rayleigh dissipation $\nu$, and a proper Morse function ${V : Q \rightarrow \left[0,\infty\right)}$ 
	%	with global minimizer 
	with a unique minimum at 
	${0_Q \in Q}$,
	both the gradient dynamics on $Q$ given by
				\begin{equation}
		\dot{q} = - \mathrm{grad}_\kappa\, V(q) 
		%		- \kappa^\sharp \nu^\flat \dot{q} 
		\label{gradient_descent}
	\end{equation}
	and Euler-Lagrange dynamics on $TQ$ given by\footnote{The maps ${\kappa^\flat, \nu^\flat : TQ \rightarrow T^*Q}$ and ${\kappa^\sharp, \nu^\sharp : T^*Q \rightarrow TQ}$ are the \textit{musical isomorphisms} with respect to the Riemannian metrics $\kappa$ and $\nu$ \cite{BulloAndLewis2004}.}
	\begin{equation}
		\overset{\kappa}{\nabla}_{\dot{q}} \dot{q} = - \mathrm{grad}_\kappa\, V(q) - \kappa^\sharp \circ \nu^\flat (\dot{q}) 
		\label{euler_lagrange}
	\end{equation}
	are almost globally asymptotically stable and locally exponentially stable around ${0_Q \in Q}$ and ${0_{TQ} = (0_Q,0) \in TQ}$ respectively, and moreover all chain recurrent points of both systems are hyperbolic equilibria. An early, influential analysis of the previous stability properties is \cite{Koditschek1989}, while a detailed modern treatment can be found in \cite[Chap. 6]{BulloAndLewis2004}. The chain recurrence claim is immediate by Theorem \ref{thm:equilibria} and the fact that the ``potential'' $V$ and the ``total energy''
	\begin{equation}
		\label{total_energy}
		W : (q,\dot{q}) \mapsto V(q) + \frac{1}{2}\kappa(\dot{q},\dot{q}),
	\end{equation}
	are respectively decreasing along all non-equilibrium trajectories of \eqref{gradient_descent} (by construction) and \eqref{euler_lagrange} \cite[Prop. 4.66]{BulloAndLewis2004}. Proofs of these facts can be found in Appendix B.
\end{remark}
	
%\textbf{Perhaps: include funneling example?}




%\begin{proof}
%	Considering the total energy function given by
%	\begin{equation}
%		\label{total_energy}
%		W : (q,\dot{q}) \mapsto V(q) + \frac{1}{2}\kappa(\dot{q},\dot{q}),
%	\end{equation}
%	we compute
%	\begin{align}
%		\dot{W} 
%		%		&= dE_{(q,\dot{q})}(\dot{q},\ddot{q}) \\ 
%		&= dV(q) \dot{q}  + \kappa(	\overset{\kappa}{\nabla}_{\dot{q}} \dot{q},\dot{q}) \\
%		&= dV(q) \dot{q}  + \kappa(	- \mathrm{grad}_\kappa\, V(q) - \kappa^\sharp \circ \nu^\flat (\dot{q})   \, , \, \dot{q} ) \\
%		%	 &= dV(q) \dot{q}  - \kappa( {grad}_\kappa\, V(q) ,\dot{q}) 
%		%	 - \kappa( \kappa^\sharp \nu^\flat \dot{q}    ,\dot{q}) \\
%		&= dV(q) \dot{q}  - dV(q) \dot{q}  - \nu(\dot{q},\dot{q}) 
%		%		\\ &= 
%		= -\nu(\dot{q},\dot{q}) \leq 0.
%	\end{align}
%		Since $V$ is proper, so is $W$, and since $W$ is nonincreasing, all forward trajectories are precompact and \eqref{euler_lagrange} induces an autonomous semiflow. Moreover, the equilibrium set of \eqref{euler_lagrange} is simply $\sigma_0(\{q \in Q : dV(q) = 0\})$, where $\sigma_0 : Q \rightarrow TQ$ is the zero section. \cite{Koditschek1989}. These equilibria can be seen to be hyperbolic, since $V$ is Morse and $\nu$ is \textbf{... ?} and hyperbolic equilibria are isolated. Moreover, only $0_{TQ}$ is (locally exponentially) stable, since $0_Q$ is the unique minimizer of $V$ \cite{Koditschek1989}.
%		Furthermore, it is clear that $\dot{W}$ is decreasing along all trajectories outside the set ${E = \sigma_0(Q) \subseteq TQ}$. 
%
%		\textbf{Problem:} it almost seems like $W$ is constant on $E$, but this is false - we only have $\dot{W} = 0$ along trajectories at $E$. So, can't apply Corollary \ref{co:E-isolated-components}. Even if we could, that would only show that chain recurrent points are in the image of the zero section. What now? Do we need a LaSalle-like approach in the corollaries themselves? 
%%		 all chain recurrent points have zero velocity. it will suffice to show that $(q,0)$ is not chain recurrent if $dV(q) \neq 0$. how to do this?
%		
%		\textbf{Below, assuming we have proved the claim that all chain recurrent points are equilibria.}
%
%		Since all forward trajectories are precompact, all trajectories converge to the chain recurrent set \cite{Mischaikow1995}. Since hyperbolic equilibria are isolated, all trajectories converge to a particular hyperbolic equilibrium. On the other hand, the global stable manifold theorem asserts that the set of points converging to an unstable equilibrium is measure zero, and thus there exists a set of full measure converging to $0_{TQ}$, proving almost global asymptotic stability.
%%	\begin{equation}
%%		\left\{ (q,\dot{q}) \in TQ : dV(q) = 0\right\}.
%%	\end{equation}
%%	Since $V$ has compact sublevel sets, so does $W$. Moreover, 
%%	
%%	
%%	Since $W$ has compact sublevels and the trajectory from any nonequilibrium initial condition converges to an equilibrium at which $W$ has a smaller value, a straightforward $\epsilon,\delta$ argument implies that nonequilibrium initial conditions are not chain recurrent (TODO: see e.g. the proof of Lemma 1.6 on page 371 of the Robinson book). On the other hand, any equilibrium is chain recurrent. Thus, the chain recurrent set for \eqref{euler_lagrange} is precisely the set of equilibria (which are all isolated by the Morse assumption on $V$, since the Hessian of a Morse function is nondegenerate by definition, while the linearization of the dynamics is given by....).
%%	
%%	To Do: finish this argument (hyperbolicity, finish stability, etc)
%\end{proof}

%In short, the main results of \cite{Koditschek1989} is to characterize the global limit behavior of \eqref{euler_lagrange}, showing that it is almost globally asymptotically stable. We characterize its recurrence behavior, showing that the chain recurrent set consists solely of hyperbolic equilibria. 
%\cite{Koditschek1989} characterized limit behavior of dissipative mechanical systems, we also characterize chain recurrence of the same. 

%\clearpage
%
%With these results in mind, in particular Corollary \ref{co:equilibria}, we now turn our attention to dynamical systems generated by descending the gradient of a ``navigation function'', i.e. a proper Morse function with a unique minimum, as well as a well-known second-order generalization of such a procedure. 
%An early and influential analysis of such systems can be found in the classic work \cite{Koditschek1989}. 
%The subsequent analysis establishes a link between the main stability result of the paper and important techniques for geometric control design, illustrating the practical implications of our results.
%
%%\textbf{Idea: prove the stability in these propositions using the fact that precompact trajectories converge to the chain recurrent set .} Basically, Corollary 2 -> CRS = EQL, EQL hyperbolic and only one stable, thus converges to minimum of $V$. Rely on properness of $V/W$ to show precompactness, and cite Mischaikow.
%
%We also consider generalization to second-order systems, which is the primary concern of \cite{Koditschek1989}, since mechanical systems are in general second-order. The following proposition extends \cite[Theorem 2]{Koditschek1989} to also characterize chain recurrence in dissipative mechanical systems.
%
%In this section, we give sufficient conditions for a chain recurrent set of this form. We then apply this result to demonstrate that this property is enjoyed by two classes of dynamical systems arising on a Riemannian manifold or its tangent bundle. 
%
%
%\clearpage

%In view of Propositions \ref{first_order_dynamics} and \ref{second_order_dynamics}, it is clear that if \eqref{inner_theorem} and \eqref{unforced} take the form of \eqref{gradient_descent} or \eqref{euler_lagrange}, this is sufficient to verify conditions 1 and 2 in the hypotheses of Theorem \ref{almost_global_stability}, so all that remains is to verify the precompactness of forward trajectories.












\section{Precompactness of Forward Trajectories}

To apply Theorem \ref{almost_global_stability} to examples, we require practical conditions certifying precompactness, since many manifolds of interest in geometric control are noncompact, such as the tangent bundle of any manifold. 
In $\mathbb{R}^n$ or any Riemannian manifold given a complete Riemannian metric, a subset is compact if and only if it is closed and bounded, so in those settings precompactness amounts to boundedness. 
Informally, such an assumption prevents the finite time escape of any trajectory before the inner loop has the chance to converge. 
In this section, we give a growth rate criterion suited to our geometric setting, characterizing the ``{interconnection term}'' (Fig. 1, system $\Sigma_h$).
%, which measures the deviation between the unforced outer loop and the forced outer loop. 
The result bears similarities to \cite[Thm. 4.7]{Sepulchre2012} certifying boundedness in systems in $\mathbb{R}^n$.

\iffalse
To apply Theorem \ref{almost_global_stability} to examples of interest, we require practical conditions certifying precompactness. Clearly, such a requirement holds automatically when the state space of the outer loop is compact; however, many manifolds of interest in geometric control, such as Euclidean space or the tangent bundle of any compact or noncompact manifold, are noncompact. 
Since any subset of Euclidean space or a manifold equipped with a complete Riemannian metric is compact if and only if it is closed and bounded, in such settings this amounts to boundedness of forward trajectories. 
Informally, such an assumption prevents the finite time escape of any trajectory before the inner loop has the chance to converge. 
In this section, we give a growth rate criterion suited for our geometric setting, characterizing the ``{interconnection term}'' (Fig. 1, system $\Sigma_h$), which measures the deviation between the unforced outer loop and the forced outer loop. The result bears similarities to \cite[Thm. 4.7]{Sepulchre2012} which certifies boundedness for systems evolving in $\mathbb{R}^n$.
\fi
%
%system in the form \eqref{outer_theorem}-\eqref{inner_theorem}, given by 
%\begin{equation}
%	h : X \times Y \rightarrow TM, \ (x,y) \mapsto f(x,y) - f(x,0_Y).
%\end{equation}
%It is immediately clear that
%\begin{equation}
%	\dot{x} = f(x,0_Y) + h(x,y),
%\end{equation}
%so the interconnection term can be understood as 
%In this section, we characterize the growth rate of the interconnection term in $x$ to provide a certificate of boundedness for the full cascade for all trajectories beginning in the inner loop's basin of attraction.
%For notational convenience, in any metric space $M$ we define the notation
%\begin{equation}
%	\big|\, \cdot \, \big|_M : M \rightarrow \mathbb{R}, \, x \mapsto d(x,0_X),
%\end{equation}
%giving the distance from some constant designated point $0_X$.

%\begin{theorem}[Forward Trajectories of Cascade]
%	\label{boundedness}
%For some manifolds and metric spaces $X$ and $Y$,
%	consider the cascade on $X \times Y$ given by
%	\begin{subequations}
	%		\begin{align}
		%			\dot{x} &= f(x,y), \label{outer_bound} \\
		%			\dot{y} &= g(y). \label{inner_bound} 
		%		\end{align}
	%	\end{subequations}
%Let ${0_Y \in N}$ be a stable hyperbolic equilibrium of \eqref{inner_bound} with basin of attraction $\mathcal{B}_Y$, and let the {\normalfont interconnection term} be given by 
%\begin{equation}
%	h : X \times Y \rightarrow TX, \ (x,y) \mapsto f(x,y) - f(x,0_Y).
%\end{equation}
%Suppose there exists a proper function
%${W : X \rightarrow \mathbb{R}_{\geq 0}}$ with ${W^{-1}(0) = 0_X}$, a function ${\alpha : \mathbb{R}_{\geq 0} \rightarrow \mathbb{R}_{\geq 0}}$ differentiable at zero and satisfying ${\alpha(0) = 0}$,  
%%a function ${\alpha \in \mathcal{K}}$,
%and a constant $a \in \mathbb{R}$,
%such that for all $(x,y)$ in the set 
%\begin{equation}
%	S = \{x \in X : W(x) > a\} \times \mathcal{B}_Y
%\end{equation}
%it holds that
%\begin{align}
%	\mathcal{L}_{f(x,0_Y)} W &\leq 0 \label{negative_semidefinite_derivative},
%	%				, \\
%	%				\big| dW_x \big| \left| x \right|_{0_X} &\leq b \, W(x)
%	%				\label{polynomial_condition}
%\end{align}
%\begin{equation}
%	\mathcal{L}_{h(x,y)} 
%	W
%	\leq 
%%	\Big(
%	\alpha \circ d_Y(y,0_Y) \, 
%	%	\circ d(0_Y,y) \cdot d(0_X,x) 
%	W(x)
%%	+ 
%%	\beta
%%	\big(\left|y\right|_N\big)
%%%	\Big)
%	,
%	\label{growth_restriction}
%\end{equation}
%where $d_Y$ is a metric on $Y$.
%		Then, the forward trajectory of \eqref{outer_bound}-\eqref{inner_bound} through any initial condition ${(x_0,y_0) \in X \times \mathcal{B}_Y}$ is precompact.
%	\end{theorem}
%
%
%
%%
%%
%%
%%\begin{theorem}[Precompact Forward Trajectories of Cascade]
%%	\label{boundedness}
%%	Consider the cascade on $X \times Y$ given by
%%	\begin{subequations}
%%		\begin{align}
	%%			\dot{x} &= f(x,y), \label{outer_bound} \\
	%%			\dot{y} &= g(y). \label{inner_bound} 
	%%		\end{align}
%%	\end{subequations}
%%	%	where ${f(0_X,0_Y) = 0}$ and ${g(0_Y) = 0}$ for some points ${0_X \in M}$ and ${0_Y \in N}$.
%%	%	 and all trajectories are bounded.
%%	For some ${0_X \in M}$, suppose the following conditions hold:
%%	%	\begin{enumerate}
%%		\setlist[enumerate,1]{leftmargin=.5cm}
%%		\begin{enumerate}[{1.}] %[{$C_1$.}]
	%%			\item ${0_Y \in N}$ is a stable hyperbolic 
	%%			%			almost-globally 
	%%			%			asymptotically 
	%%			equilibrium of \eqref{inner_bound}.
	%%			\item The {\normalfont interconnection term} given by
	%%			\begin{equation}
		%%				h : X \times Y \rightarrow TM, \ (x,y) \mapsto f(x,y) - f(x,0_Y)
		%%			\end{equation}
	%%			satisfies the growth restriction
	%%			\begin{equation}
		%%				\big|\big|
		%%				h(x,y)
		%%				\big|\big|_M \leq \alpha\big(\left|y\right|_{0_Y}\big) \left|x\right|_{0_X} + \beta\big(\left|y\right|_{0_Y}\big)
		%%				\label{growth_restriction}
		%%			\end{equation}
	%%			for some functions  ${\alpha,\beta \in \mathcal{K}}$.
	%%			\item  
	%%			%			\begin{equation}
		%%				%				{\dot{x} = f(x,0_Y)},
		%%				%				\label{unforced_bound}
		%%				%			\end{equation}
	%%			There exists a positive definite function
	%%			${W : M \rightarrow \mathbb{R}}$ with compact sublevel sets
	%%			such that for all ${\left|x\right| > a}$,
	%%			\begin{align}
		%%				\mathcal{L}_{f(x,0_Y)} W &\leq 0 \label{negative_semidefinite_derivative}, \\
		%%				\big| dW_x \big| \left| x \right|_{0_X} &\leq b \, W(x)
		%%				\label{polynomial_condition}
		%%			\end{align}
	%%			for some constants $a, b \in \mathbb{R}$.
	%%			%		and moreover the chain recurrent set of \eqref{unforced} consists solely of hyperbolic equilibria.
	%%			%		and moreover the chain recurrent set of \eqref{inner_theorem} consists solely of hyperbolic equilibria.
	%%			%		\eqref{inner_theorem} 
	%%			%			and the chain recurrent set of \eqref{unforced} consists solely of hyperbolic equilibria.
	%%			%		and the chain recurrent set of 
	%%			%		\eqref{unforced} 
	%%			%		consists solely of hyperbolic equilibria.
	%%			%			\item 
	%%			%			For any $x \in M$ and ${y \in \mathcal{B}_Y}$ (the basin of attraction of $0_Y$), the trajectory of \eqref{outer_bound}-\eqref{inner_bound} with initial condition $(x,y)$ is bounded in forward time.
	%%			%		All trajectories beginning 
	%%			%		in the set $X \times \mathcal{B}_Y$ are bounded in forward time, where ${\mathcal{B}_Y \subseteq N}$ is the basin of attraction of $0_Y$.
	%%			%		, where ${\mathcal{B}_Y \subseteq N}$ denotes the basin of attraction of $0_Y$ with respect to \eqref{inner_theorem}. 
	%%		\end{enumerate}
%%		Then, for any $x \in M$ and ${y \in \mathcal{B}_Y}$ (the basin of attraction of $0_Y$), the forward trajectory of \eqref{outer_bound}-\eqref{inner_bound} with initial condition $(x,y)$ is precompact.
%%	\end{theorem}
%	
%	\begin{proof}
%		First, we remark that $(X \times \mathcal{B}_Y) \setminus S$ is compact by the properness of $W$, and therefore the closure of any trajectory must either be contained in this compact set, or enter $S$. It will thus suffice to show that the forward trajectory starting at any given initial condition ${(x_0,y_0) \in S}$ is precompact.
%%		in the set 
%%		\begin{equation}
	%%			S = 
	%%			\left\{
	%%			x \in M : W(x) > a
	%%			\right\}
	%%			\times \mathcal{B}_Y.
	%%		\end{equation}
%		Since $			\dot{x} = f(x,0_Y) + h(x,y)$, 
%		we have 
%		\begin{equation}
	%			\dot{W} =
	%%			 \mathcal{L}_{f(x,y)} W = 
	%			 \mathcal{L}_{f(x,0_Y)} + \mathcal{L}_{h(x,y)} W \leq \mathcal{L}_{h(x,y)} W, 
	%		\end{equation}
%		where the inequality is due to \eqref{negative_semidefinite_derivative}. Thus by \eqref{growth_restriction}, 
%		\begin{align}
	%\dot{W}
	%\leq 
	%\alpha \circ d_Y(y(t),0_Y) \,
	%%\big(\left|y\right|_N\big)
	%%	\circ d(0_Y,y) \cdot d(0_X,x) 
	%%\left|x\right|_M
	%W.
	%\label{distance_w_dot}
	%		\end{align}
%%		where the first inequality follows from the given definition of the norm of a differential of a function on a Riemannian manifold, and the second is due to \eqref{growth_restriction}.
%		Since $0_Y$ is locally exponentially stable and ${y_0 \in \mathcal{B}_Y}$, there exists positive constants $K$ and $\omega$ such that 
%		\begin{equation}
	%%			\alpha\left( \left|y\right|_{N} \right)
	%d_Y(y(t),0_Y)
	%%			,\beta\left( \left|y\right|_{N} \right) 
	%			\leq
	%%			 \gamma\left(\left|y_0\right|_{N}\right) 
	%			K \, d_Y(y_0,0_Y) \, 
	%			e^{-\omega t}.
	%			\label{les_bound}
	%		\end{equation}
%Moreover, since $\alpha$ is differentiable at zero and ${\alpha(0) = 0}$, the limit
%\begin{equation}
%	\lim_{t\to \infty} \frac{\alpha(A \,e^{-\omega t})}{A \,e^{-\omega t}}
%\end{equation}
%is finite for any ${A \in \mathbb{R}_{\geq 0}}$, and therefore the quotient inside this limit is bounded. Thus there exists a ${B \in \mathbb{R}_{\geq 0}}$ such that
%\begin{equation}
%	\alpha(A \,e^{-\omega t}) \leq B \, e^{-\omega t} \text{ for all } t \in \mathbb{R}_{\geq 0}.
%\end{equation}
%%		\begin{equation}
	%%	%			\alpha\left( \left|y\right|_{N} \right)
	%%	\alpha \circ d_Y(y,0_Y)
	%%	%			,\beta\left( \left|y\right|_{N} \right) 
	%%	\leq
	%%	%			 \gamma\left(\left|y_0\right|_{N}\right) 
	%%	\gamma \circ d_Y(y_0,0_Y) \, 
	%%	e^{-\omega t}.
	%%\end{equation}
	%		It then follows from \eqref{distance_w_dot} and \eqref{les_bound} that for some ${C \in \mathbb{R}_{\geq 0}}$,
	%		\begin{align}
		%			\dot{W}  
		%			% 		&\leq \big|dW \big| \, \Big( 
		%			% 			\alpha \circ \gamma (\left|y_0\right|) \left|x\right| + \beta \circ \gamma (\left|y_0\right|) 
		%			% 		\Big) e^{-rt} \\
		%			&\leq
		%%			\gamma\left(\left|y_0\right|_{N}\right) e^{-\omega t}
		%			C \, e^{-\omega t} \,
		%			W
		%		\end{align}
	%%		for some ${K \in \mathcal{K}}$. Thus, we have 
	%%		\begin{align}
		%%			\dot{W} \leq b \,
		%%			K \big(\left|y_0\right|_{0_Y}\big) \, e^{- \omega t} \, W(x)  .
		%%		\end{align}
	%		Application of Gr\"{o}nwall's inequality provides the bound
	%		\begin{align}
		%			W\big(x(t)\big)
		%			 \leq 
		%			e^{ 
			%%			\gamma\left(\left|y_0\right|_{N}\right)  \,
			%%			\gamma \circ d_Y(y_0,0_Y) \,
			%			C \,	\int_0^t e^{- \omega s} ds
			%			} \ 
		%			W(x_0),
		%		\end{align}
	%		and since 
	%		\begin{equation}
		%			\int_0^\infty e^{- \omega s} ds = \frac{1}{\omega}
		%		\end{equation}
	%		it is clear $W\big(x(t)\big)$ is bounded for all forward time. Since $W$ is proper, $x$ is therefore contained in a compact set for all forward time. Since the forward trajectory of $y$ is precompact by asymptotic stability, this completes the proof.
	%%		Moreover, the complement of $S$ is also compact, and thus, all forward trajectories starting in $X \times \mathcal{B}_Y$ have compact closure.
	%%		To do: clean this up a little bit, with crisper argument regarding either entering $S$ or being in a compact sublevel set.
	%		% Furthermore, any initial condition in $X \times \mathcal{B}_Y$ must either remain bounded or enter $S$, so all trajectories beginning in $X \times \mathcal{B}_Y$ are bounded.
	%	\end{proof}
%






\begin{theorem}[Precompact Forward Trajectories of Cascade]
	\label{boundedness}
	Consider the cascade on $X \times Y$ given by
	\begin{subequations}
		\begin{align}
			\dot{x} &= f(x,y), \label{outer_bound} \\
			\dot{y} &= g(y). \label{inner_bound} 
		\end{align}
	\end{subequations}
	Let ${0_Y \in Y}$ be a stable hyperbolic equilibrium of \eqref{inner_bound} with basin of attraction $\mathcal{B}_Y$,
	and define the {interconnection term}
	\begin{align}
		h : X \times Y \to TX, \, (x,y) \mapsto f(x,y) - f(x,0_Y).
	\end{align}
	Suppose that the following conditions hold:
	\setlist[enumerate,1]{leftmargin=.5cm}
	\begin{enumerate}[{1.}] %[{$C_1$.}]
		\item 	The proper, differentiable function 
		\begin{equation}
			W : X \rightarrow \mathbb{R}_{\geq 0}
		\end{equation}
		is a (non-strict) Lyapunov function for the system
		\begin{equation}
			\dot{x} = f(x,0_Y). \label{unforced_bound}
		\end{equation}
		\item 
		There exists some ${c \in \mathbb{R}}$ and continuous functions
		\begin{align}
			\alpha, \, \beta : \mathcal{B}_Y \rightarrow \mathbb{R}_{\geq 0}	 	
		\end{align} 
		which are vanishing and 
		differentiable at $0_Y$ and 
		%a function ${\alpha \in \mathcal{K}}$,
		\begin{equation}
			%	 dW \big(h(x,y)\big)
			\mathcal{L}_{h(x,y)} W 
			\leq 
			\alpha(y) \, 
			%	\circ d(0_Y,y) \cdot d(0_X,x) 
			W(x) + \beta(y),
			%	+ 
			%	\beta
			%	\big(\left|y\right|_N\big)
			%%	\Big)
			\label{growth_restriction}
		\end{equation}
		for all $(x,y)$ such that $W(x) \geq c$ and $y \in \mathcal{B}_Y$.
	\end{enumerate}
	%	where $d_Y$ is a metric on $Y$.
	Then, the forward trajectory of \eqref{outer_bound}-\eqref{inner_bound} through any initial condition ${(x_0,y_0) \in X \times \mathcal{B}_Y}$ is precompact.
\end{theorem}













\begin{proof}
	
	Since $W$ is a proper Lyapunov function for \eqref{unforced_bound}, the forward trajectory through any initial condition of the form ${(x_0,0_Y)}$ is precompact, so it suffices to consider initial conditions $(x_0,y_0)$ with ${y_0\neq 0_Y}$. Fix ${(x_0,y_0)\in X\times B_Y}$ with ${y_0\neq 0_Y}$ and let $\big(x(t),y(t)\big)$ denote its forward trajectory. We prove the claim in two steps.
	
	\begin{step}
		\label{bound_alpha_beta}
		There exist positive constants $A$, $B$, and $\omega$ such that
		$\alpha\big(y(t)\big)\leq A e^{-\omega t}$ and  $\beta\big(y(t)\big)\leq Be^{-\omega t}$ for all $t\geq 0$.
	\end{step}
	\begin{proof}
		\renewcommand{\qedsymbol}{$\blacktriangledown$}
%		\renewcommand{\qedsymbol}{\raisebox{2pt}{\rotatebox[origin=c]{180}{$\triangle$}}}
%		Fix any continuous Riemannian metric on $Y$, let $\text{dist}$ denote the associated distance metric, and define 		
		Let ${d(t)\coloneqq \text{dist}(y(t),0_Y) > 0}$.
		Since a stable hyperbolic equilibrium is locally exponentially stable with respect to the distance associated to any continuous Riemannian metric, there exist $C_0, \omega > 0$ such that, for all $t\geq 0$,
		\begin{equation}\label{eq:d-bound}
			d(t)\leq C_0e^{-\omega t}.
		\end{equation}
		Next, since 
%		${d(t)\to 0}$ as ${t\to\infty}$ and 
		$\alpha$ and $\beta$ are vanishing and differentiable at $0_Y$, a local coordinate calculation (using uniform equivalence of continuous Riemannian metrics over compact sets) shows
		\begin{equation}
			\limsup_{t\to\infty}\frac{\alpha\big(y(t)\big)}{d(t)} < \infty, \quad \limsup_{t\to\infty}\frac{\beta\big(y(t)\big)}{d(t)} < \infty.
		\end{equation}
		Since the above quotients are also continuous functions of $t$, it follows that they are bounded.
		Hence there exist $C_1, C_2 > 0$ such that $\alpha(y(t))/d(t)< C_1$ and $\beta(y(t))/d(t) < C_2$ for all $t\geq 0$.
		When combined with \eqref{eq:d-bound}, we obtain the desired bounds,
		%
		%, this implies that
		%\begin{equation}\label{eq:alpha-beta-bound}
		%	\alpha(y(t))\leq A e^{-\omega t}, \qquad \beta(y(t))\leq Be^{-\omega t}
		%\end{equation}
		where $A:= C_0 C_1$ and $B:= C_0 C_2$.
	\end{proof}
	
	%\begin{step}
	%		$\dot{W} \leq Ae^{-\omega t} \, W + B e^{-\omega t}.$
	%%		$\frac{d}{dt}W\big(x(t)\big)\leq Ae^{-\omega t} \, W\big(x(t)\big) + B e^{-\omega t}.$
	%\end{step}
	\begin{step}
		$W\big(x(t)\big)$ is bounded for all $t > 0$.
	\end{step}
	\begin{proof}
		\renewcommand{\qedsymbol}{$\blacktriangledown$}
		
		Since $W$ is a Lyapunov function for \eqref{unforced_bound}, we have
		\begin{align}
			\label{lyapunov_bound}
			%	\tfrac{d}{dt}W(x(t))
			%	= dW \big(f(x,0_Y) + h(x,y)\big) 
			%	\leq dW \big(h(x,y)\big).
			\dot{W} 
			= \mathcal{L}_{f(x,0_Y) + h(x,y)}W 
			\leq \mathcal{L}_{h(x,y)}W.
		\end{align}
		Consider any $t_2 \geq t_1 \geq 0$ such that $W(x([t_1,t_2]))\subseteq [c,\infty)$.
		Then for all $t\in [t_1,t_2]$, \eqref{growth_restriction}, \eqref{lyapunov_bound}, and the conclusion of the previous step imply that
		\begin{equation}
			\tfrac{d}{dt}W\big(x(t)\big)\leq Ae^{-\omega t} \, W\big(x(t)\big) + B e^{-\omega t}.
		\end{equation}
		Thus, by the comparison principle \cite[p.~102, Lem.~3.4]{khalil1996nonlinear},
		\begin{align}
			&
			\begin{aligned}
				W\big(x(t_2)\big)&\leq e^{\int_{t_1}^{t_2}Ae^{-\omega t}\, dt}W\big(x(t_1)\big) \ + \\
				& \qquad \quad \quad  \int_{t_1}^{t_2}e^{\int_{t}^{t_2}Ae^{-\omega s}\, ds}Be^{-\omega t}\, dt
			\end{aligned}
			\\
			&\phantom{W\big(x(t_2)\big)}\leq e^\frac{A}{\omega} \Big(W\big(x(t_1)\big)+ \tfrac{B}{\omega}\Big),
		\end{align}
		where the second inequality holds because
			${\int_a^b e^{-\omega t}\, dt \leq \frac{1}{\omega}}$ for any $b \geq a \geq 0$.
%		\begin{equation}
%			\int_a^b e^{-\omega t}\, dt \leq \int_0^\infty e^{-\omega t}\, dt = \frac{1}{\omega}
%		\end{equation}
%		for any $b \geq a \geq 0$.
		This implies that for all $t\geq 0$,
		$$W(x(t))\leq C:= \big(\max\big\{c,W\big(x(t_1)\big)\big\}+\tfrac{B}{\omega}\big)e^\frac{A}{\omega}. \qedhere $$
	\end{proof}
	%		\renewcommand{\qedsymbol}{}
	%\vspace{-20pt}
	\noindent 
	Hence, by the properness of $W$ and attractiveness of $0_Y$,
%	y([0,\infty))$ is precompact since ${y(t)\to 0_Y}$ as ${t\to\infty}$.
the forward trajectory through $(x_0,y_0)$ is 
%contained in the precompact  set $x\big([0,\infty)\big)\times y\big([0,\infty)\big)$, so is 
precompact.
%	Hence, ${x\big([0,\infty)\big)\subset X}$ is precompact since $W$ is proper, and $y([0,\infty))$ is precompact since ${y(t)\to 0_Y}$ as ${t\to\infty}$.
%	The forward trajectory through $(x_0,y_0)$ is contained in the precompact  set $x\big([0,\infty)\big)\times y\big([0,\infty)\big)$, so is precompact.
\end{proof}




















%\clearpage
%
%\textbf{To Do: update proof}
%	\begin{proof}
	%	$(X \times \mathcal{B}_Y) \setminus S$ is compact by the properness of $W$, so the closure of any trajectory must either be contained in this compact set, or enter $S$. It will thus suffice to show that the forward trajectory starting at any given initial condition ${(x_0,y_0) \in S}$ is precompact.
	%	%		where the first inequality follows from the given definition of the norm of a differential of a function on a Riemannian manifold, and the second is due to \eqref{growth_restriction}.
	%	Since any basin of attraction is diffeomorphic to $\mathbb{R}^n$ \textbf{(is this true as stated?)}, and the closure of any attracting forward trajectory is diffeomorphic to the closed unit interval, there exists a smooth chart $\varphi : \mathcal{B}_Y \rightarrow \mathbb{R}^n$ and a constant vector $u \in \mathbb{R}^n$ such that 	for all $t \geq 0$,
	%	\begin{align}
		%		\varphi(0_Y) = 0 \textrm{ and } 
		%		\varphi \circ y(t) = d\big(0_Y,y(t)\big) \, u.
		%		\label{clever_chart}
		%	\end{align}
	%Moreover, since $\alpha$ is differentiable at $0_Y$, the map
	%\begin{equation}
	%	\alpha \circ \varphi^{-1} : U \subseteq \mathbb{R}^n \rightarrow \mathbb{R}_{\geq0}
	%\end{equation}
	%is differentiable at $0$, and therefore the limit
	%	\begin{equation}
		%		\lim_{h \rightarrow 0} \frac{\alpha\circ \varphi^{-1} (0+h \, u) - \overbrace{\alpha \circ \varphi^{-1}(0)}^{=0} }{h}
		%	\end{equation}
	%exists. Moreover, since $0_Y$ is attractive, 
	%%since $y_0 \in \mathcal{B}_Y$, 
	%\begin{equation}
	%	\lim_{t \rightarrow \infty}
	%	d(0_Y,y(t)) = 0
	%\end{equation}
	%for any metric $d(\cdot,\cdot)$ on $Y$. It then follows that
	%\begin{equation}
	%		\lim_{t \rightarrow \infty} \frac{\alpha\circ \varphi^{-1} \Big(d\big(0_Y,y(t)\big) \, u\Big) }{ d\big(0_Y,y(t)\big) }
	%\end{equation}
	%exists as well, and so from \eqref{clever_chart}, we have
	%\begin{equation}
	%	\lim_{t \rightarrow \infty} \frac{\alpha\big(y(t)\big) }{ d(0_Y,y(t)) } < \infty.
	%\end{equation}
	%As a result, the quotient inside the limit is bounded, i.e. 
	%	\begin{equation}
		%	%			\alpha\left( \left|y\right|_{N} \right)
		%	\alpha \big(y(t)\big)
		%	%			,\beta\left( \left|y\right|_{N} \right) 
		%	\leq
		%	%			 \gamma\left(\left|y_0\right|_{N}\right) 
		%	A \,
		%	d(0_Y,y(t)),
		%	\label{limit_bound}
		%\end{equation}
		%for some constant ${A > 0}$. Since $0_Y$ is attractive and hyperbolic, there also exist positive constants $K$ and $\omega$ such that 
		%	\begin{equation}
			%		%			\alpha\left( \left|y\right|_{N} \right)
			%		d\big(0_Y,y(t)\big)
			%		%			,\beta\left( \left|y\right|_{N} \right) 
			%		\leq
			%		%			 \gamma\left(\left|y_0\right|_{N}\right) 
			%		B \,
			%		e^{-\omega t}
			%		\text{ for all } t \geq 0.
			%		\label{les_bound}
			%	\end{equation}
		%%	Moreover, since $\alpha$ is differentiable at $0_Y$ and ${U(0_Y) = 0}$, the limit
		%%\begin{equation}
		%%	\lim_{t\to \infty} \frac{\alpha(A \,e^{-\omega t})}{A \,e^{-\omega t}}
		%%\end{equation}
		%%is finite for any ${A \in \mathbb{R}_{\geq 0}}$, and therefore the quotient inside this limit is bounded. 
		%Clearly, \eqref{limit_bound} and \eqref{les_bound} imply that
		%	\begin{equation}
			%		\alpha(y(t)) \leq C \, e^{-\omega t} 
			%		\text{ for all } t \geq 0.
			%		\label{ultimate_exponential_bound}
			%	\end{equation}
		%where $C > 0$. Moreover, 
		%	%		\begin{equation}
			%		%	%			\alpha\left( \left|y\right|_{N} \right)
			%		%	\alpha \circ d_Y(y,0_Y)
			%		%	%			,\beta\left( \left|y\right|_{N} \right) 
			%		%	\leq
			%		%	%			 \gamma\left(\left|y_0\right|_{N}\right) 
			%		%	\gamma \circ d_Y(y_0,0_Y) \, 
			%		%	e^{-\omega t}.
			%		%\end{equation}
			%by \eqref{growth_restriction}, we have
			%		\begin{align}
				%			\dot{W} = \mathcal{L}_{f(x,y)} W 
				%			\leq 
				%			\alpha\big(y(t)\big) \,
				%			%\big(\left|y\right|_N\big)
				%			%	\circ d(0_Y,y) \cdot d(0_X,x) 
				%			%\left|x\right|_M
				%			W
				%			%		\big(x(t)\big).
				%			\label{distance_w_dot}
				%		\end{align}
			%		It then follows from \eqref{distance_w_dot} and \eqref{ultimate_exponential_bound} that 
			%%		for some ${C \in \mathbb{R}_{\geq 0}}$,
			%		\begin{align}
				%			\dot{W}  
				%			% 		&\leq \big|dW \big| \, \Big( 
				%			% 			\alpha \circ \gamma (\left|y_0\right|) \left|x\right| + \beta \circ \gamma (\left|y_0\right|) 
				%			% 		\Big) e^{-rt} \\
				%			&\leq
				%			%			\gamma\left(\left|y_0\right|_{N}\right) e^{-\omega t}
				%			C \, e^{-\omega t} \,
				%			W
				%		\end{align}
			%		%		for some ${K \in \mathcal{K}}$. Thus, we have 
			%		%		\begin{align}
				%			%			\dot{W} \leq b \,
				%			%			K \big(\left|y_0\right|_{0_Y}\big) \, e^{- \omega t} \, W(x)  .
				%			%		\end{align}
			%		Application of Gr\"{o}nwall's inequality provides the bound
			%		\begin{align}
				%			W\big(x(t)\big)
				%			\leq 
				%			e^{ 
					%				%			\gamma\left(\left|y_0\right|_{N}\right)  \,
					%				%			\gamma \circ d_Y(y_0,0_Y) \,
					%				C \,	\int_0^t e^{- \omega s} ds
					%			} \ 
				%			W(x_0),
				%		\end{align}
			%		and since 
			%		\begin{equation}
				%			\int_0^\infty e^{- \omega s} ds = \frac{1}{\omega}
				%		\end{equation}
			%		it is clear $W\big(x(t)\big)$ is bounded for all forward time. Since $W$ is proper, $x$ is therefore contained in a compact set for all forward time. Since the forward trajectory of $y$ is precompact by asymptotic stability, this completes the proof.
			%		%		Moreover, the complement of $S$ is also compact, and thus, all forward trajectories starting in $X \times \mathcal{B}_Y$ have compact closure.
			%		%		To do: clean this up a little bit, with crisper argument regarding either entering $S$ or being in a compact sublevel set.
			%		% Furthermore, any initial condition in $X \times \mathcal{B}_Y$ must either remain bounded or enter $S$, so all trajectories beginning in $X \times \mathcal{B}_Y$ are bounded.
			%	\end{proof}
		%	
		%%It is easily seen that we obtain the following corollary, which admits analysis of only the ``interconnection term'', instead of the full dynamics.
		%
		%%\begin{corollary}
		%%Suppose that all assumptions of Theorem \ref{boundedness} other than \eqref{growth_restriction} hold. 
		%%Define the interconnection term by
		%%\begin{equation}
		%%	h : X \times Y \rightarrow TM, \ (x,y) \mapsto f(x,y) - f(x,0_Y)
		%%\end{equation}
		%%Suppose the function $W$ of \eqref{W_boundedness} is a LaSalle function for the dynamics
		%%\begin{equation}
		%%	\dot{x} = f(x,0_Y),
		%%\end{equation}
		%%that is, $\mathcal{L}_{f(x,0_Y)} W \leq 0$, 
		%%and moreover 
		%%\begin{equation}
		%%	\mathcal{L}_{h(x,y)} \leq \alpha(y) W(x).
		%%\end{equation}
		%%Then, the forward trajectory of \eqref{outer_bound}-\eqref{inner_bound} through any initial condition ${(x_0,y_0) \in X \times \mathcal{B}_Y}$ is precompact.
		%%
		%%\end{corollary}
		%
		%
		%
		%
		%
		%
		%
		%
		%
		
		
		
		
		
		
		
		
		
		
		
		
		
		
		
		
		%We note that in the previous theorem, the basin of attraction of $0_Y$ contains \textit{all} initial conditions attracted to $0_Y$, i.e. it extends beyond the region of exponential stability.
		
		%	We emphasize that any proper LaSalle function for the limiting dynamics ${\dot{x} = f(x,0_Y)}$ that also satisfies \eqref{polynomial_condition} will suffice for the function ${W : M \rightarrow \mathbb{R}}$ required in Theorem \ref{boundedness}.
		%	
		%	\textbf{Question: Under what kind of stability properties does a Lyapunov function satisfying \eqref{polynomial_condition} exist?}
		%	
		%	\textbf{Question: does the choice of metric matter?}
		
		
		%\textbf{Redoing this theorem without Riemannian metric:}
		%
		%\begin{theorem}[Forward Boundedness of Cascade]
		%	\label{boundedness}
		%	Consider the cascade on $X \times Y$ given by
		%	\begin{subequations}
			%		\begin{align}
				%			\dot{x} &= f(x,y), \label{outer_bound} \\
				%			\dot{y} &= g(y). \label{inner_bound} 
				%		\end{align}
			%	\end{subequations}
		%	%	where ${f(0_X,0_Y) = 0}$ and ${g(0_Y) = 0}$ for some points ${0_X \in M}$ and ${0_Y \in N}$.
		%	%	 and all trajectories are bounded.
		%	Suppose the following conditions hold:
		%	%	\begin{enumerate}
			%		\setlist[enumerate,1]{leftmargin=.5cm}
			%		\begin{enumerate}[{1.}] %[{$C_1$.}]
				%			\item ${0_Y \in N}$ is a stable hyperbolic 
				%			equilibrium of \eqref{inner_bound}.
				%			\item  
				%			%			\begin{equation}
					%				%				{\dot{x} = f(x,0_Y)},
					%				%				\label{unforced_bound}
					%				%			\end{equation}
				%			There exists a positive definite function
				%			${W : M \rightarrow \mathbb{R}}$ with compact sublevel sets
				%			such that for all ${\left|x\right| > a}$,
				%			\begin{align}
					%				\mathcal{L}_{f(x,0_Y)} W &\leq 0 \label{negative_semidefinite_derivative}, \\
					%				\big| dW_x \big| \left| x \right| &\leq b \, W(x)
					%				\label{polynomial_condition}
					%			\end{align}
				%			for some constants $a, b \in \mathbb{R}$.
				%			\item 
				%The interconnection term given by
				%\begin{equation}
				%	h : X \times Y \rightarrow TM, \ (x,y) \mapsto f(x,y) - f(x,0_Y)
				%\end{equation}
				%satisfies the growth restriction
				%\begin{equation}
				%	\mathcal{L}_{h(x,y)} W 
				%	\leq \alpha(\left|y\right|) \left|x\right| + \beta(\left|y\right|)
				%	\label{growth_restriction}
				%\end{equation}
				%for some class $\mathcal{K}$ functions $\alpha,\beta$.
				%		\end{enumerate}
			%		Then, for any $x \in M$ and ${y \in \mathcal{B}_Y}$ (the basin of attraction of $0_Y$), the trajectory of \eqref{outer_bound}-\eqref{inner_bound} with initial condition $(x,y)$ is bounded in forward time.
			%	\end{theorem}















\section{Application of the Results}



%\subsection{Revisiting the Motivating Example}

%Apply the results from the Riemannian section and the main section, i.e. show that the subsystems are of the Prop 2 form, and the interconnection term satisfies the growth rate, where we use to Sasaki metric to talk about the tangent bundle as a riemannian manifold.

%\textbf{This example needs a lot of cleaning up}

We now revisit the motivating example \eqref{outer_example}-\eqref{inner_example} evolving on 
${T\mathbb{T}^2 = T\mathbb{S}^1 \times T\mathbb{S}^1}$,
%the tangent bundle of the torus, whose trajectories are shown in Fig. \ref{trajectories}. 
using our results to show that the full cascade is in fact almost globally asymptotically stable. The system can be given explicitly in the form \eqref{outer_general}-\eqref{inner_general} by
\begin{subequations}
\begin{align}
	\label{outer_example_state_space}
	\frac{d}{dt}{\begin{bmatrix}
			\theta \\ \dot{\theta}
	\end{bmatrix}} &= {\begin{bmatrix}
			\dot{\theta} \\ -(\sin \theta + \dot{\theta}) \cos 2\phi
	\end{bmatrix}}, \\
	\label{inner_example_state_space}
	\frac{d}{dt}{\begin{bmatrix}
			\phi \\ \dot{\phi}
	\end{bmatrix}} &= {\begin{bmatrix}
			\dot{\phi} \\ -(\sin \phi + \dot{\phi})
	\end{bmatrix}}.
\end{align}
%\begin{align}
%	\label{outer_example_state_space}
%	\frac{d}{dt}\overbrace{\begin{bmatrix}
%			\theta \\ \dot{\theta}
%	\end{bmatrix}}^{x} &= \overbrace{\begin{bmatrix}
%			\dot{\theta} \\ -(\sin \theta + \dot{\theta}) \cos 2\phi
%	\end{bmatrix}}^{f(x,y)}, \\
%	\label{inner_example_state_space}
%	\frac{d}{dt}\underbrace{\begin{bmatrix}
%		\phi \\ \dot{\phi}
%\end{bmatrix}}_{y} &= \underbrace{\begin{bmatrix}
%		\dot{\phi} \\ -(\sin \phi + \dot{\phi})
%\end{bmatrix}}_{g(y)}.
%\end{align}
\end{subequations}
%and a Riemannian structure via the metric
%\begin{equation}
%	g = 
%	d\dot{\theta} \otimes d\dot{\theta} +
%	d\ddot{\theta} \otimes d\ddot{\theta}
%\end{equation}
%where $\big|\,\cdot\,\big|$ is the absolute value.
It is easily verified that \eqref{inner_example_state_space} takes the form of the Euler-Lagrange dynamics \eqref{euler_lagrange} 
%of Proposition \ref{second_order_dynamics} 
for the kinetic energy metric,  Rayleigh dissipation, and Morse potential function
\begin{gather}
	\kappa = \nu = d \phi \otimes d\phi, \quad  
	V : \mathbb{S}^1 \rightarrow \mathbb{R}, \, \phi \mapsto 1-\cos \phi.
\end{gather}
%and the Rayleigh dissipation 
%\begin{equation}
%	\nu 
%%	: T\mathbb{S}^1 \times T\mathbb{S}^1 \rightarrow T\mathbb{S}^1 
%	= 
%%	\frac{\partial}{\partial \phi } \cdot 
%	\tfrac{1}{2}d \phi \otimes
%	 d\phi.
%%	\nu : T\mathbb{S}^1 \times T\mathbb{S}^1 \rightarrow \mathbb{R}, \, \theta \mapsto \dot{\theta}^2
%\end{equation}
Thus by Remark \ref{summarize_chain_recurrence_section}, \eqref{inner_example_state_space} is almost globally asymptotically stable and locally exponentially stable with respect to ${y = (\phi,\dot{\phi}) = (0,0)}$, and moreover its chain recurrent set consists solely of hyperbolic equilibria. Clearly, the same is true for
${x = (\theta,\dot{\theta}) = (0,0)}$ with respect to \eqref{outer_example_state_space} restricted to ${y = (0,0)}$. 
%\begin{equation}
%	0_Q = \arg \min_{q \in Q} V(q)
%\end{equation}
Hence by Theorem \ref{almost_global_stability}, it  suffices to show precompactness of forward trajectories, which
we accomplish using Theorem \ref{boundedness} and the total energy function \eqref{total_energy} given by
%for the unforced outer loop, given by
\begin{equation}
	W : (\theta,\dot{\theta}) \mapsto 1-\cos \theta +  \tfrac{1}{2} \dot{\theta}^2.
\end{equation}
%which has compact sublevel sets.
%, as well as the distance metric 
%\begin{equation}
%	\begin{aligned}
%		d_{T\mathbb{S}^1}
%		%	_{T\mathbb{S}^1} 
%		: \big(
%		(\theta_1,\dot{\theta}_1),
%		(\theta_2,\dot{\theta}_2)
%		\big) \mapsto & \\
%		\big|\theta_1 - \theta_2 \big| \hspace{-6pt} & \mod 2\pi +  \big|\dot{\theta}_1 - \dot{\theta}_2\big|.
%	\end{aligned}
%\end{equation}
%$.%and
%\begin{equation}
%	\left|dW_{(\theta,\dot{\theta})}
%	\right| = \sup_{(z_1,z_2) \neq 0}
%\frac{\theta z_1 + \dot{\theta} z_2}{\sqrt{z_1^2 + z_2^2}} = \sqrt{\theta^2 + \dot{\theta}^2}.
%\end{equation}
%Therefore, 
%\begin{align}
%	\left|dW_{(\theta,\dot{\theta})}
%	\right| \, 
%	\left|{(\theta,\dot{\theta})}
%	\right|
%	&= \sqrt{\theta^2 + \dot{\theta}^2}\sqrt{\theta^2 + \dot{\theta}^2} \\
%	&= {\theta^2 + \dot{\theta}^2} \\ 
%	&\leq 2 \, W(x).
%\end{align}
The interconnection term is given by
\begin{align}
	h(x,y) = \begin{bmatrix}
		0 \\ \big(1-\cos 2\phi\big)\big(\sin \theta + \dot{\theta}\big) 
	\end{bmatrix},
\end{align}
%and therefore,
%\begin{align}
%	\left|\left|h(x,y)\right|\right|_{T\mathbb{S}^1} = (1-\cos \phi)\left|\sin \theta + \dot{\theta}\right| 
%\end{align}
%and for $W(x) > \pi$, it holds that 
and the directional derivative of $W$ along $h$ is 
\begin{equation}
	\mathcal{L}_{h(x,y)}W = 
	  {\big(1-\cos 2\phi\big)}
%	  _{
%		\alpha(\left|(\phi,\dot{\phi})\right|_{T\mathbb{S}^1})  
%	  }
%	  \underbrace
	  {\big(\sin \theta + \dot{\theta}\big) \dot{\theta}}.
%	  _{\leq a W(x)}
\label{computed_lie_derivative}
\end{equation}
We propose the 
%class $\mathcal{K}$ 
functions
\begin{equation}
	\alpha : 
%	T\mathbb{S}^1 \rightarrow \mathbb{R}_{\geq0},	 \,
 (\phi,\dot{\phi}) \mapsto 4 \, (1 - \cos 2\phi),
\quad
	\beta : 
%	T\mathbb{S}^1 \rightarrow \mathbb{R}_{\geq0},	 \,
(\phi,\dot{\phi}) \mapsto 0,
\end{equation}
which are differentiable and vanish at $(0,0)$. We compute
%\begin{align}
%	&\begin{aligned}
%			\alpha(x)	\,  W& (y)  + \beta(y)  
%		%	\alpha(\phi,\dot{\phi})	\,  W(\theta,\dot{\theta})   
%		\\
%		&= 4\, \big(1 - \cos 2 \phi\big) \big(1-\cos(\theta) +  \tfrac{1}{2} \dot{\theta}^2\big)
%	\end{aligned}
%	 \\
%	&
%		 \phantom{\alpha(x)	\,  W}
%	\geq \big(1 - \cos 2 \phi\big) \, \big(2\dot{\theta}^2\big). \label{simplified_inequality}
%\end{align}
\begin{align}
		\alpha(y)	W (x)  + \beta(y)  
		%	\alpha(\phi,\dot{\phi})	\,  W(\theta,\dot{\theta})   
		&= 4(1 - \cos 2 \phi) (1-\cos(\theta) +  \tfrac{\dot{\theta}^2}{2})
	\\
	&\geq \big(1 - \cos 2 \phi\big) \, \big(2\dot{\theta}^2\big). \label{simplified_inequality}
\end{align}
%Since ${\sin \theta \leq 1}$, we have 
% \begin{equation}
%	2\dot{\theta}^2 \geq \dot{\theta} \sin \theta + \dot{\theta}^2 \textrm{ for all } {\dot{\theta} \geq 1},
%\end{equation}
%and since ${\cos \theta \geq -1}$, we have 
%\begin{equation}
%	{\dot{\theta} \geq 1} \textrm{ for all } W(
%	x % \theta,\dot{\theta}
%	) \geq 4.
%	\label{last_inequality}
%\end{equation}
Since ${\sin \theta \leq 1}$ and ${\cos \theta \geq -1}$, we have 
\begin{equation}
	2\dot{\theta}^2 \geq \dot{\theta} \sin \theta + \dot{\theta}^2 \textrm{ for all } W(
	x % \theta,\dot{\theta}
	) \geq 4.
	\label{last_inequality}
\end{equation}
Thus in view of \eqref{computed_lie_derivative} and \eqref{simplified_inequality}-\eqref{last_inequality}, we have shown
%In view of \eqref{computed_lie_derivative}, it will thus suffice to show that 
% \begin{equation}
%	2\dot{\theta}^2 \geq \dot{\theta} \sin \theta + \dot{\theta}^2 \textrm{ for } W(x) \geq a, \, a \in \mathbb{R}.
%\end{equation}
%This is clearly true for $\dot{\theta} \geq 1$ since $\sin \theta \leq 1$, and any $x$ for which $W(x) \geq 3$ has $\dot{\theta} \geq 1$, therefore
\begin{equation}
		\mathcal{L}_{h(x,y)}W \leq 
\alpha(y)	\,  W(x) + \beta(y)
		\textrm{ for } W(x)  \geq 4,
\end{equation}
and it follows by Theorem \ref{boundedness} that all forward trajectories of \eqref{outer_example_state_space}-\eqref{inner_example_state_space} with ${y = (\phi,\dot{\phi})}$ starting in the basin of attraction of \eqref{inner_example_state_space} are precompact. Thus by Theorem \ref{almost_global_stability}, the system
%\eqref{outer_example_state_space}-\eqref{inner_example_state_space} i.e. \eqref{outer_example}-\eqref{inner_example} 
is almost globally asymptotically stable and locally exponentially stable with respect to ${(0,0,0,0) \in T\mathbb{T}^2}$.

 %\begin{equation}
%	(\sin(\phi) + \dot{\theta})\dot{\theta} \leq c(1-\cos \theta + \frac{1}{2} \dot{\theta}^2)
%\end{equation}
%\begin{equation}
%	\mathcal{L}_{h(x,y)}W \leq \alpha 
%\end{equation}

\section{Discussion}


%Discuss comparisons to GLOBAL results
%Such a result for our class of systems can be compared to the fact that a cascade of globally asymptotically stable systems in Euclidean space is globally asymptotically stable as long as all trajectories are bounded   

The disturbance robustness of systems with some similar properties was considered in \cite{Angeli2010}, and the connection to systems whose only chain recurrent points are equilibria was explored in Sec. IV therein. However, those results (when combined with \cite{Angeli2004}) can only certify the stability of a cascade if the outer loop is almost globally input to state stable. Indeed, \cite[Prop. 1]{Angeli2010} certifies robustness to small disturbances, while general robustness is achieved only under the additional assumption of ``ultimate boundedness'', a strong property which is absent from systems such as our motivating example. 
%Moreover, they assume the availability of a Lyapunov function with strictly negative time derivative off equilibria, whereas we require only a Lyapunov function which is decreasing to suitably characterize the chain recurrent set.

%Discuss Angeli paper, need to compare to \cite{Angeli2010}
%
%\begin{enumerate}
%	\item Not formulated for cascades without aISS
%	\item Assumes Lyapunov function with strictly negative derivative, not just decreasing
%	\item disturbances required to be smaller than some amount?
%	\item Prop 1: there exists a value for which disturbances bounded by this value which eventually converge cause the trajectories to converge. What if disturbances are above that value?
%	\item Def 2 (ultimate boundedness): the system never gets further away from some point than a class K function of infinity norm of signal plus constant
%	\item Prop 2: if the system satisfies prop 1 and ultimate boundedness, then aISS. 
%	\item Prop 3: a system is ultimately bounded a dissipation equality is satisfied where there is a negative $W$ term on the right. 
%\end{enumerate}

The condition that only equilibria are chain recurrent may seem restrictive, but in the stabilization of a desired state, such a property in the closed loop dynamics is quite desirable; indeed, it would run contrary to the goal of rapid convergence for states other than equilibria to exhibit chain recurrence. An obvious question asks whether our results could be extended to certify chains of more than two cascaded subsystems; any such extension would necessitate a characterization of the chain recurrent set of the full cascade and likely require additional hyperbolicity assumptions.

An interesting implication of our results is that for cascades with both subsystems being almost globally asymptotically stable and having a chain recurrent set of only hyperbolic equilibria, the only obstacle to almost global asymptotic stability of the full cascade is the precompactness of forward trajectories, a conclusion which can be compared with \cite[Prop. 4.1]{Sepulchre2012} for globally asymptotically stable systems. Theorem \ref{boundedness} casts this condition as the requirement that the transient from any initial condition injects only a finite amount of ``generalized energy'' into the outer loop. Since the former properties are enjoyed by cascades of mechanical systems with suitable dissipation and potential, we see promising directions for the constructive synthesis of cascaded geometric controllers with almost global asymptotic stability for robotic systems possessing a geometric flat output (such as quadrotors and aerial manipulators) \cite{Welde2023}, which enjoy a cascade-like structure where the evolution of the system in the shape space is uniquely determined by the evolution in the symmetry group. Indeed, for a constant reference, the error dynamics of the geometric quadrotor controller proposed in \cite{Lee2010} take the form \eqref{outer_general}-\eqref{inner_general}, with the subsystems being dissipative mechanical systems.


\section{Conclusion}

In this work, we present sufficient conditions for the almost global asymptotic stability of a cascade in which the subsystems are only \textit{almost} globally asymptotically stable, a natural setting for geometric control design. The main result relies on the assumption that the only chain recurrent points of the unforced outer loop are hyperbolic equilibria. The required precompactness condition is analogous to the assumption of forward boundedness used in related results for globally asymptotically stable systems in Euclidean space, and can similarly be verified via a growth rate inequality. The qualitative nature of the stability criteria facilitates the verification of control designs for cascades whose subsystems are governed by arbitrarily complex equations, so long as the subsystems enjoy certain fundamental dynamical properties.

%\clearpage

%
%
%\clearpage
%
%\section{Asymptotically Autonomous stuff}
%
%This won't be in this paper.
%
%\
%
%Need to argue that if 
%$$
%f(t_1,x,0_Y) = f(t_2,x,0_Y) \ \forall \ t_1,t_2 \in \overline{\mathbb{R}^+}
%$$
%and some other regularity conditions hold, then the cascade induces an asymptotically autonomous semiflow. Or just the first equation?
%
%\begin{theorem}
%	\label{simple_cascade}
%	Consider the cascade on $X \times Y$ given by
%	\begin{subequations}
%		\begin{align}
%			\dot{x} &= f(t,x,y) \label{outer_theorem_time} \\
%			\dot{y} &= g(y) \label{inner_theorem_time} 
%		\end{align}
%	\end{subequations}
%	%	where ${f(0_X,0_Y) = 0}$ and ${g(0_Y) = 0}$ for some points ${0_X \in M}$ and ${0_Y \in N}$.
%	%	 and all trajectories are bounded.
%	where all trajectories are bounded,
%	and suppose the following conditions hold:
%	\begin{enumerate}
%		\item The point ${0_Y \in N}$ is an almost-globally asymptotically stable equilibrium of \eqref{inner_theorem_time}.
%		\item For all $t_1,t_2 \in \mathbb{R}_{\geq 0}$, 
%		\begin{equation}
%			f(x,t_1,0_Y) = f(x,t_2,0_Y)
%		\end{equation}
%		i.e. the dynamics are time-invariant on ${M \times \{0_Y\}}$.
%		\item The point ${0_X \in M}$ is an almost-globally asymptotically stable equilibrium of the dynamics 
%		\begin{equation}
%			{\dot{x} = f(0,x,0_Y) \coloneqq f_0(x)}.
%			\label{unforced_time}
%		\end{equation}
%		%		and moreover the chain recurrent set of \eqref{unforced} consists solely of hyperbolic equilibria.
%		%		and moreover the chain recurrent set of \eqref{inner_theorem} consists solely of hyperbolic equilibria.
%		\item The chain recurrent sets of \eqref{inner_theorem} and \eqref{unforced} consist solely of hyperbolic equilibria.
%	\end{enumerate}
%	Then, $(0_X,0_Y)$ is an almost globally asymptotically stable equilibrium of \eqref{outer_theorem_time}-\eqref{inner_theorem_time}.
%\end{theorem}
%
%\begin{proof} Rough outline:
%	
%	\begin{enumerate}
%		\item for \textit{each} initial condition $y_0 \in N$, there is induced a nonautonomous semiflow on $M$.
%		\item if ${y_0 \in \mathcal{B}_Y}$, the semiflow is asymptotically autonomous with limiting semiflow induced by \eqref{unforced}. (need to prove) 
%		\item By Theorem 1.8 of AAS paper, the $\omega$-limit set under the AAF of each point is chain recurrent for the limiting semiflow, and the trajectories converge to the $\omega$-limit set, therefore to the chain recurrent set of the limiting flow
%		\item since the CRS of the limiting flow is solely hyperbolic equilibria, such points are isolated
%	\end{enumerate}
%\end{proof}


%We may also remark that the boundedness assumption of the previous theorem is automatically verified when $X$ is compact.

%Almost-Global Geometric Backstepping Control for Differentially Flat Systems: A Unified Approach

%
%\begin{corollary}
%	For cascades whose inner and unforced outer dynamics are gradient-like and for which the outer state space is compact, aGAS of the inner and unforced outer systems implies aGAS of the cascade.
%\end{corollary}
%

%
%\clearpage
%
%\section{old stuff!}
%\begin{remark}
%	The requirement that the interconnection term vanish at \textit{all} equilibria can be relaxed...
%\end{remark}
%
%\subsection{Cascades of Many Systems}
%
%\begin{theorem}[Upper Triangular Cascade of Nice Systems]
%	For compact manifolds without boundary $M_1,M_2,\ldots,M_n$, consider the system defined on their product given by
%	\begin{subequations}
%		\begin{align}
%			\dot{x}_1 &= f_1(x_1) + h_1(x_1,x_2,\ldots,x_n) \label{outermost} \\
%			\dot{x}_2 &= f_2(x_2) + h_2(x_2,\ldots,x_n) \\
%			& \ \, \vdots \nonumber \\ 
%%			\dot{x}_{n-1} &= f_{n-1}(x_{n-1}) + h(x_n) \\
%			\dot{x}_n &= f_n(x_n) \label{innermost}
%		\end{align}
%	\end{subequations} 
%	and suppose the vector fields have the following properties:
%	\begin{enumerate}
%		\item {\normalfont Niceness:} For $i = 1,\ldots,n$, the system $\dot{x}_i = f_i(x_i)$ is nice with equilibrium ${0_i \in X_i}$.
%		\item {\normalfont Vanishing Interconnections:} For $i = 1,\ldots,n-1$, we have
%	\begin{equation}
%	f_j(x_j) = 0 \, \forall \, i < j \leq n \implies
%	h_i(x_i,\ldots,x_{n}) = 0
%	\end{equation}
%	i.e. the interconnection term of each system vanishes over the set of simultaneous equilibria of the systems inward of that system.
%	\end{enumerate}
%	Then, the upper triangular cascade system \eqref{outermost}-\eqref{innermost} is nice with equilibrium $(0_1,0_2,\ldots,0_Y)$.
%\end{theorem}
%
%\begin{proof}
%	We prove the claim by induction. The claim is tautological, and hence true, in the case of a cascade of $n=1$ systems. Let us assume the claim is true for $k-1$ systems, and consider a cascade of $k$ systems. It can be readily seen that the last $k-1$ equations in the cascade, which govern the evolution of $x_2, \ldots, x_k$, define a cascade of $k-1$ systems with the stated properties, and therefore by the inductive assumption, this portion of the cascade together constitutes a system with state $y = (x_2,x_3,\ldots,x_k)$ which is nice with equilibrium $(0_2,0_3,\ldots,0_k)$. Thus, it follows immediately from Theorem \ref{simple_cascade} on a simple cascade of two systems that the full cascade of $k$ systems is nice with equilibrium $(0_1,0_2,\ldots,0_k)$. Hence the result follows for any $n$ via induction.
%\end{proof}
%
%%\begin{conjecture}
%%	Suppose all trajectories of the cascade  
%%	\begin{equation}
%%			\begin{aligned}
%%			\dot{x}_1 &= f_1(x_1,x_2) \\
%%			\dot{x}_2 &= f_2(x_2,x_3) \\
%%			& \ \, \vdots  \\
%%			\dot{x}_{n-1} &= f_{n-1}(x_{n-1},x_n) \\
%%			\dot{x}_n &= f_n(x_n)
%%			\label{multicascade}
%%		\end{aligned}
%%	\end{equation}
%%are bounded, and that ${0_Y \in X_n}$ is an almost-globally asymptotically stable equilibrium of the system ${\dot{x}_n = f_n(x_n)}$. Suppose also that for ${i = 1, \ldots, n - 1}$, each vector field ${\dot{x}_i = f_i(x_i,0_{i+1})}$, has an almost-globally asymptotically stable equilibrium ${0_i \in X_i}$ and a chain recurrent set consisting of only isolated hyperbolic equilibria. 
%%%. Suppose that all equilibria of ${\dot{x}_i = f_i(x_i,0_{i+1})}$
%%%
%%% all equilibria of these systems are hyperbolic, and moreover that the chain recurrent 
%%%
%%%Suppose that ${0_Y \in Y}$ and ${0_X \in X}$ are almost-globally asymptotically stable equilibria of the systems ${\dot{y} = g(y)}$ and ${\dot{x} = f(x,0_Y)}$ respectively, and suppose all respective equilibria of these systems are hyperbolic. 
%%%
%%%Finally, suppose that every chain recurrent point of $\dot{x} = f(x,0_Y)$ is an isolated equilibrium. 
%%Then, $(0_1,\ldots,0_Y)$ is an almost-globally asymptotically stable equilibrium of the cascade \eqref{multicascade}, and furthermore all equilibria of the cascade are hyperbolic.
%%
%%\end{conjecture}
%%\begin{proof}
%%	We prove the claim by induction. The claim is true by assumption when $n=1$.
%%\end{proof}
%%\clearpage
%%
%%Need to also present a theorem for boundedness of trajectories, perhaps using converse Lyapunov theorems.
%%
%%Need to connect to navigation functions.
%%
%%
%%Example: Use our theorem for backstepping control on Lie groups to obtain almost-global controllers. Likely also ISS, but still a good example and easier to prove.
%
%\clearpage
%
%\subsection{Growth Rate Conditions}
%
%Need to come up with some boundedness criteria based on interconnection term.
%
%\begin{conjecture}
%	Consider a cascade with the nice properties above. Suppose that the interconnection term satisfies the condition
%	$$\left|\left|\psi(x,y)\right|\right| \leq \gamma_1\big(\textrm{d}_Y(0_Y,y)\big)
%\textrm{d}_X(0_X,x)
%+
%\gamma_2\big(\textrm{d}_Y(0_Y,y)\big)
%$$
%where $\gamma_1$ and $\gamma_2$ are functions of a certain class.
%and there exists a Lyapunov function for $\dot{x} = f(x,0_Y)$ such that
%$$
%\left|\left|dV_x\right|\right|
%\textrm{d}_X(0_X,x)	\leq c \, V_x(x)
%$$
%Then, all trajectories are bounded. 
%\end{conjecture}
%
%or something like this!
%
%
%
%\section{Examples and Applications}
%
%\section{Conclusions}

%
%\balance 
\bibliographystyle{IEEEtran}
\bibliography{IEEEabrv,refs}




%\begin{appendix}
%\section*{Lyapunov Functions Constant on Components of Equilibria and Chain Recurrence}

\appendix

\section*{A. Chain Recurrence and Decreasing Functions}

In this appendix, we localize the chain recurrent set $\mathcal{R}(\Phi)$ of a continuous semiflow ${\Phi^t : M \rightarrow M}$ to a particular subset of the state space, with the aid of a function which is decreasing along all trajectories outside this subset. 
While similar results appear to be known \cite{Benaim1995}, we do not know of a reference providing these facts in our exact setting (e.g. for semiflows on possibly noncompact manifolds). 

%\textbf{(Is this true?)}

%https://mathworld.wolfram.com/DecreasingFunction.html

\begin{theorem}\label{th:R-subset-E}
	Assume there exists a proper continuous function ${V\colon M\to \R}$ and a subset ${S\subseteq M}$ such that $V(S)$ is nowhere dense in $\R$ and $V$ is decreasing on trajectories outside of $S$.
	Then ${\mathcal{R}(\Phi)\subseteq S}$.
\end{theorem}
\begin{proof}
	Fix any $x\not \in S$.
	Since $V(S)$ is nowhere dense and ${x\not  \in S}$, there exist $b>a>0$ such that 
	\begin{equation}
		\begin{gathered}
			[a,b]\subseteq V(\Phi^{[0,\infty)}(x)) \text{ and }
			\{a\leq V \leq b\}\cap S = \varnothing.
		\end{gathered}
	\end{equation}
	Since $V$ is decreasing along trajectories outside of $S$ and since $V$-sublevel sets are compact, this implies  the existence of $T>0$ such that $V(\Phi^T(x)) \leq a$ and $\Phi^{[T,\infty)}(\{V\leq b\})\subseteq \{V\leq a\}$.
	Compactness of $\{V\leq a\}$ also implies the existence of $\varepsilon > 0$ such that the distance between any point in $\{V\leq a\}$ and any point in $\{V \geq b\}$ is at least $2\varepsilon$. 
	By construction there does not exist an $(\varepsilon, T)$-chain from $x$ to $x$, so $x$ is not chain recurrent. 
	This completes the proof.
\end{proof}

\begin{corollary}\label{co:E-isolated-components}
	Assume there exists a proper continuous function ${V\colon M\to \R}$ and a subset ${S\subseteq M}$ such that $V$ is constant on each connected component of $S$, each connected component of $S$ is isolated, and  $V$ is decreasing on trajectories outside of $S$.
	Then ${\mathcal{R}(\Phi)\subseteq S}$.
\end{corollary}
\begin{proof}
	For each ${t\in \R}$, the set ${S_t\coloneqq S\cap \{V\leq t\}}$ is compact since $V$ is proper, so each $S_t$ has finitely many components since components of $S$ (hence also $S_t$) are isolated.
	This implies that ${V(S_{t+1}\setminus S_t)\subseteq (t,t+1]}$ is finite for each $t$.
	Thus, ${V(S)=\bigcup_{n=1}^{\infty}V(S_{n+1}\setminus S_n)}$ is not dense in any nonempty open subset of $\R$, so $V(S)$ is nowhere dense. 
	The desired result now follows from Theorem~\ref{th:R-subset-E}.
\end{proof}

The following fact is related to the analysis in \cite[Cor. 2.4]{Benaim1995}, however the setting of that work differs from our own, since it considers stochastic processes evolving on $\mathbb{R}^n$.

\begin{corollary}\label{co:equilibria}
		If $\mathcal{E}(\Phi)$ consists of isolated points and there is a proper, continuous function ${V:M\to \R}$ 
%		such that $V$ is 
that is
decreasing along nonequilibrium trajectories,
then ${\mathcal{R}(\Phi) = \mathcal{E}(\Phi)}$.
\end{corollary}
\begin{proof}
	Since $\mathcal{E}(\Phi) \subseteq \mathcal{R}(\Phi)$, it suffices to show that $\mathcal{R}(\Phi)\subseteq \mathcal{E}(\Phi)$.
	Since each connected component of ${S = \mathcal{E}(\Phi)}$ is a singleton, $V$ is automatically constant on each component of $S$, so the desired result follows from Corollary~\ref{co:E-isolated-components}.
\end{proof}

\begin{remark}
%	In this paper, the chain recurrent set is defined with respect to $(\varepsilon, T)$-chains using the given metric, rather than $(\mathcal{U},T)$-chains as in Conley's monograph \cite{Conley1978}, and where $\varepsilon$ is a constant rather than a positive function as in Hurley's work \cite{Hurley1992}.
	If the distance function on $M$ is replaced with a new one inducing the same topology, the conclusions of Theorem~\ref{th:R-subset-E} and Corollary~\ref{co:E-isolated-components} imply that the new chain recurrent set is still contained in $S$, and the conclusion of Corollary~\ref{co:equilibria} implies that the new chain recurrent set coincides with the old, i.e. $\mathcal{E}(\Phi)$. Moreover,  $\mathcal{R}(\Phi)$ is independent of the choice of Riemannian metric, since all smooth Riemannian metrics induce the same topology.
%	, the set of equilibria.
\end{remark}



\section*{B. Two Classes of Gradient-Like Systems}

%Our main result, Theorem \ref{almost_global_stability}, characterizes the stability of a class of cascades for which all chain recurrent points of the unforced outer loop are equilibria, a somewhat abstract property. In the Appendix, we present Theorem \ref{th:R-subset-E}, localizing the chain recurrent set of a dynamical system to a subset of the state space, provided there exists a function with certain technical properties which is decreasing along trajectories outside that subset. For convenience, we restate here the result from the Appendix that is most relevant to our present interests. This fact is related to the analysis in \cite[Cor. 2.4]{Benaim1995}, however the setting of that work differs from our own, since it considers stochastic processes evolving on $\mathbb{R}^n$.


		In this appendix, we consider two important classes of systems of particular relevance to geometric control design, which are already widely known to be almost globally asymptotically stable. With the aid of Corollary \ref{co:equilibria}, we verify the lesser-known fact that such systems have a chain recurrent set consisting solely of hyperbolic equilibria.
		The following facts are not particularly novel (see e.g. the discussion in \cite[Sec. IV]{Angeli2010} and \cite{Benaim1995}), but they are relevant to our larger interests, so we present them for completeness.
		
		
		\subsection{Gradient Systems}
		
		
		
		
		
		
		
		
		
		We first consider dynamical systems induced by descending the gradient of a \textit{Morse function} (i.e. a function whose critical points are all nondegenerate \cite{BulloAndLewis2004}) with a unique minimum. We note that Morse functions with unique minima, including ``perfect'' ones with the minimum possible number of critical points, are well-known for those manifolds typically encountered in geometric control \cite{Maithripala2006}. 
		
		%\textbf{Rephrase: Kod paper shows global limit behavior lifts to dissipative systems. We show recurrence behavior also lifts.}
		
		%\clearpage 
		
		%\textbf{Possibility: outsource the limit behavior arguments - keep them in the proposition, just cite the Kod paper for proof. Then, also prove the recurrence claim.}
		\begin{proposition}[Gradient System]
			\label{first_order_dynamics}
			For a Riemannian manifold ${(Q,\kappa)}$ and a proper Morse function ${V : Q \rightarrow \left[0,\infty\right)}$ 
			%	with global minimizer 
			with a unique minimum at 
			${0_Q \in Q}$, the dynamical system 
			%	on $Q$ given by
			\begin{equation}
				\dot{q} = - \mathrm{grad}_\kappa\, V(q) 
				%		- \kappa^\sharp \nu^\flat \dot{q} 
				\label{gradient_descent}
			\end{equation}
			is almost globally asymptotically stable and locally exponentially stable with respect to $0_Q$, and all chain recurrent points of \eqref{gradient_descent} are hyperbolic equilibria.
			%
			%	is gradient-like. Moreover, all of its equilibria are hyperbolic, and in particular the equilibrium ${0_{TQ} \in TQ}$, (the zero tangent vector at $0_Q$, the minimizer of $V$) is almost-globally asymptotically stable while all other equilibria are unstable.
		\end{proposition}
		
		\begin{proof}
			The almost global asymptotic stability of \eqref{gradient_descent} with respect to $0_Q$ is proved in \cite[Proposition 2.1]{Koditschek1989} for compact $Q$. However, the extension to the noncompact case is immediate since the sublevel sets of $V$ are compact and forward invariant, since by direct computation, 
			\begin{align}
				\dot{V} 
				&= dV(-\mathrm{grad}_\kappa V)
				= -\kappa\big(\mathrm{grad}_\kappa V, \mathrm{grad}_\kappa V \big) \leq 0.
			\end{align}
			Since the equilibria of \eqref{gradient_descent} are simply the critical points of $V$, the nondegeneracy of the critical points of Morse functions ensures hyperbolicity and therefore the local exponential stability of $0_Q$.
			%	is an immediate consequence of hyperbolicity
%			Thus, it only remains to show that all chain recurrent points must be equilibria. 
			%	exponential stability 
			%	It is clear that the equilibria of \eqref{gradient_descent} are the critical points of $V$.
			%	Since $V$ is Morse, it is readily verified that all these equilibria are hyperbolic \cite{BulloAndLewis2004} and therefore isolated, hence by the second countability of $Q$, there are countably many equilibria. 
			%	Moreover, only $0_Q$ is (locally exponentially) stable, since it is the unique minimizer of $V$ \cite{Koditschek1989}.
			Finally, since $V$ is decreasing on non-equilibrium trajectories and hyperbolic equilibria are isolated, Corollary \ref{co:equilibria} implies that the chain recurrent set of \eqref{gradient_descent} is exactly the set of equilibria.
			%	, and since hyperbolic equilibria are isolated, all trajectories converge to a particular hyperbolic equilibrium. On the other hand, the global stable manifold theorem asserts that the set of points converging to an unstable hyperbolic equilibrium is measure zero, and thus there exists a set of full measure converging to $0_Q$, proving almost global asymptotic stability.
		\end{proof}
		%\begin{proof}
		%	It is clear that the equilibria of \eqref{gradient_descent} are the critical points of $V$.
		%	Since $V$ is Morse, it is readily verified that all these equilibria are hyperbolic \cite{BulloAndLewis2004} and therefore isolated, hence by the second countability of $Q$, there are countably many equilibria. 
		%	Moreover, only $0_Q$ is (locally exponentially) stable, since it is the unique minimizer of $V$ \cite{Koditschek1989}.
		%	By direct computation we see that 
		%	\begin{align}
			%		\dot{V} 
			%		&= dV(-\mathrm{grad}_\kappa V)
			%		= -\kappa\big(\mathrm{grad}_\kappa V, \mathrm{grad}_\kappa V \big) \leq 0.
			%	\end{align}
		%	Since $V$ is proper non-increasing, all forward trajectories are precompact, so all trajectories converge to the chain recurrent set \cite{Mischaikow1995}. Moreover, since $V$ is decreasing on non-equilibrium trajectories, the chain recurrent set is exactly the set of equilibria	by Corollary \ref{co:equilibria}, and since hyperbolic equilibria are isolated, all trajectories converge to a particular hyperbolic equilibrium. On the other hand, the global stable manifold theorem asserts that the set of points converging to an unstable hyperbolic equilibrium is measure zero, and thus there exists a set of full measure converging to $0_Q$, proving almost global asymptotic stability.
		%\end{proof}
		
		
		
		%Discuss that the tangent bundle of a Riemannian manifold is Riemannian under e.g. the Sasaki metric?
		
		\subsection{Dissipative Mechanical Systems}
		
		We now turn our attention to the important class of dissipative mechanical systems arising from kinetic energy, potential energy, and damping. Such systems have been studied at length, since the introduction of artificial dissipation and potential shaping via feedback can result in closed loop dynamics of this form with desirable limit behavior. 
		We direct the reader to the seminal work \cite{Koditschek1989} which studies the global stability properties of such systems, as well as the more recent reference \cite[Chap. 6]{BulloAndLewis2004} which provides a comprehensive and detailed overview.
%		 
%		An early and influential analysis of the global stability properties of such systems can be found in \cite{Koditschek1989}, while a detailed and rigorous modern presentation of the same is found in \cite[Chap. 6]{BulloAndLewis2004}.
%		
		 Closed loop dynamics of this form have enabled trajectory tracking on arbitrary Lie groups \cite{Maithripala2006} and have also featured in the inner and outer loops of cascaded geometric controllers for underactuated robotic systems \cite{Lee2010,Sreenath2013}.
		
		
		
		
		
		
		%	Temporary list of relevant citations:
		%\begin{enumerate}
		%	\item Equilibria set is the critical set of $V$: \cite[Lemma 6.32]{BulloAndLewis2004}
		%	%		\item Linearization is stable at minima of $V$ and unstable at other critical points: \cite[Theorem 6.42]{BulloAndLewis2004}
		%	\item Actual system is unstable at other critical points of $V$: \cite[Proposition 6.44]{BulloAndLewis2004}
		%	\item System is locally asymptotically stable and locally exponentially stable around isolated local minima of $V$: \cite[Theorem 6.45]{BulloAndLewis2004} This is the one we want for stable one. 
		%	\item proper $V$ and finite critical points in the sublevels (which follows from properness and second countability) implies convergence to critical points of $V$: \cite[Theorem 6.47]{BulloAndLewis2004}
		%	\item \cite[Theorem 2]{Koditschek1989}
		%\end{enumerate}
		
		
		
		\begin{proposition}[Dissipative Mechanical System]
			\label{second_order_dynamics}
			For a Riemannian manifold ${(Q,\kappa)}$,
			a strict Rayleigh dissipation $\nu$, and a proper Morse function ${V : Q \rightarrow \left[0,\infty\right)}$ 
			%	with global minimizer 
			with a unique minimum at 
			${0_Q \in Q}$,
			the Euler-Lagrange dynamical system
%			\footnote{The maps ${\kappa^\flat, \nu^\flat : TQ \rightarrow T^*Q}$ and ${\kappa^\sharp, \nu^\sharp : T^*Q \rightarrow TQ}$ are the \textit{musical isomorphisms} with respect to the Riemannian metrics $\kappa$ and $\nu$ \cite{BulloAndLewis2004}.} 
			%	on $TQ$ given by
			\begin{equation}
				\overset{\kappa}{\nabla}_{\dot{q}} \dot{q} = - \mathrm{grad}_\kappa\, V(q) - \kappa^\sharp \circ \nu^\flat (\dot{q}) 
				\label{appendix:euler_lagrange}
			\end{equation}
			is almost globally asymptotically stable and locally exponentially stable with respect to ${0_{TQ} = (0_Q,0) \in TQ}$, and all chain recurrent points of \eqref{appendix:euler_lagrange} are hyperbolic equilibria.
			%
			%	is gradient-like. Moreover, all of its equilibria are hyperbolic, and in particular the equilibrium ${0_{TQ} \in TQ}$, (the zero tangent vector at $0_Q$, the minimizer of $V$) is almost-globally asymptotically stable while all other equilibria are unstable.
		\end{proposition}
		
		\begin{proof}
			It is clear that the equilibrium set of \eqref{appendix:euler_lagrange} is precisely the image of the critical points of $V$ in the zero section of $TQ$, and moreover these equilibria can be verified to be hyperbolic since $\nu$ is a strict linear dissipation and the critical points of a Morse function are nondegenerate. Moreover, only $0_{TQ}$ is (locally exponentially) stable, while all other equilibria are unstable, since $0_Q$ is the unique minimum of $V$.
			Considering the total energy function given by
			\begin{equation}
				\label{total_energy}
				W : (q,\dot{q}) \mapsto V(q) + \frac{1}{2}\kappa(\dot{q},\dot{q}),
			\end{equation}
			we compute
			\begin{align}
				\dot{W} 
				%		&= dE_{(q,\dot{q})}(\dot{q},\ddot{q}) \\ 
				&= dV(q) \dot{q}  + \kappa(	\overset{\kappa}{\nabla}_{\dot{q}} \dot{q},\dot{q}) \\
				&= dV(q) \dot{q}  + \kappa(	- \mathrm{grad}_\kappa\, V(q) - \kappa^\sharp \circ \nu^\flat (\dot{q})   \, , \, \dot{q} ) \\
				%	 &= dV(q) \dot{q}  - \kappa( {grad}_\kappa\, V(q) ,\dot{q}) 
				%	 - \kappa( \kappa^\sharp \nu^\flat \dot{q}    ,\dot{q}) \\
				&= dV(q) \dot{q}  - dV(q) \dot{q}  - \nu(\dot{q},\dot{q}) 
				%		\\ &= 
				= -\nu(\dot{q},\dot{q}) \leq 0.
			\end{align}	
			For any trajectory $t\mapsto q(t)$ of the Euler-Lagrange dynamics, \eqref{appendix:euler_lagrange} and strictness of $\nu$ imply that $\nu(\dot{q}(t),\dot{q}(t))>0$ for almost all $t$ if and only if the trajectory is nonequilibrium, so $W$ is decreasing along nonequilibrium trajectories. Thus, by Corollary \ref{co:equilibria}, the chain recurrent set of \eqref{appendix:euler_lagrange} is exactly the set of equilibria. Becuase $W$ is proper (since $V$ is proper and $\nu$ is positive definite) and nonincreasing along trajectories, all forward trajectories are precompact and therefore converge to the chain recurrent set \cite{Mischaikow1995}. Since hyperbolic equilibria are isolated, all trajectories converge to some equilibrium. Application of the global stable manifold theorem shows that almost no trajectories converge to an unstable hyperbolic equilibrium, so the unique stable equilibrium $0_{TQ}$ is almost globally asymptotically stable.
			%For compact $Q$, the almost global asymptotic stability of \eqref{euler_lagrange} with respect to ${(0_Q,0) \in TQ}$ is proved in \cite[Theorem 2]{Koditschek1989}, however we follow the more general treatment of  \cite{BulloAndLewis2004}. It is easily seen that the equilibrium set of \eqref{euler_lagrange} is simply the image of the critical points of $V$ in the zero section of $TQ$ \cite[Lemma 6.32]{BulloAndLewis2004}. Moreover, all such equilibria can be seen to be hyperbolic due to the nondegeneracy of the critical points of $V$ and the positive definiteness of $\nu$. Moreover, all are unstable except that associated with $0_Q$, since each is associated with a nondegenerate critical point of $V$ which is not a local minimum \cite[Proposition 6.44]{BulloAndLewis2004}. On the contrary, the equilibrium associated with $0_Q$ is locally exponentially stable \cite[Theorem 6.45]{BulloAndLewis2004} since $0_Q$ is an isolated local minimum of $V$. Since all trajectories of \eqref{euler_lagrange} converge to an equilibrium \cite[Theorem 6.47]{BulloAndLewis2004}, while all such equilibria are hyperbolic, the stable manifold theorem implies that almost all points converge to the stable equilibrium $0_{TQ}$.
			%	Similarly, the almost global asymptotic stability of \eqref{euler_lagrange} with respect to ${(0_Q,0) \in TQ}$ is proved in \cite[Theorem 2]{Koditschek1989} for compact $Q$, but the noncompact case follows immediately from the properness of $V$ (see \cite[Theorem 6.7]{BulloAndLewis2004} for further discussion). It is clear that the set of equilibria of \eqref{euler_lagrange} is simply the image of the critical points of $V$ in the zero section of $TQ$.  The hyperbolicity of equilibria, and therefore local exponential stability of $(0_Q,0)$, follows from [citation]. 
			%	Thus, all that remains is to show that all chain recurrent points are equilibria.
			%	\cite[Theorem 6.42]{BulloAndLewis2004} gives local asymptotic stability and local exponential stability
		\end{proof}
		
		\begin{remark}
			A primary contribution of \cite{Koditschek1989} is the observation that the global limit behavior of a dissipative mechanical system is essentially determined by the global limit behavior of the associated gradient system, which is often called the ``lifting property'' of dissipative mechanical systems \cite{BulloAndLewis2004}. Here, we have shown that a similar lifting property holds for these systems in regards to the chain recurrent set.
		\end{remark}


%\end{appendix}

%
%
%
%\clearpage
%
%
%\textbf{Probably remove this:}
%
%\subsection{Geometric Backstepping Control}
%
%In this section, we give some preliminary applications of our method of stability certification in a more general scenario.
%Consider a control system on $Q \times G$, where $Q$ is an arbitrary manifold and $G$ is a matrix Lie group, given by
%\begin{align}
%	\dot{x} &= f(x,g) \label{outer_backstepping}\\
%	\dot{g} &= g \, \xi \label{inner_backstepping}
%\end{align}
%where $\xi \in \mathfrak{g}$.
%Regarding ${g \in G}$ temporarily as a control input to \eqref{outer_backstepping}, suppose that the control ${k : Q \rightarrow G}$ renders the closed loop dynamics
%\begin{equation}
%	\dot{x} = f(x,k(x))
%\end{equation} 
%almost globally asymptotically stable with respect to some ${0_Q \in {Q}}$, with a chain recurrent set consisting solely of hyperbolic equilibria. For example, this will hold for any vector field on $Q$ of the form \eqref{gradient_descent}.
%Suppose also that the control ${p : G \rightarrow \mathfrak{g}}$ renders the identity element ${1_G \in G}$ almost globally asymptotically stable and locally exponentially stable for \eqref{inner_backstepping}.
%
%We now use a backstepping-inspired approach to construct a controller for \eqref{outer_backstepping}-\eqref{inner_backstepping}, whose stability can be certified by our main results. We introduce the change of variables
%\begin{equation}
%	y = 
%	k(x)^{-1} \, {g} 
%\end{equation}
%such that by direct computation,
%\begin{equation}
%	\dot{y} = 
%	-k(x)^{-1} 
%	\,dk\Big(f\big(x,k(x)\big)\,\Big) \, 
%	y + y \, \xi.
%\end{equation}
%%\begin{equation}
%%%	\dot{y} = -z^{-1} z u z^{-1} k(x) - z^{-1} dk(f(x,k(x)))
%%	\dot{y} = - u z^{-1} k(x) - z^{-1} dk(f(x,k(x))).
%%\end{equation}
%Proposing the control
%%\begin{equation}
%%	%	\dot{y} = -z^{-1} \dot{z} z^{-1} k(x) - z^{-1} dk(f(x),k(x))
%%	u = -z^{-1} \dot{y} k(x)^{-1}z - dk(f(x,k(x))) k(x)^{-1}z
%%\end{equation}
%\begin{equation}
%	%	\dot{y} = -z^{-1} \dot{z} z^{-1} k(x) - z^{-1} dk(f(x),k(x))
%	\xi =  p(y) + y^{-1} \, k(x)^{-1} \, dk\Big(f\big(x,k(x)\big)  \Big) \, y
%\end{equation}
%yields the dynamics
%\begin{align}
%	\dot{x} &= f(x,k(x) \, y) \label{outer_backstepping_closed_loop}\\
%	\dot{y} &= y \, p(y) \label{inner_backstepping_closed_loop}
%\end{align}
%and by our main result, this system is almost globally asymptotically stable and locally exponentially stable with respect to the equilibrium ${(0_Q,1_G)}$ as long as the trajectories of \eqref{outer_backstepping_closed_loop}-\eqref{inner_backstepping_closed_loop} beginning in ${Q \times \mathcal{B}_G}$ are precompact, where $\mathcal{B}_G$ is the basin of attraction of \eqref{inner_backstepping_closed_loop}. Given a particular system in the form \eqref{outer_backstepping} and a LaSalle function for the same, our second main result may verify that the interconnection term
%\begin{equation}
%	h(x,y) = f(x,k(x) \, y) - f(x,k(x)) 
%\end{equation}
%satisfies our growth rate condition and thereby certify precompactness. Analyzing a more specific class of dynamics of the form \eqref{outer_backstepping} could possibly lead to generalized conclusions in regards to the precompactness requirement.
%
%\textbf{This seems premature / too vague, since we can't verify the precompactness criteria. Save for next paper with flat systems, where we will be able to be more specific?}
%
%\ 
%
%\textbf{Other Idea: apply to connection-based locomotion (e.g. falling cat problem, kinematic connection, etc.)  on $G \times H$ where $H$ (the shape space) is a Lie group, with formal proof of almost global asymptotic stability for the complete system on $G \times \mathfrak{h}$, composing controller for $G$ and $\mathfrak{h}$, each of the form of Prop 2 and 1 respectively?}


%
%\iffalse
%
%
%\section{Chain Recurrence and Equilibria}
%
%Our main result, Theorem \ref{almost_global_stability}, characterizes the stability of a class of cascades for which all chain recurrent points of the unforced outer loop are equilibria, a somewhat abstract property.
%In the Appendix, we present Theorem \ref{th:R-subset-E}, localizing the chain recurrent set of a dynamical system to a subset of the state space, provided there exists a function with certain technical properties which is decreasing along trajectories outside that subset. For convenience, we restate here the result from the Appendix that is most relevant to our present interests. This fact is related to the analysis in \cite[Cor. 2.4]{Benaim1995}, however the setting of that work differs from our own, since it considers stochastic processes evolving on $\mathbb{R}^n$.
%
%\begin{restate_corollary}{2}
%	%	\begin{corollary}
%		%		\label{co:equilibria}
%		If all points in $\mathcal{E}(\Phi)$ are isolated and there exists a proper\footnote{A function ${V : M \rightarrow \mathbb{R}}$ is \textit{proper} if it has compact sublevel sets, which morally generalizes the notion of ``radially unbounded'' functions on $\mathbb{R}^n$.} continuous function ${V:M\to \R}$ 
%		%		such that $V$ is 
%		that is
%		decreasing\footnote{A function ${f : \mathbb{R} \rightarrow \mathbb{R}}$ is \textit{decreasing} if ${f(t_2) < f(t_1)}$ whenever ${t_1 < t_2 }$. Note that this does \textit{not} imply $\dot{f}(t) < 0$ for all $t$, e.g. $t \mapsto -t^3$.} along nonequilibrium trajectories,
%		then ${\mathcal{R}(\Phi) = \mathcal{E}(\Phi)}$.
%		%	\end{corollary
%		\end{restate_corollary}
%		
%		\begin{remark}
%			From Corollary \ref{co:equilibria}, it is clear that Theorem 1 also holds if its second condition is replaced by the assumption that for the system \eqref{unforced}, all equilibria are hyperbolic and there exists a Lyapunov function around $0_X$ which is decreasing along all non-equilibrium trajectories.\footnote{Some authors \cite{Benaim1995} call this a \textit{strict} Lyapunov function, but the control community tends to reserve this term for Lyapunov functions with strictly negative derivative along non-constant trajectories \cite{Santibanez1997}, a stronger condition.}
%		\end{remark}
%		
%		In the remainder of this section, we consider two important classes of systems of particular relevance to geometric control design, which are already widely known to be almost globally asymptotically stable. With the aid of Corollary \ref{co:equilibria}, we verify the lesser-known fact that such systems have a chain recurrent set consisting solely of hyperbolic equilibria.
%		The following facts are not particularly novel (see e.g. the discussion in \cite[Sec. IV]{Angeli2010} and \cite{Benaim1995}), but they are relevant to our larger discussion, so we provide them for completeness.
%		
%		
%		\subsection{Gradient Systems}
%		
%		
%		
%		
%		
%		
%		
%		
%		
%		We first consider dynamical systems induced by descending the gradient of a \textit{Morse function} (i.e. a function whose critical points are all nondegenerate \cite{BulloAndLewis2004}) with a unique minimum. We note that Morse functions with unique minima, including ``perfect'' ones with the minimum possible number of critical points, are well-known for those manifolds typically encountered in geometric control \cite{Maithripala2006}. 
%		
%		%\textbf{Rephrase: Kod paper shows global limit behavior lifts to dissipative systems. We show recurrence behavior also lifts.}
%		
%		%\clearpage 
%		
%		%\textbf{Possibility: outsource the limit behavior arguments - keep them in the proposition, just cite the Kod paper for proof. Then, also prove the recurrence claim.}
%		\begin{proposition}[Gradient System]
%			\label{first_order_dynamics}
%			For a Riemannian manifold ${(Q,\kappa)}$ and a proper Morse function ${V : Q \rightarrow \left[0,\infty\right)}$ 
%			%	with global minimizer 
%			with a unique minimum at 
%			${0_Q \in Q}$, the dynamical system 
%			%	on $Q$ given by
%			\begin{equation}
%				\dot{q} = - \mathrm{grad}_\kappa\, V(q) 
%				%		- \kappa^\sharp \nu^\flat \dot{q} 
%				\label{gradient_descent}
%			\end{equation}
%			is almost globally asymptotically stable and locally exponentially stable with respect to $0_Q$, and all chain recurrent points of \eqref{gradient_descent} are hyperbolic equilibria.
%			%
%			%	is gradient-like. Moreover, all of its equilibria are hyperbolic, and in particular the equilibrium ${0_{TQ} \in TQ}$, (the zero tangent vector at $0_Q$, the minimizer of $V$) is almost-globally asymptotically stable while all other equilibria are unstable.
%		\end{proposition}
%		
%		\begin{proof}
%			The almost global asymptotic stability of \eqref{gradient_descent} with respect to $0_Q$ is proved in \cite[Proposition 2.1]{Koditschek1989} for compact $Q$. However, the extension to the noncompact case is immediate since the sublevel sets of $V$ are compact and forward invariant, since by direct computation, 
%			\begin{align}
%				\dot{V} 
%				&= dV(-\mathrm{grad}_\kappa V)
%				= -\kappa\big(\mathrm{grad}_\kappa V, \mathrm{grad}_\kappa V \big) \leq 0.
%			\end{align}
%			Since the equilibria of \eqref{gradient_descent} are simply the critical points of $V$, the nondegeneracy of the critical points of Morse functions ensures hyperbolicity and therefore the local exponential stability of $0_Q$.
%			%	is an immediate consequence of hyperbolicity
%			Thus, it only remains to show that all chain recurrent points must be equilibria. 
%			%	exponential stability 
%			%	It is clear that the equilibria of \eqref{gradient_descent} are the critical points of $V$.
%			%	Since $V$ is Morse, it is readily verified that all these equilibria are hyperbolic \cite{BulloAndLewis2004} and therefore isolated, hence by the second countability of $Q$, there are countably many equilibria. 
%			%	Moreover, only $0_Q$ is (locally exponentially) stable, since it is the unique minimizer of $V$ \cite{Koditschek1989}.
%			Since $V$ is decreasing on non-equilibrium trajectories and hyperbolic equilibria are isolated, Corollary \ref{co:equilibria} implies that the chain recurrent set of \eqref{gradient_descent} is exactly the set of equilibria.
%			%	, and since hyperbolic equilibria are isolated, all trajectories converge to a particular hyperbolic equilibrium. On the other hand, the global stable manifold theorem asserts that the set of points converging to an unstable hyperbolic equilibrium is measure zero, and thus there exists a set of full measure converging to $0_Q$, proving almost global asymptotic stability.
%		\end{proof}
%		%\begin{proof}
%		%	It is clear that the equilibria of \eqref{gradient_descent} are the critical points of $V$.
%		%	Since $V$ is Morse, it is readily verified that all these equilibria are hyperbolic \cite{BulloAndLewis2004} and therefore isolated, hence by the second countability of $Q$, there are countably many equilibria. 
%		%	Moreover, only $0_Q$ is (locally exponentially) stable, since it is the unique minimizer of $V$ \cite{Koditschek1989}.
%		%	By direct computation we see that 
%		%	\begin{align}
%			%		\dot{V} 
%			%		&= dV(-\mathrm{grad}_\kappa V)
%			%		= -\kappa\big(\mathrm{grad}_\kappa V, \mathrm{grad}_\kappa V \big) \leq 0.
%			%	\end{align}
%		%	Since $V$ is proper non-increasing, all forward trajectories are precompact, so all trajectories converge to the chain recurrent set \cite{Mischaikow1995}. Moreover, since $V$ is decreasing on non-equilibrium trajectories, the chain recurrent set is exactly the set of equilibria	by Corollary \ref{co:equilibria}, and since hyperbolic equilibria are isolated, all trajectories converge to a particular hyperbolic equilibrium. On the other hand, the global stable manifold theorem asserts that the set of points converging to an unstable hyperbolic equilibrium is measure zero, and thus there exists a set of full measure converging to $0_Q$, proving almost global asymptotic stability.
%		%\end{proof}
%		
%		
%		
%		%Discuss that the tangent bundle of a Riemannian manifold is Riemannian under e.g. the Sasaki metric?
%		
%		\subsection{Dissipative Mechanical Systems}
%		
%		We now turn our attention to the important class of dissipative mechanical systems arising from a kinetic energy, potential energy, and damping. Such systems have been studied at length, since artificial potential shaping via feedback can result in closed loop dynamics of this form with desirable limit behavior. An early and influential analysis of the global stability properties of such systems can be found in \cite{Koditschek1989}, while a very detailed and rigorous modern presentation of the same is found in \cite[Chap. 6]{BulloAndLewis2004}. In fact, closed loop error dynamics of this form have enabled trajectory tracking on arbitrary Lie groups \cite{Maithripala2006} and have also featured in the inner and outer loops of cascaded geometric controllers for underactuated robotic systems \cite{Lee2010,Sreenath2013}.
%		
%		
%		
%		
%		
%		
%		%	Temporary list of relevant citations:
%		%\begin{enumerate}
%		%	\item Equilibria set is the critical set of $V$: \cite[Lemma 6.32]{BulloAndLewis2004}
%		%	%		\item Linearization is stable at minima of $V$ and unstable at other critical points: \cite[Theorem 6.42]{BulloAndLewis2004}
%		%	\item Actual system is unstable at other critical points of $V$: \cite[Proposition 6.44]{BulloAndLewis2004}
%		%	\item System is locally asymptotically stable and locally exponentially stable around isolated local minima of $V$: \cite[Theorem 6.45]{BulloAndLewis2004} This is the one we want for stable one. 
%		%	\item proper $V$ and finite critical points in the sublevels (which follows from properness and second countability) implies convergence to critical points of $V$: \cite[Theorem 6.47]{BulloAndLewis2004}
%		%	\item \cite[Theorem 2]{Koditschek1989}
%		%\end{enumerate}
%		
%		
%		
%		\begin{proposition}[Dissipative Mechanical System]
%			\label{second_order_dynamics}
%			For a Riemannian manifold ${(Q,\kappa)}$,
%			a strict Rayleigh dissipation $\nu$, and a proper Morse function ${V : Q \rightarrow \left[0,\infty\right)}$ 
%			%	with global minimizer 
%			with a unique minimum at 
%			${0_Q \in Q}$,
%			the Euler-Lagrange dynamical system\footnote{The maps ${\kappa^\flat, \nu^\flat : TQ \rightarrow T^*Q}$ and ${\kappa^\sharp, \nu^\sharp : T^*Q \rightarrow TQ}$ are the \textit{musical isomorphisms} with respect to the Riemannian metrics $\kappa$ and $\nu$ \cite{BulloAndLewis2004}.} 
%			%	on $TQ$ given by
%			\begin{equation}
%				\overset{\kappa}{\nabla}_{\dot{q}} \dot{q} = - \mathrm{grad}_\kappa\, V(q) - \kappa^\sharp \circ \nu^\flat (\dot{q}) 
%				\label{euler_lagrange}
%			\end{equation}
%			is almost globally asymptotically stable and locally exponentially stable with respect to ${0_{TQ} = (0_Q,0) \in TQ}$, and all chain recurrent points of \eqref{euler_lagrange} are hyperbolic equilibria.
%			%
%			%	is gradient-like. Moreover, all of its equilibria are hyperbolic, and in particular the equilibrium ${0_{TQ} \in TQ}$, (the zero tangent vector at $0_Q$, the minimizer of $V$) is almost-globally asymptotically stable while all other equilibria are unstable.
%		\end{proposition}
%		
%		\begin{proof}
%			It is clear that the equilibrium set of \eqref{euler_lagrange} is precisely the image of the critical points of $V$ in the zero section of $TQ$, and moreover these equilibria can be verified to be hyperbolic since $\nu$ is a strict dissipation and the critical points of a Morse function are nondegenerate. Moreover, only $0_{TQ}$ is (locally exponentially) stable, while all other equilibria are unstable, since $0_Q$ is the unique minimum of $V$.
%			Considering the total energy function given by
%			\begin{equation}
%				\label{total_energy}
%				W : (q,\dot{q}) \mapsto V(q) + \frac{1}{2}\kappa(\dot{q},\dot{q}),
%			\end{equation}
%			we compute
%			\begin{align}
%				\dot{W} 
%				%		&= dE_{(q,\dot{q})}(\dot{q},\ddot{q}) \\ 
%				&= dV(q) \dot{q}  + \kappa(	\overset{\kappa}{\nabla}_{\dot{q}} \dot{q},\dot{q}) \\
%				&= dV(q) \dot{q}  + \kappa(	- \mathrm{grad}_\kappa\, V(q) - \kappa^\sharp \circ \nu^\flat (\dot{q})   \, , \, \dot{q} ) \\
%				%	 &= dV(q) \dot{q}  - \kappa( {grad}_\kappa\, V(q) ,\dot{q}) 
%				%	 - \kappa( \kappa^\sharp \nu^\flat \dot{q}    ,\dot{q}) \\
%				&= dV(q) \dot{q}  - dV(q) \dot{q}  - \nu(\dot{q},\dot{q}) 
%				%		\\ &= 
%				= -\nu(\dot{q},\dot{q}) \leq 0.
%			\end{align}	
%			For any trajectory $t\mapsto q(t)$ of the Euler-Lagrange dynamics, \eqref{euler_lagrange} and nondegeneracy of $\nu$ imply that $\nu(\dot{q}(t),\dot{q}(t))>0$ for almost all $t$ if and only if the trajectory is nonequilibrium, so $W$ is decreasing along nonequilibrium trajectories. Thus, by Corollary \ref{co:equilibria}, the chain recurrent set of \eqref{euler_lagrange} is exactly the set of equilibria. Becuase $W$ is proper (since $V$ is proper and $\nu$ is positive definite) and nonincreasing along trajectories, all forward trajectories are precompact and therefore converge to the chain recurrent set \cite{Mischaikow1995}. Since hyperbolic equilibria are isolated, all trajectories converge to some equilibrium. Application of the global stable manifold theorem shows that almost no trajectories converge to an unstable hyperbolic equilibrium, so the unique stable equilibrium $0_{TQ}$ is almost globally asymptotically stable.
%			%For compact $Q$, the almost global asymptotic stability of \eqref{euler_lagrange} with respect to ${(0_Q,0) \in TQ}$ is proved in \cite[Theorem 2]{Koditschek1989}, however we follow the more general treatment of  \cite{BulloAndLewis2004}. It is easily seen that the equilibrium set of \eqref{euler_lagrange} is simply the image of the critical points of $V$ in the zero section of $TQ$ \cite[Lemma 6.32]{BulloAndLewis2004}. Moreover, all such equilibria can be seen to be hyperbolic due to the nondegeneracy of the critical points of $V$ and the positive definiteness of $\nu$. Moreover, all are unstable except that associated with $0_Q$, since each is associated with a nondegenerate critical point of $V$ which is not a local minimum \cite[Proposition 6.44]{BulloAndLewis2004}. On the contrary, the equilibrium associated with $0_Q$ is locally exponentially stable \cite[Theorem 6.45]{BulloAndLewis2004} since $0_Q$ is an isolated local minimum of $V$. Since all trajectories of \eqref{euler_lagrange} converge to an equilibrium \cite[Theorem 6.47]{BulloAndLewis2004}, while all such equilibria are hyperbolic, the stable manifold theorem implies that almost all points converge to the stable equilibrium $0_{TQ}$.
%			%	Similarly, the almost global asymptotic stability of \eqref{euler_lagrange} with respect to ${(0_Q,0) \in TQ}$ is proved in \cite[Theorem 2]{Koditschek1989} for compact $Q$, but the noncompact case follows immediately from the properness of $V$ (see \cite[Theorem 6.7]{BulloAndLewis2004} for further discussion). It is clear that the set of equilibria of \eqref{euler_lagrange} is simply the image of the critical points of $V$ in the zero section of $TQ$.  The hyperbolicity of equilibria, and therefore local exponential stability of $(0_Q,0)$, follows from [citation]. 
%			%	Thus, all that remains is to show that all chain recurrent points are equilibria.
%			%	\cite[Theorem 6.42]{BulloAndLewis2004} gives local asymptotic stability and local exponential stability
%		\end{proof}
%		
%		\begin{remark}
%			A primary contribution of \cite{Koditschek1989} is the observation that the global limit behavior of a dissipative mechanical system is essentially determined by the global limit behavior of the associated gradient system, which is often called the ``lifting property'' of dissipative mechanical systems \cite{BulloAndLewis2004}. Here, we have shown that a similar lifting property holds for these systems in regards to the chain recurrent set.
%		\end{remark}
%
%
%
%
%
%\fi

\end{document}

