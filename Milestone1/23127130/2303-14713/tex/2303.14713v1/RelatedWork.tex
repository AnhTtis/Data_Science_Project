\section{Related Survey-based and Empirical Studies}



We now review the relevant research rooted in survey-based evidence or empirical studies on (i) engineering and architecting quantum software (Section \ref{Related:QSE}), and (ii) quantum services computing (Section \ref{Related:QAAS}), as in Table \ref{tab:RelatedWork}. % can help position the scope and contributions of the proposed SMS.

\textbf{Interpretation of the Review}: To compare and summarise objectively, Table \ref{tab:RelatedWork} highlights each study using five-point self-explanatory criteria including (a) type of study (adopted from \cite{R16_petersen2008systematic}), (b) focus of study, (c) core findings, (d) QSE activity supported by the study, and (e) year of publication, each exemplified below. For example, the study \cite{X2_QSA} presents a systematic review, available since 2021, as part of evidence-based software engineering to focus on architecting quantum software. It presents the core findings about architectural life cycle and state-of-the-art on architectural modeling, patterns, and tools to architects develop quantum software. %%It mainly concerns with design and architecting activities of QSE life cycle and indicates the needs for future research on quantum-specific professional expertise (human roles in QSE) to develop quantum software. 

\subsection{Architecting and Engineering Quantum Software}\label{Related:QSE}
Quantum Software Engineering (QSE) has emerged as the most recent genre of software engineering (SE) that allows practitioners to adopt a process-centric and systematic approach develop software-intensive systems and applications for QC \cite{X1_QSE, X5_QC}. Some recently published survey-based studies on QSE highlight that existing SE principles and practices can be tailored to develop quantum software, however, issues specific to QC, such as operationalising QuBits/QuGates, quantum domain engineering, and quantum-classic software split, require new engineering methods to tackle these challenges \cite{X5_QC}. To derive new methods and processes, the QSE research community is striving to organize a body of knowledge and academic collaborations that can be stimulated via dedicated publication forums. Academic discussions at the publication fora and published results \cite{R1_ali2022software}\cite{R4_dyakonov2019will} have highlighted that in addition to the engineering principle and practices, human knowledge and expertise require fundamental knowledge of quantum mechanics to effectively design algorithmic solutions for QSE projects \cite{R15_de2022software}. The current generation of software architects and developers who lack the foundational knowledge of quantum mechanics may be hindered or find themselves under-prepared for quantum software development \cite{X2_QSA}. Software architecture is being viewed as a solution that can abstract complexities of implementation by representing software to be developed as architectural components and connectors. A recently conducted systematic review of architectural solutions puts forward five architecting activities to guide software designers to engineer quantum software solutions in an incremental manner \cite{X2_QSA}. %Architectural solution(s) can also empower quantum software engineers to exploit architectural knowledge in terms of patterns (i.e., reuse knowledge), modelling notations (i.e., software representation), and tools (i.e., automation) to address the challenges of QSE. In addition to architecting, there is a need for empirically grounded guidelines and solutions to address the aspects of quantum software programming, model-driven development, and quality assurance in QSE \cite{x4_Issues}. %To ameliorate this gap there is a need for reference architectures, patterns, processes, and empirical guidelines to systemise quantum software development.  


\subsection{Quantum Services Computing}\label{Related:QAAS}
Quantum service-orientation is a recent trend initially pushed by QC vendors to allow developers who can compose and invoke software services on remotely hosted quantum systems and infrastructures \cite{R11_leymann2020quantum}. Quantum service computing initiatives, such as Amazon Braket and Q experience, have paved the way for academic research to propose solutions for quantum cloud, quantum services computing, and quantum as a service. Recent research reviews inform about the current progress and emerging challenges for quantum services computing that include but are not limited to hardware availability, quantum noise, and quantum-classic split \cite{x7_QAAS}. Software researchers are striving to synergise existing methods of microservicing and quantum software development such as Quantum DevOps to develop quantum microservices \cite{R23_khan2022agile}. %Despite the potential for quantum service-orientation, existing research lacks a systematic process, patterns, frameworks and reference architectures that can guide developers to implement QCaaS solutions. 


