\section{Conclusions and Future Work}\label{conclusions}
Quantum service orientation allows QC vendors to offer end-users’ access to computational resources - enabling shots at QPUs - by following the utility computing model. Recently, research and development on QSE have started to synergize the principles of service-orientation and practices of QC (algorithms, simulations, QuGates etc.) to enable the adoption of QCaaS by individuals and enterprises to attain quantum supremacy in modern day computing. %To support sustained research and systematic development of QCaaS systems, there is a need to investigate existing solutions and emerging trends of research on quantum services computing. 
To this end, we used the systematic mapping study approach to investigate (i) existing solutions for quantum service development that enable or enhance QCaaS, and (ii) emerging trends that highlight the needs for ongoing and future research on QCaaS. The results of this SMS as structured information (Table \ref{tab:criteria}) and visual illustrations (Figure \ref{Fig-1:Context}-\ref{Results}) highlight the strengths, limitations, and potential for future research from the QSE perspective.  

\textbf{Implications and Future Work}: The primary implication of this SMS is to establish an evidence-based body of knowledge for service development that leverages design notations, patterns, and architectural approaches highlighting the needs for human-centric and process-driven QCaaS development. The results of this SMS establish the foundations for future work in three directions that include (a) \textit{conducting practitioners' survey}, (b) \textit{implementing the reference architecture}, and (c) \textit{mining social coding platforms} for empiricism in QCaaS computing. The literary foundations can help us to design an empirical study that aims to engage service developers and engineers in a workshop (activity and survey) to seek their feedback and synthesise the results as practitioners' perspectives to complement the evidence from academic research. We also plan to mine social coding platforms (e.g., GitHub) to empirically discover knowledge and understand the practices adopted by developers' communities in open-source QCaaS.

{\renewcommand{\arraystretch}{1}
\begin{table}[h]
\centering
\tiny
\caption{List of the selected studies of this SMS}
\label{tab:SelectedStudies}
\begin{tabular}{|c|l|c|c|}
\hline
\rowcolor[HTML]{DAE8FC} 
\textbf{\begin{tabular}[c]{@{}c@{}}Study\\ ID\end{tabular}} &
  \multicolumn{1}{c|}{\cellcolor[HTML]{DAE8FC}\textbf{\begin{tabular}[c]{@{}c@{}}Authors, Title \\ and Venue\end{tabular}}} &
  \textbf{\begin{tabular}[c]{@{}c@{}}Publication\\ Year\end{tabular}} &
  \textbf{\begin{tabular}[c]{@{}c@{}}Quality\\ Score\end{tabular}} \\ \hline
{[}S1{]} &
  \begin{tabular}[c]{@{}l@{}}G. Jose, J. Rojo, D. Valencia, E. Moguel, J. Berrocal,\\    and Juan. Murillo. \textsf{Quantum Software as a Service}\\  \textsf{through a Quantum API Gateway. \textit{IEEE Internet Computing}}\end{tabular} &
  2021 &
  4.0 \\ \hline
{[}S2{]} &
  \begin{tabular}[c]{@{}l@{}}Kumara, Indika, Willem-Jan Van Den Heuvel, and \\ Damian A. Tamburri. QSOC: \textsf{Quantum  service}\\ \textsf{oriented computing. \textit{SummerSOC}}\end{tabular} &
  2021 &
  3.0 \\ \hline
{[}S3{]} &
  \begin{tabular}[c]{@{}l@{}}Nguyen, Hoa T., Muhammad Usman, and Rajkumar \\ Buyya. \textsf{QFaaS: A Serverless Function-as-a-Service} \\ \textsf{Framework for Quantum Computing. \textit{arXiv}}\end{tabular} &
  2022 &
  5.0 \\ \hline
{[}S4{]} &
  \begin{tabular}[c]{@{}l@{}}Moguel, Enrique, Javier Rojo, David Valencia, Javier \\ Berrocal, Jose Garcia-Alonso, and Juan M. Murillo.\\ \textsf{Quantum service-oriented computing: current landscape}\\  \textsf{and challenges. \textit{Software Quality Journal}}\end{tabular} &
  2022 &
  4.0 \\ \hline
{[}S5{]} &
  \begin{tabular}[c]{@{}l@{}}Rojo, Javier, David Valencia, Javier Berrocal, Enrique \\ Moguel, Jose Garcia-Alonso, and Juan Manuel Murillo \\ Rodriguez. \textsf{Trials and tribulations of developing hybrid}\\ \textsf{quantum-classical microservices systems. \textit{Q-SET}}\end{tabular} &
  2021 &
  3.5 \\ \hline
{[}S6{]} &
  \begin{tabular}[c]{@{}l@{}}De Stefano, Manuel, Dario Di Nucci, Fabio Palomba, \\ Davide Taibi, and Andrea De Lucia. \textsf{Towards Quantum}\\ \textsf{algorithms-as-a-service. \textit{QP4SE}}\end{tabular} &
  2022 &
  2.5 \\ \hline
{[}S7{]} &
  \begin{tabular}[c]{@{}l@{}}Valencia, David, Enrique Moguel, Javier Rojo, Javier \\ Berrocal, Jose Garcia-Alonso, and Juan M. Murillo. \\ \textsf{Quantum Service-Oriented Architectures: From Hybrid}\\ \textsf{Classical Approaches to Future Stand-Alone Solutions.}\\ \textit{Quantum Software Engineering}\end{tabular} &
  2022 &
  3.5 \\ \hline
{[}S8{]} &
  \begin{tabular}[c]{@{}l@{}}Barzen, Johanna, Frank Leymann, Michael Falkenthal, \\ Daniel Vietz, Benjamin Weder, and Karoline Wild. \\ \textsf{Relevance of near-term quantum computing in the} \\ \textsf{cloud: A humanities perspective. \textit{CLOSER}}\end{tabular} &
  2021 &
  4.0 \\ \hline
{[}S9{]} &
  \begin{tabular}[c]{@{}l@{}}Valencia, David, Jose Garcia-Alonso, Javier Rojo, \\ Enrique Moguel, Javier Berrocal, and Juan \\ Manuel Murillo. \textsf{Hybrid classical-quantum} \\ \textsf{software services systems: Exploration of the}\\ \textsf{rough edges.} \textit{Quality of Information and} \\ \textit{Communications Technology}\end{tabular} &
  2021 &
  3.5 \\ \hline
\end{tabular}
\end{table}}

