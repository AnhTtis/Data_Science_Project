\section{Emerging Trends of Research on QaaS (RQ-2)}
\label{sec:RQ2}
We now answer RQ-2 that aims to report the emerging trends that indicate dimensions of potential future research on QCaaS. To identify these trends, we specifically reviewed the details about objectives/contributions, evaluation/demonstrations, and limitations/future research in each study during quality assessment (Q-1, Q-4, Q-5 in Table \ref{tab:qualitycriteria}). We have visualised the identified emerging trends in Figure \ref{FutureResearch} as per the SOA life cycle activities. For example, Figure \ref{FutureResearch} highlights that during the conception of quantum services, Quantum Significant Requirements (QSRs) should include quality aspects that complement the functional aspects to ensure the required functionality and desired quality of QCaaS as part of quantum domain engineering activity in QSE. %Functionality and quality requirements as part of quantum domain engineering could help derive a model that can enable model-driven development, i.e., modeling requirements into the design of quantum services. 
%The trends highlighted here only represents the evidence gathered from reviewed studies that may not be comprehensive, however, it is to be viewed as the basis to hypothesise future research challenges. 


%%\scriptsize{\faHandORight ~\textsf{\textbf{Process-centred approaches} such as Quantum DevOps, Quantum microservicing or agile methods can enable incremental development and delivery of quantum software services. \textbf{Human roles} can enrich the process (synergising knowledge of quantum physics practices of software engineering) to tackle activities such as quantum domain engineering, software architecting, and simulation management - reflecting expertise that currently lack in QaaS development.}}


%%\faHandORight \scriptsize{~\textsf{The notion of \textbf{QSRs} is rooted in the concept of ASRs that aims to identify, classify and document functional and quality aspects of quantum software services. QSRs such as QuBit utilisation, energy efficiency, vendor virtualisation etc. represent quality aspects that can complement the functionality of a quantum service.}}


\subsection{Process and Human Roles in QCaaS Development} 
Process-centred engineering is manifested in service life cycle to structure a multitude of activities, such as service design, development, and delivery, into a coherent process that can be executed in an incremental fashion \cite{R19_ahmadtowards}. Although the evidence from the selected studies is accumulated and organized under SOA life cycle in Table \ref{tab:criteria}, however, at the individual scale, the studies lacked a process-centric approach and highlighted the needs for processes attuned to the needs of QSE \cite{R19_ahmadtowards}. These processes, such as Quantum DevOps, Quantum  microservicing or agile methods for quantum service/software etc., are seen as quantum-genre of SE processes that are better aligned with the needs of QSE \cite{R23_khan2022agile}. For example, in the agile method for QSE, quantum domain engineering can help accumulate the domain information of a quantum system (hardware/software) to develop a design model that can act as a blueprint to implement quantum software and services. Process-centric approaches can also support tools (for automation) and professional roles (human decision support) to engineer quantum services. We identified the need for human roles as QSE professionals to effectively undertake QCaaS development.

\faComment{} 
{~\footnotesize\textsf{{\textbf{Process-centred} approaches, such as Quantum DevOps, Quantum microservicing or agile methods, can enable an incremental development and delivery of quantum software services. \textbf{Human roles} can enrich the process - synergising knowledge of quantum physics practices of software engineering - to support activities of quantum domain engineering, software architecting, and simulation management as expertise that currently lacks in quantum service-orientation.}}}   


\subsection{Quantum Significant Requirements}   
The concept of Quantum Significant Requirements (QSRs) can be traced to well-established role of Architecturally Significant Requirements (ASRs) in software and service engineering. QSRs as part of the quantum domain engineering activity can help elicit and document the functional and non-functional aspects (also quality aspects or attributes) of quantum software. Existing research primarily focuses on functional aspects of services while overlooking the non-functional aspects that include but are not limited to QuBit utilisation, energy efficiency, quantum vendor lock-ins, QPU elasticity, quantum error mitigation, that impact the development and operationalisation or QCaaS solutions. For example, during the quantum domain engineering activity the hardware aspects (operations of QuGates) can be mapped to software aspects (components and connectors) that can help with splitting the computation tasks between classical and quantum computers using the classic-quantum split pattern. 

\faComment{} 
{~\footnotesize\textsf{The notion of {\textbf{QSRs} is rooted in the concept of ASRs that aims to identify, classify, and document functional and quality aspects of quantum software services. QSRs for quantum services can complement the functional aspects with quality requirements to ensure the required functionality and desired quality of service.}}} 

\subsection{Model-driven Quantum Software Servicing}
Model-driven Service Engineering (MDSE) enables software engineers and architects to rely on models that can abstract complex and implementation-specific details with human comprehensible visual notations to design and implement software services. Specifically, by exploiting MDSE, software engineers can apply the model transformation to transform the design models into implementation (source code) and validation (test case) models. Modeling notations, such as Service-oriented architecture Modeling Language (SoaML) and more specifically the Q-UML specification, provide a metamodel and a UML profile for the specification and design of services within a service-oriented architecture. MDSE can benefit novice developers to to map the flow of algorithms and modules of source code to graphical models for low-code (model-driven) development of quantum services.

\faComment{}
{~\footnotesize\textsf{\textbf{Model-driven Quantum Service Development} can leverage the principle of MDSE and modeling notations like SoaML and Q-UML to abstract implementation-level complexities with design-level models driven by QSRs. To exploit model-driven development, there is a need for tool support that could enable automation (e.g., model transformation tools) and human decision support (e.g., quantum software architects) to quantum software servicing using MDSE.}}

\begin{figure}[]
 \centering
 \includegraphics[scale=0.7]{Images/Future.pdf} 
 	\caption{Overview of Emerging Research Trends}
	\label{FutureResearch}
\end{figure}

%%\faHandORight{\scriptsize{\textsf{~Model-driven Quantum Service Development can leverage the principle of MDSE and modelling notations like SoaML to abstract implementation-level complexities with design-level models driven by QSRs. However, to exploit model-driven development, there is a need for tool support that could enable automation and human roles that can guide the process to generate models.}}}

\begin{table*}[hbt!]
\caption{A Review of the Relevant Secondary Studies}
\begin{centering}
{\tiny
\begin{tabular}{|lcclcl|}
\hline
\rowcolor[HTML]{DAE8FC} 
\multicolumn{6}{|c|}{\cellcolor[HTML]{DAE8FC}\textbf{Quantum Software Engineering}} \\ \hline
\rowcolor[HTML]{EFEFEF} 
\multicolumn{1}{|l|}{\cellcolor[HTML]{EFEFEF}\textit{\begin{tabular}[c]{@{}l@{}}Study \\ Reference\end{tabular}}} & \multicolumn{1}{c|}{\cellcolor[HTML]{EFEFEF}\textit{\begin{tabular}[c]{@{}c@{}}Type of \\ Study  \end{tabular}}} & \multicolumn{1}{c|}{\cellcolor[HTML]{EFEFEF}\textit{\begin{tabular}[c]{@{}c@{}}Focus of \\ Study\end{tabular}}} & \multicolumn{1}{c|}{\cellcolor[HTML]{EFEFEF}\textit{\begin{tabular}[c]{@{}c@{}}Core\\ Findings\end{tabular}}} & \multicolumn{1}{l|}{\cellcolor[HTML]{EFEFEF}\begin{tabular}[c]{@{}l@{}}QSE\\ Activity\end{tabular}} & \begin{tabular}[c]{@{}l@{}}Publication \\ Year\end{tabular} \\ \hline
\multicolumn{1}{|l|}{\cite{X1_QSE}} & \multicolumn{1}{c|}{\begin{tabular}[c]{@{}c@{}}Literature \\ Review\end{tabular}} & \multicolumn{1}{c|}{\begin{tabular}[c]{@{}c@{}}Quantum \\ Software \\ Engineering\end{tabular}} & \multicolumn{1}{l|}{\begin{tabular}[c]{@{}l@{}}QSE life cycle, quantum software \\ engineering processes, methods, and tools.\end{tabular}} & \multicolumn{1}{c|}{\begin{tabular}[c]{@{}c@{}}Software  \\ Life cycle\end{tabular}} & 2020 \\ \hline
\multicolumn{1}{|l|}{\cite{X2_QSA}} & \multicolumn{1}{c|}{\begin{tabular}[c]{@{}c@{}}Systematic \\ Review\end{tabular}} & \multicolumn{1}{c|}{\begin{tabular}[c]{@{}c@{}}Quantum\\ Software \\ Architecture\end{tabular}} & \multicolumn{1}{l|}{\begin{tabular}[c]{@{}l@{}}Quantum software architecting activities, \\ modelling notations, patterns, and tools \\ for architectural development\end{tabular}} & \multicolumn{1}{c|}{\begin{tabular}[c]{@{}c@{}}Software Design \\ and \\ Architecture\end{tabular}} & 2021 \\ \hline
\multicolumn{1}{|l|}{\cite{R15_de2022software}} & \multicolumn{1}{c|}{\begin{tabular}[c]{@{}c@{}}Repository Mining\\ and   \\ Practitioner Survey\end{tabular}} & \multicolumn{1}{c|}{\begin{tabular}[c]{@{}c@{}}Quantum\\ Programming\\ Languages\end{tabular}} & \multicolumn{1}{l|}{\begin{tabular}[c]{@{}l@{}}Mining repositories and interviewing \\ practitioner to investigate quantum \\ programming languages\end{tabular}} & \multicolumn{1}{c|}{\begin{tabular}[c]{@{}c@{}}Software \\ Implementation\end{tabular}} & 2022 \\ \hline
\multicolumn{1}{|l|}{\cite{x4_Issues}} & \multicolumn{1}{c|}{\begin{tabular}[c]{@{}c@{}}Repository \\ Mining\end{tabular}} & \multicolumn{1}{c|}{\begin{tabular}[c]{@{}c@{}}Quantum\\ Issues\end{tabular}} & \multicolumn{1}{l|}{\begin{tabular}[c]{@{}l@{}}Mining repositories to identify technical\\ debts in open-source quantum software\end{tabular}} & \multicolumn{1}{c|}{\begin{tabular}[c]{@{}c@{}}Software \\ Implementation \\ and   Testing\end{tabular}} & 2022 \\ \hline
\multicolumn{1}{|l|}{\cite{X5_QC}} & \multicolumn{1}{c|}{\begin{tabular}[c]{@{}c@{}}Vision \\  Paper\end{tabular}} & \multicolumn{1}{c|}{\begin{tabular}[c]{@{}c@{}}Quantum\\ Software \\ Architecture\end{tabular}} & \multicolumn{1}{l|}{\begin{tabular}[c]{@{}l@{}}QSE life cycle and comparing   quantum\\ computers with their classical \\ counterparts and vision for future research\end{tabular}} & \multicolumn{1}{c|}{\begin{tabular}[c]{@{}c@{}}Software  \\ Life cycle\end{tabular}} & 2022 \\ \hline
\rowcolor[HTML]{DAE8FC} 
\multicolumn{6}{|c|}{\cellcolor[HTML]{DAE8FC}\textbf{Quantum Services Computing}} \\ \hline
\multicolumn{1}{|l|}{\cite{R11_leymann2020quantum}} & \multicolumn{1}{c|}{\begin{tabular}[c]{@{}c@{}}Literature \\ Survey\end{tabular}} & \multicolumn{1}{c|}{\begin{tabular}[c]{@{}c@{}}Quantum \\ Cloud \\ Computing\end{tabular}} & \multicolumn{1}{l|}{\begin{tabular}[c]{@{}l@{}}Programming quantum computers and \\ investigating hybrid software consisting \\ of classical parts  and quantum parts.\end{tabular}} & \multicolumn{1}{c|}{\begin{tabular}[c]{@{}c@{}}Service Design \\ and \\ Architecture\end{tabular}} & 2020 \\ \hline
\multicolumn{1}{|l|}{\cite{x7_QAAS}} & \multicolumn{1}{c|}{\begin{tabular}[c]{@{}c@{}}Literature\\ Survey\end{tabular}} & \multicolumn{1}{c|}{\begin{tabular}[c]{@{}c@{}}Quantum \\ Service \\ Computing\end{tabular}} & \multicolumn{1}{l|}{\begin{tabular}[c]{@{}l@{}}Potential and limitation  of integrating \\ quantum computing with cloud computing\end{tabular}} & \multicolumn{1}{c|}{\begin{tabular}[c]{@{}c@{}}Service \\ Lifecycle\end{tabular}} & 2015 \\ \hline
\end{tabular}}
\par\end{centering}
\label{tab:RelatedWork}
\end{table*}
\subsection{Empiricism in Mining Quantum Service Patterns}
Pattern-based software service engineering relies on architectural design and implementation strategies and best practices that can be reused to deliver software services \cite{R20_keen2006patterns}. Existing solutions employ a number of pattern-based solutions, such as classic-quantum split and service wrapping pattern, however, there is no evidence on an empirical discovery of patterns and tactics as reusable knowledge \cite{R27_GitQSE}. A lack of empiricism in discovering new patterns hinders the reusability of service design and implementation knowledge during quantum service engineering. One possible dimension for pattern discovery is mining software repositories or social coding platforms (e.g., GitHub) that contain raw knowledge that can be mined  as patterns. Pattern-based solutions could complement human expertise with available best practices for service design and implementation. %%More specifically, novice developers of quantum software services can rely on patterns that abstract complexities of quantum source coding with modularisation of pattern-based solutions.

\faComment{}
{~\footnotesize\textsf{\textbf{Pattern discovery} via empirically-grounded methods, i.e., mining repositories or social coding platforms of quantum software development, can help leverage reusable design rationale for quantum service engineering. Pattern languages can empower the role of quantum software architects and algorithm designers to rely on reusable knowledge and best practices as opposed to ad-hoc and once-off solutions.}}



%%\faHandORight{\scriptsize{\textsf{~Patterns and tactics represent theoretical knowledge and best practices as an empirically grounded design decisions and strategies that enable reuse and enhance quality of software services. Mining patterns (investigating service repositories and social coding platforms etc.) can help discover and document patterns specific to quantum software services.}}}

\subsection{Continuous Testing and Delivery of Quantum Services} 
With an adoption of agile software engineering in quantum software development context \cite{R23_khan2022agile}, there is a need for light and adaptive methods to ensure a continuous development and delivery of quantum software services. The literature suggested a lack of solutions on testing the software services. The continuous testing and continuous delivery can help CT/CD to test the services against QSRs more effectively and deliver them rapidly. Quantum service testing can involve simulation or regression tests against the QSRs.

%\faComment{} 
%{~\footnotesize\textsf{\TEXTBF{Continuous Testing and Continuous Delivery} (CT/CD) relies on agile software engineering methods to help deliver quantum software services rapidly and reliably. CT/CD can provide strategic benefits to vendors by adding new services to their quantum platforms.}} 

\faComment {~\footnotesize\textsf{\textbf{Continuous Testing and Continuous Delivery} (CT/CD) relies on agile software engineering methods to help deliver quantum software services rapidly and reliably. CT/CD can provide strategic benefits to vendors by adding new services to their quantum platforms.}}


%%\subsection{Continuous Delivery of Quantum Services}

