\section{Introduction}
\label{sec:introduction}
Quantum computing (QC) has started to emerge as a disruptive technology and an enabling platform – exploiting the principles of quantum mechanics – relying on Quantum Bits (QuBits) that manipulate Quantum Gates (QuGates) to tackle computationally intensive tasks efficiently \cite{R1_ali2022software}. QC systems are in a phase of continuous evolution and despite being in a state of their infancy, such systems have started to computationally outperform their classical counterparts (i.e., digital computers) in applications such as quantum information processing, bio-inspired computing, and simulation of quantum mechanics \cite{R2_harrow2017quantum}. Academic research \cite{R1_ali2022software} \cite{R3_egger2020quantum} and industrial initiatives led by technology giants such as IBM, Google, and Microsoft \cite{R4_dyakonov2019will} are striving hard to achieve strategic advantages associated with quantum systems and software technologies in a so-called `race to quantum economy’. A recent report presented at the World Economic Forum titled \textit{State of Quantum Computing: Building a Quantum Economy} highlights that by the year 2022, public and private investments in quantum computing technologies totalled \$35.5 billion \cite{R5_WorldEconomicForum}. Despite the strategic capabilities that can be attained via quantum supremacy in computing; programming, operationalising, and maintaining a quantum computer is a complex and radically distinct engineering paradigm \cite{R1_ali2022software}. To augment the quantum hardware development, state-funded projects, and global consortiums are proactively funding initiatives such as the Quantum Flagship \cite{R14_riedel2019europe} and National Quantum Initiative \cite{R22_raymer2019us} to develop software ecosystems, networking technologies, and human expertise for the alleged quantum leap in computing \cite{R13_monroe2018quantum}. 


\textbf{Motivation: Pay-per-use QCaaS -} Service-oriented systems have proven to be useful in supporting utility computing model that relies on pay-per usability approach for individuals and organisations to utilize a multitude of computing services without the need to own or maintain them \cite{R6_wei2010service}. Since their early adoption, pay-per-use service-driven systems have now grown from data storage, video streaming, and entertainment, to resource-sharing applications that represent a multi-billion dollar industry in service economies \cite{R7_bouguettaya2017service}\cite{R8_Statista}. The ‘as-a-Service’ (aaS) model has provided the impetus to wide-scale adoption of service systems that can offer a plethora of computing services including but not limited to storage, computation, infrastructure, platform, and software to end-users \cite{R7_bouguettaya2017service}. The aaS model can enable users and developers who can exploit the QC platforms (e.g., processors, memory, simulators) offered by quantum vendors, such as Amazon, Goole, and IBM \cite{R9_moguel2022quantum}. Quantum Computing as a Service (QCaaS) as in Figure \ref{Fig-2:QCaaS} is a recent and quantum-specific genre of aaS model that is built on the philosophy of service-orientation of QC, i.e., pay-per-shot at QC resources instead of owning, programming, and/or maintaining quantum computers \cite{R10_garcia2021quantum}. Quantum vendors view pay-per-shot as an opportunistic business model to generate revenue streams from their QC infrastructures, where a shot is a single execution of a quantum algorithm on a quantum processing unit (QPU). % Beyond the design philosophy, technical concepts of aaS such as service wrapping pattern, microservices architectural style, and service implementation tools can be adopted or extended to make QaaS as a solution for distributed QC. 
The QCaaS can alleviate the need to own or maintain quantum computers and can help software and service developers  to rely on  existing knowledge and best practices to develop software services  that can be executed on QC platforms \cite{R11_leymann2020quantum}.  Quantum software services enable developers to wrap data and computations inside loosely coupled, fine-grained modules of source code to execute tasks such as prime factorisation, key encryption, or bio stimulations on QC platforms \cite{R10_garcia2021quantum}. There is a growing interest in the academic community and industrial vendors to research and develop solutions for enabling quantum service-orientation \cite{R9_moguel2022quantum}\cite{R10_garcia2021quantum}.

\begin{figure}[]
 \centering
 \includegraphics[scale=0.80]{Images/QCaaS.pdf} 
 	\caption{A Generic View of the QCaaS Model}
	\label{Fig-2:QCaaS}
\end{figure}
\textbf{Needs for the mapping study:} Systematic mapping studies (SMS) rely on evidence-based software engineering approach to systematically and reproducibly identify, analyse, synthesise, and document (i.e., map the trends of) existing research on the topic under investigation \cite{R16_petersen2008systematic}. The academic community aims to exploit QC platforms for empiricism in quantum research \cite{R9_moguel2022quantum}, while the vendors pursue a revenue stream as well as validation and testing of their under-development quantum platforms, offered as a service \cite{R10_garcia2021quantum}. Synergising research and development on quantum computing with service-orientation can allow researchers and developers to leverage existing design principles, patterns, architectural styles, and modelling languages of services computing to offer QC as a service \cite{R11_leymann2020quantum}. Moreover, systemizing the efforts to architect and implement QaaS requires discovering new patterns and developing innovative frameworks rooted in empirically grounded guidelines to research emerging challenges and develop futuristic solutions \cite{R12_valencia2022quantum}. We conducted this SMS based on two research questions to investigate (i) \textit{existing solutions in terms of designing, implementing, deploying, and operationalizing quantum software services} and  (ii) \textit{ emerging trends that can indicate futuristic research on QCaaS}. %We organised and documented the SMS results, i.e., answers to the RQs, as activities of Service Oriented Architecture (SOA) life cycle \cite{R20_keen2006patterns} that support conception, modeling, assembly, and deployment of quantum services. 
The results indicate that during quantum service-orientation, software modeling languages (e.g., UML) and patterns (e.g., service wrapping, API gateway) help in mapping functional requirements to service implementation (e.g., Python code) that can be deployed on QC platforms (e.g., Amazon Braket). Emerging trends indicate non-functional aspects, model-driven engineering (low-code development), empirically discovered tactics, human roles, and process-centric development of QCaaS.

\textbf{Contributions and implications} of this SMS are to: 

\begin{itemize}
    \item identify and document a collective impact of existing research (published academic evidence)  to investigate the extent to which classical and quantum-specific service-orientation can be applied to QCaaS.
    \item highlight emerging trends - identifying existing gaps - that reflect the dimensions of future research to develop emerging and next-generation of QCaaS solutions.      
\end{itemize}

Academic researchers can rely on mapping of existing research and vision for future work to research and develop QCaaS solutions in a broader context of QSE \cite{R1_ali2022software}\cite{R14_riedel2019europe}. Practitioners who can rely on academic references about patterns (reusability), modeling notations (representation), and implementation (prototyping) to develop QCaaS \cite{R9_moguel2022quantum}.    

%%Rest of the paper is organised as follow. Research context and method are detailed in Section 2. Results of the mapping study that answer the RQs are presented in Section 3 and Section 4. Review of most relevant existing research is in Section 5. Threats to the Validity of SMS are in Section 6. Conclusions and future work are in Section 7.
