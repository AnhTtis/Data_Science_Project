\section{Engineering Software for QCaaS (RQ-1)}
\label{sec:RQ1}
We now discuss the existing solutions, reported in the literature, that support the development of quantum services to operationalise QCaaS solutions. The data extracted from the selected studies is presented in Table \ref{tab:criteria} and visualised in Figure \ref{Results}. Table \ref{tab:criteria} can be viewed as a catalogue that organises a summary of the core findings to answer RQ-1 based on the four activities of the SOA life cycle (Phase III, Figure \ref{Researchmethod}). %Figure \ref{Results} complements Table \ref{tab:criteria} with an illustrative example of the quantum service life cycle in Figure \ref{tab:criteria} a) and highlighting the core findings in Figure \ref{tab:criteria} b) - g). 

\vspace{0.5em}

\textbf{Illustrative Example}: Figure \ref{Results}a) exemplifies the service life cycle with \textit{functional aspects} that requires a quantum service to compute the prime factors of an integer. Functional aspects need \textit{modeling} and that uses Unified Modeling Language (UML) component diagram \cite{R21_perez2020towards} as the modeling notation to specify computational elements (components) and their interconnections (connectors) in a service. The Classic-Quantum split pattern \cite{R12_valencia2022quantum} is applied to slice the functionality between a classical computer (pre-/post-processing, e.g., \textsf{Gen\_Num} component) and a quantum computer (prime factorization, \textsf{Factorize} component). The model acts as a blue-print to support \textit{assembly} of a services using a programming language (Qiskit code snippet) that converts the design specifications of a service to its executable specifications. Finally, \textit{deployment} activity is shown as UML deployment diagram to configure the assembled service on a quantum computer provided by the quantum vendor (Amazon Braket). 
%We now present each of the six aspects of results visualised in Figure 3 b) – g) in the context of four activities of service lifecycle. 

\subsection{\textbf{Conception}: \textsf{Functional Aspects}} 
During the conception activity, functional aspects of a service relate to identifying and outlining the functionality to be offered by a quantum service. The functionality can be achieved via service execution on a quantum computer (provider), whereas the service is developed or invoked by the user (requester). Figure \ref{Fig-1:Context} shows that, in order to exploit QC platforms for quantum functionality, service requesters can develop/discover new and/or available services, such as quantum simulation or quantum cryptography using the SOA patterns \cite{R7_bouguettaya2017service}. In the SMS, we identified a multitude of functional aspects for quantum services and organized them into five categories namely \textit{experimental}, \textit{service delivery}, \textit{number crunching}, \textit{data searching}, and \textit{natural computing} as shown in Table \ref{tab:criteria}. For example, Figure \ref{Results}a) highlights the generic functional aspect of number crunching that involves prime factorisation of integers for Shor's algorithm \cite{R1_ali2022software}\cite{R2_harrow2017quantum}.  

The diversity of existing functional aspects is proportional to the capability of the current era of quantum computers that is considred as limited due to a number of factors such as simplistic quantum circuitry (e.g., less QuBits/QuGates) and quantum errors (e.g., NISQ) \cite{R4_dyakonov2019will}. Quantum software services that support functional aspects of QC are merely capable of checking the status of quantum circuits (experimental) or generation of random numbers using quantum hardware (number crunching). Functional aspects reflect only a partial view of system design in terms of offered functionality that should not overlook the non-functional or quality aspects of the QCaaS solutions. For example, resource efficiency in terms of utilizing minimal available QuBits to generate prime factors can ensure the required functionality and desired quality (e.g., service efficiency, execution performance) of QCaaS.  

\begin{tcolorbox} [sharp corners, boxrule=0.1mm,]
\faEdit \scriptsize{~\textsf{\textbf{Functional aspects} of quantum services, in general, are rather limited to basic quantum experimentation and numerical processing. The limitation reflects the existing capabilities of QCs and consequently the offered services by quantum vendors. Investigating the non-functional aspects can help outline the quantum significant requirements (QSRs) in terms of required functionality and desired quality of the service that currently lacks in the existing literature.}}
\end{tcolorbox}
   
   
\begin{figure*}[ht]
 \centering
% \includegraphics[width =14cm, height = 11cm]{Images/Result.pdf}
 \includegraphics[scale =0.75]{Images/Results.pdf} 
 	\caption{An Overview of the Mapping Study Results}
	\label{Results}
\end{figure*}

\subsection{\textbf{Modeling}: \textsf{Notations and Patterns}} 
%The terms \footnote{\textit{pattern} and \textit{style} in software engineering context are virtually synonymous and are often used interchangeably. A distinction should be maintained: (i) style is reusable structuring as styling of architecture (e.g., microservices architecture), and (ii) pattern represent recurring solution to design problems (e.g., service façade).}

During service modeling, modeling notations, such as Q-UML or ontological models can help create a blueprint for the implementation of functional aspects \cite{R21_perez2020towards}. Patterns can complement the notations by providing reusable knowledge and design rationale to architect the quantum services \cite{R10_garcia2021quantum} \cite{R12_valencia2022quantum}.

\begin{table*}[]
\caption{Data Extracted for SMS from Reviewed Studies (SOA lifecycle activities [20])}
\begin{centering}
{\tiny
\begin{tabular}{|c|c|cc|cc|c|}
\hline
\multicolumn{7}{|c|}{\includegraphics[scale=0.6]{Images/QaaS-Table.pdf}} \\ \hline
\multicolumn{1}{|l|}{} & \multicolumn{1}{c|}{\cellcolor[HTML]{BDD6EE}\textbf{Conception}} & \multicolumn{2}{c|}{\cellcolor[HTML]{BDD6EE}\textbf{Modeling}} & \multicolumn{2}{c|}{\cellcolor[HTML]{BDD6EE}\textbf{Assembly}} & \multicolumn{1}{c|}{\cellcolor[HTML]{BDD6EE}\textbf{Deployment}} \\ \cline{2-7}
\multirow{-2}{*}{\textbf{\begin{tabular}[c]{@{}c@{}}Study ID \\ \& \\ Research Type\end{tabular}}} &
  \cellcolor[HTML]{F2F2F2}\begin{tabular}[c]{@{}c@{}}Functional \\ Aspects\end{tabular} &
  \multicolumn{1}{c|}{\cellcolor[HTML]{F2F2F2}\begin{tabular}[c]{@{}c@{}}Modelling \\ Notation\end{tabular}} &
  \cellcolor[HTML]{F2F2F2}\begin{tabular}[c]{@{}c@{}}Software\\ Pattern\end{tabular} &
  \multicolumn{1}{c|}{\cellcolor[HTML]{F2F2F2}\begin{tabular}[c]{@{}c@{}}Service \\ Use case\end{tabular}} &
  \cellcolor[HTML]{F2F2F2}\begin{tabular}[c]{@{}c@{}}Service\\ Programming\end{tabular} &
  \cellcolor[HTML]{F2F2F2}\begin{tabular}[c]{@{}c@{}}Quantum\\ Platform/\\ Vendor\end{tabular} \\ \hline
\rowcolor[HTML]{FFFFFF} 


\begin{tabular}[c]{@{}c@{}}{\includegraphics[scale=0.11]{Images/icons/1.jpg}} \\ {[}S1{]}\end{tabular} &
  \begin{tabular}[c]{@{}c@{}}{\includegraphics[scale=0.04]{Images/icons/Empirical.png}}\\    Quantum   Service \\ Delivery\end{tabular} &
  \multicolumn{1}{c|}{\cellcolor[HTML]{FFFFFF}\begin{tabular}[c]{@{}c@{}}{\includegraphics[scale=0.04]{Images/icons/UMLS.png}}   \\ Deployment\\ Diagram\end{tabular}} &
  \begin{tabular}[c]{@{}c@{}}{\includegraphics[scale=0.04]{Images/icons/API.png}}\\ API Gateway\end{tabular} &
  \multicolumn{1}{c|}{\cellcolor[HTML]{FFFFFF}\begin{tabular}[c]{@{}c@{}}{\includegraphics[scale=0.04]{Images/icons/oPTIMISATIOPN.png}} \\ Optimal Service \\ Provider\end{tabular}} &

  \begin{tabular}[c]{@{}c@{}}{\includegraphics[scale=0.15]{Images/icons/flask1.png}} \\ Python, Flask \end{tabular} &
  
  \begin{tabular}[c]{@{}c@{}}{\includegraphics[scale=0.20]{Images/icons/amazon.png}}\end{tabular} \\ \hline
\rowcolor[HTML]{FFFFFF} 
\begin{tabular}[c]{@{}c@{}}{\includegraphics[scale=0.04]{Images/icons/opinion.png}}\\   {[}S2{]}\end{tabular} &
  \begin{tabular}[c]{@{}c@{}}{\includegraphics[scale=0.07]{Images/icons/ServiceDelivery.png}}   \\ Enterprise  Services \\ Development\end{tabular} &
  \multicolumn{1}{c|}{\cellcolor[HTML]{FFFFFF}\begin{tabular}[c]{@{}c@{}}{\includegraphics[scale=0.04]{Images/icons/bpmn.png}}\\   Business Process\end{tabular}} &

  \begin{tabular}[c]{@{}c@{}}{\includegraphics[scale=0.07]{Images/icons/layer.png}}  \\ Layered\\ Architecture\end{tabular} &
  
  \multicolumn{1}{c|}{\cellcolor[HTML]{FFFFFF}\begin{tabular}[c]{@{}c@{}}{\includegraphics[scale=0.04]{Images/icons/process.png}}   \\ Process Automation\end{tabular}} &
  
  \begin{tabular}[c]{@{}c@{}}{\includegraphics[scale=0.07]{Images/icons/Noevidence.png}}\\No Evidence \end{tabular} &
  \begin{tabular}[c]{@{}c@{}}{\includegraphics[scale=0.07]{Images/icons/Noevidence.png}}\\No Evidence \end{tabular} \\ \hline
\rowcolor[HTML]{FFFFFF} 
\begin{tabular}[c]{@{}c@{}}{\includegraphics[scale=0.11]{Images/icons/1.jpg}}\\   {[}S3{]}\end{tabular} &
  \begin{tabular}[c]{@{}c@{}}{\includegraphics[scale=0.11]{Images/icons/numb.png}}   \\ Quantum Random \\ Number Generation\\ Quantum Search Algo\end{tabular} &
  \multicolumn{1}{c|}{\cellcolor[HTML]{FFFFFF}\begin{tabular}[c]{@{}c@{}}{\includegraphics[scale=0.04]{Images/icons/UML.png}}   \\ Class, Sequence \\ Diagram\end{tabular}} &

  \begin{tabular}[c]{@{}c@{}}{\includegraphics[scale=0.07]{Images/icons/split.png}}\\Classic-
Quantum Split\end{tabular} &
  
  \multicolumn{1}{c|}{\cellcolor[HTML]{FFFFFF}\begin{tabular}[c]{@{}c@{}}{\includegraphics[scale=0.07]{Images/icons/maths.png}} \\ Mathematics\end{tabular}} &
  
  \begin{tabular}[c]{@{}c@{}}{\includegraphics[scale=0.11]{Images/icons/qsharp1.png}} \\Q sharp\end{tabular} &

  
  \begin{tabular}[c]{@{}c@{}}{\includegraphics[scale=0.11]{Images/icons/ibm.png}} \\IBM Quantum\end{tabular} \\ \hline
\rowcolor[HTML]{FFFFFF} 
\begin{tabular}[c]{@{}c@{}}{\includegraphics[scale=0.04]{Images/icons/opinion.png}} \\ {[}S4{]}\end{tabular} &
  
  \begin{tabular}[c]{@{}c@{}}{\includegraphics[scale=0.11]{Images/icons/numb.png}}\\ Integer Factorisation\end{tabular} &
  
  \multicolumn{1}{c|}{\cellcolor[HTML]{FFFFFF}\begin{tabular}[c]{@{}c@{}}{\includegraphics[scale=0.05]{Images/icons/UMLS.png}}\\ Deployment Diagram\end{tabular}} &
  \begin{tabular}[c]{@{}c@{}}{\includegraphics[scale=0.11]{Images/icons/1.jpg}} \\ Solution\end{tabular} &
  
  \multicolumn{1}{c|}{\cellcolor[HTML]{FFFFFF}\begin{tabular}[c]{@{}c@{}}{\includegraphics[scale=0.07]{Images/icons/maths.png}} \\ Mathematics\end{tabular}} &
  
  \begin{tabular}[c]{@{}c@{}}{\includegraphics[scale=0.005]{Images/icons/Python_icon_black_and_white.png}}\\Python\end{tabular} &
  
  \begin{tabular}[c]{@{}c@{}}{\includegraphics[scale=0.20]{Images/icons/amazon.png}}\end{tabular} \\ \hline
  
  \rowcolor[HTML]{FFFFFF} 
\begin{tabular}[c]{@{}c@{}}{\includegraphics[scale=0.01]{Images/icons/validate.png}}  \\ {[}S5{]}\end{tabular} &
  \begin{tabular}[c]{@{}c@{}}{\includegraphics[scale=0.07]{Images/icons/ServiceDelivery.png}}\\ Experimental Quantum\\   Service Computing)\end{tabular} &
  
  \multicolumn{1}{c|}{\cellcolor[HTML]{FFFFFF}\begin{tabular}[c]{@{}c@{}}{\includegraphics[scale=0.07]{Images/icons/NoEvidences.png}}\\No Evidence\end{tabular}} &
  
  \begin{tabular}[c]{@{}c@{}}{\includegraphics[scale=0.09]{Images/icons/servicewrapper.png}}\\Service  
Wrapping \end{tabular} &
  
  \multicolumn{1}{c|}{\cellcolor[HTML]{FFFFFF}\begin{tabular}[c]{@{}c@{}}{\includegraphics[scale=0.04]{Images/icons/oPTIMISATIOPN.png}}\\Optimisation\end{tabular}} &

  \begin{tabular}[c]{@{}c@{}}{\includegraphics[scale=0.15]{Images/icons/flask1.png}}\\Python, Flask\end{tabular} &
  
  \begin{tabular}[c]{@{}c@{}}{\includegraphics[scale=0.20]{Images/icons/amazon.png}}\end{tabular} \\ \hline

\rowcolor[HTML]{FFFFFF} 
\begin{tabular}[c]{@{}c@{}}{\includegraphics[scale=0.11]{Images/icons/1.jpg}}\\   {[}S6{]}\end{tabular} &
  \begin{tabular}[c]{@{}c@{}}{\includegraphics[scale=0.07]{Images/icons/ServiceDelivery.png}}\\ Experimental Services\\ Algorithm\end{tabular} &
  \multicolumn{1}{c|}{\cellcolor[HTML]{FFFFFF}\begin{tabular}[c]{@{}c@{}}{\includegraphics[scale=0.04]{Images/icons/UML.png}}\\ Sequence Diagrams\end{tabular}} &
  
  \begin{tabular}[c]{@{}c@{}}{\includegraphics[scale=0.04]{Images/icons/API.png}}\\ API Gateway\end{tabular} &

  \multicolumn{1}{c|}{\cellcolor[HTML]{FFFFFF}\begin{tabular}[c]{@{}c@{}}{\includegraphics[scale=0.07]{Images/icons/ServiceDevelopment.jpg}}\\ Algorithm as \\ a Service\end{tabular}} &
 
  \begin{tabular}[c]{@{}c@{}}{\includegraphics[scale=0.005]{Images/icons/Python_icon_black_and_white.png}}\\Python \end{tabular} &
  
  \begin{tabular}[c]{@{}c@{}}{\includegraphics[scale=0.07]{Images/icons/rigetti.png}} \\ Rigetti\end{tabular} \\ \hline

\rowcolor[HTML]{FFFFFF} 
\begin{tabular}[c]{@{}c@{}}{\includegraphics[scale=0.11]{Images/icons/1.jpg}}\\   {[}S7{]}\end{tabular} &
  \begin{tabular}[c]{@{}c@{}}{\includegraphics[scale=0.07]{Images/icons/numb.png}}   \\ Integer Factorisation\end{tabular} &
  
  \multicolumn{1}{c|}{\cellcolor[HTML]{FFFFFF}\begin{tabular}[c]{@{}c@{}}{\includegraphics[scale=0.15]{Images/icons/grph.png}}\\Directed Graph\end{tabular}} &
  
  \begin{tabular}[c]{@{}c@{}}{\includegraphics[scale=0.04]{Images/icons/API.png}}\\ API Gateway\end{tabular} &
  
  \multicolumn{1}{c|}{\cellcolor[HTML]{FFFFFF}\begin{tabular}[c]{@{}c@{}}{\includegraphics[scale=0.04]{Images/icons/oPTIMISATIOPN.png}} \\Optimisation \end{tabular}} &
  
  \begin{tabular}[c]{@{}c@{}}{\includegraphics[scale=0.07]{Images/icons/Noevidence.png}}\\No Evidence\end{tabular} &
  
  \begin{tabular}[c]{@{}c@{}}{\includegraphics[scale=0.20]{Images/icons/amazon.png}}\end{tabular} \\ \hline

  
\rowcolor[HTML]{FFFFFF} 
\begin{tabular}[c]{@{}c@{}}{\includegraphics[scale=0.04]{Images/icons/opinion.png}}   \\ {[}S8{]}\end{tabular} &

  \begin{tabular}[c]{@{}c@{}}{\includegraphics[scale=0.03]{Images/icons/bio.png}}\\Bio-inspired\\
Computing\end{tabular} &
  
  \multicolumn{1}{c|}{\cellcolor[HTML]{FFFFFF}\begin{tabular}[c]{@{}c@{}}{\includegraphics[scale=0.07]{Images/icons/ontology.png}}\\Ontologies\end{tabular}} &
  
  \begin{tabular}[c]{@{}c@{}}{\includegraphics[scale=0.09]{Images/icons/repository.png}}\\Repository \\
Pattern\end{tabular} &
  
  \multicolumn{1}{c|}{\cellcolor[HTML]{FFFFFF}\begin{tabular}[c]{@{}c@{}}{\includegraphics[scale=0.04]{Images/icons/simulate.png}} \\Simulation\end{tabular}} &
  
  \begin{tabular}[c]{@{}c@{}}{\includegraphics[scale=0.07]{Images/icons/Noevidence.png}}\\No Evidence\end{tabular} &
  
  \begin{tabular}[c]{@{}c@{}}{\includegraphics[scale=0.07]{Images/icons/Noevidence.png}}\\No Evidence\end{tabular} \\ \hline

\rowcolor[HTML]{FFFFFF} 
\begin{tabular}[c]{@{}c@{}}{\includegraphics[scale=0.04]{Images/icons/opinion.png}}   \\ {[}S9{]}\end{tabular} &
 
  \begin{tabular}[c]{@{}c@{}}{\includegraphics[scale=0.11]{Images/icons/numb.png}} \\Integer \\Factorisation \end{tabular} &
  
  \multicolumn{1}{c|}{\cellcolor[HTML]{FFFFFF}\begin{tabular}[c]{@{}c@{}}{\includegraphics[scale=0.07]{Images/icons/Noevidence.png}}\\No Evidence\end{tabular}} &
  
  \begin{tabular}[c]{@{}c@{}}{\includegraphics[scale=0.09]{Images/icons/servicewrapper.png}} \\Service  
Wrapping\end{tabular} &
  
  \multicolumn{1}{c|}{\cellcolor[HTML]{FFFFFF}\begin{tabular}[c]{@{}c@{}}{\includegraphics[scale=0.07]{Images/icons/Noevidence.png}}\\No Evidence\end{tabular}} &
  
  \begin{tabular}[c]{@{}c@{}}{\includegraphics[scale=0.005]{Images/icons/Python_icon_black_and_white.png}}\\Python\end{tabular} &
  \begin{tabular}[c]{@{}c@{}}{\includegraphics[scale=0.20]{Images/icons/amazon.png}}\end{tabular} \\ \hline
\end{tabular}}
\par\end{centering}
\label{tab:criteria}
\end{table*}

\textit{Modeling notations} are fundamental to the creation, maintenance, and evolution of models such as ontological structures and graph-based diagrams that provide a visual representation, whereas architectural description languages support a textual specification for software-intensive systems \cite{R21_perez2020towards}. Recent trends in software engineering that promote model-driven and low-code application development have resulted in transitioning developers’ focus from coding to software modeling for implementation \cite{R22_raymer2019us}. Low code application development process leverages the principle and practices of model-driven engineering to utilise model(s) as first-class entities in software development \cite{R15_de2022software}\cite{R19_ahmadtowards}. Investigating software models and modeling notations that help create service models is essential to support model-driven perspective to QSE, consequently facilitating quantum code developers to abstract implementation-specific complexities, via model-driven QSE, while developing quantum services \cite{R19_ahmadtowards}. This SMS indicates three main types of notations to model services in QCaaS that include the Unified Modeling Language (UML), graph-based models, and process models highlighted in Figure \ref{Results}b). UML-based models are represented via a multitude of notations, such as class and component diagrams that represent the structure, while sequence and deployment diagrams represent runtime or behavioural view of QCaaS. Graph-based models contain directed graphs and ontologies, whereas process models rely on automating the business processes of an enterprise as quantum services. For example, the study [S3] reports a class diagram as a structural view of the system to represent the attributes and methods of entities (user, service provider, authentication etc.) of quantum computing as a service. %The study also demonstrates a sequence diagram representing behavioural view of quantum as a function service. %It is unsurprising to witness UML-based diagrams representing the majority of QaaS models. 
UML diagrams and profiles represent the status-quo in software modeling and are seen as the de-facto notation in the software and service ' community to model classical software with growing adoption in QSE \cite{R21_perez2020towards}.

\textit{Design patterns} represent a concentrated wisdom of software designers that can be leveraged to address design and implementation issues, addressing functionality and quality,   effectively and efficiently. Considering a lack of professional expertise in QSE (e.g., quantum domain engineers, quantum algorithm designers, quantum software architects etc.) patterns as artifacts of reuse can help novice developers during quantum software development to rely on existing best practices \cite{R1_ali2022software}\cite{R10_garcia2021quantum}. This SMS highlights that the literature on QCaaS  reports five patterns, namely the \textit{API Gateway}, \textit{Layered Architecture}, \textit{Classic-Quantum Split}, \textit{Service Wrapping}, and \textit{Repository Pattern}. Figure \ref{Results}c) depicts pattern thumbnails as an abstract view of the identified patterns. Patterns are generally documented as templates or pattern languages, here we only focus on overviewing the reported patterns for QCaaS, while details for pattern representation and documentation can be found in \cite{R19_ahmadtowards}. The Classic-Quantum Split pattern \cite{R12_valencia2022quantum} is a quantum version of the Splitter pattern, driven by quantum workflow, that splits computation tasks into tasks that can be generated and executed on classical machines (e.g., random number generation) and tasks that can be executed on quantum machines (e.g., prime factorisation). The pattern aims to address issues like NISQ by splitting quantum software into classical and quantum parts as a hybrid application \cite{R2_harrow2017quantum}\cite{R4_dyakonov2019will}. 

\begin{tcolorbox} [sharp corners, boxrule=0.1mm,]
\faEdit \scriptsize{~\textsf{\textbf{Modeling notations} can assist software engineers to transit their focus from implementation towards design perspective. Modeling can incrementally transform functional aspects to service models leading to service implementation via model-driven engineering or low-code development. \textbf{Patterns} (classical or quantum-specific) can facilitate developers to architect and implement quantum-age software services by relying on reusable knowledge and best practices of service-orientation.}}
\end{tcolorbox}

Recently, a number of studies have focused on organising quantum software patterns as a body of knowledge in QSE, however, there is no evidence of empirically-derived methods to discover and document patterns for quantum services computing. SOA-specific patterns like API Gateway and Service Wrapping patterns can be tailored to address QCaaS solutions. There is a need for mining repositories and knowledge resources to discover reusable knowledge and best practices from quantum software development projects that can to be documented as tactics and patterns for QCaaS solutions.


\subsection{\textbf{Assembly:} \textsf{Application Domain and Programming}} 
Assembling the quantum services involves identifying the application domains and exploiting the programming languages as implementation technologies to develop executable specifications from the service model \cite{R7_bouguettaya2017service} \cite{R9_moguel2022quantum}. 

\textit{Application Domain} is also referred to as the implementation use cases or practical context to which the QCaaS solutions can be applied. For example, quantum security represents an application domain for quantum software servicing where a service can be invoked to implement cryptography protocols to generate and manage a secure quantum key \cite{R13_monroe2018quantum}. The results of this SMS indicate four application domains, namely \textit{OptimiSation}, \textit{Process Automation}, \textit{Mathematics}, and \textit{Quantum Simulation}. The application domains may impact the selection of programming languages and tools for service implementation. For example, the study [S3] uses Q\# as the programming language that can be developed and compiled in Microsoft .Net framework for executing quantum algorithms.

\textit{Service Implementation} involves programming languages that represent a system of notation or source coding scripts for implementing quantum  services to manage and operationalise QC resources \cite{R15_de2022software} \cite{R21_perez2020towards}. In recent years, a number of Quantum Programming Languages (QPLs) including but not limited to Q\# by Microsoft or Cirq by Google have emerged to provide specialized programming syntax, framework, and environments to develop, execute, and deploy quantum source code. Insights into programming languages can reveal if classical programming languages (e.g., C, Java, Python etc.) suffice for QCaaS implementation or if there is need for more specialised QPLs (Q\#, Cirq etc.). The SMS results indicate three programming languages, namely Python, Java, and Q\#, as the preferred languages to implement quantum services. Based on the details of source coding, Figure \ref{Results}f) distinguishes between native code of a language and specialised libraries/application programming interfaces (APIs) being developed using a specific language. For example, the studies [S4, S5] used native Python code to implement quantum micro-servicing for experimentation. In comparison, the study [S1] used Flask as a Web framework written in Python to develop a solution for optimal delivery of quantum services on Amazon Braket. Python is the most preferred programming language both in terms of native code as well as specialised libraries of Python that include Flask and Qiskit as open-source language frameworks and Cirq which is adopted by Google. 

\begin{tcolorbox} [sharp corners, boxrule=0.1mm,]
\faEdit \scriptsize{~\textsf{\textbf{Application domains} represent the practical context/use cases of quantum services and may impact the selection of programming languages and tools for implementation. \textbf{Programming languages} provide a system of notation for source-coding of quantum services. Classical programming languages, such as Python represent a predominant choice over QPLs to implement QCaaS due to more comprehensive documentation and familiarity of Python in service developers' community.}}
\end{tcolorbox}
    
\subsection{\textbf{Deployment:} \textsf{Quantum Platform}} 
The deployment activity supports the selection of QC platforms on which services can be deployed for their operationalisation and execution. Platform providers also referred to as quantum vendors offer computing infrastructure in terms of hardware as well as software that allows service developers to develop and/or utilise the quantum services. Deployment represents the last activity in the SOA life cycle that is represented as a UML deployment diagram in Figure \ref{Results}a). This SMS identified a total of three quantum vendors for the deployment of quantum services, namely \textit{Amazon Bracket}, \textit{IBM Quantum}, and \textit{Rigetti}. Amazon Braket (a managed Amazon Web Services (AWS)) is the most preferred platform to design, test, and run quantum algorithms. One of the reasons for selecting Amazon Braket for service deployment is that it can allow service users/developers to design their own quantum algorithms. This can be particularly handy for novice developers unfamiliar with the technicalities of quantum systems to utilise a set of pre-built algorithms, tools, and documents to develop and manage quantum services on Amazon platform. 

\begin{tcolorbox} [sharp corners, boxrule=0.1mm,]
\faEdit \scriptsize{~\textsf{\textbf{Quantum platforms} leverage cloud computing infrastructures to support quantum service-orientation. Amazon Braket is the predominant quantum vendor that can enable novice developers to exploit some pre-built algorithms, programming tools, and service documentation that lack on other quantum platforms.}}
\end{tcolorbox}
    




    
    % \begin{table*}[ht]
    % \caption{Caption}
    % \begin{centering}
    % {\tiny{}}%
    
    % \begin{tabular}{|l|l|l|l|l|l|l|}
    % \hline
    % \multicolumn{1}{|c|}{\begin{tabular}[c]{@{}c@{}}[S1]\\ {\includegraphics[scale=0.15]{Images/icons/1.jpg}}\end{tabular}} & \multicolumn{1}{c|}{\begin{tabular}[c]{@{}c@{}}{\includegraphics[scale=0.06]{Images/icons/2.png}} \\ Quantum vendor \\ Selection\end{tabular}} & \multicolumn{1}{c|}{\begin{tabular}[c]{@{}c@{}}{\includegraphics[scale=0.06]{Images/icons/3.png}}\\ Deployment \\ Diagram\end{tabular}} & \multicolumn{1}{c|}{\begin{tabular}[c]{@{}c@{}}{\includegraphics[scale=0.06]{Images/icons/4.png}}\\ API \\ Gateway\end{tabular}} & \multicolumn{1}{c|}{\begin{tabular}[c]{@{}c@{}}{\includegraphics[scale=0.07]{Images/icons/5.png}} \\ Optimal Service \\ Delivery\end{tabular}} & \multicolumn{1}{c|}{\begin{tabular}[c]{@{}c@{}}{\includegraphics[scale=0.013]{Images/icons/6.png}}\\ Some Text\end{tabular}} & \multicolumn{1}{c|}{\begin{tabular}[c]{@{}c@{}}{\includegraphics[scale=0.2]{Images/icons/7.jpg}}\\ Amazon Braket\end{tabular}} \\ \hline
    %  &  &  &  &  &  &  \\ \hline
    %  &  &  &  &  &  &  \\ \hline
    %  &  &  &  &  &  &  \\ \hline
    
    % \end{tabular}
    % \par\end{centering}
    % \label{tab:criteria}
    % \end{table*}
