\documentclass[conference]{IEEEtran}
\IEEEoverridecommandlockouts
\usepackage{cite}
\usepackage{amsmath,amssymb,amsfonts}
\usepackage{algorithmic}
\usepackage{dblfloatfix}
\usepackage{graphicx}
\usepackage{placeins}
\usepackage{textcomp}
\usepackage{color,soul}
\usepackage{dblfloatfix}
%%\usepackage[dvipsnames]{xcolor}
\usepackage[table,xcdraw]{xcolor}
\usepackage[listings]{tcolorbox}
\usepackage{xcolor}
\def\BibTeX{{\rm B\kern-.05em{\sc i\kern-.025em b}\kern-.08em
    T\kern-.1667em\lower.7ex\hbox{E}\kern-.125emX}}
\usepackage[table]{xcolor}
\usepackage{enumitem}
\usepackage{hyperref}
\usepackage{xcolor}
\usepackage{fontawesome}
\usepackage{multirow}
\usepackage{subcaption}
\usepackage{balance}

\usepackage{pgfplots}
\usepackage{pgfplotstable}
\pgfplotsset{compat=1.7}
\usepackage{tikz}
\hypersetup{
  colorlinks   = True, %Colours links instead of ugly boxes
  urlcolor     = blue, %Colour for external hyperlinks
  linkcolor    = blue, %Colour of internal links
  citecolor    = blue %Colour of citations
}

\title{Engineering Software Systems for Quantum Computing as a Service: A Mapping Study}

\author{
\IEEEauthorblockN{Aakash Ahmad\IEEEauthorrefmark{1}, Muhammad Waseem\IEEEauthorrefmark{2}, Peng Liang\IEEEauthorrefmark{3}, Mahdi Fehmideh\IEEEauthorrefmark{4}}
\IEEEauthorblockN{Arif Ali Khan\IEEEauthorrefmark{5}, David Georg Reichelt\IEEEauthorrefmark{1}, Tommi Mikkonen\IEEEauthorrefmark{2}}
\IEEEauthorblockA{\IEEEauthorrefmark{1} School of Computing and Communications, Lancaster University Leipzig, Leipzig, Germany}
\IEEEauthorblockA{\IEEEauthorrefmark{2} Faculty of Information Technology, University of Jyväskylä, Jyväskylä, Finland}
\IEEEauthorblockA{\IEEEauthorrefmark{3} School of Computer Science, Wuhan University, Wuhan, China}
\IEEEauthorblockA{\IEEEauthorrefmark{4} School of Business at University of Southern Queensland, Queensland, Australia}
\IEEEauthorblockA{\IEEEauthorrefmark{5} M3S Empirical Software Engineering Research Unit, University of Oulu, Oulu, Finland}
\IEEEauthorblockA{a.ahmad13@lancaster.ac.uk, mwaseem@jyu.fi, liangp@whu.edu.cn, mahdi.fahmideh@usq.edu.au}
\IEEEauthorblockA{arif.khan@oulu.fi, d.g.reichelt@lancaster.ac.uk, tommi.j.mikkonen@jyu.fi}
}

\begin{document}
\maketitle

\begin{abstract}
Quantum systems have started to emerge as a disruptive technology and enabling platforms – exploiting the principles of quantum mechanics – to achieve quantum supremacy in computing. Academic research, industrial projects (e.g., Amazon Braket), and consortiums like `Quantum Flagship’ are striving to develop practically capable and commercially viable quantum computing (QC) systems and technologies. Quantum Computing as a Service (QCaaS) is viewed as a solution attuned to the philosophy of service-orientation that can offer QC resources and platforms, as utility computing, to individuals and organisations who do not own quantum computers. To understand the quantum service development life cycle and pinpoint emerging trends, we used evidence-based software engineering approach to conduct a systematic mapping study (SMS) of research that enables or enhances QCaaS. The SMS process retrieved a total of 55 studies, and based on their qualitative assessment we selected 9 of them to investigate (i) the functional aspects, design models, patterns, programming languages, deployment platforms, and (ii) trends of emerging research on QCaaS. The results indicate three modelling notations and a catalogue of five design patterns to architect QCaaS, whereas Python (native code or frameworks) and Amazon Braket are the predominant solutions to implement and deploy QCaaS solutions. %Emerging trends indicate potentially futuristic research on quantum significant requirements, model-driven/low-code development, architectural tactics, services simulation, and human roles for process-centric development of QCaaS. 
From the quantum software engineering (QSE) perspective, this SMS provides empirically grounded findings that could help derive processes, patterns, and reference architectures to engineer software services for QC.
\end{abstract}

\begin{IEEEkeywords}
 Quantum Software Engineering, Quantum Service Computing, Systematic Mapping Study, Software Services.
\end{IEEEkeywords}

% Importance and appeal of children's drawings
Children's depictions of the human figure are highly expressive and varied.
As one of the very first subjects children attempt to draw, the representation begins as an almost unintelligible cloud of scribbles. 
As the child grows, their representation of the human figure becomes more developed and is extended to graphically represent many different types of characters: people, animals, and even personified objects (see Figure 1).

Who among us has not wished, either as a child or as an adult, to see such figures come to life and move around on the page?
Sadly, while it is relatively fast to produce a single drawing, creating the sequence of images necessary for animation is a much more tedious endeavor, requiring discipline, skill, patience, and sometimes complicated software.
As a result, most of these figures remain static upon the page.

% We built a system to animate them.
Inspired by the importance and appeal of the drawn human figure, we design and build a system to automatically animate it given an in-the-wild photograph of a child's drawing. 
Our system is fast, intuitive, and robust to much of the variation present in these types of drawings, making it well-suited to allow our target audience--children--to see their own characters coming to life.
The system is comprised of four stages: figure detection, segmentation masking, pose estimation/rigging, and animation. 
We describe each stage and identify common causes of failure in each. 
For object detection and pose estimation, we make use of existing computer vision models designed to detect human figures and joints in photographs; we fine-tune these models for use with children's drawings.
For segmentation, we present a straightforward, image processing-based method that, for animation purposes, is more useful and accurate than segmentation masks obtained from a fine-tuned object detection model.
During the animation step, we take advantage of the \textit{twisted perspective} commonly seen in children’s drawings to retarget motion capture data onto the character in a novel and appealing way.

% We use existing machine learning models. However, given the wide domain gap it's not clear how much fine-tuning data was needed. So we ran some experiments to find out and report it.
While our system leverages existing models and techniques, most are not directly applicable to the task due to the many differences between photographic images and simple pen and paper representations. 
To this end, we couple the presentation of our system with a set of experiments exploring the relationship between fine-tuning training set size and success rates.
We also include a perceptual study validating viewer preference for incorporating \textit{twisted perspective} into the motion retargeting step.

We validate the desirability and appeal of our system by building and publicly releasing a version of it as the \AD Demo \,\cite{animateddrawings}.
Launched in December 2021, this demo has been used by millions of people around the world to animate their children's drawings.
Inspired by this reception, our second contribution is The Amateur Drawings Dataset: \hjs{180,000 drawings and user-accepted annotations collected, with consent, through the demo. See Section \ref{sec:UI} for a description of how the annotations were generated.}
We believe this dataset will be a resource to researchers from various fields seeking to better understand the space of amateur drawings, evaluate new algorithms in this domain, or develop new drawing-based tools in general.

To summarize, our contributions are as follows:
\begin{enumerate}
    \item 
    We explore the problem of automatic sketch-to-animation for children's drawings of human figures and present a framework that achieves this effect. We also present a set of experiments determining the amount of training data necessary to achieve high levels of success and a perceptual study validating the usefulness of our motion retargeting technique.
    \item To encourage additional research in the domain of amateur drawings, we present a first-of-its-kind dataset of 180,000 user-submitted amateur drawings, along with user-accepted bounding box, segmentation mask, and joint location annotations.
\end{enumerate}

Upon acceptance of this paper, we plan to publicly release the Amateur Drawings Dataset, project code, and fine-tuned model weights.

\section{Research Context and Method}
\label{ResearchContextMethod}
We now contextualise service-orientation for QCs in Section \ref{sec:context} and present the research method in Section \ref{sec:method}. %Concepts and terminologies introduced in this section are used throughout the paper.

\subsection{Context: Service-Orientation for Quantum Computing}
\label{sec:context}
\subsubsection{Quantum Computing Systems}
We briefly overview a QC system that comprises of quantum hardware and software elements as shown in Figure \ref{Fig-1:Context}a). Fundamental to achieving quantum computations are Quantum Bits (QuBits) that represent the basic unit of quantum information processing by manipulating Quantum Gates (QuGates) \cite{R2_harrow2017quantum} \cite{R13_monroe2018quantum}. 
Traditional Binary Digits (Bits) in classical systems (i.e., digital computers) are represented as [1, 0] where 1 represents the computation state as \textsc{On} and 0 represents the state as \textsc{Off} to manipulate binary gates in digital circuits. In comparison, a QuBit represents a two-state quantum computer expressed as $|0\rangle$ and $|1\rangle$. The state of a single Qubit can be expressed as  $|0\rangle = \begin{bmatrix} 1 \\ 0 \end{bmatrix}$ and $|1\rangle = \begin{bmatrix} 0 \\ 1 \end{bmatrix}$ and quantum superposition allows a QuBit to attain a liner combination of both states: 

\begin{equation}\label{EQ-1}
 |0\rangle  =  \left[ \begin{array}{c} 1 \\ 0 \end{array} \right] ~~~~~ + ~~~~~  |1\rangle  =  \left[ \begin{array}{c} 0 \\ 1 \end{array} \right]   
\end{equation}

Based on Figure \ref{Fig-1:Context}a), we distinguish between a Bit and QuBit such that a Bit can take a value as either \textsc{`Off:0’} or \textsc{`On:1’} with 100\% probability. In comparison, a QuBit can be in a state of $|0\rangle$ or $|1\rangle$ or in a superposition state with 50\% $|0\rangle$ and 50\%  $|1\rangle$. In addition, two QuBits can be entangled, and entangled QuBits are linked in a way that observing (i.e., measuring) one of the QuBits, can reveal the state of the other QuBit. Extended details about QuBits and QuGates to develop and operate the QC systems are reported in studies like \cite{R1_ali2022software} and \cite{R9_moguel2022quantum}. To utilise the quantum computing resources, such as quantum processor and memory, there is a need for control software that can program QuBits to manage QuGates of a QC system. Quantum software systems rely on quantum source code compilers that allow quantum algorithm designers and programmers to write, build, and execute software for quantum computers. For example, a programmer can use a quantum programming language, such as Q\# (by Microsoft) or Qiskit (by Google) and use a quantum compiler to enable programable quantum computations \cite{R4_dyakonov2019will} \cite{R15_de2022software}. Software systems that can manage and control quantum hardware find their applications in areas including but not limited to quantum cryptography, bio-inspired computing, and quantum information processing \cite{R1_ali2022software}. However, scarcity of quantum hardware, lack of quantum software professionals, and economics of owning or maintaining QC are some critical factors that impede commercially viable quantum computers \cite{R2_harrow2017quantum} \cite{R3_egger2020quantum}. Vendors who offer QCaaS platforms view quantum service-orientation as an opportunistic business model that offers QC resources to customers as utility computing \cite{R6_wei2010service} \cite{R9_moguel2022quantum}.

\begin{figure}[]
 \centering
 \includegraphics[scale=0.45]{Images/Overview.pdf} 
 	\caption{An Overview of the Quantum Service-Orientation}
	\label{Fig-1:Context}
\end{figure}


\subsubsection{Service-Oriented Computing}
Services computing follows the SOA style that allows service users to discover and utilise a multitude of available software services that encapsulate computing resources and applications offered by service vendors/providers \cite{R7_bouguettaya2017service}. Figure \ref{Fig-1:Context}b) shows SOA style-based quantum servicing where a QC user (i.e., service requester) can utilise the QC resources offered by quantum vendors (i.e., service provider) by means of quantum services. In most cases, QC systems of today are not capable of executing quantum algorithms wrapped with large amounts of data, inputs, and outputs \cite{R2_harrow2017quantum} \cite{R13_monroe2018quantum}. As shown in Figure \ref{Fig-1:Context}a), large volumes of data in quantum algorithms require more QuBits and complex QuGates that result in deep quantum circuiting and consequently increased errors referred to as noisy intermediate-scale quantum (NISQ) \cite{R4_dyakonov2019will}. To address the issues like NISQ, the classic-quantum split pattern slices the overall quantum software or application into classical modules (pre/post-processing) and quantum modules (quantum computation) that result in hybrid applications\cite{R12_valencia2022quantum}. One of the prime examples of classic-quantum split patterns is Shor’s algorithm which involves quantum computations for finding the prime factors of an integer with its application in computer security and cryptography. %Quantum computing vendors such as Amazon, IBM, and Google have started to offer their QC systems and infrastructures to be utilised by individuals and organisations by means of quantum services computing \cite{R9_moguel2022quantum}. %For example, Amazon Braket allows quantum computing services to build, execute, and simulate quantum software based on pay-per-shot model. %Ongoing research and development is focused on offering algorithms, hardware, simulations, and mathematical problems as services for quantum platforms [11].
%%To harness QC as utility computing, there is a need to tailor existing principles and methods of service-orientation or develop new architectures, frameworks, and empirically grounded processes to synergise QC and SOA in the context QaaS.
Quantum service-orientation when viewed from a utility computing perspective can minimise the quantum divide, a prevailing issue highlighted at the World Economic Forum 2023, between states/entities that own or lack QC systems, technologies, and infrastructures \cite{R24_WorldEconomicForum}. 

\subsection{Research Method for the SMS}\label{sec:method}
We now discuss the research method, driven by three phases as illustrated in Figure \ref{Researchmethod}, based on the guidelines to conduct the SMS \cite{R16_petersen2008systematic}.% Extended details of the research method as the SMS protocol are provided in \cite{18_SMSProtocol}.

\subsubsection{\textsf{Phase I} – Specifying the Research Questions}
Research questions (RQs) are fundamental to conducting the SMS and documenting the results. We outlined two RQs for this SMS.

%\textbf{RQ-1}: \textit{What solutions are reported in literature to support Quantum as a Service Computing?}

\begin{tcolorbox} [sharp corners, boxrule=0.1mm,]
\small
\textbf{RQ-1}: What solutions are reported in the literature to support the development of quantum computing as a service?
\end{tcolorbox}


\textbf{Objective(s)} - To investigate state-of-the-art in terms of existing solutions that enable or enhance QCaaS computing. A multi-perspective analysis can reveal the functionality offered by the solutions, modeling languages and patterns to design the solutions, and programming technologies along with deployment platforms to implement and operationalise the solutions. 

%\textbf{RQ-2}: \textit{What are the emerging trends of research on Quantum as a Service Computing?}

\begin{tcolorbox} [sharp corners, boxrule=0.1mm,]
\small
\textbf{RQ-2}: What are the emerging trends of research on quantum computing as a service?
\end{tcolorbox}


\textbf{Objective(s)} - To identify and discuss the emerging trends that can help pinpoint the prevalent challenges and their solutions as dimensions of potentially futuristic research on QCaaS. The emerging trends can help to provide a road map for progressing research and development on QCaaS.



\vspace{0.5em}

\begin{table*}[b!] 
\scriptsize
\caption{Criteria for Screening and Qualitative Assessment of Selected Studies}
\begin{center}
{\tiny}
\begin{tabular}{|l|}
\hline
\rowcolor[HTML]{F2F2F2} 
\multicolumn{1}{|c|}{\cellcolor[HTML]{F2F2F2}\textbf{Study Selection Step I -   Screening of Identified Studies}}                                                                                \\ \hline
S1 - The study does not discuss any   solution or proposal for quantum computing as a service                                                                                                    \\ \hline
S2 - The study is not reported in  English                                                                                                                                                      \\ \hline
S3 - The study is a duplicate study. Duplicate studies are studies with overlapping contents, e.g., a conference paper extended as a journal  article.                                           \\ \hline
S4 - The study is a secondary  study/survey paper                                                                                                                                               \\ \hline
\rowcolor[HTML]{BDD6EE} 
\begin{tabular}[c]{@{}l@{}}Exclude the study if the answer to any of the criteria in Step I (S1 - S4) results in Yes, otherwise, \\ Include the study for quality assessment in Step II\end{tabular}  \\ \hline
\rowcolor[HTML]{F2F2F2} 

\multicolumn{1}{|c|}{\cellcolor[HTML]{F2F2F2}\textbf{Study Selection Step II - Quality Assessment of the Identified Studies}}                                                                    \\ \hline
Q1 - Study objectives and  Contributions are clear? {[}Yes = 1, Partially = 0.5, No = 0{]}                                                                                                      \\ \hline
Q2 - Research method to conduct  the study is reported {[}Yes = 1, Partially = 0.5, No = 0{]}                                                                                                   \\ \hline
Q3 - Design and/or implementation  details of solution are provided {[}Yes = 1, Partially = 0.5, No = 0{]}                                                                                      \\ \hline
Q4 - Details for   Experiments/Evaluation/Demonstration of Solution are provided {[}Yes = 1,  Partially = 0.5, No = 0{]}                                                                        \\ \hline
Q5 - Study limitations and needs  for future research are discussed {[}Yes = 1, Partially = 0.5, No = 0{]}                                                                                      \\ \hline
\rowcolor[HTML]{BDD6EE} 
\begin{tabular}[c]{@{}l@{}}Exclude the study that has a quality assessment score (Q1 – Q5) less than 2.0, otherwise  \\ Include the study for review and data extraction in Table 2\end{tabular} \\ \hline
\end{tabular}
\end{center}
\label{tab:qualitycriteria}
\end{table*}
\subsubsection{\textsf{Phase II} – Identifying and Selecting the Literature for SMS}

Based on the guidelines for literature search, we formulated a generic string to be executed on prominent Electronic Data Sources (EDS) \cite{R17_chen2010towards} including IEEE Xplore, ACM Digital Library, Springer Link, Science Direct, Springer Link, and Wiley Online Library. Google Scholar was used as a complementary EDS to ensure that we did not miss any relevant study for selection. The search string presented in Figure \ref{Researchmethod} is generic that combines logical operations \textsc{AND}, \textsc{OR}) to compose the key terms (e.g., Quantum \textsc{AND} Service \textsc{OR} Cloud), customized for each EDS individually. Customised search strings are provided as part of the SMS protocol \cite{18_SMSProtocol}. We conducted a pilot search to assess the need for any customisation to the search string(s) or any filters applied on specific EDS to avoid an exhaustive search resulting in a significant number of unrelated studies. For example, we limited our search on IEEE Xplore from `Full Text \& Metadata’ to `Document Title’ as searching for our defined key terms in full text and metadata yielded a significant amount of irrelevant studies (e.g., cloud services, quantum hardware). %The pilot search indicated limiting the years of publications (2011-2022) as no evidence of published research on QCaaS was found before 2011.






\textbf{Screening and Quality Assessment:} By executing the customised search strings on five selected EDS, the SMS process retrieved a total of 55 potentially relevant studies. To complement the automated search on EDS, we applied the forward snowballing process [17], as a manual effort. The forward snowballing approach involves looking up the references or bibliography sections of 55 studies, referred to as the seed set in snowballing, to see if any relevant cited literature can be found.  The forward snowballing helped us to identify a total of 13 studies resulting in a total of 68 studies (55: EDS and 13: snow-balling). To assess and select the studies for review, we performed study screening based on criteria (Step 1: S1 – S4) in Table \ref{tab:qualitycriteria}. Most of the studies identified during the snowballing failed the screening criteria S3 and S4 which means either the studies were duplicate studies or secondary/survey studies that cannot be included in the review. Based on the screening of identified studies, more specifically reading through the titles, abstracts, and conclusions we shortlisted a total of 11 potentially relevant studies to be qualitatively assessed (Step II: Q1 – Q5) for their inclusion in the review for SMS. Based on the quality assessment, we excluded 2 studies to finally select a total of 9 studies to be included in the review. The list of selected studies for the SMS is provided in \textbf{Appendix A}. 

\begin{figure}[]
 \centering
 \includegraphics[scale=0.83]{Images/Researchmethod.pdf} 
 	\caption{Overview of the Research Method}
	\label{Researchmethod}
\end{figure}

\vspace{0.5em}
\subsubsection{\textsf{Phase III} – Documenting the Results}
To document the results, i.e., answering RQs objectively, we extracted the data from the selected studies in Appendix A and documented it using a structured format, having seven criteria, in Table \ref{tab:criteria}. The criteria focus on conceptualising, designing, developing, and deploying quantum software services by following the IBM SOA foundation life cycle (SOA life cycle for short) \cite{R20_keen2006patterns}. To contextualise QCaaS from Figure \ref{Fig-1:Context} and to ensure fine-grained analysis of SMS data, we have divided the `Model' activity from SOA life cycle into two activities namely Conception and Model to distinguish between functional needs (conception) and representation (modeling) of quantum service design. Model represents the conception as the design specification of functional needs for quantum services. We do not have the `Manage' phase from SOA life cycle as we could not find any evidence in the literature that supports identity, compliance, and business metrics management of quantum services.


\begin{itemize}
\item \textbf{Conception} as the initial activity in the service life cycle aims to conceptualise the functional aspects of quantum services by capturing the details of required functionality, i.e., functional requirements. Conception aims to identify: \textit{what are the business needs of a quantum service?}

\item \textbf{Model} activity aims to translate the conception into a design that acts as a blueprint for implementing quantum services. Model focuses on: \textit{how to represent the conception as the design of the solution?}

\item \textbf{Assemble} focuses on implementing the design to produce concrete, i.e., executable specification of quantum services. Assemble aims to address: \textit{how to implement the design as executable services?}

\item \textbf{Deploy} as the last activity aims to deploy the assembled solution for operationalisation and usage of quantum services. Deploy focuses on: \textit{what platforms can be used to deploy the assembled (implemented) solution?}
\end{itemize}

\vspace{0.5em}

\subsubsection{Threats to the Validity of SMS}
Systematic literature reviews and mapping studies are prone to a number of validity threats that refer to deviation, limitations, or invalidation of study results when applied to a theoretical or practical context. %We highlight three main types of threats and discuss our efforts to minimise them. 
\textbf{Construct validity} of the SMS corresponds to the rigor of study protocol and methodological details to extract, analyze, and synthesise the data to objectively answer the RQs and present the data systematically. To avoid this threat, i.e., avoiding the bias in data extraction and documentation, we applied well-practiced guidelines \cite{R16_petersen2008systematic, R17_chen2010towards}, derived the search strings (Figure \ref{Researchmethod}), and devised a structured format (Table \ref{tab:criteria}) to collect and present the data consistently. \textbf{Internal validity} examines SMS design, conduct, and analysis to answer the RQs without bias. To minimize this threat, we synthesized the data based on the well-known IBM SOA life cycle \cite{R20_keen2006patterns} that structures the results into fine-grained life cycle activities. We documented the results while performing a quality assessment (Table \ref{tab:qualitycriteria}) and a well-defined service life cycle template. \textbf{External validity} of the SMS refers to the extent to which the findings of study can be generalised/externalised to research and development projects. It is challenging to foresee and outline the predictive implications of the study results. We have outlined the implications and generalization of study findings (Table \ref{tab:criteria}, Figure \ref{Results}, Figure \ref{FutureResearch}) can provide the basis for creating a reference architecture for QCaaS as future work. The documented results are discussed in Section \ref{sec:RQ1} (RQ-1) and Section \ref{sec:RQ2} (RQ-2).


% \begin{table*}[ht]
% \caption{Data Extracted for SMS from Reviewed Studies (SOA lifecycle activities [20])}
% \begin{centering}
% {\tiny{}}%
% \begin{tabular}{|lllllll|}
% \hline
% \multicolumn{7}{|c|}{\includegraphics[scale=0.7]{Images/QaaS-Table.pdf}} \\ \hline
% \multicolumn{1}{|l|}{} & \multicolumn{1}{c|}{\cellcolor[HTML]{BDD6EE}\textbf{Conception}} & \multicolumn{2}{c|}{\cellcolor[HTML]{BDD6EE}\textbf{Modeling}} & \multicolumn{2}{c|}{\cellcolor[HTML]{BDD6EE}\textbf{Assembly}} & \multicolumn{1}{c|}{\cellcolor[HTML]{BDD6EE}\textbf{Deployment}} \\ \cline{2-7} 
% \multicolumn{1}{|l|}{\multirow{-2}{*}{Study ID}} & \multicolumn{1}{c|}{\cellcolor[HTML]{E7E6E6}\begin{tabular}[c]{@{}c@{}}Functional\\ Aspects\end{tabular}} & \multicolumn{1}{c|}{\cellcolor[HTML]{E7E6E6}\begin{tabular}[c]{@{}c@{}}Modeling \\ Notation\end{tabular}} & \multicolumn{1}{c|}{\cellcolor[HTML]{E7E6E6}\begin{tabular}[c]{@{}c@{}}Software \\ Pattern\end{tabular}} & \multicolumn{1}{c|}{\cellcolor[HTML]{E7E6E6}\begin{tabular}[c]{@{}c@{}}Service \\ Use case\end{tabular}} & \multicolumn{1}{c|}{\cellcolor[HTML]{E7E6E6}\begin{tabular}[c]{@{}c@{}}Service   \\ Programming\end{tabular}} & \multicolumn{1}{c|}{\cellcolor[HTML]{E7E6E6}\begin{tabular}[c]{@{}c@{}}Quantum\\ Platform/Vendor\end{tabular}} \\ \hline
% \multicolumn{1}{|l|}{} & \multicolumn{1}{l|}{} & \multicolumn{1}{l|}{} & \multicolumn{1}{l|}{} & \multicolumn{1}{l|}{} & \multicolumn{1}{l|}{} &  \\ \hline
% \multicolumn{1}{|l|}{} & \multicolumn{1}{l|}{} & \multicolumn{1}{l|}{} & \multicolumn{1}{l|}{} & \multicolumn{1}{l|}{} & \multicolumn{1}{l|}{} &  \\ \hline
% \multicolumn{1}{|l|}{} & \multicolumn{1}{l|}{} & \multicolumn{1}{l|}{} & \multicolumn{1}{l|}{} & \multicolumn{1}{l|}{} & \multicolumn{1}{l|}{} &  \\ \hline
% \multicolumn{1}{|l|}{} & \multicolumn{1}{l|}{} & \multicolumn{1}{l|}{} & \multicolumn{1}{l|}{} & \multicolumn{1}{l|}{} & \multicolumn{1}{l|}{} &  \\ \hline
% \end{tabular}
% \par\end{centering}
% \label{tab:criteria}
% \end{table*}







\section{Engineering Software for QCaaS (RQ-1)}
\label{sec:RQ1}
We now discuss the existing solutions, reported in the literature, that support the development of quantum services to operationalise QCaaS solutions. The data extracted from the selected studies is presented in Table \ref{tab:criteria} and visualised in Figure \ref{Results}. Table \ref{tab:criteria} can be viewed as a catalogue that organises a summary of the core findings to answer RQ-1 based on the four activities of the SOA life cycle (Phase III, Figure \ref{Researchmethod}). %Figure \ref{Results} complements Table \ref{tab:criteria} with an illustrative example of the quantum service life cycle in Figure \ref{tab:criteria} a) and highlighting the core findings in Figure \ref{tab:criteria} b) - g). 

\vspace{0.5em}

\textbf{Illustrative Example}: Figure \ref{Results}a) exemplifies the service life cycle with \textit{functional aspects} that requires a quantum service to compute the prime factors of an integer. Functional aspects need \textit{modeling} and that uses Unified Modeling Language (UML) component diagram \cite{R21_perez2020towards} as the modeling notation to specify computational elements (components) and their interconnections (connectors) in a service. The Classic-Quantum split pattern \cite{R12_valencia2022quantum} is applied to slice the functionality between a classical computer (pre-/post-processing, e.g., \textsf{Gen\_Num} component) and a quantum computer (prime factorization, \textsf{Factorize} component). The model acts as a blue-print to support \textit{assembly} of a services using a programming language (Qiskit code snippet) that converts the design specifications of a service to its executable specifications. Finally, \textit{deployment} activity is shown as UML deployment diagram to configure the assembled service on a quantum computer provided by the quantum vendor (Amazon Braket). 
%We now present each of the six aspects of results visualised in Figure 3 b) – g) in the context of four activities of service lifecycle. 

\subsection{\textbf{Conception}: \textsf{Functional Aspects}} 
During the conception activity, functional aspects of a service relate to identifying and outlining the functionality to be offered by a quantum service. The functionality can be achieved via service execution on a quantum computer (provider), whereas the service is developed or invoked by the user (requester). Figure \ref{Fig-1:Context} shows that, in order to exploit QC platforms for quantum functionality, service requesters can develop/discover new and/or available services, such as quantum simulation or quantum cryptography using the SOA patterns \cite{R7_bouguettaya2017service}. In the SMS, we identified a multitude of functional aspects for quantum services and organized them into five categories namely \textit{experimental}, \textit{service delivery}, \textit{number crunching}, \textit{data searching}, and \textit{natural computing} as shown in Table \ref{tab:criteria}. For example, Figure \ref{Results}a) highlights the generic functional aspect of number crunching that involves prime factorisation of integers for Shor's algorithm \cite{R1_ali2022software}\cite{R2_harrow2017quantum}.  

The diversity of existing functional aspects is proportional to the capability of the current era of quantum computers that is considred as limited due to a number of factors such as simplistic quantum circuitry (e.g., less QuBits/QuGates) and quantum errors (e.g., NISQ) \cite{R4_dyakonov2019will}. Quantum software services that support functional aspects of QC are merely capable of checking the status of quantum circuits (experimental) or generation of random numbers using quantum hardware (number crunching). Functional aspects reflect only a partial view of system design in terms of offered functionality that should not overlook the non-functional or quality aspects of the QCaaS solutions. For example, resource efficiency in terms of utilizing minimal available QuBits to generate prime factors can ensure the required functionality and desired quality (e.g., service efficiency, execution performance) of QCaaS.  

\begin{tcolorbox} [sharp corners, boxrule=0.1mm,]
\faEdit \scriptsize{~\textsf{\textbf{Functional aspects} of quantum services, in general, are rather limited to basic quantum experimentation and numerical processing. The limitation reflects the existing capabilities of QCs and consequently the offered services by quantum vendors. Investigating the non-functional aspects can help outline the quantum significant requirements (QSRs) in terms of required functionality and desired quality of the service that currently lacks in the existing literature.}}
\end{tcolorbox}
   
   
\begin{figure*}[ht]
 \centering
% \includegraphics[width =14cm, height = 11cm]{Images/Result.pdf}
 \includegraphics[scale =0.75]{Images/Results.pdf} 
 	\caption{An Overview of the Mapping Study Results}
	\label{Results}
\end{figure*}

\subsection{\textbf{Modeling}: \textsf{Notations and Patterns}} 
%The terms \footnote{\textit{pattern} and \textit{style} in software engineering context are virtually synonymous and are often used interchangeably. A distinction should be maintained: (i) style is reusable structuring as styling of architecture (e.g., microservices architecture), and (ii) pattern represent recurring solution to design problems (e.g., service façade).}

During service modeling, modeling notations, such as Q-UML or ontological models can help create a blueprint for the implementation of functional aspects \cite{R21_perez2020towards}. Patterns can complement the notations by providing reusable knowledge and design rationale to architect the quantum services \cite{R10_garcia2021quantum} \cite{R12_valencia2022quantum}.

\begin{table*}[]
\caption{Data Extracted for SMS from Reviewed Studies (SOA lifecycle activities [20])}
\begin{centering}
{\tiny
\begin{tabular}{|c|c|cc|cc|c|}
\hline
\multicolumn{7}{|c|}{\includegraphics[scale=0.6]{Images/QaaS-Table.pdf}} \\ \hline
\multicolumn{1}{|l|}{} & \multicolumn{1}{c|}{\cellcolor[HTML]{BDD6EE}\textbf{Conception}} & \multicolumn{2}{c|}{\cellcolor[HTML]{BDD6EE}\textbf{Modeling}} & \multicolumn{2}{c|}{\cellcolor[HTML]{BDD6EE}\textbf{Assembly}} & \multicolumn{1}{c|}{\cellcolor[HTML]{BDD6EE}\textbf{Deployment}} \\ \cline{2-7}
\multirow{-2}{*}{\textbf{\begin{tabular}[c]{@{}c@{}}Study ID \\ \& \\ Research Type\end{tabular}}} &
  \cellcolor[HTML]{F2F2F2}\begin{tabular}[c]{@{}c@{}}Functional \\ Aspects\end{tabular} &
  \multicolumn{1}{c|}{\cellcolor[HTML]{F2F2F2}\begin{tabular}[c]{@{}c@{}}Modelling \\ Notation\end{tabular}} &
  \cellcolor[HTML]{F2F2F2}\begin{tabular}[c]{@{}c@{}}Software\\ Pattern\end{tabular} &
  \multicolumn{1}{c|}{\cellcolor[HTML]{F2F2F2}\begin{tabular}[c]{@{}c@{}}Service \\ Use case\end{tabular}} &
  \cellcolor[HTML]{F2F2F2}\begin{tabular}[c]{@{}c@{}}Service\\ Programming\end{tabular} &
  \cellcolor[HTML]{F2F2F2}\begin{tabular}[c]{@{}c@{}}Quantum\\ Platform/\\ Vendor\end{tabular} \\ \hline
\rowcolor[HTML]{FFFFFF} 


\begin{tabular}[c]{@{}c@{}}{\includegraphics[scale=0.11]{Images/icons/1.jpg}} \\ {[}S1{]}\end{tabular} &
  \begin{tabular}[c]{@{}c@{}}{\includegraphics[scale=0.04]{Images/icons/Empirical.png}}\\    Quantum   Service \\ Delivery\end{tabular} &
  \multicolumn{1}{c|}{\cellcolor[HTML]{FFFFFF}\begin{tabular}[c]{@{}c@{}}{\includegraphics[scale=0.04]{Images/icons/UMLS.png}}   \\ Deployment\\ Diagram\end{tabular}} &
  \begin{tabular}[c]{@{}c@{}}{\includegraphics[scale=0.04]{Images/icons/API.png}}\\ API Gateway\end{tabular} &
  \multicolumn{1}{c|}{\cellcolor[HTML]{FFFFFF}\begin{tabular}[c]{@{}c@{}}{\includegraphics[scale=0.04]{Images/icons/oPTIMISATIOPN.png}} \\ Optimal Service \\ Provider\end{tabular}} &

  \begin{tabular}[c]{@{}c@{}}{\includegraphics[scale=0.15]{Images/icons/flask1.png}} \\ Python, Flask \end{tabular} &
  
  \begin{tabular}[c]{@{}c@{}}{\includegraphics[scale=0.20]{Images/icons/amazon.png}}\end{tabular} \\ \hline
\rowcolor[HTML]{FFFFFF} 
\begin{tabular}[c]{@{}c@{}}{\includegraphics[scale=0.04]{Images/icons/opinion.png}}\\   {[}S2{]}\end{tabular} &
  \begin{tabular}[c]{@{}c@{}}{\includegraphics[scale=0.07]{Images/icons/ServiceDelivery.png}}   \\ Enterprise  Services \\ Development\end{tabular} &
  \multicolumn{1}{c|}{\cellcolor[HTML]{FFFFFF}\begin{tabular}[c]{@{}c@{}}{\includegraphics[scale=0.04]{Images/icons/bpmn.png}}\\   Business Process\end{tabular}} &

  \begin{tabular}[c]{@{}c@{}}{\includegraphics[scale=0.07]{Images/icons/layer.png}}  \\ Layered\\ Architecture\end{tabular} &
  
  \multicolumn{1}{c|}{\cellcolor[HTML]{FFFFFF}\begin{tabular}[c]{@{}c@{}}{\includegraphics[scale=0.04]{Images/icons/process.png}}   \\ Process Automation\end{tabular}} &
  
  \begin{tabular}[c]{@{}c@{}}{\includegraphics[scale=0.07]{Images/icons/Noevidence.png}}\\No Evidence \end{tabular} &
  \begin{tabular}[c]{@{}c@{}}{\includegraphics[scale=0.07]{Images/icons/Noevidence.png}}\\No Evidence \end{tabular} \\ \hline
\rowcolor[HTML]{FFFFFF} 
\begin{tabular}[c]{@{}c@{}}{\includegraphics[scale=0.11]{Images/icons/1.jpg}}\\   {[}S3{]}\end{tabular} &
  \begin{tabular}[c]{@{}c@{}}{\includegraphics[scale=0.11]{Images/icons/numb.png}}   \\ Quantum Random \\ Number Generation\\ Quantum Search Algo\end{tabular} &
  \multicolumn{1}{c|}{\cellcolor[HTML]{FFFFFF}\begin{tabular}[c]{@{}c@{}}{\includegraphics[scale=0.04]{Images/icons/UML.png}}   \\ Class, Sequence \\ Diagram\end{tabular}} &

  \begin{tabular}[c]{@{}c@{}}{\includegraphics[scale=0.07]{Images/icons/split.png}}\\Classic-
Quantum Split\end{tabular} &
  
  \multicolumn{1}{c|}{\cellcolor[HTML]{FFFFFF}\begin{tabular}[c]{@{}c@{}}{\includegraphics[scale=0.07]{Images/icons/maths.png}} \\ Mathematics\end{tabular}} &
  
  \begin{tabular}[c]{@{}c@{}}{\includegraphics[scale=0.11]{Images/icons/qsharp1.png}} \\Q sharp\end{tabular} &

  
  \begin{tabular}[c]{@{}c@{}}{\includegraphics[scale=0.11]{Images/icons/ibm.png}} \\IBM Quantum\end{tabular} \\ \hline
\rowcolor[HTML]{FFFFFF} 
\begin{tabular}[c]{@{}c@{}}{\includegraphics[scale=0.04]{Images/icons/opinion.png}} \\ {[}S4{]}\end{tabular} &
  
  \begin{tabular}[c]{@{}c@{}}{\includegraphics[scale=0.11]{Images/icons/numb.png}}\\ Integer Factorisation\end{tabular} &
  
  \multicolumn{1}{c|}{\cellcolor[HTML]{FFFFFF}\begin{tabular}[c]{@{}c@{}}{\includegraphics[scale=0.05]{Images/icons/UMLS.png}}\\ Deployment Diagram\end{tabular}} &
  \begin{tabular}[c]{@{}c@{}}{\includegraphics[scale=0.11]{Images/icons/1.jpg}} \\ Solution\end{tabular} &
  
  \multicolumn{1}{c|}{\cellcolor[HTML]{FFFFFF}\begin{tabular}[c]{@{}c@{}}{\includegraphics[scale=0.07]{Images/icons/maths.png}} \\ Mathematics\end{tabular}} &
  
  \begin{tabular}[c]{@{}c@{}}{\includegraphics[scale=0.005]{Images/icons/Python_icon_black_and_white.png}}\\Python\end{tabular} &
  
  \begin{tabular}[c]{@{}c@{}}{\includegraphics[scale=0.20]{Images/icons/amazon.png}}\end{tabular} \\ \hline
  
  \rowcolor[HTML]{FFFFFF} 
\begin{tabular}[c]{@{}c@{}}{\includegraphics[scale=0.01]{Images/icons/validate.png}}  \\ {[}S5{]}\end{tabular} &
  \begin{tabular}[c]{@{}c@{}}{\includegraphics[scale=0.07]{Images/icons/ServiceDelivery.png}}\\ Experimental Quantum\\   Service Computing)\end{tabular} &
  
  \multicolumn{1}{c|}{\cellcolor[HTML]{FFFFFF}\begin{tabular}[c]{@{}c@{}}{\includegraphics[scale=0.07]{Images/icons/NoEvidences.png}}\\No Evidence\end{tabular}} &
  
  \begin{tabular}[c]{@{}c@{}}{\includegraphics[scale=0.09]{Images/icons/servicewrapper.png}}\\Service  
Wrapping \end{tabular} &
  
  \multicolumn{1}{c|}{\cellcolor[HTML]{FFFFFF}\begin{tabular}[c]{@{}c@{}}{\includegraphics[scale=0.04]{Images/icons/oPTIMISATIOPN.png}}\\Optimisation\end{tabular}} &

  \begin{tabular}[c]{@{}c@{}}{\includegraphics[scale=0.15]{Images/icons/flask1.png}}\\Python, Flask\end{tabular} &
  
  \begin{tabular}[c]{@{}c@{}}{\includegraphics[scale=0.20]{Images/icons/amazon.png}}\end{tabular} \\ \hline

\rowcolor[HTML]{FFFFFF} 
\begin{tabular}[c]{@{}c@{}}{\includegraphics[scale=0.11]{Images/icons/1.jpg}}\\   {[}S6{]}\end{tabular} &
  \begin{tabular}[c]{@{}c@{}}{\includegraphics[scale=0.07]{Images/icons/ServiceDelivery.png}}\\ Experimental Services\\ Algorithm\end{tabular} &
  \multicolumn{1}{c|}{\cellcolor[HTML]{FFFFFF}\begin{tabular}[c]{@{}c@{}}{\includegraphics[scale=0.04]{Images/icons/UML.png}}\\ Sequence Diagrams\end{tabular}} &
  
  \begin{tabular}[c]{@{}c@{}}{\includegraphics[scale=0.04]{Images/icons/API.png}}\\ API Gateway\end{tabular} &

  \multicolumn{1}{c|}{\cellcolor[HTML]{FFFFFF}\begin{tabular}[c]{@{}c@{}}{\includegraphics[scale=0.07]{Images/icons/ServiceDevelopment.jpg}}\\ Algorithm as \\ a Service\end{tabular}} &
 
  \begin{tabular}[c]{@{}c@{}}{\includegraphics[scale=0.005]{Images/icons/Python_icon_black_and_white.png}}\\Python \end{tabular} &
  
  \begin{tabular}[c]{@{}c@{}}{\includegraphics[scale=0.07]{Images/icons/rigetti.png}} \\ Rigetti\end{tabular} \\ \hline

\rowcolor[HTML]{FFFFFF} 
\begin{tabular}[c]{@{}c@{}}{\includegraphics[scale=0.11]{Images/icons/1.jpg}}\\   {[}S7{]}\end{tabular} &
  \begin{tabular}[c]{@{}c@{}}{\includegraphics[scale=0.07]{Images/icons/numb.png}}   \\ Integer Factorisation\end{tabular} &
  
  \multicolumn{1}{c|}{\cellcolor[HTML]{FFFFFF}\begin{tabular}[c]{@{}c@{}}{\includegraphics[scale=0.15]{Images/icons/grph.png}}\\Directed Graph\end{tabular}} &
  
  \begin{tabular}[c]{@{}c@{}}{\includegraphics[scale=0.04]{Images/icons/API.png}}\\ API Gateway\end{tabular} &
  
  \multicolumn{1}{c|}{\cellcolor[HTML]{FFFFFF}\begin{tabular}[c]{@{}c@{}}{\includegraphics[scale=0.04]{Images/icons/oPTIMISATIOPN.png}} \\Optimisation \end{tabular}} &
  
  \begin{tabular}[c]{@{}c@{}}{\includegraphics[scale=0.07]{Images/icons/Noevidence.png}}\\No Evidence\end{tabular} &
  
  \begin{tabular}[c]{@{}c@{}}{\includegraphics[scale=0.20]{Images/icons/amazon.png}}\end{tabular} \\ \hline

  
\rowcolor[HTML]{FFFFFF} 
\begin{tabular}[c]{@{}c@{}}{\includegraphics[scale=0.04]{Images/icons/opinion.png}}   \\ {[}S8{]}\end{tabular} &

  \begin{tabular}[c]{@{}c@{}}{\includegraphics[scale=0.03]{Images/icons/bio.png}}\\Bio-inspired\\
Computing\end{tabular} &
  
  \multicolumn{1}{c|}{\cellcolor[HTML]{FFFFFF}\begin{tabular}[c]{@{}c@{}}{\includegraphics[scale=0.07]{Images/icons/ontology.png}}\\Ontologies\end{tabular}} &
  
  \begin{tabular}[c]{@{}c@{}}{\includegraphics[scale=0.09]{Images/icons/repository.png}}\\Repository \\
Pattern\end{tabular} &
  
  \multicolumn{1}{c|}{\cellcolor[HTML]{FFFFFF}\begin{tabular}[c]{@{}c@{}}{\includegraphics[scale=0.04]{Images/icons/simulate.png}} \\Simulation\end{tabular}} &
  
  \begin{tabular}[c]{@{}c@{}}{\includegraphics[scale=0.07]{Images/icons/Noevidence.png}}\\No Evidence\end{tabular} &
  
  \begin{tabular}[c]{@{}c@{}}{\includegraphics[scale=0.07]{Images/icons/Noevidence.png}}\\No Evidence\end{tabular} \\ \hline

\rowcolor[HTML]{FFFFFF} 
\begin{tabular}[c]{@{}c@{}}{\includegraphics[scale=0.04]{Images/icons/opinion.png}}   \\ {[}S9{]}\end{tabular} &
 
  \begin{tabular}[c]{@{}c@{}}{\includegraphics[scale=0.11]{Images/icons/numb.png}} \\Integer \\Factorisation \end{tabular} &
  
  \multicolumn{1}{c|}{\cellcolor[HTML]{FFFFFF}\begin{tabular}[c]{@{}c@{}}{\includegraphics[scale=0.07]{Images/icons/Noevidence.png}}\\No Evidence\end{tabular}} &
  
  \begin{tabular}[c]{@{}c@{}}{\includegraphics[scale=0.09]{Images/icons/servicewrapper.png}} \\Service  
Wrapping\end{tabular} &
  
  \multicolumn{1}{c|}{\cellcolor[HTML]{FFFFFF}\begin{tabular}[c]{@{}c@{}}{\includegraphics[scale=0.07]{Images/icons/Noevidence.png}}\\No Evidence\end{tabular}} &
  
  \begin{tabular}[c]{@{}c@{}}{\includegraphics[scale=0.005]{Images/icons/Python_icon_black_and_white.png}}\\Python\end{tabular} &
  \begin{tabular}[c]{@{}c@{}}{\includegraphics[scale=0.20]{Images/icons/amazon.png}}\end{tabular} \\ \hline
\end{tabular}}
\par\end{centering}
\label{tab:criteria}
\end{table*}

\textit{Modeling notations} are fundamental to the creation, maintenance, and evolution of models such as ontological structures and graph-based diagrams that provide a visual representation, whereas architectural description languages support a textual specification for software-intensive systems \cite{R21_perez2020towards}. Recent trends in software engineering that promote model-driven and low-code application development have resulted in transitioning developers’ focus from coding to software modeling for implementation \cite{R22_raymer2019us}. Low code application development process leverages the principle and practices of model-driven engineering to utilise model(s) as first-class entities in software development \cite{R15_de2022software}\cite{R19_ahmadtowards}. Investigating software models and modeling notations that help create service models is essential to support model-driven perspective to QSE, consequently facilitating quantum code developers to abstract implementation-specific complexities, via model-driven QSE, while developing quantum services \cite{R19_ahmadtowards}. This SMS indicates three main types of notations to model services in QCaaS that include the Unified Modeling Language (UML), graph-based models, and process models highlighted in Figure \ref{Results}b). UML-based models are represented via a multitude of notations, such as class and component diagrams that represent the structure, while sequence and deployment diagrams represent runtime or behavioural view of QCaaS. Graph-based models contain directed graphs and ontologies, whereas process models rely on automating the business processes of an enterprise as quantum services. For example, the study [S3] reports a class diagram as a structural view of the system to represent the attributes and methods of entities (user, service provider, authentication etc.) of quantum computing as a service. %The study also demonstrates a sequence diagram representing behavioural view of quantum as a function service. %It is unsurprising to witness UML-based diagrams representing the majority of QaaS models. 
UML diagrams and profiles represent the status-quo in software modeling and are seen as the de-facto notation in the software and service ' community to model classical software with growing adoption in QSE \cite{R21_perez2020towards}.

\textit{Design patterns} represent a concentrated wisdom of software designers that can be leveraged to address design and implementation issues, addressing functionality and quality,   effectively and efficiently. Considering a lack of professional expertise in QSE (e.g., quantum domain engineers, quantum algorithm designers, quantum software architects etc.) patterns as artifacts of reuse can help novice developers during quantum software development to rely on existing best practices \cite{R1_ali2022software}\cite{R10_garcia2021quantum}. This SMS highlights that the literature on QCaaS  reports five patterns, namely the \textit{API Gateway}, \textit{Layered Architecture}, \textit{Classic-Quantum Split}, \textit{Service Wrapping}, and \textit{Repository Pattern}. Figure \ref{Results}c) depicts pattern thumbnails as an abstract view of the identified patterns. Patterns are generally documented as templates or pattern languages, here we only focus on overviewing the reported patterns for QCaaS, while details for pattern representation and documentation can be found in \cite{R19_ahmadtowards}. The Classic-Quantum Split pattern \cite{R12_valencia2022quantum} is a quantum version of the Splitter pattern, driven by quantum workflow, that splits computation tasks into tasks that can be generated and executed on classical machines (e.g., random number generation) and tasks that can be executed on quantum machines (e.g., prime factorisation). The pattern aims to address issues like NISQ by splitting quantum software into classical and quantum parts as a hybrid application \cite{R2_harrow2017quantum}\cite{R4_dyakonov2019will}. 

\begin{tcolorbox} [sharp corners, boxrule=0.1mm,]
\faEdit \scriptsize{~\textsf{\textbf{Modeling notations} can assist software engineers to transit their focus from implementation towards design perspective. Modeling can incrementally transform functional aspects to service models leading to service implementation via model-driven engineering or low-code development. \textbf{Patterns} (classical or quantum-specific) can facilitate developers to architect and implement quantum-age software services by relying on reusable knowledge and best practices of service-orientation.}}
\end{tcolorbox}

Recently, a number of studies have focused on organising quantum software patterns as a body of knowledge in QSE, however, there is no evidence of empirically-derived methods to discover and document patterns for quantum services computing. SOA-specific patterns like API Gateway and Service Wrapping patterns can be tailored to address QCaaS solutions. There is a need for mining repositories and knowledge resources to discover reusable knowledge and best practices from quantum software development projects that can to be documented as tactics and patterns for QCaaS solutions.


\subsection{\textbf{Assembly:} \textsf{Application Domain and Programming}} 
Assembling the quantum services involves identifying the application domains and exploiting the programming languages as implementation technologies to develop executable specifications from the service model \cite{R7_bouguettaya2017service} \cite{R9_moguel2022quantum}. 

\textit{Application Domain} is also referred to as the implementation use cases or practical context to which the QCaaS solutions can be applied. For example, quantum security represents an application domain for quantum software servicing where a service can be invoked to implement cryptography protocols to generate and manage a secure quantum key \cite{R13_monroe2018quantum}. The results of this SMS indicate four application domains, namely \textit{OptimiSation}, \textit{Process Automation}, \textit{Mathematics}, and \textit{Quantum Simulation}. The application domains may impact the selection of programming languages and tools for service implementation. For example, the study [S3] uses Q\# as the programming language that can be developed and compiled in Microsoft .Net framework for executing quantum algorithms.

\textit{Service Implementation} involves programming languages that represent a system of notation or source coding scripts for implementing quantum  services to manage and operationalise QC resources \cite{R15_de2022software} \cite{R21_perez2020towards}. In recent years, a number of Quantum Programming Languages (QPLs) including but not limited to Q\# by Microsoft or Cirq by Google have emerged to provide specialized programming syntax, framework, and environments to develop, execute, and deploy quantum source code. Insights into programming languages can reveal if classical programming languages (e.g., C, Java, Python etc.) suffice for QCaaS implementation or if there is need for more specialised QPLs (Q\#, Cirq etc.). The SMS results indicate three programming languages, namely Python, Java, and Q\#, as the preferred languages to implement quantum services. Based on the details of source coding, Figure \ref{Results}f) distinguishes between native code of a language and specialised libraries/application programming interfaces (APIs) being developed using a specific language. For example, the studies [S4, S5] used native Python code to implement quantum micro-servicing for experimentation. In comparison, the study [S1] used Flask as a Web framework written in Python to develop a solution for optimal delivery of quantum services on Amazon Braket. Python is the most preferred programming language both in terms of native code as well as specialised libraries of Python that include Flask and Qiskit as open-source language frameworks and Cirq which is adopted by Google. 

\begin{tcolorbox} [sharp corners, boxrule=0.1mm,]
\faEdit \scriptsize{~\textsf{\textbf{Application domains} represent the practical context/use cases of quantum services and may impact the selection of programming languages and tools for implementation. \textbf{Programming languages} provide a system of notation for source-coding of quantum services. Classical programming languages, such as Python represent a predominant choice over QPLs to implement QCaaS due to more comprehensive documentation and familiarity of Python in service developers' community.}}
\end{tcolorbox}
    
\subsection{\textbf{Deployment:} \textsf{Quantum Platform}} 
The deployment activity supports the selection of QC platforms on which services can be deployed for their operationalisation and execution. Platform providers also referred to as quantum vendors offer computing infrastructure in terms of hardware as well as software that allows service developers to develop and/or utilise the quantum services. Deployment represents the last activity in the SOA life cycle that is represented as a UML deployment diagram in Figure \ref{Results}a). This SMS identified a total of three quantum vendors for the deployment of quantum services, namely \textit{Amazon Bracket}, \textit{IBM Quantum}, and \textit{Rigetti}. Amazon Braket (a managed Amazon Web Services (AWS)) is the most preferred platform to design, test, and run quantum algorithms. One of the reasons for selecting Amazon Braket for service deployment is that it can allow service users/developers to design their own quantum algorithms. This can be particularly handy for novice developers unfamiliar with the technicalities of quantum systems to utilise a set of pre-built algorithms, tools, and documents to develop and manage quantum services on Amazon platform. 

\begin{tcolorbox} [sharp corners, boxrule=0.1mm,]
\faEdit \scriptsize{~\textsf{\textbf{Quantum platforms} leverage cloud computing infrastructures to support quantum service-orientation. Amazon Braket is the predominant quantum vendor that can enable novice developers to exploit some pre-built algorithms, programming tools, and service documentation that lack on other quantum platforms.}}
\end{tcolorbox}
    




    
    % \begin{table*}[ht]
    % \caption{Caption}
    % \begin{centering}
    % {\tiny{}}%
    
    % \begin{tabular}{|l|l|l|l|l|l|l|}
    % \hline
    % \multicolumn{1}{|c|}{\begin{tabular}[c]{@{}c@{}}[S1]\\ {\includegraphics[scale=0.15]{Images/icons/1.jpg}}\end{tabular}} & \multicolumn{1}{c|}{\begin{tabular}[c]{@{}c@{}}{\includegraphics[scale=0.06]{Images/icons/2.png}} \\ Quantum vendor \\ Selection\end{tabular}} & \multicolumn{1}{c|}{\begin{tabular}[c]{@{}c@{}}{\includegraphics[scale=0.06]{Images/icons/3.png}}\\ Deployment \\ Diagram\end{tabular}} & \multicolumn{1}{c|}{\begin{tabular}[c]{@{}c@{}}{\includegraphics[scale=0.06]{Images/icons/4.png}}\\ API \\ Gateway\end{tabular}} & \multicolumn{1}{c|}{\begin{tabular}[c]{@{}c@{}}{\includegraphics[scale=0.07]{Images/icons/5.png}} \\ Optimal Service \\ Delivery\end{tabular}} & \multicolumn{1}{c|}{\begin{tabular}[c]{@{}c@{}}{\includegraphics[scale=0.013]{Images/icons/6.png}}\\ Some Text\end{tabular}} & \multicolumn{1}{c|}{\begin{tabular}[c]{@{}c@{}}{\includegraphics[scale=0.2]{Images/icons/7.jpg}}\\ Amazon Braket\end{tabular}} \\ \hline
    %  &  &  &  &  &  &  \\ \hline
    %  &  &  &  &  &  &  \\ \hline
    %  &  &  &  &  &  &  \\ \hline
    
    % \end{tabular}
    % \par\end{centering}
    % \label{tab:criteria}
    % \end{table*}

\section{Emerging Trends of Research on QaaS (RQ-2)}
\label{sec:RQ2}
We now answer RQ-2 that aims to report the emerging trends that indicate dimensions of potential future research on QCaaS. To identify these trends, we specifically reviewed the details about objectives/contributions, evaluation/demonstrations, and limitations/future research in each study during quality assessment (Q-1, Q-4, Q-5 in Table \ref{tab:qualitycriteria}). We have visualised the identified emerging trends in Figure \ref{FutureResearch} as per the SOA life cycle activities. For example, Figure \ref{FutureResearch} highlights that during the conception of quantum services, Quantum Significant Requirements (QSRs) should include quality aspects that complement the functional aspects to ensure the required functionality and desired quality of QCaaS as part of quantum domain engineering activity in QSE. %Functionality and quality requirements as part of quantum domain engineering could help derive a model that can enable model-driven development, i.e., modeling requirements into the design of quantum services. 
%The trends highlighted here only represents the evidence gathered from reviewed studies that may not be comprehensive, however, it is to be viewed as the basis to hypothesise future research challenges. 


%%\scriptsize{\faHandORight ~\textsf{\textbf{Process-centred approaches} such as Quantum DevOps, Quantum microservicing or agile methods can enable incremental development and delivery of quantum software services. \textbf{Human roles} can enrich the process (synergising knowledge of quantum physics practices of software engineering) to tackle activities such as quantum domain engineering, software architecting, and simulation management - reflecting expertise that currently lack in QaaS development.}}


%%\faHandORight \scriptsize{~\textsf{The notion of \textbf{QSRs} is rooted in the concept of ASRs that aims to identify, classify and document functional and quality aspects of quantum software services. QSRs such as QuBit utilisation, energy efficiency, vendor virtualisation etc. represent quality aspects that can complement the functionality of a quantum service.}}


\subsection{Process and Human Roles in QCaaS Development} 
Process-centred engineering is manifested in service life cycle to structure a multitude of activities, such as service design, development, and delivery, into a coherent process that can be executed in an incremental fashion \cite{R19_ahmadtowards}. Although the evidence from the selected studies is accumulated and organized under SOA life cycle in Table \ref{tab:criteria}, however, at the individual scale, the studies lacked a process-centric approach and highlighted the needs for processes attuned to the needs of QSE \cite{R19_ahmadtowards}. These processes, such as Quantum DevOps, Quantum  microservicing or agile methods for quantum service/software etc., are seen as quantum-genre of SE processes that are better aligned with the needs of QSE \cite{R23_khan2022agile}. For example, in the agile method for QSE, quantum domain engineering can help accumulate the domain information of a quantum system (hardware/software) to develop a design model that can act as a blueprint to implement quantum software and services. Process-centric approaches can also support tools (for automation) and professional roles (human decision support) to engineer quantum services. We identified the need for human roles as QSE professionals to effectively undertake QCaaS development.

\faComment{} 
{~\footnotesize\textsf{{\textbf{Process-centred} approaches, such as Quantum DevOps, Quantum microservicing or agile methods, can enable an incremental development and delivery of quantum software services. \textbf{Human roles} can enrich the process - synergising knowledge of quantum physics practices of software engineering - to support activities of quantum domain engineering, software architecting, and simulation management as expertise that currently lacks in quantum service-orientation.}}}   


\subsection{Quantum Significant Requirements}   
The concept of Quantum Significant Requirements (QSRs) can be traced to well-established role of Architecturally Significant Requirements (ASRs) in software and service engineering. QSRs as part of the quantum domain engineering activity can help elicit and document the functional and non-functional aspects (also quality aspects or attributes) of quantum software. Existing research primarily focuses on functional aspects of services while overlooking the non-functional aspects that include but are not limited to QuBit utilisation, energy efficiency, quantum vendor lock-ins, QPU elasticity, quantum error mitigation, that impact the development and operationalisation or QCaaS solutions. For example, during the quantum domain engineering activity the hardware aspects (operations of QuGates) can be mapped to software aspects (components and connectors) that can help with splitting the computation tasks between classical and quantum computers using the classic-quantum split pattern. 

\faComment{} 
{~\footnotesize\textsf{The notion of {\textbf{QSRs} is rooted in the concept of ASRs that aims to identify, classify, and document functional and quality aspects of quantum software services. QSRs for quantum services can complement the functional aspects with quality requirements to ensure the required functionality and desired quality of service.}}} 

\subsection{Model-driven Quantum Software Servicing}
Model-driven Service Engineering (MDSE) enables software engineers and architects to rely on models that can abstract complex and implementation-specific details with human comprehensible visual notations to design and implement software services. Specifically, by exploiting MDSE, software engineers can apply the model transformation to transform the design models into implementation (source code) and validation (test case) models. Modeling notations, such as Service-oriented architecture Modeling Language (SoaML) and more specifically the Q-UML specification, provide a metamodel and a UML profile for the specification and design of services within a service-oriented architecture. MDSE can benefit novice developers to to map the flow of algorithms and modules of source code to graphical models for low-code (model-driven) development of quantum services.

\faComment{}
{~\footnotesize\textsf{\textbf{Model-driven Quantum Service Development} can leverage the principle of MDSE and modeling notations like SoaML and Q-UML to abstract implementation-level complexities with design-level models driven by QSRs. To exploit model-driven development, there is a need for tool support that could enable automation (e.g., model transformation tools) and human decision support (e.g., quantum software architects) to quantum software servicing using MDSE.}}

\begin{figure}[]
 \centering
 \includegraphics[scale=0.7]{Images/Future.pdf} 
 	\caption{Overview of Emerging Research Trends}
	\label{FutureResearch}
\end{figure}

%%\faHandORight{\scriptsize{\textsf{~Model-driven Quantum Service Development can leverage the principle of MDSE and modelling notations like SoaML to abstract implementation-level complexities with design-level models driven by QSRs. However, to exploit model-driven development, there is a need for tool support that could enable automation and human roles that can guide the process to generate models.}}}

\begin{table*}[hbt!]
\caption{A Review of the Relevant Secondary Studies}
\begin{centering}
{\tiny
\begin{tabular}{|lcclcl|}
\hline
\rowcolor[HTML]{DAE8FC} 
\multicolumn{6}{|c|}{\cellcolor[HTML]{DAE8FC}\textbf{Quantum Software Engineering}} \\ \hline
\rowcolor[HTML]{EFEFEF} 
\multicolumn{1}{|l|}{\cellcolor[HTML]{EFEFEF}\textit{\begin{tabular}[c]{@{}l@{}}Study \\ Reference\end{tabular}}} & \multicolumn{1}{c|}{\cellcolor[HTML]{EFEFEF}\textit{\begin{tabular}[c]{@{}c@{}}Type of \\ Study  \end{tabular}}} & \multicolumn{1}{c|}{\cellcolor[HTML]{EFEFEF}\textit{\begin{tabular}[c]{@{}c@{}}Focus of \\ Study\end{tabular}}} & \multicolumn{1}{c|}{\cellcolor[HTML]{EFEFEF}\textit{\begin{tabular}[c]{@{}c@{}}Core\\ Findings\end{tabular}}} & \multicolumn{1}{l|}{\cellcolor[HTML]{EFEFEF}\begin{tabular}[c]{@{}l@{}}QSE\\ Activity\end{tabular}} & \begin{tabular}[c]{@{}l@{}}Publication \\ Year\end{tabular} \\ \hline
\multicolumn{1}{|l|}{\cite{X1_QSE}} & \multicolumn{1}{c|}{\begin{tabular}[c]{@{}c@{}}Literature \\ Review\end{tabular}} & \multicolumn{1}{c|}{\begin{tabular}[c]{@{}c@{}}Quantum \\ Software \\ Engineering\end{tabular}} & \multicolumn{1}{l|}{\begin{tabular}[c]{@{}l@{}}QSE life cycle, quantum software \\ engineering processes, methods, and tools.\end{tabular}} & \multicolumn{1}{c|}{\begin{tabular}[c]{@{}c@{}}Software  \\ Life cycle\end{tabular}} & 2020 \\ \hline
\multicolumn{1}{|l|}{\cite{X2_QSA}} & \multicolumn{1}{c|}{\begin{tabular}[c]{@{}c@{}}Systematic \\ Review\end{tabular}} & \multicolumn{1}{c|}{\begin{tabular}[c]{@{}c@{}}Quantum\\ Software \\ Architecture\end{tabular}} & \multicolumn{1}{l|}{\begin{tabular}[c]{@{}l@{}}Quantum software architecting activities, \\ modelling notations, patterns, and tools \\ for architectural development\end{tabular}} & \multicolumn{1}{c|}{\begin{tabular}[c]{@{}c@{}}Software Design \\ and \\ Architecture\end{tabular}} & 2021 \\ \hline
\multicolumn{1}{|l|}{\cite{R15_de2022software}} & \multicolumn{1}{c|}{\begin{tabular}[c]{@{}c@{}}Repository Mining\\ and   \\ Practitioner Survey\end{tabular}} & \multicolumn{1}{c|}{\begin{tabular}[c]{@{}c@{}}Quantum\\ Programming\\ Languages\end{tabular}} & \multicolumn{1}{l|}{\begin{tabular}[c]{@{}l@{}}Mining repositories and interviewing \\ practitioner to investigate quantum \\ programming languages\end{tabular}} & \multicolumn{1}{c|}{\begin{tabular}[c]{@{}c@{}}Software \\ Implementation\end{tabular}} & 2022 \\ \hline
\multicolumn{1}{|l|}{\cite{x4_Issues}} & \multicolumn{1}{c|}{\begin{tabular}[c]{@{}c@{}}Repository \\ Mining\end{tabular}} & \multicolumn{1}{c|}{\begin{tabular}[c]{@{}c@{}}Quantum\\ Issues\end{tabular}} & \multicolumn{1}{l|}{\begin{tabular}[c]{@{}l@{}}Mining repositories to identify technical\\ debts in open-source quantum software\end{tabular}} & \multicolumn{1}{c|}{\begin{tabular}[c]{@{}c@{}}Software \\ Implementation \\ and   Testing\end{tabular}} & 2022 \\ \hline
\multicolumn{1}{|l|}{\cite{X5_QC}} & \multicolumn{1}{c|}{\begin{tabular}[c]{@{}c@{}}Vision \\  Paper\end{tabular}} & \multicolumn{1}{c|}{\begin{tabular}[c]{@{}c@{}}Quantum\\ Software \\ Architecture\end{tabular}} & \multicolumn{1}{l|}{\begin{tabular}[c]{@{}l@{}}QSE life cycle and comparing   quantum\\ computers with their classical \\ counterparts and vision for future research\end{tabular}} & \multicolumn{1}{c|}{\begin{tabular}[c]{@{}c@{}}Software  \\ Life cycle\end{tabular}} & 2022 \\ \hline
\rowcolor[HTML]{DAE8FC} 
\multicolumn{6}{|c|}{\cellcolor[HTML]{DAE8FC}\textbf{Quantum Services Computing}} \\ \hline
\multicolumn{1}{|l|}{\cite{R11_leymann2020quantum}} & \multicolumn{1}{c|}{\begin{tabular}[c]{@{}c@{}}Literature \\ Survey\end{tabular}} & \multicolumn{1}{c|}{\begin{tabular}[c]{@{}c@{}}Quantum \\ Cloud \\ Computing\end{tabular}} & \multicolumn{1}{l|}{\begin{tabular}[c]{@{}l@{}}Programming quantum computers and \\ investigating hybrid software consisting \\ of classical parts  and quantum parts.\end{tabular}} & \multicolumn{1}{c|}{\begin{tabular}[c]{@{}c@{}}Service Design \\ and \\ Architecture\end{tabular}} & 2020 \\ \hline
\multicolumn{1}{|l|}{\cite{x7_QAAS}} & \multicolumn{1}{c|}{\begin{tabular}[c]{@{}c@{}}Literature\\ Survey\end{tabular}} & \multicolumn{1}{c|}{\begin{tabular}[c]{@{}c@{}}Quantum \\ Service \\ Computing\end{tabular}} & \multicolumn{1}{l|}{\begin{tabular}[c]{@{}l@{}}Potential and limitation  of integrating \\ quantum computing with cloud computing\end{tabular}} & \multicolumn{1}{c|}{\begin{tabular}[c]{@{}c@{}}Service \\ Lifecycle\end{tabular}} & 2015 \\ \hline
\end{tabular}}
\par\end{centering}
\label{tab:RelatedWork}
\end{table*}
\subsection{Empiricism in Mining Quantum Service Patterns}
Pattern-based software service engineering relies on architectural design and implementation strategies and best practices that can be reused to deliver software services \cite{R20_keen2006patterns}. Existing solutions employ a number of pattern-based solutions, such as classic-quantum split and service wrapping pattern, however, there is no evidence on an empirical discovery of patterns and tactics as reusable knowledge \cite{R27_GitQSE}. A lack of empiricism in discovering new patterns hinders the reusability of service design and implementation knowledge during quantum service engineering. One possible dimension for pattern discovery is mining software repositories or social coding platforms (e.g., GitHub) that contain raw knowledge that can be mined  as patterns. Pattern-based solutions could complement human expertise with available best practices for service design and implementation. %%More specifically, novice developers of quantum software services can rely on patterns that abstract complexities of quantum source coding with modularisation of pattern-based solutions.

\faComment{}
{~\footnotesize\textsf{\textbf{Pattern discovery} via empirically-grounded methods, i.e., mining repositories or social coding platforms of quantum software development, can help leverage reusable design rationale for quantum service engineering. Pattern languages can empower the role of quantum software architects and algorithm designers to rely on reusable knowledge and best practices as opposed to ad-hoc and once-off solutions.}}



%%\faHandORight{\scriptsize{\textsf{~Patterns and tactics represent theoretical knowledge and best practices as an empirically grounded design decisions and strategies that enable reuse and enhance quality of software services. Mining patterns (investigating service repositories and social coding platforms etc.) can help discover and document patterns specific to quantum software services.}}}

\subsection{Continuous Testing and Delivery of Quantum Services} 
With an adoption of agile software engineering in quantum software development context \cite{R23_khan2022agile}, there is a need for light and adaptive methods to ensure a continuous development and delivery of quantum software services. The literature suggested a lack of solutions on testing the software services. The continuous testing and continuous delivery can help CT/CD to test the services against QSRs more effectively and deliver them rapidly. Quantum service testing can involve simulation or regression tests against the QSRs.

%\faComment{} 
%{~\footnotesize\textsf{\TEXTBF{Continuous Testing and Continuous Delivery} (CT/CD) relies on agile software engineering methods to help deliver quantum software services rapidly and reliably. CT/CD can provide strategic benefits to vendors by adding new services to their quantum platforms.}} 

\faComment {~\footnotesize\textsf{\textbf{Continuous Testing and Continuous Delivery} (CT/CD) relies on agile software engineering methods to help deliver quantum software services rapidly and reliably. CT/CD can provide strategic benefits to vendors by adding new services to their quantum platforms.}}


%%\subsection{Continuous Delivery of Quantum Services}


\section{Related Work}

\textbf{Topological Map in Exploration and Navigation.} Inspired by the animal and human psychology~\cite{tolman1948cognitive}, a large amount of work has recently proposed to build topological map to represent an environment~\cite{graphtopoexplore,Murphy08ICRA,neural_topomap,beeching2020learning,savinov2018semiparametric,VisualGraphMem_ICCV21,MRTopoMap,Savarese-RSS-19}. They use the topological map for tasks such as navigation~\cite{neural_topomap,learn2explore_iclr20,savinov2018semiparametric,VisualGraphMem_ICCV21,Savarese-RSS-19}, exploration~\cite{learn2explore_iclr20,graphtopoexplore,Murphy08ICRA,savinov2018semiparametric,MRTopoMap,TSGM} and planning~\cite{beeching2020learning}. To build the topological map, they combine various sensors such as RGB image, depth map~\cite{TSGM,Savarese-RSS-19}, pose~\cite{neural_topomap,learn2explore_iclr20,beeching2020learning} and even LiDAR scanner~\cite{MRTopoMap,graphtopoexplore}. Some of them further adopt data-hungry and computation-demanding Reinforcement Learning~(RL) techniques to train the model to construct the topological map~\cite{neural_topomap,learn2explore_iclr20,VisualGraphMem_ICCV21}. Kwon \textit{et al.}~\cite{VisualGraphMem_ICCV21} combine imitation learning~(IL) and RL to train the model. Some of these methods~\cite{neural_topomap,learn2explore_iclr20,beeching2020learning} involve metric information to construct the topological map. N.~Savinov~\textit{et. al.}~\cite{savinov2018semiparametric} use the random walk to construct the topological map, which inevitably leads to an inefficient topological map. TSGM~\cite{TSGM} jointly adds surrounding objects during topological map construction. Unlike these prior works, our \textit{\acronym{}} is completely metric-free and simple in experimental configuration~(just RGB image, much smaller expert demonstration size).

\textbf{Hallucinating Future Feature.} The idea of hallucinating future latent features has been discussed in other application domains. Previous work has utilized this idea of visual anticipation in video prediction/human action prediction~\cite{16Vondrick,17Zeng,20Chang,21Fernando,Suris2021LearningTP}, and researchers have applied similar ideas to robot motion and path planning~\cite{Jain2016RecurrentNN, Koppula2016, Carlone2019, Park2016}. As stated in~\cite{16Vondrick,17Zeng,Suris2021LearningTP}, visual features in the latent space provide an efficient way to encode semantic/high-level information of scenes, allowing us to do planning in the latent space, which is considered more computationally efficient when dealing with high-dimensional data as input~\cite{Lippi2020,Ichter2019}. Different from previous robotics work, we take advantage of this efficient representation by adding deep supervision when anticipating the next visual feature, which was computationally intractable if we were to operate at the pixel level.

\textbf{Deeply-Supervised Learning} has been extensively explored~\cite{deeply_supervised_nets,knowledge_synergy,li2017deep,li2018deep} during the past several years. The main idea is to add extra supervision to various intermediate layers of a deep neural network in order to more effectively train deeper neural networks. In our work, we adopt a similar idea to deeply supervise the training of feature hallucination and action generation.


\begin{figure*}[t]
    \centering
    \includegraphics[width=0.8\linewidth]{Topo_map.pdf}
    \caption{\textbf{Training and inference for task and motion imitation.} Feature extractor $g_\psi$ takes image $I_t$ as input and generates the corresponding feature vector $f_t$. \textit{TaskPlanner} $\pi_{\theta_T}$ is a recurrent neural network (RNN) consuming a sequence of features $\{ f_{t-10}, \cdots, f_t\}$ to hallucinate the next best feature to visit $\hat{f}_{t+1}$. \textit{MotionPlanner} $\pi_{\theta_M}$ consumes the concatenation (denoted by $\bigoplus$) of ${f}_{t}$ and $\hat{f}_{t+1}$ and generates the action to move the agent towards the hallucinated feature. During training, we supervise all the intermediate outputs including the intermediate hallucinated features $\{ \hat{f}_{t-9}, \cdots, \hat{f}_{t} \}$ and the intermediate actions $\{ \hat{a}_{t-10}, \cdots, \hat{a}_{t-1} \}$, in addition to the final output $\hat{f}_{t+1}$ and $\hat{a}_t$. During inference, current observation $I_t$ is firstly encoded and fed into $\pi_{\theta_T}$ to hallucinate $\hat{f}_{t+1}$, and then $\hat{f}_{t+1}$ combined with the ${f}_{t}$ is fed into $\pi_{\theta_M}$ for motion planning. $\mathcal{L}_T$ is $L_2$ loss and $\mathcal{L}_M$ is cross entropy loss (the subscripts $T$ and $M$ denote \textbf{T}ask and \textbf{M}otion respectively). $h_t$ denotes the hidden state of RNN.} 
    \label{fig:pipeline}
    \vspace{-5mm}
\end{figure*}

\textbf{Task and Motion Planning.} Task and motion planning (TAMP) divides a robotic planning problem into high-level task allocation (task planning) and low-level action for task execution (motion planning). This hierarchical framework is adopted in many robotic tasks such as manipulation \cite{chitnis2016guided,mcdonald2022guided} exploration~\cite{Cao-RSS-21} and navigation \cite{lo2018petlon,thomas2021mptp}. Such a framework allows us to leverage high-level information about the scenes to tackle challenges in local control techniques~\cite{bansal2019-lb-wayptnav}. In this work, to perform active topological mapping of a novel environment, the agent firstly reasons at the highest level about the regions to navigate: hallucinate the next best feature point to visit. Afterward, the agent takes an action to get to the target feature. The whole procedure is totally implemented in feature space without any metric information.

\textbf{Imitation Learning} aims to mimic human behavior or expert demonstrations for a given specific task~\cite{imitation_learning,il_legged,il_planning}. The agent is trained to perform tasks by directly observing demonstrations~\cite{il_legged,il_planning}. In our work, the expert demonstration is a set of image-action pair sequences that an agent would observe along a route that efficiently covers an environment. It is widely accessible in either real-world or simulated environments~(e.g. from human experts or maps of environments). 
%input{ResearchMethod}
%\input{Processes}
%\input{HumanRoles}
%%\input{Threats}
%\section{Related Work}

\textbf{Topological Map in Exploration and Navigation.} Inspired by the animal and human psychology~\cite{tolman1948cognitive}, a large amount of work has recently proposed to build topological map to represent an environment~\cite{graphtopoexplore,Murphy08ICRA,neural_topomap,beeching2020learning,savinov2018semiparametric,VisualGraphMem_ICCV21,MRTopoMap,Savarese-RSS-19}. They use the topological map for tasks such as navigation~\cite{neural_topomap,learn2explore_iclr20,savinov2018semiparametric,VisualGraphMem_ICCV21,Savarese-RSS-19}, exploration~\cite{learn2explore_iclr20,graphtopoexplore,Murphy08ICRA,savinov2018semiparametric,MRTopoMap,TSGM} and planning~\cite{beeching2020learning}. To build the topological map, they combine various sensors such as RGB image, depth map~\cite{TSGM,Savarese-RSS-19}, pose~\cite{neural_topomap,learn2explore_iclr20,beeching2020learning} and even LiDAR scanner~\cite{MRTopoMap,graphtopoexplore}. Some of them further adopt data-hungry and computation-demanding Reinforcement Learning~(RL) techniques to train the model to construct the topological map~\cite{neural_topomap,learn2explore_iclr20,VisualGraphMem_ICCV21}. Kwon \textit{et al.}~\cite{VisualGraphMem_ICCV21} combine imitation learning~(IL) and RL to train the model. Some of these methods~\cite{neural_topomap,learn2explore_iclr20,beeching2020learning} involve metric information to construct the topological map. N.~Savinov~\textit{et. al.}~\cite{savinov2018semiparametric} use the random walk to construct the topological map, which inevitably leads to an inefficient topological map. TSGM~\cite{TSGM} jointly adds surrounding objects during topological map construction. Unlike these prior works, our \textit{\acronym{}} is completely metric-free and simple in experimental configuration~(just RGB image, much smaller expert demonstration size).

\textbf{Hallucinating Future Feature.} The idea of hallucinating future latent features has been discussed in other application domains. Previous work has utilized this idea of visual anticipation in video prediction/human action prediction~\cite{16Vondrick,17Zeng,20Chang,21Fernando,Suris2021LearningTP}, and researchers have applied similar ideas to robot motion and path planning~\cite{Jain2016RecurrentNN, Koppula2016, Carlone2019, Park2016}. As stated in~\cite{16Vondrick,17Zeng,Suris2021LearningTP}, visual features in the latent space provide an efficient way to encode semantic/high-level information of scenes, allowing us to do planning in the latent space, which is considered more computationally efficient when dealing with high-dimensional data as input~\cite{Lippi2020,Ichter2019}. Different from previous robotics work, we take advantage of this efficient representation by adding deep supervision when anticipating the next visual feature, which was computationally intractable if we were to operate at the pixel level.

\textbf{Deeply-Supervised Learning} has been extensively explored~\cite{deeply_supervised_nets,knowledge_synergy,li2017deep,li2018deep} during the past several years. The main idea is to add extra supervision to various intermediate layers of a deep neural network in order to more effectively train deeper neural networks. In our work, we adopt a similar idea to deeply supervise the training of feature hallucination and action generation.


\begin{figure*}[t]
    \centering
    \includegraphics[width=0.8\linewidth]{Topo_map.pdf}
    \caption{\textbf{Training and inference for task and motion imitation.} Feature extractor $g_\psi$ takes image $I_t$ as input and generates the corresponding feature vector $f_t$. \textit{TaskPlanner} $\pi_{\theta_T}$ is a recurrent neural network (RNN) consuming a sequence of features $\{ f_{t-10}, \cdots, f_t\}$ to hallucinate the next best feature to visit $\hat{f}_{t+1}$. \textit{MotionPlanner} $\pi_{\theta_M}$ consumes the concatenation (denoted by $\bigoplus$) of ${f}_{t}$ and $\hat{f}_{t+1}$ and generates the action to move the agent towards the hallucinated feature. During training, we supervise all the intermediate outputs including the intermediate hallucinated features $\{ \hat{f}_{t-9}, \cdots, \hat{f}_{t} \}$ and the intermediate actions $\{ \hat{a}_{t-10}, \cdots, \hat{a}_{t-1} \}$, in addition to the final output $\hat{f}_{t+1}$ and $\hat{a}_t$. During inference, current observation $I_t$ is firstly encoded and fed into $\pi_{\theta_T}$ to hallucinate $\hat{f}_{t+1}$, and then $\hat{f}_{t+1}$ combined with the ${f}_{t}$ is fed into $\pi_{\theta_M}$ for motion planning. $\mathcal{L}_T$ is $L_2$ loss and $\mathcal{L}_M$ is cross entropy loss (the subscripts $T$ and $M$ denote \textbf{T}ask and \textbf{M}otion respectively). $h_t$ denotes the hidden state of RNN.} 
    \label{fig:pipeline}
    \vspace{-5mm}
\end{figure*}

\textbf{Task and Motion Planning.} Task and motion planning (TAMP) divides a robotic planning problem into high-level task allocation (task planning) and low-level action for task execution (motion planning). This hierarchical framework is adopted in many robotic tasks such as manipulation \cite{chitnis2016guided,mcdonald2022guided} exploration~\cite{Cao-RSS-21} and navigation \cite{lo2018petlon,thomas2021mptp}. Such a framework allows us to leverage high-level information about the scenes to tackle challenges in local control techniques~\cite{bansal2019-lb-wayptnav}. In this work, to perform active topological mapping of a novel environment, the agent firstly reasons at the highest level about the regions to navigate: hallucinate the next best feature point to visit. Afterward, the agent takes an action to get to the target feature. The whole procedure is totally implemented in feature space without any metric information.

\textbf{Imitation Learning} aims to mimic human behavior or expert demonstrations for a given specific task~\cite{imitation_learning,il_legged,il_planning}. The agent is trained to perform tasks by directly observing demonstrations~\cite{il_legged,il_planning}. In our work, the expert demonstration is a set of image-action pair sequences that an agent would observe along a route that efficiently covers an environment. It is widely accessible in either real-world or simulated environments~(e.g. from human experts or maps of environments). 
\section{Conclusions}
\label{sec:conclusions}

We have demonstrated that the fraction of negative event weights in
existing large high-multiplicity samples can be reduced by more than
an order of magnitude, whilst preserving predictions for observables
within statistical uncertainties. Concretely, we have employed the cell
resampling method proposed in~\cite{Andersen:2021mvw} with NLO event
samples for Z boson production with up to three jets
and W boson production with five jets produced with \textsc{Sherpa}
and \textsc{BlackHat}.

For the first time, cell resampling has been applied to samples with
up to several billions of events. This was made possible by
algorithmic improvements leading to a speed-up by several orders of
magnitude. Our updated implementation can be retreived from
\url{https://cres.hepforge.org/}.

The advances in the development of the cell resampling method
presented in this work pave the way for future applications to processes with
high-multiplicities, in particular including parton showered
predictions. It will be necessary to quantify the uncertainty
introduced by the weight smearing. Variations in the maximum cell size
parameter and different prescriptions for weight redistribution within
a cell can serve as handles to assess this uncertainty. Another
promising avenue for further exploration is the analysis of the
information on weight distribution within phase space collected during
cell resampling. Regions with insufficient Monte Carlo statistics
could be identified by their accumulated negative weight, thereby
guiding the event generation. We leave the investigation of these
questions to future work.

\section*{Acknowledgements}

AM thanks Zahari Kassabov for encouragement to reconsider the use of nearest
neighbour search trees. The work of JRA and DM is supported by the STFC under
grant ST/P001246/1.

%%% Local Variables:
%%% mode: latex
%%% TeX-master: "main"
%%% End:


\balance
\bibliographystyle{ieeetr}
\bibliography{References}


%\begin{thebibliography}{00}
%\end{thebibliography}
%\vspace{12pt}
%\color{red}
%IEEE conference templates contain guidance text for composing and formatting conference papers. Please ensure that all template text is removed from your conference paper prior to submission to the conference. Failure to remove the template text from your paper may result in your paper not being published.

%\balance
\end{document}
