\section{Research Context and Method}
\label{ResearchContextMethod}
We now contextualise service-orientation for QCs in Section \ref{sec:context} and present the research method in Section \ref{sec:method}. %Concepts and terminologies introduced in this section are used throughout the paper.

\subsection{Context: Service-Orientation for Quantum Computing}
\label{sec:context}
\subsubsection{Quantum Computing Systems}
We briefly overview a QC system that comprises of quantum hardware and software elements as shown in Figure \ref{Fig-1:Context}a). Fundamental to achieving quantum computations are Quantum Bits (QuBits) that represent the basic unit of quantum information processing by manipulating Quantum Gates (QuGates) \cite{R2_harrow2017quantum} \cite{R13_monroe2018quantum}. 
Traditional Binary Digits (Bits) in classical systems (i.e., digital computers) are represented as [1, 0] where 1 represents the computation state as \textsc{On} and 0 represents the state as \textsc{Off} to manipulate binary gates in digital circuits. In comparison, a QuBit represents a two-state quantum computer expressed as $|0\rangle$ and $|1\rangle$. The state of a single Qubit can be expressed as  $|0\rangle = \begin{bmatrix} 1 \\ 0 \end{bmatrix}$ and $|1\rangle = \begin{bmatrix} 0 \\ 1 \end{bmatrix}$ and quantum superposition allows a QuBit to attain a liner combination of both states: 

\begin{equation}\label{EQ-1}
 |0\rangle  =  \left[ \begin{array}{c} 1 \\ 0 \end{array} \right] ~~~~~ + ~~~~~  |1\rangle  =  \left[ \begin{array}{c} 0 \\ 1 \end{array} \right]   
\end{equation}

Based on Figure \ref{Fig-1:Context}a), we distinguish between a Bit and QuBit such that a Bit can take a value as either \textsc{`Off:0’} or \textsc{`On:1’} with 100\% probability. In comparison, a QuBit can be in a state of $|0\rangle$ or $|1\rangle$ or in a superposition state with 50\% $|0\rangle$ and 50\%  $|1\rangle$. In addition, two QuBits can be entangled, and entangled QuBits are linked in a way that observing (i.e., measuring) one of the QuBits, can reveal the state of the other QuBit. Extended details about QuBits and QuGates to develop and operate the QC systems are reported in studies like \cite{R1_ali2022software} and \cite{R9_moguel2022quantum}. To utilise the quantum computing resources, such as quantum processor and memory, there is a need for control software that can program QuBits to manage QuGates of a QC system. Quantum software systems rely on quantum source code compilers that allow quantum algorithm designers and programmers to write, build, and execute software for quantum computers. For example, a programmer can use a quantum programming language, such as Q\# (by Microsoft) or Qiskit (by Google) and use a quantum compiler to enable programable quantum computations \cite{R4_dyakonov2019will} \cite{R15_de2022software}. Software systems that can manage and control quantum hardware find their applications in areas including but not limited to quantum cryptography, bio-inspired computing, and quantum information processing \cite{R1_ali2022software}. However, scarcity of quantum hardware, lack of quantum software professionals, and economics of owning or maintaining QC are some critical factors that impede commercially viable quantum computers \cite{R2_harrow2017quantum} \cite{R3_egger2020quantum}. Vendors who offer QCaaS platforms view quantum service-orientation as an opportunistic business model that offers QC resources to customers as utility computing \cite{R6_wei2010service} \cite{R9_moguel2022quantum}.

\begin{figure}[]
 \centering
 \includegraphics[scale=0.45]{Images/Overview.pdf} 
 	\caption{An Overview of the Quantum Service-Orientation}
	\label{Fig-1:Context}
\end{figure}


\subsubsection{Service-Oriented Computing}
Services computing follows the SOA style that allows service users to discover and utilise a multitude of available software services that encapsulate computing resources and applications offered by service vendors/providers \cite{R7_bouguettaya2017service}. Figure \ref{Fig-1:Context}b) shows SOA style-based quantum servicing where a QC user (i.e., service requester) can utilise the QC resources offered by quantum vendors (i.e., service provider) by means of quantum services. In most cases, QC systems of today are not capable of executing quantum algorithms wrapped with large amounts of data, inputs, and outputs \cite{R2_harrow2017quantum} \cite{R13_monroe2018quantum}. As shown in Figure \ref{Fig-1:Context}a), large volumes of data in quantum algorithms require more QuBits and complex QuGates that result in deep quantum circuiting and consequently increased errors referred to as noisy intermediate-scale quantum (NISQ) \cite{R4_dyakonov2019will}. To address the issues like NISQ, the classic-quantum split pattern slices the overall quantum software or application into classical modules (pre/post-processing) and quantum modules (quantum computation) that result in hybrid applications\cite{R12_valencia2022quantum}. One of the prime examples of classic-quantum split patterns is Shor’s algorithm which involves quantum computations for finding the prime factors of an integer with its application in computer security and cryptography. %Quantum computing vendors such as Amazon, IBM, and Google have started to offer their QC systems and infrastructures to be utilised by individuals and organisations by means of quantum services computing \cite{R9_moguel2022quantum}. %For example, Amazon Braket allows quantum computing services to build, execute, and simulate quantum software based on pay-per-shot model. %Ongoing research and development is focused on offering algorithms, hardware, simulations, and mathematical problems as services for quantum platforms [11].
%%To harness QC as utility computing, there is a need to tailor existing principles and methods of service-orientation or develop new architectures, frameworks, and empirically grounded processes to synergise QC and SOA in the context QaaS.
Quantum service-orientation when viewed from a utility computing perspective can minimise the quantum divide, a prevailing issue highlighted at the World Economic Forum 2023, between states/entities that own or lack QC systems, technologies, and infrastructures \cite{R24_WorldEconomicForum}. 

\subsection{Research Method for the SMS}\label{sec:method}
We now discuss the research method, driven by three phases as illustrated in Figure \ref{Researchmethod}, based on the guidelines to conduct the SMS \cite{R16_petersen2008systematic}.% Extended details of the research method as the SMS protocol are provided in \cite{18_SMSProtocol}.

\subsubsection{\textsf{Phase I} – Specifying the Research Questions}
Research questions (RQs) are fundamental to conducting the SMS and documenting the results. We outlined two RQs for this SMS.

%\textbf{RQ-1}: \textit{What solutions are reported in literature to support Quantum as a Service Computing?}

\begin{tcolorbox} [sharp corners, boxrule=0.1mm,]
\small
\textbf{RQ-1}: What solutions are reported in the literature to support the development of quantum computing as a service?
\end{tcolorbox}


\textbf{Objective(s)} - To investigate state-of-the-art in terms of existing solutions that enable or enhance QCaaS computing. A multi-perspective analysis can reveal the functionality offered by the solutions, modeling languages and patterns to design the solutions, and programming technologies along with deployment platforms to implement and operationalise the solutions. 

%\textbf{RQ-2}: \textit{What are the emerging trends of research on Quantum as a Service Computing?}

\begin{tcolorbox} [sharp corners, boxrule=0.1mm,]
\small
\textbf{RQ-2}: What are the emerging trends of research on quantum computing as a service?
\end{tcolorbox}


\textbf{Objective(s)} - To identify and discuss the emerging trends that can help pinpoint the prevalent challenges and their solutions as dimensions of potentially futuristic research on QCaaS. The emerging trends can help to provide a road map for progressing research and development on QCaaS.



\vspace{0.5em}

\begin{table*}[b!] 
\scriptsize
\caption{Criteria for Screening and Qualitative Assessment of Selected Studies}
\begin{center}
{\tiny}
\begin{tabular}{|l|}
\hline
\rowcolor[HTML]{F2F2F2} 
\multicolumn{1}{|c|}{\cellcolor[HTML]{F2F2F2}\textbf{Study Selection Step I -   Screening of Identified Studies}}                                                                                \\ \hline
S1 - The study does not discuss any   solution or proposal for quantum computing as a service                                                                                                    \\ \hline
S2 - The study is not reported in  English                                                                                                                                                      \\ \hline
S3 - The study is a duplicate study. Duplicate studies are studies with overlapping contents, e.g., a conference paper extended as a journal  article.                                           \\ \hline
S4 - The study is a secondary  study/survey paper                                                                                                                                               \\ \hline
\rowcolor[HTML]{BDD6EE} 
\begin{tabular}[c]{@{}l@{}}Exclude the study if the answer to any of the criteria in Step I (S1 - S4) results in Yes, otherwise, \\ Include the study for quality assessment in Step II\end{tabular}  \\ \hline
\rowcolor[HTML]{F2F2F2} 

\multicolumn{1}{|c|}{\cellcolor[HTML]{F2F2F2}\textbf{Study Selection Step II - Quality Assessment of the Identified Studies}}                                                                    \\ \hline
Q1 - Study objectives and  Contributions are clear? {[}Yes = 1, Partially = 0.5, No = 0{]}                                                                                                      \\ \hline
Q2 - Research method to conduct  the study is reported {[}Yes = 1, Partially = 0.5, No = 0{]}                                                                                                   \\ \hline
Q3 - Design and/or implementation  details of solution are provided {[}Yes = 1, Partially = 0.5, No = 0{]}                                                                                      \\ \hline
Q4 - Details for   Experiments/Evaluation/Demonstration of Solution are provided {[}Yes = 1,  Partially = 0.5, No = 0{]}                                                                        \\ \hline
Q5 - Study limitations and needs  for future research are discussed {[}Yes = 1, Partially = 0.5, No = 0{]}                                                                                      \\ \hline
\rowcolor[HTML]{BDD6EE} 
\begin{tabular}[c]{@{}l@{}}Exclude the study that has a quality assessment score (Q1 – Q5) less than 2.0, otherwise  \\ Include the study for review and data extraction in Table 2\end{tabular} \\ \hline
\end{tabular}
\end{center}
\label{tab:qualitycriteria}
\end{table*}
\subsubsection{\textsf{Phase II} – Identifying and Selecting the Literature for SMS}

Based on the guidelines for literature search, we formulated a generic string to be executed on prominent Electronic Data Sources (EDS) \cite{R17_chen2010towards} including IEEE Xplore, ACM Digital Library, Springer Link, Science Direct, Springer Link, and Wiley Online Library. Google Scholar was used as a complementary EDS to ensure that we did not miss any relevant study for selection. The search string presented in Figure \ref{Researchmethod} is generic that combines logical operations \textsc{AND}, \textsc{OR}) to compose the key terms (e.g., Quantum \textsc{AND} Service \textsc{OR} Cloud), customized for each EDS individually. Customised search strings are provided as part of the SMS protocol \cite{18_SMSProtocol}. We conducted a pilot search to assess the need for any customisation to the search string(s) or any filters applied on specific EDS to avoid an exhaustive search resulting in a significant number of unrelated studies. For example, we limited our search on IEEE Xplore from `Full Text \& Metadata’ to `Document Title’ as searching for our defined key terms in full text and metadata yielded a significant amount of irrelevant studies (e.g., cloud services, quantum hardware). %The pilot search indicated limiting the years of publications (2011-2022) as no evidence of published research on QCaaS was found before 2011.






\textbf{Screening and Quality Assessment:} By executing the customised search strings on five selected EDS, the SMS process retrieved a total of 55 potentially relevant studies. To complement the automated search on EDS, we applied the forward snowballing process [17], as a manual effort. The forward snowballing approach involves looking up the references or bibliography sections of 55 studies, referred to as the seed set in snowballing, to see if any relevant cited literature can be found.  The forward snowballing helped us to identify a total of 13 studies resulting in a total of 68 studies (55: EDS and 13: snow-balling). To assess and select the studies for review, we performed study screening based on criteria (Step 1: S1 – S4) in Table \ref{tab:qualitycriteria}. Most of the studies identified during the snowballing failed the screening criteria S3 and S4 which means either the studies were duplicate studies or secondary/survey studies that cannot be included in the review. Based on the screening of identified studies, more specifically reading through the titles, abstracts, and conclusions we shortlisted a total of 11 potentially relevant studies to be qualitatively assessed (Step II: Q1 – Q5) for their inclusion in the review for SMS. Based on the quality assessment, we excluded 2 studies to finally select a total of 9 studies to be included in the review. The list of selected studies for the SMS is provided in \textbf{Appendix A}. 

\begin{figure}[]
 \centering
 \includegraphics[scale=0.83]{Images/Researchmethod.pdf} 
 	\caption{Overview of the Research Method}
	\label{Researchmethod}
\end{figure}

\vspace{0.5em}
\subsubsection{\textsf{Phase III} – Documenting the Results}
To document the results, i.e., answering RQs objectively, we extracted the data from the selected studies in Appendix A and documented it using a structured format, having seven criteria, in Table \ref{tab:criteria}. The criteria focus on conceptualising, designing, developing, and deploying quantum software services by following the IBM SOA foundation life cycle (SOA life cycle for short) \cite{R20_keen2006patterns}. To contextualise QCaaS from Figure \ref{Fig-1:Context} and to ensure fine-grained analysis of SMS data, we have divided the `Model' activity from SOA life cycle into two activities namely Conception and Model to distinguish between functional needs (conception) and representation (modeling) of quantum service design. Model represents the conception as the design specification of functional needs for quantum services. We do not have the `Manage' phase from SOA life cycle as we could not find any evidence in the literature that supports identity, compliance, and business metrics management of quantum services.


\begin{itemize}
\item \textbf{Conception} as the initial activity in the service life cycle aims to conceptualise the functional aspects of quantum services by capturing the details of required functionality, i.e., functional requirements. Conception aims to identify: \textit{what are the business needs of a quantum service?}

\item \textbf{Model} activity aims to translate the conception into a design that acts as a blueprint for implementing quantum services. Model focuses on: \textit{how to represent the conception as the design of the solution?}

\item \textbf{Assemble} focuses on implementing the design to produce concrete, i.e., executable specification of quantum services. Assemble aims to address: \textit{how to implement the design as executable services?}

\item \textbf{Deploy} as the last activity aims to deploy the assembled solution for operationalisation and usage of quantum services. Deploy focuses on: \textit{what platforms can be used to deploy the assembled (implemented) solution?}
\end{itemize}

\vspace{0.5em}

\subsubsection{Threats to the Validity of SMS}
Systematic literature reviews and mapping studies are prone to a number of validity threats that refer to deviation, limitations, or invalidation of study results when applied to a theoretical or practical context. %We highlight three main types of threats and discuss our efforts to minimise them. 
\textbf{Construct validity} of the SMS corresponds to the rigor of study protocol and methodological details to extract, analyze, and synthesise the data to objectively answer the RQs and present the data systematically. To avoid this threat, i.e., avoiding the bias in data extraction and documentation, we applied well-practiced guidelines \cite{R16_petersen2008systematic, R17_chen2010towards}, derived the search strings (Figure \ref{Researchmethod}), and devised a structured format (Table \ref{tab:criteria}) to collect and present the data consistently. \textbf{Internal validity} examines SMS design, conduct, and analysis to answer the RQs without bias. To minimize this threat, we synthesized the data based on the well-known IBM SOA life cycle \cite{R20_keen2006patterns} that structures the results into fine-grained life cycle activities. We documented the results while performing a quality assessment (Table \ref{tab:qualitycriteria}) and a well-defined service life cycle template. \textbf{External validity} of the SMS refers to the extent to which the findings of study can be generalised/externalised to research and development projects. It is challenging to foresee and outline the predictive implications of the study results. We have outlined the implications and generalization of study findings (Table \ref{tab:criteria}, Figure \ref{Results}, Figure \ref{FutureResearch}) can provide the basis for creating a reference architecture for QCaaS as future work. The documented results are discussed in Section \ref{sec:RQ1} (RQ-1) and Section \ref{sec:RQ2} (RQ-2).


% \begin{table*}[ht]
% \caption{Data Extracted for SMS from Reviewed Studies (SOA lifecycle activities [20])}
% \begin{centering}
% {\tiny{}}%
% \begin{tabular}{|lllllll|}
% \hline
% \multicolumn{7}{|c|}{\includegraphics[scale=0.7]{Images/QaaS-Table.pdf}} \\ \hline
% \multicolumn{1}{|l|}{} & \multicolumn{1}{c|}{\cellcolor[HTML]{BDD6EE}\textbf{Conception}} & \multicolumn{2}{c|}{\cellcolor[HTML]{BDD6EE}\textbf{Modeling}} & \multicolumn{2}{c|}{\cellcolor[HTML]{BDD6EE}\textbf{Assembly}} & \multicolumn{1}{c|}{\cellcolor[HTML]{BDD6EE}\textbf{Deployment}} \\ \cline{2-7} 
% \multicolumn{1}{|l|}{\multirow{-2}{*}{Study ID}} & \multicolumn{1}{c|}{\cellcolor[HTML]{E7E6E6}\begin{tabular}[c]{@{}c@{}}Functional\\ Aspects\end{tabular}} & \multicolumn{1}{c|}{\cellcolor[HTML]{E7E6E6}\begin{tabular}[c]{@{}c@{}}Modeling \\ Notation\end{tabular}} & \multicolumn{1}{c|}{\cellcolor[HTML]{E7E6E6}\begin{tabular}[c]{@{}c@{}}Software \\ Pattern\end{tabular}} & \multicolumn{1}{c|}{\cellcolor[HTML]{E7E6E6}\begin{tabular}[c]{@{}c@{}}Service \\ Use case\end{tabular}} & \multicolumn{1}{c|}{\cellcolor[HTML]{E7E6E6}\begin{tabular}[c]{@{}c@{}}Service   \\ Programming\end{tabular}} & \multicolumn{1}{c|}{\cellcolor[HTML]{E7E6E6}\begin{tabular}[c]{@{}c@{}}Quantum\\ Platform/Vendor\end{tabular}} \\ \hline
% \multicolumn{1}{|l|}{} & \multicolumn{1}{l|}{} & \multicolumn{1}{l|}{} & \multicolumn{1}{l|}{} & \multicolumn{1}{l|}{} & \multicolumn{1}{l|}{} &  \\ \hline
% \multicolumn{1}{|l|}{} & \multicolumn{1}{l|}{} & \multicolumn{1}{l|}{} & \multicolumn{1}{l|}{} & \multicolumn{1}{l|}{} & \multicolumn{1}{l|}{} &  \\ \hline
% \multicolumn{1}{|l|}{} & \multicolumn{1}{l|}{} & \multicolumn{1}{l|}{} & \multicolumn{1}{l|}{} & \multicolumn{1}{l|}{} & \multicolumn{1}{l|}{} &  \\ \hline
% \multicolumn{1}{|l|}{} & \multicolumn{1}{l|}{} & \multicolumn{1}{l|}{} & \multicolumn{1}{l|}{} & \multicolumn{1}{l|}{} & \multicolumn{1}{l|}{} &  \\ \hline
% \end{tabular}
% \par\end{centering}
% \label{tab:criteria}
% \end{table*}






