%======================================================================
% CONTEXT, MODEL, EQUATION, CAT
%
\section{Governing equations} \label{sec:equations}
We consider 1D hyperbolic systems of conservation laws of the form
\bea \label{eq:systemPDEs_bis}
    \partial_t \U + \partial_x \mathbf{F}(\U) = \mathbf{0},
\eea
where $t\in \mathbb{R}^+$ represents the time variable, $x\in \mathbb{R}$ the space variable and $\U=\U(x,t)
\in \mathbb{R}^M$ is the vector of conserved variables while 
$\mathbf{F}(\U(x,t)) \in \mathbb{R}^M$ is the flux vector. More precisely, we focus on the 1D Euler equations of gas dynamics in which $M = 3$,  $\U=(\rho, \rho u,  \rho e)^t$ with $\rho$ the density, $u$ the velocity and 
%RAPH
%$e=\varepsilon + \frac12 \| u \|^2$ 
$e=\varepsilon + \frac12 \raph{u^2}$ the total energy \add{per unit mass}, being $\varepsilon$ the specific \raph{internal energy}.
%one.
The flux is given by 
$\mathbf{F}(\U)= \left(
\rho u, \rho u^2 +p, (\rho e +p )u\right)^t$.
%\bea
% \mathbf{F}(\U)= \left( \begin{array}{c}
%\rho u \\
%\rho u^2 +p  \\
%(\rho e +p )u  \\
%\end{array} \right).
%\eea
The system is closed with the \raph{perfect gas} equation of state:
%RAPH: !!! EOS as a function of \varepsilon not e.
%$$ p(\rho,e) = (\gamma-1)\rho e. $$
$ \raph{p(\rho,\varepsilon) = (\gamma-1)\rho \varepsilon}. $
The equations in (\ref{eq:systemPDEs_bis}) represent the conservation of mass, momentum and total energy.
An entropy inequality has to be 
%added 
\raph{supplemented to deal with discontinuous solutions}.
This system is hyperbolic with eigenvalues $\lambda^-=u - c$, $\lambda^0=u,$  $\lambda^+=u + c$ where $c = \sqrt{\gamma p/\rho}$ \raph{is the sound speed}.
The set of admissible states, i.e. states that are consistent with physics, is
\bea \label{eq:physical_states}
    \mathcal{A} = \left\{ \U\in \mathbb{R}^M, \; \text{such that} \; \rho>0, \; p>0 \right\}.
\eea
%
Together with this system, two scalar conservation laws
\raph{of the form}
\begin{equation}
    \label{sec:CAT_gov_equ}
    u_t + \partial_x f(u) = 0,
\end{equation} 
will be considered 
%RAPH: sound weird to me... but it may only be me!
%as well 
to test the methods: \raph{first} the linear advection equation corresponding to $f(u)=b u$ with $b\in \mathbb{R}$, and, \raph{secondly}, Burgers' equation corresponding to $f(u)=u^2/2$.  
\raph{These scalar conservation laws verify the maximum principle, that is
\begin{eqnarray} \label{eq:physical_states2}
    \mathcal{A} & = & \left\{ u\in \mathbb{R}, \; \text{s.t.} \; \underline{u}_0 \leq u(x,t) \leq \overline{u}_0 \right\}, 
    \nonumber\\
    \underline{u}_0 & = & \min_x( u(x,t_0)), \; \overline{u}_0=\max_x( u(x,t_0)), 
\end{eqnarray}
where $t_0$ is the initial time, and $u(x,t_0)$ the initial condition.
}


% \subsection{1D linear and non-linear scalar conservation laws}
% To drastically simplify the description of the numerical schemes we explain the CAT procedure to non-linear scalar conservation laws. For this reason, let us consider the non-linear scalar conservation law on the $Oxt$-Cartesian frame

% where $u=u(x,t):\mathbb{R}\times\mathbb{R}^+\rightarrow\mathbb{R}$ represents the scalar variable, and, $f(u)=f(u(x,t))$ the non-linear flux depending on $u$.
% $u(x,0)=u_0(x)$ is the solution at initial time, while BC depend on the test case; in this chapter we adopt periodic ones and Dirichlet ones. \\
% Eq. (\ref{sec:CAT_gov_equ}) denotes the generic model of the non-linear scalar equation. The simplest one is probably represented by

% \subsection{Mesh}
% In this chapter we consider uniform meshes in 1D. For this reasoning, the space domain can be represented by a segment $\Omega$ and it is uniformly discretized by $N$ cells.  In general, a cell, called $\omega_i,$ is identified so that $\omega_i=[x_{i-1/2};x_{i+1/2}]$ while the cell center is given by $x_i=\ha(x_{i+1/2}+x_{i-1/2})$. Then, the size of is fixes as $\Delta x$. 

% The time domain $\mathcal{T}=[0,T]$ with final time $T>0$ is split into time intervals $[t^n,t^{n+1}]$, $n\in \mathbb{N}$, and time-steps $\Delta t_n=t^{n+1}-t^n$ subject to a CFL (Courant-Friedrichs-Lewy) like condition. In practice, it is better to assign $\Delta t$ dynamically at each time step by imposing some CFL condition. The choice of constant time step here is adopted in order to simplify the notation in the description of the method.

% \subsection{Notation}
% In this article we refer with the following notations for the different type of derivatives or approximations.
% \begin{itemize}
%     \item $u_{i,j}^{(k)}$ is the $k-th$ time derivative of $u$ at time $t_n$ in position $x_{i+j},$ where $i$ refers to the cell and $j$ to the position of the stencil. In general, for a scheme of order $2P,$ $k = 1,\ldots,2P-1$ and $j = -P+1,\ldots,P.$

%     \item $f_{i,j}^{(k)}$ is the $k-th$ time derivative of $f(u)$ at time $t_n$ in position $x_{i+j},$ where $i$ refers to the cell and $j$ to the position of the stencil. In general, for a scheme of order $2P,$ $k = 0,\ldots,2P-1$ and $j = -P+1,\ldots,P,$ under the assumption that $f_{i,j}^{(0)} = f(u_{i+j}^{n})$ 

%     \item $u_{i,j}^{k,n+r}$ is the explicit Taylor expansion in time truncated at order $k$ of $u$ centered at time $t_n$ at distance $r\Delta t$ in position $x_{i+j},$ where $i$ refers to the cell and $j$ to the position of the stencil. In general, for a scheme of order $2P,$ $k = 1,\ldots,2P-1,$ while $j,r = -P+1,\ldots,P.$

%     \item $f_{i,j}^{k,n+r}$ refers to $f\Bigl(u_{i,j}^{k,n+r}\Bigr).$
% \end{itemize}