\section{Introduction} \label{sec:introduction}
Lax-Wendroff methods  for linear systems of conservation laws are based on Taylor expansions in time in which the time derivatives are transformed into spatial derivatives using the governing equations  \cite{LeVeque2007book,Toro2009}. The spatial derivatives are then discretised by means of centered high-order differentiation formulas. 

 \add{One of the} 
difficulties to extend Lax-Wendroff methods to nonlinear problems come from the transformation of time derivatives into spatial derivatives through the Cauchy-Kovalesky (CK) procedure: this approach may indeed be impractical from the computational point of view. The Lax-Wendroff Approximate Taylor (LAT) methods introduced in  \cite{ZBM2017}  circumvent the CK procedure by computing  time derivatives in a recursive way using high-order centered differentiation formulas combined with  Taylor expansions in time. Compact Approximated Taylor methods (CAT) introduced in \cite{Carrillo-Pares}  follow a similar strategy.  These methods are compact in the sense that the length of the stencils is minimal: $(2P +1)$-point stencils are used to get order $2P$ compared to $4P +1$-point stencils in LAT methods. They are also  $L^2$ linearly-stable.

\add{A second difficulty comes from the treatment of shocks and discontinuities that usually arise in quasilinear systems of conservation laws.} 
In order to avoid the spurious oscillations that Lax-Wendroff-type methods produce in presence of discontinuities or high gradients, CAT methods were combined in \cite{CPZMR2020,TesiPhD,Macca-Pares} with an \textit{a priori} order adaptive procedure.
%was developed. In these ACAT methods, 
\raph{To do so} a family of smoothness indicators were used to automatically reduce the order of the method 
%close to
\raph{in the vicinity of}
discontinuities. 
\raph{These limited CAT methods are called ACAT.}

The goal of this paper is to combine \raph{1D} CAT methods with the \textit{a posteriori} Multi-dimensional Optimal Order Detection (MOOD) paradigm introduced in \cite{CDL1,CDL0_FVCA}. \ema{This technique is expected to produce non-oscillatory high-order methods with an appropriate detection of discontinuities} The \raph{resulting} methods will be applied to scalar conservation laws and the 1D Euler equations of gas-dynamics.

% Outline
The rest of this paper is organized as follows: 
the \add{next} section introduces the system of equations we plan to solve. 
In section three, CAT methods are recalled. 
The fourth section describes how to blend CAT schemes with MOOD.
\add{Numerical results are reported in the fifth section, illustrating the behavior} of the sixth-order CAT-MOOD method.
Conclusions and perspectives are finally drawn.
% =========================================