%======================================================================
%   PRESENTATION OF CAT SCHEMES --- STATE OF THE ART
%
\section{Compact Approximate Taylor (CAT) schemes} \label{sec:CAT} 
% The focus of this section is to introduce a family of numerical methods for one-dimensional non-linear systems of conservation law, named Compact Approximate Tayor (CAT) schemes.
% This family is based on an approximate Taylor procedure that constitutes a proper generalization of Lax-Wendroff (LW) method, in the sense that it reduces to the standard high-order LW method when the flux is linear. 
In this section we recall the formulation of CAT methods. In order to simplify the notation, the methods are introduced for scalar conservation laws \eqref{sec:CAT_gov_equ}, but the expression for systems \eqref{eq:systemPDEs_bis} is similar.

We consider uniform meshes in 1D: the space domain is split into computational cells  $\omega_i=[x_{i-1/2},x_{i+1/2}]$ of constant  \add{width} $\Delta x$. $x_i=\ha(x_{i+1/2}+x_{i-1/2})$ represents the center of the $i$-th cell.
Although in practice the time step depends on the CFL condition and thus it is not constant, for the sake of simplicity in the presentation of the methods it will be assumed that the time interval $[0,T]$ is split into sub-intervals $[t^n,t^{n+1}]$ of constant length $\Delta t$.

The $2P$-CAT method is written in conservative form as
\bea \label{eq:general_LW}
 u_i^{n+1} = u_i^n + \frac{\Delta t}{\Delta x}\left( F^P_{i-1/2}- F^P_{i+1/2}\right).
\eea
To compute the numerical flux $F^P_{i+1/2}$ only the approximations in the $2P$-point stencil
\bea \label{eq:stencil2P} 
\mathcal{S}_{i+\ha}^P = \left\{ x_{i-P+1},\ldots, x_{i+P} \right\}
\eea
are used, which ensures that  $(2P + 1)$ points are 
%RAPH
%used 
\raph{employed}
to update the numerical solution. 
The following formulas of numerical differentiation are used to compute the numerical fluxes:  given two positive integers $P$, $k$, an index $i$, and a real number $q$, we consider the interpolatory formula that approximates the $k$-th derivative of a function $f$  at the point $x_i + q \Delta x$ using its values at the $2P$  points $x_{i-P+1}, \dots, x_{i+P}$:
\begin{equation}\label{upwF}
f^{(k)}(x_i + q \Delta x) \approx  \mathcal{A}^{k,j}_{P} (f, \Delta x) = \frac{1}{\Delta x^k} \sum_{j = -P + 1}^P \gamma^{k,q}_{P,j} f(x_{i+j}).
\end{equation}
\raph{Notice that the case} $k = 0$ corresponds to Lagrange interpolation.  
When the formulas are applied to approximate the partial derivatives of a function $f(x,t)$ from some approximations $f_i^n \approx f(x_i, t_n)$, the symbol $\bcdot$ will be used to indicate to which variable (space or time) the differentiation is applied. For instance:
\begin{eqnarray*}
& &  \partial^k_x f(x_{i}, t_n)  \approx 
\mathcal{A}_{P}^{k,0} (f_{\bcdot}^n,\Delta x\Bigr)
=  \frac{1}{\Delta x^k} \sum_{j=-P+1}^{P} \gamma^{k,0}_{P,l} f_{i+j}^n, \\
& &  \partial^k_t f(x_{i}, t_n)  \approx  \mathcal{A}_{P}^{k,0}(f_i^{\bcdot},\Delta t)
 = \frac{1}{\Delta t^k} \sum_{r=-P+1}^{P} \gamma^{k,0}_{P,r} f_i^{n + r}.
 \end{eqnarray*}

Using this notation, the expression of the numerical flux is as follows:
\begin{equation}\label{cat2}
F^P_{i+1/2}  = \sum_{k=1}^{m} \frac{\Delta t^{k-1}}{k!}\mathcal{A}^{0, 1/2}_{P}(f_{i,\bcdot}^{(k-1)}, \Delta x),
\end{equation}
where 
\begin{equation}\label{f(k-1)_ij}
f^{(k-1)}_{i,j}  \approx \partial_t^{k-1}f(u)(x_{i+j}, t^n), \quad j=-P+1,\dots, P
\end{equation}
are \textit{local} approximations of the time derivatives of the flux. 
By \textit{local} we mean that these approximations depend on the stencil, i.e.
\raph{for two different stencils such that 
$i_1 + j_1 = i_2 + j_2$ then $f^{(k-1)}_{i_1,j_1}$ is not necessarily equal to $f^{(k-1)}_{i_2,j_2}$.}
%$$ i_1 + j_1 = i_2 + j_2  \not \Rightarrow f^{(k-1)}_{i_1,j_1} = f^{(k-1)}_{i_2,j_2}. $$



% The $2P$-CAT method uses a Taylor expansion to update the numerical solution in the 
% \bea \label{eq:general_LW}
%  u_i^{n+1} = u_i^n + \sum_{k=1}^{2P} \frac{(\Delta t)^k}{k!} u_i^{(k)},
% \eea
% where $u_i^n$ is the approximation of $u(x_i,t_n)$ and $u_i^{(k)}$ is an approximation of $\partial_t^k u(x_i,t_n)$. 

% CAT methods,  designed by Carrillo and Pares \cite{Carrillo-Pares}, are a variant of the  Approximate Taylor method, proposed by Zorio et al. \cite{ZBM2017}, that properly generalize  the Lax-Wendroff methods for linear systems. These methods are based on the conservative expression but the difference is that now the numerical flux $F_{i\pm 1/2}$ are computed using only the values of the stencils  $\mathcal{S}_{i\pm\ha}^P$ of size $2P$

% With this requirement, $u_i^{n+1}$ is locally updated using only the values at the centered $(2P + 1)$-point stencil. 

% The flux functions $F_{i\pm \ha}^P$ are then computed, respectively, on the sets $\mathcal{S}_{i\pm\ha}^P$
Since the exact solution satisfies
$$
\partial_t^k u = - \partial_t^{k-1} f(u),
$$
local approximations of the time derivatives of the solution are obtained \raph{by} 
%as follows
:
\begin{equation*}
u^{(k)}_{i,j} = - \mathcal{A}^{1,j}_{P}(f^{(k-1)}_{i, \bcdot}, \Delta x)
= - \frac{1}{\Delta x} \sum_{r=-P +1}^P \gamma^{1,j}_{P,r} f^{(k-1)}_{i, r}.
\end{equation*}
These approximations of the time derivatives are then adopted to compute \gio{predicted values of the flux at several time levels,  via recursive use of Taylor expansions, that will be then used to numerically approximate time derivatives of the flux at time level $n$.}
The algorithm to compute $F^p_{i+1/2}$ for the cell $i$ is then:
%as follows:
\begin{enumerate}
\item  {Define}
$$
f^{(0)}_{i,j}=f(u^n_{i+j}), \quad  j = -P+1, \dots, P.
$$
\item  {For $k = 2 \dots m$:}
\begin{enumerate}
\item Compute
\begin{equation*}
 u^{(k-1)}_{i,j} = - \mathcal{A}^{1,j}_{P}(f^{(k-2)}_{i,\bcdot}, \Delta x). 
\end{equation*}
\item Compute 
$$
f^{k-1,n+r}_{i,j} = f \left(  u^n_{i+j} + \sum_{l=1}^{k-1} \frac{(r \Delta t)^l}{l!} u^{(l)}_{i,j} \right), \quad  j, r = -P+1, \dots, P.
$$
\item Compute
$$
f^{(k-1)}_{i,j} =   \mathcal{A}^{k-1,0}_{P}( f^{k-1, \bcdot}_{i,j}, \Delta t),\quad  j = -P+1, \dots, P.
$$
\end{enumerate}
\item Compute $F^P_{i+1/2}$ by 
\begin{equation}
    \label{FP}
    F_{i+\ha}^P = \sum_{k=1}^{2P}\frac{\Delta t^{k-1}}{k!}f^{(k-1)}_{i+\ha},
\end{equation} where
$$ 
f^{(k-1)}_{i+\ha} = \mathcal{A}_{P}^{0,\ha}\left(f^{(k-1)}_{i,*},\Delta x\right), 
\quad 
\text{with}
\quad
\mathcal{A}_{P}^{0,\ha}\left(f^{(k-1)}_{i,*},\Delta x\right)  = \sum_{p=-P+1}^P\gamma_{P,p}^{0,\ha} \, f_{i+p}^{(k-1)}.
$$ 
\end{enumerate}


% $\mathcal{A}_{P}^{0,\ha}$ is an interpolation formulas of order $2P-1$ based on $2P$-point stencil.
% \end{enumerate}
% Once the numerical fluxes have been computed, the numerical solution is updated by using 
% \begin{equation}
%     \label{CAT2P}
%     u_i^{n+1} = u_i^n + \frac{\Delta t}{\Delta x}\left(F_{i-\ha}^P - F_{i+\ha}^P\right).
% \end{equation} 
% For the sake of clarity, details of the second order ($P=1$) CAT2 and fourth-order ($P=2$) CAT4 methods can be founded in



% \begin{rem}
%     Observe that the computation of the numerical flux  $F_{i+1/2}^{P}$ requires the approximation of $u$ at the nodes of a space-time grid of size $2P \times 2P$:, represented by $u_{i,j}^{k,n+r}$, for $-P+1 \leq j,r =\leq P$. 
%     The approximations of the solution $u$ at times $(n-P+1)\Delta t$, \dots, $(n-1)\Delta t$ are different from the ones  already computed in the previous steps $t^{n-P},\ldots, t^{n-1}$: $u_{i+j}^{n-P}$, \ldots, $u_{i+j}^{n-1}$.  
%     In other words, the discretization in time is not based on a multi-step method but on a one-step one. In fact, it can be re-interpreted as a Runge-Kutta method whose stages are $\tilde u^{n+r}_{i,j}$, $r = -P+1, \dots, P$, see \cite{Carrillo-Pares,CPZMR2020,Macca-Pares}.
% \end{rem}
% \begin{rem}
%     These approximations are local. Indeed, suppose that $i_1+j_1 = i_2+j_2 = \ell,$ i.e. $x_{\ell}>0$ belongs to $\mathcal{S}^P_{i_1+1/2}$ and $\mathcal{S}^P_{i_2+1/2}$ with local coordinates $j_1$ and $j_2$ respectively. Then, $f^{(k)}_{i_1,j_1}$ and $f^{(k)}_{i_2,j_2}$ are, in general, two different approximations of $\partial_t^k f(u)(x_{\ell},t_n).$
% \end{rem}



%=============================================================
%
%  ADD MOOD ONTO CAT --- HERE IS THE NEW STUFF
%
\section{CATMOOD} \label{sec:CATMOOD}
% The main objective of this work is to combine the \aposteriori shock capturing technique (MOOD) \cite{CDL1} to the family of one-step spatial and temporal reconstructions of high order CAT schemes.  

% The MOOD algorithm obtains an \aposteriori solution of the high-order numerical method using a family of criteria that detect a variety of, even minimal, oscillations. In this family it is therefore possible to consider some criteria such as positivity, monotonicity and physical properties that the numerical solution, obtained by the scheme, have to satisfy even in complex cases. 
% In addition, this procedure allows to reduce and eliminate the numerical oscillations introduced by the CAT methods in the presence of shocks or large .

The essential idea of the MOOD technique is to apply a high-order method over the entire domain for a time step, then check locally, for each cell $i$, the behavior of the solution using some admissibility criteria such as positivity, monotonicity, physical \add{admissibility}, etc. If the solution computed in cell $i$ at time $t^{n+1}$ is in accordance with the selected criteria, it is kept. Otherwise it is \raph{locally} recomputed with a  lower order numerical method. This operation is repeated until acceptability, or when a robust first order scheme is 
%used. 
\raph{employed.}

Bearing this in mind, the idea is to design a cascade of CAT methods in which the order is locally adjusted according to some \aposteriori admissibility criteria thus creating a new family of adaptive CAT methods called CAT-MOOD schemes. 


\subsection{MOOD admissibility criteria} \label{ssec:MOOD}
Following \cite{CDL2,CDL3}, we select three different admissibility criteria 
 which are used to check the admissibility  of a candidate numerical solution $\left\{ u_i^{n+1} \right\}_{1\leq i \leq N}$:
\begin{enumerate}
    \item \textit{Physical Admissible Detector (PAD)}: The first detector checks the physical validity of the candidate solution. In particular, this  detector reacts to negative solution when a variable cannot take negative values: this is the case of the pressure $p$ and density $\rho$ for the 1D Euler system 
    in compliance with (\ref{eq:physical_states}). 
    % For the Euler test the physicality here is assessed from the point of view of a fluid flow, which limits the generality of the criteria as far as predicted pressures are concerned. 
    % Said differently, this physical admissibility criteria must be adapted to the model of PDEs which is solved.
    \item \textit{Numerical Admissible Detector (NAD)}: This criterion is used to ensure the essentially non-oscillatory (ENO) character of the numerical solution, namely that no large and spurious minima or maxima are introduced locally in the solution. To do this, the following relaxed variant of the Discrete Maximum Principle (see \cite{Ciarlet,CDL1}) is \raph{considered}:
    $$\min_{j\in \mathcal{C}_i^P}(u_{j}^n) - \delta_i^n \le u_{i}^{n+1} \le 
    \max_{j \in \mathcal{C}_i^P}(u^n_{j}) + \delta_i^n, $$
    where {$\mathcal{C}_i^P = \{i-P,\ldots,i+P\}$} 
     is the $(2P +1)$-point centered stencil and $\delta_i$ is a parameter that avoids wrong detections in flat region. Here,  $\delta_i$ is set as:
    \begin{equation}
        \delta_i^n = \max\left(\gio{{\rm tol}_1},\gio{{\rm tol}_2},\max_{j\in \mathcal{C}_i^P}u^n_j - \min_{j\in \mathcal{C}_i^P}u^n_j\right). 
        \label{eq:delta}
    \end{equation}
 In the Euler test of Sec.~\ref{sec:numerics} the relaxed discrete maximum principle is only computed for density $\rho$ and pressure $p$. 
    \item \textit{Computer Admissible Detector (CAD)}: The last criterion detects 
    undefined or unrepresentable quantities, usually
    not-a-number \texttt{NaN} or infinity quantity \raph{which may appear}, for instance, when a division by zero \gio{is encountered}.
\end{enumerate}  
The order in which these three criteria are applied to the candidate solution is showed in Figure~\ref{fig:MOOD_chain_cascade}-left. If one of the criteria is not satisfied, the cell is marked \raph{as 'failed'}.
A new candidate solution is then computed in marked cells using a lower order method and it is checked again.
% In case of spurious oscillations have affected a candidate solution, $w_i^*$, then it will detect by the NAD detector and it will not pass at least one of the previous criteria ordered into a chain, see figure~\ref{fig:MOOD_chain_cascade}-left.
% As a consequence, the MOOD loop drives the code to locally downgrade the order of accuracy by using an auxiliary scheme of lower accuracy.
% ---- FIG ----
\begin{figure}[!ht]
    \centering    
     \begin{subfigure}[b]{0.76\textwidth}
         \centering
    \includegraphics[width=0.99\textwidth]{FIG/chain.pdf}
    \caption{Criteria}
    \end{subfigure}
     \begin{subfigure}[b]{0.16\textwidth}
         \centering
    \includegraphics[width=0.99\textwidth,angle = 270]{FIG/cascade.pdf}
    \caption{Cascade}
    \end{subfigure}
    \caption{Left: Detection criteria of the MOOD technique for a candidate solution $u_i^*$. \textit{Computer Admissible Detector (CAD)}, \textit{Physical Admissible Detector (PAD)}  and \textit{Numerical Admissible Detector (PAD)} ---
    Right: Order cascades of CAT schemes used in the MOOD procedure. Starting from the most accurate one, CAT6, downgrading to lower order schemes, and, at last to a $1$st order accurate scheme employed to ensure robustness.}
   \label{fig:MOOD_chain_cascade}
\end{figure}
% ---- FIG ----

\subsection{CAT scheme with MOOD limiting} \label{ssec:CATMOOD} 
In this work, CAT methods are used as high-order methods within the MOOD strategy: the natural idea would be, \raph{given a target order $2P$}, to use the \raph{following} cascade of numerical methods 
$$\text{CAT}2P \to \text{CAT}2(P-1) \to \dots \text{CAT}2 \to \text{First-order method}$$ to  obtain a method with  order of accuracy $2P$ in smooth regions and an essentially non-oscillatory solution close to discontinuities or large gradients. 
%Nevertheless, 
\raph{However,}
in order to reduce the computational cost, the following cascade has been 
%used 
\raph{preferred}
in the numerical tests below
$$\text{CAT}6 \to \text{CAT}2  \to \text{First-order method}$$ 
where the first-order method is Rusanov for scalar problems and HLL for Euler equations: see Figure~\ref{fig:MOOD_chain_cascade}-right \raph{for an illustration}. Therefore, the expected order of accuracy in smooth regions is 6. The resulting scheme is referred to as CATMOOD6. 
% \ema{To obtain a very accurate numerical solution, the best approach is to use a scheme that is of high order in the smooth region and that goes gradually reducing the order of accuracy when, the considered region, is not sufficiently smooth. In this spirit, the purpose of this work is to create a numerical solution that is accurate up to the $6$th order in the smooth domain and that is reducing the order to 2 to 2 as long as a first order method is used. In this regard, the idea is to get a solution that is at most $6$th order of accuracy in the smooth region while it is $4$th order when CAT6 fails or $2$th order when CAT4 fails. At least a $1$th order scheme is adopted when all CAT2-4-6 fail. In numerical experiments, to obtain a scheme that has a lower computational cost without affecting the accuracy of the numerical solution, we preferred to use three numerical methods instead of four. In particular, as intermediate schemes between CAT6 and a first order one, only CAT2 has been adopted. In order to achieve the most accurate solution in the smooth region and the non-oscillatory solution in the region in which large gradient or discontinuities occur, the idea is adopt CAT6 as often as possible.\footnote{\ema{In the case where the numerical solution is regular up to order 6, the error will still be $O(\Delta x^6)$; while, in the case where a discontinuity or a large gradient is present in the stencil, the error will be of the order $O(\Delta x)$ or $O(\Delta x^2)$ which is greater than $O(\Delta x^4)$.}}} On the contrary, for cells that present a discontinuous solution we plan to rely on a $1$st order low accurate but robust scheme, for instance using Rusanov or HLL fluxes. Clearly, we would like to employ only when and where necessary. In our tests, between these two extremes the CAT2 schemes of $2$th order of accuracy is inserted.
% As such we build several cascades of schemes of decreasing orders but possibly increasing robustness, see Figure~\ref{fig:MOOD_chain_cascade}-right.
% This scheme is referred to as CATMOOD6.

  
% \subsection{Algorithm} \label{ssec:algo}
% Practically, for the time-step $[t^n,t^{n+1}]$ and for each cell $i$, we define a 'mask', $A_i^{n}\in\{-1,0,1\}$, such that
% \bea \label{eq:mask}
% A_i^n = \begin{cases}
%            1 \quad & \text{if} \; u_i^* \; \text{fails at least one criterion},  \\ 
%            -1\quad & \text{if} \; \exists  j \in \mathcal{N}_i \; \text{s.t} \; A_j^n=1,\\ 
%            0 \quad & \text{otherwise}.
%          \end{cases}
% \eea
% where $\mathcal{N}_i$ is the set of direct neighbor cells of cell $\omega_i$. $A_i^n=-1$ means that cell $i$ is the neighbor of an invalid cell. \\
% The algorithm designed for CATMOOD6 scheme is:
% \begin{enumerate}
%     \item[Init] Let be $\{u_i^n\}_{1\leq c\leq N_c}$ the numerical solution at $t = t^n$ over all the domain $\Omega.$
%     \item[CAT6] Let be $\{u_i^*\}_{1\leq c\leq N_c}$ the numerical solution at $t = t^{n+1}$ of order 6 obtained from CAT6 scheme. Set $A_i^n=0$.\\
%     For all cell $i$, check if $u_i^*$ satisfies all detection criteria. 
%     In this case, then $u_i^{n+1} = u_i^*$ and $A_i^n=0$.
%     Otherwise $A_i^n=1$ and, if $A_j^n=0$ then set $A_j^n=-1$ for all $j \in \mathcal{N}_i$.
%     \item[CAT2P] Only for the troubled cell $i$, i.e $A_i^n\not= 0$, recompute $\{u_i^*\}$ the numerical solution at time $t = t^{n+1}$ of order 2 obtained with CAT2 scheme. \\
%    Check if $u_i^*$ satisfies all detection criteria.   
%     In this case, then $u_i^{n+1} = u_i^*$ and $A_i^n=0$.
%     Otherwise $A_i^n=1$ and, if $A_j^n=0$ then set $A_j^n=-1$ for all $j \in \mathcal{N}_i$.
%     \item[1st-ord] Only for the remaining troubled cell $i$, i.e $A_i^n\not= 0$, recompute $\{u_i^*\}$ the numerical solution at time $t = t^{n+1}$ of order 1 obtained with a first order scheme, and set $u_i^{n+1} = u_i^*$.
% \end{enumerate}
% Notice that if a CAT4 scheme is used in the cascade in-between CAT6 and CAT2, then one adds another CAT2P step in the algorithm.
