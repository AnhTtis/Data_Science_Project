%============================================================
%    2 D   N U M E R I C S
%
\section{Numerical test cases}  \label{sec:numerics}
% In this paper only numerical tests related to the one-dimensional equations are considered. Our methodology of testing relies on classical test cases: 
% \begin{enumerate}
%     \item \textit{Scalar linear equation with smooth initial condition}. This test measures the ability of the MOOD procedure combined with CAT schemes to achieve the optimal high order for a smooth solution. We also compared CATMOOD6 with Rusanov first order scheme.
%     \item \textit{Burgers equation with non-smooth initial condition}. This test measures the ability of the MOOD technique to eliminate oscillations near discontinuities introduced by CAT methods \cite{CDL1,CPZMR2020,Carrillo-Pares,TesiPhD}.
%     \item \textit{Sod problem}. This test presents smooth regions, a rarefaction wave, a contact discontinuity and shock. The MOOD technique deletes the oscillations close to the discontinuity \cite{Toro2009}. 
% \end{enumerate}

\gio{Three tests are considered, namely linear advection equation, to check the numerical order of accuracy,  Burgers equation, and Euler equations of gas dynamics, to check shock-capturing capability of the method. 
In all our tests we used the following parameters: 
${\rm tol}_1 = 10^{-4}$ and 
${\rm tol}_2 = 10^{-3}$.}

\subsection{Scalar linear equation with smooth initial condition} \label{ssec:linear}
Let us consider the linear conservation law \eqref{sec:CAT_gov_equ} with \gio{$f(u) = u$, and periodic boundary conditions in $[0,2\pi]$}. The
smooth initial condition is given by
\begin{equation}\label{smooth_1}
u(x,0) = u_0(x) = \ha\sin(x) + 1.
\end{equation} 
The goal of this test is to check and compare the empirical order of accuracy of CATMOOD6. %and 
For smooth solutions, the detectors are expected not to spoil the sixth-order of accuracy of the CAT6 \raph{scheme}. We run this test case in the interval $[0,2\pi],$ with final time $t_{fin} = 1$, CFL$=0.9,$ and periodic boundary conditions. We apply CAT6 and CATMOOD6 method on successive refined uniform meshes going from $10$   to  $320$ cells. As expected the empirical order of accuracy of both CAT and CATMOOD is 6: see Table \ref{tab:linear_convergence}. 
\raph{We observe that below $N=80$ cells, the mesh is not fine enough to allow for a clean limiting.
\gio{Notice that this threshold depends on the parameteters ${\rm tol}_1$ and ${\rm tol}_2$ adopted in \eqref{eq:delta}. Larger values of these parameter will lower the threshold to smaller values of $N$.} 
Moreover, in this test, CATMOOD6 is $1.5$ more expensive than CAT6.}
% What we expect.
% Since the solution is smooth, it should be simulated with optimal high accuracy, in other words, the limiting/stabilization procedure employed in the scheme should not have any effect. 
% % --- TAB ---
\begin{table}[!ht]
%\numerikNine
    \centering
    \begin{tabular}{|c||ccc|ccc|}
    \hline \multicolumn{7}{|c|}{\textbf{Linear equation - Error, Rate of convergence, CPU time}} \\
\hline   
\hline   
& \multicolumn{3}{c|}{\textbf{CAT6}} &\multicolumn{3}{c|}{\textbf{CATMOOD6}}\\
 $N$ &  $L^1$ error &  order & CPU time & $L^1$ error &  order & CPU time \\ \hline
10 & 4.27$\times 10^{-5}$  &  ---  & 0.0073    & 8.34$\times 10^{-3}$ & ---  & 0.016    \\
20 & 6.93$\times 10^{-7}$    &  5.95  & 0.020    & 3.18$\times 10^{-3}$ & 1.39  & 0.031    \\
40 & 9.64$\times 10^{-9}$    &  6.17  & 0.038     & 3.16$\times 10^{-4}$ & 3.33  & 0.053    \\
80 & 1.43$\times 10^{-10}$   &  6.07  & 0.068    &  2.48$\times 10^{-6}$ &  3.67  & 0.094    \\
160 & 2.09$\times 10^{-12}$  &  6.09  & 0.13    & 2.09$\times 10^{-12}$ & 23.50  & 0.2    \\
320 & 3.29$\times 10^{-14}$  &  5.99  & 0.25    & 3.31$\times 10^{-14}$ & 5.99   & 0.38    \\
& \text{Expected} & 6 &
& \text{Expected} & 6 &
\\
    \hline 
    \end{tabular}
    \caption{Linear scalar equation \ref{ssec:linear}.  $L^1-$norm errors between the numerical solution and the exact solution of the linear equation at $t_{\text{final}} = 1$ on uniform Cartesian mesh and CFL$=0.9.$
    %\gio{The change between 80 and 160 points is dramatic! Furthremore it is evident that MOOD does not do anything with 160 points, while it dramatically affects the accuracy with just 80 points. We should at least comment about this.}
   }
    \label{tab:linear_convergence}
\end{table}


\vspace{-0.6cm}

\subsection{Burgers' equation with non-smooth initial condition} \label{ssec:burgers}
Let us consider \eqref{sec:CAT_gov_equ} with \gio{$f(u) = u^2/2$, periodic boundary conditions,} and non-smooth initial condition
\begin{equation} \label{square_step_test}
u_0(x) = \begin{cases}
1.1 \quad \;\;\;\mathrm{if}\quad 0\le x\le \ha;\\
2.1 \quad\;\; \;\mathrm{if}\quad \ha< x < \frac{3}{2};\\
0.1 \quad \;\;\; \mathrm{if} \quad \frac{3}{2}\le x\le\frac{17}{10}. 
\end{cases}
\end{equation} 
We run this test case with first-order Rusanov-flux and  CATMOOD6 method in the interval $[0,1.7],$
with final time $t_{fin} = 0.65$, 
using a 50-cell mesh, CFL$=0.9,$ and periodic boundary conditions.
Figure~\ref{fig:burgers} shows the initial condition and the numerical solutions obtained with both methods: it can be seen that CATMOOD6 provides a non-oscillatory solution \gio{still providing better resolution than the first order method. Notice that larger values of the thresholds tol$_1$ and tol$_2$ will decrease dissipation, but may not be sufficient to avoid creation of spurious oscillations.}

\begin{figure}[!ht]
\hspace{-1.cm}
    \centering    
    \begin{subfigure}[b]{0.48\textwidth}
         \centering
    \includegraphics[width=1.12\textwidth]{FIG/Burgers/u_no_smooth_mid_1.pdf}
    \vspace{-0.5cm}
    \caption{Numerical solutions at time $t=0.3$.}
    \end{subfigure}
     \begin{subfigure}[b]{0.48\textwidth}
         \centering
    \includegraphics[width=1.12\textwidth]{FIG/Burgers/u_no_smooth_fin_1.pdf}
    \vspace{-0.5cm}
    \caption{Numerical solutions at time $t=0.65$.}
    \end{subfigure}
    \caption{Burgers' equation with non-smooth initial condition \ref{ssec:burgers} --- Left: Numerical solutions at time $t = 0.3$ --- Right: Numerical solutions at final time $t_{fin} = 0.65$ obtained with Rusanov flux and CATMOOD6 on $50$ uniform mesh and CFL$=0.9$. Rusanov flux scheme has been used as first order. The reference solution has been obtained with the 1st order Rusanov scheme on $2000$ uniform cells. }
   \label{fig:burgers}
\end{figure}
%\raph{Question: maybe using a finer mesh say 500 cells to put on top of figure 2 to see the convergence could be good?}
\vspace{-0.6cm}



\subsection{Euler system: Sod problem} \label{ssec:Sod}
Let us consider Euler equations \eqref{sec:CAT_gov_equ} with SOD initial condition 

\begin{equation} \label{Sod_IC}
	(\rho, \raph{u}, p)
	= \left\{
	\begin{array}{ll}
	\displaystyle (1,0,1) & \mbox {if } x \le 0, \\
	\displaystyle (0.125,0,0.1) & \mbox {if } x > 0.
	\end{array}\right.
\end{equation}
%\vspace{-1cm}

\begin{figure}[!ht]
\hspace{-1.cm}
    \centering    
    \begin{subfigure}[b]{0.48\textwidth}
         \centering
    \includegraphics[width=1.12\textwidth]{FIG/Sod/density_1.pdf} 
    \vspace{-0.5cm}
    \caption{Density.}
    \end{subfigure}
     \begin{subfigure}[b]{0.48\textwidth}
         \centering
    \includegraphics[width=1.12\textwidth]{FIG/Sod/density_zoom_1.pdf}
    \vspace{-0.5cm}
    \caption{Zoom of the density on shock}
    \end{subfigure} 
    \\
    \hspace{-1.cm}
    \centering    
    \begin{subfigure}[b]{0.48\textwidth}
         \centering
    \includegraphics[width=1.12\textwidth]{FIG/Sod/vel.pdf} 
    \vspace{-0.5cm}
    \caption{Velocity.}
    \end{subfigure}
     \begin{subfigure}[b]{0.48\textwidth}
         \centering
    \includegraphics[width=1.12\textwidth]{FIG/Sod/press.pdf}
    \vspace{-0.5cm}
    \caption{Pressure}
    \end{subfigure}
    \caption{Euler system: Sod problem \ref{ssec:Sod} --- Left-up: Numerical solutions for density $\rho$ at final time $t=0.3$ obtained with \raph{1st order HLL} 
    %Rusanov 
    flux and CATMOOD6, on $200$ uniform mesh and CFL$=0.9$. --- Right-up: Zoom of the numerical solutions and reference one. --- 
    Left/right-down: same as left-up but for velocity and pressure variables. The reference solution is the exact one.}
   \label{fig:Sod}
\end{figure}
%\vspace{-0.5cm}
We run this test case with first-order Rusanov-flux and CATMOOD6 method in the interval $[-1,1],$ with final time $t_{fin} = 0.3$. We use  a $200-$cell mesh, CFL$=0.9,$ and free boundary conditions.
Figure~\ref{fig:Sod} shows the numerical and  the exact solutions for density, velocity and pressure obtained with both methods and a zoom of the density close to the shock. It can be seen that, as expected, CATMOOD6 solution shows a better resolution of rarefaction and contact waves. 