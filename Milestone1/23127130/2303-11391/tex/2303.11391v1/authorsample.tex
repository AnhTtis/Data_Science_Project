%%%%%%%%%%%%%%%%%%%% author.tex %%%%%%%%%%%%%%%%%%%%%%%%%%%%%%%%%%%
%
% sample root file for your "contribution" to a contributed volume
%
% Use this file as a template for your own input.
%
%%%%%%%%%%%%%%%% Springer %%%%%%%%%%%%%%%%%%%%%%%%%%%%%%%%%%


% RECOMMENDED %%%%%%%%%%%%%%%%%%%%%%%%%%%%%%%%%%%%%%%%%%%%%%%%%%%
\documentclass[graybox]{svmult}

% choose options for [] as required from the list
% in the Reference Guide



\usepackage{type1cm}        % activate if the above 3 fonts are
                            % not available on your system
%
\usepackage{makeidx}         % allows index generation
\usepackage{graphicx}        % standard LaTeX graphics tool
                             % when including figure files
\usepackage{multicol}        % used for the two-column index
\usepackage[bottom]{footmisc}% places footnotes at page bottom


\usepackage{newtxtext}       % 
\usepackage[varvw]{newtxmath}       % selects Times Roman as basic font
\usepackage{epstopdf}
%\usepackage{fullpage}
\usepackage{caption}
\usepackage{subcaption}
\newcommand{\tr}{\text{tr}}
\newcommand{\Frac}{\displaystyle\frac}
\newcommand{\Int}{\displaystyle\int}
\newcommand{\Sum}{\displaystyle\sum}
\newcommand{\matr}[1]{\mathbf{{#1}}}          
% RAPH
\newcommand{\nn}{\nonumber}
\newcommand{\dt}{\Delta t}
\newcommand{\dx}{\Delta x}
\newcommand{\dy}{\Delta y}
\newcommand{\dz}{\Delta z}
\newcommand{\sign}{{\rm sign}}
\newcommand{\CFL}{\text{CFL}}
%
\renewcommand{\S}{\mathbf{S}}
\newcommand{\U}{\mathbf{U}}
\newcommand{\W}{\mathbf{W}}
\newcommand{\F}{\mathbf{F}}
\newcommand{\f}{\mathbf{f}}
\newcommand{\z}{\mathbf{z}}
\newcommand{\x}{\mathbf{x}}
\newcommand{\n}{\mathbf{n}}
\renewcommand{\u}{\mathbf{u}}

%%%%%%%%%%%%%%%%%%%%%%%%%%%%%%%%%%%%%%%%%
%--- Emanuele newcommand
\def\bcdot{{*}}
\newcommand{\Z}{\mathbb{Z}}
\newcommand{\ha}{\frac{1}{2}}
\newcommand{\smax}{s^{\rm max}}
\newcommand{\bi}{\mathbf{i}}
\newcommand{\bj}{\mathbf{j}}
\newcommand{\bhalf}{\mathbf{\ha}}
\newcommand{\bef}{\mathbf{e}_1}
\newcommand{\bes}{\mathbf{e}_2}
\newcommand{\bzero}{\mathbf{0}}
\newcommand{\bone}{\mathbf{1}}
\newcommand{\abs}[1]{\left| #1 \right|}
\newcommand{\um}{u_{\max}}
%%%%%%%%%%%%%%%%%%%%%%%%%%%%%%%%%%%%%%%%%
\usepackage[normalem]{ulem}

%%%%%%%%%%%%%%%%%%%%%%%%%%%%%%%%%%%%%%%%%
\newtheorem{rem}{Remark}
%\newtheorem{prop}{Proposition}
%\newtheorem{thm}{Theorem}
%\newtheorem{def}{Definition}
%\newtheorem{algo}{Algorithm}
\newtheorem{com}{Comment}
\newtheorem{lem}{Lemma}
%%%%%%%%%%%%%%%%%%%%%%%%%%%%%%%%%%%%%%%%%
\def\be{\begin{equation}}
\def\ee{\end{equation}}
\newcommand{\ei}[0]{\end{itemize}}
\newcommand{\beann}[0]{\begin{eqnarray*}}
\newcommand{\eeann}[0]{\end{eqnarray*}}
\def\bea{\begin{eqnarray}}
\def\eea{\end{eqnarray}}
\def\ba{\begin{array}{l}\displaystyle}
\def\ea{\end{array}}
%%%%%%%%%%%%%%%%%%%%%%%%%%%%%%%%%%%%%%%%%%%%%%%%
% TextText
\newcommand{\ie}{\textit{i.e} }
\newcommand{\apriori}{\textit{a priori} }
\newcommand{\aposteriori}{\textit{a posteriori} }

\newcommand{\carlos}[1]{{\color{black} #1}}
\newcommand{\ema}[1]{{\color{black} #1}}
\newcommand{\raph}[1]{{\color{black} #1}}
\newcommand{\gio}[1]{{\color{black} #1}}
\newcommand{\add}[1]{{\color{black} #1}}

% see the list of further useful packages
% in the Reference Guide

\makeindex             % used for the subject index
                       % please use the style svind.ist with
                       % your makeindex program

%%%%%%%%%%%%%%%%%%%%%%%%%%%%%%%%%%%%%%%%%%%%%%%%%%%%%%%%%%%%%%%%%%%%%%%%%%%%%%%%%%%%%%%%%

\begin{document}

\title{CAT-MOOD methods for conservation laws in one space dimension}
% Use \titlerunning{Short Title} for an abbreviated version of
% your contribution title if the original one is too long
\author{R. Loubere, E. Macca, C. Pares and G. Russo}
\institute{Emanuele Macca, corresponding author, \at Universit{y} of Catania, \email{emanuele.macca@unict.it}
\and Rapha{\"e}l Loub{\`e}re \at  Universit{y} of Bordeaux, CNRS, \email{raphael.loubere@math.u-bordeaux.fr} 
\and Carlos Par{\'e}s \at  University of M{\'a}laga, \email{pares@uma.es}
\and Giovanni Russo \at Universit{y} di Catania, \email{russo@dmi.unict.it}
}



% Use the package "url.sty" to avoid
% problems with special characters
% used in your e-mail or web address
%
\maketitle

\abstract{In this paper we blend high-order Compact Approximate Taylor (CAT) numerical methods with the \textit{a posteriori} Multi-dimensional Optimal Order Detection (MOOD) paradigm to solve hyperbolic systems of conservation laws. The resulting methods 
%present
are highly accurate for smooth solutions, essentially non-oscillatory for discontinuous ones, and almost fail-safe positivity preserving. 
Some numerical results for scalar conservation laws and systems are presented to show the appropriate behavior of CAT-MOOD methods.
}

 %\tableofcontents

\section{Introduction} \label{sec:introduction}
Lax-Wendroff methods  for linear systems of conservation laws are based on Taylor expansions in time in which the time derivatives are transformed into spatial derivatives using the governing equations  \cite{LeVeque2007book,Toro2009}. The spatial derivatives are then discretised by means of centered high-order differentiation formulas. 

 \add{One of the} 
difficulties to extend Lax-Wendroff methods to nonlinear problems come from the transformation of time derivatives into spatial derivatives through the Cauchy-Kovalesky (CK) procedure: this approach may indeed be impractical from the computational point of view. The Lax-Wendroff Approximate Taylor (LAT) methods introduced in  \cite{ZBM2017}  circumvent the CK procedure by computing  time derivatives in a recursive way using high-order centered differentiation formulas combined with  Taylor expansions in time. Compact Approximated Taylor methods (CAT) introduced in \cite{Carrillo-Pares}  follow a similar strategy.  These methods are compact in the sense that the length of the stencils is minimal: $(2P +1)$-point stencils are used to get order $2P$ compared to $4P +1$-point stencils in LAT methods. They are also  $L^2$ linearly-stable.

\add{A second difficulty comes from the treatment of shocks and discontinuities that usually arise in quasilinear systems of conservation laws.} 
In order to avoid the spurious oscillations that Lax-Wendroff-type methods produce in presence of discontinuities or high gradients, CAT methods were combined in \cite{CPZMR2020,TesiPhD,Macca-Pares} with an \textit{a priori} order adaptive procedure.
%was developed. In these ACAT methods, 
\raph{To do so} a family of smoothness indicators were used to automatically reduce the order of the method 
%close to
\raph{in the vicinity of}
discontinuities. 
\raph{These limited CAT methods are called ACAT.}

The goal of this paper is to combine \raph{1D} CAT methods with the \textit{a posteriori} Multi-dimensional Optimal Order Detection (MOOD) paradigm introduced in \cite{CDL1,CDL0_FVCA}. \ema{This technique is expected to produce non-oscillatory high-order methods with an appropriate detection of discontinuities} The \raph{resulting} methods will be applied to scalar conservation laws and the 1D Euler equations of gas-dynamics.

% Outline
The rest of this paper is organized as follows: 
the \add{next} section introduces the system of equations we plan to solve. 
In section three, CAT methods are recalled. 
The fourth section describes how to blend CAT schemes with MOOD.
\add{Numerical results are reported in the fifth section, illustrating the behavior} of the sixth-order CAT-MOOD method.
Conclusions and perspectives are finally drawn.
% =========================================

%======================================================================
% CONTEXT, MODEL, EQUATION, CAT
%
\section{Governing equations} \label{sec:equations}
We consider 1D hyperbolic systems of conservation laws of the form
\bea \label{eq:systemPDEs_bis}
    \partial_t \U + \partial_x \mathbf{F}(\U) = \mathbf{0},
\eea
where $t\in \mathbb{R}^+$ represents the time variable, $x\in \mathbb{R}$ the space variable and $\U=\U(x,t)
\in \mathbb{R}^M$ is the vector of conserved variables while 
$\mathbf{F}(\U(x,t)) \in \mathbb{R}^M$ is the flux vector. More precisely, we focus on the 1D Euler equations of gas dynamics in which $M = 3$,  $\U=(\rho, \rho u,  \rho e)^t$ with $\rho$ the density, $u$ the velocity and 
%RAPH
%$e=\varepsilon + \frac12 \| u \|^2$ 
$e=\varepsilon + \frac12 \raph{u^2}$ the total energy \add{per unit mass}, being $\varepsilon$ the specific \raph{internal energy}.
%one.
The flux is given by 
$\mathbf{F}(\U)= \left(
\rho u, \rho u^2 +p, (\rho e +p )u\right)^t$.
%\bea
% \mathbf{F}(\U)= \left( \begin{array}{c}
%\rho u \\
%\rho u^2 +p  \\
%(\rho e +p )u  \\
%\end{array} \right).
%\eea
The system is closed with the \raph{perfect gas} equation of state:
%RAPH: !!! EOS as a function of \varepsilon not e.
%$$ p(\rho,e) = (\gamma-1)\rho e. $$
$ \raph{p(\rho,\varepsilon) = (\gamma-1)\rho \varepsilon}. $
The equations in (\ref{eq:systemPDEs_bis}) represent the conservation of mass, momentum and total energy.
An entropy inequality has to be 
%added 
\raph{supplemented to deal with discontinuous solutions}.
This system is hyperbolic with eigenvalues $\lambda^-=u - c$, $\lambda^0=u,$  $\lambda^+=u + c$ where $c = \sqrt{\gamma p/\rho}$ \raph{is the sound speed}.
The set of admissible states, i.e. states that are consistent with physics, is
\bea \label{eq:physical_states}
    \mathcal{A} = \left\{ \U\in \mathbb{R}^M, \; \text{such that} \; \rho>0, \; p>0 \right\}.
\eea
%
Together with this system, two scalar conservation laws
\raph{of the form}
\begin{equation}
    \label{sec:CAT_gov_equ}
    u_t + \partial_x f(u) = 0,
\end{equation} 
will be considered 
%RAPH: sound weird to me... but it may only be me!
%as well 
to test the methods: \raph{first} the linear advection equation corresponding to $f(u)=b u$ with $b\in \mathbb{R}$, and, \raph{secondly}, Burgers' equation corresponding to $f(u)=u^2/2$.  
\raph{These scalar conservation laws verify the maximum principle, that is
\begin{eqnarray} \label{eq:physical_states2}
    \mathcal{A} & = & \left\{ u\in \mathbb{R}, \; \text{s.t.} \; \underline{u}_0 \leq u(x,t) \leq \overline{u}_0 \right\}, 
    \nonumber\\
    \underline{u}_0 & = & \min_x( u(x,t_0)), \; \overline{u}_0=\max_x( u(x,t_0)), 
\end{eqnarray}
where $t_0$ is the initial time, and $u(x,t_0)$ the initial condition.
}


% \subsection{1D linear and non-linear scalar conservation laws}
% To drastically simplify the description of the numerical schemes we explain the CAT procedure to non-linear scalar conservation laws. For this reason, let us consider the non-linear scalar conservation law on the $Oxt$-Cartesian frame

% where $u=u(x,t):\mathbb{R}\times\mathbb{R}^+\rightarrow\mathbb{R}$ represents the scalar variable, and, $f(u)=f(u(x,t))$ the non-linear flux depending on $u$.
% $u(x,0)=u_0(x)$ is the solution at initial time, while BC depend on the test case; in this chapter we adopt periodic ones and Dirichlet ones. \\
% Eq. (\ref{sec:CAT_gov_equ}) denotes the generic model of the non-linear scalar equation. The simplest one is probably represented by

% \subsection{Mesh}
% In this chapter we consider uniform meshes in 1D. For this reasoning, the space domain can be represented by a segment $\Omega$ and it is uniformly discretized by $N$ cells.  In general, a cell, called $\omega_i,$ is identified so that $\omega_i=[x_{i-1/2};x_{i+1/2}]$ while the cell center is given by $x_i=\ha(x_{i+1/2}+x_{i-1/2})$. Then, the size of is fixes as $\Delta x$. 

% The time domain $\mathcal{T}=[0,T]$ with final time $T>0$ is split into time intervals $[t^n,t^{n+1}]$, $n\in \mathbb{N}$, and time-steps $\Delta t_n=t^{n+1}-t^n$ subject to a CFL (Courant-Friedrichs-Lewy) like condition. In practice, it is better to assign $\Delta t$ dynamically at each time step by imposing some CFL condition. The choice of constant time step here is adopted in order to simplify the notation in the description of the method.

% \subsection{Notation}
% In this article we refer with the following notations for the different type of derivatives or approximations.
% \begin{itemize}
%     \item $u_{i,j}^{(k)}$ is the $k-th$ time derivative of $u$ at time $t_n$ in position $x_{i+j},$ where $i$ refers to the cell and $j$ to the position of the stencil. In general, for a scheme of order $2P,$ $k = 1,\ldots,2P-1$ and $j = -P+1,\ldots,P.$

%     \item $f_{i,j}^{(k)}$ is the $k-th$ time derivative of $f(u)$ at time $t_n$ in position $x_{i+j},$ where $i$ refers to the cell and $j$ to the position of the stencil. In general, for a scheme of order $2P,$ $k = 0,\ldots,2P-1$ and $j = -P+1,\ldots,P,$ under the assumption that $f_{i,j}^{(0)} = f(u_{i+j}^{n})$ 

%     \item $u_{i,j}^{k,n+r}$ is the explicit Taylor expansion in time truncated at order $k$ of $u$ centered at time $t_n$ at distance $r\Delta t$ in position $x_{i+j},$ where $i$ refers to the cell and $j$ to the position of the stencil. In general, for a scheme of order $2P,$ $k = 1,\ldots,2P-1,$ while $j,r = -P+1,\ldots,P.$

%     \item $f_{i,j}^{k,n+r}$ refers to $f\Bigl(u_{i,j}^{k,n+r}\Bigr).$
% \end{itemize}


%======================================================================
%   PRESENTATION OF CAT SCHEMES --- STATE OF THE ART
%
\section{Compact Approximate Taylor (CAT) schemes} \label{sec:CAT} 
% The focus of this section is to introduce a family of numerical methods for one-dimensional non-linear systems of conservation law, named Compact Approximate Tayor (CAT) schemes.
% This family is based on an approximate Taylor procedure that constitutes a proper generalization of Lax-Wendroff (LW) method, in the sense that it reduces to the standard high-order LW method when the flux is linear. 
In this section we recall the formulation of CAT methods. In order to simplify the notation, the methods are introduced for scalar conservation laws \eqref{sec:CAT_gov_equ}, but the expression for systems \eqref{eq:systemPDEs_bis} is similar.

We consider uniform meshes in 1D: the space domain is split into computational cells  $\omega_i=[x_{i-1/2},x_{i+1/2}]$ of constant  \add{width} $\Delta x$. $x_i=\ha(x_{i+1/2}+x_{i-1/2})$ represents the center of the $i$-th cell.
Although in practice the time step depends on the CFL condition and thus it is not constant, for the sake of simplicity in the presentation of the methods it will be assumed that the time interval $[0,T]$ is split into sub-intervals $[t^n,t^{n+1}]$ of constant length $\Delta t$.

The $2P$-CAT method is written in conservative form as
\bea \label{eq:general_LW}
 u_i^{n+1} = u_i^n + \frac{\Delta t}{\Delta x}\left( F^P_{i-1/2}- F^P_{i+1/2}\right).
\eea
To compute the numerical flux $F^P_{i+1/2}$ only the approximations in the $2P$-point stencil
\bea \label{eq:stencil2P} 
\mathcal{S}_{i+\ha}^P = \left\{ x_{i-P+1},\ldots, x_{i+P} \right\}
\eea
are used, which ensures that  $(2P + 1)$ points are 
%RAPH
%used 
\raph{employed}
to update the numerical solution. 
The following formulas of numerical differentiation are used to compute the numerical fluxes:  given two positive integers $P$, $k$, an index $i$, and a real number $q$, we consider the interpolatory formula that approximates the $k$-th derivative of a function $f$  at the point $x_i + q \Delta x$ using its values at the $2P$  points $x_{i-P+1}, \dots, x_{i+P}$:
\begin{equation}\label{upwF}
f^{(k)}(x_i + q \Delta x) \approx  \mathcal{A}^{k,j}_{P} (f, \Delta x) = \frac{1}{\Delta x^k} \sum_{j = -P + 1}^P \gamma^{k,q}_{P,j} f(x_{i+j}).
\end{equation}
\raph{Notice that the case} $k = 0$ corresponds to Lagrange interpolation.  
When the formulas are applied to approximate the partial derivatives of a function $f(x,t)$ from some approximations $f_i^n \approx f(x_i, t_n)$, the symbol $\bcdot$ will be used to indicate to which variable (space or time) the differentiation is applied. For instance:
\begin{eqnarray*}
& &  \partial^k_x f(x_{i}, t_n)  \approx 
\mathcal{A}_{P}^{k,0} (f_{\bcdot}^n,\Delta x\Bigr)
=  \frac{1}{\Delta x^k} \sum_{j=-P+1}^{P} \gamma^{k,0}_{P,l} f_{i+j}^n, \\
& &  \partial^k_t f(x_{i}, t_n)  \approx  \mathcal{A}_{P}^{k,0}(f_i^{\bcdot},\Delta t)
 = \frac{1}{\Delta t^k} \sum_{r=-P+1}^{P} \gamma^{k,0}_{P,r} f_i^{n + r}.
 \end{eqnarray*}

Using this notation, the expression of the numerical flux is as follows:
\begin{equation}\label{cat2}
F^P_{i+1/2}  = \sum_{k=1}^{m} \frac{\Delta t^{k-1}}{k!}\mathcal{A}^{0, 1/2}_{P}(f_{i,\bcdot}^{(k-1)}, \Delta x),
\end{equation}
where 
\begin{equation}\label{f(k-1)_ij}
f^{(k-1)}_{i,j}  \approx \partial_t^{k-1}f(u)(x_{i+j}, t^n), \quad j=-P+1,\dots, P
\end{equation}
are \textit{local} approximations of the time derivatives of the flux. 
By \textit{local} we mean that these approximations depend on the stencil, i.e.
\raph{for two different stencils such that 
$i_1 + j_1 = i_2 + j_2$ then $f^{(k-1)}_{i_1,j_1}$ is not necessarily equal to $f^{(k-1)}_{i_2,j_2}$.}
%$$ i_1 + j_1 = i_2 + j_2  \not \Rightarrow f^{(k-1)}_{i_1,j_1} = f^{(k-1)}_{i_2,j_2}. $$



% The $2P$-CAT method uses a Taylor expansion to update the numerical solution in the 
% \bea \label{eq:general_LW}
%  u_i^{n+1} = u_i^n + \sum_{k=1}^{2P} \frac{(\Delta t)^k}{k!} u_i^{(k)},
% \eea
% where $u_i^n$ is the approximation of $u(x_i,t_n)$ and $u_i^{(k)}$ is an approximation of $\partial_t^k u(x_i,t_n)$. 

% CAT methods,  designed by Carrillo and Pares \cite{Carrillo-Pares}, are a variant of the  Approximate Taylor method, proposed by Zorio et al. \cite{ZBM2017}, that properly generalize  the Lax-Wendroff methods for linear systems. These methods are based on the conservative expression but the difference is that now the numerical flux $F_{i\pm 1/2}$ are computed using only the values of the stencils  $\mathcal{S}_{i\pm\ha}^P$ of size $2P$

% With this requirement, $u_i^{n+1}$ is locally updated using only the values at the centered $(2P + 1)$-point stencil. 

% The flux functions $F_{i\pm \ha}^P$ are then computed, respectively, on the sets $\mathcal{S}_{i\pm\ha}^P$
Since the exact solution satisfies
$$
\partial_t^k u = - \partial_t^{k-1} f(u),
$$
local approximations of the time derivatives of the solution are obtained \raph{by} 
%as follows
:
\begin{equation*}
u^{(k)}_{i,j} = - \mathcal{A}^{1,j}_{P}(f^{(k-1)}_{i, \bcdot}, \Delta x)
= - \frac{1}{\Delta x} \sum_{r=-P +1}^P \gamma^{1,j}_{P,r} f^{(k-1)}_{i, r}.
\end{equation*}
These approximations of the time derivatives are then adopted to compute \gio{predicted values of the flux at several time levels,  via recursive use of Taylor expansions, that will be then used to numerically approximate time derivatives of the flux at time level $n$.}
The algorithm to compute $F^p_{i+1/2}$ for the cell $i$ is then:
%as follows:
\begin{enumerate}
\item  {Define}
$$
f^{(0)}_{i,j}=f(u^n_{i+j}), \quad  j = -P+1, \dots, P.
$$
\item  {For $k = 2 \dots m$:}
\begin{enumerate}
\item Compute
\begin{equation*}
 u^{(k-1)}_{i,j} = - \mathcal{A}^{1,j}_{P}(f^{(k-2)}_{i,\bcdot}, \Delta x). 
\end{equation*}
\item Compute 
$$
f^{k-1,n+r}_{i,j} = f \left(  u^n_{i+j} + \sum_{l=1}^{k-1} \frac{(r \Delta t)^l}{l!} u^{(l)}_{i,j} \right), \quad  j, r = -P+1, \dots, P.
$$
\item Compute
$$
f^{(k-1)}_{i,j} =   \mathcal{A}^{k-1,0}_{P}( f^{k-1, \bcdot}_{i,j}, \Delta t),\quad  j = -P+1, \dots, P.
$$
\end{enumerate}
\item Compute $F^P_{i+1/2}$ by 
\begin{equation}
    \label{FP}
    F_{i+\ha}^P = \sum_{k=1}^{2P}\frac{\Delta t^{k-1}}{k!}f^{(k-1)}_{i+\ha},
\end{equation} where
$$ 
f^{(k-1)}_{i+\ha} = \mathcal{A}_{P}^{0,\ha}\left(f^{(k-1)}_{i,*},\Delta x\right), 
\quad 
\text{with}
\quad
\mathcal{A}_{P}^{0,\ha}\left(f^{(k-1)}_{i,*},\Delta x\right)  = \sum_{p=-P+1}^P\gamma_{P,p}^{0,\ha} \, f_{i+p}^{(k-1)}.
$$ 
\end{enumerate}


% $\mathcal{A}_{P}^{0,\ha}$ is an interpolation formulas of order $2P-1$ based on $2P$-point stencil.
% \end{enumerate}
% Once the numerical fluxes have been computed, the numerical solution is updated by using 
% \begin{equation}
%     \label{CAT2P}
%     u_i^{n+1} = u_i^n + \frac{\Delta t}{\Delta x}\left(F_{i-\ha}^P - F_{i+\ha}^P\right).
% \end{equation} 
% For the sake of clarity, details of the second order ($P=1$) CAT2 and fourth-order ($P=2$) CAT4 methods can be founded in



% \begin{rem}
%     Observe that the computation of the numerical flux  $F_{i+1/2}^{P}$ requires the approximation of $u$ at the nodes of a space-time grid of size $2P \times 2P$:, represented by $u_{i,j}^{k,n+r}$, for $-P+1 \leq j,r =\leq P$. 
%     The approximations of the solution $u$ at times $(n-P+1)\Delta t$, \dots, $(n-1)\Delta t$ are different from the ones  already computed in the previous steps $t^{n-P},\ldots, t^{n-1}$: $u_{i+j}^{n-P}$, \ldots, $u_{i+j}^{n-1}$.  
%     In other words, the discretization in time is not based on a multi-step method but on a one-step one. In fact, it can be re-interpreted as a Runge-Kutta method whose stages are $\tilde u^{n+r}_{i,j}$, $r = -P+1, \dots, P$, see \cite{Carrillo-Pares,CPZMR2020,Macca-Pares}.
% \end{rem}
% \begin{rem}
%     These approximations are local. Indeed, suppose that $i_1+j_1 = i_2+j_2 = \ell,$ i.e. $x_{\ell}>0$ belongs to $\mathcal{S}^P_{i_1+1/2}$ and $\mathcal{S}^P_{i_2+1/2}$ with local coordinates $j_1$ and $j_2$ respectively. Then, $f^{(k)}_{i_1,j_1}$ and $f^{(k)}_{i_2,j_2}$ are, in general, two different approximations of $\partial_t^k f(u)(x_{\ell},t_n).$
% \end{rem}



%=============================================================
%
%  ADD MOOD ONTO CAT --- HERE IS THE NEW STUFF
%
\section{CATMOOD} \label{sec:CATMOOD}
% The main objective of this work is to combine the \aposteriori shock capturing technique (MOOD) \cite{CDL1} to the family of one-step spatial and temporal reconstructions of high order CAT schemes.  

% The MOOD algorithm obtains an \aposteriori solution of the high-order numerical method using a family of criteria that detect a variety of, even minimal, oscillations. In this family it is therefore possible to consider some criteria such as positivity, monotonicity and physical properties that the numerical solution, obtained by the scheme, have to satisfy even in complex cases. 
% In addition, this procedure allows to reduce and eliminate the numerical oscillations introduced by the CAT methods in the presence of shocks or large .

The essential idea of the MOOD technique is to apply a high-order method over the entire domain for a time step, then check locally, for each cell $i$, the behavior of the solution using some admissibility criteria such as positivity, monotonicity, physical \add{admissibility}, etc. If the solution computed in cell $i$ at time $t^{n+1}$ is in accordance with the selected criteria, it is kept. Otherwise it is \raph{locally} recomputed with a  lower order numerical method. This operation is repeated until acceptability, or when a robust first order scheme is 
%used. 
\raph{employed.}

Bearing this in mind, the idea is to design a cascade of CAT methods in which the order is locally adjusted according to some \aposteriori admissibility criteria thus creating a new family of adaptive CAT methods called CAT-MOOD schemes. 


\subsection{MOOD admissibility criteria} \label{ssec:MOOD}
Following \cite{CDL2,CDL3}, we select three different admissibility criteria 
 which are used to check the admissibility  of a candidate numerical solution $\left\{ u_i^{n+1} \right\}_{1\leq i \leq N}$:
\begin{enumerate}
    \item \textit{Physical Admissible Detector (PAD)}: The first detector checks the physical validity of the candidate solution. In particular, this  detector reacts to negative solution when a variable cannot take negative values: this is the case of the pressure $p$ and density $\rho$ for the 1D Euler system 
    in compliance with (\ref{eq:physical_states}). 
    % For the Euler test the physicality here is assessed from the point of view of a fluid flow, which limits the generality of the criteria as far as predicted pressures are concerned. 
    % Said differently, this physical admissibility criteria must be adapted to the model of PDEs which is solved.
    \item \textit{Numerical Admissible Detector (NAD)}: This criterion is used to ensure the essentially non-oscillatory (ENO) character of the numerical solution, namely that no large and spurious minima or maxima are introduced locally in the solution. To do this, the following relaxed variant of the Discrete Maximum Principle (see \cite{Ciarlet,CDL1}) is \raph{considered}:
    $$\min_{j\in \mathcal{C}_i^P}(u_{j}^n) - \delta_i^n \le u_{i}^{n+1} \le 
    \max_{j \in \mathcal{C}_i^P}(u^n_{j}) + \delta_i^n, $$
    where {$\mathcal{C}_i^P = \{i-P,\ldots,i+P\}$} 
     is the $(2P +1)$-point centered stencil and $\delta_i$ is a parameter that avoids wrong detections in flat region. Here,  $\delta_i$ is set as:
    \begin{equation}
        \delta_i^n = \max\left(\gio{{\rm tol}_1},\gio{{\rm tol}_2},\max_{j\in \mathcal{C}_i^P}u^n_j - \min_{j\in \mathcal{C}_i^P}u^n_j\right). 
        \label{eq:delta}
    \end{equation}
 In the Euler test of Sec.~\ref{sec:numerics} the relaxed discrete maximum principle is only computed for density $\rho$ and pressure $p$. 
    \item \textit{Computer Admissible Detector (CAD)}: The last criterion detects 
    undefined or unrepresentable quantities, usually
    not-a-number \texttt{NaN} or infinity quantity \raph{which may appear}, for instance, when a division by zero \gio{is encountered}.
\end{enumerate}  
The order in which these three criteria are applied to the candidate solution is showed in Figure~\ref{fig:MOOD_chain_cascade}-left. If one of the criteria is not satisfied, the cell is marked \raph{as 'failed'}.
A new candidate solution is then computed in marked cells using a lower order method and it is checked again.
% In case of spurious oscillations have affected a candidate solution, $w_i^*$, then it will detect by the NAD detector and it will not pass at least one of the previous criteria ordered into a chain, see figure~\ref{fig:MOOD_chain_cascade}-left.
% As a consequence, the MOOD loop drives the code to locally downgrade the order of accuracy by using an auxiliary scheme of lower accuracy.
% ---- FIG ----
\begin{figure}[!ht]
    \centering    
     \begin{subfigure}[b]{0.76\textwidth}
         \centering
    \includegraphics[width=0.99\textwidth]{FIG/chain.pdf}
    \caption{Criteria}
    \end{subfigure}
     \begin{subfigure}[b]{0.16\textwidth}
         \centering
    \includegraphics[width=0.99\textwidth,angle = 270]{FIG/cascade.pdf}
    \caption{Cascade}
    \end{subfigure}
    \caption{Left: Detection criteria of the MOOD technique for a candidate solution $u_i^*$. \textit{Computer Admissible Detector (CAD)}, \textit{Physical Admissible Detector (PAD)}  and \textit{Numerical Admissible Detector (PAD)} ---
    Right: Order cascades of CAT schemes used in the MOOD procedure. Starting from the most accurate one, CAT6, downgrading to lower order schemes, and, at last to a $1$st order accurate scheme employed to ensure robustness.}
   \label{fig:MOOD_chain_cascade}
\end{figure}
% ---- FIG ----

\subsection{CAT scheme with MOOD limiting} \label{ssec:CATMOOD} 
In this work, CAT methods are used as high-order methods within the MOOD strategy: the natural idea would be, \raph{given a target order $2P$}, to use the \raph{following} cascade of numerical methods 
$$\text{CAT}2P \to \text{CAT}2(P-1) \to \dots \text{CAT}2 \to \text{First-order method}$$ to  obtain a method with  order of accuracy $2P$ in smooth regions and an essentially non-oscillatory solution close to discontinuities or large gradients. 
%Nevertheless, 
\raph{However,}
in order to reduce the computational cost, the following cascade has been 
%used 
\raph{preferred}
in the numerical tests below
$$\text{CAT}6 \to \text{CAT}2  \to \text{First-order method}$$ 
where the first-order method is Rusanov for scalar problems and HLL for Euler equations: see Figure~\ref{fig:MOOD_chain_cascade}-right \raph{for an illustration}. Therefore, the expected order of accuracy in smooth regions is 6. The resulting scheme is referred to as CATMOOD6. 
% \ema{To obtain a very accurate numerical solution, the best approach is to use a scheme that is of high order in the smooth region and that goes gradually reducing the order of accuracy when, the considered region, is not sufficiently smooth. In this spirit, the purpose of this work is to create a numerical solution that is accurate up to the $6$th order in the smooth domain and that is reducing the order to 2 to 2 as long as a first order method is used. In this regard, the idea is to get a solution that is at most $6$th order of accuracy in the smooth region while it is $4$th order when CAT6 fails or $2$th order when CAT4 fails. At least a $1$th order scheme is adopted when all CAT2-4-6 fail. In numerical experiments, to obtain a scheme that has a lower computational cost without affecting the accuracy of the numerical solution, we preferred to use three numerical methods instead of four. In particular, as intermediate schemes between CAT6 and a first order one, only CAT2 has been adopted. In order to achieve the most accurate solution in the smooth region and the non-oscillatory solution in the region in which large gradient or discontinuities occur, the idea is adopt CAT6 as often as possible.\footnote{\ema{In the case where the numerical solution is regular up to order 6, the error will still be $O(\Delta x^6)$; while, in the case where a discontinuity or a large gradient is present in the stencil, the error will be of the order $O(\Delta x)$ or $O(\Delta x^2)$ which is greater than $O(\Delta x^4)$.}}} On the contrary, for cells that present a discontinuous solution we plan to rely on a $1$st order low accurate but robust scheme, for instance using Rusanov or HLL fluxes. Clearly, we would like to employ only when and where necessary. In our tests, between these two extremes the CAT2 schemes of $2$th order of accuracy is inserted.
% As such we build several cascades of schemes of decreasing orders but possibly increasing robustness, see Figure~\ref{fig:MOOD_chain_cascade}-right.
% This scheme is referred to as CATMOOD6.

  
% \subsection{Algorithm} \label{ssec:algo}
% Practically, for the time-step $[t^n,t^{n+1}]$ and for each cell $i$, we define a 'mask', $A_i^{n}\in\{-1,0,1\}$, such that
% \bea \label{eq:mask}
% A_i^n = \begin{cases}
%            1 \quad & \text{if} \; u_i^* \; \text{fails at least one criterion},  \\ 
%            -1\quad & \text{if} \; \exists  j \in \mathcal{N}_i \; \text{s.t} \; A_j^n=1,\\ 
%            0 \quad & \text{otherwise}.
%          \end{cases}
% \eea
% where $\mathcal{N}_i$ is the set of direct neighbor cells of cell $\omega_i$. $A_i^n=-1$ means that cell $i$ is the neighbor of an invalid cell. \\
% The algorithm designed for CATMOOD6 scheme is:
% \begin{enumerate}
%     \item[Init] Let be $\{u_i^n\}_{1\leq c\leq N_c}$ the numerical solution at $t = t^n$ over all the domain $\Omega.$
%     \item[CAT6] Let be $\{u_i^*\}_{1\leq c\leq N_c}$ the numerical solution at $t = t^{n+1}$ of order 6 obtained from CAT6 scheme. Set $A_i^n=0$.\\
%     For all cell $i$, check if $u_i^*$ satisfies all detection criteria. 
%     In this case, then $u_i^{n+1} = u_i^*$ and $A_i^n=0$.
%     Otherwise $A_i^n=1$ and, if $A_j^n=0$ then set $A_j^n=-1$ for all $j \in \mathcal{N}_i$.
%     \item[CAT2P] Only for the troubled cell $i$, i.e $A_i^n\not= 0$, recompute $\{u_i^*\}$ the numerical solution at time $t = t^{n+1}$ of order 2 obtained with CAT2 scheme. \\
%    Check if $u_i^*$ satisfies all detection criteria.   
%     In this case, then $u_i^{n+1} = u_i^*$ and $A_i^n=0$.
%     Otherwise $A_i^n=1$ and, if $A_j^n=0$ then set $A_j^n=-1$ for all $j \in \mathcal{N}_i$.
%     \item[1st-ord] Only for the remaining troubled cell $i$, i.e $A_i^n\not= 0$, recompute $\{u_i^*\}$ the numerical solution at time $t = t^{n+1}$ of order 1 obtained with a first order scheme, and set $u_i^{n+1} = u_i^*$.
% \end{enumerate}
% Notice that if a CAT4 scheme is used in the cascade in-between CAT6 and CAT2, then one adds another CAT2P step in the algorithm.


%============================================================
%    2 D   N U M E R I C S
%
\section{Numerical test cases}  \label{sec:numerics}
% In this paper only numerical tests related to the one-dimensional equations are considered. Our methodology of testing relies on classical test cases: 
% \begin{enumerate}
%     \item \textit{Scalar linear equation with smooth initial condition}. This test measures the ability of the MOOD procedure combined with CAT schemes to achieve the optimal high order for a smooth solution. We also compared CATMOOD6 with Rusanov first order scheme.
%     \item \textit{Burgers equation with non-smooth initial condition}. This test measures the ability of the MOOD technique to eliminate oscillations near discontinuities introduced by CAT methods \cite{CDL1,CPZMR2020,Carrillo-Pares,TesiPhD}.
%     \item \textit{Sod problem}. This test presents smooth regions, a rarefaction wave, a contact discontinuity and shock. The MOOD technique deletes the oscillations close to the discontinuity \cite{Toro2009}. 
% \end{enumerate}

\gio{Three tests are considered, namely linear advection equation, to check the numerical order of accuracy,  Burgers equation, and Euler equations of gas dynamics, to check shock-capturing capability of the method. 
In all our tests we used the following parameters: 
${\rm tol}_1 = 10^{-4}$ and 
${\rm tol}_2 = 10^{-3}$.}

\subsection{Scalar linear equation with smooth initial condition} \label{ssec:linear}
Let us consider the linear conservation law \eqref{sec:CAT_gov_equ} with \gio{$f(u) = u$, and periodic boundary conditions in $[0,2\pi]$}. The
smooth initial condition is given by
\begin{equation}\label{smooth_1}
u(x,0) = u_0(x) = \ha\sin(x) + 1.
\end{equation} 
The goal of this test is to check and compare the empirical order of accuracy of CATMOOD6. %and 
For smooth solutions, the detectors are expected not to spoil the sixth-order of accuracy of the CAT6 \raph{scheme}. We run this test case in the interval $[0,2\pi],$ with final time $t_{fin} = 1$, CFL$=0.9,$ and periodic boundary conditions. We apply CAT6 and CATMOOD6 method on successive refined uniform meshes going from $10$   to  $320$ cells. As expected the empirical order of accuracy of both CAT and CATMOOD is 6: see Table \ref{tab:linear_convergence}. 
\raph{We observe that below $N=80$ cells, the mesh is not fine enough to allow for a clean limiting.
\gio{Notice that this threshold depends on the parameteters ${\rm tol}_1$ and ${\rm tol}_2$ adopted in \eqref{eq:delta}. Larger values of these parameter will lower the threshold to smaller values of $N$.} 
Moreover, in this test, CATMOOD6 is $1.5$ more expensive than CAT6.}
% What we expect.
% Since the solution is smooth, it should be simulated with optimal high accuracy, in other words, the limiting/stabilization procedure employed in the scheme should not have any effect. 
% % --- TAB ---
\begin{table}[!ht]
%\numerikNine
    \centering
    \begin{tabular}{|c||ccc|ccc|}
    \hline \multicolumn{7}{|c|}{\textbf{Linear equation - Error, Rate of convergence, CPU time}} \\
\hline   
\hline   
& \multicolumn{3}{c|}{\textbf{CAT6}} &\multicolumn{3}{c|}{\textbf{CATMOOD6}}\\
 $N$ &  $L^1$ error &  order & CPU time & $L^1$ error &  order & CPU time \\ \hline
10 & 4.27$\times 10^{-5}$  &  ---  & 0.0073    & 8.34$\times 10^{-3}$ & ---  & 0.016    \\
20 & 6.93$\times 10^{-7}$    &  5.95  & 0.020    & 3.18$\times 10^{-3}$ & 1.39  & 0.031    \\
40 & 9.64$\times 10^{-9}$    &  6.17  & 0.038     & 3.16$\times 10^{-4}$ & 3.33  & 0.053    \\
80 & 1.43$\times 10^{-10}$   &  6.07  & 0.068    &  2.48$\times 10^{-6}$ &  3.67  & 0.094    \\
160 & 2.09$\times 10^{-12}$  &  6.09  & 0.13    & 2.09$\times 10^{-12}$ & 23.50  & 0.2    \\
320 & 3.29$\times 10^{-14}$  &  5.99  & 0.25    & 3.31$\times 10^{-14}$ & 5.99   & 0.38    \\
& \text{Expected} & 6 &
& \text{Expected} & 6 &
\\
    \hline 
    \end{tabular}
    \caption{Linear scalar equation \ref{ssec:linear}.  $L^1-$norm errors between the numerical solution and the exact solution of the linear equation at $t_{\text{final}} = 1$ on uniform Cartesian mesh and CFL$=0.9.$
    %\gio{The change between 80 and 160 points is dramatic! Furthremore it is evident that MOOD does not do anything with 160 points, while it dramatically affects the accuracy with just 80 points. We should at least comment about this.}
   }
    \label{tab:linear_convergence}
\end{table}


\vspace{-0.6cm}

\subsection{Burgers' equation with non-smooth initial condition} \label{ssec:burgers}
Let us consider \eqref{sec:CAT_gov_equ} with \gio{$f(u) = u^2/2$, periodic boundary conditions,} and non-smooth initial condition
\begin{equation} \label{square_step_test}
u_0(x) = \begin{cases}
1.1 \quad \;\;\;\mathrm{if}\quad 0\le x\le \ha;\\
2.1 \quad\;\; \;\mathrm{if}\quad \ha< x < \frac{3}{2};\\
0.1 \quad \;\;\; \mathrm{if} \quad \frac{3}{2}\le x\le\frac{17}{10}. 
\end{cases}
\end{equation} 
We run this test case with first-order Rusanov-flux and  CATMOOD6 method in the interval $[0,1.7],$
with final time $t_{fin} = 0.65$, 
using a 50-cell mesh, CFL$=0.9,$ and periodic boundary conditions.
Figure~\ref{fig:burgers} shows the initial condition and the numerical solutions obtained with both methods: it can be seen that CATMOOD6 provides a non-oscillatory solution \gio{still providing better resolution than the first order method. Notice that larger values of the thresholds tol$_1$ and tol$_2$ will decrease dissipation, but may not be sufficient to avoid creation of spurious oscillations.}

\begin{figure}[!ht]
\hspace{-1.cm}
    \centering    
    \begin{subfigure}[b]{0.48\textwidth}
         \centering
    \includegraphics[width=1.12\textwidth]{FIG/Burgers/u_no_smooth_mid_1.pdf}
    \vspace{-0.5cm}
    \caption{Numerical solutions at time $t=0.3$.}
    \end{subfigure}
     \begin{subfigure}[b]{0.48\textwidth}
         \centering
    \includegraphics[width=1.12\textwidth]{FIG/Burgers/u_no_smooth_fin_1.pdf}
    \vspace{-0.5cm}
    \caption{Numerical solutions at time $t=0.65$.}
    \end{subfigure}
    \caption{Burgers' equation with non-smooth initial condition \ref{ssec:burgers} --- Left: Numerical solutions at time $t = 0.3$ --- Right: Numerical solutions at final time $t_{fin} = 0.65$ obtained with Rusanov flux and CATMOOD6 on $50$ uniform mesh and CFL$=0.9$. Rusanov flux scheme has been used as first order. The reference solution has been obtained with the 1st order Rusanov scheme on $2000$ uniform cells. }
   \label{fig:burgers}
\end{figure}
%\raph{Question: maybe using a finer mesh say 500 cells to put on top of figure 2 to see the convergence could be good?}
\vspace{-0.6cm}



\subsection{Euler system: Sod problem} \label{ssec:Sod}
Let us consider Euler equations \eqref{sec:CAT_gov_equ} with SOD initial condition 

\begin{equation} \label{Sod_IC}
	(\rho, \raph{u}, p)
	= \left\{
	\begin{array}{ll}
	\displaystyle (1,0,1) & \mbox {if } x \le 0, \\
	\displaystyle (0.125,0,0.1) & \mbox {if } x > 0.
	\end{array}\right.
\end{equation}
%\vspace{-1cm}

\begin{figure}[!ht]
\hspace{-1.cm}
    \centering    
    \begin{subfigure}[b]{0.48\textwidth}
         \centering
    \includegraphics[width=1.12\textwidth]{FIG/Sod/density_1.pdf} 
    \vspace{-0.5cm}
    \caption{Density.}
    \end{subfigure}
     \begin{subfigure}[b]{0.48\textwidth}
         \centering
    \includegraphics[width=1.12\textwidth]{FIG/Sod/density_zoom_1.pdf}
    \vspace{-0.5cm}
    \caption{Zoom of the density on shock}
    \end{subfigure} 
    \\
    \hspace{-1.cm}
    \centering    
    \begin{subfigure}[b]{0.48\textwidth}
         \centering
    \includegraphics[width=1.12\textwidth]{FIG/Sod/vel.pdf} 
    \vspace{-0.5cm}
    \caption{Velocity.}
    \end{subfigure}
     \begin{subfigure}[b]{0.48\textwidth}
         \centering
    \includegraphics[width=1.12\textwidth]{FIG/Sod/press.pdf}
    \vspace{-0.5cm}
    \caption{Pressure}
    \end{subfigure}
    \caption{Euler system: Sod problem \ref{ssec:Sod} --- Left-up: Numerical solutions for density $\rho$ at final time $t=0.3$ obtained with \raph{1st order HLL} 
    %Rusanov 
    flux and CATMOOD6, on $200$ uniform mesh and CFL$=0.9$. --- Right-up: Zoom of the numerical solutions and reference one. --- 
    Left/right-down: same as left-up but for velocity and pressure variables. The reference solution is the exact one.}
   \label{fig:Sod}
\end{figure}
%\vspace{-0.5cm}
We run this test case with first-order Rusanov-flux and CATMOOD6 method in the interval $[-1,1],$ with final time $t_{fin} = 0.3$. We use  a $200-$cell mesh, CFL$=0.9,$ and free boundary conditions.
Figure~\ref{fig:Sod} shows the numerical and  the exact solutions for density, velocity and pressure obtained with both methods and a zoom of the density close to the shock. It can be seen that, as expected, CATMOOD6 solution shows a better resolution of rarefaction and contact waves. 


\section{Conclusion and perspectives} \label{sec:conclusion}
% Conclusions
In this paper we have presented a combination between the \aposteriori shock-capturing MOOD technique  and the one-step high-order finite-difference CAT2P schemes. CAT2P schemes are of order $2P$ on smooth solutions, but, \raph{an} extra dissipative mechanism must be supplemented to deal with steep gradients or discontinuous solutions.
In this work we rely on an \aposteriori MOOD paradigm which computes an unlimited high-order candidate solution at time $t^{n+1}$, and, further detects troubled cells which are recomputed with a lower-order scheme throughout a family of detectors.
For a proof of concept, we tested the so-called \raph{CATMOOD6 scheme based on the } 'cascade': 
%of schemes: 
CAT6$\rightarrow$CAT2$\rightarrow \,$ 1st, where the last scheme is a first order robust scheme.
%The resulting MOOD scheme is called CATMOOD6. \\
%We have tested 
\raph{This scheme has been challenged}
%this scheme 
on a test suite of smooth solutions (linear scalar equation), simple shock waves (Burgers' equation), and complex self-similar solutions involving contact, shock and rarefaction waves (Sod problem).
In all test cases, CATMOOD6 has preserved the accuracy on smooth parts of the solutions, an essentially-non-oscillatory behavior close to steep gradients, and, \raph{always produces} a physically valid solution.
\gio{A detailed analysis of the improvement in the efficiency over standard slope limiters has not been performed. 
A two dimensional implementation, as well as other generalizations and  improvements are  under way.}

\vspace{-0.75cm}
\bibliographystyle{plain}
\bibliography{biblio}

\end{document}
