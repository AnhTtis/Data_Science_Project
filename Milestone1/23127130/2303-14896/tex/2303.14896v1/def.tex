% Set left margin - The default is 1 inch, so the following
% command sets a 1.25-inch left margin.
%\setlength{\oddsidemargin}{0.25in}
%\oddsidemargin .15cm
%\evensidemargin .15cm
\oddsidemargin .0cm
\evensidemargin .0cm

% Set width of the text - What is left will be the right margin.
% In this case, right margin is 8.5in - 1.25in - 6in = 1.25in.
\setlength{\textwidth}{6.5in}
%\textwidth 16.5cm

% Set top margin - The default is 1 inch, so the following
% command sets a 0.75-inch top margin.
%\setlength{\topmargin}{-0.25in}
\topmargin .5cm

% Set height of the header
\setlength{\headheight}{.5cm}

% Set vertical distance between the header and the text
\setlength{\headsep}{0.5cm}

% Set height of the text
\setlength{\textheight}{22.5cm}

% Set vertical distance between the text and the
% bottom of footer
\setlength{\footskip}{0.4in}

\usepackage{amsmath}
\usepackage{amsfonts}
\usepackage{latexsym}
\usepackage{amssymb}

\newtheorem{thm}{Theorem}[section]
%\newtheorem{theorem}{Theorem}[section]
\newtheorem{theorem}[thm]{Theorem}
\newtheorem{lem}[thm]{Lemma}
\newtheorem{lemma}[thm]{Lemma}
\newtheorem{prop}[thm]{Proposition}
\newtheorem{proposition}[thm]{Proposition}
\newtheorem{cor}[thm]{Corollary}
\newtheorem{corollary}[thm]{Corollary}
\newtheorem{definition}[thm]{Definition}
\newtheorem{defn}[thm]{Definition}
\newtheorem{example}[thm]{Example}
\newtheorem{rem}[thm]{Remark}
\newtheorem{remark}[thm]{Remark}
\newtheorem{problem}[thm]{Problem}
\newtheorem{question}[thm]{Question}
\newtheorem{exercise}[thm]{Exercise}
\newtheorem{application}[thm]{Application}
\newtheorem{observation}[thm]{Observation}

\newcommand{\beq}{\begin{equation}}
\newcommand{\eeq}{\end{equation}}
\newcommand{\beqa}{\begin{eqnarray}}
\newcommand{\eeqa}{\end{eqnarray}}
\newcommand{\beqas}{\begin{eqnarray*}}
\newcommand{\eeqas}{\end{eqnarray*}}
\newcommand{\bi}{\begin{itemize}}
\newcommand{\ei}{\end{itemize}}
\newcommand{\gap}{\hspace*{2em}}
\newcommand{\sgap}{\hspace*{1em}}
\newcommand{\vgap}{\vspace{.1in}}
\newcommand{\nn}{\nonumber}
\newcommand{\bRe}{\bar \Re}
\newcommand{\IR}{\makebox{\sf I \hspace{-9.5 pt} R \hspace{-7.0pt} }}
\newcommand{\defeq}{\stackrel{\triangle}{=}}

\setcounter{page}{1}
\def\eqnok#1{(\ref{#1})}
\def\proof{\noindent{\bf Proof}: \ignorespaces}
\def\remark{\par{\it Remark}: \ignorespaces}
\def\endproof{{\ \hfill\hbox{%
      \vrule width1.0ex height1.0ex
    }\parfillskip 0pt}\par}
\def\endremark{{\ \hfill\hbox{%
      \vrule width1.0ex height1.0ex
    }\parfillskip 0pt}\par}

\newcommand{\tos}{\rightrightarrows}
\newcommand{\R}{\mathbb{R}}
\newcommand{\calS}{{\cal S}}
\newcommand{\cG}{{\cal G}}
\newcommand{\cL}{{\cal L}}
\newcommand{\cX}{{\cal X}}
\newcommand{\cC}{{\cal C}}
\newcommand{\cZ}{{\cal Z}}
\newcommand{\Z}{{\cal Z}}
\newcommand{\lam}{{\lambda}}
\newcommand{\blam}{{\bar \lambda}}
\newcommand{\A}{{\cal A}}
\newcommand{\tA}{{\tilde A}}
\newcommand{\tC}{{\tilde C}}
\newcommand{\cN}{{\cal N}}
\newcommand{\cK}{{\cal K}}
\newcommand{\norm}[1]{\left\Vert#1\right\Vert}
\newcommand{\ind}[1]{I_{#1}}
\newcommand{\inds}[1]{I^*_{#1}}
\newcommand{\supp}[1]{\sigma_{#1}}
\newcommand{\supps}[1]{\sigma^*_{#1}}
\newcommand{\simplex}[1]{\Delta_{#1}}
\newcommand{\abs}[1]{\left\vert#1\right\vert}
\newcommand{\Si}[1]{{\cal S}^{#1}}
\newcommand{\Sip}[1]{{\cal S}^{#1}_{+}}
\newcommand{\Sipp}[1]{{\cal S}^{#1}_{++}}
\newcommand{\inner}[2]{\langle #1,#2\rangle}
\newcommand{\Inner}[2]{\big \langle #1\,,#2 \big \rangle}

\newcommand{\ttheta}{{\tilde \theta}}
\newcommand{\cl}{\mathrm{cl}\,}
\newcommand{\co}{\mathrm{co}\,}
\newcommand{\argmin}{\mathrm{argmin}\,}
\newcommand{\cco}{{\overline{\co}}\,}
\newcommand{\ri}{\mathrm{ri}\,}
\newcommand{\inte}{\mathrm{int}\,}
\newcommand{\interior}{\mathrm{int}\,}
\newcommand{\bd}{\mathrm{bd}\,}
\newcommand{\rbd}{\mathrm{rbd}\,}
\newcommand{\cone}{\mathrm{cone}\,}
\newcommand{\aff}{\mathrm{aff}\,}
\newcommand{\lin}{\mathrm{lin}\,}
\newcommand{\lineal}{\mathrm{lineal}\,}
\newcommand{\epi}{\mathrm{epi}\,}
\newcommand{\sepi}{\mbox{\rm epi}_s\,}
\newcommand{\dom}{\mathrm{dom}\,}
\newcommand{\Dom}{\mathrm{Dom}\,}
\newcommand{\Argmin}{\mathrm{Argmin}\,}
\newcommand{\lsc}{\mathrm{lsc}\,}
\newcommand{\di}{\mbox{\rm dim}\,}
\newcommand{\EConv}[1]{\mbox{\rm E-Conv}(\Re^{#1})}
\newcommand{\EConc}[1]{\mbox{\rm E-Conc}(\Re^{#1})}
\newcommand{\Conv}[1]{\mbox{\rm Conv}(\R^{#1})}
\newcommand{\Conc}[1]{\mbox{\rm Conc}(\Re^{#1})}
\newcommand{\cball}[2]{\mbox{$\bar {\it B}$}(#1;#2)}
\newcommand{\oball}[2]{\mbox{\it B}(#1;#2)}
%\newcommand{\Subl}[2]{{\cal L}_{\le #1}(#2)}
%\newcommand{\sSubl}[2]{{\cal L}_{< #1}(#2)}
\newcommand{\Subl}[2]{{#2}^{-1}[-\infty,#1]}
\newcommand{\sSubl}[2]{{#2}^{-1}[-\infty,#1)}
\newcommand{\bConv}[1]{\overline{\mbox{\rm Conv}}\,(\R^{#1})}
\newcommand{\bEConv}[1]{\mbox{\rm E-C}\overline{\mbox{\rm onv}}\,(\Re^{#1})}
\newcommand{\asympt}[1]{{#1}'_{\infty}}
\newcommand{\sym}[1]{{\cal S}^{#1}}
\newcommand{\spa}{\,:\,}
\newcommand{\barco}{\overline{\mbox{co}}\,}
\newcommand{\tx}{\tilde x}
\newcommand{\ty}{\tilde y}
\newcommand{\tz}{\tilde z}
\newcommand{\tm}{\tilde m}
\newcommand{\mConv}[1]{\overline{\mbox{\rm Conv}}_\mu\,(\R^{#1})}
\newcommand{\la}{\langle\,}
\newcommand{\ra}{\,\rangle}
\def\iff{\Leftrightarrow}
\def\solution{\noindent{\bf Solution}. \ignorespaces}
\def\endsolution{{\ \hfill\hbox{%
      \vrule width1.0ex height1.0ex
          }\parfillskip 0pt}\par}
\def\NN{{\mathbb{N}}}
