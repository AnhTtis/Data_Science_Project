\documentclass{article}
\usepackage[utf8]{inputenc}
\usepackage{amsmath}
\usepackage{amsfonts}
\usepackage{color}
\usepackage{latexsym}
\usepackage{tablefootnote}
\usepackage{algorithm}
\usepackage{algorithmic}
\newcommand{\red}[1]{\textcolor{red}{#1}}
\newcommand*{\E}{\mathbb{E}}
\newcommand*{\QEDA}{\hfill\hbox{\vrule width1.0ex height1.0ex}}

\section{Approximate Leximin Optimality}\label{sec:approx-leximin-def}
In this section, we present our definition of leximin approximation in the presence of multiplicative and additive errors, in the context of multi-objective optimization problems.
% Section \ref{} discusses other potential definitions that might be considered intuitive and the reasons we made a different choice.


% \yonatan{I think the following discussion is too detailed for the introduction - where the actual definition is not provided.}
\subsection{Motivation: Unsatisfactory  Definitions}

Which solutions should be considered approximately-optimal in terms of  leximin? 
Several definitions appear intuitive at first glance.
As an example, suppose we are interested in approximations with an allowable multiplicative error of $0.1$.
Denote the utilities in the leximin-optimal solution by $(u_1,\ldots,u_n)$.
A first potential definition is that any solution in which the sorted utility vector is at least $(0.9\cdot u_1,\ldots,0.9\cdot u_n)$ should be considered approximately-optimal.
For example, if the utilities in the optimal solution are $(1,2,3)$, then a solution with utilities $(0.9, 1.8, 2.7)$ is approximately-optimal.
However, allowing the smallest utility to take the value $0.9$ may substantially increase the maximum possible value of the second (and third) smallest utility --- e.g.~a solution that yields utilities $(0.9, 1000,1000)$ might exist. In that case, a solution with utilities  $(0.9, 1.8, 2.7)$ is very far from optimal.
We expect a good approximation notion to consider the fact that an error in one utility might change the optimal value of the others.

The following, second attempt at a definition, captures this requirement.
An approximately-optimal solution is one that yields utilities at least $(0.9\cdot m_1, 0.9 \cdot m_2, \dots, 0.9 \cdot m_n)$, where $m_1$ is the maximum value of the smallest utility, $m_2$ is the maximum value of the second-smallest utility \emph{among all solutions whose smallest utility is at least $0.9 \cdot m_1$};
$m_3$ is the maximum value of the third-smallest utility among all solutions whose smallest utility is at least $0.9 \cdot m_1$ and their second-smallest utility is at least $0.9\cdot m_2$; and so on. 
In the above example, to be considered approximately-optimal, the smallest utility should be at least $0.9$ and the second-smallest should be at least $900$.
Thus, a solution with utilities $(0.9, 1.8, 2.7)$ is not considered approximately-optimal. Unfortunately, according to this definition, even the leximin-optimal solution --- with utilities $(1,2,3)$  --- is not considered approximately-optimal.
We expect a good approximation notion to be a relaxation of leximin-optimality.
% \eden{is this paragraph relevant here? if so, to write about algorithms as well?\\
% \textit{Is there an algorithm that can be used when the single-objective solver can only approximate the optimal value?}}
% \erel{I like the examples. I am not sure where is the best place to mention the algorithms.}



\subsection{Our Definition}

\paragraph{The approximate leximin order} 
The first step in the defining  the following  strict \textit{partial} order on solutions.
%(a partial order allows two solutions with different utilities such that no one is preferred over the other).
Here, a maximal element is one over which no other solution is preferred; note that it is not equivalent to one that is preferred over all others (as in total order).

We focus on maximization problems. Given $\DEFmultApprox\in (0,1]$ and $\DEFadditiveApprox \geq 0$, 
%a value $v_1 \in \mathbb{R}$ is considered a $(\DEFmultApprox,\DEFadditiveApprox)$-approximation of a value $v_2 \in \mathbb{R}$ if $v_1 \geq \DEFmultApprox\cdot v_2 - \DEFadditiveApprox$.
value %Therefore, we say that 
$v_2$ is \emph{$(\DEFmultApprox,\DEFadditiveApprox)$-preferred} over $v_1$ if $v_2 > \frac{1}{\DEFmultApprox}(v_1 + \DEFadditiveApprox)$.

The approximate leximin order can now be described.%
\footnote{A proof that the approximate leximin order is a strict partial order can be found in appendix \ref{sec:approx-order-is-strict-partial}}
% It is determined according to the following preferences relation: we say that
A solution $y$ is \emph{$(\DEFmultApprox,\DEFadditiveApprox)$-leximin-preferred} over a solution $x$ if there exists an integer $k \in [n]$ such that the smallest $(k-1)$ objective values of $y$ are \emph{at least} those of $x$, and the $k$'th smallest objective value of $y$ is $(\DEFmultApprox,\DEFadditiveApprox)$-preferred over the $k$'th smallest objective value of $x$, that is:
\begin{align*}
    \forall j < k \colon \quad &\valBy{j}{y} \geq \valBy{j}{x}\\
    &\valBy{k}{y} > \frac{1}{\DEFmultApprox} \left( \valBy{k}{x} + \DEFadditiveApprox\right)
\end{align*}
This relation is denoted by $y \alphaBetaPreferred x$.
The corresponding relation set is defined as follows:
\begin{align*}
    \relationSetAlphaBeta = \{(y,x) \mid \forall x,y \in S \colon y \alphaBetaPreferred x\}
\end{align*}

Before describing the approximation definition, we present some observations of this new relation that will be useful later.
The proofs are straightforward and are omitted. 

\iffalse
Consider the case were $\DEFmultApprox = 1$ and $\DEFadditiveApprox=0$. 
The relations $\leximinPreferred$ and $\alphaBetaPreferredParams{1}{0}$ may look similar at first glance (as the requirement for $k$ is the same), but they are different.
In the first relation, the requirement for $j<k$ says that the values of $y$ are \emph{equal} to those of $x$, while in the second relation it says that they are \emph{at least} as good. 
In spite of this, Lemma \ref{lemma:approx-relation-prop1} proves that these two relation are equivalent.
\fi

\begin{lemma}\label{lemma:approx-relation-prop1}
    Let $x,y \in S$. Then $y \leximinPreferred x \iff y \alphaBetaPreferredParams{1}{0} x$.
\end{lemma}

\iffalse
\begin{proof}
    % The first direction $y \leximinPreferred x \Rightarrow y \alphaBetaPreferredParams{0}{0} x$ is almost trivial.
For the first direction,
    assume that $y \leximinPreferred x$. 
    By definition there exists an integer $k \in [n]$ such that $\valBy{j}{y} = \valBy{j}{x}$ for any $j < k$, and $\valBy{k}{y} > \valBy{k}{x}$.
    It is easy to verify that the same $k$ also implies that $y \alphaBetaPreferredParams{1}{0} x$. 
    
For the second direction,
    assume that $y \alphaBetaPreferredParams{1}{0} x$. 
    By definition there exists an integer $k \in [n]$ such that $\valBy{j}{y} \geq \valBy{j}{x}$ for any $j < k$, and $\valBy{k}{y} > \frac{1}{1} \left(\valBy{k}{x}+0\right) = \valBy{k}{x}$.
    Let $k'$ be the smallest integer for which $\valBy{k'}{y} > \valBy{k'}{x}$; 
    % (such a $k'$ must exist since it is true in particular for $k$) . 
    Note that $k'\leq k$.
    This means that $\valBy{j'}{y} = \valBy{j'}{x}$  for any $j' < k'$, and $\valBy{k'}{y} > \valBy{k'}{x}$, so $y \leximinPreferred x$.
\end{proof}
\fi

% \eden{is this ok?} 
% \erel{Yes}
Throughout the remainder of this section, we denote the difference between $\DEFmultApprox$ and $1$ by $\DEFmultError = 1-\DEFmultApprox$; in the context of approximations, $\DEFmultError$ can be viewed as the multiplicative \emph{error} factor.

Another important property of this relation arises from the following observation.

\begin{observation}\label{obs:approx-relation-prop2}
     If $0 \leq \DEFmultErrorOf{\DEFmultApprox_1} \leq  \DEFmultErrorOf{\DEFmultApprox_2} < 1$ and $0 \leq \DEFadditiveError_1 \leq \DEFadditiveError_2$. 
     Then:
     \begin{align*}
         y \alphaBetaPreferredParams{\DEFmultApprox_2}{\DEFadditiveApprox_2} x \Rightarrow y \alphaBetaPreferredParams{\DEFmultApprox_1}{\DEFadditiveApprox_1} x
     \end{align*}
\end{observation}
% \begin{proof}
%     Assume that $y \xPreferred{\gamma} x$.
%     By definition this means that there exists an integer $k\in [n]$ such that:
%     \begin{align*}
%         \forall j < k \colon \quad &\valBy{j}{y} \geq \valBy{j}{x}\\
%         &\valBy{k}{y} > \gamma \cdot \valBy{k}{x} 
%     \end{align*}
%     However, since $\gamma \geq \delta$, this $k$ also implies that $y \alphaBetaPreferred x$ as $\valBy{k}{y} > \gamma \cdot \valBy{k}{x} \geq \delta \cdot \valBy{k}{x}$, and for $j<k$ it is the same as for $\gamma$. 
% \end{proof}
One can easily verify that it follows directly from the definition as
% $\DEFmultApprox_1 \geq \DEFmultApprox_2$ and therefore,
$\frac{1}{\DEFmultApprox_2} \geq \frac{1}{\DEFmultApprox_1}$. 
% Lemma \ref{lemma:approx-relation-prop2} connects the different relations generated by different parameters, that is, the relations $\xPreferred{\gamma}$ and $\alphaBetaPreferred$, arising from $\gamma \geq \delta \geq 1$.
Accordingly, by considering the relation sets $\relationSetParams{\DEFmultApprox_1}{\DEFadditiveApprox_1}$ and $\relationSetParams{\DEFmultApprox_2}{\DEFadditiveApprox_2}$, we can conclude that $\relationSetParams{\DEFmultApprox_2}{\DEFadditiveApprox_2} \subseteq \relationSetParams{\DEFmultApprox_1}{\DEFadditiveApprox_1}$.
This means that as the \emph{error} parameters $\DEFmultError$ and $\DEFadditiveApprox$ increase,
the relation becomes \emph{more partial}:
when $\DEFmultError = 0$ and $\DEFadditiveApprox = 0$ it is a total order, any two elements that yield different utilities appear as a pair in $\relationSetParams{1}{0}$; but as they increase, the set $\relationSetAlphaBeta$ potentially becomes smaller, as fewer pairs are comparable.

To illustrate, consider the following example with three solutions $x,y,z$ and sorted utility vectors $u_x=(1,10,15), u_y =(1,40,60), u_z=(2,20,30)$.
It is easy to see that the maximum element according to the leximin order is $z$ and that $\relationSetParams{1}{0} = \{(z,x),(z,y),(y,x)\}$.
However, the relation set either stays the same or becomes smaller as $\DEFmultError$ increases (and the approximation factor $\DEFmultApprox$ decreases), for example $\relationSetParams{0.75}{0} = \{(z,x),(z,y),(y,x)\}$,
$\relationSetParams{0.5}{0} = \{(y,x)\}$, and $\relationSetParams{0.25}{0} = \emptyset$.
The same applies when $\DEFadditiveError$ increases, for example $\relationSetParams{1}{1} = \{(z,x),(z,y),(y,x)\}$, $\relationSetParams{1}{15} = \{(y,x)\}$ and $\relationSetParams{1}{45} = \emptyset$.
Similarly, when both $\DEFmultError$ and $\DEFadditiveError$ increase, for example $\relationSetParams{0.9}{0.5} = \{(z,x),(z,y),(y,x)\}$, $\relationSetParams{0.75}{1} = \{(y,x)\}$ and $\relationSetParams{0.75}{20} = \emptyset$.
% \eden{for us for checking my calculation:
% \begin{align*}
%     &\relationSetParams{0.25}{0} = \{(z,x),(z,y),(y,x)\} \Hquad \frac{1}{1-0.25} \approx 1.33\\
%     & \quad (z,x), (z,y) \text{ since } 2 > \frac{1}{1-0.25}1 \approx 1.33, \Hquad (y,x)  \text{ since } 40 > \frac{1}{1-0.25}10 \approx 13.33\\
%     &\relationSetParams{0.5}{0} = \{(y,x)\}  \Hquad \frac{1}{1-0.5} = 2\\
%     & \quad (y,x) \text{ since } 40 > \frac{1}{1-0.5}10\approx 1.33, \Hquad \text{NOT }(z,x), (z,y)  \text{ since the ratio is} <2\\
%      &\relationSetParams{0.75}{0} = \emptyset,  \Hquad \frac{1}{1-0.75} = 4, \Hquad \text{ the ratio is always} \leq 4\\ 
%      &\relationSetParams{0}{1} = \{(z,x),(z,y),(y,x)\}\\
%      &\relationSetParams{0}{15} = \{(y,x)\}\\
%      &\relationSetParams{0}{45} = \emptyset,\Hquad \text{ the difference is always} \leq 45\\ \\
%      &\relationSetParams{0.1}{0.5} = \{(z,x),(z,y),(y,x)\}\\
%      &\relationSetParams{0.25}{1} = \{(y,x)\}\\
%      &\relationSetParams{0.25}{20} = \emptyset
% \end{align*}
% }


The leximin approximation can now be defined.

\paragraph{Leximin approximation}
% \eden{should we say again that it is a generalization of Henzinger et al?}
We say that a solution $x\in S$ is an \emph{$(\DEFmultApprox,\DEFadditiveError)$-approximately-optimal} if it is a maximum element of the order $\alphaBetaPreferred$ in $S$ for $\DEFmultApprox\in (0,1]$ and $\DEFadditiveError \geq 0$.
That is, there is \emph{no} solution in $S$ that is $(\DEFmultApprox,\DEFadditiveError)$-leximin-preferred over it --- $y \nAlphaBetaPreferred x$ for any $y\in S$.


% \eden{==== stopped here}
This definition has some important properties.
Lemma \ref{lemma:absence-of-errors} proves that in the absence of errors ($\DEFmultError = \DEFadditiveError = 0$) it is equivalent to the exact leximin optimal definition. 
Then, Lemma \ref{lemma:beta1-beta2-approx} shows that a $(\DEFmultApprox_1,\DEFadditiveError_1)$-approximation is also a $(\DEFmultApprox_2,\DEFadditiveError_2)$-approximation when $0 \leq \DEFmultErrorOf{\DEFmultApprox_1} \leq  \DEFmultErrorOf{\DEFmultApprox_2} < 1$ and $0 \leq \DEFadditiveError_1 \leq \DEFadditiveError_2$.
Finally, Lemma \ref{lemma:exact-is-always-optimal} proves that a leximin optimal solution is always approximately-optimal (for any error factors).
% And third, the definition preserves the leximin nature according to which a solution that hurts the poorest is never preferred.

\begin{lemma}\label{lemma:absence-of-errors}
 In the absence of errors, $\DEFmultError = \DEFadditiveError = 0$, a solution is approximately-leximin-optimal if and only if it is leximin-optimal.
\end{lemma}
% \eden{should it be a corollary?}
\begin{proof}
    % We will show that $x^*$ is a leximin optimal solution if and only if it is approximately-optimal for $\DEFmultError = 0$.
    % First, assume that $x^*$ is a leximin optimal solution. 
    % From Lemma \ref{lemma:exact-is-always-optimal} it is also approximately-optimal for any $\DEFmultError \in [0,1)$, in particular for  $\DEFmultError = 0$.
    % Now, assume that $x^*$ is approximately-optimal for $\DEFmultError = 0$.
    % By definition, $x \nxPreferred{1} x^*$ for any solution $x \in S$. 
    % By Observation \ref{obs:approx-relation-prop2} we can conclude that $x \nLeximinPreferred x^*$ and therefore it is 
    %
    % The claim follows almost directly from Lemma \ref{lemma:approx-relation-prop1}, which implies that $y \nLeximinPreferred x \iff y \nAlphaBetaPreferredParams{0}{0} x$.
    By definition, a solution $x^*$ is approximately-optimal for $\DEFmultError = \DEFadditiveError = 0$ if and only if $x \nAlphaBetaPreferredParams{1}{0} x^*$ for any solution $x \in S$.
    This holds if and only if $x \nLeximinPreferred x^*$ for any solution $x \in S$ (by Lemma \ref{lemma:approx-relation-prop1}).
    This means, by definition, that $x^*$ is a leximin-optimal solution.
\end{proof}




\begin{lemma}\label{lemma:beta1-beta2-approx}
    Let $x \in S$, $0 \leq \DEFmultErrorOf{\DEFmultApprox_1} \leq  \DEFmultErrorOf{\DEFmultApprox_2} < 1$, and $0 \leq \DEFadditiveError_1 \leq \DEFadditiveError_2$. If $x$ is $(\DEFmultApprox_1,\DEFadditiveError_1)$-approximately-optimal then it is also $(\DEFmultApprox_2,\DEFadditiveError_2)$-approximately-optimal.
\end{lemma}

\begin{proof}
    Assume that $x$ is $(\DEFmultApprox_1,\DEFadditiveError_1)$-approximately-optimal.
    By definition, $y \nAlphaBetaPreferredParams{\DEFmultApprox_1}{\DEFadditiveError_1} x$ for any solution $y \in S$.
    Observation \ref{obs:approx-relation-prop2} implies
        % \footnote{Observation \ref{obs:approx-relation-prop2} says that $y \alphaBetaPreferredParams{\DEFmultError_2}{\DEFadditiveError_2} x \Rightarrow y \alphaBetaPreferredParams{\DEFmultError_1}{\DEFadditiveError_1} x$, which implies that $y \nAlphaBetaPreferredParams{\DEFmultError_1}{\DEFadditiveError_1} x \Rightarrow y \nAlphaBetaPreferredParams{\DEFmultError_2}{\DEFadditiveError_2} x$.} 
        that
        % $y \nAlphaBetaPreferredParams{\DEFmultError_1}{\DEFadditiveError_1} x \Rightarrow 
        $y \nAlphaBetaPreferredParams{\DEFmultApprox_2}{\DEFadditiveError_2} x$ 
    % Therefore, $y \nAlphaBetaPreferredParams{\DEFmultError_2}{\DEFadditiveError_2} x$ 
    for any solution $y \in S$. This means, by definition, that $x$ is $(\DEFmultApprox_2,\DEFadditiveError_2)$-approximately-optimal.
\end{proof}


\begin{lemma}\label{lemma:exact-is-always-optimal}
    Let $x^* \in S$ be a leximin optimal solution. Then $x^*$ is also $(\DEFmultApprox,\DEFadditiveError)$-approximately-optimal for any $\DEFmultError \in [0,1)$  and $\DEFadditiveError \geq 0$.
\end{lemma}

% \eden{maybe we should say somewhere that for brevity when we say "any other solution" we mean any solution that has different sorted utility vector; where is the right place to write it?}
\begin{proof}
    % Let $\DEFmultError \in [0,1)$.
    By Lemma \ref{lemma:absence-of-errors}, the solution $x^*$ is also approximately-optimal for $\DEFmultError = \DEFadditiveError = 0$.
    But this means, according to Lemma \ref{lemma:beta1-beta2-approx}, that $x^*$ is also $(\DEFmultApprox_2,\DEFadditiveError_2)$-approximately-optimal for any $0 \leq \DEFmultErrorOf{\DEFmultApprox_2} < 1$ and $\DEFadditiveError_2 \geq 0$.
    %
    % definition of a leximin optimal solution,  $x \nLeximinPreferred x^*$ for any solution $x \in S$.
    % However, as $\frac{1}{1-\DEFmultError} \geq 1$, Observation \ref{obs:approx-relation-prop2} implies\footnote{Observation \ref{obs:approx-relation-prop2} says that $y \xPreferred{\gamma} x \Rightarrow y \alphaBetaPreferred x$ for $\gamma \geq \delta \geq 1$, which implies that $y \nDeltaPreferred x \Rightarrow y \nxPreferred{\gamma} x$} that $x \nLeximinPreferred x^* \Rightarrow x \nxPreferred{\frac{1}{1-\DEFmultError}} x^*$.
    % Therefore, there is no solution that is  $\frac{1}{1-\DEFmultError}$-preferred over $x^*$ and so, by definition, $x^*$ is also $(1-\DEFmultError)$ approximately-optimal.
\end{proof}


Using the example given previously, we shall now demonstrate that as the error parameters $\DEFmultError$ and $\DEFadditiveError$ increase, the quality of the \emph{approximation} decreases.
Given solutions $x,y,z$ with sorted utility vectors $u_x=(1,10,15), u_y =(1,40,60), u_z=(2,20,30)$, we saw that $\relationSetParams{1}{0} = \{(z,x),(z,y),(y,x)\}$. 
In this case, the only solution over which no solution is preferred is $z$.
Therefore, when $\DEFmultError = \DEFadditiveError = 0$, the only approximately-optimal solution is $z$ which is also the only leximin optimal one.
We also saw that $\relationSetParams{0.75}{0} = \relationSetParams{1}{1} = \relationSetParams{0.9}{0.5} = \{(z,x),(z,y),(y,x)\}$; here, similarly, the only approximately-optimal solution for these parameters is $z$.
However, $\relationSetParams{0.5}{0} =\relationSetParams{1}{15} = \relationSetParams{0.75}{1} = \{(y,x)\}$. 
According to the relation set $\{(y,x)\}$, both $z$ and $y$ are solutions over which no solution is preferred, and therefore, they are both approximately optimal for these parameters.
Lastly, as $\relationSetParams{0.25}{0} =\relationSetParams{1}{45} = \relationSetParams{0.75}{20} = \emptyset$, \emph{all} three solutions are approximately-optimal for these parameters.



% let $\gamma \geq \delta \geq 1$ if $x \succ_{\gamma} y$ then also $x \succ_{\delta} y$.
% \eden{it is easy to see that $x \succ_{\delta} y \Rightarrow x \succ_{\delta'} y$ for $\delta \geq \delta'$}
% \eden{therefore, we can also notice the following relation: $x \succ_{\delta} y \Rightarrow x \succ y$, which also implies $x \nsucc y \Rightarrow x \nsucc_{\delta} y$}
% To illustrate, l
% In particular, this implies that for any $\delta > 1$ 



% \paragraph{Characteristics}
% \eden{need to rewrite}
% \begin{itemize}
%     \item In the absence of errors ($\DEFmultError = 1$) the approximate definition is identical to the exact definition.

%     \item For any $\DEFmultError$ the Leximin optimal solution is approximately optimal as well.

%     \item Preserves the Leximin nature/semantics.
    
% \end{itemize}


\begin{document}

\title{Stochastic Proximal Bundle Method with Max Model}
%\date{July 2022}
\maketitle


\section{Problem Setup}

Stochastic	
	convex composite optimization (SCCO) problem
	\begin{equation}\label{eq:ProbIntro}
	\phi_{*}:=\min \left\{\phi(x):=f(x)+h(x): x \in \R^n\right\}
	\end{equation}
	where 
	\begin{equation}\label{pbint2}
	f(x)=\mathbb{E}_{\xi}[F(x,\xi)].
    \end{equation}
    \begin{equation}\label{pbint2s}
	f'(x)=\mathbb{E}_{\xi}[F'(x,\xi)].
    \end{equation}
% Assume that
% \[
% \E_\xi \left [ (F(u;\xi) - f(u))^2 \right] \le \E[(F(u;\xi)-f(u))^2].
% \]
% \[
% \E_\xi \left [ \E (F(u;\xi) - F(u;\xi'))^2 \right] \le \E[(F(u;\xi)-f(u))^2]
% \]



\begin{equation}
\label{def:tell}
    \tilde \ell(u,x;\xi) := F(x;\xi)+ \inner{F'(x;\xi)}{u-x} + h(u)
\end{equation}

\begin{equation}
\label{def:ell}
     \ell(u,x;\xi) := f(x)+ \inner{F'(x;\xi)}{u-x} + h(u)
\end{equation}

The linearization satisfies
\[
\tilde \ell(u,z_i;\xi) \le \Phi(u;\xi).
\]

\subsection{Assumptions}
	
Let $\Xi$ denote the
	support of random 
	vector $\xi$ and assume that the following conditions on \eqref{eq:ProbIntro} are assumed to hold:
	\begin{itemize}
% \item[(A1)] \red{remove}
% $h \in \bConv{n}$ is
% $M_h$-Lipschitz continuous on its domain, i.e., $|h(x)-h(x')| \le M_h \|x-x'\|$ for every $x,x' \in \dom h$;
\item[(A1)]
$f$ and $ h$ are proper closed convex functions satisfying
		$\dom f \supset \dom h$;
		
\item[(A2)] for almost every $\xi \in \Xi$,
a functional oracle $F(\cdot,\xi) :\dom h \to \R$ and
a stochastic subgradient
oracle $s(\cdot,\xi):\dom h \to \R^n$ satisfying
\[
f(x) = \E[F(x,\xi)], \quad f'(x) := \E[s(x,\xi)] \in \partial f(x)
\]
for every $x \in \dom h$ are available;
		\item[(A3)]
		$M := \sup \{ \E[\|s(x,\xi)\|^2]^{1/2} : x \in \dom h \} < \infty$;
%  \item[(A5)] $\dom h$ has a finite diameter $D>0$;
 	\item[(A4)]
		the set of optimal solutions $X^*$ of
	 \eqref{eq:ProbIntro}-\eqref{pbint2}	is nonempty.
	 % \item[(A5)] $\E \left [ (F(x;\xi) - f(x))^2 \right] \le \E[(F(u;\xi)-f(u))^2]$ for every $x\in \dom h$;
    \item[(A6)] $\dom h$ has a diameter $d_0<\infty$.
	\end{itemize}
	

	
	We now make some observations about the above conditions.
	First, as in \cite{nemjudlannem09},
	condition (A2) does not require $F(\cdot,\xi)$ to be
	convex.
	Second,
condition (A3) implies that
\begin{equation}\label{ineq:fp}
 \|f'(x)\| = \|\E[s(x,\xi)]\| \le \E[\|s(x,\xi)\|] \le \left(\E[\|s(x,\xi)\|^2]\right)^{1/2} \le M \quad \forall x \in \dom h.
\end{equation}
Third, defining for every $\xi \in \Xi$ and $x \in \dom h$,
	\begin{equation}\label{def:Phi}
	\Phi(\cdot,\xi)=F(\cdot,\xi)+h(\cdot), \quad 
	\ell(\cdot,x;\xi)= f(x)+\inner{s(x,\xi)}{\cdot-x} + h(\cdot), 
	\end{equation}
it follows from (A2), the second identity in \eqref{def:Phi},
and the convexity of $f$ by (A1), that
\begin{equation}\label{eq:exp0}
    \E[\Phi(\cdot,\xi)] = \phi(\cdot)\ge f(x) + \inner{f'(x)}{\cdot-x} + h(\cdot) =
\E[\ell(\cdot;x,\xi)]
\end{equation}
where $\phi(\cdot)$ is as in \eqref{eq:ProbIntro}.
Hence, $\ell(\cdot;x,\xi)$ is
a stochastic composite linear approximation of
$\phi(\cdot)$
in the sense that its expectation is a true composite linear approximation of $\phi(\cdot)$. (The terminology ``composite" refers to the function $h$ which is included in the approximation $\ell(\cdot;x,\xi)$ as is.)




\section{Stochastic Multi-cut Bundle method}

\red{stochastic regularized cutting-plane (SRCP) method}
\red{change the algorithm}

\noindent\rule[0.5ex]{1\columnwidth}{1pt}

Stochastic Multi-cut Bundle Method (SMCB)

\noindent\rule[0.5ex]{1\columnwidth}{1pt}
{\bf Input:} Scalars $\{\lam_j\}>0$, integer $I \ge 1$, set $B$ such that 
$\{1\} \subseteq B \subseteq \{1,\ldots, \lfloor I/2 \rfloor\}$, and point $ x_0 \in \dom h $.
\begin{itemize}
\item [0.] 
 Set 
 %$j=1$, $z_0=x$
	$j=1$ and $z_0=x_0$,
 take a sample $\xi_0$
 of $\xi$, and set
 $\underline{\phi}_0(\cdot) = \tilde \ell (\cdot,z_0;\xi_0)$; 
    \item [1.] 
 %    take 
	% a sample $\xi_{j-1}$ of r.v.\ $\xi$ independent from the previous samples $\xi_0,\ldots,\xi_{j-2}$
	% and 
 compute
 %    \begin{align} \label{eq:Gammaj}
	% \bar \phi_j(\cdot) =  \left\{\begin{array}{ll}
	%     % \tilde \ell(\cdot,x;\xi_{0}), & \text { if } j=1 , \\ 
	% 	   \max \{ \fr
	%     \tilde \ell(\cdot,z_{j-1};\xi_{j-1}) +  \frac{j-1}{j} \bar \phi_{j-1}(\cdot), \tilde \ell(\cdot,z_{j-1};\xi_{j-1})\},  & \text { if } j \in B,
 %     \\ 
	% 	    \frac{1}{j}
	%     \tilde \ell(\cdot,z_{j-1};\xi_{j-1}) +  \frac{j-1}{j} \bar \phi_{j-1}(\cdot) ,  & \text { otherwise},
	% 	    \end{array}\right.
 %    \end{align}	
    \begin{align} 
	\underline{\phi}_j(\cdot) &=  \left\{\begin{array}{ll}
	    % \tilde \ell(\cdot,x;\xi_{0}), & \text { if } j=1 , \\ 
		   \max\{\frac{1}{j}
	    \tilde \ell(\cdot,z_{j-1};\xi_{j-1}) + \frac{j-1}{j}  
     % \frac{1}{j}
	    % \tilde \ell(\cdot,z_{j-1};\xi_{j-1}) +  \frac{j-1}{j}
     \underline{\phi}_{j-1}(\cdot), \tilde \ell(\cdot,z_{j-1};\xi_{j-1})\},  & \text { if } j \in B,
     \\ 
		    \frac{1}{j}
	    \tilde \ell(\cdot,z_{j-1};\xi_{j-1}) +  \frac{j-1}{j} \underline{\phi}_{j-1}(\cdot) ,  & \text { otherwise},
		    \end{array}\right. \label{eq:Gammaj}\\
  %   \end{align}	
  %   \lessgap
  %   \lessgap
		% \begin{align}
	    z_{j} &=\underset{u\in \R^n}\argmin
	    \left\lbrace  \underline{\phi}_j^{\lam_j}(u):=
	    \underline{\phi}_j(u) +\frac{1}{2\lam_j}\|u- z_0 \|^2 \right\rbrace, \label{def:xj}  \\  
% 	    \end{align}
% and
% 	    \lessgap
%     \lessgap
% 	    \begin{align} \label{def:yj}
	w_j &=  \frac{1}{j} \sum_{i=1}^j z_i \label{def:wj}
    \end{align}		 
    \lessgap
    \item [2.] 
    if $j<I$, then take 
	a sample $\xi_{j}$ of r.v.\ $\xi$ independent from the previous samples $\xi_0,\ldots,\xi_{j-1}$,
 set $j \leftarrow j+1$ and go to step 1; otherwise \textbf{stop}.
\end{itemize}
{\bf Output:} $(z_I,w_I)$.

\noindent
\rule[0.5ex]{1\columnwidth}{1pt}

--------Renato's notes---

First dual-average paper by Nesterov
\[
\lam_j = \frac{j}{\beta_j}
\]
\[
\beta_j \approx \sqrt{j}
\]

Second dual-average paper by Nesterov

 \[
\lam_j = \frac{j}{\beta_j}
\]
\[
\beta_j \approx \gamma \sqrt{I+1}
\]

-----------

\section{Only $M$ included}

\subsection{Deterministic case (convex case)}

Let $\lam>0$ and $\Gamma \le \phi$ be given
and set
\[
x = \argmin \left \{ \Gamma(u) + \frac{1}{2\lam} \|u-x_0\|^2 \right\}
\]
\[
m = \min \left \{ \Gamma(u) + \frac{1}{2\lam} \|u-x_0\|^2 \right\}
\]
Update
\[
\Gamma^+ = \tau \Gamma + (1-\tau) \ell_\phi(\cdot;x)
\]
and assume that 
\[
\lam^+ \le \frac{\lam}{\tau}
\]
Have
\begin{align*}
    m^+ &= \Gamma^+(x^+) + \frac{1}{2\lam^+} \|x^+-x_0\|^2 \\
    &=\tau \Gamma(x^+) + (1-\tau) \ell_\phi(x^+;x) + \frac{1}{2\lam^+} \|x^+-x_0\|^2 \\
    &\ge \tau \left [  \Gamma(x^+) + \frac{1}{2\lam} \|x^+-x_0\|^2 \right] + (1-\tau) \ell_\phi(x^+;x)  \\
    &\ge \tau \left [  \Gamma(x) + \frac{1}{2\lam} \|x-x_0\|^2 + \frac{1}{2\lam} \|x^+-x\|^2  \right] + (1-\tau) \ell_\phi(x^+;x)  \\
    &= \tau \left [  m + \frac{1}{2\lam} \|x^+-x\|^2  \right] + (1-\tau) \ell_\phi(x^+;x)
\end{align*}
Hence,
\begin{align*}
    m^+ - \tau m &\ge \frac{1}{2\lam} \|x^+-x\|^2  + (1-\tau) \ell_\phi(x^+;x) \\
    &= \frac{1}{2\lam} \|x^+-x\|^2  + (1-\tau) \phi(x^+) - (1-\tau) [\phi(x^+)-\ell_\phi(x^+;x) ] \\
    &\ge \frac{1}{2\lam} \|x^+-x\|^2  + (1-\tau) \phi(x^+) - (1-\tau) 2M \|x^+-x\| \\
    &\ge (1-\tau) \phi(x^+) -
    2 (1-\tau)^2 M^2 \lam
\end{align*}
Thus
\begin{align*}
    t^+- \tau t &= [ \phi(y^+)-m^+] -
\tau [ \phi(y) - m] =
[ \phi(y^+)-\tau  \phi(y)] - [m^+-\tau m] \\
&\le [ \phi(y^+)-\tau  \phi(y) - (1-\tau) \phi(x^+) ] + 2 (1-\tau)^2 M^2 \lam \\
&\le  2 (1-\tau)^2 M^2 \lam
\end{align*}
Indexing the above equation, we conclude that
\begin{align*}
    t_{j+1} - \tau_j t_j \le 2 (1-\tau_j)^2 M^2 \lam_j
\end{align*}
Assume that
\[
\tau_j = \frac{j}{j+1} \quad \forall j \ge 1
\]
Multiplying ?? by $j+1$, we then conclude that
\[
(j+1) t_{j+1} - j t_j \le
\frac{2}{j+1} M^2 \lam_j
\]
and hence that
\[
J t_J - t_1 \le 2 M^2 \sum_{j=1}^{J-1} \frac{\lam_j}{j+1}
\]
{\bf 1st choice:}
\[
\lam_j = \gamma \sqrt{j} \quad \forall j \ge 1
\]
and note that
\[
\lam_{j+1} = \gamma \sqrt{j+1} =
\gamma \sqrt{j} \sqrt{\frac{j+1}{j}}
= \lam_j \sqrt{\frac{j+1}{j}}
\le \lam_j \frac{j+1}{j} = \frac{\lam_j}{\tau_j}
\]
So, we conclude that
\[
J t_J - t_1 \le 2 M^2 \sum_{j=1}^{J-1} \frac{\lam_j}{j+1} \le 2 \gamma M^2 \sum_{j=1}^{J-1} \frac{1}{\sqrt{j}} \approx \gamma M^2 \sqrt{J}
\]
Hence
\[
t_J \le \frac{t_1}{J} +
\frac{\gamma M^2}{\sqrt{J}}
\]
Thus
\begin{align*}
    \frac{t_1}{J} +
\frac{\gamma M^2}{\sqrt{J}} &\ge
\phi(y_J) - \left[ \Gamma_J(x_J) + \frac{1}{2 \lam_J} \|x_J-x_0\|^2 \right] \\
&\ge \phi(y_J) - \left[ \Gamma_J(x_*) + \frac{1}{2 \lam_J} \|x_*-x_0\|^2 - \frac{1}{2 \lam_J} \|x_*-x_J\|^2 \right] \\
&\ge \phi(y_J) - \left[ \phi(x_*) + \frac{1}{2 \lam_J} \|x_*-x_0\|^2 - \frac{1}{2 \lam_J} \|x_*-x_J\|^2 \right]
\end{align*}
so that
\begin{align*}
    \phi(y_J) - \phi(x_*)
&\le 
\frac{t_1}{J} +
\frac{\gamma M^2}{\sqrt{J}} + \frac{1}{2 \lambda_J} \left( \|x_*-x_0\|^2 -
 \|x_*-x_J\|^2 \right)  \\
&\le \frac{t_1}{J} +
\frac{\gamma M^2}{\sqrt{J}} + \frac{1}{2 \gamma \sqrt{J}} \left( \|x_*-x_0\|^2 -
 \|x_*-x_J\|^2 \right)
\end{align*}
{\bf 2nd choice:}
\[
\lam_j = \gamma \frac{j}{\sqrt{J}} \quad \forall j \ge 1
\]
and note that
\[
\lam_{j+1} = \gamma \sqrt{j+1} =
\gamma \sqrt{j} \sqrt{\frac{j+1}{j}}
= \lam_j \sqrt{\frac{j+1}{j}}
\le \lam_j \frac{j+1}{j} = \frac{\lam_j}{\tau_j}
\]
So, we conclude that
\[
J t_J - t_1 \le 2 M^2 \sum_{j=1}^{J-1} \frac{\lam_j}{j+1} \le \frac{2 \gamma M^2}{\sqrt{J}} \sum_{j=1}^{J-1} \frac{j}{j+1} \approx \gamma M^2 \sqrt{J}
\]
Hence
\[
t_J \le \frac{t_1}{J} +
\frac{\gamma M^2}{\sqrt{J}}
\]
Thus
\begin{align*}
    \frac{t_1}{J} +
\frac{\gamma M^2}{\sqrt{J}} &\ge
\phi(y_J) - \left[ \Gamma_J(x_J) + \frac{1}{2 \lam_J} \|x_J-x_0\|^2 \right] \\
&\ge \phi(y_J) - \left[ \Gamma_J(x_*) + \frac{1}{2 \lam_J} \|x_*-x_0\|^2 - \frac{1}{2 \lam_J} \|x_*-x_J\|^2 \right] \\
&\ge \phi(y_J) - \left[ \phi(x_*) + \frac{1}{2 \lam_J} \|x_*-x_0\|^2 - \frac{1}{2 \lam_J} \|x_*-x_J\|^2 \right]
\end{align*}
so that
\begin{align*}
    \phi(y_J) - \phi(x_*)
&\le 
\frac{t_1}{J} +
\frac{\gamma M^2}{\sqrt{J}} + \frac{1}{2 \lambda_J} \left( \|x_*-x_0\|^2 -
 \|x_*-x_J\|^2 \right)  \\
&\le \frac{t_1}{J} +
\frac{\gamma M^2}{\sqrt{J}} + \frac{1}{2 \gamma \sqrt{J}} \left( \|x_*-x_0\|^2 -
 \|x_*-x_J\|^2 \right)
\end{align*}
since $\lam_J = \gamma \sqrt{J}$

\subsection{Deterministic case (strongly convex case)}

Let $\lam>0$ and $\Gamma \le \phi$ be given
and set
\[
x = \argmin \left \{ \Gamma(u) + \frac{1}{2\lam} \|u-x_0\|^2 \right\}
\]
\[
m = \min \left \{ \Gamma(u) + \frac{1}{2\lam} \|u-x_0\|^2 \right\}
\]
Update
\[
\Gamma^+ = \tau \Gamma + (1-\tau) \ell_\phi(\cdot;x)
\]
and assume that 
\[
\lam^+ \le \frac{\lam}{\tau}
\]
Have
\begin{align*}
    m^+ &= \Gamma^+(x^+) + \frac{1}{2\lam^+} \|x^+-x_0\|^2 \\
    &=\tau \Gamma(x^+) + (1-\tau) \ell_\phi(x^+;x) + \frac{1}{2\lam^+} \|x^+-x_0\|^2 \\
    &\ge \tau \left [  \Gamma(x^+) + \frac{1}{2\lam} \|x^+-x_0\|^2 \right] + (1-\tau) \ell_\phi(x^+;x)  \\
    &\ge \tau \left [  \Gamma(x) + \frac{1}{2\lam} \|x-x_0\|^2 + \frac{1}{2\lam_\mu} \|x^+-x\|^2  \right] + (1-\tau) \ell_\phi(x^+;x)  \\
    &= \tau \left [  m + \frac{1}{2\lam_\mu} \|x^+-x\|^2  \right] + (1-\tau) \ell_\phi(x^+;x)
\end{align*}
where
\[
\frac{1}{\lam_\mu} := \frac1\lam + \mu \ge \mu
\]
Hence,
\begin{align*}
    m^+ - \tau m &\ge \frac{1}{2\lam_mu} \|x^+-x\|^2  + (1-\tau) \ell_\phi(x^+;x) \\
    &= \frac{1}{2\lam_\mu} \|x^+-x\|^2  + (1-\tau) \phi(x^+) - (1-\tau) [\phi(x^+)-\ell_\phi(x^+;x) ] \\
    &\ge \frac{1}{2\lam_\mu} \|x^+-x\|^2  + (1-\tau) \phi(x^+) - (1-\tau) 2M \|x^+-x\| \\
    &\ge (1-\tau) \phi(x^+) -
    2 (1-\tau)^2 M^2 \lam^\mu
\end{align*}
Thus
\begin{align*}
    t^+- \tau t &= [ \phi(y^+)-m^+] -
\tau [ \phi(y) - m] =
[ \phi(y^+)-\tau  \phi(y)] - [m^+-\tau m] \\
&\le [ \phi(y^+)-\tau  \phi(y) - (1-\tau) \phi(x^+) ] + 2 (1-\tau)^2 M^2 \lam^\mu \\
&\le  2 (1-\tau)^2 M^2 \lam^\mu 
% \le \frac{2 (1-\tau)^2 M^2} \mu
\end{align*}
Indexing the above equation, we conclude that
\begin{align}
    t_{j+1} - \tau_j t_j \le 2 (1-\tau_j)^2 M^2 \lam^\mu_j
    % \frac{2 (1-\tau_j)^2 M^2} \mu
     \label{eq:recurv-st-conv}
\end{align}
Assume that
\[
\tau_j = \frac{j}{j+1} \quad \forall j \ge 1
\]
Multiplying \eqref{eq:recurv-st-conv} by $j+1$, we then conclude that
\[
(j+1) t_{j+1} - j t_j \le
\frac{2M^2 \lam_j^\mu}{(j+1)}
\]
and hence that
\[
J t_J - t_1 \le 2 M^2  \sum_{j=1}^{J-1} \frac{\lam^\mu_j}{j+1}
\]
Now, choose
\[
\lam_j = \gamma \sqrt{j} \left( 1 + \mu \sqrt{j} \right)  \quad \forall j \ge 1
\]
and note that
\[
\lam_{j+1} = \gamma \sqrt{j+1}\left( 1 + \mu \sqrt{j+1} \right) =
\gamma \sqrt{{\frac{j}{\tau_j}}} 
\left( 1 + \mu \sqrt{{\frac{j}{\tau_j}}} \right)
\le
\frac{1}{\tau_j} \gamma \sqrt{j} 
\left( 1 + \mu \sqrt{j} \right)
= \frac{\lam_j}{\tau_j}
\]
So, we conclude that
\[
J t_J - t_1 \le 2 M^2 \sum_{j=1}^{J-1} \frac{\lam_j^\mu}{j+1} \le \frac{2  M^2}{\mu} \sum_{j=1}^{J-1} \frac{1}{j+1} \approx \frac{ M^2}{\mu} \log J
\]
Hence
\[
t_J \le \frac{t_1}{J} +
\frac{ M^2 \log J}{\mu J}
\]
Thus
\begin{align*}
    \frac{t_1}{J} +
\frac{\gamma M^2}{\sqrt{J}} &\ge
\phi(y_J) - \left[ \Gamma_J(x_J) + \frac{1}{2 \lam_J} \|x_J-x_0\|^2 \right] \\
&\ge \phi(y_J) - \left[ \Gamma_J(x_*) + \frac{1}{2 \lam_J} \|x_*-x_0\|^2 - \frac{1}{2 \lam_J^\mu} \|x_*-x_J\|^2 \right] \\
&\ge \phi(y_J) - \left[ \phi(x_*) + \frac{1}{2 \lam_J} \|x_*-x_0\|^2 - \frac{1}{2 \lam_J^\mu} \|x_*-x_J\|^2 \right]
\end{align*}
so that
\begin{align*}
   \phi(y_J) - \phi(x_*)
&\le \frac{t_1}{J} +\frac{ M^2 \log J}{\mu J}
 + \frac{1}{2 \lam_J} \|x_*-x_0\|^2 -
\frac{1}{2 \lam_J^\mu} \|x_*-x_J\|^2 \\
&\le \frac{t_1}{J} +\frac{ M^2 \log J}{\mu J}
 + \frac{1}{2 \gamma \mu J} \|x_*-x_0\|^2
\end{align*}
since $\lam_J \ge \gamma \mu J$

\section{Including $(L,M)$}

\subsection{Deterministic case (convex case)}

Let $\lam>0$ and $\Gamma \le \phi$ be given
and set
\[
x = \argmin \left \{ \Gamma(u) + \frac{1}{2\lam} \|u-x_0\|^2 \right\}
\]
\[
m = \min \left \{ \Gamma(u) + \frac{1}{2\lam} \|u-x_0\|^2 \right\}
\]
Update
\[
\Gamma^+ = \tau \Gamma + (1-\tau) \ell_\phi(\cdot;x)
\]
and assume that 
\[
\lam^+ \le \frac{\lam}{\tau}
\]
Have
\begin{align*}
    m^+ &= \Gamma^+(x^+) + \frac{1}{2\lam^+} \|x^+-x_0\|^2 \\
    &=\tau \Gamma(x^+) + (1-\tau) \ell_\phi(x^+;x) + \frac{1}{2\lam^+} \|x^+-x_0\|^2 \\
    &\ge \tau \left [  \Gamma(x^+) + \frac{1}{2\lam} \|x^+-x_0\|^2 \right] + (1-\tau) \ell_\phi(x^+;x)  \\
    &\ge \tau \left [  \Gamma(x) + \frac{1}{2\lam} \|x-x_0\|^2 + \frac{1}{2\lam} \|x^+-x\|^2  \right] + (1-\tau) \ell_\phi(x^+;x)  \\
    &= \tau \left [  m + \frac{1}{2\lam} \|x^+-x\|^2  \right] + (1-\tau) \ell_\phi(x^+;x)
\end{align*}
Hence,
\begin{align*}
    m^+ - \tau m &\ge \frac{1}{2\lam} \|x^+-x\|^2  + (1-\tau) \ell_\phi(x^+;x) \\
    &= \frac{1}{2\lam} \|x^+-x\|^2  + (1-\tau) \phi(x^+) - (1-\tau) [\phi(x^+)-\ell_\phi(x^+;x) ] \\
    &\ge \frac{1}{2\lam} \|x^+-x\|^2  + (1-\tau) \phi(x^+) - (1-\tau) \left [  2M \|x^+-x\| + \frac{L}2 \|x^+-x\|^2 \right ] \\
    &= (1-\tau) \left[
    \phi(x^+) -  2M \|x^+-x\| + \frac12 \left(\frac{1}{(1-\tau)\lam} - L \right) \|x^+-x\|^2
    \right ] \\
    &\ge (1-\tau) \left [ \phi(x^+) -
    \frac{ 2(1-\tau) M^2 \lam}{1-L(1-\tau) \lam} \right]
\end{align*}
Thus
\begin{align*}
    t^+- \tau t &= [ \phi(y^+)-m^+] -
\tau [ \phi(y) - m] =
[ \phi(y^+)-\tau  \phi(y)] - [m^+-\tau m] \\
&\le [ \phi(y^+)-\tau  \phi(y) - (1-\tau) \phi(x^+) ] +
    \frac{ 2(1-\tau)^2 M^2 \lam}{1-L(1-\tau) \lam} \\
&\le  \frac{ 2(1-\tau)^2 M^2 \lam}{1-L(1-\tau) \lam}
\end{align*}
Assume
\beq \label{eq:con-comp}
\frac{1}{1-\tau} \ge \lam ( L + \alpha )
\eeq
or equivalently
\[
1 - (1-\tau) \lam L \ge (1-\tau) \lam \alpha
\]
Then
\[
 t^+- \tau t \le \frac{ 2(1-\tau)^2 M^2 \lam}{1-L(1-\tau) \lam} \le \frac{ 2(1-\tau) M^2 }{\alpha}
\]
Indexing the above equation, we conclude that
\begin{align*}
    t_{j+1} - \tau_j t_j \le \frac{2 (1-\tau_j) M^2}{\alpha_j}
\end{align*}
Assume that
\[
\tau_j = \frac{j}{j+1} \quad \forall j \ge 1
\]
Multiplying ?? by $j+1$, we then conclude that
\[
(j+1) t_{j+1} - j t_j \le
\frac{2 M^2}{\alpha_j}
\]
and hence that
\[
J t_J - t_1 \le 2 M^2 \sum_{j=1}^{J-1} \frac{1}{\alpha_j}
\]
{\bf 1st choice:}
\[
\lam_j = \gamma \sqrt{j+1} \quad \forall j \ge 0
\]
and note that
\[
\lam_{j+1} = \gamma \sqrt{j+2} =
\gamma \sqrt{j+1} \sqrt{\frac{j+2}{j+1}}
= \lam_j \sqrt{\frac{j+2}{j+1}}
\le \lam_j \frac{j+2}{j+1} = \frac{\lam_j}{\tau_{j+1}} \le
\frac{\lam_j}{\tau_{j}}
\]
since $\tau_j \le \tau_{j+1}$.
Also, \eqref{eq:con-comp} becomes
\[
j+1 = \frac{1}{1-\tau_j} \ge \lam_j ( L + \alpha_j ) = \gamma \sqrt{j+1} ( L + \alpha_j)
\]
or
\[
\sqrt{j+1} \ge \gamma(L + \alpha_j)
\]
or
\[
\alpha_j \le \frac{\sqrt{j+1} -\gamma L} {\gamma}
\]
Take
\[
\alpha_j = \frac{\sqrt{j+1}} {2\gamma}, \quad \gamma \le \frac{1}L
\]
So, we conclude that
\[
J t_J - t_1 \le 2 M^2 \sum_{j=1}^{J-1} \frac{1}{\alpha_j} \le 2 M^2 \sum_{j=1}^{J-1} \frac{2 \gamma}{\sqrt{j+1}} \approx \gamma M^2 \sqrt{J}
\]
Hence
\[
t_J \le \frac{t_1}{J} +
\frac{\gamma M^2}{\sqrt{J}}
\]
Thus
\begin{align*}
    \frac{t_1}{J} +
\frac{\gamma M^2}{\sqrt{J}} &\ge
\phi(y_J) - \left[ \Gamma_J(x_J) + \frac{1}{2 \lam_J} \|x_J-x_0\|^2 \right] \\
&\ge \phi(y_J) - \left[ \Gamma_J(x_*) + \frac{1}{2 \lam_J} \|x_*-x_0\|^2 - \frac{1}{2 \lam_J} \|x_*-x_J\|^2 \right] \\
&\ge \phi(y_J) - \left[ \phi(x_*) + \frac{1}{2 \lam_J} \|x_*-x_0\|^2 - \frac{1}{2 \lam_J} \|x_*-x_J\|^2 \right]
\end{align*}
so that
\begin{align*}
    \phi(y_J) - \phi(x_*)
&\le 
\frac{t_1}{J} +
\frac{\gamma M^2}{\sqrt{J}} + \frac{1}{2 \lambda_J} \left( \|x_*-x_0\|^2 -
 \|x_*-x_J\|^2 \right)  \\
&\le \frac{t_1}{J} +
\frac{\gamma M^2}{\sqrt{J}} + \frac{1}{2 \gamma \sqrt{J}} \|x_*-x_0\|^2 
\end{align*}
{\bf 2nd choice:}
\[
\lam_j = \gamma \frac{j+1}{\sqrt{J+1}} \quad \forall j \ge 1
\]
and note that
\[
\sqrt{J+1} \, \lam_{j+1} = \gamma (j+2) =
\gamma (j+1)  \frac{j+2}{j+1}
= \frac{\gamma (j+1)}{\tau_{j+1}} \le
\frac{\gamma (j+1)}{\tau_{j}} =
\frac{\sqrt{J+1} \, \lam_j}{\tau_{j}}
\]
Also, \eqref{eq:con-comp} becomes
\[
j+1 = \frac{1}{1-\tau_j} \ge \lam_j ( L + \alpha_j ) = \frac{\gamma (j+1)} {\sqrt{J+1}} ( L + \alpha_j)
\]
or
\[
\sqrt{J+1} \ge \gamma(L + \alpha_j)
\]
or
\[
\alpha_j \le \frac{\sqrt{J+1} -\gamma L} {\gamma}
\]
Take
\[
\alpha_j = \frac{\sqrt{J+1}} {2\gamma}, \quad 
\]
and $\gamma$ such that
\[
\frac{\sqrt{J+1}} {2\gamma} \ge L
\]
i.e.,
\[
\gamma = \min\left\{ \frac{\sqrt{J+1}} {2L} , \frac{d_0}M \right\}
\]
So
\[
\gamma^{-1} = \max \left\{ \frac{2L}{\sqrt{J+1}}  , \frac M{d_0} \right\}
\]
So, we conclude that
\[
J t_J - t_1 \le 2 M^2 \sum_{j=1}^{J-1} \frac{1}{\alpha_j} = \frac{4 \gamma M^2(J-1)}{\sqrt{J+1}} \approx  \gamma M^2 \sqrt{J} \le d_0 M \aqrt{J}
\]
since $\gamma \le 1$.
Hence
\[
t_J \le \frac{t_1}{J} +
\frac{d_0 M}{\sqrt{J}}
\]
Thus
\begin{align*}
    \frac{t_1}{J} +
\frac{Md_0}{\sqrt{J}} \ge t_J  &=
\phi(y_J) - \left[ \Gamma_J(x_J) + \frac{1}{2 \lam_J} \|x_J-x_0\|^2 \right] \\
&\ge \phi(y_J) - \left[ \Gamma_J(x_*) + \frac{1}{2 \lam_J} \|x_*-x_0\|^2 - \frac{1}{2 \lam_J} \|x_*-x_J\|^2 \right] \\
&\ge \phi(y_J) -  \phi(x_*) - \frac{1}{2 \lam_J} \|x_*-x_0\|^2 +\frac{1}{2 \lam_J} \|x_*-x_J\|^2
\end{align*}
so that
\begin{align*}
    \phi(y_J) - \phi(x_*)
&\le 
\frac{t_1}{J} +
\frac{Md_0}{\sqrt{J}} + \frac{1}{2 \lambda_J} \left( \|x_*-x_0\|^2 -
 \|x_*-x_J\|^2 \right)  \\
&\le \frac{t_1}{J} +
\frac{Md_0}{\sqrt{J}} + \frac{1}{2 \gamma \sqrt{J}} d_0^2 \\
&\le
\frac{t_1}{J} +
\frac{Md_0}{\sqrt{J}} +
\frac{1}{2 \sqrt{J}} d_0^2 \left( \frac{2L}{\sqrt{J+1}}  +  \frac M{d_0}\right) \\
&= {\cal O}\left( \frac{t_1}{J} +
\frac{Md_0}{\sqrt{J}}
+ \frac{Ld_0^2}{J}
\right)
\end{align*}
since $\lam_J = \gamma \sqrt{J}$

\subsection{Deterministic case (strongly convex case)}

Let $\lam>0$ and $\Gamma \le \phi$ be given
and set
\[
x = \argmin \left \{ \Gamma(u) + \frac{1}{2\lam} \|u-x_0\|^2 \right\}
\]
\[
m = \min \left \{ \Gamma(u) + \frac{1}{2\lam} \|u-x_0\|^2 \right\}
\]
Update
\[
\Gamma^+ = \tau \Gamma + (1-\tau) \ell_\phi(\cdot;x)
\]
and assume that 
\beq \label{eq:cond-cru'}
\lam^+ \le \frac{\lam}{\tau}
\eeq
Have
\begin{align*}
    m^+ &= \Gamma^+(x^+) + \frac{1}{2\lam^+} \|x^+-x_0\|^2 \\
    &=\tau \Gamma(x^+) + (1-\tau) \ell_\phi(x^+;x) + \frac{1}{2\lam^+} \|x^+-x_0\|^2 \\
   \mbox{(due to \eqref{eq:cond-cru'})} \ \ &\ge \tau \left [  \Gamma(x^+) + \frac{1}{2\lam} \|x^+-x_0\|^2 \right] + (1-\tau) \ell_\phi(x^+;x)  \\
    &\ge \tau \left [  \Gamma(x) + \frac{1}{2\lam} \|x-x_0\|^2 + \frac{1}{2\lam_\mu} \|x^+-x\|^2  \right] + (1-\tau) \ell_\phi(x^+;x)  \\
    &= \tau \left [  m + \frac{1}{2\lam_\mu} \|x^+-x\|^2  \right] + (1-\tau) \ell_\phi(x^+;x)
\end{align*}
where
\[
\frac{1}{\lam_\mu} := \frac1\lam + \mu \ge \mu
\]
Assume
\beq \label{eq:con-comp'}
\alpha:= \frac{1}{\lam_\mu(1-\tau)} -L > 0
\eeq
Hence,
\begin{align*}
    m^+ - \tau m &\ge \frac{1}{2\lam_\mu} \|x^+-x\|^2  + (1-\tau) \ell_\phi(x^+;x) \\
    &= \frac{1}{2\lam_\mu} \|x^+-x\|^2  + (1-\tau) \phi(x^+) - (1-\tau) [\phi(x^+)-\ell_\phi(x^+;x) ] \\
    &\ge \frac{1}{2\lam_\mu} \|x^+-x\|^2  + (1-\tau) \phi(x^+) - (1-\tau)  \left [  2M \|x^+-x\| + \frac{L}2 \|x^+-x\|^2 \right ]
     \\
     &= (1-\tau) \left[
    \phi(x^+) -  2M \|x^+-x\| + \frac12 \left(\frac{1}{(1-\tau)\lam_\mu} - L \right) \|x^+-x\|^2
    \right ] \\
 \mbox{(due to \eqref{eq:con-comp'})} \ \    &\ge  (1-\tau) \left[
    \phi(x^+) -  2M \|x^+-x\| + \frac{\alpha}2  \|x^+-x\|^2
    \right ] \\
     &\ge (1-\tau) \left[ \phi(x^+) -
     \frac{2M^2}{\alpha} \right]
\end{align*}
Thus
\begin{align*}
    t^+- \tau t &= [ \phi(y^+)-m^+] -
\tau [ \phi(y) - m] =
[ \phi(y^+)-\tau  \phi(y)] - [m^+-\tau m] \\
&\le [ \phi(y^+)-\tau  \phi(y) - (1-\tau) \phi(x^+) ] + \frac{2 (1-\tau) M^2}{\alpha} \\
&\le  \frac{2 (1-\tau) M^2}{\alpha} 
% \le \frac{2 (1-\tau)^2 M^2} \mu
\end{align*}
Indexing the above equation, we conclude that
\begin{align}
    t_{j+1} - \tau_j t_j \le \frac{2 (1-\tau_j) M^2 }{\alpha_j}
    % \frac{2 (1-\tau_j)^2 M^2} \mu
     \label{eq:recurv-st-conv}
\end{align}
Assume that
\[
\tau_j = \frac{j}{j+1} \quad \forall j \ge 1
\]
Multiplying \eqref{eq:recurv-st-conv} by $j+1$, we then conclude that
\[
(j+1) t_{j+1} - j t_j \le
\frac{2M^2}{\alpha_j}
\]
and hence that
\[
J t_J - t_1 \le 2 M^2 \Gamma_k
\]
where
\[
\Gamma_J := \sum_{j=1}^{J-1} \frac{1}{\alpha_j}
\]
Thus
\begin{align*}
    \frac{t_1 + 2 M^2 \Gamma_J}{J}  &\ge t_J = 
\phi(y_J) - \left[ \Gamma_J(x_J) + \frac{1}{2 \lam_J} \|x_J-x_0\|^2 \right] \\
&\ge \phi(y_J) - \left[ \Gamma_J(x_*) + \frac{1}{2 \lam_J} \|x_*-x_0\|^2 - \frac{1}{2 \lam_J^\mu} \|x_*-x_J\|^2 \right] \\
&\ge \phi(y_J) - \left[ \phi(x_*) + \frac{1}{2 \lam_J} \|x_*-x_0\|^2 - \frac{1}{2 \lam_J^\mu} \|x_*-x_J\|^2 \right]
\end{align*}

---------

\[
\frac{\lam_J^\mu}{\lam J} = \left( \frac{\lam_J}{1+\lam_J \mu} \frac{1}{\lam_J} \right) = \left( \frac{1}{1+\lam_J \mu}  \right)
\]

\[
  2 M d_0 \ge \frac{1}{\lam_J} d_0^2 - \frac{1}{\lam_J^\mu}(d_0^+)^2
\]

---------


so that
\begin{align*}
   \phi(y_J) - \phi(x_*)
&\le \frac{t_1+2 M^2 \Gamma_k}{J} 
 + \frac{1}{2 \lam_J} \|x_*-x_0\|^2 -
\frac{1}{2 \lam_J^\mu} \|x_*-x_J\|^2 \\
&\le \frac{t_1+2 M^2 \Gamma_J}{J} 
 + \frac{d_0^2}{2 \lam_J}
\end{align*}

% -------------------

% \begin{align*}
%    \phi(y_J) - \phi(x_*)
% &\le \frac{t_1}{J} +\frac{ M^2 \log J}{\mu J}
%  + \frac{1}{2 \lam_J} \|x_*-x_0\|^2 -
% \frac{1}{2 \lam_J^\mu} \|x_*-x_J\|^2 \\
% &\le \frac{t_1}{J} +\frac{ M^2 \log J}{\mu J}
%  + \frac{1}{2 \gamma \mu J} \|x_*-x_0\|^2
% \end{align*}
% since $\lam_J \ge \gamma \mu J$

% ----------------

{\bf 1st choice}
\[
\lam_j = \gamma \frac{j}{\sqrt{J}} \left( 1 + \mu \sqrt{j} \right)  \quad \forall j \ge 1
\]
Using the above definition and the fact that $\tau_j \le 1$, we have
\[
\sqrt{J} \lam_{j+1} = \gamma \sqrt{j+1}\left( 1 + \mu \sqrt{j+1} \right) =
\gamma \sqrt{{\frac{j}{\tau_j}}} 
\left( 1 + \mu \sqrt{{\frac{j}{\tau_j}}} \right)
\le
\frac{1}{\tau_j} \gamma \sqrt{j} 
\left( 1 + \mu \sqrt{j} \right)
= \sqrt{J} \frac{\lam_j}{\tau_j}
\]
and hence that \eqref{eq:cond-cru'} holds.
Also, \eqref{eq:con-comp'} becomes
\begin{align*}
\alpha_j + L &=\frac{1}{\lam_j^\mu(1-\tau_j)} =
\frac{j+1}{\lam_j^\mu} 
= (j+1) \left( \mu + \frac{1}{\lam_j} \right)
\ge  (j+1) \mu +  \frac{j}{\lam_j}
\\
&=  (j+1) \mu +  \frac{\sqrt{J}}{\gamma (1+ \mu \sqrt{j})}
\ge (j+1) \mu +  \frac{\sqrt{J}}{\gamma  (1+ \mu \sqrt{J})}
\end{align*}
Take
\[
\gamma :=\frac{\sqrt{J}}{L(1+\mu \sqrt{J})}
\]
Then
\[
\alpha_j + L \ge 
(j+1) \mu +  \frac{\sqrt{J}}{\gamma  (1+ \mu \sqrt{J})} =
(j+1) \mu + L
\]
and hence that
\[
\alpha_j \ge (j+1) \mu %+ L ( \sqrt{j}-1 )
\]
Also,
\[
\lam_J = \gamma \sqrt{J} \left( 1 + \mu \sqrt{J} \right) = \frac{J}{L}
\]


% \[
% j+1 = \frac{1}{1-\tau_j} \ge \lam_j^\mu(L+\alpha_j) = \frac{\lam_j}{1+\lam_j\mu} (L+\alpha_j)
% \]
% Since $\lam_j/(1+\lam_j\mu) \le 1/\mu$,  a sufficient condition for
% \eqref{eq:con-comp'} is
% \[
% j+1 \ge (L+\alpha_j)/\mu
% \]
% or
% \[
% \alpha_j \le \mu (j+1) -L
% \]


{\bf 2nd choice}
\[
\lam_j = \gamma \sqrt{j} \left( 1 + \mu \sqrt{j} \right)  \quad \forall j \ge 1
\]
Using the above definition and the fact that $\tau_j \le 1$, we have
\[
\lam_{j+1} = \gamma \sqrt{j+1}\left( 1 + \mu \sqrt{j+1} \right) =
\gamma \sqrt{{\frac{j}{\tau_j}}} 
\left( 1 + \mu \sqrt{{\frac{j}{\tau_j}}} \right)
\le
\frac{1}{\tau_j} \gamma \sqrt{j} 
\left( 1 + \mu \sqrt{j} \right)
= \frac{\lam_j}{\tau_j}
\]
and hence that \eqref{eq:cond-cru'} holds. Also, \eqref{eq:con-comp'} becomes
\begin{align*}
\alpha_j + L &=\frac{1}{\lam_j^\mu(1-\tau_j)} =
\frac{j+1}{\lam_j^\mu} 
= (j+1) \left( \mu + \frac{1}{\lam_j} \right)
\ge  (j+1) \mu +  \frac{j}{\lam_j}
\\&=  (j+1) \mu +  \frac{j}{\gamma \sqrt{j} (1+ \mu \sqrt{j})}
\ge
(j+1) \mu +  \frac{\sqrt{j}}{\gamma  (1+ \mu \sqrt{j})} \\
&\ge (j+1) \mu +  \frac{\sqrt{j}}{\gamma  (1+ \mu \sqrt{J})}
\end{align*}
Take
\[
\gamma :=\frac{1}{L(1+\mu \sqrt{J})}
\]
Then
\[
\alpha_j + L \ge 
(j+1) \mu +  \frac{\sqrt{j}}{\gamma  (1+ \mu \sqrt{J})} =
(j+1) \mu + L \sqrt{j}
\]
and hence
\[
\alpha_j \ge (j+1) \mu + L ( \sqrt{j}-1 )
\]
Also,
\[
\lam_J = \gamma \sqrt{J} \left( 1 + \mu \sqrt{J} \right) = \frac{\sqrt{J}}{L}
\]
% or
% \[
% \left( \mu + \frac{1}{\lam_j} \right) (j+1) \ge L + \alpha_j
% \]
% Now take
% \[
% \gamma = \frac{L }
% \]
% Since $\lam_j/(1+\lam_j\mu) \le 1/\mu$,  a sufficient condition for
% \eqref{eq:con-comp'} is
% \[
% j+1 \ge (L+\alpha_j)/\mu
% \]
% or
% \[
% \alpha_j \le \mu (j+1) -L
% \]

\subsection{Deterministic case (strongly convex case - Again)}

Let $\lam>0$ and $\Gamma \le \phi$ be given
and set
\[
x = \argmin \left \{ \Gamma(u) + \frac{1}{2\lam} \|u-x_0\|^2 \right\}
\]
\[
m = \min \left \{ \Gamma(u) + \frac{1}{2\lam} \|u-x_0\|^2 \right\}
\]
Update
\[
\Gamma^+ = \tau \Gamma + (1-\tau) \ell_\phi(\cdot;x)
\]
and assume that 
\beq \label{eq:cond-cru'}
\lam^+ \le \frac{\lam}{\tau}
\eeq
Have
\begin{align*}
    m^+ &= \Gamma^+(x^+) + \frac{1}{2\lam^+} \|x^+-x_0\|^2 \\
    &=\tau \Gamma(x^+) + (1-\tau) \ell_\phi(x^+;x) + \frac{1}{2\lam^+} \|x^+-x_0\|^2 \\
   \mbox{(due to \eqref{eq:cond-cru'})} \ \ &\ge \tau \left [  \Gamma(x^+) + \frac{1}{2\lam} \|x^+-x_0\|^2 \right] + (1-\tau) \ell_\phi(x^+;x)  \\
    &\ge \tau \left [  \Gamma(x) + \frac{1}{2\lam} \|x-x_0\|^2 + \frac{1}{2\lam_\mu} \|x^+-x\|^2  \right] + (1-\tau) \ell_\phi(x^+;x)  \\
    &= \tau \left [  m + \frac{1}{2\lam_\mu} \|x^+-x\|^2  \right] + (1-\tau) \ell_\phi(x^+;x)
\end{align*}
where
\[
\frac{1}{\lam_\mu} := \frac1\lam + \mu \ge \mu
\]
Assume
\beq \label{eq:con-comp''}
\alpha:= \frac{1}{\lam_\mu(1-\tau)} -L > 0
\eeq
Hence,
\begin{align*}
    m^+ - \tau m &\ge \frac{1}{2\lam_\mu} \|x^+-x\|^2  + (1-\tau) \ell_\phi(x^+;x) \\
    &= \frac{1}{2\lam_\mu} \|x^+-x\|^2  + (1-\tau) \phi(x^+) - (1-\tau) [\phi(x^+)-\ell_\phi(x^+;x) ] \\
    &\ge \frac{1}{2\lam_\mu} \|x^+-x\|^2  + (1-\tau) \phi(x^+) - (1-\tau)  \left [  2M \|x^+-x\| + \frac{L}2 \|x^+-x\|^2 \right ]
     \\
     &= (1-\tau) \left[
    \phi(x^+) -  2M \|x^+-x\| + \frac12 \left(\frac{1}{(1-\tau)\lam_\mu} - L \right) \|x^+-x\|^2
    \right ] \\
 \mbox{(due to \eqref{eq:con-comp'})} \ \    &\ge  (1-\tau) \left[
    \phi(x^+) -  2M \|x^+-x\| + \frac{\alpha}2  \|x^+-x\|^2
    \right ] \\
     &\ge (1-\tau) \left[ \phi(x^+) -
     \frac{2M^2}{\alpha} \right]
\end{align*}
Thus
\begin{align*}
    t^+- \tau t &= [ \phi(y^+)-m^+] -
\tau [ \phi(y) - m] =
[ \phi(y^+)-\tau  \phi(y)] - [m^+-\tau m] \\
&\le [ \phi(y^+)-\tau  \phi(y) - (1-\tau) \phi(x^+) ] + \frac{2 (1-\tau) M^2}{\alpha} \\
&\le  \frac{2 (1-\tau) M^2}{\alpha} 
% \le \frac{2 (1-\tau)^2 M^2} \mu
\end{align*}
Indexing the above equation, we conclude that
\begin{align}
    t_{j+1} - \tau_j t_j \le \frac{2 (1-\tau_j) M^2 }{\alpha_j}
    % \frac{2 (1-\tau_j)^2 M^2} \mu
     \label{eq:recurv-st-conv}
\end{align}
Assume that
\[
\tau_j = \frac{\tau j}{\tau j+1} \quad \forall j \ge 1
\]
Multiplying \eqref{eq:recurv-st-conv} by $\tau j+1$, we then conclude that
\[
(\tau j+1) t_{j+1} - \tau j t_j \le
\frac{2M^2}{\alpha_j}
\]
and hence that
\[
J t_J - t_1 \le 2 M^2 \Gamma_k
\]
where
\[
\Gamma_k := \sum_{j=1}^{J-1} \frac{1}{\alpha_j}
\]
Thus
\begin{align*}
    \frac{t_1 + 2 M^2 \Gamma_k}{J}  &\ge t_J = 
\phi(y_J) - \left[ \Gamma_J(x_J) + \frac{1}{2 \lam_J} \|x_J-x_0\|^2 \right] \\
&\ge \phi(y_J) - \left[ \Gamma_J(x_*) + \frac{1}{2 \lam_J} \|x_*-x_0\|^2 - \frac{1}{2 \lam_J^\mu} \|x_*-x_J\|^2 \right] \\
&\ge \phi(y_J) - \left[ \phi(x_*) + \frac{1}{2 \lam_J} \|x_*-x_0\|^2 - \frac{1}{2 \lam_J^\mu} \|x_*-x_J\|^2 \right]
\end{align*}
so that
\begin{align*}
   \phi(y_J) - \phi(x_*)
&\le \frac{t_1+2 M^2 \Gamma_k}{J} 
 + \frac{1}{2 \lam_J} \|x_*-x_0\|^2 -
\frac{1}{2 \lam_J^\mu} \|x_*-x_J\|^2 \\
&\le \frac{t_1+2 M^2 \Gamma_k}{J} 
 + \frac{d_0^2}{2 \lam_J}
\end{align*}

% -------------------

% \begin{align*}
%    \phi(y_J) - \phi(x_*)
% &\le \frac{t_1}{J} +\frac{ M^2 \log J}{\mu J}
%  + \frac{1}{2 \lam_J} \|x_*-x_0\|^2 -
% \frac{1}{2 \lam_J^\mu} \|x_*-x_J\|^2 \\
% &\le \frac{t_1}{J} +\frac{ M^2 \log J}{\mu J}
%  + \frac{1}{2 \gamma \mu J} \|x_*-x_0\|^2
% \end{align*}
% since $\lam_J \ge \gamma \mu J$

% ----------------

{\bf 1st choice}
\[
\lam_j = \gamma \frac{j}{\sqrt{J}} \left( 1 + \mu \sqrt{j} \right)  \quad \forall j \ge 1
\]
Using the above definition and the fact that $\tau_j \le 1$, we have
\[
\sqrt{J} \lam_{j+1} = \gamma \sqrt{j+1}\left( 1 + \mu \sqrt{j+1} \right) =
\gamma \sqrt{{\frac{j}{\tau_j}}} 
\left( 1 + \mu \sqrt{{\frac{j}{\tau_j}}} \right)
\le
\frac{1}{\tau_j} \gamma \sqrt{j} 
\left( 1 + \mu \sqrt{j} \right)
= \sqrt{J} \frac{\lam_j}{\tau_j}
\]
and hence that \eqref{eq:cond-cru'} holds.
Also, \eqref{eq:con-comp'} becomes
\begin{align*}
\alpha_j + L &=\frac{1}{\lam_j^\mu(1-\tau_j)} =
\frac{j+1}{\lam_j^\mu} 
= (j+1) \left( \mu + \frac{1}{\lam_j} \right)
\ge  (j+1) \mu +  \frac{j}{\lam_j}
\\
&=  (j+1) \mu +  \frac{\sqrt{J}}{\gamma (1+ \mu \sqrt{j})}
\ge (j+1) \mu +  \frac{\sqrt{J}}{\gamma  (1+ \mu \sqrt{J})}
\end{align*}
Take
\[
\gamma :=\frac{\sqrt{J}}{L(1+\mu \sqrt{J})}
\]
Then
\[
\alpha_j + L \ge 
(j+1) \mu +  \frac{\sqrt{J}}{\gamma  (1+ \mu \sqrt{J})} =
(j+1) \mu + L
\]
and hence that
\[
\alpha_j \ge (j+1) \mu %+ L ( \sqrt{j}-1 )
\]
Also,
\[
\lam_J = \gamma \sqrt{J} \left( 1 + \mu \sqrt{J} \right) = \frac{J}{L}
\]


% \[
% j+1 = \frac{1}{1-\tau_j} \ge \lam_j^\mu(L+\alpha_j) = \frac{\lam_j}{1+\lam_j\mu} (L+\alpha_j)
% \]
% Since $\lam_j/(1+\lam_j\mu) \le 1/\mu$,  a sufficient condition for
% \eqref{eq:con-comp'} is
% \[
% j+1 \ge (L+\alpha_j)/\mu
% \]
% or
% \[
% \alpha_j \le \mu (j+1) -L
% \]


{\bf 2nd choice}
\[
\lam_j = \gamma \sqrt{j} \left( 1 + \mu \sqrt{j} \right)  \quad \forall j \ge 1
\]
Using the above definition and the fact that $\tau_j \le 1$, we have
\[
\lam_{j+1} = \gamma \sqrt{j+1}\left( 1 + \mu \sqrt{j+1} \right) =
\gamma \sqrt{{\frac{j}{\tau_j}}} 
\left( 1 + \mu \sqrt{{\frac{j}{\tau_j}}} \right)
\le
\frac{1}{\tau_j} \gamma \sqrt{j} 
\left( 1 + \mu \sqrt{j} \right)
= \frac{\lam_j}{\tau_j}
\]
and hence that \eqref{eq:cond-cru'} holds. Also, \eqref{eq:con-comp'} becomes
\begin{align*}
\alpha_j + L &=\frac{1}{\lam_j^\mu(1-\tau_j)} =
\frac{j+1}{\lam_j^\mu} 
= (j+1) \left( \mu + \frac{1}{\lam_j} \right)
\ge  (j+1) \mu +  \frac{j}{\lam_j}
\\&=  (j+1) \mu +  \frac{j}{\gamma \sqrt{j} (1+ \mu \sqrt{j})}
\ge
(j+1) \mu +  \frac{\sqrt{j}}{\gamma  (1+ \mu \sqrt{j})} \\
&\ge (j+1) \mu +  \frac{\sqrt{j}}{\gamma  (1+ \mu \sqrt{J})}
\end{align*}
Take
\[
\gamma :=\frac{1}{L(1+\mu \sqrt{J})}
\]
Then
\[
\alpha_j + L \ge 
(j+1) \mu +  \frac{\sqrt{j}}{\gamma  (1+ \mu \sqrt{J})} =
(j+1) \mu + L \sqrt{j}
\]
and hence
\[
\alpha_j \ge (j+1) \mu + L ( \sqrt{j}-1 )
\]
Also,
\[
\lam_J = \gamma \sqrt{J} \left( 1 + \mu \sqrt{J} \right) = \frac{\sqrt{J}}{L}
\]



% -----------

% So, we conclude that
% \[
% J t_J - t_1 \le 2 M^2 \sum_{j=1}^{J-1} \frac{\lam_j^\mu}{j+1} \le \frac{2  M^2}{\mu} \sum_{j=1}^{J-1} \frac{1}{j+1} \approx \frac{ M^2}{\mu} \log J
% \]
% Hence
% \[
% t_J \le \frac{t_1}{J} +
% \frac{ M^2 \log J}{\mu J}
% \]

% Now, choose
% \[
% \lam_j = \gamma \frac{j}{\sqrt{J}} \left( 1 + \mu  \frac{j}{\sqrt{J}}\right)  \quad \forall j \ge 1
% \]


% Assume that $I$ and $C_1$ are such that
% \[
% \alpha= \frac{C_1}{2I} < 1
% \]
% Assume also that $\frac{j-1}{j}$ satisfies
% \[
% \frac{1-\frac{j-1}{j}}{1+\frac{j-1}{j}} = \alpha 
% \]
% or equivalently,
% \[
% \frac{j-1}{j} = \frac{1-\alpha}{1+\alpha}
% \]
% and hence
% \[
% 1-\frac{j-1}{j} = \frac{2\alpha}{1+\alpha}
% \]
% Then
% \[
% \frac{j-1}{j}^I + \frac{1-\frac{j-1}{j}}{1+\frac{j-1}{j}}
% = \frac{j-1}{j}^I + \alpha = \frac{j-1}{j}^I + \frac{C_1}{2I}
% \]
% Now
% \[
% \log (\frac{j-1}{j}^I ) = I \log \frac{j-1}{j}
% \le I(\frac{j-1}{j}-1)
% \]
% So
% if
% \[
% I (\frac{j-1}{j}-1) \le \log \alpha \quad (*) 
% \]
% then
% \[
% \frac{j-1}{j}^I + \frac{1-\frac{j-1}{j}}{1+\frac{j-1}{j}}
% = \frac{j-1}{j}^I + \alpha \le 2 \alpha = \frac{C_1}{I}
% \]
% Now $(*)$ is equivalent to
% \[
% I  \ge \frac{1}{j}^{-1} \log \alpha^{-1}
% = \frac{1+\alpha}{2\alpha} \log \alpha^{-1}
% \]
% Since $\alpha<1$, a sufficient condition for the above to hold is that
% \[
% I  \ge \frac{1}{\alpha} \log \alpha^{-1} = \alpha^{-1} \log \alpha^{-1} = \frac{2I}{C_1} 
% \log \left(\frac{2I}{C_1} \right)
% \]
% or equivalently,
% \[
% 1 \ge \frac{2}{C_1} 
% \log \left(\frac{2I}{C_1} \right)
% \]
% So we can also assume that
% $C_1$ and $I$ satisfy the above condition. There are many ways to satisfy this condition, e.g., $C_1$ and $I$ such that
% \[
% C_1 = 2 \log I, \quad C_1 \ge 2
% \]
% Also
% \begin{align*}
%     t_{I} &\le 2Md_0\frac{j-1}{j}^{I-1}  + \frac{\frac{1}{j}\lam_j M^2}{\frac{j-1}{j}} \\
%     &= \frac{2Md_0\frac{j-1}{j}^{I}  + \frac{1}{j}\lam_j M^2}{\frac{j-1}{j}} \\
%     &\le \frac{2Md_0\alpha  + \frac{1}{j}\lam_j M^2}{\frac{j-1}{j}} \\
%     &= \alpha \frac{2Md_0  + 2(1+\alpha)^{-1}\lam_j M^2}{\frac{j-1}{j}} \\
%     &\le 2\alpha \frac{Md_0  +\lam_j M^2}{\frac{j-1}{j}} \\
%     &\le 2\alpha(1+\alpha) \frac{Md_0  +\lam_j M^2}{1-\alpha}
% \end{align*}
\section{Dual averaging analysis}
\subsection{convex case}








\end{document}