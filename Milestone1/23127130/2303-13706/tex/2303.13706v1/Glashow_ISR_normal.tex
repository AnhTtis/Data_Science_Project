\documentclass[aps,superscriptaddress,long,notitlepage,letterpaper,balancelastpage,nofootinbib,prl,floatfix,twocolumn]{revtex4-1}
\pdfoutput=1

\usepackage{amsmath,mathtools,amssymb,amsthm,amsxtra,overpic,bbm,epsfig,subfigure,url,bm}
\usepackage{hyperref}
\usepackage{mathrsfs}
\usepackage{color,xcolor}
\usepackage{comment}
\usepackage{float}
\usepackage{enumitem}
\usepackage{slashed}
\usepackage{fixmath}

\usepackage[T1]{fontenc}    
\usepackage{lmodern}        
\usepackage{graphicx}
\renewcommand\#{\protect\scalebox{0.8}{\protect\raisebox{0.3ex}{\char"0023}}}
%\newcommand{\overbar}[1]{\mkern 1.5mu\overline{\mkern-1.5mu#1\mkern-1.5mu}\mkern 1.5mu}




\newcommand{\nunubar}{%
\mathrel{\vbox{\offinterlineskip\ialign{%
			\hfil##\hfil\cr
			\textbf{\fontsize{2.4pt}{0pt}\selectfont(---)}\cr
			\noalign{\kern0ex}
			{\scriptsize$\nu$}\cr
}}}}

\newcommand{\nunubartext}{%
\mathrel{\vbox{\offinterlineskip\ialign{%
			\hfil##\hfil\cr
			\textbf{\fontsize{2.8pt}{0pt}\selectfont(---)}\cr
			\noalign{\kern0ex}
			{$\nu$}\cr
}}}}

%\usepackage{ulem}

\newcommand{\prlsection}[2]{{\it\textbf{#1}{#2}}---}
\makeatletter
\newcommand*{\balancecolsandclearpage}{%
\close@column@grid
\cleardoublepage
\twocolumngrid
}
%\def\uwave{\bgroup \markoverwith{\lower3.5\p@\hbox{\sixly \textcolor{red}{\char58}}}\ULon}
%\font\sixly=lasy6 % does not re-load if already loaded, so no memory problem.
\makeatother


\addtolength{\arraycolsep}{-3pt} %make formulae more compact
%\renewcommand{\baselinestretch}{1.15}
\def\thefootnote{\arabic{footnote}}
\setcounter{footnote}{0}
\setlength{\parskip}{0.3mm}
\setlength{\abovedisplayskip}{0pt}
\setlength{\belowdisplayskip}{0pt}
\setlength{\belowcaptionskip}{0pt}
\setlength{\abovecaptionskip}{-0pt}
\setlist{nolistsep}



\definecolor{MyDarkBlue}{rgb}{0.1, 0.1, 0.8} %defining the color 'MyDarkBlue'
\definecolor{SBlue}{rgb}{0.2, 0.4, 0.7} %defining the color 'MyDarkBlue'
\definecolor{MyLightBlue}{rgb}{0.22,0.51,0.9}
\definecolor{MyGreen}{rgb}{0.0, 0.5, 0.0}
\definecolor{BrickRed}{rgb}{0.8, 0.25, 0.33}

\definecolor{nicered}{rgb}{0.5,0.,0.}
\definecolor{nicegreen}{rgb}{0.,0.5,0.}
\definecolor{niceblue}{rgb}{0.,0.,0.5}
\hypersetup{
colorlinks=true,
linkcolor=black,
filecolor=nicegreen,      
urlcolor=niceblue,
citecolor=niceblue,
}

\def\nv#1{\textcolor{MyDarkBlue}{#1}}
\def\gyh#1{\textcolor{BrickRed}{#1}}


%\linespread{1.0}
%\setlength{\columnsep}{4mm}
%\setlength{\footnotesep}{0.2cm}

\renewcommand*{\thefootnote}{\fnsymbol{footnote}}

\interfootnotelinepenalty=10000

%\textwidth = 17.5cm 
%\textheight = 22.8cm 
%\voffset = -5mm 
%\hoffset = -4mm
%\fontdimen3\font=1em% inter word stretch

%%%%%%%%%%%%%%%%%%%%%%%%%%%%%%%%%%%%%%%%%%%%%%%%%%
\begin{document}
\title{\Large Inferring astrophysical neutrino sources from the Glashow resonance}
%high multiplicity processes
%\fontsize{17}{16}\bfseries
\author{\bf Guo-yuan Huang}
\email{guoyuan.huang@mpi-hd.mpg.de}
\affiliation{Max-Planck-Institut f{\"u}r Kernphysik, Saupfercheckweg 1, 69117 Heidelberg, Germany} 
\author{\bf Manfred Lindner}
\email{manfred.lindner@mpi-hd.mpg.de}
\affiliation{Max-Planck-Institut f{\"u}r Kernphysik, Saupfercheckweg 1, 69117 Heidelberg, Germany} 
\author{\bf Nele Volmer}
\email{nele.volmer@mpi-hd.mpg.de}
\affiliation{Max-Planck-Institut f{\"u}r Kernphysik, Saupfercheckweg 1, 69117 Heidelberg, Germany} 


%%%%%%%%%%%%%%%%%%%%%%%%%%
\begin{abstract}
	\noindent
	We infer the ultrahigh energy neutrino source by using the Glashow resonance candidate event recently identified by the IceCube Observatory.
	For the calculation of the cross section for the Glashow resonance, we incorporate both the atomic Doppler broadening effect and initial state radiation $\overline{\nu}^{}_{e} e^- \to W^- \gamma$, which correct the original cross section considerably.
	%
	Using available experimental information, we have set a generic constraint on the $\overline{\nu}^{}_{e}$ fraction of astrophysical neutrinos, which excludes the $\mu$-damped ${\rm p}\gamma$ source around $2\sigma$ confidence level. While a weak preference has been found for the pp source, next-generation measurements will be able to distinguish between ideal pp and p$\gamma$ sources with a high significance assuming an optimistic single power-law neutrino spectrum.
	%We investigate for the first time the initial state radiation of Glashow resonant scattering, i.e., $\overline{\nu}^{}_{e}+e^- \to W^- +\gamma$.
\end{abstract}

\maketitle

\fontdimen1\font=0.0em% inter word Slant
\fontdimen2\font=0.38em% inter word space
\fontdimen3\font=0.1em% inter word stretch
\fontdimen4\font=0.1em% inter word shrink
%%%%%%%%%%%%%%%%%%%%%%%%%%


\section{I. Introduction}
\noindent 
The IceCube Observatory has successfully established the observation of ultrahigh energy (UHE) neutrino flux below a few PeV energies 
~\cite{IceCube:2021rpz,IceCube:2013low,IceCube:2013cdw,IceCube:2018cha,IceCube:2020abv,IceCube:2020wum,IceCube:2022der,Halzen:2022pez}. However, it remains a mystery as to where those neutrinos come from. 
One of the most popular mechanisms rests on the accelerated cosmic rays colliding with  ambient targets around the source~\cite{Gaisser:1994yf,Bhattacharjee:1999mup,Beatty:2009zz,gaisser_engel_resconi_2016,anchordoqui2019ultra,tjus2020closing}.
There is a variety of source models for UHE neutrinos~\cite{Murase:2019tjj,Murase:2022feu,Troitsky:2021nvu,Xing:2011zza} which can usually be classified into the p$\gamma$ and pp types depending on whether the target particle is a photon or a proton.

For both p$\gamma$ and pp sources, after traveling an astronomical distance  the fluxes of three neutrino flavors strongly mix with each other due to neutrino oscillations, which ends up with a nearly democratic flavor composition $\phi^{\oplus}_{\nu_{e}} + \phi^{\oplus}_{\overline{\nu}_{e}} : \phi^{\oplus}_{\nu_{\mu}} + \phi^{\oplus}_{\overline{\nu}_{\mu}} : \phi^{\oplus}_{\nu_{\tau}} + \phi^{\oplus}_{\overline{\nu}_{\tau}} \approx 1:1:1$ at Earth~\footnote[4]{Throughout this work, we use the superscript `$\oplus$' to denote the quantity at Earth and `S' to denote that at source. Note also that we do not assume the presence of sterile neutrinos.}.
%
It is unlikely to disentangle those two sources by traditional flavor ratio measurements~\cite{Mena:2014sja, Chen:2014gxa, Palomares-Ruiz:2015mka, Aartsen:2015ivb, Palladino:2015zua, Arguelles:2015dca, Bustamante:2015waa, Aartsen:2015knd, Brdar:2016thq, DAmico:2017dwq, Pagliaroli:2015rca, Rasmussen:2017ert, Brdar:2018tce, Bustamante:2019sdb, Palladino:2019pid, Stachurska:2019srh,Song:2020nfh}.
The difference between those two sources lies in the composition of neutrinos and antineutrinos.
For the p$\gamma$ neutrino source, cosmic rays collide with photons to produce charged pions (mostly $\pi^+$) followed by the decays $\pi^+ \to \mu^+ \nu^{}_{\mu}$ and $\mu^+ \to e^+ \overline{\nu}^{}_{\mu} \nu^{}_{e}$, which results in more neutrino flux than antineutrino flux, i.e., $\phi^{\rm S}_\nu : \phi^{\rm S}_{\overline{\nu}} = 2:1$. In comparison, the pp source will give rise to nearly equal fractions of $\pi^+$ and $\pi^-$, which leads to $\phi^{\rm S}_\nu : \phi^{\rm S}_{\overline{\nu}} = 1:1$.


The key to distinguishing those two sources is by measuring the $\overline{\nu}^{}_{e}$ fraction $f^{}_{\overline{\nu}_{e}} \equiv \phi^{}_{\overline{\nu}_{e}}/(\phi^{}_{\overline{\nu}_{e}}+\phi^{}_{{\nu}_{e}})$, thanks to the Standard Model process $\overline{\nu}^{}_{e} e^- \to W^- \to X$ predicted by  S.~L.~Glashow~\cite{Glashow:1960zz}. 
Due to the resonance enhancement, the cross section of $\overline{\nu}^{}_{e} e^-$ scattering around $E^{}_{\nu} \approx 6.3~{\rm PeV}$ is larger than that of the deep inelastic scattering (DIS) by more than two orders of magnitude.
%
This promises us an excellent channel to differentiate between the ideal pp (with $f^{\oplus}_{\overline{\nu}_{e}} \approx 0.5$) and p$\gamma$ (with $f^{\oplus}_{\overline{\nu}_{e}} \approx 0.23$) sources, as continuously anticipated in previous works~\cite{Berezinsky:1977sf,Anchordoqui:2004eb,Hummer:2010ai,Xing:2011zm,Bhattacharya:2011qu,Bhattacharya:2012fh,Barger:2012mz,Barger:2014iua,Palladino:2015uoa,Shoemaker:2015qul,Anchordoqui:2016ewn,Kistler:2016ask,Biehl:2016psj,Sahu:2016qet,Huang:2019hgs,Zhou:2020oym}.
In practice, the overall diffuse neutrino flux might be contributed by different types of sources, and the $\overline{\nu}_{e}$ fraction $f^{\oplus}_{\overline{\nu}_{e}}$ can take any reasonable values in between.

Excitingly, with its unprecedented detection volume, the IceCube Observatory has collected one candidate event with an energy deposition $E^{}_{\rm dep} = 6.05 \pm 0.72~{\rm PeV}$ in the sample of partially contained events~\cite{IceCube:2021rpz}.
The probability that this event stems from the Glashow resonance (GR) is high, around $99\,\%$ by using the best-fit neutrino flux taken from Ref.~\cite{IceCube:2015fuw}.





In this letter, a timely quantitative assessment is carried out to infer the $\overline{\nu}^{}_{e}$ fraction by taking $f^{\oplus}_{\overline{\nu}_{e}} $ as a free parameter and to explain the level that we can differentiate between p$\gamma$ and pp sources. 
We have included both the radiation of initial photons~\cite{Gauld:2019pgt,Garcia:2020jwr} and the Doppler broadening effect~\cite{Glashow:2014} while calculating the GR events.
%
Using the updated cross section, we investigate both the results for the current GR candidate in IceCube as well as the prospects of next-generation experiments.
%In the remaining part of this letter, we further explain the calculation of the GR cross section and our analysis framework.


%




\section{II. A full treatment of Glashow resonance}
\noindent 
%
As more and more UHE neutrino data have been accumulated,
it becomes increasingly important to take into account the subleading effects for the theoretical evaluation of the GR. 
There are mainly two effects that should be emphasized: (i) the initial state radiation (ISR)~\cite{Gauld:2019pgt,Garcia:2020jwr}; (ii) the Doppler broadening effect~\cite{Glashow:2014}.
%
At the leading level, the 
cross section for the process $\overline{\nu}_e e^- \to W^- \to X$ reads \cite{IceCube:2021rpz}
\begin{equation}
\label{eq:sigmaGlashow}
\sigma^{(0)}(s)= 24 \pi {\rm \Gamma}^{2}_W {\rm Br}^{}_{W^-\to \overline{\nu}_e e^-}\frac{s/M_W^2}{(s-M_W^2)^2+{\rm \Gamma}_W^2M_W^2} \;,
\end{equation}
where $M_W \approx 80.433\, \text{GeV}$ is the mass of the $W$ boson, ${\rm \Gamma}_W \approx 2.09~{\rm GeV}$ is the total decay width and ${\rm Br}^{}_{W^-\to \overline{\nu}_e e^-} \approx 10.7\,\%$ is the branching ratio of the channel $W^-\to \overline{\nu}_e e^-$. 
The ISR and the Doppler broadening effect are found to considerably modify the above picture and should be included for completion.


Let us start with the ISR.
This effect becomes increasingly notable when the center-of-mass (COM) energy is much higher than the mass of the initial charged lepton, for which the collinear emission of photons is significant. 
For instance, in the Large Electron-Positron Collider (LEP), the ISR should be taken into account when analyzing the $Z$ boson peak~\cite{ALEPH:2005ab}.
For UHE neutrino telescopes like IceCube, the ISR cross section  near the GR will receive a large enhancement factor of  $\ln(M^{}_{W}/m^{}_{e}) \approx 12$ on top of the fine structure constant $\alpha$.


The ISR can be consistently included by using the structure function approach in analogy with the DIS off hadrons. The modified cross section will be \cite{Garcia:2020jwr}
\begin{equation}
\sigma (E^{}_{\nu})=\int \mathrm{d} x \, {\rm \Gamma}^{}_{e/e}(x,Q^2)\sigma^{(0)}(x,Q^2, E^{}_{\nu}) \;,
\end{equation}
where $Q$ represents the energy scale, $x$ is the longitudinal momentum fraction of the electron after the photon radiation, $\sigma^{(0)}$ is the cross section without the initial-state photon, and ${\rm \Gamma}^{}_{e/e}$ is the structure function of the electron.
We take the structure function from Ref.~\cite{Cacciari:1992} which includes soft photons resummed to all orders and hard photons up to $\mathcal{O}(\alpha^3)$. 







The second effect of interest is the Doppler broadening due to the motion of atomic electrons~\cite{Glashow:2014}. 
The velocity of atomic electrons $\beta$ is typically of the order $\mathcal{O}(\alpha\, c)$.
A simple estimation shows that this velocity will shift the COM energy square from $s = 2 E^{}_{\nu} m^{}_{e}$ to $2 E^{}_{\nu} m^{}_{e}(1-\beta\cos \theta)$, where $\theta$ is the angle between the electron velocity and the incoming neutrino in the laboratory frame. This broadens the COM energy by around $0.6~{\rm GeV}$ in comparison to the $W$ decay width ${\rm \Gamma}_W=~2.09~{\rm GeV}$.
Non-relativistic electrons in the atom have the four-momentum
$
( m_e+{\left|\mathbold{k}\right|^2}/({2m_e}),\mathbold{k})
$,
where $\left|\mathbold{k}\right| \approx m_e \beta$. By integrating over the electron wave function, one can arrive at the total cross section~\cite{Glashow:2014}
\begin{equation} 
\label{eq:integrationDoppler}
\sigma (E^{}_{\nu})= \frac{1}{4 \pi}\int \mathrm{d}\phi \int \mathrm{d}\beta F(\beta) \int \mathrm{d} x^\prime \sigma^{(0)}[E^{}_{\nu}(1-\beta x^\prime)] \;,
\end{equation}
where $\phi$ represents the azimuth angle, $F(\beta)$ is the velocity distribution of electrons and  $x^\prime=\cos (\theta )$.
Since the calculation framework was already outlined in Ref.~\cite{Glashow:2014}, we give more details about the updated calculation in the Appendix.

Those two effects can be combined, and their joint result  is shown as the red curve in Fig.~\ref{fig:ComparisonAll} for the ${\rm H_2 O}$  target, along with the cross sections without (solid black curve) or modified by only one  (blue and orange curves) of those effects. In comparison, the charged-current (CC) and neutral-current (NC) interactions are depicted as dashed and dotted black curves, respectively.
%
Some remarks on the results are given below.
\begin{itemize}[noitemsep,topsep=0pt,leftmargin=5.5mm]
	\item The ISR will reduce the peak at the resonance energy $E^{}_{\nu} \approx 6.3~{\rm PeV}$ by almost $20\,\%$. Furthermore, the cross section above the resonance energy is enhanced by a factor of more than two. This is due to the radiative return phenomenon, for which the photon in the process $\overline{\nu}^{}_{e} e^- \to W^- \gamma$ carries away some energy such that the $W$ production will be made on shell even if~$\sqrt{s} > M^{}_{W}$.
	\item The Doppler broadening effect for the ${\rm H_2 O}$  target is small compared to the ISR in the logarithmic scale.
	To see the detailed impact we also show the result in a flat scale as Fig.~\ref{fig:DopplerBroadened} of our Appendix. 
	%
	The resonance peak is reduced slightly, while the width is broadened due to the motion of atomic electrons.
	%
	\item The combined result of the ISR and the Doppler broadening is obtained with a convolution, which reduces the peak by around $30\,\%$. However, we should note that those effects will be partly smeared by the finite energy resolution of the IceCube detector. We have checked that the eventual effect can decrease the events within the energy window near the GR by almost $10\%$.
\end{itemize}
With the full GR cross section, we are able to calculate the event rate in IceCube and compare it to both experimental data available now and those from future experiments.




\begin{figure} 
	\centering
	\includegraphics[width=0.45\textwidth]{xsec.pdf}
	\caption{Cross section for the Glashow resonance process $\overline{\nu}_e+e^- \rightarrow W^- \rightarrow X$ with and without initial state radiation and Doppler broadening effects. The black curve shows the cross section without initial state radiation and Doppler broadening, the blue dotted one includes initial state radiation and the orange dotted one includes Doppler broadening. The red curve is the cross section with both Doppler broadening and initial state radiation effects, and the tabulated result of this curve is given in our supplemental data. Both the broadening and the radiative return are visible. For the Glashow resonance curves we averaged over the electrons in H$_2$O for the target. The charged-current (black dashed) and neutral-current (black dotted) cross sections are shown as well. For these we assumed an isoscalar nucleon target.}
	\label{fig:ComparisonAll}
\end{figure}


\section{III. Analysis framework} 
\noindent
In order to constrain the $\overline{\nu}^{}_{e}$ fraction in the total diffuse neutrino flux, we calculate the likelihood by fitting models with different values of $f^{}_{\overline{\nu}_{e}} = \phi^{}_{\overline{\nu}_{e}}/(\phi^{}_{\overline{\nu}_{e}}+\phi^{}_{{\nu}_{e}})$ to the available IceCube data.
The reason why we use $f^{}_{\overline{\nu}_{e}}$ to measure the $\overline{\nu}^{}_{e}$ fraction is that it almost solely determines the spectrum of single cascade event topology at PeV energies in the IceCube detector.

%Events in IceCube and why this statistical framework.

The observed GR candidate in IceCube belongs to the PeV energy partially contained events (PEPEs), in comparison to the high energy starting events (HESEs) where the shower is fully contained inside the fiducial volume.
Even though the PEPE effective volume is nearly twice the volume of HESE at PeV energies
only one event with an energy deposition $E^{}_{\rm dep} = 6.05 \pm 0.72~{\rm PeV}$ has been observed within the energy window $4~{\rm PeV} < E^{}_{\rm dep} < 10~{\rm PeV}$.
For HESE three PeV events have been collected~\cite{IceCube:2014stg}, nicknamed Bert, Ernie and Big Bird. However, all of them have energies below $3~{\rm PeV}$, which are most likely contributed by the DIS.
Even though the GR has not significantly arisen in the HESE sample, HESE is useful to fix the normalization and shape of UHE neutrino flux which are crucial for our extraction of the $\overline{\nu}^{}_{e}$ fraction.

In Ref.~\cite{IceCube:2020wum}, the IceCube Collaboration has analyzed the overall UHE neutrino flux with HESEs collected over 7.5 years, assuming a flavor ratio $\phi^{\oplus}_{\nu_{e}} + \phi^{\oplus}_{\overline{\nu}_{e}} : \phi^{\oplus}_{\nu_{\mu}} + \phi^{\oplus}_{\overline{\nu}_{\mu}} : \phi^{\oplus}_{\nu_{\tau}} + \phi^{\oplus}_{\overline{\nu}_{\tau}} = 1:1:1$. 
%
During our analysis we will use the HESE results including uncertainties from Ref.~\cite{IceCube:2020wum} to set the spectrum of neutrino flux and use  PEPE to extract the $\overline{\nu}^{}_{e}$ fraction $f^{\oplus}_{\overline{\nu}_{e}}$.
%
Note that a more thorough analysis would assume a completely free flavor ratio. However, on the one hand, the latest IceCube HESE fit available has fixed the flavor ratio. On the other hand, ideal pp and p$\gamma$ astrophysical models reasonably prefer such a democratic ratio after neutrino oscillations over an astronomical distance. 

For demonstration, we choose two benchmark flux models in our analysis: (i) the unbroken single power-law model; (ii) the single power-law model with an exponential energy cutoff. The former one reads
\begin{align} \label{eq:dPhidE}
\frac{\mathrm{d} {\rm \Phi}^{}_{6\nu}}{\mathrm{d} E^{}_{\nu}} = {\rm \Phi}^{}_{0} \left(\frac{E^{}_{\nu}}{100~{\rm TeV}}\right)^{-\gamma}10^{-18}~ {\rm GeV^{-1} cm^{-2} s^{-1} sr^{-1}} \;,
\end{align}
which represents models consistent with the Fermi acceleration mechanism and extends to infinite energies.
In practice, the reachable energy of astrophysical accelerators always features a cutoff due to the Hillas criterion~\cite{Hillas:1984}.
For the cutoff model, the flux in Eq.~(\ref{eq:dPhidE}) will be multiplied by a suppression factor  $\exp\left({-E^{}_{\nu}/E^{}_{\rm cutoff}}\right)$.
To confine the flux parameters, we construct a likelihood based on the results in Ref.~\cite{IceCube:2020wum}:
\begin{align} \label{eq:}
-2\ln{\mathcal{L}^{}_{6\nu}} = \frac{({\rm \Phi^{}_{0}}- {\rm \Phi}^{\rm bf}_{0})^2}{ \sigma({\rm \Phi^{}_{0}})^2} + \frac{(\gamma- \gamma^{\rm bf})^2}{ \sigma(\gamma)^2} \;,
\end{align}
with the best-fit values ${\rm \Phi}^{\rm bf}_{0} = 6.37$ and $\gamma^{\rm bf} = 2.87$, as well as the $1\sigma$ errors $\sigma({\rm \Phi^{}_{0}}) = 1.54$ and $\sigma(\gamma) = 0.2$.
For the cutoff model we further derive the likelihood for $E^{}_{\rm cutoff}$ from Fig.~ VI.9 of Ref.~\cite{IceCube:2020wum} where the test-statistic has been marginalized.
Note that in this case we have ignored possible correlations among ${\rm \Phi}^{}_{0}$, $\gamma$ and $E^{}_{\rm cutoff}$, which are not provided. Nevertheless, such a choice will be more conservative because less information is utilized in our analysis.

After the prior knowledge of $\{{\rm \Phi}^{}_{0}, \gamma(, E^{}_{\rm cutoff})\}$ has been established by HESE, we continue with fitting $f^{\oplus}_{\overline{\nu}_e}$ to PEPE. 
The task is to calculate the likelihood $\mathcal{L}^{}_{\overline{\nu}_{e}} (f^{\oplus}_{\overline{\nu}_e})$ with the  GR candidate we have.
The joint likelihood can then be obtained with $\mathcal{L}^{}_{\rm tot}=\mathcal{L}^{}_{6\nu} \times \mathcal{L}^{}_{\overline{\nu}_{e}}$ for the parameter set ${\rm \Theta} \equiv \{ {\rm \Phi}^{}_{0}, \gamma(, E^{}_{\rm cutoff}), f^{\oplus}_{\overline{\nu}_e}\}$.
%
In the frame of extended likelihood analysis of unbinned data (see e.g. Chapter 6 of Ref.~\cite{Cowan:1998ji})
the likelihood is calculated with
\begin{align} \label{eq:Lnuebar}
\mathcal{L}^{}_{\overline{\nu}_{e}} =  &  \prod_{i=1}^{n} \left[ \mu^{}_{\rm DIS}{P}^{}_{\rm DIS}(\# {i}| {\rm \Theta}) + \mu^{}_{\rm GR}{P}^{}_{\rm GR}(\# {i}| {\rm \Theta})\right]  \notag\\
& \times \frac{1}{n!} \mathrm{e}^{-(\mu^{}_{\rm DIS}+ \mu^{}_{\rm GR})}\;,
\end{align}
where $\mu^{}_{\rm DIS}$ and $\mu^{}_{\rm GR}$ are the expected event numbers within the energy window $E^{}_{\rm dep} \in [4, 10]~{\rm PeV}$ for the DIS and the GR, respectively, and $\# i$ represents in general all possible GR candidates. Moreover, ${P}^{}_{\rm DIS/GR}(\# {i}| {\rm \Theta})$ is the normalized probability to have an event at $\# {i}$'s energy for the given model parameter set ${\rm \Theta}$.
Since there is only one GR candidate so far we have $n = 1$ in Eq.~(\ref{eq:Lnuebar}).

The expected event numbers $\mu^{}_{\rm DIS}$ and $\mu^{}_{\rm GR}$ can be obtained by integrating the flux and cross sections with the detector configuration.
%
The differential event distribution for the reaction type $r$ (DIS-CC, DIS-NC or GR) as a function of the energy deposition  $E^{}_{\rm dep}$ reads~\cite{Palomares-Ruiz:2015mka}
\begin{eqnarray} \label{eq:dNdE}
\frac{\mathrm{d} N_{{\nunubar}_{\! \alpha}}^{r}}{\mathrm{d} E^{}_{\rm dep}} & = & T^{}_{\rm IC}\cdot  N^{}_{\rm A} \int_{E^{}_{\rm min}}^\infty \mathrm{d} E^{}_{\nu} \frac{\mathrm{d} {\rm \Phi}_{{\nunubar}_{\! \alpha}}^{\rm IC}}{\mathrm{d} E^{}_{\nu}} \times  \\  & &\int^1_0 \mathrm{d} y \frac{\mathrm{d} \sigma^{r}_{{{\nunubar}_{\! \alpha}}}(E^{}_{\nu})}{\mathrm{d} y} \frac{\mathrm{d}P^{}_{}(E^{}_{\rm sh})}{\mathrm{d}E^{}_{\rm dep}}   M^{}_{\rm eff}(E^{}_{\rm sh}) \;,\notag
\end{eqnarray}
where $T^{}_{\rm IC}$ is the collection time of IceCube, $N^{}_{\rm A}$ is the Avogadro constant, ${\mathrm{d} {\rm \Phi}_{{\nunubar}_{\! \alpha}}^{\rm IC}}/{\mathrm{d} E^{}_{\nu}}$ is the ${{\nunubartext}_{\! \alpha}}$ flux
integrated over the incoming direction in IceCube, and the cross section is averaged over the nucleon number.
We have computed the neutrino flux in IceCube numerically by adopting the preliminary reference earth model as the density profile of Earth without uncertainties~\cite{Dziewonski:1981xy}.
Here, $M^{}_{\rm eff}$ is the effective target mass of IceCube which is available for HESE from Ref.~\cite{IceCube:2013low}.
For PEPE we scale the effective target mass according to  Monte-Carlo results of the effective area in Ref.~\cite{Lu:2017nti} which is approximately doubled with respect to HESE.
%  
Moreover, $y$ specifies the energy fraction distributed in one of the two final-state fermions, which typically reads $1-E^{}_{\ell}/E^{}_{\nu}$ for the CC interaction with $\ell$ being the charged lepton. 
%The p.d.f. of the energy deposition $\mathcal{P} (y)$ depends on the interaction type.
%For the charged-current DIS initiated by a $\nu^{}_{e}$ as well as the GR with hadronic final states, 
%almost all the energy of final states enters into the cascade.
Besides the primary neutrino energy $E^{}_{\nu}$, there are the other two energy quantities: $E^{}_{\rm sh} (E^{}_{\nu}, y)$ for the charged final-state particles initiating the shower and $E^{}_{\rm dep}$ for the energy deposition. Because neutral particles during the shower development carry away a small fraction of energies, $E^{}_{\rm dep}$ is in general smaller than  $E^{}_{\rm sh}$ and satisfies a probability distribution ${\mathrm{d}P^{}_{}(E^{}_{\rm sh})}/{\mathrm{d}E^{}_{\rm dep}}$.
%
%
Following Appendix A of Ref.~\cite{Palladino:2018evm}, we take ${\mathrm{d}P^{}_{}(E^{}_{\rm sh})}/{\mathrm{d}E^{}_{\rm dep}} $ as a Gaussian distribution
\begin{align}
\frac{\mathrm{d}P^{}_{}(E^{}_{\rm sh})}{\mathrm{d}E^{}_{\rm dep}} = N\, \mathrm{exp}\left[{-\frac{(E^{}_{\rm dep}-r E^{}_{\rm sh})^2}{2 (E^{}_{\rm sh}{\rm \Delta})^2}}\right] {\rm \Theta}(E^{}_{\rm sh} - E^{}_{\rm dep}) \;,
\end{align}
where $N$ is the normalization factor, $r = 0.95$, ${\rm \Delta} = 0.06$ and ${\rm \Theta} (x)$ is the Heaviside step function.
%Once the differential event distributions in Eq.~(\ref{eq:dNdE}) are calculated, we are ready to compare models to data.
Eq.~(\ref{eq:dNdE}) does not apply to the case of $\nu^{}_{\tau}$ CC interaction. The outgoing tau decays either hadronically or leptonically, for which the energy distribution of visible decay products must be taken into account. In principle, a $6.3$~PeV $\nu^{}_{\tau}$ CC event can easily lead to a distinctive double-cascade signature~\cite{Learned:1994wg,IceCube:2020abv} and hence can be separated from the $\nu^{}_{e}$ cascade event. 

Given the differential distributions in Eq.~(\ref{eq:dNdE}), the expected event numbers $\mu^{}_{\rm dis}$ and $\mu^{}_{\rm gr}$
simply read
\begin{align}
\mu^{}_{\rm DIS} = & \int^{}_{\rm cut} \mathrm{d} E^{}_{\rm dep}
\cdot \left(\frac{\mathrm{d} N_{{\nu}_{e}+\overline{\nu}_{e}}^{\rm CC}}{\mathrm{d} E^{}_{\rm dep}} + \sum^{}_{\alpha}  \frac{\mathrm{d} N_{{\nu}_{\alpha} + \overline{\nu}_{\alpha}}^{\rm NC}  }{\mathrm{d} E^{}_{\rm dep}}  \right) \;, \\
\mu^{}_{\rm GR} = & \int^{}_{\rm cut} \mathrm{d} E^{}_{\rm dep}
\cdot \left(\frac{\mathrm{d} N_{\overline{\nu}_{e}}^{{\rm GR},jj}}{\mathrm{d} E^{}_{\rm dep}} + \frac{\mathrm{d} N_{\overline{\nu}_{e}}^{{\rm GR},e\nu}}{\mathrm{d} E^{}_{\rm dep}} \right) \;.
\end{align}
%Since no information is available  for the tauness of the PEPE shower yet, we include the tau flavor contribution into the background $\mu^{}_{\rm DIS}$ to be conservative. 
The event cut is $E^{}_{\rm dep} \in [4, 10]~{\rm PeV}$ as we mentioned before.


Finally, let us describe how to obtain ${P}^{}_{\rm DIS/GR}(\# {i}| {\rm \Theta})$ in Eq.~(\ref{eq:Lnuebar}).
%
In practice, the energy deposition of the event $\# i$ is distributed over a range with the probability function $P(\# i| E^{}_{\rm dep})$ due to the limited energy resolution.
For the GR candidate, we take this probability function as the posterior distribution
from Fig.~3a of Ref.~\cite{IceCube:2021rpz}. 
A convolution over $E^{}_{\rm dep}$ is required to get ${P}^{}_{\rm DIS/GR}(\# {i}| {\rm \Theta})$, namely
\begin{align}
{P}^{}_{\rm DIS/GR}(\# i| {\rm \Theta}) = \int \mathrm{d} E^{}_{\rm dep}  {P}^{}_{}(\# i | E^{}_{\rm dep}) f^{}_{\rm DIS/GR}(E^{}_{\rm dep}| {\rm \Theta}) \;.
\end{align}
Here, $f^{}_{\rm DIS/GR}$ is the probability density function of events normalized within the cut, which can be directly obtained from Eq.~(\ref{eq:dNdE}) with normalization.

\section{IV. Results} 
\noindent
With the framework  above, we can compute the total likelihood $\mathcal{L}^{}_{\rm tot}$ as a function of the parameter set $\{ {\rm \Phi}^{}_{0}, \gamma(, E^{}_{\rm cutoff}), f^{\oplus}_{\overline{\nu}_e}\}$. 
The likelihood can then be used for either frequentist or Bayesian interpretations. 
For the frequentist interpretation, we 
obtain the likelihood maximum $\mathcal{L}^{\rm max}_{\rm tot} (f^{\oplus}_{\overline{\nu}_e})$
by marginalizing over the other parameters.
For the Bayesian interpretation, we need to derive the posterior distribution of $f^{\oplus}_{\overline{\nu}_e}$ by integrating over  the likelihood and priors. We choose flat priors on ${\rm \Phi}^{}_{0}$, $\gamma$, $f^{\oplus}_{\overline{\nu}_e}$ and $\ln{E^{}_{\rm cutoff}}$ for illustration.

\begin{figure}[t!]
	\centering
	%\hspace{-1cm}
\includegraphics[width=0.45\textwidth]{LH.pdf}
	\caption{The likelihood (in blue) or posterior (in brown) of the $\overline{\nu}^{}_{e}$ fraction $f^{\oplus}_{\overline{\nu}_e} $ inferred from the Glashow resonance event in IceCube with $4.6$ years of data taking. The upper panel assumes a single power-law flux model with central values and uncertainties from Ref.~\cite{IceCube:2020wum}, while the lower one has incorporated an exponential cutoff $E^{}_{\rm cutoff}$ in the neutrino spectrum. The expected $\overline{\nu}^{}_{e}$ fractions of three representative ultrahigh energy neutrino source models, including the ideal pp  ($f^{\oplus}_{\overline{\nu}_e} \approx 0.5$), the p$\gamma$  ($f^{\oplus}_{\overline{\nu}_e} \approx 0.23$) and the $\mu$-damped p$\gamma$ ($f^{\oplus}_{\overline{\nu}_e} \approx 0$) sources, are indicated by the vertical lines. The sensitivity of the future IceCube-Gen2 project with an effective exposure of ten (fifty) years is shown as the dashed (dotted) blue curves, assuming that the pp source is dominant with ${\rm \Phi}^{}_{0} = 6.37$, $\gamma = 2.7$ and $E^{}_{\rm cutoff} = 5~{\rm PeV}$.}
	\label{fig:Lnuebar}
\end{figure}






Our main results are given in Fig.~\ref{fig:Lnuebar}, which shows the likelihood function (in blue) or posterior distribution (in brown) of the $\overline{\nu}^{}_{e}$ fraction $f^{\oplus}_{\overline{\nu}_{e}} $ inferred from the IceCube 4.6-year data. Uncertainties from neutrino flux parameters have been systematically included and marginalized when we constrain $f^{\oplus}_{\overline{\nu}_{e}}$. 
The upper and lower panels stand for the assumptions of an unbroken single power-law flux model and a single power-law model with a varying exponential energy cutoff, respectively~\cite{IceCube:2020wum}.

For blue curves, the horizontal lines with $-2\ln{\mathcal{L}} = 1$ and $4$ roughly set the $1\sigma$ and $2\sigma$ confidence levels, respectively. For brown regions, the $1\sigma$ and $2\sigma$ credible intervals have been covered from dark to light colors. 
We find that for all cases, the $\mu$-damped p$\gamma$ source with $f^{\oplus}_{\overline{\nu}_{e}} \approx 0$ is  excluded by around $2\sigma$ level. The current IceCube 4.6-year data weakly favor the pp source but are not able to exclude the ideal p$\gamma$ source considerably (only at $1\sigma$ or so); see the vertical lines. 


%
\begin{figure}[t!]
	\centering 
	%\hspace{-1cm}
 \includegraphics[width=0.42\textwidth]{chi2Ecut.pdf}
	\caption{The log-likelihood of the energy cutoff $E^{}_{\rm cutoff}$. The dashed curve is taken from Fig.~ VI.9 of Ref.~\cite{IceCube:2020wum}, while the solid curve is derived from the Glashow resonance candidate event by marginalizing over the other model parameters. }
	\label{fig:lhEcut}
\end{figure}	

A few additional remarks are made below.
First, the difference between the cases of the single power law and that with an energy cutoff is understandable, because one additional parameter $E^{}_{\rm cutoff}$ is involved and will dilute the information on $f^{\oplus}_{\overline{\nu}_{e}}$. 
Second, the prediction for $f^{\oplus}_{\overline{\nu}_{e}}$ from different source models (vertical lines) is made by using the best-fit values of neutrino mixing parameters~\cite{Esteban:2020cvm}. Those include the $\mu$-damped p$\gamma$ source (no $\overline{\nu}$) with $f^{\oplus}_{\overline{\nu}_{e}}=0$, the ideal p$\gamma$ source with  $f^{\oplus}_{\overline{\nu}_{e}}=0.22$ and the pp source with $f^{\oplus}_{\overline{\nu}_{e}}=0.5$. 





Last but not least we should emphasize that the GR event can also constrain the possible energy cutoff $E^{}_{\rm cutoff}$ in the neutrino spectrum. The original best-fit value of $E^{}_{\rm cutoff}$ without GR is around $5~{\rm PeV}$ in Ref.~\cite{IceCube:2020wum}, with a $2\sigma$ lower boundary at $0.5~{\rm PeV}$. The presence of the GR candidate event will push the $2\sigma$ lower boundary to $2.2~{\rm PeV}$, as illustrated in Fig.~\ref{fig:lhEcut}.
%

%[Some comments on the astrophysical models]





\section{V. Outlook}
\noindent 
Using the recent GR candidate event identified by IceCube, we have performed an analysis to infer the $\overline{\nu}^{}_{e}$ content in UHE astrophysical neutrinos.
We treat the $\overline{\nu}^{}_{e}$ fraction as a free parameter and have set a generic constraint on it by including the uncertainties in the UHE neutrino flux. From the candidate event measured so far, we find a weak preference for the pp source.
The situation will be greatly improved by the upcoming next-generation neutrino telescopes.
%In the single power-law model with an energy cutoff, we find that the GR event can considerably improve the current constraints on $E^{}_{\rm cutoff}$.

In the future, there are many projects such as IceCube-Gen2~\cite{IceCube:2014gqr,IceCube-Gen2:2020qha}, Baikal-GVD~\cite{Baikal-GVD:2018isr}, KM3NeT~\cite{KM3Net:2016zxf}, P-ONE~\cite{P-ONE:2020ljt}, TAMBO~\cite{Romero-Wolf:2020pzh}, TRIDENT~\cite{Ye:2022vbk} and so on, which will provide very valuable sensitivities to PeV astrophysical neutrinos~\cite{Huang:2021mki,Coleman:2022abf,Ackermann:2022rqc,Valera:2022wmu}. 
We take IceCube-Gen2 for demonstration by rescaling the current IceCube target mass by ten times, and perform a count analysis in the energy window of $[4, 10]~{\rm PeV}$.
The sensitivity for ten (fifty) years of effective exposure is shown as the dashed (dotted) curves in Fig.~\ref{fig:Lnuebar}. Because the flux parameters $\{ {\rm \Phi}^{}_{0}, \gamma(, E^{}_{\rm cutoff})\}$  can be very precisely determined in the future~\cite{IceCube-Gen2:2020qha}, we choose a reasonably optimistic spectrum as ${\rm \Phi}^{}_{0} = 6.37$, $\gamma = 2.7$ and $E^{}_{\rm cutoff} = 5~{\rm PeV}$ in making the forecast; see Fig.~16 of Ref.~\cite{IceCube-Gen2:2020qha} for example. 

Assuming the pp type as the true source, i.e., $f^{\oplus}_{\overline{\nu}_{e}}=0.5$, we expect eleven GR events in IceCube-Gen2 with ten years of exposure for the best-fit single power-law model. If we take an exponential cutoff $E^{}_{\rm cutoff} = 5~{\rm PeV}$ in the spectrum, the event expectation would be reduced to three. The expected number of events is still diverse due to low statistics of events at PeV energies. 
For the best-fit single power-law model, IceCube-Gen2 with ten years of exposure can already differentiate pp  from p$\gamma$ source with a $2\sigma$ confidence level. However, if there is an exponential cutoff at $5$ PeV, an effective exposure of fifty years would be required to reach the $2\sigma$ level.
Those results can also be applied to other telescopes by adjusting the effective exposure.
%
By measuring the spectrum precisely in the future, one may go beyond the assumptions of single power-law flux model (with cutoff) and take the spectrum with a general energy dependence.
%

The hybrid cascade and early muon reconstruction in IceCube can already greatly improve the angular resolution of the GR shower.
In case of the increased statistics, GR events detected in future experiments can also be used to produce a map of the sky and identify associated PeVatrons~\cite{	LHAASO:2021cbz,LHAASO_nature,Sudoh:2022sdk}. Our main point is that knowledge about neutrino sources will be significantly improved by those upcoming facilities with large statistics, which also guarantees a robust frontier for possible new physics studies~\cite{Bustamante:2020niz,Jezo:2014kla,Babu:2019vff,Dey:2020fbx,Babu:2022fje,Xu:2022svm,Arguelles:2022xxa,Huang:2022pce,Huang:2022ebg,Heighton:2023qpg}. 


%However, for this the expected event numbers for IceCube-Gen2 for the two flux models we used are not sufficient. It is possible that some sources are more active in the following years, which would lead to higher statistics and hence a better signal. Thus it might be possible to identify very active sources that lead to a high flux of $\Bar{\nu}_e$ during the runtime of IceCube-Gen2.


\section{Acknowledgments} \noindent
GYH is supported in part by the Alexander von Humboldt Foundation.
%Due to the extended effective volume, PEPE features nearly twice the detection volume of HESE at PeV energies. Indeed, there are nine events that have passed the selection criteria of PEPE, among which only one event has an energy higher than $4~{\rm PeV}$. The remaining PEPE events all have energies below $2~{\rm PeV}$ and were not officially announced  in detail. 
%



%Prior choice
%In a simple power-law model, free parameters describing the astrophysical neutrino flux in our fit are ${\rm \Theta} = \{ {\rm \Phi}^{}_{0}, \gamma, f^{}_{\overline{\nu}_e}\}$, including the flux normalization ${\rm \Phi}^{}_{0}$, the spectral index $\gamma$, neutrino flavor ratios $f_{}$. For this model, there are five degrees of freedom in total as flavor ratios amount to one.
%In principle, an analysis taking into account the complete  IceCube datasets (both below and above PeV energies) is needed to fix those model parameters. 
%In fact, various analyses using the full spectrum have already been performed by IceCube to fit the parameters $\{ {\rm \Phi}^{}_{0}, \gamma,  f^{}_{\overline{\nu}_e + \nu_e},f^{}_{\overline{\nu}_\mu + \nu_\mu},f^{}_{\overline{\nu}_\tau + \nu_\tau}\}$ without, however, discriminating $\overline{\nu}^{}_{e}$ from others.
%
%The information of $\overline{\nu}^{}_{e}$ fraction is mostly encoded in the multi-PeV GR events, whose importance for astrophysical neutrino sources is widely recognized.
%
%However, note that during the IceCube fit~\cite{IceCube:2020abv,IceCube:2020wum}, values and uncertainties of $\{ {\rm \Phi}^{}_{0}, \gamma,  f^{}_{\overline{\nu}_e + \nu_e},f^{}_{\overline{\nu}_\mu + \nu_\mu},f^{}_{\overline{\nu}_\tau + \nu_\tau}\}$ are dominated by those events {\it below} the PeV energy scale.
%


%









%We have the following considerations.
%
%First, the visible energy distribution of this event including the uncertainties from ice and the detector is already shown, as Fig.~3a of Ref.~\cite{IceCube:2021rpz}. Second, we also need to collect other events around this energy range (i.e., set a cut 1~PeV to 10~PeV for HESE, 4~PeV to 8~PeV for PEPE; there are four such cascades fully contained in the detector if I remember correctly), which are suspected to be candidates originated from the GR.
%
%Third, we adopt the likelihood ratio test following the appendix (the p-value section) of Ref.~\cite{IceCube:2021rpz}. 






%For other scenarios of cascade (e.g., neutral-current DIS and GR with $W \to \nu^{}_{e} e$), the invisible neutrino in the final state dominates the depletion of energy deposition, \gyh{for which we obtain $\mathcal{P} (y)$ by calculating the differential cross section (need to work out and give in the appendix if possible).}




%






%

%$\mathcal{P}^{}_{}({\rm E}^{}_i | E^{}_{\nu})$ is not the same as $\mathcal{P}^{}_{}({\rm E}^{}_i | E^{}_{\rm vis})$, but they are relate as follows.
%Note that not all energy of the initial neutrino is deposited (visible) in the detector, i.e., $E^{}_{\rm vis} < E^{}_{\nu}$. A p.d.f. $\mathcal{P} (y)$ can be defined with $y \equiv E^{}_{\rm vis}/E^{}_{\nu}$ being the fraction of energy deposited in the detector. To obtain likelihood $\mathcal{P}^{}_{}({\rm E}^{}_i | E^{}_{\nu})$, a further convolution of $\mathcal{P}^{}_{}({\rm E}^{}_i | E^{}_{\rm vis}) $ with $\mathcal{P} (y)$ should be performed.
%Hence, $\mathcal{P}^{}_{}({\rm E}^{}_i | E^{}_{\nu})$ for different PeV events are obtained with
%\begin{eqnarray}
%\mathcal{P}^{}_{}({\rm E}^{}_i | E^{}_{\nu}) = \int \mathrm{d} y \mathcal{P}^{}_{}({\rm E}^{}_i | E^{}_{\rm vis})  \mathcal{P} (y) \;,
%\end{eqnarray}
%where $ y \equiv E^{}_{\rm vis}/E^{}_{\nu}$ is the fraction of energy deposited in the detector. 

%The p.d.f. of the energy deposition $\mathcal{P} (y)$ depends on the interaction type.
%For the charged-current DIS initiated by a $\nu^{}_{e}$ as well as the GR with hadronic final states, 
%almost all the energy of final states enters into the cascade.
%For them, following Appendix A of Ref.~\cite{Palladino:2018evm}, $\mathcal{P} (y)$ takes a Gaussian form,
%\begin{eqnarray}
%\mathcal{P} (y) \propto  \mathrm{exp}\left[{-\frac{(y-m)^2}{2\Delta^2}}\right] \;,
%\end{eqnarray}
%where $m = 0.95$ and $\Delta = 0.06$.
%For other scenarios of cascade (e.g., neutral-current DIS and GR with $W \to \nu^{}_{e} e$), the invisible neutrino in the final state dominates the depletion of energy deposition, \gyh{for which we obtain $\mathcal{P} (y)$ by calculating the differential cross section (need to work out and give in the appendix if possible).}









\bibliographystyle{utcaps_mod}
\bibliography{reference}

\clearpage
\appendix
\onecolumngrid
\section{Appendix: Details of the Doppler broadening effect}\label{AppendixDoppler}
\begin{figure}[h!]
	\centering
	\includegraphics[width=0.5\columnwidth]{xsecDoppler.pdf}
	\caption{Cross section for the Glashow resonance process $\Bar{\nu}_e+e^- \rightarrow W^- \rightarrow X$ with and without Doppler broadening and assuming ice (${\rm H}^{}_{2} {\rm O}$) as the target. The black curve represents the cross section without Doppler broadening and for the orange curve Doppler broadening is included.}
	\label{fig:DopplerBroadened}
\end{figure}
\noindent
We  follow the procedure outlined in Ref.~\cite{Glashow:2014} to include the Doppler broadening effect of atomic electrons. By integrating over angular variables in Eq.~(\ref{eq:integrationDoppler}), we  arrive at
\begin{align}
\label{eq:sigmaorbitalintegrated}
\sigma (E^{}_{\nu}) =\frac{6\pi {\rm \Gamma}_W^2 {\rm Br}^{}_{W^-\rightarrow\overline{\nu}_e e^-}}{M^{}_W m^{}_e E^{}_{\nu} }\int d\beta \frac{F(\beta)}{\beta}
\left\{\frac{1}{2M^{}_W}\left[ \ln (y_h^2+1)-\ln (y_l^2+1) \right]+ \frac{1}{{\rm \Gamma}^{}_W}\left[ \arctan (y^{}_h)-\arctan(y^{}_l) \right] \right\}
\end{align}
where
\begin{equation}
y^{}_h=\frac{2m^{}_eE^{}_{\nu}(1+\beta )+m_e^2-M_W^2}{{\rm \Gamma}^{}_W M^{}_W}
\text{\hspace{0.5cm} and  \hspace{0.5cm}}
y^{}_l=\frac{2 m^{}_e E^{}_{\nu} (1-\beta ) + m_e^2 - M_W^2}{{\rm \Gamma}^{}_W M^{}_{W}}\,.
\end{equation}
Now the problem is attributed to the integration over the averaged electron velocity distribution $F(\beta)$.
In terms of the wave function of an electron with quantum numbers $n$ and $l$, the distribution reads
\begin{equation}
f^{}_{n\,l}(\beta)=m_e\int \mathrm{d}{\rm \Omega}^{}_k k^2 |{\rm \Psi}^{}_{n\,l}(k)|^2
\text{\hspace{0.5cm} with  \hspace{0.5cm}}
{\rm \Psi}^{}_{n\,l}(\mathbf{k})\propto Y_{l\,m}^*({\rm \Omega}^{}_k)\int_0^{\infty} \mathrm{d}r \, r^{n+1}e^{-\mu r} j^{}_l(kr)\;,
\end{equation}
where $k=m^{}_e \beta$ and $\mu^{}_{n\,l}=\xi^{}_{n\,l}/a^{}_0$. Here, $a_0$ denotes the Bohr radius, $\xi^{}_{n\,l}=Z^{}_{\rm eff}/n=(Z-\sigma^{}_{n\,l})/n$, and $\sigma_{n\,l}$ accounts for the screening of the nuclear charge by the other electrons in the atom.



After the integration, we can get the velocity distribution for atoms up to $Z=26$~\cite{Glashow:2014}:
\begin{align}
f^{}_{1s}(k)=&\frac{32}{\pi}\frac{\mu_{1s}^5k^2}{(\mu_{1s}^2+k^2)^4} \;,\\
f^{}_{2s}(k)=&\frac{32}{3\pi}\frac{\mu_{2s}^5(3\mu_{2s}^2k-k^3)^2}{(\mu_{2s}^2+k^2)^6} \;, \label{eq:correct1}\\
f^{}_{2p}(k)=&\frac{512}{3\pi}\frac{\mu_{2p}^7k^4}{(\mu_{2p}^2+k^2)^6} \;, \\
f^{}_{3s}(k)=&\frac{1024}{5\pi}\frac{\mu_{3s}^7 (\mu^3_{3s} k -\mu^{}_{3s} k^3)^2}{(\mu_{3s}^2+k^2)^8} \;, \\
f^{}_{3p}(k)=&\frac{1024}{45\pi}\frac{\mu_{3p}^7 (5\mu^2_{3p} k^2 - k^4)^2}{(\mu_{3p}^2+k^2)^8} \;,\\
f^{}_{3d}(k)=&\frac{4096}{5\pi}\frac{\mu_{3d}^9 k^6}{(\mu_{3d}^2+k^2)^8} \;, \\
f^{}_{4s}(k)=&\frac{512}{35\pi}\frac{\mu_{4s}^9 (5\mu^4_{4s}k -10\mu^2_{4s} k^3 + k^5)^2}{(\mu_{4s}^2+k^2)^{10}} \;.\label{eq:correct2}
\end{align}
Note that we have checked the expressions in Ref.~\cite{Glashow:2014} and corrected possible discrepancies in our Eqs.~(\ref{eq:correct1}) and (\ref{eq:correct2}).




We take the ice molecule ${\rm H}^{}_{2} {\rm O}$ as an example. For oxygen, $\mu_{1s}=7.6579$, $\mu_{2s}=2.2458$ and $\mu_{2p}=2.2266$~\cite{Clementi:1963}, and for hydrogen $\mu_{1s}=1$.
We weigh the distribution functions by averaging over the electron numbers:
\begin{equation}
\label{eq:Fice}
F^{}_{\rm ice}(\beta)= \frac{2F^{}_{\rm H}(\beta )+8F^{}_{\rm O}(\beta )}{10}\;.
\end{equation}
Using Eq.~(\ref{eq:sigmaorbitalintegrated}) together with Eq.~(\ref{eq:Fice}) we get the Doppler broadened cross section for ice as the target, which is depicted in Fig.~\ref{fig:DopplerBroadened}.
The effect reduces the peak by about $12\,\%$. Even though the total cross section integrated over the initial neutrino energy is barely altered, the broadening effect will make a difference when a non-uniform neutrino spectrum is considered.
%\nv{Do we want to comment on the fact that we had to use $1/\alpha F(1/alpha \times \beta )$ instead of just transforming to $\beta$? To be honest I still don't fully understand why this was necessary and in the Doppler broadening paper they don't mention this at all.}






%\section{The event rate for tau neutrinos} \label{AppendixRates}

%The differential event rate distribution of tau neutrinos differs from Eq.~(7) in the main text, because tau does not directly develop showers but decays hadronically or leptonically. The cascade rate due to tau neutrinos can be calculated with
%%\noindent
%\begin{align}
%\frac{\mathrm{d} N_{{\nunubar}_{\! \tau}}^{r}}{\mathrm{d} E^{}_{\rm dep}}  = & T^{}_{\rm IC}\cdot  N^{}_{\rm A} \int_{E^{}_{\rm min}}^\infty \mathrm{d} E^{}_{\nu} \frac{\mathrm{d} {\rm \Phi}_{{\nunubar}_{\! \tau}}^{\rm IC}}{\mathrm{d} E^{}_{\nu}}  \int^1_0 \mathrm{d} y  \frac{\mathrm{d} \sigma^{r}_{{{\nunubar}_{\! \tau}}}(E^{}_{\nu})}{\mathrm{d} y} \sum^{}_{f = h,e}\int \mathrm{d} z  \frac{\mathrm{d}n^{}_{f}}{\mathrm{d}z}  \frac{\mathrm{d}P^{}_{}(E^{}_{\rm sh})}{\mathrm{d}E^{}_{\rm dep}}  M^{}_{\rm eff}(E^{}_{\rm sh}) D^{}_{\tau}(E^{}_{\tau})\;\notag
%\end{align}
%The energy distribution of decay products can be found in Refs.~\cite{Dutta:2000jv,gaisserbook,Lipari:1993hd}, and $D^{}_{\tau}(E^{}_{\tau})$ quantifies the fraction of tau which decays promptly. Since the information of tauness is not provided for the PEPE event, we take $D^{}_{\tau}(E^{}_{\tau})=1$. Because the ${\nu}^{}_{\tau}$ CC interaction contributes to the background of GR, this choice is actually more conservative.



\end{document}
   
