\documentclass[journal]{IEEEtran}
%DIF LATEXDIFF DIFFERENCE FILE
%DIF DEL main-old.tex   Sat Feb 26 23:49:24 2022
%DIF ADD main.tex       Sat Feb 26 23:49:48 2022

\ifCLASSINFOpdf
\else
   \usepackage[dvips]{graphicx}
\fi
\usepackage{url}

\hyphenation{op-tical net-works semi-conduc-tor}

\usepackage{graphicx}
\usepackage[ruled,vlined]{algorithm2e}
\usepackage{amsfonts,amsthm,amsmath}

\newtheorem{thm}{Theorem}
\newtheorem{prop}{Proposition}
\theoremstyle{definition}
\newtheorem{rem}{Remark}

\newcommand{\utheta}{\Theta}
\newcommand{\ltheta}{\theta}
\newcommand{\vtheta}{\vartheta}
\newcommand{\tutheta}{\tilde{\utheta}}
\newcommand{\tltheta}{\tilde{\ltheta}}
\newcommand{\cx}{\mathcal{X}}
\newcommand{\mybfC}{\mathbf{C}}
\newcommand{\updatedbn}{\texttt{Update}}

%% covariance estimation white paper
%% covariance estimation white paper
\newcommand{\bfr}{\mathbf{R}}
\newcommand{\bfrh}{\hat{\mathbf{R}}}
\newcommand{\bfrc}{\check{\bfr}}
\newcommand{\bfx}{\mathbf{X}}
\newcommand{\bX}{\mathbf{X}}
\newcommand{\bx}{\mathbf{x}}
\newcommand{\bZ}{\mathbf{Z}}
\newcommand{\bs}{\mathbf{s}}
\newcommand{\bd}{\mathbf{d}}
\newcommand{\bS}{\mathbf{S}}
\newcommand{\cX}{\mathcal{X}}
\newcommand{\cy}{\mathcal{Y}}
\newcommand{\myM}{M}
\newcommand{\mym}{m}
\newcommand{\cm}{\mathcal{M}}
\newcommand{\tr}{\mathrm{tr}}
\newcommand{\bfa}{\mathbf{A}}
\newcommand{\bfb}{\mathbf{B}}
\newcommand{\bfc}{\mathbf{C}}
\newcommand{\bD}{\mathbf{D}}
\newcommand{\bU}{\mathbf{U}}
\newcommand{\rmid}{\mathrm{id}}
\newcommand{\bfv}{\mathbf{V}}
\newcommand{\bw}{\mathbf{w}}
\newcommand{\be}{\mathbf{e}}
\newcommand{\bu}{\mathbf{u}}
\newcommand{\bv}{\mathbf{v}}
\newcommand{\cO}{\mathcal{O}}
\newcommand{\cS}{\mathcal{S}}
\newcommand{\wthree}{\kappa}
\newcommand{\trR}{\text{admissible}}
\newcommand{\snr}{\eta}
\newcommand{\snrhat}{\hat{\eta}}
\newcommand{\snrtilde}{\tilde{\eta}}
\newcommand{\um}{\breve{m}}
\newcommand{\toas}{\overset{\text{a.s.}}{\to}}
\newcommand{\rhi}{\hat{R}^{-1}}
\newcommand{\rhrrh}{\hat{R}^{-1}R\hat{R}^{-1}}
\newcommand{\buni}{\bu_{n,i}}
\newcommand{\lamni}{\lambda_{n,i}}
\newcommand{\bfrs}{\mathbf{R}^*}
\newcommand{\bfrsi}{\mathbf{R}^{*-1}}
\newcommand{\toprob}{ \overset{\mathrm{p}}{\to} }
\newcommand{\bz}{\mathbf{z}}
\newcommand{\bfT}{\mathbf{T}}
\newcommand{\bssig}{\boldsymbol{\Sigma}}
\newcommand{\bV}{\mathbf{V}}
\newcommand{\fH}{\mathcal{H}}
\newcommand{\fK}{\mathcal{K}}
\newcommand{\fB}{\mathcal{B}}
\newcommand{\bbE}{\mathbb{E}}
\newcommand{\bbP}{\mathbb{P}}
\newcommand{\mydelta}{\delta}
\newcommand{\myphi}{\varphi}
\newcommand{\ala}{\emph{\`{a} la}}
\newcommand{\effone}{f_1}
\newcommand{\effzero}{f_0}
\newcommand{\tautilde}{\tilde{\tau}}
\newcommand{\myem}{T}

%DIF 91c91
%DIF < \newcommand{\laipoor}{\texttt{Greedy}}
%DIF -------
\newcommand{\laipoor}{\texttt{Round-Robin}} %DIF > 
%DIF -------
\newcommand{\proposed}{\texttt{Proposed}}
%DIF PREAMBLE EXTENSION ADDED BY LATEXDIFF
%DIF UNDERLINE PREAMBLE %DIF PREAMBLE
\RequirePackage[normalem]{ulem} %DIF PREAMBLE
\RequirePackage{color}\definecolor{RED}{rgb}{1,0,0}\definecolor{BLUE}{rgb}{0,0,1} %DIF PREAMBLE
\providecommand{\DIFadd}[1]{{\protect\color{blue}\uwave{#1}}} %DIF PREAMBLE
\providecommand{\DIFdel}[1]{{\protect\color{red}\sout{#1}}}                      %DIF PREAMBLE
%DIF SAFE PREAMBLE %DIF PREAMBLE
\providecommand{\DIFaddbegin}{} %DIF PREAMBLE
\providecommand{\DIFaddend}{} %DIF PREAMBLE
\providecommand{\DIFdelbegin}{} %DIF PREAMBLE
\providecommand{\DIFdelend}{} %DIF PREAMBLE
\providecommand{\DIFmodbegin}{} %DIF PREAMBLE
\providecommand{\DIFmodend}{} %DIF PREAMBLE
%DIF FLOATSAFE PREAMBLE %DIF PREAMBLE
\providecommand{\DIFaddFL}[1]{\DIFadd{#1}} %DIF PREAMBLE
\providecommand{\DIFdelFL}[1]{\DIFdel{#1}} %DIF PREAMBLE
\providecommand{\DIFaddbeginFL}{} %DIF PREAMBLE
\providecommand{\DIFaddendFL}{} %DIF PREAMBLE
\providecommand{\DIFdelbeginFL}{} %DIF PREAMBLE
\providecommand{\DIFdelendFL}{} %DIF PREAMBLE
\newcommand{\DIFscaledelfig}{0.5}
%DIF HIGHLIGHTGRAPHICS PREAMBLE %DIF PREAMBLE
\RequirePackage{settobox} %DIF PREAMBLE
\RequirePackage{letltxmacro} %DIF PREAMBLE
\newsavebox{\DIFdelgraphicsbox} %DIF PREAMBLE
\newlength{\DIFdelgraphicswidth} %DIF PREAMBLE
\newlength{\DIFdelgraphicsheight} %DIF PREAMBLE
% store original definition of \includegraphics %DIF PREAMBLE
\LetLtxMacro{\DIFOincludegraphics}{\includegraphics} %DIF PREAMBLE
\newcommand{\DIFaddincludegraphics}[2][]{{\color{blue}\fbox{\DIFOincludegraphics[#1]{#2}}}} %DIF PREAMBLE
\newcommand{\DIFdelincludegraphics}[2][]{% %DIF PREAMBLE
\sbox{\DIFdelgraphicsbox}{\DIFOincludegraphics[#1]{#2}}% %DIF PREAMBLE
\settoboxwidth{\DIFdelgraphicswidth}{\DIFdelgraphicsbox} %DIF PREAMBLE
\settoboxtotalheight{\DIFdelgraphicsheight}{\DIFdelgraphicsbox} %DIF PREAMBLE
\scalebox{\DIFscaledelfig}{% %DIF PREAMBLE
\parbox[b]{\DIFdelgraphicswidth}{\usebox{\DIFdelgraphicsbox}\\[-\baselineskip] \rule{\DIFdelgraphicswidth}{0em}}\llap{\resizebox{\DIFdelgraphicswidth}{\DIFdelgraphicsheight}{% %DIF PREAMBLE
\setlength{\unitlength}{\DIFdelgraphicswidth}% %DIF PREAMBLE
\begin{picture}(1,1)% %DIF PREAMBLE
\thicklines\linethickness{2pt} %DIF PREAMBLE
{\color[rgb]{1,0,0}\put(0,0){\framebox(1,1){}}}% %DIF PREAMBLE
{\color[rgb]{1,0,0}\put(0,0){\line( 1,1){1}}}% %DIF PREAMBLE
{\color[rgb]{1,0,0}\put(0,1){\line(1,-1){1}}}% %DIF PREAMBLE
\end{picture}% %DIF PREAMBLE
}\hspace*{3pt}}} %DIF PREAMBLE
} %DIF PREAMBLE
\LetLtxMacro{\DIFOaddbegin}{\DIFaddbegin} %DIF PREAMBLE
\LetLtxMacro{\DIFOaddend}{\DIFaddend} %DIF PREAMBLE
\LetLtxMacro{\DIFOdelbegin}{\DIFdelbegin} %DIF PREAMBLE
\LetLtxMacro{\DIFOdelend}{\DIFdelend} %DIF PREAMBLE
\DeclareRobustCommand{\DIFaddbegin}{\DIFOaddbegin \let\includegraphics\DIFaddincludegraphics} %DIF PREAMBLE
\DeclareRobustCommand{\DIFaddend}{\DIFOaddend \let\includegraphics\DIFOincludegraphics} %DIF PREAMBLE
\DeclareRobustCommand{\DIFdelbegin}{\DIFOdelbegin \let\includegraphics\DIFdelincludegraphics} %DIF PREAMBLE
\DeclareRobustCommand{\DIFdelend}{\DIFOaddend \let\includegraphics\DIFOincludegraphics} %DIF PREAMBLE
\LetLtxMacro{\DIFOaddbeginFL}{\DIFaddbeginFL} %DIF PREAMBLE
\LetLtxMacro{\DIFOaddendFL}{\DIFaddendFL} %DIF PREAMBLE
\LetLtxMacro{\DIFOdelbeginFL}{\DIFdelbeginFL} %DIF PREAMBLE
\LetLtxMacro{\DIFOdelendFL}{\DIFdelendFL} %DIF PREAMBLE
\DeclareRobustCommand{\DIFaddbeginFL}{\DIFOaddbeginFL \let\includegraphics\DIFaddincludegraphics} %DIF PREAMBLE
\DeclareRobustCommand{\DIFaddendFL}{\DIFOaddendFL \let\includegraphics\DIFOincludegraphics} %DIF PREAMBLE
\DeclareRobustCommand{\DIFdelbeginFL}{\DIFOdelbeginFL \let\includegraphics\DIFdelincludegraphics} %DIF PREAMBLE
\DeclareRobustCommand{\DIFdelendFL}{\DIFOaddendFL \let\includegraphics\DIFOincludegraphics} %DIF PREAMBLE
%DIF LISTINGS PREAMBLE %DIF PREAMBLE
\RequirePackage{listings} %DIF PREAMBLE
\RequirePackage{color} %DIF PREAMBLE
\lstdefinelanguage{DIFcode}{ %DIF PREAMBLE
%DIF DIFCODE_UNDERLINE %DIF PREAMBLE
  moredelim=[il][\color{red}\sout]{\%DIF\ <\ }, %DIF PREAMBLE
  moredelim=[il][\color{blue}\uwave]{\%DIF\ >\ } %DIF PREAMBLE
} %DIF PREAMBLE
\lstdefinestyle{DIFverbatimstyle}{ %DIF PREAMBLE
	language=DIFcode, %DIF PREAMBLE
	basicstyle=\ttfamily, %DIF PREAMBLE
	columns=fullflexible, %DIF PREAMBLE
	keepspaces=true %DIF PREAMBLE
} %DIF PREAMBLE
\lstnewenvironment{DIFverbatim}{\lstset{style=DIFverbatimstyle}}{} %DIF PREAMBLE
\lstnewenvironment{DIFverbatim*}{\lstset{style=DIFverbatimstyle,showspaces=true}}{} %DIF PREAMBLE
%DIF END PREAMBLE EXTENSION ADDED BY LATEXDIFF

\begin{document}

\title{Non-Bayesian Online Anomaly Search with Switching Delays}

\author{Benjamin D. Robinson, \IEEEmembership{Member, IEEE}, Matthew Ubl, \IEEEmembership{Student Member, IEEE}, Matthew Hale, \IEEEmembership{Member, IEEE}, and Douglas Cochran \IEEEmembership{Fellow, IEEE}
\thanks{This paragraph of the first footnote will contain the date on which you submitted your paper for review. It will also contain support information, including sponsor and financial support acknowledgment. For example, ``This work was supported in part by the U.S. Department of Commerce under Grant BS123456.'' }
\thanks{The next few paragraphs should contain the authors' current affiliations, including current address and e-mail. For example, F. A. Author is with the National Institute of Standards and Technology, Boulder, CO 80305 USA (e-mail: author@boulder.nist.gov).}
\thanks{S. B. Author, Jr., was with Rice University, Houston, TX 77005 USA. He is now with the Department of Physics, Colorado State University, Fort Collins, CO 80523 USA (e-mail: author@lamar.colostate.edu).}}

\markboth{Journal of \LaTeX\ Class Files, Vol. 14, No. 8, August 2015}
{Shell \MakeLowercase{\textit{et al.}}: Bare Demo of IEEEtran.cls for IEEE Journals}
\maketitle

\begin{abstract}

Online anomaly search, also called sequential design of experiments, is a problem dating back to Chernoff (1959).  In it, multiple channels are available for sampling, one at a time, and any anomalous ones are to be identified.  The fundamental trade-off to be optimized is the trade-off between false-alarm rate and expected stopping time.  A particular challenge is a version of the problem in which the experimenter must suffer a time delay each time the observed channel index switches from one index to a different one.  Lambez and Cohen (2021) present a solution that is  asymptotically optimal as the false-alarm rate goes to zero, but only in the case where Bayesian prior probabilities of whether channels are anomalous are available.  In this paper, we propose a complementary algorithm for the non-Bayesian case, aiming to optimize worst-case, rather than average, expected detection delay subject to false-alarms.  Our algorithm makes use of recent work on multi-armed bandits by Rouyer et al. (2021). We present a theoretical bound on false-alarm rate and simulations indicating favorable performance relative to the quickest search algorithm of Lai et al. (2011).

% These instructions give you guidelines for preparing papers for IEEE Signal Processing Letters. Use this document as a template if you are using \LaTeX. Otherwise, use this document as an instruction set. The electronic file of your paper will be formatted further at IEEE. Paper titles should be written in uppercase and lowercase letters, not all uppercase. Do not write ``(Invited)'' in the title. Full names of authors are preferred in the author field, but are not required. Put a space between authors’ initials. The abstract must be a concise yet comprehensive reflection of what is in your article. In particular, the abstract must be self-contained, without abbreviations, footnotes, or references. It should be a microcosm of the full article. The abstract is typically between 100--175 words. The abstract must be written as one paragraph, and should not contain displayed mathematical equations or tabular material. The abstract should include three or four different keywords or phrases, as this will help readers to find it. It is important to avoid over-repetition of such phrases as this can result in a page being rejected by search engines. Ensure that your abstract reads well and is grammatically correct.
\end{abstract}

\begin{IEEEkeywords}
Quickest search, multi-armed bandit, controlled sensing, scanning rule
\end{IEEEkeywords}


\IEEEpeerreviewmaketitle



\section{Introduction}

\IEEEPARstart{A}{}fundamental problem in information theory and signal processing is online anomaly search.  With origins in Chernoff's sequential design of experiments \cite{chernoff1959sequential}, the aim of online anomaly search is to develop an efficient policy for sampling a subset of several channels over time that quickly and accurately identifies any anomalous ones.  An important application is clinical-trial design, where the drug tested in each trial depends on the drugs tested and outcomes obtained in all previous trials of the study.  
Another application is online search for an open channel in cognitive radio, where the sampling policy may depend on past observations.  
Many variations on Chernoff's method exist \cite{dragalin1996simple,nitinawarat2013controlled,naghshvar2013active,nitinawarat2015controlled,cohen2015active,huang2018active,tsopelakos2019sequential}, as do methods for handling an essentially infinite number of experiments \cite{lai2011quickest,malloy2012quickest,tajer2013quick,malloy2014sequential}.

Recently several authors have considered the problem of online anomaly search in which a delay is incurred any time the current channel sampled differs from the last one \cite{vaidhiyan2017neural, vaidhiyan2015active, lambez2021anomaly}.  This new formulation models online anomaly search more realistically than the traditional formulation because it accounts for possible switching delays in hardware or software.
% when a gimbeled radar searches two locations alternately with a switching time that is nonzero.  Another relevant scenario would be searching multiple locations for a target event using an autonomous platform, perhaps for disaster search-and-rescue.  
In a closely-related work to ours, Lambez and Cohen \cite{lambez2021anomaly} have proposed a method that \DIFdelbegin \DIFdel{is }\DIFdelend balances false-alarm probability, sampling cost, and switching penalties/delays, and has state-of-the-art finite-sample performance.   However, their method \DIFdelbegin \DIFdel{is Bayesian, making the }\DIFdelend \DIFaddbegin \DIFadd{relies upon the Bayesian }\DIFaddend assumption that each channel's prior probability of being anomalous is known,  and does not provide \DIFdelbegin \DIFdel{a result }\DIFdelend \DIFaddbegin \DIFadd{any insight }\DIFaddend in the case where no such priors are available.  In this paper, we address this gap by proposing a method that has provable \emph{worst-case}, rather than average-case, performance. Our method is based on an algorithm designed for multi-armed bandits with switching costs \cite{rouyer2021algorithm}.  For simplicity we assume that the number of anomalous channels and the number of samples that can be taken at each time is one.

In Section~\ref{sec:background} we provide some background and a formal problem statement.  In Section~\ref{sec:proposed} we describe our method and proofs.  In Section~\ref{sec:simulations} we present simulations of our results.%comment**: simulations plural?

\section{Background} \label{sec:background}

\subsection{Stochastic Multi-Armed Bandits with Switching Costs} \label{subsec:bandits}

The stochastic multi-armed bandit problem is a model for many sequential decision making problems, including ad-serving and clinical-trial design.  In this paper, we consider a variation of it in which sampling from a different arm than the previous one results in a switching cost $\lambda \ge 0$, as formulated by \cite{rouyer2021algorithm}.  The problem setup is that there are $K$ arms, each producing i.i.d. streams $X_1^i, X_2^i, \dots$ in a sample space $\mathcal{X}$ ($i\in\{1,2,\dots, K\}$).  Each observation is associated with a \emph{loss} $\ell_{t,i} = \ell(X^i_t)$ for some measurable function $\ell: \mathcal{X}\to [0,1]$.  
Given the sampling-loss history of $(i_1, \ell_{1,i_1}, i_2, \ell_{2,i_2},\dots, i_t, \ell_{t,i_t})$, we would like to inductively select an index $i_{t+1}\in \{1,2,\dots, K\}$ to minimize the \emph{pseudo-regret}.  This pseudo-regret is defined as the expected value of the cumulative loss: the sum of the gaps between the ideal loss and the realized ones, plus $\lambda$ times the number of switches.  In other words, denoting the pseudo-regret by $R(T,\lambda)$, we have:
% and we wish to determine a function $J_t \in \{1,2,\dots, K\}$ so that the expected cumulative regret which arm to sample next based on all the arms and observations sampled so far.  In practice, this decision is usually based on \emph{losses} observed: given $\ell:\mathcal{X}\to [0,1]$ we observe losses $\ell_{t,i} = \ell(X^i_t)\in [0,1]$, and decide at time $t+1$ to sample an arm $J_{t+1}$ based only on the $\ell_{t, J_t}$, $s=1,2,\dots, t$.  At each time $t$, the algorithm may select an arm $J_t$ based on the losses and choices that have occured so far.  The goal is to minimize the  pseudo-regret
\begin{align*}
R(T,\lambda) & = \mathbb{E}\left[\sum_{t=1}^T\ell_{t,i_t}\right] - \min_i \mathbb{E}\left[\sum_{t=1}^T\ell_{t,i}\right] \\
& + \lambda \sum_{t=1}^T \mathrm{Pr}(i_{t-1} \ne i_t).
\end{align*}

% there are several possible ``arms'' that a player may play one at a time, each of which yields a reward sampled from some unknown but stationary distribution.  Arm choices are sequential---meaning, they may depend on past arm choices and observations. The goal is choose arms so that the expected total reward is within some bound of the maximum possible expected total reward. In other words, we wish to find an allocation rule whose total expected reward approximates the total expected reward that would be earned by an ``oracle'' that knows the best arm a priori.

Optimal algorithms for minimizing $R(T,\lambda)$ over some fixed or receding time horizon are generally computationally intractable \cite{hero2015foundational}.  Instead, one often resorts to finding algorithms for which $R(T,\lambda)$ is sub-linear  in $T$. For $\lambda=0$ the UCB family of algorithms provides several examples \cite{lattimore2020bandit}. The case of $\lambda\ne 0$ is significantly more challenging, but fortunately a \DIFdelbegin \DIFdel{sub-linear solution }\DIFdelend \DIFaddbegin \DIFadd{solution with sub-linear pseudo-regret }\DIFaddend exists and is provided in \cite{rouyer2021algorithm}.  Their solution, referred to as the Tsallis-Switch algorithm, can be obtained for time horizon $T$ by setting $b=\infty$ in Algorithm~\ref{alg:tsallis-alg}.  The choice $b=\infty$ makes the functions $f_1$ and $f_0$ and the algorithm's output irrelevant, but they will become important in the next section.

% One strategy is to play each arm in succession until one has a reasonably good idea of the arm's payoff is obtained.  However, this strategy has the disadvantage that it can require a relatively large number of plays before the correct arm can be determined.  A better family of strategies is the Upper Confidence Bound (UCB) family.  This family is ``rate-optimal,'' meaning effectively that the number of incorrect arm choices is optimally small up to asymptotic constants \cite{lattimore2020bandit}.  

% When a sensing agent changes data streams to observe, there is often some time lag before the new stream can be sensed.  For example, data structures may have to be destroyed and re-initialized, a radar on a gimbel may have to rotate to a new position, or a robot may have to get near to its target**.  In this case, the agent incurs an opportunity cost of however many samples could have been taken during the switch.  If $\lambda$ is the number of samples lost per switch, there is a recent generalization of the UCB algorithm that minimizes total expected sampling cost called Tsallis-Switch \cite{rouyer2021algorithm}. If each sample is drawn from a common sample space $\mathcal{X}$ and the stage-wise loss function is $\ell:\mathcal{X}\to[0,1]$, the Tsallis-Switch algorithm for time horizon $T$ can be obtained by setting $b=\infty$ in Algorithm~\ref{alg:tsallis-alg}.  The choice $b=\infty$ makes the functions $f_1$ and $f_0$ and the algorithm's output irrelevant, but they will become important in the next section.

\begin{figure}
\begin{algorithm}[H]
\DontPrintSemicolon
\SetAlgoLined
\KwResult{Anomalous arm prediction $i$}
Input: $K, \myem \in \mathbb{N}$; $\lambda, b \ge 0$; $\ell:\mathcal{X}\to [0,1]$ \;
Initialization: $Y=t = 0$, $n=1$, $\mybfC_0 = \mathbf{0}_K$ \; 
% Initialization: $T^i = \log \frac{\effone(X^i_1)}{\effzero(X^i_1)}$ and $\hat{\mu}^i = \frac{1}{2\sqrt{\mu}} T^i$ and $n^i = 1$ for $i=1,2,\dots, K$; and $t=K$\;
 \While{$ Y \le b$ and $t < \myem$}{
$a_n = \frac{3\lambda}{2}\sqrt{\frac{n}{K}}$, $B_n = \max\{\lceil a_n \rceil, 1\}$, $\eta_n = \frac{2}{a_n+1}\sqrt{\frac{2}{n}}$ \;
$p = \mathrm{argmin}_{p\in \Delta^{K-1}} \left\{ \left\langle p, \mybfC_{n-1}\right\rangle - \sum_{i=1}^K \frac{2\sqrt{p_i} - 2p_i}{\eta_n}\right\}$ \;
Sample $I \sim p$; Let $i= I$\; 
s = 0 \;
\While{$Y \le b$ and $Y \ge 0$}{
$s = s + 1$; Sample $X_s \sim F_{i==i^*}$ \;
$Y = \max\{Y, 0\} + \log\frac{\effone(X_s)}{\effzero(X_s)}$ \;
}
\While{$Y \le b$ and $s < B_n$}{
$s = s + 1$, Sample $X_s \sim F_{i==i^*}$ \;
}
$c = \sum_{s=1}^{B_n} \ell(X_s)$\;
$\mybfC_n(i) = \mybfC_{n-1}(i) + \frac{c}{p_i}$ \;
 $n=n+1$ \;
 $t=t+1$ \;

}

 \caption{\label{alg:tsallis-alg} \proposed{}}
\end{algorithm}
\end{figure}

As promised, Tsallis-Switch induces a regret with a sub-linear bound in $T$, given explicitly in the proposition below.
\begin{prop} \label{prop:main}
Suppose $\Delta_i = \mathbb{E}\ell_{1,i}- \min_i \mathbb{E}\ell_{1,i}$, and there is a constant $\Delta$ so that $\Delta_i = \Delta$ for all but one $i$.  Then the pseudo-regret of Tsallis-Switch satisfies
\begin{align*}
R(T,\lambda)  = \DIFdelbegin \DIFdel{\frac{K}{\Delta}}\DIFdelend \DIFaddbegin \DIFadd{\frac{K-1}{\Delta}}\DIFaddend \mathcal{O}\left((\lambda K)^{2/3}T^{1/3}+\log T\right).
\end{align*}
\end{prop}
\noindent Indeed, \cite[Theorem~1]{rouyer2021algorithm} indicates the upper bound on $R(T,\lambda)$ can be taken to be the concave function of $T$
\begin{align*}
    & \left(66(\lambda K)^{2/3}T^{1/3} + 32\log T\right)\DIFdelbegin \DIFdel{\frac{K}{\Delta} }\DIFdelend \DIFaddbegin \DIFadd{\frac{K-1}{\Delta} }\DIFaddend \\
    & + \left(160\lambda^{2/3}T^{1/3}K^{1/6} + 160\lambda + 49\lambda^2+32\right)\DIFdelbegin \DIFdel{\frac{K}{\Delta} }\DIFdelend \DIFaddbegin \DIFadd{\frac{K-1}{\Delta} }\DIFaddend \\
    & + \frac{544\lambda}{\sqrt{K}} + \lambda + 66.
\end{align*}


% Notes on this section:

% **Applications where there is a switching cost is repetitive



\subsection{Anomaly Search}

% For the purpose of the Research Objectives discussed below, we introduce here the quickest search for change point problem (QSC), which is modeled similarly to SQS.

In this section, we formally describe the problem of anomaly search with switching costs, which is closely related to the bandit problem above.  Assume as before there 
 are $K$ channels (previously referred to as arms), each producing i.i.d. streams $X_1^i, X_2^i, \dots$ in a sample space $\mathcal{X}$ ($i\in\{1,2,\dots, K\}$). 
% there are $K$ channels, each  producing independent observations $X^i_1, X^i_2,\dots$ in a sample space $\mathcal{X}$ for $i=1,2,\dots, K$, and that exactly one channel can be sampled at any given time.  
Assume the observations $X^{i^*}_t$ have distribution $F_1$ (p.d.f. $\effone$) and that for $i\ne i^*$, the observations $X^i_t$ have distribution $F_0$ (p.d.f. $f_0$).  
Given that $i_1, i_2, \dots, i_t$ were the previously sampled channel indices, we would like to inductively determine from  the \emph{sampling-observation history} $H_t = (i_1, X^{i_1}_1, i_2, X^{i_2}_2, \dots, i_t, X^{i_t}_t)$ whether to sample some channel $i_{t+1}$ or stop sampling completely.    
Note that the sampling-observation history differs from the sample-loss history of Section~\ref{subsec:bandits} in that the sampling-observation history does not explicitly involve a loss function. If $\tau$ is the number of samples taken before stopping, we define $\tautilde = \tau + \sum_{t=1}^{\tau-1} \mathbf{1}(i_t \ne i_{t+1})$, the detection delay including switching costs.  Upon stopping, we make a terminal decision $\delta\in \{1,2,\dots, K\}$ based on the sampling history as to which channel is anomalous.  We define a \emph{sampling policy} to be a tuple $(\varphi, \tau, \delta)$, also called a \emph{scanning rule} by Dragalin \cite{dragalin1996simple}, who considered the case where $\lambda = 0$.

Let $\mathrm{Pr}_i$ and $\mathbb{E}_i$ denote the probability and expectation given that the anomalous stream has index $i$.  We call a sampling policy \emph{worst-case optimal}  if it minimizes  $\max_i\mathbb{E}_i\tautilde$ subject to the constraints $\mathrm{Pr}_i(\delta \ne i) \le \gamma$ for $i=1,2,\dots, K$.  This type of optimality, not Bayesian optimality, is the type that we aim to approximate in this paper.

\section{Proposed Method and False-Alarm Rate} \label{sec:proposed}

The proposed method makes use of a measurable loss function $\ell:\mathcal{X}\to[0,1]$, where the nominal channels are penalized more on average than the anomalous ones. The only difference between the proposed method and the Tsallis-Switch algorithm is that instead of infinite $b$ and finite $\myem$ in \proposed{}, we take $\myem$ to be infinite and $b$ to be finite.  This choice effectively makes the time horizon infinite and allows the algorithm to run until the detection threshold of $b$ is reached.

As in Proposition~\ref{prop:main}, let $\Delta$ be the common value of $\mathbb{E}\ell_{1,i} - \mathbb{E}\ell_{1,i^*}$ for $i\ne i^*$.  The following theorem estimates the false-alarm rate of \proposed{}.  
\begin{thm}
% Let $\Delta$ be the common value of $\mathbb{E}\ell(X^i_1) - \mathbb{E}\ell(X^{i^*}_1)$ for $i\ne i^*$. 
Let $\alpha$ and $\beta$ be the type-I and type-II error probabilities of a Wald test of $F_0$ versus $F_1$ with thresholds $0$ and $b$.  Then the false-alarm rate $P_{FA}$ of \proposed{} is bounded above according to
\[
P_{FA} \le 1 - (1-\beta)\sum_{n=1}^\infty \beta^{n-1} \left(1-\alpha\right)\DIFdelbegin \DIFdel{^{\min\{(K-1)n,\phi(n)\}}}\DIFdelend \DIFaddbegin \DIFadd{^{(K-1)\min\{n,\phi(n)\}}}\DIFaddend ,
\]
where $\phi(n)$ is the (\DIFdelbegin \DIFdel{sub-linear}\DIFdelend \DIFaddbegin \DIFadd{concave}\DIFaddend ) function that satisfies the functional equation
\begin{equation} \label{eq:thm-main}
\phi(n) = \DIFdelbegin \DIFdel{C \frac{K}{\Delta^2} }\DIFdelend \DIFaddbegin \DIFadd{\frac{C}{\Delta^2} }\DIFaddend \left(\lambda\sqrt{K(\phi(n) + n)}+\log(\phi(n) + n)\right)
\end{equation}
for some absolute constant $C$ determined by Proposition~\ref{prop:main}.
% for $n\ge 1$ and some $C$ depending only on $F_0$ and $F_1$.
\end{thm}
\begin{rem}
We note that, by a perturbative analysis, $\phi(n)$ is to first order \DIFdelbegin \DIFdel{$C\frac{K}{\Delta^2}\left(\lambda\sqrt{Kn}+\log n\right)$}\DIFdelend \DIFaddbegin \DIFadd{$\frac{C}{\Delta^2}\left(\lambda\sqrt{Kn}+\log n\right)$}\DIFaddend .  For the exact value of $\phi(n)$ required by the theorem, a numerical solver can be employed.  
\DIFdelbegin \DIFdel{We further note that the algorithm of \mbox{%DIFAUXCMD
\cite{dragalin1996simple} }\hspace{0pt}%DIFAUXCMD
(which is essentially equivalent to the one in \mbox{%DIFAUXCMD
\cite{lai2011quickest}}\hspace{0pt}%DIFAUXCMD
) can be recovered if the exponent of $1-\alpha$ is weakened to $(K-1)n$.
}\DIFdelend %DIF > We further note that the false-alarm bound for the algorithm of \cite{dragalin1996simple} (which is essentially equivalent to the algorithm in \cite{lai2011quickest}) can be recovered if the exponent of $1-\alpha$ is weakened to $(K-1)n$.
\end{rem}
\begin{proof}
Observe from \proposed{} that the procedure uses block sampling with block lengths of $B_1, B_2, \dots$.  We consider the complement of $P_{FA}$, which corresponds to probability of correct detection.  Let the random integer $N$ be the number of blocks from arm $i^*$ that are observed before a correct detection.  By disjunction $1-P_{FA} = \sum_{n=1}^\infty \mathrm{Pr}_{i^*}[N = n]$.

Consider $\mathrm{Pr}_i[N=n]$.  For $j=1,2,\dots,n$, let the random integers $M_j$ be the number of nominal blocks observed between anomalous block $j-1$ and anomalous block $j$.  Then $\mathrm{Pr}_i[N=n\mid M_1, M_2, \dots, M_n] $ is given by
\begin{align*}
(1-\alpha)^{\sum_{j=1}^n M_j} \beta^{n-1} (1-\beta),
\end{align*}
since there are $n-1$ blocks resulting in a miss, each with probability $\beta$, and a final block resulting in a detection, which occurs with probability $1-\beta$.  Setting $N_n = \sum_{j=1}^n M_j$, taking the expectation, and using convexity of exponential functions, we get
\[
\mathrm{Pr}_{i^*}[N=n] \ge (1-\alpha)^{\mathbb{E}N_n}\beta^{n-1}(1-\beta).
\]

A simple upper bound for $N_n$ is the sum of the number of incorrect samples up to the terminal number of blocks \DIFdelbegin \DIFdel{: }\DIFdelend $N_n + n$.  This follows from the fact that the number of nominal blocks observed is smaller than the number of nominal samples taken before the procedure terminates.  The time of termination $T_n$ is bounded above by $\psi(N_n+n) := \sum_{j=1}^{N_n+n} B_j$, which is of order $(N_n+n)^{3/2}\lambda /\sqrt{K}$\DIFaddbegin \DIFadd{, using an approximation by and integral and the fact that $B_j = 3\lambda\sqrt{j/K}$}\DIFaddend . We will bound $\mathbb{E}N_n$ from above in two ways: first, by conditioning on $T_n$ and using Proposition~\ref{prop:main}.

Consider $\mathbb{E}[N_n \mid T_n]$.   Since the pseudo-regret is $\Delta$ times the number of incorrect samples plus the switching costs, an upper bound for $\mathbb{E}[N_n \mid T_n]$ is $\Delta^{-1} R(\psi(N_n+n),\lambda)$.  It follows  from Proposition~\ref{prop:main} that $\Delta^{-1} R(\psi(N_n+n), \lambda)$ is order
\begin{align*}
    & \DIFdelbegin \DIFdel{\frac{K}{\Delta^2}}\DIFdelend \DIFaddbegin \DIFadd{\frac{K-1}{\Delta^2}}\DIFaddend \left\{ (\lambda K)^{2/3}\left((N_n+n)^{3/2}\lambda/\sqrt{K}\right)^{1/3} + \log(N_n+n)\right\} \\
    & = \DIFdelbegin \DIFdel{\frac{K}{\Delta^2}}\DIFdelend \DIFaddbegin \DIFadd{\frac{K-1}{\Delta^2}}\DIFaddend \mathcal{O}\left(\lambda\sqrt{K(N_n+n)}+\log(N_n+n)\right).
\end{align*}
Thus, by taking expectations and using concavity,
\begin{align*}
\mathbb{E}N_n & \le C \DIFdelbegin \DIFdel{\frac{K}{\Delta^2} }\DIFdelend \DIFaddbegin \DIFadd{\frac{K-1}{\Delta^2} }\DIFaddend \mathbb{E}\left[\lambda\sqrt{K(N_n + n)}+\log(N_n+n)\right], \\
    % \mathbb{E}[N_n] & \le \frac{1}{\Delta} \mathbb{E} R(\psi(N_n+n),\lambda) \\
    % & \le \frac{1}{\Delta} R(\psi(\mathbb{E}N_n + n), \lambda) \\
    & \le C \DIFdelbegin \DIFdel{\frac{K}{\Delta^2} }\DIFdelend \DIFaddbegin \DIFadd{\frac{K-1}{\Delta^2} }\DIFaddend \left(\lambda\sqrt{K(\mathbb{E}N_n + n)}+\log(\mathbb{E}N_n + n)\right),
\end{align*}
for some absolute constant $C$.  It follows that $\mathbb{E}N_n$ is bounded above by \DIFdelbegin \DIFdel{the function }\DIFdelend \DIFaddbegin \DIFadd{$(K-1)\phi(n)$, where }\DIFaddend $\phi(n)$ that satisfies \eqref{eq:thm-main}.  

We now bound $\mathbb{E}N_n$ a second way.  The algorithm \proposed{} treats all channels equally initially, and tends to favor sampling from the anomalous channel over the nominal ones on average over \DIFdelbegin \DIFdel{a }\DIFdelend \DIFaddbegin \DIFadd{any }\DIFaddend fixed number of blocks.  It must therefore be that if $n$ anomalous blocks are sampled, the expected number of blocks allocated to each nominal channel is at most $n$.  This means that $\mathbb{E}N_n \le (K-1)n$.

From the argument above, we have that $$1-P_{FA} \ge (1-\beta)\sum_{n=1}^\infty \beta^{n-1}(1-\alpha)\DIFdelbegin \DIFdel{^{\min\{(K-1)n,\phi(n)\}}}\DIFdelend \DIFaddbegin \DIFadd{^{(K-1)\min\{n,\phi(n)\}}}\DIFaddend ,$$
\DIFdelbegin \DIFdel{as desired}\DIFdelend \DIFaddbegin \DIFadd{which, by rearrangement of terms, is the same as the desired result}\DIFaddend .

% Assume that a correct detection is made the $(n+1)^\text{th}$ time a block of samples from arm $i^*$ is observed.  Then, the first $n$ blocks resulted in a miss, each with probability $\beta$, and the final block resulted in a detection, with probability $1-\beta$.  Assume further $N_n$ is the (random) number of blocks observed from arms with $i\ne i^*$ before detection.  

% We iterate over the number of times a block from channel $i^*$ is observed but no detection occurs.  Given that this number is $n$, the number of prior blocks from channels other than $i^*$ that are observed is a random integer $N_n$.  
% \[
% \mathrm{Pr}_i[\delta = i \mid N_1, N_2, \dots] = (1-\beta)\sum_{n=0}^\infty \beta^n (1-\alpha)^{N_n}
% \]
% Taking the expectation of both sides and using convexity gives
% \[
% \mathrm{Pr}_i[\delta = i] \ge (1-\beta)\sum_{n=0}^\infty \beta^n (1-\alpha)^{\mathbb{E}N_n}
% \]
% Using the upper bound of $R(\sum_{i=1}^n B_i, \lambda)$ on $\Delta\mathbb{E}N_n$ gives the desired result.
\end{proof}

% **It would be nice to use $\phi_\lambda$ or something similar to estimate this in some practical setting, perhaps in the simulations section.

\section{Simulations} \label{sec:simulations}

In this section we simulate our algorithm and a baseline one.  We choose $X^i_t\sim \mathcal{N}(-\mu,1)$ for $i < K$ and $\mathcal{N}(\mu,1)$ for $i=K$.  (In other words, $i^*=K$.) We choose $\ell$ to be the map \DIFaddbegin \DIFadd{$1-\Phi$ }\DIFaddend from $\mathbb{R}$ to $[0,1]$\DIFdelbegin \DIFdel{using }\DIFdelend \DIFaddbegin \DIFadd{, where $\Phi$ is }\DIFaddend the standard normal c.d.f., although it is possible improved maps exist.  A competing algorithm is the \DIFdelbegin \DIFdel{cyclical }\DIFdelend \DIFaddbegin \DIFadd{round-robin }\DIFaddend algorithm of \cite{lai2011quickest}\DIFaddbegin \DIFadd{, which is essentially identical to the one in \mbox{%DIFAUXCMD
\cite{dragalin1996simple}}\hspace{0pt}%DIFAUXCMD
}\DIFaddend .  Although this algorithm is designed for the case where the number $K$ of channels goes to infinity, it provides a reasonable baseline for other algorithms when $K$ is finite as well.  The cyclical algorithm, \laipoor{}, is presented in Algorithm~\ref{alg:linear-qs}.

\begin{figure}
\begin{algorithm}[H]
\DontPrintSemicolon
\SetAlgoLined
\KwResult{Anomalous arm prediction $i$}
Input: $K\in \mathbb{N}$, $b>0$ \;
Initialization: $Y=0$, $i=1$\; 
% Initialization: $Y^i = \log \frac{\effone(X^i_1)}{\effzero(X^i_1)}$ and $\hat{\mu}^i = \frac{1}{2\sqrt{\mu}} Y^i$ and $n^i = 1$ for $i=1,2,\dots, K$; and $t=K$\;
 \While{$ Y \le b$}{
  \If{$ Y < 0$}{
 $i = (i < K)\times(i+1) + (i==K)$\;
 }
 Sample $X\sim F_{i==K}$ \;
 $Y = \max\{Y, 0\} + \log\frac{\effone(X)}{\effzero(X)}$ \;
}
 \caption{\label{alg:linear-qs} \laipoor{}}
\end{algorithm}
\end{figure}

We compare the two algorithms \DIFdelbegin \DIFdel{for $\lambda=2$, }\DIFdelend \DIFaddbegin \DIFadd{in two settings: one ``hard'' and one ``easy.''  For the hard setting, we set $\lambda=1$, }\DIFaddend $\mu=0.1$, and $K=22$\DIFdelbegin \DIFdel{.  For each of several thresholds }\DIFdelend \DIFaddbegin \DIFadd{, in which case $\Delta \approx 0.05$.  This is similar to, but more difficult than, the hard setting in \mbox{%DIFAUXCMD
\cite{rouyer2021algorithm}}\hspace{0pt}%DIFAUXCMD
, in which $\Delta$ is $0.05$ and $K=8$.  For the easy setting, we set $\lambda=0.025$, $\mu=0.4$, and $K=8$, in which case $\Delta \approx 0.2$.  This is analogous to the easy setting in \mbox{%DIFAUXCMD
\cite{rouyer2021algorithm}}\hspace{0pt}%DIFAUXCMD
.  
%DIF >  (For $\mu = 0.1$, $\Delta \approx 0.0528$, which is considered the ``hard'' case in \cite{rouyer2021algorithm}.)  
For each setting and several values of the threshold }\DIFaddend $b$, 10,000 Monte Carlo simulations are \DIFdelbegin \DIFdel{used }\DIFdelend \DIFaddbegin \DIFadd{conducted }\DIFaddend to estimate the false-alarm probability and expected stopping time.  The results, in \DIFdelbegin \DIFdel{Figure~\ref{fig:ROC} }\DIFdelend \DIFaddbegin \DIFadd{Figures~\ref{fig:ROC} and }\fig{fig:ROC-easy}\DIFaddend , indicate a savings of roughly \DIFdelbegin \DIFdel{700 samples for low false-alarm rates.
}\DIFdelend \DIFaddbegin \DIFadd{$1/3$ to $1/2$ of the samples needed for a decision.
%DIF > 700 samples for low false-alarm rates.
}\DIFaddend 

\begin{figure}
    \centering
    \DIFdelbeginFL %DIFDELCMD < \includegraphics[width=\linewidth]{ROCbandit-crop.pdf}
%DIFDELCMD <     %%%
\DIFdelendFL \DIFaddbeginFL \includegraphics[width=\linewidth]{ROCbandit-hard-crop.pdf}
    \DIFaddendFL \caption{Comparison with $K=22$, $\mu=0.1$, \DIFdelbeginFL \DIFdelFL{$\lambda=2$}\DIFdelendFL \DIFaddbeginFL \DIFaddFL{$\lambda=1$}\DIFaddendFL , and 10,000 Monte Carlo trials per threshold $b$.}
    \label{fig:ROC}
\end{figure}
\DIFaddbegin 

\begin{figure}
    \centering
    \includegraphics[width=\linewidth]{ROCbandit-easy-crop.pdf}
    \caption{\DIFaddFL{Comparison with $K=8$, $\mu=0.4$, $\lambda=0.025$, and 10,000 Monte Carlo trials per threshold $b$.}}
    \label{fig:ROC-easy}
\end{figure}

\section{\DIFadd{Conclusion}}

\DIFadd{In this paper we have proposed a method for anomaly search using recent developments in multi-armed bandit theory.  We have shown through simulations that the method appears to dominate round-robin methods that are common in the anomaly-search literature.  Further, we have shown in Theorem~\ref{eq:thm-main} that the false-alarm rate of our method is bounded above by a nontrivial quantity that is numerically efficient to compute in terms of the detection threshold $b$.
}

\DIFadd{A natural question for future research is whether a similar bound can be obtained for the expected stopping time.  Although we have given this question some consideration, the answer appears to be a challenge: for example, summations involving the number of samples taken during various intervals appear not to be analyzable using Wald's identity, due to the dependency of the later samples on earlier ones.  Another question is whether the function $\phi(n)$ in Theorem~\ref{eq:thm-main} is order-optimal.  That is, could it be replaced with a slower-growing function, such as a logarithmic function.  And if $\phi(n)$ is order-optimal, it would be beneficial to optimize constants such as $C$ subject to computational constraints.
}\DIFaddend 

% use section* for acknowledgment
\section*{Acknowledgements}
This work was supported by the United States Air Force Research Lab. %and AFOSR grant 22RYCOR007. 
However, the views and opinions expressed in this article are those of the authors and do not necessarily reflect the official policy or position of any agency of the U.S. government. Examples of analysis performed within this article are only examples. Assumptions made within the analysis are also not reflective of the position of any U.S. Government entity. The Public Affairs approval number of this document is ?? %AFRL-2021-4155.




% \IEEEPARstart{T}{his} document is a template for \LaTeX. If you are reading a paper or PDF version of this document, please download the electronic file, trans\_jour.tex, from the IEEE Web site at \url{http://www.ieee.org/authortools/trans_jour.tex} so you can use it to prepare your manuscript. If you would prefer to use LaTeX, download IEEE's LaTeX style and sample files from the same Web page. You can also explore using the Overleaf editor at {https://www.overleaf.com/blog/278-how-to-use-overleaf-with-ieee-collabratec-your-quick-guide-to-getting-started\#.xsVp6tpPkrKM9}

% If your paper is intended for a conference, please contact your conference editor concerning acceptable word processor formats for your particular conference.  


% \section{Guidelines For Manuscript Preparation}


% The IEEEtran\_HOWTO.pdf is the complete guide of \LaTeX\ for manuscript preparation included with this stuff. 


% \subsection{Information for Authors}

% {\em IEEE Signal Processing Letters} allows only four-page articles. A fifth page is allowed for ``References'' only, though ``References'' may begin before the fifth page. Author biographies or photographs are not allowed in Signal Processing Letters. Please review the Information for Authors at for {\em IEEE Signal Processing Letters:} https://signalprocessingsociety.org/publications-resources/ieee-signal-processing-letters/information-authors-spl



% \section{Guidelines for Graphics Preparation and Submission}
% \label{sec:guidelines}

% \subsection{Types of Graphics}
% The following list outlines the different types of graphics published in 
% {\it IEEE Signal Processing Letters}. They are categorized based on their construction, and use of 
% color/shades of gray:

% \subsubsection{Color/Grayscale figures}
% {Figures that are meant to appear in color, or shades of black/gray. Such 
% figures may include photographs, illustrations, multicolor graphs, and 
% flowcharts.}

% \subsubsection{Line Art figures}
% {Figures that are composed of only black lines and shapes. These figures 
% should have no shades or half-tones of gray, only black and white.}

% \subsubsection{Tables}
% {Data charts which are typically black and white, but sometimes include 
% color.}



% \subsection{Multipart figures}
% Figures compiled of more than one sub-figure presented side-by-side, or 
% stacked. If a multipart figure is made up of multiple figure
% types (one part is lineart, and another is grayscale or color) the figure 
% should meet the stricter guidelines.

% \subsection{File Formats For Graphics}\label{formats}
% Format and save your graphics using a suitable graphics processing program 
% that will allow you to create the images as PostScript (PS), Encapsulated 
% PostScript (.EPS), Tagged Image File Format (.TIFF), Portable Document 
% Format (.PDF), Portable Network Graphics (.PNG), or Metapost (.MPS), sizes them, and adjusts 
% the resolution settings. When 
% submitting your final paper, your graphics should all be submitted 
% individually in one of these formats along with the manuscript.

% \subsection{Sizing of Graphics}
% Most charts, graphs, and tables are one column wide (3.5 inches/88 
% millimeters/21 picas) or page wide (7.16 inches/181 millimeters/43 
% picas). The maximum depth a graphic can be is 8.5 inches (216 millimeters/54
% picas). When choosing the depth of a graphic, please allow space for a 
% caption. Figures can be sized between column and page widths if the author 
% chooses, however it is recommended that figures are not sized less than 
% column width unless when necessary. 

% \begin{figure}
% \centerline{\includegraphics[width=\columnwidth]{fig1.png}}
% \caption{Magnetization as a function of applied field. Note that ``Fig.'' is abbreviated. There is a period after the figure number, followed by two spaces. It is good practice to explain the significance of the figure in the caption.}
% \end{figure}

% \begin{table}
% \caption{Units for Magnetic Properties}
% \label{table}
% \small
% \setlength{\tabcolsep}{3pt}
% \begin{tabular}{|p{25pt}|p{75pt}|p{110pt}|}
% \hline
% Symbol& 
% Quantity& 
% Conversion from Gaussian and \par CGS EMU to SI$^{\mathrm{a}}$ \\
% \hline
% $\Phi $& 
% Magnetic flux& 
% 1 Mx $\to  10^{-8}$ Wb $= 10^{-8}$ V $\cdot$ s \\
% $B$& 
% Magnetic flux density, \par magnetic induction& 
% 1 G $\to  10^{-4}$ T $= 10^{-4}$ Wb/m$^{2}$ \\
% $H$& 
% Magnetic field strength& 
% 1 Oe $\to  10^{-3}/(4\pi )$ A/m \\
% $m$& 
% Magnetic moment& 
% 1 erg/G $=$ 1 emu \par $\to 10^{-3}$ A $\cdot$ m$^{2} = 10^{-3}$ J/T \\
% $M$& 
% Magnetization& 
% 1 erg/(G $\cdot$ cm$^{3}) =$ 1 emu/cm$^{3}$ \par $\to 10^{-3}$ A/m \\
% 4$\pi M$& 
% Magnetization& 
% 1 G $\to  10^{-3}/(4\pi )$ A/m \\
% $\sigma $& 
% Specific magnetization& 
% 1 erg/(G $\cdot$ g) $=$ 1 emu/g $\to $ 1 A $\cdot$ m$^{2}$/kg \\
% $j$& 
% Magnetic dipole \par moment& 
% 1 erg/G $=$ 1 emu \par $\to 4\pi \times  10^{-10}$ Wb $\cdot$ m \\
% $J$& 
% Magnetic polarization& 
% 1 erg/(G $\cdot$ cm$^{3}) =$ 1 emu/cm$^{3}$ \par $\to 4\pi \times  10^{-4}$ T \\
% $\chi , \kappa $& 
% Susceptibility& 
% 1 $\to  4\pi $ \\
% $\chi_{\rho }$& 
% Mass susceptibility& 
% 1 cm$^{3}$/g $\to  4\pi \times  10^{-3}$ m$^{3}$/kg \\
% $\mu $& 
% Permeability& 
% 1 $\to  4\pi \times  10^{-7}$ H/m \par $= 4\pi \times  10^{-7}$ Wb/(A $\cdot$ m) \\
% $\mu_{r}$& 
% Relative permeability& 
% $\mu \to \mu_{r}$ \\
% $w, W$& 
% Energy density& 
% 1 erg/cm$^{3} \to  10^{-1}$ J/m$^{3}$ \\
% $N, D$& 
% Demagnetizing factor& 
% 1 $\to  1/(4\pi )$ \\
% \hline
% \multicolumn{3}{p{251pt}}{Vertical lines are optional in tables. Statements that serve as captions for 
% the entire table do not need footnote letters. }\\
% \multicolumn{3}{p{251pt}}{$^{\mathrm{a}}$Gaussian units are the same as cg emu for magnetostatics; Mx 
% $=$ maxwell, G $=$ gauss, Oe $=$ oersted; Wb $=$ weber, V $=$ volt, s $=$ 
% second, T $=$ tesla, m $=$ meter, A $=$ ampere, J $=$ joule, kg $=$ 
% kilogram, H $=$ henry.}
% \end{tabular}
% \label{tab1}
% \end{table}


% \subsection{Resolution }
% The proper resolution of your figures will depend on the type of figure it 
% is as defined in the ``Types of Figures'' section. Author photographs, 
% color, and grayscale figures should be at least 300dpi. Line art, including 
% tables should be a minimum of 600dpi.

% \subsection{Vector Art}
% In order to preserve the figures' integrity across multiple computer 
% platforms, we accept files in the following formats: .EPS/.PDF/.PS. All 
% fonts must be embedded or text converted to outlines in order to achieve the 
% best-quality results.


% \subsection{Accepted Fonts Within Figures}
% When preparing your graphics IEEE suggests that you use of one of the 
% following Open Type fonts: Times New Roman, Helvetica, Arial, Cambria, and 
% Symbol. If you are supplying EPS, PS, or PDF files all fonts must be 
% embedded. Some fonts may only be native to your operating system; without 
% the fonts embedded, parts of the graphic may be distorted or missing.

% A safe option when finalizing your figures is to strip out the fonts before 
% you save the files, creating ``outline'' type. This converts fonts to 
% artwork what will appear uniformly on any screen.

% \subsection{Using Labels Within Figures}

% \subsubsection{Figure Axis labels }
% Figure axis labels are often a source of confusion. Use words rather than 
% symbols. As an example, write the quantity ``Magnetization,'' or 
% ``Magnetization M,'' not just ``M.'' Put units in parentheses. Do not label 
% axes only with units. As in Fig. 1, for example, write ``Magnetization 
% (A/m)'' or ``Magnetization (A$\cdot$m$^{-1}$),'' not just ``A/m.'' Do not label axes with a ratio of quantities and 
% units. For example, write ``Temperature (K),'' not ``Temperature/K.'' 

% Multipliers can be especially confusing. Write ``Magnetization (kA/m)'' or 
% ``Magnetization (10$^{3}$ A/m).'' Do not write ``Magnetization 
% (A/m)$\,\times\,$1000'' because the reader would not know whether the top 
% axis label in Fig. 1 meant 16000 A/m or 0.016 A/m. Figure labels should be 
% legible, approximately 8 to 10 point type.

% \subsubsection{Subfigure Labels in Multipart Figures and Tables}
% Multipart figures should be combined and labeled before final submission. 
% Labels should appear centered below each subfigure in 8 point Times New 
% Roman font in the format of (a) (b) (c). 

% \subsection{File Naming}
% Figures (line artwork or photographs) should be named starting with the 
% first 5 letters of the author's last name. The next characters in the 
% filename should be the number that represents the sequential 
% location of this image in your article. For example, in author 
% ``Anderson's'' paper, the first three figures would be named ander1.tif, 
% ander2.tif, and ander3.ps.

% Tables should contain only the body of the table (not the caption) and 
% should be named similarly to figures, except that `.t' is inserted 
% in-between the author's name and the table number. For example, author 
% Anderson's first three tables would be named ander.t1.tif, ander.t2.ps, 
% ander.t3.eps.

% \subsection{Referencing a Figure or Table Within Your Paper}
% When referencing your figures and tables within your paper, use the 
% abbreviation ``Fig.'' even at the beginning of a sentence. Do not abbreviate 
% ``Table.'' Tables should be numbered with Roman Numerals.

% \subsection{Checking Your Figures: The IEEE Graphics Analyzer}
% The IEEE Graphics Analyzer enables authors to pre-screen their graphics for 
% compliance with IEEE Transactions and Journals standards before submission. 
% The online tool, located at
% \underline{http://graphicsqc.ieee.org/}, allows authors to 
% upload their graphics in order to check that each file is the correct file 
% format, resolution, size and colorspace; that no fonts are missing or 
% corrupt; that figures are not compiled in layers or have transparency, and 
% that they are named according to the IEEE Transactions and Journals naming 
% convention. At the end of this automated process, authors are provided with 
% a detailed report on each graphic within the web applet, as well as by 
% email.

% For more information on using the Graphics Analyzer or any other graphics 
% related topic, contact the IEEE Graphics Help Desk by e-mail at 
% graphics@ieee.org.

% \subsection{Submitting Your Graphics}
% Because IEEE will do the final formatting of your paper,
% you do not need to position figures and tables at the top and bottom of each 
% column. In fact, all figures, figure captions, and tables can be placed at 
% the end of your paper. In addition to, or even in lieu of submitting figures 
% within your final manuscript, figures should be submitted individually, 
% separate from the manuscript in one of the file formats listed above in 
% Section \ref{formats}. Place figure captions below the figures; place table titles 
% above the tables. Please do not include captions as part of the figures, or 
% put them in ``text boxes'' linked to the figures. Also, do not place borders 
% around the outside of your figures.

% \subsection{Color Processing/Printing in IEEE Journals}
% All IEEE Transactions, Journals, and Letters allow an author to publish 
% color figures on IEEE Xplore\textregistered\ at no charge, and automatically 
% convert them to grayscale for print versions. In most journals, figures and 
% tables may alternatively be printed in color if an author chooses to do so. 
% Please note that this service comes at an extra expense to the author. If 
% you intend to have print color graphics, include a note with your final 
% paper indicating which figures or tables you would like to be handled that 
% way, and stating that you are willing to pay the additional fee.


% \section{Conclusion}

% A conclusion section is not required. Although a conclusion may review the main points of the paper, do not replicate the abstract as the conclusion. A conclusion might elaborate on the importance of the work or suggest applications and extensions. 

% \section*{Acknowledgment}

% The preferred spelling of the word ``acknowledgment'' in American English is without an ``e'' after the ``g.'' Use the singular heading even if you have many acknowledgments. Avoid expressions such as ``One of us (S.B.A.) would like to thank . . . .'' Instead, write “F. A. Author thanks ... .” In most cases, sponsor and financial support acknowledgments are placed in the unnumbered footnote on the first page, not here.

% \section*{References and Footnotes}

% \subsection{References}

% References need not be cited in text. When they are, they appear on the line, in square brackets, inside the punctuation.  Multiple references are each numbered with separate brackets. When citing a section in a book, please give the relevant page numbers. In text, refer simply to the reference number. Do not use ``Ref.'' or ``reference'' except at the beginning of a sentence: ``Reference [3] shows . . . .'' Please do not use automatic endnotes in {\em Word}, rather, type the reference list at the end of the paper using the ``References'' style.

% Reference numbers are set flush left and form a column of their own, hanging out beyond the body of the reference. The reference numbers are on the line, enclosed in square brackets. In all references, the given name of the author or editor is abbreviated to the initial only and precedes the last name. Use them all; use {\em et al.} only if names are not given. Use commas around Jr., Sr., and III in names. Abbreviate conference titles.  When citing IEEE transactions, provide the issue number, page range, volume number, year, and/or month if available. When referencing a patent, provide the day and the month of issue, or application. References may not include all information; please obtain and include relevant information. Do not combine references. There must be only one reference with each number. If there is a URL included with the print reference, it can be included at the end of the reference.

% Other than books, capitalize only the first word in a paper title, except for proper nouns and element symbols. For papers published in translation journals, please give the English citation first, followed by the original foreign-language citation. See the end of this document for formats and examples of common references. For a complete discussion of references and their formats, see the IEEE style manual at www.ieee.org/authortools.

% \subsection{Footnotes}

% Number footnotes separately in superscripts (Insert $\mid$ Footnote).\footnote{It is recommended that footnotes be avoided (except for the unnumbered footnote with the receipt date on the first page). Instead, try to integrate the footnote information into the text.}  Place the actual footnote at the bottom of the column in which it is cited; do not put footnotes in the reference list (endnotes). Use letters for table footnotes (see Table I). 


% \section*{References}

\bibliographystyle{plain}
\bibliography{information-geometry-bib2}

% \subsection*{Basic format for books:}

% J. K. Author, ``Title of chapter in the book,'' in {\em Title of His Published Book}, xth ed. City of Publisher, (only U.S. State), Country: Abbrev. of Publisher, year, ch. x, sec. x, pp. xxx--xxx.

% \subsection*{Examples:}
% \def\refname{}
% \begin{thebibliography}{34}

% \bibitem{}G. O. Young, ``Synthetic structure of industrial plastics,'' in {\em Plastics}, 2nd ed., vol. 3, J. Peters, Ed. New York, NY, USA: McGraw-Hill, 1964, pp. 15--64.

% \bibitem{}W.-K. Chen, {\it Linear Networks and Systems}. Belmont, CA, USA: Wadsworth, 1993, pp. 123--135.

% \end{thebibliography}

% \subsection*{Basic format for periodicals:}

% J. K. Author, ``Name of paper,'' Abbrev. Title of Periodical, vol. x,   no. x, pp. xxx--xxx, Abbrev. Month, year, DOI. 10.1109.XXX.123--456.

% \subsection*{Examples:}

% \begin{thebibliography}{34}
% \setcounter{enumiv}{2}

% \bibitem{}J. U. Duncombe, ``Infrared navigation Part I: An assessment of feasibility,'' {\em IEEE Trans. Electron Devices}, vol. ED-11, no. 1, pp. 34--39, Jan. 1959,10.1109/TED.2016.2628402.

% \bibitem{}E. P. Wigner, ``Theory of traveling-wave optical laser,''
% {\em Phys. Rev.},  vol. 134, pp. A635--A646, Dec. 1965.

% \bibitem{}E. H. Miller, ``A note on reflector arrays,'' {\em IEEE Trans. Antennas Propagat.}, to be published.
% \end{thebibliography}


% \subsection*{Basic format for reports:}

% J. K. Author, ``Title of report,'' Abbrev. Name of Co., City of Co., Abbrev. State, Country, Rep. xxx, year.

% \subsection*{Examples:}
% \begin{thebibliography}{34}
% \setcounter{enumiv}{5}

% \bibitem{} E. E. Reber, R. L. Michell, and C. J. Carter, ``Oxygen absorption in the earth’s atmosphere,'' Aerospace Corp., Los Angeles, CA, USA, Tech. Rep. TR-0200 (4230-46)-3, Nov. 1988.

% \bibitem{} J. H. Davis and J. R. Cogdell, ``Calibration program for the 16-foot antenna,'' Elect. Eng. Res. Lab., Univ. Texas, Austin, TX, USA, Tech. Memo. NGL-006-69-3, Nov. 15, 1987.
% \end{thebibliography}

% \subsection*{Basic format for handbooks:}

% {\em Name of Manual/Handbook}, x ed., Abbrev. Name of Co., City of Co., Abbrev. State, Country, year, pp. xxx--xxx.

% \subsection*{Examples:}

% \begin{thebibliography}{34}
% \setcounter{enumiv}{7}

% \bibitem{} {\em Transmission Systems for Communications}, 3rd ed., Western Electric Co., Winston-Salem, NC, USA, 1985, pp. 44--60.

% \bibitem{} {\em Motorola Semiconductor Data Manual}, Motorola Semiconductor Products Inc., Phoenix, AZ, USA, 1989.
% \end{thebibliography}

% \subsection*{Basic format for books (when available online):}

% J. K. Author, ``Title of chapter in the book,'' in {\em Title of Published Book}, xth ed. City of Publisher, State, Country: Abbrev. of Publisher, year, ch. x, sec. x, pp. xxx xxx. [Online]. Available: http://www.web.com 

% \subsection*{Examples:}

% \begin{thebibliography}{34}
% \setcounter{enumiv}{9}

% \bibitem{}G. O. Young, ``Synthetic structure of industrial plastics,'' in Plastics, vol. 3, Polymers of Hexadromicon, J. Peters, Ed., 2nd ed. New York, NY, USA: McGraw-Hill, 1964, pp. 15--64. [Online]. Available: http://www.bookref.com. 

% \bibitem{} {\em The Founders Constitution}, Philip B. Kurland and Ralph Lerner, eds., Chicago, IL, USA: Univ. Chicago Press, 1987. [Online]. Available: http://press-pubs.uchicago.edu/founders/

% \bibitem{} The Terahertz Wave eBook. ZOmega Terahertz Corp., 2014. [Online]. Available: http://dl.z-thz.com/eBook/zomega\_ebook\_pdf\_1206\_sr.pdf. Accessed on: May 19, 2014. 

% \bibitem{} Philip B. Kurland and Ralph Lerner, eds., {\em The Founders Constitution}. Chicago, IL, USA: Univ. of Chicago Press, 1987, Accessed on: Feb. 28, 2010, [Online] Available: http://press-pubs.uchicago.edu/founders/ 
% \end{thebibliography}

% \subsection*{Basic format for journals (when available online):}

% J. K. Author, ``Name of paper,'' {\em Abbrev. Title of Periodical}, vol. x, no. x, pp. xxx--xxx, Abbrev. Month, year. Accessed on: Month, Day, year, doi: 10.1109.XXX.123456, [Online].

% \subsection*{Examples:}

% \begin{thebibliography}{34}
% \setcounter{enumiv}{13}

% \bibitem{}J. S. Turner, ``New directions in communications,'' {\em IEEE J. Sel. Areas Commun.}, vol. 13, no. 1, pp. 11--23, Jan. 1995. 

% \bibitem{} W. P. Risk, G. S. Kino, and H. J. Shaw, ``Fiber-optic frequency shifter using a surface acoustic wave incident at an oblique angle,'' {\em Opt. Lett.}, vol. 11, no. 2, pp. 115--117, Feb. 1986.

% \bibitem{} P. Kopyt {\em et al.}, ``Electric properties of graphene-based conductive layers from DC up to terahertz range,'' {\em IEEE THz Sci. Technol.}, to be published. doi: 10.1109/TTHZ.2016.2544142.
% \end{thebibliography}

% \subsection*{Basic format for papers presented at conferences (when available online):}

% J.K. Author. (year, month). Title. presented at abbrev. conference title. [Type of Medium]. Available: site/path/file

% \subsection*{Example:}

% \begin{thebibliography}{34}
% \setcounter{enumiv}{16}

% \bibitem{}PROCESS Corporation, Boston, MA, USA. Intranets: Internet technologies deployed behind the firewall for corporate productivity. Presented at INET96 Annual Meeting. [Online]. Available: http://home.process.com/Intranets/wp2.htp
% \end{thebibliography}

% \subsection*{Basic format for reports  and  handbooks (when available online):}

% J. K. Author. ``Title of report,'' Company. City, State, Country. Rep. no., (optional: vol./issue), Date. [Online] Available: site/path/file 

% \subsection*{Examples:}

% \begin{thebibliography}{34}
% \setcounter{enumiv}{17}

% \bibitem{}R. J. Hijmans and J. van Etten, ``Raster: Geographic analysis and modeling with raster data,'' R Package Version 2.0-12, Jan. 12, 2012. [Online]. Available: http://CRAN.R-project.org/package=raster 

% \bibitem{}Teralyzer. Lytera UG, Kirchhain, Germany [Online]. Available: http://www.lytera.de/Terahertz\_THz\_Spectroscopy.php?id=home, Accessed on: Jun. 5, 2014.
% \end{thebibliography}

% \subsection*{Basic format for computer programs and electronic documents (when available online):}

% Legislative body. Number of Congress, Session. (year, month day). {\em Number of bill or resolution, Title}. [Type of medium]. Available: site/path/file
% {\em NOTE:} ISO recommends that capitalization follow the accepted practice for the language or script in which the information is given.

% \subsection*{Example:}

% \begin{thebibliography}{34}
% \setcounter{enumiv}{19}

% \bibitem{}U. S. House. 102nd Congress, 1st Session. (1991, Jan. 11). {\em H. Con. Res. 1, Sense of the Congress on Approval of Military Action}. [Online]. Available: LEXIS Library: GENFED File: BILLS 
% \end{thebibliography}

% \subsection*{Basic format for patents (when available online):}

% Name of the invention, by inventor’s name. (year, month day). Patent Number [Type of medium]. Available:site/path/file

% \subsection*{Example:}

% \begin{thebibliography}{34}
% \setcounter{enumiv}{20}

% \bibitem{}Musical tooth brush with mirror, by L. M. R. Brooks. (1992, May 19). Patent D 326 189
% [Online]. Available: NEXIS Library: LEXPAT File:   DES 

% \end{thebibliography}

% \subsection*{Basic format for conference proceedings (published):}

% J. K. Author, ``Title of paper,'' in {\em Abbreviated Name of Conf.}, City of Conf., Abbrev. State (if given), Country, year, pp. xxx--xxx.

% \subsection*{Example:}

% \begin{thebibliography}{34}
% \setcounter{enumiv}{21}

% \bibitem{}D. B. Payne and J. R. Stern, ``Wavelength-switched passively coupled single-mode optical network,'' in {\em Proc. IOOC-ECOC}, Boston, MA, USA, 1985,
% pp. 585--590.

% \end{thebibliography}

% \subsection*{Example for papers presented at conferences (unpublished):}

% \begin{thebibliography}{34}
% \setcounter{enumiv}{22}

% \bibitem{}D. E behard and E. Voges, ``Digital single sideband detection for inter ferometric sensors,'' presented at the {\em 2nd Int. Conf. Optical Fiber Sensors}, Stuttgart, Germany, Jan. 2--5, 1984.
% \end{thebibliography}

% \subsection*{Basic formatfor patents:}

% J. K. Author, ``Title of patent,'' U. S. Patent x xxx xxx, Abbrev. Month, day, year.

% \subsection*{Example:}

% \begin{thebibliography}{34}
% \setcounter{enumiv}{23}

% \bibitem{}G. Brandli and M. Dick, ``Alternating current fed power supply,'' U. S. Patent 4 084 217, Nov. 4, 1978.
% \end{thebibliography}

% \subsection*{Basic format for theses (M.S.) and dissertations (Ph.D.):}

% a) J. K. Author, ``Title of thesis,'' M. S. thesis, Abbrev. Dept., Abbrev. Univ., City of Univ., Abbrev. State, year.

% b) J. K. Author, ``Title of dissertation,'' Ph.D. dissertation, Abbrev. Dept., Abbrev. Univ., City of Univ., Abbrev. State, year.

% \subsection*{Examples:}

% \begin{thebibliography}{34}
% \setcounter{enumiv}{24}

% \bibitem{}J. O. Williams, ``Narrow-band analyzer,'' Ph.D. dissertation, Dept. Elect. Eng., Harvard Univ., Cambridge, MA, USA, 1993.

% \bibitem{}N. Kawasaki, ``Parametric study of thermal and chemical nonequilibrium nozzle flow,'' M.S. thesis, Dept. Electron. Eng., Osaka Univ., Osaka, Japan, 1993.
% \end{thebibliography}

% \subsection*{Basic format for the most common types of unpublished references:}

% a) J. K. Author, private communication, Abbrev. Month, year.

% b) J. K. Author, ``Title of paper,'' unpublished.

% c) J. K. Author, ``Title of paper,'' to be published.

% \subsection*{Examples:}

% \begin{thebibliography}{34}
% \setcounter{enumiv}{26}

% \bibitem{}A. Harrison, private communication, May 1995.

% \bibitem{}B. Smith, ``An approach to graphs of linear forms,'' unpublished.

% \bibitem{}A. Brahms, ``Representation error for real numbers in binary computer arithmetic,'' IEEE Computer Group Repository, Paper R-67-85.
% \end{thebibliography}

% \subsection*{Basic formats for standards:}

% a) {\em Title of Standard}, Standard number, date.

% b) {\em Title of Standard}, Standard number, Corporate author, location, date.

% \subsection*{Examples:}

% \begin{thebibliography}{34}
% \setcounter{enumiv}{29}


% \bibitem{}IEEE Criteria for Class IE Electric Systems, IEEE Standard 308, 1969.

% \bibitem{} Letter Symbols for Quantities, ANSI Standard Y10.5-1968.
% \end{thebibliography}

% \subsection*{Article number in reference examples:}

% \begin{thebibliography}{34}
% \setcounter{enumiv}{31}

% \bibitem{}R. Fardel, M. Nagel, F. Nuesch, T. Lippert, and A. Wokaun, ``Fabrication of organic light emitting diode pixels by laser-assisted forward transfer,'' {\em Appl. Phys. Lett.}, vol. 91, no. 6, Aug. 2007, Art. no. 061103. 

% \bibitem{} J. Zhang and N. Tansu, ``Optical gain and laser characteristics of InGaN quantum wells on ternary InGaN substrates,'' {\em IEEE Photon.} J., vol. 5, no. 2, Apr. 2013, Art. no. 2600111
% \end{thebibliography}

% \subsection*{Example when using et al.:}

% \begin{thebibliography}{34}
% \setcounter{enumiv}{33}

% \bibitem{}S. Azodolmolky {\em et al.}, Experimental demonstration of an impairment aware network planning and operation tool for transparent/translucent optical networks,'' {\em J. Lightw. Technol.}, vol. 29, no. 4, pp. 439--448, Sep. 2011.
% \end{thebibliography}

\end{document}
