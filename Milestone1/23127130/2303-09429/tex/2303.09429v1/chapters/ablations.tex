\section{Ablation Study}
\label{sec:ablations}
In this section we ablate various key-components of our approach to examine their impact on performance. 
\Cref{tab:fiq_ablations} reports ablation results on the frequently used \fashioniq dataset. We observed similar trends also on CIRR (see suppl. material). 
We train \our without the reverse queries objective described in \Cref{sec:method_reverse_queries} (No-RQ). We observe that reverse queries improve performance by $0.5$--$1.2$\% absolute points ($\sim\!\!5$\% relative performance boost at R@1). Using surrogate Recall@K loss instead of common contrastive loss, further improves results by roughly $0.5\%$ absolute points. Finally, we examine the influence of fine-tuning our ViT parameters. We observe improvement at higher K values (R@10, R@50) that trades-off with lower R@K metrics (R@1, R@5). However, this experiment is dataset-dependent.
%: training on \fashioniq with 25.1K images (see \cref{tab:datasets}) result a R@K trade-off between different $K$ (as described above).
We observed that when training on \ourDS, with 81.6K images, and testing on the \ourDS test set, all metrics were improved by absolute $0.4-2\%$.
% ----- CVPR ver ------
% In this section we ablate various key-components of our approach to examine their impact on performance. \Cref{tab:fiq_ablations} reports the ablation results on the frequently used \fashioniq dataset.
% First, we train \our without the reverse queries objective described in \cref{sec:method_reverse_queries} (No-RQ). We observe that reverse queries improve performance by $0.5$--$1.2$\% absolute points ($\sim\!\!5$\% relative performance boost at R@1). Using surrogate Recall@K loss instead of common contrastive loss, further improves results by roughly $0.5\%$ absolute points. Finally, we examine the influence of fine-tuning our ViT parameters. We observe improvement at higher K values (R@10, R@50) that trades-off with lower R@K metrics (R@1, R@5). However, this experiment is dataset-dependent.
% %: training on \fashioniq with 25.1K images (see \cref{tab:datasets}) result a R@K trade-off between different $K$ (as described above).
% We observed that when training on \ourDS, with 81.6K images, and testing on the \ourDS test set, all metrics were improved by absolute $0.4-2\%$.

% Next, we examine the impact of our early fusion approach by building an alternative baseline with late fusion also based on BLIP (dubbed LF-B) and show the results in \Cref{tab:fiq_ablations}. Following the LF-C model presented by Baldrati \etal \mbox{\cite{cclip}}, LF-B separately encodes each modality to a global feature vector (but with BLIP encoders), then fuses both via a learnable MLP.
% The LF-B baseline shows poor results, implying the effectiveness of using our early fusion approach with cross-attention. 
% -------------------
\begin{table}[t]
\begin{center}
\resizebox{0.8\columnwidth}{!}{%
\begin{tabular}{@{}llllll@{}}
\toprule
 & R@1 & R@5 & R@10 & R@50 & R@100 \\ \midrule
% Rand Init. &  0.2 & 0.98 & 1.73 & 6.65 & 11.64 & 36.77 \\
% Image-only &  1.51 & 4.85 & 7.33 & 18.33 & 25.2 & 48.37 \\
% Text-only &  13.16 & 26.68 & 35.09 & 57.38 & 66.82 & 86.12 \\
LF-BLIP &  8.66 & 19.33 & 25.75 & 43.98 & 52.61 \\
% \bottomrule
No-RQ &  19.98 & 38.36 & 47.57 & {70.28} & 77.68 \\
% No Aug. (txt) &  19.83 & 38.71 & 47.81 & 70.25 & 77.84 & 91.87 \\
Contrastive Loss &  20.45 & 38.96 & 48.04 & 69.98 & {78.06} \\
% No Prompt &  20.61 & 38.7 & {48.64} & 70.01 & 77.91 & 92.34 \\
Freeze ViT &  \textbf{21.03} & \textbf{39.43} & 48.25 & 69.85 & 77.96\\
\our &  {20.69} & {39.38} & \textbf{48.79} & \textbf{70.68} & \textbf{78.54}  \\
\bottomrule
\end{tabular}
}
\end{center}
\caption{Ablation study for different configurations of \our, conducted on \fashioniq.}
\label{tab:fiq_ablations}
\vspace{-1em}
\end{table}
% \begin{table}[ht]
\begin{center}
\resizebox{\columnwidth}{!}{%
\begin{tabular}{@{}lllllll@{}}
\toprule
 & R@1 & R@5 & R@10 & R@50 & R@100 & R@500 \\ \midrule
LF-BLIP & 21.96 & 50.16 & 64.05 & 85.67 & 90.62 & 98.04 \\
No-RQ & 47.50 & 80.00 & 88.85 & 97.35 & 98.71 & 99.86 \\
% Add Image Aug. &  47.98 & 80.22 & 88.71 & 97.27 & {\ul98.64} & 99.78 \\
Contrastive Loss & 48.00 & 80.08 & {\bf 89.07} & {\bf 97.56} & {\bf 98.92} & 99.86 \\
Freeze ViT & {\bf 48.07} & 79.60 & 88.52 & 97.27 & 98.59 & 99.78\\
% No Img Mask &  {\ul48.43} & {\ul80.46} & \textbf{89.05} & {\ul97.39} & {\ul98.64} & 99.78 \\
\our & { 47.96} & {\bf 80.65} & 88.88 & 97.46 & 98.78 & 99.81 \\
\bottomrule
\end{tabular}
}
\end{center}
\caption{Ablation study on CIRR validation dataset.}
\label{tab:cirr_ablations}
\end{table}
% \begin{table}[ht]
\begin{center}
\resizebox{\columnwidth}{!}{%
\begin{tabular}{lcccccc|c}
\hline
Dataset & R@1 & R@5 & R@10 & R@50 & R@100 & \multicolumn{1}{l|}{R@500} & \multicolumn{1}{l}{\%CT2I} \\ \hline
COCO & 32.69 & 57.76 & 68.23 & 90.07 & 96.21 & 99.56 & 60.17 \\
CIRR & 19.97 & 44.61 & 57.07 & 80.12 & 86.70 & 97.03 & 48.23 \\
FashionIQ & 6.07 & 12.55 & 16.84 & 31.85 & 41.57 & 66.74 & 16.82 \\
LaSCo & \textbf{1.47} & \textbf{3.81} & \textbf{5.89} & \textbf{14.56} & \textbf{20.89} & \textbf{45.64} & \textbf{6.61} \\ \hline
\end{tabular}
}
\end{center}
\caption{Results of CLIP Text-to-Image (CT2I) baseline on several datasets. Retrieval on datasets that require compositional capabilities is expected to be low, since query-text is used. \%CT2I score is the average of R@\{1,10,50\}.}
\label{tab:tti_clip}
\end{table}
% ======================