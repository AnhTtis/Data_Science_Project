\documentclass[10pt,twocolumn,letterpaper]{article}

\usepackage{iccv}
\usepackage{times}
\usepackage{epsfig}
\usepackage{graphicx}
\usepackage{graphics}
\usepackage{amsmath,bm}
\usepackage{amssymb}
\usepackage{subcaption}
\usepackage{booktabs}
% Include other packages here, before hyperref.

%donggyu added
\usepackage{bbm}
\DeclareMathOperator{\E}{\mathbb{E}}
\usepackage{caption}
\usepackage{subcaption}
\usepackage{xspace}
\usepackage{wrapfig}
\usepackage[ruled,linesnumbered]{algorithm2e}
\usepackage{multirow}

\newcommand{\methodname}{BGS\xspace}
\newcommand{\methodnamefull}{Group-class \textbf{B}alanced \textbf{G}reedy \textbf{S}ampling\xspace}
\newcommand{\ours}{\methodname}

% If you comment hyperref and then uncomment it, you should delete
% egpaper.aux before re-running latex.  (Or just hit 'q' on the first latex
% run, let it finish, and you should be clear).
\usepackage[pagebackref=true,breaklinks=true,letterpaper=true,colorlinks,bookmarks=false]{hyperref}
\usepackage{cleveref}
\iccvfinalcopy % *** Uncomment this line for the final submission

\def\iccvPaperID{9134} % *** Enter the ICCV Paper ID here
\def\httilde{\mbox{\tt\raisebox{-.5ex}{\symbol{126}}}}

% Pages are numbered in submission mode, and unnumbered in camera-ready
\ificcvfinal\pagestyle{empty}\fi

\begin{document}

%%%%%%%%% TITLE
\title{Continual Learning in the Presence of Spurious Correlation}

\author{Donggyu Lee\textsuperscript{\rm 1}\thanks{Equal contribution.},\ \ Sangwon Jung\textsuperscript{\rm 2}\footnotemark[1],\ \ Taesup Moon\textsuperscript{\rm 2,3,4}\footnote{Corresponding author.}\\
\textsuperscript{\rm 1} Department of Electrical and Computer Engineering, Sungkyunkwan University\\\
\textsuperscript{\rm 2} Department of Electrical and Computer Engineering, Seoul National University\\\
\textsuperscript{\rm 3} SNU-LG AI Research Center \ \ 
\textsuperscript{\rm 4} ASRI/INMC/IPAI/AIIS, Seoul National University \\
% For a paper whose authors are all at the same institution,
% omit the following lines up until the closing ``}''.
% Additional authors and addresses can be added with ``\and'',
% just like the second author.
% To save space, use either the email address or home page, not both
{\tt\small ldk3088@skku.edu, \ \{s.jung,tsmoon\}@snu.ac.kr}} 


\maketitle
% Remove page # from the first page of camera-ready.
\ificcvfinal\thispagestyle{empty}\fi


%%%%%%%%% ABSTRACT
\begin{abstract}

Most continual learning (CL) algorithms have focused on tackling the stability-plasticity dilemma, that is, the challenge of preventing the forgetting of previous tasks while learning new ones. However, they have overlooked the impact of the knowledge transfer when the dataset in a certain task is \textit{biased} --- namely, when some unintended spurious correlations of the tasks are learned from the biased dataset. In that case, how would they affect learning future tasks or the knowledge already learned from the past tasks? In this work, we carefully design systematic experiments using one synthetic and two real-world datasets to answer the question from our empirical findings. Specifically, we first show through two-task CL experiments that standard CL methods, which are unaware of dataset bias, can transfer biases from one task to another, both forward and backward, and this transfer is exacerbated depending on whether the CL methods focus on the stability or the plasticity. We then present that the bias transfer also exists and even accumulate in longer sequences of tasks. Finally, we propose a simple, yet strong plug-in method for debiasing-aware continual learning, dubbed as \methodnamefull (\ours). As a result, we show that our \ours can always reduce the bias of a CL model, with a slight loss of CL performance at most. 
\end{abstract}


%%%%%%%%% BODY TEXT
% \begin{figure}[t]
%     % \begin{subfigure}{1\linewidth}
%     %   \centering
%     % %   \includegraphics[width=1\linewidth]{figs/fig_1_moti_textattn.pdf}  
%     % %   \includegraphics[width=1\linewidth]{figs/fig_1_moti_textattn_v2.pdf}  
%     %   \includegraphics[width=1\linewidth]{figs/fig_1_moti_textattn_v5.pdf}  
%     %   \vspace{-0.5cm}
%     %     \caption{Amount of attention added to each video clip from the source video and query text in the self-attention layers of Moment-DETR encoder.}
%     %     % \caption{Distribution of attention for source and query in Moment-DETR encoder}
%     %     % Visualization of video clip's self-attention score in Moment-DETR encoder.
%     %   \label{fig:fig1_text_attn_ex}
%     % \end{subfigure}%\hfill% or  or \hspace{0.3\textwidth}
%     \vspace{0.2cm}
%     % \begin{subfigure}{1\linewidth}
%       \centering
%     %   \includegraphics[width=1\linewidth]{figs/fig1_moti_negattn.pdf}  
%       \includegraphics[width=1\linewidth]{figs/fig1_moti_negattn_v3.pdf}  
%       \vspace{-0.4cm}
%     %   \caption{Correspondence of saliency scores on the relevance between video clips and the text query.}
%     % \caption{Predicted saliency scores against the video relevant positive query and video irrelevant negative query}
%       \label{fig:fig1_neg_attn_ex}
%     % \end{subfigure}%\hfill% or  or \hspace{0.3\textwidth}
%     \caption{
%     % 원준 원본
%     % (a) Comparison between attention scores of source and query for each video clip~(We sum the attention scores from video and text). 
%     % We observe that the attention scores are dominated by other clips in the source video. 
%     % Text queries do not account for much attention regardless of the relevance to the video clips.
%     % \textbf{(a)} Inspection of the query dependency in Moment-DETR encoder.
%     % % We visualize the attention score of video tokens in the transformer encoder and observe that text query accounts for only a low portion of attention.
%     % % This tendency occurs regardless of the relevance between the text query and video clips. 
%     % We visualize the attention score of video tokens in the transformer encoder and observe 1) text query only accounts for a low portion of attention, and 2) relevance between video-query pair does not affect the attention scores ratio of text.
%     \textbf{(b)} Comparison of highlight-ness when relevant and non-relevant queries are input.
%     As observed in , existing work only uses queries to play an insignificant role, thereby may not be capable of detecting false queries and considering the video-query relevance even when the problem in (a) is resolved. 
%     % \SE{} % 이 부분이 "not capable of" 란 용어가 세다는 피드백이 있는 듯 합니다. 이러한 능력이 없다는 것은 굉장히 강한 어조인거 같기는 하고, 이러한 경우들이 종종 있다거나 좀 약화시킬 필요가 있어보이긴 하네요.
%     On the other hand, our QD-DETR yields a query-dependent representation that the relevance between the source video and query text is updated in the saliency scores.
%     There is a large gap between positive and negative saliency scores, and scores are consistent since the clips are all highly correlated to others.
%     }
%     \label{fig:motivation_ex}
%     % \captionsetup{belowskip=13pt}
%     % \setlength{\belowcaptionskip}{-10pt}
% \end{figure}
\begin{figure}
    \centering
    \includegraphics[width=1\linewidth]{figs/fig1_moti_negattn_1111.pdf}
    % \includegraphics[width=1\linewidth]{figs/fig1_moti_negattn_1109.pdf}
    % \includegraphics[width=1\linewidth]{figs/fig1_moti_negattn_stat.pdf}
    \vspace{-0.6cm}
    \caption{
        % \SE{} % 수정 필요
        Comparison of highlight-ness~(saliency score) when relevant and non-relevant queries are given.
        We found that the existing work only uses queries to play an insignificant role, thereby may not be capable of detecting negative queries and video-query relevance; saliency scores for clips in ground-truth~(GT) moments are low and equivalent for positive and negative queries.
        % This also results in mispredicted moments when ground-truth~(GT) moment is dominated by clips unrelated to GT since their prediction is highly focused on the video.
        % \SE{} % 여기 한번 더 보면 좋을 듯 합니다. GT moment에 unrelated한 clip이 많으면? label이 틀렷을 경우를 말씀하시는건지?
        % As observed in saliency graph, existing work only uses queries to play an insignificant role, thereby may not be capable of detecting false queries and considering the video-query relevance.
        On the other hand, query-dependent representations of QD-DETR result in corresponding saliency scores to the video-query relevance and precisely localized moments.
        % On the other hand, our QD-DETR yields a query-dependent representation that the
        % saliency scores are in accordance with the relevance between the video and query.
        % text is in accordance with the saliency scores.
        % There is a large gap between positive and negative saliency scores, and scores are consistent since the clips are all highly correlated to others.
}
    \label{fig:motivation_ex}
\end{figure}


\section{Introduction}
% 원준 원본
% Along with the advance of digital devices and platforms, video is now one of the most desired data type for consumers. However, although the large information capacity of videos may be beneficial in many aspects, e.g., informative and entertaining, on the contrary perspective, videos are time-consuming, and hard to search for desirable moments. 
% This has led many creators to use extra manpower to crop and edit the video to generate highlight clips to gain the consumer’s attention.
Along with the advance of digital devices and platforms, video is now one of the most desired data types for consumers~\cite{apostolidis2021video,wu2017deep}.
% SE: Video aware deep learning application & survey papers?
Although the large information capacity of videos might be beneficial in many aspects, e.g., informative and entertaining, inspecting the videos is time-consuming, so that it is hard to capture the desired moments~\cite{anne2017localizing,apostolidis2021video}. 
% This has led many creators to use extra manpower to crop and edit the video to generate highlight clips to gain the consumer’s attention.


% On the other side, 
Indeed, the need to retrieve user-requested or highlight moments within videos is greatly raised.
Numerous research efforts were put into the search for the requested moments in the video~\cite{anne2017localizing, gao2017tall, liu2015multi, escorcia2019temporal} and summarizing the video highlights~\cite{zhang2016video, mahasseni2017unsupervised, badamdorj2022contrastive, wei2022learning}.
% Numerous research efforts were put into the search for the requested moments in the video~\cite{anne2017localizing, gao2017tall, liu2015multi, escorcia2019temporal}, summarizing the video to generate highlights was another popular topic~\cite{zhang2016video, mahasseni2017unsupervised, badamdorj2022contrastive, wei2022learning}.
Recently, Moment-DETR~\cite{momentdetr} further spotlighted the topic by proposing a QVHighlights dataset that enables the model to perform both tasks, retrieving the moments with their highlight-ness, simultaneously.

% 원준 원본
% To detect the desired moments, previous works employed transformer encoder-decoder architectural designs to fuse the text query into the video representations. Moment-DETR~\cite{mDETR} modified detection transformer to process capture the moment as a set, and UMT~\cite{umt} implemented transformer decoder as to output clip-wise saliency. 
% Yet to their outstanding breakthroughs in the literature of moment retrieval with the seminal architectures, their limitation is that the role of the given text query is insignificant in representing the query-conditioned video representation; the attention mechanism of moment DETR is not explicitly conditioned on the text query, and the text query is conditioned on multi-modal clips where the differences between the clips are smoothed after encoding process in UMT.



% \begin{figure}[t]
% \centering
%     \begin{subfigure}[l]{0.37\linewidth}
%       \centering
%       \vspace{0.20cm}
%     %   \includegraphics[width=1\linewidth]{figs/fig_1_moti_textattn.pdf}  
%     %   \includegraphics[width=1\linewidth]{figs/fig_1_moti_textattn_v2.pdf}  
%       \includegraphics[width=1\linewidth]{figs/fig1_moti_violin_a.pdf}  
%       \vspace{-0.60cm}
%     %   \caption{text attention}
%         \caption{Importance of queries in video representation}
%       \label{fig:fig1_text_attn}
%     \end{subfigure}%\hfill% or  or \hspace{0.3\textwidth}
%     \vspace{0.2cm}
%     \begin{subfigure}[r]{0.61\linewidth}
%       \centering
%     %   \includegraphics[width=1\linewidth]{figs/fig1_moti_negattn.pdf}  
%       \includegraphics[width=1\linewidth]{figs/fig1_moti_violin_b.pdf}  
%     %   \caption{neg attention}
%         % \caption{Relation between the highlight-ness and the relevance between videos and query texts.}
%         \caption{Highlight-ness~(saliency) histogram of positive and negative video-query pairs\SE{}}
%       \label{fig:fig1_neg_attn}
%     \end{subfigure}%\hfill% or  or \hspace{0.3\textwidth}
%     % \vspace{-0.2cm}
%     \caption{Overall statistics for attention scores in Fig.~\ref{fig:motivation_ex} in QVHighlights dataset. 
%     (a) For the attention scores that measure how much the text query is generally involved in video representation, we use violin plots to show the probability density. We plot the score for each layer in the encoder.
%     % (b) Using the histogram, we compare how the baseline and QD-DETR yield different salient scores given the positive and negative video-text pairs.
%     (b) Saliency histogram shows the distributional gap between positive and negative video-text query pairs of baseline~(Moment-DETR) and proposed QD-DETR.\SE{}
%     }
%     \label{fig:motivation}
%     % \captionsetup{belowskip=13pt}
%     % \setlength{\belowcaptionskip}{-10pt}
% \end{figure}

% \begin{figure}[t]
% \centering

%     \begin{subfigure}[r]{1\linewidth}
%       \centering
%       \hspace{-0.2cm}
%     %   \includegraphics[width=1\linewidth]{figs/fig1_moti_negattn.pdf}  
%       \includegraphics[width=1.1\linewidth]{figs/fig1_moti_violin_a_v2.pdf}  
%     %   \caption{neg attention}
%         % \caption{Relation between the highlight-ness and the relevance between videos and query texts.}
%         \vspace{-0.5cm}
%         % \caption{Saliency histogram of positive and negative video-query pairs}
%         \caption{We plot the histograms and its average value~(dotted line) to compare saliency scores when true and false text queries are given for each method. (left) Since the video representations do not include much textual information, both the true and false queries yield similar saliency scores. (Middle) Even when the video representation is enforced to be updated with the textual information, the issue is not much resolved. (Right) By extracting discriminative features in the text query, distributions are differentiated.
%         % \SE{} % R1@0.5 설명
%         Also, R1@0.5 indicates evaluation metric, Recall at 1 with IoU 0.5 threshold on QVhighlight \textit{val} set.
%         }
%       \label{fig:fig1_neg_attn}
%     \end{subfigure}%\hfill% or  or \hspace{0.3\textwidth}
%     \\
%     \begin{tabular}{cc}
%     \hspace{-0.2cm}
%         \begin{minipage}{.4\linewidth}
%             \begin{subfigure}[l]{1\linewidth}
%               \centering
%             %   \vspace{0.20cm}
%             %   \includegraphics[width=1\linewidth]{figs/fig_1_moti_textattn.pdf}  
%             %   \includegraphics[width=1\linewidth]{figs/fig_1_moti_textattn_v2.pdf}  
%               \includegraphics[width=1\linewidth]{figs/fig1_moti_violin_a.pdf}  
%               \vspace{-0.60cm}
%             %   \caption{text attention}
%                 \caption{Importance of queries in video representation}
%               \label{fig:fig1_text_attn}
%             \end{subfigure}%\hfill% or  or \hspace{0.3\textwidth}
%         \end{minipage}
        
%         \begin{minipage}{.6\linewidth}
%             \vspace{-0.2cm}
%             \caption{Overall statistics of Fig.~\ref{fig:motivation_ex} in QVHighlights dataset. 
%             (a) Saliency histogram shows the distributional gap between positive and negative video-text query pairs.
%             % (a) For the attention scores that measure how much the text query is generally involved in video representation, we use violin plots to show the probability density. We plot the score for each layer in the encoder.
%             % (b) Using the histogram, we compare how the baseline and QD-DETR yield different salient scores given the positive and negative video-text pairs.
%             % (b) Text ratio in self-attention layer to  of Moment-DETR
%             % (b) Ratio of text when representing video tokens in self-attention of Moment-DETR.
%             % (b) Magnitude of attention text query involved.
%             % (b) Attention score of video tokens
%             % (b) Magnitude of text query to refine the video tokens in self-attention layer of Moment-DETR.
%             (b) Probability density depicting the weight of the text query in attention score for video clips. Scores are from the self-attention layers in Moment-DETR encoder.
%             % (b) The text query ratio in attention score of video clips (Self-attention layer in Moment-DETR encoder). We use violin plots to show probability density.
%             % 텍스트 쿼리가, 비디오 피쳐에 얼만큼 attend 하는지
%             }
%         \end{minipage}
    
%     \end{tabular}
%     \vspace{-0.5cm}
%     \label{fig:moti}
%     % \captionsetup{belowskip=13pt}
%     % \setlength{\belowcaptionskip}{-10pt}
% \end{figure}


% \begin{figure}
%     \centering
%     % \includegraphics[width=1\linewidth]{figs/fig1_moti_negattn_1109.pdf}
%     \includegraphics[width=1\linewidth]{figs/fig1_moti_negattn_stat_v2.pdf}
%     \vspace{-0.8cm}
%     \caption{
%         Histogram of saliency when the positive and negative queries are given. We plot the histograms and its average value~(dotted line) to compare saliency scores when relevant~(positive) and irrelevant~(negative) text queries are given for each method. (Left) Since the video representations do not properly reflect textual information, both the positive and negative queries yield similar saliency scores. 
%         % (Middle) Even when the video representation is enforced to be updated with the textual information, the issue is not much resolved. 
%         (Right) By representing video clips in query-dependent manner, distributions are differentiated.
%     }
%     \vspace{-0.6cm}
%     \label{fig:motivation}
% \end{figure}


% One of the demanding task is moment retrieval task, which is detecting the desired moments from the given query, typically the text query.
When describing the moment, one of the most favored types of query is the natural language sentence~(text)\cite{anne2017localizing}. 
While early methods utilized convolution networks~\cite{zhang2020learning, gao2021fast, wang2020temporally}, recent approaches have shown that deploying the attention mechanism of transformer architecture is more effective to fuse the text query into the video representation.
% To handle these modalities, previous works simply employed the attention mechanism of transformer architecture to fuse the text query into the video representation.
For example, Moment-DETR~\cite{momentdetr} introduced the transformer architecture which processes both text and video tokens as input by modifying the detection transformer~(DETR), and UMT~\cite{umt} proposed transformer architectures to take multi-modal sources, e.g., video and audio. 
Also, they utilized the text queries in the transformer decoder.
Although they brought breakthroughs in the field of MR/HD with seminal architectures, they overlooked the role of the text query.
To validate our claim, we investigate the Moment-DETR~\cite{momentdetr} in terms of the impact of text query in MR/HD~(Fig.\ref{fig:motivation_ex}).
Given the video clips with a relevant positive query and an irrelevant negative query, we observe that the baseline often neglects the given text query when estimating the query-relevance scores, i.e., saliency scores, for each video clip.
% the output saliency score, i.e. query-relevance scores.
% Based on the observation, we traced the actual saliency prediction of the model against both the video-relevant query and the irrelevant dummy one where we find that the baseline often neglects the given text query when estimating the query-relevance scores of video clips.
% For example, in Fig.~\ref{fig:motivation_ex}, saliency scores are not affected even when the query is substituted with the dummy.
% % General statistics for Fig.~\ref{fig:motivation_ex} is shown in Fig.~\ref{fig:motivation}. 
% General statistics corresponding to Fig.~\ref{fig:motivation_ex} are also shown in Fig.~\ref{fig:motivation}.



% The limitation of the concrete baseline~\cite{momentdetr} is inspected in two different aspects; 1) Utilization of text-query in the encoding process and 2) the output saliency score, i.e. query-relevance scores.
% Firstly, we visualize the attention score when video clips are given as a query in self-attention. 
% We observe that the text queries have relatively small impacts compared to other video features, as shown in Fig.~\ref{fig:fig1_text_attn_ex}.
% That is, the text does not account for much in representing every video clip, although the goal of MR/HD is to detect query-relevant moments.
% Based on the observation, we traced the actual saliency prediction of the model against both the video-relevant query and the irrelevant dummy one where we find that the baseline often neglects the given text query when estimating the query-relevance scores of video clips.
% For example, in Fig.~\ref{fig:motivation_ex}, saliency scores are not affected even when the query is substituted with the dummy.
% % General statistics for Fig.~\ref{fig:motivation_ex} is shown in Fig.~\ref{fig:motivation}. 
% General statistics are also shown in Fig.~\ref{fig:motivation}.

% Consequently, in Fig.~\ref{fig:fig1_neg_attn_ex}~(b), we found that the baseline often neglects the given text query when estimating the query-relevance scores of video clips; 
% For example, 


% We validate the previous work sometimes neglects the given query when estimating the saliency of video clips.
% For example, there is an example that the saliency scores from positive and negative queries cannot be distinguishable, as shown in Fig.~\ref{fig:fig1_neg_attn_ex}.
% % 우리는 추가로 text attention을 추가도 해봤지만, 효과가 있긴 했으나, still 이슈가 있는 것을 확인하였다?
% % Still, we observe that assuring the high attendance of text queries does not resolve the overlap which motivates us to question the quality of the naive use of task-agnostic text representation~\cite{momentdetr, umt}.
% We found that introducing the text-attention for ensuring the high attendance of text queries relieve the overlap, but there still be a severe overlap.


% To validate their limitations, we inspect the impacts of text queries in the concrete baseline~\cite{momentdetr} with the two different aspects, 1) tendency of attention in self-attention layer and 2) saliency score, i.e. query-relevance scores. \SE{} % attention 이 갑자기 등장하는가?
% Firstly, we visualize the attention score when video clips are given as a query in self-attention. We observe the text queries have relatively low attention scores compared to the video features, as shown in Fig.~\ref{fig:fig1_text_attn_ex}.
% That is, the text does not account for much in representing every video clip, although the goal of MR/HD is to detect query-relevant moments.
% Based on this observation, we trace the actual saliency prediction of the model against both positive and negative text queries.
% We validate the previous work sometimes neglects the given query when estimating the saliency of video clips.
% For example, there is an example that the saliency scores from positive and negative queries cannot be distinguishable, as shown in Fig.~\ref{fig:fig1_neg_attn_ex}.
% % 우리는 추가로 text attention을 추가도 해봤지만, 효과가 있긴 했으나, still 이슈가 있는 것을 확인하였다?
% % Still, we observe that assuring the high attendance of text queries does not resolve the overlap which motivates us to question the quality of the naive use of task-agnostic text representation~\cite{momentdetr, umt}.
% We found that introducing the text-attention for ensuring the high attendance of text queries relieve the overlap, but there still be a severe overlap.



% Thus, we 
% query dependency를 높이기 위해 
% Cross-attention? text-attention? detailed explanation on text-attention should be needed?
% By handling these two issues, we find that more precise retrieval can be achieved.
% 
% 
%
% By projecting video-discriminative text features with high text attendance to source video, we f 
% We also find the need to improve the quality of query features since assuring high text attendance also results in...
% pairs are not finetuned to be discriminative that even the similarity within the pairs does not reflect the relevance between the query and the video clips.
% General statistics for Fig.~\ref{fig:motivation_ex} is shown in Fig.~\ref{fig:motivation}. 
% \SE{} % 이거 ??로 뜨는데, 위처럼 figure 그리면 label이 안되는걸까요
% \SE{}
% 형님 아래 사항 생각 좀 해보는게 좋을 거 같아요.
% fig 1. (a) 그림만 봤을 때 모든 clip에 대해 text attention이 일정이상 존재하긴 하니까, 뭔가 not assured to be conditioned가 와닿지 않는거 같아요.
% + 왜 text가 항상 attend 해야하나?
% not assured to be conditioned --> text shows relatively low affects compared to video 같이 실제 나타난 현상까지 같이 적으면 어떨까 싶어요.
% fig 1. (b) 덜 반영한다?

% \SU{}
% 일단 text가 attend 잘 되어야 한다는 것에 좀 궁금점이 생깁니다. 결국에는 text와 관련있는 frame들을 attend해서 higlight를 찾아야 하는게 아닐까요? 그리고, 현제 저희의 모델 구조상 text query가 Key와 Value로 거의 활용되고 있는데 그렇다면 결국에는 해당 모델은 text에 대한 attention이 전혀 없다고 봐도 무방하지 않을까요? 그런 면에서 text attention을 강조하는게 좀 걸리긴 합니다.

% Specifically, the text query is not assured to be explicitly conditioned on every clip of the video, and as the query texts are evenly treated, discriminative keywords may not be spotlighted.
% attention mechanism of Moment-DETR is not explicitly conditioned on the text query as shown in Fig~\ref{}(d), and in UMT, the text are only used for conditioning the queries while the video representation are refined itself by self-attention.

% \begin{figure}[t]
%     \begin{subfigure}{1\linewidth}
%       \centering
%     %   \includegraphics[width=1\linewidth]{figs/fig_1_moti_textattn.pdf}  
%     %   \includegraphics[width=1\linewidth]{figs/fig_1_moti_textattn_v2.pdf}  
%       \includegraphics[width=1\linewidth]{figs/fig_1_moti_textattn_v4.pdf}  
%       \vspace{-0.5cm}
%     %   \caption{text attention}
%         \caption{Distribution of attention scores in Moment-DETR encoder}
%       \label{fig:fig1_text_attn}
%     \end{subfigure}%\hfill% or  or \hspace{0.3\textwidth}
%     \vspace{0.2cm}
%     \begin{subfigure}{1\linewidth}
%       \centering
%     %   \includegraphics[width=1\linewidth]{figs/fig1_moti_negattn.pdf}  
%       \includegraphics[width=1\linewidth]{figs/fig1_moti_negattn_v2.pdf}  
%       \vspace{-0.5cm}
%     %   \caption{neg attention}
%         \caption{Saliency score against positive and negative text queries}
%       \label{fig:fig1_neg_attn}
%     \end{subfigure}%\hfill% or  or \hspace{0.3\textwidth}
%     \vspace{0.2cm}
%     \begin{subfigure}{1\linewidth}
%       \centering
%     %   \includegraphics[width=1\linewidth]{figs/fig1_moti_violin.pdf}  
%       \includegraphics[width=1\linewidth]{figs/fig1_moti_violin_v2.pdf}  
%       \vspace{-0.5cm}
%       \caption{violin}
%       \label{fig:fig1_violin}
%     \end{subfigure}%\hfill% or  or \hspace{0.3\textwidth}
%     \vspace{-0.2cm}
%     \caption{(a) 1. portion of text attention vs. video attention 2. relation with text query and content (e.g. fg, bg) of clip seems not to affect the attention score
%     (b) 1. high variability even though entire clips are highly correlated with the given text query 2. positive and negative query makes overlaps on saliency score distribution
%     (3) actual distribution on validation dataset.}
%     \label{fig:motivation}
%     % \captionsetup{belowskip=13pt}
%     % \setlength{\belowcaptionskip}{-10pt}
% \end{figure}

To this end, we propose Query-Dependent DETR~(QD-DETR) that produces query-dependent video representation.
% Our key focus is to ensure each clip in predicted moments is explicitly conditioned by the query, particularly on the video-descriptive portion of the text query.
% Our key focus is to ensure that query-relevant clips are predicted by enforcing each clip to be explicitly conditioned by the query.
%Our key focus is to ensure that the model prediction for each clip is highly relevant to the query.
Our key focus is to ensure that the model's prediction for each clip is highly dependent on the query.
% by enforcing each clip to be explicitly conditioned by the query. :)
% hmm...
% \SE {} % "query-relevant clips are predicted" 이 문장이 좀 애매한거 같습니다. relevant 클립을 놓지지 않고 찾는 것을 보장한다? 이런 느낌인지 아니면 높은 saliency 를 주는게 목적이다? model prediction이 query-relevance를 반영하는 것을 보장한다?
% Our key focus is to ensure that the model prediction reflects query-relevance of clips by enforcing each clip to be explicitly conditioned by the query.
First, to fully utilize the contextual information in the query, we revise the transformer encoder to be equipped with cross-attention layers at the very first layers.
% 상익's thought :  single video - query간의 관계만 고려 - 같은 word가 더 많이 쓰이는 것을 보고 
% 교수님's thought : neg pair 를 쓰면 쿼리를 보지 않고서는 video clip간만 고려하는 것이 사라짐. 왜냐면 0으로 내보내야 하기 때문. --> SE: relative difference 만 고려하다가, 
By inserting a video as the query and a text as the key and value of the cross-attention layers, our encoder enforces the engagement of the text query in extracting video representation.
% 원준 교수님 코멘트 반영해서 다시
Then, in order to not only inject a lot of textual information into the video feature but also make it fully exploited, we leverage the negative video-query pairs generated by mixing the original pairs.
Specifically, the model is learned to suppress the saliency scores of such  negative~(irrelevant) pairs.
Our expectation is the increased contribution of the text query in prediction since the videos will be sometimes required to yield high saliency scores and sometimes low ones depending on whether the text query is relevant or not.
% \SE{}
% learns to?
% By suppressing the saliency scores of the irrelevant video-query pairs, the model learns to spotlight only the video-specific discriminative words in the query.
% % \SE{} % ====================== 상익 수정 ========================
% However, this architectural design still lacks the capability of identifying the video-descriptive keywords in the query.
% % However, this architectural design still lacks in identifying proper query relevance.
% This is because the current training scheme only focuses on the interactions of video and clips within a single video while neglecting information shared throughout the entire video.
% % We argue the problem of the current training scheme that only focuses on distinguishing the clips in a single video while neglecting information shared throughout the entire video.
% Therefore, we leverage the negative video-query relationships to enhance the capability of identifying the contextual similarity of query and video clips.
% 
% 원준 원본 
% However, this architectural design heavily relies on the quality of the text query.
% Therefore, we leverage the negative video-query relationships to enable the model to emphasize key corresponding query features.
% By suppressing the saliency scores of the irrelevant video-query pairs, the model learns to spotlight only the video-specific discriminative words in the query.
% =========================================================
Lastly, to apply the dynamic criterion to mark highlights for each instance, we deploy a saliency token to represent the entire video and utilize it as an input-adaptive saliency criterion. 
With all components combined, our QD-DETR produces query-dependent video representation by integrating source and query modalities.
This further allows the use of positional queries~\cite{dabdetr} in the transformer decoder.
% Furthermore, we can exploit the advanced DETR decoder architectures using the positional information, e.g., DAB-DETR, since our encoded tokens consist of identical position representations from a single modality.
% \SE{} % ====================== 상익 수정 ========================
% Furthermore, we can exploit the advanced DETR decoder architectures using the positional information, e.g., DAB-DETR, since our video clip tokens consist of identical position representations from a single modality.
% 원준 원본
% It also enables the use of advanced DETR decoder architectures, e.g., DAB-DETR, for the first time, as these works exploit the position information within a single modality.
% =========================================================
Overall, our superior performances over the existing approaches validate the significance of the role of text query for MR/HD.
% Our extensive experiments on QVHighlights, TVSum, and Charades-STA datasets validate the significance of considering the role and the quality of text query.

% All components combined with dynamic anchor moments for the query of decoder, our FOQUE fosters the query-dependent video representation, thereby making the 
% All components combined, our modified transformer encoding process fosters the query-dependent video representation thereby achieving the state-of-the-art results on various benchmarks of moment-retrieval and highlight detection.
	
% -	Video Platform & Streamer & Consumer의 증가. 
% Video는 다른 데이터 타입보다 정보가 많아 유용하지만, 이는 다른 말로 해석하면 video를 보는 것은 time-consuming 하고, 원하는 것을 찾아보기에는 힘들 수 있음.
% 따라서, 많은 매체에서는 사람들의 더 많은 이목을 끌기 위해 highlight 비디오라는 것을 편집하여 공유도 함.
% 하지만, highlight video를 만들기 위해 사람의 노력이 필요한 현 시점에서, This spotlights the need to retrieve the user-requested / Highlight moments in the video.

% -	이전에도 이러한 문제를 해결하기 위해 (asdfasdf) for moment retrieval, (asdfasdf) for highlight detection 등이 제안 되었지만, 이들은 비디오의 특정 영역을 찾는다는 공통된 목적을 가지고 있으면서도, 데이터 셋의 한계로 인해 따로 연구되었음. 이를 문제 삼으며, 최근에는 두 task를 동시에 학습할 수 있는 dataset이 소개 되었는데, 컴퓨터비전에서 최근 각광을 받고 있는 Transformer 모델 도입과 함께 큰 발전을 거듭하고 있음.

% -	구체적으로, 이 두가지 task를 수행하기 위해서는 transformer를 두가지 방법으로 이용할 수 있는데, moment-DETR 처럼 moment 를 clip의 set 단위로 예측할 수 있고, UMT 처럼 clip-wise prediction을 할 수 있음. 하지만, 이들은 query를 condition이 아닌 video와 동등한 레벨로 취급하거나 [mDETR], 매 클립이 self-attention으로 mixing 된 후에 condition을 걸어주어 clip간의 차이를 확실하지 이용하지 못하였고, 또한, 확실하게 condition으로 주지 못하였고, video와 query 사이의 관계를 한정적으로만 이용하였다.

% -	따라서, we explore three different ways to fully exploit query information. First, we design one-way cross-attention layer to condition every clip with the query features. Then, we utilized the negative video-text pairs to better model the relationships between the video and the text embeddings. Lastly, we define the saliency token to be the video-query dependent saliency estimator.


















% ===================== neg pair 부분 ===========================
% Nevertheless, the current training scheme, only considering the given video-query pair, still disturbs the model from identifying proper query-relevance prediction.
% In detail, the model focus on learning the fine-grained discrepancy between video clips, while neglecting the information they share, which contains significant clues to understand the context of video.
% Therefore, we leverage the negative video-query relationships to enhance the capability of identifying the contextual similarity of query and video clips.
% Therefore, we leverage the negative video-query relationships by suppressing those pairs, so that enhance the capability of identifying the contextual similarity of query and video clips.
% We hypothsize the diversity in query-video pairs are insufficient to learn the general relationship between text query and video.
% Therefore, we leverage the negative video-query relationships by suppressing the saliency scores of the irrelevant video-query pairs.
% However, this architectural design still lacks in identifying proper query relevance.
% We argue that the current training scheme only focuses on learning the fine-grained discrepancy between clips in a single video, while neglecting the information they share, which contains significant clues to understand the context of the video.
% Therefore, we leverage the negative video-query relationships to enhance the capability of identifying the contextual similarity of query and video clips.
% However, this architectural design still lacks in identifying proper query relevance.
% We argue the problem of the current training scheme that only focuses on learning the fine-grained discrepancy between clips in a single video.
% That is, the current design neglects the information shared throughout the video, although it contains significant clues to understand the context of the video.
\section{Related Works}

\begin{figure*}[!ht]
\centering
\includegraphics[width=\linewidth]{body/figures/data_collection2.png}
\caption{\textbf{Hardware Setup.} We use a GelSight Wedge sensor for tactile sensing, an Intel ReslSense D405 camera mounted on the side for RGB vision sensing, and an OptiTrack setup for motion capture. \textbf{Data Collection.} The tactile finger and the camera are fixed to the table at all times. A human operator moves a test object and presses it against the finger. We show sampled tactile and RGB images as well as a reconstructed local tactile depth map on the right.}
\label{datacollection}
\end{figure*}

% \textbf{Tactile sensors}:
% Over the years, researchers have developed tactile sensors working on different sensing principles, such as resistance, capacitance, magnetic, barometric, and optic.
% We refer readers to \cite{kappassov2015tactile} for an in-depth review of different types of tactile sensors and their applications.
% Compared to other sensing principles, GelSight tactile sensors have the advantage of providing high-resolution geometrical information of the contact surface.
% They are usually constructed with an elastic silicone gel, directional colored LEDs, and a camera pointing at the gel.
% The gels are usually coated with reflective paint with printed dots.
% When in contact, the gel deforms and takes the shape of the contact surface.
% Shear force can be retrieved by tracking dot movements.
% Furthermore, the color value of a pixel is correlated with the gradient of the height of the contact surface at the specific location. 
% With a pre-calibrated color table, a depth map can be reconstructed from the color image.
% This type of tactile sensor is selected for our work for its rich output and ease to use.

Researchers of the robotics community have put forward a wide range of tactile sensing solutions.
Sensors working on different sensing principles have been adopted to solve a large set of manipulation tasks.
Among different types of tactile sensors, vision-based ones such as GelSight \cite{yuan2017gelsight} and GelSlim \cite{donlon2018gelslim} stand out for their rich output, ease to use, and affordability.
While we focus on the pose estimation and shape reconstruction task using vision-based tactile sensors, we refer readers to \cite{kappassov2015tactile} for an in-depth review of different types of tactile sensing and their applications.
In this section, we review works on three typical tasks that are most relevant to our solution: slip detection, object property inference, and SLAM.

% Researchers have found that using this class of vision-based tactile sensors can greatly increase the accuracy when reasoning about the contact surface, compared to traditional tactile sensors that are constructed with normal direction force sensors [\todo{add citation}].

\textbf{Slip detection and estimation}:
Using a similar sensor to ours, Yuan \etal compared and analyzed a GelSight tactile sensor's images collected at different stages of slip in \cite{yuan2015measurement} and showed this type of sensors' capability in detecting micro scale movements.
Li \etal and Zhang \etal trained recurrent neural networks on tactile images to detect slip between multiple time steps in a manipulation sequence \cite{li2018slip, zhang2018fingervision}.
Built on their binary slip detection model in \cite{li2018slip}, Li further added rotational slip direction prediction in \cite{li2019rotational}.
Calandra \etal improved a grasp planner for the classic robot bin-picking problem by incorporating slip detection and achieved a higher grasp success rate \cite{calandra2017feeling}. 
However, those methods only detect slip without localizing the object after the slip.
In many precision manipulation tasks we are also interested in the amount of the displacement.

\textbf{Object property inference and localization}:
% With detailed information on the contact surface provided by high-resolution tactile sensors, 
Many works have focused on inferring properties of the in-contact object, such as shape \cite{strub2014using, luo2015tactile, luo2019iclap}, texture \cite{luo2018vitac, yuan2017connecting}, and material \cite{yuan2017connecting, kroemer2011learning, kerr2018material}.
Those learned object properties can be further used for localization.
In order to localize current grasps, Bauza \etal proposed to match new tactile imprints with previously collected tactile imprints \cite{bauza2019tactile}, while Luo \etal learned to match tactile imprints directly to visual images of the whole object \cite{luo2015localizing}.
Assuming known CAD models, Bauza \etal proposed to localize by comparing contact masks generated from tactile images with a large bank of random projections of the CAD model \cite{bauza2022tac2pose}.
To solve the reverse problem, i.e. what a tactile image looks like given an object and a pose, several tactile simulators have been built to automatically generate tactile images given an object's CAD model and a finger pose \cite{si2022taxim, wang2022tacto}.
One major limitation for this category of works is that they all require a known calibrated geometry of the object: a pre-collected tactile map \cite{bauza2019tactile}, a model of the object \cite{bauza2022tac2pose}, or a global image with known geometry \cite{luo2015localizing}.
This requirement can be hard to meet in less constraint environments.

\textbf{Tactile SLAM}:
Recent studies have shown interests in working with unknown objects by leveraging methods from the SLAM problem.
With a focus on 2D shapes, Suresh \etal parameterized shapes as Gaussian Process Implicit Surfaces (GPIS), and learned its parameters from tactile signals collected during pushing \cite{suresh2021tactile}.
Assuming known contact poses, authors of \cite{suresh2022shapemap} first learned a noisy mapping from known surface geometries to corresponding tactile images, then reconstructed an object by combining many noisy local tactile measurements into an optimized global shape using factor graph optimization.
The closest prior work to ours is \cite{sodhi2022patchgraph}, where the authors learned to estimate 6D poses and 3D shapes simultaneously for unknown objects. 
They constructed a pose estimator based on tactile sensing, and a shape reconstruction pipeline that added in new tactile point clouds incrementally on the run.
However, this approach heavily relies on the performance of the tactile pose estimator, which lacks a global understanding of the object and can suffer from repeated patterns or smooth surfaces.
In contrast, our work combines vision and tactile sensing which provides us with both global and local understandings of the scene without requiring any other domain knowledge.
Furthermore, we designed a loop closure mechanism that periodically matches current tactile and vision images to stored key-frames, which significantly reduced accumulated errors.
With this, FingerSLAM is able to produce realistic reconstructions even in long sequences. 
% model 얘기 아직 안씀.
\section{Experimental setup}
\label{sec:setup}

\subsection{Notations and problem setting}
We consider the \textit{bias-aware continual learning} which is composed of a sequence of classification tasks which may contain a dataset bias. Each data sample in the $t$-th task $T_t$ consists of the tuple $(x_i,a_i,y_i)$ where $a_i\in\mathcal{A}$ and $y_i\in\mathcal{Y}_t$ are group and class labels of an input $x_i$, respectively. 
% an input, $x_i$, a group label $a_i\in\mathcal{A}$, a class label $y_i\in\mathcal{Y}_t$, and a task label $t\in [T]$.
Unless otherwise noted, we consider the single type of bias in a CL scenario for simplicity of analysis. In addition, the sets of class label, $\{\mathcal{Y}_i\}_{i\in\{ 1, 2, \dots \}}$, can be identical or not, depending on the type of CL scenarios. 

\subsection{Benchmark datasets}
We use one synthetic dataset, Split CIFAR-100S, and two real-world datasets, CelebA$^2$ (or CelebA$^8$) and Split ImageNet-100, which are applied for Task-IL, Domain-IL, and Class-IL settings, respectively. Followings are descriptions for our datasets (more details are in Appendix). 

\noindent\textbf{Split CIFAR-100S} is a modified dataset from Split CIFAR-100 \cite{si, er, vandeven2020brain}, which randomly divides CIFAR-100 \cite{cifar10} into 10 tasks with 10 distinct classes. Similarly as in \cite{wang2020towards}, we modify split CIFAR-100 such that, given a skew-ratio $\alpha\geq 0.5$, half of the classes in each task are skewed toward the grayscale group and the other half toward the color group; namely, the training images of each class are split into $\alpha$ and $1-\alpha$ ratios for each group. Thus, the ``color'' becomes the bias of the dataset. We set 7 bias levels (0-6) by dividing the range of skew-ratio from 0.5 to 0.99 evenly on a log scale for systematic control of the degree of bias. 

\noindent\textbf{CelebA} \cite{liu2015deep} contains more than 200K face images, each annotated with 40 binary attributes. It is notorious for containing representation biases towards specific attributes such as race, age, or gender \cite{torfason2017face,fabbrizzi2022survey}; for instance, the sub-populations of young women and old men are over-represented in the CelebA dataset. Unless otherwise specified, we use ``gender'' and ``young'' attributes as a group and a class label, respectively. We additionally select one or three other attributes and based on them, divide CelebA into two tasks (used in \cref{sec:two_task_studies}) or eight tasks (used in \cref{sec:longer} and \ref{sec:comparison}), which are denoted as CelebA$^2$ or CelebA$^8$, respectively. Since the representation bias in CelebA can be controlled by adjusting imbalances between the class and group labels, we set 7 bias levels varying on the degree of imbalance from 0.5 to 0.99, as the same procedure in Split CIFAR-100S.

% It is known that DNN models trained from ImageNet-1000 rely on various spurious features such as texture \cite{geirhos2018}, watermark \cite{li2022whac}, and co-occurring class pairs \cite{singla2021salient}, but we only consider watermark bias for simplicity of analysis. 
\noindent\textbf{Split ImageNet-100} divides ImageNet-100 \cite{(Imagenet)Deng09} into 10 tasks with disjoint 10 classes which are randomly sampled from original 1000 classes in ImageNet-1000. 
It is known that DNN models trained from ImageNet are biased towards watermark \cite{singla2021salient}; namely, the ImageNet-pretrained models predict an image as the carton class when injecting a watermark to it, since most of carton images in the ImageNet training dataset include the watermark. 
% only carton images in the ImageNet training dataset include the watermark. 
% Thus, 
Following the recent study, we also consider the watermark bias in Split ImageNet-100.
% To study this bias in our analysis, we compare the degree of bias of a CL model depending on the presence of the carton class. 
We utilize two types of test datasets, the original ImageNet validation dataset, and ImageNet-W with watermarks injected by style transfer \cite{li2022whac, gatys2016image}, in order to measure the degree of bias. We set a bias level of a task as 0 or 6, depending on the presence of a carton class in the task.
% , and measure the degree of bias by comparing the accuracy between them.

% \begin{figure*}[t!]
%     \centering
%     \begin{minipage}{.3\linewidth}
%         \begin{subfigure}[t]{\linewidth}
%             \includegraphics[width=\linewidth]{figures/two_cifar_forward.pdf}
%             \caption{Results on Split CIFAR-100S}
%             \label{fig:two_forward_cifar}
%         \end{subfigure}
%         \begin{subfigure}[b]{\linewidth}
%             \includegraphics[width=\linewidth]{figures/two_celeba_forward.pdf}
%             \caption{Results on CelebA$^2$}
%             \label{fig:two_forward_celeba}
%         \end{subfigure}
%         \caption{Forward transfer of bias}
%         \label{fig:two_forward}        
%     \end{minipage}
%     \hfill
%     \begin{minipage}{.3\linewidth}
%         \begin{subfigure}[t]{\linewidth}
%             \includegraphics[width=\linewidth]{figures/two_cifar_backward.pdf}
%             \caption{Results on Split CIFAR-100S}
%             \label{fig:two_backward_cifar}
%         \end{subfigure}
%         \begin{subfigure}[b]{\linewidth}
%             \includegraphics[width=\linewidth]{figures/two_celeba_backward.pdf}
%             \caption{Results on CelebA$^2$}
%             \label{fig:two_backward_celeba}
%         \end{subfigure}
%         \caption{Backward transfer of bias}
%         \label{fig:two_backward}
%     \end{minipage}    
%     \hfill
%     \begin{minipage}{.33\linewidth}
%         \centering
%         \begin{subfigure}[t]{\linewidth}
%             \includegraphics[width=\linewidth]{figures/cka_forward.pdf}
%             \caption{Forward transfer}
%         \end{subfigure}
%         \centering
%         \begin{subfigure}[b]{\linewidth}
%             \includegraphics[width=\linewidth]{figures/cka_backward.pdf}
%             \caption{Backward forward}
%         \end{subfigure}
%         \caption{CKA on Split CIFAR-100S} 
%     \end{minipage}
%     \label{fig:cka}
% \end{figure*}

\begin{figure*}[t!]
    \centering
    \begin{subfigure}[t]{0.3\linewidth}
        \includegraphics[width=0.9\linewidth]{figures/two_cifar_forward.pdf}
        \caption{\small{Split CIFAR-100S in Task-IL}}
        \label{fig:two_forward_cifar}
    \end{subfigure}
    \begin{subfigure}[t]{0.3\linewidth}
        \includegraphics[width=0.9\linewidth]{figures/two_celeba_forward.pdf}
        \caption{CelebA$^2$ in Domain-IL}
        \label{fig:two_forward_celeba}
    \end{subfigure}
    \begin{subfigure}[t]{0.3\linewidth}
        \includegraphics[width=0.9\linewidth]{figures/two_imagenet_forward.pdf}
        \caption{Split ImageNet-100 in Class-IL}
        \label{fig:two_forward_imagenet}
    \end{subfigure}
    \vspace{-.05in}
    \caption{\small {\bf Forward transfer of bias in two tasks-continual learning.}}
    \label{fig:two_forward}
\end{figure*}

\begin{figure*}[t!]
    \centering
    \begin{subfigure}[t]{0.3\linewidth}
        \includegraphics[width=0.9\linewidth]{figures/two_cifar_backward.pdf}
        \caption{Split CIFAR-100S in Task-IL}
        \label{fig:two_backward_cifar}
    \end{subfigure}
    \begin{subfigure}[t]{0.3\linewidth}
        \includegraphics[width=0.9\linewidth]{figures/two_celeba_backward.pdf}
        \caption{CelebA$^2$ in Domain-IL}
        \label{fig:two_backward_celeba}
    \end{subfigure}
    \begin{subfigure}[t]{0.3\linewidth}
        \includegraphics[width=0.9\linewidth]{figures/two_imagenet_backward.pdf}
        \caption{Split ImageNet-100 in Class-IL}
        \label{fig:two_backward_imagenet}
    \end{subfigure}    
    \vspace{-.05in}
    \caption{\small {\bf Backward transfer of bias in two tasks-continual learning.}}
    \label{fig:two_backward}
    \vspace{-.1in}
\end{figure*}


\subsection{Continual learning and debiasing baselines}
We compare two naive methods and six representative CL methods: \textit{fine-tuning} without any consideration of CL, \textit{model-freezing} with freezing model parameters updated from previous tasks, LWF \cite{li2017learning} and EWC \cite{ewc} for regularization based methods, ER \cite{er} and iCaRL \cite{rebuffi2017icarl} for rehearsal based methods and PackNet \cite{mallya2018packnet} for parameter isolation based methods. We note that each CL method can control the stability-plasticity trade-off by adjusting their own hyperparameters such as the regularization strength, the size of the exemplar memory  or the pruning ratio. We further note that PackNet is designed only for task-IL settings, LWF for task-IL and class-IL, and iCaRL for class-IL settings.
Additionally, we employ a widely used debiasing technique, Group DRO \cite{groupdro}.
For implementation details, please refer to the Appendix.

\subsection{Metrics} 
We utilize \textit{average accuracy} over learned tasks as a metric for CL performance and \textit{Normalized $\mathcal{F} - \mathcal{I}$} as a metric for the relative weight on plasticity and stability. 
% In addition, we compute forgetting ($\mathcal{F}$) and intransigence ($\mathcal{I}$) measures \cite{Chaudhry2018ECCV,cha2021cpr} for evaluating stability and plasticity of a CL method, respectively, and use \textit{Normalized $\mathcal{F} - \mathcal{I}$} as a metric for the relative weight on plasticity and stability. 
The concrete definition of Normalized $\mathcal{F} - \mathcal{I}$ is given below. 
Let $h_t$ and $ h^*_t$ be the classifiers learned up to $T_t$ tasks which are trained by a CL method and the fine-tuning, respectively. 
% =\{(x_{t}^{(i)}, a_{t}^{(i)}, y_{t}^{(i)})\}_{i=1}^{N_t}$ be a test dataset for task $T_t$ where , where $a_{t}^{(i)}\in \mathcal{A}$ is the group label of the input $x_{t}^{(i)}\in\mathcal{X}$, and $y_{t}^{(i)}\in\mathcal{Y}_t$ is the class label where $\mathcal{Y}_t$ is the set of classes of $T_t$. 
The forgetting and intransigence measures \cite{Chaudhry2018ECCV,cha2021cpr}, $\mathcal{F}_t$ and $\mathcal{I}_t$, after learning up to task $T_t$ are then defined as follows:
\begin{align}
\mathcal{F}_t &: \frac{1}{t-1} \sum_{j=1}^{t-1} \max_{l\in[t-1]} \operatorname{A}(h_l, \mathcal{D}_j)-\operatorname{A}(h_t, \mathcal{D}_j) \label{eq:F} \\
\mathcal{I}_t &: \frac{1}{t} \sum_{j=1}^{t} \operatorname{A}(h^*_j, \mathcal{D}_j)-\operatorname{A}(h_j, \mathcal{D}_j), \label{eq:I}
\end{align}
in which $\operatorname{A}(h, \mathcal{D}_t)$ is the test accuracy of a model for a test dataset of $T_t$, $\mathcal{D}_t$. Note that the two measures evaluate the stability and plasticity of a CL method, respectively. 
Then, for each CL scenario and method, the differences between the two measures are normalized by the maximum and minimum values of $\mathcal{F}-\mathcal{I}$, which are obtained by varying the hyperparameter of each CL method. Especially, in the case of regularization based methods, the maximum and minimum values of $\mathcal{F}-\mathcal{I}$ mostly correspond to $\mathcal{F}-\mathcal{I}$ of the fixed model and the fine-tuning. Notably, the Normalized $\mathcal{F}-\mathcal{I}$ indicates the model focuses more on stability as the value becomes lower and on plasticity as it becomes higher.

The degree of bias of the model can be evaluated by observing its behavior for predicting a sample when a bias feature of the sample is changed. Formally, we measure a model bias using the bias-flipped mis-classification rate (BMR):
\begin{align}
    \text{BMR} = \frac{\sum_{\{x_i \in \mathcal{D}|h(x_i)=y_i\}} \mathbb{I}(h(x_i^*)\neq y_i)}{|\{x_i \in \mathcal{D}|h(x_i)=y_i\}|},
    % \text{BMR} = \mathbb{E}_{\{x_i \in \mathcal{D}|h(x_i)=y_i\}} \mathbb{I}(h(x_i^*)\neq y_i),
\end{align}
in which $x_i^*$ is a bias-flipped sample with other features fixed, \eg, changes of presence only for color (for Split CIFAR-100S) or watermark (for Split-ImageNet). Namely, BMR considers the number of falsely predicted samples after only flipping bias features for all correctly predicted samples. 
% We note that BMR has close connection with average treatment effect (ATE) \cite{xxx} widely used in the causality literature. 

In most real-world datasets such as CelebA, it might be challenging to generate bias-flipped samples such as transforming a woman image to look like a man. Thus, for CelebA, we use the difference of classwise accuracy (DCA) \cite{berk2021fairness} as a surrogate metric: 
\begin{align}
\small
\text{DCA}(h, \mathcal{D}_{t}) &= \frac{1}{|\mathcal{Y}_{t}|}\sum_{y\in\mathcal{Y}_{t}}{\max_{a, a'\in\mathcal{A}}} \lvert \operatorname{A}(h, \mathcal{D}_{t}^{y,a}) - \operatorname{A}(h, \mathcal{D}_{t}^{y,a'}) \rvert,  \nonumber 
\end{align}
in which $\mathcal{D}^{y,a}_{t}$ is the subset of $\mathcal{D}_{t}$ that is confined to the
samples with class-group label pair $(y,a)$. DCA means the average (over class) of per-class maximum accuracy difference between domains. Informally, DCA is regarded as an approximation of BMR by calculating the difference of predictions in group levels, not in sample levels.

We note that we use BMR for Split CIFAR-100S and ImageNet-100 and DCA for CelebA. Further note that high BMR and DCA correspond to $h$ possessing large bias. 


% adjusted by the 
% min-max normalization; the maximum value is calculated with maximum $\mathcal{F}$ and minimum $\mathcal{I}$ by setting the second terms of \cref{eq:F} and \cref{eq:I} to 0 and 1, respectively, and the minimum value vice versa.
% we first compute the minimum and the maximum of $\mathcal{F}$ and $\mathcal{I}$ as: $\mathcal{F}_{min} = \operatorname{A}(h_1,\mathcal{D}_{1}) - 1.0$ and $\mathcal{I}_{min} = \operatorname{A}(h^*_2,\mathcal{D}_{2}) - 1.0$; for the maximum,  $\mathcal{F}_{max} = \operatorname{A}(h_1,\mathcal{D}_{1})$ and $\mathcal{I}_{max}= \operatorname{A}(h^*_2,\mathcal{D}_{2})$. Then we use $(\mathcal{F} - \mathcal{I})_{min}$ and $(\mathcal{F} - \mathcal{I})_{max}$ for normalization.
% that can be achieved theoretically. 


% \subsection{Continual learning scenario}
% We consider the task-incremental learning for Split CIFAR-100S and the domain-incremental learning for CelebA; the former assumes tasks with different class labels and a multi-headed model and the latter assumes tasks with the same class labels and a single-headed model. 
% % a task identifier $t \in {\mathcal{T}}\triangleq \{1,2,3,\cdots\}$ is given during inference time 
% % and the latter assumes that \cite{van2019three}
% We further assume that the domain of a training sample is known. 

% For simplicity of our analysis, in this section, we only considered the scenario of sequentially learning \textit{two} tasks. For Split CIFAR-100S, we randomly chose 2 out of 10 tasks in every run.  
% and reported the averaged results over 4 different runs
% . We denote the $t$-th task as $T_t$ with $t\in\{1,2\}$.
% We denote the first and second task as $T_1$ and $T_2$, respectively.


\section{Case for CL with two tasks}
\label{sec:two_task_studies}
We begin our analysis by examining CL scenarios consisting of two tasks in sequence. Our goal is to identify \textit{forward} and \textit{backward} transfers of the bias of a CL model through both quantitative analyses and figure out how the CL methods promote each of these transfers. 

\subsection{Forward transfer of bias}
To investigate the forward transfer of the bias, we evaluated CL methods by varying the degree of bias of $T_1$, while that of $T_2$ is fixed to level 0. 
Figure \ref{fig:two_forward} reports bias metric values of $T_2$ along with Normalized $\mathcal{F}-\mathcal{I}$ on three datasets after learning $T_2$ with two different bias levels of $T_1$, \ie, level 0 \& 6. 
In the figure, we plot the results of each CL method by varying their own hyperparameter for controlling the stability-plasticity trade-off. The upper point on each plot represents a lower regularization strength, a smaller memory size, or a lower pruning ratio. 
% We note that for PackNet, we only adjust a pruning ratio for the second task while a pruning ratio for the first task is fixed, because the knowledge learned from $T_1$ would be changed depending on the first pruning ratio. 

The following are our observations from the figures.
First, from the gap of blue triangles in Split CIFAR-100S and Split ImageNet-100 results, we observe that even with simple fine-tuning, the bias of $T_1$ adversely affects one of $T_2$, \ie, forward transfer of bias exists, which is consistent with Salman \etal \cite{salman2022does}. However, the overlapping blue triangles on CelebA show that the bias of $T_1$ does not always persist when not considering the stability, implying that the degree of forgetting for the bias of a model acquired from previous tasks can be different depending on the types of bias.
Second, we observe that when applying CL methods, the gap between colored and uncolored points for similar Normalized $\mathcal{F}-\mathcal{I}$ is mostly larger than fine-tuning. Moreover, the gap increases as Normalized $\mathcal{F}-\mathcal{I}$ becomes lower. Namely, these results clearly demonstrate that CL methods promote the forward transfer of bias because they tend to remember the knowledge of past tasks even if it contains some bias features. Furthermore, the extent of the transfer is increasing as CL methods focus on stability more. 
Finally, we observe that bias of $T_2$ is mostly better when learned after $T_1$ with bias level 0 than with bias level 6, under similar Normalized $\mathcal{F}-\mathcal{I}$. Therefore, we argue that before learning a new task, biases of a CL model obtained from previous tasks should be mitigated for learning the new tasks correctly. 

We additionally report the results on Split CIFAR-100S when the bias level of $T_2$ is 2 or 4 in Appendix, and observe similar trends from Figure \ref{fig:two_forward}. 


% \\ [Note that the model learning tasks with bias level 0 is not unbiased] \\


\subsection{Backward transfer of bias}
Now, we investigate the backward transfer of bias. Figure \ref{fig:two_backward} compares the bias of a model at $T_1$ by varying the bias of $T_2$, while the bias level of $T_1$ is fixed as level 0. We omit the results for PackNet, as it freezes the parameters updated in the previous tasks and thereby the predictions for previous tasks are not changed, \ie, any backward transfer does not occur. 

Figure \ref{fig:two_backward_cifar} and \ref{fig:two_backward_celeba} show the opposite trend of our previous results. Firstly, we observe that the bias gap for $T_1$ under similar Normalized $\mathcal{F}-\mathcal{I}$ is maximized by fine-tuning and minimized by model-freezing. Also, for each CL method, the gap becomes severer as the Normalized $\mathcal{F}-\mathcal{I}$ increases. This means that the more CL methods prioritize plasticity over stability, the more bias obtained from $T_2$ is transferred to $T_1$, i.e., the backward transfer of bias occurs more. Additionally, when the bias level of $T_2$ is 0, $T_1$ consistently exhibits lower bias, highlighting the need to address the potential bias of new incoming tasks. 

We identify that BMRs between colored and uncolored points are nearly identical in Figure \ref{fig:two_backward_imagenet}. To reason about this phenomenon, we analyzed the predictions from models and found out that it may be due to the old-new bias which is an inherent issue of Class-IL --- namely, predictions are biased towards new classes due to the imbalance between old and new task data samples. In other words, although $T_2$ does not contain the carton class, the converted image containing watermarks can be predicted to one of the new classes, leading to no difference in BMR. Please refer to Appendix for the related experimental results and more discussions. 

\noindent\textbf{Remarks on developing a new CL method}. 
From two-task analyses in Figure \ref{fig:two_forward} and \ref{fig:two_backward}, we observed that the bias of each task negatively affects the other tasks even if they do not contain the bias. This appeals that whenever we encounter a bias of an incoming task, we should consider learning the task without the bias and preventing forgetting of the previous tasks at the same time. As a straightforward solution, one may naively consider applying an existing debiasing technique (\eg, Group DRO) to the model obtained after learning a current task by a CL method. However, we can easily expect that the accuracy of previous tasks significantly drops whereas the bias of the current task can be successfully reduced
% with similar accuracy 
(we report the results of this scenario in Appendix). Hence, we argue that it is necessary to develop a novel approach for taking debiased learning into account while continual learning.

 \begin{figure}
    \centering
    \includegraphics[width=0.6\linewidth]{figures/cka_forward.pdf}
    \caption{\small\textbf{CKA on Split CIFAR-100S.} To observe the forward transfer of bias, the CKA between color and grayscale images in $T_2$ is shown according to the bias level of $T_1$ and the regularization strength, after learning up to $T_2$ by EWC.}
    \label{fig:cka_forward}
    \vspace{-.1in}
\end{figure}

\subsection{Feature representation analysis}
To provide more direct evidence of bias transfer, we analyze feature representations extracted from the penultimate layer of a DNN-based model using centered kernel alignment (CKA) with the linear kernel \cite{kornblith2019similarity}. CKA is an isotropic scaling-invariant metric for measuring the similarity between two representations of a model. The two plots in Figure \ref{fig:cka_forward} compare the CKA values on Split CIFAR-100S for EWC under the two-tasks settings similar to Figure \ref{fig:two_forward_cifar}. Namely, we evaluate models after learning $T_2$ by varying the regularization strength and the bias level of $T_1$. We then compute the CKA similarity between color and grayscale images in the test dataset of $T_2$. That means, a low CKA value indicates that representations for each group are different, \ie, the model possesses the color bias more. Figure \ref{fig:cka_forward} clearly shows that as the regularization strength is stronger and the bias of $T_1$ is more severe, CKA values decrease. Thus, we also observe the forward transfer of the bias through the analysis of feature representations. The CKA result showing the backward transfer is reported in Appendix.



\section{Case for CL with a longer sequence of tasks}
\label{sec:longer}
\begin{figure}[t!]
    \centering
    \begin{subfigure}[t]{\linewidth}
        \centering
        \includegraphics[width=0.7\linewidth]{figures/long_cifar_forward.pdf}
        \caption{\small {Forward transfer of bias}}
        \label{fig:long_forward_cifar}
    \end{subfigure}
    \begin{subfigure}[b]{\linewidth}
        \centering
        \includegraphics[width=0.7\linewidth]{figures/long_cifar_backward.pdf}
        \caption{\small {Backward transfer of bias}}
        \label{fig:long_backward_cifar}
    \end{subfigure}
    \caption{\small {\bf Bias transfers in a sequence of 10 tasks on Split-CIFAR100S}. The BMRs of $T_{10}$ or $T_1$ are shown after learning up to $T_{10}$ by CL methods.}
    \label{fig:long_cifar}
    \vspace{-.1in}
\end{figure}

We confirmed in the previous section that the bias is transferred forward and backward in tow-tasks CL scenarios. In this section, we further investigate the bias transferability in a sequence of multiple tasks. 

We note that for all figures in this section, we simplify the visual format for better clarity of the comparison; in detail, we divide a range of the normalized $\mathcal{F}-\mathcal{I}$ into three equal intervals and report a result for each interval. Given a CL scenario, we evaluate CL methods several times for varying their hyperparameters and select one of the results with the highest average accuracy for each interval. 
% Our findings indicate that biases not only transfer across tasks, but also tend to persist and accumulate while learning several tasks.

\subsection{Persistence of bias transfer in longer sequences}
\label{subsec:long_transfer}
% to verify that biases persist in longer CL scenarios,
Firstly, we consider a sequence of 10 tasks on Split CIFAR-100S to observe that the bias transfer exists in longer CL scenarios.
Similar to settings in Figure \ref{fig:two_forward}, we vary the bias level of the first or last task with level 0 or 6, while the bias level of all other tasks is fixed as level 0. The two plots in Figure \ref{fig:long_cifar} show BMR of $T_{10}$ (resp. $T_1$) according to the bias of $T_1$ (resp. $T_{10}$) for each CL method. 
% a long-term effect of the bias in a CL model. 
% we pick the result with similar accuracy among results of which the normalized $\mathcal{F}-\mathcal{I}$ corresponds to each interval. 

Figure \ref{fig:long_cifar} reveals an analogous trend with two-task analyses. Specifically, the two plots exhibit the BMR of $T_{10}$ (resp. $T_1$) always higher when the bias level of $T_1$ (resp. $T_{10}$) is 6.
Furthermore, the gap of BMR between them is at its widest when the normalized $\mathcal{F}-\mathcal{I}$ is low (resp. high). 
Namely, the bias transfers also occur in longer sequences of tasks. 
We additionally display the accuracy of $T_{10}$ and $T_{1}$ for each plot in Appendix and find that the accuracies are roughly the same, meaning that the gap of BMR is due to the bias transfer. 
Finally, we emphasize that from Figure \ref{fig:long_forward_cifar}, the color bias of $T_1$ can \textit{persist} even if a CL model learns nine additional tasks with the bias level of 0, especially when focusing on the stability.
% Hence, we can infer that the bias of a task at any point can influence the biases of all tasks in a CL scenario.

\subsection{Accumulation of the same type of bias}
We now verify whether the \textit{same} type of bias of each task would be \textit{accumulated} by CL methods; namely, when the number of previous tasks with the same type of dataset bias is increasing, the CL models make much more biased predictions for the current task. To this end, we take a sequence of 5 tasks with a bias level of 0 in Split CIFAR-100S. We then randomly select some of the four tasks except the last task and change the bias level of them to 4.

Figure \ref{fig:accumul_cifar} reports BMR values for LWF depending on the number of biased tasks. We clearly check that the BMR of $T_5$ increases as the number of biased tasks increases, and the increase is more significant in the low and middle ranges. Namely, it demonstrates that when tasks with the same type of bias are continuously upcoming, their biases accumulate in the CL model. In addition, we observe that when the number of biased tasks is low in the low range, the gap of BMR is small. It would be because if only the tasks in the middle of sequences have the dataset bias, their dataset bias could not be sufficiently learned under the low plasticity.
\begin{figure}
    \centering
    \includegraphics[width=0.75\linewidth]{figures/accumul_forward_cifar.pdf}
    \caption{\small {\bf Accumulation of the same type of bias on Split CIFAR-100S.} BMR of $T_5$ is shown depending on the number of biased tasks after up to learning $T_5$ by LWF.}
    \label{fig:accumul_cifar}
    \vspace{-.2in}
\end{figure}

\subsection{Accumulation of the different types of bias}
In order to investigate that the \textit{different} type of biases can also accumulate, we consider sequences of three tasks, each randomly picked from CelebA$^8$. We suppose that each task can include one of two kinds of dataset bias. That is, in the training datasets of each task, the class attribute, ``young'', can spuriously correlate with one of two group attributes, ``gender'' or `smiling''. Then, when bias levels of $T_3$  for both group attributes are fixed to 0, we compare the BMRs of $T_3$ depending on whether or not $T_1$ and $T_2$ have gender or smiling biases, respectively. 

The results are shown in Figure \ref{fig:accumul_celeba}, which displays the degree of gender and smiling bias at $T_3$ for EWC and ER. We find that when $T_1$ and $T_2$ are biased towards gender and smiling, respectively, BMR at $T_3$ are much higher in terms of both group attributes. Moreover, in most cases, we again observe the stronger forward transfer of bias when CL methods focus on the stability more. In Figure \ref{fig:accumul smiling bias}, the gap between colored and uncolored bars for EWC is bigger in the middle range, compared to the low range. We infer that this would be because the dataset bias of $T_2$ could be learned more under higher plasticity, resulting in the bias being more transferred to $T_3$. 

\begin{figure}
    \centering
    \begin{subfigure}{0.7\linewidth}
        \centering
        \includegraphics[width=\linewidth]{figures/accumul_gender_celeba.pdf}
        \caption{\small{DCA of $T_3$ for the gender bias}}
        \label{fig:accumul gender bias}
    \end{subfigure}
    \begin{subfigure}{0.7\linewidth}
        \centering 
        \includegraphics[width=\linewidth]{figures/accumul_smiling_celeba.pdf}
        \caption{\small{DCA of $T_3$ for the smiling bias}}
        \label{fig:accumul smiling bias}
    \end{subfigure}
    \caption{\small {\bf Accumulation of different types of bias on CelebA$^8$.} DCAs of $T_3$ for ``gender'' and ``smiling'' attributes are reported.}
    \label{fig:accumul_celeba}
    \vspace{-.1in}
\end{figure}


% \Input{$x_i$, $y_i$, $t_i$, $a_i$}
% \If{}

\section{Bias-aware Continual learning} 
\label{sec:comparison}
We demonstrated that the bias of each task can be transferred by naive CL methods in various continual learning situations, highlighting the need for a new approach to continual learning that considers debiasing each task. To address this need, we propose a simple, yet efficient baseline method, dubbed as \methodnamefull (\ours), that can be easily combined with existing CL methods while preventing forgetting of the learned tasks.


\subsection{\methodnamefull (\ours)}
Our proposed method is inspired by a recent CL and debiasing method, specifically GDumb (Greedy Sampler and Dumb Learner) \cite{prabhu2020gdumb} and DFR (Deep Feature Re-weighting) \cite{kirichenko2022last}. GDumb greedily stores a small number of data into an exemplar memory equally for each class in all tasks and uses them to train a model from scratch during the test time. By learning all tasks at the same time, GDumb can improve the average accuracy over learned tasks. On the other hand, DFR retrains only the classification head using a small number of group-balanced data after training a model from scratch using an entire training data that may contain any dataset bias. The authors demonstrate that even if the model learned the biased feature  representations, DFR can remove them by only re-training the last layer using a small portion of group-balanced data. We emphasize that both methods are simple to implement and easily applicable to various settings, and can achieve better or comparable performance for CL or debiasing, respectively, compared to more complex and recent methods in its literature. 


\begin{table}[t]
\caption{\small {\bf The comparison of methods on Split CIFAR-100S.} The average accuracy and BMR over 10 tasks are shown. The $k$ denotes the size of exemplar memory, and the numbers in parentheses stand for the standard deviations of each result obtained with different seeds. }
\centering
\resizebox{0.85\columnwidth}{!} {
\begin{tabular}{lcc}
\toprule
Method & Avg. Acc. (\%) & Avg. BMR (\%) \\
\midrule
Fine-tuning     & 19.49 (1.46)                & 51.96 (10.96)          \\
+ BGS (k=1000)   & \textbf{42.75 (2.27)}                & \textbf{34.38 (5.70)}           \\
+ BGS (k=2000)   & \textbf{47.44 (2.80)}                & \textbf{30.42 (5.71)}           \\
\midrule
LWF \cite{li2017learning}             & 59.61 (2.04)                & 28.00 (2.38)           \\
+ BGS (k=1000)   & \textbf{65.00 (1.20)}                & \textbf{18.18 (0.84)}           \\
+ BGS (k=2000)   & \textbf{66.88 (1.02)}                & \textbf{16.82 (0.90)}           \\
\midrule
EWC \cite{ewc}            & 35.57 (1.71)                & 46.65 (2.31)           \\
+ BGS (k=1000)   & \textbf{46.20 (1.08)}                & \textbf{31.65 (2.12)}           \\
+ BGS (k=2000)   & \textbf{49.30 (1.37)}                & \textbf{28.49 (1.84)}           \\
\midrule
ER (k=1000) \cite{er}    & 54.77 (1.68)                & 29.17 (3.81)           \\
+ BGS        & \textbf{59.38 (0.98)}                & \textbf{20.89 (2.28)}           \\
ER (k=2000)     & 60.75 (1.12)                & 25.47 (1.41)           \\
+ BGS    & \textbf{65.00 (0.74)}                & \textbf{17.62 (1.46)}           \\
\midrule
PackNet \cite{mallya2018packnet}            & 47.46 (1.52)                & 33.97 (6.10)           \\
+ BGS (k=1000)   & \textbf{47.69 (2.88)}                & \textbf{31.26 (3.93)}           \\
+ BGS (k=2000)   & \textbf{49.39 (2.86)}                & \textbf{29.15 (4.05)}           \\
\midrule
GDumb (k=1000) \cite{prabhu2020gdumb}  & 32.33 (1.98)                & 52.15 (4.95)           \\
GDumb (k=2000)  & 41.31 (1.99)                & 45.43 (3.56)           \\
% \midrule
% BGS (k=1000)     & 34.99 (1.29)                & 29.82 (2.55)           \\
% BGS (k=2000)     & 46.67 (8.22)                & 20.71 (1.3)            \\
\midrule
\midrule
LWF + Group DRO \cite{groupdro} & 59.39 (1.41)                & 23.35 (2.72)          \\
\bottomrule
\end{tabular}
}
\vspace{-.1in}
\label{tab:cifar}
\end{table}


In a similar spirit, the algorithm of our \ours is two steps: first, during CL using a employed typical CL method, \ours stores group-class balanced data over all seen classes and groups in a greedy manner as the same process as GDumb, which shown is in Appendix. \ours then retrains only the classification heads of a neural network trained by the CL method by using the group-class balanced exemplar memory. By doing this process, we expect that \ours can mitigate the bias of the model like DFR while preventing forgetting of previous tasks. Moreover, we emphasize that \ours can be used in conjunction with any existing CL method without any additional hyperparameters. 

\subsection{Performance comparison}
We evaluate our \ours using 10-task sequences on Split CIFAR-100S and 8-task sequences on CelebA$^8$. Each task has a random bias level ranging from 0 to 6 (we did not conduct experiments on Split ImageNet-100 due to the lack of group labels (\ie, watermark labels) in the training dataset). We compare the standard CL methods including GDumb, and combinations with the CL methods and \ours. Additionally, we evaluate naive combination with the best performing regularization CL method for each dataset and a debiasing technique, Group DRO. We tuned the hyperparameters for each CL method and Group DRO based on the average accuracy and BMR up to $T_3$, following the hyperparameter selection protocol used in \cite{mai2022online} and used them to learn the rest of the tasks. 

Table \ref{tab:cifar} and \ref{tab:celeba} present the average accuracy and BMR over all tasks for each method on Split CIFAR-100S and CelebA$^8$. We evaluate replay based methods, ER, GDumb, and \ours, with the two kinds of memory size, which correspond to 10 or 20 images per class, respectively. From the tables, we first observe that applying \ours into the standard CL methods leads to improve BMR in all cases and the average accuracy on Split CIFAR-100S, while slightly dropping the average accuracy on CelebA. In addition, we obviously see that the performance gain by \ours in terms of CL and debiasing performances increases when using a large exemplar memory. We emphasize that such improvements from \ours require any additional hyperparameter tuning for debiasing. On the other hand, although applying Group DRO to CL methods shows good performance on CelebA$^8$ in terms of BMR, it needs additional hyperparameter tuning for Group DRO, which may be prohibitive for practice. Moreover, we observet that its improvement is marginal on Split CIFAR-100S since LWF + Group DRO  may fail to find a good hyperparameter for Group DRO when tuning it with only three tasks, not ten tasks. 

\begin{table}[t]
\caption{\small {\bf The comparison of methods on CelebA$^8$.} The other settings are identical to Table \ref{tab:cifar}. If there is an improvement by \ours, the result is shown in bold.}
\centering
\resizebox{0.9\columnwidth}{!} {
\begin{tabular}{lcc}
\toprule
Method          & Avg. Acc. (\%) & Avg. DCA (\%) \\
\midrule
Fine-tuning     & 79.77 (1.06)        & 34.31 (7.53)  \\
+ BGS (k=320)    & 79.62 (1.24)       & \textbf{31.42 (7.04)}   \\
+ BGS (k=640)    & 79.33 (1.86)        & \textbf{30.30 (6.95)}   \\
\midrule
EWC \cite{ewc}         & 80.19 (1.4)        & 36.30 (9.8)   \\
+ BGS(k=320)     & 79.15 (2.10)        & \textbf{33.87 (11.74)}   \\
+ BGS (k=640)    & 78.98 (1.72)        & \textbf{32.70 (11.64)}   \\
\midrule
ER (k=320) \cite{er}     & 80.99 (0.74)        & 37.68 (6.51)   \\
+ BGS (k=320)    & 80.19 (1.89)        & \textbf{31.04 (5.11)}   \\
ER (k=640)      & 81.00 (0.80)        & 37.50 (5.92)   \\
+ BGS (k=640)    & 79.98 (1.50)        & \textbf{30.49 (2.54)}   \\
\midrule
GDumb (k=320) \cite{prabhu2020gdumb}  & 69.38 (1.19)        & 36.45 (5.54)   \\
GDumb (k=640)   & 72.55 (1.26)        & 42.25 (1.48)   \\
% \midrule
% BGS (k=320)      & 59.75 (1.8)        & 12.53 (3.8)   \\
% BGS (k=640)      & 61.11 (0.5)        & 10.05 (1.4)   \\
\midrule
\midrule
% EWC + MFD \cite{mfd}       & 72.66 (7.6)        & 11.31 (3.6)   \\
EWC + Group DRO \cite{groupdro} & 75.32 (3.58)        & 21.79 (3.98)  \\
\bottomrule
\end{tabular}
}
\label{tab:celeba}
\vspace{-.1in}
\end{table}

\noindent \textbf{Remarks for limitations}. 
Although we demonstrated that \ours can improve average bias metric values in CL scenarios without requiring additional hyperparameter, it is important to note that our method does not fundamentally solve the bias-aware CL problem. Namely, even with \ours, the feature representation of a CL model could be still biased. Furthermore, \ours does not work in the absence of group labels in the training dataset, such as Split ImageNet-100. Nevertheless, we hope that \ours serves the purpose of a standard baseline for the bias-aware CL problems.


% \label{sec:experiemnts}
% In experiments, we investigate the bias of models as well as the accuracy in various scenarios of continual learning using a synthetic dataset, Split CIFAR-100S, and two real world datasets: ACSTravelTime \cite{retiring_adult} and FMoW-WILDS \cite{fmow2018, koh2021wilds}. 
% % we compare our MBC with various approaches using a synthetic dataset, Split CIFAR-100S, and a real world dataset, ACSTravelTime \cite{retiring_adult}.

% \noindent\textbf{Baselines} \hspace{2pt} We adopt a naive finetuning approach and typical continual learning baselines of LWF \cite{li2017learning}, EWC \cite{ewc}, ER \cite{er}. Each of them maintains the past knowledge in different forms of knowledge distillation \cite{hintonKD}, structural regularization, and memory-based rehearsal respectively. In addition, we consider naive combinations of the continual learning method, LWF, and existing debiasing techniques: Group DRO \cite{groupdro}, MMD-based fair distillation (MFD) \cite{mfd}, and label bias correction (LBC) \cite{jiang2020identifying}.

% \noindent\textbf{Implementation details} For Split CIFAR-100S, we use the same setting as in Figure  \ref{sec:case_study}. The details are provided in the supplementary material. For ACSTravelTime, we adopt a fully-connected network having two hidden layers each with 128 hidden nodes. We train the network for 20 epochs using AdamW optimizer \cite{adamw} with a learning rate of 0.0003 and a weight decay of 0.02. We did the grid search for hyperparameters of each method and report the results with the best ones after trained on the first five tasks. To consider continual learning debiased models, we train models on the first task with debiased techniques for all datasets. We give the results after the debiased learning and training details of this in the supplementary material.

% \subsection{Synthetic dataset}
% In this section, we compare \ours with baselines on the whole sequence of Split CIFAR-100S. We evaluate the accuracy and DCA of the models on all learned tasks after learning each task, and report the average accuracy and the average DCA over the tasks. The results are summarized in Figure \ref{fig:splitcifar100s_overall_results}.

% Other than finetuning and EWC, our \ours and all baselines show comparable average accuracy as shown in Figure \ref{fig:splitcifar100s_overall_acc}. However, looking into Figure \ref{fig:splitcifar100s_overall_dca}, existing continual learning approaches show that they do not mitigate the bias as expected. 
% While LWF with MFD reduces the average DCA as learning more tasks, our \ours outperforms it in terms of both the average accuracy and the average DCA throughout the learning.
% For finetuning, the overall bias also decreases, but it is since catastrophic forgettings occurs in the learned tasks and the accuracy between groups become similar.

% \subsection{Real world datasets}
% \noindent\textbf{ACSTravelTime} \hspace{2pt} ACSTravelTime \cite{retiring_adult} is a binary classification task to predict whether commute time of a person in the US is longer than 20 minutes. 
% % It is originally composed of data from different states in the US, and
% For this task, we select 10 states with the most data from 2018 US-wide ACS PUMS data. Since there are large distribution shifts between different states, a model trained for a certain state needs to be retrained for another state.  
% To consider a continual learning scenario, we suppose each state's data is collected and given to a model at different time of the year.
% For the group label, we binarize the race by dividing it into two groups: Caucasian and non-Caucasian. With this dataset, we investigate the change of accuracy and DCA between the two race groups over the states.


% \noindent\textbf{FMoW-WILDS} \hspace{2pt} FMoW-WILDS \cite{fmow2018, koh2021wilds} is a dataset for land use classification based on satellite images. It contains images from different geographical regions, and each region has the different number of the images.
% Without the regions and the classes of few images, we choose 40 classes from 3 different regions (Asia,~,~) and compose 10 different tasks having 4 distinct classes. We set the region as the group label to check the bias over the regions.






 \section{Conclusion}
 In this paper, we have presented a tactile manipulation system that is able to rotate different objects without vision. We showed an end-to-end reinforcement learning framework to learn tactile dexterity over the proposed system. We carried out experiments both in simulation and real to demonstrate its effectiveness. Our work demonstrated that we are able to achieve tactile dexterity as humans in real for the first time. In the future, there are many promising future directions to investigate, such as exploring the use of a more dense contact sensor array and scaling up the system to solve more diverse tasks. We hope that our work can pave the way for more intelligent robot hands.
{\small
\bibliographystyle{ieee_fullname}
\bibliography{references}
}

\clearpage
\onecolumn

\appendix
\appendix
\numberwithin{equation}{section}
\numberwithin{figure}{section}
\numberwithin{table}{section}
\section*{Appendix}
We offer supplementary materials in this document. Specifically, we provide the detailed algorithm of \ours in Section \ref{sec_append:algo}. In Section \ref{sec_append:details}, we present implementation details including model architectures, optimization, implementations of baseline methods, and the range of hyperparameters used. In addition, we report some additional results for a naive debiasing scenario in Section \ref{sec_append:naive_debiasing} and other various settings in Section \ref{sec_append:additional_results}. Finally, we discuss the results of our study for the backward transfer on Split ImageNet-100 in Section \ref{sec_append:discussion}.
\section{\ours sampling algorithm}
\label{sec_append:algo}
\begin{algorithm}
    \caption{\methodnamefull}
    \label{alg:ours}
    \SetKwInOut{Input}{Input}
    \SetKwInOut{Output}{Output}
    \SetKwInOut{Init}{Init}
    \Init{Memory $\mathcal{M}=\{\{\}, \dots, \}$ with capacity $K$, Labels $\mathcal{L}=\{\}$, Count $C=\{0,\dots\}$ 
    }
    \Input{a data sample $(x_i, a_i, y_i)$, task id $t$}
    
    \Output{$\mathcal{M}$}
    % $N_{y} \gets N_{y} + |\mathcal{Y}_{t}|$\;
    % $\mathcal{Y} \gets \mathcal{M}_{t-1}.\mathcal{Y} \cup \mathcal{Y}_{t}$\;
    $k \gets \frac{K}{|\mathcal{L}| \times |\mathcal{A}|}$\;
    \If{$C\big[(a_i, y_i, t)\big] < k$}{
    \eIf{$\Sigma_{a\in\mathcal{A}, (y,t)\in\mathcal{L}} C\big[(a,y,t)\big] < K$}{
        $\mathcal{M}\big[(a_i, y_i, t)\big] = \mathcal{M}\big[(a_i, y_i, t)\big] \cup (x_i, a_i, y_i, t)$\;
    } 
    {
        $(a_j, y_j, t_j)$ = $\operatorname{argmax}_{(a, y, t)} C\big[(a,y,t)\big] $\;
        $\mathcal{M}\big[(a_j, y_j, t_j)\big].\operatorname{pop}()$\;
        $\mathcal{M}\big[(a_i, y_i, t)\big] = \mathcal{M}\big[(a_i, y_i, t)\big] \cup (x_i, a_i, y_i, t)$\;
        $C[(a_j, y_j, t_j)] = C[(a_j, y_j, t_j)] - 1$ \;
    }
    \If{$C\big[(a_i, y_i, t)\big]==0$}{
        $\mathcal{L} = \mathcal{L} \cup (y_i, t)$\;
    }
        $C\big[(a_i, y_i, t)\big] = C\big[(a_i, y_i, t)\big] + 1$\;
    }
\end{algorithm}
\section{More implementation details}
\label{sec_append:details}
\subsection{Model architectures and optimization}
For all datasets, we used the AdamW optimizer \cite{adamw} with the following hyperparameters: learning rate of 0.001, weight decay of 0.01, $\beta_1$ of 0.9 , $\beta_2$ of 0.999, and $\epsilon$ of $10^{-8}$. 
For Split CIFAR-100S, we trained ResNet-56 \cite{resnet} from scratch for 70 epochs using a batch size of 256. 
For CelebA and Split ImageNet-100, we trained ResNet-18 from scratch for 50 and 70 epochs, respectively, using a batch size of 128. We incorporated the cosine annealing learning rate decay, with the maximum number of iterations set to the same as the number of training epochs.

\subsection{Implementations of continual learning methods}
In the original Elastic Weight Consolidation (EWC) algorithm \cite{ewc}, the snapshot of a CL model should be stored whenever the model is updated from a new task. This stored model is then used to calculate the importance scores of model parameters in the new task. Namely, the algorithm requires a linearly growing amount of memory to store a sequence of models, which is space-inefficient. To address this issue, we implemented online EWC, proposed in \cite{schwarz2018progress}, which averages the importance scores in an online manner without storing a set of models. 

For Learning Without Forgetting (LWF) \cite{li2017learning}, we used an average of the distillation losses for each head to balance between the cross entropy loss and the distillation losses.

For Experience Replay (ER) \cite{er}, incremental classifier and representation learning (iCaRL) \cite{rebuffi2017icarl} and Packnet \cite{mallya2018packnet}, we implemented the same as their original versions. For ER, we employed the Reservoir sampling \cite{reservoir}  as a strategy for updating the exemplar memory.  

\subsection{Hyperparameters for each result}
In our experiments, we evaluated CL methods several times by varying their hyperparameters based on the sets of candidates. The  candidates are uniformly distributed on a logarithmic scale within a given range. For all figures presented in Section 4 and 5, we then plotted results for some of those hyperparameter candidates. We note that we omitted some overlapped results in Section 4 to enhance visibility. For the table results in Section 6, we tuned the hyperparameters for regularization based methods and Group DRO using the same candidates. We included Table \ref{table:hyperparams} to provide full ranges of hyperparameters tested. 
We note that the memory size specified in Table \ref{table:hyperparams} represents the fraction of the number of training images included in one of all tasks except the last one. 
\begin{table}[t]
    % \vskip -0.5in
    \caption{\small \textbf{Hyperparameter search ranges.}}
    \label{table:hyperparams}
    \centering
    \small
    \begin{tabular}{ccc}
    \toprule
        Method & Hyperparameter & Search range \\ \midrule
        EWC \cite{ewc} & Regularization strength $\lambda$ &[$10^0, 10^9$] \\\midrule
        LWF \cite{li2017learning} & Regularization strength $\lambda$ &[$10^{-2}, 3\times10^{2}$] \\\midrule
        ER \cite{er} & Memory size & [$10^{-3}, 10^{0}$] \\ \midrule
        iCaRL \cite{rebuffi2017icarl} & Memory size & [$10^{-3}, 10^{0}$]  \\ \midrule
        PackNet \cite{mallya2018packnet}  & Pruning ratio $r$ & [$10^{-1}, 8\times10^{-1}$] \\ \midrule
        Group DRO \cite{groupdro}  & Learning rate of $q$ &  [$10^{-8}, 10^2$] \\ 
        \bottomrule
    \end{tabular}
\end{table}
\subsection{Datasets}
\noindent\textbf{Split CIFAR-100S}. In the training dataset of each task, the first five classes are skewed toward the color group and the latter skewed toward the grayscale group, given the skew ratio. For the test dataset of each task, we have pairs of the same images; ones in color, and ones in grayscale. Fig \ref{fig:split_cifar100s_samples} illustrates some training and test samples in Split CIFAR-100S.

\begin{figure*}[t!]
    \centering
    \begin{subfigure}[t]{0.4\linewidth}
        \includegraphics[width=0.9\linewidth]{figures/cifar100_sample_train.pdf}
        \caption{Training samples}
        \label{fig:split_cifar100s_train}
    \end{subfigure}
    % \hspace{1cm}
    \begin{subfigure}[t]{0.4\linewidth}
        \includegraphics[width=0.9\linewidth]{figures/cifar100_sample_test.pdf}
        \caption{Test samples}
        \label{fig:split_cifar_100s_test}
    \end{subfigure}
    \caption{\small {\bf Samples in a certain task with bias level of 2 in Split CIFAR-100S.} Each row represents a specific class within the task. The top three rows represent classes biased toward the grayscale samples, while the bottom three rows contain classes biased toward the color samples. The test dataset includes pairs of images, where each pair contains one grayscale and one color version of the same image.}
    \label{fig:split_cifar100s_samples}
\end{figure*}

\begin{figure*}[t!]
    \centering
    \begin{subfigure}[t]{0.4\linewidth}
        \includegraphics[width=0.9\linewidth]{figures/Imagenet_samples_without_carton.pdf}
        \caption{Without ``Carton'' class}
        \label{fig:split_imagenet_train_wo_carton}
    \end{subfigure}
    % \hspace{1cm}
    \begin{subfigure}[t]{0.4\linewidth}
        \includegraphics[width=0.9\linewidth]{figures/Imagenet_samples_with_carton.pdf}
        \caption{With ``Carton'' class}
        \label{fig:split_imagenet_train_w_carton}
    \end{subfigure}
    \caption{\small \textbf{Training samples in Split ImageNet-100}. Two plots show training samples in a certain task with different bias levels, \ie, 0 \& 6. Each row represents a specific class within the task. The last row of the right plot is the carton class.}
    \label{fig:split_imagenet_train_samples}
\end{figure*}

\begin{figure*}[t!]
    \centering
    \begin{subfigure}[t]{0.4\linewidth}
        \includegraphics[width=0.9\linewidth]{figures/Imagenet_test_samples_without_water.pdf}
        \caption{Test samples in a task without watermark}
        \label{fig:split_imagenet_test_wo_water}
    \end{subfigure}
    % \hspace{1cm}
    \begin{subfigure}[t]{0.4\linewidth}
        \includegraphics[width=0.9\linewidth]{figures/Imagenet_test_samples_with_water.pdf}
        \caption{Test samples in a task with watermark}
        \label{fig:split_imagenet_test_w_water}
    \end{subfigure}
    \caption{\small \textbf{Test samples in Split ImageNet-100}. Both sides of samples are used to compute BMR.}
    \label{fig:split_imagenet_test_samples}
\end{figure*}

\noindent\textbf{CelebA}. Domain-IL typically assumes the input distributions vary as the number of tasks increases. To reflect this assumption in CL scenarios, we divided the CelebA dataset into several tasks based on some selected attributes and each task thereby has different facial features. For instance, the two tasks in CelebA$^2$ are defined by whether face images are ``smiling'' or not. Similarly, we utilize three attributes, ``Black Hair'', ``Oval Face'', and ``Mouth slightly open'' for CelebA$^8$. By dividing the data in this way, CL scenarios based on CelebA$^2$ or CelebA$^8$ mimic the distribution shift occurring in real-world applications.

\noindent\textbf{Split ImageNet-100}. To study the impact of the watermark bias in CL scenarios, we replace a randomly selected class with the ``Carton'' class in a certain task. To calculate BMR for the watermark bias, we make watermarked versions of each sample in the original test dataset by the style transfer used in \cite{li2022whac}. Examples of training and test samples of Split ImageNet-100 are illustrated in Fig \ref{fig:split_imagenet_train_samples} and \ref{fig:split_imagenet_test_samples}.

\section{Naive debiasing in CL scenarios}
\label{sec_append:naive_debiasing}
Here, we assume a scenario in which the bias of a model is detected after learning $T_1$ (stage 1) and $T_2$ (stage 2) continually. After that, the model is re-trained using existing debiasing techniques to obtain a debiased model (stage 3). For debiasing, we employ an existing debiasing technique, Group DRO, \cite{groupdro} which minimizes the worst-case group loss.
For this scenario, we set the bias level of $T_1$ and $T_2$ as 0 and 6 respectively. Figure
\ref{fig:naive_debiasing_gdro} show the accuracy and BMR of $T_2$ (left) and $T_1$ (right) at each stage for each baseline. We plotted the results of each baseline with hyperparameters achieving the highest average accuracy of the two tasks.
In the right plot, we observe that some points shift to the bottom left as progressing from stage 1 to stage 2, \ie, forgetting of $T_1$.
% It means that as the stability gets less focus, the bias obtained from T2 is more transferred to T1, i.e., the backward transfer of bias. 
From results of the stage 3 in the left plot, we show that BMR of $T_2$ can be reduced by Group DRO and MFD.
% with similar accuracy. 
However, we also identify that the accuracy of $T_1$ significantly drops. Thus, when debiasing after learning each task as we argued in Section \ref{sec:two_task_studies}, one should consider forgetting issue of the learned tasks at the same time, \ie, it is necessary to develop a novel debiasing method considering the stability for CL.
% As shown in \cref{fig:naive_debiasing}, 

\begin{figure}
    \centering
    \includegraphics[width=0.6\linewidth]{figures/two_task_cl_debiased_pareto_gdro.pdf}
    \caption{\small \textbf{Naive debiasing with Group DRO on Split CIFAR-100S.} The accuracy and BMR of $T_1$ and $T_2$ are shown for each stage.}
    \label{fig:naive_debiasing_gdro}
\end{figure}

% \begin{figure}
%     \centering
%     \includegraphics[width=0.6\linewidth]{figures/two_task_cl_debiased_pareto_mfd.pdf}
%     \caption{\small \textbf{Naive debiasing with MFD on Split CIFAR-100S.} The accuracy and BMR of $T_1$ and $T_2$ are shown for each stage.}
%     \label{fig:naive_debiasing_mfd}
% \end{figure}


\section{Additional experimental results}
\label{sec_append:additional_results}
% \subsection{Further analyses for CL with two-tasks}
\subsection{Forward and backward transfers of bias for CL with two tasks}
Figure \ref{fig:two_forward_more_levels} displays the forward transfer of the color bias for two task-CLs on Split CIFAR-100S. In each plot in the figure, the bias levels of $T_2$ are fixed to 2 or 4, respectively, and BMR of $T_1$ is reported for each CL method and hyperparameter. It is apparent from the figure that the difference in BMR between colored and uncolored points becomes more pronounced with the bias transfer, as compared to results obtained when the bias level of $T_2$ is fixed to 0. This would be because previously learned biases of a CL model tend to facilitate learning of the dataset bias of the current task more. 

\begin{figure*}[t!]
    \centering
    \begin{subfigure}[t]{0.4\linewidth}
        \includegraphics[width=0.9\linewidth]{figures/two_cifar_forward_level2.pdf}
        \caption{Bias level of $T_2$ : 2}
        \label{fig:two_forward_cifar_level2}
    \end{subfigure}
    \begin{subfigure}[t]{0.4\linewidth}
        \includegraphics[width=0.9\linewidth]{figures/two_cifar_forward_level4.pdf}
        \caption{Bias level of $T_2$ : 4}
        \label{fig:two_forward_cifar_level4}
    \end{subfigure}
    \caption{\small{\bf Forward transfer of bias in two tasks-continual learning on Split CIFAR-100S}. }
    \label{fig:two_forward_more_levels}
\end{figure*}


Similarly, in Figure \ref{fig:two_backward_more_levels}, we present the outcomes of two-task experiments for analysis of the backward transfer of bias. As in Figure \ref{fig:two_backward_more_levels}, the results show that if $T_1$ already contains a dataset bias, the effect of the backward transfer of the bias from $T_2$ is more pronounced. 

\begin{figure*}[t!]
    \centering
    \begin{subfigure}[t]{0.4\linewidth}
        \includegraphics[width=0.9\linewidth]{figures/two_cifar_backward_level2.pdf}
        \caption{Bias level of $T_1$ : 2}
        \label{fig:two_backward_cifar_level2}
    \end{subfigure}
    \begin{subfigure}[t]{0.4\linewidth}
        \includegraphics[width=0.9\linewidth]{figures/two_cifar_backward_level4.pdf}
        \caption{Bias level of $T_1$ : 4}
        \label{fig:two_backward_cifar_level4}
    \end{subfigure}
    \caption{\small{\bf Backward transfer of bias in two tasks-continual learning on Split CIFAR-100S}.}
    \label{fig:two_backward_more_levels}
\end{figure*}

\subsection{Feature representation analysis for backward transfer of bias}
We exhibit the results of CKA analysis for the backward transfer on Split CIFAR-100S. From Figure \ref{fig:cka_backward}, we observe a clear trend indicating the CKA value decreases as the regularization strength decreases and the bias level of $T_2$ increases. This again suggests that when a CL method focuses on learning a biased current task, \ie, plasticity, the backward transfer of bias by a CL method becomes more obvious. 
\begin{figure}
    \centering
    \includegraphics[width=0.4\linewidth]{figures/cka_backward.pdf}
    \caption{\small {\bf CKA on Split CIFAR-100S.} The CKA values between color images and grayscale images in $T_1$ are shown. Each value is calculated after learning $T_2$ by EWC.}
    \label{fig:cka_backward}
\end{figure}

\subsection{Experimental results for accuracy with a longer sequence of tasks}
Figure \ref{fig:long_cifar_acc} and \ref{fig:accumul_acc} show accuracies corresponding to each of the results in Figure 5 and 6 in the manuscript. The figures demonstrate that the accuracies in the same intervals are roughly the same, so we conclude that the gaps of BMR shown in Figure 5 and 6 are due to the bias transfers, not the accuracy gaps.
\begin{figure}
    \centering
    \begin{subfigure}{0.45\linewidth}
        \centering
        \includegraphics[width=0.9\linewidth]{figures/long_cifar_forward_acc.pdf}    
        % \caption{Forward transfer of bias}
    \end{subfigure}
    \begin{subfigure}{0.45\linewidth}
        \centering
        \includegraphics[width=0.9\linewidth]{figures/long_cifar_backward_acc.pdf}    
        % \caption{d}
    \end{subfigure}    
    \caption{\small {\bf Accuracy in longer sequences of Split CIFAR-100S.} The experimental settings in two plots are the same as in Figure 5(a) and 5(b), respectively.}
    \label{fig:long_cifar_acc}
\end{figure}

\begin{figure}
    \centering
    \includegraphics[width=0.4\linewidth]{figures/accumul_forward_cifar_acc.pdf}
    \caption{\small{\bf Accuracy of $T_5$ depending on the number of biased tasks.} The experimental settings are the same as in Figure 6.}
    \label{fig:accumul_acc}
\end{figure}

\begin{table}[]
\centering
\caption{\small\textbf{Mis-classified ratio of $T_2$ test data when watermark is added either after learning ``carton'' in $T_1$ or not.} We reported the results with memory capacity which is 100\% of $T_1$ data to consider the stability.}
\resizebox{0.5\linewidth}{!}{
\begin{tabular}{ccc}
\toprule
\multirow{2}{*}{Prediction} & \multicolumn{2}{c}{Mis-classified ratio (\%)}              \\ 
                             & \multicolumn{1}{c}{$T_1$ with ``carton''} & $T_1$ without ``carton'' \\ \midrule
Old class                  & \multicolumn{1}{c}{4.52}             & 7.78                \\ 
New class                  & \multicolumn{1}{c}{7.74}             & 6                \\ 
``Carton'' class (old)            & \multicolumn{1}{c}{4.02}             & -                   \\ \midrule 
Total (BMR)             &      16.28   &    13.78           \\ \bottomrule
\end{tabular}
}
\label{tab:forward_image_mis}
\end{table}

\begin{table}[]
\centering
\caption{\small \textbf{Mis-classified ratio of $T_1$ test data when watermark is added either after learning ``carton'' in $T_2$ or not.} We reported the results with memory capacity which is 10\% of $T_1$ data to consider the plasticity.}
\resizebox{0.5\linewidth}{!}{
\begin{tabular}{ccc}
\toprule
\multirow{2}{*}{Prediction} & \multicolumn{2}{c}{Mis-classified ratio (\%)}              \\ 
                             & \multicolumn{1}{c}{$T_2$ with ``carton''} & $T_2$ without ``carton'' \\ \midrule
Old class                  & \multicolumn{1}{c}{2.06}             & 1.38              \\ 
New class                  & \multicolumn{1}{c}{21.61}            & 28.06               \\ 
``Carton'' class (new)            & \multicolumn{1}{c}{5.69}            & -                   \\ \midrule 
Total (BMR)             &    29.36     &  29.44            \\ \bottomrule
\end{tabular}
}
\label{tab:backward_image_mis}
\end{table}

\section{Discussions about results for backward transfer on Split ImageNet-100}
\label{sec_append:discussion}
In this section, we analyze the predictions of a models in order to investigate why the backward transfer is not observed in Class-IL scenarios with two-tasks on Split ImageNet-100. 
% to investigate the effect of the old-new bias of Class-IL on the backward transfer of bias. 
Table \ref{tab:forward_image_mis} (resp. Table \ref{tab:backward_image_mis}) represents the CL scenarios for the forward (resp. backward) transfer of bias,  which evaluates the bias for $T_2$ (resp. $T_1$) by varying the bias level of $T_1$ (resp. $T_2$), \ie, whether the ``carton'' class is contained in the task or not. We employ ER in our experiments and set the memory size as 1 in Table \ref{tab:forward_image_mis} or 0.1 in Table \ref{tab:backward_image_mis}, to make forward and backward transfer of bias significantly occurs by focusing more on stability or stability, respectively. 
% to store 100\% of $T_1$ data for Table \ref{tab:forward_image_mis} and 10\% of $T_1$ data for Table \ref{tab:backward_image_mis} to consider the model highly focusing on the stability and the plasticity respectively. 
% In the tables, we report BMR for each scenario and 
To analyze BMR values in more detail, we divided the cases of misclassification  of bias-conflicted samples into three categories; a CL model falsely predicts for 1) one of the old classes (except the carton), 2) one of the new classes (except the carton), and 3) the ``carton'' class. 

The following are our observations from the tables.
First, we observe that BMRs of $T_1$ in Table \ref{tab:backward_image_mis} show almost no difference regardless of whether a CL model learns the ``carton'' class under high plasticity, while BMR of $T_2$ increases in Table \ref{tab:forward_image_mis} when $T_1$ contains the ``carton'' class. Namely, in terms of BMR, it looks like the backward transfer of the watermark bias does not occur. However, when we look at the categorized results, we can derive different trends, \ie, the backward transfer still occurs. In Table \ref{tab:backward_image_mis}, when $T_2$ contains the ``carton'' class, some samples with the watermark are misclassified into the ``carton'' class, \ie, the CL model possesses watermark bias in $T_1$ due to backward transfer. We then argue that the main cause of similar BMRs in Table \ref{tab:backward_image_mis} would be the \textit{old-new bias}, an inherent issue of Class-IL, which indicates biased predictions towards new classes. Indeed, when the watermark is injected into samples in $T_1$ which belongs to old classes, we see a disproportionately high ratio of incorrectly predicted samples for the new classes in Table \ref{tab:backward_image_mis}. On the other hand, we observe that the ratios for old and new classes are relatively similar in Table \ref{tab:forward_image_mis}, since the watermark bias is injected to samples in $T_2$ which belongs to new classes. Thus, we can infer that the old-new bias can make the model predictions for samples in $T_1$ vulnerable to watermark bias and easily shift to new classes in $T_2$, even if the model did not explicitly learn the ``carton'' class. This explains high misclassified ratios for new class regardless of the dataset bias of $T_2$ and similar BMRs in Table \ref{tab:backward_image_mis} although the backward transfer of bias occurs. 
% In addition, we calculated the CKAs from the two models in \cref{tab:backward_image_mis} and find out that a model learning $T_2$ with the ``carton'' class has lower CKA by 0.12, which can be used in another evidence for the backward transfer.


% In Table \ref{tab:forward_image_mis} and \ref{tab:backward_image_mis}, we computed mis-classified ratio of correctly predicted samples when the watermark is added. 
% To investigate how the mis-classifications are made, we divide them into three cases according to the predictions.
% In Table \ref{tab:forward_image_mis}, we report the results when the memory stored the whole $T_1$ training data, \ie, high stability, when learning ``carton'' class in $T_1$ or not. We clearly observe that adding watermark leads more images of $T_2$ to be mis-classified as ``carton'' after learning ``carton'' in $T_1$, \ie, the forward transfer of bias, while the total ratios of old and new classes are similar.
% In Table \ref{tab:backward_image_mis}, we report the results when the memory stored 10\% of $T_1$ training data, \ie, high plasticity, when learning ``carton'' class in $T_2$ or not. In this case, we again observe that some images of $T_1$ are mis-classified as ``carton'' after learning ``carton'' in $T_2$. However, even when not learning ``carton'', the model predicted the old task samples to another new class and this made BMR of two different models be almost the same. We believe this is because the model is vulnerable to a random perturbation of an image and tends to predict the samples to the recently learned classes when the old task samples was not considered enough while learning a new task, \ie, old-new bias of Class-IL.
% A model in Class-IL tends to suffer from a phenomenon which is to predict samples from old classes to the one of the recently learned classes when the old samples are not considered enough while learning a new task. 

% To interpret the phenomenon, we counted predictions of the model for the test data of two tasks $T_1$ and $T_2$ after learning $T_2$ in Table \ref{tab:backward_image_oldnew}. As shown in the table, the model predicted most of samples from old classes to the new classes in both cases when learning ``carton'' and not. Furthermore, when the watermark is added, more old samples are predicted to the new classes, which shows the vulnerability of a model to the perturbation. Similar trend is observed in Table \ref{tab:forward_image_oldnew} when the memory capacity is large. In the meantime, the table also shows that few samples are predicted to old classes when the watermark is added, and it increased slightly more in the case when learning ``carton'' in $T_1$ which represents, due to the old-new bias, robust predictions could be made to the new task samples more easily.
% more new class samples are predicted to old classes when learning 'carton' in $T_1$ than when not learning 'carton' which represents 
% Table \ref{tab:forward_image_oldnew} and \ref{tab:backward_image_oldnew} show predictions of models for the test data of two tasks $T_1$ and $T_2$ after learning $T_2$ with 100\%  and 10\% of $T_1$ training data, respectively. Settings of those two tables are correspond to table \ref{tab:forward_image_mis} and \ref{tab:backward_image_mis} respectively.

% , we investigated the predictions of models for the test data of $T_1$ and $T_2$ after learning $T_2$ when we stored 100\% and 10\% of $T_1$ training data respectively. 
% In those tables, we observe that the model predicted original samples from old classes to new classes more when the less memory capacity is used. 
% Moreover, adding watermark on the test samples, let the model 




% Please add the following required packages to your document preamble:
% \usepackage{multirow}


% Please add the following required packages to your document preamble:
% \usepackage{multirow}


% Please add the following required packages to your document preamble:
% \usepackage{multirow}

% Please add the following required packages to your document preamble:
% \usepackage{multirow}



% \begin{table}[t]
% \centering
% \caption{\small \textbf{Model predictions in the scenario to study the forward transfer of bias.} We reported the results with memory capacity which is the same as the number of $T_1$ data to consider the stability.}
% \resizebox{0.7\linewidth}{!}{
% \begin{tabular}{cccccc}
% \toprule
% \multirow{2}{*}{Label}      & \multirow{2}{*}{Prediction} & \multicolumn{2}{c}{$T_1$ with ``carton''}           & \multicolumn{2}{c}{$T_1$ without ``carton''}        \\ 
%                              &                              & \multicolumn{1}{c}{Original} & Watermark-added & \multicolumn{1}{c}{Original} & Watermark-added \\ \midrule
% \multirow{2}{*}{Old class} & Old class                  & \multicolumn{1}{c}{443}      & 410             & \multicolumn{1}{c}{446}      & 410             \\ 
%                              & New class                  & \multicolumn{1}{c}{57}       & 90              & \multicolumn{1}{c}{54}       & 90              \\ \midrule
% \multirow{2}{*}{New class} & Old class                  & \multicolumn{1}{c}{56}       & 79              & \multicolumn{1}{c}{56}       & 75              \\ 
%                              & New class                  & \multicolumn{1}{c}{444}      & 421             & \multicolumn{1}{c}{444}      & 425           \\ \bottomrule
% \end{tabular}
% }
% \label{tab:forward_image_oldnew}
% \end{table}
% \begin{table}[]
% \centering
% \caption{\small \textbf{Model predictions in the scenario to study the backward transfer of bias.} We reported the results with memory capacity which is 10\% of $T_1$ data to less consider the stability.}
% \resizebox{0.7\linewidth}{!}{
% \begin{tabular}{cccccc}
% \toprule
% \multirow{2}{*}{Label}      & \multirow{2}{*}{Prediction} & \multicolumn{2}{c}{$T_2$ with ``carton''}           & \multicolumn{2}{c}{$T_2$ without ``carton''}        \\ 
%                              &                              & \multicolumn{1}{c}{Original} & Watermark-added & \multicolumn{1}{c}{Original} & Watermark-added \\ \midrule
% \multirow{2}{*}{Old class} & Old class                  & \multicolumn{1}{c}{294}      & 208             & \multicolumn{1}{c}{280}      & 196             \\  
%                              & New class                  & \multicolumn{1}{c}{206}      & 292             & \multicolumn{1}{c}{220}      & 304             \\ \midrule
% \multirow{2}{*}{New class} & Old class                  & \multicolumn{1}{c}{6}        & 5               & \multicolumn{1}{c}{7}        & 8               \\
%                              & New class                  & \multicolumn{1}{c}{494}      & 495             & \multicolumn{1}{c}{493}      & 492             \\ \bottomrule
% \end{tabular}
% }
% \label{tab:backward_image_oldnew}
% \end{table}

%------------------------------------------------------------------------
% \section{Final copy}

% You must include your signed IEEE copyright release form when you submit
% your finished paper. We MUST have this form before your paper can be
% published in the proceedings.


\end{document}