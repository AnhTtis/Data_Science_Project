\pdfoutput=1
 % \documentclass[12pt, journal,onecolumn,draftclsnofoot]{IEEEtran}
\documentclass[12pt, draftclsnofoot, onecolumn]{IEEEtran}
% \documentclass[journal]{IEEEtran}
\IEEEoverridecommandlockouts
% \usepackage[left=0.67in,right=0.67in,top=0.7in,bottom=1.1in]{geometry}
\usepackage{amsmath,graphicx,epstopdf,amssymb,amsthm}% ,epstopdf,spconf,amssymb,epsfig,fullpage}%, ,graphicx,psfrag,url,amsmath,amsthm,comment} 
%\usepackage{acronym} 
\usepackage{cite} 
\usepackage{hyperref}
% \usepackage{ulem,color}   
\usepackage{epstopdf} 
\usepackage{amsmath,bm}    
\usepackage{verbatim} 
%\usepackage{unicode-math} 
%\usepackage{graphicx}  
%\usepackage[]{graphics} 
\newcommand{\ie}{\textit{i.e. }}
\newcommand{\eg}{\textit{e.g. }}
%\def\yesnumber{\global\@myeqnswtrue}
%\newcommand{\noteout}[1]{\textcolor{red}{\sout{}}}       
%\def\noteout    
\usepackage[utf8]{inputenc}     
\usepackage[english]{babel}   
%\usepackage{slashbox}  
\usepackage{caption}    
\usepackage{enumitem} 
\usepackage[ruled,vlined,linesnumbered]{algorithm2e}
\DeclareCaptionLabelFormat{lc}{\MakeLowercase{#1}~#2}
\captionsetup{labelfont=sc,labelformat=lc}
\usepackage[dvipsnames]{xcolor}
 
\newenvironment{skproof}{\noindent\textit{Sketch of  Proof:}}{\hfill$\blacksquare$}
\newcommand\numberthis{\addtocounter{equation}{1}\tag{\theequation}}
\newtheorem{theorem}{Theorem}
\newtheorem{lemma}{Lemma} 
\newtheorem{fact}{Fact}
\newtheorem{definition}{Definition}
\newtheorem{proposition}{Proposition}
\newtheorem{corollary}{Corollary}
\newtheorem{example}{Example}
\newtheorem{remark}{Remark}
\newtheorem{claim}{Claim} 
\newtheorem{assumption}{Assumption}
\newtheorem{condition}{Condition}
\newtheorem{observation}{Observation}
\newcommand{\note}[1]{\textcolor{blue}{#1}}
\newcommand{\rem}[1]{{\color{blue} {\bf [REMOVE: #1]}}}
\allowdisplaybreaks
\usepackage[protrusion=true,expansion=true]{microtype}
\pdfoutput=1
\renewcommand{\figurename}{Fig.}
\usepackage[font=small]{caption}
\newcommand{\ali}[1]{{\color{magenta} {{\bf}#1}}}
\newcommand{\chris}[1]{{\color{ForestGreen} {{\bf [Chris: #1]}}}}
\newcommand{\shams}[1]{{\color{cyan} {{\bf [S: #1]}}}}
\newcommand{\mathcolorbox}[2]{\colorbox{#1}{$\displaystyle #2$}}
\usepackage{float}
\usepackage{mathtools}
\newcommand{\frank}[1]{{\color{blue} {\bf FL: #1}}}
\newcommand{\addFL}[1]{\textcolor{blue}{#1}}
\DeclarePairedDelimiter\floor{\lfloor}{\rfloor}
\newcommand{\cgb}[1]{{\color{blue} {\bf CGB: #1}}}


\newcommand{\nm}[1]{{\color{red} {\bf [NM: #1]}}}
\newcommand{\add}[1]{\textcolor{red}{#1}}
\newcommand{\sst}[1]{\st{#1}}
\usepackage{cancel}
\newcommand\mst[2][red]{\setbox0=\hbox{$#2$}\rlap{\raisebox{.45\ht0}{\textcolor{#1}{\rule{\wd0}{2pt}}}}#2}   
\usepackage{soul,xcolor}
\DeclareMathOperator*{\argmin}{arg\,min}
\DeclareMathOperator*{\argmax}{arg\,max}
% \newtheorem{definition}{Definition}
% \newtheorem{lemma}{Lemma}
% \newtheorem{theorem}{Theorem}
% \newtheorem{proposition}{Proposition}
% \newtheorem{corollary}{Corollary}
% \newtheorem{assumption}{Assumption}
% \newtheorem{condition}{Condition}
% \usepackage{bbm, dsfont}

\def\BibTeX{{\rm B\kern-.05em{\sc i\kern-.025em b}\kern-.08em
    T\kern-.1667em\lower.7ex\hbox{E}\kern-.125emX}}
\begin{document}
\setulcolor{red}
\setul{red}{2pt}
\setstcolor{red}
%\newcommand\semiSmall{\fontsize{23.8}{20.38}\selectfont}
%\title{D2D-assisted Federated Learning: Hybrid Distributed Machine Learning in Two Timescales}
\title{Delay-Aware Hierarchical Federated Learning}
% : Intelligent Model Aggregation in Large-scale Wireless Edge Networks}

% Distributed model training under cooperative D2D communications.. 


% A consensus-driven distributed model training platform via cooperative device-to-device communications



% Some keywords: Device-to-device, peer-to-peer, Non-i.i.d data (to emphasize our new definition...), resource constrained, local descent method (we have multiple local descents),
% D2D, P2P, cluster-based, local cooperation.., locally cooperative devices ....



%  D2D-assisted hybrid federated learning with Aperiodic Consensus

\author{Frank Po-Chen Lin,~\IEEEmembership{Student Member,~IEEE},  Seyyedali~Hosseinalipour,~\IEEEmembership{Member,~IEEE}, Christopher G. Brinton,~\IEEEmembership{Senior~Member,~IEEE}, and Nicol\`o Michelusi, \IEEEmembership{Senior~Member,~IEEE}
\thanks{F. Lin and C. Brinton are with the School of Electrical and Computer Engineering, Purdue University, IN, USA. e-mail: \{lin1183,cgb\}@purdue.edu. Brinton and Lin acknowledge support from ONR grants N000142212305 and N000142112472.}
\thanks{S. Hosseinalipour is with the Department of Electrical Engineering, University at Buffalo, NY, USA. e-mail: alipour@buffalo.edu.}
\thanks{N. Michelusi is with the School of Electrical, Computer and Energy Engineering, Arizona State University, AZ, USA. e-mail: nicolo.michelusi@asu.edu. Part of his research has been funded by NSF under grant CNS-2129015.}
\thanks{A condensed version of this paper was presented at IEEE Globecom 2020~\cite{frank2020delay}.}}
\maketitle


\begin{abstract}
Federated learning has gained popularity as a means of training models distributed across the wireless edge.
The paper introduces delay-aware federated learning ({\tt DFL}) to improve the efficiency of distributed machine learning (ML) model training by addressing communication delays between edge and cloud. {\tt DFL} employs multiple stochastic gradient descent iterations on device datasets during each global aggregation interval and intermittently aggregates model parameters through edge servers in local subnetworks. The cloud server synchronizes the local models with the global deployed model computed via a local-global combiner at global synchronization. The convergence behavior of {\tt DFL} is theoretically investigated under a generalized data heterogeneity metric. A set of conditions is obtained to achieve the sub-linear convergence rate of $\mathcal O(1/k)$. Based on these findings, an adaptive control algorithm is developed for {\tt DFL}, implementing policies to mitigate energy consumption and edge-to-cloud communication latency while aiming for a sublinear convergence rate. Numerical evaluations show {\tt DFL}'s superior performance in terms of faster global model convergence, reduced resource consumption, and robustness against communication delays compared to existing FL algorithms. In summary, this proposed method offers improved efficiency and satisfactory results when dealing with both convex and non-convex loss functions.
%Federated learning relies on device-server communications to train a machine learning (ML) model, neglecting device-to-device (D2D) communications promoted in wireless networks. In this paper, we propose a novel ML model training architecture called \textit{two timescale hybrid federated learning} ({\tt TT-HF}), where we consider a cooperative cluster-based model training methodology and introduce a hybrid learning paradigm, where cooperative D2D communications are utilized in conjunction with device-server communications. In {\tt TT-HF}, during  each  global  aggregation interval, devices perform multiple local stochastic gradient descent (SGD) iterations and aperiodically synchronize their model parameters through local consensus via D2D  communications. This enables model training in two timescales capturing local SGD iterations and D2D communication rounds.
%We introduce a general definition on the gradient diversity, and investigate the convergence of {\tt TT-HF} that leads us to new convergence bounds for distributed ML. We exploit our theoretical findings to develop an online algorithm that actively tunes the step size, the D2D communications rounds, and the interval of global aggregations.
%Through extensive simulations, we demonstrate that the proposed framework is robust against extreme heterogeneity of the users' datasets and can achieve a higher accuracy as compared to the current art methods while imposing a smaller network cost.    
\end{abstract} 

\begin{IEEEkeywords}
\noindent Federated learning, edge intelligence, network optimization, convergence analysis, hierarchical architecture.
\end{IEEEkeywords}

\section{Introduction}

The increasing complexity of source code poses a key challenge to the reliability of large-scale software systems. Software bugs in these systems can lead to safety issues~\cite{bug_safety} for users around the world as well as cause non-negligible financial losses~\cite{bug_loss}. As such, developers have to spend a large amount of time and effort on bug fixing. Consequently, \aprfull (\apr), designed to automatically generate patches to fix software bugs, has attracted wide attention from both academia and industry~\cite{long2016prophet, legoues2012genprog, long2015spr, lou2020can, tufano2018empstudy}. 


To achieve \apr, one popular approach is known as Generate-and-Validate (G\&V)~\cite{qi2015gv, ghanbari2019prapr, lou2020can, le2016hdrepair, legoues2012genprog, wen2018capgen, hua2018sketchfix, martinez2016astor, koyuncu2020fixminder, liu2019tbar, liu2019avatar}, which is typically based on the following pipeline: First, fault localization techniques~\cite{wong2016fl, abreu2007ochiai, zhang2013injecting, papadakis2015metallaxis, li2019deepfl, li2017transforming} are applied to determine the suspicious locations in programs where bugs are likely to exist. Then, the buggy locations are used by the \apr tools to generate a list of patches that replace buggy lines with correct lines. Afterward, each patch is validated against the original test suite to identify any \emph{plausible patches} (i.e., passing all tests in the test suite). Finally, to determine the \emph{correct patches}, developers examine the list of plausible patches to see if any of them can correctly fix the bug. 

Traditional \apr tools can mainly be categorized into heuristic-based~\cite{legoues2012genprog, le2016hdrepair, wen2018capgen}, constraint-based~\cite{mechtaev2016angelix, le2017s3, demacro2014nopol, long2015spr} and \template~\cite{ghanbari2019prapr, hua2018sketchfix, martinez2016astor, liu2019tbar, liu2019avatar}. Among these traditional tools, \template \apr tools~\cite{ghanbari2019prapr, liu2019tbar, benton2020effectiveness} have been able to achieve state-of-the-art results. \Template \apr tools typically leverage pre-defined templates (e.g., adding a nullness check) for bug fixing. However, since these fix templates are typically handcrafted, the number and types of bugs they are able to fix can be limited. 



To address the limitations of traditional \apr, researchers have proposed various \learning \apr tools~\cite{li2020dlfix, chen2018sequencer, jiang2021cure, lutellier2020coconut, zhu2021recoder, ye2022rewardrepair} based on the \nmtfull (\nmt) architecture~\cite{sutskever2014mt} where the input is the buggy code snippets and the goal is to translate the buggy code snippets into a fixed version. To accomplish this, \learning \apr tools require supervised training datasets with pairs of both buggy and fixed code snippets in order to learn how to perform this translation step. These training data are usually obtained by mining historical bug fixes using heuristics/keywords~\cite{dallmeier2007benchmark}, which can be imprecise for identifying bug-fixing commits; even the actual bug-fixing commits can include irrelevant code changes, leading to further pollution in the dataset~\cite{xia2022alpharepair}.
% 
Moreover, it can be hard for such \apr tools to generalize and fix bug types unseen during training. 



To better leverage recent advances in \plmfull{s} (\plm{s}), researchers~\cite{xia2022alpharepair, xia2023repairstudy, kolak2022patch, prenner2021codexws} have directly applied \plm{s} to generate patches without bug-fixing datasets. These \llm-based \apr tools work by either directly generating a complete code function~\cite{prenner2021codexws, xia2023repairstudy} or predict/infill the correct code snippet given its surrounding context~\cite{xia2022alpharepair, xia2023repairstudy}. By directly using \llm{s} that are pre-trained on billions of open-source code snippets, \llm-based \apr tools can achieve state-of-the-art performance on many repair datasets~\cite{xia2022alpharepair}. 


% 
%
%

Traditional \apr tools have long used the insight of the \emph{plastic surgery hypothesis}~\cite{barr2014plastic} where it states that the code ingredients to fix a bug already exist within the same project. Traditional \apr tools have manually designed pattern-~\cite{ghanbari2019prapr, saha2017elixir} or heuristic-based~\cite{jiang2018simfix, legoues2012genprog} approaches to finding and using such relevant code ingredients to generate fixes for bugs. However, the plastic surgery hypothesis has been largely ignored in \llm-based \apr. In fact, \llm provides a unique opportunity to fully automate the plastic surgery hypothesis idea via fine-tuning (learning project-specific information via model updates from the buggy project) and prompting (directly providing relevant code ingredients to the model), and make it directly applicable to different languages (since the \llm{s} are typically multi-lingual).%
Moreover, despite the intensive manual efforts involved, traditional \apr tools still cannot fully leverage project-specific information due to large search space for leveraging/composing existing code ingredients. In contrast, the project-specific information can effectively leveraged by \llm{s} due to their power in code understanding/vectorization, e.g., even partial/imprecise information may still guide \llm{s} in correct patch generation!
 To this end, we ask the question: \emph{How useful is the plastic surgery hypothesis in the era of \plm{s}}?








\mypara{Our Work.} To answer the question, we present \ourtech{\xspace} -- a \llm-based approach that automatically utilizes the plastic surgery hypothesis by systematically combining multiple fine-tuning and prompting strategies for \apr. \ourtech fine-tunes \plm{s} using two novel domain-specific training strategies: \textbf{\epfinetune} -- we fine-tune using the original buggy project by aggressively masking out a high percentage of tokens, which allows \plm to learn project-specific code tokens and programming styles; and \textbf{\rofinetune} -- which only masks out a single continuous code sequence per training sample, allowing the model to get used to the final \csapr task of predicting a single continuous code sequence. Furthermore, we directly leverage the ability for \plm{s} to understand natural language instructions and introduce a novel prompting strategy, \textbf{\idprompting}, which uses information retrieval and static analysis to obtain a list of relevant identifiers for the buggy lines. While such relevant identifiers are critical for fixing some difficult bugs, they may not be seen by the \llm during inference due to limited context window size. Through the use of prompting, we directly tell the model to use these extracted identifiers (relevant code ingredients) to generate the correct code. Finally, to perform repair, we combine all four model variants (including the base model, both fine-tuned models and the base model with prompting) for the final repair.





While our insight of leveraging the plastic surgery hypothesis for \llm-based \apr is generalizable across different types of \plm{s}, to implement \ourtech, we choose a recent \plm{\xspace}, \ctfive~\cite{wang2021codet5}, which is pre-trained on millions of open-source code snippets. \ctfive is an encoder-decoder model trained using \mspfull (\msp) objective where a percentage of tokens are masked out and each continuous masked token sequence is referred to as a masked span. Also, although we only extract relevant identifiers from the current buggy project (since this paper focuses on the plastic surgery hypothesis), our work can be easily extended to obtain other code information (such as relevant statements or functions) from other sources, such as  the massive pre-training corpora~\cite{husain2020codesearchnet} or historical bug-fixing datasets~\cite{jiang2019infer}, which can provide more coding knowledge for \llm{s}. Besides, although we mainly focus on using traditional string comparison algorithms for information retrieval in this paper, these techniques can be easily replaced by other frequency-based retrieval~\cite{robertson2009probabilistic} and neural search (or embedding-based search)~\cite{reimers2019sentence}.
  In summary, this paper makes the following contributions:


%


\begin{itemize}[noitemsep, leftmargin=*, topsep=0pt]
    \item \textbf{Dimension.} This paper is the first to revisit the important plastic surgery hypothesis in the era of \llm{s}. It opens up a new dimension for \llm-based \apr to incorporate previously neglected information from the buggy project itself to boost \apr performance. Furthermore, it demonstrates the promising future of retrieval-based prompting for modern \llm-based \apr.
    \item \textbf{Implementation.} We implement \ourtech based on the recent \ctfive model. We augment the model using two novel fine-tuning strategies: \epfinetune and \rofinetune, along with a novel prompting strategy based on information retrieval and static analysis: \idprompting. We combine the patches generated by all four models together and perform patch ranking to speed up \apr.% 
    \item \textbf{Evaluation Study.} We conduct an extensive evaluation against state-of-the-art \apr tools. On the widely studied \dfj 1.2 and 2.0 datasets~\cite{just2014dfj}, \ourtech is able to achieve the new state-of-the-art results of 89 and 44 correct bug fixes (15 and 8 more than best baseline) respectively.  Furthermore, we perform a broad ablation study to justify our design. \ourtech demonstrates for the first time that the plastic surgery hypothesis can substantially boost \llm-based \apr and advance state-of-the-art \apr, while being fully automated and general. Moreover, even partial/imprecise code ingredients may still effectively guide \llm{s} for \apr!
\end{itemize}


\section{\system Framework Design}
\label{sec:system}
We now explain how \system helps author widgets that support transparent, reusable, and customizable user actions. 

% \danc{here do we want to emphasize a) we revise conventional widget design to a statefule design or b) megneton can convert your existing widgets to improve it with our proposed characteristics? It sounds more like b) to me but not sure if that's the intention. } \saj{it's actually a}
%In this section, we discuss the key components that provide the foundation for \system widgets, which instruments the design goals outlined in Section~\ref{sec:design_goal}. 
%We then explain the system architecture of \system.

\subsection{Widget Frameworks: Design and Limitations}
Widgets are interactive elements, \eg sliders, text boxes, buttons, that have representations both in the kernel, \ie where code is executed, and the front-end, \ie the notebook web interface. However, recent frameworks for authoring widgets~\cite{idomjp} also enable integration of interactive dashboards in the front-end~\cite{wu2020b2,bauerle2022symphony, zhang2023meganno}.  

 \begin{figure}[!htb] 
 \centering
  \includegraphics[width=0.8\linewidth]{figures/stateful-widget-redesign-basic.png}
  \caption{Design of basic, \ie traditional widgets.}
  \label{fig:base_widget} 
  \Description{The basic widget design.}
\end{figure}

 
 As shown in Figure~\ref{fig:base_widget}, Widgets (\eg \emph{ipywidgets}~\cite{IPyWidgets}) maintain their state both at the back-end kernel (called \emph{Widget Base}) and the front-end (called \emph{Widget Model}.) The Widget Base and Widget Model remain in-sync via the communication API called \emph{Comm}. 
 %\danc{The previous statement is bit hard to follow. Break down into defining what are widget base and model, and then explain how they work together?} 
However, only the most recent state is maintained, making the widgets essentially \emph{memoryless}. The \emph{Widget Manager} coordinates the display of the widget in the front-end \emph{Widget View}. The Widget View is a container for rendering interactive components using front-end libraries and web frameworks. The Widget View only registers low-level event listeners corresponding to user interactions on the components. %For example, a \emph{drag} interaction that updates position of slider is registered as an \emph{onChange} listener. 
 For example, a \emph{selection} interaction on the graph node in Figure~\ref{fig:teaser}B that updates the bar charts is registered as an \emph{onClick} listener.
 Therefore, these widgets are \emph{agnostic} of the user's high-level interaction types and additional context, such as where the interaction happened and which components were updated. The \emph{memoryless} and \emph{interaction agnostic} nature of widgets prevent tracking of the user's interaction history and the corresponding widget states.
Moreover, such a design primarily serves to parameterize data operations in the kernel using front-end events --- a widget state variable (\eg current node identifier) impacted by a low-level event (\eg \emph{onClick}) serves as an input parameter to a data operation (\eg distribution computation). Any change in the widget variable triggers a recomputation of the data operation. In the notebook, the users can programmatically access and update the parameters of the data operations. However, the data operations in the kernel, designed by widget developers, are neither accessible nor customizable from the front-end. The lack of affordances to override data operations limit 
the end-user's capability to customize the widgets designed by the developers. We describe enhancement of existing widgets with such features next.

\subsection{Towards Persistent, Interaction-Aware, and Customizable Widgets}

%We augment existing widgets to introduce new features such as interaction history, reusable sates, and on-demand customization of data operations. 
We create a persistent and interaction-aware widget called \emph{stateful widget} by extending the Widget Base with state and interaction history management capabilities (see Figure~\ref{fig:stateful_widget}.) Within a stateful widget, the state manager maintains each state updates corresponding to user interactions within a list called \emph{Data States}. The state manager registers the following in the \emph{action history}: (a) context of each event (\eg the front-end interaction type and the component and element where interaction occurred) and (b) the corresponding state identifier in Data States. Since the default Widget View only registers low-level events, we create a Widget View Wrapper that records each event's context as an action via an Action Wrapper. The action wrapper dispatches an action consisting of the event context mentioned earlier via the Comm API. Users can view the interaction history in a separate notebook cell which shows the details of an interaction and the corresponding data state as shown in Figure~\ref{fig:history}, thereby ensuring transparency. The history view is synchronized with the corresponding widget. Therefore, users can leverage the history to load previous states in the Widget View using the \emph{Restore} button. Moreover, users can also access the widget state as a \code{JSON} object using a declarative command as shown in Figure~\ref{fig:teaser}E, thereby ensuring reusability. Such a design also enables users to employ visualizations as a medium for capturing 
and exporting ``actions interactively
performed in the component''~\cite{batch2017interactive} --- 
the outcomes of these interactions are often utilized in 
subsequent steps of a data science workflow~\cite{rahman2022ie}.

 \begin{figure}[!htb] 
  \centering
  \includegraphics[width=\linewidth]{figures/stateful-widget-redesign-mag.png}
  \caption{Design of \system widgets. The dashed (``- -'') elements, \ie the stateful widget and widget view wrappers, are introduced by \system.}
  \label{fig:stateful_widget} 
  \Description{The stateful widget design.}
\end{figure}

% Therefore, interaction-aware state management \todo{via stateful widget} ensures transparency and reusability of user actions. \todo{elaborate}

\begin{figure*}[!htb] 
  \centering
  \includegraphics[width=\linewidth]{figures/history.png}
  \caption{The history view of a widget (\code{widget.history.show()}). Clicking the \emph{Restore} button loads previous state visualizations. }
  \label{fig:history} 
  \Description{The history view of a widget. Clicking the Restore button loads the previous states and their visualizations.}
\end{figure*}

%\hkc{why is it called 'shared'? also which opreations are customizable and which are not?} all data operations are customizable if they are defined as shared
Since data operations in the kernel correspond to user interactions in the front-end component, we introduce the concept of \emph{shared actions}.
Shared actions are data operations that end-users can override from the notebook. The operation definitions are essentially shared between the kernel and front-end. In the \system framework, developers can define a data operation to be shared. 
 For example, a shared data operation may return a distribution sorted by descending order of frequency. However, the user may prefer viewing the distribution in the alphabetic order of labels. As shown in Figure~\ref{fig:teaser}C and~\ref{fig:teaser}D, a user-defined function (UDF) written in the notebook --- which reflects the updated sort order --- is mapped to these the actions during widget instantiation time. 
In the kernel, the state manager parses the UDFs using custom serializers and overrides the data operation corresponding to the shared action. 
Such a design expands the ``events parameterizing code'' paradigm of widgets to ``operations parameterizing code'' and offers more flexible customization capabilities --- users can keep updating the shared actions to explore different objectives by modifying the function defined in the notebook.
Note that developers may implement data operations such as schema generation and distributions computation using standalone libraries or from scratch. In the case studies described in Section~\ref{sec:study}, we used an in-house graph query library, which was published as a Python package. 
% , among others. The operations are part of an in-house graph query library, which is published as a Python package for internal usage.
%We provide examples of these features in the supplementary material. 


%\todo{ADD CODE BLOCK}
\stitle{Components in \system Widget View.} 
%\danc{this paragraph renamed to technical/implementation details? Or is component a special term in megneton? } 
We use the React web framework~\cite{react} to develop the front-end components and the IDOM-Jupyter package~\cite{idomjp} for component rendering in the Widget View. 
The components are TypeScript~\cite{typescript} modules that enable the rendering of a wide range web-based visualization libraries. For example, we used a custom graph visualization library to render the schema graph~\cite{franz2016cytoscape}, Vega-lite~\cite{satyanarayan2016vega} to render the bar charts, and a JavaScript library to render tables. 
%We provide a detailed list of all the components in the supplementary material.
As TypeScript supports static typing, developers can define application-specific data types and use those across the modules. 
Therefore, using TypeScript ensures a tighter integration between the Widget Base in the kernel and the Widget Model in the front-end. 
Moreover, when customizing data operations defined as shared actions, the pre-defined types provide hints to the user about the expected return type of the customized function. Each of the components rendered in the Widget View is fully interactive. These interactions, derived from existing visualization research~\cite{yi2007toward, amar2005low} are reactively synchronized across \system components, enabling multiple-coordinated visualizations (\eg Figure~\ref{fig:teaser}B.) 
In Section~\ref{subsec:rgd_algo}, we describe the RGD algorithm for solving the alignment problem in Eq.~(\ref{eq:GPOP}). Then in \revdel{Section~\ref{subsec:loc_sub_conv} we prove the local sublinear convergence of RGD to a critical point of \revdel{$F$}\revadd{$\widetilde{F}$}. In} Section~\ref{subsec:loc_lin_conv}, using the theory of Morse\revdel{-Bott} functions \revadd{ and Proposition~\ref{prop:HessVicinity}}, we \revdel{extend the result to}\revadd{show} the local linear convergence of RGD to a non-degenerate alignment (see Section~\ref{subsec:non_deg_gen_setting}). \revadd{TODO: add one more line about contrasting conditions and exact recovery.}

\subsection{RGD Algorithm}
\label{subsec:rgd_algo}
A standard way to find a local minimum of Eq.~(\ref{eq:GPOP}) is to use RGD with a suitable initial point, step size and retraction strategy.
%In this work, our choice of retraction is based on \revdel{QR decomposition}\revadd{polar decomposition},
\revadd{In this work we use retraction based on the exponential map on $\mathbb{O}(d)$. Define,}
\revdel{
\begin{align}
    R_{\QR }: \cup_{\mathbf{S} \in \mathbb{O}(d)^m}(\{\mathbf{S}\} \times T_\mathbf{S}\mathbb{O}(d)^m) &\mapsto \mathbb{O}(d)^m\\
    R_{\QR }\left(\begin{bmatrix}\mathbf{S}_1\\\vdots\\\mathbf{S}_m\end{bmatrix}, \begin{bmatrix}\boldsymbol{\xi}_1\\\vdots\\\boldsymbol{\xi}_m\end{bmatrix}\right) &= \begin{bmatrix}\qf (\mathbf{S}_1+\boldsymbol{\xi}_1)\\\vdots\\\qf (\mathbf{S}_m+\boldsymbol{\xi}_m)\end{bmatrix}. \label{eq:R_QR}
\end{align}
}
\revadd{
\begin{align}
    R_{\EXP }: \cup_{\mathbf{S} \in \mathbb{O}(d)^m}(\{\mathbf{S}\} \times T_\mathbf{S}\mathbb{O}(d)^m) &\mapsto \mathbb{O}(d)^m\\
    R_{\EXP }\left([\mathbf{S}_i]_1^m, [\boldsymbol{\xi}_i]_1^m\right) &= [\mathbf{S}_i\exp(\mathbf{S}_i^T\boldsymbol{\xi}_i)]_1^m. \label{eq:R_PF}
\end{align}
}\revdel{where $\qf (\mathbf{A})$ denotes the $\mathbf{Q}$ factor in the thin QR decomposition of $\mathbf{A}$ \citea{van1996matrix, absil2009optimization}}\revadd{where $\exp (\mathbf{A})$ denotes the matrix exponential of $\mathbf{A}$ \citea{van1996matrix, absil2009optimization}. Then the following lemma provides a consistent definition of retraction on the quotient manifold $\mathbb{O}(d)^m/_{\sim}$.}
\revadd{
\begin{lem}
\label{lem:retraction}
Let $\widetilde{\mathbf{S}} \in \mathbb{O}(d)^{m}/_{\sim}$ and $\mathbf{S}^a, \mathbf{S}^b \in \pi^{-1}(\widetilde{\mathbf{S}})$. If $\mathbf{Z}^a \in T_{\mathbf{S}^a}\mathbb{O}(d)^m$ and $\mathbf{Z}^b \in T_{\mathbf{S}^b}\mathbb{O}(d)^m$ are the horizontal lifts of $\widetilde{\mathbf{Z}} \in T_{\widetilde{\mathbf{S}}}\mathbb{O}(d)^{m}/_{\sim}$ then $\pi(R_{\EXP }(\mathbf{S}^a, \mathbf{Z}^a)) = \pi(R_{\EXP }(\mathbf{S}^b, \mathbf{Z}^b))$. As a result, the retraction
\begin{align}
    \widetilde{R}_{\EXP }: \cup_{\widetilde{\mathbf{S}} \in \mathbb{O}(d)^m/_{\sim} }(\{\widetilde{\mathbf{S}}\} \times T_{\widetilde{\mathbf{S}}}\mathbb{O}(d)^m/_{\sim}) &\mapsto \mathbb{O}(d)^m/_{\sim}\\
    \widetilde{R}_{\EXP }\left([\widetilde{\mathbf{S}}_i]_1^m, [\widetilde{\mathbf{Z}}_i]_1^m\right) &=  \pi(R_{\EXP }(\mathbf{S}, \mathbf{Z}))\label{eq:Rtilde_PF}
\end{align}
is well defined for any $\mathbf{S} \in \pi^{-1}(\widetilde{\mathbf{S}})$ and $\mathbf{Z}$ being the horizontal lift of $\widetilde{\mathbf{Z}}$ at $\mathbf{S}$.
\end{lem}
}

\revadd{The step direction will always be the horizontal lift of $-\grad \widetilde{F}(\widetilde{\mathbf{S}})$ at some $\mathbf{S} \in \pi^{-1}(\widetilde{\mathbf{S}})$. Consequently, due to Proposition~\ref{prop:gradFS},} the step direction is $\boldsymbol{\xi} = -\grad F(\mathbf{S})$ which is the projection of the antigradient $-\nabla F(\mathbf{S})$ onto $T_\mathbf{S}\mathbb{O}(d)^m$. Recall (from the proof of Proposition~\ref{prop:gradFS}) that $\grad F(\mathbf{S}) = [[\mathbf{C}\mathbf{S}]_i - \mathbf{S}_i[\mathbf{C}\mathbf{S}]_i^T\mathbf{S}_i]_1^m$. Then the step size $\alpha$ is calculated using the Armijo-type rule with parameters $\beta,\gamma \in (0,1)$ (here $g$ is the canonical metric on $\mathbb{O}(d)^m$ as in Eq.~(\ref{eq:g_Z_W})),
\begin{equation}
    \alpha = \max_{l \geq 0}\{\beta^l\ \vertbar\ F(R_{\EXP }(\mathbf{S}, -\beta^l\grad F(\mathbf{S}))) - F(\mathbf{S}) \leq -\gamma \beta^l g(\nabla F(\mathbf{S}),  \grad F(\mathbf{S})) \}. \label{eq:armijo_step}
\end{equation}
\revadd{Since $F$ extends to a continuously differentiable non-negative function on $\mathbb{R}^{md \times d}$ containing $\mathbb{O}(d)^m$, it follows from \citeb[Proposition 2.8]{schneider2015convergence} that $\alpha$ is well-defined.}

\revdel{
\begin{algorithm}
\caption{Riemannian gradient descent for solving GPOP \label{algo:rgd_old}}
\revdel{
\begin{algorithmic}[1]
\REQUIRE $\mathbf{S}^0 \in \mathbb{O}(d)^m$, $\Gamma$, $\{\mathbf{x}_{k,i}: (k,i) \in E(\Gamma)\}$, $\beta, \gamma \in (0,1)$
\STATE Construct $\mathbf{C}$ as in Eq.~(\ref{eq:GPOP}).
\REPEAT
    \STATE calculate the descent direction $-\grad F(\mathbf{S}^k)$ at $\mathbf{S}^k$ using Eq.~(\ref{eq:gradFS}).
    \STATE calculate the step size $\alpha_k$ according to the Armijo-type rule (see Eq.~(\ref{eq:armijo_step})).
    \STATE set $\mathbf{S}^{k+1} = R_{\QR }(\mathbf{S}^k, -\alpha_k \grad F(\mathbf{S}^k))$ using Eq.~(\ref{eq:R_PF}).
    \STATE $k \leftarrow k + 1$.
\UNTIL{convergence.}
\end{algorithmic}
}
\end{algorithm}
}

\begin{algorithm}
\caption{Riemannian gradient descent for solving GPOP \label{algo:rgd}}
\revadd{
\begin{algorithmic}[1]
\REQUIRE $\widetilde{\mathbf{S}}^0 \in \mathbb{O}(d)^{m-1}$, $\Gamma$, $\{\mathbf{x}_{k,i}: (k,i) \in E(\Gamma)\}$, $\beta, \gamma \in (0,1)$
\STATE Construct $\mathbf{C}$ as in Eq.~(\ref{eq:GPOP}).
\REPEAT
    \STATE set $\mathbf{S}^k = [\mathbf{I}_d; \widetilde{\mathbf{S}}^k] \in \pi^{-1}(\widetilde{\mathbf{S}}^k) \subset \mathbb{O}(d)^m$ (Eq.~\ref{eq:pi_inv_wtS}).
    \STATE calculate the descent direction $-\grad F(\mathbf{S}^k)$ at $\mathbf{S}^k$ using Eq.~(\ref{eq:gradFS}).
    \STATE calculate the step size $\alpha_k$ according to the Armijo-type rule (see Eq.~(\ref{eq:armijo_step})).
    \STATE set $\widetilde{\mathbf{S}}^{k+1} = \widetilde{R}_{\EXP}(\mathbf{S}^k, -\alpha_k \grad F(\mathbf{S}^k))$ using Eq.~(\ref{eq:R_PF}, \ref{eq:pi}).
    \STATE $k \leftarrow k + 1$.
\UNTIL{convergence.}
\end{algorithmic}
}
\end{algorithm}

\revdel{
\subsection{Local Sublinear Convergence of RGD}
\label{subsec:loc_sub_conv}
}
\revadd{
\subsection{Local linear Convergence of RGD}
\label{subsec:loc}
}
We proceed to show the local sublinear convergence of Algorithm~\ref{algo:rgd} to a non-degenerate alignment (see Definition~\ref{def:non_deg_alignment0}). Our main tool will be the convergence analysis framework presented in \citeb[Section 2.3]{schneider2015convergence} as used in \citea{liu2019quadratic}.
\revdel{To this end, we first note that $F$ is a real-analytic function bounded from below by zero and $\mathbb{O}(d)^m$ is a compact submanifold of $\mathbb{R}^{md \times d}$. Thus, the Lojasiewicz gradient inequality \citea{lojasiewicz1965ensembles}, \citeb[Section 2.2]{schneider2015convergence} holds at every $\mathbf{S}^* \in \mathbb{O}(d)^m$ and in particular for every $\mathbf{S}^* \in \mathcal{C}$ (see Eq.~(\ref{eq:crit_pts2})) i.e. there exist $\delta, \eta > 0$ and $\theta \in (0,1/2]$ (generally dependent on $\mathbf{S}^*$) such that}
\revadd{To this end, we first note that $\widetilde{F}$ is a real-analytic function bounded from below by zero and $\mathbb{O}(d)^m/_{\sim}$ (whose elements are identified with $\mathbb{O}(d)^{m-1}$ here) is a compact submanifold of $\mathbb{R}^{(m-1)d \times d}$. Thus, the Lojasiewicz gradient inequality \citea{lojasiewicz1965ensembles}, \citeb[Section 2.2]{schneider2015convergence} holds at every $\widetilde{\mathbf{S}}^* \in \mathbb{O}(d)^m/_{\sim}$ and in particular for every $\widetilde{\mathbf{S}}^* \in \widetilde{\mathcal{C}}$ (see Eq.~(\ref{eq:crit_pts})) i.e. there exist $\delta, \eta > 0$ and $\theta \in (0,1/2]$ (generally dependent on $\widetilde{\mathbf{S}}^*$) such that}
\revdel{
\begin{equation}
    |F(\mathbf{S}) - F(\mathbf{S}^*)|^{1-\theta} \leq \eta \left\| \grad F(\mathbf{S})\right\|_F. \label{eq:Lojasiewicz_gradient_ineq_old}
\end{equation}
holds for every $\mathbf{S} \in \mathbb{O}(d)^m$ satisfying $\left\|\mathbf{S}-\mathbf{S}^*\right\|_F < \delta$.
}
\revadd{
\begin{equation}
    |\widetilde{F}(\widetilde{\mathbf{S}}) - \widetilde{F}(\widetilde{\mathbf{S}}^*)|^{1-\theta} \leq \eta \left\| \grad \widetilde{F}(\widetilde{\mathbf{S}})\right\|_F. \label{eq:Lojasiewicz_gradient_ineq}
\end{equation}
holds for every $\widetilde{\mathbf{S}} \in \mathbb{O}(d)^m/_{\sim}$ satisfying $\left\|\widetilde{\mathbf{S}}-\widetilde{\mathbf{S}}^*\right\|_F < \delta$.
}

\revdel{
Then using Theorem 2.3 in \citea{schneider2015convergence} and the fact that $\mathbb{O}(d)^m$ is compact (thus every sequence on it has a cluster point), for the Algorithm~\ref{algo:rgd} to converge at least sublinearly to a non-degenerate alignment $\mathbf{S}^*$, it suffices to show that the iterates $\{\mathbf{S}^k\}_{k \geq 0}$ generated by the algorithm satisfy the following:\\
}
\revdel{
\noindent \textbf{(A1)}. \textit{(Sufficient Descent)} There exist $\kappa_0 > 0$ and $k_1 \in \mathbb{N}$ such that, the inequality $ F(\mathbf{S}^{k+1}) - F(\mathbf{S}^k) \leq - \kappa_0 \left\|\grad F(\mathbf{S}^k)\right\|_F \cdot \left\|\mathbf{S}^{k+1}-\mathbf{S}^k\right\|_F$ holds for all $k \geq k_1$.
    % \begin{equation}
    %     F(\mathbf{S}^{k+1}) - F(\mathbf{S}^k) \leq - \kappa_0 \left\|\grad F(\mathbf{S}^k)\right\|_F \cdot \left\|\mathbf{S}^{k+1}-\mathbf{S}^k\right\|_F
    % \end{equation}
\smallskip\\
}
\revdel{
\noindent \textbf{(A2)}. \textit{(Stationarity)} There exist $k_2 \in \mathbb{N}$ such that for all $k \geq k_2$, if $\left\|\grad F(\mathbf{S}^k)\right\|_F = 0$ then $\mathbf{S}^{k+1} = \mathbf{S}^k$. The sequence $\{\mathbf{S}^{k}\}_{k \geq 0}$ satisfies this trivially.
\smallskip\\
}
\revdel{
\noindent \textbf{(A3)}. \textit{(Safeguard)} There exist a constant $\mu > 0$ and $k_3 \in \mathbb{N}$ such that the inequality $\left\|\grad F(\mathbf{S}^k)\right\|_F \leq \mu \left\|\mathbf{S}^{k+1}-\mathbf{S}^k\right\|_F$ holds for all $k \geq k_3$.
    % \begin{equation}
    %     \left\|\grad F(\mathbf{S}^k)\right\|_F \leq \mu \left\|\mathbf{S}^{k+1}-\mathbf{S}^k\right\|_F.
    % \end{equation}
\smallskip\\
}
\revdel{
To prove \textbf{(A1)} and \textbf{(A3)}, we need
\begin{prop}{\citeb[Appendix E.2]{liu2019quadratic}}
\label{prop:liu_qr}
There exist $\phi, M > 0$ such that for all $\mathbf{S}_i \in \mathbb{O}(d)$ and $\boldsymbol{\xi}_i \in T_{\mathbf{S}_i}\mathbb{O}(d)$ satisfying $\left\|\boldsymbol{\xi}_i\right\|_F \leq \phi$, $\left\|\qf (\mathbf{S}_i + \boldsymbol{\xi}_i) - (\mathbf{S}_i + \boldsymbol{\xi}_i)\right\|_F \leq M \left\|\boldsymbol{\xi}_i\right\|_F^2$. In particular, $M = \sqrt{10}/4$ and $\phi = 1/2$ satisfy the above inequality.
\end{prop}
\begin{prop}
\label{prop:second_order_boundedness_of_RQR}
There exist $\phi, M > 0$ such that for all $\mathbf{S} \in \mathbb{O}(d)^{m}$ and $\boldsymbol{\xi} \in T_{\mathbf{S}}\mathbb{O}(d)^m$ satisfying $\left\|\boldsymbol{\xi}\right\|_F \leq \phi$, $\left\|R_{\QR }(\mathbf{S},\boldsymbol{\xi}) - (S + \boldsymbol{\xi})\right\|_F \leq M \left\|\boldsymbol{\xi}\right\|_F^2$.
\end{prop}
}

\revadd{Then using Theorem 2.3 in \citea{schneider2015convergence} and the fact that $\mathbb{O}(d)^m/_{\sim}$ is compact (thus every sequence on it has a cluster point), for the Algorithm~\ref{algo:rgd} to converge at least sublinearly to a non-degenerate alignment $\widetilde{\mathbf{S}}^*$, it suffices to show that the iterates $\{\widetilde{\mathbf{S}}^k\}_{k \geq 0}$ generated by the algorithm satisfy the following:}
\smallskip

\noindent \textbf{(A1)}. \revadd{\textit{(Sufficient Descent)} There exist $\kappa_0 > 0$ and $k_1 \in \mathbb{N}$ such that, the inequality $\widetilde{F}(\widetilde{\mathbf{S}}^{k+1}) - \widetilde{F}(\widetilde{\mathbf{S}}^k) \leq - \kappa_0 \left\|\grad \widetilde{F}(\widetilde{\mathbf{S}}^k)\right\|_F \cdot \left\|\widetilde{\mathbf{S}}^{k+1}-\widetilde{\mathbf{S}}^k\right\|_F$ holds for all $k \geq k_1$.}
    % \begin{equation}
    %     F(\mathbf{S}^{k+1}) - F(\mathbf{S}^k) \leq - \kappa_0 \left\|\grad F(\mathbf{S}^k)\right\|_F \cdot \left\|\mathbf{S}^{k+1}-\mathbf{S}^k\right\|_F
    % \end{equation}
\smallskip

\noindent \textbf{(A2)}. \revadd{\textit{(Stationarity)} There exist $k_2 \in \mathbb{N}$ such that for all $k \geq k_2$, if $\left\|\grad \widetilde{F}(\widetilde{\mathbf{S}}^k)\right\|_F = 0$ then $\widetilde{\mathbf{S}}^{k+1} = \widetilde{\mathbf{S}}^k$. The sequence $\{\widetilde{\mathbf{S}}^{k}\}_{k \geq 0}$ satisfies this trivially.}

\smallskip

\noindent \textbf{(A3)}. \revadd{\textit{(Safeguard)} There exist a constant $\mu > 0$ and $k_3 \in \mathbb{N}$ such that the inequality $\left\|\grad \widetilde{F}(\widetilde{\mathbf{S}}^k)\right\|_F \leq \mu \left\|\widetilde{\mathbf{S}}^{k+1}-\widetilde{\mathbf{S}}^k\right\|_F$ holds for all $k \geq k_3$.}
    % \begin{equation}
    %     \left\|\grad F(\mathbf{S}^k)\right\|_F \leq \mu \left\|\mathbf{S}^{k+1}-\mathbf{S}^k\right\|_F.
    % \end{equation}

\revadd{To prove \textbf{(A1)} and \textbf{(A3)}, we need
\begin{prop}
\label{prop:liu_pf}
For all $\mathbf{S}_i \in \mathbb{O}(d)$ and $\boldsymbol{\xi}_i \in T_{\mathbf{S}_i}\mathbb{O}(d)$ satisfying $\left\|\boldsymbol{\xi}_i\right\|_F \leq 1$, $\left\|\mathbf{S}_i\exp (\mathbf{S}_i^T\boldsymbol{\xi}_i) - (\mathbf{S}_i + \boldsymbol{\xi}_i)\right\|_F \leq (e-1)\left\|\boldsymbol{\xi}_i\right\|_F^2$.
\end{prop}
\begin{prop}
\label{prop:second_order_boundedness_of_RPF}
For $\mathbf{S} \in \mathbb{O}(d)^m$ and $\boldsymbol{\xi} \in T_{\mathbf{S}}\mathbb{O}(d)^m$ satisfying $\left\|\boldsymbol{\xi}\right\|_F \leq 1$, $\left\|R_\EXP(\mathbf{S}, \boldsymbol{\xi})(\mathbf{S}_1\exp (\mathbf{S}_1^T\boldsymbol{\xi}_1))^T - (\mathbf{S} + \boldsymbol{\xi})\right\|_F \leq 2\sqrt{m}\left\|\boldsymbol{\xi}\right\|_F^2$.
%For all $\mathbf{S}_i \in \mathbb{O}(d)$ and $\boldsymbol{\xi}_i \in T_{\mathbf{S}_i}\mathbb{O}(d)$, $i \in [1,m]$, satisfying $\left\|\boldsymbol{\xi}_i\right\|_F \leq 1/2$, $\left\|\EXP (\mathbf{S}_i + \boldsymbol{\xi}_i)\EXP (\mathbf{S}_1 + \boldsymbol{\xi}_1)^T - (\mathbf{S}_i + \boldsymbol{\xi}_i)\right\|_F \leq \left\|\boldsymbol{\xi}_i\right\|_F^2 + \left\|\boldsymbol{\xi}_1\right\|_F^2$.
\end{prop}
}\begin{prop}
\label{prop:alpha_grad}
$(\alpha_k)_{k \geq 0}$ and $(\mathbf{S}^k)_{k \geq 0}$ satisfy $\lim \alpha_k \left\|\grad F(\mathbf{S}^k)\right\|_F = 0$.
\end{prop}
\revdel{
\subsection{Local Linear Convegence of RGD}
\label{subsec:loc_lin_conv}
}
Now we extend the above result to the R-linear convergence of the sequence $(\widetilde{\mathbf{S}}^k)$ generated through Algorithm~\ref{algo:rgd} to a non-degenerate alignment. \revadd{Since $F(\mathbf{S}\mathbf{Q}) = F(\mathbf{S})$ for all $\mathbf{Q} \in \mathbb{O}(d)$, therefore every critical point of $F$ is degenerate and in particular $F$ is not a Morse-function \citea{cohen_iga_norbury_2006}. However, if $\mathbf{S}^*$ is a non-degenerate alignment then $\widetilde{\mathbf{S}}^* = \pi(\mathbf{S}^*)$ is a non-degenerate critical point of $\widetilde{F}$, as a result $\widetilde{F}$ is a Morse function at $\widetilde{\mathbf{S}}^*$.
%due to \citeb[Remark 6.6]{usevich2020approximate}. As a result, due to \citeb[Theorem 6.7 and 6.8]{usevich2020approximate} and equivalently
Consequently, due to \citeb[Proposition 4.2]{hu2018convergence}, the Lojasiewicz gradient inequality is satisfied with $\theta = 1/2$. Precisely,} \revdel{In turn, it suffices to show that $F$ is a Morse-Bott function at $\mathbf{S}^*$ \citeb[Section 6.2]{usevich2020approximate} \citeb[Definition 1.5]{feehan2021optimal}.}
\begin{prop}
\revadd{Let $\mathbf{S}^*$ be a non-degenerate alignment and define $\widetilde{\mathbf{S}}^* = \pi(\mathbf{S}^*)$. Then there exist $\delta, \eta > 0$ such that
\begin{equation}
    |\widetilde{F}(\widetilde{\mathbf{S}}) - \widetilde{F}(\widetilde{\mathbf{S}}^*)|^{1/2} \leq \eta \left\| \grad \widetilde{F}(\widetilde{\mathbf{S}})\right\|_F. \label{eq:Lojasiewicz_gradient_ineq_half}
\end{equation}
holds for every $\widetilde{\mathbf{S}} \in \mathbb{O}(d)^m/_{\sim}$ satisfying $\left\|\widetilde{\mathbf{S}}-\widetilde{\mathbf{S}}^*\right\|_F < \delta$.}
\end{prop}
\revdel{
\begin{rmk}
Since $F(\mathbf{S}\mathbf{Q}) = F(\mathbf{S})$ for all $\mathbf{Q} \in \mathbb{O}(d)$, therefore no critical point of $F$ is non-degenerate and in particular $F$ is not a Morse-function \citea{cohen_iga_norbury_2006}.
\end{rmk}
}
% \revadd{Despite the above remark, from Proposition~\ref{prop:hlift_frob_ineq}, \ref{prop:d_g_tilde} and \ref{prop:gradFS}, we obtain $d_{\widetilde{g}}(\widetilde{\mathbf{S}},\widetilde{\mathbf{S}}^*) < d_g(\mathbf{S}, \mathbf{S}^*)$ and $\left\|\grad F(\mathbf{S})\right\|_F \leq \left\| \grad \widetilde{F}(\widetilde{\mathbf{S}})\right\|_F \leq \sqrt{(m+1)}\left\|\grad F(\mathbf{S})\right\|_F$. Combining these with the above proposition and the fact that $F(\mathbf{S}) = \widetilde{F}(\widetilde{\mathbf{S}})$ for all $\mathbf{S} \in \pi^{-1}(\widetilde{\mathbf{S}})$, we obtain the Eq.~(\ref{eq:Lojasiewicz_gradient_ineq}) with $\theta=1/2$.}
\revadd{Then, due to \citeb[Theorem 2.3]{schneider2015convergence} and the fact that non-degenerate critical points are isolated due to Morse Lemma\citeb[Proposition 4.2]{hu2018convergence}, we have the following result,}
% \begin{cor}
% \revadd{Let $\mathbf{S}^*$ be a non-degenerate alignment. Then there exist $\delta, \eta > 0$ such that
% \begin{equation}
%     |F(\mathbf{S}) - F(\mathbf{S}^*)|^{1/2} \leq \eta \left\| \grad F(\mathbf{S})\right\|_F. \label{eq:Lojasiewicz_gradient_ineq}
% \end{equation}
% holds for every $\widetilde{\mathbf{S}} \in \mathbb{O}(d)^m/_{\sim}$ satisfying $\left\|\widetilde{\mathbf{S}}-\widetilde{\mathbf{S}}^*\right\|_F < \delta$.}
% \end{cor}
%\begin{prop}
%\label{prop:morse_bott_1}
\revdel{Let $\mathbf{S}^*$ be a non-degenerate alignment. Then $\widetilde{F}$ is Morse-Bott at $\pi(\mathbf{S}^*)$ and consequently $F$ is Morse-Bott at $\mathbf{S}^*$. Combining this with \citeb[Theorem 6.3]{usevich2020approximate} (equivalently \citeb[Theorem 2.3]{schneider2015convergence}) we obtain}
%\end{prop}
% Combining previous propositions with \citea{usevich2020approximate}[Proposition $6.8$, Theorem $6.7$], \citea{hu2018convergence}[proposition $4.1$, Proposition $4.2$], we have the following result
% \begin{thm}
% If $S^* \in \mathcal{C}$ satisfies the rigidity constraints then there exist $\delta, \eta > 0$ such that for every $S \in \mathbb{O}(d)^m$ with $\left\|S-S^*\right\|_F < \delta$,
% \begin{align}
%     |F(S) - F(S^*)| \leq \eta \left\|\grad F(S)\right\|_F^2.
% \end{align}
% \end{thm}
% \begin{cor}
% We conclude that $\theta = 1/2$ in Eq.~(\ref{eq:Lojasiewicz_gradient_ineq}). Then invoking the convergence theorem in \citea{schneider2015convergence}[Theorem 2.3], we conclude that the sequence $\{\mathbf{S}^k\}_{k \geq 0}$ generated by Algorithm~\ref{algo:rgd} converges linearly to a critical point in $\mathcal{C}$.
% \end{cor}
\revdel{As a consequence of the above proposition, we have the following result}
\begin{thm}
\label{thm:rgd_conv}
Let $\mathbf{S}^*$ be a non-degenerate alignment and define $\widetilde{\mathbf{S}}^* = \pi(\mathbf{S}^*)$. Then there exist $\delta > 0$ such that RGD converges to $\widetilde{\mathbf{S}}^*$ \revadd{R-}linearly when initialized with $\widetilde{\mathbf{S}}^0$ such that $\left\|\widetilde{\mathbf{S}}^0-\widetilde{\mathbf{S}}^*\right\|_F < \delta$.
\end{thm}
\begin{rmk}
\revadd{Here the radius of convergence $\delta$ depends on the size of the neighborhood on which the Morse Lemma is applicable. In fact, due to Proposition~\ref{prop:HessFSZ}, Proposition~\ref{prop:HessVicinity}, Corollary~\ref{cor:HessVicinity} and $\min_{\mathbf{Q}\in\mathbb{O}(d)}\left\|\mathbf{O}-\mathbf{S}\mathbf{Q}\right\|_F \leq \left\|\pi(\mathbf{O})-\pi(\mathbf{S})\right\|_F$,
%the choice of $\mathbf{S}^{k} = [\mathbf{I}_d; \widetilde{\mathbf{S}}^{k}]$ (consequently, $\left\|\mathbf{S}^{k+1}-\mathbf{S}^{k}\right\|_F = \left\|\widetilde{\mathbf{S}}^{k+1}-\widetilde{\mathbf{S}}^{k}\right\|_F$),
the radius $\delta$ is given by
\begin{align}
    \delta = \begin{cases}(2(\delta_1 + 2 \delta_{3}))^{-1}|\lambda|, & \text{if } \mathbf{S}^* \text{ is a perfect alignment}\\
    (2(\delta_1 + \delta_2(\mathbf{S}^*) + 2 \delta_{3}(\mathbf{S}^*))))^{-1}|\lambda|, & \text{otherwise.}\end{cases}
\end{align}
}
\end{rmk}

Combining the above with Theorem~\ref{thm:non_deg_loc_min}, Corollary~\ref{cor:suff_cond_views_non_deg} and Theorem~\ref{thm:loc_rigid}, we obtain the following corollaries.
\begin{cor}
If $\mathbf{S}^*$ is an alignment such that $\mathbf{S}^* \in \mathcal{C}$, $\mathbb{L}(\mathbf{S}^*)$ is negative semi-definite and of rank $(m-1)d(d-1)/2$, then RGD converges locally linearly to $\widetilde{\mathbf{S}}^* = \pi(\mathbf{S}^*)$.
\end{cor}

\revadd{Exact recovery:}
\revadd{\citea{chaudhury2015global} showed that under the affine rigidity condition, the algorithms based on spectral and semidefinite relaxation exactly recovers the perfect alignment. While \citea{ling2021generalized} showed exact recovery due to generalized power method in a special setting which in fact exhibits affine rigidity where each point is represented in every local view i.e. the bipartite graph $\Gamma$ is complete. Here we advocate local linear convergence of RGD to a perfect alignment under affine rigidity as well as under weaker conditions of global and local rigidity. Precisely,}
\revadd{
\begin{cor}
Suppose $\mathbf{S}^*$ is a perfect alignment. Then RGD converges locally linearly to $\widetilde{\mathbf{S}}^* = \pi(\mathbf{S}^*)$ if any of the following conditions hold:
\begin{enumerate}
    \item $\Theta(\mathbf{S}^*)$ is affinely rigid, equivalently $\mathbf{C}$ is of rank $(m-1)d$.
    \item $\Theta(\mathbf{S}^*)$ is globally rigid, equivalently $\mathbf{S}^*$ is a unique perfect alignment.
    \item $\Theta(\mathbf{S}^*)$ is locally rigid, equivalently $\mathbb{L}(\mathbf{S}^*)$ is of rank $(m-1)d(d-1)/2$.
\end{enumerate}
Note that $1 \implies 2 \implies 3$.
\end{cor}
}
% \begin{cor}
% \label{cor:G_star_1_conv}
% If $\mathbf{S}^*$ is a perfect alignment such that $|\mathbb{G}^*(\mathbf{S})| = 1$ (see Theorem~\ref{thm:G_star_1}), then RGD converges locally linearly to $\mathbf{S}^*$.
% \end{cor}
\revadd{Finally, we state a sufficient condition on the structure of the noiseless local views that enables local linear convergence of RGD. This condition is stronger than the local rigidity condition and weaker than the global rigidity condition in the above corollary.}
\begin{cor}
\label{cor:G_conv}
If the local views are noiseless and $\mathbb{G}$ is connected then RGD converges locally linearly to a perfect alignment.
\end{cor}
\revdel{
\begin{cor}
If each local view is affinely non-degenerate (see Remark~\ref{rmk:non_deg_views}) and $\mathbf{S}^*$ is a perfect alignment such that the realization $\Theta(\mathbf{S}^*)$ is locally rigid then RGD converges locally linearly to $\mathbf{S}^*$.
\end{cor}
}

\section{Adaptive Control Algorithm for {\tt DFL}}\label{Sec:controlAlg}
\noindent In this section, we develop a control algorithm based on Theorem~\ref{thm:subLin_m} for tuning the controllable parameters in {\tt DFL}, while guaranteeing the sublinear convergence of the model. In {\tt DFL}, there are four sets of controllable parameters: (i) local model training intervals $\{\tau_k\}$, (ii) the combiner weight $\{\alpha_k\}$, (iii) the gradient descent step size $\{\eta_k\}$, and (iv) the instances of local aggregations $\{\mathcal T^\mathsf{L}_{k,c}\}$. The decisions on (i), (ii), (iii) and (iv) are made by the main server at $t=t_{k+1}-\Delta_k^{\mathsf{D}},~\forall k$ during global aggregation. 
% while the decision on (iv) is made at the edge server at the instance of local aggregations. These decisions are deployed at the system at $t=t_{k+1},~\forall k$.
% \nm{isnt the edge server responsible for the decisions on local agg??}
% whereas for (iv) the decision of the first instance (i.e., the time of first local aggregation) is determined by the main server at $t=t_k-\Delta_k^{\mathsf{D}},~\forall k$ while the rest (i.e., the time of the remaining local aggregations) are determined by the edge server at the instances when the local models are sent from the edge devices to the edge server to perform local aggregations (e.g., the time of the second local aggregation is obtained when the edge server conducts the first local aggregation).

To tune these parameters, we develop a control algorithm consisting of the following two parts. \textit{Part I:}  an adaptive technique (Sec.~\ref{subsec:learniParam}) to determine the step-size (i.e., $\eta_{max}$ and $\gamma$ in $\eta_k$ defined in Theorem~\ref{thm:subLin_m}) considering the conditions imposed by Theorem~\ref{thm:subLin_m}. \textit{Part II:} an optimization scheme (Sec.~\ref{subsec:learnduration}) to tune $\tau_k$ and $\alpha_k$ accounting for the tradeoff between the ML model performance and network resource consumption. 
% \textit{Part III:} an estimation procedure to estimate dataset and model-related parameters ($\beta, \mu, \zeta, \delta, \zeta_c, \delta_c, \sigma$, $\omega$, and $\omega_c$) used in Parts I and II (Sec.~\ref{subsub:estparam}). 
In Sec.~\ref{ssec:control}, we provide the pseudocode summarizing how Parts I and II are integrated.
% In this section, we develop a control algorithm (Sec.~\ref{ssec:control}) based on Theorem~\ref{thm:subLin} for tuning (i), (ii) at the main server at the beginning of each global aggregation, and (iii) at each device cluster in a decentralized manner. To do so, we propose an approach for determining the learning-related parameters (Sec.~\ref{subsec:learniParam}), a resource-performance tradeoff optimization for $\tau_k$ and $\Gamma_c$ (Sec.~\ref{subsec:learnduration}), and estimation procedures for dataset-related parameters (Sec.~\ref{subsec:cont}).  
  

\subsection{Step Size Parameters ($\eta_{max}$, $\gamma$)}\label{subsec:learniParam}
We first tune the step size parameters ($\eta_{max}$, $\gamma$). This is done for a given set of model-related parameters ($\beta, \mu, \zeta, \delta, \zeta_c, \delta_c, \sigma$, $\omega$ and $\omega_c$), which can be estimated by the server (e.g., see Sec. IV-C of~\cite{lin2021timescale}). 
% We assume that the latency-sensitivity of the learning application specifies a tolerable amount of time that {\tt DFL} can wait between consecutive global aggregations, i.e., the value of $\tau$. 
Given the fact that larger feasible values of $\eta_{max}$ result in larger values of step size and thus faster convergence, given the conditions mentioned in the statement of Theorem~\ref{thm:subLin_m}, we first determine the largest value for $\eta_{max}$ such that $\eta_{max}<\min\left\{\frac{2}{\beta+\mu},\frac{(\tau-\Delta)\mu}{\beta^2[(1+\lambda_+)^{\tau}-1-\tau\lambda_+]}\right\}$, where $\lambda_+$ is defined in Theorem~\ref{thm:subLin_m}. Afterward, 
% given the fact that smaller values of $\Lambda$ result in a larger value of step size and thus faster convergence, 
we arbitrarily choose the value of $\gamma$ such that $\gamma<\min\left\{1-(1-\mu\eta_{\mathrm{max}})^{2(\tau-\Delta)},C_3\eta_{\mathrm{max}}\beta\right\}$.
% \begin{align} \label{eq:gammaCond}
%        \hspace{-3mm} 
%             \gamma<\min\left\{1-(1-\mu\eta_{\mathrm{max}})^{2(\tau-\Delta)},C_3\eta_{\mathrm{max}}\beta\right\}. 
%         \hspace{-4mm}
%     \end{align} 

% The step size-related parameters will be re-computed at the server at each global aggregation using the most recent estimates of dataset and model-related parameters and used in the subsequent local model training interval.

We next introduce the optimization formulation to determine the length of local model training interval $\tau_k$ and the combiner weight $\alpha_k$ for each local model training interval. 

%With the learning-related parameters in hand, we then determine the period of local model training in the following.

% $
% \frac{\phi^2}{\beta}
% $
% $ Z_2\triangleq
%     \frac{1}{2}[\frac{\sigma^2}{\beta}+\frac{2\phi^2}{\beta}]
%     +50\beta\gamma(\tau-1)\left(1+\frac{\tau-2}{\alpha+1}\right)
%     \left(1+\frac{\tau-1}{\alpha-1}\right)^{6\beta\gamma}\left[\frac{\sigma^2}{\beta}+\frac{\phi^2}{\beta}+\frac{\delta^2}{\beta}\right]$
% and allows the value of $\tau_k$ 
%     \begin{align} \label{60}
%         1 \leq \tau_k \leq \sqrt{\Big[\mu\gamma-[3+8(1/\vartheta+(2\omega-\vartheta/2))^2]\Big]\Big[\tilde{\Gamma}-\vartheta\gamma\phi^2/2\Big]/(\gamma^2\beta^2 B)-A/B}
%     \end{align}
% to resides within a desirable range. 
% Note that satisfying \eqref{47} is equivalent to satisfy the second condition for $\nu$ in Theorem \ref{thm:subLin}.

% want to find some $\tau$ and $\phi$ such that the objective in (48) can be minimized while satisfying the second term in the lower bound of $\Gamma$ as follows:
% \begin{align}
%      &\Gamma\geq \frac{\gamma^2\beta}{2}[Q_k A+B]/[\mu\gamma-(1+16\omega^2\beta^2\gamma^2 Q_k/\kappa)]+\beta\phi\gamma/2
% \end{align}
% can be satisfied.
% which forces $\phi,\tau$ to have an upper bound. Now, intuitively, larger gamma always helps with reducing the consensus rounds, so we choose to select the largest $\Gamma$. Note that $\Gamma$ is a function of $\tau$ and $\phi$. In general, larger $\phi$ makes $\Gamma$ larger and so does larger $\tau$.
% $A=(16\omega^2/\kappa+153/16)\beta\phi+81/16\sigma^2+\delta^2$, $B=\sigma^2+\beta\phi$

% \begin{align}
%     \frac{\gamma^2\beta}{2}[97/4\sigma^2+\delta^2]/
%     \{[\mu\gamma-(1+16\omega^2\beta^2\gamma^2 Q_k/\kappa)]F(\hat{ \mathbf{w}}(0)\}\geq\alpha\geq\max\{\beta\gamma/\kappa, \beta\gamma\big[1-\kappa/4+\sqrt{(1+\kappa/4)^2+2\omega}\big]\}
% \end{align}
% \addFL{
% Or equivalently, for the given value of $\Gamma$, we want to find the maximum value of $\tau$ based on a given value of $\phi$.
%     $$
%     [(\mu\gamma-1)(\Gamma-\beta\phi\gamma/2)-B]/[\frac{\gamma^2\beta}{2}A+\frac{16\omega^2\beta^2\gamma^2(\Gamma-\beta\phi\gamma/2)}{\kappa}]>4,
%     $$
%     we obtain the range of $\tau$ as  
%     \begin{align}
%         \tau \leq \log_4\left\{[(\mu\gamma-1)(\Gamma-\beta\phi\gamma/2)-B]/[\frac{\gamma^2\beta}{2}A+\frac{16\omega^2\beta^2\gamma^2(\Gamma-\beta\phi\gamma/2)}{\kappa}]\right\}.
%     \end{align}
% }

% we keep changing $\phi$ in the interval obtained from \eqref{47} and \eqref{60} (for example, we can use line search with a certain quantization step) and 
% we compute the optimization of \eqref{eq:obj} with convex optimization techniques to obtain the corresponding value of $\tau_k$ for each round of global aggregation. 
% This will give us multiple pairs of $\tau_1$ and $\phi$. We choose the pair of $\tau_1,\phi$ that gives us the best result with respect to the objective and fix this value of $\phi$ for the rest of the training.

% Note that the dynamics of $\Upsilon_c^{(t)}$ with respect to different values of $\tau_k$ given some fixed value of $\phi$ will be obtained in the estimation phase.
 
\subsection{Length of Local Training Interval $\{\tau_k\}$ and Value of Combiner Weight $\{\alpha_k\}$}\label{subsec:learnduration}

Considering the convergence goal of {\tt DFL} (i.e., sublinear convergence with low resource consumption across edge devices), we formulate an optimization problem $\bm{\mathcal{P}}$ solved by the main server at each instance of global aggregation at $t=t_{k}-\Delta_{k-1}^{\mathsf{D}},~\forall k$ to tune $\tau_k$ and $\alpha_k$ for the subsequent local model training intervals $\mathcal{T}_k,~\forall k$. The objective function of $\bm{\mathcal{P}}$ accounts for the joint impact of three metrics: $(a)$ energy consumption of local and global model aggregation, $(b)$ communication delay of local and global aggregation, and $(c)$ the performance of global deployed model captured by the optimality gap in Theorem~\ref{thm:subLin_m}. 
% In this paper, we assume that the latency-sensitivity of the learning application in {\tt DFL} specifies a tolerable length on the local model training interval $\tau_k$ denoted as $\tau_{\textrm{max}}$, providing the formulation as follows:
% To capture this, we formulate an optimization problem $(\bm{\mathcal{P}})$ solved by the main server at the beginning of each global aggregation period $\mathcal{T}_k$, i.e., when $t=t_{k-1}$:

% \footnote{$\sum\limits_{j\in\mathcal S_c}\rho_{j,c}\mathbb E_{t_{k-1}}[\Vert \mathbf e_j^{(t)} \Vert] 
%       \leq \sum\limits_{j\in\mathcal S_c}\rho_{j,c}\mathbb E_{t_{k-1}}\Vert\mathbf e_{j}^{(t_{k-1})}\Vert\prod\limits_{u=t_{k-1}}^{t-1}(1+2\eta_{u}\beta)
%     +\sum\limits_{u=t_{k-1}}^{t-1}\eta_{u}\Big(\frac{2\omega_c\beta}{\mu}\mathbb E_{t_{k-1}}\Vert\nabla F(\bar{\mathbf w}_c^{(u)})\Vert
%     +(2\sigma+\delta_c)\Big)\prod\limits_{m=n+1}^{t-1}(1+2\eta_{m}\beta)$}
% \nm{I dont understand, what is $\Gamma_{\{c,k\}}$ for? Is it capturing the time slots that trigger local aggregation? If that is the case, shuoldnt Ak be defined as 
% $\mathcal{A}_{k} = \{t\in\mathcal T_k:  X_t> \eta_t \phi \}$
% ins tead of 
% $\mathcal{A}_{k} = \{t\in\mathcal T_k:  X_t\leq  \eta_t \phi \}$??
% }
\begin{align*} 
    &(\bm{\mathcal{P}}): ~~\min_{\tau,\alpha}  c_1\underbrace{(\frac{T-t_{k}}{\tau})\Big(E_{\textrm{GlobAgg}}+\sum\limits_{c=1}^N \vert\mathcal{A}_{\{c,k\}}\vert E_{c,\textrm{LocalAgg}}\Big)}_{(a)}+
    \nonumber \\&
    c_2\underbrace{(\frac{T-t_{k}}{\tau})\Big(\Delta_{\textrm{GlobAgg}}+\sum\limits_{c=1}^N\vert\mathcal{A}_{\{c,k\}}\vert\Delta_{c,\textrm{LocAgg}}\Big)}_{(b)}
    + c_3\underbrace{\nu(\tau,\alpha)}_{(c)}
\end{align*}  
%  network prop delay, trans delay, queiuing and processing delay, describe (44) when range of tau is empty 
\vspace{-0.3in}  
    \begin{align} 
    & \textrm{s.t.}\nonumber \\ 
    % & \;\;\; 1 \leq \Delta_k, \Delta_k\in\mathbb{Z}^+, \\
   & \;\;\; \Delta \leq \tau \leq \min{\{\tau_\textrm{max}, T-t_{k}\}}, \tau_k\in\mathbb{Z}^+, \label{eq:tauMax}\\   
   & \;\;\; \alpha<\frac{1}{\frac{C_2\eta_{\max}^2}{\eta_{\max}\beta C_3-\gamma}
         2\omega C_2(1+\gamma)
        +(1+\gamma)(1+\lambda_+)^{\tau}}, \label{eq:bound_alpha}
 %   & \;\;\; \nu(\tau_k,\alpha_k) \geq \nu^{\mathsf{min}},  
 % \label{eq:bound_min}\\
 %   & \;\;\; \nu^{\mathsf{min}} = \max\hspace{.51mm} \left\{\frac{Y_3\beta^2\gamma^2(\Lambda-\Delta)}{(\Lambda-\Delta)(Y_1\gamma-(\Delta+1))-Y_2{\gamma}^2\beta^2},\frac{\Lambda\Vert\nabla F(\widehat{\mathbf w}^{(0)})\Vert^2}{2\mu}\right\}, \label{eq:eps_min}
%   & \;\;\; 1 \leq \Delta \leq \tau, \Delta\in\mathbb{Z}^+, \label{eq:delta}\\ 
%   & \;\;\; \Delta_{\textrm{GlobAgg}} = \max_{n_i\in\mathcal{N}} M\times Q/R_{\{c,n_i\}}, \label{eq:globDelay}\\
%   & \;\;\; \Delta_{c,\textrm{LocalAgg}} = \max_{j\in\mathcal{S}_c} M\times Q/R_{\{n_c,j\}}, \label{eq:locDelay}\\
%   & \;\;\; E_{\textrm{GlobAgg}} = \sum\limits_{i=1}^Np_{\{c,n_i\}} M\times Q/R_{\{c,n_i\}}, \label{eq:globEng}\\
%   & \;\;\; E_{c,\textrm{LocalAgg}} = \sum\limits_{j=1}^{s_c}\ p_{\{n_c,j\}} M\times Q/R_{\{n_c,j\}}, \label{eq:locEng}\\
%   & \;\;\; \mathcal{A}_{\{c,k\}} = \{t\in\mathcal T_k:  (\epsilon_c^{(t)})^2> \eta_t \phi \},  \label{eq:consensusNum} \\
\end{align}
% \nm{doesnt make sense that X depebnds ob the expected terms. You need to use upper bounds for them}
where $E_{\textrm{c,LocalAgg}}= \sum_{j\in \mathcal{S}_c} M\times Q\times p_{j}/R_j^{(t)}$ is the energy consumption of conducting local model aggregation at edge server $n_c$, where $M$ denotes the size of the model (i.e., number of model parameters), $Q$ denotes the number of bits used to represent each model parameter, which is dependent on the quantization level, $p_{j}$ denotes the transmit power of device $j\in\mathcal S_c$,  and $R_j^{(t)}=W\log_2 \left(1+ \frac{p_j \vert {h}^{(t)}_{j}\vert^2}{N_0W} \right)$ is the transmission rate between device $j\in\mathcal S_c$ and its associated edge server $n_c$ at time $t$. The noise power is $N_0 W$, with $N_0$ as the white noise power spectral density, $W$ as the bandwidth, and ${h}^{(t)}_{j}$ as the channel coefficient.
% \footnote{Although the exact data rates between the devices and the server may not be known in advance due to unknown channel conditions, we can use expected data rates given the channel condition statistics (see Sec.~\ref{sec:experiments}).} 
$E_{\textrm{GlobAgg}}=\sum_{n_c \in\mathcal{N}} M\times Q \times \bar{p}_{n_c}/\bar{R}_{n_c}^{(t)}$ is the energy consumption for edge-to-main server communications, where $\bar{p}_{n_c}$ and $\bar{R}_{n_c}^{(t)}$ denote the transmit power of edge server $n_c$ and the transmission rate between the edge server $n_c\in\mathcal N,~\forall c$ and the main server, respectively. Furthermore, $\Delta_{\textrm{c,LocalAgg}}=\max_{j\in\mathcal{S}_c} \{ M\times Q/R_{j}^{(t)}\}$ is the communication delay of performing local aggregation via device $j\in\mathcal S_c$.\footnote{The device-to-edge server communications are assumed to occur in parallel, using multiple access techniques such as FDMA.}
% The edge server waits for the reception of ML models from all its associated devices before performing a local aggregation.}  
$\Delta_{\textrm{GlobAgg}}=\max_{i\in\mathcal I}\{\Delta_k / R(\xi_i^{(t)})\}$ is the device-to-main server communication delay, where $\Delta_k$ denotes the round-trip delay measured in terms of the number of conducted SGDs and $R(\xi_i^{(t)})$ denotes the processing rate (the number of SGDs conducted at each time instance measured in seconds) at edge device $i$. 

In $\bm{\mathcal{P}}$, $\vert\mathcal{A}_{\{c,k\}}\vert$ is the number of local aggregations performed by devices $i\in\mathcal S_c$ within a period of local model training interval, which we obtain as $\mathcal{A}_{\{c,k\}} = \{t\in\mathcal T_k:  \sum\limits_{c=1}^N\varrho_c(1-\Theta_c^{(t)})(2\delta_c^2+4\omega_c^2\beta^2\Vert\bar{\mathbf v}_c^{(t)}-\mathbf w^*\Vert^2)>\phi^2 \}$ (see Assumption~\ref{assump:sub_err}). To obtain $\mathcal{A}_{\{c,k\}}$ we thus need to control $\Vert\bar{\mathbf v}_c^{(t)}-\mathbf w^*\Vert^2$, to monitor the value of which, we first approximate it as $\Vert\bar{\mathbf v}_c^{(t)}-\mathbf w^*\Vert^2\approx\Vert\bar{\mathbf w}_c^{(t)}-\mathbf w^*\Vert^2$. Then, during each local aggregation, the edge server estimates the upper bound on $\Vert\bar{\mathbf w}_c^{(t)}-\mathbf w^*\Vert^2$ using strong convexity of $F(\cdot)$ (i.e., $\Vert\nabla F(\bar{\mathbf w}_c^{(t)})\Vert^2 \geq \mu^2\Vert\bar{\mathbf w}_c^{(t)}-\mathbf w^*\Vert^2$). 
% The decision variables for the $k$-th local model training interval in $(\bm{\mathcal{P}})$ are obtained using the estimated values of $\Vert\nabla F(\bar{\mathbf w}_c^{(t)})\Vert^2,~\forall t\in\mathcal{T}^\mathsf{L}_{k-1,c}$ from the previous interval.
Finally, $\nu(\tau,\alpha)=2Y_1^2 \eta_K+2Y_3^2\eta_K^2$ denotes the optimality gap upper bound derived in Theorem~\ref{thm:subLin_m} at time $t_K=T$. To compute $Y_3$, we first approximate $e_3^{(0)}\approx\mathbb \Vert\bar{\mathbf w}^{(0)}-\mathbf w^*\Vert$, and then estimate its using its upper bound $\left\Vert\nabla F(\bar{\mathbf w}^{(0)})\right\Vert \geq \mu\Vert\bar{\mathbf w}^{(0)}-\mathbf w^*\Vert$.
% Also, constraint~\eqref{eq:bound_min} guarantee a feasible value for the optimality gap.
% is the sum of the upper bounds on $e_1^{(T)}$, $e_2^{(T)}$ and $e_3^{(T)}$ after expanding the recurrence relationship from the nth term to the $K$-th term. Note that the bound in Theorem~\ref{thm:subLin_m} is not applied in the optimization since 

\textbf{Constraints.}
% The objective function of $\bm{\mathcal{P}}$ captures a tradeoff between $(a)$ energy consumption, $(b)$ communication delay, and $(c)$ the expected ML model performance, with $c_1$, $c_2$, $c_3\geq 0$ weighting coefficients. 
% In $(a)$, we account for the total amount of transmission power to perform local and global aggregation;
% in $(b)$, we consider the transmission delay for conducting local aggregation $\Delta_{c,\textrm{LocalAgg}}$, whereas we consider the propagation delay in the edge-to-cloud link for performing global aggregation $\Delta_{\textrm{GlobAgg}}$.
% Note that the reason to neglect the transmission delay for performing global aggregations is due to the fact that, in the system model, we assume device-to-edge links have much shorter range than the edge-to-cloud links such that the propagation delay becomes a dominant portion of the communication delay. Therefore, we neglect the transmission delay in device-to-edge links for global aggregations in the objective function.
% Term $(c)$ is the upper bound of the expected optimality gap derived in Theorem~\ref{thm:subLin_m}. 
% A smaller value of $(c)$ corresponds to a smaller optimality gap, implying a better ML performance. 
% Note that the optimization formulation $(\bm{\mathcal{P}})$ is solved at $t=t_k-\Delta_{k-1}^{\mathsf{D}}$ at the main server, which
% obtains the values of $\tau$ and $\alpha$ assuming that these values will be used for the rest of the training period (i.e., $\tau_{k'}=\tau$ and $\alpha_{k'}=\alpha,~\forall k'\geq k$ given the current edge-to-cloud communication delay $\Delta = \Delta_k$).
% accounts for the assumptions imposed in the convergence analysis by having a fixed value of $\tau=\tau_k$ and $\alpha=\alpha_k,~\forall k$ via optimizing the performance metrics over the remaining model training time based on a given edge-to-cloud communication delay $\Delta = \Delta_k,~\forall k$. Upon doing so, the formulation inherently assumes that the optimization variables $\tau_k$ and $\alpha_k$ along with the delay $\Delta_k$ are fixed over the remaining period from $t_{k-1}$ to $T$.
The constraint in~\eqref{eq:tauMax} guarantees that the value of $\tau_k$ is larger than the edge-to-main server communication delay, matching our assumption in Sec.~\ref{subsec:syst3}.
Constraint~\eqref{eq:bound_alpha} is a condition on $\alpha$ described in Theorem~\ref{thm:subLin_m} and ensures that the value of $\alpha$ lies within a range to guarantee the sublinear convergence of the global deployed model.
% via using the Polyak-Lojasiewicz (PL) inequality $\left\Vert\nabla F(\widehat{\mathbf w}^{(t)})\right\Vert^2 \geq 2\mu [F(\widehat{\mathbf w}^{(t)})-F(\mathbf w^*)]$.
% Also, constraint~\eqref{eq:bound_min} guarantee a feasible value for the optimality gap.

\textbf{Solution.} Formulation $\bm{\mathcal{P}}$ is a non-convex  mixed-integer programming problem. Due to the complex nature of the problem, we solve it via exhaustive search, performing line search over the integer values of $\tau$ in the range given in~\eqref{eq:tauMax} and obtain the optimum of $\alpha\in(0,1]$ corresponding to each value of $\tau$. Note that, given a value of $\tau$,  $\bm{\mathcal{P}}$  is still non-convex with respect to  $\alpha$. Therefore, we discretize the search space of $\alpha$ and perform a line search over the discretized search space of $\alpha$ for each $\tau$. Based on this approach, the search space of $\bm{\mathcal{P}}$ remains to be small due to the limited ranges/choices of $\tau$ and $\alpha$ (i.e., the time complexity of performing line search over $\tau$ is $\mathcal O(T-t_{k}-\Delta_{k-1})$ and the time complexity of performing line search over $\alpha$ is $\mathcal O( 1/S_{\alpha})$, where $S_{\alpha}$ is the discretization step used to discretize  $(0,1]$ interval, resulting in a time complexity of $\mathcal O((T-t_{k}-\Delta_{k-1})\times1/S_{\alpha})$). Thus, we are able to  solve the problem via a reasonable precision (e.g., $S_{\alpha}=0.01$) in a short duration of time (e.g., less than $3$ seconds on a laptop with Intel(R) Xeon(R) Gold 6242 CPU @ 2.80GHz).
\subsection{{\tt DFL} Control Algorithm}
\label{ssec:control}
The procedure of the {\tt DFL} control algorithm is outlined in Algorithm~\ref{GT}, which integrates the procedures described in Sec.~\ref{subsec:learniParam} and~\ref{subsec:learnduration}. 
% \clearpage

\begin{algorithm}[H]
{\scriptsize 
\SetAlgoLined
\caption{{\tt DFL} with adaptive control parameters.} \label{GT}
% \KwResult{Write here the result }
\KwIn{Desirable subnet deviation error coefficient $\phi$, length of model training $T$} 
\KwOut{Global model $\bar{\mathbf w}^{(T)}$}
% // Start of initialization by the main server\\
 Initialize $\bar{\mathbf w}^{(0)}$ and broadcast it among the edge devices through the edge server.\\
 Initialize estimates of $\zeta \ll 2\beta,\delta,\sigma$.\\
 Initialize $\eta_{max}<\min\left\{\frac{2}{\beta+\mu},\frac{(\tau-\Delta)\mu}{\beta^2[(1+\lambda_+)^{\tau}-1-\tau\lambda_+]}\right\}$ and $\gamma$ for the step size $\eta_k=\frac{\eta_{max}}{1+\gamma k}$ according to Sec.~\ref{subsec:learniParam}.\\ 
%  and $\Lambda, \gamma,\xi, T$ satisfy~\eqref{46}.\\
%  Obtain $\phi^{\mathsf{max}}$ from~\eqref{47}.\\
 % Initialize $\tau_0$ randomly, where $\tau_0\leq T-t_{k-1}.~\forall k$.\\
%  \textbf{Initialization by the server:} 
Initialize $t=0, ~k=0,~ t_0=0, ~t_1=\tau_0$, with $\tau_0$ chosen randomly, such that $\tau_0\leq T-t_{k},~\forall k$.\\ 
% // End of initialization by the main server\\
 \While{$t\leq T$}{
     \While{$t\leq t_{k+1}$}{
        \For( // Operation at the subnets){$c=1:N$}
        {
         Each device $i\in\mathcal{S}_c$ performs a local SGD update based on~\eqref{eq:SGD} and~\eqref{8} using $\mathbf w_i^{(t-1)}$ to obtain~$\widetilde{\mathbf{w}}_i^{(t)}$.\\
        %  with:\\
        %   $\widetilde{\mathbf w}_i^{(t)} = 
        %   \mathbf w_i^{(t-1)}-\eta_{t-1} \widehat{\mathbf g}_{i}^{(t-1)}$\; 
        % Devices estimate the value of $\Upsilon^{(t)}_c$ using~\eqref{eq:Ups_est} with distributed message passing.\\
        \uIf{$t\in\mathcal T^\mathsf{L}_{k,c}$}{
        Devices inside the subnet conduct local model aggregation using~\eqref{eq:local_agg} to obtain the updated local model $\mathbf w_i^{(t)}$.
        }
        \uElseIf{$t=t_{k+1}$}{
        Devices inside each subnet perform global synchronization using~\eqref{eq:aggr_alpha}.
        }
        \Else{
        Each device $i\in\mathcal{S}_c$ obtains its updated local model as $\mathbf{w}_i^{(t)} = \widetilde{\mathbf{w}}_i^{(t)}$
        }
      }
      \uIf( // Operation at the edge server){$t=t_{k+1}-\Delta_k$}{
      % // Operation at the edge server\\
    %   Each edge server $n_c$ estimates the local SGD noise 
    %   as described in Sec.~\ref{subsub:estparam}.\\
      Each edge server $n_c$ sends $\mathbf w_{i}^{(t_{k+1}-\Delta_k)}$, $\widehat{\mathbf g}_{i}^{(t_{k+1}-\Delta_k)},~\forall i\in\mathcal S_c$ to the main server.\\
    %   and the estimated local SGD noise to the main server. \\
    %  Solve Problem $(\bm{\mathcal{P}})$ to obtain $\tau_{k+1}$, and set $t_{k+1}=t+\tau_{k+1}$\\
    %  Compute $\Xi^{(k)}$ using~\eqref{eq:Xi_est} and set $\mu=\min\{ \}_{k'=1}^{k}$, and $\beta=\max\{ \}_{k'=1}^{k}$
       }
    \ElseIf( // Operation at the main server){$t=t_{k+1}-\Delta_k^{\mathsf{D}}$}{
    % // Operation at the main server\\
    %   Estimate $\delta$\;
      Compute $\bar{\mathbf w}^{(t_{k+1}-\Delta_k)}$ according to~\eqref{15}, $\hat{\beta}_k$, $\hat{\mu}_k$, $\hat{\sigma}_k$.\\ 
      % using \eqref{15},~\eqref{eq:est_beta},~\eqref{eq:est_mu} and~\eqref{eq:est_sigma}.\\
      % Compute $\hat{\beta}_k$ using~\eqref{eq:est_beta}.\\ 
      % Compute $\hat{\mu}_k$ using~\eqref{eq:est_mu}.\\
      % Compute $\hat{\Psi}_k$ using~\eqref{eq:est_psi}.\\
      Set $\hat{\zeta}_k\ll 2\hat{\beta}_k$ and $\hat{\zeta}_{c,k}\ll 2\hat{\beta}_k$, then compute $\hat{\delta}_k$ and $\hat{\delta}_{c,k}$ using the method in~\cite{lin2021timescale}. \\ 
      % via~\eqref{eq:grad_div_est} and~\eqref{eq:est_delta_c}. \\
      % Set $\hat{\zeta}_{c,k}\ll 2\hat{\beta}_k$, and compute $\hat{\delta}_{c,k}$ via~\eqref{eq:est_delta_c}. \\
      % Compute $\hat{\sigma}_k$ using~\eqref{eq:est_sigma} 
      Characterize $\eta_{max}$ and $\gamma$ for the step size $\eta_k=\frac{\eta_{max}}{1+\gamma k}$ according to Sec.~\ref{subsec:learniParam}.\\
    %  Compute the value of $\alpha$ by (to be determined)\\
     Compute the instances of local aggregation for each cluster $c$ using Sec.~\ref{subsec:learnduration}.\\
     Solve the optimization~$\bm{\mathcal{P}}$ to obtain $\tau_{k+1}$ and $\alpha_{k+1}$.\\
     Broadcast (i) $\bar{\mathbf w}^{(t_{k+1}-\Delta_k)}$, (ii) $\alpha_{k+1}$ and (iii) $\eta_k$ among the devices.
    }
     $t\gets t+1$
 }
 $k\gets k+1$ and $t_{k+1}\gets t_k+\tau_k$\\
}
}
\end{algorithm}

\iffalse
\textbf{Obtaining Tractable Solutions.}
Let the subnet deviation error coefficient $\phi$ be a desirable value chosen based on the learning application and consider the expression for $\nu$ in Theorem~\ref{thm:subLin}, the instances of local aggregations are obtained based on the condition $(\epsilon^{(t)})^2=\eta_t\phi$ imposed in Theorem~\ref{thm:subLin} along with applying the bound on $(\epsilon^{(t)})^2$ derived in Proposition~\ref{prop:clust_div_mn}. Note that the expression for $(\epsilon^{(t)})^2$ depends upon $\mathbb E\Vert\bar{\mathbf w}_c^{(t-t')}-\mathbf w^*\Vert$, where $t-t'$ is the instance of conducting local aggregation. Although the value of the optimum $\mathbf w^*$ is unknown, each edge server can still apply Proposition~\ref{prop:clust_div_mn} to determine the time instances and total number of local aggregation in a local model training interval via replacing the estimation of $\mathbb E\Vert\bar{\mathbf w}_c^{(t-t')}-\mathbf w^*\Vert$ in the expression of $(\epsilon^{(t)})^2$ with $\mathbb E\big\Vert\nabla F(\bar{\mathbf w}_c^{(t-t')})\big\Vert/\mu$, which is an upper bound of $\mathbb E\Vert\bar{\mathbf w}_c^{(t-t')}-\mathbf w^*\Vert$ that can be obtained via applying the property of strong convexity on $F(\cdot)$, i.e., $\big\Vert\nabla F(\bar{\mathbf w}_c^{(t)})\big\Vert \geq \mu \Vert\bar{\mathbf w}_c^{(t)}-\mathbf w^*\Vert$. Note that the decision for determining the first instance of local aggregation is conducted 
during global aggregation, whereas the decision of the following instance of local aggregations is performed at its precedent local aggregations. Therefore, we will use the value of $\mathbb E\Vert\bar{\mathbf w}_c^{(t_k-\Delta_k)}-\mathbf w^*\Vert$ as the estimate for $\mathbb E\big\Vert\nabla F(\bar{\mathbf w}_c^{(t-t')})\big\Vert/\mu$ to determine the first instance of local aggregation in $\mathcal T_k$ and $\mathbb E\big\Vert\nabla F(\bar{\mathbf w}_c^{(t-t')})\big\Vert/\mu$ as the estimate of $\mathbb E\Vert\bar{\mathbf w}_c^{(t-t')}-\mathbf w^*\Vert$ to determine the rest local aggregation instances.
\fi

 



%%%%%%%%%%%%%%%%%%%%%%%%%%%%%%%%%%%%%%%%%%%%%%%%%%%%%%%%%%%%%%%%%%%%%%%%%%%%%%%%%%%%%%%

%%%%%%%%%%%%%%%%%%%%%%%%%%%%%%%%%%%%%%%%%%%%%%%%%%%%%%%%%%%%%%%%%%%%%%%%%%%%%%%%%%%%%%%

% \addFL{Finally, we are going to describe Part C}
% In this section, we describe the design of the two-phase adaptive control algorithm {\tt DFL} based on the results obtained from Theorem \ref{thm:subLin} in Section III. In phase I, we design an parameter estimation scheme to estimate the value for defining the characteristics of the loss function (i.e. $\beta,\mu$), gradient diversity (i.e. $\delta,\zeta$) and upper bound on the variance of the estimated gradient used in SGD (i.e. $\sigma$). 
% the convergence parameters (i.e. $\Gamma,\gamma$) and the initial value on the length of the 1-st \emph{local model training interval} (i.e. $\tau_1$).
% In phase II, we develop an adaptive control scheme for {\tt DFL} that adaptively tunes the step size, length of the local model training interval and the number of D2D rounds through time using the parameters estimated in the first phase.

% According to \eqref{cond1}, \eqref{cond2} and \eqref{cond3}, the tuning of learning parameters requires the knowledge on the upper bound of (A.) parameter divergence $\Upsilon_c(k)$ to satisfy the condition provided by Proposition \ref{genLin} and \ref{strLin}. Therefore, we first derive an adaptive estimation for the divergence of parameters in a distributed manner for every local iteration $t$. 


% \begin{algorithm}
% \small
% \SetAlgoLined
% \caption{Two Timescale Hybrid Federated Learning ({\tt \DFL})-Phase I} \label{GT}
% % \KwResult{Write here the result }
% \KwIn{$K$} 
% \KwOut{Global model $\hat{\mathbf w}(t_K)$}
%  \textbf{Initialization operated by the server:} (i) Initialize the local model as $\mathbf w_i(0)=\hat{\mathbf w}(0),\  \forall i$, (ii) Set the step size as \add{$\eta_t=\frac{\gamma}{t+\alpha}$}, where $\alpha=\max\{\beta\gamma/\kappa, \beta\gamma\big[1-\kappa/4+\sqrt{(1+\kappa/4)^2+2\omega}\big]\}$\;
%  \For{$k=1:K$}{
%      \For{$t=t_{k-1}+1:t_k$}{
%       \uIf{$t=t_k$}{
%       Estimate $\hat{\beta}\leftarrow\sum_{c=1}^N \varrho_c^{(k)}\hat{\beta}_i$\;
%       Estimate $\hat{\mu}\leftarrow\sum_{c=1}^N \varrho_c^{(k)}\hat{\mu}_i$\;
%       Estimate $\delta,\zeta\leftarrow\Vert\nabla F(\hat{\mathbf w}(t_k))-\nabla F_i(\hat{\mathbf w}(t_k))\Vert$\;
%     %   Estimate parameter for the variance of SGD $\sigma$\;
%       Compute $\hat{\mathbf w}(t)$ with \eqref{15} at the server\;
%       Synchronize local models with the global model $\mathbf w_i(t)=\hat{\mathbf w}(t),~\forall i$ //Global Synchronization\;
%         % \uIf{$\tau_{k+1}>\log_4\left\{(\mu\gamma-1)\kappa/(16\omega^2\beta^2\gamma^2)\right\}$} 
%         % {
%         % $\tau_{k+1} = \tau_{k+1}/2$ // The initial choice of $\tau_{k+1}$ could not be satisfied\;
%         % }
%         % \Else{
%         % Tune the number of D2D consensus such that \\
%         % $\tau_k \leq \log_4\left\{[(\mu\gamma-1)\nu-B]/[\gamma^2\beta A/2 +16\omega^2\beta^2\gamma^2(\Gamma-\phi^2\gamma/2)/\kappa]\right\}$ is satisfied\;
%         % }
%           }
%     \Else{
%       For each edge device $i$ in parallel, perform local update with\\
%       $\mathbf w_i(t+1) =  
%           \mathbf w_j(t)-\eta_t\nabla F_j(\mathbf w_j(t))$\;
%           \uIf{$\hat{\mu}_i>(\nabla F_i(\mathbf w _i(t))-\nabla F_i(\hat{\mathbf w}(t_{k-1})))^\top(\mathbf w _i(t)-\hat{\mathbf w}(t_{k-1}))/\Vert\mathbf w _i(t)-\hat{\mathbf w}(t_{k-1})\Vert^2$}
%           {Estimate $\hat{\mu}_i\leftarrow(\nabla F_i(\mathbf w _i(t))-\nabla F_i(\hat{\mathbf w}(t_{k-1})))^\top(\mathbf w _i(t)-\hat{\mathbf w}(t_{k-1}))/\Vert\mathbf w _i(t)-\hat{\mathbf w}(t_{k-1})\Vert^2$\;
%           }
%           \uIf{$\hat{\beta}_i<\Vert\nabla F_i(\mathbf w _i(t))-\nabla F_i(\hat{\mathbf w}(t_{k-1}))\Vert/\Vert\mathbf w _i(t)-\hat{\mathbf w}(t_{k-1})\Vert$}
%           {
%           Estimate $\hat{\beta}_i\leftarrow\Vert\nabla F_i(\mathbf w _i(t))-\nabla F_i(\hat{\mathbf w}(t_{k-1}))\Vert/\Vert\mathbf w _i(t)-\hat{\mathbf w}(t_{k-1})\Vert$\;
%           }
%           \uIf{$\hat{\tilde{\beta}}_i<\Vert\nabla \hat{f}(\mathbf x_{i,b};\mathbf w_i(t))-\nabla \hat{f}(\mathbf x_{i,b'};\mathbf w_i(t))\Vert/\Vert\mathbf x_{i,b}-\mathbf x_{i,b'}\Vert$}
%           {
%           Estimate $\hat{\tilde{\beta}}_i\leftarrow\Vert\nabla F_i(\mathbf w _i(t))-\nabla F_i(\hat{\mathbf w}(t_{k-1}))\Vert/\Vert\mathbf w _i(t)-\hat{\mathbf w}(t_{k-1})\Vert$\;
%           }
%       }
%      }
%  }
%  Estimate $\hat{\sigma}_i^2 
% \leftarrow
% \Big(1-\frac{\vert\xi_i^{(t)}\vert}{D_i}\Big)\frac{\tilde{\beta}^2}{D_i}
% \frac{\sum_{b=1}^{D_i}\sum_{b'=1}^{D_i}\Vert b-b'\Vert^2}
% {\vert\xi_i^{(t)}\vert(D_i-1)}$\;
% \end{algorithm}

\iffalse
\subsection{Data and Model-Related Parameters ($\beta, \mu, \delta, \zeta, \delta_c, \zeta_c, \sigma^2$)} \label{subsub:estparam}
% In Phase I, we perform several rounds of the standard federated averaging {\tt FedAvg} \cite{McMahan} and estimate the parameters throughout the training process. {\tt FedAvg} generally consist of three steps repeated in sequence: (i) several iterations of parallel, local model training at each device using their own local datasets, (ii) aggregation of the local models at an edge server into a single, global model, and (iii) synchronization of the local models at each device with this global model. During the training process, we continuing to estimate $\beta,\mu,\delta,\zeta$ and $\sigma$ with the local and global models. After several rounds of training, we obtain the estimated value of all the parameters, and use them for the scheme in Phase II. The estimation scheme for each of the parameters is demonstrated as follows. In the following, we use the same notations for the system model of {\tt \DFL} as described in Section II:

We next aim to estimate the values of parameters used to describe the loss function (i.e. $\beta,\mu$), gradient diversity (i.e. $\delta,\zeta$, $\delta_c$, $\zeta_c$), and the variance of the stochastic gradients used during SGD iterations (i.e. $\sigma^2$). These parameters are estimated  at the main server based on the ML parameters of the devices transmitted at each $t=t_{k+1}-\Delta_k,~\forall k$.
% at each $t=t_k-\Delta_k^{\mathsf{D}},~\forall k$ of each global aggregation with parameters estimations obtained from

%We first overview the estimation method of the ML parameters and the divergence of local models inside the clusters, using which we develop our control algorithm for {\tt \DFL} that dynamically tunes the number of D2D communication rounds and obtains the length of interval of local model training.
\subsubsection{Estimation of $\mu$}
The main server estimates the value of $\mu$ according  to the strong convex property of $F(\cdot)$ given in~\eqref{eq:11_mu}, which can be equivalently expressed as 
$
\exists \mu>0: \Big\Vert \nabla F(\mathbf w_1)-\nabla F(\mathbf w_2)\Big\Vert \geq \mu\Big\Vert \mathbf w_1-\mathbf w_2 \Big\Vert,~\forall \mathbf w_1, \mathbf w_2.
$
To this end, 
the main server uses the received local models and gradients  to acquire the estimate of $\mu$ denoted by $\hat{\mu}_k$ as follows: 
{\small\begin{align} \label{eq:est_mu}
    \hat{\mu}_k = \min_{i,j\in\mathcal I}\left\{\big\Vert\sum_{c=1}^N\varrho_c\sum_{i\in\mathcal S_c}\rho_{i,c}\widehat{\mathbf g}_{i}^{(t_{k}-\Delta_k)}-\sum_{c=1}^N\varrho_c\sum_{j\in\mathcal S_c}\rho_{j,c}\widehat{\mathbf g}_{j}^{(t_{k}-\Delta_k)}\big\Vert/\Vert\mathbf w_i^{(t_{k}-\Delta_k)}-\mathbf w_j^{(t_{k}-\Delta_k)}\Vert\right\}.
\end{align}}


% as the candidates for $\hat{\mu_i}$ at every edge device $i$. The smallest value of candidates estimated at every edge device $i$ throughout the entire training process will be chosen as $\hat{\mu_i}$. Then, we compute $\hat{\mu}\leftarrow\sum_{c=1}^N \varrho_c^{(k)}\hat{\mu}_i$ as the estimate for $\mu$. 

\subsubsection{Estimation of $\beta$}
The main server estimates the value of $\beta$ using the $\beta$-smoothness property of $F_i(\cdot)$ expressed in Assumption \ref{beta} via~\eqref{eq:11_beta}. To this end, the main server uses the received local models and gradients to estimate $\beta$ as follows: 
 \begin{align} \label{eq:est_beta}
     \hat{\beta}_k = \max_{i,j\in\mathcal I}\left\{ \Vert\widehat{\mathbf g}_{i}^{(t_{k}-\Delta_k)}-\widehat{\mathbf g}_{j}^{(t_{k}-\Delta_k)}\big\Vert/\Vert\mathbf w_i^{(t_{k}-\Delta_k)}-\mathbf w_j^{(t_{k}-\Delta_k)}\Vert\right\}.
 \end{align}
 
%  \subsubsection{Estimation of $\Psi$} To estimate the value of $\Psi$, denoted by $\hat{\Psi}_k$, we first upper bound the optimality gap of the ML model with 
% $\Vert\bar{\mathbf w}^{(t)}-\mathbf w^*\Vert \leq \Vert\nabla F(\bar{\mathbf w}^{(t)})\Vert$ and apply it into the result in Proposition~\ref{prop:clust_div_mn} to get
% \begin{align}
%      &\mathbb E\left[\sum\limits_{c=1}^N\varrho_{c}\Vert\bar{\mathbf w}_c^{(t)}-\bar{\mathbf w}^{(t)}\Vert^2\right] + \mathbb E[\Vert\bar{\mathbf w}^{(t)}-\mathbf w^*\Vert^2] 
%      \nonumber \\&
%      \leq  
%      \mathbb E\left[\sum\limits_{c=1}^N\varrho_{c}\Vert\bar{\mathbf w}_c^{(t)}-\bar{\mathbf w}^{(t)}\Vert^2\right] + \mathbb E[\Vert\nabla F(\bar{\mathbf w}^{(t)})\Vert^2] \approx \Psi.  
% \end{align}
%     By using $\sum_{n_c\in\mathcal N}\varrho_c\Vert\bar{\mathbf w}_c^{(t_k-\Delta_k)}-\bar{\mathbf w}^{(t_k-\Delta_k)}\Vert^2$ and $\Big\Vert\sum_{n_c\in\mathcal{N}}\varrho_c\sum_{i\in\mathcal S_c}\rho_{i,c}\widehat{\mathbf g}_{i}^{(t_{k}-\Delta_k)}\Big\Vert$ as the estimate for $\mathbb E\Big[\sum_{n_c\in\mathcal N}\varrho_c\Vert\bar{\mathbf w}_c^{(t)}-\bar{\mathbf w}^{(t)}\Vert^2\Big]$ and $\mathbb E[\Vert\nabla F(\bar{\mathbf w}^{(t)})\Vert^2]$, respectively, the main server estimates the value of $\Psi$ as follows:
% \begin{align} \label{eq:est_psi}
%     \hat{\Psi}_k = \max_{k'\in\{1,\cdots,k\}}\Big\{\sum_{n_c\in\mathcal N}\varrho_c\Vert\bar{\mathbf w}_c^{(t_k-\Delta_k)}-\bar{\mathbf w}^{(t_k-\Delta_k)}\Vert^2+\Big\Vert\sum_{n_c\in\mathcal{N}}\varrho_c\sum_{i\in\mathcal S_c}\rho_{i,c}\widehat{\mathbf g}_{i}^{(t_{k}-\Delta_k)}\Big\Vert^2\Big\}.
% \end{align}

%  $\Vert\nabla F_i(\mathbf w _i(t))-\nabla F_i(\hat{\mathbf w}(t_{k-1}))\Vert/\Vert\mathbf w _i(t)-\hat{\mathbf w}(t_{k-1})\Vert,~t\in\mathcal{T}_k$ as the candidates for $\hat{\mu_i}$ at every edge device $i$. The largest value of candidates estimated at every edge device $i$ throughout the entire training process will be chosen as $\hat{\beta}_i$. Then, we compute $\hat{\beta}\leftarrow\sum_{c=1}^N \varrho_c\hat{\beta}_i$ as the estimate for $\beta$.
\subsubsection{Estimation of $\delta,\zeta$}
To estimate the values of $\delta,\zeta$, we first upper bound the gradient diversity in Definition \ref{gradDiv} as 
\begin{align} \label{eq:grad_div_est}
    \Vert\nabla\bar F_c(\mathbf w)-\nabla F(\mathbf w)\Vert
    \leq \delta+ \zeta \Vert\mathbf w-\mathbf w^*\Vert 
    \leq \delta+ \zeta/\mu \Vert\nabla F(\mathbf w)\Vert,~\forall c, \mathbf w
\end{align}
via applying the strongly convex property of $F(\cdot)$, i.e., $\big\Vert\nabla F(\mathbf w)\big\Vert \geq \mu \Vert\mathbf w-\mathbf w^*\Vert$. Based on~\eqref{eq:grad_div_est}, to obtain the estimates of $\delta$ and $\zeta$ denoted by $\hat{\delta}_k$ and $\hat{\zeta}_k$, we first select an arbitrary value of $\hat{\zeta}_k$ such that $\hat{\zeta}_k\ll 2\hat{\beta}_k$ and obtain $\hat{\delta}_k$ via computing an estimate for $\Vert\nabla\bar F_c(\mathbf w)-\nabla F(\mathbf w)\Vert$ and $\Vert\nabla F(\mathbf w)\Vert$, which can be obtained using the received local gradients (i.e., $\Big\Vert\sum_{n_c\in\mathcal{N}}\varrho_c\sum_{i\in\mathcal S_c}\rho_{i,c}\widehat{\mathbf g}_{i}^{(t_{k}-\Delta_k)} - \sum_{i\in\mathcal S_c}\rho_{i,c}\widehat{\mathbf g}_{i}^{(t_{k}-\Delta_k)}\Big\Vert$ and $\Big\Vert\sum_{n_c\in\mathcal{N}}\varrho_c\sum_{i\in\mathcal S_c}\rho_{i,c}\widehat{\mathbf g}_{i}^{(t_{k}-\Delta_k)}\Big\Vert$) as follows: 
\begin{align}
    \hat{\delta}_k = 
    \left[\max_{n_c\in\mathcal{N}}\left\{\Big\Vert\sum_{n_c\in\mathcal{N}}\varrho_c\sum_{i\in\mathcal S_c}\rho_{i,c}\widehat{\mathbf g}_{i}^{(t_{k}-\Delta_k)} - \sum_{i\in\mathcal S_c}\rho_{i,c}\widehat{\mathbf g}_{i}^{(t_{k}-\Delta_k)}\Big\Vert - \hat{\zeta}_k/\hat{\mu}_k\Big\Vert\sum_{n_c\in\mathcal{N}}\varrho_c\sum_{i\in\mathcal S_c}\rho_{i,c}\widehat{\mathbf g}_{i}^{(t_{k}-\Delta_k)}\Big\Vert\right\}\right]^+,
\end{align}
where notation $[A]^+\triangleq \max\{A,0\}$, $A\in \mathbb{R}$.
% where we use $\Big\Vert\sum_{c=1}^N\varrho_c\sum_{i\in\mathcal S_c}\rho_{i,c}\widehat{\mathbf g}_{i}^{(t_{k}-\Delta_k)} - \sum_{i\in\mathcal S_c}\rho_{i,c}\widehat{\mathbf g}_{i}^{(t_{k}-\Delta_k)}\Big\Vert$ and $\Big\Vert\sum_{c=1}^N\varrho_c\sum_{i\in\mathcal S_c}\rho_{i,c}\widehat{\mathbf g}_{i}^{(t_{k}-\Delta_k)}\Big\Vert$ as the estimates for $\Vert\nabla\bar F_c(\mathbf w)-\nabla F(\mathbf w)\Vert$ and $\Vert\nabla F(\mathbf w)\Vert$ respectively.

\subsubsection{Estimation of $\delta_c,\zeta_c$}
Similarly, to estimate the values of $\delta_c,\zeta_c$ denoted by $\hat{\delta}_{c,k}$ and $\hat{\zeta}_{c,k}$, we first upper bound the gradient diversity in Definition \ref{gradDiv_c} as 
\begin{align} 
    \left\Vert\nabla F_i(\mathbf w)-\nabla\bar F_c(\mathbf w)\right\Vert
    \leq \delta_c+\zeta_c\Vert\mathbf w-\mathbf w^*\Vert
    \leq \delta_c+\zeta_c/\mu \Vert\nabla F(\mathbf w)\Vert,~\forall i\in\mathcal{S}_c,  ~\forall c, \mathbf w.
\end{align}
% via applying the strongly convex property of $F(\cdot)$, i.e., $\big\Vert\nabla F(\mathbf w)\big\Vert \geq \mu \Vert\mathbf w-\mathbf w^*\Vert$. 
By choosing the value of $\hat{\zeta}_{c,k}$ such that $\hat{\zeta}_{c,k}\ll2\beta$ along with using $\Big\Vert\widehat{\mathbf g}_{i}^{(t_{k}-\Delta_k)} - \sum_{i\in\mathcal S_c}\rho_{i,c}\widehat{\mathbf g}_{i}^{(t_{k}-\Delta_k)}\Big\Vert$ and $\Big\Vert\sum_{n_c\in\mathcal{N}}\varrho_c\sum_{i\in\mathcal S_c}\rho_{i,c}\widehat{\mathbf g}_{i}^{(t_{k}-\Delta_k)}\Big\Vert$ as the estimates for $\Vert\nabla F_i(\mathbf w)-\nabla\bar F_c(\mathbf w)\Vert$ and $\Vert\nabla F(\mathbf w)\Vert$ respectively, we can estimate the value of $\delta$ as follows:
\begin{equation} \label{eq:est_delta_c}
    \hat{\delta}_{c,k} = 
    \left[\max_{i\in\mathcal{S}_c}\left\{\Big\Vert\widehat{\mathbf g}_{i}^{(t_{k}-\Delta_k)} - \sum_{i\in\mathcal S_c}\rho_{i,c}\widehat{\mathbf g}_{i}^{(t_{k}-\Delta_k)}\Big\Vert-\hat{\zeta}_{c,k}/\hat{\mu}_k\Big\Vert\sum_{n_c\in\mathcal{N}}\varrho_c\sum_{i\in\mathcal S_c}\rho_{i,c}\widehat{\mathbf g}_{i}^{(t_{k}-\Delta_k)}\Big\Vert\right\}\right]^+,
\end{equation}
where $\hat{\delta}_{c,k}$ is the estimated value for $\delta_c$.

\subsubsection{Estimation of $\sigma^2$}
% From Assumption~\ref{assump:SGD_noise}, a simple way of obtaining the value of $\sigma^2$ would be comparing the gradients from sampled devices with their full-batch counterparts. But this might be impractical if the local datasets $\mathcal{D}_i$ are large. Thus, we propose an approach where $\sigma^2$ is computed at each device through two independent mini-batches of data. Recall $|\xi_i|$ denotes the mini-batch size used at node $i$ during the model training.
At each instance of global aggregation, the edge servers collects the local gradients at $t=t_{k}-\Delta_k$. Since $\hat{\mathbf  g}_i^{(t_k-\Delta_k)}= \nabla F_i(\mathbf w^{(t_k-\Delta_k)})+\mathbf n_i^{(t_k-\Delta_k)}$, $\hat{\mathbf g}_j^{(t_k-\Delta_k)}= \nabla F_j(\mathbf w^{(t_k-\Delta_k)}) +\mathbf n_j^{(t_k-\Delta_k)},~\forall i\neq j$, we use the fact that $\mathbf n_i^{(t_k-\Delta_k)}$ and $\mathbf n_j^{(t_k-\Delta_k)}$ are independent random variables with $\mathbb E_t[\mathbf n_i^{(t_k-\Delta_k)}]=\mathbb E_t[\mathbf n_j^{(t_k-\Delta_k)}]\leq\sigma^2$, and thus $\mathbb E_t\Vert \hat{\mathbf  g}_i^{(t_k-\Delta_k)}-\hat{\mathbf  g}_j^{(t_k-\Delta_k)} \Vert^2 = \mathbb E_t\Vert \mathbf n_i^{(t_k-\Delta_k)} - \mathbf n_j^{(t_k-\Delta_k)} \Vert^2 \leq  2\sigma^2,~\forall i\neq j$. The estimation of $\sigma^2$ denoted by $\hat{\sigma}^2_k$ can then be obtained as follows:
\begin{align} \label{eq:est_sigma}
    \hat{\sigma}^2_k = \max_{i,j\in\mathcal I}\big\{\Vert\hat{\mathbf  g}_i^{(t_k-\Delta_k)}-\hat{\mathbf  g}_j^{(t_k-\Delta_k)}\Vert^2/2\big\}.
\end{align}
\fi


\iffalse
\textbf{Estimation of $\delta,\zeta,\sigma^2$:}\label{subsub:estparam} 
These parameters can be estimated by the main server during model training. The server can estimate $\delta$ and $\zeta$ at each global aggregation by receiving the latest gradients from SGD at the sampled devices. $\sigma^2$ can first estimated locally at the sampled devices, and then decided at the main server.



% where the main server broadcasts a few test global parameters $\mathbf{w}^{(1)},\cdots,\mathbf{w}^{(n)}$ and receives the corresponding local gradient from the nodes. 
 
% \begin{align}
%         \exists \beta>0: \Big\Vert \nabla F_i(\mathbf w_1)-\nabla F_i(\mathbf w_2)\Big\Vert \leq & \beta\Big\Vert \mathbf w_1-\mathbf w_2 \Big\Vert,~\forall i, \mathbf w_1, \mathbf w_2.
% \end{align}
%     where $\mathbf x_{i,b}\in\mathcal{D}_i,~\forall i,b$.
% During the training phase of {\tt FedAvg}, we collect the global model at the beginning of each \emph{local model training interval} at $t_{k-1}$, and 
% continually calculate 




% The 
% estimation of $\beta$ and $\mu$ can be conducted at via computing
% \begin{equation} \label{eq:Xi_est}
%   \hspace{-3mm} \begin{aligned}
%         &\Xi{(k',k'')}
%       = \frac{\left\Vert\sum\limits_{i=1}^I \varrho_i\widehat{\mathbf g}_{i}^{(k')}-\sum\limits_{i=1}^I \varrho_i\widehat{\mathbf g}_{i}^{({k''})}\right\Vert}{\left\Vert{\mathbf w}^{({k'})}-{\mathbf w}^{({k''})}\right\Vert}, 1\leq  k'\neq k''\leq n,
%     \end{aligned} \hspace{-3mm} 
% \end{equation}
% and saving the maximum values among $\Xi{(.,.)}$-s as a candidate for $\beta$ and the minimum value among $\Xi{(.,.)}$-s as a candidate for $\mu$. In~\eqref{eq:Xi_est}, $\widehat{\mathbf g}_{i}^{(k')}$ and $\widehat{\mathbf g}_{i}^{(k'')}$ denote the local SGD at the node computed based on $\mathbf w^{(k')}$ and $\mathbf w^{(k'')}$, respectively.


% as the candidates for $\hat{\mu_i}$ at every edge device $i$. The largest value of candidates estimated at every edge device $i$ throughout the entire training process will be chosen as $\hat{\beta}_i$. Then, we compute $\hat{\beta}\leftarrow\sum_{c=1}^N \varrho_c^{(k)}\hat{\beta}_i$ as the estimate for $\beta$.
% The same procedure can be used to estimate the internal value of $mu$. These values can then be sent to the server to estimate $\beta$ and $\mu$ via averaging. 
% We estimate the value of $\mu$ by the monoticity property of strongly convex of $F_i(\cdot)$ expressed as 
% \begin{align}
%     \exists \mu>0: \mu\leq\big(\nabla F_i(\mathbf w_1)-\nabla F_i(\mathbf w_2)\big)^\top\big(\mathbf w_1-\mathbf w_2\big)/\Vert\mathbf w _1-\mathbf w_2\Vert^2,~\forall \mathbf w_1, \mathbf w_2.
% \end{align}
% During the training phase of {\tt FedAvg}, we collect the global model at the beginning of each \emph{local model training interval} at $t_{k-1}$, and continually calculate $[\nabla F_i(\mathbf w _i(t))-\nabla F_i(\hat{\mathbf w}(t_{k-1}))]^\top[\mathbf w _i(t)-\hat{\mathbf w}(t_{k-1})]/\Vert\mathbf w _i(t)-\hat{\mathbf w}(t_{k-1})\Vert^2,~t\in\mathcal{T}_k$ as the candidates for $\hat{\mu_i}$ at every edge device $i$. The smallest value of candidates estimated at every edge device $i$ throughout the entire training process will be chosen as $\hat{\mu_i}$. Then, we compute $\hat{\mu}\leftarrow\sum_{c=1}^N \varrho_c^{(k)}\hat{\mu}_i$ as the estimate for $\mu$.

Specifically, to estimate $\delta,\zeta$, since the value of $\mathbf w^*$ is not known, we upper bound the gradient diversity Definition \ref{gradDiv} by introducing a new parameter $\delta'$:
\begin{align} \label{eq:estGradDiv}
  \hspace{-3mm}  \Vert\nabla\hat F_c(\mathbf w)-\nabla F(\mathbf w)\Vert \leq \delta+ \zeta \Vert\mathbf w-\mathbf w^*\Vert \leq  \delta' + \zeta \Vert\mathbf w\Vert,  \hspace{-3mm} 
\end{align}
\nm{Instead of centering it around 0 shouldnt it be better it to center it around the global aggregation parameter? Since the latter converges to Wstar?}
where $\delta'$ satisfies $\delta'\geq \delta+\zeta\Vert\mathbf w^*\Vert$\nm{how do you satrisfy this if you dont know wstar??  You said two lines ago that you used delta prime to overcome the fact that you dont know wstar, but you still need wster here.  }. Thus, a value of $\zeta\ll 2\beta$\nm{how do you set this??} is set, and then the value of $\delta'$ is estimated using~\eqref{eq:estGradDiv}\nm{I dont undestand, how do you estimate delta' using 53 without knowing wstar?}, where the server uses the SGD gradients $\widehat{\mathbf g}_{n_c}^{(t_k)}$ from the sampled devices $n_c$ at the instance of each global aggregation $k$, and chooses the smallest $\delta'$ such that $\Vert\nabla\hat F_c(\hat{\mathbf w}^{(t_k)})-\nabla F(\hat{\mathbf w}^{(t_k)})\Vert \approx \Vert \widehat{\mathbf g}_{n_c}^{(t_k)}-\sum_{c'=1}^N \varrho_{c'}\widehat{\mathbf g}_{n_{c'}}^{(t_k)}\Vert\leq \delta'+\zeta \Vert \hat{\mathbf w}^{(t_k)}\Vert$ $\forall c$.
% instead of their full-batch counterparts

% for each instance of $k'\in\{ 1,\cdots,n\}$ the maximum among which is chosen as the final value of $\delta'$.  Given $\Vert \mathbf w^{(k')}\Vert$, the server uses the SGD values instead of full-batch counterparts: $\Vert\nabla\hat F_c(\mathbf w^{(k')})-\nabla F(\mathbf w^{(k')})\Vert \approx \Vert \widehat{\mathbf g}_{i}^{(k')}-\sum_{j=1}^I \varrho_j\widehat{\mathbf g}_{j}^{(k')}\Vert\leq \delta'+\Vert \mathbf w^{(k')}\Vert$.

% To estimate the value of SGD variance $\sigma^2$, since the estimation of full-batch gradient might be impractical due to large number of local data points, each node $i$ estimates its local SGD noise at the end of each local model training round $k$ as $\sigma^2_i=\Vert\widehat{\mathbf g}_{j}^{(t_k)}-\widetilde{\nabla F}_j(\mathbf w_{j}^{(t_k)})\Vert^2$, where $\widehat{\mathbf g}_{j}^{(t_k)}$ is the latest SGD computed at the node and $\widetilde{\nabla F}_j(\mathbf w_{j}^{(t_k)})$ is a more accurate estimation of the full-batch gradient obtained via sampling a larger number of data points as compared to the mini-batch size. These scalars are then transferred to the main server, which chooses $\sigma^2=\max\{\sigma^2_1,\cdots,\sigma^2_I\}$.

From Assumption~\ref{assump:SGD_noise}, a simple way of obtaining the value of $\sigma^2$ would be comparing the gradients from sampled devices with their full-batch counterparts. But this might be impractical if the local datasets $\mathcal{D}_i$ are large. Thus, we propose an approach where $\sigma^2$ is computed at each device through two independent mini-batches of data. Recall $|\xi_i|$ denotes the mini-batch size used at node $i$ during the model training. 
At each instance of global aggregation, the sampled devices each select two mini-batches of size $|\xi_i|$ and compute two SGD realizations $\mathbf g_1$, $\mathbf g_2$ from which $\widehat{\mathbf g}_{i}^{(t_k)} = (\mathbf g_1 + \mathbf g_2) /2$. Since $\mathbf  g_1= \nabla F_i(\mathbf w^{(t_k)})+\mathbf n_1$, $\mathbf g_2= \nabla F_i(\mathbf w^{(t_k)}) +\mathbf n_2$, we use the fact that $\mathbf n_1$ and $\mathbf n_2$ are independent random variables with the same upper bound on variance $\sigma^2$, and thus $\Vert \mathbf  g_1-\mathbf  g_2 \Vert^2 = \Vert \mathbf n_1-\mathbf n_2 \Vert^2 \leq  2\sigma^2$, from which $\sigma^2$ can be approximated locally. These scalars are then transferred to the main server, where the server chooses the maximum reported $\sigma^2$ from the sampled devices.
\fi
% \subsubsection{Estimation of $\Upsilon^{(t)}_c$}
% Based on~\eqref{eq:Updef}, we propose the following approximation to estimate the value of $\Upsilon^{(t)}_c$:
% % By Definition \ref{paraDiv} and given the conditions in XXX, since the server needs to determine the learning parameters at the beginning of global aggregation, we are interesting in approximating the prediction of $\epsilon_c(k)$ for $t\in\mathcal T_k$ by $\epsilon_c(k)$ at the beginning of the global aggregation $k$. Therefore, $\epsilon_c(k)$ is approximated as
% % \begin{align*}
% %     \epsilon_c(k) &\geq \Vert\mathbf w_i(t)-\mathbf w_j(t)\Vert
% %     \\&
% %     \geq \vert\Vert\mathbf w_i(t)\Vert-\Vert\mathbf w_j(t)\Vert\vert, ~\forall i, j\in\mathcal{S}_c^{(k)}, ~\forall t\in\mathcal T_k
% % \end{align*}
%     \begin{align} \label{eq:Ups_est}
%          \Upsilon^{(t)}_c&\approx \max_{i, j\in\mathcal{S}_c,1\leq z\leq M}\{\vert[\mathbf w_i(t)]_z-[\mathbf w_j(t)]_z\vert\} 
%     \nonumber \\ &\approx \underbrace{\max_{i\in\mathcal{S}_c} \Vert  \mathbf w_i(t)\Vert_{\infty}}_{(a)}- \underbrace{\min_{j\in\mathcal{S}_c} \Vert  \mathbf w_j(t)\Vert_{\infty}}_{(b)},~ i\neq j, 
%     % \leq \max_{i, j\in\mathcal{S}^{(k)}_c,1\leq z\leq M}\{\vert[\mathbf w_i(t)]_z\vert+\vert[\mathbf w_j(t)]_z\vert\}
%     % \nonumber \\& \leq 2\underbrace{\max_{i, j\in\mathcal{S}^{(k)}_c,1\leq z\leq M}\{\vert[\mathbf w_i(t)]_z\vert\}}_{(a)},~\forall t\in \tau_k,
%     \end{align}
% % \begin{align}
% %     \Upsilon_c(k) &\approx \max_{i, j\in\mathcal{S}_c.}\{\Vert\mathbf w_i(t)\Vert-\Vert\mathbf w_j(t)\Vert\}
% %     \\& \label{epsProx}
% %     =\underbrace{\max_{i\in\mathcal{S}_c}\Vert\mathbf w_i(t)\Vert}_{(a)}-\underbrace{\min_{j\in\mathcal{S}_c}\Vert\mathbf w_j(t)\Vert}_{(b)}, ~\forall t\in\mathcal T_k,
% % \end{align}
% where we have used the lower bound $\Vert\mathbf a - \mathbf b \Vert_{\infty} \geq \Vert\mathbf a\Vert_{\infty}-\Vert\mathbf b\Vert_{\infty}$ for vectors $\mathbf a$ and $\mathbf b$, which we experimentally observe gives a better approximation of $\Upsilon^{(t)}_c$. In~\eqref{eq:Ups_est}, $(a)$ and $(b)$ can be both obtained in a distributed manner through scalar message passing, where each device $i\in\mathcal{S}_c$ computes $\max_{1\leq z\leq M}\{\vert[\mathbf w_i(t)]_z\vert\}$ and $\min_{1\leq z\leq M}\{\vert[\mathbf w_i(t)]_z\vert\}$ and shares it with its neighbors $j \in \mathcal{N}_i$. The devices update their $\max$ and $\min$ accordingly, share these updated values, and the process continues. After the rounds of message passing has exceeded the diameter of the graph, each node has the value of $(a)$ and $(b)$, and thus the estimate of $\Upsilon^{(t)}_c$. The server can obtain these values for $t \in \mathcal{T}_k$ from the node $n_c$ it samples for cluster $c$ at $t = t_k$. %The aforementioned estimated parameters are used in our control algorithm that adaptively tunes the number of D2D communication rounds inside the clusters in a distributed manner and obtains the length of local model training intervals.
  




% The values of $\tau_{\textrm{max}}$, $\phi$ and $T$ are provided as inputs.
% }
% First, estimates of different parameters are initialized, the value of $\phi$ is determined, and the first period of model training is set (lines 2-6). Then, during the local model training intervals, in each timestep, the devices (i) compute the SGD updates, (ii) estimate the cluster model divergence, (iii) determine the instants and number of local aggregations, and (iv) conduct the consensus process with their neighboring nodes (lines 12-16).

% At global aggregation instances, the sampled devices compute their estimated local SGD noise, and transmit it along with their model parameter vector, gradient vector, and estimates of cluster parameter divergence over the previous global aggregation round to the server (lines 20-21). Then, the main server (i) updates the global model, (ii) estimates $\zeta,\delta',\sigma$ for the step size, (iii) estimates the linear model coefficients used in~\eqref{eq:Up_dyn}, (iv) obtains the optimal length $\tau_{k+1}$ of the next local model training interval, and (v) broadcasts the updated global model, step size coefficients, local model training interval, and consensus coefficient, along with the indices of the sampled devices for the next global aggregation (line 23-29).

% \nm{Where are the numerical results?}





% To obtain (a) and (b) for each cluster, each device in the cluster computes $\Vert\mathbf w_i(t_k)\Vert$ and share it with its neighbors iteratively. During 
% each iteration, each device memorizes both the (i) maximum and (ii) minimum values among the exchanged values of $\Vert\mathbf w_i(t_k)\Vert$ such that the approximation of $\epsilon_c(k)$ can be estimated.

\iffalse
\begin{corollary} \label{SPEgenLin}
    Consider the case that each device performs rounds of distributed average consensus after each local gradient update, under Assumption \ref{beta}, \ref{PL}, Definition \ref{gradDiv} and Algorithm \ref{GT}, if the number of consensus rounds at different clusters of the network satisfies 
    $$
    \Gamma_c(k)\geq
    \left[\frac{\log(\sigma_c^{(k)})-2\log(s_c^{(k)}^2\varepsilon_c(k))}{2\log(\lambda_c)}\right]^+
    $$
    such that $\sigma_c^{(k)}$ satisfies the following inequalities 
    \begin{equation} \label{cond2}
        \begin{aligned}
            \begin{cases}
                \lim_{k\rightarrow\infty}B_{k+1}/B_k\leq 1, \\
                \eta_k\leq\frac{1}{\beta+\frac{2\beta^2\delta(\mathbf w)\tau_k}{\eta_k(1-\mu\eta_k)^{\tau_k-1}}\big[1-(1-\mu\eta_k)^{\tau_k-1}\big]},
            \end{cases},
        \end{aligned}
    \end{equation}
    \nm{which $\mathbf w$?}
    where $B_k = \frac{1-(1-\mu\eta_k)^{\tau_k}}{\mu}\beta^2\tau_k m^2 h_k\sum\limits_{c=1}^N\varrho_{c} \sigma_c^{(k)}$. 
    Then Algorithm \ref{GT} is guaranteed to converge:
    \begin{align*}
        &F(\mathbf w(t_{k}))-F(\mathbf w^*)
        \\&
        \leq\prod_{n=1}^k(1-\mu\eta_n)^{\tau_n} \Big(F(\mathbf w(0))-F(\mathbf w^*)\Big)+\mathcal O(C_k),
    \end{align*}
    where $C_k=\max\{B_k,\Big((1-\mu\eta_k)^{\tau_k}+\xi\Big)^{t_k}\}$ for any $\xi>0$.
\end{corollary}

For the case that each device performs rounds of distributed average consensus after each local gradient update, corollary \ref{SPEgenLin} 
provides a guideline for designing the number of consensus rounds $\Gamma_c(k)$ at different network clusters over each global aggregation $k$, the number of local updates in the $k$-th global aggregation $\tau_k$ and the step size $\eta_k$ to guarantee weak linear convergence.

The condition to guarantee convergence is more relaxed compared to the condition for the case in proposition \ref{genLin} that consensus are performed after several rounds of local gradient updates since $\sigma_c^{(k)}$ is always upper bounded by $s_c^{(k)}^2\epsilon_c^2(t)$.
It shows that by performing less number of local gradient updates between consensus, the system obtains better performance, which coincides with the result in our simulation.

\begin{corollary} \label{SPEstr}
    Consider the case that each device performs rounds of distributed average consensus after each local gradient update, under Assumption \ref{beta}, \ref{PL} and Definition \ref{gradDiv}, if the number of consensus rounds at different clusters of the network are chosen to satisfy
    $$
    \Gamma_c(k)\geq
    \left[\frac{\log(\sigma_c^{(k)})-2\log(s_c^{(k)}^2\varepsilon_c(k))}{2\log(\lambda_c)}\right]^+
    $$
    where $\sigma_c^{(k)}$, $\tau_k$ and $\eta_k$ are determined to satisfy the following inequalities   
    \begin{align} \label{cond3}
            \begin{cases} \label{35}
            h_k\sum\limits_{c=1}^N\varrho_{c} \sigma_c^{(k)}
            \leq\frac{\mu^2(\gamma_k-\lambda_k)}{2\beta^4 m^2 \tau_k\left[1-(1-\mu\eta_k)^{\tau_k}\right]}\Big\Vert\nabla F(\mathbf w(t_{k-1}))\Big\Vert^2, \\
            \eta_k\leq\frac{1}{\beta+\frac{2\beta^2\delta(\mathbf w)\tau_k}{\eta_k(1-\mu\eta_k)^{\tau_k-1}}\big[1-(1-\mu\eta_k)^{\tau_k-1}\big]},
            \end{cases}.
    \end{align}
    \nm{what is gamma and is lambdak$<$gamma?}
    Then Algorithm \ref{GT} is guaranteed to achieve linear convergence:
    \begin{align*}
        &F(\mathbf w(t_{k}))-F(\mathbf w^*)
        \leq (1-\lambda_k)\Big(F(\mathbf w(t_{k-1}))-F(\mathbf w^*)\Big)
    \end{align*}
    for $0\leq\lambda_k\leq1-(1-\mu\eta_k)^{\tau_k}$.
\end{corollary}
\nm{how do you choose tau? Why not choosing it to be constant over k?
}
Similar to corollary \ref{SPEgenLin}, consider the case that each device performs rounds of distributed average consensus after each local gradient update, corollary \ref{SPEstr} 
provides a guideline for designing the number of consensus rounds $\Gamma_c(k)$ at different network clusters over each global aggregation $k$, the number of local updates in the $k$-th global aggregation $\tau_k$ and the step size $\eta_k$ to guarantee strong linear convergence with a rate $\lambda$ upper bounded by $1-(1-\mu\eta_k)^{\tau_k}$. 

The condition to guarantee convergence is more relaxed compared to the condition for the case in proposition \ref{strLin} that consensus are performed after several rounds of local gradient updates\\
since $\sigma_c^{(k)}$ is always upper bounded by $s_c^{(k)}^2\epsilon_c^2(t)$\\
It shows that by performing less number of local gradient updates between consensus, the system obtains better performance, which coincides with the result in our simulation.

\begin{corollary} \label{inf}
    Consider the case that each device performs infinite rounds of distributed average consensus after each local gradient update, under Assumption \ref{beta}, \ref{PL} and Definition \ref{gradDiv}, Algorithm \ref{GT} is always guaranteed to achieve linear convergence when $\eta_k\leq\frac{1}{\beta+\frac{2\beta^2\delta(\mathbf w)\tau_k}{\eta_k(1-\mu\eta_k)^{\tau_k-1}}\big[1-(1-\mu\eta_k)^{\tau_k-1}\big]}$:
    \begin{align*}
        &F(\mathbf w(t_{k}))-F(\mathbf w^*)
        \leq (1-\mu\eta_k)^{\tau_k}\Big(F(\mathbf w(t_{k-1}))-F(\mathbf w^*)\Big).
    \end{align*}
\end{corollary} 

Corollary \ref{inf} asserts that each device performs consensus after local gradient update,
if each device performs infinite rounds of consensus within their cluster and the step size is properly chosen to satisfy $\eta_k\leq\frac{1}{\beta+\frac{2\beta^2\delta\tau_k}{\mu(1-\mu\eta_k)^{\tau_k-1}}}$,\nm{since both sides of the eq depend on eta, when is it satisfied? (it is for eta=0), hence there exists a  range $[0,\eta_{th}]$, where $\eta_{th}$ is a function of beta delta tau, mu, for which it is satisfied} linear convergence is always guaranteed.  

For each local model $i$, instead being updated based on its own data, it is equivalent to be updated based upon the entire dataset from the cluster it belongs to.

\section{Real-time adaptive {\tt \DFL} algorithm} \label{sec:control}
In this section, we develop a real-time adaptive {\tt \DFL} algorithm to tune the learning parameters on the step size, time span between consecutive global aggregation and the number of consensus based on the result of convergence analysis in section \ref{sec:convAnalysis}.

In Propositions \ref{genLin} and \ref{strLin}, policies to guarantee linear convergence are provided. Proposition \ref{genLin} provides condition in \eqref{cond1} and \eqref{cond2} on the learning parameters $\eta_k$, $\tau_k$ and $\Gamma_c(k)$ to guarantee a general linear convergence to the optimal whereas Proposition \ref{strLin} provides a stricter condition \eqref{cond1} and \eqref{cond3} on these learning parameters to guarantee strong linear convergence to the optimal with convergence rate $\lambda\leq1-(1-\mu\eta_k)^{\tau_k}$.

To realize the policies provided in Propositions 1 and 2, we design a real-time adaptive algorithm to tune the learning parameters in practice. It is assumed that the sever
has an estimate about the topology of each cluster, and thus the upper-bound of the
spectral radius $\lambda_c$ at each global iteration.

According to \eqref{cond1}, \eqref{cond2} and \eqref{cond3}, the tuning of learning parameters requires the knowledge on the upper bound of (A.) parameter divergence $\epsilon_c(k)$ to satisfy the condition provided by Proposition \ref{genLin} and \ref{strLin}. In addition, Proposition \ref{strLin} further requires information of the (B.) global gradient $\Vert\nabla F(\mathbf w(t_{k-1}))\Vert$. Therefore, we first derive an adaptive estimation for the divergence of parameters in a distributed manner for every local iteration $t$. Then, we focus on the approach to approximate the global gradient $\Vert\nabla F(\mathbf w(t_{k-1}))\Vert$ at each global iteration $k$. 

\subsection{Estimation of parameter divergence} 
By Definition \ref{paraDiv} and given the conditions in \eqref{cond1} and \eqref{cond2}, since the server needs to determine the learning parameters at the beginning of global aggregation, we are interesting in approximating the prediction of $\epsilon_c(k)$ for $t\in\mathcal T_k$ by $\epsilon_c(k)$ at the beginning of the global aggregation $k$. Therefore, $\epsilon_c(k)$ is approximated as
% \begin{align*}
%     \epsilon_c(k) &\geq \Vert\mathbf w_i(t)-\mathbf w_j(t)\Vert
%     \\&
%     \geq \vert\Vert\mathbf w_i(t)\Vert-\Vert\mathbf w_j(t)\Vert\vert, ~\forall i, j\in\mathcal{S}_c^{(k)}, ~\forall t\in\mathcal T_k
% \end{align*}

\begin{align}
    \epsilon_c(k) &\approx \max_{i, j\in\mathcal{S}_c^{(k)}.}\{\Vert\mathbf w_i(t_k)\Vert-\Vert\mathbf w_j(t_k)\Vert\}
    \\& \label{epsProx}
    =\underbrace{\max_{i\in\mathcal{S}_c^{(k)}}\Vert\mathbf w_i(t_k)\Vert}_{(a)}-\underbrace{\min_{j\in\mathcal{S}_c^{(k)}}\Vert\mathbf w_j(t_k)\Vert}_{(b)}, ~\forall t\in\mathcal T_k,
\end{align}

To obtain (a) and (b) for each cluster, each device in the cluster computes $\Vert\mathbf w_i(t_k)\Vert$ and share it with its neighbors iteratively. During 
each iteration, each device memorizes both the (i) maximum and (ii) minimum values among the exchanged values of $\Vert\mathbf w_i(t_k)\Vert$ such that the approximation of $\epsilon_c(k)$ can be estimated.
\fi
% \subsection{Approximation of global gradient}
% In this section, we propose a two-step method to approximate $\Vert\nabla F(\mathbf w(t_{k-1}))\Vert$. By \eqref{15}, we first approximate $\nabla F(\mathbf w(t_{k-2}))$ as 
% $\nabla \widehat{F}(\mathbf w(t_{k-2}))=\mathbf w(t_{k-2})-\mathbf w(t_{k-1})/\eta_{k-2}$.\nm{why?}
% Since $F(\cdot)$ is strongly convex, the value $\Vert\nabla F(\mathbf w(t_{k}))\Vert$ is expected to decrease over $k$. Thus, $\Vert\nabla F(\mathbf w(t_{k-1}))\Vert$ can be further approximated by $\Vert\nabla \widehat{F}(\mathbf w(t_{k-1}))\Vert=\alpha\Vert\nabla F(\mathbf w(t_{k-2}))\Vert$, where $0\leq\alpha\leq 1$. The two-step approximation can be summarized as
% \begin{align} \label{gradProx}
%     \Vert\nabla \widehat{F}(\mathbf w(t_{k-1}))\Vert=\alpha\big(\mathbf w(t_{k-2})-\mathbf w(t_{k-1})\big)/\eta_{k-2}.
% \end{align}
% \nm{norm of RHS?}

% bibtex


% \section{Possible Optimization Formulations}
% 1) Optimization formulation:
% In the first proposed formulation, we aim to optimize the next interval for the local training period at each global aggregation with a look-ahead method. In particular, we try to minimize the total cost consisting of communication energy and delay starting from the beginning of k-th global aggregation (i,e., $t=t_{k-1}$) all the way to the end (i.e., $t=T$) to determine the value of local training period for the current global aggregation period (i.e., $\tau_{k}$). We assume that the value of $\tau_{k}$ is also going to be used for the rest of the training period, i.e., $\tau_{k}=\tau_{k+1}=....$. The intuition behind this is that we want to find the best choice of $\tau_{k}$ that minimizes the overall cost for the remaining training process as the decision for $\tau_{k}$. 

% We propose the following optimization  problem:
% \begin{align}
%   &\hspace{-3mm}\min_{\tau_k,\phi}  c_1 \times \left(\left[\sum_{t=t_{k-1}}^{T}\sum_{c=1}^{N}\theta_{c}(t)*E_{\textrm{D2D}} \right]+ K*E_{\textrm{Glob}} \right) + c_2\times \left(\left[\sum_{t=t_{k-1}}^{T}\sum_{c=1}^{N}\theta_{c}(t)*\Delta_{\textrm{D2D}}\right]+K*\Delta_{\textrm{Glob}}\right) \label{obj}
%   \hspace{-3mm} \\
%   & \textrm{s.t.}\nonumber \\
%   & \theta_c^{(t)}\geq\max\Big\{\log\Big(\eta_t\phi/(s_c\Upsilon_c^{(t)})\Big)/\log\Big(\lambda_c^{(t_{k-1})}\Big),0\Big\},~t_{k-1}\leq t\leq T \\
% %   &0<\tau_k<\log_4\frac{(\mu\gamma-1)\kappa}{16\omega^2\beta^2\gamma^2} \\
% %   &(T+\alpha) \xi>\Gamma\geq \alpha [F(\hat{\mathbf w}(0))] \\
% %   &[(\mu\gamma-1)(\Gamma-\beta\phi\gamma/2)-B]/[\frac{\gamma^2\beta}{2}A+\frac{16\omega^2\beta^2\gamma^2(\Gamma-\beta\phi\gamma/2)}{\kappa}]>4 \\
%   &1 \leq \tau_k \leq \sqrt{\Big[\mu\gamma-[3+8(1/\vartheta+(2\omega-\vartheta/2))^2]\Big]\Big[\tilde{\Gamma}-\vartheta\gamma\phi^2/2\Big]/(\gamma^2\beta^2 B)-A/B} \\
% %   &T\zeta/[F(\hat{\mathbf w}(0))-\xi]>\alpha\geq\max\{\beta\gamma/\kappa, \beta\gamma\big[1-\kappa/4+\sqrt{(1+\kappa/4)^2+2\omega}\big]\} \\
% %   &\Upsilon_c^{(t_k)} = 0\\
% %   &\Upsilon_c^{(k)} \leq d(\tau)\Upsilon_c^{(k-1)} \\
%     &K=\floor{\frac{T-t_{k-1}}{\tau_k}} \\
%     & \Upsilon_c^{(t_{k'-1})}=0, k'=k,\cdots,K,~ \forall c \\
%     & \Upsilon_c^{(t)} = 1_{\{\theta^{(t)}_c =0\}} (\hat{\chi}_c^{(k')}\Upsilon_c^{(t-1)}+\bar{\chi}_c^{(k')})+ (1-1_{\{\theta^{(t)}_c =0\}}) (\tilde{\chi}_c^{(k')}), t\in \bigcup_{k'=k}^{K} \overline{\mathcal{T}_{k'}}, ~\forall c\\
%   &\hat{\chi}_c^{(k')}=\hat{R}_c \hat{\chi}_c^{(k'-1)},~k'=k,\cdots,K, ~\forall c,\\
%   & \bar{\chi}_c^{(k')} = \bar{R}_c \bar{\chi}_c^{(k'-1)},~k'=k,\cdots,K, ~\forall c\\
%   &\tilde{\chi}_c^{(k')} = \tilde{R}_c \tilde{\chi}_c^{(k'-1)},~k'=k,\cdots,K, ~\forall c\\
%   &\overline{\mathcal{T}_k'}=\mathcal{T}_k'\setminus \{t_{k'-1}\},~k'=k,\cdots,K.
%     % & \Upsilon_c^{(t_{k'}+1)} = c''(k')\Upsilon_c^{(t_{k'-1}+1)}, k'=k-1,\cdots,K  \\
%     % & c(t)=c'(t)=0, t=t_{k'},~k' = 1,\cdots,K
% \end{align}



% \textbf{Successive Estimation of Divergence of Parameters ($\Upsilon$)}: rate of increase inside a global aggregation, periodic behavior, rate of decrease in increase in rate between two global aggregation... 
% Clearly reveal how this is a function of tau!!!!

% 1.3 , 5..
% 1.2 , 6..
% 1.2 * ... , 6

% 1.2/1.3



% We can see that the above optimization problem is highly non-convex. Thus, we propose to estimate and set a few of the parameters such as $\phi,\Gamma,\alpha,\gamma$ during a first global aggregation and keep tuning the $\tau_k$ for the rest of global aggregations. Note that even fixing all the parameters except $\tau_k$ does not result in a unique $\tau_k$ since the value of $\Upsilon_c^{(t)}$ keeps changing over time and we need to keep tracking and estimating it, which changes the value of $\tau_k$ since it affects the number of D2D rounds $\theta_c^{(t)}$.

\iffalse
Formulation 2: In this formulation, the designer also wants to minimize the convergence bound at a given time instance $T$.
\begin{align}
    &\min_{\tau,\phi}  
    c_0 \xi
    +c_1 \times \left(\sum_{t=t_{k-1}}^{T}\theta_{c}(t)*E_{\textrm{D2D}} + K*E_{\textrm{Glob}} \right) 
    +c_2\times \left(\sum_{t=t_{k-1}}^{T}\theta_{c}(t)*\Delta_{\textrm{D2D}}+K*\Delta_{\textrm{Glob}}\right)
    \\
   & \textrm{s.t.}\\
   & \frac{\Gamma}{T+\alpha}\leq\xi \\
   & \theta_c^{(t)}\geq\max\Big\{\log\Big(\eta_t\phi/(s_c\Upsilon_c^{(t)})\Big)/\log\Big(\lambda_c^{(t_{k-1})}\Big),0\Big\} \\
   &0<\tau_k<\log_4\frac{(\mu\gamma-1)\kappa}{16\omega^2\beta^2\gamma^2} \\
   &(T+\alpha) \xi>\Gamma\geq \alpha [F(\hat{\mathbf w}(0))] \\
   &[(\mu\gamma-1)(\Gamma-\beta\phi\gamma/2)-B]/[\frac{\gamma^2\beta}{2}A+\frac{16\omega^2\beta^2\gamma^2(\Gamma-\beta\phi\gamma/2)}{\kappa}]>4 \\
   &\tau_k \leq \log_4\left\{[(\mu\gamma-1)(\Gamma-\beta\phi\gamma/2)-B]/[\frac{\gamma^2\beta}{2}A+\frac{16\omega^2\beta^2\gamma^2(\Gamma-\beta\phi\gamma/2)}{\kappa}]\right\} \\
   &T\zeta/[F(\hat{\mathbf w}(0))-\xi]>\alpha\geq\max\{\beta\gamma/\kappa, \beta\gamma\big[1-\kappa/4+\sqrt{(1+\kappa/4)^2+2\omega}\big]\} \\
%   &\Upsilon_c^{(t_k)} = 0\\
%   & \Upsilon_c^{(t)} \leq c\Upsilon_c^{(t-1)}+c',~~t\in \mathcal{T}_k\\
%   &\Upsilon_c^{(k)} \leq d(\tau)\Upsilon_c^{(k-1)} \\
    &K=\floor{\frac{T-t_{k-1}}{\tau}}
    \\
    &t_{k-1}\leq t\leq T
\end{align}
In this formulation, the value of $\xi$ is also one of the optimization variables. Based on the solving technique discussed above, since we have different values of $\tau_k$ for different rounds of global aggregations, we need to recompute the optimization process with different values of $\xi$ and find the corresponding optimal choice of $\tau_k$ repeatedly for every aggregation rounds. However, different decisions on the value of $\xi$ in different global aggregation round correspondingly results in different range for the options of $\alpha$ and $\Gamma$ (can be observed from \eqref{45} and \eqref{46}). And since $\eta_t=\frac{\gamma}{t+\alpha}$ depends on the value of $\alpha$, the step size would vibrate up and down across aggregations (instead of steadily decreasing linearly). Another approach is to fix all the variables as described above for a fixed $\xi$ and keep changing the $\xi$ via a line search to obtain the best combination of $(\xi,\tau_1,\phi)$ and then we fix everything except $\tau_k$ and keep recomputing it through the algorithm. 
\fi

% for the case when $\alpha>\frac{\gamma^2\beta}{2}[Q_k A+B]/\{[\mu\gamma-(1+16\omega^2\beta^2\gamma^2 Q_k/\kappa)]F(\hat{ \mathbf{w}}(0)\}$, $\Gamma$ depends on the value of $\phi$, whereas for the case when $\alpha\leq\frac{\gamma^2\beta}{2}[Q_k A+B]/\{[\mu\gamma-(1+16\omega^2\beta^2\gamma^2 Q_k/\kappa)]F(\hat{ \mathbf{w}}(0)\}$, $\Gamma$ depends both on the value of $\phi$ and $\tau$. 










We present in section~\ref{ssec:faces} an application of PnP-HVAE on face images, using a pretrained state-of-the-art hierarchical VAE. 
Next, we study the application of our framework to natural images. To that end, we introduce  in section~\ref{ssec:patchVDVAE}  a patch hierachical VAE architecture, that is able to model natural images of different resolutions. In section~\ref{ssec:app_nat}, we provide deblurring, super-resolution and inpainting experiments to demonstrate the relevance of the proposed method.

Additional results are presented in Appendix~\ref{app:add}. All experiments can be reproduced using the code available at \url{https://github.com/jprost76/PnP-HVAE}.



\subsection{Face Image restoration (FFHQ)}\label{ssec:faces}
We first demonstrate the effectiveness of PnP-HVAE on highly structured data, by performing face image restoration.
Latent variable generative models can accurately model structured images such as face images \cite{karras2019style,vahdat2020nvae,child2021very,kingma2018glow}, and then be used to produce high quality restoration of such data. 
In our experiments, we use the VDVAE model of~\cite{child2021very}, pre-trained on the FFHQ dataset~\cite{karras2019style}, as our hierarchical VAE prior.
VDVAE has $L=66$ latent variable groups in its hierarchy and generates images at resolution $256\times256$.

We compare PnP-HVAE with the intermediate layer optimization algorithm (ILO)~\cite{daras2021intermediate} that is based on a different class of generative models than HVAE. ILO is a GAN inversion method which optimizes the image latent code along with the intermediate layer representation of a StyleGAN to generate an image consistent with a degraded observation.
We use the official implementation of ILO, along with a StyleGAN2 model~\cite{karras2020analyzing, stylegan2pytorch}, that was trained for 550k iterations on images of resolution $256\times256$ from FFHQ.  
As VDVAE and StyleGAN models are not trained on the same train-test split of FFHQ, we chose to evaluate the methods on a subset of 100 images from the CelebA dataset~\cite{liu2018large}. 
For super-resolution, the degradation model corresponds to the application of a gaussian low-pass filter followed by a $\times 4$ sub-sampling, and the addition of a gaussian white noise with $\sigma=3$.
For the deblurring, we considered motion blur and  gaussian kernels, both with a noise level $\sigma=8$. %

We provide quantitative comparisons in table~\ref{table:comp_ILO}, along with a visual comparison of the results in figure~\ref{fig:face_restoration}.
PnP-HVAE has the best  PSNR and SSIM results for all the considered restoration tasks, while ILO provides better results  for the perceptual distance.
By jointly optimizing the image and its latent variable, PnP-HVAE provides  results that are both realistic and consistent with the degraded observation.
On the other hand,  ILO  only optimizes on an extended latent space. This method generates  sharp and realistic images with better LPIPS scores,   
but the results lack  of consistency with respect to the observation, which explains the overall lower PSNR performance. 






\subsection{PatchVDVAE: a HVAE for natural images}\label{ssec:patchVDVAE}
Available generative models in the literature operate on images of  fixed resolutions and
are either restrained to datasets of limited diversity, or even to registered face images~\cite{kingma2018glow,child2021very, vahdat2020nvae, karras2019style}, or requiring additional class information~\cite{brock2018large, dhariwal2021diffusion, song2020score, luhman2022optimizing}.
Fitting an unconditional model on natural images appears to be a more difficult task, as their resolution can change, and their content is highly diverse.
The complexity of the problem can be reduced by learning a prior model on patches of reduced dimension. 
For image restoration problems, the patch model can be reused on images of higher dimensions~\cite{zoran2011learning,prost2021learning,altekruger2022patchnr}. When the model is a full CNN, the prior on the set of the  patches can  be computed efficiently by applying the network on the full image~\cite{prost2021learning}.

We thus introduce  patchVDVAE, a fully convolutional hierarchical VAE.
Contrary to existing HVAE models whose resolution is constrained by the constant tensor at the input of the top-down block, patchVDVAE can generate images of different resolutions by controlling the dimension of the input latent. 
This amounts to defining a prior on patches whose dimension corresponds to the receptive field of the VAE. A similar model is used for image denoising in~\cite{prakash2021interpretable}.

 
For PatchVDVAE architecture, we use the same bottom-up and top-down blocks as VDVAE~\cite{child2021very}, and replace the constant trainable input in the first top-down block by a latent variable, to make the model fully convolutional (details on the  architecture are given in Appendix~\ref{app:details}). 
The training dataset is composed of $128\times 128$ patches extracted from a combination of DIV2K~\cite{agustsson2017ntire} and Flickr2K~\cite{Lim_2017_CVPR_workshops} datasets.
We perform data augmentation by extracting  patches at $3$ resolutions: HR-images and $\times 2$ and $\times 4$ downscaled images. 
The model is trained for $7.10^5$ iterations with a batch size of $64$. Following the recommendation of~\cite{hazami2022efficient}, we use Adamax optimizer with an exponential moving average and gradient smoothing of the variance.
We set the decoder model to be a gaussian with diagonal covariance, as in~\cite{luhman2022optimizing}.
PatchVDVAE is fully convolutional and can generate images of dimension that are multiples of $64$ as illustrated by
figure~\ref{fig:vdvae}.

\newlength{\patchwidth}
\setlength{\patchwidth}{0.135\columnwidth}
\begin{figure}[!ht]
    \centering
    \begin{subfigure}[t]{.34\columnwidth}\hspace{0.1cm}
        \setlength{\tabcolsep}{0.02pt}
\renewcommand{\arraystretch}{0}
        \begin{tabular}{*{2}{p{1.03\patchwidth}}}
            \includegraphics[width=\patchwidth]{figures_arxiv/patchVDVAE/samples/generated/64x64/setup-5-image-0018.png} &
            \includegraphics[width=\patchwidth]{figures_arxiv/patchVDVAE/samples/generated/64x64/setup-5-image-0016.png} \\
            \includegraphics[width=\patchwidth]{figures_arxiv/patchVDVAE/samples/generated/64x64/setup-5-image-0008.png} &
            \includegraphics[width=\patchwidth]{figures_arxiv/patchVDVAE/samples/generated/64x64/setup-5-image-0019.png}   
        \end{tabular}
    \end{subfigure}\hspace{-0.15cm}
    \begin{subfigure}[t]{.64\columnwidth}
\begin{tabular}{cc}\vspace{-0.1cm}
\includegraphics[width=2\patchwidth]{figures_arxiv/patchVDVAE/samples/generated/256x256/setup-2-image-0009.png}&
        \includegraphics[width=2\patchwidth]{figures_arxiv/patchVDVAE/samples/generated/256x256/setup-2-image-0002.png}\end{tabular}

    \end{subfigure}
    \caption{\label{fig:vdvae} Left: $64\times64$ patches samples from our patchVDVAE model trained on patches from natural images.
    Right: PatchVDVAE is fully convolutional and it can generate images of higher resolution (here: $128\times128$).\vspace{-0.2cm}}
\end{figure}

\subsection{Natural images restoration}\label{ssec:app_nat}
We  evaluate PnP-HVAE on natural image restoration.
For each task, we report the average value of the PSNR, the SSIM, and the LPIPS metrics on $20$ images from the test set of the BSD dataset~\cite{MartinFTM01}.\\


\noindent
{\bf Image deblurring.}
In the experiments, we consider $2$ gaussian kernels and $2$ motion blur kernels from~\cite{levin2009understanding}, with $3$ different noise levels 
$\sigma \in \{2.55, 7.65, 12.75\}$.
As a baseline we consider  EPLL~\cite{zoran2011learning}, which learns a prior on image patches with a gaussian mixture model.
We also compare PnP-HVAE  with PnP-MMO and GS-PnP, $2$ competing convergent Plug-and-Play methods based on CNN denoisers.
PnP-MMO~\cite{pesquet2021learning} restricts the denoiser to be contraction in order to guarantee the convergence of the PnP forward-backard algorithm. GS-PnP~\cite{hurault2022gradient} considers a gradient step denoiser and reaches state-of-the-art performances of non converging methods~\cite{zhang2021plug}.
We set the temperature $\tau$  in our method as $0.95$, $0.8$ and $0.6$ for noise levels $2.55$, $7.65$ and $12.75$ respectively, and we let it run for a maximum of $50$ iterations. 
For the three compared methods we use the official implementations and pre-trained models provided by the respective authors. 
Details on the choice of hyperparameters for the concurrent methods are provided in the Appendix~\ref{app:details}
Figure~\ref{fig:deblurring_bsd} illustrates that our method provides correct deblurring results. 

According to table~\ref{tab:deb}, the performance of PnP-HVAE is between those of EPLL and GS-PnP and it outperforms PnP-MMO for large noise levels.\\

\begin{table}
\begin{center}\footnotesize
    \begin{tabular}{>{\centering}m{.3cm}*{5}{c}}
    $\sigma$ &Method & PSNR$\uparrow$ & SSIM$\uparrow$ & LPIPS$\downarrow$  \\ 
    \hline
    \multirow{4}{*}{\vcell{$2.55$}}
    & PnP-HVAE & $27.75$ & $0.79$ & $0.31$\\
    & GS-PNP \cite{hurault2022gradient} & $\mathbf{29.59}$ & $\mathbf{0.84}$ & $\mathbf{0.22}$\\
    & EPLL \cite{zoran2011learning} & $26.49$ & $0.71$ & $0.36$\\ 
    & PnP-MMO \cite{pesquet2021learning} & $\underbar{29.50}$ & $\underbar{0.83}$ & $\underbar{0.20}$ \\ \hline
    \multirow{4}{*}{\vcell{$7.65$}}
    & PnP-HVAE & $\underbar{26.36}$ & $\underbar{0.72}$ & $\underbar{0.40}$\\
    & GS-PNP \cite{hurault2022gradient} & $\mathbf{27.33}$ & $\mathbf{0.77}$ & $\mathbf{0.31}$\\
    & EPLL \cite{zoran2011learning} & $24.04$ & $0.66$ & $0.45$ \\ 
    & PnP-MMO \cite{pesquet2021learning} & $25.34$ & $0.69$ & $0.34$\\
    \hline
    \multirow{4}{*}{\vcell{$12.75$}}
    & PnP-HVAE & $\underbar{25.12}$ & $\mathbf{0.73}$ & $\underbar{0.47}$\\
    & GS-PNP \cite{hurault2022gradient} & $\mathbf{26.32}$ & $\mathbf{0.73}$ & $\mathbf{0.37}$\\
    & EPLL \cite{zoran2011learning} & $23.28$ & $0.61$ & $0.51$ \\ 
    & PnP-MMO \cite{pesquet2021learning} & $22.42$ & $0.53$& $0.54$ \\
    \hline
    &\vspace*{-.3cm}\\
            \multicolumn{2}{c}{Blur and motion kernels}& \multicolumn{3}{c}{
        \includegraphics*[scale=1]{figures_arxiv/kernels/4.png}\;\includegraphics*[scale=1]{figures_arxiv/kernels/7.png}\;\includegraphics*[scale=1]{figures_arxiv/kernels/9.png}\;\includegraphics*[scale=1]{figures_arxiv/kernels/11.png}} 
    \end{tabular}
        \caption{\label{tab:deb}Comparison  of PnP-HVAE  and other restoration methods on deblurring. Results are averaged on $4$ kernels.\vspace{-0.2cm}}% on image deblurring.}
    \end{center}
\end{table}

\begin{figure}
    
    \begin{subfigure}[h]{\linewidth}
        \centering
        \includegraphics*[width=\columnwidth]{figures_arxiv/deb_s255_k7.pdf}\vspace{-0.1cm}
        \caption{Gaussian blur, $\sigma=2.55$}
    \end{subfigure}
    \begin{subfigure}[h]{\linewidth}
        \centering
        \includegraphics*[width=\columnwidth]{figures_arxiv/deb_s765_k11.pdf}\vspace{-0.1cm}
        \caption{Motion blur, $\sigma=7.65$}
    \end{subfigure}\vspace*{-0.1cm}
    \caption{\label{fig:deblurring_bsd} Natural image deblurring\vspace{-0.1cm}}
\end{figure}

\noindent {\bf Effect of the temperature.}
PnP-HVAE gives control on the temperature of the prior over the latent space.
In figure~\ref{fig:temp_effect}, we illustrate that reducing the temperature increases the strength of the regularization prior. In this example the tuning $\tau=0.7$ produces the best performance.\\
\begin{figure}[!ht]
   
    \includegraphics[width=\columnwidth]{figures_arxiv/demo_temp.pdf}\vspace{-0.15cm}
    \caption{ \label{fig:temp_effect} Effect of the temperature in PnP-VAE on a deblurring problem, with $\sigma=7.65$.\vspace{-0.15cm}}
\end{figure}


\noindent
{\bf Image inpainting.}
Next we consider the task of noisy image inpainting. 
We compose a test-set of 10 images from the validation set of BSD~\cite{MartinFTM01} and we create masks
  by occluding diverse objects of small size in the images. 
A gaussian white noise with $\sigma=3$ is added to the images.
As a comparaison, we still consider GS-PnP and EPLL.
For PnP-HVAE, the temperature is set to $\tau=0.6$, and the algorithm is run for a maximum of $200$ iterations, unless the residual $||\x_{k+1}-\x_k||$ is on a plateau.
We provide on Table~\ref{tab:inpainting_bsd} the distortion metrics with the ground truth, as well as a visual
\begin{table}



\begin{center}
    \begin{tabular}{cccc}
        & PSNR$\uparrow$ & SSIM$\uparrow$ &LPIPS$\downarrow$ \\\hline
        PnP-HVAE  & $\mathbf{29.54}$ & $\mathbf{0.93}$ & $\mathbf{0.06}$\\
        GS-PNP & $28.52$ & $\mathbf{0.93}$ & $0.09$\\
        EPLL & $\underline{29.16}$ & $\mathbf{0.93}$ & $\mathbf{0.06}$\\
    \end{tabular}
    \caption{\label{tab:inpainting_bsd}Quantitative evaluation for inpainting on BSD.}
    \end{center}
\end{table}
comparison on figure~\ref{fig:inpainting_bsd}. 
With its hierarchical structure,  PnP-HVAE outperforms the compared methods. \vspace{0.05cm}



\begin{figure}[!h]
    \includegraphics[width=\columnwidth]{figures_arxiv/demo_inp_bsd2.pdf}\vspace{-0.1cm}
    \caption{\label{fig:inpainting_bsd}Natural image inpainting\vspace{-0.3cm}}
\end{figure}













 



\section{Conclusion and Future Work}
\noindent In this work, we proposed {\tt DFL}, which is a novel methodology that aims to improve the efficiency of distributed machine learning model training by mitigating the round-trip communication delay between the edge and the cloud. {\tt DFL} quantifies the effects of delay and modifies the FL algorithm by introducing a linear local-global model combiner used in the local model synchronization steps.
% , optimizing model performance under a hierarchical model training architecture subject to communication delay. 
We investigated the convergence behavior of {\tt DFL} under a generalized data heterogeneity metric and obtained a set of conditions to achieve sub-linear convergence. Based on these characteristics, we developed an adaptive control algorithm that adjusts the learning rate, local aggregation rounds, combiner weight, and global synchronization periods. Our numerical evaluation showed that {\tt DFL} leads to a faster global model convergence, lower resource consumption, and a higher robustness against communication delay compared to existing FL algorithms.
% Future work includes the implementation of {\tt DFL} in a real-world system and testing its performance in a real-world setting. Moreover, 
Future research directions include improving the robustness of DFL against different types of network impairments, such as jitter and packet loss, and investigating its performance under flexible device participation.

\iffalse
\noindent We proposed {\tt TT-HF}, a methodology which improves the efficiency of federated learning in D2D-enabled wireless networks by augmenting global aggregations with cooperative consensus formation among device clusters. We conducted a formal convergence analysis of {\tt TT-HF}, resulting in a bound which quantifies the impact of gradient diversity, consensus error, and global aggregation periods on the convergence behavior. Using this bound, we characterized a set of conditions under which {\tt TT-HF} is guaranteed to converge sublinearly with rate of $\mathcal{O}(1/t)$. Based on these conditions, we developed an adaptive control algorithm that actively tunes the device learning rate, cluster consensus rounds, and global aggregation periods throughout the training process. Our experimental results demonstrated the robustness of {\tt TT-HF} against data heterogeneity among edge devices, and its improvement in trained model accuracy, training time, and/or network resource utilization in different scenarios compared to the current art.

There are several avenues for future work. To further enhance the flexibility of {\tt TT-HF}, one may consider (i) heterogeneity in computation capabilities across edge devices, (ii) different communication delays from the clusters to the server, and (iii) wireless interference caused by D2D communications.
\fi
% \pagebreak 
\bibliographystyle{IEEEtran}
\bibliography{ref}

\pagebreak

% 
% %\newpage
\section{Alternative Definitions}\label{sec:other-definitions-short}
In this section, we discuss other potential definitions of Leximin approximation that might be considered intuitive.
\eden{removed ack. for anonymous submission}
% \footnote{We thank Sylvain Bouveret for suggesting definitions \ref{altDef:5} and \ref{altDef:6}.}.
For each alternative, we provide an example that illustrates why we believe it is inappropriate and a conclusion based on that example.
It should be noted that in order to avoid confusion, the error parameter $\gamma \in (0,1)$ is used in the alternative definitions (instead of $\beta$), to emphasize that these are only alternatives we do not use.


\begin{potentialDefinition}\label{altDef:2}
    A solution $x$ is a $(1-\gamma)$-approximately optimal if given a Leximin-optimal solution, $x^*$, there exists an integer $k \in [n]$ such that: 
    \begin{align*}
    \forall j < k: & \valBy{j}{x} \geq (1-\gamma) \cdot \valBy{j}{x^*}\\
    & \valBy{k}{x} > \valBy{k}{x^*}
    \end{align*}
\end{potentialDefinition}

\paragraph{Bad example def. \ref{altDef:2}:} Consider the following example with three objectives:
\begin{align*}
    \max \quad &\{f_i(x) = x_i \mid \forall 1 \leq i \leq 3 \} \\ \tag{E1}\label{eq:alt-def-eaxmple-1}
    s.t. \quad  & 99 x_1 + x_2 \leq 100\\
    &  x_3 \leq 100\\
    & x \in \mathbb{R}^3_{+}
\end{align*}
The Leximin optimal solution $x^*$ is $(1,1, 100)$ and therefore, by taking $k$ to be $2$, we get that any solution that its minimum objective value is at least $(1-\gamma)$ and its second-smallest objective value is more than $1$ is considered $(1-\gamma)$-approximately optimal Leximin solution.
For instance, consider $\gamma = 0.1$, the solution $(0.9, 1.1, 1.1)$ should be considered a $0.9$-approximately optimal according to this definition.
However, it is easy to see that this solution is quite bad for $f_2$ who can achieve $10.9$ (higher by a factor $> 9$) and very bad for $f_3$ who can achieve $100$ (higher by a factor $>90$).
And so, it seem reasonable to require that a good definition will consider as many objectives as possible.
% \erel{What objective values exactly? Do you mean: as many objectives as possible?}
% \eden{yes. To myself: this comment might be relevant to other places..}

\paragraph{Conclusion def. \ref{altDef:2}:} An appropriate definition should take into account as many objectives as possible.

\begin{potentialDefinition}\label{altDef:1}
    A solution $x$ is a $(1-\gamma)$-approximately optimal if for a Leximin-optimal solution, $x^*$, and for each $j = 1, \dots, n$ the following holds: 
    \begin{align*}
        \valBy{j}{x} \geq (1-\gamma) \cdot \valBy{j}{x^*} 
    \end{align*}
\end{potentialDefinition}

\paragraph{Bad example def. \ref{altDef:1}:} 
% An error in the first objective value might cause the other values to increase significantly.
Consider example \eqref{eq:alt-def-eaxmple-1} again.
Here, as the optimal solution is $(1,1, 100)$, any solution that yields at least $(1-\gamma,1-\gamma, (1-\gamma)\cdot 100)$.
However, considering $\gamma = 0.1$, $f_2$ can again achieve $9.1$ which is higher by a factor $> 100$ than the value it got $0.1$.

\paragraph{Conclusion def. \ref{altDef:1}:} An appropriate definition should consider the fact that an error in one objective might change the optimal value of other objectives.
As a consequence, another conclusion is that an appropriate definition should not consider the optimal solution at all.



\begin{potentialDefinition}\label{altDef:3}
    A solution $x$ is a $(1-\gamma)$-approximately optimal 
    if it satisfies the following requirements:
    \begin{enumerate}
        \item The objective-function with the smallest objective value achieves at least its maximum value times $(1-\gamma)$:
        \begin{align*}
            \valBy{1}{x} \geq (1-\gamma) \cdot \valBy{1}{x^*} 
        \end{align*}
        
        \item Given all the solutions that satisfies the first condition, let $m_2$ be the highest second-smallest objective value.
        The objective-function with the second-smallest objective value achieves at least the $m_2$ times $(1-\gamma)$.
        
        \item Given all the solutions that satisfies the former conditions, let $m_3$ be the highest third-smallest objective value.
        The objective-function with the third-smallest objective value achieves at least the $m_3$ times $(1-\gamma)$.
        
        \item and so on.
    \end{enumerate}
\end{potentialDefinition}

\paragraph{Bad example def. \ref{altDef:3}:}
Consider the following example with only two objectives:
\begin{align*}
    \max \quad &\{f_1(x) = x_1, f_2(x)=x_2\} \\
    s.t. \quad  & 99 x_1 + x_2 \leq 100\\
    & x \in \mathbb{R}^2_{+}
\end{align*}
The Leximin-optimal solution is $(1,1)$. Consider $\gamma = 0.1$, according to part (1) of this definition, all solutions in which the smallest objective value is at least $(1-\gamma)=0.9$ should be considered in order to determine $m_2$.
So, in this case, $m_2$ is determined to be $100 - 0.9 \cdot 99 = 10.9$.
Then, according to part (2), in order to be considered a $0.9$-approximately optimal, the second value must be at least $0.9 \cdot 10.9 = 9.81$.
But, even the exact Leximin optimal solution does not satisfy this requirement, so this cannot be considered an approximation to Leximin optimal.

In general, this definition has the disadvantage of favoring solutions that give the lowest bounds to the objective functions considered in the earlier steps,  since this may enable to increase the values of the higher objectives.
According to the Leximin nature, the most important thing is to make the worst-off player as happy as possible (and then the second worst-off and so on), therefore, we emphasize the importance of this characteristic also in the definition of the approximated version.

\paragraph{Conclusion def. \ref{altDef:3}:} An appropriate definition should also capture the Leximin optimal solutions, and maintain the Leximin nature whenever possible.

% \eden{I think this definition is actually equivalent to our current... need to think about it again}
% \begin{potentialDefinition}\label{altDef:4}
%     A solution $x$ is a $\gamma$-approximately-optimal Leximin solution if it can be viewed as the result of this process:
%     \begin{enumerate}
%         \item Choose a solution in which the objective-function with the smallest objective value achieves at least the maximum value minus $\gamma$:
%         \begin{align*}
%             \valBy{1}{x} \geq \valBy{1}{x^*} - \gamma
%         \end{align*}
%         Let $z_1$ be the value it achieves (i.e., $\valBy{1}{x})$.
        
%         \item Consider all the solutions in which the objective-function with the smallest objective value achieves at least $z_1$ and let $m_2$ be the highest second-smallest objective value.
%         Then, choose a solution in which the objective-function with the second-smallest objective value achieves at least the $m_2$ minus $\gamma$.
%         Let $z_2$ be the value it achieves.
%         \item Consider all the solutions in which the objective-function, the smallest objective value achieves at least $z_1$ and the second-smallest objective value achieves at least $z_2$, and let $m_3$ be the highest third-smallest objective value.
%         Then, choose a solution in which the objective-function with the third-smallest objective value achieves at least the $m_3$ minus $\gamma$.
        
%         \item and so on...
%     \end{enumerate}
% \end{potentialDefinition}

% \paragraph{Bad example def. \ref{altDef:3}:} Although in this definition, the Leximin optimal solution is also approximately-optimal as we wanted, another issue arises.

% \begin{itemize}
%     \item \textbf{Bad example:} two solutions that meet this definition, but one of them is strictly better (by more than $\gamma$) than the other from some point.
%     \item \textbf{Conclusion:} an appropriate (good?) definition should determine between two solutions if possible. 
% \end{itemize}

% -----------------------------------
% \subsection{others}
% (From the correspondence of Erel with Lemaitre and Bouveret)

%----------------------------------
% need to think about a corresponding def for mult...
\begin{potentialDefinition}\label{altDef:5}
    A solution $x$ is a $(1-\gamma)$-approximately optimal if for a Leximin-optimal solution, $x^*$, and for each $j = 1, \dots, n$: 
    % in the addive version was $$$
    \begin{align*}
        % |\valBy{j}{x} - \valBy{j}{x^*}| \leq \gamma\\
         \max\{\valBy{j}{y},\valBy{j}{x}\}  \leq \frac{1}{1-\gamma} \cdot \min\{\valBy{j}{y},\valBy{j}{x}\}
    \end{align*}
\end{potentialDefinition}

\paragraph{Bad example and conclusion def. \ref{altDef:5}:}
This definition is close to definition \ref{altDef:1} but weaker, still the same example and conclusion apply.

\begin{potentialDefinition}\label{altDef:6}
    A solution $x$ is a $(1-\gamma)$-approximately optimal if given a Leximin-optimal solution, $x^*$, there exists an integer $k \in [n]$ such that: 
    \begin{align*}
    \forall j < k: & \valBy{j}{x} = \valBy{j}{x^*}\\
    & \valBy{k}{x} > (1-\gamma) \cdot \valBy{k}{x^*}
    \end{align*}
\end{potentialDefinition}
% \eden{I'm not sure it is well defined, since by decreasing $\gamma$ (for example) the second value might become smaller than the first.}

\paragraph{Bad example def. \ref{altDef:6}:} As in the case of definition \ref{altDef:2}, by taking a small $k$, we cannot distinguish between two solutions that satisfy this definition, but one of them should be definitely preferred.
Consider again the following example with three objectives, where:
\begin{align*}
    \max \quad &\{f_i(x) = x_i \mid \forall 1 \leq i \leq 3 \} \\
    s.t. \quad  & 9 x_1 + x_2 \leq 10\\
    &  x_3 \leq 100\\
    & x \in \mathbb{R}^3_{+}
\end{align*}
The Leximin optimal solution $x^*$ is $(1,1, 100)$ and therefore, by taking $k$ to be $2$, we get that any solution that its minimum value is $1$ and its second-smallest objective value is more than $(1-\gamma)$ is considered $(1-\gamma)$-approximately optimal.
As an example, the solution $(1, 1, 1)$ is considered $(1-\gamma)$-approximately-optimal Leximin solution (as $(1-\gamma) < 1$).
But it is easy to see that this solution is quite bad for $f_3$ (who can achieve $100$).

\paragraph{Conclusion def. \ref{altDef:6}:} Same as for def. \ref{altDef:2}, an appropriate definition should take into account as many objectives as possible.

% \begin{potentialDefinition}
%     OWA.
% \end{potentialDefinition}

%--------------------------

\begin{potentialDefinition}\label{altDef:7}
    A solution $x$ is a $(1-\gamma)$-approximately optimal if there is no other solution $y$ that is $(1-\gamma)$-Leximin preferred over it, where this relation is defined as follows: $y$ is preferred over $x$ if  there exists an integer $k \in [n]$ such that:
    \begin{align*}
        \forall j < k \colon \quad &   \max\{\valBy{j}{y},\valBy{j}{x}\}  \leq \frac{1}{(1-\gamma)} \cdot \min\{\valBy{j}{y},\valBy{j}{x}\}\\
        & \valBy{k}{y} > \frac{1}{(1-\gamma)} \cdot 
\valBy{k}{x}
    \end{align*}
     [This relation is related to a one suggested in \cite{kalai_lexicographic_2012}, it is described in more detail in the Related work Section]. 
\end{potentialDefinition}

\paragraph{Bad example and conclusion def. \ref{altDef:7}:} As with definition \ref{altDef:3}, here also, the Leximin optimal solution is not optimal according to this relation and it might favor solutions with lower smallest objective values. 
Consider again the following example:
\begin{align*}
    \max \quad &\{f_1(x) = x_1, f_2(x)=x_2\} \\
    s.t. \quad  & 99 x_1 + x_2 \leq 100\\
    & x \in \mathbb{R}^2_{+}
\end{align*}
Assume that $\gamma = 0.1$, the Leximin-optimal solution is $(1,1)$, but the solution $(0.9,10.9)$ is preferred over it according to this relation (since for $k=2$ we get that $\max\{0.9,1\} \leq \frac{1}{0.9}\cdot\min\{0.9,1\}$ and $10.9 > \frac{1}{0.9} \cdot 1$) and therefore, it is not approximately-optimal.



\section{The Approximate Leximin Order}\label{sec:approx-order-is-strict-partial}

Unlike the leximin order, $\leximinPreferred$, which is a strict \textbf{total} order, the approximate leximin order, $\alphaBetaPreferred$ for $\DEFmultApprox\in (0,1]$ and $\DEFadditiveApprox \geq 0$ is a strict \textbf{partial} order.
The difference is that in partial orders, not all vectors are comparable.
Consider for example the sorted vectors $(1,2)$ and $(1, 3)$. 
According to the leximin order, $(1,3)$ is clearly preferred (as $3>2$), but according to many approximate leximin orders neither one is preferred over the other, for example according to the orders $\alphaBetaPreferredParams{0.6}{0}$,$ \alphaBetaPreferredParams{1}{1}$ or $\alphaBetaPreferredParams{0.8}{0.5}$.
% (irreflexive, asymmetric and transitive).

An order is a strict partial order if it is irreflexive, transitive and asymmetric.
Lemma \ref{lemma:order-is-irreflexive} proves that the order is irreflexive, Lemma \ref{lemma:order-is-transitive} proves it is transitive, and Lemma \ref{lemma:order-is-asymmetric} proves that it is asymmetric.
% \erel{It would be good to show an example why this is not a total order.}

% need to prove irreflexive, asymmetric (we have already proved that it is transitive).

% ***[I thought it would be better to prove it on vectors (rather than "solutions") to make it as general as possible]\\

Let $\DEFmultApprox\in (0,1]$ and $\DEFadditiveApprox \geq 0$. 

\begin{lemma}\label{lemma:order-is-irreflexive}
    The approximate leximin order $\alphaBetaPreferred$ is irreflexive.
\end{lemma}

\begin{proof}
    % \eden{I used $x$ only to remind the reader what irreflexive is, maybe it should simply be in the lemma description}
    Let $x$ be a solution. We will show that $x \nAlphaBetaPreferred x$.
    As the definition requires that one component be \emph{strictly greater} than the other, it is trivial.
\end{proof}

\begin{lemma}\label{lemma:order-is-transitive}
    The approximate leximin order $\alphaBetaPreferred$ is transitive.
\end{lemma}

\begin{proof}
    Let $x,y$ and $z$ be solutions such that $x \alphaBetaPreferred y$ and $y \alphaBetaPreferred z$.
    We will prove that $x \alphaBetaPreferred z$.

    
    Since $x \alphaBetaPreferred y$, there exists an integer $ k_1 \in [n]$ such that:
    \begin{align*}
        \forall j<k_1 \colon &  \valBy{j}{x} \geq \valBy{j}{y}\\
            & \valBy{k_1}{x} > \frac{1}{\DEFmultApprox} \left( \valBy{k_1}{y} + \DEFadditiveApprox \right)
    \end{align*}
    And since $y \alphaBetaPreferred z$, there exists an integer $k_2 \in [n]$ such that:
    \begin{align*}
        \forall j<k_2 \colon &  \valBy{j}{y} \geq \valBy{j}{z}\\
            & \valBy{k_2}{y} > \frac{1}{\DEFmultApprox} \left( \valBy{k_2}{z} + \DEFadditiveApprox \right) 
    \end{align*}

    As $\DEFmultApprox \in (0,1]$ and $\DEFadditiveApprox \geq 0$, it follows that:
    \begin{align}\label{eq:trans-k-s}
        \valBy{k_1}{x} > \valBy{k_1}{y}, \Hquad \valBy{k_2}{y} >  \valBy{k_2}{z}
    \end{align}

    % Accordingly, if $k_1=k_2$, then this integer, denoted by $k$, allows us to conclude that $x \alphaBetaPreferred z$. 
    % By the definitions of $k_1$ and $k_2$, for any $j<k_1=k_2$ the required holds as $\valBy{j}{x} \geq \valBy{j}{y} \geq \valBy{j}{z}$.
    % In addition, $\valBy{k_1}{x}> \valBy{k_1}{y}$ by equation \ref{eq:trans-k-s}
    % $ > \frac{1}{\DEFmultApprox} \left( \valBy{k_1}{y} + \DEFadditiveApprox \right)$ and nd  and 
    
    Let $k = \min\{k_1,k_2\}$.
    
    If $k = k_1$, by the definition of $k_1$, $\valBy{k}{x} > \frac{1}{\DEFmultApprox} \left( \valBy{k}{y} + \DEFadditiveApprox \right)$.
    However, $\valBy{k}{y} \geq \valBy{k}{z}$, by definition if $k<k_2$ and by equation \ref{eq:trans-k-s} if $k=k_2$. \ref{eq:transitive-k}
    Therefore, $\valBy{k}{x} > \frac{1}{\DEFmultApprox} \left( \valBy{k}{z} + \DEFadditiveApprox \right)$.
    
    Otherwise, if $k=k_2$, by the definition of $k_2$, $\valBy{k}{y} > \frac{1}{\DEFmultApprox} \left( \valBy{k}{z} + \DEFadditiveApprox \right)$. But, $\valBy{k}{x} \geq \valBy{k}{y}$, by definition if $k<k_1$ and by equation \ref{eq:trans-k-s} if $k=k_1$. Again, we can conclude that $\valBy{k}{x} > \frac{1}{\DEFmultApprox} \left( \valBy{k}{z} + \DEFadditiveApprox \right)$.

     In addition, for each $j<k$, since $j< k_1$ and $j < k_2$, by definition the following holds:
    \begin{align}\label{eq:transitive-k}
        \valBy{j}{x} \geq \valBy{j}{y} \geq \valBy{j}{z}
    \end{align}
    So, $k$ is an integer that satisfy all the requirements, and so, $x \alphaBetaPreferred z$.
    \end{proof}
    

    
    \begin{lemma}\label{lemma:order-is-asymmetric}
        The approximate leximin order $\alphaBetaPreferred$ is asymmetric.
    \end{lemma}
    
    \begin{proof}
        Let $x$ and $y$ be solutions such that $x \alphaBetaPreferred y$. We will show that $y \nAlphaBetaPreferred x$. 
        Assume by contradiction that $y \alphaBetaPreferred x$. 
        From Lemma \ref{lemma:order-is-transitive}, this relation is transitive. Therefore, since $x \alphaBetaPreferred y$ and $y \alphaBetaPreferred x$, also $x \alphaBetaPreferred x$.
        But, from Lemma \ref{lemma:order-is-irreflexive}, this relation is irreflexive --- a contradiction.
    \end{proof}
\section{Proof of Theorem \ref{th:main}}\label{sec:algo-sec-proofs}
\eden{should probably change the title}

This section is dedicated to proving Theorem \ref{th:main}.
To this end, we use another equivalent representation of \eqref{eq:sums-OP}, which was also introduced by \cite{Ogryczak_2006} 
(we provide the proof of equivalence in Appendix \ref{sec:equivalent-proofs}). 
\erel{Can't we just use it directly instead of P2?}
% (here also, the variables are $\ztVar{x}$ and $x$, and $z_1, \ldots z_{t-1}$ are constants)
\begin{align*}
    \max \quad &z_t \tag{P2-compact}\label{eq:compact-OP} \;\;
        s.t. &\quad  & (1) \quad x \in S\\
                    &&& (\Tilde{2}) \quad \sum_{i=1}^{\ell} \valBy{i}{x} \geq \sum_{i=1}^{\ell}  z_i && \ell = 1,\ldots, t-1 \nonumber\\
                    &&& (\Tilde{3}) \quad \sum_{i=1}^{t} \valBy{i}{x} \geq \sum_{i=1}^{t}  z_i
\end{align*}
In this problem, constraints $(\hat{2})$ and $(\hat{3})$ are replaced by  $(\Tilde{2})$ and $(\Tilde{3})$, respectively.  
The difference is that
$(\hat{2})$ gives, for each $\ell$, a lower bound on the sum for \emph{any} set of $\ell$ objective functions; whereas $(\Tilde{2})$ only considers the sum of the $\ell$ \emph{smallest} such values.  
% However, since the constraints set the same lower bound on this sum, the constraints are equivalent.  
Similarly for $(\hat{3})$ and $(\Tilde{3})$. 
Since  the problems are equivalent, a solver, either exact or approximate, for one can be used as a solver, with the same level of accuracy, for the other (Lemma \ref{lemma:solver-equivalent-prob}). 
Therefore, as \eqref{eq:compact-OP} is equivalent to \eqref{eq:sums-OP}, which, in turn, is equivalent to \eqref{eq:vsums-OP}, in proving the theorem we may assume that \textsf{OP} is an approximation procedure for \eqref{eq:compact-OP}.  
This will simplify the proofs. \eden{I added the line from the comment back, isn't it important to explain why we need this representation?}

% \erel{*** I do not understand. We say that P1 and P2 are equivalent with an exact solver, but not with an approximate solver. Here, we claim that P3 and P2-compact are equivalent, but this is true only with an exact solver. Don't we have to prove that they are equivalent also with an approximate solver? ***}

We denote $\retSol := x_n$ = the solution $x$ attained at the last iteration ($t=n$) of the algorithm. 

Following are some observations regarding the set of feasible solutions in each iteration, their objective values, and the solution $\retSol$ that will be useful later on.

% For any constants $z_1,\ldots, z_{t-1}$,
% any vector $x \in S$ that satisfies constraint $(\Tilde{2})$ of \eqref{eq:compact-OP} 
% is feasible to this problem.
% This is because any solution $x \in S$ can satisfy constraint $(\Tilde{3})$ with a small enough assignment to the variable $z_t$. \eden{I'm not sure how to explain it....}
\begin{observation}\label{obs:feasi-and-constraint2}
For any constants $z_1,\ldots, z_{t-1}$,
any vector $x \in S$ that satisfies constraint $(\Tilde{2})$ of \eqref{eq:compact-OP} 
can be a part of a feasible solution $(x,z_t)$ for any $z_t \leq \sum_{i=1}^{t} \valBy{i}{x} - \sum_{i=1}^{t-1} z_i$.
\end{observation}

Since $\retSol$ is a feasible solution of \eqref{eq:compact-OP} in iteration $n$, and as each
iteration only adds new constraints to $(\Tilde{2})$, it follows that $\retSol$ is also a feasible solution of \eqref{eq:compact-OP} in any iteration $1 \leq t\leq n$. 
\begin{observation}\label{obs:retSol-solves-any-t}
$\retSol$ is a feasible solution of \eqref{eq:compact-OP} in any iteration $1 \leq t\leq n$.
\end{observation}

Now, consider the problem \eqref{eq:compact-OP} that was solved in iteration $t$.
Here, $z_t$ is a \emph{variable} and $z_1, \ldots z_{t-1}$ are constants.
The objective of this problem is $\max z_t$, and the only constraint that includes the variable $z_t$ is  $(\Tilde{3})$.
Therefore, rearranging it to $\sum_{i=1}^{t} \valBy{i}{x} - \sum_{i=1}^{t-1}  z_i\geq z_t$, allows us to conclude that the objective value is determined by the left side of this inequality (as $z_t$ is maximized when the inequality turns to equality).
\begin{observation}\label{obs:obj-value}
The objective value obtained by a feasible solution $x$ to the problem \eqref{eq:compact-OP} that was solved in iteration $t$ is $\sum_{i=1}^{t} \valBy{i}{x} - \sum_{i=1}^{t-1}  z_i$.
\end{observation}

Lastly, as the value obtained as a $(\multApprox, \additiveApprox)$-approximation for this problem is the \emph{constant} $z_t$, the optimal value is at most $\frac{1}{\multApprox} (z_t+\additiveError)$. 
Consequently, the objective value of any feasible solution is at most this value.
Since $\retSol$ is feasible for any iteration $t$ (Observation \ref{obs:retSol-solves-any-t}) and since its objective is $\sum_{i=1}^t \valBy{i}{\retSol} - \sum_{i=1}^{t-1} z_i$ (Observation \ref{obs:obj-value}), we can conclude:

\begin{observation}\label{obs:obj-xt-to-zt}
    The objective value obtained by $\retSol$ to the problem \eqref{eq:compact-OP} that was solved in iteration $t$ is at most $\frac{1}{\multApprox} (z_t+\additiveError)$. That is:
    \begin{align*}
        \sum_{i=1}^t \valBy{i}{\retSol} - \sum_{i=1}^{t-1} z_i \leq \frac{1}{\multApprox} \left(z_t+\additiveError \right).
    \end{align*}
\end{observation}

% This conclusion also implies that for any $1 \leq t \leq n$, the solution $(x_t, z_t)$ that that was outputted for \eqref{eq:compact-OP} in iteration $t$, satisfies constraint $(\Tilde{3})$ as equality. That is:
% \begin{observation}\label{obs:equality-xt-zt}
% For any $1 \leq t \leq n$,  $\sum_{i=1}^{t} \valBy{i}{x_t} = \sum_{i=1}^{t}  z_i$.
% \end{observation}



%%%
% OVERALL EXPLANATION 
We start with Lemmas \ref{lemma:beta-vk}-\ref{lemma:fk-to-all}, which establish a relationship between the $k$-th least objective value obtained by $\retSol$ 
% ($\valBy{k}{\retSol}$) 
and the difference between the sum of the $(k-1)$ least objective values obtained by $\retSol$ and the sum of the $(k-1)$ first $z_i$ values.
% ($\sum_{i=1}^{k-1}\valBy{k}{\retSol} - \sum_{i=1}^{k-1}z_i$). 
Theorem \ref{th:main} then uses this relation to prove that the existence of another solution that would be $\left(\frac{\multApprox^2}{1-\multApprox + \multApprox^2}, \frac{\multApprox(2-\multApprox)\additiveApprox}{1-\multApprox +\multApprox^2}\right)$-preferred over $\retSol$ would lead to a contradiction.

For clarity, throughout the proofs, we denote the multiplicative error factor by $\multError = 1-\multApprox$.

% LEMMAS.
% BLAH BLAH.

\begin{lemma}\label{lemma:beta-vk}
    For all $k\in[n]$, 
    \begin{align*}
        \multError \valBy{k}{\retSol} \geq \left(\sum_{i=1}^k \valBy{i}{\retSol} - \sum_{i=1}^k z_i\right) -\multError \left(\sum_{i=1}^{k-1} \valBy{i}{\retSol} - \sum_{i=1}^{k-1} z_i\right) -\additiveError
    \end{align*}
\end{lemma}

\begin{proof}
By Observation \ref{obs:obj-xt-to-zt},
    \begin{align*}
         &\sum_{i=1}^k \valBy{i}{\retSol} - \sum_{i=1}^{k-1} z_i \leq \frac{1}{\multApprox} \left(z_k + \additiveError \right) = \frac{1}{1-\multError} \left(z_k + \additiveError \right)\\
         &\Rightarrow z_k +\additiveError \geq (1-\multError) \left(\sum_{i=1}^{k} \valBy{i}{\retSol} - \sum_{i=1}^{k-1}  z_i\right)\\
        &\Rightarrow z_k +\additiveError\geq \left(\sum_{i=1}^{k} \valBy{i}{\retSol} - \sum_{i=1}^{k-1}  z_i\right) - \multError \left(\sum_{i=1}^{k} \valBy{i}{\retSol} - \sum_{i=1}^{k-1}  z_i\right)\\
        &\Rightarrow \multError \valBy{k}{\retSol} \geq \left(\sum_{i=1}^k \valBy{i}{\retSol} - \sum_{i=1}^k z_i\right) -\multError \left(\sum_{i=1}^{k-1} \valBy{i}{\retSol} - \sum_{i=1}^{k-1} z_i\right) -\additiveError.
        \qedhere
    \end{align*}
\end{proof}


\begin{lemma}\label{lemma:beta-sums-to-diff}
    For all $k\in[n]$, 
    \begin{align*}
        \sum_{i=1}^k \multError^{i} \valBy{k-i+1}{\retSol} \geq \sum_{i=1}^k \valBy{i}{\retSol} - \sum_{i=1}^{k} z_i -\additiveError
    \end{align*}
\end{lemma}

\begin{proof}
    The proof is by induction on $k$.
    For $k=1$ the claim follows directly from Lemma \ref{lemma:beta-vk}.
    Assuming the claim is true for $1,\ldots k-1$, we show it is true for $k$:
    \begin{align*}
        &\sum_{i=1}^k \multError^{i} \valBy{k-i+1}{\retSol} = \multError \valBy{k}{\retSol} + \sum_{i=2}^k \multError^{i} \valBy{k-i+1}{\retSol}\\
        &= \multError \valBy{k}{\retSol} + \sum_{i=1}^{k-1} \multError^{i+1} \valBy{k-(i+1)+1}{\retSol} \\
        &= \multError \valBy{k}{\retSol} + \multError \sum_{i=1}^{k-1} \multError^{i} \valBy{(k-1) -i+1}{\retSol}\\
        &= \multError \valBy{k}{\retSol} + \multError \left(\sum_{i=1}^{k-1} \valBy{i}{\retSol} - \sum_{i=1}^{k-1} z_i\right) && \text{(by induction assumption)}\\
        &\geq \left(\sum_{i=1}^k \valBy{i}{\retSol} - \sum_{i=1}^k z_i\right) -\multError \left(\sum_{i=1}^{k-1} \valBy{i}{\retSol} - \sum_{i=1}^{k-1} z_i\right)-\additiveError  \\
        & \quad +  \multError \left(\sum_{i=1}^{k-1} \valBy{i}{\retSol} - \sum_{i=1}^{k-1} z_i\right) && \text{(by Lemma \ref{lemma:beta-vk})} \\
        &= \sum_{i=1}^k \valBy{i}{\retSol} - \sum_{i=1}^{k} z_i -\additiveError.
        \qed
    \end{align*}
\end{proof}


\begin{lemma}\label{lemma:fk-to-all}
    For all $1<k \leq n$, 
    \begin{align*}
        \frac{\multError}{1-\multError} \valBy{k}{\retSol} \geq \sum_{i=1}^{k-1}\valBy{i}{\retSol} - \sum_{i=1}^{k-1}z_i - \additiveError
    \end{align*}
\end{lemma}

\begin{proof}
    First, notice that since $k \geq (k-1)-i+1$ for any $1\leq i \leq k$ and as the function $\valBy{i}$ represents the $i$-th smallest objective value, also:
    \begin{align}\label{eq:increase-by-obj-size}
        \forall 1\leq i \leq k \colon \quad \valBy{k}{\retSol} \geq \valBy{(k-1)-i+1}{\retSol}
    \end{align}
    In addition, consider the geometric series with a first element $1$, a ratio $\multError$, and a length $(k-1)$. 
    As $\multError < 1$, its sum can be bounded in the following way:
    \begin{align}\label{eq:geometric-series-beta}
        \sum_{i=1}^{k-1} \multError^{i-1} = \frac{1-\multError^{k-1}}{1-\multError} < \lim_{k \to \infty}\frac{1-\multError^{k-1}}{1-\multError} = \frac{1}{1-\multError}
    \end{align}
    
    Now, the claim can be concluded as follows:
    \begin{align*}
        & \frac{\multError}{1-\multError}\valBy{k}{\retSol} = \multError \left(\frac{1}{1-\multError} \valBy{k}{\retSol} \right)\\
        & > \multError \left(\sum_{i=1}^{k-1} \multError^{i-1} \valBy{k}{\retSol} \right) && \text{(by Equation \eqref{eq:geometric-series-beta})}\\
        & \geq  \multError \left(\sum_{i=1}^{k-1} \multError^{i-1} \valBy{(k-1)-i+1}{\retSol} \right) && \text{(by Equation \eqref{eq:increase-by-obj-size})}\\
        &= \sum_{i=1}^{k-1} \multError^{i} \valBy{(k-1)-i+1}{\retSol} \\
        &\geq \sum_{i=1}^{k-1}\valBy{i}{\retSol} - \sum_{i=1}^{k-1}z_i - \additiveError && \text{(by Lemma \ref{lemma:beta-sums-to-diff})}
\end{align*}
\erel{Formally, Lemma \ref{lemma:beta-sums-to-diff} is for $k\geq 1$, and we apply it for $k-1$, which might be $0$.}\eden{I tried to fixed it, is it better?}
\end{proof}



%------
% thm.

We are now ready to prove the Theorem \ref{th:main}.
\begin{proof}[Proof of Theorem \ref{th:main}]
% \eden{I'm not sure if we should write again about the claim with $\multApprox$}
Recall that the claim is that $\retSol$ is a $\left(\frac{\multApprox^2}{1-\multApprox + \multApprox^2}, \frac{\multApprox(2-\multApprox)\additiveApprox}{1-\multApprox +\multApprox^2}\right)$-approximation.

For brevity, we define the following constants:
\begin{align*}
    \Delta^{mult} = \frac{\multApprox}{1-\multApprox + \multApprox^2}, \quad  \Delta^{add} = \frac{\multApprox(2-\multApprox)}{1-\multApprox +\multApprox^2}
\end{align*}
Accordingly, we need to prove that $\retSol$ is a $\left(\Delta^{mult} \cdot \multApprox, \Delta^{add}\cdot\additiveApprox\right)$-approximation.

We prove the following equation, that will be helpful later:
\begin{align}\label{equ:mu}
\frac{1}{\Delta^{mult} \cdot \multApprox} = \frac{1-\multError +\multError^2}{(1-\multError)^2}
\end{align}
This is true because
\begin{align*}
    &\Delta^{mult} \cdot \multApprox =   \frac{\multApprox^2}{1-\multApprox + \multApprox^2} && \text{(Definition of $\Delta^{mult}$)} \\
    &= \frac{(1-\multError)^2}{\multError +(1-\multError)^2} = \frac{(1-\multError)^2}{1-\multError +\multError^2} &&\text{(since $\multApprox = 1-\multError$)}\\
    & \Rightarrow \frac{1}{\Delta^{mult} \cdot \multApprox} = \frac{1-\multError +\multError^2}{(1-\multError)^2}
    \end{align*}
    Another equation that will be useful later is:
    \begin{align}\label{eq:additive-error}
        \frac{\Delta^{add}}{\Delta^{mult}\cdot \multApprox}  = \frac{1+\multError}{1-\multError}.
    \end{align}
    The reason for this is that
    \begin{align*}
        &\frac{\Delta^{add}}{\Delta^{mult}\cdot \multApprox} =\frac{1-\multApprox + \multApprox^2}{\multApprox^2} \cdot \frac{\multApprox(2-\multApprox)}{1-\multApprox +\multApprox^2}&& \text{(Definitions of $\Delta^{mult}$ and $\Delta^{add}$)}\\
        &=\frac{\multApprox(2-\multApprox)}{\multApprox^2} = \frac{(1-\multError)(1 + \multError)}{(1-\multError)^2} =\frac{1+\multError}{1-\multError}  &&\text{(since $\multApprox = 1-\multError$)}
    \end{align*}

    Now, suppose by contradiction that $\retSol$ is \emph{not} $\left(\Delta^{mult} \cdot \multApprox, \Delta^{add}\cdot\additiveApprox\right)$-approximately-optimal.
    By definition, this means there exists a solution $y \in S$  that is $\left(\Delta^{mult} \cdot \multApprox, \Delta^{add}\cdot\additiveApprox\right)$-preferred over it.
    That is, there exists an integer $1 \leq k \leq n$ such that:
    \begin{align*}
        \forall j < k \colon &\valBy{j}{y} \geq \valBy{j}{\retSol};\\
        & \valBy{k}{y} > \frac{1}{\Delta^{mult} \cdot\multApprox} \left(\valBy{k}{\retSol} + \Delta^{add} \cdot\additiveError \right).
    \end{align*}

    Since $\retSol$ was obtained in \eqref{eq:compact-OP} that was solved in the last iteration $n$, it is clear that $\sum_{i=1}^k \valBy{i}{\retSol} \geq \sum_{i=1}^{k} z_i$ (by constraint $(\Tilde{2})$ if $k<n$ and $(\Tilde{3})$ otherwise).
    Which implies:
    \begin{align}\label{eq:fk-to-zk}
        \sum_{i=1}^k \valBy{i}{\retSol} - \sum_{i=1}^{k-1} z_i \geq z_k
    \end{align}

    Now, consider \eqref{eq:compact-OP} that was solved in iteration $k$.
    By Observation \ref{obs:retSol-solves-any-t}, $\retSol$ is feasible to this problem.
    As the $(k-1)$ smallest objective values of $y$ are at least as high as those of $\retSol$, it is easy to conclude that $y$ also satisfies constraints $(\Tilde{2})$ of this problem; since, for any $\ell < k$:
    \begin{align*}
        \sum_{i=1}^{\ell} \valBy{i}{y} \geq\sum_{i=1}^{\ell} \valBy{i}{\retSol} \geq \sum_{i=1}^{\ell} z_i
    \end{align*}
    Therefore, by Observation \ref{obs:feasi-and-constraint2}, $y$ is also feasible to this problem. 

    If $k=1$, the objective value $y$ in this problem is $\valBy{1}{y}$ (Observation \ref{obs:obj-value}).
    In addition, $\valBy{1}{\retSol} \geq z_1$ by equation \ref{eq:fk-to-zk}. As $\Delta^{mult}\geq 0$ and $\Delta^{add}\geq 0$, it follows that:
    \begin{align*}
        \valBy{1}{y}> \frac{1}{\Delta^{mult} \cdot\multApprox} \left(\valBy{1}{\retSol} + \Delta^{add} \cdot\additiveError \right)\geq \frac{1}{\multApprox} \left(z_1 + \additiveError \right)
    \end{align*}
    But, $z_1$ was obtained as an approximation for this problem, therefore the optimal value is at most $\frac{1}{\multApprox}\left(z_1 + \additiveError \right)$ --- a contradiction.

    
    Otherwise, $k>1$, we shall now see that in this case $y$ also satisfies the following:
    \begin{align}\label{eq:yk-to-sum}
        \valBy{k}{y} > \frac{1}{1-\multError} \valBy{k}{\retSol} + \frac{\multError}{1-\multError}\sum_{i=1}^{k-1}\valBy{i}{\retSol} - \frac{\multError}{1-\multError} \sum_{i=1}^{k-1}z_i  +\frac{1}{1-\multError}\cdot\additiveError
    \end{align}
    this is true because
    \begin{align*}
        &\valBy{k}{y} > \frac{1}{ \Delta^{mult} \cdot\multApprox} \left(\valBy{k}{\retSol} + \Delta^{add}\cdot \additiveError \right) && \text{(Definition of $y$ for $k$)}\\
        &= \frac{1-\multError +\multError^2}{(1-\multError)^2} \valBy{k}{\retSol}+ \frac{\Delta^{add}}{\Delta^{mult} \multApprox}\cdot\additiveError && \text{(by Equation \ref{equ:mu})}\\
        &= \frac{1-\multError +\multError^2}{(1-\multError)^2} \valBy{k}{\retSol}+ \frac{1+\multError}{1-\multError}\cdot\additiveError && \text{(by Equation \ref{eq:additive-error})} \erel{???}\\
        &\geq\frac{1}{1-\multError} \valBy{k}{\retSol} + \frac{\multError}{1-\multError}\left(\sum_{i=1}^{k-1}\valBy{i}{\retSol} - \sum_{i=1}^{k-1}z_i-\additiveError\right) +\frac{1+\multError}{1-\multError}\cdot\additiveError && \text{(by Lemma \ref{lemma:fk-to-all} for $k>1$)}\\
        & = \frac{1}{1-\multError} \valBy{k}{\retSol} +\frac{\multError}{1-\multError}\sum_{i=1}^{k-1}\valBy{i}{\retSol} - \frac{\multError}{1-\multError} \sum_{i=1}^{k-1}z_i +\frac{1}{1-\multError}\cdot\additiveError &&\erel{???}\text{\eden{is it more clear?}}
    \end{align*}    
    
    We compute the objective value of $y$, which is $\sum_{i=1}^k \valBy{i}{y} - \sum_{i=1}^{k-1} z_i$ (by Observation \ref{obs:obj-value}):  
    \begin{align*}
        &\sum_{i=1}^k \valBy{i}{y} - \sum_{i=1}^{k-1} z_i=\sum_{i=1}^{k-1} \valBy{i}{y} - \sum_{i=1}^{k-1} z_i + \valBy{k}{y}\\
        &\geq \sum_{i=1}^{k-1} \valBy{i}{\retSol} - \sum_{i=1}^{k-1} z_i + \valBy{k}{y} && \text{(Definition of $y$ for $j<k$)}\\
        &> \sum_{i=1}^{k-1} \valBy{i}{\retSol} - \sum_{i=1}^{k-1} z_i + \frac{1}{1-\multError} \valBy{k}{\retSol} \\
        & \quad + \frac{\multError}{1-\multError}\sum_{i=1}^{k-1}\valBy{i}{\retSol} - \frac{\multError}{1-\multError}\sum_{i=1}^{k-1}z_i +\frac{1}{1-\multError}\cdot\additiveError && \text{(by Equation \ref{eq:yk-to-sum})}\\
        & = \frac{1}{1-\multError} \left(\sum_{i=1}^k \valBy{k}{\retSol} - \sum_{i=1}^{k-1}z_i + \additiveError\right) &&\text{(since  $1+\frac{\multError}{1-\multError} = \frac{1}{1-\multError}$)}\erel{???}\text{\eden{is it more clear?}}
        \\
        &\geq \frac{1}{1-\multError} \left(z_k +\additiveError\right) && \text{(by Equation \ref{eq:fk-to-zk}) }
    \end{align*}
    \eden{I'm not sure why to comment the lines, shouldn't we explain why it is a contradiction?how is the following?}
    % \emark{However, the approximately-optimal solution obtained for this problem during the algorithm run is $z_k$, so the optimal value is at most $\frac{1}{(1-\multError)}\left(z_k+\additiveError\right)$.
    % But, as we shall see, the objective $y$ yields in this problem, $\sum_{i=1}^k \valBy{i}{y} - \sum_{i=1}^{k-1} z_i$ (by Observation \ref{obs:obj-value}), is higher than this value, which is of course a contradiction:}
    However, the approximately-optimal value obtained for this problem during the algorithm run is $z_k$, so the optimal value is at most $\frac{1}{(1-\multError)}\left(z_k+\additiveError\right)$, which is, again, a contradiction.
    
\end{proof}

\section{Proof of Theorem \ref{th:app-main}}\label{sec:app-sec-proofs}
% \eden{should probably change the title}

% Agents are assumed to care only about their own share (allowing us to use the following abuse of notation in which $u_j$ takes a bundle $b$ of items), their utilities are assumed to be normalized ($u_j(\emptyset) = 0$), monotone ($u_j(b_1) \leq u_j(b_2)$ if $b_1 \subseteq b_2$), and submodular ($u_j(b_1) + u_j(b_2) \geq u_j(b_1 \cup b_2) + u_j(b_1 \cap b_2)$ for any bundles $b_1,b_2$).
% It is assumed that each agent assigns a positive utility to the set of all items.
% The utilities $(u_i)_{i=1}^n$ are assumed to be given in the \emph{value oracle model}, meaning that we do not have a direct access to them, but only to an oracle that indicates the value of an agent from a given simple allocation.
% % \eden{z1 > 0}

This section proves Theorem \ref{th:app-main}:
suppose we are given a randomized algorithm that returns a simple allocation that approximates the utilitarian welfare with multiplicative error $\multError$ (with success probability $p$).
Then, Algorithm \ref{alg:basic-ordered-Outcomes} can be used to obtain a stochastic allocation that approximates leximin with a multiplicative error of at most $\frac{\multError}{1-\multError +\multError^2}$ (with the same probability).

% title: the specific problem as P3
As we saw in Section \ref{sec:algo-short}, an approximation to leximin can be obtained by providing a procedure \textsf{OP} to approximate \eqref{eq:vsums-OP}  (Theorem \ref{th:main}), which, under these particular settings, becomes:
% \erel{Why do you call it "configuration LP"? I think this term refers to something else: \url{https://en.wikipedia.org/wiki/Configuration_linear_program}}
\begin{align}
&\max \quad z_t \quad s.t. \tag{\progAppFirst}\label{eq:app-vsums-OP}\\
& (\text{\progAppFirst.1.1}) \Hquad \sum_{A \in \mathcal{A}} p_d(A) = 1 \nonumber\\
& (\text{\progAppFirst.1.2}) \Hquad p_d(A) \geq 0  && \forall A \in \mathcal{A} \nonumber\\
& (\text{\progAppFirst.2}) \Hquad \ell y_{\ell} - \sum_{j=1}^n m_{\ell,j}\geq \sum_{i=1}^{\ell}  z_i && \forXinY{\ell}{t-1} \nonumber \\
& (\text{\progAppFirst.3}) \Hquad t y_t - \sum_{j=1}^{n} m_{t,j} \geq \sum_{i=1}^{t}  z_i \nonumber \\
& (\text{\progAppFirst.4}) \Hquad m_{\ell,j} \geq y_{\ell} - \sum_{A \in \mathcal{A}}p_d(A) \cdot u_j(A)  && \forXinY{\ell}{t},\Hquad \forXinY{j}{n} \nonumber \\
& (\text{\progAppFirst.5}) \Hquad m_{\ell,j} \geq 0  && \forXinY{\ell}{t},\Hquad \forXinY{j}{n} \nonumber
\end{align}
Here the variables are $p_d(A)$ for any simple allocation $A \in \mathcal{A}$, $\ztVar{}$, and $y_{\ell}$ and $m_{\ell,j}$ for all $\ell \in [t]$ and $ j\in [n]$; and the values $z_1, \ldots z_{t-1}$ are constants.
Notice that it is a \emph{linear program} that has a polynomial number of constraints thanks to \eqref{eq:vsums-OP} representation, but an exponential number of variables (since there is a variable $p_d(A)$ for each simple allocation).
So, it is unclear how to approach it directly in polynomial time.
% \eden{here?}
In addition, it means that the output size is exponential in $n$.
To deal with this issue, the solutions are considered in \emph{sparse form} --- a list of the variables with positive values, along with their values.
Accordingly, if a solution has only a polynomial number of variables with positive values it can be represented by a polynomial size.
We will later see that the procedure described in this section returns such a solution in polynomial time.
% \eden{should write something about the output size, as \cite{kawase_max-min_2020}}

% title: baseline
% \erel{I would move the following paragraph upwards}
With $t=1$, \eqref{eq:app-vsums-OP} can be viewed as the problem of egalitarian welfare maximization, indeed, Kawase and Sumita \cite{kawase_max-min_2020} who studied this problem, considered a slightly simpler representation. 
% After proving that approximating the optimal value to a factor better than $(1-\frac{1}{e})$ is NP-hard, they present a dual-based algorithm that achieves this accuracy \er{w.h.p (?)}.
We now show how their dual-based technique can be applied to approximate \eqref{eq:app-vsums-OP} for any $t\geq 1$ while maintaining the same approximation factor.


To begin, consider the following program \eqref{eq:app-ver2-vsums-OP}, which is the result of modifying \eqref{eq:app-vsums-OP} in three ways. 
First, changing the objective-function to $\min 1/z_t$ instead of $\max z_t$. 
Second, replacing all the original variables and constants, except $z_t$, with new ones that are smaller by a factor $z_t$ (that is, $p'_A = p_d(A)/z_t$ for all $A \in \mathcal{A}$, $,y'_{\ell} = y_{\ell}/z_t,m'_{\ell,j} = m_{\ell,j}/z_t$ for $\ell \in [t]$ and $ j\in [n]$,  and $z'_i = z_i/z_t$ for $i \in [t-1]$).
And third, dividing all the constraints by $z_t$ ($z_t > 0$ since $z_t \geq z_1$ for any $t \geq 1$ and  $z_1 >0$).
\eden{to myself: maybe to explain why $z_1>0$}

\begin{align}
& \min \quad 1/z_t \quad s.t. \tag{\progAppSecond}\label{eq:app-ver2-vsums-OP}\\
& (\text{\progAppSecond.1.1}) \Hquad \sum_{A \in \mathcal{A}} p'_A = 1/z_t \nonumber\\
& (\text{\progAppSecond.1.2}) \Hquad p'_A \geq 0  && \forall A \in \mathcal{A} \nonumber\\
& (\text{\progAppSecond.2}) \Hquad \ell y'_{\ell} - \sum_{j=1}^n m'_{\ell,j}\geq \sum_{i=1}^{\ell}  z'_i && \forXinY{\ell}{t-1} \nonumber \\
& (\text{\progAppSecond.3}) \Hquad t y'_t - \sum_{j=1}^{n} m'_{t,j} \geq \sum_{i=1}^{t-1}  z'_i + 1 \nonumber \\
& (\text{\progAppSecond.4}) \Hquad m'_{\ell,j} \geq y'_{\ell} - \sum_{A \in \mathcal{A}}p'_A \cdot u_j(A)  && \forXinY{\ell}{t},\Hquad \forXinY{j}{n} \nonumber \\
& (\text{\progAppSecond.5}) \Hquad m'_{\ell,j} \geq 0  && \forXinY{\ell}{t},\Hquad \forXinY{j}{n} \nonumber
\end{align}
The programs \eqref{eq:app-vsums-OP} and \eqref{eq:app-ver2-vsums-OP} are related in the following way:
% \erel{I would make this a lemma:}
\begin{lemma}\label{lemma:bijection}
There exists a bijection mapping each solution of 
\eqref{eq:app-vsums-OP} with objective value $V$ to a unique solution of 
\eqref{eq:app-ver2-vsums-OP} with objective value $1/V$.
\end{lemma}
\begin{proof}
Let $p_d(A)$ for $A \in \mathcal{A}$, $\ztVar{}$, and $y_{\ell}$ and $m_{\ell,j}$ for all $\ell \in [t]$ and $ j\in [n]$ be a feasible solution to the program \eqref{eq:app-vsums-OP} with objective value $V$.
It can be easily verified that $p'_A = p_d(A)/z_t$ for $A \in \mathcal{A}$, $z_t$, and $y'_{\ell} = y_{\ell}/z_t$ and $m'_{\ell,j} = m_{\ell,j}/z_t$ for all $\ell \in [t]$ and $ j\in [n]$ is a feasible solution to the program \eqref{eq:app-ver2-vsums-OP} with objective value $1/V$.
\end{proof}
% \eden{maybe to write something about why it is a bijection (or to write that it is straightforward)}

Denote this bijection by $\Psi$, this also implies the following:
\begin{lemma}\label{lemma:approx-acc-by-bijection}
    If a solution approximates the program \eqref{eq:app-ver2-vsums-OP} with a multiplicative error of $\frac{\multError}{1-\multError}$. Then the corresponding solution to \eqref{eq:app-vsums-OP} according to the bijection $\Psi$ approximates this program with a multiplicative error of $\multError$.
\end{lemma}

\begin{proof}
    Let $V^*$ be the optimal objective value of \eqref{eq:app-vsums-OP}. 
    By Lemma \ref{lemma:bijection}, there exists a solution to \eqref{eq:app-ver2-vsums-OP} with value $1/V^{*}$.
    This solution yields the optimal value for \eqref{eq:app-ver2-vsums-OP} --- if there was a solution that had a value \emph{lower} than $1/V^{*}$ (\eqref{eq:app-ver2-vsums-OP} is a minimization problem), then the corresponding solution to \eqref{eq:app-vsums-OP} (by the bijection $\Psi$) would have a value higher than the optimal value $V^*$.
    Now, let the value of the solution that approximates the program \eqref{eq:app-ver2-vsums-OP} with a multiplicative error of $\frac{\multError}{1-\multError}$ be $1/V$. 
    Since \eqref{eq:app-ver2-vsums-OP} is a minimization problem, assuming that $1/V$ approximates $1/V^*$ with a multiplicative error of $\frac{\multError}{1-\multError}$ means that:
    \begin{align*}
        \frac{1}{V} \leq \left(1+\frac{\multError}{1-\multError}\right)\frac{1}{V^*},
    \end{align*}
 which implies that $V \geq (1-\multError)V^*$.
    As \eqref{eq:app-vsums-OP} is a maximization problem, this means that $V$ approximates this problem with multiplicative error $\multError$.
    By Lemma \ref{lemma:bijection}, $V$ is the value of the corresponding solution to \eqref{eq:app-vsums-OP} by the bijection $\Psi$.
\end{proof}

Notice that the only constraint of \eqref{eq:app-ver2-vsums-OP} that includes the variable $z_t$, (\progAppSecond.1.1), says that $\sum_{A \in \mathcal{A}}p'_A = 1/z_t$, and also that its objective function is $\min 1/z_t$.
As a result, we can reduce the need for the variable $z_t$ by removing constraint (\progAppSecond.1.1) and changing the objective function to $\min \sum_{A \in \mathcal{A}}p'_A$.
This change makes \eqref{eq:app-ver2-vsums-OP} a \emph{linear} program.
This will allow us to approximate it using its dual, as we will see.

The following observation will be useful later:
\begin{observation}\label{obs:c2-to-c1-in-poly-time}
    If a solution to \eqref{eq:app-ver2-vsums-OP} is given in a sparse form --- a list of the variables with nonzero value and their values, then the corresponding solution to \eqref{eq:app-vsums-OP} in a sparse form can be computed in time polynomial to the number of nonzero variables.
\end{observation}
\noindent For completeness, we briefly outline the process. 
When given a list of variables with nonzero values, we first iterate the list and sum all variables of the form $p'_A$, and then set $z_t$ to be $1$ divided by this sum. 
After, for each variable $\nu'$ in the list, we set the corresponding variable, $\nu$, to $z_t \cdot \nu'$.


% title: dual 
Now, let us consider the dual program of \eqref{eq:app-ver2-vsums-OP}, which can be described as follows:
% \erel{When you present an LP, it can help the reader if you mention what exactly the variables of the LP are.}
\begin{align}
    \max &&& \sum_{\ell=1}^{t-1} q_{\ell} \sum_{i=1}^{\ell} z_i + q_t (\sum_{i=1}^{t-1} z_i +1) \tag{\progAppDual}\label{eq:app-dual}\\
        s.t. &&& (\text{\progAppDual.1}) \Hquad \sum_{j=1}^n u_j(A) \sum_{\ell=1}^t v_{\ell,j} \leq 1  && \forall A \in \mathcal{A} \nonumber\\
                    &&& (\text{\progAppDual.2}) \Hquad \ell q_{\ell} - \sum_{j=1}^n v_{\ell,j} = 0 && \forXinY{\ell}{t} \nonumber \\
                    &&& (\text{\progAppDual.3}) \Hquad q_{\ell} - v_{\ell,j} \leq 0  && \forXinY{\ell}{t},\Hquad \forXinY{j}{n} \nonumber \\
                    &&& (\text{\progAppDual.4}) \Hquad v_{\ell,j} \geq 0  && \forXinY{\ell}{t},\Hquad \forXinY{j}{n} \nonumber \\
                    &&& (\text{\progAppDual.5}) \Hquad q_{\ell} \geq 0  && \forXinY{\ell}{t} \nonumber
\end{align}
Here, the variables are $q_{\ell}$ and $v_{\ell,j}$ for any $\ell \in [t]$ and $j \in [n]$; and the constants are (as before) $z_i$ for $i \in [t-1]$.
Recall that $u_j(A)$ is the utility that agent $j$ assigns to simple allocation $A$, as given by the value oracle.
% title: ellipsoid variant
This problem has an exponential number of constraints --- a constraint for each allocation (in line (\progAppDual.1)) but only a polynomial number of variables.
Using the ellipsoid method \cite{grotschel_ellipsoid_1981}, it could be solved in polynomial time 
if we had a \emph{separation oracle} ---
an oracle that given a vector $\upsilon$ either determines that $\upsilon$ is infeasible and returns a violated constraint, or asserts that $\upsilon$ is feasible.
Unfortunately, as we shall now see, it is NP-hard to compute a separation oracle to this problem.
\begin{lemma}
    Computing a separation oracle to \eqref{eq:app-dual} is NP-hard.
\end{lemma}

% very similar to what they did in yonatan's paper..
\begin{proof}
We prove that a separation oracle for \eqref{eq:app-dual} would allow us to compute a leximin optimal stochastic allocation.
    As discussed previously, computing such an allocation is NP-hard, so the same applies for computing a separation oracle for \eqref{eq:app-dual}.

    First, we prove that such a separation oracle can be used to extract an optimal solution to \eqref{eq:app-ver2-vsums-OP}.
    Assume that the ellipsoid method was operated with the given oracle to solve \eqref{eq:app-dual}.
    Let $C$ be the set of constraints that the oracle determined as being violated.
    Since the ellipsoid method operates in polynomial time, the size of the set $C$ is also polynomial.
    Let $V_C$ be the set of variables of \eqref{eq:app-ver2-vsums-OP} associated with the constraints in $C$.
    By complementary slackness, the variables in $V_C$ are the only ones that may get a \emph{positive} value in the corresponding optimal solution to \eqref{eq:app-ver2-vsums-OP}.
    Therefore, the program \eqref{eq:app-ver2-vsums-OP} with only the variables in $V_C$ (and the other variables equal to zero) has a polynomial size, and therefore can be solved exactly.


    But, by Observation \ref{obs:c2-to-c1-in-poly-time}, this would allow us to find the corresponding optimal solution to \eqref{eq:app-vsums-OP} in polynomial time.
    % \erel{Did we say that $\psi$ can be computed in polynomial time?}\eden{in the way it is written now is not, it iterate over each variable of \eqref{eq:app-vsums-OP} and there are exponential number of them. I need to think how to write it appropriately. maybe "that can be computed in time equals to the number of positive variables"?}
    % \erel{If it is not polynomial, then the reduction is not polynomial, so it does not imply NP-hardness}
    This means the described process can be used as an approximation procedure to \eqref{eq:vsums-OP} (that became \eqref{eq:app-vsums-OP} under the settings of this problem) with $\multError = \additiveError = 0$.
    Therefore, by Theorem \ref{th:main}, this means we can use Algorithm \ref{alg:basic-ordered-Outcomes} to obtain a leximin optimal solution\footnote{Actually, Theorem \ref{th:main} says that Algorithm \ref{alg:basic-ordered-Outcomes} will output a $(1,0)$-leximin-approximation; But Lemma \ref{lemma:absence-of-errors} says that such a solution is, indeed, a leximin optimal solution.}.
\end{proof}


% --- it would allow us to compute a leximin optimal stochastic allocation, which is, as discussed previously, NP-hard.

In Appendix \ref{sec:mult-variant-ellipsoid}, we present another variant of the ellipsoid method, which allows us to approximate the program \eqref{eq:app-ver2-vsums-OP} given a \emph{half-randomized approximate separation oracle} to \eqref{eq:app-dual}.
That is, an oracle that, given a multiplicative error $\multError$, a success probability $p$, and a vector $\upsilon$, either determines that $\upsilon$ is infeasible and returns a violated constraint; or determines that $\upsilon$ is $\multError$-\textit{approximately-feasible}, which means that for any constraint $a \cdot x \leq b$, the vector $\upsilon$ satisfies $a \cdot \upsilon \leq (1+\multError)\cdot b$.
When the oracle says that $\upsilon$ is $\multError$-approximately-feasible, it is correct with probability at least $p$.
Given such an oracle for the dual program, the ellipsoid method variant can be used to output a solution to the primal, that approximates it to the same factor with probability at least $p^I$, where $I$ is an upper bound on the number of iterations in any execution of the ellipsoid method variant on the dual (if it is given a deterministic oracle).
We can therefore conclude the following result:
\begin{lemma}\label{lemma:approx-sep-oracle-to-goal}
    Given a half-randomized approximate separation oracle to the problem \eqref{eq:app-dual}, with a multiplicative error of $\frac{\beta}{1-\beta}$ and a success probability $p$, a stochastic allocation that approximates leximin to a multiplicative error $\frac{\multError}{1-\multError+\multError^2}$ can be obtained with probability $p^{nI}$.
\end{lemma}

\begin{proof}
    % To begin, assume that we are given a deterministic approximate separation oracle (i.e., with failure probability $p=0$).
    As described above, we can use the ellipsoid method variant of Appendix \ref{sec:mult-variant-ellipsoid} with the given oracle to \eqref{eq:app-dual} to obtain a solution to \eqref{eq:app-ver2-vsums-OP},  that approximates it with a multiplicative error of $\frac{\multError}{1-\multError}$ with probability $p^I$.
    Then, by Observation \ref{obs:c2-to-c1-in-poly-time}, this would allow us to find the corresponding solution to \eqref{eq:app-vsums-OP}, that, with probability $p^I$, approximates it with a multiplicative error of $\multError$.
    That is, the described process can be used as a randomized approximation procedure to \eqref{eq:vsums-OP} (that became \eqref{eq:app-vsums-OP} under the settings of this problem).
    % with $\multError = \additiveError = 0$.
    Therefore, by Theorem \ref{th:main}, Algorithm \ref{alg:basic-ordered-Outcomes} can be used to obtain a leximin approximation to the original problem with only a multiplicative error of $\frac{\multError}{1-\multError+\multError^2}$ with probability $p^{nI}$ (Corollary \ref{corollary:main-with-probability}).
\end{proof}

Now, we show that such an oracle can be designed given a randomized approximation algorithm for computing a simple allocation that approximates the utilitarian welfare. Specifically, 

\begin{lemma}\label{lemma:alg-for-utilitarian-to-sep-oracle}
    Given a randomized approximation algorithm for computing a simple allocation that approximates the utilitarian welfare with multiplicative error $\multError$ and a success probability $p$, a half-randomized approximate separation oracle to \eqref{eq:app-dual} can be designed with a multiplicative error of $\frac{\beta}{1-\beta}$ and a success probability at least $\left(1-\frac{1}{nI}(1-p)\right)$.
\end{lemma}

% \eden{should say somewhere that the oracle is polynomial time and therefore everything is?...}
% FROM HERE: https://tex.stackexchange.com/a/675333/20929
\algdef{SE}[REPEATN]{REPEATN}{ENDREP}[1]{\algorithmicrepeat\ #1 \textbf{times}}{\algorithmicend\ \algorithmicrepeat}
\begin{algorithm}[!tbp]
\caption{A Half-Randomized Approximate Separation Oracle to \eqref{eq:app-dual}}
\label{alg:sep-oracle}
INPUT: variables $q_{\ell}$ and $v_{\ell,j}$ for any $\ell \in [t]$ and $j \in [n]$, an $\multApprox$-approximation algorithm for the utilitarian welfare problem (\eqref{eq:utilitarian}) with success probability $p$.
\begin{algorithmic}[1] %[1] enables line numbers
\STATE Iterate over constraints (\progAppDual.2)-(\progAppDual.5). If one of them is  violated, stop and return it.
\STATE \textbf{If} $p=1$ then set $T:=1$; \textbf{else} set $T := 1 + \lceil-\log_{(1-p)}(nI)\rceil$.

\REPEATN{$T$}
    \STATE Operate the algorithm for the utilitarian welfare problem on $n,m,(u'_j)_{j=1}^n$ to obtain an allocation $\Tilde{A}$ with value $\nu$.
    \IF{$\nu > 1$}  
        \STATE Return the corresponding violated constraint $\sum_{j=1}^n u_j(\Tilde{A}) \sum_{\ell=1}^t v_{\ell,j} > 1$
    \ENDIF
\ENDREP
\STATE Return "the assignment is approximately-feasible".

\end{algorithmic}
\end{algorithm}


Algorithm \ref{alg:sep-oracle} describes the oracle.
It accepts as input an assignment to the variables of \eqref{eq:app-dual}, that is, $q_{\ell}$ and $v_{\ell,j}$ for any $\ell \in [t]$ and $j \in [n]$, and an algorithm for approximating the maximum utilitarian welfare.
It starts by verifying constraints (\progAppDual.2)-(\progAppDual.5) one by one (this is possible as their number is polynomial in $n$ and $m$). 
If a violated constraint was found, the oracle simply returns it. Otherwise, it proceeds to check constraints (\progAppDual.1).
Although the number of constraints described by (\progAppDual.1) is exponential in $n$, they can be treated collectively in polynomial time (as in \cite{kawase_max-min_2020}).
% \eden{here maybe to say something about the randomness}.\erel{Maybe mention that \textcite{kawase_max-min_2020} ignored this issue.}
First, notice that in order to determine whether the expression $\sum_{j=1}^n u_j(A) \sum_{\ell=1}^t v_{\ell,j}$ is at most $1$ for all simple allocations ($A \in \mathcal{A}$), it is sufficient to check the allocation that maximizes this expression and compare it to $1$.
Define new utility functions for all $j \in [n]$ and $A \in \mathcal{A}$, 
\begin{align*}
u'_j(A) := \sum_{\ell=1}^t v_{\ell,j} \cdot u_j(A) 
\end{align*}
The above expression can be simplified to $\sum_{j=1}^n u'_j(A)$. An allocation that maximizes this expression is an allocation that maximizes the utilitarian welfare (i.e., the sum of utilities) when the same sets of agents and items is considered but with different utilities%
\footnote{Notice that the utilities $u'_j$ are  normalized, monotone, submodular, and can be computed using $t\leq n$ calls to the value oracle of $u_j$}
($u'_j$ instead of $u_j$ for $j \in [n]$).
Such an allocation cannot be found in polynomial time since approximating the utilitarian welfare up to a factor better than $(1-\frac{1}{e})$ in the case of submodular utilities is known to be NP-hard \cite{khot_inapproximability_2008}.
However, the oracle is given an approximation algorithm to the utilitarian welfare problem as input.
Therefore, an allocation $\Tilde{A}$ with utilitarian value at least $(1-\multError)$ of the optimal can be obtained with probability $p$.
We shall now see that it is enough.


\begin{proof}[Proof of Lemma \ref{lemma:alg-for-utilitarian-to-sep-oracle}]
First, observe that when Algorithm \ref{alg:sep-oracle} returns a violated constraint, it is always correct.
This is obvious for constraints described by (\progAppDual.2)-(\progAppDual.5), since these constraints have been verified directly.
For constraints described by (\progAppDual.1), it means that the algorithm found an allocation $\Tilde{A}$ that satisfies $\sum_{j=1}^n u'_j(\Tilde{A}) > 1$.
    By the definition of $u'$, the constraint corresponding to this allocation is, indeed, violated:
    \begin{align*}
         \sum_{j=1}^n u_j(\Tilde{A}) \sum_{\ell=1}^t v_{\ell,j} = \sum_{j=1}^n u'_j(\Tilde{A}) > 1.
    \end{align*}
Let us assume that the given algorithm for the utilitarian welfare problem is deterministic (i.e., $p=1$) and then revisit the case $p<1$.
    Assume that the oracle said that the assignment is approximately-feasible.
    This means that the algorithm for the utilitarian welfare problem found an allocation $\Tilde{A}$ with value at most $1$.
    Since $\Tilde{A}$ is approximately-optimal, the optimal utilitarian value is at most $1/(1-\multError)\cdot 1$.
    As this is an upper bound of the utilitarian value of any allocation, it follows that all the constraints described bu (\progAppDual.1) are $\frac{\multError}{1-\multError}$-approximately maintained --- that is, for any allocation $A \in \mathcal{A}$ the following holds:
    \begin{align*}
        \sum_{j=1}^n u'_j(A) = \sum_{j=1}^n u_j(A) \sum_{\ell=1}^t v_{\ell,j} \leq \frac{1}{1-\multError}\cdot 1 = \left(1+\frac{\multError}{1-\multError}\right)\cdot1
    \end{align*}
    We get that, in this case, the oracle is also deterministic, and that the success probability is at least $\left(1-\frac{1}{nI}(1-p)\right) = 1$ for $p=1$.

    Assume now that $p<1$. Then, the oracle may be incorrect when it says the assignment is approximately feasible, but only if the algorithm for the utilitarian welfare problem did not return an appropriate approximation in all $T = \lceil-\log_{(1-p)}(nI)\rceil + 1$ operations, that is, with probability at most $(1-p)^T$.
    % as each operation of the oracle is independent
    Notice that $T>1$ since $\log_{(1-p)}(nI) < 0$\footnote{
    % The fact that  $\log_{(1-p)}(nI) < 0$ can be easily concluded 
    Since $(1-p)\in(0,1)$ and $nI>1$ by change of base: $\log_{(1-p)}(nI) = \log(nI)/\log(1-p)$, the numerator is positive and the denominator is negative.}.
    Now, as $T \geq -\log_{(1-p)}(nI) + 1$ and $(1-p)<1$ we get that:
    \begin{align*}
        &(1-p)^T \leq (1-p)\cdot(1-p)^{-\log_{(1-p)}(nI)} = (1-p)(nI)^{-1}
    \end{align*}
    So, the success probability is at least $\left(1-\frac{1}{nI}(1-p)\right)$.
\end{proof}

We can now prove Theorem \ref{th:app-main}.

\begin{proof}[Proof of Theorem \ref{th:app-main}]
    Assume we are given an algorithm that returns a simple allocation that approximates the utilitarian welfare with multiplicative error $\multError$ with success probability $p$.
    By Lemma \ref{lemma:alg-for-utilitarian-to-sep-oracle} this algorithm can be used to obtain an half-randomized approximate separation oracle to \eqref{eq:app-dual} with a multiplicative error $\frac{\multError}{1-\multError}$ with success probability $\left(1-\frac{1}{nI}(1-p)\right)$.
    By Lemma \ref{lemma:approx-sep-oracle-to-goal}, with such an oracle a stochastic allocation that approximates leximin to a multiplicative error of $\frac{\multError}{1-\multError+\multError^2}$ can be obtained with probability $\left(1-\frac{1}{nI}(1-p)\right)^{nI}$.
    If $p=1$ then the success probability is $1$ too (at least $\left(1-\frac{1}{nI}(1-p)\right)^{nI}= 1$).
    However, if $p<1$, then $\frac{1}{nI}(1-p) \in (0,1)$ and therefore the success probability is at least $p$\footnote{For any $\epsilon \in (0,1)$ and $k \in \mathbb{Z}_{+} \colon \Hquad (1 - \epsilon)^k \geq 1 - k \cdot \epsilon$}:
    \begin{align*}
        \left(1-\frac{1}{nI}(1-p)\right)^{nI} \geq \left(1-nI\cdot\frac{1}{nI}(1-p)\right) = p.   \end{align*}
\end{proof}

\section{Equivalent Single-objective Optimization Problems in the Presence of Errors}\label{sec:equivalent-proofs}

Many times, when referring to two optimization problems\footnote{In this section,  we consider only single-objective optimization problems.} as equivalent, one means that they have the same optimal value.
When two problems satisfy this relation, it is clear that in order to obtain an optimal \emph{value}, a solver\footnote{It is assumed that a solver (either approximate or exact) for a single-objective optimization problem returns a solution and its objective value.\eden{maybe to explain it better..}} for one can be used as a solver for the other. 
However, if we are interested in an optimal \emph{solution} that yields this value, a solver that returns an optimal solution for another problem with the same optimal value is not enough\eden{reduction to feasibility problem}.
Moreover, when it comes to approximation, even if we are only concerned about the objective value, an approximate solver for one can no longer be used for the other.
To illustrate, consider the following problems:
\begin{align*}
    (E1) \Hquad &\max\quad x                         &&& (E2)\Hquad &  \max\quad x\\
    &\Hquad s.t.\quad  x \in \{0.9,1\}       &&&& \Hquad s.t.\quad x \in \{0.95,1\} 
\end{align*}    
Both problems have the same optimal objective value $1$.
Now, assume that a multiplicative error of $0.1$ is acceptable.
An approximate solver for the problem $(E1)$ may return the objective value $0.9$, which is not a possible value of $(E2)$; similarly, an approximate solver for the problem $(E2)$ may return the objective value $0.95$, which is not a possible value of $(E2)$.
% Thus, although both problems have the same optimal value, an approximate solver for one problem \emph{cannot} be used as an approximate solver for the other.

In this appendix, we present a new definition of equivalent optimization problems, which requires a stronger relationship.
We prove that, according to our definition, when two optimization problems are equivalent, a solver for one, either exact or approximate, can also be used for the other.

\paragraph{Equivalent problems definition} We say that two (single-objective) optimization problems, $OP1 = (S_1,f_1)$ and $OP2 = (S_2,f_2)$, are \emph{equivalent} if they are from the same type --- either both are maximization problems or both are minimization problems; and there exists a bijection, $B \colon S_1 \to S_2$, mapping each solution of $OP1$, $x \in S_1$, to a unique solution of $OP2$, $B(x) \in S_2$, and they have the same objective value $f_1(x) = f_2(B(x))$.

% It is easy to conclude that this relation is symmetric, reflexive and transitive and therefore it is, indeed, an equivalence relation.
The following observation can be easily concluded by the definition:
\begin{observation}
    The equivalent relation between problems is transitive, reflexive and symmetric.
\end{observation}
% \begin{observation}
%     The equivalent relation between problems is transitive.
% \erel{maybe also argue that it is reflexive and symmetric, so it is an equivalence relation.}
% \end{observation}
% \eden{to explain }

If we are only concerned with the objective value, then the following lemma ensures that an approximate solver for one problem can be applied, as is, to the other (it is not necessary to know what the bijection is):

\begin{lemma}\label{lemma:approx-value-equivalent-prob}
    Let $OP1 = (S_1,f_1)$ and $OP2 = (S_2,f_2)$ be equivalent optimization problems, and let $v_1 \in \mathbb{R}$ be an $(\multApprox,\additiveApprox)$-approximation of the optimal objective value of $OP1$.
    Then, $v_1$ is also an $(\multApprox,\additiveApprox)$-approximation of the optimal objective value of $OP2$.
% \erel{Add that the approximation ratios ($\alpha$,$\epsilon$) is the same}
\end{lemma}

\begin{proof}
    For brevity, we prove the claim only for maximization problems, the proof for minimization problems is similar.
    
    Let $x^*\in S_1$ and $y^*\in S_2$ be optimal solutions of the problems $OP1$ and $OP2$ respectively.
    In order to prove that $v_1$ is an $(\multApprox,\additiveApprox)$-approximation of the optimal objective value of $OP2$, we will show that there is a solution $y \in S_2$ with objective value $v_1$, and also that $v_1 \geq \multApprox f_2(y^*) - \additiveApprox$.

    First, since $v_1$ is an $(\multApprox,\additiveApprox)$-approximation of the optimal objective value of $OP1$, there exists a solution $x \in S_1$ such that $f_1(x) = v_1$ and also $v_1 \geq \multApprox f_1(x^*) - \additiveApprox$.
    By definition of equivalent problems, the corresponding solution to $OP2$ by the bijection, $B(X) \in S_2$, has the same objective value $f_2(B(x)) = v_1$.
    
    In addition, we shall now see that both problems have the same optimal objective value.
    Let $B: S_1\to S_2$ be a bijection as described in the definition of equivalent problems.    
    So $f_1(x^*)=f_2(B(x^*))$, and $f_2(B(x^*))\leq f_2(y^*)$ by optimality of $y^*$, so $f_1(x^*)\leq f_2(y^*)$. By analogous arguments $f_2(y^*)\leq f_1(x^*)$, so in fact $f_1(x^*) = f_2(y^*)$.
    
    Therefore, we can conclude that:
    \begin{align*}
        f_2(B(x)) = v_1 \geq \multApprox f_1(x^*) - \additiveApprox =  \multApprox f_2(y^*) - \additiveApprox
    \end{align*}
    as required.
\end{proof}

% \begin{lemma}\label{lemma:solver-equivalent-prob}
%     Let $OP1 = (S_1,f_1)$ and $OP2 = (S_2,f_2)$ be equivalent optimization problems. Then, in order to approximate the optimal value, an $(\multApprox,\additiveApprox)$-approximate solver for one can be used as an $(\multApprox,\additiveApprox)$-approximate solver for the other.
% % \erel{Add that the approximation ratios ($\alpha$,$\epsilon$) is the same}
% \end{lemma}

% \begin{proof}
%     For brevity, we prove the claim only for maximization problems, the proof for minimization problems is similar.
%     Let $x^*\in S_1$ and $y^*\in S_2$ be optimal solutions of the problems $OP1$ and $OP2$ respectively.
%     Let $B: S_1\to S_2$ be a bijection as described in the definition of equivalent problems.    
%     So $f_1(x^*)=f_2(B(x^*))$, and $f_2(B(x^*))\leq f_2(y^*)$ by optimality of $y^*$, so $f_1(x^*)\leq f_2(y^*)$. By analogous arguments $f_2(y^*)\leq f_1(x^*)$, so in fact $f_1(x^*) = f_2(y^*)$.
%     % , since otherwise one of them is higher, and therefore the bijection can be used to obtain a solution to the second problem with value higher than optimal. \eden{need to rewrite it..}
%     Now, 
%     % without loss of generality, 
%     assume that we have an $(\multApprox, \additiveApprox)$-approximate solver for $OP1$, for some $\multApprox\in(0,1]$ and $\additiveError\geq 0$.
%     That is, the solver returns a solution $x \in S_1$ such that $f_1(x) \geq \multApprox \cdot f_1(x^*) - \additiveError$. 
%     Consider the corresponding solution to $OP2$ by the bijection, $B(X) \in S_2$, we know that $f_1(x_1) = f_2(B(x_1))$.
%     It follows that $B(x)$ is an $(\multApprox, \additiveApprox)$-approximation to $OP2$:
%     \begin{align*}
%         f_2(B(x)) = f_1(x) \geq \multApprox \cdot f_1(x^*) - \additiveError = \multApprox \cdot f_2(y^*) - \additiveError
%     \end{align*}
% \end{proof}

Notice that the approximation value is obtained by the corresponding solution ($B(X)$), and therefore, we can also conclude the following result:
% Therefore, if we also have access to procedures to calculate the bijection and its inverse, then we can use a solver for one problem to find the solution to the other, that is:
\begin{corollary}\label{corollary:solver-equivalent-prob}
    Let $OP1 = (S_1,f_1)$ and $OP2 = (S_2,f_2)$ be equivalent optimization problems, and let $P_{1\to 2}$ be a procedure that, given a solution to $OP1$, returns the corresponding solution to $OP2$.
    Then, an $(\multApprox, \additiveApprox)$-approximate solver for $OP1$ can be used to obtain a \emph{solution} that is an $(\multApprox, \additiveApprox)$-approximation for $OP2$.
\end{corollary}

If the procedure from $OP1$ to $OP2$ operates in polynomial time we say that $OP1$ is \emph{polynomial-time equivalent} to $OP2$. 

\eden{how is the name "polynomial-time equivalent"?}

% \eden{If we will have time: "Further, if the bijection is given and can be calculated in polynomial time, then ....}



\subsection{Relationships Between Single-Objective Problems for Leximin Optimization}
\eden{I'm not sure which title to give}

For clarity, descriptions of all the problems are provided here as well (table \ref{table:prob-des}).

\begin{table}[h!]
\begin{tabular}{l}
\hline
\\
$\begin{aligned}
     \text{(P1)}\Hquad \max \quad &\ztVar{x}  \;\;
        s.t. &\quad  & (1) \quad x \in S \\
              &     & & (2) \quad \valBy{\ell}{x}\geq z_{\ell} & \ell = 1,\ldots,t-1\nonumber \\
               &    & & (3) \quad \valBy{t}{x} \geq \ztVar{x} \nonumber  \\\\
    \text{(P2)}\Hquad\max \quad &\ztVar{x}  \;\;
        s.t. &\quad  & (1) \quad x \in S  \\
        &&& (\hat{2}) \quad \sum_{i \in F'} f_i(x) \geq \sum_{i=1}^{|F'|}  z_i & \forall F' \subseteq [n], |F'| < t \\
        &&& (\hat{3}) \quad \sum_{i \in F'} f_i(x) \geq \sum_{i=1}^{t}  z_i  & \forall F' \subseteq [n], |F'| = t\\\\
     \text{(P3)}\Hquad \max \quad &\ztVar{x}  \;\;
        s.t. &\quad  & (1) \quad x \in S  \\
                    &&& (2) \quad \ell y_{\ell} - \sum_{j=1}^n m_{\ell,j}\geq \sum_{i=1}^{\ell}  z_i & \ell = 1, \ldots,t-1 \nonumber \\
                    &&& (3) \quad t y_t - \sum_{j=1}^{n} m_{t,j} \geq \sum_{i=1}^{t}  z_i  \nonumber \\
                    &&& (4) \quad m_{\ell,j} \geq y_{\ell} - f_j(x)  & \ell = 1, \ldots,t,\Hquad j = 1, \ldots,n \nonumber \\
                    &&& (5) \quad m_{\ell,j} \geq 0  & \ell = 1, \ldots,t,\Hquad j = 1, \ldots,n \nonumber\\\\
    \text{(P2-compact)}& \\
    \max \quad &z_t  \;\;
        s.t. &\quad  & (1) \quad x \in S \\
                    &&& (\Tilde{2}) \quad \sum_{i=1}^{\ell} \valBy{i}{x} \geq \sum_{i=1}^{\ell}  z_i & \ell = 1,\ldots, t-1 \nonumber\\
                    &&& (\Tilde{3}) \quad \sum_{i=1}^{t} \valBy{i}{x} \geq \sum_{i=1}^{t}  z_i\\
\end{aligned}$\\
\\
\hline
\end{tabular}
\caption{Summary description of the problems.}
\label{table:prob-des}
\end{table}


\subsubsection{Equivalence of The Problems \eqref{eq:sums-OP} and \eqref{eq:compact-OP}}\label{sec:prob-sums-and-comp}
we prove that the \emph{identity function} is an appropriate bijection between \eqref{eq:sums-OP} and \eqref{eq:compact-OP}. Therefore, they are polynomial-time equivalent to each other. 

We start by proving the following lemma:
\begin{lemma}\label{lemma:sums-to-comp-constrants}
    For any $x \in S$, any $\ell \in [n]$ and a constant $c \in \mathbb{R}$ the following two conditions are equivalent:
    \begin{align}\label{eq:sums-to-comp-constrants}
         \forall F' \subseteq [n], |F'| = \ell \colon \sum_{i \in F'} f_i(x) &\geq c 
         \\
         \sum_{i=1}^{\ell} \valBy{i}{x}&\geq c 
    \end{align}
\end{lemma}

\begin{proof}
    For the first direction, recall that the values $ \valBy{1}{x}, \dots,  \valBy{\ell}{x}$ were obtained from $\ell$ objective functions (those who yield the smallest value).
    By the assumption, the sum of any set of function with size $\ell$ is at least $c$; therefore, it is true in particular for the functions corresponding to the values $ (\valBy{1}{x})_{i=1}^{\ell}$.
    For the second direction, assume that $\sum_{i=1}^{\ell} \valBy{i}{x}\geq c$.
    Since $ \valBy{1}{x}, \dots,  \valBy{\ell}{x}$ are the $\ell$ smallest values in $\allValues{x}$, we get that:
    \begin{align*}
       \forall F' \subseteq [n],\Hquad |F'| = \ell \colon \quad \sum_{i \in F'}f_i(x) \geq \sum_{i=1}^s \valBy{i}{x}\geq c.
    \end{align*}
\end{proof}

    Now, let $(x,z_t)$ be a solution to \eqref{eq:sums-OP}. 
    As $x$ satisfies constraint (1) of \eqref{eq:sums-OP}), it is also satisfies constraint (1) of \eqref{eq:compact-OP} (as both constraints are the same, $x \in S$).
    In addition, as $x$ satisfies constraint $(\hat{2})$ of \eqref{eq:sums-OP}, for any $\ell \in [t-1]$, 
    \begin{align*}
        \forall F' \subseteq [n], |F'| = \ell \colon \sum_{i \in F'} f_i(x) \geq \sum_{i=1}^{\ell} z_i
    \end{align*}
    by Lemma \ref{lemma:sums-to-comp-constrants}, also $\sum_{i=1}^{\ell} \valBy{i}{x} \geq \sum_{i=1}^{\ell} z_i$. Therefore, $x$ satisfies constraint $(\Tilde{2})$ of \eqref{eq:compact-OP}.
    Lastly, as $x$ and $z_t$ satisfy constraint $(\hat{3})$ of \eqref{eq:sums-OP}, 
    \begin{align*}
        \forall F' \subseteq [n], |F'| = t \colon \sum_{i \in F'} f_i(x) \geq \sum_{i=1}^{t} z_i
    \end{align*}
    again, by Lemma \ref{lemma:sums-to-comp-constrants} also $\sum_{i=1}^{t} \valBy{i}{x} \geq \sum_{i=1}^{t} z_i$.
    So, $x$ ans $z_t$   satisfy constraint $(\Tilde{3})$ of \eqref{eq:compact-OP}.
    Since we saw that $x$ ans $z_t$ satisfy all the constraints of \eqref{eq:compact-OP}, it is feasible to this problem.

    As in both problems the objective value is determined by $z_t$, it is clear that $(x,z_t)$ obtains the same objective value from both \eqref{eq:sums-OP} and \eqref{eq:compact-OP}.

    Therefore, the identity function (i.e., $B((x,z_t)) = (x,z_t)$) is an appropriate bijection and so, the problems are equivalent.



%--------------------------------------------------
\subsubsection{Equivalence of The  problems \eqref{eq:compact-OP} and \eqref{eq:vsums-OP}} We prove that these problems are equivalent by describing an appropriate bijection.
We will also see that this bijection and its inverse can be calculated in polynomial time and therefore, each problem is polynomial-time equivalent to the other.

We start with the following lemma:
\begin{lemma}\label{lemma:comp-to-p3-m-sums}
    For any $x \in S$ and any constant $c \in C$,
    \begin{align*}
        \sum_{j=1}^n \max(0, c - f_j(x) ) = \sum_{j=1}^n \max(0, c - \valBy{j}{x} )
    \end{align*}
\end{lemma}
\begin{proof}
     Let $(\pi_1, \ldots, \pi_n)$ be a permutation of $\{1,\ldots,n\}$ such that $f_{\pi_i}(x) = \valBy{i}{x}$ for any $i \in [n]$ (notice that such permutation exists by the definition of $\valBy{}{}$).
     That is, the value that $f_{\pi_i}$ obtains is the ${\pi_i}$-th smallest one in the multiset of all values $\allValues{x}$.
    Since each element in the sum $\sum_{j=1}^n \max(0, c - f_j(x))$ is affected by $j$ only through $f_j(x)$, the permutation $\pi$ allows us to conclude the following:
    \erel{This argument is not clear}\eden{better?}
    \begin{align*}
        \sum_{j=1}^n \max(0, c -f_j(x)) &= \sum_{j=\pi_1}^{\pi_n} \max(0,c -f_j(x)\\ 
        &= \sum_{j=1}^n \max(0,c -f_{\pi_i}(x)  = \sum_{j=1}^{n} \max(0,c -\valBy{j}{x})
    \end{align*}
\end{proof}



Following is Lemma \ref{lemma:comp-to-p3-mapping}, which describes a function $B$ and proves that it is a mapping from the feasible region of the problem \eqref{eq:compact-OP} to the feasible region of the problem \eqref{eq:vsums-OP}.
Then, Lemma \ref{lemma:comp-to-p3-is-bij} proves that this mapping is a bijection.
Lastly, Lemma \ref{lemma:comp-to-p3-obj} shows that the same objective value is obtained.

\begin{lemma}\label{lemma:comp-to-p3-mapping}
    Let $(x,z_t)$ be a feasible solution  to \eqref{eq:compact-OP}. Then $B((x,z_t)) = (x, z_t, (y_1,\ldots,y_n), (m_{1,1},\ldots,m_{n,n}))$ is a feasible solution to \eqref{eq:vsums-OP}, where
    \begin{align*}
        \quad y_{\ell} &:= \valBy{\ell}{x} \Hquad\forall \ell \in [n], 
        \\
        m_{\ell,j} &:= \max(0,y_{\ell} -f_j(x)) \Hquad \forall \ell \in [n], \Hquad \forall 1 \leq j \leq n 
    \end{align*}
\end{lemma}

\begin{proof}
    First, since $x$ satisfies constraint (1) of \eqref{eq:compact-OP}, it is also satisfies constraint (1) of \eqref{eq:vsums-OP} (as both constraints are the same).
    Also, as $m_{\ell,j} \geq 0$ and $m_{\ell,j} \geq y_{\ell} - f_j(x)$ for any $\ell \in [n]$ and $j \in [n]$, this assignment satisfies constraints (4) and (5) of \eqref{eq:vsums-OP}.
    
    To show that this assignment also satisfies constraints (2) and (3) of problem \eqref{eq:vsums-OP}, we first prove that for any $\ell \in [n]$ and any constant $c \in \mathbb{R}$ this assignment satisfies the following:
    \begin{align}\label{eq:comp-to-p3}
        \sum_{i=1}^{\ell} \valBy{i}{x}\geq c \Hquad \Longrightarrow \Hquad \ell y_{\ell} - \sum_{j=1}^n m_{\ell,j}\geq c
    \end{align}
    As $y_{\ell} = \valBy{\ell}{x}$, also  $m_{\ell,j} = \max(0,\valBy{\ell}{x} -f_j(x))$.
    % in this way it is easy to see that $j$ affects $m$ only through $f_j(x)$.
    And so, by Lemma \ref{lemma:comp-to-p3-m-sums}, it can also be described as $\sum_{j=1}^{n} \max(0,\valBy{\ell}{x} -\valBy{j}{x})$.
    Since $\valBy{\ell}{x}$ is the $\ell$-th smallest objective, it is clear that $\valBy{\ell}{x} - \valBy{j}{x} \leq 0$ for any $j > \ell$, and $\valBy{\ell}{x} - \valBy{j}{x} \geq 0$ for any $j \leq \ell$.
    We can now conclude that $\ell y_{\ell} - \sum_{j=1}^n m_{\ell,j}\geq c$:
    \begin{align*}
        &\ell y_{\ell} - \sum_{j=1}^n m_{\ell,j} = \ell \cdot \valBy{\ell}{x} - \sum_{j=1}^n \max(0,\valBy{\ell}{x} -\valBy{j}{x}) \\
        &= \ell \cdot \valBy{\ell}{x} - \sum_{j=1}^{\ell} \max(0,\valBy{\ell}{x} -\valBy{j}{x}) - \sum_{j=\ell+1}^n \max(0,\valBy{\ell}{x} -\valBy{j}{x}) \\
        &= \ell \cdot \valBy{\ell}{x} - \sum_{j=1}^{\ell} \left(\valBy{\ell}{x} -\valBy{j}{x}\right) - \sum_{j=\ell+1}^n 0 = \ell \cdot \valBy{\ell}{x} - \ell \cdot\valBy{\ell}{x} + \sum_{j=1}^{\ell} \valBy{j}{x}\\
        &= \sum_{j=1}^{\ell} \valBy{j}{x} \geq  c \text{~~~by assumption.}
    \end{align*}

    Now, since $x$ satisfies constraint $(\Tilde{2})$ of \eqref{eq:compact-OP}, for any $\ell \in [t-1]$, $\sum_{i=1}^{\ell} \valBy{i}{x} \geq \sum_{i=1}^{\ell} z_i$ and so by equation \ref{eq:comp-to-p3}, $\ell y_{\ell} - \sum_{j=1}^n m_{\ell,j}\geq  \sum_{i=1}^{\ell} z_i$
    Therefore, this assignment constraint (2) of problem \eqref{eq:vsums-OP}.
    In addition, as $x$ and $z_t$ satisfy constraint $(\Tilde{3})$ of \eqref{eq:compact-OP}, $\sum_{i=1}^{t} \valBy{i}{x} \geq \sum_{i=1}^{t} z_i$ and so by equation \ref{eq:comp-to-p3}, \ref{eq:comp-to-p3}, $t y_{t} - \sum_{j=1}^n m_{t,j}\geq  \sum_{i=1}^{t} z_i$.
    This means that also satisfies constraints (3) of problem \eqref{eq:vsums-OP}.
\end{proof}

\begin{lemma}\label{lemma:comp-to-p3-is-bij}
    The mapping $B$ is a bijection.
\end{lemma}

\begin{proof}
    Injective ($B(a) = B(b) \Rightarrow a = b$) is trivial since $x$ and $z_t$ are part of the solution.
    
    To prove that the mapping is surjective, we will show that for any feasible solution to \eqref{eq:vsums-OP}, that is,
    \begin{align*}
        (x \in S, z_t, y_1, \ldots, y_t, m_{1,1}, \ldots, m_{1,n}, m_{2,1}, \ldots, m_{2,n},\ldots, m_{t,1}, \ldots, m_{t,n})
    \end{align*}
    there is a feasible solution to \eqref{eq:compact-OP} that is  mapped to this solution.
    In fact, we prove that $(x,z_t)$ does.

    It is easy to see that since $x$ satisfies constraint (1) of \eqref{eq:vsums-OP}, it is also satisfies constraint (1) of \eqref{eq:compact-OP} (as both are the same).
    To show that it also satisfies constraints $(\Tilde{2})$ and $(\Tilde{3})$ of \eqref{eq:compact-OP}, we start by proving that for any $\ell \in [n]$ and any constant $c \in \mathbb{R}$:
    \begin{align}\label{eq:p3-to-comp}
         \ell y_{\ell} - \sum_{j=1}^n m_{\ell,j}\geq c
         \Hquad \Longrightarrow \Hquad \sum_{i=1}^{\ell} \valBy{i}{x}\geq c
    \end{align}
    Notice that, for any $j\in [n]$ and any $\ell \in [n]$, $m_{\ell,j} \geq y_{\ell} - f_j(x)$ by constraint (4) of \eqref{eq:vsums-OP}, and also $m_{\ell,j} \geq 0$ by constraint (5) of \eqref{eq:vsums-OP}.
    Therefore, $m_{\ell,j} \geq \max(0,y_{\ell} -f_j(x))$.
    And so, by Lemma \ref{lemma:comp-to-p3-m-sums}:
    \begin{align}\label{eq:p3-to-conp-m-sum}
        \sum_{j=1}^n m_{\ell,j} \geq  \sum_{j=1}^n \max(0,y_{\ell} -f_j(x)) = \sum_{j=1}^n \max(0,y_{\ell} -\valBy{j}{x})
    \end{align}
    Now, suppose by contradiction that $\ell y_{\ell} - \sum_{j=1}^n m_{\ell,j}\geq c$ but at the same time $\sum_{i=1}^{\ell} \valBy{i}{x}< c$ (equation \ref{eq:p3-to-comp}).
    Since $\ell y_{\ell} - \sum_{j=1}^n m_{\ell,j}\geq c$, by equation \ref{eq:p3-to-conp-m-sum} also:
    \begin{align*}
        c \leq \ell y_{\ell} - \sum_{j=1}^n m_{\ell,j} \leq \ell y_{\ell} - \sum_{j=1}^n \max(0,y_{\ell} -\valBy{j}{x})
    \end{align*}
    But, as $\sum_{i=1}^{\ell} \valBy{i}{x}< c$ this lead to contradiction:
\begin{align*}
       &\sum_{i=1}^{\ell} \valBy{i}{x} < c \leq \ell y_{\ell} - \sum_{j=1}^n \max(0,y_{\ell} -\valBy{j}{x})\\
       \Rightarrow \Hquad & \ell y_{\ell} - \sum_{i=1}^{\ell} \valBy{i}{x} - \sum_{j=1}^n \max(0,y_{\ell} -\valBy{j} {x}) > 0\\
       \Rightarrow \Hquad & \sum_{i=1}^{\ell} y_{\ell} - \sum_{i=1}^{\ell} \valBy{i}{x} - \sum_{j=1}^n \max(0,y_{\ell} -\valBy{j} {x}) > 0\\
       \Rightarrow \Hquad & \sum_{i=1}^{\ell}\left( y_{\ell} - \valBy{i}{x} \right) - \sum_{j=1}^{\ell} \max(0,y_{\ell} -\valBy{j} {x}) - \sum_{j=\ell+1}^n \max(0,y_{\ell} -\valBy{j} {x}) > 0\\
        \Rightarrow \Hquad &  \sum_{j=1}^{\ell} \underbrace{\left((y_{\ell} - \valBy{j}{x}) - \max(0,y_{\ell} -\valBy{j}{x})\right)}_{\text{each element } \leq 0} - \sum_{j=\ell+1}^n \underbrace{\max(0,y_{\ell} -\valBy{j}{x})}_{\text{each element } \geq 0} >  0\\
     \Rightarrow \Hquad & 0 > 0
   \end{align*}

    Now, as constraint (2) of problem \eqref{eq:vsums-OP} is satisfied, for any $\ell \in [t-1]$,  $\ell y_{\ell} - \sum_{j=1}^n m_{\ell,j}\geq  \sum_{i=1}^{\ell} z_i$, and so by equation \ref{eq:p3-to-comp}, also $\sum_{i=1}^{\ell} \valBy{i}{x} \geq \sum_{i=1}^{\ell} z_i$.
    This implies that $x$ satisfies constraint $(\Tilde{2})$ of \eqref{eq:compact-OP}.
    Similarly, as constraint (3) of problem \eqref{eq:vsums-OP} is satisfied,  $t y_{t} - \sum_{j=1}^n m_{t,j}\geq  \sum_{i=1}^{t} z_i$, and so by equation \ref{eq:p3-to-comp}, also $\sum_{i=1}^{t} \valBy{i}{x} \geq \sum_{i=1}^{t} z_i$.
    This implies that $x$ and $z_t$ satisfy constraint $(\Tilde{3})$ of \eqref{eq:compact-OP}.
\end{proof}


\begin{lemma}\label{lemma:comp-to-p3-obj}
    $(x,z_t)$ and $B((x,z_t))$ obtain the same objective value from the problems \eqref{eq:compact-OP} and \eqref{eq:vsums-OP} respectively.
\end{lemma}

\begin{proof}
    As in both problems the objective value is determined by $z_t$, by the definition of $B$ (the variable $z_t$ is mapped to itself), it is clear that $(x,z_t)$ and $B((x,z_t))$ obtains the same objective value from \eqref{eq:compact-OP} and \eqref{eq:vsums-OP} respectively.
\end{proof}


%--------------------------------------------------
\subsubsection{Relationship Between the Problems \eqref{eq:basic-OP} and \eqref{eq:sums-OP}} 
% Both problems are depended on a set of constants $z_1, \ldots, z_{t-1}$, 
We shall now prove Lemma \ref{lemma:alg-1-can-use-sums-exact} (Section \ref{sec:algo-short}), which says that in Algorithm \ref{alg:basic-ordered-Outcomes}, a solver for \eqref{eq:sums-OP} can be used (instead of for \eqref{eq:basic-OP}), and the algorithm will still output a leximin optimal solution.

\begin{proof}[Proof of Lemma \ref{lemma:alg-1-can-use-sums-exact}]
    Contrariwise, suppose that the returned solution, $x^*$, is not leximin optimal.
    This means that there exists a solution, $y \in S$, that leximin-preferred over it.
    That is, there exists an integer $k \in [n]$ such that:
    \begin{align*}
        \forall j < k \colon &\valBy{j}{y} = \valBy{j}{\retSol};\\
        & \valBy{k}{y} > \valBy{k}{\retSol}.
    \end{align*}
    In addition, since $x^*$ is the returned solution, it is the solution of \eqref{eq:sums-OP} that was solved in the last iteration and therefore $\sum_{i=1}^{s} \valBy{i}{s} \geq \sum_{i=1}^{s} z_i$ for any $s \in [n]$ (by constraint  $(\hat{2})$ for $s<n$ and constraint  $(\hat{3})$ for $s=n$).
    Now, consider \eqref{eq:sums-OP} that was solved in iteration $t$.
    Since $y$ is a solution ($y \in S$) it satisfies constraint (1).
    It is also easy to see that $y$ satisfies constraint $(\hat{2})$ --- for any $s \in [k-1]$:
    \begin{align*}
        &\sum_{i=1}^s \valBy{i}{y} = \sum_{i=1}^s \valBy{i}{\retSol}
        && \text{since } i\leq s<k \text{ and $y$'s def.}\\
        & \geq \sum_{i=1}^s z_i
    \end{align*}
    Moreover, since $z_t$ is a variable in this problem, it satisfies constraint $(\hat{3})$ with any $z_t \geq \sum_{i=1}^t \valBy{i}{y} - \sum_{i=1}^{t-1} z_i$.
    Therefore, it is feasible to this problem. 
    But, the objective value obtained by $y$ is higher than the optimal value, $z_t$, which is a contradiction:
    \begin{align*}
        \sum_{i=1}^t \valBy{i}{y} - \sum_{i=1}^{t-1} z_i > \sum_{i=1}^t \valBy{i}{x^*} - \sum_{i=1}^{t-1} z_i \geq \sum_{i=1}^t z_t - \sum_{i=1}^{t-1} z_i = z_t
    \end{align*}
\end{proof}
% We start by proving that, for $t \in [n]$, when the constants $z_1, \ldots, z_{t-1}$ represent the optimal values of \eqref{eq:basic-OP} in iterations $1, \ldots t$ respectively, the programs \eqref{eq:basic-OP} and \eqref{eq:sums-OP} are equivalent.

\eden{alternative: is this better?
\begin{proof}
    In Section \ref{sec:algo-sec-proofs}, it was proven that if Algorithm \ref{alg:basic-ordered-Outcomes} uses an $(\multApprox, \additiveApprox)$-approximate solver for \eqref{eq:compact-OP} as \textsf{OP}, then the returned solution is an $(\multApprox, \additiveApprox)$-approximation to leximin. 
    This means that, given an exact solver to \eqref{eq:compact-OP}, the algorithm will output a leximin optimal solution.
    However, we saw that \eqref{eq:sums-OP} and \eqref{eq:compact-OP} are equivalent and that the identity function is an appropriate bijection (Section \ref{sec:prob-sums-and-comp}).
    Therefore, in each iteration, a solver for \eqref{eq:sums-OP} will output the same solution and the same result will be obtained.
\end{proof}
}

\eden{in the next version we can also prove it in a maybe more interesting way.. that when the constants $z_1, \ldots, z_{t-1}$ represent the optimal values of \eqref{eq:basic-OP} the programs are equivalent}

\section{Ellipsoid Method Variant for Approximation}\label{sec:mult-variant-ellipsoid}
This Appendix describes a variant of the ellipsoid method that can be used to approximate  LPs that cannot be solved directly due to a large number of variables.
% It requires an approximate separation oracle for the dual program.
The method combines techniques presented in \cite{grotschel_geometric_1993,grotschel_ellipsoid_1981,karmarkar_efficient_1982}.

\subsection{Using Approximate Separation Oracles (multiple error)}
Our goal is to solve the following linear program (the primal):
\begin{align}
\tag{P}
\begin{split}
\min \quad &c^T \cdot x \\
s.t. \quad &A \cdot x \geq b, \quad x\geq 0;
\end{split}
\end{align}
We assume that (P) has a small number of constraints, but may have a huge number of variables, so we cannot solve (P) directly. We consider its \emph{dual}:
\begin{align}
\tag{D}
\begin{split}
\max \quad & b^T \cdot y \\
s.t. \quad &A^T \cdot y \leq c,\quad y\geq 0.
\end{split}
\end{align}
Assume that both problems have optimal solutions and denote the optimal solutions of (P) and (D) by $x^{*}$ and $y^{*}$ respectively. By the strong duality theorem:
\begin{align}
    c^T \cdot x^{*} = b^T \cdot y^{*}
\end{align}

While (D) has a small number of variables, it has a huge number of constraints, so
we cannot solve it directly either. 
In this Appendix, we show that it can be approximately using the following tool:

\begin{definition}
An \emph{approximate separation oracle} with multiplicative error (MASO) for the dual LP is an efficient function parameterized by a constant $\multError \geq 0$.
Given a vector $y$  it returns one of the following two answers:
\begin{enumerate}
\item "$y$ is infeasible". In this case, is returns a violated constraint, that is, a row $a_i^T \in A^T$ such that $a_i^T  y > c_i$.
\item "$y$ is \emph{approximately feasible}". 
That means that $A^T y \leq (1+\multError) \cdot c$
\end{enumerate}

\end{definition}
Given the MASO, we apply the ellipsoid method as follows (this is just a sketch
to illustrate the way we use the MASO; it omits some technical details):
\begin{itemize}
    \item Let $E_0$ be a large ellipsoid, that contains the entire feasible region, that is, all $y \geq 0$ for which $A^T y \leq c$.

    \item For $k = 0,1,\dots, K$ (where $K$ is a fixed constant, as will be explained later):
    \begin{itemize}
        \item Let $y_k$ be the centroid of ellipsoid $E_k$.
        
        \item Run the MASO on $y_k$.
        
        \item If the MASO returns "$y_k$ is infeasible" and a violated constraint $a_i^T$, then make a \emph{feasibility cut} --- keep in $E_{k+1}$ only those $y \in E_k$ for which $a_i^T y \leq c_i$.
        
        \item If the MASO returns "$y$ is approximately feasible", then make an \emph{optimality cut} --- keep in $E_{k+1}$ only those $y \in E_k$ for which $b^T y \geq b^T y_k$.
    \end{itemize}
    
    \item From the set $y_0, y_1, \dots, y_K$, choose the point with the highest $b^T \cdot y_k$ among all the approximately-feasible points.
\end{itemize}
Since both cuts are through the center of the ellipsoid, the ellipsoid dilates by a factor of at least $\frac{1}{r}$ at each iteration, where $r > 1$ is some constant (see \cite{grotschel_ellipsoid_1981} for computation of $r$). Therefore, by choosing $K := \log_2 r \cdot L$, where $L$ is the
number of bits in the binary representation of the input, the last ellipsoid $E_K$ is so small that all points in it can be considered equal (up to the accuracy of the binary representation).


The solution $y'$ returned by the above algorithm satisfies the following two conditions:
\begin{equation} \label{mult:y-star-is-approximetly-feasible}
     A^T y' \leq (1+\multError)\cdot c
\end{equation}
\begin{equation} \label{mult:y-star-obj-geq-opt}
     b^T y' \geq b^T y^{*}
\end{equation}
Inequality \ref{mult:y-star-is-approximetly-feasible} holds since, by definition, $y'$ is approximately-feasible.

To prove \ref{mult:y-star-obj-geq-opt}, suppose by contradiction that $b^T y^{*} > b^T y'$. 
Since $y^{*}$ is feasible for (D), it is in the initial ellipsoid. 
It remains in the ellipsoid throughout the algorithm: it is removed neither by a feasibility cut (since it is
feasible), nor by an optimality cut (since its value is at least as large as all values used for optimality cuts).
Therefore, it remains in the final ellipsoid, and it is chosen as the highest-valued feasible point rather than $y'$ --- a contradiction.

Now, we construct a reduced version of (D), where there are only at most $K$ constraints --- only the constraints used to make feasibility cuts.
Denote the reduced constraints by $A_{red}^T \cdot y \leq c_{red}$, where $A_{red}^T$ is a matrix containing a subset of at most $K$ rows of of $A^T$, and $c_{red}$ is a vector containing the corresponding subset of the elements of $c$. The reduced-dual LP is:
\begin{equation}
\tag{RD}
\begin{split}
\max  \quad & b^T y \\
s.t. \quad & A_{red}^T \cdot y \leq c_{red}, \quad y\geq 0
\end{split}
\end{equation}
Notice that it has the same number of variables as the program (D). Further, if we had run this ellipsoid method variant on (RD) (instead of (D)), then the result would have been exactly the same --- $y'$.
Therefore, (\ref{mult:y-star-obj-geq-opt}) holds for the (RD) too:
\begin{equation} \label{mult:y-star-to-y-redopt}
    b^T y' \geq b^T y^{*}_{red}
\end{equation}
where $y^{*}_{red}$ is the optimal value of (RD).


As $A_{red}^T$ contains a subset of at most $K$ rows of $A^T$, the matrix $A_{red}$ contains a subset of \emph{columns} of $A$.
Therefore, the dual of (RD) has only at most $K$ variables, which are those who correspond to the remaining columns of $A$:
\begin{equation}
	\tag{RP}
    \begin{split}
     \min \quad &c_{red}^T \cdot x_{red} \\
            s.t. \quad &A_{red} \cdot x_{red} \geq b, \quad x_{red}\geq 0
    \end{split}
\end{equation}
%  reduced-primal
%\er{Note that $A_{red}$ is a matrix with the same number of rows as $A$, but only at most $K$ columns.}
Since (RP) has a polynomial number of variables  and constraints, it can be solved exactly by any LP solver (not necessarily the ellipsoid method).
Denote the optimal solution by $x^{*}_{red}$. 

Let $x'$ be a vector which describes an assignment to the variables of (P), in which all variables that exist in (RP) have the same value as in $x^{*}_{red}$, and all other variables are set to $0$.
It follows that $A \cdot x' = A_{red} \cdot x^{*}_{red}$, therefore, since $x^{*}_{red}$ is feasible to RD, also $x'$ is a feasible solution to (P).
\erel{
In second reading, I think this should be made more formal.
Let $x'$ be a solution to (P), in which all variables that exist in (RP) have the same value as in $x^{*}_{red}$, and all other variables are set to 0.
We have to prove that 
(1) $x'$ is feasible for (P);
(2) $c^T x' \leq (1+\epsilon)\cdot c^T\cdot x^{*}$.
}
\eden{better?}
Similarly, $c^T \cdot x' = c^T_{red} \cdot x^{*}_{red}$.
We shall now see that this implies that the objective obtained by $x'$ approximates the objective obtained by $x^{*}$:
\begin{align*} 
&c^T \cdot x' = c^T_{red} \cdot x^{*}_{red} \\
&=  b^T \cdot y^{*}_{red} & \text{(by strong duality for the reduced LPs)} \\
                     &\leq  b^T\cdot y' & \text{(By (\ref{mult:y-star-to-y-redopt}))}\\
                     &\leq  (A \cdot x^{*})^T y' & \text{(definition of (P))} \\
                     &=  (x^{*})^T (A^T\cdot y') & \text{(properties of transpose and associativity of multiplication)} \\
                     &\leq  (x^{*})^T ((1+\multError)\cdot c) & \text{(by \ref{mult:y-star-is-approximetly-feasible})} \\
                     & = (1+\multError) \cdot (c^T x^{*}) & \text{(properties of transpose)}
\end{align*}
So, $x'$ ($x^{*}_{red}$ with all missing variables set to $0$) is an approximate solution to the primal LP (P) --- as required.

\subsection{Using Half-Randomized Approximate Separation Oracles}
Here, we allow the oracle to also be \emph{half-randomized}, that is, when it says that a solution is infeasible, it is always correct; however, when it says that a solution is approximately feasible, it is only correct with some probability $p \in [0,1]$.

Since the ellipsoid method variant is iterative, and since the oracle calls are independent, there is a probability $p^T$ that the oracle answers correctly in each iteration, and so, the overall process performs as before. 
We shall now explain why, using a half-randomized oracle, this ellipsoid method variant \emph{always} returns a feasible solution to the primal (even if the oracle was incorrect).

First, notice that the oracle is always correct when it determines that a solution is infeasible.
In addition, the construction of RD is only depended by these set of constraints.
Therefore, by the same arguments, $x'$ would still be a feasible solution to P (but not necessarily with an approximately-optimal objective value).

This means that given a half-randomized approximate separation oracle for the dual with error $\multError$ and success probability $p$, this ellipsoid method variant can be used as a randomized approximation algorithm for the primal with the same error and success probability $p^I$ (where $I$ is an upper bound on the number of iteration of the method on the given input). 
% \section{Saturation Algorithm}\label{sec:saturation-algorithm}
The following algorithm was independently proposed by different researchers for different problems \cite{willson,airiau_portioning_2019,nace_max-min_2008}.
% --- by Willson \cite{willson} for the  problem of fair allocation of divisible items, Airiau et al. \cite{airiau_portioning_2019} for the problem of portioning with ordinal preferences, Bei at el. \cite{bei_truthful_2022} for a variant of cake cutting that they called cake sharing and Nace and Pioro \cite{nace_max-min_2008} for multi-commodity flow problem 
But it can be generalized to capture the following case:
\begin{enumerate}
    \item The feasible region $S$ is \textit{convex}: for any two solutions $x, y \in S$ and for any $\lambda \in [0,1]$, the convex combination of $x$ and $y$ in relation to $\lambda$ is also a solution:
    \begin{align*}
        \forall x, y \in S, \quad \forall \lambda \in [0,1] \colon \quad  
        \bigl(\lambda x + (1-\lambda)y\bigl)\in S
    \end{align*}

    \item The size of the feasible region $S$ is polynomial with respect to $n$.
    % \eden{To myself: to check if this is accurate: i.e., it can be described with a number of variables and constraints that is polynomial to $n$.}

    \item The objective-functions are \textit{additive}: let $x,y,z \in S$ be solutions for which $\alpha,\beta \in \mathbb{R}$ exist such that $z = \alpha x + \beta y$, then for each objective-function $f_i \in \allObjFunc$:
    \begin{align*}
        f_i(z) &= f_i(\alpha x + \beta y) =\\
        &= \alpha f_i(x) + \beta f_i(y)
    \end{align*}
    \erel{ 
    For this condition, we must say that the solutions are vectors (we did not say this so far). Otherwise there is no meaning to adding or multiplying by scalars.
    }

    \item The objective-functions are \textit{concave}: for any objective-function $f_i \in \allObjFunc$ the set $\{f(x) \mid x \in S\}$ is concave (equivalently, the set $\{-f(x) \mid x \in S\}$ is convex). 

    \item There is a black-box for finding  the \textit{next maximin} value (denote by $OP1$): given a subset of objective-functions ($\mathcal{A}\subset \allObjFunc$) for which lower bounds have been set ($\forall f_i \in \mathcal{A} \colon z_i \in \mathbb{R}$), finds the highest value that all other objective functions can achieve simultaneously:
    \begin{align*}
        \max \quad &z\\
        s.t. \quad  & x \in S\\
                    & f_i(x) = z_i   & f_i \in \mathcal{A}\\
                    & f_i(x) \geq z   & f_i \notin \mathcal{A}
    \end{align*}

    \item There is a black-box for solving a saturation test (denote by $OP2$):
    % \eden{I think we should name this process, but I'm not sure if it is the best name...}: 
    For each objective-function $f_k \in \allObjFunc$, a single-objective optimization version of the problem with lower bounds on the values of the other objectives ($\forall f_i \in \mathcal{A} \colon z_i \in \mathbb{R}$ and $z \in \mathbb{R}$):
    \begin{align*}
    \max \quad &f_i(x)\\
    s.t. \quad  & x \in S\\
                    & f_i(x) = z_i   & f_i \in \mathcal{A}\\
                    & f_i(x) \geq z   & f_i \notin \mathcal{A}
    \end{align*}
\end{enumerate}
The algorithm is described in detail (in our terms and notations) in Algorithm \ref{alg:willson-leximin}. 


\begin{algorithm}[!htbp]
\caption{Saturation Algorithm--- for finding the Leximin optimal solution}
\label{alg:willson-leximin}
% \textbf{Input}: A black-box for OP1 and a black-box for OP2\\
% \textbf{Output}: The Lexical optimal solution
\begin{algorithmic}[1] %[1] enables line numbers 
\STATE Initialize the set of \textit{saturated} objective-functions $\mathcal{A} = \{\}$ and initialize $t=0$ (a step counter).

\STATE increase $t$ ($t = t+1$).

\STATE Use the black-box for $OP1$ to solve the following  problem, where the variables are $x$ (a vector) and $v$ (a scalar): 
\begin{align*}
\max \quad &v\\
        s.t. \quad  & x \in S\\
                    & f_i(x) \geq z_i   & f_i \in \mathcal{A}\\
                    & f_i(x) \geq v   & f_i \notin \mathcal{A}
\end{align*}
Let $x_t$ and $v_t$ be the optimal solution. 
    
\FOR{$f_k \notin \mathcal{A}$}
    \STATE Use the black-box for $OP2$ to solve the following problem, where the variables are $x$ (a vector) and $v$ (a scalar):
    \begin{align*}
    \max \quad & v\\
            s.t. \quad  & x \in S\\
                        & f_i(x) \geq z_i   & f_i \in \mathcal{A}\\
                        & f_i(x) \geq v_t   & f_i \notin \mathcal{A}\\
                        & f_k(x) \geq v
    \end{align*}
    Let $x_t^k$ and $v_t^k$ be the optimal solution. 

    \STATE \textbf{if} $v_t^k = v_t$ \textbf{then} set $f_k$ as \textit{saturated}: add it to $\mathcal{A}$ ($\mathcal{A} = \mathcal{A} \cup \{f_k\}$) and set its value to $v_t$ ($z_k = v_t$).
    % \IF{$z_{max}^k = z_{max}$}
        % \STATE Set $f_k$ as saturated: add it to $\mathcal{A}$ ($\mathcal{A} = \mathcal{A} \cup \{f_k\}$) and set its value to $z_{max}$ ($z_k = z_{max}$).
    % \ENDIF
\ENDFOR
\STATE \textbf{if} $|\mathcal{A}| = n$ \textbf{then} return $x_t$ \textbf{else} Goto line 2.
% \IF{$|\mathcal{A}| = n$}
    % \STATE return $x$ \eden{To myself: the return part of all algorithms should be explain better}
% \ENDIF
\end{algorithmic}
\end{algorithm}

The algorithm keeps a set of objective-functions that are saturated ($\mathcal{A}$) and lower bounds on their values ($\forall f_i \in \mathcal{A} \colon z_i \in \mathbb{R}$). 
The set is initially empty. 
At each iteration, at least one function becomes saturated and its lower bound is set.
When all functions become saturated, the algorithm terminates.
Each iteration of the algorithm can be divided into two parts.
In the first part, the first black-box is used to find the \textit{next max-min} value, which is the maximum value that all functions outside of $\mathcal{A}$ can achieve at the same time, given that all functions within $\mathcal{A}$ achieve their lower bounds.
In the second part, \textit{a saturation test} is made.
% \eden{I think we should name this process, but I'm not sure if it is the best name...}
For every function not in $\mathcal{A}$, the second black-box is used to find the maximum value of this function when all saturated functions ($f_i \in \mathcal{A}$) achieve their lower bounds and all other functions (outside of $\mathcal{A}$) achieve at least the max-min value from the first part.
This value is used to determine if this objective function is saturated, that is, if its maximal value from the saturation test is equal to the max-min value obtained in the first fart.
If so, we add it to the set of saturated objective-functions ($\mathcal{A}$) and set its lower bound to this value.

% \section{Additive Variant}\label{sec:additive}

\begin{theorem}\label{thm:leximin-approx-alg-leximin-opt}
    Let $\epsilon \in [0,1]$ and \textsf{OP} be a procedure that outputs a $\epsilon$ \emph{additive} approximation to \eqref{eq:vsums-OP}. Then Algorithm \ref{alg:basic-ordered-Outcomes} outputs a $\epsilon$ additive-approximate Leximin solution.  
\end{theorem}

\begin{proof}
    Contrariwise, suppose that $\retSol$ is not an $\epsilon$-approximately Leximin-optimal solution.
    This means that there exists a solution $y$ that is $\epsilon$-preferred over $\retSol$.
    That is, there exists an integer $k \in [n]$ such that:
    \begin{align*}
        \forall j < k \colon &\valBy{j}{y} \geq \valBy{j}{\retSol};\\
        & \valBy{k}{y} > \valBy{k}{\retSol} + \epsilon.
    \end{align*}
    We get that for all $s \in [k-1]$:
    \begin{align*}
         &\sum_{i=1}^s \valBy{i}{y} \geq \sum_{i=1}^s \valBy{i}{\retSol}
        && \text{since } i\leq s<k \text{ and $y$'s def.}\\
        & \geq \sum_{i=1}^s z_i && \text{constraint (2) for $t=n$.}
    \end{align*}
    Therefore, $y$ is a solution to the OP that was solved when $t = k$.
    \erel{You proved that $y$ satisfies constraint (2), but what about constraint (3)?}
    \eden{I'm not sure how to explain that constraint (3) is not a \textbf{standard} constraint. It determines the objective value $z$, so although it is not always optimal, it is always valid.}
    
    In addition, either $k<n$ or $k=n$. 
    If $k<n$ then constraint (2) for $t=n$ says that:
    \begin{align}\label{equ:approx-sum-k-geq-z-1}
        \sum_{i=1}^k \valBy{i}{\retSol} \geq \sum_{i=1}^k z_i
    \end{align}
    If $k=n$ then since $z=z_n$ constraint (3) says it.
    In both cases, we know that equation \ref{equ:approx-sum-k-geq-z-1} holds.
    
    And so, we get that:
    \begin{align*}
         \sum_{i=1}^k \valBy{i}{y} &= \sum_{i=1}^{k-1} \valBy{i}{y} + \valBy{k}{y}\\
         &  \geq\sum_{i=1}^{k-1}\valBy{i}{\retSol} + \valBy{k}{y} &&  \text{since } i \leq k-1 < k \text{ and $y$'s def.}\\
        & > \sum_{i=1}^{k-1}\valBy{i}{\retSol} + \valBy{k}{\retSol} + \epsilon &&  \text{$y$'s def. for } k
        \\
        & = \sum_{i=1}^{k}\valBy{i}{\retSol} + \epsilon \\
        & \geq \sum_{i=1}^{k} z_i + \epsilon &&  \text{equation } \ref{equ:approx-sum-k-geq-z-1}
    \end{align*}
    Which simply means that:
    \begin{align}\label{equ:sum-y-geq-sum-z-plus-eps}
         \sum_{i=1}^k \valBy{i}{y} > \sum_{i=1}^{k} z_i +\epsilon
    \end{align}
    That means that the $z$ achieved by the solution $y$ in the OP that was solved when $t = k$ is strictly more than the value we achieved $z_k$ plus $\epsilon$:
    \begin{align*}
        &\sum_{i=1}^k \valBy{i}{y} - \sum_{i=1}^{k-1} z_i && \text{insulated } z\\
        &> \sum_{i=1}^{k} z_i + \epsilon - \sum_{i=1}^{k-1} z_i  && \text{equation } \ref{equ:sum-y-geq-sum-z-plus-eps} \\
        &= z_k + \epsilon
    \end{align*}
    But we know that the error in this OP is at most $\epsilon$ --- a contradiction.
\end{proof}
\begingroup
\let\clearpage\relax 
% \onecolumn %%% For
\onecolumn

\appendices 
\setcounter{lemma}{0}
\setcounter{proposition}{0}
\setcounter{theorem}{0}
\setcounter{definition}{0}
\setcounter{assumption}{0}

\section*{Introduction to Notations and Preliminaries used in the Proofs}\label{app:notations}
\noindent The subnet noise-free variable before global synchronization is introduced as follows:
\begin{equation}\label{eq:v_c}
    \bar{\mathbf v}_c^{(t+1)}= \bar{\mathbf v}_c^{(t)}-\eta_{k}\nabla \bar{F}_c(\bar{\mathbf v}_c^{(t)}),~\forall t\in\mathcal{T}_k\setminus{\{t_k\}},
\end{equation}
% \nm{what about global sync??} 
with the subnet noise-free variable at global synchronization is defined as 
\begin{align}
    \bar{\mathbf v}_c^{(t_{k+1})}
    =& (1-\alpha)\bar{\mathbf v}^{(t_{k+1}-\Delta)}
    +\alpha\widetilde{\mathbf v}_c^{(t_{k+1})},
\end{align}
where $\widetilde{\mathbf v}_c^{(t_{k+1})}$ is the noise-free variable right before global synchronization, as opposed to $\mathbf v_c^{(t_{k+1})}$ defined right after global synchronization. Similarly, the global noise-free variable is defined as  
\begin{equation}
    \bar{\mathbf v}^{(t+1)}=\sum\limits_{d=1}^N\varrho_{d}\bar{\mathbf v}_d^{(t+1)}~~\forall t\in\mathcal{T}_k.
\end{equation}
% \nm{please move def of e1 e2 e3 here}
The following noise terms used in the appendices are defined as follows:
\begin{align}
 &\label{eq:def_e1}   e_1^{(t)}\triangleq\Big(\mathbb E\Big[\sum\limits_{c=1}^N\varrho_c\sum\limits_{j\in\mathcal S_c}\rho_{j,c}\Vert\mathbf w_i^{(t)} - \bar{\mathbf v}_c^{(t)}\Vert^2\Big]\Big)^{1/2},\\
&\label{eq:def_e2}       e_2^{(t)} \triangleq\sum\limits_{c=1}^N\varrho_{c}\Vert\bar{\mathbf v}_c^{(t)}-\bar{\mathbf v}^{(t)}\Vert,
\\
&\label{eq:def_e3}
        e_3^{(t)}\triangleq\Vert\bar{\mathbf v}^{(t)}-\mathbf w^*\Vert.
    \end{align}
\section{Proof of Theorem~\ref{thm:subLin}} \label{app:subLin} 
% \nm{MAIN PROOF}
\begin{theorem} \label{thm:subLin}
% \nm{I think you need to define all the quantities here as well...it is difficult to read when you are pointing to equations that appear somewhere else..}
    Under Assumptions \ref{beta},~\ref{assump:SGD_noise} and~\ref{assump:sub_err}, if $\eta_{k}=\frac{\eta_{\mathrm{max}}}{1+\gamma k},~\forall k$ and $\vert\mathcal T_k\vert\leq\tau,~\forall k$, using {\tt DFL} for ML model training, the distance between the global model and the optimum at global synchronization can be bounded as 
    \begin{align}
        \mathbb E[\Vert\bar{\mathbf w}^{(t_k)}-\mathbf w^*\Vert^2]
        \leq 2Y_1^2 \eta_k+2Y_3^2\eta_k^2,
    \end{align}
    % \nm{??? Isnt it $\mathbb E[\Vert\bar{\mathbf w}^{(t_k)}-\mathbf w^*\Vert^2]\leq (e1+e3)^2\leq 2e1^2+2e3^2\leq 2Y_1^2 \eta_k+2Y_3^2\eta_k^2$? I think you need to keep them separate since they show a stronger result: the impact of delta decays faster with k.}
    where 
    $\eta_{\mathrm{max}}<\min\left\{\frac{2}{\beta+\mu},\frac{(\tau-\Delta)\mu}{\beta^2[(1+\lambda_+)^{\tau}-1-\tau\lambda_+]}\right\}$, $\gamma<\min\left\{1-(1-\mu\eta_{\mathrm{max}})^{2(\tau-\Delta)},C_3\eta_{\mathrm{max}}\beta\right\}$,
    \begin{align}
        \alpha<\alpha^* \triangleq \frac{1}{\frac{C_2\eta_{\max}^2}{\eta_{\max}\beta C_3-\gamma}
         2\omega C_2(1+\gamma)
        +(1+\gamma)(1+\lambda_+)^{\tau}},
    \end{align}
    \begin{align}\label{eq:Y1}
        Y_1 \triangleq\sqrt{\frac{(\tau-(1-\alpha)\Delta)(\sigma^2+\phi^2)\eta_{\mathrm{max}}}{C_1-\gamma}},
    \end{align}
    \begin{align}\label{eq:Y2}
    Y_2\triangleq\max\left\{
    \frac{\eta_{\mathrm{max}}^2\alpha 2\omega C_2(1+\gamma)
    e_3^{(0)}
    +\alpha K_1\delta(1+\gamma)
    }{1-\alpha(1+\gamma)(1+\lambda_+)^{\tau}},
    \frac{\frac{ K_1\delta}{\eta_{\mathrm{max}} 2\omega C_2}
    +\frac{K_2\delta}{\beta C_3-\gamma}
    }{
    \frac{[1-\alpha(1+\gamma)(1+\lambda_+)^{\tau}]}{\eta_{\mathrm{max}}\alpha 2\omega C_2(1+\gamma)}
    -\frac{C_2}{\beta C_3-\gamma}
    }\right\},
\end{align}
    % \begin{align}
    %     Y_2 \triangleq\frac{\eta_{\mathrm{max}}K_1\nm{K1(alpha??)}Y_3(1+\gamma)+K_2\delta(1+\gamma)}{1-\alpha(1+\gamma)(1+\lambda_+)^{\tau}},
    % \end{align}
    \begin{align}\label{eq:Y3}
        Y_3 \triangleq \max\left\{\eta_{\mathrm{max}}e_3^{(0)},
        \frac{[C_2Y_2+K_2\delta]\eta_{\max}}{\eta_{\max}\beta C_3-\gamma}\right\}.
    \end{align}
    with $K_1=\frac{\mu}{-\beta\lambda_+\lambda_-}[(1+\lambda_+)^{\tau}-1]$, $K_2=\frac{\beta}{\sqrt{1+8\omega}}\sum_{\ell=0}^{\tau-2}\Big(\begin{array}{c}\tau\\\ell+2\end{array}\Big)[\lambda_+^{\ell+1}-\lambda_-^{\ell+1}]$, $C_1=1-((1-\alpha)(1-\mu\eta_{\mathrm{max}})^{2(\tau-\Delta)}+\alpha(1-\mu\eta_{\mathrm{max}})^{2\tau})$, $C_2=\frac{2\beta}{\sqrt{8\omega+1}}[(1+\lambda_+)^{\tau}-1]$, $C_3=(\tau-\Delta)\mu/\beta 
        - \eta_{\mathrm{max}}\beta[(1+\lambda_+)^{\tau}-1-\tau\lambda_+]$ and $\lambda_{\pm} =\frac{1}{2}-\frac{\mu}{\beta}\pm\frac{\sqrt{8\omega+1}}{2}$.
    % \nm{you need to tell where these vars are defined.}
    % \nm{since K1 and K2 are prop to alphha redefine $K_1\gets K_1(1)$, so that $K_1(\alpha)$ becomes $K_1\alpha$. Same for K2.}
\end{theorem}
\begin{proof}
Note that
\begin{align} 
    &\sqrt{\mathbb E[\Vert\bar{\mathbf w}^{(t_{k})}-\mathbf w^*\Vert^2]} 
    =\sqrt{\mathbb E[\Vert\bar{\mathbf w}^{(t_{k})}-\bar{\mathbf v}^{(t_{k})}+\bar{\mathbf v}^{(t_{k})}-\mathbf w^*\Vert^2]} 
    \\&
    \leq
    \sqrt{\mathbb E[\Vert\bar{\mathbf w}^{(t_{k})}-\bar{\mathbf v}^{(t_{k})}\Vert]}+\Vert\bar{\mathbf v}^{(t_{k})}-\mathbf w^*\Vert
    \\&
    =
    \sqrt{\mathbb E[\Vert\sum\limits_{c=1}^N\varrho_c\sum_{j\in\mathcal S_c}\rho_{j,c}({\mathbf w}_j^{(t_{k})}-\bar{\mathbf v}_c^{(t_{k})})\Vert^2]}+\Vert\bar{\mathbf v}^{(t_{k})}-\mathbf w^*\Vert
    \\&
    \leq
    \sqrt{\sum\limits_{c=1}^N\varrho_c\sum_{j\in\mathcal S_c}\rho_{j,c}\mathbb E[\Vert{\mathbf w}_j^{(t_{k})}-\bar{\mathbf v}_c^{(t_{k})}\Vert^2]}+\Vert\bar{\mathbf v}^{(t_{k})}-\mathbf w^*\Vert
    = e_1^{(t_k)}+e_3^{(t_k)},
\end{align}
and therefore
\begin{align}\label{eq:final}
     \mathbb E[\Vert\bar{\mathbf w}^{(t_k)}-\mathbf w^*\Vert^2]\leq (e_1^{(t_k)}+e_3^{(t_k)})^2\leq 2(e_1^{(t_k)})^2+2(e_3^{(t_k)})^2.
\end{align}
We now show by induction that $e_1^{(t_k)}\leq Y_1\sqrt{\eta_k}$, $e_2^{(t_k)}\leq Y_2\eta_k$ and $e_3^{(t_k)}\leq Y_3\eta_k$ with $Y_1$, $Y_2$
and $Y_3$ defined in~\eqref{eq:Y1},~\eqref{eq:Y2} and~\eqref{eq:Y3}. The conditions trivially holds at the beginning of training at $k=0$ since $Y_1\geq0$, $Y_2\geq0$ and $Y_3\geq\eta_{\mathrm{max}}e_3^{(0)}$.
% \nm{note that e2 and e3 are of order eta...Can be seen in the constatn stepsize case, asyntotuic regime}
% \nm{/init condyion here
Now, assume $e_1^{(t_k)}\leq Y_1\sqrt{\eta_k}$, $e_2^{(t_k)}\leq Y_2\eta_k$ and $e_3^{(t_k)}\leq Y_3\eta_k$ for a certain $k\geq0$. We prove the condition holds for $k+1$ as well. 

To show $e_1^{(t_{k+1})}\leq\sqrt{\eta_{k+1}}Y_1$, we use~\eqref{eq:e1_main} of Proposition~\ref{lem:main_gap} and the induction hypothesis ($e_1^{(t_{k})}\leq\sqrt{\eta_{k}}Y_1$), yielding the sufficient condition
% For $e_1^{(t_k)}$, from~\eqref{eq:e1_main} of Proposition~\ref{lem:main_gap}, we aim to show that $(e_1^{(t_{k+1})})^2-\eta_{k+1}Y_1^2\leq 0$. Applying the induction condition into \eqref{eq:e1_main}, it is sufficient 
 \begin{align} 
&\left(1-\eta_k/\eta_{\mathrm{max}}C_1\right)\eta_{k}Y_1^2
    % \nonumber \\&
    +\eta_{k}^2(\tau-(1-\alpha)\Delta)(\sigma^2+\phi^2)
    -\eta_{k+1}Y_1^2\leq 0.
\end{align}
% \nm{you have already applyied the ind hp to get the above!}
% \hl{Applying the induction hypothesis} and 
% \nm{sentence sounds weird.. please rephrase, I would just do:}
% \sst{Transforming the condition above through the set of following algebraic steps:}
% \nm{a bit confusing and not systematic.. are you replacing the values of etak?? Then do that for ALL etak! I would say :}
Using the expression of $\eta_k=\frac{\eta_{\mathrm{max}}}{1+\gamma k}$, the above condition is equivalent to
$$
     -[C_1-\gamma]Y_1^2
    % \\&
    +\eta_{\mathrm{max}}(\tau-(1-\alpha)\Delta)(\sigma^2+\phi^2)
    -\gamma Y_1^2\frac{\gamma}{1+\gamma (k+1)}
    \leq 0.
$$
To satisfy the condition for all $k\geq 0$, the above condition is equivalent to
$$
\eta_{\mathrm{max}}(\tau-(1-\alpha)\Delta)(\sigma^2+\phi^2)
\leq 
[C_1-\gamma]Y_1^2,
$$
which is indeed verified since 
% \sst{$\gamma<1-((1-\alpha)(1-\mu\eta_{\mathrm{max}})^{2(\tau-\Delta)}+\alpha(1-\mu\eta_{\mathrm{max}})^{2\tau})\leq1-(1-\mu\eta_{\mathrm{max}})^{2(\tau-\Delta)}$}
% \nm{
$\gamma<1-(1-\mu\eta_{\mathrm{max}})^{2(\tau-\Delta)}\leq C_1$
% }
 and $Y_1^2=\frac{(\tau-(1-\alpha)\Delta)(\sigma^2+\phi^2)\eta_{\mathrm{max}}}{C_1-\gamma}.$
This completes the induction for $e_1^{(t_k)}$, showing that $(e_1^{(t_k)})^2\leq\eta_{k}Y_1^2,~\forall k$. 

% \nm{please use a similar approach for e2 as I did for e1.. Dont overcomplicate things: replace the value of etak, organize the terms properly, and then bound properly (worst case over k)}
To show $e_2^{(t_{k+1})}\leq\eta_{k+1}Y_2$, we use~\eqref{eq:x1_syn_inter2} of Proposition~\ref{lem:main_gap} and the induction hypothesis ($e_2^{(t_{k})}\leq\eta_{k}Y_2$), yielding the sufficient condition
 \begin{align} 
&\alpha(1+\lambda_+)^{\tau}\eta_k Y_2
    +\alpha 2\omega C_2\eta_k^2 Y_3
    % \nonumber\\& 
    +\alpha K_1\eta_k\delta
    -\eta_{k+1}Y_2\leq 0,
\end{align} 
Using the expression of $\eta_k=\frac{\eta_{\mathrm{max}}}{1+\gamma k}$, the above condition can be written as:
% \nm{Reorganize as
\begin{align}\label{eq:e2_finTransform}
    & \frac{\alpha(1+\lambda_+)^{\tau} Y_2}{\eta_{\mathrm{max}}}
    +\alpha 2\omega C_2Y_3
    +\frac{\alpha K_1\delta}{\eta_{\mathrm{max}}}
    - \frac{Y_2}{\eta_{\mathrm{max}}(1+\gamma)}
    - \frac{Y_2\gamma^2 k}{\eta_{\mathrm{max}}(1+\gamma(k+1))(1+\gamma)}
    - \frac{\alpha 2\omega C_2Y_3\gamma k}{1+\gamma k}
    \leq 0,
\end{align}
% which makes it immediately clear that $k=0$ is the worst case
% }
To satisfy the condition for all $k\geq 0$, the above condition is equivalent to
 \begin{align}\label{eq:e2_cond2}
    & \frac{[1-\alpha(1+\gamma)(1+\lambda_+)^{\tau}]Y_2-\alpha K_1\delta(1+\gamma)}{\eta_{\mathrm{max}}}
    -\alpha 2\omega C_2Y_3(1+\gamma)
    \geq 0.
 \end{align}
% which can be verified since $\alpha<\alpha*\leq\frac{1}{(1+\gamma)(1+\lambda_+)^{\tau}}$ and $Y_2\geq\frac{\alpha (\eta_{\mathrm{max}}2\omega C_2Y_3+K_1\delta)}{1/(1+\gamma)-\alpha(1+\lambda_+)^{\tau}}$. 

To show $e_3^{(t_{k+1})}\leq\eta_{k+1}Y_3$, we use~\eqref{eq:e3_sync} of Proposition~\ref{lem:main_gap} and the induction hypothesis ($e_3^{(t_{k})}\leq\eta_{k}Y_3$), yielding the sufficient condition
\begin{align}
    (1-\eta_k\beta C_3)Y_3\eta_k
      % \nonumber \\&
    +[C_2Y_2+K_2\delta] \eta_k^2
    - Y_3\eta_{k+1}\leq 0.
\end{align}
Using the expression of $\eta_k=\frac{\eta_{\mathrm{max}}}{1+\gamma k}$, the above condition can be written as:
\begin{align}
Y_3[\gamma -\eta_{\max}\beta C_3]
      % \nonumber \\&
    +[C_2Y_2+K_2\delta]\eta_{\max}
    % \nonumber\\&
    % +[g_{5}(\Pi_{+,t}-1)+g_{6}(\Pi_{-,t}-1)]
    % [\sigma/\beta+\sum\limits_{d=1}^N\varrho_{d}\epsilon_{d}^{(0)}] \nonumber\\&
 -Y_3\gamma^2(1+\gamma k+\gamma)\leq 0.
\end{align}
To satisfy the condition for all $k\geq 0$, the above condition is equivalent to
\begin{align} \label{eq:cond_Y3}
    Y_3[\gamma -\eta_{\max}\beta C_3]
    +[C_2Y_2+K_2\delta]\eta_{\max}
    \leq 0.
\end{align}
 To show $e_2^{(t_k)}\leq\eta_{k+1}Y_2$ and $e_3^{(t_k)}\leq\eta_{k+1}Y_3$, the conditions $e_3^{(t_k)}\geq \eta_{\mathrm{max}}e_3^{(0)}$,~\eqref{eq:e2_cond2} and~\eqref{eq:cond_Y3} need to be satisfied simultaneously. To satisfy this, we need $\gamma<C_3\eta_{\mathrm{max}}\beta$ and
 \begin{align}
     Y_3\geq \eta_{\mathrm{max}}e_3^{(0)},
 \end{align}
 \begin{align}
     Y_3\leq
    \frac{[1-\alpha(1+\gamma)(1+\lambda_+)^{\tau}]Y_2-\alpha K_1\delta(1+\gamma)}{\eta_{\mathrm{max}}\alpha 2\omega C_2(1+\gamma)},
 \end{align}
 and
 \begin{align}
     Y_3\geq\frac{[C_2Y_2+K_2\delta]\eta_{\max}}{\eta_{\max}\beta C_3-\gamma}.
 \end{align}
 Using the definition of $Y_3$ in~\eqref{eq:Y3}, the conditions above become equivalent to 
 \begin{align}
     Y_3\leq
    \frac{[1-\alpha(1+\gamma)(1+\lambda_+)^{\tau}]Y_2-\alpha K_1\delta(1+\gamma)}{\eta_{\mathrm{max}}\alpha 2\omega C_2(1+\gamma)},
 \end{align}
 yielding the sufficient conditions
 \begin{align}
     \eta_{\mathrm{max}}e_3^{(0)}\leq
    \frac{[1-\alpha(1+\gamma)(1+\lambda_+)^{\tau}]Y_2-\alpha K_1\delta(1+\gamma)}{\eta_{\mathrm{max}}\alpha 2\omega C_2(1+\gamma)},
 \end{align}
 and 
 \begin{align}
     \frac{[C_2Y_2+K_2\delta]\eta_{\max}}{\eta_{\max}\beta C_3-\gamma}\leq
    \frac{[1-\alpha(1+\gamma)(1+\lambda_+)^{\tau}]Y_2-\alpha K_1\delta(1+\gamma)}{\eta_{\mathrm{max}}\alpha 2\omega C_2(1+\gamma)},
 \end{align}
which can be verified since $\alpha<\alpha^*$ together with the definition of $Y_2$ given in~\eqref{eq:Y2}.
This completes the induction showing that $e_2^{(t_k)}\leq\eta_{k+1}Y_2$ and $e_3^{(t_k)}\leq\eta_{k+1}Y_3$. Finally, applying the result of induction for $e_1^{(t_k)}, e_2^{(t_k)}$ and $e_3^{(t_k)}$ into~\eqref{eq:final} completes the proof.
\end{proof}

\pagebreak
\section{Proof of Proposition~\ref{lem:main_gap}} \label{app:main_gap} 
% \nm{this is also an auxiliary result and should be moved after Thm1}
\begin{proposition} \label{lem:main_gap}
    Under Assumptions \ref{beta},~\ref{assump:SGD_noise} and~\ref{assump:sub_err}, if $\eta_{k}=\frac{\eta_{\mathrm{max}}}{1+\gamma k}$, where $\eta_{\mathrm{max}}<\min\left\{\frac{2}{\beta+\mu},\frac{(\tau-\Delta)\mu}{\beta^2[(1+\lambda_+)^{\tau}-1-\tau\lambda_+]}\right\}$, using {\tt DFL} for ML model training, $(e_1^{(t_{k+1})})^2$, $e_2^{(t_{k+1})}$ and $e_3^{(t_{k+1})}$ across global synchronizations can be bounded as 
\begin{align} \label{eq:e1_main}
    (e_1^{(t_{k+1})})^2
    \leq&
    \left(1-\eta_k/\eta_{\mathrm{max}}C_1\right)(e_1^{(t_k)})^2
    % \nonumber \\&
    +\eta_{k}^2 (\tau-(1-\alpha)\Delta)(\sigma^2+\phi^2),
    % \nm{ERROR, SEE\ BELOW}
\end{align} 
\begin{align} \label{eq:x1_syn_inter2}
    e_2^{(t_{k+1})}&\leq 
    % \nonumber \\&
    \alpha(1+\lambda_+)^{\tau}e_2^{(t_k)}
    +\eta_k\alpha 2\omega C_2e_3^{(t_k)}
    % \nonumber\\& 
    +\eta_k\alpha K_1\delta,
\end{align}
\begin{align} \label{eq:e3_sync}
     &  e_3^{(t_{k+1})}\leq
    (1-\eta_k\beta C_3) e_3^{(t_k)}
    +C_2 \eta_ke_2^{(t_k)}
    +\eta_k^2 K_2\delta,
\end{align}
 where 
 \begin{align}\label{eq:K1}
     C_1 \triangleq 1-((1-\alpha)(1-\mu\eta_{\mathrm{max}})^{2(\tau-\Delta)}+\alpha(1-\mu\eta_{\mathrm{max}})^{2\tau}),
 \end{align}
  \begin{align}\label{eq:C2}
     C_2\triangleq\frac{2\beta}{\sqrt{8\omega+1}}[(1+\lambda_+)^{\tau}-1],
 \end{align}
 \begin{align}\label{eq:C1}
     &C_3 \triangleq (\tau-\Delta)\mu/\beta 
        - \eta_{\mathrm{max}}\beta[(1+\lambda_+)^{\tau}-1-\tau\lambda_+],
 \end{align}
 \begin{align}\label{eq:K2}
     K_1 \triangleq \frac{\mu}{-\beta\lambda_+\lambda_-}[(1+\lambda_+)^{\tau}-1],
 \end{align}
 \begin{align}\label{eq:C3}
     K_2\triangleq \frac{\beta}{\sqrt{1+8\omega}}\sum_{\ell=0}^{\tau-2}\left(\begin{array}{c}\tau\\\ell+2\end{array}\right)[\lambda_+^{\ell+1}-\lambda_-^{\ell+1}],
 \end{align}
 with $\lambda_\pm$ defined in~\eqref{eq:eign+-} of Lemma~\ref{lem:main}.
\end{proposition}
\begin{proof}
% \nm{follow the order: e1, e2 finally e3}
We prove this result by further upper bounding $e_1^{(t_{k+1})}$ in Lemma~\ref{lem:main}. Therein, we found that
\begin{align} \label{eq:e1_lem2}
        &(e_1^{(t_{k+1})})^2
        \leq [(1-\alpha)(1-\mu\eta_{k})^{2(\tau-\Delta)}+\alpha(1-\mu\eta_{k})^{2\tau}](e_1^{(t_k)})^2
        % \nonumber \\&
        +\eta_{k}^2(\tau-(1-\alpha)\Delta)(\sigma^2+\phi^2).
    \end{align} 
To bound $(e_1^{(t_{k+1})})^2$ in~\eqref{eq:e1_lem2}, we
 use the fact that $[(1-\alpha)(1-\mu\eta_{k})^{2(\tau-\Delta)}+\alpha(1-\mu\eta_{k})^{2\tau}]$ is a convex function of $\eta_k\in[0,\eta_{\mathrm{max}}]$ (in fact, $\mu\eta_{\mathrm{max}}\leq 1$ since $\eta_{\mathrm{max}}\leq\frac{2}{\beta+\mu}$), hence
% \nm{This is only true if $\mu\epsilon\leq 1$ iff $\tau\geq (1-\alpha)\Delta$, see above}
\begin{align} \label{eq:cvx_eta}
    &(1-\alpha)(1-\mu\eta_{k})^{2(\tau-\Delta)}+\alpha(1-\mu\eta_{k})^{2\tau}
    % \nonumber \\&
    \leq
    1-\eta_k/\eta_{\mathrm{max}}C_1.
\end{align}
 Applying the result from~\eqref{eq:cvx_eta} into~\eqref{eq:e1_lem2}, gives us the result in~\eqref{eq:e1_main}.
Similarly for $e_2^{(t_{k+1})}$, we found
  \begin{align} 
    e_2^{(t_{k+1})}&\leq 
    \alpha\Pi_{+,t_{k+1}}e_2^{(t_k)}
    +\alpha\frac{4\omega}{\sqrt{8\omega+1}}[\Pi_{+,t_{k+1}}-1]e_3^{(t_k)}
    % \nonumber \\&
    % \nonumber\\& 
+\alpha\frac{\mu}{-\beta^2\lambda_+\lambda_-}(\Pi_{+,t_{k+1}}-1)\delta,
    \end{align}
    where $\Pi_{\{+,-\},t}=[1+\eta_{k}\beta\lambda_{\{+,-\}}]^{t-t_{k}}$.
    % \nm{above, first shoe the e2 term and then the e3, to be consistent and easier to follow}
    Using convexity of $\Pi_{+,t_{k+1}}$ in $\eta_k\beta\in[0,1]$, we bound it as $\Pi_{+,t_{k+1}}\leq1+ \eta_k\beta [(1+\lambda_+)^{\tau}-1]$. Applying it into the above inequality yields
    % \nm{But: if you are bounding $\eta_k\beta\leq 1$, why not simply use convexity wrt $\eta_{k}\beta\in(0,1)$ yielding
    % $(1+\eta_{k}\beta\lambda_+)^{\tau}\leq 1+\eta_{k}\beta[(1+\lambda_+)^{\tau}-1]$?? This is actually tighter since $\beta\leq 1/\eta_{\max}$. 
    % gives\nm{grammar}
    \begin{align}
    e_2^{(t_{k+1})}&\leq 
    \alpha(1+\lambda_+)^{\tau}e_2^{(t_k)}
    +\eta_{k}\alpha 2\omega C_2e_3^{(t_k)}
    % \nonumber \\&
    % \nonumber\\& 
    +\eta_{k}\alpha K_1\delta.
    \end{align}
    Finally, we found in~\eqref{eq:wc-w*_sync1} of Lemma~\ref{lem:main} that
    \begin{align} \label{eq:e3_init}
         &e_3^{(t_{k+1})}\leq
        \Psi_1(\eta_k) e_3^{(t_k)}
          \nonumber \\&
        +\underbrace{2g_{3}[(1-\alpha)\Pi_{+,t_{k+1}-\Delta}+\alpha\Pi_{+,t_{k+1}}-1]}_{(a)}e_2^{(t_k)}
        \nonumber\\&
        % +[g_{5}(\Pi_{+,t}-1)+g_{6}(\Pi_{-,t}-1)]
        % [\sigma/\beta+\sum\limits_{d=1}^N\varrho_{d}\epsilon_{d}^{(0)}] \nonumber\\&
        +\underbrace{\left[(1-\alpha)[g_{5}(\Pi_{+,t_{k+1}-\Delta}-1)+g_{6}(\Pi_{-,t_{k+1}-\Delta}-1)]
        +\alpha[g_{5}(\Pi_{+,t_{k+1}}-1)+g_{6}(\Pi_{-,t_{k+1}}-1)]\right]}_{(b)}\delta/\beta,
    \end{align}
    where $\Psi_1(\eta_k)$, $g_3$, $g_5$ and $g_6$ is defined in~\eqref{eq:Psi_1},~\eqref{eq:g3},~\eqref{eq:g5} and~\eqref{eq:g6} of Lemma~\ref{lem:main}. We bound $\Psi_1(\eta_k)$ as follows.
    Applying the binomial expansion, we have
    \begin{align} \label{eq:phi_bi}
        &\frac{g_{1}\Pi_{+,t_{k}+\ell}+g_{2}\Pi_{-,t_{k}+\ell}-1}{\eta_k\beta}
        \nonumber \\&
        =-\ell\mu/\beta+\eta_k\beta\sum\limits_{r=2}^\ell\frac{\ell!}{r!(\ell-r)!}(\eta_k\beta)^{r-2}
        \left[\frac{1}{2}(1-\frac{1}{\sqrt{8\omega+1}})\lambda_+^r+\frac{1}{2}(1+\frac{1}{\sqrt{8\omega+1}})\lambda_-^r\right].
    \end{align}
    Since $\lambda_-\leq\lambda_+$, $\eta_k\leq\eta_{\mathrm{max}}$
    % \nm{??? etamax?? PLEASE check that you are not using old variables} 
    and $\eta_k\beta\leq1$, we can further upper bound~\eqref{eq:phi_bi} with
    \begin{align}
        &\frac{g_{1}\Pi_{+,t_{k}+\ell}+g_{2}\Pi_{-,t_{k}+\ell}-1}{\eta_k\beta}
        \leq -\ell\mu/\beta +\eta_k\beta\sum\limits_{r=2}^\ell\frac{\ell!}{r!(\ell-r)!}(\eta_k\beta)^{r-2}\lambda_+^r
        \nonumber \\&
        \leq -\ell\mu/\beta +\eta_{\mathrm{max}}\beta\sum\limits_{r=2}^\ell\frac{\ell!}{r!(\ell-r)!}\lambda_+^r
        \nonumber \\&
        \overset{(a)}{=} -\ell\mu/\beta +\eta_{\mathrm{max}}\beta[(1+\lambda_+)^\ell-1-\ell\lambda_+],
    \end{align}
    where $(a)$ comes from applying the binomial theorem.
    Note that $(1+\lambda_+)^\ell-1-\ell\lambda_+\geq0,~\forall \ell\geq0$. 
    Combining this result into~\eqref{eq:Psi_1} of Lemma~\ref{lem:main}, it follows that 
    \begin{align}
        &\frac{\Psi_1(\eta_k)-1}{\eta_k\beta}
        \leq 
        -(\tau-(1-\alpha)\Delta)\mu/\beta 
        \nonumber \\&
        + \eta_{\mathrm{max}}\beta[(1-\alpha)(1+\lambda_+)^{\tau-\Delta}+\alpha(1+\lambda_+)^{\tau}-1-(\tau-(1-\alpha)\Delta)\lambda_+]
        \nonumber \\&
        \leq 
        -(\tau-\Delta)\mu/\beta 
        + \eta_{\mathrm{max}}\beta[(1+\lambda_+)^{\tau}-1-\tau\lambda_+]
        \triangleq -C_3,
    \end{align}
    where in the last inequality comes from $\tau-(1-\alpha)\Delta\geq\tau-\Delta$ and $(1-\alpha)(1+\lambda_+)^{\tau-\Delta}+\alpha(1+\lambda_+)^{\tau}-1-(\tau-(1-\alpha)\Delta)\lambda_+\leq(1+\lambda_+)^{\tau}-1-\tau\lambda_+$.
    Therefore, under $\eta_{\mathrm{max}}<\frac{(\tau-\Delta)\mu}{\beta^2[(1+\lambda_+)^{\tau}-1-\tau\lambda_+]}$, we have $C_3>0$ and
    \begin{align}
        \Psi_1(\eta_k)\leq 1-\eta_k\beta C_3<1. 
    \end{align}
    Next, we bound $(a)$ in~\eqref{eq:e3_init}.
    Convexity of $\Pi_{+,t}-1$ with respect to $\eta_k\beta$ and $\eta_{\mathrm{max}}\beta\leq1$ implies that
    \begin{align}
        &2g_3[(1-\alpha)\Pi_{+,t_{k+1}-\Delta}+\alpha\Pi_{+,t_{k+1}}-1]
        \nonumber \\&
        \leq
        \eta_{k} \frac{2\beta}{\sqrt{8\omega+1}}[(1-\alpha)(1+\lambda_+)^{\tau-\Delta}+\alpha(1+\lambda_+)^{\tau}-1]
        \leq 
        \eta_k C_2,
    \end{align}
    where $g_3$ is defined in~\eqref{eq:g3} of Lemma~\ref{lem:main},
    with $$C_2= \frac{2\beta}{\sqrt{8\omega+1}}[(1+\lambda_+)^{\tau}-1].$$
Finally, we bound $(b)$ in~\eqref{eq:e3_init}, using the binomial expansion and the expressions of $g_5$ and $g_6$ 
% \nm{need more steps. First show the binomial expansion with =, then bound}
    $$
    g_{5}(\Pi_{+,t}-1)+g_{6}(\Pi_{-,t}-1)
    =(\eta_k\beta)^2\frac{1}{\sqrt{1+8\omega}}\sum_{\ell=0}^{t-t_k-2}\left(\begin{array}{c}t-t_k\\\ell+2\end{array}\right)(\eta_k\beta)^{\ell}[\lambda_+^{\ell+1}-\lambda_-^{\ell+1}]
    \leq \eta_k^2\beta K_2.
    $$
Using these bounds in~\eqref{eq:e3_init} yield the final result in~\eqref{eq:e3_sync}.
\end{proof} 
\pagebreak


\section{Lemmas and Auxiliary Results}\label{app:lemmas}
To improve the tractability of the proofs, we provide a set of lemmas in the following, which will be used to obtain the main results of the paper.

\begin{lemma} \label{lem:An_oneSTP}
For $t\in\mathcal T_k$ before performing global synchronization, under Assumptions \ref{beta},~\ref{assump:SGD_noise} and~\ref{assump:sub_err}, if $\eta_{k}\leq\frac{2}{\mu+\beta},~\forall k$, using {\tt DFL} for ML model training, in $t\in\mathcal T_k$, the one-step behaviors of $(e_1^{(t+1)})^2$, $e_2^{(t+1)}$ and $e_3^{(t+1)}$ are presented as follows:
% \nm{move to very beginning of App}
\begin{align} \label{eq:e1_oneSTP}
    &(e_1^{(t+1)})^2
    \leq
    (1-\mu\eta_{k})^2(e_1^{(t)})^2
    +\eta_{k}^2(\sigma^2+\phi^2),
    \\\label{eq:e2_oneSTP}
    &e_2^{(t+1)}\leq
    (1+\eta_{k}(\beta-\mu))e_2^{(t)} 
    +2\omega\eta_{k}\beta e_3^{(t)}
    +\eta_{k}\delta,
\\ \label{eq:e3_oneSTP}
&   e_3^{(t+1)} \leq
     (1-\eta_{k}\mu)e_3^{(t)} 
    +\eta_{k}\beta e_2^{(t)}.
\end{align} 
%\nm{assumption on control algo (phi) missing. state as an Assumption in the main text and recall it here}
\end{lemma}

\begin{proof}
To bound $e_1^{(t)}$, we first use the definition of $\mathbf w_i^{(t+1)},~\forall i \in\mathcal S_c$ in \eqref{eq:w_i-gen} and $\mathbf v_c^{(t+1)}$ in \eqref{eq:v_c} to get, 
\begin{align} \label{eq:e_tmp1}
    &\mathbf w_i^{(t+1)} - \bar{\mathbf v}_c^{(t+1)}
    =(1-\Theta_c^{(t)})(\mathbf w_i^{(t)}-\eta_{k} \nabla F_i({\mathbf w}_i^{(t)})-\bar{\mathbf v}_c^{(t)}+\eta_{k} \nabla\bar F_c(\bar{\mathbf v}_c^{(t)}))
    \nonumber \\&
    +\Theta_c^{(t)}(\bar{\mathbf w}_c^{(t)}-\eta_{k}\sum\limits_{j\in\mathcal S_{c}}\rho_{j,c}\nabla F_j({\mathbf w}_j ^{(t)})-\bar{\mathbf v}_c^{(t)}+\eta_{k} \nabla\bar F_c(\bar{\mathbf v}_c^{(t)}))
    \nonumber \\&
    -\eta_{k}(1-\Theta_c^{(t)})\mathbf n_{i}^{(t)}-\eta_{k}\Theta_c^{(t)}\sum\limits_{j\in\mathcal S_c}\rho_{j,c}\mathbf n_{j}^{(t)}.
\end{align}
Then,
\begin{align}\label{eq:A_new}
    &(e_1^{(t+1)})^2\triangleq\mathbb E\Big[\sum\limits_{c=1}^N\varrho_c\sum\limits_{j\in\mathcal S_c}\rho_{j,c}\Vert\mathbf w_j^{(t+1)} - \bar{\mathbf v}_c^{(t+1)}\Vert^2\Big]
    \nonumber \\&
    \leq
    \mathbb E\Big[\sum\limits_{c=1}^N\varrho_c(1-\Theta_c^{(t)})\sum\limits_{j\in\mathcal S_c}\rho_{j,c}\Vert\mathbf w_j^{(t)}-\eta_{k} \nabla F_j({\mathbf w}_j^{(t)})-\bar{\mathbf v}_c^{(t)}+\eta_{k} \nabla\bar F_c(\bar{\mathbf v}_c^{(t)})\Vert^2\Big]
    % \nm{mismatch between i and j indece!!!}
    \nonumber \\&
    +\mathbb E\Big[\sum\limits_{c=1}^N\varrho_c\Theta_c^{(t)}\Vert\bar{\mathbf w}_c^{(t)}-\eta_{k}\sum\limits_{j\in\mathcal S_{c}}\rho_{j,c}\nabla F_j({\mathbf w}_j ^{(t)})-\bar{\mathbf v}_c^{(t)}+\eta_{k} \nabla\bar F_c(\bar{\mathbf v}_c^{(t)})\Vert^2\Big]+\eta_{k}^2\sigma^2
        \nonumber \\&
    \leq
    \mathbb E\Big[\sum\limits_{c=1}^N\varrho_c(1-\Theta_c^{(t)})\sum\limits_{j\in\mathcal S_c}\rho_{j,c}\Vert\mathbf w_j^{(t)}-\bar{\mathbf v}_c^{(t)}-\eta_{k} (\nabla F_j({\mathbf w}_j^{(t)})- \nabla\bar F_c(\bar{\mathbf v}_c^{(t)}))\Vert^2\Big]
    \nonumber \\&
    +\mathbb E\Big[\sum\limits_{c=1}^N\varrho_c\Theta_c^{(t)}
    \sum\limits_{j\in\mathcal S_{c}}\rho_{j,c}
    \Vert\mathbf w_j^{(t)}-\eta_{k}\nabla F_j({\mathbf w}_j ^{(t)})-\bar{\mathbf v}_c^{(t)}+\eta_{k} \nabla F_j(\bar{\mathbf v}_c^{(t)})\Vert^2\Big]+\eta_{k}^2\sigma^2,
\end{align}
where the last step
follows from
$\sum\limits_{j\in\mathcal S_c}\rho_{j,c}
 F_j(\bar{\mathbf v}_c^{(t)})
=
\bar F_c(\bar{\mathbf v}_c^{(t)}),
$
$\sum\limits_{j\in\mathcal S_c}\rho_{j,c}
{\mathbf w}_j^{(t)}
=\bar{\mathbf w}_c^{(t)}
$
and convexity of $\Vert\cdot\Vert^2$. Using again the fact that
$\sum\limits_{j\in\mathcal S_c}\rho_{j,c}
 F_j(\bar{\mathbf v}_c^{(t)})
=
\bar F_c(\bar{\mathbf v}_c^{(t)}),
$
we further bound
\begin{align}
    &\sum\limits_{j\in\mathcal S_c}\rho_{j,c}\Vert\mathbf w_j^{(t)}-\bar{\mathbf v}_c^{(t)}-\eta_{k} (\nabla F_j({\mathbf w}_j^{(t)})- \nabla\bar F_c(\bar{\mathbf v}_c^{(t)}))\Vert^2
    % \nm{mismatch between i and j indece!!!}
    \nonumber \\&
    =
\sum\limits_{j\in\mathcal S_c}\rho_{j,c}\Vert\mathbf w_j^{(t)}-\bar{\mathbf v}_c^{(t)}-\eta_{k}(\nabla F_j({\mathbf w}_j^{(t)})-\nabla F_j(\bar{\mathbf v}_c^{(t)})) -\eta_{k}\left(\nabla F_j(\bar{\mathbf v}_c^{(t)})- \nabla\bar F_c(\bar{\mathbf v}_c^{(t)})\right)\Vert^2
    \nonumber \\&
    \leq
\sum\limits_{j\in\mathcal S_c}\rho_{j,c}\Vert\mathbf w_j^{(t)}-\bar{\mathbf v}_c^{(t)}-\eta_{k} (\nabla F_j({\mathbf w}_j^{(t)})-\eta_{k} \nabla F_j(\bar{\mathbf v}_c^{(t)}))\Vert^2
    +\eta_{k}^2\sum\limits_{j\in\mathcal S_c}\rho_{j,c}\Vert\nabla F_j(\bar{\mathbf v}_c^{(t)})- \nabla\bar F_c(\bar{\mathbf v}_c^{(t)})\Vert^2.
    \nonumber
\end{align}
Furthermore, 
$\Vert\nabla F_j(\bar{\mathbf v}_c^{(t)})- \nabla\bar F_c(\bar{\mathbf v}_c^{(t)})\Vert^2
\leq
(\delta_c+2\omega_c\beta\Vert\bar{\mathbf v}_c^{(t)}-\mathbf w^*\Vert)^2
\leq
2\delta_c^2+4\omega_c\beta\Vert\bar{\mathbf v}_c^{(t)}-\mathbf w^*\Vert^2
$ (Definition~\ref{gradDiv_c}).
Combining these bounds together into~\eqref{eq:A_new} and using Fact~\ref{fact:3} yields
\begin{align}\label{eq:A_new}
    &(e_1^{(t+1)})^2
    \leq
    (1-\mu\eta_k)^2\mathbb E\Big[\sum\limits_{c=1}^N\varrho_c
    \sum\limits_{j\in\mathcal S_{c}}\rho_{j,c}
    \Vert\bar{\mathbf w}_j^{(t)}-\bar{\mathbf v}_c^{(t)}\Vert^2\Big]
    \nonumber \\&
    +\eta_{k}^2
    \sum\limits_{c=1}^N\varrho_c(1-\Theta_c^{(t)})(2\delta_c^2+4\omega_c^2\beta^2\Vert\bar{\mathbf v}_c^{(t)}-\mathbf w^*\Vert^2)
    +\eta_{k}^2\sigma^2.
\end{align}
Assuming that $\Theta_c^{(t)}$ is chosen such that $\sum\limits_{c=1}^N\varrho_c(1-\Theta_c^{(t)})(2\delta_c^2+4\omega_c^2\beta^2\Vert\bar{\mathbf v}_c^{(t)}-\mathbf w^*\Vert^2)\leq\phi^2$, (this will be part of the control algorithm, see Assumption~\ref{assump:sub_err}) we can further upper bound~\eqref{eq:A_new} and obtain the result in~\eqref{eq:e1_oneSTP}.
% \nm{sentence does not make greammatical sense: Assuming... yielding....}

% \nm{changed the order: e2 first and finally e3}
Next, we  bound $e_2$.
Using~\eqref{eq:v_c} we find that
    \begin{align} \label{eq:w-_3}
        &\bar{\mathbf v}^{(t+1)}=
        \bar{\mathbf v}^{(t)}
        -\eta_{k}\sum\limits_{d=1}^N\varrho_{d}\nabla \bar{F}_d(\bar{\mathbf v}_d^{(t)}).
    \end{align}
It then follows, after algebraic manipulations,
    \begin{align} 
        &\bar{\mathbf v}_c^{(t+1)}-\bar{\mathbf v}^{(t+1)}=\bar{\mathbf v}_c^{(t)}-\bar{\mathbf v}^{(t)}
        -\eta_{k}\Big(\nabla\bar F_c(\bar{\mathbf v}_c^{(t)})-\nabla\bar F_c(\bar{\mathbf v}^{(t)})\Big)
        % \nonumber\\&
        % -\eta_t\sum\limits_{j\in\mathcal S_{c}}\rho_{j,c}\mathbf n_{j}^{(t)}
        % +\eta_t\sum\limits_{d=1}^N\varrho_{d}\sum\limits_{j\in\mathcal S_{d}}\rho_{j,d}\mathbf n_{j}^{(t)}
        % \nonumber \\&
        % -\eta_t\sum\limits_{j\in\mathcal S_{c}}\rho_{j,c}\Big(\nabla F_j(\mathbf w_j ^{(t)})-\nabla F_j(\bar{\mathbf w}_c ^{(t)})\Big)
        % +\eta_t\sum\limits_{d=1}^N\varrho_{d}\sum\limits_{j\in\mathcal S_{d}}\rho_{j,d}\Big(\nabla F_j(\mathbf w_j ^{(t)})-\nabla F_j(\bar{\mathbf w}_d^{(t)})\Big)
        \nonumber \\&
        +\eta_{k}\sum\limits_{d=1}^N\varrho_{d}\Big(\nabla\bar F_d(\bar{\mathbf v}_d^{(t)})-\nabla\bar F_d(\bar{\mathbf v}^{(t)})\Big)
    %   \nonumber \\&
        -\eta_{k}\Big(\nabla\bar F_c(\bar{\mathbf v}^{(t)})-\nabla F(\bar{\mathbf v}^{(t)})\Big).
    \end{align}   
    Taking the norm-2 of both hand sides of the above equality and applying the triangle inequality results in
    \begin{align} \label{eq:tri_wc}
        &\Vert\bar{\mathbf v}_c^{(t+1)}-\bar{\mathbf v}^{(t+1)}\Vert\leq
        \left\Vert\bar{\mathbf v}_c^{(t)}-\bar{\mathbf v}^{(t)} -\eta_{k}[\nabla\bar F_c(\bar{\mathbf v}_c^{(t)})-\nabla\bar F_c(\bar{\mathbf v}^{(t)})]\right\Vert
        \nonumber \\&
        +\eta_{k}\sum\limits_{d=1}^N\varrho_{d}\Vert\nabla\bar F_d(\bar{\mathbf v}_d^{(t)})-\nabla\bar F_d(\bar{\mathbf v}^{(t)}))\Vert
        % \nonumber \\&
        +\eta_{k}\Vert\nabla\bar F_c(\bar{\mathbf v}^{(t)})-\nabla F(\bar{\mathbf v}^{(t)})\Vert.
    \end{align}   
    Using $\beta$-smoothness of $F_i(\cdot),\forall i$ (hence of $\bar F_d(\cdot)$), Definition \ref{gradDiv}, Fact~\ref{fact:3}, and adding over $\sum_c\rho_c$,
     we further bound the right hand side of~\eqref{eq:tri_wc} as
\begin{align} \label{eq:tri_wc2_3}
    &e_2^{(t+1)}\triangleq\sum\limits_{c=1}^N\varrho_{c}\Vert\bar{\mathbf v}_c^{(t+1)}-\bar{\mathbf v}^{(t+1)}\Vert\leq
    (1+\eta_{k}(\beta-\mu))\sum\limits_{c=1}^N\varrho_{c}\Vert\bar{\mathbf v}_c^{(t)}-\bar{\mathbf v}^{(t)}\Vert
    +2\omega\eta_{k}\beta\Vert\bar{\mathbf v}^{(t)}-\mathbf w^*\Vert
    +\eta_{k}\delta,
\end{align}   
which proves~\eqref{eq:e1_oneSTP}.
% \nm{wrong ref.. pleases check all refs}.
% \textbf{Finding the relationship between $\Vert\bar{\mathbf w}_c^{(t)}-\mathbf w^*\Vert$ and $\sum\limits_{j\in\mathcal S_c}\rho_{j,c}\Vert\mathbf w_j^{(t)}-\bar{\mathbf w}_c^{(t)}\Vert$}:
Finally, we  bound $e_3$. From \eqref{eq:w-_3}, we get 
% \nm{change sum c to sum d for consistency}
\begin{align} 
        &\bar{\mathbf v}^{(t+1)}-\mathbf w^* =~ \bar{\mathbf v}^{(t)}-\mathbf w^*-\eta_{k} \nabla F(\bar{\mathbf v}^{(t)})
        -\eta_{k} \sum\limits_{c=1}^N\varrho_{d} [\nabla \bar F_d(\bar{\mathbf v}_d^{(t)})-\nabla \bar F_d(\bar{\mathbf v}^{(t)})].
    \end{align}
    Taking the norm of both hand sides of the above equality and applying the triangle inequality gives us
    \begin{align} \label{29_3}
        &\Vert\bar{\mathbf v}^{(t+1)}-\mathbf w^*\Vert \leq \Vert\bar{\mathbf v}^{(t)}-\mathbf w^*-\eta_{k} \nabla F(\bar{\mathbf v}^{(t)})\Vert
        % +\eta_t \sum\limits_{c=1}^N\varrho_{c}\sum\limits_{j\in\mathcal S_{c}}\rho_{j,c} \Vert\mathbf n_{j}^{(t)}\Vert
        % \nonumber \\&
        % +\eta_t \sum\limits_{c=1}^N\varrho_{c}\sum\limits_{j\in\mathcal S_{c}}\rho_{j,c} \Vert\nabla F_j(\mathbf w_j^{(t)})-\nabla F_j(\bar{\mathbf w}_c^{(t)})\Vert
        % \nonumber \\&
        +\eta_{k} \sum\limits_{d=1}^N\varrho_{d} \Vert\nabla \bar F_d(\bar{\mathbf v}_d^{(t)})-\nabla \bar F_d(\bar{\mathbf v}^{(t)})\Vert.
    \end{align}
    Using $\beta$-smoothness of $F_i(\cdot)$ (hence of $\bar F_c(\cdot)$) and Fact~\ref{fact:3}, we further bound
    \begin{align}
        e_3^{(t+1)}\triangleq\Vert\bar{\mathbf v}^{(t+1)}-\mathbf w^*\Vert \leq& 
         (1-\eta_{k}\mu)
        \Vert\bar{\mathbf v}^{(t)}-\mathbf w^*\Vert
        +\eta_{k}\beta \sum\limits_{d=1}^N\varrho_{d} \Vert\bar{\mathbf v}_d^{(t)}-\bar{\mathbf v}^{(t)}\Vert,
    \end{align}
    yielding~\eqref{eq:e3_oneSTP}.
\end{proof}

% \section{Proof of Proposition~\ref{Local_disperse}} \label{app:Local_disperse}
\begin{lemma} \label{lem:main}
    Under Assumptions \ref{beta},~\ref{assump:SGD_noise} and~\ref{assump:sub_err}, if $\eta_{k}\leq\frac{2}{\beta+\mu},~\forall k$, using {\tt DFL} for ML model training, $e_1^{t_{k+1}}$, $e_2^{t_{k+1}}$ and $e_3^{(t_{k+1})}$ across global synchronization periods can be bounded as 
    % \nm{assumption on control algo (phi) missing. state as an Assumption in the main text and recall it here}
    % \nm{Use e2 and e3!}
 \begin{align} \label{eq:e1_orig}
    &(e_1^{t_{k+1}})^2
    \leq [(1-\alpha)(1-\mu\eta_{k})^{2(\tau-\Delta)}+\alpha(1-\mu\eta_{k})^{2\tau}](e_1^{(t_k)})^2
    % \nonumber\\&
    +[\tau-(1-\alpha)\Delta]\eta_{k}^2(\sigma^2+\phi^2),
\end{align} 
\begin{align} \label{eq:x1_syn_inter1}
    e_2^{(t_{k+1})}&\leq 
    \alpha\Pi_{+,t_{k+1}}e_2^{(t_k)}
    +\alpha\frac{4\omega}{\sqrt{8\omega+1}}[\Pi_{+,t_{k+1}}-1]e_3^{(t_k)}
    % \nonumber \\&
    % \nonumber\\& 
    +\alpha\frac{\mu}{-\beta^2\lambda_+\lambda_-}[\Pi_{+,t_{k+1}}-1]\delta,
\end{align}
% \nm{e3 is wrong. This is not what you are using later on... we need a tighter bound with Pi+ and Pi-}
\begin{align} \label{eq:wc-w*_sync1}
    &e_3^{(t_{k+1})}\leq
    \Psi_1(\eta_k) e_3^{(t_k)}
      % \nonumber \\&
    +2g_{3}[(1-\alpha)\Pi_{+,t_{k+1}-\Delta}+\alpha\Pi_{+,t_{k+1}}-1]e_2^{(t_k)}
    \nonumber\\&
    % +[g_{5}(\Pi_{+,t}-1)+g_{6}(\Pi_{-,t}-1)]
    % [\sigma/\beta+\sum\limits_{d=1}^N\varrho_{d}\epsilon_{d}^{(0)}] \nonumber\\&
    +\left[(1-\alpha)[g_{5}(\Pi_{+,t_{k+1}-\Delta}-1)+g_{6}(\Pi_{-,t_{k+1}-\Delta}-1)] 
    +\alpha[g_{5}(\Pi_{+,t_{k+1}}-1)+g_{6}(\Pi_{-,t_{k+1}}-1)]\right]\delta/\beta.
\end{align}
where 
\begin{align}\label{eq:Psi_1}
    \Psi_1(\eta_k)\triangleq (1-\alpha)[g_{1}\Pi_{+,t_{k+1}-\Delta}+g_{2}\Pi_{-,t_{k+1}-\Delta}]
    + \alpha[g_{1}\Pi_{+,t_{k+1}}+g_{2}\Pi_{-,t_{k+1}}]
\end{align}
and
\begin{align}\label{eq:eign+-}
    \lambda_{\pm} =\frac{1}{2}-\frac{\mu}{\beta}\pm\frac{\sqrt{8\omega+1}}{2},
\end{align}
with $\lambda_+>0$, $\lambda_-<0$, $\Pi_{\{+,-\},t}=[1+\eta_{k}\beta\lambda_{\{+,-\}}]^{t-t_{k}}$ and $g_1$, $g_2$, $g_3$, $g_5$, $g_6$ defined in~\eqref{eq:g1},~\eqref{eq:g2},~\eqref{eq:g3},~\eqref{eq:g5},~\eqref{eq:g6}.
% $\lambda_+ =\frac{1}{2}-\frac{\mu}{\beta}+\frac{\sqrt{8\omega+1}}{2}>0$ and $\lambda_-= \frac{1}{2}-\frac{\mu}{\beta}-\frac{\sqrt{8\omega+1}}{2}<0$.
\end{lemma}

\begin{proof}
% \addFL{
% \nm{follow the order, start with e1. first}
\subsection{Obtaining the upper bound of $e_1^{(t_{k+1})}$ at global synchronization} 
Using the one-step dynamics in~\eqref{eq:e1_oneSTP}, Lemma~\ref{lem:An_oneSTP}, we find before global synchronization
\begin{align}
\label{e1beforesync}
    &(e_1^{(t)})^2
    \leq
    (1-\mu\eta_{k})^{2(t-t_k)}(e_1^{(t_k)})^2
    +\sum_{\ell=0}^{t-t_k-1}(1-\mu\eta_{k})^{2\ell}\eta_{k}^2(\sigma^2+\phi^2)
    \nonumber \\&
    \leq
    (1-\mu\eta_{k})^{2(t-t_k)}(e_1^{(t_k)})^2
    +(t-t_k)\eta_{k}^2(\sigma^2+\phi^2).
\end{align}
         Next, we obtain the behavior of $e_1^{(t)}$ at global synchronization by using the definition of $\mathbf w_i^{(t)}$, $\bar{\mathbf v}_c^{(t)}$ and the global synchronization scheme in~\eqref{eq:aggr_alpha} as follows:
\begin{align}
    &\mathbf w_i^{(t_{k+1})}-\bar{\mathbf v}_c^{(t_{k+1})}
    = (1-\alpha)\sum\limits_{d=1}^N\varrho_d\sum_{j\in\mathcal S_d}\rho_{j,d}({\mathbf w}_j^{(t_{k+1}-\Delta)}-\bar{\mathbf v}_d^{(t_{k+1}-\Delta)})
    +\alpha\left(\widetilde{\mathbf w}_i^{(t_{k+1})}-\widetilde{\mathbf v}_c^{(t_{k+1})}\right), 
\end{align}
where $\widetilde{\mathbf w}_i^{(t_{k+1})}$ and $\widetilde{\mathbf v}_c^{(t_{k+1})}$ are the local model and subnet noise-free variables right before global synchronization, as opposed to $\mathbf w_i^{(t_{k+1})}$ and $\mathbf v_c^{(t_{k+1})}$ defined right after global synchronization.
  Taking the squared norm on both hand sides of the above equality and applying Jensen's inequality (convexity of $\Vert\cdot\Vert^2$) yields
\begin{align} 
    &\Vert\mathbf w_i^{(t_{k+1})}-\bar{\mathbf v}_c^{(t_{k+1})}\Vert^2
    \leq (1-\alpha)\sum\limits_{d=1}^N\varrho_d\sum_{j\in\mathcal S_d}\rho_{j,d}\Vert{\mathbf w}_j^{(t_{k+1}-\Delta)}-\bar{\mathbf v}_d^{(t_{k+1}-\Delta)}\Vert^2
    +\alpha\Vert\widetilde{\mathbf w}_i^{(t_{k+1})}-\widetilde{\mathbf v}_c^{(t_{k+1})}\Vert^2.
\end{align} 
% \nm{please fix c index. Used d}
Therefore, 
\begin{align} \label{eq:e1_sy}
    &(e_1^{(t_{k+1})})^2=\sum\limits_{c=1}^N\varrho_c\sum_{i\in\mathcal S_c}\rho_{i,c}\Vert\mathbf w_i^{(t_{k+1})}-\bar{\mathbf v}_c^{(t_{k+1})}\Vert^2
        \nonumber \\&
    \leq (1-\alpha)\sum\limits_{c=1}^N\varrho_c\sum_{j\in\mathcal S_c}\rho_{j,c}\Vert{\mathbf w}_j^{(t_{k+1}-\Delta)}-\bar{\mathbf v}_c^{(t_{k+1}-\Delta)}\Vert^2
    +\alpha\sum\limits_{c=1}^N\varrho_c\sum_{j\in\mathcal S_c}\rho_{j,c}\left\Vert\widetilde{\mathbf w}_j^{(t_{k+1})}-\widetilde{\mathbf v}_c^{(t_{k+1})}\right\Vert^2.
\end{align} 
Note that the terms above are upper bounded by \eqref{e1beforesync} before global synchronization, hence they can be bounded as
$$
\sum\limits_{c=1}^N\varrho_c\sum_{j\in\mathcal S_c}\rho_{j,c}\Vert{\mathbf w}_j^{(t_{k+1}-\Delta)}-\bar{\mathbf v}_c^{(t_{k+1}-\Delta)}\Vert^2
    \leq
    (1-\mu\eta_{k})^{2(\tau-\Delta)}(e_1^{(t_k)})^2+[\tau-\Delta]\eta_{k}^2(\sigma^2+\phi^2),
$$
$$
 \sum\limits_{c=1}^N\varrho_c\sum_{j\in\mathcal S_c}\rho_{j,c}\left\Vert\widetilde{\mathbf w}_j^{(t_{k+1})}-\widetilde{\mathbf v}_c^{(t_{k+1})}\right\Vert^2
    \leq
    (1-\mu\eta_{k})^{2\tau}(e_1^{(t_k)})^2
    +\tau\eta_{k}^2(\sigma^2+\phi^2).
$$
Using these bounds in~\eqref{eq:e1_sy} yields~\eqref{eq:e1_orig}. 

\subsection{Solving the coupled dynamics between $e_2$ and $e_3$} 

Let $\mathbf x^{(t)}=
     \begin{bmatrix}
         e_2^{(t)} & e_3^{(t)}
     \end{bmatrix}^\top$, with $e_2$ and $e_3$ defined in~\eqref{eq:def_e2} and~\eqref{eq:def_e3}. Using the one-step dynamics found in Lemma~\ref{lem:main} for $t\in\mathcal T_k$, we find (here, the vector inequality is entry-wise)
    \begin{align} \label{69}
        \mathbf x^{(t+1)}
        &\leq  
        [\mathbf I+\eta_{k}\beta\mathbf B]\mathbf  x^{(t)}
        +\eta_{k}\beta\mathbf z, 
    \end{align}
    where
    $
\mathbf z=  
\mathbf e_1\delta/\beta
    $
    ,    
    $\mathbf B=\begin{bmatrix} 1-\frac{\mu}{\beta} & 2\omega
    \\ 1 & -\frac{\mu}{\beta}\end{bmatrix}$, $\mathbf e_1=[1,0]^\top$. We also define  $\mathbf e_2=[0,1]^\top$.
    We aim to derive an upper bound on $\mathbf x^{(t)}$ denoted by $\mathbf x^{(t)}\leq\bar{\mathbf x}^{(t)}$. Using the above inequality, such upper bound is given by the recursion
     \begin{align} \label{69_3}
 \bar{\mathbf x}^{(t+1)}
=  
        [\mathbf I+\eta_{k}\beta\mathbf B]\bar{\mathbf x}^{(t)}
        +\eta_{k}\beta\mathbf z, \ \forall t\in\mathcal T_k,
    \end{align}
 initialized as $\bar{\mathbf x}^{(t_k)}={\mathbf x}^{(t_k)}$.
 To solve the coupled dynamic, we first apply eigen-decomposition on $\mathbf B$ yielding
    $\mathbf B=\mathbf U\mathbf D\mathbf U^{-1}$, where
    $$\mathbf D=\begin{bmatrix} \lambda_+ & 0
    \\ 0 & \lambda_-\end{bmatrix},\ 
    \mathbf U=\begin{bmatrix} \frac{1}{2}(1+\sqrt{8\omega+1}) & -\frac{1}{2}(\sqrt{8\omega+1}-1)
    \\ 1 & 1\end{bmatrix},\ 
\mathbf U^{-1}=\frac{1}{
    \sqrt{8\omega+1}
    }\begin{bmatrix} 1 & \frac{1}{2}(\sqrt{8\omega+1}-1)
    \\ -1 &\frac{1}{2}(\sqrt{8\omega+1}+1)\end{bmatrix}$$
with eigenvalues given by~\eqref{eq:eign+-}.
Using this decomposition in~\eqref{69_3} yields by induction
    \begin{align} 
    \bar{\mathbf x}^{(t)}=
\mathbf U(\mathbf I+\eta_{k}\beta\mathbf D)^{t-t_{k}}
         \mathbf U^{-1}{\mathbf x}^{(t_k)}
+\mathbf U\left[(\mathbf I+\eta_{k}\beta\mathbf D)^{t-t_{k}}-\mathbf I\right]
         \mathbf D^{-1}\mathbf U^{-1}\mathbf z.
    \end{align}
    % \textbf{(Part III) Finding the connection between the bound on ${\mathbf x}^{(t)}$ and the expressions for $A^{(t)}$ and $B^{(t)}$}:
Therefore,
% \triangleq \bar y^{(t)}
\begin{align} \label{eq:x2_dyn}
    e_2^{(t)}&=\mathbf e_1^\top\mathbf x^{(t)}\leq 
    \mathbf e_1^\top\bar{\mathbf x}^{(t)}
   \nonumber\\&=
  [m_{1}\Pi_{+,t}+m_{2}\Pi_{-,t}]e_3^{(t_k)}
  \nonumber \\&
+[m_{3}\Pi_{+,t}+m_{4}\Pi_{-,t}]e_2^{(t_k)}
\nonumber\\&
% +[m_{5}(\Pi_{+,t}-1)+m_{6}(\Pi_{-,t}-1)]
% [\sigma/\beta+\sum\limits_{d=1}^N\varrho_{d}\epsilon_{d}^{(0)}] \nonumber\\&
+[m_{5}(\Pi_{+,t}-1)+m_{6}(\Pi_{-,t}-1)]\delta/\beta, 
    \end{align}
    where we have defined $\Pi_{\{+,-\},t}=[1+\eta_{k}\beta\lambda_{\{+,-\}}]^{t-t_{k}}$ and constants $m_{1}$-$m_{8}$ as  
        % \nm{please align left}
    \begin{flalign}
        m_{1}\triangleq
        &[\mathbf U]_{1,1}[\mathbf U^{-1}]_{1,2}
        =
        \frac{2\omega}{\sqrt{8\omega+1}},&
    \end{flalign}
    % \nm{use $[\mathbf A]_{i,j}$ instead of $\mathbf e_i^\top\mathbf A\mathbf e_j$...}
    \begin{flalign}
        m_{2}\triangleq
        &[\mathbf U]_{1,2}[\mathbf U^{-1}]_{2,2}
        =
        -m_1, &
    \end{flalign}
    \begin{flalign}
        m_{3}\triangleq
         &[\mathbf U]_{1,1}[\mathbf U^{-1}]_{1,1}
         =\frac{\sqrt{8\omega+1}+1}{2\sqrt{8\omega+1}}, &
    \end{flalign}
    \begin{flalign}
        m_{4}\triangleq
        &[\mathbf U]_{1,2}[\mathbf U^{-1}]_{2,1}
        =1-m_3\geq 0, &
    \end{flalign}
\begin{flalign}
    m_{5}
    \triangleq
    &[\mathbf U]_{1,1}[\mathbf D^{-1}\mathbf U^{-1}]_{1,1}
    =
    \frac{\mu(\sqrt{8\omega+1}+1)+4\omega\beta}{-2\beta\lambda_+\lambda_-\sqrt{8\omega+1}}=\frac{\sqrt{8\omega+1}+1}{2\sqrt{8\omega+1}\lambda_+}\geq 0,&
\end{flalign}
\begin{flalign}
    m_{6}\triangleq
    &[\mathbf U]_{1,2}[\mathbf D^{-1}\mathbf U^{-1}]_{2,1}
    =
    \frac{\mu(\sqrt{8\omega+1}-1)-4\omega\beta}{-2\beta\lambda_+\lambda_-\sqrt{8\omega+1}}
    =\frac{\sqrt{8\omega+1}-1}{-2\sqrt{8\omega+1}\lambda_-}\leq 0.&
\end{flalign}
Since $\Pi_{-,t}\leq\Pi_{+,t}$, $\Pi_{+,t}-\Pi_{-,t}\leq 2[\Pi_{+,t}-1]$, 
% \nm{can be proved using binomial expansion}
 $m_3+m_4=1$ and $\Pi_{-,t}-1\geq-(\Pi_{+,t}-1)$, we can further upper bound
 % \nm{unless I did something wrong m6 is negative! therefore cannot use $\Pi_{-,t}-1\leq0$.
 % Use $1-\Pi_{-,t}\leq \Pi_{+,t}-1$ instead.
 % }
\begin{align} \label{eq:E_1}
    e_2^{(t)}&\leq 
    \frac{4\omega}{\sqrt{8\omega+1}}[\Pi_{+,t}-1]e_3^{(t_k)}
    % \nonumber \\&
    +\Pi_{+,t}e_2^{(t_k)}
    % \nonumber\\& 
    +\frac{\mu}{-\beta^2\lambda_+\lambda_-}(\Pi_{+,t}-1)\delta
    . 
    \end{align}
% Revisit~\eqref{eq:x2_dyn}, we get
% coupled dynamics of x1 %%%%%%%%%%%%%%%%%%%%%%%%%%%%%%%%%%%%%%%%%%%%%%%%%%%%%%%%%%%%%%%%%%
Similarly, from the expression of $\bar{\mathbf x}^{(t)}$ above, we find
% \triangleq \bar y^{(t)}
\begin{align} \label{eq:x2_dyn}
    e_3^{(t)}&=\mathbf e_2^\top\mathbf x^{(t)}\leq 
    \mathbf e_2^\top\bar{\mathbf x}^{(t)}
   \nonumber\\&=
  [g_{1}\Pi_{+,t}+g_{2}\Pi_{-,t}]e_3^{(t_k)}
  \nonumber \\&
+[g_{3}\Pi_{+,t}+g_{4}\Pi_{-,t}]e_2^{(t_k)}
\nonumber\\&
% +[g_{5}(\Pi_{+,t}-1)+g_{6}(\Pi_{-,t}-1)]
% [\sigma/\beta+\sum\limits_{d=1}^N\varrho_{d}\epsilon_{d}^{(0)}] \nonumber\\&
+[g_{5}(\Pi_{+,t}-1)+g_{6}(\Pi_{-,t}-1)]\delta/\beta, 
    \end{align}
    where we have defined $g_{1}$-$g_{8}$ as
    % \nm{use $[]_{i,j}$}
    % \nm{please align left}
    \begin{flalign}\label{eq:g1}
        g_{1}\triangleq
        &[\mathbf U]_{2,1}[\mathbf U^{-1}]_{1,2}
        =
        \frac{1}{2}(1-\frac{1}{\sqrt{8\omega+1}})\geq 0,&
    \end{flalign}
    \begin{flalign}\label{eq:g2}
        g_{2}\triangleq
        &[\mathbf U]_{2,2}[\mathbf U^{-1}]_{2,2}
        =
        \frac{1}{2}(1+\frac{1}{\sqrt{8\omega+1}})=1-g_1\geq 0,&
    \end{flalign}
    \begin{flalign}\label{eq:g3}
        g_{3}\triangleq
        &[\mathbf U]_{2,1}[\mathbf U^{-1}]_{1,1}
         =\frac{1}{\sqrt{8\omega+1}}\in [1/3,1],&
    \end{flalign}
    \begin{flalign}
        g_{4}\triangleq
        &[\mathbf U]_{2,2}[\mathbf U^{-1}]_{2,1}
        =-g_3,&
    \end{flalign}
    \begin{flalign}\label{eq:g5}
        g_{5}
        \triangleq
        &[\mathbf U]_{2,1}[\mathbf D^{-1}\mathbf U^{-1}]_{1,1}
        =
        \frac{1}{\lambda_+\sqrt{1+8\omega}}\geq 0,&
    \end{flalign}
    \begin{flalign}\label{eq:g6}
        g_{6}\triangleq
        &[\mathbf U]_{2,2}[\mathbf D^{-1}\mathbf U^{-1}]_{2,1}
        =
        \frac{1}{-\lambda_-\sqrt{1+8\omega}}\geq0.&
    \end{flalign}
Since $\Pi_{+,t}-\Pi_{-,t}\leq 2[\Pi_{+,t}-1]$ and $g_4=-g_3$, we can further upper bound $e_3$ as
\begin{align} \label{eq:E_2}
    &e_3^{(t)}\leq
     [g_{1}\Pi_{+,t}+g_{2}\Pi_{-,t}]e_3^{(t_k)}
      \nonumber \\&
    +2g_3[\Pi_{+,t}-1]e_2^{(t_k)}
    \nonumber\\&
    % +[g_{5}(\Pi_{+,t}-1)+g_{6}(\Pi_{-,t}-1)]
    % [\sigma/\beta+\sum\limits_{d=1}^N\varrho_{d}\epsilon_{d}^{(0)}] \nonumber\\&
    +[g_{5}(\Pi_{+,t}-1)+g_{6}(\Pi_{-,t}-1)]\delta/\beta.
\end{align}
Next, we use these results to bound $e_2$ and $e_3$ at global synchronization.
% \nm{seems good enough for our purpose}
%%%%%%%%%%%%%%%%%%%%%%%%%%%%%%%%%%%%%%%%%%%%%%%%%%%%%%%%%%%%%%%%%%%%%%%%%%%%%%%%%%%%%%%%%%%%
\subsection{Obtaining the upper bound of $e_2^{(t_{k+1})}$ at global synchronization.} 
% \nm{I dont know why you were doing that, so I cut it. You already have the results of solving the induction, why are you doing an additional step??}
To obtain the behavior of $\bar{\mathbf v}_c^{(t_{k+1})}-\bar{\mathbf v}^{(t_{k+1})}$ after global synchronization, we use the definition of $\bar{\mathbf v}_c^{(t)}$, $\bar{\mathbf v}^{(t)}$ and the global synchronization scheme in~\eqref{eq:aggr_alpha}, we have  
% \nm{please use constant alpha} 
\begin{align}
    \bar{\mathbf v}_c^{(t_{k+1})}-\bar{\mathbf v}^{(t_{k+1})}
    =& (1-\alpha)(\bar{\mathbf v}^{(t_{k+1}-\Delta)}-\bar{\mathbf v}^{(t_{k+1}-\Delta)})
    +\alpha\left(\widetilde{\mathbf v}_c^{(t_{k+1})}-\widetilde{\mathbf v}^{(t_{k+1})}\right)
    \nonumber \\
    =&
    \alpha\left(\widetilde{\mathbf v}_c^{(t_{k+1})}-\widetilde{\mathbf v}^{(t_{k+1})}\right),
    % \nm{fix\ conflict\ of indeces.. use\ d\ instead\ of\ c}
\end{align}
where $\widetilde{\mathbf v}^{(t_{k+1})}$ is the global noise-free variable right before global synchronization, as opposed to $\mathbf v^{(t_{k+1})}$ defined right after global synchronization.
Taking the norm of both hand sides of the above equality and adding over $\sum_c\rho_c$ yields
\begin{align} \label{eq:wc-w*_sync2}
        &e_2^{(t_{k+1})}=\sum_c\rho_c\Vert\bar{\mathbf v}_c^{(t_{k+1})}-\bar{\mathbf v}^{(t_{k+1})}\Vert
        = \alpha\sum_c\rho_c\left\Vert\widetilde{\mathbf v}_c^{(t_{k+1})}-
        \widetilde{\mathbf v}^{(t_{k+1})}\right\Vert.
\end{align} 
Note that, since $\widetilde{\mathbf v}_c^{(t_{k+1})}$ represents the noise-free variable right before the global synchronization, the right hand side above can be bounded
via \eqref{eq:E_1}, yielding the final result \eqref{eq:x1_syn_inter1}.
\subsection{Obtaining upper bound of $e_3^{(t_{k+1})}$ at global synchronization.}
% \nm{I dont know why you were doing that, so I cut it. You already have the results of solving the induction, why are you doing an additional step??}
To obtain the behavior of $\bar{\mathbf v}^{(t_{k+1})}-\mathbf w^*$ after global synchronization,  we use the definition of $\bar{\mathbf v}^{(t)}$ and the global synchronization scheme in~\eqref{eq:aggr_alpha}, we have  
\begin{align}
    & \bar{\mathbf v}^{(t_{k+1})}-\mathbf w^*
        = (1-\alpha)[\bar{\mathbf v}^{(t_{k+1}-\Delta)}-\mathbf w^*]
        +\alpha\left[\widetilde{\mathbf v}^{(t_{k+1})}-\mathbf w^*\right].
\end{align} 
Note that $\widetilde{\mathbf v}^{(t)}$ is equivalent to $\bar{\mathbf v}^{(t)}$ before conducting global synchronization. Taking the norm of both hand sides of the above equality and applying the triangle inequality
gives us
\begin{align} \label{eq:wc-w*_sync_tmp}
        &e_3^{(t_{k+1})}= \Vert\bar{\mathbf v}^{(t_{k+1})}-\mathbf w^*\Vert
        \leq (1-\alpha)\Vert\bar{\mathbf v}^{(t_{k+1}-\Delta)}-\mathbf w^*\Vert
        +\alpha\Vert\widetilde{\mathbf v}^{(t_{k+1})}-\mathbf w^*\Vert.
\end{align} 
We further upper bound the right hand using~\eqref{eq:E_2} to obtain the result in\eqref{eq:wc-w*_sync1}.
\end{proof}


\begin{fact}\label{fact:1} 
Consider $n$ random real-valued vectors $\mathbf x_1,\cdots,\mathbf x_n\in\mathbb R^m$, the following inequality holds: 
 \begin{equation}
     \sqrt{\mathbb E\left[\Big\Vert\sum\limits_{i=1}^{n} \mathbf x_i\Big\Vert^2\right]}\leq \sum\limits_{i=1}^{n} \sqrt{\mathbb E[\Vert\mathbf x_i\Vert^2]}.
 \end{equation}
\end{fact}
\begin{proof} Note that
    % Using the following result of Cauchy-Schwarz Inequality
    % \begin{align}
    %     \mathbb E[XY] \leq \sqrt{\mathbb E[X^2]\mathbb E[ Y^2]},
    % \end{align}
    % we obtain
    \begin{align}
        &\sqrt{\mathbb E\left[\Big\Vert\sum\limits_{i=1}^{n}\mathbf x_i\Big\Vert^2\right]}
        =
        \sqrt{\sum\limits_{i,j=1}^{n}\mathbb E [\mathbf x_i^\top\mathbf x_j]}
        \overset{(a)}{\leq}
\sum\limits_{i,j=1}^{n}\sqrt{\mathbb E [\Vert\mathbf x_i\Vert^2] \mathbb E[\Vert\mathbf x_j\Vert^2]]}
        % \nonumber \\&
        =
        \sum\limits_{i=1}^{n} \sqrt{\mathbb E[\Vert\mathbf x_i\Vert^2]},
    \end{align}
    where $(a)$ follows from Holder's inequality, $\mathbb E[|XY|] \leq \sqrt{\mathbb E[|X|^2]\mathbb E[ |Y|^2]}$.
\end{proof}


\begin{fact}\label{fact:3} Let $f(\cdot)$ be $\mu$-strong convex and $\beta$-smooth and $\eta\leq\frac{2}{\beta+\mu}$, the following inequality holds
\begin{align}
    \left\Vert\mathbf w_1-\mathbf w_2 -\eta(\nabla f(\mathbf w_1)-\nabla f(\mathbf w_2))\right\Vert
    \leq (1-\mu\eta)\left\Vert\mathbf w_1-\mathbf w_2\right\Vert,\ \forall \mathbf w_1,\mathbf w_2\in\mathbb R^M.
\end{align}    
\end{fact}
\begin{proof}
    \begin{align} \label{eq:stx_3}
        &\left\Vert\mathbf w_1-\mathbf w_2 -\eta(\nabla f(\mathbf w_1)-\nabla f(\mathbf w_2))\right\Vert
        \nonumber \\&
        =
        \sqrt{\Vert\mathbf w_1-\mathbf w_2\Vert^2+\eta^2\Vert\nabla f(\mathbf w_1)-\nabla f(\mathbf w_2)\Vert^2-2\eta(\mathbf w_1-\mathbf w_2)^\top(\nabla f(\mathbf w_1)-\nabla f(\mathbf w_2))}
        \nonumber \\&
        \overset{(a)}{\leq} 
        \sqrt{\left(1-\eta\frac{2\mu\beta}{\mu+\beta}\right)\Vert\mathbf w_1-\mathbf w_2\Vert^2-\eta\left(\frac{2}{\mu+\beta}-\eta\right)\Vert\nabla f(\mathbf w_1)-\nabla f(\mathbf w_2)\Vert^2}
        \nonumber \\&
        \overset{(b)}{\leq}
%        \sqrt{\left(1-2\mu\eta+\mu^2\eta^2\right)\Vert\mathbf w_1-\mathbf w_2\Vert^2}        \leq 
        (1-\eta\mu)\Vert\mathbf w_1-\mathbf w_2\Vert,
    \end{align}
    where $(a)$ comes from~\cite[Theorem 2.1.12]{Nesterov}, i.e., $(\mathbf w_1-\mathbf w_2)^\top(\nabla f(\mathbf w_1)-\nabla f(\mathbf w_2))\geq\frac{\mu\beta}{\mu+\beta}\Vert\mathbf w_1-\mathbf w_2\Vert^2+\frac{1}{\mu+\beta}\Vert\nabla f(\mathbf w_1)-\nabla f(\mathbf w_2)\Vert^2$ and $(b)$ results from
    $\Vert\nabla f(\mathbf w_1)-\nabla f(\mathbf w_2)\Vert\geq \mu\Vert\mathbf w_1-\mathbf w_2\Vert$ (strong convexity)
 and $\eta\leq \frac{2}{\mu+\beta}$.
\end{proof}

%%%%%%%%%%%%%%%%%%%%%%%%%%%%%%%%%%%%%%%%%%%%%%%%%%%%%%%%%%%%%%%%%%%%%%%%%%%%%%%%%%

\pagebreak



\end{document}

