\documentclass[twoside,11pt]{article}

% Any additional packages needed should be included after jmlr2e.
% Note that jmlr2e.sty includes epsfig, amssymb, natbib and graphicx,
% and defines many common macros, such as 'proof' and 'example'.
%
% It also sets the bibliographystyle to plainnat; for more information on
% natbib citation styles, see the natbib documentation, a copy of which
% is archived at http://www.jmlr.org/format/natbib.pdf

% Available options for package jmlr2e are:
%
%   - abbrvbib : use abbrvnat for the bibliography style
%   - nohyperref : do not load the hyperref package
%   - preprint : remove JMLR specific information from the template,
%         useful for example for posting to preprint servers.
%
% Example of using the package with custom options:
%
\usepackage[abbrvbib, preprint]{jmlr2e}

%\usepackage[abbrvbib]{jmlr2e}
%\usepackage{lineno,hyperref}
%\modulolinenumbers[5]

% insert here the call for the packages your document requires
%\usepackage{latexsym}
\usepackage{multirow,makecell}
\let\proof\relax
\let\endproof\relax
\usepackage{bm,amsmath,amsfonts,amsthm,booktabs}
\allowdisplaybreaks
\usepackage{mathtools}
\usepackage[justification=justified]{caption}
\usepackage{url}
\usepackage{enumitem}
\usepackage{xcolor}
%\usepackage[ruled,vlined,linesnumbered]{algorithm2e}
%\SetArgSty{textnormal}
\usepackage{algcompatible}
\usepackage{algpseudocode}
\usepackage{algorithm}
%\usepackage{algorithmicx}
\usepackage{subcaption}
\usepackage{rotating}
\usepackage{array}
\usepackage{verbatim}
\usepackage{tikz-cd}
\usepackage{graphbox}
\usepackage{float}
\usepackage{svg}
\usepackage{tabularx}
\usepackage{nicematrix}
\usepackage[export]{adjustbox}
\usepackage{hyperref}
\usepackage{cleveref}
\usepackage{autonum}


\newcommand*{\vertbar}{\rule[-1ex]{0.5pt}{2.5ex}}
\newcommand*{\horzbar}{\rule[.5ex]{2.5ex}{0.5pt}}
\newcolumntype{P}[1]{>{\centering\arraybackslash}p{#1}}
\newcolumntype{M}[1]{>{\centering\arraybackslash}m{#1}}
\newcommand{\makeheadbox}{\relax}
%\newtheorem{example}{Example}
%\setlength\parindent{0pt}
\setcounter{MaxMatrixCols}{20}

\def\TODO#1{{\color{red} \textbf{#1}}}

\newcommand{\ac}[1]{\textcolor{green}{Alex: #1}}

\newcommand{\dk}[1]{\textcolor{blue}{Dhruv: #1}}

\newcommand{\gm}[1]{\textcolor{purple}{Gal: #1}}

\newcommand{\edit}[1]{\textcolor{black}{#1}}

\newcommand{\editt}[1]{\textcolor{black}{#1}}

\newcommand{\edittt}[1]{\textcolor{black}{#1}}

%\DeclareSymbolFont{ebgletters}{OML}{EBGaramond-Maths}{m}{it}
%\DeclareMathSymbol{w}{\mathalpha}{ebgletters}{`k}


\newtheorem{assump}{Assumption}
\newtheorem{dfn}{Definition}
\newtheorem{thm}{Theorem}
\newtheorem{cor}[thm]{Corollary}
\newtheorem{lem}[thm]{Lemma}
\newtheorem{prop}[thm]{Proposition}
\newtheorem{rmk}{Remark}
\newtheorem{ex}{Example}
%\newdefinition{note}{Note}
%\newproof{pf}{Proof}
\newtheorem{contrib}{Contribution}

\newcommand{\proofoffirst}[1]{\noindent \textbf{Proof of #1.}}
\newcommand{\proofof}[1]{\medskip \noindent \textbf{Proof of #1.}}

\newcommand{\argmin}{\mathop{\mathrm{argmin}}\limits}
\newcommand{\argmax}{\mathop{\mathrm{argmax}}\limits}


% Definitions of handy macros can go here

\newcommand{\dataset}{{\cal D}}
\newcommand{\fracpartial}[2]{\frac{\partial #1}{\partial  #2}}

\newcommand{\QR}{{\mathrm{QR}}}
\newcommand{\qf}{{\mathrm{qf}}}
\newcommand{\grad}{{\mathrm{grad}}}
\newcommand{\diag}{{\mathrm{diag}}}
\newcommand{\vecz}{{\mathrm{vec}}}
\newcommand{\Hess}{{\mathrm{Hess}}}
\newcommand{\blockdiag}{{\mathrm{block\text{-}diag}}}
\newcommand{\Skew}{{\mathrm{Skew}}}
\newcommand{\Sym}{{\mathrm{Sym}}}
\newcommand{\rank}{{\mathrm{rank}}}

\newif\iftodos
%\todostrue % comment out to hide answers

% Heading arguments are {volume}{year}{pages}{date submitted}{date published}{paper id}{author-full-names}

\usepackage{lastpage}
\usepackage{subfiles}

\setcitestyle{square}

% \renewcommand{\citep}[2][]{[\citenum{#2}, {#1}]}
%\renewcommand{\citep}[2][]{\cite[{#1}]{#2}}
\newcommand{\citea}[2][]{[\citenum{#2}]}
\newcommand{\citeb}[2][]{[\citenum{#2}, {#1}]}

%\renewcommand{\cite}[1][2]{[\citenum{#1}]}
%\renewcommand{\ref}[1]{[\cref{#1}]}

%\jmlrheading{22}{2021}{1-\pageref{LastPage}}{2/21; Revised
%9/21}{11/21}{21-0131}{Dhruv Kohli, Alexander Cloninger and Gal Mishne}
%\ShortHeadings{LDLE: LOW DISTORTION LOCAL EIGENMAPS}{Kohli, Cloninger and Mishne}

\firstpageno{1}

\begin{document}

\title{Non-degenerate Rigid Alignment in a Patch Framework}

\author{\name Dhruv Kohli\email dhkohli@ucsd.edu\\
        \addr Department of Mathematics\\
        University of California San Diego\\
        CA 92093, USA
        \AND
        \name Gal Mishne\email gmishne@ucsd.edu\\
        \addr Halicio\u{g}lu Data Science Institute\\
        University of California San Diego\\
        CA 92093, USA
        \AND
        \name Alexander Cloninger\email acloninger@ucsd.edu\\
        \addr Department of Mathematics and\\
        Halicio\u{g}lu Data Science Institute\\
        University of California San Diego\\
        CA 92093, USA}

\editor{}

\maketitle

\begin{abstract}%

Given a set of overlapping local views (patches) of a dataset, we consider the problem of finding a rigid alignment of the views that minimizes a $2$-norm based alignment error.
%Given a patch framework, we consider the problem of finding a rigid alignment of the views
%, identified with an element of the product of $m$ orthogonal groups, $\mathbb{O}(d)^m$,
%that minimizes a $2$-norm based alignment error. In general, the views may be noisy and a perfect alignment may not exist.
In general, the views are noisy and a perfect alignment may not exist.
In this work, we characterize the non-degeneracy of an alignment in the noisy setting based on the kernel and positivity of a certain matrix. This leads to a polynomial time algorithm for testing the non-degeneracy of a given alignment. Consequently, we focus on Riemannian gradient descent for minimization of the error and obtain a sufficient condition on an alignment for the algorithm to converge (locally) linearly to it.  In the case of noiseless views, a perfect alignment exists, resulting in a realization of the points that respects the geometry of the views.
%The affine rigidity of such realizations and its connection with the overlapping structure of the views has been studied in related work [Zha and Zhang;  Chaudhary et al.]. 
Under a mild condition on the views, we show that the non-degeneracy of a perfect alignment
%, up to the action of $\mathbb{O}(d)$,
is equivalent to the local rigidity of the resulting realization.
%, thus obtaining a characterization of the latter.
By specializing the characterization of a non-degenerate alignment to the noiseless setting, we obtain necessary and sufficient conditions on the overlapping structure of the views for a locally rigid realization. Similar results are also obtained in the context of global rigidity.
\end{abstract}

\begin{keywords}
Nondegeneracy, uniqueness, rigid alignment, local rigidity, global rigidity, linear convergence, Riemannian gradient descent, Morse-Bott function.
\end{keywords}

% Optional TODOs:
% \begin{enumerate}
%     \item Counter example for theorem 31.
%     \item More references on local and global rigidity.
%     \item More references on the papers dealing with rigid alignment of views.
%     \item Remember to on the todo switch to see technical todos.
% \end{enumerate}

\section{Introduction}
\label{sec:intro}
\subfile{sections/intro}


\section{A Quadratic Program in Orthogonal Groups for Aligning Views}
\label{sec:setup}
\subfile{sections/setup}


\section{Non-degeneracy and Uniqueness in the General Setting}
\label{sec:non_deg}
\subfile{sections/non_deg}

\section{Non-degeneracy and Uniqueness in the Noiseless Regime}
\label{sec:noiseless_non_deg_results}
\subfile{sections/noiseless}

\section{Linear Convergence of RGD}
\label{sec:convergence}
\subfile{sections/convergence}

\section{Conclusion and Open Problems}
\label{sec:conc}
\subfile{sections/conc}

\acks{DK was partly supported by a grant from Kavli Institute for Brain and Mind (UCSD). GM was partially funded by NSF CCF-2217058. AC was partially funded by NSF DMS-2012266 and a gift from Intel Research.}

%\clearpage
\setcitestyle{numbers}
%\bibliography{loc_rigid}
\begin{thebibliography}{34}
\providecommand{\natexlab}[1]{#1}
\providecommand{\url}[1]{\texttt{#1}}
\expandafter\ifx\csname urlstyle\endcsname\relax
  \providecommand{\doi}[1]{doi: #1}\else
  \providecommand{\doi}{doi: \begingroup \urlstyle{rm}\Url}\fi

\bibitem[Absil et~al.(2009)Absil, Mahony, and Sepulchre]{absil2009optimization}
P.-A. Absil, R.~Mahony, and R.~Sepulchre.
\newblock Optimization algorithms on matrix manifolds.
\newblock In \emph{Optimization Algorithms on Matrix Manifolds}. Princeton
  University Press, 2009.

\bibitem[Bandeira et~al.(2014)Bandeira, Charikar, Singer, and
  Zhu]{bandeira2014multireference}
A.~S. Bandeira, M.~Charikar, A.~Singer, and A.~Zhu.
\newblock Multireference alignment using semidefinite programming.
\newblock In \emph{Proceedings of the 5th conference on Innovations in
  theoretical computer science}, pages 459--470, 2014.

\bibitem[Chaudhury et~al.(2015)Chaudhury, Khoo, and
  Singer]{chaudhury2015global}
K.~N. Chaudhury, Y.~Khoo, and A.~Singer.
\newblock Global registration of multiple point clouds using semidefinite
  programming.
\newblock \emph{SIAM Journal on Optimization}, 25\penalty0 (1):\penalty0
  468--501, 2015.

\bibitem[Clementi et~al.(2000)Clementi, Nymeyer, and
  Onuchic]{clementi2000topological}
C.~Clementi, H.~Nymeyer, and J.~N. Onuchic.
\newblock Topological and energetic factors: what determines the structural
  details of the transition state ensemble and “en-route” intermediates for
  protein folding? an investigation for small globular proteins.
\newblock \emph{Journal of molecular biology}, 298\penalty0 (5):\penalty0
  937--953, 2000.

\bibitem[Cohen et~al.(2006)Cohen, Iga, and Norbury]{cohen_iga_norbury_2006}
R.~L. Cohen, K.~Iga, and P.~Norbury.
\newblock Topics in morse theory: Lecture notes, 2006.
\newblock URL
  \url{http://math.stanford.edu/~ralph/morsecourse/biglectures.pdf}.

\bibitem[Crosilla and Beinat(2002)]{proc_algo_1}
F.~Crosilla and A.~Beinat.
\newblock {Use of generalised Procrustes analysis for the photogrammetric block
  adjustment by independent models}.
\newblock \emph{ISPRS Journal of Photogrammetry and Remote Sensing},
  56:\penalty0 195--209, 04 2002.

\bibitem[Cucuringu et~al.(2012{\natexlab{a}})Cucuringu, Lipman, and
  Singer]{cucuringu2012sensor}
M.~Cucuringu, Y.~Lipman, and A.~Singer.
\newblock Sensor network localization by eigenvector synchronization over the
  euclidean group.
\newblock \emph{ACM Transactions on Sensor Networks (TOSN)}, 8\penalty0
  (3):\penalty0 1--42, 2012{\natexlab{a}}.

\bibitem[Cucuringu et~al.(2012{\natexlab{b}})Cucuringu, Singer, and
  Cowburn]{cucuringu2012eigenvector}
M.~Cucuringu, A.~Singer, and D.~Cowburn.
\newblock Eigenvector synchronization, graph rigidity and the molecule problem.
\newblock \emph{Information and Inference: A Journal of the IMA}, 1\penalty0
  (1):\penalty0 21--67, 2012{\natexlab{b}}.

\bibitem[Fang and Toh(2012)]{fang2012using}
X.~Fang and K.-C. Toh.
\newblock Using a distributed sdp approach to solve simulated protein molecular
  conformation problems.
\newblock In \emph{Distance Geometry: Theory, Methods, and Applications}, pages
  351--376. Springer, 2012.

\bibitem[Feehan(2021)]{feehan2021optimal}
P.~M. Feehan.
\newblock Optimal {\l}ojasiewicz--simon inequalities and morse--bott
  yang--mills energy functions.
\newblock \emph{Advances in Calculus of Variations}, 2021.

\bibitem[Gentle(2007)]{gentle2007matrix}
J.~E. Gentle.
\newblock Matrix algebra.
\newblock \emph{Springer texts in statistics, Springer, New York, NY, doi},
  10:\penalty0 978--0, 2007.

\bibitem[Gortler et~al.(2010)Gortler, Gotsman, Liu, and
  Thurston]{gortler2010affine}
S.~J. Gortler, C.~Gotsman, L.~Liu, and D.~P. Thurston.
\newblock On affine rigidity.
\newblock \emph{arXiv preprint arXiv:1011.5553}, 2010.

\bibitem[Hendrickson(1992)]{hendrickson1992conditions}
B.~Hendrickson.
\newblock Conditions for unique graph realizations.
\newblock \emph{SIAM journal on computing}, 21\penalty0 (1):\penalty0 65--84,
  1992.

\bibitem[Ho and Van~Dooren(2005)]{HO2005917}
N.-D. Ho and P.~Van~Dooren.
\newblock On the pseudo-inverse of the laplacian of a bipartite graph.
\newblock \emph{Applied Mathematics Letters}, 18\penalty0 (8):\penalty0
  917--922, 2005.

\bibitem[Kohli et~al.(2021)Kohli, Cloninger, and Mishne]{kohli2021ldle}
D.~Kohli, A.~Cloninger, and G.~Mishne.
\newblock Ldle: low distortion local eigenmaps.
\newblock \emph{Journal of Machine Learning Research}, 22\penalty0
  (282):\penalty0 1--64, 2021.

\bibitem[Kovanic(1979)]{kovanic1979pseudoinverse}
P.~Kovanic.
\newblock On the pseudoinverse of a sum of symmetric matrices with applications
  to estimation.
\newblock \emph{Kybernetika}, 15\penalty0 (5):\penalty0 341--348, 1979.

\bibitem[Krishnan et~al.(2005)Krishnan, Lee, Moore, Venkatasubramanian,
  et~al.]{krishnan2005global}
S.~Krishnan, P.~Y. Lee, J.~B. Moore, S.~Venkatasubramanian, et~al.
\newblock Global registration of multiple 3d point sets via
  optimization-on-a-manifold.
\newblock In \emph{Symposium on Geometry Processing}, pages 187--196, 2005.

\bibitem[Lee(2008)]{lee2008geometric}
A.~Lee.
\newblock \emph{Geometric constraint systems with applications in CAD and
  biology}.
\newblock University of Massachusetts Amherst, 2008.

\bibitem[Ling(2021{\natexlab{a}})]{ling2021generalized}
S.~Ling.
\newblock Generalized power method for generalized orthogonal procrustes
  problem: global convergence and optimization landscape analysis.
\newblock \emph{arXiv preprint arXiv:2106.15493}, 2021{\natexlab{a}}.

\bibitem[Ling(2021{\natexlab{b}})]{ling2021near}
S.~Ling.
\newblock Near-optimal bounds for generalized orthogonal procrustes problem via
  generalized power method.
\newblock \emph{arXiv preprint arXiv:2112.13725}, 2021{\natexlab{b}}.

\bibitem[Liu et~al.(2019)Liu, So, and Wu]{liu2019quadratic}
H.~Liu, A.~M.-C. So, and W.~Wu.
\newblock Quadratic optimization with orthogonality constraint: explicit
  {\l}ojasiewicz exponent and linear convergence of retraction-based
  line-search and stochastic variance-reduced gradient methods.
\newblock \emph{Mathematical Programming}, 178\penalty0 (1):\penalty0 215--262,
  2019.

\bibitem[{\L}ojasiewicz(1965)]{lojasiewicz1965ensembles}
S.~{\L}ojasiewicz.
\newblock \emph{Ensembles semi-analytiques}.
\newblock Institut des Hautes Etudes Scientifiques, 1965.

\bibitem[Saxe(1979)]{saxe1979embeddability}
J.~B. Saxe.
\newblock Embeddability of weighted graphs in k-space is strongly np-hard.
\newblock In \emph{Proc. of 17th Allerton Conference in Communications, Control
  and Computing, Monticello, IL}, pages 480--489, 1979.

\bibitem[Schneider and Uschmajew(2015)]{schneider2015convergence}
R.~Schneider and A.~Uschmajew.
\newblock Convergence results for projected line-search methods on varieties of
  low-rank matrices via {\l}ojasiewicz inequality.
\newblock \emph{SIAM Journal on Optimization}, 25\penalty0 (1):\penalty0
  622--646, 2015.

\bibitem[Sch{\"o}nemann(1966)]{schonemann1966generalized}
P.~H. Sch{\"o}nemann.
\newblock A generalized solution of the orthogonal procrustes problem.
\newblock \emph{Psychometrika}, 31\penalty0 (1):\penalty0 1--10, 1966.

\bibitem[Sharp et~al.(2004)Sharp, Lee, and Wehe]{sharp2004multiview}
G.~C. Sharp, S.~W. Lee, and D.~K. Wehe.
\newblock Multiview registration of 3d scenes by minimizing error between
  coordinate frames.
\newblock \emph{IEEE Transactions on Pattern Analysis and Machine
  Intelligence}, 26\penalty0 (8):\penalty0 1037--1050, 2004.

\bibitem[Szeliski et~al.(2007)]{szeliski2007image}
R.~Szeliski et~al.
\newblock Image alignment and stitching: A tutorial.
\newblock \emph{Foundations and Trends{\textregistered} in Computer Graphics
  and Vision}, 2\penalty0 (1):\penalty0 1--104, 2007.

\bibitem[Usevich et~al.(2020)Usevich, Li, and Comon]{usevich2020approximate}
K.~Usevich, J.~Li, and P.~Comon.
\newblock Approximate matrix and tensor diagonalization by unitary
  transformations: convergence of jacobi-type algorithms.
\newblock \emph{SIAM Journal on Optimization}, 30\penalty0 (4):\penalty0
  2998--3028, 2020.

\bibitem[Van~Loan and Golub(1996)]{van1996matrix}
C.~F. Van~Loan and G.~Golub.
\newblock Matrix computations (johns hopkins studies in mathematical sciences).
\newblock \emph{Matrix Computations}, 1996.

\bibitem[Williams and Bennamoun(2000)]{williams2000simultaneous}
J.~A. Williams and M.~Bennamoun.
\newblock Simultaneous registration of multiple point sets using orthonormal
  matrices.
\newblock In \emph{2000 IEEE International Conference on Acoustics, Speech, and
  Signal Processing. Proceedings (Cat. No. 00CH37100)}, volume~4, pages
  2199--2202. IEEE, 2000.

\bibitem[Zha and Zhang(2009)]{zha2009spectral}
H.~Zha and Z.~Zhang.
\newblock Spectral properties of the alignment matrices in manifold learning.
\newblock \emph{SIAM review}, 51\penalty0 (3):\penalty0 545--566, 2009.

\bibitem[Zhang et~al.(2010)Zhang, Liu, Gotsman, and Gortler]{zhang2010rigid}
L.~Zhang, L.~Liu, C.~Gotsman, and S.~J. Gortler.
\newblock An as-rigid-as-possible approach to sensor network localization.
\newblock \emph{ACM Transactions on Sensor Networks (TOSN)}, 6\penalty0
  (4):\penalty0 1--21, 2010.

\bibitem[Zhang and Zha(2004)]{zhang2004principal}
Z.~Zhang and H.~Zha.
\newblock Principal manifolds and nonlinear dimensionality reduction via
  tangent space alignment.
\newblock \emph{SIAM journal on scientific computing}, 26\penalty0
  (1):\penalty0 313--338, 2004.

\bibitem[Zhu et~al.(2010)Zhu, So, and Ye]{zhu2010universal}
Z.~Zhu, A.~M.-C. So, and Y.~Ye.
\newblock Universal rigidity: Towards accurate and efficient localization of
  wireless networks.
\newblock In \emph{2010 Proceedings IEEE INFOCOM}, pages 1--9. IEEE, 2010.

\end{thebibliography}



\clearpage
\appendix
\subfile{sections/all_proofs}

\end{document}
