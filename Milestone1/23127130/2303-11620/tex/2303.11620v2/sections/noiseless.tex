As discussed in \citea{chaudhury2015global}, in the noiseless case, the patch-stress matrix $\mathbf{C}$ is constructed from $\Gamma$ and clean measurements. In particular, there exists a global minimum $\mathbf{S}$ of $F$ such $F(\mathbf{S}) = 0$ or in our terminology, there exists a perfect alignment $\mathbf{S}$.

The organization of the section is as follows. We start by deriving some important consequences of the noiseless setting in Section~\ref{subsec:noiseless_conseq} which are used in the subsequent sections. In Section~\ref{subsec:loc_glob_rigid}, under a mild assumption on the structure of the local views, we show that the non-degeneracy and uniqueness of a perfect alignment is equivalent to the local and global rigidity \citea{gortler2010affine} of the resulting realization (Theorem~\ref{thm:loc_rigid} and \ref{thm:glob_rigid}), respectively. Combined with the results in the previous section, we obtain a characterization of the local rigidity of a realization. Then, in Section~\ref{subsec:non_deg_noiseless_setting} and Section~\ref{subsec:uniq_noiseless_setting}, we provide necessary and sufficient conditions on the overlapping structure of the views for their realization to be locally/globally rigid. These conditions should be contrasted with those for the affine rigidity of a realization, as presented in \citea{zha2009spectral}.

\subsection{Consequences of Noiseless Setting}
\label{subsec:noiseless_conseq}
The following consequences of the noiseless setting will play a crucial role in the subsequent sections.
\begin{prop}
\label{prop:noiseless_setting1}
Let $\mathbf{S}$ be a perfect alignment. Then $\widehat{\mathbf{C}}(\mathbf{S}) = 0$ and consequently $\mathbf{L}(\mathbf{S}) = -\mathbf{C}(\mathbf{S})$ (see Eq.~(\ref{eq:C_of_S}, \ref{eq:C_hat}, \ref{eq:L_of_S})) and $\mathbb{L}(\mathbf{S}) = -\overline{\mathbf{P}}(\mathbf{I}_d \otimes  (\mathbf{P}\mathbf{C}(\mathbf{S})\mathbf{P}^T))\overline{\mathbf{P}}^T$ (see Eq.~(\ref{eq:mathcal_L}, \ref{eq:mathbb_L})).
\end{prop}

\begin{rmk}
\label{rmk:noiseless_setting1}
Due to the above proposition and Remark~\ref{rmk:C_S_structure}, it is easy to deduce that for a perfect alignment $\mathbf{S}$, $\mathbf{L}(\mathbf{S}) \preceq 0$ and $\mathbb{L}(\mathbf{S}) \preceq 0$.
\end{rmk}

Using the above proposition, we provide a simplified characterization of a non-degenerate perfect alignment that will be useful in proving the subsequent results. First, along the similar lines as in \citea{zha2009spectral}, we define a certificate of $\mathbf{L}(\mathbf{S})$ as follows.
\begin{dfn}
\label{def:LScertificate}
An $\boldsymbol{\Omega} \in \Skew(d)^m$ is said to be a certificate of $\mathbf{L}(\mathbf{S})$ if $\mathbf{L}(\mathbf{S})\boldsymbol{\Omega} = 0$. It is a trivial certificate if $\boldsymbol{\Omega}_i = \boldsymbol{\Omega}_0$ for all $i \in [1,m]$ and for some $\boldsymbol{\Omega}_0 \in \Skew(d)$.
\end{dfn}

Then the characterization of a non-degenerate perfect alignment (obtained trivially from Proposition~\ref{prop:one_all1} and ~\ref{prop:noiseless_setting1}) is as follows.
\begin{prop}
\label{prop:non_deg_triv_cert}
If $\mathbf{S}$ is a perfect alignment then $\mathbf{S}$ is non-degenerate if and only if every certificate of $\mathbf{L}(\mathbf{S})$ is trivial.
\end{prop}

The following corollary follows trivially from Corollary~\ref{cor:suff_non_deg_loc_min}, Proposition~\ref{prop:noiseless_setting1} and Remark~\ref{rmk:C_S_structure}.
\begin{cor}
\label{cor:noiseless_setting1}
If $\mathbf{C}$ is of rank $(m-1)d$ then every perfect alignment of $F$ is non-degenerate.
\end{cor}

\subsection{Locally and Globally Rigid Realization}
\label{subsec:loc_glob_rigid}
In the following, we reveal the relation between non-degenerate perfect alignment and local rigidity of the realization due to it, and unique perfect alignment and the global rigidity of the resulting realization. First note that due to Definition~\ref{def:realization}, for any two perfect alignments $\mathbf{S}, \mathbf{O} \in \mathbb{O}(d)^m$, we have 
\begin{align}
    \min_{\mathbf{Q} \in \mathbb{O}(d), \mathbf{t} \in \mathbb{R}^d} \left\|\Theta(\mathbf{O}) - \mathbf{Q}^T\Theta(\mathbf{S}) - \mathbf{t}\mathbf{1}_n^T\right\|_F &= \min_{\mathbf{Q} \in \mathbb{O}(d)} \left\|\Theta(\mathbf{O}) - \mathbf{Q}^T\Theta(\mathbf{S})\right\|_F\\
    &= \min_{\mathbf{Q} \in \mathbb{O}(d)} \left\|\Theta(\mathbf{O}) - \Theta(\mathbf{S}\mathbf{Q})\right\|_F. \label{eq:loc_rigid_pre}
\end{align}
Now we are ready to define the local and global rigidity of a realization $\Theta(\mathbf{S})$. Although phrased differently, the definitions are the same as those in \citea{gortler2010affine}.
% \begin{dfn}
% Let $\mathbf{S}$ be a global minimum. Then $\Theta(\mathbf{S})$ is said to be locally rigid if there exist $\epsilon > 0$ such that for any $Y = [y_k]_1^n$ with $\left\|Y-\Theta(\mathbf{S})\right\|_F < \epsilon$ and
% \begin{align}
%     y_k = O_i^Tx_{k}(\mathbf{S}) + v_i, (k,i) \in E
% \end{align}
% for another global minimum $\{O_i\}_1^m \subseteq \mathbb{O}(d)$ and $\{v_i\}_1^m \subseteq \mathbb{R}^d$ (i.e. $Y = X([\mathbf{S}_iO_i]_1^m)$), we have $Y = \Theta(\mathbf{S})Q = \Theta(\mathbf{S}\mathbf{Q})$ for some $\mathbf{Q} \in \mathbb{O}(d)$.
% \end{dfn}

\begin{dfn}
\label{def:loc_rigid}
Let $\mathbf{S}$ be a perfect alignment. Then $\Theta(\mathbf{S})$ is said to be locally rigid if there exists $\epsilon > 0$ such that for any other perfect alignment $\mathbf{O} \in \mathbb{O}(d)^m$ with $\left\|\Theta(\mathbf{O})-\Theta(\mathbf{S})\right\|_F < \epsilon$, we have $\Theta(\mathbf{O})$ to be a rigid transformation of $\Theta(\mathbf{S})$ or equivalently (due to Eq.~\ref{eq:loc_rigid_pre}) $\Theta(\mathbf{O}) = \mathbf{Q}^T\Theta(\mathbf{S}) = \Theta(\mathbf{S}\mathbf{Q})$ for some $\mathbf{Q} \in \mathbb{O}(d)$.
\end{dfn}

\begin{dfn}
\label{def:glob_rigid}
Let $\mathbf{S}$ be a perfect alignment. Then $\Theta(\mathbf{S})$ is said to be globally rigid if for any other perfect alignment $\mathbf{O} \in \mathbb{O}(d)^m$ we have $\Theta(\mathbf{O}) = \mathbf{Q}^T\Theta(\mathbf{S}) = \Theta(\mathbf{S}\mathbf{Q})$ for some $\mathbf{Q} \in \mathbb{O}(d)$.
\end{dfn}

Examples of realizations that are not locally rigid, locally rigid but not globally rigid and globally rigid are provided in Figure~\ref{fig:nec_cond_loc_rigid_of_views}, Figure~\ref{fig:suff_cond_views_non_deg} and Figure~\ref{fig:G_star_1}, respectively.

%\textbf{TODO}: Locally and globally rigid but not affinely rigid. ?
\begin{assump}
\label{assump:non_deg_views}
Let $\mathbf{S}$ and $\mathbf{O}$ be perfect alignments then $\Theta(\mathbf{S}) = \Theta(\mathbf{O})$ if and only if $\mathbf{S} = \mathbf{O}$. 
\end{assump}

\begin{rmk}
\label{rmk:non_deg_views}
Clearly, Assumption~\ref{assump:non_deg_views} holds when each view is affinely non-degenerate i.e. has at least $d+1$ points whose affine span has a rank of $d$. In this case, the perfect alignment of the local views can be uniquely determined by their realization.
\end{rmk}

In the following, we show that a non-degenerate perfect alignment gives rise to a locally rigid realization and vice versa.
\begin{thm}
\label{thm:loc_rigid}
Let $\mathbf{S}$ be a perfect alignment and suppose Assumption~\ref{assump:non_deg_views} holds. Then $\mathbf{S}$ is non-degenerate if and only if $\Theta(\mathbf{S})$ is locally rigid.
\end{thm}

\begin{rmk}
Due to Corollary~\ref{cor:noiseless_setting1} and Theorem~\ref{thm:loc_rigid}, under Assumption~\ref{assump:non_deg_views}, if $\mathbf{C}$ is of rank $(m-1)d$ then every realization is locally rigid.
\end{rmk}

The above remark is consistent with the results in \citea{chaudhury2015global,gortler2010affine} in that, if each view has at least $d+1$ affinely non-degenerate points and the rank of $\mathbf{C}$ is $(m-1)d$ then the underlying patch-stress framework is affinely rigid and thus locally (as well as globally) rigid too.

Finally, following Theorem~\ref{thm:loc_rigid}, Remark~\ref{rmk:noiseless_setting1}, and Theorem~\ref{thm:non_deg_loc_min}, we immediately obtain a characterization of the local rigidity, as follows.

\begin{cor}
Let $\mathbf{S}$ be a perfect alignment and suppose Assumption~\ref{assump:non_deg_views} holds. Then $\Theta(\mathbf{S})$ is locally rigid if and only if $\mathbb{L}(\mathbf{S})$ is of rank $(m-1)d(d-1)/2$.
\end{cor}

In contrast to the general case, here, $\mathbb{L}(\mathbf{S})$ is already negative semidefinite as a consequence of the noiseless views (see Remark~\ref{rmk:noiseless_setting1}). Finally, from Definition~\ref{def:uniq_alignment} and Definition~\ref{def:glob_rigid}, it trivially follows that a unique perfect alignment gives rise to a globally rigid realization and vice versa.
\begin{thm}
\label{thm:glob_rigid}
Let $\mathbf{S}$ be a perfect alignment and suppose Assumption~\ref{assump:non_deg_views} holds. Then $\mathbf{S}$ is unique if and only if $\Theta(\mathbf{S})$ is globally rigid.
\end{thm}

\subsection{Conditions on Overlapping Views for a Locally Rigid Realization}
\label{subsec:non_deg_noiseless_setting}
\iftodos
\textbf{TODO}:
\begin{prop}
Let $\mathbf{S}$ be a global minimum of $F$ in the noiseless setting. Then $\mathbf{S}$ is non-degenerate iff $\mathbf{P}(I_d \otimes \overline{\mathbf{P}}D_\mathbf{S}^T B\boldsymbol{\mathcal{L}}_{\Gamma}^\dagger B^TD_S \overline{\mathbf{P}}^T)\mathbf{P}^T$ is negative semidefinite and of rank $(m-1)d(d-1)/2$.
\end{prop}
Exact non-degeneracy condition in the noiseless setting. Question: Can the non-degeneracy of global minimum in the noiseless setting be reduced to rank constraint on $I_d \otimes C$ and thus on $\mathbf{C}$. I don't think so.
\fi
Under Assumption~\ref{assump:non_deg_views}, the local rigidity of a realization is equivalent to the non-degeneracy of the corresponding perfect alignment (Theorem~\ref{thm:loc_rigid}). We therefore focus on deriving necessary and sufficient conditions on the overlapping structure of the views for a perfect alignment to be non-degenerate. First, we need the following definition.

\begin{dfn}
\label{def:BSAcapB}
Let $\mathbf{S}$ be a perfect alignment. Let $A$ and $B$ be non-empty disjoint subsets of $[1,m]$. Define $\mathbf{B}(\mathbf{S})_{A,B}$ to be a matrix whose columns are $\mathbf{S}_i^T\mathbf{x}_{k,i}+\mathbf{t}_i$ (in the increasing order of $k$) where $(k,i),(k,j) \in E(\Gamma)$ for some $i \in A$ and $j \in B$, and where $\mathbf{t}_i$ is obtained using Eq.~(\ref{eq:opt_Z}). Also define
\begin{align}
    \overline{\mathbf{B}(\mathbf{S})}_{A,B} = \mathbf{B}(\mathbf{S})_{A,B}\left(\mathbf{I}_{n'} - \frac{\mathbf{1}_{n'}\mathbf{1}_{n'}^T}{n'}\right)
\end{align}
where $n' = |\{k:(k,i),(k,j) \in E(\Gamma) \text{ for some } (i,j) \in A \times B\}|$.
For brevity, we denote $\mathbf{B}(\mathbf{S})_{\{i\},\{j\}}$ and $\overline{\mathbf{B}(\mathbf{S})}_{\{i\},\{j\}}$ by $\mathbf{B}(\mathbf{S})_{i,j}$ and $\overline{\mathbf{B}(\mathbf{S})}_{i,j}$ respectively, where $i \neq j$. Note that the notation is consistent with that of Definition~\ref{def:BSicapj}.
\end{dfn}

\begin{rmk}
Since $\mathbf{S}$ is a perfect alignment $\mathbf{S}_i^T\mathbf{x}_{k,i}+\mathbf{t}_i = \mathbf{S}_j^T\mathbf{x}_{k,j}+\mathbf{t}_j$ for all $(k,i),(k,j) \in E(\Gamma)$ and thus $\mathbf{B}(\mathbf{S})_{A,B}$ is well defined. Also note that $\mathbf{B}(\mathbf{S})_{A,B} = \mathbf{B}(\mathbf{S})_{B,A}$.
\end{rmk}

\begin{rmk}
\label{rmk:BS_ijB_ij_noiseless}
Let $i,j \in [1,m]$ and $\mathbf{S}$ be a perfect alignment. Since $\mathbf{B}(\mathbf{S})_{i,j} = \mathbf{B}(\mathbf{S})_{j,i}$ we conclude the following from Remark~\ref{rmk:BS_ijB_ij},
\begin{align}
    \rank (\overline{\mathbf{B}}_{i,j}) &= \rank (\overline{\mathbf{B}(\mathbf{S})}_{i,j}) = \rank (\overline{\mathbf{B}(\mathbf{S})}_{j,i}) = \rank (\overline{\mathbf{B}}_{j,i})\\
    &= \rank (\overline{\mathbf{B}(\mathbf{S})}_{i,j}\overline{\mathbf{B}(\mathbf{S})}_{j,i}^T) = \rank (\overline{\mathbf{B}}_{i,j}\overline{\mathbf{B}}_{j,i}^T).
\end{align}
\end{rmk}

Due to the above two remarks and Theorem~\ref{thm:non_deg_two_views_gen_setting}, a necessary and sufficient condition for a perfect alignment of two views to be non-degenerate is easily obtained and is as follows.
\begin{thm}
\label{thm:nec_suff_cond_loc_rigid_two_views}
Consider $m=2$ and let $\mathbf{S}$ be a perfect alignment. Then $\mathbf{S}$ is non-degenerate if and only if $\rank(\overline{\mathbf{B}}_{1,2}) \geq d-1$.
\end{thm}

A necessary condition for a perfect alignment of $m \geq 3$ views to be non-degenerate is as follows. The converse of the theorem does not hold, as demonstrated in Figure~\ref{fig:nec_cond_loc_rigid_of_views}.
\begin{thm}
\label{thm:nec_cond_loc_rigid_of_views}
Let $\mathbf{S}$ be a perfect alignment. If $\mathbf{S}$ is non-degenerate then $\rank(\overline{\mathbf{B}(\mathbf{S})}_{A,B})$ is at least $d-1$ for all non-empty partitions $A$ and $B$ of $[1,m]$ i.e. for all $A,B \subseteq [1,m]$, $A,B \neq \emptyset$, $A \cap B = \emptyset$ and $A \cup B = [1,m]$.
\end{thm}

Now we derive a sufficient condition for a perfect alignment of $m \geq 3$ views to be non-degenerate. As in \citea{zha2009spectral}, we construct a graph $\mathbb{G}$ with $m$ vertices where each vertex corresponds to a view and an edge exists between the $i$th and $j$th vertices if and only if $\rank(\overline{\mathbf{B}}_{i,j}) \geq d-1$. The Theorem~\ref{thm:nec_suff_cond_loc_rigid_two_views} and the following propositions will play a crucial role in our next set of results,

\begin{lem}
\label{lem:subproblem_cert}
Let $\mathbf{S}$ be a perfect alignment and $\boldsymbol{\Omega}$ be a certificate of $\mathbf{L}(\mathbf{S})$. Consider removing the $i$th view and the points that lie exclusively in it. Then $\mathbf{S}_{-i} = [\mathbf{S}_j]_{j \in [1,m] \setminus \{i\}}$ is a perfect alignment of the remaining set of views and $[\boldsymbol{\Omega}_j]_{j \in [1,m] \setminus \{i\}}$ is a certificate of $\mathbf{L}_{-i}(\mathbf{S}_{-i})$, the matrix in Eq.~(\ref{eq:L_of_S}) associated with the remaining set of views.
\end{lem}

\begin{prop}
\label{prop:same_conn_comp_non_deg}
Let $\mathbf{S}$ be a perfect alignment. Let $\boldsymbol{\Omega}$ be a certificate of $\mathbf{L}(\mathbf{S})$. If $i$th and $j$th view lie in the same connected component of $\mathbb{G}$ then $\boldsymbol{\Omega}_i = \boldsymbol{\Omega}_j$.
\end{prop}

Similar to the one in \citea{zha2009spectral}, consider the following coarsening procedure on $\mathbb{G}$ given a perfect alignment $\mathbf{S}$: (i) transform all the views using $\mathbf{S}$ (and $\mathbf{t}$ computed using Eq.~\ref{eq:opt_Z}) (ii) merge the views that lie in the same connected component of $\mathbb{G}$ and replace them with a single view (iii) then construct the graph (in the same manner as $\mathbb{G}$) associated with the new set of views (iv) repeat the procedure from (ii). Let the final graph over the remaining views be $\mathbb{G}^*(\mathbf{S})$, then the following holds.
\begin{thm}
\label{thm:G_star_1}
    Let $\mathbf{S}$ be a perfect alignment. If $|\mathbb{G}^*(\mathbf{S})| = 1$ then $\mathbf{S}$ is non-degenerate.
\end{thm}
The converse of the above theorem may not hold, as demonstrated in Figure~\ref{fig:G_star_1}. The following corollary follows trivially.
\begin{cor}
\label{cor:suff_cond_views_non_deg}
If $\mathbb{G}$ is connected then every perfect alignment is non-degenerate.
\end{cor}

\begin{figure}[H]
    \centering
    \begin{tabular}{cc}
    \begin{subfigure}[b]{0.35\textwidth}
         \centering
         \includegraphics[width=0.9\textwidth,keepaspectratio]{../fig/fig0/counterex_nec_loc_rigid.png}
         \caption{Theorem~\ref{thm:nec_cond_loc_rigid_of_views}}
         \label{fig:nec_cond_loc_rigid_of_views}
     \end{subfigure}
     & 
     \begin{subfigure}[b]{0.175\textwidth}
         \centering
         \includegraphics[width=0.9\textwidth,keepaspectratio]{../fig/fig0/counterex_suff_loc_rigid_2.png}
         \caption{Theorem~\ref{thm:G_star_1}}
         \label{fig:G_star_1}
     \end{subfigure}
     \end{tabular}
    \caption{Counterexamples for the converse of various Theorems. The dotted lines represent views and the filled points represent points on the overlaps. (\ref{fig:nec_cond_loc_rigid_of_views}) Clearly for every pair of nonempty partitions $A$ and $B$ of $[1,4]$, $\rank(\overline{\mathbf{B}(\mathbf{S})}_{A,B}) \geq 1$ but $\mathbf{S}$ is degenerate. (\ref{fig:G_star_1}) Clearly $\mathbf{S}$ is non-degenerate but $|\mathbb{G}^*(\mathbf{S})| = 3$.}
    \label{fig:counterex}
\end{figure}

\subsection{Conditions on Overlapping Views for a Globally Rigid Realization}
\label{subsec:uniq_noiseless_setting}
Under Assumption~\ref{assump:non_deg_views}, the global rigidity of a realization is equivalent to the uniqueness of the corresponding perfect alignment (Theorem~\ref{thm:glob_rigid}). We therefore focus on deriving necessary and sufficient conditions on the overlapping structure of the views for a perfect alignment to be unique. From Remark~\ref{rmk:BS_ijB_ij_noiseless} and Theorem~\ref{thm:uniq_two_views_gen_setting}, it is easy to deduce the following.
\begin{thm}
\label{thm:nec_suff_cond_glob_rigid_two_views}
Consider $m=2$ and let $\mathbf{S}$ be a perfect alignment. Then $\mathbf{S}$ is unique (see Definition~\ref{def:uniq_alignment}) if and only if $\rank (\overline{\mathbf{B}}_{1,2}) = d$.
\end{thm}

A necessary condition for a perfect alignment of $m \geq 3$ views to be unique is as follows.
%The converse of it may not hold, as demonstrated in Figure~\ref{fig:nec_cond_glob_rigid_views}.
\begin{thm}
\label{thm:nec_cond_glob_rigid_views}
Let $\mathbf{S}$ be a perfect alignment. If $\mathbf{S}$ is unique then $\rank(\overline{\mathbf{B}(\mathbf{S})}_{A,B}) = d$ for all non-empty partitions $A$ and $B$ of $[1,m]$.
\end{thm}

Now we derive a sufficient condition for a perfect alignment of $m \geq 3$ to be unique. As in the previous section, we construct a graph $\overline{\mathbb{G}}$ with $m$ vertices where each vertex corresponds to a view and an edge exists between the $i$th and $j$th vertices if and only if $\rank(\overline{\mathbf{B}}_{i,j}) = d$. We need Lemma~\ref{lem:subproblem_cert}, Theorem~\ref{thm:nec_suff_cond_glob_rigid_two_views} and the following proposition for our next result,
\begin{prop}
\label{prop:same_conn_comp_uniq}
Let $\mathbf{S}$ and $\mathbf{S}'$ be perfect alignments. If $i$th and $j$th view lie in the same connected component of $\overline{\mathbb{G}}$ then there exist $\mathbf{Q} \in \mathbb{O}(d)$ such that $\mathbf{S}'_i = \mathbf{S}_i\mathbf{Q}$ and $\mathbf{S}'_j = \mathbf{S}_j\mathbf{Q}$.
\end{prop}

Now consider the same coarsening procedure as used for Theorem~\ref{thm:G_star_1}, except that $\mathbb{G}$ and $\mathbb{G}^*(\mathbf{S})$ are replaced by $\overline{\mathbb{G}}$ and $\overline{\mathbb{G}}^*$, respectively. Then the following holds.
\begin{thm}
\label{thm:overline_G_star_1}
    Let $\mathbf{S}$ be a perfect alignment. If $|\overline{\mathbb{G}}^*(\mathbf{S})| = 1$ then $\mathbf{S}$ is unique.
\end{thm}

The converse of the above theorem may not hold, as demonstrated in Figure~\ref{fig:overline_G_star_1}. The following corollary is obtained trivially from the above theorem.

\begin{cor}
\label{cor:suff_cond_views_uniq}
Let $\mathbf{S}$ be a perfect alignment. If $\overline{\mathbb{G}}$ is connected, then $\mathbf{S}$ is unique.
\end{cor}

% \begin{figure}[H]
%     \centering
%     \begin{tabular}{cc}
%     \begin{subfigure}[b]{0.35\textwidth}
%          \centering
%          \includegraphics[width=0.9\textwidth,keepaspectratio]{../fig/fig0/counterex_nec_loc_rigid.png}
%          \caption{Theorem~\ref{thm:nec_cond_glob_rigid_views}}
%          \label{fig:nec_cond_glob_rigid_views}
%      \end{subfigure}
%      & 
%      \begin{subfigure}[b]{0.175\textwidth}
%          \centering
%          \includegraphics[width=\textwidth,keepaspectratio]{../fig/fig0/counterex_suff_glob_rigid.png}
%          \caption{Theorem~\ref{thm:overline_G_star_1}}
%          \label{fig:overline_G_star_1}
%      \end{subfigure}
%      \end{tabular}
%     \caption{Counterexamples for the converse of various Theorems. The dotted lines represent views and the filled points represent points on the overlaps. (\ref{fig:nec_cond_glob_rigid_views}) Just a placeholder for now. (\ref{fig:overline_G_star_1}) Clearly $\mathbf{S}$ is unique but $|\overline{\mathbb{G}}^*(\mathbf{S})| = 3$.}
%     \label{fig:counterex2}
% \end{figure}

\begin{figure}[H]
    \centering
     \includegraphics[width=0.175\textwidth,keepaspectratio]{../fig/fig0/counterex_suff_glob_rigid.png}
    \caption{Counterexample for the converse of Theorem~\ref{thm:overline_G_star_1}. The dotted lines represent views and the filled points represent points on the overlaps. Clearly $\mathbf{S}$ is unique but $|\overline{\mathbb{G}}^*(\mathbf{S})| = 3$.}
    \label{fig:overline_G_star_1}
\end{figure}