The organization of the section is as follows. In Section~\ref{subsec:prelims} we derive an expression involving the Hessian of $\widetilde{F}$ (Eq.~(\ref{eq:eq3}), (\ref{eq:eq4})), the function induced by $F$ on a certain quotient space $\mathbb{O}(d)^m/\sim$ (along with all the necessary geometrical objects). We refer the reader to \citeb[Chapter 3 and 5]{absil2009optimization} for the definitions of differential of a mapping, metric, gradient, connection, Hessian etc. in the context of Riemannian manifolds. Then in Section~\ref{subsec:non_sing_pos_def_hess} we obtain the equations governing the non-singularity and positivity of the Hessian (Eq.~(\ref{eq:TrOmegaTLSOmega}), (\ref{eq:omega^TmbbLomega})). Subsequently, in Section~\ref{subsec:non_deg_gen_setting}, we obtain a characterization of the non-degeneracy of an alignment in the general setting (Theorem~\ref{thm:non_deg_loc_min}). We also present a necessary and sufficient condition on the overlapping structure of two views under which an alignment of the views is non-degenerate (Theorem~\ref{thm:non_deg_two_views_gen_setting}). Finally, in Section~\ref{subsec:uniq_gen_setting}, we derive a similar condition under which an optimal alignment of two views is unique (Theorem~\ref{thm:uniq_two_views_gen_setting}).

\subsection{Preliminaries}
\label{subsec:prelims}
Recall that the problem under consideration is the minimization of $F(\mathbf{S}) = \text{Tr}(\mathbf{C}\mathbf{S}\mathbf{S}^T)$ over $\mathbf{S} \in \mathbb{O}(d)^m$ where $\mathbf{C} \succeq 0$ is the patch-stress matrix defined in Eq.~(\ref{eq:GPOP}). Note that the objective is invariant to the action of $\mathbb{O}(d)$ i.e. for any $\mathbf{Q} \in \mathbb{O}(d)$, $F(\mathbf{S}) = F(\mathbf{S}\mathbf{Q})$.
\begin{assump}
If $\Gamma$ has $K$ connected components, then the objective is invariant to the action of $\mathbb{O}(d)^K$ on $\mathbb{O}(d)^m = \prod_1^K \mathbb{O}(d)^{m_j}$ where $\sum_1^K m_j = m$ and where each $\mathbb{O}(d)$ acts independently on $\mathbb{O}(d)^{m_j}$ for $j \in [1,K]$. To keep the computations clean, we assume that the bipartite graph $\Gamma$ is connected (as in Assumption~\ref{assump:connected_gamma}) throughout the rest of this work.
\end{assump}
Subsequently, we define an equivalence relation on $\mathbb{O}(d)^m$; $\mathbf{S}_1 \sim \mathbf{S}_2$ if and only if $\mathbf{S}_1 = \mathbf{S}_2\mathbf{Q}$ for some $\mathbf{Q} \in \mathbb{O}(d)$. Given $\mathbf{S} \in \mathbb{O}(d)^m$, its equivalence class is $[\mathbf{S}] = \left\{\mathbf{S}\mathbf{Q}: \mathbf{Q} \in \mathbb{O}(d)\right\}$. Clearly, there exists a bijection between $\mathbb{O}(d)^m/\sim$ and $\mathbb{O}(d)^{m-1}$, thus an element of $\mathbb{O}(d)^m/\sim$ will be identified with an element of $\mathbb{O}(d)^{m-1}$. Define the projection,
\begin{align}
    \pi: \mathbb{O}(d)^m &\mapsto \mathbb{O}(d)^m/\sim\\
    \pi(\mathbf{S}_{1:m}) &= \mathbf{S}_{2:m}\mathbf{S}_1^T. \label{eq:pi}
\end{align}
Let $\widetilde{\mathbf{S}} \in \mathbb{O}(d)^m/\sim$, then
\begin{align}
    \pi^{-1}(\widetilde{\mathbf{S}}) &= \left\{\begin{bmatrix}
    \mathbf{Q}\\\widetilde{\mathbf{S}}\mathbf{Q}
    \end{bmatrix}: \mathbf{Q} \in \mathbb{O}(d)\right\} = \{\mathbf{S} \in \mathbb{O}(d)^m: \mathbf{S}_{i+1}\mathbf{S}_1^T = \widetilde{\mathbf{S}}_{i}, i \in [1,m-1]\}. \label{eq:pi_inv_wtS}
\end{align}

The Riemannian metric $g$ on $\mathbb{O}(d)^m$ is the canonical one given by
\begin{align}
g(\mathbf{Z},\mathbf{W}) \coloneqq \text{Tr}(\mathbf{Z}^T\mathbf{W}) = \sum_1^m \text{Tr}(\mathbf{Z}_i^T\mathbf{W}_i) \text{ where } \mathbf{Z},\mathbf{W} \in T_{\mathbf{S}}\mathbb{O}(d)^m \subseteq \mathbb{R}^{md \times d}. \label{eq:g_Z_W}
\end{align}
By a simple extension of the $m=1$ case \citea{absil2009optimization}, it is easy to deduce the following result.
\begin{prop}
\label{prop:T_SOdm}
For $\mathbf{S} \in \pi^{-1}(\widetilde{\mathbf{S}})$, the tangent space to $\mathbb{O}(d)^m$ at $\mathbf{S}$ is
\begin{align}
    T_{\mathbf{S}}\mathbb{O}(d)^m &= \{[\mathbf{S}_i\boldsymbol{\Omega}_i]_1^m: \boldsymbol{\Omega}_i \in \Skew(d)\}. \label{eq:T_SOdm}
\end{align}
The orthogonal projection of $\boldsymbol{\xi} = [\boldsymbol{\xi}_i]_1^m$, where $\boldsymbol{\xi}_i \in \mathbb{R}^{d \times d}$, onto $T_{\mathbf{S}}\mathbb{O}(d)^m$ is
\begin{align}
    P_{\mathbf{S}}\left(\boldsymbol{\xi}\right) = \argmin_{\substack{[\mathbf{S}_i\boldsymbol{\Omega}_i]_1^m\\\boldsymbol{\Omega}_i \in \Skew(d)}} \sum_1^m\left\|\boldsymbol{\xi}_i - \mathbf{S}_i\boldsymbol{\Omega}_i\right\|_F^2 = [\mathbf{S}_i\Skew(\mathbf{S}_i^T\boldsymbol{\xi}_i)]_1^m. \label{eq:P_S_xi}
\end{align}
\end{prop}

Then, $\pi^{-1}(\widetilde{\mathbf{S}})$ admits a tangent space at $\mathbf{S} \in \pi^{-1}(\widetilde{\mathbf{S}})$ called the vertical space $\mathcal{V}_{\mathbf{S}}$ at $\mathbf{S}$. The horizontal space $\mathcal{H}_{\mathbf{S}}$ at $\mathbf{S}$ is the subspace of $T_{\mathbf{S}}\mathbb{O}(d)^m$ that is the orthogonal complement to the vertical space $\mathcal{V}_{\mathbf{S}}$.

\begin{prop}
\label{prop:V_S_H_S}
The vertical space $\mathcal{V}_{\mathbf{S}}$ at $\mathbf{S} \in \pi^{-1}(\widetilde{\mathbf{S}})$ is
\begin{align}
    \mathcal{V}_{\mathbf{S}} = \{\mathbf{S}\boldsymbol{\Omega}: \boldsymbol{\Omega} \in \Skew(d)\}. \label{eq:V_S}
\end{align}
The orthogonal projection of $\mathbf{Z} = [\mathbf{S}_i\boldsymbol{\Omega}_i]_1^m \in T_{\mathbf{S}}\mathbb{O}(d)^m$ onto $\mathcal{V}_{\mathbf{S}}$ is
\begin{align}
    P^{v}_{\mathbf{S}}([\mathbf{S}_i\boldsymbol{\Omega}_i]_1^m) = \left[\mathbf{S}_i\argmin_{\boldsymbol{\Omega}\in \Skew(d)} \sum_1^m\left\|\mathbf{S}_j(\boldsymbol{\Omega}_j-\boldsymbol{\Omega})\right\|_F^2\right]_1^m = \left[\mathbf{S}_i\left(\frac{\sum_1^m\boldsymbol{\Omega}_i}{m}\right)\right]_1^m. \label{eq:P^v_S}
\end{align}
The horizontal space $\mathcal{H}_{\mathbf{S}}$ at $\mathbf{S} \in \pi^{-1}(\widetilde{\mathbf{S}})$ is
\begin{align}
    \mathcal{H}_{\mathbf{S}} &= \left\{[\mathbf{S}_i\boldsymbol{\Omega}_i]_1^m, \boldsymbol{\Omega}_i \in \Skew(d), \sum_1^m \boldsymbol{\Omega}_i= 0\right\}. \label{eq:H_S}
\end{align}
The orthogonal projection of $\mathbf{Z} = [\mathbf{S}_i\boldsymbol{\Omega}_i]_1^m \in T_{\mathbf{S}}\mathbb{O}(d)^m$ to $\mathcal{H}_{\mathbf{S}}$ is
\begin{align}
    P^{h}_{\mathbf{S}}([\mathbf{S}_i\boldsymbol{\Omega}_i]_1^m) = [\mathbf{S}_i\boldsymbol{\Omega}_i]_1^m - P^{v}_{\mathbf{S}}([\mathbf{S}_i\boldsymbol{\Omega}_i]_1^m) = \left[\mathbf{S}_i\left(\boldsymbol{\Omega}_i-\frac{\sum_1^m\boldsymbol{\Omega}_i}{m}\right)\right]_1^m \label{eq:P^h_S}
\end{align}
\end{prop}

Note that $T_\mathbf{S}\mathbb{O}(d)^m$ is a vector space of dimension $md(d-1)/2$ and $\mathcal{V}_{\mathbf{S}}$ forms a $d(d-1)/2$ dimensional subspace of $T_{\mathbf{S}}\mathbb{O}(d)^m$. The dimension of $\mathcal{H}_{\mathbf{S}}$ and of $T_{\widetilde{\mathbf{S}}}\mathbb{O}(d)^m/\sim$ is $(m-1)d(d-1)/2$. In particular, $\mathcal{H}_{\mathbf{S}}$ can be identified with $T_{\widetilde{\mathbf{S}}}\mathbb{O}(d)^m/\sim$. Let $\widetilde{\mathbf{S}} \in \mathbb{O}(d)^{m}/\sim$ and $\widetilde{\mathbf{Z}} \in T_{\widetilde{\mathbf{S}}}\mathbb{O}(d)^{m}/\sim$. Then the horizontal lift of $\widetilde{\mathbf{Z}}$ at $\mathbf{S} \in \pi^{-1}(\widetilde{\mathbf{S}})$ is defined as $\overline{\widetilde{\mathbf{Z}}} \in \mathcal{H}_{\mathbf{S}}$ such that for each $i \in [1,m-1]$,
\begin{align}
    D\pi[\mathbf{S}]\left(\overline{\widetilde{\mathbf{Z}}}\right)_i = \widetilde{\mathbf{Z}}_i. \label{eq:hlift_def}
\end{align}

\begin{prop}
\label{prop:hlift_char}
Let $\widetilde{\mathbf{S}} \in \mathbb{O}(d)^{m}/\sim$ and $\widetilde{\mathbf{Z}} \in T_{\widetilde{\mathbf{S}}}\mathbb{O}(d)^{m}/\sim$. Let $\mathbf{Z}$ be the horizontal lift of $\widetilde{\mathbf{Z}}$ at $\mathbf{S} \in \pi^{-1}(\widetilde{\mathbf{S}})$. If $(\widetilde{\boldsymbol{\Omega}}_i)_1^{m-1} \subseteq \Skew(d)$ are such that $\widetilde{\mathbf{Z}}_i = \widetilde{\mathbf{S}}_i\widetilde{\boldsymbol{\Omega}}_i$, and $(\boldsymbol{\Omega}_i)_1^m \subseteq \Skew(d)$ are such that $\mathbf{Z}_i=\mathbf{S}_i\boldsymbol{\Omega}_i$ and $\sum_1^m \boldsymbol{\Omega}_i = 0$, then
\begin{align}
    \boldsymbol{\Omega}_1 &= -\frac{1}{m}\mathbf{S}_1^T\left(\sum_1^{m-1}\widetilde{\boldsymbol{\Omega}}_i\right)\mathbf{S}_1 \label{eq:hlift1}\\
    \boldsymbol{\Omega}_{i+1} &= \mathbf{S}_1^T\widetilde{\boldsymbol{\Omega}}_i\mathbf{S}_1 + \boldsymbol{\Omega}_1 \text{ for all } i \in [1,m-1]. \label{eq:hlifti}
\end{align}
Moreover, the linear system above has full rank, and thus the horizontal lift $\mathbf{Z}$ of $\widetilde{\mathbf{Z}}$ at $\mathbf{S} \in \mathbb{O}(d)^m$ is a unique element of $\mathcal{H}_{\mathbf{S}}$.
\end{prop}

\begin{prop}
\label{prop:g_tilde}
Let $\widetilde{\mathbf{Z}},\widetilde{\mathbf{W}} \in T_{\widetilde{\mathbf{S}}}\mathbb{O}(d)^{m}/\sim$ and $\mathbf{Z}, \mathbf{W} \in T_{\mathbf{S}}\mathbb{O}(d)^m$ be their the horizontal lifts at $\mathbf{S} \in \pi^{-1}(\widetilde{\mathbf{S}})$. Then
\begin{align}
    \widetilde{g}(\widetilde{\mathbf{Z}},\widetilde{\mathbf{W}}) \coloneqq g(\mathbf{Z}, \mathbf{W})
\end{align}
defines a Riemannian metric on $\mathbb{O}(d)^m/\sim$.
\end{prop}
We note that $\mathcal{H}_{\mathbf{S}}$ with the canonical metric $g$, is isometric to $T_\mathbf{S}\mathbb{O}(d)^{m}/\sim$ (equivalently $T_\mathbf{S}\mathbb{O}(d)^{m-1}$) when equipped with the above metric $\widetilde{g}$.

Now, coming back to the alignment error $F(\mathbf{S}) = \text{Tr}(\mathbf{C}\mathbf{S}\mathbf{S}^T)$ defined on $\mathbb{O}(d)^m$. It induces the following function on $\mathbb{O}(d)^{m}/\sim$ (again identified with $\mathbb{O}(d)^{m-1}$),
\begin{align}
    \widetilde{F}(\widetilde{\mathbf{S}}) = \text{Tr}\left(\mathbf{C}\begin{bmatrix}
    \mathbf{I}_d\\\widetilde{\mathbf{S}}
    \end{bmatrix}\begin{bmatrix}
    \mathbf{I}_d\\\widetilde{\mathbf{S}}
    \end{bmatrix}^T\right). \label{eq:Ftilde}
\end{align}
In particular, $F = \widetilde{F} \circ \pi$. It is easy to see that if $\mathbf{S}_1 \sim \mathbf{S}_2$ then $\widetilde{F} \circ \pi(\mathbf{S}_1) = \widetilde{F} \circ \pi(\mathbf{S}_2)$.

\begin{prop}
\label{prop:gradFS}
The horizontal lift of $\grad \widetilde{F}(\widetilde{\mathbf{S}})$ at $\mathbf{S} \in \pi^{-1}(\widetilde{\mathbf{S}})$ is
\begin{align}
    \overline{\grad \widetilde{F}(\widetilde{\mathbf{S}})} = \grad F(\mathbf{S}) = [\mathbf{S}_i\boldsymbol{\Omega}_i]_1^{m} \label{eq:gradFS}
\end{align}
where $\boldsymbol{\Omega}_i \coloneqq \mathbf{S}_i^T[\mathbf{C}\mathbf{S}]_i - [\mathbf{C}\mathbf{S}]_i^T\mathbf{S}_i \in \Skew(d)$ and $\sum_1^m \boldsymbol{\Omega}_i = \mathbf{S}^T\mathbf{C}\mathbf{S} - \mathbf{S}^T\mathbf{C}\mathbf{S} = 0$ (which validates that $\overline{\grad \widetilde{F}(\widetilde{\mathbf{S}})}$ is indeed in $\mathcal{H}_{\mathbf{S}}$, see Proposition~\ref{prop:V_S_H_S}).
\end{prop}

Due to the above proposition, the set of critical points of $\widetilde{F}$ is given by
\begin{align}
    \widetilde{\mathcal{C}} &= \{\widetilde{\mathbf{S}} \in \mathbb{O}(d)^m/\sim: \grad \widetilde{F}(\widetilde{\mathbf{S}}) = 0\}\\
    &= \{\widetilde{\mathbf{S}} \in \mathbb{O}(d)^m/\sim: \mathbf{S}_i^T[\mathbf{C}\mathbf{S}]_i = [\mathbf{C}\mathbf{S}]_i^T\mathbf{S}_i, \text{ for all } i \in [1,m], \mathbf{S} \in \pi^{-1}(\widetilde{\mathbf{S}})\}, \label{eq:crit_pts}
\end{align}
and that of $F$ is,
\begin{align}
     \mathcal{C} &= \{\mathbf{S} \in \mathbb{O}(d)^m: \grad F(\mathbf{S}) = 0\}\\
    &= \{\mathbf{S} \in \mathbb{O}(d)^m: \mathbf{S}_i^T[\mathbf{C}\mathbf{S}]_i = [\mathbf{C}\mathbf{S}]_i^T\mathbf{S}_i, \text{ for all } i \in [1,m]\}. \label{eq:crit_pts2}
\end{align}

The following remark follows trivially from Eq.~(\ref{eq:crit_pts}) and (\ref{eq:crit_pts2}).
\begin{rmk}
\label{rmk:crit_pt_F_Ftilde}
If $\mathbf{S}$ is a critical point of $F$ i.e. $\mathbf{S} \in \mathcal{C}$ then $\pi(\mathbf{S})$ is a critical point of $\widetilde{F}$ i.e. $\pi(\mathbf{S}) \in \widetilde{C}$. Similarly, if $\widetilde{\mathbf{S}} \in \widetilde{C}$ then $\mathbf{S} \in \mathcal{C}$ for all $\mathbf{S} \in \pi^{-1}(\widetilde{\mathbf{S}})$.
\end{rmk}


% Alternatively,
% \begin{align}
%     0 &= \grad F(\mathbf{S})_i\\
%     &= \mathbf{S}_{i+1}\overline{\grad F(\mathbf{S})}_1^T + \overline{\grad F(\mathbf{S})}_{i+1}\mathbf{S}_1^T\\
%     &= \mathbf{S}_{i+1}([\mathbf{C}\mathbf{S}]_1 - \mathbf{S}_1[\mathbf{C}\mathbf{S}]_1^T\mathbf{S}_1)^T + ([\mathbf{C}\mathbf{S}]_{i+1} - \mathbf{S}_{i+1}[\mathbf{C}\mathbf{S}]_{i+1}^T\mathbf{S}_{i+1})\mathbf{S}_1^T\\
%     &= \mathbf{S}_{i}C_{11} + \sum_{j=2}^{m}\mathbf{S}_{i}\mathbf{S}_{j-1}^TC_{j1} - \mathbf{S}_{i}(C_{11} + \sum_{j=2}^{m}C_{1j}\mathbf{S}_{j-1})+\\
%     &\qquad\qquad C_{i+1,1} + \sum_{j=2}^{m}C_{i+1,j}\mathbf{S}_{j-1} - (\mathbf{S}_iC_{1,i+1}\mathbf{S}_i + \sum_{j=2}^{m}\mathbf{S}_i\mathbf{S}_{j-1}^TC_{j,i+1}\mathbf{S}_i)\\
%     &= \mathbf{S}_i\left(\mathbf{S}_i^TC_{i+1,1} - C_{1,i+1}\mathbf{S}_i + \sum_{j=1}^{m-1}\mathbf{S}_i^TC_{i+1,j+1}\mathbf{S}_{j} - \mathbf{S}_{j}^{T}C_{j+1,i+1}\mathbf{S}_i + \mathbf{S}_{j}^{T}C_{j+1,1} - C_{1,j+1}\mathbf{S}_{j}\right)\\
%     &= \mathbf{S}_i\left(\sum_{j=0}^{m-1}\mathbf{S}_i^TC_{i+1,j+1}\mathbf{S}_{j} - \mathbf{S}_{j}^{T}C_{j+1,i+1}\mathbf{S}_i + \mathbf{S}_{j}^{T}C_{j+1,1} - C_{1,j+1}\mathbf{S}_{j}\right)\\
%     &= 2\mathbf{S}_i \left(\Skew\left(\mathbf{S}_i^T \left[C\begin{bmatrix}I_d\\S\end{bmatrix}\right]_{i+1}\right) - \Skew\left( \left[C\begin{bmatrix}I_d\\S\end{bmatrix}\right]_{1}\right)\right)
% \end{align}
% where $\mathbf{S}_0 = I_d$. Summing up for $i=1:m-1$ we obtain
% \begin{align}
%     \sum_{j=0}^{m-1} \mathbf{S}_j^TC_{j+1,1} - C_{1,j+1}\mathbf{S}_j = 0 \implies \left[C\begin{bmatrix}
%     I_d\\S
%     \end{bmatrix}\right]_1 = \left[C\begin{bmatrix}
%     I_d\\S
%     \end{bmatrix}\right]_1^T.
% \end{align}
% and for $i=1:m-1$,
% \begin{align}
%     \sum_{j=0}^{m-1}\mathbf{S}_i^TC_{i+1,j+1}\mathbf{S}_{j} - \mathbf{S}_{j}^{T}C_{j+1,i+1}\mathbf{S}_i = 0 \implies \mathbf{S}_i^T \left[C\begin{bmatrix}
%     I_d\\S
%     \end{bmatrix}\right]_i = \left[C\begin{bmatrix}
%     I_d\\S
%     \end{bmatrix}\right]_i^T\mathbf{S}_i
% \end{align}
% The above equations can be written in a more compact form as
% \begin{align}
%     \blockdiag\left(C\begin{bmatrix}
%     I_d\\S
%     \end{bmatrix}\begin{bmatrix}I_d & \mathbf{S}^T\end{bmatrix}\right) = \blockdiag\left(\begin{bmatrix}I_d\\S\end{bmatrix}\begin{bmatrix}
%     I_d & \mathbf{S}^T
%     \end{bmatrix}C\right)
% \end{align}

\begin{prop}
\label{prop:DgradFSZ}
Let $\widetilde{\mathbf{S}} \in \widetilde{\mathcal{C}}$ then for every $\mathbf{S} \in \pi^{-1}(\widetilde{\mathbf{S}})$ and $\mathbf{Z} \in T_\mathbf{S}\mathbb{O}(d)^m$,
\begin{align}
    D\grad F(\mathbf{S})[\mathbf{Z}] &= \left[\mathbf{S}_i(\mathbf{S}_i^T[\mathbf{C}\mathbf{Z}]_i - [\mathbf{C}\mathbf{Z}]_i^T\mathbf{S}_i - [\mathbf{C}\mathbf{S}]_i^T\mathbf{Z}_i + \mathbf{Z}_i^T[\mathbf{C}\mathbf{S}]_i)\right]_1^m.
\end{align}
\end{prop}

Let $\nabla$ be the Levi-Civita connection (also known as the Riemannian connection) on $\mathbb{O}(d)^m$ and $\widetilde{\nabla}$ be the induced connection on $\mathbb{O}(d)^m / \sim$. Then the following result follows from Proposition~\ref{prop:DgradFSZ} and the definition of the Riemannian Hessian operator \citeb[Section 5.5]{absil2009optimization}.
\begin{prop}
\label{prop:HessFSZ}
Let $\widetilde{\mathbf{S}} \in \widetilde{\mathcal{C}}$. Let $\widetilde{\mathbf{Z}} \in T_{\widetilde{\mathbf{S}}}\mathbb{O}(d)^{m}/\sim$. Let $\mathbf{Z}$ be the horizontal lift of $\widetilde{\mathbf{Z}}$ at $\mathbf{S} \in \pi^{-1}(\widetilde{\mathbf{S}})$. Then the horizontal lift of $\Hess \widetilde{F}(\widetilde{\mathbf{S}})[\widetilde{\mathbf{Z}}]$ at $\mathbf{S}$ is
\begin{align}
    \overline{\Hess \widetilde{F}(\widetilde{\mathbf{S}})[\widetilde{\mathbf{Z}}]} = [\mathbf{S}_i\widehat{\boldsymbol{\Omega}}_i]_1^m \label{eq:eq3}
\end{align}
where
\begin{align}
    \widehat{\boldsymbol{\Omega}}_i = \mathbf{S}_i^T[\mathbf{C}\mathbf{Z}]_i - [\mathbf{C}\mathbf{Z}]_i^T\mathbf{S}_i - [\mathbf{C}\mathbf{S}]_i^T\mathbf{Z}_i + \mathbf{Z}_i^T[\mathbf{C}\mathbf{S}]_i. \label{eq:Omega_hat_i}
\end{align}
\end{prop}

In the following, we obtain a compact representation for $\widehat{\boldsymbol{\Omega}}_i$. We first define certain matrices, then use them to obtain an expression for $\widehat{\boldsymbol{\Omega}}_i$ and then describe their structure. Recall that $\mathbf{C} = \mathbf{D} - \mathbf{B}\boldsymbol{\mathcal{L}}_{\Gamma}^\dagger \mathbf{B}^T$ (see Eq.~(\ref{eq:GPOP}) and Remark~\ref{rmk:L0DB}) and for convenience define
\begin{align}
    \mathbf{B}(\mathbf{S}) &\coloneqq \blockdiag((\mathbf{S}_i)_1^m)^T\ \mathbf{B} = [\mathbf{S}_i^T\mathbf{B}_i]_1^m \label{eq:BofS}\\
    \mathbf{D}(\mathbf{S}) &\coloneqq \blockdiag((\mathbf{S}_i)_1^m)^T\ \mathbf{D}\ \blockdiag((\mathbf{S}_i)_1^m) = \blockdiag((\mathbf{S}_i^T\mathbf{D}_{ii}\mathbf{S}_i)_1^m).
\end{align}
Then define
\begin{align}
    \mathbf{C}(\mathbf{S}) &\coloneqq \blockdiag((\mathbf{S}_i)_1^m)^T\ \mathbf{C}\ \blockdiag((\mathbf{S}_i)_1^m) = \mathbf{D}(\mathbf{S}) - \mathbf{B}(\mathbf{S})\boldsymbol{\mathcal{L}}_{\Gamma}^\dagger \mathbf{B}(\mathbf{S})^T\label{eq:C_of_S}\\
    \widehat{\mathbf{C}}(\mathbf{S}) &\coloneqq \blockdiag(([\mathbf{C}(\mathbf{S})\mathbf{I}_d^m]_i)_1^m) \label{eq:C_hat}\\
    \mathbf{L}(\mathbf{S}) &\coloneqq \widehat{\mathbf{C}}(\mathbf{S}) - \mathbf{C}(\mathbf{S}). \label{eq:L_of_S}
\end{align}
\begin{prop}
\label{prop:Omega_hat_compact}
Consider the same setup as in Proposition~\ref{prop:HessFSZ}. Then
\begin{align}
    \widehat{\boldsymbol{\Omega}}_i &= [\mathbf{L}(\mathbf{S})\boldsymbol{\Omega}]_i^T - [\mathbf{L}(\mathbf{S})\boldsymbol{\Omega}]_i \label{eq:eq4}
\end{align}
where $\boldsymbol{\Omega}=[\boldsymbol{\Omega}_i]_1^m$ and $\boldsymbol{\Omega}_i \in \Skew(d)$ is such that $\mathbf{Z}_i = \mathbf{S}_i\boldsymbol{\Omega}_i$ and $\sum_1^m\boldsymbol{\Omega}_i = 0$.
\end{prop}

Combining the above equation with Proposition~\ref{prop:HessFSZ}, we obtain a slightly compact representation of the horizontal lift of the Hessian. We end this subsection by giving remarks that reveal the structure of $\mathbf{C}(\mathbf{S})$, $\widehat{\mathbf{C}}(\mathbf{S})$ and $\mathbf{L}(\mathbf{S})$.

\begin{rmk}
\label{rmk:C_S_structure}
Since $\blockdiag((\mathbf{S}_i)_1^m)$ is an orthogonal matrix, $\mathbf{C}(\mathbf{S})$ is unitarily equivalent to $\mathbf{C}$. Thus, $\mathbf{C}(\mathbf{S}) \in \Sym(md)$, $\mathbf{C}(\mathbf{S}) \succeq 0$, $\rank(\mathbf{C}(\mathbf{S})) = \rank(\mathbf{C})$ and the $(i,j)$th block of size $d$ in $\mathbf{C}(\mathbf{S})$ is $\mathbf{C}(\mathbf{S})_{ij} = \delta_{ij}\mathbf{D}(\mathbf{S})_{ii} - \mathbf{B}(\mathbf{S})_i\boldsymbol{\mathcal{L}}_{\Gamma}^\dagger \mathbf{B}(\mathbf{S})_j^T$ where $\mathbf{B}(\mathbf{S})_i$ is the $i$th row block of $\mathbf{B}(\mathbf{S})$ of dimension $d \times (m+n)$.
\end{rmk}

\begin{rmk}
\label{rmk:StildeCtilde}
From the definition of $\mathcal{C}$ (see Eq.~(\ref{eq:crit_pts2})), $\mathbf{S} \in \mathcal{C}$ if and only if $\widehat{\mathbf{C}}(\mathbf{S}) \in \Sym(md)$ i.e. for each $i \in [1,m]$,
\begin{align}
    [\mathbf{C}(\mathbf{S})\mathbf{I}_d^m]_i &= \sum_{j=1}^{m}\mathbf{C}(\mathbf{S})_{ij} = \mathbf{S}_i^T[\mathbf{C}\mathbf{S}]_i = [\mathbf{C}\mathbf{S}]_i^T\mathbf{S}_i = \sum_{j=1}^{m}\mathbf{C}(\mathbf{S})_{ij}^T = [\mathbf{C}(\mathbf{S})\mathbf{I}_d^m]_i^T. \label{eq:eq2}
\end{align}
\end{rmk}

\begin{rmk}
\label{rmk:C_hat_L_structure}
If $\widetilde{\mathbf{S}} \in \widetilde{\mathcal{C}}$ then for every $\mathbf{S} \in \pi^{-1}(\mathbf{S})$, the following are easy to deduce.
\begin{enumerate}
    \item $\mathbf{L}(\mathbf{S}) \in \Sym(md)$.
    \item $\sum_{j=1}^{m}\mathbf{L}(\mathbf{S})_{ij} = \sum_{j=1}^{m}\mathbf{L}(\mathbf{S})_{ji} = 0$ for all $i \in [1,m]$ (see Eq.~(\ref{eq:eq2})).
    \item For each $i \in [1,d]$, the vector $\mathbf{1}_m \otimes \mathbf{e}^d_i$ lies in the kernel of $\mathbf{L}(\mathbf{S})$ and thus the rank of $\mathbf{L}(\mathbf{S})$ is atmost $(m-1)d$.
    \item If $\boldsymbol{\Omega} = [\boldsymbol{\Omega}_0]_1^m$ for some $\boldsymbol{\Omega}_0 \in \Skew(d)$ then $\mathbf{L}(\mathbf{S})\boldsymbol{\Omega} = 0$.
    \item The $(i,j)$th block of size $d$ in $\mathbf{L}(\mathbf{S})$ is
    \begin{align}
        \mathbf{L}(\mathbf{S})_{ij} &= -\delta_{ij}\mathbf{B}(\mathbf{S})_i\boldsymbol{\mathcal{L}}_{\Gamma}^\dagger \mathbf{B}(\mathbf{S})^T\mathbf{I}^m_d + \mathbf{B}(\mathbf{S})_i\boldsymbol{\mathcal{L}}_{\Gamma}^\dagger \mathbf{B}(\mathbf{S})_j^T\\
        &= -\delta_{ij}(\mathbf{I}^m_d)^T\mathbf{B}(\mathbf{S})\boldsymbol{\mathcal{L}}_{\Gamma}^\dagger \mathbf{B}(\mathbf{S})_i^T + \mathbf{B}(\mathbf{S})_i\boldsymbol{\mathcal{L}}_{\Gamma}^\dagger \mathbf{B}(\mathbf{S})_j^T \label{eq:L2}
    \end{align}
    and in particular, $\mathbf{L}(\mathbf{S})$ does not depend on $\mathbf{D}(\mathbf{S})$.
    \item Let $\mathbf{Q} \in \mathbb{O}(d)$ then $\mathbf{L}(\mathbf{S}\mathbf{Q}) = (\mathbf{I}_m \otimes \mathbf{Q})^T\mathbf{L}(\mathbf{S})(\mathbf{I}_m \otimes \mathbf{Q})$ (follows from Eq.~(\ref{eq:C_of_S}, \ref{eq:C_hat}, \ref{eq:eq2})) and since $\mathbf{I}_m \otimes \mathbf{Q}$ is an orthogonal matrix, $L(\mathbf{S}\mathbf{Q})$ is unitarily equivalent to $\mathbf{L}(\mathbf{S})$.
\end{enumerate}
\end{rmk}

\subsection{Non-singular and Positive Definite Hessian}
\label{subsec:non_sing_pos_def_hess}
A non-degenerate local minimum is defined to be the critical point at which the Hessian is positive definite. Similarly, a non-degenerate critical point is the one where the Hessian is non-singular. We proceed to derive the equations to identify the conditions under which the Hessian is non-singular and positive definite, which in turn characterize the non-degenerate critical points and local minima. First, we need the following result.
\begin{prop}
\label{prop:HessFSZZ}
Let $\widetilde{\mathbf{S}} \in \widetilde{\mathcal{C}}$. Let $\widetilde{\mathbf{Z}} \in T_{\widetilde{\mathbf{S}}}\mathbb{O}(d)^{m}/\sim$. Let $\mathbf{Z}$ be the horizontal lift of $\widetilde{\mathbf{Z}}$ at $\mathbf{S} \in \pi^{-1}(\widetilde{\mathbf{S}})$. Then
\begin{align}
    \widetilde{g}(\Hess \widetilde{F}(\widetilde{\mathbf{S}})[\widetilde{\mathbf{Z}}],\widetilde{\mathbf{Z}}) &= -2\text{Tr}(\boldsymbol{\Omega}^T\mathbf{L}(\mathbf{S})\boldsymbol{\Omega}) \label{eq:TrOmegaTLSOmega}
\end{align}
where $\boldsymbol{\Omega}=[\boldsymbol{\Omega}_i]_1^m$ and $\boldsymbol{\Omega}_i \in \Skew(d)$ is such that $\mathbf{Z}_i = \mathbf{S}_i\boldsymbol{\Omega}_i$ and $\sum_1^m\boldsymbol{\Omega}_i = 0$.
\end{prop}

Then the non-singularity and the positive definiteness of the Hessian amounts to the right side of Eq.~(\ref{eq:TrOmegaTLSOmega}) being non-zero and positive, respectively, for every non-zero $\boldsymbol{\Omega}$. Although $\mathbf{C}(\mathbf{S})$, and thus $\mathbf{L}(\mathbf{S})$, can be calculated from the patch framework $\Theta$ and the alignment $\mathbf{S}$, it is not obvious how to test the above practically. The main reason being that $\boldsymbol{\Omega}$ in Eq.~(\ref{eq:TrOmegaTLSOmega}) is not unconstrained, and in fact has a specific structure.

Therefore, for the above reason, we are going to manipulate Eq.~(\ref{eq:TrOmegaTLSOmega}), utilizing the structure of $\Omega$. The aim is to obtain an expression of the form $\boldsymbol{\omega}^T \mathbb{L}(\mathbf{S})\boldsymbol{\omega}$ where (i) the vector $\boldsymbol{\omega}$ is essentially unconstrained and (ii) $\boldsymbol{\Omega}$ and $\mathbf{L}(\mathbf{S})$ are related to $\boldsymbol{\omega}$ and $\mathbb{L}(\mathbf{S})$, respectively, through permutation matrices and vectorization operations (and thus the two pairs carry the same information). To achieve that, we first define certain matrices, then rewrite Eq.~(\ref{eq:TrOmegaTLSOmega}) in terms of those matrices and then describe their structure.

To this end, for $\boldsymbol{\Omega} = [\boldsymbol{\Omega}_i]_1^m$ where $\boldsymbol{\Omega}_i \in \Skew(d)$, let $\{\boldsymbol{\Omega}_{i}(r,s): 1 \leq r < s \leq d\}$ be the elements in the upper triangular region of $\boldsymbol{\Omega}_i$. For a fixed pair $(r,s)$ such that $1 \leq r < s \leq d$, define the column vector $\boldsymbol{\omega}_{r,s} \coloneqq [\boldsymbol{\Omega}_{i}(r,s)]_{i=1}^{m} \in \mathbb{R}^m$, a vertical stack of the $(r,s)$th element of each $\boldsymbol{\Omega}_i$. Then there exists a permutation matrix $\mathbf{P}$ such that
\begin{align}
    \mathbf{P}\boldsymbol{\Omega} &= \begin{bmatrix}
    \mathbf{0}_{m} & \boldsymbol{\omega}_{1,2} &  \ldots & \boldsymbol{\omega}_{1,d-1} & \boldsymbol{\omega}_{1,d}\\
    -\boldsymbol{\omega}_{1,2} & \mathbf{0}_{m}  & \ldots & \boldsymbol{\omega}_{2,d-1} & \boldsymbol{\omega}_{2,d}\\
        \vdots & \vdots & \vdots & \vdots & \vdots\\
    -\boldsymbol{\omega}_{1,d-1} & -\boldsymbol{\omega}_{2,d-1} & \ldots & \mathbf{0}_{m} & \boldsymbol{\omega}_{d-1,d}\\
    -\boldsymbol{\omega}_{1,d} & -\boldsymbol{\omega}_{2,d}  & \ldots & -\boldsymbol{\omega}_{d-1,d} & \mathbf{0}_{m}
    \end{bmatrix}. \label{eq:P0Omega}
\end{align}
In words, for $1 \leq r < s \leq d$, the $(r,s)$th block of $\mathbf{P}\boldsymbol{\Omega}$ is a vertical stack of the $(r,s)$th element of each $\boldsymbol{\Omega}_{i}$. For $r = s$, this is just a zero vector and for $r > s$, this is $-\boldsymbol{\omega}_{s,r}$.

Then, we collect the (strictly) upper triangular elements of $\mathbf{P}\boldsymbol{\Omega}$ in the column-major order in the vector $\boldsymbol{\omega}$. Note that $\mathbf{P}\boldsymbol{\Omega}$ can be fully described by $\boldsymbol{\omega}$. In particular, there exist a block matrix $\overline{\mathbf{P}}$ of size $d(d-1)/2 \times d^2$ siuch that $\vecz (\mathbf{P}\boldsymbol{\Omega}) = \overline{\mathbf{P}}^T \boldsymbol{\omega}$. The blocks of $\overline{\mathbf{P}}$ when indexed using tuples $(r,s)$ and $(p,q)$ where $1 \leq r < s \leq d$ and $p,q \in [1,d]$, are given by,
\begin{align}
    \overline{\mathbf{P}}_{(r,s),(p,q)} &= \left\{\begin{matrix}\mathbf{0}_{m \times m}, & p = q\\
    \delta_{pr}\delta_{qs}\mathbf{I}_m, & p < q\\
    -\delta_{ps}\delta_{qr}\mathbf{I}_m, & p > q.
    \end{matrix}\right.
\end{align}

Finally, we define
\begin{align}
    \mathcal{B}(\mathbf{S}) &\coloneqq \mathbf{P}\mathbf{B}(\mathbf{S}) \label{eq:mathcal_B}\\
    \boldsymbol{\mathcal{L}}(\mathbf{S}) &\coloneqq \mathbf{P}\mathbf{L}(\mathbf{S})\mathbf{P}^T \label{eq:mathcal_L}\\
    \mathbb{L}(\mathbf{S}) &\coloneqq \overline{\mathbf{P}}(\mathbf{I}_d \otimes \boldsymbol{\mathcal{L}}(\mathbf{S}))\overline{\mathbf{P}}^T \label{eq:mathbb_L}
\end{align}
and then note the following,
\begin{prop}
\label{prop:Omega^TLSOmega2}
Consider the same setup as in Proposition~\ref{prop:HessFSZZ}. Then
\begin{align}
    \text{Tr}(\boldsymbol{\Omega}^T\mathbf{L}(\mathbf{S})\boldsymbol{\Omega}) = \boldsymbol{\omega}^T\mathbb{L}(\mathbf{S})\boldsymbol{\omega} \label{eq:omega^TmbbLomega}
\end{align}
\end{prop}
The following remarks reveal the structure of $\mathcal{B}(\mathbf{S})$, $\boldsymbol{\mathcal{L}}(\mathbf{S})$ and $\mathbb{L}(\mathbf{S})$.
\begin{rmk}
\label{rmk:mathcalBS}
For $p \in [1,d]$, the $p$th row-block of $\mathcal{B}(\mathbf{S})$, $\mathcal{B}(\mathbf{S})_p$, is of size $m \times (n+m)$, and can be viewed as a vertical stack of the $p$th row of each $\mathbf{B}(\mathbf{S})_i$, $i \in [1,m]$. In particular, $\mathcal{B}(\mathbf{S})_p$ depends only on the $p$th coordinate of the local views (also see Remark~\ref{rmk:L0DB}).
\end{rmk}

\begin{rmk}
\label{rmk:mathcalLS}
% For $p,\mathbf{Q} \in [1,d]$ and $i,j \in [1,m]$, let $\boldsymbol{\mathcal{L}}(\mathbf{S})_{pq}$ be the $(p,q)$th block of size $m$ in $\boldsymbol{\mathcal{L}}(\mathbf{S})$ and $\mathbf{L}(\mathbf{S})_{ij}$ be the $(i,j)$th block of size $d$ in $\mathbf{L}(\mathbf{S})$. Then, from Eq.~(\ref{eq:mathcal_L}) it is easy to deduce
% \begin{align}
%     \boldsymbol{\mathcal{L}}(\mathbf{S})_{pq}(i,j) &= \mathbf{L}(\mathbf{S})_{ij}(p,q)\\
%     &= -\delta_{ij} \sum_{k=1}^m \mathbf{B}(\mathbf{S})_{i}(p,:)\boldsymbol{\mathcal{L}}_{\Gamma}^\dagger \mathbf{B}(\mathbf{S})_{k}(q,:)^T + \mathbf{B}(\mathbf{S})_i(p,:)\boldsymbol{\mathcal{L}}_{\Gamma}^\dagger \mathbf{B}(\mathbf{S})_j(q,:)^T\\
%     &= -\delta_{ij}\mathcal{B}(\mathbf{S})_{p}(i,:)\boldsymbol{\mathcal{L}}_{\Gamma}^\dagger \mathcal{B}(\mathbf{S})_{q}^T\mathbf{1}_m + \mathcal{B}(\mathbf{S})_{p}(i,:)\boldsymbol{\mathcal{L}}_{\Gamma}^\dagger \mathcal{B}(\mathbf{S})_{q}(j,:)^T
% \end{align}
% and thus
% \begin{align}
%     \boldsymbol{\mathcal{L}}(\mathbf{S})_{pq} &= -\diag (\mathcal{B}(\mathbf{S})_p\boldsymbol{\mathcal{L}}_{\Gamma}^\dagger \mathcal{B}(\mathbf{S})_{q}^T\mathbf{1}_m) + \mathcal{B}(\mathbf{S})_p\boldsymbol{\mathcal{L}}_{\Gamma}^\dagger \mathcal{B}(\mathbf{S})_q^T
% \end{align}
% Thus $\boldsymbol{\mathcal{L}}(\mathbf{S})_{pq}$ depends on the structure of $\Gamma$ through $\boldsymbol{\mathcal{L}}_{\Gamma}^\dagger$ and the $p$th and $q$th coordinates of the rigidly transformed local views $[\mathbf{S}_iB_i]_1^m$.
If $\widetilde{\mathbf{S}} \in \widetilde{\mathcal{C}}$ then for every $\mathbf{S} \in \pi^{-1}(\widetilde{\mathbf{S}})$, the following are easy to deduce.
\begin{enumerate}
    \item For $p,q \in [1,d]$, it is easy to deduce
    \begin{align}
        \boldsymbol{\mathcal{L}}(\mathbf{S})_{p,q} &= -\diag (\mathcal{B}(\mathbf{S})_p\boldsymbol{\mathcal{L}}_{\Gamma}^\dagger \mathcal{B}(\mathbf{S})_{q}^T\mathbf{1}_m) + \mathcal{B}(\mathbf{S})_p\boldsymbol{\mathcal{L}}_{\Gamma}^\dagger \mathcal{B}(\mathbf{S})_q^T.
    \end{align}
    Thus $\boldsymbol{\mathcal{L}}(\mathbf{S})_{p,q}$ depends on $\Gamma$ through $\boldsymbol{\mathcal{L}}_{\Gamma}^\dagger$ and the $p$th and $q$th coordinates of the rigidly transformed local views $\mathbf{B}(\mathbf{S})$.
    \item Since $\mathbf{L}(\mathbf{S})$ is symmetric, $\boldsymbol{\mathcal{L}}(\mathbf{S})_{q,p} = \boldsymbol{\mathcal{L}}(\mathbf{S})_{p,q}^T$.
    \item $\boldsymbol{\mathcal{L}}(\mathbf{S})_{p,q}$ is a Laplacian-like matrix and the constant vectors are in its kernel.
    \item For each $p \in [1,d]$, the vector $\mathbf{e}^d_p \otimes \mathbf{1}_m$ lies in the kernel of $\boldsymbol{\mathcal{L}}(\mathbf{S})$, thus the rank of $\boldsymbol{\mathcal{L}}(\mathbf{S})$ is atmost $(m-1)d$.
    \item If $\mathbf{Q} \in \mathbb{O}(d)$ then, from Remark~\ref{rmk:C_hat_L_structure} and Eq.~(\ref{eq:mathcal_L}), it follows that $\boldsymbol{\mathcal{L}}(\mathbf{S}\mathbf{Q})$ is unitarily equivalent to $\boldsymbol{\mathcal{L}}(\mathbf{S})$.
\end{enumerate}
\end{rmk}

\begin{rmk}
\label{rmk:mathbb_L_structure}
If $\widetilde{\mathbf{S}} \in \widetilde{\mathcal{C}}$ then for every $\mathbf{S} \in \pi^{-1}(\mathbf{S})$, the following are easy to deduce.
\begin{enumerate}
    \item $\mathbb{L}(\mathbf{S})$ is a block matrix of size $d(d-1)/2$ where each block is of size $m$. Indexing the rows and columns of $\mathbb{L}$ by tuples of the form $(r,s)$ where $1 \leq r < s \leq d$ we have,
    \begin{align}
        \mathbb{L}(\mathbf{S})_{(r_1,s_1),(r_2,s_2)} &= \sum_{\substack{p_1,q_1\in [1,d]\\p_2,q_2\in [1,d]}}\overline{\mathbf{P}}_{(r_1,s_1),(p_1,q_1)}(\mathbf{I}_d \otimes \boldsymbol{\mathcal{L}}(\mathbf{S}))_{(p_1,q_1),(p_2,q_2)}\overline{\mathbf{P}}_{(r_2,s_2),(p_2,q_2)}^T\\
        &= \sum_{p,q_1,q_2\in [1,d]}\overline{\mathbf{P}}_{(r_1,s_1),(p,q_1)}\boldsymbol{\mathcal{L}}(\mathbf{S})_{q_1,q_2}\overline{\mathbf{P}}_{(r_2,s_2),(p,q_2)}^T\\
        &= \left\{\begin{matrix}
        \boldsymbol{\mathcal{L}}(\mathbf{S})_{r,r}+\boldsymbol{\mathcal{L}}(\mathbf{S})_{s,s}, & r_1=r_2=r, s_1=s_2=s\\
        0, & \{r_1,s_1\} \cap \{r_2,s_2\} = \emptyset\\
        \boldsymbol{\mathcal{L}}(\mathbf{S})_{s_1,s_2}, & r_1 = r_2, s_1 \neq s_2\\
        \boldsymbol{\mathcal{L}}(\mathbf{S})_{r_1,r_2}, & s_1 = s_2, r_1 \neq r_2\\
        -\boldsymbol{\mathcal{L}}(\mathbf{S})_{r_1,s_2}, & s_1 = r_2\\
        -\boldsymbol{\mathcal{L}}(\mathbf{S})_{s_1,r_2}, & s_2 = r_1.\\
        \end{matrix}\right.
    \end{align}
    \item Since $\boldsymbol{\mathcal{L}}(\mathbf{S})$ is symmetric, $\mathbb{L}(\mathbf{S}) = \mathbb{L}(\mathbf{S})^T$.
    \item The set of vectors of the form $\boldsymbol{\omega} = [\boldsymbol{\omega}_{r,s}]_{1 \leq r < s \leq d}$ where each $\boldsymbol{\omega}_{r,s}$ is a constant vector, lie in the kernel of $\mathbb{L}(\mathbf{S})$.
    \item The rank of $\mathbb{L}(\mathbf{S})$ is at most $(m-1)d(d-1)/2$.
    \item If $\mathbf{Q} \in \mathbb{O}(d)$ then $\mathbb{L}(\mathbf{S}\mathbf{Q})$ is unitarily equivalent to $\mathbb{L}(\mathbf{S})$.
\end{enumerate}
\end{rmk}

% \begin{dfn}
% Let $\mathbf{S} \in \mathcal{C}$. Then $\mathbb{L}(\mathbf{S})$ is said to have trivial certificate if the null space of $\mathbb{L}$ contains only the vectors of the form $\boldsymbol{\omega} = [\boldsymbol{\omega}_{r,s}]_{1 \leq r < s \leq d}$ where each $\boldsymbol{\omega}_{r,s}$ is a constant vector i.e. if $\mathbb{L}(\mathbf{S})$ has a rank of $(m-1)d(d-1)/2$.
% \end{dfn}

\begin{rmk}
\label{rmk:mathbbL_examples}
For $d=2$, $3$ and $4$, $\mathbb{L}(\mathbf{S})$ is given by (for brevity, we use $\boldsymbol{\mathcal{L}}$ in place of $\boldsymbol{\mathcal{L}}(\mathbf{S})$)
\begin{align}
    [\boldsymbol{\mathcal{L}}_{1,1} + \boldsymbol{\mathcal{L}}_{2,2}],
\end{align}
\begin{align}
    &\begin{bmatrix}
    \boldsymbol{\mathcal{L}}_{1,1} + \boldsymbol{\mathcal{L}}_{2,2} & \boldsymbol{\mathcal{L}}_{2,3} & -\boldsymbol{\mathcal{L}}_{1,3}\\
    \boldsymbol{\mathcal{L}}_{3,2} & \boldsymbol{\mathcal{L}}_{1,1}+\boldsymbol{\mathcal{L}}_{3,3} & \boldsymbol{\mathcal{L}}_{1,2}\\
    -\boldsymbol{\mathcal{L}}_{3,1} & \boldsymbol{\mathcal{L}}_{2,1} & \boldsymbol{\mathcal{L}}_{2,2}+\boldsymbol{\mathcal{L}}_{3,3}
    \end{bmatrix}
\end{align}
and
{\small
\begin{align}
    \begin{bmatrix}
    \boldsymbol{\mathcal{L}}_{1,1}+\boldsymbol{\mathcal{L}}_{2,2} & \boldsymbol{\mathcal{L}}_{2,3} & \boldsymbol{\mathcal{L}}_{2,4} & -\boldsymbol{\mathcal{L}}_{1,3} & -\boldsymbol{\mathcal{L}}_{1,4} & 0\\
    \boldsymbol{\mathcal{L}}_{3,2} & \boldsymbol{\mathcal{L}}_{1,1}+\boldsymbol{\mathcal{L}}_{3,3} & \boldsymbol{\mathcal{L}}_{3,4} & \boldsymbol{\mathcal{L}}_{1,2} & 0 & -\boldsymbol{\mathcal{L}}_{1,4}\\
    \boldsymbol{\mathcal{L}}_{4,2} & \boldsymbol{\mathcal{L}}_{4,3} & \boldsymbol{\mathcal{L}}_{1,1}+\boldsymbol{\mathcal{L}}_{4,4} & 0 & \boldsymbol{\mathcal{L}}_{1,2} & \boldsymbol{\mathcal{L}}_{1,3}\\
    -\boldsymbol{\mathcal{L}}_{3,1} & \boldsymbol{\mathcal{L}}_{2,1} & 0 & \boldsymbol{\mathcal{L}}_{2,2}+\boldsymbol{\mathcal{L}}_{3,3} & \boldsymbol{\mathcal{L}}_{3,4} & -\boldsymbol{\mathcal{L}}_{2,4}\\
    -\boldsymbol{\mathcal{L}}_{4,1} & 0 & \boldsymbol{\mathcal{L}}_{2,1} & \boldsymbol{\mathcal{L}}_{4,3} & \boldsymbol{\mathcal{L}}_{2,2}+\boldsymbol{\mathcal{L}}_{4,4} & \boldsymbol{\mathcal{L}}_{2,3}\\
    0 & -\boldsymbol{\mathcal{L}}_{4,1} & \boldsymbol{\mathcal{L}}_{3,1} & -\boldsymbol{\mathcal{L}}_{4,2} & \boldsymbol{\mathcal{L}}_{3,2} & \boldsymbol{\mathcal{L}}_{3,3}+\boldsymbol{\mathcal{L}}_{4,4}
    \end{bmatrix}
\end{align}
}
respectively.
\end{rmk}

\subsection{Non-degenerate Alignment in the General Setting}
\label{subsec:non_deg_gen_setting}
As argued in Section~\ref{sec:setup}, since $F(\mathbf{S}) = F(\mathbf{S}\mathbf{Q})$ for all $\mathbf{Q} \in \mathbb{O}(d)$, every alignment $\mathbf{S}$ is degenerate in this sense. With a slight abuse of notation, we define a non-degenerate alignment as follows.
\begin{dfn}
\label{def:non_deg_alignment0}
An alignment $\mathbf{S} \in \mathbb{O}(d)^m$ is non-degenerate if $\pi(\mathbf{S})$ is a non-degenerate local minimum of $\widetilde{F}$.
\end{dfn}
With the above definition, to characterize the non-degenerate alignments, it suffices to characterize the non-degenerate local minima of $\widetilde{F}$. We accomplish the same in the following theorem. Note that we have not made any assumption about the affine non-degeneracy of the points and the noise in the local views.

\begin{thm}{\textbf{(Condition for $\widetilde{\mathbf{S}}$ to be a non-degenerate local minimum of $\widetilde{F}$)}}. Let $\widetilde{\mathbf{S}} \in \widetilde{\mathcal{C}}$ and $\mathbf{S} \in \pi^{-1}(\widetilde{\mathbf{S}})$. Then the following are equivalent,
\begin{enumerate}
    \item $\widetilde{\mathbf{S}}$ is a non-degenerate local minimum of $\widetilde{F}$.
    \item $\widetilde{g}(\Hess \widetilde{F}(\widetilde{\mathbf{S}})[\widetilde{\mathbf{Z}}],\widetilde{\mathbf{Z}}) > 0$ for all $\widetilde{\mathbf{Z}} \in T_{\widetilde{\mathbf{S}}}\mathbb{O}(d)^m/\sim$ such that $\widetilde{\mathbf{Z}} \neq 0$.
    \item $\text{Tr}(\boldsymbol{\Omega}^T\mathbf{L}(\mathbf{S})\boldsymbol{\Omega}) < 0$ for all $\boldsymbol{\Omega} = [\boldsymbol{\Omega}_i]_1^m$ where $\boldsymbol{\Omega}_i \in \Skew(d)$, $\sum_1^m \boldsymbol{\Omega}_i = 0$ and not all $\boldsymbol{\Omega}_i$ equal zero.
    \item $\text{Tr}(\boldsymbol{\Omega}^T\mathbf{L}(\mathbf{S})\boldsymbol{\Omega}) < 0$ for all $\boldsymbol{\Omega} = [\boldsymbol{\Omega}_i]_1^m$ where $\boldsymbol{\Omega}_i \in \Skew(d)$ and not all $\boldsymbol{\Omega}_i$ are equal.
    \item $\boldsymbol{\omega}^T\mathbb{L}(\mathbf{S})\boldsymbol{\omega} < 0$ for all $\boldsymbol{\omega} = [\boldsymbol{\omega}_{r,s}]_{1 \leq r < s \leq d}$ where not all $\boldsymbol{\omega}_{r,s}$ are constant vectors.
    \item $\mathbb{L}(\mathbf{S})$ is negative semi-definite and of rank $(m-1)d(d-1)/2$.
\end{enumerate}
\label{thm:non_deg_loc_min}
\end{thm}
\begin{rmk}
Given the patch framework $\Theta$ and the alignment $\mathbf{S}$, one can compute the matrix $\mathbb{L}(\mathbf{S})$ in polynomial time in $m$, $n$ and $d$, and then verify the non-degeneracy of the alignment $\mathbf{S}$ by testing the last condition in the above theorem (which again requires polynomial time in $m$ and $d$).
\end{rmk}

Although, for $\widetilde{\mathbf{S}}$ to be a non-degenerate local minimum of $\widetilde{F}$, the above theorem demands any of the equivalent conditions 3-6 to hold for every $\mathbf{S} \in \pi^{-1}(\widetilde{\mathbf{S}})$, the following result shows that if a conditions hold for one $\mathbf{S} \in \pi^{-1}(\widetilde{\mathbf{S}})$ then it holds for all other elements as well i.e. for all $\mathbf{S}\mathbf{Q}$ where $\mathbf{Q} \in \mathbb{O}(d)$ is arbitrary.
\begin{prop}
\label{prop:one_all1}
Let $\mathbf{S} \in \pi^{-1}(\widetilde{\mathbf{S}})$ and $\mathbf{Q} \in \mathbb{O}(d)$. Fix $i \in [3,6]$. Suppose condition $i$ in Theorem~\ref{thm:non_deg_loc_min} holds for $\mathbf{S}$ then it holds for $\mathbf{S}\mathbf{Q}$ also. Consequently, an alignment $\mathbf{S}$ is non-degenerate if $\mathbf{S} \in \mathcal{C}$ (see Eq.~(\ref{eq:crit_pts2})) and it satisfies any of the (equivalent) conditions 3-6 in Theorem~\ref{thm:non_deg_loc_min}.
\end{prop}

A sufficient condition for $\mathbf{S}$ to be a non-degenerate alignment is as follows.
\begin{cor}
\label{cor:suff_non_deg_loc_min}
If $\mathbf{L}(\mathbf{S}) \preceq 0$ and is of rank $(m-1)d$, then $\mathbf{S}$ is a non-degenerate alignment. The same holds when $\mathbf{L}(\mathbf{S})$ is replaced by $\boldsymbol{\mathcal{L}}(\mathbf{S})$ as the two are unitarily equivalent.
\end{cor}

Note that the rank of $\mathbf{L}(\mathbf{S})$ being $(m-1)d$ is not a necessary condition for non-degeneracy as demonstrated in Figure~\ref{fig:suff_cond_views_non_deg}.

\begin{figure}[H]
    \centering
     \includegraphics[width=0.15\textwidth,keepaspectratio]{../fig/fig0/counterex_suff_loc_rigid.png}
    \caption{The dotted lines represent views and the filled points represent points on the overlaps. Here $d=2$. It will be clear from Proposition~\ref{prop:noiseless_setting1} in Section~\ref{sec:noiseless_non_deg_results} that $\mathbf{L}(\mathbf{S}) \preceq 0$, and thus $\mathbb{L}(\mathbf{S}) \preceq 0$. Through simple calculations one can deduce that the rank of $\mathbb{L}(\mathbf{S})$ is $3$ (which equals $(m-1)d(d-1)/2$) while the rank of $\mathbf{L}(\mathbf{S})$ is $ 3 < 6 = (m-1)d$.}
    \label{fig:suff_cond_views_non_deg}
\end{figure}

\iftodos
\textbf{TODO}:
Note that $\mathbf{C}$ is unitarily equivalent to $\mathbf{C}(\mathbf{S})$ and $\mathbf{C}(\mathbf{S})$ is related to $\mathbf{L}(\mathbf{S})$ via Eq.~(\ref{eq:L_of_S}). Also note that $C \succeq 0$. Question: if the rank of $\mathbf{C}$ (and thus of $\mathbf{C}(\mathbf{S})$) is $(m-1)d$ then every global minimum is non-degenerate? This is the case in the noiseless setting, but what about in noisy case. If this is not the case, then demonstrate using a counter example.
\fi

\begin{comment}
A similar set of results for non-degenerate critical points is as follows (proofs are analogous to those of the above results).
\begin{thm}{\textbf{(Condition for $\widetilde{\mathbf{S}}$ to be a non-degenerate critical point of $\widetilde{F}$)}}. Let $\widetilde{\mathbf{S}} \in \mathbb{O}(d)^m/\sim$ and $\mathbf{S} \in \pi^{-1}(\widetilde{\mathbf{S}})$. Then the following are equivalent,
\begin{enumerate}
    \item $\widetilde{\mathbf{S}}$ is a non-degenerate critical point of $\widetilde{F}$.
    \item $\widetilde{\mathbf{S}} \in \widetilde{\mathcal{C}}$, $\mathbf{S} \in \pi^{-1}(\widetilde{\mathbf{S}})$, $Z = [\mathbf{S}_i\overline{\boldsymbol{\Omega}}_i]_1^m$ and $g(\Hess F(\mathbf{S})[\mathbf{Z}],\mathbf{Z}) = 0$ if and only if $\boldsymbol{\Omega} = 0$.
    \item $\boldsymbol{\Omega} = [\boldsymbol{\Omega}_i]_1^m$, $\boldsymbol{\Omega}_i \in \Skew(d)$, $\sum_1^m \boldsymbol{\Omega}_i = 0$ $\text{Tr}(\boldsymbol{\Omega}^T\mathbf{L}(\mathbf{S})\boldsymbol{\Omega}) = 0$ if and only if $\boldsymbol{\Omega} = 0$.
    \item $\boldsymbol{\Omega} = [\boldsymbol{\Omega}_i]_1^m$, $\boldsymbol{\Omega}_i \in \Skew(d)$, $\text{Tr}(\boldsymbol{\Omega}^T\mathbf{L}(\mathbf{S})\boldsymbol{\Omega}) = 0$ if and only if $\boldsymbol{\Omega}_i = \boldsymbol{\Omega}_j$ for all $i,j \in [1,m]$.
    \item $\boldsymbol{\omega} = [\boldsymbol{\omega}_{r,s}]_{1 \leq r < s \leq d}$, $\mathbb{L}(\mathbf{S})\boldsymbol{\omega} = 0$ if and only if for every $1 \leq r < s \leq d$, $\boldsymbol{\omega}_{r,s}$ is a constant vector.
    \item $\mathbb{L}(\mathbf{S})$ is of rank $(m-1)d(d-1)/2$.
\end{enumerate}
\label{thm:non_deg_crit_pt}
\end{thm}

\begin{prop}
\label{prop:one_all2}
Let $\mathbf{S} \in \pi^{-1}(\widetilde{\mathbf{S}})$ and $\mathbf{Q} \in \mathbb{O}(d)$. Fix $i \in [3,6]$. Suppose that condition $i$ in Theorem~\ref{thm:non_deg_crit_pt} holds for $\mathbf{S}$ then it holds for $\mathbf{S}\mathbf{Q}$ also.
\end{prop}

\begin{cor}
\label{cor:suff_non_deg_crit_pt}
If $\mathbf{L}(\mathbf{S})$ is of rank $(m-1)d$ then $\widetilde{\mathbf{S}}$ is a non-degenerate critical point of $\widetilde{F}$.
\end{cor}
\end{comment}

We end this subsection by deriving a necessary and sufficient condition for an alignment of two views to be non-degenerate. First we need the following definitions (note that the objects in these definitions are related but not identical to $\mathbf{B}_i$ (see Eq.~(\ref{eq:B}), Remark~\ref{rmk:L0DB}) and $\mathbf{B}(\mathbf{S})_i$ (see Eq.~(\ref{eq:BofS}), Remark~\ref{rmk:C_S_structure})),
\begin{dfn}
\label{def:Bij}
Let $i,j \in [1,m]$ be two views. Define $\mathbf{B}_{i,j}$ to be a matrix whose columns are $\mathbf{x}_{k,i}$ (in the increasing order of $k$) where $(k,i),(k,j) \in E(\Gamma)$. Generally, $\mathbf{B}_{i,j} \neq \mathbf{B}_{j,i}$. Also, define
\begin{align}
    \overline{\mathbf{B}}_{i,j} = \mathbf{B}_{i,j}\left(\mathbf{I}_{n'} - \frac{\mathbf{1}_{n'}\mathbf{1}_{n'}^T}{n'}\right)
\end{align}
where $n' = |\{k: (k,i),(k,j) \in E(\Gamma)\}|$ is the number of points on the overlap of the $i$th view and the $j$th view, or equivalently the number of columns in $\mathbf{B}_{i,j}$.
\end{dfn}

\begin{dfn}
\label{def:BSicapj}
Let $\mathbf{S}$ be an alignment. Let $i,j \in [1,m]$. Define $\mathbf{B}(\mathbf{S})_{i,j}$ to be a matrix whose columns are $\mathbf{S}_i^T\mathbf{x}_{k,i}+\mathbf{t}_i$ (in increasing order of $k$) where $(k,i),(k,j) \in E(\Gamma)$ and where $\mathbf{t}_i$ is obtained using Eq.~(\ref{eq:opt_Z}). Also, define
\begin{align}
    \overline{\mathbf{B}(\mathbf{S})}_{i,j} = \mathbf{B}(\mathbf{S})_{i,j}\left(\mathbf{I}_{n'} - \frac{\mathbf{1}_{n'}\mathbf{1}_{n'}^T}{n'}\right).
\end{align}
\end{dfn}

\begin{rmk}
\label{rmk:BS_ijB_ij}
Let $i,j \in [1,m]$ and $\mathbf{S}$ be an alignment. Let  $n' = |\{k:(k,i),(k,j) \in E(\Gamma)\}|$ be the number of points on the overlap of the two views. Then
\begin{align}
\mathbf{B}(\mathbf{S})_{i,j} = \mathbf{S}_i^T \mathbf{B}_{i,j} + \mathbf{t}_i\mathbf{1}_{n'}^T
\end{align}
where $\mathbf{t}_i$ is obtained using Eq.~(\ref{eq:opt_Z}). Thus, we have
\begin{align}
    \rank (\overline{\mathbf{B}}_{i,j}) = \rank (\overline{\mathbf{B}(\mathbf{S})}_{i,j}).
\end{align}
Moreover,
\begin{align}
    \mathbf{B}(\mathbf{S})_{i,j}\left(\mathbf{I}_{n'} - \frac{\mathbf{1}_{n'}\mathbf{1}_{n'}^T}{n'}\right)\mathbf{B}(\mathbf{S})_{j,i}^T = \mathbf{S}_i^T\mathbf{B}_{i,j}\left(\mathbf{I}_{n'} - \frac{\mathbf{1}_{n'}\mathbf{1}_{n'}^T}{n'}\right)\mathbf{B}_{j,i}^T\mathbf{S}_j = \mathbf{S}_1^T\overline{\mathbf{B}}_{i,j}\overline{\mathbf{B}}_{j,i}^T\mathbf{S}_2
\end{align}
and in particular $\rank (\overline{\mathbf{B}(\mathbf{S})}_{i,j}\overline{\mathbf{B}(\mathbf{S})}_{j,i}^T) = \rank (\overline{\mathbf{B}}_{i,j}\overline{\mathbf{B}}_{j,i}^T)$.
\end{rmk}

\begin{thm}
\label{thm:non_deg_two_views_gen_setting}
Consider $m=2$ and let $\mathbf{S} \in \mathbb{O}(d)^2$. Then $\mathbf{S}$ is a non-degenerate alignment if and only if all of the following hold: (see Figures~\ref{fig:nec_suff_cond_loc_rigid_two_views_1} and \ref{fig:nec_suff_cond_loc_rigid_two_views} for intuition when $d=2$)
\begin{enumerate}
    \itemsep0em 
    \item $\overline{\mathbf{B}(\mathbf{S})}_{1,2}\overline{\mathbf{B}(\mathbf{S})}_{2,1}^T$ is symmetric.
    \item $\mathrm{Tr}(\boldsymbol{\Omega}^T \overline{\mathbf{B}(\mathbf{S})}_{1,2}\overline{\mathbf{B}(\mathbf{S})}_{2,1}^T\boldsymbol{\Omega}) \geq 0$ for all $\boldsymbol{\Omega} \in \Skew(d)$.
    \item $\rank\left(\overline{\mathbf{B}(\mathbf{S})}_{1,2}\overline{\mathbf{B}(\mathbf{S})}_{2,1}^T\right) \geq d-1$ (equivalently $\rank (\overline{\mathbf{B}}_{1,2}\overline{\mathbf{B}}_{2,1}^T) \geq d-1$).
\end{enumerate}
\end{thm}

\subsection{Unique Optimal Alignment in the General Setting}
\label{subsec:uniq_gen_setting}
Since $F(\mathbf{S}) = F(\mathbf{S}\mathbf{Q})$ for all $\mathbf{Q} \in \mathbb{O}(d)$, if $\mathbf{S}$ is an optimal alignment i.e. a global minimum then so is $\mathbf{S}\mathbf{Q}$. In this sense, no optimal alignment is unique. With a slight abuse of convention we define a unique optimal alignment as follows.
\begin{dfn}
\label{def:uniq_alignment}
An alignment $\mathbf{S} \in \mathbb{O}(d)^m$ is a unique optimal alignment if $\pi(\mathbf{S})$ is the unique global minimum of $\widetilde{F}$ or equivalently, $\mathbf{S}$ is an optimal alignment that is unique up to the action of $\mathbb{O}(d)$: for each optimal alignment $\mathbf{S}'$, there exist $\mathbf{Q} \in \mathbb{O}(d)$ such that $\mathbf{S}' = \mathbf{S}\mathbf{Q}$.
\end{dfn}

Now we provide a necessary and sufficient condition for an optimal alignment of two views to be unique. Although the proof can be found in \citea{schonemann1966generalized}, for completeness, we provide a proof using the constructs derived so far.

\begin{thm}
\label{thm:uniq_two_views_gen_setting}
Consider $m=2$ and let $\mathbf{S}$ be an optimal alignment. Then $\mathbf{S}$ is unique if and only if $\rank (\overline{\mathbf{B}}_{1,2}\overline{\mathbf{B}}_{2,1}^T) = d$ (see Figures~\ref{fig:nec_suff_cond_loc_rigid_two_views} and \ref{fig:nec_suff_cond_glob_rigid_two_views} for intuition when $d=2$).
\end{thm}

\begin{figure}[H]
    \centering
    \begin{tabular}{ccc}
    \begin{subfigure}[b]{0.175\textwidth}
         \centering
         \includegraphics[width=0.9\textwidth,keepaspectratio]{../fig/fig0/nec_suff_loc_rigid_1.png}
         \caption{}
         \label{fig:nec_suff_cond_loc_rigid_two_views_1}
     \end{subfigure}
     &
     \begin{subfigure}[b]{0.175\textwidth}
         \centering
         \includegraphics[width=0.9\textwidth,keepaspectratio]{../fig/fig0/nec_suff_loc_rigid_2.png}
         \caption{}
         \label{fig:nec_suff_cond_loc_rigid_two_views}
     \end{subfigure}
     &
     \begin{subfigure}[b]{0.175\textwidth}
         \centering
         \includegraphics[width=0.9\textwidth,keepaspectratio]{../fig/fig0/nec_suff_glob_rigid.png}
         \caption{}
         \label{fig:nec_suff_cond_glob_rigid_two_views}
     \end{subfigure}
     \end{tabular}
    \caption{The dotted lines represent views and the filled points represent points on the overlaps. All the pair of views are perfectly aligned. In (\ref{fig:nec_suff_cond_loc_rigid_two_views_1}), $\rank (\overline{\mathbf{B}}_{1,2}\overline{\mathbf{B}}_{2,1}^T) = 0$ and clearly the two views can be rotated by a different amount while still being perfectly aligned. In (\ref{fig:nec_suff_cond_loc_rigid_two_views}), $\rank (\overline{\mathbf{B}}_{1,2}\overline{\mathbf{B}}_{2,1}^T) = 1$ and in order for the views to be perfectly aligned, every infinitesimal rotation of the two views must be identical. However the perfect alignment of the views is not unique because the second view can be flipped (a non-infinitesimal rotation) to obtain another perfect alignment of the views. In (\ref{fig:nec_suff_cond_glob_rigid_two_views}), $\rank (\overline{\mathbf{B}}_{1,2}\overline{\mathbf{B}}_{2,1}^T) = 2$ and the perfect alignment is unique.}
    \label{fig:geom_intuit}
\end{figure}