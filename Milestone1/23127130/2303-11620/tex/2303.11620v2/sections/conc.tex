In this work we derived a characterization of a non-degenerate alignment of possibly noisy local views (Theorem~\ref{thm:non_deg_loc_min}) that can be tested in polynomial time. Thereafter, we worked in the noiseless setting, and under a mild assumption on the structure of the local views (which is satisfied when the views are affinely non-degenerate), we established equivalence of non-degeneracy and uniqueness of a perfect alignment with the local and global rigidity of the resulting perfect realization (Theorems~\ref{thm:loc_rigid} and \ref{thm:glob_rigid}).

Then, by specializing the characterization of non-degenerate alignment to the noiseless setting, we derived certain necessary and sufficient conditions on the overlapping structure of the views for a locally rigid realization (equivalently a non-degenerate perfect alignment) (Theorems~\ref{thm:nec_cond_loc_rigid_of_views} and \ref{thm:G_star_1}). Similar conditions for a globally rigid realization (equivalently a unique perfect alignment) were also presented (Theorems~\ref{thm:nec_cond_glob_rigid_views} and \ref{thm:overline_G_star_1}).

Finally, we showed that RGD to solve the alignment problem in Eq.~(\ref{eq:GPOP}) converges linearly to a non-degenerate alignment when initialized close to it (Theorem~\ref{thm:rgd_conv}). We ended with several corollaries connecting the overlapping structure of the views and the local rigidity of the realization with the convergence speed of RGD.

The following questions remain unanswered and we aim to address them in future work.
\begin{enumerate}[leftmargin=*]
    \item A non-degenerate alignment can be characterized using both $\mathbf{L}(\mathbf{S})$ and $\mathbb{L}(\mathbf{S})$ (see Theorem~\ref{thm:non_deg_loc_min}). The physically-interpretable consequences of the algebraic structure of $\mathbf{L}(\mathbf{S})$ (see Remark~\ref{rmk:C_hat_L_structure}) are founded in the results that connect the overlapping structure of the views with the non-degeneracy of a perfect alignment (Section~\ref{subsec:non_deg_noiseless_setting}). Although, Remark~\ref{rmk:mathbb_L_structure} and \ref{rmk:mathbbL_examples} reveal the algebraic structure of $\mathbb{L}(\mathbf{S})$, the physical ramifications are still unclear.
    \item Analogous to those presented in Section~\ref{sec:noiseless_non_deg_results} but in the case when the views are noisy, necessary and sufficient conditions on the overlapping structure of $m > 2$ views for a non-degenerate alignment and for a unique optimal alignment are still unknown.
    \item The proof/counterexample of the converse of Theorem~\ref{thm:nec_cond_glob_rigid_views} is to be investigated.
    \item Requiring an alignment to be non-degenerate for the local linear convergence of RGD to it, is in a sense a strong ask. The authors of \citea{liu2019quadratic} presented a much stronger result where they showed the local linear convergence of RGD to any critical point (not necessarily non-degenerate) in a similar setting as ours. An extension of their result to our problem in Eq.~(\ref{eq:GPOP}) seems unclear. We aim to investigate the local linear convergence of RGD to arbitrary critical points using a geometric approach based on \citeb[Section 6.2]{usevich2020approximate}.
\end{enumerate}