There exist a plethora of problems \citea{fang2012using,cucuringu2012sensor,sharp2004multiview,williams2000simultaneous} which involve the task of aligning local views (also known as registration of point clouds) so as to obtain a global view of the data that respects the geometry of the local views. Although it is not uncommon for the correspondence between the points and the views to not be known apriori \citea{szeliski2007image}, nevertheless, we assume that the overlapping structure of the local views is available. A suitable example of our setup is rendered by the bottom-up manifold learning techniques, LTSA \citea{zhang2004principal} and LDLE \citea{kohli2021ldle}, which first construct low distortion local views of high dimensional data into lower dimension followed by the alignment of the views to obtain a low dimensional global embedding of the data. Here we focus on the alignment step.

The correspondence between the local views and the points is captured by a bipartite graph, which when combined with the local coordinates of the points due to the views containing them, form a \textit{patch framework} $\Theta$ \citea{chaudhury2015global,gortler2010affine}. The alignment of the views amounts to finding a suitable transformation for each view so that the local coordinates of a point in its transformed views are close to each other. While LTSA seeks affine transformations, LDLE pursues the rigid ones for alignment. %The latter is of interest to us.

In the case where the views undergo rigid transformation, the problem of aligning views can be posed as the minimization of a $2$-norm based alignment error, given by a quadratic $F$ over the product of orthogonal groups $\mathbb{O}(d)^m$ (see Section~\ref{sec:setup}).  With our definition of the alignment error, we identify an ``alignment" of the local views by an element of $\mathbb{O}(d)^m$. Note that rotating/reflecting each view by the same amount does not affect the alignment error. For a given alignment $\mathbf{S} \in \mathbb{O}(d)^m$, this translates to the alignment error being the same due to $\mathbf{S}\mathbf{Q}$ for all $\mathbf{Q} \in \mathbb{O}(d)$ i.e. $F(\mathbf{S}) = F(\mathbf{S}\mathbf{Q})$. In this sense, every alignment is degenerate and every optimal alignment is non-unique. With a slight abuse of convention, \textit{we define a non-degenerate alignment to be a local minimum of $F$ which is non-degenerate up to an orthogonal transformation}. \textit{Similarly, we say that an optimal alignment (a global minimum of $F$) is unique if every other optimal alignment can be obtained by an orthogonal transformation of it}. This brings us to our first contribution.
\begin{contrib}
In Section~\ref{sec:non_deg}, we derive a characterization of non-degenerate alignment of (possibly noisy) local views that can be tested in polynomial time. By specializing it to the case of two views, we obtain a necessary and sufficient condition for an alignment of two views to be non-degenerate.\footnote{A similar condition for the uniqueness of an optimal alignment of two views exists (see \citea{schonemann1966generalized}). Nevertheless, we provide a short proof based on the constructs defined in this work.}
\end{contrib}

Given an alignment $\mathbf{S} \in \mathbf{O}(d)^m$ of the local views, a consensus representation $\Theta(\mathbf{S})$ of the points can be obtained by averaging the local coordinates of the points due to the (rigidly transformed) views containing them. In the noiseless setting where the local views are clean measurements of the data (obtained by applying an unknown rigid transformation to a subset of data points), a perfect alignment of views is possible. Equivalently, when the views are noiseless, a value of zero for $F$ is attainable, and an $\mathbf{S}$ that achieves it is called a ``perfect alignment". Clearly, a perfect alignment is an optimal one, while the converse may not hold. To be consistent with previous works \citea{gortler2010affine,hendrickson1992conditions}, the consensus representation of the points $\Theta(\mathbf{S})$ due to a perfect alignment $\mathbf{S}$ of the views is called a realization of the framework.

An understanding of affine, global (Euclidean), or local (Euclidean) rigidity \citea{gortler2010affine} of a realization $\Theta(\mathbf{S})$ has importance in several areas such as molecular dynamics \citea{lee2008geometric,clementi2000topological,cucuringu2012eigenvector} or sensor network localization \citea{zhu2010universal,zhang2010rigid}. Under a mild assumption on the structure of the local views, \citea{chaudhury2015global,zha2009spectral,gortler2010affine} characterized the affine rigidity of a realization by the rank of a certain matrix derived from the framework $\Theta$. It was shown in \citea{saxe1979embeddability} that deriving a similar characterization of global rigidity is NP-Hard. Nevertheless a characterization of the local rigidity is useful from an algorithmic standpoint as we show in this work. Furthermore, necessary and sufficient conditions on the overlapping structure of the local views for affine rigidity were derived in \citea{zha2009spectral}. Similar results in the context of local and global rigidity form our second set of contributions.
\begin{contrib}
\ 
    \begin{enumerate}
        \item In Section~\ref{sec:noiseless_non_deg_results}, under a mild assumption on the structure of the local views, we show that the local and global rigidity of a realization $\Theta(\mathbf{S})$ are equivalent to the non-degeneracy and uniqueness of the perfect alignment $\mathbf{S}$, respectively. As a corollary, we obtain a characterization of the local rigidity of a realization.
        \item By specializing the characterization of a non-degenerate alignment to noiseless setting, we obtain necessary and sufficient conditions on the overlapping structure of the local views for a perfect alignment to be non-degenerate (equivalently, for a realization to be locally rigid). Similar conditions are also derived for the uniqueness of a perfect alignment.
    \end{enumerate}
\end{contrib}

Several algorithms exist to obtain an approximate solution of the alignment problem at hand. A few important ones are: semidefinite programming (SDP) \citea{chaudhury2015global,bandeira2014multireference}, spectral relaxation (SPEC) \citea{chaudhury2015global,bandeira2014multireference}, Procrustes analysis (PROC) \citea{kohli2021ldle,proc_algo_1}, generalized power method (GPM) \citea{ling2021generalized} and Riemannian gradient descent (RGD) \citea{krishnan2005global}. It is common to obtain stability and convergence guarantees of an algorithm under rigidity constraints over the framework. For example, \citea{chaudhury2015global} derived stability guarantees on the SPEC and SDP solutions under affine rigidity constraints. Also, \citea{ling2021generalized,ling2021near} derived stability guarantees and showed global linear convergence of GPM with SPEC initialization under the setting where each view is affinely non-degenerate and contains all the points. It is easy to deduce that this framework exhibits global rigidity when the views have no noise. Our last contribution is along the same lines.
\begin{contrib}
In Section~\ref{sec:convergence}, we show that RGD converges locally linearly to a non-degenerate alignment of views. A corollary of that is: when the local views are affinely non-degenerate and noiseless then RGD converges locally linearly to a non-degenerate perfect alignment and thus to a locally rigid realization.
\end{contrib}

The notation and proofs are provided in Appendix~\ref{sec:all_proofs}.