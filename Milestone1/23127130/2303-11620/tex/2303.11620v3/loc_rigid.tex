%\documentclass[review,hidelinks,onefignum,onetabnum]{siamart220329}
\documentclass{amsart}
\usepackage[left=3.25cm, right=3.25cm, top=3cm, bottom=3cm]{geometry}

\usepackage{lipsum}
\usepackage{amsfonts}
\usepackage{graphicx}
\usepackage{float}
\usepackage{algorithm}
\usepackage{algorithmic}
\usepackage{multirow,makecell}
\usepackage{bm,amsfonts,booktabs,amssymb}
%\allowdisplaybreaks
\usepackage{amsmath}
\makeatletter
\g@addto@macro\normalsize{%
  \setlength\abovedisplayskip{7pt}
  \setlength\belowdisplayskip{7pt}
  \setlength\abovedisplayshortskip{8pt}
  \setlength\belowdisplayshortskip{8pt}
}
\makeatother
\usepackage{mathtools}
\usepackage{enumitem}
% \usepackage{caption}
% \captionsetup{belowskip=0pt}
\usepackage{subcaption}
\usepackage[font={small,it}]{caption}
\usepackage{comment}
\usepackage{nicematrix}
\usepackage[hidelinks]{hyperref}
\usepackage{cleveref}
\usepackage{autonum}
\usepackage{longtable}
\usepackage{tikz}
\usetikzlibrary{positioning}
\usetikzlibrary{arrows}

\ifpdf
  \DeclareGraphicsExtensions{.eps,.pdf,.png,.jpg}
\else
  \DeclareGraphicsExtensions{.eps}
\fi

% Add a serial/Oxford comma by default.
\newcommand{\creflastconjunction}{, and~}

% Used for creating new theorem and remark environments
\newtheorem{claim}{Claim}


\newcommand{\retainlabel}[1]{\label{#1}\sbox0{\ref{#1}}}

\newcommand*{\vertbar}{\rule[-1ex]{0.5pt}{2.5ex}}
\newcommand*{\horzbar}{\rule[.5ex]{2.5ex}{0.5pt}}
\newcolumntype{P}[1]{>{\centering\arraybackslash}p{#1}}
\newcolumntype{M}[1]{>{\centering\arraybackslash}m{#1}}
\newcommand{\makeheadbox}{\relax}
%\newtheorem{example}{Example}
%\setlength\parindent{0pt}
\setcounter{MaxMatrixCols}{20}

\def\TODO#1{{\color{red} \textbf{#1}}}

\newcommand{\ac}[1]{\textcolor{green}{Alex: #1}}

\newcommand{\dk}[1]{\textcolor{blue}{Dhruv: #1}}

\newcommand{\acomment}[1]{\textcolor{orange}{#1}}
%\newcommand{\revadd}[1]{\textcolor{blue}{#1}}
\newcommand{\revadd}[1]{#1}
\newcommand{\revdel}[1]{\textcolor{red}{}}

\newcommand{\gm}[1]{\textcolor{purple}{Gal: #1}}

\newcommand{\edit}[1]{\textcolor{black}{#1}}

\newcommand{\editt}[1]{\textcolor{black}{#1}}

\newcommand{\edittt}[1]{\textcolor{black}{#1}}

%\DeclareSymbolFont{ebgletters}{OML}{EBGaramond-Maths}{m}{it}
%\DeclareMathSymbol{w}{\mathalpha}{ebgletters}{`k}


\newtheorem{assump}{Assumption}
\newtheorem{dfn}{Definition}[section]
\newtheorem{thm}{Theorem}[section]
\newtheorem{cor}{Corollary}[section]
\newtheorem{lem}{Lemma}[section]
\newtheorem{prop}{Proposition}[section]
\newtheorem{rmk}{Remark}[section]
\newtheorem{ex}{Example}[section]

% \newtheorem{assump}{Assumption}
% \newtheorem{dfn}{Definition}
% \newtheorem{thm}{Theorem}
% \newtheorem{cor}[thm]{Corollary}
% \newtheorem{lem}[thm]{Lemma}
% \newtheorem{prop}[thm]{Proposition}
% \newtheorem{rmk}{Remark}
% \newtheorem{ex}{Example}
%\newdefinition{note}{Note}
%\newproof{pf}{Proof}
%\newtheorem{contrib}{Contribution}

\newcommand{\proofoffirst}[1]{\noindent \underline{\textbf{Proof of #1.}}}
\newcommand{\proofof}[1]{\smallskip \noindent \underline{\textbf{Proof of #1}}.}

\newcommand{\argmin}{\mathop{\mathrm{argmin}}}
\newcommand{\argmax}{\mathop{\mathrm{argmax}}}


% Definitions of handy macros can go here

\newcommand{\dataset}{{\cal D}}
\newcommand{\fracpartial}[2]{\frac{\partial #1}{\partial  #2}}

\newcommand{\Tr}{{\mathrm{Tr}}}
\newcommand{\QR}{{\mathrm{QR}}}
\newcommand{\PF}{{\mathrm{PF}}}
\newcommand{\EXP}{{\mathrm{Exp}}}
\newcommand{\pf}{{\mathrm{pf}}}
\newcommand{\qf}{{\mathrm{qf}}}
\newcommand{\grad}{{\mathrm{grad}}}
%\newcommand{\diag}{{\mathrm{diag}}}
\newcommand{\vecz}{{\mathrm{vec}}}
\newcommand{\Hess}{{\mathrm{Hess}}}
\newcommand{\blockdiag}{{\mathrm{block\text{-}diag}}}
\newcommand{\Skew}{{\mathrm{Skew}}}
\newcommand{\Sym}{{\mathrm{Sym}}}
\newcommand{\rank}{{\mathrm{rank}}}

\newif\iftodos
%\todostrue % comment out to hide answers

% Heading arguments are {volume}{year}{pages}{date submitted}{date published}{paper id}{author-full-names}

%\setcitestyle{square}

% \renewcommand{\citep}[2][]{[\citenum{#2}, {#1}]}
%\renewcommand{\citep}[2][]{\cite[{#1}]{#2}}
\newcommand{\citea}[2][]{\cite{#2}}
\newcommand{\citeb}[2][]{\cite[#1]{#2}}

\makeatletter
\newcommand*{\addFileDependency}[1]{% argument=file name and extension
\typeout{(#1)}% latexmk will find this if $recorder=0
% however, in that case, it will ignore #1 if it is a .aux or 
% .pdf file etc and it exists! If it doesn't exist, it will appear 
% in the list of dependents regardless)
%
% Write the following if you want it to appear in \listfiles 
% --- although not really necessary and latexmk doesn't use this
%
\@addtofilelist{#1}
%
% latexmk will find this message if #1 doesn't exist (yet)
\IfFileExists{#1}{}{\typeout{No file #1.}}
}\makeatother

\newcommand*{\myexternaldocument}[1]{%
\externaldocument[nocite]{#1}%
\addFileDependency{#1.tex}%
\addFileDependency{#1.aux}%
}


% Sets running headers as well as PDF title and authors
% \title[Non-degenerate Rigid Alignment in a Patch Framework]{Non-degenerate Rigid Alignment in a Patch Framework}

% Title. If the supplement option is on, then "Supplementary Material"
% is automatically inserted before the title.
% \title{Non-degenerate Rigid Alignment in a Patch Framework\thanks{Submitted to the editors on Aug 11, 2023.}}

% Authors: full names plus addresses.
% \renewcommand{\thefootnote}{\fnsymbol{footnote}}
% \footnotetext[1]{Department of Mathematics, UC San Diego (dhkohli@ucsd.edu, acloninger@ucsd.edu)}
% \footnotetext[2]{Halicio\u{g}lu Data Science Institute, UC San Diego (gmishne@ucsd.edu).}

% \author[D. Kohli, G. Mishne, A. Cloninger]{Dhruv Kohli\thanks{Department of Mathematics, UC San Diego (dhkohli@ucsd.edu, acloninger@ucsd.edu)} \and  Gal Mishne \and Alexander Cloninger}

\usepackage{amsopn}
\DeclareMathOperator{\diag}{diag}

\setlength{\belowcaptionskip}{0pt}
\setlength{\abovecaptionskip}{0pt}
\setlength{\intextsep}{0pt}

%%% Local Variables: 
%%% mode:latex
%%% TeX-master: "ex_article"
%%% End: 

\everypar{\looseness=-1}

% Information that is shared between the article and the supplement
% (title and author information, macros, packages, etc.) goes into
% ex_shared.tex. If there is no supplement, this file can be included
% directly.

%% SIAM Shared Information Template
% This is information that is shared between the main document and any
% supplement. If no supplement is required, then this information can
% be included directly in the main document.


% Packages and macros go here
\usepackage{lipsum}
\usepackage{amsfonts}
\usepackage{graphicx}
\usepackage{epstopdf}
\usepackage{algorithmic}
\ifpdf
  \DeclareGraphicsExtensions{.eps,.pdf,.png,.jpg}
\else
  \DeclareGraphicsExtensions{.eps}
\fi

% Prevent itemized lists from running into the left margin inside theorems and proofs
\usepackage{enumitem}
\setlist[enumerate]{leftmargin=.5in}
\setlist[itemize]{leftmargin=.5in}

% Add a serial/Oxford comma by default.
\newcommand{\creflastconjunction}{, and~}

% Used for creating new theorem and remark environments
\newsiamremark{remark}{Remark}
\newsiamremark{hypothesis}{Hypothesis}
\crefname{hypothesis}{Hypothesis}{Hypotheses}
\newsiamthm{claim}{Claim}

% Sets running headers as well as PDF title and authors
\headers{Area formula for spherical polygons via prequantization}{Albert Chern and Sadashige Ishida}

% Title. If the supplement option is on, then "Supplementary Material"
% is automatically inserted before the title.
\title{Area formula for spherical polygons via prequantization 
% \thanks{Submitted to the editors April 12th, 2023. \rev{Revision submitted to the editors March 28th, 2024. 
% }
% % \funding{This project was funded in part by the European Research Council (ERC Consolidator Grant 101045083 \emph{CoDiNA})}
% }
}

% Authors: full names plus addresses.
\author{Albert Chern
\thanks{University of California San Diego
  (\email{alchern@ucsd.edu}).}
\and Sadashige Ishida\thanks{Institute of Science and Technology Austria 
  (\email{sadashige.ishida@ist.ac.at}).}
% \and Jane E. Smith\footnotemark[3]
}

\usepackage{amsopn}
\DeclareMathOperator{\diag}{diag}


%%% Local Variables: 
%%% mode:latex
%%% TeX-master: "ex_article"
%%% End: 


% Optional PDF information
\ifpdf
\hypersetup{
  pdftitle={Non-degenerate Rigid Alignment in a Patch Framework},
  pdfauthor={D. Kohli, G. Mishne and A. Cloninger}
}
\fi

% The next statement enables references to information in the
% supplement. See the xr-hyperref package for details.

%\myexternaldocument{ex_supplement}
%\externaldocument[supp:][nocite]{ex_supplement}

% FundRef data to be entered by SIAM
%<funding-group specific-use="FundRef">
%<award-group>
%<funding-source>
%<named-content content-type="funder-name"> 
%</named-content> 
%<named-content content-type="funder-identifier"> 
%</named-content>
%</funding-source>
%<award-id> </award-id>
%</award-group>
%</funding-group>

%%%%%%%%%%----------------------------
%%%%%%%%%%----------------------------

\begin{document}
% make footnote style to be symbols
\renewcommand{\thefootnote}{\fnsymbol{footnote}}
\footnotetext[1]{Department of Mathematics, UC San Diego (dhkohli@ucsd.edu, acloninger@ucsd.edu)}
\footnotetext[2]{Halicio\u{g}lu Data Science Institute, UC San Diego (gmishne@ucsd.edu).}

\title[Non-degenerate Rigid Alignment in a Patch Framework]{Non-degenerate Rigid Alignment in a Patch Framework}
\author[D. Kohli, G. Mishne, A. Cloninger]{Dhruv Kohli${}^{\ast}$, Gal Mishne${}^\dagger$, Alexander Cloninger${}^{\ast, \dagger}$}

% reset back to arabic
\renewcommand{\thefootnote}{\arabic{footnote}}

\begin{abstract}%
% Given a set of overlapping local views (patches) of a dataset, we consider the problem of finding a rigid alignment of the views that minimizes a $2$-norm based alignment error. In general, the views are noisy and a perfect alignment may not exist. In this work, we characterize the non-degeneracy of an alignment in the noisy setting based on the kernel and positivity of a certain matrix. This leads to a polynomial time algorithm for testing the non-degeneracy of a given alignment. Consequently, we focus on Riemannian gradient descent for minimization of the error and obtain a sufficient condition on an alignment for the algorithm to converge (locally) linearly to it.  In the case of noiseless views, a perfect alignment exists, resulting in a realization of the points that respects the geometry of the views. Under a mild condition on the views, we show that the non-degeneracy of a perfect alignment is \revdel{equivalent}\revadd{implies the} local rigidity of the resulting realization. By specializing the characterization of a non-degenerate alignment to the noiseless setting, we obtain necessary and sufficient conditions on the overlapping structure of the views for a \revadd{non-degenerate alignment}\revdel{locally rigid realization}. Similar results are also obtained in the context of a unique perfect alignment and global rigidity.
Given a set of overlapping local views (patches) of a dataset, we consider the problem of finding a rigid alignment of the views that minimizes a $2$-norm based alignment error. In general, the views are noisy and a perfect alignment may not exist. In this work, we characterize the non-degeneracy of an alignment in the noisy setting based on the kernel and positivity of a certain matrix. This leads to a polynomial time algorithm for testing the non-degeneracy of a given alignment. Subsequently, we focus on Riemannian gradient descent for minimizing the alignment error, providing a sufficient condition on an alignment for the algorithm to converge (locally) linearly to it. \revadd{Additionally, we provide an exact recovery and noise stability analysis of the algorithm}. In the case of noiseless views, a perfect alignment exists, resulting in a realization of the points that respects the geometry of the views. Under a mild condition on the views, we show that a non-degenerate perfect alignment \revadd{characterizes the infinitesimally rigidity of a realization, and thus the local rigidity of a generic realization}. By specializing the non-degeneracy conditions to the noiseless case, we derive necessary and sufficient conditions on the overlapping structure of the views for \revadd{a perfect alignment to be non-degenerate and equivalently, for the resulting realization to be infinitesimally rigid}. Similar results are also derived regarding the uniqueness of a perfect alignment and global rigidity.
% Given a set of overlapping local views (patches) of a dataset, we consider the problem of finding a rigid alignment of the views that minimizes a $2$-norm based alignment error. In general, the views are noisy and a perfect alignment may not exist. In this work, we characterize the non-degeneracy of an alignment in the noisy setting based on the kernel and positivity of a certain matrix. This leads to a polynomial time algorithm for testing the non-degeneracy of a given alignment. Subsequently, we focus on Riemannian gradient descent for minimizing the alignment error, providing a sufficient condition on an alignment for the algorithm to converge (locally) linearly to it. Additionally, we provide an exact recovery and noise stability analysis of our algorithm. In the case of noiseless views, a perfect alignment exists, resulting in a realization of the points that respects the geometry of the views. Under a mild condition on the views, we show that a non-degenerate perfect alignment characterizes the infinitesimally rigidity of a realization, and thus the local rigidity of a generic realization. By specializing the non-degeneracy conditions to the noiseless case, we derive necessary and sufficient conditions on the overlapping structure of the views for a perfect alignment to be non-degenerate and equivalently, for the resulting realization to be infinitesimally rigid. Similar results are also derived regarding the uniqueness of a perfect alignment and global rigidity.
\end{abstract}

% \begin{keywords}
\keywords{Nondegeneracy, rigid alignment, infinitesimal rigidity, local rigidity, affine rigidity, linear convergence, Riemannian gradient descent, quotient manifold, noise stability.}
% \end{keywords}

% \begin{MSCcodes}
\subjclass[2010]{52C25, 53B21, 53B20, 65K10, 65Y20, 40A05, 05C50.}
% \end{MSCcodes}
% Optional TODOs:
% \begin{enumerate}
%     \item Counter example for theorem 31.
%     \item More references on local and global rigidity.
%     \item More references on the papers dealing with rigid alignment of views.
%     \item Remember to on the todo switch to see technical todos.
% \end{enumerate}
\maketitle
\section{Introduction}
\label{sec:intro}
\section{Introduction}

The increasing complexity of source code poses a key challenge to the reliability of large-scale software systems. Software bugs in these systems can lead to safety issues~\cite{bug_safety} for users around the world as well as cause non-negligible financial losses~\cite{bug_loss}. As such, developers have to spend a large amount of time and effort on bug fixing. Consequently, \aprfull (\apr), designed to automatically generate patches to fix software bugs, has attracted wide attention from both academia and industry~\cite{long2016prophet, legoues2012genprog, long2015spr, lou2020can, tufano2018empstudy}. 


To achieve \apr, one popular approach is known as Generate-and-Validate (G\&V)~\cite{qi2015gv, ghanbari2019prapr, lou2020can, le2016hdrepair, legoues2012genprog, wen2018capgen, hua2018sketchfix, martinez2016astor, koyuncu2020fixminder, liu2019tbar, liu2019avatar}, which is typically based on the following pipeline: First, fault localization techniques~\cite{wong2016fl, abreu2007ochiai, zhang2013injecting, papadakis2015metallaxis, li2019deepfl, li2017transforming} are applied to determine the suspicious locations in programs where bugs are likely to exist. Then, the buggy locations are used by the \apr tools to generate a list of patches that replace buggy lines with correct lines. Afterward, each patch is validated against the original test suite to identify any \emph{plausible patches} (i.e., passing all tests in the test suite). Finally, to determine the \emph{correct patches}, developers examine the list of plausible patches to see if any of them can correctly fix the bug. 

Traditional \apr tools can mainly be categorized into heuristic-based~\cite{legoues2012genprog, le2016hdrepair, wen2018capgen}, constraint-based~\cite{mechtaev2016angelix, le2017s3, demacro2014nopol, long2015spr} and \template~\cite{ghanbari2019prapr, hua2018sketchfix, martinez2016astor, liu2019tbar, liu2019avatar}. Among these traditional tools, \template \apr tools~\cite{ghanbari2019prapr, liu2019tbar, benton2020effectiveness} have been able to achieve state-of-the-art results. \Template \apr tools typically leverage pre-defined templates (e.g., adding a nullness check) for bug fixing. However, since these fix templates are typically handcrafted, the number and types of bugs they are able to fix can be limited. 



To address the limitations of traditional \apr, researchers have proposed various \learning \apr tools~\cite{li2020dlfix, chen2018sequencer, jiang2021cure, lutellier2020coconut, zhu2021recoder, ye2022rewardrepair} based on the \nmtfull (\nmt) architecture~\cite{sutskever2014mt} where the input is the buggy code snippets and the goal is to translate the buggy code snippets into a fixed version. To accomplish this, \learning \apr tools require supervised training datasets with pairs of both buggy and fixed code snippets in order to learn how to perform this translation step. These training data are usually obtained by mining historical bug fixes using heuristics/keywords~\cite{dallmeier2007benchmark}, which can be imprecise for identifying bug-fixing commits; even the actual bug-fixing commits can include irrelevant code changes, leading to further pollution in the dataset~\cite{xia2022alpharepair}.
% 
Moreover, it can be hard for such \apr tools to generalize and fix bug types unseen during training. 



To better leverage recent advances in \plmfull{s} (\plm{s}), researchers~\cite{xia2022alpharepair, xia2023repairstudy, kolak2022patch, prenner2021codexws} have directly applied \plm{s} to generate patches without bug-fixing datasets. These \llm-based \apr tools work by either directly generating a complete code function~\cite{prenner2021codexws, xia2023repairstudy} or predict/infill the correct code snippet given its surrounding context~\cite{xia2022alpharepair, xia2023repairstudy}. By directly using \llm{s} that are pre-trained on billions of open-source code snippets, \llm-based \apr tools can achieve state-of-the-art performance on many repair datasets~\cite{xia2022alpharepair}. 


% 
%
%

Traditional \apr tools have long used the insight of the \emph{plastic surgery hypothesis}~\cite{barr2014plastic} where it states that the code ingredients to fix a bug already exist within the same project. Traditional \apr tools have manually designed pattern-~\cite{ghanbari2019prapr, saha2017elixir} or heuristic-based~\cite{jiang2018simfix, legoues2012genprog} approaches to finding and using such relevant code ingredients to generate fixes for bugs. However, the plastic surgery hypothesis has been largely ignored in \llm-based \apr. In fact, \llm provides a unique opportunity to fully automate the plastic surgery hypothesis idea via fine-tuning (learning project-specific information via model updates from the buggy project) and prompting (directly providing relevant code ingredients to the model), and make it directly applicable to different languages (since the \llm{s} are typically multi-lingual).%
Moreover, despite the intensive manual efforts involved, traditional \apr tools still cannot fully leverage project-specific information due to large search space for leveraging/composing existing code ingredients. In contrast, the project-specific information can effectively leveraged by \llm{s} due to their power in code understanding/vectorization, e.g., even partial/imprecise information may still guide \llm{s} in correct patch generation!
 To this end, we ask the question: \emph{How useful is the plastic surgery hypothesis in the era of \plm{s}}?








\mypara{Our Work.} To answer the question, we present \ourtech{\xspace} -- a \llm-based approach that automatically utilizes the plastic surgery hypothesis by systematically combining multiple fine-tuning and prompting strategies for \apr. \ourtech fine-tunes \plm{s} using two novel domain-specific training strategies: \textbf{\epfinetune} -- we fine-tune using the original buggy project by aggressively masking out a high percentage of tokens, which allows \plm to learn project-specific code tokens and programming styles; and \textbf{\rofinetune} -- which only masks out a single continuous code sequence per training sample, allowing the model to get used to the final \csapr task of predicting a single continuous code sequence. Furthermore, we directly leverage the ability for \plm{s} to understand natural language instructions and introduce a novel prompting strategy, \textbf{\idprompting}, which uses information retrieval and static analysis to obtain a list of relevant identifiers for the buggy lines. While such relevant identifiers are critical for fixing some difficult bugs, they may not be seen by the \llm during inference due to limited context window size. Through the use of prompting, we directly tell the model to use these extracted identifiers (relevant code ingredients) to generate the correct code. Finally, to perform repair, we combine all four model variants (including the base model, both fine-tuned models and the base model with prompting) for the final repair.





While our insight of leveraging the plastic surgery hypothesis for \llm-based \apr is generalizable across different types of \plm{s}, to implement \ourtech, we choose a recent \plm{\xspace}, \ctfive~\cite{wang2021codet5}, which is pre-trained on millions of open-source code snippets. \ctfive is an encoder-decoder model trained using \mspfull (\msp) objective where a percentage of tokens are masked out and each continuous masked token sequence is referred to as a masked span. Also, although we only extract relevant identifiers from the current buggy project (since this paper focuses on the plastic surgery hypothesis), our work can be easily extended to obtain other code information (such as relevant statements or functions) from other sources, such as  the massive pre-training corpora~\cite{husain2020codesearchnet} or historical bug-fixing datasets~\cite{jiang2019infer}, which can provide more coding knowledge for \llm{s}. Besides, although we mainly focus on using traditional string comparison algorithms for information retrieval in this paper, these techniques can be easily replaced by other frequency-based retrieval~\cite{robertson2009probabilistic} and neural search (or embedding-based search)~\cite{reimers2019sentence}.
  In summary, this paper makes the following contributions:


%


\begin{itemize}[noitemsep, leftmargin=*, topsep=0pt]
    \item \textbf{Dimension.} This paper is the first to revisit the important plastic surgery hypothesis in the era of \llm{s}. It opens up a new dimension for \llm-based \apr to incorporate previously neglected information from the buggy project itself to boost \apr performance. Furthermore, it demonstrates the promising future of retrieval-based prompting for modern \llm-based \apr.
    \item \textbf{Implementation.} We implement \ourtech based on the recent \ctfive model. We augment the model using two novel fine-tuning strategies: \epfinetune and \rofinetune, along with a novel prompting strategy based on information retrieval and static analysis: \idprompting. We combine the patches generated by all four models together and perform patch ranking to speed up \apr.% 
    \item \textbf{Evaluation Study.} We conduct an extensive evaluation against state-of-the-art \apr tools. On the widely studied \dfj 1.2 and 2.0 datasets~\cite{just2014dfj}, \ourtech is able to achieve the new state-of-the-art results of 89 and 44 correct bug fixes (15 and 8 more than best baseline) respectively.  Furthermore, we perform a broad ablation study to justify our design. \ourtech demonstrates for the first time that the plastic surgery hypothesis can substantially boost \llm-based \apr and advance state-of-the-art \apr, while being fully automated and general. Moreover, even partial/imprecise code ingredients may still effectively guide \llm{s} for \apr!
\end{itemize}



\section{A Quadratic Program in Orthogonal Groups for Aligning Views}
\label{sec:setup}
%% Macro setup
\definecolor{purple}{rgb}{1, 0, 1}

\newcommand{\ie}{\emph{i.e.,}\xspace}
\newcommand{\eg}{\emph{e.g.,}\xspace}
\newcommand{\abr}{\emph{abbr.}\xspace}
\newcommand{\ea}{\emph{et al.}\xspace}
\newcommand{\gensync}{\emph{GenSync}\xspace}
\newcommand{\colosseum}{\emph{Colosseum}\xspace}
\newcommand{\srep}{\emph{SREP}\xspace} % Set Reconciliation Enhances
\newcommand{\srepsim}{\emph{SREPSim}\xspace}
% Propagation
\newcommand{\esrep}{\emph{E-SREP}\xspace}
\newcommand{\epsrep}{\emph{EP-SREP}\xspace}
\newcommand{\mesrep}{\emph{ME-SREP}\xspace}
\newcommand{\mempoolsync}{\emph{MempoolSync}}

\newcommand{\fref}[1]{Fig.~\ref{#1}}
\newcommand{\tref}[1]{Table~\ref{#1}}
\newcommand{\aref}[1]{Algorithm~\ref{#1}}
\newcommand{\procref}[1]{Procedure~\ref{#1}}
\newcommand{\sref}[1]{Section~\ref{#1}}
\newcommand{\lineref}[1]{line~\ref{#1}}
\newcommand{\appref}[1]{Appendix~\ref{#1}}

% Change \eqref
\LetLtxMacro{\originaleqref}{\eqref}
\renewcommand{\eqref}{Eq.~\originaleqref}

% Theorems and corollaries
\newcounter{theoremcount}
\setcounter{theoremcount}{0}
\DeclareRobustCommand{\theorem}[1]{%
  \refstepcounter{theoremcount}%
  \noindent\textit{\textbf{Theorem \thetheoremcount\label{theorem:#1}: }}%
}
\DeclareRobustCommand{\theoremref}[1]{Theorem~\ref{theorem:#1}}

\DeclareRobustCommand{\proof}{\emph{Proof:}\xspace}
\DeclareRobustCommand{\qqed}{\hfill$\blacksquare$}

\newcounter{corollcount}
\setcounter{corollcount}{0}
\DeclareRobustCommand{\coroll}[1]{%
  \refstepcounter{corollcount}%
  \noindent\textit{\textbf{Corollary \thecorollcount\label{coroll:#1}: }}%
}
\DeclareRobustCommand{\corollref}[1]{Corollary~\ref{coroll:#1}}

\newcounter{lemmacount}
\setcounter{lemmacount}{0}
\DeclareRobustCommand{\lemma}[1]{%
  \refstepcounter{lemmacount}%
  \noindent\textit{\textbf{Lemma \thelemmacount\label{lemma:#1}: }}%
}
\DeclareRobustCommand{\lemmaref}[1]{Lemma~\ref{lemma:#1}}

\newcounter{definitioncount}
\setcounter{definitioncount}{0}
\DeclareRobustCommand{\definition}[1]{%
  \refstepcounter{definitioncount}%
  \noindent\textit{\textbf{Definition \thedefinitioncount\label{definition:#1}: }}%
}
\DeclareRobustCommand{\defref}[1]{Definition~\ref{definition:#1}}

%notes of different authors
\newif\ifnotes
\notestrue
\notesfalse

\newif\ifdiff
\difftrue
\difffalse

\newcommand{\anote}[1]{\ifnotes $\ll$\textsf{\textcolor{purple}{Ari: {#1}}}$\gg$ \fi}
\newcommand{\nnote}[1]{\ifnotes $\ll$\textsf{\textcolor{orange}{Novak: {#1}}}$\gg$ \fi}
\newcommand{\diff}[1]{\ifdiff\textcolor{orange}{#1}\else#1\fi}

%%% Local Variables:
%%% mode: latex
%%% TeX-master: "main"
%%% End:


%%% Redefining theorem-like environments
\newcounter{environments}

\newcounter{theoremCounter}
\newcounter{lemmaCounter}
\newcounter{definitionCounter}
\newcounter{propositionCounter}
\newcounter{corollaryCounter}
\newcounter{exampleCounter}
\newcounter{remarkCounter}
\newcounter{propertyCounter}
\newcounter{assumptionCounter}
\newcounter{proofCounter}

%\theorempreskip{1pt}
%\theorempostskip{1pt}

\let\proposition\relax
\let\theorem\relax
\let\lemma\relax
\let\definition\relax
\let\corollary\relax
\theoremseparator{.}
\theorembodyfont{\itshape}
\theoremsymbol{$\triangleleft$}
\newtheorem{theorem}[theoremCounter]{Theorem}
\newtheorem{lemma}[lemmaCounter]{Lemma}
\newtheorem{definition}[definitionCounter]{Definition}
\newtheorem{proposition}[propositionCounter]{Proposition}
\newtheorem{corollary}[corollaryCounter]{Corollary}

\let\remark\relax
\let\example\relax
\let\assumption\relax

\theorembodyfont{\normalfont}
\newtheorem{example}[exampleCounter]{Example}
\newtheorem{remark}[remarkCounter]{Remark}

\theoremheaderfont{\itshape}
\theoremsymbol{}
\renewtheorem{property}[remarkCounter]{Property}

\theoremheaderfont{\bfseries}
\theorembodyfont{\itshape}
\newtheorem{assumption}[assumptionCounter]{Assumption}
\theoremheaderfont{\itshape}


\theoremstyle{plain}
\theoremheaderfont{\itshape}
\theorembodyfont{\normalfont}
\let\proof\relax
\theoremseparator{.}
\theoremsymbol{\qedfull}
\newtheorem*{proof}{Proof}
\qedsymbol{\qedfull}

% Reset equation counters for each property!
\makeatletter
\@addtoreset{equation}{property}
\makeatother

% Tikz stuff
\newcommand{\seqarr}
{\begin{tikzpicture}
		\draw[-{Triangle[scale=.7]}] (0,0) --  (.3,0); 
\end{tikzpicture}}

\newcommand{\looparr}
{\begin{tikzpicture}[scale=0.7,baseline=-1.55ex]
		\draw[arrows = {-Stealth[inset=0pt, length=2pt, angle'=60]}] (0,0) arc (102:437:.2cm);
\end{tikzpicture}}

\definecolor{blue-violet}{rgb}{0.54, 0.17, 0.89}
\definecolor{cadmiumorange}{rgb}{0.93, 0.53, 0.18}
\definecolor{yellow-green}{rgb}{0.6, 0.8, 0.2}
\definecolor{green1}{rgb}{0.12, 0.3, 0.17}
\definecolor{byzantium}{rgb}{0.44, 0.16, 0.39}


\section{Non-degeneracy and Uniqueness in the General Setting}
\label{sec:non_deg}
The organization of the section is as follows. In Section~\ref{subsec:prelims} we derive an expression involving the Hessian of $\widetilde{F}$ (Eq.~(\ref{eq:eq3}), (\ref{eq:eq4})), the function induced by $F$ on a certain quotient space $\mathbb{O}(d)^m/\sim$ (along with all the necessary geometrical objects). We refer the reader to \citeb[Chapter 3 and 5]{absil2009optimization} for the definitions of differential of a mapping, metric, gradient, connection, Hessian etc. in the context of Riemannian manifolds. Then in Section~\ref{subsec:non_sing_pos_def_hess} we obtain the equations governing the non-singularity and positivity of the Hessian (Eq.~(\ref{eq:TrOmegaTLSOmega}), (\ref{eq:omega^TmbbLomega})). Subsequently, in Section~\ref{subsec:non_deg_gen_setting}, we obtain a characterization of the non-degeneracy of an alignment in the general setting (Theorem~\ref{thm:non_deg_loc_min}). We also present a necessary and sufficient condition on the overlapping structure of two views under which an alignment of the views is non-degenerate (Theorem~\ref{thm:non_deg_two_views_gen_setting}). Finally, in Section~\ref{subsec:uniq_gen_setting}, we derive a similar condition under which an optimal alignment of two views is unique (Theorem~\ref{thm:uniq_two_views_gen_setting}).

\subsection{Preliminaries}
\label{subsec:prelims}
Recall that the problem under consideration is the minimization of $F(\mathbf{S}) = \text{Tr}(\mathbf{C}\mathbf{S}\mathbf{S}^T)$ over $\mathbf{S} \in \mathbb{O}(d)^m$ where $\mathbf{C} \succeq 0$ is the patch-stress matrix defined in Eq.~(\ref{eq:GPOP}). Note that the objective is invariant to the action of $\mathbb{O}(d)$ i.e. for any $\mathbf{Q} \in \mathbb{O}(d)$, $F(\mathbf{S}) = F(\mathbf{S}\mathbf{Q})$.
\begin{assump}
If $\Gamma$ has $K$ connected components, then the objective is invariant to the action of $\mathbb{O}(d)^K$ on $\mathbb{O}(d)^m = \prod_1^K \mathbb{O}(d)^{m_j}$ where $\sum_1^K m_j = m$ and where each $\mathbb{O}(d)$ acts independently on $\mathbb{O}(d)^{m_j}$ for $j \in [1,K]$. To keep the computations clean, we assume that the bipartite graph $\Gamma$ is connected (as in Assumption~\ref{assump:connected_gamma}) throughout the rest of this work.
\end{assump}
Subsequently, we define an equivalence relation on $\mathbb{O}(d)^m$; $\mathbf{S}_1 \sim \mathbf{S}_2$ if and only if $\mathbf{S}_1 = \mathbf{S}_2\mathbf{Q}$ for some $\mathbf{Q} \in \mathbb{O}(d)$. Given $\mathbf{S} \in \mathbb{O}(d)^m$, its equivalence class is $[\mathbf{S}] = \left\{\mathbf{S}\mathbf{Q}: \mathbf{Q} \in \mathbb{O}(d)\right\}$. Clearly, there exists a bijection between $\mathbb{O}(d)^m/\sim$ and $\mathbb{O}(d)^{m-1}$, thus an element of $\mathbb{O}(d)^m/\sim$ will be identified with an element of $\mathbb{O}(d)^{m-1}$. Define the projection,
\begin{align}
    \pi: \mathbb{O}(d)^m &\mapsto \mathbb{O}(d)^m/\sim\\
    \pi(\mathbf{S}_{1:m}) &= \mathbf{S}_{2:m}\mathbf{S}_1^T. \label{eq:pi}
\end{align}
Let $\widetilde{\mathbf{S}} \in \mathbb{O}(d)^m/\sim$, then
\begin{align}
    \pi^{-1}(\widetilde{\mathbf{S}}) &= \left\{\begin{bmatrix}
    \mathbf{Q}\\\widetilde{\mathbf{S}}\mathbf{Q}
    \end{bmatrix}: \mathbf{Q} \in \mathbb{O}(d)\right\} = \{\mathbf{S} \in \mathbb{O}(d)^m: \mathbf{S}_{i+1}\mathbf{S}_1^T = \widetilde{\mathbf{S}}_{i}, i \in [1,m-1]\}. \label{eq:pi_inv_wtS}
\end{align}

The Riemannian metric $g$ on $\mathbb{O}(d)^m$ is the canonical one given by
\begin{align}
g(\mathbf{Z},\mathbf{W}) \coloneqq \text{Tr}(\mathbf{Z}^T\mathbf{W}) = \sum_1^m \text{Tr}(\mathbf{Z}_i^T\mathbf{W}_i) \text{ where } \mathbf{Z},\mathbf{W} \in T_{\mathbf{S}}\mathbb{O}(d)^m \subseteq \mathbb{R}^{md \times d}. \label{eq:g_Z_W}
\end{align}
By a simple extension of the $m=1$ case \citea{absil2009optimization}, it is easy to deduce the following result.
\begin{prop}
\label{prop:T_SOdm}
For $\mathbf{S} \in \pi^{-1}(\widetilde{\mathbf{S}})$, the tangent space to $\mathbb{O}(d)^m$ at $\mathbf{S}$ is
\begin{align}
    T_{\mathbf{S}}\mathbb{O}(d)^m &= \{[\mathbf{S}_i\boldsymbol{\Omega}_i]_1^m: \boldsymbol{\Omega}_i \in \Skew(d)\}. \label{eq:T_SOdm}
\end{align}
The orthogonal projection of $\boldsymbol{\xi} = [\boldsymbol{\xi}_i]_1^m$, where $\boldsymbol{\xi}_i \in \mathbb{R}^{d \times d}$, onto $T_{\mathbf{S}}\mathbb{O}(d)^m$ is
\begin{align}
    P_{\mathbf{S}}\left(\boldsymbol{\xi}\right) = \argmin_{\substack{[\mathbf{S}_i\boldsymbol{\Omega}_i]_1^m\\\boldsymbol{\Omega}_i \in \Skew(d)}} \sum_1^m\left\|\boldsymbol{\xi}_i - \mathbf{S}_i\boldsymbol{\Omega}_i\right\|_F^2 = [\mathbf{S}_i\Skew(\mathbf{S}_i^T\boldsymbol{\xi}_i)]_1^m. \label{eq:P_S_xi}
\end{align}
\end{prop}

Then, $\pi^{-1}(\widetilde{\mathbf{S}})$ admits a tangent space at $\mathbf{S} \in \pi^{-1}(\widetilde{\mathbf{S}})$ called the vertical space $\mathcal{V}_{\mathbf{S}}$ at $\mathbf{S}$. The horizontal space $\mathcal{H}_{\mathbf{S}}$ at $\mathbf{S}$ is the subspace of $T_{\mathbf{S}}\mathbb{O}(d)^m$ that is the orthogonal complement to the vertical space $\mathcal{V}_{\mathbf{S}}$.

\begin{prop}
\label{prop:V_S_H_S}
The vertical space $\mathcal{V}_{\mathbf{S}}$ at $\mathbf{S} \in \pi^{-1}(\widetilde{\mathbf{S}})$ is
\begin{align}
    \mathcal{V}_{\mathbf{S}} = \{\mathbf{S}\boldsymbol{\Omega}: \boldsymbol{\Omega} \in \Skew(d)\}. \label{eq:V_S}
\end{align}
The orthogonal projection of $\mathbf{Z} = [\mathbf{S}_i\boldsymbol{\Omega}_i]_1^m \in T_{\mathbf{S}}\mathbb{O}(d)^m$ onto $\mathcal{V}_{\mathbf{S}}$ is
\begin{align}
    P^{v}_{\mathbf{S}}([\mathbf{S}_i\boldsymbol{\Omega}_i]_1^m) = \left[\mathbf{S}_i\argmin_{\boldsymbol{\Omega}\in \Skew(d)} \sum_1^m\left\|\mathbf{S}_j(\boldsymbol{\Omega}_j-\boldsymbol{\Omega})\right\|_F^2\right]_1^m = \left[\mathbf{S}_i\left(\frac{\sum_1^m\boldsymbol{\Omega}_i}{m}\right)\right]_1^m. \label{eq:P^v_S}
\end{align}
The horizontal space $\mathcal{H}_{\mathbf{S}}$ at $\mathbf{S} \in \pi^{-1}(\widetilde{\mathbf{S}})$ is
\begin{align}
    \mathcal{H}_{\mathbf{S}} &= \left\{[\mathbf{S}_i\boldsymbol{\Omega}_i]_1^m, \boldsymbol{\Omega}_i \in \Skew(d), \sum_1^m \boldsymbol{\Omega}_i= 0\right\}. \label{eq:H_S}
\end{align}
The orthogonal projection of $\mathbf{Z} = [\mathbf{S}_i\boldsymbol{\Omega}_i]_1^m \in T_{\mathbf{S}}\mathbb{O}(d)^m$ to $\mathcal{H}_{\mathbf{S}}$ is
\begin{align}
    P^{h}_{\mathbf{S}}([\mathbf{S}_i\boldsymbol{\Omega}_i]_1^m) = [\mathbf{S}_i\boldsymbol{\Omega}_i]_1^m - P^{v}_{\mathbf{S}}([\mathbf{S}_i\boldsymbol{\Omega}_i]_1^m) = \left[\mathbf{S}_i\left(\boldsymbol{\Omega}_i-\frac{\sum_1^m\boldsymbol{\Omega}_i}{m}\right)\right]_1^m \label{eq:P^h_S}
\end{align}
\end{prop}

Note that $T_\mathbf{S}\mathbb{O}(d)^m$ is a vector space of dimension $md(d-1)/2$ and $\mathcal{V}_{\mathbf{S}}$ forms a $d(d-1)/2$ dimensional subspace of $T_{\mathbf{S}}\mathbb{O}(d)^m$. The dimension of $\mathcal{H}_{\mathbf{S}}$ and of $T_{\widetilde{\mathbf{S}}}\mathbb{O}(d)^m/\sim$ is $(m-1)d(d-1)/2$. In particular, $\mathcal{H}_{\mathbf{S}}$ can be identified with $T_{\widetilde{\mathbf{S}}}\mathbb{O}(d)^m/\sim$. Let $\widetilde{\mathbf{S}} \in \mathbb{O}(d)^{m}/\sim$ and $\widetilde{\mathbf{Z}} \in T_{\widetilde{\mathbf{S}}}\mathbb{O}(d)^{m}/\sim$. Then the horizontal lift of $\widetilde{\mathbf{Z}}$ at $\mathbf{S} \in \pi^{-1}(\widetilde{\mathbf{S}})$ is defined as $\overline{\widetilde{\mathbf{Z}}} \in \mathcal{H}_{\mathbf{S}}$ such that for each $i \in [1,m-1]$,
\begin{align}
    D\pi[\mathbf{S}]\left(\overline{\widetilde{\mathbf{Z}}}\right)_i = \widetilde{\mathbf{Z}}_i. \label{eq:hlift_def}
\end{align}

\begin{prop}
\label{prop:hlift_char}
Let $\widetilde{\mathbf{S}} \in \mathbb{O}(d)^{m}/\sim$ and $\widetilde{\mathbf{Z}} \in T_{\widetilde{\mathbf{S}}}\mathbb{O}(d)^{m}/\sim$. Let $\mathbf{Z}$ be the horizontal lift of $\widetilde{\mathbf{Z}}$ at $\mathbf{S} \in \pi^{-1}(\widetilde{\mathbf{S}})$. If $(\widetilde{\boldsymbol{\Omega}}_i)_1^{m-1} \subseteq \Skew(d)$ are such that $\widetilde{\mathbf{Z}}_i = \widetilde{\mathbf{S}}_i\widetilde{\boldsymbol{\Omega}}_i$, and $(\boldsymbol{\Omega}_i)_1^m \subseteq \Skew(d)$ are such that $\mathbf{Z}_i=\mathbf{S}_i\boldsymbol{\Omega}_i$ and $\sum_1^m \boldsymbol{\Omega}_i = 0$, then
\begin{align}
    \boldsymbol{\Omega}_1 &= -\frac{1}{m}\mathbf{S}_1^T\left(\sum_1^{m-1}\widetilde{\boldsymbol{\Omega}}_i\right)\mathbf{S}_1 \label{eq:hlift1}\\
    \boldsymbol{\Omega}_{i+1} &= \mathbf{S}_1^T\widetilde{\boldsymbol{\Omega}}_i\mathbf{S}_1 + \boldsymbol{\Omega}_1 \text{ for all } i \in [1,m-1]. \label{eq:hlifti}
\end{align}
Moreover, the linear system above has full rank, and thus the horizontal lift $\mathbf{Z}$ of $\widetilde{\mathbf{Z}}$ at $\mathbf{S} \in \mathbb{O}(d)^m$ is a unique element of $\mathcal{H}_{\mathbf{S}}$.
\end{prop}

\begin{prop}
\label{prop:g_tilde}
Let $\widetilde{\mathbf{Z}},\widetilde{\mathbf{W}} \in T_{\widetilde{\mathbf{S}}}\mathbb{O}(d)^{m}/\sim$ and $\mathbf{Z}, \mathbf{W} \in T_{\mathbf{S}}\mathbb{O}(d)^m$ be their the horizontal lifts at $\mathbf{S} \in \pi^{-1}(\widetilde{\mathbf{S}})$. Then
\begin{align}
    \widetilde{g}(\widetilde{\mathbf{Z}},\widetilde{\mathbf{W}}) \coloneqq g(\mathbf{Z}, \mathbf{W})
\end{align}
defines a Riemannian metric on $\mathbb{O}(d)^m/\sim$.
\end{prop}
We note that $\mathcal{H}_{\mathbf{S}}$ with the canonical metric $g$, is isometric to $T_\mathbf{S}\mathbb{O}(d)^{m}/\sim$ (equivalently $T_\mathbf{S}\mathbb{O}(d)^{m-1}$) when equipped with the above metric $\widetilde{g}$.

Now, coming back to the alignment error $F(\mathbf{S}) = \text{Tr}(\mathbf{C}\mathbf{S}\mathbf{S}^T)$ defined on $\mathbb{O}(d)^m$. It induces the following function on $\mathbb{O}(d)^{m}/\sim$ (again identified with $\mathbb{O}(d)^{m-1}$),
\begin{align}
    \widetilde{F}(\widetilde{\mathbf{S}}) = \text{Tr}\left(\mathbf{C}\begin{bmatrix}
    \mathbf{I}_d\\\widetilde{\mathbf{S}}
    \end{bmatrix}\begin{bmatrix}
    \mathbf{I}_d\\\widetilde{\mathbf{S}}
    \end{bmatrix}^T\right). \label{eq:Ftilde}
\end{align}
In particular, $F = \widetilde{F} \circ \pi$. It is easy to see that if $\mathbf{S}_1 \sim \mathbf{S}_2$ then $\widetilde{F} \circ \pi(\mathbf{S}_1) = \widetilde{F} \circ \pi(\mathbf{S}_2)$.

\begin{prop}
\label{prop:gradFS}
The horizontal lift of $\grad \widetilde{F}(\widetilde{\mathbf{S}})$ at $\mathbf{S} \in \pi^{-1}(\widetilde{\mathbf{S}})$ is
\begin{align}
    \overline{\grad \widetilde{F}(\widetilde{\mathbf{S}})} = \grad F(\mathbf{S}) = [\mathbf{S}_i\boldsymbol{\Omega}_i]_1^{m} \label{eq:gradFS}
\end{align}
where $\boldsymbol{\Omega}_i \coloneqq \mathbf{S}_i^T[\mathbf{C}\mathbf{S}]_i - [\mathbf{C}\mathbf{S}]_i^T\mathbf{S}_i \in \Skew(d)$ and $\sum_1^m \boldsymbol{\Omega}_i = \mathbf{S}^T\mathbf{C}\mathbf{S} - \mathbf{S}^T\mathbf{C}\mathbf{S} = 0$ (which validates that $\overline{\grad \widetilde{F}(\widetilde{\mathbf{S}})}$ is indeed in $\mathcal{H}_{\mathbf{S}}$, see Proposition~\ref{prop:V_S_H_S}).
\end{prop}

Due to the above proposition, the set of critical points of $\widetilde{F}$ is given by
\begin{align}
    \widetilde{\mathcal{C}} &= \{\widetilde{\mathbf{S}} \in \mathbb{O}(d)^m/\sim: \grad \widetilde{F}(\widetilde{\mathbf{S}}) = 0\}\\
    &= \{\widetilde{\mathbf{S}} \in \mathbb{O}(d)^m/\sim: \mathbf{S}_i^T[\mathbf{C}\mathbf{S}]_i = [\mathbf{C}\mathbf{S}]_i^T\mathbf{S}_i, \text{ for all } i \in [1,m], \mathbf{S} \in \pi^{-1}(\widetilde{\mathbf{S}})\}, \label{eq:crit_pts}
\end{align}
and that of $F$ is,
\begin{align}
     \mathcal{C} &= \{\mathbf{S} \in \mathbb{O}(d)^m: \grad F(\mathbf{S}) = 0\}\\
    &= \{\mathbf{S} \in \mathbb{O}(d)^m: \mathbf{S}_i^T[\mathbf{C}\mathbf{S}]_i = [\mathbf{C}\mathbf{S}]_i^T\mathbf{S}_i, \text{ for all } i \in [1,m]\}. \label{eq:crit_pts2}
\end{align}

The following remark follows trivially from Eq.~(\ref{eq:crit_pts}) and (\ref{eq:crit_pts2}).
\begin{rmk}
\label{rmk:crit_pt_F_Ftilde}
If $\mathbf{S}$ is a critical point of $F$ i.e. $\mathbf{S} \in \mathcal{C}$ then $\pi(\mathbf{S})$ is a critical point of $\widetilde{F}$ i.e. $\pi(\mathbf{S}) \in \widetilde{C}$. Similarly, if $\widetilde{\mathbf{S}} \in \widetilde{C}$ then $\mathbf{S} \in \mathcal{C}$ for all $\mathbf{S} \in \pi^{-1}(\widetilde{\mathbf{S}})$.
\end{rmk}


% Alternatively,
% \begin{align}
%     0 &= \grad F(\mathbf{S})_i\\
%     &= \mathbf{S}_{i+1}\overline{\grad F(\mathbf{S})}_1^T + \overline{\grad F(\mathbf{S})}_{i+1}\mathbf{S}_1^T\\
%     &= \mathbf{S}_{i+1}([\mathbf{C}\mathbf{S}]_1 - \mathbf{S}_1[\mathbf{C}\mathbf{S}]_1^T\mathbf{S}_1)^T + ([\mathbf{C}\mathbf{S}]_{i+1} - \mathbf{S}_{i+1}[\mathbf{C}\mathbf{S}]_{i+1}^T\mathbf{S}_{i+1})\mathbf{S}_1^T\\
%     &= \mathbf{S}_{i}C_{11} + \sum_{j=2}^{m}\mathbf{S}_{i}\mathbf{S}_{j-1}^TC_{j1} - \mathbf{S}_{i}(C_{11} + \sum_{j=2}^{m}C_{1j}\mathbf{S}_{j-1})+\\
%     &\qquad\qquad C_{i+1,1} + \sum_{j=2}^{m}C_{i+1,j}\mathbf{S}_{j-1} - (\mathbf{S}_iC_{1,i+1}\mathbf{S}_i + \sum_{j=2}^{m}\mathbf{S}_i\mathbf{S}_{j-1}^TC_{j,i+1}\mathbf{S}_i)\\
%     &= \mathbf{S}_i\left(\mathbf{S}_i^TC_{i+1,1} - C_{1,i+1}\mathbf{S}_i + \sum_{j=1}^{m-1}\mathbf{S}_i^TC_{i+1,j+1}\mathbf{S}_{j} - \mathbf{S}_{j}^{T}C_{j+1,i+1}\mathbf{S}_i + \mathbf{S}_{j}^{T}C_{j+1,1} - C_{1,j+1}\mathbf{S}_{j}\right)\\
%     &= \mathbf{S}_i\left(\sum_{j=0}^{m-1}\mathbf{S}_i^TC_{i+1,j+1}\mathbf{S}_{j} - \mathbf{S}_{j}^{T}C_{j+1,i+1}\mathbf{S}_i + \mathbf{S}_{j}^{T}C_{j+1,1} - C_{1,j+1}\mathbf{S}_{j}\right)\\
%     &= 2\mathbf{S}_i \left(\Skew\left(\mathbf{S}_i^T \left[C\begin{bmatrix}I_d\\S\end{bmatrix}\right]_{i+1}\right) - \Skew\left( \left[C\begin{bmatrix}I_d\\S\end{bmatrix}\right]_{1}\right)\right)
% \end{align}
% where $\mathbf{S}_0 = I_d$. Summing up for $i=1:m-1$ we obtain
% \begin{align}
%     \sum_{j=0}^{m-1} \mathbf{S}_j^TC_{j+1,1} - C_{1,j+1}\mathbf{S}_j = 0 \implies \left[C\begin{bmatrix}
%     I_d\\S
%     \end{bmatrix}\right]_1 = \left[C\begin{bmatrix}
%     I_d\\S
%     \end{bmatrix}\right]_1^T.
% \end{align}
% and for $i=1:m-1$,
% \begin{align}
%     \sum_{j=0}^{m-1}\mathbf{S}_i^TC_{i+1,j+1}\mathbf{S}_{j} - \mathbf{S}_{j}^{T}C_{j+1,i+1}\mathbf{S}_i = 0 \implies \mathbf{S}_i^T \left[C\begin{bmatrix}
%     I_d\\S
%     \end{bmatrix}\right]_i = \left[C\begin{bmatrix}
%     I_d\\S
%     \end{bmatrix}\right]_i^T\mathbf{S}_i
% \end{align}
% The above equations can be written in a more compact form as
% \begin{align}
%     \blockdiag\left(C\begin{bmatrix}
%     I_d\\S
%     \end{bmatrix}\begin{bmatrix}I_d & \mathbf{S}^T\end{bmatrix}\right) = \blockdiag\left(\begin{bmatrix}I_d\\S\end{bmatrix}\begin{bmatrix}
%     I_d & \mathbf{S}^T
%     \end{bmatrix}C\right)
% \end{align}

\begin{prop}
\label{prop:DgradFSZ}
Let $\widetilde{\mathbf{S}} \in \widetilde{\mathcal{C}}$ then for every $\mathbf{S} \in \pi^{-1}(\widetilde{\mathbf{S}})$ and $\mathbf{Z} \in T_\mathbf{S}\mathbb{O}(d)^m$,
\begin{align}
    D\grad F(\mathbf{S})[\mathbf{Z}] &= \left[\mathbf{S}_i(\mathbf{S}_i^T[\mathbf{C}\mathbf{Z}]_i - [\mathbf{C}\mathbf{Z}]_i^T\mathbf{S}_i - [\mathbf{C}\mathbf{S}]_i^T\mathbf{Z}_i + \mathbf{Z}_i^T[\mathbf{C}\mathbf{S}]_i)\right]_1^m.
\end{align}
\end{prop}

Let $\nabla$ be the Levi-Civita connection (also known as the Riemannian connection) on $\mathbb{O}(d)^m$ and $\widetilde{\nabla}$ be the induced connection on $\mathbb{O}(d)^m / \sim$. Then the following result follows from Proposition~\ref{prop:DgradFSZ} and the definition of the Riemannian Hessian operator \citeb[Section 5.5]{absil2009optimization}.
\begin{prop}
\label{prop:HessFSZ}
Let $\widetilde{\mathbf{S}} \in \widetilde{\mathcal{C}}$. Let $\widetilde{\mathbf{Z}} \in T_{\widetilde{\mathbf{S}}}\mathbb{O}(d)^{m}/\sim$. Let $\mathbf{Z}$ be the horizontal lift of $\widetilde{\mathbf{Z}}$ at $\mathbf{S} \in \pi^{-1}(\widetilde{\mathbf{S}})$. Then the horizontal lift of $\Hess \widetilde{F}(\widetilde{\mathbf{S}})[\widetilde{\mathbf{Z}}]$ at $\mathbf{S}$ is
\begin{align}
    \overline{\Hess \widetilde{F}(\widetilde{\mathbf{S}})[\widetilde{\mathbf{Z}}]} = [\mathbf{S}_i\widehat{\boldsymbol{\Omega}}_i]_1^m \label{eq:eq3}
\end{align}
where
\begin{align}
    \widehat{\boldsymbol{\Omega}}_i = \mathbf{S}_i^T[\mathbf{C}\mathbf{Z}]_i - [\mathbf{C}\mathbf{Z}]_i^T\mathbf{S}_i - [\mathbf{C}\mathbf{S}]_i^T\mathbf{Z}_i + \mathbf{Z}_i^T[\mathbf{C}\mathbf{S}]_i. \label{eq:Omega_hat_i}
\end{align}
\end{prop}

In the following, we obtain a compact representation for $\widehat{\boldsymbol{\Omega}}_i$. We first define certain matrices, then use them to obtain an expression for $\widehat{\boldsymbol{\Omega}}_i$ and then describe their structure. Recall that $\mathbf{C} = \mathbf{D} - \mathbf{B}\boldsymbol{\mathcal{L}}_{\Gamma}^\dagger \mathbf{B}^T$ (see Eq.~(\ref{eq:GPOP}) and Remark~\ref{rmk:L0DB}) and for convenience define
\begin{align}
    \mathbf{B}(\mathbf{S}) &\coloneqq \blockdiag((\mathbf{S}_i)_1^m)^T\ \mathbf{B} = [\mathbf{S}_i^T\mathbf{B}_i]_1^m \label{eq:BofS}\\
    \mathbf{D}(\mathbf{S}) &\coloneqq \blockdiag((\mathbf{S}_i)_1^m)^T\ \mathbf{D}\ \blockdiag((\mathbf{S}_i)_1^m) = \blockdiag((\mathbf{S}_i^T\mathbf{D}_{ii}\mathbf{S}_i)_1^m).
\end{align}
Then define
\begin{align}
    \mathbf{C}(\mathbf{S}) &\coloneqq \blockdiag((\mathbf{S}_i)_1^m)^T\ \mathbf{C}\ \blockdiag((\mathbf{S}_i)_1^m) = \mathbf{D}(\mathbf{S}) - \mathbf{B}(\mathbf{S})\boldsymbol{\mathcal{L}}_{\Gamma}^\dagger \mathbf{B}(\mathbf{S})^T\label{eq:C_of_S}\\
    \widehat{\mathbf{C}}(\mathbf{S}) &\coloneqq \blockdiag(([\mathbf{C}(\mathbf{S})\mathbf{I}_d^m]_i)_1^m) \label{eq:C_hat}\\
    \mathbf{L}(\mathbf{S}) &\coloneqq \widehat{\mathbf{C}}(\mathbf{S}) - \mathbf{C}(\mathbf{S}). \label{eq:L_of_S}
\end{align}
\begin{prop}
\label{prop:Omega_hat_compact}
Consider the same setup as in Proposition~\ref{prop:HessFSZ}. Then
\begin{align}
    \widehat{\boldsymbol{\Omega}}_i &= [\mathbf{L}(\mathbf{S})\boldsymbol{\Omega}]_i^T - [\mathbf{L}(\mathbf{S})\boldsymbol{\Omega}]_i \label{eq:eq4}
\end{align}
where $\boldsymbol{\Omega}=[\boldsymbol{\Omega}_i]_1^m$ and $\boldsymbol{\Omega}_i \in \Skew(d)$ is such that $\mathbf{Z}_i = \mathbf{S}_i\boldsymbol{\Omega}_i$ and $\sum_1^m\boldsymbol{\Omega}_i = 0$.
\end{prop}

Combining the above equation with Proposition~\ref{prop:HessFSZ}, we obtain a slightly compact representation of the horizontal lift of the Hessian. We end this subsection by giving remarks that reveal the structure of $\mathbf{C}(\mathbf{S})$, $\widehat{\mathbf{C}}(\mathbf{S})$ and $\mathbf{L}(\mathbf{S})$.

\begin{rmk}
\label{rmk:C_S_structure}
Since $\blockdiag((\mathbf{S}_i)_1^m)$ is an orthogonal matrix, $\mathbf{C}(\mathbf{S})$ is unitarily equivalent to $\mathbf{C}$. Thus, $\mathbf{C}(\mathbf{S}) \in \Sym(md)$, $\mathbf{C}(\mathbf{S}) \succeq 0$, $\rank(\mathbf{C}(\mathbf{S})) = \rank(\mathbf{C})$ and the $(i,j)$th block of size $d$ in $\mathbf{C}(\mathbf{S})$ is $\mathbf{C}(\mathbf{S})_{ij} = \delta_{ij}\mathbf{D}(\mathbf{S})_{ii} - \mathbf{B}(\mathbf{S})_i\boldsymbol{\mathcal{L}}_{\Gamma}^\dagger \mathbf{B}(\mathbf{S})_j^T$ where $\mathbf{B}(\mathbf{S})_i$ is the $i$th row block of $\mathbf{B}(\mathbf{S})$ of dimension $d \times (m+n)$.
\end{rmk}

\begin{rmk}
\label{rmk:StildeCtilde}
From the definition of $\mathcal{C}$ (see Eq.~(\ref{eq:crit_pts2})), $\mathbf{S} \in \mathcal{C}$ if and only if $\widehat{\mathbf{C}}(\mathbf{S}) \in \Sym(md)$ i.e. for each $i \in [1,m]$,
\begin{align}
    [\mathbf{C}(\mathbf{S})\mathbf{I}_d^m]_i &= \sum_{j=1}^{m}\mathbf{C}(\mathbf{S})_{ij} = \mathbf{S}_i^T[\mathbf{C}\mathbf{S}]_i = [\mathbf{C}\mathbf{S}]_i^T\mathbf{S}_i = \sum_{j=1}^{m}\mathbf{C}(\mathbf{S})_{ij}^T = [\mathbf{C}(\mathbf{S})\mathbf{I}_d^m]_i^T. \label{eq:eq2}
\end{align}
\end{rmk}

\begin{rmk}
\label{rmk:C_hat_L_structure}
If $\widetilde{\mathbf{S}} \in \widetilde{\mathcal{C}}$ then for every $\mathbf{S} \in \pi^{-1}(\mathbf{S})$, the following are easy to deduce.
\begin{enumerate}
    \item $\mathbf{L}(\mathbf{S}) \in \Sym(md)$.
    \item $\sum_{j=1}^{m}\mathbf{L}(\mathbf{S})_{ij} = \sum_{j=1}^{m}\mathbf{L}(\mathbf{S})_{ji} = 0$ for all $i \in [1,m]$ (see Eq.~(\ref{eq:eq2})).
    \item For each $i \in [1,d]$, the vector $\mathbf{1}_m \otimes \mathbf{e}^d_i$ lies in the kernel of $\mathbf{L}(\mathbf{S})$ and thus the rank of $\mathbf{L}(\mathbf{S})$ is atmost $(m-1)d$.
    \item If $\boldsymbol{\Omega} = [\boldsymbol{\Omega}_0]_1^m$ for some $\boldsymbol{\Omega}_0 \in \Skew(d)$ then $\mathbf{L}(\mathbf{S})\boldsymbol{\Omega} = 0$.
    \item The $(i,j)$th block of size $d$ in $\mathbf{L}(\mathbf{S})$ is
    \begin{align}
        \mathbf{L}(\mathbf{S})_{ij} &= -\delta_{ij}\mathbf{B}(\mathbf{S})_i\boldsymbol{\mathcal{L}}_{\Gamma}^\dagger \mathbf{B}(\mathbf{S})^T\mathbf{I}^m_d + \mathbf{B}(\mathbf{S})_i\boldsymbol{\mathcal{L}}_{\Gamma}^\dagger \mathbf{B}(\mathbf{S})_j^T\\
        &= -\delta_{ij}(\mathbf{I}^m_d)^T\mathbf{B}(\mathbf{S})\boldsymbol{\mathcal{L}}_{\Gamma}^\dagger \mathbf{B}(\mathbf{S})_i^T + \mathbf{B}(\mathbf{S})_i\boldsymbol{\mathcal{L}}_{\Gamma}^\dagger \mathbf{B}(\mathbf{S})_j^T \label{eq:L2}
    \end{align}
    and in particular, $\mathbf{L}(\mathbf{S})$ does not depend on $\mathbf{D}(\mathbf{S})$.
    \item Let $\mathbf{Q} \in \mathbb{O}(d)$ then $\mathbf{L}(\mathbf{S}\mathbf{Q}) = (\mathbf{I}_m \otimes \mathbf{Q})^T\mathbf{L}(\mathbf{S})(\mathbf{I}_m \otimes \mathbf{Q})$ (follows from Eq.~(\ref{eq:C_of_S}, \ref{eq:C_hat}, \ref{eq:eq2})) and since $\mathbf{I}_m \otimes \mathbf{Q}$ is an orthogonal matrix, $L(\mathbf{S}\mathbf{Q})$ is unitarily equivalent to $\mathbf{L}(\mathbf{S})$.
\end{enumerate}
\end{rmk}

\subsection{Non-singular and Positive Definite Hessian}
\label{subsec:non_sing_pos_def_hess}
A non-degenerate local minimum is defined to be the critical point at which the Hessian is positive definite. Similarly, a non-degenerate critical point is the one where the Hessian is non-singular. We proceed to derive the equations to identify the conditions under which the Hessian is non-singular and positive definite, which in turn characterize the non-degenerate critical points and local minima. First, we need the following result.
\begin{prop}
\label{prop:HessFSZZ}
Let $\widetilde{\mathbf{S}} \in \widetilde{\mathcal{C}}$. Let $\widetilde{\mathbf{Z}} \in T_{\widetilde{\mathbf{S}}}\mathbb{O}(d)^{m}/\sim$. Let $\mathbf{Z}$ be the horizontal lift of $\widetilde{\mathbf{Z}}$ at $\mathbf{S} \in \pi^{-1}(\widetilde{\mathbf{S}})$. Then
\begin{align}
    \widetilde{g}(\Hess \widetilde{F}(\widetilde{\mathbf{S}})[\widetilde{\mathbf{Z}}],\widetilde{\mathbf{Z}}) &= -2\text{Tr}(\boldsymbol{\Omega}^T\mathbf{L}(\mathbf{S})\boldsymbol{\Omega}) \label{eq:TrOmegaTLSOmega}
\end{align}
where $\boldsymbol{\Omega}=[\boldsymbol{\Omega}_i]_1^m$ and $\boldsymbol{\Omega}_i \in \Skew(d)$ is such that $\mathbf{Z}_i = \mathbf{S}_i\boldsymbol{\Omega}_i$ and $\sum_1^m\boldsymbol{\Omega}_i = 0$.
\end{prop}

Then the non-singularity and the positive definiteness of the Hessian amounts to the right side of Eq.~(\ref{eq:TrOmegaTLSOmega}) being non-zero and positive, respectively, for every non-zero $\boldsymbol{\Omega}$. Although $\mathbf{C}(\mathbf{S})$, and thus $\mathbf{L}(\mathbf{S})$, can be calculated from the patch framework $\Theta$ and the alignment $\mathbf{S}$, it is not obvious how to test the above practically. The main reason being that $\boldsymbol{\Omega}$ in Eq.~(\ref{eq:TrOmegaTLSOmega}) is not unconstrained, and in fact has a specific structure.

Therefore, for the above reason, we are going to manipulate Eq.~(\ref{eq:TrOmegaTLSOmega}), utilizing the structure of $\Omega$. The aim is to obtain an expression of the form $\boldsymbol{\omega}^T \mathbb{L}(\mathbf{S})\boldsymbol{\omega}$ where (i) the vector $\boldsymbol{\omega}$ is essentially unconstrained and (ii) $\boldsymbol{\Omega}$ and $\mathbf{L}(\mathbf{S})$ are related to $\boldsymbol{\omega}$ and $\mathbb{L}(\mathbf{S})$, respectively, through permutation matrices and vectorization operations (and thus the two pairs carry the same information). To achieve that, we first define certain matrices, then rewrite Eq.~(\ref{eq:TrOmegaTLSOmega}) in terms of those matrices and then describe their structure.

To this end, for $\boldsymbol{\Omega} = [\boldsymbol{\Omega}_i]_1^m$ where $\boldsymbol{\Omega}_i \in \Skew(d)$, let $\{\boldsymbol{\Omega}_{i}(r,s): 1 \leq r < s \leq d\}$ be the elements in the upper triangular region of $\boldsymbol{\Omega}_i$. For a fixed pair $(r,s)$ such that $1 \leq r < s \leq d$, define the column vector $\boldsymbol{\omega}_{r,s} \coloneqq [\boldsymbol{\Omega}_{i}(r,s)]_{i=1}^{m} \in \mathbb{R}^m$, a vertical stack of the $(r,s)$th element of each $\boldsymbol{\Omega}_i$. Then there exists a permutation matrix $\mathbf{P}$ such that
\begin{align}
    \mathbf{P}\boldsymbol{\Omega} &= \begin{bmatrix}
    \mathbf{0}_{m} & \boldsymbol{\omega}_{1,2} &  \ldots & \boldsymbol{\omega}_{1,d-1} & \boldsymbol{\omega}_{1,d}\\
    -\boldsymbol{\omega}_{1,2} & \mathbf{0}_{m}  & \ldots & \boldsymbol{\omega}_{2,d-1} & \boldsymbol{\omega}_{2,d}\\
        \vdots & \vdots & \vdots & \vdots & \vdots\\
    -\boldsymbol{\omega}_{1,d-1} & -\boldsymbol{\omega}_{2,d-1} & \ldots & \mathbf{0}_{m} & \boldsymbol{\omega}_{d-1,d}\\
    -\boldsymbol{\omega}_{1,d} & -\boldsymbol{\omega}_{2,d}  & \ldots & -\boldsymbol{\omega}_{d-1,d} & \mathbf{0}_{m}
    \end{bmatrix}. \label{eq:P0Omega}
\end{align}
In words, for $1 \leq r < s \leq d$, the $(r,s)$th block of $\mathbf{P}\boldsymbol{\Omega}$ is a vertical stack of the $(r,s)$th element of each $\boldsymbol{\Omega}_{i}$. For $r = s$, this is just a zero vector and for $r > s$, this is $-\boldsymbol{\omega}_{s,r}$.

Then, we collect the (strictly) upper triangular elements of $\mathbf{P}\boldsymbol{\Omega}$ in the column-major order in the vector $\boldsymbol{\omega}$. Note that $\mathbf{P}\boldsymbol{\Omega}$ can be fully described by $\boldsymbol{\omega}$. In particular, there exist a block matrix $\overline{\mathbf{P}}$ of size $d(d-1)/2 \times d^2$ siuch that $\vecz (\mathbf{P}\boldsymbol{\Omega}) = \overline{\mathbf{P}}^T \boldsymbol{\omega}$. The blocks of $\overline{\mathbf{P}}$ when indexed using tuples $(r,s)$ and $(p,q)$ where $1 \leq r < s \leq d$ and $p,q \in [1,d]$, are given by,
\begin{align}
    \overline{\mathbf{P}}_{(r,s),(p,q)} &= \left\{\begin{matrix}\mathbf{0}_{m \times m}, & p = q\\
    \delta_{pr}\delta_{qs}\mathbf{I}_m, & p < q\\
    -\delta_{ps}\delta_{qr}\mathbf{I}_m, & p > q.
    \end{matrix}\right.
\end{align}

Finally, we define
\begin{align}
    \mathcal{B}(\mathbf{S}) &\coloneqq \mathbf{P}\mathbf{B}(\mathbf{S}) \label{eq:mathcal_B}\\
    \boldsymbol{\mathcal{L}}(\mathbf{S}) &\coloneqq \mathbf{P}\mathbf{L}(\mathbf{S})\mathbf{P}^T \label{eq:mathcal_L}\\
    \mathbb{L}(\mathbf{S}) &\coloneqq \overline{\mathbf{P}}(\mathbf{I}_d \otimes \boldsymbol{\mathcal{L}}(\mathbf{S}))\overline{\mathbf{P}}^T \label{eq:mathbb_L}
\end{align}
and then note the following,
\begin{prop}
\label{prop:Omega^TLSOmega2}
Consider the same setup as in Proposition~\ref{prop:HessFSZZ}. Then
\begin{align}
    \text{Tr}(\boldsymbol{\Omega}^T\mathbf{L}(\mathbf{S})\boldsymbol{\Omega}) = \boldsymbol{\omega}^T\mathbb{L}(\mathbf{S})\boldsymbol{\omega} \label{eq:omega^TmbbLomega}
\end{align}
\end{prop}
The following remarks reveal the structure of $\mathcal{B}(\mathbf{S})$, $\boldsymbol{\mathcal{L}}(\mathbf{S})$ and $\mathbb{L}(\mathbf{S})$.
\begin{rmk}
\label{rmk:mathcalBS}
For $p \in [1,d]$, the $p$th row-block of $\mathcal{B}(\mathbf{S})$, $\mathcal{B}(\mathbf{S})_p$, is of size $m \times (n+m)$, and can be viewed as a vertical stack of the $p$th row of each $\mathbf{B}(\mathbf{S})_i$, $i \in [1,m]$. In particular, $\mathcal{B}(\mathbf{S})_p$ depends only on the $p$th coordinate of the local views (also see Remark~\ref{rmk:L0DB}).
\end{rmk}

\begin{rmk}
\label{rmk:mathcalLS}
% For $p,\mathbf{Q} \in [1,d]$ and $i,j \in [1,m]$, let $\boldsymbol{\mathcal{L}}(\mathbf{S})_{pq}$ be the $(p,q)$th block of size $m$ in $\boldsymbol{\mathcal{L}}(\mathbf{S})$ and $\mathbf{L}(\mathbf{S})_{ij}$ be the $(i,j)$th block of size $d$ in $\mathbf{L}(\mathbf{S})$. Then, from Eq.~(\ref{eq:mathcal_L}) it is easy to deduce
% \begin{align}
%     \boldsymbol{\mathcal{L}}(\mathbf{S})_{pq}(i,j) &= \mathbf{L}(\mathbf{S})_{ij}(p,q)\\
%     &= -\delta_{ij} \sum_{k=1}^m \mathbf{B}(\mathbf{S})_{i}(p,:)\boldsymbol{\mathcal{L}}_{\Gamma}^\dagger \mathbf{B}(\mathbf{S})_{k}(q,:)^T + \mathbf{B}(\mathbf{S})_i(p,:)\boldsymbol{\mathcal{L}}_{\Gamma}^\dagger \mathbf{B}(\mathbf{S})_j(q,:)^T\\
%     &= -\delta_{ij}\mathcal{B}(\mathbf{S})_{p}(i,:)\boldsymbol{\mathcal{L}}_{\Gamma}^\dagger \mathcal{B}(\mathbf{S})_{q}^T\mathbf{1}_m + \mathcal{B}(\mathbf{S})_{p}(i,:)\boldsymbol{\mathcal{L}}_{\Gamma}^\dagger \mathcal{B}(\mathbf{S})_{q}(j,:)^T
% \end{align}
% and thus
% \begin{align}
%     \boldsymbol{\mathcal{L}}(\mathbf{S})_{pq} &= -\diag (\mathcal{B}(\mathbf{S})_p\boldsymbol{\mathcal{L}}_{\Gamma}^\dagger \mathcal{B}(\mathbf{S})_{q}^T\mathbf{1}_m) + \mathcal{B}(\mathbf{S})_p\boldsymbol{\mathcal{L}}_{\Gamma}^\dagger \mathcal{B}(\mathbf{S})_q^T
% \end{align}
% Thus $\boldsymbol{\mathcal{L}}(\mathbf{S})_{pq}$ depends on the structure of $\Gamma$ through $\boldsymbol{\mathcal{L}}_{\Gamma}^\dagger$ and the $p$th and $q$th coordinates of the rigidly transformed local views $[\mathbf{S}_iB_i]_1^m$.
If $\widetilde{\mathbf{S}} \in \widetilde{\mathcal{C}}$ then for every $\mathbf{S} \in \pi^{-1}(\widetilde{\mathbf{S}})$, the following are easy to deduce.
\begin{enumerate}
    \item For $p,q \in [1,d]$, it is easy to deduce
    \begin{align}
        \boldsymbol{\mathcal{L}}(\mathbf{S})_{p,q} &= -\diag (\mathcal{B}(\mathbf{S})_p\boldsymbol{\mathcal{L}}_{\Gamma}^\dagger \mathcal{B}(\mathbf{S})_{q}^T\mathbf{1}_m) + \mathcal{B}(\mathbf{S})_p\boldsymbol{\mathcal{L}}_{\Gamma}^\dagger \mathcal{B}(\mathbf{S})_q^T.
    \end{align}
    Thus $\boldsymbol{\mathcal{L}}(\mathbf{S})_{p,q}$ depends on $\Gamma$ through $\boldsymbol{\mathcal{L}}_{\Gamma}^\dagger$ and the $p$th and $q$th coordinates of the rigidly transformed local views $\mathbf{B}(\mathbf{S})$.
    \item Since $\mathbf{L}(\mathbf{S})$ is symmetric, $\boldsymbol{\mathcal{L}}(\mathbf{S})_{q,p} = \boldsymbol{\mathcal{L}}(\mathbf{S})_{p,q}^T$.
    \item $\boldsymbol{\mathcal{L}}(\mathbf{S})_{p,q}$ is a Laplacian-like matrix and the constant vectors are in its kernel.
    \item For each $p \in [1,d]$, the vector $\mathbf{e}^d_p \otimes \mathbf{1}_m$ lies in the kernel of $\boldsymbol{\mathcal{L}}(\mathbf{S})$, thus the rank of $\boldsymbol{\mathcal{L}}(\mathbf{S})$ is atmost $(m-1)d$.
    \item If $\mathbf{Q} \in \mathbb{O}(d)$ then, from Remark~\ref{rmk:C_hat_L_structure} and Eq.~(\ref{eq:mathcal_L}), it follows that $\boldsymbol{\mathcal{L}}(\mathbf{S}\mathbf{Q})$ is unitarily equivalent to $\boldsymbol{\mathcal{L}}(\mathbf{S})$.
\end{enumerate}
\end{rmk}

\begin{rmk}
\label{rmk:mathbb_L_structure}
If $\widetilde{\mathbf{S}} \in \widetilde{\mathcal{C}}$ then for every $\mathbf{S} \in \pi^{-1}(\mathbf{S})$, the following are easy to deduce.
\begin{enumerate}
    \item $\mathbb{L}(\mathbf{S})$ is a block matrix of size $d(d-1)/2$ where each block is of size $m$. Indexing the rows and columns of $\mathbb{L}$ by tuples of the form $(r,s)$ where $1 \leq r < s \leq d$ we have,
    \begin{align}
        \mathbb{L}(\mathbf{S})_{(r_1,s_1),(r_2,s_2)} &= \sum_{\substack{p_1,q_1\in [1,d]\\p_2,q_2\in [1,d]}}\overline{\mathbf{P}}_{(r_1,s_1),(p_1,q_1)}(\mathbf{I}_d \otimes \boldsymbol{\mathcal{L}}(\mathbf{S}))_{(p_1,q_1),(p_2,q_2)}\overline{\mathbf{P}}_{(r_2,s_2),(p_2,q_2)}^T\\
        &= \sum_{p,q_1,q_2\in [1,d]}\overline{\mathbf{P}}_{(r_1,s_1),(p,q_1)}\boldsymbol{\mathcal{L}}(\mathbf{S})_{q_1,q_2}\overline{\mathbf{P}}_{(r_2,s_2),(p,q_2)}^T\\
        &= \left\{\begin{matrix}
        \boldsymbol{\mathcal{L}}(\mathbf{S})_{r,r}+\boldsymbol{\mathcal{L}}(\mathbf{S})_{s,s}, & r_1=r_2=r, s_1=s_2=s\\
        0, & \{r_1,s_1\} \cap \{r_2,s_2\} = \emptyset\\
        \boldsymbol{\mathcal{L}}(\mathbf{S})_{s_1,s_2}, & r_1 = r_2, s_1 \neq s_2\\
        \boldsymbol{\mathcal{L}}(\mathbf{S})_{r_1,r_2}, & s_1 = s_2, r_1 \neq r_2\\
        -\boldsymbol{\mathcal{L}}(\mathbf{S})_{r_1,s_2}, & s_1 = r_2\\
        -\boldsymbol{\mathcal{L}}(\mathbf{S})_{s_1,r_2}, & s_2 = r_1.\\
        \end{matrix}\right.
    \end{align}
    \item Since $\boldsymbol{\mathcal{L}}(\mathbf{S})$ is symmetric, $\mathbb{L}(\mathbf{S}) = \mathbb{L}(\mathbf{S})^T$.
    \item The set of vectors of the form $\boldsymbol{\omega} = [\boldsymbol{\omega}_{r,s}]_{1 \leq r < s \leq d}$ where each $\boldsymbol{\omega}_{r,s}$ is a constant vector, lie in the kernel of $\mathbb{L}(\mathbf{S})$.
    \item The rank of $\mathbb{L}(\mathbf{S})$ is at most $(m-1)d(d-1)/2$.
    \item If $\mathbf{Q} \in \mathbb{O}(d)$ then $\mathbb{L}(\mathbf{S}\mathbf{Q})$ is unitarily equivalent to $\mathbb{L}(\mathbf{S})$.
\end{enumerate}
\end{rmk}

% \begin{dfn}
% Let $\mathbf{S} \in \mathcal{C}$. Then $\mathbb{L}(\mathbf{S})$ is said to have trivial certificate if the null space of $\mathbb{L}$ contains only the vectors of the form $\boldsymbol{\omega} = [\boldsymbol{\omega}_{r,s}]_{1 \leq r < s \leq d}$ where each $\boldsymbol{\omega}_{r,s}$ is a constant vector i.e. if $\mathbb{L}(\mathbf{S})$ has a rank of $(m-1)d(d-1)/2$.
% \end{dfn}

\begin{rmk}
\label{rmk:mathbbL_examples}
For $d=2$, $3$ and $4$, $\mathbb{L}(\mathbf{S})$ is given by (for brevity, we use $\boldsymbol{\mathcal{L}}$ in place of $\boldsymbol{\mathcal{L}}(\mathbf{S})$)
\begin{align}
    [\boldsymbol{\mathcal{L}}_{1,1} + \boldsymbol{\mathcal{L}}_{2,2}],
\end{align}
\begin{align}
    &\begin{bmatrix}
    \boldsymbol{\mathcal{L}}_{1,1} + \boldsymbol{\mathcal{L}}_{2,2} & \boldsymbol{\mathcal{L}}_{2,3} & -\boldsymbol{\mathcal{L}}_{1,3}\\
    \boldsymbol{\mathcal{L}}_{3,2} & \boldsymbol{\mathcal{L}}_{1,1}+\boldsymbol{\mathcal{L}}_{3,3} & \boldsymbol{\mathcal{L}}_{1,2}\\
    -\boldsymbol{\mathcal{L}}_{3,1} & \boldsymbol{\mathcal{L}}_{2,1} & \boldsymbol{\mathcal{L}}_{2,2}+\boldsymbol{\mathcal{L}}_{3,3}
    \end{bmatrix}
\end{align}
and
{\small
\begin{align}
    \begin{bmatrix}
    \boldsymbol{\mathcal{L}}_{1,1}+\boldsymbol{\mathcal{L}}_{2,2} & \boldsymbol{\mathcal{L}}_{2,3} & \boldsymbol{\mathcal{L}}_{2,4} & -\boldsymbol{\mathcal{L}}_{1,3} & -\boldsymbol{\mathcal{L}}_{1,4} & 0\\
    \boldsymbol{\mathcal{L}}_{3,2} & \boldsymbol{\mathcal{L}}_{1,1}+\boldsymbol{\mathcal{L}}_{3,3} & \boldsymbol{\mathcal{L}}_{3,4} & \boldsymbol{\mathcal{L}}_{1,2} & 0 & -\boldsymbol{\mathcal{L}}_{1,4}\\
    \boldsymbol{\mathcal{L}}_{4,2} & \boldsymbol{\mathcal{L}}_{4,3} & \boldsymbol{\mathcal{L}}_{1,1}+\boldsymbol{\mathcal{L}}_{4,4} & 0 & \boldsymbol{\mathcal{L}}_{1,2} & \boldsymbol{\mathcal{L}}_{1,3}\\
    -\boldsymbol{\mathcal{L}}_{3,1} & \boldsymbol{\mathcal{L}}_{2,1} & 0 & \boldsymbol{\mathcal{L}}_{2,2}+\boldsymbol{\mathcal{L}}_{3,3} & \boldsymbol{\mathcal{L}}_{3,4} & -\boldsymbol{\mathcal{L}}_{2,4}\\
    -\boldsymbol{\mathcal{L}}_{4,1} & 0 & \boldsymbol{\mathcal{L}}_{2,1} & \boldsymbol{\mathcal{L}}_{4,3} & \boldsymbol{\mathcal{L}}_{2,2}+\boldsymbol{\mathcal{L}}_{4,4} & \boldsymbol{\mathcal{L}}_{2,3}\\
    0 & -\boldsymbol{\mathcal{L}}_{4,1} & \boldsymbol{\mathcal{L}}_{3,1} & -\boldsymbol{\mathcal{L}}_{4,2} & \boldsymbol{\mathcal{L}}_{3,2} & \boldsymbol{\mathcal{L}}_{3,3}+\boldsymbol{\mathcal{L}}_{4,4}
    \end{bmatrix}
\end{align}
}
respectively.
\end{rmk}

\subsection{Non-degenerate Alignment in the General Setting}
\label{subsec:non_deg_gen_setting}
As argued in Section~\ref{sec:setup}, since $F(\mathbf{S}) = F(\mathbf{S}\mathbf{Q})$ for all $\mathbf{Q} \in \mathbb{O}(d)$, every alignment $\mathbf{S}$ is degenerate in this sense. With a slight abuse of notation, we define a non-degenerate alignment as follows.
\begin{dfn}
\label{def:non_deg_alignment0}
An alignment $\mathbf{S} \in \mathbb{O}(d)^m$ is non-degenerate if $\pi(\mathbf{S})$ is a non-degenerate local minimum of $\widetilde{F}$.
\end{dfn}
With the above definition, to characterize the non-degenerate alignments, it suffices to characterize the non-degenerate local minima of $\widetilde{F}$. We accomplish the same in the following theorem. Note that we have not made any assumption about the affine non-degeneracy of the points and the noise in the local views.

\begin{thm}{\textbf{(Condition for $\widetilde{\mathbf{S}}$ to be a non-degenerate local minimum of $\widetilde{F}$)}}. Let $\widetilde{\mathbf{S}} \in \widetilde{\mathcal{C}}$ and $\mathbf{S} \in \pi^{-1}(\widetilde{\mathbf{S}})$. Then the following are equivalent,
\begin{enumerate}
    \item $\widetilde{\mathbf{S}}$ is a non-degenerate local minimum of $\widetilde{F}$.
    \item $\widetilde{g}(\Hess \widetilde{F}(\widetilde{\mathbf{S}})[\widetilde{\mathbf{Z}}],\widetilde{\mathbf{Z}}) > 0$ for all $\widetilde{\mathbf{Z}} \in T_{\widetilde{\mathbf{S}}}\mathbb{O}(d)^m/\sim$ such that $\widetilde{\mathbf{Z}} \neq 0$.
    \item $\text{Tr}(\boldsymbol{\Omega}^T\mathbf{L}(\mathbf{S})\boldsymbol{\Omega}) < 0$ for all $\boldsymbol{\Omega} = [\boldsymbol{\Omega}_i]_1^m$ where $\boldsymbol{\Omega}_i \in \Skew(d)$, $\sum_1^m \boldsymbol{\Omega}_i = 0$ and not all $\boldsymbol{\Omega}_i$ equal zero.
    \item $\text{Tr}(\boldsymbol{\Omega}^T\mathbf{L}(\mathbf{S})\boldsymbol{\Omega}) < 0$ for all $\boldsymbol{\Omega} = [\boldsymbol{\Omega}_i]_1^m$ where $\boldsymbol{\Omega}_i \in \Skew(d)$ and not all $\boldsymbol{\Omega}_i$ are equal.
    \item $\boldsymbol{\omega}^T\mathbb{L}(\mathbf{S})\boldsymbol{\omega} < 0$ for all $\boldsymbol{\omega} = [\boldsymbol{\omega}_{r,s}]_{1 \leq r < s \leq d}$ where not all $\boldsymbol{\omega}_{r,s}$ are constant vectors.
    \item $\mathbb{L}(\mathbf{S})$ is negative semi-definite and of rank $(m-1)d(d-1)/2$.
\end{enumerate}
\label{thm:non_deg_loc_min}
\end{thm}
\begin{rmk}
Given the patch framework $\Theta$ and the alignment $\mathbf{S}$, one can compute the matrix $\mathbb{L}(\mathbf{S})$ in polynomial time in $m$, $n$ and $d$, and then verify the non-degeneracy of the alignment $\mathbf{S}$ by testing the last condition in the above theorem (which again requires polynomial time in $m$ and $d$).
\end{rmk}

Although, for $\widetilde{\mathbf{S}}$ to be a non-degenerate local minimum of $\widetilde{F}$, the above theorem demands any of the equivalent conditions 3-6 to hold for every $\mathbf{S} \in \pi^{-1}(\widetilde{\mathbf{S}})$, the following result shows that if a conditions hold for one $\mathbf{S} \in \pi^{-1}(\widetilde{\mathbf{S}})$ then it holds for all other elements as well i.e. for all $\mathbf{S}\mathbf{Q}$ where $\mathbf{Q} \in \mathbb{O}(d)$ is arbitrary.
\begin{prop}
\label{prop:one_all1}
Let $\mathbf{S} \in \pi^{-1}(\widetilde{\mathbf{S}})$ and $\mathbf{Q} \in \mathbb{O}(d)$. Fix $i \in [3,6]$. Suppose condition $i$ in Theorem~\ref{thm:non_deg_loc_min} holds for $\mathbf{S}$ then it holds for $\mathbf{S}\mathbf{Q}$ also. Consequently, an alignment $\mathbf{S}$ is non-degenerate if $\mathbf{S} \in \mathcal{C}$ (see Eq.~(\ref{eq:crit_pts2})) and it satisfies any of the (equivalent) conditions 3-6 in Theorem~\ref{thm:non_deg_loc_min}.
\end{prop}

A sufficient condition for $\mathbf{S}$ to be a non-degenerate alignment is as follows.
\begin{cor}
\label{cor:suff_non_deg_loc_min}
If $\mathbf{L}(\mathbf{S}) \preceq 0$ and is of rank $(m-1)d$, then $\mathbf{S}$ is a non-degenerate alignment. The same holds when $\mathbf{L}(\mathbf{S})$ is replaced by $\boldsymbol{\mathcal{L}}(\mathbf{S})$ as the two are unitarily equivalent.
\end{cor}

Note that the rank of $\mathbf{L}(\mathbf{S})$ being $(m-1)d$ is not a necessary condition for non-degeneracy as demonstrated in Figure~\ref{fig:suff_cond_views_non_deg}.

\begin{figure}[H]
    \centering
     \includegraphics[width=0.15\textwidth,keepaspectratio]{../fig/fig0/counterex_suff_loc_rigid.png}
    \caption{The dotted lines represent views and the filled points represent points on the overlaps. Here $d=2$. It will be clear from Proposition~\ref{prop:noiseless_setting1} in Section~\ref{sec:noiseless_non_deg_results} that $\mathbf{L}(\mathbf{S}) \preceq 0$, and thus $\mathbb{L}(\mathbf{S}) \preceq 0$. Through simple calculations one can deduce that the rank of $\mathbb{L}(\mathbf{S})$ is $3$ (which equals $(m-1)d(d-1)/2$) while the rank of $\mathbf{L}(\mathbf{S})$ is $ 3 < 6 = (m-1)d$.}
    \label{fig:suff_cond_views_non_deg}
\end{figure}

\iftodos
\textbf{TODO}:
Note that $\mathbf{C}$ is unitarily equivalent to $\mathbf{C}(\mathbf{S})$ and $\mathbf{C}(\mathbf{S})$ is related to $\mathbf{L}(\mathbf{S})$ via Eq.~(\ref{eq:L_of_S}). Also note that $C \succeq 0$. Question: if the rank of $\mathbf{C}$ (and thus of $\mathbf{C}(\mathbf{S})$) is $(m-1)d$ then every global minimum is non-degenerate? This is the case in the noiseless setting, but what about in noisy case. If this is not the case, then demonstrate using a counter example.
\fi

\begin{comment}
A similar set of results for non-degenerate critical points is as follows (proofs are analogous to those of the above results).
\begin{thm}{\textbf{(Condition for $\widetilde{\mathbf{S}}$ to be a non-degenerate critical point of $\widetilde{F}$)}}. Let $\widetilde{\mathbf{S}} \in \mathbb{O}(d)^m/\sim$ and $\mathbf{S} \in \pi^{-1}(\widetilde{\mathbf{S}})$. Then the following are equivalent,
\begin{enumerate}
    \item $\widetilde{\mathbf{S}}$ is a non-degenerate critical point of $\widetilde{F}$.
    \item $\widetilde{\mathbf{S}} \in \widetilde{\mathcal{C}}$, $\mathbf{S} \in \pi^{-1}(\widetilde{\mathbf{S}})$, $Z = [\mathbf{S}_i\overline{\boldsymbol{\Omega}}_i]_1^m$ and $g(\Hess F(\mathbf{S})[\mathbf{Z}],\mathbf{Z}) = 0$ if and only if $\boldsymbol{\Omega} = 0$.
    \item $\boldsymbol{\Omega} = [\boldsymbol{\Omega}_i]_1^m$, $\boldsymbol{\Omega}_i \in \Skew(d)$, $\sum_1^m \boldsymbol{\Omega}_i = 0$ $\text{Tr}(\boldsymbol{\Omega}^T\mathbf{L}(\mathbf{S})\boldsymbol{\Omega}) = 0$ if and only if $\boldsymbol{\Omega} = 0$.
    \item $\boldsymbol{\Omega} = [\boldsymbol{\Omega}_i]_1^m$, $\boldsymbol{\Omega}_i \in \Skew(d)$, $\text{Tr}(\boldsymbol{\Omega}^T\mathbf{L}(\mathbf{S})\boldsymbol{\Omega}) = 0$ if and only if $\boldsymbol{\Omega}_i = \boldsymbol{\Omega}_j$ for all $i,j \in [1,m]$.
    \item $\boldsymbol{\omega} = [\boldsymbol{\omega}_{r,s}]_{1 \leq r < s \leq d}$, $\mathbb{L}(\mathbf{S})\boldsymbol{\omega} = 0$ if and only if for every $1 \leq r < s \leq d$, $\boldsymbol{\omega}_{r,s}$ is a constant vector.
    \item $\mathbb{L}(\mathbf{S})$ is of rank $(m-1)d(d-1)/2$.
\end{enumerate}
\label{thm:non_deg_crit_pt}
\end{thm}

\begin{prop}
\label{prop:one_all2}
Let $\mathbf{S} \in \pi^{-1}(\widetilde{\mathbf{S}})$ and $\mathbf{Q} \in \mathbb{O}(d)$. Fix $i \in [3,6]$. Suppose that condition $i$ in Theorem~\ref{thm:non_deg_crit_pt} holds for $\mathbf{S}$ then it holds for $\mathbf{S}\mathbf{Q}$ also.
\end{prop}

\begin{cor}
\label{cor:suff_non_deg_crit_pt}
If $\mathbf{L}(\mathbf{S})$ is of rank $(m-1)d$ then $\widetilde{\mathbf{S}}$ is a non-degenerate critical point of $\widetilde{F}$.
\end{cor}
\end{comment}

We end this subsection by deriving a necessary and sufficient condition for an alignment of two views to be non-degenerate. First we need the following definitions (note that the objects in these definitions are related but not identical to $\mathbf{B}_i$ (see Eq.~(\ref{eq:B}), Remark~\ref{rmk:L0DB}) and $\mathbf{B}(\mathbf{S})_i$ (see Eq.~(\ref{eq:BofS}), Remark~\ref{rmk:C_S_structure})),
\begin{dfn}
\label{def:Bij}
Let $i,j \in [1,m]$ be two views. Define $\mathbf{B}_{i,j}$ to be a matrix whose columns are $\mathbf{x}_{k,i}$ (in the increasing order of $k$) where $(k,i),(k,j) \in E(\Gamma)$. Generally, $\mathbf{B}_{i,j} \neq \mathbf{B}_{j,i}$. Also, define
\begin{align}
    \overline{\mathbf{B}}_{i,j} = \mathbf{B}_{i,j}\left(\mathbf{I}_{n'} - \frac{\mathbf{1}_{n'}\mathbf{1}_{n'}^T}{n'}\right)
\end{align}
where $n' = |\{k: (k,i),(k,j) \in E(\Gamma)\}|$ is the number of points on the overlap of the $i$th view and the $j$th view, or equivalently the number of columns in $\mathbf{B}_{i,j}$.
\end{dfn}

\begin{dfn}
\label{def:BSicapj}
Let $\mathbf{S}$ be an alignment. Let $i,j \in [1,m]$. Define $\mathbf{B}(\mathbf{S})_{i,j}$ to be a matrix whose columns are $\mathbf{S}_i^T\mathbf{x}_{k,i}+\mathbf{t}_i$ (in increasing order of $k$) where $(k,i),(k,j) \in E(\Gamma)$ and where $\mathbf{t}_i$ is obtained using Eq.~(\ref{eq:opt_Z}). Also, define
\begin{align}
    \overline{\mathbf{B}(\mathbf{S})}_{i,j} = \mathbf{B}(\mathbf{S})_{i,j}\left(\mathbf{I}_{n'} - \frac{\mathbf{1}_{n'}\mathbf{1}_{n'}^T}{n'}\right).
\end{align}
\end{dfn}

\begin{rmk}
\label{rmk:BS_ijB_ij}
Let $i,j \in [1,m]$ and $\mathbf{S}$ be an alignment. Let  $n' = |\{k:(k,i),(k,j) \in E(\Gamma)\}|$ be the number of points on the overlap of the two views. Then
\begin{align}
\mathbf{B}(\mathbf{S})_{i,j} = \mathbf{S}_i^T \mathbf{B}_{i,j} + \mathbf{t}_i\mathbf{1}_{n'}^T
\end{align}
where $\mathbf{t}_i$ is obtained using Eq.~(\ref{eq:opt_Z}). Thus, we have
\begin{align}
    \rank (\overline{\mathbf{B}}_{i,j}) = \rank (\overline{\mathbf{B}(\mathbf{S})}_{i,j}).
\end{align}
Moreover,
\begin{align}
    \mathbf{B}(\mathbf{S})_{i,j}\left(\mathbf{I}_{n'} - \frac{\mathbf{1}_{n'}\mathbf{1}_{n'}^T}{n'}\right)\mathbf{B}(\mathbf{S})_{j,i}^T = \mathbf{S}_i^T\mathbf{B}_{i,j}\left(\mathbf{I}_{n'} - \frac{\mathbf{1}_{n'}\mathbf{1}_{n'}^T}{n'}\right)\mathbf{B}_{j,i}^T\mathbf{S}_j = \mathbf{S}_1^T\overline{\mathbf{B}}_{i,j}\overline{\mathbf{B}}_{j,i}^T\mathbf{S}_2
\end{align}
and in particular $\rank (\overline{\mathbf{B}(\mathbf{S})}_{i,j}\overline{\mathbf{B}(\mathbf{S})}_{j,i}^T) = \rank (\overline{\mathbf{B}}_{i,j}\overline{\mathbf{B}}_{j,i}^T)$.
\end{rmk}

\begin{thm}
\label{thm:non_deg_two_views_gen_setting}
Consider $m=2$ and let $\mathbf{S} \in \mathbb{O}(d)^2$. Then $\mathbf{S}$ is a non-degenerate alignment if and only if all of the following hold: (see Figures~\ref{fig:nec_suff_cond_loc_rigid_two_views_1} and \ref{fig:nec_suff_cond_loc_rigid_two_views} for intuition when $d=2$)
\begin{enumerate}
    \itemsep0em 
    \item $\overline{\mathbf{B}(\mathbf{S})}_{1,2}\overline{\mathbf{B}(\mathbf{S})}_{2,1}^T$ is symmetric.
    \item $\mathrm{Tr}(\boldsymbol{\Omega}^T \overline{\mathbf{B}(\mathbf{S})}_{1,2}\overline{\mathbf{B}(\mathbf{S})}_{2,1}^T\boldsymbol{\Omega}) \geq 0$ for all $\boldsymbol{\Omega} \in \Skew(d)$.
    \item $\rank\left(\overline{\mathbf{B}(\mathbf{S})}_{1,2}\overline{\mathbf{B}(\mathbf{S})}_{2,1}^T\right) \geq d-1$ (equivalently $\rank (\overline{\mathbf{B}}_{1,2}\overline{\mathbf{B}}_{2,1}^T) \geq d-1$).
\end{enumerate}
\end{thm}

\subsection{Unique Optimal Alignment in the General Setting}
\label{subsec:uniq_gen_setting}
Since $F(\mathbf{S}) = F(\mathbf{S}\mathbf{Q})$ for all $\mathbf{Q} \in \mathbb{O}(d)$, if $\mathbf{S}$ is an optimal alignment i.e. a global minimum then so is $\mathbf{S}\mathbf{Q}$. In this sense, no optimal alignment is unique. With a slight abuse of convention we define a unique optimal alignment as follows.
\begin{dfn}
\label{def:uniq_alignment}
An alignment $\mathbf{S} \in \mathbb{O}(d)^m$ is a unique optimal alignment if $\pi(\mathbf{S})$ is the unique global minimum of $\widetilde{F}$ or equivalently, $\mathbf{S}$ is an optimal alignment that is unique up to the action of $\mathbb{O}(d)$: for each optimal alignment $\mathbf{S}'$, there exist $\mathbf{Q} \in \mathbb{O}(d)$ such that $\mathbf{S}' = \mathbf{S}\mathbf{Q}$.
\end{dfn}

Now we provide a necessary and sufficient condition for an optimal alignment of two views to be unique. Although the proof can be found in \citea{schonemann1966generalized}, for completeness, we provide a proof using the constructs derived so far.

\begin{thm}
\label{thm:uniq_two_views_gen_setting}
Consider $m=2$ and let $\mathbf{S}$ be an optimal alignment. Then $\mathbf{S}$ is unique if and only if $\rank (\overline{\mathbf{B}}_{1,2}\overline{\mathbf{B}}_{2,1}^T) = d$ (see Figures~\ref{fig:nec_suff_cond_loc_rigid_two_views} and \ref{fig:nec_suff_cond_glob_rigid_two_views} for intuition when $d=2$).
\end{thm}

\begin{figure}[H]
    \centering
    \begin{tabular}{ccc}
    \begin{subfigure}[b]{0.175\textwidth}
         \centering
         \includegraphics[width=0.9\textwidth,keepaspectratio]{../fig/fig0/nec_suff_loc_rigid_1.png}
         \caption{}
         \label{fig:nec_suff_cond_loc_rigid_two_views_1}
     \end{subfigure}
     &
     \begin{subfigure}[b]{0.175\textwidth}
         \centering
         \includegraphics[width=0.9\textwidth,keepaspectratio]{../fig/fig0/nec_suff_loc_rigid_2.png}
         \caption{}
         \label{fig:nec_suff_cond_loc_rigid_two_views}
     \end{subfigure}
     &
     \begin{subfigure}[b]{0.175\textwidth}
         \centering
         \includegraphics[width=0.9\textwidth,keepaspectratio]{../fig/fig0/nec_suff_glob_rigid.png}
         \caption{}
         \label{fig:nec_suff_cond_glob_rigid_two_views}
     \end{subfigure}
     \end{tabular}
    \caption{The dotted lines represent views and the filled points represent points on the overlaps. All the pair of views are perfectly aligned. In (\ref{fig:nec_suff_cond_loc_rigid_two_views_1}), $\rank (\overline{\mathbf{B}}_{1,2}\overline{\mathbf{B}}_{2,1}^T) = 0$ and clearly the two views can be rotated by a different amount while still being perfectly aligned. In (\ref{fig:nec_suff_cond_loc_rigid_two_views}), $\rank (\overline{\mathbf{B}}_{1,2}\overline{\mathbf{B}}_{2,1}^T) = 1$ and in order for the views to be perfectly aligned, every infinitesimal rotation of the two views must be identical. However the perfect alignment of the views is not unique because the second view can be flipped (a non-infinitesimal rotation) to obtain another perfect alignment of the views. In (\ref{fig:nec_suff_cond_glob_rigid_two_views}), $\rank (\overline{\mathbf{B}}_{1,2}\overline{\mathbf{B}}_{2,1}^T) = 2$ and the perfect alignment is unique.}
    \label{fig:geom_intuit}
\end{figure}

\section{Non-degeneracy and Uniqueness in the Noiseless Regime}
\label{sec:noiseless_non_deg_results}
%The organization of the section is as follows.
% We start by deriving some important consequences of the noiseless setting in Section~\ref{subsec:noiseless_conseq} which are used in the subsequent sections.
% In Section~\ref{subsec:loc_glob_rigid}, under a mild assumption on the structure of the local views, we show that the non-degeneracy of a perfect alignment is equivalent to the infinitesimal rigidity of the resulting realization and to the local rigidity of a generic realization. We also show that the uniqueness of a perfect alignment is equivalent to the global rigidity \citea{gortler2010affine} of the resulting realization.
% \revadd{We compare our results with previously established findings on the affine rigidity of a realization in \cite{chaudhury2015global,gortler2010affine}.}
% \revadd{Then, in Section~\ref{subsec:non_deg_noiseless_setting} and Section~\ref{subsec:uniq_noiseless_setting}, we provide necessary and sufficient conditions on the overlapping structure of the views for the non-degeneracy and uniqueness of a perfect alignment.
% Combined with the results in Section~\ref{subsec:loc_glob_rigid}, we obtain conditions on a perfect alignment for the resulting realization to be infinitesimally/locally/globally rigid. These conditions should be contrasted with those for the affine rigidity of a realization, as presented in \citea{zha2009spectral}.}
We start by deriving some important consequences of the noiseless setting. Under a mild assumption on the structure of the local views, we show that the non-degeneracy and uniqueness of a perfect alignment is equivalent to certain notions of rigidity of the resulting realization (Figure~\ref{fig:rigidity_flow}). We then provide necessary and sufficient conditions on the overlapping structure of the views for the non-degeneracy and uniqueness of a perfect alignment. \revadd{Consequently, we obtain conditions on a perfect alignment for the resulting realization to be infinitesimally/locally/globally rigid. These should be contrasted with the ones in \citea{zha2009spectral}, for the affine rigidity of a realization.}
\begin{figure}[H]
    \centering
    \resizebox{0.65\textwidth}{!}{%
    \begin{tikzpicture}[
        NodeA/.style={rectangle, draw=black!60, fill=green!5, very thick, minimum size=7mm, text width=3cm},
        NodeB/.style={rectangle, draw=black!60, fill=red!5, very thick, minimum size=5mm, text width=3cm},
        NodeC/.style={rectangle, draw=black!60, fill=blue!5, very thick, minimum size=7mm, text width=3cm},
        NodeD/.style={rectangle, draw=black!60, fill=blue!5, very thick, minimum size=7mm, text width=5cm},
        every text node part/.style={align=center}
        ]
        %Nodes
        \node[NodeB]        (irigid)  {infinitesimally rigid $\Theta(\mathbf{S})$};
        \node[NodeC]      (rankR) [above=0.85cm of irigid] {$\rank(\boldsymbol{\mathcal{R}}(\mathbf{S})) \geq nd - d(d+1)/2$};
        \node[NodeB]      (lrigid)       [below=of irigid] {locally rigid $\Theta(\mathbf{S})$};
        \node[NodeB]      (grigid)   [below=of lrigid]        {globally rigid $\Theta(\mathbf{S})$};
        \node[NodeB]        (arigid)       [below=of grigid] {affinely rigid $\Theta(\mathbf{S})$};
        
        
        \node[NodeA]        (nondegS)       [right=3cm of irigid] {non-degenerate $\mathbf{S}$};
        \node[NodeA]        (strictS)       [right=3cm of lrigid] {$\pi(\mathbf{S})$ is a strict minimum of $\widetilde{F}$};
        \node[NodeC]        (rankLbb)       [above=of nondegS] {$\rank(\boldsymbol{\mathbb{L}}(\mathbf{S})) = (m-1)d(d-1)/2$};
        \node[NodeA]        (uniqueS)       [right=3cm of grigid] {unique $\mathbf{S}$};
        \node[NodeC]        (rankC)       [right=3cm of arigid] {$\rank(\mathbf{C}) = (m-1)d$};
        \coordinate[below right=2cm and 0.5cm of rankLbb]  (rankLbb0) ;
        \coordinate[above right=2cm and 0.5cm of rankC]  (rankC0) ;

        \path (rankLbb) -- node (rankLbbiffnondegS) {Proposition~\ref{prop:noiseless_setting1}} (nondegS);
        \path (nondegS) -- node (nondegSimpstrictS) {Trivial} (strictS);
        \path (lrigid.20) -- node (lrigidimpirigid) {Generic $\Theta(\mathbf{S})$} (irigid.330);
        
        %Lines
        %\draw[-implies,double equal sign distance, line width=0.4mm] (irigid.east) --(irigidimpnondegS)-- (nondegS.west);
        \draw[implies-implies, double, line width=0.4mm] (irigid.east) -- node [midway,above] {Theorem~\ref{thm:inf_rigid}} (nondegS.west);
        \draw[-implies,double, line width=0.4mm] (irigid.230) --node[midway,left] {\citea{toth2017handbook}} (lrigid.145);
        \draw[-implies,double, line width=0.4mm] (lrigid.20) -- (lrigidimpirigid) -- (irigid.330);
        %\draw[implies-,double equal sign distance, line width=0.4mm] (strictS.west)--(lrigidiffstrictS);
        %\draw[-implies,double equal sign distance, line width=0.4mm] (lrigidiffstrictS)--(lrigid.east);
        \draw[implies-implies, double, line width=0.4mm] (lrigid.east) -- node[midway,above] {Proposition~\ref{prop:non_deg_views}} (strictS.west);
        \draw[-implies,double, line width=0.4mm] (grigid.north) -- node [midway,left] {\citea{gortler2010affine}} (lrigid.south);
        \draw[implies-implies,double, line width=0.4mm] (irigid.north) -- node [midway,left] {\citea{toth2017handbook}} (rankR.south);
        %\draw[-implies,double, line width=0.4mm](uniqueSiffgrigid)--(grigid.east);
        %\draw[implies-,double, line width=0.4mm] (uniqueS.west)--(uniqueSiffgrigid);
        \draw[implies-implies,double, line width=0.4mm] (grigid.east)--node[midway,above]{Theorem~\ref{thm:glob_rigid}} (uniqueS.west);
        \draw[-implies,double, line width=0.4mm] (arigid.north) -- node [midway,left] {\citea{gortler2010affine}} (grigid.south);

        \draw[-,double, line width=0.4mm] (rankC.east) -| (rankC0);
        \draw[-,double, line width=0.4mm] (rankLbb0) --node[above,midway,sloped] {Corollary~\ref{cor:noiseless_setting1}} (rankC0);
        \draw[-implies,double, line width=0.4mm] (rankLbb0) |- (rankLbb.east) ;

        \draw[implies-implies,double, line width=0.4mm] (rankC.west) -- node [midway, above] {\citea{chaudhury2015global,zha2009spectral}} (arigid.east);

        \draw[implies-,double, line width=0.4mm] (rankLbb.south)--(rankLbbiffnondegS);
        \draw[-implies,double, line width=0.4mm] (rankLbbiffnondegS)--(nondegS.north);
        \draw[-implies,double, line width=0.4mm] (nondegS.south) --(nondegSimpstrictS)-- (strictS.north);
        
    \end{tikzpicture}
    }
    \caption{\revadd{The implications between the type of a perfect alignment $\mathbf{S}$ and the rigidity of the resulting realization $\Theta(\mathbf{S})$.}}
    \label{fig:rigidity_flow}
\end{figure}
\subsection{Consequences of Noiseless Setting}
\label{subsec:noiseless_conseq}
As discussed in \citea{chaudhury2015global}, in the noiseless case, the patch-stress matrix $\mathbf{C}$ is constructed from $\Gamma$ and clean measurements. In particular, there exists a perfect alignment $\mathbf{S}$ such that $F(\mathbf{S}) = 0$.
%The following consequences of the noiseless setting will play a crucial role in the subsequent sections.
\begin{prop}
\label{prop:noiseless_setting1}
Let $\mathbf{S}$ be a perfect alignment. Then $\widehat{\mathbf{C}}(\mathbf{S}) = 0$, $\mathbf{L}(\mathbf{S}) = \mathbf{C}(\mathbf{S})$ (see Eq.~(\ref{eq:L_of_S})) and $\mathbb{L}(\mathbf{S}) = \overline{\mathbf{P}}(\mathbf{I}_d \otimes  (\mathbf{P}\mathbf{C}(\mathbf{S})\mathbf{P}^T))\overline{\mathbf{P}}^T$ (see Eq.~(\ref{eq:mathbb_L})). Consequently, it is easy to deduce from Remark~\ref{rmk:C_S_structure} that $\mathbf{L}(\mathbf{S}) \succeq 0$ and $\mathbb{L}(\mathbf{S}) \succeq 0$. \revadd{It follows from Theorem~\ref{thm:non_deg_loc_min} that $\mathbf{S}$ is non-degenerate if and only if $\rank(\mathbb{L}(\mathbf{S})) = (m-1)d(d-1)/2$.}
\end{prop}
% \begin{rmk}
% \label{prop:noiseless_setting1}
% Due to the above proposition and Remark~\ref{rmk:C_S_structure}, it is easy to deduce that for a perfect alignment $\mathbf{S}$, $\mathbf{L}(\mathbf{S}) \succeq 0$ and $\mathbb{L}(\mathbf{S}) \succeq 0$. Consequently, from Theorem~\ref{thm:non_deg_loc_min}, $\mathbf{S}$ is non-degenerate if and only if $\rank(\mathbb{L}(\mathbf{S})) = (m-1)d(d-1)/2$.
% \end{rmk}

\begin{rmk}
\label{rmk:C1Sp}
\revadd{
Following Remark~\ref{rmk:mathcalLpp} and due to the above proposition, it is easy to deduce that for a perfect alignment $\mathbf{S}$, $\mathbf{\mathcal{L}}(\mathbf{S})_{pp}$ is exactly the patch-stress matrix (see Eq.~(\ref{eq:GPOP})) of the one-dimensional patch framework  $(\Gamma, (\mathbf{S}_i^T\mathbf{x}_{k,i}(p)))$.}
\end{rmk}

\revadd{As a direct corollary of Proposition~\ref{prop:HessVicinity}, we obtain a bound on the neighborhood of a non-degenerate perfect alignment where the Hessian is positive~definite.}
\begin{cor}
\label{cor:HessVicinity}
\revadd{Let $\mathbf{S}$ be a non-degenerate perfect alignment.
%Then, using the fact that $\mathbf{C}\mathbf{S} = 0$,  Proposition~\ref{prop:noiseless_setting1} and Eq.~(\ref{eq:C_of_S}),
As in Proposition~\ref{prop:HessVicinity}, define $c_1 = \max_{1}^{m}\sigma_{\max}(\mathbf{C}_{k,:})$, $c_3 \coloneqq \sigma_{\max}(\mathbf{C})$, $\lambda_{0_-}(\mathbf{S}) \coloneqq \lambda_{d(d-1)/2+1}(\mathbb{L}(\mathbf{S}))$ and $\lambda_{0_+}(\mathbf{S}) \coloneqq \lambda_{md(d-1)/2}(\mathbb{L}(\mathbf{S}))$ (from Proposition~\ref{prop:noiseless_setting1} and Eq.~(\ref{eq:omega^TmbbLomega}) it is easy to deduce that $0 < \lambda_{0_-}(\mathbf{S}) \leq \lambda_{0_+}(\mathbf{S}) \leq 2\lambda_{md}(\mathbf{C})$). Let $\zeta \in (0,1)$ be fixed and define}
\begin{equation}
\label{eq:delta}
\revadd{\delta_0(\mathbf{S}) \coloneqq |\lambda_{0_-}(\mathbf{S})|/ 2(c_1 + 2 c_{3}).}
\end{equation}
\revadd{If $\mathbf{O} \in \mathbb{O}(d)^m$ satisfies $\min_{\mathbf{Q}\in\mathbb{O}(d)}\left\|\mathbf{O}-\mathbf{S}\mathbf{Q}\right\|_F \leq \zeta\delta_0(\mathbf{S})$, then for all $\widetilde{\mathbf{Z}} \in T_{\widetilde{\mathbf{O}}}\mathbb{O}(d)^m/_{\sim}$,}
\begin{equation}
    \revadd{(1-\zeta)\lambda_{0_-}(\mathbf{S})\widetilde{g}(\widetilde{\mathbf{Z}}, \widetilde{\mathbf{Z}}) \leq \widetilde{g}(\Hess \widetilde{F}(\widetilde{\mathbf{O}})[\widetilde{\mathbf{Z}}],\widetilde{\mathbf{Z}}) \leq (\lambda_{0_+}(\mathbf{S}) + \zeta\lambda_{0_-}(\mathbf{S})) \widetilde{g}(\widetilde{\mathbf{Z}}, \widetilde{\mathbf{Z}}).}
\end{equation}
% Suppose $\mathbf{S}$ is a non-degenerate perfect alignment. Then, using the fact that $\mathbf{C}\mathbf{S} = 0$,  Proposition~\ref{prop:noiseless_setting1} and Eq.~(\ref{eq:C_of_S}), we have $\delta_1 = \max_{1}^{m}\sigma_{\max}(\mathbf{C}_{k,:})$, $ \delta_2(\mathbf{S}) = 0$ and $\delta_3(\mathbf{S}) \equiv \delta_3 =  \sigma_{\max}(\mathbf{C})$.
% Let $\lambda \coloneqq \lambda_{d(d-1)/2+1}(\mathbb{L}(\mathbf{S})) < 0$ (follows from Theorem~\ref{thm:non_deg_loc_min}). If $\mathbf{O} \in \mathbb{O}(d)^m$ is such that $\min_{\mathbf{Q}\in\mathbb{O}(d)}\left\|\mathbf{O}-\mathbf{S}\mathbf{Q}\right\|_F < (2(\delta_1 + 2 \delta_{3})))^{-1}|\lambda|$, then $\Tr(\boldsymbol{\Omega}^T(\mathbf{L}(\mathbf{O})+\mathbf{L}(\mathbf{O})^T)\boldsymbol{\Omega}) < 0$ for all $\boldsymbol{\Omega} = [\boldsymbol{\Omega}_i]_1^m$ such that $\boldsymbol{\Omega}_i \in \Skew(d)$, $\sum_1^m \boldsymbol{\Omega}_i = 0$ and not all $\boldsymbol{\Omega}_i$ equal zero.
\end{cor}

Finally, \revadd{using Proposition~\ref{prop:one_all1} and Proposition~\ref{prop:noiseless_setting1},} we provide a simplified characterization of a non-degenerate perfect alignment that will be useful in proving the subsequent results. First, similar to \citea{zha2009spectral}, we define a certificate of $\mathbf{L}(\mathbf{S})$.
\begin{dfn}
\label{def:LScertificate}
An $\boldsymbol{\Omega} \in \Skew(d)^m$ is said to be a certificate of $\mathbf{L}(\mathbf{S})$ if $\mathbf{L}(\mathbf{S})\boldsymbol{\Omega} = 0$. It is a trivial certificate if $\boldsymbol{\Omega}_i = \boldsymbol{\Omega}_0$ for all $i \in [1,m]$ and for some $\boldsymbol{\Omega}_0 \in \Skew(d)$.
\end{dfn}

% Then the characterization of a non-degenerate perfect alignment (obtained trivially from Proposition~\ref{prop:one_all1} and ~\ref{prop:noiseless_setting1}) is as follows.
\begin{prop}
\label{prop:non_deg_triv_cert}
If $\mathbf{S}$ is a perfect alignment then $\mathbf{S}$ is non-degenerate iff every certificate of $\mathbf{L}(\mathbf{S})$ is trivial.
\end{prop}
% \revdel{From Corollary~\ref{cor:suff_non_deg_loc_min}, Proposition~\ref{prop:noiseless_setting1} and Remark~\ref{rmk:C_S_structure}, it follows that
% \begin{cor}
% \label{cor:noiseless_setting1_old}
% If $\mathbf{C}$ is of rank $(m-1)d$ then every perfect alignment of $F$ is non-degenerate.
% \end{cor}
% }
%%
%%
%%
%%
%%
%% 
%%
%%
%%
%%
%%
%%
% \begin{figure}[h]
%     \centering
%     \begin{tikzpicture}[
%         NodeA/.style={rectangle, draw=black!60, fill=green!5, very thick, minimum size=7mm, text width=3cm},
%         NodeB/.style={rectangle, draw=black!60, fill=red!5, very thick, minimum size=5mm, text width=3cm},
%         NodeC/.style={rectangle, draw=black!60, fill=blue!5, very thick, minimum size=7mm, text width=3cm},
%         NodeD/.style={rectangle, draw=black!60, fill=blue!5, very thick, minimum size=7mm, text width=5cm},
%         every text node part/.style={align=center}
%         ]
%         %Nodes
%         \node[NodeB]      (grigid)                              {globally rigid $\Theta(\mathbf{S})$};
%         \node[NodeB]      (lrigid)       [right=of grigid] {locally rigid $\Theta(\mathbf{S})$};
%         \node[NodeB]        (arigid)       [left=of grigid] {affinely rigid $\Theta(\mathbf{S})$};
%         \node[NodeA]        (uniqueS)       [above=of grigid] {unique $\mathbf{S}$};
%         \node[NodeA]        (strictS)       [above=of lrigid] {$\pi(\mathbf{S})$ is a strict minimum of $\widetilde{F}$};
%         \node[NodeC]        (nondegS)       [above=of strictS] {non-degenerate $\mathbf{S}$};
%         \node[NodeA]        (rankC)       [above=of arigid] {$\rank(\mathbf{C}) = (m-1)d$};
%         \node[NodeD]        (rankLbb)       [above=of uniqueS, xshift=-3.25cm, yshift=0.25cm] {$\rank(\boldsymbol{\mathbb{L}}(\mathbf{S})) = (m-1)d(d-1)/2$};
%         \node[NodeB]        (irigid)       [below=of lrigid] {infinitesimally rigid $\Theta(\mathbf{S})$};
%         \path (rankC) -- node (rankCimprankLbb) {Corollary~\ref{cor:noiseless_setting1}} (rankLbb);
%         \path (rankLbb) -- node (rankLbbiffnondegS) {Theorem~\ref{thm:loc_rigid}} (nondegS);
%         \path (nondegS) -- node (nondegSimpstrictS) {Trivial} (strictS);
%         \path (strictS) -- node (nondegSifflrigid){Proposition~\ref{prop:non_deg_views}} (lrigid);
%         \path (uniqueS) -- node (uniqueSiffgrigid) {Theorem~\ref{thm:glob_rigid}} (grigid);
%         \path (irigid) -- node (irigidimpnondegS) {Theorem~\ref{thm:inf_rigid}} (nondegS);
%         %Lines
%         \draw[implies-implies,double equal sign distance, line width=0.4mm] (rankC.south) -- node [text width=1cm,midway,right] {\citea{chaudhury2015global,zha2009spectral}} (arigid.north);
%         \draw[-implies,double equal sign distance, line width=0.4mm] (grigid.east) -- node [text width=1cm,midway,above] {\citea{gortler2010affine}} (lrigid.west);
%         \draw[-implies,double equal sign distance, line width=0.4mm] (arigid.east) -- node [text width=1cm,midway,above] {\citea{gortler2010affine}} (grigid.west);
%         \draw[implies-,double equal sign distance, line width=0.4mm] (uniqueS.south)--(uniqueSiffgrigid);
%         \draw[-implies,double equal sign distance, line width=0.4mm](uniqueSiffgrigid)--(grigid.north);
%         \draw[implies-,double equal sign distance, line width=0.4mm] (strictS.south)--(nondegSifflrigid);
%         \draw[-implies,double equal sign distance, line width=0.4mm] (nondegSifflrigid)--(lrigid.north);
%         \draw[-implies,double equal sign distance, line width=0.4mm] (rankC.north)--(rankCimprankLbb)--(rankLbb.south);
%         \draw[implies-,double equal sign distance, line width=0.4mm] (rankLbb.east)--(rankLbbiffnondegS);
%         \draw[-implies,double equal sign distance, line width=0.4mm] (rankLbbiffnondegS)--(nondegS.west);
%         \draw[-implies,double equal sign distance, line width=0.4mm] (nondegS.south) --(nondegSimpstrictS)-- (strictS.north);
%         \draw[-implies,double equal sign distance, line width=0.4mm] (irigid.south) --(irigidimpnondegS)-- (nondegS.north);
%     \end{tikzpicture}
%     \caption{The implications between the type of a perfect alignment $\mathbf{S}$ and the rigidity of the resulting realization $\Theta(\mathbf{S})$.}
%     \label{fig:rigidity_flow}
% \end{figure}

\subsection{Rigidity of a Realization}
\label{subsec:loc_glob_rigid}
In the following, we reveal the relation between non-degenerate and unique perfect alignment with the various notions of the rigidity of the resulting realization. \revadd{These are summarized in Figure~\ref{fig:rigidity_flow}.}
\revadd{Throughout the rest of this work, we assume the following.}
\begin{assump}
\label{assump:non_deg_views}
\revadd{Each view is affinely non-degenerate i.e. has at least $d+1$ points whose affine span has a rank of $d$.}
\end{assump}

\revadd{Consequently, the perfect alignment of the local views can be uniquely determined by their realization. This can be easily inferred from the following result.}
\begin{prop}
\label{prop:non_deg_views}
\revadd{Let $\mathbf{B}_{i,i}$, $i \in [1,m]$, be as in Definition~\ref{def:Bij}. Define $\varrho = (\sum_1^m 1/\sigma_{\min}(\mathbf{B}_{i,i}\mathbf{B}_{i,i}^T)^{2})^{1/2}$. Then for perfect alignments $\mathbf{S}$ and $\mathbf{O}$, and the corresponding realizations $\Theta(\mathbf{S})$ and $\Theta(\mathbf{O})$, $\left\|\mathbf{S} - \mathbf{O}\right\|_F \leq \varrho \left\|\Theta(\mathbf{S}) - \Theta(\mathbf{O})\right\|_F$. In particular, $\Theta(\mathbf{S}) = \Theta(\mathbf{O})$ if and only if $\mathbf{S} = \mathbf{O}$.}
\end{prop}
% \revadd{Due to Definition~\ref{def:realization}, for any two perfect alignments $\mathbf{S}, \mathbf{O} \in \mathbb{O}(d)^m$, we have}
% {\small
% \begin{align}
%     \mathbf{0}_{d} = \textstyle\argmmin_{\mathbf{t} \in \mathbb{R}^d} \left\|\Theta(\mathbf{O}) - \mathbf{Q}^T\Theta(\mathbf{S}) - \mathbf{t}\mathbf{1}_n^T\right\|_F &= \left\|\Theta(\mathbf{O}) - \mathbf{Q}^T\Theta(\mathbf{S})\right\|_F = \left\|\Theta(\mathbf{O}) - \Theta(\mathbf{S}\mathbf{Q})\right\|_F. \label{eq:loc_rigid_pre}
% \end{align}
% }
% \begin{equation}
%     \revadd{\textstyle\min_{\substack{\mathbf{Q} \in \mathbb{O}(d)\\\mathbf{t} \in \mathbb{R}^d}} \left\|\Theta(\mathbf{O}) - \mathbf{Q}^T\Theta(\mathbf{S}) - \mathbf{t}\mathbf{1}_n^T\right\|_F = \textstyle\min_{\mathbf{Q} \in \mathbb{O}(d)} \left\|\Theta(\mathbf{O}) - \Theta(\mathbf{S}\mathbf{Q})\right\|_F.} \label{eq:loc_rigid_pre}
% \end{equation}

Now we define various notions of the rigidity of a realization $\Theta(\mathbf{S})$. Although phrased differently, the definitions are the same as those in \citea{toth2017handbook, gortler2010affine, chaudhury2015global}.
% \begin{dfn}
% Let $\mathbf{S}$ be a global minimum. Then $\Theta(\mathbf{S})$ is said to be locally rigid if there exist $\epsilon > 0$ such that for any $Y = [y_k]_1^n$ with $\left\|Y-\Theta(\mathbf{S})\right\|_F < \epsilon$ and
% \begin{align}
%     y_k = O_i^Tx_{k}(\mathbf{S}) + v_i, (k,i) \in E
% \end{align}
% for another global minimum $\{O_i\}_1^m \subseteq \mathbb{O}(d)$ and $\{v_i\}_1^m \subseteq \mathbb{R}^d$ (i.e. $Y = X([\mathbf{S}_iO_i]_1^m)$), we have $Y = \Theta(\mathbf{S})Q = \Theta(\mathbf{S}\mathbf{Q})$ for some $\mathbf{Q} \in \mathbb{O}(d)$.
% \end{dfn}
\begin{dfn}
\label{def:inf_rigid}
\revadd{Let $\mathbf{S}$ be a perfect alignment. Then $\Theta(\mathbf{S}) = (\mathbf{x}_k(\mathbf{S}))_1^n$ is infinitesimally rigid if there does not exist a perturbation $(\mathbf{p}_k)_1^n \subseteq \mathbb{R}^d$ satisfying:
\begin{enumerate}[leftmargin=*]
    \item $(\mathbf{p}_k)_1^n$ is is not a trivial perturbation (it is a trivial perturbation if there exist $\boldsymbol{\Omega} \in \Skew(d)$ and $\mathbf{t} \in \mathbb{R}^d$ such that $\mathbf{p}_k = \boldsymbol{\Omega}\mathbf{x}_k + \mathbf{t}$),
    \item and for all $(k_1,i), (k_2,i) \in E(\Gamma)$, $(\mathbf{x}_{k_1}(\mathbf{S})-\mathbf{x}_{k_2}(\mathbf{S}))^T(\mathbf{p}_{k_1}-\mathbf{p}_{k_2}) = 0$.
\end{enumerate}
}
\begin{rmk}
\label{rmk:inf_rigid}
\revadd{The above two conditions can be described in terms of the rank of the so-called rigidity matrix $\boldsymbol{\mathcal{R}}(\mathbf{S})$ \citea{toth2017handbook}. The rigidity matrix has a row for each triplet $(k_1, k_2, i)$ satisfying $(k_1,i), (k_2,i) \in E(\Gamma)$, and the $k_1$th and $k_2$th the blocks of the row are $(\mathbf{x}_{k_1}(\mathbf{S})-\mathbf{x}_{k_2}(\mathbf{S}))^T$ and $(\mathbf{x}_{k_2}(\mathbf{S})-\mathbf{x}_{k_1}(\mathbf{S}))^T$, respectively. Overall, the sparse matrix $\boldsymbol{\mathcal{R}}(\mathbf{S})$ has $\sum_{1}^{m}{n_i \choose 2}$ rows and $nd$ columns, and the realization $\Theta(\mathbf{S}) = (\mathbf{x}_k(\mathbf{S}))_1^n$ is infinitesimally rigid if and only if $\rank(\boldsymbol{\mathcal{R}}(\mathbf{S})) \geq nd - d(d+1)/2$.}
\end{rmk}
\end{dfn}
For the following definitions, we use the facts due to Definition~\ref{def:realization}: (i) $\mathbf{Q}^T\Theta(\mathbf{S}) = \Theta(\mathbf{S}\mathbf{Q})$ for any $\mathbf{Q} \in \mathbb{R}^{d \times d}$ and (ii) $\mathbf{0}_{d} = \textstyle\argmin_{\mathbf{t} \in \mathbb{R}^d} \left\|\Theta(\mathbf{O}) - \mathbf{Q}^T\Theta(\mathbf{S}) - \mathbf{t}\mathbf{1}_n^T\right\|_F$.
\begin{dfn}
\label{def:loc_rigid}
Let $\mathbf{S}$ be a perfect alignment. Then $\Theta(\mathbf{S})$ is locally rigid if there exists $\epsilon > 0$ such that for any other perfect alignment $\mathbf{O} \in \mathbb{O}(d)^m$ with $\left\|\Theta(\mathbf{O})-\Theta(\mathbf{S})\right\|_F < \epsilon$, we have $\Theta(\mathbf{O})$ to be a rigid transformation of $\Theta(\mathbf{S})$ or equivalently $\Theta(\mathbf{O})  = \Theta(\mathbf{S}\mathbf{Q})$ for some $\mathbf{Q} \in \mathbb{O}(d)$.
\end{dfn}
\begin{dfn}
\label{def:glob_rigid}
Let $\mathbf{S}$ be a perfect alignment. Then $\Theta(\mathbf{S})$ is globally rigid if for any other perfect alignment $\mathbf{O} \in \mathbb{O}(d)^m$ we have $\Theta(\mathbf{O}) = \Theta(\mathbf{S}\mathbf{Q})$ for some $\mathbf{Q} \in \mathbb{O}(d)$.
\end{dfn}
\begin{dfn}
\label{def:affine_rigid}
\revadd{Let $\mathbf{S}$ be a perfect alignment. Then $\Theta(\mathbf{S})$ is affinely rigid if for any realization $\mathbf{Y} \in \mathbb{R}^{d \times n}$ satisfying: for each $i \in [1,m]$ there exist an affine transform $\mathbf{A}_i$ such that $\mathbf{Y}_k = \mathbf{A}_i(\mathbf{x}_{k,i})$, we have $\mathbf{Y} = \mathbf{A}\Theta(\mathbf{S})$ for some global affine transform~$\mathbf{A}$.}
\end{dfn}
% \begin{figure}[H]
%     \centering
%     \begin{subfigure}[b]{0.4\textwidth}
%          \centering
%          \includegraphics[width=0.4\textwidth,keepaspectratio]{fig/fig0/glob_rig_not_aff_rig.png}
%          \caption{}
%          \label{fig:glob_not_affine}
%      \end{subfigure}
%      \begin{subfigure}[b]{0.4\textwidth}
%          \centering
%          \includegraphics[width=0.7\textwidth,keepaspectratio]{fig/fig0/degenerate_config.png}
%          \caption{}
%          \label{fig:locrigiddeg}
%      \end{subfigure}
%     \caption{(a) An example of a realization \citea{gortler2010affine} that is globally rigid but not affinely rigid. (b) An example of a degenerate perfect alignment of three views in two dimensions ($\rank(\mathbb{L}(\mathbf{S})) < 2$) with locally rigid realization. These views should be considered affinely non-degenerate, satisfying Assumption~\ref{assump:non_deg_views}. For clarity, only the points in the overlapping regions are shown.}
% \end{figure}
%For completeness, we define the affine rigidity of a realization $\Theta(\mathbf{S})$ as stated in \citeb[Definition 3.2]{chaudhury2015global} (also see \cite{zha2009spectral} and \cite{gortler2010affine}).
\revadd{From the above definitions, it is easy to see that an affinely rigid realization is globally rigid which in turn is locally rigid.} Examples of realizations that are not locally rigid, locally rigid but not globally rigid and \revadd{globally rigid but not affinely rigid} are provided in Figure~\ref{fig:nec_cond_loc_rigid_of_views}, Figure~\ref{fig:suff_cond_views_non_deg}, Figure~\ref{fig:G_star_1} and \revadd{\cite[Figure~3]{gortler2010affine}} respectively.
% \revdel{\begin{assump}
% \label{assump:non_deg_views_old}
% For perfect alignments $\mathbf{S}$ and $\mathbf{O}$, $\Theta(\mathbf{S}) = \Theta(\mathbf{O}) \iff \mathbf{S} = \mathbf{O}$. 
% \end{assump}
% \begin{rmk}
% \label{rmk:non_deg_views_old}
% Assumption~\ref{assump:non_deg_views} holds when each view is affinely non-degenerate i.e. has at least $d+1$ points whose affine span has a rank of $d$. In this case, the perfect alignment of the local views can be uniquely determined by their realization.
% \end{rmk}}
\revadd{Follows our first result connecting type of a perfect alignment with the rigidity of the resulting realization.}
\begin{thm}
\label{thm:inf_rigid}
\revadd{Let $\mathbf{S}$ be a perfect alignment. The realization $\Theta(\mathbf{S})$ is infinitesimally rigid if and only if the alignment $\mathbf{S}$ is non-degenerate.}
\end{thm}
% \begin{rmk}
% \revadd{We conjecture that the above theorem holds in the noisy setting having replaced the ``realization'' with the ``consensus representation'' (Definition~\ref{def:realization}). Specifically, for an alignment $\mathbf{S} \in \mathcal{C}$ Eq.~(\ref{eq:crit_pts2}), we conjecture that $\mathbf{S}$ is non-degenerate if and only if the resulting consensus representation $\mathbf{\Theta}(\mathbf{S})$ is infinitesimally rigid.}
% \end{rmk}
\revadd{Due to the proof of the above theorem, Proposition~\ref{prop:noiseless_setting1} and Remark~\ref{rmk:inf_rigid}, one can derive non-trivial perturbations of a non-infinitesimally rigid realization $\Theta(\mathbf{S})$ by using the non-trivial vectors in the null space of $\boldsymbol{\mathcal{R}}(\mathbf{S})$ or $\boldsymbol{\mathbb{L}}(\mathbf{S})$. Furthermore, in the noiseless setting, one can test if a perfect alignment $\mathbf{S}$ is non-degenerate - either by checking if $\rank(\boldsymbol{\mathcal{R}}(\mathbf{S})) \geq nd-d(d+1)/2$ or if $\rank(\boldsymbol{\mathbb{L}}(\mathbf{S})) \geq md(d-1)/2$.}
%The asymptotic time complexity of the former is $\mathcal{O}((nd)^3)$ and the latter is $\mathcal{O}((md(d-1)/2)^3 + |E(\Gamma)|(n+m)d)$, with the preferred approach depending on the specifics of the problem.}

\revadd{It is well known that an infinitesimally rigid realization is also locally rigid \citea{toth2017handbook}, and the converse holds for generic realizations ($\Theta(\mathbf{S}) = (\mathbf{x}_k(\mathbf{S}))_1^n$ is generic if the coordinates do not satisfy any non-zero algebraic equation with rational coefficients). Here, we provide a result which elucidates a more clear picture.}
% \begin{thm}
% \label{thm:loc_rigid}
% Let $\mathbf{S}$ be a perfect alignment.
% %and suppose Assumption~\ref{assump:non_deg_views} holds.
% Then the realization $\Theta(\mathbf{S})$ is locally rigid iff $\mathbf{S}$ is non-degenerate.
% \end{thm}
\begin{thm}
\label{thm:loc_rigid}
Let $\mathbf{S}$ be a perfect alignment.
%and suppose Assumption~\ref{assump:non_deg_views} holds.
Then the realization $\Theta(\mathbf{S})$ is locally rigid iff $\pi(\mathbf{S})$ is a strict global minimum of $\widetilde{F}$. Consequently, if $\mathbf{S}$ is a non-degenerate perfect alignment then $\Theta(\mathbf{S})$ is locally rigid, and the converse holds if $\Theta(\mathbf{S})$ is generic.
\end{thm}

% \revadd{There exist pathological cases such as the one illustrated in Figure~\ref{fig:locrigiddeg}, where the converse may not hold i.e. the perfect alignment associated with a locally rigid realization may be degenerate. This also follows from the fact that a locally rigid realization may not be infinitesimally rigid, unless the realization is \textit{generic} \cite{toth2017handbook, gortler2010affine}.}
% \begin{dfn}
% \label{def:generic}
% \revadd{A realization $\Theta(\mathbf{S}) = (\mathbf{x}_k(\mathbf{S}))_1^n$ is generic if the coordinates do not satisfy any non-zero algebraic equation with rational coefficients.}
% \end{dfn}
% \revadd{Since a locally rigid generic realization is also infinitesimally rigid \cite{toth2017handbook} therefore its underlying perfect alignment is going to be non-degenerate.}
% % \revdel{Due to Corollary~\ref{cor:noiseless_setting1} and Theorem~\ref{thm:loc_rigid}, under Assumption~\ref{assump:non_deg_views}, if $\mathbf{C}$ is of rank $(m-1)d$ then every realization is locally rigid. This is consistent with the results in \citea{chaudhury2015global,gortler2010affine} in that, if each view has at least $d+1$ affinely non-degenerate points and the rank of $\mathbf{C}$ is $(m-1)d$ then the patch framework is affinely rigid and thus locally (as well as globally) rigid too. Finally,}
% Consequently, from Proposition~\ref{prop:noiseless_setting1} and Theorem~\ref{thm:loc_rigid}, we obtain a characterization of \revadd{generic} local rigidity.
% \begin{cor}
% \label{cor:non_deg_in_noiseless_setting}
% Let $\mathbf{S}$ be a perfect alignment.
% %and the Assumption~\ref{assump:non_deg_views} holds.
% Then the realization $\Theta(\mathbf{S})$ is locally rigid if $\mathbb{L}(\mathbf{S}) = \overline{\mathbf{P}}(\mathbf{I}_d \otimes  (\mathbf{P}\mathbf{C}(\mathbf{S})\mathbf{P}^T))\overline{\mathbf{P}}^T$ is of rank $(m-1)d(d-1)/2$. Additionally, if $\Theta(\mathbf{S})$ is generic then the converse holds too.
% \end{cor}
% \revdel{In contrast to the general case, here, $\mathbb{L}(\mathbf{S})$ is already positive semidefinite as a consequence of the noiseless views (see Proposition~\ref{prop:noiseless_setting1}).}
\revadd{Moreover, using Corollary~\ref{cor:suff_non_deg_loc_min}, Proposition~\ref{prop:noiseless_setting1}, we obtain a sufficient condition for any realization of a patch framework to be locally rigid.
\begin{cor}
\label{cor:noiseless_setting1}
If $\mathbf{C}$ is of rank $(m-1)d$ then every perfect alignment $\mathbf{S}$ of $F$ is non-degenerate and the realization $\Theta(\mathbf{S})$ is locally rigid.
\end{cor}
}

Using the Definition~\ref{def:uniq_alignment} and~\ref{def:glob_rigid}, it is easy to deduce that a unique perfect alignment results in a globally rigid realization and vice versa. Then, using the fact that affine rigidity implies global rigidity \cite{gortler2010affine,toth2017handbook}, and a realization is affinely rigid if and only if the rank of $\mathbf{C}$ is $(m-1)d$ \citea{chaudhury2015global}, it follows from Corollary~\ref{cor:noiseless_setting1} that the unique perfect alignment underlying an affinely rigid realization is also non-degenerate.
\begin{thm}
\label{thm:glob_rigid}
Let $\mathbf{S}$ be a perfect alignment.
%and suppose Assumption~\ref{assump:non_deg_views} holds.
Then $\Theta(\mathbf{S})$ is globally rigid iff $\mathbf{S}$ is unique.
% Then the realization $\Theta(\mathbf{S})$ is globally rigid iff $\pi(\mathbf{S})$ is the unique global minimum of $\widetilde{F}$ i.e. if $\mathbf{O}$ is another perfect alignments then $\mathbf{O} = \mathbf{S}\mathbf{Q}$ for some $\mathbf{Q} \in \mathbb{O}(d)$.
\end{thm}
% \revadd{It is well known that affine rigidity implies global rigidity \cite{gortler2010affine,toth2017handbook}. Moreover, a realization is affinely rigid if and only if the rank of $\mathbf{C}$ is $(m-1)d$ \citea{chaudhury2015global}. Combined with Corollary~\ref{cor:noiseless_setting1} and Theorem~\ref{thm:glob_rigid}, we obtain the following result.}
\begin{prop}
\label{prop:affine_rigid}
\revadd{Let $\mathbf{S}$ be a perfect alignment.
%and suppose Assumption~\ref{assump:non_deg_views} holds.
If the realization $\Theta(\mathbf{S})$ is affinely rigid then $\mathbf{S}$ is a non-degenerate and unique perfect alignment. The converse does not hold due to the counterexample in \cite[Figure~3]{gortler2010affine}.}
\end{prop}
\revadd{
%Finally, it is worth to note that the realization $\Theta(\mathbf{S})$ due to a perfect alignment $\mathbf{S}$ is affinely rigid if and only if the rank of $\mathbf{C}$ is $(m-1)d$, as shown in \citea{chaudhury2015global}. 
%Combined with above proposition, if the rank of $\mathbf{C}$ is $(m-1)d$ then the unique perfect alignment is non-degenerate. This is in fact consistent with our finding in Corollary~\ref{cor:noiseless_setting1}.
Finally, combining the affine rigidity rank condition ($\rank(\mathbf{C}) = (m-1)d$) with Corollary~\ref{cor:non_deg_d_2}, Remark~\ref{rmk:C1Sp} and Theorem~\ref{thm:loc_rigid}, we are also able to connect the local rigidity of a realization in two dimensions with affine rigidity of its projection in one dimension.
\begin{cor}
\label{cor:local_affine_rigid_in_d_2}
Let $d=2$ and $\mathbf{S}$ be a perfect alignment. Then $\Theta(\mathbf{S})$ is locally rigid if its projection in at least one of the two dimensions is affinely rigid.
\end{cor}}
%%
%%
%%
%%
%%
%% 
%%
%%
%%
%%
%%
%%
\subsection{Conditions on Overlapping Views for a Non-degenerate Perfect Alignment}
\label{subsec:non_deg_noiseless_setting}
\revadd{We now focus on deriving the necessary and sufficient conditions on the overlapping structure of the views for a perfect alignment to be non-degenerate. These are inspired by the affine rigidity criteria discussed in~\cite{zha2009spectral}. The main difference is that we impose relatively weaker rank constraint on the overlaps.
In fact, as indicated by our previous results (Figure~\ref{fig:rigidity_flow}), the conditions presented here can be viewed as those ensuring infinitesimal and generic local rigidity of a realization.} To begin with,

% \begin{figure}[h]
%     \centering
%     \begin{tikzpicture}[
%         NodeA/.style={rectangle, draw=black!60, fill=green!5, very thick, minimum size=7mm, text width=3cm},
%         NodeB/.style={rectangle, draw=black!60, fill=red!5, very thick, minimum size=5mm, text width=3cm},
%         NodeC/.style={rectangle, draw=black!60, fill=blue!5, very thick, minimum size=7mm, text width=3cm},
%         NodeD/.style={rectangle, draw=black!60, fill=blue!5, very thick, minimum size=7mm, text width=5cm},
%         every text node part/.style={align=center}
%         ]
%         %Nodes
%         \node[NodeB]      (grigid)                              {globally rigid $\Theta(\mathbf{S})$};
%         \node[NodeB]      (lrigid)       [right=of grigid] {locally rigid $\Theta(\mathbf{S})$};
%         \node[NodeB]        (arigid)       [above=of grigid] {affinely rigid $\Theta(\mathbf{S})$};
%         \node[NodeA]        (uniqueS)       [below=of grigid] {unique $\mathbf{S}$};
%         %\node[NodeA]        (strictS)       [above=of lrigid] {$\pi(\mathbf{S})$ is a strict minimum of $\widetilde{F}$};
%         %\node[NodeC]        (nondegS)       [above=of strictS] {non-degenerate $\mathbf{S}$};
%         \node[NodeA]        (obbG)       [left=of grigid] {A graph $\overline{\mathbb{G}}$ with $m$ vertices where\\$(i,j) \in E(\overline{\mathbb{G}}) \iff \rank(\overline{\mathbf{B}}_{i,j}) = d$.};
%         %\node[NodeD]        (rankLbb)       [above=of uniqueS, xshift=-3.25cm, yshift=0.25cm] {$\rank(\boldsymbol{\mathbb{L}}(\mathbf{S})) = (m-1)d(d-1)/2$};
%         %\path (rankC) -- node (rankCimprankLbb) {Corollary~\ref{cor:noiseless_setting1}} (rankLbb);
%         %\path (rankLbb) -- node (rankLbbiffnondegS) {Theorem~\ref{thm:loc_rigid}} (nondegS);
%         %\path (nondegS) -- node (nondegSimpstrictS) {Trivial} (strictS);
%         %\path (strictS) -- node (nondegSifflrigid){Proposition~\ref{prop:non_deg_views}} (lrigid);
%         %\path (uniqueS) -- node (uniqueSiffgrigid) {Theorem~\ref{thm:glob_rigid}} (grigid);
        
%         %Lines
%         % \draw[implies-implies,double equal sign distance, line width=0.4mm] (rankC.south) -- node [text width=1cm,midway,right] {\citea{chaudhury2015global,zha2009spectral}} (arigid.north);
%         % \draw[-implies,double equal sign distance, line width=0.4mm] (grigid.east) -- node [text width=1cm,midway,above] {\citea{gortler2010affine}} (lrigid.west);
%         % \draw[-implies,double equal sign distance, line width=0.4mm] (arigid.east) -- node [text width=1cm,midway,above] {\citea{gortler2010affine}} (grigid.west);
%         % \draw[implies-,double equal sign distance, line width=0.4mm] (uniqueS.south)--(uniqueSiffgrigid);
%         % \draw[-implies,double equal sign distance, line width=0.4mm](uniqueSiffgrigid)--(grigid.north);
%         % \draw[implies-,double equal sign distance, line width=0.4mm] (strictS.south)--(nondegSifflrigid);
%         % \draw[-implies,double equal sign distance, line width=0.4mm] (nondegSifflrigid)--(lrigid.north);
%         % \draw[-implies,double equal sign distance, line width=0.4mm] (rankC.north)--(rankCimprankLbb)--(rankLbb.south);
%         % \draw[implies-,double equal sign distance, line width=0.4mm] (rankLbb.east)--(rankLbbiffnondegS);
%         % \draw[-implies,double equal sign distance, line width=0.4mm] (rankLbbiffnondegS)--(nondegS.west);
%         % \draw[-implies,double equal sign distance, line width=0.4mm] (nondegS.south) --(nondegSimpstrictS)-- (strictS.north);
%     \end{tikzpicture}
%     \caption{The implications between the type of a perfect alignment $\mathbf{S}$ and the rigidity of the resulting realization $\Theta(\mathbf{S})$.}
%     \label{fig:cond_on_overlaps}
% \end{figure}

\begin{dfn}
\label{def:BSAcapB}
Let $\mathbf{S}$ be a perfect alignment. Let $A$ and $B$ be non-empty disjoint subsets of $[1,m]$. Define $\mathbf{B}(\mathbf{S})_{A,B}$ to be a matrix whose columns are $\mathbf{S}_i^T\mathbf{x}_{k,i}+\mathbf{t}_i$ (in the increasing order of $k$) where $(k,i),(k,j) \in E(\Gamma)$ for some $i \in A$ and $j \in B$, and where $\mathbf{t}_i$ is obtained using Eq.~(\ref{eq:opt_Z}). Also define $\overline{\mathbf{B}(\mathbf{S})}_{A,B} = \mathbf{B}(\mathbf{S})_{A,B}\left(\mathbf{I}_{n'} - (1/n')\mathbf{1}_{n'}\mathbf{1}_{n'}^T\right)$ where $n' = |\{k:(k,i),(k,j) \in E(\Gamma) \text{ for some } (i,j) \in A \times B\}|$.
For brevity, we denote $\mathbf{B}(\mathbf{S})_{\{i\},\{j\}}$ and $\overline{\mathbf{B}(\mathbf{S})}_{\{i\},\{j\}}$ by $\mathbf{B}(\mathbf{S})_{i,j}$ and $\overline{\mathbf{B}(\mathbf{S})}_{i,j}$ respectively, where $i \neq j$. Note that the notation is consistent with that of Definition~\ref{def:BSicapj}.
\end{dfn}

\begin{rmk}
\label{rmk:BS_ijB_ij_noiseless}
Since $\mathbf{S}$ is a perfect alignment $\mathbf{S}_i^T\mathbf{x}_{k,i}+\mathbf{t}_i = \mathbf{S}_j^T\mathbf{x}_{k,j}+\mathbf{t}_j$ for all $(k,i),(k,j) \in E(\Gamma)$, thus $\mathbf{B}(\mathbf{S})_{A,B}$ is well defined and $\mathbf{B}(\mathbf{S})_{A,B} = \mathbf{B}(\mathbf{S})_{B,A}$.
Let $i,j \in [1,m]$ then, since $\mathbf{B}(\mathbf{S})_{i,j} = \mathbf{B}(\mathbf{S})_{j,i}$, from Remark~\ref{rmk:BS_ijB_ij}, $\rank (\overline{\mathbf{B}}_{i,j}) = \rank (\overline{\mathbf{B}(\mathbf{S})}_{i,j}) = \rank (\overline{\mathbf{B}(\mathbf{S})}_{j,i}) = \rank (\overline{\mathbf{B}}_{j,i})$ and $\rank (\overline{\mathbf{B}}_{i,j}) = \rank (\overline{\mathbf{B}(\mathbf{S})}_{i,j}\overline{\mathbf{B}(\mathbf{S})}_{j,i}^T) = \rank (\overline{\mathbf{B}}_{i,j}\overline{\mathbf{B}}_{j,i}^T)$.
\end{rmk}

Due to the above two remarks and Theorem~\ref{thm:non_deg_two_views_gen_setting}, a necessary and sufficient condition for a perfect alignment of two views to be non-degenerate is easily obtained.
\begin{thm}
\label{thm:nec_suff_cond_loc_rigid_two_views}
Consider $m=2$ and let $\mathbf{S}$ be a perfect alignment. Then $\mathbf{S}$ is non-degenerate iff $\rank(\overline{\mathbf{B}}_{1,2}) \geq d-1$.
\end{thm}
\begin{figure}[H]
    \centering
    \begin{tabular}{cc}
    \begin{subfigure}[b]{0.35\textwidth}
         \centering
         \includegraphics[width=0.9\textwidth,keepaspectratio]{fig/fig0/counterex_nec_loc_rigid.png}         \caption{Theorem~\ref{thm:nec_cond_loc_rigid_of_views}}
         \label{fig:nec_cond_loc_rigid_of_views}
     \end{subfigure}
     & 
     \begin{subfigure}[b]{0.175\textwidth}
         \centering
         \includegraphics[width=0.9\textwidth,keepaspectratio]{fig/fig0/counterex_suff_loc_rigid_2.png}
         \caption{Theorem~\ref{thm:G_star_1}}
         \label{fig:G_star_1}
     \end{subfigure}
     \end{tabular}
    \caption{Counterexamples for the converse of various Theorems. (a) For every pair of nonempty partitions $A$ and $B$ of $[1,4]$, $\rank(\overline{\mathbf{B}(\mathbf{S})}_{A,B}) \geq 1$ but $\mathbf{S}$ is degenerate. (b) $\mathbf{S}$ is non-degenerate but $|\mathbb{G}^*(\mathbf{S})| = 3$. These views should be considered affinely non-degenerate, satisfying Assumption~\ref{assump:non_deg_views}. For clarity, only the points in the overlapping regions are shown.}
    \label{fig:counterex}
\end{figure}
A necessary condition for a perfect alignment of $m \geq 3$ views to be non-degenerate is as follows. The converse of the theorem does not hold, as demonstrated in Figure~\ref{fig:nec_cond_loc_rigid_of_views}.
\begin{thm}
\label{thm:nec_cond_loc_rigid_of_views}
Let $\mathbf{S}$ be a perfect alignment. If $\mathbf{S}$ is non-degenerate then the $\rank(\overline{\mathbf{B}(\mathbf{S})}_{A,B})$ is at least $d-1$ for all non-empty partitions $A$ and $B$ of $[1,m]$ i.e. for all $A,B \subseteq [1,m]$, $A,B \neq \emptyset$, $A \cap B = \emptyset$ and $A \cup B = [1,m]$.
\end{thm}

Now we derive a sufficient condition for a perfect alignment of $m \geq 3$ views to be non-degenerate. As in \citea{zha2009spectral}, we construct a graph $\mathbb{G}$ with $m$ vertices where each vertex corresponds to a view and an edge exists between the $i$th and $j$th vertices iff $\rank(\overline{\mathbf{B}}_{i,j}) \geq d-1$. The Theorem~\ref{thm:nec_suff_cond_loc_rigid_two_views} and the following propositions will play a crucial role in our next set of results,

\begin{lem}
\label{lem:subproblem_cert}
Let $\mathbf{S}$ be a perfect alignment and $\boldsymbol{\Omega}$ be a certificate of $\mathbf{L}(\mathbf{S})$. Consider removing the $i$th view and the points that lie exclusively in it. Then $\mathbf{S}_{-i} = [\mathbf{S}_j]_{j \in [1,m] \setminus \{i\}}$ is a perfect alignment of the remaining views and $[\boldsymbol{\Omega}_j]_{j \in [1,m] \setminus \{i\}}$ is a certificate of $\mathbf{L}_{-i}(\mathbf{S}_{-i})$, the matrix in Eq.~(\ref{eq:L_of_S}) associated with the remaining views.
\end{lem}

\begin{prop}
\label{prop:same_conn_comp_non_deg}
Let $\mathbf{S}$ be a perfect alignment. Let $\boldsymbol{\Omega}$ be a certificate of $\mathbf{L}(\mathbf{S})$. If $i$th and $j$th view lie in the same connected component of $\mathbb{G}$ then $\boldsymbol{\Omega}_i = \boldsymbol{\Omega}_j$.
\end{prop}

Similar to \citea{zha2009spectral}, consider the following coarsening procedure on $\mathbb{G}$ given a perfect alignment $\mathbf{S}$: (i) transform all the views using $\mathbf{S}$ (and $\mathbf{t}$ computed using Eq.~\ref{eq:opt_Z}), (ii) merge the views that lie in the same connected component of $\mathbb{G}$ and replace them with a single view, (iii) then construct the graph (in the same manner as $\mathbb{G}$) associated with the new set of views, (iv) repeat the procedure from (ii). Let the final graph over the remaining views be $\mathbb{G}^*(\mathbf{S})$, then the following result holds (the corollary follows trivially and the converse of the theorem may not hold, as shown in Figure~\ref{fig:G_star_1}).
\begin{thm}
\label{thm:G_star_1}
    A perfect alignment $\mathbf{S}$ is non-degenerate if $|\mathbb{G}^*(\mathbf{S})| = 1$.
\end{thm}
\begin{cor}
\label{cor:suff_cond_views_non_deg}
Every perfect alignment is non-degenerate if $\mathbb{G}$ is connected.
\end{cor}
\begin{rmk}
\label{rmk:tree_structure}
\revadd{From Theorem~\ref{thm:nec_cond_loc_rigid_of_views}, it is easy to see that the converse of the above corollary holds when a graph over views, in which two views are connected if they are overlapping (i.e. they share at least one common point), is a tree.}
\end{rmk}

\revadd{It is important to note that we have derived a necessary and sufficient rank based condition (Proposition~\ref{prop:noiseless_setting1}) that can be tested in polynomial time to assess the non-degeneracy of a given perfect alignment. While the above results offer a geometric interpretation of this condition, a complete understanding in the form of a single condition on the overlapping structure of the views, that is both necessary and sufficient, has yet to be established.}

%%
%%
%%
%%
%%
%% 
%%
%%
%%
%%
%%
%%
\subsection{Conditions on Overlapping Views for a Unique Perfect Alignment}
\label{subsec:uniq_noiseless_setting}
\revdel{We previously showed that the global rigidity of a realization is equivalent to the uniqueness of the corresponding perfect alignment (Theorem~\ref{thm:glob_rigid}). Thus}Here, we focus on deriving necessary and sufficient conditions on the overlapping structure of the views for a perfect alignment to be unique, equivalently, for the resulting realization to be globally rigid. From Remark~\ref{rmk:BS_ijB_ij_noiseless} and Theorem~\ref{thm:uniq_two_views_gen_setting},
\begin{thm}
\label{thm:nec_suff_cond_glob_rigid_two_views}
Consider $m=2$ and let $\mathbf{S}$ be a perfect alignment. Then $\mathbf{S}$ is unique (Definition~\ref{def:uniq_alignment}) iff $\rank (\overline{\mathbf{B}}_{1,2}) = d$.
\end{thm}

A necessary condition for a perfect alignment of $m \geq 3$ views to be unique is, %as follows.
%The converse of it may not hold, as demonstrated in Figure~\ref{fig:nec_cond_glob_rigid_views}.
\begin{thm}
\label{thm:nec_cond_glob_rigid_views}
If $\mathbf{S}$ is a unique perfect alignment then $\rank(\overline{\mathbf{B}(\mathbf{S})}_{A,B}) = d$ for all non-empty partitions $A$ and $B$ of $[1,m]$.
\end{thm}

\revadd{It was shown in \citea{zha2009spectral} that the above rank condition holds for affinely rigid realization $\Theta(\mathbf{S})$. In contrast, our requirement only requires the perfect alignment $\mathbf{S}$ to be unique, which is equivalent to the global rigidity of $\Theta(\mathbf{S})$ (Figure~\ref{fig:rigidity_flow}). Moreover, we conjecture that the converse of the above theorem holds, in which case we would obtain a characterization of a unique perfect alignment and an exponential-time algorithm to test it, aligning with the NP-hardness of testing global rigidity~\cite{saxe1979embeddability}.}

% \revadd{We note that the above result is stronger than the one in \citea{zha2009spectral} where the authors showed that the above rank condition holds when the realization $\Theta(\mathbf{S})$ is affinely rigid, while we require the perfect alignment $\mathbf{S}$ to be unique which is equivalent to global rigidity of $\Theta(\mathbf{S})$. Additionally, we conjecture that the converse of the above theorem holds too. If it is indeed the case then we would obtain a characterization of a unique perfect alignment and an exponential time algorithm to test it. Note that testing if a realization is globally rigid is NP-hard \cite{saxe1979embeddability}.}

Now we derive a sufficient condition for a perfect alignment of $m \geq 3$ to be unique. As in the previous section, we construct a graph $\overline{\mathbb{G}}$ with $m$ vertices, one for each view. An edge exists between the $i$th and $j$th vertices iff $\rank(\overline{\mathbf{B}}_{i,j}) = d$. We need Lemma~\ref{lem:subproblem_cert}, Theorem~\ref{thm:nec_suff_cond_glob_rigid_two_views} and the following proposition for our next result,
\begin{prop}
\label{prop:same_conn_comp_uniq}
Let $\mathbf{S}$ and $\mathbf{S}'$ be perfect alignments. If $i$th and $j$th view lie in the same connected component of $\overline{\mathbb{G}}$ then $\mathbf{S}'_i = \mathbf{S}_i\mathbf{Q}$ and $\mathbf{S}'_j = \mathbf{S}_j\mathbf{Q}$ for some $\mathbf{Q} \in \mathbb{O}(d)$.
\end{prop}
\begin{figure}[H]
    \centering
     \includegraphics[width=0.15\textwidth,keepaspectratio]{fig/fig0/counterex_suff_glob_rigid.png}
    \caption{$\mathbf{S}$ is unique but $|\overline{\mathbb{G}}^*(\mathbf{S})| = 3$, thus the converse of Theorem~\ref{thm:overline_G_star_1} may not hold.}
    \label{fig:overline_G_star_1}
\end{figure}

Consider the same coarsening procedure as in Theorem~\ref{thm:G_star_1}, except that $\mathbb{G}$ and $\mathbb{G}^*(\mathbf{S})$ are replaced by $\overline{\mathbb{G}}$ and $\overline{\mathbb{G}}^*(\mathbf{S})$, respectively. Then the following holds (the corollary follows trivially and a counterexample for the converse is shown in Figure~\ref{fig:overline_G_star_1}).
\begin{thm}
\label{thm:overline_G_star_1}
    A perfect alignment $\mathbf{S}$ is unique if $|\overline{\mathbb{G}}^*(\mathbf{S})| = 1$.
\end{thm}
\begin{cor}
\label{cor:suff_cond_views_uniq}
Every perfect alignment is unique if $\overline{\mathbb{G}}$ is connected.
\end{cor}
% \begin{figure}[H]
%     \centering
%     \begin{tabular}{cc}
%     \begin{subfigure}[b]{0.35\textwidth}
%          \centering
%          \includegraphics[width=0.9\textwidth,keepaspectratio]{fig/fig0/counterex_nec_loc_rigid.png}
%          \caption{Theorem~\ref{thm:nec_cond_glob_rigid_views}}
%          \label{fig:nec_cond_glob_rigid_views}
%      \end{subfigure}
%      & 
%      \begin{subfigure}[b]{0.175\textwidth}
%          \centering
%          \includegraphics[width=\textwidth,keepaspectratio]{fig/fig0/counterex_suff_glob_rigid.png}
%          \caption{Theorem~\ref{thm:overline_G_star_1}}
%          \label{fig:overline_G_star_1}
%      \end{subfigure}
%      \end{tabular}
%     \caption{Counterexamples for the converse of various Theorems. The dotted lines represent views and the filled points represent points on the overlaps. (\ref{fig:nec_cond_glob_rigid_views}) Just a placeholder for now. (\ref{fig:overline_G_star_1}) Clearly $\mathbf{S}$ is unique but $|\overline{\mathbb{G}}^*(\mathbf{S})| = 3$.}
%     \label{fig:counterex2}
% \end{figure}
% \revadd{We would like to acknowledge that the above result is weaker than the one presented in \citea{zha2009spectral}, where the authors demonstrated that the sufficient condition leads to an affinely rigid realization $\Theta(\mathbf{S})$, whereas we only establish global rigidity of $\Theta(\mathbf{S})$. Nevertheless, we retain this result since the the proving technique is substantially different than the one in \citea{zha2009spectral}.}
\revadd{We note that the above result is weaker than the one in \citea{zha2009spectral} where the authors showed that the sufficient condition leads to affinely rigid realization $\Theta(\mathbf{S})$, while we show global rigidity. Nevertheless, we keep the result since the the proving technique is different than the one in \citea{zha2009spectral}.}

\section{Linear Convergence of RGD}
\label{sec:convergence}
In this section, we describe the RGD algorithm for solving the alignment problem in Eq.~(\ref{eq:GPOP}). Using the theory of Morse functions, we show that if the sequence of iterates generated by RGD converges to a non-degenerate alignment then the convergence is linear. Moreover, We obtain an estimate of the radius and the rate of convergence. We also present an exact recovery and noise stability analysis of RGD, initialized using the output of the spectral algorithm (SPEC) \citea{chaudhury2015global}.
%We end with a simulation demonstrating that the alignment error decays in a manner consistent with linear order of convergence, as RGD refines the spectral alignment.
% In Section~\ref{subsec:rgd_algo}, we describe the RGD algorithm for solving the alignment problem in Eq.~(\ref{eq:GPOP}). Then in Section~\ref{subsec:loc_lin_conv}, using the theory of Morse functions and Proposition~\ref{prop:HessVicinity}, we show that if the sequence of iterates generated by RGD converges to a non-degenerate alignment then the convergence is linear (Theorem~\ref{thm:rgd_conv}). Moreover, combining \citeb[Theorem~4.2]{udriste2013convex} with the bounds on Hessian at a non-degenerate alignment in Proposition~\ref{prop:HessVicinity}, we obtain an estimate of the radius and the rate of local linear convergence of RGD (Theorem~\ref{thm:rgd_conv2}). Finally, in Section~\ref{subsec:noise_stability}, we present an exact recovery and noise stability analysis of our RGD algorithm, initialized using the output of a spectral relaxation-based algorithm \citea{chaudhury2015global} to solve Eq.~(\ref{eq:GPOP}). We further provide a simulation demonstrating that the alignment error decays in a manner consistent with linear order of convergence, as RGD refines the spectral alignment.

\subsection{RGD Algorithm}
\label{subsec:rgd_algo}
A standard way to find a local minimum of Eq.~(\ref{eq:GPOP}) is to use RGD with a suitable initial point, step size and retraction strategy. In this work we use retraction based on the exponential map on $\mathbb{O}(d)$ \citea{van1996matrix, udriste2013convex}. Define,
\begin{align}
    R_{\EXP }: \cup_{\mathbf{S} \in \mathbb{O}(d)^m}(\{\mathbf{S}\} \times T_\mathbf{S}\mathbb{O}(d)^m) &\mapsto \mathbb{O}(d)^m\\
    R_{\EXP }\left([\mathbf{S}_i]_1^m, [\boldsymbol{\xi}_i]_1^m\right) &= [\mathbf{S}_i\exp(\mathbf{S}_i^T\boldsymbol{\xi}_i)]_1^m. \label{eq:R_PF}
\end{align}
where $\exp (\mathbf{A})$ denotes the matrix exponential of $\mathbf{A}$ \citea{van1996matrix, absil2009optimization}. Then the following lemma provides a consistent definition of a retraction on the quotient manifold $\mathbb{O}(d)^m/_{\sim}$.
\begin{lem}
\label{lem:retraction}
Let $\widetilde{\mathbf{S}} \in \mathbb{O}(d)^{m}/_{\sim}$ and $\mathbf{S}^a, \mathbf{S}^b \in \pi^{-1}(\widetilde{\mathbf{S}})$. If $\mathbf{Z}^a \in T_{\mathbf{S}^a}\mathbb{O}(d)^m$ and $\mathbf{Z}^b \in T_{\mathbf{S}^b}\mathbb{O}(d)^m$ are the horizontal lifts of $\widetilde{\mathbf{Z}} \in T_{\widetilde{\mathbf{S}}}\mathbb{O}(d)^{m}/_{\sim}$ then $\pi(R_{\EXP }(\mathbf{S}^a, \mathbf{Z}^a)) = \pi(R_{\EXP }(\mathbf{S}^b, \mathbf{Z}^b))$ (see Eq.~\ref{eq:pi}). As a result, the retraction
\begin{align}
    \widetilde{R}_{\EXP }: \cup_{\widetilde{\mathbf{S}} \in \mathbb{O}(d)^m/_{\sim} }(\{\widetilde{\mathbf{S}}\} \times T_{\widetilde{\mathbf{S}}}\mathbb{O}(d)^m/_{\sim}) &\mapsto \mathbb{O}(d)^m/_{\sim}\\
    \widetilde{R}_{\EXP }\left(\widetilde{\mathbf{S}}, \widetilde{\mathbf{Z}}\right) &=  \pi(R_{\EXP }(\mathbf{S}, \mathbf{Z}))\label{eq:Rtilde_PF}
\end{align}
is well defined for any $\mathbf{S} \in \pi^{-1}(\widetilde{\mathbf{S}})$ and $\mathbf{Z}$ being the horizontal lift of $\widetilde{\mathbf{Z}}$ at $\mathbf{S}$.
\end{lem}

The step direction will always be the horizontal lift of $-\grad \widetilde{F}(\widetilde{\mathbf{S}})$ at some $\mathbf{S} \in \pi^{-1}(\widetilde{\mathbf{S}})$. Consequently, due to Proposition~\ref{prop:gradFS}, the step direction is $\boldsymbol{\xi} = -\grad F(\mathbf{S}) = [[\mathbf{C}\mathbf{S}]_i - \mathbf{S}_i[\mathbf{C}\mathbf{S}]_i^T\mathbf{S}_i]_1^m$, the projection of the antigradient $-\nabla F(\mathbf{S})$ onto $T_\mathbf{S}\mathbb{O}(d)^m$. The step size $\alpha$ is calculated using the Armijo-type rule with parameters $\beta,\gamma \in (0,1)$ (here $g$ is the canonical metric on $\mathbb{O}(d)^m$ as in Eq.~(\ref{eq:g_Z_W})),
\begin{equation}
    \alpha = \max_{l \geq 0}\{\beta^l\ \vertbar\ F(R_{\EXP }(\mathbf{S}, -\beta^l\grad F(\mathbf{S}))) - F(\mathbf{S}) \leq -\gamma \beta^l g(\nabla F(\mathbf{S}),  \grad F(\mathbf{S})) \}. \label{eq:armijo_step}
\end{equation}
Since $F$ extends to a continuously differentiable non-negative function on $\mathbb{R}^{md \times d}$ containing $\mathbb{O}(d)^m$, it follows from \citeb[Proposition 2.8]{schneider2015convergence} that $\alpha$ is well-defined.
\begin{algorithm}[H]
\caption{Riemannian gradient descent for solving GPOP \label{algo:rgd}}
\begin{algorithmic}[1]
\REQUIRE $\widetilde{\mathbf{S}}^0 \in \mathbb{O}(d)^{m-1}$, $\Gamma$, $\{\mathbf{x}_{k,i}: (k,i) \in E(\Gamma)\}$, $\beta, \gamma \in (0,1)$
\STATE Construct $\mathbf{C}$ as in Eq.~(\ref{eq:GPOP}).
\REPEAT
    \STATE set $\mathbf{S}^k = [\mathbf{I}_d; \widetilde{\mathbf{S}}^k] \in \pi^{-1}(\widetilde{\mathbf{S}}^k) \subset \mathbb{O}(d)^m$ (Eq.~\ref{eq:pi_inv_wtS}).
    \STATE calculate the descent direction $-\grad F(\mathbf{S}^k)$ at $\mathbf{S}^k$ using Eq.~(\ref{eq:gradFS}).
    \STATE calculate the step size $\alpha_k$ according to the Armijo-type rule (see Eq.~(\ref{eq:armijo_step})).
    \STATE set $\widetilde{\mathbf{S}}^{k+1} = \pi(R_{\EXP}(\mathbf{S}^k, -\alpha_k \grad F(\mathbf{S}^k)))$ using Eq.~(\ref{eq:R_PF}, \ref{eq:pi}).
    \STATE $k \leftarrow k + 1$.
\UNTIL{convergence.}
\end{algorithmic}
\end{algorithm}
\subsection{Local linear Convergence of RGD}
\label{subsec:loc_lin_conv}
% We proceed to show the local linear convergence of Algorithm~\ref{algo:rgd} to a non-degenerate alignment (see Definition~\ref{def:non_deg_alignment0}). First, using the convergence analysis framework presented in \citea{schneider2015convergence} and as used in \citea{liu2019quadratic}, we show that if the sequence of iterates $\{\widetilde{\mathbf{S}}^k\}_{k\geq 0}$ generated by Algorithm~\ref{algo:rgd} converges to a non-degenerate $\widetilde{\mathbf{S}}^*$ then the convergence is linear. Then, using \citeb[Theorem~4.2]{udriste2013convex} and Proposition~\ref{prop:HessVicinity} we obtain an estimate of the radius and rate of convergence.
We proceed to show the local linear convergence of Algorithm~\ref{algo:rgd} to a non-degenerate alignment using the convergence analysis framework presented in \citea{schneider2015convergence} and as used in \citea{liu2019quadratic}.
To this end, we note that $\widetilde{F}$ and $F$ are real-analytic functions bounded from below by zero. $\mathbb{O}(d)^m/_{\sim}$  (whose elements are identified with $\mathbb{O}(d)^{m-1}$ here) and $\mathbb{O}(d)^m$ are compact submanifolds of $\mathbb{R}^{(m-1)d \times d}$ and $\mathbb{R}^{md \times d}$, respectively.
However, since $F(\mathbf{S}\mathbf{Q}) = F(\mathbf{S})$ for all $\mathbf{Q} \in \mathbb{O}(d)$, therefore every critical point of $F$ is degenerate and in particular $F$ is not a Morse-function \citea{cohen_iga_norbury_2006}. Nevertheless, if $\mathbf{S}^*$ is a non-degenerate alignment then $\widetilde{\mathbf{S}}^* = \pi(\mathbf{S}^*)$ is a non-degenerate critical point of $\widetilde{F}$. As a result $\widetilde{F}$ is a Morse function at $\widetilde{\mathbf{S}}^*$ and, due to \citeb[Proposition 4.2]{hu2018convergence}, the Lojasiewicz gradient inequality is satisfied.
\begin{prop}
Let $\mathbf{S}^*$ be a non-degenerate alignment and define $\widetilde{\mathbf{S}}^* = \pi(\mathbf{S}^*)$. Then there exist $\delta, \eta > 0$ such that $$|\widetilde{F}(\widetilde{\mathbf{S}}) - \widetilde{F}(\widetilde{\mathbf{S}}^*)| \leq \eta \left\| \grad \widetilde{F}(\widetilde{\mathbf{S}})\right\|_F^2$$
% \begin{equation}
%     |\widetilde{F}(\widetilde{\mathbf{S}}) - \widetilde{F}(\widetilde{\mathbf{S}}^*)| \leq \eta \left\| \grad \widetilde{F}(\widetilde{\mathbf{S}})\right\|_F^2. \label{eq:Lojasiewicz_gradient_ineq_half}
% \end{equation}
holds for every $\widetilde{\mathbf{S}} \in \mathbb{O}(d)^m/_{\sim}$ satisfying $\left\|\widetilde{\mathbf{S}}-\widetilde{\mathbf{S}}^*\right\|_F < \delta$.
\end{prop}
% Thus, the Lojasiewicz gradient inequality \citea{lojasiewicz1965ensembles}, \citeb[Section 2.2]{schneider2015convergence} holds at every $\widetilde{\mathbf{S}}^* \in \mathbb{O}(d)^m/_{\sim}$ and in particular for every $\widetilde{\mathbf{S}}^* \in \widetilde{\mathcal{C}}$ (see Eq.~(\ref{eq:crit_pts})) i.e. there exist $\delta, \eta > 0$ and $\theta \in (0,1/2]$ (generally dependent on $\widetilde{\mathbf{S}}^*$) such that
% \begin{equation}
%     |\widetilde{F}(\widetilde{\mathbf{S}}) - \widetilde{F}(\widetilde{\mathbf{S}}^*)|^{1-\theta} \leq \eta \left\| \grad \widetilde{F}(\widetilde{\mathbf{S}})\right\|_F. \label{eq:Lojasiewicz_gradient_ineq}
% \end{equation}
% holds for every $\widetilde{\mathbf{S}} \in \mathbb{O}(d)^m/_{\sim}$ satisfying $\left\|\widetilde{\mathbf{S}}-\widetilde{\mathbf{S}}^*\right\|_F < \delta$.

Moreover, the iterates $\{\widetilde{\mathbf{S}}^k\}_{k \geq 0}$ generated by Algorithm~\ref{algo:rgd} satisfy the (\textbf{A1}) sufficient descent, (\textbf{A2}) stationarity and (\textbf{A3}) safeguard assumptions below. The proofs are in the appendix and we make use of the following results to prove them.
\begin{prop}
\label{prop:liu_pf}
For all $\mathbf{S}_i \in \mathbb{O}(d)$ and $\mathbf{Z}_i \in T_{\mathbf{S}_i}\mathbb{O}(d)$ satisfying $\left\|\mathbf{Z}_i\right\|_F \leq 1$,
$$\left\|\mathbf{S}_i\exp (\mathbf{S}_i^T\mathbf{Z}_i) - (\mathbf{S}_i + \mathbf{Z}_i)\right\|_F \leq (e-1)\left\|\mathbf{Z}_i\right\|_F^2.$$
\end{prop}
% \begin{prop}
% \label{prop:second_order_boundedness_of_RPF}
% For $\mathbf{S} \in \mathbb{O}(d)^m$ and $\boldsymbol{\xi} \in T_{\mathbf{S}}\mathbb{O}(d)^m$ satisfying $\left\|\boldsymbol{\xi}\right\|_F \leq 1$,
% \begin{enumerate}[label=(\alph*)]
%     \item $\left\|R_\EXP(\mathbf{S}, \boldsymbol{\xi}) - (\mathbf{S} + \boldsymbol{\xi})\right\|_F \leq (e-1)\left\|\boldsymbol{\xi}\right\|_F^2$ and
%     \item $\left\|R_\EXP(\mathbf{S}, \boldsymbol{\xi})(\mathbf{S}_1\exp (\mathbf{S}_1^T\boldsymbol{\xi}_1))^T - (\mathbf{S} + \boldsymbol{\xi})(\mathbf{S}_1 + \boldsymbol{\xi}_1)^T\right\|_F \leq 2(e-1)\sqrt{m}\left\|\boldsymbol{\xi}\right\|_F^2$.
% \end{enumerate}
% \end{prop}
\begin{prop}
\label{prop:second_order_boundedness_of_Rtilde}
For $\widetilde{\mathbf{S}} \in \mathbb{O}(d)^m/_{\sim}$ and $\widetilde{\mathbf{Z}} \in T_{\widetilde{\mathbf{S}}}\mathbb{O}(d)^m/_{\sim}$ satisfying  $\left\|\widetilde{\mathbf{Z}}\right\|_F \leq 1/2$,
\begin{enumerate}[leftmargin=*,label=(\alph*)]
    \item $\left\|R_\EXP(\mathbf{S}, \mathbf{Z}) - (\mathbf{S} + \mathbf{Z})\right\|_F \leq (e-1)\left\|\mathbf{Z}\right\|_F^2$ for any $\mathbf{S} \in \pi^{-1}(\widetilde{\mathbf{S}})$ and $\mathbf{Z} \in T_{\mathbf{S}}\mathbb{O}(d)^m$, the horizontal lift of $\widetilde{\mathbf{Z}}$ at $\mathbf{S}$. 
    \item $\left\|\widetilde{R}_\EXP(\widetilde{\mathbf{S}}, \widetilde{\mathbf{Z}}) - (\widetilde{\mathbf{S}} + \widetilde{\mathbf{Z}})\right\|_F \leq (e-1)\left\|\widetilde{\mathbf{Z}}\right\|_F^2$.
\end{enumerate}
\end{prop}
\begin{prop}
\label{prop:alpha_grad}
%$(\alpha_k)_{k \geq 0}$, $(\widetilde{\mathbf{S}}^k)_{k \geq 0}$ and $(\mathbf{S}^k)_{k \geq 0}$ satisfy 
$\lim \alpha_k \left\|\grad \widetilde{F}(\widetilde{\mathbf{S}}^k)\right\|_F = 0$ and $\lim \alpha_k \left\|\grad F(\mathbf{S}^k)\right\|_F = 0$.
\end{prop}

\noindent \textbf{(A1)}. \textit{(Sufficient Descent)} There exist $\kappa_0 > 0$ and $k_1 \in \mathbb{N}$ such that, the inequality $\widetilde{F}(\widetilde{\mathbf{S}}^{k+1}) - \widetilde{F}(\widetilde{\mathbf{S}}^k) \leq - \kappa_0 \left\|\grad \widetilde{F}(\widetilde{\mathbf{S}}^k)\right\|_F \cdot \left\|\widetilde{\mathbf{S}}^{k+1}-\widetilde{\mathbf{S}}^k\right\|_F$ holds for all $k \geq k_1$.
\smallskip

\noindent \textbf{(A2)}. \textit{(Stationarity)} There exist $k_2 \in \mathbb{N}$ such that for all $k \geq k_2$, if $\left\|\grad \widetilde{F}(\widetilde{\mathbf{S}}^k)\right\|_F = 0$ then $\widetilde{\mathbf{S}}^{k+1} = \widetilde{\mathbf{S}}^k$. The sequence $\{\widetilde{\mathbf{S}}^{k}\}_{k \geq 0}$ satisfies this trivially.
\smallskip

\noindent \textbf{(A3)}. \textit{(Safeguard)} There exist a constant $\mu > 0$ and $k_3 \in \mathbb{N}$ such that the inequality $\left\|\grad \widetilde{F}(\widetilde{\mathbf{S}}^k)\right\|_F \leq \mu \left\|\widetilde{\mathbf{S}}^{k+1}-\widetilde{\mathbf{S}}^k\right\|_F$ holds for all $k \geq k_3$.

Combined with Theorem 2.3 in \citea{schneider2015convergence} and the fact that $\mathbb{O}(d)^m/_{\sim}$ is compact (thus every sequence on it has a cluster point), we obtain the following result.
\begin{thm}
\label{thm:rgd_conv}
Let $\mathbf{S}^*$ be a non-degenerate alignment and $\widetilde{\mathbf{S}}^* = \pi(\mathbf{S}^*)$. If the sequence $\{\widetilde{\mathbf{S}}^k\}_{k \geq 0}$ due to Algorithm~\ref{algo:rgd} converges to $\widetilde{\mathbf{S}}^*$ then the convergence is linear.
\end{thm}

Finally, we obtain an estimate of the radius and rate of linear convergence. The proof follows directly from \citeb[Chapter 7, Theorem~4.2]{udriste2013convex} (here $d_{\widetilde{g}}$ is the geodesic distance induced by the metric $\widetilde{g}$ on $\mathbb{O}(d)^m/_\sim$ as defined in Proposition~\ref{prop:g_tilde}).
% \begin{thm}
% \label{thm:rgd_conv2}
% Let $\mathbf{S}^*$ be a non-degenerate alignment
% %i.e. $\mathbf{S}^* \in \mathcal{C}$ (Eq.~(\ref{eq:crit_pts2})) and $\lambda_{d(d-1)/2+1}(\mathbb{L}(\mathbf{S}^*)) > 0$, 
% and $\zeta \in (0,1)$ be fixed. Let $\lambda_{-}(\mathbf{S}^*)$, $\lambda_{+}(\mathbf{S}^*)$ and $\delta(\mathbf{S}^*)$ be as defined in Proposition~\ref{prop:HessVicinity} (Corollary~\ref{cor:HessVicinity}) for the noisy (noiseless) setting. Then Algorithm~\ref{algo:rgd} converges to $\widetilde{\mathbf{S}}^* = \pi(\mathbf{S}^*)$ linearly when initialized with $\widetilde{\mathbf{S}}^0 = \pi(\mathbf{S}^0)$ satisfying $\min_{\mathbf{Q}\in \mathbb{O}(d)}\left\|\mathbf{S}^0-\mathbf{S}^*\mathbf{Q}\right\|_F < \zeta\delta(\mathbf{S}^*)$. Moreover,
% \begin{align}
%     \widetilde{F}(\widetilde{\mathbf{S}}^k) - \widetilde{F}(\widetilde{\mathbf{S}}^*) &\leq q^{k}(\widetilde{F}(\widetilde{\mathbf{S}}^0)-\widetilde{F}(\widetilde{\mathbf{S}}^*))\label{eq:alignment_err_ratio}\\
%     %\left\|\widetilde{\mathbf{S}}^k-\widetilde{\mathbf{S}}^*\right\|_F &\leq C q^{(k-1)/2}
%     d_{\widetilde{g}}(\widetilde{\mathbf{S}}^k, \widetilde{\mathbf{S}}^*) &\leq C q^{(k-1)/2}
% \end{align}
% where $C > 0$ is a constant, $q = 1 - 2\gamma (1-\gamma)r(1+r) \in (0,1)$ and $r = \frac{(1-\zeta)\lambda_{-}(\mathbf{S}^*)}{\lambda_{+}(\mathbf{S}^*) +\zeta\lambda_{-}(\mathbf{S}^*)}$.
% \end{thm}
\begin{thm}
\label{thm:rgd_conv2}
Let $\mathbf{S}^*$ be a non-degenerate alignment
%i.e. $\mathbf{S}^* \in \mathcal{C}$ (Eq.~(\ref{eq:crit_pts2})) and $\lambda_{d(d-1)/2+1}(\mathbb{L}(\mathbf{S}^*)) > 0$, 
and $\zeta \in (0,1)$ be fixed. Let $\lambda_{-}(\mathbf{S}^*)$, $\lambda_{+}(\mathbf{S}^*)$ and $\delta(\mathbf{S}^*)$ be as defined in Proposition~\ref{prop:HessVicinity}. If  the initialization $\widetilde{\mathbf{S}}^0 = \pi(\mathbf{S}^0)$ of Algorithm~\ref{algo:rgd} and and the subsequent iterates $\widetilde{\mathbf{S}}^k = \pi(\mathbf{S}^k)$ generated by it satisfy $\min_{\mathbf{Q}\in \mathbb{O}(d)}\left\|\mathbf{S}^k-\mathbf{S}^*\mathbf{Q}\right\|_F < \min\left\{2,\frac{2}{\pi}\zeta\delta(\mathbf{S}^*)\right\}$, then the sequence $\{\widetilde{\mathbf{S}}^k\}_{k \geq 0}$ converges to $\widetilde{\mathbf{S}}^* = \pi(\mathbf{S}^*)$ linearly. Moreover,
\begin{align}
    \widetilde{F}(\widetilde{\mathbf{S}}^k) - \widetilde{F}(\widetilde{\mathbf{S}}^*) &\leq q^{k}(\widetilde{F}(\widetilde{\mathbf{S}}^0)-\widetilde{F}(\widetilde{\mathbf{S}}^*))\label{eq:alignment_err_ratio}\\
    %\left\|\widetilde{\mathbf{S}}^k-\widetilde{\mathbf{S}}^*\right\|_F &\leq C q^{(k-1)/2}
    d_{\widetilde{g}}(\widetilde{\mathbf{S}}^k, \widetilde{\mathbf{S}}^*) &\leq C q^{(k-1)/2}
\end{align}
where $C > 0$ is a constant, $q = 1 - 2\gamma (1-\gamma) r(1+r)\in (0,1)$ and $r = \frac{(1-\zeta)\lambda_{-}(\mathbf{S}^*)}{\lambda_{+}(\mathbf{S}^*) +\zeta\lambda_{-}(\mathbf{S}^*)}$.
\end{thm}

\subsection{Exact Recovery and Noise Stability}
\label{subsec:noise_stability}
A direct consequence of Theorem~\ref{thm:rgd_conv2} and Corollary~\ref{cor:HessVicinity} to the noiseless setting is that RGD converges locally linearly to a perfect alignment under a condition weaker than affine rigidity.
\begin{thm}
\label{thm:exact_recovery}
Suppose $\mathbf{S}^*$ is a perfect alignment. Then Algorithm~\ref{algo:rgd} converges locally linearly to $\widetilde{\mathbf{S}}^* = \pi(\mathbf{S}^*)$ if any of the following holds:
\begin{enumerate}
    \item $\rank(\mathbf{C}) = (m-1)d$.
    \item $\rank(\mathbb{L}(\mathbf{S}^*))  = (m-1)d(d-1)/2$.
\end{enumerate}
The first condition which characterizes affine rigidity implies the second that characterizes infinitesimal rigidity as well as generic local rigidity (Figure~\ref{fig:rigidity_flow}).
\end{thm}
% \begin{cor}
% \label{cor:rgd_conv3}
% Let $\mathbf{S}^*$ be a non-degenerate perfect alignment
% %i.e. $\mathbf{S}^* \in \mathcal{C}$ (Eq.~(\ref{eq:crit_pts2})) and $\lambda_{d(d-1)/2+1}(\mathbb{L}(\mathbf{S}^*)) > 0$, 
% and $\zeta \in (0,1)$ be fixed. Let $\delta_0(\mathbf{S}^*)$ be as defined in Corollary~\ref{cor:HessVicinity}. Then Algorithm~\ref{algo:rgd} converges to $\widetilde{\mathbf{S}}^* = \pi(\mathbf{S}^*)$ linearly when initialized with $\widetilde{\mathbf{S}}^0 = \pi(\mathbf{S}^0)$ that satisfies $\min_{\mathbf{Q}\in \mathbb{O}(d)}\left\|\mathbf{S}^0-\mathbf{S}^*\mathbf{Q}\right\|_F < \zeta\delta_0(\mathbf{S}^*)$. 
% % Moreover, $\widetilde{F}(\widetilde{\mathbf{S}}^k) \leq q^{k}\widetilde{F}(\widetilde{\mathbf{S}}^0)$ and $d_{\widetilde{g}}(\widetilde{\mathbf{S}}^k, \widetilde{\mathbf{S}}^*) \leq C q^{(k-1)/2}$
% % % \begin{align}
% % %     \widetilde{F}(\widetilde{\mathbf{S}}^k) &\leq q^{k}\widetilde{F}(\widetilde{\mathbf{S}}^0)\label{eq:alignment_err_ratio2}\\
% % %     %\left\|\widetilde{\mathbf{S}}^k-\widetilde{\mathbf{S}}^*\right\|_F &\leq C q^{(k-1)/2}
% % %     d_{\widetilde{g}}(\widetilde{\mathbf{S}}^k, \widetilde{\mathbf{S}}^*) &\leq C q^{(k-1)/2}
% % % \end{align}
% % where $C > 0$ is a constant, $q = 1 - 2\gamma (1-\gamma)r(1+r) \in (0,1)$ and $r = \frac{(1-\zeta)\lambda_{0_-}(\mathbf{S}^*)}{\lambda_{0_+}(\mathbf{S}^*) +\zeta\lambda_{0_-}(\mathbf{S}^*)}$.
% \end{cor}

While RGD achieves local linear convergence under less restrictive conditions, a key challenge lies in selecting an initial alignment that is sufficiently close to a non-degenerate perfect alignment $\mathbf{S}^*$. In \citea{chaudhury2015global}, the authors showed that under the affine rigidity constraints, SPEC recovers the perfect alignment, eliminating the need for RGD. However, under weaker non-degeneracy (equivalently, infinitesimal/local rigidity) constraints, it is yet to be established whether the output of SPEC recovers/remains close to a perfect alignment of noiseless views (similarly, to an optimal alignment of the noisy views). We aim to address this in our future work.

Nevertheless, under the bounded noise model and affine rigidity constraints, the spectral solution $\mathbf{S}_{spec}$ has been shown to approximate a perfect alignment $\mathbf{S}_0$ of the noiseless counterparts of the noisy views \citea{chaudhury2015global}. While $\mathbf{S}_0$ is generally not the optimal alignment $\mathbf{S}^*$ of the noisy views, one can expect them to be relatively close. Therefore, under affine rigidity constraints, we expect $\mathbf{S}_{spec}$ to be near the optimal alignment $\mathbf{S}^*$ of the noisy views and thus, refining $\mathbf{S}_{spec}$ using RGD could potentially yield $\mathbf{S}^*$. Here, we provide a noise stability analysis of RGD which support this idea.
%A more challenging problem is to develop a new initialization method for RGD or prove that the output of SPEC remains close to an optimal alignment under weaker infinitesimal/local rigidity constraints. We aim to address this in future work.

% Although the Algorithm~\ref{algo:rgd} enjoys local linear convergence under weaker constraints than those adopted in previous works, one limitation is how to choose an initial alignment $\widetilde{\mathbf{S}}^{0}$ so that it is close to the optimal alignment $\widetilde{\mathbf{S}}^*$. Previous works have shown that alignment obtained due to the spectral relaxation of the problem tend to be close to noiseless optimal alignment under affine rigidity constraints \citea{chaudhury2015global}. This means, under such stronger constraints one can expect the RGD algorithm initialized with spectral alignment to converge to a better if not optimal alignment. We provide a noise stability analysis of Algorithm~\ref{algo:rgd} that is consistent with the above idea. A more challenging problem is to devise a procedure to initialize RGD or show that the solution of spectral relaxation is close to a \textit{nice} alignment under weaker local rigidity constraints. We hope to deal with this in our future work.

%Here we investigate the noise stability of Algorithm~\ref{algo:rgd} when initialized with SPEC \citea{chaudhury2015global}.
We start with a set of noiseless views and inject them with bounded noise i.e.  $\mathbf{x}_{k,i} \leftarrow \mathbf{x}_{k,i} + \boldsymbol{\epsilon}_{k,i}$ where $\left\|\boldsymbol{\epsilon}_{k,i}\right\|_2 \leq \varepsilon$ for a fixed noise level $\varepsilon > 0$. Let $\mathbf{C}_0$ and $\mathbf{C}$ be the patch-stress matrices (Eq.~(\ref{eq:GPOP})) corresponding to the noiseless views and their noisy counterparts, respectively.
% Then the following lemma bounds the difference in the optimization landscapes in terms of the noise level.
% \begin{lem}
% \label{lem:landascape_diff}
% Let $\mathbf{S} \in \mathbb{O}(d)^m$ be a fixed alignment. Then,
% \begin{equation}
%     |\Tr(\mathbf{C}\mathbf{S}\mathbf{S}^T) - \Tr(\mathbf{C}_0\mathbf{S}\mathbf{S}^T)| \leq m\sqrt{d}(k_1\varepsilon + k_2\varepsilon^2)
% \end{equation}
% where the constants $k_1$ and $k_2$ are
% \begin{align}
%     k_1 &= 2\sqrt{n |E(\Gamma)|}\left(4 \max_1^n\left\|\mathbf{x}_k\right\|_2\frac{\sqrt{n|E(\Gamma)|}}{\lambda_{2}(\mathbf{\mathcal{L}}_{\Gamma})} + 1\right)\\
%     k_2 &= 2\sqrt{n |E(\Gamma)|}\left(2\frac{\sqrt{n|E(\Gamma)|}}{\lambda_{2}(\mathbf{\mathcal{L}}_{\Gamma})} + 1\right).
% \end{align}
% \end{lem}
% The proof directly follows from Cauchy-Schwarz inequality, the fact that $\left\|\mathbf{S}\mathbf{S}^T\right\|_F = m\sqrt{d}$ and the $\left\|\mathbf{C}-\mathbf{C}_0\right\|_F \leq k_1\varepsilon + k_2\varepsilon^2$ provided in \citeb[Eq.(5.12)]{chaudhury2015global}. Also note that $\lambda_2(\mathbf{\mathcal{L}}_{\Gamma}) > 0$ due to Assumption~\ref{assump:connected_gamma}.
Then the following lemma establishes a quadratic growth condition at an optimal alignment in the noiseless setting. The subsequent lemma bounds the distance between the optimal alignments of noisy and noiseless views.
\begin{lem}
\label{lem:quadgrowth}
Let $\rank(\mathbf{C}_0) = (m-1)d$, and consequently $\mathbf{S}_0$ be a unique perfect alignment of the noiseless views (Figure~\ref{fig:rigidity_flow}). Then,
\begin{equation}
    \Tr(\mathbf{C}_0\mathbf{S}\mathbf{S}^T) \geq \frac{\lambda_{d+1}(\mathbf{C}_0)}{2} \min_{\mathbf{Q} \in \mathbb{O}(d)}\left\|\mathbf{S}- \mathbf{S}_0\mathbf{Q}\right\|_F^2.
\end{equation}
% Consequently, for all $\mathbf{S} \in \mathbb{O}(d)^m$ such that
% \begin{equation}
%     \min_{\mathbf{Q} \in \mathbb{O}(d)^m} \left\|\mathbf{S}- \mathbf{S}_0\mathbf{Q}\right\|_F > \sqrt{\frac{4m\sqrt{d}(k_1\varepsilon + k_2\varepsilon^2)}{\lambda_{d+1}(\mathbf{C}_0)}}
% \end{equation}
% we have $\Tr(\mathbf{C}_0\mathbf{S}\mathbf{S}) > 2m\sqrt{d}(k_1\varepsilon + k_2\varepsilon^2)$.
\end{lem}
\begin{lem}
\label{lem:distS_0Sstar}
Let $\rank(\mathbf{C}_0) = (m-1)d$ and $\mathbf{S}_0$ be a unique perfect alignment of the noiseless views. Let $\mathbf{S}^*$ be an optimal alignment of the noisy views. Then
\begin{equation}
    \min_{\mathbf{Q}\in\mathbb{O}(d)}\left\|\mathbf{S}^* - \mathbf{S}_0\mathbf{Q}\right\|_F \leq \frac{4 m \left\|\mathbf{C}-\mathbf{C}_0\right\|_F}{\lambda_{d+1}(\mathbf{C}_0)}.
\end{equation}
\end{lem}

Finally, we obtain a bound on the noise level for RGD, initialized with  the spectral alignment, to converge locally linearly to the optimal alignment of the noisy views. 
\begin{thm}
\label{thm:rgd_noise_stability}
Let $\rank(\mathbf{C}_0) = (m-1)d$ and $\mathbf{S}_0$ be a unique perfect alignment of the noiseless views. Let $\mathbf{S}^*$ be an optimal alignment of the noisy views and suppose $\mathbf{S}^*$ is non-degenerate. Let $\zeta \in (0,1)$ be fixed. Then Algorithm~\ref{algo:rgd}, initialized with $\pi(\mathbf{S}_{spec}(\mathbf{C}))$, converges locally linearly to $\pi(\mathbf{S}^*)$ if the noise level $\varepsilon$ satisfies
\begin{equation}
    4\sqrt{m}\left(\frac{\pi\sqrt{d(d+1)}}{\lambda_{d+1}(\mathbf{C})} + \frac{\sqrt{m}}{\lambda_{d+1}(\mathbf{C}_0)}\right)(K_1 \varepsilon + K_2\varepsilon^2) < \min\left\{2, \frac{2}{\pi}\zeta\delta(\mathbf{S}^*)\right\}
\end{equation}
and the subsequent iterates satisfy $\min_{\mathbf{Q}\in \mathbb{O}(d)}\left\|\mathbf{S}^k-\mathbf{S}^*\mathbf{Q}\right\|_F < \min\left\{2, \frac{2}{\pi}\zeta\delta(\mathbf{S}^*)\right\}$.
Here,
\begin{align}
    K_1 &= 2\sqrt{n |E(\Gamma)|}\left(4 \max_1^n\left\|\mathbf{x}_k^*\right\|_2\frac{\sqrt{n|E(\Gamma)|}}{\lambda_{2}(\mathbf{\mathcal{L}}_{\Gamma})} + 1\right)\\
    K_2 &= 2\sqrt{n |E(\Gamma)|}\left(2\frac{\sqrt{n|E(\Gamma)|}}{\lambda_{2}(\mathbf{\mathcal{L}}_{\Gamma})} + 1\right),
\end{align}
$\mathbf{x}_k^*$ is the realization of the noiseless views due to $\mathbf{S}_0$ and $\delta(\mathbf{S}^*)$ is defined in Proposition~\ref{prop:HessVicinity}.
\end{thm}
In the result above, we assumed $\lambda_{d+1}(\mathbf{C}) > 0$, as was the case in \citea{chaudhury2015global}, where the authors observed (and as we validate below) that $\lambda_{d+1}(\mathbf{C})$ increases with noise level $\varepsilon$.
\begin{figure}
    \centering
    \begin{tabular}{cc}
       \includegraphics[width=0.46\linewidth,keepaspectratio]{fig/fig0/rgd_convergence3_new.pdf}
      &  \includegraphics[width=0.46\linewidth,keepaspectratio]{fig/fig0/rgd_convergence2_new.pdf}
    \end{tabular}
    \caption{(left) The eigenvalue $\lambda_{d+1}(\mathbf{C})$ against the noise levels. (right) The evolution of the ratio (Eq.~(\ref{eq:alignment_err_ratio})) due to the iterates generated by Algorithm~\ref{algo:rgd} when initialized with the output of SPEC \citea{chaudhury2015global}.}
    \label{fig:rgd_iterates}
\end{figure}

To demonstrate, we provide a simple simulation showing that when Algorithm~\ref{algo:rgd} is initialized with SPEC, it produce iterates with lower alignment error. We took about $n=5000$ points arranged in a unit square grid with a resolution of $70$ points per dimension and subsequently obtained $m=331$ overlapping views. We added random bounded noise in each view for a fixed noise level $\varepsilon$, obtained the corresponding patch-stress matrix $\mathbf{C}$, and computed the spectral alignment $\mathbf{S}_{spec}(\mathbf{C})$ of the noisy views. Finally, we refined it using Algorithm~\ref{algo:rgd} for $100$ iterations. Figure~\ref{fig:rgd_iterates} shows the eigenvalue $\lambda_{d+1}(\mathbf{C})$ against the noise level $\varepsilon$ and confirms the observation in \cite{chaudhury2015global} that the eigenvalue increases with the noise level. Figure~\ref{fig:rgd_iterates} also shows the ratio of
%$\frac{\widetilde{F}(\widetilde{\mathbf{S}}^k)- \widetilde{F}(\widetilde{\mathbf{S}}^*)}{\widetilde{F}(\widetilde{\mathbf{S}}^0) - \widetilde{F}(\widetilde{\mathbf{S}}^*)}$ for the alignments $\widetilde{\mathbf{S}}^k$
$\widetilde{F}(\widetilde{\mathbf{S}}^k)- \widetilde{F}(\widetilde{\mathbf{S}}^*)$ and $\widetilde{F}(\widetilde{\mathbf{S}}^0) - \widetilde{F}(\widetilde{\mathbf{S}}^*)$ for the alignments $\widetilde{\mathbf{S}}^k$ produced by RGD. The evolution of the ratio is consistent with the linear convergence predicted by Theorem~\ref{thm:rgd_conv2}.
% Finally, we state a sufficient condition on the structure of the noiseless local views that enables local linear convergence of RGD. This condition is stronger than the non-degeneracy condition.
% \begin{cor}
% \label{cor:G_conv}
% If the local views are noiseless and $\mathbb{G}$ is connected then RGD converges locally linearly to a perfect alignment.
% \end{cor}
% Noise stability analysis:
% \begin{thm}
% If $\mathbf{S}^*$ is an alignment such that $\mathbf{S}^* \in \mathcal{C}$, $\mathbb{L}(\mathbf{S}^*)$ is positive semi-definite and of rank $(m-1)d(d-1)/2$, then Algorithm~\ref{algo:rgd} converges locally linearly to $\widetilde{\mathbf{S}}^* = \pi(\mathbf{S}^*)$.
% \end{thm}

\section{Discussion}
\label{sec:conc}
\section{Conclusions}
\label{secConc}

We introduce a simple procedure to control for false discoveries and identify individual signals in scenarios involving many tests, dependent test statistics, and  potentially sparse signals.
			The tool is agnostic to the underlying dependence structure and scalable to deal with high dimensions.
			 Our approach is a sequential version of the global Cauchy combination test proposed by  \cite{liu2020cauchy}. 
			 By applying the global test recursively on a sequence of expanding subsets of ordered $p$-values, 
			 our sequential Cauchy combination test enables the identification of individual violations.
 
			
We show that the sequential Cauchy combination test achieves strong familywise error rate control and is less conservative compared to  popular statistical inequality-based methods (such as the Bonferroni correction and  subsequent improvements of \citeauthor{holm1979simple}, \citeyear{holm1979simple},  \citeauthor{hommel1988stagewise}, 
\citeyear{hommel1988stagewise} and   \citeauthor{hochberg1988sharper}, 
\citeyear{hochberg1988sharper}) and the Gumbel method.

The Cauchy transformation has proven its value in a genome-wide association study of Crohn's disease \citep[][Section 4.3]{liu2020cauchy}, but its applicability extends beyond genomics. 
We revisit two important 
needle-in-a-haystack problems 
in financial econometrics, where the test statistics have either serial or
cross-sectional dependence:  monitoring drift bursts and searching for nonzero alpha assets. 
The drift burst test of \citet{christensen2018drift} detects the presence of explosive trends in asset prices using  high-frequency intraday data. The test statistics are computed from overlapping windows, resulting in high autocorrelation. 
We  also revisit the 
\citet{fama2015five}
multi-factor model to identify nonzero alpha financial assets. Detecting these rare nonzero alphas among a large group of financial assets is challenging, especially when the test statistics are likely to be cross-sectionally correlated.
Without a proper controlling procedure, one might flag false discoveries or miss important signals. Our results indicate that the sequential Cauchy combination test is the a preferable method for both applications.


We emphasize that our sequential Cauchy combination test is not limited to  financial econometrics. 
We anticipate its applicability to  a wide range of hypothesis tests in fields such as economics, finance, medicine, marketing and climate studies, as it can handle various types of dependence effectively. 



\section*{Acknowledgements}
We thank Lijun Ding for providing references which assisted the noise stability analysis. DK was partly supported by a grant from Kavli Institute for Brain and Mind (UCSD). GM was partially funded by NSF CCF-2217058. AC was partially funded by NSF DMS-2012266 and a gift from Intel Research.

\appendix
\section{Notation and Proofs}
\label{supp:sec:all_proofs}
$[a,b]$ is the set $\{a,\ldots,b\}$ where $a,b \in \mathbb{Z}$.
$(a_i)_1^k$ is the sequence $a_1,\ldots,a_k$ where $a_i$ is either a scalar or a vector or a matrix.
$\mathbf{e}^p_q$ is a vector of zeros of length $p$ with $1$ at the $q$th location.
$\mathbf{1}^{p}_q$ is a vector of zeros of length $p$ whose first $q$ elements are $1$s.
$\mathbf{1}_p$ equals $\mathbf{1}^{p}_p$.
$\mathbf{0}_p$ and $\mathbf{0}_{m \times n}$ is a vector of length $p$ and a matrix of zeros with $m$ rows and $n$ columns, respectively.
$[\mathbf{A}_i]_1^n$ denotes a matrix obtained by vertically stacking the matrices $(\mathbf{A}_i)_1^m$.
$[\mathbf{A}]_1^n$ equals $[\mathbf{A}_i]_1^n$ where $\mathbf{A}_i = \mathbf{A}$ for all $i \in [1,n]$.
$\mathbf{I}_d$ and $\mathbf{I}^m_d$ denotes the identity matrix of size $d$ and $[\mathbf{I}_d]_1^m$, respectively.
$\mathbf{A}_i$ is the $i$th row block of $\mathbf{A}$ (the dimensions are contextual).
$\mathbf{A}_{ij}$ is the $(i,j)$th block of the block matrix $\mathbf{A}$ (the dimensions are contextual).
$\vecz (\mathbf{A})$ denotes the column-major vectorization of the matrix $\mathbf{A}$.
$\blockdiag((\mathbf{A}_i)_1^m)$ is a block diagonal matrix with $\mathbf{A}_i$ as the $i$th block.
$\diag (\mathbf{v})$ is a diagonal matrix with $\mathbf{v}(i)$ as the $i$th diagonal element.
%$\mathbf{A}(k,:)$ ($\mathbf{A}(:,k)$) denotes $k$th row (column) of $\mathbf{A}$.
$\mathbf{A}(i:j,:) (\mathbf{A}(:,i:j))$ denotes a stacking of $i$th to $j$th rows (columns) of $\mathbf{A}$.
$\mathbb{O}(n)$ is the set of orthogonal matrices of size $n$.
$\Sym (n)$ and $\Skew (n)$ is the set of symmetric and skew-symmetric matrices of size $n$, respectively.
$\Sym(\mathbf{A})$ and $\Skew(\mathbf{A})$ equals $(\mathbf{A}+\mathbf{A}^T)/2$ and $(\mathbf{A}-\mathbf{A}^T)/2$, respectively.
$\mathbf{A} \otimes \mathbf{B}$ denotes the Kronecker product of $\mathbf{A}$ and $\mathbf{B}$. \revadd{For a symmetric matrix $\mathbf{A}$, $\lambda_{i}(\mathbf{A})$ denotes the $i$th smallest eigenvalue of $\mathbf{A}$ and $\sigma_{\min}(\mathbf{A})$ ($\sigma_{\max}(\mathbf{A})$) denotes the smallest (largest) singular value of $\mathbf{A}$.}
% \begin{longtable}{|l|l|}
%     \hline
%     $[a,b]$& the set $\{a,\ldots,b\}$ where $a,b \in \mathbb{Z}$\\
%     $(a_i)_1^k$& the sequence $a_1,\ldots,a_k$ where $a_i$ is a scalar/vector/matrix\\
%     $\mathbf{e}^p_q$& a vector of zeros of length $p$ with $1$ at the $q$th location\\
%     $\mathbf{1}^{p}_q$& a vector of zeros of length $p$ whose first $q$ elements are $1$s\\
%     $\mathbf{1}_p$& $\mathbf{1}^{p}_p$\\
%     $\mathbf{0}_p, \mathbf{0}_{m \times n}$ & a vector and a matrix of zeros\\
%     $[\mathbf{A}_i]_1^n$& a matrix obtained by vertically stacking the matrices $(\mathbf{A}_i)_1^m$\\
%     $[\mathbf{A}]_1^n$& $[\mathbf{A}_i]_1^n$ where $\mathbf{A}_i = \mathbf{A}$ for all $i \in [1,n]$\\
%     $\mathbf{I}_d$, $\mathbf{I}^m_d$ & the identity matrix of size $d$ and $[\mathbf{I}_d]_1^m$\\
%     $\mathbf{A}_i$ & $i$th row block of $\mathbf{A}$$^\sharp$\\
%     $\mathbf{A}_{ij}$ & $(i,j)$th block of the block matrix $\mathbf{A}$$^\sharp$\\
%     $\vecz (\mathbf{A})$ & the column-major vectorization of the matrix $\mathbf{A}$\\
%     $\blockdiag((\mathbf{A}_i)_1^m)$ & a block diagonal matrix with $\mathbf{A}_i$ as the $i$th block\\
%     $\diag (\mathbf{v})$ & a diagonal matrix with $\mathbf{v}(i)$ as the $i$th diagonal element\\
%     $\mathbf{A}(k,:)$ ($\mathbf{A}(:,k)$) & $k$th row (column) of $\mathbf{A}$\\
%     $\mathbf{A}(i:j,:) (\mathbf{A}(:,i:j))$ & a stacking of $i$th to $j$th rows (columns) of $\mathbf{A}$\\
%     $\mathbb{O}(n)$ & the set of orthogonal matrices of size $n$\\
%     $\Sym (n)$, $\Skew (n)$, & the set of symmetric and skew-symmetric matrices of size $n$\\
%     $\Sym(\mathbf{A})$, $\Skew(\mathbf{A})$ & $(\mathbf{A}+\mathbf{A}^T)/2$, $(\mathbf{A}-\mathbf{A}^T)/2$\\
%     $\mathbf{A} \otimes \mathbf{B}$ & the Kronecker product of $\mathbf{A}$ and $\mathbf{B}$\\
%     \hline
%     \multicolumn{2}{l}{\footnotesize{$^\sharp$Dimensions are contextual.}}\\
%     %\caption{Notation.}
%     % \label{supp:tab:notation}
% \end{longtable}

%%%%%%%%%%%%%%%%%%%%%%%%%%%%%%%%%%%%%%%%%%
% Section 2 Proofs
%%%%%%%%%%%%%%%%%%%%%%%%%%%%%%%%%%%%%%%%%%
\proofof{Proposition~\ref{prop:A_0_2_A_1}}
By differentiating the objective in Eq.~(\ref{eq:A_1}) with respect to $\mathbf{x}_k$, the optimal $\mathbf{x}_k^* \coloneqq \mathbf{x}_k((\mathbf{S}_i)_1^m,(\mathbf{t}_i)_1^m) = n_i^{-1}\textstyle\sum_{(k,i)\in E(\Gamma)}(\mathbf{S}_i^T\mathbf{x}_{k,i}+\mathbf{t}_i)$ (where $n_i = |\{i:(k,i)\in E(\Gamma)\}|$ is the number of points in the $i$th view) which is the consensus of all the views for the $k$th point. Since $\textstyle\sum_{(k,i)\in E(\Gamma)}(\mathbf{S}_i^T\mathbf{x}_{k,i}+\mathbf{t}_i-\mathbf{x}_k^*) = 0$, by adding and subtracting $\mathbf{x}_k^*$, we conclude that 
$$\textstyle\sum_{\substack{(k,i)\in E(\Gamma)\\(k,j)\in E(\Gamma)}}\left\|(\mathbf{S}_i^T\mathbf{x}_{k,i}+\mathbf{t}_i)-(\mathbf{S}_j^T\mathbf{x}_{k,j}+\mathbf{t}_j)\right\|^2_2 = 2\textstyle\sum_{(k,i)\in E(\Gamma)}\left\|(\mathbf{S}_i^T\mathbf{x}_{k,i}+\mathbf{t}_i)-\mathbf{x}_k^*\right\|^2_2$$
% \begin{equation}
%     \textstyle\sum_{\substack{(k,i)\in E(\Gamma)\\(k,j)\in E(\Gamma)}}\left\|(\mathbf{S}_i^T\mathbf{x}_{k,i}+\mathbf{t}_i)-(\mathbf{S}_j^T\mathbf{x}_{k,j}+\mathbf{t}_j)\right\|^2_2 = 2\textstyle\sum_{(k,i)\in E(\Gamma)}\left\|(\mathbf{S}_i^T\mathbf{x}_{k,i}+\mathbf{t}_i)-\mathbf{x}_k^*\right\|^2_2
% \end{equation}
and the result follows. The above equality also shows that the minimizers
%$(\mathbf{S}_i)_1^m$ and $(\mathbf{t}_i)_1^m$
of Eq.~(\ref{eq:A_0}) and Eq.~(\ref{eq:A_1}) are the same.

\proofof{Proposition~\ref{prop:kerB}}
Let $\mathbf{u} \in \mathbb{R}^{n+m}$. Since $\boldsymbol{\mathcal{L}}_{\Gamma} \succeq 0$, $\boldsymbol{\mathcal{L}}_{\Gamma}\mathbf{u} = 0 \iff \mathbf{u}^T\boldsymbol{\mathcal{L}}_{\Gamma}\mathbf{u} = 0$. From Eq.~(\ref{eq:L_Gamma}), the latter holds iff $\mathbf{e}_{ki}^T\mathbf{u} = 0$ for all $(k,i) \in E(\Gamma)$. Thus, using Eq.~(\ref{eq:B}), $\boldsymbol{\mathcal{L}}_{\Gamma}\mathbf{u} = 0$ implies $\mathbf{B}\mathbf{u} = 0$. 

% %%%%%%%%%%%%%%%%%%%%%%%%%%%%%%%%%%%%%%%%%%
% % Section 3 Proofs
% %%%%%%%%%%%%%%%%%%%%%%%%%%%%%%%%%%%%%%%%%%
\proofoffirst{Proposition~\ref{prop:V_S_H_S}}
Using Eq.~(\ref{eq:pi_inv_wtS}),
\begin{align}
    \mathcal{V}_{\mathbf{S}} &= T_{\mathbf{S}}\pi^{-1}(\widetilde{\mathbf{S}}) = \{\mathbf{Z} \in \mathbb{R}^{md \times d}: \mathbf{Z}_i\mathbf{S}_1^T+\mathbf{S}_i\mathbf{Z}_1^T = 0, \mathbf{Z}_j\mathbf{S}_j^T+\mathbf{S}_j\mathbf{Z}_j^T=0, i \in [2,m], j \in [1,m]\}\\
    &= \{[\mathbf{S}_i\boldsymbol{\Omega}_i]_1^m: \boldsymbol{\Omega}_i \in \Skew(d), \boldsymbol{\Omega}_i+\boldsymbol{\Omega}_1^T = 0, i \in [1,m]\}\\
    &= \{\mathbf{S}\boldsymbol{\Omega}: \boldsymbol{\Omega} \in \Skew(d)\}
\end{align}
% \begin{align}
%     \mathcal{V}_{\mathbf{S}} &= T_{\mathbf{S}}\pi^{-1}(\widetilde{\mathbf{S}})\\
%     &= \{\mathbf{Z} \in \mathbb{R}^{md \times d}: \mathbf{Z}_i\mathbf{S}_1^T+\mathbf{S}_i\mathbf{Z}_1^T = 0, \mathbf{Z}_j\mathbf{S}_j^T+\mathbf{S}_j\mathbf{Z}_j^T=0, i \in [2,m], j \in [1,m]\}\\
%     &= \{[\mathbf{S}_i\boldsymbol{\Omega}_i]_1^m: \boldsymbol{\Omega}_i \in \Skew(d), \boldsymbol{\Omega}_i+\boldsymbol{\Omega}_1^T = 0, i \in [1,m]\} = \{\mathbf{S}\boldsymbol{\Omega}: \boldsymbol{\Omega} \in \Skew(d)\}.
% \end{align}

Since the objective in Eq.~(\ref{eq:P^v_S}) is strictly convex in $\boldsymbol{\Omega}$, it suffices to solve for the critical point and the Eq.~(\ref{eq:P^v_S}) follows immediately. Then, using Eq.~(\ref{eq:g_Z_W}), we obtain
\begin{align}
    \mathcal{H}_{\mathbf{S}} &= \mathcal{V}_{\mathbf{S}}^\perp = \{\mathbf{W} \in T_{\mathbf{S}}\mathbb{O}(d)^m: \Tr(\mathbf{Z}^T\mathbf{W}) = 0 \text{ for all } \mathbf{Z} \in \mathcal{V}_{\mathbf{S}}\}\\
    &= \left\{[\mathbf{S}_i\boldsymbol{\Omega}_i]_1^m: \boldsymbol{\Omega}_i \in \Skew(d) \text{ and }\Tr\left(\boldsymbol{\Omega}^T\left(\textstyle\sum_1^m\boldsymbol{\Omega}_i\right)\right) = 0 \text{ for all } \boldsymbol{\Omega}\in \Skew(d)\right\}.
\end{align}
% \begin{align}
%     \mathcal{H}_{\mathbf{S}} &= \mathcal{V}_{\mathbf{S}}^\perp = \{\mathbf{W} \in T_{\mathbf{S}}\mathbb{O}(d)^m: \Tr(\mathbf{Z}^T\mathbf{W}) = 0 \text{ for all } \mathbf{Z} \in \mathcal{V}_{\mathbf{S}}\}\\
%     &= \left\{[\mathbf{S}_i\boldsymbol{\Omega}_i]_1^m: \boldsymbol{\Omega}_i \in \Skew(d) \text{ and }\Tr\left(\boldsymbol{\Omega}^T\left(\textstyle\sum_1^m\boldsymbol{\Omega}_i\right)\right) = 0 \text{ for all } \boldsymbol{\Omega}\in \Skew(d)\right\}.
% \end{align}
The constraints on $\boldsymbol{\Omega}_i$ are equivalent to $\boldsymbol{\Omega}_i \in \Skew(d)$ and $ \textstyle\sum_1^m \boldsymbol{\Omega}_i \in \Sym(d)$, and subsequently to $\boldsymbol{\Omega}_i \in \Skew(d)$ and $\textstyle\sum_1^m \boldsymbol{\Omega}_i= 0$. Since $\mathcal{H}_{\mathbf{S}}$ is the orthogonal complement to $\mathcal{V}_{\mathbf{S}}$ in $T_{\mathbf{S}}\mathbb{O}(d)^m$, Eq.~(\ref{eq:P^h_S}) follows trivially.

\proofof{Proposition~\ref{prop:hlift_char}}
By Eq.~(\ref{eq:hlift_def}), we obtain, $\lim_{t \rightarrow 0}(\pi(\mathbf{S}+t\mathbf{Z})_i-\pi(\mathbf{S})_i)/t = \widetilde{\mathbf{Z}}_i$ for each $i \in [1,m-1]$ which further implies $\mathbf{S}_{i+1}\mathbf{Z}_1^T + \mathbf{Z}_{i+1}\mathbf{S}_1^T = \widetilde{\mathbf{Z}}_i$.
% \begin{align}
%     \lim_{t \rightarrow 0}(\pi(\mathbf{S}+t\mathbf{Z})_i-\pi(\mathbf{S})_i)/t = \widetilde{\mathbf{Z}}_i \implies \mathbf{S}_{i+1}\mathbf{Z}_1^T + \mathbf{Z}_{i+1}\mathbf{S}_1^T = \widetilde{\mathbf{Z}}_i. \label{supp:lifted_vec}
% \end{align}
Since $T_{\widetilde{\mathbf{S}}}\mathbb{O}(d)^m/_{\sim}$ is identified with $T_{\widetilde{\mathbf{S}}}\mathbb{O}(d)^{m-1}$, therefore there exist $(\widetilde{\boldsymbol{\Omega}}_i)_1^{m-1} \subseteq \Skew(d)$, such that $\widetilde{\mathbf{Z}}_i = \widetilde{\mathbf{S}}_i\widetilde{\boldsymbol{\Omega}}_i$. Also, since $\mathbf{Z} \in \mathcal{H}_{\mathbf{S}}$, there exist $(\boldsymbol{\Omega}_i)_1^m \subseteq \Skew(d)$ such that $\textstyle\sum_1^m \boldsymbol{\Omega}_i = 0$ and $\mathbf{Z}_i=\mathbf{S}_i\boldsymbol{\Omega}_i$. Substituting $\widetilde{\mathbf{Z}}_i = \widetilde{\mathbf{S}}_i\widetilde{\boldsymbol{\Omega}}_i$ and $\mathbf{Z}_i=\mathbf{S}_i\boldsymbol{\Omega}_i$,
%in the above equation
we obtain
\begin{equation}
    \mathbf{S}_{i+1}\boldsymbol{\Omega}_1^T\mathbf{S}_1^T + \mathbf{S}_{i+1}\boldsymbol{\Omega}_{i+1}\mathbf{S}_1^T = \widetilde{\mathbf{S}}_i\widetilde{\boldsymbol{\Omega}}_i \implies \boldsymbol{\Omega}_{i+1}-\boldsymbol{\Omega}_1 = \mathbf{S}_{1}^T\widetilde{\boldsymbol{\Omega}}_i\mathbf{S}_{1}, i \in [1,m-1], \label{supp:eq:eq1_}
\end{equation}
where we used the fact that $\widetilde{\mathbf{S}}_i = \mathbf{S}_{i+1}\mathbf{S}_1^T$. Observe that the linear system in $[\boldsymbol{\Omega}_i]_1^m$ is of full rank. By applying $\textstyle\sum_{i=1}^{m-1}$ and using $\textstyle\sum_1^m \mathbf{\Omega}_i = 0$ gives Eq.~(\ref{eq:hlift1}) and (\ref{eq:hlifti}).

\proofof{Proposition~\ref{prop:g_tilde}}
It suffices to show that $g(\mathbf{Z}, \mathbf{W})$ does not depend on the choice of $\mathbf{S} \in \pi^{-1}(\widetilde{\mathbf{S}})$. Let $\{\widetilde{\mathbf{U}}_i,\widetilde{\mathbf{V}}_i\}_1^{m-1},\{\mathbf{U}_i,\mathbf{V}_i\}_1^m$ be elements of $\Skew(d)$ such that $\widetilde{\mathbf{Z}}_i = \widetilde{\mathbf{S}}_i\widetilde{\mathbf{U}}_i$, $\widetilde{\mathbf{W}}_i = \widetilde{\mathbf{S}}_i\widetilde{\mathbf{V}}_i$, $\mathbf{Z}_i = \mathbf{S}_i\mathbf{U}_i$ and $\mathbf{W}_i = \mathbf{S}_i\mathbf{V}_i$. By the definition of $\mathcal{H}_{\mathbf{S}}$, $\textstyle\sum_1^m\mathbf{U}_i = \textstyle\sum_1^m \mathbf{V}_i = 0$ and the relation between $\mathbf{U}_i$ and $\widetilde{\mathbf{U}}_i$, and $\mathbf{V}_i$ and $\widetilde{\mathbf{V}}_i$ is given by Eq.~(\ref{eq:hlift1}, \ref{eq:hlifti}) (through Eq.~(\ref{supp:eq:eq1_})). Then we have
\begin{equation}
    g(\mathbf{Z}, \mathbf{W}) = \textstyle\sum_1^m \Tr(\mathbf{Z}_i^T\mathbf{W}_i) = \textstyle\sum_1^m \Tr(\mathbf{U}_i^T\mathbf{V}_i) = \textstyle\sum_1^m \Tr((\mathbf{U}_i-\mathbf{U}_1)^T\mathbf{V}_i) + \Tr(\mathbf{U}_1^T\mathbf{V}_i)
\end{equation}
Since $\textstyle\sum_1^m \mathbf{V}_i = 0$, the second term vanishes. The first reduces to 
\begin{align}
    &\textstyle\sum_1^m \Tr((\mathbf{U}_i-\mathbf{U}_1)^T\mathbf{V}_i) = \textstyle\sum_1^m \Tr((\mathbf{U}_i-\mathbf{U}_1)^T(\mathbf{V}_i-\mathbf{V}_1)) + \Tr((\mathbf{U}_i-\mathbf{U}_1)^T\mathbf{V}_1)\\
    &= \textstyle\sum_1^{m-1} \Tr((\mathbf{U}_{i+1}-\mathbf{U}_1)^T(\mathbf{V}_{i+1}-\mathbf{V}_1)) - \revadd{m}\Tr(\mathbf{U}_1^T\mathbf{V}_1)\\
    &=\textstyle\sum_1^{m-1}\Tr(\widetilde{\mathbf{U}}_{i}^T\widetilde{\mathbf{V}}_{i}) - m\Tr(\mathbf{U}_1^T\mathbf{V}_1)\\
    &= \textstyle\sum_1^{m-1}\Tr(\widetilde{\mathbf{U}}_{i}^T\widetilde{\mathbf{V}}_{i}) - \revdel{m^{-2}}\revadd{m^{-1}}\textstyle\sum_{i,j=1}^{m-1}\Tr(\widetilde{\mathbf{U}}_i^T\widetilde{\mathbf{V}}_j).
\end{align}
% \begin{align}
%     &g(\mathbf{Z}, \mathbf{W}) = \textstyle\sum_1^m \Tr(\mathbf{Z}_i^T\mathbf{W}_i) = \textstyle\sum_1^m \Tr(\mathbf{U}_i^T\mathbf{V}_i) = \textstyle\sum_1^m \Tr((\mathbf{U}_i-\mathbf{U}_1)^T\mathbf{V}_i) + \Tr(\mathbf{U}_1^T\mathbf{V}_i)\\
%     &= \textstyle\sum_1^m \Tr((\mathbf{U}_i-\mathbf{U}_1)^T\mathbf{V}_i) = \textstyle\sum_1^m \Tr((\mathbf{U}_i-\mathbf{U}_1)^T(\mathbf{V}_i-\mathbf{V}_1)) + \Tr((\mathbf{U}_i-\mathbf{U}_1)^T\mathbf{V}_1)\\
%     &= \textstyle\sum_1^{m-1} \Tr((\mathbf{U}_{i+1}-\mathbf{U}_1)^T(\mathbf{V}_{i+1}-\mathbf{V}_1)) - \Tr(\mathbf{U}_1^T\mathbf{V}_1)\\
%     &= \textstyle\sum_1^{m-1}\Tr(\widetilde{\mathbf{U}}_{i}^T\widetilde{\mathbf{V}}_{i}) - \Tr(\mathbf{U}_1^T\mathbf{V}_1) = \textstyle\sum_1^{m-1}\Tr(\widetilde{\mathbf{U}}_{i}^T\widetilde{\mathbf{V}}_{i}) - m^{-2}\textstyle\sum_{i,j=1}^{m-1}\Tr(\widetilde{\mathbf{U}}_i^T\widetilde{\mathbf{V}}_j).
% \end{align}
Since the above equation is independent of the choice of $\mathbf{S}$, the result follows.

\proofof{Proposition~\ref{prop:hlift_frob_ineq}} 
We have $\left\|\mathbf{Z}\right\|_F^2 = \sum_1^m \Tr(\mathbf{\Omega}_i\mathbf{\Omega}_i^T)$. Using Proposition~\ref{prop:hlift_char},
\begin{align}
    \left\|\widetilde{\mathbf{Z}}\right\|_F^2 &= \sum_1^{m-1} \Tr(\widetilde{\mathbf{\Omega}}_{i+1}\widetilde{\mathbf{\Omega}}_{i+1}^T) =\sum_1^{m-1} (\Tr(\mathbf{\Omega}_{i+1}\mathbf{\Omega}_{i+1}^T) -2\Tr(\mathbf{\Omega}_{i+1}\mathbf{\Omega}_1^T)) + (m-1)\Tr(\mathbf{\Omega}_1\mathbf{\Omega}_1^T)\\
    &=\sum_1^m \Tr(\mathbf{\Omega}_i\mathbf{\Omega}_i^T) + m \sum_1^m \Tr(\mathbf{\Omega}_1\mathbf{\Omega}_1^T) = \left\|\mathbf{Z}\right\|_F^2 + m\left\|\mathbf{\Omega}_1\right\|_F^2.
\end{align}

\proofof{Proposition~\ref{prop:gradFS}}
Using \citeb[Section 3.6.2]{absil2009optimization} and Proposition~\ref{prop:T_SOdm},
\begin{align}
    \overline{\grad \widetilde{F}(\widetilde{\mathbf{S}})} &= \grad F(\mathbf{S}) = \mathbf{P}_{\mathbf{S}}(\nabla F(\mathbf{S})) = \mathbf{P}_{\mathbf{S}}(2\mathbf{C}\mathbf{S}) =[2\mathbf{S}_i\text{skew}(\mathbf{S}_i^T[\mathbf{C}\mathbf{S}]_i])]_1^m\\
    &=[\mathbf{C}\mathbf{S}]_i - \mathbf{S}_i[\mathbf{C}\mathbf{S}]_i^T\mathbf{S}_i = \mathbf{S}_i (\mathbf{S}_i^T[\mathbf{C}\mathbf{S}]_i - [\mathbf{C}\mathbf{S}]_i^T\mathbf{S}_i)
\end{align}
% \begin{align}
%     \overline{\grad \widetilde{F}(\widetilde{\mathbf{S}})} &= \grad F(\mathbf{S}) = \mathbf{P}_{\mathbf{S}}(\nabla F(\mathbf{S})) = \mathbf{P}_{\mathbf{S}}(2\mathbf{C}\mathbf{S}) = [2\mathbf{S}_i\text{skew}(\mathbf{S}_i^T[\mathbf{C}\mathbf{S}]_i])]_1^m\\
%     &= [\mathbf{C}\mathbf{S}]_i - \mathbf{S}_i[\mathbf{C}\mathbf{S}]_i^T\mathbf{S}_i = \mathbf{S}_i (\mathbf{S}_i^T[\mathbf{C}\mathbf{S}]_i - [\mathbf{C}\mathbf{S}]_i^T\mathbf{S}_i).
% \end{align}
Using the fact that $\mathbf{C}$ is symmetric, we conclude that $\sum_{1}^{m} \boldsymbol{\Omega}_i = \mathbf{S}^T\mathbf{C}\mathbf{S}-\mathbf{S}^T\mathbf{C}^T\mathbf{S} = 0$.

\proofof{Proposition~\ref{prop:DgradFSZ}}
\begin{align}
    D\grad F(\mathbf{S})[\mathbf{Z}]_i &= \lim_{t \rightarrow 0}(\grad F(\mathbf{S}+t\mathbf{Z})_i-\grad F(\mathbf{S})_i)/t\\
    &= \lim_{t \rightarrow 0} t^{-1}\left\{([\mathbf{C}(\mathbf{S}+t\mathbf{Z})]_i - (\mathbf{S}_i+t\mathbf{Z}_i)[\mathbf{C}(\mathbf{S}+t\mathbf{Z})]_i^T(\mathbf{S}_i+t\mathbf{Z}_i)) - ([\mathbf{C}\mathbf{S}]_i - \mathbf{S}_i[\mathbf{C}\mathbf{S}]_i^T\mathbf{S}_i)\right\}\\
    &=[\mathbf{C}\mathbf{Z}]_i - \mathbf{S}_i[\mathbf{C}\mathbf{Z}]_i^T\mathbf{S}_i - \mathbf{S}_i[\mathbf{C}\mathbf{S}]_i^T\mathbf{Z}_i -\mathbf{Z}_i[\mathbf{C}\mathbf{S}]_i^T\mathbf{S}_i\\
    &=\mathbf{S}_i(\mathbf{S}_i^T[\mathbf{C}\mathbf{Z}]_i - [\mathbf{C}\mathbf{Z}]_i^T\mathbf{S}_i - [\mathbf{C}\mathbf{S}]_i^T\mathbf{Z}_i - \mathbf{S}_i^T\mathbf{Z}_i[\mathbf{C}\mathbf{S}]_i^T\mathbf{S}_i)
\end{align}
% \revdel{Note that we have not yet used the facts that $\widetilde{\mathbf{S}} \in \widetilde{\mathcal{C}}$ or equivalently, $[\mathbf{C}\mathbf{S}]_i = \mathbf{S}_i[\mathbf{C}\mathbf{S}]_i^T\mathbf{S}_i$ (see the proof of Proposition~\ref{prop:gradFS}) and, $\mathbf{Z} \in T_{\mathbf{S}}\mathbb{O}(d)^m$ or equivalently, $\mathbf{S}_i^T\mathbf{Z}_i + \mathbf{Z}_i^T\mathbf{S}_i = 0$ (see Proposition~\ref{prop:T_SOdm}). Combining the two, we get $\mathbf{S}_i^T\mathbf{Z}_i[\mathbf{C}\mathbf{S}]_i^T\mathbf{S}_i = -\mathbf{Z}_i^T[\mathbf{C}\mathbf{S}]_i$ and the result follows.}
\revadd{The result follows from the facts that $\mathbf{S}_i^T\mathbf{Z}_i + \mathbf{Z}_i^T\mathbf{S}_i = 0$ for $\mathbf{Z} \in T_{\mathbf{S}}\mathbb{O}(d)^m$ (see Proposition~\ref{prop:T_SOdm}), and $[\mathbf{C}\mathbf{S}]_i = \mathbf{S}_i[\mathbf{C}\mathbf{S}]_i^T\mathbf{S}_i$ for $\widetilde{\mathbf{S}} \in \widetilde{\mathcal{C}}$ (see the proof of Proposition~\ref{prop:gradFS})}.

\proofof{Proposition~\ref{prop:HessFSZ}}
Using \citeb[Chapter 5]{absil2009optimization},
$$\overline{\Hess \widetilde{F}(\widetilde{\mathbf{S}})[\widetilde{\mathbf{Z}}]} = \overline{\widetilde{\nabla}_{\widetilde{\mathbf{Z}}}\grad \widetilde{F}(\widetilde{\mathbf{S}})} = P^{h}_{\mathbf{S}}(\nabla_{\overline{\widetilde{\mathbf{Z}}}}\overline{\grad \widetilde{F}(\widetilde{\mathbf{S}})}) = P^{h}_{\mathbf{S}}(\nabla_{\mathbf{Z}}\grad F(\mathbf{S})).$$ Then from, Proposition~\ref{prop:T_SOdm}, \ref{prop:V_S_H_S} and \ref{prop:DgradFSZ}, the latter is reduced to
\begin{align}
    P^{h}_{\mathbf{S}}(\nabla_{\mathbf{Z}}\grad F(\mathbf{S})) &= P^{h}_{\mathbf{S}}(P_{\mathbf{S}}(D\grad F(\mathbf{S})[\mathbf{Z}])) = P^{h}_{\mathbf{S}}\left(\left[\mathbf{S}_i(\Skew(\boldsymbol{\xi}_i))\right]_1^m\right)\\
    &= [\mathbf{S}_i(\Skew(\boldsymbol{\xi}_i)-m^{-1}\sum_1^m\Skew(\boldsymbol{\xi}_i))]_1^m
\end{align}
For the case $\widetilde{\mathbf{S}} \in \widetilde{\mathcal{C}}$, we note that $\boldsymbol{\xi}_i = \Skew(\boldsymbol{\xi}_i) = \mathbf{S}_i^T[\mathbf{C}\mathbf{Z}]_i - [\mathbf{C}\mathbf{Z}]_i^T\mathbf{S}_i - [\mathbf{C}\mathbf{S}]_i^T\mathbf{Z}_i + \mathbf{Z}_i^T[\mathbf{C}\mathbf{S}]_i$ and $\sum_{1}^{m}\boldsymbol{\xi}_i = \mathbf{S}^T\mathbf{C}\mathbf{Z}-\mathbf{Z}^T\mathbf{C}\mathbf{S}-\mathbf{S}^T\mathbf{C}\mathbf{Z}+\mathbf{Z}^T\mathbf{C}\mathbf{S} = 0$.
% \revdel{which equals $P^{h}_{\mathbf{S}}\left(\left[\mathbf{S}_i(\mathbf{S}_i^T[\mathbf{C}\mathbf{Z}]_i - [\mathbf{C}\mathbf{Z}]_i^T\mathbf{S}_i - [\mathbf{C}\mathbf{S}]_i^T\mathbf{Z}_i + \mathbf{Z}_i^T[\mathbf{C}\mathbf{S}]_i)\right]_1^m\right)
%     = P^h_{\mathbf{S}}([\mathbf{S}_i\widehat{\boldsymbol{\Omega}}_i]_1^m) = [\mathbf{S}_i(\widehat{\boldsymbol{\Omega}}_i-\overline{\widehat{\boldsymbol{\Omega}}})]_1^m$.
% % \begin{align}
% %     \overline{\Hess \widetilde{F}(\widetilde{\mathbf{S}})[\widetilde{\mathbf{Z}}]} &= \overline{\widetilde{\nabla}_{\widetilde{\mathbf{Z}}}\grad \widetilde{F}(\widetilde{\mathbf{S}})} = P^{h}_{\mathbf{S}}(\nabla_{\overline{\widetilde{\mathbf{Z}}}}\overline{\grad \widetilde{F}(\widetilde{\mathbf{S}})}) = P^{h}_{\mathbf{S}}(\nabla_{\mathbf{Z}}\grad F(\mathbf{S}))\\
% %     &= P^{h}_{\mathbf{S}}(P_{\mathbf{S}}(D\grad F(\mathbf{S})[\mathbf{Z}]))\\
% %     &= P^{h}_{\mathbf{S}}\left(\left[\mathbf{S}_i(\mathbf{S}_i^T[\mathbf{C}\mathbf{Z}]_i - [\mathbf{C}\mathbf{Z}]_i^T\mathbf{S}_i - [\mathbf{C}\mathbf{S}]_i^T\mathbf{Z}_i + \mathbf{Z}_i^T[\mathbf{C}\mathbf{S}]_i)\right]_1^m\right)\\
% %     &= P^h_{\mathbf{S}}([\mathbf{S}_i\widehat{\boldsymbol{\Omega}}_i]_1^m) = [\mathbf{S}_i(\widehat{\boldsymbol{\Omega}}_i-\overline{\widehat{\boldsymbol{\Omega}}})]_1^m
% % \end{align}
% where $\overline{\widehat{\boldsymbol{\Omega}}} = m^{-1}\textstyle\sum_1^m\widehat{\boldsymbol{\Omega}}_i = m^{-1}\textstyle\sum_1^m (\mathbf{S}_i^T[\mathbf{C}\mathbf{Z}]_i - [\mathbf{C}\mathbf{Z}]_i^T\mathbf{S}_i - [\mathbf{C}\mathbf{S}]_i^T\mathbf{Z}_i + \mathbf{Z}_i^T[\mathbf{C}\mathbf{S}]_i)$. Thus, $\overline{\widehat{\boldsymbol{\Omega}}} = m^{-1}(\mathbf{S}^T\mathbf{C}\mathbf{Z}-\mathbf{Z}^T\mathbf{C}\mathbf{S}-\mathbf{S}^T\mathbf{C}\mathbf{Z}+\mathbf{Z}^T\mathbf{C}\mathbf{S}) = 0$.
% % \begin{align}
% %     \overline{\widehat{\boldsymbol{\Omega}}} &= m^{-1}\textstyle\sum_1^m\widehat{\boldsymbol{\Omega}}_i = m^{-1}\textstyle\sum_1^m (\mathbf{S}_i^T[\mathbf{C}\mathbf{Z}]_i - [\mathbf{C}\mathbf{Z}]_i^T\mathbf{S}_i - [\mathbf{C}\mathbf{S}]_i^T\mathbf{Z}_i + \mathbf{Z}_i^T[\mathbf{C}\mathbf{S}]_i)\\
% %     &= m^{-1}(\mathbf{S}^T\mathbf{C}\mathbf{Z}-\mathbf{Z}^T\mathbf{C}\mathbf{S}-\mathbf{S}^T\mathbf{C}\mathbf{Z}+\mathbf{Z}^T\mathbf{C}\mathbf{S}) = 0.
% % \end{align}
% This further validates that $\overline{\Hess \widetilde{F}(\widetilde{\mathbf{S}})[\widetilde{\mathbf{Z}}]} = [\mathbf{S}_i\widehat{\boldsymbol{\Omega}}_i]_1^m$ indeed lies in $\mathcal{H}_{\mathbf{S}}$ (see Proposition~\ref{prop:V_S_H_S}).
% }

\proofof{Proposition~\ref{prop:Omega_hat_compact}}
Since $\mathbf{Z} \in \mathcal{H}_{\mathbf{S}}$, using Proposition~\ref{prop:V_S_H_S}, there exist $\boldsymbol{\Omega} = [\boldsymbol{\Omega}_i]_1^m$ such that $\boldsymbol{\Omega}_i \in \Skew(d)$, $\textstyle\sum_1^m\boldsymbol{\Omega}_i = 0$ and $\mathbf{Z}_i = \mathbf{S}_i\boldsymbol{\Omega}_i$. Then,
\revadd{$\boldsymbol{\xi}_i = [\mathbf{C}(\mathbf{S})\boldsymbol{\Omega}]_i - [\mathbf{C}(\mathbf{S})\boldsymbol{\Omega}]_i^T - [\widehat{\mathbf{C}}(\mathbf{S})^T\boldsymbol{\Omega}]_i + [\widehat{\mathbf{C}}(\mathbf{S})\boldsymbol{\Omega}]_i^T$. The result follows from (i) $\mathbf{L}(\mathbf{S}) = \mathbf{C}(\mathbf{S}) - \widehat{\mathbf{C}}(\mathbf{S})$, (ii) $\mathbf{C}(\mathbf{S})$ is symmetric and (iii) for $\widetilde{\mathbf{S}}\in \widetilde{\mathcal{C}}$, $\widehat{\mathbf{C}}(\mathbf{S})$ is also symmetric (Remark~\ref{rmk:StildeCtilde}).}
% \revdel{$\widehat{\boldsymbol{\Omega}}_i$, which equals $(\mathbf{S}_i^T[\mathbf{C}\mathbf{Z}]_i - [\mathbf{C}\mathbf{Z}]_i^T\mathbf{S}_i) - ([\mathbf{C}\mathbf{S}]_i^T\mathbf{Z}_i - \mathbf{Z}_i^T[\mathbf{C}\mathbf{S}]_i)$, reduces to $([\mathbf{C}(\mathbf{S})\boldsymbol{\Omega}]_i - [\mathbf{C}(\mathbf{S})\boldsymbol{\Omega}]_i^T) - ( [\widehat{\mathbf{C}}(\mathbf{S})\boldsymbol{\Omega}]_i - [\widehat{\mathbf{C}}(\mathbf{S})\boldsymbol{\Omega}]_i^T)$. The latter is simply $[\mathbf{L}(\mathbf{S})\boldsymbol{\Omega}]_i^T - [\mathbf{L}(\mathbf{S})\boldsymbol{\Omega}]_i$.
% }
% \begin{align}
%     \widehat{\boldsymbol{\Omega}}_i &= (\mathbf{S}_i^T[\mathbf{C}\mathbf{Z}]_i - [\mathbf{C}\mathbf{Z}]_i^T\mathbf{S}_i) - ([\mathbf{C}\mathbf{S}]_i^T\mathbf{Z}_i - \mathbf{Z}_i^T[\mathbf{C}\mathbf{S}]_i)\\
%     &= ([\mathbf{C}(\mathbf{S})\boldsymbol{\Omega}]_i - [\mathbf{C}(\mathbf{S})\boldsymbol{\Omega}]_i^T) - ( [\widehat{\mathbf{C}}(\mathbf{S})\boldsymbol{\Omega}]_i - [\widehat{\mathbf{C}}(\mathbf{S})\boldsymbol{\Omega}]_i^T) = [\mathbf{L}(\mathbf{S})\boldsymbol{\Omega}]_i^T - [\mathbf{L}(\mathbf{S})\boldsymbol{\Omega}]_i.
% \end{align}

\proofof{Proposition~\ref{prop:HessFSZZ}}
We obtain $\widetilde{g}(\Hess \widetilde{F}(\widetilde{\mathbf{S}})[\widetilde{\mathbf{Z}}],\widetilde{\mathbf{Z}}) = g(\overline{\Hess \widetilde{F}(\widetilde{\mathbf{S}})[\widetilde{\mathbf{Z}}]},\overline{\widetilde{\mathbf{Z}}})$ from Proposition~\ref{prop:g_tilde}. Due to Proposition \ref{prop:HessFSZ},
$$g(\overline{\Hess \widetilde{F}(\widetilde{\mathbf{S}})[\widetilde{\mathbf{Z}}]},\overline{\widetilde{\mathbf{Z}}}) = g(\overline{\Hess \widetilde{F}(\widetilde{\mathbf{S}})[\widetilde{\mathbf{Z}}]},\mathbf{Z}) = \textstyle\sum_{1}^{m}\Tr((\mathbf{S}_i\widehat{\boldsymbol{\Omega}}_i)^T\mathbf{S}_i\boldsymbol{\Omega}_i) = \textstyle\sum_{1}^{m}\Tr(\widehat{\boldsymbol{\Omega}}_i^T\boldsymbol{\Omega}_i).$$ \revadd{The result follows from the Proposition~\ref{prop:Omega_hat_compact} and the fact that $\Tr(\mathbf{A}) = \Tr(\mathbf{A}^T)$.}
% \revdel{This, using Proposition~\ref{prop:Omega_hat_compact}, simplifies to $\textstyle\sum_{1}^{m}\Tr([\mathbf{L}(\mathbf{S})\boldsymbol{\Omega}]_i\boldsymbol{\Omega}_i - [\mathbf{L}(\mathbf{S})\boldsymbol{\Omega}]_i^T\boldsymbol{\Omega}_i) = -2\Tr(\boldsymbol{\Omega}^T\mathbf{L}(\mathbf{S})\boldsymbol{\Omega})$.}
% \begin{align}
%     &g(\overline{\Hess \widetilde{F}(\widetilde{\mathbf{S}})[\widetilde{\mathbf{Z}}]},\mathbf{Z}) = \textstyle\sum_{1}^{m}\Tr((\mathbf{S}_i\widehat{\boldsymbol{\Omega}}_i)^T\mathbf{S}_i\boldsymbol{\Omega}_i) = \textstyle\sum_{1}^{m}\Tr(\widehat{\boldsymbol{\Omega}}_i^T\boldsymbol{\Omega}_i)\\
%     &= \textstyle\sum_{1}^{m}\Tr([\mathbf{L}(\mathbf{S})\boldsymbol{\Omega}]_i\boldsymbol{\Omega}_i - [\mathbf{L}(\mathbf{S})\boldsymbol{\Omega}]_i^T\boldsymbol{\Omega}_i) = -2\Tr(\boldsymbol{\Omega}^T\mathbf{L}(\mathbf{S})\boldsymbol{\Omega})
% \end{align}

\proofof{Proposition~\ref{prop:Omega^TLSOmega2}} Using $\vecz (\mathbf{A}\mathbf{X}\mathbf{B}) = (\mathbf{B}^T \otimes \mathbf{A})\vecz (\mathbf{X})$, we obtain
\begin{align}
\Tr(\boldsymbol{\Omega}^T&\mathbf{L}(\mathbf{S})\boldsymbol{\Omega}) = \Tr((\mathbf{P}\boldsymbol{\Omega})^T\mathbf{P}\mathbf{L}(\mathbf{S})\mathbf{P}^T(\mathbf{P}\boldsymbol{\Omega})) = \vecz (\mathbf{P}\boldsymbol{\Omega})^T\vecz (\mathcal{L}(\mathbf{S}) (\mathbf{P}\boldsymbol{\Omega}))\\
&= \vecz (\mathbf{P}\boldsymbol{\Omega})^T(\mathbf{I}_d \otimes \mathcal{L}(\mathbf{S}))\vecz (\mathbf{P}\boldsymbol{\Omega}) = \boldsymbol{\omega}^T\overline{\mathbf{P}}(\mathbf{I}_d \otimes \mathcal{L}(\mathbf{S}))\overline{\mathbf{P}}^T\boldsymbol{\omega} = \boldsymbol{\omega}^T\mathbb{L}(\mathbf{S})\boldsymbol{\omega}.
\end{align}
% \begin{align}
%     &\Tr(\boldsymbol{\Omega}^T\mathbf{L}(\mathbf{S})\boldsymbol{\Omega}) = \Tr((\mathbf{P}\boldsymbol{\Omega})^T\mathbf{P}\mathbf{L}(\mathbf{S})\mathbf{P}^T(\mathbf{P}\boldsymbol{\Omega}))= \vecz (\mathbf{P}\boldsymbol{\Omega})^T\vecz (\mathcal{L}(\mathbf{S}) (\mathbf{P}\boldsymbol{\Omega}))\\
%     &= \vecz (\mathbf{P}\boldsymbol{\Omega})^T(\mathbf{I}_d \otimes \mathcal{L}(\mathbf{S}))\vecz (\mathbf{P}\boldsymbol{\Omega})= \boldsymbol{\omega}^T\overline{\mathbf{P}}(\mathbf{I}_d \otimes \mathcal{L}(\mathbf{S}))\overline{\mathbf{P}}^T\boldsymbol{\omega}= \boldsymbol{\omega}^T\mathbb{L}(\mathbf{S})\boldsymbol{\omega}.
% \end{align}

\proofof{Theorem~\ref{thm:non_deg_loc_min}}
$1$, $2$ and $3$ are equivalent by definitions and Proposition~\ref{prop:HessFSZZ}. For ($3 \iff 4$), by comparing dimensions, we note that the set of $\boldsymbol{\Omega} = [\boldsymbol{\Omega}_i]_1^m$ where $\boldsymbol{\Omega}_i \in \Skew(d)$ and $\textstyle\sum_1^m \boldsymbol{\Omega}_i = 0$, is the same as the set of $\boldsymbol{\Omega} = [\boldsymbol{\Omega}_i-\boldsymbol{\Omega}_0]_1^m$ where $\boldsymbol{\Omega}_i \in \Skew(d)$ and $\boldsymbol{\Omega}_0 = \frac{1}{m}\textstyle\sum_1^m\boldsymbol{\Omega}_i$. Using Remark~\ref{rmk:C_hat_L_structure}, we know that $\mathbf{L}(\mathbf{S})=\mathbf{L}(\mathbf{S})^T$ and $\mathbf{L}(\mathbf{S})[\boldsymbol{\Omega}_0]_1^m = 0$. Thus $(\boldsymbol{\Omega}-[\boldsymbol{\Omega}_0]_1^m)^T\mathbf{L}(\mathbf{S})(\boldsymbol{\Omega}-[\boldsymbol{\Omega}_0]_1^m) = \boldsymbol{\Omega}^T\mathbf{L}(\mathbf{S})\boldsymbol{\Omega}$ and the result follows. Subsequently, ($4 \iff 5$) follows directly from the definition of $\boldsymbol{\omega}$ and Proposition~\ref{prop:Omega^TLSOmega2}. Finally, ($5 \iff 6$) and ($6 \iff 7$) follow from Remark~\ref{rmk:mathbb_L_structure}.

\proofof{Proposition~\ref{prop:one_all1}} Suppose $\mathbf{L}(\mathbf{S})$ satisfies condition 4 and $\boldsymbol{\Omega}$ be as in the condition. Then using Remark~\ref{rmk:C_hat_L_structure}, 
% \begin{align}
%     \Tr(\boldsymbol{\Omega}^T\mathbf{L}(\mathbf{S}\mathbf{\mathbf{Q}})\boldsymbol{\Omega}) &= \Tr(\boldsymbol{\Omega}^T(\mathbf{I}_m \otimes \mathbf{Q})^T\mathbf{L}(\mathbf{S})(\mathbf{I}_m \otimes \mathbf{Q})\boldsymbol{\Omega}) = \Tr(\mathbf{Q}\boldsymbol{\Omega}^T(\mathbf{I}_m \otimes \mathbf{Q})^T\mathbf{L}(\mathbf{S})(\mathbf{I}_m \otimes \mathbf{Q})\boldsymbol{\Omega} \mathbf{Q}^T)\\
%     &= \Tr(\overline{\boldsymbol{\Omega}}^T \mathbf{L}(\mathbf{S})\overline{\boldsymbol{\Omega}})
% \end{align}
\begin{align}
    \Tr(\boldsymbol{\Omega}^T\mathbf{L}(\mathbf{S}\mathbf{\mathbf{Q}})\boldsymbol{\Omega}) &= \Tr(\boldsymbol{\Omega}^T(\mathbf{I}_m \otimes \mathbf{Q})^T\mathbf{L}(\mathbf{S})(\mathbf{I}_m \otimes \mathbf{Q})\boldsymbol{\Omega}) = \Tr(\overline{\boldsymbol{\Omega}}^T \mathbf{L}(\mathbf{S})\overline{\boldsymbol{\Omega}})
\end{align}
which is positive because $\overline{\boldsymbol{\Omega}}_i = \mathbf{Q}\boldsymbol{\Omega}_i\mathbf{Q}^T \in \Skew(d)$ for all $i \in [1,m]$ and not all $\overline{\boldsymbol{\Omega}}_i$ are equal (if $\overline{\boldsymbol{\Omega}}_i = \overline{\boldsymbol{\Omega}}_j$ then $\boldsymbol{\Omega}_i = \boldsymbol{\Omega}_j$, a contradiction). Thus, condition 4 holds for~$\mathbf{S}\mathbf{Q}$ too. The result follows.
% \revdel{It suffices to pick conditions 4 and 6 (one involving $\mathbf{L}(\mathbf{S})$ and one involving $\mathbb{L}(\mathbf{S})$). Suppose $\mathbf{L}(\mathbf{S})$ satisfies condition 4 and let $\boldsymbol{\Omega}$ be as in the condition. Then using Remark~\ref{rmk:C_hat_L_structure}, $\Tr(\boldsymbol{\Omega}^T\mathbf{L}(\mathbf{S}\mathbf{\mathbf{Q}})\boldsymbol{\Omega}) = \Tr(\boldsymbol{\Omega}^T(\mathbf{I}_m \otimes \mathbf{Q})^T\mathbf{L}(\mathbf{S})(\mathbf{I}_m \otimes \mathbf{Q})\boldsymbol{\Omega})$. The latter simplifies to $\Tr(\mathbf{Q}\boldsymbol{\Omega}^T(\mathbf{I}_m \otimes \mathbf{Q})^T\mathbf{L}(\mathbf{S})(\mathbf{I}_m \otimes \mathbf{Q})\boldsymbol{\Omega} \mathbf{Q}^T) = \Tr(\overline{\boldsymbol{\Omega}}^T \mathbf{L}(\mathbf{S})\overline{\boldsymbol{\Omega}})$ which is negative because $\overline{\boldsymbol{\Omega}}_i = \mathbf{Q}\boldsymbol{\Omega}_i\mathbf{Q}^T \in \Skew(d)$ for all $i \in [1,m]$ and not all $\overline{\boldsymbol{\Omega}}_i$ are equal (if $\overline{\boldsymbol{\Omega}}_i = \overline{\boldsymbol{\Omega}}_j$ then $\boldsymbol{\Omega}_i = \boldsymbol{\Omega}_j$, a contradiction). Thus the condition 4 holds for $\mathbf{S}\mathbf{Q}$. Now suppose $\mathbf{S}$ satisfies condition 6 then from Remark~\ref{rmk:mathbb_L_structure}, $\mathbf{S}\mathbf{Q}$ satisfies it too.} 

\proofof{Corollary~\ref{cor:suff_non_deg_loc_min}}
Since the rank of $\mathbf{L}(\mathbf{S})$ is $(m-1)d$, using Remark~\ref{rmk:C_hat_L_structure}, $\mathbf{L}(\mathbf{S})[\boldsymbol{\Omega}_i]_1^m = 0$ (equivalently, $\Tr(([\boldsymbol{\Omega}_i]_1^m)^T\mathbf{L}(\mathbf{S})[\boldsymbol{\Omega}_i]_1^m) = 0$) iff $\boldsymbol{\Omega}_i = \boldsymbol{\Omega}_0$ for all $i \in [1,m]$ and some $\boldsymbol{\Omega}_0 \in \Skew(d)$. Thus, condition~$4$ in Theorem~\ref{thm:non_deg_loc_min} is satisfied. 


\proofof{Proposition~\ref{prop:HessVicinity}} For brevity, we define $\mathbf{D}_{\mathbf{S}} \coloneqq \blockdiag([\mathbf{S}_i]_1^m)$ and $\mathbf{Z}_i \coloneqq \mathbf{O}_i\boldsymbol{\Omega}_i$, i $\in [1,m]$, which satisfies $\left\|\mathbf{Z}_i\right\|_F^2 = \left\|\boldsymbol{\Omega}_i\right\|_F^2$. Then,
\begin{align}
    \Tr(\boldsymbol{\Omega}^T(\mathbf{L}(\mathbf{O})+\mathbf{L}(\mathbf{O})^T)\boldsymbol{\Omega}) &= 2\Tr(\boldsymbol{\Omega}^T\mathbf{C}(\mathbf{O})\boldsymbol{\Omega}) - \Tr(\boldsymbol{\Omega}^T(\widehat{\mathbf{C}}(\mathbf{O})+\widehat{\mathbf{C}}(\mathbf{O})^T)\boldsymbol{\Omega})\\
    &= 2\Tr(\boldsymbol{\Omega}^T\mathbf{C}(\mathbf{O})\boldsymbol{\Omega}) - \textstyle\sum_{i=1}^{m}\Tr\left(\boldsymbol{\Omega}_i^T \left(\textstyle\sum_{j=1}^{m}\mathbf{O}_i^T\mathbf{C}_{ij}\mathbf{O}_j + \mathbf{O}_j^T\mathbf{C}_{ji}\mathbf{O}_i\right)\boldsymbol{\Omega}_i\right)\\
    &= 2\Tr(\boldsymbol{\Omega}^T\mathbf{C}(\mathbf{O})\boldsymbol{\Omega}) - \textstyle\sum_{i=1}^{m}\Tr\left(\mathbf{Z}_i^T \left(\textstyle\sum_{j=1}^{m}\mathbf{C}_{ij}\mathbf{O}_j\mathbf{O}_i^T + \mathbf{O}_i\mathbf{O}_j^T\mathbf{C}_{ji}\right)\mathbf{Z}_i\right)\label{eq:HessO1}
\end{align}
Rewriting the first term,
\begin{align}
\textstyle\sum_{i=1}^{m}\Tr&\left(\mathbf{Z}_i^T\textstyle\sum_{j=1}^{m}\mathbf{C}_{ij}\mathbf{O}_j\mathbf{O}_i^T\mathbf{Z}_i\right) = \textstyle\sum_{i=1}^{m}\Tr\left(\mathbf{Z}_i^T\textstyle\sum_{j=1}^{m}\mathbf{C}_{ij}(\mathbf{O}_j\mathbf{O}_i^T-\mathbf{S}_j\mathbf{S}_i^T + \mathbf{S}_j\mathbf{S}_i^T)\mathbf{Z}_i\right)\\
&= \textstyle\sum_{i=1}^{m}\Tr\left(\mathbf{Z}_i^T\textstyle\sum_{j=1}^{m}\mathbf{C}_{ij}(\mathbf{O}_j\mathbf{O}_i^T-\mathbf{S}_j\mathbf{S}_i^T)\mathbf{Z}_i\right) + \Tr\left(\mathbf{Z}_i^T\mathbf{S}_i\mathbf{S}_i^T\textstyle\sum_{j=1}^{m}\mathbf{C}_{ij} \mathbf{S}_j\mathbf{S}_i^T\mathbf{Z}_i\right)\\
&= \textstyle\sum_{i=1}^{m} \Tr\left(\mathbf{Z}_i^T\textstyle\sum_{j=1}^{m}\mathbf{C}_{ij}(\mathbf{O}_j\mathbf{O}_i^T-\mathbf{S}_j\mathbf{S}_i^T)\mathbf{Z}_i\right) + \Tr\left(\mathbf{Z}^T\mathbf{D}_{\mathbf{S}}\widehat{\mathbf{C}}(\mathbf{S})\mathbf{D}_{\mathbf{S}}^T\mathbf{Z}\right) \label{eq:HessO2}
\end{align}
Using Cauchy-Schwarz inequality and the fact that $\left\|\mathbf{A}_1\mathbf{A}_2\right\|_F \leq \left\|\mathbf{A}_1\right\|_2\left\|\mathbf{A}_2\right\|_F$,
\begin{align}
\left|\textstyle\sum_{i=1}^{m} \Tr\right.&\left.\left(\mathbf{Z}_i^T\textstyle\sum_{j=1}^{m}\mathbf{C}_{ij}(\mathbf{O}_j\mathbf{O}_i^T-\mathbf{S}_j\mathbf{S}_i^T)\mathbf{Z}_i\right)\right| \leq \textstyle\sum_{i=1}^{m} \left\|\textstyle\sum_{j=1}^{m}\mathbf{C}_{ij}(\mathbf{O}_j\mathbf{O}_i^T-\mathbf{S}_j\mathbf{S}_i^T)\right\|_F\left\|\mathbf{Z}_i\right\|_F^2\\
&\leq \textstyle\sum_{i=1}^{m} \left\|\textstyle\sum_{j=1}^{m}\mathbf{C}_{ij}(\mathbf{O}_j\mathbf{O}_i^T-\mathbf{S}_j\mathbf{O}_i^T + \mathbf{S}_j\mathbf{O}_i^T-\mathbf{S}_j\mathbf{S}_i^T)\right\|_F\left\|\mathbf{Z}_i\right\|_F^2\\
&\leq  \textstyle\sum_{i=1}^{m}(\max_{k=1}^{m}\left\|\mathbf{C}_{k,:}\right\|_2 \left\|\mathbf{O}-\mathbf{S}\right\|_F + \left\|[\mathbf{C}\mathbf{S}]_i\right\|_2\left\|\mathbf{O}_i-\mathbf{S}_i\right\|_F)\left\|\boldsymbol{\Omega}_i\right\|_F^2\\
&\leq   \textstyle\sum_{i=1}^{m}(c_1 \left\|\mathbf{O}-\mathbf{S}\right\|_F + c_2(\mathbf{S})\left\|\mathbf{O}_i-\mathbf{S}_i\right\|_F)\left\|\boldsymbol{\Omega}_i\right\|_F^2\\
&\leq  (c_1 + c_2(\mathbf{S}))\left\|\mathbf{O}-\mathbf{S}\right\|_F\left\|\boldsymbol{\Omega}\right\|_F^2 \label{eq:HessO3}
\end{align}
Also, due to Eq.~(\ref{eq:C_of_S}),
$$\Tr(\boldsymbol{\Omega}^T\mathbf{C}(\mathbf{O})\boldsymbol{\Omega}) = \Tr(\boldsymbol{\Omega}^T\mathbf{D}_{\mathbf{O}}^T\mathbf{D}_{\mathbf{S}}\mathbf{C}(\mathbf{S})\mathbf{D}_{\mathbf{S}}^T\mathbf{D}_{\mathbf{O}}\boldsymbol{\Omega}) = \Tr(\mathbf{Z}^T\mathbf{D}_{\mathbf{S}}\mathbf{C}(\mathbf{S})\mathbf{D}_{\mathbf{S}}^T\mathbf{Z}).$$
Combining this with Eq.~(\ref{eq:HessO1}, \ref{eq:HessO2}, \ref{eq:HessO3}), we obtain
\begin{align}
    \Tr(\boldsymbol{\Omega}^T(\mathbf{L}(\mathbf{O})+\mathbf{L}(\mathbf{O})^T)\boldsymbol{\Omega}) &\geq 2\Tr(\mathbf{Z}^T\mathbf{D}_{\mathbf{S}}\mathbf{L}(\mathbf{S})\mathbf{D}_{\mathbf{S}}^T\mathbf{Z}) - 2(c_1 + c_2(\mathbf{S}))\left\|\mathbf{O}-\mathbf{S}\right\|_F\left\|\boldsymbol{\Omega}\right\|_F^2\\
    \Tr(\boldsymbol{\Omega}^T(\mathbf{L}(\mathbf{O})+\mathbf{L}(\mathbf{O})^T)\boldsymbol{\Omega}) &\leq 2\Tr(\mathbf{Z}^T\mathbf{D}_{\mathbf{S}}\mathbf{L}(\mathbf{S})\mathbf{D}_{\mathbf{S}}^T\mathbf{Z}) + 2(c_1 + c_2(\mathbf{S}))\left\|\mathbf{O}-\mathbf{S}\right\|_F\left\|\boldsymbol{\Omega}\right\|_F^2.
\end{align}
Moreover,
\begin{align}
\Tr(\mathbf{Z}^T\mathbf{D}_{\mathbf{S}}\mathbf{L}(\mathbf{S})\mathbf{D}_{\mathbf{S}}^T\mathbf{Z}) &= \Tr(\boldsymbol{\Omega}^T\mathbf{D}_{\mathbf{O}}^T\mathbf{D}_{\mathbf{S}}\mathbf{L}(\mathbf{S})\mathbf{D}_{\mathbf{S}}^T\mathbf{D}_{\mathbf{O}}\boldsymbol{\Omega})\\
&= \Tr(\boldsymbol{\Omega}^T\mathbf{L}(\mathbf{S})\boldsymbol{\Omega}) + 2\Tr(\boldsymbol{\Omega}^T(\mathbf{D}_{\mathbf{O}}^T\mathbf{D}_{\mathbf{S}}-\mathbf{I}_{md})\mathbf{L}(\mathbf{S})\mathbf{D}_{\mathbf{S}}^T\mathbf{D}_{\mathbf{O}}\boldsymbol{\Omega}),
\end{align}
% \begin{align}
% \Tr(\mathbf{Z}^T\mathbf{D}_{\mathbf{S}}\mathbf{L}(\mathbf{S})\mathbf{D}_{\mathbf{S}}^T\mathbf{Z}) &= \Tr(\boldsymbol{\Omega}^T\mathbf{D}_{\mathbf{O}}^T\mathbf{D}_{\mathbf{S}}\mathbf{L}(\mathbf{S})\mathbf{D}_{\mathbf{S}}^T\mathbf{D}_{\mathbf{O}}\boldsymbol{\Omega})\\
% &= \Tr(\boldsymbol{\Omega}^T\mathbf{L}(\mathbf{S})\boldsymbol{\Omega}) + 2\Tr(\boldsymbol{\Omega}^T(\mathbf{D}_{\mathbf{O}}^T\mathbf{D}_{\mathbf{S}}-\mathbf{I}_{md})\mathbf{L}(\mathbf{S})\mathbf{D}_{\mathbf{S}}^T\mathbf{D}_{\mathbf{O}}\boldsymbol{\Omega}),
% \end{align}
where, for the first term,
\begin{equation}
    (\lambda_{-}(\mathbf{S})/2) \left\|\boldsymbol{\Omega}\right\|_F^2 \leq \Tr(\boldsymbol{\Omega}^T\mathbf{L}(\mathbf{S})\boldsymbol{\Omega}) =  \Tr(\boldsymbol{\omega}^T\mathbf{\mathbb{L}}(\mathbf{S})\boldsymbol{\omega}) \leq  (\lambda_{+}(\mathbf{S})/2) \left\|\boldsymbol{\Omega}\right\|_F^2.
\end{equation}
The fraction $1/2$ appears because $\left\|\boldsymbol{\omega}\right\|_F^2 = \left\|\boldsymbol{\Omega}\right\|_F^2/2$ as in Eq.~(\ref{eq:omega^TmbbLomega}). Then, for the second term, using $|\Tr(\mathbf{A}_1\mathbf{A}_2)| \leq \left\|\mathbf{A}_1\right\|_2\Tr(\mathbf{A}_2)$ and Cauchy-Schwarz inequality,
\begin{align}
\left|\Tr(\boldsymbol{\Omega}^T(\mathbf{D}_{\mathbf{O}}^T\mathbf{D}_{\mathbf{S}}-\mathbf{I}_{md})\mathbf{L}(\mathbf{S})\mathbf{D}_{\mathbf{S}}^T\mathbf{D}_{\mathbf{O}}\boldsymbol{\Omega})\right| &\leq \left\|\mathbf{L}(\mathbf{S})\right\|_2\left|\Tr(\boldsymbol{\Omega}^T(\mathbf{D}_{\mathbf{O}}^T\mathbf{D}_{\mathbf{S}}-\mathbf{I}_{md})\boldsymbol{\Omega})\right|\\
&\leq c_3(\mathbf{S})\left\|\mathbf{S}-\mathbf{O}\right\|_F\left\|\boldsymbol{\Omega}\right\|_F^2.
\end{align}
Overall,
\begin{align}
    \Tr(\boldsymbol{\Omega}^T(\mathbf{L}(\mathbf{O})+\mathbf{L}(\mathbf{O})^T)\boldsymbol{\Omega}) &\geq (\lambda_{-}(\mathbf{S}) - 2(c_1 + c_2(\mathbf{S}) + 2 c_3(\mathbf{S}))\left\|\mathbf{S}-\mathbf{O}\right\|_F)\left\|\boldsymbol{\Omega}\right\|_F^2\\
    \Tr(\boldsymbol{\Omega}^T(\mathbf{L}(\mathbf{O})+\mathbf{L}(\mathbf{O})^T)\boldsymbol{\Omega}) &\leq (\lambda_{+}(\mathbf{S}) + 2(c_1 + c_2(\mathbf{S}) + 2 c_3(\mathbf{S}))\left\|\mathbf{S}-\mathbf{O}\right\|_F)\left\|\boldsymbol{\Omega}\right\|_F^2
\end{align}
% \begin{equation}
% \begin{matrix}
% \Tr(\boldsymbol{\Omega}^T(\mathbf{L}(\mathbf{O})+\mathbf{L}(\mathbf{O})^T)\boldsymbol{\Omega}) \geq (\lambda_{-}(\mathbf{S}) - 2(c_1 + c_2(\mathbf{S}) + 2 c_3(\mathbf{S}))\left\|\mathbf{S}-\mathbf{O}\right\|_F)\left\|\boldsymbol{\Omega}\right\|_F^2\\
% \Tr(\boldsymbol{\Omega}^T(\mathbf{L}(\mathbf{O})+\mathbf{L}(\mathbf{O})^T)\boldsymbol{\Omega}) \leq (\lambda_{+}(\mathbf{S}) + 2(c_1 + c_2(\mathbf{S}) + 2 c_3(\mathbf{S}))\left\|\mathbf{S}-\mathbf{O}\right\|_F)\left\|\boldsymbol{\Omega}\right\|_F^2.
% \end{matrix}
% \end{equation}
\revadd{Consequently, if $\left\|\mathbf{S}-\mathbf{O}\right\|_F < \zeta\delta(\mathbf{S})$ (as defined in the theorem statement) then}
\begin{equation}
\revadd{(1-\zeta)\lambda_{-}(\mathbf{S})\left\|\boldsymbol{\Omega}\right\|_F^2 \leq \Tr(\boldsymbol{\Omega}^T(\mathbf{L}(\mathbf{O})+\mathbf{L}(\mathbf{O})^T)\boldsymbol{\Omega}) \leq (\lambda_{+}(\mathbf{S}) + \zeta \lambda_{-}(\mathbf{S}))\left\|\boldsymbol{\Omega}\right\|_F^2.}
\end{equation}
\revadd{Finally, due to Proposition~\ref{prop:one_all1} and the fact that $\delta, \lambda_{-}$ and $\lambda_{+}$ are invariant under the action of $\mathbb{O}(d)$, we can replace $\mathbf{S}$ by $\mathbf{S}\mathbf{Q}$ for any $\mathbf{Q} \in \mathbb{O}(d)$, and the result follows.}

\proofof{Theorem~\ref{thm:non_deg_two_views_gen_setting}} The proof is divided into three parts specialized to the three conditions in the statement.

\noindent \underline{\textbf{Part 1}}. First note that $\mathbf{S} \in \mathcal{C}$ iff $\mathbf{S}_i^T[\mathbf{C}\mathbf{S}]_i = [\mathbf{C}\mathbf{S}]_i^T\mathbf{S}_i$ for $i = 1,2$ (see Eq.~(\ref{eq:crit_pts2})). Since
% \begin{align}
%     \mathbf{S}_1^T[\mathbf{C}\mathbf{S}]_1 - [\mathbf{C}\mathbf{S}]_1^T\mathbf{S}_1 &= \mathbf{S}_1^T\mathbf{B}_1\boldsymbol{\mathcal{L}}_{\Gamma}^\dagger \mathbf{B}_2^T\mathbf{S}_2 - \mathbf{S}_2^T\mathbf{B}_2\boldsymbol{\mathcal{L}}_{\Gamma}^\dagger \mathbf{B}_1^T\mathbf{S}_1\\
%     &= \mathbf{B}(\mathbf{S})_1\boldsymbol{\mathcal{L}}_{\Gamma}^\dagger \mathbf{B}(\mathbf{S})_2^T- \mathbf{B}(\mathbf{S})_2\boldsymbol{\mathcal{L}}_{\Gamma}^\dagger \mathbf{B}(\mathbf{S})_1^T,
% \end{align}
\begin{equation}
   \mathbf{S}_1^T[\mathbf{C}\mathbf{S}]_1 - [\mathbf{C}\mathbf{S}]_1^T\mathbf{S}_1 =  [\mathbf{C}\mathbf{S}]_2^T\mathbf{S}_2 - \mathbf{S}_2^T[\mathbf{C}\mathbf{S}]_2 = \mathbf{B}(\mathbf{S})_1\boldsymbol{\mathcal{L}}_{\Gamma}^\dagger \mathbf{B}(\mathbf{S})_2^T- \mathbf{B}(\mathbf{S})_2\boldsymbol{\mathcal{L}}_{\Gamma}^\dagger \mathbf{B}(\mathbf{S})_1^T,
\end{equation}
thus $\mathbf{S} \in \mathcal{C}$ iff $\mathbf{B}(\mathbf{S})_1\boldsymbol{\mathcal{L}}_{\Gamma}^\dagger \mathbf{B}(\mathbf{S})_2^T$ is symmetric. At this point, we note that
%the forms of $\mathbf{B}(\mathbf{S})_{1}$ and $\mathbf{B}(\mathbf{S})_{2}$ are
\begin{equation}
    \begin{matrix}
        \mathbf{B}(\mathbf{S})_1 & = & [ & \mathbf{X}_1 & \mathbf{0}_{d \times n_2} & \mathbf{X}_3 & -(\mathbf{X}_1\mathbf{1}_{n_1} + \mathbf{X}_3\mathbf{1}_{n_3}) & \mathbf{0}_{d} & ]\\
        \mathbf{B}(\mathbf{S})_2 & = & [ & \mathbf{0}_{d \times n_1} & \mathbf{Y}_2 & \mathbf{Y}_3 & \mathbf{0}_{d} & -(\mathbf{Y}_2\mathbf{1}_{n_2} + \mathbf{Y}_3\mathbf{1}_{n_3}) & ]
    \end{matrix}
\end{equation}
where (see Remark~\ref{rmk:L0DB}) $\mathbf{X}_1 \in \mathbb{R}^{d \times n_1}$ and $\mathbf{X}_3 \in \mathbb{R}^{d \times n_3}$ correspond to the local coordinates, due to the first view, of the $n_1$ points that lie exclusively in the first view and the $n_3$ points that lie on the overlap of both views, respectively. Similarly, $\mathbf{Y}_2 \in \mathbb{R}^{d \times n_2}$ and $\mathbf{Y}_3 \in \mathbb{R}^{d \times n_3}$ correspond to the local coordinates, due to the second view, of the $n_2$ points that lie exclusively in the second view and the $n_3$ points which lie on the overlap of both views, respectively. In particular, $\mathbf{X}_3 = \mathbf{B}(\mathbf{S})_{1,2}$ and $\mathbf{Y}_3 = \mathbf{B}(\mathbf{S})_{2,1}$ (perhaps after permuting the points). Moreover,
\begin{equation}
    \boldsymbol{\mathcal{L}}_{\Gamma} = \begin{bmatrix}
    \mathbf{I}_{n_1} & & & -\mathbf{1}_{n_1} & \mathbf{0}_{n_1}\\
     & \mathbf{I}_{n_2} & & \mathbf{0}_{n_2} & -\mathbf{1}_{n_2}\\
    &  & 2\mathbf{I}_{n_3} & -\mathbf{1}_{n_3} & -\mathbf{1}_{n_3}\\
    -\mathbf{1}_{n_1}^T & \mathbf{0}_{n_2}^T & -\mathbf{1}_{n_3}^T & n_1+n_3 & \\
    \mathbf{0}_{n_1}^T & -\mathbf{1}_{n_2}^T & -\mathbf{1}_{n_3}^T &  & n_2+n_3\end{bmatrix}.
\end{equation}
Through simple calculations, we obtain
\begin{equation}
    \boldsymbol{\mathcal{L}}_{\Gamma}^\dagger = \frac{1}{2n_3}\begin{bmatrix}
    2n_3\mathbf{I}_{n_1} + \mathbf{1}_{n_1}\mathbf{1}_{n_1}^T & -\mathbf{1}_{n_1}\mathbf{1}_{n_2}^T & & \mathbf{1}_{n_1} & -\mathbf{1}_{n_1}\\
    -\mathbf{1}_{n_2}\mathbf{1}_{n_1}^T & 2n_3\mathbf{I}_{n_2} + \mathbf{1}_{n_2}\mathbf{1}_{n_2}^T & & -\mathbf{1}_{n_2} & \mathbf{1}_{n_2}\\
    &  & n_3\mathbf{I}_{n_3} &  & \\
    \mathbf{1}_{n_1}^T & -\mathbf{1}_{n_2}^T &  & 1 & -1\\
    -\mathbf{1}_{n_1}^T & \mathbf{1}_{n_2}^T & &  -1 & 1\end{bmatrix}.
\end{equation}
Thus,
$$\mathbf{B}(\mathbf{S})_1\boldsymbol{\mathcal{L}}_{\Gamma}^\dagger = \begin{bmatrix}\mathbf{X}_1 - \frac{\mathbf{X}_3\mathbf{1}_{n_3}\mathbf{1}_{n_1}^T}{2n_3}, & \frac{\mathbf{X}_3\mathbf{1}_{n_3}\mathbf{1}_{n_2}^T}{2n_3}, & \frac{1}{2}\mathbf{X}_3, & -\frac{\mathbf{X}_3\mathbf{1}_{n_3}}{2n_3}, & \frac{\mathbf{X}_3\mathbf{1}_{n_3}}{2n_3} \end{bmatrix}.$$
Then, using Definition~\ref{def:BSicapj}
\begin{align}
    \mathbf{B}(\mathbf{S})_1\boldsymbol{\mathcal{L}}_{\Gamma}^\dagger \mathbf{B}(\mathbf{S})_2^T &= \frac{1}{2}\mathbf{X}_3\left(\mathbf{I}_{n_3}-\frac{1}{n_3}\mathbf{1}_{n_3}\mathbf{1}_{n_3}^T\right)\mathbf{Y}_3^T\\
    &=\frac{1}{2}\mathbf{B}(\mathbf{S})_{1,2}\left(\mathbf{I}_{n_3}-\frac{1}{n_3}\mathbf{1}_{n_3}\mathbf{1}_{n_3}^T\right)\left(\mathbf{I}_{n_3}-\frac{1}{n_3}\mathbf{1}_{n_3}\mathbf{1}_{n_3}^T\right)^T\mathbf{B}(\mathbf{S})_{2,1}^T\\
    &= \frac{1}{2}\overline{\mathbf{B}(\mathbf{S})}_{1,2}\overline{\mathbf{B}(\mathbf{S})}_{2,1}^T
\end{align}
% \begin{align}
%     &\mathbf{B}(\mathbf{S})_2\boldsymbol{\mathcal{L}}_{\Gamma}^\dagger \mathbf{B}(\mathbf{S})_1^T = \mathbf{B}(\mathbf{S})_1\boldsymbol{\mathcal{L}}_{\Gamma}^\dagger \mathbf{B}(\mathbf{S})_2^T
%     %&= \begin{bmatrix}\mathbf{X}_1 - \frac{\mathbf{X}_3\mathbf{1}_{n_3}\mathbf{1}_{n_1}^T}{2n_3} & \frac{\mathbf{X}_3\mathbf{1}_{n_3}\mathbf{1}_{n_2}^T}{2n_3} & \frac{1}{2}\mathbf{X}_3 & -\frac{\mathbf{X}_3\mathbf{1}_{n_3}}{2n_3} & \frac{\mathbf{X}_3\mathbf{1}_{n_3}}{2n_3} \end{bmatrix}\begin{bmatrix}0 \\ \mathbf{Y}_2^T \\ \mathbf{Y}_3^T \\ 0 \\ -(\mathbf{1}_{n_2}^T\mathbf{Y}_2^T + \mathbf{1}_{n_3}^T\mathbf{Y}_3^T)\end{bmatrix}\\
%     = \frac{1}{2}\mathbf{X}_3\left(\mathbf{I}_{n_3}-\frac{1}{n_3}\mathbf{1}_{n_3}\mathbf{1}_{n_3}^T\right)\mathbf{Y}_3^T\\
%     &= \frac{1}{2}\mathbf{B}(\mathbf{S})_{1,2}\left(\mathbf{I}_{n_3}-\frac{1}{n_3}\mathbf{1}_{n_3}\mathbf{1}_{n_3}^T\right)\left(\mathbf{I}_{n_3}-\frac{1}{n_3}\mathbf{1}_{n_3}\mathbf{1}_{n_3}^T\right)^T\mathbf{B}(\mathbf{S})_{2,1}^T = \frac{1}{2}\overline{\mathbf{B}(\mathbf{S})}_{1,2}\overline{\mathbf{B}(\mathbf{S})}_{2,1}^T.
% \end{align}
Since $\mathbf{S} \in \mathcal{C}$ iff $\mathbf{B}(\mathbf{S})_1\boldsymbol{\mathcal{L}}_{\Gamma}^\dagger \mathbf{B}(\mathbf{S})_2^T$ is symmetric, implies $\mathbf{S} \in \mathcal{C}$ iff $\overline{\mathbf{B}(\mathbf{S})}_{1,2}\overline{\mathbf{B}(\mathbf{S})}_{2,1}^T$ is symmetric. 

\noindent \underline{\textbf{Part 2}}. For $\mathbf{S} \in \mathcal{C}$, from the Remark~\ref{rmk:C_hat_L_structure} and Part 1, we have,
% \begin{align}
%     \mathbf{L}(\mathbf{S}) &= \begin{bmatrix}
%     -\mathbf{B}(\mathbf{S})_1\boldsymbol{\mathcal{L}}_{\Gamma}^\dagger \mathbf{B}(\mathbf{S})_2^T & \mathbf{B}(\mathbf{S})_1\boldsymbol{\mathcal{L}}_{\Gamma}^\dagger \mathbf{B}(\mathbf{S})_2^T\\
%     \mathbf{B}(\mathbf{S})_2\boldsymbol{\mathcal{L}}_{\Gamma}^\dagger \mathbf{B}(\mathbf{S})_1^T & -\mathbf{B}(\mathbf{S})_2\boldsymbol{\mathcal{L}}_{\Gamma}^\dagger \mathbf{B}(\mathbf{S})_1^T
%     \end{bmatrix}\\
%     &= \begin{bmatrix}
%     -\mathbf{B}(\mathbf{S})_1\boldsymbol{\mathcal{L}}_{\Gamma}^\dagger \mathbf{B}(\mathbf{S})_2^T & \mathbf{B}(\mathbf{S})_1\boldsymbol{\mathcal{L}}_{\Gamma}^\dagger \mathbf{B}(\mathbf{S})_2^T\\
%     \mathbf{B}(\mathbf{S})_1\boldsymbol{\mathcal{L}}_{\Gamma}^\dagger \mathbf{B}(\mathbf{S})_2^T & -\mathbf{B}(\mathbf{S})_1\boldsymbol{\mathcal{L}}_{\Gamma}^\dagger \mathbf{B}(\mathbf{S})_2^T
%     \end{bmatrix}. \label{supp:eq:LS_two_views}
% \end{align}
\begin{equation}
    \mathbf{L}(\mathbf{S}) = \begin{bmatrix}
    \mathbf{B}(\mathbf{S})_1\boldsymbol{\mathcal{L}}_{\Gamma}^\dagger \mathbf{B}(\mathbf{S})_2^T & -\mathbf{B}(\mathbf{S})_1\boldsymbol{\mathcal{L}}_{\Gamma}^\dagger \mathbf{B}(\mathbf{S})_2^T\\
    -\mathbf{B}(\mathbf{S})_1\boldsymbol{\mathcal{L}}_{\Gamma}^\dagger \mathbf{B}(\mathbf{S})_2^T & \mathbf{B}(\mathbf{S})_1\boldsymbol{\mathcal{L}}_{\Gamma}^\dagger \mathbf{B}(\mathbf{S})_2^T
    \end{bmatrix}. \label{supp:eq:LS_two_views}
\end{equation}
Let $\boldsymbol{\Omega}_1,\boldsymbol{\Omega}_2 \in \Skew(d)$ such that $\boldsymbol{\Omega}_1 + \boldsymbol{\Omega}_2 = 0$. Then, using the above equations, $$\Tr\left(\begin{bmatrix}
        \boldsymbol{\Omega}_1^T & \boldsymbol{\Omega}_2^T
    \end{bmatrix} \mathbf{L}(\mathbf{S}) \begin{bmatrix}
        \boldsymbol{\Omega}_1\\\boldsymbol{\Omega}_2
    \end{bmatrix}\right) = \Tr\left(\begin{bmatrix}
        -\boldsymbol{\Omega}_1 & \boldsymbol{\Omega}_1
    \end{bmatrix} \mathbf{L}(\mathbf{S}) \begin{bmatrix}
        \boldsymbol{\Omega}_1\\-\boldsymbol{\Omega}_1
    \end{bmatrix}\right) = 2 \Tr(\boldsymbol{\Omega}_1^T\overline{\mathbf{B}(\mathbf{S})}_{1,2}\overline{\mathbf{B}(\mathbf{S})}_{2,1}^T\boldsymbol{\Omega}_1).$$
% \begin{align}
%     \Tr\left(\begin{bmatrix}
%         \boldsymbol{\Omega}_1^T & \boldsymbol{\Omega}_2^T
%     \end{bmatrix} \mathbf{L}(\mathbf{S}) \begin{bmatrix}
%         \boldsymbol{\Omega}_1\\\boldsymbol{\Omega}_2
%     \end{bmatrix}\right) &= \Tr\left(\begin{bmatrix}
%         -\boldsymbol{\Omega}_1 & \boldsymbol{\Omega}_1
%     \end{bmatrix} \mathbf{L}(\mathbf{S}) \begin{bmatrix}
%         \boldsymbol{\Omega}_1\\-\boldsymbol{\Omega}_1\\
%     \end{bmatrix}\right)\\
%     &= -2 \Tr(\boldsymbol{\Omega}_1^T\overline{\mathbf{B}(\mathbf{S})}_{1,2}\overline{\mathbf{B}(\mathbf{S})}_{2,1}^T\boldsymbol{\Omega}_1).
% \end{align}
Combining the above and Part 1 with Proposition~\ref{prop:HessFSZZ}, we conclude that $\pi(\mathbf{S})$ is a local minimum of $\widetilde{F}$ iff the first two conditions of the statement are met.

\noindent \underline{\textbf{Part 3}}. Here we deal with the non-degeneracy of $\widetilde{\mathbf{S}} = \pi(\mathbf{S})$. For $d=1$, $\widetilde{\mathbf{S}}$ is trivially non-degenerate. So we assume that $d \geq 2$. From Part 2, we note that for $\boldsymbol{\Omega}_1,\boldsymbol{\Omega}_2 \in \Skew(d)$ such that $\boldsymbol{\Omega}_1 + \boldsymbol{\Omega}_2 = 0$, $\mathbf{L}(\mathbf{S})[\boldsymbol{\Omega}_i]_1^2 = 0$ iff $\overline{\mathbf{B}(\mathbf{S})}_{1,2}\overline{\mathbf{B}(\mathbf{S})}_{2,1}^T\boldsymbol{\Omega} = 0$. Thus $\widetilde{\mathbf{S}}$ is non-degenerate iff $\overline{\mathbf{B}(\mathbf{S})}_{1,2}\overline{\mathbf{B}(\mathbf{S})}_{2,1}^T\boldsymbol{\Omega} = 0 \iff \boldsymbol{\Omega} = 0$. It suffices to show that $\overline{\mathbf{B}(\mathbf{S})}_{1,2}\overline{\mathbf{B}(\mathbf{S})}_{2,1}^T\boldsymbol{\Omega} = 0$  iff $\rank (\overline{\mathbf{B}(\mathbf{S})}_{1,2}\overline{\mathbf{B}(\mathbf{S})}_{2,1}^T) \geq d-1$.

($\impliedby$) Suppose $\rank (\overline{\mathbf{B}(\mathbf{S})}_{1,2}\overline{\mathbf{B}(\mathbf{S})}_{2,1}^T) \geq d-1$ then null space of $\overline{\mathbf{B}(\mathbf{S})}_{1,2}\overline{\mathbf{B}(\mathbf{S})}_{2,1}^T$ is at most one-dimensional. Moreover, rank of a non-zero skew symmetric matrix of size $d \geq 2$, is at least two. Thus $\overline{\mathbf{B}(\mathbf{S})}_{1,2}\overline{\mathbf{B}(\mathbf{S})}_{2,1}^T\boldsymbol{\Omega} = 0 \iff \boldsymbol{\Omega} = 0$. We conclude that
%$\mathbf{L}(\mathbf{S})$ has trivial certificates only, and thus
$\widetilde{\mathbf{S}}$ is non-degenerate.

$(\implies)$ Suppose $\rank (\overline{\mathbf{B}(\mathbf{S})}_{1,2}\overline{\mathbf{B}(\mathbf{S})}_{2,1}^T) \leq d-2$, then there exist non-zero vectors $\mathbf{u},\mathbf{v} \in \mathbb{R}^d$ in the kernel of $\overline{\mathbf{B}(\mathbf{S})}_{1,2}\overline{\mathbf{B}(\mathbf{S})}_{2,1}^T$ such that $\mathbf{u}^T\mathbf{v} = 0$. Let $\boldsymbol{\Omega} = \mathbf{u}\mathbf{v}^T - \mathbf{v}\mathbf{u}^T$ then clearly $\boldsymbol{\Omega} \in \Skew(d)$, $\boldsymbol{\Omega} \neq 0$ and $\overline{\mathbf{B}(\mathbf{S})}_{1,2}\overline{\mathbf{B}(\mathbf{S})}_{2,1}^T\boldsymbol{\Omega} = 0$. 

% \proofof{Theorem~\ref{thm:uniq_two_views_gen_setting}}
% Let $\mathbf{S}$ be an optimal alignment (a global minimum of $F$), then from proof of Theorem~\ref{thm:non_deg_two_views_gen_setting}, $\mathbf{C}(\mathbf{S})_{1,2} = -\mathbf{B}(\mathbf{S})_1\boldsymbol{\mathcal{L}}_{\Gamma}^\dagger \mathbf{B}(\mathbf{S})_2^T = -\frac{1}{2}\mathbf{S}_1^T\overline{\mathbf{B}}_{1,2}\overline{\mathbf{B}}_{2,1}^T\mathbf{S}_2$.
% % \begin{align}
% %     \mathbf{C}(\mathbf{S})_{1,2} = -\mathbf{B}(\mathbf{S})_1\boldsymbol{\mathcal{L}}_{\Gamma}^\dagger \mathbf{B}(\mathbf{S})_2^T = -\frac{1}{2}\mathbf{S}_1^T\overline{\mathbf{B}}_{1,2}\overline{\mathbf{B}}_{2,1}^T\mathbf{S}_2 .%= \frac{1}{2}\mathbf{S}_2^T\overline{\mathbf{B}}_{2,1}\overline{\mathbf{B}}_{1,2}^T\mathbf{S}_1 = \mathbf{B}(\mathbf{S})_2\boldsymbol{\mathcal{L}}_{\Gamma}^\dagger \mathbf{B}(\mathbf{S})_1^T = \mathbf{C}(\mathbf{S})_{21}.
% % \end{align}
% Then note that the minimizer of $\Tr(\mathbf{C}\mathbf{S}\mathbf{S}^T)$ is same as the minimizer of $\Tr(\mathbf{C}(\mathbf{S})_{1,2})$ because $\mathbf{S}_1\mathbf{S}_1^T = \mathbf{S}_2\mathbf{S}_2^T = \mathbf{I}_d$ and $\mathbf{C}_{2,1} = \mathbf{C}_{1,2}^T$ is symmetric. WLOG, take $\mathbf{S}_1 = \mathbf{I}_d$. Then it suffices to show that $\textstyle\min_{\mathbf{S}_2\in\mathbb{O}(d)}\Tr(\mathbf{C}(\mathbf{S})_{12}) = -0.5\textstyle\max_{\mathbf{S}_2\in\mathbb{O}(d)} \Tr(\overline{\mathbf{B}}_{1,2}\overline{\mathbf{B}}_{2,1}^T\mathbf{S}_2)$ has a unique solution iff $\rank (\overline{\mathbf{B}}_{1,2}\overline{\mathbf{B}}_{2,1}^T) = d$.

% Let $\overline{\mathbf{B}}_{1,2}\overline{\mathbf{B}}_{2,1}^T = \mathbf{U}\mathbf{\Sigma}\mathbf{V}^T$ be its singular value decomposition. Note that $\mathbf{\Sigma} \succeq 0$. If $\rank (\overline{\mathbf{B}}_{1,2}\overline{\mathbf{B}}_{2,1}^T) = d$ then $\mathbf{\Sigma} \succ 0$ and clearly, the unique optimizer is $\mathbf{S}_2=\mathbf{V}\mathbf{U}^T$. If $\rank (\overline{\mathbf{B}}_{1,2}\overline{\mathbf{B}}_{2,1}^T) \leq d-1$ then there exist orthogonal matrix $\mathbf{U}' \neq \mathbf{I}_d$ such that $\boldsymbol{\Sigma} = \mathbf{U}'\boldsymbol{\Sigma}$ and thus $\mathbf{V}\mathbf{U}^T \neq \mathbf{V}(\mathbf{U}'\mathbf{U})^T$ are both optimizers of the above objective. 

%%%%%%%%%%%%%%%%%%%%%%%%%%%%%%%%%%%%%%%%%%
% Section 4 Proofs
%%%%%%%%%%%%%%%%%%%%%%%%%%%%%%%%%%%%%%%%%%
\proofof{Proposition~\ref{prop:noiseless_setting1}}
Since $\mathbf{S}$ is a perfect alignment $F(\mathbf{S}) = \Tr(\mathbf{C}\mathbf{S}\mathbf{S}^T) = 0$. Since $\mathbf{C} \succeq 0$, the columns of $\mathbf{S}$ lie in the kernel of $\mathbf{C}$. In particular $\mathbf{C}\mathbf{S} = \mathbf{0}$. It follows that $\widehat{\mathbf{C}}(\mathbf{S}) = \mathbf{0}$ (see Eq.~(\ref{eq:C_hat})). We conclude that $\mathbf{L}(\mathbf{S}) = \mathbf{C}(\mathbf{S})$.

\proofof{Proposition~\ref{prop:non_deg_views}} \revadd{WLOG assume that each views is centered at the origin i.e. $\mathbf{B}_{i,i}\mathbf{1}_{n_i} = 0$ where $n_i$ is the number of points in the $i$th view. Due to Assumption~\ref{assump:non_deg_views}, the matrix $\mathbf{B}_{i,i}\mathbf{B}_{i,i}^T$ has a rank of $d$ and consequently $\sigma_{\min}(\mathbf{B}_{i,i}\mathbf{B}_{i,i}^T) > 0$. Let $\Theta(\mathbf{S})_i$ and $\Theta(\mathbf{O})_i$ be the realizations of the points in the $i$th views i.e. of $\mathbf{B}_{i,i}$, due to the perfect alignments $\mathbf{S}$ and $\mathbf{O}$, respectively. Also, the optimal translation of the $i$th view due to $\mathbf{S}$ is given by $\mathbf{S}^T\mathbf{B}\boldsymbol{\mathcal{L}}_{\Gamma}^\dagger\mathbf{e}^{n+m}_{n+i}$ and the translation of all the views so that $\Theta(\mathbf{S})$ is centered at the origin is given by $-\mathbf{S}^T\mathbf{B}\boldsymbol{\mathcal{L}}_{\Gamma}^\dagger\mathbf{1}^{n+m}_{n}$. For brevity, define the vector $\mathbf{v}_i\coloneqq \mathbf{B}\boldsymbol{\mathcal{L}}_{\Gamma}^\dagger(\mathbf{e}^{n+m}_{n+i}-\mathbf{1}^{n+m}_{n})$. The net translation for the $i$th view due to $\mathbf{S}$ is the sum of the two translations $\mathbf{S}^T\mathbf{v}_i$ (similarly for $\mathbf{O}$). Then the result follows from,}
\begin{align}
    &\left\|\Theta(\mathbf{S}) - \Theta(\mathbf{O})\right\|^2_F \geq \left\|\Theta(\mathbf{S})_i - \Theta(\mathbf{O})_i\right\|^2_F\\
    &= \left\|(\mathbf{S}_i^T\mathbf{B}_{i,i} + \mathbf{S}^T\mathbf{v}_i\mathbf{1}_{n_i}^T) -(\mathbf{O}_i^T\mathbf{B}_{i,i} + \mathbf{O}^T\mathbf{v}_i\mathbf{1}_{n_i}^T)  \right\|_F^2\\
    &= \left\|(\mathbf{S}_i-\mathbf{O}_i)^T\mathbf{B}_{i,i}\right\|_F^2 + \left\|(\mathbf{S}-\mathbf{O})^T\mathbf{v}_i\mathbf{1}_{n_i}^T\right\|_F^2 - 2\Tr((\mathbf{S}_i-\mathbf{O}_i)^T\mathbf{B}_{i,i}\mathbf{1}_{n_i}\mathbf{v}_i^T(\mathbf{S}-\mathbf{O}))\\
    &\geq \left\|(\mathbf{S}_i-\mathbf{O}_i)^T\mathbf{B}_{i,i}\right\|_F^2 \geq \left\|\mathbf{S}_i-\mathbf{O}_i\right\|_F^2\sigma_{\min}(\mathbf{B}_{i,i}\mathbf{B}_{i,i}^T).
\end{align}

\proofof{Theorem~\ref{thm:inf_rigid}}
\revadd{Suppose $\mathbf{S}$ is degenerate then, due to Theorem~\ref{prop:noiseless_setting1} and Definition~\ref{def:LScertificate}, there exist $\boldsymbol{\Omega}$ such that $\mathbf{C}(\mathbf{S}) \boldsymbol{\Omega} = 0$. Since $\mathbf{C}(\mathbf{S}) \succeq 0$ therefore $\Tr(\mathbf{C}(\mathbf{S}) \boldsymbol{\Omega}\boldsymbol{\Omega}^T) = 0$. Following Eq.~(\ref{eq:opt_Z}), we set the perturbations to be $\mathbf{p}_k \coloneqq \boldsymbol{\Omega}^T\blockdiag(\mathbf{S}^T)\mathbf{B}\boldsymbol{\mathcal{L}}_{\Gamma}^\dagger \mathbf{e}^{n+m}_{k}$ for $k \in [1,n+m]$. Then, following Eq.~(\ref{eq:A_1_}) and Eq.~(\ref{eq:A_1}) in that order, and the fact that $\Tr(\mathbf{C}(\mathbf{S}) \boldsymbol{\Omega}\boldsymbol{\Omega}^T) = 0$,  we obtain $\mathbf{p}_{k} = \boldsymbol{\Omega}_i^T\mathbf{S}_i^T \mathbf{x}_{k,i} + \mathbf{p}_{n+i}$. Consequently, for $(k_1,i), (k_2,i)\in E(\Gamma)$, $\mathbf{p}_{k_1} - \mathbf{p}_{k_2} = \boldsymbol{\Omega}_i^T\mathbf{S}_i^T (\mathbf{x}_{k_1,i}-\mathbf{x}_{k_2,i})$. Similarly, since $\mathbf{S}$ is a perfect alignment, $\mathbf{x}_{k_1}(\mathbf{S}) - \mathbf{x}_{k_2}(\mathbf{S}) = \mathbf{S}_i^T (\mathbf{x}_{k_1,i}-\mathbf{x}_{k_2,i})$. Finally, $(\mathbf{p}_{k_1} - \mathbf{p}_{k_2})^T(\mathbf{x}_{k_1}(\mathbf{S}) - \mathbf{x}_{k_2}(\mathbf{S})) = (\mathbf{x}_{k_1,i}-\mathbf{x}_{k_2,i})^T\mathbf{S}_i\boldsymbol{\Omega}_i\mathbf{S}_i^T (\mathbf{x}_{k_1,i}-\mathbf{x}_{k_2,i}) = 0$ since $\Tr(\boldsymbol{\Omega}_i \mathbf{A}) = 0$ for any symmetric matrix $\mathbf{A}$.}

\revadd{Now suppose $\Theta(\mathbf{S}) = (\mathbf{x}_k(\mathbf{S}))_1^n$ is not infinitesimally rigid. From Definition~\ref{def:inf_rigid}, there exist a non-trivial perturbation $(\mathbf{p}_k)_1^n$ such that $(\mathbf{p}_{k_1}-\mathbf{p}_{k_2})^T(\mathbf{x}_{k_1}(\mathbf{S})-\mathbf{x}_{k_2}(\mathbf{S})) = 0$ for all $(k_1,i), (k_2,i) \in E(\Gamma)$. Combining this with the Assumption~\ref{assump:non_deg_views} that each view is affinely non-degenerate, it follows from  \citea{schulze2010symmetric, asimow1978rigidity} that for each $i \in [1,m]$ there exist $\boldsymbol{\Omega}_i \in \Skew(d)$ and and $\mathbf{t}_i \in \mathbf{R}^d$ such that for each $(k,i) \in E(\Gamma)$, $\mathbf{p}_k = \boldsymbol{\Omega}_i^T\mathbf{x}_k(\mathbf{S}) + \mathbf{t}_i$. Therefore, following Eq.(\ref{eq:A_1}, \ref{eq:A_1_}, \ref{eq:GPOP}), we conclude that $\mathbf{C}(\mathbf{S})\boldsymbol{\Omega} = 0$. From Proposition~\ref{prop:noiseless_setting1} and Proposition~\ref{prop:non_deg_triv_cert}, it suffices to show that $\boldsymbol{\Omega}$ is non-trivial i.e. not all $\boldsymbol{\Omega}_i$'s are equal. In fact, since $\Gamma$ is connected (Assumption~\ref{assump:connected_gamma}) and for $(k,i), (k,j) \in E(\Gamma)$, $\mathbf{p}_k = \boldsymbol{\Omega}_i^T\mathbf{x}_k(\mathbf{S}) + \mathbf{t}_i = \boldsymbol{\Omega}_j^T\mathbf{x}_k(\mathbf{S}) + \mathbf{t}_j$, therefore if $\boldsymbol{\Omega}_i = \boldsymbol{\Omega}_j$ for all $i,j \in [1,m]$ then $\mathbf{t}_i = \mathbf{t}_j$ too. As a result, the perturbation $(\mathbf{p}_k)_1^n$ ends up being trivial, a contradiction.}

\proofof{Theorem~\ref{thm:loc_rigid}}
First, under Assumption~\ref{assump:non_deg_views}, the equation in Definition~\ref{def:loc_rigid}, $\Theta(\mathbf{O}) = \Theta(\mathbf{S}\mathbf{Q})$, is equivalent to $\pi(\mathbf{S}) = \pi(\mathbf{O})$.

($\impliedby$) Suppose $\mathbf{S}$ is a perfect alignment but $\pi(\mathbf{S})$ is not a strict global minimum of $\widetilde{F}$. 
%Thus $\pi(\mathbf{S})$ is degenerate (see Definition~\ref{def:non_deg_alignment0}).
Define 
$$\eta \coloneqq \left\|\mathbf{B}\boldsymbol{\mathcal{L}}_{\Gamma}^\dagger(:,1:n)\left(\mathbf{I}_n - n^{-1}\mathbf{1}_{n}\mathbf{1}_{n}^T\right)\right\|_F$$
and let $\epsilon > 0$ be arbitrary.
% \begin{align}
%     \eta \coloneqq \left\|\mathbf{B}\boldsymbol{\mathcal{L}}_{\Gamma}^\dagger(:,1:n)\left(\mathbf{I}_n - n^{-1}\mathbf{1}_{n}\mathbf{1}_{n}^T\right)\right\|_F.
% \end{align}
Then there exists another perfect alignment $\mathbf{O} \in \mathbb{O}(d)^m$ such that $\left\|\mathbf{S}-\mathbf{O}\right\|_F < \epsilon / \eta$ and $\pi(\mathbf{S}) \neq \pi(\mathbf{O})$. Due to Assumption~\ref{assump:non_deg_views}, we have $\Theta(\mathbf{O}) \neq \Theta(\mathbf{S}\mathbf{Q})$ for any $\mathbf{Q} \in \mathbb{O}(d)$, however, from Definition~\ref{def:realization},
$$\left\|\Theta(\mathbf{O})-\Theta(\mathbf{S})\right\|_F \leq \left\|\mathbf{S}-\mathbf{O}\right\|_F\left\|\mathbf{B}\boldsymbol{\mathcal{L}}_{\Gamma}^\dagger(:,1:n)\left(\mathbf{I}_n - n^{-1}\mathbf{1}_{n}\mathbf{1}_{n}^T\right)\right\|_F = \eta\left\|\mathbf{S}-\mathbf{O}\right\|_F < \epsilon.$$
% \begin{equation}
%     \left\|\Theta(\mathbf{O})-\Theta(\mathbf{S})\right\|_F \leq \left\|\mathbf{S}-\mathbf{O}\right\|_F\left\|\mathbf{B}\boldsymbol{\mathcal{L}}_{\Gamma}^\dagger(:,1:n)\left(\mathbf{I}_n - n^{-1}\mathbf{1}_{n}\mathbf{1}_{n}^T\right)\right\|_F < \eta\left\|\mathbf{S}-\mathbf{O}\right\|_F < \epsilon.
% \end{equation}
Since $\epsilon$ is arbitrary, we conclude that $\Theta(\mathbf{S})$ is not locally rigid.

($\implies$) Let $\epsilon > 0$ be arbitrary. Suppose $\Theta(\mathbf{S})$ is not locally rigid, then there exist another perfect alignment $\mathbf{O}_\epsilon \in \mathbb{O}(d)^m$ such that $\left\|\Theta(\mathbf{O}_\epsilon)-\Theta(\mathbf{S})\right\|_F < \epsilon$ but $\Theta(\mathbf{O}_\epsilon) \neq \Theta(\mathbf{S}\mathbf{Q})$. \revadd{Due to Proposition~\ref{prop:non_deg_views}, $\left\|\mathbf{O}_\epsilon-\mathbf{S}\right\|_F < \varrho\epsilon$ where the constant $\varrho > 0$, but $\pi(\mathbf{O}_\epsilon) \neq \pi(\mathbf{S})$}. Since this true for all $\epsilon > 0$, we conclude that $\pi(\mathbf{S})$ is not a strict global minimum of $\widetilde{F}$.

\revadd{Finally, we note that a non-degenerate extremum is a strict extremum and the result follows.}

\proofof{Theorem~\ref{thm:nec_cond_loc_rigid_of_views}}
Consider a partition of $[1,m]$ into two non-empty subsets $A$ and $B$. Suppose $\rank(\overline{\mathbf{B}(\mathbf{S})}_{A, B})$ is at most $d-2$. Let $\boldsymbol{\Omega}_0 \in \Skew(d)$ be such that $\boldsymbol{\Omega}_0 \neq 0$ and $\overline{\mathbf{B}(\mathbf{S})}_{A,B}^T\boldsymbol{\Omega}_0 = 0$ (its existence follows from the third part of the proof of Theorem~\ref{thm:non_deg_two_views_gen_setting}). WLOG assume that $\mathbf{B}(\mathbf{S})_{A,B}\mathbf{1}_{n'} = 0$ (here $n'$ is as in Definition~\ref{def:BSAcapB}) (perhaps by translating all aligned views by $-\mathbf{B}(\mathbf{S})_{A,B}\mathbf{1}_{n'}$). Then $\mathbf{B}(\mathbf{S})_{A,B}^T\boldsymbol{\Omega}_0 = 0$.

Let $\boldsymbol{\Omega} = [\boldsymbol{\Omega}_i]_1^m$ be such that $\boldsymbol{\Omega}_i = \boldsymbol{\Omega}_0$ for $i \in A$ and $\boldsymbol{\Omega}_i = -\boldsymbol{\Omega}_0$ for $i \in B$. Clearly, $\boldsymbol{\Omega} \in \Skew(d)^m$ such that not all $\boldsymbol{\Omega}_i$ are equal. It suffices to show that $\boldsymbol{\Omega}$ is a nontrivial certificate of $\mathbf{L}(\mathbf{S})$, equivalently $\Tr(\boldsymbol{\Omega}^T\mathbf{L}(\mathbf{S})\boldsymbol{\Omega}) = 0$. First we observe that for $i \in A$, 
$$[\mathbf{L}(\mathbf{S}) \boldsymbol{\Omega}]_i = (\mathbf{B}(\mathbf{S})_i\boldsymbol{\mathcal{L}}_{\Gamma}^\dagger \mathbf{B}(\mathbf{S})^T\mathbf{I}^m_d - \textstyle\sum_{1}^{m}(-1)^{\mathbf{1}_{B}(j)}\mathbf{B}(\mathbf{S})_i\boldsymbol{\mathcal{L}}_{\Gamma}^\dagger \mathbf{B}(\mathbf{S})_j^T)\boldsymbol{\Omega}_0 = 2 \textstyle\sum_{j \in B}\mathbf{B}(\mathbf{S})_i\boldsymbol{\mathcal{L}}_{\Gamma}^\dagger \mathbf{B}(\mathbf{S})_j^T \boldsymbol{\Omega}_0$$
where $\mathbf{1}_{B}(j) = 1$ iff $j \in B$.
% \begin{align}
%     [\mathbf{L}(\mathbf{S}) \boldsymbol{\Omega}]_i &= (-\mathbf{B}(\mathbf{S})_i\boldsymbol{\mathcal{L}}_{\Gamma}^\dagger \mathbf{B}(\mathbf{S})^T\mathbf{I}^m_d + \textstyle\sum_{j \in A}\mathbf{B}(\mathbf{S})_i\boldsymbol{\mathcal{L}}_{\Gamma}^\dagger \mathbf{B}(\mathbf{S})_j^T - \textstyle\sum_{j \in B}\mathbf{B}(\mathbf{S})_i\boldsymbol{\mathcal{L}}_{\Gamma}^\dagger \mathbf{B}(\mathbf{S})_j^T)\boldsymbol{\Omega}_0\\
%     &= -2 \textstyle\sum_{j \in B}\mathbf{B}(\mathbf{S})_i\boldsymbol{\mathcal{L}}_{\Gamma}^\dagger \mathbf{B}(\mathbf{S})_j^T \boldsymbol{\Omega}_0
% \end{align}
Similarly, for $i \in B$, 
$$[\mathbf{L}(\mathbf{S}) \boldsymbol{\Omega}]_i = -2 \textstyle\sum_{j \in A}\mathbf{B}(\mathbf{S})_i\boldsymbol{\mathcal{L}}_{\Gamma}^\dagger \mathbf{B}(\mathbf{S})_j^T \boldsymbol{\Omega}_0.$$ Denote by $\mathbf{B}_A$ and $\mathbf{B}_B$, the matrices $\textstyle\sum_{i \in A}\mathbf{B}(\mathbf{S})_i$ and $\textstyle\sum_{j \in B}\mathbf{B}(\mathbf{S})_j$, respectively. Thus,
\begin{equation}
    \Tr(\boldsymbol{\Omega}^T\mathbf{L}(\mathbf{S})\boldsymbol{\Omega}) = 4\Tr(\boldsymbol{\Omega}_0^T \mathbf{B}_A\boldsymbol{\mathcal{L}}_{\Gamma}^\dagger \mathbf{B}_B^T \boldsymbol{\Omega}_0). \label{supp:eq:Omega0TLSOmega0}
\end{equation}
% \begin{align}
%     \Tr(\boldsymbol{\Omega}^T\mathbf{L}(\mathbf{S})\boldsymbol{\Omega}) = -4\Tr\left(\boldsymbol{\Omega}_0^T \left(\textstyle\sum_{i \in A}\mathbf{B}(\mathbf{S})_i\right)\boldsymbol{\mathcal{L}}_{\Gamma}^\dagger \left(\textstyle\sum_{j \in B}\mathbf{B}(\mathbf{S})_j^T\right) \boldsymbol{\Omega}_0\right). \label{supp:eq:Omega0TLSOmega0}
% \end{align}

% Then, since $\mathbf{L}(\mathbf{S}) \preceq 0$,
% \begin{align}
%     \Tr(\boldsymbol{\Omega}_0^T \mathbf{B}_A\boldsymbol{\mathcal{L}}_{\Gamma}^\dagger \mathbf{B}_B^T \boldsymbol{\Omega}_0) \geq 0. \label{supp:eq:TrOmega0LOmega0_lb}
% \end{align}
We are going to show that the above evaluates to zero. WLOG assume that the first $n_1$ points lie in the views with indices in $A \setminus B$, next $n_2$ points  lie in the views with indices in $B \setminus A$ and the remaining $n_3$ points lie in the views with indices in $A \cap B$. Note that $n_1 + n_2 + n_3 = n$ and $|A| + |B| = m$. Then the matrices $\mathbf{B}_A$ and $\mathbf{B}_B$ (perhaps after permuting the views) have the following structure. 
\begin{align}
    \begin{matrix}
        \mathbf{B}_A & = & [ & \mathbf{X}_{1} & \mathbf{0}_{d \times n_2} & \mathbf{X}_{3} & \mathbf{U}_{1} + \mathbf{U}_{3} & \mathbf{0}_{d \times |B|} & ]\\
        \mathbf{B}_B & = & [ & \mathbf{0}_{d \times n_1} & \mathbf{Y}_{2} & \mathbf{Y}_{3} & \mathbf{0}_{d \times |A|} & \mathbf{V}_{2} + \mathbf{V}_{3} & ]
    \end{matrix}
\end{align}
where $\mathbf{X}_{1} \in \mathbb{R}^{d \times n_1}$ and $\mathbf{Y}_{2} \in \mathbb{R}^{d \times n_2}$ contain the sum of the local coordinates of the $n_1$ and $n_2$ points, respectively. Also, $\mathbf{X}_{3} \in \mathbb{R}^{d \times n_3}$ and $\mathbf{Y}_{3} \in \mathbb{R}^{d \times n_3}$ contain the sum of the local coordinates of the remaining $n_3$ points due to the views with indices in $A$ and $B$ respectively. The matrices $\mathbf{U}_{1} \in \mathbb{R}^{d \times |A|}$, $\mathbf{V}_{2} \in \mathbb{R}^{d \times |B|}$, $\mathbf{U}_{3} \in \mathbb{R}^{d \times |A|}$ and $\mathbf{V}_{3} \in \mathbb{R}^{d \times |B|}$ follow from Remark~\ref{rmk:L0DB}. Further define
\begin{equation}
    \begin{matrix}
         \mathbf{B}_{A \setminus B} & \coloneqq & [ & \mathbf{X}_{1} & 0 & 0 & \mathbf{U}_{1} & 0 & ]\\
         \mathbf{B}_{A,B} & \coloneqq & [ & 0 & 0 & \mathbf{X}_{3} & \mathbf{U}_{3} & 0 & ]\\
         \mathbf{B}_{B \setminus A} & \coloneqq & [ & 0 & \mathbf{Y}_{2} & 0 & 0 & \mathbf{V}_{2} & ]\\
         \mathbf{B}_{B,A} & \coloneqq & [ & 0 & 0& \mathbf{Y}_{3} & 0 & \mathbf{V}_{3} & ]
    \end{matrix} \label{supp:eq:eq8}
\end{equation}
then $\mathbf{B}_A = \mathbf{B}_{A \setminus B} + \mathbf{B}_{A,B}$ and $\mathbf{B}_B = \mathbf{B}_{B \setminus A} + \mathbf{B}_{B,A}$. Note that $\mathbf{1}_{n+m}$ lies in the kernel of the four matrices defined above (see Remark~\ref{rmk:L0DB}) and, $\mathbf{B}_{A \setminus B}\mathbf{B}_{B\setminus A}^T = 0$, $\mathbf{B}_{A \setminus B}\mathbf{B}_{B,A}^T = 0$ and $\mathbf{B}_{B \setminus A}\mathbf{B}_{A,B}^T = 0$. Now, the structure of $\boldsymbol{\mathcal{L}}_{\Gamma}$ is as follows,
\begin{equation}
    \boldsymbol{\mathcal{L}}_{\Gamma} = \begin{bmatrix}
        \boldsymbol{\mathcal{D}}_{1} & & & -\boldsymbol{\mathcal{K}}_{1} & \\
        & \boldsymbol{\mathcal{D}}_{2} & &  & -\boldsymbol{\mathcal{K}}_{2}\\
        & &\boldsymbol{\mathcal{D}}_{3} + \boldsymbol{\mathcal{D}}_{3} & -\boldsymbol{\mathcal{K}}_{A_3} & -\boldsymbol{\mathcal{K}}_{B_3}\\
        -\boldsymbol{\mathcal{K}}_{1}^T & & -\boldsymbol{\mathcal{K}}_{A_3}^T & \boldsymbol{\mathcal{D}}_{A} & \\
        & -\boldsymbol{\mathcal{K}}_{2}^T & -\boldsymbol{\mathcal{K}}_{B_3}^T & & \boldsymbol{\mathcal{D}}_{B}
    \end{bmatrix}
\end{equation}
where $\boldsymbol{\mathcal{K}}_{1} \in \mathbb{R}^{n_1 \times |A|}$ is the adjacency between the first $n_1$ points and the views with indices in $A$, $\boldsymbol{\mathcal{K}}_{2} \in \mathbb{R}^{n_2 \times |B|}$ is the adjacency between the next $n_2$ points and the views with indices in $B$, and $\boldsymbol{\mathcal{K}}_{A_3} \in \mathbb{R}^{n_3 \times |A|}$ and $\boldsymbol{\mathcal{K}}_{B_3} \in \mathbb{R}^{n_3 \times |B|}$ are the adjacencies between the remaining $n_3$ points and the views with indices in $A$ and $B$ respectively. As for the remaining matrices, $\boldsymbol{\mathcal{D}}_{i} = \diag(\boldsymbol{\mathcal{K}}_{i}\mathbf{1}_{|A|})$ represents the degrees of the points in the bipartite adjacency $\boldsymbol{\mathcal{K}}_{i}$, $\boldsymbol{\mathcal{D}}_{A} = \diag(\boldsymbol{\mathcal{K}}_{1}^T\mathbf{1}_{n_1} + \boldsymbol{\mathcal{K}}_{A_3}^T\mathbf{1}_{n_3})$ represents the degree of the views i.e. the number of points contained in the views with indices in $A$, Similarly, $\boldsymbol{\mathcal{D}}_{j} = \diag(\boldsymbol{\mathcal{K}}_{j}\mathbf{1}_{|B|})$. and $\boldsymbol{\mathcal{D}}_{B} = \diag(\boldsymbol{\mathcal{K}}_{2}^T\mathbf{1}_{n_2} + \boldsymbol{\mathcal{K}}_{B_3}^T\mathbf{1}_{n_3})$. 

Since $\mathbf{S}$ is a perfect alignment, the local coordinates of a point due to the views are the same. Using the fact that $\mathbf{B}(\mathbf{S})_{A,B}$ represent the local coordinates of the $n_3$ points contained in views with indices in $A \cap B$ (see Definition~\ref{def:BSAcapB}), we obtain
\begin{align}
    \begin{matrix}
        \mathbf{X}_{3} &=& \mathbf{B}(\mathbf{S})_{A,B}\boldsymbol{\mathcal{D}}_{3}, & & \mathbf{Y}_{3} &=& \mathbf{B}(\mathbf{S})_{A,B}\boldsymbol{\mathcal{D}}_{3},\\
        \mathbf{U}_{3} &=& -\mathbf{B}(\mathbf{S})_{A,B}\boldsymbol{\mathcal{K}}_{A_3},& &
        \mathbf{V}_{3} &=& -\mathbf{B}(\mathbf{S})_{A,B}\boldsymbol{\mathcal{K}}_{B_3}.
    \end{matrix}\label{supp:eq:eq9}
\end{align} 
Similarly, it follows that $\mathbf{U}_1 = \mathbf{X}_1\boldsymbol{\mathcal{D}}_{1}^{-1}\boldsymbol{\mathcal{K}}_{1}$ and $\mathbf{V}_2 = \mathbf{Y}_2\boldsymbol{\mathcal{D}}_{2}^{-1}\boldsymbol{\mathcal{K}}_{2}$. Thus,
\begin{align}
    \begin{matrix}
         \mathbf{B}_{A \setminus B} & = & [ & \mathbf{X}_{1}\boldsymbol{\mathcal{D}}_{1}^{-1} & \mathbf{0}_{d \times n_2} & \mathbf{0}_{d \times n_3} & \mathbf{0}_{d \times |A|} & \mathbf{0}_{d \times |B|} & ]\boldsymbol{\mathcal{L}}_{\Gamma}\\
         \mathbf{B}_{B \setminus A} & = & [ & \mathbf{0}_{d \times n_1} & \mathbf{Y}_{2}\boldsymbol{\mathcal{D}}_{2}^{-1} & \mathbf{0}_{d \times n_3} & \mathbf{0}_{d \times |A|} & \mathbf{0}_{d \times |B|} & ]\boldsymbol{\mathcal{L}}_{\Gamma}
    \end{matrix}\label{supp:eq:eq14_} \text{ and }
\end{align}
% A final observation regarding the matrix $\mathbf{B}_{A \setminus B}$ and $\mathbf{B}_{B \setminus A}$ in Eq.~(\ref{supp:eq:eq8}) is that it is the analog of $\mathbf{S}^T\mathbf{B}$ in Eq.~(\ref{eq:opt_Z}) when the local coordinates of the last $n_2 + n_3$ points due to the views containing them are forced to be zeros, while the local coordinates of the first $n_1$ points are the same as above and in particular (just like Eq.~(\ref{supp:eq:eq9})) equal $\mathbf{X}_{1}\boldsymbol{\mathcal{D}}_{1}^{-1}$. Similar premise holds for $\mathbf{B}_{B \setminus A}$. Since, even after forcing certain local coordinates to be zeros, the views are perfectly aligned, using Eq.~(\ref{eq:opt_Z}) and Eq.~(\ref{eq:H}), it follows that
%Subsequently,
\begin{align}
    \begin{matrix}
         \mathbf{B}_{A \setminus B}\boldsymbol{\mathcal{L}}_{\Gamma}^\dagger & = & [ & \mathbf{X}_{1}\boldsymbol{\mathcal{D}}_{1}^{-1} & \mathbf{0}_{d \times n_2} & \mathbf{0}_{d \times n_3} & \mathbf{0}_{d \times |A|} & \mathbf{0}_{d \times |B|} & ] + \mathbf{t}_{A \setminus B}\mathbf{1}_{n+m}^T\\
         \mathbf{B}_{B \setminus A}\boldsymbol{\mathcal{L}}_{\Gamma}^\dagger & = & [ & \mathbf{0}_{d \times n_1} & \mathbf{Y}_{2}\boldsymbol{\mathcal{D}}_{2}^{-1} & \mathbf{0}_{d \times n_3} & \mathbf{0}_{d \times |A|} & \mathbf{0}_{d \times |B|} & ] +  \mathbf{t}_{B \setminus A}\mathbf{1}_{n+m}^T
         %(\mathbf{B}_{A,B}+\mathbf{B}_{B,A})\boldsymbol{\mathcal{L}}_{\Gamma}^\dagger & = & [ & 0 & 0 & \mathbf{B}(\mathbf{S})_{A,B} & 0 & 0 & ] & + & \mathbf{t}\mathbf{1}_{n+m}^T\\
    \end{matrix}\label{supp:eq:eq14}
\end{align}
for some translation vectors $\mathbf{t}_{A \setminus B}, \mathbf{t}_{B \setminus A} \in \mathbb{R}^d$. Since $\mathbf{1}_{n+m}$ lies in $\ker(\mathbf{B}_{A \setminus B})$ and $\ker(\mathbf{B}_{A \setminus B})$ (see the paragraph after Eq.~(\ref{supp:eq:eq8})), thus
\begin{equation}
    \mathbf{B}_{A \setminus B}\boldsymbol{\mathcal{L}}_{\Gamma}^\dagger \mathbf{B}_{B\setminus A}^T = \mathbf{B}_{A \setminus B}\boldsymbol{\mathcal{L}}_{\Gamma}^\dagger \mathbf{B}_{B,A}^T = \mathbf{B}_{B \setminus A}\boldsymbol{\mathcal{L}}_{\Gamma}^\dagger \mathbf{B}_{A,B}^T = 0. \label{supp:eq:eq11}
\end{equation}

% Finally, note that since $\overline{\mathbf{B}(\mathbf{S})}_{A, B}^T\boldsymbol{\Omega}_0 = 0$, we have $\mathbf{B}(\mathbf{S})_{A, B}^T\boldsymbol{\Omega}_0 = \mathbf{1}_{n_3}\mathbf{v}^T$ for some $\mathbf{v} \in \mathbb{R}^d$.
% Let $\boldsymbol{\mathcal{D}}_{A} \in \mathbb{R}^{|A| \times |A|}$ and $\boldsymbol{\mathcal{D}}_{B} \in \mathbb{R}^{|B| \times |B|}$ be the diagonal matrices whose $i$th element on the diagonal is the number of points in the $i$th and $(|A|+i)$th view, respectively. Then,
% \begin{align}
%     \boldsymbol{\Omega}_0^T\mathbf{X}_{3} &= \mathbf{v}\mathbf{1}_{n_3}^T\boldsymbol{\mathcal{D}}_{3}\\
%     \boldsymbol{\Omega}_0^T\mathbf{Y}_{3} &= \mathbf{v}\mathbf{1}_{n_3}^T\boldsymbol{\mathcal{D}}_{3}\\
%     \boldsymbol{\Omega}_0^T\mathbf{U}_{3} &= -\mathbf{v}\mathbf{1}_{|A|}^T\boldsymbol{\mathcal{D}}_{A}\\
%     -\boldsymbol{\Omega}_0^T\mathbf{V}_{3} &= \mathbf{v}\mathbf{1}_{|B|}^T\boldsymbol{\mathcal{D}}_{B}
% \end{align}
% and
% \begin{align}
%     \begin{matrix}
%         \boldsymbol{\Omega}_0^T\mathbf{B}_{A,B} & = & [ & 0 & 0 & \mathbf{v}\mathbf{1}_{n_3}^T\boldsymbol{\mathcal{D}}_{3} & -\mathbf{v}\mathbf{1}_{|A|}^T\boldsymbol{\mathcal{D}}_{A} & 0 & ]\\
%         \boldsymbol{\Omega}_0^T\mathbf{B}_{B,A} & = & [ & 0 & 0 & \mathbf{v}\mathbf{1}_{n_3}^T\boldsymbol{\mathcal{D}}_{3} &  0 & -\mathbf{v}\mathbf{1}_{|B|}^T\boldsymbol{\mathcal{D}}_{B} & ]
%     \end{matrix}
% \end{align}
% Using the above equation and the fact that $\mathbf{1}_n \in \ker(\boldsymbol{\mathcal{L}}_{\Gamma}^\dagger)$ and $\boldsymbol{\mathcal{L}}_{\Gamma}^\dagger \succeq 0$, we obtain
% \begin{align}
%     \Tr(\boldsymbol{\Omega}_0^T \mathbf{B}_A\boldsymbol{\mathcal{L}}_{\Gamma}^\dagger \mathbf{B}_B^T \boldsymbol{\Omega}_0) &= \Tr(\boldsymbol{\Omega}_0^T (\mathbf{B}_{A \setminus B}+\mathbf{B}_{A,B})\boldsymbol{\mathcal{L}}_{\Gamma}^\dagger (\mathbf{B}_{B\setminus A} + \mathbf{B}_{B,A})^T \boldsymbol{\Omega}_0)\\
%     &= \Tr(\boldsymbol{\Omega}_0^T \mathbf{B}_{A,B}\boldsymbol{\mathcal{L}}_{\Gamma}^\dagger\mathbf{B}_{B,A}^T \boldsymbol{\Omega}_0)\\
%     &\leq \Tr(\boldsymbol{\Omega}_0^T (\mathbf{B}_{A,B} + \mathbf{B}_{B,A})\boldsymbol{\mathcal{L}}_{\Gamma}^\dagger(\mathbf{B}_{A,B} + \mathbf{B}_{B,A})^T \boldsymbol{\Omega}_0)\\
%     &= \Tr(\begin{bmatrix} 0 & 0 & \boldsymbol{\Omega}_0^T\mathbf{B}(\mathbf{S})_{A,B} & 0 & 0\end{bmatrix}(\mathbf{B}_{A,B} + \mathbf{B}_{B,A})^T \boldsymbol{\Omega}_0)\\
%     &= 0
% \end{align}
% where the last equation follows from the fact that $\mathbf{B}(\mathbf{S})_{A,B}^T\boldsymbol{\Omega}_0 = 0$. Combining the above equation with Eq.~(\ref{eq:TrOmega0LOmega0_lb}), we conclude that $\boldsymbol{\Omega}$ is a non-trivial certificate of $\mathbf{L}(\mathbf{S})$ and thus $\pi(\mathbf{S})$ is degenerate.

Since $\mathbf{B}(\mathbf{S})_{A,B}^T\boldsymbol{\Omega}_0 = 0$ (by assumption), combining with Eq.~(\ref{supp:eq:eq9}, \ref{supp:eq:eq8}) yields $\mathbf{B}_{A,B}^T\boldsymbol{\Omega}_0 = 0$ and $ \mathbf{B}_{B,A}^T\boldsymbol{\Omega}_0 = 0$. Substituting the above and Eq.~(\ref{supp:eq:eq11}) into Eq.~(\ref{supp:eq:Omega0TLSOmega0}), we obtain 
$$\Tr(\boldsymbol{\Omega}_0^T \mathbf{B}_A\boldsymbol{\mathcal{L}}_{\Gamma}^\dagger \mathbf{B}_B^T \boldsymbol{\Omega}_0) = \Tr(\boldsymbol{\Omega}_0^T (\mathbf{B}_{A \setminus B}+\mathbf{B}_{A,B})\boldsymbol{\mathcal{L}}_{\Gamma}^\dagger (\mathbf{B}_{B\setminus A} + \mathbf{B}_{B,A})^T \boldsymbol{\Omega}_0)
    = \Tr(\boldsymbol{\Omega}_0^T \mathbf{B}_{A,B}\boldsymbol{\mathcal{L}}_{\Gamma}^\dagger\mathbf{B}_{B,A}^T \boldsymbol{\Omega}_0) = 0.$$
We conclude that $\boldsymbol{\Omega}$ is a non-trivial certificate of $\mathbf{L}(\mathbf{S})$ and thus $\pi(\mathbf{S})$ is degenerate.

\proofof{Lemma~\ref{lem:subproblem_cert}} 
WLOG, let the $m$th vertex be removed. Let $\Gamma$ be the bipartite graph representing the correspondence between $m$ views and $n$ vertices, as described in Section~\ref{sec:setup}. Let $\Gamma_{-}$ be the bipartite graph obtained after the removal of the vertices representing the $m$th view and the points which lie exclusively in it. Let $\mathbf{D} \in \mathbb{R}^{md \times md}$, $\mathbf{B} \in \mathbb{R}^{md \times (n+m)}$, $\mathbf{D}_{-} \in \mathbb{R}^{(m-1)d\times (m-1)d}$ and $\mathbf{B}_{-} \in \mathbb{R}^{(m-1)d \times (n_1+n_2+m-1)}$ be the matrices defined in Remark~\ref{rmk:L0DB} for graphs $\Gamma$ and $\Gamma_{-}$. Also, let
\begin{itemize}[leftmargin=*]
    \item $\boldsymbol{\mathcal{K}}_1 \in \mathbb{R}^{n_1 \times (m-1)}$ is the bipartite adjacency matrix between the first $m-1$ views and the $n_1$ points which lie exclusively in them. Note that the adjacency between such points and the $m$th view is $\mathbf{0}_{n_1}$.
    \item $\boldsymbol{\mathcal{K}}_2 \in \mathbb{R}^{n_2 \times (m-1)}$ is the bipartite adjacency between the first $m-1$ views and the $n_2$ points which lie on the overlap of the $m$th view and the union of the first $m-1$ views. Note that the adjacency between such points and the $m$th view is $\mathbf{1}_{n_2}$. Also note that since $\Gamma$ is connected by Assumption~\ref{assump:connected_gamma}, $n_2 > 0$.
    \item the fifth and the third column in $\boldsymbol{\mathcal{L}}_{\Gamma}$ correspond to the  $m$th view and the $n_3$ points that lie exclusively in it, respectively. The adjacency between such points and the first $m-1$ views is $\mathbf{0}_{n_3 \times (m-1)}$, and that with the $m$th view is $\mathbf{1}_{n_3}$.
    \item $\boldsymbol{\mathcal{D}}_1 = \diag (\boldsymbol{\mathcal{K}}_1\mathbf{1}_{m-1})$, $\boldsymbol{\mathcal{D}}_2 = \diag (\boldsymbol{\mathcal{K}}_2\mathbf{1}_{m-1})$ and $\overline{\boldsymbol{\mathcal{D}}} = \diag (\boldsymbol{\mathcal{K}}_1^T\mathbf{1}_{n_1} + \boldsymbol{\mathcal{K}}_2^T\mathbf{1}_{n_2})$.
\end{itemize}
Then the structure of the combinatorial Laplacian of $\Gamma$ and $\Gamma_{-}$ are
\begin{equation}
    \boldsymbol{\mathcal{L}}_{\Gamma} = \begin{bmatrix}
        \boldsymbol{\mathcal{D}}_1 &  &  & -\boldsymbol{\mathcal{K}}_1 & \mathbf{0}_{n_1}\\
         & \boldsymbol{\mathcal{D}}_2 + \mathbf{I}_{n_2} &  & -\boldsymbol{\mathcal{K}}_2 & -\mathbf{1}_{n_2}\\
         &  & \mathbf{I}_{n_3} & \mathbf{0}_{n_3 \times (m-1)} & -\mathbf{1}_{n_3}\\
        -\boldsymbol{\mathcal{K}}_1^T & -\boldsymbol{\mathcal{K}}_2^T & \mathbf{0}_{n_3}^T & \overline{\boldsymbol{\mathcal{D}}} & \mathbf{0}_{m-1}\\
        \mathbf{0}_{n_1}^T & -\mathbf{1}_{n_2}^T & -\mathbf{1}_{n_3}^T & \mathbf{0}_{m-1}^T & n_2+n_3
    \end{bmatrix}
\end{equation}
and
$\boldsymbol{\mathcal{L}}_{\Gamma_{-}} = \begin{bsmallmatrix}
    \boldsymbol{\mathcal{D}}_1 &  & -\boldsymbol{\mathcal{K}}_1\\
     & \boldsymbol{\mathcal{D}}_2 & -\boldsymbol{\mathcal{K}}_2\\
    -\boldsymbol{\mathcal{K}}_1^T & -\boldsymbol{\mathcal{K}}_2^T & \overline{\boldsymbol{\mathcal{D}}}
    \end{bsmallmatrix}$.
Using a permutation matrix
% \begin{align}
%     \boldsymbol{\mathcal{P}}_{0} &= \begin{bmatrix}
%         \mathbf{I}_{n_1} &  &  &  & \\
%         & \mathbf{I}_{n_2} &  &  &  \\
%         & & & \mathbf{I}_{n_3}& \\
%         & & \mathbf{I}_{m-1}& & \\
%         & & & & 1
%     \end{bmatrix}
% \end{align}
%\begin{align}
$\boldsymbol{\mathcal{P}}_{0} = \begin{bsmallmatrix}
        \mathbf{I}_{n_1+n_2} &  &  & \\
        & & \mathbf{I}_{n_3}& \\
        & \mathbf{I}_{m-1}& & \\
        & & & 1
    \end{bsmallmatrix}$
%\end{align}
and a diagonal matrix $\boldsymbol{\mathcal{D}}_0 = \diag ((\mathbf{0}_{n_1},\mathbf{1}_{n_2},\mathbf{0}_{m-1}))$, we obtain
% \begin{align}
%     \boldsymbol{\mathcal{P}}_{0}\boldsymbol{\mathcal{L}}_{\Gamma}\boldsymbol{\mathcal{P}}_{0}^T &= \begin{bmatrix}\mathbf{A}_{11}&\mathbf{A}_{12}\\\mathbf{A}_{21}&\mathbf{A}_{22}\end{bmatrix} = \begin{bmatrix}
%         &&&\mathbf{0}_{n_1 \times n_3} & \mathbf{0}_{n_1}\\
%         & \boldsymbol{\mathcal{L}}_{\Gamma_{-}} + \boldsymbol{\mathcal{D}}_0 & & \mathbf{0}_{n_2 \times n_3} & -\mathbf{1}_{n_2}\\
%         &&& \mathbf{0}_{(m-1) \times n_3} & \mathbf{0}_{m-1}\\
%         \mathbf{0}_{n_1 \times n_3}^T & \mathbf{0}_{n_2 \times n_3}^T & \mathbf{0}_{(m-1) \times n_3}^T & \mathbf{I}_{n_3} & -\mathbf{1}_{n_3}\\
%         \mathbf{0}_{n_1}^T & -\mathbf{1}_{n_2}^T & \mathbf{0}_{m-1}^T & -\mathbf{1}_{n_3}^T & n_2+n_3
%     \end{bmatrix} \label{supp:eq:eq15}
% \end{align}
\begin{equation}
    \boldsymbol{\mathcal{P}}_{0}\boldsymbol{\mathcal{L}}_{\Gamma}\boldsymbol{\mathcal{P}}_{0}^T = \begin{bmatrix}\mathbf{A}_{11}&\mathbf{A}_{12}\\\mathbf{A}_{21}&\mathbf{A}_{22}\end{bmatrix} = \begin{bmatrix}
        &&& \mathbf{0}_{n_1} & \mathbf{0}_{n_1}\\
        & \boldsymbol{\mathcal{L}}_{\Gamma_{-}} + \boldsymbol{\mathcal{D}}_0 & & \mathbf{0}_{n_2} & -\mathbf{1}_{n_2}\\
        &&& \mathbf{0}_{m-1} & \mathbf{0}_{m-1}\\
        \mathbf{0}_{n_1}^T & \mathbf{0}_{n_2}^T & \mathbf{0}_{m-1}^T & \mathbf{I}_{n_3} & -\mathbf{1}_{n_3}\\
        \mathbf{0}_{n_1}^T & -\mathbf{1}_{n_2}^T & \mathbf{0}_{m-1}^T & -\mathbf{1}_{n_3}^T & n_2+n_3
    \end{bmatrix} \label{supp:eq:eq15}
\end{equation}

The rest is divided into three parts. First, we derive the pseudoinverse of the above block matrix using \citeb[Section 3.6.2]{gentle2007matrix}. Then we show that $\mathbf{S}_{-} \coloneqq \mathbf{S}_{-m}$ is a perfect alignment of the $m-1$ views and finally we show that $[\boldsymbol{\Omega}_i]_{1}^{m-1}$ is a certificate of $\mathbf{L}_{-}(\mathbf{S}_{-}) \coloneqq \mathbf{L}_{-m}(\mathbf{S}_{-m})$ when $[\boldsymbol{\Omega}_i]_{1}^{m}$ is a certificate of $\mathbf{L}(\mathbf{S})$.

\noindent \underline{\textbf{Part 1}}. Here we derive the pseudoinverse of the matrix in Eq.~(\ref{supp:eq:eq15}). First, we note
\begin{prop}
\label{supp:prop:LplusD0}
$\boldsymbol{\mathcal{L}}_{\Gamma_{-}} + \boldsymbol{\mathcal{D}}_0 \succ 0$.
\end{prop}
\textit{Proof}. Since $\boldsymbol{\mathcal{L}}_{\Gamma_{-}} \succeq 0$ and $\boldsymbol{\mathcal{D}}_0 \succeq 0$, it suffices to show that $\ker (\boldsymbol{\mathcal{L}}_{\Gamma_{-}}) \cap \ker (\boldsymbol{\mathcal{D}}_0) = \{0\}$. Recall that the $m$th view contains $n_2 + n_3$ points where $n_2>0$ points lie on the overlap of $m$th view and the union of first $m-1$ views, and $n_3$ points lie exclusively in the $m$th view. Removal of the $m$th view and the $n_3$ points that lie exclusively in it may disconnect $\Gamma$ to produce $\Gamma_{-}$ with at most $n_2$ connected components. The vectors $\mathbf{u}_i$ with ones at the indices of the vertices in the $i$th component and zeros elsewhere, form an orthogonal basis of $\ker(\boldsymbol{\mathcal{L}}_{\Gamma_{-}})$. Since there exists at least one $k \in [n_1+1, n_1+n_2]$ with $\mathbf{u}_i(k) = 1$, thus $\mathbf{u}_i^T\boldsymbol{\mathcal{D}}_0\mathbf{u}_i > 0$. Also, for $i \neq j$, $\mathbf{u}_i^T\boldsymbol{\mathcal{D}}_0\mathbf{u}_j = 0$. The result follows.~$\blacksquare$

Since $\boldsymbol{\mathcal{L}}_{\Gamma_{-}} + \boldsymbol{\mathcal{D}}_0 \succ 0$, thus $(\boldsymbol{\mathcal{L}}_{\Gamma_{-}}+\boldsymbol{\mathcal{D}}_0)^\dagger = (\boldsymbol{\mathcal{L}}_{\Gamma_{-}}+\boldsymbol{\mathcal{D}}_0)^{-1}$ and
\begin{equation}
    (\boldsymbol{\mathcal{L}}_{\Gamma_{-}}+\boldsymbol{\mathcal{D}}_0)  \begin{bmatrix}\mathbf{1}_{n_1}\\ \mathbf{1}_{n_2} \\ \mathbf{1}_{m-1}\end{bmatrix} =  \begin{bmatrix}\mathbf{0}_{n_1}\\ \mathbf{1}_{n_2} \\ \mathbf{0}_{m-1}\end{bmatrix} \implies (\boldsymbol{\mathcal{L}}_{\Gamma_{-}}+\boldsymbol{\mathcal{D}}_0)^\dagger \begin{bmatrix}\mathbf{0}_{n_1}\\ \mathbf{1}_{n_2} \\ \mathbf{0}_{m-1}\end{bmatrix} = \begin{bmatrix}\mathbf{1}_{n_1}\\ \mathbf{1}_{n_2} \\ \mathbf{1}_{m-1}\end{bmatrix}. \label{supp:eq:L_Gamma_minus__plus_D_0_1s}
\end{equation}
Using the above equation, the matrix $\mathbf{Z} \coloneqq [(\boldsymbol{\mathcal{P}}_{0}\boldsymbol{\mathcal{L}}_{\Gamma}\boldsymbol{\mathcal{P}}_{0}^T)^\dagger]_{22} = \mathbf{A}_{22} - \mathbf{A}_{21}\mathbf{A}_{11}^\dagger \mathbf{A}_{12}$ and its pseudoinverse are, $\mathbf{Z} = \begin{bmatrix}
    \mathbf{I}_{n_3} & -\mathbf{1}_{n_3}\\
    -\mathbf{1}_{n_3}^T & n_3
\end{bmatrix}$ and $\mathbf{Z}^\dagger = \begin{bmatrix}
    \mathbf{I}_{n_3}  & \mathbf{0}_{n_3}\\
    \mathbf{0}_{n_3}^T & 0
\end{bmatrix}$. Next, we have 
$$[(\boldsymbol{\mathcal{P}}_{0}\boldsymbol{\mathcal{L}}_{\Gamma}\boldsymbol{\mathcal{P}}_{0}^T)^\dagger]_{11} = (\boldsymbol{\mathcal{L}}_{\Gamma_{-}}+\boldsymbol{\mathcal{D}}_0)^\dagger + ((\boldsymbol{\mathcal{L}}_{\Gamma_{-}}+\boldsymbol{\mathcal{D}}_0)^\dagger \mathbf{A}_{12})\mathbf{Z}^{\dagger}(\mathbf{A}_{21}(\boldsymbol{\mathcal{L}}_{\Gamma_{-}}+\boldsymbol{\mathcal{D}}_0)^\dagger).$$
Using Eq.~(\ref{supp:eq:eq15}, \ref{supp:eq:L_Gamma_minus__plus_D_0_1s}),  we obtain,
\begin{align}
[(\boldsymbol{\mathcal{P}}_{0}\boldsymbol{\mathcal{L}}_{\Gamma}\boldsymbol{\mathcal{P}}_{0}^T)^\dagger]_{11} &= (\boldsymbol{\mathcal{L}}_{\Gamma_{-}}+\boldsymbol{\mathcal{D}}_0)^\dagger. \label{supp:eq:pinvA11}\\
[(\boldsymbol{\mathcal{P}}_{0}\boldsymbol{\mathcal{L}}_{\Gamma}\boldsymbol{\mathcal{P}}_{0}^T)^\dagger]_{12} &= -(\mathbf{A}_{11}^\dagger \mathbf{A}_{12}) \mathbf{Z}^\dagger = 0\\
[(\boldsymbol{\mathcal{P}}_{0}\boldsymbol{\mathcal{L}}_{\Gamma}\boldsymbol{\mathcal{P}}_{0}^T)^\dagger]_{21} &= 0.
\end{align}
Thus,
$$(\boldsymbol{\mathcal{P}}_{0}\boldsymbol{\mathcal{L}}_{\Gamma}\boldsymbol{\mathcal{P}}_{0}^T)^\dagger = \blockdiag((\boldsymbol{\mathcal{L}}_{\Gamma_{-}}+\boldsymbol{\mathcal{D}}_0)^\dagger, \mathbf{I}_{n_3}, 0).$$
% \begin{align}
%     \mathbf{Z} &= \begin{bmatrix}
%         \mathbf{I}_{n_3} & -\mathbf{1}_{n_3}\\
%         -\mathbf{1}_{n_3}^T & n_3
%     \end{bmatrix} \implies  \mathbf{Z}^\dagger = \begin{bmatrix}
%         \mathbf{I}_{n_3}  & \mathbf{0}_{n_3}\\
%         \mathbf{0}_{n_3}^T & 0
%     \end{bmatrix}.
% \end{align}
% \begin{align}
% &[(\boldsymbol{\mathcal{P}}_{0}\boldsymbol{\mathcal{L}}_{\Gamma}\boldsymbol{\mathcal{P}}_{0}^T)^\dagger]_{11} = (\boldsymbol{\mathcal{L}}_{\Gamma_{-}}+\boldsymbol{\mathcal{D}}_0)^\dagger + ((\boldsymbol{\mathcal{L}}_{\Gamma_{-}}+\boldsymbol{\mathcal{D}}_0)^\dagger \mathbf{A}_{12})\mathbf{Z}^{\dagger}(\mathbf{A}_{21}(\boldsymbol{\mathcal{L}}_{\Gamma_{-}}+\boldsymbol{\mathcal{D}}_0)^\dagger)\\
%     &= (\boldsymbol{\mathcal{L}}_{\Gamma_{-}}+\boldsymbol{\mathcal{D}}_0)^\dagger + \begin{bsmallmatrix}\mathbf{0}_{n_1 \times n_3} & \mathbf{1}_{n_1}\\ \mathbf{0}_{n_2 \times n_3} & \mathbf{1}_{n_2} \\ \mathbf{0}_{(m-1) \times n_3} & \mathbf{1}_{m-1}\end{bsmallmatrix}\begin{bsmallmatrix}
%         \mathbf{I}_{n_3}  & \mathbf{0}_{n_3}\\
%         \mathbf{0}_{n_3}^T & 0
%     \end{bsmallmatrix} \begin{bsmallmatrix}\mathbf{0}_{n_1 \times n_3}^T & \mathbf{0}_{n_2 \times n_3}^T & \mathbf{0}_{(m-1) \times n_3}^T \\ \mathbf{1}_{n_1}^T & \mathbf{1}_{n_2}^T & \mathbf{1}_{m-1}\end{bsmallmatrix}\\
%     &= (\boldsymbol{\mathcal{L}}_{\Gamma_{-}}+\boldsymbol{\mathcal{D}}_0)^\dagger \label{supp:eq:pinvA11}
% \end{align}
% \begin{align}(\boldsymbol{\mathcal{P}}_{0}\boldsymbol{\mathcal{L}}_{\Gamma}\boldsymbol{\mathcal{P}}_{0}^T)^\dagger &= 
%     \begin{bmatrix}
%        \blockdiag((\boldsymbol{\mathcal{L}}_{\Gamma_{-}}+\boldsymbol{\mathcal{D}}_0)^\dagger  & & \\
%         & \mathbf{I}_{n_3}  & \\
%         & & 0
%     \end{bmatrix}.
% \end{align}

\noindent \underline{\textbf{Part 2}}. Now, let $\mathbf{S} = [\mathbf{S}_i]_1^{m}$ and $\mathbf{S}_{-} = [\mathbf{S}_i]_1^{m-1}$. Intuitively, it should be clear that $\mathbf{S}_{-}$ is a perfect alignment for the $m-1$ views. Since $\mathbf{S}$ is a perfect alignment, the alignment error (see Eq.~(\ref{eq:GPOP}))
$$\Tr(\mathbf{S}^T(\mathbf{D}-\mathbf{B}\boldsymbol{\mathcal{L}}_{\Gamma}^\dagger\mathbf{B}^T)\mathbf{S}) = 0.$$ We show that the error after the removal of the $m$th view is still zero i.e. 
$$\Tr(\mathbf{S}_{-}^T(\mathbf{D}_{-}-\mathbf{B}_{-}\boldsymbol{\mathcal{L}}_{\Gamma_{-}}\mathbf{B}_{-}^T)\mathbf{S}_{-}) = 0.$$
Let $\mathbf{B}^*_{1} \in \mathbb{R}^{d \times n_1}$, $\mathbf{B}^*_{2} \in \mathbb{R}^{d \times n_2}$ and $\mathbf{B}^*_{3} \in \mathbb{R}^{d \times n_3}$ contain the coordinates (after alignment with $\mathbf{S}$) of the $n_1$ points that lie exclusively in the first $m-1$ views, of the $n_2$ points that lie on the overlap of the $m$th view with the remaining views, and of the $n_3$ points that lie exclusively in the $m$th view, respectively. Then it suffices to show
\begin{prop}
\label{supp:prop:subproblem_perf_alignment}
(i) $\mathbf{S}^T\mathbf{B}\boldsymbol{\mathcal{L}}_{\Gamma}^\dagger\mathbf{B}^T\mathbf{S} =  \mathbf{S}_{-}^T\mathbf{B}_{-}\boldsymbol{\mathcal{L}}_{\Gamma_{-}}^\dagger \mathbf{B}_{-}^T\mathbf{S}_{-} + \mathbf{B}^*_{2}\mathbf{B}^{*^T}_{2} + \mathbf{B}^*_{3}\mathbf{B}^{*^T}_{3}$ and (ii) $\mathbf{S}^T\mathbf{D}\mathbf{S} = \mathbf{S}_{-}^T\mathbf{D}_{-}\mathbf{S}_{-} + \mathbf{B}^*_{2}\mathbf{B}^{*^T}_{2} + \mathbf{B}^*_{3}\mathbf{B}^{*^T}_{3}$. By taking the trace of the difference of these equations, the main result follows.
\end{prop}
\textit{Proof}. The second equation follows from Eq.~(\ref{eq:D}), Remark~\ref{rmk:L0DB} and the fact that, since the $m$ views are perfectly aligned, the local coordinates of the points are the same as those in the matrices $\mathbf{B}^*_{1}$,  $\mathbf{B}^*_{2}$ and  $\mathbf{B}^*_{3}$. We proceed to prove the first equation.

Since $\mathbf{\mathcal{P}}_0$ is a permutation matrix, $\mathbf{S}^T\mathbf{B}\boldsymbol{\mathcal{L}}_{\Gamma}^\dagger\mathbf{B}^T\mathbf{S} = (\mathbf{S}^T\mathbf{B}\boldsymbol{\mathcal{L}}_{\Gamma}^\dagger\boldsymbol{\mathcal{P}}_{0}^T)(\boldsymbol{\mathcal{P}}_{0}\mathbf{B}^T\mathbf{S})$. Then, using the same idea as in Eq.~(\ref{supp:eq:eq9}, \ref{supp:eq:eq14_}, \ref{supp:eq:eq14}) in the proof of Theorem~\ref{thm:nec_cond_loc_rigid_of_views}, we obtain
\begin{align}
    \mathbf{S}^T\mathbf{B} &= \begin{bmatrix}
        \mathbf{B}^*_{1}\boldsymbol{\mathcal{D}}_1 & \mathbf{B}^*_{2}(\boldsymbol{\mathcal{D}}_2 + \mathbf{I}_{n_2}) & \mathbf{B}^*_{3} & -(\mathbf{B}^*_{1}\boldsymbol{\mathcal{K}}_1+\mathbf{B}^*_{2}\boldsymbol{\mathcal{K}}_2) & -(\mathbf{B}^*_{2}\mathbf{1}_{n_2}+\mathbf{B}^*_{3}\mathbf{1}_{n_3})
    \end{bmatrix}\\
    &= \begin{bmatrix}
        \mathbf{B}^*_{1} & \mathbf{B}^*_{2} & \mathbf{B}^*_{3} & \mathbf{0}_{d \times (m-1)} & \mathbf{0}_{d}
    \end{bmatrix}\boldsymbol{\mathcal{L}}_{\Gamma}\label{supp:eq:STB}
\end{align}
and thus,
\begin{equation}
    \mathbf{S}^T\mathbf{B}\boldsymbol{\mathcal{L}}_{\Gamma}^\dagger = \begin{bmatrix}
        \mathbf{B}^*_{1} & \mathbf{B}^*_{2} & \mathbf{B}^*_{3} & \mathbf{0}_{d \times (m-1)} & \mathbf{0}_{d}
    \end{bmatrix} + \mathbf{t}\mathbf{1}_{n+m}^T \label{supp:eq:STBL_GammaBTS}
\end{equation}
for some translation vector $\mathbf{t} \in \mathbb{R}^d$. Similarly,
\begin{equation}
    \mathbf{S}_{-}^T\mathbf{B}_{-} = \begin{bmatrix}
        \mathbf{B}^*_{1}\boldsymbol{\mathcal{D}}_1 & \mathbf{B}^*_{2} \boldsymbol{\mathcal{D}}_2 & -(\mathbf{B}^*_{1}\boldsymbol{\mathcal{K}}_1+\mathbf{B}^*_{2}\boldsymbol{\mathcal{K}}_2)
    \end{bmatrix} = \begin{bmatrix}
        \mathbf{B}^*_{1} & \mathbf{B}^*_{2} & \mathbf{0}_{d \times (m-1)}
    \end{bmatrix}\boldsymbol{\mathcal{L}}_{\Gamma_{-}} \label{supp:eq:SmTBm1}
\end{equation}
and thus 
$$\mathbf{S}_{-}^T\mathbf{B}_{-}\boldsymbol{\mathcal{L}}_{\Gamma_{-}}^\dagger  = \begin{bmatrix}
        \mathbf{B}^*_{1} & \mathbf{B}^*_{2} & \mathbf{0}_{d \times (m-1)}
    \end{bmatrix} + \mathbf{t}_{-}\mathbf{v}_{-}^T$$    
for some translation vector $\mathbf{t}_{-} \in \mathbb{R}^d$ and $\mathbf{v}_{-} \in \ker(\boldsymbol{\mathcal{L}}_{\Gamma_{-}})$. From Proposition~\ref{prop:kerB}, $\ker(\boldsymbol{\mathcal{L}}_{\Gamma_{-}}) \subseteq \ker(\mathbf{B}_{-})$, therefore,
\begin{equation}
    \mathbf{S}_{-}^T\mathbf{B}_{-}\boldsymbol{\mathcal{L}}_{\Gamma_{-}}^\dagger \mathbf{B}_{-}^T\mathbf{S}_{-}  = \begin{bmatrix}
        \mathbf{B}^*_{1} & \mathbf{B}^*_{2} & \mathbf{0}_{d \times (m-1)}
    \end{bmatrix}\mathbf{B}_{-}^T\mathbf{S}_{-}. \label{supp:eq:eq20}
\end{equation}
Now, combining Eq.~(\ref{supp:eq:STB}) and Eq.~(\ref{supp:eq:SmTBm1}) we can write 
\begin{align}
    \mathbf{S}^T\mathbf{B}\boldsymbol{\mathcal{P}}_{0}^T &= \begin{bmatrix}
    \mathbf{B}^*_{1}\boldsymbol{\mathcal{D}}_1 & \mathbf{B}^*_{2}(\boldsymbol{\mathcal{D}}_2 + \mathbf{I}_{n_2}) & -(\mathbf{B}^*_{1}\boldsymbol{\mathcal{K}}_1+\mathbf{B}^*_{2}\boldsymbol{\mathcal{K}}_2)  & \mathbf{B}^*_{3} & -(\mathbf{B}^*_{2}\mathbf{1}_{n_2}+\mathbf{B}^*_{3}\mathbf{1}_{n_3})
\end{bmatrix}\\
    &= \begin{bmatrix}
    \mathbf{S}_{-}^T\mathbf{B}_{-} & \mathbf{0}_{d \times n_3} & \mathbf{0}_{d}
\end{bmatrix} + \begin{bmatrix}
    \mathbf{0}_{d \times n_1} & \mathbf{B}^*_{2} &  \mathbf{0}_{d \times (m-1)} &  \mathbf{B}^*_{3} & -(\mathbf{B}^*_{2}\mathbf{1}_{n_2}+\mathbf{B}^*_{3}\mathbf{1}_{n_3})
\end{bmatrix}.
\end{align}
% \begin{align}
%     &\mathbf{S}^T\mathbf{B}\boldsymbol{\mathcal{P}}_{0}^T = \begin{bmatrix}
%         \mathbf{B}^*_{1}\boldsymbol{\mathcal{D}}_1 & \mathbf{B}^*_{2}(\boldsymbol{\mathcal{D}}_2 + \mathbf{I}_{n_2}) & -(\mathbf{B}^*_{1}\boldsymbol{\mathcal{K}}_1+\mathbf{B}^*_{2}\boldsymbol{\mathcal{K}}_2)  & \mathbf{B}^*_{3} & -(\mathbf{B}^*_{2}\mathbf{1}_{n_2}+\mathbf{B}^*_{3}\mathbf{1}_{n_3})
%     \end{bmatrix}\\
%     &= \begin{bmatrix}
%         \mathbf{S}_{-}^T\mathbf{B}_{-} & \mathbf{0}_{d \times n_3} & \mathbf{0}_{d}
%     \end{bmatrix} + \begin{bmatrix}
%         \mathbf{0}_{d \times n_1} & \mathbf{B}^*_{2} &  \mathbf{0}_{d \times (m-1)} &  \mathbf{B}^*_{3} & -(\mathbf{B}^*_{2}\mathbf{1}_{n_2}+\mathbf{B}^*_{3}\mathbf{1}_{n_3})
%     \end{bmatrix}.
% \end{align}
Finally, due to Eq.~(\ref{supp:eq:STBL_GammaBTS}) and $\mathbf{1}_{n+m} \in \ker(\mathbf{S}^T\mathbf{B}\boldsymbol{\mathcal{P}}_{0}^T)$, the above equation and Eq.~(\ref{supp:eq:eq20}),
\begin{align}
    \mathbf{S}^T\mathbf{B}\boldsymbol{\mathcal{L}}_{\Gamma}^\dagger\mathbf{B}^T\mathbf{S} &= (\mathbf{S}^T\mathbf{B}\boldsymbol{\mathcal{L}}_{\Gamma}^\dagger\boldsymbol{\mathcal{P}}_{0}^T)(\boldsymbol{\mathcal{P}}_{0}\mathbf{B}^T\mathbf{S})\\
    &= \left(\begin{bmatrix}
    \mathbf{B}^*_{1} & \mathbf{B}^*_{2} &  \mathbf{0}_{d \times (m-1)} & \mathbf{B}^*_{3} & \mathbf{0}_{d}
\end{bmatrix} + \mathbf{t}\mathbf{1}_{n+m}^T\right)(\boldsymbol{\mathcal{P}}_{0}\mathbf{B}^T\mathbf{S})\\
&=\begin{bmatrix}
        \mathbf{B}^*_{1} & \mathbf{B}^*_{2} &  \mathbf{0}_{d \times (m-1)} & \mathbf{B}^*_{3} & \mathbf{0}_{d}
    \end{bmatrix}\begin{bmatrix}
        \mathbf{S}_{-}^T\mathbf{B}_{-} & \mathbf{0}_{d \times n_3} & \mathbf{0}_{d}
    \end{bmatrix}^T + \mathbf{B}^*_{2}\mathbf{B}^{*^T}_{2} + \mathbf{B}^*_{3}\mathbf{B}^{*^T}_{3}\\
    &=\mathbf{S}_{-}^T\mathbf{B}_{-}\boldsymbol{\mathcal{L}}_{\Gamma_{-}}^\dagger \mathbf{B}_{-}^T\mathbf{S}_{-} + \mathbf{B}^*_{2}\mathbf{B}^{*^T}_{2} + \mathbf{B}^*_{3}\mathbf{B}^{*^T}_{3},
\end{align}
\hfill $\blacksquare$

\noindent \underline{\textbf{Part 3}}. Now let $\mathbf{L}(\mathbf{S})$ and $\mathbf{L}_{-}(\mathbf{S}_{-})$ be the matrices, as described in Eq.~(\ref{eq:L_of_S}) for the two graphs $\Gamma$ and $\Gamma_{-}$ and the corresponding views. Let $\boldsymbol{\Omega} = [\boldsymbol{\Omega}_i]_1^m$ be a certificate of $\mathbf{L}(\mathbf{S})$. By Remark~\ref{rmk:C_hat_L_structure}, $\boldsymbol{\Omega}^{'} = \boldsymbol{\Omega} - [\boldsymbol{\Omega}_m]_1^m$ is also a certificate of $\mathbf{L}(\mathbf{S})$ and in particular $\boldsymbol{\Omega}^{'}_m = 0$. Define $\boldsymbol{\Omega}_{-} = [\boldsymbol{\Omega}^{'}_i]_1^{m-1}$. We are going to show that $\Tr(\boldsymbol{\Omega}_{-}^T\mathbf{L}_{-}(\mathbf{S}_{-})\boldsymbol{\Omega}_{-}) = 0$ i.e. $\boldsymbol{\Omega}_{-}$ is a certificate of $\mathbf{L}_{-}(\mathbf{S}_{-})$. Then using Remark~\ref{rmk:C_hat_L_structure}, it follows that $[\boldsymbol{\Omega}_i]_1^{m-1}$ (which equals $\boldsymbol{\Omega}_{-} + [\boldsymbol{\Omega}_m]_1^{m-1}$) is a certificate of $\mathbf{L}_{-}(\mathbf{S}_{-})$.
% One way to show that $\Tr(\boldsymbol{\Omega}_{-}^T\mathbf{L}_{-}(\mathbf{S}_{-})\boldsymbol{\Omega}_{-}) = 0$ is: define a curve $\mathbf{S}(t) \in \mathbb{O}(d)^m$ where $t \in [0,\epsilon)$ for small enough $\epsilon$ such that $\mathbf{S}'(0) = [\mathbf{S}_i\boldsymbol{\Omega}^{'}_i]_1^m$. Similarly, define a curve $\mathbf{S}_{-}(t) \in \mathbb{O}(d)^{m-1}$ where $t \in [0,\epsilon_{-})$ for small enough $\epsilon_{-}$ such that $\mathbf{S}_{-}'(0) = [\mathbf{S}_{-_i}\boldsymbol{\Omega}_{-_i}]_1^{m-1}$. 
First, we note that
\begin{align}
    \mathbf{B}(\mathbf{S})\boldsymbol{\mathcal{P}}_{0}^T &= \begin{bmatrix}
        \mathbf{B}_{-}(\mathbf{S}_{-}) &  \mathbf{0}_{(m-1)d \times n_3} & \mathbf{0}_{(m-1)d}\\
        \begin{bmatrix}\mathbf{0}_{d \times n_1} & \mathbf{B}^*_{2} & \mathbf{0}_{d \times (m-1)} \end{bmatrix} &\mathbf{B}^*_{3} & -(\mathbf{B}^*_{2}\mathbf{1}_{n_2}+\mathbf{B}^*_{3}\mathbf{1}_{n_3})
    \end{bmatrix}\\
    \mathbf{D}(\mathbf{S}) &= \blockdiag(\mathbf{D}_{-}(\mathbf{S}_{-}), \mathbf{B}^*_{2}\mathbf{B}^{*^T}_{2} + \mathbf{B}^*_{3}\mathbf{B}^{*^T}_{3}).
\end{align}
Since $\boldsymbol{\Omega}^{'}$ is a certificate of $\mathbf{L}(\mathbf{S})$, $\Tr(\boldsymbol{\Omega}^{'^T}\mathbf{L}(\mathbf{S})\boldsymbol{\Omega}^{'}) = 0$. Then, using the definition of $\mathbf{L}(\mathbf{S})$, the above equations, the fact that $\boldsymbol{\Omega}^{'}_m = 0$, and Eq.~(\ref{supp:eq:pinvA11}), we obtain
\begin{align}
    0 &= \Tr(\boldsymbol{\Omega}^{'^T}(\mathbf{D}(\mathbf{S}) - \mathbf{B}(\mathbf{S})\boldsymbol{\mathcal{P}}_{0}^T\boldsymbol{\mathcal{P}}_{0}\boldsymbol{\mathcal{L}}_{\Gamma}^\dagger\boldsymbol{\mathcal{P}}_{0}^T\boldsymbol{\mathcal{P}}_{0} \mathbf{B}(\mathbf{S})^T)\boldsymbol{\Omega}^{'})\\
    &= \Tr(\boldsymbol{\Omega}_{-}^T(\mathbf{D}_{-}(\mathbf{S}) - \mathbf{B}_{-}(\mathbf{S}_{-}) (\boldsymbol{\mathcal{P}}_{0}\boldsymbol{\mathcal{L}}_{\Gamma}^\dagger\boldsymbol{\mathcal{P}}_{0}^T)_{11} \mathbf{B}_{-}(\mathbf{S}_{-})^T)\boldsymbol{\Omega}_{-})\\
    &= \Tr(\boldsymbol{\Omega}_{-}^T(\mathbf{D}_{-}(\mathbf{S}) - \mathbf{B}_{-}(\mathbf{S}_{-})(\boldsymbol{\mathcal{L}}_{\Gamma_{-}}+\boldsymbol{\mathcal{D}}_0)^\dagger \mathbf{B}_{-}(\mathbf{S}_{-})^T)\boldsymbol{\Omega}_{-}).
\end{align}

% \begin{align}
%     0 &= \Tr(\boldsymbol{\Omega}^{'^T}\mathbf{L}(\mathbf{S})\boldsymbol{\Omega}^{'}) = \Tr(\boldsymbol{\Omega}^{'^T}(\mathbf{B}(\mathbf{S})\boldsymbol{\mathcal{P}}_{0}^T\boldsymbol{\mathcal{P}}_{0}\boldsymbol{\mathcal{L}}_{\Gamma}^\dagger\boldsymbol{\mathcal{P}}_{0}^T\boldsymbol{\mathcal{P}}_{0} \mathbf{B}(\mathbf{S})^T - \mathbf{D}(\mathbf{S}))\boldsymbol{\Omega}^{'})\\
%     &= \Tr(\boldsymbol{\Omega}^{'^T}(\mathbf{B}(\mathbf{S})\boldsymbol{\mathcal{P}}_{0}^T(\boldsymbol{\mathcal{P}}_{0}\boldsymbol{\mathcal{L}}_{\Gamma}\boldsymbol{\mathcal{P}}_{0}^T)^\dagger\boldsymbol{\mathcal{P}}_{0} \mathbf{B}(\mathbf{S})^T - \mathbf{D}(\mathbf{S}))\boldsymbol{\Omega}^{'})\\
%     &= \Tr(\boldsymbol{\Omega}_{-}^T(\mathbf{B}_{-}(\mathbf{S}_{-}) (\boldsymbol{\mathcal{P}}_{0}\boldsymbol{\mathcal{L}}_{\Gamma}^\dagger\boldsymbol{\mathcal{P}}_{0}^T)_{11} \mathbf{B}_{-}(\mathbf{S}_{-})^T - \mathbf{D}_{-}(\mathbf{S}))\boldsymbol{\Omega}_{-})\\
%     &= \Tr(\boldsymbol{\Omega}_{-}^T(\mathbf{B}_{-}(\mathbf{S}_{-})(\boldsymbol{\mathcal{L}}_{\Gamma_{-}}+\boldsymbol{\mathcal{D}}_0)^\dagger \mathbf{B}_{-}(\mathbf{S}_{-})^T - \mathbf{D}_{-}(\mathbf{S}))\boldsymbol{\Omega}_{-})
% \end{align}
From Proposition~\ref{prop:noiseless_setting1} and Eq.~(\ref{eq:C_of_S}), 
$$\mathbf{L}_{-}(\mathbf{S}_{-}) = \mathbf{D}_{-}(\mathbf{S}_{-}) - \mathbf{B}_{-}(\mathbf{S}_{-})\boldsymbol{\mathcal{L}}_{\Gamma_{-}}^\dagger \mathbf{B}_{-}(\mathbf{S}_{-})^T,$$ thus 
$$\Tr(\boldsymbol{\Omega}_{-}^T\mathbf{L}_{-}(\mathbf{S}_{-})\boldsymbol{\Omega}_{-}) - \Tr(\boldsymbol{\Omega}_{-}^T(\mathbf{B}_{-}(\mathbf{S}_{-}) ((\boldsymbol{\mathcal{L}}_{\Gamma_{-}}+\boldsymbol{\mathcal{D}}_0)^\dagger - \boldsymbol{\mathcal{L}}_{\Gamma_{-}}^\dagger) \mathbf{B}_{-}(\mathbf{S}_{-})^T )\boldsymbol{\Omega}_{-}) = 0.$$
Since $\mathbf{L}_{-}(\mathbf{S}_{-}) \succeq 0$, to show that $\Tr(\boldsymbol{\Omega}_{-}^T\mathbf{L}_{-}(\mathbf{S}_{-})\boldsymbol{\Omega}_{-}) = 0$, it suffices to show that
\begin{prop} $\mathbf{B}_{-}(\mathbf{S}_{-}) ((\boldsymbol{\mathcal{L}}_{\Gamma_{-}}+\boldsymbol{\mathcal{D}}_0)^\dagger - \boldsymbol{\mathcal{L}}_{\Gamma_{-}}^\dagger) \mathbf{B}_{-}(\mathbf{S}_{-})^T \preceq 0$.
\end{prop}
\textit{Proof}. Since $\boldsymbol{\mathcal{L}}_{\Gamma_{-}} \succeq 0$, consider $\boldsymbol{\mathcal{L}}_{\Gamma_{-}} = \mathbf{U}\boldsymbol{\Lambda}\mathbf{U}^T = \begin{bsmallmatrix}
        \mathbf{U}_1 & \mathbf{U}_2
    \end{bsmallmatrix}\begin{bsmallmatrix}
        \mathbf{\Lambda}_1 & \\
        & \mathbf{0}
    \end{bsmallmatrix}\begin{bsmallmatrix}
        \mathbf{U}_1^T\\
        \mathbf{U}_2^T
    \end{bsmallmatrix}$
where the $1 \leq n' \leq n_2$ columns of $\mathbf{U}_2 \in \mathbb{R}^{(n_1+n_2+m-1) \times n'}$ form an orthogonal basis of the $\ker (\boldsymbol{\mathcal{L}}_{\Gamma_{-}})$ (see proof of Proposition~\ref{supp:prop:LplusD0}). Also, $\mathbf{\Lambda}_1 \succ 0$. From Proposition~\ref{prop:kerB} and Eq.~(\ref{eq:BofS}), $\mathbf{B}_{-}(\mathbf{S}_{-})\mathbf{U}_2 = 0$. Thus, $\mathbf{B}_{-}(\mathbf{S}_{-})\mathbf{U} = \begin{bmatrix}
    \mathbf{B}_{-}(\mathbf{S}_{-})\mathbf{U}_1 & 0
\end{bmatrix}$. Then note that
\begin{equation}
    ((\boldsymbol{\mathcal{L}}_{\Gamma_{-}}+\boldsymbol{\mathcal{D}}_0)^\dagger - \boldsymbol{\mathcal{L}}_{\Gamma_{-}}^\dagger) = \mathbf{U} \left\{\left(\begin{bmatrix}
        \mathbf{\Lambda}_1 & \\
        & \mathbf{0}
    \end{bmatrix} + \mathbf{U}^T\boldsymbol{\mathcal{D}}_0\mathbf{U}\right)^\dagger-\begin{bmatrix}
        \mathbf{\Lambda}_1^{-1} & \\
        & \mathbf{0}
    \end{bmatrix}\right\}\mathbf{U}^T. \label{supp:eq:eq30}
\end{equation}
Using \citeb[Eq.~(10, 11, 17, 19)]{kovanic1979pseudoinverse} and simple calculations, we obtain
$$\left(\blockdiag(\mathbf{\Lambda}_1, \mathbf{0}) + \mathbf{U}^T\boldsymbol{\mathcal{D}}_0\mathbf{U}\right)^\dagger = (\blockdiag(\mathbf{\Lambda}_1^{-1}, \mathbf{0}) + \blockdiag(\mathbf{W}_1, \mathbf{W}_2))$$
% \begin{align}
%     \left(\begin{bmatrix}
%         \mathbf{\Lambda}_1 & \\
%         & \mathbf{0}
%     \end{bmatrix} + \mathbf{U}^T\boldsymbol{\mathcal{D}}_0\mathbf{U}\right)^\dagger &= \begin{bmatrix}
%         \mathbf{\Lambda}_1^{-1} & \\
%         & \mathbf{0}
%     \end{bmatrix} + \begin{bmatrix}
%         \mathbf{W}_1 &\\
%         & \mathbf{W}_2
%     \end{bmatrix}
% \end{align}
where 
$$\mathbf{W}_1 = -\mathbf{\Lambda}_1^{-1}\mathbf{U}_1^T\boldsymbol{\mathcal{D}}_0(\mathbf{I} + \boldsymbol{\mathcal{D}}_0\mathbf{U}_1\mathbf{\Lambda}_1^{-1}\mathbf{U}_1^T\boldsymbol{\mathcal{D}}_0)^{-1}\boldsymbol{\mathcal{D}}_0\mathbf{U}_1\mathbf{\Lambda}_1^{-1}$$
and $\mathbf{W}_2 = (\mathbf{U}_2^T\boldsymbol{\mathcal{D}}_0\mathbf{U}_2)^\dagger$.
In particular $\mathbf{W}_1 \preceq 0$ and $\mathbf{W}_2 \succeq 0$.
Combining above with Eq.~(\ref{supp:eq:eq30}), we deduce that 
$$\mathbf{B}_{-}(\mathbf{S}_{-}) ((\boldsymbol{\mathcal{L}}_{\Gamma_{-}}+\boldsymbol{\mathcal{D}}_0)^\dagger - \boldsymbol{\mathcal{L}}_{\Gamma_{-}}^\dagger) \mathbf{B}_{-}(\mathbf{S}_{-})^T = \mathbf{B}_{-}(\mathbf{S}_{-})\mathbf{U}_1\mathbf{W}_1\mathbf{U}_1^T\mathbf{B}_{-}(\mathbf{S}_{-})^T \preceq 0.$$
\hfill $\blacksquare$

\noindent We conclude that $\Tr(\boldsymbol{\Omega}_{-}^T\mathbf{L}_{-}(\mathbf{S}_{-})\boldsymbol{\Omega}_{-}) = 0$ and thus $\boldsymbol{\Omega}_{-}$ is a certificate of $\mathbf{L}_{-}(\mathbf{S}_{-})$.


\proofof{Proposition~\ref{prop:same_conn_comp_non_deg}}
It suffices to show that if the $i$th and $j$th vertices are adjacent in $\mathbb{G}$ then $\boldsymbol{\Omega}_i = \boldsymbol{\Omega}_j$. For $m=2$, the result is a direct consequence of Theorem~\ref{thm:nec_suff_cond_loc_rigid_two_views}. Suppose the result holds for $m-1$ views for some $m > 2$. If there are no edges in $\mathbb{G}$, then the result holds trivially for $m$ views. Suppose $i$th and $j$th vertices are adjacent in $\mathbb{G}$. Let $r \in [1,m] \setminus \{i,j\}$. We remove the $r$th view and the points which lie exclusively in it. Then by Lemma~\ref{lem:subproblem_cert}, $[\boldsymbol{\Omega}_k]_{k \in [1,m] \setminus r}$ is a certificate of $\mathbf{L}_{-r}(\mathbf{S}_{-r})$. Now, construct $\mathbb{G}_{-r}$ (in the same way as $\mathbb{G}$) and note that the $i$th and $j$th vertices are still adjacent. Thus, by the induction hypothesis, $\boldsymbol{\Omega}_i = \boldsymbol{\Omega}_j$.

\proofof{Theorem~\ref{thm:G_star_1}}
The result holds for two views (see Theorem~\ref{thm:nec_suff_cond_loc_rigid_two_views}). Suppose it holds for $m-1$ views for some $m > 2$. Let $\boldsymbol{\Omega}$ be a certificate of $\mathbf{L}(\mathbf{S})$. We need to show that $\boldsymbol{\Omega}$ is trivial. Since $|\mathbb{G}^*(\mathbf{S})|=1$, $\mathbb{G}$ must have a connected component with at least two views. Pick one such component and note that there exist a view in it such that removing it will not disconnect the component. Let it be the $i$th view. Consider removing the $i$th view and the points which lie exclusively in it. For the new set of views we still have $|\mathbb{G}^*(\mathbf{S}_{-i})|=1$ (where $\mathbb{G}^*_{-i}(\mathbf{S}_{-i})$ is constructed in the same manner as $\mathbb{G}^*(\mathbf{S})$). By Lemma~\ref{lem:subproblem_cert} and the induction hypothesis, we conclude that $[\boldsymbol{\Omega}_j]_{j \in [1,m]\setminus \{i\}}$ must be trivial. By Proposition~\ref{prop:same_conn_comp_non_deg} we conclude that $\boldsymbol{\Omega}$ is trivial.

\proofof{Theorem~\ref{thm:nec_cond_glob_rigid_views}}
The calculations up to Eq.~(\ref{supp:eq:eq11}) in the proof of Theorem~\ref{thm:nec_cond_loc_rigid_of_views} are reused here.
Consider a partition of $[1,m]$ into two non-empty subsets $A$ and $B$. Suppose the rank of $\overline{\mathbf{B}(\mathbf{S})}_{A, B}$ is at most $d-1$. As in the proof of Theorem~\ref{thm:nec_cond_loc_rigid_of_views}, WLOG assume that $\mathbf{B}(\mathbf{S})_{A, B}\mathbf{1}_{n'} = 0$ (where $n'$ is as in Definition~\ref{def:BSAcapB}). Then the rank of $\mathbf{B}(\mathbf{S})_{A, B}$ is at most $d-1$. We are going to construct another perfect alignment $\mathbf{S}'$ such that $\pi(\mathbf{S}) \neq \pi(\mathbf{S}')$, thus concluding that $\mathbf{S}$ is not unique.

Let $\mathbf{V}_1, \mathbf{V}_2 \in \mathbb{O}(d)$ and $\mathbf{\Sigma}$ be the diagonal matrix containing the singular values of $\mathbf{B}(\mathbf{S})_{A, B}$ such that $\mathbf{B}(\mathbf{S})_{A, B} = \mathbf{V}_1\mathbf{\Sigma}\mathbf{V}_2^T$. Since rank of $\mathbf{B}(\mathbf{S})_{A, B} \leq d-1$, there exist $\mathbf{U} \in \mathbb{O}(d)$ such that $\mathbf{U} \neq \mathbf{I}_d$ and $\mathbf{\Sigma} = \mathbf{U}\mathbf{\Sigma}$. Define $\mathbf{Q} = \mathbf{V}_1\mathbf{U}^T\mathbf{V}_1^T$. Then $\mathbf{Q} \neq \mathbf{I}_d$ and 
\begin{equation}
    \mathbf{Q}^T\mathbf{B}(\mathbf{S})_{A, B} = \mathbf{B}(\mathbf{S})_{A, B}. \label{supp:eq:QTBSAB}
\end{equation}
Define $\mathbf{S}' \in \mathbb{O}(d)^m$ such that $\mathbf{S}'_i = \mathbf{S}_i\mathbf{Q}$ for all $i \in A$ and $\mathbf{S}'_j = \mathbf{S}_j$ for all $j \in B$. Clearly, $\mathbf{S}' \neq \mathbf{S}$. We will show that $\mathbf{S}'$ is another perfect alignment. It is easy to see that
% \begin{align}
%     \Tr(\mathbf{S}'^T\mathbf{C}\mathbf{S}') &= \textstyle\sum_{i = 1}^{m}\textstyle\sum_{j=1}^{m}\Tr(\mathbf{S}'^T_i\mathbf{C}_{ij}\mathbf{S}'_j)\\
%     &= \textstyle\sum_{\substack{i \in A, j \in A\\i \in B, j \in B}}\Tr(\mathbf{S}_i^T\mathbf{C}_{ij}\mathbf{S}_j) + 2 \textstyle\sum_{i \in A}\textstyle\sum_{j \in B}\Tr(\mathbf{Q}^T\mathbf{S}_i^T\mathbf{C}_{ij}\mathbf{S}_j).
% \end{align}
%\begin{equation}
$$\Tr(\mathbf{S}'^T\mathbf{C}\mathbf{S}') = \textstyle\sum_{\substack{i \in A, j \in A\\i \in B, j \in B}}\Tr(\mathbf{S}_i^T\mathbf{C}_{ij}\mathbf{S}_j) + 2 \textstyle\sum_{i \in A, j \in B}\Tr(\mathbf{Q}^T\mathbf{S}_i^T\mathbf{C}_{ij}\mathbf{S}_j).$$
%\end{equation}
Since, for $i \in A$ and $j \in B$, $\mathbf{C}_{ij} = \mathbf{B}_i \boldsymbol{\mathcal{L}}_{\Gamma}^\dagger \mathbf{B}_j^T$, it suffices to show that
$$\Tr\left(\left(\textstyle\sum_{i \in A}\mathbf{B}(\mathbf{S})_i\right)\boldsymbol{\mathcal{L}}_{\Gamma}^\dagger \left(\textstyle\sum_{j \in B}\mathbf{B}(\mathbf{S})_j^T\right)\right) = \Tr\left(\mathbf{Q}^T\left(\textstyle\sum_{i \in A}\mathbf{B}(\mathbf{S})_i\right)\boldsymbol{\mathcal{L}}_{\Gamma}^\dagger \left(\textstyle\sum_{j \in B}\mathbf{B}(\mathbf{S})_j^T\right)\right).$$
% \begin{equation}
%     \Tr\left(\left(\textstyle\sum_{i \in A}\mathbf{B}(\mathbf{S})_i\right)\boldsymbol{\mathcal{L}}_{\Gamma}^\dagger \left(\textstyle\sum_{j \in B}\mathbf{B}(\mathbf{S})_j^T\right)\right)  = \Tr\left(\mathbf{Q}^T\left(\textstyle\sum_{i \in A}\mathbf{B}(\mathbf{S})_i\right)\boldsymbol{\mathcal{L}}_{\Gamma}^\dagger \left(\textstyle\sum_{j \in B}\mathbf{B}(\mathbf{S})_j^T\right)\right) 
% \end{equation}
Define $\mathbf{B}_A$, $\mathbf{B}_B$,  $\mathbf{B}_{A\setminus B}$, $\mathbf{B}_{A,B}$, $\mathbf{B}_{B\setminus A}$ and $\mathbf{B}_{B,A}$ as in the proof of Theorem~\ref{thm:nec_cond_loc_rigid_of_views}, then it suffices to show that 
$\Tr(\mathbf{Q}^T \mathbf{B}_A\boldsymbol{\mathcal{L}}_{\Gamma}^\dagger \mathbf{B}_B^T) = \Tr( \mathbf{B}_A\boldsymbol{\mathcal{L}}_{\Gamma}^\dagger \mathbf{B}_B^T).$
Using Eq.~(\ref{supp:eq:QTBSAB}, \ref{supp:eq:eq8}, \ref{supp:eq:eq9}) we obtain $\mathbf{Q}^T \mathbf{B}_{A,B} = \mathbf{B}_{A,B}$ and this combined with Eq.~(\ref{supp:eq:eq11}), yields
\begin{align}
    \Tr(\mathbf{Q}^T \mathbf{B}_A\boldsymbol{\mathcal{L}}_{\Gamma}^\dagger \mathbf{B}_B^T) &= \Tr(\mathbf{Q}^T (\mathbf{B}_{A \setminus B}+\mathbf{B}_{A,B})\boldsymbol{\mathcal{L}}_{\Gamma}^\dagger (\mathbf{B}_{B\setminus A} + \mathbf{B}_{B,A})^T ) \\
    &= \Tr(\mathbf{Q}^T \mathbf{B}_{A,B}\boldsymbol{\mathcal{L}}_{\Gamma}^\dagger\mathbf{B}_{B,A}^T) = \Tr( \mathbf{B}_{A,B}\boldsymbol{\mathcal{L}}_{\Gamma}^\dagger\mathbf{B}_{B,A}^T)\\
    &=\Tr((\mathbf{B}_{A \setminus B}+\mathbf{B}_{A,B})\boldsymbol{\mathcal{L}}_{\Gamma}^\dagger (\mathbf{B}_{B\setminus A} + \mathbf{B}_{B,A})^T )\\
    &= \Tr(\mathbf{B}_A\boldsymbol{\mathcal{L}}_{\Gamma}^\dagger \mathbf{B}_B^T)
\end{align}
% \begin{align}
%     &\Tr(\mathbf{Q}^T \mathbf{B}_A\boldsymbol{\mathcal{L}}_{\Gamma}^\dagger \mathbf{B}_B^T) = \Tr(\mathbf{Q}^T (\mathbf{B}_{A \setminus B}+\mathbf{B}_{A,B})\boldsymbol{\mathcal{L}}_{\Gamma}^\dagger (\mathbf{B}_{B\setminus A} + \mathbf{B}_{B,A})^T )\\
%     &= \Tr(\mathbf{Q}^T \mathbf{B}_{A,B}\boldsymbol{\mathcal{L}}_{\Gamma}^\dagger\mathbf{B}_{B,A}^T) = \Tr( \mathbf{B}_{A,B}\boldsymbol{\mathcal{L}}_{\Gamma}^\dagger\mathbf{B}_{B,A}^T)\\
%     &= \Tr((\mathbf{B}_{A \setminus B}+\mathbf{B}_{A,B})\boldsymbol{\mathcal{L}}_{\Gamma}^\dagger (\mathbf{B}_{B\setminus A} + \mathbf{B}_{B,A})^T ) = \Tr(\mathbf{B}_A\boldsymbol{\mathcal{L}}_{\Gamma}^\dagger \mathbf{B}_B^T).
% \end{align}

\proofof{Proposition~\ref{prop:same_conn_comp_uniq}} By replacing $\mathbb{G}$ with $\overline{\mathbb{G}}$ and Theorem~\ref{thm:nec_suff_cond_loc_rigid_two_views} with \ref{thm:nec_suff_cond_glob_rigid_two_views}, the inductive proof is the same as of Proposition~\ref{prop:same_conn_comp_non_deg}.
%It suffices to show the result for adjacent vertices in $\overline{\mathbb{G}}$. For two views, the result follows from Theorem~\ref{thm:nec_suff_cond_glob_rigid_two_views}. Suppose the result holds for $m-1$ views for some $m > 2$. Then we show the result for $m$ views. If there are no edges in $\overline{\mathbb{G}}$ then the result is trivially valid. Suppose $i$th and $j$th vertices are adjacent in $\overline{\mathbb{G}}$. Let $r \in [1,m] \setminus \{i,j\}$. We remove the $r$th view and the points which lie exclusively in it. Then, by Lemma~\ref{lem:subproblem_cert}, $\mathbf{S}_{-r} = [\mathbf{S}_{k}]_{k \in [1,m] \setminus \{r\}}$ is a perfect alignment of the remaining views. Now, construct $\overline{\mathbb{G}}_{-r}$ (in the same way as $\overline{\mathbb{G}}$) and note that the vertices corresponsing to $i$th and $j$th views are still adjacent in $\overline{\mathbb{G}}_{-r}$. By the induction hypothesis we conclude the result.

\proofof{Theorem~\ref{thm:overline_G_star_1}} By replacing $\mathbb{G}$ with $\overline{\mathbb{G}}$, Theorem~\ref{thm:nec_suff_cond_loc_rigid_two_views} with \ref{thm:nec_suff_cond_glob_rigid_two_views} and Proposition~\ref{prop:same_conn_comp_non_deg} with \ref{prop:same_conn_comp_uniq}, the inductive proof is the same as of Theorem~\ref{thm:G_star_1}.

%The result holds for two views (see Theorem~\ref{thm:nec_suff_cond_glob_rigid_two_views}). Suppose the result holds for $m-1$ views for some $m > 2$. We will show that the result holds for $m$ views. Suppose $|\overline{\mathbb{G}}^*(\mathbf{S})|=1$. Let $\mathbf{S}'$ be a perfect alignment. We need to show that $\mathbf{S}' = \mathbf{S}\mathbf{Q}$ for some $\mathbf{Q} \in \mathbb{O}(d)$. Since $|\overline{\mathbb{G}}^*(\mathbf{S})|=1$, $\overline{\mathbb{G}}$ must have a connected component with at least two views. Pick one such component and note that there exist a view in it such that removing it will not disconnect the component. Let it be the $i$th view. Consider removing the $i$th view and the points that lie exclusively in it. Note that for the new set of views we still have $|\overline{\mathbb{G}}^*_{-i}(\mathbf{S}_{-i})|=1$ (where $\overline{\mathbb{G}}^*_{-i}(\mathbf{S}_{-i})$ is constructed in the same manner as $\overline{\mathbb{G}}^*(\mathbf{S})$). By Lemma~\ref{lem:subproblem_cert} and the induction hypothesis, we conclude that $\mathbf{S}'_{-i} = \mathbf{S}_{-i}\mathbf{Q}$ for some $\mathbf{Q} \in \mathbb{O}(d)$. Then, by Proposition~\ref{prop:same_conn_comp_uniq} we conclude that $\mathbf{S}' = \mathbf{S}\mathbf{Q}$.

%%%%%%%%%%%%%%%%%%%%%%%%%%%%%%%%%%%%%%%%%%
% Section 5 Proofs
%%%%%%%%%%%%%%%%%%%%%%%%%%%%%%%%%%%%%%%%%%
\proofof{Lemma~\ref{lem:retraction}}
For $\mathbf{S} \in \pi^{-1}(\widetilde{\mathbf{S}})$ and $\mathbf{Z} = [\mathbf{S}_i\boldsymbol{\Omega}_i]_1^m \in T_{\mathbf{S}}\mathbb{O}(d)^m$, the horizontal lift of $\widetilde{\mathbf{Z}} = [\widetilde{\mathbf{S}}_i\widetilde{\boldsymbol{\Omega}}_i] \in T_{\widetilde{\mathbf{S}}}\mathbb{O}(d)^m/_{\sim}$,
$$\pi(R_{\EXP}(\mathbf{S}, \mathbf{Z})) = [\mathbf{S}_i\exp(\boldsymbol{\Omega}_i)(\mathbf{S}_1\exp(\boldsymbol{\Omega}_1))^T]_1^m.$$
It suffices to show that $\mathbf{S}_{i+1}\exp(\boldsymbol{\Omega}_{i+1})(\mathbf{S}_1\exp(\boldsymbol{\Omega}_1))^T$ depends only on $\widetilde{\mathbf{S}}$ and $\widetilde{\boldsymbol{\Omega}}$ for all $i \in [1,m-1]$. Using Proposition~\ref{prop:hlift_char} and expanding the expression, we obtain 
$$\mathbf{S}_{i+1}\exp(\boldsymbol{\Omega}_{i+1})(\mathbf{S}_1\exp(\boldsymbol{\Omega}_1))^T = \mathbf{S}_{i+1}\exp(\mathbf{S}_1^T \widetilde{\boldsymbol{\Omega}}_i\mathbf{S}_1 + \boldsymbol{\Omega}_1) (\mathbf{S}_1\exp(\boldsymbol{\Omega}_1))^T.$$
Since $\mathbf{S}_1 \in \mathbb{O}(d)$, we have $\exp(\mathbf{S}_1^T \widetilde{\boldsymbol{\Omega}}_i\mathbf{S}_1) = \mathbf{S}_1^T\exp(\widetilde{\boldsymbol{\Omega}}_i)\mathbf{S}_1$. Also, the following identites hold: $\exp(\boldsymbol{\Omega}_1)^T = \exp(\boldsymbol{\Omega}_1^T) = \exp(-\boldsymbol{\Omega}_1)$ and $\exp(\mathbf{A}_1+\mathbf{A}_1) = \exp(\mathbf{A}_1)\exp(\mathbf{A}_2)$. By substituting back into the expression, we obtain 
$$\mathbf{S}_{i+1}\exp(\mathbf{S}_1^T \widetilde{\boldsymbol{\Omega}}_i\mathbf{S}_1 + \boldsymbol{\Omega}_1) (\mathbf{S}_1\exp(\boldsymbol{\Omega}_1))^T = \mathbf{S}_{i+1}\mathbf{S}_1^T\exp(\widetilde{\boldsymbol{\Omega}}_i) = \widetilde{\mathbf{S}}_i\exp(\widetilde{\boldsymbol{\Omega}}_i).$$

\proofof{Proposition~\ref{prop:liu_pf}}
Since $\mathbf{Z}_i = \mathbf{S}_i\boldsymbol{\Omega}_i$ where $\boldsymbol{\Omega}_i \in \Skew(d)$,
$$
    \left\|\mathbf{S}_i\exp(\mathbf{S}_i^T\mathbf{Z}_i) - (\mathbf{S}_i + \mathbf{Z}_i)\right\|_F = \left\|\exp(\boldsymbol{\Omega}_i) - (\mathbf{I}_d + \boldsymbol{\Omega}_i)\right\|_F \leq \left(\sum_{2}^{\infty}\frac{1}{k !}\right)\left\|\boldsymbol{\Omega}_i\right\|_F^2 = (e-1)\left\|\boldsymbol{\Omega}_i\right\|_F^2
$$
where the inequality follows from the triangle inequality and the fact that $\left\|\boldsymbol{\Omega}_i\right\|_F = \left\|\mathbf{Z}_i\right\|_F \leq 1$.

% \proofof{Proposition~\ref{prop:second_order_boundedness_of_RPF}}
% \revdel{Fix $\phi = 1/2$. Then $\left\|\boldsymbol{\xi}_i\right\|_F \leq 1/2$. From Proposition~\ref{prop:liu_qr}, we obtain $\left\|R_{\QR }(\mathbf{S}, \boldsymbol{\xi}) - (\mathbf{S} + \boldsymbol{\xi})\right\|_F^2 = \textstyle\sum_1^m \left\|\qf (\mathbf{S}_i + \boldsymbol{\xi}_i) - (\mathbf{S}_i + \boldsymbol{\xi}_i)\right\|_F^2 \leq M^2 \textstyle\sum_1^m \left\|\boldsymbol{\xi}_i\right\|_F^4 \leq M^2 \left\|\boldsymbol{\xi}\right\|_F^4$.
% \begin{align}
%     \left\|R_{\QR }(\mathbf{S}, \boldsymbol{\xi}) - (\mathbf{S} + \boldsymbol{\xi})\right\|_F^2 &= \textstyle\sum_1^m \left\|\qf (\mathbf{S}_i + \boldsymbol{\xi}_i) - (\mathbf{S}_i + \boldsymbol{\xi}_i)\right\|_F^2 \leq M^2 \textstyle\sum_1^m \left\|\boldsymbol{\xi}_i\right\|_F^4 \leq M^2 \left\|\boldsymbol{\xi}\right\|_F^4.
%     %&\leq M^2 \left(\textstyle\sum_1^m \left\|\boldsymbol{\xi}_i\right\|_F^2\right)^2 = M^2 \left\|\boldsymbol{\xi}\right\|_F^4.
% \end{align}
% The result follows for the same $M$ as in Proposition~\ref{prop:liu_qr}.}
% \revadd{By adding and subtracting the term $(\mathbf{S}_i + \boldsymbol{\xi}_i)\pf(\mathbf{S}_1 + \boldsymbol{\xi}_1)^T$ to $\left\|\pf(\mathbf{S}_i + \boldsymbol{\xi}_i)\pf(\mathbf{S}_1 + \boldsymbol{\xi}_1)^T - (\mathbf{S}_i + \boldsymbol{\xi}_i)\right\|_F$, using triangle inequality and then using Proposition~\ref{prop:liu_pf}, we obtain the bound $\left\|\boldsymbol{\xi}_i\right\|_F^2 + \left\|\mathbf{S}_i+\boldsymbol{\xi}_i\right\|_F \left\|\pf(\mathbf{S}_1+\boldsymbol{\xi}_1) - \mathbf{I}_d\right\|_F  \leq \left\|\boldsymbol{\xi}_i\right\|_F^2 + \left\|\boldsymbol{\xi}_1\right\|_F^2 \leq 2\left\|\boldsymbol{\xi}\right\|_F^2$. The result follows.
% }
% \revadd{The proof of part (a) follows from Proposition~\ref{prop:liu_pf} and $\left\|R_\EXP(\mathbf{S}, \boldsymbol{\xi}) - (\mathbf{S} + \boldsymbol{\xi})\right\|_F \leq \sum_1^m \left\|\mathbf{S}_i\exp(\mathbf{S}_i^T\boldsymbol{\xi}_i) - (\mathbf{S}_i + \boldsymbol{\xi}_i)\right\|_F$. For part (b), adding and subtracting $(\mathbf{S}_i + \boldsymbol{\xi}_i)(\mathbf{S}_1\exp(\mathbf{S}_1^T\boldsymbol{\xi}_1))^T$ to 
%$\left\|\mathbf{S}_i\exp(\mathbf{S}_i^T\boldsymbol{\xi}_i)(\mathbf{S}_1\exp(\mathbf{S}_1^T\boldsymbol{\xi}_1))^T - (\mathbf{S}_i + \boldsymbol{\xi}_i)(\mathbf{S}_1+\boldsymbol{\xi}_1)^T\right\|_F$
% the l.h.s, using triangle inequality and Proposition~\ref{prop:liu_pf} with the fact that $\left\|\boldsymbol{\xi}_i\right\|_F^2 \leq \left\|\boldsymbol{\xi}\right\|_F^2 \leq 1$, we obtain the bound $(e-1)\left\|\boldsymbol{\xi}_i\right\|_F^2 + \left\|\mathbf{S}_i+\boldsymbol{\xi}_i\right\|_F \left\|\mathbf{S}_1\exp(\mathbf{S}_1^T\boldsymbol{\xi}_1) - (\mathbf{S}_1+\boldsymbol{\xi}_1)\right\|_F  \leq (e-1)(\left\|\boldsymbol{\xi}_i\right\|_F^2 + 2\left\|\boldsymbol{\xi}_1\right\|_F^2)$. Since $\left\|\boldsymbol{\xi}_i\right\|_F^2 + 2\left\|\boldsymbol{\xi}_1\right\|_F^2 \leq 2\left\|\boldsymbol{\xi}\right\|_F^2$, the result follows.}

\proofof{Proposition~\ref{prop:second_order_boundedness_of_Rtilde}}
The proof of part (a) follows from Proposition~\ref{prop:liu_pf}, Proposition~\ref{prop:hlift_frob_ineq} and 
$$\left\|R_\EXP(\mathbf{S}, \boldsymbol{\xi}) - (\mathbf{S} + \boldsymbol{\xi})\right\|_F \leq \sum_1^m \left\|\mathbf{S}_i\exp(\mathbf{S}_i^T\boldsymbol{\xi}_i) - (\mathbf{S}_i + \boldsymbol{\xi}_i)\right\|_F.$$
For part (b), let $\mathbf{S} \in \pi^{-1}(\widetilde{\mathbf{S}})$ and $\mathbf{Z} = [\mathbf{S}_i\boldsymbol{\Omega}_i]_1^m \in T_{\mathbf{S}}\mathbb{O}(d)^m$ be the horizontal lift of $\widetilde{\mathbf{Z}} = [\widetilde{\mathbf{S}}_i\widetilde{\boldsymbol{\Omega}}_i]_1^m$ at $\mathbf{S}$. Note that $\widetilde{\mathbf{S}}_i = \mathbf{S}_{i+1}\mathbf{S}_1^T$ (Eq.~(\ref{eq:pi_inv_wtS})) and $\boldsymbol{\Omega}_{i+1} -  \boldsymbol{\Omega}_{1}= \mathbf{S}_1^T\widetilde{\boldsymbol{\Omega}}_{i}\mathbf{S}_1$ (Eq.~(\ref{eq:hlifti})). Then, using the identity $\exp(\boldsymbol{\Omega}_{i+1})\exp(\boldsymbol{\Omega}_{1})^T = \exp(\boldsymbol{\Omega}_{i+1}-\boldsymbol{\Omega}_{1})$, the inequality $\left\|\boldsymbol{\Omega}_{i+1}-\boldsymbol{\Omega}_{1}\right\|_F \leq \left\|\boldsymbol{\Omega}_{i+1}\right\|_F + \left\|\boldsymbol{\Omega}_{1}\right\|_F \leq 1$ and Proposition~\ref{prop:liu_pf}, we obtain
\begin{align}
    \left\|\widetilde{R}_\EXP(\widetilde{\mathbf{S}}, \widetilde{\mathbf{Z}})_i - (\widetilde{\mathbf{S}}_i + \widetilde{\mathbf{Z}}_i)\right\|_F &= \left\|\mathbf{S}_{i+1}(\exp(\boldsymbol{\Omega}_{i+1})\exp(\boldsymbol{\Omega}_{1})^T - (\mathbf{I}_d +\boldsymbol{\Omega}_{i+1} -  \boldsymbol{\Omega}_{1}))\mathbf{S}_{1}^T\right\|_F\\
    &\leq (e-1)\left\|\boldsymbol{\Omega}_{i+1} -  \boldsymbol{\Omega}_{1}\right\|_F^2 = (e-1)\left\|\widetilde{\boldsymbol{\Omega}}_{i}\right\|_F^2 = (e-1)\left\|\widetilde{\mathbf{Z}}_i\right\|_F^2.
\end{align}
Then the result follows from $\left\|\widetilde{R}_\EXP(\widetilde{\mathbf{S}}, \widetilde{\mathbf{Z}}) - (\widetilde{\mathbf{S}} + \widetilde{\mathbf{Z}})\right\|_F \leq \sum_1^m \left\|\widetilde{R}_\EXP(\widetilde{\mathbf{S}}, \widetilde{\mathbf{Z}})_i - (\widetilde{\mathbf{S}}_i + \widetilde{\mathbf{Z}}_i)\right\|_F$.

\proofof{Proposition~\ref{prop:alpha_grad}}
\revadd{The proof is similar to the one in \citea{liu2019quadratic}. Due to \textbf{(A2)} in Section~\ref{subsec:loc_lin_conv}, WLOG we assume that $\grad F(\mathbf{S}^k) \neq 0$ for all $k \geq 0$. Then, from the proof of Proposition~\ref{prop:gradFS}, since 
$$\grad F(\mathbf{S})_i = 0.5\left(\nabla F(\mathbf{S})_i - \mathbf{S}_i\nabla F(\mathbf{S})_i^T\mathbf{S}_i\right),$$
we obtain 
\begin{align}
g(\nabla F(\mathbf{S}^k)_i, \grad F(\mathbf{S}^k)_i) &= 0.5\left(\left\|\nabla F(\mathbf{S}^k)_i\right\|_F^2 -g(\nabla F(\mathbf{S}^k)_i,\mathbf{S}^k_i\nabla F(\mathbf{S}^k)_i^T\mathbf{S}^k_i)\right)\\
&= 0.25 \left\|\nabla F(\mathbf{S}^k)_i - \mathbf{S}^k_i\nabla F(\mathbf{S}^k)_i^T\mathbf{S}^k_i\right\|_F^2\\
&= \left\|\grad F(\mathbf{S}^k)_i\right\|_F^2.
\end{align}
Thus, 
$$g(\nabla F(\mathbf{S}^k), \grad F(\mathbf{S}^k)) = \textstyle\sum_1^m g(\nabla F(\mathbf{S}^k)_i, \grad F(\mathbf{S}^k)_i) = \left\|\grad F(\mathbf{S}^k)\right\|_F^2.$$ 
This together with Algorithm~\ref{algo:rgd}, and Eq.~(\ref{eq:armijo_step}), implies
\begin{equation}
    F(\mathbf{S}^{k+1})-F(\mathbf{S}^{k}) \leq -\gamma \alpha_k \left\|\grad F(\mathbf{S}^k)\right\|_F^2 \label{supp:eq:eq1}
\end{equation}
Since $\mathbb{O}(d)^m$ is compact and $F$ is analytic, thus $F$ is bounded. Since $\alpha_k \in (0,1]$ for all $k \geq 0$, 
$$\textstyle\sum_0^\infty \alpha_k^2 \left\|\grad F(\mathbf{S}^k)\right\|_F^2 \leq \textstyle\sum_0^\infty \alpha_k \left\|\grad F(\mathbf{S}^k)\right\|_F^2 \leq \frac{1}{\gamma}(F(\mathbf{S}^0) - \lim F(\mathbf{S}^k)) < \infty.$$
Thus, $\lim \alpha_k \left\|\grad F(\mathbf{S}^k)\right\|_F = 0$ and from Proposition~\ref{prop:hlift_frob_ineq}, we obtain $\lim \alpha_k \left\|\grad \widetilde{F}(\widetilde{\mathbf{S}}^k)\right\|_F = 0$.}
% \proofof{(A1) in Section~\ref{subsec:loc_lin_conv}}
% \revdel{
% From Proposition~\ref{prop:alpha_grad}, there exists $k_1 \geq 0$ such that $\alpha_k\left\|\grad F(\mathbf{S}^k)\right\| \leq 1/2$ for all $k \geq k_1$. Then note that $\left\|\mathbf{S}^{k+1} - \mathbf{S}^k\right\|_F$ equals $\left\|R_{QR}(\mathbf{S}^k, -\alpha_k\grad F(\mathbf{S}^k)) - \mathbf{S}^k\right\|_F$. From Proposition~\ref{prop:second_order_boundedness_of_RQR}, for all $k \geq k_1$, the latter is bounded by $M \alpha_k^2 \left\|\grad F(\mathbf{S}^k)\right\|_F^2 + \alpha_k\left\|\grad F(\mathbf{S}^k)\right\|_F$, which in turn is bounded by $(M/2+1)\alpha_k \left\|\grad F(\mathbf{S}^k)\right\|_F$.
% % \begin{align}
% %     &\left\|\mathbf{S}^{k+1} - \mathbf{S}^k\right\|_F = \left\|R_{QR}(\mathbf{S}^k, -\alpha_k\grad F(\mathbf{S}^k)) - \mathbf{S}^k\right\|_F\\
% %     &\leq M \alpha_k^2 \left\|\grad F(\mathbf{S}^k)\right\|_F^2 + \alpha_k\left\|\grad F(\mathbf{S}^k)\right\|_F \leq (M/2+1)\alpha_k \left\|\grad F(\mathbf{S}^k)\right\|_F.
% % \end{align}
% Finally, using Eq~(\ref{supp:eq:eq1}), for all $k \geq k_1$, $F(\mathbf{S}^{k+1})-F(\mathbf{S}^{k}) \leq -2\gamma(M+2)^{-1} \left\|\grad F(\mathbf{S}^k)\right\|_F \cdot \left\|\mathbf{S}^{k+1}-\mathbf{S}^k\right\|_F$. Thus, (\textbf{A1}) holds for $\kappa_0 = 2\gamma(M+2)^{-1}$. The result follows.
% }
% \revadd{
% For convenience, denote $\grad F(\mathbf{S}^k)$ by $[\mathbf{S}^k_i\boldsymbol{\xi}^k_i]_1^m$. From Proposition~\ref{prop:alpha_grad}, there exists $k_1 \geq 0$ such that $\alpha_k\left\|\grad F(\mathbf{S}^k)\right\|_F \leq 1/2$ for all $k \geq k_1$. Therefore, for all $k \geq k_1$, $\alpha_k \left\|\boldsymbol{\xi}^k\right\|_F \leq 1/2$. Then note that $\left\|\widetilde{\mathbf{S}}^{k+1} - \widetilde{\mathbf{S}}^k\right\|_F$ equals $\left\|R_\EXP(\mathbf{S}^k, -\alpha_k\boldsymbol{\xi}^k)(\mathbf{S}^k_1\exp(-\alpha_k\boldsymbol{\xi}^k_1))^T - \mathbf{S}^k(\mathbf{S}^k_1)^T\right\|_F$. Using Proposition~\ref{prop:second_order_boundedness_of_RPF},}
% \begin{align}
%     &\left\|R_\EXP(\mathbf{S}^k, -\alpha_k\boldsymbol{\xi}^k)(\mathbf{S}^k_1\exp(-\alpha_k\boldsymbol{\xi}^k_1))^T - \mathbf{S}^k(\mathbf{S}^k_1)^T\right\|_F\\
%     &\leq \left\|R_\EXP(\mathbf{S}^k, -\alpha_k\boldsymbol{\xi}^k)(\mathbf{S}^k_1\exp(-\alpha_k\boldsymbol{\xi}^k_1))^T - (\mathbf{S}^k+\alpha_k\boldsymbol{\xi}^k)(\mathbf{S}^k_1)^T\right\|_F + \alpha_k\left\|\boldsymbol{\xi}^k\right\|_F\\
%     &\leq \left\|R_\EXP(\mathbf{S}^k, -\alpha_k\boldsymbol{\xi}^k)(\mathbf{S}^k_1\exp(-\alpha_k\boldsymbol{\xi}^k_1))^T - (\mathbf{S}^k+\alpha_k\boldsymbol{\xi}^k)(\mathbf{S}^k_1+\alpha_k\boldsymbol{\xi}^k_1)^T\right\|_F +\\
%     &\hspace{5.5cm} (\sqrt{md} + \alpha_k \left\|\boldsymbol{\xi}^k\right\|_F)\alpha_k \left\|\boldsymbol{\xi}^k\right\|_F + \alpha_k\left\|\boldsymbol{\xi}^k\right\|_F\\
%     &\leq 2\sqrt{m}(e-1)\alpha_k^2 \left\|\boldsymbol{\xi}^k\right\|_F^2 + (\sqrt{md}+3/2)\alpha_k \left\|\boldsymbol{\xi}^k\right\|_F\\
%     &\leq (\sqrt{m}(e-1) +\sqrt{md}+3/2)\alpha_k \left\|\boldsymbol{\xi}^k\right\|_F,
% \end{align}
% for all $k \geq k_1$.
% \revadd{
% % From Proposition~\ref{prop:second_order_boundedness_of_RPF}, for all $k \geq k_1$, the latter is bounded by $2\sqrt{m} \alpha_k^2 \left\|\boldsymbol{\xi}^k\right\|_F^2 + \alpha_k\left\|\boldsymbol{\xi}^k\right\|_F$, which in turn is bounded by $(\sqrt{m}+1)\alpha_k \left\|\boldsymbol{\xi}^k\right\|_F$.
% % \begin{align}
% %     &\left\|\mathbf{S}^{k+1} - \mathbf{S}^k\right\|_F = \left\|R_{QR}(\mathbf{S}^k, -\alpha_k\grad F(\mathbf{S}^k)) - \mathbf{S}^k\right\|_F\\
% %     &\leq M \alpha_k^2 \left\|\grad F(\mathbf{S}^k)\right\|_F^2 + \alpha_k\left\|\grad F(\mathbf{S}^k)\right\|_F \leq (M/2+1)\alpha_k \left\|\grad F(\mathbf{S}^k)\right\|_F.
% % \end{align}
% Then using Eq~(\ref{supp:eq:eq1}) and the fact that $F(\mathbf{S}^k) = \widetilde{F}(\widetilde{\mathbf{S}}_k)$, for all $k \geq k_1$, $\widetilde{F}(\mathbf{S}^{k+1})-\widetilde{F}(\mathbf{S}^{k}) \leq -\gamma(\sqrt{m}(e-1) +\sqrt{md}+3/2)^{-1} \left\|\boldsymbol{\xi}^k\right\|_F \cdot \left\|\widetilde{\mathbf{S}}^{k+1}-\widetilde{\mathbf{S}}^k\right\|_F$. Finally, from Proposition~\ref{prop:hlift_frob_ineq}, $\left\|\boldsymbol{\xi}^k\right\|_F = \left\|\grad F(\mathbf{S}^k)\right\|_F \geq (m+1)^{-1/2} \left\|\grad \widetilde{F}(\widetilde{\mathbf{S}}^k)\right\|_F$. Thus, (\textbf{A1}) holds for $\kappa_0 = \gamma(\sqrt{m}(e-1) +\sqrt{md}+3/2)^{-1}(m+1)^{-1/2}$.  The result follows.
% }

\proofof{(A1) in Section~\ref{subsec:loc_lin_conv}}
From Proposition~\ref{prop:alpha_grad}, there exists $k_1 \geq 0$ such that $\alpha_k\left\|\grad \widetilde{F}(\widetilde{\mathbf{S}}^k)\right\|_F \leq 1/2$ for all $k \geq k_1$. Then note that 
$$\left\|\widetilde{\mathbf{S}}^{k+1} - \widetilde{\mathbf{S}}^k\right\|_F = \left\|\widetilde{R}_\EXP(\widetilde{\mathbf{S}}^k, -\alpha_k\grad \widetilde{F}(\widetilde{\mathbf{S}}^k)) - \widetilde{\mathbf{S}}^k\right\|_F.$$
From Proposition~\ref{prop:second_order_boundedness_of_Rtilde}, for all $k \geq k_1$,
\begin{align}
    \left\|\widetilde{R}_\EXP(\widetilde{\mathbf{S}}^k, -\alpha_k\grad \widetilde{F}(\widetilde{\mathbf{S}}^k)) - \widetilde{\mathbf{S}}^k\right\|_F &\leq (e-1) \alpha_k^2 \left\|\grad \widetilde{F}(\widetilde{\mathbf{S}}^k)\right\|_F^2 + \alpha_k\left\|\grad \widetilde{F}(\widetilde{\mathbf{S}}^k)\right\|_F\\
    &\leq \frac{1}{2}(e+1)\alpha_k \left\|\grad \widetilde{F}(\widetilde{\mathbf{S}}^k)\right\|_F
\end{align}
Finally, using Eq~(\ref{supp:eq:eq1}) and Proposition~\ref{prop:hlift_frob_ineq}, for all $k \geq k_1$, 
$$\widetilde{F}(\widetilde{\mathbf{S}}^{k+1})-\widetilde{F}(\widetilde{\mathbf{S}}^{k}) \leq -2\gamma(e+1)^{-1}(m+1)^{-1/2} \left\|\grad \widetilde{F}(\widetilde{\mathbf{S}}^k)\right\|_F \cdot \left\|\widetilde{\mathbf{S}}^{k+1}-\widetilde{\mathbf{S}}^k\right\|_F.$$
Thus, (\textbf{A1}) holds for $\kappa_0 = 2\gamma(e+1)^{-1}(m+1)^{-1/2}$. The result follows.

\proofof{(A3) in Section~\ref{subsec:loc_lin_conv}}
Since $\nabla F(\mathbf{S}) = 2\mathbf{C}\mathbf{S}$ is Lipschitz with parameter $L_F \leq 2\left\|\mathbf{C}\right\|_F$, the proof of \textbf{(A3)} is same as in \citeb[Pg. 235]{liu2019quadratic} (alternatively \citeb[Theorem 2.10]{schneider2015convergence}). For the sake of completeness, we present an adaptation of their proof to our setting.
% The proof is essentially the same as the one in \citea{liu2019quadratic}. Due to (A2) in Section~\ref{subsec:loc_sub_conv}, WLOG we assume that $\grad F(\mathbf{S}^k) \neq 0$ for all $k \geq 0$. Then, using line 5 in Algorithm~\ref{algo:rgd} and Proposition~\ref{prop:second_order_boundedness_of_RQR}, we obtain
% \begin{align}
%     \left\|\mathbf{S}^{k+1} - \mathbf{S}^k\right\|_F &= \left\|R_{QR}(\mathbf{S}^k, -\alpha_k\grad F(\mathbf{S}^k)) - \mathbf{S}^k\right\|_F\\
%     &= \left\|R_{QR}(\mathbf{S}^k, -\alpha_k\grad F(\mathbf{S}^k)) - (\mathbf{S}^k-\alpha_k\grad F(\mathbf{S}^k)) -\alpha_k\grad F(\mathbf{S}^k)\right\|_F\\
%     &\geq \left\|\alpha_k\grad F(\mathbf{S}^k)\right\|_F - \left\|R_{QR}(\mathbf{S}^k, -\alpha_k\grad F(\mathbf{S}^k)) - (\mathbf{S}^k-\alpha_k\grad F(\mathbf{S}^k))\right\|_F\\
%     &\geq \left\|\alpha_k\grad F(\mathbf{S}^k)\right\|_F - M \alpha_k^2 \left\|\grad F(\mathbf{S}^k)\right\|_F^2
% \end{align}
% Dividing by $\left\|\alpha_k\grad F(\mathbf{S}^k)\right\|_F^2$ and using Proposition~\ref{prop:alpha_grad}, we obtain
% \begin{align}
%     \lim \frac{\left\|\mathbf{S}^{k+1} - \mathbf{S}^k\right\|_F}{|\alpha_k|\left\|\grad F(\mathbf{S}^k)\right\|_F} \geq 1
% \end{align}
WLOG we assume that $\alpha_k \left\|\grad \widetilde{F}(\widetilde{\mathbf{S}}^{k}))\right\|_F \neq 0$ for all $k \geq 0$. Then, $\widetilde{\mathbf{S}}^{k+1} - \widetilde{\mathbf{S}}^{k} = \widetilde{R}_\EXP(\widetilde{\mathbf{S}}^{k}, -\alpha_k \grad \widetilde{F}(\widetilde{\mathbf{S}}^{k})) - \widetilde{\mathbf{S}}^{k}$ and we have
\begin{align}
    \left\|\widetilde{\mathbf{S}}^{k+1}-\widetilde{\mathbf{S}}^{k}\right\|_F &\geq \alpha_k \left\|\grad \widetilde{F}(\widetilde{\mathbf{S}}^{k}))\right\|_F - \left\|\widetilde{R}_\EXP(\widetilde{\mathbf{S}}^{k}, -\alpha_k \grad \widetilde{F}(\widetilde{\mathbf{S}}^{k})) - (\widetilde{\mathbf{S}}^{k} - \grad \widetilde{F}(\widetilde{\mathbf{S}}^{k})))\right\|_F\\
    \left\|\widetilde{\mathbf{S}}^{k+1}-\widetilde{\mathbf{S}}^{k}\right\|_F &\leq \alpha_k \left\|\grad \widetilde{F}(\widetilde{\mathbf{S}}^{k}))\right\|_F + \left\|\widetilde{R}_\EXP(\widetilde{\mathbf{S}}^{k}, -\alpha_k \grad \widetilde{F}(\widetilde{\mathbf{S}}^{k})) - (\widetilde{\mathbf{S}}^{k} - \grad \widetilde{F}(\widetilde{\mathbf{S}}^{k})))\right\|_F.
\end{align}
Using Proposition~\ref{prop:second_order_boundedness_of_Rtilde}, we obtain 
$$\lim \frac{\left\|\widetilde{\mathbf{S}}^{k+1}-\widetilde{\mathbf{S}}^{k}\right\|_F}{\alpha_k\left\|\grad \widetilde{F}(\widetilde{\mathbf{S}}^{k}))\right\|_F} = 1.$$
% \begin{equation}
%     \lim \frac{\left\|\widetilde{\mathbf{S}}^{k+1}-\widetilde{\mathbf{S}}^{k}\right\|_F}{\alpha_k\left\|\grad \widetilde{F}(\widetilde{\mathbf{S}}^{k}))\right\|_F} = 1.
% \end{equation}
It suffices to show that $\liminf \alpha_k > 0$. Let $\overline{\alpha}_k = \overline{\alpha}(\mathbf{S}^k) > 0$ where 
\begin{equation}
    \overline{\alpha}(\mathbf{S}) = \inf\{\alpha > 0| F(R_\EXP(\mathbf{S}, -\alpha \grad F(\mathbf{S}))) - F(\mathbf{S}) = -\gamma \alpha g(\nabla F(\mathbf{S}), \grad F(\mathbf{S}))\}.
\end{equation}
where $\overline{\alpha}(\mathbf{S})$ is well defined due to \citeb[Proposition 2.8]{schneider2015convergence} since $F$ extends to a continuously differentiable non-negative function on $\mathbb{R}^{md \times d}$ containing $\mathbb{O}(d)^m$. By Eq.~(\ref{eq:armijo_step}) and above equation, we have $\alpha_k = 1$ if $\overline{\alpha}_k \geq 1$ and $\alpha_k \geq \beta \overline{\alpha}_k$ if $\overline{\alpha}_k < 1$.
% \begin{equation}
%     \begin{matrix}
%         \alpha_k = 1 \text{ if } \overline{\alpha}_k \geq 1; & \alpha_k \geq \beta \overline{\alpha}_k \text{ if } \overline{\alpha}_k < 1.
%     \end{matrix}
% \end{equation}
It suffices to assume that $\overline{\alpha}_k < 1$ for all $k \geq 0$ and show that $\liminf \overline{\alpha}_k > 0$. From the above equation, it follows that $\overline{\alpha}_k \left\|\grad F(\mathbf{S}^{k}))\right\|_F \leq (\alpha_k/\beta)\left\|\grad F(\mathbf{S}^{k}))\right\|_F$ which combined with Proposition~\ref{prop:alpha_grad} implies that $\lim\overline{\alpha}_k \left\|\grad F(\mathbf{S}^k)\right\|_F = 0$.
% \begin{equation}
%     \lim\overline{\alpha}_k \left\|\grad F(\mathbf{S}^k)\right\|_F = 0.
% \end{equation}
By mean value theorem and the definition of $\overline{\alpha}_k$, there exist $\zeta_k \in (0,1)$ such that 
$$\mathbf{U}^k = \zeta_k(R_\EXP(\mathbf{S}^k, -\overline{\alpha}_k \grad F(\mathbf{S}^k)) - \mathbf{S}^k)$$
satisfies
\begin{align}
    (R_\EXP(\mathbf{S}, -\overline{\alpha}_k \grad F(\mathbf{S}))- \mathbf{S}^k)^T\nabla F(\mathbf{S}^k + \mathbf{U}_k) &= F(R_\EXP(\mathbf{S}, -\overline{\alpha}_k \grad F(\mathbf{S}))) - F(\mathbf{S}^k)\\
    &= -\gamma \overline{\alpha}_k g(\nabla F(\mathbf{S}^k), \grad F(\mathbf{S}^k)). \label{eq:Uk}
\end{align}
Moreover, for sufficiently large $k \geq 0$, $\overline{\alpha}_k\left\|\grad F(\mathbf{S}^k)\right\|_F < 1$, therefore using Proposition~\ref{prop:liu_pf} and the triangle inequality,
\begin{equation}
    \left\|\mathbf{U}^k\right\|_F \leq \left\|R_\EXP(\mathbf{S}^k, -\overline{\alpha}_k \grad F(\mathbf{S}^k)) - \mathbf{S}^k\right\|_F \leq e\overline{\alpha}_k\left\|\grad F(\mathbf{S}^k)\right\|_F.
\end{equation}
Then we obtain the following set of inequalities using the above inequality, the fact that $\nabla F$ is Lipschitz continuous with parameter $L_F \leq 2 \left\|\mathbf{C}\right\|_F$, using Cauchy-Schwarz inequality, Eq~(\ref{eq:Uk}) and the triangle inequality,

\begin{align}
    \overline{\alpha}_k^2&\left\|\grad F(\mathbf{S}^k)\right\|_F^2 \geq e^{-1}\left\|\mathbf{U}_k\right\|_F \left\|R_\EXP(\mathbf{S}^k, - \alpha_k \grad F(\mathbf{S}^k)) - \mathbf{S}^k\right\|_F\\
    &\geq (eL_F)^{-1} \left\|\nabla F(\mathbf{S}^k) - \nabla F(\mathbf{S}^k + \mathbf{U}_k)\right\|_F \left\|R_\EXP(\mathbf{S}^k, - \overline{\alpha}_k\grad F(\mathbf{S}^k)) - \mathbf{S}^k\right\|_F\\
    &\geq (eL_F)^{-1} |g(\nabla F(\mathbf{S}^k) - \nabla F(\mathbf{S}^k + \mathbf{U}_k), R_\EXP(\mathbf{S}^k, - \overline{\alpha}_k\grad F(\mathbf{S}^k)) - \mathbf{S}^k)|\\
    &= (eL_F)^{-1}|g(\nabla F(\mathbf{S}^k), R_\EXP(\mathbf{S}^k, - \overline{\alpha}_k\grad F(\mathbf{S}^k)) - \mathbf{S}^k) + \gamma \overline{\alpha}_k g(\nabla F(\mathbf{S}^k), \grad F(\mathbf{S}^k))|\\
    &\geq (1-\gamma)\overline{\alpha}_k(eL_F)^{-1} |g(\nabla F(\mathbf{S}^k),\grad F(\mathbf{S}^k))| -\\
    &\qquad (eL_F)^{-1}|g(\nabla F(\mathbf{S}^k), R_\EXP(\mathbf{S}^k, - \overline{\alpha}_k\grad F(\mathbf{S}^k)) - (\mathbf{S}^k - \overline{\alpha}_k\grad F(\mathbf{S}^k))|.
\end{align}
Combining with $g(\nabla F(\mathbf{S}^k),\grad F(\mathbf{S}^k)) = \left\|\grad F(\mathbf{S}^k)\right\|_F^2$ (proof of Proposition~\ref{prop:alpha_grad}), using Cauchy-Schwarz inequality and dividing by $\overline{\alpha}_k\left\|\grad F(\mathbf{S}^k)\right\|_F^2$, we obtain
\begin{equation}
    \overline{\alpha}_k \geq (1-\gamma)(eL_{F})^{-1} - (eL_{F})^{-1}  \frac{\left\|\nabla F(\mathbf{S})\right\|_F\left\|R_\EXP(\mathbf{S}^k, - \overline{\alpha}_k\grad F(\mathbf{S}^k)) - (\mathbf{S}^k - \overline{\alpha}_k\grad F(\mathbf{S}^k))\right\|_F}{\overline{\alpha}_k\left\|\grad F(\mathbf{S}^k)\right\|_F^2}.
\end{equation}
Finally, since $\left\|\nabla F(\mathbf{S})\right\|_F \leq 2\sqrt{md} \left\|\mathbf{C}\right\|_F$ and using Proposition~\ref{prop:second_order_boundedness_of_Rtilde},
% \begin{equation}
%     \left\|R_\EXP(\mathbf{S}^k, - \overline{\alpha}_k\grad F(\mathbf{S}^k)) - (\mathbf{S}^k - \overline{\alpha}_k\grad F(\mathbf{S}^k))\right\|_F \leq \overline{\alpha}_k^2\left\|\grad F(\mathbf{S}^k)\right\|_F^2,
% \end{equation}
we obtain 
\begin{equation}
    \liminf \overline{\alpha}_k \geq \frac{1-\gamma}{eL_F + 2\sqrt{md}\left\|\mathbf{C}\right\|_F} > 0.
\end{equation}
% \revdel{\proofof{Proposition~\ref{prop:morse_bott_1}}
% Since $\pi(\mathbf{S}^*)$ is non-degenerate, $\widetilde{F}$ being Morse-Bott at $\pi(\mathbf{S}^*)$ follows from \cite[Definition 6.5]{usevich2020approximate}. Since non-degenerate critical points of a smooth function are isolated from other critical points, it follows again from \citea{usevich2020approximate} that $\widetilde{F}$ is Morse-Bott in a sufficiently small neighborhood of $\pi(\mathbf{S}^*)$. 

% Then, since the action of $\mathbb{O}(d)$ on $\mathbb{O}(d)^m$ is free and the function $F$ is invariant under the action of $\mathbb{O}(d)$ ($\mathbf{S}\mathbf{Q} = \mathbf{S} \iff \mathbf{Q} = \mathbf{I}_d$ and $F(\mathbf{S}\mathbf{Q}) = F(\mathbf{S})$ for all $\mathbf{S} \in \mathbb{O}(d)^m, \mathbf{Q} \in \mathbb{O}(d)$), the result follows from \citeb[Section 15.2]{cohen_iga_norbury_2006}.% and \citea{austin1995morse}.
% }

\proofof{Lemma~\ref{lem:quadgrowth}}
Since $\mathbf{S}_0$ is a unique perfect alignment, all other perfect alignments are of the form $\mathbf{S}_0\mathbf{Q}$ where $\mathbf{Q}\in \mathbb{O}(d)$. Moreover, the null space of $\mathbf{C}_0$ is exactly the span of columns of $\mathbf{S}_0$. Therefore, we obtain the following decomposition of $\mathbf{C}_0 = \mathbf{U}_0\boldsymbol{\Lambda}_0 \mathbf{U}_0^T$ where $\mathbf{U}_0^T\mathbf{U}_0 = \mathbf{I}_{(m-1)d}$, $\mathbf{S}_0^T \mathbf{U}_0 = 0$ and $\boldsymbol{\Lambda}_0$ is a diagonal matrix containing the strictly positive eigenvalues of $\mathbf{C}_0$. Using the above decomposition, we have 
$$\Tr(\mathbf{C}_0\mathbf{S}\mathbf{S}^T) = \Tr(\mathbf{U}_0\boldsymbol{\Lambda}_0 \mathbf{U}_0^T\mathbf{S}\mathbf{S}^T) \geq \lambda_{d+1}(\mathbf{C}_0) \left\|\mathbf{U}_0^T\mathbf{S}\right\|_F^2.$$

\textbf{Claim:} $\left\|\mathbf{U}_0^T\mathbf{S}\right\|_F^2 \geq \frac{1}{2}\min_{\mathbf{Q} \in \mathbb{O}(d)}\left\|\mathbf{S}- \mathbf{S}_0\mathbf{Q}\right\|_F^2$. Since the union of the columns of $\mathbf{S}_0$ and $\mathbf{U}_0$ span $\mathbb{R}^{md}$ therefore there exist $\mathbf{R}_1 \in \mathbb{R}^{d \times d}$ and $\mathbf{R}_2 \in \mathbb{R}^{(m-1)d \times d}$ such that $\mathbf{S} = \mathbf{S}_0 \mathbf{R}_1 + \mathbf{U}_0 \mathbf{R}_2$. Let $\mathbf{Q}^* \in \mathbb{O}(d)$ be such that $\left\|\mathbf{S}- \mathbf{S}_0\mathbf{Q}^*\right\|_F^2 = \min_{\mathbf{Q} \in \mathbb{O}(d)^m} \left\|\mathbf{S}- \mathbf{S}_0\mathbf{Q}\right\|_F^2$. Then $\mathbf{Q}^* = \mathbf{U}_1\mathbf{V}_1^T$ where $\mathbf{R}_1 = \mathbf{U}_1\boldsymbol{\Sigma}_1\mathbf{V}_1^T$ is a singular vector decomposition of $\mathbf{R}_1$. Using $\mathbf{S}^T\mathbf{S} = \mathbf{S}_0^T\mathbf{S}_0 = m\mathbf{I}_d$ and $\mathbf{R}_1 = \mathbf{U}_1\boldsymbol{\Sigma}_1\mathbf{V}_1^T$,
\begin{align}
    \left\|\mathbf{U}_0^T\mathbf{S}\right\|_F^2 &= \left\|\mathbf{R}_2\right\|_F^2 = \left\|\mathbf{S}-\mathbf{S}_0\mathbf{R}_1\right\|_F^2\\
    &= md + m \left\|\mathbf{R}_1\right\|_F^2 - 2\Tr(\mathbf{S}\mathbf{R}_1^T\mathbf{S}_0^T)\\
    &= md + m\left\|\boldsymbol{\Sigma}\right\|_F^2 - 2\Tr(\mathbf{S}\mathbf{V}_1\boldsymbol{\Sigma}_1\mathbf{U}_1^T\mathbf{S}_0^T)
\end{align}
% \begin{equation}
%     \left\|\mathbf{U}_0^T\mathbf{S}\right\|_F^2 = md + m\left\|\boldsymbol{\Sigma}\right\|_F^2 - 2\Tr(\mathbf{S}\mathbf{V}_1\boldsymbol{\Sigma}_1\mathbf{U}_1^T\mathbf{S}_0^T)
% \end{equation}}
% $\left\|\mathbf{U}_0^T\mathbf{S}\right\|_F^2 = md + m\left\|\boldsymbol{\Sigma}\right\|_F^2 - 2\Tr(\mathbf{S}\mathbf{V}_1\boldsymbol{\Sigma}_1\mathbf{U}_1^T\mathbf{S}_0^T)$.
% \begin{align}
%     \left\|\mathbf{U}_0^T\mathbf{S}\right\|_F^2 = \left\|\mathbf{S}-\mathbf{S}_0\mathbf{R}_1\right\|_F^2 &= md + m \left\|\mathbf{R}_1\right\|_F^2 - 2\Tr(\mathbf{S}\mathbf{R}_1^T\mathbf{S}_0^T)\\
%     &= md + m\left\|\boldsymbol{\Sigma}\right\|_F^2 - 2\Tr(\mathbf{S}\mathbf{V}_1\boldsymbol{\Sigma}_1\mathbf{U}_1^T\mathbf{S}_0^T).
% \end{align}
Moreover, using the fact that $\mathbf{S}_0^T\mathbf{S}_0 = m\mathbf{I}_d$ and the definition of $\mathbf{Q}^*$,
% \begin{equation}
%     \revadd{\left\|\mathbf{S} - \mathbf{S}_0\mathbf{Q}^*\right\|_F^2 = \left\|\mathbf{U}_0\mathbf{R}_2\right\|_F^2 + \left\|\mathbf{S}_0(\mathbf{R}_1 - \mathbf{Q}^*)\right\|_F^2}
% \end{equation}
% \revadd{where $\left\|\mathbf{U}_0\mathbf{R}_2\right\|_F^2 = \left\|\mathbf{S} - \mathbf{S}_0\mathbf{R}_1\right\|_F^2 = \left\|\mathbf{R}_2\right\|_F^2$, and $\left\|\mathbf{S}_0(\mathbf{R}_1 - \mathbf{Q}^*)\right\|_F^2 = m\left\|\mathbf{R}_1 - \mathbf{Q}^*\right\|_F^2$, which reduces to $m\left\|\mathbf{I}_d-\boldsymbol{\Sigma}_1\right\|_F^2$ using $\mathbf{S}_0^T\mathbf{S}_0 = m\mathbf{I}_d$ and the definition of $\mathbf{Q}^*$
{\allowdisplaybreaks
\begin{align}
    \left\|\mathbf{S} - \mathbf{S}_0\mathbf{Q}^*\right\|_F^2 &= \left\|\mathbf{U}_0\mathbf{R}_2\right\|_F^2 + \left\|\mathbf{S}_0(\mathbf{R}_1 - \mathbf{Q}^*)\right\|_F^2 = \left\|\mathbf{S} - \mathbf{S}_0\mathbf{R}_1\right\|_F^2 + m\left\|\mathbf{R}_1 - \mathbf{Q}^*\right\|_F^2\\
    &= \left\|\mathbf{S} - \mathbf{S}_0\mathbf{R}_1\right\|_F^2 + m\left\|\mathbf{I}_d-\boldsymbol{\Sigma}_1\right\|_F^2\\
    &= \left\|\mathbf{S} - \mathbf{S}_0\mathbf{R}_1\right\|_F^2 + m(d + \left\|\boldsymbol{\Sigma}_1\right\|_F^2 - 2\Tr(\boldsymbol{\Sigma}_1))\\
    &= 2(md + m\left\|\boldsymbol{\Sigma}_1\right\|_F^2) - 2(\Tr(\mathbf{S}\mathbf{V}_1\boldsymbol{\Sigma}_1\mathbf{U}_1^T\mathbf{S}_0^T) + m\Tr(\boldsymbol{\Sigma}_1))
\end{align}
}
Overall, 
$$\left\|\mathbf{U}_0^T\mathbf{S}\right\|_F^2 - \frac{1}{2}\left\|\mathbf{S} - \mathbf{S}_0\mathbf{Q}^*\right\|_F^2 = m\Tr(\boldsymbol{\Sigma}_1) - \Tr(\mathbf{S}\mathbf{V}_1\boldsymbol{\Sigma}_1\mathbf{U}_1^T\mathbf{S}_0^T) = \sum_{1}^{m}(\Tr(\boldsymbol{\Sigma}_1) - \Tr(\mathbf{U}_1^T\mathbf{S}_{0_i}^T\mathbf{S}_i\mathbf{V}_1\boldsymbol{\Sigma}_1)).$$
Since $\max_{\mathbf{Q} \in \mathbb{O}(d)}\Tr(\mathbf{Q} \boldsymbol{\Sigma}_1) = \Tr(\boldsymbol{\Sigma}_1)$, the result follows.
% \begin{align}
%     \left\|\mathbf{U}_0^T\mathbf{S}\right\|_F^2 - \frac{1}{2}\left\|\mathbf{S} - \mathbf{S}_0\mathbf{Q}^*\right\|_F^2 &= m\Tr(\boldsymbol{\Sigma}_1) - \Tr(\mathbf{S}\mathbf{V}_1\boldsymbol{\Sigma}_1\mathbf{U}_1^T\mathbf{S}_0^T)\\
%     &= \sum_{1}^{m}(\Tr(\boldsymbol{\Sigma}_1) - \Tr(\mathbf{U}_1^T\mathbf{S}_{0_i}^T\mathbf{S}_i\mathbf{V}_1\boldsymbol{\Sigma}_1))\geq 0
% \end{align}
% where the last inequality follows from the fact that $\max_{\mathbf{Q} \in \mathbb{O}(d)}\Tr(\mathbf{Q} \boldsymbol{\Sigma}_1) = \Tr(\boldsymbol{\Sigma}_1)$.

\proofof{Lemma~\ref{lem:distS_0Sstar}}
The proof is motivated from \citeb[Proposition~4.32]{bonnans2013perturbation}. Define 
$$H(\mathbf{S}) = \Tr(\mathbf{C}\mathbf{S}\mathbf{S}^T) - \Tr(\mathbf{C}_0\mathbf{S}\mathbf{S}^T)$$
and note that $H(\mathbf{S})$ is Lipschitz with a Lipschitz constant bounded by $\left\|\nabla H(\mathbf{S})\right\|_F \leq 2m \left\|\mathbf{C}-\mathbf{C}_0\right\|_F$. 

Let $\mathbf{Q}^* \in \mathbb{O}(d)$ be such that 
$$\left\|\mathbf{S}^* - \mathbf{S}_0\mathbf{Q}^*\right\|_F = \min_{\mathbf{Q}\in\mathbb{O}(d)}\left\|\mathbf{S}^* - \mathbf{S}_0\mathbf{Q}\right\|_F.$$
Then, using the mean value theorem and the fact the $\mathbf{S}^*$ is an optimal alignment in the noisy setting, meaning $\Tr(\mathbf{C}\mathbf{S}^*\mathbf{S}^{*^T}) \leq \Tr(\mathbf{C}\mathbf{S}_0\mathbf{Q}^*(\mathbf{S}_0\mathbf{Q}^*)^T)$,
\begin{align}
    \Tr(\mathbf{C}_0\mathbf{S}^*\mathbf{S}^{*^T}) - \Tr(\mathbf{C}_0\mathbf{S}_0\mathbf{Q}^*(\mathbf{S}_0\mathbf{Q}^*)^T) &= H(\mathbf{S}_0\mathbf{Q}^*) - H(\mathbf{S}^*) + (\Tr(\mathbf{C}\mathbf{S}^*\mathbf{S}^{*^T}) - \Tr(\mathbf{C}\mathbf{S}_0\mathbf{Q}^*(\mathbf{S}_0\mathbf{Q}^*)^T))\\
    &\leq 2m \left\|\mathbf{C}-\mathbf{C}_0\right\|_F \left\|\mathbf{S}^*-\mathbf{S}_0\mathbf{Q}^*\right\|_F
\end{align}
% \begin{align}
%     &\Tr(\mathbf{C}_0\mathbf{S}^*\mathbf{S}^{*^T}) - \Tr(\mathbf{C}_0\mathbf{S}_0\mathbf{Q}^*(\mathbf{S}_0\mathbf{Q}^*)^T)\\
%     &= H(\mathbf{S}_0\mathbf{Q}^*) - H(\mathbf{S}^*) + (\Tr(\mathbf{C}\mathbf{S}^*\mathbf{S}^{*^T}) - \Tr(\mathbf{C}\mathbf{S}_0\mathbf{Q}^*(\mathbf{S}_0\mathbf{Q}^*)^T))\\
%     &\leq 2m \left\|\mathbf{C}-\mathbf{C}_0\right\|_F \left\|\mathbf{S}^*-\mathbf{S}_0\mathbf{Q}^*\right\|_F
% \end{align}
% where the last inequality follows from the mean value theorem and the fact the $\mathbf{S}^*$ is an optimal alignment in the noisy setting, meaning $\Tr(\mathbf{C}\mathbf{S}^*\mathbf{S}^{*^T}) \leq \Tr(\mathbf{C}\mathbf{S}_0\mathbf{Q}^*(\mathbf{S}_0\mathbf{Q}^*)^T)$. 
Combining with Lemma~\ref{lem:quadgrowth} and $\mathbf{C}_0\mathbf{S}_0 = 0$, we obtain 
$$(\lambda_{d+1}(\mathbf{C}_0)/2) \left\|\mathbf{S}^*- \mathbf{S}_0\mathbf{Q}^*\right\|_F^2 \leq 2m \left\|\mathbf{C}-\mathbf{C}_0\right\|_F \left\|\mathbf{S}^*-\mathbf{S}_0\mathbf{Q}^*\right\|_F.$$
% \begin{align}
%      \frac{\lambda_{d+1}(\mathbf{C}_0)}{2} \left\|\mathbf{S}^*- \mathbf{S}_0\mathbf{Q}^*\right\|_F^2 \leq 2m \left\|\mathbf{C}-\mathbf{C}_0\right\|_F \left\|\mathbf{S}^*-\mathbf{S}_0\mathbf{Q}^*\right\|_F.
% \end{align}
The result follows.

\proofof{Theorem~\ref{thm:rgd_noise_stability}} \revadd{The following bound holds from \citeb[Eq.~(5.8, 5.11, 5.12)]{chaudhury2015global}
\begin{equation}
    \min_{\mathbf{Q} \in \mathbb{O}(d)} \left\|\mathbf{S}_{spec}(\mathbf{C}) - \mathbf{S}_0\mathbf{Q}\right\|_F \leq \frac{4\pi\sqrt{md(d+1)}}{\lambda_{d+1}(\mathbf{C})}(K_1 \varepsilon + K_2\varepsilon^2). \label{eq:spec_bound}
\end{equation}
Let $\mathbf{Q}^*$ be such that $\left\|\mathbf{S}^* - \mathbf{S}_0\mathbf{Q}^*\right\|_F = \min_{\mathbf{Q}\in\mathbb{O}(d)}\left\|\mathbf{S}^* - \mathbf{S}_0\mathbf{Q}\right\|_F$. Then,
\begin{align}
    \left\|\mathbf{S}_{spec}(\mathbf{C}) - \mathbf{S}^*\mathbf{Q}\right\|_F &\leq \left\|\mathbf{S}_{spec}(\mathbf{C}) - \mathbf{S}_0\mathbf{Q}^*\mathbf{Q}\right\|_F + \left\|\mathbf{S}^*\mathbf{Q} - \mathbf{S}_0\mathbf{Q}^*\mathbf{Q}\right\|_F\\
    &\leq \left\|\mathbf{S}_{spec}(\mathbf{C}) - \mathbf{S}_0\mathbf{Q}^*\mathbf{Q}\right\|_F + \left\|\mathbf{S}^* - \mathbf{S}_0\mathbf{Q}^*\right\|_F
\end{align}
Minimizing the above over $\mathbf{Q}$, using the fact that 
$$\min_{\mathbf{Q}\in\mathbb{O}(d)}\left\|\mathbf{S}_{spec}(\mathbf{C}) - \mathbf{S}_0\mathbf{Q}^*\mathbf{Q}\right\|_F = \min_{\mathbf{Q}\in\mathbb{O}(d)}\left\|\mathbf{S}_{spec}(\mathbf{C}) - \mathbf{S}_0\mathbf{Q}\right\|_F,$$
followed by Eq.~(\ref{eq:spec_bound}), Lemma~\ref{lem:distS_0Sstar} and Theorem~\ref{thm:rgd_conv2}, the result follows.}
%\input{sections/todos}
%\clearpage
%\setcitestyle{numbers}
\bibliographystyle{siamplain}
\bibliography{loc_rigid}

\end{document}
