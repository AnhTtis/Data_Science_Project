% SIAM Shared Information Template
% This is information that is shared between the main document and any
% supplement. If no supplement is required, then this information can
% be included directly in the main document.


% Packages and macros go here
\usepackage{lipsum}
\usepackage{amsfonts}
\usepackage{graphicx}
\usepackage{algorithmic}
\usepackage{multirow,makecell}
\usepackage{bm,amsfonts,booktabs,amssymb}
%\allowdisplaybreaks
\usepackage{amsmath}
\makeatletter
\g@addto@macro\normalsize{%
  \setlength\abovedisplayskip{7pt}
  \setlength\belowdisplayskip{7pt}
  \setlength\abovedisplayshortskip{8pt}
  \setlength\belowdisplayshortskip{8pt}
}
\makeatother
\usepackage{mathtools}
\usepackage{enumitem}
% \usepackage{caption}
% \captionsetup{belowskip=0pt}
\usepackage{subcaption}
\usepackage[font={small,it}]{caption}
\usepackage{comment}
\usepackage{nicematrix}
\usepackage{hyperref}
\usepackage{cleveref}
\usepackage{autonum}
\usepackage{longtable}
\usepackage{tikz}
\usetikzlibrary{positioning}
\usetikzlibrary{arrows}

\ifpdf
  \DeclareGraphicsExtensions{.eps,.pdf,.png,.jpg}
\else
  \DeclareGraphicsExtensions{.eps}
\fi

% Add a serial/Oxford comma by default.
\newcommand{\creflastconjunction}{, and~}

% Used for creating new theorem and remark environments
\newsiamremark{remark}{Remark}
\newsiamremark{hypothesis}{Hypothesis}
\crefname{hypothesis}{Hypothesis}{Hypotheses}
\newsiamthm{claim}{Claim}


\newcommand{\retainlabel}[1]{\label{#1}\sbox0{\ref{#1}}}

\newcommand*{\vertbar}{\rule[-1ex]{0.5pt}{2.5ex}}
\newcommand*{\horzbar}{\rule[.5ex]{2.5ex}{0.5pt}}
\newcolumntype{P}[1]{>{\centering\arraybackslash}p{#1}}
\newcolumntype{M}[1]{>{\centering\arraybackslash}m{#1}}
\newcommand{\makeheadbox}{\relax}
%\newtheorem{example}{Example}
%\setlength\parindent{0pt}
\setcounter{MaxMatrixCols}{20}

\def\TODO#1{{\color{red} \textbf{#1}}}

\newcommand{\ac}[1]{\textcolor{green}{Alex: #1}}

\newcommand{\dk}[1]{\textcolor{blue}{Dhruv: #1}}

\newcommand{\acomment}[1]{\textcolor{orange}{#1}}
\newcommand{\revadd}[1]{\textcolor{blue}{#1}}
%\newcommand{\revadd}[1]{#1}
\newcommand{\revdel}[1]{\textcolor{red}{}}

\newcommand{\gm}[1]{\textcolor{purple}{Gal: #1}}

\newcommand{\edit}[1]{\textcolor{black}{#1}}

\newcommand{\editt}[1]{\textcolor{black}{#1}}

\newcommand{\edittt}[1]{\textcolor{black}{#1}}

%\DeclareSymbolFont{ebgletters}{OML}{EBGaramond-Maths}{m}{it}
%\DeclareMathSymbol{w}{\mathalpha}{ebgletters}{`k}


\newsiamremark{assump}{Assumption}
\newsiamremark{dfn}{Definition}
\newsiamthm{thm}{Theorem}
\newsiamthm{cor}{Corollary}
\newsiamthm{lem}{Lemma}
\newsiamthm{prop}{Proposition}
\newsiamremark{rmk}{Remark}
\newsiamremark{ex}{Example}

% \newtheorem{assump}{Assumption}
% \newtheorem{dfn}{Definition}
% \newtheorem{thm}{Theorem}
% \newtheorem{cor}[thm]{Corollary}
% \newtheorem{lem}[thm]{Lemma}
% \newtheorem{prop}[thm]{Proposition}
% \newtheorem{rmk}{Remark}
% \newtheorem{ex}{Example}
%\newdefinition{note}{Note}
%\newproof{pf}{Proof}
\newtheorem{contrib}{Contribution}

\newcommand{\proofoffirst}[1]{\noindent \underline{\textbf{Proof of #1.}}}
\newcommand{\proofof}[1]{\smallskip \noindent \underline{\textbf{Proof of #1}}.}

\newcommand{\argmin}{\mathop{\mathrm{argmin}}}
\newcommand{\argmax}{\mathop{\mathrm{argmax}}}


% Definitions of handy macros can go here

\newcommand{\dataset}{{\cal D}}
\newcommand{\fracpartial}[2]{\frac{\partial #1}{\partial  #2}}

\newcommand{\Tr}{{\mathrm{Tr}}}
\newcommand{\QR}{{\mathrm{QR}}}
\newcommand{\PF}{{\mathrm{PF}}}
\newcommand{\EXP}{{\mathrm{Exp}}}
\newcommand{\pf}{{\mathrm{pf}}}
\newcommand{\qf}{{\mathrm{qf}}}
\newcommand{\grad}{{\mathrm{grad}}}
%\newcommand{\diag}{{\mathrm{diag}}}
\newcommand{\vecz}{{\mathrm{vec}}}
\newcommand{\Hess}{{\mathrm{Hess}}}
\newcommand{\blockdiag}{{\mathrm{block\text{-}diag}}}
\newcommand{\Skew}{{\mathrm{Skew}}}
\newcommand{\Sym}{{\mathrm{Sym}}}
\newcommand{\rank}{{\mathrm{rank}}}

\newif\iftodos
%\todostrue % comment out to hide answers

% Heading arguments are {volume}{year}{pages}{date submitted}{date published}{paper id}{author-full-names}

%\setcitestyle{square}

% \renewcommand{\citep}[2][]{[\citenum{#2}, {#1}]}
%\renewcommand{\citep}[2][]{\cite[{#1}]{#2}}
\newcommand{\citea}[2][]{\cite{#2}}
\newcommand{\citeb}[2][]{\cite[#1]{#2}}

\makeatletter
\newcommand*{\addFileDependency}[1]{% argument=file name and extension
\typeout{(#1)}% latexmk will find this if $recorder=0
% however, in that case, it will ignore #1 if it is a .aux or 
% .pdf file etc and it exists! If it doesn't exist, it will appear 
% in the list of dependents regardless)
%
% Write the following if you want it to appear in \listfiles 
% --- although not really necessary and latexmk doesn't use this
%
\@addtofilelist{#1}
%
% latexmk will find this message if #1 doesn't exist (yet)
\IfFileExists{#1}{}{\typeout{No file #1.}}
}\makeatother

\newcommand*{\myexternaldocument}[1]{%
\externaldocument[nocite]{#1}%
\addFileDependency{#1.tex}%
\addFileDependency{#1.aux}%
}


% Sets running headers as well as PDF title and authors
\headers{Non-degenerate Rigid Alignment in a Patch Framework}{D. Kohli, G. Mishne and A. Cloninger}

% Title. If the supplement option is on, then "Supplementary Material"
% is automatically inserted before the title.
\title{Non-degenerate Rigid Alignment in a Patch Framework\thanks{Submitted to the editors on Aug 11, 2023.}}

% Authors: full names plus addresses.
\author{Dhruv Kohli\thanks{Department of Mathematics, UC San Diego 
  (\email{dhkohli@ucsd.edu}).}
\and  Gal Mishne\thanks{Halicio\u{g}lu Data Science Institute, UC San Diego 
  (\url{gmishne@ucsd.edu}, \url{acloninger@ucsd.edu}).}
\and Alexander Cloninger\footnotemark[2]\ \footnotemark[3]}

\usepackage{amsopn}
\DeclareMathOperator{\diag}{diag}

\setlength{\belowcaptionskip}{0pt}
\setlength{\abovecaptionskip}{0pt}
\setlength{\intextsep}{0pt}

%%% Local Variables: 
%%% mode:latex
%%% TeX-master: "ex_article"
%%% End: 

\everypar{\looseness=-1}