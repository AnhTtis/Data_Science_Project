\section{Notation and Proofs}
\label{supp:sec:all_proofs}
$[a,b]$ is the set $\{a,\ldots,b\}$ where $a,b \in \mathbb{Z}$.
$(a_i)_1^k$ is the sequence $a_1,\ldots,a_k$ where $a_i$ is either a scalar or a vector or a matrix.
$\mathbf{e}^p_q$ is a vector of zeros of length $p$ with $1$ at the $q$th location.
$\mathbf{1}^{p}_q$ is a vector of zeros of length $p$ whose first $q$ elements are $1$s.
$\mathbf{1}_p$ equals $\mathbf{1}^{p}_p$.
$\mathbf{0}_p$ and $\mathbf{0}_{m \times n}$ is a vector of length $p$ and a matrix of zeros with $m$ rows and $n$ columns, respectively.
$[\mathbf{A}_i]_1^n$ denotes a matrix obtained by vertically stacking the matrices $(\mathbf{A}_i)_1^m$.
$[\mathbf{A}]_1^n$ equals $[\mathbf{A}_i]_1^n$ where $\mathbf{A}_i = \mathbf{A}$ for all $i \in [1,n]$.
$\mathbf{I}_d$ and $\mathbf{I}^m_d$ denotes the identity matrix of size $d$ and $[\mathbf{I}_d]_1^m$, respectively.
$\mathbf{A}_i$ is the $i$th row block of $\mathbf{A}$ (the dimensions are contextual).
$\mathbf{A}_{ij}$ is the $(i,j)$th block of the block matrix $\mathbf{A}$ (the dimensions are contextual).
$\vecz (\mathbf{A})$ denotes the column-major vectorization of the matrix $\mathbf{A}$.
$\blockdiag((\mathbf{A}_i)_1^m)$ is a block diagonal matrix with $\mathbf{A}_i$ as the $i$th block.
$\diag (\mathbf{v})$ is a diagonal matrix with $\mathbf{v}(i)$ as the $i$th diagonal element.
%$\mathbf{A}(k,:)$ ($\mathbf{A}(:,k)$) denotes $k$th row (column) of $\mathbf{A}$.
$\mathbf{A}(i:j,:) (\mathbf{A}(:,i:j))$ denotes a stacking of $i$th to $j$th rows (columns) of $\mathbf{A}$.
$\mathbb{O}(n)$ is the set of orthogonal matrices of size $n$.
$\Sym (n)$ and $\Skew (n)$ is the set of symmetric and skew-symmetric matrices of size $n$, respectively.
$\Sym(\mathbf{A})$ and $\Skew(\mathbf{A})$ equals $(\mathbf{A}+\mathbf{A}^T)/2$ and $(\mathbf{A}-\mathbf{A}^T)/2$, respectively.
$\mathbf{A} \otimes \mathbf{B}$ denotes the Kronecker product of $\mathbf{A}$ and $\mathbf{B}$. \revadd{For a symmetric matrix $\mathbf{A}$, $\lambda_{i}(\mathbf{A})$ denotes the $i$th smallest eigenvalue of $\mathbf{A}$ and $\sigma_{\min}(\mathbf{A})$ ($\sigma_{\max}(\mathbf{A})$) denotes the smallest (largest) singular value of $\mathbf{A}$.}
% \begin{longtable}{|l|l|}
%     \hline
%     $[a,b]$& the set $\{a,\ldots,b\}$ where $a,b \in \mathbb{Z}$\\
%     $(a_i)_1^k$& the sequence $a_1,\ldots,a_k$ where $a_i$ is a scalar/vector/matrix\\
%     $\mathbf{e}^p_q$& a vector of zeros of length $p$ with $1$ at the $q$th location\\
%     $\mathbf{1}^{p}_q$& a vector of zeros of length $p$ whose first $q$ elements are $1$s\\
%     $\mathbf{1}_p$& $\mathbf{1}^{p}_p$\\
%     $\mathbf{0}_p, \mathbf{0}_{m \times n}$ & a vector and a matrix of zeros\\
%     $[\mathbf{A}_i]_1^n$& a matrix obtained by vertically stacking the matrices $(\mathbf{A}_i)_1^m$\\
%     $[\mathbf{A}]_1^n$& $[\mathbf{A}_i]_1^n$ where $\mathbf{A}_i = \mathbf{A}$ for all $i \in [1,n]$\\
%     $\mathbf{I}_d$, $\mathbf{I}^m_d$ & the identity matrix of size $d$ and $[\mathbf{I}_d]_1^m$\\
%     $\mathbf{A}_i$ & $i$th row block of $\mathbf{A}$$^\sharp$\\
%     $\mathbf{A}_{ij}$ & $(i,j)$th block of the block matrix $\mathbf{A}$$^\sharp$\\
%     $\vecz (\mathbf{A})$ & the column-major vectorization of the matrix $\mathbf{A}$\\
%     $\blockdiag((\mathbf{A}_i)_1^m)$ & a block diagonal matrix with $\mathbf{A}_i$ as the $i$th block\\
%     $\diag (\mathbf{v})$ & a diagonal matrix with $\mathbf{v}(i)$ as the $i$th diagonal element\\
%     $\mathbf{A}(k,:)$ ($\mathbf{A}(:,k)$) & $k$th row (column) of $\mathbf{A}$\\
%     $\mathbf{A}(i:j,:) (\mathbf{A}(:,i:j))$ & a stacking of $i$th to $j$th rows (columns) of $\mathbf{A}$\\
%     $\mathbb{O}(n)$ & the set of orthogonal matrices of size $n$\\
%     $\Sym (n)$, $\Skew (n)$, & the set of symmetric and skew-symmetric matrices of size $n$\\
%     $\Sym(\mathbf{A})$, $\Skew(\mathbf{A})$ & $(\mathbf{A}+\mathbf{A}^T)/2$, $(\mathbf{A}-\mathbf{A}^T)/2$\\
%     $\mathbf{A} \otimes \mathbf{B}$ & the Kronecker product of $\mathbf{A}$ and $\mathbf{B}$\\
%     \hline
%     \multicolumn{2}{l}{\footnotesize{$^\sharp$Dimensions are contextual.}}\\
%     %\caption{Notation.}
%     % \label{supp:tab:notation}
% \end{longtable}

%%%%%%%%%%%%%%%%%%%%%%%%%%%%%%%%%%%%%%%%%%
% Section 2 Proofs
%%%%%%%%%%%%%%%%%%%%%%%%%%%%%%%%%%%%%%%%%%
\proofof{Proposition~\ref{prop:A_0_2_A_1}}
By differentiating the objective in Eq.~(\ref{eq:A_1}) with respect to $\mathbf{x}_k$, the optimal $\mathbf{x}_k^* \coloneqq \mathbf{x}_k((\mathbf{S}_i)_1^m,(\mathbf{t}_i)_1^m) = n_i^{-1}\textstyle\sum_{(k,i)\in E(\Gamma)}(\mathbf{S}_i^T\mathbf{x}_{k,i}+\mathbf{t}_i)$ (where $n_i = |\{i:(k,i)\in E(\Gamma)\}|$ is the number of points in the $i$th view) which is the consensus of all the views for the $k$th point. Since $\textstyle\sum_{(k,i)\in E(\Gamma)}(\mathbf{S}_i^T\mathbf{x}_{k,i}+\mathbf{t}_i-\mathbf{x}_k^*) = 0$, by adding and subtracting $\mathbf{x}_k^*$, we conclude that 
$$\textstyle\sum_{\substack{(k,i)\in E(\Gamma)\\(k,j)\in E(\Gamma)}}\left\|(\mathbf{S}_i^T\mathbf{x}_{k,i}+\mathbf{t}_i)-(\mathbf{S}_j^T\mathbf{x}_{k,j}+\mathbf{t}_j)\right\|^2_2 = 2\textstyle\sum_{(k,i)\in E(\Gamma)}\left\|(\mathbf{S}_i^T\mathbf{x}_{k,i}+\mathbf{t}_i)-\mathbf{x}_k^*\right\|^2_2$$
% \begin{equation}
%     \textstyle\sum_{\substack{(k,i)\in E(\Gamma)\\(k,j)\in E(\Gamma)}}\left\|(\mathbf{S}_i^T\mathbf{x}_{k,i}+\mathbf{t}_i)-(\mathbf{S}_j^T\mathbf{x}_{k,j}+\mathbf{t}_j)\right\|^2_2 = 2\textstyle\sum_{(k,i)\in E(\Gamma)}\left\|(\mathbf{S}_i^T\mathbf{x}_{k,i}+\mathbf{t}_i)-\mathbf{x}_k^*\right\|^2_2
% \end{equation}
and the result follows. The above equality also shows that the minimizers
%$(\mathbf{S}_i)_1^m$ and $(\mathbf{t}_i)_1^m$
of Eq.~(\ref{eq:A_0}) and Eq.~(\ref{eq:A_1}) are the same.

\proofof{Proposition~\ref{prop:kerB}}
Let $\mathbf{u} \in \mathbb{R}^{n+m}$. Since $\boldsymbol{\mathcal{L}}_{\Gamma} \succeq 0$, $\boldsymbol{\mathcal{L}}_{\Gamma}\mathbf{u} = 0 \iff \mathbf{u}^T\boldsymbol{\mathcal{L}}_{\Gamma}\mathbf{u} = 0$. From Eq.~(\ref{eq:L_Gamma}), the latter holds iff $\mathbf{e}_{ki}^T\mathbf{u} = 0$ for all $(k,i) \in E(\Gamma)$. Thus, using Eq.~(\ref{eq:B}), $\boldsymbol{\mathcal{L}}_{\Gamma}\mathbf{u} = 0$ implies $\mathbf{B}\mathbf{u} = 0$. 

% %%%%%%%%%%%%%%%%%%%%%%%%%%%%%%%%%%%%%%%%%%
% % Section 3 Proofs
% %%%%%%%%%%%%%%%%%%%%%%%%%%%%%%%%%%%%%%%%%%
\proofoffirst{Proposition~\ref{prop:V_S_H_S}}
Using Eq.~(\ref{eq:pi_inv_wtS}),
\begin{align}
    \mathcal{V}_{\mathbf{S}} &= T_{\mathbf{S}}\pi^{-1}(\widetilde{\mathbf{S}}) = \{\mathbf{Z} \in \mathbb{R}^{md \times d}: \mathbf{Z}_i\mathbf{S}_1^T+\mathbf{S}_i\mathbf{Z}_1^T = 0, \mathbf{Z}_j\mathbf{S}_j^T+\mathbf{S}_j\mathbf{Z}_j^T=0, i \in [2,m], j \in [1,m]\}\\
    &= \{[\mathbf{S}_i\boldsymbol{\Omega}_i]_1^m: \boldsymbol{\Omega}_i \in \Skew(d), \boldsymbol{\Omega}_i+\boldsymbol{\Omega}_1^T = 0, i \in [1,m]\}\\
    &= \{\mathbf{S}\boldsymbol{\Omega}: \boldsymbol{\Omega} \in \Skew(d)\}
\end{align}
% \begin{align}
%     \mathcal{V}_{\mathbf{S}} &= T_{\mathbf{S}}\pi^{-1}(\widetilde{\mathbf{S}})\\
%     &= \{\mathbf{Z} \in \mathbb{R}^{md \times d}: \mathbf{Z}_i\mathbf{S}_1^T+\mathbf{S}_i\mathbf{Z}_1^T = 0, \mathbf{Z}_j\mathbf{S}_j^T+\mathbf{S}_j\mathbf{Z}_j^T=0, i \in [2,m], j \in [1,m]\}\\
%     &= \{[\mathbf{S}_i\boldsymbol{\Omega}_i]_1^m: \boldsymbol{\Omega}_i \in \Skew(d), \boldsymbol{\Omega}_i+\boldsymbol{\Omega}_1^T = 0, i \in [1,m]\} = \{\mathbf{S}\boldsymbol{\Omega}: \boldsymbol{\Omega} \in \Skew(d)\}.
% \end{align}

Since the objective in Eq.~(\ref{eq:P^v_S}) is strictly convex in $\boldsymbol{\Omega}$, it suffices to solve for the critical point and the Eq.~(\ref{eq:P^v_S}) follows immediately. Then, using Eq.~(\ref{eq:g_Z_W}), we obtain
\begin{align}
    \mathcal{H}_{\mathbf{S}} &= \mathcal{V}_{\mathbf{S}}^\perp = \{\mathbf{W} \in T_{\mathbf{S}}\mathbb{O}(d)^m: \Tr(\mathbf{Z}^T\mathbf{W}) = 0 \text{ for all } \mathbf{Z} \in \mathcal{V}_{\mathbf{S}}\}\\
    &= \left\{[\mathbf{S}_i\boldsymbol{\Omega}_i]_1^m: \boldsymbol{\Omega}_i \in \Skew(d) \text{ and }\Tr\left(\boldsymbol{\Omega}^T\left(\textstyle\sum_1^m\boldsymbol{\Omega}_i\right)\right) = 0 \text{ for all } \boldsymbol{\Omega}\in \Skew(d)\right\}.
\end{align}
% \begin{align}
%     \mathcal{H}_{\mathbf{S}} &= \mathcal{V}_{\mathbf{S}}^\perp = \{\mathbf{W} \in T_{\mathbf{S}}\mathbb{O}(d)^m: \Tr(\mathbf{Z}^T\mathbf{W}) = 0 \text{ for all } \mathbf{Z} \in \mathcal{V}_{\mathbf{S}}\}\\
%     &= \left\{[\mathbf{S}_i\boldsymbol{\Omega}_i]_1^m: \boldsymbol{\Omega}_i \in \Skew(d) \text{ and }\Tr\left(\boldsymbol{\Omega}^T\left(\textstyle\sum_1^m\boldsymbol{\Omega}_i\right)\right) = 0 \text{ for all } \boldsymbol{\Omega}\in \Skew(d)\right\}.
% \end{align}
The constraints on $\boldsymbol{\Omega}_i$ are equivalent to $\boldsymbol{\Omega}_i \in \Skew(d)$ and $ \textstyle\sum_1^m \boldsymbol{\Omega}_i \in \Sym(d)$, and subsequently to $\boldsymbol{\Omega}_i \in \Skew(d)$ and $\textstyle\sum_1^m \boldsymbol{\Omega}_i= 0$. Since $\mathcal{H}_{\mathbf{S}}$ is the orthogonal complement to $\mathcal{V}_{\mathbf{S}}$ in $T_{\mathbf{S}}\mathbb{O}(d)^m$, Eq.~(\ref{eq:P^h_S}) follows trivially.

\proofof{Proposition~\ref{prop:hlift_char}}
By Eq.~(\ref{eq:hlift_def}), we obtain, $\lim_{t \rightarrow 0}(\pi(\mathbf{S}+t\mathbf{Z})_i-\pi(\mathbf{S})_i)/t = \widetilde{\mathbf{Z}}_i$ for each $i \in [1,m-1]$ which further implies $\mathbf{S}_{i+1}\mathbf{Z}_1^T + \mathbf{Z}_{i+1}\mathbf{S}_1^T = \widetilde{\mathbf{Z}}_i$.
% \begin{align}
%     \lim_{t \rightarrow 0}(\pi(\mathbf{S}+t\mathbf{Z})_i-\pi(\mathbf{S})_i)/t = \widetilde{\mathbf{Z}}_i \implies \mathbf{S}_{i+1}\mathbf{Z}_1^T + \mathbf{Z}_{i+1}\mathbf{S}_1^T = \widetilde{\mathbf{Z}}_i. \label{supp:lifted_vec}
% \end{align}
Since $T_{\widetilde{\mathbf{S}}}\mathbb{O}(d)^m/_{\sim}$ is identified with $T_{\widetilde{\mathbf{S}}}\mathbb{O}(d)^{m-1}$, therefore there exist $(\widetilde{\boldsymbol{\Omega}}_i)_1^{m-1} \subseteq \Skew(d)$, such that $\widetilde{\mathbf{Z}}_i = \widetilde{\mathbf{S}}_i\widetilde{\boldsymbol{\Omega}}_i$. Also, since $\mathbf{Z} \in \mathcal{H}_{\mathbf{S}}$, there exist $(\boldsymbol{\Omega}_i)_1^m \subseteq \Skew(d)$ such that $\textstyle\sum_1^m \boldsymbol{\Omega}_i = 0$ and $\mathbf{Z}_i=\mathbf{S}_i\boldsymbol{\Omega}_i$. Substituting $\widetilde{\mathbf{Z}}_i = \widetilde{\mathbf{S}}_i\widetilde{\boldsymbol{\Omega}}_i$ and $\mathbf{Z}_i=\mathbf{S}_i\boldsymbol{\Omega}_i$,
%in the above equation
we obtain
\begin{equation}
    \mathbf{S}_{i+1}\boldsymbol{\Omega}_1^T\mathbf{S}_1^T + \mathbf{S}_{i+1}\boldsymbol{\Omega}_{i+1}\mathbf{S}_1^T = \widetilde{\mathbf{S}}_i\widetilde{\boldsymbol{\Omega}}_i \implies \boldsymbol{\Omega}_{i+1}-\boldsymbol{\Omega}_1 = \mathbf{S}_{1}^T\widetilde{\boldsymbol{\Omega}}_i\mathbf{S}_{1}, i \in [1,m-1], \label{supp:eq:eq1_}
\end{equation}
where we used the fact that $\widetilde{\mathbf{S}}_i = \mathbf{S}_{i+1}\mathbf{S}_1^T$. Observe that the linear system in $[\boldsymbol{\Omega}_i]_1^m$ is of full rank. By applying $\textstyle\sum_{i=1}^{m-1}$ and using $\textstyle\sum_1^m \mathbf{\Omega}_i = 0$ gives Eq.~(\ref{eq:hlift1}) and (\ref{eq:hlifti}).

\proofof{Proposition~\ref{prop:g_tilde}}
It suffices to show that $g(\mathbf{Z}, \mathbf{W})$ does not depend on the choice of $\mathbf{S} \in \pi^{-1}(\widetilde{\mathbf{S}})$. Let $\{\widetilde{\mathbf{U}}_i,\widetilde{\mathbf{V}}_i\}_1^{m-1},\{\mathbf{U}_i,\mathbf{V}_i\}_1^m$ be elements of $\Skew(d)$ such that $\widetilde{\mathbf{Z}}_i = \widetilde{\mathbf{S}}_i\widetilde{\mathbf{U}}_i$, $\widetilde{\mathbf{W}}_i = \widetilde{\mathbf{S}}_i\widetilde{\mathbf{V}}_i$, $\mathbf{Z}_i = \mathbf{S}_i\mathbf{U}_i$ and $\mathbf{W}_i = \mathbf{S}_i\mathbf{V}_i$. By the definition of $\mathcal{H}_{\mathbf{S}}$, $\textstyle\sum_1^m\mathbf{U}_i = \textstyle\sum_1^m \mathbf{V}_i = 0$ and the relation between $\mathbf{U}_i$ and $\widetilde{\mathbf{U}}_i$, and $\mathbf{V}_i$ and $\widetilde{\mathbf{V}}_i$ is given by Eq.~(\ref{eq:hlift1}, \ref{eq:hlifti}) (through Eq.~(\ref{supp:eq:eq1_})). Then we have
\begin{equation}
    g(\mathbf{Z}, \mathbf{W}) = \textstyle\sum_1^m \Tr(\mathbf{Z}_i^T\mathbf{W}_i) = \textstyle\sum_1^m \Tr(\mathbf{U}_i^T\mathbf{V}_i) = \textstyle\sum_1^m \Tr((\mathbf{U}_i-\mathbf{U}_1)^T\mathbf{V}_i) + \Tr(\mathbf{U}_1^T\mathbf{V}_i)
\end{equation}
Since $\textstyle\sum_1^m \mathbf{V}_i = 0$, the second term vanishes. The first reduces to 
\begin{align}
    &\textstyle\sum_1^m \Tr((\mathbf{U}_i-\mathbf{U}_1)^T\mathbf{V}_i) = \textstyle\sum_1^m \Tr((\mathbf{U}_i-\mathbf{U}_1)^T(\mathbf{V}_i-\mathbf{V}_1)) + \Tr((\mathbf{U}_i-\mathbf{U}_1)^T\mathbf{V}_1)\\
    &= \textstyle\sum_1^{m-1} \Tr((\mathbf{U}_{i+1}-\mathbf{U}_1)^T(\mathbf{V}_{i+1}-\mathbf{V}_1)) - \revadd{m}\Tr(\mathbf{U}_1^T\mathbf{V}_1)\\
    &=\textstyle\sum_1^{m-1}\Tr(\widetilde{\mathbf{U}}_{i}^T\widetilde{\mathbf{V}}_{i}) - m\Tr(\mathbf{U}_1^T\mathbf{V}_1)\\
    &= \textstyle\sum_1^{m-1}\Tr(\widetilde{\mathbf{U}}_{i}^T\widetilde{\mathbf{V}}_{i}) - \revdel{m^{-2}}\revadd{m^{-1}}\textstyle\sum_{i,j=1}^{m-1}\Tr(\widetilde{\mathbf{U}}_i^T\widetilde{\mathbf{V}}_j).
\end{align}
% \begin{align}
%     &g(\mathbf{Z}, \mathbf{W}) = \textstyle\sum_1^m \Tr(\mathbf{Z}_i^T\mathbf{W}_i) = \textstyle\sum_1^m \Tr(\mathbf{U}_i^T\mathbf{V}_i) = \textstyle\sum_1^m \Tr((\mathbf{U}_i-\mathbf{U}_1)^T\mathbf{V}_i) + \Tr(\mathbf{U}_1^T\mathbf{V}_i)\\
%     &= \textstyle\sum_1^m \Tr((\mathbf{U}_i-\mathbf{U}_1)^T\mathbf{V}_i) = \textstyle\sum_1^m \Tr((\mathbf{U}_i-\mathbf{U}_1)^T(\mathbf{V}_i-\mathbf{V}_1)) + \Tr((\mathbf{U}_i-\mathbf{U}_1)^T\mathbf{V}_1)\\
%     &= \textstyle\sum_1^{m-1} \Tr((\mathbf{U}_{i+1}-\mathbf{U}_1)^T(\mathbf{V}_{i+1}-\mathbf{V}_1)) - \Tr(\mathbf{U}_1^T\mathbf{V}_1)\\
%     &= \textstyle\sum_1^{m-1}\Tr(\widetilde{\mathbf{U}}_{i}^T\widetilde{\mathbf{V}}_{i}) - \Tr(\mathbf{U}_1^T\mathbf{V}_1) = \textstyle\sum_1^{m-1}\Tr(\widetilde{\mathbf{U}}_{i}^T\widetilde{\mathbf{V}}_{i}) - m^{-2}\textstyle\sum_{i,j=1}^{m-1}\Tr(\widetilde{\mathbf{U}}_i^T\widetilde{\mathbf{V}}_j).
% \end{align}
Since the above equation is independent of the choice of $\mathbf{S}$, the result follows.

\proofof{Proposition~\ref{prop:hlift_frob_ineq}} 
We have $\left\|\mathbf{Z}\right\|_F^2 = \sum_1^m \Tr(\mathbf{\Omega}_i\mathbf{\Omega}_i^T)$. Using Proposition~\ref{prop:hlift_char},
\begin{align}
    \left\|\widetilde{\mathbf{Z}}\right\|_F^2 &= \sum_1^{m-1} \Tr(\widetilde{\mathbf{\Omega}}_{i+1}\widetilde{\mathbf{\Omega}}_{i+1}^T) =\sum_1^{m-1} (\Tr(\mathbf{\Omega}_{i+1}\mathbf{\Omega}_{i+1}^T) -2\Tr(\mathbf{\Omega}_{i+1}\mathbf{\Omega}_1^T)) + (m-1)\Tr(\mathbf{\Omega}_1\mathbf{\Omega}_1^T)\\
    &=\sum_1^m \Tr(\mathbf{\Omega}_i\mathbf{\Omega}_i^T) + m \sum_1^m \Tr(\mathbf{\Omega}_1\mathbf{\Omega}_1^T) = \left\|\mathbf{Z}\right\|_F^2 + m\left\|\mathbf{\Omega}_1\right\|_F^2.
\end{align}

\proofof{Proposition~\ref{prop:gradFS}}
Using \citeb[Section 3.6.2]{absil2009optimization} and Proposition~\ref{prop:T_SOdm},
\begin{align}
    \overline{\grad \widetilde{F}(\widetilde{\mathbf{S}})} &= \grad F(\mathbf{S}) = \mathbf{P}_{\mathbf{S}}(\nabla F(\mathbf{S})) = \mathbf{P}_{\mathbf{S}}(2\mathbf{C}\mathbf{S}) =[2\mathbf{S}_i\text{skew}(\mathbf{S}_i^T[\mathbf{C}\mathbf{S}]_i])]_1^m\\
    &=[\mathbf{C}\mathbf{S}]_i - \mathbf{S}_i[\mathbf{C}\mathbf{S}]_i^T\mathbf{S}_i = \mathbf{S}_i (\mathbf{S}_i^T[\mathbf{C}\mathbf{S}]_i - [\mathbf{C}\mathbf{S}]_i^T\mathbf{S}_i)
\end{align}
% \begin{align}
%     \overline{\grad \widetilde{F}(\widetilde{\mathbf{S}})} &= \grad F(\mathbf{S}) = \mathbf{P}_{\mathbf{S}}(\nabla F(\mathbf{S})) = \mathbf{P}_{\mathbf{S}}(2\mathbf{C}\mathbf{S}) = [2\mathbf{S}_i\text{skew}(\mathbf{S}_i^T[\mathbf{C}\mathbf{S}]_i])]_1^m\\
%     &= [\mathbf{C}\mathbf{S}]_i - \mathbf{S}_i[\mathbf{C}\mathbf{S}]_i^T\mathbf{S}_i = \mathbf{S}_i (\mathbf{S}_i^T[\mathbf{C}\mathbf{S}]_i - [\mathbf{C}\mathbf{S}]_i^T\mathbf{S}_i).
% \end{align}
Using the fact that $\mathbf{C}$ is symmetric, we conclude that $\sum_{1}^{m} \boldsymbol{\Omega}_i = \mathbf{S}^T\mathbf{C}\mathbf{S}-\mathbf{S}^T\mathbf{C}^T\mathbf{S} = 0$.

\proofof{Proposition~\ref{prop:DgradFSZ}}
\begin{align}
    D\grad F(\mathbf{S})[\mathbf{Z}]_i &= \lim_{t \rightarrow 0}(\grad F(\mathbf{S}+t\mathbf{Z})_i-\grad F(\mathbf{S})_i)/t\\
    &= \lim_{t \rightarrow 0} t^{-1}\left\{([\mathbf{C}(\mathbf{S}+t\mathbf{Z})]_i - (\mathbf{S}_i+t\mathbf{Z}_i)[\mathbf{C}(\mathbf{S}+t\mathbf{Z})]_i^T(\mathbf{S}_i+t\mathbf{Z}_i)) - ([\mathbf{C}\mathbf{S}]_i - \mathbf{S}_i[\mathbf{C}\mathbf{S}]_i^T\mathbf{S}_i)\right\}\\
    &=[\mathbf{C}\mathbf{Z}]_i - \mathbf{S}_i[\mathbf{C}\mathbf{Z}]_i^T\mathbf{S}_i - \mathbf{S}_i[\mathbf{C}\mathbf{S}]_i^T\mathbf{Z}_i -\mathbf{Z}_i[\mathbf{C}\mathbf{S}]_i^T\mathbf{S}_i\\
    &=\mathbf{S}_i(\mathbf{S}_i^T[\mathbf{C}\mathbf{Z}]_i - [\mathbf{C}\mathbf{Z}]_i^T\mathbf{S}_i - [\mathbf{C}\mathbf{S}]_i^T\mathbf{Z}_i - \mathbf{S}_i^T\mathbf{Z}_i[\mathbf{C}\mathbf{S}]_i^T\mathbf{S}_i)
\end{align}
% \revdel{Note that we have not yet used the facts that $\widetilde{\mathbf{S}} \in \widetilde{\mathcal{C}}$ or equivalently, $[\mathbf{C}\mathbf{S}]_i = \mathbf{S}_i[\mathbf{C}\mathbf{S}]_i^T\mathbf{S}_i$ (see the proof of Proposition~\ref{prop:gradFS}) and, $\mathbf{Z} \in T_{\mathbf{S}}\mathbb{O}(d)^m$ or equivalently, $\mathbf{S}_i^T\mathbf{Z}_i + \mathbf{Z}_i^T\mathbf{S}_i = 0$ (see Proposition~\ref{prop:T_SOdm}). Combining the two, we get $\mathbf{S}_i^T\mathbf{Z}_i[\mathbf{C}\mathbf{S}]_i^T\mathbf{S}_i = -\mathbf{Z}_i^T[\mathbf{C}\mathbf{S}]_i$ and the result follows.}
\revadd{The result follows from the facts that $\mathbf{S}_i^T\mathbf{Z}_i + \mathbf{Z}_i^T\mathbf{S}_i = 0$ for $\mathbf{Z} \in T_{\mathbf{S}}\mathbb{O}(d)^m$ (see Proposition~\ref{prop:T_SOdm}), and $[\mathbf{C}\mathbf{S}]_i = \mathbf{S}_i[\mathbf{C}\mathbf{S}]_i^T\mathbf{S}_i$ for $\widetilde{\mathbf{S}} \in \widetilde{\mathcal{C}}$ (see the proof of Proposition~\ref{prop:gradFS})}.

\proofof{Proposition~\ref{prop:HessFSZ}}
Using \citeb[Chapter 5]{absil2009optimization},
$$\overline{\Hess \widetilde{F}(\widetilde{\mathbf{S}})[\widetilde{\mathbf{Z}}]} = \overline{\widetilde{\nabla}_{\widetilde{\mathbf{Z}}}\grad \widetilde{F}(\widetilde{\mathbf{S}})} = P^{h}_{\mathbf{S}}(\nabla_{\overline{\widetilde{\mathbf{Z}}}}\overline{\grad \widetilde{F}(\widetilde{\mathbf{S}})}) = P^{h}_{\mathbf{S}}(\nabla_{\mathbf{Z}}\grad F(\mathbf{S})).$$ Then from, Proposition~\ref{prop:T_SOdm}, \ref{prop:V_S_H_S} and \ref{prop:DgradFSZ}, the latter is reduced to
\begin{align}
    P^{h}_{\mathbf{S}}(\nabla_{\mathbf{Z}}\grad F(\mathbf{S})) &= P^{h}_{\mathbf{S}}(P_{\mathbf{S}}(D\grad F(\mathbf{S})[\mathbf{Z}])) = P^{h}_{\mathbf{S}}\left(\left[\mathbf{S}_i(\Skew(\boldsymbol{\xi}_i))\right]_1^m\right)\\
    &= [\mathbf{S}_i(\Skew(\boldsymbol{\xi}_i)-m^{-1}\sum_1^m\Skew(\boldsymbol{\xi}_i))]_1^m
\end{align}
For the case $\widetilde{\mathbf{S}} \in \widetilde{\mathcal{C}}$, we note that $\boldsymbol{\xi}_i = \Skew(\boldsymbol{\xi}_i) = \mathbf{S}_i^T[\mathbf{C}\mathbf{Z}]_i - [\mathbf{C}\mathbf{Z}]_i^T\mathbf{S}_i - [\mathbf{C}\mathbf{S}]_i^T\mathbf{Z}_i + \mathbf{Z}_i^T[\mathbf{C}\mathbf{S}]_i$ and $\sum_{1}^{m}\boldsymbol{\xi}_i = \mathbf{S}^T\mathbf{C}\mathbf{Z}-\mathbf{Z}^T\mathbf{C}\mathbf{S}-\mathbf{S}^T\mathbf{C}\mathbf{Z}+\mathbf{Z}^T\mathbf{C}\mathbf{S} = 0$.
% \revdel{which equals $P^{h}_{\mathbf{S}}\left(\left[\mathbf{S}_i(\mathbf{S}_i^T[\mathbf{C}\mathbf{Z}]_i - [\mathbf{C}\mathbf{Z}]_i^T\mathbf{S}_i - [\mathbf{C}\mathbf{S}]_i^T\mathbf{Z}_i + \mathbf{Z}_i^T[\mathbf{C}\mathbf{S}]_i)\right]_1^m\right)
%     = P^h_{\mathbf{S}}([\mathbf{S}_i\widehat{\boldsymbol{\Omega}}_i]_1^m) = [\mathbf{S}_i(\widehat{\boldsymbol{\Omega}}_i-\overline{\widehat{\boldsymbol{\Omega}}})]_1^m$.
% % \begin{align}
% %     \overline{\Hess \widetilde{F}(\widetilde{\mathbf{S}})[\widetilde{\mathbf{Z}}]} &= \overline{\widetilde{\nabla}_{\widetilde{\mathbf{Z}}}\grad \widetilde{F}(\widetilde{\mathbf{S}})} = P^{h}_{\mathbf{S}}(\nabla_{\overline{\widetilde{\mathbf{Z}}}}\overline{\grad \widetilde{F}(\widetilde{\mathbf{S}})}) = P^{h}_{\mathbf{S}}(\nabla_{\mathbf{Z}}\grad F(\mathbf{S}))\\
% %     &= P^{h}_{\mathbf{S}}(P_{\mathbf{S}}(D\grad F(\mathbf{S})[\mathbf{Z}]))\\
% %     &= P^{h}_{\mathbf{S}}\left(\left[\mathbf{S}_i(\mathbf{S}_i^T[\mathbf{C}\mathbf{Z}]_i - [\mathbf{C}\mathbf{Z}]_i^T\mathbf{S}_i - [\mathbf{C}\mathbf{S}]_i^T\mathbf{Z}_i + \mathbf{Z}_i^T[\mathbf{C}\mathbf{S}]_i)\right]_1^m\right)\\
% %     &= P^h_{\mathbf{S}}([\mathbf{S}_i\widehat{\boldsymbol{\Omega}}_i]_1^m) = [\mathbf{S}_i(\widehat{\boldsymbol{\Omega}}_i-\overline{\widehat{\boldsymbol{\Omega}}})]_1^m
% % \end{align}
% where $\overline{\widehat{\boldsymbol{\Omega}}} = m^{-1}\textstyle\sum_1^m\widehat{\boldsymbol{\Omega}}_i = m^{-1}\textstyle\sum_1^m (\mathbf{S}_i^T[\mathbf{C}\mathbf{Z}]_i - [\mathbf{C}\mathbf{Z}]_i^T\mathbf{S}_i - [\mathbf{C}\mathbf{S}]_i^T\mathbf{Z}_i + \mathbf{Z}_i^T[\mathbf{C}\mathbf{S}]_i)$. Thus, $\overline{\widehat{\boldsymbol{\Omega}}} = m^{-1}(\mathbf{S}^T\mathbf{C}\mathbf{Z}-\mathbf{Z}^T\mathbf{C}\mathbf{S}-\mathbf{S}^T\mathbf{C}\mathbf{Z}+\mathbf{Z}^T\mathbf{C}\mathbf{S}) = 0$.
% % \begin{align}
% %     \overline{\widehat{\boldsymbol{\Omega}}} &= m^{-1}\textstyle\sum_1^m\widehat{\boldsymbol{\Omega}}_i = m^{-1}\textstyle\sum_1^m (\mathbf{S}_i^T[\mathbf{C}\mathbf{Z}]_i - [\mathbf{C}\mathbf{Z}]_i^T\mathbf{S}_i - [\mathbf{C}\mathbf{S}]_i^T\mathbf{Z}_i + \mathbf{Z}_i^T[\mathbf{C}\mathbf{S}]_i)\\
% %     &= m^{-1}(\mathbf{S}^T\mathbf{C}\mathbf{Z}-\mathbf{Z}^T\mathbf{C}\mathbf{S}-\mathbf{S}^T\mathbf{C}\mathbf{Z}+\mathbf{Z}^T\mathbf{C}\mathbf{S}) = 0.
% % \end{align}
% This further validates that $\overline{\Hess \widetilde{F}(\widetilde{\mathbf{S}})[\widetilde{\mathbf{Z}}]} = [\mathbf{S}_i\widehat{\boldsymbol{\Omega}}_i]_1^m$ indeed lies in $\mathcal{H}_{\mathbf{S}}$ (see Proposition~\ref{prop:V_S_H_S}).
% }

\proofof{Proposition~\ref{prop:Omega_hat_compact}}
Since $\mathbf{Z} \in \mathcal{H}_{\mathbf{S}}$, using Proposition~\ref{prop:V_S_H_S}, there exist $\boldsymbol{\Omega} = [\boldsymbol{\Omega}_i]_1^m$ such that $\boldsymbol{\Omega}_i \in \Skew(d)$, $\textstyle\sum_1^m\boldsymbol{\Omega}_i = 0$ and $\mathbf{Z}_i = \mathbf{S}_i\boldsymbol{\Omega}_i$. Then,
\revadd{$\boldsymbol{\xi}_i = [\mathbf{C}(\mathbf{S})\boldsymbol{\Omega}]_i - [\mathbf{C}(\mathbf{S})\boldsymbol{\Omega}]_i^T - [\widehat{\mathbf{C}}(\mathbf{S})^T\boldsymbol{\Omega}]_i + [\widehat{\mathbf{C}}(\mathbf{S})\boldsymbol{\Omega}]_i^T$. The result follows from (i) $\mathbf{L}(\mathbf{S}) = \mathbf{C}(\mathbf{S}) - \widehat{\mathbf{C}}(\mathbf{S})$, (ii) $\mathbf{C}(\mathbf{S})$ is symmetric and (iii) for $\widetilde{\mathbf{S}}\in \widetilde{\mathcal{C}}$, $\widehat{\mathbf{C}}(\mathbf{S})$ is also symmetric (Remark~\ref{rmk:StildeCtilde}).}
% \revdel{$\widehat{\boldsymbol{\Omega}}_i$, which equals $(\mathbf{S}_i^T[\mathbf{C}\mathbf{Z}]_i - [\mathbf{C}\mathbf{Z}]_i^T\mathbf{S}_i) - ([\mathbf{C}\mathbf{S}]_i^T\mathbf{Z}_i - \mathbf{Z}_i^T[\mathbf{C}\mathbf{S}]_i)$, reduces to $([\mathbf{C}(\mathbf{S})\boldsymbol{\Omega}]_i - [\mathbf{C}(\mathbf{S})\boldsymbol{\Omega}]_i^T) - ( [\widehat{\mathbf{C}}(\mathbf{S})\boldsymbol{\Omega}]_i - [\widehat{\mathbf{C}}(\mathbf{S})\boldsymbol{\Omega}]_i^T)$. The latter is simply $[\mathbf{L}(\mathbf{S})\boldsymbol{\Omega}]_i^T - [\mathbf{L}(\mathbf{S})\boldsymbol{\Omega}]_i$.
% }
% \begin{align}
%     \widehat{\boldsymbol{\Omega}}_i &= (\mathbf{S}_i^T[\mathbf{C}\mathbf{Z}]_i - [\mathbf{C}\mathbf{Z}]_i^T\mathbf{S}_i) - ([\mathbf{C}\mathbf{S}]_i^T\mathbf{Z}_i - \mathbf{Z}_i^T[\mathbf{C}\mathbf{S}]_i)\\
%     &= ([\mathbf{C}(\mathbf{S})\boldsymbol{\Omega}]_i - [\mathbf{C}(\mathbf{S})\boldsymbol{\Omega}]_i^T) - ( [\widehat{\mathbf{C}}(\mathbf{S})\boldsymbol{\Omega}]_i - [\widehat{\mathbf{C}}(\mathbf{S})\boldsymbol{\Omega}]_i^T) = [\mathbf{L}(\mathbf{S})\boldsymbol{\Omega}]_i^T - [\mathbf{L}(\mathbf{S})\boldsymbol{\Omega}]_i.
% \end{align}

\proofof{Proposition~\ref{prop:HessFSZZ}}
We obtain $\widetilde{g}(\Hess \widetilde{F}(\widetilde{\mathbf{S}})[\widetilde{\mathbf{Z}}],\widetilde{\mathbf{Z}}) = g(\overline{\Hess \widetilde{F}(\widetilde{\mathbf{S}})[\widetilde{\mathbf{Z}}]},\overline{\widetilde{\mathbf{Z}}})$ from Proposition~\ref{prop:g_tilde}. Due to Proposition \ref{prop:HessFSZ},
$$g(\overline{\Hess \widetilde{F}(\widetilde{\mathbf{S}})[\widetilde{\mathbf{Z}}]},\overline{\widetilde{\mathbf{Z}}}) = g(\overline{\Hess \widetilde{F}(\widetilde{\mathbf{S}})[\widetilde{\mathbf{Z}}]},\mathbf{Z}) = \textstyle\sum_{1}^{m}\Tr((\mathbf{S}_i\widehat{\boldsymbol{\Omega}}_i)^T\mathbf{S}_i\boldsymbol{\Omega}_i) = \textstyle\sum_{1}^{m}\Tr(\widehat{\boldsymbol{\Omega}}_i^T\boldsymbol{\Omega}_i).$$ \revadd{The result follows from the Proposition~\ref{prop:Omega_hat_compact} and the fact that $\Tr(\mathbf{A}) = \Tr(\mathbf{A}^T)$.}
% \revdel{This, using Proposition~\ref{prop:Omega_hat_compact}, simplifies to $\textstyle\sum_{1}^{m}\Tr([\mathbf{L}(\mathbf{S})\boldsymbol{\Omega}]_i\boldsymbol{\Omega}_i - [\mathbf{L}(\mathbf{S})\boldsymbol{\Omega}]_i^T\boldsymbol{\Omega}_i) = -2\Tr(\boldsymbol{\Omega}^T\mathbf{L}(\mathbf{S})\boldsymbol{\Omega})$.}
% \begin{align}
%     &g(\overline{\Hess \widetilde{F}(\widetilde{\mathbf{S}})[\widetilde{\mathbf{Z}}]},\mathbf{Z}) = \textstyle\sum_{1}^{m}\Tr((\mathbf{S}_i\widehat{\boldsymbol{\Omega}}_i)^T\mathbf{S}_i\boldsymbol{\Omega}_i) = \textstyle\sum_{1}^{m}\Tr(\widehat{\boldsymbol{\Omega}}_i^T\boldsymbol{\Omega}_i)\\
%     &= \textstyle\sum_{1}^{m}\Tr([\mathbf{L}(\mathbf{S})\boldsymbol{\Omega}]_i\boldsymbol{\Omega}_i - [\mathbf{L}(\mathbf{S})\boldsymbol{\Omega}]_i^T\boldsymbol{\Omega}_i) = -2\Tr(\boldsymbol{\Omega}^T\mathbf{L}(\mathbf{S})\boldsymbol{\Omega})
% \end{align}

\proofof{Proposition~\ref{prop:Omega^TLSOmega2}} Using $\vecz (\mathbf{A}\mathbf{X}\mathbf{B}) = (\mathbf{B}^T \otimes \mathbf{A})\vecz (\mathbf{X})$, we obtain
\begin{align}
\Tr(\boldsymbol{\Omega}^T&\mathbf{L}(\mathbf{S})\boldsymbol{\Omega}) = \Tr((\mathbf{P}\boldsymbol{\Omega})^T\mathbf{P}\mathbf{L}(\mathbf{S})\mathbf{P}^T(\mathbf{P}\boldsymbol{\Omega})) = \vecz (\mathbf{P}\boldsymbol{\Omega})^T\vecz (\mathcal{L}(\mathbf{S}) (\mathbf{P}\boldsymbol{\Omega}))\\
&= \vecz (\mathbf{P}\boldsymbol{\Omega})^T(\mathbf{I}_d \otimes \mathcal{L}(\mathbf{S}))\vecz (\mathbf{P}\boldsymbol{\Omega}) = \boldsymbol{\omega}^T\overline{\mathbf{P}}(\mathbf{I}_d \otimes \mathcal{L}(\mathbf{S}))\overline{\mathbf{P}}^T\boldsymbol{\omega} = \boldsymbol{\omega}^T\mathbb{L}(\mathbf{S})\boldsymbol{\omega}.
\end{align}
% \begin{align}
%     &\Tr(\boldsymbol{\Omega}^T\mathbf{L}(\mathbf{S})\boldsymbol{\Omega}) = \Tr((\mathbf{P}\boldsymbol{\Omega})^T\mathbf{P}\mathbf{L}(\mathbf{S})\mathbf{P}^T(\mathbf{P}\boldsymbol{\Omega}))= \vecz (\mathbf{P}\boldsymbol{\Omega})^T\vecz (\mathcal{L}(\mathbf{S}) (\mathbf{P}\boldsymbol{\Omega}))\\
%     &= \vecz (\mathbf{P}\boldsymbol{\Omega})^T(\mathbf{I}_d \otimes \mathcal{L}(\mathbf{S}))\vecz (\mathbf{P}\boldsymbol{\Omega})= \boldsymbol{\omega}^T\overline{\mathbf{P}}(\mathbf{I}_d \otimes \mathcal{L}(\mathbf{S}))\overline{\mathbf{P}}^T\boldsymbol{\omega}= \boldsymbol{\omega}^T\mathbb{L}(\mathbf{S})\boldsymbol{\omega}.
% \end{align}

\proofof{Theorem~\ref{thm:non_deg_loc_min}}
$1$, $2$ and $3$ are equivalent by definitions and Proposition~\ref{prop:HessFSZZ}. For ($3 \iff 4$), by comparing dimensions, we note that the set of $\boldsymbol{\Omega} = [\boldsymbol{\Omega}_i]_1^m$ where $\boldsymbol{\Omega}_i \in \Skew(d)$ and $\textstyle\sum_1^m \boldsymbol{\Omega}_i = 0$, is the same as the set of $\boldsymbol{\Omega} = [\boldsymbol{\Omega}_i-\boldsymbol{\Omega}_0]_1^m$ where $\boldsymbol{\Omega}_i \in \Skew(d)$ and $\boldsymbol{\Omega}_0 = \frac{1}{m}\textstyle\sum_1^m\boldsymbol{\Omega}_i$. Using Remark~\ref{rmk:C_hat_L_structure}, we know that $\mathbf{L}(\mathbf{S})=\mathbf{L}(\mathbf{S})^T$ and $\mathbf{L}(\mathbf{S})[\boldsymbol{\Omega}_0]_1^m = 0$. Thus $(\boldsymbol{\Omega}-[\boldsymbol{\Omega}_0]_1^m)^T\mathbf{L}(\mathbf{S})(\boldsymbol{\Omega}-[\boldsymbol{\Omega}_0]_1^m) = \boldsymbol{\Omega}^T\mathbf{L}(\mathbf{S})\boldsymbol{\Omega}$ and the result follows. Subsequently, ($4 \iff 5$) follows directly from the definition of $\boldsymbol{\omega}$ and Proposition~\ref{prop:Omega^TLSOmega2}. Finally, ($5 \iff 6$) and ($6 \iff 7$) follow from Remark~\ref{rmk:mathbb_L_structure}.

\proofof{Proposition~\ref{prop:one_all1}} Suppose $\mathbf{L}(\mathbf{S})$ satisfies condition 4 and $\boldsymbol{\Omega}$ be as in the condition. Then using Remark~\ref{rmk:C_hat_L_structure}, 
% \begin{align}
%     \Tr(\boldsymbol{\Omega}^T\mathbf{L}(\mathbf{S}\mathbf{\mathbf{Q}})\boldsymbol{\Omega}) &= \Tr(\boldsymbol{\Omega}^T(\mathbf{I}_m \otimes \mathbf{Q})^T\mathbf{L}(\mathbf{S})(\mathbf{I}_m \otimes \mathbf{Q})\boldsymbol{\Omega}) = \Tr(\mathbf{Q}\boldsymbol{\Omega}^T(\mathbf{I}_m \otimes \mathbf{Q})^T\mathbf{L}(\mathbf{S})(\mathbf{I}_m \otimes \mathbf{Q})\boldsymbol{\Omega} \mathbf{Q}^T)\\
%     &= \Tr(\overline{\boldsymbol{\Omega}}^T \mathbf{L}(\mathbf{S})\overline{\boldsymbol{\Omega}})
% \end{align}
\begin{align}
    \Tr(\boldsymbol{\Omega}^T\mathbf{L}(\mathbf{S}\mathbf{\mathbf{Q}})\boldsymbol{\Omega}) &= \Tr(\boldsymbol{\Omega}^T(\mathbf{I}_m \otimes \mathbf{Q})^T\mathbf{L}(\mathbf{S})(\mathbf{I}_m \otimes \mathbf{Q})\boldsymbol{\Omega}) = \Tr(\overline{\boldsymbol{\Omega}}^T \mathbf{L}(\mathbf{S})\overline{\boldsymbol{\Omega}})
\end{align}
which is positive because $\overline{\boldsymbol{\Omega}}_i = \mathbf{Q}\boldsymbol{\Omega}_i\mathbf{Q}^T \in \Skew(d)$ for all $i \in [1,m]$ and not all $\overline{\boldsymbol{\Omega}}_i$ are equal (if $\overline{\boldsymbol{\Omega}}_i = \overline{\boldsymbol{\Omega}}_j$ then $\boldsymbol{\Omega}_i = \boldsymbol{\Omega}_j$, a contradiction). Thus, condition 4 holds for~$\mathbf{S}\mathbf{Q}$ too. The result follows.
% \revdel{It suffices to pick conditions 4 and 6 (one involving $\mathbf{L}(\mathbf{S})$ and one involving $\mathbb{L}(\mathbf{S})$). Suppose $\mathbf{L}(\mathbf{S})$ satisfies condition 4 and let $\boldsymbol{\Omega}$ be as in the condition. Then using Remark~\ref{rmk:C_hat_L_structure}, $\Tr(\boldsymbol{\Omega}^T\mathbf{L}(\mathbf{S}\mathbf{\mathbf{Q}})\boldsymbol{\Omega}) = \Tr(\boldsymbol{\Omega}^T(\mathbf{I}_m \otimes \mathbf{Q})^T\mathbf{L}(\mathbf{S})(\mathbf{I}_m \otimes \mathbf{Q})\boldsymbol{\Omega})$. The latter simplifies to $\Tr(\mathbf{Q}\boldsymbol{\Omega}^T(\mathbf{I}_m \otimes \mathbf{Q})^T\mathbf{L}(\mathbf{S})(\mathbf{I}_m \otimes \mathbf{Q})\boldsymbol{\Omega} \mathbf{Q}^T) = \Tr(\overline{\boldsymbol{\Omega}}^T \mathbf{L}(\mathbf{S})\overline{\boldsymbol{\Omega}})$ which is negative because $\overline{\boldsymbol{\Omega}}_i = \mathbf{Q}\boldsymbol{\Omega}_i\mathbf{Q}^T \in \Skew(d)$ for all $i \in [1,m]$ and not all $\overline{\boldsymbol{\Omega}}_i$ are equal (if $\overline{\boldsymbol{\Omega}}_i = \overline{\boldsymbol{\Omega}}_j$ then $\boldsymbol{\Omega}_i = \boldsymbol{\Omega}_j$, a contradiction). Thus the condition 4 holds for $\mathbf{S}\mathbf{Q}$. Now suppose $\mathbf{S}$ satisfies condition 6 then from Remark~\ref{rmk:mathbb_L_structure}, $\mathbf{S}\mathbf{Q}$ satisfies it too.} 

\proofof{Corollary~\ref{cor:suff_non_deg_loc_min}}
Since the rank of $\mathbf{L}(\mathbf{S})$ is $(m-1)d$, using Remark~\ref{rmk:C_hat_L_structure}, $\mathbf{L}(\mathbf{S})[\boldsymbol{\Omega}_i]_1^m = 0$ (equivalently, $\Tr(([\boldsymbol{\Omega}_i]_1^m)^T\mathbf{L}(\mathbf{S})[\boldsymbol{\Omega}_i]_1^m) = 0$) iff $\boldsymbol{\Omega}_i = \boldsymbol{\Omega}_0$ for all $i \in [1,m]$ and some $\boldsymbol{\Omega}_0 \in \Skew(d)$. Thus, condition~$4$ in Theorem~\ref{thm:non_deg_loc_min} is satisfied. 


\proofof{Proposition~\ref{prop:HessVicinity}} For brevity, we define $\mathbf{D}_{\mathbf{S}} \coloneqq \blockdiag([\mathbf{S}_i]_1^m)$ and $\mathbf{Z}_i \coloneqq \mathbf{O}_i\boldsymbol{\Omega}_i$, i $\in [1,m]$, which satisfies $\left\|\mathbf{Z}_i\right\|_F^2 = \left\|\boldsymbol{\Omega}_i\right\|_F^2$. Then,
\begin{align}
    \Tr(\boldsymbol{\Omega}^T(\mathbf{L}(\mathbf{O})+\mathbf{L}(\mathbf{O})^T)\boldsymbol{\Omega}) &= 2\Tr(\boldsymbol{\Omega}^T\mathbf{C}(\mathbf{O})\boldsymbol{\Omega}) - \Tr(\boldsymbol{\Omega}^T(\widehat{\mathbf{C}}(\mathbf{O})+\widehat{\mathbf{C}}(\mathbf{O})^T)\boldsymbol{\Omega})\\
    &= 2\Tr(\boldsymbol{\Omega}^T\mathbf{C}(\mathbf{O})\boldsymbol{\Omega}) - \textstyle\sum_{i=1}^{m}\Tr\left(\boldsymbol{\Omega}_i^T \left(\textstyle\sum_{j=1}^{m}\mathbf{O}_i^T\mathbf{C}_{ij}\mathbf{O}_j + \mathbf{O}_j^T\mathbf{C}_{ji}\mathbf{O}_i\right)\boldsymbol{\Omega}_i\right)\\
    &= 2\Tr(\boldsymbol{\Omega}^T\mathbf{C}(\mathbf{O})\boldsymbol{\Omega}) - \textstyle\sum_{i=1}^{m}\Tr\left(\mathbf{Z}_i^T \left(\textstyle\sum_{j=1}^{m}\mathbf{C}_{ij}\mathbf{O}_j\mathbf{O}_i^T + \mathbf{O}_i\mathbf{O}_j^T\mathbf{C}_{ji}\right)\mathbf{Z}_i\right)\label{eq:HessO1}
\end{align}
Rewriting the first term,
\begin{align}
\textstyle\sum_{i=1}^{m}\Tr&\left(\mathbf{Z}_i^T\textstyle\sum_{j=1}^{m}\mathbf{C}_{ij}\mathbf{O}_j\mathbf{O}_i^T\mathbf{Z}_i\right) = \textstyle\sum_{i=1}^{m}\Tr\left(\mathbf{Z}_i^T\textstyle\sum_{j=1}^{m}\mathbf{C}_{ij}(\mathbf{O}_j\mathbf{O}_i^T-\mathbf{S}_j\mathbf{S}_i^T + \mathbf{S}_j\mathbf{S}_i^T)\mathbf{Z}_i\right)\\
&= \textstyle\sum_{i=1}^{m}\Tr\left(\mathbf{Z}_i^T\textstyle\sum_{j=1}^{m}\mathbf{C}_{ij}(\mathbf{O}_j\mathbf{O}_i^T-\mathbf{S}_j\mathbf{S}_i^T)\mathbf{Z}_i\right) + \Tr\left(\mathbf{Z}_i^T\mathbf{S}_i\mathbf{S}_i^T\textstyle\sum_{j=1}^{m}\mathbf{C}_{ij} \mathbf{S}_j\mathbf{S}_i^T\mathbf{Z}_i\right)\\
&= \textstyle\sum_{i=1}^{m} \Tr\left(\mathbf{Z}_i^T\textstyle\sum_{j=1}^{m}\mathbf{C}_{ij}(\mathbf{O}_j\mathbf{O}_i^T-\mathbf{S}_j\mathbf{S}_i^T)\mathbf{Z}_i\right) + \Tr\left(\mathbf{Z}^T\mathbf{D}_{\mathbf{S}}\widehat{\mathbf{C}}(\mathbf{S})\mathbf{D}_{\mathbf{S}}^T\mathbf{Z}\right) \label{eq:HessO2}
\end{align}
Using Cauchy-Schwarz inequality and the fact that $\left\|\mathbf{A}_1\mathbf{A}_2\right\|_F \leq \left\|\mathbf{A}_1\right\|_2\left\|\mathbf{A}_2\right\|_F$,
\begin{align}
\left|\textstyle\sum_{i=1}^{m} \Tr\right.&\left.\left(\mathbf{Z}_i^T\textstyle\sum_{j=1}^{m}\mathbf{C}_{ij}(\mathbf{O}_j\mathbf{O}_i^T-\mathbf{S}_j\mathbf{S}_i^T)\mathbf{Z}_i\right)\right| \leq \textstyle\sum_{i=1}^{m} \left\|\textstyle\sum_{j=1}^{m}\mathbf{C}_{ij}(\mathbf{O}_j\mathbf{O}_i^T-\mathbf{S}_j\mathbf{S}_i^T)\right\|_F\left\|\mathbf{Z}_i\right\|_F^2\\
&\leq \textstyle\sum_{i=1}^{m} \left\|\textstyle\sum_{j=1}^{m}\mathbf{C}_{ij}(\mathbf{O}_j\mathbf{O}_i^T-\mathbf{S}_j\mathbf{O}_i^T + \mathbf{S}_j\mathbf{O}_i^T-\mathbf{S}_j\mathbf{S}_i^T)\right\|_F\left\|\mathbf{Z}_i\right\|_F^2\\
&\leq  \textstyle\sum_{i=1}^{m}(\max_{k=1}^{m}\left\|\mathbf{C}_{k,:}\right\|_2 \left\|\mathbf{O}-\mathbf{S}\right\|_F + \left\|[\mathbf{C}\mathbf{S}]_i\right\|_2\left\|\mathbf{O}_i-\mathbf{S}_i\right\|_F)\left\|\boldsymbol{\Omega}_i\right\|_F^2\\
&\leq   \textstyle\sum_{i=1}^{m}(c_1 \left\|\mathbf{O}-\mathbf{S}\right\|_F + c_2(\mathbf{S})\left\|\mathbf{O}_i-\mathbf{S}_i\right\|_F)\left\|\boldsymbol{\Omega}_i\right\|_F^2\\
&\leq  (c_1 + c_2(\mathbf{S}))\left\|\mathbf{O}-\mathbf{S}\right\|_F\left\|\boldsymbol{\Omega}\right\|_F^2 \label{eq:HessO3}
\end{align}
Also, due to Eq.~(\ref{eq:C_of_S}),
$$\Tr(\boldsymbol{\Omega}^T\mathbf{C}(\mathbf{O})\boldsymbol{\Omega}) = \Tr(\boldsymbol{\Omega}^T\mathbf{D}_{\mathbf{O}}^T\mathbf{D}_{\mathbf{S}}\mathbf{C}(\mathbf{S})\mathbf{D}_{\mathbf{S}}^T\mathbf{D}_{\mathbf{O}}\boldsymbol{\Omega}) = \Tr(\mathbf{Z}^T\mathbf{D}_{\mathbf{S}}\mathbf{C}(\mathbf{S})\mathbf{D}_{\mathbf{S}}^T\mathbf{Z}).$$
Combining this with Eq.~(\ref{eq:HessO1}, \ref{eq:HessO2}, \ref{eq:HessO3}), we obtain
\begin{align}
    \Tr(\boldsymbol{\Omega}^T(\mathbf{L}(\mathbf{O})+\mathbf{L}(\mathbf{O})^T)\boldsymbol{\Omega}) &\geq 2\Tr(\mathbf{Z}^T\mathbf{D}_{\mathbf{S}}\mathbf{L}(\mathbf{S})\mathbf{D}_{\mathbf{S}}^T\mathbf{Z}) - 2(c_1 + c_2(\mathbf{S}))\left\|\mathbf{O}-\mathbf{S}\right\|_F\left\|\boldsymbol{\Omega}\right\|_F^2\\
    \Tr(\boldsymbol{\Omega}^T(\mathbf{L}(\mathbf{O})+\mathbf{L}(\mathbf{O})^T)\boldsymbol{\Omega}) &\leq 2\Tr(\mathbf{Z}^T\mathbf{D}_{\mathbf{S}}\mathbf{L}(\mathbf{S})\mathbf{D}_{\mathbf{S}}^T\mathbf{Z}) + 2(c_1 + c_2(\mathbf{S}))\left\|\mathbf{O}-\mathbf{S}\right\|_F\left\|\boldsymbol{\Omega}\right\|_F^2.
\end{align}
Moreover,
\begin{align}
\Tr(\mathbf{Z}^T\mathbf{D}_{\mathbf{S}}\mathbf{L}(\mathbf{S})\mathbf{D}_{\mathbf{S}}^T\mathbf{Z}) &= \Tr(\boldsymbol{\Omega}^T\mathbf{D}_{\mathbf{O}}^T\mathbf{D}_{\mathbf{S}}\mathbf{L}(\mathbf{S})\mathbf{D}_{\mathbf{S}}^T\mathbf{D}_{\mathbf{O}}\boldsymbol{\Omega})\\
&= \Tr(\boldsymbol{\Omega}^T\mathbf{L}(\mathbf{S})\boldsymbol{\Omega}) + 2\Tr(\boldsymbol{\Omega}^T(\mathbf{D}_{\mathbf{O}}^T\mathbf{D}_{\mathbf{S}}-\mathbf{I}_{md})\mathbf{L}(\mathbf{S})\mathbf{D}_{\mathbf{S}}^T\mathbf{D}_{\mathbf{O}}\boldsymbol{\Omega}),
\end{align}
% \begin{align}
% \Tr(\mathbf{Z}^T\mathbf{D}_{\mathbf{S}}\mathbf{L}(\mathbf{S})\mathbf{D}_{\mathbf{S}}^T\mathbf{Z}) &= \Tr(\boldsymbol{\Omega}^T\mathbf{D}_{\mathbf{O}}^T\mathbf{D}_{\mathbf{S}}\mathbf{L}(\mathbf{S})\mathbf{D}_{\mathbf{S}}^T\mathbf{D}_{\mathbf{O}}\boldsymbol{\Omega})\\
% &= \Tr(\boldsymbol{\Omega}^T\mathbf{L}(\mathbf{S})\boldsymbol{\Omega}) + 2\Tr(\boldsymbol{\Omega}^T(\mathbf{D}_{\mathbf{O}}^T\mathbf{D}_{\mathbf{S}}-\mathbf{I}_{md})\mathbf{L}(\mathbf{S})\mathbf{D}_{\mathbf{S}}^T\mathbf{D}_{\mathbf{O}}\boldsymbol{\Omega}),
% \end{align}
where, for the first term,
\begin{equation}
    (\lambda_{-}(\mathbf{S})/2) \left\|\boldsymbol{\Omega}\right\|_F^2 \leq \Tr(\boldsymbol{\Omega}^T\mathbf{L}(\mathbf{S})\boldsymbol{\Omega}) =  \Tr(\boldsymbol{\omega}^T\mathbf{\mathbb{L}}(\mathbf{S})\boldsymbol{\omega}) \leq  (\lambda_{+}(\mathbf{S})/2) \left\|\boldsymbol{\Omega}\right\|_F^2.
\end{equation}
The fraction $1/2$ appears because $\left\|\boldsymbol{\omega}\right\|_F^2 = \left\|\boldsymbol{\Omega}\right\|_F^2/2$ as in Eq.~(\ref{eq:omega^TmbbLomega}). Then, for the second term, using $|\Tr(\mathbf{A}_1\mathbf{A}_2)| \leq \left\|\mathbf{A}_1\right\|_2\Tr(\mathbf{A}_2)$ and Cauchy-Schwarz inequality,
\begin{align}
\left|\Tr(\boldsymbol{\Omega}^T(\mathbf{D}_{\mathbf{O}}^T\mathbf{D}_{\mathbf{S}}-\mathbf{I}_{md})\mathbf{L}(\mathbf{S})\mathbf{D}_{\mathbf{S}}^T\mathbf{D}_{\mathbf{O}}\boldsymbol{\Omega})\right| &\leq \left\|\mathbf{L}(\mathbf{S})\right\|_2\left|\Tr(\boldsymbol{\Omega}^T(\mathbf{D}_{\mathbf{O}}^T\mathbf{D}_{\mathbf{S}}-\mathbf{I}_{md})\boldsymbol{\Omega})\right|\\
&\leq c_3(\mathbf{S})\left\|\mathbf{S}-\mathbf{O}\right\|_F\left\|\boldsymbol{\Omega}\right\|_F^2.
\end{align}
Overall,
\begin{align}
    \Tr(\boldsymbol{\Omega}^T(\mathbf{L}(\mathbf{O})+\mathbf{L}(\mathbf{O})^T)\boldsymbol{\Omega}) &\geq (\lambda_{-}(\mathbf{S}) - 2(c_1 + c_2(\mathbf{S}) + 2 c_3(\mathbf{S}))\left\|\mathbf{S}-\mathbf{O}\right\|_F)\left\|\boldsymbol{\Omega}\right\|_F^2\\
    \Tr(\boldsymbol{\Omega}^T(\mathbf{L}(\mathbf{O})+\mathbf{L}(\mathbf{O})^T)\boldsymbol{\Omega}) &\leq (\lambda_{+}(\mathbf{S}) + 2(c_1 + c_2(\mathbf{S}) + 2 c_3(\mathbf{S}))\left\|\mathbf{S}-\mathbf{O}\right\|_F)\left\|\boldsymbol{\Omega}\right\|_F^2
\end{align}
% \begin{equation}
% \begin{matrix}
% \Tr(\boldsymbol{\Omega}^T(\mathbf{L}(\mathbf{O})+\mathbf{L}(\mathbf{O})^T)\boldsymbol{\Omega}) \geq (\lambda_{-}(\mathbf{S}) - 2(c_1 + c_2(\mathbf{S}) + 2 c_3(\mathbf{S}))\left\|\mathbf{S}-\mathbf{O}\right\|_F)\left\|\boldsymbol{\Omega}\right\|_F^2\\
% \Tr(\boldsymbol{\Omega}^T(\mathbf{L}(\mathbf{O})+\mathbf{L}(\mathbf{O})^T)\boldsymbol{\Omega}) \leq (\lambda_{+}(\mathbf{S}) + 2(c_1 + c_2(\mathbf{S}) + 2 c_3(\mathbf{S}))\left\|\mathbf{S}-\mathbf{O}\right\|_F)\left\|\boldsymbol{\Omega}\right\|_F^2.
% \end{matrix}
% \end{equation}
\revadd{Consequently, if $\left\|\mathbf{S}-\mathbf{O}\right\|_F < \zeta\delta(\mathbf{S})$ (as defined in the theorem statement) then}
\begin{equation}
\revadd{(1-\zeta)\lambda_{-}(\mathbf{S})\left\|\boldsymbol{\Omega}\right\|_F^2 \leq \Tr(\boldsymbol{\Omega}^T(\mathbf{L}(\mathbf{O})+\mathbf{L}(\mathbf{O})^T)\boldsymbol{\Omega}) \leq (\lambda_{+}(\mathbf{S}) + \zeta \lambda_{-}(\mathbf{S}))\left\|\boldsymbol{\Omega}\right\|_F^2.}
\end{equation}
\revadd{Finally, due to Proposition~\ref{prop:one_all1} and the fact that $\delta, \lambda_{-}$ and $\lambda_{+}$ are invariant under the action of $\mathbb{O}(d)$, we can replace $\mathbf{S}$ by $\mathbf{S}\mathbf{Q}$ for any $\mathbf{Q} \in \mathbb{O}(d)$, and the result follows.}

\proofof{Theorem~\ref{thm:non_deg_two_views_gen_setting}} The proof is divided into three parts specialized to the three conditions in the statement.

\noindent \underline{\textbf{Part 1}}. First note that $\mathbf{S} \in \mathcal{C}$ iff $\mathbf{S}_i^T[\mathbf{C}\mathbf{S}]_i = [\mathbf{C}\mathbf{S}]_i^T\mathbf{S}_i$ for $i = 1,2$ (see Eq.~(\ref{eq:crit_pts2})). Since
% \begin{align}
%     \mathbf{S}_1^T[\mathbf{C}\mathbf{S}]_1 - [\mathbf{C}\mathbf{S}]_1^T\mathbf{S}_1 &= \mathbf{S}_1^T\mathbf{B}_1\boldsymbol{\mathcal{L}}_{\Gamma}^\dagger \mathbf{B}_2^T\mathbf{S}_2 - \mathbf{S}_2^T\mathbf{B}_2\boldsymbol{\mathcal{L}}_{\Gamma}^\dagger \mathbf{B}_1^T\mathbf{S}_1\\
%     &= \mathbf{B}(\mathbf{S})_1\boldsymbol{\mathcal{L}}_{\Gamma}^\dagger \mathbf{B}(\mathbf{S})_2^T- \mathbf{B}(\mathbf{S})_2\boldsymbol{\mathcal{L}}_{\Gamma}^\dagger \mathbf{B}(\mathbf{S})_1^T,
% \end{align}
\begin{equation}
   \mathbf{S}_1^T[\mathbf{C}\mathbf{S}]_1 - [\mathbf{C}\mathbf{S}]_1^T\mathbf{S}_1 =  [\mathbf{C}\mathbf{S}]_2^T\mathbf{S}_2 - \mathbf{S}_2^T[\mathbf{C}\mathbf{S}]_2 = \mathbf{B}(\mathbf{S})_1\boldsymbol{\mathcal{L}}_{\Gamma}^\dagger \mathbf{B}(\mathbf{S})_2^T- \mathbf{B}(\mathbf{S})_2\boldsymbol{\mathcal{L}}_{\Gamma}^\dagger \mathbf{B}(\mathbf{S})_1^T,
\end{equation}
thus $\mathbf{S} \in \mathcal{C}$ iff $\mathbf{B}(\mathbf{S})_1\boldsymbol{\mathcal{L}}_{\Gamma}^\dagger \mathbf{B}(\mathbf{S})_2^T$ is symmetric. At this point, we note that
%the forms of $\mathbf{B}(\mathbf{S})_{1}$ and $\mathbf{B}(\mathbf{S})_{2}$ are
\begin{equation}
    \begin{matrix}
        \mathbf{B}(\mathbf{S})_1 & = & [ & \mathbf{X}_1 & \mathbf{0}_{d \times n_2} & \mathbf{X}_3 & -(\mathbf{X}_1\mathbf{1}_{n_1} + \mathbf{X}_3\mathbf{1}_{n_3}) & \mathbf{0}_{d} & ]\\
        \mathbf{B}(\mathbf{S})_2 & = & [ & \mathbf{0}_{d \times n_1} & \mathbf{Y}_2 & \mathbf{Y}_3 & \mathbf{0}_{d} & -(\mathbf{Y}_2\mathbf{1}_{n_2} + \mathbf{Y}_3\mathbf{1}_{n_3}) & ]
    \end{matrix}
\end{equation}
where (see Remark~\ref{rmk:L0DB}) $\mathbf{X}_1 \in \mathbb{R}^{d \times n_1}$ and $\mathbf{X}_3 \in \mathbb{R}^{d \times n_3}$ correspond to the local coordinates, due to the first view, of the $n_1$ points that lie exclusively in the first view and the $n_3$ points that lie on the overlap of both views, respectively. Similarly, $\mathbf{Y}_2 \in \mathbb{R}^{d \times n_2}$ and $\mathbf{Y}_3 \in \mathbb{R}^{d \times n_3}$ correspond to the local coordinates, due to the second view, of the $n_2$ points that lie exclusively in the second view and the $n_3$ points which lie on the overlap of both views, respectively. In particular, $\mathbf{X}_3 = \mathbf{B}(\mathbf{S})_{1,2}$ and $\mathbf{Y}_3 = \mathbf{B}(\mathbf{S})_{2,1}$ (perhaps after permuting the points). Moreover,
\begin{equation}
    \boldsymbol{\mathcal{L}}_{\Gamma} = \begin{bmatrix}
    \mathbf{I}_{n_1} & & & -\mathbf{1}_{n_1} & \mathbf{0}_{n_1}\\
     & \mathbf{I}_{n_2} & & \mathbf{0}_{n_2} & -\mathbf{1}_{n_2}\\
    &  & 2\mathbf{I}_{n_3} & -\mathbf{1}_{n_3} & -\mathbf{1}_{n_3}\\
    -\mathbf{1}_{n_1}^T & \mathbf{0}_{n_2}^T & -\mathbf{1}_{n_3}^T & n_1+n_3 & \\
    \mathbf{0}_{n_1}^T & -\mathbf{1}_{n_2}^T & -\mathbf{1}_{n_3}^T &  & n_2+n_3\end{bmatrix}.
\end{equation}
Through simple calculations, we obtain
\begin{equation}
    \boldsymbol{\mathcal{L}}_{\Gamma}^\dagger = \frac{1}{2n_3}\begin{bmatrix}
    2n_3\mathbf{I}_{n_1} + \mathbf{1}_{n_1}\mathbf{1}_{n_1}^T & -\mathbf{1}_{n_1}\mathbf{1}_{n_2}^T & & \mathbf{1}_{n_1} & -\mathbf{1}_{n_1}\\
    -\mathbf{1}_{n_2}\mathbf{1}_{n_1}^T & 2n_3\mathbf{I}_{n_2} + \mathbf{1}_{n_2}\mathbf{1}_{n_2}^T & & -\mathbf{1}_{n_2} & \mathbf{1}_{n_2}\\
    &  & n_3\mathbf{I}_{n_3} &  & \\
    \mathbf{1}_{n_1}^T & -\mathbf{1}_{n_2}^T &  & 1 & -1\\
    -\mathbf{1}_{n_1}^T & \mathbf{1}_{n_2}^T & &  -1 & 1\end{bmatrix}.
\end{equation}
Thus,
$$\mathbf{B}(\mathbf{S})_1\boldsymbol{\mathcal{L}}_{\Gamma}^\dagger = \begin{bmatrix}\mathbf{X}_1 - \frac{\mathbf{X}_3\mathbf{1}_{n_3}\mathbf{1}_{n_1}^T}{2n_3}, & \frac{\mathbf{X}_3\mathbf{1}_{n_3}\mathbf{1}_{n_2}^T}{2n_3}, & \frac{1}{2}\mathbf{X}_3, & -\frac{\mathbf{X}_3\mathbf{1}_{n_3}}{2n_3}, & \frac{\mathbf{X}_3\mathbf{1}_{n_3}}{2n_3} \end{bmatrix}.$$
Then, using Definition~\ref{def:BSicapj}
\begin{align}
    \mathbf{B}(\mathbf{S})_1\boldsymbol{\mathcal{L}}_{\Gamma}^\dagger \mathbf{B}(\mathbf{S})_2^T &= \frac{1}{2}\mathbf{X}_3\left(\mathbf{I}_{n_3}-\frac{1}{n_3}\mathbf{1}_{n_3}\mathbf{1}_{n_3}^T\right)\mathbf{Y}_3^T\\
    &=\frac{1}{2}\mathbf{B}(\mathbf{S})_{1,2}\left(\mathbf{I}_{n_3}-\frac{1}{n_3}\mathbf{1}_{n_3}\mathbf{1}_{n_3}^T\right)\left(\mathbf{I}_{n_3}-\frac{1}{n_3}\mathbf{1}_{n_3}\mathbf{1}_{n_3}^T\right)^T\mathbf{B}(\mathbf{S})_{2,1}^T\\
    &= \frac{1}{2}\overline{\mathbf{B}(\mathbf{S})}_{1,2}\overline{\mathbf{B}(\mathbf{S})}_{2,1}^T
\end{align}
% \begin{align}
%     &\mathbf{B}(\mathbf{S})_2\boldsymbol{\mathcal{L}}_{\Gamma}^\dagger \mathbf{B}(\mathbf{S})_1^T = \mathbf{B}(\mathbf{S})_1\boldsymbol{\mathcal{L}}_{\Gamma}^\dagger \mathbf{B}(\mathbf{S})_2^T
%     %&= \begin{bmatrix}\mathbf{X}_1 - \frac{\mathbf{X}_3\mathbf{1}_{n_3}\mathbf{1}_{n_1}^T}{2n_3} & \frac{\mathbf{X}_3\mathbf{1}_{n_3}\mathbf{1}_{n_2}^T}{2n_3} & \frac{1}{2}\mathbf{X}_3 & -\frac{\mathbf{X}_3\mathbf{1}_{n_3}}{2n_3} & \frac{\mathbf{X}_3\mathbf{1}_{n_3}}{2n_3} \end{bmatrix}\begin{bmatrix}0 \\ \mathbf{Y}_2^T \\ \mathbf{Y}_3^T \\ 0 \\ -(\mathbf{1}_{n_2}^T\mathbf{Y}_2^T + \mathbf{1}_{n_3}^T\mathbf{Y}_3^T)\end{bmatrix}\\
%     = \frac{1}{2}\mathbf{X}_3\left(\mathbf{I}_{n_3}-\frac{1}{n_3}\mathbf{1}_{n_3}\mathbf{1}_{n_3}^T\right)\mathbf{Y}_3^T\\
%     &= \frac{1}{2}\mathbf{B}(\mathbf{S})_{1,2}\left(\mathbf{I}_{n_3}-\frac{1}{n_3}\mathbf{1}_{n_3}\mathbf{1}_{n_3}^T\right)\left(\mathbf{I}_{n_3}-\frac{1}{n_3}\mathbf{1}_{n_3}\mathbf{1}_{n_3}^T\right)^T\mathbf{B}(\mathbf{S})_{2,1}^T = \frac{1}{2}\overline{\mathbf{B}(\mathbf{S})}_{1,2}\overline{\mathbf{B}(\mathbf{S})}_{2,1}^T.
% \end{align}
Since $\mathbf{S} \in \mathcal{C}$ iff $\mathbf{B}(\mathbf{S})_1\boldsymbol{\mathcal{L}}_{\Gamma}^\dagger \mathbf{B}(\mathbf{S})_2^T$ is symmetric, implies $\mathbf{S} \in \mathcal{C}$ iff $\overline{\mathbf{B}(\mathbf{S})}_{1,2}\overline{\mathbf{B}(\mathbf{S})}_{2,1}^T$ is symmetric. 

\noindent \underline{\textbf{Part 2}}. For $\mathbf{S} \in \mathcal{C}$, from the Remark~\ref{rmk:C_hat_L_structure} and Part 1, we have,
% \begin{align}
%     \mathbf{L}(\mathbf{S}) &= \begin{bmatrix}
%     -\mathbf{B}(\mathbf{S})_1\boldsymbol{\mathcal{L}}_{\Gamma}^\dagger \mathbf{B}(\mathbf{S})_2^T & \mathbf{B}(\mathbf{S})_1\boldsymbol{\mathcal{L}}_{\Gamma}^\dagger \mathbf{B}(\mathbf{S})_2^T\\
%     \mathbf{B}(\mathbf{S})_2\boldsymbol{\mathcal{L}}_{\Gamma}^\dagger \mathbf{B}(\mathbf{S})_1^T & -\mathbf{B}(\mathbf{S})_2\boldsymbol{\mathcal{L}}_{\Gamma}^\dagger \mathbf{B}(\mathbf{S})_1^T
%     \end{bmatrix}\\
%     &= \begin{bmatrix}
%     -\mathbf{B}(\mathbf{S})_1\boldsymbol{\mathcal{L}}_{\Gamma}^\dagger \mathbf{B}(\mathbf{S})_2^T & \mathbf{B}(\mathbf{S})_1\boldsymbol{\mathcal{L}}_{\Gamma}^\dagger \mathbf{B}(\mathbf{S})_2^T\\
%     \mathbf{B}(\mathbf{S})_1\boldsymbol{\mathcal{L}}_{\Gamma}^\dagger \mathbf{B}(\mathbf{S})_2^T & -\mathbf{B}(\mathbf{S})_1\boldsymbol{\mathcal{L}}_{\Gamma}^\dagger \mathbf{B}(\mathbf{S})_2^T
%     \end{bmatrix}. \label{supp:eq:LS_two_views}
% \end{align}
\begin{equation}
    \mathbf{L}(\mathbf{S}) = \begin{bmatrix}
    \mathbf{B}(\mathbf{S})_1\boldsymbol{\mathcal{L}}_{\Gamma}^\dagger \mathbf{B}(\mathbf{S})_2^T & -\mathbf{B}(\mathbf{S})_1\boldsymbol{\mathcal{L}}_{\Gamma}^\dagger \mathbf{B}(\mathbf{S})_2^T\\
    -\mathbf{B}(\mathbf{S})_1\boldsymbol{\mathcal{L}}_{\Gamma}^\dagger \mathbf{B}(\mathbf{S})_2^T & \mathbf{B}(\mathbf{S})_1\boldsymbol{\mathcal{L}}_{\Gamma}^\dagger \mathbf{B}(\mathbf{S})_2^T
    \end{bmatrix}. \label{supp:eq:LS_two_views}
\end{equation}
Let $\boldsymbol{\Omega}_1,\boldsymbol{\Omega}_2 \in \Skew(d)$ such that $\boldsymbol{\Omega}_1 + \boldsymbol{\Omega}_2 = 0$. Then, using the above equations, $$\Tr\left(\begin{bmatrix}
        \boldsymbol{\Omega}_1^T & \boldsymbol{\Omega}_2^T
    \end{bmatrix} \mathbf{L}(\mathbf{S}) \begin{bmatrix}
        \boldsymbol{\Omega}_1\\\boldsymbol{\Omega}_2
    \end{bmatrix}\right) = \Tr\left(\begin{bmatrix}
        -\boldsymbol{\Omega}_1 & \boldsymbol{\Omega}_1
    \end{bmatrix} \mathbf{L}(\mathbf{S}) \begin{bmatrix}
        \boldsymbol{\Omega}_1\\-\boldsymbol{\Omega}_1
    \end{bmatrix}\right) = 2 \Tr(\boldsymbol{\Omega}_1^T\overline{\mathbf{B}(\mathbf{S})}_{1,2}\overline{\mathbf{B}(\mathbf{S})}_{2,1}^T\boldsymbol{\Omega}_1).$$
% \begin{align}
%     \Tr\left(\begin{bmatrix}
%         \boldsymbol{\Omega}_1^T & \boldsymbol{\Omega}_2^T
%     \end{bmatrix} \mathbf{L}(\mathbf{S}) \begin{bmatrix}
%         \boldsymbol{\Omega}_1\\\boldsymbol{\Omega}_2
%     \end{bmatrix}\right) &= \Tr\left(\begin{bmatrix}
%         -\boldsymbol{\Omega}_1 & \boldsymbol{\Omega}_1
%     \end{bmatrix} \mathbf{L}(\mathbf{S}) \begin{bmatrix}
%         \boldsymbol{\Omega}_1\\-\boldsymbol{\Omega}_1\\
%     \end{bmatrix}\right)\\
%     &= -2 \Tr(\boldsymbol{\Omega}_1^T\overline{\mathbf{B}(\mathbf{S})}_{1,2}\overline{\mathbf{B}(\mathbf{S})}_{2,1}^T\boldsymbol{\Omega}_1).
% \end{align}
Combining the above and Part 1 with Proposition~\ref{prop:HessFSZZ}, we conclude that $\pi(\mathbf{S})$ is a local minimum of $\widetilde{F}$ iff the first two conditions of the statement are met.

\noindent \underline{\textbf{Part 3}}. Here we deal with the non-degeneracy of $\widetilde{\mathbf{S}} = \pi(\mathbf{S})$. For $d=1$, $\widetilde{\mathbf{S}}$ is trivially non-degenerate. So we assume that $d \geq 2$. From Part 2, we note that for $\boldsymbol{\Omega}_1,\boldsymbol{\Omega}_2 \in \Skew(d)$ such that $\boldsymbol{\Omega}_1 + \boldsymbol{\Omega}_2 = 0$, $\mathbf{L}(\mathbf{S})[\boldsymbol{\Omega}_i]_1^2 = 0$ iff $\overline{\mathbf{B}(\mathbf{S})}_{1,2}\overline{\mathbf{B}(\mathbf{S})}_{2,1}^T\boldsymbol{\Omega} = 0$. Thus $\widetilde{\mathbf{S}}$ is non-degenerate iff $\overline{\mathbf{B}(\mathbf{S})}_{1,2}\overline{\mathbf{B}(\mathbf{S})}_{2,1}^T\boldsymbol{\Omega} = 0 \iff \boldsymbol{\Omega} = 0$. It suffices to show that $\overline{\mathbf{B}(\mathbf{S})}_{1,2}\overline{\mathbf{B}(\mathbf{S})}_{2,1}^T\boldsymbol{\Omega} = 0$  iff $\rank (\overline{\mathbf{B}(\mathbf{S})}_{1,2}\overline{\mathbf{B}(\mathbf{S})}_{2,1}^T) \geq d-1$.

($\impliedby$) Suppose $\rank (\overline{\mathbf{B}(\mathbf{S})}_{1,2}\overline{\mathbf{B}(\mathbf{S})}_{2,1}^T) \geq d-1$ then null space of $\overline{\mathbf{B}(\mathbf{S})}_{1,2}\overline{\mathbf{B}(\mathbf{S})}_{2,1}^T$ is at most one-dimensional. Moreover, rank of a non-zero skew symmetric matrix of size $d \geq 2$, is at least two. Thus $\overline{\mathbf{B}(\mathbf{S})}_{1,2}\overline{\mathbf{B}(\mathbf{S})}_{2,1}^T\boldsymbol{\Omega} = 0 \iff \boldsymbol{\Omega} = 0$. We conclude that
%$\mathbf{L}(\mathbf{S})$ has trivial certificates only, and thus
$\widetilde{\mathbf{S}}$ is non-degenerate.

$(\implies)$ Suppose $\rank (\overline{\mathbf{B}(\mathbf{S})}_{1,2}\overline{\mathbf{B}(\mathbf{S})}_{2,1}^T) \leq d-2$, then there exist non-zero vectors $\mathbf{u},\mathbf{v} \in \mathbb{R}^d$ in the kernel of $\overline{\mathbf{B}(\mathbf{S})}_{1,2}\overline{\mathbf{B}(\mathbf{S})}_{2,1}^T$ such that $\mathbf{u}^T\mathbf{v} = 0$. Let $\boldsymbol{\Omega} = \mathbf{u}\mathbf{v}^T - \mathbf{v}\mathbf{u}^T$ then clearly $\boldsymbol{\Omega} \in \Skew(d)$, $\boldsymbol{\Omega} \neq 0$ and $\overline{\mathbf{B}(\mathbf{S})}_{1,2}\overline{\mathbf{B}(\mathbf{S})}_{2,1}^T\boldsymbol{\Omega} = 0$. 

% \proofof{Theorem~\ref{thm:uniq_two_views_gen_setting}}
% Let $\mathbf{S}$ be an optimal alignment (a global minimum of $F$), then from proof of Theorem~\ref{thm:non_deg_two_views_gen_setting}, $\mathbf{C}(\mathbf{S})_{1,2} = -\mathbf{B}(\mathbf{S})_1\boldsymbol{\mathcal{L}}_{\Gamma}^\dagger \mathbf{B}(\mathbf{S})_2^T = -\frac{1}{2}\mathbf{S}_1^T\overline{\mathbf{B}}_{1,2}\overline{\mathbf{B}}_{2,1}^T\mathbf{S}_2$.
% % \begin{align}
% %     \mathbf{C}(\mathbf{S})_{1,2} = -\mathbf{B}(\mathbf{S})_1\boldsymbol{\mathcal{L}}_{\Gamma}^\dagger \mathbf{B}(\mathbf{S})_2^T = -\frac{1}{2}\mathbf{S}_1^T\overline{\mathbf{B}}_{1,2}\overline{\mathbf{B}}_{2,1}^T\mathbf{S}_2 .%= \frac{1}{2}\mathbf{S}_2^T\overline{\mathbf{B}}_{2,1}\overline{\mathbf{B}}_{1,2}^T\mathbf{S}_1 = \mathbf{B}(\mathbf{S})_2\boldsymbol{\mathcal{L}}_{\Gamma}^\dagger \mathbf{B}(\mathbf{S})_1^T = \mathbf{C}(\mathbf{S})_{21}.
% % \end{align}
% Then note that the minimizer of $\Tr(\mathbf{C}\mathbf{S}\mathbf{S}^T)$ is same as the minimizer of $\Tr(\mathbf{C}(\mathbf{S})_{1,2})$ because $\mathbf{S}_1\mathbf{S}_1^T = \mathbf{S}_2\mathbf{S}_2^T = \mathbf{I}_d$ and $\mathbf{C}_{2,1} = \mathbf{C}_{1,2}^T$ is symmetric. WLOG, take $\mathbf{S}_1 = \mathbf{I}_d$. Then it suffices to show that $\textstyle\min_{\mathbf{S}_2\in\mathbb{O}(d)}\Tr(\mathbf{C}(\mathbf{S})_{12}) = -0.5\textstyle\max_{\mathbf{S}_2\in\mathbb{O}(d)} \Tr(\overline{\mathbf{B}}_{1,2}\overline{\mathbf{B}}_{2,1}^T\mathbf{S}_2)$ has a unique solution iff $\rank (\overline{\mathbf{B}}_{1,2}\overline{\mathbf{B}}_{2,1}^T) = d$.

% Let $\overline{\mathbf{B}}_{1,2}\overline{\mathbf{B}}_{2,1}^T = \mathbf{U}\mathbf{\Sigma}\mathbf{V}^T$ be its singular value decomposition. Note that $\mathbf{\Sigma} \succeq 0$. If $\rank (\overline{\mathbf{B}}_{1,2}\overline{\mathbf{B}}_{2,1}^T) = d$ then $\mathbf{\Sigma} \succ 0$ and clearly, the unique optimizer is $\mathbf{S}_2=\mathbf{V}\mathbf{U}^T$. If $\rank (\overline{\mathbf{B}}_{1,2}\overline{\mathbf{B}}_{2,1}^T) \leq d-1$ then there exist orthogonal matrix $\mathbf{U}' \neq \mathbf{I}_d$ such that $\boldsymbol{\Sigma} = \mathbf{U}'\boldsymbol{\Sigma}$ and thus $\mathbf{V}\mathbf{U}^T \neq \mathbf{V}(\mathbf{U}'\mathbf{U})^T$ are both optimizers of the above objective. 

%%%%%%%%%%%%%%%%%%%%%%%%%%%%%%%%%%%%%%%%%%
% Section 4 Proofs
%%%%%%%%%%%%%%%%%%%%%%%%%%%%%%%%%%%%%%%%%%
\proofof{Proposition~\ref{prop:noiseless_setting1}}
Since $\mathbf{S}$ is a perfect alignment $F(\mathbf{S}) = \Tr(\mathbf{C}\mathbf{S}\mathbf{S}^T) = 0$. Since $\mathbf{C} \succeq 0$, the columns of $\mathbf{S}$ lie in the kernel of $\mathbf{C}$. In particular $\mathbf{C}\mathbf{S} = \mathbf{0}$. It follows that $\widehat{\mathbf{C}}(\mathbf{S}) = \mathbf{0}$ (see Eq.~(\ref{eq:C_hat})). We conclude that $\mathbf{L}(\mathbf{S}) = \mathbf{C}(\mathbf{S})$.

\proofof{Proposition~\ref{prop:non_deg_views}} \revadd{WLOG assume that each views is centered at the origin i.e. $\mathbf{B}_{i,i}\mathbf{1}_{n_i} = 0$ where $n_i$ is the number of points in the $i$th view. Due to Assumption~\ref{assump:non_deg_views}, the matrix $\mathbf{B}_{i,i}\mathbf{B}_{i,i}^T$ has a rank of $d$ and consequently $\sigma_{\min}(\mathbf{B}_{i,i}\mathbf{B}_{i,i}^T) > 0$. Let $\Theta(\mathbf{S})_i$ and $\Theta(\mathbf{O})_i$ be the realizations of the points in the $i$th views i.e. of $\mathbf{B}_{i,i}$, due to the perfect alignments $\mathbf{S}$ and $\mathbf{O}$, respectively. Also, the optimal translation of the $i$th view due to $\mathbf{S}$ is given by $\mathbf{S}^T\mathbf{B}\boldsymbol{\mathcal{L}}_{\Gamma}^\dagger\mathbf{e}^{n+m}_{n+i}$ and the translation of all the views so that $\Theta(\mathbf{S})$ is centered at the origin is given by $-\mathbf{S}^T\mathbf{B}\boldsymbol{\mathcal{L}}_{\Gamma}^\dagger\mathbf{1}^{n+m}_{n}$. For brevity, define the vector $\mathbf{v}_i\coloneqq \mathbf{B}\boldsymbol{\mathcal{L}}_{\Gamma}^\dagger(\mathbf{e}^{n+m}_{n+i}-\mathbf{1}^{n+m}_{n})$. The net translation for the $i$th view due to $\mathbf{S}$ is the sum of the two translations $\mathbf{S}^T\mathbf{v}_i$ (similarly for $\mathbf{O}$). Then the result follows from,}
\begin{align}
    &\left\|\Theta(\mathbf{S}) - \Theta(\mathbf{O})\right\|^2_F \geq \left\|\Theta(\mathbf{S})_i - \Theta(\mathbf{O})_i\right\|^2_F\\
    &= \left\|(\mathbf{S}_i^T\mathbf{B}_{i,i} + \mathbf{S}^T\mathbf{v}_i\mathbf{1}_{n_i}^T) -(\mathbf{O}_i^T\mathbf{B}_{i,i} + \mathbf{O}^T\mathbf{v}_i\mathbf{1}_{n_i}^T)  \right\|_F^2\\
    &= \left\|(\mathbf{S}_i-\mathbf{O}_i)^T\mathbf{B}_{i,i}\right\|_F^2 + \left\|(\mathbf{S}-\mathbf{O})^T\mathbf{v}_i\mathbf{1}_{n_i}^T\right\|_F^2 - 2\Tr((\mathbf{S}_i-\mathbf{O}_i)^T\mathbf{B}_{i,i}\mathbf{1}_{n_i}\mathbf{v}_i^T(\mathbf{S}-\mathbf{O}))\\
    &\geq \left\|(\mathbf{S}_i-\mathbf{O}_i)^T\mathbf{B}_{i,i}\right\|_F^2 \geq \left\|\mathbf{S}_i-\mathbf{O}_i\right\|_F^2\sigma_{\min}(\mathbf{B}_{i,i}\mathbf{B}_{i,i}^T).
\end{align}

\proofof{Theorem~\ref{thm:inf_rigid}}
\revadd{Suppose $\mathbf{S}$ is degenerate then, due to Theorem~\ref{prop:noiseless_setting1} and Definition~\ref{def:LScertificate}, there exist $\boldsymbol{\Omega}$ such that $\mathbf{C}(\mathbf{S}) \boldsymbol{\Omega} = 0$. Since $\mathbf{C}(\mathbf{S}) \succeq 0$ therefore $\Tr(\mathbf{C}(\mathbf{S}) \boldsymbol{\Omega}\boldsymbol{\Omega}^T) = 0$. Following Eq.~(\ref{eq:opt_Z}), we set the perturbations to be $\mathbf{p}_k \coloneqq \boldsymbol{\Omega}^T\blockdiag(\mathbf{S}^T)\mathbf{B}\boldsymbol{\mathcal{L}}_{\Gamma}^\dagger \mathbf{e}^{n+m}_{k}$ for $k \in [1,n+m]$. Then, following Eq.~(\ref{eq:A_1_}) and Eq.~(\ref{eq:A_1}) in that order, and the fact that $\Tr(\mathbf{C}(\mathbf{S}) \boldsymbol{\Omega}\boldsymbol{\Omega}^T) = 0$,  we obtain $\mathbf{p}_{k} = \boldsymbol{\Omega}_i^T\mathbf{S}_i^T \mathbf{x}_{k,i} + \mathbf{p}_{n+i}$. Consequently, for $(k_1,i), (k_2,i)\in E(\Gamma)$, $\mathbf{p}_{k_1} - \mathbf{p}_{k_2} = \boldsymbol{\Omega}_i^T\mathbf{S}_i^T (\mathbf{x}_{k_1,i}-\mathbf{x}_{k_2,i})$. Similarly, since $\mathbf{S}$ is a perfect alignment, $\mathbf{x}_{k_1}(\mathbf{S}) - \mathbf{x}_{k_2}(\mathbf{S}) = \mathbf{S}_i^T (\mathbf{x}_{k_1,i}-\mathbf{x}_{k_2,i})$. Finally, $(\mathbf{p}_{k_1} - \mathbf{p}_{k_2})^T(\mathbf{x}_{k_1}(\mathbf{S}) - \mathbf{x}_{k_2}(\mathbf{S})) = (\mathbf{x}_{k_1,i}-\mathbf{x}_{k_2,i})^T\mathbf{S}_i\boldsymbol{\Omega}_i\mathbf{S}_i^T (\mathbf{x}_{k_1,i}-\mathbf{x}_{k_2,i}) = 0$ since $\Tr(\boldsymbol{\Omega}_i \mathbf{A}) = 0$ for any symmetric matrix $\mathbf{A}$.}

\revadd{Now suppose $\Theta(\mathbf{S}) = (\mathbf{x}_k(\mathbf{S}))_1^n$ is not infinitesimally rigid. From Definition~\ref{def:inf_rigid}, there exist a non-trivial perturbation $(\mathbf{p}_k)_1^n$ such that $(\mathbf{p}_{k_1}-\mathbf{p}_{k_2})^T(\mathbf{x}_{k_1}(\mathbf{S})-\mathbf{x}_{k_2}(\mathbf{S})) = 0$ for all $(k_1,i), (k_2,i) \in E(\Gamma)$. Combining this with the Assumption~\ref{assump:non_deg_views} that each view is affinely non-degenerate, it follows from  \citea{schulze2010symmetric, asimow1978rigidity} that for each $i \in [1,m]$ there exist $\boldsymbol{\Omega}_i \in \Skew(d)$ and and $\mathbf{t}_i \in \mathbf{R}^d$ such that for each $(k,i) \in E(\Gamma)$, $\mathbf{p}_k = \boldsymbol{\Omega}_i^T\mathbf{x}_k(\mathbf{S}) + \mathbf{t}_i$. Therefore, following Eq.(\ref{eq:A_1}, \ref{eq:A_1_}, \ref{eq:GPOP}), we conclude that $\mathbf{C}(\mathbf{S})\boldsymbol{\Omega} = 0$. From Proposition~\ref{prop:noiseless_setting1} and Proposition~\ref{prop:non_deg_triv_cert}, it suffices to show that $\boldsymbol{\Omega}$ is non-trivial i.e. not all $\boldsymbol{\Omega}_i$'s are equal. In fact, since $\Gamma$ is connected (Assumption~\ref{assump:connected_gamma}) and for $(k,i), (k,j) \in E(\Gamma)$, $\mathbf{p}_k = \boldsymbol{\Omega}_i^T\mathbf{x}_k(\mathbf{S}) + \mathbf{t}_i = \boldsymbol{\Omega}_j^T\mathbf{x}_k(\mathbf{S}) + \mathbf{t}_j$, therefore if $\boldsymbol{\Omega}_i = \boldsymbol{\Omega}_j$ for all $i,j \in [1,m]$ then $\mathbf{t}_i = \mathbf{t}_j$ too. As a result, the perturbation $(\mathbf{p}_k)_1^n$ ends up being trivial, a contradiction.}

\proofof{Theorem~\ref{thm:loc_rigid}}
First, under Assumption~\ref{assump:non_deg_views}, the equation in Definition~\ref{def:loc_rigid}, $\Theta(\mathbf{O}) = \Theta(\mathbf{S}\mathbf{Q})$, is equivalent to $\pi(\mathbf{S}) = \pi(\mathbf{O})$.

($\impliedby$) Suppose $\mathbf{S}$ is a perfect alignment but $\pi(\mathbf{S})$ is not a strict global minimum of $\widetilde{F}$. 
%Thus $\pi(\mathbf{S})$ is degenerate (see Definition~\ref{def:non_deg_alignment0}).
Define 
$$\eta \coloneqq \left\|\mathbf{B}\boldsymbol{\mathcal{L}}_{\Gamma}^\dagger(:,1:n)\left(\mathbf{I}_n - n^{-1}\mathbf{1}_{n}\mathbf{1}_{n}^T\right)\right\|_F$$
and let $\epsilon > 0$ be arbitrary.
% \begin{align}
%     \eta \coloneqq \left\|\mathbf{B}\boldsymbol{\mathcal{L}}_{\Gamma}^\dagger(:,1:n)\left(\mathbf{I}_n - n^{-1}\mathbf{1}_{n}\mathbf{1}_{n}^T\right)\right\|_F.
% \end{align}
Then there exists another perfect alignment $\mathbf{O} \in \mathbb{O}(d)^m$ such that $\left\|\mathbf{S}-\mathbf{O}\right\|_F < \epsilon / \eta$ and $\pi(\mathbf{S}) \neq \pi(\mathbf{O})$. Due to Assumption~\ref{assump:non_deg_views}, we have $\Theta(\mathbf{O}) \neq \Theta(\mathbf{S}\mathbf{Q})$ for any $\mathbf{Q} \in \mathbb{O}(d)$, however, from Definition~\ref{def:realization},
$$\left\|\Theta(\mathbf{O})-\Theta(\mathbf{S})\right\|_F \leq \left\|\mathbf{S}-\mathbf{O}\right\|_F\left\|\mathbf{B}\boldsymbol{\mathcal{L}}_{\Gamma}^\dagger(:,1:n)\left(\mathbf{I}_n - n^{-1}\mathbf{1}_{n}\mathbf{1}_{n}^T\right)\right\|_F = \eta\left\|\mathbf{S}-\mathbf{O}\right\|_F < \epsilon.$$
% \begin{equation}
%     \left\|\Theta(\mathbf{O})-\Theta(\mathbf{S})\right\|_F \leq \left\|\mathbf{S}-\mathbf{O}\right\|_F\left\|\mathbf{B}\boldsymbol{\mathcal{L}}_{\Gamma}^\dagger(:,1:n)\left(\mathbf{I}_n - n^{-1}\mathbf{1}_{n}\mathbf{1}_{n}^T\right)\right\|_F < \eta\left\|\mathbf{S}-\mathbf{O}\right\|_F < \epsilon.
% \end{equation}
Since $\epsilon$ is arbitrary, we conclude that $\Theta(\mathbf{S})$ is not locally rigid.

($\implies$) Let $\epsilon > 0$ be arbitrary. Suppose $\Theta(\mathbf{S})$ is not locally rigid, then there exist another perfect alignment $\mathbf{O}_\epsilon \in \mathbb{O}(d)^m$ such that $\left\|\Theta(\mathbf{O}_\epsilon)-\Theta(\mathbf{S})\right\|_F < \epsilon$ but $\Theta(\mathbf{O}_\epsilon) \neq \Theta(\mathbf{S}\mathbf{Q})$. \revadd{Due to Proposition~\ref{prop:non_deg_views}, $\left\|\mathbf{O}_\epsilon-\mathbf{S}\right\|_F < \varrho\epsilon$ where the constant $\varrho > 0$, but $\pi(\mathbf{O}_\epsilon) \neq \pi(\mathbf{S})$}. Since this true for all $\epsilon > 0$, we conclude that $\pi(\mathbf{S})$ is not a strict global minimum of $\widetilde{F}$.

\revadd{Finally, we note that a non-degenerate extremum is a strict extremum and the result follows.}

\proofof{Theorem~\ref{thm:nec_cond_loc_rigid_of_views}}
Consider a partition of $[1,m]$ into two non-empty subsets $A$ and $B$. Suppose $\rank(\overline{\mathbf{B}(\mathbf{S})}_{A, B})$ is at most $d-2$. Let $\boldsymbol{\Omega}_0 \in \Skew(d)$ be such that $\boldsymbol{\Omega}_0 \neq 0$ and $\overline{\mathbf{B}(\mathbf{S})}_{A,B}^T\boldsymbol{\Omega}_0 = 0$ (its existence follows from the third part of the proof of Theorem~\ref{thm:non_deg_two_views_gen_setting}). WLOG assume that $\mathbf{B}(\mathbf{S})_{A,B}\mathbf{1}_{n'} = 0$ (here $n'$ is as in Definition~\ref{def:BSAcapB}) (perhaps by translating all aligned views by $-\mathbf{B}(\mathbf{S})_{A,B}\mathbf{1}_{n'}$). Then $\mathbf{B}(\mathbf{S})_{A,B}^T\boldsymbol{\Omega}_0 = 0$.

Let $\boldsymbol{\Omega} = [\boldsymbol{\Omega}_i]_1^m$ be such that $\boldsymbol{\Omega}_i = \boldsymbol{\Omega}_0$ for $i \in A$ and $\boldsymbol{\Omega}_i = -\boldsymbol{\Omega}_0$ for $i \in B$. Clearly, $\boldsymbol{\Omega} \in \Skew(d)^m$ such that not all $\boldsymbol{\Omega}_i$ are equal. It suffices to show that $\boldsymbol{\Omega}$ is a nontrivial certificate of $\mathbf{L}(\mathbf{S})$, equivalently $\Tr(\boldsymbol{\Omega}^T\mathbf{L}(\mathbf{S})\boldsymbol{\Omega}) = 0$. First we observe that for $i \in A$, 
$$[\mathbf{L}(\mathbf{S}) \boldsymbol{\Omega}]_i = (\mathbf{B}(\mathbf{S})_i\boldsymbol{\mathcal{L}}_{\Gamma}^\dagger \mathbf{B}(\mathbf{S})^T\mathbf{I}^m_d - \textstyle\sum_{1}^{m}(-1)^{\mathbf{1}_{B}(j)}\mathbf{B}(\mathbf{S})_i\boldsymbol{\mathcal{L}}_{\Gamma}^\dagger \mathbf{B}(\mathbf{S})_j^T)\boldsymbol{\Omega}_0 = 2 \textstyle\sum_{j \in B}\mathbf{B}(\mathbf{S})_i\boldsymbol{\mathcal{L}}_{\Gamma}^\dagger \mathbf{B}(\mathbf{S})_j^T \boldsymbol{\Omega}_0$$
where $\mathbf{1}_{B}(j) = 1$ iff $j \in B$.
% \begin{align}
%     [\mathbf{L}(\mathbf{S}) \boldsymbol{\Omega}]_i &= (-\mathbf{B}(\mathbf{S})_i\boldsymbol{\mathcal{L}}_{\Gamma}^\dagger \mathbf{B}(\mathbf{S})^T\mathbf{I}^m_d + \textstyle\sum_{j \in A}\mathbf{B}(\mathbf{S})_i\boldsymbol{\mathcal{L}}_{\Gamma}^\dagger \mathbf{B}(\mathbf{S})_j^T - \textstyle\sum_{j \in B}\mathbf{B}(\mathbf{S})_i\boldsymbol{\mathcal{L}}_{\Gamma}^\dagger \mathbf{B}(\mathbf{S})_j^T)\boldsymbol{\Omega}_0\\
%     &= -2 \textstyle\sum_{j \in B}\mathbf{B}(\mathbf{S})_i\boldsymbol{\mathcal{L}}_{\Gamma}^\dagger \mathbf{B}(\mathbf{S})_j^T \boldsymbol{\Omega}_0
% \end{align}
Similarly, for $i \in B$, 
$$[\mathbf{L}(\mathbf{S}) \boldsymbol{\Omega}]_i = -2 \textstyle\sum_{j \in A}\mathbf{B}(\mathbf{S})_i\boldsymbol{\mathcal{L}}_{\Gamma}^\dagger \mathbf{B}(\mathbf{S})_j^T \boldsymbol{\Omega}_0.$$ Denote by $\mathbf{B}_A$ and $\mathbf{B}_B$, the matrices $\textstyle\sum_{i \in A}\mathbf{B}(\mathbf{S})_i$ and $\textstyle\sum_{j \in B}\mathbf{B}(\mathbf{S})_j$, respectively. Thus,
\begin{equation}
    \Tr(\boldsymbol{\Omega}^T\mathbf{L}(\mathbf{S})\boldsymbol{\Omega}) = 4\Tr(\boldsymbol{\Omega}_0^T \mathbf{B}_A\boldsymbol{\mathcal{L}}_{\Gamma}^\dagger \mathbf{B}_B^T \boldsymbol{\Omega}_0). \label{supp:eq:Omega0TLSOmega0}
\end{equation}
% \begin{align}
%     \Tr(\boldsymbol{\Omega}^T\mathbf{L}(\mathbf{S})\boldsymbol{\Omega}) = -4\Tr\left(\boldsymbol{\Omega}_0^T \left(\textstyle\sum_{i \in A}\mathbf{B}(\mathbf{S})_i\right)\boldsymbol{\mathcal{L}}_{\Gamma}^\dagger \left(\textstyle\sum_{j \in B}\mathbf{B}(\mathbf{S})_j^T\right) \boldsymbol{\Omega}_0\right). \label{supp:eq:Omega0TLSOmega0}
% \end{align}

% Then, since $\mathbf{L}(\mathbf{S}) \preceq 0$,
% \begin{align}
%     \Tr(\boldsymbol{\Omega}_0^T \mathbf{B}_A\boldsymbol{\mathcal{L}}_{\Gamma}^\dagger \mathbf{B}_B^T \boldsymbol{\Omega}_0) \geq 0. \label{supp:eq:TrOmega0LOmega0_lb}
% \end{align}
We are going to show that the above evaluates to zero. WLOG assume that the first $n_1$ points lie in the views with indices in $A \setminus B$, next $n_2$ points  lie in the views with indices in $B \setminus A$ and the remaining $n_3$ points lie in the views with indices in $A \cap B$. Note that $n_1 + n_2 + n_3 = n$ and $|A| + |B| = m$. Then the matrices $\mathbf{B}_A$ and $\mathbf{B}_B$ (perhaps after permuting the views) have the following structure. 
\begin{align}
    \begin{matrix}
        \mathbf{B}_A & = & [ & \mathbf{X}_{1} & \mathbf{0}_{d \times n_2} & \mathbf{X}_{3} & \mathbf{U}_{1} + \mathbf{U}_{3} & \mathbf{0}_{d \times |B|} & ]\\
        \mathbf{B}_B & = & [ & \mathbf{0}_{d \times n_1} & \mathbf{Y}_{2} & \mathbf{Y}_{3} & \mathbf{0}_{d \times |A|} & \mathbf{V}_{2} + \mathbf{V}_{3} & ]
    \end{matrix}
\end{align}
where $\mathbf{X}_{1} \in \mathbb{R}^{d \times n_1}$ and $\mathbf{Y}_{2} \in \mathbb{R}^{d \times n_2}$ contain the sum of the local coordinates of the $n_1$ and $n_2$ points, respectively. Also, $\mathbf{X}_{3} \in \mathbb{R}^{d \times n_3}$ and $\mathbf{Y}_{3} \in \mathbb{R}^{d \times n_3}$ contain the sum of the local coordinates of the remaining $n_3$ points due to the views with indices in $A$ and $B$ respectively. The matrices $\mathbf{U}_{1} \in \mathbb{R}^{d \times |A|}$, $\mathbf{V}_{2} \in \mathbb{R}^{d \times |B|}$, $\mathbf{U}_{3} \in \mathbb{R}^{d \times |A|}$ and $\mathbf{V}_{3} \in \mathbb{R}^{d \times |B|}$ follow from Remark~\ref{rmk:L0DB}. Further define
\begin{equation}
    \begin{matrix}
         \mathbf{B}_{A \setminus B} & \coloneqq & [ & \mathbf{X}_{1} & 0 & 0 & \mathbf{U}_{1} & 0 & ]\\
         \mathbf{B}_{A,B} & \coloneqq & [ & 0 & 0 & \mathbf{X}_{3} & \mathbf{U}_{3} & 0 & ]\\
         \mathbf{B}_{B \setminus A} & \coloneqq & [ & 0 & \mathbf{Y}_{2} & 0 & 0 & \mathbf{V}_{2} & ]\\
         \mathbf{B}_{B,A} & \coloneqq & [ & 0 & 0& \mathbf{Y}_{3} & 0 & \mathbf{V}_{3} & ]
    \end{matrix} \label{supp:eq:eq8}
\end{equation}
then $\mathbf{B}_A = \mathbf{B}_{A \setminus B} + \mathbf{B}_{A,B}$ and $\mathbf{B}_B = \mathbf{B}_{B \setminus A} + \mathbf{B}_{B,A}$. Note that $\mathbf{1}_{n+m}$ lies in the kernel of the four matrices defined above (see Remark~\ref{rmk:L0DB}) and, $\mathbf{B}_{A \setminus B}\mathbf{B}_{B\setminus A}^T = 0$, $\mathbf{B}_{A \setminus B}\mathbf{B}_{B,A}^T = 0$ and $\mathbf{B}_{B \setminus A}\mathbf{B}_{A,B}^T = 0$. Now, the structure of $\boldsymbol{\mathcal{L}}_{\Gamma}$ is as follows,
\begin{equation}
    \boldsymbol{\mathcal{L}}_{\Gamma} = \begin{bmatrix}
        \boldsymbol{\mathcal{D}}_{1} & & & -\boldsymbol{\mathcal{K}}_{1} & \\
        & \boldsymbol{\mathcal{D}}_{2} & &  & -\boldsymbol{\mathcal{K}}_{2}\\
        & &\boldsymbol{\mathcal{D}}_{3} + \boldsymbol{\mathcal{D}}_{3} & -\boldsymbol{\mathcal{K}}_{A_3} & -\boldsymbol{\mathcal{K}}_{B_3}\\
        -\boldsymbol{\mathcal{K}}_{1}^T & & -\boldsymbol{\mathcal{K}}_{A_3}^T & \boldsymbol{\mathcal{D}}_{A} & \\
        & -\boldsymbol{\mathcal{K}}_{2}^T & -\boldsymbol{\mathcal{K}}_{B_3}^T & & \boldsymbol{\mathcal{D}}_{B}
    \end{bmatrix}
\end{equation}
where $\boldsymbol{\mathcal{K}}_{1} \in \mathbb{R}^{n_1 \times |A|}$ is the adjacency between the first $n_1$ points and the views with indices in $A$, $\boldsymbol{\mathcal{K}}_{2} \in \mathbb{R}^{n_2 \times |B|}$ is the adjacency between the next $n_2$ points and the views with indices in $B$, and $\boldsymbol{\mathcal{K}}_{A_3} \in \mathbb{R}^{n_3 \times |A|}$ and $\boldsymbol{\mathcal{K}}_{B_3} \in \mathbb{R}^{n_3 \times |B|}$ are the adjacencies between the remaining $n_3$ points and the views with indices in $A$ and $B$ respectively. As for the remaining matrices, $\boldsymbol{\mathcal{D}}_{i} = \diag(\boldsymbol{\mathcal{K}}_{i}\mathbf{1}_{|A|})$ represents the degrees of the points in the bipartite adjacency $\boldsymbol{\mathcal{K}}_{i}$, $\boldsymbol{\mathcal{D}}_{A} = \diag(\boldsymbol{\mathcal{K}}_{1}^T\mathbf{1}_{n_1} + \boldsymbol{\mathcal{K}}_{A_3}^T\mathbf{1}_{n_3})$ represents the degree of the views i.e. the number of points contained in the views with indices in $A$, Similarly, $\boldsymbol{\mathcal{D}}_{j} = \diag(\boldsymbol{\mathcal{K}}_{j}\mathbf{1}_{|B|})$. and $\boldsymbol{\mathcal{D}}_{B} = \diag(\boldsymbol{\mathcal{K}}_{2}^T\mathbf{1}_{n_2} + \boldsymbol{\mathcal{K}}_{B_3}^T\mathbf{1}_{n_3})$. 

Since $\mathbf{S}$ is a perfect alignment, the local coordinates of a point due to the views are the same. Using the fact that $\mathbf{B}(\mathbf{S})_{A,B}$ represent the local coordinates of the $n_3$ points contained in views with indices in $A \cap B$ (see Definition~\ref{def:BSAcapB}), we obtain
\begin{align}
    \begin{matrix}
        \mathbf{X}_{3} &=& \mathbf{B}(\mathbf{S})_{A,B}\boldsymbol{\mathcal{D}}_{3}, & & \mathbf{Y}_{3} &=& \mathbf{B}(\mathbf{S})_{A,B}\boldsymbol{\mathcal{D}}_{3},\\
        \mathbf{U}_{3} &=& -\mathbf{B}(\mathbf{S})_{A,B}\boldsymbol{\mathcal{K}}_{A_3},& &
        \mathbf{V}_{3} &=& -\mathbf{B}(\mathbf{S})_{A,B}\boldsymbol{\mathcal{K}}_{B_3}.
    \end{matrix}\label{supp:eq:eq9}
\end{align} 
Similarly, it follows that $\mathbf{U}_1 = \mathbf{X}_1\boldsymbol{\mathcal{D}}_{1}^{-1}\boldsymbol{\mathcal{K}}_{1}$ and $\mathbf{V}_2 = \mathbf{Y}_2\boldsymbol{\mathcal{D}}_{2}^{-1}\boldsymbol{\mathcal{K}}_{2}$. Thus,
\begin{align}
    \begin{matrix}
         \mathbf{B}_{A \setminus B} & = & [ & \mathbf{X}_{1}\boldsymbol{\mathcal{D}}_{1}^{-1} & \mathbf{0}_{d \times n_2} & \mathbf{0}_{d \times n_3} & \mathbf{0}_{d \times |A|} & \mathbf{0}_{d \times |B|} & ]\boldsymbol{\mathcal{L}}_{\Gamma}\\
         \mathbf{B}_{B \setminus A} & = & [ & \mathbf{0}_{d \times n_1} & \mathbf{Y}_{2}\boldsymbol{\mathcal{D}}_{2}^{-1} & \mathbf{0}_{d \times n_3} & \mathbf{0}_{d \times |A|} & \mathbf{0}_{d \times |B|} & ]\boldsymbol{\mathcal{L}}_{\Gamma}
    \end{matrix}\label{supp:eq:eq14_} \text{ and }
\end{align}
% A final observation regarding the matrix $\mathbf{B}_{A \setminus B}$ and $\mathbf{B}_{B \setminus A}$ in Eq.~(\ref{supp:eq:eq8}) is that it is the analog of $\mathbf{S}^T\mathbf{B}$ in Eq.~(\ref{eq:opt_Z}) when the local coordinates of the last $n_2 + n_3$ points due to the views containing them are forced to be zeros, while the local coordinates of the first $n_1$ points are the same as above and in particular (just like Eq.~(\ref{supp:eq:eq9})) equal $\mathbf{X}_{1}\boldsymbol{\mathcal{D}}_{1}^{-1}$. Similar premise holds for $\mathbf{B}_{B \setminus A}$. Since, even after forcing certain local coordinates to be zeros, the views are perfectly aligned, using Eq.~(\ref{eq:opt_Z}) and Eq.~(\ref{eq:H}), it follows that
%Subsequently,
\begin{align}
    \begin{matrix}
         \mathbf{B}_{A \setminus B}\boldsymbol{\mathcal{L}}_{\Gamma}^\dagger & = & [ & \mathbf{X}_{1}\boldsymbol{\mathcal{D}}_{1}^{-1} & \mathbf{0}_{d \times n_2} & \mathbf{0}_{d \times n_3} & \mathbf{0}_{d \times |A|} & \mathbf{0}_{d \times |B|} & ] + \mathbf{t}_{A \setminus B}\mathbf{1}_{n+m}^T\\
         \mathbf{B}_{B \setminus A}\boldsymbol{\mathcal{L}}_{\Gamma}^\dagger & = & [ & \mathbf{0}_{d \times n_1} & \mathbf{Y}_{2}\boldsymbol{\mathcal{D}}_{2}^{-1} & \mathbf{0}_{d \times n_3} & \mathbf{0}_{d \times |A|} & \mathbf{0}_{d \times |B|} & ] +  \mathbf{t}_{B \setminus A}\mathbf{1}_{n+m}^T
         %(\mathbf{B}_{A,B}+\mathbf{B}_{B,A})\boldsymbol{\mathcal{L}}_{\Gamma}^\dagger & = & [ & 0 & 0 & \mathbf{B}(\mathbf{S})_{A,B} & 0 & 0 & ] & + & \mathbf{t}\mathbf{1}_{n+m}^T\\
    \end{matrix}\label{supp:eq:eq14}
\end{align}
for some translation vectors $\mathbf{t}_{A \setminus B}, \mathbf{t}_{B \setminus A} \in \mathbb{R}^d$. Since $\mathbf{1}_{n+m}$ lies in $\ker(\mathbf{B}_{A \setminus B})$ and $\ker(\mathbf{B}_{A \setminus B})$ (see the paragraph after Eq.~(\ref{supp:eq:eq8})), thus
\begin{equation}
    \mathbf{B}_{A \setminus B}\boldsymbol{\mathcal{L}}_{\Gamma}^\dagger \mathbf{B}_{B\setminus A}^T = \mathbf{B}_{A \setminus B}\boldsymbol{\mathcal{L}}_{\Gamma}^\dagger \mathbf{B}_{B,A}^T = \mathbf{B}_{B \setminus A}\boldsymbol{\mathcal{L}}_{\Gamma}^\dagger \mathbf{B}_{A,B}^T = 0. \label{supp:eq:eq11}
\end{equation}

% Finally, note that since $\overline{\mathbf{B}(\mathbf{S})}_{A, B}^T\boldsymbol{\Omega}_0 = 0$, we have $\mathbf{B}(\mathbf{S})_{A, B}^T\boldsymbol{\Omega}_0 = \mathbf{1}_{n_3}\mathbf{v}^T$ for some $\mathbf{v} \in \mathbb{R}^d$.
% Let $\boldsymbol{\mathcal{D}}_{A} \in \mathbb{R}^{|A| \times |A|}$ and $\boldsymbol{\mathcal{D}}_{B} \in \mathbb{R}^{|B| \times |B|}$ be the diagonal matrices whose $i$th element on the diagonal is the number of points in the $i$th and $(|A|+i)$th view, respectively. Then,
% \begin{align}
%     \boldsymbol{\Omega}_0^T\mathbf{X}_{3} &= \mathbf{v}\mathbf{1}_{n_3}^T\boldsymbol{\mathcal{D}}_{3}\\
%     \boldsymbol{\Omega}_0^T\mathbf{Y}_{3} &= \mathbf{v}\mathbf{1}_{n_3}^T\boldsymbol{\mathcal{D}}_{3}\\
%     \boldsymbol{\Omega}_0^T\mathbf{U}_{3} &= -\mathbf{v}\mathbf{1}_{|A|}^T\boldsymbol{\mathcal{D}}_{A}\\
%     -\boldsymbol{\Omega}_0^T\mathbf{V}_{3} &= \mathbf{v}\mathbf{1}_{|B|}^T\boldsymbol{\mathcal{D}}_{B}
% \end{align}
% and
% \begin{align}
%     \begin{matrix}
%         \boldsymbol{\Omega}_0^T\mathbf{B}_{A,B} & = & [ & 0 & 0 & \mathbf{v}\mathbf{1}_{n_3}^T\boldsymbol{\mathcal{D}}_{3} & -\mathbf{v}\mathbf{1}_{|A|}^T\boldsymbol{\mathcal{D}}_{A} & 0 & ]\\
%         \boldsymbol{\Omega}_0^T\mathbf{B}_{B,A} & = & [ & 0 & 0 & \mathbf{v}\mathbf{1}_{n_3}^T\boldsymbol{\mathcal{D}}_{3} &  0 & -\mathbf{v}\mathbf{1}_{|B|}^T\boldsymbol{\mathcal{D}}_{B} & ]
%     \end{matrix}
% \end{align}
% Using the above equation and the fact that $\mathbf{1}_n \in \ker(\boldsymbol{\mathcal{L}}_{\Gamma}^\dagger)$ and $\boldsymbol{\mathcal{L}}_{\Gamma}^\dagger \succeq 0$, we obtain
% \begin{align}
%     \Tr(\boldsymbol{\Omega}_0^T \mathbf{B}_A\boldsymbol{\mathcal{L}}_{\Gamma}^\dagger \mathbf{B}_B^T \boldsymbol{\Omega}_0) &= \Tr(\boldsymbol{\Omega}_0^T (\mathbf{B}_{A \setminus B}+\mathbf{B}_{A,B})\boldsymbol{\mathcal{L}}_{\Gamma}^\dagger (\mathbf{B}_{B\setminus A} + \mathbf{B}_{B,A})^T \boldsymbol{\Omega}_0)\\
%     &= \Tr(\boldsymbol{\Omega}_0^T \mathbf{B}_{A,B}\boldsymbol{\mathcal{L}}_{\Gamma}^\dagger\mathbf{B}_{B,A}^T \boldsymbol{\Omega}_0)\\
%     &\leq \Tr(\boldsymbol{\Omega}_0^T (\mathbf{B}_{A,B} + \mathbf{B}_{B,A})\boldsymbol{\mathcal{L}}_{\Gamma}^\dagger(\mathbf{B}_{A,B} + \mathbf{B}_{B,A})^T \boldsymbol{\Omega}_0)\\
%     &= \Tr(\begin{bmatrix} 0 & 0 & \boldsymbol{\Omega}_0^T\mathbf{B}(\mathbf{S})_{A,B} & 0 & 0\end{bmatrix}(\mathbf{B}_{A,B} + \mathbf{B}_{B,A})^T \boldsymbol{\Omega}_0)\\
%     &= 0
% \end{align}
% where the last equation follows from the fact that $\mathbf{B}(\mathbf{S})_{A,B}^T\boldsymbol{\Omega}_0 = 0$. Combining the above equation with Eq.~(\ref{eq:TrOmega0LOmega0_lb}), we conclude that $\boldsymbol{\Omega}$ is a non-trivial certificate of $\mathbf{L}(\mathbf{S})$ and thus $\pi(\mathbf{S})$ is degenerate.

Since $\mathbf{B}(\mathbf{S})_{A,B}^T\boldsymbol{\Omega}_0 = 0$ (by assumption), combining with Eq.~(\ref{supp:eq:eq9}, \ref{supp:eq:eq8}) yields $\mathbf{B}_{A,B}^T\boldsymbol{\Omega}_0 = 0$ and $ \mathbf{B}_{B,A}^T\boldsymbol{\Omega}_0 = 0$. Substituting the above and Eq.~(\ref{supp:eq:eq11}) into Eq.~(\ref{supp:eq:Omega0TLSOmega0}), we obtain 
$$\Tr(\boldsymbol{\Omega}_0^T \mathbf{B}_A\boldsymbol{\mathcal{L}}_{\Gamma}^\dagger \mathbf{B}_B^T \boldsymbol{\Omega}_0) = \Tr(\boldsymbol{\Omega}_0^T (\mathbf{B}_{A \setminus B}+\mathbf{B}_{A,B})\boldsymbol{\mathcal{L}}_{\Gamma}^\dagger (\mathbf{B}_{B\setminus A} + \mathbf{B}_{B,A})^T \boldsymbol{\Omega}_0)
    = \Tr(\boldsymbol{\Omega}_0^T \mathbf{B}_{A,B}\boldsymbol{\mathcal{L}}_{\Gamma}^\dagger\mathbf{B}_{B,A}^T \boldsymbol{\Omega}_0) = 0.$$
We conclude that $\boldsymbol{\Omega}$ is a non-trivial certificate of $\mathbf{L}(\mathbf{S})$ and thus $\pi(\mathbf{S})$ is degenerate.

\proofof{Lemma~\ref{lem:subproblem_cert}} 
WLOG, let the $m$th vertex be removed. Let $\Gamma$ be the bipartite graph representing the correspondence between $m$ views and $n$ vertices, as described in Section~\ref{sec:setup}. Let $\Gamma_{-}$ be the bipartite graph obtained after the removal of the vertices representing the $m$th view and the points which lie exclusively in it. Let $\mathbf{D} \in \mathbb{R}^{md \times md}$, $\mathbf{B} \in \mathbb{R}^{md \times (n+m)}$, $\mathbf{D}_{-} \in \mathbb{R}^{(m-1)d\times (m-1)d}$ and $\mathbf{B}_{-} \in \mathbb{R}^{(m-1)d \times (n_1+n_2+m-1)}$ be the matrices defined in Remark~\ref{rmk:L0DB} for graphs $\Gamma$ and $\Gamma_{-}$. Also, let
\begin{itemize}[leftmargin=*]
    \item $\boldsymbol{\mathcal{K}}_1 \in \mathbb{R}^{n_1 \times (m-1)}$ is the bipartite adjacency matrix between the first $m-1$ views and the $n_1$ points which lie exclusively in them. Note that the adjacency between such points and the $m$th view is $\mathbf{0}_{n_1}$.
    \item $\boldsymbol{\mathcal{K}}_2 \in \mathbb{R}^{n_2 \times (m-1)}$ is the bipartite adjacency between the first $m-1$ views and the $n_2$ points which lie on the overlap of the $m$th view and the union of the first $m-1$ views. Note that the adjacency between such points and the $m$th view is $\mathbf{1}_{n_2}$. Also note that since $\Gamma$ is connected by Assumption~\ref{assump:connected_gamma}, $n_2 > 0$.
    \item the fifth and the third column in $\boldsymbol{\mathcal{L}}_{\Gamma}$ correspond to the  $m$th view and the $n_3$ points that lie exclusively in it, respectively. The adjacency between such points and the first $m-1$ views is $\mathbf{0}_{n_3 \times (m-1)}$, and that with the $m$th view is $\mathbf{1}_{n_3}$.
    \item $\boldsymbol{\mathcal{D}}_1 = \diag (\boldsymbol{\mathcal{K}}_1\mathbf{1}_{m-1})$, $\boldsymbol{\mathcal{D}}_2 = \diag (\boldsymbol{\mathcal{K}}_2\mathbf{1}_{m-1})$ and $\overline{\boldsymbol{\mathcal{D}}} = \diag (\boldsymbol{\mathcal{K}}_1^T\mathbf{1}_{n_1} + \boldsymbol{\mathcal{K}}_2^T\mathbf{1}_{n_2})$.
\end{itemize}
Then the structure of the combinatorial Laplacian of $\Gamma$ and $\Gamma_{-}$ are
\begin{equation}
    \boldsymbol{\mathcal{L}}_{\Gamma} = \begin{bmatrix}
        \boldsymbol{\mathcal{D}}_1 &  &  & -\boldsymbol{\mathcal{K}}_1 & \mathbf{0}_{n_1}\\
         & \boldsymbol{\mathcal{D}}_2 + \mathbf{I}_{n_2} &  & -\boldsymbol{\mathcal{K}}_2 & -\mathbf{1}_{n_2}\\
         &  & \mathbf{I}_{n_3} & \mathbf{0}_{n_3 \times (m-1)} & -\mathbf{1}_{n_3}\\
        -\boldsymbol{\mathcal{K}}_1^T & -\boldsymbol{\mathcal{K}}_2^T & \mathbf{0}_{n_3}^T & \overline{\boldsymbol{\mathcal{D}}} & \mathbf{0}_{m-1}\\
        \mathbf{0}_{n_1}^T & -\mathbf{1}_{n_2}^T & -\mathbf{1}_{n_3}^T & \mathbf{0}_{m-1}^T & n_2+n_3
    \end{bmatrix}
\end{equation}
and
$\boldsymbol{\mathcal{L}}_{\Gamma_{-}} = \begin{bsmallmatrix}
    \boldsymbol{\mathcal{D}}_1 &  & -\boldsymbol{\mathcal{K}}_1\\
     & \boldsymbol{\mathcal{D}}_2 & -\boldsymbol{\mathcal{K}}_2\\
    -\boldsymbol{\mathcal{K}}_1^T & -\boldsymbol{\mathcal{K}}_2^T & \overline{\boldsymbol{\mathcal{D}}}
    \end{bsmallmatrix}$.
Using a permutation matrix
% \begin{align}
%     \boldsymbol{\mathcal{P}}_{0} &= \begin{bmatrix}
%         \mathbf{I}_{n_1} &  &  &  & \\
%         & \mathbf{I}_{n_2} &  &  &  \\
%         & & & \mathbf{I}_{n_3}& \\
%         & & \mathbf{I}_{m-1}& & \\
%         & & & & 1
%     \end{bmatrix}
% \end{align}
%\begin{align}
$\boldsymbol{\mathcal{P}}_{0} = \begin{bsmallmatrix}
        \mathbf{I}_{n_1+n_2} &  &  & \\
        & & \mathbf{I}_{n_3}& \\
        & \mathbf{I}_{m-1}& & \\
        & & & 1
    \end{bsmallmatrix}$
%\end{align}
and a diagonal matrix $\boldsymbol{\mathcal{D}}_0 = \diag ((\mathbf{0}_{n_1},\mathbf{1}_{n_2},\mathbf{0}_{m-1}))$, we obtain
% \begin{align}
%     \boldsymbol{\mathcal{P}}_{0}\boldsymbol{\mathcal{L}}_{\Gamma}\boldsymbol{\mathcal{P}}_{0}^T &= \begin{bmatrix}\mathbf{A}_{11}&\mathbf{A}_{12}\\\mathbf{A}_{21}&\mathbf{A}_{22}\end{bmatrix} = \begin{bmatrix}
%         &&&\mathbf{0}_{n_1 \times n_3} & \mathbf{0}_{n_1}\\
%         & \boldsymbol{\mathcal{L}}_{\Gamma_{-}} + \boldsymbol{\mathcal{D}}_0 & & \mathbf{0}_{n_2 \times n_3} & -\mathbf{1}_{n_2}\\
%         &&& \mathbf{0}_{(m-1) \times n_3} & \mathbf{0}_{m-1}\\
%         \mathbf{0}_{n_1 \times n_3}^T & \mathbf{0}_{n_2 \times n_3}^T & \mathbf{0}_{(m-1) \times n_3}^T & \mathbf{I}_{n_3} & -\mathbf{1}_{n_3}\\
%         \mathbf{0}_{n_1}^T & -\mathbf{1}_{n_2}^T & \mathbf{0}_{m-1}^T & -\mathbf{1}_{n_3}^T & n_2+n_3
%     \end{bmatrix} \label{supp:eq:eq15}
% \end{align}
\begin{equation}
    \boldsymbol{\mathcal{P}}_{0}\boldsymbol{\mathcal{L}}_{\Gamma}\boldsymbol{\mathcal{P}}_{0}^T = \begin{bmatrix}\mathbf{A}_{11}&\mathbf{A}_{12}\\\mathbf{A}_{21}&\mathbf{A}_{22}\end{bmatrix} = \begin{bmatrix}
        &&& \mathbf{0}_{n_1} & \mathbf{0}_{n_1}\\
        & \boldsymbol{\mathcal{L}}_{\Gamma_{-}} + \boldsymbol{\mathcal{D}}_0 & & \mathbf{0}_{n_2} & -\mathbf{1}_{n_2}\\
        &&& \mathbf{0}_{m-1} & \mathbf{0}_{m-1}\\
        \mathbf{0}_{n_1}^T & \mathbf{0}_{n_2}^T & \mathbf{0}_{m-1}^T & \mathbf{I}_{n_3} & -\mathbf{1}_{n_3}\\
        \mathbf{0}_{n_1}^T & -\mathbf{1}_{n_2}^T & \mathbf{0}_{m-1}^T & -\mathbf{1}_{n_3}^T & n_2+n_3
    \end{bmatrix} \label{supp:eq:eq15}
\end{equation}

The rest is divided into three parts. First, we derive the pseudoinverse of the above block matrix using \citeb[Section 3.6.2]{gentle2007matrix}. Then we show that $\mathbf{S}_{-} \coloneqq \mathbf{S}_{-m}$ is a perfect alignment of the $m-1$ views and finally we show that $[\boldsymbol{\Omega}_i]_{1}^{m-1}$ is a certificate of $\mathbf{L}_{-}(\mathbf{S}_{-}) \coloneqq \mathbf{L}_{-m}(\mathbf{S}_{-m})$ when $[\boldsymbol{\Omega}_i]_{1}^{m}$ is a certificate of $\mathbf{L}(\mathbf{S})$.

\noindent \underline{\textbf{Part 1}}. Here we derive the pseudoinverse of the matrix in Eq.~(\ref{supp:eq:eq15}). First, we note
\begin{prop}
\label{supp:prop:LplusD0}
$\boldsymbol{\mathcal{L}}_{\Gamma_{-}} + \boldsymbol{\mathcal{D}}_0 \succ 0$.
\end{prop}
\textit{Proof}. Since $\boldsymbol{\mathcal{L}}_{\Gamma_{-}} \succeq 0$ and $\boldsymbol{\mathcal{D}}_0 \succeq 0$, it suffices to show that $\ker (\boldsymbol{\mathcal{L}}_{\Gamma_{-}}) \cap \ker (\boldsymbol{\mathcal{D}}_0) = \{0\}$. Recall that the $m$th view contains $n_2 + n_3$ points where $n_2>0$ points lie on the overlap of $m$th view and the union of first $m-1$ views, and $n_3$ points lie exclusively in the $m$th view. Removal of the $m$th view and the $n_3$ points that lie exclusively in it may disconnect $\Gamma$ to produce $\Gamma_{-}$ with at most $n_2$ connected components. The vectors $\mathbf{u}_i$ with ones at the indices of the vertices in the $i$th component and zeros elsewhere, form an orthogonal basis of $\ker(\boldsymbol{\mathcal{L}}_{\Gamma_{-}})$. Since there exists at least one $k \in [n_1+1, n_1+n_2]$ with $\mathbf{u}_i(k) = 1$, thus $\mathbf{u}_i^T\boldsymbol{\mathcal{D}}_0\mathbf{u}_i > 0$. Also, for $i \neq j$, $\mathbf{u}_i^T\boldsymbol{\mathcal{D}}_0\mathbf{u}_j = 0$. The result follows.~$\blacksquare$

Since $\boldsymbol{\mathcal{L}}_{\Gamma_{-}} + \boldsymbol{\mathcal{D}}_0 \succ 0$, thus $(\boldsymbol{\mathcal{L}}_{\Gamma_{-}}+\boldsymbol{\mathcal{D}}_0)^\dagger = (\boldsymbol{\mathcal{L}}_{\Gamma_{-}}+\boldsymbol{\mathcal{D}}_0)^{-1}$ and
\begin{equation}
    (\boldsymbol{\mathcal{L}}_{\Gamma_{-}}+\boldsymbol{\mathcal{D}}_0)  \begin{bmatrix}\mathbf{1}_{n_1}\\ \mathbf{1}_{n_2} \\ \mathbf{1}_{m-1}\end{bmatrix} =  \begin{bmatrix}\mathbf{0}_{n_1}\\ \mathbf{1}_{n_2} \\ \mathbf{0}_{m-1}\end{bmatrix} \implies (\boldsymbol{\mathcal{L}}_{\Gamma_{-}}+\boldsymbol{\mathcal{D}}_0)^\dagger \begin{bmatrix}\mathbf{0}_{n_1}\\ \mathbf{1}_{n_2} \\ \mathbf{0}_{m-1}\end{bmatrix} = \begin{bmatrix}\mathbf{1}_{n_1}\\ \mathbf{1}_{n_2} \\ \mathbf{1}_{m-1}\end{bmatrix}. \label{supp:eq:L_Gamma_minus__plus_D_0_1s}
\end{equation}
Using the above equation, the matrix $\mathbf{Z} \coloneqq [(\boldsymbol{\mathcal{P}}_{0}\boldsymbol{\mathcal{L}}_{\Gamma}\boldsymbol{\mathcal{P}}_{0}^T)^\dagger]_{22} = \mathbf{A}_{22} - \mathbf{A}_{21}\mathbf{A}_{11}^\dagger \mathbf{A}_{12}$ and its pseudoinverse are, $\mathbf{Z} = \begin{bmatrix}
    \mathbf{I}_{n_3} & -\mathbf{1}_{n_3}\\
    -\mathbf{1}_{n_3}^T & n_3
\end{bmatrix}$ and $\mathbf{Z}^\dagger = \begin{bmatrix}
    \mathbf{I}_{n_3}  & \mathbf{0}_{n_3}\\
    \mathbf{0}_{n_3}^T & 0
\end{bmatrix}$. Next, we have 
$$[(\boldsymbol{\mathcal{P}}_{0}\boldsymbol{\mathcal{L}}_{\Gamma}\boldsymbol{\mathcal{P}}_{0}^T)^\dagger]_{11} = (\boldsymbol{\mathcal{L}}_{\Gamma_{-}}+\boldsymbol{\mathcal{D}}_0)^\dagger + ((\boldsymbol{\mathcal{L}}_{\Gamma_{-}}+\boldsymbol{\mathcal{D}}_0)^\dagger \mathbf{A}_{12})\mathbf{Z}^{\dagger}(\mathbf{A}_{21}(\boldsymbol{\mathcal{L}}_{\Gamma_{-}}+\boldsymbol{\mathcal{D}}_0)^\dagger).$$
Using Eq.~(\ref{supp:eq:eq15}, \ref{supp:eq:L_Gamma_minus__plus_D_0_1s}),  we obtain,
\begin{align}
[(\boldsymbol{\mathcal{P}}_{0}\boldsymbol{\mathcal{L}}_{\Gamma}\boldsymbol{\mathcal{P}}_{0}^T)^\dagger]_{11} &= (\boldsymbol{\mathcal{L}}_{\Gamma_{-}}+\boldsymbol{\mathcal{D}}_0)^\dagger. \label{supp:eq:pinvA11}\\
[(\boldsymbol{\mathcal{P}}_{0}\boldsymbol{\mathcal{L}}_{\Gamma}\boldsymbol{\mathcal{P}}_{0}^T)^\dagger]_{12} &= -(\mathbf{A}_{11}^\dagger \mathbf{A}_{12}) \mathbf{Z}^\dagger = 0\\
[(\boldsymbol{\mathcal{P}}_{0}\boldsymbol{\mathcal{L}}_{\Gamma}\boldsymbol{\mathcal{P}}_{0}^T)^\dagger]_{21} &= 0.
\end{align}
Thus,
$$(\boldsymbol{\mathcal{P}}_{0}\boldsymbol{\mathcal{L}}_{\Gamma}\boldsymbol{\mathcal{P}}_{0}^T)^\dagger = \blockdiag((\boldsymbol{\mathcal{L}}_{\Gamma_{-}}+\boldsymbol{\mathcal{D}}_0)^\dagger, \mathbf{I}_{n_3}, 0).$$
% \begin{align}
%     \mathbf{Z} &= \begin{bmatrix}
%         \mathbf{I}_{n_3} & -\mathbf{1}_{n_3}\\
%         -\mathbf{1}_{n_3}^T & n_3
%     \end{bmatrix} \implies  \mathbf{Z}^\dagger = \begin{bmatrix}
%         \mathbf{I}_{n_3}  & \mathbf{0}_{n_3}\\
%         \mathbf{0}_{n_3}^T & 0
%     \end{bmatrix}.
% \end{align}
% \begin{align}
% &[(\boldsymbol{\mathcal{P}}_{0}\boldsymbol{\mathcal{L}}_{\Gamma}\boldsymbol{\mathcal{P}}_{0}^T)^\dagger]_{11} = (\boldsymbol{\mathcal{L}}_{\Gamma_{-}}+\boldsymbol{\mathcal{D}}_0)^\dagger + ((\boldsymbol{\mathcal{L}}_{\Gamma_{-}}+\boldsymbol{\mathcal{D}}_0)^\dagger \mathbf{A}_{12})\mathbf{Z}^{\dagger}(\mathbf{A}_{21}(\boldsymbol{\mathcal{L}}_{\Gamma_{-}}+\boldsymbol{\mathcal{D}}_0)^\dagger)\\
%     &= (\boldsymbol{\mathcal{L}}_{\Gamma_{-}}+\boldsymbol{\mathcal{D}}_0)^\dagger + \begin{bsmallmatrix}\mathbf{0}_{n_1 \times n_3} & \mathbf{1}_{n_1}\\ \mathbf{0}_{n_2 \times n_3} & \mathbf{1}_{n_2} \\ \mathbf{0}_{(m-1) \times n_3} & \mathbf{1}_{m-1}\end{bsmallmatrix}\begin{bsmallmatrix}
%         \mathbf{I}_{n_3}  & \mathbf{0}_{n_3}\\
%         \mathbf{0}_{n_3}^T & 0
%     \end{bsmallmatrix} \begin{bsmallmatrix}\mathbf{0}_{n_1 \times n_3}^T & \mathbf{0}_{n_2 \times n_3}^T & \mathbf{0}_{(m-1) \times n_3}^T \\ \mathbf{1}_{n_1}^T & \mathbf{1}_{n_2}^T & \mathbf{1}_{m-1}\end{bsmallmatrix}\\
%     &= (\boldsymbol{\mathcal{L}}_{\Gamma_{-}}+\boldsymbol{\mathcal{D}}_0)^\dagger \label{supp:eq:pinvA11}
% \end{align}
% \begin{align}(\boldsymbol{\mathcal{P}}_{0}\boldsymbol{\mathcal{L}}_{\Gamma}\boldsymbol{\mathcal{P}}_{0}^T)^\dagger &= 
%     \begin{bmatrix}
%        \blockdiag((\boldsymbol{\mathcal{L}}_{\Gamma_{-}}+\boldsymbol{\mathcal{D}}_0)^\dagger  & & \\
%         & \mathbf{I}_{n_3}  & \\
%         & & 0
%     \end{bmatrix}.
% \end{align}

\noindent \underline{\textbf{Part 2}}. Now, let $\mathbf{S} = [\mathbf{S}_i]_1^{m}$ and $\mathbf{S}_{-} = [\mathbf{S}_i]_1^{m-1}$. Intuitively, it should be clear that $\mathbf{S}_{-}$ is a perfect alignment for the $m-1$ views. Since $\mathbf{S}$ is a perfect alignment, the alignment error (see Eq.~(\ref{eq:GPOP}))
$$\Tr(\mathbf{S}^T(\mathbf{D}-\mathbf{B}\boldsymbol{\mathcal{L}}_{\Gamma}^\dagger\mathbf{B}^T)\mathbf{S}) = 0.$$ We show that the error after the removal of the $m$th view is still zero i.e. 
$$\Tr(\mathbf{S}_{-}^T(\mathbf{D}_{-}-\mathbf{B}_{-}\boldsymbol{\mathcal{L}}_{\Gamma_{-}}\mathbf{B}_{-}^T)\mathbf{S}_{-}) = 0.$$
Let $\mathbf{B}^*_{1} \in \mathbb{R}^{d \times n_1}$, $\mathbf{B}^*_{2} \in \mathbb{R}^{d \times n_2}$ and $\mathbf{B}^*_{3} \in \mathbb{R}^{d \times n_3}$ contain the coordinates (after alignment with $\mathbf{S}$) of the $n_1$ points that lie exclusively in the first $m-1$ views, of the $n_2$ points that lie on the overlap of the $m$th view with the remaining views, and of the $n_3$ points that lie exclusively in the $m$th view, respectively. Then it suffices to show
\begin{prop}
\label{supp:prop:subproblem_perf_alignment}
(i) $\mathbf{S}^T\mathbf{B}\boldsymbol{\mathcal{L}}_{\Gamma}^\dagger\mathbf{B}^T\mathbf{S} =  \mathbf{S}_{-}^T\mathbf{B}_{-}\boldsymbol{\mathcal{L}}_{\Gamma_{-}}^\dagger \mathbf{B}_{-}^T\mathbf{S}_{-} + \mathbf{B}^*_{2}\mathbf{B}^{*^T}_{2} + \mathbf{B}^*_{3}\mathbf{B}^{*^T}_{3}$ and (ii) $\mathbf{S}^T\mathbf{D}\mathbf{S} = \mathbf{S}_{-}^T\mathbf{D}_{-}\mathbf{S}_{-} + \mathbf{B}^*_{2}\mathbf{B}^{*^T}_{2} + \mathbf{B}^*_{3}\mathbf{B}^{*^T}_{3}$. By taking the trace of the difference of these equations, the main result follows.
\end{prop}
\textit{Proof}. The second equation follows from Eq.~(\ref{eq:D}), Remark~\ref{rmk:L0DB} and the fact that, since the $m$ views are perfectly aligned, the local coordinates of the points are the same as those in the matrices $\mathbf{B}^*_{1}$,  $\mathbf{B}^*_{2}$ and  $\mathbf{B}^*_{3}$. We proceed to prove the first equation.

Since $\mathbf{\mathcal{P}}_0$ is a permutation matrix, $\mathbf{S}^T\mathbf{B}\boldsymbol{\mathcal{L}}_{\Gamma}^\dagger\mathbf{B}^T\mathbf{S} = (\mathbf{S}^T\mathbf{B}\boldsymbol{\mathcal{L}}_{\Gamma}^\dagger\boldsymbol{\mathcal{P}}_{0}^T)(\boldsymbol{\mathcal{P}}_{0}\mathbf{B}^T\mathbf{S})$. Then, using the same idea as in Eq.~(\ref{supp:eq:eq9}, \ref{supp:eq:eq14_}, \ref{supp:eq:eq14}) in the proof of Theorem~\ref{thm:nec_cond_loc_rigid_of_views}, we obtain
\begin{align}
    \mathbf{S}^T\mathbf{B} &= \begin{bmatrix}
        \mathbf{B}^*_{1}\boldsymbol{\mathcal{D}}_1 & \mathbf{B}^*_{2}(\boldsymbol{\mathcal{D}}_2 + \mathbf{I}_{n_2}) & \mathbf{B}^*_{3} & -(\mathbf{B}^*_{1}\boldsymbol{\mathcal{K}}_1+\mathbf{B}^*_{2}\boldsymbol{\mathcal{K}}_2) & -(\mathbf{B}^*_{2}\mathbf{1}_{n_2}+\mathbf{B}^*_{3}\mathbf{1}_{n_3})
    \end{bmatrix}\\
    &= \begin{bmatrix}
        \mathbf{B}^*_{1} & \mathbf{B}^*_{2} & \mathbf{B}^*_{3} & \mathbf{0}_{d \times (m-1)} & \mathbf{0}_{d}
    \end{bmatrix}\boldsymbol{\mathcal{L}}_{\Gamma}\label{supp:eq:STB}
\end{align}
and thus,
\begin{equation}
    \mathbf{S}^T\mathbf{B}\boldsymbol{\mathcal{L}}_{\Gamma}^\dagger = \begin{bmatrix}
        \mathbf{B}^*_{1} & \mathbf{B}^*_{2} & \mathbf{B}^*_{3} & \mathbf{0}_{d \times (m-1)} & \mathbf{0}_{d}
    \end{bmatrix} + \mathbf{t}\mathbf{1}_{n+m}^T \label{supp:eq:STBL_GammaBTS}
\end{equation}
for some translation vector $\mathbf{t} \in \mathbb{R}^d$. Similarly,
\begin{equation}
    \mathbf{S}_{-}^T\mathbf{B}_{-} = \begin{bmatrix}
        \mathbf{B}^*_{1}\boldsymbol{\mathcal{D}}_1 & \mathbf{B}^*_{2} \boldsymbol{\mathcal{D}}_2 & -(\mathbf{B}^*_{1}\boldsymbol{\mathcal{K}}_1+\mathbf{B}^*_{2}\boldsymbol{\mathcal{K}}_2)
    \end{bmatrix} = \begin{bmatrix}
        \mathbf{B}^*_{1} & \mathbf{B}^*_{2} & \mathbf{0}_{d \times (m-1)}
    \end{bmatrix}\boldsymbol{\mathcal{L}}_{\Gamma_{-}} \label{supp:eq:SmTBm1}
\end{equation}
and thus 
$$\mathbf{S}_{-}^T\mathbf{B}_{-}\boldsymbol{\mathcal{L}}_{\Gamma_{-}}^\dagger  = \begin{bmatrix}
        \mathbf{B}^*_{1} & \mathbf{B}^*_{2} & \mathbf{0}_{d \times (m-1)}
    \end{bmatrix} + \mathbf{t}_{-}\mathbf{v}_{-}^T$$    
for some translation vector $\mathbf{t}_{-} \in \mathbb{R}^d$ and $\mathbf{v}_{-} \in \ker(\boldsymbol{\mathcal{L}}_{\Gamma_{-}})$. From Proposition~\ref{prop:kerB}, $\ker(\boldsymbol{\mathcal{L}}_{\Gamma_{-}}) \subseteq \ker(\mathbf{B}_{-})$, therefore,
\begin{equation}
    \mathbf{S}_{-}^T\mathbf{B}_{-}\boldsymbol{\mathcal{L}}_{\Gamma_{-}}^\dagger \mathbf{B}_{-}^T\mathbf{S}_{-}  = \begin{bmatrix}
        \mathbf{B}^*_{1} & \mathbf{B}^*_{2} & \mathbf{0}_{d \times (m-1)}
    \end{bmatrix}\mathbf{B}_{-}^T\mathbf{S}_{-}. \label{supp:eq:eq20}
\end{equation}
Now, combining Eq.~(\ref{supp:eq:STB}) and Eq.~(\ref{supp:eq:SmTBm1}) we can write 
\begin{align}
    \mathbf{S}^T\mathbf{B}\boldsymbol{\mathcal{P}}_{0}^T &= \begin{bmatrix}
    \mathbf{B}^*_{1}\boldsymbol{\mathcal{D}}_1 & \mathbf{B}^*_{2}(\boldsymbol{\mathcal{D}}_2 + \mathbf{I}_{n_2}) & -(\mathbf{B}^*_{1}\boldsymbol{\mathcal{K}}_1+\mathbf{B}^*_{2}\boldsymbol{\mathcal{K}}_2)  & \mathbf{B}^*_{3} & -(\mathbf{B}^*_{2}\mathbf{1}_{n_2}+\mathbf{B}^*_{3}\mathbf{1}_{n_3})
\end{bmatrix}\\
    &= \begin{bmatrix}
    \mathbf{S}_{-}^T\mathbf{B}_{-} & \mathbf{0}_{d \times n_3} & \mathbf{0}_{d}
\end{bmatrix} + \begin{bmatrix}
    \mathbf{0}_{d \times n_1} & \mathbf{B}^*_{2} &  \mathbf{0}_{d \times (m-1)} &  \mathbf{B}^*_{3} & -(\mathbf{B}^*_{2}\mathbf{1}_{n_2}+\mathbf{B}^*_{3}\mathbf{1}_{n_3})
\end{bmatrix}.
\end{align}
% \begin{align}
%     &\mathbf{S}^T\mathbf{B}\boldsymbol{\mathcal{P}}_{0}^T = \begin{bmatrix}
%         \mathbf{B}^*_{1}\boldsymbol{\mathcal{D}}_1 & \mathbf{B}^*_{2}(\boldsymbol{\mathcal{D}}_2 + \mathbf{I}_{n_2}) & -(\mathbf{B}^*_{1}\boldsymbol{\mathcal{K}}_1+\mathbf{B}^*_{2}\boldsymbol{\mathcal{K}}_2)  & \mathbf{B}^*_{3} & -(\mathbf{B}^*_{2}\mathbf{1}_{n_2}+\mathbf{B}^*_{3}\mathbf{1}_{n_3})
%     \end{bmatrix}\\
%     &= \begin{bmatrix}
%         \mathbf{S}_{-}^T\mathbf{B}_{-} & \mathbf{0}_{d \times n_3} & \mathbf{0}_{d}
%     \end{bmatrix} + \begin{bmatrix}
%         \mathbf{0}_{d \times n_1} & \mathbf{B}^*_{2} &  \mathbf{0}_{d \times (m-1)} &  \mathbf{B}^*_{3} & -(\mathbf{B}^*_{2}\mathbf{1}_{n_2}+\mathbf{B}^*_{3}\mathbf{1}_{n_3})
%     \end{bmatrix}.
% \end{align}
Finally, due to Eq.~(\ref{supp:eq:STBL_GammaBTS}) and $\mathbf{1}_{n+m} \in \ker(\mathbf{S}^T\mathbf{B}\boldsymbol{\mathcal{P}}_{0}^T)$, the above equation and Eq.~(\ref{supp:eq:eq20}),
\begin{align}
    \mathbf{S}^T\mathbf{B}\boldsymbol{\mathcal{L}}_{\Gamma}^\dagger\mathbf{B}^T\mathbf{S} &= (\mathbf{S}^T\mathbf{B}\boldsymbol{\mathcal{L}}_{\Gamma}^\dagger\boldsymbol{\mathcal{P}}_{0}^T)(\boldsymbol{\mathcal{P}}_{0}\mathbf{B}^T\mathbf{S})\\
    &= \left(\begin{bmatrix}
    \mathbf{B}^*_{1} & \mathbf{B}^*_{2} &  \mathbf{0}_{d \times (m-1)} & \mathbf{B}^*_{3} & \mathbf{0}_{d}
\end{bmatrix} + \mathbf{t}\mathbf{1}_{n+m}^T\right)(\boldsymbol{\mathcal{P}}_{0}\mathbf{B}^T\mathbf{S})\\
&=\begin{bmatrix}
        \mathbf{B}^*_{1} & \mathbf{B}^*_{2} &  \mathbf{0}_{d \times (m-1)} & \mathbf{B}^*_{3} & \mathbf{0}_{d}
    \end{bmatrix}\begin{bmatrix}
        \mathbf{S}_{-}^T\mathbf{B}_{-} & \mathbf{0}_{d \times n_3} & \mathbf{0}_{d}
    \end{bmatrix}^T + \mathbf{B}^*_{2}\mathbf{B}^{*^T}_{2} + \mathbf{B}^*_{3}\mathbf{B}^{*^T}_{3}\\
    &=\mathbf{S}_{-}^T\mathbf{B}_{-}\boldsymbol{\mathcal{L}}_{\Gamma_{-}}^\dagger \mathbf{B}_{-}^T\mathbf{S}_{-} + \mathbf{B}^*_{2}\mathbf{B}^{*^T}_{2} + \mathbf{B}^*_{3}\mathbf{B}^{*^T}_{3},
\end{align}
\hfill $\blacksquare$

\noindent \underline{\textbf{Part 3}}. Now let $\mathbf{L}(\mathbf{S})$ and $\mathbf{L}_{-}(\mathbf{S}_{-})$ be the matrices, as described in Eq.~(\ref{eq:L_of_S}) for the two graphs $\Gamma$ and $\Gamma_{-}$ and the corresponding views. Let $\boldsymbol{\Omega} = [\boldsymbol{\Omega}_i]_1^m$ be a certificate of $\mathbf{L}(\mathbf{S})$. By Remark~\ref{rmk:C_hat_L_structure}, $\boldsymbol{\Omega}^{'} = \boldsymbol{\Omega} - [\boldsymbol{\Omega}_m]_1^m$ is also a certificate of $\mathbf{L}(\mathbf{S})$ and in particular $\boldsymbol{\Omega}^{'}_m = 0$. Define $\boldsymbol{\Omega}_{-} = [\boldsymbol{\Omega}^{'}_i]_1^{m-1}$. We are going to show that $\Tr(\boldsymbol{\Omega}_{-}^T\mathbf{L}_{-}(\mathbf{S}_{-})\boldsymbol{\Omega}_{-}) = 0$ i.e. $\boldsymbol{\Omega}_{-}$ is a certificate of $\mathbf{L}_{-}(\mathbf{S}_{-})$. Then using Remark~\ref{rmk:C_hat_L_structure}, it follows that $[\boldsymbol{\Omega}_i]_1^{m-1}$ (which equals $\boldsymbol{\Omega}_{-} + [\boldsymbol{\Omega}_m]_1^{m-1}$) is a certificate of $\mathbf{L}_{-}(\mathbf{S}_{-})$.
% One way to show that $\Tr(\boldsymbol{\Omega}_{-}^T\mathbf{L}_{-}(\mathbf{S}_{-})\boldsymbol{\Omega}_{-}) = 0$ is: define a curve $\mathbf{S}(t) \in \mathbb{O}(d)^m$ where $t \in [0,\epsilon)$ for small enough $\epsilon$ such that $\mathbf{S}'(0) = [\mathbf{S}_i\boldsymbol{\Omega}^{'}_i]_1^m$. Similarly, define a curve $\mathbf{S}_{-}(t) \in \mathbb{O}(d)^{m-1}$ where $t \in [0,\epsilon_{-})$ for small enough $\epsilon_{-}$ such that $\mathbf{S}_{-}'(0) = [\mathbf{S}_{-_i}\boldsymbol{\Omega}_{-_i}]_1^{m-1}$. 
First, we note that
\begin{align}
    \mathbf{B}(\mathbf{S})\boldsymbol{\mathcal{P}}_{0}^T &= \begin{bmatrix}
        \mathbf{B}_{-}(\mathbf{S}_{-}) &  \mathbf{0}_{(m-1)d \times n_3} & \mathbf{0}_{(m-1)d}\\
        \begin{bmatrix}\mathbf{0}_{d \times n_1} & \mathbf{B}^*_{2} & \mathbf{0}_{d \times (m-1)} \end{bmatrix} &\mathbf{B}^*_{3} & -(\mathbf{B}^*_{2}\mathbf{1}_{n_2}+\mathbf{B}^*_{3}\mathbf{1}_{n_3})
    \end{bmatrix}\\
    \mathbf{D}(\mathbf{S}) &= \blockdiag(\mathbf{D}_{-}(\mathbf{S}_{-}), \mathbf{B}^*_{2}\mathbf{B}^{*^T}_{2} + \mathbf{B}^*_{3}\mathbf{B}^{*^T}_{3}).
\end{align}
Since $\boldsymbol{\Omega}^{'}$ is a certificate of $\mathbf{L}(\mathbf{S})$, $\Tr(\boldsymbol{\Omega}^{'^T}\mathbf{L}(\mathbf{S})\boldsymbol{\Omega}^{'}) = 0$. Then, using the definition of $\mathbf{L}(\mathbf{S})$, the above equations, the fact that $\boldsymbol{\Omega}^{'}_m = 0$, and Eq.~(\ref{supp:eq:pinvA11}), we obtain
\begin{align}
    0 &= \Tr(\boldsymbol{\Omega}^{'^T}(\mathbf{D}(\mathbf{S}) - \mathbf{B}(\mathbf{S})\boldsymbol{\mathcal{P}}_{0}^T\boldsymbol{\mathcal{P}}_{0}\boldsymbol{\mathcal{L}}_{\Gamma}^\dagger\boldsymbol{\mathcal{P}}_{0}^T\boldsymbol{\mathcal{P}}_{0} \mathbf{B}(\mathbf{S})^T)\boldsymbol{\Omega}^{'})\\
    &= \Tr(\boldsymbol{\Omega}_{-}^T(\mathbf{D}_{-}(\mathbf{S}) - \mathbf{B}_{-}(\mathbf{S}_{-}) (\boldsymbol{\mathcal{P}}_{0}\boldsymbol{\mathcal{L}}_{\Gamma}^\dagger\boldsymbol{\mathcal{P}}_{0}^T)_{11} \mathbf{B}_{-}(\mathbf{S}_{-})^T)\boldsymbol{\Omega}_{-})\\
    &= \Tr(\boldsymbol{\Omega}_{-}^T(\mathbf{D}_{-}(\mathbf{S}) - \mathbf{B}_{-}(\mathbf{S}_{-})(\boldsymbol{\mathcal{L}}_{\Gamma_{-}}+\boldsymbol{\mathcal{D}}_0)^\dagger \mathbf{B}_{-}(\mathbf{S}_{-})^T)\boldsymbol{\Omega}_{-}).
\end{align}

% \begin{align}
%     0 &= \Tr(\boldsymbol{\Omega}^{'^T}\mathbf{L}(\mathbf{S})\boldsymbol{\Omega}^{'}) = \Tr(\boldsymbol{\Omega}^{'^T}(\mathbf{B}(\mathbf{S})\boldsymbol{\mathcal{P}}_{0}^T\boldsymbol{\mathcal{P}}_{0}\boldsymbol{\mathcal{L}}_{\Gamma}^\dagger\boldsymbol{\mathcal{P}}_{0}^T\boldsymbol{\mathcal{P}}_{0} \mathbf{B}(\mathbf{S})^T - \mathbf{D}(\mathbf{S}))\boldsymbol{\Omega}^{'})\\
%     &= \Tr(\boldsymbol{\Omega}^{'^T}(\mathbf{B}(\mathbf{S})\boldsymbol{\mathcal{P}}_{0}^T(\boldsymbol{\mathcal{P}}_{0}\boldsymbol{\mathcal{L}}_{\Gamma}\boldsymbol{\mathcal{P}}_{0}^T)^\dagger\boldsymbol{\mathcal{P}}_{0} \mathbf{B}(\mathbf{S})^T - \mathbf{D}(\mathbf{S}))\boldsymbol{\Omega}^{'})\\
%     &= \Tr(\boldsymbol{\Omega}_{-}^T(\mathbf{B}_{-}(\mathbf{S}_{-}) (\boldsymbol{\mathcal{P}}_{0}\boldsymbol{\mathcal{L}}_{\Gamma}^\dagger\boldsymbol{\mathcal{P}}_{0}^T)_{11} \mathbf{B}_{-}(\mathbf{S}_{-})^T - \mathbf{D}_{-}(\mathbf{S}))\boldsymbol{\Omega}_{-})\\
%     &= \Tr(\boldsymbol{\Omega}_{-}^T(\mathbf{B}_{-}(\mathbf{S}_{-})(\boldsymbol{\mathcal{L}}_{\Gamma_{-}}+\boldsymbol{\mathcal{D}}_0)^\dagger \mathbf{B}_{-}(\mathbf{S}_{-})^T - \mathbf{D}_{-}(\mathbf{S}))\boldsymbol{\Omega}_{-})
% \end{align}
From Proposition~\ref{prop:noiseless_setting1} and Eq.~(\ref{eq:C_of_S}), 
$$\mathbf{L}_{-}(\mathbf{S}_{-}) = \mathbf{D}_{-}(\mathbf{S}_{-}) - \mathbf{B}_{-}(\mathbf{S}_{-})\boldsymbol{\mathcal{L}}_{\Gamma_{-}}^\dagger \mathbf{B}_{-}(\mathbf{S}_{-})^T,$$ thus 
$$\Tr(\boldsymbol{\Omega}_{-}^T\mathbf{L}_{-}(\mathbf{S}_{-})\boldsymbol{\Omega}_{-}) - \Tr(\boldsymbol{\Omega}_{-}^T(\mathbf{B}_{-}(\mathbf{S}_{-}) ((\boldsymbol{\mathcal{L}}_{\Gamma_{-}}+\boldsymbol{\mathcal{D}}_0)^\dagger - \boldsymbol{\mathcal{L}}_{\Gamma_{-}}^\dagger) \mathbf{B}_{-}(\mathbf{S}_{-})^T )\boldsymbol{\Omega}_{-}) = 0.$$
Since $\mathbf{L}_{-}(\mathbf{S}_{-}) \succeq 0$, to show that $\Tr(\boldsymbol{\Omega}_{-}^T\mathbf{L}_{-}(\mathbf{S}_{-})\boldsymbol{\Omega}_{-}) = 0$, it suffices to show that
\begin{prop} $\mathbf{B}_{-}(\mathbf{S}_{-}) ((\boldsymbol{\mathcal{L}}_{\Gamma_{-}}+\boldsymbol{\mathcal{D}}_0)^\dagger - \boldsymbol{\mathcal{L}}_{\Gamma_{-}}^\dagger) \mathbf{B}_{-}(\mathbf{S}_{-})^T \preceq 0$.
\end{prop}
\textit{Proof}. Since $\boldsymbol{\mathcal{L}}_{\Gamma_{-}} \succeq 0$, consider $\boldsymbol{\mathcal{L}}_{\Gamma_{-}} = \mathbf{U}\boldsymbol{\Lambda}\mathbf{U}^T = \begin{bsmallmatrix}
        \mathbf{U}_1 & \mathbf{U}_2
    \end{bsmallmatrix}\begin{bsmallmatrix}
        \mathbf{\Lambda}_1 & \\
        & \mathbf{0}
    \end{bsmallmatrix}\begin{bsmallmatrix}
        \mathbf{U}_1^T\\
        \mathbf{U}_2^T
    \end{bsmallmatrix}$
where the $1 \leq n' \leq n_2$ columns of $\mathbf{U}_2 \in \mathbb{R}^{(n_1+n_2+m-1) \times n'}$ form an orthogonal basis of the $\ker (\boldsymbol{\mathcal{L}}_{\Gamma_{-}})$ (see proof of Proposition~\ref{supp:prop:LplusD0}). Also, $\mathbf{\Lambda}_1 \succ 0$. From Proposition~\ref{prop:kerB} and Eq.~(\ref{eq:BofS}), $\mathbf{B}_{-}(\mathbf{S}_{-})\mathbf{U}_2 = 0$. Thus, $\mathbf{B}_{-}(\mathbf{S}_{-})\mathbf{U} = \begin{bmatrix}
    \mathbf{B}_{-}(\mathbf{S}_{-})\mathbf{U}_1 & 0
\end{bmatrix}$. Then note that
\begin{equation}
    ((\boldsymbol{\mathcal{L}}_{\Gamma_{-}}+\boldsymbol{\mathcal{D}}_0)^\dagger - \boldsymbol{\mathcal{L}}_{\Gamma_{-}}^\dagger) = \mathbf{U} \left\{\left(\begin{bmatrix}
        \mathbf{\Lambda}_1 & \\
        & \mathbf{0}
    \end{bmatrix} + \mathbf{U}^T\boldsymbol{\mathcal{D}}_0\mathbf{U}\right)^\dagger-\begin{bmatrix}
        \mathbf{\Lambda}_1^{-1} & \\
        & \mathbf{0}
    \end{bmatrix}\right\}\mathbf{U}^T. \label{supp:eq:eq30}
\end{equation}
Using \citeb[Eq.~(10, 11, 17, 19)]{kovanic1979pseudoinverse} and simple calculations, we obtain
$$\left(\blockdiag(\mathbf{\Lambda}_1, \mathbf{0}) + \mathbf{U}^T\boldsymbol{\mathcal{D}}_0\mathbf{U}\right)^\dagger = (\blockdiag(\mathbf{\Lambda}_1^{-1}, \mathbf{0}) + \blockdiag(\mathbf{W}_1, \mathbf{W}_2))$$
% \begin{align}
%     \left(\begin{bmatrix}
%         \mathbf{\Lambda}_1 & \\
%         & \mathbf{0}
%     \end{bmatrix} + \mathbf{U}^T\boldsymbol{\mathcal{D}}_0\mathbf{U}\right)^\dagger &= \begin{bmatrix}
%         \mathbf{\Lambda}_1^{-1} & \\
%         & \mathbf{0}
%     \end{bmatrix} + \begin{bmatrix}
%         \mathbf{W}_1 &\\
%         & \mathbf{W}_2
%     \end{bmatrix}
% \end{align}
where 
$$\mathbf{W}_1 = -\mathbf{\Lambda}_1^{-1}\mathbf{U}_1^T\boldsymbol{\mathcal{D}}_0(\mathbf{I} + \boldsymbol{\mathcal{D}}_0\mathbf{U}_1\mathbf{\Lambda}_1^{-1}\mathbf{U}_1^T\boldsymbol{\mathcal{D}}_0)^{-1}\boldsymbol{\mathcal{D}}_0\mathbf{U}_1\mathbf{\Lambda}_1^{-1}$$
and $\mathbf{W}_2 = (\mathbf{U}_2^T\boldsymbol{\mathcal{D}}_0\mathbf{U}_2)^\dagger$.
In particular $\mathbf{W}_1 \preceq 0$ and $\mathbf{W}_2 \succeq 0$.
Combining above with Eq.~(\ref{supp:eq:eq30}), we deduce that 
$$\mathbf{B}_{-}(\mathbf{S}_{-}) ((\boldsymbol{\mathcal{L}}_{\Gamma_{-}}+\boldsymbol{\mathcal{D}}_0)^\dagger - \boldsymbol{\mathcal{L}}_{\Gamma_{-}}^\dagger) \mathbf{B}_{-}(\mathbf{S}_{-})^T = \mathbf{B}_{-}(\mathbf{S}_{-})\mathbf{U}_1\mathbf{W}_1\mathbf{U}_1^T\mathbf{B}_{-}(\mathbf{S}_{-})^T \preceq 0.$$
\hfill $\blacksquare$

\noindent We conclude that $\Tr(\boldsymbol{\Omega}_{-}^T\mathbf{L}_{-}(\mathbf{S}_{-})\boldsymbol{\Omega}_{-}) = 0$ and thus $\boldsymbol{\Omega}_{-}$ is a certificate of $\mathbf{L}_{-}(\mathbf{S}_{-})$.


\proofof{Proposition~\ref{prop:same_conn_comp_non_deg}}
It suffices to show that if the $i$th and $j$th vertices are adjacent in $\mathbb{G}$ then $\boldsymbol{\Omega}_i = \boldsymbol{\Omega}_j$. For $m=2$, the result is a direct consequence of Theorem~\ref{thm:nec_suff_cond_loc_rigid_two_views}. Suppose the result holds for $m-1$ views for some $m > 2$. If there are no edges in $\mathbb{G}$, then the result holds trivially for $m$ views. Suppose $i$th and $j$th vertices are adjacent in $\mathbb{G}$. Let $r \in [1,m] \setminus \{i,j\}$. We remove the $r$th view and the points which lie exclusively in it. Then by Lemma~\ref{lem:subproblem_cert}, $[\boldsymbol{\Omega}_k]_{k \in [1,m] \setminus r}$ is a certificate of $\mathbf{L}_{-r}(\mathbf{S}_{-r})$. Now, construct $\mathbb{G}_{-r}$ (in the same way as $\mathbb{G}$) and note that the $i$th and $j$th vertices are still adjacent. Thus, by the induction hypothesis, $\boldsymbol{\Omega}_i = \boldsymbol{\Omega}_j$.

\proofof{Theorem~\ref{thm:G_star_1}}
The result holds for two views (see Theorem~\ref{thm:nec_suff_cond_loc_rigid_two_views}). Suppose it holds for $m-1$ views for some $m > 2$. Let $\boldsymbol{\Omega}$ be a certificate of $\mathbf{L}(\mathbf{S})$. We need to show that $\boldsymbol{\Omega}$ is trivial. Since $|\mathbb{G}^*(\mathbf{S})|=1$, $\mathbb{G}$ must have a connected component with at least two views. Pick one such component and note that there exist a view in it such that removing it will not disconnect the component. Let it be the $i$th view. Consider removing the $i$th view and the points which lie exclusively in it. For the new set of views we still have $|\mathbb{G}^*(\mathbf{S}_{-i})|=1$ (where $\mathbb{G}^*_{-i}(\mathbf{S}_{-i})$ is constructed in the same manner as $\mathbb{G}^*(\mathbf{S})$). By Lemma~\ref{lem:subproblem_cert} and the induction hypothesis, we conclude that $[\boldsymbol{\Omega}_j]_{j \in [1,m]\setminus \{i\}}$ must be trivial. By Proposition~\ref{prop:same_conn_comp_non_deg} we conclude that $\boldsymbol{\Omega}$ is trivial.

\proofof{Theorem~\ref{thm:nec_cond_glob_rigid_views}}
The calculations up to Eq.~(\ref{supp:eq:eq11}) in the proof of Theorem~\ref{thm:nec_cond_loc_rigid_of_views} are reused here.
Consider a partition of $[1,m]$ into two non-empty subsets $A$ and $B$. Suppose the rank of $\overline{\mathbf{B}(\mathbf{S})}_{A, B}$ is at most $d-1$. As in the proof of Theorem~\ref{thm:nec_cond_loc_rigid_of_views}, WLOG assume that $\mathbf{B}(\mathbf{S})_{A, B}\mathbf{1}_{n'} = 0$ (where $n'$ is as in Definition~\ref{def:BSAcapB}). Then the rank of $\mathbf{B}(\mathbf{S})_{A, B}$ is at most $d-1$. We are going to construct another perfect alignment $\mathbf{S}'$ such that $\pi(\mathbf{S}) \neq \pi(\mathbf{S}')$, thus concluding that $\mathbf{S}$ is not unique.

Let $\mathbf{V}_1, \mathbf{V}_2 \in \mathbb{O}(d)$ and $\mathbf{\Sigma}$ be the diagonal matrix containing the singular values of $\mathbf{B}(\mathbf{S})_{A, B}$ such that $\mathbf{B}(\mathbf{S})_{A, B} = \mathbf{V}_1\mathbf{\Sigma}\mathbf{V}_2^T$. Since rank of $\mathbf{B}(\mathbf{S})_{A, B} \leq d-1$, there exist $\mathbf{U} \in \mathbb{O}(d)$ such that $\mathbf{U} \neq \mathbf{I}_d$ and $\mathbf{\Sigma} = \mathbf{U}\mathbf{\Sigma}$. Define $\mathbf{Q} = \mathbf{V}_1\mathbf{U}^T\mathbf{V}_1^T$. Then $\mathbf{Q} \neq \mathbf{I}_d$ and 
\begin{equation}
    \mathbf{Q}^T\mathbf{B}(\mathbf{S})_{A, B} = \mathbf{B}(\mathbf{S})_{A, B}. \label{supp:eq:QTBSAB}
\end{equation}
Define $\mathbf{S}' \in \mathbb{O}(d)^m$ such that $\mathbf{S}'_i = \mathbf{S}_i\mathbf{Q}$ for all $i \in A$ and $\mathbf{S}'_j = \mathbf{S}_j$ for all $j \in B$. Clearly, $\mathbf{S}' \neq \mathbf{S}$. We will show that $\mathbf{S}'$ is another perfect alignment. It is easy to see that
% \begin{align}
%     \Tr(\mathbf{S}'^T\mathbf{C}\mathbf{S}') &= \textstyle\sum_{i = 1}^{m}\textstyle\sum_{j=1}^{m}\Tr(\mathbf{S}'^T_i\mathbf{C}_{ij}\mathbf{S}'_j)\\
%     &= \textstyle\sum_{\substack{i \in A, j \in A\\i \in B, j \in B}}\Tr(\mathbf{S}_i^T\mathbf{C}_{ij}\mathbf{S}_j) + 2 \textstyle\sum_{i \in A}\textstyle\sum_{j \in B}\Tr(\mathbf{Q}^T\mathbf{S}_i^T\mathbf{C}_{ij}\mathbf{S}_j).
% \end{align}
%\begin{equation}
$$\Tr(\mathbf{S}'^T\mathbf{C}\mathbf{S}') = \textstyle\sum_{\substack{i \in A, j \in A\\i \in B, j \in B}}\Tr(\mathbf{S}_i^T\mathbf{C}_{ij}\mathbf{S}_j) + 2 \textstyle\sum_{i \in A, j \in B}\Tr(\mathbf{Q}^T\mathbf{S}_i^T\mathbf{C}_{ij}\mathbf{S}_j).$$
%\end{equation}
Since, for $i \in A$ and $j \in B$, $\mathbf{C}_{ij} = \mathbf{B}_i \boldsymbol{\mathcal{L}}_{\Gamma}^\dagger \mathbf{B}_j^T$, it suffices to show that
$$\Tr\left(\left(\textstyle\sum_{i \in A}\mathbf{B}(\mathbf{S})_i\right)\boldsymbol{\mathcal{L}}_{\Gamma}^\dagger \left(\textstyle\sum_{j \in B}\mathbf{B}(\mathbf{S})_j^T\right)\right) = \Tr\left(\mathbf{Q}^T\left(\textstyle\sum_{i \in A}\mathbf{B}(\mathbf{S})_i\right)\boldsymbol{\mathcal{L}}_{\Gamma}^\dagger \left(\textstyle\sum_{j \in B}\mathbf{B}(\mathbf{S})_j^T\right)\right).$$
% \begin{equation}
%     \Tr\left(\left(\textstyle\sum_{i \in A}\mathbf{B}(\mathbf{S})_i\right)\boldsymbol{\mathcal{L}}_{\Gamma}^\dagger \left(\textstyle\sum_{j \in B}\mathbf{B}(\mathbf{S})_j^T\right)\right)  = \Tr\left(\mathbf{Q}^T\left(\textstyle\sum_{i \in A}\mathbf{B}(\mathbf{S})_i\right)\boldsymbol{\mathcal{L}}_{\Gamma}^\dagger \left(\textstyle\sum_{j \in B}\mathbf{B}(\mathbf{S})_j^T\right)\right) 
% \end{equation}
Define $\mathbf{B}_A$, $\mathbf{B}_B$,  $\mathbf{B}_{A\setminus B}$, $\mathbf{B}_{A,B}$, $\mathbf{B}_{B\setminus A}$ and $\mathbf{B}_{B,A}$ as in the proof of Theorem~\ref{thm:nec_cond_loc_rigid_of_views}, then it suffices to show that 
$\Tr(\mathbf{Q}^T \mathbf{B}_A\boldsymbol{\mathcal{L}}_{\Gamma}^\dagger \mathbf{B}_B^T) = \Tr( \mathbf{B}_A\boldsymbol{\mathcal{L}}_{\Gamma}^\dagger \mathbf{B}_B^T).$
Using Eq.~(\ref{supp:eq:QTBSAB}, \ref{supp:eq:eq8}, \ref{supp:eq:eq9}) we obtain $\mathbf{Q}^T \mathbf{B}_{A,B} = \mathbf{B}_{A,B}$ and this combined with Eq.~(\ref{supp:eq:eq11}), yields
\begin{align}
    \Tr(\mathbf{Q}^T \mathbf{B}_A\boldsymbol{\mathcal{L}}_{\Gamma}^\dagger \mathbf{B}_B^T) &= \Tr(\mathbf{Q}^T (\mathbf{B}_{A \setminus B}+\mathbf{B}_{A,B})\boldsymbol{\mathcal{L}}_{\Gamma}^\dagger (\mathbf{B}_{B\setminus A} + \mathbf{B}_{B,A})^T ) \\
    &= \Tr(\mathbf{Q}^T \mathbf{B}_{A,B}\boldsymbol{\mathcal{L}}_{\Gamma}^\dagger\mathbf{B}_{B,A}^T) = \Tr( \mathbf{B}_{A,B}\boldsymbol{\mathcal{L}}_{\Gamma}^\dagger\mathbf{B}_{B,A}^T)\\
    &=\Tr((\mathbf{B}_{A \setminus B}+\mathbf{B}_{A,B})\boldsymbol{\mathcal{L}}_{\Gamma}^\dagger (\mathbf{B}_{B\setminus A} + \mathbf{B}_{B,A})^T )\\
    &= \Tr(\mathbf{B}_A\boldsymbol{\mathcal{L}}_{\Gamma}^\dagger \mathbf{B}_B^T)
\end{align}
% \begin{align}
%     &\Tr(\mathbf{Q}^T \mathbf{B}_A\boldsymbol{\mathcal{L}}_{\Gamma}^\dagger \mathbf{B}_B^T) = \Tr(\mathbf{Q}^T (\mathbf{B}_{A \setminus B}+\mathbf{B}_{A,B})\boldsymbol{\mathcal{L}}_{\Gamma}^\dagger (\mathbf{B}_{B\setminus A} + \mathbf{B}_{B,A})^T )\\
%     &= \Tr(\mathbf{Q}^T \mathbf{B}_{A,B}\boldsymbol{\mathcal{L}}_{\Gamma}^\dagger\mathbf{B}_{B,A}^T) = \Tr( \mathbf{B}_{A,B}\boldsymbol{\mathcal{L}}_{\Gamma}^\dagger\mathbf{B}_{B,A}^T)\\
%     &= \Tr((\mathbf{B}_{A \setminus B}+\mathbf{B}_{A,B})\boldsymbol{\mathcal{L}}_{\Gamma}^\dagger (\mathbf{B}_{B\setminus A} + \mathbf{B}_{B,A})^T ) = \Tr(\mathbf{B}_A\boldsymbol{\mathcal{L}}_{\Gamma}^\dagger \mathbf{B}_B^T).
% \end{align}

\proofof{Proposition~\ref{prop:same_conn_comp_uniq}} By replacing $\mathbb{G}$ with $\overline{\mathbb{G}}$ and Theorem~\ref{thm:nec_suff_cond_loc_rigid_two_views} with \ref{thm:nec_suff_cond_glob_rigid_two_views}, the inductive proof is the same as of Proposition~\ref{prop:same_conn_comp_non_deg}.
%It suffices to show the result for adjacent vertices in $\overline{\mathbb{G}}$. For two views, the result follows from Theorem~\ref{thm:nec_suff_cond_glob_rigid_two_views}. Suppose the result holds for $m-1$ views for some $m > 2$. Then we show the result for $m$ views. If there are no edges in $\overline{\mathbb{G}}$ then the result is trivially valid. Suppose $i$th and $j$th vertices are adjacent in $\overline{\mathbb{G}}$. Let $r \in [1,m] \setminus \{i,j\}$. We remove the $r$th view and the points which lie exclusively in it. Then, by Lemma~\ref{lem:subproblem_cert}, $\mathbf{S}_{-r} = [\mathbf{S}_{k}]_{k \in [1,m] \setminus \{r\}}$ is a perfect alignment of the remaining views. Now, construct $\overline{\mathbb{G}}_{-r}$ (in the same way as $\overline{\mathbb{G}}$) and note that the vertices corresponsing to $i$th and $j$th views are still adjacent in $\overline{\mathbb{G}}_{-r}$. By the induction hypothesis we conclude the result.

\proofof{Theorem~\ref{thm:overline_G_star_1}} By replacing $\mathbb{G}$ with $\overline{\mathbb{G}}$, Theorem~\ref{thm:nec_suff_cond_loc_rigid_two_views} with \ref{thm:nec_suff_cond_glob_rigid_two_views} and Proposition~\ref{prop:same_conn_comp_non_deg} with \ref{prop:same_conn_comp_uniq}, the inductive proof is the same as of Theorem~\ref{thm:G_star_1}.

%The result holds for two views (see Theorem~\ref{thm:nec_suff_cond_glob_rigid_two_views}). Suppose the result holds for $m-1$ views for some $m > 2$. We will show that the result holds for $m$ views. Suppose $|\overline{\mathbb{G}}^*(\mathbf{S})|=1$. Let $\mathbf{S}'$ be a perfect alignment. We need to show that $\mathbf{S}' = \mathbf{S}\mathbf{Q}$ for some $\mathbf{Q} \in \mathbb{O}(d)$. Since $|\overline{\mathbb{G}}^*(\mathbf{S})|=1$, $\overline{\mathbb{G}}$ must have a connected component with at least two views. Pick one such component and note that there exist a view in it such that removing it will not disconnect the component. Let it be the $i$th view. Consider removing the $i$th view and the points that lie exclusively in it. Note that for the new set of views we still have $|\overline{\mathbb{G}}^*_{-i}(\mathbf{S}_{-i})|=1$ (where $\overline{\mathbb{G}}^*_{-i}(\mathbf{S}_{-i})$ is constructed in the same manner as $\overline{\mathbb{G}}^*(\mathbf{S})$). By Lemma~\ref{lem:subproblem_cert} and the induction hypothesis, we conclude that $\mathbf{S}'_{-i} = \mathbf{S}_{-i}\mathbf{Q}$ for some $\mathbf{Q} \in \mathbb{O}(d)$. Then, by Proposition~\ref{prop:same_conn_comp_uniq} we conclude that $\mathbf{S}' = \mathbf{S}\mathbf{Q}$.

%%%%%%%%%%%%%%%%%%%%%%%%%%%%%%%%%%%%%%%%%%
% Section 5 Proofs
%%%%%%%%%%%%%%%%%%%%%%%%%%%%%%%%%%%%%%%%%%
\proofof{Lemma~\ref{lem:retraction}}
For $\mathbf{S} \in \pi^{-1}(\widetilde{\mathbf{S}})$ and $\mathbf{Z} = [\mathbf{S}_i\boldsymbol{\Omega}_i]_1^m \in T_{\mathbf{S}}\mathbb{O}(d)^m$, the horizontal lift of $\widetilde{\mathbf{Z}} = [\widetilde{\mathbf{S}}_i\widetilde{\boldsymbol{\Omega}}_i] \in T_{\widetilde{\mathbf{S}}}\mathbb{O}(d)^m/_{\sim}$,
$$\pi(R_{\EXP}(\mathbf{S}, \mathbf{Z})) = [\mathbf{S}_i\exp(\boldsymbol{\Omega}_i)(\mathbf{S}_1\exp(\boldsymbol{\Omega}_1))^T]_1^m.$$
It suffices to show that $\mathbf{S}_{i+1}\exp(\boldsymbol{\Omega}_{i+1})(\mathbf{S}_1\exp(\boldsymbol{\Omega}_1))^T$ depends only on $\widetilde{\mathbf{S}}$ and $\widetilde{\boldsymbol{\Omega}}$ for all $i \in [1,m-1]$. Using Proposition~\ref{prop:hlift_char} and expanding the expression, we obtain 
$$\mathbf{S}_{i+1}\exp(\boldsymbol{\Omega}_{i+1})(\mathbf{S}_1\exp(\boldsymbol{\Omega}_1))^T = \mathbf{S}_{i+1}\exp(\mathbf{S}_1^T \widetilde{\boldsymbol{\Omega}}_i\mathbf{S}_1 + \boldsymbol{\Omega}_1) (\mathbf{S}_1\exp(\boldsymbol{\Omega}_1))^T.$$
Since $\mathbf{S}_1 \in \mathbb{O}(d)$, we have $\exp(\mathbf{S}_1^T \widetilde{\boldsymbol{\Omega}}_i\mathbf{S}_1) = \mathbf{S}_1^T\exp(\widetilde{\boldsymbol{\Omega}}_i)\mathbf{S}_1$. Also, the following identites hold: $\exp(\boldsymbol{\Omega}_1)^T = \exp(\boldsymbol{\Omega}_1^T) = \exp(-\boldsymbol{\Omega}_1)$ and $\exp(\mathbf{A}_1+\mathbf{A}_1) = \exp(\mathbf{A}_1)\exp(\mathbf{A}_2)$. By substituting back into the expression, we obtain 
$$\mathbf{S}_{i+1}\exp(\mathbf{S}_1^T \widetilde{\boldsymbol{\Omega}}_i\mathbf{S}_1 + \boldsymbol{\Omega}_1) (\mathbf{S}_1\exp(\boldsymbol{\Omega}_1))^T = \mathbf{S}_{i+1}\mathbf{S}_1^T\exp(\widetilde{\boldsymbol{\Omega}}_i) = \widetilde{\mathbf{S}}_i\exp(\widetilde{\boldsymbol{\Omega}}_i).$$

\proofof{Proposition~\ref{prop:liu_pf}}
Since $\mathbf{Z}_i = \mathbf{S}_i\boldsymbol{\Omega}_i$ where $\boldsymbol{\Omega}_i \in \Skew(d)$,
$$
    \left\|\mathbf{S}_i\exp(\mathbf{S}_i^T\mathbf{Z}_i) - (\mathbf{S}_i + \mathbf{Z}_i)\right\|_F = \left\|\exp(\boldsymbol{\Omega}_i) - (\mathbf{I}_d + \boldsymbol{\Omega}_i)\right\|_F \leq \left(\sum_{2}^{\infty}\frac{1}{k !}\right)\left\|\boldsymbol{\Omega}_i\right\|_F^2 = (e-1)\left\|\boldsymbol{\Omega}_i\right\|_F^2
$$
where the inequality follows from the triangle inequality and the fact that $\left\|\boldsymbol{\Omega}_i\right\|_F = \left\|\mathbf{Z}_i\right\|_F \leq 1$.

% \proofof{Proposition~\ref{prop:second_order_boundedness_of_RPF}}
% \revdel{Fix $\phi = 1/2$. Then $\left\|\boldsymbol{\xi}_i\right\|_F \leq 1/2$. From Proposition~\ref{prop:liu_qr}, we obtain $\left\|R_{\QR }(\mathbf{S}, \boldsymbol{\xi}) - (\mathbf{S} + \boldsymbol{\xi})\right\|_F^2 = \textstyle\sum_1^m \left\|\qf (\mathbf{S}_i + \boldsymbol{\xi}_i) - (\mathbf{S}_i + \boldsymbol{\xi}_i)\right\|_F^2 \leq M^2 \textstyle\sum_1^m \left\|\boldsymbol{\xi}_i\right\|_F^4 \leq M^2 \left\|\boldsymbol{\xi}\right\|_F^4$.
% \begin{align}
%     \left\|R_{\QR }(\mathbf{S}, \boldsymbol{\xi}) - (\mathbf{S} + \boldsymbol{\xi})\right\|_F^2 &= \textstyle\sum_1^m \left\|\qf (\mathbf{S}_i + \boldsymbol{\xi}_i) - (\mathbf{S}_i + \boldsymbol{\xi}_i)\right\|_F^2 \leq M^2 \textstyle\sum_1^m \left\|\boldsymbol{\xi}_i\right\|_F^4 \leq M^2 \left\|\boldsymbol{\xi}\right\|_F^4.
%     %&\leq M^2 \left(\textstyle\sum_1^m \left\|\boldsymbol{\xi}_i\right\|_F^2\right)^2 = M^2 \left\|\boldsymbol{\xi}\right\|_F^4.
% \end{align}
% The result follows for the same $M$ as in Proposition~\ref{prop:liu_qr}.}
% \revadd{By adding and subtracting the term $(\mathbf{S}_i + \boldsymbol{\xi}_i)\pf(\mathbf{S}_1 + \boldsymbol{\xi}_1)^T$ to $\left\|\pf(\mathbf{S}_i + \boldsymbol{\xi}_i)\pf(\mathbf{S}_1 + \boldsymbol{\xi}_1)^T - (\mathbf{S}_i + \boldsymbol{\xi}_i)\right\|_F$, using triangle inequality and then using Proposition~\ref{prop:liu_pf}, we obtain the bound $\left\|\boldsymbol{\xi}_i\right\|_F^2 + \left\|\mathbf{S}_i+\boldsymbol{\xi}_i\right\|_F \left\|\pf(\mathbf{S}_1+\boldsymbol{\xi}_1) - \mathbf{I}_d\right\|_F  \leq \left\|\boldsymbol{\xi}_i\right\|_F^2 + \left\|\boldsymbol{\xi}_1\right\|_F^2 \leq 2\left\|\boldsymbol{\xi}\right\|_F^2$. The result follows.
% }
% \revadd{The proof of part (a) follows from Proposition~\ref{prop:liu_pf} and $\left\|R_\EXP(\mathbf{S}, \boldsymbol{\xi}) - (\mathbf{S} + \boldsymbol{\xi})\right\|_F \leq \sum_1^m \left\|\mathbf{S}_i\exp(\mathbf{S}_i^T\boldsymbol{\xi}_i) - (\mathbf{S}_i + \boldsymbol{\xi}_i)\right\|_F$. For part (b), adding and subtracting $(\mathbf{S}_i + \boldsymbol{\xi}_i)(\mathbf{S}_1\exp(\mathbf{S}_1^T\boldsymbol{\xi}_1))^T$ to 
%$\left\|\mathbf{S}_i\exp(\mathbf{S}_i^T\boldsymbol{\xi}_i)(\mathbf{S}_1\exp(\mathbf{S}_1^T\boldsymbol{\xi}_1))^T - (\mathbf{S}_i + \boldsymbol{\xi}_i)(\mathbf{S}_1+\boldsymbol{\xi}_1)^T\right\|_F$
% the l.h.s, using triangle inequality and Proposition~\ref{prop:liu_pf} with the fact that $\left\|\boldsymbol{\xi}_i\right\|_F^2 \leq \left\|\boldsymbol{\xi}\right\|_F^2 \leq 1$, we obtain the bound $(e-1)\left\|\boldsymbol{\xi}_i\right\|_F^2 + \left\|\mathbf{S}_i+\boldsymbol{\xi}_i\right\|_F \left\|\mathbf{S}_1\exp(\mathbf{S}_1^T\boldsymbol{\xi}_1) - (\mathbf{S}_1+\boldsymbol{\xi}_1)\right\|_F  \leq (e-1)(\left\|\boldsymbol{\xi}_i\right\|_F^2 + 2\left\|\boldsymbol{\xi}_1\right\|_F^2)$. Since $\left\|\boldsymbol{\xi}_i\right\|_F^2 + 2\left\|\boldsymbol{\xi}_1\right\|_F^2 \leq 2\left\|\boldsymbol{\xi}\right\|_F^2$, the result follows.}

\proofof{Proposition~\ref{prop:second_order_boundedness_of_Rtilde}}
The proof of part (a) follows from Proposition~\ref{prop:liu_pf}, Proposition~\ref{prop:hlift_frob_ineq} and 
$$\left\|R_\EXP(\mathbf{S}, \boldsymbol{\xi}) - (\mathbf{S} + \boldsymbol{\xi})\right\|_F \leq \sum_1^m \left\|\mathbf{S}_i\exp(\mathbf{S}_i^T\boldsymbol{\xi}_i) - (\mathbf{S}_i + \boldsymbol{\xi}_i)\right\|_F.$$
For part (b), let $\mathbf{S} \in \pi^{-1}(\widetilde{\mathbf{S}})$ and $\mathbf{Z} = [\mathbf{S}_i\boldsymbol{\Omega}_i]_1^m \in T_{\mathbf{S}}\mathbb{O}(d)^m$ be the horizontal lift of $\widetilde{\mathbf{Z}} = [\widetilde{\mathbf{S}}_i\widetilde{\boldsymbol{\Omega}}_i]_1^m$ at $\mathbf{S}$. Note that $\widetilde{\mathbf{S}}_i = \mathbf{S}_{i+1}\mathbf{S}_1^T$ (Eq.~(\ref{eq:pi_inv_wtS})) and $\boldsymbol{\Omega}_{i+1} -  \boldsymbol{\Omega}_{1}= \mathbf{S}_1^T\widetilde{\boldsymbol{\Omega}}_{i}\mathbf{S}_1$ (Eq.~(\ref{eq:hlifti})). Then, using the identity $\exp(\boldsymbol{\Omega}_{i+1})\exp(\boldsymbol{\Omega}_{1})^T = \exp(\boldsymbol{\Omega}_{i+1}-\boldsymbol{\Omega}_{1})$, the inequality $\left\|\boldsymbol{\Omega}_{i+1}-\boldsymbol{\Omega}_{1}\right\|_F \leq \left\|\boldsymbol{\Omega}_{i+1}\right\|_F + \left\|\boldsymbol{\Omega}_{1}\right\|_F \leq 1$ and Proposition~\ref{prop:liu_pf}, we obtain
\begin{align}
    \left\|\widetilde{R}_\EXP(\widetilde{\mathbf{S}}, \widetilde{\mathbf{Z}})_i - (\widetilde{\mathbf{S}}_i + \widetilde{\mathbf{Z}}_i)\right\|_F &= \left\|\mathbf{S}_{i+1}(\exp(\boldsymbol{\Omega}_{i+1})\exp(\boldsymbol{\Omega}_{1})^T - (\mathbf{I}_d +\boldsymbol{\Omega}_{i+1} -  \boldsymbol{\Omega}_{1}))\mathbf{S}_{1}^T\right\|_F\\
    &\leq (e-1)\left\|\boldsymbol{\Omega}_{i+1} -  \boldsymbol{\Omega}_{1}\right\|_F^2 = (e-1)\left\|\widetilde{\boldsymbol{\Omega}}_{i}\right\|_F^2 = (e-1)\left\|\widetilde{\mathbf{Z}}_i\right\|_F^2.
\end{align}
Then the result follows from $\left\|\widetilde{R}_\EXP(\widetilde{\mathbf{S}}, \widetilde{\mathbf{Z}}) - (\widetilde{\mathbf{S}} + \widetilde{\mathbf{Z}})\right\|_F \leq \sum_1^m \left\|\widetilde{R}_\EXP(\widetilde{\mathbf{S}}, \widetilde{\mathbf{Z}})_i - (\widetilde{\mathbf{S}}_i + \widetilde{\mathbf{Z}}_i)\right\|_F$.

\proofof{Proposition~\ref{prop:alpha_grad}}
\revadd{The proof is similar to the one in \citea{liu2019quadratic}. Due to \textbf{(A2)} in Section~\ref{subsec:loc_lin_conv}, WLOG we assume that $\grad F(\mathbf{S}^k) \neq 0$ for all $k \geq 0$. Then, from the proof of Proposition~\ref{prop:gradFS}, since 
$$\grad F(\mathbf{S})_i = 0.5\left(\nabla F(\mathbf{S})_i - \mathbf{S}_i\nabla F(\mathbf{S})_i^T\mathbf{S}_i\right),$$
we obtain 
\begin{align}
g(\nabla F(\mathbf{S}^k)_i, \grad F(\mathbf{S}^k)_i) &= 0.5\left(\left\|\nabla F(\mathbf{S}^k)_i\right\|_F^2 -g(\nabla F(\mathbf{S}^k)_i,\mathbf{S}^k_i\nabla F(\mathbf{S}^k)_i^T\mathbf{S}^k_i)\right)\\
&= 0.25 \left\|\nabla F(\mathbf{S}^k)_i - \mathbf{S}^k_i\nabla F(\mathbf{S}^k)_i^T\mathbf{S}^k_i\right\|_F^2\\
&= \left\|\grad F(\mathbf{S}^k)_i\right\|_F^2.
\end{align}
Thus, 
$$g(\nabla F(\mathbf{S}^k), \grad F(\mathbf{S}^k)) = \textstyle\sum_1^m g(\nabla F(\mathbf{S}^k)_i, \grad F(\mathbf{S}^k)_i) = \left\|\grad F(\mathbf{S}^k)\right\|_F^2.$$ 
This together with Algorithm~\ref{algo:rgd}, and Eq.~(\ref{eq:armijo_step}), implies
\begin{equation}
    F(\mathbf{S}^{k+1})-F(\mathbf{S}^{k}) \leq -\gamma \alpha_k \left\|\grad F(\mathbf{S}^k)\right\|_F^2 \label{supp:eq:eq1}
\end{equation}
Since $\mathbb{O}(d)^m$ is compact and $F$ is analytic, thus $F$ is bounded. Since $\alpha_k \in (0,1]$ for all $k \geq 0$, 
$$\textstyle\sum_0^\infty \alpha_k^2 \left\|\grad F(\mathbf{S}^k)\right\|_F^2 \leq \textstyle\sum_0^\infty \alpha_k \left\|\grad F(\mathbf{S}^k)\right\|_F^2 \leq \frac{1}{\gamma}(F(\mathbf{S}^0) - \lim F(\mathbf{S}^k)) < \infty.$$
Thus, $\lim \alpha_k \left\|\grad F(\mathbf{S}^k)\right\|_F = 0$ and from Proposition~\ref{prop:hlift_frob_ineq}, we obtain $\lim \alpha_k \left\|\grad \widetilde{F}(\widetilde{\mathbf{S}}^k)\right\|_F = 0$.}
% \proofof{(A1) in Section~\ref{subsec:loc_lin_conv}}
% \revdel{
% From Proposition~\ref{prop:alpha_grad}, there exists $k_1 \geq 0$ such that $\alpha_k\left\|\grad F(\mathbf{S}^k)\right\| \leq 1/2$ for all $k \geq k_1$. Then note that $\left\|\mathbf{S}^{k+1} - \mathbf{S}^k\right\|_F$ equals $\left\|R_{QR}(\mathbf{S}^k, -\alpha_k\grad F(\mathbf{S}^k)) - \mathbf{S}^k\right\|_F$. From Proposition~\ref{prop:second_order_boundedness_of_RQR}, for all $k \geq k_1$, the latter is bounded by $M \alpha_k^2 \left\|\grad F(\mathbf{S}^k)\right\|_F^2 + \alpha_k\left\|\grad F(\mathbf{S}^k)\right\|_F$, which in turn is bounded by $(M/2+1)\alpha_k \left\|\grad F(\mathbf{S}^k)\right\|_F$.
% % \begin{align}
% %     &\left\|\mathbf{S}^{k+1} - \mathbf{S}^k\right\|_F = \left\|R_{QR}(\mathbf{S}^k, -\alpha_k\grad F(\mathbf{S}^k)) - \mathbf{S}^k\right\|_F\\
% %     &\leq M \alpha_k^2 \left\|\grad F(\mathbf{S}^k)\right\|_F^2 + \alpha_k\left\|\grad F(\mathbf{S}^k)\right\|_F \leq (M/2+1)\alpha_k \left\|\grad F(\mathbf{S}^k)\right\|_F.
% % \end{align}
% Finally, using Eq~(\ref{supp:eq:eq1}), for all $k \geq k_1$, $F(\mathbf{S}^{k+1})-F(\mathbf{S}^{k}) \leq -2\gamma(M+2)^{-1} \left\|\grad F(\mathbf{S}^k)\right\|_F \cdot \left\|\mathbf{S}^{k+1}-\mathbf{S}^k\right\|_F$. Thus, (\textbf{A1}) holds for $\kappa_0 = 2\gamma(M+2)^{-1}$. The result follows.
% }
% \revadd{
% For convenience, denote $\grad F(\mathbf{S}^k)$ by $[\mathbf{S}^k_i\boldsymbol{\xi}^k_i]_1^m$. From Proposition~\ref{prop:alpha_grad}, there exists $k_1 \geq 0$ such that $\alpha_k\left\|\grad F(\mathbf{S}^k)\right\|_F \leq 1/2$ for all $k \geq k_1$. Therefore, for all $k \geq k_1$, $\alpha_k \left\|\boldsymbol{\xi}^k\right\|_F \leq 1/2$. Then note that $\left\|\widetilde{\mathbf{S}}^{k+1} - \widetilde{\mathbf{S}}^k\right\|_F$ equals $\left\|R_\EXP(\mathbf{S}^k, -\alpha_k\boldsymbol{\xi}^k)(\mathbf{S}^k_1\exp(-\alpha_k\boldsymbol{\xi}^k_1))^T - \mathbf{S}^k(\mathbf{S}^k_1)^T\right\|_F$. Using Proposition~\ref{prop:second_order_boundedness_of_RPF},}
% \begin{align}
%     &\left\|R_\EXP(\mathbf{S}^k, -\alpha_k\boldsymbol{\xi}^k)(\mathbf{S}^k_1\exp(-\alpha_k\boldsymbol{\xi}^k_1))^T - \mathbf{S}^k(\mathbf{S}^k_1)^T\right\|_F\\
%     &\leq \left\|R_\EXP(\mathbf{S}^k, -\alpha_k\boldsymbol{\xi}^k)(\mathbf{S}^k_1\exp(-\alpha_k\boldsymbol{\xi}^k_1))^T - (\mathbf{S}^k+\alpha_k\boldsymbol{\xi}^k)(\mathbf{S}^k_1)^T\right\|_F + \alpha_k\left\|\boldsymbol{\xi}^k\right\|_F\\
%     &\leq \left\|R_\EXP(\mathbf{S}^k, -\alpha_k\boldsymbol{\xi}^k)(\mathbf{S}^k_1\exp(-\alpha_k\boldsymbol{\xi}^k_1))^T - (\mathbf{S}^k+\alpha_k\boldsymbol{\xi}^k)(\mathbf{S}^k_1+\alpha_k\boldsymbol{\xi}^k_1)^T\right\|_F +\\
%     &\hspace{5.5cm} (\sqrt{md} + \alpha_k \left\|\boldsymbol{\xi}^k\right\|_F)\alpha_k \left\|\boldsymbol{\xi}^k\right\|_F + \alpha_k\left\|\boldsymbol{\xi}^k\right\|_F\\
%     &\leq 2\sqrt{m}(e-1)\alpha_k^2 \left\|\boldsymbol{\xi}^k\right\|_F^2 + (\sqrt{md}+3/2)\alpha_k \left\|\boldsymbol{\xi}^k\right\|_F\\
%     &\leq (\sqrt{m}(e-1) +\sqrt{md}+3/2)\alpha_k \left\|\boldsymbol{\xi}^k\right\|_F,
% \end{align}
% for all $k \geq k_1$.
% \revadd{
% % From Proposition~\ref{prop:second_order_boundedness_of_RPF}, for all $k \geq k_1$, the latter is bounded by $2\sqrt{m} \alpha_k^2 \left\|\boldsymbol{\xi}^k\right\|_F^2 + \alpha_k\left\|\boldsymbol{\xi}^k\right\|_F$, which in turn is bounded by $(\sqrt{m}+1)\alpha_k \left\|\boldsymbol{\xi}^k\right\|_F$.
% % \begin{align}
% %     &\left\|\mathbf{S}^{k+1} - \mathbf{S}^k\right\|_F = \left\|R_{QR}(\mathbf{S}^k, -\alpha_k\grad F(\mathbf{S}^k)) - \mathbf{S}^k\right\|_F\\
% %     &\leq M \alpha_k^2 \left\|\grad F(\mathbf{S}^k)\right\|_F^2 + \alpha_k\left\|\grad F(\mathbf{S}^k)\right\|_F \leq (M/2+1)\alpha_k \left\|\grad F(\mathbf{S}^k)\right\|_F.
% % \end{align}
% Then using Eq~(\ref{supp:eq:eq1}) and the fact that $F(\mathbf{S}^k) = \widetilde{F}(\widetilde{\mathbf{S}}_k)$, for all $k \geq k_1$, $\widetilde{F}(\mathbf{S}^{k+1})-\widetilde{F}(\mathbf{S}^{k}) \leq -\gamma(\sqrt{m}(e-1) +\sqrt{md}+3/2)^{-1} \left\|\boldsymbol{\xi}^k\right\|_F \cdot \left\|\widetilde{\mathbf{S}}^{k+1}-\widetilde{\mathbf{S}}^k\right\|_F$. Finally, from Proposition~\ref{prop:hlift_frob_ineq}, $\left\|\boldsymbol{\xi}^k\right\|_F = \left\|\grad F(\mathbf{S}^k)\right\|_F \geq (m+1)^{-1/2} \left\|\grad \widetilde{F}(\widetilde{\mathbf{S}}^k)\right\|_F$. Thus, (\textbf{A1}) holds for $\kappa_0 = \gamma(\sqrt{m}(e-1) +\sqrt{md}+3/2)^{-1}(m+1)^{-1/2}$.  The result follows.
% }

\proofof{(A1) in Section~\ref{subsec:loc_lin_conv}}
From Proposition~\ref{prop:alpha_grad}, there exists $k_1 \geq 0$ such that $\alpha_k\left\|\grad \widetilde{F}(\widetilde{\mathbf{S}}^k)\right\|_F \leq 1/2$ for all $k \geq k_1$. Then note that 
$$\left\|\widetilde{\mathbf{S}}^{k+1} - \widetilde{\mathbf{S}}^k\right\|_F = \left\|\widetilde{R}_\EXP(\widetilde{\mathbf{S}}^k, -\alpha_k\grad \widetilde{F}(\widetilde{\mathbf{S}}^k)) - \widetilde{\mathbf{S}}^k\right\|_F.$$
From Proposition~\ref{prop:second_order_boundedness_of_Rtilde}, for all $k \geq k_1$,
\begin{align}
    \left\|\widetilde{R}_\EXP(\widetilde{\mathbf{S}}^k, -\alpha_k\grad \widetilde{F}(\widetilde{\mathbf{S}}^k)) - \widetilde{\mathbf{S}}^k\right\|_F &\leq (e-1) \alpha_k^2 \left\|\grad \widetilde{F}(\widetilde{\mathbf{S}}^k)\right\|_F^2 + \alpha_k\left\|\grad \widetilde{F}(\widetilde{\mathbf{S}}^k)\right\|_F\\
    &\leq \frac{1}{2}(e+1)\alpha_k \left\|\grad \widetilde{F}(\widetilde{\mathbf{S}}^k)\right\|_F
\end{align}
Finally, using Eq~(\ref{supp:eq:eq1}) and Proposition~\ref{prop:hlift_frob_ineq}, for all $k \geq k_1$, 
$$\widetilde{F}(\widetilde{\mathbf{S}}^{k+1})-\widetilde{F}(\widetilde{\mathbf{S}}^{k}) \leq -2\gamma(e+1)^{-1}(m+1)^{-1/2} \left\|\grad \widetilde{F}(\widetilde{\mathbf{S}}^k)\right\|_F \cdot \left\|\widetilde{\mathbf{S}}^{k+1}-\widetilde{\mathbf{S}}^k\right\|_F.$$
Thus, (\textbf{A1}) holds for $\kappa_0 = 2\gamma(e+1)^{-1}(m+1)^{-1/2}$. The result follows.

\proofof{(A3) in Section~\ref{subsec:loc_lin_conv}}
Since $\nabla F(\mathbf{S}) = 2\mathbf{C}\mathbf{S}$ is Lipschitz with parameter $L_F \leq 2\left\|\mathbf{C}\right\|_F$, the proof of \textbf{(A3)} is same as in \citeb[Pg. 235]{liu2019quadratic} (alternatively \citeb[Theorem 2.10]{schneider2015convergence}). For the sake of completeness, we present an adaptation of their proof to our setting.
% The proof is essentially the same as the one in \citea{liu2019quadratic}. Due to (A2) in Section~\ref{subsec:loc_sub_conv}, WLOG we assume that $\grad F(\mathbf{S}^k) \neq 0$ for all $k \geq 0$. Then, using line 5 in Algorithm~\ref{algo:rgd} and Proposition~\ref{prop:second_order_boundedness_of_RQR}, we obtain
% \begin{align}
%     \left\|\mathbf{S}^{k+1} - \mathbf{S}^k\right\|_F &= \left\|R_{QR}(\mathbf{S}^k, -\alpha_k\grad F(\mathbf{S}^k)) - \mathbf{S}^k\right\|_F\\
%     &= \left\|R_{QR}(\mathbf{S}^k, -\alpha_k\grad F(\mathbf{S}^k)) - (\mathbf{S}^k-\alpha_k\grad F(\mathbf{S}^k)) -\alpha_k\grad F(\mathbf{S}^k)\right\|_F\\
%     &\geq \left\|\alpha_k\grad F(\mathbf{S}^k)\right\|_F - \left\|R_{QR}(\mathbf{S}^k, -\alpha_k\grad F(\mathbf{S}^k)) - (\mathbf{S}^k-\alpha_k\grad F(\mathbf{S}^k))\right\|_F\\
%     &\geq \left\|\alpha_k\grad F(\mathbf{S}^k)\right\|_F - M \alpha_k^2 \left\|\grad F(\mathbf{S}^k)\right\|_F^2
% \end{align}
% Dividing by $\left\|\alpha_k\grad F(\mathbf{S}^k)\right\|_F^2$ and using Proposition~\ref{prop:alpha_grad}, we obtain
% \begin{align}
%     \lim \frac{\left\|\mathbf{S}^{k+1} - \mathbf{S}^k\right\|_F}{|\alpha_k|\left\|\grad F(\mathbf{S}^k)\right\|_F} \geq 1
% \end{align}
WLOG we assume that $\alpha_k \left\|\grad \widetilde{F}(\widetilde{\mathbf{S}}^{k}))\right\|_F \neq 0$ for all $k \geq 0$. Then, $\widetilde{\mathbf{S}}^{k+1} - \widetilde{\mathbf{S}}^{k} = \widetilde{R}_\EXP(\widetilde{\mathbf{S}}^{k}, -\alpha_k \grad \widetilde{F}(\widetilde{\mathbf{S}}^{k})) - \widetilde{\mathbf{S}}^{k}$ and we have
\begin{align}
    \left\|\widetilde{\mathbf{S}}^{k+1}-\widetilde{\mathbf{S}}^{k}\right\|_F &\geq \alpha_k \left\|\grad \widetilde{F}(\widetilde{\mathbf{S}}^{k}))\right\|_F - \left\|\widetilde{R}_\EXP(\widetilde{\mathbf{S}}^{k}, -\alpha_k \grad \widetilde{F}(\widetilde{\mathbf{S}}^{k})) - (\widetilde{\mathbf{S}}^{k} - \grad \widetilde{F}(\widetilde{\mathbf{S}}^{k})))\right\|_F\\
    \left\|\widetilde{\mathbf{S}}^{k+1}-\widetilde{\mathbf{S}}^{k}\right\|_F &\leq \alpha_k \left\|\grad \widetilde{F}(\widetilde{\mathbf{S}}^{k}))\right\|_F + \left\|\widetilde{R}_\EXP(\widetilde{\mathbf{S}}^{k}, -\alpha_k \grad \widetilde{F}(\widetilde{\mathbf{S}}^{k})) - (\widetilde{\mathbf{S}}^{k} - \grad \widetilde{F}(\widetilde{\mathbf{S}}^{k})))\right\|_F.
\end{align}
Using Proposition~\ref{prop:second_order_boundedness_of_Rtilde}, we obtain 
$$\lim \frac{\left\|\widetilde{\mathbf{S}}^{k+1}-\widetilde{\mathbf{S}}^{k}\right\|_F}{\alpha_k\left\|\grad \widetilde{F}(\widetilde{\mathbf{S}}^{k}))\right\|_F} = 1.$$
% \begin{equation}
%     \lim \frac{\left\|\widetilde{\mathbf{S}}^{k+1}-\widetilde{\mathbf{S}}^{k}\right\|_F}{\alpha_k\left\|\grad \widetilde{F}(\widetilde{\mathbf{S}}^{k}))\right\|_F} = 1.
% \end{equation}
It suffices to show that $\liminf \alpha_k > 0$. Let $\overline{\alpha}_k = \overline{\alpha}(\mathbf{S}^k) > 0$ where 
\begin{equation}
    \overline{\alpha}(\mathbf{S}) = \inf\{\alpha > 0| F(R_\EXP(\mathbf{S}, -\alpha \grad F(\mathbf{S}))) - F(\mathbf{S}) = -\gamma \alpha g(\nabla F(\mathbf{S}), \grad F(\mathbf{S}))\}.
\end{equation}
where $\overline{\alpha}(\mathbf{S})$ is well defined due to \citeb[Proposition 2.8]{schneider2015convergence} since $F$ extends to a continuously differentiable non-negative function on $\mathbb{R}^{md \times d}$ containing $\mathbb{O}(d)^m$. By Eq.~(\ref{eq:armijo_step}) and above equation, we have $\alpha_k = 1$ if $\overline{\alpha}_k \geq 1$ and $\alpha_k \geq \beta \overline{\alpha}_k$ if $\overline{\alpha}_k < 1$.
% \begin{equation}
%     \begin{matrix}
%         \alpha_k = 1 \text{ if } \overline{\alpha}_k \geq 1; & \alpha_k \geq \beta \overline{\alpha}_k \text{ if } \overline{\alpha}_k < 1.
%     \end{matrix}
% \end{equation}
It suffices to assume that $\overline{\alpha}_k < 1$ for all $k \geq 0$ and show that $\liminf \overline{\alpha}_k > 0$. From the above equation, it follows that $\overline{\alpha}_k \left\|\grad F(\mathbf{S}^{k}))\right\|_F \leq (\alpha_k/\beta)\left\|\grad F(\mathbf{S}^{k}))\right\|_F$ which combined with Proposition~\ref{prop:alpha_grad} implies that $\lim\overline{\alpha}_k \left\|\grad F(\mathbf{S}^k)\right\|_F = 0$.
% \begin{equation}
%     \lim\overline{\alpha}_k \left\|\grad F(\mathbf{S}^k)\right\|_F = 0.
% \end{equation}
By mean value theorem and the definition of $\overline{\alpha}_k$, there exist $\zeta_k \in (0,1)$ such that 
$$\mathbf{U}^k = \zeta_k(R_\EXP(\mathbf{S}^k, -\overline{\alpha}_k \grad F(\mathbf{S}^k)) - \mathbf{S}^k)$$
satisfies
\begin{align}
    (R_\EXP(\mathbf{S}, -\overline{\alpha}_k \grad F(\mathbf{S}))- \mathbf{S}^k)^T\nabla F(\mathbf{S}^k + \mathbf{U}_k) &= F(R_\EXP(\mathbf{S}, -\overline{\alpha}_k \grad F(\mathbf{S}))) - F(\mathbf{S}^k)\\
    &= -\gamma \overline{\alpha}_k g(\nabla F(\mathbf{S}^k), \grad F(\mathbf{S}^k)). \label{eq:Uk}
\end{align}
Moreover, for sufficiently large $k \geq 0$, $\overline{\alpha}_k\left\|\grad F(\mathbf{S}^k)\right\|_F < 1$, therefore using Proposition~\ref{prop:liu_pf} and the triangle inequality,
\begin{equation}
    \left\|\mathbf{U}^k\right\|_F \leq \left\|R_\EXP(\mathbf{S}^k, -\overline{\alpha}_k \grad F(\mathbf{S}^k)) - \mathbf{S}^k\right\|_F \leq e\overline{\alpha}_k\left\|\grad F(\mathbf{S}^k)\right\|_F.
\end{equation}
Then we obtain the following set of inequalities using the above inequality, the fact that $\nabla F$ is Lipschitz continuous with parameter $L_F \leq 2 \left\|\mathbf{C}\right\|_F$, using Cauchy-Schwarz inequality, Eq~(\ref{eq:Uk}) and the triangle inequality,

\begin{align}
    \overline{\alpha}_k^2&\left\|\grad F(\mathbf{S}^k)\right\|_F^2 \geq e^{-1}\left\|\mathbf{U}_k\right\|_F \left\|R_\EXP(\mathbf{S}^k, - \alpha_k \grad F(\mathbf{S}^k)) - \mathbf{S}^k\right\|_F\\
    &\geq (eL_F)^{-1} \left\|\nabla F(\mathbf{S}^k) - \nabla F(\mathbf{S}^k + \mathbf{U}_k)\right\|_F \left\|R_\EXP(\mathbf{S}^k, - \overline{\alpha}_k\grad F(\mathbf{S}^k)) - \mathbf{S}^k\right\|_F\\
    &\geq (eL_F)^{-1} |g(\nabla F(\mathbf{S}^k) - \nabla F(\mathbf{S}^k + \mathbf{U}_k), R_\EXP(\mathbf{S}^k, - \overline{\alpha}_k\grad F(\mathbf{S}^k)) - \mathbf{S}^k)|\\
    &= (eL_F)^{-1}|g(\nabla F(\mathbf{S}^k), R_\EXP(\mathbf{S}^k, - \overline{\alpha}_k\grad F(\mathbf{S}^k)) - \mathbf{S}^k) + \gamma \overline{\alpha}_k g(\nabla F(\mathbf{S}^k), \grad F(\mathbf{S}^k))|\\
    &\geq (1-\gamma)\overline{\alpha}_k(eL_F)^{-1} |g(\nabla F(\mathbf{S}^k),\grad F(\mathbf{S}^k))| -\\
    &\qquad (eL_F)^{-1}|g(\nabla F(\mathbf{S}^k), R_\EXP(\mathbf{S}^k, - \overline{\alpha}_k\grad F(\mathbf{S}^k)) - (\mathbf{S}^k - \overline{\alpha}_k\grad F(\mathbf{S}^k))|.
\end{align}
Combining with $g(\nabla F(\mathbf{S}^k),\grad F(\mathbf{S}^k)) = \left\|\grad F(\mathbf{S}^k)\right\|_F^2$ (proof of Proposition~\ref{prop:alpha_grad}), using Cauchy-Schwarz inequality and dividing by $\overline{\alpha}_k\left\|\grad F(\mathbf{S}^k)\right\|_F^2$, we obtain
\begin{equation}
    \overline{\alpha}_k \geq (1-\gamma)(eL_{F})^{-1} - (eL_{F})^{-1}  \frac{\left\|\nabla F(\mathbf{S})\right\|_F\left\|R_\EXP(\mathbf{S}^k, - \overline{\alpha}_k\grad F(\mathbf{S}^k)) - (\mathbf{S}^k - \overline{\alpha}_k\grad F(\mathbf{S}^k))\right\|_F}{\overline{\alpha}_k\left\|\grad F(\mathbf{S}^k)\right\|_F^2}.
\end{equation}
Finally, since $\left\|\nabla F(\mathbf{S})\right\|_F \leq 2\sqrt{md} \left\|\mathbf{C}\right\|_F$ and using Proposition~\ref{prop:second_order_boundedness_of_Rtilde},
% \begin{equation}
%     \left\|R_\EXP(\mathbf{S}^k, - \overline{\alpha}_k\grad F(\mathbf{S}^k)) - (\mathbf{S}^k - \overline{\alpha}_k\grad F(\mathbf{S}^k))\right\|_F \leq \overline{\alpha}_k^2\left\|\grad F(\mathbf{S}^k)\right\|_F^2,
% \end{equation}
we obtain 
\begin{equation}
    \liminf \overline{\alpha}_k \geq \frac{1-\gamma}{eL_F + 2\sqrt{md}\left\|\mathbf{C}\right\|_F} > 0.
\end{equation}
% \revdel{\proofof{Proposition~\ref{prop:morse_bott_1}}
% Since $\pi(\mathbf{S}^*)$ is non-degenerate, $\widetilde{F}$ being Morse-Bott at $\pi(\mathbf{S}^*)$ follows from \cite[Definition 6.5]{usevich2020approximate}. Since non-degenerate critical points of a smooth function are isolated from other critical points, it follows again from \citea{usevich2020approximate} that $\widetilde{F}$ is Morse-Bott in a sufficiently small neighborhood of $\pi(\mathbf{S}^*)$. 

% Then, since the action of $\mathbb{O}(d)$ on $\mathbb{O}(d)^m$ is free and the function $F$ is invariant under the action of $\mathbb{O}(d)$ ($\mathbf{S}\mathbf{Q} = \mathbf{S} \iff \mathbf{Q} = \mathbf{I}_d$ and $F(\mathbf{S}\mathbf{Q}) = F(\mathbf{S})$ for all $\mathbf{S} \in \mathbb{O}(d)^m, \mathbf{Q} \in \mathbb{O}(d)$), the result follows from \citeb[Section 15.2]{cohen_iga_norbury_2006}.% and \citea{austin1995morse}.
% }

\proofof{Lemma~\ref{lem:quadgrowth}}
Since $\mathbf{S}_0$ is a unique perfect alignment, all other perfect alignments are of the form $\mathbf{S}_0\mathbf{Q}$ where $\mathbf{Q}\in \mathbb{O}(d)$. Moreover, the null space of $\mathbf{C}_0$ is exactly the span of columns of $\mathbf{S}_0$. Therefore, we obtain the following decomposition of $\mathbf{C}_0 = \mathbf{U}_0\boldsymbol{\Lambda}_0 \mathbf{U}_0^T$ where $\mathbf{U}_0^T\mathbf{U}_0 = \mathbf{I}_{(m-1)d}$, $\mathbf{S}_0^T \mathbf{U}_0 = 0$ and $\boldsymbol{\Lambda}_0$ is a diagonal matrix containing the strictly positive eigenvalues of $\mathbf{C}_0$. Using the above decomposition, we have 
$$\Tr(\mathbf{C}_0\mathbf{S}\mathbf{S}^T) = \Tr(\mathbf{U}_0\boldsymbol{\Lambda}_0 \mathbf{U}_0^T\mathbf{S}\mathbf{S}^T) \geq \lambda_{d+1}(\mathbf{C}_0) \left\|\mathbf{U}_0^T\mathbf{S}\right\|_F^2.$$

\textbf{Claim:} $\left\|\mathbf{U}_0^T\mathbf{S}\right\|_F^2 \geq \frac{1}{2}\min_{\mathbf{Q} \in \mathbb{O}(d)}\left\|\mathbf{S}- \mathbf{S}_0\mathbf{Q}\right\|_F^2$. Since the union of the columns of $\mathbf{S}_0$ and $\mathbf{U}_0$ span $\mathbb{R}^{md}$ therefore there exist $\mathbf{R}_1 \in \mathbb{R}^{d \times d}$ and $\mathbf{R}_2 \in \mathbb{R}^{(m-1)d \times d}$ such that $\mathbf{S} = \mathbf{S}_0 \mathbf{R}_1 + \mathbf{U}_0 \mathbf{R}_2$. Let $\mathbf{Q}^* \in \mathbb{O}(d)$ be such that $\left\|\mathbf{S}- \mathbf{S}_0\mathbf{Q}^*\right\|_F^2 = \min_{\mathbf{Q} \in \mathbb{O}(d)^m} \left\|\mathbf{S}- \mathbf{S}_0\mathbf{Q}\right\|_F^2$. Then $\mathbf{Q}^* = \mathbf{U}_1\mathbf{V}_1^T$ where $\mathbf{R}_1 = \mathbf{U}_1\boldsymbol{\Sigma}_1\mathbf{V}_1^T$ is a singular vector decomposition of $\mathbf{R}_1$. Using $\mathbf{S}^T\mathbf{S} = \mathbf{S}_0^T\mathbf{S}_0 = m\mathbf{I}_d$ and $\mathbf{R}_1 = \mathbf{U}_1\boldsymbol{\Sigma}_1\mathbf{V}_1^T$,
\begin{align}
    \left\|\mathbf{U}_0^T\mathbf{S}\right\|_F^2 &= \left\|\mathbf{R}_2\right\|_F^2 = \left\|\mathbf{S}-\mathbf{S}_0\mathbf{R}_1\right\|_F^2\\
    &= md + m \left\|\mathbf{R}_1\right\|_F^2 - 2\Tr(\mathbf{S}\mathbf{R}_1^T\mathbf{S}_0^T)\\
    &= md + m\left\|\boldsymbol{\Sigma}\right\|_F^2 - 2\Tr(\mathbf{S}\mathbf{V}_1\boldsymbol{\Sigma}_1\mathbf{U}_1^T\mathbf{S}_0^T)
\end{align}
% \begin{equation}
%     \left\|\mathbf{U}_0^T\mathbf{S}\right\|_F^2 = md + m\left\|\boldsymbol{\Sigma}\right\|_F^2 - 2\Tr(\mathbf{S}\mathbf{V}_1\boldsymbol{\Sigma}_1\mathbf{U}_1^T\mathbf{S}_0^T)
% \end{equation}}
% $\left\|\mathbf{U}_0^T\mathbf{S}\right\|_F^2 = md + m\left\|\boldsymbol{\Sigma}\right\|_F^2 - 2\Tr(\mathbf{S}\mathbf{V}_1\boldsymbol{\Sigma}_1\mathbf{U}_1^T\mathbf{S}_0^T)$.
% \begin{align}
%     \left\|\mathbf{U}_0^T\mathbf{S}\right\|_F^2 = \left\|\mathbf{S}-\mathbf{S}_0\mathbf{R}_1\right\|_F^2 &= md + m \left\|\mathbf{R}_1\right\|_F^2 - 2\Tr(\mathbf{S}\mathbf{R}_1^T\mathbf{S}_0^T)\\
%     &= md + m\left\|\boldsymbol{\Sigma}\right\|_F^2 - 2\Tr(\mathbf{S}\mathbf{V}_1\boldsymbol{\Sigma}_1\mathbf{U}_1^T\mathbf{S}_0^T).
% \end{align}
Moreover, using the fact that $\mathbf{S}_0^T\mathbf{S}_0 = m\mathbf{I}_d$ and the definition of $\mathbf{Q}^*$,
% \begin{equation}
%     \revadd{\left\|\mathbf{S} - \mathbf{S}_0\mathbf{Q}^*\right\|_F^2 = \left\|\mathbf{U}_0\mathbf{R}_2\right\|_F^2 + \left\|\mathbf{S}_0(\mathbf{R}_1 - \mathbf{Q}^*)\right\|_F^2}
% \end{equation}
% \revadd{where $\left\|\mathbf{U}_0\mathbf{R}_2\right\|_F^2 = \left\|\mathbf{S} - \mathbf{S}_0\mathbf{R}_1\right\|_F^2 = \left\|\mathbf{R}_2\right\|_F^2$, and $\left\|\mathbf{S}_0(\mathbf{R}_1 - \mathbf{Q}^*)\right\|_F^2 = m\left\|\mathbf{R}_1 - \mathbf{Q}^*\right\|_F^2$, which reduces to $m\left\|\mathbf{I}_d-\boldsymbol{\Sigma}_1\right\|_F^2$ using $\mathbf{S}_0^T\mathbf{S}_0 = m\mathbf{I}_d$ and the definition of $\mathbf{Q}^*$
{\allowdisplaybreaks
\begin{align}
    \left\|\mathbf{S} - \mathbf{S}_0\mathbf{Q}^*\right\|_F^2 &= \left\|\mathbf{U}_0\mathbf{R}_2\right\|_F^2 + \left\|\mathbf{S}_0(\mathbf{R}_1 - \mathbf{Q}^*)\right\|_F^2 = \left\|\mathbf{S} - \mathbf{S}_0\mathbf{R}_1\right\|_F^2 + m\left\|\mathbf{R}_1 - \mathbf{Q}^*\right\|_F^2\\
    &= \left\|\mathbf{S} - \mathbf{S}_0\mathbf{R}_1\right\|_F^2 + m\left\|\mathbf{I}_d-\boldsymbol{\Sigma}_1\right\|_F^2\\
    &= \left\|\mathbf{S} - \mathbf{S}_0\mathbf{R}_1\right\|_F^2 + m(d + \left\|\boldsymbol{\Sigma}_1\right\|_F^2 - 2\Tr(\boldsymbol{\Sigma}_1))\\
    &= 2(md + m\left\|\boldsymbol{\Sigma}_1\right\|_F^2) - 2(\Tr(\mathbf{S}\mathbf{V}_1\boldsymbol{\Sigma}_1\mathbf{U}_1^T\mathbf{S}_0^T) + m\Tr(\boldsymbol{\Sigma}_1))
\end{align}
}
Overall, 
$$\left\|\mathbf{U}_0^T\mathbf{S}\right\|_F^2 - \frac{1}{2}\left\|\mathbf{S} - \mathbf{S}_0\mathbf{Q}^*\right\|_F^2 = m\Tr(\boldsymbol{\Sigma}_1) - \Tr(\mathbf{S}\mathbf{V}_1\boldsymbol{\Sigma}_1\mathbf{U}_1^T\mathbf{S}_0^T) = \sum_{1}^{m}(\Tr(\boldsymbol{\Sigma}_1) - \Tr(\mathbf{U}_1^T\mathbf{S}_{0_i}^T\mathbf{S}_i\mathbf{V}_1\boldsymbol{\Sigma}_1)).$$
Since $\max_{\mathbf{Q} \in \mathbb{O}(d)}\Tr(\mathbf{Q} \boldsymbol{\Sigma}_1) = \Tr(\boldsymbol{\Sigma}_1)$, the result follows.
% \begin{align}
%     \left\|\mathbf{U}_0^T\mathbf{S}\right\|_F^2 - \frac{1}{2}\left\|\mathbf{S} - \mathbf{S}_0\mathbf{Q}^*\right\|_F^2 &= m\Tr(\boldsymbol{\Sigma}_1) - \Tr(\mathbf{S}\mathbf{V}_1\boldsymbol{\Sigma}_1\mathbf{U}_1^T\mathbf{S}_0^T)\\
%     &= \sum_{1}^{m}(\Tr(\boldsymbol{\Sigma}_1) - \Tr(\mathbf{U}_1^T\mathbf{S}_{0_i}^T\mathbf{S}_i\mathbf{V}_1\boldsymbol{\Sigma}_1))\geq 0
% \end{align}
% where the last inequality follows from the fact that $\max_{\mathbf{Q} \in \mathbb{O}(d)}\Tr(\mathbf{Q} \boldsymbol{\Sigma}_1) = \Tr(\boldsymbol{\Sigma}_1)$.

\proofof{Lemma~\ref{lem:distS_0Sstar}}
The proof is motivated from \citeb[Proposition~4.32]{bonnans2013perturbation}. Define 
$$H(\mathbf{S}) = \Tr(\mathbf{C}\mathbf{S}\mathbf{S}^T) - \Tr(\mathbf{C}_0\mathbf{S}\mathbf{S}^T)$$
and note that $H(\mathbf{S})$ is Lipschitz with a Lipschitz constant bounded by $\left\|\nabla H(\mathbf{S})\right\|_F \leq 2m \left\|\mathbf{C}-\mathbf{C}_0\right\|_F$. 

Let $\mathbf{Q}^* \in \mathbb{O}(d)$ be such that 
$$\left\|\mathbf{S}^* - \mathbf{S}_0\mathbf{Q}^*\right\|_F = \min_{\mathbf{Q}\in\mathbb{O}(d)}\left\|\mathbf{S}^* - \mathbf{S}_0\mathbf{Q}\right\|_F.$$
Then, using the mean value theorem and the fact the $\mathbf{S}^*$ is an optimal alignment in the noisy setting, meaning $\Tr(\mathbf{C}\mathbf{S}^*\mathbf{S}^{*^T}) \leq \Tr(\mathbf{C}\mathbf{S}_0\mathbf{Q}^*(\mathbf{S}_0\mathbf{Q}^*)^T)$,
\begin{align}
    \Tr(\mathbf{C}_0\mathbf{S}^*\mathbf{S}^{*^T}) - \Tr(\mathbf{C}_0\mathbf{S}_0\mathbf{Q}^*(\mathbf{S}_0\mathbf{Q}^*)^T) &= H(\mathbf{S}_0\mathbf{Q}^*) - H(\mathbf{S}^*) + (\Tr(\mathbf{C}\mathbf{S}^*\mathbf{S}^{*^T}) - \Tr(\mathbf{C}\mathbf{S}_0\mathbf{Q}^*(\mathbf{S}_0\mathbf{Q}^*)^T))\\
    &\leq 2m \left\|\mathbf{C}-\mathbf{C}_0\right\|_F \left\|\mathbf{S}^*-\mathbf{S}_0\mathbf{Q}^*\right\|_F
\end{align}
% \begin{align}
%     &\Tr(\mathbf{C}_0\mathbf{S}^*\mathbf{S}^{*^T}) - \Tr(\mathbf{C}_0\mathbf{S}_0\mathbf{Q}^*(\mathbf{S}_0\mathbf{Q}^*)^T)\\
%     &= H(\mathbf{S}_0\mathbf{Q}^*) - H(\mathbf{S}^*) + (\Tr(\mathbf{C}\mathbf{S}^*\mathbf{S}^{*^T}) - \Tr(\mathbf{C}\mathbf{S}_0\mathbf{Q}^*(\mathbf{S}_0\mathbf{Q}^*)^T))\\
%     &\leq 2m \left\|\mathbf{C}-\mathbf{C}_0\right\|_F \left\|\mathbf{S}^*-\mathbf{S}_0\mathbf{Q}^*\right\|_F
% \end{align}
% where the last inequality follows from the mean value theorem and the fact the $\mathbf{S}^*$ is an optimal alignment in the noisy setting, meaning $\Tr(\mathbf{C}\mathbf{S}^*\mathbf{S}^{*^T}) \leq \Tr(\mathbf{C}\mathbf{S}_0\mathbf{Q}^*(\mathbf{S}_0\mathbf{Q}^*)^T)$. 
Combining with Lemma~\ref{lem:quadgrowth} and $\mathbf{C}_0\mathbf{S}_0 = 0$, we obtain 
$$(\lambda_{d+1}(\mathbf{C}_0)/2) \left\|\mathbf{S}^*- \mathbf{S}_0\mathbf{Q}^*\right\|_F^2 \leq 2m \left\|\mathbf{C}-\mathbf{C}_0\right\|_F \left\|\mathbf{S}^*-\mathbf{S}_0\mathbf{Q}^*\right\|_F.$$
% \begin{align}
%      \frac{\lambda_{d+1}(\mathbf{C}_0)}{2} \left\|\mathbf{S}^*- \mathbf{S}_0\mathbf{Q}^*\right\|_F^2 \leq 2m \left\|\mathbf{C}-\mathbf{C}_0\right\|_F \left\|\mathbf{S}^*-\mathbf{S}_0\mathbf{Q}^*\right\|_F.
% \end{align}
The result follows.

\proofof{Theorem~\ref{thm:rgd_noise_stability}} \revadd{The following bound holds from \citeb[Eq.~(5.8, 5.11, 5.12)]{chaudhury2015global}
\begin{equation}
    \min_{\mathbf{Q} \in \mathbb{O}(d)} \left\|\mathbf{S}_{spec}(\mathbf{C}) - \mathbf{S}_0\mathbf{Q}\right\|_F \leq \frac{4\pi\sqrt{md(d+1)}}{\lambda_{d+1}(\mathbf{C})}(K_1 \varepsilon + K_2\varepsilon^2). \label{eq:spec_bound}
\end{equation}
Let $\mathbf{Q}^*$ be such that $\left\|\mathbf{S}^* - \mathbf{S}_0\mathbf{Q}^*\right\|_F = \min_{\mathbf{Q}\in\mathbb{O}(d)}\left\|\mathbf{S}^* - \mathbf{S}_0\mathbf{Q}\right\|_F$. Then,
\begin{align}
    \left\|\mathbf{S}_{spec}(\mathbf{C}) - \mathbf{S}^*\mathbf{Q}\right\|_F &\leq \left\|\mathbf{S}_{spec}(\mathbf{C}) - \mathbf{S}_0\mathbf{Q}^*\mathbf{Q}\right\|_F + \left\|\mathbf{S}^*\mathbf{Q} - \mathbf{S}_0\mathbf{Q}^*\mathbf{Q}\right\|_F\\
    &\leq \left\|\mathbf{S}_{spec}(\mathbf{C}) - \mathbf{S}_0\mathbf{Q}^*\mathbf{Q}\right\|_F + \left\|\mathbf{S}^* - \mathbf{S}_0\mathbf{Q}^*\right\|_F
\end{align}
Minimizing the above over $\mathbf{Q}$, using the fact that 
$$\min_{\mathbf{Q}\in\mathbb{O}(d)}\left\|\mathbf{S}_{spec}(\mathbf{C}) - \mathbf{S}_0\mathbf{Q}^*\mathbf{Q}\right\|_F = \min_{\mathbf{Q}\in\mathbb{O}(d)}\left\|\mathbf{S}_{spec}(\mathbf{C}) - \mathbf{S}_0\mathbf{Q}\right\|_F,$$
followed by Eq.~(\ref{eq:spec_bound}), Lemma~\ref{lem:distS_0Sstar} and Theorem~\ref{thm:rgd_conv2}, the result follows.}