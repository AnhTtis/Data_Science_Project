%The organization of the section is as follows.
% We start by deriving some important consequences of the noiseless setting in Section~\ref{subsec:noiseless_conseq} which are used in the subsequent sections.
% In Section~\ref{subsec:loc_glob_rigid}, under a mild assumption on the structure of the local views, we show that the non-degeneracy of a perfect alignment is equivalent to the infinitesimal rigidity of the resulting realization and to the local rigidity of a generic realization. We also show that the uniqueness of a perfect alignment is equivalent to the global rigidity \citea{gortler2010affine} of the resulting realization.
% \revadd{We compare our results with previously established findings on the affine rigidity of a realization in \cite{chaudhury2015global,gortler2010affine}.}
% \revadd{Then, in Section~\ref{subsec:non_deg_noiseless_setting} and Section~\ref{subsec:uniq_noiseless_setting}, we provide necessary and sufficient conditions on the overlapping structure of the views for the non-degeneracy and uniqueness of a perfect alignment.
% Combined with the results in Section~\ref{subsec:loc_glob_rigid}, we obtain conditions on a perfect alignment for the resulting realization to be infinitesimally/locally/globally rigid. These conditions should be contrasted with those for the affine rigidity of a realization, as presented in \citea{zha2009spectral}.}
We start by deriving some important consequences of the noiseless setting. Under a mild assumption on the structure of the local views, we show that the non-degeneracy and uniqueness of a perfect alignment is equivalent to certain notions of rigidity of the resulting realization (Figure~\ref{fig:rigidity_flow}). We then provide necessary and sufficient conditions on the overlapping structure of the views for the non-degeneracy and uniqueness of a perfect alignment. \revadd{Consequently, we obtain conditions on a perfect alignment for the resulting realization to be infinitesimally/locally/globally rigid. These should be contrasted with the ones in \citea{zha2009spectral}, for the affine rigidity of a realization.}
\begin{figure}[H]
    \centering
    \resizebox{0.65\textwidth}{!}{%
    \begin{tikzpicture}[
        NodeA/.style={rectangle, draw=black!60, fill=green!5, very thick, minimum size=7mm, text width=3cm},
        NodeB/.style={rectangle, draw=black!60, fill=red!5, very thick, minimum size=5mm, text width=3cm},
        NodeC/.style={rectangle, draw=black!60, fill=blue!5, very thick, minimum size=7mm, text width=3cm},
        NodeD/.style={rectangle, draw=black!60, fill=blue!5, very thick, minimum size=7mm, text width=5cm},
        every text node part/.style={align=center}
        ]
        %Nodes
        \node[NodeB]        (irigid)  {infinitesimally rigid $\Theta(\mathbf{S})$};
        \node[NodeC]      (rankR) [above=0.85cm of irigid] {$\rank(\boldsymbol{\mathcal{R}}(\mathbf{S})) \geq nd - d(d+1)/2$};
        \node[NodeB]      (lrigid)       [below=of irigid] {locally rigid $\Theta(\mathbf{S})$};
        \node[NodeB]      (grigid)   [below=of lrigid]        {globally rigid $\Theta(\mathbf{S})$};
        \node[NodeB]        (arigid)       [below=of grigid] {affinely rigid $\Theta(\mathbf{S})$};
        
        
        \node[NodeA]        (nondegS)       [right=3cm of irigid] {non-degenerate $\mathbf{S}$};
        \node[NodeA]        (strictS)       [right=3cm of lrigid] {$\pi(\mathbf{S})$ is a strict minimum of $\widetilde{F}$};
        \node[NodeC]        (rankLbb)       [above=of nondegS] {$\rank(\boldsymbol{\mathbb{L}}(\mathbf{S})) = (m-1)d(d-1)/2$};
        \node[NodeA]        (uniqueS)       [right=3cm of grigid] {unique $\mathbf{S}$};
        \node[NodeC]        (rankC)       [right=3cm of arigid] {$\rank(\mathbf{C}) = (m-1)d$};
        \coordinate[below right=2cm and 0.5cm of rankLbb]  (rankLbb0) ;
        \coordinate[above right=2cm and 0.5cm of rankC]  (rankC0) ;

        \path (rankLbb) -- node (rankLbbiffnondegS) {Proposition~\ref{prop:noiseless_setting1}} (nondegS);
        \path (nondegS) -- node (nondegSimpstrictS) {Trivial} (strictS);
        \path (lrigid.20) -- node (lrigidimpirigid) {Generic $\Theta(\mathbf{S})$} (irigid.330);
        
        %Lines
        %\draw[-implies,double equal sign distance, line width=0.4mm] (irigid.east) --(irigidimpnondegS)-- (nondegS.west);
        \draw[implies-implies, double, line width=0.4mm] (irigid.east) -- node [midway,above] {Theorem~\ref{thm:inf_rigid}} (nondegS.west);
        \draw[-implies,double, line width=0.4mm] (irigid.230) --node[midway,left] {\citea{toth2017handbook}} (lrigid.145);
        \draw[-implies,double, line width=0.4mm] (lrigid.20) -- (lrigidimpirigid) -- (irigid.330);
        %\draw[implies-,double equal sign distance, line width=0.4mm] (strictS.west)--(lrigidiffstrictS);
        %\draw[-implies,double equal sign distance, line width=0.4mm] (lrigidiffstrictS)--(lrigid.east);
        \draw[implies-implies, double, line width=0.4mm] (lrigid.east) -- node[midway,above] {Proposition~\ref{prop:non_deg_views}} (strictS.west);
        \draw[-implies,double, line width=0.4mm] (grigid.north) -- node [midway,left] {\citea{gortler2010affine}} (lrigid.south);
        \draw[implies-implies,double, line width=0.4mm] (irigid.north) -- node [midway,left] {\citea{toth2017handbook}} (rankR.south);
        %\draw[-implies,double, line width=0.4mm](uniqueSiffgrigid)--(grigid.east);
        %\draw[implies-,double, line width=0.4mm] (uniqueS.west)--(uniqueSiffgrigid);
        \draw[implies-implies,double, line width=0.4mm] (grigid.east)--node[midway,above]{Theorem~\ref{thm:glob_rigid}} (uniqueS.west);
        \draw[-implies,double, line width=0.4mm] (arigid.north) -- node [midway,left] {\citea{gortler2010affine}} (grigid.south);

        \draw[-,double, line width=0.4mm] (rankC.east) -| (rankC0);
        \draw[-,double, line width=0.4mm] (rankLbb0) --node[above,midway,sloped] {Corollary~\ref{cor:noiseless_setting1}} (rankC0);
        \draw[-implies,double, line width=0.4mm] (rankLbb0) |- (rankLbb.east) ;

        \draw[implies-implies,double, line width=0.4mm] (rankC.west) -- node [midway, above] {\citea{chaudhury2015global,zha2009spectral}} (arigid.east);

        \draw[implies-,double, line width=0.4mm] (rankLbb.south)--(rankLbbiffnondegS);
        \draw[-implies,double, line width=0.4mm] (rankLbbiffnondegS)--(nondegS.north);
        \draw[-implies,double, line width=0.4mm] (nondegS.south) --(nondegSimpstrictS)-- (strictS.north);
        
    \end{tikzpicture}
    }
    \caption{\revadd{The implications between the type of a perfect alignment $\mathbf{S}$ and the rigidity of the resulting realization $\Theta(\mathbf{S})$.}}
    \label{fig:rigidity_flow}
\end{figure}
\subsection{Consequences of Noiseless Setting}
\label{subsec:noiseless_conseq}
As discussed in \citea{chaudhury2015global}, in the noiseless case, the patch-stress matrix $\mathbf{C}$ is constructed from $\Gamma$ and clean measurements. In particular, there exists a perfect alignment $\mathbf{S}$ such that $F(\mathbf{S}) = 0$.
%The following consequences of the noiseless setting will play a crucial role in the subsequent sections.
\begin{prop}
\label{prop:noiseless_setting1}
Let $\mathbf{S}$ be a perfect alignment. Then $\widehat{\mathbf{C}}(\mathbf{S}) = 0$, $\mathbf{L}(\mathbf{S}) = \mathbf{C}(\mathbf{S})$ (see Eq.~(\ref{eq:L_of_S})) and $\mathbb{L}(\mathbf{S}) = \overline{\mathbf{P}}(\mathbf{I}_d \otimes  (\mathbf{P}\mathbf{C}(\mathbf{S})\mathbf{P}^T))\overline{\mathbf{P}}^T$ (see Eq.~(\ref{eq:mathbb_L})). Consequently, it is easy to deduce from Remark~\ref{rmk:C_S_structure} that $\mathbf{L}(\mathbf{S}) \succeq 0$ and $\mathbb{L}(\mathbf{S}) \succeq 0$. \revadd{It follows from Theorem~\ref{thm:non_deg_loc_min} that $\mathbf{S}$ is non-degenerate if and only if $\rank(\mathbb{L}(\mathbf{S})) = (m-1)d(d-1)/2$.}
\end{prop}
% \begin{rmk}
% \label{prop:noiseless_setting1}
% Due to the above proposition and Remark~\ref{rmk:C_S_structure}, it is easy to deduce that for a perfect alignment $\mathbf{S}$, $\mathbf{L}(\mathbf{S}) \succeq 0$ and $\mathbb{L}(\mathbf{S}) \succeq 0$. Consequently, from Theorem~\ref{thm:non_deg_loc_min}, $\mathbf{S}$ is non-degenerate if and only if $\rank(\mathbb{L}(\mathbf{S})) = (m-1)d(d-1)/2$.
% \end{rmk}

\begin{rmk}
\label{rmk:C1Sp}
\revadd{
Following Remark~\ref{rmk:mathcalLpp} and due to the above proposition, it is easy to deduce that for a perfect alignment $\mathbf{S}$, $\mathbf{\mathcal{L}}(\mathbf{S})_{pp}$ is exactly the patch-stress matrix (see Eq.~(\ref{eq:GPOP})) of the one-dimensional patch framework  $(\Gamma, (\mathbf{S}_i^T\mathbf{x}_{k,i}(p)))$.}
\end{rmk}

\revadd{As a direct corollary of Proposition~\ref{prop:HessVicinity}, we obtain a bound on the neighborhood of a non-degenerate perfect alignment where the Hessian is positive~definite.}
\begin{cor}
\label{cor:HessVicinity}
\revadd{Let $\mathbf{S}$ be a non-degenerate perfect alignment.
%Then, using the fact that $\mathbf{C}\mathbf{S} = 0$,  Proposition~\ref{prop:noiseless_setting1} and Eq.~(\ref{eq:C_of_S}),
As in Proposition~\ref{prop:HessVicinity}, define $c_1 = \max_{1}^{m}\sigma_{\max}(\mathbf{C}_{k,:})$, $c_3 \coloneqq \sigma_{\max}(\mathbf{C})$, $\lambda_{0_-}(\mathbf{S}) \coloneqq \lambda_{d(d-1)/2+1}(\mathbb{L}(\mathbf{S}))$ and $\lambda_{0_+}(\mathbf{S}) \coloneqq \lambda_{md(d-1)/2}(\mathbb{L}(\mathbf{S}))$ (from Proposition~\ref{prop:noiseless_setting1} and Eq.~(\ref{eq:omega^TmbbLomega}) it is easy to deduce that $0 < \lambda_{0_-}(\mathbf{S}) \leq \lambda_{0_+}(\mathbf{S}) \leq 2\lambda_{md}(\mathbf{C})$). Let $\zeta \in (0,1)$ be fixed and define}
\begin{equation}
\label{eq:delta}
\revadd{\delta_0(\mathbf{S}) \coloneqq |\lambda_{0_-}(\mathbf{S})|/ 2(c_1 + 2 c_{3}).}
\end{equation}
\revadd{If $\mathbf{O} \in \mathbb{O}(d)^m$ satisfies $\min_{\mathbf{Q}\in\mathbb{O}(d)}\left\|\mathbf{O}-\mathbf{S}\mathbf{Q}\right\|_F \leq \zeta\delta_0(\mathbf{S})$, then for all $\widetilde{\mathbf{Z}} \in T_{\widetilde{\mathbf{O}}}\mathbb{O}(d)^m/_{\sim}$,}
\begin{equation}
    \revadd{(1-\zeta)\lambda_{0_-}(\mathbf{S})\widetilde{g}(\widetilde{\mathbf{Z}}, \widetilde{\mathbf{Z}}) \leq \widetilde{g}(\Hess \widetilde{F}(\widetilde{\mathbf{O}})[\widetilde{\mathbf{Z}}],\widetilde{\mathbf{Z}}) \leq (\lambda_{0_+}(\mathbf{S}) + \zeta\lambda_{0_-}(\mathbf{S})) \widetilde{g}(\widetilde{\mathbf{Z}}, \widetilde{\mathbf{Z}}).}
\end{equation}
% Suppose $\mathbf{S}$ is a non-degenerate perfect alignment. Then, using the fact that $\mathbf{C}\mathbf{S} = 0$,  Proposition~\ref{prop:noiseless_setting1} and Eq.~(\ref{eq:C_of_S}), we have $\delta_1 = \max_{1}^{m}\sigma_{\max}(\mathbf{C}_{k,:})$, $ \delta_2(\mathbf{S}) = 0$ and $\delta_3(\mathbf{S}) \equiv \delta_3 =  \sigma_{\max}(\mathbf{C})$.
% Let $\lambda \coloneqq \lambda_{d(d-1)/2+1}(\mathbb{L}(\mathbf{S})) < 0$ (follows from Theorem~\ref{thm:non_deg_loc_min}). If $\mathbf{O} \in \mathbb{O}(d)^m$ is such that $\min_{\mathbf{Q}\in\mathbb{O}(d)}\left\|\mathbf{O}-\mathbf{S}\mathbf{Q}\right\|_F < (2(\delta_1 + 2 \delta_{3})))^{-1}|\lambda|$, then $\Tr(\boldsymbol{\Omega}^T(\mathbf{L}(\mathbf{O})+\mathbf{L}(\mathbf{O})^T)\boldsymbol{\Omega}) < 0$ for all $\boldsymbol{\Omega} = [\boldsymbol{\Omega}_i]_1^m$ such that $\boldsymbol{\Omega}_i \in \Skew(d)$, $\sum_1^m \boldsymbol{\Omega}_i = 0$ and not all $\boldsymbol{\Omega}_i$ equal zero.
\end{cor}

Finally, \revadd{using Proposition~\ref{prop:one_all1} and Proposition~\ref{prop:noiseless_setting1},} we provide a simplified characterization of a non-degenerate perfect alignment that will be useful in proving the subsequent results. First, similar to \citea{zha2009spectral}, we define a certificate of $\mathbf{L}(\mathbf{S})$.
\begin{dfn}
\label{def:LScertificate}
An $\boldsymbol{\Omega} \in \Skew(d)^m$ is said to be a certificate of $\mathbf{L}(\mathbf{S})$ if $\mathbf{L}(\mathbf{S})\boldsymbol{\Omega} = 0$. It is a trivial certificate if $\boldsymbol{\Omega}_i = \boldsymbol{\Omega}_0$ for all $i \in [1,m]$ and for some $\boldsymbol{\Omega}_0 \in \Skew(d)$.
\end{dfn}

% Then the characterization of a non-degenerate perfect alignment (obtained trivially from Proposition~\ref{prop:one_all1} and ~\ref{prop:noiseless_setting1}) is as follows.
\begin{prop}
\label{prop:non_deg_triv_cert}
If $\mathbf{S}$ is a perfect alignment then $\mathbf{S}$ is non-degenerate iff every certificate of $\mathbf{L}(\mathbf{S})$ is trivial.
\end{prop}
% \revdel{From Corollary~\ref{cor:suff_non_deg_loc_min}, Proposition~\ref{prop:noiseless_setting1} and Remark~\ref{rmk:C_S_structure}, it follows that
% \begin{cor}
% \label{cor:noiseless_setting1_old}
% If $\mathbf{C}$ is of rank $(m-1)d$ then every perfect alignment of $F$ is non-degenerate.
% \end{cor}
% }
%%
%%
%%
%%
%%
%% 
%%
%%
%%
%%
%%
%%
% \begin{figure}[h]
%     \centering
%     \begin{tikzpicture}[
%         NodeA/.style={rectangle, draw=black!60, fill=green!5, very thick, minimum size=7mm, text width=3cm},
%         NodeB/.style={rectangle, draw=black!60, fill=red!5, very thick, minimum size=5mm, text width=3cm},
%         NodeC/.style={rectangle, draw=black!60, fill=blue!5, very thick, minimum size=7mm, text width=3cm},
%         NodeD/.style={rectangle, draw=black!60, fill=blue!5, very thick, minimum size=7mm, text width=5cm},
%         every text node part/.style={align=center}
%         ]
%         %Nodes
%         \node[NodeB]      (grigid)                              {globally rigid $\Theta(\mathbf{S})$};
%         \node[NodeB]      (lrigid)       [right=of grigid] {locally rigid $\Theta(\mathbf{S})$};
%         \node[NodeB]        (arigid)       [left=of grigid] {affinely rigid $\Theta(\mathbf{S})$};
%         \node[NodeA]        (uniqueS)       [above=of grigid] {unique $\mathbf{S}$};
%         \node[NodeA]        (strictS)       [above=of lrigid] {$\pi(\mathbf{S})$ is a strict minimum of $\widetilde{F}$};
%         \node[NodeC]        (nondegS)       [above=of strictS] {non-degenerate $\mathbf{S}$};
%         \node[NodeA]        (rankC)       [above=of arigid] {$\rank(\mathbf{C}) = (m-1)d$};
%         \node[NodeD]        (rankLbb)       [above=of uniqueS, xshift=-3.25cm, yshift=0.25cm] {$\rank(\boldsymbol{\mathbb{L}}(\mathbf{S})) = (m-1)d(d-1)/2$};
%         \node[NodeB]        (irigid)       [below=of lrigid] {infinitesimally rigid $\Theta(\mathbf{S})$};
%         \path (rankC) -- node (rankCimprankLbb) {Corollary~\ref{cor:noiseless_setting1}} (rankLbb);
%         \path (rankLbb) -- node (rankLbbiffnondegS) {Theorem~\ref{thm:loc_rigid}} (nondegS);
%         \path (nondegS) -- node (nondegSimpstrictS) {Trivial} (strictS);
%         \path (strictS) -- node (nondegSifflrigid){Proposition~\ref{prop:non_deg_views}} (lrigid);
%         \path (uniqueS) -- node (uniqueSiffgrigid) {Theorem~\ref{thm:glob_rigid}} (grigid);
%         \path (irigid) -- node (irigidimpnondegS) {Theorem~\ref{thm:inf_rigid}} (nondegS);
%         %Lines
%         \draw[implies-implies,double equal sign distance, line width=0.4mm] (rankC.south) -- node [text width=1cm,midway,right] {\citea{chaudhury2015global,zha2009spectral}} (arigid.north);
%         \draw[-implies,double equal sign distance, line width=0.4mm] (grigid.east) -- node [text width=1cm,midway,above] {\citea{gortler2010affine}} (lrigid.west);
%         \draw[-implies,double equal sign distance, line width=0.4mm] (arigid.east) -- node [text width=1cm,midway,above] {\citea{gortler2010affine}} (grigid.west);
%         \draw[implies-,double equal sign distance, line width=0.4mm] (uniqueS.south)--(uniqueSiffgrigid);
%         \draw[-implies,double equal sign distance, line width=0.4mm](uniqueSiffgrigid)--(grigid.north);
%         \draw[implies-,double equal sign distance, line width=0.4mm] (strictS.south)--(nondegSifflrigid);
%         \draw[-implies,double equal sign distance, line width=0.4mm] (nondegSifflrigid)--(lrigid.north);
%         \draw[-implies,double equal sign distance, line width=0.4mm] (rankC.north)--(rankCimprankLbb)--(rankLbb.south);
%         \draw[implies-,double equal sign distance, line width=0.4mm] (rankLbb.east)--(rankLbbiffnondegS);
%         \draw[-implies,double equal sign distance, line width=0.4mm] (rankLbbiffnondegS)--(nondegS.west);
%         \draw[-implies,double equal sign distance, line width=0.4mm] (nondegS.south) --(nondegSimpstrictS)-- (strictS.north);
%         \draw[-implies,double equal sign distance, line width=0.4mm] (irigid.south) --(irigidimpnondegS)-- (nondegS.north);
%     \end{tikzpicture}
%     \caption{The implications between the type of a perfect alignment $\mathbf{S}$ and the rigidity of the resulting realization $\Theta(\mathbf{S})$.}
%     \label{fig:rigidity_flow}
% \end{figure}

\subsection{Rigidity of a Realization}
\label{subsec:loc_glob_rigid}
In the following, we reveal the relation between non-degenerate and unique perfect alignment with the various notions of the rigidity of the resulting realization. \revadd{These are summarized in Figure~\ref{fig:rigidity_flow}.}
\revadd{Throughout the rest of this work, we assume the following.}
\begin{assump}
\label{assump:non_deg_views}
\revadd{Each view is affinely non-degenerate i.e. has at least $d+1$ points whose affine span has a rank of $d$.}
\end{assump}

\revadd{Consequently, the perfect alignment of the local views can be uniquely determined by their realization. This can be easily inferred from the following result.}
\begin{prop}
\label{prop:non_deg_views}
\revadd{Let $\mathbf{B}_{i,i}$, $i \in [1,m]$, be as in Definition~\ref{def:Bij}. Define $\varrho = (\sum_1^m 1/\sigma_{\min}(\mathbf{B}_{i,i}\mathbf{B}_{i,i}^T)^{2})^{1/2}$. Then for perfect alignments $\mathbf{S}$ and $\mathbf{O}$, and the corresponding realizations $\Theta(\mathbf{S})$ and $\Theta(\mathbf{O})$, $\left\|\mathbf{S} - \mathbf{O}\right\|_F \leq \varrho \left\|\Theta(\mathbf{S}) - \Theta(\mathbf{O})\right\|_F$. In particular, $\Theta(\mathbf{S}) = \Theta(\mathbf{O})$ if and only if $\mathbf{S} = \mathbf{O}$.}
\end{prop}
% \revadd{Due to Definition~\ref{def:realization}, for any two perfect alignments $\mathbf{S}, \mathbf{O} \in \mathbb{O}(d)^m$, we have}
% {\small
% \begin{align}
%     \mathbf{0}_{d} = \textstyle\argmmin_{\mathbf{t} \in \mathbb{R}^d} \left\|\Theta(\mathbf{O}) - \mathbf{Q}^T\Theta(\mathbf{S}) - \mathbf{t}\mathbf{1}_n^T\right\|_F &= \left\|\Theta(\mathbf{O}) - \mathbf{Q}^T\Theta(\mathbf{S})\right\|_F = \left\|\Theta(\mathbf{O}) - \Theta(\mathbf{S}\mathbf{Q})\right\|_F. \label{eq:loc_rigid_pre}
% \end{align}
% }
% \begin{equation}
%     \revadd{\textstyle\min_{\substack{\mathbf{Q} \in \mathbb{O}(d)\\\mathbf{t} \in \mathbb{R}^d}} \left\|\Theta(\mathbf{O}) - \mathbf{Q}^T\Theta(\mathbf{S}) - \mathbf{t}\mathbf{1}_n^T\right\|_F = \textstyle\min_{\mathbf{Q} \in \mathbb{O}(d)} \left\|\Theta(\mathbf{O}) - \Theta(\mathbf{S}\mathbf{Q})\right\|_F.} \label{eq:loc_rigid_pre}
% \end{equation}

Now we define various notions of the rigidity of a realization $\Theta(\mathbf{S})$. Although phrased differently, the definitions are the same as those in \citea{toth2017handbook, gortler2010affine, chaudhury2015global}.
% \begin{dfn}
% Let $\mathbf{S}$ be a global minimum. Then $\Theta(\mathbf{S})$ is said to be locally rigid if there exist $\epsilon > 0$ such that for any $Y = [y_k]_1^n$ with $\left\|Y-\Theta(\mathbf{S})\right\|_F < \epsilon$ and
% \begin{align}
%     y_k = O_i^Tx_{k}(\mathbf{S}) + v_i, (k,i) \in E
% \end{align}
% for another global minimum $\{O_i\}_1^m \subseteq \mathbb{O}(d)$ and $\{v_i\}_1^m \subseteq \mathbb{R}^d$ (i.e. $Y = X([\mathbf{S}_iO_i]_1^m)$), we have $Y = \Theta(\mathbf{S})Q = \Theta(\mathbf{S}\mathbf{Q})$ for some $\mathbf{Q} \in \mathbb{O}(d)$.
% \end{dfn}
\begin{dfn}
\label{def:inf_rigid}
\revadd{Let $\mathbf{S}$ be a perfect alignment. Then $\Theta(\mathbf{S}) = (\mathbf{x}_k(\mathbf{S}))_1^n$ is infinitesimally rigid if there does not exist a perturbation $(\mathbf{p}_k)_1^n \subseteq \mathbb{R}^d$ satisfying:
\begin{enumerate}[leftmargin=*]
    \item $(\mathbf{p}_k)_1^n$ is is not a trivial perturbation (it is a trivial perturbation if there exist $\boldsymbol{\Omega} \in \Skew(d)$ and $\mathbf{t} \in \mathbb{R}^d$ such that $\mathbf{p}_k = \boldsymbol{\Omega}\mathbf{x}_k + \mathbf{t}$),
    \item and for all $(k_1,i), (k_2,i) \in E(\Gamma)$, $(\mathbf{x}_{k_1}(\mathbf{S})-\mathbf{x}_{k_2}(\mathbf{S}))^T(\mathbf{p}_{k_1}-\mathbf{p}_{k_2}) = 0$.
\end{enumerate}
}
\begin{rmk}
\label{rmk:inf_rigid}
\revadd{The above two conditions can be described in terms of the rank of the so-called rigidity matrix $\boldsymbol{\mathcal{R}}(\mathbf{S})$ \citea{toth2017handbook}. The rigidity matrix has a row for each triplet $(k_1, k_2, i)$ satisfying $(k_1,i), (k_2,i) \in E(\Gamma)$, and the $k_1$th and $k_2$th the blocks of the row are $(\mathbf{x}_{k_1}(\mathbf{S})-\mathbf{x}_{k_2}(\mathbf{S}))^T$ and $(\mathbf{x}_{k_2}(\mathbf{S})-\mathbf{x}_{k_1}(\mathbf{S}))^T$, respectively. Overall, the sparse matrix $\boldsymbol{\mathcal{R}}(\mathbf{S})$ has $\sum_{1}^{m}{n_i \choose 2}$ rows and $nd$ columns, and the realization $\Theta(\mathbf{S}) = (\mathbf{x}_k(\mathbf{S}))_1^n$ is infinitesimally rigid if and only if $\rank(\boldsymbol{\mathcal{R}}(\mathbf{S})) \geq nd - d(d+1)/2$.}
\end{rmk}
\end{dfn}
For the following definitions, we use the facts due to Definition~\ref{def:realization}: (i) $\mathbf{Q}^T\Theta(\mathbf{S}) = \Theta(\mathbf{S}\mathbf{Q})$ for any $\mathbf{Q} \in \mathbb{R}^{d \times d}$ and (ii) $\mathbf{0}_{d} = \textstyle\argmin_{\mathbf{t} \in \mathbb{R}^d} \left\|\Theta(\mathbf{O}) - \mathbf{Q}^T\Theta(\mathbf{S}) - \mathbf{t}\mathbf{1}_n^T\right\|_F$.
\begin{dfn}
\label{def:loc_rigid}
Let $\mathbf{S}$ be a perfect alignment. Then $\Theta(\mathbf{S})$ is locally rigid if there exists $\epsilon > 0$ such that for any other perfect alignment $\mathbf{O} \in \mathbb{O}(d)^m$ with $\left\|\Theta(\mathbf{O})-\Theta(\mathbf{S})\right\|_F < \epsilon$, we have $\Theta(\mathbf{O})$ to be a rigid transformation of $\Theta(\mathbf{S})$ or equivalently $\Theta(\mathbf{O})  = \Theta(\mathbf{S}\mathbf{Q})$ for some $\mathbf{Q} \in \mathbb{O}(d)$.
\end{dfn}
\begin{dfn}
\label{def:glob_rigid}
Let $\mathbf{S}$ be a perfect alignment. Then $\Theta(\mathbf{S})$ is globally rigid if for any other perfect alignment $\mathbf{O} \in \mathbb{O}(d)^m$ we have $\Theta(\mathbf{O}) = \Theta(\mathbf{S}\mathbf{Q})$ for some $\mathbf{Q} \in \mathbb{O}(d)$.
\end{dfn}
\begin{dfn}
\label{def:affine_rigid}
\revadd{Let $\mathbf{S}$ be a perfect alignment. Then $\Theta(\mathbf{S})$ is affinely rigid if for any realization $\mathbf{Y} \in \mathbb{R}^{d \times n}$ satisfying: for each $i \in [1,m]$ there exist an affine transform $\mathbf{A}_i$ such that $\mathbf{Y}_k = \mathbf{A}_i(\mathbf{x}_{k,i})$, we have $\mathbf{Y} = \mathbf{A}\Theta(\mathbf{S})$ for some global affine transform~$\mathbf{A}$.}
\end{dfn}
% \begin{figure}[H]
%     \centering
%     \begin{subfigure}[b]{0.4\textwidth}
%          \centering
%          \includegraphics[width=0.4\textwidth,keepaspectratio]{fig/fig0/glob_rig_not_aff_rig.png}
%          \caption{}
%          \label{fig:glob_not_affine}
%      \end{subfigure}
%      \begin{subfigure}[b]{0.4\textwidth}
%          \centering
%          \includegraphics[width=0.7\textwidth,keepaspectratio]{fig/fig0/degenerate_config.png}
%          \caption{}
%          \label{fig:locrigiddeg}
%      \end{subfigure}
%     \caption{(a) An example of a realization \citea{gortler2010affine} that is globally rigid but not affinely rigid. (b) An example of a degenerate perfect alignment of three views in two dimensions ($\rank(\mathbb{L}(\mathbf{S})) < 2$) with locally rigid realization. These views should be considered affinely non-degenerate, satisfying Assumption~\ref{assump:non_deg_views}. For clarity, only the points in the overlapping regions are shown.}
% \end{figure}
%For completeness, we define the affine rigidity of a realization $\Theta(\mathbf{S})$ as stated in \citeb[Definition 3.2]{chaudhury2015global} (also see \cite{zha2009spectral} and \cite{gortler2010affine}).
\revadd{From the above definitions, it is easy to see that an affinely rigid realization is globally rigid which in turn is locally rigid.} Examples of realizations that are not locally rigid, locally rigid but not globally rigid and \revadd{globally rigid but not affinely rigid} are provided in Figure~\ref{fig:nec_cond_loc_rigid_of_views}, Figure~\ref{fig:suff_cond_views_non_deg}, Figure~\ref{fig:G_star_1} and \revadd{\cite[Figure~3]{gortler2010affine}} respectively.
% \revdel{\begin{assump}
% \label{assump:non_deg_views_old}
% For perfect alignments $\mathbf{S}$ and $\mathbf{O}$, $\Theta(\mathbf{S}) = \Theta(\mathbf{O}) \iff \mathbf{S} = \mathbf{O}$. 
% \end{assump}
% \begin{rmk}
% \label{rmk:non_deg_views_old}
% Assumption~\ref{assump:non_deg_views} holds when each view is affinely non-degenerate i.e. has at least $d+1$ points whose affine span has a rank of $d$. In this case, the perfect alignment of the local views can be uniquely determined by their realization.
% \end{rmk}}
\revadd{Follows our first result connecting type of a perfect alignment with the rigidity of the resulting realization.}
\begin{thm}
\label{thm:inf_rigid}
\revadd{Let $\mathbf{S}$ be a perfect alignment. The realization $\Theta(\mathbf{S})$ is infinitesimally rigid if and only if the alignment $\mathbf{S}$ is non-degenerate.}
\end{thm}
% \begin{rmk}
% \revadd{We conjecture that the above theorem holds in the noisy setting having replaced the ``realization'' with the ``consensus representation'' (Definition~\ref{def:realization}). Specifically, for an alignment $\mathbf{S} \in \mathcal{C}$ Eq.~(\ref{eq:crit_pts2}), we conjecture that $\mathbf{S}$ is non-degenerate if and only if the resulting consensus representation $\mathbf{\Theta}(\mathbf{S})$ is infinitesimally rigid.}
% \end{rmk}
\revadd{Due to the proof of the above theorem, Proposition~\ref{prop:noiseless_setting1} and Remark~\ref{rmk:inf_rigid}, one can derive non-trivial perturbations of a non-infinitesimally rigid realization $\Theta(\mathbf{S})$ by using the non-trivial vectors in the null space of $\boldsymbol{\mathcal{R}}(\mathbf{S})$ or $\boldsymbol{\mathbb{L}}(\mathbf{S})$. Furthermore, in the noiseless setting, one can test if a perfect alignment $\mathbf{S}$ is non-degenerate - either by checking if $\rank(\boldsymbol{\mathcal{R}}(\mathbf{S})) \geq nd-d(d+1)/2$ or if $\rank(\boldsymbol{\mathbb{L}}(\mathbf{S})) \geq md(d-1)/2$.}
%The asymptotic time complexity of the former is $\mathcal{O}((nd)^3)$ and the latter is $\mathcal{O}((md(d-1)/2)^3 + |E(\Gamma)|(n+m)d)$, with the preferred approach depending on the specifics of the problem.}

\revadd{It is well known that an infinitesimally rigid realization is also locally rigid \citea{toth2017handbook}, and the converse holds for generic realizations ($\Theta(\mathbf{S}) = (\mathbf{x}_k(\mathbf{S}))_1^n$ is generic if the coordinates do not satisfy any non-zero algebraic equation with rational coefficients). Here, we provide a result which elucidates a more clear picture.}
% \begin{thm}
% \label{thm:loc_rigid}
% Let $\mathbf{S}$ be a perfect alignment.
% %and suppose Assumption~\ref{assump:non_deg_views} holds.
% Then the realization $\Theta(\mathbf{S})$ is locally rigid iff $\mathbf{S}$ is non-degenerate.
% \end{thm}
\begin{thm}
\label{thm:loc_rigid}
Let $\mathbf{S}$ be a perfect alignment.
%and suppose Assumption~\ref{assump:non_deg_views} holds.
Then the realization $\Theta(\mathbf{S})$ is locally rigid iff $\pi(\mathbf{S})$ is a strict global minimum of $\widetilde{F}$. Consequently, if $\mathbf{S}$ is a non-degenerate perfect alignment then $\Theta(\mathbf{S})$ is locally rigid, and the converse holds if $\Theta(\mathbf{S})$ is generic.
\end{thm}

% \revadd{There exist pathological cases such as the one illustrated in Figure~\ref{fig:locrigiddeg}, where the converse may not hold i.e. the perfect alignment associated with a locally rigid realization may be degenerate. This also follows from the fact that a locally rigid realization may not be infinitesimally rigid, unless the realization is \textit{generic} \cite{toth2017handbook, gortler2010affine}.}
% \begin{dfn}
% \label{def:generic}
% \revadd{A realization $\Theta(\mathbf{S}) = (\mathbf{x}_k(\mathbf{S}))_1^n$ is generic if the coordinates do not satisfy any non-zero algebraic equation with rational coefficients.}
% \end{dfn}
% \revadd{Since a locally rigid generic realization is also infinitesimally rigid \cite{toth2017handbook} therefore its underlying perfect alignment is going to be non-degenerate.}
% % \revdel{Due to Corollary~\ref{cor:noiseless_setting1} and Theorem~\ref{thm:loc_rigid}, under Assumption~\ref{assump:non_deg_views}, if $\mathbf{C}$ is of rank $(m-1)d$ then every realization is locally rigid. This is consistent with the results in \citea{chaudhury2015global,gortler2010affine} in that, if each view has at least $d+1$ affinely non-degenerate points and the rank of $\mathbf{C}$ is $(m-1)d$ then the patch framework is affinely rigid and thus locally (as well as globally) rigid too. Finally,}
% Consequently, from Proposition~\ref{prop:noiseless_setting1} and Theorem~\ref{thm:loc_rigid}, we obtain a characterization of \revadd{generic} local rigidity.
% \begin{cor}
% \label{cor:non_deg_in_noiseless_setting}
% Let $\mathbf{S}$ be a perfect alignment.
% %and the Assumption~\ref{assump:non_deg_views} holds.
% Then the realization $\Theta(\mathbf{S})$ is locally rigid if $\mathbb{L}(\mathbf{S}) = \overline{\mathbf{P}}(\mathbf{I}_d \otimes  (\mathbf{P}\mathbf{C}(\mathbf{S})\mathbf{P}^T))\overline{\mathbf{P}}^T$ is of rank $(m-1)d(d-1)/2$. Additionally, if $\Theta(\mathbf{S})$ is generic then the converse holds too.
% \end{cor}
% \revdel{In contrast to the general case, here, $\mathbb{L}(\mathbf{S})$ is already positive semidefinite as a consequence of the noiseless views (see Proposition~\ref{prop:noiseless_setting1}).}
\revadd{Moreover, using Corollary~\ref{cor:suff_non_deg_loc_min}, Proposition~\ref{prop:noiseless_setting1}, we obtain a sufficient condition for any realization of a patch framework to be locally rigid.
\begin{cor}
\label{cor:noiseless_setting1}
If $\mathbf{C}$ is of rank $(m-1)d$ then every perfect alignment $\mathbf{S}$ of $F$ is non-degenerate and the realization $\Theta(\mathbf{S})$ is locally rigid.
\end{cor}
}

Using the Definition~\ref{def:uniq_alignment} and~\ref{def:glob_rigid}, it is easy to deduce that a unique perfect alignment results in a globally rigid realization and vice versa. Then, using the fact that affine rigidity implies global rigidity \cite{gortler2010affine,toth2017handbook}, and a realization is affinely rigid if and only if the rank of $\mathbf{C}$ is $(m-1)d$ \citea{chaudhury2015global}, it follows from Corollary~\ref{cor:noiseless_setting1} that the unique perfect alignment underlying an affinely rigid realization is also non-degenerate.
\begin{thm}
\label{thm:glob_rigid}
Let $\mathbf{S}$ be a perfect alignment.
%and suppose Assumption~\ref{assump:non_deg_views} holds.
Then $\Theta(\mathbf{S})$ is globally rigid iff $\mathbf{S}$ is unique.
% Then the realization $\Theta(\mathbf{S})$ is globally rigid iff $\pi(\mathbf{S})$ is the unique global minimum of $\widetilde{F}$ i.e. if $\mathbf{O}$ is another perfect alignments then $\mathbf{O} = \mathbf{S}\mathbf{Q}$ for some $\mathbf{Q} \in \mathbb{O}(d)$.
\end{thm}
% \revadd{It is well known that affine rigidity implies global rigidity \cite{gortler2010affine,toth2017handbook}. Moreover, a realization is affinely rigid if and only if the rank of $\mathbf{C}$ is $(m-1)d$ \citea{chaudhury2015global}. Combined with Corollary~\ref{cor:noiseless_setting1} and Theorem~\ref{thm:glob_rigid}, we obtain the following result.}
\begin{prop}
\label{prop:affine_rigid}
\revadd{Let $\mathbf{S}$ be a perfect alignment.
%and suppose Assumption~\ref{assump:non_deg_views} holds.
If the realization $\Theta(\mathbf{S})$ is affinely rigid then $\mathbf{S}$ is a non-degenerate and unique perfect alignment. The converse does not hold due to the counterexample in \cite[Figure~3]{gortler2010affine}.}
\end{prop}
\revadd{
%Finally, it is worth to note that the realization $\Theta(\mathbf{S})$ due to a perfect alignment $\mathbf{S}$ is affinely rigid if and only if the rank of $\mathbf{C}$ is $(m-1)d$, as shown in \citea{chaudhury2015global}. 
%Combined with above proposition, if the rank of $\mathbf{C}$ is $(m-1)d$ then the unique perfect alignment is non-degenerate. This is in fact consistent with our finding in Corollary~\ref{cor:noiseless_setting1}.
Finally, combining the affine rigidity rank condition ($\rank(\mathbf{C}) = (m-1)d$) with Corollary~\ref{cor:non_deg_d_2}, Remark~\ref{rmk:C1Sp} and Theorem~\ref{thm:loc_rigid}, we are also able to connect the local rigidity of a realization in two dimensions with affine rigidity of its projection in one dimension.
\begin{cor}
\label{cor:local_affine_rigid_in_d_2}
Let $d=2$ and $\mathbf{S}$ be a perfect alignment. Then $\Theta(\mathbf{S})$ is locally rigid if its projection in at least one of the two dimensions is affinely rigid.
\end{cor}}
%%
%%
%%
%%
%%
%% 
%%
%%
%%
%%
%%
%%
\subsection{Conditions on Overlapping Views for a Non-degenerate Perfect Alignment}
\label{subsec:non_deg_noiseless_setting}
\revadd{We now focus on deriving the necessary and sufficient conditions on the overlapping structure of the views for a perfect alignment to be non-degenerate. These are inspired by the affine rigidity criteria discussed in~\cite{zha2009spectral}. The main difference is that we impose relatively weaker rank constraint on the overlaps.
In fact, as indicated by our previous results (Figure~\ref{fig:rigidity_flow}), the conditions presented here can be viewed as those ensuring infinitesimal and generic local rigidity of a realization.} To begin with,

% \begin{figure}[h]
%     \centering
%     \begin{tikzpicture}[
%         NodeA/.style={rectangle, draw=black!60, fill=green!5, very thick, minimum size=7mm, text width=3cm},
%         NodeB/.style={rectangle, draw=black!60, fill=red!5, very thick, minimum size=5mm, text width=3cm},
%         NodeC/.style={rectangle, draw=black!60, fill=blue!5, very thick, minimum size=7mm, text width=3cm},
%         NodeD/.style={rectangle, draw=black!60, fill=blue!5, very thick, minimum size=7mm, text width=5cm},
%         every text node part/.style={align=center}
%         ]
%         %Nodes
%         \node[NodeB]      (grigid)                              {globally rigid $\Theta(\mathbf{S})$};
%         \node[NodeB]      (lrigid)       [right=of grigid] {locally rigid $\Theta(\mathbf{S})$};
%         \node[NodeB]        (arigid)       [above=of grigid] {affinely rigid $\Theta(\mathbf{S})$};
%         \node[NodeA]        (uniqueS)       [below=of grigid] {unique $\mathbf{S}$};
%         %\node[NodeA]        (strictS)       [above=of lrigid] {$\pi(\mathbf{S})$ is a strict minimum of $\widetilde{F}$};
%         %\node[NodeC]        (nondegS)       [above=of strictS] {non-degenerate $\mathbf{S}$};
%         \node[NodeA]        (obbG)       [left=of grigid] {A graph $\overline{\mathbb{G}}$ with $m$ vertices where\\$(i,j) \in E(\overline{\mathbb{G}}) \iff \rank(\overline{\mathbf{B}}_{i,j}) = d$.};
%         %\node[NodeD]        (rankLbb)       [above=of uniqueS, xshift=-3.25cm, yshift=0.25cm] {$\rank(\boldsymbol{\mathbb{L}}(\mathbf{S})) = (m-1)d(d-1)/2$};
%         %\path (rankC) -- node (rankCimprankLbb) {Corollary~\ref{cor:noiseless_setting1}} (rankLbb);
%         %\path (rankLbb) -- node (rankLbbiffnondegS) {Theorem~\ref{thm:loc_rigid}} (nondegS);
%         %\path (nondegS) -- node (nondegSimpstrictS) {Trivial} (strictS);
%         %\path (strictS) -- node (nondegSifflrigid){Proposition~\ref{prop:non_deg_views}} (lrigid);
%         %\path (uniqueS) -- node (uniqueSiffgrigid) {Theorem~\ref{thm:glob_rigid}} (grigid);
        
%         %Lines
%         % \draw[implies-implies,double equal sign distance, line width=0.4mm] (rankC.south) -- node [text width=1cm,midway,right] {\citea{chaudhury2015global,zha2009spectral}} (arigid.north);
%         % \draw[-implies,double equal sign distance, line width=0.4mm] (grigid.east) -- node [text width=1cm,midway,above] {\citea{gortler2010affine}} (lrigid.west);
%         % \draw[-implies,double equal sign distance, line width=0.4mm] (arigid.east) -- node [text width=1cm,midway,above] {\citea{gortler2010affine}} (grigid.west);
%         % \draw[implies-,double equal sign distance, line width=0.4mm] (uniqueS.south)--(uniqueSiffgrigid);
%         % \draw[-implies,double equal sign distance, line width=0.4mm](uniqueSiffgrigid)--(grigid.north);
%         % \draw[implies-,double equal sign distance, line width=0.4mm] (strictS.south)--(nondegSifflrigid);
%         % \draw[-implies,double equal sign distance, line width=0.4mm] (nondegSifflrigid)--(lrigid.north);
%         % \draw[-implies,double equal sign distance, line width=0.4mm] (rankC.north)--(rankCimprankLbb)--(rankLbb.south);
%         % \draw[implies-,double equal sign distance, line width=0.4mm] (rankLbb.east)--(rankLbbiffnondegS);
%         % \draw[-implies,double equal sign distance, line width=0.4mm] (rankLbbiffnondegS)--(nondegS.west);
%         % \draw[-implies,double equal sign distance, line width=0.4mm] (nondegS.south) --(nondegSimpstrictS)-- (strictS.north);
%     \end{tikzpicture}
%     \caption{The implications between the type of a perfect alignment $\mathbf{S}$ and the rigidity of the resulting realization $\Theta(\mathbf{S})$.}
%     \label{fig:cond_on_overlaps}
% \end{figure}

\begin{dfn}
\label{def:BSAcapB}
Let $\mathbf{S}$ be a perfect alignment. Let $A$ and $B$ be non-empty disjoint subsets of $[1,m]$. Define $\mathbf{B}(\mathbf{S})_{A,B}$ to be a matrix whose columns are $\mathbf{S}_i^T\mathbf{x}_{k,i}+\mathbf{t}_i$ (in the increasing order of $k$) where $(k,i),(k,j) \in E(\Gamma)$ for some $i \in A$ and $j \in B$, and where $\mathbf{t}_i$ is obtained using Eq.~(\ref{eq:opt_Z}). Also define $\overline{\mathbf{B}(\mathbf{S})}_{A,B} = \mathbf{B}(\mathbf{S})_{A,B}\left(\mathbf{I}_{n'} - (1/n')\mathbf{1}_{n'}\mathbf{1}_{n'}^T\right)$ where $n' = |\{k:(k,i),(k,j) \in E(\Gamma) \text{ for some } (i,j) \in A \times B\}|$.
For brevity, we denote $\mathbf{B}(\mathbf{S})_{\{i\},\{j\}}$ and $\overline{\mathbf{B}(\mathbf{S})}_{\{i\},\{j\}}$ by $\mathbf{B}(\mathbf{S})_{i,j}$ and $\overline{\mathbf{B}(\mathbf{S})}_{i,j}$ respectively, where $i \neq j$. Note that the notation is consistent with that of Definition~\ref{def:BSicapj}.
\end{dfn}

\begin{rmk}
\label{rmk:BS_ijB_ij_noiseless}
Since $\mathbf{S}$ is a perfect alignment $\mathbf{S}_i^T\mathbf{x}_{k,i}+\mathbf{t}_i = \mathbf{S}_j^T\mathbf{x}_{k,j}+\mathbf{t}_j$ for all $(k,i),(k,j) \in E(\Gamma)$, thus $\mathbf{B}(\mathbf{S})_{A,B}$ is well defined and $\mathbf{B}(\mathbf{S})_{A,B} = \mathbf{B}(\mathbf{S})_{B,A}$.
Let $i,j \in [1,m]$ then, since $\mathbf{B}(\mathbf{S})_{i,j} = \mathbf{B}(\mathbf{S})_{j,i}$, from Remark~\ref{rmk:BS_ijB_ij}, $\rank (\overline{\mathbf{B}}_{i,j}) = \rank (\overline{\mathbf{B}(\mathbf{S})}_{i,j}) = \rank (\overline{\mathbf{B}(\mathbf{S})}_{j,i}) = \rank (\overline{\mathbf{B}}_{j,i})$ and $\rank (\overline{\mathbf{B}}_{i,j}) = \rank (\overline{\mathbf{B}(\mathbf{S})}_{i,j}\overline{\mathbf{B}(\mathbf{S})}_{j,i}^T) = \rank (\overline{\mathbf{B}}_{i,j}\overline{\mathbf{B}}_{j,i}^T)$.
\end{rmk}

Due to the above two remarks and Theorem~\ref{thm:non_deg_two_views_gen_setting}, a necessary and sufficient condition for a perfect alignment of two views to be non-degenerate is easily obtained.
\begin{thm}
\label{thm:nec_suff_cond_loc_rigid_two_views}
Consider $m=2$ and let $\mathbf{S}$ be a perfect alignment. Then $\mathbf{S}$ is non-degenerate iff $\rank(\overline{\mathbf{B}}_{1,2}) \geq d-1$.
\end{thm}
\begin{figure}[H]
    \centering
    \begin{tabular}{cc}
    \begin{subfigure}[b]{0.35\textwidth}
         \centering
         \includegraphics[width=0.9\textwidth,keepaspectratio]{fig/fig0/counterex_nec_loc_rigid.png}         \caption{Theorem~\ref{thm:nec_cond_loc_rigid_of_views}}
         \label{fig:nec_cond_loc_rigid_of_views}
     \end{subfigure}
     & 
     \begin{subfigure}[b]{0.175\textwidth}
         \centering
         \includegraphics[width=0.9\textwidth,keepaspectratio]{fig/fig0/counterex_suff_loc_rigid_2.png}
         \caption{Theorem~\ref{thm:G_star_1}}
         \label{fig:G_star_1}
     \end{subfigure}
     \end{tabular}
    \caption{Counterexamples for the converse of various Theorems. (a) For every pair of nonempty partitions $A$ and $B$ of $[1,4]$, $\rank(\overline{\mathbf{B}(\mathbf{S})}_{A,B}) \geq 1$ but $\mathbf{S}$ is degenerate. (b) $\mathbf{S}$ is non-degenerate but $|\mathbb{G}^*(\mathbf{S})| = 3$. These views should be considered affinely non-degenerate, satisfying Assumption~\ref{assump:non_deg_views}. For clarity, only the points in the overlapping regions are shown.}
    \label{fig:counterex}
\end{figure}
A necessary condition for a perfect alignment of $m \geq 3$ views to be non-degenerate is as follows. The converse of the theorem does not hold, as demonstrated in Figure~\ref{fig:nec_cond_loc_rigid_of_views}.
\begin{thm}
\label{thm:nec_cond_loc_rigid_of_views}
Let $\mathbf{S}$ be a perfect alignment. If $\mathbf{S}$ is non-degenerate then the $\rank(\overline{\mathbf{B}(\mathbf{S})}_{A,B})$ is at least $d-1$ for all non-empty partitions $A$ and $B$ of $[1,m]$ i.e. for all $A,B \subseteq [1,m]$, $A,B \neq \emptyset$, $A \cap B = \emptyset$ and $A \cup B = [1,m]$.
\end{thm}

Now we derive a sufficient condition for a perfect alignment of $m \geq 3$ views to be non-degenerate. As in \citea{zha2009spectral}, we construct a graph $\mathbb{G}$ with $m$ vertices where each vertex corresponds to a view and an edge exists between the $i$th and $j$th vertices iff $\rank(\overline{\mathbf{B}}_{i,j}) \geq d-1$. The Theorem~\ref{thm:nec_suff_cond_loc_rigid_two_views} and the following propositions will play a crucial role in our next set of results,

\begin{lem}
\label{lem:subproblem_cert}
Let $\mathbf{S}$ be a perfect alignment and $\boldsymbol{\Omega}$ be a certificate of $\mathbf{L}(\mathbf{S})$. Consider removing the $i$th view and the points that lie exclusively in it. Then $\mathbf{S}_{-i} = [\mathbf{S}_j]_{j \in [1,m] \setminus \{i\}}$ is a perfect alignment of the remaining views and $[\boldsymbol{\Omega}_j]_{j \in [1,m] \setminus \{i\}}$ is a certificate of $\mathbf{L}_{-i}(\mathbf{S}_{-i})$, the matrix in Eq.~(\ref{eq:L_of_S}) associated with the remaining views.
\end{lem}

\begin{prop}
\label{prop:same_conn_comp_non_deg}
Let $\mathbf{S}$ be a perfect alignment. Let $\boldsymbol{\Omega}$ be a certificate of $\mathbf{L}(\mathbf{S})$. If $i$th and $j$th view lie in the same connected component of $\mathbb{G}$ then $\boldsymbol{\Omega}_i = \boldsymbol{\Omega}_j$.
\end{prop}

Similar to \citea{zha2009spectral}, consider the following coarsening procedure on $\mathbb{G}$ given a perfect alignment $\mathbf{S}$: (i) transform all the views using $\mathbf{S}$ (and $\mathbf{t}$ computed using Eq.~\ref{eq:opt_Z}), (ii) merge the views that lie in the same connected component of $\mathbb{G}$ and replace them with a single view, (iii) then construct the graph (in the same manner as $\mathbb{G}$) associated with the new set of views, (iv) repeat the procedure from (ii). Let the final graph over the remaining views be $\mathbb{G}^*(\mathbf{S})$, then the following result holds (the corollary follows trivially and the converse of the theorem may not hold, as shown in Figure~\ref{fig:G_star_1}).
\begin{thm}
\label{thm:G_star_1}
    A perfect alignment $\mathbf{S}$ is non-degenerate if $|\mathbb{G}^*(\mathbf{S})| = 1$.
\end{thm}
\begin{cor}
\label{cor:suff_cond_views_non_deg}
Every perfect alignment is non-degenerate if $\mathbb{G}$ is connected.
\end{cor}
\begin{rmk}
\label{rmk:tree_structure}
\revadd{From Theorem~\ref{thm:nec_cond_loc_rigid_of_views}, it is easy to see that the converse of the above corollary holds when a graph over views, in which two views are connected if they are overlapping (i.e. they share at least one common point), is a tree.}
\end{rmk}

\revadd{It is important to note that we have derived a necessary and sufficient rank based condition (Proposition~\ref{prop:noiseless_setting1}) that can be tested in polynomial time to assess the non-degeneracy of a given perfect alignment. While the above results offer a geometric interpretation of this condition, a complete understanding in the form of a single condition on the overlapping structure of the views, that is both necessary and sufficient, has yet to be established.}

%%
%%
%%
%%
%%
%% 
%%
%%
%%
%%
%%
%%
\subsection{Conditions on Overlapping Views for a Unique Perfect Alignment}
\label{subsec:uniq_noiseless_setting}
\revdel{We previously showed that the global rigidity of a realization is equivalent to the uniqueness of the corresponding perfect alignment (Theorem~\ref{thm:glob_rigid}). Thus}Here, we focus on deriving necessary and sufficient conditions on the overlapping structure of the views for a perfect alignment to be unique, equivalently, for the resulting realization to be globally rigid. From Remark~\ref{rmk:BS_ijB_ij_noiseless} and Theorem~\ref{thm:uniq_two_views_gen_setting},
\begin{thm}
\label{thm:nec_suff_cond_glob_rigid_two_views}
Consider $m=2$ and let $\mathbf{S}$ be a perfect alignment. Then $\mathbf{S}$ is unique (Definition~\ref{def:uniq_alignment}) iff $\rank (\overline{\mathbf{B}}_{1,2}) = d$.
\end{thm}

A necessary condition for a perfect alignment of $m \geq 3$ views to be unique is, %as follows.
%The converse of it may not hold, as demonstrated in Figure~\ref{fig:nec_cond_glob_rigid_views}.
\begin{thm}
\label{thm:nec_cond_glob_rigid_views}
If $\mathbf{S}$ is a unique perfect alignment then $\rank(\overline{\mathbf{B}(\mathbf{S})}_{A,B}) = d$ for all non-empty partitions $A$ and $B$ of $[1,m]$.
\end{thm}

\revadd{It was shown in \citea{zha2009spectral} that the above rank condition holds for affinely rigid realization $\Theta(\mathbf{S})$. In contrast, our requirement only requires the perfect alignment $\mathbf{S}$ to be unique, which is equivalent to the global rigidity of $\Theta(\mathbf{S})$ (Figure~\ref{fig:rigidity_flow}). Moreover, we conjecture that the converse of the above theorem holds, in which case we would obtain a characterization of a unique perfect alignment and an exponential-time algorithm to test it, aligning with the NP-hardness of testing global rigidity~\cite{saxe1979embeddability}.}

% \revadd{We note that the above result is stronger than the one in \citea{zha2009spectral} where the authors showed that the above rank condition holds when the realization $\Theta(\mathbf{S})$ is affinely rigid, while we require the perfect alignment $\mathbf{S}$ to be unique which is equivalent to global rigidity of $\Theta(\mathbf{S})$. Additionally, we conjecture that the converse of the above theorem holds too. If it is indeed the case then we would obtain a characterization of a unique perfect alignment and an exponential time algorithm to test it. Note that testing if a realization is globally rigid is NP-hard \cite{saxe1979embeddability}.}

Now we derive a sufficient condition for a perfect alignment of $m \geq 3$ to be unique. As in the previous section, we construct a graph $\overline{\mathbb{G}}$ with $m$ vertices, one for each view. An edge exists between the $i$th and $j$th vertices iff $\rank(\overline{\mathbf{B}}_{i,j}) = d$. We need Lemma~\ref{lem:subproblem_cert}, Theorem~\ref{thm:nec_suff_cond_glob_rigid_two_views} and the following proposition for our next result,
\begin{prop}
\label{prop:same_conn_comp_uniq}
Let $\mathbf{S}$ and $\mathbf{S}'$ be perfect alignments. If $i$th and $j$th view lie in the same connected component of $\overline{\mathbb{G}}$ then $\mathbf{S}'_i = \mathbf{S}_i\mathbf{Q}$ and $\mathbf{S}'_j = \mathbf{S}_j\mathbf{Q}$ for some $\mathbf{Q} \in \mathbb{O}(d)$.
\end{prop}
\begin{figure}[H]
    \centering
     \includegraphics[width=0.15\textwidth,keepaspectratio]{fig/fig0/counterex_suff_glob_rigid.png}
    \caption{$\mathbf{S}$ is unique but $|\overline{\mathbb{G}}^*(\mathbf{S})| = 3$, thus the converse of Theorem~\ref{thm:overline_G_star_1} may not hold.}
    \label{fig:overline_G_star_1}
\end{figure}

Consider the same coarsening procedure as in Theorem~\ref{thm:G_star_1}, except that $\mathbb{G}$ and $\mathbb{G}^*(\mathbf{S})$ are replaced by $\overline{\mathbb{G}}$ and $\overline{\mathbb{G}}^*(\mathbf{S})$, respectively. Then the following holds (the corollary follows trivially and a counterexample for the converse is shown in Figure~\ref{fig:overline_G_star_1}).
\begin{thm}
\label{thm:overline_G_star_1}
    A perfect alignment $\mathbf{S}$ is unique if $|\overline{\mathbb{G}}^*(\mathbf{S})| = 1$.
\end{thm}
\begin{cor}
\label{cor:suff_cond_views_uniq}
Every perfect alignment is unique if $\overline{\mathbb{G}}$ is connected.
\end{cor}
% \begin{figure}[H]
%     \centering
%     \begin{tabular}{cc}
%     \begin{subfigure}[b]{0.35\textwidth}
%          \centering
%          \includegraphics[width=0.9\textwidth,keepaspectratio]{fig/fig0/counterex_nec_loc_rigid.png}
%          \caption{Theorem~\ref{thm:nec_cond_glob_rigid_views}}
%          \label{fig:nec_cond_glob_rigid_views}
%      \end{subfigure}
%      & 
%      \begin{subfigure}[b]{0.175\textwidth}
%          \centering
%          \includegraphics[width=\textwidth,keepaspectratio]{fig/fig0/counterex_suff_glob_rigid.png}
%          \caption{Theorem~\ref{thm:overline_G_star_1}}
%          \label{fig:overline_G_star_1}
%      \end{subfigure}
%      \end{tabular}
%     \caption{Counterexamples for the converse of various Theorems. The dotted lines represent views and the filled points represent points on the overlaps. (\ref{fig:nec_cond_glob_rigid_views}) Just a placeholder for now. (\ref{fig:overline_G_star_1}) Clearly $\mathbf{S}$ is unique but $|\overline{\mathbb{G}}^*(\mathbf{S})| = 3$.}
%     \label{fig:counterex2}
% \end{figure}
% \revadd{We would like to acknowledge that the above result is weaker than the one presented in \citea{zha2009spectral}, where the authors demonstrated that the sufficient condition leads to an affinely rigid realization $\Theta(\mathbf{S})$, whereas we only establish global rigidity of $\Theta(\mathbf{S})$. Nevertheless, we retain this result since the the proving technique is substantially different than the one in \citea{zha2009spectral}.}
\revadd{We note that the above result is weaker than the one in \citea{zha2009spectral} where the authors showed that the sufficient condition leads to affinely rigid realization $\Theta(\mathbf{S})$, while we show global rigidity. Nevertheless, we keep the result since the the proving technique is different than the one in \citea{zha2009spectral}.}