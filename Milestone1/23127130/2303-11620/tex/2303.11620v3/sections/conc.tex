The following questions remain unanswered and we hope to address them in future work.
\begin{enumerate}[leftmargin=*]
    \item As we showed in Section~\ref{subsec:loc_glob_rigid}, in the case of noiseless views, a non-degenerate perfect alignment  characterizes the infinitesimal rigidity of the resulting realization and also its local rigidity if the realization is generic (Theorem~\ref{thm:inf_rigid} and \ref{thm:loc_rigid}). These results serve as a geometric interpretation of the non-degeneracy conditions in the noiseless case. Although a similar interpretation in the case of noisy views is still missing, it seems natural to expect that the non-degeneracy of an alignment would be associated with infinitesimal rigidity of the consensus representation of the framework (Definition~\ref{def:realization}). We conjecture that the two notions are equivalent. Notably, such a result would offer physically interpretable insights into the algebraic structure of $\mathbb{L}(\mathbf{S})$ (Remark~\ref{rmk:mathbb_L_structure}).
    \item In relation to the previous problem, and analogous to those presented in Section~\ref{subsec:non_deg_noiseless_setting} and Section~\ref{subsec:uniq_noiseless_setting}, necessary and sufficient conditions on the overlapping structure of $m > 2$ \textit{noisy} views for a non-degenerate alignment and for a unique optimal alignment are still unknown.
    \item The proof/counterexample of the converse of Theorem~\ref{thm:nec_cond_glob_rigid_views} is to be investigated.
    \item The relationship between the eigenvalues of $\mathbb{L}(\mathbf{S})$ and those of $\mathbf{C}(\mathbf{S})$ remains unclear in both the noisy and noiseless settings. Specifically, the connection between $\lambda_{d(d-1)/2+2}(\mathbb{L}(\mathbf{S}))$ and the eigenvalues of $\mathbf{C}$ should be explored as it would aid in determining the radius of convergence $\delta(\mathbf{S})$ for RGD (Theorem~\ref{thm:rgd_conv2}) in terms of the eigenvalues of $\mathbf{C}$ instead of $\mathbb{L}(\mathbf{S})$. It is important to note that a direct application of Ostrowski's theorem \cite{higham1998modifying} does not seem to provide the desired link.
    \item Requiring an alignment to be non-degenerate for the local linear convergence of RGD is, in a sense, a strong ask. The convergence to arbitrary critical points using a geometric approach based on \citeb[Section 6.2]{usevich2020approximate} is to be investigated.
    \item It remains unclear whether the solution of the spectral relaxation of Eq.~(\ref{eq:GPOP}) remains close to an optimal alignment under weaker infinitesimal or local rigidity constraints.
\end{enumerate}