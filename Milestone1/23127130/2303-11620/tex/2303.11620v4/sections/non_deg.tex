In this section, we derive the Hessian of $\widetilde{F}$, the function induced by $F$ on a certain quotient space $\mathbb{O}(d)^m/_{\sim}$. We refer the reader to \citeb[Chapter 3 and 5]{absil2009optimization} for the definitions of differential of a mapping, metric, gradient, connection and Hessian in the context of Riemannian manifolds, as well as to \citea{zhu2023rotation,luo2022nonconvex,dong2022analysis,zhao2015riemannian}, which also adopt a quotient manifold approach in their analyses. We also obtain the equations governing the non-singularity and positivity of the Hessian and consequently, a characterization of a non-degenerate alignment in the general (noisy) setting. \revadd{Moreover, we identify the vicinity of a non-degenerate alignment in which the Hessian is positive definite.}
%We also present a necessary and sufficient condition on the overlapping structure of two views under which an alignment of the views is non-degenerate. Finally, in Section~\ref{subsec:uniq_gen_setting}, we derive a similar condition under which an optimal alignment of two views is unique.
% The organization is as follows. In Section~\ref{subsec:prelims} we derive the Hessian of $\widetilde{F}$ (Eq.~(\ref{eq:eq3}), (\ref{eq:eq4})), the function induced by $F$ on a certain quotient space $\mathbb{O}(d)^m/_{\sim}$ (along with all the geometrical constructs needed). We refer the reader to \citeb[Chapter 3 and 5]{absil2009optimization} for the definitions of differential of a mapping, metric, gradient, connection and Hessian in the context of Riemannian manifolds, as well as to \citea{zhu2023rotation,luo2022nonconvex,dong2022analysis,zhao2015riemannian}, which also adopt a quotient manifold approach in their analyses. In Section~\ref{subsec:non_sing_pos_def_hess} we obtain the equations governing the non-singularity and positivity of the Hessian. Subsequently, in Section~\ref{subsec:non_deg_gen_setting}, we obtain a characterization of a non-degenerate alignment in the general (noisy) setting. \revadd{Additionally, we derive a bound on the size of the neighborhood around a non-degenerate alignment within which the Hessian is positive definite.} We also present a necessary and sufficient condition on the overlapping structure of two views under which an alignment of the views is non-degenerate. Finally, in Section~\ref{subsec:uniq_gen_setting}, we derive a similar condition under which an optimal alignment of two views is unique.
% The organization is as follows. In Section~\ref{subsec:prelims} we derive the Hessian of $\widetilde{F}$ (Eq.~(\ref{eq:eq3}), (\ref{eq:eq4})), the function induced by $F$ on a certain quotient space $\mathbb{O}(d)^m/_{\sim}$ (along with all the geometrical constructs needed). We refer the reader to \citeb[Chapter 3 and 5]{absil2009optimization} for the definitions of differential of a mapping, metric, gradient, connection and Hessian in the context of Riemannian manifolds, as well as to \citea{zhu2023rotation,luo2022nonconvex,dong2022analysis,zhao2015riemannian}, which also adopt a quotient manifold approach in their analyses. In Section~\ref{subsec:non_sing_pos_def_hess} we obtain the equations governing the non-singularity and positivity of the Hessian (Eq.~(\ref{eq:TrOmegaTLSOmega}), (\ref{eq:omega^TmbbLomega})). Subsequently, in Section~\ref{subsec:non_deg_gen_setting}, we obtain a characterization of a non-degenerate alignment in the general setting (Theorem~\ref{thm:non_deg_loc_min}). \revadd{Additionally, we derive a bound on the size of the neighborhood around a non-degenerate alignment within which the Hessian is non-singular and positive definite (Proposition~\ref{prop:HessVicinity}).} We also present a necessary and sufficient condition on the overlapping structure of two views under which an alignment of the views is non-degenerate (Theorem~\ref{thm:non_deg_two_views_gen_setting}). Finally, in Section~\ref{subsec:uniq_gen_setting}, we derive a similar condition under which an optimal alignment of two views is unique (Theorem~\ref{thm:uniq_two_views_gen_setting}).

\subsection{Preliminaries}
\label{subsec:prelims}
Recall that the problem under consideration is the minimization of $F(\mathbf{S}) = \Tr(\mathbf{C}\mathbf{S}\mathbf{S}^T)$ over $\mathbf{S} \in \mathbb{O}(d)^m$ where $\mathbf{C} \succeq 0$ is the patch-stress matrix defined in Eq.~(\ref{eq:GPOP}). Note that the objective is invariant to the action of $\mathbb{O}(d)$ i.e. for any $\mathbf{Q} \in \mathbb{O}(d)$, $F(\mathbf{S}) = F(\mathbf{S}\mathbf{Q})$.
\begin{assump}
If $\Gamma$ has $K$ connected components, then the objective is invariant to the action of $\mathbb{O}(d)^K$ on $\mathbb{O}(d)^m = \prod_1^K \mathbb{O}(d)^{m_j}$ where $\textstyle\sum_1^K m_j = m$ and where each $\mathbb{O}(d)$ acts independently on $\mathbb{O}(d)^{m_j}$ for $j \in [1,K]$. To keep the computations clean, we assume that the bipartite graph $\Gamma$ is connected (as in Assumption~\ref{assump:connected_gamma}) throughout.
\end{assump}
Subsequently, we define an equivalence relation on $\mathbb{O}(d)^m$; $\mathbf{S}_1 \sim \mathbf{S}_2$ iff $\mathbf{S}_1 = \mathbf{S}_2\mathbf{Q}$ for some $\mathbf{Q} \in \mathbb{O}(d)$. Given $\mathbf{S} \in \mathbb{O}(d)^m$, its equivalence class is $[\mathbf{S}] = \left\{\mathbf{S}\mathbf{Q}: \mathbf{Q} \in \mathbb{O}(d)\right\}$. Clearly, there exists a bijection between $\mathbb{O}(d)^m/_{\sim}$ and $\mathbb{O}(d)^{m-1}$, thus an element of $\mathbb{O}(d)^m/_{\sim}$ will be identified with an element of $\mathbb{O}(d)^{m-1}$. Define the projection, 
\begin{align}
    \pi: \mathbb{O}(d)^m &\mapsto \mathbb{O}(d)^m/_{\sim}\\
    \pi(\mathbf{S}_{1:m}) &= \mathbf{S}_{2:m}\mathbf{S}_1^T. \label{eq:pi}
\end{align}
% \begin{align}
%     \pi: \mathbb{O}(d)^m &\mapsto \mathbb{O}(d)^m/_{\sim}\\
%     \pi(\mathbf{S}_{1:m}) &= \mathbf{S}_{2:m}\mathbf{S}_1^T. \label{eq:pi}
% \end{align}
Let $\widetilde{\mathbf{S}} \in \mathbb{O}(d)^m/_{\sim}$, then
\begin{equation}
    \pi^{-1}(\widetilde{\mathbf{S}}) = \left\{\begin{bmatrix}
    \mathbf{Q}\\\widetilde{\mathbf{S}}\mathbf{Q}
    \end{bmatrix}: \mathbf{Q} \in \mathbb{O}(d)\right\} = \{\mathbf{S} \in \mathbb{O}(d)^m: \mathbf{S}_{i+1}\mathbf{S}_1^T = \widetilde{\mathbf{S}}_{i}, i \in [1,m-1]\}. \label{eq:pi_inv_wtS}
\end{equation}

The Riemannian metric $g$ on $\mathbb{O}(d)^m$ is the canonical one given by
\begin{equation}
g(\mathbf{Z},\mathbf{W}) \coloneqq \Tr(\mathbf{Z}^T\mathbf{W}) = \textstyle\sum_1^m \Tr(\mathbf{Z}_i^T\mathbf{W}_i) \text{ where } \mathbf{Z},\mathbf{W} \in T_{\mathbf{S}}\mathbb{O}(d)^m \subseteq \mathbb{R}^{md \times d}. \label{eq:g_Z_W}
\end{equation}
By a simple extension of the $m=1$ case \citea{absil2009optimization}, it is easy to deduce the following result.
\begin{prop}
\label{prop:T_SOdm}
For $\mathbf{S} \in \pi^{-1}(\widetilde{\mathbf{S}})$, the tangent space to $\mathbb{O}(d)^m$ at $\mathbf{S}$ is given by
\begin{equation}
    T_{\mathbf{S}}\mathbb{O}(d)^m = \{[\mathbf{S}_i\boldsymbol{\Omega}_i]_1^m: \boldsymbol{\Omega}_i \in \Skew(d)\}. \label{eq:T_SOdm}
\end{equation}
The orthogonal projection of $\boldsymbol{\xi} = [\boldsymbol{\xi}_i]_1^m$, where $\boldsymbol{\xi}_i \in \mathbb{R}^{d \times d}$, onto $T_{\mathbf{S}}\mathbb{O}(d)^m$ is
\begin{align}
    P_{\mathbf{S}}\left(\boldsymbol{\xi}\right) = \argmin_{[\mathbf{S}_i\boldsymbol{\Omega}_i]_1^m,\boldsymbol{\Omega}_i \in \Skew(d)} \textstyle\sum_1^m\left\|\boldsymbol{\xi}_i - \mathbf{S}_i\boldsymbol{\Omega}_i\right\|_F^2 = [\mathbf{S}_i\Skew(\mathbf{S}_i^T\boldsymbol{\xi}_i)]_1^m. \label{eq:P_S_xi}
\end{align}
\end{prop}

Then, $\pi^{-1}(\widetilde{\mathbf{S}})$ admits a tangent space at $\mathbf{S} \in \pi^{-1}(\widetilde{\mathbf{S}})$ called the vertical space $\mathcal{V}_{\mathbf{S}}$ at $\mathbf{S}$. The horizontal space $\mathcal{H}_{\mathbf{S}}$ at $\mathbf{S}$ is the subspace of $T_{\mathbf{S}}\mathbb{O}(d)^m$ that is the orthogonal complement to the vertical space $\mathcal{V}_{\mathbf{S}}$.
\begin{prop}
\label{prop:V_S_H_S}
The vertical space $\mathcal{V}_{\mathbf{S}}$ at $\mathbf{S} \in \pi^{-1}(\widetilde{\mathbf{S}})$ is
%\begin{align}
 $$   \mathcal{V}_{\mathbf{S}} = \{\mathbf{S}\boldsymbol{\Omega}: \boldsymbol{\Omega} \in \Skew(d)\}.$$
%\end{align}
The orthogonal projection of $\mathbf{Z} = [\mathbf{S}_i\boldsymbol{\Omega}_i]_1^m \in T_{\mathbf{S}}\mathbb{O}(d)^m$ onto $\mathcal{V}_{\mathbf{S}}$ is
\begin{equation}
    P^{v}_{\mathbf{S}}([\mathbf{S}_i\boldsymbol{\Omega}_i]_1^m) = \left[\mathbf{S}_i\argmin_{\boldsymbol{\Omega}\in \Skew(d)} \textstyle\sum_1^m\left\|\mathbf{S}_j(\boldsymbol{\Omega}_j-\boldsymbol{\Omega})\right\|_F^2\right]_1^m = \left[\mathbf{S}_i\left(m^{-1}\textstyle\sum_1^m\boldsymbol{\Omega}_i\right)\right]_1^m. \label{eq:P^v_S}
\end{equation}
The horizontal space at $\mathbf{S} \in \pi^{-1}(\widetilde{\mathbf{S}})$ is
$$\mathcal{H}_{\mathbf{S}} = \left\{[\mathbf{S}_i\boldsymbol{\Omega}_i]_1^m, \boldsymbol{\Omega}_i \in \Skew(d), \textstyle\sum_1^m \boldsymbol{\Omega}_i= 0\right\}.$$
The orthogonal projection of $\mathbf{Z} = [\mathbf{S}_i\boldsymbol{\Omega}_i]_1^m \in T_{\mathbf{S}}\mathbb{O}(d)^m$ to $\mathcal{H}_{\mathbf{S}}$ is
\begin{equation}
    P^{h}_{\mathbf{S}}([\mathbf{S}_i\boldsymbol{\Omega}_i]_1^m) = [\mathbf{S}_i\boldsymbol{\Omega}_i]_1^m - P^{v}_{\mathbf{S}}([\mathbf{S}_i\boldsymbol{\Omega}_i]_1^m) = \left[\mathbf{S}_i\left(\boldsymbol{\Omega}_i-m^{-1}\textstyle\sum_1^m\boldsymbol{\Omega}_i\right)\right]_1^m.\label{eq:P^h_S}
\end{equation}
\end{prop}

Note that $T_\mathbf{S}\mathbb{O}(d)^m$ is a vector space of dimension $md(d-1)/2$ and $\mathcal{V}_{\mathbf{S}}$ forms a $d(d-1)/2$ dimensional subspace of $T_{\mathbf{S}}\mathbb{O}(d)^m$. The dimension of $\mathcal{H}_{\mathbf{S}}$ and $T_{\widetilde{\mathbf{S}}}\mathbb{O}(d)^m/_{\sim}$ is $(m-1)d(d-1)/2$. In particular, $\mathcal{H}_{\mathbf{S}}$ can be identified with $T_{\widetilde{\mathbf{S}}}\mathbb{O}(d)^m/_{\sim}$. Let $\widetilde{\mathbf{S}} \in \mathbb{O}(d)^{m}/_{\sim}$ and $\widetilde{\mathbf{Z}} \in T_{\widetilde{\mathbf{S}}}\mathbb{O}(d)^{m}/_{\sim}$. Then the horizontal lift of $\widetilde{\mathbf{Z}}$ at $\mathbf{S} \in \pi^{-1}(\widetilde{\mathbf{S}})$ is defined as $\overline{\widetilde{\mathbf{Z}}} \in \mathcal{H}_{\mathbf{S}}$ such that for each $i \in [1,m-1]$,
\begin{equation}
    D\pi[\mathbf{S}]\left(\overline{\widetilde{\mathbf{Z}}}\right)_i = \widetilde{\mathbf{Z}}_i. \label{eq:hlift_def}
\end{equation}

\begin{prop}
\label{prop:hlift_char}
Let $\widetilde{\mathbf{S}} \in \mathbb{O}(d)^{m}/_{\sim}$ and $\widetilde{\mathbf{Z}} \in T_{\widetilde{\mathbf{S}}}\mathbb{O}(d)^{m}/_{\sim}$. Let $\mathbf{Z}$ be the horizontal lift of $\widetilde{\mathbf{Z}}$ at $\mathbf{S} \in \pi^{-1}(\widetilde{\mathbf{S}})$. If $(\widetilde{\boldsymbol{\Omega}}_i)_1^{m-1} \subseteq \Skew(d)$ are such that $\widetilde{\mathbf{Z}}_i = \widetilde{\mathbf{S}}_i\widetilde{\boldsymbol{\Omega}}_i$, and $(\boldsymbol{\Omega}_i)_1^m \subseteq \Skew(d)$ are such that $\mathbf{Z}_i=\mathbf{S}_i\boldsymbol{\Omega}_i$ and $\textstyle\sum_1^m \boldsymbol{\Omega}_i = 0$, then
\begin{align}
    \boldsymbol{\Omega}_1 &= -m^{-1}\mathbf{S}_1^T\left(\textstyle\sum_1^{m-1}\widetilde{\boldsymbol{\Omega}}_i\right)\mathbf{S}_1 \label{eq:hlift1}\\
    \boldsymbol{\Omega}_{i+1} &= \mathbf{S}_1^T\widetilde{\boldsymbol{\Omega}}_i\mathbf{S}_1 + \boldsymbol{\Omega}_1 \text{ for all } i \in [1,m-1]. \label{eq:hlifti}
\end{align}
Moreover, the linear system above has full rank, and thus the horizontal lift $\mathbf{Z}$ of $\widetilde{\mathbf{Z}}$ at $\mathbf{S} \in \mathbb{O}(d)^m$ is a unique element of $\mathcal{H}_{\mathbf{S}}$.
\end{prop}

\begin{prop}
\label{prop:g_tilde}
Let $\widetilde{\mathbf{Z}},\widetilde{\mathbf{W}} \in T_{\widetilde{\mathbf{S}}}\mathbb{O}(d)^{m}/_{\sim}$ and $\mathbf{Z}, \mathbf{W} \in T_{\mathbf{S}}\mathbb{O}(d)^m$ be their horizontal lifts at $\mathbf{S} \in \pi^{-1}(\widetilde{\mathbf{S}})$. Then
%\begin{align}
 $   \widetilde{g}(\widetilde{\mathbf{Z}},\widetilde{\mathbf{W}}) \coloneqq g(\mathbf{Z}, \mathbf{W})$
%\end{align}
defines a Riemannian metric on $\mathbb{O}(d)^m/_{\sim}$.
\end{prop}
We note that $\mathcal{H}_{\mathbf{S}}$ with the canonical metric $g$, is isometric to $T_{\widetilde{\mathbf{S}}}\mathbb{O}(d)^{m}/_{\sim}$ (equivalently $T_\mathbf{S}\mathbb{O}(d)^{m-1}$) when equipped with the above metric $\widetilde{g}$. \revadd{The following relation holds between the Frobenius norm of an element of $T_{\widetilde{\mathbf{S}}}\mathbb{O}(d)^{m}/_{\sim}$ and of its horizontal lift.}
\begin{prop}
\label{prop:hlift_frob_ineq}
\revadd{Let $\widetilde{\mathbf{Z}} \in T_{\widetilde{\mathbf{S}}}\mathbb{O}(d)^{m}/_{\sim}$ and $\mathbf{Z} \in T_{\mathbf{S}}\mathbb{O}(d)^m$ be the horizontal lift of $\mathbf{\widetilde{Z}}$ at $\mathbf{S} \in \pi^{-1}(\widetilde{\mathbf{S}})$. Then $\left\|\mathbf{Z}\right\|_F \leq \left\|\widetilde{\mathbf{Z}}\right\|_F \leq \sqrt{m+1}\left\|\mathbf{Z}\right\|_F$.}
\end{prop}
% \begin{prop}
% \label{prop:d_g_tilde}
% \revadd{Let $d_{g}$ and $d_{\widetilde{g}}$ be geodesic distances on induced by the metric $g$ on $\mathbb{O}(d)^m$ and $\widetilde{g}$ on $\mathbb{O}(d)^{m}/_{\sim}$, respectively. Then, $d_{\widetilde{g}}(\pi(\mathbf{S}_1), \pi(\mathbf{S}_2)) \leq d_g(\mathbf{S}_1, \mathbf{S}_2)$ for all $\mathbf{S}_1, \mathbf{S}_2 \in \mathbb{O}(d)^{m}$.}
% \end{prop}=

Now, coming back to the alignment error $F(\mathbf{S}) = \Tr(\mathbf{C}\mathbf{S}\mathbf{S}^T)$ defined on $\mathbb{O}(d)^m$. It induces the following function $\widetilde{F}(\widetilde{\mathbf{S}}) = \Tr\left(\mathbf{C}\begin{bsmallmatrix}
    \mathbf{I}_d\\\widetilde{\mathbf{S}}
    \end{bsmallmatrix}\begin{bsmallmatrix}
    \mathbf{I}_d\\\widetilde{\mathbf{S}}
    \end{bsmallmatrix}^T\right)$ on $\mathbb{O}(d)^{m}/_{\sim}$ (again identified with $\mathbb{O}(d)^{m-1}$).
% \begin{align}
%     \widetilde{F}(\widetilde{\mathbf{S}}) = \Tr\left(\mathbf{C}\begin{bmatrix}
%     \mathbf{I}_d\\\widetilde{\mathbf{S}}
%     \end{bmatrix}\begin{bmatrix}
%     \mathbf{I}_d\\\widetilde{\mathbf{S}}
%     \end{bmatrix}^T\right). \label{eq:Ftilde}
% \end{align}
In particular, $F = \widetilde{F} \circ \pi$ and $\widetilde{F} \circ \pi(\mathbf{S}_1) = \widetilde{F} \circ \pi(\mathbf{S}_2)$ if $\mathbf{S}_1 \sim \mathbf{S}_2$.

\begin{prop}
\label{prop:gradFS}
The horizontal lift of $\grad \widetilde{F}(\widetilde{\mathbf{S}})$ at $\mathbf{S} \in \pi^{-1}(\widetilde{\mathbf{S}})$ is
\begin{equation}
    \overline{\grad \widetilde{F}(\widetilde{\mathbf{S}})} = \grad F(\mathbf{S}) = [\mathbf{S}_i\boldsymbol{\Omega}_i]_1^{m} \label{eq:gradFS}
\end{equation}
where $\boldsymbol{\Omega}_i \coloneqq \mathbf{S}_i^T[\mathbf{C}\mathbf{S}]_i - [\mathbf{C}\mathbf{S}]_i^T\mathbf{S}_i \in \Skew(d)$ and $\textstyle\sum_1^m \boldsymbol{\Omega}_i = \mathbf{S}^T\mathbf{C}\mathbf{S} - \mathbf{S}^T\mathbf{C}\mathbf{S} = 0$ (which validates that $\overline{\grad \widetilde{F}(\widetilde{\mathbf{S}})}$ is indeed in $\mathcal{H}_{\mathbf{S}}$, see Proposition~\ref{prop:V_S_H_S}). Consequently, the set of critical points of $\widetilde{F}$ is given by $\widetilde{\mathcal{C}} = \{\widetilde{\mathbf{S}} \in \mathbb{O}(d)^m/_{\sim}: \grad \widetilde{F}(\widetilde{\mathbf{S}}) = 0\}$, equivalently,
\begin{equation}
    \widetilde{\mathcal{C}} = \{\widetilde{\mathbf{S}} \in \mathbb{O}(d)^m/_{\sim}: \mathbf{S}_i^T[\mathbf{C}\mathbf{S}]_i = [\mathbf{C}\mathbf{S}]_i^T\mathbf{S}_i, \text{ for all } i \in [1,m], \mathbf{S} \in \pi^{-1}(\widetilde{\mathbf{S}})\}, \label{eq:crit_pts}
\end{equation}
and that of $F$ is $\mathcal{C} = \{\mathbf{S} \in \mathbb{O}(d)^m: \grad F(\mathbf{S}) = 0\}$, equivalently,
\begin{equation}
     \mathcal{C} = \{\mathbf{S} \in \mathbb{O}(d)^m: \mathbf{S}_i^T[\mathbf{C}\mathbf{S}]_i = [\mathbf{C}\mathbf{S}]_i^T\mathbf{S}_i, \text{ for all } i \in [1,m]\}. \label{eq:crit_pts2}
\end{equation}
\end{prop}
% The following remark follows trivially from Eq.~(\ref{eq:crit_pts}) and (\ref{eq:crit_pts2}).
% \begin{rmk}
% \label{rmk:crit_pt_F_Ftilde}
% If $\mathbf{S}$ is a critical point of $F$ i.e. $\mathbf{S} \in \mathcal{C}$ then $\pi(\mathbf{S})$ is a critical point of $\widetilde{F}$ i.e. $\pi(\mathbf{S}) \in \widetilde{C}$. Similarly, if $\widetilde{\mathbf{S}} \in \widetilde{C}$ then $\mathbf{S} \in \mathcal{C}$ for all $\mathbf{S} \in \pi^{-1}(\widetilde{\mathbf{S}})$.
% \end{rmk}
% From Eq.~(\ref{eq:crit_pts}) and (\ref{eq:crit_pts2}), it is easy to see that if $\mathbf{S}$ is a critical point of $F$ i.e. $\mathbf{S} \in \mathcal{C}$ then $\pi(\mathbf{S})$ is a critical point of $\widetilde{F}$ i.e. $\pi(\mathbf{S}) \in \widetilde{C}$. Similarly, if $\widetilde{\mathbf{S}} \in \widetilde{C}$ then $\mathbf{S} \in \mathcal{C}$ for all $\mathbf{S} \in \pi^{-1}(\widetilde{\mathbf{S}})$.
From Eq.~(\ref{eq:crit_pts}) and (\ref{eq:crit_pts2}), it is easy to see that if $\mathbf{S} \in \mathcal{C}$ then $\pi(\mathbf{S}) \in \widetilde{C}$. Similarly, if $\widetilde{\mathbf{S}} \in \widetilde{C}$ then $\mathbf{S} \in \mathcal{C}$ for all $\mathbf{S} \in \pi^{-1}(\widetilde{\mathbf{S}})$.

% Alternatively,
% \begin{align}
%     0 &= \grad F(\mathbf{S})_i\\
%     &= \mathbf{S}_{i+1}\overline{\grad F(\mathbf{S})}_1^T + \overline{\grad F(\mathbf{S})}_{i+1}\mathbf{S}_1^T\\
%     &= \mathbf{S}_{i+1}([\mathbf{C}\mathbf{S}]_1 - \mathbf{S}_1[\mathbf{C}\mathbf{S}]_1^T\mathbf{S}_1)^T + ([\mathbf{C}\mathbf{S}]_{i+1} - \mathbf{S}_{i+1}[\mathbf{C}\mathbf{S}]_{i+1}^T\mathbf{S}_{i+1})\mathbf{S}_1^T\\
%     &= \mathbf{S}_{i}C_{11} + \textstyle\sum_{j=2}^{m}\mathbf{S}_{i}\mathbf{S}_{j-1}^TC_{j1} - \mathbf{S}_{i}(C_{11} + \textstyle\sum_{j=2}^{m}C_{1j}\mathbf{S}_{j-1})+\\
%     &\qquad\qquad C_{i+1,1} + \textstyle\sum_{j=2}^{m}C_{i+1,j}\mathbf{S}_{j-1} - (\mathbf{S}_iC_{1,i+1}\mathbf{S}_i + \textstyle\sum_{j=2}^{m}\mathbf{S}_i\mathbf{S}_{j-1}^TC_{j,i+1}\mathbf{S}_i)\\
%     &= \mathbf{S}_i\left(\mathbf{S}_i^TC_{i+1,1} - C_{1,i+1}\mathbf{S}_i + \textstyle\sum_{j=1}^{m-1}\mathbf{S}_i^TC_{i+1,j+1}\mathbf{S}_{j} - \mathbf{S}_{j}^{T}C_{j+1,i+1}\mathbf{S}_i + \mathbf{S}_{j}^{T}C_{j+1,1} - C_{1,j+1}\mathbf{S}_{j}\right)\\
%     &= \mathbf{S}_i\left(\textstyle\sum_{j=0}^{m-1}\mathbf{S}_i^TC_{i+1,j+1}\mathbf{S}_{j} - \mathbf{S}_{j}^{T}C_{j+1,i+1}\mathbf{S}_i + \mathbf{S}_{j}^{T}C_{j+1,1} - C_{1,j+1}\mathbf{S}_{j}\right)\\
%     &= 2\mathbf{S}_i \left(\Skew\left(\mathbf{S}_i^T \left[C\begin{bmatrix}I_d\\S\end{bmatrix}\right]_{i+1}\right) - \Skew\left( \left[C\begin{bmatrix}I_d\\S\end{bmatrix}\right]_{1}\right)\right)
% \end{align}
% where $\mathbf{S}_0 = I_d$. Summing up for $i=1:m-1$ we obtain
% \begin{align}
%     \textstyle\sum_{j=0}^{m-1} \mathbf{S}_j^TC_{j+1,1} - C_{1,j+1}\mathbf{S}_j = 0 \implies \left[C\begin{bmatrix}
%     I_d\\S
%     \end{bmatrix}\right]_1 = \left[C\begin{bmatrix}
%     I_d\\S
%     \end{bmatrix}\right]_1^T.
% \end{align}
% and for $i=1:m-1$,
% \begin{align}
%     \textstyle\sum_{j=0}^{m-1}\mathbf{S}_i^TC_{i+1,j+1}\mathbf{S}_{j} - \mathbf{S}_{j}^{T}C_{j+1,i+1}\mathbf{S}_i = 0 \implies \mathbf{S}_i^T \left[C\begin{bmatrix}
%     I_d\\S
%     \end{bmatrix}\right]_i = \left[C\begin{bmatrix}
%     I_d\\S
%     \end{bmatrix}\right]_i^T\mathbf{S}_i
% \end{align}
% The above equations can be written in a more compact form as
% \begin{align}
%     \blockdiag\left(C\begin{bmatrix}
%     I_d\\S
%     \end{bmatrix}\begin{bmatrix}I_d & \mathbf{S}^T\end{bmatrix}\right) = \blockdiag\left(\begin{bmatrix}I_d\\S\end{bmatrix}\begin{bmatrix}
%     I_d & \mathbf{S}^T
%     \end{bmatrix}C\right)
% \end{align}

\begin{prop}
\label{prop:DgradFSZ}
Let \revadd{$\widetilde{\mathbf{S}} \in \mathbb{O}(d)^{m}/_{\sim}$} then for every $\mathbf{S} \in \pi^{-1}(\widetilde{\mathbf{S}})$, $\mathbf{Z} \in T_\mathbf{S}\mathbb{O}(d)^m$,
$$D\grad F(\mathbf{S})[\mathbf{Z}] = \left[\mathbf{S}_i(\mathbf{S}_i^T[\mathbf{C}\mathbf{Z}]_i - [\mathbf{C}\mathbf{Z}]_i^T\mathbf{S}_i - [\mathbf{C}\mathbf{S}]_i^T\mathbf{Z}_i + \mathbf{Z}_i^T\mathbf{S}_i[\mathbf{C}\mathbf{S}]_i^T\mathbf{S}_i)\right]_1^m.$$
% \revadd{
% \begin{equation}
%     D\grad F(\mathbf{S})[\mathbf{Z}] = \left[\mathbf{S}_i(\mathbf{S}_i^T[\mathbf{C}\mathbf{Z}]_i - [\mathbf{C}\mathbf{Z}]_i^T\mathbf{S}_i - [\mathbf{C}\mathbf{S}]_i^T\mathbf{Z}_i + \mathbf{Z}_i^T\mathbf{S}_i[\mathbf{C}\mathbf{S}]_i^T\mathbf{S}_i)\right]_1^m.
% \end{equation}
% }
\revadd{Moreover, if $\widetilde{\mathbf{S}} \in \widetilde{\mathcal{C}}$ then} 
$$D\grad F(\mathbf{S})[\mathbf{Z}] = \left[\mathbf{S}_i(\mathbf{S}_i^T[\mathbf{C}\mathbf{Z}]_i - [\mathbf{C}\mathbf{Z}]_i^T\mathbf{S}_i - [\mathbf{C}\mathbf{S}]_i^T\mathbf{Z}_i + \mathbf{Z}_i^T[\mathbf{C}\mathbf{S}]_i)\right]_1^m.$$
% \begin{equation}
%     D\grad F(\mathbf{S})[\mathbf{Z}] = \left[\mathbf{S}_i(\mathbf{S}_i^T[\mathbf{C}\mathbf{Z}]_i - [\mathbf{C}\mathbf{Z}]_i^T\mathbf{S}_i - [\mathbf{C}\mathbf{S}]_i^T\mathbf{Z}_i + \mathbf{Z}_i^T[\mathbf{C}\mathbf{S}]_i)\right]_1^m.
% \end{equation}
\end{prop}

Let $\nabla$ be the Levi-Civita connection (also known as the Riemannian connection) on $\mathbb{O}(d)^m$ and $\widetilde{\nabla}$ be the induced connection on $\mathbb{O}(d)^m /_{\sim}$. Then, from Proposition~\ref{prop:DgradFSZ} and the definition of the Riemannian Hessian operator \citeb[Section 5.5]{absil2009optimization}, we obtain
\begin{prop}
\label{prop:HessFSZ}
Let \revdel{$\widetilde{\mathbf{S}} \in \widetilde{\mathcal{C}}$}\revadd{$\widetilde{\mathbf{S}} \in \mathbb{O}(d)^m/_{\sim}$} and $\widetilde{\mathbf{Z}} \in T_{\widetilde{\mathbf{S}}}\mathbb{O}(d)^{m}/_{\sim}$. Let $\mathbf{Z}$ be the horizontal lift of $\widetilde{\mathbf{Z}}$ at $\mathbf{S} \in \pi^{-1}(\widetilde{\mathbf{S}})$. Then the horizontal lift of $\Hess \widetilde{F}(\widetilde{\mathbf{S}})[\widetilde{\mathbf{Z}}]$ at $\mathbf{S}$ is
\begin{equation}
    \overline{\Hess \widetilde{F}(\widetilde{\mathbf{S}})[\widetilde{\mathbf{Z}}]} = [\mathbf{S}_i\widehat{\boldsymbol{\Omega}}_i]_1^m \label{eq:eq3}
\end{equation}
where \revadd{$\widehat{\boldsymbol{\Omega}}_i = \Skew(\boldsymbol{\xi}_i) - m^{-1}\sum_1^m\Skew(\boldsymbol{\xi}_i)$ and $$\boldsymbol{\xi}_i = \mathbf{S}_i^T[\mathbf{C}\mathbf{Z}]_i - [\mathbf{C}\mathbf{Z}]_i^T\mathbf{S}_i - [\mathbf{C}\mathbf{S}]_i^T\mathbf{Z}_i + \mathbf{Z}_i^T\mathbf{S}_i[\mathbf{C}\mathbf{S}]_i^T\mathbf{S}_i,$$ and if $\widetilde{\mathbf{S}} \in \widetilde{\mathcal{C}}$ then }
%\begin{align}
$$\widehat{\boldsymbol{\Omega}}_i = \mathbf{S}_i^T[\mathbf{C}\mathbf{Z}]_i - [\mathbf{C}\mathbf{Z}]_i^T\mathbf{S}_i - [\mathbf{C}\mathbf{S}]_i^T\mathbf{Z}_i + \mathbf{Z}_i^T[\mathbf{C}\mathbf{S}]_i.$$ %\label{eq:Omega_hat_i}
%\end{align}
\end{prop}

Now we obtain a compact representation for $\widehat{\boldsymbol{\Omega}}_i$. We first define certain matrices, then use them to obtain an expression for $\widehat{\boldsymbol{\Omega}}_i$ and then describe their structure. Recall that $\mathbf{C} = \mathbf{D} - \mathbf{B}\boldsymbol{\mathcal{L}}_{\Gamma}^\dagger \mathbf{B}^T$ (see Eq.~(\ref{eq:GPOP}) and Remark~\ref{rmk:L0DB}) and for convenience define
\begin{equation}
    \mathbf{B}(\mathbf{S}) \coloneqq \blockdiag((\mathbf{S}_i)_1^m)^T\ \mathbf{B}  = [\mathbf{S}_i^T\mathbf{B}_i]_1^m \label{eq:BofS}
\end{equation}
and
$$\mathbf{D}(\mathbf{S}) \coloneqq \blockdiag((\mathbf{S}_i)_1^m)^T\ \mathbf{D}\ \blockdiag((\mathbf{S}_i)_1^m) = \blockdiag((\mathbf{S}_i^T\mathbf{D}_{ii}\mathbf{S}_i)_1^m).$$ Using these matrices, we also define,
% \begin{equation}
%     \mathbf{D}(\mathbf{S}) \coloneqq \blockdiag((\mathbf{S}_i)_1^m)^T\ \mathbf{D}\ \blockdiag((\mathbf{S}_i)_1^m) = \blockdiag((\mathbf{S}_i^T\mathbf{D}_{ii}\mathbf{S}_i)_1^m).
% \end{equation}
%Then define
\begin{align}
    \mathbf{C}(\mathbf{S}) &\coloneqq \blockdiag((\mathbf{S}_i)_1^m)^T\ \mathbf{C}\ \blockdiag((\mathbf{S}_i)_1^m) = \mathbf{D}(\mathbf{S}) - \mathbf{B}(\mathbf{S})\boldsymbol{\mathcal{L}}_{\Gamma}^\dagger \mathbf{B}(\mathbf{S})^T\label{eq:C_of_S}\\
    \widehat{\mathbf{C}}(\mathbf{S}) &\coloneqq \blockdiag(([\mathbf{C}(\mathbf{S})\mathbf{I}_d^m]_i)_1^m) \label{eq:C_hat}\\
    \revadd{\mathbf{L}(\mathbf{S})} &\coloneqq \revadd{\mathbf{C}(\mathbf{S}) - \widehat{\mathbf{C}}(\mathbf{S})}. \label{eq:L_of_S}
\end{align}
\begin{rmk}
\label{rmk:StildeCtilde}
From the definition of $\mathcal{C}$ (see Eq.~(\ref{eq:crit_pts2})), $\mathbf{S} \in \mathcal{C}$ iff $\widehat{\mathbf{C}}(\mathbf{S}) \in \Sym(md)$ i.e. for each $i \in [1,m]$, $[\mathbf{C}(\mathbf{S})\mathbf{I}_d^m]_i = [\mathbf{C}(\mathbf{S})\mathbf{I}_d^m]_i^T$ or equivalently,
\begin{equation}
     \textstyle\sum_{j=1}^{m}\mathbf{C}(\mathbf{S})_{ij} = \mathbf{S}_i^T[\mathbf{C}\mathbf{S}]_i = [\mathbf{C}\mathbf{S}]_i^T\mathbf{S}_i = \textstyle\sum_{j=1}^{m}\mathbf{C}(\mathbf{S})_{ij}^T. \label{eq:eq2}
\end{equation}
\end{rmk}
%Now we present some remarks on the structure of $\mathbf{C}(\mathbf{S})$ and $\mathbf{L}(\mathbf{S})$.
\begin{rmk}
\label{rmk:C_S_structure}
Since $\blockdiag((\mathbf{S}_i)_1^m)$ is an orthogonal matrix, $\mathbf{C}(\mathbf{S})$ is unitarily equivalent to $\mathbf{C}$. Thus, $\mathbf{C}(\mathbf{S}) \in \Sym(md)$, $\mathbf{C}(\mathbf{S}) \succeq 0$, $\rank(\mathbf{C}(\mathbf{S})) = \rank(\mathbf{C})$ and the $(i,j)$th block of size $d$ in $\mathbf{C}(\mathbf{S})$ is $\mathbf{C}(\mathbf{S})_{ij} = \delta_{ij}\mathbf{D}(\mathbf{S})_{ii} - \mathbf{B}(\mathbf{S})_i\boldsymbol{\mathcal{L}}_{\Gamma}^\dagger \mathbf{B}(\mathbf{S})_j^T$ where $\mathbf{B}(\mathbf{S})_i$ is the $i$th row block of $\mathbf{B}(\mathbf{S})$ of dimension $d \times (m+n)$.
\end{rmk}
% \begin{rmk}
% \label{rmk:C_hat_L_structure}
% If $\widetilde{\mathbf{S}} \in \widetilde{\mathcal{C}}$ then for every $\mathbf{S} \in \pi^{-1}(\mathbf{S})$, the following hold.
% \begin{enumerate}[leftmargin=*]
%     \item $\mathbf{L}(\mathbf{S}) \in \Sym(md)$.
%     \item $\textstyle\sum_{j=1}^{m}\mathbf{L}(\mathbf{S})_{ij} = \textstyle\sum_{j=1}^{m}\mathbf{L}(\mathbf{S})_{ji} = 0$ for all $i \in [1,m]$ (see Eq.~(\ref{eq:eq2})).
%     \item For each $i \in [1,d]$, the vector $\mathbf{1}_m \otimes \mathbf{e}^d_i$ lies in the kernel of $\mathbf{L}(\mathbf{S})$ and thus the rank of $\mathbf{L}(\mathbf{S})$ is atmost $(m-1)d$.
%     \item If $\boldsymbol{\Omega} = [\boldsymbol{\Omega}_0]_1^m$ for some $\boldsymbol{\Omega}_0 \in \Skew(d)$ then $\mathbf{L}(\mathbf{S})\boldsymbol{\Omega} = 0$.
%     \item The $(i,j)$th block of size $d$ in $\mathbf{L}(\mathbf{S})$ is $-\delta_{ij}\mathbf{B}(\mathbf{S})_i\boldsymbol{\mathcal{L}}_{\Gamma}^\dagger \mathbf{B}(\mathbf{S})^T\mathbf{I}^m_d + \mathbf{B}(\mathbf{S})_i\boldsymbol{\mathcal{L}}_{\Gamma}^\dagger \mathbf{B}(\mathbf{S})_j^T$, equivalently $-\delta_{ij}(\mathbf{I}^m_d)^T\mathbf{B}(\mathbf{S})\boldsymbol{\mathcal{L}}_{\Gamma}^\dagger \mathbf{B}(\mathbf{S})_i^T + \mathbf{B}(\mathbf{S})_i\boldsymbol{\mathcal{L}}_{\Gamma}^\dagger \mathbf{B}(\mathbf{S})_j^T $. In particular, $\mathbf{L}(\mathbf{S})$ does not depend on $\mathbf{D}(\mathbf{S})$.
%     \item Let $\mathbf{Q} \in \mathbb{O}(d)$ then $\mathbf{L}(\mathbf{S}\mathbf{Q}) = (\mathbf{I}_m \otimes \mathbf{Q})^T\mathbf{L}(\mathbf{S})(\mathbf{I}_m \otimes \mathbf{Q})$ (follows from Eq.~(\ref{eq:C_of_S}, \ref{eq:C_hat}, \ref{eq:eq2})) and since $\mathbf{I}_m \otimes \mathbf{Q}$ is orthogonal, $L(\mathbf{S}\mathbf{Q})$ is unitarily equivalent to $\mathbf{L}(\mathbf{S})$.
% \end{enumerate}
% \end{rmk}
\begin{rmk}
\label{rmk:C_hat_L_structure}
\revadd{For a general $\widetilde{\mathbf{S}} \in \mathbb{O}(d)^m/_{\sim}$ and $\mathbf{S} \in \pi^{-1}(\mathbf{S})$, $\mathbf{L}(\mathbf{S})$ may not be symmetric and the $(i,j)$th block of size $d$ in $\mathbf{L}(\mathbf{S})$ is $\delta_{ij}\mathbf{B}(\mathbf{S})_i\boldsymbol{\mathcal{L}}_{\Gamma}^\dagger \mathbf{B}(\mathbf{S})^T\mathbf{I}^m_d - \mathbf{B}(\mathbf{S})_i\boldsymbol{\mathcal{L}}_{\Gamma}^\dagger \mathbf{B}(\mathbf{S})_j^T$. In particular, $\mathbf{L}(\mathbf{S})$ does not depend on $\mathbf{D}(\mathbf{S})$. Since the sum $\sum_{j=1}^{m}\mathbf{L}(\mathbf{S})_{ij} = 0$, for each $k \in [1,d]$, the vector $\mathbf{1}_m \otimes \mathbf{e}^d_k$ lies in the kernel of $\mathbf{L}(\mathbf{S})$ and thus the rank of $\mathbf{L}(\mathbf{S})$ is at most $(m-1)d$. Equivalently, if $\boldsymbol{\Omega} = [\boldsymbol{\Omega}_0]_1^m$ for some $\boldsymbol{\Omega}_0 \in \Skew(d)$ then $\mathbf{L}(\mathbf{S})\boldsymbol{\Omega} = 0$. Finally, if $\mathbf{Q} \in \mathbb{O}(d)$ then $\mathbf{L}(\mathbf{S}\mathbf{Q}) = (\mathbf{I}_m \otimes \mathbf{Q})^T\mathbf{L}(\mathbf{S})(\mathbf{I}_m \otimes \mathbf{Q})$ (follows from Eq.~(\ref{eq:C_of_S}, \ref{eq:C_hat}, \ref{eq:eq2})) and since $\mathbf{I}_m \otimes \mathbf{Q}$ is orthogonal, $L(\mathbf{S}\mathbf{Q})$ is unitarily equivalent to $\mathbf{L}(\mathbf{S})$.} Now, if $\widetilde{\mathbf{S}} \in \widetilde{\mathcal{C}}$ then for every $\mathbf{S} \in \pi^{-1}(\mathbf{S})$, $\mathbf{L}(\mathbf{S}) \in \Sym(md)$ and $\textstyle\sum_{j=1}^{m}\mathbf{L}(\mathbf{S})_{ij} = \textstyle\sum_{j=1}^{m}\mathbf{L}(\mathbf{S})_{ji} = 0$ for all $i \in [1,m]$ (see Eq.~(\ref{eq:eq2})).
%Then, the $(i,j)$th block of size $d$ in $\mathbf{L}(\mathbf{S})$ is $-\delta_{ij}\mathbf{B}(\mathbf{S})_i\boldsymbol{\mathcal{L}}_{\Gamma}^\dagger \mathbf{B}(\mathbf{S})^T\mathbf{I}^m_d + \mathbf{B}(\mathbf{S})_i\boldsymbol{\mathcal{L}}_{\Gamma}^\dagger \mathbf{B}(\mathbf{S})_j^T$, equivalently $-\delta_{ij}(\mathbf{I}^m_d)^T\mathbf{B}(\mathbf{S})\boldsymbol{\mathcal{L}}_{\Gamma}^\dagger \mathbf{B}(\mathbf{S})_i^T + \mathbf{B}(\mathbf{S})_i\boldsymbol{\mathcal{L}}_{\Gamma}^\dagger \mathbf{B}(\mathbf{S})_j^T $.
\end{rmk}

%\subsection{Non-singular and Positive Definite Hessian}
%\label{subsec:non_sing_pos_def_hess}
%A non-degenerate local minimum is defined to be the critical point at which the Hessian is positive definite. Similarly, a non-degenerate critical point is the one where the Hessian is non-singular.
We proceed to derive equations that will be used to determine the conditions under which the Hessian is non-singular and positive definite.
\begin{prop}
\label{prop:Omega_hat_compact}
Consider the same setup as in Proposition~\ref{prop:HessFSZ}. Then
\revadd{
\begin{equation}
    \widehat{\boldsymbol{\Omega}}_i = \left\{\begin{matrix} [\mathbf{L}(\mathbf{S})\boldsymbol{\Omega}]_i - [\mathbf{L}(\mathbf{S})\boldsymbol{\Omega}]_i^T, & \text{if } \widetilde{\mathbf{S}} \in \widetilde{\mathcal{C}}\\
    \frac{1}{2}\left\{[(\mathbf{L}(\mathbf{S}) + \mathbf{L}(\mathbf{S})^T)\boldsymbol{\Omega}]_i - [(\mathbf{L}(\mathbf{S}) + \mathbf{L}(\mathbf{S})^T)\boldsymbol{\Omega}]_i^T\right\}, & \text{otherwise}.\end{matrix}\right. \label{eq:eq4}
\end{equation}
}where $\boldsymbol{\Omega}=[\boldsymbol{\Omega}_i]_1^m$ and $\boldsymbol{\Omega}_i \in \Skew(d)$ is such that $\mathbf{Z}_i = \mathbf{S}_i\boldsymbol{\Omega}_i$ and $\textstyle\sum_1^m\boldsymbol{\Omega}_i = 0$.
\end{prop}

Combining the above with Proposition~\ref{prop:HessFSZ}, one can obtain a slightly compact representation of the horizontal lift of the Hessian, and the following result.
%, which in turn characterize the non-degenerate critical points and local minima. First, we need the following result.
\begin{prop}
\label{prop:HessFSZZ}
Let \revadd{$\widetilde{\mathbf{S}} \in \mathbf{O}(d)^m/_{\sim}$}\revdel{$\widetilde{\mathbf{S}} \in \widetilde{\mathcal{C}}$} and $\widetilde{\mathbf{Z}} \in T_{\widetilde{\mathbf{S}}}\mathbb{O}(d)^{m}/_{\sim}$. Let $\mathbf{Z}$ be the horizontal lift of $\widetilde{\mathbf{Z}}$ at $\mathbf{S} \in \pi^{-1}(\widetilde{\mathbf{S}})$. Then
\begin{equation}
    \widetilde{g}(\Hess \widetilde{F}(\widetilde{\mathbf{S}})[\widetilde{\mathbf{Z}}],\widetilde{\mathbf{Z}}) = \revadd{\Tr(\boldsymbol{\Omega}^T(\mathbf{L}(\mathbf{S})+\mathbf{L}(\mathbf{S})^T)\boldsymbol{\Omega})} =  2\Tr(\boldsymbol{\Omega}^T\mathbf{L}(\mathbf{S})\boldsymbol{\Omega}) \label{eq:TrOmegaTLSOmega}
\end{equation}
where $\boldsymbol{\Omega}=[\boldsymbol{\Omega}_i]_1^m$ and $\boldsymbol{\Omega}_i \in \Skew(d)$ is such that $\mathbf{Z}_i = \mathbf{S}_i\boldsymbol{\Omega}_i$ and $\textstyle\sum_1^m\boldsymbol{\Omega}_i = 0$.
\end{prop}
%
Then the non-singularity and the positive definiteness of the Hessian amounts to the right side of Eq.~(\ref{eq:TrOmegaTLSOmega}) being non-zero and positive, respectively, for every non-zero $\boldsymbol{\Omega}$. Although $\mathbf{C}(\mathbf{S})$, and thus $\mathbf{L}(\mathbf{S})$, can be calculated from the patch framework $\Theta$ and the alignment $\mathbf{S}$, it is not obvious how to test the above practically. The main issue is that $\boldsymbol{\Omega}$ in Eq.~(\ref{eq:TrOmegaTLSOmega}) is not unconstrained, and in fact has a specific structure.

For the above reason, we are going to manipulate Eq.~(\ref{eq:TrOmegaTLSOmega}), utilizing the structure of $\boldsymbol{\Omega}$. The aim is to obtain an expression of the form $\boldsymbol{\omega}^T \mathbb{L}(\mathbf{S})\boldsymbol{\omega}$ where (i) the vector $\boldsymbol{\omega}$ is essentially unconstrained and (ii) $\boldsymbol{\Omega}$ and $\mathbf{L}(\mathbf{S})$ are related to $\boldsymbol{\omega}$ and $\mathbb{L}(\mathbf{S})$, respectively, through permutation matrices and vectorization operations (and thus the two pairs carry the same information). To achieve that, we first define certain matrices, then rewrite Eq.~(\ref{eq:TrOmegaTLSOmega}) in terms of those matrices and then describe their structure.

% In the following, we are going to manipulate Eq.~(\ref{eq:TrOmegaTLSOmega}), utilizing the structure of $\boldsymbol{\Omega}$, so as to obtain an expression of the form $\boldsymbol{\omega}^T \mathbb{L}(\mathbf{S})\boldsymbol{\omega}$ where (i) the vector $\boldsymbol{\omega}$ is essentially unconstrained and (ii) $\boldsymbol{\Omega}$ and $\mathbf{L}(\mathbf{S})$ are related to $\boldsymbol{\omega}$ and $\mathbb{L}(\mathbf{S})$, respectively, through permutation matrices and vectorization operations (and thus the two pairs carry the same information).
%To achieve that, we first define certain matrices, then rewrite Eq.~(\ref{eq:TrOmegaTLSOmega}) in terms of those matrices and then describe their structure.

To this end, for $\boldsymbol{\Omega} = [\boldsymbol{\Omega}_i]_1^m$ where $\boldsymbol{\Omega}_i \in \Skew(d)$, let $\{\boldsymbol{\Omega}_{i}(r,s): 1 \leq r < s \leq d\}$ be the elements in the upper triangular region of $\boldsymbol{\Omega}_i$. For a fixed pair $(r,s)$ such that $1 \leq r < s \leq d$, define the column vector $\boldsymbol{\omega}_{r,s} \coloneqq [\boldsymbol{\Omega}_{i}(r,s)]_{i=1}^{m} \in \mathbb{R}^m$, a vertical stack of the $(r,s)$th element of each $\boldsymbol{\Omega}_i$. Then there exists a permutation matrix $\mathbf{P}$ such that
\begin{align}
    \mathbf{P}\boldsymbol{\Omega} &= \begin{bmatrix}
    \mathbf{0}_{m} & \boldsymbol{\omega}_{1,2} &  \ldots & \boldsymbol{\omega}_{1,d-1} & \boldsymbol{\omega}_{1,d}\\
    -\boldsymbol{\omega}_{1,2} & \mathbf{0}_{m}  & \ldots & \boldsymbol{\omega}_{2,d-1} & \boldsymbol{\omega}_{2,d}\\
        \vdots & \vdots & \vdots & \vdots & \vdots\\
    -\boldsymbol{\omega}_{1,d-1} & -\boldsymbol{\omega}_{2,d-1} & \ldots & \mathbf{0}_{m} & \boldsymbol{\omega}_{d-1,d}\\
    -\boldsymbol{\omega}_{1,d} & -\boldsymbol{\omega}_{2,d}  & \ldots & -\boldsymbol{\omega}_{d-1,d} & \mathbf{0}_{m}
    \end{bmatrix}. \label{eq:P0Omega}
\end{align}
In words, for $1 \leq r < s \leq d$, the $(r,s)$th block of $\mathbf{P}\boldsymbol{\Omega}$ is a vertical stack of the $(r,s)$th element of each $\boldsymbol{\Omega}_{i}$. For $r = s$, this is just a zero vector and for $r > s$, this is $-\boldsymbol{\omega}_{s,r}$.

Then, we collect the (strictly) upper triangular elements of $\mathbf{P}\boldsymbol{\Omega}$ in the column-major order in the vector $\boldsymbol{\omega}$. Note that $\mathbf{P}\boldsymbol{\Omega}$ can be fully described by $\boldsymbol{\omega}$. In particular, there exist a block matrix $\overline{\mathbf{P}}$ of size $d(d-1)/2 \times d^2$ siuch that $\vecz (\mathbf{P}\boldsymbol{\Omega}) = \overline{\mathbf{P}}^T \boldsymbol{\omega}$. The blocks of $\overline{\mathbf{P}}$ when indexed using tuples $(r,s)$ and $(p,q)$ where $1 \leq r < s \leq d$ and $p,q \in [1,d]$, are as follows: $\mathbf{0}_{m \times m}$ when $p=q$, $\delta_{pr}\delta_{qs}\mathbf{I}_m$ when $p < q$, and $-\delta_{ps}\delta_{qr}\mathbf{I}_m$ when $p > q$. Finally, we define $\mathcal{B}(\mathbf{S}) \coloneqq \mathbf{P}\mathbf{B}(\mathbf{S})$,
\begin{align}
    \boldsymbol{\mathcal{L}}(\mathbf{S}) &\coloneqq \mathbf{P}\mathbf{L}(\mathbf{S})\mathbf{P}^T \label{eq:mathcal_L}\\
    \mathbb{L}(\mathbf{S}) &\coloneqq \overline{\mathbf{P}}(\mathbf{I}_d \otimes \boldsymbol{\mathcal{L}}(\mathbf{S}))\overline{\mathbf{P}}^T \label{eq:mathbb_L}
\end{align}
%and then note the following,

\begin{prop}
\label{prop:Omega^TLSOmega2}
Consider the same setup as in Proposition~\ref{prop:HessFSZZ}. Then
\begin{equation}
    \Tr(\boldsymbol{\Omega}^T\mathbf{L}(\mathbf{S})\boldsymbol{\Omega}) = \boldsymbol{\omega}^T\mathbb{L}(\mathbf{S})\boldsymbol{\omega} \label{eq:omega^TmbbLomega}
\end{equation}
\end{prop}

The following remarks reveal the structure of $\mathcal{B}(\mathbf{S})$, $\boldsymbol{\mathcal{L}}(\mathbf{S})$ and $\mathbb{L}(\mathbf{S})$.
\begin{rmk}
\label{rmk:mathcalBS}
For $p \in [1,d]$, the $p$th row-block of $\mathcal{B}(\mathbf{S})$, $\mathcal{B}(\mathbf{S})_p$, is of size $m \times (n+m)$, and can be viewed as a vertical stack of the $p$th rows of $\mathbf{B}(\mathbf{S})_i$, $i \in [1,m]$. In particular, $\mathcal{B}(\mathbf{S})_p$ depends only on the $p$th coordinate of the local views (see Remark~\ref{rmk:L0DB}).
\end{rmk}
\begin{rmk}
\label{rmk:mathcalLS}
% For $p,\mathbf{Q} \in [1,d]$ and $i,j \in [1,m]$, let $\boldsymbol{\mathcal{L}}(\mathbf{S})_{pq}$ be the $(p,q)$th block of size $m$ in $\boldsymbol{\mathcal{L}}(\mathbf{S})$ and $\mathbf{L}(\mathbf{S})_{ij}$ be the $(i,j)$th block of size $d$ in $\mathbf{L}(\mathbf{S})$. Then, from Eq.~(\ref{eq:mathcal_L}) it is easy to deduce
% \begin{align}
%     \boldsymbol{\mathcal{L}}(\mathbf{S})_{pq}(i,j) &= \mathbf{L}(\mathbf{S})_{ij}(p,q)\\
%     &= -\delta_{ij} \textstyle\sum_{k=1}^m \mathbf{B}(\mathbf{S})_{i}(p,:)\boldsymbol{\mathcal{L}}_{\Gamma}^\dagger \mathbf{B}(\mathbf{S})_{k}(q,:)^T + \mathbf{B}(\mathbf{S})_i(p,:)\boldsymbol{\mathcal{L}}_{\Gamma}^\dagger \mathbf{B}(\mathbf{S})_j(q,:)^T\\
%     &= -\delta_{ij}\mathcal{B}(\mathbf{S})_{p}(i,:)\boldsymbol{\mathcal{L}}_{\Gamma}^\dagger \mathcal{B}(\mathbf{S})_{q}^T\mathbf{1}_m + \mathcal{B}(\mathbf{S})_{p}(i,:)\boldsymbol{\mathcal{L}}_{\Gamma}^\dagger \mathcal{B}(\mathbf{S})_{q}(j,:)^T
% \end{align}
% and thus
% \begin{align}
%     \boldsymbol{\mathcal{L}}(\mathbf{S})_{pq} &= -\diag (\mathcal{B}(\mathbf{S})_p\boldsymbol{\mathcal{L}}_{\Gamma}^\dagger \mathcal{B}(\mathbf{S})_{q}^T\mathbf{1}_m) + \mathcal{B}(\mathbf{S})_p\boldsymbol{\mathcal{L}}_{\Gamma}^\dagger \mathcal{B}(\mathbf{S})_q^T
% \end{align}
% Thus $\boldsymbol{\mathcal{L}}(\mathbf{S})_{pq}$ depends on the structure of $\Gamma$ through $\boldsymbol{\mathcal{L}}_{\Gamma}^\dagger$ and the $p$th and $q$th coordinates of the rigidly transformed local views $[\mathbf{S}_iB_i]_1^m$.
\revadd{For a general $\widetilde{\mathbf{S}} \in \mathbb{O}(d)^m/_{\sim}$ and $\mathbf{S} \in \pi^{-1}(\widetilde{\mathbf{S}})$, the matrix $\mathbf{\mathcal{L}}(\mathbf{S})$ may not be symmetric and for $p,q \in [1,d]$, $\boldsymbol{\mathcal{L}}(\mathbf{S})_{p,q} = \diag (\mathcal{B}(\mathbf{S})_p\boldsymbol{\mathcal{L}}_{\Gamma}^\dagger \mathcal{B}(\mathbf{S})_{q}^T\mathbf{1}_m) - \mathcal{B}(\mathbf{S})_p\boldsymbol{\mathcal{L}}_{\Gamma}^\dagger \mathcal{B}(\mathbf{S})_q^T$. Thus $\boldsymbol{\mathcal{L}}(\mathbf{S})_{p,q}$ depends on $\Gamma$ through $\boldsymbol{\mathcal{L}}_{\Gamma}^\dagger$ and the $p$th and $q$th coordinates of the rigidly transformed local views $\mathbf{B}(\mathbf{S})$. Since constant vectors are in the kernel of $\boldsymbol{\mathcal{L}}(\mathbf{S})_{p,q}$, therefore for each $p \in [1,d]$, the vector $\mathbf{e}^d_p \otimes \mathbf{1}_m$ lies in the kernel of $\boldsymbol{\mathcal{L}}(\mathbf{S})$. Thus the rank of $\boldsymbol{\mathcal{L}}(\mathbf{S})$ is atmost $(m-1)d$. Fianally, if $\mathbf{Q} \in \mathbb{O}(d)$ then, from Remark~\ref{rmk:C_hat_L_structure} and Eq.~(\ref{eq:mathcal_L}), it follows that $\boldsymbol{\mathcal{L}}(\mathbf{S}\mathbf{Q})$ is unitarily equivalent to $\boldsymbol{\mathcal{L}}(\mathbf{S})$. Now, if  $\widetilde{\mathbf{S}} \in \widetilde{\mathcal{C}}$ then $\mathbf{\mathcal{L}}(\mathbf{S})$ is symmetric and in particular $\diag (\mathcal{B}(\mathbf{S})_p\boldsymbol{\mathcal{L}}_{\Gamma}^\dagger \mathcal{B}(\mathbf{S})_{q}^T\mathbf{1}_m) = \diag (\mathcal{B}(\mathbf{S})_q\boldsymbol{\mathcal{L}}_{\Gamma}^\dagger \mathcal{B}(\mathbf{S})_{p}^T\mathbf{1}_m)$ which results in $\boldsymbol{\mathcal{L}}(\mathbf{S})_{p,q}^T = \boldsymbol{\mathcal{L}}(\mathbf{S})_{q,p}$ for all $p,q \in [1,d]$.
}\revdel{If $\widetilde{\mathbf{S}} \in \widetilde{\mathcal{C}}$ then for every $\mathbf{S} \in \pi^{-1}(\widetilde{\mathbf{S}})$, the following hold for $\mathbf{L}(\mathbf{S})$. First, since $\mathbf{L}(\mathbf{S})$ is symmetric therefore $\boldsymbol{\mathcal{L}}(\mathbf{S})_{q,p} = \boldsymbol{\mathcal{L}}(\mathbf{S})_{p,q}^T$. In particular, for $p,q \in [1,d]$, $\boldsymbol{\mathcal{L}}(\mathbf{S})_{p,q} = -\diag (\mathcal{B}(\mathbf{S})_p\boldsymbol{\mathcal{L}}_{\Gamma}^\dagger \mathcal{B}(\mathbf{S})_{q}^T\mathbf{1}_m) + \mathcal{B}(\mathbf{S})_p\boldsymbol{\mathcal{L}}_{\Gamma}^\dagger \mathcal{B}(\mathbf{S})_q^T$. Thus $\boldsymbol{\mathcal{L}}(\mathbf{S})_{p,q}$ depends on $\Gamma$ through $\boldsymbol{\mathcal{L}}_{\Gamma}^\dagger$ and the $p$th and $q$th coordinates of the rigidly transformed local views $\mathbf{B}(\mathbf{S})$. Moreover, $\boldsymbol{\mathcal{L}}(\mathbf{S})_{p,q}$ is a Laplacian-like matrix and the constant vectors are in its kernel. Moreover, for each $p \in [1,d]$, the vector $\mathbf{e}^d_p \otimes \mathbf{1}_m$ lies in the kernel of $\boldsymbol{\mathcal{L}}(\mathbf{S})$, thus the rank of $\boldsymbol{\mathcal{L}}(\mathbf{S})$ is atmost $(m-1)d$. Finally, if $\mathbf{Q} \in \mathbb{O}(d)$ then, from Remark~\ref{rmk:C_hat_L_structure} and Eq.~(\ref{eq:mathcal_L}), it follows that $\boldsymbol{\mathcal{L}}(\mathbf{S}\mathbf{Q})$ is unitarily equivalent to $\boldsymbol{\mathcal{L}}(\mathbf{S})$.}
\end{rmk}

\revadd{
The following characterization of the matrices $\mathcal{L}(\mathbf{S})_{p,p}$ will be useful for understanding the geometrical aspects of non-degenerate alignments in the case of $d=2$.}
\begin{rmk}
\label{rmk:mathcalLpp}
\revadd{For a general $\widetilde{\mathbf{S}} \in \mathbb{O}(d)^m/_{\sim}$ and $\mathbf{S} \in \pi^{-1}(\widetilde{\mathbf{S}})$, the symmetric Laplacian-like matrix $\mathcal{L}(\mathbf{S})_{p,p} = \diag (\mathcal{B}(\mathbf{S})_p\boldsymbol{\mathcal{L}}_{\Gamma}^\dagger \mathcal{B}(\mathbf{S})_{p}^T\mathbf{1}_m) - \mathcal{B}(\mathbf{S})_p\boldsymbol{\mathcal{L}}_{\Gamma}^\dagger \mathcal{B}(\mathbf{S})_p^T$ as defined in above remark can be constructed in the following way. Consider transforming all the views by $\mathbf{S}$ and projecting them to the $p$th coordinate, $p \in [1,d]$, i.e. replacing $\mathbf{x}_{k,i} \in \mathbb{R}^d$ to $\mathbf{S}_i^T\mathbf{x}_{k,i}(p) \in \mathbb{R}$. Now, to construct $\mathcal{L}(\mathbf{S})_{p,p}$, we (i) construct the positive semidefinite kernel $\mathbf{B}\boldsymbol{\mathcal{L}}_{\Gamma}^{\dagger}\mathbf{B}^T$ as in (Eq.~\ref{eq:GPOP}) using the one-dimensional patch framework $(\Gamma, (\mathbf{S}_i^T\mathbf{x}_{k,i}(p)))$, and then (ii) construct an unnormalized graph Laplacian matrix \citea{belkin2003laplacian} using the constructed kernel. Consequently, $\boldsymbol{\mathcal{L}}(\mathbf{S})_{p,p} \succeq 0$ and $\rank(\boldsymbol{\mathcal{L}}(\mathbf{S})_{p,p}) \leq m-1$.}
%For convenience, will denote the latter by $\mathbf{C}^{1}(\mathbf{S})_p$ and thus,
% \begin{equation}
% \label{eq:C1Sp}
%     \mathbf{C}^{1}(\mathbf{S})_p = -\mathcal{L}(\mathbf{S})_{p,p}.
% \end{equation}
\end{rmk}
\begin{rmk}
\label{rmk:mathbb_L_structure}
\revadd{For $\widetilde{\mathbf{S}} \in \mathbb{O}(d)^m/_{\sim}$ and $\mathbf{S} \in \pi^{-1}(\widetilde{\mathbf{S}})$}, the following hold for $\mathbb{L}(\mathbf{S})$. First, $\mathbb{L}(\mathbf{S})$ is a block matrix of size $d(d-1)/2$ where each block is of size $m$. Indexing the rows and columns of $\mathbb{L}$ by tuples of the form $(r,s)$ where $1 \leq r < s \leq d$ we have,
\begin{align}
    \mathbb{L}(\mathbf{S})_{(r_1,s_1),(r_2,s_2)} 
    %&= \textstyle\sum_{\substack{p_1,q_1\in [1,d]\\p_2,q_2\in [1,d]}}\overline{\mathbf{P}}_{(r_1,s_1),(p_1,q_1)}(\mathbf{I}_d \otimes \boldsymbol{\mathcal{L}}(\mathbf{S}))_{(p_1,q_1),(p_2,q_2)}\overline{\mathbf{P}}_{(r_2,s_2),(p_2,q_2)}^T\\
    %&= \textstyle\sum_{p,q_1,q_2\in [1,d]}\overline{\mathbf{P}}_{(r_1,s_1),(p,q_1)}\boldsymbol{\mathcal{L}}(\mathbf{S})_{q_1,q_2}\overline{\mathbf{P}}_{(r_2,s_2),(p,q_2)}^T\\
    &= \left\{\begin{matrix}
    \boldsymbol{\mathcal{L}}(\mathbf{S})_{r,r}+\boldsymbol{\mathcal{L}}(\mathbf{S})_{s,s}, & r_1=r_2=r, s_1=s_2=s\\
    0, & \{r_1,s_1\} \cap \{r_2,s_2\} = \emptyset\\
    \boldsymbol{\mathcal{L}}(\mathbf{S})_{s_1,s_2}, & r_1 = r_2, s_1 \neq s_2\\
    \boldsymbol{\mathcal{L}}(\mathbf{S})_{r_1,r_2}, & s_1 = s_2, r_1 \neq r_2\\
    -\boldsymbol{\mathcal{L}}(\mathbf{S})_{r_1,s_2}, & s_1 = r_2\\
    -\boldsymbol{\mathcal{L}}(\mathbf{S})_{s_1,r_2}, & s_2 = r_1.\\
    \end{matrix}\right.
\end{align}
\revadd{For $d=2$, $3$ and $4$, $\mathbb{L}(\mathbf{S})$ is given by $\boldsymbol{\mathcal{L}}(\mathbf{S})_{1,1} + \boldsymbol{\mathcal{L}}(\mathbf{S})_{2,2}$,
\begin{equation}
    \begin{bmatrix}
    \boldsymbol{\mathcal{L}}(\mathbf{S})_{1,1} + \boldsymbol{\mathcal{L}}(\mathbf{S})_{2,2} & \boldsymbol{\mathcal{L}}(\mathbf{S})_{2,3} & -\boldsymbol{\mathcal{L}}(\mathbf{S})_{1,3}\\
    \boldsymbol{\mathcal{L}}(\mathbf{S})_{3,2} & \boldsymbol{\mathcal{L}}(\mathbf{S})_{1,1}+\boldsymbol{\mathcal{L}}(\mathbf{S})_{3,3} & \boldsymbol{\mathcal{L}}(\mathbf{S})_{1,2}\\
    -\boldsymbol{\mathcal{L}}(\mathbf{S})_{3,1} & \boldsymbol{\mathcal{L}}(\mathbf{S})_{2,1} & \boldsymbol{\mathcal{L}}(\mathbf{S})_{2,2}+\boldsymbol{\mathcal{L}}(\mathbf{S})_{3,3}
    \end{bmatrix}
\end{equation}
and (for brevity, here $\boldsymbol{\mathcal{L}} \equiv \boldsymbol{\mathcal{L}}(\mathbf{S})$)
\begin{equation}
    \begin{bmatrix}
    \boldsymbol{\mathcal{L}}_{1,1}+\boldsymbol{\mathcal{L}}_{2,2} & \boldsymbol{\mathcal{L}}_{2,3} & \boldsymbol{\mathcal{L}}_{2,4} & -\boldsymbol{\mathcal{L}}_{1,3} & -\boldsymbol{\mathcal{L}}_{1,4} & 0\\
    \boldsymbol{\mathcal{L}}_{3,2} & \boldsymbol{\mathcal{L}}_{1,1}+\boldsymbol{\mathcal{L}}_{3,3} & \boldsymbol{\mathcal{L}}_{3,4} & \boldsymbol{\mathcal{L}}_{1,2} & 0 & -\boldsymbol{\mathcal{L}}_{1,4}\\
    \boldsymbol{\mathcal{L}}_{4,2} & \boldsymbol{\mathcal{L}}_{4,3} & \boldsymbol{\mathcal{L}}_{1,1}+\boldsymbol{\mathcal{L}}_{4,4} & 0 & \boldsymbol{\mathcal{L}}_{1,2} & \boldsymbol{\mathcal{L}}_{1,3}\\
    -\boldsymbol{\mathcal{L}}_{3,1} & \boldsymbol{\mathcal{L}}_{2,1} & 0 & \boldsymbol{\mathcal{L}}_{2,2}+\boldsymbol{\mathcal{L}}_{3,3} & \boldsymbol{\mathcal{L}}_{3,4} & -\boldsymbol{\mathcal{L}}_{2,4}\\
    -\boldsymbol{\mathcal{L}}_{4,1} & 0 & \boldsymbol{\mathcal{L}}_{2,1} & \boldsymbol{\mathcal{L}}_{4,3} & \boldsymbol{\mathcal{L}}_{2,2}+\boldsymbol{\mathcal{L}}_{4,4} & \boldsymbol{\mathcal{L}}_{2,3}\\
    0 & -\boldsymbol{\mathcal{L}}_{4,1} & \boldsymbol{\mathcal{L}}_{3,1} & -\boldsymbol{\mathcal{L}}_{4,2} & \boldsymbol{\mathcal{L}}_{3,2} & \boldsymbol{\mathcal{L}}_{3,3}+\boldsymbol{\mathcal{L}}_{4,4}
    \end{bmatrix},
\end{equation}
respectively.
% \revadd{For $d=2$, $\mathbb{L}(\mathbf{S})$ is given by $\boldsymbol{\mathcal{L}}(\mathbf{S})_{1,1} + \boldsymbol{\mathcal{L}}(\mathbf{S})_{2,2}$.
Here $\mathcal{L}(\mathbf{S})_{p,q}$ depends only on the $p$th and $q$th coordinates of the points in the local views, the structure of $\Gamma$ and the alignment $\mathbf{S}$.} Additionally, the set of vectors of the form $\boldsymbol{\omega} = [\boldsymbol{\omega}_{r,s}]_{1 \leq r < s \leq d}$ where each $\boldsymbol{\omega}_{r,s}$ is a constant vector, lie in the kernel of $\mathbb{L}(\mathbf{S})$. Therefore, the rank of $\mathbb{L}(\mathbf{S})$ is at most $(m-1)d(d-1)/2$.
%Finally, if $\mathbf{Q} \in \mathbb{O}(d)$ then $\mathbb{L}(\mathbf{S}\mathbf{Q})$ is unitarily equivalent to $\mathbb{L}(\mathbf{S})$.
\revadd{Now, if $\widetilde{\mathbf{S}} \in \widetilde{\mathcal{C}}$, then $\mathbb{L}(\mathbf{S})$ is also symmetric and in particular $\mathbb{L}(\mathbf{S})_{(r_1,s_1),(r_2,s_2)}^T = \mathbb{L}(\mathbf{S})_{(r_2,s_2),(r_1,s_1)}$}.
\end{rmk}
% \begin{rmk}
% \label{rmk:mathbb_L_structure}
% If $\widetilde{\mathbf{S}} \in \widetilde{\mathcal{C}}$ then for every $\mathbf{S} \in \pi^{-1}(\mathbf{S})$, the following hold.
% \begin{enumerate}
%     \item $\mathbb{L}(\mathbf{S})$ is a block matrix of size $d(d-1)/2$ where each block is of size $m$. Indexing the rows and columns of $\mathbb{L}$ by tuples of the form $(r,s)$ where $1 \leq r < s \leq d$ we have,
%     \begin{align}
%         \mathbb{L}(\mathbf{S})_{(r_1,s_1),(r_2,s_2)} 
%         %&= \textstyle\sum_{\substack{p_1,q_1\in [1,d]\\p_2,q_2\in [1,d]}}\overline{\mathbf{P}}_{(r_1,s_1),(p_1,q_1)}(\mathbf{I}_d \otimes \boldsymbol{\mathcal{L}}(\mathbf{S}))_{(p_1,q_1),(p_2,q_2)}\overline{\mathbf{P}}_{(r_2,s_2),(p_2,q_2)}^T\\
%         %&= \textstyle\sum_{p,q_1,q_2\in [1,d]}\overline{\mathbf{P}}_{(r_1,s_1),(p,q_1)}\boldsymbol{\mathcal{L}}(\mathbf{S})_{q_1,q_2}\overline{\mathbf{P}}_{(r_2,s_2),(p,q_2)}^T\\
%         &= \left\{\begin{matrix}
%         \boldsymbol{\mathcal{L}}(\mathbf{S})_{r,r}+\boldsymbol{\mathcal{L}}(\mathbf{S})_{s,s}, & r_1=r_2=r, s_1=s_2=s\\
%         0, & \{r_1,s_1\} \cap \{r_2,s_2\} = \emptyset\\
%         \boldsymbol{\mathcal{L}}(\mathbf{S})_{s_1,s_2}, & r_1 = r_2, s_1 \neq s_2\\
%         \boldsymbol{\mathcal{L}}(\mathbf{S})_{r_1,r_2}, & s_1 = s_2, r_1 \neq r_2\\
%         -\boldsymbol{\mathcal{L}}(\mathbf{S})_{r_1,s_2}, & s_1 = r_2\\
%         -\boldsymbol{\mathcal{L}}(\mathbf{S})_{s_1,r_2}, & s_2 = r_1.\\
%         \end{matrix}\right.
%     \end{align}
%     \item Since $\boldsymbol{\mathcal{L}}(\mathbf{S})$ is symmetric, $\mathbb{L}(\mathbf{S}) = \mathbb{L}(\mathbf{S})^T$.
%     \item The set of vectors of the form $\boldsymbol{\omega} = [\boldsymbol{\omega}_{r,s}]_{1 \leq r < s \leq d}$ where each $\boldsymbol{\omega}_{r,s}$ is a constant vector, lie in the kernel of $\mathbb{L}(\mathbf{S})$.
%     \item The rank of $\mathbb{L}(\mathbf{S})$ is at most $(m-1)d(d-1)/2$.
%     \item If $\mathbf{Q} \in \mathbb{O}(d)$ then $\mathbb{L}(\mathbf{S}\mathbf{Q})$ is unitarily equivalent to $\mathbb{L}(\mathbf{S})$.
% \end{enumerate}
% \end{rmk}
% \begin{dfn}
% Let $\mathbf{S} \in \mathcal{C}$. Then $\mathbb{L}(\mathbf{S})$ is said to have trivial certificate if the null space of $\mathbb{L}$ contains only the vectors of the form $\boldsymbol{\omega} = [\boldsymbol{\omega}_{r,s}]_{1 \leq r < s \leq d}$ where each $\boldsymbol{\omega}_{r,s}$ is a constant vector i.e. if $\mathbb{L}(\mathbf{S})$ has a rank of $(m-1)d(d-1)/2$.
% \end{dfn}
% \revadd{
% \begin{rmk}
% \label{rmk:mathbbL_examples}
% For $d=2$ and $3$, $\mathbb{L}(\mathbf{S})$ is given by $\boldsymbol{\mathcal{L}}(\mathbf{S})_{1,1} + \boldsymbol{\mathcal{L}}(\mathbf{S})_{2,2} = -(\mathbf{C}^{1}(\mathbf{S})_1 + \mathbf{C}^{1}(\mathbf{S})_2)$ and
% \begin{equation}
%     \begin{bmatrix}
%     \boldsymbol{\mathcal{L}}(\mathbf{S})_{1,1} + \boldsymbol{\mathcal{L}}(\mathbf{S})_{2,2} & \boldsymbol{\mathcal{L}}(\mathbf{S})_{2,3} & -\boldsymbol{\mathcal{L}}(\mathbf{S})_{1,3}\\
%     \boldsymbol{\mathcal{L}}(\mathbf{S})_{3,2} & \boldsymbol{\mathcal{L}}(\mathbf{S})_{1,1}+\boldsymbol{\mathcal{L}}(\mathbf{S})_{3,3} & \boldsymbol{\mathcal{L}}(\mathbf{S})_{1,2}\\
%     -\boldsymbol{\mathcal{L}}(\mathbf{S})_{3,1} & \boldsymbol{\mathcal{L}}(\mathbf{S})_{2,1} & \boldsymbol{\mathcal{L}}(\mathbf{S})_{2,2}+\boldsymbol{\mathcal{L}}(\mathbf{S})_{3,3}
%     \end{bmatrix},
% \end{equation}
% %and
% % {\small
% % \begin{equation}
% %     \begin{bmatrix}
% %     \boldsymbol{\mathcal{L}}_{1,1}+\boldsymbol{\mathcal{L}}_{2,2} & \boldsymbol{\mathcal{L}}_{2,3} & \boldsymbol{\mathcal{L}}_{2,4} & -\boldsymbol{\mathcal{L}}_{1,3} & -\boldsymbol{\mathcal{L}}_{1,4} & 0\\
% %     \boldsymbol{\mathcal{L}}_{3,2} & \boldsymbol{\mathcal{L}}_{1,1}+\boldsymbol{\mathcal{L}}_{3,3} & \boldsymbol{\mathcal{L}}_{3,4} & \boldsymbol{\mathcal{L}}_{1,2} & 0 & -\boldsymbol{\mathcal{L}}_{1,4}\\
% %     \boldsymbol{\mathcal{L}}_{4,2} & \boldsymbol{\mathcal{L}}_{4,3} & \boldsymbol{\mathcal{L}}_{1,1}+\boldsymbol{\mathcal{L}}_{4,4} & 0 & \boldsymbol{\mathcal{L}}_{1,2} & \boldsymbol{\mathcal{L}}_{1,3}\\
% %     -\boldsymbol{\mathcal{L}}_{3,1} & \boldsymbol{\mathcal{L}}_{2,1} & 0 & \boldsymbol{\mathcal{L}}_{2,2}+\boldsymbol{\mathcal{L}}_{3,3} & \boldsymbol{\mathcal{L}}_{3,4} & -\boldsymbol{\mathcal{L}}_{2,4}\\
% %     -\boldsymbol{\mathcal{L}}_{4,1} & 0 & \boldsymbol{\mathcal{L}}_{2,1} & \boldsymbol{\mathcal{L}}_{4,3} & \boldsymbol{\mathcal{L}}_{2,2}+\boldsymbol{\mathcal{L}}_{4,4} & \boldsymbol{\mathcal{L}}_{2,3}\\
% %     0 & -\boldsymbol{\mathcal{L}}_{4,1} & \boldsymbol{\mathcal{L}}_{3,1} & -\boldsymbol{\mathcal{L}}_{4,2} & \boldsymbol{\mathcal{L}}_{3,2} & \boldsymbol{\mathcal{L}}_{3,3}+\boldsymbol{\mathcal{L}}_{4,4}
% %     \end{bmatrix},
% % \end{equation}
% %}
% respectively. Here $\mathcal{L}(\mathbf{S})_{i,j}$ depends only on the $i$th and $j$th coordinates of the points in the local views, the structure of $\Gamma$ and the alignment $\mathbf{S}$.
% \end{rmk}
% }

\subsection{Non-degenerate Alignment in the General Setting}
\label{subsec:non_deg_gen_setting}
As argued in Section~\ref{sec:setup}, since $F(\mathbf{S}) = F(\mathbf{S}\mathbf{Q})$ for all $\mathbf{Q} \in \mathbb{O}(d)$, every alignment $\mathbf{S}$ is degenerate in this sense. With a slight abuse of notation, we define a non-degenerate alignment as,
\begin{dfn}
\label{def:non_deg_alignment0}
An alignment $\mathbf{S} \in \mathbb{O}(d)^m$ is non-degenerate if $\pi(\mathbf{S})$ is a non-degenerate local minimum of $\widetilde{F}$.
\end{dfn}
With the above definition, to characterize the non-degenerate alignments, it suffices to characterize the non-degenerate local minima of $\widetilde{F}$. We accomplish the same in the following theorem. Note that we have not made any assumption about the affine non-degeneracy of the points and the noise in the local views.

\begin{thm}{\textbf{(Condition for $\widetilde{\mathbf{S}}$ to be a non-degenerate local minimum of $\widetilde{F}$)}}. Let $\widetilde{\mathbf{S}} \in \widetilde{\mathcal{C}}$ and $\mathbf{S} \in \pi^{-1}(\widetilde{\mathbf{S}})$. Then the following are equivalent.
\begin{enumerate}[leftmargin=*]
    \item $\widetilde{\mathbf{S}}$ is a non-degenerate local minimum of $\widetilde{F}$.
    \item $\widetilde{g}(\Hess \widetilde{F}(\widetilde{\mathbf{S}})[\widetilde{\mathbf{Z}}],\widetilde{\mathbf{Z}}) > 0$ for all $\widetilde{\mathbf{Z}} \in T_{\widetilde{\mathbf{S}}}\mathbb{O}(d)^m/_{\sim}$ such that $\widetilde{\mathbf{Z}} \neq 0$.
    \item $\Tr(\boldsymbol{\Omega}^T\mathbf{L}(\mathbf{S})\boldsymbol{\Omega}) > 0$ for all $\boldsymbol{\Omega} = [\boldsymbol{\Omega}_i]_1^m$ where $\boldsymbol{\Omega}_i \in \Skew(d)$, $\textstyle\sum_1^m \boldsymbol{\Omega}_i = 0$ and not all $\boldsymbol{\Omega}_i$ equal zero.
    \item $\Tr(\boldsymbol{\Omega}^T\mathbf{L}(\mathbf{S})\boldsymbol{\Omega}) > 0$ for all $\boldsymbol{\Omega} = [\boldsymbol{\Omega}_i]_1^m$ where $\boldsymbol{\Omega}_i \in \Skew(d)$ and not all $\boldsymbol{\Omega}_i$ are equal.
    \item $\boldsymbol{\omega}^T\mathbb{L}(\mathbf{S})\boldsymbol{\omega} > 0$ for all $\boldsymbol{\omega} = [\boldsymbol{\omega}_{r,s}]_{1 \leq r < s \leq d}$ where not all $\boldsymbol{\omega}_{r,s}$ are constant vectors.
    \item $\mathbb{L}(\mathbf{S})$ is positive semi-definite and of rank $(m-1)d(d-1)/2$.
    \item \revadd{$\lambda_{d(d-1)/2+1}(\mathbb{L}(\mathbf{S})) > 0$.}
\end{enumerate}
\label{thm:non_deg_loc_min}
\end{thm}
\begin{rmk}
\label{rmk:non_deg_loc_min}
Given the patch framework $\Theta$ and the alignment $\mathbf{S} \in \mathcal{C}$, one can compute the matrix $\mathbb{L}(\mathbf{S})$ in polynomial time in $m$, $n$ and $d$, and then check the non-degeneracy of the alignment $\mathbf{S}$ by testing the last condition in the above theorem (which again requires polynomial time in $m$ and $d$).
\end{rmk}

Although, $\widetilde{\mathbf{S}}$ is a non-degenerate local minimum of $\widetilde{F}$ if any of the equivalent conditions in the above theorem hold for every $\mathbf{S} \in \pi^{-1}(\widetilde{\mathbf{S}})$, the following result shows that if a conditions hold for one $\mathbf{S} \in \pi^{-1}(\widetilde{\mathbf{S}})$ then it holds for all other elements as well i.e. for all $\mathbf{S}\mathbf{Q}$ where $\mathbf{Q} \in \mathbb{O}(d)$ is arbitrary.
\begin{prop}
\label{prop:one_all1}
Let $\mathbf{S} \in \pi^{-1}(\widetilde{\mathbf{S}})$ and $\mathbf{Q} \in \mathbb{O}(d)$. Suppose \revadd{a} condition in Theorem~\ref{thm:non_deg_loc_min} holds for $\mathbf{S}$ then it holds for $\mathbf{S}\mathbf{Q}$ also. Consequently, an alignment $\mathbf{S}$ is non-degenerate if $\mathbf{S} \in \mathcal{C}$ (see Eq.~(\ref{eq:crit_pts2})) and it satisfies any of the (equivalent) conditions 3-7 in Theorem~\ref{thm:non_deg_loc_min}.
\end{prop}

\revadd{Following Remark~\ref{rmk:mathbb_L_structure} and Remark~\ref{rmk:mathcalLpp}, in the case of $d=2$ and an arbitrary number of views, the non-degeneracy of an alignment can be checked by investigating the the second smallest eigenvalues of certain laplacian-like matrices.
\begin{cor}
\label{cor:non_deg_d_2}
If $d=2$ then $\mathbf{S}$ is a non-degenerate alignment iff $\mathbf{\mathcal{L}}(\mathbf{S})_{1,1}+\mathbf{\mathcal{L}}(\mathbf{S})_{2,2}$ has rank of $m-1$ or equivalently, $\lambda_2(\mathbf{\mathcal{L}}(\mathbf{S})_{1,1} + \mathbf{\mathcal{L}}(\mathbf{S})_{2,2}) > 0$. In particular, $\mathbf{S}$ is a non-degenerate alignment if either $\mathbf{\mathcal{L}}(\mathbf{S})_{1,1}$ or $\mathbf{\mathcal{L}}(\mathbf{S})_{2,2}$ has a rank of $m-1$, equivalently $\max \{\lambda_2(\mathbf{\mathcal{L}}(\mathbf{S})_{1,1}), \lambda_2(\mathbf{\mathcal{L}}(\mathbf{S})_{2,2})\} > 0$.
\end{cor}
}
\begin{figure}[H]
    \centering
     \includegraphics[width=0.15\textwidth,keepaspectratio]{fig/fig0/counterex_suff_loc_rigid.png}
    \caption{The dotted lines represent views and the filled points represent points on the overlaps. Here $d=2$ and all the pair of views are perfectly aligned (same is the case for the rest of the figures). It will be clear from Proposition~\ref{prop:noiseless_setting1} in Section~\ref{sec:noiseless_non_deg_results} that $\mathbf{L}(\mathbf{S}) \succeq 0$, and thus $\mathbb{L}(\mathbf{S}) \succeq 0$. Through simple calculations one can deduce that the rank of $\mathbb{L}(\mathbf{S})$ is $3$ (which equals $(m-1)d(d-1)/2$) while the rank of $\mathbf{L}(\mathbf{S})$ is $ 3 < 6 = (m-1)d$.}
    \label{fig:suff_cond_views_non_deg}
\end{figure}
A sufficient condition for a non-degenerate alignment in arbitrary dimensions is as follows. It is not a necessary condition though as demonstrated in Figure~\ref{fig:suff_cond_views_non_deg}.
\begin{cor}
\label{cor:suff_non_deg_loc_min}
If $\mathbf{L}(\mathbf{S}) \succeq 0$ and of rank $(m-1)d$, then $\mathbf{S}$ is a non-degenerate alignment. The same holds for $\boldsymbol{\mathcal{L}}(\mathbf{S})$ as it unitarily equivalent to $\mathbf{L}(\mathbf{S})$.
\end{cor}

\revadd{In the following we characterize the neighborhood of a non-degenerate alignment in which the Hessian remains non-singular and positive definite. This will be useful in characterizing the radius of convergence of Riemannian gradient descent in Section~\ref{sec:convergence}.}
\begin{prop}
\label{prop:HessVicinity}
\revadd{Let $\mathbf{S}$ be a non-degenerate alignment. For brevity, define $c_1 \coloneqq \max_{1}^{m}\sigma_{\max}(\mathbf{C}_{k,:})$, $c_2(\mathbf{S}) \coloneqq \max_{1}^{m}\sigma_{\max}([\mathbf{C}\mathbf{S}]_i)$, $c_3(\mathbf{S}) \coloneqq \sigma_{\max}(\mathbf{L}(\mathbf{S}))$, $\lambda_{-}(\mathbf{S}) \coloneqq \lambda_{d(d-1)/2+1}(\mathbb{L}(\mathbf{S})) > 0$ (follows from Theorem~\ref{thm:non_deg_loc_min}) and $\lambda_{+}(\mathbf{S}) \coloneqq \lambda_{md(d-1)/2}(\mathbb{L}(\mathbf{S})) > 0$. Note that the above functions are invariant under the action of $\mathbb{O}(d)$ i.e. they have the same value at $\mathbf{S}\mathbf{Q}$ for all $\mathbf{Q} \in \mathbb{O}(d)$. Let $\zeta \in (0,1)$ be fixed and define}
\begin{equation}
\label{eq:delta}
\revadd{\delta(\mathbf{S}) \coloneqq \lambda_{-}(\mathbf{S})/ 2(c_1 + c_2(\mathbf{S}) + 2 c_3(\mathbf{S})).}
\end{equation}
\revadd{If $\mathbf{O} \in \mathbb{O}(d)^m$ satisfies $\min_{\mathbf{Q}\in\mathbb{O}(d)}\left\|\mathbf{O}-\mathbf{S}\mathbf{Q}\right\|_F \leq \zeta\delta(\mathbf{S})$, then for all $\widetilde{\mathbf{Z}} \in T_{\widetilde{\mathbf{O}}}\mathbb{O}(d)^m/_{\sim}$,}
\begin{equation}
    \revadd{(1-\zeta)\lambda_{-}(\mathbf{S})\widetilde{g}(\widetilde{\mathbf{Z}}, \widetilde{\mathbf{Z}}) \leq \widetilde{g}(\Hess \widetilde{F}(\widetilde{\mathbf{O}})[\widetilde{\mathbf{Z}}],\widetilde{\mathbf{Z}}) \leq (\lambda_{+}(\mathbf{S})+\zeta \lambda_{-}(\mathbf{S})) \widetilde{g}(\widetilde{\mathbf{Z}}, \widetilde{\mathbf{Z}}).}
\end{equation}
\end{prop}
We end this subsection by deriving a necessary and sufficient condition for an alignment of two views to be non-degenerate. First we need the following definitions (note that the objects in these definitions are related but not identical to $\mathbf{B}_i$ (see Eq.~(\ref{eq:B}), Remark~\ref{rmk:L0DB}) and $\mathbf{B}(\mathbf{S})_i$ (see Eq.~(\ref{eq:BofS}), Remark~\ref{rmk:C_S_structure})),
\begin{dfn}
\label{def:Bij}
Let $i,j \in [1,m]$ be indices of two views. Define $\mathbf{B}_{i,j}$ to be a matrix whose columns are $\mathbf{x}_{k,i}$ (in the increasing order of $k$) where $(k,i),(k,j) \in E(\Gamma)$. Generally, $\mathbf{B}_{i,j} \neq \mathbf{B}_{j,i}$. Also, define $\overline{\mathbf{B}}_{i,j} = \mathbf{B}_{i,j}\left(\mathbf{I}_{n'} - (1/n')\mathbf{1}_{n'}\mathbf{1}_{n'}^T\right)$ where $n' = |\{k: (k,i),(k,j) \in E(\Gamma)\}|$ is the number of points on the overlap of the $i$th view and the $j$th view, equivalently the number of columns in $\mathbf{B}_{i,j}$.
\end{dfn}
\begin{dfn}
\label{def:BSicapj}
In a similar manner as above, let $i,j \in [1,m]$ be indices of two views and let $\mathbf{S}$ be an alignment. Define $\mathbf{B}(\mathbf{S})_{i,j}$ to be a matrix whose columns are $\mathbf{S}_i^T\mathbf{x}_{k,i}+\mathbf{t}_i$ (in increasing order of $k$) where $(k,i),(k,j) \in E(\Gamma)$ and where $\mathbf{t}_i$ is obtained using Eq.~(\ref{eq:opt_Z}). Also, define $\overline{\mathbf{B}(\mathbf{S})}_{i,j} = \mathbf{B}(\mathbf{S})_{i,j}\left(\mathbf{I}_{n'} - (1/n')\mathbf{1}_{n'}\mathbf{1}_{n'}^T\right)$.
\end{dfn}
\begin{rmk}
\label{rmk:BS_ijB_ij}
Let $i,j \in [1,m]$ be indices of two views and $\mathbf{S}$ be an alignment. Let  $n' = |\{k:(k,i),(k,j) \in E(\Gamma)\}|$ be the number of points on the overlap of the two views. Then $\mathbf{B}(\mathbf{S})_{i,j} = \mathbf{S}_i^T \mathbf{B}_{i,j} + \mathbf{t}_i\mathbf{1}_{n'}^T$ where $\mathbf{t}_i$ is obtained using Eq.~(\ref{eq:opt_Z}). Thus, we have $\rank (\overline{\mathbf{B}}_{i,j}) = \rank (\overline{\mathbf{B}(\mathbf{S})}_{i,j})$.
Moreover, $\mathbf{B}(\mathbf{S})_{i,j}\left(\mathbf{I}_{n'} - (1/n'){\mathbf{1}_{n'}\mathbf{1}_{n'}^T}\right)\mathbf{B}(\mathbf{S})_{j,i}^T$ equals $\mathbf{S}_i^T\mathbf{B}_{i,j}\left(\mathbf{I}_{n'} - (1/n'){\mathbf{1}_{n'}\mathbf{1}_{n'}^T}\right)\mathbf{B}_{j,i}^T\mathbf{S}_j$ which in turn equals $\mathbf{S}_1^T\overline{\mathbf{B}}_{i,j}\overline{\mathbf{B}}_{j,i}^T\mathbf{S}_2$.
Consequently, the $\rank (\overline{\mathbf{B}(\mathbf{S})}_{i,j}\overline{\mathbf{B}(\mathbf{S})}_{j,i}^T) = \rank (\overline{\mathbf{B}}_{i,j}\overline{\mathbf{B}}_{j,i}^T)$.
\end{rmk}
\begin{figure}[H]
    \centering
    \begin{tabular}{ccc}
    \begin{subfigure}[b]{0.175\textwidth}
         \centering
         \includegraphics[width=0.9\textwidth,keepaspectratio]{fig/fig0/nec_suff_loc_rigid_1.png}
         \caption{}
         \label{fig:nec_suff_cond_loc_rigid_two_views_1}
     \end{subfigure}
     &
     \begin{subfigure}[b]{0.175\textwidth}
         \centering
         \includegraphics[width=0.9\textwidth,keepaspectratio]{fig/fig0/nec_suff_loc_rigid_2.png}
         \caption{}
         \label{fig:nec_suff_cond_loc_rigid_two_views}
     \end{subfigure}
     &
     \begin{subfigure}[b]{0.175\textwidth}
         \centering
         \includegraphics[width=0.9\textwidth,keepaspectratio]{fig/fig0/nec_suff_glob_rigid.png}
         \caption{}
         \label{fig:nec_suff_cond_glob_rigid_two_views}
     \end{subfigure}
     \end{tabular}
    \caption{(a) $\rank (\overline{\mathbf{B}}_{1,2}\overline{\mathbf{B}}_{2,1}^T) = 0$ and the two views can be rotated by a different amount while still being perfectly aligned. (b) $\rank (\overline{\mathbf{B}}_{1,2}\overline{\mathbf{B}}_{2,1}^T) = 1$ and in order for the views to be perfectly aligned, every infinitesimal rotation of the two views must be identical. However the perfect alignment of the views is not unique because the second view can be flipped (a non-infinitesimal rotation) to obtain another perfect alignment of the views. (c) $\rank (\overline{\mathbf{B}}_{1,2}\overline{\mathbf{B}}_{2,1}^T) = 2$ and the perfect alignment is unique.}
    \label{fig:geom_intuit}
\end{figure}
\begin{thm}
\label{thm:non_deg_two_views_gen_setting}
Consider $m=2$ and let $\mathbf{S} \in \mathbb{O}(d)^2$. Then $\mathbf{S}$ is a non-degenerate alignment iff all of the following hold: (see Figures~\ref{fig:nec_suff_cond_loc_rigid_two_views_1} and \ref{fig:nec_suff_cond_loc_rigid_two_views} for intuition when $d=2$)
\begin{enumerate}[leftmargin=*]
    \itemsep0em 
    \item $\overline{\mathbf{B}(\mathbf{S})}_{1,2}\overline{\mathbf{B}(\mathbf{S})}_{2,1}^T$ is symmetric.
    \item $\mathrm{Tr}(\boldsymbol{\Omega}^T \overline{\mathbf{B}(\mathbf{S})}_{1,2}\overline{\mathbf{B}(\mathbf{S})}_{2,1}^T\boldsymbol{\Omega}) \geq 0$ for all $\boldsymbol{\Omega} \in \Skew(d)$.
    \item $\rank\left(\overline{\mathbf{B}(\mathbf{S})}_{1,2}\overline{\mathbf{B}(\mathbf{S})}_{2,1}^T\right) \geq d-1$ (equivalently $\rank (\overline{\mathbf{B}}_{1,2}\overline{\mathbf{B}}_{2,1}^T) \geq d-1$).
\end{enumerate}
\end{thm}

\subsection{Unique Optimal Alignment in the General Setting}
\label{subsec:uniq_gen_setting}
Since $F(\mathbf{S}) = F(\mathbf{S}\mathbf{Q})$ for all $\mathbf{Q} \in \mathbb{O}(d)$, if $\mathbf{S}$ is an optimal alignment (a global minimum) then so is $\mathbf{S}\mathbf{Q}$. In this sense, no optimal alignment is unique. With a slight abuse of convention we define a unique optimal alignment below.
% \begin{dfn}
% \label{def:uniq_alignment}
% An alignment $\mathbf{S} \in \mathbb{O}(d)^m$ is a unique optimal alignment if $\pi(\mathbf{S})$ is the unique global minimum of $\widetilde{F}$ or equivalently, if $\mathbf{S}$ is an optimal alignment that is unique up to the action of $\mathbb{O}(d)$: if $\mathbf{S}'$ is optimal then $\mathbf{S}' = \mathbf{S}\mathbf{Q}$ for some $\mathbf{Q} \in \mathbb{O}(d)$.
% \end{dfn}
\begin{dfn}
\label{def:uniq_alignment}
An alignment $\mathbf{S} \in \mathbb{O}(d)^m$ is a unique optimal alignment if $\pi(\mathbf{S})$ is the unique global minimum of $\widetilde{F}$ i.e. if $\mathbf{O} \in \mathbb{O}(d)^m$ is also an optimal alignment then $\pi(\mathbf{O}) = \pi(\mathbf{S})$, equivalently, $\mathbf{O} = \mathbf{S}\mathbf{Q}$ for some $\mathbf{Q} \in \mathbb{O}(d)$.
\end{dfn}


%Now we provide a necessary and sufficient condition for an optimal alignment of two views to be unique. Although the proof can be found in \citea{schonemann1966generalized}, for completeness, we provide a proof using the constructs derived so far.
%The following result is trivial (see \citea{schonemann1966generalized} for the proof)
\begin{thm}
\label{thm:uniq_two_views_gen_setting}
Let $m=2$ and $\mathbf{S}$ be an optimal alignment. Then $\mathbf{S}$ is unique iff $\rank (\overline{\mathbf{B}}_{1,2}\overline{\mathbf{B}}_{2,1}^T) = d$ (see Figures~\ref{fig:nec_suff_cond_loc_rigid_two_views} and \ref{fig:nec_suff_cond_glob_rigid_two_views} for $d=2$, and \citea{schonemann1966generalized} for the proof).
\end{thm}