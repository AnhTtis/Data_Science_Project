\documentclass[conference]{IEEEtran}
\IEEEoverridecommandlockouts
% The preceding line is only needed to identify funding in the first footnote. If that is unneeded, please comment it out.
\usepackage{cite}
\usepackage{amsmath,amssymb,amsfonts}
\usepackage{algorithmic}
\usepackage{graphicx}
\usepackage{textcomp}
\usepackage{xcolor}
\def\BibTeX{{\rm B\kern-.05em{\sc i\kern-.025em b}\kern-.08em
    T\kern-.1667em\lower.7ex\hbox{E}\kern-.125emX}}

\usepackage{pifont}
\usepackage{amssymb}
\usepackage[numbers]{natbib}
\usepackage[hyphens]{url}
\usepackage{color, colortbl}
\usepackage{caption}
\usepackage{subcaption}
\usepackage{lipsum}
\usepackage{booktabs}
\usepackage{multirow}
\usepackage{makecell}

\definecolor{Gray}{gray}{0.9}
\newcommand{\cmark}{\ding{51}}
\newcommand{\xmark}{\ding{55}}
\newcommand{\val}[1]{\begin{tabular}[c]{@{}l@{}} #1 \\~\\~ \end{tabular}}
\newcommand{\fullcite}[1]{\citeauthor{#1}~\citep{#1}}
\newcommand{\tildex}[1]{$_{\widetilde{~}}$#1}
\newcommand{\GHicon}[1]{\raisebox{-.15\height}{\,\includegraphics[scale=0.02]{img/github-mark-logo.png}}~\texttt{#1}}
\newcommand{\so}{\textsc{Stack Overflow}}
\newcommand{\gh}{\textsc{GitHub}}
\newcommand{\se}{Software Engineering}

\newcommand\spaceBelowAuthors{-0.7cm}

\begin{document}

\title{A Lot of Talk and a Badge: An Empirical Analysis\\of Personal Achievements in GitHub}
\author{\IEEEauthorblockN{Fabio Calefato}
\IEEEauthorblockA{%\textit{dept. name of organization (of Aff.)} \\
\textit{University of Bari}\\
Bari, Italy \\
0000-0003-2654-1588}
\and
\IEEEauthorblockN{Luigi Quaranta}
\IEEEauthorblockA{%\textit{dept. name of organization (of Aff.)} \\
\textit{University of Bari}\\
Bari, Italy \\
0000-0002-9221-0739}
\and
\IEEEauthorblockN{Filippo Lanubile}
\IEEEauthorblockA{%\textit{dept. name of organization (of Aff.)} \\
\textit{University of Bari}\\
Bari, Italy \\
0000-0003-3373-7589}
}


\maketitle


\begin{abstract}
\gh\ has introduced gamification via personal achievements, whereby badges are unlocked and displayed on developers' personal profile pages in recognition of their development activities.
In this paper, we present a mixed-methods empirical investigation to study the diffusion of personal badges in \gh\ in addition to the effects of and the reactions to their introduction. 
First, we conducted an observational study by mining longitudinal data for over 6,000 developers and performed correlation as well as regression analysis.
Then, we analyzed 33 answers to a survey and 312 \gh\ community discussions about the topic of personal badges to gauge how the community reacted to the introduction of the new feature.
We found that most of the sampled developers own at least a badge, but also observed an increasing number of users who choose to keep their profile private and opt out from displaying badges. 
Besides, badges are in general poorly correlated with developers' qualities and dispositions such as timeliness and desire to collaborate.
We also found that, with the exception of the \texttt{Starstruck} badge and the number of followers, their introduction to \gh\ had no effects.
Finally, the reaction of the community has been in  general mixed, as developers find them appealing in principle but without a clear purpose and hardly reflecting their abilities in the current form.

\end{abstract}

\begin{IEEEkeywords}
gamification, achievements, badges, social translucence, signaling theory, open source software
\end{IEEEkeywords}

\section{Introduction}

Gamification refers to the practice of incorporating game-design elements and mechanics into non-game contexts~\cite{hamari2007,hunter2012}.
The goal of gamification is to increase user engagement and motivation as well as drive desired behaviors by creating a sense of play and/or competition~\cite{deterding2011,robson2015}.
There have been several attempts at proposing frameworks for the gamification of \se\ activities~\cite{garcia2017,sasso2017} and methods for engineering gamified software~\cite{morschheuser2018}.
Previous research has also found evidence that gamification mechanisms are in part responsible for the success of websites like \so~\cite{vasilescu13}.
Gamification, in fact, has always been a staple of the technical Q\&A site whose members gain badges, medals, and privileges by being good community citizens (e.g., answering questions, editing poorly written posts, etc.).
Another popular source of user-generated content in the \se\ domain is \gh, which, in contrast, has always incorporated several social features (e.g., developers have their personal profiles and can follow each other, repositories can be starred) but never implemented gamification elements in its website.
However, this suddenly changed in Jun. 2022, when a blog post~\cite{brooks2022github} announced the introduction of \textit{personal achievements}, a new feature whereby \textit{badges} are unlocked in recognition of developers' activities and displayed on their personal profile pages.

In this paper, we investigated the effects of introducing gamification into the largest code hosting platform, which has earned the reputation of being the one-stop shop for software development~\cite{metz2015}.
In addition, differently from \so, neither the list of available badges is publicly disclosed by \gh\ nor are the rules to unlock them.
Specifically, we conducted a large-scale, mixed-methods study of personal achievement badges (in short \textit{personal badges} hereinafter).
We characterized the distribution of badges in \gh\ (RQ$_1$), investigated what developer qualities they are intended to signal and how well they correlate with such qualities (RQ$_2$), and looked into the  effects, if any, that the introduction of this feature has had on developers' activities  (RQ$_3$).
Finally, we gauged how the \gh\ community feels about the new feature (RQ$_4$).

After mining longitudinal data from over 6,000 \gh\ developers, we found that while most own at least one badge, the number of those who chose to have a private page or opted out from displaying badges on their profile has been steadily increasing.  
%After mining longitudinal data from over 6,000 \gh\ developers, we analyzed the distribution of badges over a six-month time span (from Jun. to Dec. 2022) and found that after an initial interest in Jul., when the number of users with at least one badge reached  85.74\%, the same number started  decreasing while the number of those who chose to have a private page or opted out from displaying badges on their profile steadily increased, reaching 17.34\% in Nov.  
Furthermore, the correlation and regression analyses revealed, respectively, that only one out of the five types of personal badges analyzed (i.e., \texttt{Starstruck}) provides a reliable signal of the associated developer quality (popularity) and that its introduction positively affected the number of followers for developers displaying the badge.
Finally, the analysis of 33 responses to a survey and 312 community discussions showed that, while some users find badges somewhat appealing and nice to display, most find them unreliable as visual cues of developers' activity or skills.
%because, unlike \so, they are too easily achieved and not properly `earned.'


The remainder of this paper is organized as follows.
In Section~\ref{sec:background}, we illustrate the theoretical background of the paper.
In Section~\ref{sec:res-quest}, we present the four research questions.
The empirical study is described in Section~\ref{sec:study}, along with the dataset and methods used to carry out the investigation.
Results are presented in Section~\ref{sec:results} and discussed in Section~\ref{sec:discussion} along with related work.
Finally, we conclude in Section~\ref{sec:conclusions}.



\section{Background}\label{sec:background}

\subsection{Social Translucence}
One of the reasons behind the success of \gh\ is that it has been designed as a `socially translucent' system~\citep{erickson2000}, which makes socially salient information (e.g., stars, followers) and activities of participants (e.g., contribution calendar) visible, thus providing the basis for inferences, planning, and coordination.
On the contrary, previous OSS forges like SourceForge were generally opaque to social information. 

According to \fullcite{erickson2000}, there are three building blocks to designing socially-translucent knowledge platforms, namely \textit{visibility}, \textit{awareness}, and \textit{accountability};
visibility and awareness make sure that users modulate their actions appropriately (e.g., seeing who made the latest changes to a file, and who is assigned to fix a bug report), and accountability ensures that  norms and rules come into play as effective mechanisms of social control (e.g., feeling the pressure of fixing a build-breaking commit).

We argue that profile pages, which show developers' salient personal information, activity traces, and now also personal badges, act as a hub for socially translucent\footnote{\citeauthor{erickson2000} use the adjective \textit{translucent} instead of \textit{transparent} to highlight the privacy concern and constant tension between showing just enough and too much information. We use the term consistently.} \textit{signals} that support visibility, awareness, and accountability in \gh.

\subsection{Signaling Theory}

When choosing with whom to collaborate, especially online, much of what we want to know about others is not directly observable. Instead, we rely on \textit{signals}, i.e., perceivable features and actions that indicate the presence of the qualities of interest~\citep{donath2008}.
For example, in collaborative development platforms like \gh\, we can rely on the overall number of code-review comments as a signal of a developer's timeliness and commitment; similarly, a large number of contributed pull requests can signal technical skills and high productivity.

\textit{Signaling theory}, initially developed in the job market~\citep{spence1978job} and biology~\citep{zahavi1975mate} fields, is useful to model the relationship between signals and qualities, and explain how only certain signals can be considered reliable.
At its core, signaling theory is fundamentally concerned with reducing information asymmetry between the \textit{signaler}---i.e., the party who possesses some pieces of information/personal attributes and benefits %(e.g., a mate or a job applicant selected over some alternatives) 
from the actions taken by the other party---and the \textit{receiver}---i.e., an outsider who lacks information and would in turn benefit from making decisions based on it.  
For example, in biology, a peacock (the signaler) shows off the size and shape of his tail to be selected in favor of alternative partners (the signaler's benefit) by a peahen (the receiver) because a larger tail is a signal of a healthy bird and a better chance for healthy offspring (the receiver's benefit)~\citep{zahavi1975mate}. 

However, not all signals are \textit{reliable}. For a signal to be considered reliable, the cost of deceptively producing it must outweigh the potential benefits.
In terms of reliability, we distinguish two classes of signals.
The \textit{assessment signals}, which are inherently reliable because the signal can be produced only if the signaler possesses the indicated quality.
For example, owning jewelry and luxury cars are reliable signals of a person being rich---a person without money simply cannot afford them. 
The other class of signals, named \textit{conventional signals}, is not inherently reliable because the link between the signal and the quality is arbitrary and established by social conventions.
These types of signals are widespread in online interaction; for example, the descriptions on online profile pages are conventional signals, which are unreliable because they are more open to deception.
Therefore, for signals to be reliable, they have to be \textit{observable}---receivers need to be able to notice them--- and they have to be \textit{costly} to produce.
In the labor market scenario, where employers lack information about job applicants, high-quality prospective employees distinguish themselves from the other low-quality applicants by highlighting in their resumé their education;
holding a degree with honors from a prestigious institution is a signal that is both easily observable for recruiters and also costly to produce and hard to fake without being both smart and hard working~\citep{spence1978job}. 

Previous research has shown that the easily observable signals conveyed through online profiles in translucent, social coding platforms (e.g., \so\ reputation points, \gh\ activity graph) can act as proxies of expertise~\citep{shami2009,marlow2013}, awareness~\citep{dabbish2012}, and commitment~\citep{tsay2013,tsay2014}.
Accordingly, here we want to investigate the extent to which personal badges in \gh\ can be considered reliable as  signals (whether assessment or conventional) of desirable yet directly unobservable qualities and personal dispositions of developers (e.g., timeliness, popularity, desire to collaborate, etc.) in the scenario of open-source software development collaboration (i.e., in short, for `people sensemaking'~\citep{shami2009}). 

\subsection{Gamification in SE}

%Gamification is the application of game elements and mechanics to non-game activities, to improve users’ engagement, motivation, and participation~\citep{deterding2011}.
The use of gamification in software engineering is common, especially in the social programmer software ecosystem~\citep{singer2013}, where several kinds of game design elements are often used~\cite{depaulaport2021}.
For example, \so\ implements a reward system that combines reputation scores and leaderboards with  badges to publicly reward users for their contributions to the Q\&A community. 
\gh\ has recently implemented badges to reward achievements unlocked by developers through their activities on the coding platform.

%The effect of gamification elements are unclear,
%ranking and leaderboards on one hand can motivate; on the other hand, they might foster unhealthy competition and even toxicity.
%this may depend on the personality of users.

\so\ provides a complete list of existing achievements, as well as the rules and even community guidelines~\citep{calefato2018} for detailing how to obtain them.  
However, an undesired side effect of knowing which achievements are there and how to unlock them is that the behavior of users may change after obtaining a badge. 
\fullcite{grant2013} observed that the number of edits to posts in \so\ drastically drops down once users obtain the related achievements.
In \gh\ neither the list of achievements nor the rules for obtaining them are publicly disclosed.
We speculate that this design choice may help to avoid the risk that making the list of achievements public steers the developers' coding behavior and, conversely, ensures that the regular coding activity on the platform is rewarded with achievement badges. 

\section{Research Questions}\label{sec:res-quest}
The overall goal of this paper is to characterize personal badges diffusion, usefulness, and perception in \gh.
Accordingly, we define and answer the following four research questions.

\textit{RQ$_1$ -- What is the distribution of personal badges unlocked in \gh?}\\ 
With the first research question, we wanted not only to uncover the distribution of unlocked personal badges by type but also to assess whether and how many developers opted out from displaying them on their public profile page.

\textit{RQ$_2$ -- Are personal badges reliable signals of developers' qualities?}\\
%The second research question builds on the signaling theory and seeks to understand whether personal badges are reliable signals of developers' characteristics, such as timeliness and popularity.
The second research question builds on the signaling theory.
Because there is no official description of personal badges, we first sought to understand what developers' qualities they are signaling  (e.g., timeliness, popularity, desire to collaborate).
Then, we assessed whether these signals are reliable.

\textit{RQ$_3$ -- Did the introduction of personal badges affect developers' activities?}\\
The rules for unlocking and the list of badges are not disclosed by \gh. 
Nonetheless, there are several, unofficial resources, such as repositories\footnote{\url{https://github.com/Schweinepriester/github-profile-achievements}} and  community discussions,\footnote{e.g., \url{https://github.com/community/community/discussions/28656}} which provide insights on the badges discovered so far and how to obtain them.
As such, we sought to understand the effects of personal badges introduction on the activities typically performed by developers in \gh.
As detailed in Section~\ref{sec:res-rq2}, we defined activities in terms of several scenarios of selection and collaboration that are typical of OSS development.

\textit{RQ$_4$ -- How does the \gh\ community feel about personal badges?}\\
Finally, with the fourth research question, we investigated the response of the \gh\ community to the introduction and the perceived usefulness of personal badges. 


\section{Empirical Study}\label{sec:study}


\subsection{Data}\label{sec:data} 
We collected a multidimensional, longitudinal dataset of badges and other activities for 6,022 \gh\ developers, using the following procedure.

We started from the opportunistic selection of 51 \gh\ organizations. 
The sampling was guided by the intention to select projects varying in terms of size (i.e., number of files and developers), development language (e.g., Python, Java, JavaScript), and application domain (e.g., npm, docker, rails, TensorFlow); then, we retrieved the list of all organizations' members (6,022).
Next, we detail the data collected and measures for each research question.


%%%%%%%%%%%%%%%%%%%%%%%%%%%%%%%%%%%%%%%%%%%%     Data RQ1    %%%%%%%%%%%%%%%%%%%%%%%%%%%%%%%%%%%%%%%%%%%%%%% 

To investigate RQ$_1$ (diffusion of personal badges), for each user we scraped the profile page to retrieve the list of personal badges unlocked and the time when the achievement was first obtained.
The personal badges were introduced by \gh\ in Jun. 2022; we repeated the scraping on a monthly basis for six months, from Jun. to Dec.
We also counted how many \gh\ developers chose not to have a public profile page or opted out from displaying personal badges from their public profile page.


%%%%%%%%%%%%%%%%%%%%%%%%%%%%%%%%%%%%%%%%%%%%     Data RQ2 + RQ3   %%%%%%%%%%%%%%%%%%%%%%%%%%%%%%%%%%%%%%%%%%%%%%% 

To investigate the reliability of badges as signals of developers’ qualities (RQ$_2$) and the potential effects of badges on developers' activities (RQ$_3$), we collected several repeated measures concerning their development activities and popularity.
Regarding the development-related activities, we used the \gh\ API to mine the number of issues and pull requests (PRs) opened and closed (merged) and worked on (as assignee) by developers in our sample, the time to close issues and merge pull requests, and the number of commits. % and repositories contributed to. 
As a measure of developers' popularity in the \gh\ community, we also collected the number of followers. % and following; 
All these metrics were collected on a monthly basis for twelve months, from Jan. to Dec. 2022.

%%%%%%%%%%%%%%%%%%%%%%%%%%%%%%%%%%%%%%%%%%%%     Data RQ4   %%%%%%%%%%%%%%%%%%%%%%%%%%%%%%%%%%%%%%%%%%%%%%%% 

To gauge the perceptions  of personal badges in \gh\ (RQ$_4$), we used Google Forms to design an online survey, consisting of 16 questions (both closed- and open-ended).
The survey questions are partially based on a similar work by~\fullcite{trockman2018} who investigated repository badges in \gh.
After collecting the basic demographics, the questions focused on why respondents choose (not) to display badges on their profile, whether they consider them as indicators of development skills, and what inferences they make about other developers who display badges on their profile pages.
The survey was advertised on social media and posted as a question to the \gh\ community discussions space.\footnote{\url{https://github.com/community/community/discussions/37346}}
%Overall, we received 33 answers. 
Respondents received no monetary compensation.
An anonymized copy of the survey is available in the replication package.\footnote{\url{https://doi.org/10.5281/zenodo.7501582}}
%After providing an overview of profile badge adoption in \gh\ and collecting through the survey the hypotheses about their effects, we now test the degree to which the presence of such badges correlates with certain expected qualities (i.e. productivity) of \gh\ developers, which we operationalize with measures such as the number of followers and commits.
Furthermore, to complement the survey and also further assess how the \gh\ community reacted to the introduction of personal badges, we mined the questions and answers posted to  the discussions space of the \GHicon{community/community} repository. 
Specifically, we retrieved all the posts listed in the \texttt{profile} category\footnote{\url{https://github.com/community/community/discussions/categories/profile}} that also contained the keywords \texttt{badge} or \texttt{achievement.}
%Overall, we mined X,xxx Q\&A threads containing XXX questions and YYY answers.


\subsection{Methods}\label{sec:methods}

%%%%%%%%%%%%%%%%%%%%%%%%%%%%%%%%%%%%%%%%%%%%     Methods RQ1    %%%%%%%%%%%%%%%%%%%%%%%%%%%%%%%%%%%%%%%%%%%%%%% 
To answer RQ$_1$ (diffusion of personal badges), we computed the number and percentages of badges unlocked by the developers in our experimental sample for every month, from Jun. to Dec. 2022, along with the number of developers who opted out from visualizing their badges on the profile page or making their profile page public.
Besides, we plotted the monthly progression of the number of unlocked badges.

%%%%%%%%%%%%%%%%%%%%%%%%%%%%%%%%%%%%%%%%%%%%     Methods RQ2  %%%%%%%%%%%%%%%%%%%%%%%%%%%%%%%%%%%%%%%%%%%% 
%With respect to RQ$_2$ (reliability of badges as signals), we looked for correlations between the presence of personal badges and differences in the quality they are signaling. 
With respect to RQ$_2$ (reliability of badges as signals), we first hypothesized and then looked for correlations between the presence of personal badges and differences in the qualities they might be signaling. 
We collected several measures, such as the number of followers, commits, and issues and PRs opened and closed.
After verifying the presence of non-normal distributions with the Shapiro-Wilt test, we used the non-parametric Wilcoxon-Mann-Whitney (WMW) test to compare distributions and  Cliff's $\delta$ to assess the effect size.



%%%%%%%%%%%%%%%%%%%%%%%%%%%%%%%%%%%%%%%%%%%%     Methods RQ3   %%%%%%%%%%%%%%%%%%%%%%%%%%%%%%%%%%%%%%%%%%%%

Personal badges may correlate with various developers' qualities, however, the analysis performed for the second research question cannot assess whether and how the introduction of personal badges has any effect on developers’ activities.
Accordingly, for RQ$_3$ we performed a difference-in-differences (DiD) regression analysis~\cite{angrist2009}, a quasi-experimental approach that uses longitudinal data from observational studies to compare the changes that occur over time in the outcome variable (e.g., the number of commits) between the treatment group (i.e., the developers who unlocked the \texttt{Galaxy Brain} badge) and the control group (i.e., the developers who did not).

%Regarding (), for each of the metrics computed (e.g., the number of PRs merged), we perform a trend analysis, that is comparing the historical values of such variables to establish and compare the trends before and after a badge (e.g., \texttt{Pull Shark}) has been obtained.


%%%%%%%%%%%%%%%%%%%%%%%%%%%%%%%%%%%%%%%%%%%%     Methods RQ4   %%%%%%%%%%%%%%%%%%%%%%%%%%%%%%%%%%%%%%%%%%%% 

Finally, to answer RQ$_4$ (perceptions of personal badges), we first quantitatively and qualitatively analyzed the answers to the survey; then, we performed a thematic analysis of the discussion posts mentioning personal badges.
%afterward, we qualitatively analyzed those entire threads (i.e., questions and answers),
Specifically, we focused on the questions expressing positive/negative feedback, reporting shortcomings, and giving suggestions for the improvement of personal badges.
%Then, we performed both automatic sentiment analysis and manual inspection of the Q\&A discussion threads collected from community discussions related to badges to gauge further the positive/negative reactions to the introduction of badges. 
%To perform the sentiment analysis task, we used the sentiment analysis module~\cite{calefato2018} of the EMTk toolkit~\cite{calefato2019}, which is specifically trained in the software engineering domain.
%statistically representative sample (95\% confidence level, 5\% margin of error) 

\section{Results}\label{sec:results}

%%%%%%%%%%%%%%%%%%%%%%%%%%%%%%%%%%%%%%%%%%%%     Results RQ1    %%%%%%%%%%%%%%%%%%%%%%%%%%%%%%%%%%%%%%%%%%%% 

\begin{table*}[h!]
%\vspace{-0.1cm}
\caption{Categories of personal achievements badges unlocked as of Dec. 2022 and their distribution in our dataset. Badges highlighted in gray are excluded from the empirical analysis.}
\label{tab:badges_tiers}
\begin{tabular}{clclr}
\multicolumn{5}{c}{\textsc{Achievements}}                                                                                                                                                                                               \\ \hline
\textbf{Badge}                     & \textbf{Title}                                                         & \textbf{Earnable?} & \textbf{Description}                                                      & \textbf{Unlocked (\%)} \\ \hline \vspace{-4mm}
\includegraphics[width=7mm, height=7mm]{img/badges/pair-extraordinaire-default.png}                         & \val{Pair Extraordinaire}                                                      & \val{\cmark}             & \val{Coauthored a merged pull request}                                       & \val{847 (4.99\%)}   \\ \vspace{-4mm}
\includegraphics[width=7mm, height=7mm]{img/badges/quickdraw-default.png}                                   & \val{Quickdraw}                                                                & \val{\cmark}             & \val{Closed an issue or a pull request within 5 min of opening}              & \val{1441 (8.48\%)}  \\ \vspace{-4mm}
\includegraphics[width=7mm, height=7mm]{img/badges/starstruck-default.png}                                  & \val{Starstruck}                                                               & \val{\cmark}             & \val{Created a repository that has 16 stars}                                 & \val{1,122 (6.60\%)} \\ \vspace{-4mm}
\includegraphics[width=7mm, height=7mm]{img/badges/galaxy-brain-default.png}                                & \val{Galaxy Brain}                                                             & \val{\cmark}             & \val{2 accepted answers}                                                     & \val{192 (1.13\%)}   \\ \vspace{-4mm}
\includegraphics[width=7mm, height=7mm]{img/badges/pull-shark-default.png}                                  & \val{Pull Shark}                                                               & \val{\cmark}             & \val{2 pull requests merged}                                                 & \val{838 (4.93\%)}   \\ 
\rowcolor{Gray}\vspace{-4mm}
\includegraphics[width=7mm, height=7mm]{img/badges/yolo-default.png}                                        & \val{YOLO}                                                                     & \val{\cmark}             & \val{Merged a pull request without code review}                              & \val{1324 (7.79\%)}  \\ 
\rowcolor{Gray}\vspace{-4mm}
\includegraphics[width=7mm, height=7mm]{img/badges/arctic-code-vault-contributor-default.png}               & \val{Arctic Code Vault Contributor}                                            & \val{\xmark}             & \val{Contributed code to repositories in the 2020 \gh\ Archive Program}      & \val{4,591 (27.02\%)}\\ 
\rowcolor{Gray}\vspace{-4mm}
\includegraphics[width=7mm, height=7mm]{img/badges/public-sponsor-default.png}                              & \val{Public Sponsor}                                                           & \val{\cmark}             & \val{Sponsoring open source work via \gh\ Sponsors}                          & \val{118 (0.69\%)}   \\
\rowcolor{Gray}
\includegraphics[width=7mm, height=7mm]{img/badges/mars-2020-contributor-default.png}                      & \val{Mars 2020 Contributor}                                                     & \val{\xmark}             & \val{Contributed code to repositories used in Mars 2020 Helicopter Mission}  & \val{282 (1.66\%)}   \\
\multicolumn{5}{c}{\textsc{Tiers}}                                                                                                                                                                                                      \\ \hline
\multicolumn{1}{l}{\textbf{Badge}} & \textbf{Title}                                                         & \textbf{Tier}      & \textbf{Description}                                                        & \textbf{Unlocked (\%)}  \\ \hline \vspace{-4mm}
\includegraphics[width=7mm, height=7mm]{img/tiers/pair-extraordinaire-bronze.png}                           & \val{Pair Extraordinaire x2}                                                   & \val{Bronze}             & \val{Coauthored 10 merged pull requests}                                     & \val{455 (2.68\%)}     \\ \vspace{-4mm}
\includegraphics[width=7mm, height=7mm]{img/tiers/pair-extraordinaire-silver.png}                           & \val{Pair Extraordinaire x3}                                                   & \val{Silver}             & \val{Coauthored 24 merged pull requests}                                     & \val{469 (2.76\%)}     \\ \vspace{-4mm}
\includegraphics[width=7mm, height=7mm]{img/tiers/pair-extraordinaire-gold.png}                             & \val{Pair Extraordinaire x4}                                                   & \val{Gold}               & \val{Coauthored 48 merged pull requests}                                     & \val{53 (0.31\%)}      \\ \vspace{-4mm}
\includegraphics[width=7mm, height=7mm]{img/tiers/starstruck-bronze.png}                                    & \val{Starstruck x2}                                                            & \val{Bronze}             & \val{Created a repository w/ 128 stars}                                      & \val{598 (3.52\%)}     \\ \vspace{-4mm}
\includegraphics[width=7mm, height=7mm]{img/tiers/starstruck-silver.png}                                    & \val{Starstruck x3}                                                            & \val{Silver}             & \val{Created a repository w/ 512 stars}                                      & \val{507 (2.98\%)}     \\ \vspace{-4mm}
\includegraphics[width=7mm, height=7mm]{img/tiers/starstruck-gold.png}                                      & \val{Starstruck x4}                                                            & \val{Gold}               & \val{Created a repository w/ 4096 stars}                                     & \val{158 (0.93\%)}     \\ \vspace{-4mm}
\includegraphics[width=7mm, height=7mm]{img/tiers/galaxy-brain-bronze.png}                                  & \val{Galaxy Brain x2}                                                          & \val{Bronze}             & \val{8 accepted answers}                                                     & \val{41 (0.24\%)}      \\ \vspace{-4mm}
\includegraphics[width=7mm, height=7mm]{img/tiers/galaxy-brain-silver.png}                                  & \val{Galaxy Brain x3}                                                          & \val{Silver}             & \val{16 accepted answers}                                                    & \val{18 (0.11\%)}      \\ \vspace{-4mm}
\includegraphics[width=7mm, height=7mm]{img/tiers/galaxy-brain-gold.png}                                    & \val{Galaxy Brain x4}                                                          & \val{Gold}               & \val{32 accepted answers}                                                    & \val{12 (0.07\%)}      \\ \vspace{-4mm}
\includegraphics[width=7mm, height=7mm]{img/tiers/pull-shark-bronze.png}                                    & \val{Pull Shark x2}                                                            & \val{Bronze}             & \val{16 pull requests merged}                                                & \val{1,382 (8.13\%)}   \\ \vspace{-4mm}
\includegraphics[width=7mm, height=7mm]{img/tiers/pull-shark-silver.png}                                    & \val{Pull Shark x3}                                                            & \val{Silver}             & \val{128 pull requests merged}                                               & \val{1,897 (11.17\%)}  \\ \vspace{-4mm}
\includegraphics[width=7mm, height=7mm]{img/tiers/pull-shark-gold.png}                                      & \val{Pull Shark x4}                                                            & \val{Gold}               & \val{1024 pull requests merged}                                              & \val{645 (3.80\%)}     \\ \hline
\end{tabular}
\vspace{-0.2cm}
\end{table*}



\subsection{RQ$_1$ -- Diffusion of Personal Badges}\label{sec:res-rq1}

Table~\ref{tab:badges_tiers} shows the distribution of badges unlocked in our dataset as of Dec. 2022.
The statistics in the table reveal that the most popular badge is \texttt{Arctic Code Vault Contributor} (4591 (27.02\%).
This result is `inflated' because this one is among the badges that already existed before the launch of the personal badges in Jun. 2022; besides, this badge is not earnable anymore as it was awarded to anyone contributing code to GitHub in 2020. 
The other previously existing badges are \texttt{Mars 2020 Contributor} (282, 1.66\%), and \texttt{Public Sponsor} (118, 0.69\%),  unlocked respectively by developers who contributed code to the repositories used in the Mars 2020 helicopter mission and by those who have provided sponsorships to projects via the  \gh\ \textsc{Sponsors} program.
Because these three badges predate the introduction of personal achievement badges in Jun. 2022, they were excluded from subsequent analyses.

Regarding the other, recently introduced types of badges, the most common (aggregated by tier) is \texttt{Pull Shark} (4,762, 28.03\%), unlocked by merging a certain number of pull requests.
The second most common type of badge in our dataset is \texttt{Starstruck} (2,385, 14.03\%), which is awarded by owning repositories that receive more and more stars from other users.
\texttt{Pair Extraordinaire} (1,824, 10.74\%) and \texttt{Galaxy Brain} (263, 1.55\%) are unlocked respectively by coauthoring merged pull requests and by answering questions in project discussions. 
Finally, \texttt{YOLO} (1324, 7.79\%) is the personal badge awarded to those who have merged a pull request without performing a code review.
Because the action that earns developers this badge is a development practice that should not be encouraged, it was also excluded from subsequent analyses.

As of Dec. 2022, most of the 6,022 developers in our dataset had at least one personal badge displayed on their profile page (4,977, 82.65\%), only 1 (0.02\%) had no personal badge, and 1,044 (17.34\%) had opted out from displaying badges or having a public profile.
Figure~\ref{fig:plotrq1} plots over a six-month time span the distributions of the number of developers with badges (also filtered, i.e., with pre-existing badges excluded) versus those who either opted out from displaying badges on their profile page or without a public profile altogether.
We observe an initial increase in the number of users displaying badges from Jun. (4,825, 80.12\%) to Jul. (5,163, 85.74\%); afterward, we notice a decreasing trend (4,452, 73.93\% in Nov.) with more and more developers choosing not to display badges, either by opting out or because they chose not to have a public profile page (1,412, 17.34\% in Nov.).

\begin{figure}[t]
    \centering
    \vspace{-0.75cm}
    \includegraphics[scale=0.58]{img/results/rq1_users_badges.png}
    \vspace{-0.7cm}
    \caption{Number of users with badges vs. those who opted out from displaying badges or without a public profile page (the filter refers to the exclusion of pre-existing badges.)}
    \vspace{-0.3cm}
    \label{fig:plotrq1}
\end{figure}

%%%%%%%%%%%%%%%%%%%%%%%%%%%%%%%%%%%%%%%%%%%%     RQ2    %%%%%%%%%%%%%%%%%%%%%%%%%%%%%%%%%%%%%%%%%%%%%%% 

\begin{figure*}[!ht]
     \centering
     \begin{subfigure}[b]{0.20\textwidth}
         \centering
         \includegraphics[width=\textwidth]{img/results/rq2_violinplot_pair-extraordinaire_opened_issues_prs.png}
         \caption{Pair Extraordinaire\\(\# issues and PRs opened)}
         \label{fig:rq2_vp_pairextra_opened_issues_prs}
     \end{subfigure}
    \begin{subfigure}[b]{0.20\textwidth}
         \centering
         \includegraphics[width=\textwidth]{img/results/rq2_violinplot_pair-extraordinaire_authored_commits.png}
         \caption{Pair Extraordinaire\\(\# authored commits)}
         \label{fig:rq2_vp_pairextra_authored_commits}
     \end{subfigure}
     \begin{subfigure}[b]{0.20\textwidth}
         \centering
         \includegraphics[width=\textwidth]{img/results/rq2_violinplot_quickdraw.png}
         \caption{Quickdraw\\(time to close issues \& PRs)}
         \label{fig:rq2_vp_quickdraw}
     \end{subfigure}
     \begin{subfigure}[b]{0.20\textwidth}
         \centering
         \includegraphics[width=\textwidth]{img/results/rq2_violinplot_starstruck.png}
         \caption{Starstruck\\(\# followers)}
         \label{fig:rq2_vp_starstruck}
     \end{subfigure}
     \begin{subfigure}[b]{0.20\textwidth}
         \centering
         \includegraphics[width=\textwidth]{img/results/rq2_violinplot_galaxy-brain_opened_issues_prs.png}
         \caption{Galaxy Brain\\(\# issues and PRs opened)}
         \label{fig:rq2_vp_galaxybrain_opened_issues_prs}
     \end{subfigure}
     %\begin{subfigure}[b]{0.20\textwidth}
     %    \centering
     %    \includegraphics[width=\textwidth]{img/results/rq2_violinplot_galaxy-brain_authored_commits.png}
     %    \caption{Galaxy Brain\\(\# authored commits)}
     %    \label{fig:rq2_vp_galaxybrain_authored_commits}
     %\end{subfigure}
     \begin{subfigure}[b]{0.20\textwidth}
         \centering
         \includegraphics[width=\textwidth]{img/results/rq2_violinplot_pull-shark_closed_issues.png}
         \caption{Pull Shark\\(\# closed issues)}
         \label{fig:rq2_vp_pullshark_closed_issues}
     \end{subfigure}
     \begin{subfigure}[b]{0.20\textwidth}
         \centering
         \includegraphics[width=\textwidth]{img/results/rq2_violinplot_pull-shark_committed_commits.png}
         \caption{Pull Shark\\(\# committed commits)}
         \label{fig:rq2_vp_pullshark_committed_commits}
     \end{subfigure}
     \caption{Distributions of response variables w/ and w/o badges. WMW $U$, $p$-value, and Cliff's $\delta$ statistics below each figure.}
     \vspace{-0.3cm}
     \label{fig:rq2_violinsplots}
\end{figure*}

\subsection{RQ$_2$ -- Reliability of Personal Badges as Signals}\label{sec:res-rq2}

To uncover what badges might be signaling, we investigated the associations of their presence with desirable qualities and dispositions of OSS developers.
Below are the results of the correlation analysis between the badges and their hypothesized qualities.
The distributions are illustrated in Figure~\ref{fig:rq2_violinsplots}, where we report the results of the WMW tests with $p$-values. 
Cliff's $\delta$ values are also reported to gauge the effect size (the magnitude is assessed using the thresholds provided in~\cite{romano2006}, i.e., $|\delta|<.147$ `\textit{negligible}', $|\delta|<.33$ `\textit{small}', $|\delta|<.474$, `\textit{medium}', otherwise `\textit{large}').

The \texttt{Pair Extraordinaire} badge is awarded to developers who coauthor merged pull requests.
We hypothesized that owning the badge might signal an increased \textit{desire to collaborate}, e.g., by contributing code, reporting issues, and opening PRs; accordingly, we tested its correlation with the number of opened issues and PRs as well as the number of authored commits.
The results of the WMW tests show the existence of a statistically significant difference between developers with and without the badge and the number of opened issues and PRs (Figure~\ref{fig:rq2_vp_pairextra_opened_issues_prs}, $p<.001$) with a negligible effect size ($\delta=.132$).
Instead, there is no statistically significant difference in the case of the number of authored commits between developers with and without the \texttt{Pair Extraordinaire} badge (Figure~\ref{fig:rq2_vp_pairextra_authored_commits}).

Regarding the \texttt{Quickdraw} badge, gained by developers who close issues and PRs within 5 minutes, we speculated that its presence could be a signal of \textit{timeliness} and, accordingly, tested its association with an overall shorter time taken to close issues and PRs as compared to those who do not own the badge.
The result (Figure~\ref{fig:rq2_vp_quickdraw})
shows a statistically significant difference in the WMW test but a negligible effect size ($p<.001$, $\delta < -0.031$).

The most substantial finding is observed for the \texttt{Starstruck} badge, which was hypothesized as a signal of \textit{popularity} since it is unlocked when repositories are starred; hence, we tested the correlation of its presence on developers' profile pages with the number of their followers; the results in Figure~\ref{fig:rq2_vp_starstruck} show that there is a statistically significant difference between the distributions of developers with and without the badge in terms of the number of followers ($p<.001$) and that this difference has a large effect size ($\delta=.693$).

The \texttt{Galaxy Brain} badge is unlocked by having answers posted to project discussions accepted; as such, we hypothesized that its presence might signal an increased \textit{willingness to help} projects grow; accordingly, we tested the correlation between its presence and the number of opened issues and PRs. 
The results in Figure~\ref{fig:rq2_vp_galaxybrain_opened_issues_prs} show that there is a statistically significant difference ($p<0.001$) and a small effect size ($\delta=0.156)$ between developers with and without the badge in terms of opened issues and PRs.%, whereas there is no significant result in terms of the number of authored commits.

Finally, the \texttt{Pull Shark} badge is obtained by having PRs merged; therefore, we hypothesized that its presence could represent another signal of increased \textit{disposition to manage} projects and tested its correlation with the number of closed issues and with the number of committed commits.
The results of the WMW tests show a statistically significant difference and a negligible effect size between developers with and without the badge in terms of the number of closed issues (Figure~\ref{fig:rq2_vp_pullshark_closed_issues}, $p<0.01$, $\delta=0.084$) and the lack of statistical significance in terms of the number of committed commits (Figure~\ref{fig:rq2_vp_pullshark_committed_commits}).


%%%%%%%%%%%%%%%%%%%%%%%%%%%%%%%%%%%%%%%%%%%%     RQ3   %%%%%%%%%%%%%%%%%%%%%%%%%%%%%%%%%%%%%%%%%%%%%%% 


\subsection{RQ$_3$ -- Effects of Personal Badges on Developers’ Activities}\label{sec:res-rq3}

This section presents the results of the difference-in-differences (DiD) regression analysis. 
The fundamental assumption for applying a DiD regression is that in the absence of treatment (i.e., before the introduction of personal badges) the dependent variable trend would be the same in both the treatment group (i.e., the developers who will eventually unlock a given type of badge) and the control group (i.e., the developers without the badges).
In Figure~\ref{fig:rq3_starstruck_boxplot}, through the boxplots we visually confirm the \textit{parallel} (or \textit{common}) \textit{trends} assumption for the \texttt{Starstruck} badge; Figure~\ref{fig:rq3_quickdraw_boxplot} shows instead a violation of the premise for the \texttt{Quickdraw} badge, which is therefore excluded from this analysis.
The other badges excluded from this analysis for the same reason are \texttt{Pair Extraordinaire}, with respect to the number of authored commits, and \texttt{Pull Shark}, in terms of both the number of issues closed and committed commits.
Please, refer to the supplemental material provided in the replication package for the other figures.

A difference-in-differences regression model is built to estimate equations like (\ref{eq:dd})
 
\begin{equation}\label{eq:dd}
y_{it} = \alpha + \beta B_i + \gamma BA_t + \delta (B \times BA)_{it} + \epsilon_{it}
\end{equation}
\\
where $y_{it}$ is the dependent variable (outcome) for developer $i$ at time $t$, $B_i$ is the binary variable for developers who own a badge (the treatment group, i.e., $Badger_i=1$) or not (the control group, i.e., $Badger_i=0$), and $BA_t$ is a time dummy that switches on for observations obtained after Jun. 2022 when personal badges became available (i.e., $BadgeAvail_t=1$ when $t > June$, otherwise $BadgeAvail_t=0$). 
Finally, $\epsilon_{it}$ is the residual term.


\begin{figure}[t]
    \vspace{-0.1cm}
    \centering
    \begin{subfigure}[b]{0.4\textwidth}
        \centering
        \vspace{-0.3cm}
        \includegraphics[width=\textwidth]{img/results/rq3_starstruck_grouped_boxplot.png}
        \vspace{-0.5cm}
        \caption{Starstruck (Jan.-Dec. '22)}
        \label{fig:rq3_starstruck_boxplot}
    \end{subfigure}
    \begin{subfigure}[b]{0.4\textwidth}
        \centering
        \includegraphics[width=\textwidth]{img/results/rq3_quickdraw_grouped_boxplot.png}
        \vspace{-0.5cm}
        \caption{Quickdraw (Jan.-Dec. '22)}
        \label{fig:rq3_quickdraw_boxplot}
    \end{subfigure}
    \vspace{-0.2cm}
    \caption{Visual confirmation (a) and violation (b) of the parallel trends assumption before the introduction of personal badges.}
    \vspace{-0.3cm}
    \label{fig:rq3_boxplots_parallel}
\end{figure}

The resulting DiD model is an interaction model interpreted as follows~\cite{angrist2009}.
The estimate of the intercept $\alpha$ is interpreted as the mean of the outcome (i.e., the dependent variable $y$) for the control group (i.e., the developers without badge) in the month(s) before the introduction of the personal badges feature in \gh\ (i.e., $Badger_i=0$ and $BadgesAvail_t=0$).
The coefficient $\beta$ of $Badger$ is the expected mean change in the outcome $y$ between treatment and control groups in the pre-treatment period ($Badger_i=1$ and $BadgesAvail_t=0$); this can be viewed as the `baseline difference' in the outcome variable between the two groups before the treatment.
The coefficient $\gamma$ of $BadgesAvail$ is the expected mean difference in $y$ before and after the introduction of the personal badges among the control group; 
the main effect for $BadgesAvail$ %(i.e., the post-treatment variable) 
is the effect of the simple passage of time in the absence of the treatment.
Finally, the estimated coefficient $\delta$ of the interaction term is an estimate of the treatment effect. This is the DiD coefficient and the focus of this analysis because the interaction is testing whether the expected mean change in the outcome $y$ before and after the introduction of the new feature is different for the treatment and control groups, respectively, the developers with ($Badger_i=1$) and without ($Badger_i=0$) the badge of interest.

Table~\ref{tab:dd-regressions} reports the results of the DiD regressions.
They show that the interaction term is significant only for the \texttt{Starstruck} badge, on which we focus the presentation of findings.
The DiD  coefficient $\delta$ is significant and different from zero (0.169); this means that the introduction of the new feature of \texttt{Starstruck} badge, which counts the number of starred repositories of a developer, caused an increase in the average number of their followers; specifically, because the regression is in logs, the average count of followers in the post-treatment time window is 18.4\% (i.e., $(e^{coef} -1 )\times 100$) higher than it would have been without the introduction of the \texttt{Starstruck} badge in \gh~\cite{cameron2005}.
In addition, %the intercept coefficient $\alpha$ (the constant) is significant and different from zero; this indicates that developers in the control group have an average number of followers of 3.618 in the pre-treatment time window.
we observe that the coefficient $\beta$ (treatment group) is significant and different from zero, which means that in the pre-treatment time window the developers in the treatment and control groups had a different number of followers; specifically, developers with the badge had on average 299.9\% more followers than those without. % and the average number of followers in the treatment group was 5.004 ($\alpha+\beta$).
Finally, the coefficient $\gamma$ is also significant and different from zero (-0.1), meaning that the average number of followers in the control group from the pre-treatment to the post-treatment time window decreased by 9.5\%. % to 3.518 ($\alpha+\gamma$).
%Finally, based on these results, we can calculate the counterfactual ($\alpha + \beta + \gamma = 4.904$), which represents the average number in logs of followers in the treatment group, had the introduction of the \texttt{Starstruck} badge never happened; however, because the treatment did occur, and the actual average number of followers in logs is equal to 5.073 (the counterfactual  $+ \delta$).

\begin{table}[t]
\centering
\vspace{-0.1cm}
\caption{Results of the difference-in-difference regression for each badge. Significant results are highlighted in bold.}
\label{tab:dd-regressions}
\begin{tabular}{llcc}
\hline
\multicolumn{4}{l}{\textbf{Pair Extraordinaire, log(y=no. issues and PRs opened)}}                          \\ \hline
                             & \multicolumn{1}{c}{\textbf{Coef (S.E.)}} & \textbf{[0.25}  & \textbf{0.75]}  \\ \cline{2-4} 
Intercept ($\alpha$)         & \textbf{0.295 (0.142)*}                  & \textbf{0.015}  & \textbf{0.574}  \\
Badger ($\beta$)             & 0.054 (0.198)                            & -0.335          & 0.443           \\
BadgeAvail ($\gamma$)        & \textbf{0.913 (0.169)***}                & \textbf{0.581}  & \textbf{1.245}  \\
Badger:BadgeAvail ($\delta$) & 0.374 (0.229)                            & -0.077          & 0.825           \\ \cline{2-4} 
                             & \multicolumn{3}{l}{Adj. R$^2$=0.212}                                         \\ \hline
%\multicolumn{4}{l}{\textbf{Quickdraw, log(y=time to close issues and PRs)}}                                 \\ \hline
%                             & \multicolumn{1}{c}{\textbf{Coef (S.E.)}} & \textbf{[0.25}  & \textbf{0.75]}  \\ \cline{2-4} 
%Intercept ($\alpha$)         & \textbf{10.158 (0.014)***}               & \textbf{10.131} & \textbf{10.186} \\
%Badger ($\beta$)             & \textbf{0.090 (0.020)***}                & \textbf{0.051}  & \textbf{0.130}  \\
%BadgeAvail ($\gamma$)        & \textbf{0.079 (0.016)***}                & \textbf{0.049}  & \textbf{0.110}  \\
%Badger:BadgeAvail ($\delta$) & \textbf{-0.205 (0.022)***}               & \textbf{-0.249} & \textbf{-0.161} \\ \cline{2-4} 
 %                            & \multicolumn{3}{l}{Adj. R$^2$=0.002}                                         \\ \hline
\multicolumn{4}{l}{\textbf{Starstruck, log(y=no. followers)}}                                               \\ \hline
                             & \multicolumn{1}{c}{\textbf{Coef (S.E.)}} & \textbf{[0.25}  & \textbf{0.75]}  \\ \cline{2-4} 
Intercept ($\alpha$)         & \textbf{3.618 (0.011)***}                & \textbf{3.596}  & \textbf{3.639}  \\
Badger ($\beta$)             & \textbf{1.386 (0.016)***}                & \textbf{1.354}  & \textbf{1.418}  \\
BadgeAvail ($\gamma$)        & \textbf{-0.100 (0.016)***}               & \textbf{-0.131} & \textbf{-0.070} \\
Badger:BadgeAvail ($\delta$) & \textbf{0.169 (0.023)***}                & \textbf{0.124}  & \textbf{0.213}  \\ \cline{2-4} 
                             & \multicolumn{3}{l}{Adj. R$^2$=0.228}                                         \\ \hline
\multicolumn{4}{l}{\textbf{Galaxy Brain, log(y=no. issues and PRs opened)}}                                               \\ \hline
                             & \multicolumn{1}{c}{\textbf{Coef (S.E.)}} & \textbf{[0.25}  & \textbf{0.75]}  \\ \cline{2-4} 
Intercept ($\alpha$)         & \textbf{0.322 (0.104)**}                 & \textbf{0.117}  & \textbf{0.527}  \\
Badger ($\beta$)             & 0.007 (0.399)                            & -0.778          & 0.792           \\
BadgeAvail ($\gamma$)        & \textbf{1.166 (0.121)***}                & \textbf{0.929}  & \textbf{1.404}  \\
Badger:BadgeAvail ($\delta$) & -0.135 (0.436)                           & -0.991          & 0.721           \\ \cline{2-4} 
                             & \multicolumn{3}{l}{Adj. R$^2$=0.187}                                         \\ \hline
%\multicolumn{4}{l}{\textbf{Pull Shark, log(y=no. issues closed)}}                                   \\ \hline
%                             & \multicolumn{1}{c}{\textbf{Coef (S.E.)}} & \textbf{[0.25}  & \textbf{0.75]}  \\ \cline{2-4} 
%Intercept ($\alpha$)         & 0.641 (0.336)                            & -0.019          & 1.300           \\
%Badger ($\beta$)             & 0.129 (0.367)                            & -0.592          & 0.850           \\
%BadgeAvail ($\gamma$)        & \textbf{2.122 (0.370)***}                & \textbf{1.395}  & \textbf{2.850}  \\
%Badger:BadgeAvail ($\delta$) & -0.504 (0.407)                           & -1.303          & 0.296           \\ \cline{2-4} 
%                             & \multicolumn{3}{l}{Adj. R$^2$=0.193}                                         \\ \hline
\multicolumn{4}{l}{***$p<0.001$, **$p<0.01$, *$p<0.5$}                                                     
\end{tabular}
\vspace{-0.4cm}
\end{table}

%%%%%%%%%%%%%%%%%%%%%%%%%%%%%%%%%%%%%%%%%%%%     RQ4   %%%%%%%%%%%%%%%%%%%%%%%%%%%%%%%%%%%%%%%%%%%%%%%% 


\subsection{RQ$_4$ -- Community Perception of Personal Badges}\label{sec:res-rq4}

\subsubsection{Survey analysis}
We received 33 answers. 
Regarding the demographics (Q$_{1}$-Q$_{4}$), participants in the survey reported having on average 9 years of experience (min. 1, max. 31). 
The participants also maintain an average of 5 open source software repositories (min. 0, max. 16) and have contributed to \tildex{20} repositories (min. 1, max. \tildex{100}).
Finally, they reported having an average of \tildex{26} followers (min. 0, max. \tildex{100}).
Three of the respondents were disqualified because they reported (Q$_{5}$) having opted out from displaying badges (1) or not having a public profile page (2). 
They motivated their choices (Q$_{16}$) by saying that they ``\textit{simply don't like the idea of badges}'' (P$_{22}$) and that \gh\ ``\textit{may be a social coding site but it's not (or shouldn't be or become) like social media}'' (P$_{33}$).
The other users who reported keeping the badge feature turned on (Q$_{6}$, n=30) mentioned, among the most cited reasons, that they mostly did it out of curiosity (9), because badges look nice (9), and because they saw them on other developers' profiles (12). 
Next, we analyze the remaining valid responses.

Question Q$_{7}$ (n=30, see Figure~\ref{fig:survey_q7-10}) aimed to assess whether the participants consider the presence of personal badges an indicator of their own coding skills in general.
Most of the respondents (12) strongly disagreed.
Then, questions Q$_8$-Q$_{10}$ (n=30) asked the participants to report on the effects that badges may have on collaboration with other developers.
Results are also reported in Figure~\ref{fig:survey_q7-10}.
Albeit the distribution of answers is varied, overall the figure shows that about half of the respondents feel that neither badges can be an indicator of others' ability nor they are influenced by their presence during collaboration with others.
Question Q$_{11}$ asked the respondents to motivate their previous answers. 
Participants were critical of the current implementation of  badges. 
Participant P$_7$ found that ``\textit{these badges can be too easily achieved, not like in Stack Overflow},'' and P$_{2}$ that there are too few of them (``\textit{we all have almost the same badges}''); others added that they fail to properly capture developers' experience -- e.g., ``\textit{I have 30+ years of development experience and almost none is visible through [badges]}'' (P$_{1}$), ``\textit{I have lots of experience and almost no badges}'' (P$_{8}$).
Also, P$_5$ answered with a provocative question: ``\textit{Are developers with these badges better developers? Vice versa and more critical: is a developer bad just because they don't have some of these badges?}''

Next, we delve into understanding what respondents feel about specific badges.
Question Q$_{13}$ (n=30) asked participants to indicate the most relevant badges.
The answers clustered around two main groups.
The first group contains answers (9) from users who feel that none of these badges can tell anything useful about others.
The other group of remaining answers (21) thinks instead that badges do give information about who unlocks them.
Specifically, 5 respondents mentioned the \textsc{Galaxy Brain} badge (awarded to those who have accepted answers in project discussions) because it reflects ``\textit{how serious [one is] in the community''} (P$_{28}$).
A group of 8 participants mentioned the \textsc{Starstruck} badge (unlocked by receiving project stars) because it gives tangible ``\textit{evidence of acknowledgment and interest by the community}'' (P$_{17}$).
The remaining respondents (8) highlighted the importance of \textsc{Pair Extraordinaire} (awarded to those coauthoring merged pull requests) because the badge may reflect ``\textit{a positive attitude [toward] collaboration with others''} (P$_{19}$).
Question Q$_{14}$ asked the participants to indicate the perceived effects of each badge on collaboration.
We received few answers (n=5).
This lack of responses arguably suggests that, in general, the \gh\  community does not have a clear understanding of what signals, if any, showing badges can have on developers; as P$_{30}$ wrote: ``\textit{they're supposed to show me some kind of evidence but I don't know}.'' 

Finally, question Q$_{15}$ (n=12) asked participants to indicate any missing badge, besides those available.
The suggestions converged on the following ideas for new badges that would display: i) the years of work in a project, ii) how many repositories you are a maintainer of, iii) how many repositories you have contributed to, and iv) more badges in general, as long as they are challenging to unlock.
%Finally, the last two questions asked whether the participants noticed any differences when interacting with others after displaying any personal badge (Q$_{16}$) and any difference in developers' practices or ability depending on the presence of personal achievement badges on their profile page(Q$_{17}$). They strongly disagreed.

\begin{figure}[t]
    \centering
    \vspace{-0.3cm}
    \includegraphics[scale=0.313]{img/results/rq4_q7-10.png}
    \vspace{-1.5cm}
    \caption{The extent to which participants find personal badges to be an indicator of coding skills (Q$_{7}$) and their perceived effects on collaboration (Q$_{8}$-Q$_{10}$).}
    \vspace{-0.6cm}
    \label{fig:survey_q7-10}
\end{figure}

\subsubsection{Community discussions analysis}

Overall, we mined all the Q\&A threads related to personal badges, containing 312 questions, 937 answers, and 781 replies. 
Afterward, by conducting a thematic analysis of the retrieved questions, we uncovered 4 main themes (see Table~\ref{tab:badge_discussions}), i.e., \emph{Info requests}, \emph{Feedback}, \emph{Improvements}, and \emph{Other}. 
In the following paragraphs, we describe each theme and the corresponding codes, exemplifying the most relevant concepts with excerpts from the discussion threads.

The prevalent theme that emerged from our analysis is the request for information about personal badges -- namely, \emph{Info requests}. 
This theme encompasses 122 distinct questions, a number that doubles those recorded for the next two main themes, i.e., \emph{Improvements} and \emph{Feedback}. 
Arguably, the prevalence of questions from the \emph{Info requests} theme can be explained by the choice of \gh\ not to disclose explicit information on personal badges; this choice has led several users of the platform to seek unofficial information in the community forum.
Within this theme, the most frequent code is \emph{How to get a badge}, assigned to 79 questions concerning the requirements needed to earn a specific badge (e.g., \emph{``How do you get the Galaxy Brain Badge?''}). 
Similarly, 17 more questions were coded as \emph{How to get badges}; these represent requests for help on how to earn more badges in general (\emph{``How to get a new badge on GitHub?''}). 
The remaining 26 questions from this theme were coded as \emph{General info}; these are generic requests for information about the badges feature, the most common being the full list of the currently available personal badges in \gh\ (e.g., \emph{``How can I see all available badges in GitHub? Please Help!!''}).

The next major theme identified is the \emph{Improvements} category, which includes 62 questions and  2 codes: \emph{Badge(s) proposals} and \emph{UX suggestions}.
Under \emph{Badge(s) proposals}, we collected 22 questions representing requests for the addition of new personal badges in \gh. 
For instance, three users recommended adding an additional badge to reward code-review activity (e.g., \emph{``I've learned a lot from code reviewing others' code and being code reviewed by others''}); %{[}\ldots{]} So I think it's fair to have one {[}badge{]} for the code review as well''}; 
furthermore, three other users proposed the addition of organization-specific badges (e.g., \emph{``to celebrate the first PR merge of a new employee''}). 
Some users even suggested creative names for their new badge ideas, such as the \emph{``Issue Muncher''} badge---keeping track of \emph{``how many issues one has helped to close through linked PRs''}---and the \emph{``Bug Hunter''} badge---rewarding reports of bugs in \gh. 
Also, a user suggested extending the badge feature to showcase actual professional achievements such as the Linux Foundation Certified Kubernetes Administrator certificate or the Google Certification Academy certificate. 
Finally, a couple more questions with this code concerned how to suggest new types of badges, confirming that some users are willing to participate in the definition of future personal achievements.
As for the \emph{UX suggestions} code, we identified 40 threads. 
In particular, 21 questions concerned a feature request, the most popular (with 8 questions) being the possibility to selectively hide/show the earned badges on user profile pages; other recurring feature requests are: enabling custom badge ordering in profiles (2), taking historical data into account when defining badge unlocking rules (2), and enable a summary view displaying all possible achievements (2). Another common type of UX-related question (11 questions) represents negative feedback about the graphical appearance of badges (e.g., \emph{``These new badges are too cartoonish''}, \emph{``The designs could be more professional''}) or suggestions on how to improve the display of badges (e.g., \emph{``Make the old badges look like the new ones.''})

The most significant theme in relation to RQ4 is \emph{Feedback}, encompassing the codes \emph{Positive}, \emph{Negative (in general)}, and \emph{Negative (specific)}, for a total of 53 questions. 
While there are some questions (12) reporting positive feedback, in the form of short and generic statements of appreciation for the personal badges feature, most of the questions gathered within this category were coded as negative feedback, either generic (25 questions) or feature-specific (16 questions). 
As for the generic negative feedback, several users reported disregarding the new feature, with 17 out of 25 of them asking how to opt out (or declaring the intention to do so).
Frequent causes for the negative feedback were: i) aversion to the gamification of a professional environment like \gh\ and the \emph{``childish''} style %and ``trivial'' nature 
of badges -- e.g., \emph{``Gamification has no place here. Useless twaddle''} %\emph{``Pull Shark? That's pretty patronizing. I do this for a living.''}; \emph{``{[}\ldots{]} turning GitHub into a social network style site is a terrible idea''} 
(11 questions); ii) skepticism about the ability of personal badges to convey the \emph{``performance/dedication/talent/achievement of a developer''} (2 questions); iii) concerns about the potential negative effects of badges on user behaviors -- e.g., \emph{``PRs created not for the purpose of making a meaningful contribution, but simply to get another badge checked off their list.''} -- and the whole community 
-- e.g., \emph{``{[}Badges{]} risk becoming a driver for ivory tower superiority and community tribalism, which is something the developer community already struggles with''} (5 questions).  %(I'm not listening to your feature request because I have badges and you don't)

As for the badge-specific negative feedback, most of the complaints concerned the \texttt{YOLO} badge (7 questions), which is perceived by several developers as a source of shame and an unfair mark to get, especially if earned within single-person projects (e.g., \emph{``I feel having the YOLO badge does not pass the right message, especially to recruiters''}). Another common criticism (4 questions) concerned the unclear requirements for unlocking the \texttt{Pull Shark} badge, sometimes perceived as biased (e.g., \emph{``Pull Shark achievement should work {[}only{]} for repositories I don't own. {[}\ldots{]} we will always merge our own pull requests, but not everyone will merge our pull request.''}) % and that's the point

Finally, the theme \emph{Other} gathered 46 questions, mostly coded as requests of support for a \emph{Technical issue} (e.g., \emph{%``Pull Shark details not correct {[}\ldots{]} 
Pull Shark shows the wrong merged PR number for me''}), with only 4 questions coded as \emph{Miscellaneous}.






\begin{table}[t]
    \centering
    \caption{Results from the thematic analysis of questions.}
    \label{tab:badge_discussions}
    \begin{tabular}{clc}
    \hline
    \textbf{Category} & \textbf{Code} & \textbf{Frequency} \\
    \hline
    \addlinespace
    \multirow{3}{*}{\makecell{\textbf{\textit{Info requests}}\\ (122)}} & General info & 26 \\
    & How to get badges & 17 \\
    & How to get a badge & 79 \\
    \addlinespace
    \multirow{3}{*}{\makecell{\textbf{\textit{Feedback}}\\ (53)}} & Positive & 12 \\
    & Negative (in general) & 25 \\
    & Negative (badge-specific) & 16 \\
    \addlinespace
    \multirow{2}{*}{\makecell{\textbf{\textit{Improvements}}\\ (63)}} & Badge(s) proposal & 23 \\
    & UX suggestion & 37 \\
    \addlinespace
    \multirow{2}{*}{\makecell{\textbf{\textit{Other}}\\ (46)}} & Technical issue & 42 \\
    & Miscellaneous & 4 \\
    % & Excluded & 31 \\
    \addlinespace
    \hline
    \end{tabular}
    \vspace{-0.4cm}
\end{table}


\section{Discussion}\label{sec:discussion}

In this paper, we studied personal achievement badges, a new feature introduced into \gh, with unknown effects.

\textbf{Research questions}.
We answered four research questions.
First (RQ$_1$), we explored the diffusion of badges among  a sample of 6,022 \gh\ developers as of Dec. 2022 (Table~\ref{tab:badges_tiers}) and their evolution over a six-month time span, since their introduction in Jun. 2022 (Figure~\ref{fig:plotrq1}).
We found that all developers except one own at least one badge and that the number of those who opted out from displaying badges and chose to make their profile pages private has been steadily increasing.
Then (RQ$_2$), we investigated whether owning badges is associated with signaling certain hypothesized developers' qualities and dispositions. 
The results (Figure~\ref{fig:rq2_violinsplots}) showed a small difference between developers who own the \texttt{Pair Extraordinaire badge}---gained by coauthoring merged PRs and a signal of an increased desire to collaborate---and a large difference for those owning the \texttt{Starstruck} badge---gained by those who get their repositories starred and a signal of increased popularity.
Furthermore, the regression analysis (RQ$_3$) revealed that the introduction of badges caused a large increase in the number of followers of developers who own the \texttt{Starstruck} badge (Table~\ref{tab:dd-regressions}).
Finally (RQ$_4$), the analysis of 33 survey responses showed several shortcomings in the current implementation of the badges, such as the limited types of badges currently existing and the lack of those more accurately reflecting contributions to projects and years of experience.
Besides, the analysis of 312 community discussions about personal badges revealed that \gh\ users are willing to know more about badges and the requirements set out to earn them; also, part of the community is willing to get involved in the definition of future achievements.
Nonetheless, the analyzed discussions showed the clear prevalence of negative opinions on badges, confirming the skepticism of several users about their ability to adequately convey the skills of a developer and the concerns of the community on the potential drawbacks of gamifying \gh.

\textbf{Badges as signals}.
The results of the user survey confirmed that \gh\ developers do look at profile pages and are aware of the elements therein---so much so that several even decided to opt out from displaying badges.
As such, these pages have the potential to act as hubs that make socially translucent signals visible and readily available, and support awareness of collaborators’ behavior~\cite{erickson2000}.
This result is consistent with those reported by \fullcite{shami2009} who used the signaling theory as a conceptual framework to investigate how users of online social platforms rely on digital artifacts for `people sensemaking,' i.e., use portions of profile pages as proxies used to infer unknown coworkers' expertise.

However, the visibility of personal badges is not sufficient to ensure their reliability.
According to the signaling theory~\cite{donath2008}, to be reliable, assessment signals must be both observable and costly to produce.
Our findings suggest that most of the personal badges currently available, even though they reflect possessed properties, do not send reliable assessment signals because they may not be costly enough to produce. %; one possible explanation is that \gh\ users are not aware of the different tiers for the same badge.
The only exception to the general unreliability is the \texttt{Starstruck} badge signaling popularity.
One potential explanation for this finding is that the \texttt{Starstruck} badge is awarded to a developer as a result of the interest of others (stars) in their repositories: creating a repository implies an effort broader in scope, if not harder, as compared to others badges, which require the developers to take one action (e.g., answer a question, merge a pull request).
This result is somewhat consistent with those reported by \fullcite{trockman2018} who found repository badges related to popularity to be more reliable as assessment signals.
One major difference with their work is that repository badges are \textit{chosen}---maintainers select what they intend to signal by adding them to README pages (e.g., the count of downloads badge to signal popularity, up-to-date dependency badge to signal the lack of known security risks).
Conversely, personal badges are a true gamification element---they are unlocked and displayed on personal pages---and the only control developers have over them is choosing whether to opt out from showing them.

\textbf{Badges as a gamification mechanism}.
%Compare to the papers discussing gamification frameworks for \se. 
%Because repository badges are not gained based on developers' activity, personal badges are a more proper type of gamification mechanism.
According to \fullcite{hunter2012}, users of a gamified environment go through a journey, from \textit{onboarding} to \textit{scaffolding} and, finally, \textit{mastery}.
As argued by~\fullcite{dalsasso2017}, implementing game mechanisms such as badges to gamify a software development environment is a far-from-trivial, iterative endeavor.
Our analyses suggest that the leveling process of personal badges in \gh\ needs to be adjusted, especially during the onboarding and mastery stages.
On the one hand, some junior developers among the survey respondents reported that it \textit{``felt nice}'' to discover the first badge on their profile.
However, the easier-to-achieve, onboarding badges (i.e., \texttt{YOLO} and \texttt{Quickdraw}) are obtained through practices that are not to be encouraged in software development (i.e., closing issues and PRs quickly or without code review)---in the community discussions, some users referred to these badges  as ``\textit{shameful}'' and ``\textit{trivial}.''
Conversely, more senior developers argued that they feel that current badges are ``\textit{too easily achieved, not like in Stack Overflow}'' and  complained about the lack of badges reflecting their years of experience and project contributions.

\textbf{Implications for platform designers}.
The negative feedback collected from the survey responses and the discussion analysis highlights some criticalities in the design of personal badges.
Platform designers might leverage these insights to make informed decisions about future improvements. For instance, they might take into account popular feature requests, like the ability to selectively hide/show badges on user profile pages; this might alleviate the discontent caused by badges -- like \texttt{YOLO} -- which mark and publicly expose undesirable user behaviors. Moreover, they might consider making the progress state towards the achievement of badges more explicit, especially in case of debated badge requirements (e.g., those of the \texttt{Pull Shark} badge).
Also, to meet the expectations of users, graphic designers might revise the appearance of badges to give them a more professional and consistent look and feel. % (e.g., inspired by the look of actual professional achievements).

In addition, as discussed above, the current number of badges and the implementation of the leveling system do not seem to fit well the onboarding and mastery stages typical of a gamified environment. 
As such, we suggest platform designers consider adding more: i) \textit{onboarding badges}, that are appealing to `newcomers' (e.g., junior developers) without fostering bad software development practices such as no or quick code reviews; ii) \textit{mastery badges}, that reflect long-running, broad-in-scope achievements that are appealing to and feel `earned' by more experienced, senior developers.

Furthermore, the current implementation of badges in \gh\ appears to be underdeveloped.
First, \gh\ platform designers should consider rolling out more of them, all at once rather than slowly, as some users complained that they ``\emph{all have almost the same badges}.''
Second, they should also reflect on the decision not to provide any official documentation regarding badges.
This decision implies that no clear indication is given of what cues badges are intended to signal beyond their name and image.
As of this writing, the community has discovered two more badges, namely \texttt{Heart On Your Sleeve} and \texttt{Open Sourcerer} but no one has uncovered what they mean or how to unlock them yet.


\textbf{Limitations}.
First, we acknowledge that we focused on studying a limited sample of 6,022 \gh\ developers retrieved from 51 organizations.
Because \gh\ currently hosts millions of OSS projects that are the result of the contributions of over 94M developers,\footnote{\url{https://octoverse.github.com/#the-world-of-open-source}} we acknowledge that our findings may not generalize.
Nonetheless, the current findings helped us provide an initial understanding of both the effects of badges and the developers’ perception of the new gamification element introduced into \gh.
Also, we made the complete replication package available to facilitate replications and extensions of the current study.

All the descriptions provided regarding the badges have been inferred by \gh\ community members.
As such, another limitation affecting the correlation and regression analyses is that we arbitrarily hypothesized the cues that badges might send as (conventional) signals.
For example, we hypothesized that displaying the \texttt{Quickdraw} badge might be a cue of timeliness.
However, one might also read the very short time taken to close an issue or a pull request as a negative indication that the developer might not always perform thorough code reviews.
Still, we notice that the arbitrariness of the cues sent by badges is an intrinsic limitation of the current implementation of the available badges, which lack official documentation.

Finally, also related to the limited significant results obtained from the correlation and regression analyses, we acknowledge that this might be the effect of aggregating all the badge tiers (see Table~\ref{tab:badges_tiers}) and conducting the analyses distinguishing only between developers with and without badges.
It is, therefore, possible that tier-3 badges, which are harder to achieve, might send more reliable signals than tier-1 badges of the same type.


\section{Conclusions}\label{sec:conclusions}
In this paper, we analyzed the introduction of a new gamification feature in \gh, namely personal achievement badges.
We studied the distribution of badges, the signals they send, and the effects of their introduction.
We also collected evidence of the reaction of the developers' community to the new feature.
As the implementation of the feature appears to be at the initial stage, we will consider furthering this study as changes (e.g., more badges) are introduced into \gh.

%\section*{Acknowledgment}
%Omitted to ensure anonymity.
%We are grateful to the anonymous respondents to the survey.
%We also thank Claudia Capozza for her guidance and feedback on the regression modeling.



\bibliographystyle{IEEEtranN}
\bibliography{biblio}


\end{document}
